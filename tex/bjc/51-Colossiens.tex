\ShortTitle{Col.}\BookTitle{Colossiens}\BFont
\noindent\hrulefill
{\footnotesize
\textit{
\bigskip
{\centering{}
\\Auteur~: Paul
\\Thème~: La prééminence de Christ
\\Date de rédaction~: Env. 60 ap. J.-C.\\}
}
\textit{
\\Située en Asie Mineure, Colosses était une ville de Phrygie qui se trouvait à environ deux cents kilomètres d'Ephèse.
\\Rédigée lors de la première captivité romaine de Paul, la lettre aux Colossiens a pour but de rétablir la suprématie de Christ. En effet, cette église - dont Epaphras, le probable fondateur, s'était converti à Ephèse au cours des trois années que Paul y passa - était sous l'influence d'enseignements séducteurs basés sur le gnosticisme. Cette philosophie à la fois attrayante et très dangereuse prônait entre autres le salut par la connaissance et le dualisme.\bigskip
}
}
\par\nobreak\noindent\hrulefill
\begin{multicols}{2}
\Chap{1}
\TextTitle{Introduction}
\VerseOne{}Paul, apôtre de Jésus-Christ, par la volonté de Dieu, et le frère Timothée,
\VS{2}aux saints et frères, fidèles en Christ, qui sont à Colosses~: Que la grâce et la paix vous soient données de la part de Dieu notre Père et de la part du Seigneur Jésus-Christ~!
\VS{3}Nous rendons grâces à Dieu, qui est le Père de notre Seigneur Jésus-Christ, et nous prions toujours pour vous,
\VS{4}ayant entendu parler de votre foi en Jésus-Christ, et de votre charité envers tous les saints,
\VS{5}à cause de l'espérance des biens qui vous sont réservés dans les cieux, et dont vous avez eu précédemment connaissance par la parole de la vérité, c'est-à-dire par l'Evangile,
\VS{6}qui est parvenu jusqu'à vous, comme il l'est aussi dans le monde entier. Et il porte des fruits, comme aussi parmi vous, depuis le jour où vous avez entendu et connu la grâce de Dieu dans la vérité,
\VS{7}ainsi que vous en avez aussi été instruits par Epaphras, notre cher compagnon de service, qui est pour vous un fidèle serviteur de Christ,
\VS{8}et qui nous a fait connaître votre charité par le Saint-Esprit.
\TextTitle{Prière de Paul pour les Colossiens}
\VS{9}C'est pourquoi depuis le jour où nous l'avons appris, nous ne cessons pas de prier pour vous, et de demander à Dieu que vous soyez remplis de la connaissance de sa volonté, en toute sagesse et intelligence spirituelle,
\VS{10}afin que vous vous conduisiez d'une manière digne du Seigneur, pour lui plaire en toutes choses, portant des fruits en toutes sortes de bonnes œuvres, et croissant dans la connaissance de Dieu,
\VS{11}étant fortifiés en toute force, selon la puissance de sa gloire, pour toute patience et constance, avec joie.
\TextTitle{Le salut de Dieu}
\VS{12}Rendant grâces au Père, qui nous a rendus capables d'avoir part à l'héritage des saints dans la lumière,
\VS{13}qui nous a délivrés de la puissance des ténèbres, et nous a transportés dans le Royaume du Fils de son amour,
\VS{14}en qui nous avons la rédemption par son sang, à savoir la rémission des péchés.
\TextTitle{Jésus, le Créateur}
\VS{15}Lequel est l'image du Dieu invisible, le premier-né\FTNT{Dans les Ecritures, l'expression «~premier-né~» est appliquée au Seigneur pour exprimer trois réalités. Tout d'abord, on parle de Jésus en tant que premier-né de Marie, c'est-à-dire son fils aîné (Lu. 2:6-7). Ensuite, on trouve cette expression au sens figuré, pour marquer une distinction (par exemple concernant Israël~; Ex. 4:2) ou désigner la particularité et la suprématie d'une personne. Ainsi, bien que David était le dernier-né de son père Isaï (Ps. 89:28), Dieu en fit un premier-né, «~le plus élevé des rois de la terre~» (Ps. 89:28). Il en va de même pour Jésus-Christ. Il n'est pas le premier-né de la création dans le sens de rang de naissance ou de création, autrement Paul aurait employé le terme grec «~prôtoktisis~» qui signifie «~premier-créé~», au lieu de «~prôtotokos~», c'est-à-dire «~premier-né~». Il faut donc voir dans cette expression un titre de supériorité et d'hiérarchie, pour marquer sa prééminence. En effet, la Parole de Dieu déclare clairement que le Seigneur Jésus-Christ est l'Alpha et le Commencement de toutes choses (Ap. 1:8~; Ap. 21:6~; Ap. 22:13), le Créateur suprême (Ge. 1:1~; Ge. 2:7~; Es. 45:11-18~; Ps. 104:30~; Job. 33:4~; Jn. 1:3~; 1 Co. 8:6~; Col. 1:12-16~; Ap. 22:3~; Ap. 14:6). D'ailleurs, il l'a lui-même affirmé sans ambiguïté~: «~Avant qu'Abraham fût, Je suis~» (Jn. 8:58). Enfin, Jésus-Christ est aussi appelé le premier-né d'entre les morts (Col. 1:18). Cela ne signifie pas qu'il a été le premier à ressusciter, car il y a eu plusieurs résurrections avant la sienne, mais il fut le premier à ressusciter avec un corps glorieux. Sa résurrection est donc le gage de la promesse de la résurrection de tous ceux qui ont foi en lui (Jn. 3:16).} de toute la création.
\VS{16}Car par lui ont été créées toutes les choses qui sont dans les cieux et sur la terre, les visibles et les invisibles, soit les trônes, ou les dominations, ou les principautés, ou les puissances, toutes choses ont été créées par lui, et pour lui.
\VS{17}Et il est avant toutes choses, et toutes choses subsistent par lui.
\VS{18}Et c'est lui qui est le Chef du corps de l'Eglise, et qui est le commencement et le premier-né d'entre les morts, afin qu'il tienne le premier rang en toutes choses,
\VS{19}car le bon plaisir du Père a été que toute plénitude habitât en lui.
\VS{20}Et de réconcilier par lui toutes choses avec lui même, ayant fait la paix par le sang de sa croix, à savoir, tant les choses qui sont dans les cieux que celles qui sont sur la terre.
\VS{21}Et vous, qui étiez autrefois étrangers, et qui étiez ses ennemis dans votre entendement, et dans les mauvaises œuvres, il vous a maintenant réconciliés
\VS{22}par le corps de sa chair, par sa mort, pour vous présenter saints, et sans tache, et irrépréhensibles devant lui~;
\VS{23}si toutefois vous demeurez dans la foi, étant fondés et fermes, et n'étant pas transportés hors de l'espérance de l'Evangile que vous avez entendu, lequel est prêché à toute créature qui est sous le ciel, dont moi Paul, j'ai été fait le serviteur.
\VS{24}Je me réjouis donc maintenant dans mes souffrances pour vous~; et j'accomplis le reste des afflictions de Christ dans ma chair, pour son corps, qui est l'Eglise.
\VS{25}C'est d'elle que j'ai été fait le serviteur, selon la gestion que Dieu m'a donnée auprès de vous, afin que j'exécute pleinement la parole de Dieu,
\VS{26}à savoir le mystère qui avait été caché dans tous les siècles et dans tous les âges, mais qui est maintenant manifesté à ses saints~;
\VS{27}auxquels Dieu a voulu faire connaître quelle est la richesse de la gloire de ce mystère parmi les Gentils, à savoir Christ en vous, l'espérance de la gloire~;
\VS{28}c'est lui que nous annonçons, exhortant tout homme et enseignant tout homme en toute sagesse, afin que nous puissions présenter tout homme parfait en Jésus-Christ.
\TextTitle{Le combat de Paul}
\VS{29}C'est aussi à quoi je travaille, en combattant selon son efficacité\FTNT{le terme «~efficacité~» vient du grec «~energeia~» qui signifie «~action~», «~fonctionnement~», «~compétence~», ou encore «~force à l'œuvre dans~». Ce mot est utilisé seulement pour parler du pouvoir surhumain que ce soit celui de Dieu ou celui du diable. (Ep. 1:19~; Ph. 3:21~; 2 Ti. 2:9).}, qui agit puissamment en moi.
\Chap{2}
\VerseOne{}Or je veux, en effet, que vous sachiez combien est grand le combat que j'ai pour vous, et pour ceux qui sont à Laodicée, et pour tous ceux qui n'ont pas vu mon visage dans la chair,
\VS{2}afin que leurs cœurs soient consolés, étant unis ensemble dans la charité, et enrichis d'une pleine intelligence, pour la connaissance du mystère de notre Dieu et Père, et de Christ,
\VS{3}en qui sont cachés tous les trésors de la sagesse et de la connaissance.
\TextTitle{Mise en garde contre les discours séduisants et la philosophie\FTNTT{1 Co. 2:4~; Ro. 16:17-18~; 2 Pi. 2:3.}}
\VS{4}Or je dis cela afin que personne ne vous trompe par des discours séduisants.
\VS{5}Car bien que je sois absent de corps, toutefois je suis avec vous en esprit, me réjouissant, et voyant votre ordre et la fermeté de votre foi, que vous avez en Christ.
\VS{6}Ainsi, comme vous avez reçu le Seigneur Jésus-Christ, marchez en lui,
\VS{7}étant enracinés et édifiés en lui, et affermis dans la foi, selon que vous avez été enseignés, abondant en elle avec reconnaissance.
\VS{8}Prenez garde que personne ne fasse de vous sa proie par la philosophie, et par de vaines tromperies conformes à la tradition des hommes et aux rudiments du monde, et non pas à la doctrine de Christ.
\TextTitle{La divinité du Christ}
\VS{9}Car en lui habite corporellement toute la plénitude de la divinité\FTNT{En Jésus-Christ habite toute la plénitude de la divinité. Il est le Dieu Tout-Puissant.}.
\VS{10}Et vous êtes comblés en lui, qui est le Chef de toute principauté et puissance.
\TextTitle{L’œuvre de la croix}
\VS{11}En lui aussi vous êtes circoncis d'une circoncision faite sans main, qui consiste à dépouiller le corps des péchés de la chair, ce qui est la circoncision de Christ.
\VS{12}Etant ensevelis avec lui par le baptême, en lui aussi vous êtes ressuscités ensemble par la foi de l'efficacité de Dieu, qui l'a ressuscité des morts.
\VS{13}Et lorsque vous étiez morts dans vos offenses, et dans l'incirconcision de votre chair, il vous a vivifiés ensemble avec lui, vous ayant gratuitement pardonné toutes vos offenses.
\VS{14}Il a effacé l'acte qui était contre nous, qui consistait en des ordonnances, et qui nous était contraire, et il l'a entièrement aboli en le clouant à la croix.
\VS{15}Il a dépouillé les principautés et les puissances, et les a exposées publiquement en spectacle, en triomphant d'elles par la croix.
\TextTitle{Mise en garde contre les commandements et les doctrines des hommes}
\VS{16}Que personne donc ne vous juge au sujet du manger ou du boire, ou au sujet d'un jour de fête, ou d'un jour de nouvelle lune, ou de sabbat,
\VS{17}qui sont l'ombre des choses qui devaient venir, mais le corps est en Christ.
\VS{18}Que personne ne vous enlève à son gré le prix de la course, sous l'apparence d'humilité d'esprit et par un culte des anges, s'ingérant dans les choses qu'il n'a pas vues, étant témérairement enflé par ses pensées charnelles,
\VS{19}sans s'attacher au Chef, dont tout le corps étant joint et ajusté ensemble par des jointures et des liens, s'accroît d'un accroissement de Dieu.
\VS{20}Si donc vous êtes morts avec Christ quant aux rudiments du monde, pourquoi vous impose-t-on ces ordonnances, comme si vous viviez dans le monde~?
\VS{21}A savoir~: Ne prends pas~! ne goûte pas~! ne touche pas~!
\VS{22}lesquelles sont toutes périssables par l'usage, et établies suivant les commandements et les doctrines des hommes~;
\VS{23}et qui ont pourtant quelque apparence de sagesse en dévotion volontaire, et en humilité d'esprit, et en ce qu'elles n'épargnent pas le corps, et n'ont aucun égard à la satisfaction de la chair.
\Chap{3}
\TextTitle{Rechercher les choses d'en haut}
\VerseOne{}Si donc vous êtes ressuscités avec Christ, cherchez les choses qui sont en haut, où Christ est assis à la droite de Dieu.
\VS{2}Pensez aux choses d'en haut, et non à celles qui sont sur la terre.
\VS{3}Car vous êtes morts, et votre vie est cachée avec Christ en Dieu.
\VS{4}Quand Christ, qui est votre vie, apparaîtra, alors vous paraîtrez aussi avec lui dans la gloire.
\TextTitle{La mort à soi en pratique}
\VS{5}Faites donc mourir vos membres qui sont sur la terre~: La fornication, l'impureté, les passions, les mauvais désirs, et la cupidité, qui est une idolâtrie.
\VS{6}C'est à cause de ces choses que la colère de Dieu vient sur les fils de la rébellion,
\VS{7}parmi lesquels vous marchiez autrefois, quand vous viviez dans ces choses.
\VS{8}Mais maintenant, vous aussi, rejetez toutes ces choses~: La colère, l'animosité, la médisance, et les paroles déshonnêtes qui pourraient sortir de votre bouche.
\VS{9}Ne mentez pas les uns aux autres, vous étant dépouillés du vieil homme et de ses œuvres,
\VS{10}et ayant revêtu le nouvel homme, qui se renouvelle dans la connaissance, selon l'image de celui qui l'a créé,
\VS{11}en qui il n'y a ni Grec ni Juif, ni circoncis ni incirconcis, ni barbare ni Scythe, ni esclave ni libre~; mais Christ y est tout et en tous.
\VS{12}Ainsi donc, comme des élus de Dieu, saints et bien-aimés, revêtez-vous des entrailles de miséricorde, de bonté, d'humilité, de douceur, de patience~;
\VS{13}vous supportant les uns les autres, et vous pardonnant les uns aux autres~; et si l'un a querelle contre l'autre, comme Christ vous a pardonné, vous aussi faites-en de même.
\VS{14}Mais par-dessus toutes ces choses, revêtez-vous de la charité, qui est le lien de la perfection.
\VS{15}Et que la paix de Dieu, à laquelle aussi vous êtes appelés pour être un seul corps, tienne le principal lieu dans vos cœurs. Et soyez reconnaissants.
\VS{16}Que la parole de Christ habite abondamment en vous en toute sagesse~; instruisez-vous et exhortez-vous les uns les autres par des psaumes, par des hymnes et des cantiques spirituels, chantant dans votre cœur au Seigneur avec reconnaissance.
\VS{17}Et quoi que vous fassiez, en parole ou en œuvre, faites tout au Nom du Seigneur Jésus, rendant grâces par lui à notre Dieu et Père.
\TextTitle{La famille selon Dieu}
\VS{18}Femmes, soyez soumises à vos maris, comme il convient dans le Seigneur\FTNT{Ep. 5:22.}.
\VS{19}Maris, aimez vos femmes, et ne vous aigrissez pas contre elles\FTNT{Ep. 5:25.}.
\VS{20}Enfants, obéissez à vos pères et à vos mères en toutes choses, car cela est agréable au Seigneur\FTNT{Ep. 6:1-2.}.
\VS{21}Pères, n'irritez pas vos enfants\FTNT{Ep. 6:4.}, afin qu'ils ne se découragent pas.
\TextTitle{Les rapports entre serviteurs et maîtres selon Dieu}
\VS{22}Serviteurs, obéissez en toutes choses à ceux qui sont vos maîtres selon la chair, ne servant pas seulement sous leurs yeux, comme voulant complaire aux hommes, mais en simplicité de cœur, craignant Dieu\FTNT{Ep. 6:5-6.}.
\VS{23}Et quoi que vous fassiez, faites tout de bon cœur, comme le faisant pour le Seigneur, et non pas pour les hommes,
\VS{24}sachant que vous recevrez du Seigneur l'héritage pour récompense. Car vous servez Christ, le Seigneur.
\VS{25}Mais celui qui agit injustement recevra ce qu'il aura fait injustement, car en Dieu il n'y a pas d'égard à l'apparence des personnes.
\Chap{4}
\VerseOne{}Maîtres, accordez à vos serviteurs ce qui est juste et équitable, sachant que vous avez, vous aussi, un Maître dans les cieux.
\TextTitle{La persévérance dans la prière}
\VS{2}Persévérez dans la prière, veillant dans cet exercice avec des actions de grâces.
\VS{3}Priez aussi tous ensemble pour nous, afin que Dieu nous ouvre une porte pour la parole, afin d'annoncer le mystère de Christ pour lequel aussi je suis prisonnier,
\VS{4}afin que je le fasse connaître comme je dois en parler.
\VS{5}Conduisez-vous sagement envers ceux du dehors, et rachetez le temps.
\VS{6}Que votre parole soit toujours assaisonnée de sel, avec grâce, afin que vous sachiez comment vous avez à répondre à chacun.
\TextTitle{Salutations}
\VS{7}Tychique, notre frère bien-aimé, et fidèle serviteur, et mon compagnon de service en notre Seigneur, vous fera savoir tout mon état.
\VS{8}Je l'envoie vers vous expressément, afin qu'il connaisse quel est votre état, et qu'il console vos cœurs~;
\VS{9}avec Onésime, notre fidèle et bien-aimé frère, qui est des vôtres. Ils vous feront connaître toutes les choses d'ici.
\VS{10}Aristarque, qui est prisonnier avec moi, vous salue aussi, et Marc qui est le cousin de Barnabas, au sujet duquel vous avez reçu un ordre, s'il vient chez vous, recevez-le.
\VS{11}Et Jésus, appelé Justus, vous salue aussi. Ils sont du nombre des circoncis, et les seuls qui travaillent avec moi pour le Royaume de Dieu, et qui ont été pour moi une consolation.
\VS{12}Epaphras, qui est des vôtres, et serviteur de Jésus-Christ, vous salue~; il ne cesse de combattre pour vous dans ses prières, afin que vous demeuriez parfaits et accomplis dans toute la volonté de Dieu.
\VS{13}Car je lui rends témoignage qu'il a un grand zèle pour vous, et pour ceux de Laodicée, et pour ceux d'Hiérapolis.
\VS{14}Luc, le médecin bien-aimé, vous salue, ainsi que Démas.
\VS{15}Saluez les frères qui sont à Laodicée, et Nymphas, avec l'église qui est dans sa maison.
\VS{16}Et quand cette lettre aura été lue entre vous, faites en sorte qu'elle soit aussi lue dans l'église des Laodicéens, et que vous lisiez aussi celle qui viendra de Laodicée.
\VS{17}Et dites à Archippe~: Prends garde au service que tu as reçu dans le Seigneur afin de bien le remplir.
\VS{18}Je vous salue, moi Paul, de ma propre main. Souvenez-vous de mes liens. Que la grâce soit avec vous~! Amen~!
\PPE{}
\end{multicols}
