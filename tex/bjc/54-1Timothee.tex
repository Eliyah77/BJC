\ShortTitle{1 Ti.}\BookTitle{1 Timothée}\BFont
\noindent\hrulefill
{\footnotesize
\textit{
\bigskip
{\centering{}
\\Auteur~: Paul
\\(Gr.~: Timotheos)
\\Signification~: Qui adore ou honore Dieu
\\Thème~: Comment se conduire dans l'église
\\Date de rédaction~: Env. 64 ap. J.-C.\\}
}
\textit{
\\Cette lettre s'adresse à Timothée dont le père était grec et la mère juive. Le jeune homme se convertit à Christ avec sa mère et sa grand-mère dès le premier voyage missionnaire de Paul au cours duquel il passa à Lystre.
\\Cette épître fut rédigée après la première captivité de Paul à Rome. Alors que les églises connaissaient une certaine expansion, Paul s'adresse à Timothée, jeune et fidèle compagnon d'œuvre qu'il a lui-même formé, sur des questions d'ordre disciplinaire et sur la pureté de la foi. Dans cette épître, dite pastorale, Paul donne des instructions précises à Timothée pour enseigner, exhorter, diriger le culte public et choisir ses collaborateurs.\bigskip
}
}
\par\nobreak\noindent\hrulefill
\begin{multicols}{2}
\Chap{1}
\TextTitle{Introduction}
\VerseOne{}Paul, apôtre de Jésus-Christ par l'ordre de Dieu, notre Sauveur, et du Seigneur Jésus-Christ, notre espérance,
\VS{2}à Timothée mon véritable fils dans la foi~: Que la grâce, la miséricorde et la paix te soient données de la part de Dieu notre Père, et de Jésus-Christ, notre Seigneur~!
\TextTitle{Mise en garde contre les erreurs doctrinales~; le but de la loi}
\VS{3}Suivant la prière que je te fis de demeurer à Ephèse, lorsque j'allais en Macédoine, je te prie encore d'ordonner à certaines personnes de ne pas enseigner une autre doctrine,
\VS{4}et de ne pas s'adonner aux fables et aux généalogies sans fin, qui produisent des disputes plutôt que l'édification en Dieu qui consiste dans la foi.
\VS{5}Or le but du commandement c'est la charité qui procède d'un cœur pur, d'une bonne conscience, et d'une foi sincère.
\VS{6}Quelques-uns, s'étant détournés de ces choses, se sont écartés dans de vains discours,
\VS{7}voulant être docteurs de la loi~; mais ils ne comprennent ni ce qu'ils disent ni ce qu'ils affirment.
\VS{8}Or nous savons que la loi est bonne pour celui qui en fait un usage légitime,
\VS{9}sachant ceci, que ce n'est pas pour le juste que la loi a été établie, mais pour les méchants et les rebelles, pour les impies et les pécheurs, pour les irréligieux et les profanes, pour les parricides, les meurtriers,
\VS{10}pour les fornicateurs, pour les homosexuels, pour les voleurs d'hommes, pour les menteurs, pour les parjures, et contre telle autre chose qui est contraire à la saine doctrine,
\VS{11}selon l'Evangile de la gloire du Dieu béni, Evangile qui m'a été confié.
\TextTitle{Témoignage de Paul}
\VS{12}Je rends grâces à celui qui m'a fortifié, c'est-à-dire à Jésus-Christ, notre Seigneur, de ce qu'il m'a estimé fidèle en m'établissant dans le service,
\VS{13}moi qui auparavant étais un blasphémateur, un persécuteur, et un homme violent~; mais j'ai obtenu miséricorde parce que j'agissais par ignorance, étant dans l'incrédulité.
\VS{14}Or la grâce de notre Seigneur a surabondé en moi, avec la foi et l'amour qui est en Jésus-Christ.
\VS{15}Cette parole est certaine et entièrement digne d'être reçue, que Jésus-Christ est venu dans le monde pour sauver les pécheurs, dont je suis le premier.
\VS{16}Mais j'ai obtenu miséricorde, afin que Jésus-Christ fasse voir en moi le premier, toute sa clémence, pour que je serve d'exemple à ceux qui croiraient en lui pour la vie éternelle.
\VS{17}Or au Roi des siècles, immortel, invisible, à Dieu seul sage, soient honneur et gloire aux siècles des siècles~! Amen~!
\TextTitle{Recommandations à Timothée}
\VS{18}Mon fils Timothée, je te recommande ce commandement que conformément aux prophéties qui auparavant ont été faites sur toi, tu t'acquittes, selon elles, du devoir de combattre dans cette bonne guerre,
\VS{19}en gardant la foi et une bonne conscience, que quelques-uns ont rejetée, et ils ont fait naufrage quant à la foi.
\VS{20}De ce nombre sont Hyménée et Alexandre, que j'ai livrés à Satan, afin qu'ils apprennent par ce châtiment à ne plus blasphémer.
\Chap{2}
\TextTitle{Instructions sur la prière}
\VerseOne{}J'exhorte donc, avant toutes choses, à faire des requêtes, des prières, des supplications, et des actions de grâces pour tous les hommes,
\VS{2}pour les rois et pour tous ceux qui sont constitués en dignité, afin que nous menions une vie paisible et tranquille, en toute piété et honnêteté.
\VS{3}Car cela est bon et agréable devant Dieu, notre Sauveur,
\VS{4}qui veut que tous les hommes soient sauvés et qu'ils viennent à la connaissance de la vérité.
\TextTitle{Jésus-Christ, le seul médiateur}
\VS{5}Car Dieu est un\FTNT{Ga. 3:20 ; Ja. 2:19 ; Mc. 12:29.}, et aussi le Médiateur entre Dieu et les hommes est un, à savoir Jésus-Christ homme,
\VS{6}qui s'est donné lui-même en rançon pour tous. C'est le témoignage qui a été rendu en son propre temps.
\VS{7}C'est dans cette vue que j'ai été établi prédicateur, apôtre (je dis la vérité en Christ, je ne mens pas) et docteur des Gentils dans la foi et dans la vérité.
\VS{8}Je veux donc que les hommes prient en tout lieu, levant leurs mains pures, sans colère, et sans dispute.
\TextTitle{La tenue de la femme}
\VS{9}Et de même, que les femmes, vêtues d'une manière décente, avec pudeur et modestie, ne se parent ni de tresses, ni d'or, ni de perles, ni d'habits somptueux,
\VS{10}mais qu'elles se parent de bonnes œuvres, comme il convient à des femmes qui font profession de servir Dieu.
\TextTitle{Le comportement de la femme envers son mari}
\VS{11}Que la femme apprenne dans le silence, en toute soumission.
\VS{12}Car je ne permets pas à la femme d'enseigner ni d'user d'autorité\FTNT{Le mot «~autorité~» vient du grec «~authenteo~» et signifie «~celui qui tue de ses propres mains un autre ou lui-même~; celui qui agit de sa propre autorité, autocrate~; un maître absolu~; gouverneur, exercer une domination~».} sur le mari~; mais elle doit demeurer dans le silence\FTNT{Le mot grec traduit par «~silence~» est «~hesuchia~» qui signifie «~en silence~; paisiblement~». La racine de ce terme est «~hesuchios~»~: tranquille, paisible.}.
\VS{13}Car Adam a été formé le premier, Eve ensuite.
\VS{14}Et ce n'est pas Adam qui a été séduit, mais la femme, ayant été séduite, a été la cause de la transgression.
\VS{15}Elle sera néanmoins sauvée en mettant des enfants au monde\FTNT{Il est évident que le salut ne dépend pas du fait d'enfanter puisque nous sommes sauvés par grâce et non par les œuvres. Ce verset fait référence à Eve, la mère de tous les vivants. Par elle le péché et la mort sont entrés dans le monde (Ro. 5:12) mais c'est aussi par sa postérité, à savoir Christ (Ge. 3:15), que le salut a été apporté.}, pourvu qu'elle persévère dans la foi, dans la charité, et dans la sanctification, avec modestie.
\Chap{3}
\TextTitle{Les évêques et les diacres doivent manifester le caractère de Christ}
\VerseOne{}Cette parole est certaine, si quelqu'un désire la charge d'évêque\FTNT{Evêque, du grec «~episkope~», signifie «~investigation, inspection, visite d'inspection~». C'est un acte par lequel Dieu visite les hommes, observe leurs voies, leurs caractères, pour leur accorder en partage joie ou tristesse. Ce terme signifie également surveillance, charge, contrôle, fonction, la fonction d'un ancien. Voir Ac. 1:20.}, il désire une œuvre excellente.
\VS{2}Mais il faut que l'évêque soit irrépréhensible, mari\FTNT{Paul ne dit pas que les évêques ne peuvent pas être célibataires. Il y a en effet une différence entre mari et marié. L'apôtre met l'accent sur la monogamie. Un homme célibataire peut en effet être évêque s'il remplit les caractéristiques décrites dans ce passage.} d'une seule femme, vigilant, modéré, honorable, hospitalier, propre à enseigner.
\VS{3}Il faut qu'il ne soit ni adonné au vin, ni violent, ni porté au gain déshonnête, mais modéré, éloigné des querelles, exempt d'avarice.
\VS{4}Il faut qu'il dirige honnêtement sa propre maison, et qu'il tienne ses enfants dans la soumission et dans une parfaite honnêteté~;
\VS{5}car si quelqu'un ne sait pas diriger sa propre maison, comment pourra-t-il gouverner l'église de Dieu~?
\VS{6}Il ne faut pas qu'il soit un nouveau converti, de peur qu'enflé d'orgueil, il ne tombe sous le jugement du diable.
\VS{7}Il faut aussi qu'il reçoive un bon témoignage de ceux du dehors, afin de ne pas tomber dans l'opprobre et dans les pièges du diable.
\VS{8}Que les diacres aussi soient honnêtes, éloignés de la duplicité, des excès du vin, d'un gain sordide,
\VS{9}conservant le mystère de la foi dans une conscience pure.
\VS{10}Ils doivent aussi être premièrement éprouvés, et qu'ensuite ils exercent leur service, après avoir été trouvés sans reproche.
\VS{11}Leurs femmes, de même, doivent être honnêtes, non médisantes, sobres, fidèles en toutes choses.
\VS{12}Les diacres doivent être maris d'une seule femme, dirigeant honnêtement leurs enfants, et leurs propres maisons.
\VS{13}Car ceux qui auront bien servi s'acquièrent un rang honorable, et une grande liberté dans la foi qui est en Jésus-Christ.
\VS{14}Je t'écris ces choses espérant que j'irai bientôt vers toi~;
\VS{15}et afin que tu saches, si je tarde, comment il faut se conduire dans la maison de Dieu, qui est l'Eglise du Dieu vivant, la colonne et l'appui de la vérité.
\TextTitle{Le mystère de la piété}
\VS{16}Et sans contredit, le mystère de la piété\FTNT{Le mystère de la piété. Il s'agit de la connaissance de Dieu manifestée en chair dans la personne de Jésus-Christ, 100\% homme et 100\% Dieu. C'est l'incarnation du Dieu Tout-Puissant dans le seul but de sauver les hommes et de produire dans leurs cœurs la véritable piété.} est grand~: Dieu a été manifesté en chair, justifié par l'Esprit, vu des anges, prêché aux Gentils, cru dans le monde, et élevé dans la gloire.
\Chap{4}
\TextTitle{L'apostasie et la séduction~: Signes des derniers temps}
\VerseOne{}Mais l'Esprit dit expressément que dans les derniers temps, quelques-uns se détourneront de la foi pour s'attacher à des esprits séducteurs et à des doctrines de démons\FTNT{Il est indéniable que nous vivons les dernières minutes avant le retour glorieux de Jésus-Christ. Toutes les conditions sont pratiquement réunies pour que le Seigneur revienne, c'est pourquoi chaque enfant de Dieu doit se préparer à la rencontre avec l'Epoux. Les prophètes, notamment Paul, ont annoncé que la fin des temps serait caractérisée par la séduction et l'abandon de la foi de beaucoup de chrétiens.},
\VS{2}par l'hypocrisie de faux docteurs, ayant leur propre conscience marquée au fer rouge\FTNT{L'expression «~marqué au fer~» ou «~marque de la flétrissure~» se dit «~kauteriazo~» en grec et veut dire «~ceux dont l'âme est stigmatisée par les marques du péché~». Dans un sens médical, ce mot signifie «~cautériser~». Ce passage fait allusion à la marque de la bête qui sera imprimée dans la conscience des hommes~; voilà pourquoi Dieu nous demande de garder sa parole dans nos cœurs (Ps. 119:11). Les Juifs devaient avoir sur leurs mains et sur leurs fronts la marque de Dieu qui est sa parole (De. 6:6-8). La main se dit «~yad~» en hébreu, ce qui signifie «~pouvoir~», «~force~» ou encore «~autorité~»~; elle symbolise donc l'action. Le front se dit «~towphaphah~» en hébreu, ce qui signifie «~marque~»~; il s'agit de la pensée.}~;
\VS{3}défendant de se marier et ordonnant de s'abstenir des viandes que Dieu a créées afin que les fidèles, et ceux qui ont connu la vérité, en usent avec actions de grâces.
\VS{4}Car tout ce que Dieu a créé est bon, et rien ne doit être rejeté, pourvu qu'on le prenne avec actions de grâces,
\VS{5}parce que tout est sanctifié par la parole de Dieu et par la prière.
\VS{6}En exposant ces choses aux frères, tu seras un bon serviteur de Jésus-Christ, nourri des paroles de la foi et de la bonne doctrine que tu as exactement suivie.
\VS{7}Mais rejette les fables profanes, et semblables aux récits de vieilles femmes.
\TextTitle{S'exercer à la piété}
\VS{8}Exerce-toi à la piété~; car l'exercice corporel est utile à peu de chose, tandis que la piété est utile à toutes choses, ayant les promesses de la vie présente et de celle qui est à venir.
\VS{9}C'est là une parole certaine et digne d'être entièrement reçue.
\VS{10}Car c'est aussi à cause de cela que nous endurons des travaux et des opprobres, parce que nous espérons dans le Dieu vivant, qui est le Sauveur de tous les hommes, mais principalement des fidèles.
\VS{11}Déclare ces choses et enseigne-les.
\VS{12}Que personne ne méprise ta jeunesse~; mais sois le modèle pour les fidèles en paroles, en conduite, en charité, en esprit, en foi, en pureté.
\VS{13}Applique-toi à la lecture, à l'exhortation et à l'enseignement, jusqu'à ce que je vienne.
\VS{14}Ne néglige pas le don qui est en toi, et qui t'a été donné par prophétie, par l'imposition des mains de l'assemblée des anciens.
\VS{15}Pratique ces choses et donne-toi tout entier à elles, afin que tes progrès soient évidents pour tous.
\VS{16}Veille sur toi-même et sur la doctrine~; persévère dans ces choses, car en agissant ainsi, tu te sauveras toi-même et tu sauveras ceux qui t'écoutent.
\Chap{5}
\TextTitle{Recommandations concernant les veuves}
\VerseOne{}Ne reprends pas rudement le vieillard, mais exhorte-le comme un père~; les jeunes gens comme des frères,
\VS{2}les femmes âgées comme des mères, celles qui sont jeunes comme des sœurs, en toute pureté.
\VS{3}Honore les veuves qui sont véritablement veuves.
\VS{4}Mais si une veuve a des enfants, ou des petits enfants, qu'ils apprennent avant tout à exercer la piété envers leur propre famille, et à rendre à leurs parents ce qu'ils ont reçu d'eux~; car cela est bon et agréable à Dieu.
\VS{5}Or celle qui est véritablement veuve, et qui est laissée seule, espère en Dieu, et persévère nuit et jour dans les supplications et les prières.
\VS{6}Mais celle qui vit dans les plaisirs est morte quoique vivante.
\VS{7}Avertis-les donc de ces choses, afin qu'elles soient irrépréhensibles.
\VS{8}Si quelqu'un n'a pas soin des siens, et principalement de ceux de sa famille, il a renié la foi, et il est pire qu'un infidèle.
\VS{9}Qu'une veuve, pour être enregistrée sur le rôle\FTNT{Inscription sur le rôle~: Expression qui s'apparente à l'enrôlement des soldats. Il est question des veuves ayant une place importante dans l'église, du fait qu'elles exercent une certaine responsabilité sur le reste des femmes, et ayant en charge les veuves et les orphelins pris en compte pour la dépense publique.}, n'ait pas moins de soixante ans, qu'elle ait été la femme d'un seul mari,
\VS{10}ayant le témoignage d'avoir fait de bonnes œuvres, comme d'avoir bien élevé ses propres enfants, d'avoir exercé l'hospitalité envers les étrangers, d'avoir lavé les pieds des saints, d'avoir secouru les affligés, et de s'être ainsi constamment appliquée à toutes sortes de bonnes œuvres.
\VS{11}Mais refuse les veuves qui sont plus jeunes~; car quand elles sont devenues lascives\FTNT{Ce mot vient du grec «~katastreniao~»~: «~ressentir les pulsions du désir sexuel~».} contre Christ, elles veulent se marier,
\VS{12}ayant leur condamnation, en ce qu'elles ont violé leur première foi.
\VS{13}Et avec cela aussi, étant oisives, elles apprennent à aller de maison en maison~; et non seulement elles sont oisives, mais encore causeuses, et curieuses, et parlant de choses qui ne sont pas bienséantes.
\VS{14}Je veux donc que les jeunes veuves se marient, qu'elles aient des enfants, qu'elles gouvernent leur ménage, et qu'elles ne donnent à l'adversaire aucune occasion de médire.
\VS{15}Car quelques-unes se sont déjà détournées pour suivre Satan.
\VS{16}Si quelque fidèle, homme ou femme, a des veuves, qu'ils les assistent, et que l'église n'en soit pas chargée, afin qu'elle puisse assister celles qui sont véritablement veuves.
\TextTitle{Recommandations concernant les anciens}
\VS{17}Que les anciens qui dirigent\FTNT{Du grec «~proistemi~»~: «~disposer~» ou «~placer devant~», «~diriger~», «~présider~» (1 Th. 5:12~; Ro. 12:8~; 1 Ti. 3:4-5,12.} convenablement soient jugés dignes d'un double honneur, spécialement ceux qui travaillent à la prédication et à l'enseignement.
\VS{18}Car l'Ecriture dit~: Tu n'emmuselleras pas le bœuf quand il foule le grain\FTNT{De. 25:4.}. Et l'ouvrier mérite son salaire\FTNT{Lu. 10:7.}.
\VS{19}Ne reçois pas d'accusation contre un ancien, si ce n'est sur la déposition de deux ou de trois témoins\FTNT{De. 19:15~; Mt. 18:16~; 2 Co. 13:1.}.
\VS{20}Reprends publiquement ceux qui pèchent, afin que les autres aussi en aient de la crainte.
\VS{21}Je te conjure devant Dieu, et devant le Seigneur Jésus-Christ, et devant les anges élus, d'observer ces choses sans préférer l'un à l'autre, et de ne rien faire avec partialité.
\VS{22}N'impose les mains à personne avec précipitation, et ne participe pas aux péchés d'autrui~; toi-même, conserve-toi pur.
\VS{23}Ne bois plus uniquement de l'eau~; mais use d'un peu de vin, à cause de ton estomac et de tes fréquentes maladies.
\VS{24}Les péchés de certains hommes sont manifestes, même avant tout jugement, alors que chez d'autres, ils ne se découvrent qu'après.
\VS{25}De même, les bonnes œuvres sont manifestes, et celles qui ne le sont pas ne peuvent pas rester cachées\FTNT{Mt. 10:26~; Mc. 4:22~; Lu. 8:17~; Lu. 12:2.}.
\Chap{6}
\TextTitle{L'attitude du serviteur envers son maître}
\VerseOne{}Que tous les esclaves qui sont sous le joug sachent qu'ils doivent à leurs maîtres toute sorte d'honneur, afin qu'on ne blasphème pas le Nom de Dieu et sa doctrine.
\VS{2}Et que ceux qui ont des fidèles pour maîtres ne les méprisent pas sous prétexte qu'ils sont leurs frères, mais qu'ils les servent d'autant mieux que ce sont des fidèles et des bien-aimés de Dieu, étant participants de la grâce. Enseigne ces choses et recommande-les.
\VS{3}Si quelqu'un enseigne des fausses doctrines, et ne se soumet pas aux saines paroles de notre Seigneur Jésus-Christ, et à la doctrine qui est selon la piété,
\VS{4}il est enflé d'orgueil, il ne sait rien~; mais il a la maladie des questions et des disputes de mots, d'où naissent l'envie, les querelles, les médisances et les mauvais soupçons,
\VS{5}les vaines disputes d'hommes corrompus d'entendement et privés de la vérité, qui estiment que la piété est un moyen de gagner. Sépare-toi de ces sortes de gens.
\TextTitle{L'amour de l'argent~: La racine de tous les maux}
\VS{6}Or la piété avec le contentement d'esprit est un grand gain.
\VS{7}Car nous n'avons rien apporté dans le monde, et aussi il est évident que nous n'en pouvons rien emporter.
\VS{8}Si nous avons la nourriture et le vêtement, cela nous suffira.
\VS{9}Mais ceux qui veulent devenir riches tombent dans la tentation\FTNT{La tentation se rapporte à l'envie de toujours posséder, de s'enrichir et de gagner plus d'argent. Elle pousse l'homme à l'orgueil, au mensonge, à la duplicité, à la fornication, etc.}, dans le piège\FTNT{Le mot «~piège~» vient du grec «~pagis~» qui donne en français «~trappe~», «~filet~». «~Car il surprendra comme un filet tous ceux qui habitent sur la surface de toute la terre.~» Lu. 21:35. Ce mot suggère l'inattendu, l'improviste, la surprise, car les oiseaux et autres animaux pris dans le filet sont attrapés par surprise. Les conséquences de la cupidité sont nombreuses, notamment le mensonge et l'adultère. En effet, une personne cupide finit en général par tromper son conjoint.}, et dans beaucoup de désirs insensés et pernicieux\FTNT{Les désirs insensés et pernicieux sont multiples~: l'envie de toujours posséder plus que les autres, la convoitise, les rivalités, la concurrence, la folie des grandeurs. Ces choses sortent les gens de la vision du Seigneur (Mc. 4:19).} qui plongent les hommes dans la ruine et la perdition\FTNT{Une personne cupide se perd en s'éloignant du Seigneur (2 Pi. 2). Selon Salomon, l'argent ne rassasie personne (Ec. 5:9). L'enfant de Dieu ne doit pas être cupide et s'appuyer sur l'argent car le système bancaire mondial s'écroulera dans les prochaines années (Ap. 18).}.
\VS{10}Car l'amour de l'argent est la racine de tous les maux\FTNT{L'amour de l'argent est la racine de tous les maux. Ceux qui espèrent en une sécurité divine doivent renoncer à la sécurité matérielle et financière que la chair désire.}~; et quelques-uns en étant possédés, se sont détournés de la foi et se sont jetés eux-mêmes dans bien des tourments.
\VS{11}Mais toi, homme de Dieu, fuis ces choses, et recherche la justice, la piété, la foi, la charité, la patience, la douceur.
\VS{12}Combats le bon combat de la foi, saisis la vie éternelle, à laquelle aussi tu as été appelé, et pour laquelle tu as fait une belle confession en présence de plusieurs témoins.
\VS{13}Je t'ordonne, devant Dieu qui donne la vie à toutes choses, et devant Jésus-Christ qui a fait cette belle confession devant Ponce Pilate,
\VS{14}de garder ce commandement, en te conservant sans tache et irrépréhensible, jusqu'à l'apparition de notre Seigneur Jésus-Christ,
\VS{15}qui sera manifesté en son temps, qui est le Béni et seul Prince, le Roi des rois, et le Seigneur des seigneurs,
\VS{16}qui seul possède l'immortalité, et qui habite une lumière inaccessible, que nul homme n'a vu ni ne peut voir, à qui appartiennent l'honneur et la puissance éternelle. Amen~!
\VS{17}Ordonne à ceux qui sont riches dans ce monde, qu'ils ne soient pas hautains, et qu'ils ne mettent pas leur confiance dans l'incertitude des richesses, mais dans le Dieu vivant, qui nous donne toutes choses abondamment pour en jouir.
\VS{18}Qu'ils fassent du bien, qu'ils soient riches en bonnes œuvres, qu'ils soient prompts à donner, avec libéralité,
\VS{19}s'amassant ainsi pour l'avenir un trésor placé sur un fondement solide, afin qu'ils obtiennent la vie éternelle.
\TextTitle{Conclusion}
\VS{20}Timothée, garde le dépôt, en fuyant les discours vains et profanes, et les contradictions d'une science faussement ainsi nommée,
\VS{21}dont font profession quelques-uns qui se sont détournés de la foi. Que la grâce soit avec toi~! Amen~!
\PPE{}
\end{multicols}
