\ShortTitle{Mi.}\BookTitle{Michée}\BFont
\noindent\hrulefill
{\footnotesize
\textit{
\bigskip
{\centering{}
\\Auteur~: Michée
\\(Heb.~: Miykayehuw)
\\Signification~: Qui est comme Dieu~?
\\Thème~: Le jugement et le royaume
\\Date de rédaction~: 8\up{ème} siècle av. J.-C.\\}
}
\textit{
\\Originaire de Moréscheth, Michée exerça son service dans le royaume du sud au temps d'Ezéchias, roi de Juda. Il était contemporain d'Osée, d'Amos et d'Esaïe. Alors que la corruption et l'idolâtrie régnaient en Samarie et à Jérusalem, Michée appela le peuple à se détourner de ses iniquités et le prévint du danger qui le menaçait. Il prédit également le rétablissement final de la nation juive et mit en exergue la miséricorde divine.\bigskip
}
}
\par\nobreak\noindent\hrulefill
\begin{multicols}{2}
\Chap{1}
\TextTitle{Jugement de Yahweh sur l'infidèle Israël}
\VerseOne{}La Parole de Yahweh qui vint à Michée, de Moréscheth, au temps de Jotham, Achaz, et Ezéchias, rois de Juda, laquelle  lui fut révélée au sujet de Samarie et de Jérusalem.
\VS{2}Vous tous, peuples, écoutez~! Et toi, terre, et tout ce qui est en elle, soyez attentifs~! Et que le Seigneur, Yahweh, soit témoin contre vous, le Seigneur, sortant du palais de sa sainteté.
\VS{3}Car voici, Yahweh sortira de son lieu, il descendra, et marchera sur les hauts lieux de la terre\FTNT{Cette prophétie annonce à la fois la destruction du royaume du nord par Salmanasar V (régna de 727-722 av. J.-C.) en 722 av. J.-C. (2 R. 17:1-23), l'invasion de Sanchérib (régna de 705 à 681 av. J.-C.), (2 R. 18:13 à 2 R. 19:37) ainsi que celle de Nebucadnetsar (régna de 604 à 562 av. J.-C.), (2 R. 24 et 25).}~;
\VS{4}et les montagnes se fondront sous lui, et les vallées se fendront, elles seront comme de la cire devant le feu et comme des eaux qui coulent sur une pente.
\VS{5}Tout cela arrivera à cause du crime de Jacob, et à cause des péchés de la maison d'Israël. Or quel est le crime de Jacob~? N'est-ce pas Samarie~? Et quels sont les hauts lieux de Juda~? N'est-ce pas Jérusalem~?
\TextTitle{Chute de Samarie et de Jérusalem}
\VS{6}C'est pourquoi je réduirai Samarie en un monceau de pierres dans les champs, un lieu où l'on plante des vignes~; et je ferai rouler ses pierres dans la vallée, et je découvrirai ses fondements.
\VS{7}Toutes ses images taillées seront brisées, tous ses salaires de prostitution seront brûlés au feu, et je mettrai tous ses faux dieux en désolation~; parce qu'elle les a entassés par le moyen du salaire de sa prostitution, ils serviront de salaire à une prostituée.
\VS{8}C'est pourquoi je gémirai, et je hurlerai~; je m'en irai dépouillé et nu~; je ferai une lamentation comme celle des dragons, et je mènerai le deuil comme celui des autruches.
\VS{9}Car sa plaie est incurable, elle est venue jusqu'en Juda, et est parvenue jusqu'à la porte de mon peuple, jusqu'à Jérusalem.
\VS{10}Ne l'annoncez point dans Gath, et ne pleurez nullement~! Vautre-toi dans la poussière à Beth-Leaphra.
\VS{11}Passe? habitante de Schaphir, dans la nudité et la honte~! L'habitante de Tsaanan n'est pas sortie~; le deuil de Beth-Haëtsel vous prive de son abri.
\VS{12}L'habitante de Maroth est dans l'angoisse à cause de son bien~; parce que le mal est descendu de par Yahweh sur la porte de Jérusalem.
\VS{13}Attelle le cheval au char, habitante de Lakisch~! Toi qui es le commencement du péché de la fille de Sion~; car en toi ont été trouvés les crimes d'Israël.
\VS{14}C'est pourquoi donne des présents à cause de Moréschet-Gath~; les maisons d'Aczib mentiront aux rois d'Israël.
\VS{15}Je t'amènerai un autre héritier, habitante de Maréscha~; et la gloire d'Israël s'en ira jusqu'à Adullam.
\VS{16}Arrache tes cheveux et fais-toi tondre, à cause de tes fils qui font tes délices~; arrache tout le poil de ton corps, comme un aigle qui mue, car ils sont emmenés prisonniers loin de toi.
\Chap{2}
\TextTitle{Le jugement de Yahweh}
\VerseOne{}Malheur à ceux qui projetent la méchanceté et qui forgent le mal sur leurs lits, et qui l'exécutent dès le point du jour, parce qu'ils ont le pouvoir en main.
\VS{2}Ils convoitent des possessions et s'en emparent, des maisons, et ils les enlèvent~; ainsi ils oppriment l'homme et sa maison, l'homme et son héritage.
\VS{3}C'est pourquoi ainsi parle Yahweh~: Voici, je médite contre cette famille-ci un mal dont vous ne pourrez point préserver votre cou, et vous ne marcherez point la tête levée, car ce temps est mauvais.
\VS{4}En ce temps-là on fera de vous un proverbe lugubre, et l'on gémira d'un gémissement lamentable, en disant~: Nous sommes entièrement détruits~; la part de mon peuple, il la change de mains~! Comment nous enlève-t-il et partage-t-il notre terre à l'infidèle~?
\VS{5}C'est pourquoi il n'y aura personne qui jettera le cordeau pour ton lot, dans l'assemblée de Yahweh.
\VS{6}Ne prophétisez point, disent-ils, ne prophétisez point de telles choses~; l'opprobre ne s'éloignera point.
\VS{7}Or toi qui es appelée maison de Jacob, l'Esprit de Yahweh est-il amoindri~? Sont-ce là ses actes~? Mes paroles ne sont-elles pas bonnes pour celui qui marche droitement~?
\VS{8}Mais celui qui était hier mon peuple, s'élève à la manière d'un ennemi~; vous dépouillez le manteau avec le vêtement à ceux qui passent en assurance, de retour de la guerre.
\VS{9}Vous chassez les femmes de mon peuple hors des maisons de leurs délices~; vous ôtez pour toujours ma gloire de dessus leurs enfants.
\VS{10}Levez-vous et marchez, car ce pays n'est plus pour vous un lieu de repos~; à cause de la souillure, il vous détruira d'une violente destruction.
\VS{11}S'il y a quelque homme qui marche selon le vent et le mensonge, et qui mente en disant~: Je te prophétiserai sur du vin et sur les boissons fortes, ce sera le prophète de ce peuple.
\TextTitle{Yahweh, le Dieu qui rassemble son peuple}
\VS{12}Mais je t'assemblerai tout entier, ô Jacob~! Et je ramasserai entièrement le reste d'Israël, et le mettrai tout ensemble comme les brebis d'une bergerie, comme un troupeau au milieu de son pâturage~; il y aura un grand bruit pour la foule des hommes.
\VS{13}Celui qui fera la brèche montera devant eux, on brisera, et on passera outre, et ils sortiront par la porte~; et leur Roi marchera devant eux, Yahweh sera à leur tête.
\Chap{3}
\TextTitle{Yahweh annonce la destruction de Jérusalem}
\VerseOne{}C'est pourquoi je dis~: Ecoutez, chefs de Jacob, et vous conducteurs de la maison d'Israël~! N'est-ce point à vous de connaître ce qui est juste~?
\VS{2}Ils haïssent le bien, et aiment le mal~; ils ravissent la peau de ces gens-ci de dessus eux, et leur chair de dessus leurs os.
\VS{3} Et ce qu'ils mangent~; c'est la chair de mon peuple, et ils ont écorché leur peau et lui brisent les os~; ils le mettent en pièces comme dans un pot, comme de la viande dans une chaudière.
\VS{4}Alors ils crieront à Yahweh, mais il ne les exaucera point, et il leur cachera sa face en ce temps-là, parce qu'ils se sont mal conduits dans leurs actions
\VS{5}Ainsi parle Yahweh contre les prophètes qui égarent mon peuple, qui proclament la paix quand leurs dents ont de quoi mordre, et si quelqu'un ne leur met rien dans la bouche, ils publient la guerre contre lui.
\VS{6}C'est pourquoi la nuit sera sur vous, afin que vous n'ayez plus de vision~; et elle s'obscurcira, afin que vous ne deviniez plus~; le soleil se couchera sur ces prophètes-là, et le jour leur sera ténébreux.
\VS{7}Les voyants seront honteux, et les devins seront confondus~; tous se couvriront la barbe, parce qu'il n'y aura aucune réponse de Dieu.
\VS{8}Mais moi, je suis rempli de force, de justice et de courage, par l'Esprit de Yahweh, pour déclarer à Jacob son crime, et à Israël son péché.
\TextTitle{Future destruction de Jérusalem}
\VS{9}Ecoutez maintenant ceci, chefs de la maison de Jacob, et vous conducteurs de la maison d'Israël, vous qui avez la justice en abomination, et qui pervertissez tout ce qui est droit,
\VS{10}vous qui bâtissez Sion avec le sang, et Jérusalem avec l'injustice~!
\VS{11}Ses chefs jugent pour des présents, ses prêtres enseignent pour un salaire, et ses prophètes devinent pour de l'argent, ensuite ils s'appuient sur Yahweh, en disant~: Yahweh n'est-il pas parmi nous~? Le mal ne nous atteindra pas.
\VS{12}C'est pourquoi, à cause de vous, Sion sera labourée comme un champ, et Jérusalem sera réduite en ruines, et la montagne du temple en hauts lieux de forêt.
\Chap{4}
\TextTitle{Restauration d'Israël}
\VerseOne{}Mais il arrivera dans les derniers jours\FTNT{Voir commentaire en Ge. 49:1-2.}, que la montagne de la maison de Yahweh\FTNT{Dans les Ecritures, les montagnes symbolisent parfois une grande puissance terrestre, et les collines celles de moindre importance. Cette prophétie confirme l'établissement du Royaume messianique dont la capitale sera Jérusalem (2 S. 7:14-16). Esaïe avait reçu la même prophétie que l'on peut découvrir au chapitre 2 de son livre.} sera affermie au sommet des montagnes, et sera élevée par-dessus les collines~; les peuples y afflueront.
\VS{2}Et plusieurs nations iront et diront~: Venez, et montons à la montagne de Yahweh, à la maison du Dieu de Jacob~; et il nous enseignera sur ses voies, et nous marcherons dans ses sentiers~; car la loi sortira de Sion, et la parole de Yahweh de Jérusalem.
\VS{3}Il exercera le jugement parmi plusieurs peuples, et il réprimandera fortement les grandes nations jusqu'aux pays les plus éloignés~; et de leurs épées, elles forgeront des hoyaux~; et de leurs lances, des serpes~; une nation ne lèvera plus l'épée contre une autre, et elles ne s'adonneront plus à la guerre.
\VS{4}Mais chacun s'assiéra sous sa vigne et sous son figuier, et il n'y aura personne pour les épouvanter~; car la bouche de Yahweh des armées a parlé.
\VS{5}Certainement tous les peuples marchent chacun au nom de leur dieu~; mais nous, nous marchons au Nom de Yahweh, notre Dieu, à toujours et à perpétuité.
\TextTitle{Yahweh, le Dieu qui rachète son peuple}
\VS{6}En ce jour-là, dit Yahweh, je rassemblerai les boiteux, je recueillerai ceux que j'avais chassés et ceux que j'avais maltraités.
\VS{7}Je ferai des boiteux un reste, et de ceux qui étaient éloignés une nation puissante~; Yahweh régnera sur eux, à la montagne de Sion, dès lors et à toujours.
\VS{8}Et toi, tour du troupeau, la colline de la fille de Sion viendra jusqu'à toi, et la première domination viendra~; le royaume, dis-je, sera à la fille de Jérusalem.
\VS{9}Pourquoi maintenant pousses-tu des cris~? N'y a-t-il point de roi au milieu de toi~? Ou ton conseiller est-il mort, que la douleur t'ait saisie comme celle qui enfante~?
\VS{10}Sois en travail et gémis, fille de Sion, comme celle qui enfante~; car tu sortiras bientôt de la ville, tu demeureras aux champs, et tu iras jusqu'à Babylone~; là tu seras délivrée~; là Yahweh te rachètera de la main de tes ennemis.
\TextTitle{Les nations rassemblées pour la bataille d'Harmaguédon}
\VS{11}Maintenant plusieurs nations se sont rassemblées contre toi\FTNT{Il est ici question de la guerre d'Harmaguédon. Voir commentaire en Ap. 16:12-16.} et disent~: Qu'elle soit profanée~! Et que notre œil voie en Sion ce qu'il y voudrait voir\FTNT{Jérusalem est une horloge de Dieu. Le Seigneur a fait en sorte que les nations aient les yeux toujours tournés vers ce bout de terre, car c'est là que débutera la troisième guerre mondiale, l'Harmaguédon (Mt. 24:15-28~; Ap. 16:12-16~; Ap. 19:11-21). Là aura lieu le jugement des nations, dans la vallée de Josaphat (Joë. 3:2-12). Le Messie reviendra (Es. 59:20-21~; Za. 14:1-8~; Ac. 1:10-11) et gouvernera le monde depuis Jérusalem (Za. 14:9-21).}.
\VS{12}Mais ils ne connaissent point les pensées de Yahweh, et ne comprennent pas ses desseins~; car il les a assemblées comme des gerbes dans l'aire.
\VS{13}Lève-toi, et foule, fille de Sion~! Car je te ferai une corne de fer, et te mettrai des ongles d'airain~; et tu écraseras des peuples nombreux, et tu consacreras par interdit leurs profits à Yahweh, et leurs richesses au Seigneur de toute la terre.
\VS{14}Maintenant rassemble tes troupes, fille de troupes~; on a mis le siège contre nous, on frappera le juge d'Israël avec la verge sur la joue.
\Chap{5}
\TextTitle{Naissance du Messie\FTNTT{Cp. Mt. 2:1-6~; 27:24-37.}}
\VerseOne{}Mais toi, Bethléhem Ephrata, petite pour être entre les milliers de Juda, de toi sortira pour moi quelqu'un pour être dominateur en Israël, dont l'origine remonte aux temps anciens, aux jours de l'éternité\FTNT{Il est question ici de Jésus-Christ. Ce passage nous parle de sa préexistence éternelle. Les pharisiens, scribes et principaux prêtres avaient la connaissance de cette prophétie concernant le Messie (Mt. 2:1-6).}.
\VS{2}C'est pourquoi il les livrera jusqu'au temps où enfantera celle qui doit enfanter~; et le reste de ses frères retournera avec les enfants d'Israël.
\VS{3}Et il se maintiendra et gouvernera par la force de Yahweh, avec la magnificence du Nom de Yahweh, son Dieu~; et ils auront une demeure assurée, car dès lors il sera élevé jusqu'aux extrémités de la terre.
\VS{4}C'est lui qui sera la paix. Après que l'Assyrien sera entré dans notre pays, et qu'il aura mis le pied dans nos palais, nous élèverons contre lui sept pasteurs et huit princes du peuple.
\VS{5}Ils ravageront le pays d'Assyrie avec l'épée, et le pays de Nimrod à ses portes. Il nous délivrera ainsi des Assyriens, quand ils seront entrés dans notre pays, et qu'ils auront mis le pied dans nos quartiers.
\VS{6}Et le reste de Jacob sera au milieu de peuples nombreux, comme une rosée qui vient de Yahweh, et comme une pluie menue qui tombe sur l'herbe, qui ne s'attend à aucun homme, et qui n'espère pas des enfants des hommes.
\VS{7}Aussi le reste de Jacob sera parmi les nations, et au milieu de peuples nombreux, comme un lion parmi les bêtes de la forêt, et comme un lionceau parmi les troupeaux de brebis~; qui, en passant, foule et déchire, sans que personne ne puisse les sauver.
\TextTitle{Jugement de Dieu sur les nations ennemies}
\VS{8}Ta main se lèvera sur tes adversaires, et tous tes ennemis seront exterminés.
\VS{9}Et il arrivera en ce temps-là, dit Yahweh, que j'exterminerai du milieu de toi tes chevaux, et ferai périr tes chars.
\VS{10}J'exterminerai les villes de ton pays et renverserai toutes tes forteresses.
\VS{11}J'exterminerai aussi de ta main les sorcelleries, et tu n'auras plus aucun devin.
\VS{12}Et j'exterminerai du milieu de toi tes idoles et tes statues, et tu ne te prosterneras plus devant l'ouvrage de tes mains.
\VS{13}J'arracherai aussi du milieu de toi les Asherah\FTNT{Ex. 34:13.}, et détruirai tes ennemis.
\VS{14}Et j'exercerai ma vengeance avec colère et avec fureur contre toutes les nations qui ne m'auront pas écouté.
\Chap{6}
\TextTitle{Yahweh appelle son peuple à l'humilité}
\VerseOne{}Ecoutez maintenant ce que dit Yahweh~: Lève-toi, plaide devant les montagnes, et que les collines entendent ta voix~!
\VS{2}Ecoutez, montagnes, le procès de Yahweh, vous qui êtes les plus fermes fondements de la terre~! Car Yahweh a un procès avec son peuple, et il plaidera avec Israël.
\VS{3}Mon peuple, que t'ai-je fait, ou en quoi t'ai-je causé de la peine~? Réponds-moi~!
\VS{4}Car je t'ai fait monter hors du pays d'Egypte, je t'ai délivré de la maison de servitude, et j'ai envoyé devant toi Moïse, Aaron et Marie.
\VS{5}Mon peuple, rappelle-toi, je te prie, du conseil que Balak, roi de Moab, avait pris contre toi, et de ce que Balaam, fils de Beor, lui répondit~; et de ce que j'ai fait depuis Sittim jusqu'à Guilgal, afin que tu connaisses la justice de Yahweh.
\TextTitle{Pratiquer la justice}
\VS{6}Avec quoi me présenterai-je devant Yahweh, et me prosternerai-je devant le Dieu Très-Haut~? Me présenterai-je avec des holocaustes, et avec des veaux d'un an~?
\VS{7}Yahweh prendra-t-il plaisir à des milliers de béliers ou à des myriades de torrents d'huile~? Donnerai-je pour mon crime mon premier-né\FTNT{Selon la loi (Ex. 13:2~; Ex. 13:12), les premiers-nés de l'homme et des animaux appartenaient au Seigneur. Ceux des animaux étaient offerts en sacrifice~; et quant à ceux des hommes, le sacrifice était fait de manière symbolique par une présentation au temple. En Israël, le sacrifice des enfants était formellement interdit sous peine de mort (Lé. 18:21~; Lé. 20:2-5~; De. 12:31~; De. 18:10).}, le fruit de mes entrailles pour le péché de mon âme~?
\VS{8}Ô homme~! Il t'a fait connaître ce qui est bon, et ce que Yahweh exige de toi~: Que tu fasses ce qui est juste, que tu aimes la miséricorde, et que tu marches en toute humilité avec ton Dieu.
\VS{9}La voix de Yahweh crie à la ville, et le sage reconnaît son Nom. Ecoutez la verge, et celui qui la dirige~!
\VS{10}Y a-t-il encore dans la maison du méchant des trésors iniques, et un épha court et détestable~?
\VS{11}Tiendrai-je pour pur celui qui a de fausses balances et de faux poids dans son sac~?
\VS{12}Ses riches sont pleins de violence, ses habitants usent de mensonge, et ils ont une langue trompeuse dans leur bouche.
\VS{13}C'est pourquoi je te rendrai languissante en te frappant, et te ravagerai à cause de tes péchés.
\VS{14}Tu mangeras, mais tu ne seras pas rassasiée, et la faim sera au-dedans de toi-même~; tu mettras de côté, mais tu ne sauveras point, et ce que tu auras sauvé je le livrerai à l'épée.
\VS{15}Tu sèmeras, mais tu ne moissonneras point~; tu presseras l'olive, mais tu ne t'oindras point d'huile~; et tu presseras le moût, mais tu ne boiras pas le vin.
\VS{16}Car tu as gardé les ordonnances d'Omri, et toutes les œuvres de la maison d'Achab, et tu as marché dans leurs conseils. C'est pourquoi je te livrerai à la désolation, je ferai de tes habitants un objet de raillerie, et vous porterez l'opprobre de mon peuple.
\Chap{7}
\TextTitle{Exhortation à pratiquer le bien}
\VerseOne{}Malheur à moi~! Car je suis comme quand on a cueilli les fruits d'été et les grappillages de la vendange~: Il n'y a ni grappe pour manger, ni les premiers fruits que mon âme désirait.
\VS{2}Le fidèle est exterminé du pays, et il n'y a plus de juste entre les hommes~; ils sont tous en embûche pour verser le sang, chacun chasse son frère avec des filets.
\VS{3}Pour faire du mal, leurs mains sont prêtes~: Le gouverneur exige, le juge demande un salaire, le grand déclare ce qu'il convoite, et ils s'unissent.
\VS{4}Le meilleur d'entre eux est comme une ronce, et le plus juste est pire qu'une haie d'épines. Le jour annoncé par tes sentinelles, ton châtiment arrive. C'est alors qu'ils seront dans la confusion.
\VS{5}Ne crois pas à ton ami intime, et ne te confie pas en tes conducteurs~; garde-toi d'ouvrir ta bouche devant la femme qui dort dans ton sein.
\VS{6}Car le fils déshonore le père, la fille s'élève contre sa mère, la belle-fille contre sa belle-mère, et chacun a pour ennemis les gens de sa maison.
\TextTitle{Espérance en Yahweh, le Dieu de notre salut}
\VS{7}Mais moi, je regarderai vers Yahweh, je m'attendrai au Dieu de mon salut~; mon Dieu m'exaucera.
\VS{8}Toi, mon ennemie, ne te réjouis pas à mon sujet~; si je suis tombée, je me relèverai~; si j'ai été gisante dans les ténèbres, Yahweh m'éclairera.
\VS{9}Je supporterai la colère de Yahweh, car j'ai péché contre lui, jusqu'à ce qu'il défende ma cause, et qu'il me fasse justice~; il me conduira à la lumière, je verrai sa justice.
\VS{10}Et mon ennemie le verra, et la honte la couvrira~; elle qui me disait~: Où est Yahweh, ton Dieu~? Mes yeux la verront, et alors elle sera foulée aux pieds comme la boue des rues.
\VS{11}Le jour où il rebâtira tes murs, en ce jour-là tes limites seront reculées.
\VS{12}En ce jour-là on viendra jusqu'à toi d'Assyrie et des villes d'Egypte, et depuis les villes d'Egypte jusqu'au fleuve, et depuis une mer jusqu'à l'autre mer, et depuis une montagne jusqu'à l'autre montagne~;
\VS{13}après que le pays aura été en désolation à cause de ses habitants, et du fruit de leurs actions.
\VS{14}Pais ton peuple avec ta houlette, le troupeau de ton héritage, qui demeure seul dans les forêts au milieu de Carmel~! Et fais qu'ils paissent en Basan et en Galaad, comme aux temps anciens.
\VS{15}Je lui ferai voir des choses merveilleuses, comme au jour où tu sortis du pays d'Egypte.
\VS{16}Les nations le verront, et elles seront honteuses avec toute leur force~; elles mettront la main sur la bouche, et leurs oreilles seront sourdes.
\VS{17}Elles lécheront la poussière comme le serpent, comme les reptiles de la terre~; elles trembleront dans leurs forteresses et accourront toutes effrayées vers Yahweh, notre Dieu, et te craindront.
\VS{18}Quel dieu est semblable à toi, qui est le Dieu qui pardonne l'iniquité, et qui passe par-dessus les péchés du reste de son héritage~? Il ne garde pas à toujours sa colère, parce qu'il prend plaisir à la miséricorde.
\VS{19}Il aura encore compassion de nous~; il effacera nos iniquités, et jettera tous nos péchés au fond de la mer.
\VS{20}Tu feras voir ta fidélité à Jacob, et ta miséricorde à Abraham, comme tu l'as juré à nos pères dès les temps anciens\FTNT{Les versets 18 à 20 de Mi. 7 sont lus chaque année dans les synagogues le jour des expiations.}.
\PPE{}
\end{multicols}
