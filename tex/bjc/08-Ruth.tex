\ShortTitle{Ru.}\BookTitle{Ruth}\BFont
\noindent\hrulefill
{\footnotesize
\textit{
\bigskip
{\centering{}
\\Auteur~: Inconnu
\\(Heb.~: Ruwth)
\\Signification~: Amitié, une amie
\\Thème~: Les origines de la famille messianique
\\Date de rédaction~: 11\up{ème} siècle av. J.-C.\\}
}
\textit{
\\Au temps des juges, la famine qui frappa le pays de Juda poussa Elimélec, sa femme Naomi et leurs deux fils à s'installer dans le pays de Moab. Ils y rencontrèrent Ruth qui devint ensuite la belle-fille de Naomi. Après la mort de son mari, cette Moabite montra son attachement à sa belle-mère et au Dieu de celle-ci, qui devint le sien. Sa détermination, sa fidélité, son obéissance et son humilité bouleversèrent sa destinée.
\\Son histoire est l'image du rachat des nations par Jésus-Christ, dont elle fut l'une des ancêtres.\bigskip
}
}
\par\nobreak\noindent\hrulefill
\begin{multicols}{2}
\Chap{1}
\TextTitle{De Juda à Moab}
\VerseOne{}Et il arriva au temps où les juges gouvernaient, qu'il y eut une famine dans le pays~; et un homme de Bethléhem de Juda s'en alla pour séjourner dans la campagne de Moab, lui, sa femme et ses deux fils.
\VS{2}Et le nom de cet homme était Elimélec, le nom de sa femme Naomi et les noms de ses deux fils, Machlon et Kiljon, Ephratiens, de Bethléhem de Juda. Ils arrivèrent dans la campagne de Moab, et s'y établirent.
\VS{3}Or Elimélec, mari de Naomi, mourut~; et elle resta avec ses deux fils,
\VS{4}qui prirent pour eux des femmes Moabites, dont l'une s'appelait Orpa et l'autre Ruth\FTNT{Ruth, la Moabite, dont l'ancêtre était issu d'une relation incestueuse (Ge. 19:36-37), est devenue l'ancêtre du Messie (Mt. 1:5-6).}, et ils demeurèrent là environ dix ans.
\VS{5}Puis ses deux fils Machlon et Kiljon moururent~; ainsi cette femme resta seule, privée de ses deux fils et de son mari.
\TextTitle{Naomi renvoie ses belles-filles dans leur famille}
\VS{6}Puis elle se leva avec ses belles-filles afin de quitter la campagne de Moab, car elle y avait appris que Yahweh avait visité son peuple en lui donnant du pain.
\VS{7}Ainsi elle partit du lieu où elle était, avec ses deux belles-filles, et elles se mirent en chemin pour retourner dans le pays de Juda.
\VS{8}Naomi dit à ses deux belles-filles~: Allez, retournez chacune à la maison de sa mère~! Que Yahweh vous fasse du bien, comme vous en avez fait à ceux qui sont morts et à moi.
\VS{9}Que Yahweh vous fasse trouver du repos à chacune à la maison de son mari. Et elle les baisa~; mais elles élevèrent leur voix et pleurèrent~;
\VS{10}et lui dirent~: Nous retournerons avec toi vers ton peuple.
\TextTitle{Fidélité de Ruth à Naomi}
\VS{11}Et Naomi répondit~: Retournez, mes filles~! Pourquoi viendriez-vous avec moi~? Ai-je encore des fils dans mon sein, afin que vous les ayez pour maris~?
\VS{12}Retournez, mes filles, allez-vous-en~! Je suis trop âgée pour avoir un mari. Et quand je dirais que j'en ai l'espérance, quand même dès cette nuit je serais avec un mari, et que j'enfanterais des fils,
\VS{13}les attendriez-vous jusqu'à ce qu'ils soient grands~? Resteriez-vous pour cela sans être à un mari~? Non, mes filles~! Certes je suis dans une plus grande amertume que vous, parce que la main de Yahweh s'est déployée contre moi.
\VS{14}Alors elles élevèrent leur voix, et pleurèrent encore. Orpa baisa sa belle-mère, mais Ruth s'attacha à elle.
\VS{15}Et Naomi dit à Ruth~: Voici, ta belle-sœur est retournée vers son peuple et vers ses dieux~; retourne après ta belle-sœur.
\VS{16}Mais Ruth répondit~: Ne me prie pas de te laisser, pour m'éloigner de toi, car où tu iras, j'irai~; et où tu demeureras, je demeurerai~; ton peuple sera mon peuple et ton Dieu sera mon Dieu.
\VS{17}Là où tu mourras je mourrai, et j'y serai ensevelie. Qu'ainsi me fasse Yahweh et plus encore~; il n'y aura que la mort qui me séparera de toi~!
\VS{18}Naomi donc, voyant qu'elle était résolue d'aller avec elle, cessa de lui en parler.
\TextTitle{Arrivée à Bethléhem}
\VS{19}Et elles marchèrent toutes deux jusqu'à ce qu'elles arrivèrent à Bethléhem, et comme elles furent entrées dans Bethléhem, toute la ville se mit à parler sur son sujet, et les femmes dirent~: N'est-ce pas ici Naomi~?
\VS{20}Et elle leur répondit~: Ne m'appelez pas Naomi, appelez-moi Mara, car le Tout-Puissant m'a remplie d'amertume.
\VS{21}Je suis partie pleine de biens et Yahweh me ramène à vide. Pourquoi m'appelleriez-vous Naomi\FTNT{Le nom Naomi signifie «~ma gracieuse~».}, puisque Yahweh m'a abattue, et que le Tout-Puissant m'a affligée~?
\VS{22}C'est ainsi que revint Naomi et avec elle, Ruth, sa belle-fille, la Moabite, qui était venue du pays de Moab~; et elles entrèrent dans Bethléhem au commencement de la moisson des orges.
\Chap{2}
\TextTitle{Ruth trouve grâce aux yeux de Boaz}
\VerseOne{}Or le mari de Naomi avait là un parent, un homme puissant et riche, de la famille d'Elimélec, qui s'appelait Boaz.
\VS{2}Et Ruth, la Moabite, dit à Naomi~: Je te prie, laisse-moi aller glaner des épis dans le champ de celui aux yeux duquel je trouverai grâce. Et elle lui répondit~: Va, ma fille.
\VS{3}Elle s'en alla donc, entra dans un champ et glana après les moissonneurs. Et il arriva qu'elle se rencontra dans un champ qui appartenait à Boaz, qui était de la famille d'Elimélec.
\VS{4}Or voici, Boaz vint de Bethléhem, et il dit aux moissonneurs~: Yahweh soit avec vous~! Et ils lui répondirent~: Yahweh te bénisse~!
\VS{5}Et Boaz dit à son serviteur qui était établi sur les moissonneurs~: A qui est cette jeune fille~?
\VS{6}Et le serviteur qui était établi sur les moissonneurs répondit, et dit~: C'est une jeune femme Moabite, qui est venue avec Naomi du pays de Moab.
\VS{7}Et elle nous a dit~: Je vous prie que je glane, et que je ramasse quelques gerbes, après les moissonneurs. Et elle est venue, et elle a été debout depuis le matin jusqu'à cette heure, et ne s'est reposée qu'un moment dans la maison.
\VS{8}Alors Boaz dit à Ruth~: Ecoute, ma fille, ne va pas glaner dans un autre champ, et même, ne pars pas au loin, mais reste ici auprès de mes jeunes filles.
\VS{9}Regarde le champ où l'on moissonne, et va après elles~; n'ai-je pas défendu à mes garçons de te toucher~? Et si tu as soif, tu iras aux vases, et tu boiras de ce que les garçons auront puisé.
\VS{10}Alors elle tomba sur sa face, et se prosterna contre terre, et elle lui dit~: Comment ai-je trouvé grâce à tes yeux, pour que tu prêtes attention à moi, moi qui suis une étrangère~?
\VS{11}Boaz lui répondit, et dit~: On m'a rapporté, on m'a raporté tout ce que tu as fait pour ta belle-mère depuis que ton mari est mort, comment tu as laissé ton père et ta mère, le pays de ta naissance, et tu es venue vers un peuple que tu ne connaissais pas auparavant.
\VS{12}Que Yahweh récompense ton œuvre, et que ton salaire soit entier de la part de Yahweh, le Dieu d'Israël, sous les ailes duquel tu es venue te réfugier~!
\VS{13}Et elle dit~: Mon seigneur, je trouve grâce à tes yeux~; car tu m'as consolée, et tu as parlé selon le cœur de ta servante, bien que je ne sois pas, moi, comme l'une de tes servantes.
\VS{14}Boaz lui dit encore à l'heure du repas~: Approche-toi d'ici, et mange du pain, et trempe ton morceau dans le vinaigre. Et elle s'assit à côté des moissonneurs, et il lui donna du grain rôti, et elle en mangea, et fut rassasiée, et garda le reste.
\VS{15}Puis, elle se leva pour glaner et Boaz ordonna à ses garçons~: Qu'elle glane même entre les gerbes, et ne lui faites pas honte.
\VS{16}Et même vous lui retirerez quelques poignées~; vous les lui laisserez, et elle les recueillera, et vous ne l'en censurerez point.
\VS{17}Elle glana donc dans le champ jusqu'au soir, et elle battit ce qu'elle avait recueilli, et il y eut environ un épha d'orge.
\VS{18}Et elle l'emporta, entra dans la ville, et sa belle-mère vit ce qu'elle avait glané. Elle sortit aussi ce qu'elle avait gardé de reste, après avoir été rassasiée, et elle le lui donna.
\VS{19}Alors sa belle-mère lui dit~: Où as-tu glané aujourd'hui, et où as-tu travaillé~? Béni soit celui qui t'a reconnue~! Et elle raconta à sa belle-mère chez qui elle avait travaillé, et dit~: L'homme chez qui j'ai travaillé aujourd'hui s'appelle Boaz.
\VS{20}Et Naomi dit à sa belle-fille~: Qu'il soit béni de Yahweh, qui n'abandonne pas sa bonté envers les vivants et les morts~! Et Naomi lui dit~: Cet homme est un proche parent et il est un de ceux qui ont sur nous le droit de rachat\FTNT{En hébreu, le verbe «~racheter~» se dit «~ga'al~», ce qui signifie «~racheter, être racheté, venger, se venger, vengeur de sang~». Ce terme est employé pour désigner le fait d'épouser la veuve d'un frère pour lui susciter une descendance (De. 25:5-6), racheter une terre, un bien, un esclave (Lé. 25:24-55), ou encore venger une personne assassinée (No. 35:21). Sous l'Ancienne Alliance, la loi prévoyait qu'un proche parent puisse exercer le droit de rachat dans l'un des cas évoqués afin que justice soit rendue et pour éviter que les biens acquis par une famille ne soient dispersés en dehors du clan familial. Le rédempteur était donc celui qui exerçait le droit de rachat par le paiement d'une rançon. Sous la Nouvelle Alliance, le Seigneur Jésus est le rédempteur suprême qui nous a rachetés de l'esclavage imposé par le diable en donnant sa propre vie en rançon (Ga. 3:13~; Ro. 3:23-24).}.
\VS{21}Et Ruth, la Moabite, dit~: Il m'a même dit~: Reste avec mes garçons jusqu'à ce qu'ils aient achevé toute la moisson qui m'appartient.
\VS{22}Et Naomi dit à Ruth, sa belle-fille~: Ma fille, il est bon que tu sortes avec ses jeunes filles, et qu'on ne te rencontre pas dans un autre champ.
\VS{23}Elle resta donc avec les jeunes filles de Boaz, afin de glaner jusqu'à la fin de la moisson des orges et la moisson des froments, puis elle demeurait avec sa belle-mère.
\Chap{3}
\TextTitle{Ruth obéit soigneusement aux instructions}
\VerseOne{}Et Naomi, sa belle-mère, lui dit~: Ma fille, ne te chercherai-je pas du repos, afin que tu sois heureuse~?
\VS{2}Maintenant donc Boaz, avec les jeunes filles duquel tu as été, n'est-il pas de notre parenté~? Voici, il vanne cette nuit les orges qui ont été foulées dans l'aire.
\VS{3}C'est pourquoi lave-toi et oins-toi, puis mets sur toi tes plus beaux habits, et descends dans l'aire~; mais ne te fais pas connaître à lui jusqu'à ce qu'il ait achevé de manger et de boire.
\VS{4}Et quand il se couchera, remarque le lieu où il se couche; puis entre, découvre ses pieds et couche-toi. Alors il te dira ce que tu auras à faire.
\VS{5}Et elle lui répondit~: Je ferai tout ce que tu as dit.
\VS{6}Elle descendit donc à l'aire, et fit tout ce que sa belle-mère lui avait ordonné.
\VS{7}Et Boaz mangea et but~; et son cœur était joyeux, il vint se coucher à l'extrémité d'un tas de gerbes. Ruth vint secrètement, découvrit ses pieds, et se coucha.
\VS{8}Au milieu de la nuit, cet homme eut peur et se tourna, et voici, une femme était couchée à ses pieds.
\VS{9}Et il lui dit~: Qui es-tu~? Et elle répondit~: Je suis Ruth, ta servante~; étends le pan de ta robe sur ta servante, car tu as le droit de rachat.
\VS{10}Et il dit~: Ma fille, que Yahweh te bénisse~! Ce dernier trait de bonté me réjouit plus que le premier, car tu n'es pas allée après des jeunes gens, pauvres ou riches.
\VS{11}Et maintenant, ma fille, ne crains pas, je te ferai tout ce que tu me diras, car toute la porte de mon peuple sait que tu es une femme vertueuse.
\VS{12}Il est bien vrai que j'ai droit de rachat, mais il existe un autre plus proche que moi, qui a le droit de rachat.
\VS{13}Passe ici la nuit, et demain, s'il veut user envers toi du droit de rachat, à la bonne heure, qu'il te rachète~; mais s'il ne lui plaît pas de te racheter, moi je te rachèterai, Yahweh est vivant~! Couche-toi jusqu'au matin.
\VS{14}Elle se coucha à ses pieds jusqu'au matin, puis elle se leva avant qu'on puisse se reconnaître l'un l'autre~; car il dit~: Qu'on ne sache pas qu'une femme est entrée dans l'aire.
\VS{15}Puis il dit~: Donne-moi le manteau qui est sur toi, et tiens-le. Elle le tint, et il mesura six mesures d'orge, qu'il posa sur elle~; puis il entra dans la ville.
\VS{16}Et elle vint vers sa belle-mère, qui lui dit~: Qui es-tu, ma fille~? Et elle lui raconta tout ce que cet homme avait fait pour elle.
\VS{17}Et elle dit~: Il m'a donné ces six mesures d'orge, car il m'a dit~: Tu ne retourneras pas à vide vers ta belle-mère.
\VS{18}Et Naomi dit~: Ma fille, reste ici, jusqu'à ce que tu saches comment l'affaire se terminera~; car cet homme ne se donnera pas de repos, qu'il n'ait achevé cette affaire aujourd'hui.
\Chap{4}
\TextTitle{Boaz exerce son droit de rachat}
\VerseOne{}Boaz monta donc à la porte, et s'y assit. Et voici, celui qui avait le droit de rachat, et dont Boaz avait parlé, passa. Boaz lui dit~: Ah~! Toi un tel, détourne-toi, et assieds-toi ici. Et il se détourna, et s'assit.
\VS{2}Et Boaz prit dix hommes d'entre les anciens de la ville, et leur dit~: Asseyez-vous ici. Et ils s'assirent.
\VS{3}Puis il dit à celui qui avait le droit de rachat~: Naomi, qui est revenue de la terre de Moab, a vendu la parcelle qui appartenait à notre frère Elimélec.
\VS{4}Et moi, je me suis dit qu'il fallait t'en informer, et te dire~: Acquiers-la en présence de ceux qui sont ici assis, et en présence des anciens de mon peuple. Si tu veux la racheter par droit de rachat, rachète-la~; mais si tu ne veux pas la racheter, déclare-le moi, afin que je le sache~: Car il n'y a pas d'autre que toi qui ait le droit de rachat, et moi je suis après toi. Il répondit~: Je rachèterai.
\VS{5}Et Boaz dit~: Le jour où tu acquerras le champ de la main de Naomi, tu l'acquerras aussi de Ruth, la Moabite, femme du défunt, pour maintenir le nom du défunt dans son héritage.
\VS{6}Et celui qui avait le droit de rachat dit~: Je ne puis pas racheter pour mon compte, de peur de détruire mon héritage~; prends pour toi le droit de rachat, car je ne puis pas le racheter.
\VS{7}Autrefois en Israël, pour confirmer une affaire quelconque relative à un rachat ou à un échange, l'homme ôtait son soulier et le donnait à son parent~; c'était là, en Israël, un témoignage qu'on cédait son droit.
\VS{8}Celui qui avait le droit de rachat dit à Boaz~: Acquiers-le pour toi~! Et il ôta son soulier.
\VS{9}Alors Boaz dit aux anciens et à tout le peuple~: Vous êtes aujourd'hui témoins que j'ai acquis de la main de Naomi tout ce qui appartenait à Elimélec, à Kiljon et à Machlon.
\VS{10}Et que je me suis également acquis pour femme Ruth, la Moabite, femme de Machlon, pour maintenir le nom du défunt dans son héritage, et afin que le nom du défunt ne soit pas retranché d'entre ses frères et de la porte de sa ville. Vous en êtes témoins aujourd'hui~!
\TextTitle{Boaz prend Ruth pour femme}
\VS{11}Et tout le peuple qui était à la porte et les anciens dirent~: Nous en sommes témoins~! Que Yahweh fasse que la femme qui entre dans ta maison, soit comme Rachel et comme Léa, qui toutes les deux ont bâti la maison d'Israël~! Montre ta puissance dans Ephrata, et rends ton nom célèbre dans Bethléhem~!
\VS{12}Et que de la postérité que Yahweh te donnera de cette jeune femme, ta maison soit comme la maison de Pérets, que Tamar enfanta à Juda~!
\VS{13}Ainsi Boaz prit Ruth, et elle fut sa femme~; et il vint vers elle~; et Yahweh lui fit la grâce de concevoir, et elle enfanta un fils.
\VS{14}Et les femmes dirent à Naomi~: Béni soit Yahweh qui ne t'a pas laissé manquer aujourd'hui d'un homme, ayant droit de rachat, et dont le nom sera nommé en Israël~!
\VS{15}Et il restaurera ton âme, et sera le soutien de ta vieillesse~; car ta belle-fille, qui t'aime, a enfanté, et elle te vaut mieux que sept fils.
\VS{16}Alors Naomi prit l'enfant et le mit dans son sein, et elle lui tenait lieu de nourrice.
\TextTitle{Obed, le grand-père de David}
\VS{17}Et les voisines lui donnèrent un nom, en disant~: Un fils est né à Naomi~! Et elles l'appelèrent du nom d'Obed. Ce fut le père d'Isaï, père de David.
\VS{18}Or c'est ici la généalogie de Pérets. Pérets engendra Hetsron~;
\VS{19}Hetsron engendra Ram~; Ram engendra Amminadab~;
\VS{20}Amminadab engendra Nachschon~; Nachschon engendra Salmon~;
\VS{21}Salmon engendra Boaz~; Boaz engendra Obed~;
\VS{22}Obed engendra Isaï, et Isaï engendra David.
\PPE{}
\end{multicols}
