\begin{center}{\LARGE Le Nom de Dieu, le tétragramme YHWH}\end{center}
\begin{small}

Le tétragramme YHWH identifie clairement le nom du Dieu d'Israël. Le texte biblique hébreu laisse apparaître ces quatre lettres qui pour nous, ne sont que de simples consonnes, mais en réalité elles expriment toute la plénitude du Dieu vivant.\bigskip

« \emph{Et il lui dit : Je suis YHWH qui t'ai fait sortir d'Ur en Chaldée, afin de te donner ce pays-ci pour le posséder.} » Genèse 15:7.
Dans ce verset, Dieu se présente par son nom à Abraham. Il utilise le même nom pour se présenter à Moïse dans Ex. 3:14. Finalement, ce nom est utilisé plus de 5000 fois dans le Tanakh pour faire référence au Dieu d'Israël et est traduit par « JE SUIS ».\bigskip

Le nom divin (dans sa forme hébraïque) s'écrit avec les lettres YOD HE WAV HE.\newline
La lettre « yod », peut se prononcer par les sons « ya », « yé », « ye » ou « yi » ; le « he » qui se prononce « hé » peut être muet comme en français et servir à séparer deux sons différents ; la lettre « wav » quant à elle peut se prononcer  « o », « ou », ou encore « é ».\bigskip

Plusieurs siècles avant le début de notre ère, les Israélites s'étaient interdit de prononcer le nom de Dieu. Ils justifiaient cela par leur désir d'obéir au troisième commandement : « \emph{Tu ne prendras point le Nom de YHWH, ton Dieu, en vain ; car YHWH ne tiendra point pour innocent celui qui aura pris son Nom en vain} » (Exode 20:7), mais il s'agissait aussi d'une superstition.\newline
Ils remplaçaient le tétragramme dans le cadre liturgique, par « Adonai » qui signifie « Seigneur » ou par « Elohim » qui veut dire « Dieu », et dans un contexte plus courant, par « Hachem » qui se traduit par « Le nom ». En réalité, ce commandement n'interdisait pas la prononciation du nom de Dieu, mais en soulignait le caractère sacré et insistait sur la révérence qui lui est due.\bigskip

Aussi, plusieurs éléments permettent d'affirmer que le nom de Dieu était connu et se prononçait.\newline
Tout d'abord, on peut noter que des hommes ont prononcé et invoqué le nom de Dieu tout au long du Tanakh. C'est le cas d'Abraham dans Ge. 15:2, d'Isaac dans Ge. 26:25, de Jacob dans Ge. 32:9 et de Moïse dans la Torah. Les prophètes ont également prononcé « le nom », notamment au travers de la formule : « Ainsi parle YHWH… », présente dans les nombreux messages de Dieu à son peuple.\bigskip

Par ailleurs, il est intéressant de rappeler que plusieurs noms propres hébraïques contiennent le tétragramme ou sa forme contractée. A titre d'exemple, Elie de l'hébreu « EliYah » signifie « YHWH est mon Dieu » et Jérémie, de l'hébreu « YirmeYah» signifie « celui que YHWH a désigné ».\bigskip

Enfin, le nom de Dieu devait être prononcé lors de différents rituels et dans des versets d'injonction, comme en témoigne le passage suivant, où Dieu donna à Aaron et ses fils, l'ordre de bénir les enfants d'Israël en son nom : « \emph{YHWH parla à Moïse, en disant : Parle à Aaron et à ses fils, et dis-leur : Vous bénirez ainsi les enfants d'Israël, en leur disant : YHWH te bénisse, et te garde ! YHWH fasse luire sa face sur toi, et te fasse grâce ! YHWH tourne sa face vers toi, et te donne la paix ! Ils mettront donc mon Nom sur les enfants d'Israël, et je les bénirai.} » Nombre 6:22-27.\bigskip

De ~200 à ~1000 après J.-C, les massorètes, des savants juifs « maîtres de la tradition », travaillèrent à la conservation des écrits du Tanakh. Ainsi, ils  fixèrent la prononciation des mots et les sens des textes en ajoutant des voyelles. Ils placèrent sous les quatre consonnes du tétragramme les voyelles incluses dans le nom « Adonaï », indiquant qu'il fallait lire le titre et non pas le nom sacré.\bigskip

Pour s'aligner sur la tradition hébraïque qui utilisait le terme « Adonai », les traductions de la Vulgate et de la Septante ont remplacé, dans le Tanakh, le tétragramme YHWH par « Seigneur ». Aussi, les textes grecs communément appelés « Nouveau Testament », mentionnant des extraits du Tanakh, ont traduit le tétragramme par « Kurios » qui signifie « Seigneur » ou « Theos » qui veut dire « Dieu ».\bigskip

En 1533, encouragé par la Réforme naissante, Pierre Robert Olivétan (1506-1538), érudit et humaniste français, entreprit une traduction de la bible à partir des textes massorétiques pour le Tanakh et des écrits d'Érasme de Rotterdam pour les Evangiles et le Testament de Jésus. Dans cette entreprise, Olivétan choisit de remplacer le tétragramme par « l'Eternel » et le justifia ainsi dans la préface de sa version de 1535 :\newline
« \emph{Désirant montrer la vraie propriété et signification de ce mot \textbf{YHWH (...) je l'ai exprimé} selon son origine, au plus près qu'il m'a été possible \textbf{par le mot Éternel}. Car YHVH vient de \textbf{HWH} qui veut dire « est ». Or, il n'y a que lui qui soit vraiment et qui fasse être toute chose (...) De le nommer comme les Juifs Adonaï c'est-à-dire Seigneur, ce n'est pas remplir et satisfaire à la signification et majesté du mot. Car Adonaï en l'Ecriture est communicable, étant aux hommes comme à Dieu. Mais \textbf{Yahvé} est incommunicable, ne se pouvant approprier et attribuer, sinon qu'à Dieu seul selon son essence.} ».\bigskip

Beaucoup de versions françaises de la Bible ont suivi les travaux d'Olivétan et ont retranscrit le tétragramme par « l'Eternel ».\bigskip

Dans sa vocation d'un retour aux sources, la version de la Bible de Jésus Christ a pris le parti de se rapprocher au plus près des textes originaux tout en rendant leur sens accessible aux lecteurs ; les quatre lettres du nom de Dieu ont été retranscrites dans leur équivalent en alphabet latin : YHWH. Il a donc été décidé d'opter pour la vocalisation la plus largement répandue du tétragramme, à savoir YaHWeH, qui est considérée comme sa prononciation la plus probable. Ainsi, c'est sous la forme « Yahweh » que le nom de Dieu apparait dans cette bible. Le nom divin apparait donc dans les écrits du Tanakh.\bigskip

Lorsque des extraits du Tanakh sont cités dans les évangiles et le Testament de Jésus, le tétragramme figure sous la forme « Seigneur » ou « Dieu » conformément aux textes majoritaires grecs. Aux renvois de versets, des commentaires ont été ajoutés dans cette édition (ex. : Luc 4:18-19) pour permettre au lecteur de faire le lien entre les différents textes et comprendre que de Genèse à Apocalypse, il n'est question que d'un seul message, un seul Dieu, un seul Sauveur.\bigskip

« \emph{Déclarez-le, et faites-les approcher ! Qu'ils prennent conseil ensemble ! Qui a fait entendre ces choses dès l'origine, et les a déclarées dès longtemps ? N'est-ce pas moi, Yahweh ? Or il n'y a point d'autre Dieu à part moi ; un Dieu juste et un Sauveur, il n'y en a pas d'autre à part moi. Vous tous qui êtes aux extrémités de la terre, regardez vers moi, et soyez sauvés ; car JE SUIS Dieu, et il n'y en a point d'autre.} » Esaïe 45:21-22.
\end{small}
