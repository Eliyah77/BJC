\ShortTitle{Joë.}\BookTitle{Joël}\BFont
\noindent\hrulefill
{\footnotesize
\textit{
\bigskip
{\centering{}
\\Auteur~: Joël
\\(Heb.~: Yow'el)
\\Signification~: Yahweh est Dieu
\\Thème~: Le jour de Yahweh
\\Date de rédaction~: 9\up{ème} ou 8\up{ème} siècle av. J.-C.\\}
}
\textit{
\\Joël, fils de Pethuel, exerça son service dans le royaume de Juda. Son message faisait suite à deux fléaux qui s'étaient abattus sur Juda, à savoir une invasion de sauterelles et la sécheresse. Il s'agissait d'un avertissement de Yahweh qui appelait le peuple à revenir à lui avec la promesse de le restaurer dans tout ce qu'il avait perdu. Joël annonça en outre l'effusion de l'Esprit sur toute chair dans un avenir lointain~; prophétie ayant trouvé son accomplissement à la naissance de l'Eglise lors de la Pentecôte.\bigskip
}
}
\par\nobreak\noindent\hrulefill
\begin{multicols}{2}
\Chap{1}
\VerseOne{}La parole de Yahweh qui vint à Joël, fils de Pethuel.
\VS{2}Anciens, écoutez ceci~! Et vous, tous les habitants du pays, prêtez l'oreille~! Est-il arrivé de votre temps, ou même du temps de vos pères, une chose comme celle-ci~?
\VS{3}Faites-en le récit à vos enfants, et que vos enfants le fassent à leurs enfants, et leurs enfants à la génération suivante~!
\TextTitle{Désolation après l'invasion des sauterelles}
\VS{4}La sauterelle a dévoré les restes du gazam, le jélek a dévoré les restes de la sauterelle, et le hasil a dévoré les restes du jélek.
\VS{5}Ivrognes, réveillez-vous, et pleurez~! Et vous tous, buveurs de vin, hurlez à cause du vin nouveau, parce qu'il est retranché de votre bouche~!
\VS{6}Car une nation puissante et innombrable est montée contre mon pays. Elle a les dents d'un lion et les mâchoires d'un vieux lion.
\VS{7}Elle a réduit ma vigne en désert~; et a ôté l'écorce de mes figuiers~; elle les a dépouillés, elle les a dépouillés et les a abattus, leurs branches sont devenues toutes blanches.
\VS{8}Lamente-toi, comme une jeune fille qui se ceint d'un sac pour pleurer le mari de sa jeunesse~!
\VS{9}L'offrande et la libation sont retranchées de la maison de Yahweh, et les prêtres qui font le service de Yahweh mènent le deuil.
\VS{10}Les champs sont ravagés, la terre est dans le deuil~; parce que le blé est détruit, le vin est tari et l'huile est desséchée.
\VS{11}Les laboureurs sont confus, les vignerons gémissent, à cause du froment et de l'orge, car la moisson des champs est perdue.
\VS{12}La vigne est desséchée, le figuier languissant~; le grenadier, le palmier, le pommier et tous les arbres des champs ont séché, c'est pourquoi la joie a cessé parmi les fils de l'homme~!
\VS{13}Prêtres, ceignez-vous et gémissez~! Poussez des cris, vous qui faites le service de l'autel, hurlez, vous qui faites le service de mon Dieu~! Entrez, passez la nuit vêtus de sacs~! Car il est défendu à l'offrande et à la libation d'entrer dans la maison de votre Dieu.
\TextTitle{Désolation après la sécheresse et la famine}
\VS{14}Sanctifiez le jeûne, publiez l'assemblée solennelle, assemblez les anciens, et tous les habitants du pays dans la maison de Yahweh votre Dieu, et criez à Yahweh en disant~:
\VS{15}Hélas~! Quel jour~! Car le jour de Yahweh\FTNT{Jour de Yahweh~: Voir commentaire en Za. 14:1.} est proche~: Il vient comme un ravage fait par le Tout-Puissant.
\VS{16}La nourriture n'est-elle pas retranchée sous nos yeux~? Et la joie et l'allégresse de la maison de notre Dieu~?
\VS{17}Les semences sont pourries sous leurs mottes, les magasins sont dévastés, les greniers sont renversés parce que le blé a manqué.
\VS{18}Ô combien ont gémi les bêtes, et dans quelle peine ont été les troupeaux de bœufs, parce qu'ils n'ont point de pâturage~! Aussi les troupeaux de brebis sont dévastés.
\VS{19}Ô Yahweh, je crierai à toi, car le feu a consumé les pâturages du désert, et la flamme a brûlé tous les arbres des champs.
\VS{20}Même toutes les bêtes des champs crient aussi vers toi~; car les torrents d'eau sont à sec, et le feu a consumé les pâturages du désert.
\Chap{2}
\TextTitle{Le jour de Yahweh, invasion future}
\VerseOne{}Sonnez du shofar en Sion, et sonnez avec un retentissement bruyant dans la montagne de ma sainteté~! Que tous les habitants du pays tremblent~! Car le jour de Yahweh vient~; car il est proche,
\VS{2}jour de ténèbres et d'obscurité, jour de nuées et de brouillards, il vient comme l'aurore s'étend sur les montagnes. Voici un peuple nombreux et puissant, tel qu'il n'y en a jamais eu, et qu'il n'y en aura jamais dans la suite des siècles.
\VS{3}Devant lui est un feu dévorant, et derrière lui une flamme brûle~; devant lui le pays était comme le jardin d'Eden, et derrière lui, la désolation d'un désert~; et rien ne lui échappe.
\VS{4}Leur aspect est comme l'aspect des chevaux, et ils courent comme des cavaliers.
\VS{5}C'est comme le bruit de chariots, quand ils sautent au sommet des montagnes, comme le bruit d'une flamme de feu, qui dévore le chaume, comme un peuple puissant rangé en bataille.
\VS{6}Les peuples tremblent en le voyant~; tous les visages en deviennent pâles et livides.
\VS{7}Ils courent comme des hommes vaillants, et montent sur les murailles comme des gens de guerre~; chacun va son chemin, sans se détourner de son sentier.
\VS{8}Ils ne se pressent point les uns les autres, chacun va son chemin~; ils se jettent au travers des épées sans être blessés.
\VS{9}Ils courent çà et là dans la ville, se précipitent sur les murailles, montent sur les maisons, entrent par les fenêtres comme le voleur.
\VS{10}La terre tremble devant eux, les cieux sont ébranlés, le soleil et la lune s'obscurcissent, et les étoiles retirent leur éclat.
\VS{11}Aussi Yahweh fait entendre sa voix devant son armée~; parce que son camp est très grand, car l'exécuteur de sa parole est puissant. Certainement le jour de Yahweh est grand et terrible. Qui pourra le supporter~?
\TextTitle{Repentance et miséricorde}
\VS{12}Maintenant encore, dit Yahweh, revenez à moi de tout votre cœur, avec des jeûnes, avec des pleurs et des lamentations~!
\VS{13}Déchirez vos cœurs et non vos vêtements, et revenez à Yahweh, votre Dieu~; car il est compatissant et miséricordieux, lent à la colère et riche en bonté, et il se repent d'avoir affligé.
\VS{14}Qui sait si Yahweh, votre Dieu, ne reviendra pas et ne se repentira pas, et s'il ne laissera point après lui la bénédiction, des offrandes et des libations~?
\VS{15}Sonnez du shofar en Sion~! Sanctifiez le jeûne, publiez l'assemblée solennelle~!
\VS{16}Assemblez le peuple, sanctifiez la congrégation~! Réunissez les anciens, assemblez les enfants, même les nourrissons à la mamelle~! Que l'époux sorte de sa demeure, et l'épouse de sa chambre nuptiale~!
\VS{17}Que les prêtres qui font le service de Yahweh pleurent entre le portique et l'autel, et qu'ils disent~: Yahweh~! Epargne ton peuple~! N'expose pas ton héritage à l'opprobre, que les nations n'en fassent pas un sujet de railleries~! Pourquoi dirait-on parmi les peuples~: Où est leur Dieu~?
\TextTitle{Promesse de restauration}
\VS{18}Or Yahweh est jaloux pour son pays, et il est ému de compassion envers son peuple.
\VS{19}Yahweh répond et il dit à son peuple~: Voici, je vous enverrai du blé, du vin, et de l'huile, et vous en serez rassasiés~; et je ne vous exposerai plus à l'opprobre parmi les nations.
\VS{20}J'éloignerai de vous l'armée venue du nord, je la chasserai vers une terre aride et déserte, son avant-garde dans la mer orientale, son arrière-garde dans la Mer Occidentale~; et sa puanteur montera, et son infection s'élèvera, après avoir fait de grandes choses.
\VS{21}Terre, ne crains pas, sois dans l'allégresse et réjouis-toi, car Yahweh fait de grandes choses~!
\VS{22}Ne craignez point, bêtes des champs, car les pâturages du désert ont poussé leur jet, et même les arbres portent leur fruit~; le figuier et la vigne ont poussé avec vigueur.
\VS{23}Et vous, enfants de Sion, soyez dans l'allégresse et réjouissez-vous en Yahweh, votre Dieu, car il vous donnera la pluie selon sa justice, il vous enverra la pluie de la première\FTNT{La pluie de la première saison~: En orient, la première pluie tombe au moment des semailles d'automne. Elle est nécessaire afin que la semence puisse germer. Sous l'influence des pluies fertilisantes, les tendres pousses sortent du sol.} et de l'arrière-saison\FTNT{La pluie de l'arrière-saison~: Elle tombe vers la fin de la saison, mûrit le grain et le prépare pour la moisson. C'est la pluie du printemps. Voir Jé. 5:24~; Os. 6:1-3~; Za. 10:1.}, au premier mois.
\VS{24}Et les aires se rempliront de blé, et les cuves regorgeront de vin et d'huile.
\VS{25}Ainsi je vous rendrai les fruits des années qu'ont dévorés la sauterelle, le jélek, le hasil et le gazam, ma grande armée que j'avais envoyée contre vous.
\VS{26}Vous aurez donc abondamment de quoi manger et être rassasiés, et vous louerez le Nom de Yahweh, votre Dieu, qui aura fait pour vous des choses merveilleuses~; et mon peuple ne sera plus jamais dans la confusion.
\VS{27}Et vous saurez que je suis au milieu d'Israël, que je suis Yahweh, votre Dieu, et qu'il n'y en a point d'autre, et mon peuple ne sera plus jamais dans la confusion.
\TextTitle{La promesse de l'Esprit}
\VS{28}Et il arrivera après cela, que je répandrai mon Esprit sur toute chair\FTNT{Cette promesse s'est réalisée en Actes 2, et elle se réalisera pleinement lors du retour du Messie en Israël (Za. 12:10-14)~; puisque cette prophétie annonce la repentance nationale d'Israël (Ro. 11:26-27).}~; et vos fils et vos filles prophétiseront~; vos vieillards auront des songes, et vos jeunes gens verront des visions.
\VS{29}Et même en ces jours-là, je répandrai mon Esprit sur les serviteurs et sur les servantes.
\TextTitle{Prodiges précédant le jour de Yahweh\FTNTT{Es. 13:9-10~; 24:21-23~; Ez. 32:7-10~; Mt. 24:29-30.}}
\VS{30}Et je ferai des prodiges dans les cieux et sur la terre, du sang et du feu, et des colonnes de fumée~;
\VS{31}le soleil se changera en ténèbres, et la lune en sang, avant que le grand et terrible jour de Yahweh vienne.
\VS{32}Et il arrivera que quiconque invoquera le Nom de Yahweh\FTNT{Cette prophétie fait écho aux propos de Paul en Ro. 10:9-13. Le Nom de Yahweh qui nous a été révélé est Jésus-Christ (Jn. 17:26~; Jn. 10:30~; Ap. 19:13). Ainsi, conformément aux Ecrits de la Nouvelle Alliance, quiconque invoquera le nom du Seigneur Jésus sera sauvé (Ac. 4:12~; Ac. 9:21~; 1 Co. 1:2).} sera sauvé~; car le salut sera sur la montagne de Sion et dans Jérusalem, comme l'a dit Yahweh, et parmi le reste que Yahweh appellera.
\Chap{3}
\TextTitle{Rétablissement d'Israël\FTNTT{Es. 11:10-12~; Jé. 23:5-8~; Ez. 37:21-28~; Ac. 15:15-17.}}
\VerseOne{}Car voici, en ces jours-là, et en ce temps-là, quand je ramènerai les captifs de Juda et de Jérusalem,
\TextTitle{Jugements des nations étrangères\FTNTT{Za. 12:2-3.}}
\VS{2}je rassemblerai toutes les nations\FTNT{Dieu rassemblera les nations dans la vallée de Josaphat (de l'hébreu «~Yehowshaphat~»~: «~Yahweh a jugé~») pour leur jugement. Cette vallée est peut-être celle où le roi Josaphat remporta une grande victoire, avec beaucoup de facilité, sur les Moabites, les Ammonites et les Maonites (2 Ch. 20). Cette vallée s'étend à l'orient de Jérusalem, entre la ville et le Mont des Oliviers, et traverse le torrent de Cédron.}, et je les ferai descendre dans la vallée de Josaphat et là, j'entrerai en jugement avec elles, à cause de mon peuple, d'Israël, mon héritage, lequel ils ont dispersé parmi les nations, et parce qu'ils ont partagé entre eux mon pays~;
\VS{3}et qu'ils ont tiré mon peuple au sort~; ils ont donné l'enfant pour une prostituée, ils ont vendu la jeune fille pour du vin, et ils ont bu.
\VS{4}Et qu'ai-je aussi affaire de vous, Tyr et Sidon, et de vous, toutes les limites de la Palestine~? Me rendrez-vous ma récompense, ou voulez-vous m'irriter~? Je vous rendrai promptement et sans délai votre récompense sur votre tête.
\VS{5}Car vous avez pris mon argent et mon or~; et vous avez emporté dans vos temples mes belles choses désirables.
\VS{6}Vous avez vendu les enfants de Juda et de Jérusalem aux enfants des Grecs, afin de les éloigner de leur territoire.
\VS{7}Voici, je les ferai lever\FTNT{Le verbe «~lever~» vient de l'hébreu «~'uwr~» qui signifie «~se réveiller~», «~éveiller~», «~être éveillé~», «~inciter~», «~veiller~», «~se lever~», «~sortir de l'assoupissement~», «~prendre courage~». Yahweh annonce le réveil des hébreux depuis les nations, d'où ils sont établis. Ce réveil est une prise de conscience qui aboutira au retour à la terre sainte.} du lieu où ils ont été transportés après que vous les avez vendus~; et je ferai retourner votre récompense sur votre tête.
\VS{8}Je vendrai donc vos fils et vos filles entre les mains des enfants de Juda, et ils les vendront à ceux de Séba, qui les transporteront vers une nation éloignée~; car Yahweh a parlé.
\VS{9}Publiez ceci parmi les nations~! Préparez la guerre~! Réveillez les hommes vaillants~! Qu'ils s'approchent, et qu'ils montent, tous les hommes de guerre~!
\VS{10}Forgez des épées de vos hoyaux, et des lances de vos serpes~! Et que le faible dise~: Je suis fort~!
\VS{11}Hâtez-vous et venez, vous toutes les nations d'alentour, et rassemblez-vous~! Là, ô Yahweh, fais descendre tes hommes vaillants~!
\VS{12}Que les nations se réveillent, et qu'elles montent à la vallée de Josaphat~! Car là je siégerai pour juger toutes les nations d'alentour.
\VS{13}Saisissez la faucille, car la moisson est mûre~! Venez, et descendez, car le pressoir est plein, les cuves regorgent~! Car leur méchanceté est grande,
\VS{14}des multitudes, des multitudes, dans la vallée du jugement~; car le jour de Yahweh est proche, dans la vallée du jugement.
\VS{15}Le soleil et la lune s'obscurcissent, et les étoiles retirent leur éclat.
\VS{16}De Sion Yahweh rugit, de Jérusalem il fait entendre sa voix~; les cieux et la terre sont ébranlés. Mais Yahweh est le refuge pour son peuple, et la forteresse\FTNT{Jésus-Christ est notre rocher (commentaire Es. 8:14~; Ps. 78:35~; 1 Co. 10:4).} pour les enfants d'Israël.
\VS{17}Et vous saurez que je suis Yahweh, votre Dieu, qui habite à Sion, ma sainte montagne. Jérusalem sera sainte, et les étrangers n'y passeront plus.
\TextTitle{Restauration finale et pleine bénédiction du royaume}
\VS{18}Et il arrivera en ce jour-là, que le vin ruissellera des montagnes, le lait coulera des collines, il y aura de l'eau dans tous les torrents de Juda~; et une source\FTNT{Jésus est celui qui fait jaillir en nous une source d'eau qui étanche notre soif à jamais et nous donne la vie éternelle (Jé. 2:13~; Jé. 17:13~; Ez. 47:1-12~; Za. 14:8~; Jn. 4:14~; Ap. 22:1).} sortira de la maison de Yahweh, et arrosera la vallée de Sittim.
\VS{19}L'Egypte sera dévastée, Edom sera réduit en désert de désolation, à cause de la violence faite aux enfants de Juda, dont ils ont répandu le sang innocent dans leur pays.
\VS{20}Mais là, Juda sera éternellement habitée, et Jérusalem, d'âge en âge.
\VS{21}Et je nettoierai leur sang que je n'avais point nettoyé~; car Yahweh habite en Sion.
\PPE{}
\end{multicols}
