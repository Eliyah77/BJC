\ShortTitle{Am.}\BookTitle{Amos}\BFont
\noindent\hrulefill
{\footnotesize
\textit{
\bigskip
{\centering{}
\\Auteur~: Amos
\\(Heb.~: 'Amowc)
\\Signification~: Fardeau, porteur de fardeau
\\Thème~: Jugement sur le péché
\\Date de rédaction~: 8\up{ème} siècle av. J.-C.\\}
}
\textit{
\\Originaire de Tekoa, Amos exerça son service dans le royaume du nord, au temps d'Ozias, roi de Juda et de Jéroboam II, roi d'Israël. Il fut aussi le contemporain des prophètes Osée, Michée, Jonas et Esaïe.
\\Alors que le peuple juif jouissait d'une certaine prospérité, l'immoralité et les sacrilèges prirent place dans le royaume. Amos avertit le peuple de son péché et du jugement qu'il encourait. Il lui rappela la bonté de Dieu et l'invita à revenir à Yahweh et à lui rester fidèle.\bigskip
}
}
\par\nobreak\noindent\hrulefill
\begin{multicols}{2}
\Chap{1}
\VerseOne{}Les paroles d'Amos qui était parmi les bergers de Tekoa, visions qu'il eut concernant Israël, aux jours d'Ozias, roi de Juda, et de Jéroboam, fils de Joas, roi d'Israël, deux ans avant le tremblement de terre\FTNT{Za. 14:5.}.
\VS{2}Il dit donc~: Yahweh rugit de Sion, et fait entendre sa voix de Jérusalem, et les pâturages des bergers se lamentent, et le sommet du Carmel est desséché\FTNT{Jé. 25:30~; Joë 3:16.}.
\TextTitle{Yahweh annonce ses jugements sur les villes et les pays d'alentour}
\VS{3}Ainsi parle Yahweh~: A cause de trois crimes de Damas, et même de quatre, je ne rappellerai point cela, mais je le ferai\FTNT{Il est question du jugement de Dieu.}, parce qu'ils ont foulé Galaad avec des herses de fer\FTNT{Es. 17:1.}.
\VS{4}Et j'enverrai le feu dans la maison de Hazaël, et il dévorera le palais de Ben-Hadad.
\VS{5}Je briserai aussi les verrous de Damas, j'exterminerai de Bikath-Aven ses habitants, et de Beth-Eden celui qui tient le sceptre~; et le peuple de Syrie sera mené captif à Kir, dit Yahweh.
\VS{6}Ainsi parle Yahweh~: A cause de trois crimes de Gaza, et même de quatre, je ne rappellerai point cela, mais je le ferai\FTNT{Voir commentaire en Am. 1:3.} parce qu'ils ont emmené des captifs en grand nombre pour les livrer à Edom\FTNT{Ez. 25:13-17.}.
\VS{7}Et j'enverrai le feu dans les murailles de Gaza, et il dévorera ses palais.
\VS{8}Et j'exterminerai d'Asdod les habitants, et d'Askalon celui qui tient le sceptre~; je tournerai ma main contre Ekron, et le reste des Philistins périra, dit le Seigneur, Yahweh.
\VS{9}Ainsi parle Yahweh~: A cause de trois crimes de Tyr, et même de quatre, je ne rappellerai point cela, mais je le ferai\FTNT{Voir commentaire en Am. 1:3.}, parce qu'ils ont livré à Edom beaucoup d'exilés et ne se sont pas souvenus de l'alliance fraternelle\FTNT{Ez. 26:2.}.
\VS{10}Et j'enverrai le feu dans les murailles de Tyr, et il dévorera ses palais.
\VS{11}Ainsi parle Yahweh~: A cause de trois crimes d'Edom, et même de quatre, je ne rappellerai point cela, mais je le ferai\FTNT{Voir commentaire en Am. 1:3.}, parce qu'il a poursuivi son frère avec l'épée, et qu'il a altéré ses compassions, parce que sa colère déchire continuellement et qu'il garde sa fureur éternellement.
\VS{12}Et j'enverrai le feu dans Théman, et il dévorera les palais de Botsra\FTNT{Jé. 49:7~; Ab. 1:9.}.
\VS{13}Ainsi parle Yahweh~: A cause de trois crimes des enfants d'Ammon, et même de quatre, je ne rappellerai point cela, mais je le ferai\FTNT{Voir commentaire en Am. 1:3.}, parce qu'ils ont fendu le ventre des femmes enceintes de Galaad pour élargir leurs frontières\FTNT{Ez. 21:33~; So. 2:8.}.
\VS{14}Et j'allumerai le feu avec des cris de guerre, au jour de la bataille avec tourbillon au jour de la tempête, dans les murailles de Rabba, et il dévorera les palais.
\VS{15}Et leur roi ira en captivité, et avec lui ses chefs de son pays, dit Yahweh.
\Chap{2}
\TextTitle{Suite des jugements prononcés sur les villes et les pays d'alentour}
\VerseOne{}Ainsi parle Yahweh~: A cause de trois crimes de Moab, et même de quatre, je ne rappellerai point cela, mais je le ferai, parce qu'il a brûlé les os du roi d'Edom jusqu'à les calciner.
\VS{2}Et j'enverrai le feu dans Moab, et il dévorera les palais de Kerijoth~; et Moab mourra dans le tumulte, au milieu des cris de guerre et du bruit du shofar\FTNT{Ez. 25:8-9.}.
\VS{3}Et j'exterminerai les gouverneurs du milieu de son pays, et je tuerai ensemble avec lui tous les chefs du pays, dit Yahweh.
\TextTitle{Juda et Israël jugés à cause de leurs iniquités}
\VS{4}Ainsi parle Yahweh~: A cause de trois crimes de Juda, et même de quatre, je ne rappellerai point cela, mais je le ferai, parce qu'ils ont rejeté la loi de Yahweh et n'ont point gardé ses ordonnances~; mais leurs mensonges que leurs pères avaient suivis les ont fait égarer.
\VS{5}Et j'enverrai le feu dans Juda, et il dévorera les palais de Jérusalem.
\VS{6}Ainsi parle Yahweh~: A cause de trois crimes d'Israël, et même de quatre, je ne rappellerai point cela, mais je le ferai, parce qu'ils ont vendu le juste pour de l'argent, et le pauvre pour une paire de souliers.
\VS{7}Soupirant après la poussière de la terre pour la jeter sur la tête des faibles~; et ils pervertissent la voie des pauvres. Le fils et le père vont vers la même jeune fille, pour profaner mon saint Nom.
\VS{8}Et ils se couchent près de chaque autel, sur les vêtements qu'ils ont pris en gage, et boivent dans la maison de leurs dieux le vin des condamnés.
\VS{9}J'ai pourtant détruit devant eux les Amoréens qui étaient hauts comme les cèdres et forts comme les chênes~; et j'ai ruiné son fruit par-dessus et ses racines par-dessous\FTNT{No. 21:24~; Jos. 24:8.}.
\VS{10}Je vous ai fait monter du pays d'Egypte et je vous ai conduits dans le désert quarante ans pour que vous possédiez le pays des Amoréens.
\VS{11}J'ai suscité davantage des prophètes parmi vos fils et des nazaréens\FTNT{Le mot «~nazaréen~» vient de l'hébreu «~nâzîr~», de la racine «~nâzar~» qui signifie «~séparer~». Il y avait deux types de nazaréens. Premièrement ceux qui étaient appelés par Dieu. Par exemple~: Samson (Jg. 13:1-7)~; Samuel (1 S. 1:11)~; Jean-Baptiste (Lu. 1:15). Deuxièmement les personnes qui voulaient se consacrer à Dieu (No. 6:13).} parmi vos jeunes gens. N'en est-il pas ainsi, enfants d'Israël~? dit Yahweh.
\VS{12}Mais vous avez fait boire du vin aux nazaréens, et vous avez donné cet ordre aux prophètes leur disant~: Ne prophétisez plus\FTNT{Es. 30:10~; Jé. 11:21~; Mi. 2:6.}~!
\VS{13}Voici, je m'en vais fouler le lieu où vous habitez, comme un chariot plein de gerbes foule tout par où il passe.
\VS{14}Tellement que l'homme agile ne pourra pas fuir, et le fort ne pourra pas faire usage de sa vigueur, et le vaillant ne sauvera pas sa vie\FTNT{Jé. 46:6.}.
\VS{15}Et celui qui manie l'arc ne pourra pas tenir ferme, et celui qui a les pieds légers n'échappera pas, et le cavalier ne sauvera pas sa vie.
\VS{16}Et le plus courageux d'entre les hommes vaillants s'enfuira tout nu en ce jour-là, dit Yahweh.
\Chap{3}
\TextTitle{La maison de Jacob coupable devant Yahweh}
\VerseOne{}Enfants d'Israël écoutez la parole que Yahweh a prononcée contre vous, contre toutes les familles, dis-je, que j'ai fait monter du pays d'Egypte.
\VS{2}Je vous ai connus, vous seuls parmi toutes les familles de la terre~; c'est pourquoi je vous châtierai pour toutes vos iniquités\FTNT{Ex. 19:5-6~; Ps. 147:19-20.}.
\VS{3}Deux hommes marchent-ils ensemble s'ils ne sont pas accordés~?
\VS{4}Le lion rugit-il dans la forêt sans qu'il n'ait de proie~? Le lionceau fait-il retentir son cri de sa tanière s'il n'a pas pris quelque chose~?
\VS{5}L'oiseau tombe-t-il dans le filet qui est à terre sans qu'il lui ait tendu des pièges~? Lève-t-on le filet de dessus la terre sans avoir rien pris du tout~?
\VS{6}Le shofar sonne-t-il dans une ville sans que le peuple ait peur~? Arrive-t-il un mal dans une ville sans que Yahweh ne l'ait ordonné\FTNT{Es. 45:7~; La. 3:37-38.}~?
\VS{7}Car le Seigneur Yahweh ne fait rien sans avoir révélé son secret à ses serviteurs les prophètes.
\VS{8}Le lion rugit, qui ne craindrait~? Le Seigneur Yahweh parle, qui ne prophétiserait\FTNT{Lorsque Yahweh parle, des prophètes sont suscités~: Jé. 20:9~; Mi. 3:8~; Ac. 4:20.}~?
\VS{9}Faites entendre ceci dans les palais d'Asdod, et dans les palais du pays d'Egypte, et dites~: Assemblez-vous sur les montagnes de Samarie, et regardez les grands désordres au milieu d'elle et ceux auxquels on a fait tort dans son sein~!
\VS{10}Car ils ne savent pas faire ce qui est juste, dit Yahweh, ils amassent la violence et la rapine dans leurs palais.
\VS{11}C'est pourquoi ainsi parle le Seigneur, Yahweh~: L'ennemi viendra, il cernera le pays, il t'ôtera ta force et tes palais seront pillés.
\VS{12}Ainsi parle Yahweh~: Comme un berger sauvait de la gueule d'un lion les deux jambes ou le bout d'une oreille, ainsi les enfants d'Israël qui habitent dans Samarie seront arrachés de l'angle d'un lit et de l'asile de Damas.
\VS{13}Ecoutez et protestez contre la maison de Jacob, dit le Seigneur, Yahweh, le Dieu des armées~:
\VS{14}Le jour où je punirai Israël pour ses péchés, j'exercerai aussi mon châtiment sur les autels de Béthel~; et les cornes de l'autel seront retranchées et tomberont à terre.
\VS{15}Et je frapperai la maison d'hiver et la maison d'été~; et les maisons d'ivoire seront détruites et les grandes maisons prendront fin, dit Yahweh.
\Chap{4}
\TextTitle{Yahweh condamne les sacrifices du peuple}
\VerseOne{}Ecoutez cette parole-ci, vaches de Basan, qui vous tenez sur la montagne de Samarie, qui faites tort aux faibles, et qui opprimez les pauvres, qui dites à leurs maîtres~: Apportez, et que nous buvions~!
\VS{2}Le Seigneur, Yahweh, l'a juré par sa sainteté que voici, les jours viennent sur vous, où l'on vous enlèvera avec des hameçons, et votre postérité avec des crochets de pêche\FTNT{Jé. 16:16~; Ha. 1:14-16.}.
\VS{3}Et vous sortirez dehors par les brèches, chacune devant soi, et vous jetterez là ce que vous avez amassé dans les palais, dit Yahweh.
\VS{4}Entrez dans Béthel, et péchez~! Commettez-y vos crimes~; multipliez vos péchés dans Guilgal, et amenez vos sacrifices dès le matin, et vos dîmes au bout de trois ans accomplis\FTNT{Voir commentaires en No. 18:21 et Mal. 3:10.}~!
\VS{5}Et brûlez de l'encens avec du pain levé pour l'offrande de remerciement~; proclamez et publiez les offrandes volontaires~! Car c'est là ce que vous aimez, enfants d'Israël, dit le Seigneur, Yahweh\FTNT{Lé. 2:1.}.
\TextTitle{Endurcissement du peuple malgré les châtiments de Yahweh}
\VS{6}C'est pourquoi, je vous ai envoyé la famine dans toutes vos villes, et la disette de pain dans toutes vos demeures~. Mais malgré cela, vous n'êtes pas revenus vers moi, dit Yahweh.
\VS{7}Je vous ai aussi privés de pluie, quand il restait encore trois mois jusqu'à la moisson~; j'ai fait pleuvoir sur une ville et je n'ai pas fait pleuvoir sur une autre ville~; une parcelle a été arrosée par la pluie, et l'autre parcelle, sur laquelle il n'a pas plu, est desséchée\FTNT{1 R. 8:35~; 1 R. 17:1~; Es. 5:6~; Ag. 1:11.}.
\VS{8}Et deux, même trois villes sont allées vers une autre ville pour boire de l'eau et n'ont pas été désaltérées, mais vous n'êtes pas revenus vers moi, dit Yahweh.
\VS{9}Je vous ai frappés par la brûlure et par la rouille, et la sauterelle a brouté autant de jardins et de vignes, de figuiers et d'oliviers que vous aviez, mais vous n'êtes pas revenus vers moi, dit Yahweh\FTNT{De. 28:22-39~; 1 R. 8:37~; Ag. 2:17~; 2 Ch. 6:28.}.
\VS{10}J'ai envoyé parmi vous la peste de la même manière que je l'avais envoyée en Egypte~; et j'ai fait mourir par l'épée vos jeunes hommes et vos chevaux en captivité~; j'ai fait remonter, jusque dans votre nez, la puanteur de vos camps~; mais vous n'êtes pas revenus vers moi, dit Yahweh\FTNT{Ez. 14:19.}.
\VS{11}Je vous ai renversés de la même manière que Dieu renversa Sodome et Gomorrhe, et vous avez été comme un tison arraché du feu, mais vous n'êtes pas revenus vers moi, dit Yahweh\FTNT{Ge. 19:24~; Jé. 49:18~; Za. 3:2.}.
\VS{12}C'est pourquoi je te traiterai de la même manière, ô Israël~; et parce que je te traiterai ainsi, prépare-toi à la rencontre de ton Dieu, ô Israël~!
\VS{13}Car voici celui qui a formé les montagnes et créé le vent, et qui déclare à l'homme quelle est sa pensée, qui fait l'aube et l'obscurité, et qui marche sur les hauteurs de la terre~; YAHWEH, LE DIEU DES ARMÉES est son nom.
\Chap{5}
\TextTitle{Israël invité à revenir entièrement à Yahweh}
\VerseOne{}Ecoutez cette parole, qui est la complainte que je prononce à haute voix sur vous, ô maison d'Israël~!
\VS{2}Elle est tombée, elle ne se relèvera plus, la vierge d'Israël~; elle est abandonnée sur la terre, et il n'y a personne qui la relève.
\VS{3}Car ainsi parle le Seigneur, Yahweh à la maison d'Israël~: La ville de laquelle il en sortait par millier, en aura cent de reste~; et celle de laquelle il en sortait cent, en aura dix de reste.
\VS{4}Car ainsi a parlé Yahweh à la maison d'Israël~: Cherchez-moi, et vous vivrez~!
\VS{5}Ne cherchez pas Béthel, et n'allez pas à Guilgal, et ne passez point à Beer-Schéba. Car Guilgal sera transportée en captivité, et Béthel sera détruite\FTNT{Os. 4:15.}.
\VS{6}Cherchez Yahweh, et vous vivrez, de peur qu'il ne saisisse comme un feu la maison de Joseph, et que ce feu ne la consume, sans qu'il y ait personne à Béthel pour l'éteindre.
\VS{7}Parce qu'ils changent le jugement en absinthe, et qu'ils renversent la justice\FTNT{Es. 5:26-28~; Ha. 1:1-3.}.
\VS{8}Cherchez celui qui a fait les Pléiades et l'Orion, qui change les profonds ténèbres en aube du matin, et qui obscurcit le jour en nuit, qui appelle les eaux de la mer, et les répand sur la surface de la Terre, Yahweh est son nom\FTNT{Es. 58:8-10~; Job 9:9~; Job 38:31.}~;
\VS{9}qui renforce le ravage contre le fort, tellement que le ravage entrera dans la forteresse.
\VS{10}Ils haïssent à la porte ceux qui les reprennent, et ils ont en abomination celui qui parle en intégrité.
\VS{11}C'est pourquoi, puisque vous opprimez le pauvre et lui enlevez la charge de froment, vous avez bâti des maisons en pierres de taille, mais vous n'y habiterez pas~; vous avez planté des vignes délicieuses, mais vous n'en boirez pas le vin.
\VS{12}Car j'ai connu vos crimes, ils sont en grand nombre, et vos péchés se sont multipliés~; vous êtes des oppresseurs du juste, et des preneurs de rançon, et vous pervertissez à la porte le droit des pauvres.
\VS{13}C'est pourquoi l'homme prudent se tient dans le silence en ce temps-ci, car les temps sont mauvais.
\VS{14}Recherchez le bien et non le mal, afin que vous viviez~; et qu'ainsi Yahweh, le Dieu des armées, soit avec vous, comme vous l'avez dit.
\VS{15}Haïssez le mal, et aimez le bien, et établissez la justice à la porte~; peut-être Yahweh, le Dieu des armées, aura pitié des restes de Joseph.
\TextTitle{Le jour de Yahweh}
\VS{16}C'est pourquoi ainsi parle Yahweh, le Dieu des armées, le Seigneur, parle ainsi~: Dans toutes les places on se lamentera, dans toutes les rues on dira~: Hélas~! Hélas~! On appellera au deuil le laboureur, et à la lamentation ceux qui en savent le métier.
\VS{17}Et il y aura des lamentations dans les vignes, car je passerai au milieu de toi, dit Yahweh.
\VS{18}Malheur à vous qui désirez le jour de Yahweh\FTNT{L'expression «~le jour du Seigneur~» ou «~le jour de Yahweh~» est une période durant laquelle Jésus-Christ interviendra ouvertement dans les affaires des hommes. Elle est utilisée dix-neuf fois dans le Tanakh (Es. 2:12~; Es. 13:6-9~; Ez. 13:5~; Ez. 30:3~; Joë. 1:15~; Joë. 2:1,11,31~; Joë. 3:14~; Am. 5:18, 20~; Ab. 1:15~; So. 1:7, 14~; Za. 14:1~; Mal. 4:5) et quatre fois dans le Testament de Jésus (Ac. 2:20~; 2 Th. 2:2~; 2 Pi. 3:10). On y fait également allusion dans d'autres passages (Ap. 6:17~; Ap. 16:14).}. A quoi vous servira le jour de Yahweh~? Il sera ténèbres et non lumière.
\VS{19}C'est comme si un homme s'enfuit devant un lion, et qu'un ours le rencontre, ou qui entre dans sa maison, appuie sa main sur le mur et un serpent le mord.
\TextTitle{Mépris du droit et de la justice}
\VS{20}Le jour de Yahweh n'est-il pas ténèbres et non lumière~? Et l'obscurité n'est-elle point en lui, et non la clarté~?
\VS{21}Je hais et rejette vos fêtes solennelles, et je ne flairerai pas l'odeur de vos parfums dans vos assemblées solennelles.
\VS{22}Même si vous m'offrez des holocaustes et des gâteaux, je ne les accepterai point~; et je ne regarderai pas les bêtes grasses de vos offrandes de paix\FTNT{Voir commentaire en Lé. 3:1.}.
\VS{23}Ôte de devant moi le bruit de tes chansons~; je n'écouterai pas la mélodie de tes luths.
\VS{24}Mais que le jugement coule comme de l'eau, et la justice comme un torrent intarissable.
\VS{25}Est-ce à moi, maison d'Israël, que vous avez offert des sacrifices et des gâteaux dans le désert pendant quarante ans~?
\VS{26}Au contraire, vous avez porté la tente de votre Moloc, et de vos idoles Kijun\FTNT{«~Kijun~»~ est une divinité de l'antiquité ainsi que l'un des nombreux nom donnés au soleil par les civilisations antiques}, l'étoile de votre dieu que vous vous êtes fabriquée.
\VS{27}C'est pourquoi je vous transporterai au-delà de Damas, dit Yahweh, dont le nom est le Dieu des armées.
\Chap{6}
\TextTitle{Ceux qui prospèrent seront emmenés captifs}
\VerseOne{}Hélas, vous qui êtes à votre aise en Sion, et qui vous confiez en la montagne de Samarie, lieux les plus renommés d'entre les principaux des nations, auprès desquels va la maison d'Israël.
\VS{2}Passez à Calné, et regardez~; allez de là à Hamath la grande, puis descendez à Gath chez les Philistins. Ces villes sont-elles plus prospères que vos deux royaumes, ou leur pays n'est-il pas plus étendu que votre pays~?
\VS{3}Vous qui éloignez le jour du malheur, et qui approchez le règne de la violence.
\VS{4}Vous qui vous couchez sur des lits d'ivoire, et qui êtes étendus sur vos coussins~; qui mangez les agneaux du troupeau, et les veaux pris du lieu où on les engraisse~;
\VS{5}qui fredonnez au son du luth~; qui inventez des instruments de musique comme David,
\VS{6}qui buvez le vin dans de grandes coupes, qui parfumez des parfums les plus exquis, et qui n'êtes pas affligés pour la plaie de Joseph.
\VS{7}A cause de cela, ils s'en iront incessamment en captivité parmi les premiers qui s'en iront en captivité~; et le luxe de ces personnes voluptueuses prendra fin.
\VS{8}Le Seigneur, Yahweh, l'a juré par lui-même. Yahweh, Dieu des armées, dit~: Je déteste l'orgueil de Jacob, et je hais ses palais, c'est pourquoi je livrerai la ville, et tout ce qui est en elle.
\VS{9}Et s'il arrive qu'il reste dix hommes dans une maison, ils mourront.
\VS{10}Et l'oncle de chacun d'eux les prendra, et les brûlera et enlèvera les os hors de la maison, et il dira à celui qui est au fond de la maison~: Y a-t-il encore quelqu'un avec toi~? Et il répondra~: Non. Puis son oncle lui dira~: Tais-toi, car nous ne pouvons faire mention du nom de Yahweh.
\VS{11}Car voici, Yahweh ordonne~: Et il frappera les grandes maisons par des débordements d'eau, et la petite maison en débris.
\VS{12}Les chevaux courent-ils sur les rochers~? Y laboure-t-on avec des bœufs, pour que vous ayez changé la droiture en poison, et le fruit de la justice en absinthe~?
\VS{13}Vous vous réjouissez de choses qui ne sont que néant, vous dites~: N'est-ce pas par notre force que nous avons acquis de la puissance~?
\VS{14}Mais, j'élèverai contre vous, ô maison d'Israël, dit Yahweh, le Dieu des armées, une nation qui vous opprimera depuis l'entrée de Hamath jusqu'au torrent du désert.
\Chap{7}
\TextTitle{Avertissement\FTNTT{Am. 8:1~; 9:10.}}
\VerseOne{}Le Seigneur, Yahweh, me fit voir cette vision~: Voici, il formait des sauterelles au temps où le regain commençait à croître~; et voici le regain poussait après les récoltes du roi.
\VS{2}Et quand elles eurent achevé de dévorer l'herbe de la terre, je dis~: Seigneur Yahweh, pardonne, je te prie~! Comment Jacob subsistera-t-il~? Car il est faible.
\VS{3}Yahweh se repentit de cela. Cela n'arrivera pas, dit Yahweh.
\VS{4}Puis le Seigneur, Yahweh, me fit voir cette vision~: Voici, le Seigneur, Yahweh, criait tout haut, qu'on fasse le jugement par le feu. Et le feu dévorait le grand abîme et dévorait les champs.
\VS{5}Et je dis~: Seigneur Yahweh~! Arrête, je te prie~! Comment Jacob se relèvera-t-il~? Car il est petit.
\VS{6}Et Yahweh se repentit de cela. Cela non plus n'arrivera pas, dit le Seigneur, Yahweh.
\VS{7}Puis il me fit voir cette vision~: Voici, le Seigneur se tenait debout sur un mur fait au niveau, et il avait un niveau dans la main.
\VS{8}Et Yahweh me dit~: Que vois-tu, Amos~? Et je répondis~: Un niveau. Et le Seigneur me dit~: Je mettrai le niveau au milieu de mon peuple d'Israël, je ne lui pardonnerai plus.
\VS{9}Et les hauts lieux d'Isaac seront ravagés, et les sanctuaires d'Israël seront détruits~; et je me lèverai contre la maison de Jéroboam avec l'épée.
\TextTitle{Amatsia accuse Amos devant Jéroboam}
\VS{10}Alors Amatsia, prêtre de Béthel, fit dire à Jéroboam roi d'Israël~: Amos conspire contre toi au milieu de la maison d'Israël~; le pays ne pourrait supporter toutes ses paroles.
\VS{11}Car ainsi a dit Amos~: Jéroboam mourra par l'épée, et Israël ne manquera pas d'être transporté hors de son pays.
\VS{12}Et Amatsia dit à Amos~: Voyant\FTNT{Voyant ou prophète.}, va-t'en, fuis dans le pays de Juda, et manges-y ton pain, et là tu prophétiseras.
\VS{13}Mais ne continue pas à prophétiser à Béthel\FTNT{Béthel, qui signifie «~maison de Dieu~», était devenue le sanctuaire du roi Jéroboam.}, car c'est le sanctuaire du roi, et c'est une maison royale.
\TextTitle{Amos répond}
\VS{14}Amos répondit à Amatsia~: Je n'étais ni prophète ni fils de prophète~; j'étais un berger, et je cueillais des figues sauvages.
\VS{15}Et Yahweh m'a pris derrière le troupeau, et Yahweh m'a dit~: Va, prophétise à mon peuple d'Israël.
\VS{16}Ecoute donc maintenant la parole de Yahweh: Tu me dis~: Ne prophétise plus contre Israël, et ne fait plus tomber ta parole contre la maison d'Isaac.
\VS{17}C'est pourquoi ainsi parle Yahweh~: Ta femme se prostituera dans la ville, tes fils et tes filles tomberont par l'épée, ton champ sera partagé au cordeau, et toi, tu mourras sur une terre souillée, et Israël ne manquera pas d'être transporté hors de son pays.
\Chap{8}
\TextTitle{Vision du panier de fruit, la fin pour le peuple d'Israël}
\VerseOne{}Le Seigneur, Yahweh, me fit voir cette vision~: Voici, je vis un panier de fruits d'été.
\VS{2}Il dit~: Que vois-tu, Amos~? Et je répondis~: Un panier de fruits. Et Yahweh me dit~: La fin est venue pour mon peuple d'Israël, je ne lui en passerai plus.
\VS{3}Les cantiques du temple seront des hurlements en ce jour-là, dit le Seigneur, Yahweh~; en tout lieu, il y aura beaucoup de cadavres que l'on jettera en silence.
\VS{4}Ecoutez ceci vous qui dévorez les indigents, même jusqu'à exterminer les pauvres du pays,
\VS{5}et qui dites~: Quand la nouvelle lune sera-t-elle, pour que nous vendions le grain~? Et le shabbat, pour que nous exposions le grain en vente, en faisant l'épha plus petit, en augmentant le sicle, et falsifiant la balance pour tromper~?
\VS{6}Afin que nous acquérions les faibles pour de l'argent, et le pauvre pour une paire de souliers, et que nous vendions la criblure du froment~?
\VS{7}Yahweh l'a juré par la gloire de Jacob~: Jamais je n'oublierai toutes leurs actions~!
\VS{8}La terre ne sera-t-elle point émue d'une telle chose, et tous ses habitants ne se lamenteront-ils point~? Ne s'écoulera-t-elle pas toute comme un fleuve, et ne sera-t-elle pas emportée et submergée comme par le fleuve d'Egypte~?
\VS{9}Il arrivera en ce jour-là, dit le Seigneur, Yahweh, que je ferai coucher le soleil à midi, et que j'obscurcirai la terre en plein jour.
\VS{10}Je changerai vos fêtes en deuil, et tous vos chants en lamentations~; je couvrirai de sacs tous les reins, et je rendrai chauves toutes les têtes~; je mettrai le pays dans le deuil comme pour un fils unique, et sa fin sera un jour d'amertume.
\VS{11}Voici, les jours viennent, dit le Seigneur, Yahweh, où j'enverrai la famine dans le pays~; non une famine de pain, ni une soif d'eau, mais d'entendre les paroles de Yahweh.
\VS{12}Ils courront depuis une mer jusqu'à l'autre, ils iront de tous côtés depuis le nord jusqu'à l'orient pour chercher la parole de Yahweh, et ils ne la trouveront pas.
\VS{13}En ce jour-là, les belles vierges et les jeunes hommes s'évanouiront de soif.
\VS{14}Ceux qui jurent par le péché de Samarie disent~: Ô Dan~! Ton dieu est vivant, et Vive la voie de Beer-Schéba~! Ils tomberont et ne se relèveront plus.
\Chap{9}
\TextTitle{Prophétie annonçant la destruction\FTNTT{De. 28:63-68.}}
\VerseOne{}Je vis le Seigneur se tenant debout sur l'autel, et disant~: Frappe le linteau de la porte, et que les poteaux soient ébranlés~; et blesse les tous à la tête~; et je tuerai le dernier d'entre eux avec l'épée~; celui d'entre eux qui s'enfuit ne s'enfuira pas, et celui d'entre eux qui s'échappe ne sera pas délivré.
\VS{2}Quand ils auraient creusé jusqu'au scheol, ma main les enlèvera de là~; et quand ils monteraient jusqu'aux cieux, je les en ferai descendre.
\VS{3}Et quand ils se cacheraient au sommet du Carmel, je les y chercherai, et je les enlèverai de là~; et quand ils se seraient cachés à ma vue dans le fond de la mer, de là je commanderai au serpent, et il les mordra.
\VS{4}Et lorsqu'ils iraient en captivité devant leurs ennemis, de là j'ordonnerai à l'épée, et elle les tuera~; je fixerai mes yeux sur eux pour leur faire du mal, et non pour leur faire du bien.
\VS{5}Car c'est le Seigneur Yahweh des armées qui touche la terre, et elle se fond, et tous ceux qui l'habitent mènent deuil, et elle s'écoule toute comme un fleuve, et elle est submergée comme par le fleuve d'Egypte.
\VS{6}C'est lui qui a bâti ses étages dans les cieux, et qui a établi ses armées sur la terre~; qui appelle les eaux de la mer, et qui les répand sur le dessus de la terre~; son nom est Yahweh.
\VS{7}N'êtes-vous pas pour moi comme les enfants de Cusch, enfants d'Israël~? dit Yahweh. N'ai-je pas fait monter Israël du pays d'Egypte, les Philistins de Caphtor et les Syriens de Kir~?
\VS{8}Voici, les yeux du Seigneur Yahweh sont sur ce royaume pécheur. Je le détruirai de dessus la surface de la terre. Cependant, je ne détruirai pas entièrement la maison de Jacob, dit Yahweh.
\VS{9}Car voici, je donnerai mes ordres, et je secouerai la maison d'Israël parmi toutes les nations, comme on secoue le blé dans le crible, sans qu'il en tombe un grain à terre.
\VS{10}Tous les pécheurs de mon peuple mourront par l'épée, ceux qui disent~: Le mal n'approchera pas, il ne nous atteindra pas.
\TextTitle{Yahweh relève la maison de David}
\VS{11}En ce temps-là, je relèverai le tabernacle de David qui est tombé, j'en réparerai les brèches, j'en redresserai les ruines, et je le rebâtirai comme il était autrefois~;
\VS{12}afin qu'ils possèdent le reste d'Edom et toutes les nations sur lesquelles mon nom a été invoqué, dit Yahweh, qui accomplira cela.
\TextTitle{Restauration d'Israël}
\VS{13}Voici, les jours viennent, dit Yahweh, où le laboureur suivra de près le moissonneur, et celui qui foule les raisins atteindra celui qui répand la semence~; et le moût ruissellera des montagnes et découlera de toutes les collines.
\VS{14}Et je ramènerai ceux de mon peuple d'Israël qui auront été emmenés captifs~; et ils rebâtiront les villes dévastées, et y habiteront, ils planteront des vignes, et ils en boiront le vin~; ils feront aussi des jardins, et ils en mangeront les fruits.
\VS{15}Je les planterai sur leur terre, et ils ne seront plus arrachés du pays que je leur ai donné\FTNT{Cette prophétie annonce la restauration de la maison de David. Personne ne chassera Israël de sa terre, aucune nation n'a le pouvoir de le déloger, car c'est le Seigneur qui l'a établi.}, dit Yahweh, ton Dieu.
\PPE{}
\end{multicols}
