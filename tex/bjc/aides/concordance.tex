\begin{multicols}{3}
{\fontsize{6pt}{0.7em}\selectfont

\ConcordanceEntry{Aaron}
\vspace{-2mm}
\begin{listverse}
\item[\vref{Ex 6:20}] qui lui enfanta A. et Moïse ; les
\item[\vref{Ex 7:7}] quatre-vingts ans, et A. de quatre-vingt-trois ans
\end{listverse}
\begin{legend}
\NoAutoSpaceBeforeFDP{
\item Issu de la tribu de Lévite, frère de Moïse et Marie
\item Premier Grand prêtre de Yahweh et Porte parole de Moîse : Ex 4:14-17; 7:1; Ps 99:6; 105:26, Hé 5:4
\item Ses descendants  : 1 Ch 6:50
\item Sa verge fleurie : No 17:8; Hé 9:4
\item Les eaux changées en sang : Ex 7:20
\item Le Veau d’or : Ex 32:3 à 4; De 9:20
\item Sa prêtrise : Ex 28:1; 29:4; Lé 8:2 à 13; Hé 5:14; 7:11
\item Son assistance à Moïse : Ex 5:1; 8:1,2,12; 17:12
\item Incrédule et rebelle  à Yahweh : No 20:10 à 13, 24
\item Parla contre Moïse : No 12:1,2,11
\item Sa Mort : No 20:28; De 10:6
}
\end{legend}

\ConcordanceEntry{Abaddon}
\vspace{-2mm}
\begin{listverse}
\item[\vref{Ap 9:11}] en hébreu est A., mais en grec
\end{listverse}

\ConcordanceEntry{Abaisser}
\vspace{-2mm}
\begin{listverse}
\item[\vref{2 S 22:10}] Il a. les cieux, et descendit : Il y
\item[\vref{Ps 147:6}] malheureux, mais il a. les méchants jusqu'à
\item[\vref{Pr 29:23}] L'orgueil de l'hom. l'a., mais celui qui
\item[\vref{Es 2:9}] de qualité sont a. ; ne lr. pardonne
\item[\vref{Es 2:17}] et les hommes qui s'élèvent seront a. :
\item[\vref{Es 31:4}] lr. cri, ni a. par lr. bruit ;
\item[\vref{Ez 21:31}] est bas, et j'a. ce qui est
\item[\vref{Da 4:37}] et qui peut a. ceux qui marchent
\item[\vref{Mt 23:12}] quiconque s'élèvera sera a. ; et quiconque s'abaissera
\item[\vref{2 Co 11:7}] je me suis a. moi-mm, afin que
\item[\vref{Ph 2:8}] hom., il s'est a. lui-mm, en se
\item[\vref{Ph 4:12}] Je sais être a., je sais aussi
\end{listverse}

\ConcordanceEntry{Abandonner}
\vspace{-2mm}
\begin{listverse}
\item[\vref{Ge 28:15}] car je ne t'a. point que je
\item[\vref{De 31:6}] ne te délaissera point et ne t'a. point.
\item[\vref{De 32:15}] et il a a. Dieu qui l'a
\item[\vref{Jos 24:16}] Dieu ns. garde d'a. Yahweh pour servir
\item[\vref{Jg 2:13}] Ils a. dc Yahweh, et servirent Baal et
\item[\vref{Jg 10:10}] certes, ns. avons a. notre Dieu et
\item[\vref{1 Ch 28:9}] mais si tu l'a., il te rejettera
\item[\vref{Né 10:31}] pour les vendre ; d'a. la septième année
\item[\vref{Ps 9:11}] toi, car tu n'a. point ceux qui
\item[\vref{Ps 22:2}] Dieu ! pourquoi m'as-tu a., t'éloignant de ma
\item[\vref{Ps 37:25}] vu le juste a., ni sa postérité
\item[\vref{Ps 37:28}] juste, et il n'a. point ses fidèles ;
\item[\vref{Ps 138:8}] demeure toujours ; tu n'a. pas l'œuvre de
\item[\vref{Pr 19:4}] est pauvre est a. mm par son
\item[\vref{Es 55:7}] Que le méchant a. sa voie, et
\item[\vref{Jé 2:19}] mauvaise et amère d'a. Yahweh, ton Dieu,
\item[\vref{Za 11:17}] pasteur inutile qui a. les brebis ! Que
\item[\vref{Mt 26:56}] ts les disciples l'a. et s'enfuirent.
\item[\vref{Mt 27:46}] Mon Dieu, mon Dieu, pourquoi m'as-tu a. ?
\item[\vref{2 Co 4:9}] persécutés, mais non a. ; abattus, mais non
\item[\vref{2 Ti 4:16}] mais ts m'ont a. ; toutefois que cela
\item[\vref{Hé 10:35}] N'a. dc pas cette fermeté que vs.
\item[\vref{Jud 1:6}] mais qui ont a. lr. propre demeure ;
\item[\vref{Ap 2:4}] que tu as a. ta première charité.
\end{listverse}

\ConcordanceEntry{Abattre}
\vspace{-2mm}
\begin{listverse}
\item[\vref{Ge 4:5}] fort irrité, et son visage fut a.
\item[\vref{No 22:27}] Yahweh, et elle s'a. sous Balaam. Balaam
\item[\vref{Jg 6:28}] est dessus était a., et le deuxième
\item[\vref{Jg 16:5}] le lierons pour l'a., et ns. te
\item[\vref{Ps 42:6}] Mon âme, pourquoi t'a.-tu, et murmures-tu
\item[\vref{Ps 44:26}] notre âme est a. ds la poussière
\item[\vref{Ps 143:4}] mon esprit est a. au-dedans de moi,
\item[\vref{Pr 17:22}] remède, mais l'esprit a. dessèche les os.
\item[\vref{Es 61:3}] lieu d'un esprit a., afin qu'on les
\item[\vref{Es 66:2}] qui a l'esprit a., et qui tremble
\item[\vref{Ag 2:22}] les montent seront a., chacun par l'épée
\item[\vref{Lu 12:18}] que je ferai : J'a. mes greniers, j'en
\item[\vref{2 Co 4:9}] mais non abandonnés ; a., mais non perdus ;
\item[\vref{1 Th 5:14}] qui ont l'esprit a., supportez les faibles,
\end{listverse}

\ConcordanceEntry{Abba}
\vspace{-2mm}
\begin{listverse}
\item[\vref{Mc 14:36}] Il disait : A., Père, ttes choses
\item[\vref{Ro 8:15}] lequel ns. crions: A. ! C'est-à-dire Père.
\item[\vref{Ga 4:6}] cœurs, lequel crie : A. ! C'est-à-dire Père.
\end{listverse}

\ConcordanceEntry{Abdias}
\vspace{-2mm}
\begin{listverse}
\item[\vref{1 R 18:3}] Achab avait appelé A., chef de sa
\item[\vref{1 R 18:4}] prophètes de Yahweh, A. prit cent prophètes
\item[\vref{Ab 1:1}] La vision d'A.. Ainsi parle le
\end{listverse}

\ConcordanceEntry{Abdon}
\vspace{-2mm}
\begin{listverse}
\item[\vref{Jg 12:13}] Après lui, A. fils d'Hillel, le
\end{listverse}

\ConcordanceEntry{Abed-Négo}
\vspace{-2mm}
\begin{listverse}
\item[\vref{Da 1:7}] de Méschac et à Azaria celui d'A.
\item[\vref{Da 3:23}] Schadrac, Méschac, et A., tombèrent ts liés
\end{listverse}

\ConcordanceEntry{Abel}
\vspace{-2mm}
\begin{listverse}
\item[\vref{Ge 4:2}] Elle enfanta encore A., son frère ; et
\end{listverse}
\begin{legend}
\NoAutoSpaceBeforeFDP{
\item Fils cadet d'Adam et Eve, tué par Caïn son frère : Ge 4:2, 8
\item Son offrande agréée et sa foi : Ge 4:4; Hé 11:4
\item Autres : Mt 23:35; 1 Jn 3:12; Lu 11:51; Hé 12:24
}
\end{legend}

\ConcordanceEntry{Abiathar}
\vspace{-2mm}
\begin{listverse}
\item[\vref{1 S 22:20}] d'Achithub, qui s'appelait A., se sauva, et
\item[\vref{2 S 15:29}] Ainsi, Tsadok et A. rapportèrent l'arche de
\item[\vref{1 R 2:27}] Ainsi, Salomon dépouilla A. de ses fonctions,
\item[\vref{Mc 2:26}] temps du grand-prêtre A., et mangea les
\end{listverse}

\ConcordanceEntry{Abiézer}
\vspace{-2mm}
\begin{listverse}
\item[\vref{Jg 6:11}] de la famille d'A.. Gédéon, son fils,
\item[\vref{Jg 8:32}] Ophra, qui appartenait à la famille d'A.
\end{listverse}

\ConcordanceEntry{Abigaïl}
\vspace{-2mm}
\begin{listverse}
\item[\vref{1 S 25:14}] ce rapport à A., fem. de Nabal,
\item[\vref{1 S 25:39}] pour parler à A., afin de la
\end{listverse}

\ConcordanceEntry{Abija, Abijam}
\vspace{-2mm}
\begin{listverse}
\item[\vref{1 R 14:31}] Naama, l'Ammonite. Et A., son fils, régna
\item[\vref{1 R 15:1}] fils de Nebath, A. commença à régner
\item[\vref{1 Ch 3:10}] Salomon fut Roboam. A., son fils ; Asa,
\item[\vref{2 Ch 13:1}] du roi Jéroboam, A. commença à régner
\end{listverse}

\ConcordanceEntry{Abîme}
\vspace{-2mm}
\begin{listverse}
\item[\vref{Ge 1:2}] la surface de l'a., et l'Esprit de
\item[\vref{Ge 7:11}] sources du grand a. furent rompues, et
\item[\vref{De 33:13}] rosée, et de l'a. qui est en
\item[\vref{Job 38:16}] T'es-tu promené ds les profondeurs de l'a. ?
\item[\vref{Ps 36:7}] sont un grand a.. Yahweh, tu sauves
\item[\vref{Ps 107:26}] ils descendent ds l'a. ; lr. âme se
\item[\vref{Pr 8:27}] le cercle à la surface des a. ;
\item[\vref{Es 44:27}] Qui dit à l'a. : Sois asséchée, et
\item[\vref{Jon 2:6}] environné jusqu'à l'âme. L'a. m'a enveloppé, les
\item[\vref{Ha 3:10}] d'eau se précipitent, l'a. fait retentir sa
\item[\vref{Lu 8:31}] ne pas lr. ordonner d'aller ds l'a.
\item[\vref{Lu 16:26}] vs. un grand a., en sorte que
\item[\vref{Ro 10:7}] Qui descendra ds l'a. ? C'est faire remonter
\item[\vref{2 Pi 2:4}] a précipités ds l'a., et les a
\item[\vref{Ap 9:1}] du puits de l'a. fut donnée à
\item[\vref{Ap 9:11}] roi l'ange de l'a., dont le nom
\item[\vref{Ap 11:7}] qui monte de l'a. lr. fera la
\item[\vref{Ap 20:3}] le jeta ds l'a., et il l'enferma
\end{listverse}

\ConcordanceEntry{Abimélec}
\vspace{-2mm}
\begin{listverse}
\item[\vref{Ge 20:2}] ma sœur. Et A., roi de Guérar,
\end{listverse}
\begin{legend}
\NoAutoSpaceBeforeFDP{
\item Roi de Guérar (1) Qui convoita Saraï femme d'Abraham : Ge 20:1-17, 21:28-32
\item Roi de Guérar (2) qui fit alliance avec Isaac : Ge 26:1-11, 26-30
\item Fils de Jerubbaal (Gédéon) qui tua ses 70 frères pour devenir roi de Sichem : Jg 9:1-53
}
\end{legend}

\ConcordanceEntry{Abinadab}
\vspace{-2mm}
\begin{listverse}
\item[\vref{1 S 7:1}] ds la maison d'A., sur la colline,
\item[\vref{1 S 16:8}] Puis Isaï appela A., et le fit
\item[\vref{2 S 6:3}] de la maison d'A. qui était sur
\item[\vref{1 Ch 2:13}] Eliab, le second A., le troisième Schimea,
\end{listverse}

\ConcordanceEntry{Abiram}
\vspace{-2mm}
\begin{listverse}
\item[\vref{No 16:1}] avec Dathan et A., fils d'Eliab, et
\item[\vref{No 26:9}] Nemuel, Dathan et A.. Ce Dathan et
\item[\vref{De 11:6}] Dathan et à A., fils d'Eliab, fils
\item[\vref{Ps 106:17}] et engloutit Dathan ; et recouvrit l'assemblée d'A.
\end{listverse}

\ConcordanceEntry{Abischag}
\vspace{-2mm}
\begin{listverse}
\item[\vref{1 R 1:3}] et on trouva A., la Sunamite, que
\item[\vref{1 R 2:22}] Et pourquoi demandes-tu A., la Sunamite, pour
\end{listverse}

\ConcordanceEntry{Abischaï}
\vspace{-2mm}
\begin{listverse}
\item[\vref{1 S 26:9}] David dit à A. : Ne le tue
\item[\vref{2 S 2:18}] de Tseruja, Joab, A. et Asaël étaient
\item[\vref{2 S 3:30}] Ainsi Joab et A., son frère, tuèrent
\item[\vref{2 S 23:18}] y avait aussi A., frère de Joab,
\end{listverse}

\ConcordanceEntry{Ablutions}
\vspace{-2mm}
\begin{listverse}
\item[\vref{Hé 9:10}] breuvages, en diverses a., et en des
\end{listverse}

\ConcordanceEntry{Abner}
\vspace{-2mm}
\begin{listverse}
\item[\vref{1 S 14:50}] son armée était A., fils de Ner,
\end{listverse}
\begin{legend}
\NoAutoSpaceBeforeFDP{
\item Fils de Ner, oncle du roi Saül et  chef de son l'armée : 1 S 14:50; 17:55
\item Tué par Joab et pleuré par David : 2 S 3:27-31
\item Autres : 2 Samuel 2:8-9; 2 S 3:9-12
}
\end{legend}

\ConcordanceEntry{Abolir}
\vspace{-2mm}
\begin{listverse}
\item[\vref{Est 9:28}] ville ; et qu'on n'a. point ces jours
\item[\vref{Ps 119:126}] temps que Yahweh opère ; ils ont a. ta loi.
\item[\vref{Da 9:24}] ville sainte, pour a. la transgression et
\item[\vref{So 3:15}] Yahweh a a. ta condamnation, il a éloigné ton
\item[\vref{Mt 5:17}] je sois venu a. la loi ou
\item[\vref{Ro 4:14}] est anéantie, et la promesse est a.,
\item[\vref{1 Co 1:28}] sont pas, pour a. celles qui sont,
\item[\vref{1 Co 13:8}] Les prophéties seront a. et les langues
\item[\vref{1 Co 13:10}] ce qui est en partie sera a.
\item[\vref{1 Co 13:11}] devenu hom., j'ai a. ce qui était
\item[\vref{1 Co 15:24}] Père, après avoir a. tt empire, tte
\item[\vref{2 Co 3:14}] voile, qui est a. par Christ, demeure
\item[\vref{Ga 5:11}] scandale de la croix est dc a.
\item[\vref{Ep 2:15}] ayant a. ds sa chair l'inimitié, à savoir
\item[\vref{Col 2:14}] il l'a entièrement a. en le clouant
\item[\vref{Hé 8:13}] vieux et ancien, est près d'être a.
\item[\vref{Hé 10:9}] Il a. ainsi le premier afin d'établir le
\end{listverse}

\ConcordanceEntry{Abominable}
\vspace{-2mm}
\begin{listverse}
\item[\vref{Lé 11:43}] point vos personnes a. par ts ces
\item[\vref{De 14:3}] Tu ne mangeras d'aucune chose a.
\item[\vref{1 R 21:26}] se rendit fort a., allant après les
\item[\vref{Job 15:16}] combien plus est a. et corrompu l'hom.
\item[\vref{Es 66:5}] com. une chose a., à cause de
\item[\vref{Es 66:17}] et des choses a., com. des souris,
\item[\vref{Ez 16:52}] été rendue plus a. qu'elles ; elles sont
\item[\vref{Da 9:27}] moyen des ailes a., qui causeront la
\item[\vref{Os 9:10}] ils sont devenus a. com. ce qu'ils
\item[\vref{Tit 1:16}] car ils sont a., rebelles, et réprouvés
\item[\vref{2 Pi 3:17}] la séduction des a., vs. ne veniez
\item[\vref{Ap 21:8}] les incrédules, les a., les meurtriers, les
\end{listverse}

\ConcordanceEntry{Abomination}
\vspace{-2mm}
\begin{listverse}
\item[\vref{Ge 43:32}] que c'est à leurs yeux une a.
\item[\vref{Ge 46:34}] Egyptiens ont en a. les bergers.
\item[\vref{Lé 18:22}] couche avec une fem.. C'est une a.
\item[\vref{De 18:12}] choses est en a. à Yahweh ; et
\item[\vref{De 25:16}] injustice, est en a. à Yahweh, ton
\item[\vref{2 R 23:13}] bâtis à Astarté, l'a. des Sidoniens, à
\item[\vref{1 Ch 21:6}] parole du roi en l'ayant en a.,
\item[\vref{Pr 11:20}] pervers sont en a. à Yahweh, mais
\item[\vref{Pr 15:9}] méchant est en a. à Yahweh, mais
\item[\vref{Pr 16:5}] Yahweh a en a. tt hom. hautain
\item[\vref{Pr 21:27}] méchants est une a. ; combien plus qnd
\item[\vref{Pr 26:25}] y a sept a. ds son cœur.
\item[\vref{Pr 28:9}] loi, sa prière mm est une a.
\item[\vref{Pr 29:27}] inique est en a. aux justes, et
\item[\vref{Es 1:13}] parfum m'est en a., quant aux nouvelles
\item[\vref{Es 44:19}] reste ferais-je une a. ? Adorerais-je une branche
\item[\vref{Jé 4:1}] tu ôtes tes a. de dvt moi,
\item[\vref{Jé 6:15}] d'avoir commis des a. ? Ils n'en ont
\item[\vref{Jé 32:35}] qu'ils feraient cette a. pour faire pécher
\item[\vref{Ez 7:20}] images de leurs a., et de leurs
\item[\vref{Ez 16:51}] as multiplié tes a. plus qu'elle, et
\item[\vref{Ez 22:11}] L'un commet l'a. avec la fem.
\item[\vref{Da 11:31}] on y dressera l'a. qui causera la
\item[\vref{Da 12:11}] où sera dressée l'a. de la désolation,
\item[\vref{Mi 3:9}] la justice en a., et qui pervertissez
\item[\vref{Mal 2:11}] infidèle, et une a. a été commise
\item[\vref{Mt 24:15}] qnd vs. verrez l'a. qui causera la
\item[\vref{Lu 16:15}] hommes est une a. dvt Dieu.
\item[\vref{Ro 2:22}] qui as en a. les idoles, tu
\item[\vref{Ap 21:27}] qui s'abandonne à l'a. et au mensonge ;
\end{listverse}

\ConcordanceEntry{Abondamment}
\vspace{-2mm}
\begin{listverse}
\item[\vref{Ge 17:2}] toi, et je te multiplierai très a.
\item[\vref{Ps 36:9}] Ils seront a. rassasiés de la
\item[\vref{Pr 11:25}] celui qui arrose a. sera lui-mm arrosé.
\item[\vref{Es 55:7}] et à notre Dieu qui pardonne a.
\item[\vref{Ro 5:15}] Jésus-Christ, ont-ils été a. répandus sur plusieurs.
\item[\vref{1 Co 14:12}] que vs. cherchiez à en posséder a.
\item[\vref{2 Co 9:6}] celui qui sème a. moissonnera aussi abondamment.
\item[\vref{Col 3:16}] de Christ habite a. en vs. en
\item[\vref{1 Ti 6:17}] donne ttes choses a. pour en jouir.
\item[\vref{Tit 3:6}] qu'il a répandu a. sur ns. par
\end{listverse}

\ConcordanceEntry{Abondance}
\vspace{-2mm}
\begin{listverse}
\item[\vref{Ge 1:21}] produisirent en tte a. selon lr. espèce ;
\item[\vref{Ge 27:28}] du blé et du vin en a. !
\item[\vref{Ge 41:29}] années de grande a. ds tt le
\item[\vref{2 Ch 31:5}] d'Israël amenèrent en a. les prémices du
\item[\vref{Est 1:7}] vin royal en a. selon la libéralité
\item[\vref{Ps 65:10}] tu lui donnes l'a., tu la combles
\item[\vref{Ps 66:12}] as fait entrer ds un lieu d'a.
\item[\vref{Pr 1:23}] mon Esprit en a., et je vs.
\item[\vref{Mt 12:34}] Car c'est de l'a. du cœur que
\item[\vref{Mt 13:12}] il sera ds l'a., mais à celui
\item[\vref{Lu 6:45}] car c'est de l'a. du cœur que
\item[\vref{Lu 15:17}] du pain en a., et moi je
\item[\vref{Jn 10:10}] vie, et qu'elles l'aient mm en a.
\item[\vref{Ro 5:17}] ceux qui reçoivent l'a. de la grâce
\item[\vref{2 Co 8:13}] temps présent, votre a. supplée à leurs
\item[\vref{Ph 4:12}] aussi être ds l'a. ; partout et en
\item[\vref{Ph 4:18}] je suis ds l'a., et j'ai été
\end{listverse}

\ConcordanceEntry{Abonder}
\vspace{-2mm}
\begin{listverse}
\item[\vref{De 30:9}] Dieu, te fera a. en bien ds
\item[\vref{Ps 62:11}] qnd les richesses a., n'y mettez point
\item[\vref{Pr 28:20}] L'hom. fidèle a. en bénédictions, mais
\item[\vref{Pr 28:27}] détourne ses yeux a. en malédictions.
\item[\vref{Lu 12:15}] biens de quelqu'un a., il n'a pas
\item[\vref{Ro 5:20}] afin que l'offense a., mais là où
\item[\vref{2 Co 1:5}] souffrances de Christ a. en ns., de
\item[\vref{Ep 1:8}] qu'il a fait a. sur ns. en
\item[\vref{1 Th 3:12}] fasse croître et a. de plus en
\item[\vref{2 Pi 1:8}] vs. et y a., elles ne vs.
\end{listverse}

\ConcordanceEntry{Abram, Abraham}
\vspace{-2mm}
\begin{listverse}
\item[\vref{Ge 11:27}] Térach : Térach engendra A., Nachor, et Haran ;
\item[\vref{Hé 11:8}] Par la foi, A., étant appelé, obéit,
\end{listverse}
\begin{legend}
\NoAutoSpaceBeforeFDP{
\item Fils de Térach, originaire d'Ur en Chaldée : Ge 11:27; 17:5
\item Appel d’Abram : La promesse de Yahweh :  Ge 12:1-3; Hé 11:8; Lu 1:73
\item Ses descendants  : 1 Ch 1:27
\item Abram en Egypte :  Ge 12:10
\item Faute d’Abraham à Guérar : Ge 12:13; 20:2
\item Abram se sépare de Lot : Ge 13:8-9
\item Yahweh traite alliance avec Abram : Ge 13:14-16; 15:18-21; 17:4-7
\item Abram va au secours de Lot : Ge 14:14-17
\item Melchisédek, prêtre d’El Elyon ( Dieu Très-Haut) : Ge 14:18-20; Jn 8:56-58; Hé 7:1-6
\item La foi d’Abraham : Ge 15:6; Ro 4:3; Gal 3:6
\item La circoncision, signe de l’alliance : Ge 17:14
\item Intercession d’Abraham : Ge 18:23-32
\item Abraham chasse Agar avec Ismaël : Ge 21:14; Ga 4:30
\item Présentation d’Isaac en sacrifice à Yahweh : Ge 22:10-13; Ja 2:21
\item Une épouse pour Isaac : Ge 24:3-4
\item Abraham le modèle de la foi : Ro 4:11-12; Ga 3:7
\item Sa mort : Ge 25:8
\item Autres : Ps 105:6; Es 41:8; Es 51:2; Mt 3:9,8:11; Lu 16:23; Jn 8:33; Hé 11:17-19
}
\end{legend}

\ConcordanceEntry{Abri}
\vspace{-2mm}
\begin{listverse}
\item[\vref{Jg 6:11}] le mettre à l'a. de Madian.
\item[\vref{2 S 22:3}] je trouve un a., mon bouclier et
\item[\vref{1 Ch 4:10}] me mets à l'a. du mal, en
\item[\vref{Job 5:21}] Tu seras à l'a. du fléau de
\item[\vref{Job 14:13}] me gardes à l'a. jusqu'à ce que
\item[\vref{Ps 27:5}] tiendra caché sous l'a. de sa tente ;
\item[\vref{Ps 31:21}] les caches sous l'a. de ta face,
\end{listverse}

\ConcordanceEntry{Absalom}
\vspace{-2mm}
\begin{listverse}
\item[\vref{2 S 3:3}] Nabal ; le troisième, A., fils de Maaca,
\item[\vref{2 S 13:28}] Or A. avait donné cet ordre à ses
\item[\vref{2 S 15:1}] arriva qu'après cela, A. se procura des
\item[\vref{2 S 16:22}] une tente pour A. sur le toit
\item[\vref{2 S 18:14}] ds le cœur d'A. qui était encore
\end{listverse}

\ConcordanceEntry{Absent}
\vspace{-2mm}
\begin{listverse}
\item[\vref{1 Co 5:3}] Mais moi, étant a. de corps, mais
\item[\vref{2 Co 5:8}] aimons mieux être a. de ce corps
\item[\vref{2 Co 13:10}] ces choses étant a., afin que qnd
\item[\vref{Ph 1:27}] que je sois a., j'entende quant à
\item[\vref{Col 2:5}] que je sois a. de corps, toutefois
\end{listverse}

\ConcordanceEntry{Absinthe}
\vspace{-2mm}
\begin{listverse}
\item[\vref{De 29:18}] qui produise du poison et de l'a.
\item[\vref{Pr 5:4}] amer com. de l'a., et aigu com.
\item[\vref{Jé 9:15}] nourrir ce peuple d'a., et je lr.
\item[\vref{Jé 23:15}] faire manger de l'a., et lr. ferai
\item[\vref{La 3:15}] m'a rassasié d'amertume, il m'a enivré d'a.
\item[\vref{Am 5:7}] le jugement en a., et qu'ils renversent
\item[\vref{Ap 8:11}] de l'étoile est A. ; et le tiers
\end{listverse}

\ConcordanceEntry{Absoudre}
\vspace{-2mm}
\begin{listverse}
\item[\vref{Ex 21:19}] l'aura frappé sera a. ; toutefois, il le
\item[\vref{Ex 21:28}] mais le maître du bœuf sera a.
\item[\vref{Lu 6:37}] pas condamnés ; absolvez, et vs. serez a.
\end{listverse}

\ConcordanceEntry{Abstenir (s')}
\vspace{-2mm}
\begin{listverse}
\item[\vref{Pr 20:3}] à l'hom. de s'a. des disputes, mais
\item[\vref{Ac 15:29}] savoir, de vs. a. des viandes sacrifiées
\item[\vref{Ro 14:21}] vin, et de s'a. de ce qui
\item[\vref{1 Ti 4:3}] et ordonnant de s'a. des viandes que
\item[\vref{1 Pi 2:11}] voyageurs, à vs. a. des convoitises charnelles
\end{listverse}

\ConcordanceEntry{Abuser}
\vspace{-2mm}
\begin{listverse}
\item[\vref{Jg 19:25}] la connurent et a. d'elle tte la
\item[\vref{2 R 19:10}] te confies, ne t'a. pas en te
\item[\vref{Ez 13:10}] parce qu'ils ont a. mon peuple, en
\item[\vref{1 Co 3:18}] Que personne ne s'a. lui-mm : Si quelqu'un
\item[\vref{1 Co 9:18}] Christ gratuitement, sans a. des  droits que
\item[\vref{Ga 6:3}] soit rien, il s'a. lui-mm.
\end{listverse}

\ConcordanceEntry{Acan}
\vspace{-2mm}
\begin{listverse}
\item[\vref{Jos 7:1}] de l'interdit. Car A., fils de Carmi,
\item[\vref{Jos 7:24}] avec lui, prirent A., fils de Zérach,
\end{listverse}

\ConcordanceEntry{Accabler}
\vspace{-2mm}
\begin{listverse}
\item[\vref{Jg 4:21}] car il était a. de fatigue. Et
\item[\vref{Job 16:7}] il m'a mntnt a. ; tu as dévasté
\item[\vref{Ps 106:27}] d'a. lr. postérité parmi les nations, et
\item[\vref{Pr 12:25}] cœur de l'hom. l'a. ; mais la bonne
\item[\vref{Pr 28:19}] les fainéants sera a. de misère.
\item[\vref{Es 8:21}] ds le pays, a. et affamé ; et
\item[\vref{Es 50:4}] celui qui est a. de maux ; chaque
\item[\vref{Lu 9:32}] avec lui étaient a. de sommeil ; et
\item[\vref{2 Co 1:8}] ns. avons été a. excessivement, au-delà de
\item[\vref{2 Co 2:7}] ne soit pas a. par une trop
\item[\vref{2 Co 5:4}] ns. gémissons, étant a., vu que ns.
\end{listverse}

\ConcordanceEntry{Accès}
\vspace{-2mm}
\begin{listverse}
\item[\vref{Ep 2:18}] et les autres a. auprès du Père
\item[\vref{Ep 3:12}] avons hardiesse et a. avec confiance, par
\end{listverse}

\ConcordanceEntry{Accomplir}
\vspace{-2mm}
\begin{listverse}
\item[\vref{Ge 50:20}] en bien, pour a. ce qui arrive
\item[\vref{1 R 8:15}] et qui a a. par sa puissance
\item[\vref{2 Ch 36:21}] de Jérémie, soit a. ; jusqu'à ce que
\item[\vref{Ps 50:14}] la reconnaissance et a. tes vœux envers
\item[\vref{Ps 119:38}] A. ta parole envers ton serviteur, parole
\item[\vref{Ps 119:112}] mon cœur à a. toujours tes statuts
\item[\vref{Ec 5:3}] diffère point de l'a. ; car il ne
\item[\vref{Es 42:9}] auparavant se sont a.. Et je vs.
\item[\vref{Es 44:26}] son serviteur, et a. le conseil de
\item[\vref{Jé 29:10}] soixante-dix ans seront a. pour Babylone, je
\item[\vref{Ez 20:21}] que l'hom. doit a., pour vivre par
\item[\vref{Da 11:14}] se révolteront pour a. la vision, mais
\item[\vref{Mt 3:15}] est ainsi convenable d'a. tt ce qui
\item[\vref{Mt 5:17}] pas venu les abolir, mais les a.
\item[\vref{Mt 26:54}] Mais comment dc s'a. les Ecritures qui
\item[\vref{Lu 2:39}] qnd ils eurent a. tt ce qui
\item[\vref{Lu 24:44}] qu'il fallait que s'a. tt ce qui
\item[\vref{Jn 4:34}] m'a envoyé, et d'a. son œuvre.
\item[\vref{Jn 8:44}] et vs. voulez a. les désirs de
\item[\vref{Jn 19:30}] dit : Tout est a.. Et ayant baissé
\item[\vref{Ac 3:18}] Dieu a ainsi a. les choses qu'il
\item[\vref{Ac 13:33}] Dieu l'a a. pour ns., leurs enfants, en ressuscitant
\item[\vref{Ro 4:21}] la promesse était aussi puissant pour l'a.
\item[\vref{Ro 7:18}] je ne trouve pas le moyen d'a. le bien.
\item[\vref{Ro 13:14}] la chair pour a. ses convoitises.
\item[\vref{2 Ti 3:17}] de Dieu soit a. et parfaitement instruit
\item[\vref{Ja 1:4}] que la patience a. parfaitement son œuvre,
\item[\vref{Ap 10:7}] de Dieu sera a., com. il l'a
\end{listverse}

\ConcordanceEntry{Accorder}
\vspace{-2mm}
\begin{listverse}
\item[\vref{Jos 1:13}] Dieu vs. a a. du repos, et
\item[\vref{1 S 1:17}] Dieu d'Israël veuille t'a. la demande que
\item[\vref{Esd 7:6}] le roi lui a. tte sa requête,
\item[\vref{Est 5:6}] Elle te sera a.. Quelle est ta
\item[\vref{Ps 85:8}] ta miséricorde et a.-ns. ta délivrance !
\item[\vref{Pr 10:24}] arrive ; mais Dieu a. aux justes ce
\item[\vref{Mt 5:25}] A.-toi rapidement avec ta partie adverse,
\item[\vref{Mt 18:19}] deux d'entre vs. s'a. sur la terre,
\item[\vref{Mc 15:8}] ce qu'il avait coutume de lr. a.
\item[\vref{Ac 11:17}] Dieu lr. a a. le mm don
\item[\vref{Ac 15:15}] Et avec cela s'a. les paroles des
\item[\vref{Ro 11:11}] le salut est a. aux Gentils, pour
\item[\vref{1 Co 16:2}] que Dieu lui a., afin qu'on n'attende
\item[\vref{Ep 3:9}] ns. a été a. du mystère qui
\item[\vref{Col 4:1}] Maîtres, a. à vos serviteurs ce qui est
\item[\vref{Ja 4:6}] Il vs. a., au contraire, une plus grande grâce ;
\item[\vref{2 Pi 1:11}] et Sauveur Jésus-Christ vs. sera abondamment a.
\end{listverse}

\ConcordanceEntry{Accoucher}
\vspace{-2mm}
\begin{listverse}
\item[\vref{Ge 25:24}] où elle devait a. s'accomplirent ; et voici,
\item[\vref{Ge 38:27}] fut au moment d'a., voici, des jumeaux
\item[\vref{Ex 1:19}] vigoureuses, elles ont a. avant que la
\item[\vref{1 S 4:19}] sur le point d'a.. Lorsqu'elle apprit la
\item[\vref{1 R 3:17}] maison et j'ai a. près d'elle ds
\item[\vref{Lu 1:57}] où Elisabeth devait a. arriva, et elle
\item[\vref{Lu 2:6}] là, le temps où Marie devait a. arriva,
\item[\vref{Jn 16:21}] Quand une fem. a., elle a des
\item[\vref{Ap 12:4}] fem. qui devait a., afin de dévorer
\end{listverse}

\ConcordanceEntry{Accroissement}
\vspace{-2mm}
\begin{listverse}
\item[\vref{1 Co 3:6}] mais c'est Dieu qui a donné l'a.
\item[\vref{1 Co 3:7}] qq chose, mais Dieu qui donne l'a.
\item[\vref{Ep 4:16}] assistance, tire son a. selon la force
\item[\vref{Col 2:19}] jointures et des liens, s'accroît d'un a. de Dieu.
\end{listverse}

\ConcordanceEntry{Accroître}
\vspace{-2mm}
\begin{listverse}
\item[\vref{Ge 7:18}] eaux grossirent et s'a. beaucoup sur la
\item[\vref{Job 5:25}] verras ta postérité s'a., et tes descendants
\item[\vref{Es 9:6}] pour a. l'empire, et une paix sans fin
\item[\vref{Da 8:24}] Sa puissance s'a., mais non par
\item[\vref{Ac 9:31}] Seign. ; et elles s'a. par le rafraîchissement
\item[\vref{Col 2:19}] et des liens, s'a. d'un accroissement de
\end{listverse}

\ConcordanceEntry{Accueillir}
\vspace{-2mm}
\begin{listverse}
\item[\vref{Es 14:9}] ses profondeurs, pour t'a. à ton arrivée ;
\item[\vref{Lu 9:11}] suivirent. Jésus les a., et il lr.
\item[\vref{Ro 14:3}] qui en mange, car Dieu l'a a.
\item[\vref{Ro 15:7}] Christ ns. a a., pour la gloire
\end{listverse}

\ConcordanceEntry{Accusateur}
\vspace{-2mm}
\begin{listverse}
\item[\vref{Ap 12:10}] son Christ ; car l'a. de nos frères,
\end{listverse}

\ConcordanceEntry{Accusation}
\vspace{-2mm}
\begin{listverse}
\item[\vref{Esd 4:6}] ils écrivirent une a. calomnieuse contre les
\item[\vref{Jn 18:29}] lr. dit : Quelle a. portez-vs. contre cet
\item[\vref{Ac 25:7}] nombreuses et graves a., qu'ils ne pouvaient
\item[\vref{Ro 8:33}] Qui intentera une a. contre les élus
\item[\vref{1 Ti 5:19}] Ne reçois pas d'a. contre un ancien,
\end{listverse}

\ConcordanceEntry{Accuser}
\vspace{-2mm}
\begin{listverse}
\item[\vref{Da 6:4}] cherchèrent une occasion d'a. Daniel en ce
\item[\vref{Za 3:1}] tenait debout à sa droite, pour l'a.
\item[\vref{Mt 12:10}] avoir sujet de l'a., ils l'interrogèrent, en
\item[\vref{Mc 3:2}] le jour du sabbat, afin de l'a.
\item[\vref{Mc 15:3}] Les principaux prêtres l'a. de plusieurs choses,
\item[\vref{Lu 23:10}] étaient là, et l'a. avec une grande
\item[\vref{Jn 5:45}] que je vs. a. dvt mon Père ;
\item[\vref{Jn 8:6}] afin de pouvoir l'a.. Mais Jésus, s'étant
\item[\vref{Jn 8:9}] ils entendirent cela, a. par lr. conscience,
\item[\vref{Ac 24:2}] Tertulle commença à l'a., en disant :
\item[\vref{Ro 2:15}] et leurs pensées s'a. entre elles ou
\end{listverse}

\ConcordanceEntry{Achab}
\vspace{-2mm}
\begin{listverse}
\item[\vref{1 R 16:29}] A., fils d'Omri, régna sur Israël la
\end{listverse}
\begin{legend}
\NoAutoSpaceBeforeFDP{
\item Roi d'Israël, épouse Jézabel : 1 R 16:29, 31
\item Elie à la rencontre d'A. : 1 R 18:1-17
\item A. monte contre la Syrie et emporte la victoire : 1 R 20:15-29
\item Faute d'A. qui épargne Ben-Hadad : 1 R 20:42
\item La vigne de Naboth convoitée par A. : 1 R 21:16
\item A. s'humilie devant Dieu : 1 R 21:27-29
\item Sa mort : 1 R 22:34-35
\item Achab fils de Kolaja,un faux prophète : Jé 29:21
}
\end{legend}

\ConcordanceEntry{Achaz}
\vspace{-2mm}
\begin{listverse}
\item[\vref{2 R 16:2}] A. était âgé de vingt ans lorsqu'il
\item[\vref{2 R 16:12}] Quand le roi A. revint de Damas
\item[\vref{2 R 16:20}] A. se coucha avec ses pères, et
\item[\vref{2 Ch 28:24}] Or A. rassembla les ustensiles de la maison
\item[\vref{Es 7:10}] de nouveau à A., en disant :
\end{listverse}

\ConcordanceEntry{Achazia}
\vspace{-2mm}
\begin{listverse}
\item[\vref{1 R 22:40}] ses pères. Et A., son fils, régna
\item[\vref{1 R 22:50}] Alors A., fils d'Achab, dit à Josaphat : Que
\item[\vref{2 R 1:2}] Or A. tomba par le treillis de sa
\item[\vref{2 R 8:25}] d'Achab, roi d'Israël, A., fils de Joram,
\item[\vref{2 R 9:21}] d'Israël, sortit avec A., roi de Juda,
\item[\vref{2 R 9:27}] A., roi de Juda, ayant vu cela,
\end{listverse}

\ConcordanceEntry{Acheter}
\vspace{-2mm}
\begin{listverse}
\item[\vref{Ge 42:3}] Joseph descendirent pour a. du blé en
\item[\vref{De 2:6}] Vous a. d'eux la nourriture à prix d'argent
\item[\vref{Jg 9:4}] s'en servit pour a. des hommes misérables
\item[\vref{2 S 24:21}] David répondit : Pour a. ton aire, et
\item[\vref{Job 28:15}] pur, elle ne s'a. pas au poids
\item[\vref{Pr 17:16}] du fou pour a. la sagesse, vu
\item[\vref{Es 55:1}] pas d'argent, venez, a. et mangez ; venez,
\item[\vref{Mt 13:46}] tt ce qu'il avait, et l'a a.
\item[\vref{Mt 14:15}] les villages, pour s'a. des vivres.
\item[\vref{Mt 25:10}] qu'elles allaient en a., l'époux arriva. Celles
\item[\vref{Mt 27:10}] ont données pour a. le champ d'un
\item[\vref{Mc 6:36}] des environs pour s'a. des pains ; car
\item[\vref{1 Co 6:20}] vs. avez été a. à un grand
\item[\vref{1 Co 7:30}] pas, ceux qui a. com. ne possédant
\item[\vref{Ap 3:18}] Je te conseille d'a. de moi de
\item[\vref{Ap 13:17}] personne ne puisse a. ni vendre, sans
\item[\vref{Ap 18:11}] que plus personne n'a. leurs marchandises,
\end{listverse}

\ConcordanceEntry{Achever}
\vspace{-2mm}
\begin{listverse}
\item[\vref{Ge 2:1}] la terre furent a., avec tte lr.
\item[\vref{Ex 39:32}] Ainsi fut a. tt l'ouvrage du
\item[\vref{1 S 3:12}] contre sa maison ; je commencerai et j'a.
\item[\vref{1 S 13:10}] Comme il a. d'offrir l'holoc., Samuel arriva, et Saül
\item[\vref{Né 4:2}] faire ? Sacrifieront-ils ? Et a.-ils tt en
\item[\vref{Né 6:15}] la muraille fut a. le vingt-cinquième jour
\item[\vref{Ps 19:7}] des cieux et a. sa course à
\item[\vref{Jé 5:18}] je ne vs. a. pas entièrement.
\item[\vref{Jé 43:1}] que Jérémie eut a. de prononcer à
\item[\vref{Za 4:9}] et ses mains l'a. ; et tu sauras
\item[\vref{Lu 14:30}] bâtir, et il n'a pas pu a. ?
\item[\vref{Jn 17:4}] la terre, j'ai a. l'œuvre que tu
\item[\vref{2 Co 8:6}] commencé auparavant, qu'il a. aussi cette grâce
\item[\vref{Ph 1:6}] œuvre en vs., l'a. jusqu'au jour de
\item[\vref{2 Ti 4:7}] bon combat, j'ai a. la course, j'ai
\item[\vref{Hé 4:3}] œuvres aient été a. depuis la fondation
\item[\vref{Hé 8:5}] qnd il devait a. le tabernacle : Prends
\item[\vref{Ap 11:7}] qnd ils auront a. de rendre lr.
\end{listverse}

\ConcordanceEntry{Achija}
\vspace{-2mm}
\begin{listverse}
\item[\vref{1 R 11:29}] chemin le prophète A. de Silo, revêtu
\item[\vref{1 R 14:5}] Yahweh dit à A. : Voici la fem.
\end{listverse}

\ConcordanceEntry{Achimélec}
\vspace{-2mm}
\begin{listverse}
\item[\vref{1 S 21:1}] à Nob, vers A., le prêtre, qui
\item[\vref{1 S 22:16}] mourras, tu mourras, A., toi et tte
\end{listverse}

\ConcordanceEntry{Achitophel}
\vspace{-2mm}
\begin{listverse}
\item[\vref{2 S 15:12}] ville de Guilo, A., le Guilonite, conseiller
\item[\vref{2 S 15:31}] dire à David : A. est parmi ceux
\item[\vref{2 S 16:20}] Absalom dit à A. : Donnez un conseil
\item[\vref{2 S 17:14}] que le conseil d'A.. Car Yahweh avait
\item[\vref{2 S 17:23}] Or A. voyant qu'on n'avait point fait ce
\end{listverse}

\ConcordanceEntry{Acor}
\vspace{-2mm}
\begin{listverse}
\item[\vref{Jos 7:24}] les firent monter ds la vallée d'A.
\end{listverse}

\ConcordanceEntry{Acquérir}
\vspace{-2mm}
\begin{listverse}
\item[\vref{Ge 4:1}] elle dit : J'ai a. un hom. de
\item[\vref{Ge 31:1}] père qu'il s'est a. tte cette richesse.
\item[\vref{Ru 4:8}] dit à Boaz : A.-le pour toi !
\item[\vref{Ps 74:2}] que tu as a. autrefois. Tu t'es
\item[\vref{Pr 4:5}] A. la sagesse, acquiers l'intelligence ; n'oublie pas
\item[\vref{Pr 31:16}] un champ, et l'a. ; et elle plante
\item[\vref{Ec 2:7}] J'ai a. des hommes et des femmes esclaves ;
\item[\vref{Es 63:14}] peuple, afin de t'a. un nom glorieux.
\item[\vref{Ez 28:4}] Tu t'es a. de la puissance par ta sagesse
\item[\vref{Ac 1:18}] Mais après avoir a. un champ avec
\item[\vref{Ac 20:28}] Dieu, qu'il a a. par son propre
\item[\vref{Ac 22:28}] lui dit : J'ai a. ce droit de
\item[\vref{Ep 1:14}] ceux qu'il s'est a. à la louange
\item[\vref{1 Ti 3:13}] auront bien servi s'a. un rang honorable,
\item[\vref{1 Pi 2:9}] sainte, le peuple a., afin que vs.
\end{listverse}

\ConcordanceEntry{Acsa}
\vspace{-2mm}
\begin{listverse}
\item[\vref{Jos 15:16}] donnerai ma fille A. pour fem. à
\item[\vref{Jg 1:13}] donna sa fille A. pour fem.
\end{listverse}

\ConcordanceEntry{Acte}
\vspace{-2mm}
\begin{listverse}
\item[\vref{Ps 99:8}] et qui fait vengeance de leurs a.
\item[\vref{Pr 18:13}] entendu, fait un a. de folie et
\item[\vref{Es 59:18}] Selon leurs a., il rendra à
\item[\vref{Lu 23:51}] conseil et aux a. des autres ; il
\item[\vref{2 Co 10:11}] sommes ds nos a., étant présents.
\item[\vref{Col 2:14}] Il a effacé l'a. qui était contre
\end{listverse}

\ConcordanceEntry{Action}
\vspace{-2mm}
\begin{listverse}
\item[\vref{Ge 44:15}] lr. dit : Quelle a. avez-vs. faite ? Ne
\item[\vref{De 13:11}] fasse plus une a. aussi méchante au
\item[\vref{Jg 19:24}] faites pas cette a. infâme à l'égard
\item[\vref{1 S 24:7}] commettre une telle a. contre mon seigneur,
\item[\vref{Est 1:17}] Car l'a. de la reine parviendra à la
\item[\vref{Ps 33:15}] qui prend garde à ttes leurs a.
\item[\vref{Ps 112:5}] qui règle ses a. avec justice !
\item[\vref{Ps 141:4}] commette quelques méchantes a. par malice, avec
\item[\vref{Pr 20:11}] connaître par ses a. si son œuvre
\item[\vref{Ec 4:3}] vu les mauvaises a. qui se font
\item[\vref{Es 1:16}] méchanceté de vos a. ; cessez de faire
\item[\vref{Za 1:6}] et selon nos a., ainsi a-t-il agi
\item[\vref{Mt 26:10}] fait une bonne a. à mon égard ;
\item[\vref{Ro 8:13}] faites mourir les a. du corps, vs.
\item[\vref{Ro 13:3}] pour une bonne a., mais pour une
\item[\vref{1 Co 5:2}] a commis cette a. soit retranché du
\item[\vref{1 Co 14:16}] Amen ! à ton a. de grâces, puisqu'il
\item[\vref{Col 4:2}] exercice avec des a. de grâces.
\item[\vref{1 Ti 2:1}] supplications, et des a. de grâces pour
\item[\vref{1 Ti 4:3}] en usent avec a. de grâces.
\item[\vref{Tit 3:1}] à faire ttes sortes de bonnes a.,
\item[\vref{2 Pi 2:8}] et entendait dire de leurs méchantes a. ;
\item[\vref{Jud 1:15}] ttes leurs méchantes a. qu'ils ont commises
\item[\vref{Ap 7:12}] la sagesse, les a. de grâces, l'honneur,
\end{listverse}

\ConcordanceEntry{Adam}
\vspace{-2mm}
\begin{listverse}
\item[\vref{Ge 2:23}] Alors A. dit : Voici cette fois celle qui
\item[\vref{Ge 3:20}] Et A. appela sa fem. Eve, parce qu'elle
\item[\vref{Ge 5:3}] Et A. vécut cent trente ans, et engendra
\item[\vref{Ro 5:14}] a régné depuis A. jusqu'à Moïse, mm
\item[\vref{1 Co 15:22}] ts meurent en A., de mm aussi
\item[\vref{1 Co 15:45}] Le premier hom., A., a été fait
\end{listverse}

\ConcordanceEntry{Administrer}
\vspace{-2mm}
\begin{listverse}
\item[\vref{Ez 44:15}] qui ont soigneusement a. ce qu'il fallait
\item[\vref{Lu 16:2}] plus le pouvoir d'a. mes biens.
\item[\vref{2 Co 8:20}] collecte, qui est a. par ns. ;
\item[\vref{1 Pi 4:11}] Dieu ; si quelqu'un a., qu'il administre com.
\end{listverse}

\ConcordanceEntry{Admirable}
\vspace{-2mm}
\begin{listverse}
\item[\vref{Ps 139:14}] étrange et si a. manière ; tes œuvres
\item[\vref{Es 9:5}] épaule : On l'appellera l'A., le Conseiller, le
\item[\vref{Es 28:29}] armées qui est a. en conseil et
\item[\vref{Ap 15:1}] signe, grand et a. : Sept anges qui
\end{listverse}

\ConcordanceEntry{Admiration}
\vspace{-2mm}
\begin{listverse}
\item[\vref{Ex 33:16}] peuple serons en a. plus que ts
\item[\vref{Mt 15:31}] foule était ds l'a. de voir que
\item[\vref{Mc 6:51}] en eux-mêmes excessivement étonnés et remplis d'a.
\item[\vref{Mc 12:17}] ils furent remplis d'a. pour lui.
\item[\vref{Jn 5:20}] celles-ci, afin que vs. soyez ds l'a.
\item[\vref{Ac 3:10}] ils furent remplis d'a. et d'étonnement de
\item[\vref{Ac 13:12}] crut, étant rempli d'a. pour la doctrine
\item[\vref{Ap 13:3}] fut guérie. Remplie d'a., la terre entière
\end{listverse}

\ConcordanceEntry{Admirer}
\vspace{-2mm}
\begin{listverse}
\item[\vref{Ps 27:4}] Yahweh et pour a. son temple.
\item[\vref{Mt 22:33}] ayant entendu cela, a. sa doctrine.
\item[\vref{2 Th 1:10}] et pour être a. ds ts ceux
\item[\vref{Jud 1:16}] enflés, et qui a. les personnes pour
\end{listverse}

\ConcordanceEntry{Adoni-Bézek}
\vspace{-2mm}
\begin{listverse}
\item[\vref{Jg 1:5}] Et ils trouvèrent A. à Bézek ; ils
\end{listverse}

\ConcordanceEntry{Adonija}
\vspace{-2mm}
\begin{listverse}
\item[\vref{2 S 3:4}] le quatrième, A., fils de Haggith ;
\item[\vref{1 R 1:5}] Alors A., fils de Haggith, se laissa emporter
\item[\vref{1 R 1:11}] N'as-tu pas entendu qu'A., fils de Haggith,
\item[\vref{1 R 1:25}] lui ; ils disent : Vive le roi A. !
\item[\vref{1 R 2:13}] Alors A., fils de Haggith, vint vers Bath-Schéba,
\item[\vref{1 R 2:25}] de Jehojada, qui le frappa, et A. mourut.
\end{listverse}

\ConcordanceEntry{Adoption}
\vspace{-2mm}
\begin{listverse}
\item[\vref{Ro 8:15}] avez reçu l'Esprit d'a., par lequel ns.
\item[\vref{Ro 8:23}] ns.-mêmes, en attendant l'a., c'est-à-dire la rédemption
\item[\vref{Ro 9:4}] à qui appartiennent l'a., la gloire, les
\item[\vref{Ga 4:5}] la loi, afin que ns. recevions l'a.
\end{listverse}

\ConcordanceEntry{Adorer}
\vspace{-2mm}
\begin{listverse}
\item[\vref{Ge 22:5}] irons jsq.-là pour a., après quoi ns.
\item[\vref{Es 44:15}] en fait une image taillée et l'a.
\item[\vref{Da 3:5}] terre et vs. a. la statue d'or
\item[\vref{Da 3:28}] de servir et d'a. aucun autre dieu
\item[\vref{Za 14:16}] chaque année pour a. le Roi, Yahweh
\item[\vref{Mt 2:2}] en orient, et ns. sommes venus l'a.
\item[\vref{Mt 4:10}] est écrit : Tu a. le Seign., ton
\item[\vref{Jn 4:20}] Nos pères ont a. sur cette montagne,
\item[\vref{Jn 4:23}] les vrais adorateurs a. le Père en
\item[\vref{Ac 8:27}] ses richesses, venu à Jérus. pour a.,
\item[\vref{Hé 1:6}] Que ts les anges de Dieu l'a. !
\item[\vref{Ap 9:20}] ne cessèrent pas d'a. les démons, les
\item[\vref{Ap 14:9}] forte : Si quelqu'un a. la bête et
\item[\vref{Ap 19:10}] ses pieds pour l'a., mais il me
\end{listverse}

\ConcordanceEntry{Adullam}
\vspace{-2mm}
\begin{listverse}
\item[\vref{1 S 22:1}] ds la caverne d'A.. Ses frères et
\item[\vref{1 Ch 11:15}] ds la caverne d'A., lorsque l'armée des
\end{listverse}

\ConcordanceEntry{Adultère}
\vspace{-2mm}
\begin{listverse}
\item[\vref{Ex 20:14}] Tu ne commettras pas d'a.
\item[\vref{Lé 20:10}] qui commet un a. avec la fem.
\item[\vref{Pr 6:26}] et la fem. a. chasse après l'âme
\item[\vref{Pr 6:32}] qui commet un a. avec une fem.
\item[\vref{Pr 30:20}] de la fem. a. : Elle mange et
\item[\vref{Jé 3:9}] a commis un a. avec la pierre
\item[\vref{Jé 23:10}] d'hommes qui commettent l'a. ; et le pays
\item[\vref{Ez 16:38}] juge les femmes a., et celles qui
\item[\vref{Os 3:1}] ami, et néanmoins a. ; selon l'amour de
\item[\vref{Mt 12:39}] génération méchante et a. demande un miracle,
\item[\vref{Lu 16:18}] autre, commet un a., et quiconque prend
\item[\vref{Jn 8:3}] lui amenèrent une fem. surprise en a. ;
\item[\vref{Ro 2:22}] doit pas commettre a., tu commets adultère !
\item[\vref{Ro 7:3}] elle sera appelée a. ; mais si son
\item[\vref{1 Co 6:9}] fornicateurs, ni les idolâtres, ni les a.,
\item[\vref{Ga 5:19}] évidentes : Ce sont l'a., la fornication, l'impureté,
\item[\vref{Hé 13:4}] Dieu jugera les fornicateurs et les a.
\item[\vref{Ja 4:4}] Hommes et femmes a. ! Ne savez-vs. pas
\item[\vref{Ap 2:22}] ceux qui commettent l'a. avec elle, s'ils
\end{listverse}

\ConcordanceEntry{Adversaire}
\vspace{-2mm}
\begin{listverse}
\item[\vref{De 32:41}] vengeance à mes a. et je rétribuerai
\item[\vref{1 R 5:4}] je n'ai plus d'a., plus de calamités !
\item[\vref{Ps 8:3}] cause de tes a., afin de faire
\item[\vref{Ps 23:5}] face de mes a. ; tu oins d'huile
\item[\vref{Ps 44:6}] ns. battrons nos a., par ton Nom
\item[\vref{Ps 89:43}] droite de ses a., tu as réjoui
\item[\vref{Es 50:8}] Qui est mon a. ? Qu'il s'approche de
\item[\vref{Lu 12:58}] vas avec ton a. dvt le magistrat,
\item[\vref{Lu 21:15}] à laquelle vos a. ne pourront contredire,
\item[\vref{1 Co 16:9}] ouverte, et les a. sont nombreux.
\item[\vref{Ph 1:27}] n'étant en rien épouvantés par les a.
\item[\vref{1 Ti 5:14}] ne donnent à l'a. aucune occasion de
\item[\vref{Hé 10:27}] d'un feu qui doit dévorer les a.
\item[\vref{1 Pi 5:8}] le diable, votre a., tourne autour de
\end{listverse}

\ConcordanceEntry{Adversité}
\vspace{-2mm}
\begin{listverse}
\item[\vref{Ps 138:7}] au milieu de l'a., tu me rends
\item[\vref{Ec 7:14}] au jour de l'a., prends-y garde ; car
\item[\vref{Es 45:7}] et je crée l'a. ; moi, Yahweh, je
\item[\vref{Jé 29:11}] et non pas d'a., pour vs. donner
\end{listverse}

\ConcordanceEntry{Affaire}
\vspace{-2mm}
\begin{listverse}
\item[\vref{Ge 24:50}] et dirent : Cette a. vient de Yahweh,
\item[\vref{Ex 18:26}] et juger de ttes les petites a.
\item[\vref{Ru 4:7}] pour confirmer une a. quelconque relative à
\item[\vref{Esd 10:4}] Lève-toi, car cette a. te regarde et
\item[\vref{Né 2:16}] reste de ceux qui s'occupaient des a.
\item[\vref{Pr 16:3}] Recommande tes a. à Yahweh, et
\item[\vref{Ec 3:1}] et à tte a. sous les cieux
\item[\vref{Ec 8:6}] Car ds tte a. il y a
\item[\vref{Mt 27:19}] mêle pas de l'a. de ce juste,
\item[\vref{Lu 2:49}] je m'occupe des a. de mon Père ?
\item[\vref{1 Co 6:1}] vs. a une a. contre un autre,
\item[\vref{1 Co 9:11}] est-ce une grosse a. si ns. moissonnons
\item[\vref{2 Co 7:11}] de tte manière purs ds cette a.
\item[\vref{2 Ti 2:4}] qui s'embarrasse des a. de cette vie
\item[\vref{Hé 4:13}] de celui dvt lequel ns. avons a.
\item[\vref{1 Pi 4:15}] se mêlant des a. d'autrui,
\end{listverse}

\ConcordanceEntry{Affamé}
\vspace{-2mm}
\begin{listverse}
\item[\vref{Ge 41:55}] pays d'Egypte fut a., et le peuple
\item[\vref{Ps 107:9}] et rassasié de ses biens l'âme a.
\item[\vref{Ps 146:7}] du pain aux a. ; Yahweh délie ceux
\item[\vref{Es 8:21}] pays, accablé et a. ; et il arrivera
\item[\vref{Lu 1:53}] de biens les a., et il a
\end{listverse}

\ConcordanceEntry{Affection}
\vspace{-2mm}
\begin{listverse}
\item[\vref{1 S 19:2}] avait une grande a. pour David. C'est
\item[\vref{Ps 16:3}] pieux sont l'objet de tte mon a.
\item[\vref{Jé 15:1}] n'aurais pourtant point d'a. pour ce peuple ;
\item[\vref{Ro 1:31}] ont promis, sans a. naturelle, implacables, sans
\item[\vref{Ro 8:6}] Or l'a. de la chair, c'est la mort,
\item[\vref{Ro 10:1}] à la bonne a. de mon cœur,
\item[\vref{2 Co 7:15}] et tremblement, son a. pour vs. en
\item[\vref{Ph 3:19}] lr. confusion, n'ayant d'a. que pour les
\item[\vref{1 Th 2:8}] voulu, ds notre a. envers vs., non
\item[\vref{1 Th 5:13}] pour eux beaucoup d'a. à cause de
\item[\vref{2 Ti 3:3}] sans a. naturelle, sans fidélité, calomniateurs, intempérants, cruels,
\item[\vref{1 Pi 5:2}] gain déshonnête, mais par un principe d'a.
\end{listverse}

\ConcordanceEntry{Affermir}
\vspace{-2mm}
\begin{listverse}
\item[\vref{De 3:28}] Josué, fortifie-le et a.-le ; car c'est
\item[\vref{2 S 5:12}] reconnut que Yahweh l'a. com. roi sur
\item[\vref{1 R 9:5}] j'a. le trône de ton royaume sur
\item[\vref{2 Ch 27:6}] parce qu'il avait a. ses voies dvt
\item[\vref{Ps 30:8}] faveur tu avais a. ma montagne… Tu
\item[\vref{Ps 68:29}] tu sois puissant. A., ô Dieu, ce
\item[\vref{Pr 29:4}] Le roi a. le pays par la justice, mais
\item[\vref{Es 9:6}] son royaume, pour l'a. et le soutenir
\item[\vref{Lu 22:32}] un jour converti, a. tes frères.
\item[\vref{Ro 1:11}] don spirituel, afin que vs. soyez a.,
\item[\vref{Ro 14:4}] chrétien faible sera a., car Dieu est
\item[\vref{2 Th 3:3}] fidèle, il vs. a. et vs. gardera
\item[\vref{Ja 5:8}] attendez patiemment, et a. vos cœurs, car
\item[\vref{2 Pi 2:14}] les âmes mal a. ; ils ont le
\item[\vref{2 Pi 3:16}] ignorantes et mal a. tordent, com. elles
\item[\vref{Ap 3:2}] Sois vigilant, et a. le reste qui
\end{listverse}

\ConcordanceEntry{Affliction}
\vspace{-2mm}
\begin{listverse}
\item[\vref{Ge 16:11}] Ismaël, car Yahweh a entendu ton a.
\item[\vref{Ex 3:7}] vu, j'ai vu l'a. de mon peuple
\item[\vref{De 16:3}] levain, du pain d'a., parce que tu
\item[\vref{1 S 1:11}] tu regardes attentivement l'a. de ta servante,
\item[\vref{2 S 16:12}] Yahweh regardera mon a., et que Yahweh
\item[\vref{2 R 14:26}] Yahweh vit que l'a. d'Israël était à
\item[\vref{Esd 9:5}] sein de mon a., et ayant mes
\item[\vref{Né 9:9}] Car tu vis l'a. de nos pères
\item[\vref{Ps 119:153}] Resh.] Regarde mon a. et sauve-moi, car
\item[\vref{Ro 5:3}] mm ds les a., sachant que l'affliction
\item[\vref{2 Co 1:4}] ds tte notre a., afin que par
\item[\vref{2 Co 4:17}] Car notre légère a., qui ne fait
\item[\vref{2 Co 6:4}] patience, ds les a., ds les nécessités,
\item[\vref{2 Th 1:6}] Dieu qu'il rende l'a. à ceux qui
\item[\vref{1 Pi 2:19}] Dieu, endure des a. en souffrant injustement.
\item[\vref{Ap 2:9}] tes œuvres, ton a. et ta pauvreté,
\item[\vref{Ap 2:10}] vs. aurez une a. de dix jours.
\end{listverse}

\ConcordanceEntry{Affligé}
\vspace{-2mm}
\begin{listverse}
\item[\vref{Job 29:12}] car je délivrais l'a. qui criait au
\item[\vref{Ps 18:28}] sauves le peuple a., et tu abaisses
\item[\vref{Ps 82:3}] faites justice à l'a. et au pauvre,
\item[\vref{Ps 113:7}] Il relève l'a. de la poussière,
\item[\vref{Mc 3:5}] indignation, et étant a. de l'endurcissement de
\end{listverse}

\ConcordanceEntry{Affliger}
\vspace{-2mm}
\begin{listverse}
\item[\vref{Ge 6:6}] et il fut a. en son cœur.
\item[\vref{Ex 1:11}] commissaires d'impôts, pour l'a. en le surchargeant ;
\item[\vref{Ru 1:21}] abattue, et que le Tout-Puissant m'a a. ?
\item[\vref{1 S 8:6}] Samuel fut a. de ce qu'ils
\item[\vref{1 S 20:34}] car il était a. à cause de
\item[\vref{Ps 119:107}] je suis extrêmement a., fais-moi revivre selon
\item[\vref{Es 66:2}] celui qui est a., qui a l'esprit
\item[\vref{La 3:33}] pas sa volonté d'a. et d'humilier les
\item[\vref{Mt 9:15}] de l'époux peuvent-ils s'a. pendant que l'époux
\item[\vref{Mc 3:5}] indignation, et étant a. de l'endurcissement de
\item[\vref{Ac 20:38}] l'embrassèrent, étant principalement a. de ce qu'il
\item[\vref{2 Co 1:6}] si ns. sommes a., c'est pour votre
\item[\vref{Hé 11:25}] choisissant plutôt d'être a. avec le peuple
\item[\vref{1 Pi 1:6}] vs. soyez mntnt a. pour un peu
\item[\vref{2 Pi 2:8}] au milieu d'eux, a. ts les jours
\end{listverse}

\ConcordanceEntry{Affranchi}
\vspace{-2mm}
\begin{listverse}
\item[\vref{Ac 6:9}] la synagogue des a., de celle des
\item[\vref{1 Co 7:22}] Seign. est un a. du Seign. ; de
\end{listverse}

\ConcordanceEntry{Affranchir}
\vspace{-2mm}
\begin{listverse}
\item[\vref{Jn 8:36}] le Fils vs. a., vs. serez véritablement
\item[\vref{Ro 6:18}] Ayant dc été a. du péché, vs.
\item[\vref{Ro 8:2}] en Jésus-Christ, m'a a. de la loi
\item[\vref{Ro 8:21}] qu'elle aussi sera a. de la servitude
\item[\vref{Ga 5:1}] Christ ns. a a., et ne vs.
\end{listverse}

\ConcordanceEntry{Agabus}
\vspace{-2mm}
\begin{listverse}
\item[\vref{Ac 11:28}] L'un d'eux, nommé A., se leva, et
\item[\vref{Ac 21:10}] un prophète, nommé A., arriva de Judée
\end{listverse}

\ConcordanceEntry{Agag}
\vspace{-2mm}
\begin{listverse}
\item[\vref{No 24:7}] roi s'élève au-dessus d'A., et son royaume
\item[\vref{1 S 15:8}] mais il épargna A., roi d'Amalek.
\end{listverse}

\ConcordanceEntry{Agape}
\vspace{-2mm}
\begin{listverse}
\item[\vref{Jud 1:12}] écueils ds vos a., lorsqu'ils prennent leurs
\end{listverse}

\ConcordanceEntry{Agar}
\vspace{-2mm}
\begin{listverse}
\item[\vref{Ge 16:3}] fem. d'Abram, prit A., sa servante égyptienne,
\item[\vref{Ge 16:15}] Et A. enfanta un fils à Abram ; et
\item[\vref{Ge 21:14}] les donna à A. en les mettant
\item[\vref{Ga 4:24}] n'enfante que des esclaves, et c'est A.
\end{listverse}

\ConcordanceEntry{Age}
\vspace{-2mm}
\begin{listverse}
\item[\vref{Ps 146:10}] ton Dieu subsiste d'â. en âge ! Louez
\item[\vref{Ec 12:2}] car le jeune â. et l'adolescence ne
\item[\vref{Es 51:8}] justice demeurera toujours, et mon salut d'â. en âge.
\item[\vref{Ga 1:14}] ceux de mon â. et de ma
\end{listverse}

\ConcordanceEntry{Agé}
\vspace{-2mm}
\begin{listverse}
\item[\vref{Job 42:17}] Puis Job mourut â. et rassasié de
\item[\vref{Es 65:20}] celui qui mourra â. de cent ans
\item[\vref{Jn 8:9}] depuis les plus â. jusqu'aux derniers ; et
\item[\vref{1 Ti 5:2}] les femmes â. com. des mères,
\end{listverse}

\ConcordanceEntry{Aggée}
\vspace{-2mm}
\begin{listverse}
\item[\vref{Esd 5:1}] Alors A., le prophète, et Zacharie, fils d'Iddo,
\item[\vref{Esd 6:14}] selon les prophéties d'A., le prophète, et
\item[\vref{Ag 1:13}] Et A., messager de Yahweh, parla au peuple,
\end{listverse}

\ConcordanceEntry{Agir}
\vspace{-2mm}
\begin{listverse}
\item[\vref{Ge 4:7}] Si tu a. bien, tu relèveras ton visage, et
\item[\vref{Jos 1:8}] et nuit, pour a. fidèlement selon tt
\item[\vref{1 S 12:17}] vs. avez mal a. aux yeux de
\item[\vref{1 S 14:6}] incirconcis. Peut-être Yahweh a.-t-il pour ns.,
\item[\vref{Ps 37:5}] Yahweh, confie-toi en lui et il a.
\item[\vref{Ps 39:10}] bouche, parce que c'est toi qui a.
\item[\vref{Ps 66:5}] redoutable qnd il a. sur les fils
\item[\vref{Ps 119:124}] A. envers ton serviteur suivant ta miséricorde
\item[\vref{Mt 24:46}] en arrivant trouvera a. de cette manière !
\item[\vref{Ac 26:9}] j'avais cru devoir a. vigoureusement contre le
\item[\vref{Ro 12:19}] bien-aimés, mais laissez a. la colère de
\item[\vref{1 Th 2:13}] de Dieu, qui a. avec efficacité aussi
\item[\vref{Ap 13:5}] donné le pouvoir d'a. pendant quarante-deux mois.
\end{listverse}

\ConcordanceEntry{Agiter}
\vspace{-2mm}
\begin{listverse}
\item[\vref{Ex 29:24}] et tu les a. de côté et
\item[\vref{Jg 9:9}] honorés, pour aller m'a. sur les arbres ?
\item[\vref{Jg 13:25}] Yahweh commença à l'a. à Machané-Dan, entre
\item[\vref{Job 4:13}] de la nuit a. la pensée, qnd
\item[\vref{Ps 39:7}] ombre, certainement on s'a. inutilement ; on amasse
\item[\vref{Jé 5:22}] pas ; ses vagues s'a., mais elles sont
\item[\vref{Lu 10:41}] t'inquiètes et tu t'a. pour beaucoup de
\item[\vref{Jn 5:4}] le lavoir, et a. l'eau ; et alors
\item[\vref{Ac 17:13}] ils vinrent y a. la foule.
\item[\vref{Ja 1:6}] de la mer, a. et poussé çà
\end{listverse}

\ConcordanceEntry{Agneau}
\vspace{-2mm}
\begin{listverse}
\item[\vref{Ge 22:7}] mais où est l'a. pour l'holoc. ?
\item[\vref{No 28:3}] à Yahweh : Deux a. d'un an sans
\item[\vref{Es 11:6}] loup habitera avec l'a., et le léopard
\item[\vref{Es 53:7}] semblable à un a. qu'on mène à
\item[\vref{Es 65:25}] Le loup et l'a. paîtront ensemble, le
\item[\vref{Jé 11:19}] moi, com. un a., ou com. un
\item[\vref{Mc 14:12}] où l'on sacrifiait l'a. de Pâque, ses
\item[\vref{Lu 10:3}] envoie com. des a. au milieu des
\item[\vref{Jn 1:29}] il dit : Voici l'A. de Dieu, qui
\item[\vref{Jn 1:36}] Jésus qui marchait, il dit : Voici l'A. de Dieu.
\item[\vref{Jn 21:15}] t'aime. Il lui dit : Pais mes a.
\item[\vref{Ac 8:32}] et com. un a. muet dvt celui
\item[\vref{1 Pi 1:19}] Christ, com. d'un a. sans défaut et
\item[\vref{Ap 5:6}] des anciens, un A. qui se tenait
\item[\vref{Ap 5:12}] à haute voix : L'A. qui a été
\item[\vref{Ap 7:14}] longues robes ds le sang de l'A.
\item[\vref{Ap 13:8}] de vie de l'A. immolé dès la
\item[\vref{Ap 14:4}] ceux qui suivent l'A. partout où il
\item[\vref{Ap 19:7}] les noces de l'A. sont venues, et
\item[\vref{Ap 21:23}] Dieu l'éclaire, et l'A. est son flambeau.
\end{listverse}

\ConcordanceEntry{Agonie}
\vspace{-2mm}
\begin{listverse}
\item[\vref{Lu 22:44}] Etant en a., il priait plus instamment, et sa
\end{listverse}

\ConcordanceEntry{Agréable}
\vspace{-2mm}
\begin{listverse}
\item[\vref{Ge 3:6}] et qu'il était a. à la vue
\item[\vref{Ps 51:18}] t'en donnerais ; l'holoc. ne t'est point a.
\item[\vref{Ps 133:1}] c'est une chose a. que des frères
\item[\vref{Ec 2:26}] qui lui est a., de la sagesse,
\item[\vref{Ro 12:1}] sacrifice vivant, saint, a. à Dieu, ce
\item[\vref{2 Co 5:9}] de lui être a., et présents et
\item[\vref{Ep 5:10}] ce qui est a. au Seign. ;
\item[\vref{Col 3:20}] car cela est a. au Seign.
\item[\vref{1 Ti 2:3}] est bon et a. dvt Dieu, notre
\item[\vref{2 Ti 4:3}] par des discours a., ils chercheront des
\item[\vref{Hé 11:6}] de lui être a. sans la foi ;
\item[\vref{Hé 13:21}] qui lui est a. par Jésus-Christ ; auquel
\item[\vref{1 Pi 2:5}] des sacrifices spirituels, a. à Dieu par
\end{listverse}

\ConcordanceEntry{Agréer}
\vspace{-2mm}
\begin{listverse}
\item[\vref{Lé 19:7}] une abomination. Il ne sera point a.
\item[\vref{Ps 119:108}] je te prie, a. les offrandes volontaires
\item[\vref{Es 56:7}] leurs sacrifices seront a. sur mon autel,
\item[\vref{Mal 1:8}] à ton gouverneur ! T'a.-t-il, te recevra-t-il
\item[\vref{Mal 2:13}] ne peut rien a. de vos mains.
\end{listverse}

\ConcordanceEntry{Agrippa}
\vspace{-2mm}
\begin{listverse}
\item[\vref{Ac 25:13}] après, le roi A. et Bérénice arrivèrent
\item[\vref{Ac 25:23}] Le lendemain dc, A. et Bérénice étant
\item[\vref{Ac 26:1}] A. dit à Paul : Il t'est permis
\item[\vref{Ac 26:28}] Et A. répondit à Paul : Tu vas bientôt
\end{listverse}

\ConcordanceEntry{Aï}
\vspace{-2mm}
\begin{listverse}
\item[\vref{Ge 12:8}] à l'occident, et A. à l'orient ; et
\item[\vref{Jos 7:2}] des hommes vers A., qui est près
\item[\vref{Jos 8:1}] et monte contre A.. Regarde, j'ai livré
\end{listverse}

\ConcordanceEntry{Aide}
\vspace{-2mm}
\begin{listverse}
\item[\vref{Ge 2:18}] lui ferai une a. semblable à lui.
\item[\vref{De 33:26}] te venir en a., et sur les
\item[\vref{2 Ch 14:10}] ou sans force ! A.-ns., Yahweh, notre
\item[\vref{Esd 1:4}] lr. viennent en a. avec de l'argent,
\item[\vref{Ps 72:12}] et celui qui n'a personne qui l'a.
\item[\vref{Ps 79:9}] de notre délivrance ! a.-ns. pour l'amour
\item[\vref{Da 2:34}] se détacha sans l'a. d'une main, frappa
\item[\vref{Ac 26:22}] été secouru par l'a. de Dieu, je
\item[\vref{Hé 13:6}] Seign. est mon a., et je ne
\end{listverse}

\ConcordanceEntry{Aider}
\vspace{-2mm}
\begin{listverse}
\item[\vref{2 Ch 25:8}] a la puissance d'a. et de faire
\item[\vref{2 Ch 32:8}] Dieu, pour ns. a. et pour combattre
\item[\vref{Es 63:5}] avait personne pour m'a. ; et j'étais étonné,
\item[\vref{Da 11:1}] de lui pour l'a. et le fortifier.
\item[\vref{Lu 10:40}] Dis-lui dc de m'a. de son côté.
\item[\vref{Ro 8:26}] aussi l'Esprit ns. a. ds notre faiblesse,
\end{listverse}

\ConcordanceEntry{Aigle}
\vspace{-2mm}
\begin{listverse}
\item[\vref{Ex 19:4}] sur des ailes d'a. et vs. ai
\item[\vref{De 32:11}] com. l'a. éveille sa nichée, couve ses petits,
\item[\vref{Job 9:26}] jonc ; com. un a. qui se précipite
\item[\vref{Ps 103:5}] jeunesse est renouvelée com. celle de l'a.
\item[\vref{Pr 30:19}] la trace de l'a. ds le ciel,
\item[\vref{Es 40:31}] ailes, com. des a. ; ils courent, et
\item[\vref{Jé 48:40}] vole com. un a., et il étend
\item[\vref{Ez 1:10}] la face d'un a. à ts les
\item[\vref{Ez 10:14}] et la quatrième la face d'un a.
\item[\vref{Ez 17:3}] Yahweh : Un grand a. à grandes ailes,
\item[\vref{Ap 4:7}] semblable à un a. qui vole.
\item[\vref{Ap 12:14}] ailes d'un grand a. furent données à
\end{listverse}

\ConcordanceEntry{Aigrir}
\vspace{-2mm}
\begin{listverse}
\item[\vref{Ps 73:21}] Quand mon cœur s'a. et que je
\item[\vref{Col 3:19}] et ne vs. a. pas contre elles.
\end{listverse}

\ConcordanceEntry{Aiguille}
\vspace{-2mm}
\begin{listverse}
\item[\vref{Mt 19:24}] le trou d'une a., qu'il ne l'est
\end{listverse}

\ConcordanceEntry{Aiguillon}
\vspace{-2mm}
\begin{listverse}
\item[\vref{1 S 13:21}] était émoussé, mm pour redresser un a.
\item[\vref{Ac 9:5}] serait dur de regimber contre les a.
\item[\vref{1 Co 15:55}] victoire ? Ô mort, où est ton a. ?
\end{listverse}

\ConcordanceEntry{Aimer}
\vspace{-2mm}
\begin{listverse}
\item[\vref{Ge 22:2}] celui que tu a., Isaac, et va-t'en
\item[\vref{Ex 20:6}] à ceux qui m'a. et qui gardent
\item[\vref{Ex 21:5}] l'esclave dit franchement: J'a. mon maître, ma
\item[\vref{Lé 19:18}] peuple ; mais tu a. ton prochain com.
\item[\vref{De 4:37}] parce qu'il a a. tes pères, il
\item[\vref{De 6:5}] Tu a. dc Yahweh, ton Dieu, de tt
\item[\vref{De 7:8}] que Yahweh vs. a., et qu'il garde
\item[\vref{De 10:19}] Vous a. dc l'étranger ; car vs. avez été
\item[\vref{Jos 23:11}] vos âmes, afin d'a. Yahweh, votre Dieu.
\item[\vref{Jg 5:31}] que ceux qui t'a. soient com. le
\item[\vref{1 S 2:9}] pieds de ses bien-a., et les méchants
\item[\vref{Ps 18:2}] ma force, je t'a. d'une affection cordiale.
\item[\vref{Ps 26:8}] Yahweh, j'a. la demeure de
\item[\vref{Ps 34:13}] la vie, qui a. la prolonger pour
\item[\vref{Ps 45:10}] sont parmi tes bien-a. ; la reine est
\item[\vref{Ps 91:14}] Puisqu'il m'a. avec affection, je
\item[\vref{Ps 97:10}] Vous qui a. Yahweh, haïssez le mal ! Il garde
\item[\vref{Ps 127:2}] donne du repos à celui qu'il a.
\item[\vref{Ps 145:20}] ts ceux qui l'a., mais il exterminera
\item[\vref{Pr 8:17}] J'a. ceux qui m'aiment ; et ceux qui
\item[\vref{Pr 17:17}] L'ami intime a. en tt temps,
\item[\vref{Ec 3:8}] un temps pour a. et un temps
\item[\vref{Ec 5:9}] Celui qui a. l'argent n'est point
\item[\vref{Ec 9:9}] fem. que tu a., qui t'a été
\item[\vref{Ca 1:3}] parfum répandu ; c'est pourquoi les filles t'a.
\item[\vref{Ca 3:1}] que mon âme a. ; je l'ai cherché,
\item[\vref{Es 41:8}] élu, la race d'Abraham qui m'a a. !
\item[\vref{Jé 31:3}] m'a dit : Je t'a. d'un amour éternel,
\item[\vref{Da 9:23}] tu es un bien-a.. Sois attentif à
\item[\vref{Os 14:4}] rébellion, je les a. volontairement ; parce que
\item[\vref{Mt 3:17}] est mon Fils bien-a., en qui j'ai
\item[\vref{Mt 5:44}] je vs. dis : A. vos ennemis, bénissez
\item[\vref{Mt 6:24}] haïra l'un, et a. l'autre ; ou il
\item[\vref{Mt 10:37}] Celui qui a. son père ou
\item[\vref{Mc 10:21}] Jésus, l'ayant regardé, l'a., et lui dit :
\item[\vref{Mc 12:6}] un fils, son bien-a., il le lr.
\item[\vref{Mc 12:33}] et que l'a. de tt son cœur, de tte
\item[\vref{Lu 7:47}] elle a beaucoup a.. Or celui à
\item[\vref{Jn 3:16}] Dieu a tant a. le monde qu'il
\item[\vref{Jn 10:17}] Le Père m'a., parce que je
\item[\vref{Jn 12:25}] Celui qui a. sa vie la
\item[\vref{Jn 13:1}] Père, et ayant a. les siens qui
\item[\vref{Jn 13:23}] celui que Jésus a., était à table
\item[\vref{Jn 13:34}] un nouveau commandement : A.-vs. les uns
\item[\vref{Jn 14:23}] dit : Si quelqu'un m'a., il gardera ma
\item[\vref{Jn 16:27}] Père lui-mm vs. a., parce que vs.
\item[\vref{Jn 17:23}] que tu les a. com. tu m'as
\item[\vref{Jn 21:15}] fils de Jonas, m'a.-tu plus que
\item[\vref{Ep 1:6}] ns. a rendus agréables en son bien-a.
\item[\vref{Ep 2:4}] grande charité dont il ns. a a.,
\item[\vref{Ep 5:25}] Et vs. maris, a. vos femmes, com.
\item[\vref{1 Pi 1:8}] lequel vs. a. quoique vs. ne
\item[\vref{1 Pi 1:22}] soit sans hypocrisie, a.-vs. ardemment les
\item[\vref{1 Pi 3:10}] celui qui veut a. sa vie et
\item[\vref{1 Jn 2:10}] Celui qui a. son frère demeure
\item[\vref{1 Jn 2:15}] N'a. point le monde, ni les choses
\item[\vref{1 Jn 4:19}] Nous l'a., parce qu'il ns. a aimés le
\item[\vref{1 Jn 4:20}] Si quelqu'un dit : J'a. Dieu, et qu'il
\item[\vref{Ap 1:6}] qui ns. a a., et qui ns.
\item[\vref{Ap 3:19}] ts ceux que j'a.. Aie dc du
\item[\vref{Ap 20:9}] et la ville bien-a.. Mais Dieu fit
\end{listverse}

\ConcordanceEntry{Air}
\vspace{-2mm}
\begin{listverse}
\item[\vref{1 Co 9:26}] combats, mais non pas com. battant l'a.
\item[\vref{Ep 2:2}] la puissance de l'a., qui est l'esprit
\item[\vref{1 Th 4:17}] Seign., ds les a. et ainsi ns.
\item[\vref{Ap 9:2}] le soleil et l'a. furent obscurcis par
\item[\vref{Ap 16:17}] sa coupe ds l'a. ; et il sortit
\end{listverse}

\ConcordanceEntry{Airain}
\vspace{-2mm}
\begin{listverse}
\item[\vref{Ge 4:22}] ttes sortes d'instruments d'a. et de fer.
\item[\vref{Ex 26:37}] tu fondras pour eux cinq bases d'a.
\item[\vref{Ex 27:3}] encensoirs ; tu feras ts ses ustensiles d'a.
\item[\vref{De 8:9}] et des montagnes desquelles tu tailleras l'a.
\item[\vref{Ps 107:16}] brisé les portes d'a. et cassé les
\item[\vref{Jé 1:18}] et un mur d'a., contre les rois
\item[\vref{Da 2:32}] son ventre et ses cuisses étaient d'a. ;
\item[\vref{Mi 4:13}] mettrai des ongles d'a. ; et tu écraseras
\item[\vref{1 Co 13:1}] je suis un a. qui résonne ou
\item[\vref{Ap 1:15}] semblables à de l'a. ardent, com. s'ils
\end{listverse}

\ConcordanceEntry{Aire}
\vspace{-2mm}
\begin{listverse}
\item[\vref{Ge 50:11}] ce deuil ds l'a. d'Athad, dirent : Ce
\item[\vref{1 S 23:1}] guerre à Keïla, et pillent les a.
\item[\vref{2 S 24:24}] Ainsi, David acheta l'a. et les bœufs
\item[\vref{Mi 4:12}] a assemblées com. des gerbes ds l'a.
\item[\vref{Mt 3:12}] nettoiera entièrement son a., et il assemblera
\end{listverse}

\ConcordanceEntry{Ajalon}
\vspace{-2mm}
\begin{listverse}
\item[\vref{Jos 10:12}] et toi lune, sur la vallée d'A. !
\end{listverse}

\ConcordanceEntry{Ajouter}
\vspace{-2mm}
\begin{listverse}
\item[\vref{De 4:2}] Vous n'a. rien à la parole que je
\item[\vref{2 Ch 28:13}] Yahweh, voulez-vs. en r. à nos péchés
\item[\vref{Job 34:37}] Car il a. péché sur péché ; il applaudira au
\item[\vref{Pr 10:22}] et il n'y a. aucune peine.
\item[\vref{Pr 30:6}] N'a. rien à ses paroles, de peur
\item[\vref{Ec 3:14}] rien à y a. et rien à
\item[\vref{Es 38:5}] tes larmes. Voici, j'a. à tes jours
\item[\vref{Jé 36:32}] de paroles semblables y furent encore a.
\item[\vref{Mt 6:27}] par ses inquiétudes, a. une coudée à
\item[\vref{Ac 2:41}] ce jour-là, furent a. à l'Eglise environ
\item[\vref{Ga 3:15}] annulé par personne, et personne n'y a.
\item[\vref{Ap 22:18}] Si quelqu'un y a. qq chose, Dieu
\end{listverse}

\ConcordanceEntry{Akisch}
\vspace{-2mm}
\begin{listverse}
\item[\vref{1 S 21:10}] s'en alla vers A., roi de Gath.
\item[\vref{1 S 27:2}] il passa chez A., fils de Maoc,
\item[\vref{1 S 29:2}] ses gens marchèrent à l'arrière-garde avec A.
\end{listverse}

\ConcordanceEntry{Alexandre}
\vspace{-2mm}
\begin{listverse}
\item[\vref{Ac 4:6}] grand-prêtre, Caïphe, Jean, A., et ts ceux
\item[\vref{1 Ti 1:20}] sont Hyménée et A., que j'ai livrés
\item[\vref{2 Ti 4:14}] A., le forgeron m'a fait beaucoup de
\end{listverse}

\ConcordanceEntry{Aliment}
\vspace{-2mm}
\begin{listverse}
\item[\vref{Ge 6:21}] de ts les a. que l'on mange,
\item[\vref{Lé 3:16}] l'autel. C'est un a. d'offrande consumée par
\item[\vref{Ag 2:12}] l'huile, ou un a. quelconque, cela devient-il
\item[\vref{Mal 1:12}] rapporte est un a. méprisable.
\item[\vref{Mc 7:19}] purifient le corps de ts les a.
\item[\vref{1 Co 6:13}] Les a. sont pour le ventre, et le
\item[\vref{Hé 9:10}] ordonnés seulement en a., et en breuvages,
\item[\vref{Hé 12:16}] qui pour un a. vendit son droit
\end{listverse}

\ConcordanceEntry{Allaiter}
\vspace{-2mm}
\begin{listverse}
\item[\vref{Ex 2:7}] les Hébreux qui a. ? Et elle t'allaitera
\item[\vref{1 R 3:21}] suis levée pour a. mon fils. Et
\item[\vref{Es 40:11}] son sein ; il conduira celles qui a.
\item[\vref{Es 49:15}] son enfant qu'elle a. de sorte qu'elle
\item[\vref{Mt 24:19}] à celles qui a. en ces jours-là !
\end{listverse}

\ConcordanceEntry{Allégorique}
\vspace{-2mm}
\begin{listverse}
\item[\vref{Ga 4:24}] ont une valeur a., car ces deux
\end{listverse}

\ConcordanceEntry{Allégresse}
\vspace{-2mm}
\begin{listverse}
\item[\vref{1 Ch 16:31}] terre soit ds l'a. ! Et que l'on
\item[\vref{Ps 30:12}] mon deuil en a., tu as détaché
\item[\vref{Ps 32:11}] Yahweh, soyez ds l'a. ! Criez de joie,
\item[\vref{Ps 43:4}] joie et mon a., et je te
\item[\vref{Ps 51:10}] la joie et l'a., et les os
\item[\vref{Ps 53:7}] Jacob sera ds l'a., Israël se réjouira.
\item[\vref{Ps 81:2}] Chantez avec a. à notre Dieu,
\item[\vref{Ps 100:2}] Servez Yahweh avec a., venez dvt lui
\item[\vref{Ps 105:43}] son peuple ds l'a., ses élus au
\item[\vref{Ps 126:5}] sèment avec larmes, moissonneront avec chants d'a.
\item[\vref{Es 25:9}] attendons ; soyons ds l'a., et réjouissons-ns. de
\item[\vref{Es 49:13}] terre, sois ds l'a. ! Et vs., ô
\item[\vref{Es 65:18}] à toujours ds l'a., à cause de
\item[\vref{Jé 15:16}] la joie et l'a. de mon cœur ;
\item[\vref{Jé 31:7}] triomphe, et avec a. à cause de
\item[\vref{Za 9:9}] Sois transportée d'a., fille de Sion !
\item[\vref{Mt 5:12}] et soyez ds l'a., parce que votre
\item[\vref{Lu 6:23}] jour-là, et tressaillez d'a., parce que votre
\item[\vref{Jud 1:24}] paraître dvt sa gloire, irréprochables ds l'a.,
\end{listverse}

\ConcordanceEntry{Alléluia}
\vspace{-2mm}
\begin{listverse}
\item[\vref{Ap 19:1}] foule nombreuse, disant : A. ! Le salut, la
\item[\vref{Ap 19:3}] ils dirent encore : A. ! Et sa fumée
\item[\vref{Ap 19:4}] sur le trône, en disant : Amen ! A. !
\item[\vref{Ap 19:6}] grands tonnerres, disant : A. ! Car le Seign.
\end{listverse}

\ConcordanceEntry{Alliance}
\vspace{-2mm}
\begin{listverse}
\item[\vref{Ge 9:9}] voici, j'établis mon a. avec vs., et
\item[\vref{Ge 15:18}] jour-là, Yahweh traita a. avec Abram, en
\item[\vref{Ge 17:4}] moi, voici, mon a. est avec toi,
\item[\vref{Ge 17:13}] circoncis et mon a. sera ds votre
\item[\vref{Ge 17:14}] peuple parce qu'il aura violé mon a.
\item[\vref{Ex 2:24}] se souvint de l'a. qu'il avait traitée
\item[\vref{Ex 23:32}] ne traiteras point d'a. avec eux ni
\item[\vref{Ex 24:8}] le sang de l'A. que Yahweh a
\item[\vref{Lé 26:9}] et j'établirai mon a. avec vs.
\item[\vref{Lé 26:15}] commandements, et que vs. rompiez mon a.,
\item[\vref{No 18:19}] Yahweh. C'est une a. de sel et
\item[\vref{De 8:18}] de confirmer son a., qu'il a jurée
\item[\vref{De 9:9}] les tables de l'a. que Yahweh a
\item[\vref{De 29:12}] tu entres ds l'a. de Yahweh, ton
\item[\vref{Jos 9:7}] et comment traiterions-ns. a. avec vs. ?
\item[\vref{Jos 9:11}] et mntnt traitez a. avec ns.
\item[\vref{Jos 9:15}] avec eux une a. par laquelle il
\item[\vref{Jos 24:25}] jour-là, Josué traita a. avec le peuple,
\item[\vref{1 S 18:3}] Alors Jonathan fit a. avec David, parce
\item[\vref{2 S 23:5}] avec moi une a. éternelle, bien ordonnée,
\item[\vref{2 R 17:15}] lois, et son a. qu'il avait traitée
\item[\vref{2 R 18:12}] avaient transgressé son a., parce qu'ils n'avaient
\item[\vref{2 Ch 29:10}] cœur de traiter a. avec Yahweh, le
\item[\vref{Esd 10:3}] traitons mntnt cette a. avec notre Dieu,
\item[\vref{Ps 50:5}] qui ont traité a. avec moi par
\item[\vref{Ps 89:4}] J'ai traité a. avec mon élu,
\item[\vref{Ps 105:10}] pour être une a. éternelle,
\item[\vref{Ps 132:12}] fils gardent mon a. et mon témoignage
\item[\vref{Es 54:10}] toi, et mon a. de paix ne
\item[\vref{Jé 31:31}] traiterai une nouvelle a. avec la maison
\item[\vref{Jé 33:20}] pouvez rompre mon a. avec le jour
\item[\vref{Ez 34:25}] avec elles une a. de paix ; et
\item[\vref{Da 11:28}] contre la sainte a., il agira contre
\item[\vref{Ag 2:5}] La parole de l'A. que je traitai
\item[\vref{Za 11:10}] pour rompre mon a. que j'avais traitée
\item[\vref{Mal 2:4}] afin que mon a. avec Lévi subsiste,
\item[\vref{Mal 3:1}] cherchez ; l'Ange de l'A., que vs. désirez,
\item[\vref{Mt 26:28}] de la Nouvelle A., qui est répandu
\item[\vref{Lu 22:20}] est la Nouvelle A. en mon sang,
\item[\vref{Ac 7:8}] donna à Abraham l'a. de la circoncision ;
\item[\vref{1 Co 11:25}] est la Nouvelle A. en mon sang.
\item[\vref{2 Co 3:6}] de la Nouvelle A., non de la
\item[\vref{2 Co 3:14}] ôté ds la lecture de l'Ancienne A.
\item[\vref{Ga 3:17}] que j'entends : Une a., que Dieu a
\item[\vref{Hé 7:22}] C'est dc d'une a. d'autant plus excellente
\item[\vref{Hé 8:6}] le Médiateur d'une a. plus excellente, qui
\item[\vref{Hé 13:20}] le sang de l'A. éternelle, notre Seign.
\item[\vref{Ap 11:19}] l'arche de son a. apparut ds son
\end{listverse}

\ConcordanceEntry{Aloès}
\vspace{-2mm}
\begin{listverse}
\item[\vref{No 24:6}] com. des arbres d'a. que Yahweh a
\item[\vref{Ps 45:9}] parfumés de myrrhe, d'a. et de casse.
\item[\vref{Pr 7:17}] parfumé de myrrhe, d'a. et de cinnamome.
\item[\vref{Ca 4:14}] la myrrhe et l'a., avec ts les
\item[\vref{Jn 19:39}] de myrrhe et d'a. d'environ cent livres.
\end{listverse}

\ConcordanceEntry{Alpha}
\vspace{-2mm}
\begin{listverse}
\item[\vref{Ap 1:8}] Je suis l'A. et l'Oméga, le
\item[\vref{Ap 21:6}] accompli. Je suis l'A. et l'Oméga, le
\end{listverse}

\ConcordanceEntry{Alphée}
\vspace{-2mm}
\begin{listverse}
\item[\vref{Mt 10:3}] péager ; Jacques, fils d'A., et Lebbée, surnommé
\item[\vref{Mc 3:18}] Thomas, Jacques, fils d'A., Thaddée, Simon le
\end{listverse}

\ConcordanceEntry{Amalek, Amalécites}
\vspace{-2mm}
\begin{listverse}
\item[\vref{Ge 36:12}] enfanta à Eliphaz A.. Ce sont là
\end{listverse}
\begin{legend}
\NoAutoSpaceBeforeFDP{
\item Ancêtre d'Esaü : Ge 36:12; 1 Ch 1: 35-36
\item Combat Israël sous Moïse : Ex 17:8-13
\item Sa condamnation par Yahweh : De 25:17-19; 1 S 15:2-7, 30:16-18; 2 S 1:8-16
}
\end{legend}

\ConcordanceEntry{Amandier}
\vspace{-2mm}
\begin{listverse}
\item[\vref{Jé 1:11}] je répondis : Je vois une branche d'a.
\end{listverse}

\ConcordanceEntry{Amasser}
\vspace{-2mm}
\begin{listverse}
\item[\vref{Ps 39:7}] s'agite inutilement ; on a. des biens et
\item[\vref{Pr 10:5}] L'enfant prudent a. en été, mais
\item[\vref{Pr 28:8}] et l'usure, les a. pour celui qui
\item[\vref{Ec 2:8}] me suis aussi a. de l'argent et
\item[\vref{Ha 2:9}] à celui qui a. pour sa maison
\item[\vref{Mt 6:19}] Ne vs. a. pas des trésors sur la terre,
\item[\vref{Mt 6:20}] Mais a.-vs. des trésors ds le ciel,
\item[\vref{Jn 4:36}] le salaire, et a. le fruit pour
\item[\vref{Ro 2:5}] sans repentance, tu t'a. la colère pour
\item[\vref{Ro 12:20}] faisant cela, tu a. des charbons ardents
\item[\vref{2 Co 12:14}] enfants qui doivent a. pour leurs pères,
\item[\vref{Ja 5:3}] feu. Vous avez a. des trésors pour
\end{listverse}

\ConcordanceEntry{Amatsia}
\vspace{-2mm}
\begin{listverse}
\item[\vref{2 R 14:1}] Joachaz, roi d'Israël, A., fils de Joas,
\item[\vref{2 R 14:8}] Alors A. envoya des messagers vers Joas, fils
\item[\vref{2 Ch 25:1}] A. devint roi à l'âge de vingt-cinq
\item[\vref{2 Ch 25:11}] Alors A. prit courage, conduisit son peuple et
\item[\vref{2 Ch 25:20}] Mais A. ne l'écouta point ; Dieu avait résolu
\item[\vref{2 Ch 25:21}] face, lui et A., roi de Juda,
\item[\vref{2 Ch 25:27}] le moment où A. se détourna de
\item[\vref{Am 7:10}] Alors A., prêtre de Béthel, fit dire à
\end{listverse}

\ConcordanceEntry{Ambassadeur}
\vspace{-2mm}
\begin{listverse}
\item[\vref{Jos 9:4}] et contrefirent les a. et prirent de
\item[\vref{Pr 13:17}] le mal, mais l'a. fidèle apporte la
\item[\vref{2 Co 5:20}] Nous sommes dc a. pour Christ, et
\item[\vref{Ep 6:20}] lequel je suis a. quoique chargé de
\end{listverse}

\ConcordanceEntry{Ame}
\vspace{-2mm}
\begin{listverse}
\item[\vref{Ge 2:7}] de vie ; et l'hom. devint une â. vivante.
\item[\vref{Lé 17:11}] Car l'â. de la chair est ds le
\item[\vref{De 6:5}] de tte ton â., et de tte
\item[\vref{1 S 18:1}] parler à Saül, l'â. de Jonathan fut
\item[\vref{1 R 17:21}] te prie que l'â. de cet enfant
\item[\vref{1 R 17:22}] la voix d'Elie, l'â. de l'enfant revint
\item[\vref{2 R 4:27}] Laisse-la, car son â. est ds l'amertume,
\item[\vref{Job 12:10}] ds sa main l'â. de tt ce
\item[\vref{Job 27:8}] l'hypocrite qnd Dieu lui retire son â. ?
\item[\vref{Job 33:22}] son â. s'approche de la fosse, et sa
\item[\vref{Ps 16:10}] n'abandonneras point mon â. au scheol, tu
\item[\vref{Ps 19:8}] parfaite, elle restaure l'â. ; le témoignage de
\item[\vref{Ps 23:3}] Il restaure mon â., et me conduit
\item[\vref{Ps 24:4}] livre pas son â. à la fausseté,
\item[\vref{Ps 30:4}] fait remonter mon â. du scheol, tu
\item[\vref{Ps 33:20}] Notre â. espère en Yahweh ; il est notre
\item[\vref{Ps 34:23}] Pe.] Yahweh rachète l'â. de ses serviteurs,
\item[\vref{Ps 42:2}] d'eau, ainsi mon â. soupire ardemment après
\item[\vref{Ps 42:6}] Mon â., pourquoi t'abats-tu, et murmures-tu au-dedans de
\item[\vref{Ps 49:9}] rachat de lr. â. est trop considérable,
\item[\vref{Ps 49:16}] Dieu rachètera mon â. du pouvoir du
\item[\vref{Ps 62:2}] en soit, mon â. se repose en
\item[\vref{Ps 86:2}] Garde mon â., car je suis
\item[\vref{Ps 103:1}] de David. Mon â., bénis Yahweh ! Et
\item[\vref{Ps 107:9}] qu'il a désaltéré l'â. altérée, et rassasié
\item[\vref{Ps 119:25}] [Daleth.] Mon â. est attachée à
\item[\vref{Ps 121:7}] de tt mal, il gardera ton â.
\item[\vref{Ps 130:6}] Mon â. attend le Seign. plus que les
\item[\vref{Ps 142:5}] personne qui prend soin de mon â.
\item[\vref{Pr 11:30}] qui gagne les â. est sage.
\item[\vref{Pr 23:14}] tu sauveras son â. du scheol.
\item[\vref{Ec 6:3}] cependant si son â. ne s'est point
\item[\vref{Ca 1:7}] toi que mon â. aime, où tu
\item[\vref{Es 1:14}] Mon â. hait vos nouvelles lunes et vos
\item[\vref{Es 55:3}] écoutez, et votre â. vivra ; et je
\item[\vref{Es 58:10}] tu ouvres ton â. à celui qui
\item[\vref{Es 61:10}] Yahweh, et mon â. sera joyeuse en
\item[\vref{La 2:12}] mort, ils rendaient l'â. sur le sein
\item[\vref{La 3:25}] à lui, pour l'â. qui le cherche.
\item[\vref{Ez 14:14}] ils sauveraient leurs â. par lr. justice,
\item[\vref{Ez 18:4}] Voici, ttes les â. sont à moi ;
\item[\vref{Ez 18:20}] L'â. qui pèche est celle qui mourra.
\item[\vref{Jon 2:6}] m'ont environné jusqu'à l'â.. L'abîme m'a enveloppé,
\item[\vref{Mt 10:28}] ne peuvent tuer l'â. ; mais craignez plutôt
\item[\vref{Mt 16:26}] s'il perdait son â. ? Ou, que donnerait
\item[\vref{Mt 22:37}] de tte ton â., et de tte
\item[\vref{Mt 26:38}] lr. dit : Mon â. est de ttes
\item[\vref{Lu 1:46}] Marie dit : Mon â. magnifie le Seign.,
\item[\vref{Lu 12:20}] mm nuit ton â. te sera redemandée ;
\item[\vref{Lu 21:26}] seront com. rendant l'â. de frayeur, ds
\item[\vref{Jn 12:27}] Maintenant mon â. est troublée. Et
\item[\vref{Ac 2:27}] laisseras pas mon â. en enfer et
\item[\vref{Ac 20:10}] pas, car son â. est en lui.
\item[\vref{1 Co 15:45}] été fait en â. vivante. Le dernier
\item[\vref{1 Th 5:23}] être entier, l'esprit, l'â. et le corps
\item[\vref{Hé 4:12}] la division de l'â. et de l'esprit,
\item[\vref{Hé 13:17}] veillent pour vos â., com. dvt en
\item[\vref{Ja 1:21}] vs. et qui peut sauver vos â.
\item[\vref{Ja 5:20}] égarement, sauvera une â. de la mort
\item[\vref{1 Pi 1:9}] à savoir le salut de vos â.
\item[\vref{1 Pi 2:11}] charnelles qui font la guerre à l'â. ;
\item[\vref{2 Pi 2:14}] ils attirent les â. mal affermies ; ils
\item[\vref{3 Jn 1:2}] santé, com. ton â. est en prospérité.
\item[\vref{Ap 6:9}] sous l'autel les â. de ceux qui
\end{listverse}

\ConcordanceEntry{Amen}
\vspace{-2mm}
\begin{listverse}
\item[\vref{De 27:15}] tt le peuple répondra, et dira : A. !
\item[\vref{Né 5:13}] tte l'assemblée répondit : A. ! Et ils louèrent
\item[\vref{Ps 41:14}] le Dieu d'Israël, d'éternité en éternité ! A. ! Amen !
\item[\vref{Ps 89:53}] à toujours Yahweh ; a. ! Oui, amen !
\item[\vref{Ps 106:48}] le peuple dise: A. ! Louez Yahweh !
\item[\vref{Jé 11:5}] répondis et dis : A. ! Ô Yahweh !
\item[\vref{Mt 6:13}] règne, la puissance et la gloire. A. !
\item[\vref{Jn 21:25}] pourrait contenir les livres qu'on écrirait. A. !
\item[\vref{1 Co 14:16}] simple peuple dira-t-il : A. ! à ton action
\item[\vref{2 Co 1:20}] en lui, et a. en lui, à
\item[\vref{Ap 1:18}] siècles des siècles. A. ! Et je tiens
\item[\vref{Ap 3:14}] ce que dit l'A., le témoin fidèle
\item[\vref{Ap 5:14}] quatre animaux disaient : A. ! Et les vingt-quatre
\item[\vref{Ap 7:12}] en disant : A. ! La louange, la
\item[\vref{Ap 19:4}] trône, en disant : A. ! Alléluia !
\end{listverse}

\ConcordanceEntry{Amer}
\vspace{-2mm}
\begin{listverse}
\item[\vref{Ex 12:8}] sans levain, et avec des herbes a.
\item[\vref{No 5:18}] main les eaux a., qui apportent la
\item[\vref{Job 13:26}] tu écrives contre moi des choses a.,
\item[\vref{Ec 7:26}] j'ai trouvé plus a. que la mort,
\item[\vref{Jé 31:15}] lamentations, des larmes a. ; Rachel pleure ses
\item[\vref{Ac 8:23}] un fiel très a. et ds un
\item[\vref{Ja 3:11}] mm ouverture l'eau douce et l'eau a. ?
\item[\vref{Ap 8:11}] les eaux, parce qu'elles étaient devenues a.
\end{listverse}

\ConcordanceEntry{Amertume}
\vspace{-2mm}
\begin{listverse}
\item[\vref{Ru 1:20}] Mara, car le Tout-Puissant m'a remplie d'a.
\item[\vref{1 S 1:10}] le cœur rempli d'a., pria Yahweh en
\item[\vref{Job 9:18}] reprendre mon souffle ; il me rassasie d'a.
\item[\vref{Job 21:25}] l'autre meurt ds l'a. de son âme,
\item[\vref{Job 27:2}] remplit mon âme d'a., est vivant.
\item[\vref{Pr 14:10}] de chacun connaît l'a. de son âme,
\item[\vref{Pr 17:25}] son père, et l'a. de celle qui
\item[\vref{Es 5:20}] ténèbres ; qui font l'a. douceur, et la
\item[\vref{Am 8:10}] et sa fin sera un jour d'a.
\item[\vref{Ro 3:14}] bouche est pleine de malédictions et d'a. ;
\item[\vref{Ep 4:31}] Que tte a., tte colère, tte
\item[\vref{Hé 12:15}] ce qu'aucune racine d'a., poussant des rejetons,
\item[\vref{Ap 10:9}] remplira tes entrailles d'a., mais il sera
\end{listverse}

\ConcordanceEntry{Ami}
\vspace{-2mm}
\begin{listverse}
\item[\vref{Ex 33:11}] avec son intime a.. Puis Moïse retournait
\item[\vref{Job 12:4}] risée de ses a., mais qui invoque
\item[\vref{Ps 55:14}] mon égal, mon confident et mon a. !
\item[\vref{Ps 88:19}] de moi mon a. intime et mon
\item[\vref{Pr 2:17}] qui abandonne l'a. de sa jeunesse
\item[\vref{Pr 14:20}] mm par son a., mais les amis
\item[\vref{Pr 17:17}] L'a. intime aime en tt temps, et
\item[\vref{Pr 18:24}] a des intimes a., se tienne à
\item[\vref{Pr 19:4}] un grand nombre d'a., mais celui qui
\item[\vref{Pr 27:10}] quitte point ton a. ni l'ami de
\item[\vref{Ca 5:16}] tel est mon a., filles de Jérus. !
\item[\vref{Mi 7:5}] pas à ton a. intime, et ne
\item[\vref{Mt 9:15}] lr. répondit : Les a. de l'époux peuvent-ils
\item[\vref{Mt 11:19}] un buveur, un a. des publicains et
\item[\vref{Mt 20:13}] lui dit : Mon a., je ne te
\item[\vref{Mt 22:12}] lui dit : Mon a., comment es-tu entré
\item[\vref{Lu 14:12}] n'invite pas tes a., ni tes frères,
\item[\vref{Lu 15:6}] il appelle ses a. et ses voisins,
\item[\vref{Lu 16:9}] dis : Faites-vs. des a. avec les richesses
\item[\vref{Jn 3:29}] est l'Epoux ; mais l'a. de l'Epoux qui
\item[\vref{Jn 15:13}] qui donne sa vie pour ses a.
\item[\vref{Jn 15:15}] ai appelés mes a., parce que je
\item[\vref{Jn 19:12}] tu n'es pas a. de César. Car
\item[\vref{Ja 2:23}] à justice, et il fut appelé a. de Dieu.
\item[\vref{Ja 4:4}] qui veut être a. du monde, se
\end{listverse}

\ConcordanceEntry{Amitié}
\vspace{-2mm}
\begin{listverse}
\item[\vref{Pr 15:17}] y a de l'a., qu'un repas de
\item[\vref{Pr 17:9}] les fautes cherche l'a., mais celui qui
\item[\vref{Pr 18:24}] tienne à lr. a. parce qu'il y
\item[\vref{Os 2:9}] dont elle recherche l'a., mais ne les
\item[\vref{Ja 4:4}] savez-vs. pas que l'a. du monde est
\end{listverse}

\ConcordanceEntry{Ammon, Ammonites}
\vspace{-2mm}
\begin{listverse}
\item[\vref{Ge 19:38}] le père des A. jusqu'à ce jour.
\end{listverse}
\begin{legend}
\NoAutoSpaceBeforeFDP{
\item Ben-Ammi, Fils de Lot et sa fille (né de l'inceste) : Ge 19:30-38; De 2:19
\item Ennemis d'Israël et évincés par Yahweh : De 23:3; Jé 49:1; Ez 25:1-7; Né 13:1
\item Autres : Jg 11:4,32-33; 1 S 11:11; 2 S 12:26-29
}
\end{legend}

\ConcordanceEntry{Amnon}
\vspace{-2mm}
\begin{listverse}
\item[\vref{2 S 3:2}] Son premier-né fut A., d'Achinoam de Jizreel ;
\item[\vref{2 S 13:1}] nommait Tamar ; et A., fils de David,
\item[\vref{2 S 13:15}] Après cela, A. eut pour elle
\item[\vref{2 S 13:29}] serviteurs d'Absalom traitèrent A. com. Absalom l'avait
\end{listverse}

\ConcordanceEntry{Amon}
\vspace{-2mm}
\begin{listverse}
\item[\vref{2 R 21:18}] le jardin d'Uzza. A., son fils, régna
\item[\vref{2 R 21:19}] A. était âgé de vingt-deux ans lorsqu'il
\item[\vref{2 R 21:23}] Les serviteurs d'A. firent une conspiration
\item[\vref{2 R 21:24}] contre le roi A. ; et ils établirent
\item[\vref{Mt 1:10}] Manassé ; Manassé engendra A. ; Amon engendra Josias ;
\end{listverse}

\ConcordanceEntry{Amoréens}
\vspace{-2mm}
\begin{listverse}
\item[\vref{Ge 10:16}] les Jébusiens, les A., les Guirgasiens,
\item[\vref{Ge 15:16}] car l'iniquité des A. n'est pas encore
\item[\vref{Ex 33:2}] les Cananéens, les A., les Héthiens, les
\item[\vref{No 21:32}] et chassèrent les A. qui y étaient.
\item[\vref{Jg 11:23}] a dépossédé les A. de dvt son
\end{listverse}

\ConcordanceEntry{Amos}
\vspace{-2mm}
\begin{listverse}
\item[\vref{Am 1:1}] Les paroles d'A. qui était parmi
\item[\vref{Am 7:11}] ainsi a dit A. : Jéroboam mourra par
\item[\vref{Am 7:14}] A. répondit à Amatsia : Je n'étais ni
\item[\vref{Am 8:2}] dit : Que vois-tu, A. ? Et je répondis :
\end{listverse}

\ConcordanceEntry{Amour}
\vspace{-2mm}
\begin{listverse}
\item[\vref{Ge 18:32}] Je ne la détruirai point pour l'a. des dix.
\item[\vref{2 S 1:26}] tt mon plaisir ; l'a. que j'avais pour
\item[\vref{Ps 122:8}] Pour l'a. de mes frères et de mes
\item[\vref{Pr 5:19}] d'une biche des a., et d'une chevrette
\item[\vref{Pr 6:26}] Car pour l'a. de la fem.
\item[\vref{Pr 27:5}] Une réprimande ouverte vaut mieux qu'un a. caché.
\item[\vref{Pr 28:2}] pays, mais pour l'a. de l'hom. avisé
\item[\vref{Ec 9:1}] ne connaissent ni l'a. ni la haine
\item[\vref{Ca 1:2}] Jérus. :] Car tes a. sont plus agréables
\item[\vref{Ca 5:8}] lui direz-vs. ? Que je suis malade d'a.
\item[\vref{Ca 8:6}] ton bras ; car l'a. est fort com.
\item[\vref{Jé 2:2}] ta jeunesse, de l'a. de tes fiançailles,
\item[\vref{Jé 31:3}] Je t'aime d'un a. éternel, c'est pourquoi
\item[\vref{Os 11:4}] avec des cordages d'a., et je fus
\item[\vref{So 3:17}] cause de son a., et se réjouira
\item[\vref{Mt 10:39}] sa vie pour l'a. de moi la
\item[\vref{Jn 13:35}] vs. avez de l'a. les uns pour
\item[\vref{Jn 15:13}] de plus grand a. que celui qui
\item[\vref{Jn 17:26}] connaître, afin que l'a. dont tu m'as
\item[\vref{Ph 1:18}] ostentation, ou par a. de la vérité,
\item[\vref{2 Th 2:10}] n'ont pas reçu l'a. de la vérité
\item[\vref{1 Ti 6:10}] Car l'a. de l'argent est la racine de
\end{listverse}

\ConcordanceEntry{Amour de Christ, Amour de Dieu}
\vspace{-2mm}
\begin{listverse}
\item[\vref{1 S 12:22}] son peuple, pour l'a. de son grand
\item[\vref{Lu 11:42}] la justice et l'a. de Dieu. C'est
\item[\vref{Jn 5:42}] vs. n'avez pas l'a. de Dieu en
\item[\vref{Jn 15:9}] vs. ai aimés. Demeurez ds mon a.
\item[\vref{Ro 5:5}] pas, parce que l'a. de Dieu est
\item[\vref{Ro 5:8}] Dieu prouve son a. envers ns., en
\item[\vref{Ro 8:35}] ns. séparera de l'a. de Christ ? Sera-ce
\item[\vref{Ro 8:39}] ns. séparer de l'a. de Dieu manifesté
\item[\vref{1 Co 4:10}] sommes fous pour l'a. de Christ, mais
\item[\vref{2 Co 5:20}] supplions dc pour l'a. de Christ de
\item[\vref{Ph 3:7}] à cause de l'a. de Christ.
\item[\vref{2 Th 3:5}] vos cœurs vers l'a. de Dieu et
\item[\vref{Tit 3:4}] Sauveur et son a. envers les hommes
\item[\vref{1 Jn 2:5}] garde sa parole, l'a. de Dieu est
\item[\vref{1 Jn 5:3}] ceci que consiste l'a. de Dieu : Que
\item[\vref{Jud 1:21}] les autres ds l'a. de Dieu, en
\end{listverse}

\ConcordanceEntry{Amour fraternel}
\vspace{-2mm}
\begin{listverse}
\item[\vref{1 Pi 1:22}] vs. ayez un amour fraternel qui soit sans
\item[\vref{1 Pi 3:8}] envers les autres, d'amour fraternel, miséricordieux et doux.
\item[\vref{2 Pi 1:7}] à la piété l'amour fraternel, et à l'amour
\end{listverse}

\ConcordanceEntry{Amram}
\vspace{-2mm}
\begin{listverse}
\item[\vref{Ex 6:18}] fils de Kehath : A., Jitsehar, Hébron et
\item[\vref{1 Ch 6:3}] Et les fils d'A. furent Aaron, Moïse
\end{listverse}

\ConcordanceEntry{An, année}
\vspace{-2mm}
\begin{listverse}
\item[\vref{Ge 1:14}] pour les jours, et pour les a. ;
\item[\vref{Ge 7:11}] En l'a. six cent de la vie de
\item[\vref{Ge 47:9}] Les jours des a. de mes pèlerinages
\item[\vref{Ex 30:10}] fera une fois l'a. la propitiation sur
\item[\vref{Lé 16:34}] une fois par a.. On fit com.
\item[\vref{Lé 25:5}] Ce sera une a. de repos total
\item[\vref{Ps 65:12}] Tu couronnes l'a. de tes biens,
\item[\vref{Ps 77:6}] d'autrefois et aux a. des siècles passés.
\item[\vref{Ps 78:33}] vanité et leurs a. par une fin
\item[\vref{Ps 90:10}] jours de nos a. reviennent à soixante-dix
\item[\vref{Ps 102:28}] mm, et tes a. ne seront jamais
\item[\vref{Pr 3:2}] jours et des a. de vie et
\item[\vref{Es 34:8}] pour Yahweh, une a. de rétribution pour
\item[\vref{Es 38:5}] Voici, j'ajouterai à tes jours quinze a.
\item[\vref{Es 38:10}] de ce qui restait de mes a.
\item[\vref{Es 61:2}] pour publier une a. de grâce de
\item[\vref{Es 63:4}] mon cœur, et l'a. de mes rachetés
\item[\vref{Jé 25:11}] au roi de Babylone pendant soixante-dix a.
\item[\vref{Da 9:2}] La première a., dis-je, de son
\item[\vref{Za 14:16}] en foule chaque a. pour adorer le
\item[\vref{Lu 3:23}] avait environ trente a., lorsqu'il commença son
\item[\vref{Lu 4:19}] pour publier une a. de grâce du
\item[\vref{Lu 12:19}] assemblés pour beaucoup d'a., repose-toi, mange, bois,
\item[\vref{Lu 13:8}] encore pour cette a., je creuserai tt
\item[\vref{Lu 15:29}] y a tant d'a. que je te
\item[\vref{Ac 18:11}] y demeura un a. et six mois,
\item[\vref{Ga 4:10}] les mois, les temps et les a.
\item[\vref{Hé 9:7}] une fois par a., non sans y
\item[\vref{Hé 9:25}] des saints, chaque a., avec un autre
\item[\vref{Ja 4:13}] y passerons une a., ns. trafiquerons et
\item[\vref{Ap 9:15}] le mois et l'a., afin de tuer
\item[\vref{Ap 20:2}] Satan, et le lia pour mille a.
\end{listverse}

\ConcordanceEntry{Anak, Anakim}
\vspace{-2mm}
\begin{listverse}
\item[\vref{No 13:33}] géants, des enfants d'A., de la race
\item[\vref{De 2:10}] et de haute taille com. les A.
\item[\vref{De 2:21}] taille, com. les A., Yahweh les détruisit
\item[\vref{Jos 11:21}] il extermina les A. des montagnes d'Hébron,
\item[\vref{Jos 14:12}] s'y trouve des A., et qu'il y
\item[\vref{Jos 15:13}] Arba était père d'A. ; et Kirjath-Arba c'est
\item[\vref{Jos 15:14}] les trois fils d'A. : Schéschaï, Ahiman, et
\end{listverse}

\ConcordanceEntry{Ananias}
\vspace{-2mm}
\begin{listverse}
\item[\vref{Ac 9:10}] un disciple, nommé A., à qui le
\item[\vref{Ac 22:12}] Or un nommé A., hom. pieux selon
\item[\vref{Ac 24:1}] cinq jours après, A., le grand-prêtre, descendit
\end{listverse}

\ConcordanceEntry{Ananias, Saphira}
\vspace{-2mm}
\begin{listverse}
\item[\vref{Ac 5:1}] un hom. appelé A., et Saphira, sa
\end{listverse}

\ConcordanceEntry{Anathème}
\vspace{-2mm}
\begin{listverse}
\item[\vref{Ro 9:3}] je souhaiterais être a. et séparé de
\item[\vref{1 Co 12:3}] dit : Jésus est a. ! Et personne ne
\item[\vref{1 Co 16:22}] Jésus-Christ, qu'il soit a. ! Maranatha !
\item[\vref{Ga 1:8}] ns. vs. avons prêché, qu'il soit a. !
\item[\vref{Ap 22:3}] n'y aura plus d'a.. Le trône de
\end{listverse}

\ConcordanceEntry{Anathoth}
\vspace{-2mm}
\begin{listverse}
\item[\vref{Jos 21:18}] A., avec ses faubourgs, et Almon, avec
\item[\vref{1 R 2:26}] prêtre : Va-t-en à A. sur tes terres,
\item[\vref{Jé 1:1}] qui étaient à A., ds le pays
\item[\vref{Jé 11:21}] contre les gens d'A., qui cherchent ta
\end{listverse}

\ConcordanceEntry{Ancien (un)}
\vspace{-2mm}
\begin{listverse}
\item[\vref{Ex 3:16}] et rassemble les a. d'Israël, et dis
\item[\vref{No 11:16}] soixante-dix hommes des a. d'Israël, que tu
\item[\vref{Jos 20:4}] ses raisons aux a. de cette ville-là,
\item[\vref{Jos 23:2}] tt Israël, ses a., ses chefs, ses
\item[\vref{Jos 24:31}] le temps des a. qui survécurent à
\item[\vref{Ru 4:2}] hommes d'entre les a. de la ville,
\item[\vref{1 S 8:4}] pourquoi ts les a. d'Israël s'assemblèrent, et
\item[\vref{Esd 6:14}] Et les a. des Juifs bâtirent avec succès, selon
\item[\vref{Pr 31:23}] qnd il est assis avec les a. du pays.
\item[\vref{Ez 8:11}] Soixante-dix hommes des a. de la maison
\item[\vref{Da 7:9}] soient placés. Et l'A. des jours s'assit.
\item[\vref{Ac 14:23}] ils établirent des a. ds chaque église,
\item[\vref{Ac 20:17}] à Ephèse les a. de l'église.
\item[\vref{1 Ti 4:14}] l'imposition des mains de l'assemblée des a.
\item[\vref{1 Ti 5:17}] Que les a. qui dirigent convenablement soient jugés dignes
\item[\vref{1 Ti 5:19}] d'accusation contre un a., si ce n'est
\item[\vref{Tit 1:5}] tu établisses des a. de ville en
\item[\vref{1 Pi 5:1}] Je prie les a. qui sont parmi
\item[\vref{2 Jn 1:1}] L'a., à Kyria l'élue et à ses
\item[\vref{Ap 4:4}] ces trônes vingt-quatre a. assis, vêtus de
\item[\vref{Ap 7:13}] Et l'un des a. prit la parole
\item[\vref{Ap 11:16}] Alors les vingt-quatre a. qui étaient assis
\end{listverse}

\ConcordanceEntry{Ancien, ancienne}
\vspace{-2mm}
\begin{listverse}
\item[\vref{De 32:7}] Souviens-toi des a. jours, considère les
\item[\vref{Es 44:7}] établi le peuple a. ? Qu'ils déclarent les
\item[\vref{2 Co 3:14}] la lecture de l'A. Alliance.
\item[\vref{2 Co 5:17}] créature ; les choses a. sont passées ; voici,
\item[\vref{Hé 8:13}] devient vieux et a., est près d'être
\item[\vref{2 Pi 2:5}] n'a pas épargné l'a. monde, mais s'il
\item[\vref{1 Jn 2:7}] mais un commandement a., que vs. avez
\item[\vref{Ap 12:9}] dragon, le serpent a., appelé le diable
\end{listverse}

\ConcordanceEntry{Ancre}
\vspace{-2mm}
\begin{listverse}
\item[\vref{Hé 6:19}] tenons com. une a. sûre et ferme
\end{listverse}

\ConcordanceEntry{André}
\vspace{-2mm}
\begin{listverse}
\item[\vref{Mt 4:18}] appelé Pierre, et A., son frère, qui
\item[\vref{Mt 10:2}] nommé Pierre, et A., son frère ; Jacques,
\item[\vref{Mc 13:3}] Jacques, Jean et A., l'interrogèrent en particulier,
\item[\vref{Jn 1:40}] Or A., frère de Simon Pierre, était l'un
\item[\vref{Jn 6:8}] de ses disciples, A., frère de Simon
\item[\vref{Jn 12:22}] le dit à A., et André et
\end{listverse}

\ConcordanceEntry{Ane (un)}
\vspace{-2mm}
\begin{listverse}
\item[\vref{Ge 16:12}] farouche com. un â. sauvage ; sa main
\item[\vref{Ge 49:14}] Issacar est un â. robuste, couché entre
\item[\vref{Ex 23:4}] ennemi, ou son â. égaré, tu ne
\item[\vref{Ex 34:20}] le premier-né d'un â.. Si tu ne
\item[\vref{De 22:10}] point avec un â. et un bœuf
\item[\vref{Jg 15:15}] trouva une mâchoire d'â. fraîche, il étendit
\item[\vref{1 R 13:28}] ds le chemin, l'â. et le lion
\item[\vref{Es 1:3}] son possesseur, et l'â. la crèche de
\item[\vref{Za 9:9}] monté sur un â., sur un âne,
\item[\vref{Lu 13:15}] bœuf ou son â. de la crèche
\end{listverse}

\ConcordanceEntry{Anéantir}
\vspace{-2mm}
\begin{listverse}
\item[\vref{Job 40:3}] A.-tu mon jugement ? me condamneras-tu pour
\item[\vref{Ps 33:10}] des nations, il a. les desseins des
\item[\vref{Es 51:6}] et ma justice ne sera point a.
\item[\vref{Ez 30:13}] aussi les idoles, j'a. les faux dieux
\item[\vref{Jn 10:35}] et cependant l'Ecriture ne peut être a. ;
\item[\vref{Ro 3:3}] cru, lr. incrédulité a.-t-elle la fidélité
\item[\vref{Ro 3:31}] A.-ns. dc la loi par la
\item[\vref{Ro 4:14}] la foi est a., et la promesse
\item[\vref{1 Co 1:19}] des sages et j'a. l'intelligence des hommes
\item[\vref{2 Co 3:13}] fin de ce qui devait être a.
\item[\vref{2 Th 2:8}] bouche et qu'il a. par l'éclat de
\end{listverse}

\ConcordanceEntry{Anesse, ânon}
\vspace{-2mm}
\begin{listverse}
\item[\vref{No 22:27}] Et l'â. vit l'Ange de Yahweh, et elle
\item[\vref{No 22:28}] Yahweh fit parler l'â., et elle dit
\item[\vref{Mt 21:2}] Vous trouverez une â. attachée, et son
\item[\vref{Jn 12:14}] Jésus trouva un â., s'assit dessus, selon
\item[\vref{2 Pi 2:16}] car une â. muette, parlant d'une voix humaine, arrêta
\end{listverse}

\ConcordanceEntry{Ange}
\vspace{-2mm}
\begin{listverse}
\item[\vref{Ex 14:19}] Et l'A. de Dieu qui allait dvt le
\item[\vref{Ex 23:20}] Voici, j'envoie un A. dvt toi, afin
\item[\vref{Ps 103:20}] Yahweh, vs., ses a. puissants en force,
\item[\vref{Es 63:9}] en détresse, et l'a. qui est dvt
\item[\vref{Da 3:28}] a envoyé son a. et délivré ses
\item[\vref{Da 6:22}] a envoyé son a., et a tellement
\item[\vref{Os 12:5}] en luttant avec l'A., et il fut
\item[\vref{Mal 3:1}] que vs. cherchez ; l'A. de l'Alliance, que
\item[\vref{Mt 1:20}] y pensait, voici, l'A. du Seign. lui
\item[\vref{Mt 4:11}] Et voici, des a. s'approchèrent et le
\item[\vref{Mt 13:39}] monde, et les moissonneurs sont les a.
\item[\vref{Mt 18:10}] les cieux leurs a. voient continuellement la
\item[\vref{Mt 22:30}] sera com. les a. de Dieu ds
\item[\vref{Mt 25:31}] ts les saints a., alors il s'assiéra
\item[\vref{Mt 28:5}] Mais l'a. prit la parole et dit aux
\item[\vref{Lu 1:26}] au sixième mois, l'a. Gabriel fut envoyé
\item[\vref{Lu 2:15}] qu'après que les a. s'en furent allés
\item[\vref{Lu 22:43}] Et un a. lui apparut du ciel, pour le
\item[\vref{Jn 1:51}] ouvert, et les a. de Dieu monter
\item[\vref{Jn 5:4}] Car un a. descendait à un certain moment ds
\item[\vref{Jn 12:29}] autres disaient : Un a. lui a parlé.
\item[\vref{Jn 20:12}] elle vit deux a. vêtus de blanc,
\item[\vref{Ac 5:19}] Mais l'A. du Seign. ouvrit pendant la nuit
\item[\vref{Ac 7:53}] une ordon. des a., et qui ne
\item[\vref{Ac 12:7}] Et voici, l'A. du Seign. survint,
\item[\vref{Ac 12:16}] dirent : C'est son a.. Cependant Pierre continuait
\item[\vref{1 Co 6:3}] ns. jugerons les a. ? Combien plus les
\item[\vref{1 Co 13:1}] et mm des a., si je n'ai
\item[\vref{2 Co 11:14}] se déguise en a. de lumière.
\item[\vref{2 Co 12:7}] la chair, un a. de Satan pour
\item[\vref{Ga 1:8}] ns.-mêmes, ou un a. du ciel vs.
\item[\vref{Ga 3:19}] promulguée par des a. au moyen d'un
\item[\vref{Col 2:18}] un culte des a., s'ingérant ds les
\item[\vref{Hé 1:4}] d'autant supérieur aux a., qu'il a hérité
\item[\vref{Hé 1:6}] Que ts les a. de Dieu l'adorent !
\item[\vref{Hé 2:2}] prononcée par les a. a été ferme,
\item[\vref{Hé 2:16}] nullement secouru les a., mais il a
\item[\vref{Hé 13:2}] ont logé des a. sans le savoir.
\item[\vref{1 Pi 1:12}] ds lesquelles les a. désirent plonger leurs
\item[\vref{2 Pi 2:4}] pas épargné les a. qui ont péché,
\item[\vref{2 Pi 2:11}] alors que les a. qui sont supérieurs
\item[\vref{Ap 1:1}] envoyant par son a. à Jean, son
\item[\vref{Ap 1:20}] étoiles sont les a. des sept églises ;
\item[\vref{Ap 22:16}] j'ai envoyé mon a. pour vs. confirmer
\end{listverse}

\ConcordanceEntry{Ange de Yahweh}
\vspace{-2mm}
\begin{listverse}
\item[\vref{Ge 16:7}] Mais l'Ange de Yahweh la trouva près d'une fontaine d'eau
\item[\vref{Ge 22:11}] Mais l'Ange de Yahweh l'appela des cieux, et dit : Abraham,
\item[\vref{Ex 3:2}] Et l'Ange de Yahweh lui apparut ds une flamme de
\item[\vref{No 22:22}] était parti ; et l'Ange de Yahweh se plaça sur
\item[\vref{No 22:27}] Et l'ânesse vit l'Ange de Yahweh, et elle s'abattit
\item[\vref{Jg 2:1}] Or l'Ange de Yahweh monta de Guilgal à Bokim, et
\item[\vref{Jg 6:11}] Puis l'Ange de Yahweh vint et s'assit sous le térébinthe
\item[\vref{Jg 6:22}] voyant que c'était l'Ange de Yahweh dit : Ah ! malheur
\item[\vref{Jg 13:17}] Manoach dit à l'Ange de Yahweh : Quel est ton
\item[\vref{2 S 24:16}] ta main. Or l'Ange de Yahweh était près de
\item[\vref{1 R 19:7}] Et l'Ange de Yahweh vint une seconde fois, le toucha
\item[\vref{2 R 1:3}] Mais l'Ange de Yahweh dit à Elie, le Thischbite : Lève-toi,
\item[\vref{2 R 19:35}] arriva nuit-là que l'Ange de Yahweh sortit et frappa
\item[\vref{1 Ch 21:12}] le pays et l'Ange de Yahweh portera la destruction
\item[\vref{1 Ch 21:15}] ta main. Et l'Ange de Yahweh se tenait près
\item[\vref{1 Ch 21:18}] Alors l'Ange de Yahweh ordonna à Gad de dire à
\item[\vref{Ps 34:8}] [Heth.] L'Ange de Yahweh campe tt autour
\item[\vref{Za 3:1}] tenant debout dvt l'Ange de Yahweh, et Satan qui
\item[\vref{Za 12:8}] com. Dieu, com. l'Ange de Yahweh dvt lr. face.
\end{listverse}

\ConcordanceEntry{Angoisse}
\vspace{-2mm}
\begin{listverse}
\item[\vref{Ge 32:7}] effrayé et rempli d'a. ; et il partagea
\item[\vref{Ge 42:21}] ns. avons vu l'a. de son âme
\item[\vref{Ex 6:9}] à cause de l'a. de lr. esprit,
\item[\vref{1 S 13:6}] pris d'une grande a., car ils étaient
\item[\vref{Né 9:37}] que ns. sommes ds une grande a. !
\item[\vref{Est 9:22}] mois où lr. a. fut changée en
\item[\vref{Job 7:11}] je parlerai ds l'a. de mon esprit,
\item[\vref{Ps 107:13}] il les a délivrés de leurs a.
\item[\vref{Ps 142:3}] je déclare mon a. dvt lui.
\item[\vref{Es 37:3}] est un jour d'a., de répréhension et
\item[\vref{Da 9:25}] seront rebâties, mais en des temps d'a.
\item[\vref{Lu 2:48}] père et moi te cherchions avec a.
\item[\vref{Ro 2:9}] aura tribulation et a. sur tte âme
\item[\vref{Ro 8:35}] Sera-ce l'oppression, ou l'a., ou la persécution,
\end{listverse}

\ConcordanceEntry{Animal}
\vspace{-2mm}
\begin{listverse}
\item[\vref{Ge 1:24}] terre produise des a. selon lr. espèce,
\item[\vref{Ge 2:20}] à ts les a. des champs ; mais
\item[\vref{Ps 8:8}] les bœufs, les a. des champs,
\item[\vref{Ez 1:5}] ressemblance de quatre a. et voici lr.
\item[\vref{Ez 10:14}] Chaque a. avait quatre faces : La première face
\item[\vref{1 Co 2:14}] Or l'hom. a. ne comprend pas
\item[\vref{1 Co 15:44}] est semé corps a., il ressuscitera corps
\item[\vref{Ja 3:7}] ttes les espèces d'a. sauvages, d'oiseaux, de
\item[\vref{Ap 4:7}] Et le premier a. était semblable à
\end{listverse}

\ConcordanceEntry{Anne}
\vspace{-2mm}
\begin{listverse}
\item[\vref{1 S 1:20}] qq temps après, qu'A. conçut et enfanta
\end{listverse}
\begin{legend}
\NoAutoSpaceBeforeFDP{
\item la mère du prophète Samuel : 1 S 1:20
\item la prophétesse : Lu 2:36
\item le grand-prêtre : Lu 3:2; Jn 18:13; Ac 4:6
}
\end{legend}

\ConcordanceEntry{Anneau}
\vspace{-2mm}
\begin{listverse}
\item[\vref{Ge 24:22}] l'hom. prit un a. d'or, du poids
\item[\vref{Ge 41:42}] Pharaon ôta son a. de sa main
\item[\vref{Ex 32:2}] en pièces les a. d'or qui sont
\item[\vref{Est 3:10}] roi ôta son a. de sa main
\item[\vref{Est 8:2}] roi ôta son a., qu'il avait repris
\item[\vref{Job 42:11}] une pièce d'argent, et chacun un a. d'or.
\item[\vref{Pr 11:22}] est com. un a. d'or au groin
\item[\vref{Ez 16:12}] Je mis un a. à ton nez,
\item[\vref{Da 6:17}] scella de son a., et de l'anneau
\item[\vref{Lu 15:22}] et mettez-lui un a. au doigt, et
\item[\vref{Ja 2:2}] qui porte un a. d'or et un
\end{listverse}

\ConcordanceEntry{Annoncer}
\vspace{-2mm}
\begin{listverse}
\item[\vref{Ge 49:1}] et je vs. a. ce qui vs.
\item[\vref{1 S 16:4}] dirent : Ton arrivée a.-t-elle la paix ?
\item[\vref{1 R 8:20}] com. Yahweh l'avait a., et j'ai bâti
\item[\vref{Ps 51:17}] et ma bouche a. ta louange.
\item[\vref{Ps 92:3}] Afin d'a. chaque matin ta bonté, et ta
\item[\vref{Es 42:9}] je vs. en a. de nouvelles ; et
\item[\vref{Es 43:9}] d'entre eux a a. ces choses-là ? Et
\item[\vref{Es 61:3}] pour a. à ceux de Sion qui mènent
\item[\vref{Jé 33:3}] répondrai et je t'a. des choses grandes,
\item[\vref{Da 9:25}] la parole a a. que Jérus. sera
\item[\vref{Mt 28:8}] joie ; et coururent l'a. à ses disciples.
\item[\vref{Jn 16:13}] et il vs. a. les choses à
\item[\vref{Ac 4:29}] à tes serviteurs d'a. ta parole avec
\item[\vref{Ac 16:6}] Saint-Esprit lr. défendit d'a. la parole ds
\item[\vref{Ac 16:17}] et ils vs. a. la voie du
\item[\vref{Ac 17:30}] des temps d'ignorance, a. mntnt à ts
\item[\vref{1 Co 9:14}] que ceux qui a. l'Evangile vivent de
\item[\vref{1 Co 11:26}] cette coupe, vs. a. la mort du
\item[\vref{Col 4:3}] la parole, afin d'a. le mystère de
\item[\vref{1 Th 2:2}] Dieu, pour vs. a. l'Evangile de Dieu
\item[\vref{Hé 2:12}] disant : J'a. ton Nom à
\item[\vref{1 Pi 1:12}] que vs. ont a. mntnt ceux qui
\end{listverse}

\ConcordanceEntry{Annuler}
\vspace{-2mm}
\begin{listverse}
\item[\vref{No 30:9}] entendue, alors il a. le vœu par
\item[\vref{Mt 15:6}] Vous a. ainsi le commandement de Dieu par
\item[\vref{Ga 3:15}] un hom., n'est a. par personne, et
\item[\vref{Ga 3:17}] peut pas être a., et ainsi la
\end{listverse}

\ConcordanceEntry{Antéchrist, Antichrist}
\vspace{-2mm}
\begin{listverse}
\item[\vref{1 Jn 2:18}] avez entendu que l'a. viendra, il y
\item[\vref{1 Jn 2:22}] Christ ? Celui-là est l'a., qui nie le
\item[\vref{1 Jn 4:3}] c'est l'esprit de l'a., dont vs. avez
\item[\vref{2 Jn 1:7}] hom. est un séducteur et un a.
\end{listverse}

\ConcordanceEntry{Antioche}
\vspace{-2mm}
\begin{listverse}
\item[\vref{Ac 11:19}] Chypre, et à A., n'annonçant la parole
\item[\vref{Ac 11:26}] il l'amena à A.. Pendant tte une
\item[\vref{Ac 13:1}] qui était à A. des prophètes et
\item[\vref{Ac 13:14}] et arrivèrent à A., ville de Pisidie,
\item[\vref{Ac 14:19}] survinrent quelques Juifs d'A. et d'Icone qui
\item[\vref{Ac 15:22}] et d'envoyer à A. avec Paul et
\item[\vref{Ga 2:11}] Pierre vint à A., je lui résistai
\end{listverse}

\ConcordanceEntry{Antipas}
\vspace{-2mm}
\begin{listverse}
\item[\vref{Ap 2:13}] mm aux jours d'A., mon fidèle martyr,
\end{listverse}

\ConcordanceEntry{Apaiser}
\vspace{-2mm}
\begin{listverse}
\item[\vref{2 S 21:14}] cela, Dieu fut a. envers le pays.
\item[\vref{Pr 21:14}] fait en secret a. la colère, et
\item[\vref{Es 12:1}] ta colère s'est a., et tu m'as
\item[\vref{Es 66:13}] consolerai pour vs. a., com. quelqu'un que
\item[\vref{Za 6:8}] du nord ont a. mon Esprit ds
\item[\vref{Lu 8:24}] les flots qui s'a., et le calme
\item[\vref{Lu 18:13}] Ô Dieu, sois a. envers moi qui
\item[\vref{Ac 11:18}] ces choses, ils s'a., et ils glorifièrent
\item[\vref{Ac 19:36}] vs. devez vs. a. et ne rien
\end{listverse}

\ConcordanceEntry{Aplanir}
\vspace{-2mm}
\begin{listverse}
\item[\vref{Ex 34:1}] dit à Moïse : A.-toi deux tables
\item[\vref{Pr 15:19}] le chemin des hommes droits est a.
\item[\vref{Pr 16:17}] Le chemin a. des hommes droits,
\item[\vref{Es 28:25}] il en aura a. la surface, ne
\item[\vref{Es 40:3}] chemin de Yahweh, a. parmi les lieux
\item[\vref{Es 42:5}] étendus, qui a a. la terre avec
\item[\vref{Es 45:3}] dvt toi, et j'a. les lieux tortueux ;
\item[\vref{Za 4:7}] Zorobabel ? Tu seras a.. Il fera sortir
\item[\vref{Mt 3:3}] chemin du Seign., a. ses sentiers.
\item[\vref{Lu 3:5}] redressé, et les chemins raboteux seront a.
\end{listverse}

\ConcordanceEntry{Apollos}
\vspace{-2mm}
\begin{listverse}
\item[\vref{Ac 18:24}] un Juif, nommé A., originaire d'Alexandrie, hom.
\item[\vref{Ac 19:1}] Pendant qu'A. était à Corinthe,
\item[\vref{1 Co 1:12}] moi, je suis d'A. ! Et moi, de
\item[\vref{1 Co 3:6}] J'ai planté ; A. a arrosé ; mais
\end{listverse}

\ConcordanceEntry{Apollyon}
\vspace{-2mm}
\begin{listverse}
\item[\vref{Ap 9:11}] mais en grec son nom est A.
\end{listverse}

\ConcordanceEntry{Apostasie}
\vspace{-2mm}
\begin{listverse}
\item[\vref{Pr 1:32}] Car l'a. des stupides les tue, et la
\item[\vref{2 Th 2:3}] il faut que l'a. soit arrivée auparavant
\end{listverse}

\ConcordanceEntry{Apostolat}
\vspace{-2mm}
\begin{listverse}
\item[\vref{Ac 1:25}] et à cet a. que Judas a
\item[\vref{Ro 1:5}] la grâce et l'a., afin de porter
\item[\vref{1 Co 9:2}] sceau de mon a. ds le Seign.
\item[\vref{2 Co 12:12}] marques de mon a. se sont accomplies
\end{listverse}

\ConcordanceEntry{Apôtre}
\vspace{-2mm}
\begin{listverse}
\item[\vref{Mt 10:2}] noms des douze a. : Le premier est
\item[\vref{Mc 6:30}] Les a. se rassemblèrent auprès de Jésus, et
\item[\vref{Lu 11:49}] prophètes et des a., et ils tueront
\item[\vref{Jn 13:16}] son maître, ni l'a. plus grand que
\item[\vref{Ac 2:43}] de prodiges se faisaient par les a.
\item[\vref{Ac 5:18}] main sur les a., ils les jetèrent
\item[\vref{Ro 1:1}] appelé à être a., mis à part
\item[\vref{Ro 11:13}] Gentils, en tant qu'a. des Gentils, je
\item[\vref{1 Co 1:1}] Dieu à être a. de Jésus-Christ, et
\item[\vref{1 Co 4:9}] sommes les derniers a., com. des gens
\item[\vref{1 Co 9:1}] Ne suis-je pas a. ? Ne suis-je pas
\item[\vref{1 Co 12:28}] l'Eglise, premièrement des a., deuxièmement des prophètes,
\item[\vref{Ga 1:1}] Paul, a., non de la part des hommes,
\item[\vref{Ga 1:17}] ceux qui furent a. avant moi, mais
\item[\vref{Ga 2:8}] ds la charge d'a. pour les circoncis,
\item[\vref{Ep 4:11}] uns pour être a., les autres pour
\item[\vref{1 Ti 2:7}] été établi prédicateur, a. (je dis la
\item[\vref{Hé 3:1}] considérez attentivement Jésus-Christ, l'A. et le Grand-Prêtre
\item[\vref{1 Pi 1:1}] Pierre, a. de Jésus-Christ, à ceux qui sont
\item[\vref{Ap 2:2}] se disent être a. et qui ne
\item[\vref{Ap 21:14}] noms des douze a. de l'Agneau étaient
\end{listverse}

\ConcordanceEntry{Apparaître}
\vspace{-2mm}
\begin{listverse}
\item[\vref{Ge 35:1}] au Dieu qui t'a. lorsque tu fuyais
\item[\vref{Ex 3:16}] de Jacob m'est a., en disant : Certainement
\item[\vref{Lé 9:4}] à l'huile. Car aujourd'hui Yahweh vs. a.
\item[\vref{No 14:14}] et que tu a., ô Yahweh à
\item[\vref{Es 60:2}] lève sur toi, et sa gloire a. sur toi.
\item[\vref{Za 9:14}] Yahweh au-dessus d'eux a., et ses dards
\item[\vref{Mc 16:9}] de la semaine, a. d'abord à Marie
\item[\vref{Lu 9:8}] d'autres, qu'Elie était a. ; et d'autres, que
\item[\vref{Ac 7:2}] Dieu de gloire a. à notre père
\item[\vref{Ac 23:11}] suivante, le Seign. a. à Paul, et
\item[\vref{1 Co 15:8}] ts, il m'est a. à moi aussi
\item[\vref{Col 3:4}] est votre vie, a., alors vs. paraîtrez
\item[\vref{Hé 9:28}] péchés de plusieurs, a. sans péché une
\item[\vref{1 Pi 5:4}] le souv. Pasteur a., vs. obtiendrez la
\item[\vref{1 Jn 2:28}] que qnd il a. ns. ayons de
\item[\vref{1 Jn 3:2}] Fils de Dieu a., ns. serons semblables
\end{listverse}

\ConcordanceEntry{Apparence}
\vspace{-2mm}
\begin{listverse}
\item[\vref{De 1:17}] point d'égard à l'a. de la personne
\item[\vref{1 S 16:7}] attention à son a. ni à la
\item[\vref{Ps 82:2}] aurez-vs. égard à l'a. de la personne
\item[\vref{Pr 18:5}] d'avoir égard à l'a. de la personne
\item[\vref{Pr 24:23}] d'avoir égard à l'a. des personnes ds
\item[\vref{Es 11:3}] jugera point sur l'a. et il ne
\item[\vref{Es 52:14}] hom., et son a. plus que celle
\item[\vref{Es 53:2}] le regardions, ni a. qui ns. le
\item[\vref{Da 2:31}] toi et son a. était terrible.
\item[\vref{Da 10:16}] quelqu'un qui avait l'a. des fils de
\item[\vref{Mal 2:9}] avez égard à l'a. des personnes qnd
\item[\vref{Mt 22:16}] regardes pas à l'a. des hommes.
\item[\vref{Lu 20:47}] qui font pour l'a. de longues prières.
\item[\vref{Jn 7:24}] pas selon les a., mais jugez selon
\item[\vref{Ac 10:34}] pas égard à l'a. des personnes,
\item[\vref{Ro 2:11}] pas égard à l'a. des personnes.
\item[\vref{Ro 2:28}] en a les a. ; et la circoncision,
\item[\vref{2 Co 5:12}] se glorifient de l'a., et non du
\item[\vref{Ph 2:8}] étant trouvé en a. com. un hom.,
\item[\vref{Col 2:18}] la course, sous l'a. d'humilité d'esprit et
\item[\vref{Col 2:23}] ont pourtant qq a. de sagesse en
\item[\vref{2 Ti 3:5}] ayant l'a. de la piété, mais en ayant
\item[\vref{Ja 2:1}] pas égard à l'a. des personnes.
\end{listverse}

\ConcordanceEntry{Appartenir}
\vspace{-2mm}
\begin{listverse}
\item[\vref{Ge 15:13}] qui ne lr. a. point, et qu'ils
\item[\vref{Ge 25:5}] Abraham donna tt ce qui lui a. à Isaac.
\item[\vref{Ex 13:12}] auras ; les mâles a. à Yahweh.
\item[\vref{Job 12:13}] force ; à lui a. le conseil et
\item[\vref{Ps 50:11}] qui se meut ds les champs m'a.
\item[\vref{Mt 6:13}] c'est à toi qu'a., ds ts les
\item[\vref{Mt 25:25}] Voici, tu as ici ce qui t'a.
\item[\vref{Lu 6:20}] car le Royaume de Dieu vs. a. !
\item[\vref{Lu 15:12}] de bien qui m'a.. Et il lr.
\item[\vref{Lu 19:42}] les choses qui a. à ta paix !
\item[\vref{Jn 10:12}] berger, à qui n'a. pas les brebis,
\item[\vref{Ro 8:9}] l'Esprit de Christ, il ne lui a. pas.
\item[\vref{1 Co 6:20}] corps et ds votre esprit, qui a. à Dieu.
\item[\vref{2 Ti 2:19}] ceux qui lui a. ; et : Quiconque invoque
\item[\vref{Tit 2:14}] peuple qui lui a. en propre, et
\item[\vref{Hé 10:30}] que la vengeance a., et je le
\item[\vref{Hé 11:9}] lui avait pas a., demeurant sous des
\item[\vref{1 Pi 4:11}] par Jésus-Christ, auquel a. la gloire et
\item[\vref{2 Pi 1:3}] tt ce qui a. à la vie
\item[\vref{Ap 19:1}] et la puissance a. au Seign., notre
\end{listverse}

\ConcordanceEntry{Appel}
\vspace{-2mm}
\begin{listverse}
\item[\vref{1 Co 7:17}] Dieu, chacun selon l'a. qu'il a reçu
\end{listverse}

\ConcordanceEntry{Appeler}
\vspace{-2mm}
\begin{listverse}
\item[\vref{Ge 1:5}] Dieu a. la lumière jour, et il appela
\item[\vref{Ge 3:20}] Et Adam a. sa fem. Eve, parce qu'elle a
\item[\vref{Es 42:6}] Yahweh, je t'ai a. en justice, et
\item[\vref{Es 65:12}] parce que j'ai a., et vs. n'avez
\item[\vref{Mt 9:13}] suis pas venu a. à la repentance
\item[\vref{Mt 23:8}] vs. faites pas a., notre maître ; car
\item[\vref{Ac 25:12}] répondit : En as-tu a. à César ? Tu
\item[\vref{Ro 8:30}] les a aussi a. ; et ceux qu'il
\item[\vref{Ro 9:11}] des œuvres, mais par celui qui a.,
\item[\vref{Ro 9:25}] dit ds Osée : J'a. mon peuple celui
\item[\vref{1 Co 7:20}] il était qnd il a été a.
\item[\vref{1 Th 2:12}] Dieu, qui vs. a. à son Royaume
\item[\vref{1 Th 4:7}] ns. a pas a. à l'impureté, mais
\item[\vref{1 Th 5:24}] Celui qui vs. a. est fidèle, c'est
\item[\vref{2 Ti 1:9}] et ns. a a. par une sainte
\item[\vref{Hé 2:11}] honte de les a. ses frères,
\item[\vref{Hé 5:4}] celui qui est a. de Dieu, com.
\item[\vref{Ja 5:14}] est-il malade ? Qu'il a. les anciens de
\item[\vref{1 Pi 2:21}] Car vs. êtes a. à cela, puisque
\item[\vref{1 Jn 3:1}] que ns. soyons a. enfants de Dieu !
\item[\vref{Ap 19:9}] ceux qui sont a. au festin des
\item[\vref{Ap 19:13}] et son Nom s'a. LA PAROLE DE
\end{listverse}

\ConcordanceEntry{Appesantir}
\vspace{-2mm}
\begin{listverse}
\item[\vref{1 S 5:6}] main de Yahweh s'a. sur les Asdodiens
\item[\vref{Ps 32:4}] nuit ta main s'a. sur moi, ma
\item[\vref{Es 47:6}] tu as durement a. ton joug sur
\item[\vref{Za 7:11}] arrière, ils ont a. leurs oreilles pour
\item[\vref{Mt 26:43}] encore endormis ; car leurs yeux étaient a.
\item[\vref{Lu 21:34}] cœurs ne soient a. par la gourmandise
\end{listverse}

\ConcordanceEntry{Appliquer}
\vspace{-2mm}
\begin{listverse}
\item[\vref{Ex 30:16}] expiations, et tu l'a. à l'œuvre de
\item[\vref{1 Ch 22:19}] Maintenant dc, a. vos cœurs et
\item[\vref{Job 38:5}] Ou qui a a. sur elle le
\item[\vref{Pr 22:17}] des sages, et a. ton cœur à
\item[\vref{Ec 1:13}] Et j'ai a. mon cœur à rechercher et à
\item[\vref{Da 10:12}] où tu as a. ton cœur à
\item[\vref{Lu 1:1}] plusieurs se sont a. à mettre par
\item[\vref{1 Ti 4:13}] A.-toi à la lecture, à l'exhortation
\item[\vref{Tit 3:8}] soin principalement de s'a. à pratiquer les
\item[\vref{2 Pi 3:14}] attendant ces choses, a.-vs. à être
\end{listverse}

\ConcordanceEntry{Apprendre}
\vspace{-2mm}
\begin{listverse}
\item[\vref{De 18:9}] te donne, tu n'a. point à faire
\item[\vref{De 31:13}] l'entendront, et ils a. à craindre Yahweh,
\item[\vref{Jos 4:22}] Vous l'a. à vos enfants, en lr. disant :
\item[\vref{Ps 119:7}] cœur qnd j'aurai a. les ordonnances de
\item[\vref{Pr 19:27}] ce qui pourrait t'a. à te détourner
\item[\vref{Es 1:17}] A. à bien faire, recherchez la droiture,
\item[\vref{Da 8:19}] Voici, je vais t'a. ce qui arrivera
\item[\vref{Mt 3:7}] qui vs. a a. à fuir la
\item[\vref{1 Co 14:35}] si elles veulent a. qq chose, qu'elles
\item[\vref{Hé 5:8}] il a pourtant a. l'obéissance par les
\item[\vref{Ap 14:3}] personne ne pouvait a. le cantique, si
\end{listverse}

\ConcordanceEntry{Approcher}
\vspace{-2mm}
\begin{listverse}
\item[\vref{Ex 3:5}] Et Dieu dit : N'a. point d'ici ; déchausse
\item[\vref{Ex 28:1}] Et toi, fais a. de toi Aaron,
\item[\vref{Ex 29:4}] Puis, tu feras a. Aaron et ses
\item[\vref{Jos 7:18}] qnd il fit a. la maison de
\item[\vref{Ps 73:28}] Mais pour moi, m'a. de Dieu, c'est
\item[\vref{Es 58:2}] prennent plaisir à s'a. de Dieu, et
\item[\vref{Es 65:5}] dit : Retire-toi, ne m'a. pas, car je
\item[\vref{Joë 3:9}] hommes vaillants ! Qu'ils s'a., et qu'ils montent,
\item[\vref{Mt 4:3}] le tentateur, s'étant a., lui dit : Si
\item[\vref{Mt 9:20}] depuis douze ans, s'a. par-derrière, et toucha
\item[\vref{Mt 24:1}] temple, ses disciples s'a. de lui pour
\item[\vref{Mc 14:42}] allons ; voici, celui qui me trahit s'a.
\item[\vref{Ac 8:29}] Philippe : Avance, et a.-toi de ce
\item[\vref{Ro 13:12}] et le jour a.. Rejetons dc les
\item[\vref{Hé 4:16}] A. dc avec assurance du trône de
\item[\vref{Hé 10:22}] a.-ns. de lui avec un cœur
\item[\vref{Hé 12:18}] vs. êtes pas a. d'une montagne qu'on
\item[\vref{Ja 4:8}] A.-vs. de Dieu, et il s'approchera
\end{listverse}

\ConcordanceEntry{Approuver}
\vspace{-2mm}
\begin{listverse}
\item[\vref{2 S 3:36}] peuple l'entendit et l'a., et tt le
\item[\vref{Ac 2:22}] de Nazareth, hom. a. de Dieu parmi
\item[\vref{Ro 1:32}] mais encore ils a. ceux qui les
\item[\vref{Ro 7:15}] Car je n'a. pas ce que je fais, puisque
\item[\vref{Ro 14:18}] et il est a. des hommes.
\item[\vref{1 Co 11:19}] sont dignes d'être a. soient reconnus parmi
\item[\vref{2 Co 10:18}] lui-mm qui est a., mais celui que
\item[\vref{2 Ti 2:15}] de te rendre a. dvt Dieu, com.
\item[\vref{2 Ti 4:17}] prédication soit pleinement a. et que ts
\end{listverse}

\ConcordanceEntry{Appui}
\vspace{-2mm}
\begin{listverse}
\item[\vref{2 S 22:19}] ma détresse, mais Yahweh fut mon a.
\item[\vref{Job 6:13}] secours, et tt a. n'est-il pas éloigné
\item[\vref{Ps 18:19}] ma détresse, mais Yahweh fut mon a.
\item[\vref{Ps 69:3}] bourbier profond, sans a. ; je suis entré
\item[\vref{Es 3:1}] de Juda tt a. et tte ressource,
\item[\vref{Es 59:16}] et sa propre justice lui sert d'a.
\item[\vref{1 Ti 3:15}] la colonne et l'a. de la vérité.
\end{listverse}

\ConcordanceEntry{Appuyer}
\vspace{-2mm}
\begin{listverse}
\item[\vref{2 R 5:18}] prosterner et qu'il s'a. sur ma main,
\item[\vref{2 Ch 13:18}] parce qu'ils s'étaient a. sur Yahweh, le
\item[\vref{2 Ch 16:7}] que tu t'es a. sur le roi
\item[\vref{Es 10:20}] de Jacob ne s'a. plus sur celui
\item[\vref{Es 31:1}] l'aide, et qui s'a. sur les chevaux,
\item[\vref{Es 50:10}] Yahweh, et qu'il s'a. sur son Dieu.
\item[\vref{Es 59:4}] la vérité ; ils s'a. sur des choses
\item[\vref{Ez 33:26}] Vous vs. a. sur votre épée,
\item[\vref{Hé 11:21}] et adora Dieu, a. sur l'extrémité de
\end{listverse}

\ConcordanceEntry{Aqueduc}
\vspace{-2mm}
\begin{listverse}
\item[\vref{2 R 18:17}] ils s'arrêtèrent à l'a. de l'étang supérieur,
\item[\vref{2 R 20:20}] fit l'étang et l'a. par lequel il
\end{listverse}

\ConcordanceEntry{Aquilas, Priscille}
\vspace{-2mm}
\begin{listverse}
\item[\vref{Ac 18:2}] un Juif, nommé A., originaire du Pont,
\item[\vref{Ro 16:3}] Saluez P. et Aquilas, mes compagnons d'œuvre en
\end{listverse}

\ConcordanceEntry{Arabe, Arabie}
\vspace{-2mm}
\begin{listverse}
\item[\vref{2 Ch 26:7}] et contre les A. qui habitaient à
\item[\vref{Né 4:7}] et Tobija, les A., les Ammonites et
\item[\vref{Es 13:20}] génération ; mm les A. n'y dresseront point
\item[\vref{Es 21:13}] Prophétie contre l'A.. Vous passerez pêle-mêle
\item[\vref{Jé 25:24}] ts les rois d'A., et à ts
\item[\vref{Ga 1:17}] je partis pour l'A., puis je revins
\item[\vref{Ga 4:25}] une montagne en A. correspondant à la
\end{listverse}

\ConcordanceEntry{Ararat}
\vspace{-2mm}
\begin{listverse}
\item[\vref{Ge 8:4}] mois, l'arche s'arrêta sur les montagnes d'A.
\item[\vref{2 R 19:37}] sauvèrent au pays d'A. ; et Esar-Haddon, son
\item[\vref{Jé 51:27}] elle les royaumes d'A., de Minni et
\end{listverse}

\ConcordanceEntry{Arbre}
\vspace{-2mm}
\begin{listverse}
\item[\vref{Ge 2:9}] la terre tt a. désirable à la
\item[\vref{Ge 2:16}] mangeras de tt a. du jardin.
\item[\vref{Ge 2:17}] Mais quant à l'a. de la connaissance
\item[\vref{Ge 3:22}] ne prenne de l'a. de vie, et
\item[\vref{De 20:19}] détruiras point les a. à coups de
\item[\vref{Ps 1:3}] est com. un a. planté près des
\item[\vref{Pr 3:18}] Elle est l'a. de vie pour
\item[\vref{Pr 13:12}] un désir accompli est com. un a. de vie.
\item[\vref{Ec 11:3}] et qnd un a. tombe, au sud
\item[\vref{Es 57:5}] dieux, sous tt a. vert ; égorgeant les
\item[\vref{Jé 11:19}] en disant : Détruisons l'a. avec son fruit !
\item[\vref{Da 4:14}] parla ainsi : Coupez l'a., et ébranchez-le ! Secouez
\item[\vref{Mt 3:10}] la racine des a. ; c'est pourquoi tt
\item[\vref{Mt 7:17}] Ainsi tt bon a. porte de bons
\item[\vref{Mt 12:33}] Ou dites que l'a. est bon et
\item[\vref{Ap 2:7}] à manger de l'a. de vie, qui
\item[\vref{Ap 8:7}] le tiers des a. fut brûlé, et
\item[\vref{Ap 9:4}] ni à aucun a., mais seulement aux
\item[\vref{Ap 22:2}] du fleuve, était l'a. de vie, portant
\item[\vref{Ap 22:14}] d'avoir droit à l'a. de vie, et
\end{listverse}

\ConcordanceEntry{Arc}
\vspace{-2mm}
\begin{listverse}
\item[\vref{1 S 2:4}] L'a. des puissants est brisé, mais ceux
\item[\vref{2 S 1:18}] le cantique de l'a. : Il est écrit
\item[\vref{2 S 1:22}] L'a. de Jonathan ne revenait jamais sans
\item[\vref{2 R 9:24}] Mais Jéhu saisit l'a. de sa main,
\item[\vref{Ps 7:13}] il bande son a., et vise.
\item[\vref{Ps 11:2}] les méchants bandent l'a., ils ajustent lr.
\item[\vref{Ps 44:7}] point en mon a., et ce n'est
\item[\vref{Ps 78:57}] tournèrent com. un a. trompeur.
\item[\vref{Za 9:13}] Juda com. mon a., et que j'aurai
\item[\vref{Ap 6:2}] dessus avait un a., et il lui
\end{listverse}

\ConcordanceEntry{Arc-en-ciel}
\vspace{-2mm}
\begin{listverse}
\item[\vref{Ge 9:13}] J'ai placé mon arc ds la nuée, et il servira
\item[\vref{Ap 4:3}] était environné d'un a. semblable à de
\item[\vref{Ap 10:1}] sa tête était l'a., son visage était
\end{listverse}

\ConcordanceEntry{Archange}
\vspace{-2mm}
\begin{listverse}
\item[\vref{1 Th 4:16}] et une voix d'a., et avec la
\item[\vref{Jud 1:9}] Or l'a. Michel, lorsqu'il contestait avec le diable
\end{listverse}

\ConcordanceEntry{Arche}
\vspace{-2mm}
\begin{listverse}
\item[\vref{Ex 25:10}] ils feront une a. de bois d'acacia ;
\item[\vref{Ex 37:1}] Puis Betsaleel fit l'a. de bois d'acacia.
\item[\vref{Ex 40:3}] tu y mettras l'a. du témoignage, au-dvt
\item[\vref{No 10:35}] qu'au départ de l'a., Moïse disait : Lève-toi,
\item[\vref{Jos 3:6}] en disant : Portez l'a. de l'alliance, et
\item[\vref{Jos 6:11}] L'a. de Yahweh fit ainsi le tour
\item[\vref{1 S 5:1}] Les Philistins prirent l'a. de Dieu, et
\item[\vref{1 S 5:11}] en disant : Renvoyez l'a. du Dieu d'Israël,
\item[\vref{2 S 15:25}] à Tsadok : Rapporte l'a. de Dieu ds
\item[\vref{1 R 3:15}] se tint dvt l'A. de l'alliance de
\item[\vref{1 Ch 13:7}] Ils mirent l'a. de Dieu sur
\item[\vref{1 Ch 13:10}] sa main sur l'a.. Uzza mourut en
\item[\vref{2 Ch 6:41}] repos, toi et l'a. de ta puissance.
\item[\vref{2 Ch 35:3}] à Yahweh : Mettez l'a. sainte ds la
\item[\vref{Jé 3:16}] parlera plus de l'a. de l'alliance de
\item[\vref{Hé 9:5}] Et au-dessus de l'a. étaient les chérubins
\item[\vref{Ap 11:19}] le ciel, et l'a. de son alliance
\end{listverse}

\ConcordanceEntry{Arche de Noé}
\vspace{-2mm}
\begin{listverse}
\item[\vref{Ge 6:14}] Fais-toi une a. de bois de
\item[\vref{Ge 7:16}] ordonné à Noé, puis Yahweh ferma l'a. sur lui.
\item[\vref{Ge 8:4}] du septième mois, l'a. s'arrêta sur les
\item[\vref{Ge 8:16}] Sors de l'a., toi et ta
\item[\vref{Mt 24:38}] jusqu'au jour où Noé entra ds l'a. ;
\item[\vref{Hé 11:7}] craignit, et bâtit l'a. pour la conservation
\item[\vref{1 Pi 3:20}] Noé, tandis que l'a. se préparait ds
\end{listverse}

\ConcordanceEntry{Archélaüs}
\vspace{-2mm}
\begin{listverse}
\item[\vref{Mt 2:22}] il eut appris qu'A. régnait en Judée,
\end{listverse}

\ConcordanceEntry{Archippe}
\vspace{-2mm}
\begin{listverse}
\item[\vref{Col 4:17}] Et dites à A. : Prends garde au
\item[\vref{Phm 1:2}] notre bien-aimée, à A., notre compagnon de
\end{listverse}

\ConcordanceEntry{Architecte}
\vspace{-2mm}
\begin{listverse}
\item[\vref{2 R 12:11}] et pour les a. qui travaillaient à
\item[\vref{Ps 118:22}] Pierre que les a. avaient rejetée, est
\item[\vref{1 Co 3:10}] com. un sage a., et un autre
\item[\vref{Hé 11:10}] dont Dieu est l'a. et le constructeur.
\end{listverse}

\ConcordanceEntry{Aréopage}
\vspace{-2mm}
\begin{listverse}
\item[\vref{Ac 17:22}] au milieu de l'A., lr. dit : Hommes
\end{listverse}

\ConcordanceEntry{Argent}
\vspace{-2mm}
\begin{listverse}
\item[\vref{Ge 37:28}] pour vingt pièces d'a. aux Ismaélites, qui
\item[\vref{Ge 42:25}] et qu'on remette l'a. de chacun d'eux
\item[\vref{Ge 47:14}] Joseph amassa tt l'a. qui se trouva
\item[\vref{Ex 22:25}] tu prêtes de l'a. à mon peuple,
\item[\vref{2 R 5:26}] de prendre de l'a., de prendre des
\item[\vref{Esd 4:5}] engagé à prix d'a. des conseillers pour
\item[\vref{Job 22:25}] ton or, et l'a. de tes richesses.
\item[\vref{Ps 12:7}] pures, c'est un a. éprouvé sur terre
\item[\vref{Ps 15:5}] d'intérêt de son a. ; et qui n'accepte
\item[\vref{Pr 8:10}] plutôt que de l'a., la connaissance plutôt
\item[\vref{Pr 27:21}] est pour éprouver l'a., et le creuset
\item[\vref{Ec 5:9}] Celui qui aime l'a. n'est point rassasié
\item[\vref{Ec 10:19}] les vivants, et l'a. répond à tt.
\item[\vref{Es 1:22}] Ton a. s'est changé en scories ; ton breuvage
\item[\vref{Es 52:3}] et vs. serez aussi rachetés sans a.
\item[\vref{Es 55:1}] qui n'avez pas d'a., venez, achetez et
\item[\vref{Jé 6:30}] les appelle de l'a. réprouvé, car Yahweh
\item[\vref{Jé 10:4}] l'embellit avec de l'a. et de l'or,
\item[\vref{Da 2:45}] fer, l'airain, l'argile, l'a. et l'or. Le
\item[\vref{Joë 3:5}] avez pris mon a. et mon or ;
\item[\vref{Mi 3:11}] devinent pour de l'a., ensuite ils s'appuient
\item[\vref{Ha 2:19}] couverte d'or et d'a., et il n'y
\item[\vref{Ag 2:8}] L'a. est à moi, et l'or est
\item[\vref{Za 11:12}] pesèrent pour mon salaire trente pièces d'a.
\item[\vref{Za 11:13}] les trente pièces d'a., et les jetai
\item[\vref{Mal 3:3}] raffine et purifie l'a. ; il nettoiera les
\item[\vref{Mt 10:9}] ni or, ni a., ni monnaie ds
\item[\vref{Mt 25:18}] et y cacha l'a. de son maître.
\item[\vref{Mt 26:15}] Et ils lui comptèrent trente pièces d'a.
\item[\vref{Mt 28:12}] une forte somme d'a. aux soldats,
\item[\vref{Mc 12:41}] y mettait de l'a.. Plusieurs riches y
\item[\vref{Lu 9:3}] ni pain, ni a. ; et n'ayez pas
\item[\vref{Lu 22:5}] et convinrent de lui donner de l'a.
\item[\vref{Ac 3:6}] Je n'ai ni a., ni or ; mais
\item[\vref{Ac 8:20}] dit : Que ton a. périsse avec toi,
\item[\vref{1 Co 3:12}] de l'or, de l'a., des pierres précieuses,
\item[\vref{1 Ti 6:10}] Car l'amour de l'a. est la racine
\item[\vref{2 Ti 3:2}] d'eux-mêmes, amis de l'a., fanfarons, orgueilleux, blasphémateurs,
\item[\vref{Ja 5:3}] or et votre a. sont rouillés ; et
\end{listverse}

\ConcordanceEntry{Argile}
\vspace{-2mm}
\begin{listverse}
\item[\vref{Ge 11:3}] pierre, et le bitume lr. servit d'a.
\item[\vref{Ps 22:16}] se dessèche com. l'a., et ma langue
\item[\vref{Es 29:16}] pas réputé com. l'a. d'un potier ? Même
\item[\vref{Es 45:9}] de terre ! Mais l'a. dira-t-elle à celui
\item[\vref{Es 64:7}] Père ; ns. sommes l'a., et c'est toi
\item[\vref{Jé 18:6}] Yahweh. Voici, com. l'a. est ds la
\item[\vref{Da 2:33}] partie de fer et en partie d'a.
\item[\vref{Ro 9:20}] Dieu ? Le vase d'a. dira-t-il à celui
\end{listverse}

\ConcordanceEntry{Aride}
\vspace{-2mm}
\begin{listverse}
\item[\vref{Job 30:3}] rongent les lieux a. depuis longtemps désolés
\item[\vref{Ps 63:2}] sur cette terre a., desséchée, et sans
\item[\vref{Es 35:1}] et le lieu a. seront ds la
\item[\vref{Es 40:3}] parmi les lieux a. un chemin pour
\item[\vref{Es 51:3}] et sa terre a. à un jardin
\item[\vref{Mt 12:43}] par des lieux a., cherchant du repos,
\end{listverse}

\ConcordanceEntry{Ariel}
\vspace{-2mm}
\begin{listverse}
\item[\vref{Es 29:1}] Malheur à A., à Ariel, la
\end{listverse}

\ConcordanceEntry{Arimathée}
\vspace{-2mm}
\begin{listverse}
\item[\vref{Mt 27:57}] un hom. riche d'A., appelé Joseph, qui
\item[\vref{Lu 23:51}] autres ; il était d'A., ville des Juifs,
\item[\vref{Jn 19:38}] ces choses, Joseph d'A., qui était disciple
\end{listverse}

\ConcordanceEntry{Aristarque}
\vspace{-2mm}
\begin{listverse}
\item[\vref{Ac 19:29}] enlevèrent Gaïus et A., Macédoniens, compagnons de
\item[\vref{Ac 20:4}] Sopater de Bérée, A. et Second de
\item[\vref{Ac 27:2}] ayant avec ns. A., un Macédonien de
\end{listverse}

\ConcordanceEntry{Arme}
\vspace{-2mm}
\begin{listverse}
\item[\vref{Né 4:17}] et de l'autre ils tenaient une a.
\item[\vref{Es 54:17}] Aucune a. forgée contre toi ne réussira, et
\item[\vref{Lu 11:22}] enlève ttes ses a. ds lesquelles il
\item[\vref{Ro 13:12}] soyons revêtus des a. de lumière.
\item[\vref{2 Co 10:4}] Car les a. de notre guerre ne sont pas
\item[\vref{Ep 6:11}] de ttes les a. de Dieu, afin
\end{listverse}

\ConcordanceEntry{Armée}
\vspace{-2mm}
\begin{listverse}
\item[\vref{Jos 5:14}] le Chef de l'a. de Yahweh, je
\item[\vref{1 S 17:26}] incirconcis, pour insulter l'a. du Dieu vivant ?
\item[\vref{Ps 27:3}] Si tte une a. campait contre moi,
\item[\vref{Ps 33:16}] par une grande a., l'hom. puissant n'échappe
\item[\vref{Ps 68:12}] de bonnes nouvelles sont une grande a.
\item[\vref{Ps 103:21}] vs. ttes ses a., qui êtes ses
\item[\vref{Ps 110:3}] tu rassembles ton a. ; avec des ornements
\item[\vref{Es 6:3}] est Yahweh des a. ! Toute la terre
\item[\vref{Ez 37:10}] pieds ; c'était une a. extrêmement grande.
\item[\vref{Joë 2:11}] voix dvt son a. ; parce que son
\item[\vref{Lu 2:13}] une multitude de l'a. céleste, louant Dieu,
\item[\vref{Lu 21:20}] environnée par les a., sachez alors que
\item[\vref{Hé 11:34}] en fuite des a. étrangères.
\item[\vref{Ap 19:19}] terre, et leurs a. rassemblées pour faire
\end{listverse}

\ConcordanceEntry{Armure}
\vspace{-2mm}
\begin{listverse}
\item[\vref{2 S 21:16}] d'airain, et il était ceint d'une a. neuve.
\item[\vref{1 R 20:11}] qui endosse une a. ne se glorifie
\item[\vref{Jé 46:4}] vos casques, polissez vos lances, revêtez l'a. !
\item[\vref{Jé 51:3}] s'élève ds son a. ; et n'épargnez pas
\end{listverse}

\ConcordanceEntry{Aromates}
\vspace{-2mm}
\begin{listverse}
\item[\vref{Ex 30:34}] Moïse : Prends des a., à savoir de
\item[\vref{1 R 10:10}] très grande quantité d'a. et des pierres
\item[\vref{Est 2:12}] mois avec des a. et d'autres préparatifs
\item[\vref{Ca 3:6}] de ts les a. du parfumeur ?
\item[\vref{Ez 27:22}] ts les meilleurs a., de tte sorte
\item[\vref{Mc 16:1}] Salomé, achetèrent des a. pour venir l'embaumer.
\item[\vref{Jn 19:40}] linges, avec des a., com. les Juifs
\end{listverse}

\ConcordanceEntry{Arracher}
\vspace{-2mm}
\begin{listverse}
\item[\vref{Ge 8:11}] d'olivier qu'elle avait a. ; et Noé connut
\item[\vref{1 R 11:31}] Voici, je vais a. le royaume d'entre
\item[\vref{2 Ch 7:20}] je vs. a. de mon pays que je vs.
\item[\vref{Ec 3:2}] un temps pour a. ce qui est
\item[\vref{Jé 1:10}] pour que tu a. et que tu
\item[\vref{Jé 24:6}] les planterai et je ne les a. pas.
\item[\vref{Mt 5:29}] occasion de chute, a.-le et jette-le
\item[\vref{Mt 12:1}] se mirent à a. des épis et
\item[\vref{Mt 13:41}] ses anges qui a. de son Royaume
\item[\vref{Ga 1:4}] afin de ns. a. du présent siècle
\end{listverse}

\ConcordanceEntry{Arrêt}
\vspace{-2mm}
\begin{listverse}
\item[\vref{Es 10:32}] Encore un jour d'a. à Nob, et
\item[\vref{Es 24:3}] pillé car Yahweh a prononcé cet a.
\item[\vref{Da 2:14}] du conseil et l'a. donné à Arjoc,
\end{listverse}

\ConcordanceEntry{Arrêter}
\vspace{-2mm}
\begin{listverse}
\item[\vref{Ge 19:17}] toi, et ne t'a. en aucun endroit
\item[\vref{Ge 41:32}] la chose est a. de la part
\item[\vref{Ex 3:5}] où tu es a. est une terre
\item[\vref{Ex 14:13}] Ne craignez point, a.-vs. et voyez
\item[\vref{Jos 4:7}] Jourdain ont été a. ; c'est pourquoi ces
\item[\vref{Jos 10:12}] présence d'Israël : Soleil, a.-toi sur Gabaon,
\item[\vref{Ps 1:1}] et qui ne s'a. pas sur la
\item[\vref{Ps 46:11}] A., et sachez que je suis Dieu :
\item[\vref{Es 62:7}] Et ne vs. a. pas de l'invoquer
\item[\vref{Jon 1:15}] Et la fureur de la mer s'a.
\item[\vref{Mc 14:48}] des épées et des bâtons, pour m'a.
\item[\vref{Jn 1:33}] l'Esprit descendre et s'a., c'est celui qui
\item[\vref{Ac 4:3}] les ayant fait a., ils les mirent
\item[\vref{Ac 8:38}] Il fit a. le char ; Philippe et l'eunuque descendirent
\item[\vref{Ac 12:3}] il fit aussi a. Pierre. C'était pendant
\item[\vref{Ro 9:11}] que le dessein a. selon l'élection de
\item[\vref{Ga 5:7}] Qui vs. a a. pour vs. empêcher
\item[\vref{Ap 12:4}] Puis le dragon s'a. dvt la fem.
\end{listverse}

\ConcordanceEntry{Arrhes (de l'Esprit)}
\vspace{-2mm}
\begin{listverse}
\item[\vref{2 Co 1:22}] a donné les a. de l'Esprit ds
\item[\vref{2 Co 5:5}] a donné les a. de l'Esprit.
\end{listverse}

\ConcordanceEntry{Arrière}
\vspace{-2mm}
\begin{listverse}
\item[\vref{Ge 19:26}] Lot regarda en a., et elle devint
\item[\vref{Ps 78:57}] se retirèrent en a. et furent infidèles
\item[\vref{Ps 114:3}] et s'enfuit, le Jourdain retourna en a.
\item[\vref{Es 1:4}] d'Israël, ils se sont retirés en a.
\item[\vref{Es 38:8}] dix degrés en a. avec le soleil
\item[\vref{Es 53:3}] caché notre visage a. de lui, tant
\item[\vref{Os 6:3}] la pluie de l'a.-saison qui arrose
\item[\vref{Mt 16:23}] dit à Pierre : A. de moi, Satan !
\item[\vref{Mt 24:18}] retourne pas en a. pour prendre ses
\item[\vref{Lu 9:62}] et regarde en a., n'est pas bien
\item[\vref{Ph 3:14}] qui sont en a., et me portant
\end{listverse}

\ConcordanceEntry{Arrogance}
\vspace{-2mm}
\begin{listverse}
\item[\vref{Ps 10:4}] méchant marchant avec a. ne cherche rien !
\item[\vref{Pr 8:13}] haine l'orgueil et l'a., la voie de
\item[\vref{Es 2:17}] Et l'a. des hommes sera humiliée, et les
\item[\vref{Da 8:25}] Il aura de l'a. ds le cœur,
\item[\vref{So 2:10}] usé d'insultes et d'a., contre le peuple
\end{listverse}

\ConcordanceEntry{Arroser}
\vspace{-2mm}
\begin{listverse}
\item[\vref{Ge 2:6}] la terre qui a. tte la surface
\item[\vref{De 11:10}] ta semence, et l'a. avec ton pied,
\item[\vref{Job 37:11}] nuages à force d'a., il écarte les
\item[\vref{Ps 6:7}] mon lit est a. de mes pleurs.
\item[\vref{Pr 11:25}] et celui qui a. abondamment sera lui-mm
\item[\vref{Ec 2:6}] réservoirs d'eaux pour a. la forêt où
\item[\vref{Es 27:3}] la garde, je l'a. à chaque instant,
\item[\vref{Es 55:10}] retournent plus, mais a. la terre, et
\item[\vref{Es 58:11}] com. un jardin a., et com. une
\item[\vref{Da 5:21}] son corps fut a. de la rosée
\item[\vref{1 Co 3:6}] planté ; Apollos a a. ; mais c'est Dieu
\end{listverse}

\ConcordanceEntry{Artaxerxès}
\vspace{-2mm}
\begin{listverse}
\item[\vref{Esd 4:7}] Et du temps d'A., Bischlam, Mithredath, Thabeel,
\item[\vref{Esd 6:14}] de Darius, et d'A., roi de Perse.
\item[\vref{Né 2:1}] année du roi A., com. le vin
\end{listverse}

\ConcordanceEntry{Asa}
\vspace{-2mm}
\begin{listverse}
\item[\vref{1 R 15:9}] Jéroboam, roi d'Israël, A. commença à régner
\item[\vref{1 R 15:13}] idole pour Astarté. A. mit en pièces
\item[\vref{1 R 15:16}] eut guerre entre A. et Baescha, roi
\item[\vref{2 Ch 14:7}] Or A. avait ds son armée trois cent
\item[\vref{2 Ch 14:11}] les Ethiopiens dvt A. et dvt Juda ;
\item[\vref{2 Ch 16:12}] A. fut malade des pieds la trente-neuvième
\end{listverse}

\ConcordanceEntry{Asaël}
\vspace{-2mm}
\begin{listverse}
\item[\vref{2 S 2:18}] Joab, Abischaï et A. étaient là. Asaël
\item[\vref{2 S 2:23}] Mais A. refusa de se détourner. Abner le
\item[\vref{2 S 3:27}] cause du sang d'A., frère de Joab.
\item[\vref{2 S 23:24}] A., frère de Joab, était des trente.
\end{listverse}

\ConcordanceEntry{Asaph}
\vspace{-2mm}
\begin{listverse}
\item[\vref{2 Ch 5:12}] qui étaient chantres, A., Héman, Jeduthun, leurs
\item[\vref{Ps 50:1}] Psaume d'A.. Le Dieu puissant, Dieu, Yahweh a
\item[\vref{Ps 73:1}] Psaume d'A.. Quoi qu'il en soit, Dieu est
\item[\vref{Ps 83:1}] Cantique et psaume d'A.
\end{listverse}

\ConcordanceEntry{Asdod}
\vspace{-2mm}
\begin{listverse}
\item[\vref{Jos 13:3}] de Gaza, celui d'A., celui d'Askalon, celui
\item[\vref{Jos 15:46}] les villes près d'A., et leurs villages.
\item[\vref{1 S 5:1}] de Dieu, et l'emmenèrent d'Eben-Ezer à A.
\item[\vref{1 S 5:6}] frappa d'hémorroïdes à A. et ds tt
\item[\vref{Jé 25:20}] Gaza, à Ekron, et au reste d'A. ;
\item[\vref{So 2:4}] chassera les habitants d'A. en plein midi,
\end{listverse}

\ConcordanceEntry{Aser}
\vspace{-2mm}
\begin{listverse}
\item[\vref{Ge 30:13}] c'est pourquoi elle l'appela du nom d'A.
\item[\vref{Ge 49:20}] pain excellent viendra d'A., et il fournira
\item[\vref{No 1:40}] Des fils d'A., selon leurs générations,
\item[\vref{No 1:41}] de la tribu d'A., qui furent dénombrés,
\item[\vref{De 33:24}] Il dit aussi d'A. : Aser sera béni
\item[\vref{Jos 19:24}] tribu des fils d'A., selon leurs familles.
\item[\vref{Jos 19:31}] tribu des fils d'A., selon leurs familles ;
\item[\vref{Ap 7:6}] de la tribu d'A., douze mille marqués
\end{listverse}

\ConcordanceEntry{Asie}
\vspace{-2mm}
\begin{listverse}
\item[\vref{Ac 16:6}] lr. défendit d'annoncer la parole ds l'A.
\item[\vref{1 Co 16:19}] Les églises d'A. vs. saluent. Aquilas
\end{listverse}

\ConcordanceEntry{Asile}
\vspace{-2mm}
\begin{listverse}
\item[\vref{Ps 32:7}] Tu es mon a., tu me gardes
\item[\vref{Ps 59:17}] retraite, et mon a. au jour de
\item[\vref{Es 4:6}] de refuge et d'a. contre la tempête
\end{listverse}

\ConcordanceEntry{Askalon}
\vspace{-2mm}
\begin{listverse}
\item[\vref{Jos 13:3}] celui d'Asdod, celui d'A., celui de Gath,
\item[\vref{Jg 14:19}] il descendit à A.. Il tua trente
\item[\vref{Jé 25:20}] des Philistins, à A., à Gaza, à
\item[\vref{Am 1:8}] les habitants, et d'A. celui qui tient
\end{listverse}

\ConcordanceEntry{Aspect}
\vspace{-2mm}
\begin{listverse}
\item[\vref{Jg 13:6}] et il avait l'a. d'un ange de
\item[\vref{Es 3:9}] L'a. de lr. visage témoigne contre eux,
\item[\vref{Ez 1:13}] Et quant à l'a. des animaux, lr.
\item[\vref{Ez 10:22}] Quant à l'a. de leurs faces,
\item[\vref{Ez 40:3}] un hom., dont l'a. était com. de
\item[\vref{Mt 16:3}] savez bien discerner l'a. du ciel, et
\item[\vref{Lu 9:29}] com. il priait, l'a. de son visage
\end{listverse}

\ConcordanceEntry{Aspersion}
\vspace{-2mm}
\begin{listverse}
\item[\vref{Ex 29:21}] tu en feras l'a. sur Aaron, sur
\item[\vref{Lé 4:17}] et en fera a. dvt Yahweh en
\item[\vref{Lé 8:11}] il en fit l'a. sur l'autel par
\item[\vref{No 8:7}] purifier. Tu feras a. sur eux de
\item[\vref{No 19:4}] fera sept fois l'a. du sang vers
\item[\vref{Hé 9:19}] il en fit l'a. sur le livre
\item[\vref{Hé 11:28}] la Pâque et l'a. du sang, afin
\item[\vref{1 Pi 1:2}] qui participent à l'a. de son sang :
\end{listverse}

\ConcordanceEntry{Aspic}
\vspace{-2mm}
\begin{listverse}
\item[\vref{Job 20:14}] ds ses entrailles en un fiel d'a.
\item[\vref{Job 20:16}] sucera du venin d'a., la langue de
\item[\vref{Ps 91:13}] lion et sur l'a., tu piétineras le
\item[\vref{Es 11:8}] sur l'antre de l'a., et l'enfant sevré
\item[\vref{Ro 3:13}] a du venin d'a. sous leurs lèvres ;
\end{listverse}

\ConcordanceEntry{Assemblée}
\vspace{-2mm}
\begin{listverse}
\item[\vref{Ge 28:3}] tu deviennes une a. de peuples.
\item[\vref{Lé 23:36}] ce sera une a. solennelle. Vous ne
\item[\vref{No 15:26}] pardonné à tte l'a. des enfants d'Israël,
\item[\vref{No 16:3}] ts ceux de l'a. sont saints, et
\item[\vref{No 20:10}] et Aaron convoquèrent l'a. dvt le rocher.
\item[\vref{Jos 9:15}] les chefs de l'a. le lr. jurèrent.
\item[\vref{1 R 8:55}] et bénit tte l'a. d'Israël à haute
\item[\vref{2 Ch 1:5}] Et Salomon et l'a. y cherchèrent Yahweh.
\item[\vref{Né 8:2}] la loi dvt l'a., composée d'hommes et
\item[\vref{Ps 1:5}] les pécheurs ds l'a. des justes.
\item[\vref{Ps 22:23}] je te louerai au milieu de l'a.
\item[\vref{Ps 26:12}] droiture ; je bénirai Yahweh ds les a.
\item[\vref{Ps 35:18}] ds la grande a., je te louerai
\item[\vref{Ps 82:1}] se tient ds l'a. de Dieu, il
\item[\vref{Jé 15:17}] pas assis ds l'a. des moqueurs, et
\item[\vref{Ac 11:26}] se réunirent aux a. de l'Eglise, et
\item[\vref{1 Co 11:18}] vs. réunissez en a., j'apprends qu'il y
\item[\vref{1 Ti 4:14}] des mains de l'a. des anciens.
\item[\vref{Hé 2:12}] je te louerai au milieu de l'a.
\item[\vref{Hé 10:25}] N'abandonnons pas notre a., com. c'est la
\item[\vref{Hé 12:23}] et de l'a. et de l'Eglise des premiers-nés qui
\item[\vref{Ja 2:2}] entre ds votre a. un hom. qui
\item[\vref{Ap 19:6}] voix d'une grande a., et com. le
\end{listverse}

\ConcordanceEntry{Assembler}
\vspace{-2mm}
\begin{listverse}
\item[\vref{Ge 49:1}] et lr. dit : A.-vs., et je
\item[\vref{Ec 2:26}] recueillir et à a., afin que cela
\item[\vref{Mt 12:30}] et celui qui n'a. pas avec moi
\item[\vref{Mt 18:20}] ou trois sont a. en mon Nom,
\item[\vref{Mt 24:28}] le cadavre, là s'a. les vautours.
\item[\vref{Mc 2:2}] et aussitôt il s'a. un si grand
\item[\vref{1 Co 5:4}] mon esprit étant a. au Nom de
\item[\vref{1 Co 14:26}] que vs. vs. a., selon que chacun
\item[\vref{1 Co 16:2}] ce qu'il pourra a., selon la prospérité
\item[\vref{Ap 16:14}] afin de les a. pour le combat
\end{listverse}

\ConcordanceEntry{Asseoir}
\vspace{-2mm}
\begin{listverse}
\item[\vref{1 S 2:8}] pour le faire a. avec les nobles.
\item[\vref{1 R 2:24}] qui m'a fait a. sur le trône
\item[\vref{Ps 1:1}] pécheurs, qui ne s'a. pas ds l'assemblée
\item[\vref{Ps 110:1}] à mon Seign. : A.-toi à ma
\item[\vref{Ps 139:2}] sais qnd je m'a. et qnd je
\item[\vref{Os 14:7}] Ils reviendront s'a. à son ombre,
\item[\vref{Mt 4:16}] ce peuple, a. ds les ténèbres,
\item[\vref{Mt 14:19}] la foule de s'a. sur l'herbe, il
\item[\vref{Mt 20:21}] sont ici, soient a. l'un à ta
\item[\vref{Mt 25:31}] anges, alors il s'a. sur le trône
\item[\vref{Ep 1:20}] qu'il l'a fait a. à sa droite
\item[\vref{Ep 2:6}] ns. a fait a. ensemble ds les
\item[\vref{Ap 3:21}] je le ferai a. avec moi sur
\end{listverse}

\ConcordanceEntry{Asservir}
\vspace{-2mm}
\begin{listverse}
\item[\vref{Ge 15:13}] et qu'ils seront a. aux habitants du
\item[\vref{Ge 25:23}] plus grand sera a. au plus petit.
\item[\vref{1 S 4:9}] vs. ont été a. ; agissez en hommes,
\item[\vref{Ps 18:44}] que je ne connais point m'est a.
\item[\vref{Jé 25:11}] ces nations seront a. au roi de
\item[\vref{Jé 27:7}] nations lui seront a., à lui, à
\item[\vref{Ro 6:18}] vs. avez été a. à la justice.
\item[\vref{Ro 6:22}] du péché et a. à Dieu, vs.
\item[\vref{2 Co 11:20}] si quelqu'un vs. a., si quelqu'un vs.
\item[\vref{Ga 4:9}] voulez encore vs. a. com. auparavant ?
\item[\vref{Tit 3:3}] insensés, désobéissants, égarés, a. à tte espèce
\end{listverse}

\ConcordanceEntry{Assistance}
\vspace{-2mm}
\begin{listverse}
\item[\vref{Ep 4:16}] jointures de son a., tire son accroissement
\end{listverse}

\ConcordanceEntry{Assister}
\vspace{-2mm}
\begin{listverse}
\item[\vref{Mt 15:5}] dont j'aurais pu t'a. est une offrande
\item[\vref{Mt 15:25}] elle vint et l'adora, disant : Seign., a.-moi !
\item[\vref{Lu 8:3}] plusieurs autres qui l'a. de leurs biens.
\item[\vref{Jn 3:29}] de l'Epoux qui a., et qui l'entend,
\item[\vref{Ro 16:2}] et que vs. l'a. ds tt ce
\item[\vref{1 Ti 5:16}] veuves, qu'ils les a., et que l'église
\item[\vref{2 Ti 4:16}] Personne ne m'a a. ds ma première
\item[\vref{2 Ti 4:17}] le Seign. m'a a. et fortifié, afin
\item[\vref{Hé 7:13}] laquelle nul n'a a. à l'autel ;
\end{listverse}

\ConcordanceEntry{Assuérus}
\vspace{-2mm}
\begin{listverse}
\item[\vref{Esd 4:6}] pendant le règne d'A., au commencement de
\item[\vref{Est 1:1}] arriva au temps d'A., de cet Assuérus
\end{listverse}

\ConcordanceEntry{Assujettir}
\vspace{-2mm}
\begin{listverse}
\item[\vref{Ge 1:28}] la terre, et a.-la ; et dominez
\item[\vref{2 S 22:48}] donne vengeance, qui m'a. les peuples,
\item[\vref{Ps 47:4}] Il ns. a. des peuples et des nations sous
\item[\vref{Da 7:27}] royaumes lui seront a. et lui obéiront.
\item[\vref{Ph 3:21}] il peut mm s'a. ttes choses.
\item[\vref{Hé 2:15}] la mort, étaient a. tte lr. vie
\item[\vref{1 Pi 3:22}] et auquel sont a. les anges, les
\end{listverse}

\ConcordanceEntry{Assurance}
\vspace{-2mm}
\begin{listverse}
\item[\vref{Pr 3:26}] Yahweh sera ton a., et il gardera
\item[\vref{Pr 14:26}] a une ferme a., et une retraite
\item[\vref{Es 14:30}] pauvres reposeront en a. ; mais je ferai
\item[\vref{Ac 13:46}] lr. dirent avec a. : C'est à vs.
\item[\vref{Ac 14:3}] Icone, parlant avec a. du Seign., qui
\item[\vref{Ro 8:38}] Car j'ai l'a. que ni la
\item[\vref{Ph 1:20}] mais qu'en tte a., Christ sera mntnt,
\item[\vref{1 Th 2:2}] avons pris de l'a. en notre Dieu,
\item[\vref{Hé 3:6}] jusqu'à la fin l'a., et l'espérance qui
\item[\vref{Hé 4:16}] Approchons dc avec a. du trône de
\item[\vref{Hé 13:6}] pouvons dire avec a. : Le Seign. est
\item[\vref{1 Jn 2:28}] ns. ayons de l'a., et que ns.
\item[\vref{1 Jn 3:21}] ns. avons de l'a. dvt Dieu.
\item[\vref{1 Jn 4:17}] ns. ayons de l'a. au jour du
\item[\vref{1 Jn 5:14}] Et c'est ici l'a. que ns. avons
\end{listverse}

\ConcordanceEntry{Assyrie, Assyriens}
\vspace{-2mm}
\begin{listverse}
\item[\vref{Ge 2:14}] qui coule vers l'A. ; et le quatrième
\item[\vref{2 R 16:7}] à Tiglath-Piléser, roi d'A., pour lui dire :
\item[\vref{2 R 18:11}] Le roi d'A. emmena Israël en Assyrie et il
\item[\vref{Ps 83:9}] l'A. aussi se joint à eux ; ils
\item[\vref{Es 10:5}] Malheur à l'A., verge de ma
\item[\vref{Es 19:23}] de l'Egypte en A. ; l'Assyrie viendra en
\item[\vref{Es 30:31}] Car l'A., qui frappait du bâton, sera effrayé
\item[\vref{Es 31:8}] Et l'A. tombera par l'épée qui n'est pas
\item[\vref{Os 5:13}] vers le roi d'A., et s'est adressé
\item[\vref{Os 12:2}] traite alliance avec l'A., et l'on porte
\item[\vref{Za 10:11}] desséchées ; l'orgueil de l'A. sera abattu, et
\end{listverse}

\ConcordanceEntry{Astarté, Asherah}
\vspace{-2mm}
\begin{listverse}
\item[\vref{Ex 34:13}] statues, et vs. couperez leurs emblèmes d'A.
\item[\vref{De 16:21}] planteras point d'arbre d'A., près de l'autel
\item[\vref{Jg 2:13}] Yahweh, et servirent Baal et les A.
\item[\vref{1 S 7:3}] étrangers, et les A., dirigez votre cœur
\item[\vref{1 R 11:5}] Salomon alla après A., la divinité des
\item[\vref{1 R 11:33}] sont prosternés dvt A., la déesse des
\item[\vref{1 R 15:13}] une idole pour A.. Asa mit en
\item[\vref{1 R 16:33}] fit une idole d'A. ; de sorte qu'Achab
\item[\vref{1 R 18:19}] quatre cents prophètes d'A. qui mangent à
\item[\vref{2 R 17:10}] statues et des A. sur ttes les
\item[\vref{2 R 18:4}] abattit les idoles d'A., et il brisa
\item[\vref{2 R 23:7}] les femmes tissaient des tentes pour A.
\item[\vref{2 R 23:13}] avait bâtis à A., l'abomination des Sidoniens,
\item[\vref{2 Ch 33:19}] dressa des idoles d'A. et des images
\item[\vref{Es 27:9}] lorsque les idoles d'A. et les statues
\item[\vref{Mi 5:13}] de toi les A., et détruirai tes
\end{listverse}

\ConcordanceEntry{Astre}
\vspace{-2mm}
\begin{listverse}
\item[\vref{Es 13:10}] cieux et leurs a. ne feront plus
\item[\vref{Es 14:12}] tombé du ciel, a. brillant, fils de
\end{listverse}

\ConcordanceEntry{Astrologue}
\vspace{-2mm}
\begin{listverse}
\item[\vref{Da 2:10}] à qq magicien, a. ou Chaldéen que
\item[\vref{Da 2:27}] les sages, les a., les magiciens et
\item[\vref{Da 5:7}] fasse venir les a., les Chaldéens et
\end{listverse}

\ConcordanceEntry{Athalie}
\vspace{-2mm}
\begin{listverse}
\item[\vref{2 R 8:26}] Sa mère s'appelait A., fille d'Omri, roi
\item[\vref{2 R 11:1}] A., mère d'Achazia, ayant vu que son
\item[\vref{2 R 11:3}] de Yahweh. Cependant A. régnait sur le
\item[\vref{2 R 11:20}] mis à mort A. par l'épée ds
\end{listverse}

\ConcordanceEntry{Athènes}
\vspace{-2mm}
\begin{listverse}
\item[\vref{Ac 17:15}] le conduisirent jusqu'à A.. Puis ils s'en
\item[\vref{Ac 17:21}] qui demeuraient à A., ne passaient lr.
\item[\vref{1 Th 3:1}] trouvé bon de demeurer seuls à A.
\end{listverse}

\ConcordanceEntry{Attacher}
\vspace{-2mm}
\begin{listverse}
\item[\vref{Ge 2:24}] sa mère et s'a. à sa fem.,
\item[\vref{De 4:4}] qui vs. êtes a. à Yahweh, votre
\item[\vref{De 10:15}] Et Yahweh s'est a. à tes pères,
\item[\vref{1 R 9:9}] qu'ils se sont a. à d'autres dieux,
\item[\vref{Ps 22:16}] et ma langue s'a. à mon palais ;
\item[\vref{Es 64:6}] se réveille pour s'a. fortement à toi ;
\item[\vref{Os 11:7}] mon peuple est a. à sa rébellion
\item[\vref{Mt 6:24}] l'autre ; ou il s'a. à l'un, et
\item[\vref{Lu 19:30}] trouverez un ânon a., sur lequel aucun
\item[\vref{Ro 7:21}] faire le bien, le mal est a. à moi.
\item[\vref{Ga 3:10}] ts ceux qui s'a. aux œuvres de
\item[\vref{Col 2:19}] sans s'a. au Chef, dont tt le corps
\item[\vref{1 Ti 4:1}] la foi pour s'a. à des esprits
\item[\vref{Tit 1:14}] et qu'ils ne s'a. pas aux fables
\end{listverse}

\ConcordanceEntry{Attaquer}
\vspace{-2mm}
\begin{listverse}
\item[\vref{Ge 49:19}] des troupes viendront l'a., mais il ravagera
\item[\vref{Est 8:11}] armes pour les a., ainsi que leurs
\item[\vref{Job 40:27}] tu ne te souviendras plus de l'a.
\item[\vref{Ps 39:11}] consumé par les a. de ta main.
\end{listverse}

\ConcordanceEntry{Atteindre}
\vspace{-2mm}
\begin{listverse}
\item[\vref{Ge 47:9}] et n'ont point a. les jours des
\item[\vref{Ex 15:9}] disait : Je poursuivrai, j'a., je partagerai le
\item[\vref{Lé 15:4}] celui qui est a. d'un flux sera
\item[\vref{De 28:15}] qui viendront sur toi, et qui t'a. :
\item[\vref{Job 3:25}] et ce que j'appréhende le plus m'a.
\item[\vref{Ps 32:6}] déluges de grandes eaux, elles ne l'a. point.
\item[\vref{Ps 36:6}] Yahweh ! ta bonté a. jusqu'aux cieux, ta
\item[\vref{Ps 91:7}] ta droite, tu ne seras pas a. ;
\item[\vref{Ph 3:12}] que j'aie déjà a. le but, ou
\end{listverse}

\ConcordanceEntry{Attendre}
\vspace{-2mm}
\begin{listverse}
\item[\vref{Job 14:14}] l'hom. meurt, revivra-t-il ? J'a. dc ts les
\item[\vref{Job 30:26}] Cependant lorsque j'a. le bien, le
\item[\vref{Ps 27:14}] A.-toi à Yahweh et demeure ferme,
\item[\vref{Ps 40:2}] J'ai a. patiemment Yahweh, et il s'est tourné
\item[\vref{La 3:26}] bon d'espérer et d'a. en silence la
\item[\vref{Da 12:12}] est celui qui a. et qui parviendra
\item[\vref{Mt 11:3}] ou devons-ns. en a. un autre ?
\item[\vref{Mc 15:43}] conseiller honorable, qui a. aussi le Royaume
\item[\vref{Lu 2:25}] et pieux, il a. la consolation d'Israël,
\item[\vref{Ac 1:4}] de Jérus., mais d'a. la promesse du
\item[\vref{Ro 8:25}] c'est que ns. l'a. avec patience.
\item[\vref{1 Co 11:33}] assemblez pour manger, a.-vs. les uns
\item[\vref{1 Th 1:10}] et pour a. des cieux son Fils Jésus, qu'il
\item[\vref{Tit 2:13}] en a. la bienheureuse espérance, et l'apparition de
\item[\vref{2 Pi 3:13}] Mais ns. a., selon sa promesse,
\end{listverse}

\ConcordanceEntry{Attentif}
\vspace{-2mm}
\begin{listverse}
\item[\vref{1 R 8:28}] mon Dieu, sois a. à la prière
\item[\vref{Ps 119:95}] mais je suis a. à tes préceptes.
\item[\vref{Pr 4:20}] Mon fils, sois a. à mes paroles,
\item[\vref{Pr 7:24}] écoutez-moi et soyez a. aux paroles de
\item[\vref{Es 34:1}] vs. peuples, soyez a. ! Que la terre
\item[\vref{Es 48:18}] Si tu étais a. à mes commandements,
\item[\vref{Jé 6:17}] qui disent : Soyez a. au son du
\item[\vref{Da 9:23}] un bien-aimé. Sois a. à la parole,
\item[\vref{Mi 1:2}] en elle, soyez a. ! Et que le
\item[\vref{Mal 3:16}] et Yahweh fut a., et il écouta ;
\item[\vref{2 Pi 1:19}] faites bien d'être a., com. à une
\end{listverse}

\ConcordanceEntry{Attention}
\vspace{-2mm}
\begin{listverse}
\item[\vref{Ex 2:25}] et il fit a. à lr. état.
\item[\vref{De 16:19}] ne prêteras point a. à l'apparence des
\item[\vref{Ru 2:10}] que tu prêtes a. à moi, moi
\item[\vref{1 S 16:7}] Ne prête pas a. à son apparence
\item[\vref{1 R 8:25}] Dieu d'Israël, prête a. à la promesse
\item[\vref{2 R 3:14}] ne ferais aucune a. à toi et
\item[\vref{2 Ch 30:22}] prêtaient une grande a. et de l'intelligence
\item[\vref{Pr 23:1}] gouverneur, considère avec a. celui qui est
\item[\vref{Mt 24:15}] qui lit ce prophète y fasse a. !
\item[\vref{Mc 13:33}] Faites a. à tt, veillez et priez ; car
\item[\vref{Lu 17:20}] vient pas de manière à attirer l'a.
\item[\vref{Ac 2:14}] ceci, et faites a. à mes paroles !
\end{listverse}

\ConcordanceEntry{Attester}
\vspace{-2mm}
\begin{listverse}
\item[\vref{Ac 10:42}] au peuple et d'a. que c'est lui
\item[\vref{1 Co 15:31}] la mort, je l'a., par la gloire
\item[\vref{1 Th 4:6}] com. ns. vs. l'avons dit et a.
\end{listverse}

\ConcordanceEntry{Attirer}
\vspace{-2mm}
\begin{listverse}
\item[\vref{Pr 1:10}] les pécheurs veulent t'a., ne t'y accorde
\item[\vref{Pr 18:13}] de folie et a. la confusion.
\item[\vref{Ez 20:7}] les abominations qui a. ses regards, et
\item[\vref{Os 2:16}] Néanmoins, voici, je l'a. après que je
\item[\vref{Lu 17:20}] de manière à a. l'attention.
\item[\vref{Jn 6:44}] m'a envoyé ne l'a. ; et je le
\item[\vref{Jn 12:32}] de la terre, j'a. ts les hommes
\item[\vref{Ro 12:16}] élevé, mais laissez-vs. a. par ce qui
\item[\vref{2 Pi 2:1}] les a rachetés, a. sur eux-mêmes une
\end{listverse}

\ConcordanceEntry{Attrister}
\vspace{-2mm}
\begin{listverse}
\item[\vref{No 11:10}] s'enflamma fortement et Moïse en fut a.
\item[\vref{Ps 78:40}] de fois l'ont-ils a. ds ce lieu
\item[\vref{Es 63:10}] et ils ont a. son Esprit saint,
\item[\vref{Mt 17:23}] Et les disciples en furent fort a.
\item[\vref{Mc 14:19}] Ils commencèrent à s'a., et ils lui
\item[\vref{Jn 21:17}] m'aimes-tu ? Pierre fut a. de ce qu'il
\item[\vref{2 Co 6:10}] com. a. et toutefois toujours joyeux ; com. pauvres
\item[\vref{2 Co 7:8}] je vs. aie a. par ma lettre,
\item[\vref{Ep 4:30}] Et n'a. pas le Saint-Esprit de Dieu, par
\item[\vref{1 Th 4:13}] ne soyez pas a. com. les autres
\end{listverse}

\ConcordanceEntry{Augmenter}
\vspace{-2mm}
\begin{listverse}
\item[\vref{Ge 3:16}] à la fem. : J'a. beaucoup la souffrance
\item[\vref{Esd 10:10}] que vs. avez a. le crime d'Israël.
\item[\vref{Job 36:9}] connaître que leurs péchés se sont a.
\item[\vref{Pr 13:11}] qui amasse peu à peu les a.
\item[\vref{Pr 16:21}] douceur des lèvres a. l'instruction.
\item[\vref{Pr 28:8}] Celui qui a. ses biens par
\item[\vref{Ec 1:18}] et celui qui a. sa connaissance, augmente
\item[\vref{Da 12:4}] ça et là, et la connaissance a.
\item[\vref{Lu 17:5}] dirent au Seign. : A.-ns. la foi.
\item[\vref{Ac 16:5}] la foi, et a. en nombre chaque
\item[\vref{2 Co 9:10}] votre semence, et a. les revenus de
\end{listverse}

\ConcordanceEntry{Aumône}
\vspace{-2mm}
\begin{listverse}
\item[\vref{Mt 6:2}] tu fais ton a., ne fais pas
\item[\vref{Lu 11:41}] Donnez plutôt en a. ce qui est
\item[\vref{Lu 12:33}] et donnez-le en a.. Faites-vs. des bourses
\item[\vref{Ac 3:2}] Belle, pour demander l'a. à ceux qui
\item[\vref{Ac 9:36}] faisait beaucoup de bonnes œuvres et d'a.
\item[\vref{Ac 10:2}] faisait aussi beaucoup d'a. au peuple, et
\item[\vref{Ac 24:17}] pour faire des a. et des offrandes
\end{listverse}

\ConcordanceEntry{Aurore}
\vspace{-2mm}
\begin{listverse}
\item[\vref{Ge 32:24}] lutta avec lui jusqu'au lever de l'a.
\item[\vref{Ps 110:3}] du sein de l'a., ta jeunesse vient
\item[\vref{Ps 119:147}] Je devance l'a. et je crie ;
\item[\vref{Ps 139:9}] les ailes de l'a., et que je
\item[\vref{Es 8:20}] aura certainement point d'a. pour le peuple.
\item[\vref{Es 14:12}] brillant, fils de l'a. ? Toi qui foulais
\item[\vref{Es 58:8}] lumière éclatera com. l'a., et ta guérison
\item[\vref{Da 6:19}] du jour, avec l'a., et il alla
\item[\vref{Os 6:3}] que celle de l'a.. Il viendra pour
\item[\vref{Joë 2:2}] il vient com. l'a. s'étend sur les
\end{listverse}

\ConcordanceEntry{Autel}
\vspace{-2mm}
\begin{listverse}
\item[\vref{Ge 8:20}] Noé bâtit un a. à Yahweh, il
\item[\vref{Ge 12:7}] bâtit là un a. à Yahweh qui
\item[\vref{Ge 22:9}] bâtit là un a., et rangea le
\item[\vref{Ge 26:25}] bâtit là un a., et invoqua le
\item[\vref{Ge 33:20}] il dressa un a. qu'il appela El-Elohé-Israël (
\item[\vref{Ge 35:3}] je dresserai un a. au Dieu qui
\item[\vref{Ex 27:1}] feras aussi un a. de bois d'acacia,
\item[\vref{Ex 34:13}] vs. démolirez leurs a., vs. briserez leurs
\item[\vref{Ex 38:30}] tente d'assignation, et l'a. d'airain avec sa
\item[\vref{Lé 16:19}] son doigt sur l'a., et le purifiera
\item[\vref{Jos 22:26}] à bâtir un a., non pour des
\item[\vref{Jg 6:25}] ans ; et démolis l'a. de Baal qui
\item[\vref{1 R 13:2}] il cria contre l'a. selon la parole
\item[\vref{1 R 18:30}] et il répara l'a. de Yahweh, qui
\item[\vref{2 R 11:18}] démolirent avec ses a. ; et ils brisèrent
\item[\vref{2 R 23:12}] roi démolit les a. qui étaient sur
\item[\vref{2 Ch 7:9}] la dédicace de l'a. pendant sept jours,
\item[\vref{2 Ch 14:2}] Il ôta les a. étrangers et les
\item[\vref{2 Ch 15:8}] et il rétablit l'a. de Yahweh, qui
\item[\vref{2 Ch 33:15}] et ts les a. qu'il avait bâtis
\item[\vref{Esd 3:3}] Et ils posèrent l'a. de Dieu sur
\item[\vref{Es 6:6}] avait pris sur l'a. avec des pincettes.
\item[\vref{Mal 1:7}] offrez sur mon a. du pain souillé,
\item[\vref{Mt 5:23}] ton offrande à l'a., et que là
\item[\vref{Mt 23:18}] encore, jure par l'a., ce n'est rien ;
\item[\vref{Ac 17:23}] mm trouvé un a. sur lequel était
\item[\vref{1 Co 9:13}] qui servent à l'a. ont part à
\item[\vref{1 Co 10:18}] sacrifices ne sont-ils pas participants de l'a. ?
\item[\vref{Hé 13:10}] Nous avons un a. dont ceux qui
\item[\vref{Ja 2:21}] il offrit son fils Isaac sur l'a. ?
\item[\vref{Ap 6:9}] je vis sous l'a. les âmes de
\item[\vref{Ap 8:3}] se tint dvt l'a., ayant un encensoir
\end{listverse}

\ConcordanceEntry{Autorité}
\vspace{-2mm}
\begin{listverse}
\item[\vref{Mt 7:29}] com. ayant de l'a., et non com.
\item[\vref{Mt 21:23}] dirent : Par quelle a. fais-tu ces choses ;
\item[\vref{Mc 13:34}] sa maison, remet l'a. à ses serviteurs,
\item[\vref{Ac 1:7}] Père a fixés de sa propre a.
\item[\vref{Ro 13:2}] qui résiste à l'a. résiste à l'ordre
\item[\vref{1 Co 11:10}] une marque de l'a. de son mari
\item[\vref{1 Ti 2:12}] d'enseigner ni d'user d'a. sur le mari ;
\item[\vref{Tit 2:15}] avec une pleine a.. Et que personne
\item[\vref{Jud 1:8}] lr. chair, méprisent l'a. et blasphèment contre
\item[\vref{Ap 2:26}] je lui donnerai a. sur les nations.
\item[\vref{Ap 13:4}] qu'il avait donné l'a. à la bête,
\item[\vref{Ap 17:12}] mais ils recevront a. com. rois en
\item[\vref{Ap 20:4}] s'assirent, à qui l'a. de juger fut
\end{listverse}

\ConcordanceEntry{Autorités}
\vspace{-2mm}
\begin{listverse}
\item[\vref{Lu 12:11}] magistrats et les a., ne vs. inquiétez
\item[\vref{Ro 13:1}] soit soumise aux a. supérieures ; car il
\item[\vref{Tit 3:1}] magistrats et aux a., d'obéir aux gouverneurs,
\end{listverse}

\ConcordanceEntry{Avantage}
\vspace{-2mm}
\begin{listverse}
\item[\vref{1 S 17:9}] mais si j'ai l'a. sur lui, et
\item[\vref{Ec 1:3}] Quel a. a l'hom. de tt son travail
\item[\vref{Ec 2:11}] l'hom. n'a aucun a. de ce qui
\item[\vref{Ec 3:9}] Quel a. celui qui travaille a-t-il de sa
\item[\vref{Ec 6:11}] de vanités. Quel a. en a l'hom. ?
\item[\vref{Ec 7:12}] science a cet a., que la sagesse
\item[\vref{Ec 10:10}] la sagesse a l'a. de donner de
\item[\vref{Ro 3:1}] Quel est dc l'a. du Juif, ou
\item[\vref{Ro 14:16}] Que l'a. dont vs. jouissez ne soit dc
\end{listverse}

\ConcordanceEntry{Avare}
\vspace{-2mm}
\begin{listverse}
\item[\vref{Ps 10:3}] il estime heureux l'a. et il méprise
\item[\vref{Es 32:5}] appelé libéral, et l'a. trompeur ne sera
\item[\vref{Es 32:7}] Les instruments de l'a. sont pernicieux ; il
\item[\vref{Lu 16:14}] aussi, qui étaient a., entendaient ttes ces
\item[\vref{1 Co 6:10}] voleurs, ni les a., ni les ivrognes,
\end{listverse}

\ConcordanceEntry{Avarice}
\vspace{-2mm}
\begin{listverse}
\item[\vref{Lu 12:15}] soin de tte a. ; car quoique les
\item[\vref{Ro 1:29}] d'impureté, de méchanceté, d'a., de malignité, pleins
\item[\vref{2 Co 9:5}] non pas com. un fruit de l'a.
\item[\vref{1 Ti 3:3}] mais modéré, éloigné des querelles, exempt d'a.
\item[\vref{Hé 13:5}] conduite soit sans a., étant contents de
\end{listverse}

\ConcordanceEntry{Avènement}
\vspace{-2mm}
\begin{listverse}
\item[\vref{Mt 24:3}] signe de ton a., et de la
\item[\vref{Ac 7:52}] qui annonçaient d'avance l'a. du Juste, dont
\item[\vref{1 Co 15:23}] Christ seront vivifiés lors de son a.
\item[\vref{1 Th 2:19}] notre Seign. Jésus-Christ lors de son a. ?
\item[\vref{1 Th 3:13}] Père, lors de l'a. de notre Seign.
\item[\vref{1 Th 4:15}] et resterons pour l'a. du Seign., ne
\item[\vref{2 Th 2:1}] ce qui concerne l'a. de notre Seign.
\item[\vref{Ja 5:8}] vos cœurs, car l'a. du Seign. est
\item[\vref{2 Pi 1:16}] la puissance et l'a. de notre Seign.
\item[\vref{2 Pi 3:4}] promesse de son a. ? Car depuis que
\item[\vref{1 Jn 2:28}] confus dvt lui lors de son a.
\end{listverse}

\ConcordanceEntry{Avenir}
\vspace{-2mm}
\begin{listverse}
\item[\vref{Ex 13:14}] fils t'interrogera à l'a., en disant : Que
\item[\vref{Jos 4:6}] fils interrogeront à l'a. leurs pères, en
\item[\vref{2 R 21:6}] et qui prédisaient l'a.. Il fit de
\item[\vref{Es 19:3}] les morts et ceux qui prédisent l'a.
\item[\vref{Es 42:23}] s'y rendra attentif et l'écoutera à l'a. ?
\item[\vref{Es 43:12}] ai fait entendre l'a., qnd il n'y
\item[\vref{Na 2:1}] vœux ; car à l'a. les hommes violents
\item[\vref{2 Co 1:10}] espérons qu'il ns. délivrera aussi à l'a.
\item[\vref{1 Ti 6:19}] s'amassant ainsi pour l'a. un trésor placé
\item[\vref{Ap 4:1}] les choses qui doivent arriver à l'a.
\end{listverse}

\ConcordanceEntry{Avertir}
\vspace{-2mm}
\begin{listverse}
\item[\vref{Ge 14:13}] Un fuyard vint a. Abram, l'Hébreu, qui
\item[\vref{Ge 48:2}] On a. Jacob et on lui dit : Voici
\item[\vref{1 S 3:13}] Car je l'ai a. que je vais
\item[\vref{1 S 8:9}] voix ; mais les a.-les, avertis-les en
\item[\vref{2 R 17:13}] Yahweh fit a. Israël et Juda
\item[\vref{Né 4:12}] dix fois ns. a., de ts les
\item[\vref{Né 9:30}] et tu les a. par ton Esprit,
\item[\vref{Ps 50:7}] Israël ! Et je t'a.. JE SUIS Dieu,
\item[\vref{Ez 3:18}] Si tu ne l'a. pas, et si
\item[\vref{Ez 33:8}] au méchant pour l'a. de se détourner
\item[\vref{Ac 10:22}] témoignage, a été a. de Dieu par
\item[\vref{Ac 20:23}] ville, le Saint-Esprit m'a. que des liens
\item[\vref{Ac 20:31}] nuit et jour d'a. chacun de vs.
\item[\vref{Ac 22:26}] le tribun pour l'a., disant : Prends garde
\item[\vref{1 Co 4:14}] mais je vs. a. com. mes chers
\item[\vref{1 Co 10:28}] qui vs. a a., et à cause
\item[\vref{Hé 11:7}] ayant été divinement a. des choses qui
\item[\vref{2 Pi 3:17}] vs. êtes déjà a., prenez garde qu'étant
\end{listverse}

\ConcordanceEntry{Avertissement}
\vspace{-2mm}
\begin{listverse}
\item[\vref{No 26:10}] deux cent cinquante hommes qui servirent d'a.
\item[\vref{2 R 17:15}] pères, et ses a., qu'il lr. avait
\item[\vref{2 Ch 24:19}] lui par leurs a. ; mais ils ne
\item[\vref{Tit 3:10}] après le premier et le second a.,
\item[\vref{2 Pi 1:13}] réveille par des a., pendant que je
\item[\vref{2 Pi 3:1}] l'autre, par mes a., les sentiments purs
\end{listverse}

\ConcordanceEntry{Aveugle}
\vspace{-2mm}
\begin{listverse}
\item[\vref{Lé 19:14}] point d'achoppement dvt l'a., mais tu craindras
\item[\vref{De 27:18}] qui égare un a. ds le chemin !
\item[\vref{2 S 5:8}] le canal, ces a. et ces boiteux,
\item[\vref{Ps 146:8}] les yeux des a. ; Yahweh redresse ceux
\item[\vref{Es 35:5}] les yeux des a. seront ouverts, et
\item[\vref{Es 42:16}] Je conduirai les a. sur un chemin
\item[\vref{Es 42:19}] Qui, dis-je, est a., sinon mon serviteur ?
\item[\vref{Es 59:10}] tâtonnons com. des a. le long du
\item[\vref{Mt 15:14}] ce sont des a., conducteurs d'aveugles ; si
\item[\vref{Mt 15:31}] marchaient, que les a. voyaient ; et elle
\item[\vref{Mt 20:30}] Et voici, deux a. qui étaient assis
\item[\vref{Mc 10:46}] grande foule, un a., appelé Bartimée, c'est-à-dire
\item[\vref{Lu 6:39}] cette parabole : Un a. peut-il conduire un
\item[\vref{Lu 7:21}] il rendit la vue à plusieurs a.
\item[\vref{Jn 9:1}] vit un hom. a. de naissance.
\item[\vref{Jn 9:25}] c'est que j'étais a. et que mntnt
\item[\vref{Ac 13:11}] toi, tu seras a., et pour un
\item[\vref{Ro 2:19}] le conducteur des a., la lumière de
\item[\vref{Ap 3:17}] que tu es malheureux, misérable, pauvre, a. et nu.
\end{listverse}

\ConcordanceEntry{Aveuglement}
\vspace{-2mm}
\begin{listverse}
\item[\vref{Ge 19:11}] Et ils frappèrent d'a. les hommes qui
\item[\vref{De 28:28}] frappera de folie, d'a., et d'égarement d'esprit ;
\item[\vref{2 R 6:18}] frappe ces gens d'a. ! Et Dieu les
\item[\vref{Job 36:12}] par l'épée, ils expirent ds lr. a.
\item[\vref{Za 12:4}] et je frapperai d'a. ts les chevaux
\end{listverse}

\ConcordanceEntry{Aveugler}
\vspace{-2mm}
\begin{listverse}
\item[\vref{Ex 23:8}] car le présent a. les plus éclairés,
\item[\vref{Jn 12:40}] Il a a. leurs yeux ; et il a endurci
\item[\vref{2 Co 4:4}] ce siècle a a. l'entendement, afin qu'ils
\end{listverse}

\ConcordanceEntry{Avidité}
\vspace{-2mm}
\begin{listverse}
\item[\vref{Pr 10:3}] repousse au loin l'a. des méchants.
\end{listverse}

\ConcordanceEntry{Avilir}
\vspace{-2mm}
\begin{listverse}
\item[\vref{Es 16:14}] de Moab sera a., avec tte cette
\item[\vref{Es 23:9}] noblesse, et pour a. ts les honorables
\end{listverse}

\ConcordanceEntry{Avocat}
\vspace{-2mm}
\begin{listverse}
\item[\vref{1 Jn 2:1}] ns. avons un a. auprès du Père,
\end{listverse}

\ConcordanceEntry{Avorter}
\vspace{-2mm}
\begin{listverse}
\item[\vref{Ex 23:26}] de fem. qui a., ou qui soit
\item[\vref{Job 21:10}] se décharge de son veau et n'a. pas.
\item[\vref{Os 9:14}] un sein qui a. et des mamelles
\end{listverse}

\ConcordanceEntry{Avorton}
\vspace{-2mm}
\begin{listverse}
\item[\vref{Job 3:16}] été com. un a. caché, com. les
\item[\vref{Ps 58:9}] le soleil com. l'a. d'une fem. !
\item[\vref{Ec 6:3}] je dis qu'un a. vaut mieux que
\item[\vref{1 Co 15:8}] à moi aussi com. à un a. ;
\end{listverse}

\ConcordanceEntry{Avouer}
\vspace{-2mm}
\begin{listverse}
\item[\vref{Ps 32:5}] iniquité ; j'ai dit : J'a. mes transgressions à
\end{listverse}

\ConcordanceEntry{Azaria}
\vspace{-2mm}
\begin{listverse}
\item[\vref{2 R 15:1}] Jéroboam, roi d'Israël, A., fils d'Amatsia, roi
\item[\vref{2 R 15:7}] A. se coucha avec ses pères, et
\item[\vref{1 Ch 2:38}] Obed engendra Jéhu ; Jéhu engendra A. ;
\item[\vref{2 Ch 15:1}] Dieu fut sur A., fils d'Oded.
\item[\vref{2 Ch 26:17}] Mais A. le prêtre, entra après lui, et
\item[\vref{2 Ch 26:20}] A., le principal prêtre, le regarda ainsi
\item[\vref{Da 1:6}] de Juda, Daniel, Hanania, Mischaël et A.
\item[\vref{Da 1:7}] Méschac et à A. celui d'Abed-Négo.
\item[\vref{Da 1:19}] Hanania, Mischaël et A. ; et ils entrèrent
\item[\vref{Da 2:17}] Hanania, Mischaël et A., ses compagnons,
\end{listverse}

\ConcordanceEntry{Azazel}
\vspace{-2mm}
\begin{listverse}
\item[\vref{Lé 16:8}] pour le bouc qui doit être A.
\item[\vref{Lé 16:10}] tombé pour être A., sera présenté vivant
\item[\vref{Lé 16:26}] bouc pour être A. lavera ses vêtements
\end{listverse}

\ConcordanceEntry{Baal}
\vspace{-2mm}
\begin{listverse}
\item[\vref{Jg 2:13}] Yahweh, et servirent B. et les Astartés.
\item[\vref{Jg 6:25}] démolis l'autel de B. qui est à
\item[\vref{Jg 8:33}] et ils établirent B.-Berith pour lr.
\item[\vref{2 S 5:20}] David vint à B.-Peratsim, où il
\item[\vref{1 R 18:19}] cinquante prophètes de B. et les quatre
\item[\vref{1 R 18:40}] les prophètes de B. et qu'il n'en
\item[\vref{1 R 19:18}] les genoux dvt B., et dont la
\item[\vref{2 R 1:2}] dit : Allez, consultez B.-Zebub, dieu d'Ekron,
\item[\vref{2 Ch 26:7}] qui habitaient à Gur-B., et contre les
\item[\vref{Jé 2:8}] ont prophétisé par B., et sont allés
\item[\vref{Jé 19:5}] hauts lieux à B., afin de brûler
\item[\vref{Os 2:10}] avec lesquels ils ont façonné un B.
\end{listverse}

\ConcordanceEntry{Baal-Hermon}
\vspace{-2mm}
\begin{listverse}
\item[\vref{Jg 3:3}] la montagne de B., jusqu'à l'entrée de
\end{listverse}

\ConcordanceEntry{Baal-Peor, Peor}
\vspace{-2mm}
\begin{listverse}
\item[\vref{No 25:3}] Israël s'accoupla à B., c'est pourquoi la
\item[\vref{Jos 22:17}] chose l'iniquité de P., dont ns. ne
\item[\vref{Ps 106:28}] aux adorateurs de B., et mangèrent les
\end{listverse}

\ConcordanceEntry{Baal-Peratsim}
\vspace{-2mm}
\begin{listverse}
\item[\vref{2 S 5:20}] David vint à B., où il les
\end{listverse}

\ConcordanceEntry{Baal-Zebub}
\vspace{-2mm}
\begin{listverse}
\item[\vref{2 R 1:2}] dit : Allez, consultez B., dieu d'Ekron, pour
\item[\vref{2 R 1:16}] messagers pour consulter B., dieu d'Ekron, com.
\end{listverse}

\ConcordanceEntry{Babel}
\vspace{-2mm}
\begin{listverse}
\item[\vref{Ge 10:10}] son règne fut B., Erec, Accad, et
\item[\vref{Ge 11:9}] du nom de B., car c'est là
\end{listverse}

\ConcordanceEntry{Babylone}
\vspace{-2mm}
\begin{listverse}
\item[\vref{2 R 24:10}] Nebucadnetsar, roi de B., montèrent contre Jérus.,
\end{listverse}
\begin{legend}
\NoAutoSpaceBeforeFDP{
\item Ville de : 2 R 20:12; 24:10
\item Destruction de Jérusalem et captivité du peuple : 2 R 25:11; Jé 39:9
\item Les nations asservies par Nebucadnetsar : Jé 27:2-6; Da 4:30
\item Division de l'empire babylonien : Da 5:28-30
\item Prophéties et jugements sur B : Es 13:1; 14:4; Jé 50:1; 51:1
\item Chute de Babylone la grande : Ap 14:8; 17:5; 18:21
}
\end{legend}

\ConcordanceEntry{Baca}
\vspace{-2mm}
\begin{listverse}
\item[\vref{Ps 84:7}] la vallée de B., ils la réduisent
\end{listverse}

\ConcordanceEntry{Baescha}
\vspace{-2mm}
\begin{listverse}
\item[\vref{1 R 15:16}] entre Asa et B., roi d'Israël, pendant
\item[\vref{1 R 15:27}] Et B., fils d'Achija, de la maison d'Issacar,
\item[\vref{1 R 15:33}] roi de Juda, B., fils d'Achija, commença
\item[\vref{1 R 16:1}] de Hanani, contre B., en disant :
\item[\vref{1 R 16:3}] vais entièrement consumer B. et sa maison,
\end{listverse}

\ConcordanceEntry{Baiser (un)}
\vspace{-2mm}
\begin{listverse}
\item[\vref{Pr 27:6}] fidèles, mais les b. d'un ennemi sont
\item[\vref{Mt 26:48}] je donnerai un b., c'est lui, saisissez-le.
\item[\vref{Lu 7:45}] pas donné un b., mais elle, depuis
\item[\vref{Lu 22:48}] c'est par un b. que tu trahis
\item[\vref{Ro 16:16}] par un saint b.. Les églises de
\item[\vref{1 Co 16:20}] uns les autres par un saint b.
\item[\vref{2 Co 13:12}] par un saint b.. Tous les saints
\item[\vref{1 Th 5:26}] ts les frères par un saint b.
\item[\vref{1 Pi 5:14}] autres par un b. de charité. Que
\end{listverse}

\ConcordanceEntry{Baiser}
\vspace{-2mm}
\begin{listverse}
\item[\vref{1 R 19:18}] dont la bouche ne l'a point b.
\item[\vref{Ca 1:2}] Sulamithe :] Qu'il me b. des baisers de
\item[\vref{Lu 7:45}] pas donné un b., mais elle, depuis
\item[\vref{Lu 22:47}] Il s'approcha de Jésus pour le b.
\end{listverse}

\ConcordanceEntry{Balaam}
\vspace{-2mm}
\begin{listverse}
\item[\vref{No 22:5}] messagers auprès de B., fils de Beor,
\end{listverse}
\begin{legend}
\NoAutoSpaceBeforeFDP{
\item Balak cherche à maudire Israël : No 22:5-7
\item Balaam et l'ânesse : No 22:28; 2 Pi 2:16
\item B. bénit Israël des hauts lieux de Baal : No 23:19-24; 24:15-24
\item La doctrine de B : No 25:1-3; Jud 1:11; Ap 2:14
\item Son décès : No 31:8
}
\end{legend}

\ConcordanceEntry{Balak}
\vspace{-2mm}
\begin{listverse}
\item[\vref{No 22:2}] B., fils de Tsippor, vit tt ce
\item[\vref{Jos 24:9}] B. aussi, fils de Tsippor, roi de
\end{listverse}

\ConcordanceEntry{Balance}
\vspace{-2mm}
\begin{listverse}
\item[\vref{Lé 19:36}] Vous aurez des b. justes, des poids
\item[\vref{Ps 62:10}] l'hom. ! Dans une b., ils monteraient ts
\item[\vref{Pr 11:1}] La fausse b. est une abomination
\item[\vref{Pr 20:23}] Yahweh, et la b. fausse n'est pas
\item[\vref{Ez 45:10}] Ayez la b. juste, l'épha juste, et le bath
\item[\vref{Da 5:27}] pesé ds la b., et tu as
\item[\vref{Am 8:5}] et falsifiant la b. pour tromper ?
\item[\vref{Mi 6:11}] a de fausses b. et de faux
\item[\vref{Ap 6:5}] dessus avait une b. ds sa main.
\end{listverse}

\ConcordanceEntry{Balayures}
\vspace{-2mm}
\begin{listverse}
\item[\vref{1 Co 4:13}] devenus com. les b. du monde, com.
\end{listverse}

\ConcordanceEntry{Bannière}
\vspace{-2mm}
\begin{listverse}
\item[\vref{Ex 17:15}] autel et le nomma Yahweh, ma b.
\item[\vref{No 1:52}] chacun sous sa b., selon leurs armées.
\item[\vref{Ps 20:6}] ns. lèverons la b. au Nom de
\item[\vref{Ps 60:6}] as donné une b. à ceux qui
\item[\vref{Ca 2:4}] festin ; et sa b. que je porte
\item[\vref{Es 5:26}] Il élève une b. pour les nations
\item[\vref{Es 49:22}] je dresserai ma b. vers les peuples ;
\item[\vref{Es 62:10}] pierres ! Elevez une b. vers les peuples.
\item[\vref{Jé 50:2}] entendez-le, levez une b. ! Entendez-le, ne le
\end{listverse}

\ConcordanceEntry{Baptême}
\vspace{-2mm}
\begin{listverse}
\item[\vref{Mt 3:7}] venir à son b. beaucoup de pharisiens
\item[\vref{Mt 21:25}] Le b. de Jean, d'où venait-il ? Du ciel
\item[\vref{Mc 10:38}] être baptisés du b. dont je dois
\item[\vref{Lu 12:50}] Il est un b. dont je dois
\item[\vref{Lu 20:4}] le b. de Jean était-il du ciel ou
\item[\vref{Ac 13:24}] avait prêché le b. de repentance à
\item[\vref{Ac 18:25}] bien qu'il ne connaisse que le b. de Jean.
\item[\vref{Ac 19:3}] dit : De quel b. dc avez-vs. été
\item[\vref{Ro 6:4}] lui par le b. en sa mort,
\item[\vref{Ep 4:5}] Seign., une seule foi, un seul b.,
\item[\vref{Col 2:12}] lui par le b., en lui aussi
\item[\vref{Hé 6:2}] la doctrine des b., et de l'imposition
\item[\vref{1 Pi 3:21}] laquelle répond le b., qui n'est pas
\end{listverse}

\ConcordanceEntry{Baptiser}
\vspace{-2mm}
\begin{listverse}
\item[\vref{Mt 3:6}] ils se faisaient b. par lui ds
\item[\vref{Mt 3:11}] moi, je vs. b. d'eau en signe
\item[\vref{Mt 3:14}] J'ai besoin d'être b. par toi, et
\item[\vref{Mt 3:16}] Jésus eut été b., il sortit aussitôt
\item[\vref{Mc 16:16}] et qui sera b., sera sauvé ; mais
\item[\vref{Lu 3:7}] foule pour être b. par lui : Races
\item[\vref{Jn 1:33}] qui m'a envoyé b. d'eau m'avait dit :
\item[\vref{Jn 3:23}] Or Jean b. aussi à Enon, près de Salim,
\item[\vref{Jn 4:2}] Toutefois Jésus ne b. pas lui-mm, mais
\item[\vref{Ac 2:38}] de vs. soit b. au Nom de
\item[\vref{Ac 8:12}] les hommes que les femmes furent b.
\item[\vref{Ac 8:16}] seulement ils étaient b. au Nom du
\item[\vref{Ac 10:48}] ordonna qu'ils soient b. au Nom du
\item[\vref{Ac 19:4}] que Jean a b. du baptême de
\item[\vref{Ro 6:3}] qui avons été b. en Jésus-Christ, avons
\item[\vref{1 Co 1:17}] pas envoyé pour b., mais pour évangéliser,
\item[\vref{1 Co 10:2}] ont ts été b. en Moïse ds
\item[\vref{1 Co 15:29}] qui se font b. pour les morts ?
\item[\vref{Ga 3:27}] qui avez été b. en Christ, vs.
\end{listverse}

\ConcordanceEntry{Barabbas}
\vspace{-2mm}
\begin{listverse}
\item[\vref{Mt 27:16}] avait alors un prisonnier fameux, nommé B.
\item[\vref{Lu 23:18}] ensemble, disant : Ôte celui-ci, et relâche-ns. B.
\item[\vref{Jn 18:40}] pas celui-ci, mais B.. Or Barabbas était
\end{listverse}

\ConcordanceEntry{Barak}
\vspace{-2mm}
\begin{listverse}
\item[\vref{Jg 4:6}] Elle envoya appeler B., fils d'Abinoam, de
\item[\vref{Jg 4:15}] en déroute dvt B., Sisera, ts ses
\item[\vref{Jg 5:1}] ce cantique avec B., fils d'Abinoam, en
\item[\vref{Hé 11:32}] Gédéon, et de B., et de Samson,
\end{listverse}

\ConcordanceEntry{Barbare}
\vspace{-2mm}
\begin{listverse}
\item[\vref{Ps 114:1}] maison de Jacob s'éloigna d'un peuple b.,
\item[\vref{Ez 3:5}] à la langue b. ; c'est vers la
\item[\vref{Ez 3:6}] ou une langue b., dont tu ne
\item[\vref{Ac 28:2}] Les b. ns. traitèrent avec beaucoup d'humanité ; ils
\item[\vref{Ro 1:14}] aux Grecs qu'aux B., tant aux sages
\item[\vref{1 Co 14:11}] je serai un b. pour celui qui
\item[\vref{Col 3:11}] ni incirconcis, ni b. ni Scythe, ni
\end{listverse}

\ConcordanceEntry{Barbe}
\vspace{-2mm}
\begin{listverse}
\item[\vref{Lé 14:9}] sa tête, sa b., les sourcils de
\item[\vref{Lé 19:27}] raserez point les coins de votre b.
\item[\vref{2 S 10:4}] moitié de lr. b., et couper la
\item[\vref{Ps 133:2}] coule sur la b. d'Aaron, sur le
\item[\vref{Es 7:20}] pieds, et il enlèvera aussi la b.
\item[\vref{Jé 41:5}] hommes, ayant la b. rasée et les
\item[\vref{Jé 49:32}] coin de la b., et je ferai
\item[\vref{Ez 5:1}] et sur ta b.. Puis, tu prendras
\item[\vref{Ez 24:17}] couvre pas la b., et ne mange
\item[\vref{Mi 3:7}] se couvriront la b., parce qu'il n'y
\end{listverse}

\ConcordanceEntry{Bar-Jésus}
\vspace{-2mm}
\begin{listverse}
\item[\vref{Ac 13:6}] certain magicien, faux prophète Juif, nommé B.,
\end{listverse}

\ConcordanceEntry{Barnabas}
\vspace{-2mm}
\begin{listverse}
\item[\vref{Ac 4:36}] par les apôtres B., c'est-à-dire, fils de
\item[\vref{Ac 9:27}] Alors B., l'ayant pris avec lui, le conduisit
\item[\vref{Ac 11:25}] B. s'en alla à Tarse pour chercher
\item[\vref{Ac 11:30}] les mains de B. et de Saul.
\item[\vref{Ac 13:1}] et des docteurs : B., Siméon, appelé Niger,
\item[\vref{Ac 13:2}] dit : Séparez-moi mntnt B. et Saul pour
\item[\vref{Ac 13:4}] B. et Saul, envoyés par le Saint-Esprit,
\item[\vref{Ac 15:2}] Paul et B. eurent avec eux un débat et
\item[\vref{Ac 15:39}] l'un de l'autre. B., prenant Marc avec
\item[\vref{Ga 2:9}] moi et à B., la main d'association,
\item[\vref{Col 4:10}] le cousin de B., au sujet duquel
\end{listverse}

\ConcordanceEntry{Barque}
\vspace{-2mm}
\begin{listverse}
\item[\vref{Mt 4:22}] aussitôt quitté lr. b. et lr. père,
\item[\vref{Mt 8:24}] tempête que la b. était couverte de
\item[\vref{Mt 14:13}] là ds une b., pour se retirer
\item[\vref{Mt 14:32}] montés ds la b., le vent s'apaisa.
\item[\vref{Lu 5:3}] et de la b. il enseignait la
\item[\vref{Lu 8:23}] le lac, la b. se remplissait d'eau,
\item[\vref{Jn 6:21}] plaisir ds la b., et aussitôt la
\item[\vref{Jn 21:3}] aussitôt ds une b., mais ils ne
\end{listverse}

\ConcordanceEntry{Barthélemy}
\vspace{-2mm}
\begin{listverse}
\item[\vref{Mt 10:3}] Philippe, et B. ; Thomas, et Matthieu,
\item[\vref{Mc 3:18}] André, Philippe, B., Matthieu, Thomas, Jacques,
\item[\vref{Ac 1:13}] Philippe et Thomas, B. et Matthieu, Jacques,
\end{listverse}

\ConcordanceEntry{Bartimée}
\vspace{-2mm}
\begin{listverse}
\item[\vref{Mc 10:46}] un aveugle, appelé B., c'est-à-dire le fils
\end{listverse}

\ConcordanceEntry{Baruc}
\vspace{-2mm}
\begin{listverse}
\item[\vref{Jé 32:12}] contrat d'acquisition à B., fils de Nérija,
\item[\vref{Jé 36:4}] Jérémie dc appela B., fils de Nérija,
\item[\vref{Jé 43:3}] Mais B., fils de Nérija, t'incite contre ns.,
\item[\vref{Jé 43:6}] le prophète, et B., fils de Nérija.
\item[\vref{Jé 45:1}] prophète, adressa à B., fils de Nérija,
\end{listverse}

\ConcordanceEntry{Barzillaï}
\vspace{-2mm}
\begin{listverse}
\item[\vref{2 S 17:27}] de Lodebar, et B., le Galaadite de
\item[\vref{2 S 19:31}] B., le Galaadite, descendit de Roguelim, et
\end{listverse}

\ConcordanceEntry{Basan}
\vspace{-2mm}
\begin{listverse}
\item[\vref{No 21:33}] le chemin de B.. Og, roi de
\item[\vref{No 32:33}] Og, roi de B., le pays avec
\item[\vref{De 3:4}] contrée d'Argob, le royaume d'Og en B.
\item[\vref{Jos 13:30}] depuis Mahanaïm, tt B., et tt le
\end{listverse}

\ConcordanceEntry{Bâtard}
\vspace{-2mm}
\begin{listverse}
\item[\vref{De 23:2}] Le b. n'entrera point ds l'assemblée de Yahweh ;
\item[\vref{Za 9:6}] Et le b. habitera à Asdod ; et j'abattrai l'orgueil
\end{listverse}

\ConcordanceEntry{Bath}
\vspace{-2mm}
\begin{listverse}
\item[\vref{Es 5:10}] ne produiront qu'un b., et un omer
\item[\vref{Ez 45:10}] balance juste, l'épha juste, et le b. juste.
\item[\vref{Ez 45:11}] L'épha et le b. seront de mm
\item[\vref{Ez 45:14}] Le b. est la mesure pour l'huile, l'offrande
\end{listverse}

\ConcordanceEntry{Bath-Schéba}
\vspace{-2mm}
\begin{listverse}
\item[\vref{2 S 11:3}] dit : N'est-ce pas B., fille d'Eliam, fem.
\item[\vref{2 S 12:24}] consola sa fem. B., et il alla
\item[\vref{1 R 1:11}] Nathan parla à B., mère de Salomon,
\item[\vref{1 R 1:15}] B. se rendit ds la chambre du
\end{listverse}

\ConcordanceEntry{Bâtir}
\vspace{-2mm}
\begin{listverse}
\item[\vref{Ge 11:4}] ils dirent : Allons ! B.-ns. une ville,
\item[\vref{2 S 7:13}] sera lui qui b. une maison à
\item[\vref{1 R 9:15}] Salomon leva pour b. la maison de
\item[\vref{2 Ch 14:6}] dit à Juda : B. ces villes, et
\item[\vref{2 Ch 36:23}] ordonné de lui b. une maison à
\item[\vref{Esd 6:14}] anciens des Juifs b. avec succès, selon
\item[\vref{Né 2:20}] lèverons et ns. b. ; mais vs., vs.
\item[\vref{Ps 122:3}] Jérus., qui est b. com. une ville
\item[\vref{Ps 127:1}] Si Yahweh ne b. la maison, ceux
\item[\vref{Ec 3:3}] pour démolir et un temps pour b. ;
\item[\vref{Jé 31:28}] sur eux pour b. et pour planter,
\item[\vref{Ha 2:12}] à celui qui b. des villes avec
\item[\vref{Ag 1:8}] du bois, et b. cette maison ; et
\item[\vref{Za 1:16}] maison y sera r., dit Yahweh des
\item[\vref{Mt 7:24}] prudent qui a b. sa maison sur
\item[\vref{Lu 11:48}] et vs., vs. b. leurs sépulcres.
\item[\vref{Jn 2:20}] quarante-six ans pour b. ce temple, et
\item[\vref{Ro 15:20}] de ne pas b. sur le fondement
\item[\vref{Hé 11:7}] encore, craignit, et b. l'arche pour la
\item[\vref{1 Pi 2:7}] que ceux qui b. ont rejetée est
\item[\vref{Ap 21:16}] la ville était b. en carré, et
\end{listverse}

\ConcordanceEntry{Bâton}
\vspace{-2mm}
\begin{listverse}
\item[\vref{Ge 32:10}] Jourdain avec mon b., et mntnt je
\item[\vref{Ge 38:18}] cordon, et ton b. que tu as
\item[\vref{Ge 49:10}] Juda, ni le b. de législateur d'entre
\item[\vref{Ex 12:11}] pieds, et votre b. à la main,
\item[\vref{2 R 4:31}] avait mis le b. sur le visage
\item[\vref{Ps 23:4}] avec moi : Ton b. et ta houlette
\item[\vref{Os 4:12}] bois, et son b. lui répond ; car
\item[\vref{Za 8:4}] chacun aura son b. à la main,
\item[\vref{Mt 10:10}] ni souliers, ni b. ; car l'ouvrier mérite
\item[\vref{Mt 26:55}] épées et des b., com. après un
\item[\vref{Hé 11:21}] Dieu, appuyé sur l'extrémité de son b.
\end{listverse}

\ConcordanceEntry{Battre}
\vspace{-2mm}
\begin{listverse}
\item[\vref{De 25:3}] Il le fera b. de quarante coups,
\item[\vref{Jg 15:8}] Et il les b. dos et ventre,
\item[\vref{1 S 17:32}] ira et se b. contre lui.
\item[\vref{Ps 44:6}] Avec toi ns. b. nos adversaires, par
\item[\vref{Pr 23:35}] On m'a b., diras-tu, et je n'en suis pas
\item[\vref{Ca 5:7}] rencontrée ; ils m'ont b., ils m'ont blessée ;
\item[\vref{Es 53:4}] considéré com. frappé, b. par Dieu et
\item[\vref{Mt 20:19}] de lui, le b. de verges, et
\item[\vref{Mt 23:34}] les uns, vs. b. de verges les
\item[\vref{Mt 24:49}] se met à b. ses compagnons de
\item[\vref{Lu 12:47}] sa volonté, sera b. de plusieurs coups.
\item[\vref{Ac 5:40}] ils les firent b. de verges, ils
\item[\vref{Ac 16:22}] ordonnèrent qu'ils soient b. de verges.
\item[\vref{Ac 22:19}] en prison et b. de verges ds
\item[\vref{2 Co 11:25}] j'ai été b. de verges trois fois, j'ai été
\end{listverse}

\ConcordanceEntry{Beau}
\vspace{-2mm}
\begin{listverse}
\item[\vref{Ge 39:6}] Or Joseph était b. de taille et
\item[\vref{Ex 2:2}] Voyant qu'il était b., elle le cacha
\item[\vref{1 S 9:2}] Saül, jeune et b., et aucun des
\item[\vref{1 S 17:42}] garçon, roux et b. de figure.
\item[\vref{1 R 1:6}] Il était très b. d'apparence, il était
\item[\vref{Ps 45:3}] es le plus b. des fils de
\item[\vref{Pr 15:13}] rend le visage b., mais l'esprit est
\item[\vref{Ca 1:16}] Sulamithe :] Te voilà b., mon bien-aimé, que
\item[\vref{Ac 7:20}] qui fut divinement b.. Et il fut
\item[\vref{Hé 11:23}] que l'enfant était b., et ils ne
\end{listverse}

\ConcordanceEntry{Beauté}
\vspace{-2mm}
\begin{listverse}
\item[\vref{2 S 14:25}] qu'Absalom pour sa b. ; depuis la plante
\item[\vref{Est 1:11}] de montrer sa b. aux peuples et
\item[\vref{Ps 27:4}] pour contempler la b. de Yahweh et
\item[\vref{Ps 45:12}] désirs sur ta b. ; puisqu'il est ton
\item[\vref{Ps 50:2}] qui est d'une b. parfaite.
\item[\vref{Pr 6:25}] ton cœur sa b. et ne te
\item[\vref{Pr 31:30}] trompeuse, et la b. vaine ; mais la
\item[\vref{Es 33:17}] roi ds sa b. ; et ils regarderont
\item[\vref{Es 44:13}] hom., selon la b. d'un hom., afin
\item[\vref{Es 53:2}] en lui ni b., ni splendeur, qnd
\item[\vref{Ez 16:14}] cause de ta b., car elle était
\item[\vref{Ez 27:3}] tu disais : Je suis parfaite en b. !
\item[\vref{Ez 28:17}] cause de ta b., tu as corrompu
\end{listverse}

\ConcordanceEntry{Béelzébul}
\vspace{-2mm}
\begin{listverse}
\item[\vref{Mt 10:25}] père de famille B., à combien plus
\item[\vref{Mt 12:24}] démons que par B., prince des démons.
\item[\vref{Mc 3:22}] est possédé par B. ; c'est par le
\end{listverse}

\ConcordanceEntry{Beer-Schéba}
\vspace{-2mm}
\begin{listverse}
\item[\vref{Ge 21:14}] et fut errante au désert de B.
\item[\vref{Ge 21:31}] appela ce lieu-là B., car ts deux
\item[\vref{Ge 22:19}] allèrent ensemble à B. ; car Abraham demeurait
\item[\vref{Ge 46:1}] et vint à B., et il offrit
\item[\vref{Ge 46:5}] Jacob partit de B., et les enfants
\item[\vref{Jos 19:2}] ds lr. héritage B., Schéba, Molada,
\item[\vref{1 R 19:3}] Il arriva à B., qui appartient à
\item[\vref{Am 5:5}] passez point à B.. Car Guilgal sera
\end{listverse}

\ConcordanceEntry{Bel}
\vspace{-2mm}
\begin{listverse}
\item[\vref{Es 46:1}] B. s'incline sur ses genoux, Nebo est
\item[\vref{Jé 50:2}] Babylone est prise ! B. est confus, Merodac
\item[\vref{Jé 51:44}] Je punirai aussi B. à Babylone, je
\end{listverse}

\ConcordanceEntry{Bélier}
\vspace{-2mm}
\begin{listverse}
\item[\vref{Ge 15:9}] trois ans, un b. de trois ans,
\item[\vref{Ge 22:13}] derrière lui un b. qui était retenu
\item[\vref{Lé 5:15}] à savoir un b. sans défaut, pris
\item[\vref{Lé 9:4}] bœuf et un b. pour l'offrande de
\item[\vref{Lé 16:5}] péché et un b. pour l'holoc.
\item[\vref{Lé 19:21}] à savoir un b. pour la culpabilité.
\item[\vref{Da 8:3}] et voici, un b. se tenait dvt
\item[\vref{Mi 6:7}] des milliers de b. ou à des
\end{listverse}

\ConcordanceEntry{Belle}
\vspace{-2mm}
\begin{listverse}
\item[\vref{Ge 6:2}] des hommes étaient b., les prirent pour
\item[\vref{Ge 11:31}] et Saraï, sa b.-fille, fem. d'Abram,
\item[\vref{Ge 24:16}] fille était très b. de figure ; elle
\item[\vref{Ge 29:17}] mais Rachel était b. de taille et
\item[\vref{1 S 25:3}] bon sens et b. de visage, mais
\item[\vref{2 S 11:2}] fem. était très b. de figure.
\item[\vref{2 S 13:1}] sœur qui était b. et qui se
\item[\vref{1 R 1:4}] fem. était fort b.. Elle prit soin
\item[\vref{1 Ch 2:4}] Et Tamar, b.-fille de Juda,
\item[\vref{Est 1:11}] car elle était b. de figure.
\item[\vref{Est 2:7}] jeune fille était b. de taille et
\item[\vref{Ps 92:2}] C'est une b. chose que de
\item[\vref{Pr 11:22}] Une b. fem. qui se détourne de la
\item[\vref{Ec 3:11}] ttes choses sont b. en lr. temps ;
\item[\vref{Ca 1:5}] noire, je suis b.. Je suis com.
\item[\vref{Lu 15:22}] Apportez la plus b. robe et revêtez-le,
\end{listverse}

\ConcordanceEntry{Belschatsar}
\vspace{-2mm}
\begin{listverse}
\item[\vref{Da 5:1}] Le roi B. donna un grand festin à ses
\item[\vref{Da 5:30}] Cette mm nuit, B., roi des Chaldéens,
\item[\vref{Da 7:1}] première année de B., roi de Babylone,
\end{listverse}

\ConcordanceEntry{Beltschatsar}
\vspace{-2mm}
\begin{listverse}
\item[\vref{Da 1:7}] le nom de B., à Hanania celui
\item[\vref{Da 4:8}] moi Daniel, nommé B., selon le nom
\end{listverse}

\ConcordanceEntry{Benaja}
\vspace{-2mm}
\begin{listverse}
\item[\vref{2 S 23:20}] B., fils de Jehojada, fils d'un vaillant
\item[\vref{1 R 2:35}] le roi établit B., fils de Jehojada,
\item[\vref{1 Ch 11:22}] B. aussi, fils de Jehojada, fils d'un
\end{listverse}

\ConcordanceEntry{Bénédiction}
\vspace{-2mm}
\begin{listverse}
\item[\vref{Ge 27:35}] tromperie, et il a enlevé ta b.
\item[\vref{Ge 39:5}] Joseph ; et la b. de Yahweh fut
\item[\vref{Ge 49:28}] d'eux selon la b. qui lui était
\item[\vref{De 11:26}] dvt vs. la b. et la malédiction :
\item[\vref{De 11:27}] La b., si vs. obéissez aux commandements de
\item[\vref{De 23:5}] la malédiction en b., parce que Yahweh,
\item[\vref{De 28:2}] Voici ttes les b. qui viendront sur
\item[\vref{Jos 8:34}] loi, tant les b. que les malédictions,
\item[\vref{Né 9:5}] au-dessus de tte b. et de tte
\item[\vref{Né 13:2}] Dieu avait changé la malédiction en b.
\item[\vref{Ps 24:5}] Il obtiendra la b. de Yahweh et
\item[\vref{Ps 109:17}] plaisir à la b., que la bénédiction
\item[\vref{Pr 10:6}] Les b. seront sur la tête du juste,
\item[\vref{Pr 10:22}] La b. de Yahweh est celle qui enrichit,
\item[\vref{Ag 2:19}] depuis ce jour-ci, je donnerai la b.
\item[\vref{Za 8:13}] vs. serez en b.. Ne craignez pas,
\item[\vref{Mal 2:2}] je maudirai vos b. ; et déjà mm
\item[\vref{Mal 3:10}] votre faveur la b., jusqu'à ce qu'il
\item[\vref{Ac 20:35}] a plus de b. à donner qu'à
\item[\vref{Ro 4:6}] David exprime la b. de l'hom. à
\item[\vref{1 Co 10:16}] La coupe de b., que ns. bénissons,
\item[\vref{Ga 3:14}] afin que la b. d'Abraham ait son
\item[\vref{Ep 1:3}] bénis de ttes b. spirituelles ds les
\item[\vref{Hé 6:7}] labourée, reçoit la b. de Dieu ;
\item[\vref{Hé 12:17}] désirant hériter la b., il fut rejeté,
\item[\vref{Ja 3:10}] bouche sortent la b. et la malédiction.
\item[\vref{1 Pi 3:9}] vs. êtes appelés, afin d'hériter la b.
\end{listverse}

\ConcordanceEntry{Ben-Hadad}
\vspace{-2mm}
\begin{listverse}
\item[\vref{1 R 15:18}] les envoya vers B., fils de Thabrimmon,
\item[\vref{1 R 15:20}] Et B. écouta le roi Asa ; il envoya
\item[\vref{1 R 20:1}] Alors B., roi de Syrie rassembla tte son
\item[\vref{1 R 20:20}] Israël les poursuivit. B., roi de Syrie,
\item[\vref{1 R 20:30}] de reste. Et B. s'enfuit, entra ds
\item[\vref{2 R 6:24}] après cela que B., roi de Syrie,
\item[\vref{2 R 13:3}] les mains de B., fils de Hazaël,
\item[\vref{2 R 13:24}] Syrie, mourut, et B., son fils, régna
\item[\vref{2 R 13:25}] des mains de B., fils d'Hazaël, les
\end{listverse}

\ConcordanceEntry{Bénir}
\vspace{-2mm}
\begin{listverse}
\item[\vref{Ge 5:2}] femelle, et les b., et il lr.
\item[\vref{Ge 22:18}] la terre seront b. en ta postérité,
\item[\vref{Ge 27:33}] et je l'ai b.. Aussi sera-t-il béni !
\item[\vref{Ge 49:28}] dit en les b.. Il bénit chacun
\item[\vref{No 6:24}] Yahweh te b., et te garde !
\item[\vref{No 23:20}] la parole pour b. : Puisqu'il a béni,
\item[\vref{De 2:7}] ton Dieu, t'a b. ds tt le
\item[\vref{De 28:3}] Tu seras b. ds la ville, et tu seras
\item[\vref{Né 8:6}] Puis Esdras b. Yahweh, le grand
\item[\vref{Job 10:3}] mains, et à b. les desseins des
\item[\vref{Job 42:12}] Ainsi Yahweh b. le dernier état
\item[\vref{Ps 1:1}] B. est l'hom. qui ne marche pas
\item[\vref{Ps 16:7}] Je b. Yahweh qui me donne conseil ; je
\item[\vref{Ps 32:2}] B. est l'hom. à qui Yahweh n'impute
\item[\vref{Ps 63:5}] ainsi je te b. dc tte ma
\item[\vref{Ps 84:6}] B. est l'hom. dont la force est
\item[\vref{Ps 103:1}] David. Mon âme, b. Yahweh ! Et que
\item[\vref{Es 61:9}] sont la race que Yahweh aura b.
\item[\vref{Da 2:19}] nuit. Et Daniel b. le Dieu des
\item[\vref{Da 12:12}] B. est celui qui attendra et qui
\item[\vref{Mt 5:3}] B. sont les pauvres en esprit, car
\item[\vref{Mt 5:4}] B. sont ceux qui pleurent, car ils
\item[\vref{Mt 5:11}] B. serez-vs., lorsqu'on vs. outragera, qu'on vs.
\item[\vref{Mt 5:44}] Aimez vos ennemis, b. ceux qui vs.
\item[\vref{Mt 13:16}] Mais b. sont vos yeux, car ils voient ;
\item[\vref{Mt 15:36}] et après avoir b. Dieu, il les
\item[\vref{Mt 16:17}] dit : Tu es b., Simon, fils de
\item[\vref{Mt 21:9}] Fils de David ! B. soit celui qui
\item[\vref{Mt 24:46}] B. est ce serviteur que son maître
\item[\vref{Mt 26:27}] la coupe, et b. Dieu, il la
\item[\vref{Ro 1:25}] Créateur, qui est b. éternellement. Amen !
\item[\vref{Ro 4:8}] B. est l'hom. à qui le Seign.
\item[\vref{Ap 22:7}] à tte vitesse. B. est celui qui
\end{listverse}

\ConcordanceEntry{Benjamin}
\vspace{-2mm}
\begin{listverse}
\item[\vref{Ge 35:18}] de Ben-Oni, mais son père l'appela B.
\item[\vref{Ge 43:15}] leurs mains, et B., ils se levèrent
\item[\vref{Ge 44:12}] fut trouvée ds le sac de B.
\item[\vref{Ge 49:27}] B. est un loup qui déchirera ; le
\item[\vref{No 1:36}] Des fils de B., selon leurs générations,
\item[\vref{De 33:12}] Il dit de B. : Le bien-aimé de
\item[\vref{Jos 18:11}] des fils de B. selon leurs familles,
\item[\vref{Jg 20:43}] Ils environnèrent B., le poursuivirent, l'écrasèrent
\item[\vref{Jg 20:45}] parmi ceux de B. qui tournèrent le
\item[\vref{Ac 13:21}] la tribu de B. ; et ainsi se
\item[\vref{Ap 7:8}] la tribu de B., douze mille marqués
\end{listverse}

\ConcordanceEntry{Bérée}
\vspace{-2mm}
\begin{listverse}
\item[\vref{Ac 17:10}] et Silas pour B.. Lorsqu'ils furent arrivés,
\item[\vref{Ac 20:4}] Asie : Sopater de B., Aristarque et Second
\end{listverse}

\ConcordanceEntry{Berger}
\vspace{-2mm}
\begin{listverse}
\item[\vref{Ge 4:2}] et Abel fut b., et Caïn laboureur.
\item[\vref{Ge 46:32}] ces hommes sont b., ils se sont
\item[\vref{Ex 3:1}] Or Moïse fut b. du troupeau de
\item[\vref{1 S 17:40}] le sac de b. et ds sa
\item[\vref{Ps 23:1}] Yahweh est mon b.: Je ne manquerai
\item[\vref{Es 40:11}] troupeau com. un b., il rassemblera les
\item[\vref{Es 44:28}] Il est mon b., et il accomplira
\item[\vref{Jé 31:10}] gardera com. un b. garde son troupeau.
\item[\vref{Am 3:12}] Yahweh : Comme un b. sauvait de la
\item[\vref{Am 7:14}] prophète ; j'étais un b., et je cueillais
\item[\vref{Za 13:7}] réveille-toi contre mon B., et sur l'hom.
\item[\vref{Mt 25:32}] autres, com. le b. sépare les brebis
\item[\vref{Mt 26:31}] Je frapperai le b., et les brebis
\item[\vref{Lu 2:8}] mm contrée, des b. qui couchaient ds
\item[\vref{Jn 10:2}] porte est le b. des brebis.
\item[\vref{Jn 10:11}] suis le bon b. ; le bon berger
\item[\vref{Jn 10:16}] un seul troupeau, et un seul b.
\end{listverse}

\ConcordanceEntry{Besoin}
\vspace{-2mm}
\begin{listverse}
\item[\vref{Ps 104:14}] plantes pour le b. de l'hom., faisant
\item[\vref{Mt 6:8}] quoi vs. avez b., avant que vs.
\item[\vref{Mt 9:12}] santé qui ont b. de médecin, mais
\item[\vref{Mt 21:3}] Seign. en a b.. Et aussitôt il
\item[\vref{Jn 2:25}] qu'il n'avait pas b. qu'on lui rende
\item[\vref{Jn 16:30}] tu n'as pas b. que quelqu'un t'interroge ;
\item[\vref{Ac 17:25}] com. s'il avait b. de quoi que
\item[\vref{Ac 20:34}] pourvu à mes b. et à ceux
\item[\vref{1 Co 12:21}] Je n'ai pas b. de toi. Ni
\item[\vref{1 Co 12:24}] n'en ont pas b.. Mais Dieu a
\item[\vref{Ep 4:28}] à celui qui est ds le b.
\item[\vref{Ph 2:25}] envoyé de quoi pourvoir à mes b.
\item[\vref{Ph 4:11}] cause de mes b., car j'ai appris
\item[\vref{Ph 4:19}] dont vs. aurez b. selon ses richesses,
\item[\vref{1 Th 4:12}] du dehors, et que vs. n'ayez b. de rien.
\item[\vref{Hé 5:12}] vs. avez encore b. qu'on vs. enseigne
\item[\vref{Hé 7:11}] la loi) quel b. était-il après cela
\item[\vref{Hé 10:36}] que vs. avez b. de patience, afin
\item[\vref{1 Jn 2:27}] vs. n'avez pas b. qu'on vs. enseigne ;
\item[\vref{Ap 3:17}] et je n'ai b. de rien ; mais
\item[\vref{Ap 21:23}] ville n'a pas b. du soleil ni
\item[\vref{Ap 22:5}] et ils n'auront b. ni de lumière,
\end{listverse}

\ConcordanceEntry{Bétail}
\vspace{-2mm}
\begin{listverse}
\item[\vref{Ge 1:24}] lr. espèce, le b., les reptiles, et
\item[\vref{Ge 9:10}] oiseaux que le b., et ts les
\item[\vref{Ge 13:2}] très riche en b., en argent, et
\item[\vref{Ge 46:6}] emmenèrent aussi lr. b. et leurs biens
\item[\vref{Ex 9:4}] Yahweh distinguera le b. des Israélites du
\item[\vref{Ex 34:19}] tant du gros que du menu b.
\item[\vref{1 S 15:3}] bœufs et menu b., chameaux et ânes.
\item[\vref{Ps 104:14}] l'herbe pour le b., et les plantes
\item[\vref{Ps 107:38}] il ne laisse point diminuer lr. b.
\item[\vref{Ps 148:10}] et tt le b., reptiles et oiseaux
\item[\vref{Ec 2:7}] et de menu b. que ts ceux
\item[\vref{So 1:3}] l'hom. et le b. ; je consumerai les
\item[\vref{Za 13:5}] à gouverner du b. dès ma jeunesse.
\item[\vref{Jn 4:12}] ainsi que ses enfants et son b. ?
\end{listverse}

\ConcordanceEntry{Bête}
\vspace{-2mm}
\begin{listverse}
\item[\vref{Ge 1:28}] et sur tte b. qui se meut
\item[\vref{Ge 37:20}] ns. dirons qu'une b. féroce l'a dévoré,
\item[\vref{Ex 23:29}] et que les b. des champs ne
\item[\vref{Lé 18:23}] aussi avec une b. pour te souiller
\item[\vref{Lé 20:25}] pourquoi séparez les b. pures de celles
\item[\vref{Lé 26:22}] contre vs. les b. des champs, qui
\item[\vref{No 3:41}] prendras aussi les b. des Lévites, à
\item[\vref{De 14:4}] sont ici les b. que vs. mangerez :
\item[\vref{Ps 49:13}] est semblable aux b. que l'on égorge.
\item[\vref{Ps 73:22}] j'étais com. une b. ds ta présence.
\item[\vref{Pr 12:10}] vie de sa b., mais les entrailles
\item[\vref{Ec 3:19}] sort de la b. est un mm
\item[\vref{Es 43:20}] Les b. des champs me glorifieront, les serpents
\item[\vref{Jé 15:3}] cieux et les b. de la terre
\item[\vref{Da 4:25}] demeure avec les b. des champs, et
\item[\vref{Da 7:3}] Puis quatre grandes b. montèrent de la
\item[\vref{Mal 1:8}] vs. amenez une b. aveugle pour la
\item[\vref{Mc 1:13}] était avec les b. sauvages et les
\item[\vref{Ac 28:4}] barbares virent cette b. suspendue à sa
\item[\vref{1 Co 15:32}] combattu contre les b. à Ephèse ds
\item[\vref{Hé 12:20}] si mm une b. touche la montagne,
\item[\vref{Ap 6:8}] et par les b. sauvages de la
\item[\vref{Ap 13:1}] la mer une b. qui avait sept
\item[\vref{Ap 13:11}] vis une autre b. qui montait de
\item[\vref{Ap 13:18}] nombre de la b., car c'est un
\item[\vref{Ap 15:2}] avaient vaincu la b. et son image,
\item[\vref{Ap 16:2}] marque de la b., et ceux qui
\item[\vref{Ap 16:13}] gueule de la b., et de la
\item[\vref{Ap 17:3}] assise sur une b. écarlate, pleine de
\item[\vref{Ap 19:20}] Et la b. fut prise, et avec elle le
\end{listverse}

\ConcordanceEntry{Béthanie}
\vspace{-2mm}
\begin{listverse}
\item[\vref{Mt 21:17}] pour aller à B., où il passa
\item[\vref{Mc 11:1}] Bethphagé et de B., vers le Mont
\item[\vref{Jn 11:1}] qui était de B., village de Marie
\end{listverse}

\ConcordanceEntry{Beth-Aven}
\vspace{-2mm}
\begin{listverse}
\item[\vref{Jos 7:2}] est près de B., à l'orient de
\item[\vref{Os 4:15}] montez pas à B., et ne jurez
\item[\vref{Os 10:5}] jeunes vaches de B. ; car le peuple
\end{listverse}

\ConcordanceEntry{Béthel}
\vspace{-2mm}
\begin{listverse}
\item[\vref{Ge 12:8}] à l'orient de B., et il dressa
\item[\vref{Ge 13:3}] du midi à B., jusqu'au lieu où
\item[\vref{Ge 28:19}] le nom de B. ; mais auparavant la
\item[\vref{Ge 35:1}] Lève-toi, monte à B., et demeures-y ; là,
\item[\vref{Ge 35:8}] ensevelie au-dessous de B., sous un chêne,
\item[\vref{Jg 20:18}] vers Dieu à B. pour le consulter,
\item[\vref{1 R 12:29}] ces veaux à B., et il mit
\item[\vref{2 R 23:15}] qui était à B. et le haut
\item[\vref{Jé 48:13}] à cause de B., qui était sa
\item[\vref{Os 10:15}] B. vs. fera de mm, à cause
\item[\vref{Am 5:5}] Ne cherchez pas B., et n'allez pas
\end{listverse}

\ConcordanceEntry{Béthesda}
\vspace{-2mm}
\begin{listverse}
\item[\vref{Jn 5:2}] appelé en hébreu B., ayant cinq portiques ;
\end{listverse}

\ConcordanceEntry{Bethléhem}
\vspace{-2mm}
\begin{listverse}
\item[\vref{Ge 35:19}] sur le chemin d'Ephrata, qui est B.
\item[\vref{Jg 17:7}] jeune hom. de B. de Juda, ville
\item[\vref{1 S 16:4}] il alla à B.. Les anciens de
\item[\vref{1 S 17:12}] hom. Ephratien, de B. de Juda, nommé
\item[\vref{Mi 5:1}] Mais toi, B. Ephrata, petite pour
\item[\vref{Mt 2:1}] étant né à B., ville de Juda,
\item[\vref{Mt 2:16}] qui étaient à B., et ds tt
\item[\vref{Jn 7:42}] du village de B., d'où était David ?
\end{listverse}

\ConcordanceEntry{Bethsaïda}
\vspace{-2mm}
\begin{listverse}
\item[\vref{Mt 11:21}] malheur à toi, B. ! car si les
\item[\vref{Mc 8:22}] se rendirent à B., et on lui
\item[\vref{Lu 9:10}] désert, près de la ville appelée B.
\item[\vref{Jn 1:44}] Philippe était de B., la ville d'André
\end{listverse}

\ConcordanceEntry{Beth-Schémesch}
\vspace{-2mm}
\begin{listverse}
\item[\vref{1 S 6:9}] l'arche monte vers B., par le chemin
\item[\vref{1 S 6:19}] des gens de B. parce qu'ils avaient
\item[\vref{2 R 14:11}] de Juda, à B., qui est à
\item[\vref{2 R 14:13}] fils d'Achazia, à B.. Puis il vint
\end{listverse}

\ConcordanceEntry{Bethuel}
\vspace{-2mm}
\begin{listverse}
\item[\vref{Ge 22:23}] B. a engendré Rebecca. Milca enfanta ces
\end{listverse}

\ConcordanceEntry{Betsaleel}
\vspace{-2mm}
\begin{listverse}
\item[\vref{Ex 31:2}] par son nom B., fils d'Uri, fils
\item[\vref{Ex 36:1}] Et B. et Oholiab, et ts les hommes
\item[\vref{Ex 37:1}] Puis B. fit l'arche de bois d'acacia. Sa
\item[\vref{Ex 38:22}] Et B., fils d'Uri, fils de Hur, de
\end{listverse}

\ConcordanceEntry{Biche}
\vspace{-2mm}
\begin{listverse}
\item[\vref{Ge 49:21}] Nephtali est une b. en liberté ; il
\item[\vref{2 S 22:34}] à ceux des b., et il me
\item[\vref{Job 39:4}] bas ? Observes-tu les b. de rochers qnd
\item[\vref{Ps 42:2}] Comme une b. soupire après des
\item[\vref{Pr 5:19}] com. d'une b. des amours, et
\item[\vref{Ca 2:17}] le faon des b., sur les montagnes
\item[\vref{Jé 14:5}] Même la b. met bas son faon ds le
\item[\vref{Ha 3:19}] à ceux des b., et me fait
\end{listverse}

\ConcordanceEntry{Bien}
\vspace{-2mm}
\begin{listverse}
\item[\vref{Ge 2:9}] la connaissance du b. et du mal.
\item[\vref{Ge 3:5}] dieux, connaissant le b. et le mal.
\item[\vref{Ge 4:7}] Si tu agis b., tu relèveras ton
\item[\vref{Ge 12:5}] avec ts les b. qu'ils avaient acquis,
\item[\vref{Ge 24:10}] disposition ts les b.. Il partit dc
\item[\vref{Ge 32:9}] parenté, et je te ferai du b.
\item[\vref{Ge 44:4}] avez-vs. rendu le mal pour le b. ?
\item[\vref{Ge 50:20}] l'a changé en b., pour accomplir ce
\item[\vref{Ex 1:20}] Dieu fit du b. aux sages-femmes ; et
\item[\vref{Ex 18:17}] Ce que tu fais n'est pas b.
\item[\vref{De 8:16}] t'éprouver, pour te faire ensuite du b.,
\item[\vref{De 30:9}] fera abonder en b. ds tte l'œuvre
\item[\vref{De 30:15}] vie et le b., la mort et
\item[\vref{Ru 1:21}] partie pleine de b. et Yahweh me
\item[\vref{1 S 17:27}] dit : C'est le b. qu'on fera à
\item[\vref{1 S 24:18}] m'as rendu le b. pour le mal
\item[\vref{2 S 16:12}] me rendra le b. au lieu des
\item[\vref{1 R 22:13}] celle de chacun d'eux ! Annonce du b. !
\item[\vref{2 R 7:9}] Nous n'agissons pas b. ! Ce jour est
\item[\vref{Né 6:19}] racontaient mm du b. de lui en
\item[\vref{Est 10:3}] frères, procurant le b. de son peuple
\item[\vref{Job 2:10}] de Dieu les b., et ns. n'en
\item[\vref{Ps 14:3}] qui fasse le b., pas mm un
\item[\vref{Ps 16:2}] mon Seign., mon b. ne te revient
\item[\vref{Ps 17:14}] ventre de tes b. ; leurs enfants sont
\item[\vref{Ps 34:11}] qui cherchent Yahweh ne manquent d'aucun b.
\item[\vref{Ps 34:15}] mal, et fais-le b. ; cherche la paix
\item[\vref{Ps 35:12}] mal pour le b., tâchant de m'ôter
\item[\vref{Ps 65:12}] l'année de tes b., et tes voies
\item[\vref{Pr 8:21}] pour donner des b. durables en héritage
\item[\vref{Pr 11:17}] doux fait du b. à son âme,
\item[\vref{Pr 12:2}] L'hom. de b. obtient la faveur
\item[\vref{Pr 12:27}] gibier ; mais les b. précieux de l'hom.
\item[\vref{Pr 15:16}] un peu de b. avec la crainte
\item[\vref{Ec 9:18}] seul hom. pécheur détruit beaucoup de b.
\item[\vref{Es 5:20}] appellent le mal b. et le bien
\item[\vref{Es 7:15}] rejeter le mal et choisir le b.
\item[\vref{Es 48:7}] dises pas : Voici, je les savais b.
\item[\vref{Am 5:15}] et aimez le b., et établissez la
\item[\vref{Za 8:15}] de faire du b. à Jérus. et
\item[\vref{Mt 12:29}] et piller ses b., sans avoir auparavant
\item[\vref{Mt 19:22}] triste, parce qu'il avait de grands b.
\item[\vref{Mt 24:47}] vérité, il l'établira sur ts ses b.
\item[\vref{Lu 6:9}] de faire du b. les jours de
\item[\vref{Lu 6:26}] hommes diront du b. de vs., car
\item[\vref{Lu 8:3}] plusieurs autres qui l'assistaient de leurs b.
\item[\vref{Lu 15:12}] la part de b. qui m'appartient. Et
\item[\vref{Lu 16:1}] accusé dvt lui com. dissipant ses b.
\item[\vref{Lu 19:8}] moitié de mes b. aux pauvres ; et
\item[\vref{Ac 2:45}] possessions et leurs b., et les distribuaient
\item[\vref{Ro 7:19}] fais pas le b. que je veux,
\item[\vref{Ro 7:21}] veux faire le b., le mal est
\item[\vref{Ro 8:28}] choses concourent au b. de ceux qui
\item[\vref{1 Co 9:11}] parmi vs. des b. spirituels, est-ce une
\item[\vref{2 Co 5:10}] reçoive selon le b. ou le mal
\item[\vref{2 Co 12:14}] demande pas vos b., mais c'est vs.-mêmes
\item[\vref{Ga 6:9}] de faire le b. ; car ns. moissonnerons
\item[\vref{Ga 6:10}] temps, faisons du b. envers ts, mais
\item[\vref{Ep 6:8}] du Seign. le b. qu'il aura fait.
\item[\vref{Col 1:5}] de l'espérance des b. qui vs. sont
\item[\vref{Hé 9:11}] com. Grand-Prêtre des b. à venir ; il
\item[\vref{Hé 10:34}] l'enlèvement de vos b., sachant en vs.-mêmes
\item[\vref{Hé 13:16}] part de vos b., car Dieu prend
\item[\vref{1 Pi 2:15}] qu'en faisant le b. vs. fermiez la
\item[\vref{1 Pi 2:20}] vs. faites le b. et que vs.
\item[\vref{1 Pi 3:6}] ce qui est b. et sans vs.
\item[\vref{1 Pi 3:11}] et fasse le b., qu'il recherche la
\item[\vref{1 Jn 3:17}] quelqu'un possède les b. du monde, et
\item[\vref{3 Jn 1:11}] B.-aimé, n'imite pas le mal, mais
\end{listverse}

\ConcordanceEntry{Bienfait}
\vspace{-2mm}
\begin{listverse}
\item[\vref{Jg 9:16}] fait selon les b. qu'il a rendus
\item[\vref{1 S 12:7}] sur ts les b. que Yahweh vs.
\item[\vref{2 Ch 32:25}] pas reconnaissant du b. qu'il avait reçu ;
\item[\vref{Job 37:13}] la terre, soit pour répandre ses b.
\item[\vref{Ps 5:8}] comblé de tes b., j'entrerai ds ta
\item[\vref{Ps 51:20}] ta grâce tes b. sur Sion, édifie
\item[\vref{Ps 103:2}] et n'oublie pas un de ses b. !
\item[\vref{Ps 116:12}] Yahweh ? Tous ses b. sont sur moi ?
\item[\vref{Pr 19:17}] à Yahweh, qui lui rendra son b.
\item[\vref{Es 63:7}] pour ts les b. que Yahweh ns.
\item[\vref{Ac 4:9}] aujourd'hui sur un b. accordé à un
\end{listverse}

\ConcordanceEntry{Bienheureux}
\vspace{-2mm}
\begin{listverse}
\item[\vref{Job 29:11}] disait que j'étais b., et l'œil qui
\item[\vref{Ps 146:5}] Ô que b. est celui à qui le Dieu
\item[\vref{Es 30:18}] de jugement : Ô b. sont ts ceux
\item[\vref{Tit 2:13}] en attendant la b. espérance, et l'apparition
\end{listverse}

\ConcordanceEntry{Bienveillance}
\vspace{-2mm}
\begin{listverse}
\item[\vref{1 R 2:7}] Tu traiteras avec b. les fils de
\item[\vref{1 R 3:6}] usé d'une grande b. envers ton serviteur
\item[\vref{2 Ch 1:8}] usé d'une grande b. envers David, mon
\item[\vref{Esd 9:9}] a incliné la b. des rois de
\item[\vref{Est 2:17}] grâces et sa b. plus que ttes
\item[\vref{Ps 5:13}] l'environnes de ta b. com. d'un bouclier.
\item[\vref{Ps 106:4}] moi selon la b. que tu portes
\item[\vref{Pr 14:9}] les hommes droits se trouve la b.
\item[\vref{Es 49:8}] temps de la b., et je t'ai
\item[\vref{Ac 27:3}] traitait Paul avec b., lui permit d'aller
\item[\vref{1 Co 7:3}] sa fem. la b. qui lui est
\item[\vref{Ga 5:22}] la bonté, la b., la foi, la
\item[\vref{Ep 6:7}] servant avec b., com. servant le
\end{listverse}

\ConcordanceEntry{Bildad}
\vspace{-2mm}
\begin{listverse}
\item[\vref{Job 2:11}] Eliphaz de Théman, B. de Schuach, et
\item[\vref{Job 8:1}] Alors B. de Schuach prit la parole et
\item[\vref{Job 18:1}] Alors B. de Schuach prit la parole et
\end{listverse}

\ConcordanceEntry{Bilha}
\vspace{-2mm}
\begin{listverse}
\item[\vref{Ge 29:29}] Et Laban donna B., sa servante, à
\item[\vref{Ge 30:3}] Voici ma servante B. ; va vers elle ;
\item[\vref{Ge 30:7}] B., servante de Rachel, conçut encore et
\item[\vref{Ge 35:22}] et coucha avec B., concubine de son
\end{listverse}

\ConcordanceEntry{Bithynie}
\vspace{-2mm}
\begin{listverse}
\item[\vref{Ac 16:7}] à entrer en B. ; mais l'Esprit de
\item[\vref{1 Pi 1:1}] Galatie, la Cappadoce, l'Asie et la B.,
\end{listverse}

\ConcordanceEntry{Blanc}
\vspace{-2mm}
\begin{listverse}
\item[\vref{Ge 40:16}] corbeilles de pain b. sur ma tête.
\item[\vref{No 12:10}] frappée d'une lèpre b. com. la neige ;
\item[\vref{1 S 12:2}] vieux et tt b., et voici, mes
\item[\vref{Est 8:15}] pourpre et de b., avec une grande
\item[\vref{Job 6:6}] fade ? Trouve-t-on du goût ds un b. d'œuf ?
\item[\vref{Ps 51:9}] je serai plus b. que la neige.
\item[\vref{Ps 68:15}] pays, il devint b. com. la neige
\item[\vref{Da 7:9}] Son vêt. était b. com. la neige,
\item[\vref{Mt 5:36}] peux pas rendre b. ou noir un
\item[\vref{Mt 28:3}] et son vêt. b. com. de la
\item[\vref{Jn 20:12}] anges vêtus de b., assis à la
\item[\vref{Ap 2:17}] donnerai un caillou b., et sur ce
\item[\vref{Ap 6:2}] vis un cheval b. ; celui qui était
\item[\vref{Ap 15:6}] lin pur et b., et ayant des
\item[\vref{Ap 19:11}] parut un cheval b.. Et celui qui
\item[\vref{Ap 19:14}] sur des chevaux b., revêtues de fin
\item[\vref{Ap 20:11}] un grand trône b., et celui qui
\end{listverse}

\ConcordanceEntry{Blanchir}
\vspace{-2mm}
\begin{listverse}
\item[\vref{Job 41:23}] pour une tête b. de vieillesse.
\item[\vref{Es 1:18}] l'écarlate, ils seront b. com. la neige ;
\item[\vref{Da 11:35}] épurés, purifiés et b., jusqu'au temps de
\item[\vref{Da 12:10}] Plusieurs seront purifiés, b. et éprouvés ; mais
\item[\vref{Mt 23:27}] semblables aux sépulcres b., qui paraissent beaux
\item[\vref{Mc 9:3}] foulon sur la terre qui puisse b. ainsi.
\item[\vref{Jn 4:35}] champs qui déjà b. pour la moisson.
\item[\vref{Ap 7:14}] ont lavé et b. leurs longues robes
\end{listverse}

\ConcordanceEntry{Blasphémateur}
\vspace{-2mm}
\begin{listverse}
\item[\vref{1 Ti 1:13}] auparavant étais un b., un persécuteur, et
\end{listverse}

\ConcordanceEntry{Blasphème}
\vspace{-2mm}
\begin{listverse}
\item[\vref{Es 37:3}] répréhension et de b. ; car les enfants
\item[\vref{Mt 12:31}] péché et tt b. sera pardonné aux
\item[\vref{Mt 26:65}] entendu mntnt son b.. Que vs. en
\item[\vref{Jn 10:33}] mais pour un b., et parce que,
\item[\vref{Ap 2:9}] riche, et le b. de ceux qui
\item[\vref{Ap 13:1}] sur ses têtes des noms de b.
\item[\vref{Ap 13:5}] d'orgueil, et des b. ; et il lui
\item[\vref{Ap 17:3}] de noms de b., ayant sept têtes
\end{listverse}

\ConcordanceEntry{Blasphémer}
\vspace{-2mm}
\begin{listverse}
\item[\vref{Lé 24:11}] la fem. israélite b. et maudit le
\item[\vref{Lé 24:16}] celui qui aura b. le Nom de
\item[\vref{1 S 17:45}] de l'armée d'Israël, que tu as b.
\item[\vref{2 S 12:14}] Yahweh de le b., à cause de
\item[\vref{Job 1:5}] péché, et ont-ils b. contre Dieu ds
\item[\vref{Job 2:5}] s'il ne te b. pas en face.
\item[\vref{Ps 74:18}] Que l'ennemi a b. Yahweh, et qu'un
\item[\vref{Es 52:5}] mon Nom est b. continuellement chaque jour.
\item[\vref{Es 65:7}] et qui m'ont b. sur les collines ;
\item[\vref{Mt 9:3}] disaient au dedans d'eux : Cet hom. b.
\item[\vref{Jn 10:36}] dites-vs. que je b., moi que le
\item[\vref{Ac 26:11}] les forçais à b.. Dans mes excès
\item[\vref{Ro 2:24}] de Dieu est b. parmi les Gentils
\item[\vref{1 Ti 1:20}] par ce châtiment à ne plus b.
\item[\vref{1 Ti 6:1}] afin qu'on ne b. pas le Nom
\item[\vref{1 Pi 4:14}] vs., lequel est b. par ceux qui
\item[\vref{2 Pi 2:2}] la voie de la vérité sera b.
\item[\vref{Jud 1:8}] méprisent l'autorité et b. contre les dignités.
\item[\vref{Ap 13:6}] sa bouche pour b. contre Dieu, pour
\item[\vref{Ap 16:11}] Et ils b. le Dieu du ciel à cause
\end{listverse}

\ConcordanceEntry{Blé}
\vspace{-2mm}
\begin{listverse}
\item[\vref{Ge 41:57}] pour acheter du b. auprès de Joseph ;
\item[\vref{De 8:8}] un pays de b., d'orge, de vignes,
\item[\vref{De 33:28}] un pays de b. et de vin,
\item[\vref{Jos 5:11}] ils mangèrent du b. du pays, savoir,
\item[\vref{Né 13:5}] les dîmes du b., du vin et
\item[\vref{Ps 72:16}] Les b. abonderont ds le pays, au sommet
\item[\vref{Ps 78:24}] nourriture et il lr. donna le b. du ciel.
\item[\vref{Ca 7:3}] un tas de b. entouré de lis.
\item[\vref{Es 28:28}] Le b. avec lequel on fait le pain
\item[\vref{Ez 36:29}] souillures, j'appellerai le b., je le multiplierai,
\item[\vref{Joë 1:17}] parce que le b. a manqué.
\item[\vref{Mt 13:25}] l'ivraie parmi le b., puis s'en alla.
\item[\vref{Mt 13:30}] mais amassez le b. ds mon grenier.
\item[\vref{Mc 2:23}] des champs de b.. Ses disciples en
\item[\vref{Jn 12:24}] le grain de b. qui est tombé
\item[\vref{Ac 7:12}] y avait du b. en Egypte, il
\item[\vref{1 Co 9:10}] qui foule le b., le foule avec
\item[\vref{Ap 6:6}] Une mesure de b. pour un denier,
\item[\vref{Ap 18:13}] de farine, du b., des bœufs, des
\end{listverse}

\ConcordanceEntry{Blesser}
\vspace{-2mm}
\begin{listverse}
\item[\vref{De 32:39}] fais vivre, je b. et je guéris ;
\item[\vref{Jg 20:39}] frapper et à b. à mort environ
\item[\vref{1 R 22:34}] sortir du camp, car je suis b.
\item[\vref{2 Ch 13:17}] d'Israël cinq cent mille hommes d'élite b. à mort.
\item[\vref{Job 5:18}] la bande ; il b. et ses mains
\item[\vref{Ps 109:22}] mon cœur est b. au-dedans de moi.
\item[\vref{Pr 12:18}] dont les paroles b. com. des pointes
\item[\vref{Da 11:26}] torrent, et beaucoup de gens tomberont b. à mort.
\item[\vref{Joë 2:8}] au travers des épées sans être b.
\item[\vref{Za 11:16}] celles qui sont b., et il ne
\item[\vref{Lu 20:12}] mais ils le b. aussi, et le
\item[\vref{1 Co 8:12}] et que vs. b. lr. conscience qui
\end{listverse}

\ConcordanceEntry{Blessure}
\vspace{-2mm}
\begin{listverse}
\item[\vref{Ge 4:23}] hom. pour ma b., et un jeune
\item[\vref{Lé 24:19}] aura fait une b. à son prochain,
\item[\vref{Es 1:6}] n'y a que b., meurtrissures et plaies
\item[\vref{Es 30:26}] Yahweh bandera la b. de son peuple,
\item[\vref{Jé 10:19}] cause de ma b. ! Ma plaie est
\item[\vref{Jé 30:12}] parle Yahweh : Ta b. est incurable, ta
\item[\vref{Da 6:23}] sur lui aucune b., parce qu'il avait
\item[\vref{Na 3:19}] remède à ta b., ta plaie est
\item[\vref{Za 13:6}] dc dire ces b. que tu as
\item[\vref{2 Co 11:23}] plus ; par les b., plus qu'eux ; par
\item[\vref{Ap 13:3}] mort, mais sa b. mortelle fut guérie.
\end{listverse}

\ConcordanceEntry{Boanergès}
\vspace{-2mm}
\begin{listverse}
\item[\vref{Mc 3:17}] le nom de B., ce qui veut
\end{listverse}

\ConcordanceEntry{Boaz}
\vspace{-2mm}
\begin{listverse}
\item[\vref{Ru 2:1}] de la famille d'Elimélec, qui s'appelait B.
\item[\vref{Ru 4:9}] Alors B. dit aux anciens et à tt
\item[\vref{Ru 4:13}] Ainsi B. prit Ruth, et elle fut sa
\item[\vref{Ru 4:21}] Salmon engendra B. ; Boaz engendra Obed ;
\item[\vref{2 Ch 3:17}] droite Jakin, et celle de gauche B.
\item[\vref{Mt 1:5}] Salmon engendra B., de Rahab ; Boaz
\end{listverse}

\ConcordanceEntry{Bœuf}
\vspace{-2mm}
\begin{listverse}
\item[\vref{Ge 46:32}] brebis et leurs b., et tt ce
\item[\vref{Ex 10:9}] brebis et nos b. ; car ns. avons
\item[\vref{No 18:17}] le premier-né du b., ni le premier-né
\item[\vref{De 25:4}] n'emmuselleras point ton b. lorsqu'il foulera le
\item[\vref{Job 42:12}] mille couples de b. et mille ânesses.
\item[\vref{Ps 106:20}] la figure d'un b. qui mange l'herbe.
\item[\vref{Pr 7:22}] elle, com. un b. qui va à
\item[\vref{Pr 14:4}] a point de b., la grange est
\item[\vref{Es 1:3}] Le b. connaît son possesseur, et l'âne la
\item[\vref{Es 65:25}] lion com. le b. mangeront de la
\item[\vref{Es 66:3}] qui égorge un b. est com. celui
\item[\vref{Ez 1:10}] la face d'un b. à la gauche
\item[\vref{Lu 14:5}] âne ou son b. tombe ds un
\item[\vref{1 Co 9:9}] muselleras pas le b. qui foule le
\end{listverse}

\ConcordanceEntry{Boire}
\vspace{-2mm}
\begin{listverse}
\item[\vref{Ge 9:21}] Et il b. du vin, s'enivra, et se découvrit
\item[\vref{Ge 19:33}] du vin à b. à lr. père
\item[\vref{Ex 15:24}] murmura contre Moïse en disant : Que b.-ns. ?
\item[\vref{Jg 7:5}] qui se mettront à genoux pour b.
\item[\vref{Jg 15:19}] de l'eau. Samson b., l'Esprit lui revint,
\item[\vref{Est 4:16}] sans manger ni b. pendant trois jours,
\item[\vref{Job 15:16}] corrompu l'hom. qui b. l'iniquité com. de
\item[\vref{Ps 78:15}] lr. donna à b. d'abondantes eaux, com.
\item[\vref{Pr 25:21}] soif, donne-lui à b. de l'eau.
\item[\vref{Pr 31:4}] aux rois de b. le vin, ni
\item[\vref{Ec 5:17}] de manger, de b. et de jouir
\item[\vref{Es 65:13}] voici, mes serviteurs b., et vs. aurez
\item[\vref{Da 1:10}] devez manger et b. ; car pourquoi verrait-il
\item[\vref{Am 2:12}] vs. avez fait b. du vin aux
\item[\vref{Ha 2:15}] celui qui fait b. son compagnon en
\item[\vref{Ag 1:6}] rassasiés. Vous avez b., mais vs. n'avez
\item[\vref{Mt 10:42}] aura donné à b. seulement un verre
\item[\vref{Mt 20:22}] vs. demandez. Pouvez-vs. b. la coupe que
\item[\vref{Mt 25:35}] m'avez donné à b. ; j'étais étranger, et
\item[\vref{Mt 26:27}] la lr. donna, en lr. disant : B.-en ts,
\item[\vref{Mt 26:29}] heure je ne b. pas de ce
\item[\vref{Mt 27:34}] lui donnèrent à b. du vinaigre mêlé
\item[\vref{Mc 16:18}] main, et s'ils b. qq breuvage mortel,
\item[\vref{Jn 4:7}] l'eau, Jésus lui dit : Donne-moi à b.
\item[\vref{Jn 7:37}] qu'il vienne à moi, et qu'il b.
\item[\vref{Jn 18:11}] le fourreau. Ne b.-je pas la
\item[\vref{Ac 23:21}] ne manger ni b. jusqu'à ce qu'ils
\item[\vref{Ro 14:17}] viande et le b., mais il est
\item[\vref{1 Co 10:4}] qu'ils ont ts b. le mm breuvage
\item[\vref{1 Co 10:21}] ne pouvez pas b. la coupe du
\item[\vref{1 Co 11:25}] que vs. en b., en mémoire de
\item[\vref{Col 2:16}] manger ou du b., ou au sujet
\item[\vref{1 Pi 4:3}] manger et du b., et aux idolâtries
\item[\vref{Ap 14:10}] il b., lui aussi, du vin de la
\end{listverse}

\ConcordanceEntry{Bois}
\vspace{-2mm}
\begin{listverse}
\item[\vref{Ex 15:25}] montra un certain b. qu'il jeta ds
\item[\vref{Lé 14:51}] Il prendra le b. de cèdre, l'hysope,
\item[\vref{No 15:32}] qui ramassait du b. le jour du
\item[\vref{1 R 18:34}] et sur le b.. Puis il dit :
\item[\vref{Pr 26:20}] s'éteint faute de b., ainsi qnd il
\item[\vref{Pr 26:21}] braise, et le b. pour faire du
\item[\vref{Jé 2:27}] Ils disent au b. : Tu es mon
\item[\vref{Jé 10:3}] On coupe le b. ds la forêt ;
\item[\vref{Ez 37:19}] vais prendre le b. de Joseph qui
\item[\vref{Da 5:23}] de fer, de b. et de pierre,
\item[\vref{Os 4:12}] avis à son b., et son bâton
\item[\vref{Ha 2:19}] qui disent au b. : Réveille-toi ! Et à
\item[\vref{Ag 1:8}] montagne, apportez du b., et bâtissez cette
\item[\vref{Za 5:4}] consumera avec son b. et ses pierres.
\item[\vref{Za 12:6}] feu parmi du b., et com. une
\item[\vref{Lu 23:31}] ces choses au b. vert, que sera-t-il
\item[\vref{Ac 5:30}] fait mourir en le pendant au b.
\item[\vref{Ga 3:13}] Maudit est quiconque est pendu au b.,
\item[\vref{Ja 3:5}] feu, combien de b. allume-t-il ?
\item[\vref{1 Pi 2:24}] corps sur le b., afin qu'étant morts
\item[\vref{Ap 9:20}] pierre, et de b., qui ne peuvent
\item[\vref{Ap 18:12}] tte sorte de b. odoriférant, de tte
\end{listverse}

\ConcordanceEntry{Boisson}
\vspace{-2mm}
\begin{listverse}
\item[\vref{Lé 10:9}] vin, ni de b. forte, ni toi
\item[\vref{No 6:3}] vin et de b. forte, il ne
\item[\vref{No 28:7}] la libation de b. forte à Yahweh.
\item[\vref{1 S 1:15}] ni vin ni b. forte ; mais je
\item[\vref{Ps 102:10}] j'ai mêlé des larmes à ma b.,
\item[\vref{Pr 20:1}] moqueur et la b. forte mutine, et
\item[\vref{Pr 31:4}] ni aux princes de boire la b. forte ;
\item[\vref{Es 5:11}] qui recherchent les b. fortes, qui demeurent
\item[\vref{Es 5:22}] vin et vaillants pour mêler des b. fortes ;
\item[\vref{Es 32:6}] faire tarir la b. de celui qui
\item[\vref{Es 56:12}] ns. enivrerons de b. fortes ! Nous en
\item[\vref{Mi 2:11}] et sur les b. fortes, ce sera
\item[\vref{Lu 1:15}] ni vin, ni b. forte, et il
\end{listverse}

\ConcordanceEntry{Boiter}
\vspace{-2mm}
\begin{listverse}
\item[\vref{Ge 32:31}] passa Peniel. Jacob b. de la hanche.
\end{listverse}

\ConcordanceEntry{Boiteux}
\vspace{-2mm}
\begin{listverse}
\item[\vref{2 S 5:8}] aveugles et ces b., qui sont haïs
\item[\vref{2 S 9:13}] roi. Il était b. des deux pieds.
\item[\vref{Job 29:15}] de l'aveugle et les pieds du b.
\item[\vref{Pr 26:7}] marcher un hom. b., ainsi il en
\item[\vref{Es 35:6}] Alors le b. sautera com. un cerf, et la
\item[\vref{Mi 4:6}] je rassemblerai les b., je recueillerai ceux
\item[\vref{Mi 4:7}] Je ferai des b. un reste, et
\item[\vref{Mal 1:13}] ce qui est b., et malade, ce
\item[\vref{Mt 11:5}] la vue, les b. marchent, les lépreux
\item[\vref{Mt 15:31}] guéris, que les b. marchaient, que les
\item[\vref{Mt 18:8}] que tu entres b. ou manchot ds
\item[\vref{Mt 21:14}] aveugles et des b. s'approchèrent de lui
\item[\vref{Jn 5:3}] des aveugles, des b., des paralytiques, attendant
\item[\vref{Ac 3:2}] avait un hom. b. de naissance, qu'on
\item[\vref{Ac 8:7}] paralytiques et de b. furent guéris.
\item[\vref{Ac 14:8}] impotent des pieds, b. dès sa naissance,
\item[\vref{Hé 12:13}] ce qui est b. ne dévie pas,
\end{listverse}

\ConcordanceEntry{Bon}
\vspace{-2mm}
\begin{listverse}
\item[\vref{Ge 1:10}] et Dieu vit que cela était b.
\item[\vref{Ge 2:18}] Il n'est pas b. que l'hom. soit
\item[\vref{Ge 3:6}] de l'arbre était b. à manger et
\item[\vref{Jg 19:24}] com. il semblera b. à vos yeux.
\item[\vref{1 S 3:18}] qu'il fasse ce qui lui semblera b. !
\item[\vref{2 S 22:26}] celui qui est b. tu es bon,
\item[\vref{1 R 22:8}] prophétise rien de b., mais seulement du
\item[\vref{1 R 22:18}] prophétise rien de b., mais seulement du
\item[\vref{Ps 18:26}] celui qui est b., tu te montres
\item[\vref{Ps 22:9}] puisqu'il prend son b. plaisir en toi !
\item[\vref{Ps 25:8}] Teth.] Yahweh est b. et droit : C'est
\item[\vref{Ps 34:9}] combien Yahweh est b. ! Béni est l'hom.
\item[\vref{Ps 119:68}] Tu es b. et bienfaisant, enseigne-moi tes statuts.
\item[\vref{Pr 24:23}] Il n'est pas b. d'avoir égard à
\item[\vref{Pr 25:27}] il n'est pas b. de manger trop
\item[\vref{Pr 31:13}] elle travaille de b. cœur avec ses
\item[\vref{Ec 2:1}] et prends du b. temps. Et voici,
\item[\vref{Ec 6:12}] ce qui est b. à l'hom. ds
\item[\vref{Es 55:2}] ce qui est b., et votre âme
\item[\vref{Es 62:4}] on t'appellera mon b. plaisir en elle ;
\item[\vref{Jé 33:11}] car Yahweh est b., parce que sa
\item[\vref{So 3:7}] sont levés de b. matin, ils ont
\item[\vref{Za 11:12}] S'il vs. semble b., donnez-moi mon salaire ;
\item[\vref{Mt 7:17}] Ainsi tt b. arbre porte de
\item[\vref{Mt 12:35}] bonnes choses du b. trésor de son
\item[\vref{Mt 17:4}] Seign., il est b. que ns. soyons
\item[\vref{Mt 19:16}] Maître, qui est b., quel bien ferai-je
\item[\vref{Mt 19:17}] répondit : Pourquoi m'appelles-tu b. ? Dieu est le
\item[\vref{Jn 1:46}] qq chose de b. de Nazareth ? Philippe
\item[\vref{Jn 10:11}] Je suis le b. berger ; le bon
\item[\vref{Jn 16:33}] monde ; mais ayez b. courage, j'ai vaincu
\item[\vref{Ac 2:41}] qui reçurent de b. cœur sa parole
\item[\vref{Ro 7:18}] Ce qui est b., je le sais,
\item[\vref{Ro 12:2}] ce qui est b., agréable et parfait.
\item[\vref{Ep 1:5}] Jésus-Christ, selon le b. plaisir de sa
\item[\vref{1 Th 5:21}] ttes choses ; retenez ce qui est b.
\item[\vref{1 Ti 2:3}] Car cela est b. et agréable dvt
\item[\vref{1 Ti 3:7}] qu'il reçoive un b. témoignage de ceux
\item[\vref{1 Ti 4:4}] a créé est b., et rien ne
\item[\vref{1 Ti 6:12}] Combats le b. combat de la
\item[\vref{Hé 11:2}] ont obtenu un b. témoignage.
\item[\vref{1 Pi 2:3}] avez goûté combien le Seign. est b.
\end{listverse}

\ConcordanceEntry{Bonheur}
\vspace{-2mm}
\begin{listverse}
\item[\vref{Ge 30:11}] Léa dit : Le b. est arrivé, c'est
\item[\vref{De 30:9}] nouveau de ton b., com. il s'est
\item[\vref{Est 8:16}] les Juifs du b. et de la
\item[\vref{Job 7:7}] mes yeux ne reverront plus le b.
\item[\vref{Job 22:21}] en repos, tu atteindras ainsi le b.
\item[\vref{Job 36:11}] jours ds le b., leurs années ds
\item[\vref{Ps 4:7}] fera voir le b. ? Lève sur ns.
\item[\vref{Ps 25:13}] demeurera ds le b., et sa postérité
\item[\vref{Ps 34:13}] aime la prolonger pour jouir du b. ?
\item[\vref{Ps 81:16}] flatteraient, et le b. de mon peuple
\item[\vref{Ps 122:9}] je fais une requête pour ton b.
\item[\vref{Pr 18:22}] fem. trouve le b. et il obtient
\item[\vref{Pr 19:8}] prend garde à l'intelligence trouvera le b.
\item[\vref{Pr 28:10}] ceux qui sont intègres héritent le b.
\item[\vref{Ec 8:12}] y aura du b. pour ceux qui
\item[\vref{Ec 8:13}] Mais le b. n'est pas pour le méchant, et
\item[\vref{Ga 4:15}] dc est votre b. ? Car je vs.
\end{listverse}

\ConcordanceEntry{Bonté}
\vspace{-2mm}
\begin{listverse}
\item[\vref{Ge 47:29}] envers moi de b. et de fidélité :
\item[\vref{Ex 33:19}] passer tte ma b. dvt ta face,
\item[\vref{Ex 34:6}] colère, abondant en b. et en fidélité ;
\item[\vref{No 14:18}] et riche en b., il ôte l'iniquité
\item[\vref{Né 9:17}] et abondant en b., et tu ne
\item[\vref{Ps 17:7}] Signale ta b., toi qui sauves
\item[\vref{Ps 27:13}] de voir la b. de Yahweh sur
\item[\vref{Ps 31:20}] Ô que ta b. est grande ! Toi qui la réserves
\item[\vref{Ps 36:6}] Yahweh ! ta b. atteint jusqu'aux cieux,
\item[\vref{Ps 51:3}] moi ds ta b. ; selon ta grande
\item[\vref{Ps 63:4}] Car ta b. vaut mieux que la vie, mes
\item[\vref{Ps 66:20}] n'a point éloigné de moi sa b.
\item[\vref{Ps 77:9}] Sa b. est-elle disparue pour toujours ? Sa parole
\item[\vref{Ps 85:11}] La b. et la vérité se rencontrent, la
\item[\vref{Ps 89:3}] je dis : Ta b. a des fondements
\item[\vref{Ps 90:14}] matin de ta b., afin que ns.
\item[\vref{Ps 92:3}] chaque matin ta b., et ta fidélité
\item[\vref{Ps 94:18}] pied chancelle ! ta b. me soutient, ô
\item[\vref{Ps 98:3}] souvenu de sa b. et de sa
\item[\vref{Ps 100:5}] est bon ; sa b. demeure à toujours,
\item[\vref{Ps 136:7}] luminaires, car sa b. demeure à toujours !
\item[\vref{Pr 3:3}] Que la b. et la vérité ne t'abandonnent pas :
\item[\vref{Pr 14:22}] pas ? Mais la b. et la vérité
\item[\vref{Pr 20:6}] gens prêchent lr. b. ; mais qui trouvera
\item[\vref{Es 54:8}] toi avec une b. éternelle, dit Yahweh,
\item[\vref{Es 63:7}] ferai mention des b. de Yahweh, qui
\item[\vref{Jé 31:3}] j'ai prolongé ma b. envers toi.
\item[\vref{Ro 2:4}] richesses de sa b., et de sa
\item[\vref{Ro 11:22}] Considère dc la b. et la sévérité
\item[\vref{Ep 5:9}] consiste en tte b., justice et vérité,
\item[\vref{Col 3:12}] de miséricorde, de b., d'humilité, de douceur,
\item[\vref{2 Th 1:11}] plaisir de sa b., et l'œuvre de
\item[\vref{Tit 3:4}] Mais, qnd la b. de Dieu notre
\end{listverse}

\ConcordanceEntry{Bouc}
\vspace{-2mm}
\begin{listverse}
\item[\vref{Lé 9:3}] disant : Prenez un b. pour l'offrande d'expiation,
\item[\vref{Lé 16:8}] sur les deux b., un sort pour
\item[\vref{Lé 16:22}] Et le b. portera sur lui ttes leurs iniquités
\item[\vref{Lé 16:26}] aura conduit le b. pour être Azazel
\item[\vref{Pr 30:31}] bien troussés ; le b., et le roi
\item[\vref{Ez 43:22}] en expiation un b., sans défaut, et
\item[\vref{Da 8:5}] attentivement, voici, un b. d'entre les chèvres
\item[\vref{Da 8:21}] et le b. velu, c'est le roi de Javan ;
\end{listverse}

\ConcordanceEntry{Bouche}
\vspace{-2mm}
\begin{listverse}
\item[\vref{Ge 4:11}] a ouvert sa b. pour recevoir de
\item[\vref{Ex 4:10}] car j'ai la b. et la langue
\item[\vref{Ex 4:16}] il sera ta b., et tu seras
\item[\vref{No 12:8}] parle avec lui b. à bouche, et
\item[\vref{No 16:30}] terre ouvre sa b. pour les engloutir
\item[\vref{1 R 22:22}] mensonge ds la b. de ts ses
\item[\vref{2 Ch 35:22}] venaient de la b. de Dieu. Il
\item[\vref{Ps 8:3}] Par la b. des petits enfants et de ceux
\item[\vref{Ps 39:2}] frein à ma b., tant que le
\item[\vref{Ps 40:4}] mis ds ma b. un cantique nouveau,
\item[\vref{Ps 62:5}] bénissent de lr. b., mais au-dedans ils
\item[\vref{Ps 78:2}] J'ouvrirai ma b. en une parabole,
\item[\vref{Ps 141:3}] garde à ma b., garde l'entrée de
\item[\vref{Pr 13:3}] qui garde sa b., garde son âme ;
\item[\vref{Pr 30:20}] et s'essuie la b., puis elle dit :
\item[\vref{Pr 31:8}] Ouvre ta b. en faveur du muet, pour le
\item[\vref{Ec 5:5}] pas à ta b. de faire pécher
\item[\vref{Es 34:16}] car c'est ma b. qui l'a ordonné,
\item[\vref{Es 51:16}] paroles ds ta b., et je te
\item[\vref{Es 52:15}] rois fermeront la b. sur lui ; car
\item[\vref{Es 53:7}] point ouvert sa b., semblable à un
\item[\vref{Es 55:11}] sort de ma b., elle ne retourne
\item[\vref{Da 7:8}] d'hom., et une b. qui proférait de
\item[\vref{Os 2:19}] j'ôterai de sa b. les noms des
\item[\vref{Mal 2:7}] c'est de sa b. qu'on demande la
\item[\vref{Mt 12:34}] de l'abondance du cœur que la b. parle.
\item[\vref{Mt 15:8}] moi de sa b. et m'honore des
\item[\vref{Mt 15:11}] entre ds la b. qui souille l'hom. ;
\item[\vref{Mt 15:18}] sortent de la b. partent du cœur,
\item[\vref{Mt 17:27}] viendra ; ouvre-lui la b., tu trouveras un
\item[\vref{Mt 18:16}] que par la b. de deux ou
\item[\vref{Mt 21:16}] louanges de la b. des enfants, et
\item[\vref{Mt 22:12}] de noces ? Cet hom. eut la b. fermée.
\item[\vref{Lu 21:15}] vs. donnerai une b. et une sagesse
\item[\vref{Ro 3:19}] afin que tte b. soit fermée, et
\item[\vref{Ep 4:29}] sorte de votre b., mais seulement celui
\item[\vref{Col 3:8}] déshonnêtes qui pourraient sortir de votre b.
\item[\vref{2 Th 2:8}] souffle de sa b. et qu'il anéantira
\item[\vref{Tit 1:11}] faut fermer la b., et qui renversent
\item[\vref{Ja 3:10}] De la mm b. sortent la bénédiction
\item[\vref{1 Pi 2:15}] vs. fermiez la b. à l'ignorance des
\item[\vref{Ap 3:16}] bouillant, je te vomirai de ma b.
\item[\vref{Ap 13:6}] Elle ouvrit sa b. pour blasphémer contre
\item[\vref{Ap 14:5}] Et ds lr. b. il ne s'est
\item[\vref{Ap 19:15}] De sa b. sortait une épée tranchante, pour frapper
\end{listverse}

\ConcordanceEntry{Bouclier}
\vspace{-2mm}
\begin{listverse}
\item[\vref{Ge 15:1}] je suis ton b., ta grande et
\item[\vref{De 33:29}] par Yahweh, le b. de ton secours
\item[\vref{2 S 22:3}] un abri, mon b. et la force
\item[\vref{2 S 22:31}] il est le b. de ts ceux
\item[\vref{2 S 22:36}] me donnes le b. de ton salut,
\item[\vref{Ps 3:4}] tu es un b. autour de moi !
\item[\vref{Ps 5:13}] l'environnes de ta bienveillance com. d'un b.
\item[\vref{Ps 7:11}] Mon b. est en Dieu, qui délivre ceux
\item[\vref{Ps 18:36}] me donnes le b. de ton salut,
\item[\vref{Ps 28:7}] force et mon b. ; mon cœur se
\item[\vref{Ps 33:20}] il est notre aide et notre b.
\item[\vref{Ps 35:2}] et le grand b., et lève-toi pour
\item[\vref{Ps 47:10}] d'Abraham, car les b. de la terre
\item[\vref{Ps 89:19}] Car notre b. est Yahweh, et
\item[\vref{Ps 91:4}] fidélité est un b. et une cuirasse.
\item[\vref{Pr 2:7}] il est le b. de ceux qui
\item[\vref{Pr 30:5}] il est un b. pour ceux qui
\item[\vref{Jé 46:3}] Préparez le b. et l'écu et
\item[\vref{Na 2:4}] Le b. de ses hommes forts est rouge ;
\item[\vref{Ep 6:16}] tt, prenez le b. de la foi,
\end{listverse}

\ConcordanceEntry{Boue}
\vspace{-2mm}
\begin{listverse}
\item[\vref{Job 30:19}] jeté ds la b., et je ressemble
\item[\vref{Ps 40:3}] fond de la b. ; il a mis
\item[\vref{Es 41:25}] foule com. le potier foule la b.
\item[\vref{Es 57:20}] eaux rejettent la b. et le bourbier.
\item[\vref{Jé 38:6}] mais de la b. ; et ainsi Jérémie
\item[\vref{Jé 38:22}] enfoncés ds la b., ils se sont
\item[\vref{Ha 2:6}] lui de la b. épaisse ?
\item[\vref{Za 10:5}] hommes foulant la b. des rues ds
\item[\vref{Jn 9:6}] fit de la b. avec sa salive,
\end{listverse}

\ConcordanceEntry{Bourse}
\vspace{-2mm}
\begin{listverse}
\item[\vref{Pr 1:14}] n'y aura qu'une b. pour ns. ts.
\item[\vref{Es 46:6}] l'or de la b., et pèsent l'argent
\item[\vref{Lu 10:4}] Ne portez ni b., ni sac, ni
\item[\vref{Lu 10:35}] tira de sa b. deux deniers, et
\item[\vref{Lu 22:35}] ai envoyés sans b., sans sac, et
\item[\vref{Jn 12:6}] que, tenant la b., il prenait ce
\end{listverse}

\ConcordanceEntry{Branche}
\vspace{-2mm}
\begin{listverse}
\item[\vref{Ge 30:38}] il mit les b. qu'il avait pelées
\item[\vref{Ge 49:22}] d'une fontaine ; ses b. se sont étendues
\item[\vref{Es 44:19}] abomination ? Adorerais-je une b. de bois ?
\item[\vref{Jé 1:11}] Je vois une b. d'amandier.
\item[\vref{Da 4:12}] habitaient ds ses b., et tt être
\item[\vref{Za 4:12}] signifient ces deux b. d'olivier qui sont
\item[\vref{Mt 13:32}] et font leurs nids ds ses b.
\item[\vref{Mt 24:32}] que ses jeunes b. deviennent tendres et
\item[\vref{Mc 11:8}] et d'autres des b. qu'ils coupèrent ds
\item[\vref{Jn 19:29}] au bout d'une b. d'hysope, et la
\item[\vref{Ro 11:16}] est sainte, les b. le sont aussi.
\item[\vref{Ro 11:21}] pas épargné les b. naturelles, prends garde
\end{listverse}

\ConcordanceEntry{Bras}
\vspace{-2mm}
\begin{listverse}
\item[\vref{Ex 15:16}] grandeur de ton b., jusqu'à ce que
\item[\vref{De 11:2}] grandeur, sa main puissante, et son b. étendu,
\item[\vref{2 S 22:35}] combat, et mes b. tendent l'arc d'airain.
\item[\vref{Job 40:4}] As-tu un b. com. celui de Dieu ; tonnes-tu de
\item[\vref{Ps 77:16}] délivré par ton b. ton peuple, les
\item[\vref{Ps 89:14}] Ton b. est puissant, ta main est forte,
\item[\vref{Ps 98:1}] droite et le b. de sa sainteté
\item[\vref{Pr 31:17}] reins de force, et affermit ses b.
\item[\vref{Ca 8:6}] sceau sur ton b. ; car l'amour est
\item[\vref{Es 52:10}] Yahweh manifeste le b. de sa sainteté
\item[\vref{Es 53:1}] à qui le b. de Yahweh a-t-il
\item[\vref{Es 59:16}] c'est pourquoi son b. lui vient en
\item[\vref{Da 2:32}] poitrine et ses b. étaient d'argent ; son
\item[\vref{Za 11:17}] fonde sur son b. et sur son
\item[\vref{Mc 9:36}] pris entre ses b., il lr. dit :
\item[\vref{Lu 1:51}] opéré par son b. ; il a dissipé
\item[\vref{Lu 2:28}] prit ds ses b., bénit Dieu, et
\item[\vref{Jn 12:38}] été révélé le b. du Seign. ?
\end{listverse}

\ConcordanceEntry{Brasier}
\vspace{-2mm}
\begin{listverse}
\item[\vref{No 4:14}] pour l'autel, les b., les fourchettes, les
\item[\vref{Jé 36:23}] au feu du b., jusqu'à ce que
\end{listverse}

\ConcordanceEntry{Brebis}
\vspace{-2mm}
\begin{listverse}
\item[\vref{Lé 5:6}] bétail, soit une b., soit une chèvre,
\item[\vref{No 18:17}] premier-né de la b., ni le premier-né
\item[\vref{1 S 16:11}] fait paître les b.. Alors Samuel dit
\item[\vref{1 S 17:20}] et laissa les b. aux soins d'un
\item[\vref{1 S 17:35}] et j'arrachais la b. de sa gueule.
\item[\vref{2 S 12:2}] riche avait des b. et des bœufs
\item[\vref{Ps 44:23}] regardés com. des b. destinées à la
\item[\vref{Ps 119:176}] errant com. une b. perdue ; cherche ton
\item[\vref{Es 53:6}] errants com. des b., ns. ns. sommes
\item[\vref{Es 53:7}] boucherie, à une b. muette dvt celui
\item[\vref{Jé 50:6}] un troupeau de b. perdues ; leurs pasteurs
\item[\vref{Jé 50:17}] est com. une b. égarée que les
\item[\vref{Ez 34:4}] point fortifié les b. languissantes, vs. n'avez
\item[\vref{Ez 34:15}] je paîtrai mes b. et les ferai
\item[\vref{Ez 34:20}] à part la b. grasse et la
\item[\vref{Jon 3:7}] bœufs et les b., ne goûtent de
\item[\vref{Za 11:4}] Dieu : Pais les b. exposées au carnage !
\item[\vref{Za 11:17}] qui abandonne les b. ! Que l'épée fonde
\item[\vref{Za 13:7}] Berger, et les b. seront dispersées ; et
\item[\vref{Mt 7:15}] en habits de b., mais au-dedans ce
\item[\vref{Mt 9:36}] errantes, com. des b. qui n'ont pas
\item[\vref{Mt 10:6}] plutôt vers les b. perdues de la
\item[\vref{Mt 10:16}] envoie com. des b. au milieu des
\item[\vref{Mt 12:11}] s'il n'a qu'une b., et qu'elle vienne
\item[\vref{Mt 15:24}] été envoyé qu'aux b. perdues de la
\item[\vref{Mt 18:12}] hom. a cent b., et que l'une
\item[\vref{Mt 25:32}] berger sépare les b. d'avec les boucs.
\item[\vref{Lu 15:6}] j'ai trouvé ma b. qui était perdue.
\item[\vref{Jn 2:14}] de bœufs, de b., et de pigeons,
\item[\vref{Jn 10:3}] lui ouvre, les b. entendent sa voix,
\item[\vref{Jn 10:14}] Je connais mes b., et mes brebis
\item[\vref{Jn 10:16}] J'ai encore d'autres b. qui ne sont
\item[\vref{Jn 10:27}] Mes b. entendent ma voix ; je les connais,
\item[\vref{Jn 21:16}] t'aime. Il lui dit : Pais mes b.
\item[\vref{Ro 8:36}] estimés com. des b. de la boucherie.
\item[\vref{Hé 13:20}] grand Pasteur des b., par le sang
\item[\vref{1 Pi 2:25}] étiez com. des b. errantes, mais mntnt
\end{listverse}

\ConcordanceEntry{Brèche}
\vspace{-2mm}
\begin{listverse}
\item[\vref{Ge 38:29}] sage-fem. dit : Quelle b. tu as faite !
\item[\vref{Jg 21:15}] avait fait une b. ds les tribus
\item[\vref{2 S 6:8}] avait fait une b. en la personne
\item[\vref{Né 6:1}] n'y restait aucune b. (bien que jusqu'à
\item[\vref{Ps 106:23}] tint à la b. dvt lui, pour
\item[\vref{Pr 15:4}] est com. une b. ds l'esprit.
\item[\vref{Pr 25:28}] y a une b., et qui est
\item[\vref{Es 7:6}] ville, battons-la en b., et établissons pour
\item[\vref{Es 58:12}] le réparateur des b. et le restaurateur
\item[\vref{Es 59:16}] tienne à la b. ; c'est pourquoi son
\item[\vref{Ez 22:30}] tiendrait à la b. dvt moi pour
\item[\vref{Am 9:11}] j'en réparerai les b., j'en redresserai les
\item[\vref{Mi 2:13}] qui fera la b. montera dvt eux,
\end{listverse}

\ConcordanceEntry{Brigand}
\vspace{-2mm}
\begin{listverse}
\item[\vref{Mt 26:55}] com. après un b., pour me prendre.
\item[\vref{Mt 27:38}] furent crucifiés deux b., l'un à sa
\item[\vref{Lu 10:30}] les mains des b. qui le dépouillèrent,
\item[\vref{Jn 10:1}] ailleurs, est un voleur et un b.
\item[\vref{Jn 10:8}] moi sont des b. et des voleurs ;
\item[\vref{Jn 18:40}] mais Barabbas. Or Barabbas était un b.
\item[\vref{Ac 21:38}] emmené ds le désert quatre mille b. ?
\item[\vref{2 Co 11:26}] la part des b., en péril de
\end{listverse}

\ConcordanceEntry{Briller}
\vspace{-2mm}
\begin{listverse}
\item[\vref{Job 37:15}] comment il fait b. la lumière de
\item[\vref{Job 41:23}] Il fait b. son sentier après lui, et on
\item[\vref{Ps 18:29}] toi qui fais b. ma lumière ; Yahweh,
\item[\vref{Ps 80:2}] les chérubins, fais b. ta splendeur !
\item[\vref{Ps 94:1}] des vengeances, fais b. ta splendeur !
\item[\vref{Ec 8:1}] de l'hom. fait b. son visage, et
\item[\vref{Es 13:10}] ne feront plus b. lr. lumière ; le
\item[\vref{Ez 21:20}] est faite pour b. et réservée pour
\item[\vref{Da 12:3}] auront été intelligents, b. com. la splendeur
\item[\vref{Lu 17:24}] Car, com. l'éclair b. et resplendit d'une
\item[\vref{Ph 2:15}] parmi laquelle vs. b. com. des flambeaux
\item[\vref{2 Pi 1:19}] une lampe qui b. ds un lieu
\item[\vref{Ap 1:16}] au soleil lorsqu'il b. ds sa force.
\item[\vref{Ap 18:23}] la lampe ne b. plus chez toi,
\end{listverse}

\ConcordanceEntry{Brique}
\vspace{-2mm}
\begin{listverse}
\item[\vref{Ge 11:3}] Allons ! Faisons des b., et cuisons-les très
\item[\vref{Ex 1:14}] du mortier, des b., et tte sorte
\item[\vref{Es 65:3}] des parfums sur les autels de b.,
\item[\vref{Jé 43:9}] le four à b. qui est à
\item[\vref{Ez 4:1}] l'hom., prends une b. et place-la dvt
\item[\vref{Na 3:14}] l'argile ! Et fortifie le four à b. !
\end{listverse}

\ConcordanceEntry{Briser}
\vspace{-2mm}
\begin{listverse}
\item[\vref{Ge 19:9}] ils s'approchèrent pour b. la porte.
\item[\vref{Lé 26:19}] Je b. l'orgueil de votre force et je
\item[\vref{Jg 7:19}] du shofar et b. les cruches qu'ils
\item[\vref{Job 34:24}] Il b. les hommes puissants par des voies
\item[\vref{Ps 2:9}] Tu les b. avec un sceptre de fer, et
\item[\vref{Ps 110:5}] ta droite, il b. les rois au
\item[\vref{Ps 147:3}] ont le cœur b., et il bande
\item[\vref{Pr 29:1}] cou, sera subitement b. sans qu'il y
\item[\vref{Es 38:13}] un lion, qui b. ainsi ts mes
\item[\vref{Es 42:3}] Il ne b. point le roseau cassé, et il
\item[\vref{Es 53:5}] pour nos péchés, b. pour nos iniquités,
\item[\vref{Es 53:10}] Yahweh de le b. ; il l'a mis
\item[\vref{Jé 49:35}] Voici, je vais b. l'arc d'Elam, qui
\item[\vref{Jon 1:4}] sorte que le navire semblait se b.
\item[\vref{Mc 5:4}] les chaînes et b. les fers, et
\item[\vref{Jn 19:36}] Aucun de ses os ne sera b.
\item[\vref{Ac 2:24}] l'a ressuscité, ayant b. les liens de
\item[\vref{Ro 16:20}] Dieu de paix b. bientôt Satan sous
\item[\vref{Ap 2:27}] et elles seront b. com. les vases
\end{listverse}

\ConcordanceEntry{Broncher}
\vspace{-2mm}
\begin{listverse}
\item[\vref{Pr 3:23}] ton chemin, et ton pied ne b. pas.
\item[\vref{Pr 25:26}] Le juste qui b. dvt le méchant
\item[\vref{Jé 20:10}] observent si je b., et disent : Peut-être
\item[\vref{Jé 31:9}] où ils ne b. pas ; car j'ai
\item[\vref{Mal 2:8}] vs. avez fait b. plusieurs ds la
\item[\vref{Jn 11:9}] jour, il ne b. pas ; car il
\item[\vref{Ro 11:11}] je demande : Ont-il b. afin de tomber ?
\item[\vref{Ro 14:4}] ferme ou s'il b., c'est pour son
\item[\vref{2 Pi 1:10}] car, en faisant cela, vs. ne b. jamais.
\end{listverse}

\ConcordanceEntry{Bruit}
\vspace{-2mm}
\begin{listverse}
\item[\vref{Ex 23:1}] point de faux b., et tu ne
\item[\vref{Ex 32:17}] faisait un grand b., dit à Moïse :
\item[\vref{1 R 18:41}] se fait un b. qui annonce la
\item[\vref{1 Ch 14:15}] des mûriers un b. com. des gens
\item[\vref{Job 26:14}] pourra comprendre le b. éclatant de sa
\item[\vref{Ps 93:3}] fleuves augmentent lr. b., les fleuves élèvent
\item[\vref{Ec 12:6}] qnd s'abaisse le b. de la meule,
\item[\vref{Es 65:19}] entendra plus le b. des pleurs et
\item[\vref{Ez 1:24}] Puis j'entendis le b. que faisaient leurs
\item[\vref{So 1:15}] un jour de b. éclatant et effrayant,
\item[\vref{Mt 9:26}] Et le b. s'en répandit ds tte cette contrée.
\item[\vref{Mt 28:15}] données. Et ce b. s'est répandu parmi
\item[\vref{Jn 3:8}] en entends le b. ; mais tu ne
\item[\vref{Jn 21:23}] Là-dessus, le b. courut parmi les
\item[\vref{Ac 2:2}] se fit un b. du ciel com.
\item[\vref{Ap 1:15}] était com. le b. des grandes eaux.
\item[\vref{Ap 18:22}] chez toi le b. de la meule,
\end{listverse}

\ConcordanceEntry{Brûler}
\vspace{-2mm}
\begin{listverse}
\item[\vref{Lé 6:6}] Le feu b. continuellement sur l'autel, on ne le
\item[\vref{Lé 6:23}] ne sera mangée, mais elle sera b. au feu.
\item[\vref{De 12:31}] et mm elles b. au feu leurs
\item[\vref{Jé 7:31}] de Ben-Hinnom, pour b. au feu leurs
\item[\vref{Jé 17:4}] colère, et il b. à toujours.
\item[\vref{Da 3:27}] lr. tête n'était b., et leurs caleçons
\item[\vref{Mi 1:7}] de prostitution seront b. au feu, et
\item[\vref{Jn 15:6}] le met au feu, et il b.
\item[\vref{Ac 19:19}] livres et les b. dvt ts. On
\item[\vref{1 Co 3:15}] l'œuvre de quelqu'un b., il en fera
\item[\vref{1 Co 7:9}] vaut mieux se marier que de b.
\item[\vref{1 Co 13:3}] corps pour être b., si je n'ai
\item[\vref{2 Pi 3:10}] qu'elle renferme sera b. entièrement.
\item[\vref{Ap 16:9}] les hommes furent b. par de grandes
\item[\vref{Ap 17:16}] et mangeront sa chair, et la b. au feu.
\end{listverse}

\ConcordanceEntry{Buffle}
\vspace{-2mm}
\begin{listverse}
\item[\vref{No 24:8}] la vigueur du b. ; il consumera les
\item[\vref{De 33:17}] les cornes du b. ; il poussera ts
\item[\vref{Job 39:12}] Le b. voudra-t-il te servir, ou demeurera-t-il à
\item[\vref{Ps 22:22}] du lion, délivre-moi des cornes du b. !
\item[\vref{Ps 92:11}] com. celle d'un b. ; je serai oint
\end{listverse}

\ConcordanceEntry{Buisson}
\vspace{-2mm}
\begin{listverse}
\item[\vref{Ge 22:13}] retenu à un b. par ses cornes ;
\item[\vref{Ex 3:2}] du milieu d'un b.. Il regarda, et
\item[\vref{Mc 12:26}] parla ds le b., en disant : Je
\item[\vref{Ac 7:30}] de Sinaï, ds la flamme d'un b. en feu.
\end{listverse}

\ConcordanceEntry{But}
\vspace{-2mm}
\begin{listverse}
\item[\vref{Ac 20:30}] corrompues, ds le b. d'attirer les disciples
\item[\vref{Ro 9:17}] suscité ds le b. de démontrer en
\item[\vref{Ro 9:28}] Seign. amènera au b. et abrégera l'affaire
\item[\vref{Ph 3:12}] déjà atteint le b., ou que je
\item[\vref{Ph 3:14}] cours vers le b., pour remporter le
\item[\vref{1 Ti 1:5}] Or le b. du commandement c'est la charité qui
\item[\vref{1 Pi 1:9}] remportant le b. de votre foi,
\end{listverse}

\ConcordanceEntry{Butin}
\vspace{-2mm}
\begin{listverse}
\item[\vref{Ge 49:27}] sur le soir il partagera le b.
\item[\vref{Jos 11:14}] eux tt le b. de ces villes
\item[\vref{1 S 15:19}] jeté sur le b., et as-tu fait
\item[\vref{1 S 30:26}] une partie du b. aux anciens de
\item[\vref{1 Ch 20:2}] un très grand b. de la ville.
\item[\vref{2 Ch 28:15}] captifs, utilisèrent le b. pour revêtir ts
\item[\vref{Ps 68:13}] à la maison a partagé le b.
\item[\vref{Ps 119:162}] celui qui aurait trouvé un grand b.
\item[\vref{Es 10:2}] veuves pour lr. b., et de piller
\item[\vref{Es 49:24}] Le b. sera-t-il ôté à l'hom. puissant ? Et
\item[\vref{Es 53:12}] il partagera le b. avec les puissants,
\item[\vref{Jé 39:18}] vie sera ton b., parce que tu
\item[\vref{Da 11:24}] il distribuera le b., le pillage et
\item[\vref{So 3:8}] lèverai pour le b. ; car j'ai résolu
\item[\vref{Hé 7:4}] le patriarche, donna la dîme du b.
\end{listverse}

\ConcordanceEntry{Buveur}
\vspace{-2mm}
\begin{listverse}
\item[\vref{Ps 69:13}] moi, et les b. de boissons fortes
\item[\vref{Joë 1:5}] Et vs. ts, b. de vin, hurlez
\item[\vref{Mt 11:19}] mangeur et un b., un ami des
\end{listverse}

\ConcordanceEntry{Cacher}
\vspace{-2mm}
\begin{listverse}
\item[\vref{Ge 3:8}] sa fem. se c. loin de la
\item[\vref{Ge 18:17}] Et Yahweh dit : C.-je à Abraham
\item[\vref{Ex 2:3}] pouvant le tenir c. plus longtemps, elle
\item[\vref{De 29:29}] Les choses c. sont à Yahweh,
\item[\vref{Jos 2:6}] et les avait c. sous des tiges
\item[\vref{1 R 18:4}] prophètes et les c., cinquante ds une
\item[\vref{2 R 4:27}] Yahweh me l'a c., et ne me
\item[\vref{Job 28:11}] il fait sortir ce qui est c.
\item[\vref{Ps 10:1}] éloigné ? Pourquoi te c.-tu au temps
\item[\vref{Ps 17:8}] prunelle de l'œil, c.-moi à l'ombre
\item[\vref{Ps 19:13}] par erreur ? Purifie-moi de mes fautes c.
\item[\vref{Ps 27:5}] Car il me c. ds son tabernacle
\item[\vref{Ps 32:5}] je n'ai point c. mon iniquité ; j'ai
\item[\vref{Ps 89:47}] ô Yahweh ? Te c.-tu à jamais ?
\item[\vref{Pr 22:3}] mal et se c., mais les stupides
\item[\vref{Pr 25:2}] Dieu est de c. les choses, et
\item[\vref{Pr 28:13}] Celui qui c. ses transgressions ne
\item[\vref{Es 29:15}] à ceux qui c. profondément leurs desseins,
\item[\vref{Es 45:15}] Dieu qui te c., le Dieu d'Israël,
\item[\vref{Es 50:6}] je n'ai pas c. mon visage aux
\item[\vref{Jé 33:3}] grandes, des choses c., que tu ne
\item[\vref{Ez 28:3}] que Daniel, aucun secret ne t'est c.
\item[\vref{Mt 10:26}] a rien de c. qui ne doive
\item[\vref{Mt 11:25}] que tu as c. ces choses aux
\item[\vref{Mt 25:25}] suis allé le c. ds la terre.
\item[\vref{Lu 19:42}] mntnt elles sont c. dvt tes yeux.
\item[\vref{Ac 20:27}] conseil de Dieu, sans en rien c.
\item[\vref{1 Co 4:5}] lumière les choses c. ds les ténèbres
\item[\vref{Col 3:3}] votre vie est c. avec Christ en
\item[\vref{Hé 4:13}] créature qui soit c. dvt lui, mais
\item[\vref{Ap 6:15}] hom. libre se c. ds les cavernes
\end{listverse}

\ConcordanceEntry{Cachette}
\vspace{-2mm}
\begin{listverse}
\item[\vref{Ps 64:5}] l'innocent ds sa c. ; ils tirent soudainement
\item[\vref{Pr 21:14}] présent fait en c. calme une fureur
\item[\vref{Ac 26:26}] n'est pas en c. qu'elles se sont
\end{listverse}

\ConcordanceEntry{Cadavre}
\vspace{-2mm}
\begin{listverse}
\item[\vref{Lé 5:2}] souillée, soit le c. d'un animal impur,
\item[\vref{No 14:29}] Vos c. tomberont ds ce désert, et ts
\item[\vref{De 14:21}] ne mangerez aucun c. ; tu le donneras
\item[\vref{2 R 9:37}] et le c. de Jézabel sera com. du fumier
\item[\vref{Job 39:33}] où sont des c., il s'y trouve
\item[\vref{Ps 110:6}] remplira tt de c. ; il brisera le
\item[\vref{Am 8:3}] aura beaucoup de c. que l'on jettera
\item[\vref{Mt 24:28}] où est le c., là s'assembleront les
\item[\vref{Hé 3:17}] et dont les c. tombèrent ds le
\item[\vref{Ap 11:8}] Et leurs c. seront étendus sur les places de
\end{listverse}

\ConcordanceEntry{Caille}
\vspace{-2mm}
\begin{listverse}
\item[\vref{Ex 16:13}] il monta des c. qui couvrirent le
\item[\vref{No 11:31}] qui amena des c. et les répandit
\item[\vref{No 11:32}] et amassa des c. ; celui qui en
\item[\vref{Ps 105:40}] fit venir des c. ; et il les
\end{listverse}

\ConcordanceEntry{Caillou}
\vspace{-2mm}
\begin{listverse}
\item[\vref{Es 50:7}] semblable à un c., car je sais
\item[\vref{Ap 2:17}] lui donnerai un c. blanc, et sur
\end{listverse}

\ConcordanceEntry{Caïn}
\vspace{-2mm}
\begin{listverse}
\item[\vref{Ge 4:1}] conçut, et enfanta C. ; et elle dit :
\item[\vref{Ge 4:2}] fut berger, et C. laboureur.
\item[\vref{Ge 4:3}] qq temps, que C. offrit à Yahweh
\item[\vref{Ge 4:5}] point d'égard à C. ni à son
\item[\vref{Ge 4:8}] Or C. parla avec Abel son frère, et
\item[\vref{Ge 4:15}] pourquoi quiconque tuera C. sera puni sept
\item[\vref{Hé 11:4}] plus excellent que C. ; et par elle
\item[\vref{1 Jn 3:12}] soyons pas com. C., qui était de
\item[\vref{Jud 1:11}] la voie de C., et ont couru
\end{listverse}

\ConcordanceEntry{Caïphe}
\vspace{-2mm}
\begin{listverse}
\item[\vref{Jn 11:49}] l'un d'eux, appelé C., qui était le
\item[\vref{Jn 18:13}] le beau-père de C., qui était le
\item[\vref{Ac 4:6}] Anne, le grand-prêtre, C., Jean, Alexandre, et
\end{listverse}

\ConcordanceEntry{Calamité}
\vspace{-2mm}
\begin{listverse}
\item[\vref{Job 6:2}] mette ensemble ds une balance ma c. !
\item[\vref{Job 6:21}] vs. voyez ma c. étonnante, et vs.
\item[\vref{Ps 55:12}] Les c. sont au milieu d'elle, et la
\item[\vref{Ps 57:2}] ce que les c. soient passées.
\item[\vref{Ps 141:5}] ma prière sera pour eux lr. c.
\item[\vref{Pr 1:27}] et que votre c. viendra com. un
\item[\vref{Pr 6:15}] C'est pourquoi sa c. viendra subitement, il
\item[\vref{Pr 24:22}] car lr. c. s'élèvera tt d'un coup ; et qui
\item[\vref{Pr 27:10}] temps de la c. ; car le voisin
\item[\vref{Pr 28:14}] endurcit son cœur tombera ds la c.
\item[\vref{Jé 4:6}] nord le malheur et une grande c.
\item[\vref{Jé 8:21}] cause de la c. de la fille
\item[\vref{Jé 49:32}] les côtés lr. c., dit Yahweh.
\item[\vref{La 3:47}] dégât et la c. ns. sont arrivés.
\item[\vref{Ha 2:9}] pour échapper à l'atteinte de la c. !
\item[\vref{Lu 21:23}] aura une grande c. sur le pays,
\end{listverse}

\ConcordanceEntry{Caleb}
\vspace{-2mm}
\begin{listverse}
\item[\vref{No 13:6}] tribu de Juda : C., fils de Jephunné ;
\item[\vref{No 13:30}] C. fit taire le peuple dvt Moïse,
\item[\vref{No 32:12}] excepté C., fils de Jephunné, le Kénizien, et
\item[\vref{Jos 15:13}] on donna à C., fils de Jephunné,
\item[\vref{Jos 15:14}] Et C. chassa de là les trois fils
\end{listverse}

\ConcordanceEntry{Calme}
\vspace{-2mm}
\begin{listverse}
\item[\vref{Job 3:26}] repos, ni de c., depuis que ce
\item[\vref{Ps 65:2}] Dieu ! ds le c., on te louera
\item[\vref{Ps 89:10}] qnd ses vagues s'élèvent, tu les c.
\item[\vref{Ps 107:29}] la changeant en c., et les ondes
\item[\vref{Pr 17:27}] est d'un esprit c. est un hom.
\item[\vref{Pr 21:14}] fait en cachette c. une fureur violente.
\item[\vref{Jon 1:11}] la mer se c. ? Car la mer
\item[\vref{Mt 8:26}] et il se fit un grand c.
\item[\vref{Mc 4:39}] cessa, et il eut un grand c.
\item[\vref{Lu 8:24}] les flots qui s'apaisèrent, et le c. revint.
\end{listverse}

\ConcordanceEntry{Calomnie}
\vspace{-2mm}
\begin{listverse}
\item[\vref{Lé 19:16}] répandras point de c. parmi ton peuple.
\item[\vref{Ps 15:3}] qui ne c. point avec sa langue ; qui ne
\item[\vref{Ps 31:14}] Car j'entends les c. de plusieurs, la
\item[\vref{Ps 101:5}] retrancherai celui qui c. en secret son
\item[\vref{Pr 10:18}] qui répand la c. est un insensé.
\item[\vref{Pr 30:10}] Ne c. pas un serviteur dvt son maître,
\item[\vref{Jé 9:4}] tt intime ami marche ds la c.
\item[\vref{2 Co 6:8}] milieu de la c. et de la
\end{listverse}

\ConcordanceEntry{Calomnier}
\vspace{-2mm}
\begin{listverse}
\item[\vref{Lé 19:16}] répandras point de c. parmi ton peuple.
\item[\vref{2 S 19:27}] Et il a c. ton serviteur auprès du roi, mon
\item[\vref{Ps 15:3}] qui ne c. point avec sa langue ; qui ne
\item[\vref{Ps 31:14}] Car j'entends les c. de plusieurs, la
\item[\vref{Ps 101:5}] retrancherai celui qui c. en secret son
\item[\vref{Pr 10:18}] qui répand la c. est un insensé.
\item[\vref{Pr 30:10}] Ne c. pas un serviteur dvt son maître,
\item[\vref{Jé 9:4}] tt intime ami marche ds la c.
\item[\vref{Ro 3:8}] quelques-uns, qui ns. c., déclarent que ns.
\item[\vref{1 Co 4:13}] ns. sommes c., et ns. prions ;
\item[\vref{2 Co 6:8}] milieu de la c. et de la
\item[\vref{1 Pi 2:12}] où ils vs. c. com. si vs.
\item[\vref{1 Pi 4:4}] étrange, ils vs. c. de ce que
\end{listverse}

\ConcordanceEntry{Camp}
\vspace{-2mm}
\begin{listverse}
\item[\vref{Ge 32:2}] C'est ici le c. de Dieu ! Et
\item[\vref{Ge 32:10}] bâton, et mntnt je forme deux c.
\item[\vref{Ex 14:19}] allait dvt le c. d'Israël partit, et
\item[\vref{Ex 14:24}] nuée, regarda le c. des Egyptiens et
\item[\vref{Ex 16:13}] qui couvrirent le c., et au matin
\item[\vref{Ex 29:14}] excréments, hors du c.. C'est un sacrifice
\item[\vref{Lé 24:14}] Fais sortir du c. celui qui a
\item[\vref{No 5:2}] mettent hors du c. tt lépreux, tt
\item[\vref{No 12:15}] enfermée hors du c. sept jours ; et
\item[\vref{De 23:10}] sortira hors du c., et n'entrera point
\item[\vref{De 23:14}] milieu de ton c. pour te délivrer
\item[\vref{1 S 4:7}] entré ds le c.. Et ils dirent :
\item[\vref{Am 4:10}] puanteur de vos c. ; mais vs. n'êtes
\item[\vref{Za 14:15}] seront ds ces c., cette plaie sera
\item[\vref{Hé 13:11}] le péché, sont brûlés hors du c.
\item[\vref{Hé 13:13}] lui, hors du c., en portant son
\item[\vref{Ap 20:9}] ils environnèrent le c. des saints, et
\end{listverse}

\ConcordanceEntry{Camper}
\vspace{-2mm}
\begin{listverse}
\item[\vref{Ge 26:17}] de là, et c. ds la vallée
\item[\vref{Ge 33:18}] Canaan, et il c. dvt la ville.
\item[\vref{No 2:2}] Les enfants d'Israël c. chacun sous sa
\item[\vref{No 9:18}] Yahweh, et ils c. sur le commandement
\item[\vref{No 12:16}] Hatséroth, et il c. ds le désert
\item[\vref{No 33:16}] de Sinaï et c. à Kibroth-Hattaava.
\item[\vref{2 S 23:13}] de Philistins était c. ds la vallée
\item[\vref{Esd 8:15}] Ahava, et ns. c. là trois jours.
\item[\vref{Job 19:12}] et se sont c. autour de ma
\item[\vref{Ps 27:3}] tte une armée c. contre moi, mon
\item[\vref{Ps 34:8}] L'Ange de Yahweh c. tt autour de
\item[\vref{Ps 53:6}] de celui qui c. contre toi. Tu
\item[\vref{Es 29:3}] Car je c. en rond contre toi, et je
\item[\vref{Jé 50:29}] qui tendez l'arc, c.-vs. contre elle
\item[\vref{Jé 52:4}] armée ; et ils c. dvt elle et
\item[\vref{Za 9:8}] Je c. autour de ma maison, pour la
\end{listverse}

\ConcordanceEntry{Cana}
\vspace{-2mm}
\begin{listverse}
\item[\vref{Jn 2:1}] des noces à C. en Galilée, et
\item[\vref{Jn 2:11}] premier miracle à C. en Galilée. Et
\end{listverse}

\ConcordanceEntry{Canaan}
\vspace{-2mm}
\begin{listverse}
\item[\vref{Ge 9:18}] Japhet. Cham fut le père de C.
\end{listverse}
\begin{legend}
\NoAutoSpaceBeforeFDP{
\item Fils de Cham, petit-fils de Noé : Ge 9:18
\item C. maudit par Noé  : Ge 9:25; Le 18:3-28
\item Abram s'établit dans le pays de Canaan : Ge 12: 4-7
\item Yahweh donne à Abram et à ses descendants le pays en héritage : Ge 17:8; 13:14-15; 15:18-21, ; Ex  6:3-5
\item Autres: Ge 37:1; De 32:49; Jos 1:1-4, 5:12, 11:23, 13:7, 18:3-6, 21:43; Ps 105:11; Ez 16:3; So 2:5; Ac 13:19
}
\end{legend}

\ConcordanceEntry{Canne}
\vspace{-2mm}
\begin{listverse}
\item[\vref{Ez 40:3}] lin, et une c. à mesurer, et
\item[\vref{Ez 40:5}] la main une c. à mesurer longue
\end{listverse}

\ConcordanceEntry{Cantique}
\vspace{-2mm}
\begin{listverse}
\item[\vref{Ex 15:1}] d'Israël chantèrent ce c. à Yahweh, et
\item[\vref{No 21:17}] Israël chanta ce c. : Monte, puits ! Chantez-lui
\item[\vref{De 31:19}] dc, écrivez ce c.. Enseigne-le aux enfants
\item[\vref{De 31:21}] et d'angoisses, ce c., qui ne sera
\item[\vref{Jg 5:1}] Débora chanta ce c. avec Barak, fils
\item[\vref{Jg 5:12}] réveille-toi, dit le c., lève-toi Barak et
\item[\vref{2 S 1:18}] Juda. C'est le c. de l'arc : Il
\item[\vref{1 R 4:32}] mille paraboles et composa cinq mille c.
\item[\vref{1 Ch 13:8}] en chantant des c. et en jouant
\item[\vref{2 Ch 29:28}] en chantant le c., et les trompettes
\item[\vref{Ps 33:3}] Chantez-lui un c. nouveau ! Jouez de
\item[\vref{Ps 40:4}] ma bouche un c. nouveau, qui est
\item[\vref{Ps 137:4}] Comment chanterions-ns. les c. de Yahweh sur
\item[\vref{Ca 1:1}] Le C. des cantiques qui est de Salomon.
\item[\vref{Es 5:1}] mon bien-aimé le c. de mon bien-aimé
\item[\vref{Mt 26:30}] eurent chanté le c., ils se rendirent
\item[\vref{Ep 5:19}] hymnes et des c. spirituels, chantant et
\item[\vref{Ap 5:9}] ils chantaient un c. nouveau, en disant :
\item[\vref{Ap 14:3}] chantaient com. un c. nouveau dvt le
\item[\vref{Ap 15:3}] Ils chantaient le c. de Moïse, serviteur
\end{listverse}

\ConcordanceEntry{Capable}
\vspace{-2mm}
\begin{listverse}
\item[\vref{De 9:28}] Yahweh n'était pas c. de les conduire
\item[\vref{Né 8:2}] ceux qui étaient c. d'entendre, afin qu'on
\item[\vref{Né 10:28}] ceux qui étaient c. de connaissance et
\item[\vref{Job 41:1}] et qui est c. de se tenir
\item[\vref{Ec 8:17}] il n'est pas c. de trouver ; et
\item[\vref{Da 1:4}] pleins d'intelligence, et c. de se tenir
\item[\vref{Da 2:27}] ne sont pas c. de révéler au
\item[\vref{Mt 19:11}] ne sont pas c. de cela, mais
\item[\vref{2 Co 3:5}] que ns. soyons c. par ns.-mêmes de
\item[\vref{Col 1:12}] ns. a rendus c. d'avoir part à
\item[\vref{2 Ti 2:2}] fidèles, qui soient c. de les enseigner
\item[\vref{Tit 1:9}] afin qu'il soit c. tant d'exhorter par
\item[\vref{Hé 5:2}] étant c. d'avoir de l'indulgence pour les ignorants
\item[\vref{Hé 13:21}] vs. rende c. de tte bonne
\end{listverse}

\ConcordanceEntry{Capacité}
\vspace{-2mm}
\begin{listverse}
\item[\vref{Lé 19:35}] poids, ni ds les mesures de c.
\item[\vref{Mt 25:15}] chacun selon sa c. ; et aussitôt après
\item[\vref{2 Co 3:5}] ns.-mêmes, mais notre c. vient de Dieu,
\end{listverse}

\ConcordanceEntry{Capernaüm}
\vspace{-2mm}
\begin{listverse}
\item[\vref{Mt 4:13}] alla demeurer à C., ville maritime, sur
\item[\vref{Mt 8:5}] fut entré ds C., un centenier vint
\item[\vref{Mt 11:23}] Et toi, C., qui as été élevée jusqu'au ciel,
\item[\vref{Mt 17:24}] lorsqu'ils arrivèrent à C., ceux qui percevaient
\item[\vref{Mc 1:21}] ils entrèrent ds C.. Et le jour
\item[\vref{Jn 4:46}] y avait à C. un officier du
\end{listverse}

\ConcordanceEntry{Cappadoce}
\vspace{-2mm}
\begin{listverse}
\item[\vref{Ac 2:9}] la Judée, la C., le Pont, l'Asie,
\item[\vref{1 Pi 1:1}] la Galatie, la C., l'Asie et la
\end{listverse}

\ConcordanceEntry{Captif}
\vspace{-2mm}
\begin{listverse}
\item[\vref{De 30:3}] Dieu, ramènera tes c. et aura compassion
\item[\vref{2 R 17:23}] Israël fut emmené c. loin de son
\item[\vref{Ps 14:7}] ramènera son peuple c., Jacob se réjouira,
\item[\vref{Ps 68:19}] as emmené des c., tu as pris
\item[\vref{Ps 79:11}] le gémissement des c. parviennent jusqu'à toi !
\item[\vref{Ps 107:10}] la mort, vivaient c. ds l'affliction et
\item[\vref{Ps 126:4}] Yahweh ! ramène nos c., com. des ruisseaux
\item[\vref{Es 5:13}] peuple sera emmené c., parce qu'il n'a
\item[\vref{Es 45:13}] et libérera mes c., sans rançon ni
\item[\vref{Es 61:1}] pour proclamer aux c. la liberté, et
\item[\vref{Jé 13:17}] le troupeau de Yahweh sera emmené c.
\item[\vref{Jé 29:14}] je ramènerai vos c. ; et je vs.
\item[\vref{Jé 33:7}] je ramènerai les c. de Juda, et
\item[\vref{Ez 29:14}] je ramènerai les c. d'Egypte, et les
\item[\vref{Os 6:11}] je ramènerai les c. de mon peuple.
\item[\vref{Am 9:14}] auront été emmenés c. ; et ils rebâtiront
\item[\vref{Za 9:11}] je retirerai tes c. de la fosse
\item[\vref{Za 9:12}] à la forteresse, c. pleins d'espérance ! Aujourd'hui
\item[\vref{Lu 4:19}] pour proclamer aux c. la délivrance, et
\item[\vref{Lu 21:24}] ils seront emmenés c. parmi ttes les
\item[\vref{Ep 4:8}] grande multitude de c., et il a
\end{listverse}

\ConcordanceEntry{Captivité}
\vspace{-2mm}
\begin{listverse}
\item[\vref{De 28:41}] à toi, car ils iront en c.
\item[\vref{2 R 24:14}] Il emmena en c. tt Jérus., à
\item[\vref{Esd 10:6}] du péché de ceux de la c.
\item[\vref{Job 42:10}] Job de sa c., qnd il eut
\item[\vref{Ps 49:16}] scheol, qnd il m'enlèvera de sa c.. Sélah.
\item[\vref{Jé 15:2}] destinés à la c. iront en captivité !
\item[\vref{Jé 29:28}] à Babylone : La c. sera longue ; bâtissez
\item[\vref{Jé 49:3}] s'en va en c. avec ses prêtres
\item[\vref{La 1:18}] mes jeunes hommes sont allés en c.
\item[\vref{Za 14:2}] ville ira en c., mais le reste
\item[\vref{Ap 13:10}] destiné à la c., il ira en
\end{listverse}

\ConcordanceEntry{Carmel}
\vspace{-2mm}
\begin{listverse}
\item[\vref{1 R 18:19}] sur le mont C., les quatre cent
\item[\vref{1 R 18:20}] rassembla les prophètes sur le mont C.
\item[\vref{1 R 18:42}] au sommet du C. ; et se penchant
\item[\vref{2 R 2:25}] la montagne de C., d'où il retourna
\item[\vref{2 R 4:25}] la montagne de C.. L'hom. de Dieu,
\item[\vref{Es 32:16}] et la justice se tiendra en C.
\item[\vref{Jé 46:18}] montagnes, com. le C. qui s'avance ds
\item[\vref{Am 1:2}] le sommet du C. est desséché.
\end{listverse}

\ConcordanceEntry{Carrefour}
\vspace{-2mm}
\begin{listverse}
\item[\vref{Pr 1:21}] crie ds les c., là où on
\item[\vref{Pr 8:2}] lieux élevés, sur le chemin, aux c.
\item[\vref{Es 51:20}] défaillance gisaient aux c. de ttes les
\item[\vref{Ez 21:26}] se tient au c., à l'entrée des
\item[\vref{Na 3:10}] été écrasés aux c. de ttes les
\item[\vref{Mt 22:9}] dc ds les c. des chemins, et
\end{listverse}

\ConcordanceEntry{Casque}
\vspace{-2mm}
\begin{listverse}
\item[\vref{1 S 17:5}] Il avait un c. d'airain sur sa
\item[\vref{1 S 17:38}] lui mit son c. d'airain sur sa
\item[\vref{Es 59:17}] cuirasse, et le c. du salut est
\item[\vref{Jé 46:4}] Présentez-vs. avec vos c., polissez vos lances,
\item[\vref{Ez 23:24}] bouclier, avec les c., et je lr.
\item[\vref{Ez 38:5}] ts ont des boucliers et des c. ;
\item[\vref{Ep 6:17}] prenez aussi le c. du salut, et
\item[\vref{1 Th 5:8}] et ayant pour c. l'espérance du salut.
\end{listverse}

\ConcordanceEntry{Cause}
\vspace{-2mm}
\begin{listverse}
\item[\vref{Ex 18:22}] ttes les petites c. ; ainsi ils te
\item[\vref{De 1:17}] dvt moi la c. qui sera trop
\item[\vref{1 S 24:16}] et plaidera ma c., il me rendra
\item[\vref{Ps 9:5}] droit et ma c., tu sièges sur
\item[\vref{Ps 43:1}] Et défends ma c. contre une nation
\item[\vref{Ps 74:22}] lève-toi, défends ta c. ! Souviens-toi de l'opprobre
\item[\vref{Ps 119:154}] Soutiens ma c. et rachète-moi ; fais-moi
\item[\vref{Pr 22:23}] Yahweh défendra lr. c., et enlèvera l'âme
\item[\vref{Pr 29:7}] connaissance de la c. des pauvres, mais
\item[\vref{Es 1:17}] l'orphelin, défendez la c. de la veuve.
\item[\vref{Es 41:21}] Plaidez votre c., dit Yahweh ; et
\item[\vref{Jé 11:20}] eux, car je t'ai révélé ma c.
\item[\vref{Jé 22:16}] Il jugeait la c. du pauvre et
\item[\vref{Jé 30:13}] qui défende ta c., pour panser ta
\item[\vref{Jé 50:34}] avec chaleur lr. c., pour donner du
\item[\vref{La 3:58}] as plaidé la c. de mon âme,
\item[\vref{Mi 7:9}] qu'il défende ma c., et qu'il me
\item[\vref{Mt 27:37}] écriteau, où la c. de sa condamnation
\end{listverse}

\ConcordanceEntry{Cautionner}
\vspace{-2mm}
\begin{listverse}
\item[\vref{Pr 6:1}] si tu as c. envers ton ami,
\item[\vref{Pr 22:26}] de ceux qui c. pour des dettes.
\item[\vref{Pr 27:13}] Quand quelqu'un aura c. pour l'étranger, prends
\end{listverse}

\ConcordanceEntry{Caverne}
\vspace{-2mm}
\begin{listverse}
\item[\vref{Ge 19:30}] retira ds une c. avec ses deux
\item[\vref{Jos 10:16}] cachèrent ds une c. à Makkéda.
\item[\vref{Jg 6:2}] montagnes, ds des c. et sur les
\item[\vref{1 S 22:1}] sauva ds la c. d'Adullam. Ses frères
\item[\vref{1 S 24:9}] sortit de la c., et cria après
\item[\vref{Es 2:19}] entreront ds les c. des rochers et
\item[\vref{Jé 7:11}] vos yeux qu'une c. de voleurs, cette
\item[\vref{Mt 21:13}] avez fait une c. de voleurs.
\item[\vref{Ap 6:15}] cachèrent ds les c. et entre les
\end{listverse}

\ConcordanceEntry{Cèdre}
\vspace{-2mm}
\begin{listverse}
\item[\vref{Lé 14:4}] du bois de c., du cramoisi et
\item[\vref{1 R 4:33}] arbres, depuis le c. du Liban jusqu'à
\item[\vref{1 R 6:15}] Il revêtit de c. les murs de
\item[\vref{1 R 6:18}] Le bois de c. à l'intérieur de
\item[\vref{Ps 29:5}] Yahweh brise les c., Yahweh brise les
\item[\vref{Ps 92:13}] croît com. le c. au Liban.
\item[\vref{Ca 5:15}] Liban, elle est précieuse com. les c.
\item[\vref{Es 41:19}] au désert le c., l'acacia, le myrte
\item[\vref{Ez 17:3}] Liban, et enleva la cime d'un c.
\item[\vref{Za 11:1}] et que le feu dévore tes c. !
\end{listverse}

\ConcordanceEntry{Ceindre}
\vspace{-2mm}
\begin{listverse}
\item[\vref{Ex 12:11}] Vos reins seront c., vs. aurez vos
\item[\vref{Lé 16:4}] corps ; il se c. de la ceinture
\item[\vref{Ps 18:33}] Dieu qui me c. de force, et
\item[\vref{Ps 30:12}] détaché mon sac, et tu m'as c. de joie,
\item[\vref{Ps 65:7}] force, il est c. de puissance ;
\item[\vref{Es 45:5}] Dieu. Je t'ai c. avant que tu
\item[\vref{La 2:10}] ils se sont c. de sacs ; les
\item[\vref{Lu 12:35}] vos reins soient c., et vos lampes
\item[\vref{Jn 21:18}] un autre te c., et te mènera
\end{listverse}

\ConcordanceEntry{Ceinture}
\vspace{-2mm}
\begin{listverse}
\item[\vref{Ge 3:7}] de figuier, et s'en firent des c.
\item[\vref{Ex 28:4}] tiare, et la c.. Ils feront dc
\item[\vref{Lé 16:4}] ceindra de la c. de lin, et
\item[\vref{1 S 2:4}] qui chancellent ont la force pour c.
\item[\vref{2 R 1:8}] poil, ayant une c. de cuir, ceinte
\item[\vref{Job 12:18}] Il détache la c. des rois, et
\item[\vref{Pr 31:24}] elle livre des c. au marchand.
\item[\vref{Es 11:5}] justice sera la c. de ses reins,
\item[\vref{Jé 2:32}] ornements, l'épouse sa c. ? Mais mon peuple
\item[\vref{Jé 13:1}] et achète-toi une c. de lin et
\item[\vref{Da 10:5}] les reins une c. d'or fin d'Uphaz.
\item[\vref{Mt 3:4}] chameau, et une c. de cuir autour
\item[\vref{Mt 10:9}] ni argent, ni monnaie ds vos c. ;
\item[\vref{Ac 12:8}] dit : Mets ta c. et tes sandales.
\item[\vref{Ac 21:11}] Il prit la c. de Paul, se
\item[\vref{Ep 6:14}] la vérité pour c., ayant revêtu la
\item[\vref{Ap 15:6}] et ayant des c. d'or autour de
\end{listverse}

\ConcordanceEntry{Célébrer}
\vspace{-2mm}
\begin{listverse}
\item[\vref{Ex 5:1}] afin qu'il me c. une fête solennelle
\item[\vref{Ex 9:16}] mon Nom soit c. sur tte la
\item[\vref{Ex 31:16}] le sabbat pour c. le jour du
\item[\vref{No 9:2}] les enfants d'Israël c. la Pâque au
\item[\vref{1 Ch 16:4}] de Yahweh, pour c., remercier, et louer
\item[\vref{1 Ch 16:35}] les nations, pour c. ton saint Nom,
\item[\vref{2 Ch 5:13}] mm accord, pour c. et pour louer
\item[\vref{Né 12:24}] pour louer et c., selon l'ordre de
\item[\vref{Ps 9:2}] Je c. de tt mon cœur Yahweh, je
\item[\vref{Ps 30:5}] ses bien-aimés, et c. la mémoire de
\item[\vref{Ps 68:5}] Chantez à Dieu, c. son Nom ! Exaltez
\item[\vref{Ps 92:2}] chose que de c. Yahweh, et de
\item[\vref{Ps 119:62}] nuit pour te c. à cause des
\item[\vref{Za 14:18}] monteront pas pour c. la fête des
\item[\vref{Ro 15:9}] est écrit : Je c. à cause de
\end{listverse}

\ConcordanceEntry{Céleste}
\vspace{-2mm}
\begin{listverse}
\item[\vref{Mt 6:14}] offenses, votre Père c. vs. pardonnera aussi
\item[\vref{Jn 3:12}] si je vs. parle des choses c. ?
\item[\vref{Ac 26:19}] pas été désobéissant à la vision c.
\item[\vref{1 Co 15:40}] aussi des corps c., et des corps
\item[\vref{1 Co 15:48}] tel qu'est le c., tels aussi sont
\item[\vref{1 Co 15:49}] poussière, ns. porterons aussi l'image du c.
\item[\vref{2 Co 5:2}] ardeur d'être revêtus de notre domicile c.,
\item[\vref{Ep 1:3}] ds les lieux c. en Christ !
\item[\vref{Ph 3:14}] de la vocation c. de Dieu en
\item[\vref{Hé 3:1}] de la vocation c., considérez attentivement Jésus-Christ,
\item[\vref{Hé 11:16}] meilleure, c'est-à-dire une c.. C'est pourquoi Dieu
\item[\vref{Hé 12:22}] vivant, la Jérus. c., d'une multitude innombrable
\end{listverse}

\ConcordanceEntry{Cendre}
\vspace{-2mm}
\begin{listverse}
\item[\vref{Ge 18:27}] qui ne suis que poussière et c.
\item[\vref{Lé 4:12}] l'on répand les c., et il le
\item[\vref{Lé 6:3}] il enlèvera la c. de l'holoc. que
\item[\vref{No 19:9}] pur ramassera les c. de la jeune
\item[\vref{Job 2:8}] se gratter et s'assit sur la c.
\item[\vref{Job 13:12}] des sentences de c., et vos éminences
\item[\vref{Es 44:20}] se repaît de c., et son cœur
\item[\vref{Es 58:5}] sac et la c. ? Appelleras-tu cela un
\item[\vref{Es 61:3}] lieu de la c., une huile de
\item[\vref{Jé 6:26}] roule-toi ds la c., prends le deuil
\item[\vref{Jé 13:18}] asseyez-vs. sur la c. ! Car elle est
\item[\vref{Mt 11:21}] en prenant le sac et la c.
\item[\vref{Hé 9:13}] boucs, et la c. de la génisse,
\end{listverse}

\ConcordanceEntry{Centenier}
\vspace{-2mm}
\begin{listverse}
\item[\vref{Mt 8:5}] ds Capernaüm, un c. vint à lui,
\item[\vref{Mt 27:54}] Le c. et ceux qui étaient avec lui
\item[\vref{Mc 15:44}] fit venir le c., et lui demanda
\item[\vref{Ac 10:22}] ils dirent : Corneille, c., hom. juste et
\item[\vref{Ac 22:25}] Paul dit au c. qui était près
\item[\vref{Ac 27:11}] Mais le c. écouta plus le pilote et le
\item[\vref{Ac 27:43}] Mais le c., voulant sauver Paul, les empêcha d'exécuter
\end{listverse}

\ConcordanceEntry{Cep}
\vspace{-2mm}
\begin{listverse}
\item[\vref{Ge 40:9}] y avait un c. dvt moi.
\item[\vref{Ez 17:6}] et devint un c. de vigne étendu,
\item[\vref{Jn 15:1}] SUIS le vrai c., et mon Père
\end{listverse}

\ConcordanceEntry{Céphas}
\vspace{-2mm}
\begin{listverse}
\item[\vref{Jn 1:42}] tu seras appelé C., c'est-à-dire Pierre.
\item[\vref{1 Co 1:12}] Et moi, de C. ! Et moi, de
\item[\vref{1 Co 9:5}] et les frères du Seign., et C. ?
\item[\vref{1 Co 15:5}] été vu par C., et ensuite par
\item[\vref{Ga 2:9}] Jacques, dis-je, C., et Jean, qui
\end{listverse}

\ConcordanceEntry{César}
\vspace{-2mm}
\begin{listverse}
\item[\vref{Mt 22:17}] le tribut à C., ou non ?
\item[\vref{Mt 22:21}] De C., lui répondirent-ils. Alors il lr. dit :
\item[\vref{Lu 2:1}] un édit par C. Auguste, ordonnant un
\item[\vref{Jn 19:12}] pas ami de C.. Car quiconque se
\item[\vref{Ac 25:11}] livrer à eux. J'en appelle à C.
\item[\vref{Ac 28:19}] d'en appeler à C., n'ayant du reste
\end{listverse}

\ConcordanceEntry{Césarée}
\vspace{-2mm}
\begin{listverse}
\item[\vref{Ac 8:40}] il alla jusqu'à C., en évangélisant ttes
\end{listverse}

\ConcordanceEntry{Chagrin}
\vspace{-2mm}
\begin{listverse}
\item[\vref{Ge 21:12}] N'aie point de c. au sujet de
\item[\vref{Job 17:7}] obscurci par le c., et ts les
\item[\vref{Ps 6:8}] usé par le c. ; il vieillit à
\item[\vref{Ps 31:10}] âme et mon corps dépérissent de c.
\item[\vref{Pr 12:25}] Le c. qui est au cœur de l'hom.
\item[\vref{Ec 1:18}] a beaucoup de c., et celui qui
\item[\vref{Ec 7:3}] vaut mieux le c. que le rire ;
\item[\vref{Ec 12:2}] Ôte le c. de ton cœur, et éloigne de
\item[\vref{Ez 12:19}] lr. pain avec c., et ils boiront
\item[\vref{2 Co 9:7}] non pas avec c., ni par contrainte,
\end{listverse}

\ConcordanceEntry{Chaîne}
\vspace{-2mm}
\begin{listverse}
\item[\vref{Jg 16:21}] lièrent de deux c. d'airain. Il tournait
\item[\vref{2 Ch 33:11}] lièrent d'une double c. d'airain, et l'emmenèrent
\item[\vref{Ps 107:10}] captifs ds l'affliction et ds les c.,
\item[\vref{Jé 40:4}] t'affranchis aujourd'hui des c. que tu as
\item[\vref{Ez 7:23}] Fais une c. ! Car le pays est plein de
\item[\vref{Da 4:15}] liez-le avec des c. de fer et
\item[\vref{Mc 5:3}] le lier, pas mm avec des c.
\item[\vref{Ac 12:6}] lié de deux c. ; et les gardes
\item[\vref{Ac 21:33}] lier de deux c.. Puis il demanda
\item[\vref{Ac 28:20}] l'espérance d'Israël que je porte cette c.
\item[\vref{Ep 6:20}] quoique chargé de c., afin, dis-je, que
\item[\vref{2 Ti 1:16}] n'a pas eu honte de mes c.
\item[\vref{2 Ti 2:9}] mis ds les c. com. un malfaiteur ;
\item[\vref{Hé 11:36}] le fouet, les c. et la prison ;
\item[\vref{Ap 20:1}] et une grande c. ds sa main.
\end{listverse}

\ConcordanceEntry{Chair}
\vspace{-2mm}
\begin{listverse}
\item[\vref{Ge 2:21}] et referma la c. à la place
\item[\vref{Ge 2:23}] mes os et c. de ma chair ;
\item[\vref{Ge 2:24}] fem., et ils deviendront une seule c.
\item[\vref{Ge 6:3}] ne sont que c., et leurs jours
\item[\vref{Ge 6:13}] fin de tte c. est venue dvt
\item[\vref{Ge 9:4}] mangerez point de c. avec son âme,
\item[\vref{Ge 9:17}] moi et tte c. qui est sur
\item[\vref{Ge 17:11}] vs. circoncirez la c. de votre prépuce ;
\item[\vref{Ge 17:13}] sera ds votre c. pour être une
\item[\vref{Ex 12:8}] en mangeront la c. rôtie au feu
\item[\vref{Lé 6:20}] Quiconque touchera sa c. sera saint. Et
\item[\vref{Lé 17:11}] l'âme de la c. est ds le
\item[\vref{Lé 17:14}] l'âme de tte c. est ds son
\item[\vref{Lé 21:5}] ni ne feront d'incisions ds lr. c.
\item[\vref{2 R 9:36}] chiens mangeront la c. de Jézabel ;
\item[\vref{Job 10:11}] peau et de c., et tu m'as
\item[\vref{Job 12:10}] qui vit, et l'esprit de tte c. d'hom.
\item[\vref{Ps 38:4}] sain ds ma c., à cause de
\item[\vref{Ps 78:39}] qu'ils n'étaient que c., qu'un vent qui
\item[\vref{Ps 145:21}] Yahweh, et tte c. bénira le Nom
\item[\vref{Pr 11:17}] âme, mais le cruel trouble sa c.
\item[\vref{Pr 14:30}] vie de la c., mais l'envie est
\item[\vref{Ec 4:5}] les mains et dévore sa propre c.
\item[\vref{Es 40:5}] manifestée, et tte c. en mm temps
\item[\vref{Jé 12:12}] n'y a de paix pour aucune c.
\item[\vref{Jé 17:5}] fait de la c. sa force, et
\item[\vref{Jé 32:27}] Dieu de tte c.. Y a-t-il qq
\item[\vref{La 3:4}] fait vieillir ma c. et ma peau,
\item[\vref{Ez 21:4}] Toute c. verra que moi, Yahweh, j'ai allumé
\item[\vref{Ez 36:26}] j'ôterai de votre c. le cœur de
\item[\vref{Za 2:13}] Que tte c. se taise dvt
\item[\vref{Za 11:9}] se dévorent la c. les unes les
\item[\vref{Mt 16:17}] sont pas la c. et le sang
\item[\vref{Mt 19:5}] les deux ne seront qu'une seule c. ?
\item[\vref{Mt 26:41}] prompt, mais la c. est faible.
\item[\vref{Mc 14:38}] prompt, mais la c. est faible.
\item[\vref{Lu 3:6}] Et tte c. verra le salut de Dieu.
\item[\vref{Lu 24:39}] esprit n'a ni c. ni os, com.
\item[\vref{Jn 1:13}] volonté de la c., ni de la
\item[\vref{Jn 1:14}] a été faite c., elle a habité
\item[\vref{Jn 3:6}] né de la c. est chair, et
\item[\vref{Jn 6:51}] donnerai, c'est ma c. que je donnerai
\item[\vref{Jn 6:53}] mangez pas la c. du Fils de
\item[\vref{Jn 6:54}] qui mange ma c. et qui boit
\item[\vref{Jn 6:55}] Car ma c. est une véritable nourriture, et mon
\item[\vref{Jn 6:63}] qui vivifie ; la c. ne sert de
\item[\vref{Ac 2:17}] Esprit sur tte c. ; vos fils et
\item[\vref{Ac 2:26}] de plus, ma c. reposera avec espérance.
\item[\vref{Ro 6:19}] l'infirmité de votre c.. Comme dc vs.
\item[\vref{Ro 7:5}] étions ds la c., les passions des
\item[\vref{Ro 7:25}] suis par la c. esclave de la
\item[\vref{Ro 8:4}] pas selon la c., mais selon l'Esprit.
\item[\vref{Ro 8:6}] l'affection de la c., c'est la mort,
\item[\vref{Ro 8:7}] l'affection de la c. est inimitié contre
\item[\vref{Ro 8:13}] vivez selon la c., vs. mourrez ; mais
\item[\vref{Ro 9:5}] sorti selon la c., Christ, qui est
\item[\vref{1 Co 15:39}] Toute c. n'est pas de la mm chair,
\item[\vref{1 Co 15:50}] c'est que la c. et le sang
\item[\vref{2 Co 12:7}] écharde ds la c., un ange de
\item[\vref{Ga 1:16}] consultai ni la c. ni le sang,
\item[\vref{Ga 2:20}] mntnt ds la c., je vis ds
\item[\vref{Ga 4:23}] naquit selon la c., et celui de
\item[\vref{Ga 5:17}] Car la c. a des désirs contraires à ceux
\item[\vref{Ga 5:19}] œuvres de la c. sont évidentes : Ce
\item[\vref{Ga 5:24}] ont crucifié la c. avec ses passions
\item[\vref{Ep 2:3}] convoitises de notre c., accomplissant les désirs
\item[\vref{Ep 2:15}] aboli ds sa c. l'inimitié, à savoir
\item[\vref{Ep 6:12}] lutter contre la c. et le sang,
\item[\vref{Ph 1:24}] vs. que je demeure ds la c.
\item[\vref{1 Ti 3:16}] été manifesté en c., justifié par l'Esprit,
\item[\vref{Hé 2:14}] participent à la c. et au sang,
\item[\vref{Hé 10:20}] travers le voile, c'est-à-dire sa propre c.
\item[\vref{1 Pi 3:18}] mort en la c., mais vivifié par
\item[\vref{1 Jn 2:16}] convoitise de la c., et la convoitise
\item[\vref{1 Jn 4:2}] est venu en c. est de Dieu,
\item[\vref{Ap 17:16}] et mangeront sa c., et la brûleront
\end{listverse}

\ConcordanceEntry{Chaire}
\vspace{-2mm}
\begin{listverse}
\item[\vref{Mt 23:2}] assis ds la c. de Moïse.
\end{listverse}

\ConcordanceEntry{Chaldée}
\vspace{-2mm}
\begin{listverse}
\item[\vref{Ge 11:28}] de sa naissance, à Ur en C.
\item[\vref{Ge 15:7}] sortir d'Ur en C., afin de te
\item[\vref{Jé 50:10}] Et la C. sera abandonnée au pillage ; ts ceux
\item[\vref{Ez 23:15}] de Babylone en C., terre de lr.
\end{listverse}

\ConcordanceEntry{Chaldéens}
\vspace{-2mm}
\begin{listverse}
\item[\vref{2 R 24:2}] des troupes de C., des armées de
\item[\vref{2 R 25:4}] pendant que les C. environnaient la ville.
\item[\vref{2 R 25:13}] Les C. brisèrent les colonnes d'airain qui étaient
\item[\vref{Es 13:19}] et l'orgueil des C., sera com. Sodome
\item[\vref{Es 43:14}] le cri des C. sera ds les
\item[\vref{Es 47:1}] la fille des C. ! Car tu ne
\item[\vref{Jé 25:12}] le pays des C., que je mettrai
\item[\vref{Jé 37:9}] en disant : Les C. s'en iront loin
\item[\vref{Da 2:2}] enchanteurs et les C., pour qu'ils lui
\item[\vref{Da 4:7}] les astrologues, les C. et les devins.
\item[\vref{Ha 1:6}] vais susciter les C., ce peuple cruel
\end{listverse}

\ConcordanceEntry{Chaleur}
\vspace{-2mm}
\begin{listverse}
\item[\vref{Ge 8:22}] froid et la c., l'été et l'hiver,
\item[\vref{Ge 31:40}] Le jour la c. me consumait, et
\item[\vref{Ex 16:21}] et lorsque la c. du soleil était
\item[\vref{De 28:22}] fièvre, d'inflammation, de c. brûlante, de l'épée,
\item[\vref{Job 6:17}] temps où la c. donne dessus, défaillent ;
\item[\vref{Ps 19:7}] Rien ne se dérobe à sa c.
\item[\vref{Ec 4:11}] auront plus de c. ; mais celui qui
\item[\vref{Es 4:6}] l'ombre contre la c. du jour, pour
\item[\vref{Es 18:4}] demeure, par la c. de la lumière,
\item[\vref{Es 25:4}] l'ombrage contre la c. ; car le souffle
\item[\vref{Es 49:10}] pas soif ; la c. et le soleil
\item[\vref{Mt 20:12}] le poids du jour et la c.
\item[\vref{Ap 7:16}] ne les frappera plus, ni aucune c.
\item[\vref{Ap 16:9}] par de grandes c., et ils blasphémèrent
\end{listverse}

\ConcordanceEntry{Cham}
\vspace{-2mm}
\begin{listverse}
\item[\vref{Ge 5:32}] ans, engendra Sem, C., et Japhet.
\item[\vref{Ge 7:13}] jour, Noé, Sem, C. et Japhet, fils
\item[\vref{Ge 9:18}] l'arche étaient Sem, C., et Japhet. Cham
\item[\vref{Ge 9:22}] Et C., père de Canaan, vit la nudité
\item[\vref{Ge 10:6}] les fils de C. furent : Cusch, Mitsraïm,
\item[\vref{1 Ch 4:40}] habitaient là auparavant étaient descendus de C.
\end{listverse}

\ConcordanceEntry{Chambre}
\vspace{-2mm}
\begin{listverse}
\item[\vref{1 R 17:19}] porta ds la c. haute où il
\item[\vref{2 R 4:10}] prie, une petite c. haute avec des
\item[\vref{2 R 11:2}] nourrice ds la c. des lits. Il
\item[\vref{Né 13:5}] disposé une grande c., où on mettait
\item[\vref{Ez 8:12}] chacun ds sa c. pleine de figures ?
\item[\vref{Joë 2:16}] l'épouse de sa c. nuptiale !
\item[\vref{Mt 6:6}] entre ds ta c., et ayant fermé
\item[\vref{Mt 24:26}] est ds les c.. Ne le croyez
\item[\vref{Lu 12:3}] l'oreille ds les c. sera prêché sur
\item[\vref{Lu 22:12}] montrera une grande c. haute, meublée ; c'est
\item[\vref{Ac 1:13}] montèrent ds une c. haute où demeuraient
\item[\vref{Ac 9:39}] conduisit ds la c. haute. Toutes les
\item[\vref{Ac 20:8}] lampes ds la c. haute où ils
\end{listverse}

\ConcordanceEntry{Chameau}
\vspace{-2mm}
\begin{listverse}
\item[\vref{Ge 24:64}] vit Isaac, et descendit de son c. ;
\item[\vref{Ge 31:34}] le bât d'un c., et s'était assise
\item[\vref{Lé 11:4}] fendu, com. le c., car il rumine
\item[\vref{Mt 3:4}] de poils de c., et une ceinture
\item[\vref{Mt 19:24}] aisé à un c. de passer par
\item[\vref{Mt 23:24}] le moucheron et vs. engloutissez le c.
\item[\vref{Mc 10:25}] facile à un c. de passer par
\end{listverse}

\ConcordanceEntry{Chanceler}
\vspace{-2mm}
\begin{listverse}
\item[\vref{Job 12:25}] il les fait c. com. des gens
\item[\vref{Ps 17:5}] ds tes sentiers, mes pieds ne c. point.
\item[\vref{Ps 27:2}] mes ennemis qui c. et tombent.
\item[\vref{Ps 35:15}] Mais qnd je c., ils se réjouissent
\item[\vref{Ps 94:18}] dis : Mon pied c. ! ta bonté me
\item[\vref{Ps 121:3}] que ton pied c., celui qui te
\item[\vref{Ps 125:1}] Sion : Elle ne c. point et est
\end{listverse}

\ConcordanceEntry{Chandelier}
\vspace{-2mm}
\begin{listverse}
\item[\vref{Ex 25:31}] feras aussi un c. d'or pur. Le
\item[\vref{Lé 24:4}] lampes sur le c. pur, dvt Yahweh.
\item[\vref{1 R 7:49}] les c. d'or pur, cinq à droite et
\item[\vref{Jé 52:19}] les chaudrons, les c., les tasses et
\item[\vref{Za 4:2}] y a un c. tt en or,
\item[\vref{Mt 5:15}] mais sur un c. et elle éclaire
\item[\vref{Hé 9:2}] lequel étaient le c., et la table,
\item[\vref{Ap 1:20}] et les sept c. d'or. Les sept
\item[\vref{Ap 2:1}] qui marche au milieu des sept c. d'or :
\item[\vref{Ap 2:5}] et j'ôterai ton c. de sa place
\item[\vref{Ap 11:4}] et les deux c. qui se tiennent
\end{listverse}

\ConcordanceEntry{Changement}
\vspace{-2mm}
\begin{listverse}
\item[\vref{Job 14:14}] combat, jusqu'à ce qu'il m'arrive du c.
\item[\vref{Ps 55:20}] a point de c. en eux, et
\item[\vref{Hé 7:12}] nécessaire qu'il y ait aussi un c. de loi.
\item[\vref{Hé 12:27}] encore, marquent le c. des choses ébranlées,
\item[\vref{Ja 1:17}] n'y a ni c. ni ombre de
\end{listverse}

\ConcordanceEntry{Changer}
\vspace{-2mm}
\begin{listverse}
\item[\vref{Ge 31:7}] moi et a c. dix fois mon
\item[\vref{Ge 35:2}] vs., purifiez-vs., et c. de vêtements.
\item[\vref{Ge 50:20}] mal : Dieu l'a c. en bien, pour
\item[\vref{1 S 10:6}] et tu seras c. en un autre
\item[\vref{1 S 10:9}] de Samuel, Dieu c. son cœur, et
\item[\vref{Né 13:2}] notre Dieu avait c. la malédiction en
\item[\vref{Ps 89:35}] et je ne c. point ce qui
\item[\vref{Ps 102:27}] vêt. ; tu les c. com. un habit,
\item[\vref{Jé 2:11}] une nation qui c. ses dieux, quoiqu'ils
\item[\vref{Da 2:21}] C'est lui qui c. les temps et
\item[\vref{Da 7:25}] aura l'intention de c. les temps et
\item[\vref{Joë 2:31}] le soleil se c. en ténèbres, et
\item[\vref{Mal 3:6}] je n'ai point c. ; à cause de
\item[\vref{Lu 9:29}] de son visage c., et son vêt.
\item[\vref{Jn 4:46}] où il avait c. l'eau en vin.
\item[\vref{Jn 16:20}] dis-je, attristés ; mais votre tristesse sera c. en joie.
\item[\vref{Ro 1:23}] et ils ont c. la gloire du
\item[\vref{Ro 1:26}] parmi eux ont c. l'usage naturel en
\item[\vref{1 Co 15:51}] pas ts, mais ts ns. serons c.,
\item[\vref{Hé 6:6}] s'ils retombent, soient c. de nouveau par
\item[\vref{Hé 7:12}] la prêtrise étant c., il est nécessaire
\item[\vref{Ja 4:9}] votre rire se c. en pleurs, et
\item[\vref{Jud 1:4}] des impies, qui c. la grâce de
\item[\vref{Ap 8:11}] des eaux fut c. en absinthe, et
\item[\vref{Ap 11:6}] le pouvoir de c. les eaux en
\end{listverse}

\ConcordanceEntry{Chant}
\vspace{-2mm}
\begin{listverse}
\item[\vref{No 23:21}] en lui un c. de triomphe royal.
\item[\vref{2 S 1:17}] et sur Jonathan, son fils, un c. funèbre,
\item[\vref{1 Ch 15:21}] à huit cordes, pour conduire le c.
\item[\vref{2 Ch 20:22}] ils commencèrent le c. et la louange,
\item[\vref{Né 12:27}] et par des c. sur des cymbales,
\item[\vref{Job 8:21}] tes lèvres de c. d'allégresse.
\item[\vref{Ps 28:7}] pourquoi je le loue par mes c.
\item[\vref{Ps 32:7}] tu m'environnes de c. de triomphe à
\item[\vref{Ps 100:2}] lui avec un c. de joie !
\item[\vref{Ps 137:3}] des paroles de c., et nos oppresseurs
\item[\vref{Es 35:10}] en Sion avec c. de triomphe, et
\item[\vref{So 3:14}] Réjouis-toi avec c. de triomphe, fille
\end{listverse}

\ConcordanceEntry{Chanter}
\vspace{-2mm}
\begin{listverse}
\item[\vref{Ex 15:1}] les enfants d'Israël c. ce cantique à
\item[\vref{Ex 15:21}] Marie lr. répondait : C. à Yahweh, car
\item[\vref{Ex 32:18}] j'entends une voix de gens qui c.
\item[\vref{Jg 5:3}] l'oreille ! Moi, je c. à Yahweh, je
\item[\vref{1 S 18:7}] Les femmes c., se répondant les
\item[\vref{1 Ch 16:9}] C.-le, célébrez-le ! Parlez de ttes ses
\item[\vref{2 Ch 31:2}] service, célébrer et c. les louanges aux
\item[\vref{Job 35:10}] donne de quoi c. pendant la nuit,
\item[\vref{Ps 13:6}] m'auras donnée ; je c. à Yahweh, parce
\item[\vref{Ps 18:50}] nations ! et je c. des psaumes à
\item[\vref{Ps 33:3}] C.-lui un cantique nouveau ! Jouez de
\item[\vref{Ps 47:7}] C. à Dieu, chantez ! Chantez à notre
\item[\vref{Ps 55:1}] chantres, pour le c. avec instruments à
\item[\vref{Ps 71:22}] du luth, je c. ta fidélité, mon
\item[\vref{Ps 96:1}] C. à Yahweh un cantique nouveau ! Vous
\item[\vref{Ps 101:1}] de David. Je c. la miséricorde et
\item[\vref{Ps 104:33}] Je c. à Yahweh durant ma vie ; je
\item[\vref{Ps 138:1}] cœur, je te c. des louanges ds
\item[\vref{Ps 147:1}] est beau de c. à notre Dieu !
\item[\vref{Es 5:1}] Je c. mntnt pour mon bien-aimé le cantique
\item[\vref{Mt 26:30}] Quand ils eurent c. le cantique, ils
\item[\vref{Mc 13:35}] où le coq c., ou le matin ;
\item[\vref{Mc 14:30}] que le coq c. deux fois, tu
\item[\vref{Lu 7:32}] ns. vs. avons c. des complaintes, et
\item[\vref{Ac 16:25}] Silas priaient et c. les louanges de
\item[\vref{1 Co 14:15}] être entendu ; je c. par l'esprit, mais
\item[\vref{Ep 5:19}] des cantiques spirituels, c. et psalmodiant de
\item[\vref{Col 3:16}] des cantiques spirituels, c. ds votre cœur
\item[\vref{Ja 5:13}] Quelqu'un est-il ds la joie ? Qu'il c.
\item[\vref{Ap 5:9}] Et ils c. un cantique nouveau, en disant : Tu
\item[\vref{Ap 14:3}] Et ils c. com. un cantique nouveau dvt le
\item[\vref{Ap 15:3}] Ils c. le cantique de Moïse, serviteur de
\end{listverse}

\ConcordanceEntry{Chantre}
\vspace{-2mm}
\begin{listverse}
\item[\vref{2 S 23:1}] de Jacob, du c. agréable d'Israël :
\item[\vref{1 R 10:12}] luths pour les c. ; il ne vint
\item[\vref{1 Ch 6:32}] le service com. c. dvt le tabernacle,
\item[\vref{2 Ch 9:11}] luths pour les c.. On n'en avait
\item[\vref{2 Ch 20:21}] il établit des c. de Yahweh, qui
\item[\vref{2 Ch 23:13}] des trompettes ; les c., avec des instruments
\item[\vref{Né 7:44}] C. : Les fils d'Asaph, cent quarante-huit.
\item[\vref{Né 11:22}] les fils d'Asaph, c., pour l'ouvrage de
\item[\vref{Ps 13:1}] de David, donné au chef des c.
\item[\vref{Ps 36:1}] de Yahweh, donné au chef des c.
\item[\vref{Ps 39:1}] au chef des c.. A Jeduthun.
\item[\vref{Ps 42:1}] de Koré, donné au chef des c.
\item[\vref{Ps 51:1}] Psaume de David, au chef des c.
\item[\vref{Ps 68:26}] Les c. allaient dvt, ensuite les joueurs d'instruments,
\end{listverse}

\ConcordanceEntry{Char}
\vspace{-2mm}
\begin{listverse}
\item[\vref{Ge 45:27}] il vit les c. que Joseph avait
\item[\vref{Ex 14:6}] fit atteler son c. et il prit
\item[\vref{Ex 15:4}] la mer les c. de Pharaon et
\item[\vref{Jos 11:6}] chevaux, et brûleras au feu leurs c.
\item[\vref{Jos 17:16}] vallée ont des c. de fer, et
\item[\vref{Jg 4:3}] avait neuf cents c. de fer, et
\item[\vref{1 S 6:11}] mirent sur le c. l'arche de Yahweh,
\item[\vref{2 S 6:3}] Dieu sur un c. tt neuf, et
\item[\vref{1 R 22:38}] Lorsqu'on lava le c. à l'étang de
\item[\vref{2 R 2:11}] parlant, voici, un c. de feu et
\item[\vref{2 R 2:12}] père ! Mon père ! C. d'Israël et sa
\item[\vref{2 R 6:17}] chevaux et de c. de feu autour
\item[\vref{2 R 7:6}] un bruit de c., et un bruit
\item[\vref{2 R 18:24}] à cause des c. et des cavaliers.
\item[\vref{2 R 23:30}] mort sur un c. ; ils l'amenèrent de
\item[\vref{Ps 20:8}] vantent de leurs c., et les autres
\item[\vref{Ps 68:18}] Les c. de Dieu se comptent par vingt-mille,
\item[\vref{Ag 2:22}] je renverserai les c. et ceux qui
\item[\vref{Ac 8:38}] fit arrêter le c. ; Philippe et l'eunuque
\item[\vref{Ap 18:13}] des chevaux, des c., des esclaves, et
\end{listverse}

\ConcordanceEntry{Charan}
\vspace{-2mm}
\begin{listverse}
\item[\vref{Ge 11:31}] ils vinrent jusqu'à C., et ils y
\item[\vref{Ge 27:43}] et enfuis-toi à C., vers Laban, mon
\item[\vref{Ge 28:10}] de Beer-Schéba et s'en alla à C.
\item[\vref{Ac 7:2}] qu'il s'établisse à C., et lui dit :
\end{listverse}

\ConcordanceEntry{Chardons}
\vspace{-2mm}
\begin{listverse}
\item[\vref{Ge 3:18}] épines et des c. ; et tu mangeras
\item[\vref{Jg 8:7}] épines du désert et avec des c.
\item[\vref{Pr 24:31}] était monté en c., les ronces avaient
\item[\vref{Mt 7:16}] épines, ou des figues sur des c. ?
\item[\vref{Hé 6:8}] épines et des c., est rejetée et
\end{listverse}

\ConcordanceEntry{Charge}
\vspace{-2mm}
\begin{listverse}
\item[\vref{Ge 40:13}] rétablira ds ta c., et tu mettras
\item[\vref{Ex 6:6}] de dessous les c. des Egyptiens, et
\item[\vref{Ex 23:5}] abattu sous sa c., tu t'arrêteras pour
\item[\vref{No 1:50}] aux Lévites la c. du tabernacle du
\item[\vref{No 4:49}] porter, et la c. de chacun fut
\item[\vref{No 11:17}] avec toi la c. du peuple, et
\item[\vref{De 3:28}] Donnes-en la c. à Josué, fortifie-le
\item[\vref{2 S 15:33}] me seras à c. si tu viens
\item[\vref{2 R 5:17}] ton serviteur, une c. de deux mulets ;
\item[\vref{2 R 7:17}] il s'appuyait, la c. de garder la
\item[\vref{2 Ch 30:17}] Lévites eurent la c. d'immoler la Pâque
\item[\vref{Né 5:18}] étaient à la c. de ce peuple.
\item[\vref{Ps 109:8}] courte, et qu'un autre prenne sa c. !
\item[\vref{Es 30:27}] et une pesante c. ; ses lèvres sont
\item[\vref{Ac 1:20}] l'habite, et qu'un autre prenne sa c.
\item[\vref{1 Co 9:17}] moi, c'est une c. qui m'est confiée.
\item[\vref{2 Co 11:9}] d'être à votre c. et je m'en
\item[\vref{Ga 2:8}] Pierre ds la c. d'apôtre pour les
\item[\vref{1 Ti 3:1}] quelqu'un désire la c. d'évêque, il désire
\item[\vref{Ap 2:24}] ne mettrai pas sur vs. d'autre c.
\end{listverse}

\ConcordanceEntry{Charger}
\vspace{-2mm}
\begin{listverse}
\item[\vref{Lé 19:17}] tu ne te c. point d'un péché
\item[\vref{De 15:14}] Tu c., chargeras de qq chose de ton
\item[\vref{Jos 4:5}] chacun de vs. c. une pierre sur
\item[\vref{1 S 16:20}] un âne, qu'il c. de pain, et
\item[\vref{1 R 14:6}] d'autre ? Je suis c. de t'annoncer des
\item[\vref{2 R 22:5}] ceux qui sont c. de faire exécuter
\item[\vref{Job 36:16}] ta table sera c. de viandes grasses.
\item[\vref{Es 53:4}] et il s'est c. de nos douleurs ;
\item[\vref{Ez 7:4}] mais je te c. de tes voies,
\item[\vref{Mt 11:28}] êtes fatigués et c., et je vs.
\item[\vref{Mt 16:24}] et qu'il se c. de sa croix,
\item[\vref{Lu 11:46}] loi ! Car vs. c. les hommes de
\item[\vref{Lu 23:26}] champs, et le c. de la croix
\item[\vref{Ac 16:23}] qu'on les eut c. de coups de
\item[\vref{Ep 6:20}] suis ambassadeur quoique c. de chaînes, afin,
\end{listverse}

\ConcordanceEntry{Charité}
\vspace{-2mm}
\begin{listverse}
\item[\vref{Pr 10:12}] querelles, mais la c. couvre ttes les
\item[\vref{Pr 31:26}] loi de la c. est sur sa
\item[\vref{Mt 24:12}] sera multipliée, la c. de plusieurs se
\item[\vref{Ro 12:10}] portés par la c. fraternelle à vs.
\item[\vref{Ro 13:10}] La c. ne fait pas de mal au
\item[\vref{Ro 15:30}] et par la c. de l'Esprit, que
\item[\vref{1 Co 4:21}] verge, ou avec c. et ds un
\item[\vref{1 Co 8:1}] connaissance. La connaissance enfle, mais la c. édifie.
\item[\vref{1 Co 13:1}] n'ai pas la c., je suis un
\item[\vref{1 Co 13:4}] La c. est patiente, la charité est douce,
\item[\vref{1 Co 13:8}] La c. ne périt jamais. Les prophéties seront
\item[\vref{1 Co 13:13}] l'espérance et la c. ; mais la plus
\item[\vref{1 Co 14:1}] Recherchez la c.. Désirez avec ardeur
\item[\vref{1 Co 16:14}] ttes vos affaires se fassent avec c.
\item[\vref{2 Co 2:4}] vs. connaissiez la c. tte particulière que
\item[\vref{2 Co 2:8}] de confirmer publiquement envers lui votre c.
\item[\vref{2 Co 5:14}] Car la c. de Christ ns. unit étroitement, car
\item[\vref{2 Co 6:6}] douceur, par le Saint-Esprit, par une c. sincère,
\item[\vref{2 Co 8:7}] et ds la c. que vs. avez
\item[\vref{2 Co 8:24}] preuve de votre c. et du sujet
\item[\vref{2 Co 13:11}] le Dieu de c. et de paix
\item[\vref{2 Co 13:13}] Seign. Jésus-Christ, la c. de Dieu et
\item[\vref{Ga 5:6}] la foi qui opère par la c.
\item[\vref{Ga 5:13}] servez-vs. les uns les autres avec c.
\item[\vref{Ga 5:22}] l'Esprit c'est la c., la joie, la
\item[\vref{Ep 1:4}] et irrépréhensibles dvt lui ds la c.,
\item[\vref{Ep 1:15}] et de la c. que vs. avez
\item[\vref{Ep 2:4}] de sa grande c. dont il ns.
\item[\vref{Ep 3:17}] qu'étant enracinés et fondés ds la c.,
\item[\vref{Ep 3:19}] et connaître la c. de Christ qui
\item[\vref{Ep 4:2}] les uns les autres ds la c.,
\item[\vref{Ep 4:15}] vérité avec la c., ns. croissions en
\item[\vref{Ep 4:16}] afin qu'il soit édifié ds la c.
\item[\vref{Ep 5:2}] marchez ds la c., ainsi que Christ
\item[\vref{Ep 6:23}] frères, et la c. avec la foi,
\item[\vref{Ph 1:8}] conformément à la c. de Jésus-Christ.
\item[\vref{Ph 1:9}] grâce : Que votre c. abonde encore de
\item[\vref{Ph 1:17}] le font par c., sachant que je
\item[\vref{Ph 2:1}] soulagement ds la c., s'il y a
\item[\vref{Col 1:4}] et de votre c. envers ts les
\item[\vref{Col 1:8}] fait connaître votre c. par le Saint-Esprit.
\item[\vref{Col 2:2}] ensemble ds la c., et enrichis d'une
\item[\vref{Col 3:14}] revêtez-vs. de la c., qui est le
\item[\vref{1 Th 1:3}] travail de votre c., et l'immuabilité de
\item[\vref{1 Th 3:6}] et de votre c., et ns. a
\item[\vref{1 Th 3:12}] en plus en c. les uns envers
\item[\vref{1 Th 4:9}] Quant à la c. fraternelle, vs. n'avez
\item[\vref{1 Th 5:8}] et de la c., et ayant pour
\item[\vref{2 Th 1:3}] et que votre c. mutuelle fait des
\item[\vref{1 Ti 1:5}] commandement c'est la c. qui procède d'un
\item[\vref{1 Ti 2:15}] foi, ds la c., et ds la
\item[\vref{1 Ti 4:12}] en conduite, en c., en esprit, en
\item[\vref{1 Ti 6:11}] la foi, la c., la patience, la
\item[\vref{2 Ti 1:7}] de force, de c. et de sagesse.
\item[\vref{2 Ti 1:13}] et ds la c. qui est en
\item[\vref{2 Ti 2:22}] la foi, la c. et la paix
\item[\vref{2 Ti 3:10}] ma douceur, ma c., ma persévérance.
\item[\vref{Tit 2:2}] foi, ds la c., et ds la
\item[\vref{Phm 1:5}] et de ta c. envers ts les
\item[\vref{Phm 1:9}] plutôt par la c., bien que je
\item[\vref{Hé 6:10}] travail de la c. que vs. avez
\item[\vref{Hé 10:24}] exciter à la c. et aux bonnes
\item[\vref{Hé 13:1}] Que la c. fraternelle demeure ds vos cœurs.
\item[\vref{1 Pi 4:8}] autres une ardente c., car la charité
\item[\vref{1 Pi 5:14}] un baiser de c.. Que la paix
\item[\vref{2 Pi 1:7}] fraternel, et à l'amour fraternel la c.
\item[\vref{1 Jn 3:1}] Voyez quelle c. le Père ns.
\item[\vref{1 Jn 3:16}] avons connu la c., en ce qu'il
\item[\vref{1 Jn 4:7}] autres, car la c. est de Dieu ;
\item[\vref{1 Jn 4:8}] pas connu Dieu, car Dieu est c.
\item[\vref{1 Jn 4:9}] La c. de Dieu a été manifestée envers
\item[\vref{1 Jn 4:16}] cru en la c. que Dieu a
\item[\vref{2 Jn 1:6}] c'est ici la c., que ns. marchions
\item[\vref{3 Jn 1:6}] témoignage de ta c.. Et tu feras
\item[\vref{Ap 2:4}] que tu as abandonné ta première c.
\item[\vref{Ap 2:19}] tes œuvres, ta c., ton service, ta
\end{listverse}

\ConcordanceEntry{Charnel}
\vspace{-2mm}
\begin{listverse}
\item[\vref{Ro 7:14}] moi, je suis c., vendu au péché.
\item[\vref{Ro 15:27}] doivent aussi lr. faire part des c.
\item[\vref{1 Co 3:1}] à des hommes c., c'est-à-dire com. à
\item[\vref{1 Co 3:3}] divisions, n'êtes-vs. pas c., et ne vs.
\item[\vref{1 Co 3:4}] moi, je suis d'Apollos ! N'êtes-vs. pas c. ?
\item[\vref{2 Co 10:4}] ne sont pas c., mais elles sont
\item[\vref{Hé 7:16}] loi du commandement c., mais selon la
\item[\vref{1 Pi 2:11}] abstenir des convoitises c. qui font la
\end{listverse}

\ConcordanceEntry{Charpentier}
\vspace{-2mm}
\begin{listverse}
\item[\vref{2 S 5:11}] de cèdre, des c. et des tailleurs
\item[\vref{2 R 12:11}] argent pour les c. et pour les
\item[\vref{2 R 22:6}] pour les c., les architectes et les maçons, pour
\item[\vref{Esd 3:7}] pierres et aux c., ils donnèrent aussi
\item[\vref{Es 44:13}] Le c. étend sa règle, il trace sa
\item[\vref{Jé 24:1}] Juda, avec les c. et les serruriers,
\item[\vref{Jé 29:2}] Jérus., et les c. et les serruriers.
\item[\vref{Mt 13:55}] le fils du c. ? Sa mère ne
\end{listverse}

\ConcordanceEntry{Charrue}
\vspace{-2mm}
\begin{listverse}
\item[\vref{Lu 9:62}] main à la c., et regarde en
\end{listverse}

\ConcordanceEntry{Chasser}
\vspace{-2mm}
\begin{listverse}
\item[\vref{Ge 3:23}] Yahweh Dieu le c. du jardin d'Eden
\item[\vref{Ge 27:3}] les champs, et c.-moi du gibier.
\item[\vref{Ex 6:1}] puissante, il les c. de son pays.
\item[\vref{Ex 23:30}] Mais je les c. peu à peu
\item[\vref{Lé 18:24}] que je vais c. de dvt vs.
\item[\vref{De 7:1}] et qu'il aura c. de dvt toi
\item[\vref{Jos 24:18}] Yahweh a c. dvt ns. ts les peuples, et
\item[\vref{2 S 13:17}] il dit : Qu'on c. cette fem. loin
\item[\vref{1 Ch 17:21}] des prodiges, en c. les nations dvt
\item[\vref{Job 22:23}] tu seras rétabli. C. l'iniquité loin de
\item[\vref{Ps 35:5}] et que l'Ange de Yahweh les c. !
\item[\vref{Pr 6:26}] la fem. adultère c. après l'âme précieuse
\item[\vref{Pr 22:10}] C. le moqueur, et le débat prendra
\item[\vref{Ez 34:16}] celle qui était c., je banderai la
\item[\vref{Da 4:25}] On te c. du milieu des hommes, tu auras
\item[\vref{Os 9:15}] aversion. Je les c. de ma maison
\item[\vref{Jon 2:5}] disais : Je suis c. loin de tes
\item[\vref{Mt 7:22}] Nom ? N'avons-ns. pas c. les démons en
\item[\vref{Mt 8:16}] démoniaques. Et il c. par sa parole
\item[\vref{Mt 10:1}] le pouvoir de c. les esprits impurs
\item[\vref{Mt 12:26}] Or si Satan c. Satan, il est
\item[\vref{Mt 17:19}] particulier : Pourquoi n'avons-ns. pas pu le c. ?
\item[\vref{Mc 3:23}] forme de paraboles : Comment Satan peut-il c. Satan ?
\item[\vref{Mc 6:13}] Ils c. beaucoup de démons hors des possédés,
\item[\vref{Mc 11:15}] se mit à c. dehors ceux qui
\item[\vref{Lu 6:22}] haïront, et vs. c., et vs. outrageront,
\item[\vref{Lu 9:40}] disciples de le c., mais ils n'ont
\item[\vref{Jn 2:15}] cordes, il les c. ts du temple,
\item[\vref{Jn 16:2}] Ils vs. c. des synagogues ; mm l'heure vient où
\item[\vref{Ga 4:30}] que dit l'Ecriture ? C. l'esclave et son
\item[\vref{3 Jn 1:10}] recevoir et les c. de l'église.
\end{listverse}

\ConcordanceEntry{Châtier}
\vspace{-2mm}
\begin{listverse}
\item[\vref{Lé 26:18}] point, je vs. c. sept fois plus
\item[\vref{De 8:5}] ton Dieu, te c. com. un hom.
\item[\vref{1 Ch 16:21}] il a mm c. des rois à
\item[\vref{2 Ch 10:14}] père vs. a c. avec des fouets ;
\item[\vref{Job 5:17}] l'hom. que Dieu c. ! Ne rejette dc
\item[\vref{Job 33:19}] L'hom. est aussi c. par des douleurs
\item[\vref{Ps 6:2}] et ne me c. pas ds ta
\item[\vref{Ps 9:6}] Tu c. les nations, tu détruis le méchant,
\item[\vref{Ps 39:12}] Aussitôt que tu c. quelqu'un, en le
\item[\vref{Pr 3:12}] Car Yahweh c. celui qu'il aime,
\item[\vref{Pr 13:24}] qui l'aime se hâte de le c.
\item[\vref{Pr 19:18}] C. ton fils tandis qu'il y a
\item[\vref{Es 26:14}] tu les as c. et exterminés, et
\item[\vref{Jé 2:19}] Ta méchanceté te c., et tes débauches
\item[\vref{Jé 30:11}] entièrement ; je te c. avec équité, je
\item[\vref{Jé 46:28}] et je te c. avec justice, je
\item[\vref{Za 10:3}] pasteurs, et je c. ces boucs ; car
\item[\vref{Lu 23:16}] Je le relâcherai dc, après l'avoir c.
\item[\vref{2 Co 6:9}] ns. vivons ; com. c. et toutefois non
\item[\vref{Hé 12:7}] le fils dont le père ne c. pas ?
\item[\vref{Ap 3:19}] reprends et je c. ts ceux que
\end{listverse}

\ConcordanceEntry{Châtiment}
\vspace{-2mm}
\begin{listverse}
\item[\vref{Ge 19:15}] périsses ds le c. de la ville.
\item[\vref{De 11:2}] vos fils, le c. de Yahweh, votre
\item[\vref{2 R 7:9}] du matin, le c. ns. atteindra. Venez
\item[\vref{2 R 19:3}] jour d'angoisse, de c. et d'opprobre ; car
\item[\vref{Job 5:17}] dc pas le c. du Tout-Puissant.
\item[\vref{Ps 73:14}] jours, et ts les matins mon c. est là.
\item[\vref{Pr 15:10}] Le c. est fâcheux à celui qui quitte
\item[\vref{Es 10:3}] au jour du c., et de la
\item[\vref{Es 53:5}] nos iniquités, le c. qui ns. apporte
\item[\vref{Jé 30:14}] plaie d'ennemi, d'un c. d'hom. cruel, à
\item[\vref{Ez 25:17}] vengeances par des c. de fureur ; et
\item[\vref{Mt 25:46}] ceux-ci iront au c. éternel, mais les
\item[\vref{2 Th 1:9}] Ils auront pour c. une ruine éternelle,
\item[\vref{Hé 12:5}] méprise pas le c. du Seign., et
\item[\vref{Hé 12:11}] Or tt c. ne semble pas sur l'heure être
\item[\vref{1 Jn 4:18}] peur suppose un c. ; or celui qui
\end{listverse}

\ConcordanceEntry{Chauve}
\vspace{-2mm}
\begin{listverse}
\item[\vref{Lé 13:40}] cheveux, c'est un c.. Il est pur.
\item[\vref{De 14:1}] point de place c. entre les yeux
\item[\vref{2 R 2:23}] lui disaient : Monte c. ! Monte chauve !
\item[\vref{Es 3:17}] Yahweh rendra c. le sommet de
\item[\vref{Ez 27:31}] rendront lr. tête c. à cause de
\item[\vref{Am 8:10}] et je rendrai c. ttes les têtes ;
\end{listverse}

\ConcordanceEntry{Chef}
\vspace{-2mm}
\begin{listverse}
\item[\vref{Ge 17:16}] nations ; des rois, c. de peuples, sortiront
\item[\vref{Ge 39:21}] grâce auprès du c. de la prison.
\item[\vref{Ge 39:22}] Et le c. de la prison mit entre les
\item[\vref{Ge 40:2}] eunuques, contre le c. des échansons, et
\item[\vref{Ge 47:6}] tu les établiras c. de ts mes
\item[\vref{No 1:4}] qui est le c. de la maison
\item[\vref{No 14:4}] l'autre : Etablissons-ns. un c., et retournons en
\item[\vref{Jg 11:6}] et sois notre c., afin que ns.
\item[\vref{1 S 22:2}] il devint lr. c.. Et il y
\item[\vref{2 S 6:21}] maison pour m'établir c. sur le peuple
\item[\vref{Né 9:17}] ils s'attribuèrent un c. pour retourner à
\item[\vref{Pr 25:15}] Le c. est fléchi par la patience, et
\item[\vref{Es 1:23}] Les c. de ton peuple sont rebelles et
\item[\vref{Es 3:4}] jeunes gens pour c., et des enfants
\item[\vref{Jé 30:21}] Et son c. sera tiré de son sein, son
\item[\vref{Da 8:11}] s'éleva mm jusqu'au c. de l'armée, lui
\item[\vref{Mt 2:6}] toi sortira le C. qui paîtra mon
\item[\vref{Mt 9:18}] voici, arriva un c. qui se prosterna
\item[\vref{Jn 7:18}] de son propre c. cherche sa propre
\item[\vref{Jn 12:42}] mm parmi les c., plusieurs crurent en
\item[\vref{1 Co 11:3}] Christ est le c. de tt hom.,
\item[\vref{1 Co 11:4}] chose sur la tête, déshonore son c.
\item[\vref{Ep 1:22}] pour être le C. de l'Eglise,
\item[\vref{Ep 4:15}] qui est le C., c'est-à-dire Christ,
\item[\vref{Ep 5:23}] mari est le c. de la fem.,
\item[\vref{Col 1:18}] qui est le C. du corps de
\item[\vref{Col 2:10}] qui est le C. de tte principauté
\item[\vref{Col 2:19}] sans s'attacher au C., dont tt le
\item[\vref{Hé 12:2}] sur Jésus, le c. et le consommateur
\end{listverse}

\ConcordanceEntry{Chemin}
\vspace{-2mm}
\begin{listverse}
\item[\vref{Ge 3:24}] pour garder le c. de l'arbre de
\item[\vref{Ex 13:17}] point par le c. du pays des
\item[\vref{Ex 13:21}] conduire par le c. ; et de nuit
\item[\vref{Ex 23:20}] garde ds le c., et qu'il t'introduise
\item[\vref{No 22:22}] plaça sur le c. pour lui résister.
\item[\vref{De 8:2}] de tt le c. par lequel Yahweh,
\item[\vref{De 11:28}] vs. détournez du c. que je vs.
\item[\vref{De 17:16}] Vous ne retournerez plus par ce c.
\item[\vref{De 28:7}] toi par un c., et ils s'enfuiront
\item[\vref{Jos 23:14}] aujourd'hui par le c. de tte la
\item[\vref{1 S 12:23}] enseignerai le bon et le droit c.
\item[\vref{2 S 22:37}] Tu élargis le c. sous mes pas,
\item[\vref{1 R 13:26}] avait ramené du c. l'hom. de Dieu,
\item[\vref{Né 9:12}] éclairer ds le c. par où ils
\item[\vref{Job 28:23}] en sait le c., et qui sait
\item[\vref{Ps 18:37}] Tu élargis le c. sous mes pas,
\item[\vref{Ps 25:12}] lui enseignera le c. qu'il doit choisir.
\item[\vref{Ps 84:6}] desquels sont les c. tt tracés !
\item[\vref{Ps 107:7}] sur le droit c. pour aller ds
\item[\vref{Ps 143:8}] Fais-moi connaître le c. par lequel je
\item[\vref{Pr 6:23}] instruire sont le c. de la vie:
\item[\vref{Pr 8:20}] marche ds le c. de la justice,
\item[\vref{Pr 10:17}] l'instruction tient le c. qui rend à
\item[\vref{Pr 16:17}] Le c. aplani des hommes droits, c'est de
\item[\vref{Es 30:21}] disant : Voici le c., marchez-y, soit que
\item[\vref{Es 35:8}] sentier et un c., qu'on appellera le
\item[\vref{Es 40:3}] est : Préparez le c. de Yahweh, aplanissez
\item[\vref{Es 57:14}] frayez, préparez le c., enlevez tt obstacle
\item[\vref{Es 59:8}] connaissent point le c. de la paix,
\item[\vref{Jé 21:8}] dvt vs. le c. de la vie
\item[\vref{Jé 50:5}] Ils s'informeront du c. de Sion vers
\item[\vref{Mal 3:1}] il préparera le c. dvt moi. Et
\item[\vref{Mt 7:14}] et resserré le c. qui mènent à
\item[\vref{Mt 13:4}] le long du c. ; et les oiseaux
\item[\vref{Mc 6:8}] prendre pour le c., si ce n'est
\item[\vref{Lu 3:5}] redressé, et les c. raboteux seront aplanis.
\item[\vref{Jn 14:4}] vais, et vs. en savez le c.
\item[\vref{Jn 14:6}] JE SUIS le c., la vérité, et
\item[\vref{Ac 8:26}] Midi, sur le c. qui descend de
\item[\vref{Ac 25:3}] des embûches pour le tuer en c.
\item[\vref{1 Th 3:11}] ns. ouvrir le c. pour ns. rendre
\item[\vref{Hé 9:8}] là que le c. du Saint des
\item[\vref{Hé 10:20}] qui est le c. nouveau et vivant
\item[\vref{2 Pi 2:15}] laissé le droit c., se sont égarés
\end{listverse}

\ConcordanceEntry{Chercher}
\vspace{-2mm}
\begin{listverse}
\item[\vref{Ge 37:16}] Joseph répondit : Je c. mes frères ; je
\item[\vref{Ex 2:15}] appris ce fait-là, c. à faire mourir
\item[\vref{Ex 4:24}] le rencontra et c. à le faire
\item[\vref{De 4:29}] de là, tu c. Yahweh, ton Dieu,
\item[\vref{Jg 14:4}] Yahweh : Car Samson c. une occasion de
\item[\vref{Ru 3:1}] fille, ne te c.-je pas du
\item[\vref{1 S 16:16}] dvt toi. Ils c. un hom. qui
\item[\vref{1 S 28:7}] à ses serviteurs : C.-moi une fem.
\item[\vref{1 Ch 15:13}] ne l'avons pas c. selon la loi.
\item[\vref{1 Ch 16:10}] de ceux qui c. Yahweh se réjouisse !
\item[\vref{1 Ch 16:11}] et sa force, c. continuellement sa face !
\item[\vref{1 Ch 28:9}] Si tu le c., il se laissera
\item[\vref{2 Ch 15:4}] d'Israël ; ils l'ont c., et ils l'ont
\item[\vref{2 Ch 15:12}] ds l'alliance pour c. Yahweh, le Dieu
\item[\vref{Né 7:64}] Ils c. lr. registre généalogique, mais ils n'y
\item[\vref{Est 3:6}] était Mardochée. Haman c. à exterminer ts
\item[\vref{Ps 7:2}] mon Dieu ! je c. en toi mon
\item[\vref{Ps 9:11}] ceux qui te c., ô Yahweh !
\item[\vref{Ps 34:5}] [Daleth.] J'ai c. Yahweh et il
\item[\vref{Ps 63:2}] Dieu, je te c. au point du
\item[\vref{Ps 78:34}] repentaient et ils c. Dieu dès le
\item[\vref{Ps 119:2}] et qui le c. de tt lr.
\item[\vref{Pr 1:28}] point ; on me c. de grand matin,
\item[\vref{Pr 8:17}] ceux qui me c. soigneusement me trouveront.
\item[\vref{Pr 15:14}] de l'hom. prudent c. la science ; mais
\item[\vref{Pr 28:5}] mais ceux qui c. Yahweh comprennent tt.
\item[\vref{Ec 3:6}] un temps pour c. et un temps
\item[\vref{Ca 3:1}] La Sulamithe :] J'ai c. pendant les nuits,
\item[\vref{Es 45:19}] postérité de Jacob : C.-moi vainement ! JE
\item[\vref{Es 55:6}] C. Yahweh pendant qu'il se trouve, invoquez-le
\item[\vref{Es 65:1}] qui ne me c. pas ; j'ai dit
\item[\vref{Jé 29:7}] Et c. la paix de la ville où
\item[\vref{Jé 29:13}] Vous me c., et vs. me trouverez, après que
\item[\vref{La 3:25}] à lui, pour l'âme qui le c.
\item[\vref{Da 9:3}] Dieu, pour le c. par la prière
\item[\vref{Os 10:12}] est temps de c. Yahweh, jusqu'à ce
\item[\vref{Mt 6:33}] Mais c. premièrement le Royaume de Dieu et
\item[\vref{Mt 7:7}] vs. sera donné ; c., et vs. trouverez ;
\item[\vref{Mt 28:5}] sais que vs. c. Jésus, qui a
\item[\vref{Mc 1:37}] trouvé, ils lui dirent : Tous te c.
\item[\vref{Lu 17:33}] Quiconque c. à sauver sa
\item[\vref{Lu 19:10}] l'hom. est venu c. et sauver ce
\item[\vref{Lu 24:5}] lr. dirent : Pourquoi c.-vs. parmi les
\item[\vref{Jn 1:38}] lr. dit : Que c.-vs. ? Ils lui
\item[\vref{Jn 5:30}] car je ne c. pas ma volonté,
\item[\vref{Jn 6:26}] dis : Vous me c., non parce que
\item[\vref{Jn 8:50}] Or je ne c. pas ma gloire ; il y en
\item[\vref{Jn 18:8}] dc vs. me c., laissez aller ceux-ci.
\item[\vref{Ac 17:27}] a voulu qu'ils c. le Seign., et
\item[\vref{1 Co 7:27}] une fem. ? Ne c. pas à rompre
\item[\vref{1 Co 10:24}] Que personne ne c. son propre intérêt,
\item[\vref{Ga 5:26}] Ne c. pas une vaine gloire en ns.
\item[\vref{Col 3:1}] ressuscités avec Christ, c. les choses qui
\item[\vref{2 Ti 4:3}] discours agréables, ils c. des docteurs qui
\item[\vref{Hé 11:6}] le rémunérateur de ceux qui le c.
\item[\vref{Hé 13:14}] permanente, mais ns. c. celle qui est
\item[\vref{1 Pi 5:8}] un lion rugissant, c. qui il pourra
\item[\vref{Ap 9:6}] jours-là, les hommes c. la mort, mais
\end{listverse}

\ConcordanceEntry{Chérubin}
\vspace{-2mm}
\begin{listverse}
\item[\vref{Ge 3:24}] jardin d'Eden des c. qui tournaient ça
\item[\vref{Ex 25:18}] tu feras deux c. d'or ; tu les
\item[\vref{Ex 26:1}] feras semés de c. d'un ouvrage exquis.
\item[\vref{Ex 37:9}] Et les c. étendaient leurs ailes en haut, couvrant
\item[\vref{1 S 4:4}] habite entre les c.. Les deux fils
\item[\vref{2 S 22:11}] monté sur un c., et il volait,
\item[\vref{1 R 6:27}] Salomon plaça les c. à l'intérieur, au
\item[\vref{2 R 19:15}] assis entre les c., c'est toi qui
\item[\vref{1 Ch 13:6}] de Yahweh, qui habite entre les c.
\item[\vref{Ps 18:11}] monté sur un c., et il volait,
\item[\vref{Ps 80:2}] assis entre les c., fais briller ta
\item[\vref{Ps 99:1}] assis entre les c. : Que la terre
\item[\vref{Ez 9:3}] d'Israël s'éleva du c. sur lequel elle
\item[\vref{Ez 28:14}] Tu étais un c., oint pour servir
\item[\vref{Hé 9:5}] l'arche étaient les c. de la gloire,
\end{listverse}

\ConcordanceEntry{Cheval}
\vspace{-2mm}
\begin{listverse}
\item[\vref{Ge 49:17}] les talons du c., pour que le
\item[\vref{Ex 14:9}] et ts les c. des chars de
\item[\vref{Ex 15:1}] la mer le c. et celui qui
\item[\vref{De 17:16}] pas de nombreux c., et il ne
\item[\vref{2 R 6:17}] était pleine de c. et de chars
\item[\vref{Est 6:8}] lui amène le c. que le roi
\item[\vref{Est 6:11}] vêt. et le c., il revêtit Mardochée,
\item[\vref{Ps 20:8}] autres de leurs c. ; mais ns., ns.
\item[\vref{Ps 32:9}] pas com. le c. ni com. le
\item[\vref{Ps 33:17}] le c. est impuissant pour sauver, et ne
\item[\vref{Pr 26:3}] est pour le c., le mors pour
\item[\vref{Ca 1:9}] beau couple de c. que j'ai aux
\item[\vref{Es 58:14}] monter com. à c. par-dessus les lieux
\item[\vref{Es 66:20}] nations, sur des c., sur des chars
\item[\vref{Jé 12:5}] lutteras-tu avec les c. ? Et si tu
\item[\vref{Za 6:2}] y avait des c. roux ; au deuxième
\item[\vref{Za 9:10}] de Jérus. les c. ; et les arcs
\item[\vref{Ap 6:2}] je vis un c. blanc ; celui qui
\item[\vref{Ap 9:19}] le pouvoir des c. était ds leurs
\item[\vref{Ap 19:14}] suivaient sur des c. blancs, revêtues de
\end{listverse}

\ConcordanceEntry{Cheveu}
\vspace{-2mm}
\begin{listverse}
\item[\vref{Ge 44:29}] ferez descendre mes c. blancs avec douleur
\item[\vref{Lé 13:40}] tête dépouillée de c., c'est un chauve.
\item[\vref{No 6:5}] laissera croître les c. de sa tête.
\item[\vref{Jg 16:22}] Les c. de sa tête commencèrent à repousser,
\item[\vref{1 S 14:45}] un seul des c. de sa tête,
\item[\vref{Pr 16:31}] Les c. blancs sont une couronne d'honneur ; elle
\item[\vref{Pr 20:29}] gloire, et les c. blancs sont l'honneur
\item[\vref{Ez 8:3}] prit par les c. de ma tête.
\item[\vref{Da 3:27}] corps, et aucun c. de lr. tête
\item[\vref{Da 7:9}] neige, et les c. de sa tête
\item[\vref{Mt 5:36}] rendre blanc ou noir un seul c.
\item[\vref{Mt 10:30}] Et mm les c. de votre tête
\item[\vref{Lu 7:38}] avec ses propres c., et lui baisa
\item[\vref{Lu 21:18}] perdra pas un c. de votre tête.
\item[\vref{1 Co 11:14}] honte pour l'hom. d'avoir de longs c.,
\item[\vref{1 Co 11:15}] porter des longs c., parce que la
\item[\vref{1 Pi 3:3}] consiste ds les c. tressés, les ornements
\item[\vref{Ap 1:14}] tête et ses c. étaient blancs com.
\item[\vref{Ap 9:8}] Elles avaient les c. com. des cheveux
\end{listverse}

\ConcordanceEntry{Chèvre}
\vspace{-2mm}
\begin{listverse}
\item[\vref{Ge 15:9}] trois ans, une c. de trois ans,
\item[\vref{Ge 31:38}] brebis et tes c. n'ont point avorté,
\item[\vref{Ex 26:7}] de poils de c. pour servir de
\item[\vref{Lé 3:12}] offrande est une c., il l'offrira dvt
\item[\vref{De 14:4}] Le bœuf, la brebis et la c. ;
\item[\vref{1 S 19:13}] une peau de c. à son chevet
\item[\vref{Pr 27:27}] du lait des c. sera pour ta
\item[\vref{Da 8:5}] bouc d'entre les c. venait de l'occident,
\end{listverse}

\ConcordanceEntry{Chien}
\vspace{-2mm}
\begin{listverse}
\item[\vref{Jg 7:5}] com. lape le c., tu les sépareras
\item[\vref{1 S 17:43}] David : Suis-je un c., pour que tu
\item[\vref{2 S 9:8}] tu regardes un c. mort tel que
\item[\vref{1 R 21:19}] Yahweh : Comme les c. ont léché le
\item[\vref{1 R 21:23}] en disant : Les c. mangeront Jézabel près
\item[\vref{Ps 22:21}] l'épée, mon unique du pouvoir des c. !
\item[\vref{Pr 26:11}] Comme le c. retourne à ce qu'il a vomi,
\item[\vref{Pr 26:17}] qui prend un c. par les oreilles.
\item[\vref{Ec 9:4}] et mm un c. vivant vaut mieux
\item[\vref{Es 56:11}] Ce sont des c. voraces et insatiables ;
\item[\vref{Mt 7:6}] choses saintes aux c. et ne jetez
\item[\vref{Mt 15:27}] Cependant les petits c. mangent des miettes
\item[\vref{Lu 16:21}] et mm les c. venaient encore lécher
\item[\vref{Ph 3:2}] Prenez garde aux c., prenez garde aux
\item[\vref{2 Pi 2:22}] est arrivé : Le c. est retourné à
\item[\vref{Ap 22:15}] laissés dehors les c., les empoisonneurs, les
\end{listverse}

\ConcordanceEntry{Choisir}
\vspace{-2mm}
\begin{listverse}
\item[\vref{Ge 13:11}] Et Lot c. pour lui tte la plaine du
\item[\vref{Ex 17:9}] dit à Josué : C.-ns. des hommes,
\item[\vref{No 17:5}] l'hom. que j'aurai c. fleurira ; et je
\item[\vref{De 7:6}] ton Dieu, t'a c. pour que tu
\item[\vref{De 12:11}] Yahweh, votre Dieu, c. pour y faire
\item[\vref{De 30:19}] et la malédiction. C. dc la vie,
\item[\vref{Jos 24:15}] de servir Yahweh, c. aujourd'hui qui vs.
\item[\vref{1 S 12:13}] que vs. avez c., que vs. avez
\item[\vref{1 R 8:16}] d'Egypte, je n'ai c. aucune ville d'entre
\item[\vref{Ps 4:4}] que Yahweh s'est c. un bien-aimé. Yahweh
\item[\vref{Ps 25:12}] lui enseignera le chemin qu'il doit c.
\item[\vref{Ps 65:5}] celui que tu c. et que tu
\item[\vref{Ps 78:70}] Il c. David, son serviteur, et le prit
\item[\vref{Ps 132:13}] Car Yahweh a c. Sion, il l'a
\item[\vref{Es 7:15}] qu'il sache rejeter le mal et c. le bien.
\item[\vref{Es 14:1}] de Jacob, il c. encore Israël, et
\item[\vref{Es 44:1}] serviteur, et toi Israël que j'ai c. !
\item[\vref{Lu 10:42}] et Marie a c. la bonne part,
\item[\vref{Jn 6:70}] vs. ai-je pas c., vs. les douze ?
\item[\vref{Jn 15:16}] vs. qui m'avez c. ; mais moi, je
\item[\vref{Ac 15:14}] les nations pour c. du milieu d'elles
\item[\vref{Ga 1:15}] Dieu, qui m'avait c. dès le ventre
\item[\vref{1 Pi 2:4}] les hommes, mais c. et précieuse dvt
\end{listverse}

\ConcordanceEntry{Chorazin}
\vspace{-2mm}
\begin{listverse}
\item[\vref{Mt 11:21}] Malheur à toi, C. ! malheur à toi,
\item[\vref{Lu 10:13}] Malheur à toi, C. ! Malheur à toi,
\end{listverse}

\ConcordanceEntry{Chose}
\vspace{-2mm}
\begin{listverse}
\item[\vref{Ge 41:32}] c'est que la c. est arrêtée de
\item[\vref{Lé 22:20}] Vous n'offrirez aucune c. qui ait un
\item[\vref{De 4:9}] n'oublies point les c. que tes yeux
\item[\vref{De 29:29}] Les c. cachées sont à Yahweh, notre Dieu ;
\item[\vref{1 Ch 21:7}] cette c. déplut à Dieu, c'est pourquoi il
\item[\vref{Job 9:10}] fait de grandes c. qu'on ne peut
\item[\vref{Job 39:35}] lui apprendra-t-il qq c. ? Que celui qui
\item[\vref{Ps 33:9}] dit, et la c. arrive ; il ordonne,
\item[\vref{Ps 87:3}] Dieu, sont des c. glorieuses ! Sélah.
\item[\vref{Ps 92:2}] C'est une belle c. que de célébrer
\item[\vref{Ps 126:3}] fait de grandes c. pour ns. ; ns.
\item[\vref{Ps 131:1}] m'occupe pas de c. trop grandes et
\item[\vref{Ps 133:1}] Que c'est une c. bonne, et que
\item[\vref{Pr 4:7}] La principale c., c'est la sagesse ;
\item[\vref{Pr 16:4}] a fait ttes c. pour lui-mm ; et
\item[\vref{Pr 25:2}] de cacher les c., et la gloire
\item[\vref{Ec 7:8}] la fin d'une c. que son commencement.
\item[\vref{Es 16:14}] sera peu de c., ce ne sera
\item[\vref{Es 40:26}] a créé ces c. ? C'est lui qui
\item[\vref{Es 43:9}] a annoncé ces c.-là ? Et qui
\item[\vref{Es 43:19}] vais faire une c. nouvelle, qui paraîtra
\item[\vref{Es 64:3}] fît de telles c. pour ceux qui
\item[\vref{Jé 32:27}] Y a-t-il qq c. d'étonnant de ma
\item[\vref{Jé 33:3}] je t'annoncerai des c. grandes, des choses
\item[\vref{Mt 7:11}] enfants de bonnes c., à combien plus
\item[\vref{Mt 9:33}] disaient : Jamais pareille c. ne s'est vue
\item[\vref{Mt 11:27}] Toutes c. m'ont été données par mon Père,
\item[\vref{Mt 13:52}] son trésor des c. nouvelles et des
\item[\vref{Mt 19:26}] à Dieu ttes c. sont possibles.
\item[\vref{Mt 23:23}] vs. laissez les c. les plus importantes
\item[\vref{Mt 25:21}] en peu de c., je t'établirai sur
\item[\vref{Mc 10:21}] te manque une c. : Va et vends
\item[\vref{Lu 5:26}] vu aujourd'hui des c. étranges.
\item[\vref{Lu 10:42}] Mais une c. est nécessaire ; et Marie a choisi
\item[\vref{Lu 24:48}] Et vs. êtes témoins de ces c.
\item[\vref{Jn 1:3}] Toutes c. ont été faites par elle, et
\item[\vref{Jn 1:50}] tu verras des c. plus grandes encore.
\item[\vref{Jn 3:27}] peut recevoir aucune c. si elle ne
\item[\vref{Jn 3:35}] a remis ttes c. entre ses mains.
\item[\vref{Jn 8:26}] J'ai beaucoup de c. à dire de
\item[\vref{Jn 14:25}] ai dit ces c. pendant que je
\item[\vref{Jn 19:28}] sachant que ttes c. étaient déjà accomplies,
\item[\vref{Ac 2:37}] avoir entendu ces c., ils eurent le
\item[\vref{Ro 4:17}] qui appelle les c. qui ne sont
\item[\vref{Ro 8:28}] aussi que ttes c. concourent au bien
\item[\vref{Ro 8:32}] pas aussi ttes c. avec lui ?
\item[\vref{1 Co 1:27}] a choisi les c. folles de ce
\item[\vref{1 Co 2:9}] Ce sont des c. que l'œil n'a
\item[\vref{1 Co 2:11}] effet, connaît les c. de l'hom., sinon
\item[\vref{1 Co 2:15}] spirituel discerne ttes c. et il n'est
\item[\vref{2 Co 4:18}] regardons pas aux c. visibles, mais aux
\item[\vref{2 Co 5:17}] nouvelle créature ; les c. anciennes sont passées ;
\item[\vref{Ga 3:12}] qui mettra ces c. en pratique vivra
\item[\vref{Ep 3:9}] a créé ttes c. par Jésus-Christ,
\item[\vref{Ph 1:10}] le discernement des c. contraires, afin que
\item[\vref{Ph 3:14}] je fais une c. : Oubliant les choses
\item[\vref{Ph 4:13}] Je puis ttes c. en Christ qui
\item[\vref{Col 1:17}] est avant ttes c., et ttes choses
\item[\vref{Col 2:17}] sont l'ombre des c. qui devaient venir,
\item[\vref{Col 3:2}] Pensez aux c. d'en haut, et
\item[\vref{1 Th 5:21}] Eprouvez ttes c. ; retenez ce qui
\item[\vref{Tit 2:15}] Enseigne ces c., exhorte, et reprends
\item[\vref{Hé 3:4}] a construit ttes c., c'est Dieu.
\item[\vref{Hé 6:9}] bien-aimés, de meilleures c., et convenables au
\item[\vref{Hé 11:1}] rend présentes les c. qu'on espère, et
\item[\vref{1 Pi 1:18}] non  par des c. corruptibles, com. l'argent
\item[\vref{1 Jn 2:15}] monde, ni les c. qui sont ds
\item[\vref{Ap 1:1}] ses serviteurs les c. qui doivent arriver
\item[\vref{Ap 2:4}] Mais j'ai qq c. contre toi, c'est
\item[\vref{Ap 21:5}] je fais ttes c. nouvelles. Puis il
\end{listverse}

\ConcordanceEntry{Chrétien}
\vspace{-2mm}
\begin{listverse}
\item[\vref{Ac 11:26}] première fois, les disciples furent appelés c.
\item[\vref{Ac 26:28}] vas bientôt me persuader de devenir c. !
\item[\vref{Ro 14:4}] Et mm ce c. faible sera affermi,
\item[\vref{1 Pi 4:16}] quelqu'un souffre com. c., qu'il n'en ait
\end{listverse}

\ConcordanceEntry{Chrysolithe}
\vspace{-2mm}
\begin{listverse}
\item[\vref{Ex 28:20}] quatrième rangée, un c., un onyx et
\item[\vref{Ca 5:14}] y a des c. enchâssées ; son ventre
\item[\vref{Ez 1:16}] la couleur d'un c., et ttes les
\item[\vref{Ez 10:9}] roues avaient l'aspect d'une pierre de c.
\item[\vref{Ez 28:13}] de diamant, de c., d'onyx, de jaspe,
\item[\vref{Da 10:6}] était com. de c., et son visage
\item[\vref{Ap 21:20}] le septième de c., le huitième de
\end{listverse}

\ConcordanceEntry{Chute}
\vspace{-2mm}
\begin{listverse}
\item[\vref{2 Ch 28:23}] cause de sa c. et de celle
\item[\vref{Ps 56:14}] pieds de la c., afin que je
\item[\vref{Ps 64:9}] a causé lr. c. ; ts ceux qui
\item[\vref{Ps 116:8}] larmes et mes pieds de la c.
\item[\vref{Jé 49:21}] bruit de lr. c. ; le bruit de
\item[\vref{Mt 5:29}] une occasion de c., arrache-le et jette-le
\item[\vref{Mt 13:57}] une occasion de c.. Mais Jésus lr.
\item[\vref{Mt 18:8}] une occasion de c., coupe-les et jette-les
\item[\vref{Mc 9:43}] une occasion de c., coupe-la ; mieux vaut
\item[\vref{Lu 2:34}] établi pour la c. et pour la
\item[\vref{Ro 11:9}] une occasion de c., et cela pour
\item[\vref{Ro 11:11}] Mais, par lr. c., le salut est
\item[\vref{Ro 14:21}] une occasion de c., ou de scandale
\item[\vref{Jud 1:24}] vs. fassiez aucune c. et vs. faire
\end{listverse}

\ConcordanceEntry{Chypre}
\vspace{-2mm}
\begin{listverse}
\item[\vref{Ac 4:36}] fils de consolation, Lévite, originaire de C.,
\item[\vref{Ac 11:19}] ds l'île de C., et à Antioche,
\item[\vref{Ac 13:4}] là ils s'embarquèrent pour l'île de C.
\item[\vref{Ac 15:39}] avec lui, s'embarqua pour l'île de C.
\item[\vref{Ac 27:4}] longeâmes l'île de C., parce que les
\end{listverse}

\ConcordanceEntry{Ciel}
\vspace{-2mm}
\begin{listverse}
\item[\vref{Ge 1:1}] Dieu créa les c. et la terre.
\item[\vref{Ge 1:8}] Dieu appela l'étendue c.. Ainsi fut le
\item[\vref{Ge 8:2}] les écluses des c. furent fermées et
\item[\vref{Ge 15:5}] les yeux au c. et compte les
\item[\vref{Ex 16:4}] faire pleuvoir des c. du pain, et
\item[\vref{Ex 20:11}] a fait les c., la terre, la
\item[\vref{Lé 26:19}] ferai que votre c. soit pour vs.
\item[\vref{De 4:39}] haut ds les c. et sur la
\item[\vref{De 32:1}] C. ! Prêtez l'oreille, et je parlerai. Terre !
\item[\vref{1 R 8:27}] terre ? Voilà, les c., mm les cieux
\item[\vref{2 R 1:10}] feu descende du c. et te consume,
\item[\vref{2 R 2:1}] enleva Elie au c. ds un tourbillon,
\item[\vref{Job 11:8}] les hauteurs des c. : Qu'y feras-tu ? C'est
\item[\vref{Job 15:15}] saints et les c. ne sont pas
\item[\vref{Job 41:2}] sous ts les c. est à moi.
\item[\vref{Ps 2:4}] habite ds les c. se rit d'eux,
\item[\vref{Ps 11:4}] trône ds les c. ; ses yeux contemplent,
\item[\vref{Ps 14:2}] Yahweh regarde des c. les fils de
\item[\vref{Ps 19:2}] Les c. racontent la gloire de Dieu, et
\item[\vref{Ps 33:6}] Les c. ont été faits par la parole
\item[\vref{Ps 50:6}] Les c. aussi annonceront sa justice parce que
\item[\vref{Ps 68:9}] trembla et les c. répandirent leurs eaux
\item[\vref{Ps 73:25}] autre ai-je au c. ? Or sur la
\item[\vref{Ps 89:6}] Les c. célèbrent tes merveilles, ô Yahweh ! Ta
\item[\vref{Ps 89:7}] qui, ds le c., peut se comparer
\item[\vref{Ps 102:26}] terre, et les c. sont l'ouvrage de
\item[\vref{Ps 115:3}] Dieu est au c., il fait tt
\item[\vref{Pr 8:27}] il disposa les c., j'étais là ; lorsqu'il
\item[\vref{Ec 5:1}] Dieu est au c., et toi sur
\item[\vref{Es 14:12}] es-tu tombé du c., astre brillant, fils
\item[\vref{Es 49:13}] Ô c., réjouissez-vs. avec des chants de triomphe !
\item[\vref{Es 55:9}] Mais autant les c. sont élevés au-dessus
\item[\vref{Es 65:17}] créer de nouveaux c. et une nouvelle
\item[\vref{Es 66:1}] parle Yahweh : Le c. est mon trône,
\item[\vref{Jé 10:12}] a étendu les c. par son intelligence.
\item[\vref{Mt 3:16}] Et voici, les c. lui furent ouverts,
\item[\vref{Mt 5:18}] tant que le c. et la terre
\item[\vref{Mt 6:20}] trésors ds le c., où les vers
\item[\vref{Mt 11:25}] Père ! Seign. du c. et de la
\item[\vref{Mt 16:3}] aujourd'hui, car le c. est d'un rouge
\item[\vref{Mt 18:18}] lié ds le c. ; et tt ce
\item[\vref{Mt 24:30}] paraîtra ds le c., ttes les tribus
\item[\vref{Mt 24:35}] Le c. et la terre passeront, mais mes
\item[\vref{Mc 13:31}] Le c. et la terre passeront, mais mes
\item[\vref{Mc 16:19}] fut enlevé au c., et il s'assit
\item[\vref{Lu 3:22}] fit entendre du c. ces paroles : Tu
\item[\vref{Lu 10:21}] Père, Seign. du c. et de la
\item[\vref{Jn 1:51}] vs. verrez le c. ouvert, et les
\item[\vref{Jn 3:13}] n'est monté au c., si ce n'est
\item[\vref{Jn 3:31}] est venu du c. est au-dessus de
\item[\vref{Jn 6:38}] suis descendu du c., non pas pour
\item[\vref{Ac 3:21}] faut que le c. reçoive, jusqu'au temps
\item[\vref{Ac 4:12}] a sous le c. aucun autre Nom
\item[\vref{Ro 1:18}] révèle pleinement du c. contre tte impiété
\item[\vref{Ro 10:6}] Qui montera au c. ? C'est en faire
\item[\vref{2 Co 12:2}] qui a été ravi jusqu'au troisième c.
\item[\vref{2 Th 1:7}] se révélera du c. avec les anges
\item[\vref{Hé 9:24}] entré ds le c. mm, afin de
\item[\vref{2 Pi 3:10}] ce jour-là, les c. passeront avec le
\item[\vref{Ap 5:3}] ni ds le c., ni sur la
\item[\vref{Ap 6:14}] Et le c. se retira com. un livre qu'on
\item[\vref{Ap 12:7}] guerre ds le c.. Michel et ses
\item[\vref{Ap 19:11}] je vis le c. ouvert, et voici
\item[\vref{Ap 20:11}] terre et le c. s'enfuirent dvt sa
\item[\vref{Ap 21:1}] vis un nouveau c. et une nouvelle
\item[\vref{Ap 21:2}] qui descendait du c., d'auprès de Dieu,
\end{listverse}

\ConcordanceEntry{Cigogne}
\vspace{-2mm}
\begin{listverse}
\item[\vref{Lé 11:19}] la c., le héron selon lr. espèce, la
\item[\vref{Ps 104:17}] Quant à la c., les sapins sont
\item[\vref{Jé 8:7}] Même la c. connaît ds les cieux ses saisons ;
\item[\vref{Za 5:9}] ailes de la c.. Et elles enlevèrent
\end{listverse}

\ConcordanceEntry{Cilicie}
\vspace{-2mm}
\begin{listverse}
\item[\vref{Ac 6:9}] avec ceux de C. et d'Asie, se
\item[\vref{Ac 15:23}] à Antioche, en Syrie, et en C., salut !
\item[\vref{Ac 15:41}] Syrie et la C., fortifiant les églises.
\item[\vref{Ac 21:39}] renommée de la C.. Permets-moi, je te
\item[\vref{Ac 27:5}] la Mer de C. et de Pamphylie,
\end{listverse}

\ConcordanceEntry{Circoncire}
\vspace{-2mm}
\begin{listverse}
\item[\vref{Ge 17:10}] garderez : Tout mâle parmi vs. sera c.
\item[\vref{Ge 17:12}] huit jours sera c. parmi vs. ds
\item[\vref{De 10:16}] C. dc le prépuce de votre cœur,
\item[\vref{De 30:6}] Yahweh, ton Dieu, c. ton cœur, et
\item[\vref{Jos 5:2}] pierre tranchants, et c. de nouveau les
\item[\vref{Jé 9:25}] je punirai tt c. ayant le prépuce,
\item[\vref{Lu 1:59}] ils vinrent pour c. le petit enfant,
\item[\vref{Jn 7:22}] des pères, vs. c. bien un hom.
\item[\vref{Ac 15:1}] vs. n'êtes pas c. selon le rite
\item[\vref{Ac 16:3}] pris, il le c., à cause des
\item[\vref{Ro 3:30}] la foi les c., et aussi les
\item[\vref{Ro 4:10}] n'était pas encore c., mais il était
\item[\vref{1 Co 7:18}] été appelé étant c. ? Qu'il ne ramène
\item[\vref{Ga 2:3}] de se faire c. quoiqu'il fût Grec.
\item[\vref{Ga 5:3}] qui se fait c. qu'il est tenu
\item[\vref{Ep 2:11}] ceux qu'on appelle c., et qui le
\item[\vref{Ph 3:2}] mauvais ouvriers, prenez garde aux faux c.
\item[\vref{Ph 3:3}] qui sommes les c., ns. qui rendons
\item[\vref{Col 3:11}] ni Juif, ni c. ni incirconcis, ni
\end{listverse}

\ConcordanceEntry{Circoncision}
\vspace{-2mm}
\begin{listverse}
\item[\vref{Ex 4:26}] de sang. A cause de la c.
\item[\vref{Jn 7:22}] a donné la c., non qu'elle vienne
\item[\vref{Ac 7:8}] l'alliance de la c. ; et après cela
\item[\vref{Ro 2:25}] vrai que la c. est profitable, si
\item[\vref{Ro 2:28}] apparences ; et la c., ce n'est pas
\item[\vref{Ro 2:29}] intérieurement ; et la c., c'est celle du
\item[\vref{Ro 3:1}] ou quelle est l'utilité de la c. ?
\item[\vref{Ro 4:11}] signe de la c., com. sceau de
\item[\vref{1 Co 7:19}] La c. n'est rien, et le prépuce aussi
\item[\vref{Ga 5:6}] Jésus-Christ ni la c. ni le prépuce
\item[\vref{Ga 6:15}] Jésus-Christ ni la c., ni l'incirconcision n'ont
\item[\vref{Col 2:11}] êtes circoncis d'une c. faite sans main,
\item[\vref{Tit 1:10}] principalement ceux qui sont de la c.,
\end{listverse}

\ConcordanceEntry{Cité}
\vspace{-2mm}
\begin{listverse}
\item[\vref{2 S 5:7}] Sion : C'est la c. de David.
\item[\vref{2 S 6:12}] d'Obed-Edom jusqu'à la c. de David, au
\item[\vref{2 S 6:16}] entrait ds la c. de David, Mical,
\item[\vref{Ps 46:5}] ruisseaux réjouissent la c. de Dieu, qui
\item[\vref{Ps 87:3}] dit de toi, c. de Dieu, sont
\item[\vref{Ps 101:8}] d'exterminer de la c. de Yahweh ts
\item[\vref{Es 1:26}] cela, on t'appellera c. de la justice,
\item[\vref{Es 25:2}] et de la c. forte une ruine ;
\item[\vref{Es 48:2}] de la sainte c., et ils s'appuient
\item[\vref{Ep 2:12}] du droit de c. en Israël, étant
\item[\vref{Ph 3:20}] pour ns., notre c. est ds les
\item[\vref{Hé 11:10}] il attendait la c. qui a des
\item[\vref{Hé 12:22}] Sion, de la C. du Dieu vivant,
\item[\vref{Hé 13:14}] pas ici-bas de c. permanente, mais ns.
\item[\vref{Ap 3:12}] nom de la c. de mon Dieu,
\end{listverse}

\ConcordanceEntry{Citerne}
\vspace{-2mm}
\begin{listverse}
\item[\vref{Ge 37:20}] l'une de ces c. ; et ns. dirons
\item[\vref{1 S 13:6}] ds les tours et ds des c.
\item[\vref{1 S 19:22}] à la grande c. qui est à
\item[\vref{2 S 23:15}] l'eau de la c. qui est à
\item[\vref{2 R 18:31}] chacun boira de l'eau de sa c.,
\item[\vref{Pr 5:15}] eaux de ta c. et de ce
\item[\vref{Ec 12:8}] que la roue s'écrase sur la c. ;
\item[\vref{Es 30:14}] pour puiser de l'eau à la c.
\item[\vref{Es 36:16}] boirez chacun de l'eau de sa c.,
\item[\vref{Es 51:1}] creux de la c. dont vs. avez
\item[\vref{Jé 2:13}] se creuser des c., des citernes crevassées
\item[\vref{Jé 14:3}] petits vont aux c., ne trouvent pas
\end{listverse}

\ConcordanceEntry{Citoyen}
\vspace{-2mm}
\begin{listverse}
\item[\vref{Ac 21:39}] Juif, de Tarse, c. de la ville
\item[\vref{Ac 22:28}] ce droit de c. pour une grande
\end{listverse}

\ConcordanceEntry{Clef}
\vspace{-2mm}
\begin{listverse}
\item[\vref{Jg 3:25}] ils prirent la c. et ouvrirent ; et
\item[\vref{Es 22:22}] je mettrai la c. de la maison
\item[\vref{Mt 16:19}] te donnerai les c. du Royaume des
\item[\vref{Lu 11:52}] avez enlevé la c. de la science.
\item[\vref{Ap 1:18}] je tiens les c. de Hadès et
\item[\vref{Ap 3:7}] qui a la c. de David, qui
\item[\vref{Ap 9:1}] terre, et la c. du puits de
\item[\vref{Ap 20:1}] qui avait la c. de l'abîme et
\end{listverse}

\ConcordanceEntry{Clémence}
\vspace{-2mm}
\begin{listverse}
\item[\vref{Es 16:5}] s'affermira par la c. ; et sur ce
\item[\vref{1 Ti 1:16}] premier, tte sa c., pour que je
\end{listverse}

\ConcordanceEntry{Clou}
\vspace{-2mm}
\begin{listverse}
\item[\vref{1 Ch 22:3}] d'en faire des c. pour les battants
\item[\vref{2 Ch 3:9}] le poids des c. montait à cinquante
\item[\vref{Esd 9:8}] a donné un c. ds son saint
\item[\vref{Ec 12:13}] sont com. des c. plantés, et ces
\item[\vref{Es 22:23}] l'enfoncerai com. un c. ds un lieu
\item[\vref{Es 41:7}] l'idole avec des c., afin qu'elle ne
\item[\vref{Jé 10:4}] tenir avec des c. et à coups
\item[\vref{Ez 15:3}] Ou prendra-t-on un c. pour y pendre
\item[\vref{Za 10:4}] lui sortira le c., de lui sortira
\item[\vref{Jn 20:25}] les marques des c. ds ses mains,
\end{listverse}

\ConcordanceEntry{Cœur}
\vspace{-2mm}
\begin{listverse}
\item[\vref{Ge 6:5}] pensées de lr. c. n'était que mal
\item[\vref{Ge 6:6}] et il fut affligé en son c.
\item[\vref{Ge 50:21}] les consola en parlant à lr. c.
\item[\vref{Ex 7:3}] J'endurcirai le c. de Pharaon, et
\item[\vref{Ex 28:29}] portera sur son c. les noms des
\item[\vref{Lé 19:17}] frère ds ton c.. Tu reprendras soigneusement
\item[\vref{De 5:29}] toujours ce mm c. pour me craindre
\item[\vref{De 6:5}] de tt ton c., de tte ton
\item[\vref{De 8:2}] était ds ton c., et si tu
\item[\vref{De 11:18}] dc ds votre c. et ds votre
\item[\vref{De 30:2}] de tt ton c., de tte ton
\item[\vref{De 30:6}] Dieu, circoncira ton c., et le cœur
\item[\vref{De 32:46}] dit : Appliquez votre c. à ttes ces
\item[\vref{Jos 2:11}] entendu, et notre c. a fondu, et
\item[\vref{Jg 16:17}] ouvrit tt son c., et lui dit :
\item[\vref{Ru 2:13}] parlé selon le c. de ta servante,
\item[\vref{1 S 1:13}] parlait ds son c., elle ne faisait
\item[\vref{1 S 2:1}] et dit : Mon c. se réjouit en
\item[\vref{1 S 2:35}] agira selon mon c., et selon mon
\item[\vref{1 S 10:9}] Dieu changea son c., et ts ces
\item[\vref{1 S 13:14}] hom. selon son c., et Yahweh l'a
\item[\vref{1 S 14:7}] as ds le c., vas-y, voici je
\item[\vref{1 S 16:7}] ses yeux, mais Yahweh regarde au c.
\item[\vref{2 S 6:16}] Yahweh, elle le méprisa en son c.
\item[\vref{1 R 3:9}] ton serviteur un c. intelligent pour juger
\item[\vref{1 R 8:61}] Que votre c. soit intègre envers
\item[\vref{1 R 9:3}] yeux et mon c. seront toujours là.
\item[\vref{1 R 11:4}] firent détourner son c. vers d'autres dieux ;
\item[\vref{2 R 5:26}] lui dit : Mon c. non plus n'était
\item[\vref{2 R 22:19}] Parce que ton c. a été touché,
\item[\vref{2 R 23:25}] de tt son c., de tte son
\item[\vref{1 Ch 12:33}] et prêts à livrer bataille d'un c. assuré.
\item[\vref{1 Ch 16:10}] Nom ! Que le c. de ceux qui
\item[\vref{1 Ch 17:2}] as ds le c., car Dieu est
\item[\vref{1 Ch 28:9}] sers-le avec un c. droit et une
\item[\vref{2 Ch 16:9}] ceux dont le c. est tt entier
\item[\vref{2 Ch 17:6}] Son c. grandit ds les voies de Yahweh,
\item[\vref{2 Ch 22:9}] de tt son c.. Et il n'y
\item[\vref{2 Ch 26:16}] fut puissant, son c. s'éleva pour le
\item[\vref{2 Ch 29:10}] dc j'ai à c. de traiter alliance
\item[\vref{2 Ch 32:25}] reçu ; car son c. s'éleva, et il
\item[\vref{Né 2:12}] mis ds mon c. de faire pour
\item[\vref{Né 4:6}] peuple avait le c. au travail.
\item[\vref{Job 9:4}] est sage de c., et puissant en
\item[\vref{Job 22:22}] et mets ses paroles ds ton c.
\item[\vref{Ps 9:2}] de tt mon c. Yahweh, je raconterai
\item[\vref{Ps 16:9}] C'est pourquoi mon c. se réjouit, mon
\item[\vref{Ps 17:3}] as sondé mon c., tu l'as visité
\item[\vref{Ps 20:5}] ce que ton c. désire, et qu'il
\item[\vref{Ps 33:15}] forme également lr. c. et qui prend
\item[\vref{Ps 51:12}] en moi un c. pur, et renouvelle
\item[\vref{Ps 51:19}] méprises point un c. brisé et contrit.
\item[\vref{Ps 57:8}] Mon c. est affermi, ô Dieu ! Mon cœur
\item[\vref{Ps 62:9}] temps, déchargez votre c. sur lui ! Dieu
\item[\vref{Ps 64:7}] l'hom., et au c. le plus profond.
\item[\vref{Ps 73:21}] Quand mon c. s'aigrissait et que
\item[\vref{Ps 78:37}] car lr. c. n'était point droit envers lui, et
\item[\vref{Ps 86:11}] vérité ; lie mon c. à la crainte
\item[\vref{Ps 95:10}] peuple dont le c. s'égare ; et ils
\item[\vref{Ps 119:11}] parole ds mon c. afin de ne
\item[\vref{Ps 119:36}] Incline mon c. à tes préceptes
\item[\vref{Ps 131:1}] n'ai ni un c. qui s'élève ni
\item[\vref{Pr 2:2}] tu inclines ton c. à l'intelligence ;
\item[\vref{Pr 3:5}] de tt ton c. en Yahweh et
\item[\vref{Pr 4:23}] Garde ton c. de tt ce
\item[\vref{Pr 14:10}] Le c. de chacun connaît l'amertume de son
\item[\vref{Pr 14:30}] Un c. sain est la vie de la
\item[\vref{Pr 15:13}] Le c. joyeux rend le visage beau, mais
\item[\vref{Pr 16:9}] Le c. de l'hom. médite sur sa voie,
\item[\vref{Pr 17:22}] Le c. joyeux est un remède, mais l'esprit
\item[\vref{Pr 21:1}] Le c. du roi est ds la main
\item[\vref{Pr 23:12}] Applique ton c. à l'instruction, et
\item[\vref{Pr 23:15}] fils, si ton c. est sage, mon
\item[\vref{Pr 23:26}] fils, donne-moi ton c., et que tes
\item[\vref{Pr 27:11}] et réjouis mon c., afin que j'aie
\item[\vref{Pr 27:19}] visage, ainsi le c. de l'hom. répond
\item[\vref{Ec 1:13}] j'ai appliqué mon c. à rechercher et
\item[\vref{Ec 3:11}] l'éternité ds lr. c., sans toutefois que
\item[\vref{Ec 8:5}] mal ; et le c. du sage discerne
\item[\vref{Ca 4:9}] me ravis le c., ma sœur, mon
\item[\vref{Es 29:13}] mais que son c. est éloigné de
\item[\vref{Es 35:4}] qui ont le c. troublé : Prenez courage
\item[\vref{Es 61:1}] qui ont le c. brisé, pour proclamer
\item[\vref{Jé 4:14}] Jérus., lave ton c. du mal afin
\item[\vref{Jé 5:23}] peuple-ci a un c. indocile et rebelle ;
\item[\vref{Jé 11:20}] reins et le c., fais que je
\item[\vref{Jé 17:9}] Le c. est rusé et désespérément malin par-dessus
\item[\vref{Jé 17:10}] qui sonde le c., et qui éprouve
\item[\vref{Jé 20:9}] a ds mon c. com. un feu
\item[\vref{Jé 24:7}] lr. donnerai un c. pour me connaître,
\item[\vref{Jé 31:33}] l'écrirai ds lr. c. ; et je serai
\item[\vref{Jé 32:39}] donnerai un mm c. et une mm
\item[\vref{Ez 11:19}] donnerai un mm c., et je mettrai
\item[\vref{Ez 28:5}] commerce, puis ton c. s'est élevé à
\item[\vref{Ez 36:26}] donnerai un nouveau c., je mettrai au-dedans
\item[\vref{Da 4:16}] Que son c. d'hom. soit changé, et qu'un cœur
\item[\vref{Da 5:20}] Mais lorsque son c. s'éleva et que
\item[\vref{Da 7:4}] hom., et un c. d'hom. lui fut
\item[\vref{Da 10:12}] as appliqué ton c. à comprendre, et
\item[\vref{Os 2:16}] désert, et je parlerai à son c.
\item[\vref{Joë 2:13}] Déchirez vos c. et non vos
\item[\vref{Mal 4:6}] Il ramènera le c. des pères à
\item[\vref{Mt 5:8}] sont purs de c., car ils verront
\item[\vref{Mt 12:34}] de l'abondance du c. que la bouche
\item[\vref{Mt 15:19}] Car c'est du c. que sortent les
\item[\vref{Mc 7:21}] dedans, c'est-à-dire du c. des hommes, que
\item[\vref{Lu 8:15}] retiennent ds un c. honnête et bon,
\item[\vref{Lu 12:34}] votre trésor, là aussi sera votre c.
\item[\vref{Lu 24:25}] et dont le c. est lent à
\item[\vref{Lu 24:32}] à l'autre : Notre c. ne brûlait-il pas
\item[\vref{Jn 13:2}] mis ds le c. de Judas Iscariot,
\item[\vref{Jn 14:1}] Que votre c. ne se trouble
\item[\vref{Jn 16:22}] encore, et votre c. se réjouira, et
\item[\vref{Ac 2:37}] ils eurent le c. touché de componction,
\item[\vref{Ac 4:32}] croyaient n'était qu'un c. et qu'une âme ;
\item[\vref{Ac 8:37}] de tt ton c., cela t'est permis ;
\item[\vref{Ac 15:9}] ayant purifié leurs c. par la foi.
\item[\vref{Ac 16:14}] lui ouvrit le c., afin qu'elle soit
\item[\vref{Ro 1:21}] discours, et lr. c. destitué d'intelligence a
\item[\vref{Ro 2:5}] et par ton c. qui est sans
\item[\vref{Ro 2:29}] c'est celle du c., selon l'Esprit et
\item[\vref{Ro 10:9}] crois ds ton c. que Dieu l'a
\item[\vref{Ro 10:10}] en croyant du c. qu'on parvient à
\item[\vref{2 Co 3:15}] Moïse, le voile demeure sur lr. c.
\item[\vref{Ph 2:3}] que l'humilité de c. vs. fasse regarder
\item[\vref{Hé 3:12}] vs. n'ait un c. mauvais et incrédule,
\item[\vref{Hé 4:7}] entendez sa voix, n'endurcissez pas vos c.
\item[\vref{Hé 8:10}] écrirai ds lr. c., je serai lr.
\item[\vref{Hé 10:22}] lui avec un c. sincère, et une
\item[\vref{Hé 13:9}] bon que le c. soit affermi par
\item[\vref{1 Jn 3:20}] Si notre c. ns. condamne, certes Dieu est plus
\end{listverse}

\ConcordanceEntry{Coffre}
\vspace{-2mm}
\begin{listverse}
\item[\vref{1 S 6:15}] Yahweh, et le c. ds lequel étaient
\item[\vref{2 R 12:9}] Jehojada prit un c., et le perça
\item[\vref{2 Ch 24:11}] Lévites apportaient le c. aux inspecteurs du
\item[\vref{Esd 6:2}] trouva ds un c., au palais royal,
\item[\vref{Ez 27:24}] contenues ds des c. attachés avec des
\end{listverse}

\ConcordanceEntry{Cohéritier}
\vspace{-2mm}
\begin{listverse}
\item[\vref{Ro 8:17}] de Dieu, et c. de Christ, si
\item[\vref{Ep 3:6}] les Gentils sont c. et d'un mm
\end{listverse}

\ConcordanceEntry{Colère}
\vspace{-2mm}
\begin{listverse}
\item[\vref{Ge 27:45}] ce que la c. de ton frère
\item[\vref{Ge 49:7}] Maudite soit lr. c., car elle a
\item[\vref{Ex 23:21}] de provoquer sa c., et écoute sa
\item[\vref{Ex 34:6}] lent à la c., abondant en bonté
\item[\vref{No 11:10}] sa tente. La c. de Yahweh s'enflamma
\item[\vref{De 29:24}] D'où vient l'ardeur de cette grande c. ?
\item[\vref{De 32:22}] feu de ma c. s'est allumé, et
\item[\vref{Jos 7:26}] l'ardeur de sa c.. C'est pourquoi ce
\item[\vref{2 R 22:13}] grande est la c. de Yahweh, qui
\item[\vref{Job 5:2}] En vérité, la c. tue l'insensé, et
\item[\vref{Job 20:28}] Tout s'écoulera au jour de la c. de Dieu.
\item[\vref{Ps 2:12}] conduite, qnd sa c. s'embrasera promptement. Bénis
\item[\vref{Ps 6:2}] pas ds ta c., et ne me
\item[\vref{Ps 7:7}] Yahweh ! Dans ta c., lève-toi contre la
\item[\vref{Ps 30:6}] Car sa c. dure un instant, mais sa grâce
\item[\vref{Ps 37:8}] He.] Laisse la c., et abandonne la
\item[\vref{Ps 38:2}] pas ds ta c. et ne me
\item[\vref{Ps 78:38}] détourna souvent sa c. et ne réveilla
\item[\vref{Ps 103:9}] ne garde point à toujours sa c.
\item[\vref{Pr 14:17}] prompt à la c. agit follement, et
\item[\vref{Pr 14:29}] lent à la c. a une grande
\item[\vref{Pr 15:1}] mais la parole douloureuse excite la c.
\item[\vref{Pr 15:18}] lent à la c. apaise la dispute.
\item[\vref{Pr 19:11}] l'hom. retient sa c. ; c'est un honneur
\item[\vref{Pr 19:12}] La c. du roi est com. le rugissement
\item[\vref{Pr 29:8}] ville, mais les sages apaisent la c.
\item[\vref{Es 57:17}] caché ds ma c. ; et le rebelle
\item[\vref{Jé 23:20}] La c. de Yahweh ne se détournera pas
\item[\vref{Jon 3:9}] l'ardeur de sa c., en sorte que
\item[\vref{Mi 7:9}] Je supporterai la c. de Yahweh, car
\item[\vref{Za 10:3}] Ma c. s'est enflammée contre ces pasteurs, et
\item[\vref{Mt 3:7}] à fuir la c. à venir ?
\item[\vref{Mt 5:22}] se met en c. sans cause contre
\item[\vref{Jn 3:36}] vie, mais la c. de Dieu demeure
\item[\vref{Ro 1:18}] Car la c. de Dieu se révèle pleinement du
\item[\vref{Ro 2:5}] tu t'amasses la c. pour le jour
\item[\vref{Ro 4:15}] loi produit la c., et là où
\item[\vref{Ro 5:9}] son sang, serons-ns. sauvés de la c. par lui.
\item[\vref{Ro 9:22}] voulant montrer sa c., et faire connaître
\item[\vref{Ep 2:3}] des enfants de c. com. les autres.
\item[\vref{Ep 4:26}] vs. mettez en c., ne péchez pas,
\item[\vref{Col 3:8}] ces choses : La c., l'animosité, la médisance,
\item[\vref{1 Th 1:10}] et qui ns. délivre de la c. à venir.
\item[\vref{1 Th 2:16}] péchés. Mais la c. de Dieu est
\item[\vref{1 Th 5:9}] destinés à la c., mais à l'acquisition
\item[\vref{1 Ti 2:8}] mains pures, sans c., et sans dispute.
\item[\vref{Hé 3:11}] jurai ds ma c. : Ils n'entreront pas
\item[\vref{Ja 1:19}] à parler et lent à la c. ;
\item[\vref{Ja 1:20}] car la c. de l'hom. n'accomplit pas la justice
\item[\vref{Ap 6:17}] jour de sa c. est venu, et
\item[\vref{Ap 15:1}] c'est par eux que s'accomplit la c. de Dieu.
\end{listverse}

\ConcordanceEntry{Collecte}
\vspace{-2mm}
\begin{listverse}
\item[\vref{1 Co 16:1}] l'égard de la c. en faveur des
\item[\vref{2 Co 8:20}] de cette abondante c., qui est administrée
\item[\vref{2 Co 9:1}] est de la c. qui se fait
\end{listverse}

\ConcordanceEntry{Colline}
\vspace{-2mm}
\begin{listverse}
\item[\vref{Ex 17:9}] sommet de la c., et la verge
\item[\vref{Ps 48:3}] Belle est la c., joie de tte
\item[\vref{Ps 148:9}] et ttes les c., arbres fruitiers et
\item[\vref{Pr 8:25}] avant que les c. existent, j'ai été
\item[\vref{Es 40:4}] montagne et tte c. seront abaissées, et
\item[\vref{Es 40:12}] montagnes et les c. à la balance ?
\item[\vref{Es 54:10}] iraient, qnd les c. chancelleraient, ma bonté
\item[\vref{Es 55:12}] montagnes et les c. éclateront de joie
\item[\vref{Ez 6:3}] montagnes et aux c., aux cours des
\item[\vref{Os 10:8}] Couvrez-ns. ! Et aux c. : Tombez sur ns. !
\item[\vref{Na 1:5}] lui, et les c. se fondent ; la
\item[\vref{Lu 3:5}] montagne et tte c. seront abaissées ; et
\item[\vref{Lu 23:30}] ns.. Et aux c. : Couvrez-ns. !
\end{listverse}

\ConcordanceEntry{Colombe}
\vspace{-2mm}
\begin{listverse}
\item[\vref{Ge 8:11}] le soir, la c. revint à lui ;
\item[\vref{Ps 55:7}] des ailes de c. ? Je m'envolerais, et
\item[\vref{Ca 2:14}] Ma c. qui te tiens ds les fentes
\item[\vref{Ca 5:2}] grande amie, ma c., ma parfaite ! Car
\item[\vref{Ca 6:9}] ma c., ma parfaite est unique ; elle est
\item[\vref{Es 60:8}] nuées, com. des c. vers lr. colombier ?
\item[\vref{Jé 48:28}] Soyez com. les c. qui font lr.
\item[\vref{Mt 3:16}] descendant com. une c. et venant sur
\item[\vref{Mt 10:16}] des serpents, et simples com. des c.
\end{listverse}

\ConcordanceEntry{Colonne}
\vspace{-2mm}
\begin{listverse}
\item[\vref{Ex 13:21}] jour ds une c. de nuée pour
\item[\vref{Ex 14:24}] étant ds la c. de feu et
\item[\vref{No 12:5}] descendit ds la c. de nuée et
\item[\vref{Jg 16:29}] embrassa les deux c. du milieu sur
\item[\vref{1 R 7:21}] dressa dc les c. au portique du
\item[\vref{Né 9:19}] désert ; et la c. de nuée ne
\item[\vref{Job 26:11}] Les c. du ciel s'ébranlent et s'étonnent à
\item[\vref{Ps 99:7}] parlait de la c. de nuée ; ils
\item[\vref{Pr 9:1}] maison, elle a taillé ses sept c.
\item[\vref{Jé 1:18}] ville forte, une c. de fer, et
\item[\vref{1 Ti 3:15}] Dieu vivant, la c. et l'appui de
\item[\vref{Ap 3:12}] de lui une c. ds le temple
\item[\vref{Ap 10:1}] soleil, et ses pieds com. des c. de feu.
\end{listverse}

\ConcordanceEntry{Combat}
\vspace{-2mm}
\begin{listverse}
\item[\vref{De 2:9}] pas ds un c. avec lui ; car
\item[\vref{2 S 22:35}] mes mains au c., et mes bras
\item[\vref{Ps 18:40}] force pour le c., tu fais plier
\item[\vref{Ps 24:8}] et puissant, Yahweh puissant ds les c.
\item[\vref{Pr 18:6}] querelle, et sa bouche appelle les c.
\item[\vref{Ec 8:8}] délivrance ds ce c., et la méchanceté
\item[\vref{Ez 7:14}] pour aller au c., parce que l'ardeur
\item[\vref{Os 1:7}] ni par les c., ni par les
\item[\vref{Col 2:1}] est grand le c. que j'ai pour
\item[\vref{1 Ti 6:12}] C. le bon combat de la foi,
\item[\vref{2 Ti 4:7}] combattu le bon c., j'ai achevé la
\item[\vref{Hé 10:32}] soutenu un grand c. de souffrances,
\item[\vref{Ap 16:14}] assembler pour le c. de ce grand
\end{listverse}

\ConcordanceEntry{Combattre}
\vspace{-2mm}
\begin{listverse}
\item[\vref{Ex 14:14}] Yahweh c. pour vs. et vs. resterez tranquilles.
\item[\vref{Ex 17:9}] et sors pour c. contre Amalek ; et
\item[\vref{De 3:22}] Yahweh, votre Dieu, c. lui-mm pour vs.
\item[\vref{Jos 10:14}] hom. ; car Yahweh c. pour Israël.
\item[\vref{2 Ch 20:15}] à vs. de c., mais à Dieu.
\item[\vref{Es 63:10}] il a lui-mm c. contre eux.
\item[\vref{Jé 1:19}] Et ils c. contre toi, mais ils ne seront
\item[\vref{Ac 5:39}] trouve que vs. c. contre Dieu.
\item[\vref{Ro 7:23}] autre loi, qui c. contre la loi
\item[\vref{Ro 15:30}] l'Esprit, que vs. c. avec moi ds
\item[\vref{2 Co 10:3}] chair, ns. ne c. pas selon la
\item[\vref{Ph 1:27}] un mm esprit, c. ensemble d'une mm
\item[\vref{Col 1:29}] je travaille, en c. selon son efficacité,
\item[\vref{Col 4:12}] ne cesse de c. pour vs. ds
\item[\vref{1 Ti 1:18}] du devoir de c. ds cette bonne
\item[\vref{1 Ti 6:12}] C. le bon combat de la foi,
\item[\vref{2 Ti 2:5}] mm, l'athlète qui c. n'est pas couronné
\item[\vref{2 Ti 4:7}] J'ai c. le bon combat, j'ai achevé la
\item[\vref{Hé 11:33}] par la foi c. des royaumes, exercèrent
\item[\vref{Hé 12:4}] votre sang en c. contre le péché.
\item[\vref{Ja 4:1}] vos voluptés, qui c. ds vos membres ?
\item[\vref{Jud 1:3}] vs. exhorter à c. pour la foi
\item[\vref{Ap 12:7}] et ses anges c. contre le dragon.
\item[\vref{Ap 17:14}] Ils c. contre l'Agneau et l'Agneau les vaincra,
\item[\vref{Ap 19:11}] il juge et c. avec justice.
\end{listverse}

\ConcordanceEntry{Combler}
\vspace{-2mm}
\begin{listverse}
\item[\vref{Ge 24:35}] Yahweh a c. de bénédictions mon seigneur qui est
\item[\vref{Ge 33:11}] car Dieu m'a c. de grâce, et
\item[\vref{2 S 7:29}] ton serviteur sera c. de bénédictions éternellement.
\item[\vref{Ps 21:7}] bénédictions, tu le c. de joie dvt
\item[\vref{Ps 65:10}] l'abondance, tu la c. de richesses ; le
\item[\vref{Ps 68:20}] les jours ns. c. de ses biens !
\item[\vref{Ez 34:26}] Je les c. de bénédictions, elles, et ts les
\item[\vref{2 Co 9:8}] Tout-Puissant pour vs. c. de ttes sortes
\item[\vref{Ph 4:18}] et j'ai été c. de biens en
\item[\vref{1 Th 2:16}] qu'ils soient sauvés, c. ainsi toujours plus
\end{listverse}

\ConcordanceEntry{Commandement}
\vspace{-2mm}
\begin{listverse}
\item[\vref{Ex 16:28}] de garder mes c. et mes lois ?
\item[\vref{Lé 22:31}] Gardez mes c. et pratiquez-les. Je
\item[\vref{De 4:40}] lois et ses c. que je t'ordonne
\item[\vref{De 7:11}] Garde les c., les lois, et
\item[\vref{De 8:6}] Et garde les c. de Yahweh, ton
\item[\vref{De 30:11}] Car ce c. que je t'ordonne aujourd'hui n'est pas
\item[\vref{Est 1:12}] de venir au c. que le roi
\item[\vref{Ps 19:9}] le cœur ; les c. de Yahweh sont
\item[\vref{Ps 111:7}] Nun.] Tous ses c. sont véritables,
\item[\vref{Ps 119:60}] ne diffère point de garder tes c.
\item[\vref{Ps 119:86}] Tous tes c. ne sont que fidélité ; on me
\item[\vref{Ps 119:96}] parfait, mais tes c. n'ont point de
\item[\vref{Ps 119:143}] m'atteignent, mais tes c. font mes délices.
\item[\vref{Pr 6:23}] Car le c. est une lampe ; et l'enseignement une
\item[\vref{Pr 7:2}] Garde mes c. et tu vivras ;
\item[\vref{Pr 13:13}] qui craint le c. en sera récompensé.
\item[\vref{Pr 19:16}] qui garde le c. garde son âme,
\item[\vref{Ec 12:15}] et garde ses c. ; car c'est là
\item[\vref{Es 48:18}] attentif à mes c., ta paix serait
\item[\vref{Mt 15:9}] sont que des c. d'hommes.
\item[\vref{Mt 22:36}] le plus grand c. de la loi ?
\item[\vref{Mt 22:40}] De ces deux c. dépendent tte la
\item[\vref{Mc 7:9}] rejetez bien le c. de Dieu, afin
\item[\vref{Mc 12:31}] a pas d'autre c. plus grand que
\item[\vref{Jn 13:34}] donne un nouveau c. : Aimez-vs. les uns
\item[\vref{Ro 7:9}] mais qnd le c. vint, le péché
\item[\vref{Ro 7:12}] sainte, et le c. est saint, et
\item[\vref{Ep 6:2}] c'est le premier c. avec une promesse,
\item[\vref{1 Ti 1:5}] le but du c. c'est la charité
\item[\vref{Ja 2:10}] contre un seul c. devient coupable de
\item[\vref{1 Jn 4:21}] ns. avons ce c. de sa part :
\item[\vref{1 Jn 5:3}] ns. gardions ses c. ; et ses commandements
\end{listverse}

\ConcordanceEntry{Commander}
\vspace{-2mm}
\begin{listverse}
\item[\vref{Ge 42:6}] Joseph c. ds le pays, et c'était lui
\item[\vref{De 12:32}] que je vs. c.. Vous n'y ajouterez
\item[\vref{Job 38:12}] au monde, as-tu c. au matin et
\item[\vref{Ps 148:5}] Car il a c., et ils ont
\item[\vref{Jé 7:23}] je lr. ai c., disant : Ecoutez ma
\item[\vref{Mt 15:35}] Alors il c. aux foules de s'asseoir par terre.
\item[\vref{Mc 1:27}] nouvelle doctrine ? Il c. avec autorité mm
\item[\vref{Mc 6:8}] Il lr. c. de ne rien prendre pour le
\item[\vref{Lu 5:14}] Et il lui c. de ne le
\item[\vref{Lu 8:29}] Car Jésus c. à l'esprit impur
\item[\vref{Lu 9:54}] veux-tu que ns. c. que le feu
\item[\vref{Jn 14:31}] le Père m'a c., levez-vs., partons d'ici.
\item[\vref{Jn 15:17}] que je vs. c., c'est de vs.
\end{listverse}

\ConcordanceEntry{Commencement}
\vspace{-2mm}
\begin{listverse}
\item[\vref{Ge 1:1}] Au c., Dieu créa les cieux et la
\item[\vref{De 21:17}] il est le c. de sa vigueur,
\item[\vref{Pr 8:22}] acquise dès le c. de ses voies,
\item[\vref{Pr 9:10}] Le c. de la sagesse est la crainte
\item[\vref{Ec 7:8}] chose que son c.. Mieux vaut l'hom.
\item[\vref{Za 4:10}] jour des faibles c. ? Ils se réjouiront
\item[\vref{Mt 19:4}] le Créateur, au c., fit l'hom. et
\item[\vref{Jn 1:1}] Au c. était la Parole, et la Parole
\item[\vref{Jn 1:2}] Elle était au c. avec Dieu.
\item[\vref{Col 1:18}] qui est le c. et le premier-né
\item[\vref{2 Th 2:13}] élus dès le c. pour le salut
\item[\vref{Hé 7:3}] généalogie, n'ayant ni c. de jours ni
\item[\vref{2 Pi 3:4}] été dès le c. de la création.
\item[\vref{1 Jn 3:8}] pèche dès le c.. Or le Fils
\item[\vref{Ap 1:8}] et l'Oméga, le c. et la fin,
\item[\vref{Ap 3:14}] et véritable, le c. de la création
\item[\vref{Ap 21:6}] et l'Oméga, le c. et la fin.
\item[\vref{Ap 22:13}] le dernier, le c. et la fin.
\end{listverse}

\ConcordanceEntry{Commencer}
\vspace{-2mm}
\begin{listverse}
\item[\vref{Ge 6:1}] les hommes eurent c. à se multiplier
\item[\vref{Ge 11:6}] langage, et ils c. à travailler ; et
\item[\vref{Jos 3:7}] Josué : Aujourd'hui je c. à t'élever aux
\item[\vref{Jg 13:5}] sera lui qui c. à délivrer Israël
\item[\vref{2 Ch 29:17}] Ils c. à sanctifier le temple le premier
\item[\vref{Esd 5:2}] se levèrent et c. à rebâtir la
\item[\vref{Ps 22:26}] Ta louange c. par moi ds
\item[\vref{Ez 9:6}] lettre Tav, et c. par mon lieu
\item[\vref{Mc 14:19}] Ils c. à s'attrister, et ils lui dirent
\item[\vref{Lu 21:28}] Quand ces choses c. à arriver, regardez
\item[\vref{Lu 24:47}] les nations, à c. par Jérus.
\item[\vref{Ac 2:4}] du Saint-Esprit, et c. à parler des
\item[\vref{1 Co 11:21}] à table, chacun c. par prendre son
\item[\vref{2 Co 8:6}] com. il avait c. auparavant, qu'il achève
\item[\vref{Ga 3:3}] insensés, après avoir c. par l'Esprit, voulez-vs.
\item[\vref{Ph 1:6}] celui qui a c. cette bonne œuvre
\item[\vref{Hé 2:3}] salut, qui a c. d'être annoncé d'abord
\item[\vref{1 Pi 4:17}] que le jugement c. par la maison
\item[\vref{Ap 10:7}] ange, qnd il c. à sonner de
\end{listverse}

\ConcordanceEntry{Commun}
\vspace{-2mm}
\begin{listverse}
\item[\vref{Ex 19:8}] peuple répondit d'un c. accord, en disant :
\item[\vref{Jos 22:24}] Qu'y a-t-il de c. entre vs. et
\item[\vref{1 R 10:27}] rendit l'argent aussi c. à Jérus. que
\item[\vref{1 R 22:13}] prophètes parlent d'un c. accord au sujet
\item[\vref{So 3:9}] Yahweh, pour qu'elles le servent d'un c. accord.
\item[\vref{Ac 1:14}] Tous ceux-ci, d'un c. accord, persévéraient ds
\item[\vref{Ac 2:44}] lieu, et ils avaient tt en c. ;
\item[\vref{Ac 4:32}] ttes choses étaient c. entre eux.
\item[\vref{Ac 15:25}] assemblés ts d'un c. accord, d'envoyer vers
\item[\vref{Ac 18:12}] se soulevèrent d'un c. accord contre Paul,
\item[\vref{Ro 1:12}] par la foi qui ns. est c.
\item[\vref{1 Co 12:7}] la manifestation de l'Esprit pour l'utilité c.
\item[\vref{1 Co 14:23}] des gens du c. peuple ou des
\item[\vref{Tit 1:4}] qui ns. est c.: Que la grâce,
\item[\vref{Jud 1:3}] de notre salut c., j'ai jugé nécessaire
\end{listverse}

\ConcordanceEntry{Communion}
\vspace{-2mm}
\begin{listverse}
\item[\vref{Ac 2:42}] apôtres, ds la c. fraternelle, ds la
\item[\vref{1 Co 1:9}] appelés à la c. de son Fils
\item[\vref{1 Co 10:16}] n'est-elle pas la c. du sang de
\item[\vref{1 Co 10:20}] vs. soyez en c. avec des démons.
\item[\vref{2 Co 6:14}] infidèles, car quelle c. y a-t-il entre
\item[\vref{Ga 6:6}] parole soit en c. avec celui qui
\item[\vref{Ep 3:9}] quelle est la c. qui ns. a
\item[\vref{Ph 2:1}] y a qq c. d'esprit, s'il y
\item[\vref{Ph 3:10}] résurrection, et la c. de ses souffrances,
\item[\vref{1 Jn 1:3}] vs. soyez en c. avec ns., et
\item[\vref{1 Jn 1:6}] ns. sommes en c. avec lui, et
\item[\vref{1 Jn 1:7}] ns. sommes en c. les uns avec
\end{listverse}

\ConcordanceEntry{Compagnon}
\vspace{-2mm}
\begin{listverse}
\item[\vref{Ex 31:6}] ai donné pour c. Oholiab, fils d'Ahisamac,
\item[\vref{De 19:5}] manche, trouve son c., et s'il en
\item[\vref{Jg 7:13}] racontait à son c. un songe. Il
\item[\vref{Ps 45:8}] de joie par privilège sur tes c.
\item[\vref{Ps 88:19}] intime et mon c. ; mes connaissances ont
\item[\vref{Pr 28:24}] un péché, est c. de l'hom. destructeur.
\item[\vref{Ec 4:10}] l'autre relèvera son c. ; mais malheur à
\item[\vref{Ez 37:19}] tribus d'Israël, ses c. ; je les joindrai
\item[\vref{Da 2:13}] Daniel et ses c. pour les tuer.
\item[\vref{Ha 2:15}] fait boire son c. en lui approchant
\item[\vref{Za 13:7}] qui est mon c. ! dit Yahweh des
\item[\vref{Mt 11:16}] publiques, et qui crient à leurs c.,
\item[\vref{Mt 18:28}] un de ses c. de service, qui
\item[\vref{Mt 24:49}] à battre ses c. de service, s'il
\item[\vref{Ro 16:21}] Timothée, mon c. d'œuvre, vs. salue,
\item[\vref{2 Co 8:23}] associé et mon c. d'œuvre auprès de
\item[\vref{Phm 1:1}] Timothée, à Philémon notre bien-aimé et c. d'œuvre,
\item[\vref{Ap 6:11}] nombre de leurs c. de service, et
\item[\vref{Ap 19:10}] Je suis ton c. de service, et
\item[\vref{Ap 22:9}] je suis ton c. de service et
\end{listverse}

\ConcordanceEntry{Comparaître}
\vspace{-2mm}
\begin{listverse}
\item[\vref{Ex 34:23}] mâle d'entre vs. c. dvt le Seign.
\item[\vref{De 19:17}] hommes en contestation c. dvt Yahweh, en
\item[\vref{Job 9:19}] justice, qui est-ce qui m'y fera c. ?
\item[\vref{Job 11:10}] emprisonne et fait c., qui l'en détournera ?
\item[\vref{Ac 4:7}] Et ayant fait c. dvt eux Pierre
\item[\vref{Ac 12:4}] de le faire c. dvt le peuple
\item[\vref{Ac 24:19}] auraient dû eux-mêmes c. dvt toi et
\item[\vref{Ro 14:10}] frère ? Certes ns. c. ts dvt le
\item[\vref{2 Co 4:14}] et ns. fera c. en sa présence
\item[\vref{2 Co 5:10}] ns. faut ts c. dvt le tribunal
\item[\vref{Hé 9:24}] mm, afin de c. mntnt pour ns.
\end{listverse}

\ConcordanceEntry{Compassion}
\vspace{-2mm}
\begin{listverse}
\item[\vref{Ex 2:6}] fut touchée de c., et dit : C'est
\item[\vref{Ex 33:19}] grâce, et j'aurai c. de celui de
\item[\vref{De 30:3}] captifs et aura c. de toi ; il
\item[\vref{2 S 24:14}] Yahweh, car ses c. sont en grand
\item[\vref{Job 6:14}] droit à la c. de son ami ;
\item[\vref{Ps 25:6}] Souviens-toi de tes c. et de ta
\item[\vref{Ps 37:21}] le juste a c., et donne.
\item[\vref{Ps 40:12}] m'épargne point tes c.. Que ta bonté
\item[\vref{Ps 102:14}] et tu auras c. de Sion ; car
\item[\vref{Ps 119:156}] Tes c. sont en grand nombre, ô Yahweh !
\item[\vref{Es 49:13}] peuple, il a c. de ceux qu'il
\item[\vref{Es 54:7}] je te rassemblerai avec de grandes c.
\item[\vref{Es 63:15}] et de tes c. se retiennent-ils envers
\item[\vref{Jé 15:6}] te détruis, je suis las d'avoir c.
\item[\vref{Jé 33:26}] je ramènerai leurs captifs et j'aurai c. d'eux.
\item[\vref{La 3:22}] parce que ses c. ne sont pas
\item[\vref{La 3:32}] il a aussi c. selon la grandeur
\item[\vref{Ez 8:18}] n'aurai point de c. ; qnd ils crieront
\item[\vref{Da 9:18}] c'est à cause de tes grandes c.
\item[\vref{Am 1:11}] a altéré ses c., parce que sa
\item[\vref{Mi 7:19}] Il aura encore c. de ns. ; il
\item[\vref{Ha 3:2}] ds ta colère souviens-toi de tes c.
\item[\vref{Za 10:6}] parce que j'aurai c. d'eux, et ils
\item[\vref{Mt 9:36}] fut ému de c., parce qu'elles étaient
\item[\vref{Mc 9:22}] tu peux qq chose, secours-ns., aie c. de ns.
\item[\vref{Lu 10:33}] fut ému de c. lorsqu'il le vit.
\item[\vref{Lu 15:20}] fut ému de c., et il courut
\item[\vref{Ro 12:1}] frères, par les c. de Dieu, à
\item[\vref{Ep 4:32}] autres, pleins de c., et vs. pardonnant
\item[\vref{Ph 2:1}] a quelques cordiales affections et quelques c.,
\item[\vref{Hé 4:15}] ne puisse avoir c. de nos infirmités ;
\end{listverse}

\ConcordanceEntry{Compatissant}
\vspace{-2mm}
\begin{listverse}
\item[\vref{Ex 34:6}] Yahweh ! le Dieu c., miséricordieux, lent à
\item[\vref{2 Ch 30:9}] votre Dieu, est c. et miséricordieux ; et
\item[\vref{Né 9:17}] qui pardonne, miséricordieux, c., lent à la
\item[\vref{Ps 37:26}] Il est c. tt le temps, et il prête ;
\item[\vref{Ps 78:38}] com. il est c., il pardonna lr.
\item[\vref{Ps 86:15}] es le Dieu c., miséricordieux, lent à
\item[\vref{Ps 103:8}] Yahweh est c., miséricordieux, lent à
\item[\vref{Ps 111:4}] mémorables. [Heth.] Yahweh est miséricordieux et c.
\item[\vref{Ps 112:4}] Heth.] il est c., miséricordieux et juste.
\item[\vref{Ps 116:5}] Yahweh est c. et juste, et
\item[\vref{Ps 145:8}] est miséricordieux et c., lent à la
\item[\vref{Joë 2:13}] car il est c. et miséricordieux, lent
\item[\vref{Jon 4:2}] es un Dieu c., miséricordieux, lent à
\end{listverse}

\ConcordanceEntry{Complainte}
\vspace{-2mm}
\begin{listverse}
\item[\vref{2 S 3:33}] roi fit une c. sur Abner, et
\item[\vref{2 Ch 35:25}] parlèrent ds leurs c. sur Josias jusqu'à
\item[\vref{Ps 142:3}] dvt lui ma c., je déclare mon
\item[\vref{Jé 7:29}] haute voix ta c. sur les lieux
\item[\vref{Jé 9:10}] montagnes, et une c. à cause des
\item[\vref{Jé 9:20}] à sa compagne à faire des c. !
\item[\vref{Ez 19:1}] haute voix une c. concernant les princes
\item[\vref{Ez 19:14}] C'est là une c., et cela servira
\item[\vref{Ez 26:17}] haute voix une c. sur toi, et
\item[\vref{Ez 27:2}] l'hom., prononce à haute voix une c. sur Tyr.
\item[\vref{Ez 27:32}] sur toi une c. ds lr. lamentation,
\item[\vref{Ez 28:12}] haute voix une c. sur le roi
\item[\vref{Ez 32:2}] haute voix une c. sur Pharaon, roi
\item[\vref{Ez 32:16}] C'est ici la c. qu'on fera sur
\item[\vref{Am 5:1}] qui est la c. que je prononce
\item[\vref{Mt 11:17}] avons chanté des c. et vs. ne
\end{listverse}

\ConcordanceEntry{Complot}
\vspace{-2mm}
\begin{listverse}
\item[\vref{Ps 31:21}] face, loin du c. des hommes, tu
\item[\vref{Ps 64:3}] Cache-moi des c. des méchants, de
\item[\vref{Pr 24:8}] mal, on l'appelle le maître des c.
\item[\vref{Jé 11:15}] y faire leurs c. ? La chair sainte
\item[\vref{Jé 18:18}] et faisons des c. contre Jérémie ! Car
\item[\vref{Ez 22:25}] y a un c. de ses prophètes
\item[\vref{Da 11:25}] on formera des c. contre lui.
\item[\vref{Ac 5:9}] avez-vs. fait un c. entre vs. pour
\item[\vref{Ac 9:24}] et lr. c. parvint à la connaissance de Saul.
\item[\vref{Ac 23:12}] Juifs formèrent un c., et firent des
\end{listverse}

\ConcordanceEntry{Comploter}
\vspace{-2mm}
\begin{listverse}
\item[\vref{Ge 37:18}] près d'eux, ils c. contre lui pour
\item[\vref{Né 6:2}] Or ils avaient c. de me faire
\item[\vref{Ps 37:12}] Zayin.] Le méchant c. contre le juste,
\item[\vref{Ps 140:2}] les jours ils c. des guerres !
\item[\vref{Es 54:15}] manquera pas de c. contre toi, cela
\item[\vref{Os 6:9}] prêtres, après avoir c., tuent les gens
\end{listverse}

\ConcordanceEntry{Comprendre}
\vspace{-2mm}
\begin{listverse}
\item[\vref{De 28:49}] dont tu ne c. pas la langue,
\item[\vref{Né 8:8}] l'intelligence, la faisant c. par l'Ecriture elle-mm.
\item[\vref{Job 37:5}] grandes choses que ns. ne saurions c.
\item[\vref{Job 42:3}] et je n'y c. rien ; ces choses
\item[\vref{Pr 20:24}] dc l'hom. peut-il c. sa voie ?
\item[\vref{Pr 28:5}] droit ; mais ceux qui cherchent Yahweh c. tt.
\item[\vref{Ec 3:11}] que l'hom. puisse c. du commencement à
\item[\vref{Es 6:9}] mais vs. ne c. point ; et voyez,
\item[\vref{Es 33:19}] et de langue bégayante qu'on ne c. pas.
\item[\vref{Es 44:18}] et les cœurs pour qu'ils ne c. point.
\item[\vref{Jé 30:24}] cœur ; vs. le c. ds les derniers
\item[\vref{Da 12:8}] mais je ne c. pas ; et je
\item[\vref{Mt 13:15}] oreilles, qu'ils ne c. de lr. cœur,
\item[\vref{Mt 13:23}] parole et la c.. Il porte du
\item[\vref{Mt 13:51}] lr. dit : Avez-vs. c. ttes ces choses ?
\item[\vref{Mt 16:23}] car tu ne c. pas les choses
\item[\vref{Mt 19:12}] celui qui peut c. ceci, le comprenne !
\item[\vref{Mc 8:33}] Car tu ne c. pas les choses
\item[\vref{Jn 13:7}] dit : Tu ne c. pas mntnt ce
\item[\vref{Ac 7:25}] que ses frères c. par là que
\item[\vref{Ac 8:30}] il lui dit : C.-tu ce que
\item[\vref{1 Co 2:14}] l'hom. animal ne c. pas les choses
\item[\vref{1 Co 14:2}] personne ne le c., et c'est en
\item[\vref{Ep 3:18}] vs. puissiez c. avec ts les
\item[\vref{2 Pi 3:16}] points difficiles à c., que les personnes
\end{listverse}

\ConcordanceEntry{Compte}
\vspace{-2mm}
\begin{listverse}
\item[\vref{De 18:19}] mon Nom, je lui en demanderai c.
\item[\vref{Jos 22:23}] que Yahweh lui-mm ns. en demande c. !
\item[\vref{Job 33:13}] ne rend pas c. de ttes ses
\item[\vref{Ps 144:3}] l'hom. pour que tu en tiennes c. ?
\item[\vref{Da 3:12}] ne tiennent aucun c. de toi ; ils
\item[\vref{Mt 12:36}] Les hommes rendront c. au jour du
\item[\vref{Mt 25:19}] serviteurs revint et lr. fit rendre c.
\item[\vref{Lu 16:2}] de toi ? Rends c. de ton administration,
\item[\vref{Ac 17:30}] Dieu, sans tenir c. des temps d'ignorance,
\item[\vref{Ro 14:12}] de ns. rendra c. à Dieu pour
\item[\vref{Ph 4:17}] le fruit qui abonde pour votre c.
\item[\vref{Phm 1:18}] doit qq chose, mets-le sur mon c.
\item[\vref{Hé 13:17}] dvt en rendre c. ; afin que ce
\item[\vref{1 Pi 4:5}] Mais ils rendront c. à celui qui
\end{listverse}

\ConcordanceEntry{Compter}
\vspace{-2mm}
\begin{listverse}
\item[\vref{Ge 13:16}] si quelqu'un peut c. la poussière de
\item[\vref{Ge 15:5}] au ciel et c. les étoiles si
\item[\vref{Job 5:9}] merveilleuses qu'il est impossible de les c.
\item[\vref{Job 14:16}] Mais mntnt tu c. mes pas, et
\item[\vref{Job 38:37}] peut avec intelligence c. les nuages, et
\item[\vref{Ps 22:18}] Je pourrais c. ts mes os
\item[\vref{Ps 56:9}] Tu c. mes allées et venues ; recueille mes
\item[\vref{Ps 90:12}] Enseigne-ns. à c. nos jours, afin
\item[\vref{Ps 147:4}] Il c. le nombre des étoiles, il les
\item[\vref{Ec 1:15}] ce qui manque ne peut être c.
\item[\vref{Jé 33:22}] on ne peut c. l'armée des cieux,
\item[\vref{Da 5:25}] a été gravée : C., compté, pesé et
\item[\vref{Os 2:1}] mesurer ni se c. ; et il arrivera
\item[\vref{Ag 1:9}] Vous c. sur beaucoup, et voici, il y
\item[\vref{Mt 10:30}] cheveux de votre tête sont ts c.
\item[\vref{Mt 15:38}] mille hommes, sans c. les femmes et
\item[\vref{Mt 18:24}] se mit à c., on lui en
\item[\vref{Mt 26:15}] Et ils lui c. trente pièces d'argent.
\item[\vref{Ap 7:9}] personne ne pouvait c., de tte nation,
\item[\vref{Ap 13:18}] a de l'intelligence c. le nombre de
\end{listverse}

\ConcordanceEntry{Concevoir}
\vspace{-2mm}
\begin{listverse}
\item[\vref{Ge 4:1}] fem. et elle c., et enfanta Caïn ;
\item[\vref{Ge 16:4}] Agar, et elle c.. Quand Agar se
\item[\vref{Ge 27:41}] Esaü c. de la haine contre Jacob, à
\item[\vref{Ge 29:32}] Léa c. et enfanta un fils à qui
\item[\vref{Ex 2:2}] Cette fem. c. et enfanta un
\item[\vref{No 11:12}] moi qui ai c. tt ce peuple
\item[\vref{Ru 4:13}] la grâce de c., et elle enfanta
\item[\vref{1 S 1:20}] temps après, qu'Anne c. et enfanta un
\item[\vref{Ps 7:15}] le mal, il c. l'iniquité, et il
\item[\vref{Ps 51:7}] ma mère m'a c. ds le péché.
\item[\vref{Ps 66:18}] Si j'avais c. l'iniquité ds mon
\item[\vref{Es 8:3}] la prophétesse ; elle c. et elle enfanta
\item[\vref{Es 25:1}] merveilleuses ; tes conseils c. d'avance sont fidèlement
\item[\vref{Es 33:11}] Vous avez c. du foin, et
\item[\vref{Es 59:4}] des faussetés, ils c. le mal et
\item[\vref{Os 1:3}] de Diblaïm. Elle c., et lui enfanta
\item[\vref{Mt 1:20}] l'enfant qu'elle a c. est du Saint-Esprit.
\item[\vref{Lu 1:24}] Elisabeth, sa fem., c., et elle se
\item[\vref{Ro 9:10}] de Rébecca, lorsqu'elle c. seulement d'Isaac, notre
\item[\vref{Hé 11:11}] la force de c. un enfant, et
\item[\vref{Ja 1:15}] la convoitise a c., elle enfante le
\end{listverse}

\ConcordanceEntry{Concitoyen}
\vspace{-2mm}
\begin{listverse}
\item[\vref{Lu 19:14}] Or ses c. le haïssaient, c'est pourquoi ils envoyèrent
\item[\vref{Ep 2:19}] de dehors, mais c. des saints et
\end{listverse}

\ConcordanceEntry{Condamnation}
\vspace{-2mm}
\begin{listverse}
\item[\vref{Es 53:8}] et de la c. ; mais qui racontera
\item[\vref{Mt 27:37}] cause de sa c. était marquée en
\item[\vref{Mc 3:29}] et subira une c. éternelle .
\item[\vref{Jn 5:24}] vient pas en c., mais il est
\item[\vref{Jn 5:29}] fait le mal ressusciteront pour la c.
\item[\vref{Ro 3:8}] le disons. La c. de ces gens
\item[\vref{Ro 5:16}] jugement est devenu c., tandis que le
\item[\vref{Ro 8:1}] dc mntnt aucune c. pour ceux qui
\item[\vref{Ro 13:2}] résistent attireront la c. sur eux-mêmes.
\item[\vref{1 Co 11:29}] et boit sa c., ne distinguant pas
\item[\vref{2 Co 3:9}] service de la c. a été glorieux,
\item[\vref{Ga 5:10}] quel qu'il soit, en portera la c.
\item[\vref{1 Ti 5:12}] ayant lr. c., en ce qu'elles
\item[\vref{2 Pi 2:3}] déguisées, mais la c. qui lr. est
\item[\vref{Jud 1:4}] hommes, dont la c. est écrite depuis
\item[\vref{Ap 18:10}] puissante, comment ta c. est-elle venue en
\end{listverse}

\ConcordanceEntry{Condamner}
\vspace{-2mm}
\begin{listverse}
\item[\vref{De 25:1}] juste, et on c. le méchant.
\item[\vref{1 R 8:32}] Juge tes serviteurs, c. le coupable en
\item[\vref{Ps 94:21}] du juste, et c. le sang innocent.
\item[\vref{Pr 12:2}] Yahweh, mais Yahweh c. l'hom. qui a
\item[\vref{Pr 17:26}] pas bon de c. l'innocent à l'amende,
\item[\vref{Es 54:17}] toi, tu la c.. Tel est l'héritage
\item[\vref{Jé 26:11}] hom. mérite d'être c. à la mort ;
\item[\vref{Mt 12:7}] vs. n'auriez pas c. ceux qui ne
\item[\vref{Mt 12:37}] et tu seras c. par tes paroles.
\item[\vref{Mt 12:41}] génération et la c., parce qu'ils se
\item[\vref{Mc 14:64}] Alors ts le c. com. étant digne
\item[\vref{Mc 16:16}] celui qui ne croira pas sera c.
\item[\vref{Lu 6:37}] pas jugés ; ne c. pas, et vs.
\item[\vref{Lu 23:40}] car tu es c. au mm supplice ?
\item[\vref{Lu 24:20}] livré pour être c. à mort, et
\item[\vref{Jn 3:18}] ne sera pas c. ; mais celui qui
\item[\vref{Jn 7:51}] Notre loi c.-t-elle un hom.
\item[\vref{Jn 8:11}] Je ne te c. pas non plus ;
\item[\vref{Ac 13:27}] et, en le c., ils ont accompli
\item[\vref{Ro 2:1}] autres, tu te c. toi-mm, puisque toi
\item[\vref{Ro 8:3}] péché, il a c. le péché ds
\item[\vref{Ro 8:34}] Qui les c. ? Christ est mort ; et bien plus,
\item[\vref{Ro 14:23}] qu'il mange est c., parce qu'il n'agit
\item[\vref{1 Co 11:32}] ne soyons pas c. avec le monde.
\item[\vref{Ep 5:11}] infructueuses des ténèbres, mais au contraire c.-les !
\item[\vref{2 Th 2:12}] ont pris plaisir à l'iniquité soient c.
\item[\vref{Ja 5:9}] ne soyez pas c.. Voici, le Juge
\item[\vref{2 Pi 2:6}] et s'il a c. à la destruction
\item[\vref{1 Jn 3:20}] notre cœur ns. c., certes Dieu est
\end{listverse}

\ConcordanceEntry{Conducteur}
\vspace{-2mm}
\begin{listverse}
\item[\vref{1 S 9:16}] pour être le c. de mon peuple
\item[\vref{Pr 28:16}] Le c. qui manque d'intelligence fait beaucoup d'extorsions,
\item[\vref{Da 9:25}] jusqu'au Messie, le C., il y a
\item[\vref{Mt 15:14}] sont des aveugles, c. d'aveugles ; si un
\item[\vref{Mt 23:16}] Malheur à vs. c. aveugles, qui dites :
\item[\vref{Ro 2:19}] crois être le c. des aveugles, la
\item[\vref{Hé 13:7}] Souvenez-vs. de vos c. qui vs. ont
\item[\vref{Hé 13:17}] Obéissez à vos c., et soyez-lr. soumis,
\end{listverse}

\ConcordanceEntry{Conduire}
\vspace{-2mm}
\begin{listverse}
\item[\vref{Ge 24:27}] chemin, Yahweh m'a c. ds la maison
\item[\vref{Ge 48:15}] Dieu qui m'a c. depuis que j'existe
\item[\vref{Ex 13:21}] nuée pour les c. par le chemin ;
\item[\vref{De 29:5}] Je t'ai c. pendant quarante ans par le désert ;
\item[\vref{De 32:12}] Yahweh seul l'a c., et il n'y
\item[\vref{2 Ch 1:10}] je sache me c. dvt ce peuple ;
\item[\vref{Job 38:32}] du zodiaque, et c.-tu la Grande
\item[\vref{Ps 78:14}] Il les c. de jour par la nuée, et
\item[\vref{Ps 136:16}] Il a c. son peuple ds le désert, car
\item[\vref{Ps 139:10}] ta main me c., et ta droite
\item[\vref{Pr 11:19}] Ainsi la justice c. à la vie,
\item[\vref{Pr 16:9}] médite sur sa voie, mais Yahweh c. ses pas.
\item[\vref{Pr 19:23}] crainte de Yahweh c. à la vie,
\item[\vref{Ca 8:2}] Je te c., je t'introduirais ds la maison de
\item[\vref{Es 3:12}] ceux qui te c. t'égarent, ils corrompent
\item[\vref{Es 42:16}] Je c. les aveugles sur un chemin qu'ils
\item[\vref{Ez 39:2}] arrière, je te c., je te ferai
\item[\vref{Mi 7:9}] justice ; il me c. à la lumière,
\item[\vref{Mt 15:14}] si un aveugle c. un autre aveugle,
\item[\vref{Lu 4:1}] et il fut c. par l'Esprit ds
\item[\vref{Jn 16:13}] vérité, il vs. c. ds tte la
\item[\vref{Ro 6:16}] du péché qui c. à la mort,
\item[\vref{Ro 7:10}] le commandement qui c. à la vie
\item[\vref{Ep 4:17}] ne plus vs. c. com. le reste
\item[\vref{Ph 1:27}] Seulement, c.-vs. d'une manière
\item[\vref{Col 4:5}] C.-vs. sagement envers ceux du dehors,
\item[\vref{1 Th 2:12}] conjurant de vs. c. d'une manière digne
\item[\vref{1 Ti 3:15}] il faut se c. ds la maison
\item[\vref{1 Pi 1:17}] chacun, sans favoritisme, c.-vs. avec crainte
\item[\vref{Ap 7:17}] paîtra, et les c. aux sources des
\end{listverse}

\ConcordanceEntry{Conduite}
\vspace{-2mm}
\begin{listverse}
\item[\vref{Ps 77:21}] troupeau, sous la c. de Moïse et
\item[\vref{Ps 101:2}] attentif à une c. pure jusqu'à ce
\item[\vref{Ez 14:22}] vs. verrez lr. c. et leurs actions,
\item[\vref{Ag 1:5}] Yahweh des armées : Considérez attentivement votre c. !
\item[\vref{Ga 1:13}] été autrefois ma c. ds le judaïsme,
\item[\vref{Ep 4:22}] est de votre c. précédente, qui se
\item[\vref{1 Ti 4:12}] en paroles, en c., en charité, en
\item[\vref{2 Ti 3:10}] ma doctrine, ma c., mon intention, ma
\item[\vref{Hé 13:5}] Que votre c. soit sans avarice,
\item[\vref{Ja 3:13}] par une bonne c. avec douceur et
\item[\vref{1 Pi 1:15}] mm soyez saints ds tte votre c.,
\item[\vref{1 Pi 2:12}] ayant une c. honnête avec les
\item[\vref{1 Pi 3:2}] pureté de votre c., accompagnée de crainte.
\item[\vref{2 Pi 2:7}] de ces abominables par lr. infâme c. ;
\item[\vref{2 Pi 3:11}] sainteté de votre c. et votre piété ?
\end{listverse}

\ConcordanceEntry{Confesser}
\vspace{-2mm}
\begin{listverse}
\item[\vref{Lé 16:21}] vivant, et il c. sur lui ttes
\item[\vref{No 5:7}] alors ils c. lr. péché, qu'ils
\item[\vref{Né 1:6}] enfants d'Israël, en c. les péchés des
\item[\vref{Né 9:2}] ils se présentèrent c. leurs péchés et
\item[\vref{Pr 28:13}] celui qui les c. et les délaisse,
\item[\vref{Da 9:20}] je priais, je c. mon péché, et
\item[\vref{Mt 10:32}] Quiconque dc me c. dvt les hommes,
\item[\vref{Mc 1:5}] vers lui, et c. leurs péchés, ils
\item[\vref{Lu 12:8}] dis, quiconque me c. dvt les hommes,
\item[\vref{Jn 12:42}] ils ne le c. pas, de peur
\item[\vref{Ac 19:18}] avaient cru venaient, c. et déclarant ce
\item[\vref{Ro 10:9}] pourquoi, si tu c. de ta bouche
\item[\vref{Ro 10:10}] et c'est en c. de la bouche
\item[\vref{Ph 2:11}] que tte langue c. que Jésus-Christ est
\item[\vref{Hé 13:15}] c'est-à-dire, le fruit des lèvres, en c. son Nom.
\item[\vref{Ja 5:16}] C. dc vos fautes les uns les
\item[\vref{1 Jn 1:9}] Si ns. c. nos péchés, il est fidèle et
\item[\vref{1 Jn 4:2}] Tout esprit qui c. que Jésus-Christ est
\item[\vref{Ap 3:5}] vie, mais je c. son nom dvt
\end{listverse}

\ConcordanceEntry{Confession}
\vspace{-2mm}
\begin{listverse}
\item[\vref{Jos 7:19}] d'Israël, et fais-lui c.. Déclare-moi je te
\item[\vref{Esd 10:1}] et faisait cette c., pleurant et étant
\item[\vref{Né 9:3}] quart, ils faisaient c. de leurs péchés,
\item[\vref{Da 9:4}] lui fis ma c. : Ah ! Seign., Dieu
\item[\vref{1 Ti 6:13}] fait cette belle c. dvt Ponce Pilate,
\item[\vref{Hé 4:14}] ferme ds la c. de notre foi.
\end{listverse}

\ConcordanceEntry{Confiance}
\vspace{-2mm}
\begin{listverse}
\item[\vref{De 9:23}] vs. n'eûtes point c., et vs. n'obéîtes
\item[\vref{2 R 18:19}] Quelle est cette c. sur laquelle tu
\item[\vref{2 R 18:20}] as-tu placé ta c., pour te rebeller
\item[\vref{Job 31:24}] au fin or : Tu es ma c. ;
\item[\vref{Ps 41:10}] qui avait ma c. et qui mangeait
\item[\vref{Ps 49:7}] Ils mettent lr. c. ds leurs biens
\item[\vref{Pr 22:19}] afin que ta c. soit en Yahweh.
\item[\vref{Pr 31:11}] mari a entièrement c. en elle, ainsi
\item[\vref{Es 10:20}] ils s'appuieront avec c. sur Yahweh, le
\item[\vref{Es 12:2}] mon salut, j'aurai c. et je ne
\item[\vref{Da 3:28}] qui ont eu c. en lui, et
\item[\vref{So 3:12}] il mettra sa c. ds le Nom
\item[\vref{2 Co 1:9}] n'ayons pas de c. en ns.-mêmes, mais
\item[\vref{2 Co 5:6}] avons dc toujours c., et ns. savons
\item[\vref{Ep 3:12}] et accès avec c., par la foi
\item[\vref{Ph 2:24}] Et j'ai cette c. en notre Seign.
\item[\vref{Ph 3:4}] puisse aussi avoir c. ds la chair.
\item[\vref{2 Th 3:4}] votre égard cette c. ds le Seign.,
\item[\vref{1 Ti 6:17}] mettent pas lr. c. ds l'incertitude des
\end{listverse}

\ConcordanceEntry{Confier}
\vspace{-2mm}
\begin{listverse}
\item[\vref{De 32:37}] le rocher en qui ils se c.,
\item[\vref{Né 13:13}] Je c. la surveillance du trésor à Schélémia,
\item[\vref{Ps 2:12}] Bénis sont ts ceux qui se c. en lui !
\item[\vref{Ps 5:12}] ceux qui se c. en toi se
\item[\vref{Ps 22:5}] pères se sont c. en toi ; ils
\item[\vref{Ps 32:10}] l'hom. qui se c. en Yahweh.
\item[\vref{Ps 34:9}] bon ! Béni est l'hom. qui se c. en lui !
\item[\vref{Ps 37:3}] [Beth.] C.-toi en Yahweh, et fais ce
\item[\vref{Ps 44:7}] je ne me c. point en mon
\item[\vref{Ps 52:10}] verdoyant. Je me c. ds la bonté
\item[\vref{Ps 56:12}] Je me c. en Dieu, je ne craindrai rien :
\item[\vref{Ps 62:9}] Peuples, c.-vs. en lui en tt temps,
\item[\vref{Ps 91:2}] mon Dieu en qui je me c. !
\item[\vref{Ps 146:3}] Ne vs. c. pas aux grands, ni en aucun
\item[\vref{Pr 3:5}] C.-toi de tt ton cœur en
\item[\vref{Pr 11:28}] Celui qui se c. ds ses richesses
\item[\vref{Pr 28:26}] Celui qui se c. en son propre
\item[\vref{Es 42:17}] ceux qui se c. aux images taillées,
\item[\vref{Es 50:10}] de clarté, se c. ds le Nom
\item[\vref{Jé 17:5}] l'hom. qui se c. ds l'hom., et
\item[\vref{Jé 17:7}] l'hom. qui se c. en Yahweh, et
\item[\vref{Ez 33:13}] que lui, se c. sur sa justice,
\item[\vref{Na 1:7}] et il connaît ceux qui se c. en lui.
\item[\vref{So 3:2}] ne s'est point c. en Yahweh, elle
\item[\vref{Mt 25:20}] Seign., tu m'as c. cinq talents ; voici,
\item[\vref{Mt 27:43}] Il se c. en Dieu ; mais si Dieu l'aime,
\item[\vref{Mc 10:24}] ceux qui se c. ds les richesses
\item[\vref{Ro 3:2}] oracles de Dieu lr. ont été c.
\item[\vref{1 Co 9:17}] moi, c'est une charge qui m'est c.
\item[\vref{Ga 2:7}] incirconcis m'avait été c. com. à Pierre
\item[\vref{1 Th 2:4}] dignes de ns. c. la prédication de
\item[\vref{2 Ti 2:2}] dvt plusieurs témoins, c.-les à des
\item[\vref{Hé 2:13}] encore : Je me c. en lui ; et
\end{listverse}

\ConcordanceEntry{Confirmer}
\vspace{-2mm}
\begin{listverse}
\item[\vref{De 8:18}] richesses, afin de c. son alliance, qu'il
\item[\vref{Ru 4:7}] en Israël, pour c. une affaire quelconque
\item[\vref{2 S 7:25}] ô Yahweh Dieu, c. pour toujours la
\item[\vref{2 Ch 6:17}] déclarée à David, ton serviteur, soit c. !
\item[\vref{Est 9:32}] Ainsi l'édit d'Esther c. l'institution des Purim,
\item[\vref{Es 44:26}] C'est lui qui c. la parole de
\item[\vref{Da 9:27}] Et il c. l'alliance à plusieurs pour une semaine,
\item[\vref{Mc 16:20}] avec eux, et c. la parole par
\item[\vref{1 Co 1:6}] le témoignage de Jésus-Christ a été c. en vs.,
\item[\vref{2 Co 2:8}] vs. prie de c. publiquement envers lui
\item[\vref{2 Co 13:1}] de trois témoins, tte parole sera c.
\item[\vref{Ga 3:17}] que Dieu a c. antérieurement, ne peut
\item[\vref{Hé 2:4}] Dieu c. aussi lr. témoignage par des prodiges,
\item[\vref{Hé 9:18}] n'a pas été c. sans le sang.
\item[\vref{Ap 22:16}] ange pour vs. c. ces choses ds
\end{listverse}

\ConcordanceEntry{Confondre}
\vspace{-2mm}
\begin{listverse}
\item[\vref{Ge 11:7}] Descendons, et là c. lr. langage afin
\item[\vref{Ge 11:9}] là que Yahweh c. le langage de
\item[\vref{Ps 44:10}] rejettes, tu ns. c., et tu ne
\item[\vref{Ps 53:6}] toi. Tu les c., car Dieu les
\item[\vref{Pr 22:12}] connaissance, mais il c. les paroles du
\item[\vref{Es 45:24}] viendront, pour être c., ts ceux qui
\item[\vref{Ac 9:22}] en plus, et c. les Juifs qui
\item[\vref{Ro 5:5}] Or l'espérance ne c. pas, parce que
\end{listverse}

\ConcordanceEntry{Conformer}
\vspace{-2mm}
\begin{listverse}
\item[\vref{No 17:11}] ainsi ; il se c. à l'ordre que
\item[\vref{Ro 12:2}] Et ne vs. c. pas au siècle
\item[\vref{1 Pi 1:14}] obéissants, ne vs. c. pas à vos
\end{listverse}

\ConcordanceEntry{Confus}
\vspace{-2mm}
\begin{listverse}
\item[\vref{Ex 14:3}] d'Israël : Ils sont c. ds le pays,
\item[\vref{Esd 9:6}] je suis trop c., ô mon Dieu,
\item[\vref{Ps 22:6}] toi, et ils n'ont point été c.
\item[\vref{Ps 31:18}] ne sois point c. puisque je t'ai
\item[\vref{Ps 34:6}] illuminé, et la face n'est point c.
\item[\vref{Ps 71:1}] refuge : Que je ne sois jamais c. !
\item[\vref{Ps 97:7}] des idoles soient c. ! Vous dieux, prosternez-vs.
\item[\vref{Es 45:17}] ni honteux ni c. jsq. ds l'éternité.
\item[\vref{Es 49:23}] confient en moi ne seront point c.
\item[\vref{Jé 8:9}] Les sages sont c., ils sont épouvantés
\item[\vref{Za 10:5}] avec eux ; et les cavaliers seront c.
\item[\vref{Lu 13:17}] ses adversaires étaient c. ; mais ttes les
\item[\vref{Ac 19:32}] car l'assemblée était c., et la plupart
\item[\vref{Ro 9:33}] croit en lui ne sera pas c.
\item[\vref{Ro 10:11}] croit en lui ne sera pas c.
\item[\vref{1 Co 1:27}] monde pour rendre c. les sages ; et
\item[\vref{2 Co 7:14}] ai pas été c. ; mais com. ns.
\item[\vref{Ph 1:20}] je ne serai c. en rien, mais
\item[\vref{Tit 2:8}] contredit, soit rendu c., n'ayant aucun mal
\item[\vref{1 Pi 2:6}] croit en elle ne sera pas c.
\item[\vref{1 Pi 3:16}] en Christ soient c. de ce qu'ils
\end{listverse}

\ConcordanceEntry{Confusion}
\vspace{-2mm}
\begin{listverse}
\item[\vref{De 28:20}] la malédiction, la c., et la ruine
\item[\vref{2 S 19:5}] couvres aujourd'hui de c. les faces de
\item[\vref{Ps 44:16}] Ma c. est tt le jour dvt moi,
\item[\vref{Pr 18:13}] acte de folie et attire la c.
\item[\vref{Ez 44:13}] ils porteront lr. c. et leurs abominations
\item[\vref{Da 9:7}] à ns. la c. de face, en
\item[\vref{Os 10:6}] sera ds la c., et Israël aura
\item[\vref{Joë 2:26}] ne sera plus jamais ds la c.
\item[\vref{Ac 19:29}] fut remplie de c. ; et ils se
\item[\vref{1 Co 14:33}] un Dieu de c., mais de paix,
\item[\vref{Ph 3:19}] est ds lr. c., n'ayant d'affection que
\end{listverse}

\ConcordanceEntry{Connaissance}
\vspace{-2mm}
\begin{listverse}
\item[\vref{Ge 3:6}] donner de la c. ; prit de son
\item[\vref{1 R 4:29}] intelligence, et des c. multipliées com. le
\item[\vref{1 Ch 12:32}] intelligents ds la c. des temps, pour
\item[\vref{Né 10:28}] étaient capables de c. et d'intelligence,
\item[\vref{Job 38:2}] mes décisions par des paroles sans c. ?
\item[\vref{Ps 53:5}] n'ont-ils point de c. ? Ils mangent mon
\item[\vref{Ps 73:22}] je n'avais aucune c. ; j'étais com. une
\item[\vref{Pr 1:29}] auront haï la c., et qu'ils n'auront
\item[\vref{Pr 2:5}] de Yahweh, et tu trouveras la c. de Dieu.
\item[\vref{Pr 2:10}] et si la c. est agréable à
\item[\vref{Pr 8:12}] je possède la c. de la réflexion.
\item[\vref{Pr 22:17}] et applique ton cœur à ma c.
\item[\vref{Ec 1:18}] qui augmente sa c., augmente son chagrin.
\item[\vref{Ec 9:10}] ni pensée, ni c., ni sagesse.
\item[\vref{Es 11:2}] force, Esprit de c. et de crainte
\item[\vref{Es 11:9}] remplie de la c. de Yahweh, com.
\item[\vref{Es 53:11}] d'hommes par la c. qu'ils auront de
\item[\vref{Da 12:4}] là, et la c. augmentera.
\item[\vref{Os 4:1}] miséricorde, ni de c. de Dieu ds
\item[\vref{Os 4:6}] qu'il est sans c.. Parce que tu
\item[\vref{Ha 2:14}] remplie de la c. de la gloire
\item[\vref{Lu 1:77}] son peuple la c. du salut, par
\item[\vref{Ac 22:3}] instruit ds la c. exacte de la
\item[\vref{1 Co 1:5}] don de parole, et de tte c.,
\item[\vref{1 Co 8:1}] ts de la c.. La connaissance enfle,
\item[\vref{1 Co 8:10}] as de la c., être à table
\item[\vref{1 Co 12:8}] le mm Esprit, la parole de c. ;
\item[\vref{1 Co 13:8}] langues cesseront, la c. sera abolie.
\item[\vref{2 Co 2:14}] l'odeur de sa c. en tt lieu.
\item[\vref{2 Co 10:5}] s'élève contre la c. de Dieu, et
\item[\vref{Ep 1:17}] révélation, ds ce qui regarde sa c.
\item[\vref{Ep 4:13}] et de la c. du Fils de
\item[\vref{Ph 1:9}] en plus avec c. et tte intelligence,
\item[\vref{Col 1:10}] bonnes œuvres, et croissant ds la c. de Dieu,
\item[\vref{Col 2:2}] intelligence, pour la c. du mystère de
\item[\vref{Col 3:10}] renouvelle ds la c., selon l'image de
\item[\vref{1 Ti 2:4}] viennent à la c. de la vérité.
\item[\vref{Hé 10:26}] avoir reçu la c. de la vérité,
\item[\vref{2 Pi 1:6}] à la c. la tempérance, à la tempérance la
\end{listverse}

\ConcordanceEntry{Connaître}
\vspace{-2mm}
\begin{listverse}
\item[\vref{Ge 15:8}] Yahweh, à quoi c.-je que je
\item[\vref{Ge 32:29}] te prie, fais-moi c. ton Nom. Et
\item[\vref{Ge 45:1}] il se fit c. à ses frères.
\item[\vref{Ex 1:8}] roi sur l'Egypte, qui n'avait point c. Joseph.
\item[\vref{Ex 7:5}] Les Egyptiens c. que je suis
\item[\vref{Ex 33:12}] m'as point fait c. celui que tu
\item[\vref{1 S 2:12}] et ils ne c. pas Yahweh.
\item[\vref{1 S 3:7}] Or Samuel ne c. pas encore Yahweh,
\item[\vref{Est 2:10}] Esther n'avait fait c. ni son peuple
\item[\vref{Job 28:13}] L'hom. ne c. pas sa valeur,
\item[\vref{Ps 19:13}] Qui c. ses fautes commises par erreur ? Purifie-moi
\item[\vref{Ps 51:8}] tu me fais c. la sagesse au-dedans
\item[\vref{Ps 89:16}] le peuple qui c. le son de
\item[\vref{Ps 90:11}] Qui c., selon ta crainte, la force de
\item[\vref{Ps 91:14}] mettrai sur les hauteurs, parce qu'il c. mon Nom.
\item[\vref{Ps 94:11}] Yahweh c. les pensées des hommes qui ne
\item[\vref{Pr 1:23}] je vs. ferai c. mes paroles.
\item[\vref{Pr 14:10}] cœur de chacun c. l'amertume de son
\item[\vref{Pr 20:11}] enfant mm, fait c. par ses actions
\item[\vref{Ec 9:12}] Car l'hom. ne c. pas son heure,
\item[\vref{Es 1:3}] Le bœuf c. son possesseur, et l'âne la crèche
\item[\vref{Es 19:21}] Yahweh se fera c. aux Egyptiens, et
\item[\vref{Es 52:6}] pourquoi mon peuple c. mon Nom ; c'est
\item[\vref{Es 55:13}] et ceci fera c. le Nom de
\item[\vref{Es 63:16}] Abraham ne ns. c. pas, et Israël
\item[\vref{Jé 1:5}] mère, je te c., et avant que
\item[\vref{Jé 9:6}] tromperie, de me c., dit Yahweh.
\item[\vref{Jé 9:24}] et de me c., car je suis
\item[\vref{Jé 17:9}] malin par-dessus tt : Qui peut le c. ?
\item[\vref{Jé 31:34}] frère, en disant : C. Yahweh ! Car ts
\item[\vref{Ez 35:11}] je me ferai c. au milieu d'eux,
\item[\vref{Da 2:45}] Dieu a fait c. au Roi ce
\item[\vref{Os 2:22}] t'épouserai par la fermeté, et tu c. Yahweh.
\item[\vref{Os 6:3}] Alors ns. c. Yahweh, et ns.
\item[\vref{Os 8:2}] Dieu, ns. te c., dira Israël !
\item[\vref{Mi 6:8}] Il t'a fait c. ce qui est
\item[\vref{Mt 7:23}] vs. ai jamais c., retirez-vs. de moi,
\item[\vref{Mt 11:27}] et personne ne c. le Fils si
\item[\vref{Mt 13:11}] été donné de c. les mystères du
\item[\vref{Mt 22:29}] que vs. ne c. ni les Ecritures,
\item[\vref{Mt 25:12}] dis en vérité, je ne vs. c. pas.
\item[\vref{Mt 26:72}] disant : Je ne c. pas cet hom.
\item[\vref{Mc 2:8}] Jésus, ayant aussitôt c. par son esprit
\item[\vref{Lu 21:7}] à quel signe c.-t-on que ces
\item[\vref{Lu 22:34}] n'aies nié trois fois de me c.
\item[\vref{Jn 2:24}] pas à eux, parce qu'il les c. ts,
\item[\vref{Jn 7:28}] disant : Vous me c., et vs. savez
\item[\vref{Jn 8:32}] Vous c. la vérité, et la vérité vs.
\item[\vref{Jn 10:15}] le Père me c., je connais aussi
\item[\vref{Jn 14:7}] Si vs. me c., vs. connaîtriez aussi
\item[\vref{Jn 14:20}] ce jour-là, vs. c. que je suis
\item[\vref{Jn 17:3}] c'est qu'ils te c., toi, le seul
\item[\vref{Jn 17:26}] lr. ai fait c. ton Nom, et
\item[\vref{Ac 1:7}] à vs. de c. les temps et
\item[\vref{Ac 15:8}] Et Dieu, qui c. les cœurs, lr.
\item[\vref{Ro 1:19}] ce qu'on peut c. de Dieu est
\item[\vref{Ro 8:29}] ceux qu'il a c. d'avance, il les
\item[\vref{1 Co 1:21}] sagesse, n'a pas c. Dieu, ds la
\item[\vref{1 Co 2:11}] hommes, en effet, c. les choses de
\item[\vref{1 Co 8:2}] n'a encore rien c. com. il faut
\item[\vref{1 Co 13:9}] Car ns. c. en partie et ns. prophétisons en
\item[\vref{1 Co 13:12}] face. Aujourd'hui je c. en partie, mais
\item[\vref{2 Co 5:21}] qui n'a pas c. le péché, il
\item[\vref{Ep 6:19}] hardiesse, pour faire c. le mystère de
\item[\vref{2 Ti 2:19}] sceau : Le Seign. c. ceux qui lui
\item[\vref{Hé 9:8}] Le Saint-Esprit faisant c. par là que
\item[\vref{1 Pi 1:11}] témoignage, lr. faisant c. les souffrances de
\item[\vref{1 Jn 3:6}] pas vu, et ne l'a pas c.
\item[\vref{1 Jn 3:16}] Nous avons c. la charité, en
\item[\vref{1 Jn 3:20}] cœur, et il c. ttes choses.
\item[\vref{1 Jn 4:7}] prochain est né de Dieu et c. Dieu.
\item[\vref{Ap 1:1}] les a fait c. en les envoyant
\item[\vref{Ap 19:12}] que personne ne c., si ce n'est
\end{listverse}

\ConcordanceEntry{Consacrer}
\vspace{-2mm}
\begin{listverse}
\item[\vref{Ex 13:12}] tu c. à Yahweh tt premier-né issu du
\item[\vref{Ex 28:41}] oindras, tu les c. et tu les
\item[\vref{Ex 32:29}] Moïse avait dit : C. aujourd'hui vos mains
\item[\vref{No 6:8}] naziréat, il sera c. à Yahweh.
\item[\vref{De 15:19}] Tu c. à Yahweh, ton Dieu, tt premier-né
\item[\vref{Jg 11:31}] fils d'Ammon, sera c. à Yahweh et
\item[\vref{Jg 17:12}] Mica c. le Lévite, qui lui servit de
\item[\vref{1 S 7:1}] colline, et ils c. Eléazar, son fils,
\item[\vref{1 S 21:5}] qu'aujourd'hui on en c. de nouveau pour
\item[\vref{2 S 8:11}] roi David les c. à Yahweh, avec
\item[\vref{2 R 23:11}] de Juda avaient c. au soleil, près
\item[\vref{Esd 8:28}] dis : Vous êtes c. à Yahweh ; et
\item[\vref{Ps 2:6}] moi qui ai c. mon Roi sur
\item[\vref{Es 13:3}] qui me sont c., j'ai appelé mes
\item[\vref{Es 27:9}] et les statues c. au soleil ne
\item[\vref{Jé 1:5}] sein, je t'avais c., je t'avais établi
\item[\vref{Jé 31:40}] à l'orient, seront c. à Yahweh, et
\item[\vref{Ez 43:26}] le purifiera et chacun d'eux sera c.
\item[\vref{Os 9:10}] ils se sont c. à l'infâme idole,
\item[\vref{Mi 4:13}] nombreux, et tu c. par interdit leurs
\item[\vref{Za 14:21}] ds Juda, sera c. à Yahweh des
\item[\vref{Mal 2:11}] ce qui est c. à Yahweh, ce
\item[\vref{Ac 15:14}] d'elles un peuple c. à son Nom.
\item[\vref{Hé 2:10}] à la gloire, c. le Prince de
\item[\vref{Hé 5:9}] Après avoir été c., il est devenu
\end{listverse}

\ConcordanceEntry{Conscience}
\vspace{-2mm}
\begin{listverse}
\item[\vref{Jn 8:9}] accusés par lr. c., ils se retirèrent
\item[\vref{Ac 24:16}] avoir toujours une c. pure dvt Dieu,
\item[\vref{Ro 2:15}] leurs cœurs, lr. c. lr. rendant témoignage,
\item[\vref{Ro 13:5}] mais aussi à cause de la c.
\item[\vref{1 Co 8:7}] jusqu'à présent ont c. de l'idole, ils
\item[\vref{1 Co 10:25}] enquérir de rien par motif de c.,
\item[\vref{1 Co 10:29}] non de votre c., mais de celle
\item[\vref{2 Co 1:12}] témoignage de notre c., que ns. ns.
\item[\vref{2 Co 4:2}] approuvés à tte c. d'hom. dvt Dieu,
\item[\vref{2 Co 5:11}] que ds vos c., vs. ns. connaissez
\item[\vref{1 Ti 1:5}] pur, d'une bonne c., et d'une foi
\item[\vref{1 Ti 3:9}] mystère de la foi ds une c. pure.
\item[\vref{1 Ti 4:2}] ayant lr. propre c. marquée au fer
\item[\vref{2 Ti 1:3}] sers avec une c. pure, faisant sans
\item[\vref{Tit 1:15}] entendement et lr. c. sont souillés.
\item[\vref{Hé 10:2}] n'auraient plus eu c. des péchés.
\item[\vref{Hé 13:18}] avons une bonne c., désirant ns. conduire
\item[\vref{1 Pi 2:19}] cause de la c. qu'il a envers
\item[\vref{1 Pi 3:16}] ayant une bonne c., afin que ceux
\item[\vref{1 Pi 3:21}] à Dieu d'une c. pure, et qui
\end{listverse}

\ConcordanceEntry{Consécration}
\vspace{-2mm}
\begin{listverse}
\item[\vref{Ex 29:22}] droite ; car c'est le bélier de c.
\item[\vref{Lé 7:37}] culpabilité, de la c. et du sacrifice
\item[\vref{Lé 8:28}] fut l'offrande de c. de bonne odeur,
\item[\vref{No 6:7}] sa tête la c. de son Dieu.
\end{listverse}

\ConcordanceEntry{Conseil}
\vspace{-2mm}
\begin{listverse}
\item[\vref{Ge 49:6}] point ds lr. c. secret, que ma
\item[\vref{Ex 18:19}] Ecoute dc mon c. ; je te conseillerai
\item[\vref{2 S 15:31}] Yahweh, abolis les c. d'Achitophel !
\item[\vref{2 Ch 10:8}] il laissa le c. que les vieillards
\item[\vref{2 Ch 22:3}] car sa mère lui donnait des c. impies.
\item[\vref{Esd 10:3}] d'elles, selon le c. du Seign., et
\item[\vref{Job 12:13}] lui appartient le c. et l'intelligence.
\item[\vref{Job 26:3}] donnes de bons c. à l'hom. qui
\item[\vref{Ps 1:1}] pas selon le c. des méchants, et
\item[\vref{Ps 33:10}] Yahweh rompt le c. des nations, il
\item[\vref{Ps 33:11}] mais le c. de Yahweh subsiste à toujours, les
\item[\vref{Ps 73:24}] conduiras par ton c., et tu me
\item[\vref{Ps 81:13}] et ils ont suivi leurs propres c.
\item[\vref{Ps 89:8}] terrible ds le c. secret des saints,
\item[\vref{Ps 107:11}] avaient rejeté le c. du Très-Haut.
\item[\vref{Pr 1:25}] rejetez tt mon c., et que vs.
\item[\vref{Pr 8:14}] moi appartiennent le c. et le succès ;
\item[\vref{Pr 12:15}] qui écoute le c. est sage.
\item[\vref{Pr 20:5}] Le c. ds le cœur d'un hom. est
\item[\vref{Pr 21:30}] ni intelligence, ni c., contre Yahweh.
\item[\vref{Es 5:19}] voyions ! Que le c. du Saint d'Israël
\item[\vref{Es 19:11}] Pharaon forment un c. stupide. Comment osez-vs.
\item[\vref{Es 44:26}] et accomplit le c. de ses messagers ;
\item[\vref{Es 45:21}] approcher ! Qu'ils prennent c. ensemble ! Qui a
\item[\vref{Es 46:10}] qui dis : Mon c. tiendra, et j'exécuterai
\item[\vref{Jé 7:24}] ont suivi d'autres c., les penchants de
\item[\vref{Jé 32:19}] es grand en c. et puissant en
\item[\vref{Ez 11:2}] donnent un mauvais c. ds cette ville.
\item[\vref{Da 4:27}] roi, que mon c. te soit agréable :
\item[\vref{Mi 6:5}] te prie, du c. que Balak, roi
\item[\vref{Mt 5:22}] puni par le c. ; et celui qui
\item[\vref{Jn 18:14}] avait donné ce c. aux Juifs : Il
\item[\vref{Ac 4:28}] main et ton c. avaient auparavant déterminé
\item[\vref{Ac 20:27}] annoncé tt le c. de Dieu, sans
\item[\vref{1 Co 7:6}] dis ceci par c., et non par
\item[\vref{Ep 1:11}] efficacité selon le c. de sa volonté,
\end{listverse}

\ConcordanceEntry{Conseil de paix}
\vspace{-2mm}
\begin{listverse}
\item[\vref{Za 6:13}] y aura un conseil de paix entre les deux.
\end{listverse}

\ConcordanceEntry{Conseiller (un)}
\vspace{-2mm}
\begin{listverse}
\item[\vref{1 Ch 27:33}] Achitophel était le c. du roi ; et
\item[\vref{2 Ch 22:4}] qu'ils furent ses c. après la mort
\item[\vref{Esd 4:5}] prix d'argent des c. pour faire échouer
\item[\vref{Esd 7:15}] roi et ses c. ont offert volontairement
\item[\vref{Job 3:14}] rois et les c. de la terre,
\item[\vref{Job 12:17}] emmène dépouillés les c. et il met
\item[\vref{Ps 119:24}] font mes délices, ce sont mes c.
\item[\vref{Pr 11:14}] délivrance est ds la multitude de c.
\item[\vref{Pr 15:22}] la fermeté ds la multitude des c.
\item[\vref{Pr 24:6}] consiste ds le grand nombre des c.
\item[\vref{Es 9:5}] l'appellera l'Admirable, le C., le Dieu Puissant,
\item[\vref{Mi 4:9}] toi ? Ou ton c. est-il mort, que
\item[\vref{Mc 15:43}] arriva Joseph d'Arimathée, c. honorable, qui attendait
\item[\vref{Ro 11:34}] Seign., ou qui a été son c. ?
\end{listverse}

\ConcordanceEntry{Considérer}
\vspace{-2mm}
\begin{listverse}
\item[\vref{Ge 30:33}] les agneaux, sera c. com. un vol
\item[\vref{Ex 16:29}] C. que Yahweh vs. a ordonné le
\item[\vref{2 R 5:1}] puissant et très c. aux yeux de
\item[\vref{Né 13:13}] parce qu'ils étaient c. com. très fidèles.
\item[\vref{Job 1:8}] Satan : N'as-tu point c. mon serviteur Job,
\item[\vref{Ps 73:17}] et que j'aie c. la fin de
\item[\vref{Ec 3:10}] J'ai c. cette occupation que Dieu a donnée
\item[\vref{Es 53:4}] et ns. l'avons c. com. frappé, battu
\item[\vref{Jé 2:31}] de cette génération, c. la parole de
\item[\vref{Ag 1:5}] Yahweh des armées : C. attentivement votre conduite !
\item[\vref{Mt 6:26}] C. les oiseaux du ciel ; car ils
\item[\vref{Ac 7:31}] il approchait pour c. ce que c'était,
\item[\vref{Ro 1:20}] qnd on les c. ds ses ouvrages,
\item[\vref{Ro 6:11}] mm, vs. aussi, c.-vs. com. morts
\item[\vref{1 Co 1:26}] Car, mes frères, c. que parmi vs.
\item[\vref{2 Co 10:7}] C.-vs. les choses selon l'apparence ? Si
\item[\vref{Ga 2:2}] sont les plus c., afin de ne
\item[\vref{1 Th 2:4}] Dieu ns. a c. dignes de ns.
\item[\vref{Hé 3:1}] la vocation céleste, c. attentivement Jésus-Christ, l'Apôtre
\item[\vref{Hé 7:4}] Et c. dc combien est grand celui à
\item[\vref{Hé 12:3}] C'est pourquoi, c. soigneusement celui qui
\item[\vref{Hé 13:7}] parole de Dieu ; c. quelle a été
\item[\vref{2 Pi 3:15}] Et c. la patience du Seign. com. une
\end{listverse}

\ConcordanceEntry{Consolateur}
\vspace{-2mm}
\begin{listverse}
\item[\vref{2 S 10:3}] David t'envoie des c. ? N'est-ce pas pour
\item[\vref{Job 16:2}] pareils discours ; vs. êtes ts des c. fâcheux.
\item[\vref{Ps 69:21}] J'ai attendu des c., mais je n'en
\item[\vref{Ec 4:1}] tort, et ils n'ont point de c.
\item[\vref{La 1:16}] larmes ; car le c. qui restaurait ma
\item[\vref{Na 3:7}] compassion d'elle ? D'où te chercherai-je des c. ?
\item[\vref{Jn 14:16}] donnera un autre c., pour demeurer avec
\item[\vref{Jn 14:26}] Mais le C., qui est le Saint-Esprit, que le
\item[\vref{Jn 15:26}] sera venu le C., que je vs.
\item[\vref{Jn 16:7}] m'en vais, le C. ne viendra pas
\end{listverse}

\ConcordanceEntry{Consolation}
\vspace{-2mm}
\begin{listverse}
\item[\vref{Job 15:11}] Les c. de Dieu te semblent-elles trop petites ?
\item[\vref{Ps 94:19}] de moi, tes c. font les délices
\item[\vref{Ps 119:76}] bonté soit ma c., com. tu l'as
\item[\vref{Ec 4:1}] n'ont point de c.. Et la force
\item[\vref{Lu 2:25}] il attendait la c. d'Israël, et le
\item[\vref{Ac 15:31}] réjouis de la c. qu'elle lr. apportait.
\item[\vref{Ac 20:12}] ce fut le sujet d'une grande c.
\item[\vref{Ro 15:4}] patience, et la c. des Ecritures, ns.
\item[\vref{2 Co 1:3}] miséricordes et le Dieu de tte c.,
\item[\vref{2 Co 1:6}] c'est pour votre c. et votre salut,
\item[\vref{2 Co 7:13}] fait pour notre c.. Mais ns. ns.
\item[\vref{Ph 2:1}] y a qq c. en Christ, s'il
\item[\vref{1 Th 3:7}] ns. une grande c. à cause de
\item[\vref{2 Th 2:16}] a donné une c. éternelle, et une
\item[\vref{Hé 6:18}] ayons une ferme c., ns. qui avons
\end{listverse}

\ConcordanceEntry{Consoler}
\vspace{-2mm}
\begin{listverse}
\item[\vref{Ge 5:29}] disant : Celui-ci ns. c. de notre œuvre,
\item[\vref{Ge 24:67}] Ainsi Isaac fut c. après la mort
\item[\vref{Ge 37:35}] vinrent pour le c., mais il rejeta
\item[\vref{Ge 50:21}] et il les c. en parlant à
\item[\vref{Ru 2:13}] car tu m'as c., et tu as
\item[\vref{2 S 12:24}] David c. sa fem. Bath-Schéba, et il alla
\item[\vref{Job 2:11}] pour venir le plaindre et le c.
\item[\vref{Job 42:11}] état, ils le c. de tt le
\item[\vref{Ps 23:4}] Ton bâton et ta houlette me c.
\item[\vref{Ps 71:21}] ma grandeur et c.-moi encore !
\item[\vref{Ps 86:17}] tu m'aideras, ô Yahweh ! Tu me c. !
\item[\vref{Es 12:1}] colère s'est apaisée, et tu m'as c.
\item[\vref{Es 40:1}] C., consolez mon peuple, dit votre Dieu.
\item[\vref{Es 51:3}] Car Yahweh c. Sion, il consolera
\item[\vref{Es 51:12}] moi qui vs. c.. Qui es-tu pour
\item[\vref{Es 61:2}] notre Dieu ; pour c. ts ceux qui
\item[\vref{Es 66:13}] Je vs. c. pour vs. apaiser, com. quelqu'un que
\item[\vref{Jé 31:15}] elle refuse d'être c. sur ses fils,
\item[\vref{La 1:2}] amis qui la c. ; ses intimes amis
\item[\vref{Za 1:17}] biens, et Yahweh c. encore Sion, et
\item[\vref{Za 10:2}] songes vains et c. par la vanité.
\item[\vref{Mt 2:18}] pas voulu être c., parce qu'ils ne
\item[\vref{Mt 5:4}] ceux qui pleurent, car ils seront c. !
\item[\vref{Lu 16:25}] il est ici c., et toi, tu
\item[\vref{Jn 11:19}] Marie pour les c. au sujet de
\item[\vref{1 Co 14:3}] édifie, exhorte et c. les hommes qui
\item[\vref{1 Co 14:31}] soient instruits et que ts soient c.
\item[\vref{2 Co 1:4}] qui ns. c. ds tte notre affliction, afin que
\item[\vref{2 Co 7:6}] Mais Dieu qui c. les abattus ns.
\item[\vref{2 Co 13:11}] frères, réjouissez-vs., perfectionnez-vs., c.-vs., ayez un
\item[\vref{Ep 6:22}] et pour qu'il c. vos cœurs.
\item[\vref{Col 2:2}] leurs cœurs soient c., étant unis ensemble
\item[\vref{1 Th 4:18}] C'est pourquoi c.-vs. les uns
\item[\vref{1 Th 5:14}] ds le désordre, c. ceux qui ont
\item[\vref{2 Th 2:17}] c. vos cœurs, et vs. affermisse en
\end{listverse}

\ConcordanceEntry{Constamment}
\vspace{-2mm}
\begin{listverse}
\item[\vref{Ps 16:8}] J'ai c. Yahweh sous mes yeux ; et puisqu'il
\item[\vref{Ps 74:3}] vers les lieux c. dévastés ! L'ennemi a
\item[\vref{Jé 52:33}] mangea du pain c. en sa présence,
\item[\vref{Da 6:16}] que tu sers c. sera celui qui
\item[\vref{Ac 2:25}] lui : Je contemplais c. le Seign. dvt
\item[\vref{1 Ti 5:10}] de s'être ainsi c. appliquée à ttes
\item[\vref{Hé 12:1}] aisément, et poursuivons c. la course qui
\end{listverse}

\ConcordanceEntry{Consulter}
\vspace{-2mm}
\begin{listverse}
\item[\vref{Ge 25:22}] pourquoi suis-je enceinte ? Et elle alla c. Yahweh.
\item[\vref{De 18:11}] d'enchantements, personne qui c. les médiums ou
\item[\vref{Jg 20:18}] Béthel pour le c., en disant : Qui
\item[\vref{1 S 10:22}] On c. de nouveau Yahweh : Est-il encore venu
\item[\vref{1 S 22:10}] Il a c. Yahweh pour lui, il lui a
\item[\vref{1 S 23:2}] Et David c. Yahweh en disant : Irai-je, et frapperai-je
\item[\vref{1 S 28:16}] Pourquoi dc me c.-tu, puisque Yahweh
\item[\vref{2 S 5:23}] David c. Yahweh. Et Yahweh dit : Tu ne
\item[\vref{2 S 20:18}] dire : Que l'on c. Abel ! Et tt
\item[\vref{1 R 12:6}] Le roi Roboam c. les vieillards qui
\item[\vref{1 R 22:7}] de Yahweh, afin que ns. le c. ?
\item[\vref{2 R 1:2}] lr. dit : Allez, c. Baal-Zebub, dieu d'Ekron,
\item[\vref{Esd 2:63}] qu'un prêtre ait c. l'urim et le
\item[\vref{Né 7:65}] le prêtre eût c. l'urim et le
\item[\vref{Ps 13:3}] Jusqu'à qnd c.-je mon âme,
\item[\vref{Es 8:19}] l'on vs. dit : C. ceux qui évoquent
\item[\vref{Es 19:3}] conseil ; et ils c. les idoles et
\item[\vref{Es 34:16}] C. le livre de Yahweh et lisez :
\item[\vref{Jé 21:2}] C. mntnt Yahweh pour ns. ; car Nebucadnetsar,
\item[\vref{Ez 14:3}] ds l'iniquité. Serais-je c. par eux sérieusement ?
\item[\vref{So 1:6}] cherché Yahweh, qui ne l'ont point c.
\item[\vref{Mt 12:14}] et ils se c. sur les moyens
\item[\vref{Mt 22:15}] pharisiens allèrent se c. ensemble sur les
\item[\vref{Mc 3:6}] aussitôt, ils se c. contre lui avec
\item[\vref{Ac 5:33}] les dents, et c. pour les faire
\item[\vref{Ga 1:16}] aussitôt, je ne c. ni la chair
\end{listverse}

\ConcordanceEntry{Consumer}
\vspace{-2mm}
\begin{listverse}
\item[\vref{Ex 3:2}] feu, et le buisson ne se c. point.
\item[\vref{Ex 33:3}] je ne te c. en chemin.
\item[\vref{Ps 39:4}] feu intérieur me c., et la parole
\item[\vref{Ps 90:9}] nos années se c. ds un soupir.
\item[\vref{Ps 119:139}] Mon zèle me c. parce que mes
\item[\vref{Ps 143:7}] Mon esprit se c. ! Ne me cache
\item[\vref{Pr 5:11}] ta chair et ton corps seront c. ;
\item[\vref{Pr 13:23}] tel qui est c. faute de règles.
\item[\vref{Es 1:28}] et ceux qui abandonnent Yahweh seront c.
\item[\vref{Jé 14:15}] Ces prophètes-là seront c. par l'épée et
\item[\vref{Na 3:6}] abominations, je te c. et je te
\item[\vref{So 1:18}] se hâtera de c. ts les habitants
\item[\vref{Za 9:4}] et elle sera c. par le feu.
\item[\vref{Mal 3:6}] de Jacob, vs. n'avez point été c.
\item[\vref{Lu 9:54}] ciel, et les c., com. fit Elie ?
\end{listverse}

\ConcordanceEntry{Contempler}
\vspace{-2mm}
\begin{listverse}
\item[\vref{No 23:9}] et je le c. du haut des
\item[\vref{Job 35:5}] les cieux, et c.-les ! Vois les
\item[\vref{Job 36:25}] Tout hom. les voit, chacun les c. de loin.
\item[\vref{Ps 11:7}] aime la justice ; les hommes droits c. sa face.
\item[\vref{Ps 27:4}] ma vie, pour c. la beauté de
\item[\vref{Ps 46:9}] Venez, c. les œuvres de Yahweh, et voyez
\item[\vref{Ps 63:3}] Ainsi je te c. ds ton lieu
\item[\vref{Ps 142:5}] Je c. à ma droite, et je regarde !
\item[\vref{Es 33:17}] Tes yeux c. le roi ds sa beauté ; et
\item[\vref{Es 47:13}] des cieux qui c. les étoiles, et
\item[\vref{Jn 1:14}] et ns. avons c. sa gloire, une
\item[\vref{Jn 7:3}] tes disciples aussi c. les œuvres que
\item[\vref{Jn 17:24}] suis, afin qu'ils c. la gloire que
\item[\vref{Ac 1:11}] que vs. l'avez c. montant au ciel.
\item[\vref{Ac 2:25}] de lui : Je c. constamment le Seign.
\item[\vref{2 Co 3:18}] ns. ts qui c., com. ds un
\item[\vref{1 Jn 1:1}] que ns. avons c. et que nos
\end{listverse}

\ConcordanceEntry{Content}
\vspace{-2mm}
\begin{listverse}
\item[\vref{2 R 14:10}] cœur s'est élevé. C.-toi de ta
\item[\vref{Est 5:9}] et le cœur c.. Mais aussitôt qu'il
\item[\vref{Es 57:6}] offrandes ; puis-je être c. de ces choses ?
\item[\vref{Ph 4:11}] appris à être c. de l'état où
\item[\vref{Hé 13:5}] sans avarice, étant c. de ce que
\item[\vref{3 Jn 1:10}] et n'étant pas c. de cela, non
\end{listverse}

\ConcordanceEntry{Contenter}
\vspace{-2mm}
\begin{listverse}
\item[\vref{Ex 22:11}] la bête se c. du serment, et
\item[\vref{2 R 14:10}] cœur s'est élevé. C.-toi de ta
\item[\vref{Lu 3:14}] envers personne, mais c.-vs. de votre
\item[\vref{Hé 13:5}] sans avarice, étant c. de ce que
\end{listverse}

\ConcordanceEntry{Contester}
\vspace{-2mm}
\begin{listverse}
\item[\vref{Ge 6:3}] Mon Esprit ne c. point à toujours
\item[\vref{No 20:3}] Et le peuple c. contre Moïse et
\item[\vref{No 20:13}] les enfants d'Israël c. avec Yahweh, qui
\item[\vref{De 33:8}] qui tu as c. aux eaux de
\item[\vref{1 S 2:10}] Ceux qui c. contre Yahweh seront effrayés ; des cieux
\item[\vref{Job 23:6}] C.-il avec moi par la grandeur
\item[\vref{Job 39:35}] Celui qui c. avec le Tout-Puissant,
\item[\vref{Ps 103:9}] Il ne c. pas éternellement, et il ne garde
\item[\vref{Pr 3:30}] Ne c. pas sans motif avec quelqu'un, à
\item[\vref{Es 57:16}] ne veux pas c. à toujours, et
\item[\vref{Jé 2:9}] je veux encore c. avec vs., dit
\item[\vref{Jé 12:1}] Yahweh, qnd je c. avec toi, tu
\item[\vref{Jé 25:31}] nations, et il c. contre tte chair.
\item[\vref{Ez 20:35}] peuples, et je c. là contre vs.,
\item[\vref{Mt 12:19}] Il ne c. pas, il ne criera pas et
\item[\vref{Ac 23:9}] se levèrent et c., disant : Nous ne
\item[\vref{Ro 9:20}] es-tu, toi qui c. contre Dieu ? Le
\item[\vref{1 Co 11:16}] quelqu'un aime à c., ns. n'avons pas
\item[\vref{Jud 1:9}] l'archange Michel, lorsqu'il c. avec le diable
\end{listverse}

\ConcordanceEntry{Contradiction}
\vspace{-2mm}
\begin{listverse}
\item[\vref{Lu 2:34}] être un signe qui provoquera la c.,
\end{listverse}

\ConcordanceEntry{Contredire}
\vspace{-2mm}
\begin{listverse}
\item[\vref{Lu 21:15}] adversaires ne pourront c., ni résister.
\item[\vref{Ac 4:14}] présent avec eux, ils ne pouvaient c. en rien.
\item[\vref{Ac 28:22}] il ns. est connu qu'on la c. partout.
\item[\vref{1 Ti 3:16}] Et sans c., le mystère de la piété est
\item[\vref{Tit 2:8}] celui qui vs. c., soit rendu confus,
\item[\vref{Hé 7:7}] Or sans c., celui qui est le moindre est
\end{listverse}

\ConcordanceEntry{Convaincre}
\vspace{-2mm}
\begin{listverse}
\item[\vref{Job 9:20}] parfait, il me c. d'être coupable.
\item[\vref{Job 24:25}] ainsi, qui me c. de mensonge, qui
\item[\vref{Jn 8:46}] de vs. me c. de péché ? Et
\item[\vref{Jn 16:8}] sera venu, il c. le monde de
\item[\vref{2 Ti 3:16}] pour enseigner, pour c., pour corriger et
\item[\vref{Jud 1:15}] hommes, et pour c. ts les méchants
\end{listverse}

\ConcordanceEntry{Conversion}
\vspace{-2mm}
\begin{listverse}
\item[\vref{Ac 15:3}] Samarie, racontant la c. des Gentils ; et
\item[\vref{Ac 26:20}] repentance et la c. à Dieu, avec
\end{listverse}

\ConcordanceEntry{Convertir}
\vspace{-2mm}
\begin{listverse}
\item[\vref{Ps 7:13}] méchant ne se c. pas, Dieu aiguise
\item[\vref{Ps 22:28}] souviendront, ils se c. à Yahweh, et
\item[\vref{Es 1:27}] ceux qui s'y c. seront rachetés par
\item[\vref{Es 6:10}] qu'il ne se c., et qu'il ne
\item[\vref{Es 10:21}] Le reste se c., le reste, dis-je,
\item[\vref{Es 59:20}] Jacob qui se c. de lr. péché,
\item[\vref{Jé 3:14}] Enfants rebelles, c.-vs., dit Yahweh,
\item[\vref{Jé 5:3}] qu'un rocher, ils refusent de se c.
\item[\vref{Jé 8:5}] tromperie, et ils refusent de se c.
\item[\vref{La 5:21}] C.-ns. à toi, ô Yahweh ! Et
\item[\vref{Ez 18:32}] le Seign. Yahweh. C.-vs. dc, et
\item[\vref{Os 7:10}] se sont pas c. à Yahweh, lr.
\item[\vref{Za 1:6}] sorte que s'étant c., ils ont dit :
\item[\vref{Mal 3:18}] C.-vs. dc, et vs. verrez la
\item[\vref{Mt 18:3}] vs. ne vs. c. pas et si
\item[\vref{Mc 4:12}] qu'ils ne se c., et que leurs
\item[\vref{Lu 22:32}] seras un jour c., affermis tes frères.
\item[\vref{Ac 3:19}] Repentez-vs. dc et c.-vs., afin que
\item[\vref{Ac 9:35}] et ils se c. au Seign.
\item[\vref{Ac 11:21}] crurent et se c. au Seign.
\item[\vref{Ac 14:15}] vaines, pour vs. c. au Dieu vivant,
\item[\vref{Ac 15:19}] à ceux des Gentils qui se c. à Dieu ;
\item[\vref{Ro 11:12}] plus en sera-t-il qnd ils se c. ts ?
\item[\vref{2 Co 3:16}] le cœur se c. au Seign., le
\item[\vref{1 Th 1:9}] vs. vs. êtes c. à Dieu, en
\item[\vref{1 Ti 3:6}] soit un nouveau c., de peur qu'enflé
\end{listverse}

\ConcordanceEntry{Convocation}
\vspace{-2mm}
\begin{listverse}
\item[\vref{Ex 12:16}] aura une sainte c., et il y
\item[\vref{Lé 23:3}] aura une sainte c.. Vous ne ferez
\item[\vref{No 29:7}] aurez une sainte c., et vs. affligerez
\item[\vref{Es 1:13}] publication de vos c. ; je ne puis
\end{listverse}

\ConcordanceEntry{Convoiter}
\vspace{-2mm}
\begin{listverse}
\item[\vref{Ex 20:17}] Tu ne c. pas la maison de ton prochain ;
\item[\vref{No 11:34}] ensevelit là le peuple qui avait c.
\item[\vref{De 5:21}] Tu ne c. point la fem. de ton prochain ;
\item[\vref{De 7:25}] dieux. Tu ne c. point et tu
\item[\vref{Jos 7:21}] je les ai c., je les ai
\item[\vref{Job 20:20}] rien de ce qu'il aura tant c.
\item[\vref{Pr 6:25}] Ne c. pas en ton cœur sa beauté
\item[\vref{Es 13:17}] et qui ne c. point l'or.
\item[\vref{Mi 2:2}] Ils c. des possessions et s'en emparent, des
\item[\vref{Mi 7:3}] déclare ce qu'il c., et ils s'unissent.
\item[\vref{Mt 5:28}] fem. pour la c. a déjà commis
\item[\vref{Ro 7:7}] loi n'avait pas dit : Tu ne c. pas.
\item[\vref{Ro 13:9}] pas, tu ne c. pas, et tt
\item[\vref{1 Co 10:6}] que ns. ne c. pas des choses
\item[\vref{Ja 4:2}] Vous c., et vs. n'obtenez pas ce que
\end{listverse}

\ConcordanceEntry{Convoitise}
\vspace{-2mm}
\begin{listverse}
\item[\vref{No 11:4}] fut épris de c. ; et mm, les
\item[\vref{Ps 106:14}] furent épris de c. au désert et
\item[\vref{Mc 4:19}] richesses et les c. des autres choses
\item[\vref{Ro 1:24}] a livrés aux c. de leurs propres
\item[\vref{Ro 6:12}] pour que vs. obéissiez à ses c.
\item[\vref{Ro 7:7}] pas connu la c., si la loi
\item[\vref{Ro 13:14}] de la chair pour accomplir ses c.
\item[\vref{Ep 2:3}] autrefois, selon les c. de notre chair,
\item[\vref{Ep 4:22}] corrompt par les c. qui séduisent ;
\item[\vref{1 Th 4:5}] désirs de la c., com. les Gentils
\item[\vref{2 Ti 3:6}] de péchés et agitées de diverses c.,
\item[\vref{Tit 3:3}] tte espèce de c. et de voluptés,
\item[\vref{Ja 1:14}] attiré et amorcé par sa propre c.
\item[\vref{Ja 1:15}] Puis qnd la c. a conçu, elle
\item[\vref{1 Pi 2:11}] vs. abstenir des c. charnelles qui font
\item[\vref{1 Pi 4:3}] aux impudicités, aux c., à l'ivrognerie, aux
\item[\vref{2 Pi 1:4}] règne ds le monde par la c.
\item[\vref{2 Pi 3:3}] moqueurs, se conduisant selon leurs propres c.,
\item[\vref{1 Jn 2:16}] monde, c'est-à-dire la c. de la chair,
\item[\vref{Jud 1:16}] marchent selon leurs c., dont la bouche
\end{listverse}

\ConcordanceEntry{Coq}
\vspace{-2mm}
\begin{listverse}
\item[\vref{Mt 26:34}] avant que le c. ait chanté, tu
\item[\vref{Mc 13:35}] l'heure où le c. chante, ou le
\item[\vref{Lu 22:34}] dis que le c. ne chantera pas
\item[\vref{Lu 22:60}] instant, com. il parlait encore, le c. chanta.
\item[\vref{Jn 18:27}] nia de nouveau. Et aussitôt le c. chanta.
\end{listverse}

\ConcordanceEntry{Corbeau}
\vspace{-2mm}
\begin{listverse}
\item[\vref{Ge 8:7}] il lâcha le c., qui sortit, allant
\item[\vref{Lé 11:15}] tt c., selon son espèce ;
\item[\vref{De 14:14}] le c., selon son espèce ;
\item[\vref{1 R 17:4}] j'ai commandé aux c. de t'y nourrir.
\item[\vref{1 R 17:6}] Les c. lui apportaient du pain et de
\item[\vref{Job 39:3}] la nourriture au c., qnd ses petits
\item[\vref{Ps 147:9}] aux petits du c. qui crient.
\item[\vref{Pr 30:17}] sa mère, les c. des torrents le
\item[\vref{Es 34:11}] chouette et le c. y habiteront ; et
\item[\vref{Lu 12:24}] Considérez les c., ils ne sèment,
\end{listverse}

\ConcordanceEntry{Corbeille}
\vspace{-2mm}
\begin{listverse}
\item[\vref{Ge 40:16}] mon songe trois c. de pain blanc
\item[\vref{Lé 8:26}] aussi de la c. des pains sans
\item[\vref{No 6:17}] Yahweh, avec la c. des pains sans
\item[\vref{De 26:2}] mettras ds une c., et tu iras
\item[\vref{De 28:5}] Ta c. et ta huche seront bénies.
\item[\vref{2 R 10:7}] têtes ds des c., ils les envoyèrent
\item[\vref{Ps 81:7}] et ses mains ont lâché les c.
\item[\vref{Mt 16:10}] et combien de c. vs. avez emportées ?
\item[\vref{Mc 8:8}] l'on remporta sept c. pleines des morceaux
\item[\vref{Ac 9:25}] descendirent par la muraille ds une c.
\item[\vref{2 Co 11:33}] muraille ds une c., par une fenêtre
\end{listverse}

\ConcordanceEntry{Cordeau}
\vspace{-2mm}
\begin{listverse}
\item[\vref{2 S 8:2}] les mesura au c., en les faisant
\item[\vref{Jé 31:39}] Le c. à mesurer sera encore tiré vis-à-vis
\item[\vref{La 2:8}] a étendu le c., il n'a pas
\item[\vref{Ez 40:3}] sa main un c. de lin, et
\item[\vref{Ez 47:3}] sa main un c. ; et il mesura
\item[\vref{Am 7:17}] sera partagé au c., et toi, tu
\item[\vref{Mi 2:5}] qui jettera le c. pour ton lot,
\item[\vref{Za 1:16}] armées ; et le c. sera étendu sur
\item[\vref{Za 2:1}] la main un c. pour mesurer,
\end{listverse}

\ConcordanceEntry{Corinthe}
\vspace{-2mm}
\begin{listverse}
\item[\vref{Ac 18:1}] partit d'Athènes, et se rendit à C.
\item[\vref{Ac 18:18}] assez longtemps à C.. Ensuite il prit
\item[\vref{1 Co 1:2}] qui est à C., à ceux qui
\end{listverse}

\ConcordanceEntry{Corne}
\vspace{-2mm}
\begin{listverse}
\item[\vref{Ge 22:13}] buisson par ses c. ; et Abraham alla
\item[\vref{Lé 4:7}] sang sur les c. de l'autel des
\item[\vref{Jos 6:5}] sonneront avec la c. de bélier, aussitôt
\item[\vref{1 S 16:1}] Israël ? Remplis ta c. d'huile, et viens ;
\item[\vref{1 S 16:13}] Samuel prit la c. d'huile, et l'oignit
\item[\vref{Ps 92:11}] tu élèveras ma c. com. celle d'un
\item[\vref{Ps 112:9}] perpétuité ; [Qof.] sa c. s'élève en gloire.
\item[\vref{Ps 118:27}] et amenez-la jusqu'aux c. de l'autel.
\item[\vref{Ps 132:17}] elle germera une c. à David ; je
\item[\vref{Ez 29:21}] ferai germer la c. de la maison
\item[\vref{Da 7:7}] avant elle, et elle avait dix c.
\item[\vref{Da 7:24}] Mais les dix c. sont dix rois
\item[\vref{Da 8:8}] puissant, sa grande c. se brisa. Quatre
\item[\vref{Am 3:14}] Béthel ; et les c. de l'autel seront
\item[\vref{Mi 4:13}] te ferai une c. de fer, et
\item[\vref{Za 1:18}] et je regardai ; et voici, quatre c.
\item[\vref{Za 11:16}] il déchirera jusqu'aux c. de leurs pieds.
\item[\vref{Ap 5:6}] immolé, ayant sept c., et sept yeux,
\item[\vref{Ap 12:3}] têtes et dix c., et sur ses
\item[\vref{Ap 17:3}] blasphème, ayant sept têtes et dix c.
\item[\vref{Ap 17:16}] Les dix c. que tu as vues sur la
\end{listverse}

\ConcordanceEntry{Corneille}
\vspace{-2mm}
\begin{listverse}
\item[\vref{Ac 10:1}] un hom. nommé C., centenier d'une cohorte
\item[\vref{Ac 11:12}] ns. entrâmes ds la maison de C.
\end{listverse}

\ConcordanceEntry{Corps}
\vspace{-2mm}
\begin{listverse}
\item[\vref{Ps 16:9}] réjouit et mon c. repose en sécurité.
\item[\vref{Ps 109:24}] jeûne, et mon c. est épuisé de
\item[\vref{Ps 139:15}] Mon c. n'était pas caché dvt toi, lorsque
\item[\vref{Es 51:23}] as exposé ton c. com. la terre,
\item[\vref{Da 3:28}] et livré lr. c. plutôt que de
\item[\vref{Mt 5:29}] et que ton c. entier ne soit
\item[\vref{Mt 6:25}] ni pour votre c., de quoi vs.
\item[\vref{Mt 10:28}] qui tuent le c. et qui ne
\item[\vref{Mt 26:26}] dit : Prenez, mangez, ceci est mon c.
\item[\vref{Mt 27:52}] s'ouvrirent et plusieurs c. des saints qui
\item[\vref{Mt 27:58}] et demanda le c. de Jésus. En
\item[\vref{Lu 17:37}] où est le c., là aussi s'assembleront
\item[\vref{Lu 24:3}] trouvèrent pas le c. du Seign. Jésus.
\item[\vref{Jn 2:21}] il parlait du temple de son c.
\item[\vref{Jn 11:52}] en un seul c. les enfants de
\item[\vref{Ac 9:40}] tournant vers le c., il dit : Tabitha,
\item[\vref{Ac 19:12}] avaient touché son c., et ils étaient
\item[\vref{Ro 1:24}] l'impureté, déshonorant entre eux-mêmes leurs propres c. ;
\item[\vref{Ro 4:19}] égard à son c. qui était déjà
\item[\vref{Ro 6:6}] afin que le c. du péché soit
\item[\vref{Ro 7:4}] été, par le c. de Christ, mis
\item[\vref{Ro 8:10}] en vs., le c. est bien mort
\item[\vref{Ro 8:11}] vie à vos c. mortels à cause
\item[\vref{Ro 8:23}] l'adoption, c'est-à-dire la rédemption de notre c.
\item[\vref{Ro 12:1}] à offrir vos c. en sacrifice vivant,
\item[\vref{Ro 12:5}] formons un seul c. en Christ, et
\item[\vref{1 Co 5:3}] étant absent de c., mais présent en
\item[\vref{1 Co 6:13}] autres. Or le c. n'est pas pour
\item[\vref{1 Co 6:19}] pas que votre c. est le temple
\item[\vref{1 Co 7:4}] sur son propre c., mais c'est son
\item[\vref{1 Co 7:34}] d'être sainte de c. et d'esprit ; mais
\item[\vref{1 Co 9:27}] je mortifie mon c. et je le
\item[\vref{1 Co 10:16}] la communion du c. de Christ ?
\item[\vref{1 Co 12:12}] Car com. le c. est un, et
\item[\vref{1 Co 12:27}] Vous êtes le c. de Christ, et
\item[\vref{1 Co 13:3}] je livrerais mon c. pour être brûlé,
\item[\vref{1 Co 15:35}] et avec quel c. viennent-ils ?
\item[\vref{1 Co 15:38}] lui donne le c. com. il veut,
\item[\vref{1 Co 15:40}] a aussi des c. célestes, et des
\item[\vref{1 Co 15:44}] il est semé c. animal, il ressuscitera
\item[\vref{2 Co 4:10}] partout ds notre c. la mort du
\item[\vref{2 Co 5:6}] demeurant ds ce c., ns. sommes loin
\item[\vref{2 Co 12:2}] fut ds le c. je ne sais,
\item[\vref{Ga 6:17}] porte sur mon c. les marques du
\item[\vref{Ep 1:23}] qui est son c., et la plénitude
\item[\vref{Ep 2:16}] former un seul c. par sa croix,
\item[\vref{Ep 4:4}] a un seul c., un seul Esprit,
\item[\vref{Ep 4:16}] dont tt le c. bien ajusté et
\item[\vref{Ph 3:21}] qui transformera notre c. vil pour le
\item[\vref{Col 1:18}] le Chef du c. de l'Eglise, et
\item[\vref{Col 1:22}] par le c. de sa chair, par sa mort,
\item[\vref{Col 2:11}] à dépouiller le c. des péchés de
\item[\vref{Col 2:19}] dont tt le c. étant joint et
\item[\vref{1 Th 4:4}] sache posséder son c. ds la sanctification
\item[\vref{1 Th 5:23}] l'âme et le c. soient conservés sans
\item[\vref{Hé 10:5}] d'offrande, mais tu m'as formé un c. ;
\item[\vref{Hé 10:10}] par l'offrande du c. de Jésus-Christ qui
\item[\vref{Ja 2:26}] Car, com. le c. sans esprit est
\item[\vref{1 Pi 2:24}] péchés ds son c. sur le bois,
\item[\vref{Jud 1:9}] lui disputait le c. de Moïse, n'osa
\end{listverse}

\ConcordanceEntry{Correction}
\vspace{-2mm}
\begin{listverse}
\item[\vref{Lé 26:23}] recevez pas ma c., et que vs.
\item[\vref{Job 20:3}] J'ai entendu la c. dont tu veux
\item[\vref{Ps 50:17}] qui hais la c., et qui jettes
\item[\vref{Pr 5:12}] pu haïr la c., et comment mon
\item[\vref{Pr 10:17}] vie ; mais celui qui néglige la c. s'égare.
\item[\vref{Pr 12:1}] qui aime la c. aime la connaissance,
\item[\vref{Pr 15:31}] qui écoute la c. qui donne la
\item[\vref{Pr 22:15}] verge de la c. l'éloignera de lui.
\item[\vref{Pr 23:13}] N'éloigne pas la c. du jeune enfant ;
\item[\vref{Os 5:9}] jour de la c. ; je le fais
\item[\vref{2 Co 2:6}] hom., de la c. qui lui a
\end{listverse}

\ConcordanceEntry{Corriger}
\vspace{-2mm}
\begin{listverse}
\item[\vref{Pr 9:7}] Celui qui c. le moqueur en
\item[\vref{Pr 15:4}] La langue qui c. le prochain est
\item[\vref{Pr 29:17}] C. ton fils, et il te mettra
\item[\vref{Pr 29:19}] L'esclave ne se c. pas par des
\item[\vref{Jé 7:3}] le Dieu d'Israël : C. vos voies et
\item[\vref{Jé 7:5}] Mais c., corrigez vos voies et vos actions,
\item[\vref{2 Ti 3:16}] pour convaincre, pour c. et pour instruire
\end{listverse}

\ConcordanceEntry{Corrompre}
\vspace{-2mm}
\begin{listverse}
\item[\vref{Ge 6:11}] la terre était c. dvt Dieu, et
\item[\vref{Ge 6:12}] voici elle était c. ; car tte chair
\item[\vref{De 4:16}] vs. ne vs. c. et que vs.
\item[\vref{De 31:29}] mort vs. vs. c., et que vs.
\item[\vref{Jg 2:19}] mourrait, ils se c. de nouveau plus
\item[\vref{2 Ch 26:16}] s'éleva pour le c.. Et il pécha
\item[\vref{2 Ch 27:2}] de Yahweh. Néanmoins, le peuple se c. encore.
\item[\vref{Pr 11:9}] sa bouche l'impie c. son prochain, mais
\item[\vref{Es 1:4}] font que se c. ! Ils ont abandonné
\item[\vref{Es 3:12}] conduisent t'égarent, ils c. le chemin ds
\item[\vref{Da 11:32}] Et il c. par des flatteries ceux qui agissent
\item[\vref{1 Co 15:33}] Les mauvaises compagnies c. les bonnes mœurs.
\item[\vref{2 Co 11:3}] aussi ne se c. en se détournant
\item[\vref{Ep 4:22}] précédente, qui se c. par les convoitises
\item[\vref{Jud 1:10}] et ils se c. ds tt ce
\item[\vref{Ap 11:18}] détruire ceux qui c. la terre.
\end{listverse}

\ConcordanceEntry{Corruptible}
\vspace{-2mm}
\begin{listverse}
\item[\vref{Ro 1:23}] images représentant l'hom. c., et des oiseaux,
\item[\vref{1 Co 9:25}] obtenir une couronne c. ; mais ns., faisons-le
\item[\vref{1 Co 15:42}] corps est semé c., il ressuscitera incorruptible ;
\item[\vref{1 Co 15:53}] faut que ce c. revête l'incorruptibilité, et
\item[\vref{1 Pi 1:23}] par une semence c., mais par une
\end{listverse}

\ConcordanceEntry{Corruption}
\vspace{-2mm}
\begin{listverse}
\item[\vref{Lé 22:25}] Dieu ; car la c. qui est en
\item[\vref{Ps 16:10}] point que ton bien-aimé voie la c.
\item[\vref{Ac 2:27}] pas que ton Saint voie la c.
\item[\vref{Ac 13:36}] ses pères, et a vu la c.
\item[\vref{Ro 8:21}] servitude de la c., pour avoir part
\item[\vref{1 Co 15:50}] et que la c. n'hérite pas l'incorruptibilité.
\item[\vref{Ga 6:8}] la chair la c. ; mais celui qui
\item[\vref{2 Pi 1:4}] en fuyant la c. qui règne ds
\item[\vref{2 Pi 2:19}] esclaves de la c., car on est
\end{listverse}

\ConcordanceEntry{Côte}
\vspace{-2mm}
\begin{listverse}
\item[\vref{Ge 2:21}] une de ses c., et referma la
\item[\vref{Ge 2:22}] fem. de la c. qu'il avait prise
\item[\vref{Da 7:5}] elle avait trois c. ds la gueule
\end{listverse}

\ConcordanceEntry{Côté}
\vspace{-2mm}
\begin{listverse}
\item[\vref{Ps 124:2}] était de notre c. qnd les hommes
\item[\vref{Jn 19:34}] lui perça le c. avec une lance,
\item[\vref{Jn 20:27}] mets-la ds mon c. ; et ne sois
\item[\vref{Ac 12:7}] le frappant au c., et en disant :
\end{listverse}

\ConcordanceEntry{Côté nord}
\vspace{-2mm}
\begin{listverse}
\item[\vref{Ex 26:20}] de l'autre côté du tabernacle, du côté nord.
\item[\vref{Ex 38:11}] Et pour le côté nord, il fit des
\item[\vref{Jos 15:10}] elle traversera le côté nord de la montagne
\item[\vref{Jos 18:18}] passer sur le côté nord en face d'Araba,
\item[\vref{1 S 14:5}] était située du côté nord vis-à-vis de Micmasch,
\item[\vref{Ps 48:3}] de Sion ; le côté nord, c'est la ville
\item[\vref{Ez 8:3}] porte intérieure, du côté nord, où était posée
\item[\vref{Ez 42:1}] parvis extérieur, du côté nord ; et il me
\item[\vref{Ez 48:16}] les mesures : Du côté nord, quatre mille cinq
\end{listverse}

\ConcordanceEntry{Côté sud}
\vspace{-2mm}
\begin{listverse}
\item[\vref{1 S 14:5}] et l'autre, du côté sud vis-à-vis de Guéba.
\item[\vref{1 R 7:39}] de la maison, vers l'orient du côté sud.
\item[\vref{Ez 40:24}] me conduisit du côté sud, où se trouvait
\item[\vref{Ez 42:12}] des chambres du côté sud. Il y avait
\item[\vref{Ez 48:16}] cents cannes, du côté sud, quatre mille cinq
\end{listverse}

\ConcordanceEntry{Cou}
\vspace{-2mm}
\begin{listverse}
\item[\vref{Ge 27:16}] mains et son c., qui étaient sans
\item[\vref{Ex 32:9}] et voici, c'est un peuple au c. raide.
\item[\vref{De 10:16}] et vs. ne raidirez plus votre c.
\item[\vref{Jos 10:24}] pieds sur les c. de ces rois.
\item[\vref{1 S 4:18}] se rompit le c. et mourut ; car
\item[\vref{Né 9:16}] et raidirent lr. c.. Ils n'écoutèrent point
\item[\vref{Job 41:13}] est ds son c., et la terreur
\item[\vref{Pr 3:3}] Lie-les à ton c., et écris-les sur
\item[\vref{Pr 6:21}] ton cœur, et attache-les à ton c.
\item[\vref{Pr 29:1}] repris, raidit son c., sera subitement brisé
\item[\vref{Ca 1:10}] atours, et ton c. avec les colliers.
\item[\vref{Jé 28:12}] de dessus le c. de Jérémie, le
\item[\vref{Ez 16:11}] mains, et un collier à ton c.
\item[\vref{Da 5:16}] porteras à ton c. un collier d'or,
\item[\vref{Os 10:11}] de son superbe c. ; j'attellerai Ephraïm, Juda
\item[\vref{Mi 2:3}] point préserver votre c., et vs. ne
\item[\vref{Mt 18:6}] mette à son c. une meule d'âne,
\item[\vref{Lu 15:20}] jeter à son c. et le baisa.
\item[\vref{Ac 7:51}] Hommes au c. raide, et incirconcis
\item[\vref{Ac 20:37}] en larmes, et se jetant au c. de Paul,
\item[\vref{Ro 16:4}] ont exposé lr. c. pour ma vie ;
\end{listverse}

\ConcordanceEntry{Coucher}
\vspace{-2mm}
\begin{listverse}
\item[\vref{Ge 19:32}] notre père, et c. avec lui afin
\item[\vref{Ge 19:35}] se leva et c. avec lui ; mais
\item[\vref{Ge 26:10}] du peuple n'ait c. avec ta fem.,
\item[\vref{Ge 35:22}] Ruben vint, et c. avec Bilha, concubine
\item[\vref{Ge 39:7}] elle lui dit : C. avec moi !
\item[\vref{Ge 39:10}] il refusa de c. auprès d'elle, d'être
\item[\vref{Ex 22:16}] non fiancée, et c. avec elle, il
\item[\vref{Ex 22:19}] Celui qui c. avec une bête,
\item[\vref{Lé 18:20}] Tu ne c. point avec la fem. de ton
\item[\vref{Lé 18:22}] Tu ne c. pas aussi avec un hom., com.
\item[\vref{Lé 19:20}] Si un hom. c. et a commerce
\item[\vref{Lé 20:11}] L'hom. qui c. avec la fem.
\item[\vref{Lé 20:20}] Si un hom. c. avec sa tante,
\item[\vref{De 22:22}] trouve un hom. c. avec une fem.
\item[\vref{Jg 21:12}] connu d'hom. en c. avec lui, et
\item[\vref{1 S 2:22}] Israël, et qu'ils c. avec les femmes
\item[\vref{2 S 11:4}] lui, et il c. avec elle. Après
\item[\vref{2 S 12:11}] ta maison, qui c. avec elles à
\item[\vref{2 S 13:11}] lui dit : Viens, c. avec moi, ma
\item[\vref{2 S 13:14}] fit violence et c. avec elle.
\item[\vref{Ec 4:11}] Si deux aussi c. ensemble, ils en
\end{listverse}

\ConcordanceEntry{Coucher (se)}
\vspace{-2mm}
\begin{listverse}
\item[\vref{Ru 3:7}] il vint se c. à l'extrémité d'un
\item[\vref{Job 14:12}] ainsi l'hom. est c. par terre et
\item[\vref{Ps 4:9}] Je me c. et je m'endors en paix, car
\item[\vref{Ps 31:18}] confus, qu'ils soient c. ds le scheol !
\item[\vref{Ps 139:3}] qnd je me c. ; tu connais parfaitement
\item[\vref{Pr 3:24}] Si tu te c., tu seras sans crainte, et qnd
\item[\vref{Pr 6:22}] qnd tu te c., ils te garderont ;
\item[\vref{Ec 1:5}] le soleil se c. ; il soupire après
\item[\vref{Es 60:20}] soleil ne se c. plus, et ta
\item[\vref{La 2:21}] les vieillards sont c. par terre ds
\item[\vref{Ez 4:4}] que tu seras c. sur ce côté.
\item[\vref{Am 8:9}] que je ferai c. le soleil à
\item[\vref{Jon 1:5}] du navire, se c. et s'endormit profondément.
\item[\vref{Mt 8:14}] vit la belle-mère c. et ayant la
\item[\vref{Mt 9:2}] présenta un paralytique c. sur un lit.
\item[\vref{Mt 28:6}] le lieu où le Seign. était c.,
\item[\vref{Lu 2:7}] l'emmaillota, et le c. ds une crèche,
\item[\vref{Lu 16:20}] pauvre, nommé Lazare, c. à la porte
\item[\vref{Jn 13:23}] était à table c. sur le sein
\item[\vref{Ac 9:33}] Enée, qui était c. ds un petit
\item[\vref{Ep 4:26}] soleil ne se c. pas sur votre
\end{listverse}

\ConcordanceEntry{Coudée}
\vspace{-2mm}
\begin{listverse}
\item[\vref{Ge 6:15}] de trois cents c. ; sa largeur de
\item[\vref{Ge 7:20}] s'élevèrent de quinze c. au-dessus des montagnes
\item[\vref{Ex 25:10}] sera de deux c. et demie, et
\item[\vref{No 11:31}] hauteur de deux c. sur la terre.
\item[\vref{Jos 3:4}] d'environ deux mille c.. Elle vs. fera
\item[\vref{Né 3:13}] ils bâtirent mille c. de muraille, jusqu'à
\item[\vref{Jé 52:22}] était de cinq c., il y avait
\item[\vref{Ez 40:5}] longue de six c., chaque coudée étant
\item[\vref{Ez 47:3}] il mesura mille c., puis il me
\item[\vref{Da 3:1}] était de soixante c., et la largeur
\item[\vref{Za 5:2}] est de vingt c., et la largeur
\item[\vref{Mt 6:27}] inquiétudes, ajouter une c. à sa stature ?
\item[\vref{Jn 21:8}] environ deux cents c., traînant le filet
\item[\vref{Ap 21:17}] de cent quarante-quatre c., de la mesure
\end{listverse}

\ConcordanceEntry{Coup}
\vspace{-2mm}
\begin{listverse}
\item[\vref{De 25:3}] battre de quarante c., pas plus, de
\item[\vref{1 S 26:8}] terre d'un seul c., et je n'y
\item[\vref{Pr 19:29}] et les grands c. pour le dos
\item[\vref{Es 44:12}] le forme à c. de marteau ; il
\item[\vref{Jé 4:20}] détruites tt à c., mes pavillons en
\item[\vref{Mt 26:67}] lui donnèrent des c. de poing et
\item[\vref{Jn 12:29}] que c'était un c. de tonnerre ; les
\item[\vref{Ac 16:23}] eut chargés de c. de fouet, ils
\item[\vref{2 Co 6:5}] sous les c., ds les prisons, ds les troubles,
\item[\vref{2 Co 11:24}] des Juifs quarante c. moins un,
\item[\vref{Ap 13:14}] avait reçu le c. mortel de l'épée,
\end{listverse}

\ConcordanceEntry{Coupable}
\vspace{-2mm}
\begin{listverse}
\item[\vref{Ge 26:10}] fem., et tu ns. aurais rendus c.
\item[\vref{Ge 42:21}] Nous sommes certainement c. à l'égard de
\item[\vref{Ex 34:7}] tient point le c. pour innocent, et
\item[\vref{Lé 5:2}] reptile impur, il sera souillé et c.
\item[\vref{Jos 7:1}] d'Israël se rendirent c. au sujet de
\item[\vref{Jg 21:22}] vs. en seriez c. en ce temps.
\item[\vref{2 S 14:13}] roi se déclare c. en ce qu'il
\item[\vref{Esd 9:7}] ns. sommes grandement c., et c'est à
\item[\vref{Né 1:7}] Certainement ns. sommes c. dvt toi, ns.
\item[\vref{Job 11:11}] il discerne par le regard les c.
\item[\vref{Ez 5:6}] s'est rendue plus c. que les nations
\item[\vref{Ez 22:4}] Tu t'es rendue c. par ton sang
\item[\vref{Os 4:15}] se rende pas c. ! N'entrez dc pas
\item[\vref{Os 13:1}] il se rendit c. par Baal, et
\item[\vref{Na 1:3}] tient nullement le c. pour innocent. Yahweh
\item[\vref{Ha 1:11}] et se rend c., car sa force
\item[\vref{Za 11:5}] les tienne pour c., et celui qui
\item[\vref{Mt 12:5}] ds le temple, sans se rendre c. ?
\item[\vref{Mt 15:5}] Dieu, n'est pas c., quoiqu'il n'honore pas
\item[\vref{Lu 13:4}] qu'elles étaient plus c. que ts les
\item[\vref{Lu 23:14}] ne l'ai trouvé c. d'aucun des crimes
\item[\vref{Ac 25:5}] qq chose de c. contre cet hom.,
\item[\vref{Ac 25:8}] rien fait de c., ni contre la
\item[\vref{Ro 3:19}] le monde soit c. dvt Dieu.
\item[\vref{1 Co 4:3}] moi-mm, car je ne me sens c. de rien,
\item[\vref{1 Co 11:27}] Seign. indignement sera c. envers le corps
\item[\vref{2 Co 12:21}] l'impudicité dont ils se sont rendus c.
\item[\vref{Ja 2:10}] pèche contre un seul commandement devient c. de ts.
\end{listverse}

\ConcordanceEntry{Coupe}
\vspace{-2mm}
\begin{listverse}
\item[\vref{Ge 40:13}] tu mettras la c. ds sa main,
\item[\vref{Ps 23:5}] oins d'huile ma tête et ma c. déborde.
\item[\vref{Ps 75:9}] y a une c. ds la main
\item[\vref{Ps 116:13}] J'élèverai la c. des délivrances et
\item[\vref{Pr 23:31}] couleur ds la c., et qu'il coule
\item[\vref{Es 51:22}] la main la c. d'étourdissement, la lie
\item[\vref{Jé 25:15}] ma main cette c. de vin, à
\item[\vref{Ez 23:33}] douleur, par la c. de désolation et
\item[\vref{Za 12:2}] de Jérus. une c. d'étourdissement pour ts
\item[\vref{Za 14:20}] seront com. les c. dvt l'autel.
\item[\vref{Mt 20:22}] Pouvez-vs. boire la c. que je dois
\item[\vref{Mt 26:27}] ayant pris la c., et béni Dieu,
\item[\vref{Mt 26:39}] fais que cette c. passe loin de
\item[\vref{Mc 7:4}] le lavage des c., de cruches, des
\item[\vref{Mc 7:8}] cruches et des c., et vs. faites
\item[\vref{Lu 11:39}] dehors de la c. et du plat,
\item[\vref{Lu 22:20}] il prit la c. après le souper,
\item[\vref{Jn 18:11}] boirai-je pas la c. que le Père
\item[\vref{1 Co 10:16}] La c. de bénédiction, que ns. bénissons, n'est-elle
\item[\vref{1 Co 10:21}] pas boire la c. du Seign. et
\item[\vref{1 Co 11:25}] il prit la c., en disant : Cette
\item[\vref{1 Co 11:27}] boira de la c. du Seign. indignement
\item[\vref{Ap 5:8}] harpes et des c. d'or pleines de
\item[\vref{Ap 15:7}] sept anges sept c. d'or, pleines de
\item[\vref{Ap 16:1}] la terre les c. de la colère
\item[\vref{Ap 17:4}] la main une c. d'or, pleine des
\item[\vref{Ap 18:6}] ds la mm c. où elle vs.
\item[\vref{Ap 21:9}] tenaient les sept c. pleines des sept
\end{listverse}

\ConcordanceEntry{Courage}
\vspace{-2mm}
\begin{listverse}
\item[\vref{Ex 23:12}] de ta servante et l'étranger reprennent c.
\item[\vref{No 13:20}] non. Ayez bon c., et prenez du
\item[\vref{De 31:6}] dc et prenez c. ! Ne craignez point
\item[\vref{De 31:23}] Fortifie-toi et prends c., car c'est toi
\item[\vref{Jos 1:6}] Fortifie-toi et prends c., car c'est toi
\item[\vref{Jos 2:9}] du pays perdent c. à cause de
\item[\vref{Jos 2:24}] habitants ont perdu c. à notre vue.
\item[\vref{Jos 5:1}] avait plus de c. en eux à
\item[\vref{1 Ch 28:20}] fils : Fortifie-toi, prends c. et travaille ; ne
\item[\vref{Esd 10:4}] ns. serons avec toi. Prends dc c. et agis.
\item[\vref{Pr 24:10}] Si tu perds c. au jour de
\item[\vref{Es 35:4}] cœur troublé : Prenez c. et ne craignez
\item[\vref{Ez 7:13}] qu'il vivra ; ils ne reprendront jamais c.
\item[\vref{Da 11:32}] qui connaîtront lr. Dieu agiront avec c.
\item[\vref{Ab 1:1}] elles ont dit : C., levons-ns. contre lui
\item[\vref{Mi 3:8}] justice et de c., par l'Esprit de
\item[\vref{Mt 9:2}] au paralytique : Prends c., mon enfant, tes
\item[\vref{Mt 9:22}] la voyant : Prends c., ma fille, ta
\item[\vref{Mc 10:49}] lui disant : Prends c., lève-toi, il t'appelle.
\item[\vref{Jn 16:33}] mais ayez bon c., j'ai vaincu le
\item[\vref{Ac 23:11}] lui dit : Prends c. ; car, de mm
\item[\vref{Ac 27:22}] exhorte à prendre c. ; car aucun de
\item[\vref{Ac 28:15}] rendit grâces à Dieu et prit c.
\item[\vref{Ph 2:20}] personne d'un pareil c., et qui soit
\item[\vref{Hé 12:5}] ne perds pas c. lorsqu'il te reprend ;
\end{listverse}

\ConcordanceEntry{Courber}
\vspace{-2mm}
\begin{listverse}
\item[\vref{Ge 49:15}] magnifique ; et il c. son épaule sous
\item[\vref{Ps 38:7}] Je suis c. et abattu outre mesure ; je marche
\item[\vref{Ps 72:9}] des déserts se c. dvt lui, et
\item[\vref{Ps 145:14}] et redresse ts ceux qui sont c.
\item[\vref{Ps 146:8}] ceux qui sont c. ; Yahweh aime les
\item[\vref{Pr 16:26}] sa bouche se c. dvt lui.
\item[\vref{Ec 1:15}] Ce qui est c. ne peut se
\item[\vref{Ec 12:5}] hommes forts se c., et que celles
\item[\vref{Es 10:4}] Les uns seront c. parmi les prisonniers,
\item[\vref{Es 46:2}] Elles se sont c., elles se sont
\item[\vref{Es 51:23}] à ton âme : C.-toi, et ns.
\item[\vref{Es 58:5}] jour ? Est-ce en c. sa tête com.
\item[\vref{Es 60:14}] toi en se c., et ts ceux
\item[\vref{Lu 13:11}] ans ; elle était c., et ne pouvait
\item[\vref{Lu 24:12}] sépulcre et s'étant c. pour regarder, il
\item[\vref{Ro 11:10}] pas voir, et c. continuellement lr. dos !
\end{listverse}

\ConcordanceEntry{Courir}
\vspace{-2mm}
\begin{listverse}
\item[\vref{Ge 18:2}] les vit, il c. au-dvt d'eux depuis
\item[\vref{Jg 7:21}] se mit à c. ça et là,
\item[\vref{2 S 18:27}] La manière de c. du premier me
\item[\vref{2 R 5:21}] Et Guéhazi c. après Naaman. Naaman,
\item[\vref{Job 1:7}] Je viens de c. çà et là
\item[\vref{Pr 6:18}] les pieds qui se hâtent de c. au mal,
\item[\vref{Ca 1:4}] Jérus. :] et ns. c. après toi ! [La
\item[\vref{Jé 12:5}] Si tu c. avec des piétons et qu'ils te
\item[\vref{Mt 28:8}] grande joie ; et c. l'annoncer à ses
\item[\vref{Ac 24:22}] était parfaitement au c. de ce qui
\item[\vref{Ro 9:16}] de celui qui c., mais de Dieu
\item[\vref{1 Co 9:24}] que ceux qui c. ds le stade,
\item[\vref{1 Co 9:26}] Je c. dc, mais non pas sans savoir
\item[\vref{Ga 2:2}] de ne pas c. ou avoir couru
\item[\vref{Ga 5:7}] Vous c. bien : Qui vs. a arrêtés pour
\item[\vref{Ph 2:16}] de n'avoir pas c. en vain, ni
\item[\vref{Ph 3:14}] en avant, je c. vers le but,
\end{listverse}

\ConcordanceEntry{Couronne}
\vspace{-2mm}
\begin{listverse}
\item[\vref{Ex 29:6}] tiare et la c. de sainteté sur
\item[\vref{1 Ch 20:2}] David enleva la c. de dessus la
\item[\vref{Est 2:17}] Il mit la c. royale sur sa
\item[\vref{Job 31:36}] si je ne l'attache com. une c.,
\item[\vref{Ps 21:4}] sa tête une c. d'or pur.
\item[\vref{Ps 65:12}] Tu c. l'année de tes biens, et tes
\item[\vref{Pr 4:9}] ta tête une c. de grâce, et
\item[\vref{Pr 12:4}] vertueuse est la c. de son mari,
\item[\vref{Pr 16:31}] blancs sont une c. d'honneur ; elle se
\item[\vref{Pr 17:6}] petits-fils sont la c. des vieillards, et
\item[\vref{Ca 3:11}] Salomon avec la c. dont sa mère
\item[\vref{Es 62:3}] Tu seras une c. de gloire ds
\item[\vref{Ez 16:12}] oreilles, et une c. de gloire sur
\item[\vref{Za 9:16}] les pierres d'une c. qui brilleront ds
\item[\vref{Mt 27:29}] ayant fait une c. d'épines entrelacées, ils
\item[\vref{1 Co 9:25}] pour obtenir une c. corruptible ; mais ns.,
\item[\vref{Ph 4:1}] joie et ma c., demeurez ainsi fermes
\item[\vref{1 Th 2:19}] joie, ou notre c. de gloire ? N'est-ce
\item[\vref{2 Ti 4:8}] Au reste, la c. de justice m'est
\item[\vref{Ja 1:12}] il recevra la c. de vie, que
\item[\vref{1 Pi 5:4}] vs. obtiendrez la c. incorruptible de la
\item[\vref{Ap 3:11}] afin que personne ne t'enlève ta c.
\item[\vref{Ap 4:4}] et ayant sur leurs têtes des c. d'or.
\item[\vref{Ap 4:10}] ils jetaient leurs c. dvt le trône,
\item[\vref{Ap 9:7}] avait com. des c. semblables à de
\item[\vref{Ap 12:1}] sa tête une c. de douze étoiles.
\item[\vref{Ap 14:14}] sa tête une c. d'or, et ds
\end{listverse}

\ConcordanceEntry{Couronner}
\vspace{-2mm}
\begin{listverse}
\item[\vref{Ps 65:12}] Tu c. l'année de tes biens, et tes
\item[\vref{Ps 103:4}] fosse, qui te c. de bonté et
\item[\vref{Es 23:8}] Tyr, celle qui c. les siens, dont
\item[\vref{2 Ti 2:5}] combat n'est pas c. s'il n'a pas
\item[\vref{Hé 2:7}] anges, tu l'as c. de gloire et
\item[\vref{Hé 2:9}] ns. le voyons c. de gloire et
\end{listverse}

\ConcordanceEntry{Course}
\vspace{-2mm}
\begin{listverse}
\item[\vref{Ps 19:7}] et achève sa c. à l'autre extrémité:
\item[\vref{Ec 9:11}] soleil que la c. n'est point aux
\item[\vref{Jé 23:10}] sont desséchés, lr. c. ne va qu'au
\item[\vref{Jé 46:6}] léger à la c. ne s'enfuie pas,
\item[\vref{Ac 13:25}] Jean achevait sa c., il disait : Qui
\item[\vref{Ac 20:24}] que j'achève ma c. avec joie, et
\item[\vref{Col 2:18}] prix de la c., sous l'apparence d'humilité
\item[\vref{2 Th 3:1}] Seign. poursuive sa c., et qu'elle soit
\item[\vref{2 Ti 4:7}] j'ai achevé la c., j'ai gardé la
\item[\vref{Hé 12:1}] poursuivons constamment la c. qui ns. est
\end{listverse}

\ConcordanceEntry{Coutume}
\vspace{-2mm}
\begin{listverse}
\item[\vref{Ge 19:31}] ns., selon la c. de ts les
\item[\vref{Ge 31:35}] les femmes ont c. d'avoir ; et il
\item[\vref{Ge 50:3}] car c'était la c. d'embaumer les corps
\item[\vref{Lé 18:30}] aucune de ces c. abominables qui ont
\item[\vref{Jg 11:39}] ce fut une c. en Israël,
\item[\vref{2 S 20:18}] Autrefois on avait c. de dire : Que
\item[\vref{2 R 11:14}] l'estrade, selon la c. des rois. Les
\item[\vref{2 R 17:8}] Ils suivirent les c. des nations que
\item[\vref{Mt 27:15}] le gouverneur avait c. de relâcher un
\item[\vref{Mc 10:1}] et selon sa c., il se mit
\item[\vref{Lu 1:9}] selon la c. d'exercer la prêtrise, à entrer ds
\item[\vref{Lu 4:16}] et selon sa c., il entra ds
\item[\vref{Lu 22:39}] alla, selon sa c., au Mont des
\item[\vref{Jn 19:40}] les Juifs ont c. d'ensevelir.
\item[\vref{Ac 6:14}] et changera les c. que Moïse ns.
\item[\vref{Ac 16:21}] ils annoncent des c. qu'il ne ns.
\item[\vref{Ac 17:2}] entra, selon sa c.. Pendant trois sabbats,
\item[\vref{Ac 25:16}] n'est pas la c. des Romains de
\item[\vref{Ac 26:3}] connais parfaitement leurs c. et leurs discussions.
\item[\vref{Ac 28:17}] ni contre les c. des pères, j'ai
\item[\vref{1 Co 11:16}] pas une telle c., ni les églises
\item[\vref{Hé 10:25}] com. c'est la c. de quelques-uns ; mais
\end{listverse}

\ConcordanceEntry{Couvrir}
\vspace{-2mm}
\begin{listverse}
\item[\vref{Ge 9:23}] à reculons, ils c. la nudité de
\item[\vref{Ge 24:65}] elle prit son voile et se c.
\item[\vref{Ge 27:16}] Elle c. ses mains et son cou, qui
\item[\vref{Ex 8:2}] grenouilles montèrent et c. le pays d'Egypte.
\item[\vref{Ex 24:15}] et une nuée c. la montagne.
\item[\vref{Ex 33:22}] rocher, et te c. de ma main,
\item[\vref{Ex 40:34}] Et la nuée c. la tente d'assignation,
\item[\vref{No 4:5}] rideau, et en c. l'arche du témoignage ;
\item[\vref{No 9:15}] dressé, la nuée c. le tabernacle de
\item[\vref{1 R 6:21}] Salomon c. d'or pur l'intérieur de la maison,
\item[\vref{Est 4:1}] vêtements et se c. d'un sac et
\item[\vref{Est 7:8}] roi, aussitôt on c. le visage d'Haman.
\item[\vref{Job 31:33}] com. Adam, pour c. mon iniquité en
\item[\vref{Ps 91:4}] Il te c. de ses plumes, et tu trouveras
\item[\vref{Ps 104:6}] Tu l'avais c. de l'abîme com.
\item[\vref{Ps 139:11}] les ténèbres me c., la nuit mm
\item[\vref{Pr 10:12}] mais la charité c. ttes les transgressions.
\item[\vref{Pr 17:9}] Celui qui c. les fautes cherche
\item[\vref{Ez 12:6}] pendant l'obscurité. Tu c. aussi ton visage,
\item[\vref{Ez 16:8}] robe, et je c. ta nudité. Je
\item[\vref{Os 10:8}] dira aux montagnes : C.-ns. ! Et aux
\item[\vref{Jon 3:6}] son manteau, se c. d'un sac et
\item[\vref{Lu 1:35}] du Très-Haut te c. de son ombre.
\item[\vref{Lu 9:34}] nuée vint les c. de son ombre ;
\item[\vref{Ro 4:7}] pardonnées, et dont les péchés sont c. !
\item[\vref{1 Co 11:7}] ne doit pas c. sa tête, vu
\item[\vref{Hé 9:5}] de la gloire, c. de lr. ombre
\item[\vref{Ja 5:20}] la mort et c. une multitude de
\end{listverse}

\ConcordanceEntry{Cracher}
\vspace{-2mm}
\begin{listverse}
\item[\vref{Lé 15:8}] a ce flux c. sur celui qui
\item[\vref{No 12:14}] père lui avait c. au visage, ne
\item[\vref{De 25:9}] pied, et lui c. au visage. Et
\item[\vref{Job 30:10}] pas de me c. au visage.
\item[\vref{Mt 26:67}] Alors ils lui c. au visage, et
\item[\vref{Mt 27:30}] Et ils c. contre lui, prenaient le roseau et
\item[\vref{Mc 10:34}] battront de verges, c. sur lui, et
\item[\vref{Lu 18:32}] et on lui c. au visage,
\item[\vref{Jn 9:6}] ces paroles, il c. à terre et
\end{listverse}

\ConcordanceEntry{Craindre}
\vspace{-2mm}
\begin{listverse}
\item[\vref{Ge 22:12}] sais que tu c. Dieu, puisque tu
\item[\vref{Ge 42:18}] Faites ceci, et vs. vivrez. Je c. Dieu !
\item[\vref{Ex 14:31}] et le peuple c. Yahweh, ils crurent
\item[\vref{De 5:29}] cœur pour me c. et pour garder
\item[\vref{De 6:13}] Tu c. Yahweh, ton Dieu, tu le serviras
\item[\vref{De 28:58}] ce livre, en c. le Nom glorieux
\item[\vref{Jos 4:14}] et ils le c., com. ils avaient
\item[\vref{1 S 15:24}] paroles ; car je c. le peuple et
\item[\vref{1 R 18:3}] maison ; (or Abdias c. beaucoup Yahweh ;
\item[\vref{2 R 17:35}] disant : Vous ne c. point d'autres dieux ;
\item[\vref{2 R 17:39}] Mais vs. c. Yahweh, votre Dieu,
\item[\vref{Né 1:11}] prennent plaisir à c. ton Nom ! Je
\item[\vref{Job 3:25}] Ce que je c. le plus m'arrive,
\item[\vref{Job 11:15}] Tu seras ferme et tu ne c. rien ;
\item[\vref{Ps 22:24}] Vous qui c. Yahweh, louez-le ! Toute la race de
\item[\vref{Ps 23:4}] mort, je ne c. aucun mal, car
\item[\vref{Ps 25:12}] est l'hom. qui c. Yahweh ? Yahweh lui
\item[\vref{Ps 25:14}] ceux qui le c., et son alliance
\item[\vref{Ps 31:20}] ceux qui te c., tu lr. fais
\item[\vref{Ps 34:8}] ceux qui le c., et les délivre.
\item[\vref{Ps 49:6}] Pourquoi c.-je au jour du malheur, qnd
\item[\vref{Ps 56:12}] Dieu, je ne c. rien : Que me
\item[\vref{Ps 61:6}] me donnes l'héritage de ceux qui c. ton Nom.
\item[\vref{Ps 66:16}] Vous ts qui c. Dieu, venez, écoutez,
\item[\vref{Ps 67:8}] les extrémités de la terre le c.
\item[\vref{Ps 91:5}] Tu ne c. ni les terreurs de la nuit,
\item[\vref{Ps 112:1}] est l'hom. qui c. Yahweh [Beth.] et
\item[\vref{Ps 118:6}] moi, je ne c. point. Que me
\item[\vref{Ps 147:11}] ceux qui le c., ceux qui s'attendent
\item[\vref{Pr 3:7}] à tes yeux ; c. Yahweh, et détourne-toi
\item[\vref{Pr 27:6}] les baisers d'un ennemi sont à c.
\item[\vref{Ec 3:14}] afin que dvt lui, on le c.
\item[\vref{Ec 7:18}] car celui qui c. Dieu sort de
\item[\vref{Ec 12:15}] a été entendu : C. Dieu, et garde
\item[\vref{Es 7:25}] bêche, on ne c. plus de voir
\item[\vref{Es 8:12}] dit conjuration ; ne c. point ce qu'il
\item[\vref{Es 8:13}] que vs. devez c. et redouter.
\item[\vref{Es 12:2}] et je ne c. rien ; car Yahweh,
\item[\vref{Es 29:23}] Jacob, et ils c. le Dieu d'Israël.
\item[\vref{Es 54:14}] et tu ne c. rien ; tu seras,
\item[\vref{Es 57:11}] qui redoutais-tu, qui c.-tu pour que
\item[\vref{Es 59:19}] Et on c. le Nom de Yahweh depuis l'occident,
\item[\vref{Jé 10:7}] Qui ne te c., Roi des nations ?
\item[\vref{Jon 1:9}] Hébreu et je c. Yahweh, le Dieu
\item[\vref{So 3:16}] à Jérus. : Ne c. point Sion, que
\item[\vref{Mal 3:16}] Alors ceux qui c. Yahweh se parlèrent
\item[\vref{Mal 4:2}] pour vs. qui c. mon Nom, se
\item[\vref{Mt 1:20}] de David, ne c. pas de prendre
\item[\vref{Mt 10:28}] Et ne c. pas ceux qui tuent le corps
\item[\vref{Mt 21:46}] lui, mais ils c. la foule, parce
\item[\vref{Mc 5:36}] la synagogue : Ne c. pas, crois seulement.
\item[\vref{Mc 9:32}] discours, et ils c. de l'interroger.
\item[\vref{Lu 23:40}] et disait : Ne c.-tu pas Dieu,
\item[\vref{Hé 11:23}] et ils ne c. pas l'ordre du
\item[\vref{Hé 13:6}] et je ne c. pas ce que
\item[\vref{Ap 15:4}] qui ne te c., et qui ne
\end{listverse}

\ConcordanceEntry{Crainte}
\vspace{-2mm}
\begin{listverse}
\item[\vref{Ge 20:8}] choses, et ils furent saisis de c.
\item[\vref{Ex 20:20}] afin que sa c. soit dvt vs.,
\item[\vref{De 2:25}] frayeur et la c. de toi sur
\item[\vref{De 11:25}] frayeur et la c. de vs. sur
\item[\vref{De 25:18}] et, sans aucune c. de Dieu, attaqua
\item[\vref{Jos 4:24}] ayez toujours la c. de Yahweh, votre
\item[\vref{1 S 12:18}] eut une grande c. de Yahweh, et
\item[\vref{2 Ch 19:9}] ainsi ds la c. de Yahweh, avec
\item[\vref{Né 5:9}] marcher ds la c. de notre Dieu,
\item[\vref{Né 5:15}] cause de la c. de mon Dieu.
\item[\vref{Est 9:2}] résister, car la c. qu'on avait d'eux
\item[\vref{Job 15:4}] tu abolis la c. de Dieu, et
\item[\vref{Job 28:28}] l'hom. : Voici, la c. du Seign. est
\item[\vref{Ps 2:11}] Servez Yahweh avec c., et réjouissez-vs. avec
\item[\vref{Ps 5:8}] les sentiments d'une c. respectueuse.
\item[\vref{Ps 19:10}] La c. de Yahweh est pure, elle subsiste
\item[\vref{Ps 34:12}] vs. enseignerai la c. de Yahweh.
\item[\vref{Ps 36:2}] a point de c. de Dieu dvt
\item[\vref{Ps 38:19}] suis ds la c. à cause de
\item[\vref{Ps 52:8}] auront de la c., et ils se
\item[\vref{Ps 86:11}] cœur à la c. de ton Nom.
\item[\vref{Ps 111:10}] sagesse c'est la c. de Yahweh : [Shin.]
\item[\vref{Pr 1:7}] La c. de Yahweh est la principale de
\item[\vref{Pr 2:5}] tu connaîtras la c. de Yahweh, et
\item[\vref{Pr 3:24}] tu seras sans c., et qnd tu
\item[\vref{Pr 8:13}] La c. de Yahweh c'est la haine du
\item[\vref{Pr 9:10}] sagesse est la c. de Yahweh ; et
\item[\vref{Pr 10:27}] La c. de Yahweh ajoute au nombre de
\item[\vref{Pr 14:27}] La c. de Yahweh est une source de
\item[\vref{Pr 15:16}] bien avec la c. de Yahweh, qu'un
\item[\vref{Pr 16:6}] mal par la c. de Yahweh.
\item[\vref{Pr 19:23}] La c. de Yahweh conduit à la vie,
\item[\vref{Pr 22:4}] et de la c. de Yahweh sont
\item[\vref{Pr 28:14}] continuellement ds la c., mais celui qui
\item[\vref{Pr 29:25}] La c. qu'on a des hommes tend un
\item[\vref{Es 11:3}] Il respirera la c. de Yahweh, il
\item[\vref{Es 29:13}] parce que la c. qu'il a de
\item[\vref{Es 33:6}] ton salut ; la c. de Yahweh est
\item[\vref{Jé 32:40}] je mettrai ma c. ds lr. cœur,
\item[\vref{Da 6:26}] ait de la c. et de la
\item[\vref{Ag 1:12}] eut de la c. dvt Yahweh.
\item[\vref{Mal 1:6}] où est la c. qu'on a de
\item[\vref{Mt 28:8}] du sépulcre avec c. et grande joie ;
\item[\vref{Mc 10:32}] le suivaient avec c.. Et Jésus prit
\item[\vref{Lu 1:12}] vit, et il fut saisi de c.
\item[\vref{Jn 7:13}] cause de la c. qu'on avait des
\item[\vref{Ac 2:43}] avait de la c., et beaucoup de
\item[\vref{Ac 5:5}] causa une grande c. à ts ceux
\item[\vref{Ro 8:15}] encore ds la c. ; mais vs. avez
\item[\vref{1 Co 2:3}] faiblesse, ds la c., et ds un
\item[\vref{2 Co 7:15}] l'avez reçu avec c. et tremblement, son
\item[\vref{Ph 2:12}] propre salut avec c. et tremblement, non
\item[\vref{1 Ti 5:20}] autres aussi en aient de la c.
\item[\vref{Hé 2:15}] ceux qui, par c. de la mort,
\item[\vref{Hé 12:28}] soyons agréables avec respect et avec c.,
\item[\vref{1 Pi 1:17}] favoritisme, conduisez-vs. avec c. pendant le temps
\item[\vref{1 Pi 3:6}] sans vs. laisser troubler par aucune c.
\item[\vref{Ap 11:11}] et une grande c. saisit ceux qui
\item[\vref{Ap 18:10}] éloignés ds la c. de son tourment,
\end{listverse}

\ConcordanceEntry{Cramoisi}
\vspace{-2mm}
\begin{listverse}
\item[\vref{Ge 38:28}] attacha un fil c., en disant : Celui-ci
\item[\vref{Ex 26:1}] d'écarlate, et de c. ; et tu les
\item[\vref{Ex 39:1}] d'écarlate, et de c. les vêtements du
\item[\vref{Ex 39:29}] d'écarlate et de c., d'ouvrage de broderie,
\item[\vref{No 4:8}] drap teint de c., ils le couvriront
\item[\vref{2 S 1:24}] revêtait magnifiquement de c., qui mettait des
\item[\vref{2 Ch 2:7}] en écarlate, en c. et en pourpre,
\item[\vref{2 Ch 3:14}] pourpre, d'écarlate, de c. et de fin
\end{listverse}

\ConcordanceEntry{Crâne}
\vspace{-2mm}
\begin{listverse}
\item[\vref{Jg 9:53}] tête d'Abimélec et lui brisa le c.
\item[\vref{2 R 9:35}] d'elle que le c., les pieds et
\item[\vref{Mt 27:33}] appelé Golgotha, c'est-à-dire le lieu du c.,
\item[\vref{Lu 23:33}] appelé Calvaire (le C.), ils le crucifièrent
\item[\vref{Jn 19:17}] lieu appelé le C., qui se dit
\end{listverse}

\ConcordanceEntry{Créancier}
\vspace{-2mm}
\begin{listverse}
\item[\vref{Ex 22:25}] avec lui en c., vs. ne lui
\item[\vref{1 S 22:2}] qui avaient des c., et qui avaient
\item[\vref{2 R 4:1}] Yahweh ; or son c. est venu pour
\item[\vref{Ps 109:11}] Que le c. usant d'exaction attrape tt ce qui
\item[\vref{Es 24:2}] qui emprunte, du c. com. du débiteur.
\item[\vref{Es 50:1}] auquel de mes c. vs. ai-je vendus ?
\item[\vref{Lu 7:41}] Un c. avait deux débiteurs : L'un lui devait
\end{listverse}

\ConcordanceEntry{Créateur}
\vspace{-2mm}
\begin{listverse}
\item[\vref{Job 32:22}] de flatterie ; mon C. m'enlèverait tt aussitôt.
\item[\vref{Job 36:3}] et je défendrai la justice du C.
\item[\vref{Ec 12:3}] souviens-toi de ton C. pendant les jours
\item[\vref{Es 43:15}] votre Saint, le C. d'Israël, votre Roi.
\item[\vref{Es 45:11}] qui est son C. : Interrogez-moi sur les
\item[\vref{Es 54:5}] Car ton c. est ton époux : Yahweh des armées
\item[\vref{Mt 19:4}] lu que le C., au commencement, fit
\item[\vref{Ro 1:25}] au lieu du C., qui est béni
\item[\vref{1 Pi 4:19}] recommandent leurs âmes, com. au fidèle C.
\end{listverse}

\ConcordanceEntry{Création}
\vspace{-2mm}
\begin{listverse}
\item[\vref{Mc 10:6}] commencement de la c., Dieu fit un
\item[\vref{Ro 1:20}] nu, depuis la c. du monde, qnd
\item[\vref{Ro 8:19}] En effet, la c. attend avec un
\item[\vref{Ro 8:20}] Car la c. a été soumise à la vanité,
\item[\vref{Ro 8:22}] jour, tte la c. soupire et souffre
\item[\vref{Col 1:15}] invisible, le premier-né de tte la c.
\item[\vref{Hé 9:11}] c'est-à-dire, qui n'est pas de cette c. ;
\item[\vref{2 Pi 3:4}] été dès le commencement de la c.
\item[\vref{Ap 3:14}] commencement de la c. de Dieu :
\end{listverse}

\ConcordanceEntry{Créature}
\vspace{-2mm}
\begin{listverse}
\item[\vref{Ps 145:16}] et tu rassasies à souhait tte c. vivante.
\item[\vref{Ez 44:31}] ne mangeront aucune c. volante, aucun animal
\item[\vref{Mc 16:15}] monde, et prêchez l'Evangile à tte c.
\item[\vref{Ro 1:25}] et servi la c., au lieu du
\item[\vref{Ro 8:39}] ni aucune autre c., ne pourra ns.
\item[\vref{2 Co 5:17}] est une nouvelle c. ; les choses anciennes
\item[\vref{Ga 6:15}] n'ont aucune efficacité, mais la nouvelle c.
\item[\vref{Col 1:23}] prêché à tte c. qui est sous
\item[\vref{Hé 4:13}] n'y a aucune c. qui soit cachée
\item[\vref{Ja 1:18}] soyons com. les prémices de ses c.
\item[\vref{Ap 5:13}] aussi ttes les c. qui sont ds
\item[\vref{Ap 8:9}] le tiers des c. vivantes qui étaient
\end{listverse}

\ConcordanceEntry{Crèche}
\vspace{-2mm}
\begin{listverse}
\item[\vref{Job 39:12}] te servir, ou demeurera-t-il à ta c. ?
\item[\vref{Es 1:3}] et l'âne la c. de son maître,
\item[\vref{Lu 2:7}] coucha ds une c., parce qu'il n'y
\item[\vref{Lu 13:15}] âne de la c. le jour du
\end{listverse}

\ConcordanceEntry{Créer}
\vspace{-2mm}
\begin{listverse}
\item[\vref{Ge 1:1}] Au commencement, Dieu c. les cieux et
\item[\vref{Ge 1:27}] Dieu c. l'hom. à son image, il le
\item[\vref{Ge 5:2}] il les c. mâle et femelle, et les bénit,
\item[\vref{Ps 51:12}] Ô Dieu ! C. en moi un cœur pur, et
\item[\vref{Ps 102:19}] peuple qui sera c. louera Yahweh.
\item[\vref{Ps 104:30}] Esprit, ils sont c. ; et tu renouvelles
\item[\vref{Ps 139:13}] Tu as c. mes reins, tu me couvres ds
\item[\vref{Ec 7:29}] que Dieu a c. l'hom. juste ; mais
\item[\vref{Es 40:28}] d'éternité, Yahweh, a c. les extrémités de
\item[\vref{Es 43:7}] je les ai c. pour ma gloire ;
\item[\vref{Es 45:12}] et qui ai c. l'hom. sur elle ;
\item[\vref{Es 65:17}] voici, je vais c. de nouveaux cieux
\item[\vref{Es 65:18}] que je vais c. ; car voici je
\item[\vref{Jé 31:22}] rebelle ? Car Yahweh c. une chose nouvelle
\item[\vref{Ez 28:15}] où tu fus c., jusqu'à celui où
\item[\vref{Mal 2:10}] qui ns. a c. ? Pourquoi dc agissons-ns.
\item[\vref{Ep 2:10}] ouvrage, ayant été c. en Jésus-Christ pour
\item[\vref{Ep 2:15}] ordonnances, afin de c. les deux en
\item[\vref{Ep 3:9}] Dieu, qui a c. ttes choses par
\item[\vref{Col 1:16}] lui ont été c. ttes les choses
\item[\vref{1 Ti 4:4}] que Dieu a c. est bon, et
\end{listverse}

\ConcordanceEntry{Crème}
\vspace{-2mm}
\begin{listverse}
\item[\vref{De 32:14}] la c. des vaches, le lait des brebis,
\item[\vref{Jg 5:25}] présenté de la c. ds la coupe
\item[\vref{2 S 17:29}] miel, de la c., des brebis, et
\item[\vref{Ps 55:22}] douces que la c., mais la guerre
\end{listverse}

\ConcordanceEntry{Crète}
\vspace{-2mm}
\begin{listverse}
\item[\vref{Ac 27:7}] dessous de la C., vers Salmone.
\item[\vref{Tit 1:5}] t'ai laissé en C., c'est afin que
\end{listverse}

\ConcordanceEntry{Creuset}
\vspace{-2mm}
\begin{listverse}
\item[\vref{Ps 12:7}] sur terre au c., et sept fois
\item[\vref{Ps 26:2}] Fais passer au c. mes reins et
\item[\vref{Ps 66:10}] fait passer au c. com. l'argent.
\item[\vref{Pr 17:3}] Le c. est pour éprouver l'argent, et le
\item[\vref{Pr 27:21}] l'argent, et le c. pour l'or ; ainsi
\item[\vref{Es 48:10}] t'ai éprouvé au c. de l'affliction.
\item[\vref{Ez 22:18}] plomb ds un c. ; ils sont devenus
\item[\vref{Ez 22:20}] l'étain ds un c., afin d'y souffler
\end{listverse}

\ConcordanceEntry{Creux}
\vspace{-2mm}
\begin{listverse}
\item[\vref{Ex 33:22}] mettrai ds un c. du rocher, et
\item[\vref{Jg 15:19}] Dieu fendit un c. qui était ds
\item[\vref{1 R 20:10}] pour remplir le c. de la main
\item[\vref{Pr 30:4}] vent ds le c. de sa main,
\item[\vref{Ec 4:6}] Mieux vaut le c. de la main
\item[\vref{Es 2:21}] et ds les c. des rochers, à
\item[\vref{Es 40:12}] eaux avec le c. de sa main,
\item[\vref{Es 51:1}] taillés, et au c. de la citerne
\item[\vref{Jé 49:8}] avez fait des c. pour y habiter !
\item[\vref{Jé 49:16}] habites ds le c. des rochers, et
\end{listverse}

\ConcordanceEntry{Cri}
\vspace{-2mm}
\begin{listverse}
\item[\vref{Ge 18:20}] Yahweh dit : Le c. contre Sodome et
\item[\vref{Ex 2:23}] crièrent ; et lr. c. monta jusqu'à Dieu,
\item[\vref{Ex 3:7}] j'ai entendu le c. qu'ils ont poussé
\item[\vref{Ex 14:15}] à Moïse : Pourquoi c.-tu vers moi ?
\item[\vref{Ex 22:23}] crient à moi, certainement j'entendrai lr. c.
\item[\vref{Ex 32:18}] voix ni un c. de gens qui
\item[\vref{Lé 9:24}] ils poussèrent des c. de joie et
\item[\vref{No 20:16}] a entendu nos c.. Il a envoyé
\item[\vref{Jos 6:16}] peuple : Poussez des c. de joie, car
\item[\vref{1 S 5:12}] sorte que le c. de la ville
\item[\vref{2 S 22:7}] voix, et mon c. est parvenu à
\item[\vref{Esd 3:11}] poussait de grands c. de joie en
\item[\vref{Est 4:1}] ville en poussant avec force des c. amers,
\item[\vref{Job 34:28}] fait monter le c. du pauvre jusqu'à
\item[\vref{Ps 17:1}] attentif à mon c., prête l'oreille à
\item[\vref{Ps 33:1}] justes, poussez un c. de joie à
\item[\vref{Ps 40:2}] vers moi et a entendu mon c.
\item[\vref{Ps 47:6}] monté avec un c. de réjouissance, Yahweh
\item[\vref{Ps 66:1}] terre, poussez des c. de triomphe à
\item[\vref{Ps 145:19}] il entend lr. c. et les délivre.
\item[\vref{Pr 1:20}] La souveraine sagesse c. hautement au-dehors, elle
\item[\vref{Pr 8:1}] La sagesse ne c.-t-elle pas ? Et
\item[\vref{Pr 21:13}] pas entendre le c. du pauvre, criera
\item[\vref{Ec 9:17}] paisiblement que le c. de celui qui
\item[\vref{Jé 4:31}] Car j'entends un c. com. celui d'une
\item[\vref{Jon 2:10}] sacrifices avec un c. de louange, j'accomplirai
\item[\vref{Mt 25:6}] se fit un c., disant : Voici, l'époux
\item[\vref{Mc 15:37}] poussé un grand c., rendit l'esprit.
\item[\vref{Lu 23:23}] insistèrent à grands c., demandant qu'il soit
\item[\vref{1 Th 4:16}] lui-mm, avec un c. de commandement, et
\item[\vref{Hé 5:7}] avec de grands c. et avec larmes
\item[\vref{Ap 21:4}] ni deuil, ni c., ni douleur, car
\end{listverse}

\ConcordanceEntry{Cribler}
\vspace{-2mm}
\begin{listverse}
\item[\vref{Am 9:9}] blé ds le c., sans qu'il en
\item[\vref{Lu 22:31}] réclamés pour vs. c. com. le froment ;
\end{listverse}

\ConcordanceEntry{Crier}
\vspace{-2mm}
\begin{listverse}
\item[\vref{Ge 4:10}] de ton frère c. de la terre
\item[\vref{Ge 39:14}] moi ; mais j'ai c. à haute voix.
\item[\vref{Ex 2:23}] servitude, et ils c. ; et lr. cri
\item[\vref{Ex 8:8}] chez Pharaon ; Moïse c. à Yahweh au
\item[\vref{Ex 14:10}] grande frayeur et c. à Yahweh.
\item[\vref{Ex 33:19}] face, et je c. le Nom de
\item[\vref{Jg 3:9}] les enfants d'Israël c. à Yahweh, et
\item[\vref{1 S 7:8}] cesse pas de c. pour ns. à
\item[\vref{1 S 15:11}] irrité, et il c. à Yahweh tte
\item[\vref{2 S 22:7}] invoqué Yahweh, j'ai c. à mon Dieu ;
\item[\vref{1 R 13:2}] Et il c. contre l'autel selon la parole de
\item[\vref{1 R 18:27}] d'eux, et dit : C. à haute voix,
\item[\vref{Job 30:20}] Je c. à toi, et tu ne m'exauces
\item[\vref{Job 35:9}] On fait c. les opprimés par la grandeur des
\item[\vref{Ps 3:5}] ma voix je c. à Yahweh, et
\item[\vref{Ps 22:3}] Mon Dieu ! je c. le jour, mais
\item[\vref{Ps 69:4}] suis las de c., mon gosier se
\item[\vref{Ps 107:13}] Alors ils ont c. vers Yahweh ds
\item[\vref{Ps 119:147}] l'aurore et je c. ; je m'attends à
\item[\vref{Ps 142:2}] Je c. de ma voix à Yahweh, je
\item[\vref{Pr 8:1}] La sagesse ne c.-t-elle pas ? Et
\item[\vref{Es 42:2}] Il ne c. point, et il ne haussera, ni
\item[\vref{Es 58:1}] C. à plein gosier, ne te retiens
\item[\vref{Jé 33:3}] C. vers moi, je te répondrai et
\item[\vref{Jon 3:8}] bêtes aussi, qu'ils c. à Dieu avec
\item[\vref{Ha 1:2}] Yahweh ! Jusqu'à qnd c.-je sans que
\item[\vref{Mt 3:3}] de celui qui c. ds le désert :
\item[\vref{Mt 20:31}] taire ; mais ils c. encore plus fort :
\item[\vref{Mc 10:47}] se mit à c. et à dire :
\item[\vref{Lu 18:7}] ses élus, qui c. à lui jour
\item[\vref{Ac 8:7}] impurs sortaient, en c. à haute voix,
\item[\vref{Ac 16:17}] et ns., en c., et disant : Ces
\end{listverse}

\ConcordanceEntry{Crime}
\vspace{-2mm}
\begin{listverse}
\item[\vref{Ge 31:36}] Quel est mon c. ? Quel est mon
\item[\vref{Ge 50:17}] supplie, pardonne le c. des serviteurs du
\item[\vref{Ex 34:7}] ôtant l'iniquité, le c., et le péché,
\item[\vref{Lé 18:17}] sont tes proches parentes : C'est un c.
\item[\vref{1 Ch 10:13}] mourut pour le c. qu'il avait commis
\item[\vref{Esd 10:10}] avez augmenté le c. d'Israël.
\item[\vref{Job 13:23}] péchés ? Montre-moi mon c. et mon péché.
\item[\vref{Pr 28:21}] morceau de pain l'hom. commet un c.
\item[\vref{Es 24:6}] peine de leurs c. ; c'est pourquoi les
\item[\vref{Jé 44:9}] Avez-vs. oublié les c. de vos pères,
\item[\vref{Ez 9:9}] ville remplie de c. ; car ils ont
\item[\vref{Ez 18:24}] cause de son c. qu'il aura commis,
\item[\vref{Ez 18:31}] vs. ts les c. par lesquels vs.
\item[\vref{Am 1:3}] cause de trois c. de Damas, et
\item[\vref{Mi 1:5}] à cause du c. de Jacob, et
\item[\vref{Lu 23:4}] ne trouve aucun c. en cet hom.
\item[\vref{Lu 23:41}] qu'ont mérité nos c. ; mais celui-ci n'a
\item[\vref{Jn 19:4}] sachiez que je ne trouve aucun c. en lui.
\item[\vref{Ac 25:16}] défendre sur le c. dont on l'accuse.
\item[\vref{Ac 28:18}] en moi aucun c. qui mérite la
\end{listverse}

\ConcordanceEntry{Criminel}
\vspace{-2mm}
\begin{listverse}
\item[\vref{Ps 41:9}] Quelque action c. pèse sur lui ;
\item[\vref{1 Pi 4:3}] et du boire, et aux idolâtries c.;
\end{listverse}

\ConcordanceEntry{Cristal}
\vspace{-2mm}
\begin{listverse}
\item[\vref{Job 28:18}] corail ni du c. auprès d'elle ; la
\item[\vref{Ez 1:22}] semblable à un c. étincelant et terrible
\item[\vref{Ap 4:6}] semblable à du c. ; et au milieu
\item[\vref{Ap 21:11}] pierre de jaspe transparente com. du c.
\item[\vref{Ap 22:1}] transparent com. du c., qui sortait du
\end{listverse}

\ConcordanceEntry{Croire}
\vspace{-2mm}
\begin{listverse}
\item[\vref{Ex 4:1}] ils ne me c. pas et n'obéiront
\item[\vref{No 14:11}] jusqu'à qnd ne c.-t-il point en
\item[\vref{No 20:12}] vs. n'avez pas c. en moi, pour
\item[\vref{Ps 78:32}] encore et ne c. point à ses
\item[\vref{Ps 106:12}] Alors ils c. à ses paroles,
\item[\vref{Pr 14:15}] Le simple c. à tte parole ;
\item[\vref{Es 9:15}] dc qui font c. à ce peuple
\item[\vref{Es 53:1}] Qui a c. à notre prédication ? Et à qui
\item[\vref{Mt 9:28}] Jésus lr. dit : C.-vs. que je
\item[\vref{Mt 13:21}] en lui-mm, il c. pour un temps,
\item[\vref{Mt 21:32}] ne l'avez pas c.. Mais les publicains
\item[\vref{Mc 5:36}] Ne crains pas, c. seulement.
\item[\vref{Mc 9:23}] tu peux le c., ttes choses sont
\item[\vref{Mc 11:23}] son cœur, mais c. que ce qu'il
\item[\vref{Mc 16:16}] Celui qui c. et qui sera
\item[\vref{Lu 24:25}] est lent à c. tt ce que
\item[\vref{Jn 1:7}] à la lumière, afin que ts c. par lui.
\item[\vref{Jn 1:12}] à ceux qui c. en son Nom,
\item[\vref{Jn 2:22}] cela, et ils c. à l'Ecriture et
\item[\vref{Jn 2:23}] de Pâque, plusieurs c. en son Nom,
\item[\vref{Jn 3:12}] vs. ne les c. pas, comment croirez-vs.
\item[\vref{Jn 3:16}] afin que quiconque c. en lui ne
\item[\vref{Jn 3:18}] Celui qui c. en lui ne
\item[\vref{Jn 3:36}] Celui qui c. au Fils a
\item[\vref{Jn 5:46}] Mais si vs. c. Moïse, vs. me
\item[\vref{Jn 6:29}] Dieu, que vs. c. en celui qu'il
\item[\vref{Jn 7:38}] Celui qui c. en moi, des
\item[\vref{Jn 10:38}] vouliez pas me c., croyez à ces
\item[\vref{Jn 11:26}] quiconque vit et c. en moi ne
\item[\vref{Jn 14:10}] Ne c.-tu pas que JE SUIS en
\item[\vref{Jn 17:8}] et ils ont c. que tu m'as
\item[\vref{Jn 17:21}] que le monde c. que c'est toi
\item[\vref{Jn 20:29}] Thomas, tu as c.. Bénis sont ceux
\item[\vref{Jn 20:31}] afin que vs. c. que Jésus est
\item[\vref{Ac 4:4}] entendu la parole c. ; et le nombre
\item[\vref{Ac 17:29}] ne devons pas c. que la divinité
\item[\vref{Ac 19:4}] au peuple de c. en celui qui
\item[\vref{Ro 4:3}] l'Ecriture ? Qu'Abraham a c. en Dieu, et
\item[\vref{Ro 10:10}] Car c'est en c. du cœur qu'on
\item[\vref{Ro 10:11}] Quiconque c. en lui ne
\item[\vref{1 Co 3:18}] quelqu'un d'entre vs. c. être sage selon
\item[\vref{1 Co 8:2}] Et si quelqu'un c. savoir qq chose,
\item[\vref{1 Co 13:7}] couvre tt, elle c. tt, elle espère
\item[\vref{1 Co 14:37}] Si quelqu'un c. être prophète, ou
\item[\vref{2 Co 4:13}] est écrit : J'ai c., c'est pourquoi j'ai
\item[\vref{Ph 1:29}] non seulement de c. en lui, mais
\item[\vref{2 Th 2:11}] d'égarement, pour qu'ils c. au mensonge,
\item[\vref{1 Ti 3:16}] prêché aux Gentils, c. ds le monde,
\item[\vref{2 Ti 1:12}] en qui j'ai c. et je suis
\item[\vref{Ja 2:19}] Tu c. que Dieu est un, tu fais
\item[\vref{1 Pi 1:8}] en qui vs. c., quoique mntnt vs.
\item[\vref{1 Jn 5:13}] à vs. qui c. au Nom du
\item[\vref{Jud 1:5}] détruisit ensuite ceux qui n'avaient pas c.,
\end{listverse}

\ConcordanceEntry{Croître}
\vspace{-2mm}
\begin{listverse}
\item[\vref{Ge 17:6}] je te ferai c. très abondamment, et
\item[\vref{Ex 1:12}] il multipliait et c. en tte abondance ;
\item[\vref{1 S 2:26}] jeune garçon Samuel c. et il était
\item[\vref{Job 17:9}] les mains pures c. en force.
\item[\vref{Ps 92:13}] le palmier, il c. com. le cèdre
\item[\vref{Pr 9:9}] juste et il c. en science.
\item[\vref{Es 5:6}] les épines y c. ; et je commanderai
\item[\vref{Es 17:11}] tu as fait c. ce que tu
\item[\vref{Ez 16:7}] Je t'ai fait c. par millions com.
\item[\vref{Ez 37:6}] vs., je ferai c. de la chair
\item[\vref{Ez 44:20}] ne laisseront point c. leurs cheveux, mais
\item[\vref{Jon 4:6}] un ricin de c. au-dessus de Jonas,
\item[\vref{Mt 13:30}] Laissez-les c. ts deux ensemble
\item[\vref{Mc 4:27}] semence germe et c., sans qu'il sache
\item[\vref{Lu 2:40}] le petit enfant c. et se fortifiait
\item[\vref{Lu 2:52}] Et Jésus c. en sagesse, en stature, et en
\item[\vref{Lu 12:27}] Considérez comment c. les lis, ils
\item[\vref{Jn 3:30}] Il faut qu'il c. et que je
\item[\vref{Ac 6:7}] parole de Dieu c., et le nombre
\item[\vref{2 Co 10:15}] foi venant à c. en vs., ns.
\item[\vref{Col 1:10}] bonnes œuvres, et c. ds la connaissance
\item[\vref{1 Th 3:12}] Seign. vs. fasse c. et abonder de
\item[\vref{1 Pi 2:2}] spirituel et pur, afin que vs. c. par lui,
\item[\vref{2 Pi 3:18}] Mais c. ds la grâce et ds la
\end{listverse}

\ConcordanceEntry{Croix}
\vspace{-2mm}
\begin{listverse}
\item[\vref{Mt 10:38}] prend pas sa c., et ne vient
\item[\vref{Mt 16:24}] charge de sa c., et qu'il me
\item[\vref{Mt 27:32}] à porter la c. de Jésus.
\item[\vref{Mt 27:40}] Fils de Dieu, descends de la c. !
\item[\vref{Mc 15:21}] à porter la c. de Jésus.
\item[\vref{Jn 19:17}] Jésus, portant sa c., arriva au lieu
\item[\vref{Jn 19:19}] mit sur la c., où étaient écrits
\item[\vref{Jn 19:25}] près de la c. de Jésus se
\item[\vref{Jn 19:31}] pas sur la c. le jour du
\item[\vref{Ac 2:23}] mis à la c., vs. l'avez fait
\item[\vref{1 Co 1:17}] afin que la c. de Christ ne
\item[\vref{1 Co 1:18}] prédication de la c. est une folie
\item[\vref{Ga 5:11}] scandale de la c. est dc aboli.
\item[\vref{Ga 6:12}] persécutés pour la c. de Christ.
\item[\vref{Ga 6:14}] que de la c. de notre Seign.
\item[\vref{Ep 2:16}] corps par sa c., ayant détruit par
\item[\vref{Ph 2:8}] mm jusqu'à la mort de la c.
\item[\vref{Ph 3:18}] ennemis de la c. de Christ.
\item[\vref{Col 2:14}] aboli en le clouant à la c.
\item[\vref{Col 2:15}] spectacle, en triomphant d'elles par la c.
\item[\vref{Hé 12:2}] a souffert la c., ayant méprisé la
\end{listverse}

\ConcordanceEntry{Croyant}
\vspace{-2mm}
\begin{listverse}
\item[\vref{Mc 9:23}] croire, ttes choses sont possibles au c.
\item[\vref{Ro 10:4}] loi pour la justification de tt c.
\item[\vref{1 Co 14:22}] non pour les c., mais pour les
\item[\vref{1 Co 14:23}] peuple ou des non-c., ne diront-ils pas
\item[\vref{Ga 3:9}] foi sont bénis avec Abraham le c.
\end{listverse}

\ConcordanceEntry{Cruche}
\vspace{-2mm}
\begin{listverse}
\item[\vref{Ge 24:15}] que sortit, sa c. sur l'épaule, Rebecca,
\item[\vref{Jg 7:16}] main et des c. vides, avec des
\item[\vref{1 S 26:11}] chevet et la c. d'eau, et allons-ns.-en.
\item[\vref{1 R 17:14}] est ds la c. ne diminuera point,
\item[\vref{1 R 18:34}] dit : Remplissez quatre c. d'eau, puis versez-les
\item[\vref{1 R 19:6}] chauffées et une c. d'eau. Il mangea
\item[\vref{Ec 12:8}] brise, que la c. se rompe sur
\item[\vref{Mc 7:8}] le lavage des c. et des coupes,
\item[\vref{Mc 14:13}] hom. portant une c. d'eau, suivez-le.
\item[\vref{Jn 4:28}] ayant laissé sa c., s'en alla ds
\end{listverse}

\ConcordanceEntry{Crucifier}
\vspace{-2mm}
\begin{listverse}
\item[\vref{Mt 20:19}] verges, et le c. ; et le troisième
\item[\vref{Mt 26:2}] de l'hom. sera livré pour être c.
\item[\vref{Mt 27:22}] Ils lui dirent ts : Qu'il soit c. !
\item[\vref{Mt 27:31}] ses vêtements, et l'amenèrent pour le c.
\item[\vref{Mt 27:38}] Avec lui furent c. deux brigands, l'un
\item[\vref{Mc 15:20}] habits, et l'emmenèrent dehors pour le c.
\item[\vref{Mc 15:25}] la troisième heure, qnd ils le c.
\item[\vref{Jn 19:10}] pouvoir de te c., et que j'ai
\item[\vref{Ac 2:36}] ce Jésus, dis-je, que vs. avez c.
\item[\vref{Ac 4:10}] que vs. avez c., et que Dieu
\item[\vref{Ro 6:6}] hom. a été c. avec lui, afin
\item[\vref{1 Co 1:13}] Paul a-t-il été c. pour vs. ? Ou
\item[\vref{1 Co 1:23}] ns. prêchons Christ c., qui est un
\item[\vref{1 Co 2:2}] autre chose que Jésus-Christ et Jésus-Christ c.
\item[\vref{1 Co 2:8}] ils n'auraient pas c. le Seign. de
\item[\vref{Ga 2:20}] Je suis c. avec Christ ; et si je vis,
\item[\vref{Ga 5:24}] à Christ ont c. la chair avec
\item[\vref{Ga 6:14}] le monde est c. pour moi, com.
\item[\vref{Hé 6:6}] à eux, ils c. de nouveau le
\item[\vref{Ap 11:8}] où aussi notre Seign. a été c.
\end{listverse}

\ConcordanceEntry{Cruel}
\vspace{-2mm}
\begin{listverse}
\item[\vref{De 32:33}] venin de dragon, et du poison c. d'aspic.
\item[\vref{1 S 25:3}] mais l'hom. était c. et méchant ds
\item[\vref{Pr 5:9}] et tes années à un hom. c.
\item[\vref{Pr 11:17}] âme, mais le c. trouble sa chair.
\item[\vref{Pr 17:11}] mais le messager c. sera envoyé contre
\item[\vref{Es 13:9}] Yahweh arrive, jour c., jour de colère
\item[\vref{Es 19:4}] et un roi c. dominera sur eux,
\item[\vref{Jé 6:23}] javelot ; ils sont c. et n'ont pas
\item[\vref{Jé 30:14}] d'un châtiment d'hom. c., à cause de
\item[\vref{Da 8:23}] lèvera un roi c. et artificieux.
\item[\vref{Ha 1:6}] Chaldéens, ce peuple c. et impétueux, marchant
\item[\vref{2 Ti 3:3}] fidélité, calomniateurs, intempérants, c., haïssant les gens
\end{listverse}

\ConcordanceEntry{Cuirasse}
\vspace{-2mm}
\begin{listverse}
\item[\vref{1 S 17:38}] tête, et lui fit endosser une c.
\item[\vref{1 R 22:34}] jointures de la c.. Et le roi
\item[\vref{Né 4:16}] arcs et des c.. Les gouverneurs suivaient
\item[\vref{Ps 91:4}] fidélité est un bouclier et une c.
\item[\vref{Es 59:17}] justice com. d'une c., et le casque
\item[\vref{Ep 6:14}] ayant revêtu la c. de la justice ;
\item[\vref{1 Th 5:8}] ayant revêtu la c. de la foi
\item[\vref{Ap 9:9}] Elles avaient des c. com. des cuirasses
\item[\vref{Ap 9:17}] dessus, ayant des c. de feu, d'hyacinthe
\end{listverse}

\ConcordanceEntry{Culte}
\vspace{-2mm}
\begin{listverse}
\item[\vref{Jn 16:2}] vs. fera mourir croira rendre un c. à Dieu.
\item[\vref{Ac 7:42}] les livra au c. de l'armée du
\item[\vref{Ro 12:1}] qui est votre c. raisonnable.
\item[\vref{Ph 3:3}] à Dieu notre c. en Esprit, et
\item[\vref{Col 2:18}] et par un c. des anges, s'ingérant
\end{listverse}

\ConcordanceEntry{Cultiver}
\vspace{-2mm}
\begin{listverse}
\item[\vref{Ge 2:5}] il n'y avait pas d'hom. pour c. le sol.
\item[\vref{Ge 2:15}] d'Eden pour le c. et pour le
\item[\vref{Ge 3:23}] d'Eden pour qu'il c. la terre d'où
\item[\vref{Ge 4:12}] Quand tu c. la terre, elle ne te donnera
\item[\vref{De 28:39}] et tu les c. ; mais tu n'en
\item[\vref{Jos 24:13}] vs. n'aviez point c., des villes que
\item[\vref{2 S 9:10}] C'est pourquoi tu c. pour lui ces
\item[\vref{1 Ch 27:26}] la campagne et c. la terre.
\item[\vref{Pr 12:11}] Celui qui c. son champ sera
\item[\vref{Es 5:6}] plus taillée, ni c. ; les ronces et
\item[\vref{Es 7:25}] montagnes que l'on c. avec la bêche,
\item[\vref{Jé 27:11}] pour qu'elle le c. et qu'elle y
\item[\vref{Ez 36:34}] terre désolée sera c., tandis qu'elle n'était
\item[\vref{Os 10:13}] Vous avez c. la méchanceté, et
\end{listverse}

\ConcordanceEntry{Cupidité}
\vspace{-2mm}
\begin{listverse}
\item[\vref{Mc 7:22}] les vols, les c., les méchancetés, la
\item[\vref{Ep 4:19}] pour commettre tte sorte d'impureté avec c.
\item[\vref{Ep 5:3}] impureté, ni la c., ne soient pas
\item[\vref{Col 3:5}] désirs, et la c., qui est une
\item[\vref{1 Th 2:5}] pour prétexte la c., Dieu en est
\item[\vref{1 Th 4:6}] frère et de c. ds les affaires,
\item[\vref{2 Pi 2:3}] Par c., ils trafiqueront de vs. au moyen
\item[\vref{2 Pi 2:14}] exercé à la c. ; ce sont des
\end{listverse}

\ConcordanceEntry{Cusch}
\vspace{-2mm}
\begin{listverse}
\item[\vref{Ge 2:13}] en entourant tt le pays de C.
\item[\vref{Ge 10:6}] de Cham furent : C., Mitsraïm, Puth, et
\item[\vref{Ge 10:8}] C. engendra aussi Nimrod, c'est lui qui
\end{listverse}

\ConcordanceEntry{Cuve}
\vspace{-2mm}
\begin{listverse}
\item[\vref{Ex 30:18}] Fais aussi une c. d'airain, avec sa
\item[\vref{Job 24:11}] vendange ds les c. souffrent la soif.
\item[\vref{Pr 3:10}] d'abondance, et tes c. regorgeront de vin
\item[\vref{Es 5:2}] creusa aussi une c.. Puis il espéra
\item[\vref{Ag 2:16}] puiser de la c. cinquante mesures, il
\item[\vref{Ap 14:19}] ds la grande c. de la colère
\item[\vref{Ap 14:20}] Et la c. fut foulée hors de la ville ;
\item[\vref{Ap 19:15}] il foulera la c. du vin de
\end{listverse}

\ConcordanceEntry{Cymbale}
\vspace{-2mm}
\begin{listverse}
\item[\vref{2 S 6:5}] de tambourins, de sistres et de c.
\item[\vref{1 Ch 25:1}] luths et des c.. Et voici le
\item[\vref{Esd 3:10}] d'Asaph, avec les c., pour qu'ils célèbrent
\item[\vref{Né 12:27}] chants sur des c., des luths et
\item[\vref{Ps 150:5}] Louez-le avec les c. retentissantes ! Louez-le avec
\item[\vref{1 Co 13:1}] résonne ou une c. qui retentit.
\end{listverse}

\ConcordanceEntry{Cyrène}
\vspace{-2mm}
\begin{listverse}
\item[\vref{Mt 27:32}] un hom. de C., appelé Simon, et
\item[\vref{Ac 2:10}] est près de C., et ceux qui
\item[\vref{Ac 11:20}] Chypre et de C., qui, étant venus
\end{listverse}

\ConcordanceEntry{Cyrus}
\vspace{-2mm}
\begin{listverse}
\item[\vref{2 Ch 36:22}] première année de C., roi de Perse,
\item[\vref{Esd 1:1}] année dc de C., roi de Perse,
\item[\vref{Esd 5:13}] première année de C., roi de Babylone,
\item[\vref{Es 44:28}] Qui dit de C. : Il est mon
\item[\vref{Es 45:1}] parle Yahweh à son oint, à C.,
\item[\vref{Da 6:28}] le règne de C., roi de Perse.
\end{listverse}

\ConcordanceEntry{Dagon}
\vspace{-2mm}
\begin{listverse}
\item[\vref{Jg 16:23}] grand sacrifice à D., lr. dieu, et
\item[\vref{1 S 5:2}] la maison de D., et la posèrent
\item[\vref{1 Ch 10:10}] sa tête ds la maison de D.
\end{listverse}

\ConcordanceEntry{Damas}
\vspace{-2mm}
\begin{listverse}
\item[\vref{2 R 8:7}] se rendit à D.. Ben-Hadad, roi de
\end{listverse}
\begin{legend}
\NoAutoSpaceBeforeFDP{
\item Capitale de la Syrie : Ge 14:15; 1R 11:24
\item Les Syriens assujettis à David : 2 S 8:5-6
\item Reconquête de D par Jéroboam : 2 R 14:28
\item L'Assyrie vient en aide à Achaz : 2 R 16:7-9
\item Jésus se révèle à Saul : Ac 9:2; 26:12
\item Prophéties sur la chute de Damas : Es 8:4; 17:1; Jé 49:23
}
\end{legend}

\ConcordanceEntry{Dan}
\vspace{-2mm}
\begin{listverse}
\item[\vref{Ge 30:6}] pourquoi elle l'appela du nom de D.
\item[\vref{Ge 49:16}] D. jugera son peuple, com. l'une des
\item[\vref{No 1:39}] la tribu de D. qui furent dénombrés,
\item[\vref{De 33:22}] il dit de D. : Dan est un
\item[\vref{Jos 19:40}] des fils de D. selon leurs familles.
\end{listverse}

\ConcordanceEntry{Daniel}
\vspace{-2mm}
\begin{listverse}
\item[\vref{Da 1:6}] fils de Juda, D., Hanania, Mischaël et
\end{listverse}
\begin{legend}
\NoAutoSpaceBeforeFDP{
\item Juda livré à la captivité babylonienne : Da 1:1-7
\item Interpretations des songes : Da 1:17; 2:36; 4:19
\item Ecriture sur la muraille et interpretation : Da 5:5,25-28
\item Dieu sauve D de la fosse aux lions : Da 6:16-23
\item Vision des quatre animaux : Da 7-12; Mt 24:15
\item Autres : Ez 14:14-20; Da 1:8; 2:48; 6:2-3
}
\end{legend}

\ConcordanceEntry{Danser}
\vspace{-2mm}
\begin{listverse}
\item[\vref{Jg 21:21}] Silo sortiront pour d., alors vs. sortirez
\item[\vref{1 S 18:6}] en chantant et d. dvt le roi
\item[\vref{2 S 6:16}] David sauter et d. dvt Yahweh, elle
\item[\vref{Mt 11:17}] vs. n'avez pas d. ; ns. vs. avons
\item[\vref{Mt 14:6}] la fille d'Hérodias d. au milieu de
\end{listverse}

\ConcordanceEntry{Darius}
\vspace{-2mm}
\begin{listverse}
\item[\vref{Esd 6:1}] Alors le roi D. donna ses ordres
\item[\vref{Da 5:31}] et D., le Mède, reçut le royaume, étant
\item[\vref{Da 6:1}] Or D. trouva bon d'établir sur le royaume
\end{listverse}

\ConcordanceEntry{David}
\vspace{-2mm}
\begin{listverse}
\item[\vref{Ru 4:22}] Obed engendra Isaï, et Isaï engendra D.
\item[\vref{2 S 2:4}] là ils oignirent D. pour roi sur
\end{listverse}
\begin{legend}
\NoAutoSpaceBeforeFDP{
\item Issu de la Tribu de Juda, dernier fils d'Isaï : Ru 4:22; 1 Ch 2:13; Mt 1:6
\item Choisit par Yahweh : 1 S 2:35; 16:12-13; Ps 89:4; Ac 13:22
\item Ses descendants : 1 Ch 3:1
\item D. entre au service de Saül :  1 S 16:16-23; 2S 23:1
\item David tue Goliath :  1 S 17:49-51
\item Jalousie et assauts de Saül : 1 S 18:9,25; 20:33
\item D. échappe à Saül :  1 S 22:23; 23:14
\item D. épargne la vie de Saül : 1 S 24:11; 26:9-10
\item D. dans le pays des Philistins : 1 S 27:1-6; 30:1-6
\item David oint roi de Juda : 2 S 2:1; 4-11
\item Guerre entre Juda et Israël : 2 S 2:8-9; 2 S 3:1
\item David oint roi sur tout Israël : 2 S 5:1-5
\item D. s’empare de la forteresse de Sion : 2 S 5:1-10
\item L’arche accueillie à Jérusalem : 2 S 6; 1 Ch 13:15
\item Yahweh traite alliance avec David : 2 S 7:12-16
\item Péché de David : 2 S  11:2-5,14-17; 2 S 12:9
\item Nathan envoyé pour reprendre D. : 2 S 12:9-14
\item Repentance de David : 2 S 12:13; Ps 32:51
\item Conspiration d’Absalom son fils : 2 S 15
\item D. fuit son fils Absalom : 2 S 15:13-14,30; Ps 3
\item Retour du roi D. à Jérusalem : 2 S 19:14; 2 S 19:15
\item Yahweh donne la victoire à D. sur ses ennemis : 2 S 8; 21:15-22
\item Les vaillants hommes de David : 2 S 23
\item Dénombrement d’Israël et de Juda : 2 S 24; 1 Ch 21
\item Préparatifs (Construction du temple) : 1 Ch 22-26; 28
\item Mort de David, début du règne de Salomon : 1 R 2:10, 1 R 2:11
\item Jésus-Christ homme, Fils de David : Mt 1:1; 21:9
}
\end{legend}

\ConcordanceEntry{Débauche}
\vspace{-2mm}
\begin{listverse}
\item[\vref{Ec 10:17}] non pour se livrer à la d. !
\item[\vref{Lu 15:13}] son bien en vivant ds la d.
\item[\vref{Ro 13:13}] et de la d., des querelles et
\end{listverse}

\ConcordanceEntry{Débiteur}
\vspace{-2mm}
\begin{listverse}
\item[\vref{Es 24:2}] qui emprunte, du créancier com. du d.
\item[\vref{Ez 18:7}] gage à son d., qui ne ravit
\item[\vref{Lu 7:41}] créancier avait deux d. : L'un lui devait
\item[\vref{Lu 16:5}] appela chacun des d. de son maître,
\item[\vref{Ro 1:14}] Je suis d. tant aux Grecs qu'aux Barbares, tant
\end{listverse}

\ConcordanceEntry{Débora}
\vspace{-2mm}
\begin{listverse}
\item[\vref{Ge 35:8}] D., nourrice de Rebecca, mourut ; et elle
\item[\vref{Jg 4:4}] Dans ce temps-là, D., prophétesse, fem. de
\item[\vref{Jg 5:1}] En ce jour-là, D. chanta ce cantique
\item[\vref{Jg 5:7}] sois levée, moi D., jusqu'à ce que
\end{listverse}

\ConcordanceEntry{Debout}
\vspace{-2mm}
\begin{listverse}
\item[\vref{Jos 5:13}] main, se tenait d. dvt lui. Josué
\item[\vref{Ru 2:7}] elle a été d. depuis le matin
\item[\vref{2 S 22:34}] me fait tenir d. sur mes lieux
\item[\vref{1 R 8:14}] tte l'assemblée d'Israël se tenait là d.
\item[\vref{Esd 3:9}] Lévites, se tenaient d. pour surveiller ceux
\item[\vref{Job 41:1}] de se tenir d. dvt moi ?
\item[\vref{Ps 20:9}] ns., ns. tenons ferme, et restons d.
\item[\vref{Ps 39:6}] hom. quoiqu'il soit d. n'est que vanité.
\item[\vref{Es 3:13}] il se tient d. pour juger les
\item[\vref{Ez 8:11}] Schaphan, se tenaient d. dvt ces idoles,
\item[\vref{Ez 13:5}] de vs. tenir d. pour le combat
\item[\vref{Am 9:1}] Seign. se tenant d. sur l'autel, et
\item[\vref{Ha 2:1}] en sentinelle, j'étais d. ds la forteresse
\item[\vref{Za 3:1}] grand-prêtre, se tenant d. dvt l'Ange de
\item[\vref{Mt 6:5}] en se tenant d. ds les synagogues
\item[\vref{Lu 1:11}] et se tint d. à droite de
\item[\vref{Ro 11:20}] et tu es d. par la foi.
\item[\vref{1 Co 10:12}] qui pense demeurer d. prenne garde qu'il
\item[\vref{Ja 2:3}] Toi, tiens-toi là d. ! Ou : Assieds-toi ici
\end{listverse}

\ConcordanceEntry{Décapiter}
\vspace{-2mm}
\begin{listverse}
\item[\vref{Mt 14:10}] Et il envoya d. Jean ds la
\item[\vref{Mc 6:16}] que j'ai fait d., il est ressuscité
\item[\vref{Ap 20:4}] qui avaient été d. pour le témoignage
\end{listverse}

\ConcordanceEntry{Décapole}
\vspace{-2mm}
\begin{listverse}
\item[\vref{Mt 4:25}] Galilée, de la D., de Jérus., de
\item[\vref{Mc 5:20}] publier ds la D. les grandes choses
\item[\vref{Mc 7:31}] en traversant le pays de la D.
\end{listverse}

\ConcordanceEntry{Décharger}
\vspace{-2mm}
\begin{listverse}
\item[\vref{1 S 17:22}] Alors David se d. de son bagage,
\item[\vref{Ps 62:9}] en tt temps, d. votre cœur sur
\item[\vref{Ps 119:22}] D.-moi de l'opprobre et du mépris,
\end{listverse}

\ConcordanceEntry{Déchirer}
\vspace{-2mm}
\begin{listverse}
\item[\vref{Ge 31:39}] rapporté de bêtes d. par les bêtes
\item[\vref{Ge 49:27}] un loup qui d. ; le matin il
\item[\vref{Ex 22:31}] de la chair d. ds les champs,
\item[\vref{Jg 14:6}] sa main, il d. le lion com.
\item[\vref{1 S 15:28}] lui dit : Yahweh d. aujourd'hui le royaume
\item[\vref{1 R 11:11}] t'avais prescrites, je d. le royaume afin
\item[\vref{1 R 19:11}] vent impétueux qui d. les montagnes et
\item[\vref{Esd 9:3}] j'entendis cela, je d. mes vêtements et
\item[\vref{Job 18:4}] toi qui te d. toi-mm ds ta
\item[\vref{Ps 34:19}] ont le cœur d. par la douleur,
\item[\vref{Ec 3:7}] un temps pour d. et un temps
\item[\vref{Ez 13:20}] vs. ; et je d. ces coussins de
\item[\vref{Os 6:1}] lui qui a d., mais il ns.
\item[\vref{Os 13:8}] petits, et je d. l'enveloppe de lr.
\item[\vref{Joë 2:13}] D. vos cœurs et non vos vêtements,
\item[\vref{Za 11:16}] grasses, et il d. jusqu'aux cornes de
\item[\vref{Mt 7:6}] ne se retournent et ne vs. d.
\item[\vref{Lu 5:36}] autrement, le neuf d. le vieux, et
\item[\vref{Ac 14:14}] ayant appris cela, d. leurs vêtements et
\item[\vref{Ac 16:22}] préteurs, ayant fait d. leurs vêtements, ordonnèrent
\end{listverse}

\ConcordanceEntry{Déchirure}
\vspace{-2mm}
\begin{listverse}
\item[\vref{Mt 9:16}] l'habit, et la d. serait pire.
\end{listverse}

\ConcordanceEntry{Décision}
\vspace{-2mm}
\begin{listverse}
\item[\vref{Jg 20:7}] la question et prenez ici une d. !
\item[\vref{Job 38:2}] qui obscurcit mes d. par des paroles
\end{listverse}

\ConcordanceEntry{Déclarer}
\vspace{-2mm}
\begin{listverse}
\item[\vref{No 23:19}] Ce qu'il a d., ne l'exécutera-t-il pas ?
\item[\vref{Jos 7:19}] et fais-lui confession. D.-moi je te
\item[\vref{Jg 16:13}] dit des mensonges. D.-moi avec quoi
\item[\vref{1 R 8:24}] que tu as d. de ta bouche,
\item[\vref{Ps 22:23}] Je d. ton Nom à mes frères, je
\item[\vref{Ps 37:30}] et sa langue d. la justice.
\item[\vref{Da 9:23}] pour te la d., car tu es
\item[\vref{Am 4:13}] vent, et qui d. à l'hom. quelle
\item[\vref{Jon 3:10}] mal qu'il avait d. de lr. faire,
\item[\vref{Mi 3:8}] de Yahweh, pour d. à Jacob son
\item[\vref{Mt 13:35}] en paraboles, je d. les choses qui
\item[\vref{Mt 16:21}] Jésus commença à d. à ses disciples
\item[\vref{Ac 11:9}] que Dieu a d. pur, ne le
\item[\vref{Ro 1:4}] a été pleinement d. Fils de Dieu
\item[\vref{Ga 1:11}] Je vs. le d. dc, mes frères,
\item[\vref{1 Ti 4:11}] D. ces choses et enseigne-les.
\item[\vref{Hé 8:13}] Alliance, il a d. vieille la première ;
\item[\vref{Ap 10:7}] com. il l'a d. à ses serviteurs
\item[\vref{Ap 22:18}] Je le d. à quiconque entend les paroles de
\end{listverse}

\ConcordanceEntry{Décourager}
\vspace{-2mm}
\begin{listverse}
\item[\vref{No 32:7}] Pourquoi voulez-vs. d. les enfants d'Israël
\item[\vref{Jé 38:4}] hom. ! Car il d. les mains des
\item[\vref{Col 3:21}] vos enfants, afin qu'ils ne se d. pas.
\end{listverse}

\ConcordanceEntry{Décret}
\vspace{-2mm}
\begin{listverse}
\item[\vref{Esd 5:13}] Cyrus prit un d. pour rebâtir cette
\item[\vref{Da 4:17}] sentence est le d. de ceux qui
\item[\vref{Da 4:24}] roi, voici le d. du Très-Haut, qui
\item[\vref{Da 6:10}] sut que le d. était écrit, il
\item[\vref{Jon 3:7}] ds Ninive par d. du roi et
\item[\vref{So 2:2}] avant que le d. enfante, et que
\end{listverse}

\ConcordanceEntry{Dédaigner}
\vspace{-2mm}
\begin{listverse}
\item[\vref{Job 14:15}] te répondrai ; ne d. point l'ouvrage de
\item[\vref{Ps 89:39}] l'as rejeté et d. ! Tu t'es mis
\item[\vref{Pr 5:12}] mon cœur a-t-il d. les réprimandes ?
\item[\vref{La 2:7}] autel, il a d. son sanctuaire ; il
\item[\vref{Ez 16:45}] mère, qui a d. son mari et
\end{listverse}

\ConcordanceEntry{Dédicace}
\vspace{-2mm}
\begin{listverse}
\item[\vref{1 R 8:63}] d'Israël firent la d. de la maison
\item[\vref{2 Ch 7:5}] peuple firent la d. de la maison
\item[\vref{Esd 6:16}] captivité, célébrèrent la d. de cette maison
\item[\vref{Né 12:27}] Lors de la d. de la muraille
\item[\vref{Jn 10:22}] fête de la d. à Jérus., et
\end{listverse}

\ConcordanceEntry{Défaite}
\vspace{-2mm}
\begin{listverse}
\item[\vref{1 S 4:10}] sa tente. La d. fut très grande,
\item[\vref{2 S 18:7}] lieu, une grande d. de vingt mille
\item[\vref{1 R 20:21}] éprouver une grande d. aux Syriens.
\item[\vref{Hé 7:1}] revenait de la d. des rois, et
\end{listverse}

\ConcordanceEntry{Défaut}
\vspace{-2mm}
\begin{listverse}
\item[\vref{Ex 12:5}] un chevreau sans d., mâle, âgé d'un
\item[\vref{Lé 22:25}] eux est un d. en elles. Elles
\item[\vref{Ps 18:31}] Dieu sont sans d. ; la parole de
\item[\vref{Da 1:4}] n'y avait aucun d. corporel, beaux de
\item[\vref{1 Co 6:7}] déjà un grand d. en vs., que
\item[\vref{1 Pi 1:19}] d'un agneau sans d. et sans tache,
\end{listverse}

\ConcordanceEntry{Défendre}
\vspace{-2mm}
\begin{listverse}
\item[\vref{Ge 3:11}] dont je t'avais d. de manger ?
\item[\vref{Ps 35:1}] de David. Yahweh ! D.-moi contre mes
\item[\vref{Pr 22:23}] Car Yahweh d. lr. cause, et
\item[\vref{Es 1:17}] justice à l'orphelin, d. la cause de
\item[\vref{Mt 12:16}] Et il lr. d. avec menaces de
\item[\vref{Ac 4:17}] parmi le peuple, d.-lr. avec menaces
\end{listverse}

\ConcordanceEntry{Défense}
\vspace{-2mm}
\begin{listverse}
\item[\vref{Lu 21:14}] cœurs de ne pas préméditer votre d.
\item[\vref{1 Co 9:3}] C'est là ma d. contre ceux qui
\item[\vref{Ph 1:17}] établi pour la d. de l'Evangile.
\item[\vref{2 Ti 4:16}] ds ma première d., mais ts m'ont
\end{listverse}

\ConcordanceEntry{Dégoût}
\vspace{-2mm}
\begin{listverse}
\item[\vref{No 11:20}] en ayez du d., parce que vs.
\item[\vref{Ps 95:10}] cette génération en d. durant quarante ans,
\item[\vref{Ps 119:158}] Je vois avec d. les traîtres et
\item[\vref{Ez 6:9}] prendront eux-mêmes en d., à cause du
\item[\vref{Ez 20:43}] prendrez vs.-mêmes en d. à cause de
\item[\vref{Za 11:8}] âme aussi avait pour moi du d.
\end{listverse}

\ConcordanceEntry{Déguiser}
\vspace{-2mm}
\begin{listverse}
\item[\vref{1 R 14:2}] Lève-toi mntnt et d.-toi, en sorte
\item[\vref{2 Ch 18:29}] Je vais me d. pour aller au
\item[\vref{Pr 26:24}] la haine se d. par ses discours,
\item[\vref{2 Co 11:14}] Satan lui-mm se d. en ange de
\end{listverse}

\ConcordanceEntry{Dehors}
\vspace{-2mm}
\begin{listverse}
\item[\vref{Ge 15:5}] l'ayant fait sortir d., il lui dit :
\item[\vref{Esd 10:13}] moyen de demeurer d. ; et cette affaire
\item[\vref{Job 31:32}] passé la nuit d. ; j'ai ouvert ma
\item[\vref{Ps 69:9}] un hom. de d. pour les fils
\item[\vref{Pr 1:20}] sagesse crie hautement au-d., elle fait retentir
\item[\vref{Pr 7:12}] tantôt d., tantôt sur les places, elle se
\item[\vref{Es 43:8}] Amène d. le peuple aveugle qui a des
\item[\vref{Jé 38:13}] ils tirèrent Jérémie d. avec les cordes,
\item[\vref{Ez 2:10}] dedans et en d. ; des lamentations, des
\item[\vref{Za 5:6}] l'épha qui sort d.. Puis il dit :
\item[\vref{Mt 5:13}] qu'à être jeté d., et foulé aux
\item[\vref{Mt 12:46}] frères se tenaient d., cherchant à lui
\item[\vref{Mt 21:12}] Dieu. Il chassa d. ts ceux qui
\item[\vref{Mt 22:13}] les ténèbres de d., où il y
\item[\vref{Mt 23:25}] vs. nettoyez le d. de la coupe
\item[\vref{Mc 1:45}] il se tenait d., ds des lieux
\item[\vref{Lu 13:28}] Dieu, et que vs. serez jetés d.
\item[\vref{Jn 6:37}] ne mettrai pas d. celui qui viendra
\item[\vref{Jn 10:4}] ttes ses brebis d., il marche dvt
\item[\vref{Jn 11:43}] cria à haute voix : Lazare, sors d. !
\item[\vref{Jn 15:6}] il est jeté d., com. le sarment,
\item[\vref{Ap 22:15}] Mais seront laissés d. les chiens, les
\end{listverse}

\ConcordanceEntry{Délaisser}
\vspace{-2mm}
\begin{listverse}
\item[\vref{Jos 1:5}] je ne te d. point, et je
\item[\vref{1 Ch 28:20}] il ne te d. point, et il
\item[\vref{Ps 94:14}] Car Yahweh ne d. point son peuple,
\item[\vref{Pr 28:13}] confesse et les d., obtient miséricorde.
\item[\vref{Es 54:6}] com. une fem. d. et à l'esprit
\item[\vref{Es 62:4}] nommera plus la d., et on ne
\end{listverse}

\ConcordanceEntry{Délices}
\vspace{-2mm}
\begin{listverse}
\item[\vref{Né 9:25}] vécurent ds les d. de ta grande
\item[\vref{Job 22:26}] tu trouveras tes d. ds le Tout-Puissant,
\item[\vref{Ps 16:11}] ta face, des d. éternels à ta
\item[\vref{Ps 36:9}] les abreuveras au fleuve de tes d.
\item[\vref{Ps 37:4}] de Yahweh tes d., et il t'accordera
\item[\vref{Ps 94:19}] consolations font les d. de mon âme.
\item[\vref{Pr 8:30}] lui, j'étais ses d. de ts les
\item[\vref{Za 7:14}] d'un pays de d. ils ont fait
\end{listverse}

\ConcordanceEntry{Délier}
\vspace{-2mm}
\begin{listverse}
\item[\vref{Ps 102:21}] des prisonniers, pour d. ceux qui étaient
\item[\vref{Ps 146:7}] aux affamés ; Yahweh d. ceux qui sont
\item[\vref{Mt 16:19}] ce que tu d. sur la terre
\item[\vref{Mc 1:7}] pas digne de d. en me baissant
\item[\vref{Lu 13:16}] ne fallait-il pas d. de ce lien
\item[\vref{Jn 11:44}] Jésus lr. dit : D.-le, et laissez-le
\item[\vref{Ap 9:14}] avait la trompette : D. les quatre anges
\item[\vref{Ap 20:3}] faut qu'il soit d. pour un peu
\end{listverse}

\ConcordanceEntry{Delila}
\vspace{-2mm}
\begin{listverse}
\item[\vref{Jg 16:4}] vallée de Sorek. Elle se nommait D.
\item[\vref{Jg 16:8}] Philistins emmenèrent à D. sept cordes fraîches,
\end{listverse}

\ConcordanceEntry{Délivrance}
\vspace{-2mm}
\begin{listverse}
\item[\vref{Ge 45:7}] vs. faire vivre par une grande d.
\item[\vref{Ex 14:13}] et voyez la d. que Yahweh vs.
\item[\vref{Jg 15:18}] serviteur cette grande d. ; et mntnt mourrais-je
\item[\vref{2 S 22:51}] la tour de d. de son roi,
\item[\vref{Ps 3:9}] La d. vient de Yahweh ! Que ta bénédiction
\item[\vref{Ps 14:7}] de Sion la d. d'Israël ? Quand Yahweh
\item[\vref{Ps 25:5}] Dieu de ma d., je m'attends à
\item[\vref{Ps 32:7}] de triomphe à cause de ta d.. Sélah.
\item[\vref{Ps 35:3}] à mon âme : Je suis ta d. !
\item[\vref{Ps 62:2}] c'est de lui que vient ma d.
\item[\vref{Ps 68:20}] de ses biens ! Dieu est notre d.. Sélah.
\item[\vref{Ps 68:21}] le Dieu de d., et les issues
\item[\vref{Ps 116:13}] la coupe des d. et j'invoquerai le
\item[\vref{Pr 11:14}] prudence, mais la d. est ds la
\item[\vref{Pr 21:31}] bataille, mais la d. vient de Yahweh.
\item[\vref{Ec 8:8}] a point de d. ds ce combat,
\item[\vref{Es 33:2}] matin et notre d. au temps de
\item[\vref{La 3:26}] en silence la d. de Yahweh.
\item[\vref{Lu 2:38}] qui attendaient la d. de Jérus.
\item[\vref{Lu 4:19}] aux captifs la d., et aux aveugles
\item[\vref{Lu 21:28}] parce que votre d. approche.
\end{listverse}

\ConcordanceEntry{Délivrer}
\vspace{-2mm}
\begin{listverse}
\item[\vref{Ge 37:21}] cela et le d. de leurs mains,
\item[\vref{Ge 48:16}] l'Ange qui m'a d. de tt mal,
\item[\vref{Ex 3:8}] descendu pour le d. de la main
\item[\vref{Ex 5:23}] tu n'as point d. ton peuple.
\item[\vref{De 32:39}] personne qui puisse d. de ma main.
\item[\vref{Jg 6:14}] as et tu d. Israël de la
\item[\vref{Jg 6:15}] Seign., avec quoi d.-je Israël ? Voici,
\item[\vref{Jg 6:36}] Si tu veux d. Israël par ma
\item[\vref{1 S 4:8}] ns. ! Qui ns. d. de la main
\item[\vref{1 S 17:37}] Yahweh qui m'a d. de la griffe
\item[\vref{1 S 26:24}] et il me d. de ttes les
\item[\vref{2 S 3:18}] serviteur, que je d. mon peuple d'Israël
\item[\vref{2 S 22:1}] où Yahweh l'eut d. de la main
\item[\vref{2 R 18:29}] pourra pas vs. d. de ma main.
\item[\vref{2 Ch 25:15}] qui n'ont point d. lr. peuple de
\item[\vref{Est 4:14}] seront secourus et d. par qq autre
\item[\vref{Job 5:19}] Il te d. ds six afflictions, et à la
\item[\vref{Ps 22:9}] Yahweh ! Qu'il te d., et qu'il te
\item[\vref{Ps 34:20}] en grand nombre, mais Yahweh le d. de ts.
\item[\vref{Ps 50:15}] détresse, je te d., et tu me
\item[\vref{Ps 71:4}] Mon Dieu, d.-moi de la
\item[\vref{Ps 144:11}] retire-moi et d.-moi de la
\item[\vref{Pr 20:22}] mais attends Yahweh, et il te d.
\item[\vref{Pr 24:11}] retiens pas de d. ceux qu'on traîne
\item[\vref{Ec 8:8}] la méchanceté ne d. point son maître.
\item[\vref{Es 36:19}] de Sepharvaïm ? Ont-ils d. Samarie de ma
\item[\vref{Jé 15:21}] Et je te d. de la main des malins, et
\item[\vref{Da 3:17}] servons peut ns. d. de la fournaise
\item[\vref{Da 3:29}] dieu qui puisse d. com. lui.
\item[\vref{Da 6:16}] sers constamment sera celui qui te d.
\item[\vref{So 1:18}] ne pourront les d. au jour de
\item[\vref{Lu 1:74}] que ns. serions d. de la main
\item[\vref{Jn 12:27}] dirai-je ? Ô Père, d.-moi de cette
\item[\vref{Ac 7:34}] descendu pour les d.. Maintenant dc, va,
\item[\vref{Ro 7:2}] meurt, elle est d. de la loi
\item[\vref{Ro 7:6}] mntnt, ns. sommes d. de la loi,
\item[\vref{Col 1:13}] qui ns. a d. de la puissance
\item[\vref{2 Ti 4:18}] Seign. aussi me d. de tte mauvaise
\item[\vref{Hé 2:15}] et qu'il d. ts ceux qui, par crainte de
\item[\vref{2 Pi 2:7}] et s'il a d. le juste Lot,
\item[\vref{2 Pi 2:9}] Seign. sait ainsi d. de l'épreuve ceux
\item[\vref{Jud 1:5}] Seign. après avoir d. le peuple du
\end{listverse}

\ConcordanceEntry{Déluge}
\vspace{-2mm}
\begin{listverse}
\item[\vref{Ge 6:17}] ferai venir un d. d'eau sur la
\item[\vref{Ge 7:17}] Et le d. vint pendant quarante jours sur la
\item[\vref{Ps 29:10}] assis lors du d. ; Yahweh est assis
\item[\vref{Mt 24:39}] pas que le d. viendrait, jusqu'à ce
\item[\vref{2 Pi 2:5}] fait venir le d. sur le monde
\end{listverse}

\ConcordanceEntry{Demander}
\vspace{-2mm}
\begin{listverse}
\item[\vref{1 R 3:5}] Dieu lui dit : D. ce que tu
\item[\vref{Est 4:8}] roi pour lui d. grâce, et lui
\item[\vref{Ps 2:8}] D.-moi, et je te donnerai les
\item[\vref{Ps 104:21}] la proie, pour d. à Dieu lr.
\item[\vref{Es 47:13}] à force de d. des conseils. Que
\item[\vref{Es 65:1}] qui ne me d. pas, et je
\item[\vref{Jé 38:14}] Je vais te d. une chose, ne
\item[\vref{Mt 6:8}] besoin, avant que vs. le lui d.
\item[\vref{Mt 7:7}] D., et il vs. sera donné ; cherchez,
\item[\vref{Mt 18:19}] tt ce qu'ils d. lr. sera donné
\item[\vref{Mt 20:22}] ce que vs. d.. Pouvez-vs. boire la
\item[\vref{Mt 21:22}] quoi que vs. d. en priant Dieu,
\item[\vref{Mc 11:24}] ce que vs. d. en priant, croyez
\item[\vref{Jn 14:13}] ce que vs. d. en mon Nom,
\item[\vref{Jn 15:7}] demeurent en vs., d. tt ce que
\item[\vref{Jn 21:12}] disciples n'osait lui d. : Qui es-tu ? Sachant
\item[\vref{Ac 4:7}] Jean, ils lr. d. : Par quelle puissance,
\item[\vref{Ac 13:28}] de mort, ils d. à Pilate de
\item[\vref{Ro 8:26}] ns. convient de d. ds nos prières.
\item[\vref{Col 1:9}] vs., et de d. à Dieu que
\item[\vref{Ja 4:3}] Vous d. et vs. ne recevez pas, parce
\end{listverse}

\ConcordanceEntry{Démas}
\vspace{-2mm}
\begin{listverse}
\item[\vref{Col 4:14}] médecin bien-aimé, vs. salue, ainsi que D.
\item[\vref{2 Ti 4:10}] Car D. m'a abandonné, ayant aimé le présent
\end{listverse}

\ConcordanceEntry{Démétrius}
\vspace{-2mm}
\begin{listverse}
\item[\vref{Ac 19:24}] certain hom., nommé D., orfèvre, fabriquait de
\item[\vref{3 Jn 1:12}] rendent témoignage à D., et la vérité
\end{listverse}

\ConcordanceEntry{Demeure}
\vspace{-2mm}
\begin{listverse}
\item[\vref{Ex 15:13}] force à la d. de ta sainteté.
\item[\vref{De 12:5}] chercherez ds sa d., et vs. irez
\item[\vref{De 26:15}] de ta sainte d., des cieux, et
\item[\vref{2 S 7:10}] j'ai établi une d. à mon peuple,
\item[\vref{1 Ch 23:25}] il établira sa d. ds Jérus. à
\item[\vref{2 Ch 6:2}] qui sera ta d., et un domicile
\item[\vref{Ps 26:8}] Yahweh, j'aime la d. de ta maison,
\item[\vref{Ps 33:14}] lieu de sa d., il observe ts
\item[\vref{Ps 37:3}] le pays pour d. et la fidélité
\item[\vref{Ps 68:6}] des veuves. Dieu est ds sa d. sainte.
\item[\vref{Ps 74:7}] et profané la d. dédiée à ton
\item[\vref{Ps 76:3}] tente est à Salem, et sa d. à Sion.
\item[\vref{Ps 84:2}] armées, que tes d. sont aimables !
\item[\vref{Ps 104:3}] rencontre de sa d. ; il fait des
\item[\vref{Es 63:15}] vois de ta d. sainte et glorieuse :
\item[\vref{Ez 37:27}] Ma d. sera parmi eux ; je serai lr.
\item[\vref{Da 8:11}] et renversa la d. de son sanctuaire.
\item[\vref{Jn 14:2}] y a plusieurs d. ds la maison
\item[\vref{Ac 1:20}] Psaumes : Que sa d. soit déserte, que
\item[\vref{Jud 1:6}] mais qui ont abandonné lr. propre d. ;
\item[\vref{Ap 18:2}] est devenue la d. de démons, et
\end{listverse}

\ConcordanceEntry{Demeurer}
\vspace{-2mm}
\begin{listverse}
\item[\vref{Ge 13:6}] les porter pour d. ensemble ; car leurs
\item[\vref{De 1:6}] Vous avez assez d. ds cette montagne.
\item[\vref{Ps 91:1}] Celui qui d. sous la couverture
\item[\vref{Ps 112:3}] et sa justice d. à perpétuité.
\item[\vref{Ps 133:1}] que des frères d. unis ensemble !
\item[\vref{Pr 10:30}] les méchants ne d. pas sur la
\item[\vref{Jé 23:6}] sera sauvé, Israël d. en sécurité ; et
\item[\vref{Lu 19:5}] faut que je d. aujourd'hui ds ta
\item[\vref{Jn 1:38}] lui répondirent : Rabbi, c'est-à-dire Maître, où d.-tu ?
\item[\vref{Jn 6:56}] boit mon sang d. en moi, et
\item[\vref{Jn 8:31}] lui : Si vs. d. ds ma parole,
\item[\vref{Jn 12:34}] que le Christ d. éternellement ; et comment
\item[\vref{Jn 14:16}] autre consolateur, pour d. avec vs. éternellement,
\item[\vref{Jn 14:17}] connaissez, car il d. avec vs., et
\item[\vref{Jn 15:4}] D. en moi, et je demeurerai en
\item[\vref{Jn 15:9}] vs. ai aimés. D. ds mon amour.
\item[\vref{Ro 6:1}] Que dirons-ns. dc ? D.-ns. ds le
\item[\vref{1 Co 7:11}] s'en sépare, qu'elle d. sans être mariée,
\item[\vref{1 Co 7:20}] Que chacun d. ds la condition
\item[\vref{2 Co 3:15}] Moïse, le voile d. sur lr. cœur.
\item[\vref{Ga 5:1}] D. dc fermes ds la liberté pour
\item[\vref{2 Th 2:15}] pourquoi, mes frères, d. fermes, et retenez
\item[\vref{2 Ti 2:13}] sommes infidèles, il d. fidèle, car il
\item[\vref{Hé 7:3}] de Dieu. Il d. Prêtre continuellement.
\item[\vref{1 Pi 1:25}] parole du Seign. d. éternellement. Et cette
\item[\vref{1 Jn 2:24}] dès le commencement d. en vs., car
\item[\vref{1 Jn 3:6}] Quiconque d. en lui ne
\item[\vref{1 Jn 4:15}] de Dieu, Dieu d. en lui, et
\item[\vref{1 Jn 4:16}] et celui qui d. ds la charité,
\item[\vref{2 Jn 1:9}] et ne lui d. pas fidèle n'a
\item[\vref{Ap 17:10}] il faut qu'il d. pour un peu
\end{listverse}

\ConcordanceEntry{Démon}
\vspace{-2mm}
\begin{listverse}
\item[\vref{Lé 17:7}] leurs sacrifices aux d., avec lesquels ils
\item[\vref{Mt 7:22}] pas chassé les d. en ton Nom ?
\item[\vref{Mt 8:31}] Et les d. le priaient, en disant : Si tu
\item[\vref{Mt 9:34}] Il chasse les d. par le prince
\item[\vref{Mt 11:18}] et ils disent : Il a un d.
\item[\vref{Mt 12:22}] hom. tourmenté d'un d., aveugle et muet,
\item[\vref{Mt 17:18}] parla sévèrement au d., qui sortit de
\item[\vref{Mt 17:21}] cette sorte de d. ne sort que
\item[\vref{Mc 7:29}] parole, va, le d. est sorti de
\item[\vref{Mc 16:9}] de laquelle il avait chassé sept d.
\item[\vref{Mc 16:17}] Ils chasseront les d. en mon Nom ;
\item[\vref{Lu 4:41}] Les d. aussi sortirent de beaucoup de personnes,
\item[\vref{Lu 8:27}] possédé de plusieurs d.. Il ne portait
\item[\vref{Lu 9:42}] il approchait, le d. l'agita violemment com.
\item[\vref{Lu 10:17}] disant : Seign., les d. mêmes ns. sont
\item[\vref{Jn 6:70}] toutefois l'un de vs. est un d.
\item[\vref{Jn 7:20}] Tu as un d.. Qui est-ce qui
\item[\vref{1 Co 10:20}] les sacrifient aux d., et non à
\item[\vref{Ja 2:19}] fais bien ; les d. le croient aussi,
\item[\vref{Ap 9:20}] pas d'adorer les d., les idoles d'or,
\item[\vref{Ap 16:14}] des esprits de d., qui font des
\item[\vref{Ap 18:2}] la demeure de d., et la retraite
\end{listverse}

\ConcordanceEntry{Démoniaque}
\vspace{-2mm}
\begin{listverse}
\item[\vref{Mt 4:24}] diverses maladies, des d., des lunatiques, des
\item[\vref{Mt 8:28}] des Gadaréniens, deux d., sortant des sépulcres,
\item[\vref{Mt 9:32}] à Jésus un hom. muet et d.
\item[\vref{Lu 8:36}] racontèrent comment le d. avait été délivré.
\item[\vref{Jn 10:21}] les paroles d'un d. ; un démon peut-il
\end{listverse}

\ConcordanceEntry{Démonstration}
\vspace{-2mm}
\begin{listverse}
\item[\vref{1 Co 2:4}] humaine, mais en d. d'Esprit et de
\item[\vref{2 Th 1:5}] sont une manifeste d. du juste jugement
\item[\vref{Hé 11:1}] elle est une d. de celles qu'on
\end{listverse}

\ConcordanceEntry{Denier}
\vspace{-2mm}
\begin{listverse}
\item[\vref{Mt 18:28}] lui devait cent d.. Et l'ayant pris,
\item[\vref{Mt 20:2}] ouvriers à un d. par jour, il
\item[\vref{Mt 20:10}] mais ils reçurent aussi chacun un d.
\item[\vref{Mt 22:19}] tribut. Et ils lui présentèrent un d.
\item[\vref{Mc 6:37}] pour deux cents d., afin de lr.
\item[\vref{Mc 12:15}] tentez-vs. ? Apportez-moi un d. afin que je
\item[\vref{Lu 7:41}] devait cinq cents d., et l'autre cinquante.
\item[\vref{Ap 6:6}] blé pour un d., et les trois
\end{listverse}

\ConcordanceEntry{Dénombrement}
\vspace{-2mm}
\begin{listverse}
\item[\vref{Ex 30:12}] tu feras le d. des fils d'Israël,
\item[\vref{No 1:46}] on fit le d., furent six cent
\item[\vref{No 3:39}] on fit le d., lesquels Moïse et
\item[\vref{No 3:43}] dessus, selon lr. d., furent vingt-deux mille
\item[\vref{No 26:4}] Qu'on fasse le d. depuis l'âge de
\item[\vref{2 S 24:1}] Va, fais le d. d'Israël et de
\item[\vref{1 Ch 7:7}] vaillants, et lr. d. selon lr. généalogie
\item[\vref{1 Ch 21:1}] à faire le d. d'Israël.
\item[\vref{Né 7:5}] en faire le d. selon leurs généalogies.
\end{listverse}

\ConcordanceEntry{Dent}
\vspace{-2mm}
\begin{listverse}
\item[\vref{Ge 49:12}] vin, et les d. blanches de lait.
\item[\vref{Ex 21:24}] œil pour œil, d. pour dent, main
\item[\vref{No 11:33}] encore entre leurs d., avant qu'elle fût
\item[\vref{1 S 14:4}] y avait une d. de rocher d'un
\item[\vref{Ps 3:8}] tu brises les d. des méchants.
\item[\vref{Pr 25:19}] détresse, est une d. qui se rompt,
\item[\vref{Jé 31:29}] verts, et les d. des fils en
\item[\vref{Ez 18:2}] verts et les d. des enfants ont
\item[\vref{Da 7:7}] avait de grandes d. de fer, elle
\item[\vref{Mi 3:5}] paix qnd leurs d. ont de quoi
\item[\vref{Mt 8:12}] des pleurs et des grincements de d.
\item[\vref{Ac 7:54}] ils grinçaient des d. contre lui.
\end{listverse}

\ConcordanceEntry{Départ}
\vspace{-2mm}
\begin{listverse}
\item[\vref{No 10:2}] et pour le d. des camps.
\item[\vref{No 33:2}] Moïse écrivit leurs d., et leurs étapes,
\item[\vref{Ps 105:38}] réjouit à lr. d., car la peur
\item[\vref{Ps 121:8}] Yahweh gardera ton d. et ton arrivée,
\item[\vref{Ac 20:29}] sais qu'après mon d., il s'introduira parmi
\item[\vref{2 Ti 4:6}] temps de mon d. est proche.
\end{listverse}

\ConcordanceEntry{Dépenser}
\vspace{-2mm}
\begin{listverse}
\item[\vref{Es 55:2}] Pourquoi d.-vs. de l'argent pour ce qui
\item[\vref{Lu 10:35}] ce que tu d. de plus, je
\end{listverse}

\ConcordanceEntry{Déplaire}
\vspace{-2mm}
\begin{listverse}
\item[\vref{Ge 21:11}] Et cette parole d. fort à Abraham
\item[\vref{De 31:29}] fait ce qui d. aux yeux de
\item[\vref{Ps 51:6}] fait ce qui d. à tes yeux,
\end{listverse}

\ConcordanceEntry{Dépôt}
\vspace{-2mm}
\begin{listverse}
\item[\vref{1 Ti 6:20}] Timothée, garde le d., en fuyant les
\item[\vref{2 Ti 1:12}] pour garder mon d. jusqu'à ce jour-là.
\item[\vref{2 Ti 1:14}] Garde le bon d. par le Saint-Esprit
\end{listverse}

\ConcordanceEntry{Dépouiller}
\vspace{-2mm}
\begin{listverse}
\item[\vref{Ge 37:23}] frères, ils le d. de sa tunique,
\item[\vref{Ex 3:22}] filles : Ainsi vs. d. les Egyptiens.
\item[\vref{Lé 13:40}] a la tête d. de cheveux, c'est
\item[\vref{Pr 22:22}] Ne d. pas le pauvre, parce qu'il est
\item[\vref{Es 33:1}] à toi qui d. et qui n'as
\item[\vref{Jé 30:16}] ceux qui te d. seront dépouillés, et
\item[\vref{Da 7:12}] autres bêtes furent d. de lr. domination,
\item[\vref{2 Co 5:4}] non pas d'être d., mais d'être revêtus,
\item[\vref{Ep 4:22}] savoir que vs. d. le vieil hom.,
\item[\vref{Col 2:11}] qui consiste à d. le corps des
\item[\vref{Col 2:15}] Il a d. les principautés et les puissances, et
\item[\vref{Ap 17:16}] et nue, la d., et mangeront sa
\end{listverse}

\ConcordanceEntry{Déraciner}
\vspace{-2mm}
\begin{listverse}
\item[\vref{Ps 52:7}] tente ; il te d. de la terre
\item[\vref{Mt 13:29}] l'ivraie, vs. ne d. le blé en
\item[\vref{Mt 15:13}] Père céleste n'a pas plantée sera d.
\item[\vref{Lu 17:6}] à ce sycomore : D.-toi, et plante-toi
\item[\vref{Jud 1:12}] sans fruits, deux fois morts, et d. ;
\end{listverse}

\ConcordanceEntry{Derbe}
\vspace{-2mm}
\begin{listverse}
\item[\vref{Ac 14:6}] à Lystre, à D., et ds les
\item[\vref{Ac 16:1}] se rendit à D. et à Lystre.
\end{listverse}

\ConcordanceEntry{Dernier}
\vspace{-2mm}
\begin{listverse}
\item[\vref{Ge 49:1}] ce qui vs. arrivera ds les d. jours.
\item[\vref{No 24:14}] fera à ton peuple, ds les d. jours.
\item[\vref{De 4:30}] alors, ds les d. jours, tu retourneras
\item[\vref{Job 19:25}] se lèvera le d. sur la terre.
\item[\vref{Es 41:4}] premier, et JE SUIS avec les d.
\item[\vref{Es 44:6}] je suis le d. ; et à part
\item[\vref{Da 10:14}] peuple ds les d. jours, car la
\item[\vref{Mi 4:1}] arrivera ds les d. jours, que la
\item[\vref{Mt 20:16}] Ainsi les d. seront les premiers
\item[\vref{Jn 6:39}] mais que je le ressuscite au d. jour.
\item[\vref{Jn 8:9}] plus âgés jusqu'aux d. ; et Jésus resta
\item[\vref{Ac 2:17}] arrivera ds les d. jours, dit Dieu,
\item[\vref{1 Co 4:9}] qui sommes les d. apôtres, com. des
\item[\vref{1 Co 15:45}] âme vivante. Le d. Adam en Esprit
\item[\vref{2 Ti 3:1}] que ds les d. jours il surviendra
\item[\vref{Hé 1:2}] parlé ds ces d. jours par son
\item[\vref{Ja 5:3}] avez amassé des trésors pour les d. jours.
\item[\vref{1 Pi 1:20}] manifesté ds les d. temps pour vs.
\item[\vref{2 Pi 3:3}] d'abord ceci, qu'aux d. jours, il viendra
\item[\vref{1 Jn 2:18}] c'est ici la d. heure ; et com.
\item[\vref{Ap 1:11}] premier et le d.. Ecris ds un
\item[\vref{Ap 2:19}] et que tes d. œuvres surpassent les
\end{listverse}

\ConcordanceEntry{Dérober}
\vspace{-2mm}
\begin{listverse}
\item[\vref{Ge 31:30}] père, pourquoi as-tu d. mes dieux ?
\item[\vref{Ex 20:15}] Tu ne d. pas.
\item[\vref{De 5:19}] Tu ne d. point.
\item[\vref{Pr 9:17}] Les eaux d. sont douces et le pain pris
\item[\vref{Jé 7:9}] Ne d.-vs. pas ? Ne tuez-vs. pas ? Ne
\item[\vref{Jé 23:30}] prophètes qui se d. mes paroles l'un
\item[\vref{Mt 6:19}] et où les voleurs percent et d. ;
\item[\vref{Mt 27:64}] nuit, et ne d. son corps, et
\item[\vref{Mt 28:13}] de nuit le d., pendant que ns.
\item[\vref{Jn 10:10}] vient que pour d., tuer et détruire ;
\item[\vref{Ro 2:21}] ne doit pas d., tu dérobes !
\item[\vref{Ep 4:28}] Que celui qui d. ne dérobe plus ;
\end{listverse}

\ConcordanceEntry{Derrière}
\vspace{-2mm}
\begin{listverse}
\item[\vref{Ge 22:13}] voici, il vit d. lui un bélier
\item[\vref{Ex 14:19}] et s'en alla d. eux ; et la
\item[\vref{Ex 33:23}] tu me verras par-d., mais ma face
\item[\vref{De 25:18}] Dieu, attaqua par d. ceux qui étaient
\item[\vref{1 R 14:9}] tu m'as rejeté d. ton dos !
\item[\vref{Ps 78:66}] ses adversaires par d., et les mit
\item[\vref{Es 26:20}] ferme ta porte d. toi ; cache-toi pour
\item[\vref{Es 30:21}] celui qui sera d. toi, disant : Voici
\item[\vref{Es 38:17}] tu as jeté ts mes péchés d. ton dos.
\item[\vref{Joë 2:3}] feu dévorant, et d. lui une flamme
\item[\vref{Mt 9:20}] douze ans, s'approcha par-d., et toucha le
\item[\vref{Lu 23:26}] la croix pour qu'il la porte d. Jésus.
\item[\vref{Ap 1:10}] Seign., et j'entendis d. moi une voix
\item[\vref{Ap 4:6}] quatre animaux, pleins d'yeux dvt et d.
\end{listverse}

\ConcordanceEntry{Désapprouver}
\vspace{-2mm}
\begin{listverse}
\item[\vref{No 30:12}] ne l'ait pas d., alors ts ses
\item[\vref{1 Co 9:27}] peur d'être moi-mm d. après avoir prêché
\end{listverse}

\ConcordanceEntry{Descendant}
\vspace{-2mm}
\begin{listverse}
\item[\vref{Ge 15:13}] certaine que tes d. habiteront quatre cents
\item[\vref{No 18:23}] perpétuelle parmi vos d., et ils ne
\item[\vref{Jos 22:27}] et entre nos d. et les vôtres,
\item[\vref{Ps 68:27}] vs. qui êtes d. d'Israël !
\item[\vref{Jé 33:26}] dominent sur les d. d'Abraham, d'Isaac et
\item[\vref{Hé 11:18}] été dit : Les d. d'Isaac seront ta
\end{listverse}

\ConcordanceEntry{Descendre}
\vspace{-2mm}
\begin{listverse}
\item[\vref{Ge 11:5}] Alors Yahweh d. pour voir la
\item[\vref{Ge 11:7}] Allons ! D., et là confondons lr. langage afin
\item[\vref{Ge 46:4}] Je d. avec toi en Egypte, et je
\item[\vref{De 28:43}] et toi, tu d. toujours plus bas.
\item[\vref{Jos 2:15}] les fit dc d. avec une corde
\item[\vref{1 S 2:6}] vivre, qui fait d. au scheol et
\item[\vref{Jé 49:16}] je t'en ferai d., dit Yahweh.
\item[\vref{Joë 3:11}] ô Yahweh, fais d. tes hommes vaillants !
\item[\vref{Mt 27:40}] Fils de Dieu, d. de la croix !
\item[\vref{Mc 1:10}] et le Saint-Esprit d. sur lui com.
\item[\vref{Lu 10:30}] dit : Un hom. d. de Jérus. à
\item[\vref{Lu 19:5}] Zachée, hâte-toi de d. ; car il faut
\item[\vref{Jn 1:51}] Dieu monter et d. sur le Fils
\item[\vref{Jn 6:38}] car je suis d. du ciel, non
\item[\vref{Ro 10:6}] montera au ciel ? C'est en faire d. Christ ;
\item[\vref{Ep 4:9}] qu'il est premièrement d. ds les parties
\item[\vref{1 Th 4:16}] trompette de Dieu, d. du ciel, et
\item[\vref{Ap 12:12}] le diable est d. vers vs. animé
\item[\vref{Ap 18:1}] choses, je vis d. du ciel un
\end{listverse}

\ConcordanceEntry{Désert (le)}
\vspace{-2mm}
\begin{listverse}
\item[\vref{Ge 37:22}] qui est au d., mais ne mettez
\item[\vref{Ex 3:1}] troupeau derrière le d., et vint à
\item[\vref{Ex 4:27}] Va ds le d., au-dvt de Moïse.
\item[\vref{Ex 16:2}] murmura ds ce d. contre Moïse et
\item[\vref{Lé 16:21}] et l'enverra au d. par un hom.
\item[\vref{De 1:19}] grand et affreux d. que vs. avez
\item[\vref{De 29:5}] ans par le d. ; tes vêtements ne
\item[\vref{Jos 5:6}] marché ds le d. quarante ans jusqu'à
\item[\vref{Ps 78:40}] l'ont-ils irrité au d., et combien de
\item[\vref{Ps 106:14}] de convoitise au d. et ils tentèrent
\item[\vref{Ps 107:35}] Il transforme le d. en étangs d'eaux,
\item[\vref{Ca 8:5}] qui monte du d., mollement appuyée sur
\item[\vref{Es 32:15}] et que le d. devienne un Carmel
\item[\vref{Es 35:1}] Le d. et le lieu aride seront ds
\item[\vref{Es 40:3}] qui crie au d. est : Préparez le
\item[\vref{Es 43:19}] chemin ds le d., et des fleuves
\item[\vref{Jé 2:6}] conduits par un d., par un pays
\item[\vref{Jé 2:31}] Ai-je été un d. pour Israël, ou
\item[\vref{Jé 17:6}] bruyère ds le d., et il ne
\item[\vref{Ez 5:14}] de toi un d., un sujet d'opprobre
\item[\vref{Mt 3:3}] crie ds le d. : Préparez le chemin
\item[\vref{Mt 4:1}] l'Esprit ds le d., pour être tenté
\item[\vref{Mt 11:7}] voir ds le d. ? Un roseau agité
\item[\vref{Lu 5:16}] retiré ds les d., et priait.
\item[\vref{Jn 11:54}] contrée voisine du d., ds une ville
\item[\vref{Hé 3:8}] jour de la tentation ds le d.,
\item[\vref{Ap 12:6}] s'enfuit ds un d., où elle avait
\end{listverse}

\ConcordanceEntry{Désert}
\vspace{-2mm}
\begin{listverse}
\item[\vref{Lé 26:33}] pays sera dévasté, et vos villes d.
\item[\vref{Pr 21:19}] ds une terre d. qu'avec une fem.
\item[\vref{Mt 14:15}] Ce lieu est d. et l'heure est
\item[\vref{Mt 23:38}] Voici, votre maison va devenir d. ;
\item[\vref{Ac 1:20}] sa demeure soit d., que personne ne
\end{listverse}

\ConcordanceEntry{Déshonorer}
\vspace{-2mm}
\begin{listverse}
\item[\vref{Ge 34:5}] apprit qu'il avait d. Dina, sa fille.
\item[\vref{Lé 21:7}] fem. prostituée ou d. ; ils ne prendront
\item[\vref{Jg 19:24}] dehors ; vs. les d., et vs. ferez
\item[\vref{1 S 17:36}] car il a d. l'armée du Dieu
\item[\vref{2 S 13:12}] frère, ne me d. pas, car cela
\item[\vref{Pr 14:31}] tort au pauvre d. celui qui l'a
\item[\vref{La 5:11}] Ils ont d. les femmes ds Sion, les vierges
\item[\vref{Ez 23:8}] jeunesse, ils avaient d. sa virginité et
\item[\vref{Mi 7:6}] Car le fils d. le père, la
\item[\vref{Ro 1:24}] abandonnés à l'impureté, d. entre eux-mêmes leurs
\item[\vref{Ro 2:23}] la loi, tu d. Dieu par la
\item[\vref{1 Co 11:4}] sur la tête, d. son chef.
\end{listverse}

\ConcordanceEntry{Désir}
\vspace{-2mm}
\begin{listverse}
\item[\vref{Ge 3:16}] tes enfants ; tes d. se rapporteront vers
\item[\vref{No 15:39}] suivrez pas les d. de vos cœurs
\item[\vref{1 S 23:20}] puisque tt le d. de ton âme
\item[\vref{Ps 38:10}] Seign., tt mon d. est dvt toi,
\item[\vref{Ps 45:12}] roi porte ses d. sur ta beauté ;
\item[\vref{Ps 112:10}] consume ; [Tav.] les d. des méchants périssent.
\item[\vref{Ps 119:20}] brisée par le d. qui me porte
\item[\vref{Ps 145:19}] Il accomplit le d. de ceux qui
\item[\vref{Pr 13:4}] paresseux a des d. qu'il ne peut
\item[\vref{Ca 7:11}] bien-aimé et son d. est vers moi.
\item[\vref{Es 26:8}] souvenir sont le d. de notre âme.
\item[\vref{Jn 17:24}] Père, mon d. est que ceux
\item[\vref{Ro 8:19}] avec un ardent d., que les fils
\item[\vref{Ga 5:17}] chair a des d. contraires à ceux
\item[\vref{Ga 5:24}] chair avec ses passions et ses d.
\item[\vref{Col 3:5}] passions, les mauvais d., et la cupidité,
\item[\vref{1 Ti 6:9}] ds beaucoup de d. insensés et pernicieux
\item[\vref{Ap 18:14}] les fruits du d. de ton âme
\end{listverse}

\ConcordanceEntry{Désirer}
\vspace{-2mm}
\begin{listverse}
\item[\vref{De 5:21}] prochain ; tu ne d. point la maison
\item[\vref{Ps 20:5}] que ton cœur d., et qu'il fasse
\item[\vref{Ps 27:4}] Yahweh, que je d. ardemment : C'est d'habiter
\item[\vref{Ps 40:7}] Tu ne d. ni sacrifice ni offrande ; tu m'as
\item[\vref{Pr 10:24}] Dieu accorde aux justes ce qu'ils d.
\item[\vref{Pr 19:22}] que l'hom. doit d., c'est la miséricorde ;
\item[\vref{Es 26:9}] nuit, je te d. de mon âme,
\item[\vref{Mal 3:1}] l'Alliance, que vs. d., voici, il vient,
\item[\vref{Mt 13:17}] de justes ont d. voir les choses
\item[\vref{Lu 17:22}] viendront où vs. d. voir un des
\item[\vref{Lu 22:15}] lr. dit : J'ai d. vivement manger cet
\item[\vref{Ac 20:33}] Je n'ai d. ni l'argent, ni l'or, ni les
\item[\vref{Hé 11:16}] mntnt, ils en d. une meilleure, c'est-à-dire
\item[\vref{1 Pi 2:2}] d. ardemment, com. des enfants nouveau-nés, le
\item[\vref{Ap 9:6}] pas ; et ils d. mourir, mais la
\end{listverse}

\ConcordanceEntry{Désobéissance}
\vspace{-2mm}
\begin{listverse}
\item[\vref{Ro 5:19}] com. par la d. d'un seul hom.,
\item[\vref{2 Co 10:6}] vengeance de tte d., lorsque votre obéissance
\item[\vref{Hé 2:2}] transgression et tte d. a reçu une
\end{listverse}

\ConcordanceEntry{Désolation}
\vspace{-2mm}
\begin{listverse}
\item[\vref{De 32:10}] désert, ds la d. des hurlements d'une
\item[\vref{Est 4:3}] eut une grande d. parmi les Juifs ;
\item[\vref{Ps 106:14}] tentèrent Dieu ds le lieu de d.
\item[\vref{Es 51:3}] consolera ttes ses d., il rendra son
\item[\vref{Es 62:4}] ta terre la d. ; mais on t'appellera
\item[\vref{Jé 50:13}] sera plus qu'une d.. Quiconque passera près
\item[\vref{Da 9:2}] pour finir les d. de Jérus., était
\item[\vref{Da 12:11}] l'abomination de la d., il y aura
\item[\vref{Mt 24:15}] qui causera la d., qui a été
\item[\vref{Lu 21:20}] alors que sa d. est proche.
\end{listverse}

\ConcordanceEntry{Désordre}
\vspace{-2mm}
\begin{listverse}
\item[\vref{2 Ch 28:19}] avait mis le d. en Juda, et
\item[\vref{1 Th 5:14}] vivent ds le d., consolez ceux qui
\item[\vref{2 Th 3:7}] marché ds le d. parmi vs.,
\item[\vref{2 Th 3:11}] marchent ds le d., qui ne travaillent
\item[\vref{Ja 3:16}] là est le d., et tte sorte
\end{listverse}

\ConcordanceEntry{Dessécher}
\vspace{-2mm}
\begin{listverse}
\item[\vref{Ps 22:16}] Ma force se d. com. l'argile, et
\item[\vref{Ps 63:2}] cette terre aride, d., et sans eau.
\item[\vref{Ps 69:4}] mon gosier se d., mes yeux se
\item[\vref{Pr 17:22}] est un remède, mais l'esprit abattu d. les os.
\item[\vref{Ez 37:4}] et dis-lr. : Ossements d., écoutez la parole
\item[\vref{Za 11:17}] son bras se d., qu'il se dessèche
\end{listverse}

\ConcordanceEntry{Dessein}
\vspace{-2mm}
\begin{listverse}
\item[\vref{Ge 31:20}] pas de son d., parce qu'il s'enfuyait.
\item[\vref{De 31:21}] Je connais ses d., qu'il a déjà
\item[\vref{1 S 23:9}] connaissance des mauvais d. de Saül à
\item[\vref{Job 42:3}] entreprend d'obscurcir mes d. ? J'ai dc parlé,
\item[\vref{Ps 20:5}] désire, et qu'il fasse réussir tes d. !
\item[\vref{Ps 33:10}] il anéantit les d. des peuples ;
\item[\vref{Ps 33:11}] à toujours, les d. de son cœur
\item[\vref{Ps 40:6}] merveilles et tes d. envers ns. ; nul
\item[\vref{Ps 146:4}] mm jour ses d. périssent.
\item[\vref{Es 29:15}] cachent profondément leurs d., pour les dissimuler
\item[\vref{Es 48:15}] amené, et ses d. réussiront.
\item[\vref{Da 11:24}] il formera des d. contre les places
\item[\vref{Mi 4:12}] comprennent pas ses d. ; car il les
\item[\vref{Lu 1:51}] a dissipé les d. que les orgueilleux
\item[\vref{Lu 7:30}] lui, rendirent le d. de Dieu inutile
\item[\vref{Ac 2:23}] livré selon le d. arrêté et selon
\item[\vref{Ac 5:4}] cœur un pareil d. ? Tu n'as pas
\item[\vref{Ac 13:36}] son temps au d. de Dieu, est
\item[\vref{Ro 8:28}] ceux qui sont appelés selon son d.
\item[\vref{1 Co 4:5}] et manifestera les d. des cœurs. Alors
\item[\vref{Ep 3:11}] suivant le d. arrêté dès les
\item[\vref{2 Ti 1:9}] selon son propre d. et selon la
\item[\vref{Ap 17:17}] former un mm d., et de donner
\end{listverse}

\ConcordanceEntry{Destinée}
\vspace{-2mm}
\begin{listverse}
\item[\vref{Ps 31:16}] Ma d. est entre tes mains ; délivre-moi de
\item[\vref{Hé 9:9}] était une figure d. pour le temps
\end{listverse}

\ConcordanceEntry{Destiner}
\vspace{-2mm}
\begin{listverse}
\item[\vref{Ge 24:14}] que tu as d. à ton serviteur
\item[\vref{Ps 44:23}] com. des brebis d. à la boucherie.
\item[\vref{Pr 31:8}] ceux qui sont d. à la destruction.
\item[\vref{Jé 43:11}] Ceux qui sont d. à la mort
\item[\vref{Ac 13:48}] ceux qui étaient d. à la vie
\item[\vref{Ac 22:14}] nos pères t'a d. à connaître sa
\item[\vref{1 Th 5:9}] ns. a pas d. à la colère,
\item[\vref{1 Pi 1:10}] qui vs. était d., ont fait leurs
\end{listverse}

\ConcordanceEntry{Destructeur}
\vspace{-2mm}
\begin{listverse}
\item[\vref{Ex 12:23}] point que le d. entre ds vos
\item[\vref{Job 15:21}] croit que le d. se jette sur
\item[\vref{Pr 28:24}] un péché, est compagnon de l'hom. d.
\item[\vref{Es 54:16}] créé aussi le d. pour détruire.
\item[\vref{Jé 4:7}] la caverne, le d. des nations est
\item[\vref{1 Co 10:10}] eux murmurèrent et périrent par le d.
\item[\vref{Hé 11:28}] afin que le d. qui tuait les
\end{listverse}

\ConcordanceEntry{Détacher}
\vspace{-2mm}
\begin{listverse}
\item[\vref{Job 38:31}] des pléiades ou d. les chaînes d'orient ?
\item[\vref{Es 58:6}] choisi : Que tu d. les liens de
\item[\vref{Da 2:34}] qu'une pierre se d. sans l'aide d'une
\item[\vref{Na 1:13}] toi, et je d. tes liens.
\item[\vref{Mt 21:2}] ânon avec elle. D.-les et amenez-les-moi.
\item[\vref{Mc 11:5}] lr. dirent : Pourquoi d.-vs. cet ânon ?
\item[\vref{Ga 4:17}] ils veulent vs. d. de ns. afin
\end{listverse}

\ConcordanceEntry{Détour}
\vspace{-2mm}
\begin{listverse}
\item[\vref{2 R 3:9}] ils firent un d., et après une
\item[\vref{Es 30:12}] et ds les d., et que vs.
\end{listverse}

\ConcordanceEntry{Détourner}
\vspace{-2mm}
\begin{listverse}
\item[\vref{Ex 3:4}] que Moïse s'était d. pour regarder ; et
\item[\vref{Ex 5:4}] et Aaron, pourquoi d.-vs. le peuple
\item[\vref{No 14:43}] vs. vs. êtes d. de Yahweh, Yahweh
\item[\vref{De 2:27}] chemin, sans me d. ni à droite
\item[\vref{De 5:32}] ne vs. en d. ni à droite
\item[\vref{De 7:15}] Yahweh d. de toi tte maladie ; il ne
\item[\vref{De 16:19}] Tu ne te d. point de la justice, tu ne
\item[\vref{1 R 11:2}] certainement elles feraient d. vos cœurs pour
\item[\vref{1 R 11:4}] ses femmes firent d. son cœur vers
\item[\vref{Job 28:28}] sagesse, et se d. du mal c'est
\item[\vref{Ps 39:11}] D. de moi tes coups ! Je suis
\item[\vref{Ps 125:5}] ds des voies d., que Yahweh les
\item[\vref{Pr 13:14}] vie, pour se d. des pièges de
\item[\vref{Pr 19:27}] t'apprendre à te d. des paroles de
\item[\vref{Pr 22:6}] sera devenu vieux, il ne s'en d. pas.
\item[\vref{Es 14:27}] Sa main est étendue : Qui la d. ?
\item[\vref{Es 36:9}] Et comment ferais-tu d. le visage à
\item[\vref{Es 53:6}] ns. ns. sommes d., chacun suivait son
\item[\vref{Jé 8:4}] si on se d., ne revient-on pas ?
\item[\vref{Da 9:16}] ton indignation se d. de ta ville
\item[\vref{Ac 13:8}] résistait, cherchant à d. de la foi
\item[\vref{1 Ti 4:1}] temps, quelques-uns se d. de la foi
\item[\vref{1 Ti 6:21}] qui se sont d. de la foi.
\item[\vref{2 Pi 2:21}] connue et se d. du saint commandement
\end{listverse}

\ConcordanceEntry{Détresse}
\vspace{-2mm}
\begin{listverse}
\item[\vref{Ge 35:3}] jour de ma d., et qui a
\item[\vref{De 4:30}] seras ds la d., et que ttes
\item[\vref{Jg 2:15}] juré, ils furent ds une grande d.
\item[\vref{1 S 22:2}] étaient ds la d., qui avaient des
\item[\vref{2 S 22:7}] Dans ma d., j'ai invoqué Yahweh, j'ai crié à
\item[\vref{1 R 1:29}] délivré de tte d., est vivant !
\item[\vref{2 Ch 15:4}] Mais ds lr. d., ils sont revenus
\item[\vref{2 Ch 20:9}] toi ds notre d., et tu exauceras
\item[\vref{Né 9:27}] temps de lr. d., ils crièrent à
\item[\vref{Job 36:15}] c'est par la d. qu'il lui ouvre
\item[\vref{Ps 18:7}] Dans ma d., j'ai invoqué Yahweh, j'ai crié à
\item[\vref{Ps 22:12}] moi, car la d. est près de
\item[\vref{Ps 25:17}] [Tsade.] Les d. de mon cœur
\item[\vref{Ps 34:18}] il les délivre de ttes leurs d.
\item[\vref{Ps 50:15}] jour de ta d., je te délivrerai,
\item[\vref{Ps 71:20}] éprouver bien des d. et des malheurs,
\item[\vref{Ps 107:6}] Yahweh ds lr. d. et il les
\item[\vref{Pr 21:23}] langue garde son âme de la d.
\item[\vref{Es 8:23}] a de la d. ; si au commencement
\item[\vref{Es 63:9}] ds ttes leurs d., il a été
\item[\vref{Jé 16:19}] jour de la d. ! Les nations viendront
\item[\vref{Da 12:1}] un temps de d., tel qu'il n'y
\item[\vref{Na 1:9}] à néant ; la d. ne se lèvera
\item[\vref{Mt 24:21}] Car alors, la d. sera si grande
\item[\vref{Ac 7:11}] et une grande d., en sorte que
\item[\vref{2 Co 6:4}] afflictions, ds les nécessités, ds les d.,
\end{listverse}

\ConcordanceEntry{Détruire}
\vspace{-2mm}
\begin{listverse}
\item[\vref{Ge 6:17}] la terre, pour d. tte chair ds
\item[\vref{Ge 18:32}] Je ne la d. point pour l'amour
\item[\vref{Ex 15:9}] tirerai mon épée, ma main les d.
\item[\vref{De 2:15}] eux pour les d. du milieu du
\item[\vref{De 9:14}] Laisse-moi les d. et effacer lr.
\item[\vref{De 9:26}] Seign., Yahweh, ne d. point ton peuple,
\item[\vref{Job 10:8}] de mon corps ; et tu me d. !
\item[\vref{Ps 9:6}] les nations, tu d. le méchant, tu
\item[\vref{Ps 39:12}] son iniquité, tu d. com. la teigne
\item[\vref{Jé 1:10}] et que tu d., pour que tu
\item[\vref{Jé 45:4}] Voici, je vais d. ce que j'ai
\item[\vref{La 2:8}] avait projeté de d. les murailles de
\item[\vref{Za 12:9}] je chercherai à d. ttes les nations
\item[\vref{Mt 26:61}] dit : Je puis d. le temple de
\item[\vref{Mc 1:24}] venu pour ns. d. ? Je sais qui
\item[\vref{Jn 2:19}] et lr. dit : D. ce temple, et
\item[\vref{Jn 10:10}] dérober, tuer et d. ; moi, je suis
\item[\vref{Ac 5:39}] pourrez pas la d.. Et prenez garde
\item[\vref{1 Co 3:17}] Si quelqu'un d. le temple de
\item[\vref{Ga 5:15}] vs. ne soyez d. les uns par
\item[\vref{2 Th 2:8}] que le Seign. d. par le souffle
\item[\vref{2 Pi 2:12}] être prises et d., ils parlent d'une
\item[\vref{1 Jn 3:8}] apparu afin de d. les œuvres du
\end{listverse}

\ConcordanceEntry{Dette}
\vspace{-2mm}
\begin{listverse}
\item[\vref{2 R 4:7}] et paye ta d. ; et vs. vivrez,
\item[\vref{Né 5:10}] blé. Abandonnons je vs. prie, cette d. !
\item[\vref{Pr 22:26}] de ceux qui cautionnent pour des d.
\item[\vref{Mt 6:12}] remets ns. nos d., com. ns. aussi
\item[\vref{Mt 18:25}] et que la d. soit payée.
\item[\vref{Mt 18:27}] le relâcha, et lui remit la d.
\item[\vref{Mt 18:30}] jusqu'à ce qu'il ait payé la d.
\item[\vref{Mt 18:32}] en entier ta d., parce que tu
\item[\vref{Lu 7:42}] ts deux lr. d.. Lequel l'aimera le
\end{listverse}

\ConcordanceEntry{Deuil}
\vspace{-2mm}
\begin{listverse}
\item[\vref{Ge 50:11}] pays, voyant ce d. ds l'aire d'Athad,
\item[\vref{No 14:39}] le peuple fut ds un grand d.
\item[\vref{De 34:8}] pleurs et de d. sur Moïse furent
\item[\vref{2 S 1:12}] furent ds le d., ils pleurèrent et
\item[\vref{Ec 7:2}] une maison de d. que d'aller ds
\item[\vref{Jé 6:26}] cendre, prends le d. com. pour un
\item[\vref{Jé 31:13}] je changerai lr. d. en joie, et
\item[\vref{Ez 27:31}] ds lr. âme, en menant un d. amer.
\item[\vref{Da 10:2}] fus ds le d. pendant trois semaines
\item[\vref{Za 12:11}] aura un grand d. à Jérus., com.
\item[\vref{Lu 6:25}] serez ds le d. et ds les
\item[\vref{1 Co 5:2}] plutôt ds le d., afin que celui
\item[\vref{Ja 4:9}] soyez ds le d. et ds les
\item[\vref{Ap 18:8}] la mort, le d., et la famine,
\item[\vref{Ap 18:19}] pleurant et menant d., ils crieront, en
\item[\vref{Ap 21:4}] aura plus ni d., ni cri, ni
\end{listverse}

\ConcordanceEntry{Dévastateur}
\vspace{-2mm}
\begin{listverse}
\item[\vref{Es 16:4}] refuge contre le d. ! Car celui qui
\item[\vref{Jé 6:26}] amère ! Car le d. vient subitement sur
\item[\vref{Jé 48:8}] Et le d. entrera ds ttes les villes, et
\item[\vref{Na 2:3}] parce que les d. les ont vidés,
\end{listverse}

\ConcordanceEntry{Dévastation}
\vspace{-2mm}
\begin{listverse}
\item[\vref{Job 5:21}] peur de la d. qnd elle arrivera.
\item[\vref{Es 16:4}] d'extorsion cessera, la d. finira, celui qui
\item[\vref{Da 9:26}] déterminé que les d. dureront jusqu'à la
\item[\vref{Os 9:6}] cause de la d. ; l'Egypte les recueillera,
\end{listverse}

\ConcordanceEntry{Dévaster}
\vspace{-2mm}
\begin{listverse}
\item[\vref{Lé 26:32}] Je d. le pays, et vos ennemis qui
\item[\vref{2 Ch 36:21}] jours qu'elle demeura d. ; elle se reposa
\item[\vref{Es 64:10}] ns. avions de précieux a été d.
\item[\vref{Jé 4:27}] le pays sera d., mais je ne
\item[\vref{Ez 6:4}] Vos autels seront d., vos autels d'encens
\item[\vref{Da 9:17}] briller ta face sur ton sanctuaire d. !
\item[\vref{Os 10:1}] est une vigne d., il ne fait
\end{listverse}

\ConcordanceEntry{Devin}
\vspace{-2mm}
\begin{listverse}
\item[\vref{Lé 19:31}] ni vers les d. ; ne cherchez point
\item[\vref{No 22:7}] quoi payer le d.. Ils arrivèrent auprès
\item[\vref{De 18:14}] pronostiqueurs et les d. ; mais à toi,
\item[\vref{1 S 28:3}] qui évoquaient les morts, et les d.
\item[\vref{Es 3:2}] le prophète, le d. et l'ancien,
\item[\vref{Es 44:25}] rends insensés les d. ; qui renverse l'esprit
\item[\vref{Jé 29:8}] vs., et vos d., ne vs. séduisent
\item[\vref{Mi 3:7}] honteux, et les d. seront confondus ; ts
\item[\vref{Za 10:2}] vaines, et les d. prophétisent le mensonge,
\end{listverse}

\ConcordanceEntry{Dévorer}
\vspace{-2mm}
\begin{listverse}
\item[\vref{Ge 37:20}] bête féroce l'a d., et ns. verrons
\item[\vref{No 13:32}] un pays qui d. ses habitants et
\item[\vref{Ps 27:2}] contre moi pour d. ma chair, ce
\item[\vref{Ps 69:10}] ta maison me d., et les outrages
\item[\vref{Pr 20:25}] l'hom. que de d. les choses saintes,
\item[\vref{Ec 4:5}] les mains et d. sa propre chair.
\item[\vref{Jé 15:16}] les ai aussitôt d. ; tes paroles ont
\item[\vref{Jé 30:16}] ceux qui te d. seront dévorés, et
\item[\vref{Jé 51:34}] dira : Jérus. m'a d. et m'a brisée ;
\item[\vref{Os 5:7}] suffira pour les d. avec leurs biens.
\item[\vref{Ha 1:13}] qnd le méchant d. son prochain qui
\item[\vref{Mt 23:14}] hypocrites ! car vs. d. les maisons des
\item[\vref{Ga 5:15}] mordez et vs. d. les uns les
\item[\vref{Hé 10:27}] feu qui doit d. les adversaires.
\item[\vref{1 Pi 5:8}] lion rugissant, cherchant qui il pourra d.
\item[\vref{Ap 12:4}] accoucher, afin de d. son enfant, dès
\end{listverse}

\ConcordanceEntry{Dévouer}
\vspace{-2mm}
\begin{listverse}
\item[\vref{Ex 22:20}] Yahweh seul sera d. à la façon
\item[\vref{Lé 27:28}] Or tte chose d., que quelqu'un dévouera
\item[\vref{No 21:2}] mes mains, je d. ses villes par
\item[\vref{De 13:17}] ce qui sera d. ne s'attachera à
\item[\vref{De 20:17}] point de les d. par interdit : Héthiens,
\item[\vref{Jos 6:17}] La ville sera d. par le moyen
\item[\vref{1 S 15:9}] voulurent pas les d. par interdit, détruisant
\item[\vref{1 R 9:21}] d'Israël n'avaient pu d. par le moyen
\item[\vref{1 R 20:42}] l'hom. que j'avais d. par le moyen
\end{listverse}

\ConcordanceEntry{Diable}
\vspace{-2mm}
\begin{listverse}
\item[\vref{Mt 4:5}] Alors le d. le transporta ds la sainte ville,
\item[\vref{Mt 4:11}] Alors le d. le laissa. Et voici, des anges
\item[\vref{Mt 13:39}] semée, c'est le d. ; la moisson, c'est
\item[\vref{Mt 25:41}] préparé pour le d. et pour ses
\item[\vref{Jn 8:44}] issus c'est le d., et vs. voulez
\item[\vref{Jn 13:2}] alors que le d. avait déjà mis
\item[\vref{Ac 10:38}] sous l'empire du d., car Dieu était
\item[\vref{Ep 4:27}] pas lieu au d. de vs. perdre.
\item[\vref{Hé 2:14}] pouvoir de la mort, c'est-à-dire le d.,
\item[\vref{Ja 4:7}] Dieu ; résistez au d., et il s'enfuira
\item[\vref{1 Pi 5:8}] veillez : Car le d., votre adversaire, tourne
\item[\vref{1 Jn 3:8}] péché est du d., car le diable
\item[\vref{Ap 2:10}] arrivera que le d. mettra quelques-uns d'entre
\item[\vref{Ap 12:9}] ancien, appelé le d. et Satan, celui
\item[\vref{Ap 12:12}] mer ! Car le d. est descendu vers
\item[\vref{Ap 20:2}] qui est le d. et Satan, et
\item[\vref{Ap 20:10}] Et le d. qui les séduisait fut jeté ds
\end{listverse}

\ConcordanceEntry{Diaconesse}
\vspace{-2mm}
\begin{listverse}
\item[\vref{Ro 16:1}] Phœbé, qui est d. de l'église de
\end{listverse}

\ConcordanceEntry{Diacre}
\vspace{-2mm}
\begin{listverse}
\item[\vref{Ph 1:1}] Philippes, avec les évêques et les d. :
\item[\vref{1 Ti 3:8}] Que les d. aussi soient honnêtes, éloignés de la
\end{listverse}

\ConcordanceEntry{Diadème}
\vspace{-2mm}
\begin{listverse}
\item[\vref{Ex 39:30}] lame du saint d. d'or pur, sur
\item[\vref{Pr 4:9}] grâce, et elle t'ornera d'un magnifique d.
\item[\vref{Ap 12:3}] cornes, et sur ses têtes sept d.
\item[\vref{Ap 19:12}] sa tête plusieurs d., et il avait
\end{listverse}

\ConcordanceEntry{Diamant}
\vspace{-2mm}
\begin{listverse}
\item[\vref{Jé 17:1}] une pointe de d. ; il est gravé
\item[\vref{Ez 3:9}] semblable à un d., plus dur que
\item[\vref{Ez 28:13}] de topaze, de d., de chrysolithe, d'onyx,
\item[\vref{Za 7:12}] dur com. le d., pour ne pas
\end{listverse}

\ConcordanceEntry{Diane}
\vspace{-2mm}
\begin{listverse}
\item[\vref{Ac 19:24}] temples d'argent de D., et apportait beaucoup
\item[\vref{Ac 19:27}] de la grande D. ne tombe ds
\item[\vref{Ac 19:28}] Grande est la D. des Ephésiens !
\end{listverse}

\ConcordanceEntry{Dieu}
\vspace{-2mm}
\begin{listverse}
\item[\vref{Ge 1:1}] Au commencement, D. créa les cieux
\item[\vref{Ge 2:7}] Or Yahweh D. forma l'hom. de
\item[\vref{Ge 3:5}] mais D. sait que le jour où vs.
\item[\vref{Ge 17:1}] Je suis le D. Tout-Puissant. Marche dvt
\item[\vref{Ge 31:5}] auparavant ; toutefois le D. de mon père
\item[\vref{Ex 3:6}] Je suis le D. de ton père,
\item[\vref{Ex 4:16}] et tu seras d. pour lui.
\item[\vref{Ex 7:1}] établi pour être d. pour Pharaon, et
\item[\vref{Ex 29:45}] enfants d'Israël, et je serai lr. D.
\item[\vref{Ex 29:46}] suis Yahweh, lr. D., qui les ai
\item[\vref{Ex 32:16}] étaient l'ouvrage de D., et l'écriture était
\item[\vref{No 15:41}] suis Yahweh, votre D., qui vs. ai
\item[\vref{No 16:22}] et dirent : Ô D. ! Dieu des esprits
\item[\vref{De 5:26}] la voix du D. vivant parlant du
\item[\vref{De 6:4}] Israël ! Yahweh, notre D., Yahweh est Un.
\item[\vref{De 7:9}] c'est Yahweh, ton D., qui est Dieu.
\item[\vref{De 8:5}] que Yahweh, ton D., te châtie com.
\item[\vref{De 29:29}] à Yahweh, notre D. ; les choses révélées
\item[\vref{Jos 2:11}] Car Yahweh, votre D., est le Dieu
\item[\vref{Jos 22:22}] D., Dieu Yahweh, Dieu, DieuYahweh, le sait,
\item[\vref{Jos 24:2}] parle Yahweh, le D. d'Israël : Vos pères,
\item[\vref{Jg 13:9}] Et D. exauça la prière de Manoach, et
\item[\vref{Ru 2:12}] de Yahweh, le D. d'Israël, sous les
\item[\vref{1 S 2:3}] Yahweh est le D. qui sait tt,
\item[\vref{1 S 14:45}] car c'est avec D. qu'il a agi
\item[\vref{1 S 17:26}] cet incirconcis, pour insulter l'armée du D. vivant ?
\item[\vref{1 S 17:46}] la terre saura qu'Israël a un D.
\item[\vref{1 S 28:13}] Je vois un d. qui monte de
\item[\vref{2 S 7:22}] grand, ô Yahweh D. ! Car nul n'est
\item[\vref{2 S 22:30}] armes, avec mon D. je franchis une
\item[\vref{2 S 22:31}] La voie de D. est parfaite, la
\item[\vref{1 R 8:60}] Yahweh qui est D. et qu'il n'y
\item[\vref{1 R 11:4}] cœur vers d'autres d. ; et son cœur
\item[\vref{2 R 19:15}] dit : Ô Yahweh, D. d'Israël ! Qui est
\item[\vref{Esd 1:5}] ts ceux dont D. réveilla l'esprit, afin
\item[\vref{Né 1:5}] prie, ô Yahweh ! D. des cieux, Dieu
\item[\vref{Né 2:20}] lr. dit : Le D. des cieux lui-mm
\item[\vref{Né 4:20}] vers ns. ; notre D. combattra pour ns.
\item[\vref{Job 8:20}] D. ne rejette pas l'hom. intègre, il
\item[\vref{Job 12:4}] mais qui invoque D. et Dieu lui
\item[\vref{Job 22:12}] D. n'habite-t-il pas au plus haut des
\item[\vref{Job 36:22}] D. est élevé par sa puissance ; qui
\item[\vref{Job 36:26}] Voici, D. est grand et ns. ne le
\item[\vref{Ps 14:1}] a point de D. ! Ils se sont
\item[\vref{Ps 18:32}] Car qui est D., si ce n'est
\item[\vref{Ps 31:6}] ô Yahweh, le D. de vérité !
\item[\vref{Ps 31:15}] Yahweh ! Je dis : Tu es mon D. !
\item[\vref{Ps 42:6}] moi ? Attends-toi à D., car je le
\item[\vref{Ps 45:8}] C'est pourquoi, ô D., ton Dieu t'a
\item[\vref{Ps 46:2}] D. est notre retraite, notre force, et
\item[\vref{Ps 46:11}] que je suis D. : Je suis élevé
\item[\vref{Ps 48:2}] ville de notre D., sur la montagne
\item[\vref{Ps 50:7}] t'avertirai. JE SUIS D., ton Dieu, moi.
\item[\vref{Ps 72:18}] Béni soit Yahweh D., le Dieu d'Israël,
\item[\vref{Ps 75:8}] Car c'est D. qui gouverne ; il
\item[\vref{Ps 81:2}] allégresse à notre D., notre force ! Poussez
\item[\vref{Ps 84:10}] Ô D., notre bouclier, vois et regarde la
\item[\vref{Ps 84:12}] Car Yahweh D. est un soleil
\item[\vref{Ps 86:10}] merveilleuses. Tu es D., toi seul.
\item[\vref{Ps 90:2}] monde, d'éternité en éternité, tu es D.
\item[\vref{Ps 115:3}] Certes notre D. est au ciel,
\item[\vref{Ec 12:9}] l'esprit retourne à D. qui l'a donné.
\item[\vref{Es 9:5}] le Conseiller, le D. Puissant, le Père
\item[\vref{Es 41:10}] je suis ton D. ; je te fortifie,
\item[\vref{Es 41:17}] exaucerai ; moi, le D. d'Israël, je ne
\item[\vref{Es 43:10}] été formé de D., et il n'y
\item[\vref{Es 44:6}] moi il n'y a point de D.
\item[\vref{Es 45:15}] tu es le D. qui te caches,
\item[\vref{Es 45:21}] a point d'autre D. à part moi ;
\item[\vref{Es 45:22}] car JE SUIS D., et il n'y
\item[\vref{Es 54:5}] sera appelé le D. de tte la
\item[\vref{Jé 3:13}] contre Yahweh, ton D., tu t'es prostituée
\item[\vref{Jé 7:23}] je serai votre D., et vs. serez
\item[\vref{Jé 10:10}] Yahweh est le D. de vérité, c'est
\item[\vref{Jé 23:23}] Suis-je un D. de près, dit
\item[\vref{Ez 8:4}] la gloire du D. d'Israël était là,
\item[\vref{Os 11:9}] car je suis D., et non pas
\item[\vref{Am 4:13}] terre ; YAHWEH, LE D. DES ARMEES est
\item[\vref{Na 1:2}] Yahweh est un D. jaloux, il se
\item[\vref{Mal 2:10}] pas un seul D. qui ns. a
\item[\vref{Mt 15:31}] elle glorifiait le D. d'Israël.
\item[\vref{Mt 16:16}] es le Christ, le Fils du D. vivant.
\item[\vref{Mt 22:21}] César, et à D., ce qui est
\item[\vref{Mt 22:32}] Je suis le D. d'Abraham, le Dieu
\item[\vref{Mc 12:32}] la vérité, que D. est un, et
\item[\vref{Lu 1:30}] car tu as trouvé grâce dvt D.
\item[\vref{Lu 1:37}] Car rien n'est impossible à D.
\item[\vref{Lu 6:12}] passa tte la nuit à prier D.
\item[\vref{Lu 9:20}] répondit : Tu es le Christ de D.
\item[\vref{Jn 1:1}] Parole était avec D., et la Parole
\item[\vref{Jn 1:29}] Voici l'Agneau de D., qui ôte le
\item[\vref{Jn 3:16}] Car D. a tant aimé le monde qu'il
\item[\vref{Jn 4:24}] D. est Esprit, et il faut que
\item[\vref{Jn 8:41}] fornication ; ns. avons un seul père, D.
\item[\vref{Jn 10:33}] n'étant qu'un hom., tu te fais D.
\item[\vref{Jn 17:3}] le seul vrai D., et celui que
\item[\vref{Jn 20:28}] lui dit : Mon Seign., et mon D. !
\item[\vref{Ac 2:24}] Mais D. l'a ressuscité, ayant brisé les liens
\item[\vref{Ac 14:15}] vs. convertir au D. vivant, qui a
\item[\vref{Ac 24:15}] et ayant en D. cette espérance, com.
\item[\vref{Ro 3:30}] a un seul D., qui justifiera par
\item[\vref{Ro 5:8}] Mais D. prouve son amour envers ns., en
\item[\vref{Ro 8:31}] ces choses ? Si D. est pour ns.,
\item[\vref{Ro 15:33}] Que le D. de paix soit avec vs. ts !
\item[\vref{Ro 16:20}] Le D. de paix brisera bientôt Satan sous
\item[\vref{1 Co 1:9}] D., qui vs. a appelés à la
\item[\vref{1 Co 8:4}] a aucun autre D. qu'un seul.
\item[\vref{1 Co 14:33}] Car D. n'est pas un Dieu de confusion,
\item[\vref{1 Co 15:28}] choses, afin que D. soit tt en
\item[\vref{2 Co 3:5}] ns.-mêmes, mais notre capacité vient de D.,
\item[\vref{2 Co 5:19}] Car D. était en Christ, réconciliant le monde
\item[\vref{2 Co 9:8}] Et D. est Tout-Puissant pour vs. combler de
\item[\vref{Ga 3:6}] Abraham crut à D., et cela lui
\item[\vref{Ga 4:7}] aussi héritier de D. par Christ.
\item[\vref{Ga 4:8}] ne connaissant pas D., vs. serviez des
\item[\vref{Ep 4:6}] un seul D. et Père de ts, qui est
\item[\vref{Ph 2:13}] Car c'est D. qui produit en
\item[\vref{1 Th 1:9}] êtes convertis à D., en vs. séparant
\item[\vref{2 Th 2:4}] qui est appelé D., ou qu'on adore,
\item[\vref{1 Ti 1:17}] immortel, invisible, à D. seul sage, soient
\item[\vref{1 Ti 2:3}] et agréable dvt D., notre Sauveur,
\item[\vref{1 Ti 2:5}] Car D. est un, et aussi le Médiateur
\item[\vref{1 Ti 4:4}] tt ce que D. a créé est
\item[\vref{Hé 6:18}] est impossible que D. mente, ns. ayons
\item[\vref{Hé 9:14}] offert lui-mm à D. sans nulle tache,
\item[\vref{Hé 11:6}] qui vient à D., croie que Dieu
\item[\vref{Hé 11:10}] fondements, celle dont D. est l'architecte et
\item[\vref{1 Pi 5:5}] d'humilité ; parce que D. résiste aux orgueilleux,
\item[\vref{1 Jn 5:18}] est né de D. ne pèche point ;
\item[\vref{1 Jn 5:19}] sommes nés de D., mais le monde
\item[\vref{1 Jn 5:20}] le Fils de D. est venu, et
\item[\vref{Jud 1:25}] à D., seul sage, notre Sauveur, par Jésus-Christ,
\item[\vref{Ap 7:10}] est à notre D., qui est assis
\item[\vref{Ap 11:13}] furent épouvantés et donnèrent gloire au D. du ciel.
\item[\vref{Ap 16:14}] grand jour du D. Tout-Puissant.
\item[\vref{Ap 17:17}] Car D. a mis ds leurs cœurs de
\item[\vref{Ap 21:7}] je serai son D., et il sera
\item[\vref{Ap 22:5}] que le Seign. D. les éclairera, et
\end{listverse}

\ConcordanceEntry{dieu}
\vspace{-2mm}
\begin{listverse}
\item[\vref{Ex 20:3}] n'auras point d'autres d. dvt ma face.
\item[\vref{Ex 34:17}] te feras aucun d. de métal fondu.
\item[\vref{Jg 16:24}] il loua son d., en disant : Notre
\item[\vref{1 R 18:27}] voix, puisqu'il est d. ; mais il pense
\item[\vref{1 R 20:23}] lui dirent : Leur d. est un dieu
\item[\vref{Ps 81:10}] de toi de d. étranger, et tu
\item[\vref{Es 44:15}] fait aussi un d. et se prosterne
\item[\vref{Es 45:20}] et invoquent un d. qui ne sauve
\item[\vref{Da 3:15}] qui est le d. qui vs. délivrera
\item[\vref{Mi 4:5}] nom de lr. d. ; mais ns., ns.
\item[\vref{2 Co 4:4}] incrédules dont le d. de ce siècle
\item[\vref{Ph 3:19}] qui ont pour d. lr. ventre, et
\end{listverse}

\ConcordanceEntry{Différence}
\vspace{-2mm}
\begin{listverse}
\item[\vref{Ex 8:19}] je ferai la d. entre ton peuple
\item[\vref{Mal 3:18}] vs. verrez la d. qu'il y a
\item[\vref{Ac 15:9}] n'a fait aucune d. entre ns. et
\item[\vref{Ro 3:22}] croient. Car il n'y a nulle d.,
\item[\vref{Ro 10:12}] a pas de d. entre le Juif
\item[\vref{1 Co 4:7}] met de la d. entre toi et
\item[\vref{1 Co 7:34}] de mm une d. entre la fem.
\item[\vref{Ja 2:4}] pas fait de d. en vs.-mêmes, et
\end{listverse}

\ConcordanceEntry{Difficile}
\vspace{-2mm}
\begin{listverse}
\item[\vref{Ge 18:14}] chose qui soit d. à Yahweh ? Je
\item[\vref{Ex 18:26}] Moïse les choses d., et juger de
\item[\vref{De 1:17}] qui sera trop d. pour vs., et
\item[\vref{De 30:11}] n'est pas trop d. pour toi et
\item[\vref{2 R 2:10}] demandes une chose d.. Mais si tu
\item[\vref{2 R 5:13}] qq chose de d., ne l'aurais-tu pas
\item[\vref{Da 4:9}] secret ne t'est d., écoute les visions
\item[\vref{Mc 10:24}] enfants, qu'il est d. à ceux qui
\item[\vref{Lu 11:46}] hommes de fardeaux d. à porter, et
\item[\vref{2 Ti 3:1}] derniers jours il surviendra des temps d.
\item[\vref{Hé 5:11}] mais elles sont d. à expliquer, parce
\item[\vref{2 Pi 3:16}] a des points d. à comprendre, que
\end{listverse}

\ConcordanceEntry{Digne}
\vspace{-2mm}
\begin{listverse}
\item[\vref{Ex 15:11}] magnifique en sainteté, d. d'être révéré et
\item[\vref{Ps 113:3}] de Yahweh est d. de louanges depuis
\item[\vref{Mt 3:11}] ne suis pas d. de porter ses
\item[\vref{Mt 8:8}] ne suis pas d. que tu entres
\item[\vref{Mt 10:13}] maison en est d., que votre paix
\item[\vref{Mt 10:37}] moi n'est pas d. de moi, et
\item[\vref{Mt 22:8}] mais les conviés n'en étaient pas d.
\item[\vref{Lu 7:7}] suis pas cru d. d'aller moi-mm vers
\item[\vref{Lu 15:19}] ne suis plus d. d'être appelé ton
\item[\vref{1 Th 2:4}] ns. a considérés d. de ns. confier
\item[\vref{1 Ti 4:9}] parole certaine et d. d'être entièrement reçue.
\item[\vref{Hé 3:3}] a été jugé d. d'une gloire d'autant
\item[\vref{Hé 11:38}] monde n'était pas d., errant ds les
\item[\vref{Ap 4:11}] Seign., tu es d. de recevoir gloire,
\item[\vref{Ap 5:2}] forte : Qui est d. d'ouvrir le livre,
\item[\vref{Ap 5:12}] à mort est d. de recevoir puissance,
\end{listverse}

\ConcordanceEntry{Dignité}
\vspace{-2mm}
\begin{listverse}
\item[\vref{Ge 49:3}] qui excelle en d. et qui excelle
\item[\vref{1 R 15:13}] il ôta la d. de reine à
\item[\vref{Da 2:48}] éleva Daniel en d. et lui fit
\item[\vref{Ep 1:21}] puissance, de tte d. et de tte
\item[\vref{1 Ti 2:2}] sont constitués en d., afin que ns.
\item[\vref{Jud 1:8}] méprisent l'autorité et blasphèment contre les d.
\end{listverse}

\ConcordanceEntry{Diligent}
\vspace{-2mm}
\begin{listverse}
\item[\vref{Pr 10:4}] la main des d. enrichit.
\item[\vref{Pr 12:24}] La main des d. dominera, mais la
\item[\vref{Pr 12:27}] biens précieux de l'hom. sont au d.
\item[\vref{Pr 13:4}] mais l'âme des d. sera engraissée.
\item[\vref{Pr 21:5}] pensées d'un hom. d. le mènent à
\end{listverse}

\ConcordanceEntry{Dîme}
\vspace{-2mm}
\begin{listverse}
\item[\vref{Ge 14:20}] mains. Et Abram lui donna la d. de tt.
\item[\vref{Ge 28:22}] te donnerai la d., je donnerai la
\item[\vref{Lé 27:30}] Toute d. de la terre, tant du grain
\item[\vref{No 18:21}] Lévi, ttes les d. d'Israël, pour le
\item[\vref{No 18:26}] enfants d'Israël les d. que je vs.
\item[\vref{De 12:17}] tes portes la d. de ton blé,
\item[\vref{Né 10:37}] Dieu, et la d. de notre terre
\item[\vref{Mal 3:10}] Apportez ttes les d. aux magasins, afin
\item[\vref{Mt 23:23}] vs. payez la d. de la menthe,
\item[\vref{Lu 18:12}] je donne la d. de tt ce
\item[\vref{Hé 7:2}] sa part la d. de tt. Son
\item[\vref{Hé 7:9}] qui prend des d., les a payées
\end{listverse}

\ConcordanceEntry{Dimension}
\vspace{-2mm}
\begin{listverse}
\item[\vref{Lé 19:35}] les mesures de d., ni ds les
\item[\vref{Es 40:12}] a pris les d. des cieux avec
\end{listverse}

\ConcordanceEntry{Diminuer}
\vspace{-2mm}
\begin{listverse}
\item[\vref{Ge 8:8}] les eaux avaient d. à la surface
\item[\vref{Ex 5:8}] sans en rien d. ; car ils sont
\item[\vref{1 R 17:14}] la cruche ne d. point, jusqu'à ce
\item[\vref{Pr 13:11}] la fraude seront d., mais celui qui
\item[\vref{Ec 12:5}] parce qu'elles sont d., et qnd ceux
\item[\vref{Jé 29:6}] des filles ; multipliez-vs. là, et ne d. pas.
\item[\vref{Jn 3:30}] faut qu'il croisse et que je d.
\end{listverse}

\ConcordanceEntry{Dina}
\vspace{-2mm}
\begin{listverse}
\item[\vref{Ge 30:21}] enfanta une fille et la nomma D.
\item[\vref{Ge 34:1}] Or D., la fille que Léa avait enfantée
\item[\vref{Ge 34:25}] Lévi, frères de D., prirent leurs épées,
\end{listverse}

\ConcordanceEntry{Diotrèphe}
\vspace{-2mm}
\begin{listverse}
\item[\vref{3 Jn 1:9}] à l'église ; mais D., qui aime être
\end{listverse}

\ConcordanceEntry{Dire}
\vspace{-2mm}
\begin{listverse}
\item[\vref{1 R 22:14}] vivant ! J'annoncerai ce que Yahweh me d.
\item[\vref{Mt 7:4}] Ou comment peux-tu d. à ton frère :
\item[\vref{Mc 1:44}] et lui d. : Garde-toi de ne rien dire à
\item[\vref{Mc 14:58}] Nous l'avons entendu d. : Je détruirai ce
\item[\vref{Mc 16:8}] et elles ne d. rien à personne,
\item[\vref{Jn 3:11}] vérité, je te d., ns. disons ce
\item[\vref{Jn 12:49}] que je dois d. et annoncer.
\item[\vref{Jn 18:34}] toi-mm que tu d. cela, ou d'autres
\item[\vref{1 Co 12:3}] de Dieu, ne d. : Jésus est anathème !
\end{listverse}

\ConcordanceEntry{Directeur}
\vspace{-2mm}
\begin{listverse}
\item[\vref{Pr 6:7}] ni chef, ni d., ni gouverneur,
\end{listverse}

\ConcordanceEntry{Diriger}
\vspace{-2mm}
\begin{listverse}
\item[\vref{Ge 48:17}] d'Ephraïm, et la d. sur celle de
\item[\vref{De 32:10}] entouré, il l'a d., il l'a gardé
\item[\vref{Ps 23:2}] pâturages, il me d. près des eaux
\item[\vref{Pr 20:24}] de l'hom. sont d. par Yahweh, comment
\item[\vref{Es 9:15}] qui se laissent d. par eux se
\item[\vref{Es 40:13}] Qui a d. l'Esprit de Yahweh, ou qui a
\item[\vref{Jé 10:23}] pouvoir de l'hom. qui marche de d. ses pas.
\item[\vref{Mi 6:9}] la verge, et celui qui la d. !
\item[\vref{1 Th 5:12}] parmi vs., qui d. ds le Seign.,
\item[\vref{2 Th 3:5}] le Seign. veuille d. vos cœurs vers
\item[\vref{1 Ti 3:4}] Il faut qu'il d. honnêtement sa propre
\item[\vref{1 Ti 3:5}] ne sait pas d. sa propre maison,
\item[\vref{1 Ti 5:17}] les anciens qui d. convenablement soient jugés
\end{listverse}

\ConcordanceEntry{Discernement}
\vspace{-2mm}
\begin{listverse}
\item[\vref{Pr 1:4}] Pour donner du d. aux simples, aux
\item[\vref{Pr 8:5}] stupides, apprenez le d., et vs. ts,
\item[\vref{Pr 8:12}] j'habite avec le d., et je possède
\item[\vref{Ro 14:13}] usez plutôt de d. en ceci, qui
\item[\vref{Ph 1:10}] pour le d. des choses contraires, afin que vs.
\end{listverse}

\ConcordanceEntry{Discerner}
\vspace{-2mm}
\begin{listverse}
\item[\vref{Lé 10:10}] que vs. puissiez d. entre ce qui
\item[\vref{1 R 3:9}] ton peuple, pour d. le bien du
\item[\vref{Job 34:3}] Car l'oreille d. les discours com.
\item[\vref{Ps 139:2}] me lève ; tu d. de loin ma
\item[\vref{Pr 1:2}] et l'instruction, pour d. les paroles d'intelligence ;
\item[\vref{Ec 8:5}] cœur du sage d. le temps et
\item[\vref{Mt 16:3}] Vous savez bien d. l'aspect du ciel,
\item[\vref{Ro 2:18}] et tu sais d. ce qui est
\item[\vref{1 Co 2:15}] Mais l'hom. spirituel d. ttes choses et
\item[\vref{1 Co 12:10}] le don de d. les esprits ; à
\item[\vref{Hé 5:14}] sens exercés à d. le bien et
\end{listverse}

\ConcordanceEntry{Disciple}
\vspace{-2mm}
\begin{listverse}
\item[\vref{Es 8:16}] témoignage, scelle cette loi parmi mes d.
\item[\vref{Mt 9:14}] Alors les d. de Jean vinrent
\item[\vref{Mt 10:1}] appelé ses douze d., lr. donna le
\item[\vref{Mt 10:24}] Le d. n'est pas au-dessus du maître, ni
\item[\vref{Mt 10:25}] Il suffit au d. d'être traité com.
\item[\vref{Mt 10:42}] qu'il est mon d., je vs. le
\item[\vref{Mt 12:2}] dirent : Voici, tes d. font ce qu'il
\item[\vref{Mt 28:13}] disant : Dites : Ses d. sont venus de
\item[\vref{Mt 28:19}] les nations des d., les baptisant au
\item[\vref{Mc 16:7}] dites à ses d., et à Pierre,
\item[\vref{Lu 6:40}] Le d. n'est pas au-dessus de son maître ;
\item[\vref{Lu 10:1}] désigna soixante-dix autres d., et il les
\item[\vref{Lu 14:26}] vie, il ne peut être mon d.
\item[\vref{Jn 1:37}] Et les deux d. l'entendirent tenant ce
\item[\vref{Jn 2:11}] gloire, et ses d. crurent en lui.
\item[\vref{Jn 4:1}] baptisait plus de d. que Jean.
\item[\vref{Jn 6:66}] plusieurs de ses d. l'abandonnèrent, et ils
\item[\vref{Jn 8:31}] ma parole, vs. serez vraiment mes d.
\item[\vref{Jn 13:35}] vs. êtes mes d., si vs. avez
\item[\vref{Jn 15:8}] glorifié, et vs. serez alors mes d.
\item[\vref{Jn 18:15}] avec un autre d., suivait Jésus. Et
\item[\vref{Jn 20:2}] et vers l'autre d. que Jésus aimait,
\item[\vref{Jn 21:24}] C'est ce d. qui rend témoignage de ces choses,
\item[\vref{Ac 6:1}] jours-là, com. les d. se multipliaient, il
\item[\vref{Ac 9:26}] se joindre aux d. ; mais ts le
\item[\vref{Ac 9:36}] Joppé une fem. d., appelée Tabitha, qui
\item[\vref{Ac 14:22}] fortifiant l'esprit des d., et les exhortant
\item[\vref{Ac 15:10}] voulant imposer aux d. un joug que
\item[\vref{Ac 19:9}] d'eux, sépara les d., et enseigna ts
\item[\vref{Ac 20:30}] but d'attirer les d. après eux.
\end{listverse}

\ConcordanceEntry{Discoureur}
\vspace{-2mm}
\begin{listverse}
\item[\vref{Ac 17:18}] veut dire ce d. ? Les autres disaient :
\item[\vref{Tit 1:10}] se soumettre, vains d., et séducteurs d'esprits,
\end{listverse}

\ConcordanceEntry{Discours}
\vspace{-2mm}
\begin{listverse}
\item[\vref{Ge 37:11}] lui, mais son père retenait ses d.
\item[\vref{Ex 19:6}] sont là les d. que tu tiendras
\item[\vref{Ps 139:23}] mon cœur ! Eprouve-moi et considère mes d. !
\item[\vref{Mt 7:28}] eut achevé ce d., la foule fut
\item[\vref{Ac 6:5}] Et ce d. plut à tte l'assemblée qui était
\item[\vref{Ac 10:44}] prononçait encore ce d., le Saint-Esprit descendit
\item[\vref{Ac 20:9}] pendant le long d. de Paul ; entraîné
\item[\vref{1 Co 1:17}] pas avec des d. de la sagesse
\item[\vref{1 Co 2:1}] venu avec des d. pompeux, remplis de
\item[\vref{Ep 5:6}] par de vains d. ; car à cause
\item[\vref{Col 2:4}] trompe par des d. séduisants.
\item[\vref{1 Ti 1:6}] se sont écartés ds de vains d.,
\item[\vref{2 Ti 2:16}] Mais évite les d. vains et profanes ;
\item[\vref{2 Ti 4:3}] oreilles par des d. agréables, ils chercheront
\item[\vref{Hé 11:14}] qui tiennent ces d. montrent clairement qu'ils
\item[\vref{Ja 1:22}] vs. trompant vs.-mêmes par de vains d.
\item[\vref{2 Pi 2:18}] en prononçant des d. fort enflés de
\end{listverse}

\ConcordanceEntry{Discussion}
\vspace{-2mm}
\begin{listverse}
\item[\vref{Ac 15:2}] et une vive d. ; et les frères
\item[\vref{Ac 15:7}] après une grande d., Pierre se leva,
\item[\vref{Ac 25:19}] avec lui des d. relatives à leurs
\item[\vref{Tit 3:9}] Mais évite les d. folles, les généalogies,
\end{listverse}

\ConcordanceEntry{Discuter}
\vspace{-2mm}
\begin{listverse}
\item[\vref{Mc 8:11}] se mirent à d. avec lui, et
\item[\vref{Mc 9:16}] disant : De quoi d.-vs. avec eux ?
\item[\vref{Mc 9:34}] car ils avaient d. entre eux en
\item[\vref{Mc 12:28}] les avait entendus d., voyant qu'il lr.
\end{listverse}

\ConcordanceEntry{Disette}
\vspace{-2mm}
\begin{listverse}
\item[\vref{De 28:48}] et ds la d. de ttes choses,
\item[\vref{Job 30:3}] cause de la d. et de la
\item[\vref{Pr 6:11}] voyageur, et ta d. com. un soldat.
\item[\vref{Pr 13:25}] le ventre des méchants aura la d.
\item[\vref{Pr 14:23}] vains discours ne tournent qu'à la d.
\item[\vref{Pr 28:27}] n'aura point de d., mais celui qui
\item[\vref{Am 4:6}] villes, et la d. de pain ds
\item[\vref{Ph 4:12}] l'abondance et à être ds la d.
\end{listverse}

\ConcordanceEntry{Disparaître}
\vspace{-2mm}
\begin{listverse}
\item[\vref{No 16:33}] recouvrit, et ils d. au milieu de
\item[\vref{Jg 6:21}] l'Ange de Yahweh d. à ses yeux.
\item[\vref{2 S 21:5}] pour ns. faire d. de tt le
\item[\vref{2 R 23:19}] Josias fit encore d. ttes les maisons
\item[\vref{Es 19:3}] L'esprit de l'Egypte d. du milieu d'elle,
\item[\vref{Es 29:14}] et l'intelligence de ses hommes intelligents d.
\item[\vref{Es 59:15}] la vérité a d., et quiconque se
\item[\vref{Mt 5:18}] pas, il ne d. pas de la
\item[\vref{Lu 24:31}] reconnurent ; mais il d. de dvt eux.
\item[\vref{Ac 16:19}] la servante voyant d. l'espoir de lr.
\item[\vref{Ap 21:1}] première terre avaient d., et la mer
\end{listverse}

\ConcordanceEntry{Disperser}
\vspace{-2mm}
\begin{listverse}
\item[\vref{Ge 11:4}] ns. ne soyons d. sur tte la
\item[\vref{Ge 11:8}] Ainsi, Yahweh les d. de là par
\item[\vref{Lé 26:33}] Je vs. d. parmi les nations, et je tirerai
\item[\vref{De 30:4}] Quand tu seras d. à l'extrémité des
\item[\vref{2 S 5:20}] dit : Yahweh a d. mes ennemis dvt
\item[\vref{1 R 22:17}] vu tt Israël d. par les montagnes,
\item[\vref{Ps 44:12}] tu ns. as d. parmi les nations.
\item[\vref{Es 11:12}] il recueillera les d. de Juda des
\item[\vref{Es 33:3}] tu te lèves, les nations se d.
\item[\vref{Jé 23:2}] peuple : Vous avez d. mes brebis, et
\item[\vref{Za 13:7}] les brebis seront d. ; et je tournerai
\item[\vref{Mt 9:36}] parce qu'elles étaient d. et errantes, com.
\item[\vref{Mt 12:30}] celui qui n'assemble pas avec moi d.
\item[\vref{Jn 7:35}] ceux qui sont d. chez les Grecs,
\item[\vref{Jn 11:52}] seul corps les enfants de Dieu d.
\item[\vref{Jn 16:32}] où vs. serez d. chacun de son
\item[\vref{Ac 5:37}] s'étaient joints à lui ont été d.
\end{listverse}

\ConcordanceEntry{Disposer}
\vspace{-2mm}
\begin{listverse}
\item[\vref{2 Ch 29:36}] Dieu avait ainsi d. le peuple ; car
\item[\vref{2 Ch 30:19}] pour quiconque a d. son cœur à
\item[\vref{Job 11:13}] Si tu d. ton cœur, et que tu étends
\item[\vref{Job 38:33}] lois du ciel ? D.-tu de son
\item[\vref{Ps 51:14}] qu'un esprit bien d. me soutienne.
\item[\vref{Ps 108:1}] Mon cœur est d., ô Dieu ! Ma
\item[\vref{Pr 3:19}] sagesse, il a d. les cieux par
\item[\vref{Jé 30:21}] moi ; car qui d. son cœur pour
\item[\vref{Mt 6:23}] œil est mal d., tt ton corps
\item[\vref{Lu 1:17}] préparer au Seign. un peuple bien d.
\item[\vref{Lu 9:62}] n'est pas bien d. pour le Royaume
\item[\vref{1 Co 12:24}] Mais Dieu a d. le corps de
\item[\vref{2 Co 8:12}] Un esprit bien d., qnd il existe,
\item[\vref{Hé 9:6}] choses étant ainsi d., les prêtres qui
\end{listverse}

\ConcordanceEntry{Dispute}
\vspace{-2mm}
\begin{listverse}
\item[\vref{Ge 13:8}] ait point de d. entre moi et
\item[\vref{Jg 14:4}] une occasion de d. de la part
\item[\vref{Job 39:35}] Que celui qui d. avec Dieu réponde
\item[\vref{Ps 80:7}] un sujet de d. entre nos voisins,
\item[\vref{Pr 15:18}] lent à la colère apaise la d.
\item[\vref{Pr 17:14}] vienne à la d., retire-toi.
\item[\vref{Pr 20:3}] de s'abstenir des d., mais chaque insensé
\item[\vref{Pr 23:29}] Pour qui les d. ? Pour qui les
\item[\vref{Jé 15:10}] un hom. de d. pour tt le
\item[\vref{Ro 2:8}] un esprit de d., et qui se
\item[\vref{1 Co 3:3}] la jalousie, des d., et des divisions,
\item[\vref{Ga 5:20}] les animosités, les d., les divisions, les
\item[\vref{Ph 1:15}] un esprit de d. ; et que les
\item[\vref{Ph 2:14}] ttes choses sans murmures et sans d.,
\item[\vref{1 Ti 2:8}] mains pures, sans colère, et sans d.
\item[\vref{Tit 3:9}] querelles et les d. de la loi ;
\item[\vref{Ja 4:1}] parmi vs. les d. et les querelles ?
\end{listverse}

\ConcordanceEntry{Dissimulation}
\vspace{-2mm}
\begin{listverse}
\item[\vref{Ga 2:13}] aussi usèrent de d. com. lui, de
\item[\vref{1 Pi 2:1}] tte fraude, de d., d'envie et de
\end{listverse}

\ConcordanceEntry{Dissiper}
\vspace{-2mm}
\begin{listverse}
\item[\vref{2 S 17:14}] avait résolu de d. le conseil d'Achitophel,
\item[\vref{Job 7:9}] La nuée se d. et s'en va,
\item[\vref{Pr 20:8}] trône de justice d. tt mal par
\item[\vref{Pr 29:3}] les femmes prostituées d. ses richesses.
\item[\vref{Es 8:10}] et il sera d. ; dites la parole,
\item[\vref{Es 44:25}] qui d. les signes des menteurs, qui rends
\item[\vref{Lu 15:13}] éloigné, où il d. son bien en
\end{listverse}

\ConcordanceEntry{Distinction}
\vspace{-2mm}
\begin{listverse}
\item[\vref{Est 6:3}] honneur et quelle d. a-t-on accordé à
\item[\vref{Pr 19:10}] de dominer sur les personnes de d. !
\item[\vref{Ac 17:12}] femmes grecques de d., et des hommes
\item[\vref{Ro 14:5}] Tel fait une d. entre les jours,
\end{listverse}

\ConcordanceEntry{Distinguer}
\vspace{-2mm}
\begin{listverse}
\item[\vref{Ex 8:18}] Mais je d. ce jour-là le pays de Gosen,
\item[\vref{Ps 89:18}] notre pouvoir est d. par ta faveur.
\item[\vref{Pr 17:7}] La parole d. ne convient pas
\item[\vref{Jon 4:11}] ne savent point d. lr. main droite
\item[\vref{Ac 13:17}] pères. Il a d. glorieusement ce peuple
\end{listverse}

\ConcordanceEntry{Distribuer}
\vspace{-2mm}
\begin{listverse}
\item[\vref{2 Ch 31:15}] des prêtres, pour d. fidèlement les portions
\item[\vref{Ps 68:19}] dons pour les d. parmi les hommes,
\item[\vref{Lu 18:22}] tu as, et d.-le aux pauvres,
\item[\vref{1 Co 12:11}] ttes ces choses, d. à chacun ses
\end{listverse}

\ConcordanceEntry{Divination}
\vspace{-2mm}
\begin{listverse}
\item[\vref{Lé 20:27}] livrent à la d., ils mourront, ils
\item[\vref{No 23:23}] Jacob, ni la d. contre Israël. Au
\item[\vref{De 18:10}] qui pratique la d., l'astrologie, l'augure, la
\item[\vref{1 S 15:23}] autant que la d., et la résistance
\item[\vref{2 R 17:17}] s'adonnèrent à la d. et aux enchantements,
\item[\vref{Jé 14:14}] de mensonge, des d. de néant et
\item[\vref{Ez 13:9}] vanité et des d. de mensonge ; ils
\item[\vref{Ez 21:27}] droite est la d. contre Jérus., pour
\end{listverse}

\ConcordanceEntry{Divinité}
\vspace{-2mm}
\begin{listverse}
\item[\vref{Ac 17:29}] croire que la d. soit semblable à
\item[\vref{Ro 1:20}] éternelle et sa d., se voient com.
\item[\vref{Col 2:9}] corporellement tte la plénitude de la d.
\end{listverse}

\ConcordanceEntry{Diviser}
\vspace{-2mm}
\begin{listverse}
\item[\vref{Ge 10:5}] De ceux-là furent d. les îles des
\item[\vref{Ge 49:7}] cruelle ; je les d. ds Jacob, et
\item[\vref{2 R 2:8}] eaux, qui se d. çà et là,
\item[\vref{Pr 17:9}] rapporte la chose d. les plus grands
\item[\vref{Da 2:41}] ce royaume sera d., mais il y
\item[\vref{Mt 12:25}] dit : Tout royaume d. contre lui-mm sera
\item[\vref{Lu 12:52}] une maison seront d., trois contre deux,
\item[\vref{1 Co 1:13}] Christ est-il d. ? Paul a-t-il été
\end{listverse}

\ConcordanceEntry{Division}
\vspace{-2mm}
\begin{listverse}
\item[\vref{Mt 10:35}] venu mettre en d. le fils contre
\item[\vref{Lu 12:51}] Non, vs. dis-je ; mais plutôt la d.
\item[\vref{Jn 7:43}] dc de la d. parmi la foule
\item[\vref{Ac 24:5}] qui sème des d. parmi ts les
\item[\vref{1 Co 1:10}] ait pas de d. entre vs., mais
\item[\vref{1 Co 11:18}] y a des d. parmi vs. et
\item[\vref{Ga 5:20}] les disputes, les d., les sectes,
\item[\vref{Hé 4:12}] atteignant jusqu'à la d. de l'âme et
\end{listverse}

\ConcordanceEntry{Divorce}
\vspace{-2mm}
\begin{listverse}
\item[\vref{De 24:1}] une lettre de d., et après la
\item[\vref{Es 50:1}] la lettre de d. par laquelle j'ai
\item[\vref{Mt 19:7}] la lettre de d., et de répudier
\end{listverse}

\ConcordanceEntry{Docteur}
\vspace{-2mm}
\begin{listverse}
\item[\vref{Mt 23:8}] seul est votre D. ; et vs. êtes
\item[\vref{Lu 2:46}] au milieu des d., les écoutant et
\item[\vref{Lu 10:25}] Alors voici, un d. de la loi
\item[\vref{Jn 3:2}] tu es un d. venu de Dieu,
\item[\vref{Jn 3:10}] dit : Tu es d. d'Israël, et tu
\item[\vref{Ro 2:20}] le d. des insensés, le maître des ignorants,
\item[\vref{1 Co 12:28}] prophètes, troisièmement des d., ensuite ceux qui
\item[\vref{Ep 4:11}] les autres pour être pasteurs et d.,
\item[\vref{1 Ti 1:7}] voulant être d. de la loi ;
\item[\vref{2 Ti 4:3}] ils chercheront des d. qui répondent à
\item[\vref{Ja 3:1}] à devenir des d., sachant que ns.
\item[\vref{2 Pi 2:1}] vs. de faux d., qui introduiront secrètement
\end{listverse}

\ConcordanceEntry{Doctrine}
\vspace{-2mm}
\begin{listverse}
\item[\vref{Pr 4:2}] donne une bonne d., ne rejetez dc
\item[\vref{Mt 7:28}] la foule fut frappée de sa d. ;
\item[\vref{Jn 7:16}] et dit : Ma d. n'est pas de
\item[\vref{Jn 7:17}] connaîtra si ma d. est de Dieu,
\item[\vref{Ac 2:42}] ts ds la d. des apôtres, ds
\item[\vref{Ac 13:12}] d'admiration pour la d. du Seign.
\item[\vref{Ac 22:4}] à mort cette d., liant et mettant
\item[\vref{Ep 4:14}] ts vents de d., par la tromperie
\item[\vref{1 Ti 1:3}] de ne pas enseigner une autre d.,
\item[\vref{1 Ti 1:10}] qui est contraire à la saine d.,
\item[\vref{1 Ti 4:1}] et à des d. de démons,
\item[\vref{1 Ti 4:6}] de la bonne d. que tu as
\item[\vref{2 Ti 4:3}] pas la saine d., mais aimant qu'on
\item[\vref{Tit 2:1}] choses qui conviennent à la saine d.
\item[\vref{Tit 2:10}] ttes choses la d. de Dieu, notre
\item[\vref{Hé 13:9}] là par des d. diverses et étrangères ;
\item[\vref{2 Jn 1:9}] Quiconque transgresse la d. de Jésus-Christ et
\item[\vref{Ap 2:14}] attachés à la d. de Balaam, qui
\item[\vref{Ap 2:24}] n'ont pas cette d., et qui n'ont
\end{listverse}

\ConcordanceEntry{Doigt}
\vspace{-2mm}
\begin{listverse}
\item[\vref{Ex 8:15}] C'est ici le d. de Dieu ! Toutefois,
\item[\vref{Ex 31:18}] témoignage ; tables de pierre, écrites du d. de Dieu.
\item[\vref{Lé 4:6}] prêtre trempera son d. ds le sang,
\item[\vref{1 Ch 20:6}] qui avait six d. à chaque main,
\item[\vref{Ps 8:4}] l'ouvrage de tes d., la lune et
\item[\vref{Pr 7:3}] Lie-les à tes d., écris-les sur la
\item[\vref{Es 58:9}] de lever le d. et de dire
\item[\vref{Da 5:5}] la muraille des d. d'une main d'hom.,
\item[\vref{Lu 11:20}] démons par le d. de Dieu, alors
\item[\vref{Lu 15:22}] un anneau au d., et des souliers
\item[\vref{Lu 16:24}] bout de son d. ds l'eau et
\item[\vref{Jn 8:6}] écrivait avec son d. sur la terre.
\item[\vref{Jn 20:27}] Thomas : Mets ton d. ici, et regarde
\end{listverse}

\ConcordanceEntry{Domicile}
\vspace{-2mm}
\begin{listverse}
\item[\vref{2 Ch 6:2}] demeure, et un d. afin que tu
\item[\vref{Pr 24:15}] n'épie pas le d. du juste, et
\item[\vref{2 Co 5:2}] avec ardeur d'être revêtus de notre d. céleste,
\end{listverse}

\ConcordanceEntry{Domination}
\vspace{-2mm}
\begin{listverse}
\item[\vref{1 Ch 29:12}] tu as la d. sur ttes choses ;
\item[\vref{Ps 145:13}] siècles, et ta d. subsiste ds ts
\item[\vref{Es 41:2}] a donné la d. sur les rois ?
\item[\vref{Da 4:3}] éternel, et sa d. subsiste de génération
\item[\vref{Da 4:34}] celui dont la d. est une domination
\item[\vref{Ep 1:21}] et de tte d., et au-dessus de
\item[\vref{Col 1:16}] trônes, ou les d., ou les principautés,
\item[\vref{1 Pi 3:22}] les anges, les d. et les puissances.
\end{listverse}

\ConcordanceEntry{Dominer}
\vspace{-2mm}
\begin{listverse}
\item[\vref{Ge 1:26}] ressemblance, et qu'il d. sur les poissons
\item[\vref{Ge 3:16}] rapporteront vers ton mari, et il d. sur toi.
\item[\vref{Ge 37:8}] sur ns. ? Et d.-tu sur ns. ?
\item[\vref{De 15:6}] sur gage ; tu d. sur beaucoup de
\item[\vref{Jg 5:13}] Yahweh a fait d. un reste du
\item[\vref{Jg 8:23}] répondit : Je ne d. pas sur vs.
\item[\vref{Né 5:15}] leurs serviteurs eurent d. sur le peuple ;
\item[\vref{Est 9:1}] des Juifs espéraient d., ce fut le
\item[\vref{Ps 66:7}] Il d. par sa puissance éternellement, ses yeux
\item[\vref{Ps 72:8}] Il d. depuis une mer jusqu'à l'autre, et
\item[\vref{Ps 89:10}] Tu d. l'élévation des flots de la mer ;
\item[\vref{Ps 136:8}] Le soleil pour d. sur le jour,
\item[\vref{Ps 136:9}] les étoiles pour d. la nuit, car
\item[\vref{Ec 8:9}] temps où l'hom. d. sur l'autre pour
\item[\vref{La 5:8}] Les esclaves d. sur ns., et
\item[\vref{Za 6:13}] Il s'assiéra et d. sur son trône,
\item[\vref{Mt 20:25}] des nations les d., et que les
\item[\vref{Ro 6:14}] le péché ne d. pas sur vs.,
\item[\vref{Ro 14:9}] vie, afin qu'il d. tant sur les
\end{listverse}

\ConcordanceEntry{Dompter}
\vspace{-2mm}
\begin{listverse}
\item[\vref{Jg 16:19}] commença à le d.. Sa force partit.
\item[\vref{Pr 25:28}] qui ne sait d. son esprit est
\item[\vref{Jé 31:18}] qui n'est pas d.. Fais-moi revenir et
\item[\vref{Mc 5:4}] fers, et personne ne pouvait le d.
\item[\vref{Ja 3:8}] hom. ne peut d. la langue ; c'est
\end{listverse}

\ConcordanceEntry{Don}
\vspace{-2mm}
\begin{listverse}
\item[\vref{Ge 30:20}] donné un beau d. ; mntnt mon mari
\item[\vref{No 18:11}] t'appartiendra : Tous les d. que les enfants
\item[\vref{Ps 68:19}] as pris des d. pour les distribuer
\item[\vref{Pr 19:14}] prudente est un d. de Yahweh.
\item[\vref{Pr 21:14}] Le d. fait en secret apaise la colère,
\item[\vref{Ec 3:13}] de tt son travail, c'est un d. de Dieu.
\item[\vref{Da 5:17}] roi : Que tes d. restent à toi,
\item[\vref{Jn 4:10}] tu connaissais le d. de Dieu et
\item[\vref{Ac 2:38}] vs. recevrez le d. du Saint-Esprit.
\item[\vref{Ac 8:20}] estimé que le d. de Dieu s'acquérait
\item[\vref{Ac 11:17}] accordé le mm d. qu'à ns. qui
\item[\vref{Ro 5:16}] est pas du d. com. de ce
\item[\vref{Ro 6:23}] mort ; mais le d. gratuit de Dieu,
\item[\vref{Ro 11:29}] Car les d. et la vocation de Dieu sont
\item[\vref{1 Co 1:7}] vs. manque aucun d., pendant que vs.
\item[\vref{1 Co 12:1}] qui concerne les d. spirituels, je ne
\item[\vref{1 Co 12:4}] a diversité de d., mais il n'y
\item[\vref{1 Co 12:31}] avec ardeur des d. plus excellents, et
\item[\vref{1 Co 16:2}] pas mon arrivée pour recueillir les d.
\item[\vref{2 Co 9:15}] Dieu pour son d. inexprimable !
\item[\vref{Ga 3:18}] à Abraham ce d. de sa grâce.
\item[\vref{Ep 2:8}] vs., c'est le d. de Dieu ;
\item[\vref{Ep 4:7}] la mesure du d. de Christ.
\item[\vref{Ep 4:8}] a donné des d. aux hommes.
\item[\vref{1 Ti 4:14}] néglige pas le d. qui est en
\item[\vref{2 Ti 1:6}] à ranimer le d. de Dieu que
\item[\vref{Hé 6:4}] ont goûté le d. céleste, et qui
\item[\vref{Ja 1:17}] d'excellent et tt d. parfait viennent d'en
\item[\vref{1 Pi 4:10}] autres selon le d. qu'il a reçu,
\end{listverse}

\ConcordanceEntry{Donner}
\vspace{-2mm}
\begin{listverse}
\item[\vref{Ge 50:24}] a juré de d. à Abraham, Isaac
\item[\vref{No 8:19}] Et j'ai entièrement d., d'entre les enfants
\item[\vref{De 15:10}] Tu lui d., lui donneras et que ton cœur
\item[\vref{Jos 1:3}] je vs. l'ai d., com. je l'ai
\item[\vref{2 Ch 25:9}] talents que j'ai d. à la troupe
\item[\vref{Esd 2:69}] Ils d. au trésor de l'ouvrage, selon leurs
\item[\vref{Job 1:21}] retournerai ; Yahweh a d., Yahweh a enlevé ;
\item[\vref{Ez 36:26}] Je vs. d. un nouveau cœur, je mettrai au-dedans
\item[\vref{Mal 2:2}] à cœur de d. gloire à mon
\item[\vref{Mt 5:42}] D. à celui qui te demande, et
\item[\vref{Mt 6:33}] choses vs. seront d. par-dessus.
\item[\vref{Mt 7:11}] vs. l'êtes, savez d. à vos enfants
\item[\vref{Mt 13:12}] Car on d. à celui qui a, et il
\item[\vref{Mt 20:28}] et afin de d. sa vie en
\item[\vref{Jn 6:37}] mon Père me d. viendront à moi,
\item[\vref{Ac 3:6}] je te le d. : Au Nom de
\item[\vref{Ac 20:35}] de bénédiction à d. qu'à recevoir.
\item[\vref{Ro 12:8}] que celui qui d. le fasse ds
\item[\vref{2 Co 8:5}] ils se sont d. premièrement eux-mêmes au
\item[\vref{Ga 1:4}] qui s'est d. lui-mm pour nos
\item[\vref{1 Ti 6:17}] vivant, qui ns. d. ttes choses abondamment
\item[\vref{Ja 1:5}] Dieu, qui la d. à ts libéralement,
\item[\vref{1 Jn 3:16}] ce qu'il a d. sa vie pour
\item[\vref{Ap 16:9}] ne se repentirent pas pour lui d. gloire.
\end{listverse}

\ConcordanceEntry{Dormir}
\vspace{-2mm}
\begin{listverse}
\item[\vref{Ps 13:4}] que je ne d. du sommeil de
\item[\vref{Ps 77:5}] mes yeux de d. ; je suis troublé,
\item[\vref{Pr 10:5}] mais celui qui d. durant la moisson
\item[\vref{Ec 5:11}] du riche ne le laisse point d.
\item[\vref{Da 12:2}] de ceux qui d. ds la poussière
\item[\vref{Mt 8:24}] était couverte de flots. Et Jésus d.
\item[\vref{Mt 13:25}] que les hommes d., son ennemi vint
\item[\vref{Mt 26:45}] et lr. dit : D. mntnt, et reposez-vs. !
\item[\vref{Mc 4:27}] qu'il d. ou qu'il veille, nuit et jour,
\item[\vref{Jn 11:11}] Notre ami Lazare d., mais je vais
\item[\vref{1 Co 15:20}] est les prémices de ceux qui d.
\item[\vref{1 Co 15:51}] mystère : Nous ne d. pas ts, mais
\item[\vref{1 Th 4:14}] aussi ceux qui d. en Jésus, Dieu
\item[\vref{1 Th 5:6}] Ne d. dc pas com. les autres, mais
\end{listverse}

\ConcordanceEntry{Dos}
\vspace{-2mm}
\begin{listverse}
\item[\vref{Ex 23:27}] ennemis tourneront le d. dvt toi.
\item[\vref{Pr 10:13}] est pour le d. de celui qui
\item[\vref{Es 50:6}] J'ai exposé mon d. à ceux qui
\item[\vref{Jé 32:33}] m'ont tourné le d., et non la
\item[\vref{Ro 11:10}] pas voir, et courbe continuellement lr. d. !
\end{listverse}

\ConcordanceEntry{Double}
\vspace{-2mm}
\begin{listverse}
\item[\vref{Ex 16:5}] y ait le d. de ce qu'ils
\item[\vref{Ex 16:22}] du pain en d., deux omers pour
\item[\vref{Ex 22:4}] brebis ou chèvre, il rendra le d.
\item[\vref{De 21:17}] lui donnera la d. portion de tt
\item[\vref{1 S 1:5}] Anne une portion d. ; car il aimait
\item[\vref{2 R 2:9}] que j'aie, une d. portion de ton
\item[\vref{Job 42:10}] lui ajouta le d. de tt ce
\item[\vref{Ps 12:3}] et ils parlent avec un cœur d.
\item[\vref{Es 40:2}] de Yahweh le d. pour ts ses
\item[\vref{Es 61:7}] en auront le d., et elles crieront
\item[\vref{Jé 16:18}] rendrai d'abord le d. de lr. iniquité
\item[\vref{1 Ti 5:17}] jugés dignes d'un d. honneur, spécialement ceux
\item[\vref{Ja 1:8}] L'hom. d. de cœur est inconstant ds ttes
\item[\vref{Ap 18:6}] et payez-lui au d. selon ses œuvres ;
\end{listverse}

\ConcordanceEntry{Douceur}
\vspace{-2mm}
\begin{listverse}
\item[\vref{Jg 9:11}] Renoncerais-je à ma d. et à mon
\item[\vref{Ps 45:5}] de vérité, de d., et de justice,
\item[\vref{Ec 10:4}] condition ; car la d. fait pardonner de
\item[\vref{Es 5:20}] qui font l'amertume d., et la douceur
\item[\vref{Mt 21:5}] toi, plein de d., et monté sur
\item[\vref{Ga 5:22}] la foi, la d., la tempérance.
\item[\vref{Ga 6:1}] un esprit de d.. Prends garde à
\item[\vref{Ph 4:5}] Que votre d. soit connue de
\item[\vref{2 Ti 2:25}] enseignant avec d. ceux qui ont
\item[\vref{Ja 1:21}] méchanceté, recevez avec d. la parole qui
\item[\vref{1 Pi 3:15}] à répondre avec d. et avec respect,
\end{listverse}

\ConcordanceEntry{Douleur}
\vspace{-2mm}
\begin{listverse}
\item[\vref{Ge 3:16}] enfanteras ds la d. tes enfants ; tes
\item[\vref{Ex 3:7}] leurs oppresseurs, car je connais leurs d.
\item[\vref{1 S 1:16}] l'excès de ma d. et de mon
\item[\vref{Job 6:10}] consolation, quoique la d. me consume, et
\item[\vref{Ps 31:11}] consume ds la d., et mes années
\item[\vref{Ps 32:10}] Beaucoup de d. atteindront le méchant,
\item[\vref{Ps 34:19}] déchiré par la d., et il délivre
\item[\vref{Ps 39:3}] malheureux ; et ma d. n'était pas moins
\item[\vref{Ps 127:2}] le pain de d. ; certes c'est Dieu
\item[\vref{Es 35:10}] et l'allégresse ; la d. et le gémissement
\item[\vref{Es 53:3}] hommes, hom. de d., et sachant ce
\item[\vref{Es 53:4}] chargé de nos d. ; et ns. l'avons
\item[\vref{Jé 8:18}] pour soutenir la d., mais mon cœur
\item[\vref{La 1:12}] s'il est une d. com. ma douleur,
\item[\vref{Mt 24:8}] ne seront que le commencement des d.
\item[\vref{Ap 12:2}] souffrant les grandes d. de l'enfantement.
\item[\vref{Ap 16:10}] cause de la d. qu'ils ressentaient.
\item[\vref{Ap 21:4}] ni cri, ni d., car les premières
\end{listverse}

\ConcordanceEntry{Doute}
\vspace{-2mm}
\begin{listverse}
\item[\vref{Mc 11:23}] et s'il ne d. pas ds son
\item[\vref{Ro 14:23}] qui a des d. au sujet de
\item[\vref{Ja 1:6}] car celui qui d. est semblable au
\end{listverse}

\ConcordanceEntry{Douter}
\vspace{-2mm}
\begin{listverse}
\item[\vref{Mt 14:31}] de peu de foi, pourquoi as-tu d. ?
\item[\vref{Mt 21:21}] que vs. ne d. pas, non seulement
\item[\vref{Mt 28:17}] le virent, ils l'adorèrent, mais quelques-uns d.
\item[\vref{Mc 11:23}] et s'il ne d. pas ds son
\item[\vref{Ro 4:20}] Et il ne d. pas à l'égard de la promesse
\end{listverse}

\ConcordanceEntry{Doux}
\vspace{-2mm}
\begin{listverse}
\item[\vref{No 12:3}] un hom. fort d., plus que ts
\item[\vref{Jg 14:14}] est sorti le d.. Pendant trois jours,
\item[\vref{1 R 19:12}] vint un murmure d. et léger.
\item[\vref{Ps 147:1}] Car il est d. et bienséant de
\item[\vref{Pr 11:17}] L'hom. d. fait du bien à son âme,
\item[\vref{Ez 3:3}] et il fut d. ds ma bouche
\item[\vref{Mt 11:29}] car je suis d. et humble de
\item[\vref{Ac 2:13}] C'est qu'ils sont pleins de vin d.
\item[\vref{Ep 4:32}] Mais soyez d. les uns envers
\item[\vref{1 Pi 3:4}] l'incorruptibilité d'un esprit d. et paisible, qui
\item[\vref{Ap 10:10}] mangeai ; il fut d. ds ma bouche
\end{listverse}

\ConcordanceEntry{Drachme}
\vspace{-2mm}
\begin{listverse}
\item[\vref{Mt 17:24}] percevaient les deux d. s'adressèrent à Pierre,
\item[\vref{Lu 15:9}] j'ai trouvé la d. que j'avais perdue.
\end{listverse}

\ConcordanceEntry{Dragon}
\vspace{-2mm}
\begin{listverse}
\item[\vref{Ps 91:13}] tu piétineras le lionceau et le d.
\item[\vref{Es 51:9}] l'Egypte, et qui blessa mortellement le d. ?
\item[\vref{Ap 12:3}] voici un grand d. rouge feu ayant
\item[\vref{Ap 12:7}] combattirent contre le d.. Et le dragon
\item[\vref{Ap 12:9}] précipité le grand d., le serpent ancien,
\item[\vref{Ap 13:2}] lion. Et le d. lui donna sa
\item[\vref{Ap 13:4}] ils adorèrent le d., parce qu'il avait
\item[\vref{Ap 20:2}] Il saisit le d., le serpent ancien,
\end{listverse}

\ConcordanceEntry{Droit}
\vspace{-2mm}
\begin{listverse}
\item[\vref{Ex 15:26}] ce qui est d. dvt lui, si
\item[\vref{De 6:18}] ce qui est d. et bon aux
\item[\vref{Jg 17:6}] lui semblait être d. à ses yeux.
\item[\vref{Job 1:1}] était intègre et d., craignant Dieu et
\item[\vref{Ps 111:1}] compagnie des hommes d. et ds l'assemblée.
\item[\vref{Pr 2:21}] ceux qui sont d. habiteront la terre,
\item[\vref{Pr 8:6}] mes lèvres pour enseigner des choses d.
\item[\vref{Ha 2:4}] s'élève n'est pas d. en lui ; mais
\item[\vref{Ac 8:21}] cœur n'est pas d. dvt Dieu.
\item[\vref{2 Pi 2:15}] ayant laissé le d. chemin, se sont
\end{listverse}

\ConcordanceEntry{Droit (le)}
\vspace{-2mm}
\begin{listverse}
\item[\vref{Ge 25:31}] Vends-moi aujourd'hui ton d. d'aînesse.
\item[\vref{Ge 25:34}] Esaü méprisa son d. d'aînesse.
\item[\vref{Ge 43:33}] premier-né selon son d. d'aînesse, et le
\item[\vref{Ex 23:6}] pervertiras point le d. de l'indigent, qui
\item[\vref{De 21:17}] sa vigueur, le d. d'aînesse lui appartient.
\item[\vref{2 S 8:15}] et il faisait d. et justice à
\item[\vref{1 Ch 5:1}] son père, son d. d'aînesse fut donné
\item[\vref{Esd 4:13}] d'impôt, ni de d. de passage, et
\item[\vref{Job 8:3}] Dieu renverserait-il le d., et le Tout-Puissant
\item[\vref{Ps 37:6}] lumière, et ton d. com. le soleil
\item[\vref{Ps 140:12}] au malheureux et d. aux indigents.
\item[\vref{Ec 5:7}] et que le d. et la justice
\item[\vref{Es 10:2}] et ravir lr. d. aux malheureux de
\item[\vref{Da 7:22}] jours vint donner d. aux saints du
\item[\vref{1 Co 9:4}] N'avons-ns. pas le d. de manger et
\item[\vref{Hé 12:16}] aliment vendit son d. d'aînesse.
\item[\vref{Ap 22:14}] robes afin d'avoir d. à l'arbre de
\end{listverse}

\ConcordanceEntry{Droite (la)}
\vspace{-2mm}
\begin{listverse}
\item[\vref{Ge 13:9}] gauche, j'irai à d. ; et si tu
\item[\vref{Ge 48:14}] étendit sa main d. et la posa
\item[\vref{Ex 15:6}] Ta d., ô Yahweh, s'est montrée magnifique en
\item[\vref{De 5:32}] détournerez ni à d. ni à gauche.
\item[\vref{Ps 18:36}] ton salut, ta d. me soutient, et
\item[\vref{Ps 33:4}] de Yahweh est d., et ttes ses
\item[\vref{Ps 44:4}] mais c'est ta d., c'est ton bras,
\item[\vref{Ps 91:7}] mille à ta d., tu ne seras
\item[\vref{Ps 98:1}] choses merveilleuses. Sa d. et le bras
\item[\vref{Ps 109:31}] tient à la d. du misérable pour
\item[\vref{Ps 110:1}] Assieds-toi à ma d., jusqu'à ce que
\item[\vref{Ps 118:16}] La d. de Yahweh est élevée ! La droite
\item[\vref{Ps 139:10}] conduira, et ta d. me saisira.
\item[\vref{Es 41:13}] soutiens ta main d., et te dis :
\item[\vref{Za 3:1}] debout à sa d., pour l'accuser.
\item[\vref{Mt 5:30}] Si ta main d. est pour toi
\item[\vref{Mt 20:21}] l'un à ta d., et l'autre à
\item[\vref{Mt 25:33}] brebis à sa d., et les boucs
\item[\vref{Mc 14:62}] assis à la d. de la puissance
\item[\vref{Mc 16:19}] ciel, et il s'assit à la d. de Dieu.
\item[\vref{Hé 1:3}] assis à la d. de la Majesté
\item[\vref{Ap 1:16}] ds sa main d. sept étoiles, et
\item[\vref{Ap 5:1}] ds la main d. de celui qui
\end{listverse}

\ConcordanceEntry{Droiture}
\vspace{-2mm}
\begin{listverse}
\item[\vref{De 9:5}] ni pour la d. de ton cœur
\item[\vref{2 S 22:21}] traité selon ma d., il m'a rendu
\item[\vref{1 R 9:4}] cœur et avec d., en faisant tt
\item[\vref{1 Ch 29:17}] plaisir à la d.. C'est pourquoi j'ai
\item[\vref{Job 33:3}] répondront à la d. de mon cœur,
\item[\vref{Job 42:7}] de moi avec d. com. Job, mon
\item[\vref{Ps 9:9}] justice, il juge les peuples avec d.
\item[\vref{Ps 18:26}] l'hom. droit tu agis selon la d.
\item[\vref{Ps 27:11}] sentier de la d., à cause de
\item[\vref{Ps 75:3}] que j'aurai fixé, je jugerai avec d.
\item[\vref{Pr 14:2}] marche ds la d. craint Yahweh, mais
\item[\vref{Es 5:7}] attendait de la d., et voici du
\end{listverse}

\ConcordanceEntry{Duplicité}
\vspace{-2mm}
\begin{listverse}
\item[\vref{1 Ti 3:8}] éloignés de la d., des excès du
\end{listverse}

\ConcordanceEntry{Dur}
\vspace{-2mm}
\begin{listverse}
\item[\vref{Ex 6:9}] cause de lr. d. servitude.
\item[\vref{Job 39:19}] Elle est d. envers ses petits, com. s'ils n'étaient
\item[\vref{Ps 114:8}] la pierre très d. en une source
\item[\vref{Es 14:3}] et de la d. servitude qui te
\item[\vref{Ez 2:4}] à la face d. et au cœur
\item[\vref{Jn 6:60}] Cette parole est d., qui peut l'écouter ?
\end{listverse}

\ConcordanceEntry{Durement}
\vspace{-2mm}
\begin{listverse}
\item[\vref{1 S 12:3}] Qui ai-je traité d. ? Et de la
\item[\vref{1 R 12:13}] le roi répondit d. au peuple, laissant
\item[\vref{Es 47:6}] eux, tu as d. appesanti ton joug
\end{listverse}

\ConcordanceEntry{Dureté}
\vspace{-2mm}
\begin{listverse}
\item[\vref{Lé 25:43}] sur lui avec d., et tu craindras
\item[\vref{Ez 34:4}] avez maîtrisées avec d. et rigueur.
\item[\vref{Mt 19:8}] cause de la d. de votre cœur
\item[\vref{Mc 16:14}] incrédulité et lr. d. de cœur, parce
\item[\vref{Ro 2:5}] Mais, par ta d., et par ton
\end{listverse}

\ConcordanceEntry{Eau}
\vspace{-2mm}
\begin{listverse}
\item[\vref{Ge 1:2}] de Dieu se mouvait au-dessus des e.
\item[\vref{Ge 7:24}] Et les e. furent grosses sur la terre pendant
\item[\vref{Ge 21:15}] Quand l'e. de l'outre fut épuisée, elle jeta
\item[\vref{Ge 24:32}] il apporta de l'e. pour laver les
\item[\vref{Ge 26:19}] vallée et y trouvèrent un puits d'e. vive.
\item[\vref{Ex 7:17}] ma main les e. du fleuve, et
\item[\vref{Ex 14:22}] sec, et les e. lr. servaient de
\item[\vref{Ex 17:1}] n'y avait point d'e. à boire pour
\item[\vref{Ex 34:28}] et sans boire d'e. ; et Yahweh écrivit
\item[\vref{No 5:24}] la fem. les e. amères qui apportent
\item[\vref{No 20:2}] n'y avait point d'e. pour l'assemblée ; et
\item[\vref{No 20:8}] il donnera son e. ; ainsi tu lr.
\item[\vref{Jos 3:16}] Les e. qui descendent d'en haut, s'arrêtèrent, et
\item[\vref{Jg 7:6}] Ceux qui lapèrent l'e. en la portant
\item[\vref{Jg 15:19}] en sortit de l'e.. Samson but, l'Esprit
\item[\vref{2 S 21:10}] jusqu'à ce que l'e. du ciel tombât
\item[\vref{2 S 22:17}] saisit, il me retira des grandes e. ;
\item[\vref{2 S 23:15}] fera boire de l'e. de la citerne
\item[\vref{1 R 13:8}] ni ne boirai d'e. en ce lieu.
\item[\vref{1 R 18:38}] il absorba tte l'e. qui était ds
\item[\vref{2 R 2:8}] en frappa les e., qui se divisèrent
\item[\vref{2 R 2:22}] Les e. furent assainies, jusqu'à ce jour, selon
\item[\vref{2 R 3:17}] vallée sera remplie d'e., et vs. boirez,
\item[\vref{Esd 10:6}] ne but point d'e., parce qu'il se
\item[\vref{Né 9:15}] fis sortir de l'e. du rocher qnd
\item[\vref{Job 9:30}] laverais ds de l'e. de neige, et
\item[\vref{Job 36:27}] qu'il met les e. en petites gouttes,
\item[\vref{Ps 1:3}] près des ruisseaux d'e., qui rend son
\item[\vref{Ps 23:2}] dirige près des e. paisibles.
\item[\vref{Ps 32:6}] déluges de grandes e., elles ne l'atteindront
\item[\vref{Ps 42:2}] après des courants d'e., ainsi mon âme
\item[\vref{Ps 77:18}] versé un déluge d'e., les nuées ont
\item[\vref{Ps 93:4}] bruit des grandes e., et que les
\item[\vref{Ps 124:4}] Alors les e. ns. auraient submergés,
\item[\vref{Pr 21:1}] com. des ruisseaux d'e. ; il l'incline à
\item[\vref{Pr 25:25}] Comme de l'e. fraîche pour une
\item[\vref{Ca 8:7}] à Salomon) :] Beaucoup d'e. ne pourraient éteindre
\item[\vref{Es 35:6}] triomphe. Car des e. jailliront ds le
\item[\vref{Es 41:17}] qui cherchent des e., et n'en ont
\item[\vref{Es 43:2}] passes par les e., je serai avec
\item[\vref{Jé 2:18}] Egypte, pour boire l'e. du Schichor ? Qu'as-tu
\item[\vref{La 5:4}] Nous buvons notre e. à prix d'argent,
\item[\vref{Ez 36:25}] sur vs. une e. pure, et vs.
\item[\vref{Ez 47:1}] et voici, des e. sortaient sous le
\item[\vref{Ez 47:3}] fit traverser ces e.-là, et j'avais
\item[\vref{Am 8:11}] ni une soif d'e., mais d'entendre les
\item[\vref{Mt 3:11}] je vs. baptise d'e. en signe de
\item[\vref{Mt 10:42}] seulement un verre d'e. froide à l'un
\item[\vref{Mt 14:28}] que j'aille vers toi sur les e.
\item[\vref{Lu 5:4}] Avance en pleine e., et jetez vos
\item[\vref{Lu 16:24}] son doigt ds l'e. et me rafraîchisse
\item[\vref{Jn 2:9}] d'hôtel eut goûté l'e. qui avait été
\item[\vref{Jn 3:5}] quelqu'un ne naît d'e. et d'Esprit, il
\item[\vref{Jn 4:10}] boire, et il t'aurait donné de l'e. vive.
\item[\vref{Jn 5:3}] des paralytiques, attendant le mouvement de l'e.
\item[\vref{Jn 7:38}] moi, des fleuves d'e. vive couleront de
\item[\vref{Jn 13:5}] il mit de l'e. ds un bassin,
\item[\vref{Jn 19:34}] il sortit du sang et de l'e.
\item[\vref{Ac 8:36}] y avait de l'e.. Et l'eunuque dit :
\item[\vref{1 Co 10:4}] ils buvaient de l'e. du rocher spirituel
\item[\vref{Ep 5:26}] la lavant par l'e. de la parole ;
\item[\vref{1 Pi 3:20}] savoir huit personnes, furent sauvées par l'e.
\item[\vref{2 Pi 3:5}] est sortie de l'e., et qu'elle subsiste
\item[\vref{1 Jn 5:6}] est venu avec l'e. et le sang ;
\item[\vref{1 Jn 5:8}] à savoir l'Esprit, l'e., et le sang,
\item[\vref{Ap 8:11}] le tiers des e. fut changé en
\item[\vref{Ap 17:15}] me dit : Les e. que tu as
\item[\vref{Ap 21:6}] de la source d'e. vive gratuitement.
\item[\vref{Ap 22:1}] montra un fleuve d'e. de la vie,
\item[\vref{Ap 22:17}] prenne gratuitement de l'e. de la vie.
\end{listverse}

\ConcordanceEntry{Eben-Ezer}
\vspace{-2mm}
\begin{listverse}
\item[\vref{1 S 4:1}] Ils campèrent près d'E., et les Philistins
\item[\vref{1 S 5:1}] prirent l'arche de Dieu, et l'emmenèrent d'E. à Asdod.
\item[\vref{1 S 7:12}] appela ce lieu E., en disant : Yahweh
\end{listverse}

\ConcordanceEntry{Ebranler}
\vspace{-2mm}
\begin{listverse}
\item[\vref{Job 9:6}] sa place, et ses piliers sont é.
\item[\vref{Ps 10:6}] ne serai pas é., car d'âge en
\item[\vref{Ps 16:8}] ma droite, je ne serai pas é.
\item[\vref{Ps 30:7}] je disais : Je ne serai jamais é. !
\item[\vref{Ps 46:6}] Elle n'est point é.. Dieu la secourt
\item[\vref{Ps 62:3}] retraite ; je ne serai pas entièrement é.
\item[\vref{Ps 93:1}] ferme, tellement qu'il ne sera point é.
\item[\vref{Ps 99:1}] les chérubins : Que la terre soit é. !
\item[\vref{Ps 104:5}] ses bases, elle ne sera jamais é.
\item[\vref{Pr 10:30}] ne sera jamais é., mais les méchants
\item[\vref{Es 13:13}] C'est pourquoi j'é. les cieux, et
\item[\vref{Ac 16:26}] la prison furent é. ; au mm instant,
\item[\vref{2 Th 2:2}] vs. laisser subitement é. ds votre entendement,
\item[\vref{Hé 12:26}] dont la voix é. alors la terre,
\end{listverse}

\ConcordanceEntry{Ecaille}
\vspace{-2mm}
\begin{listverse}
\item[\vref{1 S 17:5}] d'une cuirasse à é. pesant cinq mille
\item[\vref{Ac 9:18}] yeux com. des é. ; et à l'instant
\end{listverse}

\ConcordanceEntry{Echapper}
\vspace{-2mm}
\begin{listverse}
\item[\vref{Ge 7:7}] ses fils, pour é. aux eaux du
\item[\vref{1 R 20:42}] tu as laissé é. de tes mains
\item[\vref{Est 4:13}] pas que tu é. seule d'entre ts
\item[\vref{Ps 55:9}] Je m'é. en tte hâte, plus rapide que
\item[\vref{Ps 124:7}] Notre âme s'est é. com. l'oiseau du
\item[\vref{Jé 39:18}] je te ferai é., et tu ne
\item[\vref{Jé 44:28}] ceux qui auront é. à l'épée, retourneront
\item[\vref{Da 2:5}] La chose m'a é. ; si vs. ne
\item[\vref{Ha 2:9}] lieu élevé, pour é. à l'atteinte de
\item[\vref{Lu 21:36}] soyez trouvés dignes d'é. à ttes ces
\item[\vref{Jn 10:39}] saisir, mais il s'é. de leurs mains.
\item[\vref{1 Th 5:3}] surprennent la fem. enceinte, et ils n'é. pas.
\item[\vref{Hé 2:3}] comment é.-ns., si ns. négligeons un si
\item[\vref{Hé 12:25}] terre, n'ont pas é., ns. serons punis
\end{listverse}

\ConcordanceEntry{Echarde}
\vspace{-2mm}
\begin{listverse}
\item[\vref{2 Co 12:7}] été mis une é. ds la chair,
\end{listverse}

\ConcordanceEntry{Echelle}
\vspace{-2mm}
\begin{listverse}
\item[\vref{Ge 28:12}] et voici, une é. dressée sur la
\end{listverse}

\ConcordanceEntry{Eclair}
\vspace{-2mm}
\begin{listverse}
\item[\vref{Ex 19:16}] tonnerres, et des é., et une grosse
\item[\vref{2 S 22:15}] il lança des é., et les mit
\item[\vref{Job 28:26}] un chemin à l'é. et au tonnerre,
\item[\vref{Job 37:3}] cieux, et son é. brille jusqu'aux extrémités
\item[\vref{Ps 77:19}] le tourbillon, les é. ont éclairé le
\item[\vref{Ps 97:4}] Ses é. illuminent le monde, et la terre
\item[\vref{Ps 135:7}] il fait les é. et la pluie ;
\item[\vref{Jé 10:13}] il fait les é. et la pluie,
\item[\vref{Da 10:6}] visage brillait com. l'é., ses yeux étaient
\item[\vref{Za 10:1}] Yahweh produira des é., et il vs.
\item[\vref{Mt 24:27}] Car, com. l'é. part de l'orient
\item[\vref{Mt 28:3}] était com. un é., et son vêt.
\item[\vref{Lu 10:18}] Satan tomber du ciel com. un é.
\item[\vref{Ap 4:5}] trône sortaient des é., des tonnerres, et
\end{listverse}

\ConcordanceEntry{Eclairer}
\vspace{-2mm}
\begin{listverse}
\item[\vref{Ge 1:15}] du ciel afin d'é. la terre ; et
\item[\vref{Ex 25:37}] allumera afin qu'elles é. vis-à-vis du chandelier.
\item[\vref{2 S 22:29}] Yahweh ! Et Yahweh é. mes ténèbres.
\item[\vref{Ps 119:130}] de tes paroles é., elle donne de
\item[\vref{Es 60:19}] la lune ne t'é. plus, mais Yahweh
\item[\vref{Mt 6:22}] bon état, tt ton corps sera é.
\item[\vref{Jn 1:9}] ds le monde é. tt hom.
\item[\vref{2 Co 4:4}] ne soient pas é. par la lumière
\item[\vref{Ep 5:14}] relève-toi d'entre les morts, et Christ t'é.
\item[\vref{Hé 10:32}] après avoir été é., vs. avez soutenu
\item[\vref{Ap 21:23}] la lune pour l'é., car la gloire
\item[\vref{Ap 22:5}] Seign. Dieu les é., et ils régneront
\end{listverse}

\ConcordanceEntry{Eclat}
\vspace{-2mm}
\begin{listverse}
\item[\vref{Pr 4:18}] lumière resplendissante, dont l'é. augmente jusqu'à ce
\item[\vref{Ez 28:17}] cause de ton é. ; je te jette
\item[\vref{Ac 22:11}] à cause de l'é. de cette lumière,
\item[\vref{1 Co 15:40}] mais autre est l'é. des corps célestes,
\item[\vref{2 Th 2:8}] qu'il anéantira par l'é. de son avènement.
\item[\vref{Ap 19:6}] eaux, et com. l'é. de grands tonnerres,
\item[\vref{Ap 21:11}] de Dieu. Son é. était semblable à
\end{listverse}

\ConcordanceEntry{Econome}
\vspace{-2mm}
\begin{listverse}
\item[\vref{Lu 12:42}] Quel est dc l'é. fidèle et prudent,
\item[\vref{Lu 16:1}] qui avait un é., qui fut accusé
\item[\vref{Ro 16:23}] vs. salue. Eraste, l'é. de la ville,
\item[\vref{Tit 1:7}] irrépréhensible, com. étant é. ds la maison
\end{listverse}

\ConcordanceEntry{Ecraser}
\vspace{-2mm}
\begin{listverse}
\item[\vref{Ge 3:15}] sa postérité ; celle-ci t'é. la tête, et
\item[\vref{Ps 68:22}] Certainement, Dieu é. la tête de
\item[\vref{Ps 137:9}] et qui les é. contre le rocher !
\item[\vref{Mi 4:13}] d'airain ; et tu é. des peuples nombreux,
\item[\vref{Mt 21:44}] celui sur qui elle tombera sera é.
\end{listverse}

\ConcordanceEntry{Ecrire}
\vspace{-2mm}
\begin{listverse}
\item[\vref{Ex 17:14}] dit à Moïse : E. cela pour mémorial
\item[\vref{Ex 24:12}] commandements que j'ai é. pour les enseigner.
\item[\vref{Ex 31:18}] tables de pierre, é. du doigt de
\item[\vref{Est 9:32}] et cela fut é. ds le livre.
\item[\vref{Ps 40:8}] viens ; il est é. de moi ds
\item[\vref{Ec 12:12}] en a été é. ici, est la
\item[\vref{Es 44:5}] et un autre é. de sa main :
\item[\vref{Jé 31:33}] au-dedans d'eux, je l'é. ds lr. cœur ;
\item[\vref{Jn 8:6}] penché en bas, é. avec son doigt
\item[\vref{Jn 19:22}] Ce que j'ai é., je l'ai écrit.
\item[\vref{1 Co 4:6}] ce qui est é., de peur que
\item[\vref{1 Co 10:11}] elles ont été é. pour notre instruction,
\item[\vref{Ap 20:15}] fut pas trouvé é. ds le Livre
\item[\vref{Ap 22:19}] choses qui sont é. ds ce livre.
\end{listverse}

\ConcordanceEntry{Ecrit (un)}
\vspace{-2mm}
\begin{listverse}
\item[\vref{Jg 8:14}] lui donna par é. le nom des
\item[\vref{1 Ch 28:19}] été données par é., de la part
\item[\vref{2 Ch 2:11}] répondit ds un é. qu'il envoya à
\item[\vref{2 Ch 21:12}] lui vint un é. de la part
\item[\vref{Jn 5:47}] pas à ses é., comment croirez-vs. à
\end{listverse}

\ConcordanceEntry{Ecriture}
\vspace{-2mm}
\begin{listverse}
\item[\vref{Ex 32:16}] de Dieu, et l'é. était l'écriture de
\item[\vref{Da 5:17}] toutefois je lirai l'é. au roi, et
\item[\vref{Mt 21:42}] lu ds les E. : La pierre qu'ont
\item[\vref{Mt 26:56}] afin que les E. des prophètes soient
\item[\vref{Jn 2:22}] ils crurent à l'E. et à la
\item[\vref{Jn 5:39}] Vous sondez les E., car vs. pensez
\item[\vref{Jn 7:15}] Comment connaît-il les E., lui qui n'a
\item[\vref{Jn 7:42}] L'E. ne dit-elle pas que le Christ
\item[\vref{Jn 10:35}] adressée, et cependant l'E. ne peut être
\item[\vref{Jn 20:9}] encore que, selon l'E., Jésus devait ressusciter
\item[\vref{Ac 8:32}] Le passage de l'E. qu'il lisait était
\item[\vref{Ac 17:11}] les jours les E., pour voir si
\item[\vref{Ac 18:24}] puissant ds les E., vint à Ephèse.
\item[\vref{Ac 18:28}] démontrant par les E. que Jésus était
\item[\vref{Ro 15:4}] la consolation des E., ns. ayons espérance.
\item[\vref{Ro 16:26}] manifesté par les é. des prophètes, d'après
\item[\vref{1 Co 15:3}] mort pour nos péchés, selon les E.,
\item[\vref{Ga 3:8}] Aussi, l'E. prévoyant que Dieu justifierait les Gentils
\item[\vref{Ga 3:22}] Mais l'E. a renfermé ts les hommes sous
\item[\vref{2 Ti 3:16}] Toute l'E. est inspirée de Dieu et utile
\item[\vref{Ja 4:5}] Pensez-vs. que l'E. parle en vain ?
\item[\vref{2 Pi 3:16}] aussi les autres E., à lr. propre
\end{listverse}

\ConcordanceEntry{Eden}
\vspace{-2mm}
\begin{listverse}
\item[\vref{Ge 2:8}] un jardin en E., du côté de
\item[\vref{Ge 3:23}] chassa du jardin d'E. pour qu'il cultive
\item[\vref{Es 51:3}] désert semblable à E., et sa terre
\item[\vref{Ez 28:13}] tu étais en E., le jardin de
\item[\vref{Ez 31:9}] ts les arbres d'E., qui étaient ds
\item[\vref{Joë 2:3}] com. le jardin d'E., et derrière lui,
\end{listverse}

\ConcordanceEntry{Edification}
\vspace{-2mm}
\begin{listverse}
\item[\vref{Ro 14:19}] paix et à l'é. mutuelle.
\item[\vref{1 Co 14:5}] afin que l'église en reçoive de l'é.
\item[\vref{1 Co 14:12}] ce soit pour l'é. de l'Eglise que
\item[\vref{1 Co 14:26}] interprétation, que tt se fasse pour l'é.
\item[\vref{2 Co 10:8}] donnée pour votre é. et non votre
\item[\vref{2 Co 12:19}] mes très chers frères, pour votre é.
\item[\vref{2 Co 13:10}] m'a donnée pour l'é. et non pour
\item[\vref{Ep 4:12}] du service, pour l'é. du corps de
\item[\vref{1 Ti 1:4}] disputes plutôt que l'é. en Dieu qui
\end{listverse}

\ConcordanceEntry{Edifice}
\vspace{-2mm}
\begin{listverse}
\item[\vref{1 Co 3:9}] êtes le champ de Dieu et l'é. de Dieu.
\item[\vref{2 Co 5:1}] ns. avons un é. de Dieu qui
\item[\vref{Ep 2:21}] en qui tt l'é., bien ajusté ensemble,
\end{listverse}

\ConcordanceEntry{Edifier}
\vspace{-2mm}
\begin{listverse}
\item[\vref{1 S 2:35}] et je lui é. une maison stable,
\item[\vref{Ac 9:31}] la Samarie, étant é. et marchant ds
\item[\vref{Ac 20:32}] achever de vs. é., et pour vs.
\item[\vref{1 Co 3:10}] et un autre é. dessus. Mais que
\item[\vref{1 Co 8:1}] La connaissance enfle, mais la charité é.
\item[\vref{1 Co 14:3}] celui qui prophétise, é., exhorte et console
\item[\vref{1 Co 14:4}] une langue inconnue s'é. lui-mm, mais celui
\item[\vref{Ep 2:20}] étant é. sur le fondement des apôtres et
\item[\vref{Ep 2:22}] qui vs. êtes é. ensemble, pour être
\item[\vref{Ep 4:16}] afin qu'il soit é. ds la charité.
\item[\vref{Col 2:7}] étant enracinés et é. en lui, et
\item[\vref{1 Pi 2:5}] vivantes, vs. êtes é. pour être une
\item[\vref{Jud 1:20}] vs., mes bien-aimés, é.-vs. vs.-mêmes sur
\end{listverse}

\ConcordanceEntry{Edom}
\vspace{-2mm}
\begin{listverse}
\item[\vref{Ge 25:30}] C'est pourquoi on appela son nom E.
\item[\vref{Ge 36:1}] voici la postérité d'Esaü, qui est E.
\end{listverse}

\ConcordanceEntry{Effacer}
\vspace{-2mm}
\begin{listverse}
\item[\vref{Ex 17:14}] de Josué, car j'e., j'effacerai la mémoire
\item[\vref{Ex 32:32}] lr. péché ! Sinon, e.-moi mntnt de
\item[\vref{Ps 51:3}] ta grande miséricorde, e. mes transgressions ;
\item[\vref{Ps 69:29}] Qu'ils soient e. du livre de
\item[\vref{Es 43:25}] SUIS celui qui e. tes transgressions pour
\item[\vref{Es 44:22}] J'e. tes transgressions com. une nuée épaisse,
\item[\vref{Ac 3:19}] convertissez-vs., afin que vos péchés soient e. ;
\item[\vref{Col 2:14}] Il a e. l'acte qui était contre ns., qui
\item[\vref{Ap 3:5}] blancs, et je n'e. pas son nom
\end{listverse}

\ConcordanceEntry{Effrayer}
\vspace{-2mm}
\begin{listverse}
\item[\vref{De 1:21}] dit ; ne crains point et ne t'e. point.
\item[\vref{2 S 14:15}] le peuple m'a e.. Et ta servante
\item[\vref{Né 6:9}] gens voulaient ns. e., en disant : Leurs
\item[\vref{Né 6:19}] Et Tobija envoyait des lettres pour m'e.
\item[\vref{Da 3:24}] roi Nebucadnetsar fut e., et se leva
\item[\vref{Da 7:15}] et les visions de ma tête m'e.
\item[\vref{Mc 14:33}] commença à être e. et fort agité.
\item[\vref{Lu 24:37}] tt terrifiés et e. croyaient voir un
\item[\vref{Ac 9:6}] tremblant et tt e., il dit : Seign.,
\item[\vref{Ac 16:38}] préteurs, qui furent e. en apprenant qu'ils
\item[\vref{Ac 22:9}] moi furent tt e., ils virent bien
\end{listverse}

\ConcordanceEntry{Effroi}
\vspace{-2mm}
\begin{listverse}
\item[\vref{1 S 14:15}] eut un grand e. au camp, à
\item[\vref{Ps 10:18}] tiré de la terre cesse d'inspirer l'e.
\item[\vref{Ps 119:120}] chair frissonne de l'e. que tu m'inspires
\item[\vref{Pr 1:27}] Quand votre e. surviendra com. une
\item[\vref{Jé 17:17}] moi un sujet d'e., toi, mon refuge
\item[\vref{Da 10:16}] vision m'a rempli d'e., et j'ai perdu
\end{listverse}

\ConcordanceEntry{Egal}
\vspace{-2mm}
\begin{listverse}
\item[\vref{Job 1:8}] qui n'a point d'é. sur la terre ;
\item[\vref{Job 2:3}] qui n'a point d'é. sur la terre ;
\item[\vref{Ps 55:14}] que j'estimais mon é., mon confident et
\item[\vref{Mt 20:12}] les as faits é. à ns., qui
\item[\vref{Jn 5:18}] était son propre Père, se faisant é. à Dieu.
\item[\vref{Ro 14:5}] les estime ts é.. Que chacun soit
\end{listverse}

\ConcordanceEntry{Egard}
\vspace{-2mm}
\begin{listverse}
\item[\vref{Ge 4:4}] graisse. Yahweh eut é. à Abel, et
\item[\vref{Ge 19:19}] bonté à mon é. en préservant ma
\item[\vref{Lé 19:15}] Tu n'auras point d'é. à la personne
\item[\vref{De 1:17}] Vous n'aurez point d'é. à l'apparence de
\item[\vref{Job 34:19}] qui n'a point d'é. à la personne
\item[\vref{Da 11:32}] agissent méchamment à l'é. de l'alliance. Mais
\item[\vref{Mal 2:9}] et vs. avez é. à l'apparence des
\item[\vref{Ph 2:30}] n'ayant eu aucun é. à sa propre
\item[\vref{Ja 2:9}] si vs. avez é. à l'apparence des
\end{listverse}

\ConcordanceEntry{Egarement}
\vspace{-2mm}
\begin{listverse}
\item[\vref{Ge 6:3}] hommes ds lr. é., car aussi ils
\item[\vref{De 28:28}] folie, d'aveuglement, et d'é. d'esprit ;
\item[\vref{Jé 8:5}] à de perpétuels é. ? Ils tiennent ferme
\item[\vref{Ro 1:27}] récompense de lr. é. qui lr. était
\item[\vref{2 Th 2:11}] envoie une puissance d'é., pour qu'ils croient
\item[\vref{Ja 5:20}] pécheur de son é., sauvera une âme
\item[\vref{2 Pi 2:18}] retirés de ceux qui vivent ds l'é. ;
\item[\vref{Jud 1:11}] couru par un é. tel que celui
\end{listverse}

\ConcordanceEntry{Egarer}
\vspace{-2mm}
\begin{listverse}
\item[\vref{De 27:18}] soit celui qui é. un aveugle ds
\item[\vref{Ps 14:3}] se sont ts é., ils se sont
\item[\vref{Ps 58:4}] méchants se sont é. dès le sein
\item[\vref{Ps 119:10}] me laisse pas m'é. loin de tes
\item[\vref{Es 3:12}] qui te conduisent t'é., ils corrompent le
\item[\vref{Am 2:4}] pères avaient suivis les ont fait é.
\item[\vref{Mt 18:13}] quatre-vingt-dix-neuf qui ne se sont pas é.
\item[\vref{Ro 3:11}] se sont ts é., ils se sont
\item[\vref{Tit 3:3}] autrefois insensés, désobéissants, é., asservis à tte
\item[\vref{Hé 3:10}] dis : Leur cœur s'é. toujours. Et ils
\item[\vref{Hé 5:2}] ignorants et les é., puisqu'il est aussi
\item[\vref{Ja 5:19}] parmi vs. s'est é. loin de la
\item[\vref{2 Pi 2:15}] chemin, se sont é. et ont suivi
\end{listverse}

\ConcordanceEntry{Eglise}
\vspace{-2mm}
\begin{listverse}
\item[\vref{Mt 16:18}] je bâtirai mon E., et les portes
\item[\vref{Mt 18:17}] écouter, dis-le à l'é. ; et s'il refuse
\item[\vref{Ac 2:47}] les jours à l'E. des gens pour
\item[\vref{Ac 8:3}] Mais Saul ravageait l'é., entrant ds ttes
\item[\vref{Ac 12:5}] la prison ; mais l'E. faisait sans cesse
\item[\vref{Ac 15:41}] Syrie et la Cilicie, fortifiant les é.
\item[\vref{Ac 20:28}] évêques, pour paître l'E. de Dieu, qu'il
\item[\vref{Ro 16:5}] Saluez aussi l'é. qui est ds
\item[\vref{1 Co 12:28}] a établi ds l'E., premièrement des apôtres,
\item[\vref{1 Co 14:4}] lui-mm, mais celui qui prophétise édifie l'E.
\item[\vref{2 Co 8:1}] a faite aux é. de la Macédoine.
\item[\vref{2 Co 11:28}] souci que j'ai de ttes les é.
\item[\vref{Ga 1:13}] persécutais à outrance l'E. de Dieu et
\item[\vref{Ga 2:4}] et glissés ds l'é. pour épier la
\item[\vref{Ep 1:22}] choses pour être le Chef de l'E.,
\item[\vref{Ep 5:23}] le Chef de l'E., qui est son
\item[\vref{Ep 5:27}] dvt lui cette E. glorieuse, sans tache,
\item[\vref{Ep 5:32}] je parle de Christ et de l'E.
\item[\vref{Ph 3:6}] au zèle, persécutant l'E. ; et quant à
\item[\vref{Col 1:18}] du corps de l'E., et qui est
\item[\vref{1 Th 2:14}] les imitateurs des é. de Dieu qui
\item[\vref{1 Ti 3:15}] Dieu, qui est l'E. du Dieu vivant,
\item[\vref{Phm 1:2}] combat, et à l'é. qui est ds
\item[\vref{Hé 12:23}] l'assemblée et de l'E. des premiers-nés qui
\item[\vref{Ap 1:4}] Jean aux sept é. qui sont en
\item[\vref{Ap 2:7}] l'Esprit dit aux é. ! A celui qui
\item[\vref{Ap 22:16}] choses ds les é.. Je suis le
\end{listverse}

\ConcordanceEntry{Egorger}
\vspace{-2mm}
\begin{listverse}
\item[\vref{Ge 22:10}] le couteau pour é. son fils.
\item[\vref{Ex 12:6}] de l'assemblée d'Israël l'é. entre les deux
\item[\vref{1 R 18:40}] de Kison, où il les fit é.
\end{listverse}

\ConcordanceEntry{Egypte}
\vspace{-2mm}
\begin{listverse}
\item[\vref{Ge 42:5}] d'Israël allèrent en E. pour acheter du
\end{listverse}
\begin{legend}
\NoAutoSpaceBeforeFDP{
\item Abram en Egypte : Ge 12:10; 13:1
\item Joseph esclave puis établi  en E : Ge 37:28; 41:41
\item Jacob en E : Ge 46:3-7
\item Israël esclave en E : Ex 1:11-12; 5:9
\item Israël sort d'E : Ex 12:41; 13:9
\item Fuite en E de la famille de Jésus : Mt 2:13
\item Paroles de Yahweh sur l'E : Es 19:1,21; Jé 46:2; Ez 29:2
\item Autres : 2 R 18:21; Es 30:2; Jé 2:18,36; Ez 29:14; Ap 11:8; Ps 68:32; Es 30:3; 43:3
}
\end{legend}

\ConcordanceEntry{Ehud}
\vspace{-2mm}
\begin{listverse}
\item[\vref{Jg 3:15}] suscita un libérateur, E., fils de Guéra,
\end{listverse}

\ConcordanceEntry{Ekron}
\vspace{-2mm}
\begin{listverse}
\item[\vref{Jos 13:3}] jusqu'à la frontière d'E. au nord, contrée
\item[\vref{Jg 1:18}] ses territoires ; et E. avec ses territoires.
\item[\vref{1 S 5:10}] de Dieu à E.. Or com. l'arche
\item[\vref{2 R 1:16}] consulter Baal-Zebub, dieu d'E., com. s'il n'y
\item[\vref{Jé 25:20}] à Gaza, à E., et au reste
\item[\vref{So 2:4}] plein midi, et E. sera arrachée.
\end{listverse}

\ConcordanceEntry{Ela}
\vspace{-2mm}
\begin{listverse}
\item[\vref{1 R 16:8}] roi de Juda, E., fils de Baescha,
\end{listverse}

\ConcordanceEntry{Elath}
\vspace{-2mm}
\begin{listverse}
\item[\vref{De 2:8}] de la plaine, d'E. et d'Etsjon-Guéber, et
\item[\vref{2 R 14:22}] Azaria bâtit E. et la fit
\item[\vref{2 R 16:6}] Syrie, fit rentrer E. au pouvoir des
\end{listverse}

\ConcordanceEntry{Eléazar}
\vspace{-2mm}
\begin{listverse}
\item[\vref{Ex 6:25}] E., fils d'Aaron, prit pour fem. une
\item[\vref{No 4:16}] Et E. fils d'Aaron, le prêtre, aura la
\item[\vref{No 16:39}] Ainsi E., le prêtre, prit les encensoirs d'airain,
\item[\vref{No 20:26}] fais-les revêtir à E., son fils. C'est
\item[\vref{No 34:17}] partageront le pays : E. le prêtre, et
\item[\vref{De 10:6}] il fut enseveli ; E., son fils, exerça
\item[\vref{Jos 17:4}] dvt le prêtre E., dvt Josué, fils
\item[\vref{1 S 7:1}] et ils consacrèrent E., son fils, pour
\end{listverse}

\ConcordanceEntry{Election}
\vspace{-2mm}
\begin{listverse}
\item[\vref{Ro 9:11}] dessein arrêté selon l'é. de Dieu demeurât,
\item[\vref{Ro 11:5}] un reste, selon l'é. de la grâce.
\item[\vref{Ro 11:28}] ce qui concerne l'é., ils sont aimés
\item[\vref{1 Th 1:4}] mes frères bien-aimés de Dieu, votre é.
\item[\vref{2 Pi 1:10}] vocation et votre é. ; car, en faisant
\end{listverse}

\ConcordanceEntry{EL-Elohé-Israël}
\vspace{-2mm}
\begin{listverse}
\item[\vref{Ge 33:20}] autel qu'il appela E. (le Dieu Fort,
\end{listverse}

\ConcordanceEntry{Elever}
\vspace{-2mm}
\begin{listverse}
\item[\vref{Ge 7:17}] eaux crûrent et é. l'arche, et elle
\item[\vref{Lé 19:16}] peuple. Tu ne t'é. point contre le
\item[\vref{No 16:3}] d'eux, pourquoi vs. é.-vs. au-dessus de
\item[\vref{Jos 3:7}] je commencerai à t'é. aux yeux de
\item[\vref{Job 11:15}] certainement tu pourras é. ton visage sans
\item[\vref{Ps 75:8}] qui gouverne ; il abaisse l'un, et é. l'autre.
\item[\vref{Ps 97:9}] tu es fort é. au-dessus de ts
\item[\vref{Ps 103:11}] les cieux sont é. au-dessus de la
\item[\vref{Ps 124:2}] qnd les hommes s'é. contre ns.,
\item[\vref{Pr 11:11}] La ville est é. par la bénédiction
\item[\vref{Es 2:17}] les hommes qui s'é. seront abaissés :
\item[\vref{Ez 21:31}] sera plus celle-ci ; j'é. ce qui est
\item[\vref{Da 8:25}] paix, et il s'é. contre le Prince
\item[\vref{Da 11:3}] Mais il s'é. un vaillant roi,
\item[\vref{Za 14:13}] main de l'un s'é. contre la main
\item[\vref{Mt 23:12}] Car quiconque s'é. sera abaissé ; et
\item[\vref{Jn 8:28}] Quand vs. aurez é. le Fils de
\item[\vref{Jn 12:32}] qnd je serai é. de la terre,
\item[\vref{Ph 2:9}] Dieu l'a souverainement é. et lui a
\item[\vref{2 Th 2:4}] lequel s'oppose et s'é. contre tt ce
\item[\vref{Hé 7:26}] des pécheurs, et é. au-dessus des cieux,
\item[\vref{Ja 4:10}] présence du Seign., et il vs. é.
\item[\vref{Ja 5:3}] et lr. rouille s'é. en témoignage contre
\item[\vref{1 Pi 5:6}] afin qu'il vs. é. qnd le temps
\end{listverse}

\ConcordanceEntry{Eli}
\vspace{-2mm}
\begin{listverse}
\item[\vref{Mt 27:46}] d'une voix forte : E., Eli, lama sabachthani ?
\end{listverse}

\ConcordanceEntry{Eli}
\vspace{-2mm}
\begin{listverse}
\item[\vref{1 S 1:9}] Et le prêtre E. était assis sur
\item[\vref{1 S 2:27}] Dieu vint auprès d'E., et lui dit :
\item[\vref{1 S 4:15}] Or E. était âgé de quatre-vingt-dix-huit ans, ses
\item[\vref{1 S 4:18}] l'arche de Dieu, E. tomba à la
\item[\vref{1 R 2:27}] prononcée à Silo contre la maison d'E.
\end{listverse}

\ConcordanceEntry{Eliab}
\vspace{-2mm}
\begin{listverse}
\item[\vref{1 S 16:6}] entrée, il remarqua E., et se dit :
\item[\vref{1 S 17:13}] la guerre étaient E., le premier-né, Abinadab,
\item[\vref{1 S 17:28}] Et qnd E., son frère aîné, entendit qu'il parlait
\end{listverse}

\ConcordanceEntry{Eliakim}
\vspace{-2mm}
\begin{listverse}
\item[\vref{2 R 18:18}] tt haut ; alors E., fils de Hilkija,
\item[\vref{2 R 19:2}] Puis il envoya E., chef de la
\end{listverse}

\ConcordanceEntry{Eliaschib}
\vspace{-2mm}
\begin{listverse}
\item[\vref{Né 3:1}] E., le grand-prêtre, se leva dc avec
\item[\vref{Né 13:4}] que ceci arrive, E., prêtre établi sur
\end{listverse}

\ConcordanceEntry{Elie}
\vspace{-2mm}
\begin{listverse}
\item[\vref{Ja 5:17}] E. était un hom. sujet aux mêmes
\end{listverse}
\begin{legend}
\NoAutoSpaceBeforeFDP{
\item Le prophète : 1 R 17:1
\item Annonce de trois ans de sécheresse : 1 R 17:1; Ja 5:17-18
\item E au torrent de Kérith : 1 R 17:3-6
\item Résurrection du fils de la veuve de Sarepta : 1 R 17:21-22
\item Confrontation au mont Carmel : 1 R 18
\item Hazaël, Jéhu, Elisée; oints : 1 R 19: 15-17
\item Jugement d'Achab et de Jézabel : 1 R 21:21-24; 22:38; 2 R 9:36
\item Division des eaux du Jourdain : 2 R 2:8
\item Enlèvement d'Elie au ciel : 2 R 2:11
\item Participation à la transfiguration : Mt 17:3
\item L'Esprit d'Elie en Jean-Baptiste : Mal 4; Mt 11:14; 17:12-13; Lu 1:13-17
}
\end{legend}

\ConcordanceEntry{Eliézer}
\vspace{-2mm}
\begin{listverse}
\item[\vref{Ge 15:2}] ma maison c'est E. de Damas.
\item[\vref{Ex 18:4}] et l'autre E., car il avait
\item[\vref{2 Ch 20:37}] Alors E., fils de Dodava, de Maréscha, prophétisa
\end{listverse}

\ConcordanceEntry{Elihu}
\vspace{-2mm}
\begin{listverse}
\item[\vref{Job 32:2}] s'enflamma la colère d'E., fils de Barakeël
\item[\vref{Job 34:1}] E. dc reprit la parole, et dit :
\item[\vref{Job 35:1}] E. poursuivit son discours et dit :
\item[\vref{Job 36:1}] E. continua de parler, et dit :
\end{listverse}

\ConcordanceEntry{Elimélec}
\vspace{-2mm}
\begin{listverse}
\item[\vref{Ru 1:2}] cet hom. était E., le nom de
\item[\vref{Ru 1:3}] Or E., mari de Naomi, mourut ; et elle
\end{listverse}

\ConcordanceEntry{Eliphaz}
\vspace{-2mm}
\begin{listverse}
\item[\vref{Job 2:11}] intimes de Job, E. de Théman, Bildad
\item[\vref{Job 4:1}] Alors E. de Théman prit la parole et
\item[\vref{Job 15:1}] Alors E. de Théman prit la parole et
\item[\vref{Job 22:1}] Alors E. de Théman prit la parole et
\item[\vref{Job 42:7}] il dit à E. de Théman : Ma
\item[\vref{Job 42:9}] Ainsi, E. de Théman, Bildad de Schuach, et
\end{listverse}

\ConcordanceEntry{Elisabeth}
\vspace{-2mm}
\begin{listverse}
\item[\vref{Lu 1:13}] est exaucée, et E., ta fem. t'enfantera
\item[\vref{Lu 1:36}] Voici, E., ta cousine, a conçu elle aussi,
\item[\vref{Lu 1:57}] Le temps où E. devait accoucher arriva,
\end{listverse}

\ConcordanceEntry{Elisée}
\vspace{-2mm}
\begin{listverse}
\item[\vref{1 R 19:16}] et tu oindras E., fils de Schaphath,
\end{listverse}
\begin{legend}
\NoAutoSpaceBeforeFDP{
\item Prophète à la place d'Elie : 1 R 19:16; 2 R 2:15
\item Division des eaux du Jourdain : 2 R 2:13-14
\item Puissance de Yahweh en E : 2 R 2:20-22; 3:17-18; 4:1-44; 5:10-14; 6:6-18; 7:1-18
\item Prophétise le règne d'Hazaël : 2 R 8:13
\item Jéhu oint roi d'Israël : 2 R 9:2-3
\item Sa mort; un cadavre reprend vie dans le sépulcre d'E. : 2 R 13:20-21
}
\end{legend}

\ConcordanceEntry{Elkana}
\vspace{-2mm}
\begin{listverse}
\item[\vref{1 S 1:19}] maison à Rama. E. connut Anne, sa
\item[\vref{1 S 1:21}] Puis E. son mari monta avec tte sa
\item[\vref{1 Ch 6:28}] de Samuel, fils d'E., son fils aîné
\end{listverse}

\ConcordanceEntry{Eloigner}
\vspace{-2mm}
\begin{listverse}
\item[\vref{Ge 4:16}] Alors Caïn s'é. de la présence
\item[\vref{Ge 49:10}] Le sceptre ne s'é. point de Juda,
\item[\vref{2 S 12:10}] Maintenant, l'épée ne s'é. jamais de ta
\item[\vref{Ps 22:2}] pourquoi m'as-tu abandonné, t'é. de ma délivrance
\item[\vref{Ps 73:27}] voilà, ceux qui s'é. de toi périront ;
\item[\vref{Ps 103:12}] Il é. de ns. nos transgressions, autant que
\item[\vref{Es 29:13}] son cœur est é. de moi ; et
\item[\vref{Ez 12:27}] pour des temps qui sont encore é.
\item[\vref{Am 6:3}] Vous qui é. le jour du malheur, et qui
\item[\vref{Mt 26:42}] Il s'é. encore pour la seconde fois, et
\item[\vref{Ro 16:17}] vs. avez apprise et de vs. é. d'eux.
\item[\vref{Ep 2:13}] qui étiez autrefois é., vs. avez été
\item[\vref{2 Ti 3:5}] renié la force. E.-toi dc de
\item[\vref{Ap 18:10}] ils se tiendront é. ds la crainte
\end{listverse}

\ConcordanceEntry{Elu (un)}
\vspace{-2mm}
\begin{listverse}
\item[\vref{1 Ch 16:13}] son serviteur, fils de Jacob, ses é. !
\item[\vref{Ps 89:4}] alliance avec mon é., j'ai fait serment
\item[\vref{Ps 106:5}] bien de tes é., que je me
\item[\vref{Ps 106:23}] mais Moïse, son é., se tint à
\item[\vref{Es 42:1}] soutiens, c'est mon é., en qui mon
\item[\vref{Es 45:4}] et d'Israël mon é. ; je t'ai, dis-je,
\item[\vref{Es 65:9}] montagnes ; et mes é. hériteront le pays,
\item[\vref{Mt 22:14}] y a beaucoup d'appelés, mais peu d'é.
\item[\vref{Mt 24:22}] à cause des é., ces jours seront
\item[\vref{Mt 24:31}] ils rassembleront ses é., des quatre vents,
\item[\vref{Mc 13:22}] séduire mm les é., s'il était possible.
\item[\vref{Lu 18:7}] justice à ses é., qui crient à
\item[\vref{Lu 23:35}] est le Christ, l'é. de Dieu !
\item[\vref{Ro 8:33}] accusation contre les é. de Dieu ? Dieu
\item[\vref{Ro 11:7}] obtenu, mais les é. l'ont obtenu, tandis
\item[\vref{Ep 1:4}] qu'il ns. a é. en lui avant
\item[\vref{Col 3:12}] dc, com. des é. de Dieu, saints
\item[\vref{Tit 1:1}] la foi des é. de Dieu et
\item[\vref{1 Pi 1:2}] é. selon la prescience de Dieu le
\item[\vref{1 Pi 5:13}] et qui est é. com. vs. et
\item[\vref{2 Jn 1:1}] L'ancien, à Kyria l'é. et à ses
\item[\vref{Ap 17:14}] les appelés, les é. et les fidèles
\end{listverse}

\ConcordanceEntry{Elymas}
\vspace{-2mm}
\begin{listverse}
\item[\vref{Ac 13:8}] Mais E., le magicien, car c'est ce que
\end{listverse}

\ConcordanceEntry{Embarrasser}
\vspace{-2mm}
\begin{listverse}
\item[\vref{2 R 2:17}] qu'il en était e.. Il lr. dit
\item[\vref{2 Ti 2:4}] de soldat qui s'e. des affaires de
\end{listverse}

\ConcordanceEntry{Embaumer}
\vspace{-2mm}
\begin{listverse}
\item[\vref{Ge 50:2}] qui étaient médecins d'e. son père ; et
\item[\vref{Mc 14:8}] elle a d'avance e. mon corps pour
\item[\vref{Mc 16:1}] Salomé, achetèrent des aromates pour venir l'e.
\end{listverse}

\ConcordanceEntry{Embrasser}
\vspace{-2mm}
\begin{listverse}
\item[\vref{Ge 31:28}] m'as pas laissé e. mes fils et
\item[\vref{1 R 19:20}] t'en prie, laisse-moi e. mon père et
\item[\vref{Ps 85:11}] rencontrent, la justice et la paix s'e.
\item[\vref{Ec 3:5}] un temps pour e. et un temps
\item[\vref{Mt 28:9}] Et elles s'approchèrent, e. ses pieds et
\item[\vref{Mc 14:44}] signe : Celui que j'e., c'est lui ; saisissez-le,
\end{listverse}

\ConcordanceEntry{Embûche}
\vspace{-2mm}
\begin{listverse}
\item[\vref{Ex 21:13}] a point dressé d'e., mais que Dieu
\item[\vref{Esd 8:31}] ennemis et des e. sur le chemin.
\item[\vref{Pr 1:18}] ceux-ci dressent des e. contre le sang
\item[\vref{Pr 3:26}] il gardera ton pied de tte e.
\item[\vref{Jé 9:8}] mais au-dedans il lui dresse des e.
\item[\vref{Os 7:6}] appliqué à leurs e. lr. cœur embrasé
\item[\vref{Mi 7:2}] sont ts en e. pour verser le
\item[\vref{Ac 23:30}] été averti des e. que les Juifs
\end{listverse}

\ConcordanceEntry{Emeraude}
\vspace{-2mm}
\begin{listverse}
\item[\vref{Ex 28:17}] une sardoine, une topaze, et une é.
\item[\vref{Ex 39:10}] une sardoine, une topaze et une é.
\item[\vref{Ez 28:13}] de saphir, d'escarboucle, d'é., et d'or ; tes
\item[\vref{Ap 4:3}] environné d'un arc-en-ciel semblable à de l'é.
\item[\vref{Ap 21:19}] le troisième de calcédoine, le quatrième d'é.,
\end{listverse}

\ConcordanceEntry{Emmanuel}
\vspace{-2mm}
\begin{listverse}
\item[\vref{Es 7:14}] et elle lui donnera le nom d'E.
\item[\vref{Es 8:8}] la largeur de ton pays, ô E. !
\item[\vref{Mt 1:23}] donnera le Nom d'E., ce qui signifie,
\end{listverse}

\ConcordanceEntry{Emmaüs}
\vspace{-2mm}
\begin{listverse}
\item[\vref{Lu 24:13}] un village nommé E., éloigné de Jérus.
\end{listverse}

\ConcordanceEntry{Emonder}
\vspace{-2mm}
\begin{listverse}
\item[\vref{Jn 15:2}] du fruit, il l'é., afin qu'il porte
\end{listverse}

\ConcordanceEntry{Empêcher}
\vspace{-2mm}
\begin{listverse}
\item[\vref{Ge 11:6}] rien ne les e. d'exécuter ce qu'ils
\item[\vref{Ge 20:6}] cœur, aussi ai-je e. que tu ne
\item[\vref{Ge 30:2}] de Dieu pour t'e. d'avoir des enfants ?
\item[\vref{Ex 4:10}] j'ai la bouche et la langue e.
\item[\vref{1 S 14:6}] on ne saurait e. Yahweh de délivrer
\item[\vref{1 S 25:34}] d'Israël, qui m'a e. de te faire
\item[\vref{Esd 4:4}] ils l'intimidèrent pour l'e. de bâtir,
\item[\vref{Né 13:19}] aux portes, afin d'e. l'entrée des fardeaux
\item[\vref{Job 42:2}] qu'on ne saurait t'e. de faire ce
\item[\vref{Es 43:13}] main ; je ferai l'œuvre, qui m'en e. ?
\item[\vref{Es 66:9}] postérité aux autres, l'e.-je d'enfanter ? dit
\item[\vref{Da 4:35}] a personne qui e. sa main, et
\item[\vref{Mt 19:14}] et ne les e. pas ; car le
\item[\vref{Lu 9:49}] ns. l'en avons e., parce qu'il ne
\item[\vref{Lu 11:52}] et vs. avez e. ceux qui entraient.
\item[\vref{1 Co 14:39}] de prophétiser, et n'e. pas de parler
\item[\vref{Ga 5:7}] arrêtés pour vs. e. d'obéir à la
\item[\vref{1 Th 2:18}] Paul ; mais Satan ns. en a e.
\end{listverse}

\ConcordanceEntry{Empire}
\vspace{-2mm}
\begin{listverse}
\item[\vref{Es 9:5}] été donné, et l'e. reposera sur son
\item[\vref{Es 9:6}] pour accroître l'e., et une paix
\item[\vref{Ac 10:38}] qui étaient sous l'e. du diable, car
\item[\vref{1 Co 15:24}] avoir aboli tt e., tte puissance, et
\end{listverse}

\ConcordanceEntry{Empirer}
\vspace{-2mm}
\begin{listverse}
\item[\vref{Mc 5:26}] soulagement, mais était allée plutôt en e.
\item[\vref{1 Co 11:17}] non pour devenir meilleurs, mais pour e.
\item[\vref{2 Ti 3:13}] imposteurs iront en e., séduisant les autres
\end{listverse}

\ConcordanceEntry{Emporter}
\vspace{-2mm}
\begin{listverse}
\item[\vref{Ex 2:9}] Pharaon lui dit : E. cet enfant, et
\item[\vref{Ex 36:3}] lesquels e. de dvt Moïse tte l'offrande que
\item[\vref{1 S 17:34}] un ours venait e. une brebis du
\item[\vref{Ps 49:18}] lorsqu'il mourra, il n'e. rien, ses trésors
\item[\vref{Ps 90:5}] Tu les e. semblables à un songe qui, le
\item[\vref{Pr 14:29}] est prompt à s'e. excite la folie.
\item[\vref{Ez 38:13}] grand pillage, pour e. de l'argent et
\item[\vref{Mt 9:16}] car la pièce e. une partie de
\item[\vref{Mt 24:17}] descende pas pour e. quoi que ce
\item[\vref{Ac 27:17}] qu'on se laissa e. par le vent.
\item[\vref{Ep 4:14}] enfants flottants et e. çà et là
\item[\vref{1 Ti 6:7}] évident que ns. n'en pouvons rien e.
\item[\vref{2 Ti 3:4}] traîtres, e., enflés d'orgueil, amis
\item[\vref{Hé 13:9}] Ne soyez pas e. çà et là
\item[\vref{2 Pi 3:17}] prenez garde qu'étant e. avec les autres
\item[\vref{Jud 1:12}] nuées sans eau, e. par des vents
\end{listverse}

\ConcordanceEntry{Empreinte}
\vspace{-2mm}
\begin{listverse}
\item[\vref{Hé 1:3}] sa gloire, et l'e. de son être,
\end{listverse}

\ConcordanceEntry{Emprunter}
\vspace{-2mm}
\begin{listverse}
\item[\vref{Ex 22:14}] Si quelqu'un a e. à son prochain
\item[\vref{De 15:6}] nations, et tu n'e. point sur gage ;
\item[\vref{De 28:12}] à beaucoup de nations, et tu n'e. point.
\item[\vref{2 R 6:5}] dit : Ah ! Mon seigneur ! Je l'avais e. !
\item[\vref{Ps 37:21}] Lamed.] Le méchant e., et ne rend
\item[\vref{Pr 22:7}] et celui qui e. est l'esclave de
\item[\vref{Jé 15:10}] le pays ! Je n'e. ni ne prête,
\item[\vref{Mt 5:42}] détourne pas de celui qui veut e. de toi.
\end{listverse}

\ConcordanceEntry{En haut}
\vspace{-2mm}
\begin{listverse}
\item[\vref{Ge 27:39}] et de la rosée du ciel, d'en haut.
\item[\vref{Ge 49:25}] bénédictions des cieux en haut, des bénédictions des
\item[\vref{Ex 25:20}] étendront les ailes en haut, couvrant de leurs
\item[\vref{De 4:39}] Yahweh est Dieu, en haut ds les cieux
\item[\vref{De 5:8}] choses qui sont en haut ds les cieux,
\item[\vref{De 28:13}] tu seras toujours en haut et jamais en
\item[\vref{2 R 19:22}] porté tes yeux en haut, vers le Saint
\item[\vref{Job 31:28}] jugée ; car j'aurais renié le Dieu d'en haut.
\item[\vref{Ps 50:4}] appellera les cieux d'en haut, et la terre
\item[\vref{Es 40:26}] Elevez vos yeux en haut et regardez ! Qui
\item[\vref{Es 58:4}] pour que votre voix soit exaucée d'en haut.
\item[\vref{Jé 31:37}] Si les cieux en haut peuvent être mesurés,
\item[\vref{Lu 1:78}] le Soleil Levant ns. a visités d'en haut,
\item[\vref{Lu 21:28}] à arriver, regardez en haut et levez vos
\item[\vref{Lu 24:49}] vs. soyez revêtus de la puissance d'en haut.
\item[\vref{Lu 24:50}] levant ses mains en haut, il les bénit.
\item[\vref{Jn 3:3}] quelqu'un ne naît d'en haut, il ne peut
\item[\vref{Jn 3:7}] dit : Il faut que vs. naissiez d'en haut.
\item[\vref{Jn 3:31}] Celui qui vient d'en haut est au-dessus de
\item[\vref{Jn 8:23}] moi, je suis d'en haut. Vous êtes de
\item[\vref{Jn 19:11}] t'avait été donné d'en haut. C'est pourquoi celui
\item[\vref{Ac 2:19}] des choses merveilleuses en haut ds le ciel,
\item[\vref{Ga 4:26}] Mais la Jérus. d'en haut est la fem.
\item[\vref{Ep 4:8}] dit : Etant monté en haut, il a emmené
\item[\vref{Col 3:1}] choses qui sont en haut, où Christ est
\item[\vref{Col 3:2}] Pensez aux choses d'en haut, et non à
\item[\vref{Ja 1:17}] don parfait viennent d'en haut et descendent du
\item[\vref{Ja 3:17}] Mais la sagesse d'en haut est premièrement pure,
\end{listverse}

\ConcordanceEntry{Enceinte}
\vspace{-2mm}
\begin{listverse}
\item[\vref{Ge 25:22}] ainsi, pourquoi suis-je e. ? Et elle alla
\item[\vref{Ex 21:22}] frappe une fem. e., et qu'elle en
\item[\vref{Ec 11:5}] celle qui est e., ainsi tu ne
\item[\vref{Es 7:14}] une vierge sera e., et elle enfantera
\item[\vref{Jé 20:17}] Et pourquoi n'est-elle pas restée éternellement e. ?
\item[\vref{Mt 1:18}] Joseph, se trouva e., par l'opération du
\item[\vref{Mt 24:19}] Malheur aux femmes e., et à celles
\item[\vref{Ap 12:2}] Elle était e., et elle criait,
\end{listverse}

\ConcordanceEntry{Encens}
\vspace{-2mm}
\begin{listverse}
\item[\vref{Ex 30:34}] préparé, et de l'e. pur, le tt
\item[\vref{Lé 2:1}] de l'huile dessus, et mettra de l'e.
\item[\vref{Lé 24:7}] tu mettras de l'e. pur sur chaque
\item[\vref{Ca 3:6}] de myrrhe et d'e., et de ts
\item[\vref{Jé 6:20}] Pourquoi m'offrir de l'e. venu de Séba,
\item[\vref{Jé 32:29}] a brûlé de l'e. à Baal, et
\item[\vref{Jé 44:17}] bouche, brûler de l'e. à la reine
\item[\vref{Ha 1:16}] il offre de l'e. à ses rets,
\item[\vref{Mal 1:11}] on brûle de l'e. en l'honneur de
\item[\vref{Mt 2:11}] De l'or, de l'e. et de la
\end{listverse}

\ConcordanceEntry{Encensoir}
\vspace{-2mm}
\begin{listverse}
\item[\vref{Lé 10:1}] prirent chacun lr. e., mirent du feu,
\item[\vref{Ez 8:11}] ces idoles, chacun l'e. à la main,
\item[\vref{Hé 9:4}] ayant un e. d'or, et l'arche de l'alliance, entièrement
\item[\vref{Ap 8:3}] l'autel, ayant un e. d'or, et plusieurs
\end{listverse}

\ConcordanceEntry{Enchantement}
\vspace{-2mm}
\begin{listverse}
\item[\vref{Ex 8:14}] mm par leurs e., pour produire des
\item[\vref{No 23:23}] n'y a pas d'e. contre Jacob, ni
\item[\vref{De 18:11}] d'enchanteur qui use d'e., personne qui consulte
\item[\vref{2 R 17:17}] divination et aux e., et ils se
\item[\vref{Jé 8:17}] n'y a pas d'e., et ils vs.
\item[\vref{Ap 9:21}] ni de leurs e., ni de lr.
\item[\vref{Ap 18:23}] que par tes e. ttes les nations
\end{listverse}

\ConcordanceEntry{Enchanteur}
\vspace{-2mm}
\begin{listverse}
\item[\vref{Ex 7:11}] sages et les e. ; et les magiciens
\item[\vref{De 18:11}] ni d'e. qui use d'enchantements, personne qui consulte
\item[\vref{Es 3:3}] l'expert d'entre les artisans et l'habile e.
\item[\vref{Es 19:3}] idoles et les e., ceux qui évoquent
\item[\vref{Da 2:2}] les astrologues, les e. et les Chaldéens,
\item[\vref{Mal 3:5}] témoigner contre les e. et les adultères,
\item[\vref{Ac 8:9}] qui exerçait l'art d'e., et ensorcelait le
\end{listverse}

\ConcordanceEntry{Encre}
\vspace{-2mm}
\begin{listverse}
\item[\vref{Jé 36:18}] écrivais avec de l'e. ds le livre.
\item[\vref{2 Co 3:3}] non avec de l'e., mais avec l'Esprit
\item[\vref{2 Jn 1:12}] papier et de l'e., mais j'espère aller
\item[\vref{3 Jn 1:13}] pas t'écrire avec l'e. et la plume.
\end{listverse}

\ConcordanceEntry{En-Dor}
\vspace{-2mm}
\begin{listverse}
\item[\vref{1 S 28:7}] une fem. à E. qui évoque les
\item[\vref{Ps 83:11}] furent détruits à E. et servirent de
\end{listverse}

\ConcordanceEntry{Endormir}
\vspace{-2mm}
\begin{listverse}
\item[\vref{Ge 2:21}] sur Adam, qui s'e. ; et Dieu prit
\item[\vref{Jg 16:19}] Elle l'e. sur ses genoux. Et ayant appelé
\item[\vref{Ps 3:6}] me couche, je m'e., je me réveille,
\item[\vref{Ca 5:2}] La Sulamithe :] J'étais e., mais mon cœur
\item[\vref{Jon 1:5}] se coucha et s'e. profondément.
\item[\vref{Mt 25:5}] tardait à venir, elles s'assoupirent et s'e. ttes.
\item[\vref{Mt 26:40}] disciples, qu'il trouva e., et il dit
\item[\vref{Mc 13:36}] ne vs. trouve e. à son arrivée
\item[\vref{Lu 8:23}] qu'ils naviguaient, il s'e.. Un vent impétueux
\item[\vref{Ac 20:9}] sur une fenêtre, s'e. profondément pendant le
\end{listverse}

\ConcordanceEntry{Endurcir}
\vspace{-2mm}
\begin{listverse}
\item[\vref{Ex 4:21}] dvt Pharaon ; mais j'e. son cœur et
\item[\vref{Ex 7:13}] cœur de Pharaon s'e. et il ne
\item[\vref{Ex 14:17}] je m'en vais e. le cœur des
\item[\vref{De 15:7}] te donne, tu n'e. point ton cœur,
\item[\vref{Ps 95:8}] n'e. point votre cœur, com. à Meriba,
\item[\vref{Pr 28:14}] mais celui qui e. son cœur tombera
\item[\vref{Es 46:12}] avez le cœur e. et qui êtes
\item[\vref{Mc 6:52}] pains, parce que lr. cœur était e.
\item[\vref{Jn 12:40}] et il a e. lr. cœur, de
\item[\vref{Ac 19:9}] com. quelques-uns restaient e. et rebelles, décriant
\item[\vref{Ro 9:18}] veut, et il e. qui il veut.
\item[\vref{Ro 11:7}] tandis que les autres ont été e.,
\item[\vref{2 Co 3:14}] leurs entendements sont e., car jusqu'à aujourd'hui
\item[\vref{Hé 3:8}] n'e. pas vos cœurs, com. il arriva
\item[\vref{Hé 3:13}] d'entre vs. ne s'e. par la séduction
\end{listverse}

\ConcordanceEntry{Endurcissement}
\vspace{-2mm}
\begin{listverse}
\item[\vref{La 3:65}] livre-les à l'e. de lr. cœur,
\item[\vref{Mc 3:5}] étant affligé de l'e. de lr. cœur,
\item[\vref{Ro 11:25}] est tombée ds l'e., jusqu'à ce que
\item[\vref{Ep 4:18}] en eux, par l'e. de lr. cœur.
\end{listverse}

\ConcordanceEntry{Enée}
\vspace{-2mm}
\begin{listverse}
\item[\vref{Ac 9:33}] un hom. appelé E., qui était couché
\item[\vref{Ac 9:34}] Pierre lui dit : E., Jésus-Christ te guérit !
\end{listverse}

\ConcordanceEntry{Enfant}
\vspace{-2mm}
\begin{listverse}
\item[\vref{Ge 21:17}] la voix de l'e., et l'Ange de
\item[\vref{Ge 42:36}] privez de mes e. ! Joseph n'est plus,
\item[\vref{No 14:33}] mais vos e. paîtront ds ce désert quarante ans
\item[\vref{Jos 4:21}] il parla aux e. d'Israël et lr.
\item[\vref{Jg 13:5}] tête, parce que l'e. sera Naziréen pour
\item[\vref{Ru 4:16}] Alors Naomi prit l'e. et le mit
\item[\vref{1 S 1:27}] pour avoir cet e., et Yahweh m'a
\item[\vref{2 S 12:15}] maison. Yahweh frappa l'e. que la fem.
\item[\vref{1 R 3:25}] Partagez en deux l'e. qui vit, et
\item[\vref{Pr 10:5}] L'e. prudent amasse en été, mais celui
\item[\vref{Pr 20:11}] Un jeune e. mm, fait connaître par ses actions
\item[\vref{Pr 22:6}] Instruis le jeune e. à l'entrée de
\item[\vref{Pr 22:15}] cœur du jeune e., mais la verge
\item[\vref{Pr 23:13}] correction du jeune e. ; qnd tu l'auras
\item[\vref{Pr 29:15}] la sagesse, mais l'e. livré à lui-mm
\item[\vref{Ec 4:13}] Un e. pauvre et sage vaut mieux qu'un
\item[\vref{Es 3:4}] chefs, et des e. domineront sur eux.
\item[\vref{Es 8:4}] Car avant que l'e. sache dire : Mon
\item[\vref{Es 9:5}] Car un e. ns. est né, un Fils ns.
\item[\vref{Es 11:6}] et un petit e. les conduira.
\item[\vref{Es 49:15}] peut-elle oublier son e. qu'elle allaite de
\item[\vref{Es 65:23}] n'engendreront plus des e. pour être exposés
\item[\vref{Jé 1:6}] pas parler, car je suis un e.
\item[\vref{Jé 1:7}] Je suis un e.. Car tu iras
\item[\vref{Os 11:1}] Israël était jeune e., je l'ai aimé,
\item[\vref{Mt 2:16}] tuer ts les e. qui étaient à
\item[\vref{Mt 7:11}] donner à vos e. de bonnes choses,
\item[\vref{Mt 10:21}] le père son e. ; et les enfants
\item[\vref{Mt 18:3}] com. les petits e., vs. n'entrerez pas
\item[\vref{Mt 19:14}] moi les petits e. et ne les
\item[\vref{Mc 10:15}] com. un petit e. le Royaume de
\item[\vref{Lu 1:41}] que le petit e. tressaillit ds son
\item[\vref{Jn 1:12}] a donné le pouvoir de devenir e. de Dieu,
\item[\vref{Jn 8:39}] Si vs. étiez e. d'Abraham, vs. feriez
\item[\vref{Jn 12:36}] que vs. soyez e. de lumière. Jésus
\item[\vref{Ro 8:16}] à notre esprit que ns. sommes e. de Dieu.
\item[\vref{Ro 9:8}] ceux qui sont e. de la chair
\item[\vref{1 Co 7:14}] mari ; autrement vos e. seraient impurs, or
\item[\vref{1 Co 13:11}] Quand j'étais e., je parlais com.
\item[\vref{Ga 4:27}] stérile, toi qui n'e. pas ! Eclate et
\item[\vref{Ep 6:1}] E., obéissez à vos pères et à
\item[\vref{1 Ti 3:4}] qu'il tienne ses e. ds la soumission
\item[\vref{Hé 2:13}] moi et les e. que Dieu m'a
\item[\vref{Hé 11:11}] de concevoir un e., et elle enfanta
\item[\vref{Hé 11:23}] qu'ils virent que l'e. était beau, et
\item[\vref{1 Jn 3:1}] ns. soyons appelés e. de Dieu ! Mais
\item[\vref{1 Jn 4:4}] Mes petits-e., vs. êtes de
\item[\vref{Ap 12:4}] de dévorer son e., dès qu'elle l'aurait
\end{listverse}

\ConcordanceEntry{Enfanter}
\vspace{-2mm}
\begin{listverse}
\item[\vref{Ge 3:16}] tes grossesses ; tu e. ds la douleur
\item[\vref{2 R 19:3}] d'opprobre ; car les e. sont près du
\item[\vref{Pr 27:1}] sais pas ce qu'un jour peut e.
\item[\vref{Es 54:1}] stérile, toi qui n'e. point, toi qui
\item[\vref{Es 66:8}] travail, elle a e. ses enfants !
\item[\vref{Es 66:9}] Moi qui fais e. les autres, ne
\item[\vref{Mi 5:2}] jusqu'au temps où e. celle qui doit
\item[\vref{Mt 1:21}] Elle e. un fils, et tu lui donneras
\item[\vref{Ja 1:15}] a conçu, elle e. le péché ; et
\end{listverse}

\ConcordanceEntry{Enfoncer}
\vspace{-2mm}
\begin{listverse}
\item[\vref{Es 8:22}] angoisses : Il sera e. ds l'obscurité.
\item[\vref{Jé 38:22}] pieds se sont e. ds la boue,
\item[\vref{La 2:9}] Ses portes sont e. ds la terre ;
\item[\vref{Mt 14:30}] il commençait à e., il s'écria : Seign. !
\end{listverse}

\ConcordanceEntry{Engendrer}
\vspace{-2mm}
\begin{listverse}
\item[\vref{Ps 2:7}] Fils ! Je t'ai e. aujourd'hui.
\item[\vref{Pr 8:24}] J'ai été e. lorsqu'il n'y avait pas encore d'abîmes,
\item[\vref{Es 49:21}] cœur : Qui m'a e. ceux-ci vu que
\item[\vref{Os 5:7}] car ils ont e. des fils étrangers ;
\item[\vref{1 Co 4:15}] qui vs. ai e. en Jésus-Christ par
\item[\vref{Ja 1:18}] Il ns. a e. de sa propre
\end{listverse}

\ConcordanceEntry{Engloutir}
\vspace{-2mm}
\begin{listverse}
\item[\vref{Ex 7:12}] la verge d'Aaron e. leurs verges.
\item[\vref{No 26:10}] bouche et les e., ainsi que Koré,
\item[\vref{De 11:6}] bouche et les e., avec leurs maisons
\item[\vref{Job 24:19}] ainsi le scheol e. les pécheurs !
\item[\vref{Ps 124:3}] ils ns. auraient e. tt vivants, qnd
\item[\vref{Pr 1:12}] E.-les tt vifs, com. le scheol ;
\item[\vref{La 2:16}] disent : Nous l'avons e. ! C'est ici le
\item[\vref{Jon 2:1}] un grand poisson d'e. Jonas, et Jonas
\item[\vref{Hé 11:29}] tenter, ils furent e. ds les eaux.
\item[\vref{Ap 12:16}] bouche, et elle e. le fleuve que
\end{listverse}

\ConcordanceEntry{En-Guédi}
\vspace{-2mm}
\begin{listverse}
\item[\vref{Jos 15:62}] Nibschan, Ir-Hammélach, et E. : Six villes et
\item[\vref{1 S 24:1}] et demeura ds les lieux forts d'E.
\end{listverse}

\ConcordanceEntry{Enigme}
\vspace{-2mm}
\begin{listverse}
\item[\vref{Jg 14:12}] vs. propose une é.. Si vs. me
\item[\vref{1 R 10:1}] de Yahweh, vint l'éprouver par des é.
\item[\vref{2 Ch 9:1}] Salomon par des é.. Elle avait une
\item[\vref{Ps 78:2}] je proférerai les é. cachées des temps
\item[\vref{Pr 1:6}] paraboles et les é. ; les discours des
\item[\vref{Ez 17:2}] l'hom., propose une é., une parabole à
\item[\vref{Ha 2:6}] de railleries et d'é. ? Et ne dira-t-on
\end{listverse}

\ConcordanceEntry{Enivrer}
\vspace{-2mm}
\begin{listverse}
\item[\vref{Es 56:12}] et ns. ns. e. de boissons fortes !
\item[\vref{Es 63:6}] je les ai e. ds ma fureur ;
\item[\vref{Jé 46:10}] se rassasie, elle s'e. de lr. sang.
\item[\vref{Jé 51:57}] J'e. ses princes et ses sages, ses
\item[\vref{Ha 2:15}] bouteille, et qui l'e. afin qu'on voie
\item[\vref{Lu 12:45}] à manger, à boire et à s'e.,
\item[\vref{Jn 2:10}] après qu'on s'est e. ; mais toi, tu
\item[\vref{Ep 5:18}] Et ne vs. e. pas du vin
\item[\vref{1 Th 5:7}] et ceux qui s'e., s'enivrent la nuit.
\item[\vref{Ap 17:2}] terre ont été e. du vin de
\end{listverse}

\ConcordanceEntry{Enlever}
\vspace{-2mm}
\begin{listverse}
\item[\vref{Ex 21:16}] Si quelqu'un e. un hom. et
\item[\vref{2 R 2:9}] que je sois e. d'avec toi. Elisée
\item[\vref{Job 1:21}] donné, Yahweh a e. ; que le nom
\item[\vref{Ps 26:9}] N'e. pas mon âme avec les pécheurs,
\item[\vref{Ps 102:25}] Mon Dieu, ne m'e. pas au milieu
\item[\vref{Ez 8:3}] ma tête. L'Esprit m'e. entre la terre
\item[\vref{Za 3:4}] Josué : Regarde, je t'e. ton iniquité, et
\item[\vref{Mt 9:15}] l'époux lr. sera e., alors ils jeûneront.
\item[\vref{Mc 4:15}] Satan vient et e. la parole qui
\item[\vref{Mc 16:19}] la sorte, fut e. au ciel, et
\item[\vref{Jn 6:15}] qu'ils allaient venir l'e. pour le faire
\item[\vref{Jn 20:2}] dit : On a e. le Seign. hors
\item[\vref{Ac 1:11}] qui a été e. du milieu de
\item[\vref{Ac 1:22}] il a été e. du milieu de
\item[\vref{Ac 8:39}] l'Esprit du Seign. e. Philippe, et l'eunuque
\item[\vref{Col 2:18}] personne ne vs. e. à son gré
\item[\vref{1 Th 4:17}] qui resterons, serons e. ensemble avec eux
\item[\vref{Hé 11:5}] foi, Hénoc fut e. pour ne pas
\item[\vref{Ap 12:5}] son enfant fut e. vers Dieu et
\end{listverse}

\ConcordanceEntry{Ennemi}
\vspace{-2mm}
\begin{listverse}
\item[\vref{Ge 14:20}] a livré tes e. entre tes mains.
\item[\vref{Ge 49:8}] nuque de tes e. ; les fils de
\item[\vref{Ex 23:22}] dirai, je serai l'e. de tes ennemis,
\item[\vref{Lé 26:25}] serez livrés entre les mains de l'e.
\item[\vref{No 10:35}] Yahweh, et tes e. seront dispersés, et
\item[\vref{Jos 5:13}] Es-tu des nôtres ou de nos e. ?
\item[\vref{Jg 5:31}] périssent ts tes e. ô Yahweh ! Et
\item[\vref{1 R 21:20}] M'as-tu trouvé mon e. ? Mais il lui
\item[\vref{Ps 13:5}] peur que mon e. ne dise : J'ai
\item[\vref{Ps 27:2}] adversaires et mes e. qui chancellent et
\item[\vref{Ps 68:2}] lève, et ses e. seront dispersés, et
\item[\vref{Ps 74:18}] de ceci : Que l'e. a blasphémé Yahweh,
\item[\vref{Ps 143:3}] Car l'e. poursuit mon âme, il foule ma
\item[\vref{Pr 16:7}] il apaise envers lui mm ses e.
\item[\vref{Es 63:10}] est devenu lr. e., et il a
\item[\vref{La 4:12}] cru que l'adversaire, l'e. entrerait ds les
\item[\vref{Mi 7:8}] Toi, mon e., ne te réjouis pas à mon
\item[\vref{Mt 5:44}] dis : Aimez vos e., bénissez ceux qui
\item[\vref{Mt 10:36}] propres domestiques d'un hom. seront ses e.
\item[\vref{Mt 13:25}] hommes dormaient, son e. vint et sema
\item[\vref{Ac 13:10}] fils du diable, e. de tte justice,
\item[\vref{Ro 5:10}] lorsque ns. étions e., ns. avons été
\item[\vref{Ro 11:28}] Ils sont certes e. par rapport à
\item[\vref{1 Co 15:26}] L'e. qui sera détruit le dernier c'est
\item[\vref{Ga 4:16}] dc devenu votre e. en vs. disant
\item[\vref{Ph 3:18}] pleurant, qu'ils sont e. de la croix
\item[\vref{1 Th 2:15}] et qui sont e. de ts les
\item[\vref{2 Th 3:15}] pas com. un e., mais avertissez-le com.
\item[\vref{Hé 1:13}] j'aie mis tes e. pour le marchepied
\item[\vref{Ja 4:4}] être ami du monde, se rend e. de Dieu.
\end{listverse}

\ConcordanceEntry{Enraciner}
\vspace{-2mm}
\begin{listverse}
\item[\vref{Ep 3:17}] foi ; afin qu'étant e. et fondés ds
\item[\vref{Col 2:7}] étant e. et édifiés en lui, et affermis
\end{listverse}

\ConcordanceEntry{Enseignement}
\vspace{-2mm}
\begin{listverse}
\item[\vref{De 32:2}] Que mon e. tombe com. la pluie, que ma
\item[\vref{Pr 1:8}] et n'abandonne pas l'e. de ta mère.
\item[\vref{Pr 3:1}] en oubli mon e., et que ton
\item[\vref{Pr 4:2}] doctrine, ne rejetez dc pas mon e.
\item[\vref{Pr 6:20}] et n'abandonne pas l'e. de ta mère ;
\item[\vref{Pr 6:23}] une lampe ; et l'e. une lumière ; et
\item[\vref{Pr 13:14}] L'e. du sage est une source de
\item[\vref{Mc 4:2}] et il lr. dit ds son e. :
\item[\vref{Mc 12:38}] disait ds son e. : Gardez-vs. des scribes
\item[\vref{2 Th 2:15}] et retenez les e. que vs. avez
\item[\vref{2 Th 3:6}] non selon les e. qu'il a reçus
\item[\vref{1 Ti 4:13}] l'exhortation et à l'e., jusqu'à ce que
\item[\vref{1 Ti 5:17}] travaillent à la prédication et à l'e.
\item[\vref{1 Jn 2:27}] lui selon les e. qu'elle vs. a
\end{listverse}

\ConcordanceEntry{Enseigner}
\vspace{-2mm}
\begin{listverse}
\item[\vref{Ex 4:12}] bouche et je t'e. ce que tu
\item[\vref{Ex 35:34}] de la tribu de Dan, de l'e.
\item[\vref{De 4:9}] de ton cœur ; e.-les à tes
\item[\vref{De 33:10}] Ils e. tes ordonnances à Jacob, et ta
\item[\vref{Job 36:22}] puissance ; qui saurait e. com. lui ?
\item[\vref{Ps 51:15}] J'e. tes voies aux transgresseurs et les
\item[\vref{Ps 119:12}] tu es béni ; e.-moi tes statuts.
\item[\vref{Ps 119:66}] E.-moi le bon sens et la
\item[\vref{Pr 8:6}] mes lèvres pour e. des choses droites.
\item[\vref{Es 40:13}] qui a été son conseiller pour l'e. ?
\item[\vref{Jé 32:33}] je les ai e., je les ai
\item[\vref{Mi 4:2}] et il ns. e. sur ses voies,
\item[\vref{Mt 11:1}] là, pour aller e. et prêcher ds
\item[\vref{Mt 22:16}] véritable, que tu e. la voie de
\item[\vref{Mt 28:20}] Et e.-lr. à garder tt ce que
\item[\vref{Mc 4:1}] de nouveau à e. près de la
\item[\vref{Lu 12:12}] le Saint-Esprit vs. e. à l'heure mm
\item[\vref{Jn 6:45}] Ils seront ts e. de Dieu. Donc
\item[\vref{Jn 14:26}] mon Nom, vs. e. ttes choses, et
\item[\vref{Jn 15:3}] la parole que je vs. ai e.
\item[\vref{Ac 5:28}] pas défendu expressément d'e. en ce Nom-là ?
\item[\vref{Ac 5:42}] ils ne cessaient d'e., et d'annoncer l'Evangile
\item[\vref{Ro 2:21}] toi dc, qui e. les autres, tu
\item[\vref{Ro 12:7}] est appelé à e., qu'il enseigne ;
\item[\vref{1 Co 2:13}] la sagesse humaine e., mais avec celles
\item[\vref{1 Co 11:32}] jugés, ns. sommes e. par le Seign.,
\item[\vref{Ga 6:6}] à qui l'on e. la parole soit
\item[\vref{1 Ti 1:3}] de ne pas e. une autre doctrine,
\item[\vref{1 Ti 2:12}] à la fem. d'e. ni d'user d'autorité
\item[\vref{1 Ti 4:11}] Déclare ces choses et e.-les.
\item[\vref{2 Ti 2:2}] capables de les e. aussi à d'autres.
\item[\vref{2 Ti 3:16}] et utile pour e., pour convaincre, pour
\item[\vref{Hé 5:12}] besoin qu'on vs. e. quels sont les
\item[\vref{Hé 8:11}] Personne n'e. plus son prochain,
\item[\vref{1 Jn 2:27}] besoin qu'on vs. e. ; mais com. la
\item[\vref{Ap 2:20}] se dit prophétesse, e. et séduire mes
\end{listverse}

\ConcordanceEntry{Ensevelir}
\vspace{-2mm}
\begin{listverse}
\item[\vref{Ec 8:10}] vu les méchants e. et s'en aller ;
\item[\vref{Mt 8:21}] permets-moi d'aller d'abord e. mon père.
\item[\vref{Lu 9:60}] Laisse les morts e. leurs morts ; mais
\item[\vref{Lu 16:22}] riche mourut aussi, et il fut e.
\item[\vref{Jn 19:40}] aromates, com. les Juifs ont coutume d'e.
\item[\vref{Ac 2:29}] qu'il a été e., et que son
\item[\vref{Ac 5:10}] l'emportèrent dehors, et l'e. auprès de son
\item[\vref{Ro 6:4}] avons dc été e. avec lui par
\item[\vref{1 Co 15:4}] qu'il a été e., et qu'il est
\item[\vref{Col 2:12}] Etant e. avec lui par le baptême, en
\end{listverse}

\ConcordanceEntry{Entendre}
\vspace{-2mm}
\begin{listverse}
\item[\vref{Ge 3:10}] Il répondit : J'ai e. ta voix ds
\item[\vref{Ge 16:11}] car Yahweh a e. ton affliction.
\item[\vref{Ge 21:17}] Dieu e. la voix de l'enfant, et l'Ange
\item[\vref{Ex 3:7}] Egypte et j'ai e. le cri qu'ils
\item[\vref{Ex 32:17}] Et Josué, e. la voix du
\item[\vref{No 9:8}] dit : Arrêtez-vs., et j'e. ce que Yahweh
\item[\vref{De 4:33}] qu'un peuple a e. la voix de
\item[\vref{De 13:11}] que tt Israël e. et craigne, et
\item[\vref{De 18:16}] disais : Que je n'e. plus la voix
\item[\vref{Jos 6:10}] ne ferez point e. votre voix. Et
\item[\vref{1 R 4:34}] les peuples pour e. la sagesse de
\item[\vref{Né 8:2}] qui étaient capables d'e., afin qu'on l'écoutât.
\item[\vref{Job 42:5}] J'avais e. de mes oreilles parler de toi ;
\item[\vref{Ps 5:4}] le matin tu e. ma voix, dès
\item[\vref{Ps 31:14}] Car j'e. les calomnies de plusieurs, la crainte
\item[\vref{Ps 76:9}] Tu fais e. des cieux le jugement ; la terre
\item[\vref{Ps 81:6}] pays d'Egypte, où j'e. un langage que
\item[\vref{Ps 143:8}] Fais-moi e. dès le matin ta miséricorde ! Car
\item[\vref{Ps 145:19}] le craignent, il e. lr. cri et
\item[\vref{Pr 1:24}] que vs. refusez d'e. ; parce que j'étends
\item[\vref{Pr 21:13}] pour ne pas e. le cri du
\item[\vref{Ec 1:8}] et l'oreille ne se lasse pas d'e.
\item[\vref{Ec 12:11}] il a fait e., il a recherché
\item[\vref{Es 30:19}] grâce dès qu'il e. ton cri ; dès
\item[\vref{Es 42:9}] vs. les fais e. avant qu'elles arrivent.
\item[\vref{Es 43:12}] vs. ai fait e. l'avenir, qnd il
\item[\vref{Es 66:8}] Qui a jamais e. une telle chose ?
\item[\vref{Es 66:19}] qui n'ont point e. ma renommée, et
\item[\vref{Ez 12:2}] des oreilles pour e., et n'entendent point ;
\item[\vref{Da 12:8}] J'e., mais je ne compris pas ; et
\item[\vref{Mt 11:4}] choses que vs. e. et que vs.
\item[\vref{Mt 12:19}] pas et personne n'e. sa voix ds
\item[\vref{Mt 13:13}] pas, et qu'en e. ils n'entendent pas
\item[\vref{Mt 13:43}] des oreilles pour e., qu'il entende !
\item[\vref{Mc 4:24}] ce que vs. e.. De la mesure
\item[\vref{Jn 8:26}] choses que j'ai e. de lui, je
\item[\vref{Jn 10:3}] ouvre, les brebis e. sa voix, il
\item[\vref{Jn 10:16}] les amène ; elles e. ma voix, et
\item[\vref{Ac 9:4}] terre et il e. une voix qui
\item[\vref{Ac 10:36}] qu'il a fait e. aux enfants d'Israël,
\item[\vref{Ro 10:17}] de ce qu'on e., et ce qu'on
\item[\vref{2 Co 12:4}] paradis, et a e. des secrets qu'il
\item[\vref{1 Jn 1:1}] que ns. avons e., ce que ns.
\item[\vref{Ap 1:10}] du Seign., et j'e. derrière moi une
\item[\vref{Ap 3:20}] frappe. Si quelqu'un e. ma voix et
\item[\vref{Ap 22:17}] que celui qui e. dise : Viens ! Et
\end{listverse}

\ConcordanceEntry{Entourer}
\vspace{-2mm}
\begin{listverse}
\item[\vref{De 32:10}] solitude, il l'a e., il l'a dirigé,
\item[\vref{Job 26:10}] Il a e. les eaux avec des bornes jusqu'à
\item[\vref{Ps 18:6}] du scheol m'avaient e., les filets de
\item[\vref{Ps 22:17}] assemblée de méchants m'e., ils ont percé
\item[\vref{Ps 48:13}] E. Sion, faites-en le tour, comptez ses
\item[\vref{Ez 5:7}] nations qui vs. e., et que vs.
\item[\vref{Da 9:16}] opprobre à ts ceux qui ns. e.
\item[\vref{Os 7:2}] leurs œuvres les e., elles sont dvt
\item[\vref{Lu 8:45}] la foule qui t'e. te presse, et
\item[\vref{Jn 10:24}] Et les Juifs l'e., et lui dirent :
\end{listverse}

\ConcordanceEntry{Entrailles}
\vspace{-2mm}
\begin{listverse}
\item[\vref{Ge 15:4}] sortira de tes e. sera ton héritier.
\item[\vref{De 7:13}] fruit de tes e., et le fruit
\item[\vref{De 28:53}] fruit de tes e., la chair de
\item[\vref{2 S 7:12}] sorti de tes e. et j'affermirai son
\item[\vref{1 R 3:26}] vivant sentit ses e. s'émouvoir pour son
\item[\vref{Ps 71:6}] tiré hors des e. de ma mère.
\item[\vref{Ps 132:11}] fruit de tes e. sur ton trône.
\item[\vref{Pr 12:10}] bête, mais les e. des méchants sont
\item[\vref{Es 63:15}] son de tes e. et de tes
\end{listverse}

\ConcordanceEntry{Entrée}
\vspace{-2mm}
\begin{listverse}
\item[\vref{Ex 26:36}] Et à l'e. du tabernacle, tu feras un rideau
\item[\vref{Ex 40:29}] de l'holoc. à l'e. du tabernacle de
\item[\vref{Pr 1:21}] de bruit, aux e. des portes, elle
\item[\vref{Mt 27:60}] grande pierre à l'e. du sépulcre et
\item[\vref{2 Pi 1:11}] par ce moyen, l'e. au Royaume éternel
\end{listverse}

\ConcordanceEntry{Entreprise}
\vspace{-2mm}
\begin{listverse}
\item[\vref{Jos 1:8}] succès ds tes e., c'est alors que
\item[\vref{2 R 18:7}] ds ttes ses e.. Il se révolta
\item[\vref{Job 5:12}] viennent pas à bout de leurs e. ;
\item[\vref{Da 8:24}] réussira ds ses e., et il détruira
\item[\vref{Ac 5:38}] Car si cette e. ou cette œuvre
\end{listverse}

\ConcordanceEntry{Entrer}
\vspace{-2mm}
\begin{listverse}
\item[\vref{Ge 6:18}] toi ; et tu e. ds l'arche toi
\item[\vref{Ex 14:22}] les enfants d'Israël e. au milieu de
\item[\vref{Ex 40:35}] Moïse ne put e. ds la tente
\item[\vref{Lé 16:2}] et dis-lui qu'il n'e. point en tt
\item[\vref{De 6:10}] Dieu, te fera e. ds le pays
\item[\vref{De 31:23}] toi qui feras e. les enfants d'Israël
\item[\vref{Jos 1:2}] ce peuple, pour e. ds le pays
\item[\vref{Ps 24:7}] éternelles, et le Roi de gloire e. !
\item[\vref{Mal 3:1}] moi. Et soudain e. ds son temple
\item[\vref{Mt 7:13}] E. par la porte étroite. Car c'est
\item[\vref{Mt 15:11}] pas ce qui e. ds la bouche
\item[\vref{Mt 23:13}] car vs.-mêmes n'y e. pas, et vs.
\item[\vref{Lu 13:24}] Efforcez-vs. d'e. par la porte
\item[\vref{Lu 14:23}] tu trouveras, contrains-les d'e., afin que ma
\item[\vref{Lu 22:3}] Mais Satan e. ds Judas, surnommé
\item[\vref{Jn 3:5}] il ne peut e. ds le Royaume
\item[\vref{Jn 10:2}] Mais celui qui e. par la porte
\item[\vref{Jn 10:9}] porte. Si quelqu'un e. par moi, il
\item[\vref{Ac 14:22}] qu'il ns. faut e. ds le Royaume
\item[\vref{Ro 5:12}] le péché est e. ds le monde,
\item[\vref{Hé 3:19}] ne purent y e. à cause de
\item[\vref{Hé 10:19}] avons la liberté d'e. ds le Saint
\item[\vref{Ap 21:27}] Il n'e. chez elle rien de souillé, ni
\item[\vref{Ap 22:14}] de vie, et d'e. par les portes
\end{listverse}

\ConcordanceEntry{Envie}
\vspace{-2mm}
\begin{listverse}
\item[\vref{Ge 37:11}] frères eurent de l'e. contre lui, mais
\item[\vref{Ps 73:3}] car j'ai porté e. aux insensés, en
\item[\vref{Pr 3:31}] Ne porte pas e. à l'hom. violent,
\item[\vref{Pr 14:30}] la chair, mais l'e. est la pourriture
\item[\vref{Pr 23:17}] ne porte pas d'e. aux pécheurs, mais
\item[\vref{Pr 24:1}] Ne porte pas e. aux hommes malins,
\item[\vref{Ec 8:11}] en eux de l'e. de faire le
\item[\vref{Ec 9:6}] haine, et lr. e. ont déjà péri,
\item[\vref{Mt 27:18}] savait bien qu'ils l'avaient livré par e.
\item[\vref{Ro 1:29}] de malignité, pleins d'e., de meurtre, de
\item[\vref{Ga 5:26}] en ns. portant e. les uns aux
\item[\vref{Ph 1:15}] prêchent Christ par e. et par un
\item[\vref{1 Ti 6:4}] mots, d'où naissent l'e., les querelles, les
\item[\vref{Tit 3:3}] méchanceté et ds l'e., dignes d'être haïs,
\item[\vref{Ja 3:14}] vs. avez une e. amère et un
\item[\vref{Ja 4:2}] vs. avez une e. mortelle, vs. êtes
\item[\vref{1 Pi 2:1}] fraude, de dissimulation, d'e. et de tte
\end{listverse}

\ConcordanceEntry{Envier}
\vspace{-2mm}
\begin{listverse}
\item[\vref{Ja 4:2}] vs. avez une e. mortelle, vs. êtes
\end{listverse}

\ConcordanceEntry{Envoyé (un)}
\vspace{-2mm}
\begin{listverse}
\item[\vref{2 Co 8:23}] ils sont les e. des églises et
\end{listverse}

\ConcordanceEntry{Envoyer}
\vspace{-2mm}
\begin{listverse}
\item[\vref{Ge 45:5}] car Dieu m'a e. dvt vs. pour
\item[\vref{Ex 3:13}] vos pères m'a e. vers vs., s'ils
\item[\vref{Ex 4:13}] répondit : Hélas ! Seign. ! E., je te prie,
\item[\vref{Ex 23:20}] Voici, j'e. un Ange dvt toi, afin qu'il
\item[\vref{Ex 33:12}] que tu dois e. avec moi ; tu
\item[\vref{Jg 6:14}] main des Madianites ; ne t'ai-je pas e. ?
\item[\vref{Ps 107:20}] Il e. sa parole, et les guérit ; et
\item[\vref{Es 6:8}] Seign., disant : Qui e.-je et qui
\item[\vref{Es 9:7}] Le Seign. e. une parole à
\item[\vref{Jé 26:15}] vérité Yahweh m'a e. vers vs. pour
\item[\vref{Jé 44:4}] je vs. ai e. ts mes serviteurs,
\item[\vref{Mi 6:4}] servitude, et j'ai e. dvt toi Moïse,
\item[\vref{Mt 15:24}] Je n'ai été e. qu'aux brebis perdues
\item[\vref{Mc 6:7}] commença à les e. deux à deux,
\item[\vref{Lu 9:48}] celui qui m'a e.. Car celui qui
\item[\vref{Lu 9:52}] Il e. dvt lui des messagers, qui se
\item[\vref{Lu 24:49}] Et voici, j'e. sur vs. la
\item[\vref{Jn 1:6}] un hom. appelé Jean, qui fut e. de Dieu.
\item[\vref{Jn 8:29}] celui qui m'a e. est avec moi ;
\item[\vref{Jn 16:7}] si je m'en vais, je vs. l'e.
\item[\vref{Jn 20:21}] mon Père m'a e., ainsi je vs.
\item[\vref{Ac 3:26}] Fils Jésus, l'a e. pour vs. bénir,
\item[\vref{Ac 10:20}] car c'est moi qui les ai e.
\item[\vref{Ac 22:21}] Va, car je t'e. au loin vers
\item[\vref{Ro 10:15}] ne sont pas e. ? Selon qu'il est
\item[\vref{Ga 4:6}] fils, Dieu a e. l'Esprit de son
\item[\vref{1 Pi 1:12}] par le Saint-Esprit e. du ciel, et
\item[\vref{1 Jn 4:14}] le Père a e. le Fils pour
\item[\vref{Ap 22:16}] Moi, Jésus, j'ai e. mon ange pour
\end{listverse}

\ConcordanceEntry{Epaphras}
\vspace{-2mm}
\begin{listverse}
\item[\vref{Col 1:7}] été instruits par E., notre cher compagnon
\item[\vref{Col 4:12}] E., qui est des vôtres, et serviteur
\item[\vref{Phm 1:23}] E., qui est prisonnier avec moi en
\end{listverse}

\ConcordanceEntry{Epaphrodite}
\vspace{-2mm}
\begin{listverse}
\item[\vref{Ph 2:25}] de vs. envoyer E., mon frère, mon
\item[\vref{Ph 4:18}] biens en recevant d'E. ce qui vient
\end{listverse}

\ConcordanceEntry{Epargner}
\vspace{-2mm}
\begin{listverse}
\item[\vref{Ge 19:16}] que Yahweh voulait l'é. ; et ils l'emmenèrent
\item[\vref{1 S 15:9}] et le peuple é. Agag, les meilleures
\item[\vref{1 S 24:11}] mais je t'ai é., et j'ai dit :
\item[\vref{2 Ch 36:15}] car il voulait é. son peuple et
\item[\vref{Pr 11:24}] et tel qui é. outre mesure, n'aura
\item[\vref{Joë 2:17}] qu'ils disent : Yahweh ! E. ton peuple ! N'expose
\item[\vref{Jon 4:11}] Et moi, n'é.-je point Ninive,
\item[\vref{Ac 20:29}] très dangereux, qui n'é. pas le troupeau,
\item[\vref{Ro 8:32}] qui n'a pas é. son propre Fils,
\item[\vref{1 Co 7:28}] chair ; or je voudrais vs. les é.
\item[\vref{Col 2:23}] en ce qu'elles n'é. pas le corps,
\item[\vref{2 Pi 2:4}] Dieu n'a pas é. les anges qui
\end{listverse}

\ConcordanceEntry{Epaule}
\vspace{-2mm}
\begin{listverse}
\item[\vref{Ge 9:23}] sur leurs deux é., et marchant à
\item[\vref{Ge 24:15}] sa cruche sur l'é., Rebecca, fille de
\item[\vref{Ge 49:15}] il courbe son é. sous le fardeau,
\item[\vref{Lé 7:33}] de paix aura pour sa part l'é. droite.
\item[\vref{De 33:12}] protégera toujours, et demeurera entre ses é.
\item[\vref{Jos 4:5}] pierre sur son é., selon le nombre
\item[\vref{Jg 16:3}] mit sur ses é. et les porta
\item[\vref{1 Ch 15:15}] Dieu sur leurs é., avec les barres
\item[\vref{Es 9:5}] reposera sur son é. : On l'appellera l'Admirable,
\item[\vref{Es 10:27}] de dessus ton é. et son joug
\item[\vref{Es 22:22}] David sur son é. ; et il ouvrira,
\item[\vref{Ez 12:6}] porteras sur tes é., sous leurs yeux,
\item[\vref{Mt 23:4}] mettent sur les é. des hommes, mais
\item[\vref{Lu 15:5}] la met avec joie sur ses é.,
\end{listverse}

\ConcordanceEntry{Epée}
\vspace{-2mm}
\begin{listverse}
\item[\vref{Ge 3:24}] et là une é. flamboyante pour garder
\item[\vref{No 22:23}] le chemin, son é. nue ds la
\item[\vref{De 32:41}] l'éclair de mon é., et si ma
\item[\vref{Jos 5:13}] qui avait son é. nue à la
\item[\vref{Jg 7:20}] et ils s'écrièrent : L'é. de Yahweh et
\item[\vref{1 S 13:22}] combat, nul n'avait d'é. ni de lance
\item[\vref{1 S 17:47}] délivre pas par l'é. ni par la
\item[\vref{2 S 12:10}] Maintenant, l'é. ne s'éloignera jamais
\item[\vref{1 Ch 10:4}] armes : Tire ton é., et transperce-moi, de
\item[\vref{Né 4:18}] bâtissaient avait son é. ceinte autour des
\item[\vref{Job 33:18}] la fosse, et sa vie de l'é.
\item[\vref{Ps 55:22}] l'huile, néanmoins elles sont tt autant d'é. nues.
\item[\vref{Es 2:4}] peuples. De leurs é. ils forgeront des
\item[\vref{Es 27:1}] grande et forte é. le Léviathan, le
\item[\vref{Jé 5:12}] ne verrons ni l'é. ni la famine.
\item[\vref{Jé 34:17}] contre vs. à l'é., à la peste,
\item[\vref{Jé 47:6}] Ah ! E. de Yahweh, qnd te reposeras-tu ? Rentre
\item[\vref{Jé 51:50}] avez échappé à l'é., allez, ne vs.
\item[\vref{Ez 7:15}] L'é. est au-dehors, la peste et la
\item[\vref{Za 11:17}] les brebis ! Que l'é. fonde sur son
\item[\vref{Za 13:7}] E., réveille-toi contre mon Berger, et sur
\item[\vref{Mt 10:34}] pas venu apporter la paix, mais l'é.
\item[\vref{Mt 26:52}] dit : Remets ton é. à sa place ;
\item[\vref{Lu 2:35}] pour toi, une é. te transpercera l'âme.
\item[\vref{Lu 22:38}] voici ici deux é.. Et il lr.
\item[\vref{Jn 18:11}] Pierre : Remets ton é. ds le fourreau.
\item[\vref{Ac 12:2}] fit mourir par l'é. Jacques, frère de
\item[\vref{Ro 8:35}] la nudité, ou le péril, ou l'é. ?
\item[\vref{Ro 13:4}] vain qu'il porte l'é., étant serviteur de
\item[\vref{Ep 6:17}] du salut, et l'é. de l'Esprit, qui
\item[\vref{Hé 4:12}] plus pénétrante qu'aucune é. à deux tranchants,
\item[\vref{Hé 11:37}] le tranchant de l'é., ils errèrent çà
\item[\vref{Ap 1:16}] bouche sortait une é. aiguë à deux
\item[\vref{Ap 6:8}] pour tuer par l'é., par la famine,
\item[\vref{Ap 13:14}] coup mortel de l'é., et qui était
\item[\vref{Ap 19:15}] bouche sortait une é. tranchante, pour frapper
\end{listverse}

\ConcordanceEntry{Epha}
\vspace{-2mm}
\begin{listverse}
\item[\vref{Ex 16:36}] omer est la dixième partie d'un é.
\item[\vref{De 25:14}] ta maison deux é. différents, un grand
\item[\vref{Jg 6:19}] fit avec un é. de farine des
\item[\vref{Am 8:5}] vente, en faisant l'é. plus petit, en
\item[\vref{Za 5:7}] fem. était assise au milieu de l'é.
\end{listverse}

\ConcordanceEntry{Ephèse}
\vspace{-2mm}
\begin{listverse}
\item[\vref{Ac 18:19}] Ils arrivèrent à E., et Paul y
\item[\vref{Ac 19:35}] que la ville d'E. est la gardienne
\item[\vref{Ac 20:17}] envoya chercher à E. les anciens de
\item[\vref{1 Co 15:32}] les bêtes à E. ds des vues
\item[\vref{Ap 1:11}] à savoir à E., à Smyrne, à
\item[\vref{Ap 2:1}] l'ange de l'église d'E. : Voici ce que
\end{listverse}

\ConcordanceEntry{Ephod}
\vspace{-2mm}
\begin{listverse}
\item[\vref{Ex 28:4}] feront : Le pectoral, l'é., la robe, la
\item[\vref{Ex 28:31}] la robe de l'é. entièrement de pourpre.
\item[\vref{Jg 8:27}] en fit un é., et le mit
\item[\vref{Jg 17:5}] il fit un é. et des téraphim,
\item[\vref{1 S 2:18}] Yahweh, étant jeune garçon, vêtu d'un é. de lin.
\item[\vref{1 S 23:9}] égard, dit au prêtre Abiathar : Apporte l'é. !
\item[\vref{2 S 6:14}] Yahweh, et il était ceint d'un é. de lin.
\item[\vref{Os 3:4}] sans statue, sans é., et sans théraphim.
\end{listverse}

\ConcordanceEntry{Ephraïm}
\vspace{-2mm}
\begin{listverse}
\item[\vref{Ge 41:52}] second le nom d'E., parce que, dit-il,
\item[\vref{Ge 48:5}] seront à moi : E. et Manassé seront
\item[\vref{Ge 48:20}] te traite com. E. et com. Manassé !
\item[\vref{No 1:33}] de la tribu d'E., qui furent dénombrés,
\item[\vref{No 26:35}] Voici les fils d'E., selon leurs familles :
\item[\vref{Jos 16:5}] frontière des fils d'E., selon leurs familles,
\item[\vref{Jé 31:18}] très bien entendu E. se plaignant, et
\item[\vref{Os 4:17}] E. s'est associé aux idoles ; abandonne-le !
\item[\vref{Os 5:3}] sais qui est E., et Israël ne
\item[\vref{Os 9:3}] pays de Yahweh ; E. retournera en Egypte,
\item[\vref{Za 9:10}] Je retrancherai d'E. les chars et
\end{listverse}

\ConcordanceEntry{Ephrata}
\vspace{-2mm}
\begin{listverse}
\item[\vref{Ge 35:16}] certaine distance jusqu'à E. lorsque Rachel accoucha.
\item[\vref{1 Ch 4:4}] de Hur, premier-né d'E., père de Bethléhem.
\item[\vref{Ps 132:6}] parler d'elle à E., ns. l'avons trouvée
\item[\vref{Mi 5:1}] Mais toi, Bethléhem E., petite pour être
\end{listverse}

\ConcordanceEntry{Epi}
\vspace{-2mm}
\begin{listverse}
\item[\vref{Ge 41:5}] songe. Voici, sept é. gras et beaux
\item[\vref{Ex 34:18}] saison où les é. mûrissent ; car c'est
\item[\vref{De 23:25}] pourras arracher des é. avec ta main ;
\item[\vref{Ru 2:2}] aller glaner des é. ds le champ
\item[\vref{Ps 72:16}] montagnes, et leurs é. s'agiteront com. les
\item[\vref{Mt 12:1}] à arracher des é. et à les
\item[\vref{Mc 4:28}] premièrement l'herbe, ensuite l'é., et puis le
\end{listverse}

\ConcordanceEntry{Epine}
\vspace{-2mm}
\begin{listverse}
\item[\vref{Ge 3:18}] te produira des é. et des chardons ;
\item[\vref{No 33:55}] seront com. des é. à vos yeux,
\item[\vref{Jg 9:14}] arbres dirent à l'é. : Viens, toi, et
\item[\vref{2 S 23:6}] ts com. des é. que l'on jette
\item[\vref{Pr 22:5}] y a des é. et des pièges
\item[\vref{Pr 26:9}] Comme une é. ds la main
\item[\vref{Es 27:4}] des ronces, des é. pour les combattre !
\item[\vref{Jé 4:3}] et ne semez pas parmi les é.
\item[\vref{Ez 2:6}] perçantes com. des é. soient avec toi,
\item[\vref{Os 10:8}] d'Israël, seront détruits ; l'é. et la ronce
\item[\vref{Mt 7:16}] raisins sur des é., ou des figues
\item[\vref{Mt 13:7}] tomba parmi les é. ; les épines montèrent
\item[\vref{Mt 27:29}] fait une couronne d'é. entrelacées, ils la
\item[\vref{Hé 6:8}] qui produit des é. et des chardons,
\end{listverse}

\ConcordanceEntry{Epouse}
\vspace{-2mm}
\begin{listverse}
\item[\vref{Es 62:4}] appellera ta terre l'é. ; car Yahweh prend
\item[\vref{Joë 2:16}] sa demeure, et l'é. de sa chambre
\item[\vref{Jn 3:29}] Celui qui possède l'E. est l'Epoux ; mais
\item[\vref{Ap 19:7}] venues, et son E. s'est préparée.
\item[\vref{Ap 21:9}] je te montrerai l'E., la fem. de
\item[\vref{Ap 22:17}] Et l'Esprit et l'E. disent : Viens ! Et
\end{listverse}

\ConcordanceEntry{Epouvante}
\vspace{-2mm}
\begin{listverse}
\item[\vref{Ps 55:6}] La crainte et l'é. m'atteignent, et le
\item[\vref{Ps 68:31}] E. les bêtes sauvages des roseaux, la
\end{listverse}

\ConcordanceEntry{Epouvanter}
\vspace{-2mm}
\begin{listverse}
\item[\vref{2 S 22:5}] environné, les torrents des méchants m'avaient é. ;
\item[\vref{Job 23:16}] brisé mon cœur, le Tout-Puissant m'a é.
\item[\vref{Ps 68:31}] E. les bêtes sauvages des roseaux, la
\item[\vref{Jé 23:4}] peur, et ne s'é. plus, et il
\item[\vref{Jé 46:27}] crains pas ; ne t'é. pas Israël ! Car
\item[\vref{Mc 16:5}] d'une robe blanche, et elles furent é.
\item[\vref{Lu 21:9}] soulèvements, ne vs. é. pas, car il
\item[\vref{Hé 12:21}] dit : Je suis é. et tt tremblant !
\end{listverse}

\ConcordanceEntry{Epoux}
\vspace{-2mm}
\begin{listverse}
\item[\vref{Ex 4:25}] pour moi un é. de sang !
\item[\vref{Ps 19:6}] semblable à un é. sortant de sa
\item[\vref{Es 54:5}] créateur est ton é. : Yahweh des armées
\item[\vref{Es 62:4}] toi, et ta terre aura un é.
\item[\vref{Joë 2:16}] la mamelle ! Que l'é. sorte de sa
\item[\vref{Mt 9:15}] Les amis de l'é. peuvent-ils s'affliger pendant
\item[\vref{Mt 25:6}] cri, disant : Voici, l'é. vient, allez à
\item[\vref{Jn 3:29}] possède l'Epouse est l'E. ; mais l'ami de
\item[\vref{2 Co 11:2}] à un seul é., pour vs. présenter
\item[\vref{Ap 18:23}] la voix de l'é. et de l'épouse
\end{listverse}

\ConcordanceEntry{Epreuve}
\vspace{-2mm}
\begin{listverse}
\item[\vref{De 13:3}] vs. met à l'é. pour savoir si
\item[\vref{Jg 6:39}] faire encore une é. avec la toison :
\item[\vref{Ro 5:4}] et la persévérance l'é., et l'épreuve l'espérance.
\item[\vref{2 Co 9:13}] Glorifiant Dieu pour l'é. qu'ils font de
\item[\vref{Ph 2:22}] été mis à l'é., puisqu'il a servi
\item[\vref{Ja 1:3}] sachant que l'é. de votre foi
\item[\vref{Ja 5:11}] qui ont enduré l'é. avec patience. Vous
\item[\vref{1 Pi 1:7}] afin que l'é. de votre foi,
\item[\vref{1 Pi 4:12}] fournaise pour votre é., com. s'il vs.
\item[\vref{2 Pi 2:9}] ainsi délivrer de l'é. ceux qui l'honorent,
\end{listverse}

\ConcordanceEntry{Eprouver}
\vspace{-2mm}
\begin{listverse}
\item[\vref{Ge 22:1}] choses, que Dieu é. Abraham et lui
\item[\vref{Ex 15:25}] ordon. et une loi, et il l'é. là,
\item[\vref{Ex 20:20}] venu pour vs. é., et afin que
\item[\vref{De 8:2}] t'humilier et de t'é., pour connaître ce
\item[\vref{Jg 3:1}] Yahweh laissa pour é. par elles Israël,
\item[\vref{Ps 12:7}] c'est un argent é. sur terre au
\item[\vref{Ps 18:31}] de Yahweh est é. ; il est un
\item[\vref{Ps 26:2}] Sonde-moi et é.-moi, Yahweh ! Fais
\item[\vref{Ps 66:10}] tu ns. as é., tu ns. a
\item[\vref{Es 28:16}] pierre, une pierre é., la pierre angulaire
\item[\vref{Jé 17:10}] cœur, et qui é. les reins ; mm
\item[\vref{Za 13:9}] purifie l'argent, je l'é. com. on éprouve
\item[\vref{Mal 3:10}] et dès mntnt é.-moi en ceci,
\item[\vref{Mt 16:1}] lui, et pour l'é., lui demandèrent de
\item[\vref{1 Co 3:13}] et le feu é. de quelle façon
\item[\vref{1 Co 10:13}] ne vs. a é., qui n'ait été
\item[\vref{1 Co 11:28}] Que chacun dc s'é. soi-mm, et ainsi
\item[\vref{2 Co 13:5}] ds la foi ; é.-vs. vs.-mêmes. Ne
\item[\vref{1 Th 5:21}] E. ttes choses ; retenez ce qui est
\item[\vref{Hé 11:17}] foi, Abraham étant é., offrit Isaac ; celui
\item[\vref{1 Jn 4:1}] tt esprit ; mais é. les esprits pour
\item[\vref{Ap 2:2}] que tu as é. ceux qui se
\item[\vref{Ap 3:10}] monde entier, pour é. les habitants de
\end{listverse}

\ConcordanceEntry{Epurer}
\vspace{-2mm}
\begin{listverse}
\item[\vref{Jg 7:4}] là je les é. ; et celui dont
\item[\vref{Ps 12:7}] terre au creuset, et sept fois é.
\item[\vref{Es 48:10}] Voici, je t'ai é., mais non pas
\item[\vref{Da 11:35}] afin qu'ils soient é., purifiés et blanchis,
\item[\vref{Mal 3:3}] Lévi, il les é. com. l'or et
\end{listverse}

\ConcordanceEntry{Equité}
\vspace{-2mm}
\begin{listverse}
\item[\vref{Ps 45:7}] de ton règne est un sceptre d'é.
\item[\vref{Ps 72:2}] ton peuple, et tes malheureux avec é. !
\item[\vref{Ps 89:15}] La justice et l'é. sont la base
\item[\vref{Ps 96:10}] ébranlé ; il jugera les peuples avec é.
\item[\vref{Ps 98:9}] avec justice, et les peuples avec é.
\item[\vref{Pr 1:3}] sens, de justice, de jugement et d'é.
\item[\vref{Pr 2:9}] justice, le jugement, l'é., et tt bon
\item[\vref{Es 32:1}] justice, et les princes gouverneront avec é.
\item[\vref{Jé 10:24}] Châtie-moi, mais avec é., et non ds
\item[\vref{Jé 30:11}] te châtierai avec é., je ne te
\item[\vref{Hé 1:8}] de ton Royaume est un sceptre d'é. ;
\end{listverse}

\ConcordanceEntry{Errer}
\vspace{-2mm}
\begin{listverse}
\item[\vref{No 32:13}] il les fit e. ds le désert
\item[\vref{Ps 55:3}] Ecoute-moi, et réponds-moi ! J'e. çà et là
\item[\vref{Ps 107:40}] et les fait e. ds des lieux
\item[\vref{Ps 119:176}] Je suis e. com. une brebis perdue ; cherche ton
\item[\vref{Es 30:28}] mâchoires des peuples, qui les fera e.
\item[\vref{Es 53:6}] avons ts été e. com. des brebis,
\item[\vref{Jé 4:1}] ne seras plus e. ça et là.
\item[\vref{Jé 14:10}] qu'ils aiment à e. ainsi çà et
\item[\vref{Os 4:12}] les a fait e., et ils ont
\end{listverse}

\ConcordanceEntry{Erreur}
\vspace{-2mm}
\begin{listverse}
\item[\vref{Ps 19:13}] fautes commises par e. ? Purifie-moi de mes
\item[\vref{Ec 10:5}] soleil, com. une e. qui procède du
\item[\vref{Mt 22:29}] Vous êtes ds l'e., parce que vs.
\item[\vref{Mc 12:27}] Vous êtes dc ds une grande e.
\item[\vref{1 Jn 4:6}] l'Esprit de vérité et l'esprit de l'e.
\end{listverse}

\ConcordanceEntry{Esaïe}
\vspace{-2mm}
\begin{listverse}
\item[\vref{Es 1:1}] La vision d'E., fils d'Amots, qu'il
\end{listverse}
\begin{legend}
\NoAutoSpaceBeforeFDP{
\item Prophète : 2 R 19:2; Es 1:1
\item Consulte Yahweh pour Ezéchias : 2 R 19:5-6; 20-21
\item Le remède pour Ezéchias : 2 R 20:7
\item Prophétie sur la captivité Babylonienne : 2 R 20:14-18
\item Prophéties et jugements : Es 13-23; 45-47
\item Annonce la venue du Messie : Es 7:14; 8:23; 9:1; 35:5-6; 52:13; 53:12; 63:1-6
}
\end{legend}

\ConcordanceEntry{Esaü, Edom}
\vspace{-2mm}
\begin{listverse}
\item[\vref{Ge 25:25}] et on lui donna le nom d'E.
\end{listverse}
\begin{legend}
\NoAutoSpaceBeforeFDP{
\item Fils d'Isaac et Rébecca et frère jumeau de Jacob : Ge 25:25-26; Mal 1:2
\item vend son droit d'aînesse et Profane : Ge 25:32-33; Hé 12:16
\item Ancêtre des Edomites : De 2:5; Jos 24:4
\item Autres :  Ge 27:35-36,27:41,33:4; Abd 1:9-15
}
\end{legend}

\ConcordanceEntry{Escarboucle}
\vspace{-2mm}
\begin{listverse}
\item[\vref{Ex 28:18}] seconde rangée, une e., un saphir, et
\item[\vref{Ex 39:11}] seconde rangée une e., un saphir, et
\item[\vref{1 Ch 29:2}] enchâssées, des pierres d'e., et des pierres
\item[\vref{Ez 27:16}] pourvoyait tes marchés d'e., d'écarlate, de broderie,
\item[\vref{Ez 28:13}] jaspe, de saphir, d'e., d'émeraude, et d'or ;
\end{listverse}

\ConcordanceEntry{Esclave}
\vspace{-2mm}
\begin{listverse}
\item[\vref{Ge 41:12}] un garçon Hébreu, e. du chef des
\item[\vref{Ex 21:2}] tu achètes un e. hébreu, il te
\item[\vref{Ex 21:5}] Si l'e. dit franchement: J'aime mon maître, ma
\item[\vref{Lé 25:55}] enfants d'Israël sont e. ; ce sont mes
\item[\vref{De 5:15}] tu as été e. au pays d'Egypte,
\item[\vref{De 23:15}] à son maître l'e. qui se sera
\item[\vref{De 24:18}] tu as été e. en Egypte, et
\item[\vref{Ps 105:17}] dvt eux ; Joseph fut vendu pour e.
\item[\vref{Pr 19:10}] sied-il à un e. de dominer sur
\item[\vref{Pr 22:7}] qui emprunte est l'e. de celui qui
\item[\vref{Jé 2:14}] Israël est-il un e., ou un esclave
\item[\vref{Mc 10:44}] le premier parmi vs., qu'il soit l'e. de ts.
\item[\vref{Lu 12:46}] maître de cet e.-là viendra en
\item[\vref{Jn 8:34}] au péché est e. du péché.
\item[\vref{Jn 8:35}] Or l'e. ne demeure pas toujours ds la
\item[\vref{Ro 6:16}] à quelqu'un com. e. pour lui obéir,
\item[\vref{1 Co 7:22}] Car l'e. qui a été appelé par notre
\item[\vref{1 Co 7:23}] devenez pas les e. des hommes.
\item[\vref{Ga 3:28}] a plus ni e. ni libre, il
\item[\vref{Ga 4:7}] tu n'es plus e., mais fils ; or
\item[\vref{Ep 6:8}] que chacun, soit e., soit libre, recevra
\item[\vref{Phm 1:16}] plus com. un e., mais com. étant
\item[\vref{2 Pi 2:19}] qu'ils sont eux-mêmes e. de la corruption,
\end{listverse}

\ConcordanceEntry{Esdras}
\vspace{-2mm}
\begin{listverse}
\item[\vref{1 Ch 4:17}] Les fils d'E. furent Jéther, Méred,
\item[\vref{Esd 7:1}] roi de Perse, E., fils de Seraja,
\item[\vref{Esd 7:6}] E., dis-je, qui était un scribe bien
\item[\vref{Esd 7:10}] Car E. avait disposé son cœur à étudier
\item[\vref{Esd 7:11}] Artaxerxès donna à E., prêtre et scribe,
\item[\vref{Esd 7:25}] quant à toi, E., établis des magistrats
\item[\vref{Esd 10:1}] Et com. E. priait et faisait cette confession, pleurant
\item[\vref{Né 8:9}] Néhémie, le gouverneur, E., le prêtre et
\item[\vref{Né 12:36}] hom. de Dieu. E., le scribe, marchait
\end{listverse}

\ConcordanceEntry{Espace}
\vspace{-2mm}
\begin{listverse}
\item[\vref{No 22:26}] n'y avait point d'e. pour se détourner
\item[\vref{Job 37:10}] et il réduit l'e. où se répandaient
\item[\vref{Es 5:8}] n'y ait plus d'e. et qu'ils habitent
\item[\vref{Es 54:2}] Elargis l'e. de ta tente,
\item[\vref{Za 10:10}] aura point assez d'e. pour eux.
\item[\vref{Mc 2:2}] personnes, tellement que l'e. mm dvt la
\end{listverse}

\ConcordanceEntry{Espagne}
\vspace{-2mm}
\begin{listverse}
\item[\vref{Ro 15:24}] pour aller en E., et j'espère que
\item[\vref{Ro 15:28}] je partirai pour l'E., en passant par
\end{listverse}

\ConcordanceEntry{Espérance}
\vspace{-2mm}
\begin{listverse}
\item[\vref{Ge 27:42}] se console ds l'e. qu'il a de
\item[\vref{Esd 10:2}] ne reste pas pour cela sans e.
\item[\vref{Job 4:6}] pas été ton e. ? Et l'intégrité de
\item[\vref{Job 14:7}] y a de l'e., et il poussera
\item[\vref{Ps 9:19}] oublié à jamais, l'e. des affligés ne
\item[\vref{Ps 52:11}] je mettrai mon e. en ton Nom,
\item[\vref{Pr 10:28}] L'e. des justes n'est que joie, mais
\item[\vref{Pr 23:18}] fin, et ton e. ne sera pas
\item[\vref{Ec 9:4}] y a de l'e. pour ts ceux
\item[\vref{Jé 14:8}] Toi qui es l'e. d'Israël, son Sauveur
\item[\vref{Jé 17:7}] en Yahweh, et dont Yahweh est l'e. !
\item[\vref{Jé 31:17}] y a de l'e. pour tes derniers
\item[\vref{La 3:21}] cœur, et c'est pourquoi j'aurai de l'e. :
\item[\vref{Os 12:7}] et aie continuellement e. en ton Dieu.
\item[\vref{Ac 24:15}] en Dieu cette e., com. ils l'ont
\item[\vref{Ro 5:2}] ns. glorifions ds l'e. de la gloire
\item[\vref{Ro 5:5}] Or l'e. ne confond pas, parce que l'amour
\item[\vref{Ro 8:24}] Car c'est en e. que ns. sommes
\item[\vref{Ro 15:4}] la consolation des Ecritures, ns. ayons e.
\item[\vref{Ro 15:13}] le Dieu de l'e. vs. remplisse de
\item[\vref{1 Co 9:10}] doit labourer avec e., et celui qui
\item[\vref{2 Co 1:7}] Et l'e. que ns. avons de vs. est
\item[\vref{Ga 5:5}] attendons par l'Esprit l'e. d'être justifiés par
\item[\vref{Ep 1:18}] sachiez quelle est l'e. de sa vocation,
\item[\vref{Ep 2:12}] promesse, n'ayant pas d'e., et étant sans
\item[\vref{Ep 4:4}] à une seule e. par votre vocation ;
\item[\vref{Ph 1:20}] attente et mon e., je ne serai
\item[\vref{Col 1:5}] à cause de l'e. des biens qui
\item[\vref{Col 1:27}] Christ en vs., l'e. de la gloire ;
\item[\vref{1 Th 5:8}] ayant pour casque l'e. du salut.
\item[\vref{2 Th 2:16}] et une bonne e. par sa grâce,
\item[\vref{1 Ti 1:1}] Sauveur, et du Seign. Jésus-Christ, notre e.,
\item[\vref{Tit 2:13}] attendant la bienheureuse e., et l'apparition de
\item[\vref{Hé 3:6}] fin l'assurance, et l'e. qui est notre
\item[\vref{Hé 6:18}] refuge à obtenir l'e. qui ns. est
\item[\vref{1 Pi 1:3}] régénérés pour une e. vivante, par la
\item[\vref{1 Pi 1:21}] foi et votre e. reposent sur Dieu.
\item[\vref{1 Pi 3:15}] une parole concernant l'e. qui est en
\end{listverse}

\ConcordanceEntry{Espérer}
\vspace{-2mm}
\begin{listverse}
\item[\vref{Ge 49:18}] Ô Yahweh ! J'e. en ton salut !
\item[\vref{Job 5:16}] ce qu'il a e., mais l'iniquité a
\item[\vref{Job 13:15}] ne cesserai pas d'e. en lui ; et
\item[\vref{Ps 119:43}] de vérité, car j'e. en tes jugements.
\item[\vref{Es 51:5}] peuples ; les îles e. en moi, elles
\item[\vref{La 3:26}] Il est bon d'e. et d'attendre en
\item[\vref{Ez 13:6}] et ils font e. que lr. parole
\item[\vref{Mt 12:21}] Et les nations e. en son Nom.
\item[\vref{Lu 6:35}] prêtez sans rien e., et votre récompense
\item[\vref{Ro 4:18}] Et Abraham ayant e. contre tte espérance,
\item[\vref{Ro 8:24}] espérance, car ce qu'on voit, peut-on l'e. encore ?
\item[\vref{1 Co 13:7}] croit tt, elle e. tt, elle supporte
\item[\vref{Ep 1:12}] avons les premiers e. en Christ.
\item[\vref{1 Ti 4:10}] parce que ns. e. ds le Dieu
\item[\vref{1 Ti 5:5}] est laissée seule, e. en Dieu, et
\item[\vref{Hé 11:1}] les choses qu'on e., et elle est
\end{listverse}

\ConcordanceEntry{Espion}
\vspace{-2mm}
\begin{listverse}
\item[\vref{Ge 42:9}] Vous êtes des e., vs. êtes venus
\item[\vref{Jg 1:24}] Les e. virent un hom. qui sortait de
\item[\vref{1 S 26:4}] il envoya des e., et apprit avec
\item[\vref{2 S 15:10}] Absalom envoya des e. ds ttes les
\item[\vref{Lu 20:20}] ils envoyèrent des e., qui feignaient être
\item[\vref{Hé 11:31}] avait reçu les e. et les avait
\end{listverse}

\ConcordanceEntry{Espoir}
\vspace{-2mm}
\begin{listverse}
\item[\vref{Ps 65:6}] de notre salut, e. de ttes les
\item[\vref{Pr 13:12}] Un e. différé fait languir le cœur, mais
\item[\vref{Es 57:10}] pas : C'est sans e. ! Tu trouves encore
\item[\vref{Jé 2:25}] Non, c'est sans e. ! Car j'aime les
\item[\vref{Os 2:17}] porte de son e., et là, elle
\item[\vref{Za 9:5}] aussi, car son e. sera confondu. Et
\item[\vref{Ac 16:19}] servante voyant disparaître l'e. de lr. gain,
\end{listverse}

\ConcordanceEntry{Esprit}
\vspace{-2mm}
\begin{listverse}
\item[\vref{Ge 45:27}] le porter ; et l'e. de Jacob, lr.
\item[\vref{Ex 28:3}] j'ai remplis de l'e. de science, afin
\item[\vref{No 11:17}] je mettrai de l'E. qui est sur
\item[\vref{No 14:24}] animé d'un autre e., et qu'il a
\item[\vref{No 27:18}] en qui est l'E., et tu poseras
\item[\vref{De 2:30}] avait endurci son e., et raidit son
\item[\vref{De 28:28}] frappera de folie, d'aveuglement, et d'égarement d'e. ;
\item[\vref{De 34:9}] fut rempli de l'E. de sagesse, parce
\item[\vref{Jg 9:23}] envoya un mauvais e. entre Abimélec et
\item[\vref{1 S 16:14}] et un mauvais e. envoyé par Yahweh
\item[\vref{1 R 22:21}] Alors un e. s'avança et se tint dvt Yahweh,
\item[\vref{2 R 2:9}] j'aie, une double portion de ton e. !
\item[\vref{2 R 2:15}] vu, ils dirent : L'e. d'Elie repose sur
\item[\vref{2 R 23:24}] qui évoquaient les e. des morts et
\item[\vref{2 Ch 18:21}] je serai un e. de mensonge ds
\item[\vref{2 Ch 36:22}] accomplie, Yahweh réveilla l'e. de Cyrus, roi
\item[\vref{Esd 1:1}] accomplie, Yahweh réveilla l'e. de Cyrus, roi
\item[\vref{Né 9:20}] donnas ton bon E. pour les rendre
\item[\vref{Job 4:15}] Un e. passa dvt moi, et mes cheveux
\item[\vref{Job 12:10}] qui vit, et l'e. de tte chair
\item[\vref{Job 32:8}] y a un e. ds l'hom., mais
\item[\vref{Job 34:14}] à lui son E. et son souffle,
\item[\vref{Ps 34:19}] et il délivre ceux qui ont l'e. abattu.
\item[\vref{Ps 51:12}] pur, et renouvelle en moi un e. ferme.
\item[\vref{Ps 104:29}] troublés. Retires-tu lr. e. ? Ils défaillent et
\item[\vref{Pr 17:22}] un remède, mais l'e. abattu dessèche les
\item[\vref{Pr 17:27}] qui est d'un e. calme est un
\item[\vref{Pr 18:14}] L'e. d'un hom. fort le soutiendra ds
\item[\vref{Pr 20:27}] L'e. de l'hom. est une lampe de
\item[\vref{Ec 1:14}] voici tt est vanité et tourment d'e.
\item[\vref{Ec 3:21}] Qui sait si l'e. des fils de
\item[\vref{Ec 7:9}] point en ton e. de t'irriter, car
\item[\vref{Ec 12:9}] été, et que l'e. retourne à Dieu
\item[\vref{Ez 37:9}] dit : Prophétise à l'E. ! Prophétise, fils de
\item[\vref{Da 4:8}] a en lui l'E. des dieux saints.
\item[\vref{Da 4:9}] je sais que l'E. des dieux saints
\item[\vref{Za 12:1}] qui a formé l'e. de l'hom. au-dedans
\item[\vref{Za 12:10}] habitants de Jérus., l'E. de grâce et
\item[\vref{Mt 10:1}] de chasser les e. impurs et de
\item[\vref{Mt 12:43}] Lorsque l'e. impur est sorti
\item[\vref{Mt 26:41}] en tentation, car l'e. est prompt, mais
\item[\vref{Mt 27:50}] nouveau un grand cri, et rendit l'e.
\item[\vref{Mc 5:2}] hom. possédé d'un e. impur, sortit d'abord
\item[\vref{Mc 9:17}] mon fils qui est possédé d'un e. muet.
\item[\vref{Lu 1:17}] lui animé de l'e. et de la
\item[\vref{Lu 6:18}] tourmentés par des e. impurs furent guéris.
\item[\vref{Lu 7:21}] et d'infirmités, et d'e. malins, et il
\item[\vref{Lu 8:29}] Jésus commandait à l'e. impur de sortir
\item[\vref{Lu 8:55}] Et son e. revint en elle, et à l'instant
\item[\vref{Lu 9:55}] pas de quel e. vs. êtes animés.
\item[\vref{Lu 23:46}] je remets mon e. entre tes mains !
\item[\vref{Lu 24:39}] voyez : Car un e. n'a ni chair
\item[\vref{Lu 24:45}] il lr. ouvrit l'e. afin qu'ils comprennent
\item[\vref{Ac 2:4}] étrangères selon que l'E. lr. donnait de
\item[\vref{Ac 8:7}] car les e. impurs sortaient, en criant à haute
\item[\vref{Ac 16:16}] qui avait un e. de python, et
\item[\vref{Ac 17:16}] sentit au-dedans son e. s'irriter, à la
\item[\vref{Ac 19:15}] Mais l'e. malin lr. répondit : Je connais Jésus,
\item[\vref{Ac 23:8}] ni d'ange, ni d'e., mais les pharisiens
\item[\vref{Ro 7:6}] Dieu ds un e. nouveau, et non
\item[\vref{Ro 8:2}] la loi de l'E. de vie, qui
\item[\vref{Ro 8:6}] mais l'affection de l'E. c'est la vie
\item[\vref{Ro 8:9}] chair, mais selon l'E., si toutefois l'Esprit
\item[\vref{Ro 8:13}] mais si par l'E. vs. faites mourir
\item[\vref{Ro 11:8}] a donné un e. d'assoupissement, des yeux
\item[\vref{Ro 15:30}] la charité de l'E., que vs. combattiez
\item[\vref{1 Co 2:11}] de l'hom., sinon l'e. de l'hom. qui
\item[\vref{1 Co 2:12}] n'avons pas reçu l'e. de ce monde,
\item[\vref{1 Co 5:4}] Vous et mon e. étant assemblés au
\item[\vref{1 Co 5:5}] chair, afin que l'e. soit sauvé au
\item[\vref{1 Co 6:17}] Seign. est avec lui un seul e.
\item[\vref{1 Co 14:14}] langue inconnue mon e. est en prière,
\item[\vref{1 Co 15:45}] dernier Adam en E. vivifiant.
\item[\vref{2 Co 1:22}] les arrhes de l'E. ds nos cœurs.
\item[\vref{2 Co 11:4}] recevez un autre e. que celui que
\item[\vref{Ep 1:18}] yeux de votre e., afin que vs.
\item[\vref{Ep 2:18}] auprès du Père ds un mm E.
\item[\vref{Ep 4:4}] corps, un seul E., com. aussi vs.
\item[\vref{Ep 5:18}] la dissolution, mais soyez remplis de l'E.
\item[\vref{Ph 2:1}] a qq communion d'e., s'il y a
\item[\vref{1 Th 5:23}] votre être entier, l'e., l'âme et le
\item[\vref{2 Ti 1:7}] pas donné un e. de timidité, mais
\item[\vref{Hé 1:14}] pas ts des e. administrateurs, envoyés pour
\item[\vref{Ja 2:26}] le corps sans e. est mort, de
\item[\vref{1 Pi 1:11}] circonstances marquées par l'E. prophétique de Christ
\item[\vref{1 Pi 3:4}] cœur, l'incorruptibilité d'un e. doux et paisible,
\item[\vref{1 Jn 2:13}] vs. avez vaincu l'e. du malin.
\item[\vref{1 Jn 4:1}] pas à tt e. ; mais éprouvez les
\item[\vref{1 Jn 4:6}] que ns. connaissons l'E. de vérité et
\item[\vref{Jud 1:19}] eux-mêmes, des gens sensuels, n'ayant pas l'E.
\item[\vref{Ap 1:10}] fus ravi en e. au jour du
\item[\vref{Ap 11:11}] jours et demi, l'E. de vie venant
\item[\vref{Ap 13:15}] de mettre un e. à l'image de
\item[\vref{Ap 17:9}] qu'il faut un e. intelligent et qui
\item[\vref{Ap 18:2}] retraite de tt e. impur, et le
\end{listverse}

\ConcordanceEntry{Esprit de Dieu, Esprit de Yahweh}
\vspace{-2mm}
\begin{listverse}
\item[\vref{Ge 1:2}] de l'abîme, et l'E. de Dieu se
\item[\vref{Ge 6:3}] Yahweh dit : Mon E. ne contestera point
\item[\vref{Ex 31:3}] l'ai rempli de l'E. de Dieu, de
\item[\vref{No 11:29}] Yahweh mît son E. sur eux !
\item[\vref{Jg 11:29}] L'E. de Yahweh fut sur Jephthé. Il
\item[\vref{Jg 14:6}] Et l'E. de Yahweh saisit Samson ; sans avoir
\item[\vref{1 S 11:6}] Et l'E. de Dieu saisit Saül, lorsqu'il entendit
\item[\vref{1 S 16:13}] depuis ce jour-là, l'E. de Yahweh saisit
\item[\vref{1 S 16:14}] L'E. de Yahweh se retira de Saül,
\item[\vref{2 S 23:2}] L'E. de Yahweh parle par moi, et
\item[\vref{1 R 18:12}] je t'aurai quitté, l'E. de Yahweh te
\item[\vref{2 R 2:16}] de peur que l'E. de Yahweh ne
\item[\vref{2 Ch 15:1}] Alors l'E. de Dieu fut sur Azaria, fils
\item[\vref{2 Ch 24:20}] Alors l'E. de Dieu revêtit Zacharie, fils de
\item[\vref{Né 9:30}] avertissais par ton E., par la main
\item[\vref{Job 26:13}] cieux par son E., et de sa
\item[\vref{Job 27:3}] respiration et que l'E. de Dieu sera
\item[\vref{Ps 32:2}] iniquité, et ds l'e. duquel il n'y
\item[\vref{Ps 51:13}] face, et ne m'ôte pas ton E. Saint.
\item[\vref{Ps 104:30}] Tu envoies ton E., ils sont créés ;
\item[\vref{Ps 106:33}] résistèrent à son e., et il parla
\item[\vref{Ps 139:7}] loin de ton E., et où fuirai-je
\item[\vref{Ps 143:10}] Que ton bon E. me conduise sur
\item[\vref{Es 11:2}] L'E. de Yahweh reposera sur lui, Esprit
\item[\vref{Es 34:16}] ordonné, et son E. qui les rassemblera.
\item[\vref{Es 40:13}] Qui a dirigé l'E. de Yahweh, ou
\item[\vref{Es 44:3}] je répandrai mon E. sur ta postérité,
\item[\vref{Es 59:19}] un fleuve, mais l'E. de Yahweh lèvera
\item[\vref{Es 61:1}] L'E. du Seign. Yahweh est sur moi,
\item[\vref{Ez 1:12}] allaient partout où l'E. les poussait à
\item[\vref{Ez 1:20}] allaient partout où l'E. les poussait à
\item[\vref{Ez 2:2}] Alors l'E. entra en moi, après qu'il m'eut
\item[\vref{Ez 3:14}] L'E. dc m'enleva, et me prit, et
\item[\vref{Ez 11:5}] L'E. de Yahweh tomba sur moi. Et
\item[\vref{Ez 11:24}] Puis l'E. m'enleva et me transporta en Chaldée,
\item[\vref{Ez 36:27}] Je mettrai mon E. au dedans de
\item[\vref{Ez 37:1}] transporta par son E. et me déposa
\item[\vref{Ez 37:14}] Je mettrai mon E. en vs., et
\item[\vref{Joë 2:28}] je répandrai mon E. sur tte chair ;
\item[\vref{Mi 3:8}] de courage, par l'E. de Yahweh, pour
\item[\vref{Za 4:6}] mais par mon E., dit Yahweh des
\item[\vref{Mt 4:1}] fut emmené par l'E. ds le désert,
\item[\vref{Mt 10:20}] parlerez, mais c'est l'E. de votre Père
\item[\vref{Mt 12:28}] les démons par l'E. de Dieu, certes
\item[\vref{Mt 12:32}] parlera contre le Saint-E., il ne lui
\item[\vref{Lu 1:15}] sera rempli du Saint-E. dès le ventre
\item[\vref{Lu 1:35}] et dit : Le Saint-E. viendra sur toi,
\item[\vref{Lu 4:1}] Jésus, rempli du Saint-E., revint du Jourdain,
\item[\vref{Lu 4:18}] L'E. du Seign. est sur moi, parce
\item[\vref{Lu 11:13}] du ciel donnera-t-il l'E. Saint à ceux
\item[\vref{Jn 1:32}] disant : J'ai vu l'E. descendre du ciel
\item[\vref{Jn 3:6}] est né de l'E. est esprit.
\item[\vref{Jn 3:34}] lui donne pas l'E. par mesure.
\item[\vref{Jn 4:24}] Dieu est E., et il faut que ceux qui
\item[\vref{Jn 6:63}] C'est l'E. qui vivifie ; la chair ne sert
\item[\vref{Jn 7:39}] dit cela de l'E. que devaient recevoir
\item[\vref{Jn 14:17}] l'E. de vérité que le monde ne
\item[\vref{Jn 16:13}] sera venu, lui, l'E. de vérité, il
\item[\vref{Jn 20:22}] eux, et lr. dit : Recevez le Saint-E.
\item[\vref{Ac 2:17}] répandrai de mon E. sur tte chair ;
\item[\vref{Ac 2:33}] la promesse du Saint-E., il a répandu
\item[\vref{Ac 6:3}] témoignage, pleins du Saint-E. et de sagesse,
\item[\vref{Ac 8:39}] sortis de l'eau, l'E. du Seign. enleva
\item[\vref{Ac 10:38}] a oint du Saint-E. et de force
\item[\vref{Ac 13:4}] envoyés par le Saint-E., descendirent à Séleucie,
\item[\vref{Ac 15:8}] lr. donnant le Saint-E., de mm qu'à
\item[\vref{Ac 19:2}] Avez-vs. reçu le Saint-E. qnd vs. avez
\item[\vref{Ro 1:4}] avec puissance, selon l'E. de sainteté, par
\item[\vref{Ro 5:5}] cœurs par le Saint-E. qui ns. a
\item[\vref{Ro 8:9}] chair, mais selon l'E., si toutefois l'Esprit
\item[\vref{Ro 8:14}] sont conduits par l'E. de Dieu sont
\item[\vref{Ro 8:16}] L'E. lui-mm rend témoignage à notre esprit
\item[\vref{Ro 8:26}] De mm aussi l'E. ns. aide ds
\item[\vref{1 Co 2:4}] mais en démonstration d'E. et de puissance,
\item[\vref{1 Co 2:10}] révélées par son E.. Car l'Esprit sonde
\item[\vref{1 Co 3:16}] Dieu et que l'E. de Dieu habite
\item[\vref{1 Co 12:3}] s'il parle par l'E. de Dieu, ne
\item[\vref{1 Co 12:11}] seul et mm E. opère ttes ces
\item[\vref{2 Co 13:13}] la communication du Saint-E. soient avec vs.
\item[\vref{Ga 4:6}] Dieu a envoyé l'E. de son Fils
\item[\vref{Ga 5:16}] dc : Marchez selon l'E., et vs. n'accomplirez
\item[\vref{Ga 5:17}] à ceux de l'E., et l'Esprit en
\item[\vref{Ga 5:22}] le fruit de l'E. c'est la charité,
\item[\vref{Ga 5:25}] ns. vivons par l'E., marchons aussi par
\item[\vref{Ep 3:16}] fortifiés par son E. ds l'hom. intérieur,
\item[\vref{Ep 4:30}] n'attristez pas le Saint-E. de Dieu, par
\item[\vref{1 Th 5:19}] N'éteignez pas l'E.
\item[\vref{1 Ti 3:16}] chair, justifié par l'E., vu des anges,
\item[\vref{1 Ti 4:1}] Mais l'E. dit expressément que ds les derniers
\item[\vref{Hé 6:4}] qui ont été faits participants au Saint-E.,
\item[\vref{Hé 9:14}] Christ, qui, par l'E. éternel, s'est offert
\item[\vref{Hé 10:29}] qui aura outragé l'E. de grâce ?
\item[\vref{Ja 4:5}] parle en vain ? L'E. qui habite en
\item[\vref{1 Pi 4:14}] êtes bénis, car l'E. de gloire, l'Esprit
\item[\vref{2 Pi 1:21}] poussés par le Saint-E. que les saints
\item[\vref{1 Jn 4:2}] à cette marque l'E. de Dieu : Tout
\item[\vref{1 Jn 4:13}] qu'il ns. a donné de son E.
\item[\vref{1 Jn 5:6}] sang ; et c'est l'E. qui rend témoignage,
\item[\vref{Ap 1:4}] part des sept E. qui sont dvt
\item[\vref{Ap 2:7}] entende ce que l'E. dit aux églises !
\item[\vref{Ap 22:17}] Et l'E. et l'Epouse disent : Viens ! Et que
\end{listverse}

\ConcordanceEntry{Esprit de Jésus, Esprit de Christ}
\vspace{-2mm}
\begin{listverse}
\item[\vref{Mc 2:8}] connu par son e. qu'ils raisonnaient ainsi
\item[\vref{Ac 16:7}] en Bithynie ; mais l'E. de Jésus ne
\item[\vref{Ro 8:9}] chair, mais selon l'E., si toutefois l'Esprit
\item[\vref{Ph 1:19}] le secours de l'E. de Jésus-Christ,
\item[\vref{1 Pi 1:11}] circonstances marquées par l'E. prophétique de Christ
\item[\vref{Ap 19:10}] de Jésus est l'E. de la prophétie.
\end{listverse}

\ConcordanceEntry{Esther}
\vspace{-2mm}
\begin{listverse}
\item[\vref{Est 2:7}] Hadassa qui est E., fille de son
\item[\vref{Est 2:17}] le roi aima E. plus que ttes
\item[\vref{Est 2:18}] festin en l'honneur d'E. ; il donna du
\item[\vref{Est 8:2}] donna à Mardochée ; E. établit Mardochée sur
\item[\vref{Est 8:7}] à la reine E. et au Juif
\end{listverse}

\ConcordanceEntry{Etablir}
\vspace{-2mm}
\begin{listverse}
\item[\vref{Ge 6:18}] Mais j'é. mon alliance avec toi ; et tu
\item[\vref{Ge 17:5}] car je t'ai é. père d'une multitude
\item[\vref{Jé 1:10}] Regarde, je t'é. aujourd'hui sur les
\item[\vref{Am 5:15}] le bien, et é. la justice à
\item[\vref{Mt 24:15}] le prophète, être é. ds le lieu
\item[\vref{Mt 24:45}] son maître a é. sur ts ses
\item[\vref{Mt 25:21}] de choses, je t'é. sur beaucoup ; viens
\item[\vref{Jn 15:16}] je vs. ai é., afin que vs.
\item[\vref{1 Co 12:28}] Et Dieu a é. ds l'Eglise, premièrement
\item[\vref{Ep 1:22}] pieds, et l'a é. sur ttes choses
\item[\vref{Ph 1:17}] que je suis é. pour la défense
\item[\vref{1 Ti 1:9}] loi a été é., mais pour les
\item[\vref{Hé 1:2}] Fils, qu'il a é. héritier de ttes
\item[\vref{Hé 10:9}] le premier afin d'é. le second.
\end{listverse}

\ConcordanceEntry{Etang}
\vspace{-2mm}
\begin{listverse}
\item[\vref{Ap 19:20}] jetés vivants ds l'é. ardent de feu
\item[\vref{Ap 20:10}] fut jeté ds l'é. de feu et
\item[\vref{Ap 20:14}] furent jetés ds l'é. de feu. C'est
\item[\vref{Ap 21:8}] part sera ds l'é. ardent de feu
\end{listverse}

\ConcordanceEntry{Eté}
\vspace{-2mm}
\begin{listverse}
\item[\vref{Ge 8:22}] et la chaleur, l'é. et l'hiver, le
\item[\vref{Ps 74:17}] tu as formé l'é. et l'hiver.
\item[\vref{Pr 6:8}] elle prépare en é. son pain et
\item[\vref{Pr 10:5}] prudent amasse en é., mais celui qui
\item[\vref{Pr 26:1}] convient pas en é., ni la pluie
\item[\vref{Pr 30:25}] néanmoins préparent durant l'é. lr. nourriture ;
\item[\vref{Jé 8:20}] moisson est passée, l'é. est fini, et
\item[\vref{Mt 24:32}] vs. savez que l'é. est proche.
\end{listverse}

\ConcordanceEntry{Eteindre}
\vspace{-2mm}
\begin{listverse}
\item[\vref{Lé 6:5}] brûler, on ne l'é. point ; le prêtre
\item[\vref{2 Ch 34:25}] lieu, et elle ne sera point é.
\item[\vref{Pr 26:20}] Le feu s'é. faute de bois, ainsi qnd il
\item[\vref{Pr 31:18}] sa lampe ne s'é. pas la nuit.
\item[\vref{Es 34:10}] ne sera point é. ni jour ni
\item[\vref{Mt 3:12}] paille ds un feu qui ne s'é. pas.
\item[\vref{Mt 12:20}] roseau cassé et n'é. pas le lumignon
\item[\vref{Mc 9:43}] géhenne, ds le feu qui ne s'é. pas ;
\item[\vref{Ep 6:16}] lequel vs. pourrez é. ts les dards
\item[\vref{1 Th 5:19}] N'é. pas l'Esprit.
\item[\vref{Hé 11:34}] é. la force du feu, échappèrent au
\end{listverse}

\ConcordanceEntry{Eternel, Eternelle}
\vspace{-2mm}
\begin{listverse}
\item[\vref{Ge 17:7}] sera une alliance é. en vertu de
\item[\vref{Ec 12:7}] vers sa maison é., et ceux qui
\item[\vref{Es 35:10}] et une joie é. sera sur lr.
\item[\vref{Es 55:3}] vs. une alliance é., les miséricordes immuables
\item[\vref{Es 56:5}] chacun un nom é. qui ne périra
\item[\vref{Es 60:19}] toi la lumière é., et ton Dieu
\item[\vref{Es 63:12}] eux pour se faire un nom é. ;
\item[\vref{Jé 10:10}] et le Roi é. ; la terre tremble
\item[\vref{Da 4:3}] est un règne é., et sa domination
\item[\vref{Da 7:14}] est une domination é. qui ne passera
\item[\vref{Da 9:24}] amener la justice é., pour mettre le
\item[\vref{Mt 25:41}] ds le feu é., qui a été
\item[\vref{Mc 3:29}] est coupable et subira une condamnation é. .
\item[\vref{Jn 3:16}] pas, mais qu'il ait la vie é.
\item[\vref{Jn 10:28}] donne la vie é., elles ne périront
\item[\vref{Jn 17:3}] Or la vie é., c'est qu'ils te
\item[\vref{Ro 1:20}] savoir sa puissance é. et sa divinité,
\item[\vref{Ro 6:22}] sanctification et pour fin la vie é.
\item[\vref{2 Co 4:17}] ns. un poids é. d'une gloire souverainement
\item[\vref{2 Co 4:18}] un temps, mais les invisibles sont é.
\item[\vref{2 Co 5:1}] savoir une maison é. ds les cieux.
\item[\vref{2 Th 1:9}] châtiment une ruine é., loin de la
\item[\vref{1 Ti 6:12}] saisis la vie é., à laquelle aussi
\item[\vref{Tit 3:7}] de la vie é. selon notre espérance.
\item[\vref{Hé 5:9}] l'auteur du salut é. pour ts ceux
\item[\vref{Hé 6:2}] résurrection des morts, et du jugement é.
\item[\vref{Hé 9:12}] sang, après avoir obtenu une rédemption é.
\item[\vref{Hé 9:14}] qui, par l'Esprit é., s'est offert lui-mm
\item[\vref{1 Jn 1:2}] annonçons la vie é., qui était avec
\item[\vref{Jud 1:7}] ayant reçu la punition du feu é.
\item[\vref{Ap 14:6}] il avait l'Evangile é. pour évangéliser les
\end{listverse}

\ConcordanceEntry{Eternellement}
\vspace{-2mm}
\begin{listverse}
\item[\vref{Ge 3:22}] qu'il n'en mange, et ne vive é.
\item[\vref{Ex 3:15}] ici mon Nom é., et c'est ici
\item[\vref{De 32:40}] ciel, et je dis : Je vis é.
\item[\vref{1 Ch 17:12}] une maison, et j'affermirai son trône é.
\item[\vref{Ps 29:10}] déluge ; Yahweh est assis com. Roi é.
\item[\vref{Ps 48:15}] est notre Dieu é. et à jamais ;
\item[\vref{Ps 61:5}] Je séjournerai é. ds ta tente,
\item[\vref{Ps 102:13}] Yahweh ! tu demeures é., et ta mémoire
\item[\vref{Ps 110:4}] tu es prêtre é., à la manière
\item[\vref{Es 40:8}] la parole de notre Dieu demeure é.
\item[\vref{Es 51:6}] mon salut demeurera é., et ma justice
\item[\vref{Jé 20:17}] n'est-elle pas restée é. enceinte ?
\item[\vref{Da 2:44}] ces royaumes-là, et lui-mm sera établi é.
\item[\vref{Da 6:26}] et il subsiste é. ; son Royaume ne
\item[\vref{Da 7:18}] posséderont le Royaume é., d'éternité en éternité.
\item[\vref{Lu 1:33}] maison de Jacob é., et son règne
\item[\vref{Jn 6:51}] pain, il vivra é. ; et le pain
\item[\vref{Jn 12:34}] le Christ demeure é. ; et comment dc
\item[\vref{Ro 9:5}] Dieu au-dessus de ttes choses, béni é.. Amen !
\item[\vref{Hé 5:6}] Tu es prêtre é., selon l'ordre de
\item[\vref{Hé 7:24}] parce qu'il demeure é., possède une prêtrise
\item[\vref{Hé 13:8}] hier, aujourd'hui, et il l'est aussi é.
\item[\vref{1 Pi 1:25}] du Seign. demeure é.. Et cette parole
\item[\vref{2 Jn 1:2}] ns., et qui sera avec ns. é.:
\item[\vref{Jud 1:13}] qui l'obscurité des ténèbres est réservée é.
\end{listverse}

\ConcordanceEntry{Eternité}
\vspace{-2mm}
\begin{listverse}
\item[\vref{Ge 21:33}] nom de Yahweh, le Dieu de l'é.
\item[\vref{De 33:27}] Le Dieu d'é. est un refuge,
\item[\vref{Ps 41:14}] le Dieu d'Israël, d'é. en éternité ! Amen !
\item[\vref{Ps 55:20}] qui de tte é. est assis sur
\item[\vref{Ps 90:2}] et le monde, d'é. en éternité, tu
\item[\vref{Ps 93:2}] dès lors, tu es de tte é.
\item[\vref{Ps 133:3}] la bénédiction et la vie, pour l'é.
\item[\vref{Ps 139:24}] voie ; conduis-moi sur la voie de l'é.
\item[\vref{Pr 8:23}] déclarée princesse depuis l'é., dès le commencement,
\item[\vref{Ec 3:11}] aussi a-t-il mis l'é. ds lr. cœur,
\item[\vref{Es 9:5}] Puissant, le Père d'é., le Prince de
\item[\vref{Es 40:28}] que le Dieu d'é., Yahweh, a créé
\item[\vref{Da 2:20}] Nom de Dieu, d'é. en éternité, car
\item[\vref{Da 7:18}] le Royaume éternellement, d'é. en éternité.
\item[\vref{Mi 5:1}] aux temps anciens, aux jours de l'é.
\item[\vref{Ha 1:12}] pas de tte é., ô Yahweh ! Mon
\item[\vref{Ac 15:18}] Dieu lui sont connues de tte é.
\end{listverse}

\ConcordanceEntry{Ethiopie}
\vspace{-2mm}
\begin{listverse}
\item[\vref{2 R 19:9}] de Tirhaka, roi d'E. ; on lui dit :
\item[\vref{2 Ch 14:8}] Mais Zérach, l'E., sortit contre eux
\item[\vref{Est 1:1}] les Indes jusqu'en E., sur cent vingt-sept
\item[\vref{Ps 68:32}] seigneurs viendront d'Egypte ; l'E. se hâtera d'étendre
\item[\vref{Es 18:1}] qui est au-delà des fleuves de l'E. ;
\item[\vref{Jé 13:23}] L'E. peut-il changer sa peau et le
\item[\vref{Na 3:9}] L'E. et l'Egypte étaient sa force, et
\item[\vref{Ac 8:27}] voici, un hom. E., un eunuque, qui
\end{listverse}

\ConcordanceEntry{Etienne}
\vspace{-2mm}
\begin{listverse}
\item[\vref{Ac 6:5}] et ils élurent E., hom. plein de
\item[\vref{Ac 6:8}] Or E., plein de foi et de puissance,
\item[\vref{Ac 6:9}] d'Asie, se levèrent pour disputer contre E.
\item[\vref{Ac 7:55}] Mais E., rempli du Saint-Esprit, et fixant les
\item[\vref{Ac 7:59}] Et ils lapidaient E. qui priait, et
\end{listverse}

\ConcordanceEntry{Etoile}
\vspace{-2mm}
\begin{listverse}
\item[\vref{Ge 1:16}] la nuit ; il fit aussi les é.
\item[\vref{Ge 15:5}] et compte les é. si tu peux
\item[\vref{Ge 37:9}] lune et onze é. se prosternaient dvt
\item[\vref{No 24:17}] de près ; une E. est sortie de
\item[\vref{De 4:19}] lune, et les é., tte l'armée des
\item[\vref{Job 9:9}] pléiades, et les é. des régions australes.
\item[\vref{Job 25:5}] luit pas ; les é. ne sont pas
\item[\vref{Job 38:7}] Quand les é. du matin se
\item[\vref{Ps 147:4}] le nombre des é., il les appelle
\item[\vref{Es 14:13}] trône au-dessus des é. de Dieu ; je
\item[\vref{Da 12:3}] brilleront com. les é., à toujours et
\item[\vref{Mt 2:2}] avons vu son é. en orient, et
\item[\vref{Mt 24:29}] lumière, et les é. tomberont du ciel,
\item[\vref{1 Co 15:41}] autre l'éclat des é. ; mm une étoile
\item[\vref{2 Pi 1:19}] paraître et que l'E. du matin se
\item[\vref{Jud 1:13}] leurs impuretés ; des é. errantes, à qui
\item[\vref{Ap 1:16}] main droite sept é., et de sa
\item[\vref{Ap 1:20}] mystère des sept é. que tu as
\item[\vref{Ap 6:13}] Et les é. du ciel tombèrent sur la terre,
\item[\vref{Ap 8:10}] ciel une grande é. ardente com. un
\item[\vref{Ap 12:1}] sa tête une couronne de douze é.
\item[\vref{Ap 22:16}] postérité de David, l'é. brillante du matin.
\end{listverse}

\ConcordanceEntry{Etonnement}
\vspace{-2mm}
\begin{listverse}
\item[\vref{Ge 43:33}] regardaient les uns les autres avec é.
\item[\vref{Mt 9:8}] elle fut saisie d'é., et elle glorifiait
\item[\vref{Mc 6:2}] l'entendaient étaient ds l'é., et ils disaient :
\item[\vref{Mc 9:15}] elle fut saisie d'é., et accourut pour
\item[\vref{Mc 11:18}] foule était ds l'é. à l’égard
\item[\vref{Lu 1:63}] son nom. Et ts furent ds l'é.
\item[\vref{Lu 2:18}] entendirent furent ds l'é. de ce que
\item[\vref{Lu 2:48}] ils furent saisis d'é., et sa mère
\item[\vref{Lu 11:38}] pharisien vit avec é. qu'il ne s'était
\item[\vref{Lu 24:12}] chez lui, ds l'é. de ce qui
\item[\vref{Ac 3:10}] remplis d'admiration et d'é. de ce qui
\item[\vref{2 Co 11:15}] un grand sujet d'é. si ses serviteurs
\item[\vref{Ap 17:6}] vis, je fus saisi d'un grand é.
\end{listverse}

\ConcordanceEntry{Etonner}
\vspace{-2mm}
\begin{listverse}
\item[\vref{Ps 97:4}] terre le voit et tremble tt é.
\item[\vref{Ec 5:7}] soient violés, ne t'é. point de cela ;
\item[\vref{Es 41:10}] ne sois pas é., car je suis
\item[\vref{Es 63:5}] m'aider ; et j'étais é., et il n'y
\item[\vref{Jé 2:12}] Cieux, soyez é. de cela ; frémissez
\item[\vref{Da 8:27}] du roi. J'étais é. de la vision,
\item[\vref{Ha 1:5}] nations, voyez, soyez é. et stupéfaits ! Car
\item[\vref{Mt 8:10}] entendu, Jésus fut é. et dit à
\item[\vref{Mt 13:54}] qui l'entendirent étaient é., et disaient : D'où
\item[\vref{Mt 19:25}] choses furent très é., et dirent : Qui
\item[\vref{Mc 15:5}] donna plus aucune réponse, ce qui é. Pilate.
\item[\vref{Lu 20:26}] le peuple ; mais, é. de sa réponse,
\item[\vref{Ac 4:13}] peuple ; ils s'en é., et ils reconnaissaient
\item[\vref{Ac 10:45}] avec Pierre, furent é. de ce que
\item[\vref{Ac 12:16}] virent, et furent é. de le voir.
\item[\vref{Ac 13:41}] vs. mépriseurs, soyez é. et disparaissez ; car
\item[\vref{Ga 1:6}] Je m'é. que vs. abandonniez si promptement celui
\item[\vref{1 Jn 3:13}] frères, ne vs. é. pas si le
\item[\vref{Ap 17:8}] fondation du monde, s'é. en voyant la
\end{listverse}

\ConcordanceEntry{Etourdissement}
\vspace{-2mm}
\begin{listverse}
\item[\vref{Ps 60:5}] tu ns. as abreuvés d'un vin d'é.
\item[\vref{Es 51:17}] sucé la lie de la coupe d'é. !
\item[\vref{Es 51:22}] main la coupe d'é., la lie de
\item[\vref{Da 8:18}] je restai frappé d'é., la face contre
\item[\vref{Da 10:9}] je tombai frappé d'é., la face contre
\item[\vref{Za 12:2}] Jérus. une coupe d'é. pour ts les
\item[\vref{Za 12:4}] Yahweh, je frapperai d'é. ts les chevaux,
\end{listverse}

\ConcordanceEntry{Etrange}
\vspace{-2mm}
\begin{listverse}
\item[\vref{Ps 139:14}] fait d'une si é. et si admirable
\item[\vref{Da 11:36}] proférera des choses é. contre le Dieu
\item[\vref{Lu 5:26}] ns. avons vu aujourd'hui des choses é.
\item[\vref{Jn 9:30}] c'est une chose é. que vs. ne
\item[\vref{Ac 17:20}] de certaines choses é. ; ns. voudrions dc
\item[\vref{1 Pi 4:4}] Gentils trouvent fort é., ils vs. calomnient
\item[\vref{1 Pi 4:12}] ne trouvez pas é. qnd vs. êtes
\end{listverse}

\ConcordanceEntry{Etranger}
\vspace{-2mm}
\begin{listverse}
\item[\vref{Ge 15:13}] cents ans com. é. ds un pays
\item[\vref{Ge 17:8}] tu demeures com. é., à savoir tt
\item[\vref{Ge 23:4}] Je suis é. et habitant parmi vs. ; donnez-moi une
\item[\vref{Ex 22:21}] ni n'opprimeras point l'é. ; car vs. avez
\item[\vref{Lé 24:22}] jugement. Vous traiterez l'é. com. celui qui
\item[\vref{De 10:19}] Vous aimerez dc l'é. ; car vs. avez
\item[\vref{De 24:17}] pas d'injustice à l'é. ni à l'orphelin,
\item[\vref{Jos 8:35}] petits-enfants, et des é. qui marchaient au
\item[\vref{Ru 2:10}] à moi, moi qui suis une é. ?
\item[\vref{1 R 11:1}] aima plusieurs femmes é., outre la fille
\item[\vref{1 Ch 16:19}] peu nombreux, et é. ds le pays,
\item[\vref{1 Ch 29:15}] dvt toi des é. et des habitants,
\item[\vref{Job 31:32}] L'é. n'a pas passé la nuit dehors ;
\item[\vref{Ps 69:9}] suis devenu un é. pour mes frères,
\item[\vref{Ps 119:19}] Je suis un é. sur la terre,
\item[\vref{Pr 27:2}] que ce soit l'é., et non pas
\item[\vref{Es 14:1}] lr. pays ; les é. se joindront à
\item[\vref{Es 56:3}] que l'enfant de l'é. qui se joint
\item[\vref{Os 5:7}] engendré des fils é. ; mntnt un mois
\item[\vref{Os 7:9}] Les é. ont dévoré sa force, et il
\item[\vref{Mt 25:35}] à boire ; j'étais é., et vs. m'avez
\item[\vref{Mt 27:7}] d'un potier pour la sépulture des é.
\item[\vref{Lu 17:18}] eu que cet é. qui soit revenu
\item[\vref{Jn 10:5}] suivront pas un é. ; au contraire, elles
\item[\vref{Ac 7:29}] discours, et fut é. ds le pays
\item[\vref{Ac 17:18}] annonce des divinités é.. Parce qu'il lr.
\item[\vref{Ep 2:12}] en Israël, étant é. des alliances de
\item[\vref{Ep 2:19}] n'êtes plus des é. ni des gens
\item[\vref{Col 1:21}] qui étiez autrefois é., et qui étiez
\item[\vref{Hé 11:13}] profession qu'ils étaient é. et voyageurs sur
\item[\vref{Hé 13:9}] doctrines diverses et é. ; car il est
\item[\vref{1 Pi 2:11}] vs. exhorte, com. é. et voyageurs, à
\end{listverse}

\ConcordanceEntry{Etre}
\vspace{-2mm}
\begin{listverse}
\item[\vref{Ex 3:14}] à Moïse : JE S. QUI JE SUIS.
\item[\vref{Ec 1:9}] Ce qui a é., c'est ce qui sera, et ce
\item[\vref{Za 8:23}] entendu que Dieu e. avec vs.
\item[\vref{Mt 16:13}] ses disciples : Qui s.-je, aux dires
\item[\vref{Mt 16:15}] Et vs., qui dites-vs. que je s. ?
\item[\vref{Jn 8:23}] mais moi, je s. d'en haut. Vous
\item[\vref{Jn 8:24}] C'e. pourquoi je vs. ai dit que
\item[\vref{Jn 10:35}] parole de Dieu e. adressée, et cependant
\item[\vref{Jn 17:24}] Père, mon désir e. que ceux que
\item[\vref{Ac 13:23}] C'e. de la postérité de David que
\item[\vref{Ro 8:29}] aussi prédestinés à ê. conformes à l'image
\item[\vref{Ro 8:31}] choses ? Si Dieu e. pour ns., qui
\item[\vref{1 Co 1:2}] de Dieu qui e. à Corinthe, à
\item[\vref{1 Co 12:13}] ts, en effet, é. baptisés d'un mm
\item[\vref{1 Co 14:37}] Si quelqu'un croit ê. prophète, ou spirituel,
\item[\vref{1 Co 15:10}] de Dieu, je s. ce que je
\item[\vref{Ep 2:15}] en lui-mm pour ê. un hom. nouveau,
\item[\vref{Ep 2:21}] ensemble, s'élève pour ê. un temple saint
\item[\vref{Ph 1:23}] Car je s. pressé des deux côtés : Mon désir
\item[\vref{Ph 4:1}] C'e. pourquoi, mes très chers frères bien-aimés,
\item[\vref{2 Th 2:5}] ces choses, lorsque j'é. encore chez vs. ?
\item[\vref{1 Ti 1:7}] voulant ê. docteurs de la loi ; mais ils
\item[\vref{Tit 2:4}] jeunes femmes à ê. modestes, à aimer
\item[\vref{Hé 5:12}] vs. qui devriez ê. des maîtres, vu
\item[\vref{Ja 1:13}] Que personne, lorsqu'il e. tenté, ne dise :
\item[\vref{2 Pi 3:14}] C'e. pourquoi, mes bien-aimés, en attendant ces
\item[\vref{1 Jn 4:6}] Nous s. de Dieu ; celui qui connaît Dieu
\item[\vref{Ap 2:2}] qui se disent ê. apôtres et qui
\item[\vref{Ap 6:11}] frères qui doivent ê. mis à mort
\end{listverse}

\ConcordanceEntry{Etroit}
\vspace{-2mm}
\begin{listverse}
\item[\vref{2 R 6:1}] toi est trop é. pour ns.
\item[\vref{Ps 4:2}] Quand j'étais à l'é., tu m'as mis
\item[\vref{Mt 7:13}] par la porte é.. Car c'est la
\end{listverse}

\ConcordanceEntry{Eunice}
\vspace{-2mm}
\begin{listverse}
\item[\vref{2 Ti 1:5}] grand-mère et en E., ta mère, et
\end{listverse}

\ConcordanceEntry{Eunuque}
\vspace{-2mm}
\begin{listverse}
\item[\vref{Es 56:3}] peuple ! Et que l'e. ne dise pas :
\item[\vref{Mt 19:12}] y a des e. qui sont ainsi
\item[\vref{Ac 8:27}] hom. Ethiopien, un e., qui était un
\end{listverse}

\ConcordanceEntry{Euphrate}
\vspace{-2mm}
\begin{listverse}
\item[\vref{Ge 2:14}] l'Assyrie ; et le quatrième fleuve est l'E.
\item[\vref{Ge 15:18}] d'Egypte jusqu'au grand fleuve, le fleuve d'E. ;
\item[\vref{De 11:24}] le fleuve de l'E., jusqu'à la Mer
\item[\vref{Ap 9:14}] sont liés sur le grand fleuve, l'E.
\item[\vref{Ap 16:12}] le grand fleuve, l'E.. Et son eau
\end{listverse}

\ConcordanceEntry{Eutychus}
\vspace{-2mm}
\begin{listverse}
\item[\vref{Ac 20:9}] jeune hom. nommé E., qui était assis
\end{listverse}

\ConcordanceEntry{Evangéliste}
\vspace{-2mm}
\begin{listverse}
\item[\vref{Ep 4:11}] autres pour être é., les autres pour
\item[\vref{2 Ti 4:5}] fais l'œuvre d'un é., rends ton service
\end{listverse}

\ConcordanceEntry{Evangile}
\vspace{-2mm}
\begin{listverse}
\item[\vref{Mt 4:23}] leurs synagogues, prêchant l'E. du Royaume, et
\item[\vref{Mt 24:14}] Cet E. du Royaume sera prêché ds tte
\item[\vref{Mc 1:15}] est proche. Repentez-vs., et croyez à l'E.
\item[\vref{Lu 7:22}] les morts ressuscitent, l'E. est annoncé aux
\item[\vref{Ac 8:12}] lr. annonçait concernant l'E. du Royaume de
\item[\vref{Ro 1:9}] mon esprit ds l'E. de son Fils,
\item[\vref{Ro 1:15}] vs. annoncer aussi l'E., à vs. qui
\item[\vref{Ro 15:19}] tt rempli de l'E. de Christ.
\item[\vref{Ro 16:25}] affermir selon mon E., et selon la
\item[\vref{1 Co 4:15}] vs. ai engendrés en Jésus-Christ par l'E.
\item[\vref{1 Co 9:12}] créer d'obstacle à l'E. de Christ.
\item[\vref{1 Co 9:14}] ceux qui annoncent l'E. vivent de l'Evangile.
\item[\vref{2 Co 4:3}] Si notre E. est encore voilé, il ne l'est
\item[\vref{2 Co 4:4}] la lumière de l'E. de la gloire
\item[\vref{2 Co 11:7}] vs. ai annoncé l'E. de Dieu sans
\item[\vref{Ga 1:7}] ait un autre é., mais il y
\item[\vref{Ga 1:11}] mes frères, que l'E. que j'ai annoncé
\item[\vref{Ga 2:14}] la vérité de l'E., je dis à
\item[\vref{Ep 1:13}] vérité, qui est l'E. de votre salut,
\item[\vref{Ep 3:6}] à sa promesse en Christ par l'E.,
\item[\vref{Ep 6:15}] chaussés, prêts pour l'E. de paix ;
\item[\vref{Ph 1:17}] suis établi pour la défense de l'E.
\item[\vref{Ph 1:27}] manière digne de l'E. de Christ, afin
\item[\vref{Ph 4:3}] avec moi pour l'E., avec Clément, et
\item[\vref{Col 1:23}] de l'espérance de l'E. que vs. avez
\item[\vref{1 Th 2:4}] la prédication de l'E., ainsi ns. parlons
\item[\vref{1 Th 3:2}] compagnon d'œuvre ds l'E. de Christ, pour
\item[\vref{1 Ti 1:11}] selon l'E. de la gloire du Dieu béni,
\item[\vref{1 Pi 4:17}] n'obéissent pas à l'E. de Dieu ?
\item[\vref{Ap 14:6}] ciel, il avait l'E. éternel pour évangéliser
\end{listverse}

\ConcordanceEntry{Eve}
\vspace{-2mm}
\begin{listverse}
\item[\vref{Ge 3:20}] appela sa fem. E., parce qu'elle a
\item[\vref{Ge 4:1}] Or Adam connut E. sa fem. et
\item[\vref{2 Co 11:3}] le serpent séduisit E. par sa ruse,
\item[\vref{1 Ti 2:13}] Adam a été formé le premier, E. ensuite.
\end{listverse}

\ConcordanceEntry{Evêque}
\vspace{-2mm}
\begin{listverse}
\item[\vref{Ac 20:28}] vs. a établis é., pour paître l'Eglise
\item[\vref{1 Ti 3:1}] désire la charge d'é., il désire une
\item[\vref{1 Ti 3:2}] il faut que l'é. soit irrépréhensible, mari
\item[\vref{Tit 1:7}] il faut que l'é. soit irrépréhensible, com.
\item[\vref{1 Pi 2:25}] le Pasteur et l'E. de vos âmes.
\end{listverse}

\ConcordanceEntry{Exalter}
\vspace{-2mm}
\begin{listverse}
\item[\vref{Ex 15:2}] le Dieu de mon père, je l'e.
\item[\vref{Ps 30:2}] Yahweh, je t'e. parce que tu
\item[\vref{Ps 99:9}] E. Yahweh, notre Dieu, prosternez-vs. sur la
\item[\vref{Ps 107:32}] Et qu'ils l'e. ds l'assemblée du
\item[\vref{Pr 4:8}] E.-la, et elle t'élèvera ; elle te
\item[\vref{Es 25:1}] mon Dieu ; je t'e., je célébrerai ton
\item[\vref{Es 33:10}] mntnt je serai e., mntnt je serai
\item[\vref{Es 52:13}] il sera fort e., élevé et glorifié.
\item[\vref{Da 4:37}] Nebucadnetsar, je loue, j'e., et je glorifie
\item[\vref{Za 14:10}] et Jérus. sera e. et restera à
\end{listverse}

\ConcordanceEntry{Examiner}
\vspace{-2mm}
\begin{listverse}
\item[\vref{Ps 17:3}] nuit, tu m'as e., tu n'as rien
\item[\vref{Ps 22:18}] un. Eux, ils m'e., ils me regardent.
\item[\vref{Pr 18:17}] mais sa partie adverse vient, et e. le tt.
\item[\vref{So 2:1}] E.-vs., examinez-vs. avec soin ô nation
\item[\vref{Ac 17:11}] promptitude, et ils e. ts les jours
\item[\vref{2 Co 13:5}] E.-vs. vs.-mêmes pour savoir si vs.
\item[\vref{Ga 6:4}] Que chacun e. ses propres œuvres,
\end{listverse}

\ConcordanceEntry{Exaucer}
\vspace{-2mm}
\begin{listverse}
\item[\vref{Ge 30:22}] de Rachel, il l'e. et il ouvrit
\item[\vref{1 R 8:30}] Daigne e. la supplication de ton serviteur et
\item[\vref{Ps 17:6}] t'invoque, car tu m'e., ô Dieu ! Incline
\item[\vref{Ps 28:6}] Yahweh ! Car il e. la voix de
\item[\vref{Ps 34:18}] crient, Yahweh les e. et il les
\item[\vref{Ps 61:6}] ô Dieu ! tu e. mes vœux, et
\item[\vref{Ps 69:18}] car je suis en détresse. Hâte-toi, e.-moi !
\item[\vref{Ps 91:15}] m'invoquera et je l'e. ; je serai avec
\item[\vref{Es 1:15}] je ne les e. pas ; vos mains
\item[\vref{Es 65:24}] crient, je les e. ; et lorsqu'encore ils
\item[\vref{Jé 29:12}] vs. me prierez, et je vs. e.
\item[\vref{Da 9:19}] Seign., e. ! Seign. pardonne ! Seign.
\item[\vref{Jon 2:3}] et il m'a e. ; du sein du
\item[\vref{Za 13:9}] Nom, et je l'e. ; je dirai : C'est
\item[\vref{Mt 6:7}] qu'à force de paroles, ils seront e.
\item[\vref{Jn 9:31}] savons que Dieu n'e. pas les pécheurs ;
\item[\vref{Jn 11:41}] grâces de ce que tu m'as e.
\item[\vref{2 Co 6:2}] dit : Je t'ai e. au temps favorable
\item[\vref{Hé 5:7}] il a été e. à cause de
\item[\vref{1 Jn 5:14}] chose selon sa volonté, il ns. e.
\end{listverse}

\ConcordanceEntry{Excellence}
\vspace{-2mm}
\begin{listverse}
\item[\vref{1 Ch 22:5}] être magnifique en e., en réputation, et
\item[\vref{Job 4:21}] L'e. qui était en eux, n'a-t-elle pas
\item[\vref{Es 4:2}] de grandeur et d'e. pour les réchappés
\item[\vref{2 Co 4:7}] terre, afin que l'e. de cette puissance
\item[\vref{2 Co 12:7}] à cause de l'e. des révélations, il
\item[\vref{Ph 3:8}] en comparaison de l'e. de la connaissance
\end{listverse}

\ConcordanceEntry{Excellent}
\vspace{-2mm}
\begin{listverse}
\item[\vref{Ge 49:11}] et au cep e. le petit de
\item[\vref{Pr 16:16}] combien est-il plus e. que l'argent, d'acquérir
\item[\vref{Ca 7:2}] colliers travaillés de la main d'un e. ouvrier.
\item[\vref{Es 62:8}] plus ton vin e. pour lequel tu
\item[\vref{Hé 1:4}] d'un nom plus e. que le lr.
\item[\vref{Hé 9:11}] un tabernacle plus e. et plus parfait,
\item[\vref{Hé 11:4}] un sacrifice plus e. que Caïn ; et
\item[\vref{Ja 1:17}] ns. est donné d'e. et tt don
\end{listverse}

\ConcordanceEntry{Excès}
\vspace{-2mm}
\begin{listverse}
\item[\vref{De 21:20}] se livre à l'e. et à l'ivrognerie.
\item[\vref{Pr 5:23}] il s'égarera par l'e. de sa folie.
\item[\vref{Pr 20:1}] quiconque en fait e., n'est pas sage.
\item[\vref{Ec 7:17}] point méchant à l'e., et ne sois
\item[\vref{Ez 20:13}] ils profanèrent à l'e. mes sabbats. C'est
\item[\vref{Ac 26:11}] blasphémer. Dans mes e. de fureur contre
\item[\vref{Ga 5:21}] meurtres, l'ivrognerie, les e. de table, et
\item[\vref{1 Ti 3:8}] la duplicité, des e. du vin, d'un
\item[\vref{1 Pi 4:3}] à l'ivrognerie, aux e. du manger et
\item[\vref{Ap 18:3}] sont enrichis par l'e. de son luxe.
\end{listverse}

\ConcordanceEntry{Exciter}
\vspace{-2mm}
\begin{listverse}
\item[\vref{De 9:7}] que tu as e. la colère de
\item[\vref{1 R 21:25}] Yahweh, et sa fem. Jézabel l'y e. ;
\item[\vref{Ps 106:32}] Ils e. aussi sa colère près des eaux
\item[\vref{Pr 10:12}] La haine e. les querelles, mais la charité couvre
\item[\vref{Pr 15:18}] L'hom. furieux e. la querelle, mais
\item[\vref{Lu 23:2}] trouvé cet hom. e. notre nation à
\item[\vref{Ro 10:19}] le premier dit : J'e. votre jalousie par
\item[\vref{Ro 11:11}] Gentils, pour les e. à la jalousie.
\item[\vref{Hé 10:24}] autres pour ns. e. à la charité
\end{listverse}

\ConcordanceEntry{Excuse}
\vspace{-2mm}
\begin{listverse}
\item[\vref{Lu 14:18}] aller le voir ; je te prie, e.-moi.
\item[\vref{Jn 15:22}] ils n'ont pas d'e. de lr. péché.
\end{listverse}

\ConcordanceEntry{Excuser}
\vspace{-2mm}
\begin{listverse}
\item[\vref{Lu 14:18}] ts unanimement à s'e.. Le premier lui
\end{listverse}

\ConcordanceEntry{Exécration}
\vspace{-2mm}
\begin{listverse}
\item[\vref{Jé 42:18}] serez un sujet d'e., d'épouvante, de malédiction
\item[\vref{Jé 44:12}] ils seront en e., en étonnement, en
\item[\vref{Ac 23:21}] un vœu avec e. de serment, de
\end{listverse}

\ConcordanceEntry{Exécuter}
\vspace{-2mm}
\begin{listverse}
\item[\vref{Ge 11:6}] ne les empêchera d'e. ce qu'ils ont
\item[\vref{Ec 8:11}] mauvaises œuvres ne s'e. point promptement, à
\item[\vref{Es 10:23}] des armées, va l'e. au milieu de
\item[\vref{Es 46:11}] un hom. pour e. mon conseil. Oui,
\item[\vref{Jé 1:12}] je me hâte d'e. ma parole.
\item[\vref{Da 11:36}] car ce qui est décrété sera e.
\item[\vref{Ha 1:12}] l'as établi pour e. tes jugements ; et
\item[\vref{Ac 27:43}] Paul, les empêcha d'e. ce conseil. Il
\item[\vref{Ro 9:28}] avec justice ; il e. rapidement sa parole
\end{listverse}

\ConcordanceEntry{Exemple}
\vspace{-2mm}
\begin{listverse}
\item[\vref{Jn 13:15}] ai donné un e., afin que vs.
\item[\vref{1 Co 10:6}] ont été des e. pour ns., afin
\item[\vref{1 Co 10:11}] arrivées pour servir d'e., et elles ont
\item[\vref{1 Ti 1:16}] que je serve d'e. à ceux qui
\item[\vref{Ja 5:10}] frères, prenez pour e. de patience ds
\item[\vref{2 Pi 2:6}] pour être un e. à ceux qui
\item[\vref{Jud 1:7}] mises pour servir d'e., ayant reçu la
\end{listverse}

\ConcordanceEntry{Exercer}
\vspace{-2mm}
\begin{listverse}
\item[\vref{2 S 22:35}] Il e. mes mains au combat, et mes
\item[\vref{1 Ch 16:14}] Dieu ; ses jugements s'e. sur tte la
\item[\vref{Jé 23:5}] il prospérera, et e. le droit et
\item[\vref{Jn 9:39}] ce monde pour e. le jugement, afin
\item[\vref{Ro 7:1}] que la loi e. son pouvoir sur
\item[\vref{Ro 12:8}] que celui qui e. la miséricorde le
\item[\vref{2 Co 9:11}] pleinement enrichis pour e. une parfaite libéralité,
\item[\vref{2 Th 1:8}] de feu, pour e. la vengeance contre
\item[\vref{1 Ti 4:8}] E.-toi à la piété ; car l'exercice
\item[\vref{1 Ti 5:4}] avant tt à e. la piété envers
\item[\vref{Hé 5:14}] ont leurs sens e. à discerner le
\item[\vref{Ap 16:5}] que tu as e. ce jugement.
\end{listverse}

\ConcordanceEntry{Exhortation}
\vspace{-2mm}
\begin{listverse}
\item[\vref{Lu 3:18}] aussi plusieurs autres e., il évangélisait le
\item[\vref{Ac 13:15}] avez qq parole d'e. pour le peuple,
\item[\vref{Ac 20:2}] en adressant aux disciples de nombreuses e.
\item[\vref{1 Ti 4:13}] la lecture, à l'e. et à l'enseignement,
\item[\vref{Hé 12:5}] vs. avez oublié l'e. qui vs. est
\item[\vref{Hé 13:22}] supporter la parole d'e., car je vs.
\end{listverse}

\ConcordanceEntry{Exhorter}
\vspace{-2mm}
\begin{listverse}
\item[\vref{Ac 11:23}] et il les e. ts à demeurer
\item[\vref{Ac 13:43}] Barnabas, qui les e. à persévérer ds
\item[\vref{Ac 15:32}] étaient eux-mêmes prophètes, e. les frères par
\item[\vref{Ro 12:1}] Je vs. e. dc, mes frères, par les compassions
\item[\vref{Ro 12:8}] est appelé à e., qu'il exhorte ; que
\item[\vref{Ro 15:14}] pouvez mm vs. e. les uns les
\item[\vref{Ro 16:17}] Je vs. e., mes frères, à prendre garde à
\item[\vref{1 Co 14:3}] qui prophétise, édifie, e. et console les
\item[\vref{2 Co 5:20}] si Dieu vs. e. par notre service ;
\item[\vref{Col 3:16}] sagesse ; instruisez-vs. et e.-vs. les uns
\item[\vref{1 Th 5:11}] C'est pourquoi e.-vs. réciproquement, et
\item[\vref{1 Ti 2:1}] J'e. dc, avant ttes choses, à faire
\item[\vref{2 Ti 4:2}] non. Reprends, censure, e. avec tte douceur
\item[\vref{Tit 1:9}] soit capable tant d'e. par la saine
\item[\vref{Tit 2:6}] E. aussi les jeunes hommes à être
\item[\vref{Tit 2:15}] Enseigne ces choses, e., et reprends avec
\item[\vref{Hé 3:13}] mais e.-vs. les uns les autres chaque
\item[\vref{Hé 10:25}] de quelques-uns ; mais e.-ns. les uns
\item[\vref{1 Pi 2:11}] bien-aimés, je vs. e., com. étrangers et
\item[\vref{Jud 1:3}] faire pour vs. e. à combattre pour
\end{listverse}

\ConcordanceEntry{Exiger}
\vspace{-2mm}
\begin{listverse}
\item[\vref{De 23:19}] Tu n'e. aucun intérêt à ton frère, ni
\item[\vref{Né 5:7}] Vous prêtez en e. un intérêt chacun
\item[\vref{Job 11:6}] dc que Dieu e. de toi beaucoup
\item[\vref{Mi 6:8}] ce que Yahweh e. de toi : Que
\item[\vref{Lu 3:13}] il lr. dit : N'e. rien au-delà de
\item[\vref{Lu 12:48}] donné, et on e. plus de celui
\end{listverse}

\ConcordanceEntry{Exiler}
\vspace{-2mm}
\begin{listverse}
\item[\vref{Es 11:12}] il rassemblera les e. d'Israël qui auront
\item[\vref{Es 27:13}] ceux qui étaient e. au pays d'Assyrie,
\item[\vref{Es 56:8}] qui rassemble les e. d'Israël : Je réunirai
\end{listverse}

\ConcordanceEntry{Exister}
\vspace{-2mm}
\begin{listverse}
\item[\vref{Ps 104:33}] chanterai à mon Dieu tant que j'e.
\item[\vref{Pr 8:25}] que les collines e., j'ai été engendrée.
\item[\vref{Ec 6:10}] Ce qui e. a déjà été appelé par son
\item[\vref{Da 12:1}] que les nations e. jusqu'à ce temps-là.
\item[\vref{Ro 13:1}] les autorités qui e. ont été instituées
\item[\vref{Hé 11:6}] croie que Dieu e., et qu'il est
\item[\vref{Ap 4:11}] ta volonté qu'elles e. et qu'elles ont
\end{listverse}

\ConcordanceEntry{Exorciste}
\vspace{-2mm}
\begin{listverse}
\item[\vref{Ac 19:13}] Alors quelques e. Juifs ambulants essayèrent
\end{listverse}

\ConcordanceEntry{Expiation}
\vspace{-2mm}
\begin{listverse}
\item[\vref{Ex 30:10}] de l'offrande pour l'e. faite pour les
\item[\vref{Lé 4:25}] de l'offrande pour l'e., et le mettra
\item[\vref{Lé 6:18}] loi du sacrifice d'e.. L'offrande pour l'expiation
\item[\vref{Lé 16:15}] est l'offrande pour l'e., et il apportera
\item[\vref{No 28:22}] en sacrifice pour l'e., afin de faire
\item[\vref{Jos 22:27}] et nos sacrifices d'e. et d'offrande de
\item[\vref{2 Ch 29:24}] et offrirent en e. lr. sang vers
\item[\vref{Né 10:33}] pour les sacrifices d'e. afin de faire
\item[\vref{Es 27:9}] C'est pourquoi l'e. de l'iniquité de
\end{listverse}

\ConcordanceEntry{Expiatoire}
\vspace{-2mm}
\begin{listverse}
\item[\vref{Esd 6:17}] boucs com. victimes e. pour tt Israël,
\item[\vref{Ps 40:7}] holoc. ni victime e. pour le péché.
\item[\vref{Ez 45:25}] le mm sacrifice e., le mm holoc.
\end{listverse}

\ConcordanceEntry{Expier}
\vspace{-2mm}
\begin{listverse}
\item[\vref{De 21:8}] le meurtre sera e. pour eux.
\item[\vref{La 4:22}] ton iniquité est e. ; il ne t'enverra
\end{listverse}

\ConcordanceEntry{Expirer}
\vspace{-2mm}
\begin{listverse}
\item[\vref{Ge 6:17}] ce qui est sur la terre e.
\item[\vref{No 17:13}] de Yahweh, meurt. Serons-ns. ts entièrement e. ?
\item[\vref{Job 3:11}] Pourquoi n'ai-je pas e. aussitôt que je
\item[\vref{Job 10:18}] la matrice ? J'aurais e., et aucun œil
\item[\vref{Job 14:10}] sa force ; il e. et puis où
\item[\vref{Job 36:12}] par l'épée, ils e. ds lr. aveuglement.
\end{listverse}

\ConcordanceEntry{Expliquer}
\vspace{-2mm}
\begin{listverse}
\item[\vref{Ge 41:15}] personne ne peut l'e. ; or j'ai appris
\item[\vref{De 1:5}] dc commença à e. cette loi, de
\item[\vref{Jg 14:14}] ne purent pas e. l'énigme.
\item[\vref{2 Ch 9:2}] Salomon lui e. tt ce qu'elle
\item[\vref{Da 5:12}] les songes, pour e. les questions obscures
\item[\vref{Mc 4:34}] en particulier, il e. tt à ses
\item[\vref{Lu 24:27}] prophètes, il lr. e. ds ttes les
\item[\vref{Hé 5:11}] sont difficiles à e., parce que vs.
\end{listverse}

\ConcordanceEntry{Exploit}
\vspace{-2mm}
\begin{listverse}
\item[\vref{No 31:14}] revenaient de cet e. de guerre.
\item[\vref{Ps 60:14}] ns. ferons des e., et il foulera
\end{listverse}

\ConcordanceEntry{Extérieur}
\vspace{-2mm}
\begin{listverse}
\item[\vref{Ez 40:17}] ds le parvis e., où se trouvaient
\item[\vref{Mt 23:26}] plat, afin que l'e. aussi devienne net.
\item[\vref{Ac 21:5}] leurs enfants, jusqu'à l'e. de la ville.
\item[\vref{2 Co 4:16}] si notre hom. e. se détruise, toutefois
\item[\vref{Tit 2:3}] âgées règlent lr. e. d'une manière convenable
\item[\vref{Ap 11:2}] côté le parvis e. du temple, et
\end{listverse}

\ConcordanceEntry{Exterminer}
\vspace{-2mm}
\begin{listverse}
\item[\vref{Ge 6:7}] Et Yahweh dit : J'e. de la face
\item[\vref{Ge 9:11}] ne sera plus e. par les eaux
\item[\vref{Esd 9:14}] ns., jusqu'à ns. e., sans aucun reste
\item[\vref{Est 3:6}] Haman chercha à e. ts les Juifs,
\item[\vref{Da 11:44}] pour détruire et e. beaucoup de gens.
\end{listverse}

\ConcordanceEntry{Extrémité}
\vspace{-2mm}
\begin{listverse}
\item[\vref{1 S 2:10}] Yahweh jugera les e. de la terre.
\item[\vref{Né 1:9}] été chassés jusqu'à l'e. du ciel, je
\item[\vref{Ps 2:8}] héritage, et les e. de la terre
\item[\vref{Ps 19:5}] est allée jusqu'aux e. du monde. Il
\item[\vref{Ps 19:7}] se lève à l'e. des cieux et
\item[\vref{Ps 22:28}] Toutes les e. de la terre
\item[\vref{Ps 59:14}] Jacob et jusqu'aux e. de la terre !
\item[\vref{Ps 65:6}] de ttes les e. lointaines de la
\item[\vref{Ps 98:3}] d'Israël ; ttes les e. de la terre
\item[\vref{Pr 17:24}] fou sont à l'e. de la terre.
\item[\vref{Es 41:9}] t'ai pris aux e. de la terre,
\item[\vref{Jé 31:8}] les rassemblerai des e. de la terre ;
\item[\vref{Mc 13:27}] quatre vents, de l'e. de la terre
\item[\vref{Ac 1:8}] Samarie, et jusqu'aux e. de la terre.
\item[\vref{2 Co 4:8}] réduits entièrement à l'e. ; ds la perplexité,
\end{listverse}

\ConcordanceEntry{Ezéchias}
\vspace{-2mm}
\begin{listverse}
\item[\vref{2 R 18:1}] d'Ela, roi d'Israël, E., fils d'Achaz, roi
\end{listverse}
\begin{legend}
\NoAutoSpaceBeforeFDP{
\item Fils d'Achaz, roi de Juda : 2 R 16:20; 18:1-5
\item Consécration du temple : 2 Ch 29:5-36
\item Rétablissement de la Pâque : 2 Ch 30
\item L'ange de Yahweh camp des Assyriens : 2 R 18:13; 19:35-36
\item E guérit par Yahweh : 2 R 20:7
}
\end{legend}

\ConcordanceEntry{Ezéchiel}
\vspace{-2mm}
\begin{listverse}
\item[\vref{Ez 1:3}] vint expressément à E., le prêtre, fils
\end{listverse}
\begin{legend}
\NoAutoSpaceBeforeFDP{
\item Prêtre et prophète : Ez 1:3; 3:11; 33:7
\item Visions divines et jugements : Ez 5:5-17; 8:14-17; 9:9; 25:35
\item Vision des ossements desséchés : Ez 37:1-14
\item Restauration d'Israël : Ez 36:25-27; 37:15-28
\item Le futur temple : Ez 40-48
}
\end{legend}

\ConcordanceEntry{Fable}
\vspace{-2mm}
\begin{listverse}
\item[\vref{1 Ti 1:4}] pas s'adonner aux f. et aux généalogies
\item[\vref{2 Ti 4:4}] vérité et se tourneront vers les f.
\item[\vref{Tit 1:14}] s'attachent pas aux f. judaïques et aux
\item[\vref{2 Pi 1:16}] en suivant des f. composées avec artifice,
\end{listverse}

\ConcordanceEntry{Face}
\vspace{-2mm}
\begin{listverse}
\item[\vref{Ge 17:1}] Marche dvt ma f., et sois intègre.
\item[\vref{Ge 32:30}] j'ai vu Dieu f. à face, et
\item[\vref{Ex 20:3}] n'auras point d'autres dieux dvt ma f.
\item[\vref{Ex 23:15}] nul ne se présentera dvt ma f. à vide.
\item[\vref{Ex 33:11}] parlait à Moïse f. à face, com.
\item[\vref{Ex 33:20}] pas voir ma f., car nul hom.
\item[\vref{No 6:25}] fasse luire sa f. sur toi, et
\item[\vref{No 6:26}] Yahweh tourne sa f. vers toi, et
\item[\vref{De 5:4}] Yahweh vs. parla f. à face sur
\item[\vref{De 34:10}] prophète com. Moïse, que Yahweh connaissait f. à face ;
\item[\vref{Jg 6:22}] car j'ai vu l'Ange de Yahweh f. à face.
\item[\vref{2 S 21:1}] David chercha la f. de Yahweh, et
\item[\vref{1 R 8:23}] marchent dvt ta f. de tt lr.
\item[\vref{1 Ch 16:11}] et sa force, cherchez continuellement sa f. !
\item[\vref{Job 1:11}] s'il ne te maudit pas en f.
\item[\vref{Job 33:26}] fera voir sa f. avec joie, et
\item[\vref{Ps 17:15}] je verrai ta f., et je me
\item[\vref{Ps 27:9}] cache point ta f., ne rejette point
\item[\vref{Ps 51:11}] Détourne ta f. de mes péchés,
\item[\vref{Ps 80:4}] fais briller ta f. ! Et ns. serons
\item[\vref{Ps 105:4}] et sa puissance, cherchez continuellement sa f. !
\item[\vref{Es 54:8}] moment caché ma f., mais j'aurai compassion
\item[\vref{Es 59:2}] vs. cachent sa f., afin qu'il ne
\item[\vref{Ez 1:6}] d'eux avait quatre f., et chacun avait
\item[\vref{Ez 3:8}] Voici, j'endurcirai ta f. contre leurs faces,
\item[\vref{Ez 10:14}] animal avait quatre f. : La première face
\item[\vref{Da 9:3}] je tournai ma f. vers le Seign.
\item[\vref{Da 9:17}] fais briller ta f. sur ton sanctuaire
\item[\vref{Os 5:15}] qu'ils cherchent ma f.. Ils me chercheront
\item[\vref{Za 2:13}] taise dvt la f. de Yahweh ! Car
\item[\vref{Mt 18:10}] voient continuellement la f. de mon Père
\item[\vref{Lu 7:27}] messager dvt ta f., et il préparera
\item[\vref{Jn 5:37}] voix, vs. n'avez jamais vu sa f.,
\item[\vref{1 Co 13:12}] alors ns. verrons f. à face. Aujourd'hui
\item[\vref{2 Co 2:10}] vs., dvt la f. de Christ,
\item[\vref{Ga 2:11}] lui résistai en f. parce qu'il méritait
\item[\vref{Ap 4:7}] animal avait la f. com. un hom. ;
\item[\vref{Ap 22:4}] ils verront sa f., et son Nom
\end{listverse}

\ConcordanceEntry{Facile}
\vspace{-2mm}
\begin{listverse}
\item[\vref{1 S 18:23}] Pensez-vs. qu'il soit f. de devenir le
\item[\vref{Mc 10:25}] Il est plus f. à un chameau
\end{listverse}

\ConcordanceEntry{Faible}
\vspace{-2mm}
\begin{listverse}
\item[\vref{Jg 16:7}] sèches, je deviendrais f. et je serais
\item[\vref{Es 25:4}] la force du f., la force du
\item[\vref{Joë 3:10}] Et que le f. dise : Je suis
\item[\vref{Am 7:2}] Comment Jacob subsistera-t-il ? Car il est f.
\item[\vref{Za 12:8}] et le plus f. parmi eux sera
\item[\vref{Mt 26:41}] est prompt, mais la chair est f.
\item[\vref{Mc 14:38}] est prompt, mais la chair est f.
\item[\vref{Ro 4:19}] Et, n'étant pas f. ds la foi,
\item[\vref{Ro 14:1}] celui qui est f. en la foi,
\item[\vref{Ro 14:4}] mm ce chrétien f. sera affermi, car
\item[\vref{1 Co 1:27}] choisi les choses f. de ce monde
\item[\vref{1 Co 8:7}] lr. conscience, étant f., est souillée.
\item[\vref{1 Co 8:11}] frère, qui est f., et pour lequel
\item[\vref{1 Co 9:22}] J'ai été f. avec les faibles, afin de gagner
\item[\vref{2 Co 10:10}] de corps est f., et la parole
\item[\vref{2 Co 11:29}] Qui est a., que je ne sois aussi affaibli ?
\item[\vref{2 Co 12:10}] qnd je suis f., c'est alors que
\item[\vref{2 Co 13:3}] qui n'est pas f. envers vs., mais
\item[\vref{1 Th 5:14}] abattu, supportez les f., et soyez patients
\end{listverse}

\ConcordanceEntry{Faiblesse}
\vspace{-2mm}
\begin{listverse}
\item[\vref{Ez 16:30}] Quelle f. de cœur tu as eu, dit
\item[\vref{Mt 8:17}] a pris nos f. et a porté
\item[\vref{Ro 8:26}] aide ds notre f., car ns. ne
\item[\vref{Ro 14:21}] chute, ou de scandale ou de f.
\item[\vref{1 Co 1:25}] hommes, et la f. de Dieu est
\item[\vref{1 Co 2:3}] vs. ds la f., ds la crainte,
\item[\vref{1 Co 15:43}] est semé en f., il ressuscite plein
\item[\vref{2 Co 12:9}] s'accomplit ds la f.. Je me glorifierai
\item[\vref{2 Co 13:4}] été crucifié par f., il est néanmoins
\item[\vref{Hé 7:18}] cause de sa f. et de son
\end{listverse}

\ConcordanceEntry{Faim}
\vspace{-2mm}
\begin{listverse}
\item[\vref{Ex 16:3}] faire mourir de f. tte cette assemblée.
\item[\vref{De 8:3}] t'a laissé avoir f., mais il t'a
\item[\vref{2 S 17:29}] souffrir de la f., de la fatigue
\item[\vref{Né 9:15}] qnd ils avaient f., et tu fis
\item[\vref{Ps 50:12}] Si j'avais f., je ne t'en
\item[\vref{Pr 10:3}] du juste avoir f., mais il repousse
\item[\vref{Pr 25:21}] te hait a f., donne-lui à manger
\item[\vref{Pr 27:7}] l'âme qui a f., tte chose amère
\item[\vref{Es 29:8}] celui qui a f. rêve qu'il mange,
\item[\vref{Es 49:10}] Ils n'auront pas f. et ils n'auront
\item[\vref{Es 58:7}] celui qui a f. ? Et que tu
\item[\vref{La 2:19}] qui meurent de f. aux coins de
\item[\vref{Ez 18:7}] celui qui a f. et qui couvre
\item[\vref{Mt 4:2}] et quarante nuits, finalement il eut f.
\item[\vref{Mt 5:6}] ceux qui ont f. et soif de
\item[\vref{Mt 12:1}] disciples, qui avaient f., se mirent à
\item[\vref{Mt 12:3}] qnd il eut f., lui et ceux
\item[\vref{Mt 25:42}] Car j'ai eu f., et vs. ne
\item[\vref{Lu 6:3}] qnd il eut f., lui et ceux
\item[\vref{Lu 6:21}] vs. qui avez f. mntnt, car vs.
\item[\vref{Lu 15:17}] abondance, et moi je meurs de f. !
\item[\vref{Jn 6:35}] moi n'aura jamais f. ; et celui qui
\item[\vref{Ro 12:20}] ton ennemi a f., donne-lui à manger ;
\item[\vref{1 Co 4:11}] ns. souffrons la f., la soif, la
\item[\vref{1 Co 11:21}] et l'un a f. tandis que l'autre
\item[\vref{2 Co 11:27}] veilles, ds la f., ds la soif,
\item[\vref{Ph 4:12}] et à avoir f., à être ds
\item[\vref{Ap 7:16}] Ils n'auront plus f. ni soif, et
\end{listverse}

\ConcordanceEntry{Faire}
\vspace{-2mm}
\begin{listverse}
\item[\vref{Ex 18:18}] tu ne saurais f. cela toi seul.
\item[\vref{Ex 19:8}] en disant : Nous f. tt ce que
\item[\vref{Lé 18:3}] Vous ne f. point ce qui se fait ds
\item[\vref{No 14:16}] le pouvoir de f. entrer ce peuple
\item[\vref{No 23:19}] dit, ne le f.-t-il pas ? Ce
\item[\vref{Jos 1:7}] prennes garde de f. selon tte la
\item[\vref{Jg 6:39}] je voudrais seulement f. encore une épreuve
\item[\vref{Jg 21:25}] en Israël. L'hom. f. ce qui lui
\item[\vref{Ru 3:4}] dira ce que tu auras à f.
\item[\vref{1 S 2:8}] fumier, pour le f. asseoir avec les
\item[\vref{1 S 3:11}] Voici, je vais f. une chose en
\item[\vref{2 S 12:23}] jeûnerais-je ? Puis-je le f. revenir ? J'irai vers
\item[\vref{1 R 8:54}] eut achevé de f. cette prière et
\item[\vref{1 Ch 21:1}] incita David à f. le dénombrement d'Israël.
\item[\vref{Esd 4:5}] des conseillers pour f. échouer lr. projet,
\item[\vref{Né 1:9}] choisi pour y f. habiter mon Nom.
\item[\vref{Est 6:6}] dit : Que faudrait-il f. à un hom.
\item[\vref{Job 9:12}] qui le lui f. rendre ? Et qui
\item[\vref{Ps 56:5}] Que peuvent me f. les hommes ?
\item[\vref{Ps 115:3}] au ciel, il f. tt ce qu'il
\item[\vref{Ps 143:10}] Enseigne-moi à f. ta volonté ! Car
\item[\vref{Ec 1:9}] ce qui s'est f., c'est ce qui
\item[\vref{Ec 3:11}] Il a f. que ttes choses sont belles en
\item[\vref{Ec 9:10}] main trouve à f., fais-le selon ton
\item[\vref{Es 1:17}] Apprenez à bien f., recherchez la droiture,
\item[\vref{Es 5:4}] avait-il encore à f. à ma vigne
\item[\vref{Es 26:12}] ce que ns. f., c'est toi qui
\item[\vref{Jé 32:35}] de Ben-Hinnom, pour f. passer par le
\item[\vref{Jé 39:16}] Voici, je vais f. venir mes paroles
\item[\vref{Da 2:26}] capable de me f. connaître le songe
\item[\vref{Da 10:21}] je veux te f. connaître ce qui
\item[\vref{Ha 1:5}] Car je vais f. en vos jours
\item[\vref{Za 8:16}] que vs. devez f. : Que chacun dise
\item[\vref{Mt 19:16}] bon, quel bien f.-je pour avoir
\item[\vref{Mc 3:4}] Est-il permis de f. du bien les
\item[\vref{Mc 6:39}] commanda de les f. ts asseoir par
\item[\vref{Lu 5:34}] lr. répondit : Pouvez-vs. f. jeûner les amis
\item[\vref{Lu 10:37}] lui dit : Va, et toi aussi, f. de mm.
\item[\vref{Lu 13:32}] et j'achève de f. des guérisons aujourd'hui
\item[\vref{Jn 4:34}] nourriture est de f. la volonté de
\item[\vref{Jn 5:19}] ne peut rien f. de lui-mm, il
\item[\vref{Jn 6:28}] dc : Que devons-ns. f. pour accomplir les
\item[\vref{Jn 12:7}] lui dit : Laisse-la f. ; elle l'a gardé
\item[\vref{Jn 13:27}] Ce que tu f., fais-le promptement.
\item[\vref{Ac 2:37}] aux autres apôtres : Hommes frères, que f.-ns. ?
\item[\vref{Ac 19:36}] et ne rien f. avec précipitation.
\item[\vref{Ro 7:15}] ce que je f., puisque je ne
\item[\vref{Ro 11:32}] rébellion, afin de f. miséricorde à ts.
\item[\vref{1 Co 9:22}] Je me suis f. tt à ts,
\item[\vref{Ga 6:9}] lassons pas de f. le bien ; car
\item[\vref{Ep 3:20}] avec efficacité, peut f. infiniment au-delà de
\item[\vref{Col 3:17}] ou en œuvre, f. tt au Nom
\item[\vref{2 Th 3:13}] frères, ne vs. lassez pas de f. le bien.
\item[\vref{Ja 4:17}] celui qui sait f. le bien, et
\item[\vref{Ap 9:10}] le pouvoir de f. du mal aux
\item[\vref{Ap 19:10}] Garde-toi de le f. ! Je suis ton
\end{listverse}

\ConcordanceEntry{Fait (un)}
\vspace{-2mm}
\begin{listverse}
\item[\vref{Ps 145:4}] tes œuvres, et publie tes hauts f. !
\item[\vref{Ps 145:5}] et de tes f. merveilleux !
\item[\vref{Ps 150:2}] pour ses hauts f. ! Louez-le selon la
\item[\vref{Ga 4:24}] Ces f. ont une valeur allégorique, car ces
\end{listverse}

\ConcordanceEntry{Famille}
\vspace{-2mm}
\begin{listverse}
\item[\vref{Ge 12:3}] et ttes les f. de la terre
\item[\vref{Jos 2:18}] et tte la f. de ton père.
\item[\vref{Ps 68:7}] Dieu donne une f. à ceux qui
\item[\vref{Es 60:22}] La petite f. deviendra un millier
\item[\vref{Jé 3:14}] ville, deux d'une f., et je vs.
\item[\vref{Za 12:12}] le deuil, chaque f. à part : La
\item[\vref{Mt 21:33}] un père de f. qui planta une
\item[\vref{Mt 24:43}] un père de f. savait à quelle
\item[\vref{Ac 7:14}] et tte sa f., composée de soixante-quinze
\item[\vref{Ac 16:15}] baptisée, avec sa f., elle ns. fit
\item[\vref{Ac 16:31}] tu seras sauvé, toi et ta f.
\item[\vref{1 Ti 5:4}] envers lr. propre f., et à rendre
\item[\vref{1 Ti 5:8}] ceux de sa f., il a renié
\item[\vref{Hé 11:7}] conservation de sa f. ; et par cette
\end{listverse}

\ConcordanceEntry{Famine}
\vspace{-2mm}
\begin{listverse}
\item[\vref{Ge 12:10}] Mais la f. étant survenue ds le pays, Abram
\item[\vref{Ge 47:20}] parce que la f. les pressait. Et
\item[\vref{2 S 21:1}] de David, une f. qui dura trois
\item[\vref{2 S 24:13}] Sept ans de f. sur ton pays,
\item[\vref{1 R 18:2}] alors une grande f. en Samarie.
\item[\vref{2 R 8:1}] a appelé la f., et mm elle
\item[\vref{Job 5:20}] En temps de f. il te garantira
\item[\vref{Ps 33:19}] et les fasse vivre durant la f.
\item[\vref{Ps 105:16}] appela aussi la f. sur la terre,
\item[\vref{Jé 15:2}] destinés à la f. iront à la
\item[\vref{La 4:9}] morts par la f., qui eux sont
\item[\vref{Ez 5:12}] consumé par la f. au milieu de
\item[\vref{Am 4:6}] ai envoyé la f. ds ttes vos
\item[\vref{Am 8:11}] où j'enverrai la f. ds le pays ;
\item[\vref{Mt 24:7}] y aura des f., des pestes, et
\item[\vref{Lu 4:25}] eut une grande f. ds tt le
\item[\vref{Lu 15:14}] gaspillé, une grande f. survint ds ce
\item[\vref{Ac 11:28}] l'Esprit qu'une grande f. devait arriver sur
\item[\vref{Ro 8:35}] persécution, ou la f., ou la nudité,
\item[\vref{Ap 6:8}] l'épée, par la f., par la mortalité,
\item[\vref{Ap 18:8}] deuil, et la f., viendront en un
\end{listverse}

\ConcordanceEntry{Fantôme}
\vspace{-2mm}
\begin{listverse}
\item[\vref{Mt 14:26}] dirent : C'est un f. ! Et, ds lr.
\end{listverse}

\ConcordanceEntry{Fardeau}
\vspace{-2mm}
\begin{listverse}
\item[\vref{Ge 49:15}] épaule sous le f., il s'assujettit à
\item[\vref{2 Ch 34:13}] qui portaient les f., et dirigeaient ts
\item[\vref{Né 13:19}] d'empêcher l'entrée des f. le jour du
\item[\vref{Ps 38:5}] com. un pesant f., au-delà de mes
\item[\vref{Ps 66:11}] mis sur nos reins un pesant f.
\item[\vref{Ps 81:7}] son épaule du f., et ses mains
\item[\vref{Jé 17:21}] ne portez aucun f. le jour du
\item[\vref{Jé 17:22}] vos maisons aucun f. le jour du
\item[\vref{Jé 23:33}] Quel est le f. de Yahweh ? Tu
\item[\vref{Ez 12:10}] Seign. Yahweh : Ce f. dont je suis
\item[\vref{Mt 11:30}] doux et mon f. est léger.
\item[\vref{Mt 23:4}] lient ensemble des f. pesants et insupportables
\item[\vref{Ga 6:2}] Portez les f. les uns des
\item[\vref{Ga 6:5}] Car chacun portera son propre f.
\item[\vref{Hé 12:1}] témoins, rejetons tt f., et le péché
\end{listverse}

\ConcordanceEntry{Farine}
\vspace{-2mm}
\begin{listverse}
\item[\vref{Ge 18:6}] de fleur de f., pétris-les, et fais
\item[\vref{Ex 29:2}] feras de fine f. de froment.
\item[\vref{Lé 2:1}] sera de fine f. ; il versera de
\item[\vref{1 R 17:14}] Dieu d'Israël : La f. qui est ds
\item[\vref{2 R 4:41}] Apportez-moi de la f. ; et il en
\item[\vref{Ap 18:13}] fine fleur de f., du blé, des
\end{listverse}

\ConcordanceEntry{Fatigue}
\vspace{-2mm}
\begin{listverse}
\item[\vref{Ex 18:8}] et tte la f. qu'ils avaient soufferte
\item[\vref{Jos 7:3}] battront Aï. Ne f. pas tt le
\item[\vref{Jg 4:21}] était accablé de f.. Et ainsi il
\item[\vref{2 S 17:29}] faim, de la f. et de la
\item[\vref{Ec 10:15}] de l'insensé le f., parce qu'il ne
\item[\vref{Es 40:28}] il ne se f. point, il ne
\item[\vref{Mal 1:13}] ô que de f. ! Et vs. soufflez
\item[\vref{Lu 7:6}] Seign., ne te f. pas ; car je
\end{listverse}

\ConcordanceEntry{Fatiguer}
\vspace{-2mm}
\begin{listverse}
\item[\vref{Ge 25:29}] arriva des champs, et il était f.
\item[\vref{Jg 8:4}] étaient avec lui, f., mais poursuivant toujours
\item[\vref{Ec 8:17}] a beau se f. à chercher, il
\item[\vref{Es 5:27}] Nul n'est f., nul ne chancelle
\item[\vref{Es 28:12}] celui qui est f. ; voici le soulagement !
\item[\vref{Es 57:10}] Tu te f. par la longueur du chemin, et
\item[\vref{Jé 2:24}] pas à se f. ; ils la trouvent
\item[\vref{Jé 12:13}] ils se sont f. sans profit. Soyez
\item[\vref{Mal 2:17}] Vous f. Yahweh par vos paroles, et vs.
\item[\vref{Mt 11:28}] ts qui êtes f. et chargés, et
\item[\vref{Jn 4:6}] Jacob ; et Jésus, f. du voyage, se
\end{listverse}

\ConcordanceEntry{Fausseté}
\vspace{-2mm}
\begin{listverse}
\item[\vref{Job 13:7}] Dieu, direz-vs. qq f. pour lui ?
\item[\vref{Ps 24:4}] âme à la f., et qui ne
\item[\vref{Es 28:15}] ns. ns. sommes cachés sous la f.
\item[\vref{Jé 3:10}] cœur ; c'est avec f. qu'elle l'a fait,
\item[\vref{Jé 8:8}] des scribes est une plume de f.
\item[\vref{Da 6:4}] occasion, ni aucune f., parce qu'il était
\item[\vref{Da 11:27}] ils parleront avec f.. Mais cela ne
\item[\vref{Os 7:1}] ont pratiqué la f. ; le voleur entre,
\end{listverse}

\ConcordanceEntry{Faute}
\vspace{-2mm}
\begin{listverse}
\item[\vref{Ge 41:9}] rappellerai aujourd'hui le souvenir de mes f.
\item[\vref{Lé 5:26}] que soit la f. dont il se
\item[\vref{No 5:31}] sera exempt de f., mais cette fem.
\item[\vref{De 32:5}] n'est point la f. ; la faute est
\item[\vref{Esd 10:11}] confession de votre f. à Yahweh, le
\item[\vref{Né 6:13}] à commettre cette f., afin qu'ils aient
\item[\vref{Job 19:4}] j'ai péché, la f. serait pour moi.
\item[\vref{Job 31:19}] le malheureux périr f. de vêtements, le
\item[\vref{Pr 10:21}] instruisent plusieurs, mais les insensés mourront f. de sens.
\item[\vref{Pr 11:14}] peuple tombe par f. de prudence, mais
\item[\vref{Pr 26:20}] Le feu s'éteint f. de bois, ainsi
\item[\vref{Es 50:2}] poissons se corrompent f. d'eau, et ils
\item[\vref{Da 6:4}] ne se trouvait en lui ni f. ni vice.
\item[\vref{2 Co 7:12}] a commis la f., ni à cause
\item[\vref{Ga 6:1}] surpris en qq f., vs. qui êtes
\end{listverse}

\ConcordanceEntry{Faux dieux}
\vspace{-2mm}
\begin{listverse}
\item[\vref{2 S 5:21}] laissèrent là leurs faux dieux que David et
\item[\vref{2 Ch 24:18}] Asherah et les faux dieux ; et la colère
\item[\vref{Ps 106:36}] Ils servirent leurs faux dieux qui furent un
\item[\vref{Ps 106:38}] qu'ils sacrifièrent aux faux dieux de Canaan ; et
\item[\vref{Es 46:1}] est renversé ; leurs faux dieux sont mis sur
\item[\vref{Es 57:5}] s'échauffant près des faux dieux, sous tt arbre
\item[\vref{Ez 30:13}] idoles, j'anéantirai les faux dieux de Noph, et
\item[\vref{Mi 1:7}] mettrai ts ses faux dieux en désolation ; parce
\item[\vref{So 1:4}] des prêtres des faux dieux et les prêtres,
\item[\vref{Za 13:2}] les noms des faux dieux, et on n'en
\end{listverse}

\ConcordanceEntry{Faux prophètes}
\vspace{-2mm}
\begin{listverse}
\item[\vref{Za 13:2}] du pays les faux prophètes et l'esprit d'impureté.
\item[\vref{Mt 7:15}] Gardez-vs. des faux prophètes. Ils viennent à
\item[\vref{Mt 24:11}] il s'élèvera plusieurs faux prophètes, qui en séduiront
\item[\vref{Mt 24:24}] christs et de faux prophètes, ils feront de
\item[\vref{Lu 6:26}] pères en faisaient de mm aux faux prophètes !
\item[\vref{2 Pi 2:1}] a eu de faux prophètes parmi le peuple,
\item[\vref{1 Jn 4:1}] Dieu, car plusieurs faux prophètes sont venus ds
\end{listverse}

\ConcordanceEntry{Faux témoin}
\vspace{-2mm}
\begin{listverse}
\item[\vref{Ex 23:1}] pour être un faux témoin, afin que violence
\item[\vref{De 19:16}] Quand un faux témoin s'élèvera contre un hom. pour témoigner
\item[\vref{Pr 6:19}] le faux témoin qui profère des mensonges et celui
\item[\vref{Pr 12:17}] juste, mais le faux témoin fait des rapports
\item[\vref{Pr 14:5}] jamais, mais le faux témoin avance volontiers des
\item[\vref{Pr 19:5}] Le faux témoin ne restera pas impuni, et celui
\end{listverse}

\ConcordanceEntry{Faveur}
\vspace{-2mm}
\begin{listverse}
\item[\vref{Lé 8:34}] pour faire la propitiation en votre f.
\item[\vref{Lé 26:45}] souviendrai en lr. f. de la Première
\item[\vref{Est 4:8}] une requête en f. de son peuple.
\item[\vref{Job 13:7}] discours injustes en f. de Dieu, direz-vs.
\item[\vref{Ps 30:8}] Yahweh ! par ta f. tu avais affermi
\item[\vref{Ps 89:18}] notre pouvoir est distingué par ta f.
\item[\vref{Ps 109:4}] n'ai fait que prier en lr. f. !
\item[\vref{Ps 141:5}] me sera une f.. Et qu'il me
\item[\vref{Pr 8:35}] et obtient la f. de Yahweh.
\item[\vref{Pr 11:27}] bien cherche la f., mais le mal
\item[\vref{Pr 12:2}] bien obtient la f. de Yahweh, mais
\item[\vref{Pr 18:22}] il obtient une f. de Yahweh.
\item[\vref{Pr 31:8}] ta bouche en f. du muet, pour
\item[\vref{Ca 7:10}] qui coule en f. de mon bien-aimé
\item[\vref{Es 8:19}] aux morts en f. des vivants ?
\item[\vref{Jé 14:11}] N'intercède pas en f. de ce peuple.
\item[\vref{Da 1:9}] trouver à Daniel f. et grâce auprès
\item[\vref{Da 9:20}] mon Dieu, en f. de la sainte
\item[\vref{Za 13:1}] source ouverte en f. de la maison
\item[\vref{Ro 8:27}] il intercède en f. des saints, selon
\item[\vref{1 Co 16:1}] la collecte en f. des saints, faites
\item[\vref{2 Co 8:4}] cette contribution en f. des saints.
\item[\vref{Hé 1:14}] pour servir en f. de ceux qui
\item[\vref{Ap 21:24}] marcheront à la f. de sa lumière,
\end{listverse}

\ConcordanceEntry{Favorable}
\vspace{-2mm}
\begin{listverse}
\item[\vref{Ex 28:38}] dvt Yahweh, pour qu'il lr. soit f.
\item[\vref{1 S 20:12}] et s'il est f. envers David, et
\item[\vref{1 S 25:8}] ds un jour f.. Nous te prions
\item[\vref{2 S 24:23}] Que Yahweh, ton Dieu, te soit f. !
\item[\vref{Esd 8:22}] notre Dieu est f. sur ts ceux
\item[\vref{Ps 69:14}] soit le temps f., ô Dieu ! Par
\item[\vref{Ps 77:8}] pour toujours ? Ne me sera-t-il plus f. ?
\item[\vref{Ps 85:2}] tu as été f. à ta terre,
\item[\vref{Jé 24:6}] regarderai d'un œil f., et je les
\item[\vref{Jé 42:2}] notre supplication soit f. dvt toi ! Intercède
\item[\vref{Mt 26:16}] cherchait une occasion f. pour le livrer.
\item[\vref{Ro 1:10}] Dieu, qq moyen f. pour aller vers
\item[\vref{2 Co 6:2}] exaucé au temps f. et t'ai secouru
\item[\vref{2 Ti 4:2}] en tte occasion, f. ou non. Reprends,
\end{listverse}

\ConcordanceEntry{Favoritisme}
\vspace{-2mm}
\begin{listverse}
\item[\vref{1 Pi 1:17}] de chacun, sans f., conduisez-vs. avec crainte
\end{listverse}

\ConcordanceEntry{Fécond}
\vspace{-2mm}
\begin{listverse}
\item[\vref{Ge 1:22}] en disant : Soyez f., multipliez, et remplissez
\item[\vref{Ge 9:1}] lr. dit : Soyez f., multipliez, et remplissez
\item[\vref{Ge 28:3}] bénisse, te rende f. et te multiplie,
\item[\vref{Ge 35:11}] Dieu Tout-Puissant. Sois f. et multiplie ; une
\item[\vref{Ge 47:27}] possessions, ils furent f. et multiplièrent beaucoup.
\item[\vref{Ps 105:24}] son peuple très f. et le rendit
\item[\vref{Ez 36:11}] multiplieront et seront f. ; je veux que
\end{listverse}

\ConcordanceEntry{Félix}
\vspace{-2mm}
\begin{listverse}
\item[\vref{Ac 23:24}] ils le mènent sûrement au gouverneur F.. .
\item[\vref{Ac 24:3}] Très excellent F., ns. reconnaissons en
\item[\vref{Ac 24:25}] jugement à venir, F. tt effrayé répondit :
\end{listverse}

\ConcordanceEntry{Femme}
\vspace{-2mm}
\begin{listverse}
\item[\vref{Ge 2:22}] Dieu forma une f. de la côte
\item[\vref{Ge 2:23}] chair ; on l'appellera f., parce qu'elle a
\item[\vref{Ge 3:20}] Adam appela sa f. Eve, parce qu'elle
\item[\vref{Ge 19:26}] Mais la f. de Lot regarda en arrière, et
\item[\vref{Ex 1:21}] Parce que les sages-f. craignirent Dieu, il
\item[\vref{Lé 18:19}] t'approcheras point d'une f. durant son impureté
\item[\vref{No 12:1}] sujet de la f. éthiopienne qu'il avait
\item[\vref{De 22:5}] La f. ne portera point l'habit d'un hom.
\item[\vref{De 25:5}] fils, alors la f. du défunt ne
\item[\vref{Jos 2:1}] la maison d'une f. prostituée, nommée Rahab,
\item[\vref{Jg 4:4}] temps-là, Débora, prophétesse, f. de Lappidoth, était
\item[\vref{Ru 3:11}] tu es une f. vertueuse.
\item[\vref{1 S 1:19}] connut Anne, sa f., et Yahweh se
\item[\vref{1 R 16:31}] qu'il prit pour f. Jézabel, fille d'Ethbaal,
\item[\vref{2 R 22:14}] la prophétesse Hulda, f. de Schallum, fils
\item[\vref{Ps 128:3}] Ta f. est ds ta maison com. une
\item[\vref{Pr 5:18}] réjouis-toi de la f. de ta jeunesse,
\item[\vref{Pr 11:16}] La f. gracieuse obtient de l'honneur, et les
\item[\vref{Pr 14:1}] Toute f. sage bâtit sa maison, mais la
\item[\vref{Pr 18:22}] qui trouve une f. trouve le bonheur
\item[\vref{Pr 19:14}] richesses, mais la f. prudente est un
\item[\vref{Pr 31:10}] Qui trouvera une f. vertueuse ? Car son
\item[\vref{Ec 7:26}] la mort, la f. dont le cœur
\item[\vref{Ec 7:28}] mais pas une f. entre elles ttes.
\item[\vref{Es 3:12}] enfants, et des f. dominent sur lui.
\item[\vref{Es 4:1}] ce jour sept f. saisiront un seul
\item[\vref{Es 49:15}] Une f. peut-elle oublier son enfant qu'elle allaite
\item[\vref{Es 54:6}] t'appelle com. une f. délaissée et à
\item[\vref{Jé 16:2}] prendras pas de f. et tu n'auras
\item[\vref{Ez 18:11}] s'il déshonore la f. de son prochain,
\item[\vref{Ez 24:18}] matin, et ma f. mourut le soir ;
\item[\vref{Os 1:2}] Va, prends une f. prostituée et des
\item[\vref{Mal 2:14}] toi et la f. de ta jeunesse,
\item[\vref{Mt 1:20}] toi Marie, ta f., car l'enfant qu'elle
\item[\vref{Mt 11:11}] sont nés de f., il n'en a
\item[\vref{Mt 15:22}] Et voici, une f. cananéenne, qui venait
\item[\vref{Mt 19:5}] s'attachera à sa f., et les deux
\item[\vref{Mt 22:30}] ne donnera de f. en mariage, mais
\item[\vref{Mt 24:41}] de deux f. qui moudront au moulin, l'une sera
\item[\vref{Mc 12:20}] premier prit une f. et mourut sans
\item[\vref{Mc 14:3}] à table, une f. vint à lui
\item[\vref{Lu 1:42}] bénie entre les f., et béni est
\item[\vref{Lu 7:37}] la ville une f. pécheresse, qui, ayant
\item[\vref{Lu 8:43}] y avait une f. atteinte d'une perte
\item[\vref{Lu 16:18}] Quiconque répudie sa f. et se marie
\item[\vref{Lu 17:32}] Souvenez-vs. de la f. de Lot.
\item[\vref{Jn 2:4}] moi et toi, f. ? Mon heure n'est
\item[\vref{Jn 4:7}] Et une f. samaritaine vint puiser de l'eau, Jésus
\item[\vref{Jn 8:3}] lui amenèrent une f. surprise en adultère ;
\item[\vref{Ro 7:2}] Car la f. qui est soumise à un mari,
\item[\vref{1 Co 7:2}] chacun ait sa f., et que chaque
\item[\vref{1 Co 7:10}] Seign., que la f. ne se sépare
\item[\vref{1 Co 7:39}] La f. est liée par la loi pendant
\item[\vref{1 Co 11:5}] Toute f., au contraire, qui prie, ou qui
\item[\vref{1 Co 11:9}] créé pour la f., mais la fem.
\item[\vref{Ep 5:24}] à Christ, les f. aussi doivent l'être
\item[\vref{1 Ti 2:11}] Que la f. apprenne ds le silence, en tte
\item[\vref{1 Ti 2:14}] séduit, mais la f., ayant été séduite,
\item[\vref{Tit 2:4}] instruisent les jeunes f. à être modestes,
\item[\vref{Hé 11:35}] Des f. recouvrèrent leurs morts par le moyen
\item[\vref{Ap 2:20}] tu laisses cette f. Jézabel, qui se
\item[\vref{Ap 12:1}] le ciel : Une f. revêtue du soleil,
\item[\vref{Ap 17:3}] je vis une f. assise sur une
\item[\vref{Ap 17:6}] je vis cette f. ivre du sang
\item[\vref{Ap 21:9}] montrerai l'Epouse, la f. de l'Agneau.
\end{listverse}

\ConcordanceEntry{Fenêtre}
\vspace{-2mm}
\begin{listverse}
\item[\vref{Ge 6:16}] Tu feras une f. à l'arche, et
\item[\vref{Jos 2:15}] corde par la f. ; car sa maison
\item[\vref{Jg 5:28}] regardait par la f. et s'écriait en
\item[\vref{1 S 19:12}] David par une f., et ainsi il
\item[\vref{2 R 7:2}] Yahweh ferait des f. au ciel, cela
\item[\vref{Ec 12:5}] regardent par les f. sont obscurcis.
\item[\vref{Da 6:10}] maison, où les f. de sa chambre
\item[\vref{2 Co 11:33}] corbeille, par une f. et ainsi j'échappai
\end{listverse}

\ConcordanceEntry{Fer}
\vspace{-2mm}
\begin{listverse}
\item[\vref{De 28:48}] un joug de f. sur ton cou,
\item[\vref{2 R 6:6}] mm endroit, et fit surnager le f.
\item[\vref{Pr 27:17}] Comme le f. aiguise le fer, ainsi l'hom. aiguise
\item[\vref{Ec 10:10}] Si le f. est émoussé, et qu'on n'en ait
\item[\vref{Da 2:33}] jambes étaient de f. et ses pieds
\item[\vref{Da 2:40}] fort com. du f. ; de mm que
\item[\vref{Mc 5:4}] avait eu les f. aux pieds et
\item[\vref{1 Ti 4:2}] ayant lr. propre conscience marquée au f. rouge ;
\item[\vref{Ap 12:5}] un sceptre de f.. Et son enfant
\end{listverse}

\ConcordanceEntry{Ferme}
\vspace{-2mm}
\begin{listverse}
\item[\vref{Ge 49:24}] arc est demeuré f., et ses mains
\item[\vref{De 25:8}] parleront. S'il demeure f., et qu'il dit :
\item[\vref{Jos 3:17}] s'arrêtèrent de pied f. sur le sec,
\item[\vref{1 S 24:21}] royaume d'Israël sera f. entre tes mains.
\item[\vref{1 S 25:28}] d'établir une maison f. à mon seigneur ;
\item[\vref{1 Ch 17:23}] sa maison, soit f. à jamais, et
\item[\vref{Job 11:15}] tache. Tu seras f. et tu ne
\item[\vref{Job 17:9}] le juste tient f. ds sa voie,
\item[\vref{Job 41:14}] Sa chair est f., tt est massif
\item[\vref{Ps 20:9}] ns., ns. tenons f., et restons debout.
\item[\vref{Ps 51:12}] et renouvelle en moi un esprit f.
\item[\vref{Ps 89:22}] Ma main sera f. avec lui, et
\item[\vref{Ps 93:1}] le monde est f., tellement qu'il ne
\item[\vref{Ps 112:7}] son cœur est f., confiant en Yahweh.
\item[\vref{Ps 119:90}] établi la terre, et elle demeure f.
\item[\vref{Ec 1:4}] vient, mais la terre demeure toujours f.
\item[\vref{Jé 8:5}] égarements ? Ils tiennent f. à la tromperie,
\item[\vref{Jé 49:19}] pasteur qui tiendra f. contre moi ?
\item[\vref{Da 6:7}] et un décret f., portant que quiconque,
\item[\vref{Da 12:1}] chef qui tient f. pour les enfants
\item[\vref{Mt 18:16}] ou trois témoins tte parole soit f.
\item[\vref{Ro 5:2}] laquelle ns. tenons f., et ns. ns.
\item[\vref{Ro 14:4}] S'il se tient f. ou s'il bronche,
\item[\vref{1 Co 7:37}] celui qui demeure f. ds son cœur,
\item[\vref{1 Co 15:58}] frères bien-aimés, soyez f., inébranlables, vs. appliquant
\item[\vref{2 Co 1:7}] de vs. est f., sachant que com.
\item[\vref{2 Co 1:24}] puisque vs. demeurez f. ds la foi.
\item[\vref{Ep 6:13}] jour, et tenir f. après avoir tt
\item[\vref{Ph 1:20}] selon ma f. attente et mon espérance, je ne
\item[\vref{2 Ti 2:19}] de Dieu demeure f., ayant ce sceau :
\item[\vref{2 Ti 3:14}] Mais toi, demeure f. ds les choses
\item[\vref{Hé 3:14}] que ns. gardions f. jusqu'à la fin
\item[\vref{Hé 4:14}] les cieux, demeurons f. ds la confession
\item[\vref{Hé 6:18}] ns. ayons une f. consolation, ns. qui
\item[\vref{Hé 11:27}] car il demeura f., com. voyant celui
\item[\vref{2 Pi 1:19}] qui est très f., à laquelle vs.
\item[\vref{Ap 3:11}] tte vitesse. Tiens f. ce que tu
\end{listverse}

\ConcordanceEntry{Fermer}
\vspace{-2mm}
\begin{listverse}
\item[\vref{Ge 7:16}] Noé, puis Yahweh f. l'arche sur lui.
\item[\vref{1 S 12:3}] présents, afin de f. les yeux sur
\item[\vref{Né 4:7}] avait commencé à f. les brèches, ils
\item[\vref{Es 22:22}] aura personne qui f. ; et il fermera,
\item[\vref{La 3:9}] de taille pour f. mes chemins, il
\item[\vref{Mt 23:13}] pharisiens hypocrites ! qui f. le Royaume des
\item[\vref{Lu 4:25}] le ciel fut f. trois ans et
\item[\vref{Tit 1:11}] auxquels il faut f. la bouche, et
\item[\vref{Ap 3:7}] et nul ne f., qui ferme et
\item[\vref{Ap 3:8}] ne peut la f. ; parce que tu
\item[\vref{Ap 11:6}] le pouvoir de f. le ciel, afin
\end{listverse}

\ConcordanceEntry{Fermeté}
\vspace{-2mm}
\begin{listverse}
\item[\vref{Pr 15:22}] a de la f. ds la multitude
\item[\vref{Os 2:22}] t'épouserai par la f., et tu connaîtras
\item[\vref{Col 2:5}] ordre et la f. de votre foi,
\item[\vref{Hé 6:17}] la promesse la f. immuable de sa
\item[\vref{Hé 10:35}] dc pas cette f. que vs. avez
\item[\vref{2 Pi 3:17}] ne veniez à déchoir de votre f.
\end{listverse}

\ConcordanceEntry{Fervent}
\vspace{-2mm}
\begin{listverse}
\item[\vref{Ac 18:25}] du Seign., et f. d'esprit ; il expliquait
\item[\vref{Ro 12:11}] pour autrui ; soyez f. d'esprit ; servez le
\end{listverse}

\ConcordanceEntry{Festin}
\vspace{-2mm}
\begin{listverse}
\item[\vref{Ge 19:3}] lr. fit un f., et fit cuire
\item[\vref{Est 1:5}] roi fît un f. pendant sept jours,
\item[\vref{Job 1:4}] et faisaient des f. les uns chez
\item[\vref{Pr 15:15}] gai, c'est un f. perpétuel.
\item[\vref{Ec 7:2}] une maison de f. ; car c'est là
\item[\vref{Jé 16:8}] une maison de f. pour t'asseoir avec
\item[\vref{Da 5:1}] donna un grand f. à ses grands
\item[\vref{Mt 22:4}] j'ai préparé mon f. ; mes bœufs et
\item[\vref{Mt 23:6}] places ds les f., et les premiers
\item[\vref{Lu 5:29}] fit un grand f. ds sa maison ;
\item[\vref{Lu 14:13}] tu donneras un f., convie les pauvres,
\item[\vref{Ap 19:9}] sont appelés au f. des noces de
\item[\vref{Ap 19:17}] Venez et rassemblez-vs. pour le grand f. de Dieu,
\end{listverse}

\ConcordanceEntry{Festus}
\vspace{-2mm}
\begin{listverse}
\item[\vref{Ac 24:27}] pour successeur Porcius F., qui, voulant faire
\item[\vref{Ac 25:6}] F. ne passa que dix jours parmi
\item[\vref{Ac 25:12}] Alors F. ayant conféré avec le conseil, lui
\item[\vref{Ac 25:24}] Et F. dit : Roi Agrippa, et vs. ts
\end{listverse}

\ConcordanceEntry{Fête}
\vspace{-2mm}
\begin{listverse}
\item[\vref{Né 8:18}] On célébra la f. pendant sept jours,
\item[\vref{Es 1:14}] lunes et vos f. solennelles ; elles sont
\item[\vref{Za 14:16}] pour célébrer la f. des tabernacles.
\item[\vref{Mt 26:5}] pas pendant la f., de peur qu'il
\item[\vref{Jn 7:2}] Or la f. des Juifs, appelée la fête des
\item[\vref{Jn 7:37}] jour de la f., Jésus, se tenant
\item[\vref{Jn 10:22}] on célébrait la f. de la dédicace
\item[\vref{1 Co 5:8}] célébrons dc la f., non avec du
\item[\vref{Col 2:16}] d'un jour de f., ou d'un jour
\end{listverse}

\ConcordanceEntry{Fête solennelle}
\vspace{-2mm}
\begin{listverse}
\item[\vref{Ex 5:1}] me célèbre une fête solennelle ds le désert.
\item[\vref{Ex 23:14}] par année, tu me célébreras une fête solennelle.
\item[\vref{Ex 23:16}] Et la fête solennelle de la moisson des premiers fruits
\item[\vref{Ex 34:18}] Tu garderas la fête solennelle des pains sans
\item[\vref{Ex 34:22}] Tu feras la fête solennelle des semaines au
\item[\vref{Ex 34:25}] sacrifice de la fête solennelle de la Pâque
\item[\vref{Lé 23:34}] mois sera la fête solennelle des tabernacles pendant
\item[\vref{Lé 23:39}] vs. célébrerez la fête solennelle de Yahweh pendant
\item[\vref{1 R 8:65}] Salomon célébra une fête solennelle ; et tt Israël
\item[\vref{2 R 10:20}] dit : Publiez une fête solennelle en l'honneur de
\item[\vref{Esd 3:4}] célébrèrent aussi la fête solennelle des tabernacles, de
\item[\vref{Né 8:14}] tentes pendant la fête solennelle au septième mois.
\item[\vref{Es 30:29}] l'on célèbre une fête solennelle ; vs. aurez le
\item[\vref{La 2:6}] ds Sion la fête solennelle et le sabbat,
\item[\vref{Ez 45:21}] aurez la Pâque, fête solennelle qui durera sept
\item[\vref{Na 2:1}] Juda, célèbre tes fêtes solennelles, accomplis tes vœux ;
\end{listverse}

\ConcordanceEntry{Feu}
\vspace{-2mm}
\begin{listverse}
\item[\vref{Ge 11:3}] très bien au f.. Et la brique
\item[\vref{Ge 22:7}] dit : Voici le f. et le bois,
\item[\vref{Ex 3:2}] une flamme de f., du milieu d'un
\item[\vref{Ex 9:23}] grêle, et le f. se promenait sur
\item[\vref{Ex 19:18}] était descendu en f. ; et sa fumée
\item[\vref{Lé 6:6}] Le f. brûlera continuellement sur l'autel, on ne
\item[\vref{Lé 10:1}] encensoir, mirent du f., et ils posèrent
\item[\vref{No 31:23}] passer par le f., vs. le ferez
\item[\vref{De 4:36}] montré son grand f. sur la terre,
\item[\vref{De 12:3}] vs. brûlerez au f. leurs asheras, vs.
\item[\vref{De 18:10}] passer par le f. son fils ou
\item[\vref{1 R 18:24}] répondra par le f., soit reconnu pour
\item[\vref{2 R 1:10}] Dieu, que le f. descende du ciel
\item[\vref{2 R 1:12}] Dieu, que le f. descende du ciel
\item[\vref{1 Ch 21:26}] l'exauça par le f. envoyé des cieux
\item[\vref{Job 1:16}] et dit : Le f. de Dieu est
\item[\vref{Ps 97:3}] Le f. marche dvt lui, et embrase tt
\item[\vref{Ps 104:4}] des flammes de f. ses serviteurs.
\item[\vref{Pr 6:27}] peut-il prendre du f. ds son sein,
\item[\vref{Pr 26:20}] Le f. s'éteint faute de bois, ainsi qnd
\item[\vref{Es 43:2}] marches ds le f., tu ne seras
\item[\vref{Es 66:24}] point, et lr. f. ne s'éteindra point ;
\item[\vref{Jé 23:29}] pas com. un f., dit Yahweh, et
\item[\vref{Da 3:6}] au milieu de la fournaise de f. ardent.
\item[\vref{Da 3:27}] hommes-là, et le f. n'avait eu aucun
\item[\vref{Joë 2:3}] lui est un f. dévorant, et derrière
\item[\vref{Am 7:4}] jugement par le f.. Et le feu
\item[\vref{Ha 2:13}] travaillent pour le f., et que les
\item[\vref{Mt 3:11}] vs. baptisera du Saint-Esprit et de f.
\item[\vref{Mt 18:9}] jeté ds le f. de la géhenne.
\item[\vref{Mt 25:41}] allez ds le f. éternel, qui a
\item[\vref{Mc 9:22}] jeté ds le f. et ds l'eau
\item[\vref{Lu 3:16}] vs. baptisera du Saint-Esprit et de f.
\item[\vref{Lu 9:54}] commandions que le f. descende du ciel,
\item[\vref{Lu 12:49}] venu jeter un f. sur la terre,
\item[\vref{Ac 2:3}] séparées, com. de f., qui se posèrent
\item[\vref{1 Co 3:13}] manifestée par le f., et le feu
\item[\vref{1 Co 3:15}] sera sauvé, toutefois com. par le f.
\item[\vref{Hé 1:7}] de ses serviteurs des flammes de f.
\item[\vref{Hé 12:29}] est aussi un f. dévorant.
\item[\vref{Ja 3:5}] Voici, un petit f., combien de bois
\item[\vref{Ja 5:3}] chairs com. un f.. Vous avez amassé
\item[\vref{2 Pi 3:7}] réservés pour le f. au jour du
\item[\vref{Jud 1:7}] d'exemple, ayant reçu la punition du f. éternel.
\item[\vref{Jud 1:23}] com. hors du f., et haïssez mm
\item[\vref{Ap 13:13}] faire descendre le f. du ciel sur
\item[\vref{Ap 14:10}] tourmenté ds le f. et le soufre
\item[\vref{Ap 14:18}] autorité sur le f., sortit de l'autel,
\item[\vref{Ap 16:8}] de brûler les hommes par le f.,
\item[\vref{Ap 20:10}] ds l'étang de f. et de soufre,
\end{listverse}

\ConcordanceEntry{Feuille}
\vspace{-2mm}
\begin{listverse}
\item[\vref{Ge 3:7}] cousirent ensemble des f. de figuier, et
\item[\vref{Ge 8:11}] son bec une f. d'olivier qu'elle avait
\item[\vref{Es 64:5}] flétris com. la f., et nos iniquités
\item[\vref{Jé 17:8}] pas, et sa f. restera verte ; il
\item[\vref{Mt 21:19}] trouva que des f., et il lui
\item[\vref{Mt 24:32}] et que ses f. poussent, vs. savez
\item[\vref{Ap 22:2}] mois et les f. de l'arbre servaient
\end{listverse}

\ConcordanceEntry{Fiancé, Fiancée}
\vspace{-2mm}
\begin{listverse}
\item[\vref{Ex 22:16}] une vierge non f., et couche avec
\item[\vref{De 20:7}] celui qui a f. une fem. et
\item[\vref{Lu 2:5}] avec Marie, sa f., qui était enceinte.
\item[\vref{2 Co 11:2}] je vs. ai f. à un seul
\end{listverse}

\ConcordanceEntry{Fidèle}
\vspace{-2mm}
\begin{listverse}
\item[\vref{No 12:7}] Moïse, qui est f. ds tte ma
\item[\vref{De 7:9}] Dieu. Ce Dieu f. garde son alliance
\item[\vref{1 S 22:14}] un com. David, f., et gendre du
\item[\vref{Né 13:13}] considérés com. très f.. Ils furent chargés
\item[\vref{Ps 12:2}] n'existent plus, les f. disparaissent parmi les
\item[\vref{Ps 19:8}] de Yahweh est f., il donne la
\item[\vref{Ps 32:6}] C'est pourquoi tt f. te prie au
\item[\vref{Ps 37:28}] n'abandonne point ses f. ; c'est pourquoi ils
\item[\vref{Pr 14:25}] Le témoin f. délivre les âmes,
\item[\vref{Pr 20:6}] bonté ; mais qui trouvera un hom. f. ?
\item[\vref{Pr 28:20}] L'hom. f. abondera en bénédictions, mais celui qui
\item[\vref{Da 6:4}] parce qu'il était f., et il ne
\item[\vref{Os 3:3}] je serai aussi f. envers toi.
\item[\vref{Os 12:1}] Dieu, et est f. avec les saints.
\item[\vref{Mt 24:45}] dc le serviteur f. et prudent, que
\item[\vref{Lu 16:10}] Celui qui est f. en très peu
\item[\vref{Lu 19:17}] tu as été f. en peu de
\item[\vref{Ac 16:15}] vs. me jugez f. au Seign., entrez
\item[\vref{1 Co 1:9}] son Fils Jésus-Christ, notre Seign., est f.
\item[\vref{1 Co 4:2}] des gestionnaires que chacun soit trouvé f.
\item[\vref{1 Co 10:13}] Dieu qui est f. ne permettra pas
\item[\vref{2 Co 6:15}] part a le f. avec l'infidèle ?
\item[\vref{2 Th 3:3}] Le Seign. est f., il vs. affermira
\item[\vref{1 Ti 1:12}] qu'il m'a estimé f. en m'établissant ds
\item[\vref{1 Ti 4:3}] afin que les f., et ceux qui
\item[\vref{1 Ti 6:2}] qui ont des f. pour maîtres ne
\item[\vref{2 Ti 2:2}] à des personnes f., qui soient capables
\item[\vref{2 Ti 2:13}] infidèles, il demeure f., car il ne
\item[\vref{Hé 3:2}] qui est f. à celui qui l'a établi, com.
\item[\vref{1 Pi 4:19}] âmes, com. au f. Créateur.
\item[\vref{Ap 1:5}] est le témoin f., le premier-né d'entre
\item[\vref{Ap 2:10}] dix jours. Sois f. jusqu'à la mort,
\item[\vref{Ap 2:13}] jours d'Antipas, mon f. martyr, qui a
\item[\vref{Ap 3:14}] l'Amen, le témoin f. et véritable, le
\end{listverse}

\ConcordanceEntry{Fidèlement}
\vspace{-2mm}
\begin{listverse}
\item[\vref{Ge 24:48}] qui m'a conduit f., afin que je
\item[\vref{Jos 1:8}] nuit, pour agir f. selon tt ce
\item[\vref{2 Ch 34:12}] Ces hommes s'employaient f. à cet ouvrage.
\item[\vref{Pr 12:22}] ceux qui agissent f. lui sont agréables.
\item[\vref{Es 25:1}] conçus d'avance sont f. accomplis.
\item[\vref{3 Jn 1:5}] Bien-aimé, tu agis f. ds tt ce
\end{listverse}

\ConcordanceEntry{Fidélité}
\vspace{-2mm}
\begin{listverse}
\item[\vref{Jos 24:14}] intégrité et avec f.. Ôtez les dieux
\item[\vref{1 S 26:23}] et selon sa f. ; car il t'avait
\item[\vref{Né 9:33}] as agi avec f., mais ns., ns.
\item[\vref{Ps 36:6}] jusqu'aux cieux, ta f. jusqu'aux nues.
\item[\vref{Ps 37:3}] demeure et la f. pour pâture.
\item[\vref{Ps 89:3}] tu établis ta f. ds les cieux
\item[\vref{Ps 91:4}] ses ailes ; sa f. est un bouclier
\item[\vref{Ps 96:13}] monde, et les peuples selon sa f.
\item[\vref{Ps 98:3}] et de sa f. envers la maison
\item[\vref{Ps 100:5}] toujours, et sa f. de génération en
\item[\vref{Ps 119:75}] que tu m'as humilié par ta f.
\item[\vref{Ps 119:86}] ne sont que f. ; on me persécute
\item[\vref{Ro 3:3}] incrédulité anéantira-t-elle la f. de Dieu ?
\item[\vref{2 Ti 3:3}] affection naturelle, sans f., calomniateurs, intempérants, cruels,
\item[\vref{Tit 2:10}] paraître une grande f., afin de rendre
\end{listverse}

\ConcordanceEntry{Fiel}
\vspace{-2mm}
\begin{listverse}
\item[\vref{Job 16:13}] il répand mon f. par terre.
\item[\vref{Job 20:25}] sortira de son f. ; ttes sortes de
\item[\vref{Ps 69:22}] contraire donné du f. pour mon repas ;
\item[\vref{La 3:19}] état qui n'est qu'absinthe et que f. ;
\item[\vref{Mt 27:34}] mêlé avec du f. ; mais qnd il
\item[\vref{Ac 8:23}] es ds un f. très amer et
\end{listverse}

\ConcordanceEntry{Fièvre}
\vspace{-2mm}
\begin{listverse}
\item[\vref{De 28:22}] de tuberculose, de f., d'inflammation, de chaleur
\item[\vref{Mt 8:14}] la belle-mère couchée et ayant la f.
\item[\vref{Lu 4:39}] il menaça la f., et la fièvre
\item[\vref{Jn 4:52}] septième heure, la f. l'a quitté.
\item[\vref{Ac 28:8}] malade de la f. et de la
\end{listverse}

\ConcordanceEntry{Figue}
\vspace{-2mm}
\begin{listverse}
\item[\vref{2 R 20:7}] une masse de f. sèches. Et ils
\item[\vref{Es 38:21}] une masse de f. sèches et qu'on
\item[\vref{Jé 8:13}] aura plus de f. au figuier, les
\item[\vref{Jé 24:1}] deux paniers de f. étaient posés dvt
\item[\vref{Jé 29:17}] devenir com. des f. affreuses qui ne
\item[\vref{Am 7:14}] je cueillais des f. sauvages.
\item[\vref{Na 3:12}] seront com. des f., et com. des
\item[\vref{Mc 11:13}] ce n'était pas la saison des f.
\item[\vref{Ap 6:13}] laisse tomber ses f. encore vertes.
\end{listverse}

\ConcordanceEntry{Figuier}
\vspace{-2mm}
\begin{listverse}
\item[\vref{Jg 9:10}] arbres dirent au f. : Viens, toi, règne
\item[\vref{Ca 2:13}] Le f. produit ses premiers fruits, et les
\item[\vref{Joë 2:22}] lr. fruit ; le f. et la vigne
\item[\vref{Mi 4:4}] et sous son f., et il n'y
\item[\vref{Ha 3:17}] Car le f. ne fleurira pas, et il n'y
\item[\vref{Ag 2:19}] la vigne, au f., au grenadier, et
\item[\vref{Za 3:10}] sous la vigne et sous le f.
\item[\vref{Mt 21:19}] Et voyant un f. qui était sur
\item[\vref{Mt 24:32}] la parabole du f.. Dès que ses
\item[\vref{Lu 13:7}] fruit à ce f., et je n'en
\item[\vref{Lu 21:29}] comparaison : Voyez le f. et ts les
\item[\vref{Jn 1:48}] étais sous le f., je t'ai vu.
\item[\vref{Ja 3:12}] Mes frères, un f. peut-il produire des
\item[\vref{Ap 6:13}] com. lorsque le f. est agité par
\end{listverse}

\ConcordanceEntry{Figure}
\vspace{-2mm}
\begin{listverse}
\item[\vref{1 S 6:5}] ferez dc des f. de vos hémorroïdes,
\item[\vref{1 S 17:42}] jeune garçon, roux et beau de f.
\item[\vref{Est 2:7}] très belle de f.. Après la mort
\item[\vref{Ps 106:20}] gloire contre la f. d'un bœuf qui
\item[\vref{Ca 2:14}] fais-moi voir ta f., fais-moi entendre ta
\item[\vref{Ez 1:26}] apparaissait com. une f. d'hom. placée dessus
\item[\vref{Ez 8:12}] chambre pleine de f. ? Car ils disent :
\item[\vref{Da 1:4}] corporel, beaux de f., instruits en tte
\item[\vref{Ac 7:43}] qui sont des f. que vs. avez
\item[\vref{Ro 5:14}] lequel est la f. de celui qui
\item[\vref{1 Co 7:31}] pas, car la f. de ce monde
\item[\vref{Hé 9:24}] n'était que la f. du véritable, mais
\end{listverse}

\ConcordanceEntry{Filet}
\vspace{-2mm}
\begin{listverse}
\item[\vref{2 S 22:6}] m'avaient entouré, les f. de la mort
\item[\vref{Job 19:6}] a tendu son f. autour de moi.
\item[\vref{Ps 9:16}] se prend au f. qu'elles ont caché.
\item[\vref{Ps 31:5}] Tire-moi hors du f. qu'ils m'ont tendu
\item[\vref{Ps 91:3}] te délivre du f. de l'oiseleur, de
\item[\vref{Ps 124:7}] com. l'oiseau du f. des oiseleurs ; le
\item[\vref{Ps 141:10}] chacun ds son f., jusqu'à ce que
\item[\vref{Pr 1:17}] qu'on jette le f. dvt les yeux
\item[\vref{Ec 9:12}] sont pris au f. de malheur et
\item[\vref{Os 9:8}] prophète est un f. d'oiseleur sur ttes
\item[\vref{Am 3:5}] tombe-t-il ds le f. qui est à
\item[\vref{Ha 1:16}] sacrifie à son f., et il offre
\item[\vref{Mt 4:20}] aussitôt quitté leurs f., ils le suivirent.
\item[\vref{Mt 13:47}] semblable à un f. jeté ds la
\item[\vref{Lu 5:4}] et jetez vos f. pour pêcher.
\item[\vref{Lu 21:35}] surprendra com. un f. ts ceux qui
\item[\vref{Jn 21:6}] dit : Jetez le f. du côté droit
\item[\vref{Jn 21:11}] et tira le f. à terre, plein
\item[\vref{Ro 11:9}] pour eux un f., un piège, une
\end{listverse}

\ConcordanceEntry{Fille}
\vspace{-2mm}
\begin{listverse}
\item[\vref{Ge 20:12}] est ma sœur, f. de mon père ;
\item[\vref{Ge 24:3}] fils parmi les f. des Cananéens, au
\item[\vref{Ge 24:23}] De qui es-tu f. ? Je te prie,
\item[\vref{Ge 29:10}] Jacob vit Rachel, f. de Laban, frère
\item[\vref{Ge 34:1}] Or Dina, la f. que Léa avait
\item[\vref{Ge 38:11}] à Tamar, sa belle-f. : Demeure veuve ds
\item[\vref{De 22:23}] Si une jeune f. vierge est fiancée
\item[\vref{Ru 1:22}] elle, Ruth, sa belle-f., la Moabite, qui
\item[\vref{Ru 4:15}] vieillesse ; car ta belle-f., qui t'aime, a
\item[\vref{Est 2:4}] et la jeune f. qui plaira au
\item[\vref{Ps 45:11}] Ecoute, jeune f., vois et prête
\item[\vref{Es 22:4}] désastre de la f. de mon peuple.
\item[\vref{Es 62:11}] Dites à la f. de Sion : Voici,
\item[\vref{Jé 6:2}] la délicate, la f. de Sion, je
\item[\vref{Ez 16:44}] toi, en disant : Telle mère, telle f. !
\item[\vref{Mi 7:6}] le père, la f. s'élève contre sa
\item[\vref{Za 2:10}] joie et réjouis-toi, f. de Sion ! Car
\item[\vref{Mt 10:37}] fils ou sa f. plus que moi
\item[\vref{Mt 15:28}] l'heure mm, sa f. fut guérie.
\item[\vref{Lu 8:48}] lui dit : Ma f., rassure-toi. Ta foi
\item[\vref{Lu 13:16}] fem. qui est f. d'Abraham, et que
\item[\vref{Ac 7:21}] à l'abandon, la f. de Pharaon le
\item[\vref{Ac 16:18}] sortir de cette f.. Et il sortit
\item[\vref{1 Co 7:36}] déshonneur à sa f. de dépasser la
\end{listverse}

\ConcordanceEntry{Fils}
\vspace{-2mm}
\begin{listverse}
\item[\vref{Ge 6:2}] les f. de Dieu voyant que les filles
\item[\vref{Ge 6:10}] Noé engendra trois f. : Sem, Cham, et
\item[\vref{Ge 11:31}] Térach prit son f. Abram, et Lot
\item[\vref{Ge 17:17}] cœur : Naîtrait-il un f. à un hom.
\item[\vref{Ge 18:10}] fem., aura un f.. Et Sara écoutait
\item[\vref{Ge 24:6}] dit : Garde-toi bien d'y ramener mon f. !
\item[\vref{Ge 35:22}] Israël l'apprit. Or Jacob avait douze f.
\item[\vref{Ge 48:8}] Israël vit les f. de Joseph, et
\item[\vref{Ge 49:1}] Jacob appela ses f., et lr. dit :
\item[\vref{Ex 4:22}] Israël est mon f., mon premier-né.
\item[\vref{Ex 10:9}] vieillards, avec nos f. et nos filles ;
\item[\vref{Ex 13:14}] Et qnd ton f. t'interrogera à l'avenir,
\item[\vref{Ex 28:43}] Aaron et ses f. seront ainsi habillés
\item[\vref{De 21:15}] hait enfantent des f., et que le
\item[\vref{De 21:18}] hom. a un f. indocile et rebelle,
\item[\vref{Ru 4:13}] de concevoir, et elle enfanta un f.
\item[\vref{1 S 16:5}] Isaï et ses f., et les invita
\item[\vref{2 S 7:14}] pour moi un f.. S'il fait le
\item[\vref{1 R 5:7}] à David, un f., sage pour chef
\item[\vref{1 R 13:2}] Yahweh : Voici, un f. naîtra à la
\item[\vref{1 R 17:23}] disant : Regarde, ton f. est vivant.
\item[\vref{2 R 4:16}] tu embrasseras un f.. Elle répondit : Mon
\item[\vref{2 R 4:38}] pays, et les f. des prophètes étaient
\item[\vref{1 Ch 1:32}] Quant aux f. de Ketura, concubine
\item[\vref{Job 1:6}] jour que les f. de Dieu vinrent
\item[\vref{Job 38:7}] que ts les f. de Dieu poussaient
\item[\vref{Ps 2:7}] Tu es mon F. ! Je t'ai engendré
\item[\vref{Ps 82:6}] vs. êtes ts f. du Très-Haut.
\item[\vref{Ps 89:7}] Yahweh parmi les f. de Dieu ?
\item[\vref{Ps 89:31}] Mais si ses f. abandonnent ma loi,
\item[\vref{Ps 103:17}] en faveur des f. de leurs fils ;
\item[\vref{Ps 106:37}] ils sacrifièrent leurs f. et leurs filles
\item[\vref{Ps 127:4}] tels sont les f. de la jeunesse.
\item[\vref{Pr 1:8}] Mon f., écoute l'instruction de ton père, et
\item[\vref{Pr 5:1}] Mon f., sois attentif à ma sagesse, incline
\item[\vref{Pr 8:31}] délices avec les f. de l'hom.
\item[\vref{Pr 10:1}] de Salomon. Le f. sage réjouit son
\item[\vref{Pr 13:1}] Un f. sage écoute l'instruction de son père,
\item[\vref{Es 7:6}] pour roi le f. de Tabeel au
\item[\vref{Es 9:5}] est né, un F. ns. a été
\item[\vref{Es 14:12}] ciel, astre brillant, f. de l'aurore ? Toi
\item[\vref{Es 39:7}] prendra de tes f. qui sortiront de
\item[\vref{Es 60:4}] vers toi ; tes f. viennent de loin,
\item[\vref{Jé 7:31}] au feu leurs f. et leurs filles,
\item[\vref{Jé 16:2}] n'auras pas de f. ni de filles
\item[\vref{Jé 31:15}] Rachel pleure ses f. ; elle refuse d'être
\item[\vref{Jé 31:20}] moi un cher f., un fils qui
\item[\vref{La 3:33}] et d'humilier les f. des hommes.
\item[\vref{Ez 2:1}] Il me dit : F. de l'hom., tiens-toi
\item[\vref{Da 3:25}] quatrième est semblable à celle d'un f. de Dieu.
\item[\vref{Os 2:1}] Vous êtes les f. du Dieu vivant !
\item[\vref{Joë 2:28}] chair ; et vos f. et vos filles
\item[\vref{Mi 7:6}] Car le f. déshonore le père, la fille s'élève
\item[\vref{Za 4:14}] sont les deux f. oints, qui se
\item[\vref{Mal 3:17}] pardonne à son f. qui le sert.
\item[\vref{Mt 1:1}] généalogie de Jésus-Christ, f. de David, fils
\item[\vref{Mt 1:25}] ait enfanté son f. premier-né, auquel il
\item[\vref{Mt 2:15}] J'ai appelé mon F. hors d'Egypte.
\item[\vref{Mt 3:17}] Celui-ci est mon F. bien-aimé, en qui
\item[\vref{Mt 9:6}] sachiez que le F. de l'hom. a
\item[\vref{Mt 10:37}] qui aime son f. ou sa fille
\item[\vref{Mt 16:16}] le Christ, le F. du Dieu vivant.
\item[\vref{Mt 22:42}] De qui est-il F. ? Ils lui répondirent :
\item[\vref{Mt 22:45}] l'appelle son Seign., comment est-il son F. ?
\item[\vref{Mt 23:15}] vs. le rendez f. de la géhenne,
\item[\vref{Mt 28:19}] du Père, du F. et du Saint-Esprit.
\item[\vref{Mc 13:29}] sachez que le F. de l'hom. est
\item[\vref{Lu 1:32}] sera appelé le F. du Très-Haut, et
\item[\vref{Lu 7:12}] terre un mort, f. unique de sa
\item[\vref{Lu 18:8}] Mais qnd le F. de l'hom. viendra,
\item[\vref{Lu 19:9}] celui-ci aussi est f. d'Abraham.
\item[\vref{Jn 3:16}] a donné son F. unique, afin que
\item[\vref{Jn 4:50}] dit : Va, ton f. vit. Cet hom.
\item[\vref{Jn 8:35}] la maison ; le f. y demeure toujours.
\item[\vref{Jn 8:36}] Si dc le F. vs. affranchit, vs.
\item[\vref{Ac 13:10}] de tte ruse, f. du diable, ennemi
\item[\vref{Ac 23:6}] Je suis pharisien, f. de pharisien, c'est
\item[\vref{Ro 1:3}] concernant son F., qui est né
\item[\vref{Ro 8:3}] envoyant son propre F. ds une chair
\item[\vref{Ro 8:14}] conduits par l'Esprit de Dieu sont f. de Dieu.
\item[\vref{Ro 9:27}] le nombre des f. d'Israël serait com.
\item[\vref{1 Co 15:28}] assujetties, alors le F. lui-mm sera assujetti
\item[\vref{Ga 3:26}] vs. êtes ts f. de Dieu par
\item[\vref{Ga 4:4}] a envoyé son F., né d'une fem.,
\item[\vref{Ep 2:2}] efficacité ds les f. rebelles à Dieu,
\item[\vref{Ep 4:13}] la connaissance du F. de Dieu, à
\item[\vref{Col 1:13}] le Royaume du F. de son amour,
\item[\vref{Col 3:6}] vient sur les f. de la rébellion,
\item[\vref{2 Th 2:3}] de péché, le f. de la perdition,
\item[\vref{Hé 1:2}] jours par son F., qu'il a établi
\item[\vref{Hé 12:6}] ts ceux qu'il reconnaît pour ses f.
\item[\vref{1 Jn 2:22}] qui nie le Père et le F.
\item[\vref{1 Jn 5:9}] qu'il a rendu témoignage à son F.
\item[\vref{1 Jn 5:12}] qui a le F. a la vie ;
\item[\vref{Ap 1:13}] ressemblait à un f. d'hom., vêtu d'une
\item[\vref{Ap 2:18}] que dit le F. de Dieu, qui
\item[\vref{Ap 12:5}] elle accoucha d'un f., qui doit gouverner
\item[\vref{Ap 21:7}] son Dieu, et il sera mon f.
\end{listverse}

\ConcordanceEntry{Fin}
\vspace{-2mm}
\begin{listverse}
\item[\vref{Ge 6:13}] à Noé : La f. de tte chair
\item[\vref{Job 6:11}] en est la f. pour que je
\item[\vref{Ps 39:5}] à connaître ma f. et quelle est
\item[\vref{Pr 20:21}] ne sera pas béni à la f.
\item[\vref{Ec 4:8}] ne met nulle f. à son travail ;
\item[\vref{Ec 7:2}] c'est là la f. de tt hom.,
\item[\vref{Ec 7:8}] Mieux vaut la f. d'une chose que
\item[\vref{Es 9:6}] une paix sans f. au trône de
\item[\vref{Es 29:20}] Car l'oppresseur prendra f., le moqueur sera
\item[\vref{La 1:9}] de sa dernière f. ; elle a été
\item[\vref{Da 8:17}] est pour le temps de la f.
\item[\vref{Da 9:24}] transgression et mettre f. aux péchés, faire
\item[\vref{Da 11:45}] arrivera à la f., et personne ne
\item[\vref{Da 12:4}] temps de la f.. Beaucoup courront ça
\item[\vref{Mt 10:22}] persévérera jusqu'à la f. sera sauvé.
\item[\vref{Mt 13:39}] moisson, c'est la f. du monde, et
\item[\vref{Mt 24:3}] et de la f. du monde ?
\item[\vref{Mt 24:14}] les nations, et alors viendra la f.
\item[\vref{Mt 26:60}] Mais à la f., deux faux témoins
\item[\vref{Mt 28:20}] jours jusqu'à la f. du monde. Amen !
\item[\vref{Ro 10:4}] Christ est la f. de la loi
\item[\vref{1 Co 1:8}] aussi jusqu'à la f. pour que vs.
\item[\vref{1 Co 15:24}] Ensuite viendra la f., qnd il aura
\item[\vref{Ph 3:19}] Eux dont la f. est la perdition,
\item[\vref{Hé 7:3}] de jours ni f. de vie, mais
\item[\vref{Ja 5:11}] avez vu la f. du Seign., car
\item[\vref{1 Pi 4:7}] Or la f. de ttes choses est proche : Soyez
\item[\vref{1 Pi 4:17}] quelle sera la f. de ceux qui
\item[\vref{Ap 22:13}] le dernier, le commencement et la f.
\end{listverse}

\ConcordanceEntry{Finir}
\vspace{-2mm}
\begin{listverse}
\item[\vref{Ge 27:30}] Isaac avait f. de bénir Jacob,
\item[\vref{1 R 17:14}] le pot ne f. point et l'huile
\item[\vref{Job 16:22}] mon compte vont f., et j'entre ds
\item[\vref{Es 16:4}] cessera, la dévastation f., celui qui foule
\item[\vref{Da 9:2}] prophète Jérémie pour f. les désolations de
\item[\vref{Mt 26:58}] les officiers pour voir comment cela f.
\item[\vref{Ga 3:3}] l'Esprit, voulez-vs. mntnt f. par la chair ?
\item[\vref{Hé 1:12}] le mm, et tes années ne f. pas.
\end{listverse}

\ConcordanceEntry{Fixer}
\vspace{-2mm}
\begin{listverse}
\item[\vref{Ps 75:3}] temps que j'aurai f., je jugerai avec
\item[\vref{Da 9:24}] a soixante-dix semaines f. sur ton peuple
\item[\vref{Ac 1:7}] le Père a f. de sa propre
\item[\vref{Ac 11:6}] Les regards f. sur cette nappe,
\item[\vref{2 Co 3:7}] d'Israël ne pouvaient f. les yeux sur
\end{listverse}

\ConcordanceEntry{Flambeau}
\vspace{-2mm}
\begin{listverse}
\item[\vref{Jg 7:16}] vides, avec des f. ds les cruches.
\item[\vref{Es 5:24}] pourquoi, com. le f. de feu consume
\item[\vref{Na 2:5}] sont com. des f., et courent com.
\item[\vref{Jn 18:3}] des lanternes, des f. et des armes.
\item[\vref{Ph 2:15}] brillez com. des f. ds le monde,
\item[\vref{Ap 8:10}] ardente com. un f., et elle tomba
\item[\vref{Ap 21:23}] Dieu l'éclaire, et l'Agneau est son f.
\end{listverse}

\ConcordanceEntry{Flamme}
\vspace{-2mm}
\begin{listverse}
\item[\vref{Ge 15:17}] fumante, et des f. passèrent entre les
\item[\vref{Ex 3:2}] apparut ds une f. de feu, du
\item[\vref{Job 41:12}] charbons, et une f. sort de sa
\item[\vref{Ps 104:4}] messagers, et des f. de feu ses
\item[\vref{Ca 8:6}] des embrasements de feu et une f. de Yah.
\item[\vref{Es 30:30}] colère, avec une f. de feu dévorant,
\item[\vref{Es 33:14}] séjourner avec les f. éternelles ?
\item[\vref{Da 3:22}] extraordinairement chauffée, la f. tua les hommes
\item[\vref{Da 11:33}] et à la f., à la captivité
\item[\vref{Joë 2:3}] derrière lui une f. brûle ; dvt lui
\item[\vref{Ha 3:5}] lui, et une f. ardente sort sous
\item[\vref{Lu 16:24}] je suis grièvement tourmenté ds cette f.
\item[\vref{Ac 7:30}] Sinaï, ds la f. d'un buisson en
\item[\vref{2 Th 1:8}] avec des f. de feu, pour exercer la vengeance
\item[\vref{Ap 2:18}] yeux com. une f. de feu, et
\item[\vref{Ap 19:12}] étaient com. une f. de feu ; il
\end{listverse}

\ConcordanceEntry{Fléau}
\vspace{-2mm}
\begin{listverse}
\item[\vref{1 R 8:37}] y aura un f. ou une maladie
\item[\vref{2 Ch 6:28}] y aura un f., une maladie quelconque ;
\item[\vref{Ps 91:10}] de toi, aucun f. n'approchera de ta
\item[\vref{Es 28:15}] scheol ; qnd le f. débordé passera, il
\item[\vref{Es 28:18}] pas ; qnd le f. débordé passera, vs.
\item[\vref{Mc 5:29}] corps qu'elle était guérie de son f.
\item[\vref{Ap 15:1}] les sept derniers f., car c'est par
\item[\vref{Ap 16:21}] à cause du f. de la grêle,
\item[\vref{Ap 22:18}] le frappera des f. décrits ds ce
\end{listverse}

\ConcordanceEntry{Flèche}
\vspace{-2mm}
\begin{listverse}
\item[\vref{1 S 20:36}] trouve mntnt les f. que je m'en
\item[\vref{2 S 22:15}] il lança des f. et dispersa mes
\item[\vref{2 R 13:17}] dit : C'est la f. de la délivrance
\item[\vref{Job 6:4}] Parce que les f. du Tout-Puissant sont
\item[\vref{Ps 11:2}] ils ajustent lr. f. sur la corde,
\item[\vref{Ps 38:3}] Car tes f. m'ont atteint, et ta main s'est
\item[\vref{Ps 91:5}] nuit, ni la f. qui vole le
\item[\vref{Ps 127:4}] Telles sont les f. ds la main
\item[\vref{Pr 7:23}] ce que la f. lui ait transpercé
\item[\vref{Pr 25:18}] un marteau, une épée et une f. aiguë.
\item[\vref{Es 37:33}] n'y jettera aucune f., il ne se
\item[\vref{Es 49:2}] semblable à une f. bien polie, il
\item[\vref{Jé 9:8}] langue est une f. meurtrière, elle profère
\item[\vref{La 3:12}] placé com. une cible pour sa f.
\end{listverse}

\ConcordanceEntry{Fleurir}
\vspace{-2mm}
\begin{listverse}
\item[\vref{No 17:8}] verge d'Aaron, avait f., pour la maison
\item[\vref{Ps 72:7}] temps, le juste f., et il y
\item[\vref{Ps 90:6}] Elle f. au matin, et reverdit ; le soir
\item[\vref{Ps 103:15}] com. l'herbe, il f. com. la fleur
\item[\vref{Pr 14:11}] mais la tente des hommes droits f.
\item[\vref{Ec 12:7}] chemin, qnd l'amandier f., et qnd les
\item[\vref{Es 35:1}] se réjouira et f. com. une rose.
\end{listverse}

\ConcordanceEntry{Fleuve}
\vspace{-2mm}
\begin{listverse}
\item[\vref{Ge 2:10}] Et un f. sortait d'Eden pour arroser le jardin ;
\item[\vref{Ex 7:17}] les eaux du f., et elles seront
\item[\vref{Jos 24:15}] pères au-delà du f., ou les dieux
\item[\vref{Ps 46:5}] Le f. et ses ruisseaux réjouissent la cité
\item[\vref{Ps 66:6}] a passé le f. à pied sec ;
\item[\vref{Ps 98:8}] Que les f. frappent des mains, et que les
\item[\vref{Ps 107:33}] Il réduit les f. en désert, et
\item[\vref{Es 43:20}] désert, et des f. ds la solitude,
\item[\vref{Es 59:19}] viendra com. un f., mais l'Esprit de
\item[\vref{Es 66:12}] paix com. un f., et la gloire
\item[\vref{Ez 29:9}] a dit : Les f. sont à moi,
\item[\vref{Da 7:10}] Un f. de feu sortait et se répandait
\item[\vref{Da 10:4}] bord du grand f. qui est Hiddékel.
\item[\vref{Za 10:11}] les profondeurs du f. seront desséchées ; l'orgueil
\item[\vref{Mc 1:5}] lui ds le f. du Jourdain.
\item[\vref{Ap 9:14}] sur le grand f., l'Euphrate.
\item[\vref{Ap 22:1}] me montra un f. d'eau de la
\item[\vref{Ap 22:2}] deux côtés du f., était l'arbre de
\end{listverse}

\ConcordanceEntry{Flot}
\vspace{-2mm}
\begin{listverse}
\item[\vref{2 S 22:5}] Car les f. de la mort m'avaient environné, les
\item[\vref{Job 38:11}] loin ; ici s'arrêtera l'orgueil de tes f. ?
\item[\vref{Ps 42:8}] vagues et tes f. passent sur moi.
\item[\vref{Ps 65:8}] mugissement de leurs f., et le tumulte
\item[\vref{Ps 93:3}] lr. bruit, les fleuves élèvent leurs f.
\item[\vref{Es 48:18}] justice com. les f. de la mer,
\item[\vref{Es 63:13}] à travers les f., com. un cheval
\item[\vref{Jé 51:55}] magnifique ; et leurs f. mugiront com. de
\item[\vref{Ez 26:3}] com. la mer fait monter ses f.
\item[\vref{Jon 2:4}] environné ; ts tes f. et ttes tes
\item[\vref{Za 10:11}] il frappera les f. de la mer ;
\item[\vref{Mt 8:24}] était couverte de f.. Et Jésus dormait.
\item[\vref{Mt 14:24}] battue par les f. ; car le vent
\item[\vref{Mc 4:37}] vent, et les f. se jetaient ds
\item[\vref{Lu 8:24}] vent et les f. qui s'apaisèrent, et
\item[\vref{Lu 21:25}] bruit de la mer et des f.,
\item[\vref{Ja 1:6}] est semblable au f. de la mer,
\end{listverse}

\ConcordanceEntry{Foi}
\vspace{-2mm}
\begin{listverse}
\item[\vref{Ha 2:4}] mais le juste vivra de sa f.
\item[\vref{Mt 6:30}] forte raison, ô gens de petite f. ?
\item[\vref{Mt 8:10}] n'ai pas trouvé une aussi grande f.
\item[\vref{Mt 8:13}] fait selon ta f.. Et à l'heure
\item[\vref{Mt 9:2}] Jésus, voyant lr. f., dit au paralytique :
\item[\vref{Mt 9:22}] ma fille, ta f. t'a sauvée. Et
\item[\vref{Mt 14:31}] de peu de f., pourquoi as-tu douté ?
\item[\vref{Mt 15:28}] Ô fem. ! Ta f. est grande. Qu'il
\item[\vref{Mt 17:20}] aviez de la f. com. un grain
\item[\vref{Mt 21:21}] vs. aviez la f., et que vs.
\item[\vref{Mc 5:34}] Ma fille, ta f. t'a sauvée. Va
\item[\vref{Mc 11:22}] Jésus répondant, lr. dit : Ayez f. en Dieu.
\item[\vref{Lu 7:9}] mm en Israël, une si grande f.
\item[\vref{Lu 8:25}] Où est votre f. ? Saisis de frayeur
\item[\vref{Lu 8:48}] fille, rassure-toi. Ta f. t'a guérie. Va
\item[\vref{Lu 17:5}] apôtres dirent au Seign. : Augmente-ns. la f.
\item[\vref{Lu 18:8}] qu'il trouvera la f. sur la terre ?
\item[\vref{Lu 22:32}] afin que ta f. ne défaille pas ;
\item[\vref{Jn 5:31}] mon témoignage n'est pas digne de f.
\item[\vref{Ac 3:16}] C'est par la f. en son Nom,
\item[\vref{Ac 6:5}] hom. plein de f. et du Saint-Esprit,
\item[\vref{Ac 6:7}] aussi de prêtres obéissait à la f.
\item[\vref{Ac 6:8}] Etienne, plein de f. et de puissance,
\item[\vref{Ac 13:8}] détourner de la f. le proconsul.
\item[\vref{Ac 14:9}] qu'il avait la f. pour être guéri,
\item[\vref{Ac 14:22}] persévérer ds la f., disant que c'est
\item[\vref{Ac 14:27}] aux Gentils la porte de la f.
\item[\vref{Ac 16:5}] affermies ds la f., et augmentaient en
\item[\vref{Ro 1:17}] Dieu pleinement de f. en foi, selon
\item[\vref{Ro 3:26}] qui a la f. en Jésus.
\item[\vref{Ro 3:28}] justifié par la f., sans les œuvres
\item[\vref{Ro 3:30}] justifiera par la f. les circoncis, et
\item[\vref{Ro 4:5}] le méchant, sa f. lui est imputée
\item[\vref{Ro 4:16}] dc par la f., afin que ce
\item[\vref{Ro 4:19}] faible ds la f., il n'eut pas
\item[\vref{Ro 5:2}] amenés par la f. à cette grâce,
\item[\vref{Ro 10:8}] la parole de f. que ns. prêchons.
\item[\vref{Ro 10:17}] Ainsi la f. vient de ce qu'on entend, et
\item[\vref{Ro 12:3}] la mesure de f. que Dieu a
\item[\vref{Ro 14:1}] faible en la f., recevez-le, et n'ayez
\item[\vref{Ro 14:23}] n'agit pas avec f.. Or tt ce
\item[\vref{1 Co 2:5}] afin que votre f. ne soit pas
\item[\vref{1 Co 13:2}] mm tte la f. qu'on puisse avoir,
\item[\vref{1 Co 13:13}] choses demeurent : La f., l'espérance et la
\item[\vref{1 Co 15:14}] vaine, et votre f. aussi est vaine.
\item[\vref{1 Co 16:13}] fermes ds la f., agissez courageusement, fortifiez-vs.
\item[\vref{2 Co 5:7}] marchons par la f. et non par
\item[\vref{2 Co 8:7}] ttes choses, en f., en parole, en
\item[\vref{2 Co 13:5}] êtes ds la f. ; éprouvez-vs. vs.-mêmes. Ne
\item[\vref{Ga 1:23}] annonce mntnt la f. qu'il détruisait autrefois.
\item[\vref{Ga 2:20}] vis ds la f. au Fils de
\item[\vref{Ga 3:8}] Gentils par la f., a prêché d'avance
\item[\vref{Ga 3:9}] qui ont la f. sont bénis avec
\item[\vref{Ga 3:11}] dit : Le juste vivra de la f.
\item[\vref{Ga 3:14}] recevions par la f. l'Esprit qui avait
\item[\vref{Ga 3:22}] donné par la f. en Jésus-Christ à
\item[\vref{Ga 3:23}] avant que la f. vienne, ns. étions
\item[\vref{Ga 5:6}] mais seulement la f. qui opère par
\item[\vref{Ep 2:8}] grâce, par la f. ; et cela ne
\item[\vref{Ep 3:12}] confiance, par la f. que ns. avons
\item[\vref{Ep 4:5}] Seign., une seule f., un seul baptême,
\item[\vref{Ep 4:13}] l'unité de la f. et de la
\item[\vref{Ep 6:16}] bouclier de la f., avec lequel vs.
\item[\vref{Ph 1:27}] âme pour la f. de l'Evangile, et
\item[\vref{Col 1:4}] parler de votre f. en Jésus-Christ, et
\item[\vref{Col 1:23}] demeurez ds la f., étant fondés et
\item[\vref{1 Th 5:8}] cuirasse de la f. et de la
\item[\vref{2 Th 1:3}] parce que votre f. augmente beaucoup, et
\item[\vref{2 Th 3:2}] pervers, car ts n'ont pas la f.
\item[\vref{1 Ti 1:5}] pur, d'une bonne conscience, et d'une f. sincère.
\item[\vref{1 Ti 1:19}] en gardant la f. et une bonne
\item[\vref{1 Ti 2:7}] Gentils ds la f. et ds la
\item[\vref{1 Ti 2:15}] persévère ds la f., ds la charité,
\item[\vref{1 Ti 5:8}] a renié la f., et il est
\item[\vref{1 Ti 5:12}] ce qu'elles ont violé lr. première f.
\item[\vref{1 Ti 6:10}] détournés de la f. et se sont
\item[\vref{1 Ti 6:12}] combat de la f., saisis la vie
\item[\vref{2 Ti 1:5}] souvenant de la f. sincère qui est
\item[\vref{2 Ti 2:18}] qui renversent la f. de quelques-uns.
\item[\vref{2 Ti 4:7}] achevé la course, j'ai gardé la f.
\item[\vref{Phm 1:6}] communication de ta f. devienne efficace, en
\item[\vref{Hé 4:2}] mêlée avec la f. ds ceux qui
\item[\vref{Hé 10:22}] sincère, et une f. inébranlable, ayant les
\item[\vref{Hé 10:38}] vivra de la f. ; mais si quelqu'un
\item[\vref{Hé 10:39}] qui ont la f. pour le salut
\item[\vref{Hé 11:1}] Or la f. rend présentes les choses qu'on espère,
\item[\vref{Hé 11:3}] Par la f., ns. comprenons que l'univers a été
\item[\vref{Hé 11:33}] qui par la f. combattirent des royaumes,
\item[\vref{Hé 12:2}] consommateur de la f. qui en échange
\item[\vref{Hé 13:7}] de lr. vie, et imitez lr. f.
\item[\vref{Ja 1:3}] l'épreuve de votre f. produit la patience.
\item[\vref{Ja 1:6}] la demande avec f., ne doutant nullement ;
\item[\vref{Ja 2:26}] de mm la f. sans les œuvres
\item[\vref{Ja 5:15}] prière faite avec f. sauvera le malade,
\item[\vref{1 Pi 1:7}] l'épreuve de votre f., beaucoup plus précieuse
\item[\vref{1 Pi 5:9}] fermes ds la f., sachant que les
\item[\vref{2 Pi 1:1}] en partage une f. du mm prix
\item[\vref{1 Jn 5:4}] victoire sur le monde, c'est notre f.
\item[\vref{Jud 1:3}] combattre pour la f. qui a été
\item[\vref{Jud 1:20}] votre très sainte f., et priez par
\item[\vref{Ap 2:13}] pas renié ma f., mm aux jours
\item[\vref{Ap 13:10}] persévérance et la f. des saints.
\end{listverse}

\ConcordanceEntry{Fois}
\vspace{-2mm}
\begin{listverse}
\item[\vref{Ex 23:17}] Trois f. par année, ts les mâles d'entre
\item[\vref{Job 33:14}] parle une première f. et une seconde
\item[\vref{Job 33:29}] deux et trois f. envers l'hom.,
\item[\vref{Job 39:38}] J'ai parlé une f., mais je ne
\item[\vref{2 Co 11:24}] cinq f. j'ai reçu des Juifs quarante coups
\item[\vref{Hé 9:7}] le second une f. par an, non
\item[\vref{Hé 9:26}] ait souffert plusieurs f. depuis la création
\item[\vref{Hé 9:27}] mourir une seule f., et après cela
\end{listverse}

\ConcordanceEntry{Folie}
\vspace{-2mm}
\begin{listverse}
\item[\vref{De 28:28}] te frappera de f., d'aveuglement, et d'égarement
\item[\vref{Ps 49:14}] lr. chemin, lr. f., et ceux qui
\item[\vref{Ps 69:6}] tu connais ma f., et mes fautes
\item[\vref{Ps 85:9}] jamais ils ne retournent à lr. f.
\item[\vref{Pr 12:23}] le cœur des insensés publie la f.
\item[\vref{Pr 18:13}] un acte de f. et attire la
\item[\vref{Pr 19:3}] La f. de l'hom. renverse son chemin ; et
\item[\vref{Pr 22:15}] La f. est liée au cœur du jeune
\item[\vref{Pr 26:4}] l'insensé selon sa f., de peur que
\item[\vref{Pr 26:5}] l'insensé selon sa f., de peur qu'il
\item[\vref{Pr 27:22}] un pilon, sa f. ne se détournerait
\item[\vref{Ec 2:3}] c'est que la f., jusqu'à ce que
\item[\vref{Ec 10:1}] un peu de f. produit le mm
\item[\vref{Ec 10:6}] C'est que la f. est mise aux
\item[\vref{Es 44:25}] et qui change lr. science en f.
\item[\vref{Jé 23:13}] vu de la f. ds les prophètes
\item[\vref{Za 12:4}] chevaux, et de f. ceux qui les
\item[\vref{Mc 7:22}] envieux, les discours outrageux, l'orgueil, la f.
\item[\vref{1 Co 1:18}] croix est une f. pour ceux qui
\item[\vref{1 Co 1:20}] pas convaincu de f. la sagesse de
\item[\vref{1 Co 1:21}] croyants par la f. de la prédication.
\item[\vref{1 Co 1:23}] Juifs et une f. pour les Grecs.
\item[\vref{1 Co 1:25}] Parce que la f. de Dieu est
\item[\vref{1 Co 2:14}] elles sont une f. pour lui, et
\item[\vref{1 Co 3:19}] monde est une f. dvt Dieu ; car
\item[\vref{2 Co 11:1}] peu ds ma f. ! Je vs. prie,
\item[\vref{2 Co 11:17}] selon le Seign., mais com. par f.
\item[\vref{2 Ti 3:9}] progrès, car lr. f. sera manifestée à
\item[\vref{2 Pi 2:16}] humaine, arrêta la f. du prophète.
\end{listverse}

\ConcordanceEntry{Fonction}
\vspace{-2mm}
\begin{listverse}
\item[\vref{No 8:24}] Lévite entrera en f. ds la tente
\item[\vref{No 18:7}] vs. observerez la f. de votre prêtrise
\item[\vref{1 Ch 9:26}] Car selon cette f., il y avait
\item[\vref{1 Ch 24:3}] fils d'Ithamar, en f. de leurs charges
\item[\vref{2 Ch 8:14}] prêtres selon lr. f., et les Lévites
\item[\vref{2 Ch 13:10}] sont les Lévites qui tiennent cette f.
\item[\vref{Ro 12:4}] les membres n'ont pas la mm f.,
\end{listverse}

\ConcordanceEntry{Fondation}
\vspace{-2mm}
\begin{listverse}
\item[\vref{Mt 13:35}] cachées dès la f. du monde.
\item[\vref{Mt 25:34}] préparé dès la f. du monde.
\item[\vref{Jn 17:24}] aimé avant la f. du monde.
\item[\vref{Ep 1:4}] lui avant la f. du monde, afin
\item[\vref{Hé 4:3}] achevées depuis la f. du monde.
\item[\vref{1 Pi 1:20}] prédestiné avant la f. du monde, et
\item[\vref{Ap 13:8}] immolé dès la f. du monde.
\item[\vref{Ap 17:8}] vie dès la f. du monde, s'étonneront
\end{listverse}

\ConcordanceEntry{Fondement}
\vspace{-2mm}
\begin{listverse}
\item[\vref{Esd 3:10}] bâtissaient posèrent les f. du temple de
\item[\vref{Esd 5:16}] a posé les f. de la maison
\item[\vref{Job 19:28}] persécutons-ns. ? Puisque le f. de mes paroles
\item[\vref{Ps 11:3}] Puisque les f. sont renversés, que
\item[\vref{Ps 82:5}] ténèbres ; ts les f. de la terre
\item[\vref{Ps 119:160}] Le f. de ta parole est la vérité,
\item[\vref{Pr 8:29}] lorsqu'il posa les f. de la terre,
\item[\vref{Pr 10:25}] juste est un f. perpétuel.
\item[\vref{Es 28:16}] je mettrai pour f. en Sion une
\item[\vref{Jé 51:26}] pour servir de f., car tu seras
\item[\vref{Ez 41:8}] partir de lr. f., avaient une canne
\item[\vref{Lu 6:48}] a mis le f. sur le roc.
\item[\vref{Ro 15:20}] bâtir sur le f. qu'un autre a
\item[\vref{1 Co 3:10}] j'ai posé le f. com. un sage
\item[\vref{1 Co 3:11}] poser un autre f. que celui qui
\item[\vref{1 Co 3:12}] édifie sur ce f. avec de l'or,
\item[\vref{Ep 2:20}] édifiés sur le f. des apôtres et
\item[\vref{1 Ti 6:19}] placé sur un f. solide, afin qu'ils
\item[\vref{2 Ti 2:19}] Toutefois, le f. de Dieu demeure
\item[\vref{Hé 6:1}] de nouveau le f. de la repentance
\item[\vref{Ap 21:14}] ville avait douze f., et les noms
\end{listverse}

\ConcordanceEntry{Fonder}
\vspace{-2mm}
\begin{listverse}
\item[\vref{Esd 3:8}] Jérus., commencèrent à f. le temple ; et
\item[\vref{Job 38:4}] étais-tu qnd je f. la terre ? Dis-le,
\item[\vref{Ps 8:3}] tètent, tu as f. ta puissance, à
\item[\vref{Ps 102:26}] Tu as jadis f. la terre, et
\item[\vref{Es 51:16}] cieux, que je f. la terre, et
\item[\vref{Jé 10:12}] puissance, qui a f. le monde habitable
\item[\vref{Col 1:23}] la foi, étant f. et fermes, et
\item[\vref{Hé 1:10}] Seign., tu as f. la terre dès
\end{listverse}

\ConcordanceEntry{Force}
\vspace{-2mm}
\begin{listverse}
\item[\vref{Ex 15:2}] Yahweh est ma f. et ma louange,
\item[\vref{De 8:17}] ton cœur : Ma f. et la puissance
\item[\vref{De 8:18}] donne de la f. pour acquérir ces
\item[\vref{Jg 6:14}] Va avec cette f. que tu as
\item[\vref{1 S 2:10}] il donnera la f. à son Roi,
\item[\vref{2 S 22:3}] bouclier et la f. qui me sauve,
\item[\vref{2 S 22:40}] me ceins de f. pour le combat,
\item[\vref{2 R 18:20}] conseil et la f.. Mais ce ne
\item[\vref{1 Ch 16:11}] Yahweh et sa f., cherchez continuellement sa
\item[\vref{1 Ch 29:12}] ttes choses ; la f. et la puissance
\item[\vref{2 Ch 20:12}] ns. sommes sans f. dvt cette grande
\item[\vref{Né 8:10}] la joie de Yahweh est votre f.
\item[\vref{Job 12:13}] sagesse et la f. ; à lui appartient
\item[\vref{Ps 6:3}] suis sans aucune f.. Guéris-moi, ô Yahweh !
\item[\vref{Ps 18:2}] qui es ma f., je t'aimerai d'une
\item[\vref{Ps 28:8}] Yahweh est la f. de son peuple,
\item[\vref{Ps 31:11}] les soupirs ; ma f. chancelle à cause
\item[\vref{Ps 33:16}] puissant n'échappe point par sa grande f. ;
\item[\vref{Ps 33:17}] point par la grandeur de sa f.
\item[\vref{Ps 81:2}] notre Dieu, notre f. ! Poussez des cris
\item[\vref{Ps 84:8}] Ils marchent avec f. pour se présenter
\item[\vref{Ps 118:14}] Yahweh est ma f. et le sujet
\item[\vref{Pr 20:29}] La f. des jeunes gens est lr. gloire,
\item[\vref{Pr 24:10}] la détresse, ta f. sera diminuée.
\item[\vref{Ec 9:16}] mieux que la f.. Cependant, la sagesse
\item[\vref{Es 12:2}] Yahweh est ma f. et ma louange ;
\item[\vref{Es 14:10}] tu es sans f. com. ns., tu
\item[\vref{Es 30:3}] Car la f. de Pharaon sera pour vs. une
\item[\vref{Es 40:29}] donne de la f. à celui qui
\item[\vref{Es 40:30}] mm les jeunes hommes tombent sans f.
\item[\vref{Es 40:31}] Yahweh renouvellent lr. f.. Ils s'élèvent avec
\item[\vref{Es 51:9}] réveille-toi, revêts-toi de f., bras de Yahweh !
\item[\vref{Da 12:7}] s'accompliront qnd la f. du peuple saint
\item[\vref{Mi 3:8}] suis rempli de f., de justice et
\item[\vref{Mi 5:3}] gouvernera par la f. de Yahweh, avec
\item[\vref{Ha 1:11}] coupable, car sa f. est son dieu.
\item[\vref{Ha 3:19}] Seign., est ma f., et il rend
\item[\vref{Za 4:6}] ni par la f., mais par mon
\item[\vref{Za 12:5}] Jérus. sont notre f., par Yahweh des
\item[\vref{Mt 5:41}] Si quelqu'un te f. à faire un
\item[\vref{Mc 5:30}] en lui-mm qu'une f. était sortie de
\item[\vref{Lu 6:19}] toucher, parce qu'une f. sortait de lui
\item[\vref{Ac 8:3}] et traînant par f. hommes et femmes,
\item[\vref{Ac 10:38}] Saint-Esprit et de f. Jésus de Nazareth,
\item[\vref{Ro 5:6}] étions encore sans f., Christ est mort,
\item[\vref{1 Co 15:43}] en faiblesse, il ressuscite plein de f. ;
\item[\vref{Ep 4:16}] accroissement selon la f. qu'il distribue à
\item[\vref{2 Ti 1:7}] timidité, mais de f., de charité et
\item[\vref{2 Ti 3:5}] ayant renié la f.. Eloigne-toi dc de
\item[\vref{Hé 9:17}] puisqu'il n'a aucune f. tant que le
\item[\vref{Hé 11:11}] Sara reçut la f. de concevoir un
\item[\vref{Hé 11:34}] éteignirent la f. du feu, échappèrent
\end{listverse}

\ConcordanceEntry{Forgeron}
\vspace{-2mm}
\begin{listverse}
\item[\vref{Es 44:12}] Le f. fait une hache, et il travaille
\item[\vref{Es 54:16}] ai créé le f. soufflant le charbon
\item[\vref{Za 1:20}] Puis Yahweh me montra quatre f.
\item[\vref{2 Ti 4:14}] Alexandre, le f. m'a fait beaucoup
\end{listverse}

\ConcordanceEntry{Forme}
\vspace{-2mm}
\begin{listverse}
\item[\vref{Job 38:14}] terre prenne une f. com. l'argile qui
\item[\vref{Es 44:13}] il trace sa f. au crayon avec
\item[\vref{Mc 3:23}] lr. dit sous f. de paraboles : Comment
\item[\vref{Mc 16:12}] sous une autre f. à deux d'entre
\item[\vref{Lu 3:22}] lui sous une f. corporelle, com. celle
\item[\vref{Ro 6:17}] cœur à cette f. de doctrine qui
\item[\vref{Ga 3:15}] testament en bonne f., bien que fait
\item[\vref{Ph 2:6}] lequel étant en f. de Dieu, n'a
\item[\vref{Ph 2:7}] ayant pris la f. de serviteur, fait
\end{listverse}

\ConcordanceEntry{Former}
\vspace{-2mm}
\begin{listverse}
\item[\vref{Job 10:8}] Tes mains m'ont f. et elles ont
\item[\vref{Job 33:6}] j'ai aussi été f. de la terre
\item[\vref{Ps 94:9}] Celui qui a f. l'œil, ne verrait-il
\item[\vref{Ec 11:5}] ni comment se f. les os ds
\item[\vref{Es 22:11}] qui les a f. il y a
\item[\vref{Es 43:10}] n'a pas été f. de Dieu, et
\item[\vref{Es 45:7}] Je f. la lumière, et je crée les
\item[\vref{Es 45:18}] Dieu qui a f. la terre, qui
\item[\vref{2 Co 5:5}] qui ns. a f. à cela mm,
\item[\vref{Ga 4:19}] l'enfantement, jusqu'à ce que Christ soit f. en vs.,
\item[\vref{Ep 2:16}] avec Dieu pour f. un seul corps
\item[\vref{1 Ti 2:13}] Adam a été f. le premier, Eve
\item[\vref{Hé 10:5}] mais tu m'as f. un corps ;
\item[\vref{Ap 17:17}] plaît, et de f. un mm dessein,
\end{listverse}

\ConcordanceEntry{Fornicateur}
\vspace{-2mm}
\begin{listverse}
\item[\vref{1 Co 5:9}] ne pas vs. mêler avec les f.,
\item[\vref{1 Co 5:11}] nomme frère, est f., ou cupide, ou
\item[\vref{1 Co 6:9}] vs.-mêmes : Ni les f., ni les idolâtres,
\item[\vref{Ep 5:5}] sachez-le bien qu'aucun f., ni impur, ni
\item[\vref{1 Ti 1:10}] pour les f., pour les homosexuels, pour les voleurs
\item[\vref{Hé 12:16}] parmi vs. ni f., ni profane com.
\item[\vref{Hé 13:4}] Dieu jugera les f. et les adultères.
\item[\vref{Ap 22:15}] les empoisonneurs, les f., les meurtriers, les
\end{listverse}

\ConcordanceEntry{Fornication}
\vspace{-2mm}
\begin{listverse}
\item[\vref{No 25:1}] à commettre la f. avec les filles
\item[\vref{Jé 3:9}] légèreté de sa f., elle a souillé
\item[\vref{Ez 6:9}] adonné à la f., qui s'est détourné
\item[\vref{Ez 23:8}] pas abandonné ses f. d'Egypte, car ils
\item[\vref{Ez 23:29}] et de tes f., sera découverte.
\item[\vref{Ez 43:7}] rois, par leurs f. ; mais ils souilleront
\item[\vref{Os 4:12}] car l'esprit de f. les a fait
\item[\vref{Os 5:4}] que l'esprit de f. est au milieu
\item[\vref{Mt 15:19}] les adultères, les f., les vols, les
\item[\vref{Jn 8:41}] issus de la f. ; ns. avons un
\item[\vref{Ac 15:20}] et de la f., des animaux étouffés
\item[\vref{Ac 15:29}] et de la f. ; choses contre lesquelles
\item[\vref{1 Co 6:13}] pas pour la f., mais pour le
\item[\vref{1 Co 6:18}] Fuyez la f.. Quelque autre péché qu'un hom. commette,
\item[\vref{1 Co 7:2}] pour éviter la f., que chacun ait
\item[\vref{1 Co 10:8}] pas à la f., com. quelques-uns d'entre
\item[\vref{2 Co 12:21}] l'impureté, de la f. et de l'impudicité
\item[\vref{Ga 5:19}] sont l'adultère, la f., l'impureté, l'impudicité,
\item[\vref{Col 3:5}] la terre : La f., l'impureté, les passions,
\item[\vref{1 Th 4:3}] que vs. vs. absteniez de la f.,
\item[\vref{Ap 2:14}] et qu'ils se livrent à la f.
\item[\vref{Ap 17:2}] ont commis la f., et les habitants
\item[\vref{Ap 18:3}] ont commis la f. avec elle, et
\end{listverse}

\ConcordanceEntry{Fort}
\vspace{-2mm}
\begin{listverse}
\item[\vref{Ge 25:23}] peuples sera plus f. que l'autre, et
\item[\vref{Ge 33:20}] El-Elohé-Israël (le Dieu F., le Dieu d'Israël).
\item[\vref{Ex 17:11}] alors le plus f., mais qnd il
\item[\vref{Jg 6:12}] lui dit : Très f. et vaillant héros,
\item[\vref{Jg 14:14}] mange, et du f. est sorti le
\item[\vref{1 S 17:50}] David fut plus f. que le Philistin ;
\item[\vref{2 S 3:1}] plus en plus f., et la maison
\item[\vref{Ps 24:8}] gloire ? C'est Yahweh f. et puissant, Yahweh
\item[\vref{Ec 6:10}] qui est plus f. que lui ?
\item[\vref{Ec 12:5}] que les hommes f. se courbent, et
\item[\vref{Ca 8:6}] car l'amour est f. com. la mort,
\item[\vref{Jé 9:23}] sagesse, que le f. ne se glorifie
\item[\vref{Jé 31:11}] de la main d'un ennemi plus f. que lui.
\item[\vref{Jé 50:34}] Leur Rédempteur est f., son Nom est
\item[\vref{Joë 3:10}] que le faible dise : Je suis f. !
\item[\vref{Mt 12:29}] maison d'un hom. f. et piller ses
\item[\vref{2 Co 12:10}] faible, c'est alors que je suis f.
\item[\vref{1 Jn 2:14}] que vs. êtes f. et que la
\end{listverse}

\ConcordanceEntry{Forteresse}
\vspace{-2mm}
\begin{listverse}
\item[\vref{2 S 5:7}] s'empara de la f. de Sion : C'est
\item[\vref{2 S 22:2}] mon rocher, ma f., mon libérateur.
\item[\vref{Ps 18:3}] mon Rocher, ma f. et mon libérateur !
\item[\vref{Ps 31:4}] mon Rocher, ma f. ; tu me dirigeras
\item[\vref{Ps 91:2}] retraite et ma f., tu es mon
\item[\vref{Jé 16:19}] ma force, ma f., et mon refuge
\item[\vref{Na 1:7}] il est une f. au jour de
\item[\vref{Ha 2:1}] debout ds la f. et je faisais
\item[\vref{Ac 21:34}] ordonna de mener Paul ds la f.
\item[\vref{2 Co 10:4}] de Dieu, pour la destruction des f. ;
\end{listverse}

\ConcordanceEntry{Fortifier}
\vspace{-2mm}
\begin{listverse}
\item[\vref{Ge 18:5}] de pain pour f. votre cœur. Après
\item[\vref{De 1:38}] sert, y entrera ; f.-le, car c'est
\item[\vref{De 31:7}] de tt Israël : F.-toi et prends
\item[\vref{1 R 2:2}] tte la terre, f.-toi et comporte-toi
\item[\vref{Né 6:9}] fera point. Maintenant dc, ô Dieu, f.-moi !
\item[\vref{Ps 31:25}] fermes et il f. votre esprit !
\item[\vref{Ps 138:3}] rassuré, tu m'as f. d'une nouvelle force
\item[\vref{Es 41:10}] Dieu ; je te f., et je t'aide,
\item[\vref{Ez 34:4}] Vous n'avez point f. les brebis languissantes,
\item[\vref{Da 10:19}] soit avec toi ! F.-toi, fortifie-toi ! Et
\item[\vref{Ag 2:4}] Maintenant dc Zorobabel, f.-toi ! dit Yahweh.
\item[\vref{Za 10:6}] Car je f. la maison de Juda, et je
\item[\vref{Lu 22:43}] lui apparut du ciel, pour le f.
\item[\vref{Ac 14:22}] f. l'esprit des disciples, et les exhortant
\item[\vref{Ac 15:41}] et la Cilicie, f. les églises.
\item[\vref{Ep 3:16}] donne d'être puissamment f. par son Esprit
\item[\vref{Ep 6:10}] reste, mes frères, f.-vs. ds le
\item[\vref{Ph 4:13}] ttes choses en Christ qui me f.
\item[\vref{Col 1:11}] étant f. en tte force, selon la puissance
\item[\vref{1 Ti 1:12}] celui qui m'a f., c'est-à-dire à Jésus-Christ,
\item[\vref{2 Ti 4:17}] m'a assisté et f., afin que ma
\item[\vref{1 Pi 5:10}] vs. affermisse, vs. f. et vs. établisse !
\end{listverse}

\ConcordanceEntry{Fosse}
\vspace{-2mm}
\begin{listverse}
\item[\vref{Ex 21:34}] maître de la f. donnera satisfaction, et
\item[\vref{Job 33:18}] âme de la f., et sa vie
\item[\vref{Job 33:22}] s'approche de la f., et sa vie
\item[\vref{Ps 30:4}] je ne descende point ds la f.
\item[\vref{Ps 30:10}] descends ds la f. ? La poussière te
\item[\vref{Ps 40:3}] retiré de la f. de destruction, du
\item[\vref{Ps 49:10}] n'éviteront pas la vue de la f.
\item[\vref{Ps 88:5}] descendent ds la f. ; je suis devenu
\item[\vref{Ps 103:4}] vie de la f., qui te couronne
\item[\vref{Pr 22:14}] étrangers est une f. profonde, celui contre
\item[\vref{Pr 26:27}] qui creuse la f. y tombe ; et
\item[\vref{Pr 28:10}] tombe ds la f. qu'il a faite,
\item[\vref{Ec 10:8}] qui creuse la f. y tombera, et
\item[\vref{Es 38:17}] pas ds la f. de la pourriture,
\item[\vref{Es 38:18}] descendus ds la f. ne s'attendent plus
\item[\vref{Jé 18:20}] ont creusé une f. pour mon âme.
\item[\vref{Jé 38:10}] hors de la f. Jérémie, le prophète,
\item[\vref{Ez 26:20}] descendent ds la f., vers le peuple
\item[\vref{Ez 28:8}] descendre ds la f., et tu mourras
\item[\vref{Da 6:7}] jeté ds la f. aux lions.
\item[\vref{Jon 2:7}] vivant de la f., Yahweh, mon Dieu !
\item[\vref{Za 9:11}] captifs de la f. où il n'y
\item[\vref{Mt 12:11}] tomber ds une f. le jour du
\item[\vref{Mt 15:14}] ils tomberont ts deux ds la f.
\item[\vref{Mc 12:1}] y creusa une f. pour un pressoir,
\item[\vref{Lu 6:39}] tomberont-ils pas ts deux ds la f. ?
\end{listverse}

\ConcordanceEntry{Fou, Folle}
\vspace{-2mm}
\begin{listverse}
\item[\vref{De 28:34}] Tu deviendras f. à cause de
\item[\vref{Pr 7:22}] boucherie, com. le f. qu'on lie pour
\item[\vref{Pr 9:13}] La fem. f. est bruyante, stupide et elle ne
\item[\vref{Pr 14:1}] maison, mais la f. la ruine de
\item[\vref{Pr 17:28}] Même le f., qnd il se tait, est réputé
\item[\vref{Jé 29:26}] hom. qui est f. et se donne
\item[\vref{Mt 25:2}] en avait cinq sages et cinq f.
\item[\vref{Jn 10:20}] démon, il est f. ! Pourquoi l'écoutez-vs. ?
\item[\vref{Ac 12:15}] dirent : Tu es f.. Mais elle affirma
\item[\vref{Ac 26:24}] voix : Tu es f. Paul ! Ton grand
\item[\vref{Ro 1:22}] disant être sages, ils sont devenus f. ;
\item[\vref{1 Co 3:18}] qu'il se rende f., afin de devenir
\item[\vref{1 Co 4:10}] Nous sommes f. pour l'amour de
\end{listverse}

\ConcordanceEntry{Foule}
\vspace{-2mm}
\begin{listverse}
\item[\vref{Za 14:16}] Jérus., monteront en f. chaque année pour
\item[\vref{Mt 4:25}] Une grande f. le suivit, de
\item[\vref{Mt 14:15}] avancée, renvoie la f., afin qu'elle aille
\item[\vref{Mt 15:36}] disciples, qui les distribuèrent à la f.
\item[\vref{Mt 26:55}] dit à la f. : Vous êtes venus
\item[\vref{Mc 5:21}] barque, une grande f. s'assembla près de
\item[\vref{Mc 8:6}] ordonna à la f. de s'asseoir par
\item[\vref{Mc 9:25}] Jésus vit la f. accourir ensemble, il
\item[\vref{Mc 15:15}] voulant satisfaire la f., lr. relâcha Barabbas ;
\item[\vref{Lu 3:7}] qui venaient en f. pour être baptisés
\item[\vref{Lu 3:10}] Alors la f. l'interrogeait, disant : Que ferons-ns. dc ?
\item[\vref{Lu 5:3}] de la barque il enseignait la f.
\item[\vref{Lu 8:42}] allait, il était pressé par la f.
\item[\vref{Lu 19:3}] cause de la f., car il était
\item[\vref{Jn 7:40}] Plusieurs de la f. ayant entendu ce
\item[\vref{Jn 12:18}] et la f. alla au-dvt de lui, parce qu'elle
\item[\vref{Ac 13:45}] voyant tte cette f., furent remplis de
\item[\vref{Ac 16:22}] La f. se souleva aussi contre eux, et
\item[\vref{Ac 19:29}] se jetèrent en f. ds le théâtre,
\item[\vref{Ap 19:1}] voix forte d'une f. nombreuse, disant : Alléluia !
\end{listverse}

\ConcordanceEntry{Fouler}
\vspace{-2mm}
\begin{listverse}
\item[\vref{De 11:24}] Tout lieu que f. la plante de
\item[\vref{1 Ch 21:23}] ces instruments à f. du blé pour
\item[\vref{Né 13:15}] quelques-uns de Juda f. aux pressoirs le
\item[\vref{Job 39:18}] les bêtes des champs peuvent les f.
\item[\vref{Es 3:15}] vs. revient-il de f. mon peuple, et
\item[\vref{Es 63:3}] été seul à f. au pressoir, et
\item[\vref{La 3:34}] [Lamed.] Lorsqu'on f. aux pieds ts
\item[\vref{Ez 34:18}] pour que vs. f. de vos pieds
\item[\vref{Os 10:11}] qui aime à f. le blé, mais
\item[\vref{Am 2:13}] je m'en vais f. le lieu où
\item[\vref{Mal 4:3}] Et vs. f. les méchants, car ils seront com.
\item[\vref{Lu 21:24}] et Jérus. sera f. par les nations,
\item[\vref{1 Co 9:10}] et celui qui f. le blé, le
\item[\vref{1 Ti 5:18}] bœuf qnd il f. le grain. Et
\item[\vref{Hé 10:29}] celui qui aura f. aux pieds le
\item[\vref{Ap 11:2}] Gentils, et ils f. aux pieds la
\end{listverse}

\ConcordanceEntry{Fourmi}
\vspace{-2mm}
\begin{listverse}
\item[\vref{Pr 6:6}] paresseux, vers la f., regarde ses voies
\item[\vref{Pr 30:25}] Les f., qui sont un peuple sans puissance,
\end{listverse}

\ConcordanceEntry{Fournaise}
\vspace{-2mm}
\begin{listverse}
\item[\vref{Ge 15:17}] ce fut une f. fumante, et des
\item[\vref{1 R 8:51}] sortir hors d'Egypte, du milieu d'une f. de fer !
\item[\vref{Ps 21:10}] rendras tels qu'une f. ardente le jour
\item[\vref{Da 3:6}] milieu de la f. de feu ardent.
\item[\vref{Da 3:19}] de chauffer la f. sept fois plus
\item[\vref{Mal 4:1}] ardent com. une f.. Tous les orgueilleux
\item[\vref{Mt 13:42}] jetteront ds la f. ardente, où il
\item[\vref{1 Pi 4:12}] com. ds une f. pour votre épreuve,
\item[\vref{Ap 1:15}] embrasés ds une f. ; et sa voix
\item[\vref{Ap 9:2}] fumée d'une grande f. ; et le soleil
\end{listverse}

\ConcordanceEntry{Frapper}
\vspace{-2mm}
\begin{listverse}
\item[\vref{Ex 2:13}] avait tort : Pourquoi f.-tu ton prochain ?
\item[\vref{Ex 12:23}] Yahweh passera pour f. l'Egypte et il
\item[\vref{Lé 24:17}] Celui qui aura f. mortellement un hom.,
\item[\vref{No 20:11}] sa main, et f. deux fois le
\item[\vref{De 28:28}] Yahweh te f. de folie, d'aveuglement,
\item[\vref{Jos 9:19}] mntnt ns. ne pouvons pas les f.
\item[\vref{Jg 6:16}] avec toi, tu f. les Madianites com.
\item[\vref{1 S 18:7}] disant : Saül a f. ses mille, et
\item[\vref{1 S 19:10}] Saül voulut f. David avec sa
\item[\vref{1 S 26:10}] seul qui le f., soit que son
\item[\vref{2 S 17:2}] s'enfuira, et je f. seulement le roi ;
\item[\vref{2 R 13:18}] au roi d'Israël : F. contre terre. Et
\item[\vref{Ps 73:5}] ne sont point f. avec les autres
\item[\vref{Pr 23:35}] malade ! On m'a f., et je ne
\item[\vref{Es 10:20}] celui qui les f., mais ils s'appuieront
\item[\vref{Es 53:4}] l'avons considéré com. f., battu par Dieu
\item[\vref{Es 60:10}] car je t'ai f. ds ma colère,
\item[\vref{Jé 37:15}] contre Jérémie, le f. et le mirent
\item[\vref{Ez 7:9}] saurez que je suis Yahweh qui f.
\item[\vref{Za 14:12}] plaie dont Yahweh f. ts les peuples
\item[\vref{Mt 7:7}] et vs. trouverez ; f., et l'on vs.
\item[\vref{Mt 26:68}] prophétise-ns. qui est celui qui t'a f.
\item[\vref{Mt 27:30}] le roseau et f. sur sa tête.
\item[\vref{Lu 13:25}] vs. mettrez à f. à la porte,
\item[\vref{Jn 18:23}] si j'ai bien parlé, pourquoi me f.-tu ?
\item[\vref{Ac 12:16}] Pierre continuait à f.. Et qnd ils
\item[\vref{Ac 21:32}] et les soldats, ils cessèrent de f. Paul.
\item[\vref{Hé 12:6}] aime, et il f. de la verge
\item[\vref{Ap 3:20}] porte, et je f.. Si quelqu'un entend
\item[\vref{Ap 19:15}] épée tranchante, pour f. les nations ; il
\end{listverse}

\ConcordanceEntry{Fraternité}
\vspace{-2mm}
\begin{listverse}
\item[\vref{Za 11:14}] pour rompre la f. entre Juda et
\end{listverse}

\ConcordanceEntry{Fraude}
\vspace{-2mm}
\begin{listverse}
\item[\vref{Ps 32:2}] duquel il n'y a point de f. !
\item[\vref{Ps 55:12}] tromperie et la f. ne partent point
\item[\vref{Ps 72:14}] âme de la f. et de la
\item[\vref{Ps 109:2}] bouche remplie de f. se sont ouvertes
\item[\vref{Pr 12:5}] conseils des méchants ne sont que f.
\item[\vref{Pr 13:11}] provenues de la f. seront diminuées, mais
\item[\vref{Es 53:9}] point eu de f. ds sa bouche.
\item[\vref{Jé 5:27}] sont remplies de f. ; c'est par ce
\item[\vref{Ez 18:18}] a usé de f., et qu'il a
\item[\vref{Da 8:25}] fera prospérer la f. ds sa main.
\item[\vref{So 1:9}] violence et de f. la maison de
\item[\vref{Mc 7:22}] les méchancetés, la f., l'impudicité, le regard
\item[\vref{Lu 3:14}] ni extorsion ni f. envers personne, mais
\item[\vref{Jn 1:47}] lequel il n'y a pas de f.
\item[\vref{Ac 13:10}] plein de tte f. et de tte
\item[\vref{Ro 1:29}] de querelle, de f., de mauvaises mœurs,
\item[\vref{1 Th 2:3}] ni séduction, ni motif impur, ni f.
\item[\vref{1 Th 4:6}] personne n'use de f. envers son frère
\item[\vref{1 Pi 2:1}] malice, de tte f., de dissimulation, d'envie
\item[\vref{1 Pi 2:22}] il ne s'est pas trouvé de f. ;
\item[\vref{1 Pi 3:10}] et ses lèvres de prononcer aucune f.,
\item[\vref{Ap 14:5}] pas trouvé de f., car ils sont
\end{listverse}

\ConcordanceEntry{Frayeur}
\vspace{-2mm}
\begin{listverse}
\item[\vref{Ge 15:12}] et voici, une f. d'une grande obscurité
\item[\vref{Ex 14:10}] eurent une grande f. et crièrent à
\item[\vref{No 22:3}] eut une grande f. du peuple, parce
\item[\vref{De 26:8}] avec une grande f., avec des signes
\item[\vref{Jos 10:2}] eut une grande f.. Parce que Gabaon
\item[\vref{1 S 14:15}] troublé que cela fut com. une f. de Dieu.
\item[\vref{1 Ch 14:17}] Yahweh remplit de f. ttes ces nations-là,
\item[\vref{Est 8:17}] parce que la f. des Juifs les
\item[\vref{Job 9:34}] et que la f. que j'ai de
\item[\vref{Job 39:25}] rit de la f., il ne s'épouvante
\item[\vref{Ps 34:5}] il m'a délivré de ttes mes f.
\item[\vref{Es 24:17}] La f., la fosse, et le piège sont
\item[\vref{Jé 20:10}] de plusieurs, la f. m'a saisi de
\item[\vref{Jé 46:5}] derrière eux… La f. les environne, dit
\item[\vref{Jé 48:39}] moquerie et de f. pour ts ceux
\item[\vref{La 3:47}] La f. et la fosse, le dégât et
\item[\vref{Ez 30:13}] je mettrai la f. ds le pays
\item[\vref{Da 6:26}] et de la f. pour le Dieu
\item[\vref{Da 10:7}] saisis d'une grande f., et ils s'enfuirent
\item[\vref{Mt 14:26}] Et, ds lr. f., ils poussèrent des
\item[\vref{Mt 27:54}] saisis d'une grande f., et dirent : Certainement
\item[\vref{Mt 28:4}] tellement saisis de f., qu'ils devinrent com.
\item[\vref{Lu 5:9}] Parce que la f. l'avait saisi, lui
\item[\vref{Lu 8:25}] foi ? Saisis de f. et d'étonnement, ils
\item[\vref{Lu 9:34}] furent saisis de f. en les voyant
\item[\vref{Lu 21:26}] rendant l'âme de f., ds l'attente des
\item[\vref{Lu 24:5}] Saisies de f., elles baissèrent le
\item[\vref{Jud 1:23}] autres par la f., les arrachant com.
\end{listverse}

\ConcordanceEntry{Frère}
\vspace{-2mm}
\begin{listverse}
\item[\vref{Ge 4:9}] est Abel ton f. ? Et il lui
\item[\vref{Ge 37:16}] Je cherche mes f. ; je te prie,
\item[\vref{Ex 7:1}] et Aaron, ton f., sera ton prophète.
\item[\vref{Lé 25:25}] Si ton f. est devenu pauvre et vend qq
\item[\vref{2 S 13:12}] répondit : Non, mon f., ne me déshonore
\item[\vref{1 R 12:24}] point contre vos f., les fils d'Israël !
\item[\vref{Né 5:7}] chacun de son f. ; et je fis
\item[\vref{Pr 17:17}] naîtra com. un f. ds la détresse.
\item[\vref{Pr 18:19}] Un f. offensé se rend plus difficile qu'une
\item[\vref{Ca 8:1}] es com. mon f., qui a sucé
\item[\vref{Es 19:2}] guerre contre son f., et chacun contre
\item[\vref{Jé 9:4}] confiez en aucun f. ; car tt frère
\item[\vref{Jé 31:34}] ni personne son f., en disant : Connaissez
\item[\vref{Za 7:10}] ds vos cœurs chacun contre son f.
\item[\vref{Mt 5:23}] souviennes que ton f. a qq chose
\item[\vref{Mt 12:50}] celui-là est mon f., et ma sœur,
\item[\vref{Mt 18:15}] Et si ton f. a péché contre
\item[\vref{Mt 18:21}] pardonnerai-je à mon f. lorsqu'il aura péché
\item[\vref{Mt 18:35}] chacun à son f., ses fautes.
\item[\vref{Mt 22:24}] sans enfants, son f. épousera sa fem.,
\item[\vref{Mt 23:8}] votre Docteur ; et vs. êtes ts f.
\item[\vref{Mt 28:10}] dites à mes f. d'aller en Galilée,
\item[\vref{Mc 6:3}] fils de Marie, f. de Jacques, de
\item[\vref{Mc 13:12}] Or le f. livrera son frère à la mort,
\item[\vref{Lu 12:13}] dis à mon f. qu'il partage avec
\item[\vref{Lu 15:32}] parce que ton f. que voici était
\item[\vref{Lu 17:3}] Si dc ton f. a péché contre
\item[\vref{Lu 22:32}] seras un jour converti, affermis tes f.
\item[\vref{Jn 7:5}] Car ses f. non plus ne croyaient pas en
\item[\vref{Jn 11:21}] été ici mon f. ne serait pas
\item[\vref{Ac 1:14}] mère de Jésus, et avec ses f.
\item[\vref{Ac 7:25}] croyait que ses f. comprendraient par là
\item[\vref{Ro 14:10}] pourquoi juges-tu ton f. ? Ou toi, aussi,
\item[\vref{1 Co 5:11}] qui se nomme f., est fornicateur, ou
\item[\vref{1 Co 6:6}] Mais un f. a des procès contre son frère,
\item[\vref{1 Co 7:15}] se sépare ; le f. ou la sœur
\item[\vref{1 Co 8:11}] Et ainsi ton f., qui est faible,
\item[\vref{1 Co 8:13}] viande scandalise mon f., je ne mangerai
\item[\vref{1 Co 15:6}] de cinq cents f. à la fois,
\item[\vref{Ga 2:4}] cause des faux f. qui s'étaient furtivement
\item[\vref{1 Th 4:6}] fraude envers son f. et de cupidité
\item[\vref{1 Th 5:26}] Saluez ts les f. par un saint
\item[\vref{2 Th 3:6}] recommandons aussi, mes f., au Nom de
\item[\vref{2 Th 3:15}] un ennemi, mais avertissez-le com. un f.
\item[\vref{Ja 1:9}] Que le f. de basse condition se glorifie ds
\item[\vref{2 Pi 3:15}] com. Paul, notre f. bien-aimé vs. l'a
\item[\vref{1 Jn 3:10}] n'aime pas son f. n'est point de
\item[\vref{1 Jn 3:12}] qui tua son f.. Et pourquoi le
\item[\vref{1 Jn 3:15}] Quiconque hait son f. est un meurtrier,
\item[\vref{1 Jn 3:17}] que voyant son f. ds la nécessité,
\item[\vref{1 Jn 4:21}] qui aime Dieu, aime aussi son f.
\item[\vref{1 Jn 5:16}] quelqu'un voit son f. commettre un péché
\item[\vref{Ap 1:9}] suis aussi votre f. et qui participe
\end{listverse}

\ConcordanceEntry{Froid}
\vspace{-2mm}
\begin{listverse}
\item[\vref{Ge 8:22}] les moissons, le f. et la chaleur,
\item[\vref{Ge 31:40}] la nuit le f. ; et le sommeil
\item[\vref{Job 37:9}] midi, et le f. vient des vents
\item[\vref{Ps 147:17}] morceaux ; qui peut résister dvt son f. ?
\item[\vref{Pr 25:20}] ds un jour f., et com. du
\item[\vref{Jn 18:18}] parce qu'il faisait f., et ils se
\item[\vref{Ac 28:2}] pluie tombait et qu'il faisait très f.
\item[\vref{2 Co 11:27}] jeûnes, ds le f. et ds la
\item[\vref{Ap 3:15}] tu n'es ni f. ni bouillant ; puisses-tu
\end{listverse}

\ConcordanceEntry{Front}
\vspace{-2mm}
\begin{listverse}
\item[\vref{Ex 28:38}] sera sur le f. d'Aaron ; et Aaron
\item[\vref{1 S 17:49}] le Philistin au f. que la pierre
\item[\vref{2 Ch 26:19}] parut sur son f., en présence des
\item[\vref{Es 48:4}] et que ton f. est d'airain,
\item[\vref{Ez 3:7}] d'Israël a le f. dur et le
\item[\vref{Ez 3:8}] et j'endurcirai ton f. contre leurs fronts.
\item[\vref{Ez 9:4}] Tav sur les f. des hommes qui
\item[\vref{Ap 7:3}] serviteurs de notre Dieu sur leurs f.
\item[\vref{Ap 13:16}] lr. main droite, ou sur lr. f. ;
\item[\vref{Ap 14:1}] de son Père écrit sur leurs f.
\item[\vref{Ap 14:9}] marque sur son f. ou sur sa
\item[\vref{Ap 17:5}] avait sur son f. un nom écrit,
\end{listverse}

\ConcordanceEntry{Fruit}
\vspace{-2mm}
\begin{listverse}
\item[\vref{Ge 3:2}] Nous mangeons du f. des arbres du
\item[\vref{Ge 3:12}] m'a donné du f. de l'arbre, et
\item[\vref{De 1:25}] leurs mains des f. du pays, et
\item[\vref{De 28:11}] biens ds le f. de tes entrailles,
\item[\vref{Job 31:39}] j'ai mangé son f. sans argent ; si
\item[\vref{Ps 1:3}] qui rend son f. en sa saison,
\item[\vref{Pr 8:19}] Mon f. est meilleur que l'or fin, mm
\item[\vref{Pr 11:30}] Le f. du juste est un arbre de
\item[\vref{Pr 12:14}] biens par le f. de sa bouche,
\item[\vref{Ca 2:3}] assise et son f. a été doux
\item[\vref{Es 4:2}] gloire, et le f. de la terre
\item[\vref{Jé 17:8}] et ne cessera de porter du f.
\item[\vref{Ez 36:30}] Je multiplierai le f. des arbres et
\item[\vref{Ez 47:12}] trouvera toujours du f.. Tous les mois,
\item[\vref{Os 14:8}] de moi que tu recevras ton f.
\item[\vref{Joë 2:22}] arbres portent lr. f. ; le figuier et
\item[\vref{Mt 3:8}] Produisez dc des f. convenables à la
\item[\vref{Mt 7:16}] reconnaîtrez à leurs f.. Cueille-t-on des raisins
\item[\vref{Mt 13:8}] elle donna du f., un grain en
\item[\vref{Mt 21:43}] une nation qui en rendra les f.
\item[\vref{Mt 26:29}] pas de ce f. de vigne, jusqu'au
\item[\vref{Lu 13:6}] y chercher du f., mais il n'en
\item[\vref{Lu 22:18}] boirai plus du f. de la vigne,
\item[\vref{Jn 4:36}] et amasse le f. pour la vie
\item[\vref{Jn 12:24}] s'il meurt, il porte beaucoup de f.
\item[\vref{Jn 15:5}] porte beaucoup de f., car hors de
\item[\vref{Jn 15:16}] vs. produisiez du f., et que votre
\item[\vref{Ro 1:13}] de recueillir qq f. aussi bien parmi
\item[\vref{Ro 6:21}] Quel f. dc aviez-vs. alors ? Des fruits dont
\item[\vref{Ro 6:22}] vs. avez pour f. la sanctification et
\item[\vref{Ga 5:22}] Mais le f. de l'Esprit c'est la charité, la
\item[\vref{Ep 5:9}] car le f. de l'Esprit consiste en tte bonté,
\item[\vref{Ph 4:17}] je cherche le f. qui abonde pour
\item[\vref{Col 1:6}] il porte des f., com. aussi parmi
\item[\vref{Col 1:10}] choses, portant des f. en ttes sortes
\item[\vref{Hé 12:11}] il produit un f. paisible de justice
\item[\vref{Ja 5:7}] attend le précieux f. de la terre,
\item[\vref{Ap 22:2}] vie, portant douze f., et rendant son
\end{listverse}

\ConcordanceEntry{Fugitif}
\vspace{-2mm}
\begin{listverse}
\item[\vref{Ge 4:12}] seras vagabond et f. sur la terre.
\item[\vref{Es 21:14}] viennent au-dvt du f. avec du pain
\item[\vref{Ez 33:22}] avant l'arrivée du f., et Yahweh ouvrit
\end{listverse}

\ConcordanceEntry{Fuir, S'enfuir}
\vspace{-2mm}
\begin{listverse}
\item[\vref{Ge 16:8}] elle répondit : Je m'e. de dvt Saraï,
\item[\vref{Lé 26:17}] vs. ; et vs. f. sans que personne
\item[\vref{No 35:32}] pour le laisser s'e. de sa ville
\item[\vref{Jos 8:20}] aucune force pour f. ça ou là.
\item[\vref{2 S 24:13}] trois mois tu f. dvt tes ennemis
\item[\vref{Né 6:11}] tel que moi s'e.-il ? Et quel
\item[\vref{Job 27:22}] sa main, il ne cessera de f.
\item[\vref{Ps 68:2}] qui le haïssent s'e. dvt lui.
\item[\vref{Ps 139:7}] Esprit, et où f.-je loin de
\item[\vref{Os 7:13}] parce qu'ils me f. ! Ils seront exposés
\item[\vref{Am 2:14}] ne pourra pas f., et le fort
\item[\vref{Jon 1:3}] se leva pour s'e. à Tarsis, loin
\item[\vref{Mt 3:7}] a appris à f. la colère à
\item[\vref{Mt 10:23}] ds une ville, f. ds une autre.
\item[\vref{Mt 26:56}] Alors ts les disciples l'abandonnèrent et s'e.
\item[\vref{Lu 3:7}] a appris à f. la colère à
\item[\vref{Ja 4:7}] Dieu ; résistez au diable, et il s'e. de vs.
\end{listverse}

\ConcordanceEntry{Fuite}
\vspace{-2mm}
\begin{listverse}
\item[\vref{Ge 31:27}] as-tu pris la f. secrètement, m'as-tu trompé
\item[\vref{1 S 19:18}] David prit la f. et qu'il s'échappa.
\item[\vref{2 S 10:14}] avaient pris la f., ils s'enfuirent aussi
\item[\vref{2 S 13:34}] Absalom prit la f.. Or le jeune
\item[\vref{2 S 19:3}] d'avoir pris la f. ds la bataille.
\item[\vref{Ps 104:7}] mirent promptement en f. au son de
\item[\vref{Pr 28:1}] méchant prend la f. sans qu'on le
\item[\vref{Jé 26:21}] peur, prit la f., et se retira
\item[\vref{Ez 26:18}] seront terrifiées à cause de ta f.
\item[\vref{Mt 24:20}] pour que votre f. n'arrive pas en
\item[\vref{Mc 13:18}] Dieu que votre f. n'arrive pas en
\item[\vref{Hé 11:34}] et mirent en f. des armées étrangères.
\end{listverse}

\ConcordanceEntry{Fumée}
\vspace{-2mm}
\begin{listverse}
\item[\vref{Ge 19:28}] la terre une f. com. la fumée
\item[\vref{Ex 19:18}] tt couvert de f., parce que Yahweh
\item[\vref{Job 41:11}] Une f. sort de ses narines com. d'un
\item[\vref{Ps 37:20}] beaux pâturages, s'évanouissent ; ils s'évanouissent en f.
\item[\vref{Ps 68:3}] chasseras com. la f. est chassée par
\item[\vref{Pr 10:26}] dents et la f. aux yeux, tel
\item[\vref{Ca 3:6}] des colonnes de f. en forme de
\item[\vref{Es 6:4}] et la maison fut remplie de f.
\item[\vref{Es 14:31}] d'Aquilon vient une f., et il ne
\item[\vref{Es 65:5}] Ceux-là sont une f. ds mes narines,
\item[\vref{Ap 8:4}] Et la f. des parfums monta avec les prières
\item[\vref{Ap 9:2}] l'abîme, et une f. monta du puits
\item[\vref{Ap 14:11}] Et la f. de lr. tourment montera aux siècles
\item[\vref{Ap 18:9}] ils verront la f. de son embrasement ;
\end{listverse}

\ConcordanceEntry{Fureur}
\vspace{-2mm}
\begin{listverse}
\item[\vref{Ge 27:44}] ce que la f. de ton frère
\item[\vref{2 Ch 36:16}] ce que la f. de Yahweh monta
\item[\vref{Job 16:9}] Sa f. me déchire, il se déclare mon
\item[\vref{Job 18:4}] toi-mm ds ta f., la terre sera-t-elle
\item[\vref{Ps 7:7}] lève-toi contre la f. de mes adversaires.
\item[\vref{Ps 38:2}] ne me châtie pas ds ta f.
\item[\vref{Ps 76:11}] mm ds sa f., qnd tu te
\item[\vref{Ps 79:6}] Répands ta f. sur les nations
\item[\vref{Pr 6:34}] mari est une f., il n'épargnera pas
\item[\vref{Pr 15:1}] douce apaise la f. ; mais la parole
\item[\vref{Da 3:13}] colère et de f., ordonna qu'on amène
\item[\vref{Da 3:19}] fut rempli de f., et il changea
\item[\vref{Na 1:2}] est plein de f. ; Yahweh se venge
\item[\vref{So 1:15}] un jour de f., un jour de
\item[\vref{Lu 6:11}] furent remplis de f., et ils s'entretenaient
\item[\vref{Ac 26:11}] mes excès de f. contre eux, je
\item[\vref{Hé 11:27}] sans craindre la f. du roi ; car
\item[\vref{Ap 12:12}] animé d'une grande f., sachant qu'il a
\item[\vref{Ap 14:8}] vin de la f. de son impudicité !
\end{listverse}

\ConcordanceEntry{Furieux}
\vspace{-2mm}
\begin{listverse}
\item[\vref{2 R 19:27}] comment tu es f. contre moi.
\item[\vref{2 R 19:28}] que tu es f. contre moi et
\item[\vref{Ps 102:9}] ceux qui sont f. contre moi, jurent
\item[\vref{Pr 15:18}] L'hom. f. excite la querelle, mais l'hom. lent
\item[\vref{Pr 29:22}] querelles, et l'hom. f. commet plusieurs transgressions.
\end{listverse}

\ConcordanceEntry{Gabaon, Gabaonites}
\vspace{-2mm}
\begin{listverse}
\item[\vref{Jos 9:3}] les habitants de G., ayant entendu ce
\end{listverse}
\begin{legend}
\NoAutoSpaceBeforeFDP{
\item Trompent Israël et concluent une alliance : Josué 9:3-14;  11:19
\item Asservis par Israël : Jos 9:19--21;  21:17
\item Autres : 2 S 21:1-9; 1 R 3:4; 1 Ch 16:39
}
\end{legend}

\ConcordanceEntry{Gabriel}
\vspace{-2mm}
\begin{listverse}
\item[\vref{Da 8:16}] cria et dit : G., explique-lui la vision.
\item[\vref{Da 9:21}] prière, qnd l'hom. G., que j'avais vu
\item[\vref{Lu 1:19}] dit : Je suis G., je me tiens
\item[\vref{Lu 1:26}] sixième mois, l'ange G. fut envoyé par
\end{listverse}

\ConcordanceEntry{Gad}
\vspace{-2mm}
\begin{listverse}
\item[\vref{Ge 30:11}] pourquoi elle l'appela du nom de G.
\item[\vref{Ge 49:19}] Quant à G., des troupes viendront l'attaquer, mais il
\item[\vref{No 1:25}] la tribu de G., qui furent dénombrés,
\item[\vref{Jos 13:24}] la tribu de G., pour les fils
\item[\vref{Jos 13:28}] des fils de G., selon leurs familles ;
\item[\vref{1 S 22:5}] Or G., le prophète, dit à David : Ne
\item[\vref{1 Ch 21:9}] Yahweh parla à G., le voyant de
\item[\vref{1 Ch 29:29}] le livre de G. le prophète,
\item[\vref{Ap 7:5}] la tribu de G., douze mille marqués
\end{listverse}

\ConcordanceEntry{Gadaréniens}
\vspace{-2mm}
\begin{listverse}
\item[\vref{Mt 8:28}] le pays des G., deux démoniaques, sortant
\item[\vref{Mc 5:1}] la mer, ds le pays des G.
\item[\vref{Lu 8:26}] le pays des G. qui est vis-à-vis
\end{listverse}

\ConcordanceEntry{Gaïus}
\vspace{-2mm}
\begin{listverse}
\item[\vref{Ac 19:29}] théâtre, et enlevèrent G. et Aristarque, Macédoniens,
\item[\vref{Ac 20:4}] Second de Thessalonique, G. de Derbe, Timothée,
\item[\vref{Ro 16:23}] G., mon hôte, et celui de tte
\item[\vref{3 Jn 1:1}] L'ancien, à G. le bien-aimé, que
\end{listverse}

\ConcordanceEntry{Galaad}
\vspace{-2mm}
\begin{listverse}
\item[\vref{No 26:29}] Makirites. Makir engendra G.. De Galaad descend
\item[\vref{No 32:1}] le pays de G. étaient un lieu
\item[\vref{No 32:40}] Moïse dc donna G. à Makir, fils
\item[\vref{Jos 12:2}] la moitié de G., jusqu'au torrent de
\item[\vref{Jos 17:1}] et père de G., il avait eu
\item[\vref{2 S 2:4}] de Jabès en G. ont enseveli Saül.
\item[\vref{Ps 60:9}] G. est à moi, Manassé aussi est
\end{listverse}

\ConcordanceEntry{Galatie, Galates}
\vspace{-2mm}
\begin{listverse}
\item[\vref{Ac 16:6}] le pays de G., le Saint-Esprit lr.
\end{listverse}
\begin{legend}
\NoAutoSpaceBeforeFDP{
\item Province d'Asie Mineure évangélisée par Paul : Ac 16:6, 18:23;  Ga 1:8-10, 3:1, 4:13
\item Épîtres de Paul et de Pierre adressées aux disciples de Galatie : Ga et 1 Pi
\item Se sont attachés à un autre Evangile : Ga 1:6-9, 4:17, 5:2-8
}
\end{legend}

\ConcordanceEntry{Galilée}
\vspace{-2mm}
\begin{listverse}
\item[\vref{Jos 20:7}] dc Kédesch, en G., ds la montagne
\item[\vref{1 R 9:11}] vingt villes ds le pays de G.
\item[\vref{Mt 4:25}] le suivit, de G., de la Décapole,
\item[\vref{Mt 26:32}] serai ressuscité, je vs. précéderai en G.
\item[\vref{Mc 1:14}] alla ds la G., prêchant l'Evangile du
\item[\vref{Mc 1:39}] par tte la G., et chassait les
\item[\vref{Lu 1:26}] une ville de G., appelée Nazareth,
\item[\vref{Lu 4:31}] Capernaüm, ville de G., et il les
\item[\vref{Jn 4:46}] à Cana de G., où il avait
\item[\vref{Jn 7:52}] qu'aucun prophète n'est sorti de la G.
\end{listverse}

\ConcordanceEntry{Gamaliel}
\vspace{-2mm}
\begin{listverse}
\item[\vref{Ac 5:34}] un pharisien nommé G., docteur de la
\item[\vref{Ac 22:3}] aux pieds de G., et instruit ds
\end{listverse}

\ConcordanceEntry{Garde}
\vspace{-2mm}
\begin{listverse}
\item[\vref{Né 4:9}] ns. établîmes une g. contre eux, jour
\item[\vref{Es 62:6}] j'ai placé des g. sur tes murailles
\item[\vref{Mt 28:4}] Les g. furent tellement saisis de frayeur, qu'ils
\item[\vref{Ac 12:10}] et la seconde g., ils arrivèrent à
\item[\vref{Ga 3:23}] renfermés sous la g. de la loi,
\end{listverse}

\ConcordanceEntry{Garder}
\vspace{-2mm}
\begin{listverse}
\item[\vref{Ge 17:9}] à Abraham : Tu g. dc mon alliance,
\item[\vref{Ge 28:15}] et je te g. partout où tu
\item[\vref{Ex 19:5}] et si vs. g. mon alliance, alors
\item[\vref{No 6:24}] Yahweh te bénisse, et te g. !
\item[\vref{De 7:9}] Ce Dieu fidèle g. son alliance et
\item[\vref{De 8:2}] et si tu g. ses commandements ou
\item[\vref{De 27:9}] disant : Ecoute et g. le silence, Israël !
\item[\vref{Jos 22:5}] Prenez seulement bien g. d'observer les ordonnances
\item[\vref{1 S 2:9}] Il g. les pieds de ses bien-aimés, et
\item[\vref{1 S 22:23}] tienne ; avec moi, tu seras bien g.
\item[\vref{Ps 25:10}] pour ceux qui g. son alliance et
\item[\vref{Ps 78:10}] Ils ne g. point l'alliance de Dieu et refusèrent
\item[\vref{Ps 91:11}] anges de te g. ds ttes tes
\item[\vref{Ps 97:10}] le mal ! Il g. les âmes de
\item[\vref{Ps 103:18}] pour ceux qui g. son alliance, et
\item[\vref{Ps 119:2}] sont ceux qui g. ses préceptes et
\item[\vref{Ps 119:34}] de l'intelligence ; je g. ta loi et
\item[\vref{Ps 121:3}] celui qui te g. ne sommeillera point.
\item[\vref{Ps 121:5}] celui qui te g., Yahweh est ton
\item[\vref{Ps 121:8}] Yahweh g. ton départ et ton arrivée, dès
\item[\vref{Ps 127:1}] si Yahweh ne g. la ville, celui
\item[\vref{Pr 3:21}] dvt tes yeux ; g. la sagesse et
\item[\vref{Pr 6:20}] Mon fils, g. le commandement de
\item[\vref{Es 58:8}] la gloire de Yahweh sera ton arrière-g.
\item[\vref{Jé 3:12}] Yahweh, je ne g. pas ma colère
\item[\vref{Jé 31:10}] et il le g. com. un berger
\item[\vref{Da 9:4}] redoutable, toi qui g. ton alliance et
\item[\vref{Mi 7:5}] en tes conducteurs ; g.-toi d'ouvrir ta
\item[\vref{Mc 6:28}] Le g. alla décapiter Jean ds la prison,
\item[\vref{Lu 11:28}] parole de Dieu, et qui la g. !
\item[\vref{Lu 17:3}] Prenez g. à vs.-mêmes. Si dc ton frère
\item[\vref{Jn 8:51}] dis : Si quelqu'un g. ma parole, il
\item[\vref{Jn 14:15}] Si vs. m'aimez, g. mes commandements.
\item[\vref{Jn 15:10}] Si vs. g. mes commandements, vs. demeurerez ds mon
\item[\vref{Jn 15:20}] aussi ; s'ils ont g. ma parole, ils
\item[\vref{Jn 17:6}] et ils ont g. ta parole.
\item[\vref{Jn 17:11}] toi. Père saint, g.-les en ton
\item[\vref{Jn 17:12}] monde, je les g. en ton Nom.
\item[\vref{Ac 20:28}] Prenez dc g. à vs.-mêmes, et
\item[\vref{Col 4:17}] à Archippe : Prends g. au service que
\item[\vref{2 Ti 1:14}] G. le bon dépôt par le Saint-Esprit
\item[\vref{1 Pi 1:5}] qui sommes g. par la puissance
\item[\vref{1 Jn 2:4}] et qui ne g. point ses commandements,
\item[\vref{1 Jn 2:5}] Mais celui qui g. sa parole, l'amour
\item[\vref{1 Jn 5:3}] Dieu : Que ns. g. ses commandements ; et
\item[\vref{Ap 1:3}] prophétie, et qui g. les choses qui
\item[\vref{Ap 3:8}] que tu as g. ma parole, et
\item[\vref{Ap 3:10}] que tu as g. la parole de
\item[\vref{Ap 14:12}] sont ceux qui g. les commandements de
\item[\vref{Ap 16:15}] veille et qui g. ses vêtements, afin
\item[\vref{Ap 22:7}] est celui qui g. les paroles de
\item[\vref{Ap 22:9}] il me dit : G.-toi de le
\end{listverse}

\ConcordanceEntry{Gaspiller}
\vspace{-2mm}
\begin{listverse}
\item[\vref{Mc 5:26}] médecins. Elle avait g. tt ce qu'elle
\item[\vref{Lu 15:14}] qu'il eut tt g., une grande famine
\end{listverse}

\ConcordanceEntry{Gâteau}
\vspace{-2mm}
\begin{listverse}
\item[\vref{Ex 12:39}] firent cuire des g. sans levain avec
\item[\vref{Lé 2:4}] une offrande de g. cuits au four,
\item[\vref{Lé 23:18}] Yahweh, avec leurs g. et leurs libations,
\item[\vref{Lé 24:5}] feras cuire douze g., chaque gâteau sera
\item[\vref{No 6:15}] sans levain, de g. de fine farine,
\item[\vref{No 29:18}] avec les g. et les libations pour les jeunes
\item[\vref{Jg 7:13}] me semblait qu'un g. de pain d'orge
\item[\vref{1 R 14:3}] dix pains, des g. et un vase
\item[\vref{1 R 19:6}] son chevet, un g. cuit sur des
\item[\vref{Jé 7:18}] pour faire des g. à la reine
\item[\vref{Os 7:8}] est com. un g. qui n'a pas
\end{listverse}

\ConcordanceEntry{Gath}
\vspace{-2mm}
\begin{listverse}
\item[\vref{Jos 11:22}] qu'à Gaza, à G. et à Asdod.
\item[\vref{Jos 13:3}] d'Askalon, celui de G., celui d'Ekron, et
\item[\vref{1 S 5:8}] Qu'on transporte à G. l'arche du Dieu
\item[\vref{1 S 17:4}] la ville de G., haut de six
\item[\vref{1 S 17:23}] le Philistin de G., nommé Goliath, sortit
\item[\vref{1 S 27:4}] s'était enfui à G. ; et il cessa
\item[\vref{2 S 21:19}] tua Goliath de G., qui avait une
\item[\vref{2 S 21:22}] étaient nés à G., de la race
\item[\vref{1 Ch 20:5}] de Goliath de G., qui avait une
\end{listverse}

\ConcordanceEntry{Gaza}
\vspace{-2mm}
\begin{listverse}
\item[\vref{Ge 10:19}] vers Guérar, jusqu'à G., en allant vers
\item[\vref{Jos 13:3}] Philistins, celui de G., celui d'Asdod, celui
\item[\vref{Jg 16:1}] s'en alla à G. ; il y vit
\item[\vref{Jg 16:2}] aux gens de G. : Samson est venu
\item[\vref{Jg 16:21}] le descendirent à G., et le lièrent
\item[\vref{1 S 6:17}] Asdod, un pour G., un pour Askalon,
\item[\vref{1 R 4:24}] depuis Thiphsach jusqu'à G., sur ts les
\item[\vref{2 R 18:8}] les Philistins jusqu'à G. et ravagea lr.
\item[\vref{1 Ch 7:28}] son ressort, jusqu'à G. avec les villes
\item[\vref{Am 1:6}] trois crimes de G., et mm de
\item[\vref{So 2:4}] Mais G. sera abandonnée, et Askalon sera en
\end{listverse}

\ConcordanceEntry{Géant}
\vspace{-2mm}
\begin{listverse}
\item[\vref{Ge 6:4}] ce temps-là des g. sur la terre,
\item[\vref{No 13:33}] vu aussi des g., des enfants d'Anak,
\end{listverse}

\ConcordanceEntry{Gédéon}
\vspace{-2mm}
\begin{listverse}
\item[\vref{Jg 6:11}] la famille d'Abiézer. G., son fils, battait
\end{listverse}
\begin{legend}
\NoAutoSpaceBeforeFDP{
\item Issu de la tribu de Manassé et fut juge en Israêl : Jg 6:11, 8:28; Hé 11:32
\item Bâtit l'autel de Yahweh-Shalom : Jg 6:24
\item Renverse l'autel de Baal et son nom devient Jerubbaal : Jg 6: 25-27,32
\item Gedeon et ses 300 hommes : Jg 7:1
\item Ephod de G. idolâtré par Israël : Jg 8:31
\item Sa mort : Jg 8:32
\item Autres : Jg 6:11-16,36-40, 7:13-14, 22-25, 8:24-27,30
}
\end{legend}

\ConcordanceEntry{Géhenne}
\vspace{-2mm}
\begin{listverse}
\item[\vref{Mt 5:22}] puni par le feu de la g.
\item[\vref{Mt 5:29}] ne soit pas jeté ds la g.
\item[\vref{Mt 5:30}] ne soit pas jeté ds la g.
\item[\vref{Mt 10:28}] corps en les jetant ds la g.
\item[\vref{Mt 18:9}] jeté ds le feu de la g.
\item[\vref{Mt 23:15}] fils de la g., deux fois plus
\item[\vref{Mt 23:33}] Comment éviterez-vs. le supplice de la g. ?
\item[\vref{Mc 9:43}] d'aller ds la g., ds le feu
\item[\vref{Ja 3:6}] enflammée par le feu de la g.
\end{listverse}

\ConcordanceEntry{Gémir}
\vspace{-2mm}
\begin{listverse}
\item[\vref{Job 24:12}] Ils font g. les gens ds la ville l'âme
\item[\vref{Ps 6:7}] à force de g. ; chaque nuit ma
\item[\vref{Ps 32:3}] n'ai fait que g. tt le jour ;
\item[\vref{Ps 55:18}] plains et je g., et il entendra
\item[\vref{Ps 77:4}] Dieu, et je g. ; je médite, et
\item[\vref{Pr 5:11}] que tu ne g. qnd tu seras
\item[\vref{Es 59:11}] ne cessons de g. com. des colombes ;
\item[\vref{Ez 24:17}] Garde-toi de g., et ne fais
\item[\vref{2 Co 5:2}] cela que ns. g., désirant avec ardeur
\item[\vref{Ja 5:1}] riches ! Pleurez et g. à cause des
\end{listverse}

\ConcordanceEntry{Généalogie}
\vspace{-2mm}
\begin{listverse}
\item[\vref{1 Ch 4:33}] sont là leurs habitations et lr. g. :
\item[\vref{1 Ch 5:1}] enregistré ds la g. selon le droit
\item[\vref{1 Ch 5:17}] inscrits ds la g. du temps de
\item[\vref{1 Ch 7:5}] ts selon lr. g., furent quatre-vingt-sept mille.
\item[\vref{1 Ch 7:9}] dénombrement selon lr. g., selon leurs générations,
\item[\vref{1 Ch 9:1}] furent enregistrés par g. et inscrits ds
\item[\vref{1 Ch 9:22}] familles ds la g., selon leurs villages ;
\item[\vref{Esd 8:3}] dénombrement par lr. g. selon les hommes,
\item[\vref{Mt 1:1}] Livre de la g. de Jésus-Christ, fils
\item[\vref{1 Ti 1:4}] fables et aux g. sans fin, qui
\item[\vref{Hé 7:3}] sans mère, sans g., n'ayant ni commencement
\end{listverse}

\ConcordanceEntry{Génération}
\vspace{-2mm}
\begin{listverse}
\item[\vref{Ge 15:16}] à la quatrième g., ils reviendront ici ;
\item[\vref{Ex 20:5}] à la quatrième g. de ceux qui
\item[\vref{No 32:13}] que tte la g. qui avait fait
\item[\vref{Jg 2:10}] Toute cette g. fut recueillie auprès
\item[\vref{1 Ch 16:15}] de ses promesses établies pour mille g. ;
\item[\vref{Job 8:8}] prie, interroge les g. précédentes, et applique-toi
\item[\vref{Ps 22:31}] du Seign. de g. en génération.
\item[\vref{Ec 1:4}] Une g. passe et une autre génération vient,
\item[\vref{Mt 11:16}] qui comparerai-je cette g. ? Elle est semblable
\item[\vref{Mt 12:39}] et dit : Une g. méchante et adultère
\item[\vref{Mt 23:36}] ttes ces choses viendront sur cette g.
\item[\vref{Ac 2:40}] Sauvez-vs. de cette g. perverse.
\item[\vref{Ph 2:15}] milieu de la g. corrompue et perverse,
\end{listverse}

\ConcordanceEntry{Génézareth}
\vspace{-2mm}
\begin{listverse}
\item[\vref{Mt 14:34}] ils vinrent ds le pays de G.
\item[\vref{Lu 5:1}] sur le bord du lac de G.,
\end{listverse}

\ConcordanceEntry{Genou}
\vspace{-2mm}
\begin{listverse}
\item[\vref{1 R 18:42}] il mit son visage entre ses g. ;
\item[\vref{1 R 19:18}] point fléchi les g. dvt Baal, et
\item[\vref{Ps 95:6}] et mettons-ns. à g. dvt Yahweh qui
\item[\vref{Es 35:3}] et fortifiez les g. tremblants.
\item[\vref{Es 45:23}] point révoquée : Tout g. fléchira dvt moi,
\item[\vref{Es 46:1}] s'incline sur ses g., Nebo est renversé ;
\item[\vref{Ez 47:4}] de l'eau jusqu'aux g. ; puis il mesura
\item[\vref{Da 6:10}] se mettait à g., il priait, et
\item[\vref{Mt 17:14}] se mit à g. dvt lui,
\item[\vref{Mc 15:19}] et fléchissant les g., ils se prosternaient
\item[\vref{Lu 22:41}] s'étant mis à g., il pria,
\item[\vref{Ac 9:40}] se mit à g., et pria ; puis
\item[\vref{Ac 20:36}] se mit à g. et il pria
\item[\vref{Ro 11:4}] pas fléchi le g. dvt Baal.
\item[\vref{Ro 14:11}] le Seign., tt g. fléchira dvt moi,
\item[\vref{Ep 3:14}] je fléchis mes g. dvt le Père
\item[\vref{Ph 2:10}] de Jésus, tt g. fléchisse, tant de
\item[\vref{Hé 12:12}] languissantes et vos g. affaiblis ;
\end{listverse}

\ConcordanceEntry{Gentils}
\vspace{-2mm}
\begin{listverse}
\item[\vref{Es 8:23}] du Jourdain, ds la Galilée des G.
\item[\vref{Mc 10:33}] à mort, et le livreront aux G.
\item[\vref{Ac 11:1}] apprirent que les G. aussi avaient reçu
\item[\vref{Ac 18:6}] pur ! Dès mntnt, j'irai vers les G.
\item[\vref{Ro 2:24}] blasphémé parmi les G. à cause de
\item[\vref{Ro 11:25}] la totalité des G. soit entrée.
\item[\vref{Ro 15:9}] sorte que les G. glorifient Dieu pour
\item[\vref{Ga 3:8}] Dieu justifierait les G. par la foi,
\item[\vref{Ep 2:11}] qui étiez autrefois G. ds la chair,
\end{listverse}

\ConcordanceEntry{Geôlier}
\vspace{-2mm}
\begin{listverse}
\item[\vref{Ac 16:23}] en recommandant au g. de les garder
\item[\vref{Ac 16:27}] Le g. se réveilla, et, voyant les portes
\item[\vref{Ac 16:29}] Alors le g., ayant demandé de la lumière, entra
\item[\vref{Ac 16:35}] pour dire au g. : Relâche ces hommes.
\end{listverse}

\ConcordanceEntry{Gerbe}
\vspace{-2mm}
\begin{listverse}
\item[\vref{Ge 37:7}] à lier des g. au milieu d'un
\item[\vref{Lé 23:10}] au prêtre une g. des premiers fruits
\item[\vref{Lé 23:11}] il agitera cette g.-là dvt Yahweh,
\item[\vref{De 24:19}] auras oublié une g. ds ton champ,
\item[\vref{Job 5:26}] un monceau de g. s'entasse en sa
\item[\vref{Ps 126:6}] chants d'allégresse qnd il porte ses g.
\item[\vref{Jé 9:22}] et com. une g. après le moissonneur,
\item[\vref{Mi 4:12}] assemblées com. des g. ds l'aire.
\item[\vref{Mt 13:30}] et liez-la en g. pour la brûler,
\end{listverse}

\ConcordanceEntry{Germe}
\vspace{-2mm}
\begin{listverse}
\item[\vref{Job 5:6}] le travail ne g. pas de la
\item[\vref{Ps 65:11}] la pluie, et tu bénis son g.
\item[\vref{Ps 85:12}] La vérité g. de la terre,
\item[\vref{Es 4:2}] ce temps-là, le g. de Yahweh sera
\item[\vref{Jé 23:5}] à David un G. juste, qui régnera
\item[\vref{Jé 33:15}] à David le G. de justice, qui
\item[\vref{Za 3:8}] je ferai venir mon serviteur, le G.
\item[\vref{Za 6:12}] le nom est G., qui germera de
\item[\vref{Mc 4:27}] jour, la semence g. et croît, sans
\end{listverse}

\ConcordanceEntry{Gestionnaires}
\vspace{-2mm}
\begin{listverse}
\item[\vref{1 Co 4:1}] Christ et des g. des mystères de
\item[\vref{1 Co 4:2}] est exigé des g. que chacun soit
\item[\vref{1 Pi 4:10}] com. de bons g. des diverses grâces
\end{listverse}

\ConcordanceEntry{Gethsémané}
\vspace{-2mm}
\begin{listverse}
\item[\vref{Mt 26:36}] un lieu appelé G., et il dit
\item[\vref{Mc 14:32}] un lieu appelé G., et Jésus dit
\end{listverse}

\ConcordanceEntry{Glaive}
\vspace{-2mm}
\begin{listverse}
\item[\vref{Ge 49:5}] sont frères, leurs g. sont des instruments
\end{listverse}

\ConcordanceEntry{Gloire}
\vspace{-2mm}
\begin{listverse}
\item[\vref{Ge 45:13}] quelle est ma g. en Egypte, et
\item[\vref{Ex 16:7}] vs. verrez la g. de Yahweh, parce
\item[\vref{Ex 33:18}] Je te prie, fais-moi voir ta g. !
\item[\vref{Ex 40:34}] d'assignation, et la g. de Yahweh remplit
\item[\vref{Lé 9:23}] peuple. Et la g. de Yahweh apparut
\item[\vref{No 14:21}] vivant, et la g. de Yahweh remplira
\item[\vref{De 5:24}] fait voir sa g. et sa grandeur,
\item[\vref{Jos 7:19}] te prie donne g. à Yahweh, le
\item[\vref{Jg 13:17}] ns. te rendions g. lorsque ta parole
\item[\vref{1 S 4:21}] en disant : La g. s'en est allée
\item[\vref{1 R 8:11}] nuée ; car la g. de Yahweh remplissait
\item[\vref{1 Ch 16:24}] Racontez sa g. parmi les nations,
\item[\vref{1 Ch 16:28}] donnez à Yahweh g. et force !
\item[\vref{Ps 8:6}] l'as couronné de g. et d'honneur.
\item[\vref{Ps 19:2}] cieux racontent la g. de Dieu, et
\item[\vref{Ps 26:8}] lequel est le tabernacle de ta g.
\item[\vref{Ps 29:9}] forêts. Dans son palais tt s'écrie : G. !
\item[\vref{Ps 66:2}] chantez la g. de son Nom,
\item[\vref{Ps 106:20}] Ils changèrent lr. g. contre la figure
\item[\vref{Ps 115:1}] ton Nom donne g., à cause de
\item[\vref{Pr 3:35}] sages hériteront la g. ; mais la honte
\item[\vref{Pr 25:27}] a-t-il pas de g. pour ceux qui
\item[\vref{Pr 26:1}] moisson, ainsi la g. ne convient pas
\item[\vref{Es 6:3}] la terre est pleine de sa g. !
\item[\vref{Es 40:5}] Alors la g. de Yahweh sera manifestée, et tte
\item[\vref{Es 42:8}] donnerai pas ma g. à un autre,
\item[\vref{Ez 1:28}] représentation de la g. de Yahweh. A
\item[\vref{Ez 11:23}] La g. de Yahweh s'éleva du milieu de
\item[\vref{Ez 43:2}] Et voici, la g. du Dieu d'Israël
\item[\vref{Os 4:7}] Je changerai lr. g. en ignominie.
\item[\vref{Ha 2:14}] connaissance de la g. de Yahweh, com.
\item[\vref{Ag 2:9}] La g. de cette dernière maison sera plus
\item[\vref{Mt 6:29}] ds tte sa g., n'a pas été
\item[\vref{Mt 16:27}] venir ds la g. de son Père,
\item[\vref{Mt 25:31}] environné de sa g. et accompagné de
\item[\vref{Lu 2:9}] eux, et la g. du Seign. resplendit
\item[\vref{Lu 2:14}] G. soit à Dieu ds les lieux
\item[\vref{Lu 9:31}] apparurent environnés de g., et ils parlaient
\item[\vref{Lu 9:32}] ils virent sa g., et les deux
\item[\vref{Jn 1:14}] avons contemplé sa g., une gloire com.
\item[\vref{Jn 5:41}] tire pas ma g. des hommes.
\item[\vref{Jn 5:44}] vs. recevez la g. les uns des
\item[\vref{Jn 8:50}] cherche pas ma g. ; il y en
\item[\vref{Jn 11:40}] tu verras la g. de Dieu ?
\item[\vref{Jn 12:41}] il vit sa g., et qu'il parla
\item[\vref{Jn 12:43}] ils aimèrent la g. des hommes, plus
\item[\vref{Jn 17:5}] toi, de la g. que j'avais auprès
\item[\vref{Jn 17:22}] ai donné la g. que tu m'as
\item[\vref{Jn 17:24}] qu'ils contemplent la g. que tu m'as
\item[\vref{Ro 1:23}] ont changé la g. du Dieu incorruptible
\item[\vref{Ro 2:10}] Mais g., honneur et paix à tt hom.
\item[\vref{Ro 3:23}] atteignent pas la g. de Dieu ;
\item[\vref{Ro 5:2}] ns. glorifions ds l'espérance de la g. de Dieu.
\item[\vref{Ro 8:18}] comparables à la g. à venir qui
\item[\vref{Ro 11:36}] lui soit la g. éternellement ! Amen !
\item[\vref{1 Co 2:8}] n'auraient pas crucifié le Seign. de g.
\item[\vref{1 Co 10:31}] autre chose, faites tt à la g. de Dieu.
\item[\vref{1 Co 11:7}] l'image et la g. de Dieu, mais
\item[\vref{2 Co 3:18}] un miroir la g. du Seign. à
\item[\vref{2 Co 4:6}] connaissance de la g. de Dieu qui
\item[\vref{2 Co 4:17}] poids éternel d'une g. souverainement excellente,
\item[\vref{Ga 5:26}] pas une vaine g. en ns. provoquant
\item[\vref{Ep 1:14}] acquis à la louange de sa g.
\item[\vref{Ep 3:21}] lui soit la g. ds l'Eglise, en
\item[\vref{Col 1:11}] puissance de sa g., pour tte patience
\item[\vref{1 Ti 3:16}] le monde, et élevé ds la g.
\item[\vref{Tit 2:13}] l'apparition de la g. du grand Dieu
\item[\vref{Hé 1:3}] splendeur de sa g., et l'empreinte de
\item[\vref{Hé 2:10}] enfants à la g., consacre le Prince
\item[\vref{Hé 3:3}] jugé digne d'une g. d'autant supérieure à
\item[\vref{1 Pi 4:14}] car l'Esprit de g., l'Esprit de Dieu
\item[\vref{1 Pi 5:10}] appelés à sa g. éternelle en Jésus-Christ,
\item[\vref{2 Pi 1:17}] Père honneur et g., lorsque cette voix
\item[\vref{Ap 4:11}] digne de recevoir g., honneur et puissance ;
\item[\vref{Ap 15:8}] cause de la g. de Dieu et
\item[\vref{Ap 19:1}] Le salut, la g., l'honneur et la
\end{listverse}

\ConcordanceEntry{Glorieux}
\vspace{-2mm}
\begin{listverse}
\item[\vref{1 Ch 29:13}] célébrons et ns. louons ton Nom g.
\item[\vref{Esd 4:10}] le grand et g. Osnappar a transportés
\item[\vref{Né 9:5}] bénisse ton Nom g., qui est au-dessus
\item[\vref{Ps 72:19}] éternellement son Nom g. ! Et que tte
\item[\vref{Es 33:21}] Yahweh ns. est g. ; c'est le lieu
\item[\vref{Es 60:13}] et je rendrai g. le lieu de
\item[\vref{Es 63:12}] par son bras g. ; qui fendit les
\item[\vref{Es 63:14}] peuple, afin de t'acquérir un nom g.
\item[\vref{Es 64:10}] maison sainte et g., où nos pères
\item[\vref{1 Co 15:43}] déshonneur, il ressuscite g. ; il est semé
\item[\vref{2 Co 3:7}] des pierres, était g. au point que
\item[\vref{2 Co 3:11}] fin a été g., ce qui est
\item[\vref{Ep 5:27}] lui cette Eglise g., sans tache, ni
\item[\vref{Ph 3:21}] à son corps g. selon le pouvoir
\end{listverse}

\ConcordanceEntry{Glorifier}
\vspace{-2mm}
\begin{listverse}
\item[\vref{2 Ch 25:19}] élevé pour te g.. Maintenant, reste ds
\item[\vref{Es 60:9}] et du Saint d'Israël qui te g.
\item[\vref{Da 5:23}] tu n'as pas g. le Dieu ds
\item[\vref{Mt 9:8}] d'étonnement, et elle g. Dieu qui a
\item[\vref{Lu 4:15}] et il était g. par ts.
\item[\vref{Lu 5:25}] et s'en alla ds sa maison, g. Dieu.
\item[\vref{Lu 7:16}] crainte, et ils g. Dieu, disant : Certainement
\item[\vref{Jn 8:54}] Si je me g. moi-mm, ma gloire
\item[\vref{Jn 11:4}] de Dieu soit g. par elle.
\item[\vref{Jn 12:16}] Jésus eut été g., ils se souvinrent
\item[\vref{Jn 12:28}] Père, g. ton Nom ! Alors une voix vint
\item[\vref{Jn 13:31}] de l'hom. est g. ; et Dieu est
\item[\vref{Jn 13:32}] Si Dieu est g. en lui, Dieu
\item[\vref{Jn 17:4}] Je t'ai g. sur la terre, j'ai achevé l'œuvre
\item[\vref{Ac 3:13}] nos pères, a g. son Fils Jésus,
\item[\vref{Ac 4:21}] parce que ts g. Dieu de ce
\item[\vref{Ac 10:46}] les entendaient parler diverses langues et g. Dieu.
\item[\vref{Ac 19:17}] le Nom du Seign. Jésus était g.
\item[\vref{Ro 1:21}] ne l'ont pas g. com. Dieu, et
\item[\vref{Ro 3:27}] sujet de se g. ? Il est exclu.
\item[\vref{Ro 5:2}] et ns. ns. g. ds l'espérance de
\item[\vref{Ro 8:17}] ns. soyons aussi g. avec lui.
\item[\vref{Ro 8:30}] a justifiés, il les a aussi g.
\item[\vref{1 Co 1:29}] chair ne se g. dvt lui.
\item[\vref{1 Co 1:31}] celui qui se g., se glorifie ds
\item[\vref{1 Co 4:7}] reçu, pourquoi te g.-tu com. si
\item[\vref{1 Co 6:20}] un grand prix. G. dc Dieu ds
\item[\vref{2 Co 5:12}] l'occasion de vs. g. de ns., afin
\item[\vref{2 Co 8:24}] sujet que ns. avons de ns. g. de vs.
\item[\vref{2 Co 10:8}] je veux me g. davantage de l'autorité
\item[\vref{2 Co 11:17}] j'aurais de me g., je ne le
\item[\vref{2 Co 12:6}] je voudrais me g., je ne serais
\item[\vref{Ga 1:24}] Et elles g. Dieu à cause de moi.
\item[\vref{Ga 6:4}] de quoi se g. pour lui-mm seulement,
\item[\vref{Ph 1:20}] l'a toujours été g. ds mon corps,
\item[\vref{Ph 1:26}] sujet de vs. g. de plus en
\item[\vref{2 Th 1:10}] viendra pour être g. en ce jour-là
\item[\vref{1 Pi 4:11}] choses Dieu soit g. par Jésus-Christ, auquel
\end{listverse}

\ConcordanceEntry{Gog}
\vspace{-2mm}
\begin{listverse}
\item[\vref{1 Ch 5:4}] Schemaeja, son fils ; G., son fils ; Schimeï,
\item[\vref{Ez 38:2}] ta face vers G. au pays de
\item[\vref{Ez 38:3}] veux à toi, G., prince des chefs
\item[\vref{Ez 39:1}] l'hom., prophétise contre G., et dis : Ainsi
\item[\vref{Ez 39:11}] je donnerai à G. ds ces quartiers-là
\item[\vref{Ap 20:8}] de la terre, G. et Magog, afin
\end{listverse}

\ConcordanceEntry{Golgotha}
\vspace{-2mm}
\begin{listverse}
\item[\vref{Mt 27:33}] au lieu appelé G., c'est-à-dire le lieu
\item[\vref{Mc 15:22}] au lieu appelé G., c'est-à-dire, le lieu
\item[\vref{Jn 19:17}] Crâne, qui se dit en hébreu G.,
\end{listverse}

\ConcordanceEntry{Goliath}
\vspace{-2mm}
\begin{listverse}
\item[\vref{1 S 17:4}] armées, il s'appelait G., de la ville
\item[\vref{1 S 17:23}] de Gath, nommé G., sortit des rangs
\item[\vref{1 S 21:9}] Voici l'épée de G., le Philistin, que
\item[\vref{1 S 22:10}] que l'épée de G., le Philistin.
\item[\vref{2 S 21:19}] de Bethléhem, tua G. de Gath, qui
\item[\vref{1 Ch 20:5}] Lachmi, frère de G. de Gath, qui
\end{listverse}

\ConcordanceEntry{Gomorrhe}
\vspace{-2mm}
\begin{listverse}
\item[\vref{Ge 10:19}] vers Sodome et G., Adma et Tseboïm,
\item[\vref{Ge 13:10}] détruit Sodome et G., c'était, jusqu'à Tsoar,
\item[\vref{Ge 19:24}] Sodome et sur G., du soufre et
\item[\vref{Es 1:9}] serions com. Sodome, ns. ressemblerions à G.
\item[\vref{Es 13:19}] com. Sodome et G. que Dieu détruisit.
\item[\vref{Jé 23:14}] les habitants de la ville com. G.
\item[\vref{Mt 10:15}] Sodome et de G. seront traités moins
\item[\vref{Mc 6:11}] Sodome et de G. seront traités moins
\item[\vref{Ro 9:29}] et ns. aurions été semblables à G.
\item[\vref{2 Pi 2:6}] Sodome et de G., les réduisant en
\end{listverse}

\ConcordanceEntry{Gosen}
\vspace{-2mm}
\begin{listverse}
\item[\vref{Ge 45:10}] la contrée de G., et tu seras
\item[\vref{Ex 8:18}] le pays de G., où se tient
\item[\vref{Ex 9:26}] la contrée de G., ds laquelle étaient
\item[\vref{Jos 10:41}] le pays de G. jusqu'à Gabaon.
\item[\vref{Jos 11:16}] le pays de G., la vallée et
\end{listverse}

\ConcordanceEntry{Goûter}
\vspace{-2mm}
\begin{listverse}
\item[\vref{Ps 34:9}] [Teth.] G. et voyez combien Yahweh est bon !
\item[\vref{Mt 27:34}] qnd il l'eut g., il ne voulut
\item[\vref{Lu 14:24}] été conviés ne g. de mon souper.
\item[\vref{Jn 2:9}] maître d'hôtel eut g. l'eau qui avait
\item[\vref{Col 2:21}] prends pas ! ne g. pas ! ne touche
\item[\vref{Hé 6:4}] et qui ont g. le don céleste,
\item[\vref{Hé 6:5}] qui ont g. la bonne parole de Dieu, et
\item[\vref{1 Pi 2:3}] toutefois vs. avez g. combien le Seign.
\end{listverse}

\ConcordanceEntry{Gouvernement}
\vspace{-2mm}
\begin{listverse}
\item[\vref{1 Ch 22:12}] t'instruise concernant le g. d'Israël, et comment
\item[\vref{Pr 28:2}] il y aura prolongation du mm g.
\item[\vref{Da 5:7}] troisième ds le g. du royaume.
\item[\vref{Ha 1:7}] et terrible, son g. et son autorité
\item[\vref{Lu 19:17}] choses, reçois le g. de dix villes.
\end{listverse}

\ConcordanceEntry{Gouverner}
\vspace{-2mm}
\begin{listverse}
\item[\vref{Ge 19:9}] il veut ns. g. ! Maintenant ns. te
\item[\vref{Mi 5:3}] se maintiendra et g. par la force
\item[\vref{Za 13:5}] m'a appris à g. du bétail dès
\item[\vref{Lu 22:26}] et celui qui g., com. celui qui
\item[\vref{1 Co 12:28}] de secourir, de g., de parler diverses
\item[\vref{1 Ti 3:5}] maison, comment pourra-t-il g. l'église de Dieu ?
\item[\vref{Ap 12:5}] fils, qui doit g. ttes les nations
\end{listverse}

\ConcordanceEntry{Gouverneur}
\vspace{-2mm}
\begin{listverse}
\item[\vref{Ge 45:8}] sa maison, et g. de tt le
\item[\vref{Jg 9:30}] Zebul, g. de la ville, entendit les paroles
\item[\vref{Esd 4:8}] Rehum dc, le g., et Schimschaï, le
\item[\vref{Esd 5:6}] Darius par Thathnaï, g. de ce côté
\item[\vref{Né 5:14}] commandé d'être lr. g. au pays de
\item[\vref{Da 2:48}] présents ; il l'établit g. sur tte la
\item[\vref{Ag 1:1}] fils de Schealthiel, g. de Juda, et
\item[\vref{Ag 2:21}] Parle à Zorobabel, g. de Juda, et
\item[\vref{Mt 27:27}] Les soldats du g. amenèrent Jésus ds
\item[\vref{Mt 28:14}] Et si le g. l'apprend, ns. l'apaiserons et ns. vs.
\item[\vref{Lu 2:2}] que Quirinius était g. de Syrie.
\item[\vref{Lu 3:1}] Ponce Pilate était g. de la Judée,
\item[\vref{Ac 24:1}] comparurent dvt le g. contre Paul.
\item[\vref{Ac 24:10}] après que le g. eut fait signe
\item[\vref{Ac 26:30}] leva, avec le g. et Bérénice, et
\item[\vref{2 Co 11:32}] A Damas, le g. du roi Arétas
\item[\vref{1 Pi 2:14}] soit aux g., com. à ceux qui sont envoyés
\end{listverse}

\ConcordanceEntry{Grâce}
\vspace{-2mm}
\begin{listverse}
\item[\vref{Ge 6:8}] Mais Noé trouva g. aux yeux de
\item[\vref{Ex 3:21}] ce peuple trouve g. envers les Egyptiens,
\item[\vref{Ex 33:17}] tu as trouvé g. dvt mes yeux,
\item[\vref{Ex 33:19}] et je ferai g. à qui je
\item[\vref{No 6:25}] face sur toi, et te fasse g. !
\item[\vref{2 S 7:15}] mais ma g. ne se retirera point de lui,
\item[\vref{Est 2:15}] femmes. Esther trouva g. aux yeux de
\item[\vref{Ps 30:6}] instant, mais sa g. tte la vie.
\item[\vref{Ps 45:3}] de l'hom., la g. est répandue sur
\item[\vref{Ps 84:12}] Yahweh donne la g. et la gloire,
\item[\vref{Pr 3:4}] tu trouveras la g. et la prudence
\item[\vref{Pr 22:1}] et la bonne g. plus que l'argent
\item[\vref{Pr 31:30}] [Shin.] La g. est trompeuse, et
\item[\vref{Es 26:10}] Est-il fait g. au méchant ? Il
\item[\vref{Es 30:18}] pour vs. faire g., et ainsi il
\item[\vref{Lu 1:28}] à qui une g. a été faite ;
\item[\vref{Lu 2:52}] stature, et en g., auprès de Dieu
\item[\vref{Jn 1:14}] ns., pleine de g. et de vérité ;
\item[\vref{Jn 1:16}] sa plénitude, et g. pour grâce ;
\item[\vref{Jn 1:17}] par Moïse, la g. et la vérité
\item[\vref{Ac 13:43}] les exhortèrent à persévérer ds la g. de Dieu.
\item[\vref{Ac 20:24}] rendre témoignage à l'Evangile de la g. de Dieu.
\item[\vref{Ro 5:20}] a abondé, la g. a surabondé,
\item[\vref{Ro 5:21}] mort, ainsi la g. règne par la
\item[\vref{Ro 6:14}] sous la loi, mais sous la g.
\item[\vref{Ro 11:6}] c'est par la g., ce n'est plus
\item[\vref{1 Co 15:10}] Mais par la g. de Dieu, je
\item[\vref{2 Co 6:1}] pas recevoir la g. de Dieu en
\item[\vref{2 Co 12:9}] m'a dit : Ma g. te suffit, car
\item[\vref{2 Co 13:13}] Que la g. du Seign. Jésus-Christ, la charité de
\item[\vref{Ga 5:4}] loi ; vs. êtes déchus de la g.
\item[\vref{Ep 1:7}] offenses, selon les richesses de sa g.,
\item[\vref{Ep 2:8}] sauvés par la g., par la foi ;
\item[\vref{Col 4:6}] de sel, avec g., afin que vs.
\item[\vref{1 Th 1:1}] Seign. : Que la g. et la paix
\item[\vref{Hé 12:15}] prive de la g. de Dieu ; à
\item[\vref{Hé 12:28}] ébranlé, retenons la g. par laquelle ns.
\item[\vref{Hé 13:9}] affermi par la g., et non pas
\item[\vref{Hé 13:25}] Que la g. soit avec vs. ts ! Amen !
\item[\vref{Ja 4:6}] une plus grande g. ; c'est pourquoi l'Ecriture
\item[\vref{1 Pi 1:2}] sang : Que la g. et la paix
\item[\vref{1 Pi 1:10}] prophétisé concernant la g. qui vs. était
\item[\vref{1 Pi 2:20}] patiemment, c'est une g. dvt Dieu.
\item[\vref{1 Pi 5:12}] attester que la g. de Dieu ds
\item[\vref{2 Pi 3:18}] croissez ds la g. et ds la
\item[\vref{2 Jn 1:3}] Que la g., la miséricorde et la paix de
\item[\vref{Jud 1:4}] qui changent la g. de notre Dieu
\end{listverse}

\ConcordanceEntry{Grand}
\vspace{-2mm}
\begin{listverse}
\item[\vref{Ge 12:2}] ferai devenir une g. nation, et je
\item[\vref{Ge 25:23}] et le plus g. sera asservi au
\item[\vref{Jos 7:9}] terre. Et que feras-tu à ton g. Nom ?
\item[\vref{2 S 5:10}] plus en plus g., et Yahweh, le
\item[\vref{1 R 10:23}] Salomon fut plus g. que ts les
\item[\vref{1 Ch 16:25}] Car Yahweh est g. et très digne
\item[\vref{2 Ch 2:5}] vais bâtir sera g. ; car notre Dieu
\item[\vref{Job 33:12}] sera toujours plus g. que l'hom.
\item[\vref{Job 37:23}] comprendre le Tout-Puissant, g. en puissance, en
\item[\vref{Ps 95:3}] Yahweh est un g. Dieu, et il
\item[\vref{Jé 10:6}] Yahweh ! Tu es g., et ton Nom
\item[\vref{Ez 36:23}] Je sanctifierai mon g. Nom, qui a
\item[\vref{Mal 1:11}] mon Nom est g. parmi les nations,
\item[\vref{Mal 1:14}] je suis un g. Roi, dit Yahweh
\item[\vref{Mt 11:11}] paru de plus g. que Jean-Baptiste. Toutefois,
\item[\vref{Mt 12:6}] quelqu'un de plus g. que le temple.
\item[\vref{Mt 18:1}] est le plus g. ds le Royaume
\item[\vref{Mt 23:11}] est le plus g. entre vs. soit
\item[\vref{Mc 9:34}] sur celui qui serait le plus g.
\item[\vref{Mc 10:43}] être le plus g. parmi vs., qu'il
\item[\vref{Lu 1:32}] Il sera g., et sera appelé le Fils du
\item[\vref{Jn 4:12}] Es-tu plus g. que Jacob, notre
\item[\vref{Jn 10:29}] données, est plus g. que ts ; personne
\item[\vref{Jn 14:28}] Père, car le Père est plus g. que moi.
\item[\vref{1 Jn 3:20}] Dieu est plus g. que notre cœur,
\item[\vref{1 Jn 4:4}] vs. est plus g. que celui qui
\item[\vref{Ap 12:3}] et voici un g. dragon rouge feu
\item[\vref{Ap 20:11}] je vis un g. trône blanc, et
\end{listverse}

\ConcordanceEntry{Grandeur}
\vspace{-2mm}
\begin{listverse}
\item[\vref{Ge 19:19}] as montré la g. de ta bonté
\item[\vref{Ex 15:7}] ruiné par la g. de ta majesté
\item[\vref{No 14:19}] peuple, selon la g. de ta miséricorde,
\item[\vref{De 32:3}] Yahweh ; attribuez la g. à notre Dieu.
\item[\vref{Ps 2:5}] terreur par la g. de son courroux:
\item[\vref{Ps 66:3}] cause de la g. de ta force.
\item[\vref{Ps 71:21}] Relève ma g. et console-moi encore !
\item[\vref{Ps 145:3}] n'est pas possible de sonder sa g.
\item[\vref{Ps 145:6}] puissance redoutable, et je raconterai ta g.
\item[\vref{La 3:32}] compassion selon la g. de sa miséricorde.
\end{listverse}

\ConcordanceEntry{Grand-prêtre}
\vspace{-2mm}
\begin{listverse}
\item[\vref{Lé 21:10}] Le g. d'entre ses frères, sur la tête
\item[\vref{2 R 22:8}] Alors Hilkija, le g., dit à Schaphan,
\item[\vref{2 R 23:4}] à Hilkija, le g., aux prêtres du
\item[\vref{2 Ch 19:11}] voici, Amaria, le g., sera au-dessus de
\item[\vref{Né 3:1}] Eliaschib, le g., se leva dc
\item[\vref{Né 3:20}] porte de la maison d'Eliaschib, le g.
\item[\vref{Né 13:28}] Jojada, fils d'Eliaschib, g., était gendre de
\item[\vref{Ag 2:2}] de Jotsadak, le g., et au reste
\item[\vref{Za 3:1}] voir Josué, le g., se tenant debout
\item[\vref{Za 3:8}] Ecoute mntnt Josué, g., toi et tes
\item[\vref{Mt 26:57}] chez Caïphe, le g., où les scribes
\item[\vref{Mc 2:26}] au temps du g. Abiathar, et mangea
\item[\vref{Mc 14:63}] Alors le g. déchira ses vêtements, et dit : Qu'avons-ns.
\item[\vref{Jn 18:19}] Et le g. interrogea Jésus sur ses disciples et
\item[\vref{Jn 18:24}] Anne l'envoya lié à Caïphe, le g.
\item[\vref{Ac 4:6}] avec Anne, le g., Caïphe, Jean, Alexandre,
\item[\vref{Ac 23:2}] Le g. Ananias ordonna à ceux qui étaient
\item[\vref{Hé 3:1}] l'Apôtre et le G. que ns. confessons,
\item[\vref{Hé 4:14}] avons un Souverain G., Jésus, le Fils
\item[\vref{Hé 6:20}] ayant été fait G. éternellement, selon l'ordre
\end{listverse}

\ConcordanceEntry{Gratuitement}
\vspace{-2mm}
\begin{listverse}
\item[\vref{Mt 10:8}] Vous l'avez reçu g., donnez-le gratuitement.
\item[\vref{Ro 3:24}] et ils sont g. justifiés par sa
\item[\vref{Ph 1:29}] vs. a été g. donné en ce
\item[\vref{Col 2:13}] lui, vs. ayant g. pardonné ttes vos
\item[\vref{2 Th 3:8}] ns. n'avons mangé g. le pain de
\item[\vref{Ap 21:6}] donnerai de la source d'eau vive g.
\item[\vref{Ap 22:17}] qui veut, prenne g. de l'eau de
\end{listverse}

\ConcordanceEntry{Graver}
\vspace{-2mm}
\begin{listverse}
\item[\vref{Ex 28:9}] d'onyx, et tu g. sur elles les
\item[\vref{Ex 28:36}] sur laquelle tu g. ces mots, com.
\item[\vref{Ex 32:16}] l'écriture de Dieu, g. sur les tables.
\item[\vref{Job 19:23}] Dieu qu'ils soient g. ds un livre,
\item[\vref{Es 49:16}] Voici, je t'ai g. sur les paumes
\item[\vref{Jé 17:1}] diamant ; il est g. sur la table
\item[\vref{Da 5:25}] qui a été g. : Compté, compté, pesé
\item[\vref{Ha 2:2}] la vision, et g.-la sur des
\item[\vref{Za 3:9}] yeux. Voici, je g. moi-mm ce qui
\item[\vref{2 Co 3:3}] lettre de Christ, g. par notre service,
\item[\vref{2 Co 3:7}] des lettres et g. sur des pierres,
\end{listverse}

\ConcordanceEntry{Grec}
\vspace{-2mm}
\begin{listverse}
\item[\vref{Mc 7:26}] cette fem. était g., syro-phénicienne d'origine. Elle
\item[\vref{Jn 12:20}] y avait quelques G. d'entre ceux qui
\item[\vref{Jn 19:20}] en hébreu, en g. et en latin.
\item[\vref{Ac 9:36}] qui signifie en g. Dorcas ; elle faisait
\item[\vref{Ac 14:1}] grande multitude de Juifs et de G. crut.
\item[\vref{Ac 16:1}] fem. juive fidèle et d'un père g.
\item[\vref{Ro 1:14}] débiteur tant aux G. qu'aux Barbares, tant
\item[\vref{Ro 1:16}] du Juif premièrement, puis aussi du G.,
\item[\vref{Ro 2:9}] du Juif premièrement, puis aussi du G.
\item[\vref{Ro 10:12}] Juif et le G., puisqu'ils ont un
\item[\vref{1 Co 1:22}] miracles et les G. cherchent la sagesse,
\item[\vref{Ga 2:3}] de se faire circoncire quoiqu'il fût G.
\item[\vref{Ga 3:28}] ni Juif ni G., il n'y a
\item[\vref{Col 3:11}] n'y a ni G. ni Juif, ni
\item[\vref{Ap 9:11}] Abaddon, mais en g. son nom est
\end{listverse}

\ConcordanceEntry{Grèce}
\vspace{-2mm}
\begin{listverse}
\item[\vref{Ac 20:3}] se rendit en G. où il séjourna
\end{listverse}

\ConcordanceEntry{Grêle}
\vspace{-2mm}
\begin{listverse}
\item[\vref{Ex 9:18}] mm heure, une g. tellement forte qu'il
\item[\vref{Ex 10:15}] arbres que la g. avait laissé ; il
\item[\vref{Lé 21:20}] sera bossu ou g., qui aura une
\item[\vref{Jos 10:11}] des pierres de g. furent plus nombreux
\item[\vref{Job 38:22}] neige ? As-tu vu les trésors de g.,
\item[\vref{Ps 18:14}] avec de la g. et des charbons
\item[\vref{Ps 148:8}] Feu et g., neige et brouillard, vent impétueux qui
\item[\vref{Ap 8:7}] eut de la g. et du feu
\item[\vref{Ap 11:19}] tremblement de terre, et une grosse g.
\item[\vref{Ap 16:21}] Une grosse g., dont les grêlons
\end{listverse}

\ConcordanceEntry{Grenier}
\vspace{-2mm}
\begin{listverse}
\item[\vref{Pr 3:10}] Alors tes g. seront remplis d'abondance,
\item[\vref{Joë 1:17}] sont dévastés, les g. sont renversés parce
\item[\vref{Mt 3:12}] froment ds le g. mais il brûlera
\item[\vref{Mt 6:26}] n'assemblent ds des g., et cependant votre
\item[\vref{Mt 13:30}] mais amassez le blé ds mon g.
\item[\vref{Lu 3:17}] froment ds son g., mais il brûlera
\item[\vref{Lu 12:18}] ferai : J'abattrai mes g., j'en bâtirai de
\item[\vref{Lu 12:24}] cellier, ni de g., et cependant Dieu
\end{listverse}

\ConcordanceEntry{Guéhazi}
\vspace{-2mm}
\begin{listverse}
\item[\vref{2 R 4:12}] il dit à G., son serviteur : Appelle
\end{listverse}
\begin{legend}
\NoAutoSpaceBeforeFDP{
\item Serviteur d'Elisée : 2 R 4:12
\item Convoita les biens de Naaman : 2 R 5:19
\item Fut frappé de la lèpre: 2 R 5:27
}
\end{legend}

\ConcordanceEntry{Guérar}
\vspace{-2mm}
\begin{listverse}
\item[\vref{Ge 20:1}] et il habita com. étranger à G.
\item[\vref{Ge 26:1}] vers Abimélec, roi des Philistins, à G.
\item[\vref{Ge 26:6}] Isaac dc demeura à G.
\item[\vref{Ge 26:17}] la vallée de G., où il s'établit.
\end{listverse}

\ConcordanceEntry{Guérir}
\vspace{-2mm}
\begin{listverse}
\item[\vref{Ex 15:26}] car je suis Yahweh qui te g.
\item[\vref{No 12:13}] je te prie, g.-la, je t'en
\item[\vref{2 R 20:5}] Voici je te g. ; ds trois jours
\item[\vref{Job 5:18}] bande ; il blesse et ses mains g.
\item[\vref{Ps 6:3}] sans aucune force. G.-moi, ô Yahweh !
\item[\vref{Ps 41:5}] pitié de moi ! G. mon âme ! car
\item[\vref{Ps 103:3}] tes iniquités, qui g. ttes tes infirmités ;
\item[\vref{Ps 107:20}] parole, et les g. ; et il les
\item[\vref{Ps 147:3}] il g. ceux qui ont le cœur brisé,
\item[\vref{Ec 3:3}] un temps pour g. ; un temps pour
\item[\vref{Es 19:22}] mais il les g. ; et ils retourneront
\item[\vref{Es 57:18}] toutefois je le g. ; je le conduirai
\item[\vref{Es 61:1}] m'a envoyé pour g. ceux qui ont
\item[\vref{Jé 17:14}] Yahweh, g.-moi et je serai guéri ; sauve-moi
\item[\vref{La 2:13}] com. une mer : Qui pourrait te g. ?
\item[\vref{Ez 30:21}] bandé pour le g., on ne lui
\item[\vref{Os 5:13}] pourra ni vs. g., ni panser vos
\item[\vref{Os 6:1}] mais il ns. g. ; il a frappé,
\item[\vref{Os 7:1}] Lorsque je voulais g. Israël, l'iniquité d'Ephraïm
\item[\vref{Os 11:3}] n'ont pas vu que je les g.
\item[\vref{Mt 4:23}] du Royaume, et g. ttes sortes de
\item[\vref{Mt 8:7}] lui dit : J'irai et je le g.
\item[\vref{Mt 10:1}] impurs et de g. ttes sortes de
\item[\vref{Mt 12:10}] Est-il permis de g. les jours du
\item[\vref{Mt 17:16}] mais ils n'ont pas pu le g.
\item[\vref{Mc 3:2}] voir s'il le g. le jour du
\item[\vref{Mc 3:15}] la puissance de g. les maladies, et
\item[\vref{Mc 5:29}] corps qu'elle était g. de son fléau.
\item[\vref{Lu 4:18}] m'a envoyé pour g. ceux qui ont
\item[\vref{Lu 7:3}] prier de venir g. son serviteur.
\item[\vref{Lu 9:1}] le pouvoir de g. les malades.
\item[\vref{Lu 10:9}] g. les malades qui s'y trouveront, et
\item[\vref{Lu 17:15}] d'eux se voyant g., revint sur ses
\item[\vref{Jn 4:47}] de descendre pour g. son fils qui
\item[\vref{Ac 3:11}] qui avait été g., tenait par la
\item[\vref{Ac 5:16}] des esprits impurs ; et ts étaient g.
\item[\vref{Ac 10:38}] du bien et g. ts ceux qui
\item[\vref{1 Co 12:28}] les dons de g., de secourir, de
\end{listverse}

\ConcordanceEntry{Guérison}
\vspace{-2mm}
\begin{listverse}
\item[\vref{Pr 13:17}] mal, mais l'ambassadeur fidèle apporte la g.
\item[\vref{Es 53:5}] ses meurtrissures que ns. avons la g.
\item[\vref{Es 58:8}] l'aurore, et ta g. germera rapidement ; ta
\item[\vref{Jé 8:15}] le temps de g., et voici la
\item[\vref{Jé 8:22}] Pourquoi dc la g. de la fille
\item[\vref{Mal 4:2}] justice, et la g. sera sous ses
\item[\vref{Lu 5:17}] du Seign. se manifestait par des g.
\item[\vref{Lu 13:32}] de faire des g. aujourd'hui et demain,
\item[\vref{Lu 14:3}] de faire une g. le jour du
\item[\vref{Ac 4:30}] se fasse des g., des prodiges, et
\item[\vref{1 Co 12:9}] les dons de g. par ce mm
\item[\vref{1 Co 12:30}] les dons de g. ? Tous parlent-ils diverses
\item[\vref{Ap 22:2}] servaient à la g. des nations.
\end{listverse}

\ConcordanceEntry{Guerre}
\vspace{-2mm}
\begin{listverse}
\item[\vref{No 32:6}] iront-ils à la g., et vs., demeurerez-vs.
\item[\vref{2 S 3:1}] eut une longue g. entre la maison
\item[\vref{2 R 18:20}] faut pour la g. le conseil et
\item[\vref{Ps 55:22}] crème, mais la g. est ds son
\item[\vref{Ps 120:7}] j'en parle, ils sont pour la g.
\item[\vref{Ec 9:11}] légers, ni la g. aux héros, ni
\item[\vref{Es 2:4}] ils ne s'adonneront plus à la g.
\item[\vref{Da 9:26}] dureront jusqu'à la fin de la g.
\item[\vref{Mt 24:6}] entendrez parler de g. et de bruits
\item[\vref{Lu 14:31}] va faire la g. à un autre
\item[\vref{Ap 12:17}] alla faire la g. contre les autres
\item[\vref{Ap 13:7}] de faire la g. aux saints et
\end{listverse}

\ConcordanceEntry{Guerrier}
\vspace{-2mm}
\begin{listverse}
\item[\vref{Ex 15:3}] est un vaillant g., son Nom est
\item[\vref{1 S 16:18}] vaillant, c'est un g. qui parle bien,
\item[\vref{1 R 5:3}] à cause des g. qui l'ont encerclé,
\item[\vref{2 Ch 17:17}] Benjamin, Eliada, vaillant g., avec deux cent
\item[\vref{Es 9:4}] tte bataille de g. se fait ds
\end{listverse}

\ConcordanceEntry{Guerschom, Guerschon}
\vspace{-2mm}
\begin{listverse}
\item[\vref{Ge 46:11}] fils de Lévi : G., Kehath, et Merari.
\end{listverse}
\begin{legend}
\NoAutoSpaceBeforeFDP{
\item Fils de Lévi : Ge 46:11; 1 Ch 6:17
\item Familles des Guerschonites : No 3:21-26;  4:41
\item Fils de Moïse et Séphora : Ex 2:21-22; 18:3; 1 Ch 23:15
\item Descendance de G. Fils de Moïse : 1 Ch 23:16; 26:24
\item Fils de Phinées : Esd 8:2
}
\end{legend}

\ConcordanceEntry{Guibea}
\vspace{-2mm}
\begin{listverse}
\item[\vref{Jg 19:12}] d'enfants d'Israël mais ns. passerons par G.
\item[\vref{Jg 19:14}] furent près de G., qui appartient à
\item[\vref{Jg 20:5}] Les seigneurs de G. se sont élevés
\item[\vref{1 S 10:5}] tu arriveras à G.-Elohim, où se
\item[\vref{1 S 10:10}] ils arrivèrent à G., voici, une troupe
\item[\vref{1 S 10:26}] chez lui à G.. Il fut accompagné
\item[\vref{1 S 15:34}] Saül monta ds sa maison à G. de Saül.
\item[\vref{1 S 26:1}] de Saül à G., en disant : David
\item[\vref{2 S 21:6}] dvt Yahweh à G. de Saül, l'élu
\item[\vref{Os 9:9}] aux jours de G. ; Yahweh se souviendra
\end{listverse}

\ConcordanceEntry{Guihon}
\vspace{-2mm}
\begin{listverse}
\item[\vref{Ge 2:13}] second fleuve est G. ; c'est celui qui
\item[\vref{1 R 1:33}] ma mule et faites-le descendre à G.
\item[\vref{1 R 1:38}] roi David et le menèrent à G.
\item[\vref{1 R 1:45}] pour roi à G., d'où ils sont
\end{listverse}

\ConcordanceEntry{Guilgal}
\vspace{-2mm}
\begin{listverse}
\item[\vref{Jos 4:19}] il campa à G., à l'orient de
\item[\vref{Jos 9:6}] au camp de G., et lui dirent,
\item[\vref{Jos 10:43}] avec lui, retourna au camp à G.
\item[\vref{1 S 11:14}] Venez, allons à G., et ns. y
\item[\vref{1 S 13:4}] fut convoqué auprès de Saül, à G.
\item[\vref{1 S 13:8}] venait pas à G. et le peuple
\item[\vref{Os 9:15}] s'est manifestée à G. ; c'est là que
\item[\vref{Am 4:4}] vos péchés ds G., et amenez vos
\end{listverse}

\ConcordanceEntry{Guirgasiens}
\vspace{-2mm}
\begin{listverse}
\item[\vref{Ge 10:16}] et les Jébusiens, les Amoréens, les G.,
\item[\vref{Ge 15:21}] des Cananéens, des G., et des Jébusiens.
\item[\vref{De 7:1}] Les Héthiens, les G., les Amoréens, les
\item[\vref{Jos 24:11}] les Héthiens, les G., les Héviens et
\item[\vref{Né 9:8}] Jébusiens, et des G.. Et tu as
\end{listverse}

\ConcordanceEntry{Habakuk}
\vspace{-2mm}
\begin{listverse}
\item[\vref{Ha 1:1}] Prophétie qu'H. le prophète a
\end{listverse}
\begin{legend}
\NoAutoSpaceBeforeFDP{
\item Prophète de Yahweh établi en Juda : Ha 1:1
\item Ses cris et sa prière : Ha 1:2-4; 3:1-19
\item Sentinelle : Ha 2:1
\item Foi du juste : Ha 2:4; Ro 1:17; Ga 3:11
}
\end{legend}

\ConcordanceEntry{Habile}
\vspace{-2mm}
\begin{listverse}
\item[\vref{Ge 25:27}] Esaü devint un h. chasseur, et un
\item[\vref{2 S 14:2}] Tekoa une fem. h., et il lui
\item[\vref{2 Ch 2:7}] envoie-moi un hom. h. pour travailler l'or,
\item[\vref{2 Ch 2:13}] dc un hom. h. et intelligent, Huram-Abi,
\item[\vref{Ps 45:2}] la plume d'un h. écrivain !
\item[\vref{Pr 22:29}] vu un hom. h. en son travail ?
\item[\vref{Es 3:3}] les artisans et l'h. enchanteur.
\item[\vref{Es 40:20}] se cherche un h. ouvrier pour faire
\end{listverse}

\ConcordanceEntry{Habit}
\vspace{-2mm}
\begin{listverse}
\item[\vref{De 22:5}] ne portera point l'h. d'un hom. ni
\item[\vref{Mt 7:15}] à vs. en h. de brebis, mais
\item[\vref{Mt 11:8}] qui portent des h. précieux sont ds
\item[\vref{Ac 10:30}] hom. vêtu d'un h. resplendissant se présenta
\end{listverse}

\ConcordanceEntry{Habiter}
\vspace{-2mm}
\begin{listverse}
\item[\vref{Ge 19:9}] est venu pour h. ici com. étranger,
\item[\vref{Ex 29:45}] Et j'h. au milieu des enfants d'Israël, et
\item[\vref{Ex 29:46}] pays d'Egypte, pour h. au milieu d'eux.
\item[\vref{De 26:2}] ton Dieu, choisira pour y faire h. son Nom.
\item[\vref{1 Ch 17:4}] bâtiras point de maison pour y h.
\item[\vref{Esd 4:10}] transportés et fait h. ds la ville
\item[\vref{Né 1:9}] que j'aurai choisi pour y faire h. mon Nom.
\item[\vref{Ps 23:6}] ma vie, et j'h. ds la maison
\item[\vref{Ps 27:4}] désire ardemment : C'est d'h. ds la maison
\item[\vref{Ps 140:13}] les hommes droits h. dvt ta face.
\item[\vref{Pr 21:9}] Il vaut mieux h. au coin d'un
\item[\vref{Pr 21:19}] Il vaut mieux h. ds une terre
\item[\vref{Es 45:18}] pour qu'elle soit h. ; JE SUIS Yahweh,
\item[\vref{Es 57:15}] et élevé, qui h. ds l'éternité et
\item[\vref{Jé 2:6}] n'était passé, et où personne n'avait h. ?
\item[\vref{Jn 1:14}] chair, elle a h. parmi ns., pleine
\item[\vref{Ro 7:17}] cela, mais c'est le péché qui h. en moi.
\item[\vref{Ro 8:9}] l'Esprit de Dieu h. en vs.. Mais
\item[\vref{1 Co 7:12}] qu'elle consente à h. avec lui, qu'il
\item[\vref{1 Co 7:13}] qu'il consente à h. avec elle, qu'elle
\item[\vref{Ep 3:17}] sorte que Christ h. ds vos cœurs
\item[\vref{Col 1:19}] Père a été que tte plénitude h. en lui.
\end{listverse}

\ConcordanceEntry{Hadad}
\vspace{-2mm}
\begin{listverse}
\item[\vref{Ge 25:15}] H., Théma, Jethur, Naphisch, et Kedma.
\item[\vref{Ge 36:35}] Huscham mourut, et H., fils de Bédad,
\item[\vref{Ge 36:36}] H. mourut, et Samla, de Masréka, régna
\item[\vref{1 R 11:14}] à Salomon, savoir H., l'Edomite, qui était
\item[\vref{1 R 20:1}] Alors Ben-H., roi de Syrie
\item[\vref{2 R 13:3}] les mains de Ben-H., fils de Hazaël,
\end{listverse}

\ConcordanceEntry{Hadès}
\vspace{-2mm}
\begin{listverse}
\item[\vref{1 Co 15:55}] Ô H., où est ta victoire ? Ô mort,
\item[\vref{Ap 1:18}] les clefs de H. et de la
\item[\vref{Ap 6:8}] Mort, et le H. l'accompagnait. Il lr.
\end{listverse}

\ConcordanceEntry{Haine}
\vspace{-2mm}
\begin{listverse}
\item[\vref{Ge 27:41}] conçut de la h. contre Jacob, à
\item[\vref{Ge 49:23}] les archers l'ont poursuivi de lr. h.
\item[\vref{2 S 13:15}] une très grande h., en sorte que
\item[\vref{Ps 25:19}] me haïssent d'une h. pleine de violence.
\item[\vref{Ps 139:21}] n'aurais-je point en h. ceux qui te
\item[\vref{Pr 8:13}] Yahweh c'est la h. du mal. J'ai
\item[\vref{Pr 10:12}] La h. excite les querelles, mais la charité
\item[\vref{Pr 10:18}] qui couvre la h. a des lèvres
\item[\vref{Ec 9:1}] l'amour ni la h. de tt ce
\item[\vref{Ec 9:6}] lr. amour, lr. h., et lr. envie
\item[\vref{Jé 12:8}] c'est pourquoi je l'ai pris en h.
\end{listverse}

\ConcordanceEntry{Haïr}
\vspace{-2mm}
\begin{listverse}
\item[\vref{Ge 37:8}] Et ils le h. encore plus pour
\item[\vref{No 10:35}] ceux qui te h. s'enfuiront de dvt
\item[\vref{De 32:41}] et je rétribuerai ceux qui me h.
\item[\vref{2 Ch 19:2}] aimer ceux qui h. Yahweh ? A cause
\item[\vref{Ps 11:5}] et son âme h. celui qui aime
\item[\vref{Ps 97:10}] qui aimez Yahweh, h. le mal ! Il
\item[\vref{Pr 1:29}] Parce qu'ils auront h. la connaissance, et
\item[\vref{Pr 5:12}] dc ai-je pu h. la correction, et
\item[\vref{Pr 6:16}] choses que Dieu h., et il y
\item[\vref{Pr 13:24}] épargne sa verge h. son fils, mais
\item[\vref{Ec 2:17}] C'est pourquoi j'ai h. cette vie, car
\item[\vref{Ec 3:8}] un temps pour h. ; un temps pour
\item[\vref{Am 5:10}] Ils h. à la porte ceux qui les
\item[\vref{Mi 3:2}] Ils h. le bien, et aiment le mal ;
\item[\vref{Mal 1:3}] mais j'ai h. Esaü, et j'ai
\item[\vref{Mt 5:44}] ceux qui vs. h., et priez pour
\item[\vref{Lu 6:22}] les hommes vs. h., et vs. chasseront,
\item[\vref{Lu 19:14}] ses concitoyens le h., c'est pourquoi ils
\item[\vref{Jn 3:20}] fait le mal, h. la lumière, et
\item[\vref{Jn 7:7}] peut pas vs. h., mais il me
\item[\vref{Jn 15:23}] Celui qui me h., hait aussi mon
\item[\vref{Jn 15:25}] loi : Ils m'ont h. sans cause.
\item[\vref{Ro 7:15}] mais je fais ce que je h.
\item[\vref{Ep 5:29}] personne n'a jamais h. sa propre chair,
\item[\vref{Tit 3:3}] l'envie, dignes d'être h., et ns. haïssant
\item[\vref{1 Jn 2:9}] lumière et qui h. son frère, est
\item[\vref{Jud 1:23}] du feu, et h. mm la robe
\end{listverse}

\ConcordanceEntry{Haman}
\vspace{-2mm}
\begin{listverse}
\item[\vref{Est 3:1}] grands honneurs à H., fils d'Hammedatha, l'Agaguite ;
\item[\vref{Est 3:2}] se prosternaient dvt H., car le roi
\item[\vref{Est 3:6}] peuple était Mardochée. H. chercha à exterminer
\item[\vref{Est 4:7}] la somme d'argent qu'H. avait promis de
\item[\vref{Est 7:10}] Et ils pendirent H. au bois qu'il
\item[\vref{Est 8:3}] que la malice d'H., l'Agaguite, et ce
\item[\vref{Est 8:5}] concernaient la machination d'H. fils d'Hammedatha, l'Agaguite,
\item[\vref{Est 9:25}] la méchante machination qu'H. avait faite contre
\end{listverse}

\ConcordanceEntry{Hanania}
\vspace{-2mm}
\begin{listverse}
\item[\vref{1 Ch 3:19}] furent Meschullam et H. ; et Schelomith était
\item[\vref{1 Ch 3:21}] Les fils de H. furent Pelathia et
\item[\vref{1 Ch 25:4}] Uziel, Schebuel, Jerimoth, H., Hanani, Eliatha, Guiddalthi,
\item[\vref{Jé 28:1}] quatrième année, que H., fils d'Azzur, prophète
\item[\vref{Jé 28:17}] Et H., le prophète, mourut cette année-là, ds
\item[\vref{Jé 37:13}] Schélémia, fils de H., qui saisit Jérémie,
\item[\vref{Da 1:6}] de Juda, Daniel, H., Mischaël et Azaria.
\item[\vref{Da 2:17}] de cette affaire H., Mischaël et Azaria,
\end{listverse}

\ConcordanceEntry{Harmaguédon}
\vspace{-2mm}
\begin{listverse}
\item[\vref{Ap 16:16}] lieu qui est appelé en hébreu H.
\end{listverse}

\ConcordanceEntry{Harpe}
\vspace{-2mm}
\begin{listverse}
\item[\vref{Ge 4:21}] jouent de la h. et du chalumeau.
\item[\vref{1 S 10:5}] et de la h., et qui prophétisent.
\item[\vref{1 S 16:23}] David prenait la h., et en jouait
\item[\vref{Job 30:31}] C'est pourquoi ma h. s'est changée en
\item[\vref{Ps 33:2}] Yahweh avec la h., chantez-lui des psaumes
\item[\vref{Ps 71:22}] célèbrerai avec la h., Saint d'Israël !
\item[\vref{Ps 81:3}] le tambour, la h. mélodieuse et le
\item[\vref{Ps 137:2}] avions suspendu nos h. au milieu des
\item[\vref{Es 24:8}] pris fin, la joie de la h. a cessé.
\item[\vref{1 Co 14:7}] flûte ou une h., ne rendent pas
\item[\vref{Ap 14:2}] de joueurs de h. jouant de leurs
\item[\vref{Ap 18:22}] des joueurs de h., des musiciens, des
\end{listverse}

\ConcordanceEntry{Hâter}
\vspace{-2mm}
\begin{listverse}
\item[\vref{2 Ch 35:21}] dit de me h.. Désiste-toi dc de
\item[\vref{Ps 143:7}] Ô Yahweh ! h.-toi, réponds-moi ! Mon
\item[\vref{Es 5:19}] qui disent : Qu'il h. et qu'il fasse
\end{listverse}

\ConcordanceEntry{Hatsor}
\vspace{-2mm}
\begin{listverse}
\item[\vref{Jos 11:1}] Jabin, roi de H., eut appris ces
\item[\vref{Jos 11:10}] temps, Josué prit H., et frappa son
\item[\vref{Jos 15:25}] H.-Hadattha, Kerijoth-Hetsron qui est Hatsor,
\item[\vref{Jos 19:37}] Kédesch, Edréï, En-H.,
\item[\vref{1 S 12:9}] de l'armée de H., et entre les
\item[\vref{2 S 13:23}] les tondeurs à Baal-H., près d'Ephraïm, invita
\item[\vref{1 R 9:15}] muraille de Jérus., H., Meguiddo et Guézer.
\item[\vref{Jé 49:28}] les royaumes de H., que Nebucadnetsar, roi
\end{listverse}

\ConcordanceEntry{Haut}
\vspace{-2mm}
\begin{listverse}
\item[\vref{Ge 14:19}] par le Dieu Très-H., Maître du ciel
\item[\vref{1 R 10:19}] partie supérieure, le h. du trône était
\item[\vref{1 R 11:7}] Salomon bâtit un h. lieu à Kemosch,
\item[\vref{1 Ch 21:29}] temps-là ds le h. lieu de Gabaon.
\item[\vref{Job 31:28}] car j'aurais renié le Dieu d'en h.
\item[\vref{Ps 7:18}] psalmodierai le Nom de Yahweh, du Très-H.
\item[\vref{Ps 73:11}] saurait-il ? Comment le Très-H. connaîtrait-il ?
\item[\vref{Ps 83:19}] tu es le Très-H. sur tte la
\item[\vref{Ps 107:11}] qu'ils avaient rejeté le conseil du Très-H.
\item[\vref{Ps 138:6}] Car Yahweh est h. élevé, il voit
\item[\vref{Es 14:14}] au dessus des h. lieux des nuées,
\item[\vref{La 3:38}] procèdent-ils pas de la bouche du Très-H. ?
\item[\vref{Mc 5:7}] Fils du Dieu Très-H. ? Je te conjure
\item[\vref{Lu 4:9}] plaça sur le h. du temple, et
\item[\vref{Lu 6:35}] les fils du Très-H., car il est
\item[\vref{Jn 3:3}] ne naît d'en h., il ne peut
\item[\vref{Jn 3:7}] Il faut que vs. naissiez d'en h.
\item[\vref{Jn 3:31}] qui vient d'en h. est au-dessus de
\item[\vref{Jn 8:23}] je suis d'en h.. Vous êtes de
\item[\vref{Ac 7:48}] Mais le Très-H. n'habite pas ds
\item[\vref{Ac 16:17}] serviteurs du Dieu Très-H., et ils vs.
\item[\vref{Col 3:2}] aux choses d'en h., et non à
\end{listverse}

\ConcordanceEntry{Hautain}
\vspace{-2mm}
\begin{listverse}
\item[\vref{1 S 2:3}] tant de paroles h. ; qu'il ne sorte
\item[\vref{Ps 18:28}] affligé, et tu abaisses les yeux h.
\item[\vref{Ps 131:1}] ni un regard h. ; je ne m'occupe
\item[\vref{Pr 16:5}] abomination tt hom. h. de cœur ; assurément,
\item[\vref{Pr 21:4}] Les regards h. et le cœur
\item[\vref{Ec 7:8}] que l'hom. qui est d'un esprit h.
\item[\vref{Es 2:11}] Les yeux h. des hommes seront abaissés et les
\item[\vref{Es 2:12}] hom. orgueilleux et h., et contre tt
\item[\vref{Es 9:8}] avec orgueil et avec un cœur h. :
\item[\vref{Jé 48:29}] arrogance, sa fierté, et son cœur h.
\end{listverse}

\ConcordanceEntry{Hauteur}
\vspace{-2mm}
\begin{listverse}
\item[\vref{1 R 7:15}] dix-huit coudées de h. ; et un cordon
\item[\vref{Job 11:8}] Ce sont les h. des cieux : Qu'y
\item[\vref{Job 22:12}] Regarde dc la h. des étoiles ; et
\item[\vref{Ez 41:3}] deux coudées, la h. de cette ouverture,
\item[\vref{Ro 8:39}] ni la h., ni la profondeur, ni aucune autre
\item[\vref{2 Co 10:5}] raisonnements et tte h. qui s'élève contre
\item[\vref{Ep 3:18}] la longueur, la profondeur et la h.,
\item[\vref{Ap 21:16}] largeur et la h. étaient égales.
\end{listverse}

\ConcordanceEntry{Havila}
\vspace{-2mm}
\begin{listverse}
\item[\vref{Ge 2:11}] le pays de H. où se trouve
\item[\vref{Ge 10:7}] de Cusch : Saba, H., Sabta, Raema, et
\item[\vref{Ge 10:29}] Ophir, H., et Jobab. Tous ceux-là sont les
\item[\vref{Ge 25:18}] descendants habitèrent depuis H. jusqu'à Schur, qui
\item[\vref{1 S 15:7}] les Amalécites depuis H. jusqu'à Schur, qui
\item[\vref{1 Ch 1:9}] Cusch furent : Saba, H., Sabta, Raema et
\end{listverse}

\ConcordanceEntry{Hazaël}
\vspace{-2mm}
\begin{listverse}
\item[\vref{1 R 19:15}] arrivé, tu oindras H. pour roi de
\item[\vref{2 R 8:9}] Et H. s'en alla au-dvt d'Elisée, ayant pris
\item[\vref{2 R 8:14}] Alors H. quitta Elisée et revint vers son
\item[\vref{2 R 8:28}] la guerre contre H., roi de Syrie,
\item[\vref{2 R 10:32}] territoire d'Israël, et H. battit les Israélites
\item[\vref{2 R 13:3}] les mains de H., roi de Syrie,
\end{listverse}

\ConcordanceEntry{Héber}
\vspace{-2mm}
\begin{listverse}
\item[\vref{Ge 10:21}] ts les fils d'H., et frère aîné
\end{listverse}
\begin{legend}
\NoAutoSpaceBeforeFDP{
\item Fils de Sem, Ancêtre des Hébreux : Ge 10:21,24; 1 Ch 1:18
\item Sa Descendance : Ge 10:25; 1 Ch 1:19-23
\item Fils de Béria, fils d'Aser, fils de Jacob : Ge 46:17
\item La famille des Hébrites : No 26:45; 1 Ch 7:30-32
\item Le Kénien, mari de Jaël : Jg 4:11,17; 5:24
\item Père de Soco : 1 Ch 4:18
}
\end{legend}

\ConcordanceEntry{Hébreux}
\vspace{-2mm}
\begin{listverse}
\item[\vref{Ge 14:13}] vint avertir Abram, l'H., qui demeurait ds
\end{listverse}
\begin{legend}
\NoAutoSpaceBeforeFDP{
\item Descendants d'Héber : Ge 11:16-26
\item Yahweh, le Dieu des H. : Ex 3:18; 5:3
\item Loi sur les esclaves H. Ex 15:12
\item Traversée du Jourdain par les H. : 1 S 13:7
\item Murmure des Héllénistes contre les H. : Ac 6:1
\item Epître aux H. dont l'auteur est inconnu
\item Autres : 40:15, 43:32; Ex 9:1; De 15:121; 14:11; 2 Co 11:22; Ph 3:5
}
\end{legend}

\ConcordanceEntry{Hébron}
\vspace{-2mm}
\begin{listverse}
\item[\vref{Ge 23:2}] Kirjath-Arba, qui est H., ds le pays
\end{listverse}
\begin{legend}
\NoAutoSpaceBeforeFDP{
\item Ville Cananéenne (Kirjath-Arba) située en Juda : Ge 23:2; Jos 10:36-37; 14; 15
\item Conquête d’H. : Jos 10:3, 23-27, 39, Jg 1:10
\item Conquise par Caleb, héritage pour Juda : Jos 14:13-15; Jos 15:14; Jg 1:20
\item Donnée par Juda aux Lévites et ville pour refuge  : Jos 21.9-11,13
\item David est oint roi à H. et régna 7 ans et demi : 2 S  5:3-5; 1 Ch 11:3
\item Fils de Kehath : 1 Ch 6:2
\item Autres : Ge 13:18; No 13:22; 2 S 15:9-12; 2 Ch 11:10-11; Né 11:25
}
\end{legend}

\ConcordanceEntry{Hénoc}
\vspace{-2mm}
\begin{listverse}
\item[\vref{Ge 4:17}] conçut et enfanta H.. Il bâtit une
\item[\vref{Ge 5:24}] H. marcha avec Dieu ; mais il ne
\item[\vref{No 26:5}] Fils de Ruben : H., de qui descend
\item[\vref{Hé 11:5}] Par la foi, H. fut enlevé pour
\item[\vref{Jud 1:14}] aussi pour eux qu'H., le septième hom.
\end{listverse}

\ConcordanceEntry{Herbe}
\vspace{-2mm}
\begin{listverse}
\item[\vref{Ge 1:29}] vs. donne tte h. portant de la
\item[\vref{Ex 12:8}] pains sans levain, et avec des h. amères.
\item[\vref{De 29:23}] et que nulle h. n'en sortira, ainsi
\item[\vref{1 R 18:5}] ns. trouverons de l'h., ns. garderons ainsi
\item[\vref{2 R 19:26}] sont devenus com. l'h. des champs et
\item[\vref{Ps 37:2}] foin, et ils se faneront com. l'h. verte.
\item[\vref{Ps 72:16}] les villes com. l'h. de la terre.
\item[\vref{Ps 90:5}] songe qui, le matin, passe com. l'h. :
\item[\vref{Ps 103:15}] jours sont com. l'h., il fleurit com.
\item[\vref{Ps 105:35}] qui dévorèrent tte l'h. du pays, et
\item[\vref{Es 40:6}] chair est com. l'h., et tte sa
\item[\vref{Es 40:7}] L'h. sèche, et la fleur tombe, parce
\item[\vref{Za 10:1}] à chacun de l'h. ds son champ.
\item[\vref{Mt 6:30}] Dieu revêt ainsi l'h. des champs, qui
\item[\vref{Mc 6:39}] faire ts asseoir par groupes sur l'h. verte.
\item[\vref{Jn 6:10}] y avait beaucoup d'h. ds ce lieu.
\item[\vref{Hé 6:7}] qui produit des h. propres à ceux
\item[\vref{Ja 1:10}] il passera com. la fleur de l'h.
\item[\vref{Ja 1:11}] levé, et que l'h. a séché, que
\item[\vref{1 Pi 1:24}] chair est com. l'h., et tte la
\item[\vref{Ap 8:7}] brûlé, et tte h. verte aussi fut
\item[\vref{Ap 9:4}] de mal à l'h. de la terre,
\end{listverse}

\ConcordanceEntry{Héritage}
\vspace{-2mm}
\begin{listverse}
\item[\vref{Ge 48:6}] nom de leurs frères ds lr. h.
\item[\vref{Ex 15:17}] montagne de ton h., au lieu que
\item[\vref{Lé 25:46}] aurez com. un h. pour les laisser
\item[\vref{No 18:24}] qu'ils n'auront point d'h. parmi les fils
\item[\vref{No 26:53}] entre ceux-ci en h., selon le nombre
\item[\vref{No 32:32}] possédions pour notre h. ce qui est
\item[\vref{No 36:7}] Ainsi l'h. ne sera point transporté entre les
\item[\vref{De 9:29}] peuple et ton h., que tu as
\item[\vref{De 10:9}] ni portion ni d'h. avec ses frères :
\item[\vref{De 18:2}] Ils n'auront point d'h. parmi leurs frères :
\item[\vref{De 25:19}] te donne en h. afin que tu
\item[\vref{Jos 13:6}] ce pays en h. par le sort
\item[\vref{Jos 13:7}] ce pays en h. aux neuf tribus,
\item[\vref{Jos 13:14}] ne donna point d'h. à la tribu
\item[\vref{1 R 21:3}] de te donner l'h. de mes pères !
\item[\vref{Ps 2:8}] les nations pour h., et les extrémités
\item[\vref{Ps 16:6}] un h. délicieux m'est échu, une belle possession
\item[\vref{Ps 28:9}] et bénis ton h. ! Nourris-les et élève-les
\item[\vref{Ps 33:12}] le peuple qu'il s'est choisi pour h. !
\item[\vref{Ps 37:18}] intègres, et lr. h. demeure à jamais.
\item[\vref{Ps 61:6}] tu me donnes l'h. de ceux qui
\item[\vref{Ps 68:10}] abondante sur ton h., et qnd il
\item[\vref{Ps 79:1}] entrées ds ton h. ; on a profané
\item[\vref{Ps 127:3}] enfants sont un h. donné par Yahweh,
\item[\vref{Pr 20:21}] L'h. pour lequel on s'est trop hâté
\item[\vref{Mt 21:38}] Venez, tuons-le et emparons-ns. de son h.
\item[\vref{Mt 25:34}] Père, possédez en h. le Royaume qui
\item[\vref{Mc 12:7}] venez, tuons-le, et l'h. sera à ns.
\item[\vref{Lu 12:13}] frère qu'il partage avec moi notre h.
\item[\vref{Ac 20:32}] pour vs. donner l'h. avec ts les
\item[\vref{Ac 26:18}] aient part à l'h. des saints.
\item[\vref{Ga 3:18}] Car si l'h. venait de la loi, il ne
\item[\vref{Ep 1:14}] gage de notre h. jusqu'à la rédemption
\item[\vref{Ep 1:18}] gloire de son h. qu'il réserve aux
\item[\vref{Col 1:12}] d'avoir part à l'h. des saints ds
\item[\vref{Col 3:24}] recevrez du Seign. l'h. pour récompense. Car
\item[\vref{Hé 9:15}] été appelés reçoivent l'h. éternel qui lr.
\item[\vref{Hé 11:8}] devait recevoir en h., et il partit
\item[\vref{1 Pi 1:4}] pour un h. incorruptible, et qui ne peut ni
\end{listverse}

\ConcordanceEntry{Hériter}
\vspace{-2mm}
\begin{listverse}
\item[\vref{Ge 21:10}] de cette servante n'h. point avec mon
\item[\vref{1 Ch 28:8}] vs. le fassiez h. à vos fils
\item[\vref{Esd 9:12}] vs. le laisserez h. à vos fils
\item[\vref{Pr 13:22}] laissera de quoi h. aux fils de
\item[\vref{Pr 19:14}] On peut h. de ses pères une maison et
\item[\vref{Es 65:9}] Juda celui qui h. de mes montagnes ;
\item[\vref{Za 8:12}] et je ferai h. ttes ces choses
\item[\vref{Mt 5:5}] humbles, car ils h. la terre !
\item[\vref{Mc 10:17}] que ferai-je pour h. la vie éternelle ?
\item[\vref{Lu 18:18}] dois-je faire pour h. la vie éternelle ?
\item[\vref{1 Co 6:9}] que les injustes n'h. pas le Royaume
\item[\vref{1 Co 15:50}] sang ne peuvent h. le Royaume de
\item[\vref{Ga 4:30}] fils de l'esclave n'h. pas avec le
\item[\vref{Hé 12:17}] plus tard, désirant h. la bénédiction, il
\item[\vref{1 Pi 3:9}] êtes appelés, afin d'h. la bénédiction.
\item[\vref{Ap 21:7}] Celui qui vaincra h. ttes choses ; je
\end{listverse}

\ConcordanceEntry{Héritier}
\vspace{-2mm}
\begin{listverse}
\item[\vref{Ge 15:4}] sera pas ton h., mais celui qui
\item[\vref{2 S 14:7}] ns. exterminions mm l'h. ! Ils veulent ainsi
\item[\vref{Jé 49:1}] fils ? N'a-t-il pas d'h. ? Pourquoi dc Malcom
\item[\vref{Mt 21:38}] entre eux : Voici l'h.. Venez, tuons-le et
\item[\vref{Ga 3:29}] postérité d'Abraham, et h. selon la promesse.
\item[\vref{Ga 4:1}] aussi longtemps que l'h. est enfant, je
\item[\vref{Ga 4:7}] tu es aussi h. de Dieu par
\item[\vref{Ep 1:11}] sommes aussi devenus h., ayant été prédestinés,
\item[\vref{Tit 3:7}] ns. soyons les h. de la vie
\item[\vref{Hé 1:2}] qu'il a établi h. de ttes choses,
\item[\vref{Hé 11:7}] monde, et devint h. de la justice
\item[\vref{Hé 11:9}] Jacob, qui étaient h. avec lui de
\item[\vref{Ja 2:5}] la foi, et h. du Royaume qu'il
\item[\vref{1 Pi 3:7}] étant aussi ensemble h. de la grâce
\end{listverse}

\ConcordanceEntry{Hérode}
\vspace{-2mm}
\begin{listverse}
\item[\vref{Mt 2:1}] temps du roi H., voici des mages
\end{listverse}
\begin{legend}
\NoAutoSpaceBeforeFDP{
\item H. le grand, roi de Judée : Mt 2:1,3,7
\item Instigateur du massacre des enfants de Bethléhem: Mt 2:13-16
\item Mort d'H. : Mt 2:19
\item H. Antipas II, tétrarque : Mt 14:1; Lu 3:1
\item Fut repris par J-B au sujet d'Hérodias : Lu 3:19-20; Mt 14:1-9; Mc 6:17-18
\item H. Agrippa I, il persécuta les apôtres : Ac 12:1-3
\item H. Agrippa II : Ac 25 et 26
\item Autres : Mt 4:9-10;  Mc 6:14; Lu 6:14-15; 9:7-9; 13:31-32; 23:6-12; Ac 12:20-23; 26:24-28
}
\end{legend}

\ConcordanceEntry{Hérodias}
\vspace{-2mm}
\begin{listverse}
\item[\vref{Mt 14:3}] prison, à cause d'H., fem. de Philippe,
\end{listverse}
\begin{legend}
\NoAutoSpaceBeforeFDP{
\item Femme de Philippe, frère d'Hérode, elle quitta son mari pour épouser Hérode: Mt 14:3; Mc 6:17; Lu 3:19
\item Demanda par sa fille la tête de Jean-Baptiste : Mt 14:6-11; Mc 6:18-19
\item H. récupère la tête de J-B : Mt 14.11; Mc 6:28
}
\end{legend}

\ConcordanceEntry{Hérodien}
\vspace{-2mm}
\begin{listverse}
\item[\vref{Mt 22:16}] disciples, avec des h., qui dirent : Maître,
\item[\vref{Mc 3:6}] lui avec les H., sur comment ils
\item[\vref{Mc 12:13}] pharisiens et des h. auprès de lui
\end{listverse}

\ConcordanceEntry{Héros}
\vspace{-2mm}
\begin{listverse}
\item[\vref{Jg 5:13}] Yahweh m'a fait dominer sur les h.
\item[\vref{Jg 5:21}] tu as foulé aux pieds les h.
\item[\vref{Jg 6:12}] fort et vaillant h., Yahweh est avec
\item[\vref{1 S 17:51}] voyant que lr. h. était mort, prirent
\item[\vref{2 S 1:19}] collines ! Comment des h. sont-ils tombés ?
\item[\vref{2 S 1:25}] Comment les h. sont-ils tombés au
\item[\vref{2 S 1:27}] sont tombés les h. ? Comment se sont
\item[\vref{1 Ch 28:1}] et ts les h. et ts les
\item[\vref{Jé 20:11}] moi com. un h. puissant ; c'est pourquoi
\item[\vref{Za 10:7}] seront com. un h., et lr. cœur
\end{listverse}

\ConcordanceEntry{Hesbon}
\vspace{-2mm}
\begin{listverse}
\item[\vref{No 21:26}] Or H. était la ville de Sihon, roi
\item[\vref{No 21:28}] est sorti de H., et la flamme
\item[\vref{No 32:3}] Dibon, Jaezer, Nimra, H., et Elealé, Sebam,
\item[\vref{De 1:4}] qui habitait à H., et Og, roi
\item[\vref{De 2:24}] Sihon, roi de H., l'Amoréen, et son
\item[\vref{Jos 13:10}] qui régnait à H., jusqu'à la frontière
\item[\vref{Jg 11:26}] qu'Israël demeure à H., et ds les
\end{listverse}

\ConcordanceEntry{Heth, Héthiens}
\vspace{-2mm}
\begin{listverse}
\item[\vref{Ge 10:15}] Canaan engendra Sidon, son premier-né, et H. ;
\item[\vref{Ge 23:10}] les fils de H.. Et Ephron, l'Héthien,
\item[\vref{Ge 27:46}] des filles de H.. Si Jacob prend
\item[\vref{Ex 3:8}] les Cananéens, les H., les Amoréens, les
\item[\vref{Ex 23:23}] des Amoréens, des H., des Phéréziens, des
\item[\vref{Ex 23:28}] Cananéens, et les H., de dvt ta
\item[\vref{Ex 33:2}] les Amoréens, les H., les Phéréziens, les
\item[\vref{No 13:29}] du midi ; les H., les Jébusiens et
\item[\vref{De 7:1}] de nations : Les H., les Guirgasiens, les
\item[\vref{De 20:17}] dévouer par interdit : H., Amoréens, Cananéens, Phéréziens,
\item[\vref{Esd 9:1}] des Cananéens, des H., des Phéréziens, des
\end{listverse}

\ConcordanceEntry{Heure}
\vspace{-2mm}
\begin{listverse}
\item[\vref{Ex 9:18}] à cette mm h., une grêle tellement
\item[\vref{Jos 11:6}] à cette mm h., je les livrerai
\item[\vref{1 S 9:16}] à cette mm h., je t'enverrai un
\item[\vref{1 R 19:2}] demain, à cette h.-ci, je ne
\item[\vref{2 R 10:6}] demain à cette h.-ci, à Jizreel.
\item[\vref{Ec 9:12}] connaît pas son h., com. les poissons
\item[\vref{Mt 14:15}] est désert et l'h. est déjà avancée,
\item[\vref{Mt 20:12}] n'ont travaillé qu'une h., et tu les
\item[\vref{Mt 24:44}] l'hom. viendra à l'h. où vs. n'y
\item[\vref{Mt 26:40}] pu veiller une h. avec moi ?
\item[\vref{Mt 26:45}] et reposez-vs. ! Voici, l'h. est proche, et
\item[\vref{Mt 27:45}] Depuis la sixième h. jusqu'à la neuvième,
\item[\vref{Mc 14:35}] était possible, cette h. s'éloigne de lui.
\item[\vref{Mc 15:33}] La sixième h. étant venue, il
\item[\vref{Lu 12:40}] l'hom. viendra à l'h. où vs. n'y
\item[\vref{Lu 22:53}] c'est ici votre h., et la puissance
\item[\vref{Jn 2:4}] toi, fem. ? Mon h. n'est pas encore
\item[\vref{Jn 12:27}] délivre-moi de cette h. ? Mais c'est pour
\item[\vref{Jn 13:1}] sachant que son h. était venue pour
\item[\vref{Jn 17:1}] il dit : Père, l'h. est venue ! Glorifie
\item[\vref{Ac 10:30}] jours, à cette h.-ci, j'étais en
\item[\vref{Ro 13:11}] est déjà l’h. de ns. réveiller
\item[\vref{1 Co 4:11}] Jusqu'à cette h., ns. souffrons la
\item[\vref{Ap 3:3}] pas à quelle h. je viendrai contre
\item[\vref{Ap 3:10}] garderai aussi de l'h. de la tentation
\item[\vref{Ap 17:12}] temps avec la bête, pour une h.
\item[\vref{Ap 18:10}] condamnation est-elle venue en une seule h. ?
\end{listverse}

\ConcordanceEntry{Heureux}
\vspace{-2mm}
\begin{listverse}
\item[\vref{Ge 30:13}] pour me rendre h., car les filles
\item[\vref{Ge 40:14}] qnd tu seras h., et use de
\item[\vref{De 4:40}] que tu sois h., toi et tes
\item[\vref{De 19:13}] le sang innocent, et tu seras h.
\item[\vref{De 33:29}] que tu es h., Israël ! Qui est
\item[\vref{1 R 10:8}] H. sont tes gens ! Heureux tes serviteurs
\item[\vref{Esd 8:21}] ns. donner un h. voyage, pour nos
\item[\vref{Job 5:17}] Voici, h. est l'hom. que Dieu châtie ! Ne
\item[\vref{Pr 3:13}] H. l'hom. qui a trouvé la sagesse,
\item[\vref{Pr 8:34}] Ô ! H. est l'hom. qui m'écoute, qui veille
\item[\vref{Es 32:20}] H. vs. qui semez sur ttes les
\item[\vref{Jé 7:23}] vs. ordonne, afin que vs. soyez h.
\item[\vref{Mal 3:12}] nations vs. diront h., car vs. serez
\end{listverse}

\ConcordanceEntry{Hiddékel}
\vspace{-2mm}
\begin{listverse}
\item[\vref{Ge 2:14}] troisième fleuve est H., qui coule vers
\item[\vref{Da 10:4}] bord du grand fleuve qui est H.
\end{listverse}

\ConcordanceEntry{Hin}
\vspace{-2mm}
\begin{listverse}
\item[\vref{Ex 29:40}] quatrième partie d'un h. d'huile vierge, et
\item[\vref{Lé 19:36}] juste et un h. juste. Je suis
\item[\vref{No 15:6}] farine, pétrie ds un tiers de h. d'huile,
\item[\vref{No 28:14}] libations seront d'un demi-h. de vin pour
\item[\vref{Ez 4:11}] sixième partie d'un h. ; tu la boiras
\item[\vref{Ez 46:14}] la troisième d'un h. d'huile pour pétrir
\end{listverse}

\ConcordanceEntry{Hinnom}
\vspace{-2mm}
\begin{listverse}
\item[\vref{Jos 15:8}] la vallée de Ben-H., au côté du
\item[\vref{Jos 18:16}] la vallée de Ben-H., ds la vallée
\item[\vref{2 R 23:10}] des fils de H., afin que personne
\item[\vref{2 Ch 28:3}] du fils de H., et il brûla
\item[\vref{Né 11:30}] depuis Beer-Schéba jusqu'à la vallée de H.
\item[\vref{Jé 7:31}] la vallée de Ben-H., pour brûler au
\item[\vref{Jé 7:32}] la vallée de Ben-H., mais la vallée
\item[\vref{Jé 32:35}] la vallée de Ben-H., pour faire passer
\end{listverse}

\ConcordanceEntry{Hiram}
\vspace{-2mm}
\begin{listverse}
\item[\vref{2 S 5:11}] H., roi de Tyr, envoya des messagers
\item[\vref{1 R 5:1}] H., roi de Tyr, envoya ses serviteurs
\item[\vref{1 R 5:11}] Salomon donna à H. vingt mille cors
\item[\vref{1 R 5:12}] eut paix entre H. et Salomon, et
\item[\vref{1 R 5:18}] Salomon et ceux d'H., taillèrent les pierres
\item[\vref{1 R 7:13}] roi Salomon fit venir de Tyr H. ;
\item[\vref{1 R 7:14}] d'un père tyrien, H. travaillait le cuivre ;
\item[\vref{1 R 9:11}] H., roi de Tyr, avait fourni à
\item[\vref{1 R 10:11}] les navires de H., qui amenèrent de
\end{listverse}

\ConcordanceEntry{Holocauste}
\vspace{-2mm}
\begin{listverse}
\item[\vref{Ge 8:20}] il offrit des h. sur l'autel.
\item[\vref{Ge 22:6}] le bois de l'h. et le mit
\item[\vref{Ex 29:25}] sur l'autel, sur l'h., pour être une
\item[\vref{Lé 1:3}] offrande pour un h. est de gros
\item[\vref{Lé 4:24}] où l'on égorge l'h. dvt Yahweh. C'est
\item[\vref{No 28:3}] chaque jour, en h. perpétuel.
\item[\vref{No 28:24}] offrira cela outre l'h. perpétuel, et sa
\item[\vref{No 29:8}] vs. offrirez en h., de bonne odeur
\item[\vref{Jos 22:26}] non pour des h. ni pour des
\item[\vref{1 S 6:14}] jeunes vaches en h. à Yahweh.
\item[\vref{1 S 13:9}] dit : Amenez-moi un h. et des sacrifices
\item[\vref{1 S 15:22}] prend-il plaisir aux h. et aux sacrifices,
\item[\vref{2 S 24:24}] mon Dieu, des h. qui ne me
\item[\vref{2 R 16:15}] Urie : Fais brûler l'h. du matin et
\item[\vref{Job 1:5}] il offrait des h. selon le nombre
\item[\vref{Ps 40:7}] ne demandes ni h. ni victime expiatoire
\item[\vref{Ps 51:18}] je t'en donnerais ; l'h. ne t'est point
\item[\vref{Ez 46:2}] prêtres prépareront son h. et ses sacrifices
\item[\vref{Mc 12:33}] que ts les h. et les sacrifices.
\item[\vref{Hé 10:6}] pris plaisir aux h., ni aux sacrifices
\end{listverse}

\ConcordanceEntry{Homme}
\vspace{-2mm}
\begin{listverse}
\item[\vref{Ge 1:26}] Dieu dit : Faisons l'h. à notre image,
\item[\vref{Ge 1:27}] Dieu créa l'h. à son image,
\item[\vref{Ge 2:7}] Yahweh Dieu forma l'h. de la poussière
\item[\vref{Ge 2:18}] pas bon que l'h. soit seul ; je
\item[\vref{Ge 2:24}] C'est pourquoi l'h. quittera son père
\item[\vref{Ge 3:22}] Dieu dit : Voici, l'h. est devenu com.
\item[\vref{Ge 3:24}] ainsi qu'il chassa l'h., et il mit
\item[\vref{Ge 5:1}] où Dieu créa l'h.. Il le fit
\item[\vref{Ge 5:2}] donna le nom d'h., le jour où
\item[\vref{Ge 6:6}] repentit d'avoir fait l'h. sur la terre,
\item[\vref{Ge 6:9}] Noé était un h. juste et intègre
\item[\vref{Ge 32:24}] seul. Alors un h. lutta avec lui
\item[\vref{Ex 10:11}] mntnt, vs. les h., et servez Yahweh ;
\item[\vref{Ex 33:20}] face, car nul h. ne peut me
\item[\vref{No 12:3}] Or cet h. Moïse était un hom. fort doux,
\item[\vref{De 33:1}] bénédiction dont Moïse, h. de Dieu, bénit
\item[\vref{Jos 5:13}] regarda. Voici, un h. qui avait son
\item[\vref{Jg 8:21}] car tel est l'h., telle est sa
\item[\vref{Jg 13:6}] en disant : Un h. de Dieu est
\item[\vref{1 S 9:6}] cette ville un h. de Dieu, qui
\item[\vref{1 S 16:7}] pas ce que l'h. considère ; car l'hom.
\item[\vref{2 S 12:7}] Tu es cet h.-là ! Ainsi parle
\item[\vref{1 R 8:46}] n'y a point d'h. qui ne pèche,
\item[\vref{1 R 13:6}] et dit à l'h. de Dieu : Implore
\item[\vref{2 R 1:9}] avec ses cinquante h.. Ce chef monta
\item[\vref{2 R 1:12}] je suis un h. de Dieu, que
\item[\vref{2 R 8:2}] la parole de l'h. de Dieu. Elle
\item[\vref{Job 9:2}] ainsi ; et comment l'h. mortel se justifierait-il
\item[\vref{Job 14:10}] Mais l'h. meurt et perd tte sa force ;
\item[\vref{Job 15:14}] Qu'est-ce que l'h. mortel pour qu'il
\item[\vref{Job 15:16}] abominable et corrompu l'h. qui boit l'iniquité
\item[\vref{Job 33:23}] a pour cet h. un messager qui
\item[\vref{Ps 8:5}] Qu'est-ce que l'h., pour que tu
\item[\vref{Ps 56:5}] rien : Que peuvent me faire les h. ?
\item[\vref{Ps 82:7}] mourrez com. des h., et vs. les
\item[\vref{Pr 20:24}] Les pas de l'h. sont dirigés par
\item[\vref{Ec 9:1}] Dieu ; mais les h. ne connaissent ni
\item[\vref{Es 31:3}] Egyptiens sont des h. et non Dieu ;
\item[\vref{Es 59:16}] n'y a aucun h., il s'étonne que
\item[\vref{Ez 21:14}] Fils de l'h., prophétise, et dis :
\item[\vref{Za 8:23}] jours-là que dix h. de ttes les
\item[\vref{Za 13:7}] Berger, et sur l'h. qui est mon
\item[\vref{Mt 16:26}] servirait-il à un h. de gagner tt
\item[\vref{Mt 24:30}] du Fils de l'h. paraîtra ds le
\item[\vref{Mt 24:40}] Alors, de deux h. qui seront ds
\item[\vref{Mt 26:72}] disant : Je ne connais pas cet h.
\item[\vref{Mc 5:8}] Sors de cet h., esprit impur.
\item[\vref{Mc 7:20}] qui sort de l'h., c'est ce qui
\item[\vref{Mc 8:24}] Et cet h. ayant regardé, dit : Je vois des
\item[\vref{Lu 10:30}] et dit : Un h. descendait de Jérus.
\item[\vref{Lu 22:22}] Le Fils de l'h. s'en va selon
\item[\vref{Jn 2:25}] rende témoignage d'aucun h. ; car il savait
\item[\vref{Jn 4:29}] Venez voir un h. qui m'a dit
\item[\vref{Jn 7:23}] Si un h. reçoit la circoncision le jour du
\item[\vref{Jn 10:33}] que, n'étant qu'un h., tu te fais
\item[\vref{Jn 12:34}] le Fils de l'h. soit élevé ? Qui
\item[\vref{Jn 19:5}] pourpre. Et Pilate lr. dit : Voici l'h.
\item[\vref{Ac 23:18}] t'amener ce jeune h., qui a qq
\item[\vref{Ro 1:27}] et les h., laissant de mm l'usage naturel de
\item[\vref{Ro 5:19}] désobéissance d'un seul h., plusieurs ont été
\item[\vref{1 Co 2:11}] Car lequel des h., en effet, connaît
\item[\vref{1 Co 11:3}] chef de tt h., que l'hom. est
\item[\vref{1 Co 15:21}] par un seul h., c'est aussi par
\item[\vref{1 Co 15:45}] écrit : Le premier h., Adam, a été
\item[\vref{2 Co 4:16}] mm si notre h. extérieur se détruise,
\item[\vref{Ga 3:28}] a plus ni h. ni fem. ; car
\item[\vref{1 Th 4:8}] rejette pas un h., mais Dieu qui
\item[\vref{2 Th 2:3}] auparavant et que l'h. de péché, le
\item[\vref{1 Ti 2:5}] Dieu et les h. est un, à
\item[\vref{1 Ti 6:11}] Mais toi, h. de Dieu, fuis
\item[\vref{2 Ti 3:17}] afin que l'h. de Dieu soit
\item[\vref{Ja 1:12}] Béni est l'h. qui endure la
\end{listverse}

\ConcordanceEntry{Honnêtement}
\vspace{-2mm}
\begin{listverse}
\item[\vref{Ro 13:13}] Marchons h., com. en plein jour, loin des
\item[\vref{1 Th 4:12}] vs. vs. conduisiez h. envers ceux du
\item[\vref{1 Ti 3:4}] faut qu'il dirige h. sa propre maison,
\item[\vref{1 Ti 3:12}] seule fem., dirigeant h. leurs enfants, et
\item[\vref{Hé 13:18}] désirant ns. conduire h. parmi ts.
\end{listverse}

\ConcordanceEntry{Honneur}
\vspace{-2mm}
\begin{listverse}
\item[\vref{Ge 50:10}] Joseph fit en l'h. de son père
\item[\vref{No 22:17}] te rendrai beaucoup d'h., et je ferai
\item[\vref{2 S 6:22}] je serai en h. auprès des servantes
\item[\vref{2 R 23:21}] la Pâque en l'h. de Yahweh, votre
\item[\vref{Est 6:3}] roi dit : Quel h. et quelle distinction
\item[\vref{Ps 49:13}] qui est en h. n'a point de
\item[\vref{Ez 16:21}] le feu, en l'h. de ces idoles.
\item[\vref{Da 2:6}] et un grand h.. Quoi qu'il en
\item[\vref{Da 5:18}] royaume, la magnificence, la gloire et l'h.
\item[\vref{Mal 1:6}] Père, où est l'h. qui m'appartient ? Et
\item[\vref{Mal 1:11}] de l'encens en l'h. de mon Nom,
\item[\vref{Mc 6:4}] prophète n'est sans h. que ds sa
\item[\vref{Lu 14:10}] cela te fera h. dvt ts ceux
\item[\vref{Ro 2:7}] la gloire, et l'h. et l'immortalité ;
\item[\vref{Ro 9:21}] terre un vase d'h. et un vase
\item[\vref{Ro 12:10}] prévenant les uns les autres par h. ;
\item[\vref{Ro 13:7}] devez la crainte, l'h. à qui vs.
\item[\vref{1 Ti 1:17}] seul sage, soient h. et gloire aux
\item[\vref{1 Ti 5:17}] dignes d'un double h., spécialement ceux qui
\item[\vref{1 Ti 6:1}] maîtres tte sorte d'h., afin qu'on ne
\item[\vref{Hé 2:9}] de gloire et d'h. par la passion
\item[\vref{Hé 3:3}] maison, a plus d'h. que la maison
\item[\vref{Hé 5:4}] ne s'attribue cet h., si ce n'est
\item[\vref{1 Pi 3:7}] les traitant avec h. com. étant aussi
\item[\vref{2 Pi 1:17}] Dieu le Père h. et gloire, lorsque
\item[\vref{Ap 21:26}] la gloire et l'h. des nations.
\end{listverse}

\ConcordanceEntry{Honorable}
\vspace{-2mm}
\begin{listverse}
\item[\vref{Es 3:5}] vieillard, et l'hom. de rien contre l'h.
\item[\vref{Es 43:4}] tu es rendu h. et je t'aime,
\item[\vref{Es 58:13}] tes délices, et h. ce qui est
\item[\vref{Es 60:13}] ensemble pour rendre h. le lieu de
\item[\vref{Za 11:13}] potier, ce prix h. auquel ils m'ont
\item[\vref{Mc 15:43}] Joseph d'Arimathée, conseiller h., qui attendait aussi
\item[\vref{Lu 14:8}] parmi les conviés une personne plus h. que toi,
\item[\vref{1 Co 12:23}] être les moins h. au corps, ns.
\item[\vref{Ph 4:8}] celles qui sont h., ttes celles qui
\item[\vref{1 Ti 3:2}] fem., vigilant, modéré, h., hospitalier, propre à
\item[\vref{1 Ti 3:13}] s'acquièrent un rang h., et une grande
\item[\vref{Tit 2:10}] afin de rendre h. en ttes choses
\item[\vref{Hé 13:4}] Le mariage est h. entre ts, et
\end{listverse}

\ConcordanceEntry{Honorer}
\vspace{-2mm}
\begin{listverse}
\item[\vref{Ex 20:12}] H. ton père et ta mère, afin
\item[\vref{1 S 2:30}] pas ainsi ! Car j'h. ceux qui m'honorent,
\item[\vref{1 S 15:30}] J'ai péché ! Mais h.-moi mntnt, je
\item[\vref{2 S 10:3}] ce soit pour h. ton père que
\item[\vref{1 Ch 19:3}] ce soit pour h. ton père que
\item[\vref{Esd 7:27}] du roi, pour h. la maison de
\item[\vref{Est 1:20}] ttes les femmes h. leurs maris, depuis
\item[\vref{Est 6:6}] prend plaisir à h. ? Or Haman dit
\item[\vref{Est 6:9}] prend plaisir à h., et qu'on le
\item[\vref{Est 6:11}] que le roi prend plaisir à h.
\item[\vref{Ps 15:4}] méprisable, mais qui h. ceux qui craignent
\item[\vref{Pr 3:9}] H. Yahweh avec tes biens et les
\item[\vref{Es 29:13}] bouche et qu'il m'h. de ses lèvres,
\item[\vref{Es 49:5}] toutefois je serai h. aux yeux de
\item[\vref{Mal 1:6}] Un fils h. son père, et un serviteur son
\item[\vref{Jn 5:23}] afin que ts h. le Fils com.
\item[\vref{Jn 8:49}] de démon, mais j'h. mon Père, et
\item[\vref{1 Pi 2:14}] méchants et pour h. les gens de
\item[\vref{1 Pi 2:17}] H. tt le monde ; aimez ts vos
\end{listverse}

\ConcordanceEntry{Honte}
\vspace{-2mm}
\begin{listverse}
\item[\vref{Ge 2:25}] nus, et ils n'en avaient point h.
\item[\vref{2 Ch 32:21}] pays, ds la h.. Il entra ds
\item[\vref{Esd 9:6}] Mon Dieu ! J'ai h., et je suis
\item[\vref{Job 8:22}] seront revêtus de h., et la tente
\item[\vref{Ps 35:26}] soient couverts de h. et de confusion !
\item[\vref{Ps 44:16}] moi, et la h. couvre ma face,
\item[\vref{Ps 69:8}] souffert l'opprobre, la h. a couvert mon
\item[\vref{Ps 119:80}] je ne sois pas couvert de h. !
\item[\vref{Pr 10:5}] moisson est un enfant qui fait h.
\item[\vref{Es 30:5}] il sera lr. h. et lr. opprobre.
\item[\vref{Es 54:4}] tu oublieras la h. de ta jeunesse,
\item[\vref{Jé 3:24}] Car la h. a dévoré dès notre jeunesse le
\item[\vref{Jé 8:12}] ont mm aucune h., et ils ne
\item[\vref{Jé 20:11}] Ce sera une h. éternelle qui ne
\item[\vref{Os 4:19}] et ils auront h. de leurs sacrifices.
\item[\vref{Mi 7:10}] verra, et la h. la couvrira ; elle
\item[\vref{Mc 8:38}] Car quiconque aura h. de moi et
\item[\vref{Lu 9:26}] Car quiconque aura h. de moi et
\item[\vref{Ro 1:16}] je n'ai pas h. de l'Evangile de
\item[\vref{Ro 6:21}] mntnt vs. avez h.. Certes la fin
\item[\vref{1 Co 11:14}] que c'est une h. pour l'hom. d'avoir
\item[\vref{2 Th 3:14}] lui, afin qu'il éprouve de la h.
\item[\vref{2 Ti 1:8}] N'aie dc pas h. du témoignage à
\item[\vref{2 Ti 1:16}] n'a pas eu h. de mes chaînes.
\item[\vref{Hé 11:16}] Dieu n'a pas h. d'être appelé lr.
\item[\vref{Hé 12:2}] ayant méprisé la h., et s'est assis
\item[\vref{Ap 16:15}] et qu'on ne voie pas sa h. !
\end{listverse}

\ConcordanceEntry{Honteux}
\vspace{-2mm}
\begin{listverse}
\item[\vref{Job 6:20}] mais ils sont h. d'y avoir espéré ;
\item[\vref{Ps 83:18}] Qu'ils soient h. et épouvantés à
\item[\vref{Es 45:17}] ne serez ni h. ni confus jsq.
\item[\vref{Es 50:7}] que je ne serai point rendu h.
\item[\vref{Jé 12:13}] sans profit. Soyez h. de vos récoltes,
\item[\vref{Jé 51:17}] tt fondeur est h. par les images
\item[\vref{2 Co 4:2}] rejeté les choses h. que l'on cache,
\item[\vref{Ep 5:12}] Car il est h. de dire les
\end{listverse}

\ConcordanceEntry{Hophni}
\vspace{-2mm}
\begin{listverse}
\item[\vref{1 S 1:3}] deux fils d'Eli, H. et Phinées, prêtres
\item[\vref{1 S 2:34}] tes deux fils, H. et Phinées ; ils
\item[\vref{1 S 4:4}] deux fils d'Eli, H. et Phinées, étaient
\item[\vref{1 S 4:11}] deux fils d'Eli, H. et Phinées, moururent.
\item[\vref{1 S 4:17}] tes deux fils, H. et Phinées, sont
\end{listverse}

\ConcordanceEntry{Horeb}
\vspace{-2mm}
\begin{listverse}
\item[\vref{Ex 3:1}] à la montagne de Dieu à H.
\item[\vref{Ex 17:6}] sur le rocher d'H. ; et tu frapperas
\item[\vref{Ex 33:6}] de leurs ornements vers la montagne d'H.
\item[\vref{De 1:6}] a parlé à H., en disant : Vous
\item[\vref{De 4:10}] ton Dieu, à H., après que Yahweh
\item[\vref{De 5:2}] traité avec ns. une alliance en H.
\item[\vref{1 R 8:9}] y déposa en H., lorsque Yahweh fit
\item[\vref{1 R 19:8}] quarante nuits jusqu'à H., la montagne de
\item[\vref{Ps 106:19}] un veau en H., et se prosternèrent
\item[\vref{Mal 4:4}] j'ai prescrit en H., pour tt Israël,
\end{listverse}

\ConcordanceEntry{Horreur}
\vspace{-2mm}
\begin{listverse}
\item[\vref{Job 6:21}] calamité étonnante, et vs. en avez h. !
\item[\vref{Ez 16:5}] de ta naissance, parce qu'on avait h. de toi.
\end{listverse}

\ConcordanceEntry{Horrible}
\vspace{-2mm}
\begin{listverse}
\item[\vref{Jé 5:30}] le pays une chose étonnante et h. :
\item[\vref{Jé 18:13}] d'Israël a fait une chose très h.
\item[\vref{Jé 23:14}] vu des choses h. ds les prophètes
\end{listverse}

\ConcordanceEntry{Hosanna}
\vspace{-2mm}
\begin{listverse}
\item[\vref{Mt 21:9}] criaient, en disant : H. au Fils de
\item[\vref{Mt 21:15}] ds le temple : H. au Fils de
\item[\vref{Mc 11:9}] criaient, en disant : H. ! Béni soit celui
\item[\vref{Mc 11:10}] Nom du Seign. ! H. ds les lieux
\item[\vref{Jn 12:13}] lui, en criant : H. ! Béni soit le
\end{listverse}

\ConcordanceEntry{Hospitalité}
\vspace{-2mm}
\begin{listverse}
\item[\vref{Ro 12:13}] participez aux nécessités des saints ; exercez l'h.
\item[\vref{Ro 16:2}] elle a exercé l'h. à l'égard de
\item[\vref{1 Ti 5:10}] enfants, d'avoir exercé l'h. envers les étrangers,
\item[\vref{Hé 13:2}] N'oubliez pas l'h. ; car, par elle,
\end{listverse}

\ConcordanceEntry{Houlette}
\vspace{-2mm}
\begin{listverse}
\item[\vref{Ps 23:4}] bâton et ta h. me consolent.
\item[\vref{Mi 7:14}] peuple avec ta h., le troupeau de
\end{listverse}

\ConcordanceEntry{Huile}
\vspace{-2mm}
\begin{listverse}
\item[\vref{Ge 35:14}] une libation et y versa de l'h.
\item[\vref{Ex 25:6}] de l'h. pour le luminaire, des aromates pour
\item[\vref{Ex 30:25}] en feras de l'h. pour l'onction sainte,
\item[\vref{Lé 2:1}] il versera de l'h. dessus, et mettra
\item[\vref{Lé 10:7}] ne mouriez, car l'h. de l'onction de
\item[\vref{No 4:16}] la surveillance de l'h. du luminaire, du
\item[\vref{No 11:8}] Elle avait le goût d'une liqueur d'h. fraîche.
\item[\vref{No 28:12}] farine pétrie à l'h., pour le gâteau
\item[\vref{De 32:13}] miel du rocher, l'h. du rocher le
\item[\vref{1 S 16:1}] Remplis ta corne d'h., et viens ; je
\item[\vref{1 R 17:12}] et un peu d'h. ds une cruche.
\item[\vref{2 R 4:6}] de vase. Et l'h. s'arrêta.
\item[\vref{Ps 23:5}] adversaires ; tu oins d'h. ma tête et
\item[\vref{Ps 45:8}] t'a oint d'une h. de joie par
\item[\vref{Ps 55:22}] plus douces que l'h., néanmoins elles sont
\item[\vref{Ps 89:21}] je l'ai oint de ma sainte h.
\item[\vref{Ps 92:11}] d'un buffle ; je serai oint d'une h. fraîche.
\item[\vref{Ps 133:2}] C'est com. cette h. précieuse, répandue sur
\item[\vref{Ez 46:5}] aura un hin d'h. pour chaque épha.
\item[\vref{Mt 25:4}] sages prirent de l'h. ds leurs vases
\item[\vref{Mt 25:8}] Donnez-ns. de votre h., car nos lampes
\item[\vref{Mc 6:13}] et ils oignirent d'h. beaucoup de malades
\item[\vref{Lu 7:38}] oignit de cette h. odoriférante.
\item[\vref{Lu 7:46}] oint ma tête d'h. ; mais elle, elle
\item[\vref{Lu 10:34}] y versant de l'h. et du vin ;
\item[\vref{Lu 16:6}] dit : Cent mesures d'h.. Et il lui
\item[\vref{Hé 1:9}] t'a oint d'une h. de joie par-dessus
\item[\vref{Ja 5:14}] lui en l'oignant d'h. au Nom du
\item[\vref{Ap 6:6}] de mal au vin et à l'h.
\item[\vref{Ap 18:13}] du vin, de l'h., de la fine
\end{listverse}

\ConcordanceEntry{Hulda}
\vspace{-2mm}
\begin{listverse}
\item[\vref{2 R 22:14}] de la prophétesse H., fem. de Schallum,
\item[\vref{2 Ch 34:22}] roi allèrent vers H., la prophétesse, fem.
\end{listverse}

\ConcordanceEntry{Humble}
\vspace{-2mm}
\begin{listverse}
\item[\vref{Ps 25:9}] Il conduit les h. ds la justice,
\item[\vref{Ps 138:6}] il voit les h. et il reconnaît
\item[\vref{Pr 29:23}] celui qui est h. d'esprit obtient la
\item[\vref{Es 57:15}] et qui est h. d'esprit, afin de
\item[\vref{So 3:12}] toi un peuple h. et faible, et
\item[\vref{Mt 11:29}] suis doux et h. de cœur ; et
\item[\vref{Mt 18:4}] pourquoi, quiconque deviendra h., com. ce petit
\item[\vref{Ro 12:16}] ce qui est h.. Ne soyez pas
\end{listverse}

\ConcordanceEntry{Humiliation}
\vspace{-2mm}
\begin{listverse}
\item[\vref{Né 6:16}] éprouvèrent une grande h., et ils reconnurent
\item[\vref{Ac 8:33}] Dans son h., son jugement a été levé ; mais
\end{listverse}

\ConcordanceEntry{Humilier}
\vspace{-2mm}
\begin{listverse}
\item[\vref{Ex 10:3}] qnd refuseras-tu de t'h. dvt moi ? Laisse
\item[\vref{Lé 23:29}] personne qui ne s'h. point en ce
\item[\vref{De 8:2}] désert, afin de t'h. et de t'éprouver,
\item[\vref{De 8:16}] connue, afin de t'h. et de t'éprouver,
\item[\vref{2 S 22:28}] le peuple qui s'h., et de ton
\item[\vref{2 R 22:19}] que tu t'es h. dvt Yahweh en
\item[\vref{2 Ch 32:26}] Mais Ezéchias s'h. de l'élévation de
\item[\vref{2 Ch 33:12}] Dieu, et il s'h. profondément dvt le
\item[\vref{Esd 8:21}] afin de ns. h. dvt notre Dieu,
\item[\vref{Ps 107:12}] Il h. lr. cœur par le travail, ils
\item[\vref{Ps 119:67}] Avant d'avoir été h., je m'égarais, mais
\item[\vref{Ps 119:71}] que je sois h. afin que j'apprenne
\item[\vref{Ps 119:75}] que tu m'as h. par ta fidélité.
\item[\vref{Pr 3:34}] il fait grâce à ceux qui s'h.
\item[\vref{La 3:33}] volonté d'affliger et d'h. les fils des
\item[\vref{Da 10:12}] comprendre, et à t'h. dvt ton Dieu,
\item[\vref{Ja 4:10}] H.-vs. ds la présence du Seign.,
\item[\vref{1 Pi 5:6}] H.-vs. dc sous la puissante main
\end{listverse}

\ConcordanceEntry{Humilité}
\vspace{-2mm}
\begin{listverse}
\item[\vref{Pr 15:33}] la sagesse, et l'h. précède la gloire.
\item[\vref{Pr 18:12}] ruine arrive, mais l'h. précède la gloire.
\item[\vref{Pr 22:4}] La récompense de l'h. et de la
\item[\vref{Mi 6:8}] marches en tte h. avec ton Dieu.
\item[\vref{So 2:3}] la justice, cherchez l'h. ; peut-être serez-vs. protégés
\item[\vref{Ac 20:19}] Seign. en tte h., avec beaucoup de
\item[\vref{Ep 4:2}] avec tte h. et douceur, avec
\item[\vref{Ph 2:3}] gloire, mais que l'h. de cœur vs.
\item[\vref{Col 2:18}] course, sous l'apparence d'h. d'esprit et par
\item[\vref{Col 2:23}] volontaire, et en h. d'esprit, et en
\item[\vref{Col 3:12}] miséricorde, de bonté, d'h., de douceur, de
\item[\vref{1 Pi 5:5}] les autres, revêtez-vs. d'h. ; parce que Dieu
\end{listverse}

\ConcordanceEntry{Hur}
\vspace{-2mm}
\begin{listverse}
\item[\vref{Ex 17:12}] dessus ; Aaron et H. soutenaient ses mains,
\item[\vref{Ex 24:14}] voici, Aaron et H. sont avec vs.;
\item[\vref{Ex 31:2}] d'Uri, fils de H., de la tribu
\item[\vref{No 31:8}] Evi, Rékem, Tsur, H., et Réba, cinq
\item[\vref{Jos 13:21}] Evi, Rékem, Tsur, H., et Réba, princes
\item[\vref{1 R 4:8}] Le fils de H., sur la montagne
\item[\vref{1 Ch 2:20}] H. engendra Uri, et Uri engendra Betsaleel.
\item[\vref{1 Ch 2:50}] Caleb, fils de H., premier-né d'Ephrata : Schobal,
\item[\vref{2 Ch 1:5}] d'Uri, fils de H., avait fait, était
\item[\vref{Né 3:9}] Rephaja, fils de H., chef d'un demi-quartier
\end{listverse}

\ConcordanceEntry{Hyménée}
\vspace{-2mm}
\begin{listverse}
\item[\vref{1 Ti 1:20}] ce nombre sont H. et Alexandre, que
\item[\vref{2 Ti 2:17}] ce nombre sont H. et Philète,
\end{listverse}

\ConcordanceEntry{Hypocrisie}
\vspace{-2mm}
\begin{listverse}
\item[\vref{Es 32:6}] pour exécuter son h. et pour proférer
\item[\vref{Da 11:34}] plusieurs se joindront à eux par h.
\item[\vref{Mt 23:28}] vs. êtes pleins d'h. et d'iniquité.
\item[\vref{Mc 12:15}] Jésus, connaissant lr. h., lr. dit : Pourquoi
\item[\vref{Lu 12:1}] du levain des pharisiens qui est l'h.
\item[\vref{Ga 2:13}] mm se laissa entraîner par lr. h.
\item[\vref{1 Ti 4:2}] par l'h. de faux docteurs, ayant lr. propre
\item[\vref{Ja 3:17}] bons fruits, sans partialité et sans h.
\item[\vref{1 Pi 1:22}] qui soit sans h., aimez-vs. ardemment les
\end{listverse}

\ConcordanceEntry{Hypocrite}
\vspace{-2mm}
\begin{listverse}
\item[\vref{Job 8:13}] qui oublient Dieu, et l'espérance de l'h. périra.
\item[\vref{Job 13:16}] ma délivrance ; mais l'h. ne viendra pas
\item[\vref{Job 20:5}] la joie de l'h. n'est que pour
\item[\vref{Job 27:8}] attente reste-t-il à l'h. qnd Dieu lui
\item[\vref{Job 34:30}] Afin que l'h. ne règne pas,
\item[\vref{Pr 8:13}] de la méchanceté et la bouche h.
\item[\vref{Pr 14:14}] a un cœur h. sera rassasié de
\item[\vref{Pr 17:20}] le bien ; et l'h. tombe ds le
\item[\vref{Mt 6:2}] com. font les h. ds les synagogues
\item[\vref{Mt 7:5}] H. ! Ôte premièrement de ton œil la
\item[\vref{Mt 22:18}] lr. malice, dit : H. ! Pourquoi me tentez-vs. ?
\item[\vref{Mt 23:13}] scribes et pharisiens h. ! qui fermez le
\item[\vref{Mt 24:51}] au rang des h. ; là il y
\item[\vref{Lu 6:42}] ds ton œil ? H. ! Ôte premièrement la
\item[\vref{Lu 13:15}] H. ! Lui répondit le Seign., chacun de
\end{listverse}

\ConcordanceEntry{Hysope}
\vspace{-2mm}
\begin{listverse}
\item[\vref{Ex 12:22}] prendrez un bouquet d'h. et le tremperez
\item[\vref{Lé 14:51}] bois de cèdre, l'h., le cramoisi et
\item[\vref{No 19:18}] pur prendra de l'h., et la trempera
\item[\vref{1 R 4:33}] du Liban jusqu'à l'h. qui sort de
\item[\vref{Ps 51:9}] péché avec de l'h., et je serai
\item[\vref{Jn 19:29}] bout d'une branche d'h., et la lui
\item[\vref{Hé 9:19}] écarlate, et de l'h. ; et il en
\end{listverse}

\ConcordanceEntry{Ibtsan}
\vspace{-2mm}
\begin{listverse}
\item[\vref{Jg 12:8}] Après lui, I. de Bethléhem fut
\item[\vref{Jg 12:10}] Puis I. mourut, et fut enterré à Bethléhem.
\end{listverse}

\ConcordanceEntry{Icone}
\vspace{-2mm}
\begin{listverse}
\item[\vref{Ac 13:51}] de leurs pieds et allèrent à I.,
\item[\vref{Ac 14:1}] arriva qu'étant à I., ils entrèrent ensemble
\item[\vref{Ac 14:19}] Juifs d'Antioche et d'I. qui gagnèrent la
\item[\vref{Ac 16:2}] de Lystre et d'I. rendaient de lui
\item[\vref{2 Ti 3:11}] à Antioche, à I. et à Lystre.
\end{listverse}

\ConcordanceEntry{Iddo}
\vspace{-2mm}
\begin{listverse}
\item[\vref{1 R 4:14}] Achinadab, fils d'I., à Mahanaïm.
\item[\vref{1 Ch 6:21}] Joach son fils, I. son fils, Zérach
\item[\vref{2 Ch 12:15}] le prophète, et d'I. le voyant, parmi
\item[\vref{2 Ch 13:22}] écrites ds les mémoires du prophète I.
\item[\vref{Esd 5:1}] et Zacharie, fils d'I., le prophète, prophétisèrent
\item[\vref{Esd 8:17}] pour le chef I., demeurant à Casiphia,
\item[\vref{Né 12:16}] pour I., Zacharie ; pour Guinnethon, Meschullam ;
\item[\vref{Za 1:1}] de Bérékia, fils d'I., en disant :
\end{listverse}

\ConcordanceEntry{Idolâtre}
\vspace{-2mm}
\begin{listverse}
\item[\vref{1 Co 5:10}] ravisseurs, ou les i. ; autrement, il vs.
\item[\vref{1 Co 5:11}] ou cupide, ou i., ou médisant, ou
\item[\vref{1 Co 6:9}] fornicateurs, ni les i., ni les adultères,
\item[\vref{1 Co 10:7}] Ne devenez pas i., com. quelques-uns d'entre
\item[\vref{Ep 5:5}] qui est un i., n'a d'héritage ds
\item[\vref{Ap 21:8}] les sorciers, les i. et ts les
\end{listverse}

\ConcordanceEntry{Idolâtrie}
\vspace{-2mm}
\begin{listverse}
\item[\vref{1 S 15:23}] pas moins que l'i. et les théraphim.
\item[\vref{Ac 17:16}] de cette ville entièrement adonnée à l'i.
\item[\vref{1 Co 10:14}] C'est pourquoi, mes bien-aimés, fuyez l'i.
\item[\vref{Ga 5:20}] l'i., la sorcellerie, les inimitiés, les querelles,
\item[\vref{Col 3:5}] et la cupidité, qui est une i.
\item[\vref{1 Pi 4:3}] boire, et aux i. criminelles;
\end{listverse}

\ConcordanceEntry{Idole}
\vspace{-2mm}
\begin{listverse}
\item[\vref{Lé 26:1}] vs. ferez point d'i., vs. ne vs.
\item[\vref{Jg 6:28}] était démoli, et l'i. d'Astarté qui est
\item[\vref{1 R 16:33}] Achab fit une i. d'Astarté ; de sorte
\item[\vref{2 R 17:12}] Ils servirent les i., au sujet desquelles
\item[\vref{2 R 21:3}] il fit une i. d'Asherah, com. avait
\item[\vref{1 Ch 16:26}] peuples sont des i., mais Yahweh a
\item[\vref{2 Ch 15:16}] avait fait une i. pour Asherah. Asa
\item[\vref{2 Ch 33:7}] image taillée, une i. qu'il avait faite,
\item[\vref{Ps 115:4}] Leurs i. sont des dieux d'or et d'argent,
\item[\vref{Es 2:20}] aux chauves-souris leurs i. d'argent et leurs
\item[\vref{Es 41:7}] puis il fixe l'i. avec des clous,
\item[\vref{Es 66:3}] qui bénirait une i. ; ts ceux-là ont
\item[\vref{Ez 8:5}] l'autel, était cette i. de jalousie, à
\item[\vref{Ez 14:3}] gens élèvent leurs i. ds leurs cœurs,
\item[\vref{Ez 20:8}] n'abandonnèrent point les i. de l'Egypte. Et
\item[\vref{Ez 36:25}] vos souillures et de ttes vos i.
\item[\vref{Os 9:10}] consacrés à l'infâme i., et ils sont
\item[\vref{Ac 7:41}] des sacrifices à l'i., et se réjouirent
\item[\vref{1 Co 8:4}] choses sacrifiées aux i., ns. savons que
\item[\vref{1 Co 8:7}] ont conscience de l'i., ils la mangent
\item[\vref{1 Co 10:19}] dis-je dc ? Que l'i. soit qq chose ?
\item[\vref{2 Co 6:16}] Dieu et les i. ? Car vs. êtes
\item[\vref{1 Th 1:9}] vs. séparant des i., pour servir le
\item[\vref{1 Jn 5:21}] Mes petits enfants, gardez-vs. des i.. Amen !
\item[\vref{Ap 2:20}] faire manger des choses sacrifiées aux i.
\end{listverse}

\ConcordanceEntry{Ignominie}
\vspace{-2mm}
\begin{listverse}
\item[\vref{Job 10:15}] Je suis rempli d'i., mais regarde mon
\item[\vref{Ps 69:20}] honte, et mon i. ; ts mes ennemis
\item[\vref{Ps 83:17}] Couvre leurs visages d'i. afin qu'on cherche
\item[\vref{Pr 14:34}] le péché est l'i. des peuples.
\item[\vref{Es 45:16}] vont ts avec i., les fabricants d'idoles.
\item[\vref{Jé 3:25}] honte, et notre i. ns. couvrira ; parce
\item[\vref{La 1:8}] ont vu son i. ; elle en a
\item[\vref{Ez 36:6}] que vs. portez l'i. des nations.
\item[\vref{Ez 39:26}] avoir porté lr. i., et tt lr.
\item[\vref{Os 4:7}] moi : Je changerai lr. gloire en i.
\item[\vref{Os 4:18}] Apportez ; ce n'est qu'i. que ses protecteurs.
\item[\vref{2 Co 6:8}] gloire et de l'i., au milieu de
\end{listverse}

\ConcordanceEntry{Ignorance}
\vspace{-2mm}
\begin{listverse}
\item[\vref{Ac 3:17}] avez agi par i., de mm que
\item[\vref{Ac 17:30}] compte des temps d'i., annonce mntnt à
\item[\vref{Ep 4:18}] à cause de l'i. qui est en
\item[\vref{1 Th 4:13}] vs. soyez ds l'i. au sujet de
\item[\vref{1 Ti 1:13}] que j'agissais par i., étant ds l'incrédulité.
\item[\vref{1 Pi 1:14}] à vos convoitises d'autrefois, pendant votre i.
\item[\vref{1 Pi 2:15}] la bouche à l'i. des hommes insensés ;
\end{listverse}

\ConcordanceEntry{Ignorant}
\vspace{-2mm}
\begin{listverse}
\item[\vref{Ac 20:22}] vais à Jérus., i. ce qui m'y
\item[\vref{Ro 2:20}] le maître des i., ayant le modèle
\item[\vref{1 Co 14:38}] si quelqu'un est i., qu'il soit ignorant.
\item[\vref{2 Co 11:6}] suis com. quelqu'un d'i. par rapport au
\item[\vref{2 Pi 3:16}] que les personnes i. et mal affermies
\end{listverse}

\ConcordanceEntry{I-Kabod}
\vspace{-2mm}
\begin{listverse}
\item[\vref{1 S 4:21}] elle appela l'enfant I., en disant : La
\item[\vref{1 S 14:3}] fils d'Achithub, frère d'I., fils de Phinées,
\end{listverse}

\ConcordanceEntry{Illégitime}
\vspace{-2mm}
\begin{listverse}
\item[\vref{Hé 12:8}] dc des enfants i., et non pas
\end{listverse}

\ConcordanceEntry{Illuminer}
\vspace{-2mm}
\begin{listverse}
\item[\vref{Ps 34:6}] regarde, on est i., et la face
\item[\vref{Ep 1:18}] Qu'il i. les yeux de votre esprit, afin
\end{listverse}

\ConcordanceEntry{Image}
\vspace{-2mm}
\begin{listverse}
\item[\vref{Ge 1:26}] l'hom. à notre i., selon notre ressemblance,
\item[\vref{Ge 1:27}] l'hom. à son i., il le créa
\item[\vref{De 27:15}] qui fait une i. taillée ou une
\item[\vref{Jg 17:3}] mon fils une i. taillée, et une
\item[\vref{Jg 18:14}] des théraphim, une i. taillée et une
\item[\vref{1 R 14:23}] et firent des i., et des asheras,
\item[\vref{Ps 17:15}] rassasierai de ton i., dès mon réveil.
\item[\vref{Ps 73:20}] tu méprises lr. i. à ton réveil.
\item[\vref{Ps 97:7}] qui servent les i., et qui se
\item[\vref{Ps 106:19}] prosternèrent dvt une i. de métal fondu.
\item[\vref{Es 40:19}] L'ouvrier fond l'i., et l'orfèvre la
\item[\vref{Es 42:8}] ma louange aux i. taillées.
\item[\vref{Es 48:5}] ces choses ; mon i. taillée, et mon
\item[\vref{Ac 19:35}] et de son i. tombée de Jupiter ?
\item[\vref{Ro 1:23}] Dieu incorruptible en i. représentant l'hom. corruptible,
\item[\vref{Ro 8:29}] être conformes à l'i. de son Fils,
\item[\vref{1 Co 15:49}] ns. avons porté l'i. de celui qui
\item[\vref{2 Co 3:18}] en la mm i., de gloire en
\item[\vref{2 Co 4:4}] la gloire de Christ, lequel est l'i. de Dieu.
\item[\vref{Col 1:15}] Lequel est l'i. du Dieu invisible,
\item[\vref{Col 3:10}] la connaissance, selon l'i. de celui qui
\item[\vref{Hé 8:5}] qui n'est que l'i. et l'ombre des
\item[\vref{Ap 13:15}] un esprit à l'i. de la bête,
\item[\vref{Ap 15:2}] bête et son i., et sa marque,
\item[\vref{Ap 19:20}] et adoré son i.. Et ils furent
\end{listverse}

\ConcordanceEntry{Imitateur}
\vspace{-2mm}
\begin{listverse}
\item[\vref{1 Co 4:16}] Je vs. prie dc d'être mes i.
\item[\vref{1 Co 11:1}] Soyez mes i. com. je le
\item[\vref{Ep 5:1}] Soyez dc les i. de Dieu, com.
\item[\vref{Ph 3:17}] ts ensemble mes i., mes frères, et
\item[\vref{1 Th 1:6}] avez-vs. été nos i. et ceux du
\item[\vref{1 Th 2:14}] êtes devenus les i. des églises de
\item[\vref{1 Pi 3:13}] vs. êtes les i. de celui qui
\end{listverse}

\ConcordanceEntry{Imiter}
\vspace{-2mm}
\begin{listverse}
\item[\vref{2 R 17:15}] et que Yahweh lr. avait défendu d'i.
\item[\vref{Pr 6:12}] L'hom. qui i. le démon, est
\item[\vref{Es 24:16}] et ils ont i. la mauvaise foi
\item[\vref{2 Th 3:7}] il faut ns. i., puisque ns. n'avons
\item[\vref{2 Th 3:9}] donner en ns.-mêmes un modèle à i.
\item[\vref{Hé 4:11}] ne tombe en i. une semblable rébellion.
\item[\vref{Hé 6:12}] mais que vs. i. ceux qui, par
\item[\vref{Hé 13:7}] lr. vie, et i. lr. foi.
\item[\vref{3 Jn 1:11}] Bien-aimé, n'i. pas le mal,
\end{listverse}

\ConcordanceEntry{Immoler}
\vspace{-2mm}
\begin{listverse}
\item[\vref{1 R 13:2}] nom Josias ; il i. sur toi les
\item[\vref{2 R 23:20}] Il i. sur les autels ts les prêtres
\item[\vref{2 Ch 30:15}] Puis on i. la Pâque, au quatorzième jour du
\item[\vref{2 Ch 35:1}] Jérus., et on i. la Pâque le
\item[\vref{2 Ch 35:6}] I. la Pâque, sanctifiez-vs., et préparez-la pour
\item[\vref{2 Ch 35:11}] Puis on i. la Pâque ; et les prêtres répandaient
\item[\vref{Esd 6:20}] c'est pourquoi ils i. la Pâque pour
\item[\vref{Lu 22:7}] où l'on devait i. la Pâque, arriva.
\item[\vref{Ap 5:6}] tenait là com. i., ayant sept cornes,
\item[\vref{Ap 13:8}] vie de l'Agneau i. dès la fondation
\end{listverse}

\ConcordanceEntry{Immortalité}
\vspace{-2mm}
\begin{listverse}
\item[\vref{Ro 2:7}] cherchent la gloire, et l'honneur et l'i. ;
\item[\vref{1 Co 15:53}] l'incorruptibilité, et que ce mortel revête l'i.
\item[\vref{1 Co 15:54}] mortel aura revêtu l'i., alors cette parole
\item[\vref{1 Ti 6:16}] qui seul possède l'i., et qui habite
\item[\vref{2 Ti 1:10}] la vie et l'i. par l'Evangile,
\end{listverse}

\ConcordanceEntry{Impie}
\vspace{-2mm}
\begin{listverse}
\item[\vref{2 Ch 20:35}] roi d'Israël, dont la conduite était i.
\item[\vref{2 Ch 24:7}] Car l'i. Athalie et ses fils ont ravagé
\item[\vref{Job 17:8}] sont étonnés, et l'innocent s'élève contre l'i.
\item[\vref{Pr 11:9}] Par sa bouche l'i. corrompt son prochain,
\item[\vref{Es 10:6}] contre une nation i., et je l'ai
\item[\vref{Ro 5:6}] temps, pour ns. qui étions des i.
\item[\vref{2 Th 2:9}] L'avènement de cet i., se fera par
\item[\vref{1 Ti 1:9}] rebelles, pour les i. et les pécheurs,
\item[\vref{1 Pi 4:18}] sauvé, que deviendront l'i. et le pécheur ?
\item[\vref{2 Pi 3:7}] et de la destruction des hommes i.
\end{listverse}

\ConcordanceEntry{Impiété}
\vspace{-2mm}
\begin{listverse}
\item[\vref{Ro 1:18}] ciel contre tte i. et injustice des
\item[\vref{2 Ti 2:16}] les tiennent avanceront toujours plus ds l'i.,
\item[\vref{Tit 2:12}] à renoncer à l'i. et aux passions
\item[\vref{2 Pi 2:6}] exemple à ceux qui vivraient ds l'i.,
\end{listverse}

\ConcordanceEntry{Implorer}
\vspace{-2mm}
\begin{listverse}
\item[\vref{De 3:23}] ce mm temps, j'i. la grâce de
\item[\vref{1 R 13:6}] l'hom. de Dieu : I. Yahweh, ton Dieu,
\item[\vref{2 R 8:3}] Philistins, et alla i. le roi au
\item[\vref{2 R 13:4}] Mais Joachaz i. Yahweh. Et Yahweh
\item[\vref{Ps 88:14}] moi, ô Yahweh ! j'i. ton secours, ma
\item[\vref{Jé 26:19}] craignit-il pas Yahweh ? N'i.-t-il pas Yahweh ?
\item[\vref{Da 2:18}] pour i. la miséricorde du Dieu des cieux
\item[\vref{Da 9:13}] ns. n'avons pas i. Yahweh, notre Dieu,
\item[\vref{Za 8:21}] disant : Allons, allons i. Yahweh et chercher
\end{listverse}

\ConcordanceEntry{Imposer}
\vspace{-2mm}
\begin{listverse}
\item[\vref{Ex 5:8}] Néanmoins, vs. lr. i. la quantité de
\item[\vref{De 34:9}] Moïse lui avait i. les mains. Les
\item[\vref{2 R 5:13}] le prophète t'avait i. qq chose de
\item[\vref{Job 38:10}] que je lui i. ma loi, et
\item[\vref{Es 14:3}] la dure servitude qui te fut i.,
\item[\vref{La 3:28}] et silencieux parce qu'on le lui i.
\item[\vref{Mt 9:18}] mais viens, et i.-lui ta main
\item[\vref{Mt 19:15}] Puis il lr. i. les mains et
\item[\vref{Mc 6:5}] malades en lr. i. les mains.
\item[\vref{Mc 16:18}] de mal ; ils i. les mains aux
\item[\vref{Lu 4:40}] amenèrent. Et il i. les mains à
\item[\vref{Ac 6:6}] avoir prié, lr. i. les mains.
\item[\vref{Ac 8:17}] Puis ils lr. i. les mains, et
\item[\vref{Ac 9:12}] entrant et lui i. les mains, afin
\item[\vref{Ac 13:3}] prié, ils lr. i. les mains, et
\item[\vref{Ac 15:10}] Dieu en voulant i. aux disciples un
\item[\vref{Ac 15:28}] de ne vs. i. d'autre charge que
\item[\vref{Ac 19:6}] Paul lr. eut i. les mains, le
\item[\vref{Ac 28:8}] lui, pria, lui i. les mains, et
\item[\vref{Ro 15:26}] et d'Achaïe de s'i. une contribution pour
\item[\vref{1 Co 9:16}] nécessité m'en est i., et malheur à
\item[\vref{Col 2:20}] monde, pourquoi vs. i.-t-on ces ordonnances,
\item[\vref{1 Th 2:6}] aurions pu ns. i. com. apôtres de
\item[\vref{1 Ti 5:22}] N'i. les mains à personne avec précipitation,
\end{listverse}

\ConcordanceEntry{Imposition}
\vspace{-2mm}
\begin{listverse}
\item[\vref{Ac 8:18}] était donné par l'i. des mains des
\item[\vref{1 Ti 4:14}] par prophétie, par l'i. des mains de
\item[\vref{2 Ti 1:6}] as reçu par l'i. de mes mains.
\item[\vref{Hé 6:2}] baptêmes, et de l'i. des mains, et
\end{listverse}

\ConcordanceEntry{Impossible}
\vspace{-2mm}
\begin{listverse}
\item[\vref{Mt 17:20}] transporterait ; et rien ne vs. serait i.
\item[\vref{Mt 19:26}] hommes, cela est i., mais quant à
\item[\vref{Mc 10:27}] dit : Cela est i. quant aux hommes,
\item[\vref{Lu 1:37}] Car rien n'est i. à Dieu.
\item[\vref{Lu 18:27}] Ce qui est i. aux hommes est
\item[\vref{Ro 8:3}] ce qui était i. à la loi,
\item[\vref{Hé 6:4}] Or il est i. que ceux qui
\item[\vref{Hé 6:18}] lesquelles il est i. que Dieu mente,
\item[\vref{Hé 10:4}] car il est i. que le sang
\item[\vref{Hé 11:6}] Or il est i. de lui être
\end{listverse}

\ConcordanceEntry{Imposteur}
\vspace{-2mm}
\begin{listverse}
\item[\vref{2 Ti 3:13}] hommes méchants et i. iront en empirant,
\end{listverse}

\ConcordanceEntry{Impôt}
\vspace{-2mm}
\begin{listverse}
\item[\vref{2 Ch 17:11}] Philistins, et un i. en argent ; et
\item[\vref{2 Ch 24:6}] et de Jérus., l'i. sur l'assemblée d'Israël,
\item[\vref{2 Ch 24:9}] apportât à Yahweh l'i. mis par Moïse,
\item[\vref{Esd 4:13}] de tribut, ni d'i., ni de droit
\item[\vref{Esd 4:20}] on payait tribut, i. et droit de
\item[\vref{Esd 7:24}] ni tribut, ni i., ni droit de
\item[\vref{Os 8:10}] à cause de l'i. pour le roi
\item[\vref{Mt 17:25}] tributs ou des i. ? Est-ce de leurs
\item[\vref{Ro 13:6}] vs. payez les i., parce que les
\item[\vref{Ro 13:7}] lr. est dû : L'i. à qui vs.
\end{listverse}

\ConcordanceEntry{Impotent}
\vspace{-2mm}
\begin{listverse}
\item[\vref{Ac 4:9}] à un hom. i., afin que ns.
\item[\vref{Ac 14:8}] assis un hom. i. des pieds, boiteux
\end{listverse}

\ConcordanceEntry{Imprécation}
\vspace{-2mm}
\begin{listverse}
\item[\vref{No 5:21}] avec un serment d'i. et lui dira :
\item[\vref{Mt 26:74}] à faire des i. et à jurer,
\end{listverse}

\ConcordanceEntry{Impuni}
\vspace{-2mm}
\begin{listverse}
\item[\vref{1 R 2:9}] le laisseras point i., car tu es
\item[\vref{Pr 6:29}] quiconque la touchera ne restera pas i.
\item[\vref{Pr 11:21}] ne demeurera point i., mais la race
\item[\vref{Pr 16:5}] cœur ; assurément, il ne demeurera pas i.
\item[\vref{Pr 17:5}] réjouit de l'affliction ne demeurera pas i.
\item[\vref{Pr 19:5}] ne restera pas i., et celui qui
\item[\vref{Pr 19:9}] ne restera pas i., et celui qui
\item[\vref{Pr 28:20}] hâte de s'enrichir ne demeurera pas i.
\item[\vref{Jé 49:12}] toi, tu resterais i. ! Tu ne resteras
\end{listverse}

\ConcordanceEntry{Impur}
\vspace{-2mm}
\begin{listverse}
\item[\vref{Lé 5:2}] cadavre d'un animal i., soit le cadavre
\item[\vref{Lé 10:10}] ce qui est i. et ce qui
\item[\vref{Lé 11:35}] soit, cela sera i. ; le four et
\item[\vref{Lé 11:43}] vs. rendez point i. par eux, ne
\item[\vref{Lé 13:30}] prêtre le jugera i.. C'est de la
\item[\vref{Lé 13:44}] lépreux, il est i.. Impur, le prêtre
\item[\vref{Lé 13:45}] moustache et criera : I. ! Impur !
\item[\vref{Lé 14:45}] de la ville, ds un lieu i.
\item[\vref{Lé 21:11}] se rendra point i. pour son père
\item[\vref{No 9:6}] quelques-uns qui étaient i. à cause d'un
\item[\vref{No 9:10}] votre postérité, est i. à cause d'un
\item[\vref{No 19:20}] l'hom. qui sera i., et qui ne
\item[\vref{No 19:22}] ce que l'hom. i. touchera sera souillé,
\item[\vref{De 12:15}] celui qui sera i. et celui qui
\item[\vref{De 26:14}] pour un usage i., et je n'en
\item[\vref{2 Ch 29:5}] du sanctuaire tt ce qui est i.
\item[\vref{Es 52:11}] Ne touchez rien d'i. ! Sortez du milieu
\item[\vref{Os 9:3}] mangeront en Assyrie ce qui est i.
\item[\vref{Mt 12:43}] Lorsque l'esprit i. est sorti d'un
\item[\vref{Mc 1:26}] Alors l'esprit i. sortit de cet
\item[\vref{Mc 1:27}] mm aux esprits i., et ils lui
\item[\vref{Mc 6:7}] lr. donnant pouvoir sur les esprits i.
\item[\vref{Mc 9:25}] reprit sévèrement l'esprit i., et lui dit :
\item[\vref{Lu 4:33}] esprit de démon i., et qui s'écria
\item[\vref{Lu 8:29}] commandait à l'esprit i. de sortir de
\item[\vref{Lu 11:24}] Quand l'esprit i. est sorti d'un
\item[\vref{Ac 5:16}] par des esprits i. ; et ts étaient
\item[\vref{Ac 10:28}] aucun hom. être i. ou souillé.
\item[\vref{Ac 11:8}] de souillé ni d'i. n'est entré ds
\item[\vref{1 Co 7:14}] vos enfants seraient i., or mntnt ils
\item[\vref{Ep 5:5}] qu'aucun fornicateur, ni i., ni cupide, qui
\item[\vref{1 Th 2:3}] séduction, ni motif i., ni fraude.
\item[\vref{Ap 18:2}] de tt esprit i., et le repaire
\end{listverse}

\ConcordanceEntry{Impureté}
\vspace{-2mm}
\begin{listverse}
\item[\vref{Lé 5:3}] aura touché à l'i. d'un hom., quelle
\item[\vref{Lé 7:20}] sur elle son i., cette personne-là sera
\item[\vref{Lé 15:30}] à cause du flux de son i.
\item[\vref{Lé 20:21}] frère, c'est une i. ; il a découvert
\item[\vref{Lé 22:5}] hom. atteint d'une i. quelconque, il en
\item[\vref{No 19:13}] sur lui, son i. demeure encore sur
\item[\vref{2 Ch 29:16}] Yahweh ttes les i. qu'ils trouvèrent ds
\item[\vref{Esd 6:21}] s'étaient séparés de l'i. des nations du
\item[\vref{Ez 24:13}] L'i. est ds ta souillure ; car je
\item[\vref{Ez 36:17}] la souillure d'une fem. pendant son i. ;
\item[\vref{Za 13:2}] pays les faux prophètes et l'esprit d'i.
\item[\vref{Ro 6:19}] pour servir à l'i. et à l'iniquité,
\item[\vref{2 Co 12:21}] pas repentis de l'i., de la fornication
\item[\vref{Ga 5:19}] l'adultère, la fornication, l'i., l'impudicité,
\item[\vref{Ep 4:19}] commettre tte sorte d'i. avec cupidité.
\item[\vref{Ep 5:3}] fornication, ni aucune i., ni la cupidité,
\item[\vref{Col 3:5}] terre : La fornication, l'i., les passions, les
\item[\vref{1 Th 4:7}] pas appelés à l'i., mais à la
\item[\vref{2 Pi 2:10}] la passion de l'i., et qui méprisent
\item[\vref{Jud 1:7}] que celles-ci à l'i. et qui avaient
\item[\vref{Jud 1:13}] l'écume de leurs i. ; des étoiles errantes,
\item[\vref{Ap 17:4}] des abominations de l'i. de sa prostitution.
\end{listverse}

\ConcordanceEntry{Imputer}
\vspace{-2mm}
\begin{listverse}
\item[\vref{Ge 15:6}] Yahweh qui lui i. cela à justice.
\item[\vref{Lé 17:4}] le sang sera i. à cet hom.-là ;
\item[\vref{De 24:13}] cela te sera i. à justice dvt
\item[\vref{1 S 22:15}] Que le roi n'i. aucun tort à
\item[\vref{Job 34:23}] Mais Dieu n'i. rien à l'hom.
\item[\vref{Ps 32:2}] à qui Yahweh n'i. point son iniquité,
\item[\vref{Ps 106:31}] cela lui fut i. à justice de
\item[\vref{Ac 7:60}] Seign., ne lr. i. pas ce péché !
\item[\vref{Ac 25:18}] présentés, ne lui i. aucun des crimes
\item[\vref{Ro 4:5}] foi lui est i. à justice.
\item[\vref{Ro 5:13}] péché n'est pas i., qnd il n'y
\item[\vref{1 Co 13:5}] s'irrite pas, elle n'i. pas le mal,
\item[\vref{2 Co 5:19}] en ne lr. i. pas leurs péchés,
\item[\vref{Ga 3:6}] cela lui fut i. à justice,
\item[\vref{2 Ti 4:16}] que cela ne lr. soit pas i. !
\item[\vref{Ja 2:23}] cela lui fut i. à justice, et
\end{listverse}

\ConcordanceEntry{Incirconcis}
\vspace{-2mm}
\begin{listverse}
\item[\vref{Ge 17:14}] Et le mâle i. qui n'aura pas
\item[\vref{Ex 12:48}] pays ; mais aucun i. n'en mangera.
\item[\vref{Lé 19:23}] son fruit com. i. ; il vs. sera
\item[\vref{Jos 5:7}] parce qu'ils étaient i. ; car on ne
\item[\vref{Jg 14:3}] les Philistins, ces i. ? Et Samson dit
\item[\vref{Jg 15:18}] et tomberais-je entre les mains des i. ?
\item[\vref{1 S 17:26}] ce Philistin, cet i., pour insulter l'armée
\item[\vref{2 S 1:20}] les filles des i. n'en tressaillent de
\item[\vref{1 Ch 10:4}] peur que ces i. ne viennent et
\item[\vref{Es 52:1}] vêtements magnifiques ! Car l'i. et le souillé
\item[\vref{Ez 32:19}] beauté ? Descends, et couche-toi avec les i. !
\item[\vref{Ez 44:7}] fils de l'étranger, i. de cœur et
\item[\vref{Ac 7:51}] cou raide, et i. de cœur et
\item[\vref{Ac 11:3}] chez des hommes i., et tu as
\item[\vref{Ro 2:26}] Si dc l'i. observe les ordonnances
\item[\vref{Ro 2:27}] L'i. de nature, qui accomplit la loi,
\item[\vref{Ro 3:30}] et aussi les i. par la foi.
\item[\vref{Ro 4:11}] qnd il était i., afin d'être le
\item[\vref{1 Co 7:18}] a-t-il été appelé i. ? Qu'il ne se
\item[\vref{Ga 2:7}] l'Evangile pour les i. m'avait été confiée
\item[\vref{Ep 2:11}] qui étiez appelés i. par ceux qu'on
\item[\vref{Col 3:11}] ni circoncis ni i., ni barbare ni
\end{listverse}

\ConcordanceEntry{Incliner(s')}
\vspace{-2mm}
\begin{listverse}
\item[\vref{Ge 24:48}] je me suis i., j'ai adoré Yahweh,
\item[\vref{Ex 4:31}] affliction ; et ils s'i. et se prosternèrent.
\item[\vref{Ex 12:27}] Alors le peuple s'i. et se prosterna.
\item[\vref{No 22:31}] nue ; et il s'i. et se prosterna
\item[\vref{1 R 8:58}] mais qu'il i. nos cœurs vers
\item[\vref{2 Ch 29:30}] des réjouissances, et s'i. pour se prosterner.
\item[\vref{Né 8:6}] Amen ! Et ils s'i. et se prosternèrent
\item[\vref{Est 3:2}] porte du roi s'i. et se prosternaient
\item[\vref{Ps 22:30}] ds la poussière s'i., mm celui qui
\item[\vref{Ps 31:3}] I. ton oreille vers moi, hâte-toi de
\item[\vref{Ps 95:6}] Venez, prosternons-ns., i.-ns., et mettons-ns.
\item[\vref{Ps 119:36}] I. mon cœur à tes préceptes et
\item[\vref{Ps 119:112}] J'ai i. mon cœur à accomplir toujours tes
\item[\vref{Ps 141:4}] N'i. point mon cœur à des choses
\item[\vref{Pr 2:2}] et que tu i. ton cœur à
\item[\vref{Pr 4:20}] à mes paroles, i. ton oreille à
\item[\vref{Pr 5:1}] à ma sagesse, i. ton oreille à
\item[\vref{Pr 5:13}] et n'ai-je point i. mon oreille à
\item[\vref{Pr 21:1}] ruisseaux d'eaux ; il l'i. à tt ce
\item[\vref{Es 46:1}] Bel s'i. sur ses genoux, Nebo est renversé ;
\item[\vref{Es 55:3}] I. l'oreille, et venez à moi, écoutez,
\end{listverse}

\ConcordanceEntry{Inconnu}
\vspace{-2mm}
\begin{listverse}
\item[\vref{2 R 25:16}] ces ustensiles d'airain avaient un poids i.
\item[\vref{Job 19:15}] tiennent pour un i., et me traitent
\item[\vref{Es 33:19}] peuple au langage i. qu'on n'entend pas,
\item[\vref{Ez 3:5}] peuple au langage i., ou à la
\item[\vref{Ez 3:6}] ayant un langage i. ou une langue
\item[\vref{Os 5:3}] ne m'est point i. ; car mntnt, Ephraïm,
\item[\vref{Ac 17:23}] écrit : Au dieu i. ! Celui que vs.
\item[\vref{Ga 1:22}] Or j'étais i. de visage aux
\end{listverse}

\ConcordanceEntry{Incorruptible}
\vspace{-2mm}
\begin{listverse}
\item[\vref{Ro 1:23}] gloire du Dieu i. en images représentant
\item[\vref{1 Co 9:25}] mais ns., faisons-le pour une couronne i.
\item[\vref{1 Co 15:42}] corps est semé corruptible, il ressuscitera i. ;
\item[\vref{1 Pi 1:4}] pour un héritage i., et qui ne
\item[\vref{1 Pi 1:23}] par une semence i., par la parole
\item[\vref{1 Pi 5:4}] obtiendrez la couronne i. de la gloire.
\end{listverse}

\ConcordanceEntry{Incrédule}
\vspace{-2mm}
\begin{listverse}
\item[\vref{Mt 17:17}] dit : Ô race i. et perverse, jusqu'à
\item[\vref{Mc 9:19}] dit : Ô génération i. ! Jusqu'à qnd serai-je
\item[\vref{Lu 9:41}] répondit : Ô génération i. et perverse, jusqu'à
\item[\vref{Jn 20:27}] ne sois pas i., mais fidèle.
\item[\vref{1 Co 7:12}] a une fem. i. et qu'elle consente
\item[\vref{1 Co 7:13}] a un mari i. et qu'il consente
\item[\vref{1 Co 7:14}] Car le mari i. est sanctifié par
\item[\vref{1 Co 7:15}] Que si l'i. se sépare, qu'il se sépare ; le
\item[\vref{2 Co 4:4}] pour les i. dont le dieu de ce siècle
\item[\vref{Hé 3:12}] cœur mauvais et i., au point de
\item[\vref{1 Pi 3:20}] avaient été autrefois i., qnd la patience
\item[\vref{Ap 21:8}] les timides, les i., les abominables, les
\end{listverse}

\ConcordanceEntry{Incrédulité}
\vspace{-2mm}
\begin{listverse}
\item[\vref{Mt 13:58}] de miracles, à cause de lr. i.
\item[\vref{Mt 17:20}] cause de votre i.. Je vs. le
\item[\vref{Mc 6:6}] s'étonnait de lr. i.. Il parcourait les
\item[\vref{Mc 9:24}] Je crois, Seign. ! Secours-moi ds mon i.
\item[\vref{Mc 16:14}] lr. reprocha lr. i. et lr. dureté
\item[\vref{Ro 3:3}] pas cru, lr. i. anéantira-t-elle la fidélité
\item[\vref{Ro 4:20}] de Dieu, par i. ; mais il fut
\item[\vref{Ro 11:23}] pas ds lr. i., ils seront greffés ;
\item[\vref{1 Ti 1:13}] que j'agissais par ignorance, étant ds l'i.
\item[\vref{Hé 3:19}] y entrer à cause de lr. i.
\end{listverse}

\ConcordanceEntry{Indigent}
\vspace{-2mm}
\begin{listverse}
\item[\vref{Ex 23:6}] le droit de l'i., qui est au
\item[\vref{De 15:4}] n'y ait point d'i. chez toi, car
\item[\vref{De 15:9}] envers ton frère i., afin de ne
\item[\vref{De 15:11}] aura toujours des i. ds le pays ;
\item[\vref{De 24:14}] le pauvre et l'i., d'entre tes frères,
\item[\vref{Job 24:14}] le pauvre et l'i., et la nuit
\item[\vref{Pr 14:31}] a pitié de l'i. honore Yahweh.
\item[\vref{Pr 21:17}] les réjouissances est i. ; et celui qui
\item[\vref{Jé 22:16}] pauvre et de l'i., alors il a
\item[\vref{Ac 4:34}] parmi eux aucun i. ; parce que ts
\end{listverse}

\ConcordanceEntry{Indignation}
\vspace{-2mm}
\begin{listverse}
\item[\vref{De 29:28}] avec une grande i., et il les
\item[\vref{2 R 3:27}] eut une grande i. en Israël ; ainsi
\item[\vref{Ps 85:4}] es revenu de l'ardeur de ton i.
\item[\vref{Es 34:2}] Car l'i. de Yahweh est sur ttes les
\item[\vref{Na 1:6}] subsistera dvt son i. ? Et qui peut
\item[\vref{So 3:8}] sur eux mon i., et tte l'ardeur
\item[\vref{Mc 3:5}] regardant ts avec i., et étant affligé
\item[\vref{Ro 2:8}] y aura de l'i. et de la
\item[\vref{2 Co 7:11}] Quelle justification, quelle i., quelle crainte, quel
\end{listverse}

\ConcordanceEntry{Indignement}
\vspace{-2mm}
\begin{listverse}
\item[\vref{Lu 20:11}] après l'avoir traité i., ils le renvoyèrent
\item[\vref{1 Co 11:27}] coupe du Seign. i. sera coupable envers
\item[\vref{1 Co 11:29}] qui en boit i., mange et boit
\end{listverse}

\ConcordanceEntry{Indigner}
\vspace{-2mm}
\begin{listverse}
\item[\vref{Job 1:22}] pas et n'attribua rien à Dieu d'i. de lui.
\item[\vref{Job 24:12}] mourir crie ; Dieu ne fait rien d'i. de lui.
\item[\vref{Pr 24:24}] les nations seront i. contre lui.
\item[\vref{Es 41:11}] ceux qui sont i. contre toi seront
\item[\vref{Mt 20:24}] entendu cela, furent i. contre les deux
\item[\vref{Mt 21:15}] les scribes furent i. à la vue
\item[\vref{Mt 26:8}] cela, en furent i., et dirent : A
\item[\vref{Mc 10:14}] voyant cela, fut i., et lr. dit :
\item[\vref{Ac 13:46}] vs. jugez vs.-mêmes i. de la vie
\item[\vref{1 Co 6:2}] par vs., êtes-vs. i. de rendre les
\end{listverse}

\ConcordanceEntry{Inébranlable}
\vspace{-2mm}
\begin{listverse}
\item[\vref{1 Co 15:58}] bien-aimés, soyez fermes, i., vs. appliquant toujours
\item[\vref{Hé 10:22}] et une foi i., ayant les cœurs
\item[\vref{Hé 12:27}] celles qui sont i. demeurent.
\end{listverse}

\ConcordanceEntry{Inexcusable}
\vspace{-2mm}
\begin{listverse}
\item[\vref{Ro 1:20}] ses ouvrages, de sorte qu'ils sont i. ;
\item[\vref{Ro 2:1}] autres, tu es i. ; car, en jugeant
\end{listverse}

\ConcordanceEntry{Infâme}
\vspace{-2mm}
\begin{listverse}
\item[\vref{Lé 7:18}] sera une chose i., et la personne
\item[\vref{Lé 20:17}] c'est une chose i. ; ils seront dc
\item[\vref{Jg 19:24}] pas cette action i. à l'égard de
\item[\vref{Ez 22:5}] moqueront de toi, i. de réputation, et
\item[\vref{Os 9:10}] sont consacrés à l'i. idole, et ils
\item[\vref{Lu 6:22}] votre nom com. i., à cause du
\item[\vref{Ro 1:26}] à leurs affections i., car mm les
\item[\vref{Ro 1:27}] hom. des choses i., et recevant en
\item[\vref{2 Pi 2:7}] abominables par lr. i. conduite ;
\end{listverse}

\ConcordanceEntry{Inférieur}
\vspace{-2mm}
\begin{listverse}
\item[\vref{1 R 6:6}] L'étage i. était large de cinq coudées, celui
\item[\vref{1 R 6:8}] chambres de l'étage i. était au côté
\item[\vref{Job 12:3}] vs. suis point i. ; et qui est-ce
\item[\vref{Job 13:2}] aussi ; je ne vs. suis pas i.
\item[\vref{Ps 8:6}] fait de peu i. aux anges, et
\item[\vref{Es 22:9}] vs. assemblez les eaux de l'étang i.
\item[\vref{Ez 40:18}] longueur des portes ; c'était le pavé i.
\item[\vref{Ez 41:7}] montait de l'étage i. à l'étage supérieur
\item[\vref{Ez 43:14}] sol jusqu'à l'encadrement i., il y avait
\item[\vref{2 Co 12:11}] je n'ai été i. en aucune chose
\end{listverse}

\ConcordanceEntry{Infidèle}
\vspace{-2mm}
\begin{listverse}
\item[\vref{Ex 21:8}] peuple étranger, après lui avoir été i.
\item[\vref{No 5:12}] quelqu'un se détourne et lui devienne i. ;
\item[\vref{Ps 43:1}] contre une nation i. ! Délivre-moi de l'hom.
\item[\vref{Ps 78:57}] arrière et furent i. com. leurs pères ;
\item[\vref{Jé 3:8}] j'aie répudié Israël, l'i., à cause de
\item[\vref{Jé 3:12}] dis : Reviens, Israël, l'i., dit Yahweh. Je
\item[\vref{Jé 3:20}] une fem. est i. à son compagnon,
\item[\vref{Mal 2:11}] Juda s'est montré i., et une abomination
\item[\vref{Lu 12:46}] lui donnera sa part avec les i.
\item[\vref{Lu 16:8}] maître loua l'économe i. de ce qu'il
\item[\vref{2 Co 6:15}] quelle part a le fidèle avec l'i. ?
\item[\vref{1 Ti 5:8}] foi, et il est pire qu'un i.
\item[\vref{2 Ti 2:13}] Si ns. sommes i., il demeure fidèle,
\end{listverse}

\ConcordanceEntry{Infidélité}
\vspace{-2mm}
\begin{listverse}
\item[\vref{Ex 22:9}] Dans tte affaire d'i. concernant un bœuf,
\item[\vref{Jos 22:16}] Quelle est cette i. que vs. avez
\item[\vref{Jos 22:20}] commit-il pas une i. en prenant des
\item[\vref{Jos 22:22}] rébellion et par i. envers Yahweh, alors
\item[\vref{Jos 22:31}] point commis cette i. contre Yahweh ; vs.
\item[\vref{Esd 9:4}] à cause de l'i. de ceux de
\item[\vref{Jé 5:6}] nombreuses, et leurs i. se sont renforcées.
\item[\vref{Ez 14:13}] en commettant une i., et que j'aurai
\item[\vref{Ez 15:8}] ont commis une i., dit le Seign.
\end{listverse}

\ConcordanceEntry{Infirme}
\vspace{-2mm}
\begin{listverse}
\item[\vref{Lu 13:11}] qui la rendait i. depuis dix-huit ans ;
\item[\vref{1 Co 11:30}] parmi vs. beaucoup d'i. et de malades,
\end{listverse}

\ConcordanceEntry{Infirmité}
\vspace{-2mm}
\begin{listverse}
\item[\vref{Pr 18:14}] soutiendra ds son i. ; mais l'esprit abattu,
\item[\vref{Mt 9:35}] et ttes sortes d'i. parmi le peuple.
\item[\vref{Lu 13:12}] Femme, tu es délivrée de ton i.
\item[\vref{Ro 6:19}] à cause de l'i. de votre chair.
\item[\vref{2 Co 11:30}] des choses qui sont de mon i.
\item[\vref{Ga 4:13}] à cause d'une i. de la chair
\item[\vref{Hé 5:2}] égarés, puisqu'il est aussi lui-mm enveloppé d'i.
\item[\vref{Hé 5:3}] cause de cette i., qu'il doit offrir
\end{listverse}

\ConcordanceEntry{Influence}
\vspace{-2mm}
\begin{listverse}
\item[\vref{Ga 5:8}] Cette i. ne vient pas de celui qui
\end{listverse}

\ConcordanceEntry{Ingrat}
\vspace{-2mm}
\begin{listverse}
\item[\vref{Lu 6:35}] bon envers les i. et les méchants.
\item[\vref{2 Ti 3:2}] à leurs parents, i., irréligieux,
\end{listverse}

\ConcordanceEntry{Inimitié}
\vspace{-2mm}
\begin{listverse}
\item[\vref{Ge 3:15}] Et je mettrai i. entre toi et
\item[\vref{No 35:21}] ou si par i. il le frappe
\item[\vref{No 35:22}] pousse subitement, sans i., ou s'il jette
\item[\vref{Ez 35:5}] as eu une i. immortelle, et que
\item[\vref{Ro 8:7}] la chair est i. contre Dieu, car
\item[\vref{Ep 2:15}] ds sa chair l'i., à savoir la
\item[\vref{Ep 2:16}] sa croix, ayant détruit par elle l'i.
\item[\vref{Ja 4:4}] du monde est i. contre Dieu ? Celui
\end{listverse}

\ConcordanceEntry{Inique}
\vspace{-2mm}
\begin{listverse}
\item[\vref{Ps 89:23}] surprendra point, et l'i. ne l'affligera point ;
\item[\vref{Pr 11:7}] espoir périt ; et l'espérance des hommes i. périt.
\item[\vref{Pr 29:27}] L'hom. i. est en abomination aux justes, et
\item[\vref{So 3:5}] celui qui est i. ne sait ce
\item[\vref{Lu 18:6}] Ecoutez ce que dit le juge i.
\end{listverse}

\ConcordanceEntry{Iniquité}
\vspace{-2mm}
\begin{listverse}
\item[\vref{Ge 15:16}] reviendront ici ; car l'i. des Amoréens n'est
\item[\vref{Ge 44:16}] Dieu a trouvé l'i. de tes serviteurs ;
\item[\vref{Ex 20:5}] jaloux, qui punis l'i. des pères sur
\item[\vref{Ex 28:38}] et Aaron portera l'i. commise par les
\item[\vref{Ex 34:7}] mille générations, ôtant l'i., le crime, et
\item[\vref{Lé 10:17}] donnée pour porter l'i. de l'assemblée, afin
\item[\vref{Lé 16:22}] lui ttes leurs i. ds une terre
\item[\vref{No 14:18}] bonté, il ôte l'i. et pardonne la
\item[\vref{No 23:21}] n'a point aperçu d'i. en Jacob, il
\item[\vref{1 Ch 21:8}] prie, pardonne mntnt l'i. de ton serviteur,
\item[\vref{2 Ch 19:7}] n'y a point d'i. chez Yahweh, notre
\item[\vref{Esd 9:6}] toi ; car nos i. se sont multipliées
\item[\vref{Job 10:14}] tiens pas pour innocent de mon i.
\item[\vref{Job 11:6}] moins que ton i. ne mérite.
\item[\vref{Job 33:9}] suis net, il n'y a pas d'i. en moi.
\item[\vref{Ps 18:24}] suis tenu en garde contre mon i.
\item[\vref{Ps 25:11}] me pardonneras mon i., quoiqu'elle soit grande.
\item[\vref{Ps 32:2}] n'impute point son i., et ds l'esprit
\item[\vref{Ps 32:5}] point caché mon i. ; j'ai dit : J'avouerai
\item[\vref{Ps 38:5}] Car mes i. s'élèvent au-dessus de ma tête, elles
\item[\vref{Ps 38:19}] je reconnais mon i., et je suis
\item[\vref{Ps 59:3}] Délivre-moi des ouvriers d'i. et garde-moi des
\item[\vref{Ps 66:18}] Si j'avais conçu l'i. ds mon cœur,
\item[\vref{Ps 78:38}] il pardonna lr. i., au point qu'il
\item[\vref{Ps 85:3}] Tu as pardonné l'i. de ton peuple,
\item[\vref{Ps 103:3}] pardonne ttes tes i., qui guérit ttes
\item[\vref{Ps 107:42}] réjouissent, mais tte i. a la bouche
\item[\vref{Es 5:18}] ceux qui tirent l'i. avec des cordes
\item[\vref{Es 6:7}] c'est pourquoi ton i. est ôtée, et
\item[\vref{Es 50:1}] cause de vos i., et votre mère
\item[\vref{Jé 2:22}] de savon, ton i. resterait encore marquée
\item[\vref{Jé 3:13}] Mais reconnais ton i., car tu t'es
\item[\vref{Ez 3:18}] mourra ds son i., mais je redemanderai
\item[\vref{Ez 9:9}] Il me dit : L'i. de la maison
\item[\vref{Ez 21:30}] au temps où l'i. est à son
\item[\vref{Da 9:24}] la propitiation pour l'i., pour amener la
\item[\vref{Za 5:8}] dit : C'est là l'i. ; puis il la
\item[\vref{Mt 7:23}] retirez-vs. de moi, vs. qui commettez l'i.
\item[\vref{Mt 23:28}] au-dedans vs. êtes pleins d'hypocrisie et d'i.
\item[\vref{Mt 24:12}] Et, parce que l'i. sera multipliée, la
\item[\vref{Ro 6:13}] être des instruments d'i. ; mais donnez-vs. vs.-mêmes
\item[\vref{Ro 6:19}] l'impureté et à l'i., ainsi livrez mntnt
\item[\vref{2 Co 6:14}] la justice et l'i. ? Ou quelle communion
\item[\vref{2 Th 2:7}] le mystère de l'i. opère déjà, seulement
\item[\vref{2 Ti 2:19}] Nom du Seign., qu'il s'éloigne de l'i.
\item[\vref{Tit 2:14}] racheter de tte i., et de ns.
\item[\vref{Hé 8:12}] de leurs péchés, ni de leurs i.
\item[\vref{Ja 3:6}] le monde de l'i.. La langue est
\item[\vref{1 Jn 1:9}] et pour ns. purifier de tte i.
\item[\vref{1 Jn 5:17}] Toute i. est un péché, mais il y
\end{listverse}

\ConcordanceEntry{Injure}
\vspace{-2mm}
\begin{listverse}
\item[\vref{1 S 25:7}] avons fait aucune i., et ils n'ont
\item[\vref{1 S 25:15}] ont fait aucune i., et rien ne
\item[\vref{Lu 22:65}] ils proféraient contre lui beaucoup d'autres i.
\item[\vref{2 Co 12:10}] faiblesses, ds les i., ds les nécessités,
\item[\vref{1 Pi 3:9}] pour mal, ou i. pour injure ; mais
\item[\vref{1 Pi 4:14}] vs. dit des i. pour le Nom
\end{listverse}

\ConcordanceEntry{Injurier}
\vspace{-2mm}
\begin{listverse}
\item[\vref{Ps 44:17}] et qui ns. i., et à cause
\item[\vref{Mt 27:39}] passaient par là, l'i. et secouaient la
\item[\vref{Jn 9:28}] Alors ils l'i., et lui dirent :
\end{listverse}

\ConcordanceEntry{Injuste}
\vspace{-2mm}
\begin{listverse}
\item[\vref{Job 13:7}] Tiendrez-vs. des discours i. en faveur de
\item[\vref{Job 16:11}] Dieu m'enferme chez l'i., il me fait
\item[\vref{Job 24:20}] plus de lui ; l'i. est brisé com.
\item[\vref{Job 27:4}] ne prononceront rien d'i., et ma langue
\item[\vref{Job 31:3}] n'est-elle pas pour l'i., et les accidents
\item[\vref{Es 55:7}] voie, et l'hom. i. ses pensées ; et
\item[\vref{Es 56:11}] à son gain i. ds son quartier,
\item[\vref{Lu 16:10}] celui qui est i. en très peu
\item[\vref{Ac 24:15}] une résurrection des justes et des i.
\item[\vref{Ro 3:5}] dirons-ns. ? Dieu est-il i. qnd il déchaîne
\item[\vref{1 Co 6:1}] jugement dvt les i., et il ne
\item[\vref{1 Co 6:9}] pas que les i. n'hériteront pas le
\item[\vref{Col 3:25}] celui qui agit i. recevra ce qu'il
\item[\vref{Hé 6:10}] Dieu n'est pas i., pour oublier votre
\item[\vref{1 Pi 3:18}] juste pour les i., afin de ns.
\item[\vref{2 Pi 2:9}] et réserver les i. pour être punis
\item[\vref{Ap 22:11}] celui qui est i. soit encore injuste,
\end{listverse}

\ConcordanceEntry{Injustice}
\vspace{-2mm}
\begin{listverse}
\item[\vref{1 S 24:12}] ni mal ni i. ds ma conduite,
\item[\vref{Job 34:10}] d'agir avec méchanceté, loin du Tout-Puissant l'i. !
\item[\vref{Ps 58:3}] vs. tramez des i. ds votre cœur.
\item[\vref{Ps 92:16}] Rocher, et il n'y a point d'i. en lui.
\item[\vref{Pr 22:8}] Celui qui sème l'i. moissonne le tourment,
\item[\vref{Jé 22:13}] sa maison par l'i., et ses chambres
\item[\vref{Ez 28:15}] jusqu'à celui où l'i. fut trouvée en
\item[\vref{Ez 28:18}] tes iniquités, par l'i. de ton commerce ;
\item[\vref{Jn 7:18}] véritable, et il n'y a pas d'i. en lui.
\item[\vref{Ac 25:11}] j'ai commis qq i., ou un crime
\item[\vref{Ro 1:18}] tte impiété et i. des hommes qui
\item[\vref{Ro 1:29}] de tte espèce d'i., d'impureté, de méchanceté,
\item[\vref{Ro 3:5}] Or si notre i. établit la justice
\item[\vref{Ro 9:14}] Y a-t-il de l'i. en Dieu ? A
\item[\vref{1 Co 13:6}] réjouit pas de l'i., mais elle se
\end{listverse}

\ConcordanceEntry{Innocence}
\vspace{-2mm}
\begin{listverse}
\item[\vref{1 R 8:32}] à l'innocent et traite-le selon son i. !
\item[\vref{Job 6:29}] sans injustice ; revenez, et reconnaissez mon i.
\item[\vref{Ps 17:15}] moi, ds mon i., je verrai ta
\item[\vref{Ps 25:21}] [Tav.] Que l'i. et la droiture
\item[\vref{Ps 26:6}] mes mains ds l'i. et je fais
\item[\vref{Ps 73:13}] que j'ai lavé mes mains ds l'i.:
\item[\vref{Os 8:5}] qnd ne pourront-ils pas s'adonner à l'i. ?
\end{listverse}

\ConcordanceEntry{Innocent}
\vspace{-2mm}
\begin{listverse}
\item[\vref{Ex 20:7}] tiendra point pour i. celui qui aura
\item[\vref{Ex 23:7}] feras point mourir l'i. et le juste ;
\item[\vref{Ex 34:7}] le coupable pour i., et qui punit
\item[\vref{No 14:18}] le coupable pour i., et il punit
\item[\vref{De 19:10}] que le sang i. ne soit versé
\item[\vref{De 27:25}] versant le sang i. ! Et tt le
\item[\vref{1 S 19:5}] contre le sang i. en faisant mourir
\item[\vref{2 S 3:28}] suis à jamais i., mon royaume et
\item[\vref{Job 9:28}] que tu ne me jugeras pas i.
\item[\vref{Pr 6:17}] les mains qui répandent le sang i.,
\item[\vref{Jé 26:15}] mettrez du sang i. sur vs., sur
\item[\vref{Da 6:22}] j'ai été trouvé i. dvt lui ; et
\item[\vref{Jon 1:14}] ns. le sang i. ! Car toi, Yahweh,
\item[\vref{Mt 27:24}] disant : Je suis i. du sang de
\item[\vref{Hé 7:26}] tel Grand-Prêtre, saint, i., sans tache, séparé
\end{listverse}

\ConcordanceEntry{Inquiéter}
\vspace{-2mm}
\begin{listverse}
\item[\vref{1 R 22:3}] ns. ne ns. i. pas de la
\item[\vref{Mt 6:25}] dis : Ne vs. i. pas pour votre
\item[\vref{Mt 6:28}] Et pourquoi vs. i. au sujet du
\item[\vref{Mt 6:31}] Ne vs. i. dc pas, en disant : Que mangerons-ns. ?
\item[\vref{Mt 6:34}] Ne vs. i. dc pas pour le lendemain ; car
\item[\vref{Mt 10:19}] livreront, ne vs. i. pas de ce
\item[\vref{Mt 22:16}] la vérité, sans t'i. de personne ; car
\item[\vref{Lu 10:41}] Marthe, Marthe, tu t'i. et tu t'agites
\item[\vref{Ph 4:6}] Ne vs. i. de rien, mais en ttes choses
\item[\vref{1 Pi 5:7}] qui peut vs. i., car il prend
\end{listverse}

\ConcordanceEntry{Inquiétude}
\vspace{-2mm}
\begin{listverse}
\item[\vref{Ez 12:18}] ton eau en tremblant et avec i.
\item[\vref{Mt 6:27}] puisse, par ses i., ajouter une coudée
\item[\vref{1 Co 7:32}] vs. soyez sans i.. Celui qui n'est
\item[\vref{Ga 4:20}] ds une grande i. à votre sujet.
\item[\vref{1 Th 3:5}] plus soutenir cette i., j'ai envoyé Timothée
\end{listverse}

\ConcordanceEntry{Insatiable}
\vspace{-2mm}
\begin{listverse}
\item[\vref{Pr 30:15}] choses qui sont i., il y en
\item[\vref{Ha 2:5}] et il est i. com. la mort,
\end{listverse}

\ConcordanceEntry{Inscription}
\vspace{-2mm}
\begin{listverse}
\item[\vref{Mt 22:20}] demanda : De qui porte-t-il l'image et l'i. ?
\item[\vref{Mc 12:16}] porte-t-il l'image et l'i. ? De César, lui
\item[\vref{Lu 20:24}] a-t-il l'image et l'i. ? Ils lui répondirent :
\end{listverse}

\ConcordanceEntry{Insensé}
\vspace{-2mm}
\begin{listverse}
\item[\vref{Ge 31:28}] filles ! C'est en i. que tu as
\item[\vref{De 32:6}] récompenses Yahweh, peuple i. et dépourvu de
\item[\vref{1 S 26:21}] j'ai agi en i., et j'ai commis
\item[\vref{2 S 3:33}] Abner devait-il mourir com. meurt un i. ?
\item[\vref{Job 2:10}] com. une fem. i. ! Nous recevons de
\item[\vref{Job 5:2}] la colère tue l'i., et le dépit
\item[\vref{Ps 14:1}] chef des chantres. L'i. dit en son
\item[\vref{Pr 3:35}] gloire ; mais la honte élève des i.
\item[\vref{Pr 10:8}] celui qui est i. des lèvres tombera.
\item[\vref{Pr 10:23}] jeu pour un i. de pratiquer la
\item[\vref{Pr 12:16}] Quand à l'i., sa colère est
\item[\vref{Pr 14:9}] Les i. se moquent du péché, mais parmi
\item[\vref{Pr 15:5}] L'i. méprise l'instruction de son père, mais
\item[\vref{Pr 26:4}] réponds pas à l'i. selon sa folie,
\item[\vref{Pr 26:5}] Réponds à l'i. selon sa folie,
\item[\vref{Pr 26:11}] a vomi, ainsi l'i. réitère sa folie.
\item[\vref{Ec 2:14}] sa tête, et l'i. marche ds les
\item[\vref{Ec 5:3}] de plaisir aux i. ; accomplis dc le
\item[\vref{Ec 7:17}] ne sois point i. : Pourquoi mourrais-tu avant
\item[\vref{Jé 4:22}] mon peuple est i. ; ils ne m'ont
\item[\vref{Jé 17:11}] à la fin il sera trouvé i.
\item[\vref{Za 11:15}] dit : Prends-toi encore l'équipage d'un pasteur i.
\item[\vref{Mt 5:22}] qui lui dira : I. ! sera puni par
\item[\vref{Mt 7:26}] à un hom. i. qui a bâti
\item[\vref{Mt 23:17}] I. et aveugles ! Car lequel est le
\item[\vref{Lu 12:20}] Dieu lui dit : I. ! Cette mm nuit
\item[\vref{1 Co 15:36}] Ô i. ! Ce que tu sèmes ne reprend
\item[\vref{2 Co 11:16}] regarde com. un i. ; ou bien, supportez-moi
\item[\vref{2 Co 11:21}] je parle en i., j'ai la mm
\item[\vref{2 Co 12:11}] J'ai été i. en me glorifiant, mais vs. m'y
\item[\vref{Tit 3:3}] ns. étions autrefois i., désobéissants, égarés, asservis
\item[\vref{1 Pi 2:15}] la bouche à l'ignorance des hommes i. ;
\end{listverse}

\ConcordanceEntry{Insensible}
\vspace{-2mm}
\begin{listverse}
\item[\vref{Ps 119:70}] Leur cœur est i. com. la graisse,
\item[\vref{Ac 28:27}] peuple est devenu i. ; ils ont endurci
\end{listverse}

\ConcordanceEntry{Insister}
\vspace{-2mm}
\begin{listverse}
\item[\vref{2 R 5:23}] deux talents. Il i., puis il serra
\item[\vref{Lu 23:5}] Mais ils i. encore davantage, et dirent : Il soulève
\item[\vref{Lu 23:23}] Mais ils i. à grands cris, demandant qu'il soit
\item[\vref{Ac 21:14}] pas persuader, ns. n'i. pas, et ns.
\item[\vref{2 Ti 4:2}] prêche la parole, i. en tte occasion,
\end{listverse}

\ConcordanceEntry{Insondable}
\vspace{-2mm}
\begin{listverse}
\item[\vref{Job 36:26}] nombre de ses années, il est i.
\item[\vref{Ro 11:33}] ses jugements sont i., et ses voies
\end{listverse}

\ConcordanceEntry{Inspirer}
\vspace{-2mm}
\begin{listverse}
\item[\vref{Ps 10:18}] la terre cesse d'i. l'effroi.
\item[\vref{2 Ti 3:16}] Toute l'Ecriture est i. de Dieu et
\end{listverse}

\ConcordanceEntry{Instruction}
\vspace{-2mm}
\begin{listverse}
\item[\vref{Ps 2:10}] l'intelligence ! Juges de la terre, recevez i. !
\item[\vref{Pr 1:2}] la sagesse et l'i., pour discerner les
\item[\vref{Pr 1:7}] les fous méprisent la sagesse et l'i.
\item[\vref{Pr 1:8}] Mon fils, écoute l'i. de ton père,
\item[\vref{Pr 4:13}] Embrasse l'i., ne la lâche
\item[\vref{Pr 5:23}] Il mourra faute d'i. et il s'égarera
\item[\vref{Pr 8:10}] Recevez mon i. plutôt que de
\item[\vref{Pr 9:9}] Donne l'i. au sage et il deviendra encore
\item[\vref{Pr 13:1}] fils sage écoute l'i. de son père,
\item[\vref{Pr 15:5}] L'insensé méprise l'i. de son père,
\item[\vref{Pr 23:12}] ton cœur à l'i., et tes oreilles
\item[\vref{Pr 23:23}] achète la sagesse, l'i. et l'intelligence.
\item[\vref{Jé 5:3}] refusent de recevoir l'i. ; ils endurcissent leurs
\item[\vref{Jé 6:8}] Jérus., reçois l'i., de peur que
\item[\vref{So 3:7}] craindras, tu recevras i., et sa demeure
\item[\vref{Ac 4:13}] des hommes sans i. et du commun
\item[\vref{Ro 15:4}] écrites pour notre i., afin que, par
\item[\vref{1 Co 10:11}] écrites pour notre i., com. étant ceux
\item[\vref{1 Co 14:26}] psaume, ou une i., ou une langue,
\item[\vref{2 Ti 2:23}] qui sont sans i., sachant qu'elles ne
\end{listverse}

\ConcordanceEntry{Instruire}
\vspace{-2mm}
\begin{listverse}
\item[\vref{De 4:36}] des cieux pour t'i. ; et il t'a
\item[\vref{Ps 19:3}] Un jour en i. un autre jour,
\item[\vref{Ps 94:12}] Yahweh ! que tu i. par ta loi,
\item[\vref{Ps 105:22}] désirs, et pour i. ses anciens.
\item[\vref{Pr 6:23}] réprimandes propres à i. sont le chemin
\item[\vref{Pr 22:6}] I. le jeune enfant à l'entrée de
\item[\vref{Es 40:14}] et qui l'a i., et lui a
\item[\vref{Mt 13:52}] pourquoi, tt scribe i. de ce qui
\item[\vref{1 Co 2:16}] Seign. pour pouvoir l'i. ? Mais ns., ns.
\item[\vref{1 Co 14:19}] être entendu, afin d'i. aussi les autres,
\item[\vref{2 Ti 3:16}] corriger et pour i. selon la justice,
\end{listverse}

\ConcordanceEntry{Insulte}
\vspace{-2mm}
\begin{listverse}
\item[\vref{Né 4:4}] Fais retourner leurs i. sur lr. tête,
\item[\vref{Jé 20:10}] j'ai entendu les i. de plusieurs, la
\item[\vref{Ez 36:3}] langue et aux i. des nations,
\item[\vref{So 2:8}] J'ai entendu les i. de Moab, et
\item[\vref{So 2:10}] qu'ils ont usé d'i. et d'arrogance, contre
\item[\vref{Ac 23:4}] lui dirent : Tu i. le grand-prêtre de
\end{listverse}

\ConcordanceEntry{Insulter}
\vspace{-2mm}
\begin{listverse}
\item[\vref{1 S 17:26}] cet incirconcis, pour i. l'armée du Dieu
\item[\vref{Ac 23:4}] lui dirent : Tu i. le grand-prêtre de
\end{listverse}

\ConcordanceEntry{Intègre}
\vspace{-2mm}
\begin{listverse}
\item[\vref{Ge 6:9}] hom. juste et i. en son temps ;
\item[\vref{Ge 17:1}] Marche dvt ma face, et sois i.
\item[\vref{Ge 25:27}] fut un hom. i., se tenant ds
\item[\vref{De 18:13}] Tu seras i. avec Yahweh, ton Dieu.
\item[\vref{2 S 22:26}] bon, avec l'hom. i. tu es intègre,
\item[\vref{1 R 8:61}] votre cœur soit i. envers Yahweh, notre
\item[\vref{1 R 11:4}] ne fut point i. dvt Yahweh, son
\item[\vref{1 R 15:14}] cœur d'Asa fut i. envers Yahweh pendant
\item[\vref{Job 1:1}] Cet hom. était i. et droit, craignant
\item[\vref{Job 1:8}] la terre ; hom. i. et droit, craignant
\item[\vref{Job 8:16}] Mais l'hom. i. est plein de
\item[\vref{Ps 119:1}] ceux qui sont i. ds lr. voie,
\item[\vref{Ps 119:80}] mon cœur soit i. ds tes statuts
\item[\vref{Pr 13:6}] celui qui est i. ds sa voie,
\item[\vref{Pr 28:10}] ceux qui sont i. héritent le bonheur.
\end{listverse}

\ConcordanceEntry{Intégrité}
\vspace{-2mm}
\begin{listverse}
\item[\vref{Ge 20:5}] fait ceci ds l'i. de mon cœur
\item[\vref{Ge 20:6}] l'as fait ds l'i. de ton cœur,
\item[\vref{Jos 24:14}] et servez-le avec i. et avec fidélité.
\item[\vref{Jg 9:19}] vérité et avec i. envers Jerubbaal et
\item[\vref{1 R 9:4}] a marché, avec i. de cœur et
\item[\vref{2 R 20:3}] avec fidélité et i. de cœur, et
\item[\vref{Job 2:9}] Persévéreras-tu encore ton i. ? Bénis Dieu, et
\item[\vref{Job 4:6}] ton espérance ? Et l'i. de tes voies
\item[\vref{Job 31:6}] balances justes, et Dieu connaîtra mon i.
\item[\vref{Ps 26:1}] marche ds mon i., je me confie
\item[\vref{Ps 41:13}] cause de mon i., et tu m'as
\item[\vref{Ps 78:72}] les dirigea selon l'i. de son cœur,
\item[\vref{Ps 84:12}] bien à ceux qui marchent ds l'i.
\item[\vref{Ps 101:2}] je marcherai ds l'i. de mon cœur
\item[\vref{Pr 2:7}] bouclier de ceux qui marchent ds l'i.,
\item[\vref{Pr 10:9}] qui marche ds l'i. marche avec assurance,
\item[\vref{Pr 11:3}] L'i. des hommes droits les conduit, mais
\item[\vref{Pr 20:7}] marchent ds son i. seront heureux après
\item[\vref{Pr 28:18}] qui marche ds l'i. est sauvé, mais
\item[\vref{Es 38:3}] vérité et en i. de cœur, et
\item[\vref{Am 5:10}] en abomination celui qui parle en i.
\item[\vref{Tit 2:7}] de tte altération, en pureté, en i.,
\end{listverse}

\ConcordanceEntry{Intelligence}
\vspace{-2mm}
\begin{listverse}
\item[\vref{De 4:6}] sagesse et votre i. aux yeux de
\item[\vref{De 32:28}] il n'y a en eux aucune i.
\item[\vref{1 R 4:29}] une très grande i., et des connaissances
\item[\vref{Job 12:13}] à lui appartient le conseil et l'i.
\item[\vref{Job 17:4}] à lr. cœur l'i., c'est pourquoi tu
\item[\vref{Job 28:20}] sagesse ? Où est la demeure de l'i. ?
\item[\vref{Job 28:28}] et se détourner du mal c'est l'i.
\item[\vref{Ps 49:21}] qui n'a pas d'i., est semblable aux
\item[\vref{Ps 119:34}] Donne-moi de l'i. ; je garderai ta
\item[\vref{Ps 119:130}] elle donne de l'i. aux simples.
\item[\vref{Ps 119:144}] justice éternelle ; donne-moi l'i. afin que je
\item[\vref{Ps 147:5}] sa force, son i. n'a point de
\item[\vref{Pr 2:6}] sa bouche procèdent la connaissance et l'i.
\item[\vref{Pr 2:11}] sur toi, et l'i. te gardera,
\item[\vref{Pr 3:19}] il a disposé les cieux par l'i.
\item[\vref{Pr 9:10}] et la connaissance des saints, c'est l'i.
\item[\vref{Pr 24:3}] par la sagesse, et affermie par l'i. ;
\item[\vref{Es 11:2}] de sagesse et d'i., Esprit de conseil
\item[\vref{Es 40:28}] a pas moyen de sonder son i.
\item[\vref{Es 44:19}] la connaissance ni l'i. pour dire : J'en
\item[\vref{Jé 9:24}] glorifie d'avoir de l'i. et de me
\item[\vref{Da 1:4}] les sciences, pleins d'i., et capables de
\item[\vref{Da 1:17}] science et de l'i. ds ttes les
\item[\vref{Da 1:20}] sagesse et de l'i., et sur lesquelles
\item[\vref{Da 5:12}] connaissance et de l'i., pour interpréter les
\item[\vref{Da 5:14}] une lumière, une i. et une sagesse
\item[\vref{Da 9:22}] suis venu mntnt pour ouvrir ton i.
\item[\vref{Da 10:1}] et il eut l'i. de la vision.
\item[\vref{Mt 15:16}] dit : Vous aussi, êtes-vs. encore sans i. ?
\item[\vref{Mc 12:33}] de tte son i., de tte son
\item[\vref{Ro 1:31}] sans i., ne tenant pas ce qu'ils ont
\item[\vref{Ro 3:11}] qui ait de l'i., il n'y a
\item[\vref{1 Co 1:19}] sages et j'anéantirai l'i. des hommes intelligents.
\item[\vref{1 Co 14:14}] en prière, mais l'i. que j'en ai,
\item[\vref{Ep 4:18}] Ils ont l'i. obscurcie par les
\item[\vref{Ph 1:9}] en plus avec connaissance et tte i.,
\item[\vref{Ph 4:7}] qui surpasse tte i., gardera vos cœurs
\item[\vref{Col 1:9}] tte sagesse et i. spirituelle,
\item[\vref{Col 2:2}] enrichis d'une pleine i., pour la connaissance
\item[\vref{2 Ti 2:7}] te donne de l'i. en ttes choses.
\item[\vref{1 Jn 5:20}] ns. a donné l'i. pour connaître le
\item[\vref{Ap 13:18}] qui a de l'i. compte le nombre
\end{listverse}

\ConcordanceEntry{Intelligent}
\vspace{-2mm}
\begin{listverse}
\item[\vref{Ge 41:33}] choisisse un hom. i. et sage, et
\item[\vref{1 R 3:9}] serviteur un cœur i. pour juger ton
\item[\vref{1 R 3:12}] cœur sage et i., de sorte qu'il
\item[\vref{1 R 7:14}] cuivre ; fort expert, i. et savant pour
\item[\vref{1 Ch 27:32}] conseiller, hom. très i. et scribe ; et
\item[\vref{2 Ch 2:13}] hom. habile et i., Huram-Abi,
\item[\vref{Esd 8:18}] ns., un hom. i., d'entre les fils
\item[\vref{Job 11:12}] de sens devient i., quoique l'hom. naisse
\item[\vref{Job 32:8}] l'inspiration du Tout-Puissant qui le rend i. ;
\item[\vref{Ps 14:2}] quelqu'un qui soit i., qui cherche Dieu.
\item[\vref{Ps 119:104}] Je suis devenu i. par tes commandements,
\item[\vref{Pr 1:5}] savoir, et l'hom. i. acquerra de la
\item[\vref{Pr 8:9}] clairs à l'hom. i., et droits pour
\item[\vref{Pr 14:33}] cœur de l'hom. i., et elle est
\item[\vref{Pr 18:15}] cœur de l'hom. i. acquiert la connaissance,
\item[\vref{Pr 19:25}] tu reprends l'hom. i., il discernera ce
\item[\vref{Pr 20:5}] profondes, et l'hom. i. sait y puiser.
\item[\vref{Pr 28:11}] pauvre qui est i. le sondera.
\item[\vref{Ec 9:11}] ceux qui sont i., ni la grâce
\item[\vref{Es 10:13}] car je suis i. ; j'ai reculé les
\item[\vref{Da 12:3}] qui auront été i., brilleront com. la
\item[\vref{Da 12:10}] comprendra, mais les i. comprendront.
\item[\vref{Ac 13:7}] Sergius Paulus, hom. i., qui fit appeler
\item[\vref{Ja 3:13}] hom. sage et i. ? Qu'il fasse voir
\item[\vref{Ap 17:9}] faut un esprit i. et qui ait
\end{listverse}

\ConcordanceEntry{Intercéder}
\vspace{-2mm}
\begin{listverse}
\item[\vref{Es 53:12}] et qu'il a i. pour les transgresseurs.
\item[\vref{Jé 7:16}] ni prière, et n'i. pas auprès de
\item[\vref{Jé 42:2}] favorable dvt toi ! I. auprès de Yahweh,
\item[\vref{Ro 8:27}] l'Esprit, car il i. en faveur des
\item[\vref{Hé 7:25}] toujours vivant pour i. pour eux.
\end{listverse}

\ConcordanceEntry{Interdit}
\vspace{-2mm}
\begin{listverse}
\item[\vref{Ex 22:20}] sera dévoué à la façon de l'i.
\item[\vref{Lé 27:28}] la façon de l'i. à Yahweh, de
\item[\vref{De 3:6}] le moyen de l'i., com. ns. l'avions
\item[\vref{De 7:26}] chose, dévoué par i. ; tu la détesteras
\item[\vref{De 20:17}] les dévouer par i. : Héthiens, Amoréens, Cananéens,
\item[\vref{Jos 6:18}] soit gardez-vs. de l'i., de peur que
\item[\vref{Jos 7:1}] au sujet de l'i.. Car Acan, fils
\item[\vref{Jos 7:12}] sont devenus un i.. Je ne serai
\item[\vref{Jos 7:15}] été saisi avec l'i. sera brûlé au
\item[\vref{1 S 15:8}] le dévouant par i. ; mais il épargna
\item[\vref{1 R 9:21}] le moyen de l'i., Salomon les fit
\item[\vref{1 Ch 4:41}] la façon de l'i. jusqu'à ce jour,
\item[\vref{2 Ch 20:23}] les dévouer par i. et les exterminer ;
\item[\vref{Es 34:5}] le peuple que j'ai voué à l'i.
\item[\vref{Jé 25:9}] la façon de l'i., je les mettrai
\item[\vref{Ez 44:29}] culpabilité ; et tt i. en Israël lr.
\item[\vref{Mi 4:13}] tu consacreras par i. leurs profits à
\item[\vref{Mal 4:6}] la terre à la façon de l'i.
\end{listverse}

\ConcordanceEntry{Intérêt}
\vspace{-2mm}
\begin{listverse}
\item[\vref{Lé 25:36}] lui d'usure ni d'i., mais tu craindras
\item[\vref{De 23:19}] Tu n'exigeras aucun i. à ton frère,
\item[\vref{De 23:20}] Tu prêteras avec i. à l'étranger, mais
\item[\vref{Ps 15:5}] qui n'exige point d'i. de son argent ;
\item[\vref{Pr 28:8}] ses biens par l'i. et l'usure, les
\item[\vref{Ez 18:13}] s'il prête à i., et tire une
\item[\vref{Ez 18:17}] ni usure ni i., s'il garde mes
\item[\vref{Mt 25:27}] mon retour, je l'aurais retiré avec l'i.
\item[\vref{1 Co 10:24}] cherche son propre i., mais que chacun
\item[\vref{1 Co 13:5}] cherche pas son i., elle ne s'irrite
\item[\vref{Ph 2:4}] chacun, à votre i. particulier, mais que
\item[\vref{Ph 2:21}] ts cherchent lr. i. particulier, et non
\end{listverse}

\ConcordanceEntry{Intérieur}
\vspace{-2mm}
\begin{listverse}
\item[\vref{Ge 41:48}] villes, mettant ds l'i. de chaque ville
\item[\vref{Ex 28:26}] sera du côté de l'éphod à l'i.
\item[\vref{Lé 10:18}] été porté à l'i. du sanctuaire ; vs.
\item[\vref{Lé 14:41}] la maison à l'i., tt autour ; et
\item[\vref{De 21:12}] la conduiras à l'i. de ta maison,
\item[\vref{1 R 22:35}] sang de sa blessure coulait à l'i. du char.
\item[\vref{Ps 39:4}] moi, un feu i. me consumait, et
\item[\vref{Mt 23:26}] aveugle ! nettoie premièrement l'i. de la coupe
\item[\vref{Mc 15:16}] emmenèrent Jésus ds l'i. de la cour,
\item[\vref{Lu 11:7}] et si, de l'i. de sa maison,
\item[\vref{Lu 11:39}] plat, et à l'i. vs. êtes pleins
\item[\vref{Ro 7:22}] loi de Dieu, quant à l'hom. i. ;
\item[\vref{2 Co 4:16}] se détruise, toutefois l'i. est renouvelé de
\item[\vref{Ep 3:16}] fortifiés par son Esprit ds l'hom. i.,
\end{listverse}

\ConcordanceEntry{Interprétation}
\vspace{-2mm}
\begin{listverse}
\item[\vref{Ge 40:12}] dit : Voici son i. : Les trois sarments
\item[\vref{Ge 41:11}] ns. reçut une i. en rapport avec
\item[\vref{Jg 7:15}] songe et son i., il se prosterna,
\item[\vref{Da 2:6}] songe et son i., vs. recevrez de
\item[\vref{Da 2:25}] donnera au roi l'i. de son songe.
\item[\vref{Da 2:26}] songe que j'ai eu et son i. ?
\item[\vref{Da 2:30}] donner au roi l'i. de son songe
\item[\vref{Da 2:45}] véritable et son i. est certaine.
\item[\vref{1 Co 12:10}] autre, le don d'i. les langues.
\item[\vref{1 Co 14:26}] révélation, ou une i., que tt se
\item[\vref{2 Pi 1:20}] ne procède d'une i. particulière,
\end{listverse}

\ConcordanceEntry{Interroger}
\vspace{-2mm}
\begin{listverse}
\item[\vref{Jg 4:20}] si l'on vient t'i., en disant : Y
\item[\vref{Job 8:8}] je te prie, i. les générations précédentes,
\item[\vref{Mc 9:32}] ce discours, et ils craignaient de l'i.
\item[\vref{Mc 12:34}] de Dieu. Et personne n'osait plus l'i.
\item[\vref{Mc 14:60}] levant au milieu, i. Jésus, disant : Ne
\item[\vref{Lu 9:45}] ils craignaient de l'i. à ce sujet.
\item[\vref{Lu 23:14}] voici, je l'ai i. dvt vs., et
\item[\vref{Jn 1:19}] des Lévites pour l'i., et lui dire :
\item[\vref{Jn 8:7}] ils continuaient à l'i., s'étant relevé, il
\item[\vref{Jn 9:21}] a de l'âge, i.-le, il parlera
\item[\vref{Jn 16:19}] sachant qu'ils voulaient l'i., lr. dit : Vous
\item[\vref{Jn 16:23}] jour-là, vs. ne m'i. plus sur rien.
\item[\vref{Jn 18:21}] Pourquoi m'i.-tu ? Interroge ceux
\item[\vref{1 Co 14:35}] qq chose, qu'elles i. leurs maris à
\end{listverse}

\ConcordanceEntry{Intime}
\vspace{-2mm}
\begin{listverse}
\item[\vref{Ge 38:12}] Thimna, avec Hira, l'Adullamite, son ami i.
\item[\vref{Ex 33:11}] parle avec son i. ami. Puis Moïse
\item[\vref{De 13:6}] bien-aimée, ou ton i. ami, qui est
\item[\vref{Job 16:21}] com. un hom. avec son ami i. !
\item[\vref{Ps 88:19}] moi mon ami i. et mon compagnon ;
\item[\vref{Pr 6:3}] mains de ton i. ami, va, prosterne-toi,
\item[\vref{Pr 17:17}] L'ami i. aime en tt temps, et il
\item[\vref{Jé 9:5}] de son ami i., et on ne
\item[\vref{Mi 7:5}] à ton ami i., et ne te
\end{listverse}

\ConcordanceEntry{Inutile}
\vspace{-2mm}
\begin{listverse}
\item[\vref{Za 11:17}] Malheur au pasteur i. qui abandonne les
\item[\vref{Mt 25:30}] dc le serviteur i. ds les ténèbres
\item[\vref{Lu 7:30}] dessein de Dieu i. à lr. égard.
\item[\vref{Lu 17:10}] sommes des serviteurs i. ; et ce que
\item[\vref{1 Co 15:58}] ne sera pas i. ds le Seign.
\item[\vref{1 Th 3:5}] que notre travail ne soit devenu i.
\item[\vref{Phm 1:11}] t'a été autrefois i., mais qui mntnt
\end{listverse}

\ConcordanceEntry{Invisible}
\vspace{-2mm}
\begin{listverse}
\item[\vref{2 Co 4:18}] visibles, mais aux i. ; car les choses
\item[\vref{Col 1:15}] l'image du Dieu i., le premier-né de
\item[\vref{Col 1:16}] visibles et les i., soit les trônes,
\item[\vref{1 Ti 1:17}] des siècles, immortel, i., à Dieu seul
\item[\vref{Hé 11:27}] ferme, com. voyant celui qui est i.
\end{listverse}

\ConcordanceEntry{Inviter}
\vspace{-2mm}
\begin{listverse}
\item[\vref{Ge 31:54}] la montagne et i. ses frères pour
\item[\vref{Ex 34:15}] dieux, quelqu'un ne t'i. et que tu
\item[\vref{De 20:10}] la guerre, tu l'i. à la paix.
\item[\vref{1 S 16:3}] Et tu i. Isaï au sacrifice, et je te
\item[\vref{1 S 16:5}] fils, et les i. au sacrifice.
\item[\vref{2 S 11:13}] David l'i. à manger et à boire en
\item[\vref{2 S 13:23}] Baal-Hatsor, près d'Ephraïm, i. ts les fils
\item[\vref{2 S 15:11}] qui avaient été i., s'en allèrent avec
\item[\vref{1 R 1:9}] auprès d'En-Roguel ; il i. ts ses frères,
\item[\vref{1 R 1:41}] et ts les i. qui étaient avec
\item[\vref{1 R 1:49}] Alors, ts les i. d'Adonija furent saisis
\item[\vref{Est 3:14}] chaque province, et i. publiquement ts les
\item[\vref{So 1:7}] sacrifice, il a i. ses conviés.
\item[\vref{Lu 7:39}] pharisien qui l'avait i., voyant cela, dit
\item[\vref{Lu 14:12}] ou un souper, n'i. pas tes amis,
\item[\vref{Ac 10:24}] attendait, et avait i. ses parents et
\item[\vref{1 Co 10:27}] des infidèles vs. i. et que vs.
\end{listverse}

\ConcordanceEntry{Involontairement}
\vspace{-2mm}
\begin{listverse}
\item[\vref{Lé 4:2}] personne aura péché i. contre l'un des
\item[\vref{Lé 4:13}] d'Israël a péché i., et que la
\item[\vref{Lé 4:27}] pays a péché i., en violant l'un
\item[\vref{No 15:27}] qui a péché i., elle offrira une
\item[\vref{No 35:11}] à mort quelqu'un i., s'y enfuie.
\item[\vref{De 4:42}] tué son prochain i., sans l'avoir haï
\item[\vref{De 19:4}] frappé son prochain i., et sans l'avoir
\item[\vref{Jos 20:3}] aura tué quelqu'un i. sans y penser,
\item[\vref{Ez 45:20}] hommes qui pèchent i. et à cause
\end{listverse}

\ConcordanceEntry{Invoquer}
\vspace{-2mm}
\begin{listverse}
\item[\vref{Ge 4:26}] l'on commença à i. le Nom de
\item[\vref{Ge 12:8}] à Yahweh, et i. le Nom de
\item[\vref{Jg 15:18}] extrêmement soif, et i. Yahweh, en disant :
\item[\vref{2 S 22:7}] ma détresse, j'ai i. Yahweh, j'ai crié
\item[\vref{1 R 18:24}] Puis i. le nom de vos dieux, et
\item[\vref{1 Ch 4:10}] Jaebets i. le Dieu d'Israël, en disant : Ô,
\item[\vref{1 Ch 21:30}] cet autel pour i. Dieu, parce qu'il
\item[\vref{2 Ch 7:14}] mon Nom est i., s'humilie, prie, et
\item[\vref{Esd 4:2}] vs. ; car ns. i. votre Dieu com.
\item[\vref{Job 21:15}] Et quel bien ns. reviendra-t-il de l'i. ?
\item[\vref{Ps 17:6}] Je t'i., car tu m'exauces, ô Dieu ! Incline
\item[\vref{Ps 24:6}] sont ceux qui l'i., ceux qui cherchent
\item[\vref{Ps 50:15}] I.-moi au jour de ta détresse,
\item[\vref{Ps 91:15}] Il m'i. et je l'exaucerai ; je serai avec
\item[\vref{Ps 118:5}] la détresse, j'ai i. Yahweh, et Yahweh
\item[\vref{Es 12:4}] jour-là : Louez Yahweh, i. son Nom, publiez
\item[\vref{Es 55:6}] qu'il se trouve, i.-le tandis qu'il
\item[\vref{Es 62:7}] arrêtez pas de l'i. jusqu'à ce qu'il
\item[\vref{Es 64:6}] a personne qui i. ton Nom, qui
\item[\vref{Joë 2:32}] arrivera que quiconque i. le Nom de
\item[\vref{Am 9:12}] nom a été i., dit Yahweh, qui
\item[\vref{So 3:9}] pures, afin qu'elles i. ttes le Nom
\item[\vref{Ac 2:21}] arrivera que quiconque i. le Nom du
\item[\vref{Ac 19:13}] Juifs ambulants essayèrent d'i. le Nom du
\item[\vref{1 Co 1:2}] que ce soit i. le Nom de
\item[\vref{2 Ti 2:22}] avec ceux qui i. le Seign. d'un
\item[\vref{1 Pi 1:17}] Et si vs. i. com. votre Père
\end{listverse}

\ConcordanceEntry{Iota}
\vspace{-2mm}
\begin{listverse}
\item[\vref{Mt 5:18}] loi un seul i. ou un seul
\end{listverse}

\ConcordanceEntry{Irrépréhensible}
\vspace{-2mm}
\begin{listverse}
\item[\vref{1 Co 1:8}] que vs. soyez i. au jour de
\item[\vref{Ep 1:4}] soyons saints et i. dvt lui ds
\item[\vref{Ph 2:15}] enfants de Dieu i. au milieu de
\item[\vref{1 Ti 3:2}] que l'évêque soit i., mari d'une seule
\item[\vref{1 Ti 6:14}] sans tache et i., jusqu'à l'apparition de
\item[\vref{Tit 1:6}] hom. qui soit i., mari d'une seule
\end{listverse}

\ConcordanceEntry{Irréprochable}
\vspace{-2mm}
\begin{listverse}
\item[\vref{Ep 5:27}] rien de semblable, mais sainte et i.
\item[\vref{1 Th 2:10}] croyez a été sainte, juste, et i.
\item[\vref{Hé 8:7}] Alliance avait été i., il n'y aurait
\item[\vref{Jud 1:24}] dvt sa gloire, i. ds l'allégresse,
\end{listverse}

\ConcordanceEntry{Irritation}
\vspace{-2mm}
\begin{listverse}
\item[\vref{Pr 27:3}] est lourd ; mais l'i. de l'insensé est
\item[\vref{Ec 7:9}] de t'irriter, car l'i. repose ds le
\item[\vref{Ep 4:31}] tte colère, tte i., tte clameur, tte
\end{listverse}

\ConcordanceEntry{Irriter}
\vspace{-2mm}
\begin{listverse}
\item[\vref{Ge 4:5}] Caïn fut fort i., et son visage
\item[\vref{De 32:16}] étrangers, ils l'ont i. par des abominations.
\item[\vref{2 S 6:8}] David fut i. de ce que
\item[\vref{1 R 16:7}] dvt Yahweh, en l'i. par l'œuvre de
\item[\vref{1 R 16:33}] avant lui, pour i. Yahweh, le Dieu
\item[\vref{2 R 17:11}] ils firent des choses mauvaises pour i. Yahweh.
\item[\vref{2 R 17:18}] Yahweh fut très i. contre Israël et
\item[\vref{2 R 21:6}] mal aux yeux de Yahweh pour l'i.
\item[\vref{2 Ch 34:25}] autres dieux, pour m'i. par ttes les
\item[\vref{Job 36:18}] Certainement Dieu est i. ; prends garde qu'il
\item[\vref{Ps 2:12}] peur qu'il ne s'i. et que vs.
\item[\vref{Ps 18:8}] ils furent ébranlés, parce qu'il était i.
\item[\vref{Ps 37:1}] David. [Aleph.] Ne t'i. pas contre les
\item[\vref{Ps 78:40}] de fois l'ont-ils i. au désert, et
\item[\vref{Pr 19:3}] c'est contre Yahweh que son cœur s'i.
\item[\vref{Ec 7:9}] ton esprit de t'i., car l'irritation repose
\item[\vref{Es 54:9}] de ne plus m'i. contre toi, et
\item[\vref{Es 64:4}] tu as été i. parce que ns.
\item[\vref{Es 65:3}] un peuple qui m'i. continuellement en face,
\item[\vref{Jé 7:18}] libations aux dieux étrangers, afin de m'i.
\item[\vref{Jé 25:7}] dit Yahweh, pour m'i. par les œuvres
\item[\vref{Jé 32:32}] ont fait pour m'i., eux, leurs rois,
\item[\vref{Jé 44:3}] ont faites pour m'i., en allant brûler
\item[\vref{Jon 4:9}] fais bien de m'i. jusqu'à la mort.
\item[\vref{Za 1:2}] a été extrêmement i. contre vos pères.
\item[\vref{Za 1:15}] je suis extrêmement i. contre les nations
\item[\vref{Jn 7:23}] violée, pourquoi êtes-vs. i. contre moi de
\item[\vref{Ac 17:16}] au-dedans son esprit s'i., à la vue
\item[\vref{1 Co 13:5}] intérêt, elle ne s'i. pas, elle n'impute
\item[\vref{Ep 6:4}] Et vs., pères, n'i. pas vos enfants,
\item[\vref{Ap 12:17}] le dragon fut i. contre la fem.,
\end{listverse}

\ConcordanceEntry{Isaac}
\vspace{-2mm}
\begin{listverse}
\item[\vref{Ge 17:19}] appelleras son nom I. ; et j'établirai mon
\end{listverse}

\ConcordanceEntry{Isaï}
\vspace{-2mm}
\begin{listverse}
\item[\vref{Ru 4:17}] fut le père d'I., père de David.
\item[\vref{1 S 16:1}] je t'enverrai chez I., Bethléhémite, car je
\item[\vref{1 S 20:27}] Pourquoi le fils d'I. n'a-t-il été ni
\item[\vref{1 Ch 2:13}] I. engendra son premier-né Eliab, le second
\item[\vref{Ps 72:20}] Fin des prières de David, fils d'I.
\item[\vref{Ro 15:12}] Il sortira d'I. un rejeton, qui
\end{listverse}

\ConcordanceEntry{Iscariot}
\vspace{-2mm}
\begin{listverse}
\item[\vref{Mt 10:4}] Cananite, et Judas I., celui qui le
\item[\vref{Mc 14:10}] Alors Judas I., l'un des douze,
\item[\vref{Lu 6:16}] Jacques, et Judas I., qui devint traître.
\item[\vref{Jn 6:71}] parlait de Judas I., fils de Simon ;
\item[\vref{Jn 13:2}] cœur de Judas I., fils de Simon,
\item[\vref{Jn 14:22}] Jude, non pas I., lui dit : Seign.,
\end{listverse}

\ConcordanceEntry{Isch-Boscheth}
\vspace{-2mm}
\begin{listverse}
\item[\vref{2 S 2:8}] de Saül, prit I., fils de Saül,
\item[\vref{2 S 2:10}] I., fils de Saül, était âgé de
\item[\vref{2 S 4:8}] apportèrent la tête d'I. à David ds
\end{listverse}

\ConcordanceEntry{Ismaël}
\vspace{-2mm}
\begin{listverse}
\item[\vref{Ge 16:11}] que tu appelleras I., car Yahweh a
\end{listverse}

\ConcordanceEntry{Ismaélites}
\vspace{-2mm}
\begin{listverse}
\item[\vref{Ge 37:25}] voici, une caravane d'I. qui passait et
\item[\vref{Ge 37:27}] vendons-le à ces I., et ne mettons
\item[\vref{Ge 37:28}] pièces d'argent aux I., qui emmenèrent Joseph
\item[\vref{Ge 39:1}] la main des I. qui l'y avaient
\item[\vref{Jg 8:24}] des anneaux d'or car ils étaient I.
\item[\vref{Ps 83:7}] d'Edom et des I., des Moabites et
\end{listverse}

\ConcordanceEntry{Israël}
\vspace{-2mm}
\begin{listverse}
\item[\vref{Ge 32:28}] tu seras appelé I. ; car tu as
\end{listverse}
\begin{legend}
\NoAutoSpaceBeforeFDP{
\item Peuple issu de Jacob : Ex 4:22
\item Séjourne en Egypte : Ge 15:13; 46:1
\item Les souffrances : Ex 1:13; 2:23
\item Sortie d’Egypte : Ex 12:40; 12:41; 1 S 10:18
\item Séjour au Sinaï : Ex 19:1; No 10:11,12
\item Alliance : Ex 24:8; Ga 4:24
\item Un peuple conquérant : Jos 10:40-42; 11:23
\item Les juges: Ac 13:20
\item Règne: 1 S 8:5,20-21; 1 S 11:14-15
\item Schisme : 1 R 12:20-21
\item Idolâtrie : No 25:3; 2 R 17; Ez 8
\item Captivité : 2 R 17:6, 25:11,22
\item Promesses de Yahweh : De 30:3-6;  Ps 130 :8; Jé 33:7; Es 5:25; 45 :25;  Ez 11:17, 20 :42
\item Autres : 2 Ch 9:8; Ps 73:1; Es 1:3; 45:4; Os 11:1; 12:13
}
\end{legend}

\ConcordanceEntry{Israélite}
\vspace{-2mm}
\begin{listverse}
\item[\vref{Ex 15:22}] fit partir les I. de la Mer
\item[\vref{Jos 8:22}] encerclés par les I., les uns d'un
\item[\vref{2 R 10:32}] Hazaël battit les I. sur ttes les
\item[\vref{2 R 17:6}] emmena captifs les I. en Assyrie. Il
\item[\vref{1 Ch 27:23}] le dénombrement des I., depuis l'âge de
\item[\vref{Ac 2:22}] Hommes I., écoutez ces paroles ! Jésus de Nazareth,
\item[\vref{Ac 3:12}] au peuple : Hommes I., pourquoi vs. étonnez-vs.
\item[\vref{Ro 9:4}] qui sont I., à qui appartiennent l'adoption, la gloire,
\item[\vref{Ro 11:1}] je suis aussi I., de la postérité
\item[\vref{2 Co 11:22}] suis aussi ! Sont-ils I. ? Je le suis
\item[\vref{Hé 11:28}] ne touchât pas aux premiers-nés des I.
\end{listverse}

\ConcordanceEntry{Issacar}
\vspace{-2mm}
\begin{listverse}
\item[\vref{Ge 30:18}] c'est pourquoi elle l'appela du nom d'I.
\end{listverse}
\begin{legend}
\NoAutoSpaceBeforeFDP{
\item Fils de Jacob et Léa : Ge 30:18; 49:14
\item Héritage : De 33:18, Jos 19:17
\item Autres : No 1:28; 26:23; Jg 5:15; Ap  7:7
}
\end{legend}

\ConcordanceEntry{Issue}
\vspace{-2mm}
\begin{listverse}
\item[\vref{Ps 37:37}] y a une i. pour l'hom. de
\item[\vref{Pr 14:12}] l'hom., mais dont l'i. sont les voies
\item[\vref{Es 41:22}] ns. saurons lr. i., ou faites-ns. entendre
\item[\vref{Da 12:8}] seigneur, quelle sera l'i. de ces choses ?
\end{listverse}

\ConcordanceEntry{Italie}
\vspace{-2mm}
\begin{listverse}
\item[\vref{Ac 18:2}] Pont, récemment arrivé d'I., avec Priscille, sa
\item[\vref{Ac 27:1}] ns. embarquerions pour l'I., on remit Paul
\item[\vref{Ac 27:6}] qui allait en I., ds lequel il
\item[\vref{Hé 13:24}] les saints. Ceux d'I. vs. saluent.
\end{listverse}

\ConcordanceEntry{Ithamar}
\vspace{-2mm}
\begin{listverse}
\item[\vref{Ex 6:23}] lui enfanta Nadab, Abihu, Eléazar et I.
\item[\vref{Ex 38:21}] sous la conduite d'I., fils du prêtre
\item[\vref{Lé 10:16}] contre Eléazar et I., les fils d'Aaron
\item[\vref{No 3:2}] qui était l'aîné, Abihu, Eléazar, et I.
\item[\vref{No 3:4}] mais Eléazar et I. exercèrent la prêtrise
\item[\vref{No 4:28}] sous la conduite d'I., fils d'Aaron, le
\item[\vref{1 Ch 24:2}] et Eléazar et I. exercèrent la prêtrise.
\item[\vref{1 Ch 24:4}] parmi les fils d'I., et on en
\end{listverse}

\ConcordanceEntry{Ittaï}
\vspace{-2mm}
\begin{listverse}
\item[\vref{2 S 15:19}] roi dit à I. de Gath : Pourquoi
\item[\vref{2 S 15:21}] Mais I. répondit au roi, et dit : Yahweh
\item[\vref{2 S 15:22}] dc dit à I. : Viens, et marche !
\item[\vref{2 S 18:2}] sous le commandement d'I., de Gath. Et
\item[\vref{2 S 18:5}] Abischaï, et à I., et dit : Epargnez-moi
\item[\vref{2 S 18:12}] Abischaï et à I., en disant : Prenez
\item[\vref{2 S 23:29}] Baana, de Nethopha. I., fils de Ribaï,
\item[\vref{1 Ch 11:31}] I. fils de Ribaï, de Guibea des
\end{listverse}

\ConcordanceEntry{Ivraie}
\vspace{-2mm}
\begin{listverse}
\item[\vref{Job 31:40}] blé, et de l'i. au lieu de
\item[\vref{Mt 13:25}] et sema de l'i. parmi le blé,
\item[\vref{Mt 13:26}] du fruit, alors l'i. parut aussi.
\item[\vref{Mt 13:27}] vient dc qu'il y a de l'i. ?
\item[\vref{Mt 13:29}] peur qu'en arrachant l'i., vs. ne déraciniez
\item[\vref{Mt 13:30}] moissonneurs : Arrachez premièrement l'i., et liez-la en
\item[\vref{Mt 13:36}] la parabole de l'i. du champ.
\item[\vref{Mt 13:38}] du Royaume, et l'i. ce sont les
\item[\vref{Mt 13:40}] com. on arrache l'i. et qu'on la
\end{listverse}

\ConcordanceEntry{Ivre}
\vspace{-2mm}
\begin{listverse}
\item[\vref{1 S 1:13}] C'est pourquoi Eli estima qu'elle était i.,
\item[\vref{1 S 1:14}] Jusqu'à qnd seras-tu i. ? Eloigne-toi du vin.
\item[\vref{1 S 1:15}] ne suis pas i., mon seigneur, je
\item[\vref{1 S 25:36}] il était complètement i.. C'est pourquoi elle
\item[\vref{Ps 107:27}] com. un hom. i. ; et tte lr.
\item[\vref{Pr 26:9}] main d'un hom. i., ainsi est un
\item[\vref{Es 19:14}] com. un hom. i. se vautre ds
\item[\vref{Es 24:20}] com. un hom. i., elle est transportée
\item[\vref{Es 29:9}] aveugles ! Ils sont i., mais non de
\item[\vref{Es 51:21}] ceci, ô affligée, i., mais non pas
\item[\vref{Jé 23:9}] com. un hom. i., et com. un
\item[\vref{Ac 2:15}] ne sont pas i., com. vs. le
\item[\vref{1 Co 11:21}] a faim tandis que l'autre est i.
\item[\vref{Ap 17:6}] vis cette fem. i. du sang des
\end{listverse}

\ConcordanceEntry{Ivrogne}
\vspace{-2mm}
\begin{listverse}
\item[\vref{Pr 23:21}] Car l'i. et le gourmand s'appauvrissent, et la
\item[\vref{Es 28:1}] de fierté des i. d'Ephraïm, la noblesse
\item[\vref{Joë 1:5}] I., réveillez-vs., et pleurez ! Et vs. ts,
\item[\vref{Mt 24:49}] s'il mange et boit avec les i.,
\item[\vref{1 Co 5:11}] ou médisant, ou i., ou ravisseur, de
\item[\vref{1 Co 6:10}] avares, ni les i., ni les médisants,
\end{listverse}

\ConcordanceEntry{Jabès}
\vspace{-2mm}
\begin{listverse}
\item[\vref{Jg 21:8}] aucun hom. de J. en Galaad n'était
\item[\vref{1 S 11:1}] vint et assiégea J. en Galaad. Les
\item[\vref{2 S 2:4}] Les gens de J. en Galaad ont
\item[\vref{2 S 21:12}] les habitants de J. en Galaad, qui
\end{listverse}

\ConcordanceEntry{Jachaziel}
\vspace{-2mm}
\begin{listverse}
\item[\vref{1 Ch 12:4}] trente, et Jérémie, J., Jochanan et Jozabad
\item[\vref{1 Ch 16:6}] Benaja et J., les prêtres, étaient
\item[\vref{1 Ch 23:19}] Amaria le second, J. le troisième, Jekameam
\item[\vref{2 Ch 20:14}] milieu de l'assemblée J., fils de Zacharie,
\item[\vref{Esd 8:5}] le fils de J., et avec lui
\end{listverse}

\ConcordanceEntry{Jacob}
\vspace{-2mm}
\begin{listverse}
\item[\vref{Ge 25:26}] il fut appelé J.. Isaac était âgé
\end{listverse}
\begin{legend}
\NoAutoSpaceBeforeFDP{
\item Fils cadet d’Isaac et Rebecca : Ge 25:25-26
\item S’octroie le droit d’aînesse d’Esaü : Ge 25:33
\item Reçois la bénédiction de l’aîné : Ge 27:27-30
\item Se réfugie chez le futur beau-père : Ge 27:43; 28:1-2
\item Visitation par vision : Ge 28:10-22
\item Retour en Canaan : Ge 31:1-3; 35:9
\item Combat avec Yahweh : Ge 32:24-32
\item Fais la paix avec Esaü : Ge 33
\item Dresse un autel à Yahweh : Ge 35:1
\item L’Egypte : Ge 46:6-7; Ps 105:23
\item Bénédiction des générations futures : Ge 48; Ge 49
\item Son décès : Ge 49:33
\item Autres : Ex 3:6; Ps 135:4; 147:19; Ro 9:13; Hé 11:21
}
\end{legend}

\ConcordanceEntry{Jacques}
\vspace{-2mm}
\begin{listverse}
\item[\vref{Mt 4:21}] deux autres frères, J., fils de Zébédée,
\end{listverse}
\begin{legend}
\NoAutoSpaceBeforeFDP{
\item Apôtre, frère de Jean fils de Zébédée : Mt 4:21; 10:2
\item Assiste à la transfiguration : Mt 17:1-2
\item Condamnation à mort par Hérode : Ac 12:2
\item Frère du Seigneur : Mt 13 :55
\item Ancien de l’Eglise de Jérusalem : Ac 12:17; 15:13 ; 21:18; Ga 1:19
\item Auteur de l’épître : Ja 1:1
\item Frère de l’apôtre Jude : Mt 13:55; Mc 6:3; Jud 1:1
\item Fils d’Alphée, Apôtre surnommé le mineur : Mt 10:3;  Mc 3 :18, 15:40; Ac 1:13
\item Frère de l’apôtre Jude surnommé Thadée et Lebbée : Mt 10:3; Mc 3:18; Lu 6:16; Ac 1:13
\item Ne pas confondre avec Jude appelé Barsabas : Ac 15:22,32
}
\end{legend}

\ConcordanceEntry{Jaebets}
\vspace{-2mm}
\begin{listverse}
\item[\vref{1 Ch 2:55}] qui habitaient à J. : Les Thireathiens, les
\item[\vref{1 Ch 4:9}] lesquels il eut J. plus distingué que
\end{listverse}

\ConcordanceEntry{Jaël}
\vspace{-2mm}
\begin{listverse}
\item[\vref{Jg 4:17}] la tente de J., fem. de Héber,
\item[\vref{Jg 5:6}] aux jours de J., les grandes routes
\end{listverse}

\ConcordanceEntry{Jaezer}
\vspace{-2mm}
\begin{listverse}
\item[\vref{No 21:32}] gens pour reconnaître J., ils prirent les
\item[\vref{No 32:1}] le pays de J. et le pays
\item[\vref{Jos 13:25}] lr. pays fut J., et ttes les
\end{listverse}

\ConcordanceEntry{Jaïr}
\vspace{-2mm}
\begin{listverse}
\item[\vref{De 3:14}] J., fils de Manassé, prit tte la
\item[\vref{Jos 13:30}] les villages de J. qui sont en
\item[\vref{Jg 10:3}] lui, se leva J., le Galaadite, qui
\item[\vref{Jg 10:4}] jour, villages de J., lesquels sont situées
\item[\vref{1 Ch 20:5}] Elchanan, fils de J., frappa Lachmi, frère
\item[\vref{Est 2:5}] Mardochée, fils de J., fils de Schimeï,
\end{listverse}

\ConcordanceEntry{Jaïrus}
\vspace{-2mm}
\begin{listverse}
\item[\vref{Mc 5:22}] la synagogue, nommé J., qui l'ayant aperçu,
\item[\vref{Lu 8:41}] un hom. appelé J., qui était chef
\end{listverse}

\ConcordanceEntry{Jalousie}
\vspace{-2mm}
\begin{listverse}
\item[\vref{No 5:14}] que l'esprit de j. saisisse son mari,
\item[\vref{De 32:16}] ont provoqué sa j. par des dieux
\item[\vref{De 32:21}] ont excité ma j. par ce qui
\item[\vref{1 R 14:22}] ils excitèrent sa j. plus que leurs
\item[\vref{Ps 78:58}] l'émurent à la j. par leurs images
\item[\vref{Pr 6:34}] Car la j. d'un mari est une fureur, il
\item[\vref{Pr 27:4}] mais qui pourra subsister dvt la j. ?
\item[\vref{Ec 4:4}] travail n'est que j. de l'un à
\item[\vref{Ca 8:6}] mort, et la j. est cruelle com.
\item[\vref{Es 59:17}] couvre de la j. com. d'un manteau.
\item[\vref{Ez 8:3}] posée l'idole de j. qui provoque la
\item[\vref{Ez 36:5}] feu de ma j. contre les autres
\item[\vref{So 1:18}] feu de sa j., car il se
\item[\vref{So 3:8}] dévoré par le feu de ma j.
\item[\vref{Za 8:2}] Sion d'une grande j., et je suis
\item[\vref{Ac 5:17}] sadducéens, et ils furent remplis de j. ;
\item[\vref{Ac 13:45}] furent remplis de j., et ils s'opposaient
\item[\vref{Ro 10:19}] dit : J'exciterai votre j. par ce qui
\item[\vref{Ro 11:11}] Gentils, pour les exciter à la j.
\item[\vref{Ro 11:14}] nation à la j., et en sauver
\item[\vref{1 Co 3:3}] vs. de la j., des disputes, et
\item[\vref{1 Co 10:22}] Seign. à la j. ? Sommes-ns. plus forts
\item[\vref{2 Co 11:2}] de vs. d'une j. de Dieu, parce
\end{listverse}

\ConcordanceEntry{Jaloux}
\vspace{-2mm}
\begin{listverse}
\item[\vref{Ex 20:5}] Dieu qui est j., qui punis l'iniquité
\item[\vref{Ex 34:14}] nomme le Dieu j. ; c'est le Dieu
\item[\vref{No 11:29}] lui répondit : Es-tu j. pour moi ? Plût
\item[\vref{De 4:24}] est un feu dévorant, un Dieu j.
\item[\vref{De 6:15}] est un Dieu j. au milieu de
\item[\vref{Jos 24:19}] Saint, qui est j., il ne pardonnera
\item[\vref{Es 11:13}] ne sera plus j. de Juda, et
\item[\vref{Joë 2:18}] Or Yahweh est j. pour son pays,
\item[\vref{Ac 7:9}] Les patriarches, j. de Joseph, le
\item[\vref{Ac 17:5}] Juifs, rebelles et j., prirent avec eux
\item[\vref{2 Co 11:2}] Car je suis j. de vs. d'une
\end{listverse}

\ConcordanceEntry{Jambe}
\vspace{-2mm}
\begin{listverse}
\item[\vref{Ez 34:4}] qui avait la j. rompue, vs. n'avez
\item[\vref{Ez 34:16}] qui a la j. rompue, et je
\item[\vref{Jn 19:33}] ils ne lui rompirent pas les j. ;
\end{listverse}

\ConcordanceEntry{Jambrès}
\vspace{-2mm}
\begin{listverse}
\item[\vref{2 Ti 3:8}] com. Jannès et J. ont résisté à
\end{listverse}

\ConcordanceEntry{Jannès}
\vspace{-2mm}
\begin{listverse}
\item[\vref{2 Ti 3:8}] Et com. J. et Jambrès ont résisté à Moïse,
\end{listverse}

\ConcordanceEntry{Japhet}
\vspace{-2mm}
\begin{listverse}
\item[\vref{Ge 5:32}] cents ans, engendra Sem, Cham, et J.
\item[\vref{Ge 6:10}] engendra trois fils : Sem, Cham, et J.
\item[\vref{Ge 7:13}] Sem, Cham et J., fils de Noé
\item[\vref{Ge 10:2}] Les fils de J. furent : Gomer, Magog,
\item[\vref{1 Ch 1:5}] Les fils de J. furent : Gomer, Magog,
\end{listverse}

\ConcordanceEntry{Japho}
\vspace{-2mm}
\begin{listverse}
\item[\vref{Jos 19:46}] le territoire qui est vis-à-vis de J.
\item[\vref{2 Ch 2:16}] la mer, jusqu'à J., et tu les
\item[\vref{Esd 3:7}] la mer de J., selon la permission
\item[\vref{Jon 1:3}] il descendit à J., où il trouva
\end{listverse}

\ConcordanceEntry{Jardin}
\vspace{-2mm}
\begin{listverse}
\item[\vref{Ge 2:8}] Dieu planta un j. en Eden, du
\item[\vref{Ge 2:9}] au milieu du j., et l'arbre de
\item[\vref{Ge 3:8}] promenait ds le j. ; et Adam et
\item[\vref{Ge 3:23}] le chassa du j. d'Eden pour qu'il
\item[\vref{Ge 13:10}] Tsoar, com. le j. de Yahweh, et
\item[\vref{2 R 21:18}] enseveli ds le j. de sa maison,
\item[\vref{Né 3:15}] Siloé, vers le j. du roi, et
\item[\vref{Est 7:7}] entra ds le j. du palais. Mais
\item[\vref{Ca 4:12}] tu es un j. clos, une source
\item[\vref{Ca 5:1}] J'entre ds mon j., ma sœur, mon
\item[\vref{Es 51:3}] aride à un j. de Yahweh. En
\item[\vref{Es 58:11}] seras com. un j. arrosé, et com.
\item[\vref{Es 61:11}] et com. un j. fait germer ses
\item[\vref{Ez 28:13}] en Eden, le j. de Dieu ; ta
\item[\vref{Ez 36:35}] devenue com. le j. d'Eden ; et ces
\item[\vref{Lu 13:19}] jeté ds son j. ; et il pousse,
\item[\vref{Jn 18:1}] y avait un j., ds lequel il
\item[\vref{Jn 19:41}] y avait un j. ds le lieu
\end{listverse}

\ConcordanceEntry{Jardinier}
\vspace{-2mm}
\begin{listverse}
\item[\vref{Jn 20:15}] que c'était le j., lui dit : Seign.,
\end{listverse}

\ConcordanceEntry{Jason}
\vspace{-2mm}
\begin{listverse}
\item[\vref{Ac 17:5}] la maison de J., et ils cherchèrent
\item[\vref{Ac 17:6}] trouvés, ils traînèrent J. et quelques frères
\item[\vref{Ac 17:9}] ne laissèrent aller J. et les autres
\item[\vref{Ro 16:21}] que Lucius, et J. et Sosipater, mes
\end{listverse}

\ConcordanceEntry{Jaspe}
\vspace{-2mm}
\begin{listverse}
\item[\vref{Ex 28:18}] une escarboucle, un saphir, et un j.
\item[\vref{Ex 39:11}] une escarboucle, un saphir, et un j.
\item[\vref{Ez 28:13}] chrysolithe, d'onyx, de j., de saphir, d'escarboucle,
\item[\vref{Ap 4:3}] une pierre de j. et de sardoine ;
\item[\vref{Ap 21:11}] une pierre de j. transparente com. du
\item[\vref{Ap 21:18}] muraille était de j., mais la ville
\item[\vref{Ap 21:19}] fondement était de j., le second de
\end{listverse}

\ConcordanceEntry{Javan}
\vspace{-2mm}
\begin{listverse}
\item[\vref{Ge 10:2}] Gomer, Magog, Madaï, J., Tubal, Méschec, et
\item[\vref{Es 66:19}] Tubal et à J., et vers les
\item[\vref{Ez 27:13}] J., Tubal, et Méschec trafiquaient avec toi ;
\item[\vref{Da 8:21}] le roi de J. ; et la grande
\item[\vref{Da 11:2}] le monde contre le royaume de J.
\item[\vref{Za 9:13}] tes enfants, ô J., et que je
\end{listverse}

\ConcordanceEntry{Javelot}
\vspace{-2mm}
\begin{listverse}
\item[\vref{Jos 8:18}] vers Aï le j. qui est ds
\item[\vref{1 S 17:6}] d'airain, et un j. d'airain entre ses
\item[\vref{1 S 17:45}] lance, et le j. ; mais moi, je
\item[\vref{2 S 18:14}] sa main trois j., et les enfonça
\item[\vref{Job 39:26}] luisant de la lance et du j.
\item[\vref{Job 41:20}] se moque du j. qu'on lance sur
\item[\vref{Ps 35:3}] lance et le j. contre mes persécuteurs !
\item[\vref{Jé 6:23}] l'arc et le j. ; ils sont cruels
\item[\vref{Jé 50:42}] l'arc et le j. ; ils sont cruels,
\end{listverse}

\ConcordanceEntry{Jean}
\vspace{-2mm}
\begin{listverse}
\item[\vref{Mt 4:21}] de Zébédée, et J., son frère, ds
\item[\vref{Lu 1:63}] tablettes écrivit, disant : J. est son nom.
\end{listverse}
\begin{legend}
\NoAutoSpaceBeforeFDP{
\item Jean-Baptiste : Mt 3:1
\item Fils de Zacharie : Lu 1:13; 57-63
\item L’Esprit d’Elie en J-B : Mt 17:10-13 ; Mal 3:1
\item J-B et la repentance : Mc 1:4; Ac 13:24
\item Baptême : Mt 3:6-11 ; Ac 1:5; 19:3-4
\item Témoignage de J-B : Jn 1:29-34; 3:26
\item Témoignage de Jésus : Mt 11:2-14
\item Sa mort : Lu 3:19; Lu 3:20; Mt 14:1-12
\item Apôtre Jean : Mt 4:21; 10:2
\item L’appel de Jean : Mc 1:19-20
\item Le caractère de Jean : Mc 3:17; Lu 9:49-55
\item Le disciple que Jésus aimait : Jn 21:7,20,24
\item Témoin de la transfiguration : Mt 17:1-2
\item et de la vie de Jésus : : Mc 5:35-40; 14:32-34
\item Jean accueille la mère de Jésus : Jn 19:26-27
\item La résurrection de Jésus : Jn 20:1-8
\item Associé à Pierre : Ac 3:11; 8:14-17
}
\end{legend}

\ConcordanceEntry{Jébus}
\vspace{-2mm}
\begin{listverse}
\item[\vref{Jg 19:10}] jsq. vis-à-vis de J., qui est Jérus.,
\item[\vref{Jg 19:11}] étaient près de J., le jour avait
\item[\vref{1 Ch 11:4}] Jérus., qui est J.. Là étaient les
\item[\vref{1 Ch 11:5}] qui habitaient à J. dirent à David :
\end{listverse}

\ConcordanceEntry{Jébusiens}
\vspace{-2mm}
\begin{listverse}
\item[\vref{Ge 10:16}] et les J., les Amoréens, les Guirgasiens,
\item[\vref{Ge 15:21}] des Cananéens, des Guirgasiens, et des J.
\item[\vref{Ex 3:8}] les Phéréziens, les Héviens et les J.
\item[\vref{Ex 33:2}] les Phéréziens, les Héviens et les J.,
\item[\vref{No 13:29}] les Héthiens, les J. et les Amoréens
\item[\vref{Jos 15:63}] pas chasser les J. qui habitaient à
\item[\vref{2 S 5:6}] Jérus. contre les J. qui habitaient ce
\end{listverse}

\ConcordanceEntry{Jéconia}
\vspace{-2mm}
\begin{listverse}
\item[\vref{1 Ch 3:16}] de Jojakim furent J., son fils, qui
\item[\vref{Est 2:6}] emmenés captifs avec J., roi de Juda,
\item[\vref{Jé 22:24}] Yahweh, que qnd J., fils de Jojakim,
\item[\vref{Jé 24:1}] transporté de Jérus., J., fils de Jojakim,
\item[\vref{Jé 27:20}] Jérus. à Babylone, J., fils de Jojakim,
\item[\vref{Jé 28:4}] lieu, dit Yahweh, J., fils de Jojakim,
\end{listverse}

\ConcordanceEntry{Jedidja}
\vspace{-2mm}
\begin{listverse}
\item[\vref{2 S 12:25}] le nom de J., à cause de
\end{listverse}

\ConcordanceEntry{Jeduthun}
\vspace{-2mm}
\begin{listverse}
\item[\vref{1 Ch 9:16}] Galal, fils de J. ; et Bérékia, fils
\item[\vref{1 Ch 16:41}] étaient Héman et J., et les autres
\item[\vref{1 Ch 25:1}] d'Héman et de J. qui prophétisaient avec
\item[\vref{1 Ch 25:6}] à Asaph, à J. et à Héman.
\item[\vref{2 Ch 35:15}] d'Héman et de J., le voyant du
\item[\vref{Ps 39:1}] donné au chef des chantres. A J.
\item[\vref{Ps 62:1}] donné au chef des chantres, d'après J.
\end{listverse}

\ConcordanceEntry{Jehojada}
\vspace{-2mm}
\begin{listverse}
\item[\vref{2 R 11:4}] La septième année, J. envoya chercher les
\end{listverse}
\begin{legend}
\NoAutoSpaceBeforeFDP{
\item Souverain sacrificateur : 2 Ch 24:6
\item Etablit le roi Joas : 2 R 11:9-12
\item Frappe Athalie : 2 R 11:13-16
\item Répare le temple : 2 R 12:9-16
}
\end{legend}

\ConcordanceEntry{Jéhu}
\vspace{-2mm}
\begin{listverse}
\item[\vref{1 R 16:1}] Yahweh vint à J., fils de Hanani,
\end{listverse}
\begin{legend}
\NoAutoSpaceBeforeFDP{
\item Prophète : 2 Ch 19:2; 20:34
\item Yahweh avertit Baescha : 1 R 16:1-4
\item Roi d’Israël : 1 R 19:16; 9:1-8
\item Accomplissement du jugement de Dieu sur la maison d’Achab : 2 R 9:24; 10:1-11
\item Extermination des prophètes de Baal : 2 R 10:18-27
}
\end{legend}

\ConcordanceEntry{Jephthé}
\vspace{-2mm}
\begin{listverse}
\item[\vref{Jg 11:1}] Or J., le Galaadite, était un fort et
\item[\vref{Jg 12:7}] J. fut juge en Israël pendant six
\end{listverse}

\ConcordanceEntry{Jérémie}
\vspace{-2mm}
\begin{listverse}
\item[\vref{Jé 1:1}] Les Paroles de J., fils de Hilkija,
\end{listverse}
\begin{legend}
\NoAutoSpaceBeforeFDP{
\item Prophète : 2 Ch 36:12 ; Jé 1:1
\item Son appel : Jé 1:4-10; 1:17-19
\item Dénonce le péché : Jé 2:25;  6:16-17
\item Appel à la repentance : Jé 3:12-14;  4:1-14
\item La prophétie de la captivité par Babylone : Jé 25:11
\item Annonce le retour : Jé 29:10-14; 30:3
\item Mise en lumière des faux prophètes : Jé 28:15-17;  29:31
\item Ecriture d’un livre : Jé 36:2,32
\item Souffrances et combats dans l’appel : Jé 15:10;  18:18;  20:7-18
\item Périodes d’emprisonnement : Jé 37:11-16;  38:1-6; 38:28
\item Emmené en Egypte : Jé 43:1-7
\item Ses prophéties : Jé 31:15-17; 31-34; 33:14-16
\item
}
\end{legend}

\ConcordanceEntry{Jéricho}
\vspace{-2mm}
\begin{listverse}
\item[\vref{Jos 2:1}] le pays, et J.. Ils partirent dc
\end{listverse}
\begin{legend}
\NoAutoSpaceBeforeFDP{
\item Inspectée  par les espions :  Jos 2:1
\item Saisie sous  Josué : Jos 6:20, 1 R 16:34
\item Anathématisée : Jos 6:1, 6:26; 1 R 16:34;
\item Autres : 2 R 2:4, 2 Ch 28:15 ; Mt 20:29; Hé 11:30
}
\end{legend}

\ConcordanceEntry{Jéroboam}
\vspace{-2mm}
\begin{listverse}
\item[\vref{1 R 11:26}] J. aussi, serviteur de Salomon, s'éleva également
\end{listverse}
\begin{legend}
\NoAutoSpaceBeforeFDP{
\item Roi d’Israël après le schisme : 1 R 11:26-31; 12:20
\item Idôlatre : 1 R 12:28-33; 13:1
\item La vie de péché : 1 R 14:16; 15:26; 22:53
\item Bafoue le culte à Yahweh : 2 Ch 11:13; 11:14
\item Condamnation et mort : 1 R 14:10; 15:29-30
\item Autre roi d’Israël : 2 R 13:13
\item Yahweh délivre Israël : 2 R 14:23-28
}
\end{legend}

\ConcordanceEntry{Jerubbaal}
\vspace{-2mm}
\begin{listverse}
\item[\vref{Jg 6:32}] le nom de J., en disant : Que
\item[\vref{Jg 7:1}] J. qui est Gédéon, et tt le
\end{listverse}

\ConcordanceEntry{Jérusalem}
\vspace{-2mm}
\begin{listverse}
\item[\vref{Mt 5:35}] pieds ; ni par J., parce que c'est
\end{listverse}
\begin{legend}
\NoAutoSpaceBeforeFDP{
\item Ville en Canaan : Jos 10:1
\item Habitants : les Jébusiens : Jos 15:63
\item Capital de David : 2S 5:6-9
\item Le choix de Yahweh : 1R 11:13; 2R 21:4
\item L’arche : 1 Ch 15:3,29
\item Le temple : 2Ch 3:1
\item Pillages et brèches : 1R 14:25; 2R 14:13
\item Prise et destruction : 2R 25:8-11; Jé 9:11
\item Reconstruction : Né 3; Es 44:28
\item Prophétie de Jésus sur la ville : Lu 19:41-44
\item Autres prophéties sur la ville : Ps 128:5 ; Ps 147:2 ; Es 65:18; Joë 3:1; Za 12:2 -3; Lu 21:20-22
\item La Jérusalem céleste : Ga 4:26; Hé 12:22
\item La nouvelle Jérusalem : Ap 3:12; Ap 21:2
}
\end{legend}

\ConcordanceEntry{Jésus, Christ, Jésus-Christ}
\vspace{-2mm}
\begin{listverse}
\item[\vref{Mt 1:1}] la généalogie de J.-Christ, fils de
\item[\vref{Mt 1:16}] laquelle est né J., qui est appelé
\item[\vref{Mt 1:18}] la naissance de J.-Christ. Comme Marie,
\item[\vref{Mt 1:21}] le Nom de J.; car c'est lui
\item[\vref{Mt 16:16}] Tu es le C., le Fils du
\item[\vref{Mt 22:42}] Que pensez-vs. du C. ? De qui est-il
\item[\vref{Mt 23:8}] notre maître ; car C. seul est votre
\item[\vref{Mc 10:21}] J., l'ayant regardé, l'aima, et lui dit :
\item[\vref{Mc 12:35}] Comme J. enseignait ds le temple, il prit
\item[\vref{Mc 14:62}] Et J. lui répondit : Je le suis. Et
\item[\vref{Lu 4:41}] Tu es le C., le Fils de
\item[\vref{Jn 1:20}] n'est pas moi qui suis le C.
\item[\vref{Jn 1:41}] avons trouvé le Messie, c'est-à-dire le C.
\item[\vref{Jn 4:42}] est véritablement le C., le Sauveur du
\item[\vref{Jn 12:34}] loi que le C. demeure éternellement ; et
\item[\vref{Jn 19:5}] J. dc sortit, portant la couronne d'épines
\item[\vref{Jn 20:31}] vs. croyiez que J. est le Christ,
\item[\vref{Ac 2:36}] fait Seign. et C., ce Jésus, dis-je,
\item[\vref{Ac 8:5}] la ville de Samarie, lr. prêcha C.
\item[\vref{Ac 9:5}] dit : Je suis J., que tu persécutes.
\item[\vref{Ac 10:38}] et de force J. de Nazareth, qui
\item[\vref{Ac 13:23}] promesse, a suscité J. pour être le
\item[\vref{Ac 17:7}] a un autre Roi, qu'ils nomment J.
\item[\vref{Ac 18:5}] aux Juifs que J. était le Christ.
\item[\vref{Ac 24:24}] Il l'entendit sur la foi en C.
\item[\vref{Ac 26:9}] le Nom de J. de Nazareth.
\item[\vref{Ro 1:16}] de l'Evangile de C., vu qu'il est
\item[\vref{Ro 5:6}] encore sans force, C. est mort, en
\item[\vref{Ro 8:9}] pas l'Esprit de C., il ne lui
\item[\vref{1 Co 2:2}] autre chose que J.-Christ et Jésus-Christ
\item[\vref{1 Co 2:16}] ns., ns. avons la pensée de C.
\item[\vref{1 Co 6:15}] les membres de C. ? Prendrai-je dc les
\item[\vref{1 Co 10:4}] les suivait, et ce rocher était C.
\item[\vref{1 Co 12:3}] Dieu, ne dit : J. est anathème ! Et
\item[\vref{1 Co 15:3}] à savoir que C. est mort pour
\item[\vref{2 Co 5:17}] quelqu'un est en C., il est une
\item[\vref{2 Co 11:4}] prêcher un autre J. que ns. n'avons
\item[\vref{2 Co 13:5}] reconnaissez-vs. pas que J.-Christ est en
\item[\vref{Ga 3:13}] C. ns. a rachetés de la malédiction
\item[\vref{Ga 3:16}] seule, et à sa postérité, c'est-à-dire C.
\item[\vref{Ga 3:26}] de Dieu par la foi en J.-Christ,
\item[\vref{Ga 4:14}] un ange de Dieu, et com. J.-Christ.
\item[\vref{Ga 4:19}] jusqu'à ce que C. soit formé en
\item[\vref{Ga 5:1}] liberté pour laquelle C. ns. a affranchis,
\item[\vref{Ga 6:14}] de notre Seign. J.-Christ, par qui
\item[\vref{Ep 1:10}] réunit tt en C., tant ce qui
\item[\vref{Ep 2:10}] été créés en J.-Christ pour les
\item[\vref{Ep 3:17}] en sorte que C. habite ds vos
\item[\vref{Ep 4:20}] Mais vs. n'avez pas ainsi appris C.,
\item[\vref{Ep 4:21}] selon que la vérité est en J. ;
\item[\vref{Ep 5:2}] charité, ainsi que C. ns. a aimés
\item[\vref{Ep 5:24}] est soumise à C., les femmes aussi
\item[\vref{Ph 1:21}] Car C. est ma vie, et la mort
\item[\vref{Ph 2:10}] qu'au Nom de J., tt genou fléchisse,
\item[\vref{Col 1:27}] Gentils, à savoir C. en vs., l'espérance
\item[\vref{Col 1:28}] puissions présenter tt hom. parfait en J.-Christ.
\item[\vref{Col 2:17}] venir, mais le corps est en C.
\item[\vref{Col 3:11}] ni libre ; mais C. y est tt
\item[\vref{1 Th 4:14}] ns. croyons que J. est mort, et
\item[\vref{1 Th 4:16}] les morts en C. ressusciteront premièrement.
\item[\vref{Hé 9:24}] Car C. n'est pas entré ds un sanctuaire
\item[\vref{Hé 12:2}] les yeux sur J., le chef et
\item[\vref{Hé 13:8}] J.-Christ est le mm hier, aujourd'hui,
\item[\vref{1 Pi 1:19}] sang précieux de C., com. d'un agneau
\item[\vref{1 Pi 2:21}] cela, puisque mm C. a souffert pour
\item[\vref{1 Pi 4:1}] Ainsi dc C. ayant souffert pour
\item[\vref{1 Jn 2:22}] qui nie que J. est le Christ ?
\item[\vref{1 Jn 4:2}] qui confesse que J.-Christ est venu
\item[\vref{1 Jn 4:3}] confesse point que J.-Christ est venu
\item[\vref{1 Jn 4:15}] Quiconque confessera que J. est le Fils
\item[\vref{1 Jn 5:1}] Quiconque croit que J. est le Christ,
\item[\vref{1 Jn 5:6}] C'est ce J., le Christ, qui est venu avec
\item[\vref{Jud 1:4}] le seul dominateur, J.-Christ, notre Dieu
\item[\vref{Jud 1:21}] de notre Seign. J.-Christ pour obtenir
\item[\vref{Jud 1:25}] notre Sauveur, par J.-Christ, notre Seign.,
\item[\vref{Ap 12:17}] et qui ont le témoignage de J.-Christ.
\item[\vref{Ap 20:4}] le témoignage de J., et pour la
\item[\vref{Ap 22:16}] Moi, J., j'ai envoyé mon ange pour vs.
\item[\vref{Ap 22:20}] à tte vitesse. Amen ! Oui, Seign. J., viens !
\item[\vref{Ap 22:21}] de notre Seign. J.-Christ soit avec
\end{listverse}
\begin{legend}
\NoAutoSpaceBeforeFDP{
\item Dieu incarné : Es 9:5,  7:14, Col  2:9, Ph 2:7, Col 1:16;Jn 1:1, 14; Ap 19:13; Ex 3:14; Jn 8:58; Es 41:4, Ap 1:8, 22:13
\item Naissance : Mt 1:18-23; Lu 2:1-12
\item Baptême : Mt 3:13-17; Mc 1:9-11
\item Tentation : Mt 4:1-11; Hé 2:18; Hé 4:15
\item Passage en Galilée : : Mt 4:1-12; Mc 1:39; Jn 7:1
\item Se transfigure devant 3 disciples : Mc 9:2-13
\item Pouvoir de guérir : Mt 4:23; 8:16; Mc 7:37
\item Pouvoir de purifier les lépreux :   Mc 1:40-44; Lu 17:11-19
\item Pouvoir de chasser les démons :   Mt 8:16; 9:32-33; Lu 11:20
\item Pouvoir de ressusciter les morts : Lu 7:11-15; Jn 12:9
\item Pouvoir de marcher sur l’eau : Mt 14:25
\item Pouvoir sur le vent et la tempête : Mt 8:23-27
\item Pouvoir de pardonner les péchés : Mt 9:2-6; Ac 10:43
\item Pouvoir de multiplier des pains :  Mt 14:17-21; 15:32-38
\item Mort et résurrection : Mt 16:21; 20:17-19; Mc 9:31; Lu 9:22
\item Crucifixion : Mc 15:21-27; Ac 4:10
\item La puissance de la résurrection : Lu 24:34; Ro 1:4
\item Jésus apparait aux disciples : Jn 21:14 ; Ac 1:3
\item Annonce de la venue de l’Esprit : Lu 24:49; Ac 1:8; 2:33
\item Ascension : Ac 1:9
\item Glorification : Ac 2:36; Jn 7:37-39
\item Jésus à la droite de Dieu : Mc 16:19; Ep 1:20
\item Souverain sacrificateur : Hé 5:5; 7:24-26
\item Sa venue prochaine : Ac 1:11; 1 Th 4:15-17
\item Les noces de l’Agneau : Ap 19:7
}
\end{legend}

\ConcordanceEntry{Jeter}
\vspace{-2mm}
\begin{listverse}
\item[\vref{Za 11:13}] Yahweh me dit : J.-le au potier,
\item[\vref{Mt 4:6}] Fils de Dieu, j.-toi en bas ;
\item[\vref{Mt 8:12}] du Royaume seront j. ds les ténèbres
\item[\vref{Mt 21:21}] de là et j.-toi ds la
\item[\vref{Jn 15:6}] moi, il est j. dehors, com. le
\end{listverse}

\ConcordanceEntry{Jéthro}
\vspace{-2mm}
\begin{listverse}
\item[\vref{Ex 3:1}] du troupeau de J., son beau-père, prêtre
\item[\vref{Ex 4:18}] et retourna vers J., son beau-père, et
\item[\vref{Ex 18:1}] Or J., prêtre de Madian, beau-père de Moïse,
\item[\vref{Ex 18:2}] J., beau-père de Moïse, prit Séphora la
\end{listverse}

\ConcordanceEntry{Jeune}
\vspace{-2mm}
\begin{listverse}
\item[\vref{1 R 3:7}] ne suis qu'un j. hom., je ne
\item[\vref{Ps 37:25}] Nun.] J'ai été j. et j'ai vieilli ;
\item[\vref{Ps 119:9}] quel moyen le j. hom. rendra-t-il pure
\item[\vref{Pr 1:4}] aux simples, aux j. gens de la
\item[\vref{Pr 20:29}] La force des j. gens est lr.
\item[\vref{Ec 12:1}] J. hom., réjouis-toi ds ton jeune âge,
\item[\vref{Es 3:4}] lr. donnerai de j. gens pour chefs,
\item[\vref{Da 1:4}] quelques j. garçons en qui il n'y avait
\item[\vref{Mc 14:51}] Et un certain j. hom. le suivait,
\item[\vref{Lu 7:14}] Et il dit : J. hom., je te
\item[\vref{Jn 21:18}] tu étais plus j., tu te ceignais
\item[\vref{1 Pi 5:5}] De mm, vs. j. gens, soyez soumis
\end{listverse}

\ConcordanceEntry{Jeûne}
\vspace{-2mm}
\begin{listverse}
\item[\vref{1 R 21:9}] lettres : Publiez un j. et placez Naboth
\item[\vref{Esd 8:21}] publiai là un j. près de la
\item[\vref{Ps 35:13}] âme par le j., je priais ds
\item[\vref{Es 58:6}] plutôt ici le j. que j'ai choisi :
\item[\vref{Jon 3:5}] ils publièrent un j. et se vêtirent
\item[\vref{Za 8:19}] des armées : Le j. du quatrième mois,
\item[\vref{Lu 2:37}] jour ds le j. et ds les
\item[\vref{Ac 27:9}] le temps du j. était déjà passé.
\item[\vref{1 Co 7:5}] vs. vaquiez au j. et à la
\item[\vref{2 Co 6:5}] travaux, ds les veilles, ds les j.,
\item[\vref{2 Co 11:27}] souvent ds les j., ds le froid
\end{listverse}

\ConcordanceEntry{Jeûner}
\vspace{-2mm}
\begin{listverse}
\item[\vref{Né 1:4}] plusieurs jours. Je j. et je priai
\item[\vref{Es 58:3}] Pourquoi j.-ns., et tu ne le vois
\item[\vref{Za 7:5}] Quand vs. avez j. et pleuré au
\item[\vref{Mt 4:2}] Après avoir j. quarante jours et
\item[\vref{Mt 6:16}] Et qnd vs. j., ne prenez pas
\item[\vref{Mt 9:15}] l'époux lr. sera enlevé, alors ils j.
\item[\vref{Lu 18:12}] Je j. deux fois la semaine, et je
\item[\vref{Ac 13:3}] Alors, après avoir j. et prié, ils
\end{listverse}

\ConcordanceEntry{Jeunesse}
\vspace{-2mm}
\begin{listverse}
\item[\vref{Ge 8:21}] mauvaise dès lr. j. ; et je ne
\item[\vref{1 R 18:12}] ton serviteur craint Yahweh dès sa j.
\item[\vref{Ps 25:7}] péchés de ma j. ni de mes
\item[\vref{Ps 71:5}] Seign. Yahweh ! ma confiance dès ma j.
\item[\vref{Ps 71:17}] enseigné dès ma j., et j'ai annoncé
\item[\vref{Ps 110:3}] de l'aurore, ta j. vient à toi
\item[\vref{Ps 129:1}] Ils m'ont souvent tourmenté dès ma j.
\item[\vref{Ec 12:1}] jours de ta j., et marche com.
\item[\vref{Es 54:4}] honte de ta j., et tu ne
\item[\vref{Jé 3:4}] Tu as été l'ami de ma j. !
\item[\vref{Jé 31:19}] car je porte l'opprobre de ma j.
\item[\vref{1 Ti 4:12}] ne méprise ta j. ; mais sois le
\end{listverse}

\ConcordanceEntry{Jézabel}
\vspace{-2mm}
\begin{listverse}
\item[\vref{1 R 16:31}] prit pour fem. J., fille d'Ethbaal, roi
\end{listverse}
\begin{legend}
\NoAutoSpaceBeforeFDP{
\item Fille d'Ethbaal, roi de Sidon et Epouse d'Achab roi d'Israël :  1 R 16:31
\item Fausse prophètesse et idolâtre : Ap 2:20-23
\item J. fait exterminer les prophètes de Yahweh : 1 R 18:4
\item Menace Elie :  1 R 19:1-3
\item J. fait tué Naboth pour sa vigne : 1 R 21:8-14
\item Sa mort : 1 R 21:23; 2 R 9:30-37
\item Autres : Ge 9:22-27; 10:15;  1 R 21:15; 2 R 9:10; 2 R 9:22
}
\end{legend}

\ConcordanceEntry{Jizreel}
\vspace{-2mm}
\begin{listverse}
\item[\vref{Jos 17:16}] qui habitent ds la vallée de J.
\end{listverse}
\begin{legend}
\NoAutoSpaceBeforeFDP{
\item Ville : 2 S 2:9; 1 R 18:45 ; 2 R 9:16
\item Vallée : Jos 17:16
\item Lieu de bataille : Jg 6:33; 2 R 10:1-11 ; Os 1:5
\item Fils d’Osée : Os 1:4
}
\end{legend}

\ConcordanceEntry{Joab}
\vspace{-2mm}
\begin{listverse}
\item[\vref{2 S 2:13}] J., fils de Tseruja, et les gens
\end{listverse}
\begin{legend}
\NoAutoSpaceBeforeFDP{
\item Chef de l’armée de David : 2 S 2:13; 8:16
\item Soutien Adonija : 1 R 1:5-7
\item Tué par Salomon : 1 R 2:5 -5; 28-35
\item Autres : 2 S 12:26; 14:1,19; 18:14
}
\end{legend}

\ConcordanceEntry{Joachaz}
\vspace{-2mm}
\begin{listverse}
\item[\vref{2 R 13:1}] roi de Juda, J., fils de Jéhu,
\end{listverse}

\ConcordanceEntry{Joas}
\vspace{-2mm}
\begin{listverse}
\item[\vref{1 Ch 3:11}] Achazia, son fils ; J., son fils ;
\end{listverse}
\begin{legend}
\NoAutoSpaceBeforeFDP{
\item Roi de Juda : 2 R 11:2; 12:1
\item Ordonne des réparations dans le temple : . 2 R 12:4 -5; 11-12
\item Impiété : 2 Ch 24:17-22
\item Soulèvement de ses serviteurs : 2 R 12:20
\item Règne sur Israël : 2 R 13:10
\item Joas rend visite à Elisée : 2 R 13:14-19
\item Victoire sur les Syriens : 2 R 13:25
\item Frappe Juda : 2 R 14:12-14
}
\end{legend}

\ConcordanceEntry{Job}
\vspace{-2mm}
\begin{listverse}
\item[\vref{Job 1:1}] un hom. appelé J.. Cet hom. était
\end{listverse}
\begin{legend}
\NoAutoSpaceBeforeFDP{
\item Un homme intègre : Job 1:1,8; 2:3
\item Attaques de Satan : Job 1:12; 2:7; 30:17-19
\item Souffrances et lamentations : Job 3:1; 6:1-4; 7:11-21
\item Réponse de Job sur son innocence : Job 9:21; 23:7,10
\item Yahweh interroge Job : Job 38:1; 40:1
\item S’humilie devant la toute-puissance de Yahweh : Job 42:1-6
\item Le rétablissement de Job : Job 42:10-17
\item Autres : Ez 14:14,20; Ja 5:11
}
\end{legend}

\ConcordanceEntry{Joël}
\vspace{-2mm}
\begin{listverse}
\item[\vref{1 S 8:2}] fils premier-né s'appelait J., et le second
\item[\vref{1 Ch 4:35}] J., Jéhu fils de Joschibia, fils de
\item[\vref{Joë 1:1}] qui vint à J., fils de Pethuel.
\item[\vref{Ac 2:16}] a été dit par le prophète J. :
\end{listverse}

\ConcordanceEntry{Joie}
\vspace{-2mm}
\begin{listverse}
\item[\vref{Ge 31:27}] laissé partir avec j. et avec des
\item[\vref{Lé 9:24}] des cris de j. et tombèrent sur
\item[\vref{Jg 18:20}] eut de la j. ds son cœur ;
\item[\vref{2 R 11:14}] pays éclatait de j., et on sonnait
\item[\vref{2 Ch 20:27}] à Jérus. avec j. ; car Yahweh les
\item[\vref{Esd 6:22}] ils célébrèrent avec j. la fête des
\item[\vref{Né 8:10}] tristes, car la j. de Yahweh est
\item[\vref{Job 20:5}] et que la j. de l'hypocrite n'est
\item[\vref{Ps 4:8}] mets plus de j. ds mon cœur
\item[\vref{Ps 16:11}] y a d'abondantes j. dvt ta face,
\item[\vref{Ps 21:7}] le combles de j. dvt ta face.
\item[\vref{Ps 51:14}] Rends-moi la j. de ton salut
\item[\vref{Ps 100:2}] dvt lui avec un chant de j. !
\item[\vref{Pr 14:13}] triste, et la j. finit par l'ennui.
\item[\vref{Ec 5:19}] répond par la j. de son cœur.
\item[\vref{Es 16:10}] Et la j. et l'allégresse se sont retirées du
\item[\vref{Es 35:10}] triomphe, et une j. éternelle sera sur
\item[\vref{Es 55:12}] vs. sortirez avec j., et vs. serez
\item[\vref{Es 61:3}] une huile de j. au lieu du
\item[\vref{Es 65:18}] pour n'être que j., et son peuple
\item[\vref{Mt 13:20}] parole et la reçoit aussitôt avec j. ;
\item[\vref{Mt 25:21}] participer à la j. de ton Seign.
\item[\vref{Lu 10:17}] soixante-dix revinrent avec j., disant : Seign., les
\item[\vref{Lu 15:7}] aura plus de j. ds le ciel
\item[\vref{Jn 3:29}] c'est pourquoi cette j. que j'ai est
\item[\vref{Jn 8:56}] a tressailli de j. de ce qu'il
\item[\vref{Jn 15:11}] afin que ma j. demeure en vs.,
\item[\vref{Jn 16:22}] et personne ne vs. ôtera votre j.
\item[\vref{Jn 20:20}] furent ds la j. en voyant le
\item[\vref{Ac 2:26}] est ds la j., et ma langue
\item[\vref{Ac 13:52}] étaient remplis de j. et du Saint-Esprit.
\item[\vref{Ac 14:17}] et en remplissant nos cœurs de j.
\item[\vref{2 Co 9:7}] Dieu aime celui qui donne avec j.
\item[\vref{Ga 5:22}] la charité, la j., la paix, la
\item[\vref{Ph 2:2}] rendez ma j. parfaite, ayant un
\item[\vref{1 Th 1:6}] reçu avec la j. du Saint-Esprit, la
\item[\vref{1 Th 2:20}] vs. êtes notre gloire et notre j.
\item[\vref{Hé 12:2}] échange de la j. qui lui était
\item[\vref{Ja 1:2}] sujet d'une parfaite j. qnd vs. êtes
\item[\vref{1 Pi 1:8}] vs. réjouissez d'une j. ineffable et glorieuse,
\end{listverse}

\ConcordanceEntry{Jojakim}
\vspace{-2mm}
\begin{listverse}
\item[\vref{2 R 23:34}] en celui de J.. Il prit Joachaz,
\end{listverse}
\begin{legend}
\NoAutoSpaceBeforeFDP{
\item Roi de Juda : 2 R 23:34
\item Rejet de la Parole de Yahweh : Jé 26:20-23; 36:2,23
\item Condamnation par Yahweh : Jé 22:18-19; Jé 36:30-31
}
\end{legend}

\ConcordanceEntry{Jojakin}
\vspace{-2mm}
\begin{listverse}
\item[\vref{2 R 24:6}] ses pères. Et J., son fils, régna
\end{listverse}
\begin{legend}
\NoAutoSpaceBeforeFDP{
\item Roi de Juda : 2R 24:6
\item Captivité à Babylone : 2R 24:12,15; 25:27-30
}
\end{legend}

\ConcordanceEntry{Jokébed}
\vspace{-2mm}
\begin{listverse}
\item[\vref{Ex 6:20}] Or Amram prit J., sa tante, pour
\item[\vref{No 26:59}] fem. d'Amram était J., fille de Lévi,
\end{listverse}

\ConcordanceEntry{Jonadab}
\vspace{-2mm}
\begin{listverse}
\item[\vref{2 S 13:3}] un ami, nommé J., fils de Schimea,
\item[\vref{2 S 13:32}] J., fils de Schimea, frère de David,
\item[\vref{2 R 10:15}] là, il rencontra J., fils de Récab,
\item[\vref{2 R 10:23}] Alors Jéhu, et J., fils de Récab,
\item[\vref{Jé 35:6}] de vin ; car J., fils de Récab,
\item[\vref{Jé 35:19}] le Dieu d'Israël : J., fils de Récab,
\end{listverse}

\ConcordanceEntry{Jonas}
\vspace{-2mm}
\begin{listverse}
\item[\vref{Jon 1:1}] Yahweh vint à J., fils d'Amitthaï, en
\item[\vref{Jon 1:15}] Alors ils prirent J., et le jetèrent
\item[\vref{Jon 2:1}] grand poisson d'engloutir J., et Jonas fut
\item[\vref{Jon 2:11}] le poisson vomit J. sur la terre
\item[\vref{Jon 3:1}] Yahweh vint à J. une seconde fois,
\item[\vref{Jon 4:1}] déplut fortement à J., et il fut
\item[\vref{Jon 4:6}] croître au-dessus de J., pour donner de
\item[\vref{Jon 4:8}] la tête de J., au point qu'il
\item[\vref{Mt 12:39}] que celui de J., le prophète.
\item[\vref{Mt 12:40}] de mm que J. fut trois jours
\end{listverse}

\ConcordanceEntry{Jonathan}
\vspace{-2mm}
\begin{listverse}
\item[\vref{1 S 13:16}] avec son fils J., et le peuple
\end{listverse}
\begin{legend}
\NoAutoSpaceBeforeFDP{
\item Fils du roi Saül : 1 S 14:1
\item Victoire sur les Philistins : : 1 S 13:3; 14:13-15
\item Ami de David : 1 S 18:1-4; 23:16-18; 2 S 1:26
\item Tué par les Philistins : 1 S 31:2; 2 S 1:17
\item Fils d’Abiathar : 2 S 15:27; 1 R 1:42
\item Lévite : Jg 17:7-12; Jg 18:30-31
}
\end{legend}

\ConcordanceEntry{Jonc}
\vspace{-2mm}
\begin{listverse}
\item[\vref{Ex 2:3}] une arche de j., et l'enduisit de
\item[\vref{Job 8:11}] sans marais ? Le j. pousse-t-il sans eau ?
\item[\vref{Job 9:26}] des barques de j. ; com. un aigle
\item[\vref{Job 40:21}] Mettras-tu un j. ds ses narines ?
\item[\vref{Es 18:2}] des navires de j., voguant à la
\item[\vref{Es 58:5}] tête com. le j. et en étendant
\end{listverse}

\ConcordanceEntry{Joppé}
\vspace{-2mm}
\begin{listverse}
\item[\vref{Ac 9:36}] y avait à J. une fem. disciple,
\item[\vref{Ac 10:5}] des gens à J., et fais venir
\item[\vref{Ac 11:5}] la ville de J., et, pendant que
\end{listverse}

\ConcordanceEntry{Joram}
\vspace{-2mm}
\begin{listverse}
\item[\vref{1 R 22:51}] son père. Et J., son fils, régna
\end{listverse}

\ConcordanceEntry{Josaphat}
\vspace{-2mm}
\begin{listverse}
\item[\vref{1 R 22:41}] Or J., fils d'Asa, régna sur Juda, la
\end{listverse}
\begin{legend}
\NoAutoSpaceBeforeFDP{
\item Roi de Juda : 1 R 22:41; 2 Ch 17:1
\item Piété : 2 Ch 17:1-13; 22:9
\item Alliance avec Achab : 2 Ch 18:1-3
\item Victoire sur Moab : 2 Ch 20:1, 22-25; 1 R 22:43-51; 2 R 3:6-7
}
\end{legend}

\ConcordanceEntry{Joseph}
\vspace{-2mm}
\begin{listverse}
\item[\vref{Ge 30:24}] le nom de J., en disant : Que
\end{listverse}
\begin{legend}
\NoAutoSpaceBeforeFDP{
\item Fils de Jacob et  Rachel : Ge 30:24
\item Les songes : Ge 37:5-10
\item Jalousie de ses frères : Ge 37:8,11
\item Vendu à des marchands : Ge 37:23-28
\item Esclave de Potiphar : Ge 39:1
\item Fidèle à Yahweh devant la tentation : Ge 39:7-12
\item Joseph demeure en prison : Ge 39:19,20
\item Les songes des deux eunuques : Ge 40:9-23
\item Le songe de Pharaon : Ge 41:25-36
\item Joseph élevé : Ge 41:41
\item Joseph révèle son identité :  Ge 45:1-5
\item Le pardon et la venue de Jacob : Ge 45:8-12
\item Prophéties : Ge 50:25; Hé 11:22
\item Epoux de Marie : Mt 1:16,19-21
\item Fuite en Egypte : Mt 2:13,14
\item Installation à Nazareth : Mt 2:19-23
\item Homme d’Arimathée : Mc 15:43-46
}
\end{legend}

\ConcordanceEntry{Josias}
\vspace{-2mm}
\begin{listverse}
\item[\vref{2 R 22:1}] J. était âgé de huit ans lorsqu'il
\end{listverse}
\begin{legend}
\NoAutoSpaceBeforeFDP{
\item Roi de Juda : 2 R 22:1
\item Prophéties des hommes de Dieu : 1 R 13:2; 23:16
\item Droiture et piété : 2 R 22:2
\item Restauration du temple : 2 Ch 34:8
\item Redécouverte du livre de la loi : 2 Ch 34:14
\item Renouvellement de l’alliance avec Yahweh : 2 Ch34:31
\item Haine de l’idôlatrie : 2 Ch 34:3,33
\item Rétablissement de la Pâque : 2 Ch 35:1
\item Tué par Néco roi d’Egypte : 2 Ch 35:20-27
}
\end{legend}

\ConcordanceEntry{Josué}
\vspace{-2mm}
\begin{listverse}
\item[\vref{Ex 24:13}] se leva avec J. qui le servait ;
\end{listverse}
\begin{legend}
\NoAutoSpaceBeforeFDP{
\item Serviteur de Moïse : Ex 24:13; 33:11
\item Victoire contre Amalek : Ex 17:8-13
\item Douze espions envoyés : No 13:8; 13:16
\item Un homme d’Esprit : No 27:18; De 34:9
\item Successeur de Moïse : No 27:18-23; De 1:38
\item Appel de Yahweh : Jos 1:1
\item Traversée du Jourdain : Jos 3:1, 16
\item Elévation par Yahweh devant Israël : Jos 4:14
\item Jéricho livré à Israël : Jos 6:2,24
\item Conquête des territoires : Jos 10:10-42; 11:16-23
\item Partage des territoires : Jos 13:7; 18:10
\item Yahweh donne le repos à Israël : Jos 21:44; Hé 4:8
\item Avertissements de Josué : Jos 23
\item Sa mort : Jos 24:29; Jg 2:8
}
\end{legend}

\ConcordanceEntry{Jotham}
\vspace{-2mm}
\begin{listverse}
\item[\vref{Jg 9:5}] ne resta que J., le plus jeune
\end{listverse}

\ConcordanceEntry{Joue}
\vspace{-2mm}
\begin{listverse}
\item[\vref{1 R 22:24}] Michée sur la j., et dit : Par
\item[\vref{2 Ch 18:23}] Michée sur la j., et dit : Par
\item[\vref{Job 16:10}] soufflets sur la j. pour m'outrager ; ils
\item[\vref{Ps 3:8}] frappes à la j. ts mes ennemis,
\item[\vref{Es 50:6}] frappaient et mes j. à ceux qui
\item[\vref{La 3:30}] Il présentera la j. à celui qui
\item[\vref{Mi 4:14}] d'Israël avec la verge sur la j.
\item[\vref{Mt 5:39}] frappe sur ta j. droite, présente-lui aussi
\item[\vref{Lu 6:29}] frappe sur une j., présente-lui aussi l'autre.
\end{listverse}

\ConcordanceEntry{Jouer}
\vspace{-2mm}
\begin{listverse}
\item[\vref{Ex 32:6}] boire, puis ils se levèrent pour j.
\item[\vref{Jg 16:27}] hommes et femmes, qui regardaient Samson j.
\item[\vref{1 S 16:16}] hom. qui sache j. de la harpe ;
\item[\vref{1 S 16:17}] qui sache bien j. et amenez-le-moi.
\item[\vref{1 S 16:18}] Bethléhémite, qui sait j. des instruments, il
\item[\vref{Ps 104:26}] que tu as formé pour y j.
\item[\vref{1 Co 10:7}] boire, puis ils se levèrent pour j.
\item[\vref{1 Co 14:7}] ce qui est j. sur la flûte
\end{listverse}

\ConcordanceEntry{Joug}
\vspace{-2mm}
\begin{listverse}
\item[\vref{Ge 27:40}] tu briseras son j. de dessus ton
\item[\vref{De 28:48}] Il mettra un j. de fer sur
\item[\vref{Jé 2:20}] j'ai brisé ton j., et rompu tes
\item[\vref{Jé 27:2}] liens et des j., et mets-les sur
\item[\vref{La 3:27}] de porter le j. ds sa jeunesse.
\item[\vref{Os 11:4}] qui enlèveraient le j. de dessus lr.
\item[\vref{Mt 11:29}] Prenez mon j. sur vs. et
\item[\vref{Mt 11:30}] Car mon j. est doux et mon fardeau est
\item[\vref{Mt 19:6}] Dieu a mis ensemble sous un j.
\item[\vref{Mc 10:9}] Dieu a mis ensemble sous un j.
\item[\vref{Ac 15:10}] aux disciples un j. que ni nos
\item[\vref{2 Co 6:14}] pas un mm j. avec les infidèles,
\item[\vref{Ga 5:1}] plus sous le j. de la servitude.
\item[\vref{1 Ti 6:1}] sont sous le j. sachent qu'ils doivent
\end{listverse}

\ConcordanceEntry{Jouir}
\vspace{-2mm}
\begin{listverse}
\item[\vref{Jos 22:9}] les avait fait j., suivant ce que
\item[\vref{Ps 34:13}] la prolonger pour j. du bonheur ?
\item[\vref{Ps 128:2}] Tu j. du travail de tes mains ; tu
\item[\vref{Ec 2:24}] que son âme j. du bien ds
\item[\vref{Ec 5:17}] boire et de j. du bien-être de
\item[\vref{Es 17:11}] l'on voulait en j., et il y
\item[\vref{1 Ti 6:17}] donne ttes choses abondamment pour en j.
\item[\vref{Hé 11:25}] Dieu, que de j. pour un peu
\end{listverse}

\ConcordanceEntry{Jour}
\vspace{-2mm}
\begin{listverse}
\item[\vref{Ge 1:5}] appela la lumière j., et il appela
\item[\vref{Ge 2:2}] acheva au septième j. son œuvre qu'il
\item[\vref{Ge 2:3}] bénit le septième j., et le sanctifia,
\item[\vref{Ge 2:17}] point, car le j. où tu en
\item[\vref{Ge 3:5}] sait que le j. où vs. en
\item[\vref{Ge 3:8}] au vent du j. la voix de
\item[\vref{Ge 5:1}] d'Adam, depuis le j. où Dieu créa
\item[\vref{Ge 8:22}] que dureront les j. de la terre,
\item[\vref{Ge 15:18}] En ce j.-là, Yahweh traita alliance avec Abram,
\item[\vref{Ge 29:20}] yeux com. quelques j., parce qu'il l'aimait.
\item[\vref{Ge 40:20}] arriva au troisième j., qui était le
\item[\vref{Ge 47:9}] à Pharaon : Les j. des années de
\item[\vref{Ex 12:15}] mangerez pendant sept j. des pains sans
\item[\vref{Ex 20:10}] Mais le septième j. est le repos
\item[\vref{Ex 20:11}] Car en six j. Yahweh a fait
\item[\vref{Lé 23:32}] un sabbat, un j. de repos, et
\item[\vref{No 11:19}] mangerez pas un j., ni deux jours,
\item[\vref{No 28:17}] Et au quinzième j. du mm mois
\item[\vref{No 28:26}] Et au j. des prémices, qnd vs. offrirez à
\item[\vref{De 5:12}] Garde le j. du sabbat pour le sanctifier, com.
\item[\vref{De 29:4}] Mais, jusqu'à ce j., Yahweh ne vs.
\item[\vref{De 32:7}] Souviens-toi des anciens j., considère les années,
\item[\vref{Jos 1:8}] bouche, mais médite-le j. et nuit, pour
\item[\vref{Jos 5:10}] Pâque le quatorzième j. du mois, sur
\item[\vref{Jos 10:14}] point eu de j. semblable à celui-là,
\item[\vref{1 R 2:37}] sache que, le j. où tu en
\item[\vref{1 R 20:29}] Sept j. durant ils campèrent vis-à-vis les uns
\item[\vref{2 R 19:3}] parle Ezéchias : Ce j. est un jour
\item[\vref{1 Ch 29:15}] pères ; et nos j. sont com. l'ombre
\item[\vref{Né 8:9}] le peuple : Ce j. est consacré à
\item[\vref{Ps 1:2}] médite sa loi j. et nuit !
\item[\vref{Ps 19:3}] Un j. en instruit un autre jour, et
\item[\vref{Ps 39:5}] mesure de mes j. ; que je sache
\item[\vref{Ps 39:6}] as réduit mes j. à la largeur
\item[\vref{Ps 42:4}] sont ma nourriture j. et nuit, qnd
\item[\vref{Ps 55:24}] moitié de leurs j.. C'est en toi
\item[\vref{Ps 84:11}] mieux vaut un j. ds tes parvis,
\item[\vref{Ps 90:4}] yeux, com. le j. d'hier qui est
\item[\vref{Ps 102:25}] milieu de mes j., toi dont les
\item[\vref{Ps 139:16}] inscrits ts les j. qui m'étaient destinés.
\item[\vref{Pr 3:2}] t'apportent de longs j. et des années
\item[\vref{Pr 27:1}] pas ce qu'un j. peut enfanter.
\item[\vref{Ec 7:10}] vient que les j. passés ont été
\item[\vref{Ec 12:3}] Créateur pendant les j. de ta jeunesse,
\item[\vref{Es 2:12}] y a un j. assigné par Yahweh
\item[\vref{Es 13:9}] Voici, le j. de Yahweh arrive,
\item[\vref{Es 34:10}] point éteinte ni j. ni nuit ; sa
\item[\vref{Es 53:10}] et prolongera ses j. ; et le bon
\item[\vref{Jé 33:20}] alliance avec le j. et mon alliance
\item[\vref{Joë 2:31}] grand et terrible j. de Yahweh vienne.
\item[\vref{Jon 2:1}] du poisson trois j. et trois nuits.
\item[\vref{So 1:14}] Le grand j. de Yahweh est proche, il est
\item[\vref{Za 3:9}] j'ôterai en un j. l'iniquité de ce
\item[\vref{Za 14:7}] Ce sera un j. unique, connu de
\item[\vref{Mal 3:2}] pourra soutenir le j. de sa venue ?
\item[\vref{Mal 4:5}] avant que le j. grand et redoutable
\item[\vref{Mt 6:34}] lui-mm. A chaque j. suffit sa peine.
\item[\vref{Mt 7:22}] diront en ce j.-là : Seign. ! Seign. !
\item[\vref{Mt 10:15}] moins rigoureusement au j. du jugement que
\item[\vref{Mt 11:22}] que vs., au j. du jugement.
\item[\vref{Mt 11:24}] que toi, au j. du jugement.
\item[\vref{Mt 16:21}] mort, et qu'il ressuscite le troisième j.
\item[\vref{Mt 17:23}] mais le troisième j. il ressuscitera. Et
\item[\vref{Mt 20:19}] et le troisième j. il ressuscitera.
\item[\vref{Mt 24:36}] qui est du j. et de l'heure,
\item[\vref{Mt 24:38}] com. ds les j. avant le déluge,
\item[\vref{Mt 25:13}] savez ni le j. ni l'heure en
\item[\vref{Mt 28:20}] vs. ts les j. jusqu'à la fin
\item[\vref{Mc 1:21}] Capernaüm. Et le j. du sabbat, Jésus
\item[\vref{Mc 3:2}] le guérirait le j. du sabbat, afin
\item[\vref{Mc 9:31}] à mort, il ressuscitera le troisième j.
\item[\vref{Lu 6:7}] une guérison le j. du sabbat ; c'était
\item[\vref{Lu 9:22}] mort et qu'il ressuscite le troisième j.
\item[\vref{Lu 9:23}] se charge chaque j. de sa croix,
\item[\vref{Lu 10:12}] dis qu'en ce j. Sodome sera traitée
\item[\vref{Lu 10:14}] que vs. au j. du jugement.
\item[\vref{Lu 11:3}] Donne-ns. chaque j. notre pain quotidien.
\item[\vref{Lu 17:22}] ses disciples : Des j. viendront où vs.
\item[\vref{Lu 17:30}] de mm au j. où le Fils
\item[\vref{Lu 17:31}] En ce j.-là, que celui qui sera sur
\item[\vref{Jn 6:40}] pourquoi je le ressusciterai au dernier j.
\item[\vref{Jn 7:23}] la circoncision le j. du sabbat, afin
\item[\vref{Jn 7:37}] Le dernier j., le grand jour
\item[\vref{Jn 9:4}] tandis qu'il est j., les œuvres de
\item[\vref{Jn 14:20}] En ce j.-là, vs. connaîtrez que je suis
\item[\vref{Jn 19:31}] la croix le j. du sabbat, parce
\item[\vref{Ac 2:1}] Et com. le j. de la Pentecôte
\item[\vref{Ac 2:15}] car c'est la troisième heure du j.
\item[\vref{Ac 2:20}] grand et notable j. du Seign. vienne.
\item[\vref{Ac 2:29}] existe encore parmi ns. jusqu'à ce j.
\item[\vref{Ac 7:8}] circoncit le huitième j.. Isaac engendra Jacob,
\item[\vref{Ac 10:3}] neuvième heure du j., il vit clairement
\item[\vref{Ac 20:16}] à Jérus. le j. de la Pentecôte.
\item[\vref{Ac 27:33}] Avant que le j. paraisse, Paul les
\item[\vref{Ro 8:22}] que, jusqu'à ce j., tte la création
\item[\vref{Ro 14:6}] a égard au j., y a égard
\item[\vref{1 Co 1:8}] soyez irrépréhensibles au j. de notre Seign.
\item[\vref{1 Co 3:13}] car le j. la fera connaître, parce qu'elle sera
\item[\vref{1 Co 5:5}] soit sauvé au j. du Seign. Jésus.
\item[\vref{2 Co 6:2}] t'ai secouru au j. du salut. Voici
\item[\vref{Ga 4:10}] Vous observez les j., les mois, les
\item[\vref{Ph 1:10}] irréprochables pour le j. de Christ,
\item[\vref{1 Th 2:9}] travaillant nuit et j., pour n'être à
\item[\vref{1 Th 5:2}] bien que le j. du Seign. viendra
\item[\vref{1 Th 5:4}] pour que ce j.-là vs. surprenne
\item[\vref{2 Th 2:2}] com. si le j. de Christ était
\item[\vref{1 Ti 5:5}] persévère nuit et j. ds les supplications
\item[\vref{Hé 10:25}] plus que vs. voyez approcher le j.
\item[\vref{1 Pi 3:10}] et voir des j. heureux, qu'il préserve
\item[\vref{2 Pi 3:7}] le feu au j. du jugement et
\item[\vref{2 Pi 3:8}] pas ceci, qu'un j. est dvt le
\item[\vref{2 Pi 3:10}] Or le j. du Seign. viendra com. un voleur
\item[\vref{Jud 1:6}] jugement du grand j., les anges qui
\item[\vref{Ap 1:10}] en esprit au j. du Seign., et
\item[\vref{Ap 2:10}] affliction de dix j.. Sois fidèle jusqu'à
\item[\vref{Ap 4:8}] pas de dire j. et nuit : Saint !
\item[\vref{Ap 6:17}] car le grand j. de sa colère
\item[\vref{Ap 7:15}] ils le servent j. et nuit ds
\item[\vref{Ap 8:12}] soit obscurci ; le j. fut privé d'un
\item[\vref{Ap 12:10}] dvt notre Dieu j. et nuit, a
\item[\vref{Ap 14:11}] de repos ni j. ni nuit, ceux
\item[\vref{Ap 16:14}] de ce grand j. du Dieu Tout-Puissant.
\item[\vref{Ap 18:8}] en un mm j., et elle sera
\item[\vref{Ap 20:10}] ils seront tourmentés j. et nuit, aux
\item[\vref{Ap 21:25}] fermeront pas le j., car il n'y
\end{listverse}

\ConcordanceEntry{Jourdain}
\vspace{-2mm}
\begin{listverse}
\item[\vref{Ge 32:10}] j'ai passé ce J. avec mon bâton,
\item[\vref{Jos 3:8}] des eaux du J., vs. vs. arrêterez
\item[\vref{Jos 3:17}] au milieu du J., pendant que tt
\item[\vref{2 S 17:22}] ils passèrent le J. ; à la lumière
\item[\vref{2 R 2:6}] Yahweh m'envoie jusqu'au J.. Mais Elisée répondit :
\item[\vref{2 R 2:13}] et s'arrêta sur le bord du J.
\item[\vref{2 R 5:10}] fois ds le J., et ta chair
\item[\vref{1 Ch 19:17}] Israël, passa le J., alla au-dvt d'eux
\item[\vref{Job 40:18}] pas qnd le J. se dégorgerait ds
\item[\vref{Ps 114:3}] et s'enfuit, le J. retourna en arrière.
\item[\vref{Ps 114:5}] t'enfuir ? Et toi, J., pour retourner en
\item[\vref{Jé 12:5}] paix, que feras-tu dvt l'orgueil du J. ?
\item[\vref{Mt 3:5}] des environs du J., vinrent à lui ;
\item[\vref{Lu 3:3}] des environs du J., prêchant le baptême
\item[\vref{Lu 4:1}] Saint-Esprit, revint du J., et il fut
\item[\vref{Jn 1:28}] Béthanie, au-delà du J., où Jean baptisait.
\end{listverse}

\ConcordanceEntry{Journée}
\vspace{-2mm}
\begin{listverse}
\item[\vref{Ex 5:3}] de faire trois j. de marche ds
\item[\vref{1 S 14:24}] furent épuisés cette j.-là. Mais Saül
\item[\vref{Ps 95:8}] com. à la j. de Massa, au
\item[\vref{Ps 118:24}] C'est ici la j. que Yahweh a
\item[\vref{Es 21:8}] sentinelle tte la j. et je suis
\item[\vref{Os 2:2}] pays ; car la j. de Jizreel sera
\item[\vref{Jon 3:4}] le chemin d'une j. de marche ; il
\item[\vref{Mt 20:6}] ici tte la j. sans rien faire ?
\item[\vref{Lu 2:44}] ils marchèrent une j., puis ils le
\end{listverse}

\ConcordanceEntry{Joyeux}
\vspace{-2mm}
\begin{listverse}
\item[\vref{Ru 3:7}] son cœur était j., il vint se
\item[\vref{Ps 90:14}] que ns. soyons j. tt le long
\item[\vref{Pr 15:13}] Le cœur j. rend le visage beau, mais l'esprit
\item[\vref{Pr 17:22}] Le cœur j. est un remède, mais l'esprit abattu
\item[\vref{Jn 14:28}] vs. seriez certes j. de ce que
\item[\vref{Ac 5:41}] dvt le sanhédrin, j. d'avoir été jugés
\item[\vref{Ro 12:12}] Soyez j. ds l'espérance ; patients ds la tribulation ;
\item[\vref{2 Co 6:10}] et toutefois toujours j. ; com. pauvres et
\item[\vref{1 Th 5:16}] Soyez toujours j.
\end{listverse}

\ConcordanceEntry{Jubal}
\vspace{-2mm}
\begin{listverse}
\item[\vref{Ge 4:21}] son frère était J. : Il fut le
\end{listverse}

\ConcordanceEntry{Jubilé}
\vspace{-2mm}
\begin{listverse}
\item[\vref{Lé 25:11}] sera l'année du j.. Vous ne sèmerez
\item[\vref{Lé 25:54}] sortira l'année du j., lui et ses
\item[\vref{No 36:4}] qnd viendra le j. pour les enfants
\end{listverse}

\ConcordanceEntry{Juda}
\vspace{-2mm}
\begin{listverse}
\item[\vref{Ge 29:35}] le nom de J.. Et elle cessa
\end{listverse}
\begin{legend}
\NoAutoSpaceBeforeFDP{
\item Supplications auprès de Joseph : Ge 44:18-34
\item Bénédictions : Ge 49:8-12; 38:1; De 33:7; 1 Ch 5:2; 28:4
\item Tribu : No 1:26-:27; 26:22
\item Héritage: Jos 15; Jos 19:1,9; Jg 1:1-20
\item Choix de Yahweh : Ps 78:68
\item Le sanctuaire de Yahweh: Ps 78:69; 114:2
\item David pour roi : 2 S 5:1-4
\item Royaume : 1 R 12:20-21; 14:21
\item Péché : Jé 2:28; 3:10; 17:1; Ez 9:9
\item Captivité : 2R 25:22; Jé 40:1
\item Promesses : Jé 23:6; 29:11-14; 33:7; 2 R 17:18
\item Autres : 2 Ch 12:12; Ps 60:9; Ap 5:5
}
\end{legend}

\ConcordanceEntry{Judas}
\vspace{-2mm}
\begin{listverse}
\item[\vref{Mt 10:4}] le Cananite, et J. Iscariot, celui qui
\end{listverse}

\ConcordanceEntry{Jude}
\vspace{-2mm}
\begin{listverse}
\item[\vref{Jn 14:22}] J., non pas Iscariot, lui dit : Seign.,
\item[\vref{Jud 1:1}] J., serviteur de Jésus-Christ, et frère de
\end{listverse}

\ConcordanceEntry{Judée}
\vspace{-2mm}
\begin{listverse}
\item[\vref{Mt 2:5}] Bethléhem, ville de J. ; car voici ce
\end{listverse}

\ConcordanceEntry{Juge}
\vspace{-2mm}
\begin{listverse}
\item[\vref{Ge 16:5}] Que Yahweh soit j. entre moi et
\item[\vref{Ex 2:14}] établi prince et j. sur ns. ? Veux-tu
\item[\vref{De 16:18}] Tu t'établiras des j. et des officiers
\item[\vref{Jg 2:16}] lr. suscita des j. et ils les
\item[\vref{1 S 8:1}] ses fils pour j. sur Israël.
\item[\vref{Ps 2:10}] ayez de l'intelligence ! J. de la terre,
\item[\vref{Ps 7:12}] est un juste j., Dieu s'irrite en
\item[\vref{Ps 50:6}] justice parce que Dieu est le j.. Sélah.
\item[\vref{Ps 94:2}] Toi, j. de la terre, élève-toi ! Rends aux
\item[\vref{Mi 4:14}] on frappera le j. d'Israël avec la
\item[\vref{Mt 12:27}] C'est pourquoi ils seront eux-mêmes vos j.
\item[\vref{Lu 18:2}] une ville un j. qui ne craignait
\item[\vref{Lu 18:6}] dit : Ecoutez ce que dit le j. inique.
\item[\vref{Jn 5:22}] le Père ne j. personne, mais il
\item[\vref{Ac 10:42}] établi par Dieu, j. des vivants et
\item[\vref{Ac 13:20}] lr. donna des j., jusqu'à Samuel, le
\item[\vref{Ac 17:34}] et crurent : Denys, j. de l'Aéropage, une
\item[\vref{Ro 14:3}] mange pas ne j. pas celui qui
\item[\vref{1 Co 4:4}] Celui qui me j., c'est le Seign.
\item[\vref{2 Ti 4:8}] Seign., le juste J., me la donnera
\item[\vref{Hé 4:12}] mœlles ; et elle j. les pensées et
\item[\vref{Hé 12:23}] qui est le j. de ts, et
\item[\vref{Ja 4:11}] la loi et j. la loi. Or
\item[\vref{Ja 5:9}] condamnés. Voici, le J. se tient à
\item[\vref{1 Pi 1:17}] Père celui qui j. selon l'œuvre de
\item[\vref{1 Pi 2:23}] à celui qui j. justement ;
\item[\vref{Ap 19:11}] VERITABLE, et il j. et combat avec
\end{listverse}

\ConcordanceEntry{Jugement}
\vspace{-2mm}
\begin{listverse}
\item[\vref{Ex 12:12}] et j'exercerai des j. sur ts les
\item[\vref{Ex 28:30}] le pectoral de j. l'urim et le
\item[\vref{Lé 19:15}] d'iniquité ds vos j.. Tu n'auras point
\item[\vref{Ps 10:5}] tt temps ; tes j. sont éloignés de
\item[\vref{Ps 72:1}] Dieu, donne tes j. au roi et
\item[\vref{Ps 110:6}] Il exercera le j. sur les nations,
\item[\vref{Ps 119:75}] Yahweh, que tes j. sont justes, et
\item[\vref{Ps 143:2}] N'entre point en j. avec ton serviteur !
\item[\vref{Ec 12:1}] ttes ces choses Dieu t'amènera en j.
\item[\vref{Es 5:16}] haut élevé en j., et le Dieu
\item[\vref{Es 26:9}] car lorsque tes j. s'exercent sur la
\item[\vref{Es 42:1}] il manifestera le j. aux nations.
\item[\vref{Ha 1:4}] on rend des j. corrompus.
\item[\vref{Ha 1:12}] pour exécuter tes j. ; et toi, mon
\item[\vref{So 3:5}] en lumière son j., il n'y manque
\item[\vref{Jn 5:22}] personne, mais il a donné tt j. au Fils,
\item[\vref{Jn 5:30}] j'entends, et mon j. est juste, car
\item[\vref{Jn 8:16}] je juge, mon j. est digne de
\item[\vref{Jn 9:39}] pour exercer le j., afin que ceux
\item[\vref{Jn 12:31}] est venu le j. de ce monde ;
\item[\vref{Jn 16:8}] de péché, de justice, et de j. :
\item[\vref{Ro 2:2}] savons que le j. de Dieu est
\item[\vref{Ro 2:3}] tu échapperas au j. de Dieu ?
\item[\vref{Ro 11:33}] Dieu ! que ses j. sont insondables, et
\item[\vref{2 Th 1:5}] démonstration du juste j. de Dieu, afin
\item[\vref{1 Ti 3:6}] tombe sous le j. du diable.
\item[\vref{Hé 6:2}] la résurrection des morts, et du j. éternel.
\item[\vref{Hé 9:27}] fois, et après cela vient le j.,
\item[\vref{Hé 10:27}] attente terrible du j. et l'ardeur d'un
\item[\vref{1 Pi 4:17}] temps que le j. commence par la
\item[\vref{2 Pi 2:4}] livrés pour y être gardés jusqu'au j. ;
\item[\vref{Jud 1:6}] liens éternels, jusqu'au j. du grand jour,
\item[\vref{Jud 1:9}] contre lui un j. blasphématoire, mais il
\item[\vref{Ap 14:7}] l'heure de son j. est venue ; et
\item[\vref{Ap 17:1}] te montrerai le j. de la grande
\end{listverse}

\ConcordanceEntry{Juger}
\vspace{-2mm}
\begin{listverse}
\item[\vref{Ge 18:25}] point. Celui qui j. tte la terre
\item[\vref{Ex 18:13}] Moïse siégeait pour j. le peuple, et
\item[\vref{Lé 19:15}] grand, mais tu j. ton prochain selon
\item[\vref{De 16:18}] Tu t'établiras des j. et des officiers
\item[\vref{Ps 35:24}] J.-moi selon ta justice, Yahweh, mon
\item[\vref{Ps 58:2}] fils des hommes, j.-vs. avec droiture ?
\item[\vref{Ps 58:12}] un Dieu qui j. sur la terre.
\item[\vref{Ps 67:5}] joie ; car tu j. les peuples avec
\item[\vref{Ps 75:3}] j'aurai fixé, je j. avec droiture.
\item[\vref{Ec 3:16}] lieu établi pour j., il y a
\item[\vref{Ec 3:17}] mon cœur : Dieu j. le juste et
\item[\vref{Es 3:13}] tient debout pour j. les peuples.
\item[\vref{Es 11:3}] Yahweh, il ne j. point sur l'apparence
\item[\vref{Ez 21:35}] fourreau ? Je te j. sur le lieu
\item[\vref{Joë 3:12}] je siégerai pour j. ttes les nations
\item[\vref{Mt 7:1}] Ne j. pas, afin que vs. ne soyez
\item[\vref{Lu 22:30}] des trônes, pour j. les douze tribus
\item[\vref{Jn 5:22}] le Père ne j. personne, mais il
\item[\vref{Jn 5:27}] le pouvoir de j. parce qu'il est
\item[\vref{Jn 7:24}] Ne j. pas selon les apparences, mais jugez
\item[\vref{Jn 12:47}] je ne le j. pas ; car je
\item[\vref{Jn 12:48}] paroles a son j. ; la parole que
\item[\vref{Ac 17:31}] jour où il j. le monde selon
\item[\vref{Ro 2:1}] tu sois, qui j. les autres, tu
\item[\vref{Ro 14:10}] Mais toi, pourquoi j.-tu ton frère ?
\item[\vref{1 Co 2:15}] et il n'est j. par personne.
\item[\vref{1 Co 4:4}] Celui qui me j., c'est le Seign.
\item[\vref{1 Co 4:5}] C'est pourquoi ne j. de rien avant
\item[\vref{1 Co 5:12}] effet, qu'ai-je à j. ceux du dehors ?
\item[\vref{1 Co 6:2}] que les saints j. le monde ? Or
\item[\vref{1 Co 6:5}] seul qui puisse j. entre frères ?
\item[\vref{1 Co 11:31}] si ns. ns. j. ns.-mêmes, ns. ne
\item[\vref{2 Ti 4:1}] Jésus-Christ, qui doit j. les vivants et
\item[\vref{Hé 4:12}] mœlles ; et elle j. les pensées et
\item[\vref{Ja 4:11}] la loi et j. la loi. Or
\item[\vref{1 Pi 2:23}] à celui qui j. justement ;
\item[\vref{1 Pi 4:6}] afin qu'ils soient j. selon les hommes
\item[\vref{Jud 1:15}] par millions, pour j. ts les hommes,
\item[\vref{Ap 6:10}] et véritable, ne j.-tu pas et
\item[\vref{Ap 11:18}] est venu de j. les morts, et
\item[\vref{Ap 19:11}] VERITABLE, et il j. et combat avec
\item[\vref{Ap 20:4}] qui l'autorité de j. fut donnée. Et
\item[\vref{Ap 20:13}] et ils furent j. chacun selon ses
\end{listverse}

\ConcordanceEntry{Juif}
\vspace{-2mm}
\begin{listverse}
\item[\vref{Est 6:13}] la race des J., tu n'auras pas
\item[\vref{Est 8:7}] Esther et au J. Mardochée : Voici, j'ai
\item[\vref{Za 8:23}] robe d'un hom. J., et diront : Nous
\item[\vref{Jn 4:9}] toi qui es J., me demandes-tu à
\item[\vref{Jn 18:35}] Pilate répondit : Suis-je J. ? Ta nation et
\item[\vref{Ac 21:39}] dit : Je suis J., de Tarse, citoyen
\item[\vref{Ro 2:9}] le mal, du J. premièrement, puis aussi
\item[\vref{Ro 2:29}] Mais le J., c'est celui qui l'est intérieurement ; et
\item[\vref{Ro 10:12}] différence entre le J. et le Grec,
\item[\vref{1 Co 9:20}] été com. un J. avec les Juifs,
\item[\vref{Ga 2:14}] toi qui es J., tu vis com.
\item[\vref{Ga 3:28}] a plus ni J. ni Grec, il
\item[\vref{Col 3:11}] ni Grec ni J., ni circoncis ni
\end{listverse}

\ConcordanceEntry{Julius}
\vspace{-2mm}
\begin{listverse}
\item[\vref{Ac 27:1}] à un nommé J., centenier d'une cohorte
\item[\vref{Ac 27:3}] à Sidon ; et J., qui traitait Paul
\end{listverse}

\ConcordanceEntry{Jumeaux}
\vspace{-2mm}
\begin{listverse}
\item[\vref{Ge 25:24}] y avait deux j. ds son ventre.
\item[\vref{Ge 38:27}] d'accoucher, voici, des j. étaient ds son
\item[\vref{Ca 4:5}] faons, com. les j. d'une gazelle qui
\item[\vref{Ca 7:4}] com. deux faons j. d'une gazelle.
\end{listverse}

\ConcordanceEntry{Jupiter}
\vspace{-2mm}
\begin{listverse}
\item[\vref{Ac 14:12}] ils appelaient Barnabas J., et Paul Mercure,
\item[\vref{Ac 14:13}] Le prêtre de J., qui était à
\item[\vref{Ac 19:35}] et de son image tombée de J. ?
\end{listverse}

\ConcordanceEntry{Jurer}
\vspace{-2mm}
\begin{listverse}
\item[\vref{Ge 22:16}] dit : Je le j. par moi-mm, parole
\item[\vref{Ge 24:3}] je te ferai j. par Yahweh, le
\item[\vref{De 6:13}] serviras et tu j. par son Nom.
\item[\vref{1 S 14:24}] Saül avait fait j. le peuple, en
\item[\vref{2 S 3:9}] David selon ce que Yahweh a j. à David,
\item[\vref{1 R 8:31}] pour le faire j., et que le
\item[\vref{2 Ch 15:14}] Et ils j. à Yahweh, à haute voix, avec
\item[\vref{Ps 89:36}] J'ai une fois j. par ma sainteté :
\item[\vref{Ps 95:11}] c'est pourquoi j'ai j. ds ma colère,
\item[\vref{Ps 102:9}] furieux contre moi, j. contre moi.
\item[\vref{Ps 110:4}] Yahweh l'a j., et il ne
\item[\vref{Ps 119:106}] J'ai j. et je le tiendrai, d'observer les
\item[\vref{Ec 9:2}] bien ; celui qui j., com. celui qui
\item[\vref{Es 45:23}] Je le j. par moi-mm, la parole est sortie
\item[\vref{Jé 4:2}] Alors tu j. avec vérité, avec droiture et avec
\item[\vref{Jé 5:2}] En cela, ils j. faussement.
\item[\vref{Jé 12:16}] mon peuple, pour j. par mon Nom,
\item[\vref{Da 12:7}] cieux, et il j. par celui qui
\item[\vref{Za 5:4}] de celui qui j. faussement en mon
\item[\vref{Mt 5:34}] dis de ne j. aucunement, ni par
\item[\vref{Mt 23:16}] dites : Si quelqu'un j. par le temple,
\item[\vref{Mt 23:22}] et celui qui j. par le ciel,
\item[\vref{Mt 26:74}] imprécations et à j., en disant : Je
\item[\vref{Mc 14:71}] maudire, et à j., disant : Je ne
\item[\vref{Hé 6:13}] Abraham, ne pouvant j. par un plus
\item[\vref{Hé 6:16}] Or les hommes j. par celui qui
\end{listverse}

\ConcordanceEntry{Juste}
\vspace{-2mm}
\begin{listverse}
\item[\vref{Ge 6:9}] était un hom. j. et intègre en
\item[\vref{Ge 18:23}] Feras-tu périr le j. avec le méchant ?
\item[\vref{Ge 18:24}] y a-t-il cinquante j. ds la ville,
\item[\vref{Ge 18:25}] fasses mourir le j. avec le méchant,
\item[\vref{De 32:4}] ses voies sont j.. C'est un Dieu
\item[\vref{Job 32:1}] parce qu'il se croyait un hom. j.
\item[\vref{Ps 1:6}] la voie des j., mais la voie
\item[\vref{Ps 5:13}] tu bénis le j., ô Yahweh ! Et
\item[\vref{Ps 7:10}] et affermis le j., toi qui sondes
\item[\vref{Ps 37:16}] Mieux vaut au j. le peu qu'il
\item[\vref{Ps 37:25}] point vu le j. abandonné, ni sa
\item[\vref{Ps 51:6}] que tu seras j. ds ta sentence,
\item[\vref{Ps 64:11}] Le j. se réjouira en Yahweh, et se
\item[\vref{Ps 97:11}] faite pour le j., et la joie
\item[\vref{Ps 140:13}] en soit, les j. célébreront ton Nom,
\item[\vref{Ps 141:5}] Que le j. me frappe ! Ce me sera une
\item[\vref{Ps 145:17}] Tsade.] Yahweh est j. ds ttes ses
\item[\vref{Ps 146:8}] qui sont courbés ; Yahweh aime les j.
\item[\vref{Pr 10:16}] L'œuvre du j. est pour la
\item[\vref{Pr 12:5}] Les pensées des j. ne sont que
\item[\vref{Pr 12:21}] aucun outrage aux j., mais les méchants
\item[\vref{Pr 15:6}] la maison du j., mais il y
\item[\vref{Pr 15:28}] Le cœur du j. médite ce qu'il
\item[\vref{Pr 15:29}] mais il exauce la requête des j.
\item[\vref{Pr 18:10}] tour forte, le j. y court et
\item[\vref{Pr 21:15}] joie pour le j. de faire ce
\item[\vref{Pr 28:1}] poursuive, mais les j. seront assurés com.
\item[\vref{Ec 7:16}] crois pas trop j., et ne te
\item[\vref{Ec 7:20}] a point d'hom. j. sur la terre
\item[\vref{Ec 7:29}] a créé l'hom. j. ; mais ils ont
\item[\vref{Es 26:7}] Le sentier du j. est la droiture ;
\item[\vref{Es 45:21}] moi ; un Dieu j. et un Sauveur,
\item[\vref{Es 53:11}] rassasié ; mon serviteur j. justifiera beaucoup d'hommes
\item[\vref{Es 57:1}] Le j. périt, et nul ne le prend
\item[\vref{Es 60:21}] ils seront ts j., ils posséderont la
\item[\vref{Jé 23:5}] David un Germe j., qui régnera en
\item[\vref{Ez 45:10}] Ayez la balance j., l'épha juste, et
\item[\vref{Da 9:14}] notre Dieu, est j. ds ttes les
\item[\vref{Mi 6:8}] ce qui est j., que tu aimes
\item[\vref{Ha 2:4}] lui ; mais le j. vivra de sa
\item[\vref{Mt 1:19}] était un hom. j. et qui ne
\item[\vref{Mt 9:13}] la repentance les j., mais les pécheurs.
\item[\vref{Mt 10:41}] qui reçoit un j. en qualité de
\item[\vref{Mt 23:28}] au-dehors vs. paraissez j. aux hommes, mais
\item[\vref{Mt 23:35}] sang d'Abel le j., jusqu'au sang de
\item[\vref{Mt 25:46}] éternel, mais les j. à la vie
\item[\vref{Mt 27:19}] l'affaire de ce j., car j'ai beaucoup
\item[\vref{Mt 27:24}] sang de ce j.. Cela vs. regarde.
\item[\vref{Mc 6:20}] c'était un hom. j. et saint ; il
\item[\vref{Lu 1:6}] étaient ts deux j. dvt Dieu, marchant
\item[\vref{Lu 23:41}] Pour ns., c'est j., car ns. recevons
\item[\vref{Lu 23:47}] et dit : Certes, cet hom. était j.
\item[\vref{Ac 7:52}] d'avance l'avènement du J., dont vs. avez
\item[\vref{Ac 10:22}] Corneille, centenier, hom. j. et craignant Dieu,
\item[\vref{Ac 22:14}] à voir le J., et à entendre
\item[\vref{Ro 1:17}] est écrit : Le j. vivra de la
\item[\vref{Ro 3:10}] a pas de j., pas mm un
\item[\vref{Ro 3:26}] à être trouvé j. tt en justifiant
\item[\vref{Ro 5:7}] meure pour un j. ; mais encore il
\item[\vref{Ro 7:12}] et le commandement est saint, et j., et bon.
\item[\vref{2 Co 5:21}] que ns. devenions j. dvt Dieu par
\item[\vref{Ga 3:11}] est dit : Le j. vivra de la
\item[\vref{1 Ti 1:9}] pas pour le j. que la loi
\item[\vref{2 Ti 4:8}] le Seign., le j. Juge, me la
\item[\vref{Hé 11:4}] le témoignage d'être j., parce que Dieu
\item[\vref{Hé 12:23}] des esprits des j. qui ont été
\item[\vref{Ja 5:16}] la prière du j. faite avec ferveur
\item[\vref{1 Pi 3:18}] les péchés, lui j. pour les injustes,
\item[\vref{1 Pi 4:18}] Et si le j. est difficilement sauvé, que deviendront l'impie
\item[\vref{2 Pi 2:8}] car cet hom. j., qui habitait au
\item[\vref{1 Jn 1:9}] est fidèle et j. pour ns. les
\item[\vref{1 Jn 2:1}] avocat auprès du Père, Jésus-Christ le J.
\item[\vref{1 Jn 3:7}] ce qui est j. est une personne
\item[\vref{Ap 16:5}] VIENS, tu es j., parce que tu
\item[\vref{Ap 22:11}] celui qui est j. pratique encore la
\end{listverse}

\ConcordanceEntry{Justice}
\vspace{-2mm}
\begin{listverse}
\item[\vref{De 9:5}] point pour ta j. ni pour la
\item[\vref{1 S 26:23}] chacun selon sa j. et selon sa
\item[\vref{Job 19:7}] et il n'y a point de j. !
\item[\vref{Job 35:2}] as dit ma j. est au-dessus de
\item[\vref{Job 36:3}] je défendrai la j. du Créateur.
\item[\vref{Ps 7:9}] les peuples : Rends-moi j., ô Yahweh ! selon
\item[\vref{Ps 23:3}] sentiers de la j., pour l'amour de
\item[\vref{Ps 40:10}] Je prêche ta j. ds la grande
\item[\vref{Ps 71:19}] Car ta j., ô Dieu ! est haut élevée, car
\item[\vref{Ps 82:3}] à l'orphelin, faites j. à l'affligé et
\item[\vref{Ps 85:11}] se rencontrent, la j. et la paix
\item[\vref{Ps 85:14}] La j. marchera dvt lui, et il la
\item[\vref{Ps 89:15}] La j. et l'équité sont la base de
\item[\vref{Ps 106:3}] qui observent la j., qui font en
\item[\vref{Ps 111:3}] Vav.] et sa j. demeure à perpétuité.
\item[\vref{Pr 2:9}] tu comprendras la j., le jugement, l'équité,
\item[\vref{Pr 10:2}] pas, mais la j. délivre de la
\item[\vref{Pr 11:5}] La j. de l'hom. intègre rend droite sa
\item[\vref{Pr 11:19}] Ainsi la j. conduit à la vie, mais celui
\item[\vref{Pr 14:34}] La j. élève une nation, mais le péché
\item[\vref{Pr 16:12}] le trône est affermi par la j.
\item[\vref{Es 1:26}] cité de la j., ville fidèle.
\item[\vref{Es 26:9}] les habitants du monde apprennent la j.
\item[\vref{Es 32:17}] L'oeuvre de la j. sera la paix,
\item[\vref{Es 46:13}] fais approcher ma j., elle ne s'éloignera
\item[\vref{Es 59:16}] et sa propre j. lui sert d'appui.
\item[\vref{Es 59:17}] revêt de la j. com. d'une cuirasse,
\item[\vref{Es 64:5}] et tte notre j. est com. le
\item[\vref{Jé 23:5}] droit et la j. ds le pays.
\item[\vref{Jé 23:6}] nom dont on l'appellera : Yahweh notre j.
\item[\vref{Ez 18:5}] qui pratique la droiture et la j.,
\item[\vref{Ez 33:12}] ton peuple : La j. du juste ne
\item[\vref{Da 12:3}] plusieurs à la j. brilleront com. les
\item[\vref{Mt 5:20}] que si votre j. ne surpasse celle
\item[\vref{Mt 6:1}] de pratiquer votre j. dvt les hommes,
\item[\vref{Lu 11:42}] vs. négligez la j. et l'amour de
\item[\vref{Lu 18:3}] lui dire : Fais-moi j. de ma partie
\item[\vref{Lu 18:7}] ne ferait-il pas j. à ses élus,
\item[\vref{Jn 16:8}] de péché, de j., et de jugement :
\item[\vref{Jn 16:10}] de j., parce que je m'en vais à
\item[\vref{Ac 28:4}] la mer, la J. ne permet pas
\item[\vref{Ro 3:22}] la j., dis-je, de Dieu par la foi
\item[\vref{Ro 4:5}] sa foi lui est imputée à j.
\item[\vref{Ro 5:17}] don de la j. régneront-ils ds la
\item[\vref{Ro 6:20}] étiez libres à l'égard de la j.
\item[\vref{Ro 8:4}] afin que la j. de la loi
\item[\vref{Ro 9:30}] cherchaient pas la j., ont obtenu la
\item[\vref{Ro 9:31}] loi de la j., n'est pas parvenu
\item[\vref{Ro 10:3}] connaissant pas la j. de Dieu, et
\item[\vref{Ro 10:10}] parvient à la j., et c'est en
\item[\vref{1 Co 1:30}] de Dieu, sagesse, j., sanctification et rédemption,
\item[\vref{2 Co 6:14}] a-t-il entre la j. et l'iniquité ? Ou
\item[\vref{Ga 2:21}] car si la j. vient de la
\item[\vref{Ga 3:6}] et cela lui fut imputé à j.,
\item[\vref{Ep 6:14}] ayant revêtu la cuirasse de la j. ;
\item[\vref{Ph 1:11}] de fruits de j., qui sont par
\item[\vref{1 Ti 6:11}] et recherche la j., la piété, la
\item[\vref{2 Ti 4:8}] la couronne de j. m'est réservée, et
\item[\vref{Hé 1:9}] as aimé la j., et tu as
\item[\vref{Hé 11:7}] héritier de la j. qui est selon
\item[\vref{Hé 12:11}] fruit paisible de j. à ceux qui
\end{listverse}

\ConcordanceEntry{Justification}
\vspace{-2mm}
\begin{listverse}
\item[\vref{Ro 4:25}] offenses, et est ressuscité pour notre j.
\item[\vref{Ro 5:16}] don gratuit devient j. après plusieurs offenses.
\item[\vref{Ro 5:18}] hommes recevront la j. qui donne la
\item[\vref{Ro 10:4}] loi pour la j. de tt croyant.
\item[\vref{Ga 5:4}] qui cherchez la j. ds la loi ;
\end{listverse}

\ConcordanceEntry{Justifier}
\vspace{-2mm}
\begin{listverse}
\item[\vref{Job 9:2}] l'hom. mortel se j.-il dvt Dieu ?
\item[\vref{Job 33:32}] réponds-moi ! Parle, car je désire te j.
\item[\vref{Job 40:3}] mon jugement ? me condamneras-tu pour te j. ?
\item[\vref{Es 5:23}] qui j. le méchant pour des présents, et
\item[\vref{Es 50:8}] Celui qui me j. est proche ; qui
\item[\vref{Es 53:11}] mon serviteur juste j. beaucoup d'hommes par
\item[\vref{Mt 12:37}] Car tu seras j. par tes paroles
\item[\vref{Lu 10:29}] lui, voulant se j., dit à Jésus :
\item[\vref{Lu 18:14}] ds sa maison j., plutôt que l'autre.
\item[\vref{Ac 13:39}] quiconque croit est j. par lui, de
\item[\vref{Ac 19:40}] aucune raison pour j. ce rassemblement. Après
\item[\vref{Ro 3:20}] chair ne sera j. dvt lui par
\item[\vref{Ro 3:26}] juste tt en j. celui qui a
\item[\vref{Ro 3:28}] que l'hom. est j. par la foi,
\item[\vref{Ro 3:30}] seul Dieu, qui j. par la foi
\item[\vref{Ro 5:9}] dc, étant mntnt j. par son sang,
\item[\vref{Ro 8:30}] les a aussi j. ; et ceux qu'il
\item[\vref{Ro 8:33}] de Dieu ? Dieu est celui qui j. !
\item[\vref{2 Co 12:19}] ns. voulions ns. j. envers vs. ? Nous
\item[\vref{Ga 2:16}] l'hom. n'est pas j. par les œuvres
\item[\vref{Ga 3:11}] nul ne soit j. dvt Dieu par
\item[\vref{Ja 2:21}] ne fut-il pas j. par les œuvres,
\item[\vref{Ja 2:24}] que l'hom. est j. par les œuvres,
\end{listverse}

\ConcordanceEntry{Justus}
\vspace{-2mm}
\begin{listverse}
\item[\vref{Ac 1:23}] appelé Barsabbas, surnommé J., et Matthias.
\item[\vref{Ac 18:7}] d'un hom. appelé J., hom. craignant Dieu,
\item[\vref{Col 4:11}] Et Jésus, appelé J., vs. salue aussi.
\end{listverse}

\ConcordanceEntry{Kadès, Kadès-Barnéa}
\vspace{-2mm}
\begin{listverse}
\item[\vref{No 13:26}] de Padan à K.. Ils lr. firent
\item[\vref{No 20:16}] ns. sommes à K., ville qui est
\item[\vref{No 32:8}] les envoyai de K.-Barnéa pour examiner
\item[\vref{No 34:4}] du sud de K.-Barnéa ; et sortira
\item[\vref{De 1:46}] vs. demeurâtes à K. plusieurs jours, autant
\item[\vref{De 2:14}] avons marché de K.-Barnéa, jusqu'à ce
\item[\vref{Jos 10:41}] les battit depuis K.-Barnéa jusqu'à Gaza,
\item[\vref{Jg 11:16}] Mer Rouge et il a atteint K.
\end{listverse}

\ConcordanceEntry{Kebar}
\vspace{-2mm}
\begin{listverse}
\item[\vref{Ez 1:1}] le fleuve de K., que les cieux
\item[\vref{Ez 3:23}] du fleuve de K., et je tombai
\end{listverse}

\ConcordanceEntry{Kédar}
\vspace{-2mm}
\begin{listverse}
\item[\vref{Ge 25:13}] fut Nebajoth, puis K., Adbeel, Mibsam,
\item[\vref{Ps 120:5}] et de demeurer aux tentes de K. !
\item[\vref{Ca 1:5}] les tentes de K., et com. les
\item[\vref{Es 21:16}] la gloire de K. prendra fin.
\item[\vref{Jé 2:10}] Envoyez quelqu'un à K. ; observez bien, et
\item[\vref{Jé 49:28}] Sur K. et les royaumes de Hatsor, que
\item[\vref{Ez 27:21}] les princes de K., étaient des marchands
\end{listverse}

\ConcordanceEntry{Kédesch}
\vspace{-2mm}
\begin{listverse}
\item[\vref{Jos 20:7}] Ils consacrèrent dc K., en Galilée, ds
\item[\vref{Jos 21:32}] pour les meurtriers, K. en Galilée avec
\item[\vref{Jg 4:6}] fils d'Abinoam, de K.-Nephthali et elle
\item[\vref{Jg 4:9}] et elle alla avec Barak à K.
\end{listverse}

\ConcordanceEntry{Kedorlaomer}
\vspace{-2mm}
\begin{listverse}
\item[\vref{Ge 14:1}] roi d'Ellasar, de K., roi d'Elam, et
\item[\vref{Ge 14:4}] douze années à K., et la treizième
\item[\vref{Ge 14:5}] la quatorzième année, K. et les rois
\item[\vref{Ge 14:17}] revenait vainqueur de K., et des rois
\end{listverse}

\ConcordanceEntry{Kehath, Kehathites}
\vspace{-2mm}
\begin{listverse}
\item[\vref{Ge 46:11}] de Lévi : Guerschon, K., et Merari.
\item[\vref{No 4:4}] des fils de K. à la tente
\item[\vref{No 4:15}] les fils de K. viendront pour le
\item[\vref{Jos 21:5}] des enfants de K., dix villes des
\end{listverse}

\ConcordanceEntry{Keïla}
\vspace{-2mm}
\begin{listverse}
\item[\vref{Jos 15:44}] K., Aczib et Maréscha ; neuf villes, et
\item[\vref{1 S 23:1}] la guerre à K., et pillent les
\end{listverse}

\ConcordanceEntry{Kemosch}
\vspace{-2mm}
\begin{listverse}
\item[\vref{No 21:29}] Moab ! Peuple de K., tu es perdu !
\item[\vref{Jg 11:24}] que ton dieu K. te donne à
\item[\vref{1 R 11:7}] haut lieu à K., l'abomination des Moabites,
\item[\vref{2 R 23:13}] des Sidoniens, à K., l'abomination des Moabites,
\item[\vref{Jé 48:7}] seras pris, et K. sortira pour être
\end{listverse}

\ConcordanceEntry{Kerith}
\vspace{-2mm}
\begin{listverse}
\item[\vref{1 R 17:3}] du torrent de K., qui est en
\item[\vref{1 R 17:5}] au torrent de K., vis-à-vis du Jourdain.
\end{listverse}

\ConcordanceEntry{Kesita}
\vspace{-2mm}
\begin{listverse}
\item[\vref{Jos 24:32}] Sichem, pour cent k., et qui appartint
\end{listverse}

\ConcordanceEntry{Ketura}
\vspace{-2mm}
\begin{listverse}
\item[\vref{Ge 25:1}] Abraham prit une autre fem. nommée K.
\item[\vref{1 Ch 1:32}] aux fils de K., concubine d'Abraham, elle
\end{listverse}

\ConcordanceEntry{Kis}
\vspace{-2mm}
\begin{listverse}
\item[\vref{1 S 9:1}] de Benjamin, nommé K., fort et vaillant,
\item[\vref{1 S 9:3}] Les ânesses de K., père de Saül,
\end{listverse}

\ConcordanceEntry{Kison}
\vspace{-2mm}
\begin{listverse}
\item[\vref{Jg 4:7}] au torrent de K., Sisera, chef de
\item[\vref{Jg 5:21}] Le torrent de K. les a emportés,
\end{listverse}

\ConcordanceEntry{Koré}
\vspace{-2mm}
\begin{listverse}
\item[\vref{No 16:1}] Or K., fils de Jitsehar, fils de Kehath,
\item[\vref{No 16:16}] Moïse dit à K. : Toi et ts
\item[\vref{Ps 42:1}] des fils de K., donné au chef
\item[\vref{Jud 1:11}] une rébellion semblable à celle de K.
\end{listverse}

\ConcordanceEntry{Laban}
\vspace{-2mm}
\begin{listverse}
\item[\vref{Ge 24:29}] un frère nommé L., qui courut dehors
\item[\vref{Ge 29:22}] L. réunit ts les gens du lieu
\item[\vref{Ge 29:29}] Et L. donna Bilha, sa servante, à Rachel,
\item[\vref{Ge 29:30}] servit encore chez L. sept autres années.
\item[\vref{Ge 31:1}] des fils de L. qui disaient : Jacob
\item[\vref{Ge 31:2}] le visage de L., et voici, il
\item[\vref{Ge 31:20}] Et Jacob trompa L., le Syrien, en
\item[\vref{Ge 31:43}] L. répondit à Jacob, et dit : Ces
\item[\vref{Ge 31:49}] Mitspa ; parce que L. dit : Que Yahweh
\end{listverse}

\ConcordanceEntry{Labourer}
\vspace{-2mm}
\begin{listverse}
\item[\vref{1 S 8:12}] de cinquante, pour l. ses terres, pour
\item[\vref{Job 4:8}] vu, ceux qui l. l'iniquité et qui
\item[\vref{Job 39:13}] une corde pour l. ? Ou rompra-t-il les
\item[\vref{Ps 129:3}] Des laboureurs ont l. mon dos, ils
\item[\vref{Pr 20:4}] Le paresseux ne l. pas à cause
\item[\vref{Es 28:24}] Celui qui l. pour semer, laboure-t-il
\item[\vref{Jé 26:18}] armées : Sion sera l. com. un champ,
\item[\vref{1 Co 9:10}] que celui qui l. doit labourer avec
\end{listverse}

\ConcordanceEntry{Laboureur}
\vspace{-2mm}
\begin{listverse}
\item[\vref{Ge 4:2}] et Abel fut berger, et Caïn l.
\item[\vref{Ge 9:20}] Et Noé, l. de la terre, commença à planter
\item[\vref{Ps 129:3}] Des l. ont labouré mon dos, ils y
\item[\vref{Es 28:28}] menuise, car le l. ne le foule
\item[\vref{Es 61:5}] l'étranger seront vos l. et vos vignerons.
\item[\vref{Jé 51:23}] en pièces le l. et ses bœufs
\item[\vref{Am 5:16}] au deuil le l., et à la
\item[\vref{Am 9:13}] Yahweh, où le l. suivra de près
\item[\vref{Za 13:5}] mais je suis l., car on m'a
\item[\vref{2 Ti 2:6}] aussi que le l. travaille premièrement, et
\item[\vref{Ja 5:7}] Seign.. Voici, le l. attend le précieux
\end{listverse}

\ConcordanceEntry{Laisser}
\vspace{-2mm}
\begin{listverse}
\item[\vref{Ex 4:23}] je t'ai dit : L. aller mon fils,
\item[\vref{Ex 7:14}] a refusé de l. aller le peuple.
\item[\vref{Ex 12:10}] Et ne l. aucun reste jusqu'au matin, mais s'il
\item[\vref{Jg 11:20}] Israël pour le l. passer sur son
\item[\vref{2 R 10:11}] prêtres, sans en l. échapper un seul.
\item[\vref{Mt 4:11}] le diable le l.. Et voici, des
\item[\vref{Mt 5:24}] l. là ton offrande dvt l'autel, et
\item[\vref{Mt 15:14}] L.-les, ce sont des aveugles, conducteurs
\item[\vref{Mt 24:40}] champ ; l'un sera pris, et l'autre l. ;
\item[\vref{Mc 12:21}] et mourut sans l. de postérité. Il
\item[\vref{Lu 6:9}] sauver une personne, ou de la l. mourir ?
\item[\vref{1 Th 4:5}] et sans se l. aller aux désirs
\item[\vref{2 Th 2:2}] ne pas vs. l. subitement ébranler ds
\item[\vref{Hé 6:1}] C'est pourquoi, l. la parole qui
\end{listverse}

\ConcordanceEntry{Lait}
\vspace{-2mm}
\begin{listverse}
\item[\vref{Ex 3:8}] où coulent le l. et le miel ;
\item[\vref{Ex 23:19}] chevreau ds le l. de sa mère.
\item[\vref{Lé 20:24}] où coulent le l. et le miel.
\item[\vref{No 13:27}] où coulent le l. et le miel,
\item[\vref{Jg 5:25}] a donné du l. ; elle lui a
\item[\vref{Job 10:10}] coulé com. du l. ? Et ne m'as-tu
\item[\vref{Pr 30:33}] qui bat le l. en fait sortir
\item[\vref{Es 60:16}] tu suceras le l. des nations, et
\item[\vref{1 Co 3:2}] ai donné du l. à boire, et
\item[\vref{Hé 5:12}] encore besoin de l., et non d’
\item[\vref{Hé 5:13}] quiconque use de l., ne sait pas
\item[\vref{1 Pi 2:2}] enfants nouveau-nés, le l. spirituel et pur,
\end{listverse}

\ConcordanceEntry{Lakis}
\vspace{-2mm}
\begin{listverse}
\item[\vref{Jos 10:3}] Japhia, roi de L., et à Debir,
\item[\vref{Jos 10:5}] le roi de L., et le roi
\item[\vref{Jos 10:23}] le roi de L. et le roi
\item[\vref{Jos 10:31}] de Libna à L., campa dvt elle,
\item[\vref{2 Ch 11:9}] Adoraïm, L., Azéka,
\item[\vref{2 Ch 25:27}] il s'enfuit à L. ; mais on le
\item[\vref{Jé 34:7}] à savoir, contre L. et contre Azéka,
\end{listverse}

\ConcordanceEntry{Lamentation}
\vspace{-2mm}
\begin{listverse}
\item[\vref{Jé 31:15}] à Rama, des l., des larmes amères ;
\item[\vref{Ez 2:10}] en dehors ; des l., des soupirs, et
\item[\vref{Ez 27:32}] complainte ds lr. l., et feront lr.
\item[\vref{Ez 32:18}] l'hom., dresse une l. sur la multitude
\item[\vref{Am 5:16}] et à la l. ceux qui en
\item[\vref{Mi 1:8}] je ferai une l. com. celle des
\end{listverse}

\ConcordanceEntry{Lamenter (se)}
\vspace{-2mm}
\begin{listverse}
\item[\vref{2 S 14:2}] semblant de te l., et revêts des
\item[\vref{Ec 3:4}] temps pour se l. et un temps
\item[\vref{Jé 9:20}] filles à se l., et chacune à
\item[\vref{Jé 16:5}] vas pas te l. ni te plaindre
\item[\vref{Mt 11:17}] et vs. ne vs. êtes pas l.
\item[\vref{Mt 24:30}] la terre se l. en se frappant
\item[\vref{Lu 23:27}] se frappaient la poitrine, et se l. sur lui.
\end{listverse}

\ConcordanceEntry{Lampe}
\vspace{-2mm}
\begin{listverse}
\item[\vref{Ex 25:37}] aussi ses sept l., et on les
\item[\vref{1 S 3:3}] avant que les l. de Dieu soient
\item[\vref{1 R 11:36}] serviteur, ait une l. à toujours dvt
\item[\vref{Ps 119:105}] parole est une l. à mes pieds
\item[\vref{Pr 20:27}] l'hom. est une l. de Yahweh, il
\item[\vref{Mt 5:15}] n'allume pas la l. pour la mettre
\item[\vref{Mt 6:22}] L'œil est la l. du corps. Si
\item[\vref{Mt 25:8}] huile, car nos l. s'éteignent.
\item[\vref{Mc 4:21}] aussi : Apporte-t-on la l. pour la mettre
\item[\vref{Lu 8:16}] avoir allumé la l., ne la couvre
\item[\vref{Lu 12:35}] ceints, et vos l. allumées.
\item[\vref{Jn 5:35}] Il était une l. ardente et brillante ;
\item[\vref{2 Pi 1:19}] com. à une l. qui brille ds
\item[\vref{Ap 4:5}] le trône sept l. de feu ardentes,
\item[\vref{Ap 18:23}] lumière de la l. ne brillera plus
\item[\vref{Ap 22:5}] lumière, ni de l., ni du soleil,
\end{listverse}

\ConcordanceEntry{Lance}
\vspace{-2mm}
\begin{listverse}
\item[\vref{1 S 17:7}] bois de sa l. était com. une
\item[\vref{1 S 17:45}] avec l'épée, la l., et le javelot ;
\item[\vref{1 S 17:47}] ni par la l.. Car la victoire
\item[\vref{1 S 18:10}] Saül avait une l. ds sa main.
\item[\vref{2 S 23:18}] Il brandit sa l. sur trois cents
\item[\vref{Ps 35:3}] Brandis la l. et le javelot
\item[\vref{Ps 46:10}] et rompu la l., il a consumé
\item[\vref{Ps 64:8}] Mais Dieu l. contre eux ses
\item[\vref{Es 2:4}] et de leurs l. des serpes ; une
\item[\vref{Jn 19:34}] côté avec une l., et aussitôt il
\end{listverse}

\ConcordanceEntry{Langage}
\vspace{-2mm}
\begin{listverse}
\item[\vref{Ge 11:1}] avait un mm l. et une mm
\item[\vref{Ge 11:7}] là confondons lr. l. afin qu'ils n'entendent
\item[\vref{Ps 19:4}] n'est pas un l., ce ne sont
\item[\vref{Ps 52:6}] discours pernicieux, le l. trompeur !
\item[\vref{Ez 3:5}] un peuple au l. inconnu, ou à
\item[\vref{Mt 26:73}] gens-là, car ton l. te fait connaître.
\item[\vref{Mc 14:70}] Galiléen, et ton l. s'y rapporte.
\item[\vref{Jn 8:43}] comprenez-vs. pas mon l. ? C'est parce que
\item[\vref{Ac 28:6}] ils changèrent de l. et dirent que
\item[\vref{1 Co 1:10}] ts un mm l., et qu'il n'y
\end{listverse}

\ConcordanceEntry{Langue}
\vspace{-2mm}
\begin{listverse}
\item[\vref{Job 5:21}] fléau de la l., et tu n'auras
\item[\vref{Ps 45:2}] le roi ! Ma l. sera com. la
\item[\vref{Ps 120:2}] et de la l. trompeuse.
\item[\vref{Pr 6:17}] yeux hautains, la l. mensongère, les mains
\item[\vref{Pr 10:20}] La l. du juste est un argent choisi,
\item[\vref{Pr 12:18}] d'épée, mais la l. des sages apporte
\item[\vref{Pr 12:19}] toujours, mais la l. fausse n'est que
\item[\vref{Pr 15:4}] La l. qui corrige le prochain est com.
\item[\vref{Pr 18:21}] pouvoir de la l., et celui qui
\item[\vref{Pr 21:23}] bouche et sa l. garde son âme
\item[\vref{Pr 26:28}] La fausse l. hait celui qu'elle
\item[\vref{Es 28:11}] balbutient et une l. étrangère.
\item[\vref{Es 32:4}] science, et la l. de ceux qui
\item[\vref{Es 50:4}] m'a donné la l. des savants, pour
\item[\vref{Jé 18:18}] tuons-le avec la l., et ne soyons
\item[\vref{Ez 3:26}] Et j'attacherai ta l. à ton palais,
\item[\vref{So 3:9}] je transformerai les l. des nations en
\item[\vref{Mc 7:35}] lien de sa l. se délia, et
\item[\vref{Ac 2:3}] lr. apparut des l. séparées, com. de
\item[\vref{Ac 2:4}] à parler des l. étrangères selon que
\item[\vref{Ac 2:11}] chacun ds notre l., des merveilles de
\item[\vref{Ac 10:46}] entendaient parler diverses l. et glorifier Dieu.
\item[\vref{Ro 14:11}] moi, et tte l. donnera gloire à
\item[\vref{1 Co 12:10}] la diversité de l. ; et à un
\item[\vref{1 Co 12:30}] Tous parlent-ils diverses l. ? Tous interprètent-ils ?
\item[\vref{1 Co 13:1}] parlerais ttes les l. des hommes, et
\item[\vref{1 Co 13:8}] abolies et les l. cesseront, la connaissance
\item[\vref{1 Co 14:13}] qui parle une l. inconnue prie pour
\item[\vref{1 Co 14:14}] prie ds une l. inconnue mon esprit
\item[\vref{1 Co 14:18}] parle plus de l. que vs. ts.
\item[\vref{1 Co 14:21}] gens d'une autre l., et par des
\item[\vref{Ph 2:11}] et que tte l. confesse que Jésus-Christ
\item[\vref{Ja 1:26}] tient pas sa l. en bride, mais
\item[\vref{Ja 3:5}] ainsi de la l., c'est un petit
\item[\vref{Ja 3:6}] La l. aussi est un feu ; c'est le
\item[\vref{Ja 3:8}] peut dompter la l. ; c'est un mal
\item[\vref{1 Pi 3:10}] qu'il préserve sa l. du mal, et
\item[\vref{1 Jn 3:18}] et avec la l., mais par des
\item[\vref{Ap 5:9}] tribu, de tte l., de tt peuple,
\item[\vref{Ap 16:10}] se mordaient la l. à cause de
\end{listverse}

\ConcordanceEntry{Laodicée}
\vspace{-2mm}
\begin{listverse}
\item[\vref{Col 2:1}] qui sont à L., et pour ts
\item[\vref{Col 4:13}] pour ceux de L., et pour ceux
\item[\vref{Col 4:15}] qui sont à L., et Nymphas, avec
\item[\vref{Col 4:16}] lisiez aussi celle qui viendra de L.
\item[\vref{Ap 1:11}] à Sardes, à Philadelphie, et à L.
\item[\vref{Ap 3:14}] de l'église de L. : Voici ce que
\end{listverse}

\ConcordanceEntry{Lapider}
\vspace{-2mm}
\begin{listverse}
\item[\vref{Ex 8:22}] Egyptiens ne ns. l.-ils pas ?
\item[\vref{Lé 20:2}] mourra. Le peuple du pays le l.
\item[\vref{Lé 24:16}] Toute l'assemblée le l., le lapidera. On
\item[\vref{No 14:10}] parlait de les l. ; mais la gloire
\item[\vref{No 15:35}] tte l'assemblée le l. hors du camp.
\item[\vref{1 S 30:6}] parlait de le l., car tt le
\item[\vref{1 R 21:13}] ville, ils le l., et il mourut.
\item[\vref{Jn 8:5}] la loi de l. celles qui sont
\item[\vref{Jn 10:31}] de nouveau des pierres pour le l.
\item[\vref{Jn 11:8}] cherchaient à te l., et tu retournes
\item[\vref{Ac 5:26}] avaient peur d'être l. par le peuple.
\item[\vref{Ac 14:5}] pour outrager et l. les apôtres,
\item[\vref{Hé 12:20}] montagne, elle sera l. ou percée d'un
\end{listverse}

\ConcordanceEntry{Large}
\vspace{-2mm}
\begin{listverse}
\item[\vref{Ps 18:20}] m'a mis au l., il m'a délivré,
\item[\vref{Mt 7:13}] c'est la porte l. et le chemin
\end{listverse}

\ConcordanceEntry{Larme}
\vspace{-2mm}
\begin{listverse}
\item[\vref{2 R 20:5}] j'ai vu tes l.. Voici je te
\item[\vref{Job 16:20}] œil fond en l. dvt Dieu.
\item[\vref{Ps 42:4}] Mes l. sont ma nourriture jour et nuit,
\item[\vref{Ps 56:9}] venues ; recueille mes l. ds tes outres :
\item[\vref{Ps 80:6}] de pain de l., et tu les
\item[\vref{Ps 126:5}] qui sèment avec l., moissonneront avec chants
\item[\vref{Es 25:8}] Yahweh essuie les l. de ts les
\item[\vref{Jé 9:1}] vive fontaine de l., et je pleurerais
\item[\vref{Mal 2:13}] de Yahweh de l., de plaintes et
\item[\vref{Lu 7:38}] mouilla de ses l., puis elle les
\item[\vref{2 Ti 1:4}] souvenant de tes l., je désire fort
\item[\vref{Hé 5:7}] cris et avec l. des prières et
\item[\vref{Ap 21:4}] Dieu essuiera tte l. de leurs yeux,
\end{listverse}

\ConcordanceEntry{Lasser}
\vspace{-2mm}
\begin{listverse}
\item[\vref{Ec 1:8}] l'oreille ne se l. pas d'entendre.
\item[\vref{Es 7:13}] pour vs. de l. les hommes, que
\item[\vref{Es 40:30}] jeunes gens se l. et se fatiguent,
\item[\vref{Es 43:22}] car tu t'es l. de moi, ô
\item[\vref{Ga 6:9}] Ne ns. l. pas de faire le bien ; car
\item[\vref{Ap 2:3}] et que tu ne t'es pas l.
\end{listverse}

\ConcordanceEntry{Laver}
\vspace{-2mm}
\begin{listverse}
\item[\vref{Ge 24:32}] de l'eau pour l. les pieds de
\item[\vref{Ex 29:4}] et tu les l. avec de l'eau.
\item[\vref{Lé 11:40}] de son cadavre l. ses vêtements et
\item[\vref{2 R 5:10}] dire : Va, et l.-toi sept fois
\item[\vref{Ps 51:4}] l.-moi parfaitement de mon iniquité et
\item[\vref{Ps 51:9}] je serai pur ; l.-moi, et je
\item[\vref{Es 1:16}] L.-vs., purifiez-vs., ôtez de dvt mes
\item[\vref{Jé 2:22}] Quand tu te l. avec du nitre,
\item[\vref{Mc 7:5}] repas sans se l. les mains ?
\item[\vref{Lu 7:44}] donné d'eau pour l. mes pieds ; mais
\item[\vref{Lu 11:38}] s'était pas premièrement l. avant le dîner.
\item[\vref{Jn 9:7}] dit : Va, et l.-toi au réservoir
\item[\vref{Jn 13:5}] se mit à l. les pieds de
\item[\vref{Jn 13:8}] Tu ne me l. jamais les pieds.
\item[\vref{Jn 13:14}] le Maître, j'ai l. vos pieds, vs.
\item[\vref{1 Co 6:11}] vs. avez été l., mais vs. avez
\item[\vref{Hé 10:22}] et le corps l. d'une eau pure.
\item[\vref{Ap 22:14}] sont ceux qui l. leurs robes afin
\end{listverse}

\ConcordanceEntry{Lazare}
\vspace{-2mm}
\begin{listverse}
\item[\vref{Lu 16:20}] un pauvre, nommé L., couché à la
\item[\vref{Lu 16:23}] loin Abraham et L. ds son sein.
\item[\vref{Lu 16:24}] moi, et envoie L., pour qu'il trempe
\item[\vref{Lu 16:25}] vie, et que L. a eu ses
\item[\vref{Jn 11:1}] hom. malade, appelé L., qui était de
\item[\vref{Jn 11:14}] lr. dit ouvertement : L. est mort.
\item[\vref{Jn 11:17}] arrivé, trouva que L. était déjà depuis
\item[\vref{Jn 11:43}] à haute voix : L., sors dehors !
\item[\vref{Jn 12:1}] Béthanie où était L. qui avait été
\item[\vref{Jn 12:9}] aussi pour voir L., qu'il avait ressuscité
\item[\vref{Jn 12:17}] qnd il appela L. du sépulcre et
\end{listverse}

\ConcordanceEntry{Léa}
\vspace{-2mm}
\begin{listverse}
\item[\vref{Ge 29:16}] filles : L'aînée s'appelait L., et la cadette
\item[\vref{Ge 29:17}] L. avait les yeux délicats, mais Rachel
\item[\vref{Ge 29:23}] venu, il prit L., sa fille, et
\item[\vref{Ge 29:31}] Yahweh vit que L. était haïe, et
\item[\vref{Ge 30:17}] Dieu exauça L., et elle conçut
\item[\vref{Ge 31:4}] appeler Rachel et L. qui étaient aux
\item[\vref{Ge 33:2}] avec leurs enfants ; L. et ses enfants
\item[\vref{Ge 49:31}] et c'est là que j'ai enterré L.
\item[\vref{Ru 4:11}] Rachel et com. L., qui ttes les
\end{listverse}

\ConcordanceEntry{Lecture}
\vspace{-2mm}
\begin{listverse}
\item[\vref{Né 8:3}] attentives à la l. du livre de
\item[\vref{Lu 4:16}] il se leva pour faire la l.,
\item[\vref{Ac 13:15}] Après la l. de la loi et des prophètes,
\item[\vref{2 Co 3:14}] ôté ds la l. de l'Ancienne Alliance.
\item[\vref{1 Ti 4:13}] Applique-toi à la l., à l'exhortation et
\end{listverse}

\ConcordanceEntry{Léger}
\vspace{-2mm}
\begin{listverse}
\item[\vref{1 R 19:12}] feu, vint un murmure doux et l.
\item[\vref{Ps 62:10}] ts ensemble, plus l. qu'un souffle.
\item[\vref{Jé 46:6}] Que l'hom. l. à la course
\item[\vref{Da 5:27}] balance, et tu as été trouvé l.
\item[\vref{Mt 11:30}] est doux et mon fardeau est l.
\end{listverse}

\ConcordanceEntry{Légion}
\vspace{-2mm}
\begin{listverse}
\item[\vref{Mt 26:53}] plus de douze l. d'anges ?
\item[\vref{Mc 5:9}] est ton nom ? L. est mon nom,
\item[\vref{Mc 5:15}] avait eu la l., assis et vêtu,
\item[\vref{Lu 8:30}] est ton nom ? L., répondit-il. Car plusieurs
\item[\vref{Ac 10:1}] cohorte de la l. appelée Italienne.
\item[\vref{Ac 27:1}] cohorte de la l. appelée Auguste.
\end{listverse}

\ConcordanceEntry{Législateur}
\vspace{-2mm}
\begin{listverse}
\item[\vref{Ge 49:10}] le bâton de l. d'entre ses pieds,
\item[\vref{No 21:18}] creusés, avec le l., avec leurs bâtons !
\item[\vref{De 33:21}] la portion du l., et il est
\item[\vref{Ps 108:8}] sommet de ma forteresse, Juda, mon l.
\item[\vref{Es 33:22}] Yahweh est notre L., Yahweh est notre
\item[\vref{Ja 4:12}] a qu'un seul L., qui peut sauver
\end{listverse}

\ConcordanceEntry{Lémec}
\vspace{-2mm}
\begin{listverse}
\item[\vref{Ge 4:18}] Mehujaël engendra Metuschaël, et Metuschaël engendra L.
\item[\vref{Ge 4:19}] L. prit deux femmes ; le nom de
\item[\vref{Ge 4:23}] Et L. dit à Ada et à Tsilla
\item[\vref{Ge 4:24}] sept fois davantage, L. le sera soixante-dix-sept
\item[\vref{Ge 5:25}] ayant vécu cent quatre-vingt-sept ans, engendra L.
\item[\vref{Ge 5:26}] qu'il eut engendré L., vécut sept cent
\item[\vref{Ge 5:28}] L. aussi vécut cent quatre-vingt-deux ans, et
\item[\vref{Ge 5:30}] Et L., après qu'il eut engendré Noé, vécut
\end{listverse}

\ConcordanceEntry{Lendemain}
\vspace{-2mm}
\begin{listverse}
\item[\vref{Lé 23:11}] prêtre l'agitera le l. du sabbat.
\item[\vref{Lé 23:15}] aussi dès le l. du sabbat, à
\item[\vref{Jos 5:12}] cessa dès le l. de la Pâque,
\item[\vref{Pr 27:1}] vante pas du l., car tu ne
\item[\vref{Mt 6:34}] pas pour le l. ; car le lendemain
\item[\vref{Lu 10:35}] Et le l., en partant il tira de sa
\item[\vref{Jn 1:29}] Le l., Jean vit Jésus venir à lui,
\item[\vref{Ac 4:3}] en prison jusqu'au l., parce qu'il était
\item[\vref{Ac 25:23}] Le l. dc, Agrippa et Bérénice étant venus
\item[\vref{Ja 4:14}] qui arrivera le l. ! Car qu'est-ce que
\end{listverse}

\ConcordanceEntry{Lentilles}
\vspace{-2mm}
\begin{listverse}
\item[\vref{Ge 25:34}] du potage de l. ; et il mangea
\end{listverse}

\ConcordanceEntry{Léopard}
\vspace{-2mm}
\begin{listverse}
\item[\vref{Es 11:6}] l'agneau, et le l. se couchera avec
\item[\vref{Jé 5:6}] détruit, et le l. est aux aguets
\item[\vref{Jé 13:23}] peau et le l. ses taches ? Pourriez-vs.
\item[\vref{Da 7:6}] semblable à un l., qui avait sur
\item[\vref{Os 13:7}] épierai sur la route com. un l.
\item[\vref{Ap 13:2}] semblable à un l., ses pieds étaient
\end{listverse}

\ConcordanceEntry{Lèpre}
\vspace{-2mm}
\begin{listverse}
\item[\vref{Ex 4:6}] était blanche de l. com. la neige.
\item[\vref{Lé 13:3}] une plaie de l.. Le prêtre dc
\item[\vref{Lé 14:34}] une plaie de l. sur une maison
\item[\vref{No 12:10}] était frappée d'une l. blanche com. la
\item[\vref{2 R 5:3}] Samarie, il le guérirait de sa l. !
\item[\vref{2 R 5:27}] C'est pourquoi la l. de Naaman s'attachera
\item[\vref{2 Ch 26:19}] les prêtres, la l. parut sur son
\item[\vref{Mt 8:3}] mm, il fut purifié de sa l.
\item[\vref{Mc 1:42}] Aussitôt la l. quitta cet hom.
\item[\vref{Lu 5:12}] hom. plein de l., voyant Jésus, se
\item[\vref{Lu 5:13}] pur. Aussitôt la l. le quitta.
\end{listverse}

\ConcordanceEntry{Lépreux}
\vspace{-2mm}
\begin{listverse}
\item[\vref{Lé 13:45}] Or le l. qui sera atteint de la plaie
\item[\vref{Lé 14:2}] la loi du l. pour le jour
\item[\vref{No 5:2}] du camp tt l., tt hom. ayant
\item[\vref{2 R 5:1}] cet hom. fort et vaillant était l.
\item[\vref{2 R 5:11}] sur la plaie, et guérira le l.
\item[\vref{2 R 7:3}] porte quatre hommes l., et ils se
\item[\vref{2 R 15:5}] roi, qui fut l. jusqu'au jour de
\item[\vref{Mt 8:2}] Et voici, un l. vint et se
\item[\vref{Mt 10:8}] rendez purs les l., ressuscitez les morts,
\item[\vref{Mt 11:5}] boiteux marchent, les l. sont purifiés, les
\item[\vref{Mt 26:6}] ds la maison de Simon le l.,
\item[\vref{Mc 14:3}] de Simon le l., et pendant qu'il
\item[\vref{Lu 4:27}] avait aussi plusieurs l. en Israël du
\item[\vref{Lu 7:22}] boiteux marchent, les l. sont purifiés, les
\item[\vref{Lu 17:12}] village, dix hommes l. vinrent à sa
\end{listverse}

\ConcordanceEntry{Lettre}
\vspace{-2mm}
\begin{listverse}
\item[\vref{Ez 9:6}] qui ont la l. Tav, et commencez
\item[\vref{Mt 5:18}] seul trait de l. jusqu'à ce que
\item[\vref{Mt 5:31}] lui donne une l. de divorce.
\item[\vref{Ro 2:27}] en ayant la l. de la loi
\item[\vref{Ro 2:29}] non selon la l.. La louange de
\item[\vref{Ro 7:6}] non selon la l. qui a vieilli.
\item[\vref{2 Co 3:1}] com. quelques-uns, de l. de recommandation auprès
\item[\vref{2 Co 3:3}] vs. êtes la l. de Christ, gravée
\item[\vref{2 Co 3:6}] non de la l., mais de l'Esprit ;
\item[\vref{2 Th 2:15}] par notre parole, soit par notre l.
\item[\vref{2 Ti 3:15}] connaissance des saintes l., qui peuvent te
\item[\vref{2 Pi 3:1}] ici la seconde l. que je vs.
\item[\vref{2 Pi 3:16}] ds ttes ses l., où il parle
\end{listverse}

\ConcordanceEntry{Levain}
\vspace{-2mm}
\begin{listverse}
\item[\vref{Ge 19:3}] des pains sans l., et ils mangèrent.
\item[\vref{Ex 12:8}] des pains sans l., et avec des
\item[\vref{Lé 2:11}] faite avec du l. ; car vs. ne
\item[\vref{De 16:4}] verra point de l. chez toi, sur
\item[\vref{Mt 13:33}] semblable à du l. qu'une fem. a
\item[\vref{Mt 16:6}] avec soin du l. des pharisiens et
\item[\vref{Mt 26:17}] des pains sans l., les disciples s'approchèrent
\item[\vref{Mc 8:15}] avec soin du l. des pharisiens et
\item[\vref{Lu 12:1}] gardez-vs. surtout du l. des pharisiens qui
\item[\vref{Lu 13:21}] est semblable au l. qu'une fem. a
\item[\vref{1 Co 5:6}] qu'un peu de l. fait lever tte
\item[\vref{1 Co 5:7}] dc le vieux l., afin que vs.
\item[\vref{1 Co 5:8}] avec du vieux l., non avec un
\item[\vref{Ga 5:9}] Un peu de l. fait lever tte
\end{listverse}

\ConcordanceEntry{Lever}
\vspace{-2mm}
\begin{listverse}
\item[\vref{No 10:35}] l'arche, Moïse disait : L.-toi, ô Yahweh,
\item[\vref{Ps 19:7}] il se l. à l'extrémité des cieux et achève
\item[\vref{Ps 127:2}] que vs. vs. l. de grand matin,
\item[\vref{Jé 1:17}] ceins tes reins, l.-toi, et dis-lr.
\item[\vref{Mt 9:5}] ou de dire : L.-toi et marche ?
\item[\vref{Mt 12:41}] Les Ninivites se l. au jour du
\item[\vref{Mc 5:41}] dire : Jeune fille, je te dis l.-toi.
\item[\vref{Lu 15:18}] Je me l., j'irai vers mon père, et je
\item[\vref{Lu 18:13}] n'osait mm pas l. les yeux vers
\item[\vref{Ac 3:7}] il le fit l. ; et aussitôt les
\item[\vref{Ac 9:41}] et la fit l.. Puis ayant appelé
\item[\vref{1 Co 5:6}] de levain fait l. tte la pâte ?
\item[\vref{Ga 5:9}] de levain fait l. tte la pâte.
\end{listverse}

\ConcordanceEntry{Lévi}
\vspace{-2mm}
\begin{listverse}
\item[\vref{Ge 29:34}] on lui donna le nom de L.
\end{listverse}
\begin{legend}
\NoAutoSpaceBeforeFDP{
\item Fils de Jacob et  Léa : Ge 29:34; 34:25-31;  49:5
\item Apôtre Matthieu surnommé Lévi : Mt 9:9-13; Mc 2:14-17; Lu 5:27-29
}
\end{legend}

\ConcordanceEntry{Lévi (tribu de), Lévite(s)}
\vspace{-2mm}
\begin{listverse}
\item[\vref{Ex 6:16}] des fils de L. selon lr. naissance :
\end{listverse}
\begin{legend}
\NoAutoSpaceBeforeFDP{
\item Descendants de Lévi consacrés au service de Yahweh : Ex 6:16; 32:26-29; No 3:12,45
\item Dénombrement des Lévites : No 3:39
\item Héritages : De 18:1-8; Jos 13:14,33; Jos 21:41
\item Dénombrement des chantres et Répartition par David : 1 Ch 25:1-31
\item Fonctions des Lévites :  No 3:6-10; No 8:6-26; 18:2-6; 1 Ch 23:24-32
\item Les 24 classes de prêtres : 1 Ch 24:1-19
\item Commis sur les trésors du temple : 1 Ch 26:20-28
\item Autres : De 12:19;  33:8-11; Lu 10:32; Hé 7:9, 21; Ap 7:7
}
\end{legend}

\ConcordanceEntry{Léviathan}
\vspace{-2mm}
\begin{listverse}
\item[\vref{Job 3:8}] qui sont prêts à réveiller le l. !
\item[\vref{Job 40:20}] Attireras-tu le l. à l'hameçon ? Saisiras-tu
\item[\vref{Ps 74:14}] les têtes du l., tu l'as donné
\item[\vref{Ps 104:26}] navires, et ce l. que tu as
\item[\vref{Es 27:1}] forte épée le L., le serpent fuyard,
\end{listverse}

\ConcordanceEntry{Lèvre}
\vspace{-2mm}
\begin{listverse}
\item[\vref{Ps 17:1}] faite avec des l. sans tromperie !
\item[\vref{Ps 31:19}] Que les l. menteuses soient muettes, elles profèrent des
\item[\vref{Ps 45:3}] répandue sur tes l. : C'est pourquoi Dieu
\item[\vref{Ps 106:33}] il parla avec légèreté de ses l.
\item[\vref{Ps 141:3}] ma bouche, garde l'entrée de mes l. !
\item[\vref{Pr 10:19}] qui retient ses l. est prudent.
\item[\vref{Pr 12:19}] La l. véridique est affermie pour toujours, mais
\item[\vref{Pr 12:22}] Les fausses l. sont une abomination
\item[\vref{Pr 16:13}] prendre plaisir aux l. de justice, et
\item[\vref{Pr 17:4}] attentif à la l. trompeuse, et le
\item[\vref{Es 6:5}] hom. dont les l. sont impures, j'habite
\item[\vref{1 Co 14:21}] et par des l. étrangères, et ils
\item[\vref{Hé 13:15}] le fruit des l., en confessant son
\end{listverse}

\ConcordanceEntry{Liban}
\vspace{-2mm}
\begin{listverse}
\item[\vref{De 3:25}] Jourdain, ces bonnes montagnes et le L.
\item[\vref{De 11:24}] du désert au L., et du fleuve,
\item[\vref{Jos 13:5}] et tt le L., vers l'orient, depuis
\item[\vref{Jg 3:3}] la montagne du L., depuis la montagne
\item[\vref{Ps 29:5}] cèdres, Yahweh brise les cèdres du L.,
\item[\vref{Es 10:34}] forêt, et le L. tombe sous le
\item[\vref{Es 33:9}] terre languit. Le L. est honteux et
\item[\vref{Es 60:13}] La gloire du L. viendra vers toi,
\item[\vref{Za 10:10}] Galaad, et au L., et il n'y
\end{listverse}

\ConcordanceEntry{Libation}
\vspace{-2mm}
\begin{listverse}
\item[\vref{Ge 35:14}] fit dessus une l. et y versa
\item[\vref{Lé 23:13}] Yahweh ; et sa l. de vin sera
\item[\vref{Lé 23:18}] gâteaux et leurs l., des sacrifices consumés
\item[\vref{No 4:7}] les calices de l.. Le pain continuel
\item[\vref{2 R 16:13}] sacrifice, versa ses l. et répandit sur
\item[\vref{1 Ch 29:21}] agneaux, avec leurs l. ; et des sacrifices
\item[\vref{2 Ch 29:35}] et avec les l. des holocaustes. Ainsi,
\item[\vref{Jé 44:17}] lui faire des l., com. ns. l'avons
\item[\vref{Jé 44:19}] lui faisions des l., est-ce à l'insu
\item[\vref{Ez 45:17}] offrandes et les l. qu'il faudra offrir
\item[\vref{Joë 1:9}] L'offrande et la l. sont retranchées de
\item[\vref{Joë 1:13}] et à la l. d'entrer ds la
\item[\vref{Joë 2:14}] la bénédiction, des offrandes et des l. ?
\item[\vref{Ph 2:17}] je sers de l. sur le sacrifice
\item[\vref{2 Ti 4:6}] mntnt servir de l. et le temps
\end{listverse}

\ConcordanceEntry{Libéralité}
\vspace{-2mm}
\begin{listverse}
\item[\vref{Ex 35:21}] porté à la l., apporta l'offrande de
\item[\vref{Ex 35:29}] incita à la l. pour apporter de
\item[\vref{Est 1:7}] vin royal en abondance selon la l. du roi.
\item[\vref{Pr 25:14}] vante d'une fausse l., est com. les
\item[\vref{Es 32:8}] des conseils de l. et se lève
\item[\vref{1 Co 16:3}] pour porter votre l. à Jérus.
\item[\vref{2 Co 8:2}] répandue en richesses par lr. prompte l.
\item[\vref{2 Co 9:5}] de préparer votre l. que vs. avez
\item[\vref{2 Co 9:11}] exercer une parfaite l., laquelle fait que
\item[\vref{2 Co 9:13}] votre prompte et l. communication envers eux
\item[\vref{1 Ti 6:18}] qu'ils soient prompts à donner, avec l.,
\end{listverse}

\ConcordanceEntry{Libérateur}
\vspace{-2mm}
\begin{listverse}
\item[\vref{Jg 3:9}] lr. suscita un l. qui les délivra,
\item[\vref{2 S 22:2}] est mon rocher, ma forteresse, mon l.
\item[\vref{2 R 13:5}] donna dc un l. à Israël, et
\item[\vref{Né 9:27}] lr. donnas des l. qui les délivrèrent
\item[\vref{Ps 18:3}] forteresse et mon l. ! Mon Dieu est
\item[\vref{Ps 70:6}] secours et mon l.. Ô Yahweh ! ne
\item[\vref{Ps 78:35}] et Dieu, le Très-Haut, était lr. l.
\item[\vref{Ps 106:21}] oublièrent Dieu, lr. l., qui avait fait
\item[\vref{Ps 144:2}] haute retraite, mon l., mon bouclier, mon
\item[\vref{Ac 7:35}] prince et com. l. par le moyen
\item[\vref{Ro 11:26}] est écrit : Le L. viendra de Sion,
\end{listverse}

\ConcordanceEntry{Liberté}
\vspace{-2mm}
\begin{listverse}
\item[\vref{Lé 25:10}] et publierez la l. ds le pays
\item[\vref{Es 61:1}] aux captifs la l., et aux prisonniers
\item[\vref{Ac 28:31}] Jésus-Christ, en tte l. ds les paroles
\item[\vref{Ro 8:21}] part à la l. de la gloire
\item[\vref{1 Co 8:9}] garde que cette l. que vs. avez
\item[\vref{1 Co 10:29}] Car pourquoi ma l. serait-elle condamnée par
\item[\vref{2 Co 3:12}] usons d'une grande l. ds les paroles.
\item[\vref{2 Co 3:17}] l'Esprit du Seign., là est la l.
\item[\vref{Ga 2:4}] pour épier la l. que ns. avons
\item[\vref{Ga 5:1}] fermes ds la l. pour laquelle Christ
\item[\vref{Ga 5:13}] appelés à la l., seulement ne faites
\item[\vref{Ep 6:19}] parler en tte l. et avec hardiesse,
\item[\vref{1 Ti 3:13}] et une grande l. ds la foi
\item[\vref{Hé 10:19}] ns. avons la l. d'entrer ds le
\item[\vref{Ja 1:25}] loi de la l., et qui aura
\item[\vref{Ja 2:12}] jugés par la loi de la l.,
\item[\vref{1 Pi 2:16}] n'utilisant pas votre l. com. un voile
\item[\vref{2 Pi 2:19}] lr. promettent la l., alors qu'ils sont
\end{listverse}

\ConcordanceEntry{Libre}
\vspace{-2mm}
\begin{listverse}
\item[\vref{Ex 21:2}] sortira pour être l., sans rien payer.
\item[\vref{Es 58:6}] tu laisses aller l. les opprimés, et
\item[\vref{Jn 8:36}] Fils vs. affranchit, vs. serez véritablement l.
\item[\vref{Ro 6:7}] est mort est l. du péché.
\item[\vref{Ro 6:20}] péché, vs. étiez l. à l'égard de
\item[\vref{1 Co 7:39}] meurt, elle est l. de se marier
\item[\vref{1 Co 9:1}] Ne suis-je pas l. ? N'ai-je pas vu
\item[\vref{1 Co 9:19}] que je sois l. à l'égard de
\item[\vref{Ga 3:28}] ni esclave ni l., il n'y a
\item[\vref{Ga 4:22}] l'esclave, et un de la fem. l.
\item[\vref{Ga 4:26}] est la fem. l., et c'est notre
\item[\vref{Ep 6:8}] soit esclave, soit l., recevra du Seign.
\item[\vref{Col 3:11}] ni esclave ni l. ; mais Christ y
\item[\vref{1 Pi 2:16}] com. l., et n'utilisant pas votre liberté com.
\item[\vref{Ap 6:15}] et tt hom. l. se cachèrent ds
\item[\vref{Ap 13:16}] riches et pauvres, l. et esclaves, reçoivent
\end{listverse}

\ConcordanceEntry{Libye, Libyen}
\vspace{-2mm}
\begin{listverse}
\item[\vref{2 Ch 12:3}] lui d'Egypte, des L., des Sukkiens et
\item[\vref{2 Ch 16:8}] Ethiopiens et les L. n'étaient-ils pas une
\item[\vref{Da 11:43}] de l'Egypte ; les L. et les Ethiopiens
\item[\vref{Ac 2:10}] territoire de la L. qui est près
\end{listverse}

\ConcordanceEntry{Lien}
\vspace{-2mm}
\begin{listverse}
\item[\vref{2 S 22:6}] les l. du scheol m'avaient entouré, les filets
\item[\vref{Ps 2:3}] Rompons leurs l., et jetons loin
\item[\vref{Ps 18:5}] Les l. de la mort m'avaient environné et
\item[\vref{Ps 116:16}] ta servante. Tu as délié mes l.
\item[\vref{Jé 27:2}] Yahweh : Fais-toi des l. et des jougs,
\item[\vref{Da 3:25}] quatre hommes sans l. qui marchent au
\item[\vref{Os 11:4}] tirai avec des l. d'humanité, et avec
\item[\vref{Mc 7:35}] s'ouvrirent, et le l. de sa langue
\item[\vref{Lu 8:29}] il rompait les l., et il était
\item[\vref{Lu 13:16}] délier de ce l. le jour du
\item[\vref{Ac 2:24}] ayant brisé les l. de la mort,
\item[\vref{Ac 8:23}] et ds un l. d'iniquité.
\item[\vref{Ac 16:26}] s'ouvrirent, et les l. de ts furent
\item[\vref{Ac 20:23}] m'avertit que des l. et des tribulations
\item[\vref{1 Co 7:27}] à rompre ce l.. N'es-tu pas lié
\item[\vref{Ep 4:3}] l'Esprit par le l. de la paix.
\item[\vref{Ph 1:13}] sorte que mes l. en Christ ont
\item[\vref{Col 3:14}] qui est le l. de la perfection.
\item[\vref{Col 4:18}] Souvenez-vs. de mes l.. Que la grâce
\end{listverse}

\ConcordanceEntry{Lier}
\vspace{-2mm}
\begin{listverse}
\item[\vref{Ge 22:9}] et ensuite il l. Isaac, son fils,
\item[\vref{Ge 37:7}] ns. étions à l. des gerbes au
\item[\vref{No 30:3}] par serment, pour l. son âme par
\item[\vref{Jg 15:10}] sommes montés pour l. Samson, afin que
\item[\vref{Ps 149:8}] pour l. leurs rois avec des chaînes, et
\item[\vref{Pr 6:21}] Tiens-les continuellement l. à ton cœur,
\item[\vref{Mt 16:19}] ce que tu l. sur la terre
\item[\vref{Mt 22:13}] dit aux serviteurs : L.-lui les pieds
\item[\vref{Mt 27:2}] Après l'avoir l., ils l'amenèrent et
\item[\vref{Mc 3:27}] sans avoir auparavant l. cet hom. fort ;
\item[\vref{Mc 5:3}] pouvait plus le l., pas mm avec
\item[\vref{Lu 13:16}] que Satan tenait l. depuis dix-huit ans ?
\item[\vref{Ac 10:28}] Juif de se l. avec un étranger,
\item[\vref{Ac 20:22}] mntnt, voici, étant l. par l'Esprit, je
\item[\vref{Ac 22:29}] la crainte parce qu'il l'avait fait l.
\item[\vref{Ro 7:2}] un mari, est l. à son mari
\item[\vref{1 Co 7:27}] Es-tu l. à une fem. ? Ne cherche pas
\item[\vref{2 Ti 2:9}] la parole de Dieu n'est pas l.
\item[\vref{Ap 20:2}] Satan, et le l. pour mille ans.
\end{listverse}

\ConcordanceEntry{Lieu}
\vspace{-2mm}
\begin{listverse}
\item[\vref{Ge 28:16}] est en ce l.-ci, et moi,
\item[\vref{Ex 26:33}] séparation entre le l. saint et le
\item[\vref{Lé 26:30}] détruirai vos hauts l., j'abattrai vos statues
\item[\vref{No 10:29}] Nous allons au l. dont Yahweh a
\item[\vref{De 12:11}] y aura un l. que Yahweh, votre
\item[\vref{Jos 5:15}] pieds ; car le l. sur lequel tu
\item[\vref{2 S 22:34}] me fait tenir debout sur mes l. élevés.
\item[\vref{1 R 3:4}] grand des hauts l.. Et Salomon offrit
\item[\vref{1 R 8:8}] se voyaient du l. saint dvt le
\item[\vref{1 R 8:10}] prêtres sortirent du l. saint, la nuée
\item[\vref{2 R 18:4}] disparaître les hauts l., mit en pièces
\item[\vref{2 Ch 7:12}] je choisis ce l. com. une maison
\item[\vref{2 Ch 14:4}] Juda les hauts l. et les colonnes
\item[\vref{Ps 132:5}] j'aie trouvé un l. pour Yahweh, une
\item[\vref{Ec 3:20}] ds un mm l. ; tt a été
\item[\vref{Es 66:1}] quel serait le l. de mon repos ?
\item[\vref{Ha 3:19}] marcher sur mes l. élevés. Au chef
\item[\vref{Mt 26:36}] eux ds un l. appelé Gethsémané, et
\item[\vref{Mt 27:33}] étant arrivés au l. appelé Golgotha, c'est-à-dire
\item[\vref{Mt 28:6}] et voyez le l. où le Seign.
\item[\vref{Mc 14:32}] allèrent ds un l. appelé Gethsémané, et
\item[\vref{Mc 15:22}] le conduisirent au l. appelé Golgotha, c'est-à-dire,
\item[\vref{Lu 9:18}] était ds un l. retiré pour prier,
\item[\vref{Jn 19:17}] croix, arriva au l. appelé le Crâne,
\item[\vref{Ac 2:1}] étaient ts ensemble ds un mm l.
\item[\vref{Ac 4:31}] eurent prié, le l. où ils étaient
\item[\vref{Ac 10:38}] qui allait de l. en lieu, faisant
\item[\vref{Ro 1:25}] la créature, au l. du Créateur, qui
\item[\vref{2 Co 2:14}] l'odeur de sa connaissance en tt l.
\item[\vref{Ep 1:3}] spirituelles ds les l. célestes en Christ !
\item[\vref{Ep 2:6}] ensemble ds les l. célestes en Jésus-Christ,
\item[\vref{Ep 4:27}] Ne donnez pas l. au diable de
\item[\vref{Col 3:15}] tienne le principal l. ds vos cœurs.
\item[\vref{1 Ti 2:8}] prient en tt l., levant leurs mains
\item[\vref{Hé 3:8}] arriva ds le l. de la rébellion,
\item[\vref{Hé 9:2}] tabernacle, appelé le l. saint, ds lequel
\item[\vref{Hé 12:17}] trouva pas de l. à la repentance,
\item[\vref{2 Pi 1:19}] brille ds un l. obscur, jusqu'à ce
\item[\vref{Ap 2:13}] œuvres, et le l. où tu habites,
\item[\vref{Ap 12:6}] elle avait un l. préparé par Dieu,
\item[\vref{Ap 16:16}] assemblèrent ds le l. qui est appelé
\item[\vref{Ap 18:17}] naviguent vers ce l., ts les marins,
\end{listverse}

\ConcordanceEntry{Liguer (se)}
\vspace{-2mm}
\begin{listverse}
\item[\vref{2 R 12:20}] soulevèrent et se l. ; ils frappèrent Joas
\item[\vref{Né 4:8}] Et ils se l. ts ensemble pour
\item[\vref{Ps 2:2}] les princes se l.-ils avec eux
\item[\vref{Ac 4:26}] princes se sont l. ensemble contre le
\end{listverse}

\ConcordanceEntry{Limite}
\vspace{-2mm}
\begin{listverse}
\item[\vref{No 22:36}] est sur la l. de l'Arnon, à
\item[\vref{De 32:8}] il fixa les l. des peuples selon
\item[\vref{Jos 15:47}] la grande mer, qui sert de l.
\item[\vref{Jos 19:41}] La l. de lr. héritage fut, Tsorea, Eschthaol,
\item[\vref{Job 38:20}] prennes à lr. l., et que tu
\item[\vref{Ps 74:17}] posé ttes les l. de la terre ;
\item[\vref{Ps 104:9}] as posé une l. que les eaux
\item[\vref{Ps 119:96}] mais tes commandements n'ont point de l.
\item[\vref{Pr 8:29}] lorsqu'il donna une l. à la mer,
\item[\vref{2 Co 10:13}] seulement ds la l. du champ d'action
\end{listverse}

\ConcordanceEntry{Lin}
\vspace{-2mm}
\begin{listverse}
\item[\vref{Ex 26:36}] et de fin l. retors, d'ouvrage de
\item[\vref{Da 10:5}] hom. vêtu de l., et ayant sur
\item[\vref{Ap 15:6}] temple, revêtus d'un l. pur et blanc,
\item[\vref{Ap 18:12}] perles, de fin l., de pourpre, de
\item[\vref{Ap 19:8}] revêtir d'un fin l. pur et éclatant.
\item[\vref{Ap 19:14}] revêtues de fin l. blanc et pur.
\end{listverse}

\ConcordanceEntry{Linceul}
\vspace{-2mm}
\begin{listverse}
\item[\vref{Mt 27:59}] prit le corps et l'enveloppa d'un l. pur,
\item[\vref{Mc 14:51}] suivait, enveloppé d'un l. sur le corps
\item[\vref{Mc 14:52}] il abandonna son l., et se sauva
\item[\vref{Mc 15:46}] ayant acheté un l., descendit Jésus de
\item[\vref{Lu 23:53}] il l'enveloppa d'un l., et le mit
\end{listverse}

\ConcordanceEntry{Linge}
\vspace{-2mm}
\begin{listverse}
\item[\vref{Es 64:5}] est com. le l. le plus souillé ;
\item[\vref{Lu 19:20}] que j'ai gardée enveloppée ds un l. ;
\item[\vref{Jn 11:44}] était enveloppé d'un l.. Jésus lr. dit :
\item[\vref{Jn 13:4}] et prit un l., dont il se
\item[\vref{Jn 13:5}] essuyer avec le l. dont il était
\item[\vref{Jn 20:7}] et le l. qu'on avait mis sur la tête
\end{listverse}

\ConcordanceEntry{Lion}
\vspace{-2mm}
\begin{listverse}
\item[\vref{Ge 49:9}] est un jeune l.. Mon fils, tu
\item[\vref{Jg 14:6}] il déchira le l. com. on déchire
\item[\vref{Jg 14:18}] fort que le l. ? Et il lr.
\item[\vref{1 S 17:34}] père, qnd un l. ou un ours
\item[\vref{1 S 17:36}] tué et le l., et l'ours, et
\item[\vref{1 S 17:37}] la griffe du l. et de la
\item[\vref{2 S 23:20}] il frappa un l., un jour de
\item[\vref{2 R 17:25}] contre eux des l., qui les tuaient.
\item[\vref{Ps 91:13}] marcheras sur le l. et sur l'aspic,
\item[\vref{Pr 22:13}] paresseux dit : Le l. est dehors ! Je
\item[\vref{Ec 9:4}] un chien vivant vaut mieux qu'un l. mort.
\item[\vref{Es 11:7}] gîte, et le l., com. le bœuf,
\item[\vref{Es 31:4}] Yahweh : Comme le l., com. le lionceau
\item[\vref{Jé 2:30}] prophètes com. un l. qui ravage tt.
\item[\vref{Ez 1:10}] la face d'un l. à la main
\item[\vref{Da 6:22}] la gueule des l., qu'ils ne m'ont
\item[\vref{Da 7:4}] semblable à un l., et avait des
\item[\vref{Os 5:14}] serai com. un l. pour Ephraïm, com.
\item[\vref{Os 11:10}] rugira com. un l., et qnd il
\item[\vref{2 Ti 4:17}] été délivré de la gueule du l.
\item[\vref{1 Pi 5:8}] vs. com. un l. rugissant, cherchant qui
\item[\vref{Ap 4:7}] semblable à un l. ; le second animal
\item[\vref{Ap 5:5}] pas, voici le L. qui vient de
\item[\vref{Ap 10:3}] forte, com. lorsqu'un l. rugit. Et qnd
\item[\vref{Ap 13:2}] la gueule d'un l.. Et le dragon
\end{listverse}

\ConcordanceEntry{Lionceau}
\vspace{-2mm}
\begin{listverse}
\item[\vref{Ps 17:12}] déchirer, et au l. qui se tient
\item[\vref{Ps 34:11}] [Kaf.] Les l. éprouvent la disette
\item[\vref{Ps 35:17}] de leurs tempêtes, mon unique des l.
\item[\vref{Ps 91:13}] tu piétineras le l. et le dragon.
\item[\vref{Es 11:6}] le veau, le l., et le bétail
\item[\vref{Es 31:4}] lion, com. le l. rugit sur sa
\item[\vref{Jé 25:38}] tabernacle com. un l. ; car lr. pays
\item[\vref{Ez 32:2}] semblable à un l., et com. un
\item[\vref{Os 5:14}] Ephraïm, com. un l. pour la maison
\item[\vref{Am 3:4}] de proie ? Le l. fait-il retentir son
\item[\vref{Mi 5:7}] et com. un l. parmi les troupeaux
\end{listverse}

\ConcordanceEntry{Liqueur}
\vspace{-2mm}
\begin{listverse}
\item[\vref{No 6:3}] ne boira d'aucune l. de raisins, et
\item[\vref{No 11:8}] le goût d'une l. d'huile fraîche.
\item[\vref{De 29:6}] vin ni de l. forte, afin que
\item[\vref{Jg 13:4}] ni vin ni l. forte, et de
\item[\vref{Jg 13:7}] ni vin ni l. forte, et ne
\end{listverse}

\ConcordanceEntry{Lire}
\vspace{-2mm}
\begin{listverse}
\item[\vref{Ex 24:7}] l'Alliance et le l., et le peuple
\item[\vref{De 17:19}] lui et la l. ts les jours
\item[\vref{De 31:11}] aura choisi, tu l. alors cette loi
\item[\vref{Jos 8:34}] après cela, il l. tt haut ttes
\item[\vref{Jos 8:35}] que Josué ne l. tt haut dvt
\item[\vref{Né 8:3}] Et il l. ds le livre, sur la place
\item[\vref{Né 8:8}] Et ils l. ds le livre de la loi
\item[\vref{Es 29:12}] qui répond : Je ne sais pas l.
\item[\vref{Jé 51:61}] l'auras vue, tu l. ttes ces paroles-là ;
\item[\vref{Jé 51:63}] auras achevé de l. ce livre, tu
\item[\vref{Da 5:8}] ne purent pas l. l'écriture et en
\item[\vref{Da 5:16}] si tu peux l. cette écriture, et
\item[\vref{Ha 2:2}] afin qu'on la l. couramment.
\item[\vref{Mt 12:3}] dit : N'avez-vs. pas l. ce que fit
\item[\vref{Mt 24:15}] que celui qui l. ce prophète y
\item[\vref{Ac 15:21}] prêchent, puisqu'on le l. ts les jours
\item[\vref{Ap 1:3}] est celui qui l. et ceux qui
\item[\vref{Ap 5:4}] ni de le l., ni de le
\end{listverse}

\ConcordanceEntry{Lis}
\vspace{-2mm}
\begin{listverse}
\item[\vref{1 R 7:19}] des fleurs de l. hautes de quatre
\item[\vref{Ca 2:1}] Saron, et le l. des vallées.
\item[\vref{Os 14:5}] fleurira com. le l., et il poussera
\item[\vref{Mt 6:28}] comment croissent les l. des champs : Ils
\item[\vref{Lu 12:27}] comment croissent les l., ils ne travaillent,
\end{listverse}

\ConcordanceEntry{Lit}
\vspace{-2mm}
\begin{listverse}
\item[\vref{Ge 47:31}] se prosterna sur le chevet du l.
\item[\vref{De 3:11}] Rephaïm. Voici, son l., un lit de
\item[\vref{Ps 6:7}] mes larmes, mon l. est arrosé de
\item[\vref{Ps 41:4}] soutient sur son l. de douleur ; tu
\item[\vref{Pr 26:14}] ainsi fait le paresseux sur son l.
\item[\vref{Es 28:20}] Car le l. sera trop court, et on ne
\item[\vref{Ez 23:17}] vers elle au l. de ses prostitutions,
\item[\vref{Mt 9:2}] couché sur un l.. Et Jésus, voyant
\item[\vref{Mt 9:6}] paralytique, prends ton l. et va ds
\item[\vref{Mc 2:4}] descendirent le petit l. ds lequel le
\item[\vref{Mc 2:9}] prends ton petit l. et marche ?
\item[\vref{Mc 2:11}] Lève-toi, prends ton l., et va ds
\item[\vref{Mc 2:12}] ayant pris son l., il sortit en
\item[\vref{Mc 4:21}] ou sous un l. ? N'est-ce pas pour
\item[\vref{Lu 5:24}] lève-toi, prends ton l., et va ds
\item[\vref{Lu 5:25}] eux, prit le l. sur lequel il
\item[\vref{Lu 17:34}] ds un mm l. : L'un sera pris,
\item[\vref{Jn 5:8}] Lève-toi, prends ton l., et marche.
\item[\vref{Jn 5:12}] dit : Prends ton l., et marche ?
\item[\vref{Ac 9:33}] ds un petit l. depuis huit ans,
\item[\vref{Ac 28:8}] Publius était au l., malade de la
\item[\vref{Hé 13:4}] ts, et le l. sans souillure ; mais
\item[\vref{Ap 2:22}] jeter sur un l., et mettre ds
\end{listverse}

\ConcordanceEntry{Livre}
\vspace{-2mm}
\begin{listverse}
\item[\vref{Ex 24:7}] il prit le l. de l'Alliance et
\item[\vref{No 21:14}] dit ds le l. des batailles de
\item[\vref{De 30:10}] écrites ds ce l. de la loi,
\item[\vref{Jos 1:8}] Que ce l. de la loi ne s'éloigne point
\item[\vref{Jos 8:34}] écrit ds le l. de la loi.
\item[\vref{Jos 10:13}] écrit ds le l. du Juste ? Le
\item[\vref{2 R 22:8}] J'ai trouvé le l. de la loi
\item[\vref{2 Ch 17:9}] avec eux le l. de la loi
\item[\vref{2 Ch 25:26}] écrit ds le l. des rois de
\item[\vref{Né 8:1}] scribe, d'apporter le l. de la loi
\item[\vref{Ps 69:29}] soient effacés du l. de vie, et
\item[\vref{Es 29:11}] les paroles d'un l. cacheté que l'on
\item[\vref{Es 34:16}] Consultez le l. de Yahweh et
\item[\vref{Da 7:10}] tint, et les l. furent ouverts.
\item[\vref{Da 12:1}] inscrits ds le l. seront sauvés.
\item[\vref{Da 12:4}] et scelle le l. jusqu'au temps de
\item[\vref{Mal 3:16}] écouta ; et un l. de souvenir fut
\item[\vref{Mt 1:1}] L. de la généalogie de Jésus-Christ, fils
\item[\vref{Mc 12:26}] lu ds le l. de Moïse, comment
\item[\vref{Lu 3:4}] écrit ds le l. des paroles d'Esaïe,
\item[\vref{Lu 4:17}] lui donna le l. du prophète Esaïe.
\item[\vref{Ac 7:42}] écrit ds le l. des prophètes : Maison
\item[\vref{Ga 3:10}] écrites ds le l. de la loi
\item[\vref{Ph 4:3}] les noms sont écrits ds le l. de vie.
\item[\vref{Hé 9:19}] l'aspersion sur le l. et sur tt
\item[\vref{Ap 1:11}] Ecris ds un l. ce que tu
\item[\vref{Ap 3:5}] son nom du L. de vie, mais
\item[\vref{Ap 5:1}] le trône, un l. écrit en dedans
\item[\vref{Ap 5:2}] digne d'ouvrir le l., et d'en rompre
\item[\vref{Ap 5:3}] pouvait ouvrir le l., ni le regarder.
\item[\vref{Ap 5:4}] digne d'ouvrir le l., ni de le
\item[\vref{Ap 10:2}] main un petit l. ouvert, et il
\item[\vref{Ap 10:8}] prends le petit l. ouvert qui est
\item[\vref{Ap 13:8}] écrits ds le l. de vie de
\item[\vref{Ap 20:12}] dvt Dieu. Des l. furent ouverts. Et
\item[\vref{Ap 21:27}] écrits ds le L. de vie de
\item[\vref{Ap 22:7}] paroles de la prophétie de ce l. !
\item[\vref{Ap 22:9}] paroles de ce l.. Adore Dieu !
\item[\vref{Ap 22:10}] prophétie de ce l.. Car le temps
\item[\vref{Ap 22:18}] prophétie de ce l. : Si quelqu'un y
\item[\vref{Ap 22:19}] des paroles du l. de cette prophétie,
\end{listverse}

\ConcordanceEntry{Livrer}
\vspace{-2mm}
\begin{listverse}
\item[\vref{De 1:27}] afin de ns. l. entre les mains
\item[\vref{De 2:30}] afin de le l. entre tes mains,
\item[\vref{De 2:31}] commencé à te l. Sihon et son
\item[\vref{Jos 2:24}] Certainement, Yahweh a l. tt le pays
\item[\vref{2 R 18:16}] Yahweh, pour les l. au roi d'Assyrie.
\item[\vref{2 R 23:35}] l'or qu'il devait l. à pharaon Néco.
\item[\vref{1 Ch 12:17}] trahir et me l. à mes ennemis,
\item[\vref{Mt 26:16}] cherchait une occasion favorable pour le l.
\item[\vref{Mt 26:45}] l'hom. va être l. entre les mains
\item[\vref{Mc 13:11}] emmèneront pour vs. l., ne soyez pas
\item[\vref{Mc 14:10}] vers les principaux prêtres pour le l.
\item[\vref{Mc 14:11}] cherchait une occasion favorable pour le l.
\item[\vref{Lu 20:20}] afin de le l. au magistrat et
\item[\vref{Lu 22:6}] pour le lr. l. à l'insu de
\item[\vref{Jn 8:34}] dis : Quiconque se l. au péché est
\item[\vref{Ac 25:11}] droit de me l. à eux. J'en
\item[\vref{Ac 25:16}] des Romains de l. quelqu'un à la
\item[\vref{1 Co 5:5}] tel hom. soit l. à Satan pour
\end{listverse}

\ConcordanceEntry{Log}
\vspace{-2mm}
\begin{listverse}
\item[\vref{Lé 14:10}] gâteau, pétrie à l'huile, et un l. d'huile.
\item[\vref{Lé 14:12}] culpabilité avec un l. d'huile ; il agitera
\item[\vref{Lé 14:15}] prêtre prendra du l. d'huile et en
\item[\vref{Lé 14:21}] gâteau, avec un l. d'huile ;
\item[\vref{Lé 14:24}] culpabilité et le l. d'huile, et les
\end{listverse}

\ConcordanceEntry{Loi}
\vspace{-2mm}
\begin{listverse}
\item[\vref{Ex 15:25}] ordon. et une l., et il l'éprouva
\item[\vref{Esd 7:6}] versé ds la l. de Moïse, que
\item[\vref{Ps 19:8}] La l. de Yahweh est parfaite, elle restaure
\item[\vref{Ps 37:31}] La l. de son Dieu est ds son
\item[\vref{Ps 40:9}] volonté, et ta l. est au fond
\item[\vref{Ps 50:16}] Tu énumères mes l. et tu as
\item[\vref{Ps 119:1}] marchent selon la l. de Yahweh.
\item[\vref{Ps 119:70}] moi, je prends plaisir ds ta l.
\item[\vref{Ps 119:97}] Combien j'aime ta l. ! Elle est tt
\item[\vref{Es 2:3}] sentiers ; car la l. sortira de Sion,
\item[\vref{Es 8:20}] A la l. et au témoignage ! Si l'on ne
\item[\vref{Es 42:21}] a magnifié la l. et l'a rendu
\item[\vref{Jé 10:3}] Car les l. des peuples ne sont que vanité.
\item[\vref{Da 6:8}] irrévocable, selon la l. des Mèdes et
\item[\vref{Ha 1:4}] Parce que la l. est sans force,
\item[\vref{Mt 5:17}] venu abolir la l. ou les prophètes ;
\item[\vref{Mt 5:18}] pas de la l. un seul iota
\item[\vref{Mt 7:12}] car c'est la l. et les prophètes.
\item[\vref{Jn 1:17}] car la l. a été donnée par Moïse, la
\item[\vref{Jn 7:19}] pas donné la l. ? Cependant, nul de
\item[\vref{Jn 7:51}] Notre l. condamne-t-elle un hom. avant qu'on l'entende
\item[\vref{Ro 2:13}] qui écoutent la l. qui sont justes
\item[\vref{Ro 2:14}] n'ont pas la l., font naturellement les
\item[\vref{Ro 4:15}] car la l. produit la colère, et là où
\item[\vref{Ro 7:3}] délivrée de la l., de sorte qu'elle
\item[\vref{Ro 7:4}] qui concerne la l., pour être à
\item[\vref{Ro 7:6}] délivrés de la l., étant morts à
\item[\vref{Ro 7:7}] dirons-ns. dc ? La l. est-elle péché ? A
\item[\vref{Ro 7:8}] que sans la l., le péché est
\item[\vref{Ro 7:12}] La l. dc est sainte, et le commandement
\item[\vref{Ro 7:22}] plaisir à la l. de Dieu, quant
\item[\vref{Ro 7:23}] membres une autre l., qui combat contre
\item[\vref{Ro 8:2}] Parce que la l. de l'Esprit de
\item[\vref{Ro 8:3}] impossible à la l., parce qu'elle était
\item[\vref{1 Co 9:21}] qui sont sans l., com. si j'étais
\item[\vref{1 Co 15:56}] la puissance du péché c'est la l.
\item[\vref{Ga 2:19}] c'est par la l. que je suis
\item[\vref{Ga 3:23}] garde de la l., en vue de
\item[\vref{Ga 4:4}] né d'une fem., né sous la l.,
\item[\vref{Ga 4:21}] être sous la l., ne comprenez-vs. pas
\item[\vref{Ga 5:3}] de pratiquer la l. tt entière.
\item[\vref{Ga 5:14}] Car tte la l. est accomplie ds
\item[\vref{1 Ti 1:8}] savons que la l. est bonne pour
\item[\vref{1 Ti 1:9}] juste que la l. a été établie,
\item[\vref{Hé 7:12}] y ait aussi un changement de l.
\item[\vref{Hé 9:19}] commandements de la l., prit le sang
\item[\vref{Hé 10:1}] Car la l. qui possède l'ombre des biens à
\item[\vref{Ja 1:25}] regards ds la l. parfaite, la loi
\item[\vref{Ja 2:8}] vs. accomplissez la l. royale qui est
\item[\vref{Ja 2:10}] observe tte la l., mais pèche contre
\item[\vref{Ja 2:11}] tues, tu deviens transgresseur de la l.
\item[\vref{Ja 2:12}] jugés par la l. de la liberté,
\item[\vref{Ja 4:11}] médit de la l. et juge la
\item[\vref{1 Jn 3:4}] pèche transgresse la l., car le péché
\end{listverse}

\ConcordanceEntry{Loin}
\vspace{-2mm}
\begin{listverse}
\item[\vref{Ps 139:2}] tu discernes de l. ma pensée.
\item[\vref{Ps 139:7}] Où irai-je l. de ton Esprit,
\item[\vref{Es 33:13}] Vous qui êtes l., écoutez ce que
\item[\vref{Jé 12:2}] mais tu es l. de leurs cœurs.
\item[\vref{Jé 23:23}] suis-je pas aussi un Dieu de l. ?
\item[\vref{Os 10:5}] parce que sa gloire est transportée l. d'elle.
\item[\vref{Jon 1:3}] s'enfuir à Tarsis, l. de la face
\item[\vref{Mt 26:39}] cette coupe passe l. de moi. Toutefois
\item[\vref{Mt 26:58}] le suivit de l. jusqu'à la cour
\item[\vref{Mt 27:55}] qui regardaient de l., et qui avaient
\item[\vref{Mc 12:34}] Tu n'es pas l. du Royaume de
\item[\vref{Lu 15:20}] il était encore l., son père le
\item[\vref{Lu 22:42}] éloigner cette coupe l. de moi ! Toutefois,
\item[\vref{Lu 24:28}] faisait com. s'il voulait aller plus l.
\item[\vref{Ac 2:39}] ceux qui sont l., autant que le
\item[\vref{Ac 17:27}] ne soit pas l. de chacun de
\item[\vref{Ac 22:21}] je t'enverrai au l. vers les Gentils.
\item[\vref{2 Co 5:6}] corps, ns. sommes l. du Seign.,
\item[\vref{Ep 2:17}] vs. qui étiez l., et à ceux
\item[\vref{2 Th 1:9}] une ruine éternelle, l. de la face
\item[\vref{Hé 11:13}] ont vues de l., crues, et saluées,
\item[\vref{Ja 5:19}] vs. s'est égaré l. de la vérité,
\item[\vref{2 Pi 1:9}] voit pas de l., ayant oublié la
\item[\vref{Ap 9:6}] désireront mourir, mais la mort fuira l. d'eux.
\end{listverse}

\ConcordanceEntry{Loïs}
\vspace{-2mm}
\begin{listverse}
\item[\vref{2 Ti 1:5}] premièrement habité en L., ta grand-mère et
\end{listverse}

\ConcordanceEntry{Longtemps}
\vspace{-2mm}
\begin{listverse}
\item[\vref{Es 41:26}] le connaissions ? Et l. d'avance, que ns.
\item[\vref{Es 42:14}] suis tu dès l. ; me tiendrais-je en
\item[\vref{Es 46:10}] la fin, et l. auparavant, les choses
\item[\vref{Es 57:11}] et mm depuis l., et tu ne
\item[\vref{Mt 25:19}] L. après, le maître de ces serviteurs
\item[\vref{Mc 15:44}] lui demanda s'il était mort depuis l.
\item[\vref{Lu 8:27}] ville, qui depuis l. était possédé de
\item[\vref{Lu 12:45}] Mon maître tarde l. à venir. S'il
\item[\vref{Jn 5:6}] déjà malade depuis l., lui dit : Veux-tu
\item[\vref{Jn 14:9}] SUIS depuis si l. avec vs., et
\item[\vref{Ac 28:6}] mais après avoir l. attendu, voyant qu'il
\item[\vref{Ga 4:1}] Or aussi l. que l'héritier est enfant, je dis
\item[\vref{Ep 6:3}] que tu vives l. sur la terre.
\item[\vref{Hé 4:7}] par David si l. après, selon ce
\item[\vref{2 Pi 2:3}] est destinée depuis l. ne tarde pas,
\item[\vref{Jud 1:4}] est écrite depuis l., des impies, qui
\end{listverse}

\ConcordanceEntry{Longue attente}
\vspace{-2mm}
\begin{listverse}
\item[\vref{Ro 2:4}] et de sa longue attente, ne reconnaissant pas
\end{listverse}

\ConcordanceEntry{Lot}
\vspace{-2mm}
\begin{listverse}
\item[\vref{Ge 11:31}] fils Abram, et L. fils de son
\item[\vref{Ge 12:4}] avait dit, et L. alla avec lui.
\item[\vref{Ge 13:1}] lui appartenait, et L. avec lui.
\item[\vref{Ge 13:5}] L. aussi, qui marchait avec Abram, avait
\item[\vref{Ge 14:12}] Ils prirent aussi L., fils du frère
\item[\vref{Ge 14:16}] il ramena aussi L., son frère, ses
\item[\vref{Ge 19:26}] la fem. de L. regarda en arrière,
\item[\vref{Ge 19:29}] d'Abraham ; il envoya L. hors du milieu
\item[\vref{Ge 19:36}] deux filles de L. conçurent de lr.
\item[\vref{De 2:9}] Ar en héritage aux fils de L.
\item[\vref{Ps 83:9}] servi de bras aux fils de L.. Sélah.
\item[\vref{Lu 17:28}] aux jours de L. : On mangeait, on
\item[\vref{Lu 17:29}] le jour où L. sortit de Sodome,
\item[\vref{2 Pi 2:7}] délivré le juste L., qui avait eu
\end{listverse}

\ConcordanceEntry{Louange}
\vspace{-2mm}
\begin{listverse}
\item[\vref{Lé 19:24}] sainte à la l. de Yahweh.
\item[\vref{2 Ch 30:21}] retentissaient à la l. de Yahweh.
\item[\vref{Né 9:5}] de tte bénédiction et de tte l. !
\item[\vref{Ps 22:4}] au milieu des l. d'Israël.
\item[\vref{Ps 33:1}] de Yahweh ! Sa l. sied aux hommes
\item[\vref{Ps 34:2}] tt temps ; sa l. sera continuellement ds
\item[\vref{Ps 78:4}] à venir les l. de Yahweh, sa
\item[\vref{Ps 109:1}] Dieu de ma l., ne te tais
\item[\vref{Ps 119:171}] lèvres publieront ta l. qnd tu m'auras
\item[\vref{Es 43:21}] je me suis formé racontera mes l.
\item[\vref{Es 61:3}] un manteau de l. au lieu d'un
\item[\vref{Mt 21:16}] as tiré des l. de la bouche
\item[\vref{Ac 16:25}] et chantaient les l. de Dieu, et
\item[\vref{Ro 2:29}] la lettre. La l. de ce Juif
\item[\vref{Ro 15:9}] de cela ta l. parmi les Gentils,
\item[\vref{1 Co 4:5}] de Dieu la l. qui lui sera
\item[\vref{2 Co 8:18}] frère dont la l., qu'il s'est acquise
\item[\vref{Ep 1:12}] soyons à la l. de sa gloire,
\item[\vref{Ep 1:14}] acquis à la l. de sa gloire.
\item[\vref{Ph 1:11}] à la gloire et à la l. de Dieu.
\item[\vref{Ph 4:8}] vertu et qq l. ; soient l'objet de
\item[\vref{Hé 13:15}] un sacrifice de l., c'est-à-dire, le fruit
\item[\vref{1 Pi 1:7}] pour résultat la l., l'honneur et la
\item[\vref{Ap 5:13}] à l'Agneau, soient l., honneur, gloire, et
\item[\vref{Ap 7:12}] disant : Amen ! La l., la gloire, la
\end{listverse}

\ConcordanceEntry{Louer}
\vspace{-2mm}
\begin{listverse}
\item[\vref{Ge 29:35}] Cette fois je l. Yahweh. C'est pourquoi
\item[\vref{2 S 22:4}] Je m'écrie : L. soit Yahweh ! Et
\item[\vref{1 Ch 16:4}] célébrer, remercier, et l. le Dieu d'Israël.
\item[\vref{2 Ch 5:13}] célébrer et pour l. Yahweh, et firent
\item[\vref{Né 12:24}] vis-à-vis d'eux, pour l. et célébrer, selon
\item[\vref{Ps 30:13}] ma langue te l. et ne se
\item[\vref{Ps 65:2}] calme, on te l. ds Sion, et
\item[\vref{Ps 96:4}] et digne d'être l., il est redoutable
\item[\vref{Ps 148:2}] L.-le, vs. ts ses anges ! Louez-le,
\item[\vref{Ps 150:2}] L.-le pour ses hauts faits ! Louez-le
\item[\vref{Pr 27:2}] Qu'un autre te l., et non pas
\item[\vref{Es 64:10}] nos pères te l., a été brûlée
\item[\vref{Da 5:4}] vin, et ils l. leurs dieux d'or,
\item[\vref{Da 6:10}] priait, et il l. son Dieu, com.
\item[\vref{Mt 20:1}] jour afin de l. des ouvriers pour
\item[\vref{Lu 2:13}] de l'armée céleste, l. Dieu, et disant :
\item[\vref{Lu 16:8}] Et le maître l. l'économe infidèle de
\item[\vref{Lu 18:43}] Et tt le peuple voyant cela, l. Dieu.
\item[\vref{Lu 19:37}] se mit à l. Dieu à haute
\end{listverse}

\ConcordanceEntry{Loup}
\vspace{-2mm}
\begin{listverse}
\item[\vref{Ge 49:27}] Benjamin est un l. qui déchirera ; le
\item[\vref{Es 11:6}] Le l. habitera avec l'agneau, et le léopard
\item[\vref{Es 65:25}] Le l. et l'agneau paîtront ensemble, le lion
\item[\vref{Jé 5:6}] les tue, le l. du soir les
\item[\vref{Mt 7:15}] ce sont des l. ravisseurs.
\item[\vref{Jn 10:12}] voit venir le l., abandonne les brebis,
\item[\vref{Ac 20:29}] parmi vs. des l. très dangereux, qui
\end{listverse}

\ConcordanceEntry{Luc}
\vspace{-2mm}
\begin{listverse}
\item[\vref{Col 4:14}] L., le médecin bien-aimé, vs. salue, ainsi
\item[\vref{2 Ti 4:11}] L. est seul avec moi ; prends Marc
\item[\vref{Phm 1:24}] Aristarque, Démas, et L., mes compagnons d'œuvre.
\end{listverse}

\ConcordanceEntry{Lumière}
\vspace{-2mm}
\begin{listverse}
\item[\vref{Ge 1:3}] dit : Que la l. apparaisse ! Et la
\item[\vref{Ge 1:4}] vit que la l. était bonne ; et
\item[\vref{Ex 10:23}] eut de la l. ds le lieu
\item[\vref{Job 24:16}] marquées le jour, ils haïssent la l.
\item[\vref{Ps 4:7}] sur ns. la l. de ta face,
\item[\vref{Ps 27:1}] Yahweh est ma l. et mon salut :
\item[\vref{Ps 36:10}] et par ta l. ns. voyons la
\item[\vref{Ps 43:3}] Envoie ta l. et ta vérité,
\item[\vref{Ps 90:8}] et à la l. de ta face
\item[\vref{Ps 97:11}] La l. est faite pour le juste, et
\item[\vref{Ps 104:2}] Il s'enveloppe de l. com. d'un vêt. ;
\item[\vref{Pr 4:18}] est com. la l. resplendissante, dont l'éclat
\item[\vref{Ec 11:7}] vrai que la l. est douce, et
\item[\vref{Es 2:5}] marchons ds la l. de Yahweh.
\item[\vref{Es 5:20}] font les ténèbres l., et la lumière
\item[\vref{Es 9:1}] voit une grande l., et la lumière
\item[\vref{Es 42:6}] peuple et la l. des nations,
\item[\vref{Es 42:16}] les ténèbres en l., et les choses
\item[\vref{Es 45:7}] Je forme la l., et je crée
\item[\vref{Es 60:1}] illuminée, car ta l. arrive, et la
\item[\vref{Os 6:5}] bouche, et mes jugements apporteront la l.
\item[\vref{Za 14:7}] soir il y aura de la l.
\item[\vref{Mt 4:16}] vu une grande l. ; et pour ceux
\item[\vref{Mt 5:14}] Vous êtes la l. du monde. Une
\item[\vref{Mt 5:16}] Ainsi, que votre l. luise dvt les
\item[\vref{Lu 11:34}] La l. du corps c'est l'œil. Si dc
\item[\vref{Lu 12:3}] entendu ds la l. ; et ce que
\item[\vref{Lu 16:8}] lr. génération, que les enfants de l.
\item[\vref{Jn 1:5}] Et la l. luit ds les ténèbres, mais les
\item[\vref{Jn 1:9}] Cette l. était la véritable lumière, qui, en
\item[\vref{Jn 3:20}] mal, hait la l., et ne vient
\item[\vref{Jn 5:35}] un peu de temps à sa l.
\item[\vref{Jn 8:12}] Je suis la l. du monde ; celui
\item[\vref{Jn 12:35}] lr. dit : La l. est encore avec
\item[\vref{Ac 9:3}] à coup une l. resplendit du ciel
\item[\vref{Ac 26:18}] ténèbres à la l., et de la
\item[\vref{1 Co 4:5}] il mettra en l. les choses cachées
\item[\vref{2 Co 4:4}] éclairés par la l. de l'Evangile de
\item[\vref{2 Co 6:14}] a-t-il entre la l. et les ténèbres ?
\item[\vref{2 Co 11:14}] lui-mm se déguise en ange de l.
\item[\vref{Ep 5:8}] mntnt vs. êtes l. ds le Seign.
\item[\vref{Ep 5:13}] manifesté par la l., car la lumière
\item[\vref{1 Th 5:5}] enfants de la l., et des enfants
\item[\vref{1 Ti 6:16}] qui habite une l. inaccessible, que nul
\item[\vref{1 Pi 2:9}] appelés des ténèbres à sa merveilleuse l.
\item[\vref{1 Jn 1:5}] que Dieu est l. et qu'il n'y
\item[\vref{1 Jn 2:8}] que la véritable l. paraît déjà.
\item[\vref{Ap 22:5}] besoin ni de l., ni de lampe,
\end{listverse}

\ConcordanceEntry{Lunatique}
\vspace{-2mm}
\begin{listverse}
\item[\vref{Mt 17:15}] fils qui est l. et misérablement affligé ;
\end{listverse}

\ConcordanceEntry{Lune}
\vspace{-2mm}
\begin{listverse}
\item[\vref{Ge 37:9}] le soleil, la l. et onze étoiles
\item[\vref{De 17:3}] soleil, dvt la l., ou dvt tte
\item[\vref{Jos 10:12}] Gabaon, et toi l., sur la vallée
\item[\vref{Jos 10:13}] s'arrêta, et la l. aussi s'arrêta, jusqu'à
\item[\vref{1 Ch 23:31}] sabbat, aux nouvelles l., et aux fêtes
\item[\vref{Job 25:5}] aille jusqu'à la l., elle ne luit
\item[\vref{Job 31:26}] plus, et la l. qnd elle marchait
\item[\vref{Ps 8:4}] tes doigts, la l. et les étoiles
\item[\vref{Ps 121:6}] point, ni la l. pendant la nuit.
\item[\vref{Ps 148:3}] vs., soleil et l. ! Louez-le, vs. ttes,
\item[\vref{Joë 2:31}] ténèbres, et la l. en sang, avant
\item[\vref{Mt 24:29}] soleil s'obscurcira, la l. ne donnera plus
\item[\vref{Ac 2:20}] ténèbres, et la l. en sang, avant
\item[\vref{1 Co 15:41}] l'éclat de la l., autre l'éclat des
\item[\vref{Col 2:16}] jour de nouvelle l., ou de sabbat,
\item[\vref{Ap 6:12}] crin, et la l. entière devint com.
\item[\vref{Ap 8:12}] tiers de la l., et le tiers
\item[\vref{Ap 12:1}] du soleil, la l. sous ses pieds,
\item[\vref{Ap 21:23}] ni de la l. pour l'éclairer, car
\end{listverse}

\ConcordanceEntry{Lutter}
\vspace{-2mm}
\begin{listverse}
\item[\vref{Ge 32:24}] Alors un hom. l. avec lui jusqu'au
\item[\vref{Ge 32:28}] le vainqueur en l. avec Dieu et
\item[\vref{Os 12:5}] fut vainqueur en l. avec l'Ange, et
\item[\vref{1 Co 9:25}] Or quiconque l., vit entièrement de
\item[\vref{Ep 6:12}] n'avons pas à l. contre la chair
\end{listverse}

\ConcordanceEntry{Lydie}
\vspace{-2mm}
\begin{listverse}
\item[\vref{Ac 16:14}] L'une d'elles, appelée L., marchande de pourpre,
\item[\vref{Ac 16:40}] ils entrèrent chez L., et, après avoir
\end{listverse}

\ConcordanceEntry{Lystre}
\vspace{-2mm}
\begin{listverse}
\item[\vref{Ac 14:6}] de Lycaonie, à L., à Derbe, et
\item[\vref{Ac 14:8}] A L., se tenait assis un hom. impotent
\item[\vref{Ac 14:21}] ils retournèrent à L., à Icone, et
\item[\vref{Ac 16:1}] Derbe et à L.. Et voici, il
\item[\vref{2 Ti 3:11}] Icone et à L.. Quelles persécutions n'ai-je
\end{listverse}

\ConcordanceEntry{Maaca}
\vspace{-2mm}
\begin{listverse}
\item[\vref{2 S 3:3}] Absalom, fils de M., fille de Talmaï,
\item[\vref{1 R 15:13}] à sa mère M., parce qu'elle avait
\item[\vref{2 Ch 11:20}] elle, il prit M., fille d'Absalom, qui
\end{listverse}

\ConcordanceEntry{Macédoine}
\vspace{-2mm}
\begin{listverse}
\item[\vref{Ac 16:9}] disant : Passe en M. et secours-ns. !
\item[\vref{2 Co 1:16}] chez vs. en M., puis de Macédoine
\item[\vref{2 Co 8:1}] a faite aux églises de la M.
\end{listverse}

\ConcordanceEntry{Macpéla}
\vspace{-2mm}
\begin{listverse}
\item[\vref{Ge 23:17}] qui était à M., vis-à-vis de Mamré,
\item[\vref{Ge 49:30}] du champ de M., vis-à-vis de Mamré,
\item[\vref{Ge 50:13}] du champ de M., qu'Abraham avait achetée
\end{listverse}

\ConcordanceEntry{Madian, Madianites}
\vspace{-2mm}
\begin{listverse}
\item[\vref{Ge 37:28}] com. les marchands M. passaient, ils tirèrent
\item[\vref{Ex 2:15}] au pays de M. et s'assit près
\item[\vref{No 31:2}] d'Israël sur les M., puis tu seras
\item[\vref{Jg 6:1}] les mains de M. pendant sept ans.
\item[\vref{Jg 7:14}] Dieu a livré M. et tt le
\item[\vref{Ac 7:29}] le pays de M., où il eut
\end{listverse}

\ConcordanceEntry{Magicien}
\vspace{-2mm}
\begin{listverse}
\item[\vref{Da 2:10}] chose à qq m., astrologue ou Chaldéen
\item[\vref{Ac 13:6}] là un certain m., faux prophète Juif,
\end{listverse}

\ConcordanceEntry{Magie}
\vspace{-2mm}
\begin{listverse}
\item[\vref{2 Ch 33:6}] il pratiquait la m., les sorcelleries et
\item[\vref{Ac 8:11}] il les avait éblouis par sa m.
\end{listverse}

\ConcordanceEntry{Magistrat}
\vspace{-2mm}
\begin{listverse}
\item[\vref{Ro 13:3}] mauvaise que les m. sont à craindre.
\item[\vref{Tit 3:1}] d'être soumis aux m. et aux autorités,
\end{listverse}

\ConcordanceEntry{Magnificence}
\vspace{-2mm}
\begin{listverse}
\item[\vref{1 Ch 29:11}] toi qu'appartiennent la m., la puissance, la
\item[\vref{Job 40:5}] Pare-toi mntnt de m. et de grandeur,
\item[\vref{Ps 96:6}] splendeur et la m. marchent dvt lui,
\item[\vref{Pr 31:25}] force et la m. sont son vêt.,
\item[\vref{Es 61:3}] deuil, que la m. lr. sera donnée
\item[\vref{1 Pi 3:3}] d'or ou la m. des habits,
\item[\vref{Jud 1:25}] Seign., soient gloire, m., force et puissance,
\end{listverse}

\ConcordanceEntry{Magnifique}
\vspace{-2mm}
\begin{listverse}
\item[\vref{Ex 15:11}] est com. toi, m. en sainteté, digne
\item[\vref{1 Ch 22:5}] Yahweh doit être m. en excellence, en
\item[\vref{Ps 8:2}] ton Nom est m. sur tte la
\item[\vref{Lu 9:43}] de la puissance m. de Dieu. Et
\item[\vref{2 Pi 1:17}] de la gloire m. : Celui-ci est mon
\item[\vref{Ap 21:24}] ont de plus m. et de plus
\end{listverse}

\ConcordanceEntry{Magog}
\vspace{-2mm}
\begin{listverse}
\item[\vref{Ez 38:2}] au pays de M., vers le prince
\item[\vref{Ap 20:8}] terre, Gog et M., afin de les
\end{listverse}

\ConcordanceEntry{Mahanaïm}
\vspace{-2mm}
\begin{listverse}
\item[\vref{Ge 32:2}] à ce lieu le nom de M.
\item[\vref{2 S 17:24}] David arriva à M.. Et Absalom passa
\item[\vref{2 S 19:32}] avait séjourné à M., car c'était un
\end{listverse}

\ConcordanceEntry{Main}
\vspace{-2mm}
\begin{listverse}
\item[\vref{Ge 3:22}] qu'il n'avance sa m., et aussi qu'il
\item[\vref{Ge 14:22}] Je lève ma m. vers Yahweh, le
\item[\vref{Ex 6:1}] contraint par une m. puissante, il les
\item[\vref{Ex 17:11}] Moïse élevait sa m., Israël était alors
\item[\vref{1 S 7:13}] territoire d'Israël. La m. de Yahweh fut
\item[\vref{1 S 16:23}] jouait de sa m. ; et Saül en
\item[\vref{1 Ch 13:9}] Uzza étendit sa m. pour retenir l'arche,
\item[\vref{Ps 119:173}] Que ta m. me soit en aide, parce que
\item[\vref{Ps 127:4}] flèches ds la m. d'un hom. puissant,
\item[\vref{Ps 139:10}] là aussi ta m. me conduira, et
\item[\vref{Pr 10:4}] La m. paresseuse fait devenir pauvre, mais la
\item[\vref{Ec 2:24}] aussi que cela vient de la m. de Dieu.
\item[\vref{Es 26:11}] Yahweh, qnd ta m. est élevée, ils
\item[\vref{Es 53:10}] plaisir de Yahweh prospérera en sa m.
\item[\vref{Es 59:1}] Voici, la m. de Yahweh n'est
\item[\vref{Jé 1:9}] Yahweh avança sa m., et toucha ma
\item[\vref{Mt 5:30}] Si ta m. droite est pour toi une occasion
\item[\vref{Lu 22:21}] Cependant voici, la m. de celui qui
\item[\vref{Ac 7:50}] Ma m. n'a-t-elle pas fait ttes ces choses ?
\item[\vref{Ac 18:10}] ne mettra la m. sur toi pour
\item[\vref{Ga 2:9}] à Barnabas, la m. d'association, afin que
\item[\vref{1 Ti 5:22}] N'impose les m. à personne avec
\item[\vref{2 Ti 1:6}] as reçu par l'imposition de mes m.
\item[\vref{Hé 10:31}] tomber entre les m. du Dieu vivant.
\item[\vref{Hé 12:12}] Fortifiez dc vos m. languissantes et vos
\item[\vref{1 Pi 5:6}] sous la puissante m. de Dieu, afin
\item[\vref{Ap 17:4}] tenait à la m. une coupe d'or,
\end{listverse}

\ConcordanceEntry{Maison}
\vspace{-2mm}
\begin{listverse}
\item[\vref{Ge 7:1}] et tte ta m. ds l'arche ; car
\item[\vref{Ge 12:1}] et de la m. de ton père,
\item[\vref{Ge 28:17}] C'est ici la m. de Dieu, et
\item[\vref{1 S 2:35}] lui édifierai une m. stable, et il
\item[\vref{1 S 5:2}] l'emmenèrent ds la m. de Dagon, et
\item[\vref{1 R 6:1}] Salomon bâtit la m. de Yahweh, la
\item[\vref{2 Ch 7:16}] je sanctifie cette m., afin que mon
\item[\vref{Ps 23:6}] j'habiterai ds la m. de Yahweh pour
\item[\vref{Ps 127:1}] ne bâtit la m., ceux qui la
\item[\vref{Pr 19:14}] ses pères une m. et des richesses,
\item[\vref{Pr 24:3}] La m. est établie par la sagesse, et
\item[\vref{Es 64:10}] Notre m. sainte et glorieuse, où nos pères
\item[\vref{Es 66:1}] mes pieds. Quelle m. me bâtiriez-vs., et
\item[\vref{Am 5:25}] Est-ce à moi, m. d'Israël, que vs.
\item[\vref{Ag 2:9}] de cette dernière m. sera plus grande
\item[\vref{Mal 3:10}] provision ds ma m. ; et dès mntnt
\item[\vref{Mt 7:24}] a bâti sa m. sur le roc.
\item[\vref{Mt 10:6}] perdues de la m. d'Israël.
\item[\vref{Mt 12:44}] retournerai ds ma m., d'où je suis
\item[\vref{Mt 21:13}] est écrit : Ma m. sera appelée une
\item[\vref{Lu 14:23}] afin que ma m. soit remplie.
\item[\vref{Jn 2:16}] pas de la m. de mon Père
\item[\vref{Ep 2:19}] gens de la m. de Dieu ;
\item[\vref{1 Ti 3:5}] diriger sa propre m., comment pourra-t-il gouverner
\item[\vref{2 Ti 2:20}] ds une grande m., il n'y a
\item[\vref{Hé 3:6}] Fils sur sa m. ; et ns. sommes
\item[\vref{1 Pi 2:5}] pour être une m. spirituelle, et une
\item[\vref{1 Pi 4:17}] commence par la m. de Dieu. Or
\end{listverse}

\ConcordanceEntry{Maître}
\vspace{-2mm}
\begin{listverse}
\item[\vref{Ge 14:19}] le Dieu Très-Haut, M. du ciel et
\item[\vref{Ge 27:37}] l'ai établi ton m., et lui ai
\item[\vref{Pr 16:32}] celui qui est m. de son cœur,
\item[\vref{Mt 9:38}] Priez dc le M. de la moisson
\item[\vref{Mt 23:8}] pas appeler, notre m. ; car Christ seul
\item[\vref{Mc 5:35}] donnes-tu encore de la peine au M. ?
\item[\vref{Lu 8:24}] réveillèrent, en disant : M., Maître ! Nous périssons !
\item[\vref{Jn 8:4}] ils lui dirent : M., cette fem. a
\item[\vref{Jn 11:28}] lui disant : Le M. est ici, et
\item[\vref{Ac 16:16}] profit à ses m., ns. rencontra,
\item[\vref{Ac 19:16}] eux, se rendit m. de deux d'entre
\item[\vref{Ro 2:20}] des insensés, le m. des ignorants, ayant
\item[\vref{Ga 4:1}] rien d'un esclave, quoiqu'il soit le m. de tt.
\item[\vref{Ep 6:5}] obéissez à vos m. selon la chair
\item[\vref{Col 4:1}] M., accordez à vos serviteurs ce qui
\item[\vref{Hé 5:12}] devriez être des m., vu le temps,
\item[\vref{1 Pi 2:18}] crainte à vos m., non seulement à
\end{listverse}

\ConcordanceEntry{Majesté}
\vspace{-2mm}
\begin{listverse}
\item[\vref{Ex 15:7}] grandeur de ta m. ceux qui s'élevaient
\item[\vref{Job 37:22}] en Dieu une m. redoutable.
\item[\vref{Ps 8:2}] as établi ta m. au-dessus des cieux.
\item[\vref{Ps 93:1}] est revêtu de m., Yahweh est revêtu
\item[\vref{Za 6:13}] sera rempli de m.. Il s'assiéra et
\item[\vref{Hé 1:3}] droite de la M. divine ds les
\item[\vref{2 Pi 1:16}] ayant vu sa m. de nos propres
\end{listverse}

\ConcordanceEntry{Mal (le)}
\vspace{-2mm}
\begin{listverse}
\item[\vref{Ge 2:9}] la connaissance du bien et du m.
\item[\vref{Ge 4:7}] si tu agis m., le péché est
\item[\vref{Ge 6:5}] cœur n'était que m. en tt temps.
\item[\vref{Ge 39:9}] un si grand m. et pécherais-je contre
\item[\vref{Ge 50:15}] rendre tt le m. que ns. lui
\item[\vref{Ex 32:12}] repens-toi de ce m. que tu veux
\item[\vref{1 S 12:25}] à faire le m., vs. serez détruits
\item[\vref{1 Ch 16:22}] faites point de m. à mes prophètes !
\item[\vref{Né 6:2}] avaient comploté de me faire du m.
\item[\vref{Job 5:19}] la septième le m. ne te touchera
\item[\vref{Job 35:6}] tu pèches, quel m. fais-tu à Dieu ?
\item[\vref{Job 42:11}] de tt le m., que Yahweh avait
\item[\vref{Ps 34:14}] ta langue du m., et tes lèvres
\item[\vref{Ps 97:10}] Yahweh, haïssez le m. ! Il garde les
\item[\vref{Pr 3:7}] yeux ; crains Yahweh, et détourne-toi du m.
\item[\vref{Pr 11:19}] qui poursuit le m. aboutit à sa
\item[\vref{Es 59:15}] se retire du m. est exposé au
\item[\vref{Am 5:15}] Haïssez le m., et aimez le
\item[\vref{Lu 23:41}] crimes ; mais celui-ci n'a fait aucun m.
\item[\vref{Jn 3:20}] quiconque fait le m., hait la lumière,
\item[\vref{Jn 5:29}] auront fait le m. ressusciteront pour la
\item[\vref{Jn 17:15}] monde, mais de les préserver du m.
\item[\vref{Ac 28:5}] ds le feu, ne ressentit aucun m.
\item[\vref{Ro 12:9}] en horreur le m., attachez-vs. à ce
\item[\vref{Ja 1:13}] tenté par le m., et aussi ne
\item[\vref{Ja 3:16}] le désordre, et tte sorte de m.
\item[\vref{Ja 4:3}] que vs. demandez m., afin de tt
\item[\vref{1 Pi 3:9}] Ne rendez pas m. pour mal, ou
\item[\vref{2 Pi 2:14}] attirent les âmes m. affermies ; ils ont
\item[\vref{1 Jn 5:19}] monde entier est plongé ds le m.
\item[\vref{Ap 11:5}] lr. faire du m., du feu sort
\end{listverse}

\ConcordanceEntry{Malachie}
\vspace{-2mm}
\begin{listverse}
\item[\vref{Mal 1:1}] contre Israël, par le moyen de M.
\end{listverse}

\ConcordanceEntry{Malade}
\vspace{-2mm}
\begin{listverse}
\item[\vref{Ge 48:1}] ton père est m.. Et il prit
\item[\vref{2 R 20:1}] temps-là, Ezéchias fut m. à la mort.
\item[\vref{Ca 2:5}] avec des pommes, car je suis m. d'amour.
\item[\vref{Ez 34:16}] celle qui est m. ; mais je détruirai
\item[\vref{Mt 25:36}] m'avez vêtu ; j'étais m., et vs. m'avez
\item[\vref{Mc 1:32}] amena ts les m. et les démoniaques.
\item[\vref{Mc 2:17}] médecin, mais les m.. Je ne suis
\item[\vref{Lu 9:1}] avec le pouvoir de guérir les m.
\item[\vref{Jn 5:5}] là un hom. m. depuis trente-huit ans.
\item[\vref{Jn 11:2}] c'était son frère Lazare qui était m.
\item[\vref{Ja 5:14}] parmi vs. est-il m. ? Qu'il appelle les
\end{listverse}

\ConcordanceEntry{Maladie}
\vspace{-2mm}
\begin{listverse}
\item[\vref{Ex 23:25}] et j'ôterai les m. du milieu de
\item[\vref{De 7:15}] de toi tte m. ; il ne t'enverra
\item[\vref{De 28:59}] et persistantes, des m. malignes et persistantes.
\item[\vref{2 R 13:14}] atteint de la m. dont il mourut ;
\item[\vref{Ps 41:4}] tu le soulages ds ttes ses m.
\item[\vref{Es 53:4}] a porté nos m., et il s'est
\item[\vref{Mt 8:17}] nos faiblesses et a porté nos m.
\item[\vref{Mt 10:1}] ttes sortes de m. et ttes sortes
\item[\vref{Lu 5:15}] être guéries par lui de leurs m.
\item[\vref{Lu 8:2}] malins et de m. : Marie de Magdala,
\item[\vref{Jn 5:4}] que soit la m. dont il souffrait.
\item[\vref{Ac 19:12}] guéris de leurs m., et les esprits
\item[\vref{1 Ti 6:4}] il a la m. des questions et
\end{listverse}

\ConcordanceEntry{Malchus}
\vspace{-2mm}
\begin{listverse}
\item[\vref{Jn 18:10}] coupa l'oreille droite. Ce serviteur s'appelait M.
\end{listverse}

\ConcordanceEntry{Malcom, Milcom}
\vspace{-2mm}
\begin{listverse}
\item[\vref{1 R 11:5}] Sidoniens, et après M., l'abomination des Ammonites.
\item[\vref{1 R 11:33}] Moab, et dvt M., le dieu des
\item[\vref{So 1:5}] lui, et qui jurent aussi par M.,
\end{listverse}

\ConcordanceEntry{Malédiction}
\vspace{-2mm}
\begin{listverse}
\item[\vref{Ge 27:12}] sur moi sa m. et non pas
\item[\vref{De 11:26}] dvt vs. la bénédiction et la m. :
\item[\vref{De 28:15}] voici ttes les m. qui viendront sur
\item[\vref{2 R 22:19}] d'épouvante et de m., et parce que
\item[\vref{Ps 10:7}] est pleine de m., de tromperies et
\item[\vref{Ps 59:13}] des discours de m. et de mensonge.
\item[\vref{Ps 109:17}] Puisqu'il aime la m., que la malédiction
\item[\vref{Pr 3:33}] La m. de Yahweh est ds la maison
\item[\vref{Pr 26:2}] s'envoler, ainsi la m. sans cause n'atteint
\item[\vref{Jé 24:9}] raillerie, et en m., par ts les
\item[\vref{Za 5:3}] dit : C'est la m. qui sort sur
\item[\vref{Za 5:4}] Je déploierai cette m. dit Yahweh des
\item[\vref{Mal 2:2}] sur vs. la m., et je maudirai
\item[\vref{Ro 3:14}] est pleine de m. et d'amertume ;
\item[\vref{Ga 3:10}] sont sous la m. ; car il est
\item[\vref{Ga 3:13}] rachetés de la m. de la loi
\item[\vref{Ja 3:10}] bénédiction et la m.. Il ne faut
\item[\vref{2 Pi 2:14}] cupidité ; ce sont des enfants de m.,
\end{listverse}

\ConcordanceEntry{Malfaiteur}
\vspace{-2mm}
\begin{listverse}
\item[\vref{Mc 15:28}] a été mis au rang des m.
\item[\vref{Lu 23:33}] là, et les m. aussi, l'un à
\item[\vref{Jn 18:30}] n'était pas un m., ns. ne te
\item[\vref{2 Ti 2:9}] chaînes com. un m. ; cependant, la parole
\item[\vref{1 Pi 2:12}] vs. étiez des m., ils remarquent vos
\item[\vref{1 Pi 3:16}] vs., com. si vs. étiez des m.
\item[\vref{1 Pi 4:15}] ou voleur, ou m. ou com. se
\end{listverse}

\ConcordanceEntry{Malheur}
\vspace{-2mm}
\begin{listverse}
\item[\vref{Job 31:29}] suis réjoui du m. de mon ennemi,
\item[\vref{Ps 27:5}] au jour du m., il me tiendra
\item[\vref{Ps 37:19}] au jour du m., mais ils sont
\item[\vref{Ps 41:2}] Yahweh le délivrera au jour du m. ;
\item[\vref{Ps 90:15}] d'années que ns. avons vu le m.
\item[\vref{Ps 91:10}] Aucun m. ne s'approchera de toi, aucun fléau
\item[\vref{Pr 1:26}] serez ds le m., je me moquerai
\item[\vref{Es 45:9}] M. à celui qui plaide contre celui
\item[\vref{Jé 17:16}] le jour du m., tu le sais ;
\item[\vref{Lu 6:26}] M. à vs. qnd ts les hommes
\item[\vref{Lu 17:1}] de scandales ; mais m. à celui par
\item[\vref{1 Co 9:16}] est imposée, et m. à moi si
\item[\vref{Ja 5:1}] à cause des m. qui vont tomber
\item[\vref{Jud 1:11}] M. à eux ! Car ils ont suivi
\item[\vref{Ap 9:12}] Le premier m. est passé, et
\item[\vref{Ap 12:12}] y habitez. Mais m. à vs. habitants
\end{listverse}

\ConcordanceEntry{Malheureux (le)}
\vspace{-2mm}
\begin{listverse}
\item[\vref{Ps 9:13}] se souvient des m., il n'oublie pas
\item[\vref{Ps 12:6}] l'on fait aux m., à cause du
\item[\vref{Ps 14:6}] le conseil du m. parce que Yahweh
\item[\vref{Ps 22:27}] Les m. mangeront et seront rassasiés, ceux qui
\item[\vref{Ps 72:4}] fasse droit aux m. du peuple, qu'il
\item[\vref{Ps 140:12}] fera justice au m. et droit aux
\item[\vref{Ps 147:6}] Yahweh soutient les m., mais il abaisse
\item[\vref{Pr 22:22}] n'opprime pas le m. à la porte.
\item[\vref{Es 61:1}] pour évangéliser les m. ; il m'a envoyé
\item[\vref{Ez 30:2}] le Seign. Yahweh : Hurlez, et dites : M. jour !
\item[\vref{Za 10:2}] brebis, ils sont m., parce qu'il n'y
\item[\vref{Ap 3:17}] que tu es m., misérable, pauvre, aveugle
\end{listverse}

\ConcordanceEntry{Malice}
\vspace{-2mm}
\begin{listverse}
\item[\vref{1 S 17:28}] orgueil et la m. de ton cœur,
\item[\vref{Est 8:3}] faire que la m. d'Haman, l'Agaguite, et
\item[\vref{Ps 5:10}] est rempli de m., lr. gosier est
\item[\vref{Ps 7:10}] Que la m. des méchants prenne fin, et affermis
\item[\vref{Ps 28:3}] pendant que la m. est ds lr.
\item[\vref{Ps 141:4}] méchantes actions par m., avec les hommes
\item[\vref{Pr 14:32}] chassé par sa m., mais le juste
\item[\vref{Pr 26:26}] La m. de celui qui la cache com.
\item[\vref{Jon 1:2}] elle ! Car lr. m. est montée jusqu'à
\item[\vref{Mt 22:18}] Jésus connaissant lr. m., dit : Hypocrites ! Pourquoi
\item[\vref{1 Co 5:8}] méchanceté et de m., mais avec les
\item[\vref{Ep 4:31}] médisance, et tte m. soient bannies du
\item[\vref{1 Pi 2:1}] tte sorte de m., de tte fraude,
\end{listverse}

\ConcordanceEntry{Malin}
\vspace{-2mm}
\begin{listverse}
\item[\vref{1 S 18:10}] lendemain que l'esprit m. envoyé de Dieu
\item[\vref{Job 2:7}] Job d'un ulcère m., depuis la plante
\item[\vref{Pr 15:26}] Les pensées du m. sont en abomination
\item[\vref{Pr 28:22}] qui a l'œil m. se hâte pour
\item[\vref{Jé 17:9}] rusé et désespérément m. par-dessus tt : Qui
\item[\vref{Mt 5:37}] qui est de plus vient du m.
\item[\vref{Mt 13:19}] comprend pas, le m. vient et ravit
\item[\vref{Ac 19:15}] Mais l'esprit m. lr. répondit : Je
\item[\vref{Ep 6:16}] éteindre ts les dards enflammés du m. ;
\item[\vref{1 Jn 2:13}] que vs. avez vaincu l'esprit du m.
\item[\vref{1 Jn 3:12}] était de l'esprit m. et qui tua
\item[\vref{1 Jn 5:18}] lui-mm, et le m. ne le touche
\item[\vref{Ap 16:2}] Et un ulcère m. et dangereux frappa
\end{listverse}

\ConcordanceEntry{Malte}
\vspace{-2mm}
\begin{listverse}
\item[\vref{Ac 28:1}] ils reconnurent alors que l'île s'appelait M.
\end{listverse}

\ConcordanceEntry{Maltraiter}
\vspace{-2mm}
\begin{listverse}
\item[\vref{Ge 16:6}] Saraï dc la m., et Agar s'enfuit
\item[\vref{Ex 5:22}] pourquoi as-tu fait m. ce peuple ? Pourquoi
\item[\vref{Ps 10:14}] afflige ou qu'on m. quelqu'un, tu regardes
\item[\vref{Jé 22:3}] de l'oppresseur ; ne m. pas l'orphelin, ni
\item[\vref{Mt 5:44}] ceux qui vs. m. et vs. persécutent,
\item[\vref{Ac 12:1}] se mit à m. quelques membres de
\item[\vref{Ac 19:16}] eux, et les m. de telle sorte
\item[\vref{Hé 13:3}] ceux qui sont m., com. étant aussi
\item[\vref{1 Pi 3:13}] Et qui vs. m., si vs. êtes
\end{listverse}

\ConcordanceEntry{Mammon}
\vspace{-2mm}
\begin{listverse}
\item[\vref{Mt 6:24}] ne pouvez pas servir Dieu et M.
\end{listverse}

\ConcordanceEntry{Mamré}
\vspace{-2mm}
\begin{listverse}
\item[\vref{Ge 13:18}] les plaines de M., qui sont près
\item[\vref{Ge 14:13}] les plaines de M., l'Amoréen, frère d'Eschcol,
\item[\vref{Ge 18:1}] les plaines de M., com. il était
\end{listverse}

\ConcordanceEntry{Manassé}
\vspace{-2mm}
\begin{listverse}
\item[\vref{Ge 41:51}] le nom de M., parce que, dit-il,
\item[\vref{No 1:35}] la tribu de M., qui furent dénombrés,
\item[\vref{Jos 13:29}] la demi-tribu de M. un héritage, qui
\item[\vref{Jg 6:15}] plus pauvre en M. et je suis
\item[\vref{2 R 20:21}] ses pères. Et M., son fils, régna
\item[\vref{2 R 21:16}] M. répandit aussi beaucoup de sang innocent,
\item[\vref{2 Ch 33:11}] d'Assyrie, qui mirent M. ds les fers ;
\item[\vref{Ap 7:6}] la tribu de M., douze mille marqués
\end{listverse}

\ConcordanceEntry{Mandragore}
\vspace{-2mm}
\begin{listverse}
\item[\vref{Ge 30:14}] blés, trouva des m. aux champs, et
\item[\vref{Ca 7:14}] Les m. répandent lr. parfum, et à nos
\end{listverse}

\ConcordanceEntry{Manger}
\vspace{-2mm}
\begin{listverse}
\item[\vref{Ge 2:9}] et bon à m., et l'arbre de
\item[\vref{Ge 2:17}] mal, tu n'en m. point, car le
\item[\vref{Ge 3:6}] était bon à m. et qu'il était
\item[\vref{Ex 24:11}] Dieu, et ils m. et burent.
\item[\vref{Lé 21:22}] Il pourra m. la nourriture de
\item[\vref{No 6:4}] naziréat il ne m. d'aucun fruit de
\item[\vref{Jg 14:14}] De celui qui m. est sorti ce
\item[\vref{2 R 4:43}] gens, et qu'ils m. ; car ainsi parle
\item[\vref{2 R 7:2}] de tes yeux, mais tu n'en m. pas.
\item[\vref{Pr 18:21}] aime à parler m. de ses fruits.
\item[\vref{Pr 31:27}] maison, et ne m. pas le pain
\item[\vref{Es 1:19}] obéissez volontairement, vs. m. le meilleur du
\item[\vref{Es 55:1}] venez, achetez et m. ; venez, dis-je, achetez
\item[\vref{Es 65:13}] Voici, mes serviteurs m., et vs. aurez
\item[\vref{Ez 3:2}] il me fit m. ce rouleau.
\item[\vref{Da 1:12}] des légumes à m. et de l'eau
\item[\vref{Os 4:10}] Et ils m. mais ils ne seront point rassasiés,
\item[\vref{Mt 6:31}] en disant : Que m.-ns. ? Ou : Que
\item[\vref{Mt 9:11}] Pourquoi votre Maître m.-t-il avec les
\item[\vref{Mt 13:4}] et les oiseaux vinrent, et la m. tte.
\item[\vref{Mt 14:16}] de s'en aller ; donnez-lr. vs.-mêmes à m.
\item[\vref{Mt 15:20}] l'hom. ; mais de m. sans avoir les
\item[\vref{Mt 24:38}] déluge, les hommes m. et buvaient, se
\item[\vref{Mt 25:35}] m'avez donné à m. ; j'ai eu soif,
\item[\vref{Mt 26:26}] Pendant qu'ils m., Jésus prit le
\item[\vref{Mc 6:31}] qu'ils n'avaient mm pas l'occasion de m.
\item[\vref{Mc 7:3}] les Juifs ne m. pas sans s'être
\item[\vref{Mc 7:28}] les petits chiens m. sous la table
\item[\vref{Lu 4:2}] Et il ne m. rien durant ces
\item[\vref{Lu 24:41}] dit : Avez-vs. ici qq chose à m. ?
\item[\vref{Jn 6:53}] Si vs. ne m. pas la chair
\item[\vref{Jn 21:5}] petit poisson à m. ? Ils lui répondirent :
\item[\vref{Ac 9:9}] sans voir, sans m. ni boire.
\item[\vref{Ac 10:10}] lui préparait à m., il tomba en
\item[\vref{Ro 14:6}] Et celui qui m. de ttes choses,
\item[\vref{1 Co 11:20}] n'est pas pour m. le repas du
\item[\vref{1 Co 15:32}] ne ressuscitent pas, m. et buvons, car
\item[\vref{2 Co 9:10}] du pain à m., multiplier votre semence,
\item[\vref{Col 2:16}] au sujet du m. ou du boire,
\item[\vref{2 Th 3:10}] travailler, qu'il ne m. pas non plus.
\item[\vref{1 Pi 4:3}] aux excès du m. et du boire,
\item[\vref{Ap 2:7}] lui donnerai à m. de l'arbre de
\item[\vref{Ap 2:17}] lui donnerai à m. de la manne
\end{listverse}

\ConcordanceEntry{Manifestation}
\vspace{-2mm}
\begin{listverse}
\item[\vref{Ro 2:5}] et de la m. du juste jugement
\item[\vref{1 Co 1:7}] vs. attendez la m. de notre Seign.
\item[\vref{1 Co 12:7}] est donnée la m. de l'Esprit pour
\item[\vref{2 Co 4:2}] Dieu, par la m. de la vérité.
\end{listverse}

\ConcordanceEntry{Manne}
\vspace{-2mm}
\begin{listverse}
\item[\vref{Ex 16:31}] nomma ce pain m.. Elle était com.
\item[\vref{Ex 16:35}] d'Israël mangèrent la m. durant quarante ans,
\item[\vref{No 11:7}] Or la m. était com. la graine de coriandre,
\item[\vref{De 8:3}] nourri de la m., que tu ne
\item[\vref{Jos 5:12}] Et la m. cessa dès le lendemain de la
\item[\vref{Né 9:20}] retiras point ta m. de lr. bouche,
\item[\vref{Ps 78:24}] fit pleuvoir la m. sur eux pour
\item[\vref{Jn 6:31}] ont mangé la m. ds le désert,
\item[\vref{Hé 9:4}] où était la m., et la verge
\item[\vref{Ap 2:17}] manger de la m. qui est cachée,
\end{listverse}

\ConcordanceEntry{Manoach}
\vspace{-2mm}
\begin{listverse}
\item[\vref{Jg 13:2}] le nom était M.. Sa fem. était
\end{listverse}

\ConcordanceEntry{Manquer}
\vspace{-2mm}
\begin{listverse}
\item[\vref{Ge 18:28}] Peut-être en m.-t-il cinq des
\item[\vref{Ge 50:24}] Mais Dieu ne m. pas de vs.
\item[\vref{Lé 2:13}] offrande de gâteau m. de sel, signe
\item[\vref{De 2:7}] ces quarante années, et tu n'as m. de rien.
\item[\vref{Ru 4:14}] t'a pas laissé m. aujourd'hui d'un hom.,
\item[\vref{Né 9:21}] rien ne lr. m.. Leurs vêtements ne
\item[\vref{Job 31:19}] vêtements, le pauvre m. de couverture,
\item[\vref{Ps 23:1}] Yahweh est mon berger: Je ne m. de rien.
\item[\vref{Ps 34:10}] Car rien ne m. à ceux qui
\item[\vref{Ps 34:11}] cherchent Yahweh ne m. d'aucun bien.
\item[\vref{Ps 46:2}] secours qui ne m. jamais ds les
\item[\vref{Pr 10:19}] le péché ne m. pas, mais celui
\item[\vref{Pr 22:16}] au riche, ne m. pas de tomber
\item[\vref{Pr 28:16}] Le conducteur qui m. d'intelligence fait beaucoup
\item[\vref{Ec 1:15}] et ce qui m. ne peut être
\item[\vref{Jé 35:19}] de Récab, ne m. jamais de descendants
\item[\vref{Ez 4:17}] et l'eau lr. m., ils seront épouvantés,
\item[\vref{Lu 16:9}] vs. viendrez à m., ils vs. reçoivent
\item[\vref{Lu 18:22}] dit : Il te m. encore une chose :
\item[\vref{Lu 22:35}] sans souliers, avez-vs. m. de qq chose ?
\item[\vref{Jn 2:3}] le vin ayant m., la mère de
\item[\vref{1 Co 1:7}] qu'il ne vs. m. aucun don, pendant
\item[\vref{1 Co 7:5}] tente par votre m. de contrôle.
\item[\vref{1 Co 12:24}] plus d'honneur à ce qui en m.,
\item[\vref{1 Th 3:10}] compléter ce qui m. à votre foi.
\item[\vref{Hé 11:32}] le temps me m. si je voulais
\item[\vref{Ja 1:5}] quelqu'un de vs. m. de sagesse, qu'il
\item[\vref{Ja 2:15}] sont nus et m. de ce qui
\end{listverse}

\ConcordanceEntry{Manteau}
\vspace{-2mm}
\begin{listverse}
\item[\vref{Ge 25:25}] velu, com. un m. de poils ; et
\item[\vref{Jos 7:21}] butin un beau m. de Schinear, deux
\item[\vref{1 S 15:27}] pan de son m. qui se déchira.
\item[\vref{1 S 24:5}] coupa tt doucement le pan du m. de Saül.
\item[\vref{1 R 19:13}] visage de son m., il sortit et
\item[\vref{1 R 19:19}] lui, il jeta sur lui son m.
\item[\vref{Es 59:17}] couvre de la jalousie com. d'un m.
\item[\vref{Es 61:3}] du deuil, un m. de louange au
\item[\vref{Es 61:10}] m'a couvert du m. de la justice,
\item[\vref{Mt 5:40}] prendre ta tunique, laisse-lui encore ton m.
\item[\vref{Mt 27:28}] le revêtirent d'un m. d'écarlate.
\item[\vref{Ac 12:8}] Enveloppe-toi de ton m. et suis-moi.
\end{listverse}

\ConcordanceEntry{Mara}
\vspace{-2mm}
\begin{listverse}
\item[\vref{Ex 15:23}] ils vinrent à M., mais ils ne
\item[\vref{No 33:8}] le désert d'Etham et campèrent à M.
\item[\vref{Ru 1:20}] pas Naomi, appelez-moi M., car le Tout-Puissant
\end{listverse}

\ConcordanceEntry{Maranatha}
\vspace{-2mm}
\begin{listverse}
\item[\vref{1 Co 16:22}] le Seign. Jésus-Christ, qu'il soit anathème ! M. !
\end{listverse}

\ConcordanceEntry{Marc}
\vspace{-2mm}
\begin{listverse}
\item[\vref{Ac 12:12}] de Jean, surnommé M., où plusieurs personnes
\item[\vref{Ac 15:39}] l'autre. Barnabas, prenant M. avec lui, s'embarqua
\item[\vref{Col 4:10}] salue aussi, et M. qui est le
\item[\vref{2 Ti 4:11}] avec moi ; prends M. et amène-le avec
\item[\vref{Phm 1:24}] M. aussi, Aristarque, Démas, et Luc, mes
\item[\vref{1 Pi 5:13}] com. vs. et M., mon fils vs.
\end{listverse}

\ConcordanceEntry{Marchand}
\vspace{-2mm}
\begin{listverse}
\item[\vref{Ge 37:28}] Et com. les m. Madianites passaient, ils
\item[\vref{Pr 31:14}] les navires d'un m., elle amène sa
\item[\vref{Za 14:21}] aura plus de m. ds la maison
\item[\vref{Mt 13:45}] semblable à un m. qui cherche de
\item[\vref{Ap 18:3}] elle, et les m. de la terre
\end{listverse}

\ConcordanceEntry{Marche}
\vspace{-2mm}
\begin{listverse}
\item[\vref{Ex 5:3}] trois journées de m. ds le désert,
\item[\vref{De 2:7}] a connu ta m. ds ce grand
\item[\vref{1 R 19:4}] une journée de m., il s'assit sous
\item[\vref{2 R 3:9}] et après une m. de sept jours,
\item[\vref{Ps 68:25}] Ils voient ta m., ô Dieu ! Ils
\item[\vref{Ps 110:7}] torrent pendant la m. : C'est pourquoi il
\item[\vref{Jon 3:3}] dvt Dieu, de trois jours de m.
\end{listverse}

\ConcordanceEntry{Marchepied}
\vspace{-2mm}
\begin{listverse}
\item[\vref{1 Ch 28:2}] et pour le m. de notre Dieu,
\item[\vref{Ps 99:5}] prosternez-vs. dvt son m. ! Il est saint !
\item[\vref{Ps 110:1}] tes ennemis le m. de tes pieds.
\item[\vref{Es 66:1}] terre est le m. de mes pieds.
\item[\vref{Mt 5:35}] que c'est le m. de ses pieds ;
\item[\vref{Mt 22:44}] ennemis pour le m. de tes pieds ?
\item[\vref{Mc 12:36}] ennemis pour le m. de tes pieds.
\item[\vref{Ac 7:49}] terre est le m. de mes pieds :
\item[\vref{Hé 1:13}] ennemis pour le m. de tes pieds ?
\item[\vref{Hé 10:13}] mis pour le m. de ses pieds.
\item[\vref{Ja 2:3}] debout ! Ou : Assieds-toi ici sur mon m. !
\end{listverse}

\ConcordanceEntry{Marcher}
\vspace{-2mm}
\begin{listverse}
\item[\vref{Ge 3:14}] des champs ; tu m. sur ton ventre,
\item[\vref{Ge 5:22}] eut engendré Metuschélah, m. avec Dieu trois
\item[\vref{Ge 17:1}] le Dieu Tout-Puissant. M. dvt ma face,
\item[\vref{Ex 14:15}] Parle aux enfants d'Israël, et qu'ils m.
\item[\vref{Lé 26:12}] Mais je m. au milieu de vs., je serai
\item[\vref{De 8:6}] ton Dieu, pour m. ds ses voies,
\item[\vref{Ps 15:2}] sera celui qui m. ds l'intégrité, qui
\item[\vref{Ps 23:4}] Même qnd je m. ds la vallée
\item[\vref{Ps 91:13}] Tu m. sur le lion et sur l'aspic,
\item[\vref{Ps 119:35}] Fais-moi m. sur le sentier de tes commandements
\item[\vref{Ps 139:3}] sais qnd je m. et qnd je
\item[\vref{Pr 28:26}] mais celui qui m. sagement sera délivré.
\item[\vref{Ec 12:1}] ta jeunesse, et m. com. ton cœur
\item[\vref{Es 6:8}] enverrai-je et qui m. pour ns. ? Je
\item[\vref{Es 40:31}] fatiguent point ; ils m., et ne se
\item[\vref{Es 59:8}] ceux qui y m. ignorent la paix.
\item[\vref{Ez 36:12}] Je ferai m. sur vs. des hommes, mon peuple
\item[\vref{Da 3:25}] sans liens qui m. au milieu du
\item[\vref{Da 9:10}] notre Dieu, pour m. ds ses lois,
\item[\vref{Am 3:3}] Deux hommes m.-ils ensemble s'ils
\item[\vref{Mi 4:5}] ts les peuples m. chacun au nom
\item[\vref{Mt 14:25}] alla vers eux, m. sur la mer.
\item[\vref{Mt 14:29}] de la barque, m. sur les eaux
\item[\vref{Lu 10:19}] le pouvoir de m. sur les serpents
\item[\vref{Lu 13:33}] il me faut m. aujourd'hui et demain,
\item[\vref{Ac 14:16}] ttes les nations m. ds leurs voies,
\item[\vref{Ro 6:4}] ns. aussi ns. m. en nouveauté de
\item[\vref{Ro 13:13}] M. honnêtement, com. en plein jour, loin
\item[\vref{Ep 4:1}] le Seign., à m. d'une manière digne
\item[\vref{Ep 5:2}] et m. ds la charité, ainsi que Christ
\item[\vref{1 Jn 1:7}] Mais si ns. m. ds la lumière,
\item[\vref{3 Jn 1:3}] et comment tu m. ds la vérité.
\item[\vref{Jud 1:16}] plaignent toujours, qui m. selon leurs convoitises,
\item[\vref{Ap 9:20}] peuvent ni voir, ni entendre, ni m.
\end{listverse}

\ConcordanceEntry{Mardochée}
\vspace{-2mm}
\begin{listverse}
\item[\vref{Est 2:5}] hom. Juif nommé M., fils de Jaïr,
\item[\vref{Est 8:2}] le donna à M. ; Esther établit Mardochée
\end{listverse}

\ConcordanceEntry{Mari}
\vspace{-2mm}
\begin{listverse}
\item[\vref{Ge 3:6}] aussi à son m. qui était auprès
\item[\vref{Ge 29:32}] et mntnt mon m. m'aimera.
\item[\vref{Pr 12:4}] couronne de son m., mais celle qui
\item[\vref{Pr 31:11}] cœur de son m. a entièrement confiance
\item[\vref{Pr 31:23}] [Nun.] Son m. est reconnu aux
\item[\vref{Os 2:18}] tu m'appelleras: Mon M. ! Et tu ne
\item[\vref{Jn 4:16}] Va, appelle ton m., et viens ici.
\item[\vref{Ro 7:2}] soumise à un m., est liée à
\item[\vref{1 Co 7:2}] et que chaque fem. ait son m.
\item[\vref{1 Co 7:34}] monde, comment elle plaira à son m.
\item[\vref{Ep 5:23}] car le m. est le chef de la fem.,
\item[\vref{Ep 5:25}] Et vs. m., aimez vos femmes, com. Christ a
\item[\vref{Ep 5:33}] et que la fem. respecte son m.
\item[\vref{Col 3:19}] M., aimez vos femmes, et ne vs.
\item[\vref{1 Ti 3:2}] l'évêque soit irrépréhensible, m. d'une seule fem.,
\item[\vref{1 Ti 3:12}] diacres doivent être m. d'une seule fem.,
\item[\vref{Tit 1:6}] qui soit irrépréhensible, m. d'une seule fem.,
\item[\vref{Tit 2:4}] à aimer leurs m., à aimer leurs
\item[\vref{1 Pi 3:1}] soumises à vos m., afin que si
\item[\vref{1 Pi 3:7}] Vous de mm, m., montrez de la
\item[\vref{Ap 21:2}] épouse qui s'est ornée pour son m.
\end{listverse}

\ConcordanceEntry{Mariage}
\vspace{-2mm}
\begin{listverse}
\item[\vref{De 7:3}] t'allieras point par m. avec elles, tu
\item[\vref{Hé 13:4}] Le m. est honorable entre ts, et le
\end{listverse}

\ConcordanceEntry{Marie}
\vspace{-2mm}
\begin{listverse}
\item[\vref{Ex 15:20}] Et M., la prophétesse, sœur d'Aaron, prit un
\item[\vref{Ac 1:14}] les femmes, avec M., mère de Jésus,
\end{listverse}
\begin{legend}
\NoAutoSpaceBeforeFDP{
\item Sœur de Moïse et d'Aaron : No 26:59; Ex 2:4-10
\item Une prophétesse : Ex 15:20
\item M. et Aaron murmure contre Moïse : No 12:1-15; 20:1
\item M. mère de Jésus : Mt 1:16; Lu 1:26-38; Ac 1:14
\item Sa foi : Lu 1:38; Jn 2:5
\item La famille spirituelle : Mt 12:46; Mc 3:21
\item M. et les disciples : Ac 1:14
\item M. de Magdala : Mt 27:56
\item M. de Magdala délivrée de sept démons : Lu 8:2
\item Jésus apparaît à M de Magdala : Mc 16:9; Jn 20:11-18
\item M. mère de Jacques le mineur : Mc 15:40; Jn 19:25
\item M. sœur de Marthe : Jn 11:1
\item Attitude de M. et Marthe : Lu 10:42
\item M. de Magdala oint les pieds de Jésus : Mc 14:3-9; Jn 11:2; 12:3
\item M. mère de Jean, surnommé Marc : Ac 12:12
}
\end{legend}

\ConcordanceEntry{Marier}
\vspace{-2mm}
\begin{listverse}
\item[\vref{Mt 19:10}] il ne convient pas de se m.
\item[\vref{Mt 22:25}] Le premier se m., et mourut ; et,
\item[\vref{Mt 24:38}] et buvaient, se m., et donnaient en
\item[\vref{1 Co 7:9}] maîtrise, qu'ils se m. ; car il vaut
\item[\vref{1 Co 7:36}] qu'il faille la m., qu'il fasse ce
\item[\vref{1 Ti 4:3}] défendant de se m. et ordonnant de
\item[\vref{1 Ti 5:11}] lascives contre Christ, elles veulent se m.,
\item[\vref{1 Ti 5:14}] jeunes veuves se m., qu'elles aient des
\end{listverse}

\ConcordanceEntry{Marteau}
\vspace{-2mm}
\begin{listverse}
\item[\vref{Jé 23:29}] et com. un m. qui brise le
\end{listverse}

\ConcordanceEntry{Marthe}
\vspace{-2mm}
\begin{listverse}
\item[\vref{Lu 10:38}] une fem., nommée M., le reçut ds
\item[\vref{Jn 11:1}] Marie et de M., sa sœur.
\end{listverse}

\ConcordanceEntry{Massa, Meriba}
\vspace{-2mm}
\begin{listverse}
\item[\vref{Ex 17:7}] nomma le lieu M. et Meriba, à
\item[\vref{De 6:16}] Dieu, com. vs. l'avez tenté à M.
\item[\vref{De 9:22}] à Tabeéra, à M., et à Kibroth-Hattaava.
\item[\vref{De 33:8}] as éprouvé à M., et avec qui
\item[\vref{Ps 81:8}] t'ai éprouvé auprès des eaux de M.. Sélah.
\item[\vref{Ps 95:8}] cœur, com. à M., com. à la
\item[\vref{Ps 106:32}] des eaux de M., et Moïse fut
\end{listverse}

\ConcordanceEntry{Masse}
\vspace{-2mm}
\begin{listverse}
\item[\vref{2 R 20:7}] dit : Prenez une m. de figues sèches.
\item[\vref{Es 38:21}] Qu'on prenne une m. de figues sèches
\item[\vref{Za 5:7}] on portait une m. de plomb, et
\item[\vref{Ro 9:21}] avec la mm m. de terre un
\item[\vref{Ro 11:16}] sont saintes, la m. l'est aussi ; et
\end{listverse}

\ConcordanceEntry{Matin}
\vspace{-2mm}
\begin{listverse}
\item[\vref{Ge 1:5}] ainsi fut le m. ; ce fut le
\item[\vref{Ps 5:4}] Yahweh, le m. tu entends ma
\item[\vref{Ps 92:3}] Afin d'annoncer chaque m. ta bonté, et
\item[\vref{Ps 127:2}] levez de grand m., que vs. vs.
\item[\vref{Ca 7:13}] Levons-ns. dès le m. pour aller aux
\item[\vref{Es 21:12}] sentinelle répond : Le m. vient et la
\item[\vref{Jé 11:7}] avertis dès le m., en disant : Ecoutez
\item[\vref{La 3:23}] se renouvellent chaque m.. C'est une chose
\item[\vref{Mc 16:9}] étant ressuscité, le m. du premier jour
\item[\vref{2 Pi 1:19}] que l'Etoile du m. se lève ds
\item[\vref{Ap 22:16}] postérité de David, l'étoile brillante du m.
\end{listverse}

\ConcordanceEntry{Matthias}
\vspace{-2mm}
\begin{listverse}
\item[\vref{Ac 1:26}] sort tomba sur M., qui, d'une commune
\end{listverse}

\ConcordanceEntry{Matthieu}
\vspace{-2mm}
\begin{listverse}
\item[\vref{Mt 9:9}] un hom. nommé M., assis au bureau
\end{listverse}

\ConcordanceEntry{Maudire}
\vspace{-2mm}
\begin{listverse}
\item[\vref{Ge 3:14}] cela, tu seras m. entre tt le
\item[\vref{Ge 8:21}] cœur : Je ne m. plus la terre
\item[\vref{Ge 12:3}] béniront, et je m. ceux qui te
\item[\vref{Ex 21:17}] Celui qui aura m. son père ou
\item[\vref{No 23:8}] Mais comment le m.-je ? Dieu ne
\item[\vref{Jos 6:26}] jura, en disant : M. soit dvt Yahweh
\item[\vref{Ps 62:5}] de lr. bouche, mais au-dedans ils m.. Sélah.
\item[\vref{Ps 109:28}] Ils m., mais tu béniras ; ils s'élèveront, mais
\item[\vref{Ec 10:20}] Ne m. point le roi, mm ds ta
\item[\vref{Es 8:21}] faim, il s'irritera, m. son roi et
\item[\vref{Jé 15:10}] néanmoins ts me m. et me méprisent.
\item[\vref{Jé 17:5}] Ainsi parle Yahweh : M. soit l'hom. qui
\item[\vref{Mt 5:44}] ceux qui vs. m., faites du bien
\item[\vref{Mc 14:71}] mit à se m., et à jurer,
\item[\vref{Ga 3:13}] il est écrit : M. est quiconque est
\item[\vref{Ja 3:9}] par elle ns. m. les hommes faits
\end{listverse}

\ConcordanceEntry{Mauvais}
\vspace{-2mm}
\begin{listverse}
\item[\vref{Ge 40:7}] dit : Pourquoi avez-vs. m. visage aujourd'hui ?
\item[\vref{1 S 16:15}] dirent : Voici, un m. esprit envoyé de
\item[\vref{1 S 23:9}] eu connaissance des m. desseins de Saül
\item[\vref{2 R 4:41}] plus rien de m. ds le pot.
\item[\vref{Né 2:1}] n'avais jamais eu m. visage dvt lui.
\item[\vref{Job 30:25}] qui passait de m. jours ; et mon
\item[\vref{Ps 12:6}] A cause du m. traitement que l'on
\item[\vref{Ps 21:12}] ont conçu de m. desseins dont ils
\item[\vref{Ps 37:7}] bout de ses m. desseins.
\item[\vref{Ps 52:9}] qui mettait sa force ds ses m. désirs.
\item[\vref{Ps 140:11}] l'hom. violent et m., qu'on le chasse
\item[\vref{Pr 2:12}] te délivrer du m. chemin, et de
\item[\vref{Pr 12:12}] filet des hommes m., mais la racine
\item[\vref{Pr 15:15}] de l'affligé sont m., mais qnd on
\item[\vref{Pr 17:4}] L'hom. m. est attentif à la lèvre trompeuse,
\item[\vref{Pr 20:14}] Il est m., il est mauvais, dit l'acheteur ; puis
\item[\vref{Pr 26:23}] et le cœur m. sont com. un
\item[\vref{Pr 28:10}] droits ds le m. chemin tombe ds
\item[\vref{Ec 12:3}] que les jours m. arrivent et que
\item[\vref{Am 5:13}] ce temps-ci, car les temps sont m.
\item[\vref{Mt 7:17}] fruits, mais le m. arbre porte de
\item[\vref{Lu 6:45}] méchant tire de m. choses du mauvais
\item[\vref{1 Co 15:33}] pas séduits : Les m. compagnies corrompent les
\item[\vref{Ga 1:4}] du présent siècle m., selon la volonté
\item[\vref{Ep 5:16}] le temps, car les jours sont m.
\item[\vref{Ep 6:13}] résister ds le m. jour, et tenir
\item[\vref{Ph 3:2}] prenez garde aux m. ouvriers, prenez garde
\item[\vref{Col 3:5}] les passions, les m. désirs, et la
\item[\vref{1 Ti 6:4}] médisances et les m. soupçons,
\item[\vref{2 Ti 2:24}] propre à enseigner, supportant patiemment les m.,
\item[\vref{Hé 3:12}] n'ait un cœur m. et incrédule, au
\end{listverse}

\ConcordanceEntry{Méchanceté}
\vspace{-2mm}
\begin{listverse}
\item[\vref{Ge 6:5}] vit que la m. des hommes était
\item[\vref{De 9:4}] cause de la m. de ces nations-là
\item[\vref{Jg 20:12}] pour dire : Quelle m. a été faite
\item[\vref{1 R 2:44}] fait retomber ta m. sur ta tête.
\item[\vref{Job 35:8}] toi que ta m. peut nuire, et
\item[\vref{Ps 36:4}] ne sont que m. et tromperie, il
\item[\vref{Ps 45:8}] tu hais la m. : C'est pourquoi, ô
\item[\vref{Ps 125:3}] verge de la m. ne restera pas
\item[\vref{Pr 4:17}] le pain de m., et qu'ils boivent
\item[\vref{Pr 10:2}] Les trésors de m. ne profitent pas,
\item[\vref{Pr 12:3}] affermi par la m., mais la racine
\item[\vref{Ec 3:16}] a de la m. ; et qu'au lieu
\item[\vref{Ec 8:8}] combat, et la m. ne délivrera point
\item[\vref{Es 1:13}] plus supporter votre m. ni vos assemblées
\item[\vref{Es 58:6}] liens de la m., que tu délies
\item[\vref{Jé 2:19}] Ta m. te châtiera, et tes débauches te
\item[\vref{La 1:22}] Que tte lr. m. vienne dvt toi,
\item[\vref{Ez 33:19}] détournera de sa m., et qu'il fera
\item[\vref{Os 9:15}] Toute lr. m. s'est manifestée à
\item[\vref{Joë 3:13}] regorgent ! Car lr. m. est grande,
\item[\vref{Na 3:19}] continuellement éprouvé les effets de ta m. ?
\item[\vref{Ha 1:3}] fais-tu voir la m., et vois-tu la
\item[\vref{Mal 3:15}] qui commettent la m., sont établis ; mm
\item[\vref{Lu 11:39}] êtes pleins de rapine et de m.
\item[\vref{Ac 8:22}] dc de cette m., et prie Dieu,
\item[\vref{Ro 1:29}] d'injustice, d'impureté, de m., d'avarice, de malignité,
\item[\vref{1 Co 5:8}] un levain de m. et de malice,
\item[\vref{1 Co 14:20}] l'égard de la m., et à l'égard
\item[\vref{Tit 3:3}] vivant ds la m. et ds l'envie,
\item[\vref{Ja 1:21}] tt résidu de m., recevez avec douceur
\item[\vref{1 Pi 2:16}] qui couvre la m., mais agissant com.
\end{listverse}

\ConcordanceEntry{Méchant (le)}
\vspace{-2mm}
\begin{listverse}
\item[\vref{Ge 18:23}] Feras-tu périr le juste avec le m. ?
\item[\vref{Ex 23:1}] joindras point au m. pour être un
\item[\vref{De 13:5}] Tu ôteras le m. du milieu de
\item[\vref{Job 36:6}] pas vivre le m., et il fait
\item[\vref{Job 36:17}] le jugement du m., mais le jugement
\item[\vref{Ps 5:5}] au mal ; le m. n'a point sa
\item[\vref{Ps 7:13}] Si le m. ne se convertit pas, Dieu aiguise
\item[\vref{Ps 10:3}] Car le m. se glorifie du désir de son
\item[\vref{Ps 32:10}] douleurs atteindront le m., mais la bonté
\item[\vref{Ps 37:12}] [Zayin.] Le m. complote contre le
\item[\vref{Ps 50:16}] Dieu dit au m. : Quoi dc ? Tu
\item[\vref{Ps 140:4}] la main du m. ! Préserve-moi de l'hom.
\item[\vref{Pr 3:33}] la maison du m. ; mais il bénit
\item[\vref{Pr 10:16}] le revenu du m. est pour le
\item[\vref{Pr 10:24}] que redoute le m., c'est ce qui
\item[\vref{Pr 11:5}] voie, mais le m. tombe par sa
\item[\vref{Pr 16:4}] et mm le m. pour le jour
\item[\vref{Pr 21:10}] L'âme du m. souhaite le mal, et son prochain
\item[\vref{Pr 29:2}] mais qnd le m. domine, le peuple
\item[\vref{Ec 3:17}] juste et le m. ; car il y
\item[\vref{Es 26:10}] fait grâce au m. ? Il n'en apprend
\item[\vref{Es 55:7}] Que le m. abandonne sa voie, et l'hom. injuste
\item[\vref{Ez 3:18}] je dirai au m. : Tu mourras, tu
\item[\vref{Ez 18:23}] la mort du m., dit le Seign.
\item[\vref{Ez 33:12}] son péché, le m. ne tombera point
\item[\vref{Mal 3:18}] juste et le m., entre celui qui
\item[\vref{Mt 5:39}] résistez pas au m.. Si quelqu'un te
\item[\vref{Ro 4:5}] qui justifie le m., sa foi lui
\item[\vref{1 Co 5:13}] Ôtez dc le m. du milieu de
\item[\vref{2 Th 2:8}] sera révélé le m., que le Seign.
\end{listverse}

\ConcordanceEntry{Méchant, Méchante}
\vspace{-2mm}
\begin{listverse}
\item[\vref{Ge 38:7}] de Juda, était m. dvt Yahweh, et
\item[\vref{No 14:35}] je traiterai cette m. assemblée, qui s'est
\item[\vref{De 1:35}] hommes de cette m. génération ne verra
\item[\vref{De 15:9}] ton œil soit m. envers ton frère
\item[\vref{Pr 16:27}] L'hom. m. creuse le mal, et il y
\item[\vref{Ec 7:17}] Ne sois point m. à l'excès, et
\item[\vref{Ec 10:13}] fin de son discours est une m. folie.
\item[\vref{Jé 13:10}] ce peuple très m., qui refuse d'écouter
\item[\vref{Mt 12:35}] cœur ; et l'hom. m. tire de mauvaises
\item[\vref{Mt 12:39}] dit : Une génération m. et adultère demande
\item[\vref{Mt 18:32}] et lui dit : M. serviteur, je t'avais
\item[\vref{Mt 25:26}] répondant, lui dit : M. et lâche serviteur,
\item[\vref{Lu 19:22}] il lui dit : M. serviteur, je te
\end{listverse}

\ConcordanceEntry{Médecin}
\vspace{-2mm}
\begin{listverse}
\item[\vref{Jé 8:22}] pas là de m. ? Pourquoi dc la
\item[\vref{Mt 9:12}] ont besoin de m., mais les malades.
\item[\vref{Mc 2:17}] ont besoin de m., mais les malades.
\item[\vref{Lu 4:23}] direz ce proverbe : M., guéris-toi toi-mm. Et
\item[\vref{Lu 5:31}] pas besoin de m., mais ceux qui
\item[\vref{Col 4:14}] Luc, le m. bien-aimé, vs. salue, ainsi que Démas.
\end{listverse}

\ConcordanceEntry{Mèdes}
\vspace{-2mm}
\begin{listverse}
\item[\vref{Est 1:3}] Perses et des M., aux nobles et
\item[\vref{Es 13:17}] contre eux les M., qui ne font
\item[\vref{Da 5:28}] et donné aux M. et aux Perses.
\item[\vref{Da 8:20}] les rois des M. et des Perses ;
\item[\vref{Ac 2:9}] Parthes, M., Elamites, et ceux qui habitent la
\end{listverse}

\ConcordanceEntry{Médiateur}
\vspace{-2mm}
\begin{listverse}
\item[\vref{Ga 3:19}] par des anges au moyen d'un m.
\item[\vref{1 Ti 2:5}] et aussi le M. entre Dieu et
\item[\vref{Hé 8:6}] qu'il est le M. d'une alliance plus
\item[\vref{Hé 9:15}] il est le M. de la Nouvelle
\item[\vref{Hé 12:24}] qui est le M. de la Nouvelle
\end{listverse}

\ConcordanceEntry{Méditer}
\vspace{-2mm}
\begin{listverse}
\item[\vref{Ge 50:20}] Vous aviez m. de me faire
\item[\vref{Jos 1:8}] ta bouche, mais m.-le jour et
\item[\vref{Ps 1:2}] Yahweh, et qui m. sa loi jour
\item[\vref{Ps 119:148}] la nuit pour m. ta parole.
\item[\vref{Ps 143:5}] jours anciens, je m. sur ttes tes
\item[\vref{Pr 3:29}] Ne m. pas le mal contre ton prochain,
\item[\vref{Pr 14:22}] Ceux qui m. le mal ne s'égarent-ils pas ? Mais
\item[\vref{Pr 15:28}] cœur du juste m. ce qu'il doit
\item[\vref{Pr 16:30}] des yeux pour m. des choses perverses,
\item[\vref{Mi 2:3}] Yahweh : Voici, je m. contre cette famille-ci
\item[\vref{Na 1:11}] sorti celui qui m. du mal contre
\item[\vref{Mc 13:11}] et ne le m. pas ; mais dites
\end{listverse}

\ConcordanceEntry{Meguiddo}
\vspace{-2mm}
\begin{listverse}
\item[\vref{Jos 17:11}] les habitants de M. et les villes
\item[\vref{Jg 1:27}] les habitants de M. et des villes
\item[\vref{Jg 5:19}] des eaux de M. ; mais ils ne
\item[\vref{2 R 23:29}] le vit, il le tua à M.
\end{listverse}

\ConcordanceEntry{Meilleur}
\vspace{-2mm}
\begin{listverse}
\item[\vref{Ge 45:18}] vs. donnerai le m. du pays d'Egypte ;
\item[\vref{1 S 15:9}] épargnèrent Agag, les m. brebis, les meilleurs
\item[\vref{1 S 15:28}] donne à un autre, qui est m. que toi.
\item[\vref{1 R 19:4}] ne suis pas m. que mes pères.
\item[\vref{Ps 81:17}] le nourrirait du m. froment, et je
\item[\vref{Ps 147:14}] contrées, et qui te rassasie du m. froment.
\item[\vref{Pr 3:14}] faire d'elle est m. que le trafic
\item[\vref{Pr 8:19}] Mon fruit est m. que l'or fin,
\item[\vref{Ec 3:22}] a rien de m. pour l'hom. que
\item[\vref{Ec 4:9}] ils ont un m. salaire de lr.
\item[\vref{Ec 11:6}] lequel sera le m., ceci ou cela ;
\item[\vref{Es 1:19}] vs. obéissez volontairement, vs. mangerez le m. du pays.
\item[\vref{Da 1:15}] visages parurent en m. état et avaient
\item[\vref{Mi 7:4}] Le m. d'entre eux est com. une ronce,
\item[\vref{Lu 5:39}] car il dit : Le vieux est m.
\item[\vref{Ph 1:23}] Christ, ce qui me serait beaucoup m. ;
\item[\vref{Hé 11:40}] qq chose de m. pour ns., afin
\end{listverse}

\ConcordanceEntry{Melchisédek}
\vspace{-2mm}
\begin{listverse}
\item[\vref{Ge 14:18}] M., roi de Salem, fit apporter du
\item[\vref{Ps 110:4}] prêtre éternellement, à la manière de M.
\item[\vref{Hé 5:6}] es prêtre éternellement, selon l'ordre de M.
\item[\vref{Hé 7:1}] En effet, ce M. était Roi de
\end{listverse}

\ConcordanceEntry{Membre}
\vspace{-2mm}
\begin{listverse}
\item[\vref{Ro 6:13}] livrez pas vos m. au péché pour
\item[\vref{Ro 12:5}] ns. sommes ts m. les uns des
\item[\vref{1 Co 6:15}] corps sont les m. de Christ ? Prendrai-je
\item[\vref{1 Co 12:12}] cependant a plusieurs m., et com. ts
\item[\vref{Ep 4:16}] distribue à chaque m., afin qu'il soit
\item[\vref{Col 3:5}] dc mourir vos m. qui sont sur
\item[\vref{Ja 3:5}] c'est un petit m., et cependant elle
\end{listverse}

\ConcordanceEntry{Même}
\vspace{-2mm}
\begin{listverse}
\item[\vref{Ps 102:28}] es toujours le m., et tes années
\item[\vref{Hé 13:8}] Jésus-Christ est le m. hier, aujourd'hui, et
\end{listverse}

\ConcordanceEntry{Mémoire}
\vspace{-2mm}
\begin{listverse}
\item[\vref{De 25:19}] tu effaceras la m. d'Amalek de dessous
\item[\vref{Ps 97:12}] et célébrez la m. de sa sainteté !
\item[\vref{Ps 102:13}] éternellement, et ta m. est de génération
\item[\vref{Ps 112:6}] jamais. [Lamed.] La m. du juste dure
\item[\vref{Pr 10:7}] La m. du juste est en bénédiction, mais
\item[\vref{Ec 9:5}] rien ; car lr. m. est mise en
\item[\vref{Es 43:26}] Réveille ma m., et plaidons ensemble ;
\item[\vref{Mt 26:13}] racontera aussi en m. de cette fem.
\item[\vref{Mc 14:9}] racontera aussi en m. de cette fem.
\item[\vref{Lu 22:19}] donné pour vs. ; faites ceci en m. de moi.
\item[\vref{1 Co 11:24}] rompu pour vs.. Faites ceci en m. de moi.
\item[\vref{1 Co 11:25}] fois que vs. en boirez, en m. de moi.
\item[\vref{2 Ti 2:14}] ces choses en m., protestant dvt Dieu
\end{listverse}

\ConcordanceEntry{Menace}
\vspace{-2mm}
\begin{listverse}
\item[\vref{2 S 22:16}] découvert, par la m. de Yahweh, par
\item[\vref{Job 26:11}] ciel s'ébranlent et s'étonnent à sa m.
\item[\vref{Ps 18:16}] découverts, par ta m., ô Yahweh ! par
\item[\vref{Ps 104:7}] s'enfuirent à ta m., et se mirent
\item[\vref{Es 50:2}] Voici, par ma m., je dessèche la
\item[\vref{Es 66:15}] fureur, et sa m. en flamme de
\item[\vref{Ac 4:21}] les relâchèrent avec m., ne trouvant pas
\item[\vref{Ac 9:1}] respirant encore la m. et le carnage
\item[\vref{1 Pi 2:23}] n'usait pas de m., mais il se
\end{listverse}

\ConcordanceEntry{Menacer}
\vspace{-2mm}
\begin{listverse}
\item[\vref{Ps 76:7}] sont endormis qnd tu les as m.
\item[\vref{Ps 106:9}] Car il m. la Mer Rouge et elle se
\item[\vref{Es 54:9}] toi, et de ne plus te m.
\item[\vref{Mt 8:26}] s'étant levé, il m. les vents et
\end{listverse}

\ConcordanceEntry{Menahem}
\vspace{-2mm}
\begin{listverse}
\item[\vref{2 R 15:14}] M., fils de Gadi, monta de Thirtsa
\end{listverse}

\ConcordanceEntry{Mendier}
\vspace{-2mm}
\begin{listverse}
\item[\vref{Ps 37:25}] ni sa postérité m. son pain.
\item[\vref{Mc 10:46}] assis au bord du chemin et m.
\item[\vref{Lu 16:3}] ne le puis. M. ? J'en ai honte.
\item[\vref{Jn 9:8}] l'avaient connu com. m., disaient : N'est-ce pas
\end{listverse}

\ConcordanceEntry{Mensonge}
\vspace{-2mm}
\begin{listverse}
\item[\vref{Ex 5:9}] prêtent plus attention aux paroles de m.
\item[\vref{1 R 22:23}] un esprit de m. ds la bouche
\item[\vref{Ps 52:5}] le bien, le m. plutôt que de
\item[\vref{Ps 62:10}] fils de l'hom. ! M., les fils de
\item[\vref{Ps 119:29}] la voie du m. et accorde-moi la
\item[\vref{Ps 119:104}] pourquoi je hais tte voie de m.
\item[\vref{Ps 119:128}] commandements, je hais tte voie de m.
\item[\vref{Ps 119:163}] en abomination le m. ; j'aime ta loi.
\item[\vref{Ps 144:11}] bouche profère le m. et dont la
\item[\vref{Pr 8:7}] mes lèvres ont en horreur le m.
\item[\vref{Es 57:4}] enfants de rébellion, une race de m.,
\item[\vref{Es 59:3}] lèvres profèrent le m., et votre langue
\item[\vref{Es 59:13}] prononcer du cœur des paroles de m.
\item[\vref{Jé 5:31}] prophètes prophétisent le m., et les prêtres
\item[\vref{Jé 14:14}] dit : C'est le m. que ces prophètes
\item[\vref{Jé 16:19}] ont hérité le m. et la vanité,
\item[\vref{Jé 23:25}] disent, prophétisant le m. en mon Nom,
\item[\vref{Ez 13:6}] des divinations de m., ils disent : Yahweh
\item[\vref{Os 10:13}] le fruit du m. ; parce que vs.
\item[\vref{Os 12:2}] chaque jour le m. et la violence,
\item[\vref{Na 3:1}] est pleine de m., pleine de violence ;
\item[\vref{Ha 2:18}] fonte, docteur de m., à quoi sert-elle
\item[\vref{So 3:13}] proféreront point de m., et il n'y
\item[\vref{Za 10:2}] devins prophétisent le m., ils profèrent des
\item[\vref{Jn 8:44}] qu'il profère le m., il parle de
\item[\vref{Ro 1:25}] de Dieu en m., et qui ont
\item[\vref{Ro 3:7}] si, par mon m., la vérité de
\item[\vref{Ep 4:25}] ayant dépouillé le m., parlez en vérité
\item[\vref{2 Th 2:11}] puissance d'égarement, pour qu'ils croient au m.,
\item[\vref{1 Jn 2:21}] connaissez, et qu'aucun m. ne vient de
\item[\vref{1 Jn 2:27}] n'est pas un m., demeurez en lui
\item[\vref{Ap 21:27}] l'abomination et au m. ; mais seulement ceux
\item[\vref{Ap 22:15}] et quiconque aime et pratique le m.
\end{listverse}

\ConcordanceEntry{Menteur}
\vspace{-2mm}
\begin{listverse}
\item[\vref{Ps 116:11}] ds ma précipitation : Tout hom. est m. !
\item[\vref{Pr 17:4}] trompeuse, et le m. écoute la mauvaise
\item[\vref{Pr 19:22}] et le pauvre vaut mieux qu'un m.
\item[\vref{Pr 21:28}] Le témoin m. périra, mais l'hom.
\item[\vref{Pr 30:6}] et que tu ne sois trouvé m.
\item[\vref{Jn 8:44}] car il est m. et le père
\item[\vref{Jn 8:55}] à vs., un m. ; mais je le
\item[\vref{Ro 3:4}] tt hom. pour m., selon ce qui
\item[\vref{1 Jn 1:10}] ns. le faisons m., et sa parole
\item[\vref{1 Jn 2:4}] commandements, est un m., et il n'y
\item[\vref{1 Jn 2:22}] Qui est le m., sinon celui qui
\item[\vref{1 Jn 4:20}] frère, c'est un m. ; car comment celui
\item[\vref{1 Jn 5:10}] Dieu le fait m., car il ne
\item[\vref{Ap 2:2}] et que tu les as trouvés m. ;
\item[\vref{Ap 21:8}] et ts les m., lr. part sera
\end{listverse}

\ConcordanceEntry{Mentir}
\vspace{-2mm}
\begin{listverse}
\item[\vref{No 23:19}] un hom. pour m. ni fils d'un
\item[\vref{1 R 13:18}] boive de l'eau ; mais il lui m.
\item[\vref{Ps 78:36}] et ils lui m. de lr. langue ;
\item[\vref{Ps 89:36}] par ma sainteté : M.-je à David ?
\item[\vref{Es 59:13}] pécher et de m. contre Yahweh, de
\item[\vref{Ac 5:3}] jusqu'à t'inciter à m. au Saint-Esprit, et
\item[\vref{Col 3:9}] Ne m. pas les uns aux autres, vs.
\item[\vref{Tit 1:2}] qui ne peut m., avait promise avant
\item[\vref{1 Jn 1:6}] les ténèbres, ns. m., et ns. n'agissons
\end{listverse}

\ConcordanceEntry{Mephiboscheth}
\vspace{-2mm}
\begin{listverse}
\item[\vref{2 S 4:4}] et devint boiteux ; son nom était M.
\item[\vref{2 S 21:7}] Le roi épargna M., fils de Jonathan,
\end{listverse}

\ConcordanceEntry{Mépris}
\vspace{-2mm}
\begin{listverse}
\item[\vref{Ge 16:4}] enceinte, elle regarda sa maîtresse avec m.
\item[\vref{Est 1:18}] une marque de m., ce sera aussi
\item[\vref{Job 12:21}] Il répand le m. sur les grands ;
\item[\vref{Job 31:34}] et que le m. des familles m'inspirait
\item[\vref{Ps 22:25}] il n'a ni m. ni dédain pour
\item[\vref{Ps 31:19}] le juste, avec orgueil et avec m. !
\item[\vref{Ps 107:40}] Il répand le m. sur les princes
\item[\vref{Ps 119:22}] l'opprobre et du m., car j'ai gardé
\item[\vref{Ps 123:3}] car ns. sommes assez rassasiés de m. !
\item[\vref{Ps 123:4}] des orgueilleux, du m. des hautains.
\item[\vref{Pr 12:8}] le cœur pervers est l'objet du m.
\item[\vref{Pr 18:3}] méchant vient, le m. vient aussi, et
\item[\vref{Pr 20:20}] qui traite avec m. son père ou
\item[\vref{Es 1:4}] irrité par lr. m. le Saint d'Israël,
\item[\vref{Ez 25:6}] avec tt le m. que tu as
\item[\vref{Os 12:15}] a répandu, et lui rendra ses m.
\item[\vref{Mc 9:12}] beaucoup, et qu'il soit chargé de m.
\item[\vref{Lu 23:11}] le traita avec m. ; et, après s'être
\item[\vref{Ac 19:27}] tombe ds le m., et que sa
\item[\vref{1 Co 4:10}] l'estime, et ns. sommes ds le m. !
\end{listverse}

\ConcordanceEntry{Mépriser}
\vspace{-2mm}
\begin{listverse}
\item[\vref{Ge 25:34}] alla ; ainsi Esaü m. son droit d'aînesse.
\item[\vref{No 15:31}] Parce qu'elle a m. la parole de
\item[\vref{De 27:16}] soit celui qui m. son père et
\item[\vref{1 S 2:17}] car les hommes m. l'offrande de Yahweh.
\item[\vref{2 S 6:16}] Yahweh, elle le m. en son cœur.
\item[\vref{Né 4:4}] comment ns. sommes m. ! Fais retourner leurs
\item[\vref{Est 1:17}] les portera à m. leurs maris ; elles
\item[\vref{Ps 10:3}] il estime heureux l'avare et il m. Yahweh.
\item[\vref{Ps 10:13}] Pourquoi le méchant m.-t-il Dieu ? Il
\item[\vref{Ps 22:7}] hommes et le m. du peuple.
\item[\vref{Ps 51:19}] Dieu ! tu ne m. point un cœur
\item[\vref{Ps 106:24}] Ils m. le pays désirable, et ne crurent
\item[\vref{Pr 1:7}] mais les fous m. la sagesse et
\item[\vref{Ec 9:16}] du pauvre est m., et ses paroles
\item[\vref{Es 53:3}] Il était le m. et le rejeté
\item[\vref{Jé 33:24}] élues ? Et ils m. tellement mon peuple,
\item[\vref{Mal 1:6}] vs. prêtres, qui m. mon Nom, et
\item[\vref{Mt 13:57}] Un prophète n'est m. que ds sa
\item[\vref{Mt 18:10}] Gardez-vs. de m. un seul de
\item[\vref{Lu 16:13}] à l'un, et m. l'autre. Vous ne
\item[\vref{Ac 6:1}] leurs veuves étaient m. ds le service
\item[\vref{Ro 2:4}] Ou m.-tu les richesses de sa bonté,
\item[\vref{1 Co 1:28}] monde et les m., mm celles qui
\item[\vref{1 Co 11:22}] pour boire ? Ou m.-vs. l'Eglise de
\item[\vref{1 Th 5:20}] Ne m. pas les prophéties.
\item[\vref{1 Ti 4:12}] Que personne ne m. ta jeunesse ; mais
\item[\vref{Hé 12:25}] de ne pas m. celui qui vs.
\item[\vref{2 Pi 2:10}] l'impureté, et qui m. l'autorité. Gens audacieux
\item[\vref{Jud 1:8}] souillent lr. chair, m. l'autorité et blasphèment
\end{listverse}

\ConcordanceEntry{Mer}
\vspace{-2mm}
\begin{listverse}
\item[\vref{Ge 1:26}] poissons de la m., sur les oiseaux
\item[\vref{Ex 14:21}] main sur la m., et Yahweh fit
\item[\vref{1 R 7:23}] fit aussi la m. de fonte. Elle
\item[\vref{Job 36:30}] jsq. ds les profondeurs de la m.
\item[\vref{Job 38:8}] a renfermé la m. ds ses bords,
\item[\vref{Job 38:16}] sources de la m. ? T'es-tu promené ds
\item[\vref{Ps 66:6}] fait de la m. une terre sèche,
\item[\vref{Ps 72:8}] dominera depuis une m. jusqu'à l'autre, et
\item[\vref{Ps 106:9}] il menaça la M. Rouge et elle
\item[\vref{Ps 114:3}] La m. le vit et s'enfuit, le Jourdain
\item[\vref{Ps 136:13}] a fendu la M. Rouge en deux,
\item[\vref{Pr 8:29}] limite à la m., pour que les
\item[\vref{Pr 30:19}] milieu de la m., et la trace
\item[\vref{Ec 1:7}] vont à la m., et la mer
\item[\vref{La 2:13}] grande com. une m. : Qui pourrait te
\item[\vref{Jon 1:11}] pour que la m. se calme ? Car
\item[\vref{Mc 11:23}] jette-toi ds la m. ; et s'il ne
\item[\vref{Jn 6:18}] vent, et la m. était agitée.
\item[\vref{Ro 9:27}] sable de la m., un petit reste
\item[\vref{1 Co 10:1}] ts passé au travers de la m.,
\item[\vref{2 Co 11:25}] nuit ds la m. profonde.
\item[\vref{Hé 11:29}] ils traversèrent la M. Rouge, com. un
\item[\vref{Ja 1:6}] flot de la m., agité et poussé
\item[\vref{Jud 1:13}] impétueuses de la m., rejetant l'écume de
\item[\vref{Ap 10:2}] droit sur la m., et le pied
\item[\vref{Ap 16:3}] coupe sur la m., et elle devint
\item[\vref{Ap 20:13}] Et la m. rendit les morts qui étaient en
\item[\vref{Ap 21:1}] disparu, et la m. n'était plus.
\end{listverse}

\ConcordanceEntry{Merari, Merarites}
\vspace{-2mm}
\begin{listverse}
\item[\vref{Ge 46:11}] fils de Lévi : Guerschon, Kehath, et M.
\item[\vref{No 3:33}] Et de M. est sortie la famille des Machlites,
\item[\vref{No 4:33}] des fils de M., pour tt lr.
\item[\vref{No 7:8}] aux fils de M. quatre chars et
\end{listverse}

\ConcordanceEntry{Mercenaire}
\vspace{-2mm}
\begin{listverse}
\item[\vref{De 24:14}] n'opprimeras point le m., le pauvre et
\item[\vref{Job 14:6}] que com. un m. il ait achevé
\item[\vref{Mal 3:5}] le salaire du m., qui oppriment la
\item[\vref{Jn 10:12}] Mais le m., qui n'est pas le berger, à
\end{listverse}

\ConcordanceEntry{Mère}
\vspace{-2mm}
\begin{listverse}
\item[\vref{Ge 2:24}] père et sa m. et s'attachera à
\item[\vref{Ge 3:20}] a été la m. de ts les
\item[\vref{Lé 19:3}] vs. craindra sa m. et son père,
\item[\vref{Jg 5:7}] levée pour être m. en Israël.
\item[\vref{1 R 3:27}] mourir. C'est elle qui est sa m.
\item[\vref{2 R 4:19}] dit au serviteur : Porte-le à sa m.
\item[\vref{Est 2:7}] ni père ni m.. La jeune fille
\item[\vref{Ps 27:10}] père et ma m. m'abandonnent, mais Yahweh
\item[\vref{Ps 50:20}] couvres d'opprobre le fils de ta m.
\item[\vref{Ps 51:7}] l'iniquité, et ma m. m'a conçu ds
\item[\vref{Ps 69:9}] dehors pour les fils de ma m.
\item[\vref{Ps 113:9}] en fait une m. joyeuse au milieu
\item[\vref{Ps 139:13}] couvres ds le sein de ma m.
\item[\vref{Pr 1:8}] et n'abandonne pas l'enseignement de ta m.
\item[\vref{Pr 6:20}] et n'abandonne pas l'enseignement de ta m. ;
\item[\vref{Pr 10:1}] fils insensé est l'ennui de sa m.
\item[\vref{Pr 15:20}] et un hom. insensé méprise sa m.
\item[\vref{Pr 19:26}] père et met en fuite sa m.
\item[\vref{Pr 23:25}] père et ta m. se réjouissent, que
\item[\vref{Pr 28:24}] père ou sa m., et qui dit
\item[\vref{Pr 29:15}] à lui-mm fait honte à sa m.
\item[\vref{Pr 30:11}] et qui ne bénit pas sa m.
\item[\vref{Pr 31:1}] l'instruction que sa m. lui donna.
\item[\vref{Ec 5:14}] ventre de sa m., il s'en retournera
\item[\vref{Ca 3:4}] maison de ma m. et ds la
\item[\vref{Es 66:13}] quelqu'un que sa m. caresse pour l'apaiser,
\item[\vref{Jé 1:5}] ventre de ta m., je te connaissais,
\item[\vref{Ez 16:44}] en disant : Telle m., telle fille !
\item[\vref{Mt 1:18}] Comme Marie, sa m., ayant été fiancée
\item[\vref{Mt 12:48}] Qui est ma m. et qui sont
\item[\vref{Lu 2:51}] soumis. Et sa m. gardait ttes ces
\item[\vref{Jn 3:4}] sein de sa m. et naître une
\item[\vref{Jn 19:27}] disciple : Voilà ta m.. Et dès ce
\item[\vref{Ac 1:14}] femmes, avec Marie, m. de Jésus, et
\item[\vref{Ga 1:15}] ventre de ma m., et qui m'a
\item[\vref{Ep 6:2}] père et ta m., c'est le premier
\item[\vref{2 Ti 1:5}] en Loïs, ta grand-m. et en Eunice,
\item[\vref{Hé 7:3}] sans père, sans m., sans généalogie, n'ayant
\item[\vref{Ap 17:5}] la grande, la m. des impudicités et
\end{listverse}

\ConcordanceEntry{Merveille}
\vspace{-2mm}
\begin{listverse}
\item[\vref{Ex 3:20}] par ttes les m. que je ferai
\item[\vref{Ex 34:10}] je ferai des m. qui n'ont point
\item[\vref{Né 9:17}] souvinrent point des m. que tu avais
\item[\vref{Job 37:14}] Considère encore les m. de Dieu !
\item[\vref{Ps 40:6}] as multiplié tes m. et tes desseins
\item[\vref{Ps 71:17}] et j'ai annoncé jusqu'à présent tes m.
\item[\vref{Ps 77:12}] souvenu de tes m. d'autrefois.
\item[\vref{Ps 89:6}] cieux célèbrent tes m., ô Yahweh ! Ta
\item[\vref{Ps 107:8}] bonté et ses m. envers les fils
\item[\vref{Ps 107:24}] Yahweh, et ses m. ds les lieux
\item[\vref{Ps 119:18}] je regarde aux m. de ta loi !
\item[\vref{Mc 7:37}] fait tt à m. ; mm il fait
\item[\vref{Ac 2:11}] notre langue, des m. de Dieu ?
\item[\vref{Ac 4:30}] prodiges, et des m. par le Nom
\end{listverse}

\ConcordanceEntry{Merveilleux}
\vspace{-2mm}
\begin{listverse}
\item[\vref{Jg 13:18}] Pourquoi demandes-tu mon nom ? Il est m.
\item[\vref{1 S 6:6}] eut fait de m. exploits parmi eux,
\item[\vref{Ps 119:129}] Tes préceptes sont m., c'est pourquoi mon
\item[\vref{Ps 145:5}] ta majesté, et de tes faits m. !
\end{listverse}

\ConcordanceEntry{Mésopotamie}
\vspace{-2mm}
\begin{listverse}
\item[\vref{Ge 24:10}] s'en alla en M., à la ville
\item[\vref{Jg 3:8}] Cuschan-Rischeathaïm, roi de M.. Et les enfants
\item[\vref{Ac 2:9}] qui habitent la M., la Judée, la
\item[\vref{Ac 7:2}] lorsqu'il était en M., avant qu'il s'établisse
\end{listverse}

\ConcordanceEntry{Messager}
\vspace{-2mm}
\begin{listverse}
\item[\vref{Ge 32:3}] dvt lui des m. vers Esaü, son
\item[\vref{Jos 6:17}] caché soigneusement les m. que ns. avions
\item[\vref{Job 1:13}] frère aîné, un m. vint vers Job,
\item[\vref{Ps 104:4}] des vents ses m., et des flammes
\item[\vref{Pr 13:17}] Le méchant m. tombe ds le
\item[\vref{Pr 16:14}] sont autant de m. de mort que
\item[\vref{Pr 25:13}] Le m. fidèle est à ceux qui l'envoient,
\item[\vref{Ec 5:5}] point dvt le m. de Dieu que
\item[\vref{Es 42:19}] sourd, com. mon m. que j'envoie ? Qui
\item[\vref{Mal 3:1}] Voici, j'enverrai mon m. ; il préparera le
\item[\vref{Ja 2:25}] lorsqu'elle reçut les m., et qu'elle les
\end{listverse}

\ConcordanceEntry{Messie}
\vspace{-2mm}
\begin{listverse}
\item[\vref{Ps 2:2}] eux contre Yahweh, et contre son M. ?
\item[\vref{Da 9:25}] sera rebâtie jusqu'au M., le Conducteur, il
\item[\vref{Da 9:26}] soixante-deux semaines, le M. sera retranché, mais
\item[\vref{Jn 1:41}] avons trouvé le M., c'est-à-dire le Christ.
\item[\vref{Jn 4:25}] sais que le M., c'est-à-dire le Christ,
\end{listverse}

\ConcordanceEntry{Mesure}
\vspace{-2mm}
\begin{listverse}
\item[\vref{Ge 18:6}] Hâte-toi, prends trois m. de fleur de
\item[\vref{Ps 39:5}] quelle est la m. de mes jours ;
\item[\vref{Jé 13:25}] que je te m., dit Yahweh, parce
\item[\vref{Mt 7:2}] mesurera avec la m. dont vs. mesurez.
\item[\vref{Jn 2:6}] dont chacun contenait deux ou trois m.
\item[\vref{Jn 3:34}] ne lui donne pas l'Esprit par m.
\item[\vref{Ac 1:10}] le ciel, à m. qu'il s'en allait,
\item[\vref{Ro 12:3}] modestes, selon la m. de foi que
\item[\vref{2 Co 10:12}] à lr. propre m. et en se
\item[\vref{Ep 4:7}] ns. selon la m. du don de
\item[\vref{Ep 4:13}] parfait, à la m. de la parfaite
\item[\vref{1 Th 2:16}] toujours plus la m. de leurs péchés.
\item[\vref{Ap 6:6}] qui disait : Une m. de blé pour
\end{listverse}

\ConcordanceEntry{Mesurer}
\vspace{-2mm}
\begin{listverse}
\item[\vref{Jé 31:37}] haut peuvent être m., si les fondements
\item[\vref{Za 2:2}] répondit : Je vais m. Jérus., pour voir
\item[\vref{Mt 7:2}] et l'on vs. m. avec la mesure
\item[\vref{Mc 4:24}] entendez. De la m. dont vs. mesurerez,
\item[\vref{2 Co 10:12}] Mais en se m. à lr. propre
\item[\vref{Ap 11:1}] dit : Lève-toi et m. le temple de
\item[\vref{Ap 21:15}] roseau d'or pour m. la ville, ses
\end{listverse}

\ConcordanceEntry{Mets}
\vspace{-2mm}
\begin{listverse}
\item[\vref{Ge 27:4}] Apprête-moi un m. com. j'aime, et
\item[\vref{Ge 49:20}] il fournira les m. délicats des rois.
\item[\vref{Ps 63:6}] rassasiée com. de m. gras et succulents,
\item[\vref{Da 10:3}] ne mangeai aucun m. délicat, il n'entra
\end{listverse}

\ConcordanceEntry{Metuschélah}
\vspace{-2mm}
\begin{listverse}
\item[\vref{Ge 5:21}] Hénoc vécut soixante-cinq ans, et engendra M.
\end{listverse}

\ConcordanceEntry{Meurtre}
\vspace{-2mm}
\begin{listverse}
\item[\vref{Ex 20:13}] Tu ne commettras pas de m.
\item[\vref{Os 4:2}] parjures et mensonges, m., vols et adultères ;
\item[\vref{Mc 15:7}] ds laquelle ils avaient commis un m.
\item[\vref{Ro 1:29}] pleins d'envie, de m., de querelle, de
\item[\vref{Ga 5:21}] les envies, les m., l'ivrognerie, les excès
\item[\vref{Ap 9:21}] aussi de leurs m., ni de leurs
\end{listverse}

\ConcordanceEntry{Meurtrier}
\vspace{-2mm}
\begin{listverse}
\item[\vref{No 35:6}] pour que le m. s'y enfuie, et
\item[\vref{No 35:19}] fera mourir le m. qnd il le
\item[\vref{Es 1:21}] mais mntnt elle est pleine de m. !
\item[\vref{Mt 22:7}] fit périr ces m. et brûla lr.
\item[\vref{Jn 8:44}] Il a été m. dès le commencement,
\item[\vref{Ac 3:14}] avez demandé qu'on vs. relâche un m.
\item[\vref{Ac 7:52}] avez été les traîtres et les m.,
\item[\vref{Ac 28:4}] hom. est un m. ; puisque après être
\item[\vref{1 Ti 1:9}] les profanes, pour les parricides, les m.,
\item[\vref{1 Pi 4:15}] ne souffre com. m., ou voleur, ou
\item[\vref{1 Jn 3:15}] frère est un m., et vs. savez
\item[\vref{Ap 21:8}] les abominables, les m., les fornicateurs, les
\item[\vref{Ap 22:15}] les fornicateurs, les m., les idolâtres et
\end{listverse}

\ConcordanceEntry{Meurtrissure}
\vspace{-2mm}
\begin{listverse}
\item[\vref{Ge 4:23}] et un jeune hom. pour ma m.
\item[\vref{Es 53:5}] c'est par ses m. que ns. avons
\item[\vref{1 Pi 2:24}] lui par la m. duquel vs. avez
\end{listverse}

\ConcordanceEntry{Michel}
\vspace{-2mm}
\begin{listverse}
\item[\vref{Da 10:13}] jours ; mais voici, M., l'un des principaux
\item[\vref{Jud 1:9}] Or l'archange M., lorsqu'il contestait avec
\item[\vref{Ap 12:7}] ds le ciel. M. et ses anges
\end{listverse}

\ConcordanceEntry{Mical}
\vspace{-2mm}
\begin{listverse}
\item[\vref{1 S 18:20}] Mais M., fille de Saül, aima David ; ce
\item[\vref{1 S 19:11}] mourir au matin. M., fem. de David,
\item[\vref{1 S 25:44}] Saül avait donné M., sa fille, fem.
\item[\vref{2 S 6:21}] David répondit à M. : C'est dvt Yahweh,
\end{listverse}

\ConcordanceEntry{Michée}
\vspace{-2mm}
\begin{listverse}
\item[\vref{1 R 22:8}] du mal, c'est M., fils de Jimla.
\item[\vref{1 R 22:19}] Et M. lui dit : Ecoute néanmoins la parole
\item[\vref{Mi 1:1}] qui vint à M., de Moréscheth, au
\end{listverse}

\ConcordanceEntry{Miel}
\vspace{-2mm}
\begin{listverse}
\item[\vref{Ex 3:17}] où coulent le lait et le m.
\item[\vref{De 32:13}] à sucer le m. du rocher, l'huile
\item[\vref{Jg 14:18}] doux que le m. et qu'y a-t-il
\item[\vref{1 S 14:25}] y avait du m. à la surface
\item[\vref{Ps 19:11}] doux que le m., que celui qui
\item[\vref{Ps 81:17}] le rassasierais du m. du rocher.
\item[\vref{Ps 119:103}] douce que le m. à ma bouche.
\item[\vref{Pr 5:3}] des rayons de m., et son palais
\item[\vref{Pr 16:24}] des rayons de m., douces à l'âme
\item[\vref{Pr 24:13}] fils, mange le m., car il est
\item[\vref{Pr 25:27}] manger trop de m., aussi n'y a-t-il
\item[\vref{Pr 27:7}] les rayons de m. ; mais pour l'âme
\item[\vref{Ca 4:11}] des rayons de m. ; le miel et
\item[\vref{Es 7:15}] lait et du m., jusqu'à ce qu'il
\item[\vref{Ez 3:3}] doux ds ma bouche com. du m.
\item[\vref{Mc 1:6}] et mangeait des sauterelles et du m. sauvage.
\item[\vref{Lu 24:42}] poisson rôti, et un rayon de m.
\item[\vref{Ap 10:9}] doux ds ta bouche com. du m.
\end{listverse}

\ConcordanceEntry{Miette}
\vspace{-2mm}
\begin{listverse}
\item[\vref{Mt 15:27}] chiens mangent des m. qui tombent de
\item[\vref{Mc 7:28}] la table les m. que les enfants
\item[\vref{Lu 16:21}] se rassasier des m. qui tombaient de
\end{listverse}

\ConcordanceEntry{Milliers}
\vspace{-2mm}
\begin{listverse}
\item[\vref{Ge 24:60}] puisses-tu devenir des m. de myriades, et
\item[\vref{Ex 18:21}] établis-les chefs de m., chefs de centaines,
\item[\vref{1 S 23:23}] le chercherai soigneusement parmi ts les m. de Juda.
\item[\vref{Ps 68:18}] par vingt-mille, par m. et par milliers ;
\item[\vref{Ps 144:13}] troupeaux multiplient par m., mm par dix
\item[\vref{Da 7:10}] dvt lui. Mille m. le servaient, et
\item[\vref{Da 11:12}] fera tomber des m., mais il ne
\item[\vref{Mi 5:1}] être entre les m. de Juda, de
\item[\vref{Lu 12:1}] s'étaient rassemblés par m., au point de
\item[\vref{Ac 21:20}] frère, combien de m. de Juifs ont
\end{listverse}

\ConcordanceEntry{Millions}
\vspace{-2mm}
\begin{listverse}
\item[\vref{Ez 16:7}] fait croître par m. com. l'herbe des
\item[\vref{Da 7:10}] et dix mille m. se tenaient en
\item[\vref{Jud 1:15}] qui sont par m., pour juger ts
\item[\vref{Ap 5:11}] et lr. nombre était de plusieurs m.
\item[\vref{Ap 9:16}] de deux cents m., car j'en entendis
\end{listverse}

\ConcordanceEntry{Ministre}
\vspace{-2mm}
\begin{listverse}
\item[\vref{2 S 8:18}] et les fils de David étaient ministres d'Etat.
\item[\vref{2 S 20:26}] de Jaïr était ministre d'Etat de David.
\end{listverse}

\ConcordanceEntry{Miracle}
\vspace{-2mm}
\begin{listverse}
\item[\vref{Ex 4:21}] à ts les m. que j'ai mis
\item[\vref{Ex 7:3}] signes et mes m. ds le pays
\item[\vref{Né 9:10}] Tu fis des m. et des prodiges
\item[\vref{Ps 78:43}] des m. qu'il accomplit en Egypte, et de
\item[\vref{Ps 105:5}] faites, de ses m., et des jugements
\item[\vref{Es 8:18}] signe et un m. en Israël, de
\item[\vref{Mt 7:22}] fait beaucoup de m. en ton Nom ?
\item[\vref{Mt 11:21}] car si les m. qui ont été
\item[\vref{Mt 12:38}] voudrions bien te voir faire qq m.
\item[\vref{Mt 13:58}] que peu de m., à cause de
\item[\vref{Mt 16:4}] adultère demande un m. ; mais il ne
\item[\vref{Mc 6:52}] pas compris le m. des pains, parce
\item[\vref{Mc 16:17}] Et voici les m. qui accompagneront ceux
\item[\vref{Lu 11:30}] Jonas fut un m. pour les Ninivites,
\item[\vref{Jn 2:11}] fit ce premier m. à Cana en
\item[\vref{Jn 4:48}] prodiges et des m., vs. ne croyez
\item[\vref{Jn 4:54}] encore ce second m. qnd il fut
\item[\vref{Jn 6:26}] avez vu des m., mais parce que
\item[\vref{Jn 12:18}] avait appris qu'il avait fait ce m.
\item[\vref{Jn 20:30}] disciples, beaucoup d'autres m. qui ne sont
\item[\vref{Ac 2:43}] et beaucoup de m. et de prodiges
\item[\vref{Ac 15:12}] racontèrent ts les m. et les prodiges
\item[\vref{Ro 15:19}] prodiges et des m., par la puissance
\item[\vref{1 Co 1:22}] Juifs demandent des m. et les Grecs
\item[\vref{1 Co 12:10}] les opérations des m. ; à un autre,
\item[\vref{1 Co 12:28}] le don des m., puis ceux qui
\item[\vref{2 Th 2:9}] ttes sortes de m., de signes, et
\item[\vref{Hé 2:4}] prodiges, et des m., et par plusieurs
\end{listverse}

\ConcordanceEntry{Miroir}
\vspace{-2mm}
\begin{listverse}
\item[\vref{Ex 38:8}] d'airain avec les m. des femmes qui
\item[\vref{1 Co 13:12}] au moyen d'un m., de manière obscure,
\item[\vref{2 Co 3:18}] com. ds un m. la gloire du
\item[\vref{Ja 1:23}] regarde ds un m. son visage naturel,
\end{listverse}

\ConcordanceEntry{Mischaël}
\vspace{-2mm}
\begin{listverse}
\item[\vref{Da 1:6}] Juda, Daniel, Hanania, M. et Azaria.
\item[\vref{Da 2:17}] cette affaire Hanania, M. et Azaria, ses
\end{listverse}

\ConcordanceEntry{Misérable}
\vspace{-2mm}
\begin{listverse}
\item[\vref{No 21:5}] âme est dégoûtée de cette nourriture m.
\item[\vref{Ps 22:25}] les peines du m., et il ne
\item[\vref{Ps 40:18}] suis affligé et m., mais le Seign.
\item[\vref{Ps 69:34}] Yahweh exauce les m. et ne méprise
\item[\vref{Ps 70:6}] suis affligé et m.. Ô Dieu ! hâte-toi
\item[\vref{Es 25:4}] la force du m. ds sa détresse,
\item[\vref{Es 41:17}] affligés et aux m. qui cherchent des
\item[\vref{Ro 7:24}] Ah, m. que je suis ! Qui me délivrera
\item[\vref{1 Co 15:19}] sommes les plus m. de ts les
\item[\vref{Ap 3:17}] tu es malheureux, m., pauvre, aveugle et
\end{listverse}

\ConcordanceEntry{Misère}
\vspace{-2mm}
\begin{listverse}
\item[\vref{Ps 9:14}] Yahweh ! Vois la m. où me réduisent
\item[\vref{Ps 25:18}] Resh.] Vois ma m. et ma peine,
\item[\vref{Ps 90:10}] que peine et m. ; car il passe
\item[\vref{Ps 107:41}] délivre de la m., il établit les
\item[\vref{Pr 28:19}] suit les fainéants sera accablé de m.
\item[\vref{La 1:9}] consolateur. Vois ma m., ô Yahweh ! Car
\item[\vref{Ro 3:16}] destruction et la m. sont ds leurs
\item[\vref{Hé 11:37}] réduits à la m., affligés, tourmentés,
\item[\vref{Ja 4:9}] Sentez vos m. ; et soyez ds
\end{listverse}

\ConcordanceEntry{Miséricorde}
\vspace{-2mm}
\begin{listverse}
\item[\vref{Ex 15:13}] conduit par ta m. ce peuple que
\item[\vref{Ex 20:6}] et qui fais m. jusqu'à mille générations
\item[\vref{Né 9:19}] ds ton immense m., tu ne les
\item[\vref{Ps 18:51}] roi, qui fait m. à son oint,
\item[\vref{Ps 23:6}] bonté et la m. m'accompagneront ts les
\item[\vref{Ps 25:7}] moi selon ta m., à cause de
\item[\vref{Ps 51:3}] selon ta grande m., efface mes transgressions ;
\item[\vref{Ps 101:1}] Je chanterai la m. et la justice.
\item[\vref{Ps 103:17}] Mais la m. de Yahweh est de tt temps,
\item[\vref{Ps 109:21}] parce que ta m. est grande, délivre-moi !
\item[\vref{Ps 109:26}] mon Dieu ! aide-moi, délivre-moi selon ta m.
\item[\vref{Ps 112:5}] qui exerce la m. et prête, [Yod.]
\item[\vref{Ps 119:159}] commandements, Yahweh ! Fais-moi revivre selon ta m. !
\item[\vref{Pr 28:13}] les confesse et les délaisse, obtient m.
\item[\vref{Es 30:18}] pour vs. faire m. ; car Yahweh est
\item[\vref{Os 4:1}] vérité, ni de m., ni de connaissance
\item[\vref{Mt 5:7}] sont les miséricordieux, car ils obtiendront m. !
\item[\vref{Mt 9:13}] plaisir à la m., et non aux
\item[\vref{Mt 12:7}] Je veux la m., et non pas
\item[\vref{Lu 1:50}] et sa m. s'étend de génération en génération en
\item[\vref{Lu 1:54}] et il s'est souvenu de sa m.,
\item[\vref{Lu 10:37}] a usé de m. envers lui. Jésus
\item[\vref{Ro 1:31}] promis, sans affection naturelle, implacables, sans m.
\item[\vref{Ro 9:18}] dc, il fait m. à qui il
\item[\vref{Ro 11:32}] sous la rébellion, afin de faire m. à ts.
\item[\vref{Ro 12:8}] qui exerce la m. le fasse avec
\item[\vref{2 Co 1:3}] le Père des m. et le Dieu
\item[\vref{2 Co 4:1}] service selon la m. que ns. avons
\item[\vref{Ep 2:4}] est riche en m., à cause de
\item[\vref{Col 3:12}] des entrailles de m., de bonté, d'humilité,
\item[\vref{1 Ti 1:13}] mais j'ai obtenu m. parce que j'agissais
\item[\vref{Hé 10:28}] il mourait sans m., sur la déposition
\item[\vref{Ja 2:13}] un jugement sans m. pour celui qui
\item[\vref{Ja 3:17}] conciliante, pleine de m. et de bons
\item[\vref{1 Pi 1:3}] par sa grande m. ns. a régénérés
\item[\vref{1 Pi 2:10}] n'aviez pas obtenu m., mais qui mntnt
\item[\vref{Jud 1:21}] en attendant la m. de notre Seign.
\end{listverse}

\ConcordanceEntry{Miséricordieux}
\vspace{-2mm}
\begin{listverse}
\item[\vref{Ex 22:27}] moi, je l'entendrai ; car je suis m.
\item[\vref{Ex 34:6}] le Dieu compatissant, m., lent à la
\item[\vref{De 4:31}] Dieu puissant et m., il ne t'abandonnera
\item[\vref{Né 9:17}] Dieu qui pardonne, m., compatissant, lent à
\item[\vref{Ps 130:7}] car Yahweh est m. et la rédemption
\item[\vref{Ps 145:8}] Heth.] Yahweh est m. et compatissant, lent
\item[\vref{Jé 3:12}] car je suis m., dit Yahweh, je
\item[\vref{Joë 2:13}] est compatissant et m., lent à la
\item[\vref{Jon 4:2}] un Dieu compatissant, m., lent à la
\item[\vref{Mt 5:7}] Bénis sont les m., car ils obtiendront
\item[\vref{Lu 6:36}] Soyez dc m., com. aussi votre
\item[\vref{Hé 2:17}] soit un Grand-Prêtre m. et fidèle ds
\item[\vref{1 Pi 3:8}] uns envers les autres, d'amour fraternel, m. et doux.
\end{listverse}

\ConcordanceEntry{Mitspa, Mitspé}
\vspace{-2mm}
\begin{listverse}
\item[\vref{Ge 31:49}] fut aussi appelé M. ; parce que Laban
\item[\vref{Jos 18:26}] M., Kephira, Motsa,
\item[\vref{Jg 10:17}] d'Israël se rassemblèrent et campèrent à M.
\item[\vref{Jg 11:11}] dvt Yahweh, à M., ttes les paroles
\item[\vref{1 S 7:5}] tt Israël à M., et je prierai
\item[\vref{1 S 10:17}] convoqua le peuple dvt Yahweh à M.
\end{listverse}

\ConcordanceEntry{Moab, Moabites}
\vspace{-2mm}
\begin{listverse}
\item[\vref{Ge 19:37}] du nom de M. ; c'est le père
\item[\vref{Ex 15:15}] les puissants de M., ts les habitants
\item[\vref{No 25:1}] la fornication avec les filles de M.
\item[\vref{De 2:8}] par le chemin du désert de M.
\item[\vref{2 S 8:2}] battit aussi les M. et les mesura
\item[\vref{2 R 1:1}] la mort d'Achab, M. se révolta contre
\item[\vref{Es 15:1}] Prophétie sur M.. La nuit mm
\end{listverse}

\ConcordanceEntry{Modèle}
\vspace{-2mm}
\begin{listverse}
\item[\vref{Ex 25:9}] montrer, selon le m. du tabernacle et
\item[\vref{1 Ch 28:11}] son fils, le m. du portique, de
\item[\vref{Ph 3:17}] marchent selon le m. que vs. avez
\item[\vref{1 Th 1:7}] avez été des m. à ts les
\item[\vref{2 Th 3:9}] en ns.-mêmes un m. à imiter.
\item[\vref{1 Ti 4:12}] mais sois le m. pour les fidèles
\item[\vref{2 Ti 1:13}] en Jésus-Christ, le m. des saines paroles
\item[\vref{Tit 2:7}] montrant toi-mm un m. de bonnes œuvres
\item[\vref{Hé 8:5}] choses selon le m. qui t'a été
\item[\vref{1 Pi 2:21}] ns. laissant un m., afin que vs.
\item[\vref{1 Pi 5:3}] vs. soyez les m. du troupeau.
\end{listverse}

\ConcordanceEntry{Moindre}
\vspace{-2mm}
\begin{listverse}
\item[\vref{Es 60:22}] personnes, et la m. deviendra une nation
\item[\vref{Da 2:39}] un autre royaume, m. que le tien ;
\item[\vref{Lu 12:26}] pas mm la m. chose, pourquoi êtes-vs.
\item[\vref{1 Co 15:9}] je suis le m. des apôtres, je
\item[\vref{2 Co 11:5}] été en rien m. que les plus
\item[\vref{Ep 3:8}] qui suis le m. de ts les
\item[\vref{Hé 2:7}] fait un peu m. que les anges,
\item[\vref{Hé 7:7}] qui est le m. est béni par
\end{listverse}

\ConcordanceEntry{Mois}
\vspace{-2mm}
\begin{listverse}
\item[\vref{Ge 7:11}] Noé, au second m., le dix-septième jour
\item[\vref{Ex 12:2}] Ce m.-ci sera pour vs. le premier
\item[\vref{Lé 16:29}] jour du septième m., vs. affligerez vos
\item[\vref{No 11:20}] mais jusqu'à un m. entier, jusqu'à ce
\item[\vref{Jn 4:35}] a encore quatre m. jusqu'à la moisson ?
\item[\vref{Ga 4:10}] les jours, les m., les temps et
\item[\vref{Ap 13:5}] donné le pouvoir d'agir pendant quarante-deux m.
\item[\vref{Ap 22:2}] son fruit chaque m. et les feuilles
\end{listverse}

\ConcordanceEntry{Moïse}
\vspace{-2mm}
\begin{listverse}
\item[\vref{Ex 2:10}] le nom de M. parce que, dit-elle,
\item[\vref{1 Ch 6:3}] d'Amram furent Aaron, M. et Marie. Les
\end{listverse}
\begin{legend}
\NoAutoSpaceBeforeFDP{
\item Naissance de M : Ex 2:2-10; Ac 7:20; Hé 11:23
\item Recueillit par la fille de pharaon : Ac 7:20-22
\item Sa foi : Hé 11:24-26
\item Sa fuite : Ex 2:15
\item Yahweh se révèle dans le buisson ardent : Ex 3
\item Retour en Egypte : Ex 4:19-20
\item Pharaon s'oppose à M : Ex 5:1-4; 11:5-8
\item La première Pâque : Ex 12:3,11; Hé 11:28
\item Israël sort d'Egypte : Ex 12:37-42; Hé 11:27
\item Les dix paroles : Ex 20
\item Yahweh fait monter M sur la montagne de Sinaï : Ex 24:15-18; 34:2,28
\item Le tabernacle : Ex 26:1; 40:1-2
\item M. implore Yahweh pour le peuple : Ex 32:11; No 11:2; 14:13-19; Ps 106:23
\item La gloire de Yahweh sur son visage : Ex 34:29; 2 Co 3:7-13
\item L'homme le plus doux de la terre : No 12:3
\item Un homme fidèle : No 12:7; Hé 3:2-5
\item Incrédule à Meriba : No 20:8-12; De 32:50-51
\item Le serpent d'airain : No 21:8-9; Jn 3:14
\item Josué, successeur de M : De 3:28
\item Sa mort : De 34:5-8; Jud 1:9
\item Yahweh parle face à face à M : No 12:7-8; De 34:10-12
\item Apparition durant la transfiguration : Mt 17:3
\item Annonce la venue du Messie : De 18:15-18
\item Autres: Ps 90:1; 103:7; Lu 16:29; 24:27
}
\end{legend}

\ConcordanceEntry{Moisi}
\vspace{-2mm}
\begin{listverse}
\item[\vref{Jos 9:5}] avaient pour nourriture était sec et m.
\item[\vref{Jos 9:12}] voici, il est devenu sec et m.
\end{listverse}

\ConcordanceEntry{Moisson}
\vspace{-2mm}
\begin{listverse}
\item[\vref{Ex 34:21}] temps du labourage et de la m.
\item[\vref{Pr 10:5}] dort durant la m. est un enfant
\item[\vref{Jé 8:20}] La m. est passée, l'été est fini, et
\item[\vref{Joë 3:13}] faucille, car la m. est mûre ! Venez,
\item[\vref{Mt 9:37}] ses disciples : La m. est grande, mais
\item[\vref{Mt 13:39}] le diable ; la m., c'est la fin
\item[\vref{Jn 4:35}] mois jusqu'à la m. ? Voici, je vs.
\item[\vref{Ap 14:15}] parce que la m. de la terre
\end{listverse}

\ConcordanceEntry{Moissonner}
\vspace{-2mm}
\begin{listverse}
\item[\vref{Job 4:8}] l'iniquité et qui sèment l'outrage les m. ;
\item[\vref{Ps 126:5}] sèment avec larmes, m. avec chants d'allégresse.
\item[\vref{Pr 22:8}] qui sème l'injustice m. le tourment, et
\item[\vref{Ec 11:4}] celui qui regarde les nuées, ne m. point.
\item[\vref{Mt 6:26}] sèment ni ne m., ni n'assemblent ds
\item[\vref{Mt 25:24}] hom. dur, qui m. où tu n'as
\item[\vref{Jn 4:38}] vs. ai envoyés m. ce à quoi
\item[\vref{1 Co 9:11}] affaire si ns. m. vos biens charnels ?
\item[\vref{2 Co 9:6}] qui sème peu m. aussi peu, et
\item[\vref{Ga 6:7}] qu'un hom. aura semé, il le m. aussi.
\item[\vref{Ga 6:8}] pour sa chair m. de la chair
\item[\vref{Ga 6:9}] bien ; car ns. m. au temps convenable,
\item[\vref{Ja 5:4}] ouvriers qui ont m. vos champs; et
\item[\vref{Ap 14:15}] ta faucille, et m. ; car c'est ton
\end{listverse}

\ConcordanceEntry{Moissonneur}
\vspace{-2mm}
\begin{listverse}
\item[\vref{Ru 2:3}] glana après les m.. Et il arriva
\item[\vref{Mt 13:39}] monde, et les m. sont les anges.
\item[\vref{Ja 5:4}] les cris des m. sont parvenus aux
\end{listverse}

\ConcordanceEntry{Moloc, Moloch}
\vspace{-2mm}
\begin{listverse}
\item[\vref{Lé 18:21}] le feu dvt M., et tu ne
\item[\vref{Lé 20:2}] ses enfants à M., il mourra, il
\item[\vref{2 R 23:10}] par le feu, en l'honneur de M.
\item[\vref{Jé 32:35}] leurs filles à M. ; ce que je
\item[\vref{Ac 7:43}] la tente de M., et l'étoile de
\end{listverse}

\ConcordanceEntry{Monde}
\vspace{-2mm}
\begin{listverse}
\item[\vref{Ps 50:12}] rien, car le m. est à moi
\item[\vref{Ps 93:1}] ceint. Aussi le m. est ferme, tellement
\item[\vref{Es 13:11}] Je punirai le m. habitable à cause
\item[\vref{Jé 10:12}] a fondé le m. habitable par sa
\item[\vref{Mt 13:38}] champ, c'est le m. ; la bonne semence
\item[\vref{Mc 16:15}] par tt le m., et prêchez l'Evangile
\item[\vref{Jn 1:10}] était ds le m., et le monde
\item[\vref{Jn 1:29}] Dieu, qui ôte le péché du m.
\item[\vref{Jn 3:16}] tant aimé le m. qu'il a donné
\item[\vref{Jn 4:42}] véritablement le Christ, le Sauveur du m.
\item[\vref{Jn 8:12}] la lumière du m. ; celui qui me
\item[\vref{Jn 8:23}] êtes de ce m., mais moi, je
\item[\vref{Jn 12:19}] rien ? Voici, le m. va après lui.
\item[\vref{Jn 12:47}] pour juger le m., mais pour sauver
\item[\vref{Jn 13:1}] passer de ce m. au Père, et
\item[\vref{Jn 14:17}] vérité que le m. ne peut recevoir,
\item[\vref{Jn 15:18}] Si le m. vs. hait, sachez qu'il m'a haï
\item[\vref{Jn 16:8}] il convaincra le m. de péché, de
\item[\vref{Jn 17:14}] parole ; et le m. les a haïs,
\item[\vref{Jn 21:25}] pas que le m. mm pourrait contenir
\item[\vref{Ac 17:6}] ont bouleversé le m., sont aussi venus
\item[\vref{Ro 1:20}] la création du m., qnd on les
\item[\vref{Ro 3:19}] que tt le m. soit coupable dvt
\item[\vref{Ro 5:12}] entré ds le m., et par le
\item[\vref{1 Co 7:31}] usent de ce m. com. n'en usant
\item[\vref{Ga 6:14}] par qui le m. est crucifié pour
\item[\vref{Ep 2:2}] train de ce m., selon le prince
\item[\vref{Col 2:8}] aux rudiments du m., et non pas
\item[\vref{1 Ti 1:15}] venu ds le m. pour sauver les
\item[\vref{1 Ti 6:7}] apporté ds le m., et aussi il
\item[\vref{Hé 11:38}] eux dont le m. n'était pas digne,
\item[\vref{Ja 1:27}] conserver pur des souillures de ce m.
\item[\vref{Ja 3:6}] feu ; c'est le m. de l'iniquité. La
\item[\vref{Ja 4:4}] que l'amitié du m. est inimitié contre
\item[\vref{1 Pi 2:17}] Honorez tt le m. ; aimez ts vos
\item[\vref{1 Pi 5:9}] vos frères qui sont ds le m.
\item[\vref{2 Pi 1:4}] règne ds le m. par la convoitise.
\item[\vref{2 Pi 3:6}] ces choses-là, le m. d'alors périt, étant
\item[\vref{1 Jn 2:15}] N'aimez point le m., ni les choses
\item[\vref{1 Jn 5:19}] Dieu, mais le m. entier est plongé
\item[\vref{2 Jn 1:7}] venus ds le m., qui ne confessent
\item[\vref{Ap 3:10}] arriver ds le m. entier, pour éprouver
\item[\vref{Ap 13:8}] l'Agneau immolé dès la fondation du m.
\end{listverse}

\ConcordanceEntry{Mont}
\vspace{-2mm}
\begin{listverse}
\item[\vref{Ex 19:18}] Or le m. Sinaï était tt couvert de fumée,
\item[\vref{De 27:4}] pierres-là sur le m. Ebal, selon ce
\item[\vref{De 27:12}] tiendront sur le m. Garizim, pour bénir
\item[\vref{De 27:13}] tiendront sur le m. Ebal, pour maudire.
\item[\vref{De 32:49}] M. sur cette montagne d'Abarim, sur le
\item[\vref{De 34:1}] Moab sur le m. Nebo, au sommet
\item[\vref{Jos 8:33}] du côté du m. Garizim, et l'autre
\item[\vref{Jos 24:4}] à Esaü le m. de Séir, pour
\item[\vref{1 R 18:19}] moi, sur le m. Carmel, les quatre
\item[\vref{1 R 18:20}] il rassembla les prophètes sur le m. Carmel.
\item[\vref{Ps 68:9}] de Dieu, le m. Sinaï trembla à
\item[\vref{Ps 68:16}] Dieu est un m. de Basan ; une
\item[\vref{Es 4:5}] tte l'étendue du m. Sion et sur
\item[\vref{Ha 3:3}] Saint vient du m. de Paran ; Sélah.
\item[\vref{Mt 21:1}] Bethphagé vers le M. des Oliviers, Jésus
\item[\vref{Mc 11:1}] Béthanie, vers le M. des Oliviers, Jésus
\item[\vref{Mc 13:3}] s'assit sur le M. des Oliviers, en
\item[\vref{Lu 22:39}] sa coutume, au M. des Oliviers ; et
\item[\vref{Ga 4:24}] alliances : L'une du M. Sinaï, qui n'enfante
\end{listverse}

\ConcordanceEntry{Montagne}
\vspace{-2mm}
\begin{listverse}
\item[\vref{Ge 22:14}] aujourd'hui : Dans la m. de Yahweh il
\item[\vref{Ex 3:1}] vint à la m. de Dieu à
\item[\vref{Ex 19:18}] et tte la m. tremblait fort.
\item[\vref{2 R 6:17}] Et voici la m. était pleine de
\item[\vref{Ps 2:6}] sur Sion, la m. de ma sainteté !
\item[\vref{Ps 3:5}] répond de la m. de sa sainteté.
\item[\vref{Ps 24:3}] monter à la m. de Yahweh ? Et
\item[\vref{Ps 30:8}] avais affermi ma m.… Tu cachas ta
\item[\vref{Ps 125:1}] sont com. la m. de Sion : Elle
\item[\vref{Es 2:2}] jours, que la m. de la maison
\item[\vref{Es 56:7}] amènerai sur ma m. sainte, et je
\item[\vref{Da 2:35}] devint une grande m. et remplit tte
\item[\vref{Za 4:7}] Qui es-tu, grande m., dvt Zorobabel ? Tu
\item[\vref{Mt 4:8}] une forte haute m., et lui montra
\item[\vref{Mt 17:20}] diriez à cette m. : Transporte-toi d'ici là,
\item[\vref{Mt 26:30}] rendirent à la M. des Oliviers.
\item[\vref{Mc 11:23}] dit à cette m. : Quitte ta place,
\item[\vref{Lu 3:5}] comblée, et tte m. et tte colline
\item[\vref{Jn 4:20}] adoré sur cette m., et vs., vs.
\item[\vref{Ac 1:12}] Jérus. de la m. appelée la Montagne
\item[\vref{Hé 12:18}] pas approchés d'une m. qu'on pouvait toucher
\item[\vref{2 Pi 1:18}] étions avec lui sur la sainte m.
\item[\vref{Ap 8:8}] com. une grande m. embrasée de feu,
\end{listverse}

\ConcordanceEntry{Monter}
\vspace{-2mm}
\begin{listverse}
\item[\vref{Ge 2:6}] Et il m. une vapeur de la terre qui
\item[\vref{Ge 13:1}] Abram dc m. d'Egypte vers le
\item[\vref{Ge 28:12}] anges de Dieu m. et descendaient par
\item[\vref{Ex 2:23}] et lr. cri m. jusqu'à Dieu, à
\item[\vref{Ex 19:3}] Et Moïse m. vers Dieu, car Yahweh l'avait appelé
\item[\vref{Ex 19:12}] diras : Gardez-vs. de m. sur la montagne
\item[\vref{De 30:12}] pour dire : Qui m. pour ns. aux
\item[\vref{Jos 8:1}] et lève-toi, et m. contre Aï. Regarde,
\item[\vref{Ps 18:9}] Une fumée m. de ses narines,
\item[\vref{Ps 24:3}] Qui pourra m. à la montagne
\item[\vref{Ps 66:12}] Tu as fait m. des hommes sur
\item[\vref{Ps 68:19}] Tu es m. ds les hauteurs, tu as emmené
\item[\vref{Ps 81:11}] qui t'ai fait m. hors du pays
\item[\vref{Ps 107:26}] Ils m. vers les cieux, ils descendent ds
\item[\vref{Ps 139:8}] Si je m. aux cieux, tu y es ; si
\item[\vref{Pr 8:34}] portes, et qui m. la garde aux
\item[\vref{Es 14:13}] ton cœur : Je m. aux cieux, je
\item[\vref{Da 7:3}] quatre grandes bêtes m. de la mer,
\item[\vref{Jon 1:2}] lr. malice est m. jusqu'à moi.
\item[\vref{Mc 4:7}] et les épines m., et l'étouffèrent, et
\item[\vref{Lu 2:42}] ans, ses parents m. à Jérus. selon
\item[\vref{Lu 14:10}] dise : Mon ami, m. plus haut. Alors
\item[\vref{Lu 18:10}] Deux hommes m. au temple pour
\item[\vref{Jn 1:51}] anges de Dieu m. et descendre sur
\item[\vref{Jn 3:13}] Car personne n'est m. au ciel, si
\item[\vref{Ac 2:34}] David n'est pas m. au ciel ; mais
\item[\vref{Ro 10:6}] ton cœur : Qui m. au ciel ? C'est
\item[\vref{1 Co 2:9}] ne sont pas m. au cœur de
\item[\vref{Ep 4:8}] est dit : Etant m. en haut, il
\item[\vref{Ap 4:1}] moi, me dit : M. ici, et je
\item[\vref{Ap 13:1}] Et je vis m. de la mer
\item[\vref{Ap 19:11}] celui qui était m. dessus s'appelle FIDELE
\end{listverse}

\ConcordanceEntry{Monument}
\vspace{-2mm}
\begin{listverse}
\item[\vref{Ge 28:18}] la dressa pour m., et versa de
\item[\vref{Ge 35:20}] Jacob dressa un m. sur son sépulcre.
\item[\vref{2 S 18:18}] son vivant, un m. ds la vallée
\end{listverse}

\ConcordanceEntry{Moquer}
\vspace{-2mm}
\begin{listverse}
\item[\vref{Jg 16:13}] Samson : Tu t'es m. de moi, jusqu'ici
\item[\vref{1 R 18:27}] midi, Elie se m. d'eux, et dit :
\item[\vref{2 R 2:23}] ville et se m. de lui. Ils
\item[\vref{2 Ch 36:16}] Mais ils se m. des messagers de
\item[\vref{Né 2:19}] l'ayant appris, se m. de ns. et
\item[\vref{Ps 2:4}] se rit d'eux, le Seign. se m. d'eux.
\item[\vref{Ps 22:8}] me voient se m. de moi, ils
\item[\vref{Pr 1:26}] malheur, je me m. qnd la terreur
\item[\vref{Pr 3:34}] Certes il se m. des moqueurs, mais
\item[\vref{Pr 14:9}] Les insensés se m. du péché, mais
\item[\vref{Pr 17:5}] Celui qui se m. du pauvre déshonore
\item[\vref{Pr 30:17}] celui qui se m. de son père
\item[\vref{Mt 27:41}] les anciens, se m., disaient :
\item[\vref{Lu 23:36}] soldats aussi se m. de lui ; s'approchant
\item[\vref{Ac 2:13}] les autres se m., et disaient : C'est
\item[\vref{Ac 17:32}] les uns se m., et les autres
\item[\vref{Ga 6:7}] on ne se m. pas de Dieu.
\end{listverse}

\ConcordanceEntry{Moquerie}
\vspace{-2mm}
\begin{listverse}
\item[\vref{1 R 9:7}] sarcasme et de m. parmi ts les
\item[\vref{2 Ch 29:8}] et à la m., com. vs. le
\item[\vref{Job 34:7}] qui boit la m. com. de l'eau,
\item[\vref{Ps 44:14}] en sujet de m. auprès de ceux
\item[\vref{Ps 79:4}] nos voisins, de m. et de risée
\item[\vref{Pr 1:22}] plaisir à la m., et les insensés
\item[\vref{Jé 19:8}] désolation et de m. ; quiconque passera près
\item[\vref{Jé 20:7}] un objet de m. chaque jour, chacun
\item[\vref{Jé 20:8}] d'opprobre et de m. chaque jour.
\item[\vref{Jé 29:18}] un étonnement, une m. et un opprobre
\item[\vref{Jé 48:26}] et deviendra aussi un sujet de m. !
\item[\vref{Jé 48:27}] un objet de m. ? Avait-il été trouvé
\item[\vref{Jé 48:39}] un objet de m. et de frayeur
\item[\vref{Ez 23:32}] un sujet de risée et de m.
\item[\vref{Ez 36:4}] un sujet de m. aux autres nations
\item[\vref{Os 7:16}] un objet de m. ds le pays
\item[\vref{Hé 11:36}] d'autres subirent les m. et le fouet,
\end{listverse}

\ConcordanceEntry{Moqueur}
\vspace{-2mm}
\begin{listverse}
\item[\vref{Pr 9:7}] qui corrige le m. en reçoit de
\item[\vref{Pr 9:8}] reprends pas le m., de peur qu'il
\item[\vref{Pr 9:12}] si tu es m., tu en porteras
\item[\vref{Pr 13:1}] père, mais le m. n'écoute pas la
\item[\vref{Pr 14:6}] Le m. cherche la sagesse et ne la
\item[\vref{Pr 15:12}] Le m. n'aime pas qu'on le reprenne, et
\item[\vref{Pr 19:25}] tu bats le m., le stupide en
\item[\vref{Pr 20:1}] Le vin est m. et la boisson
\item[\vref{Pr 21:11}] on punit le m., le sot devient
\item[\vref{Pr 21:24}] On appelle m. un superbe arrogant,
\item[\vref{Pr 22:10}] Chasse le m., et le débat
\item[\vref{Pr 24:9}] péché, et le m. est en abomination
\item[\vref{Es 29:20}] prendra fin, le m. sera consumé, et
\item[\vref{2 Pi 3:3}] il viendra des m., se conduisant selon
\item[\vref{Jud 1:18}] y aurait des m. qui marcheraient selon
\end{listverse}

\ConcordanceEntry{Morceau}
\vspace{-2mm}
\begin{listverse}
\item[\vref{Ge 15:10}] et mit chaque m. l'un vis-à-vis de
\item[\vref{Pr 17:1}] Mieux vaut un m. de pain sec
\item[\vref{Pr 28:21}] car pour un m. de pain l'hom.
\item[\vref{Jn 13:27}] eut pris le m., Satan entra en
\end{listverse}

\ConcordanceEntry{Morija}
\vspace{-2mm}
\begin{listverse}
\item[\vref{Ge 22:2}] au pays de M. ; et là, offre-le
\end{listverse}

\ConcordanceEntry{Mort}
\vspace{-2mm}
\begin{listverse}
\item[\vref{Ex 4:19}] en voulaient à ta vie sont m.
\item[\vref{Ex 12:30}] où il n'y ait eu un m.
\item[\vref{Ex 16:3}] ne sommes-ns. point m. par la main
\item[\vref{No 16:48}] tenait entre les m. et les vivants,
\item[\vref{De 18:11}] bonne aventure, personne qui interroge les m.
\item[\vref{De 30:15}] le bien, la m. et le mal.
\item[\vref{Ru 2:11}] ton mari est m., m'a été entièrement
\item[\vref{2 S 12:5}] qui a fait cela mérite la m.
\item[\vref{2 R 4:40}] de Dieu, la m. est ds le
\item[\vref{Est 4:11}] la peine de m. contre quiconque, hom.
\item[\vref{Job 30:23}] conduis à la m. et ds la
\item[\vref{Ps 13:4}] ne dorme du sommeil de la m.,
\item[\vref{Ps 18:5}] liens de la m. m'avaient environné et
\item[\vref{Ps 23:4}] l'ombre de la m., je ne craindrais
\item[\vref{Ps 33:19}] délivre de la m., et les fasse
\item[\vref{Ps 56:14}] âme de la m., tu as garanti
\item[\vref{Ps 116:15}] La m. des bien-aimés de Yahweh est précieuse
\item[\vref{Pr 2:18}] penche vers la m., et son chemin
\item[\vref{Pr 8:36}] ceux qui me haïssent, aiment la m.
\item[\vref{Pr 10:2}] mais la justice délivre de la m.
\item[\vref{Pr 14:12}] l'issue sont les voies de la m.
\item[\vref{Ec 4:2}] j'estime plus les m. qui sont déjà
\item[\vref{Ec 7:1}] jour de la m. que le jour
\item[\vref{Ec 9:4}] chien vivant vaut mieux qu'un lion m.
\item[\vref{Ec 9:5}] mourront, mais les m. ne savent rien,
\item[\vref{Ca 8:6}] fort com. la m., et la jalousie
\item[\vref{Es 6:1}] L'année de la m. du roi Ozias,
\item[\vref{Es 8:19}] qui évoquent les m. et les diseurs
\item[\vref{Es 19:3}] qui évoquent les m. et ceux qui
\item[\vref{Es 25:8}] Il détruit la m. par sa victoire ;
\item[\vref{Es 53:12}] âme à la m., qu'il a été
\item[\vref{Jé 8:3}] Et la m. sera plus désirable que la vie
\item[\vref{Ez 18:32}] plaisir à la m. de celui qui
\item[\vref{Ez 37:9}] souffle sur ces m., et qu'ils revivent !
\item[\vref{Os 13:14}] délivrerai de la m.. Ô mort, où
\item[\vref{Mt 4:16}] l'ombre de la m., la lumière elle-mm
\item[\vref{Mt 8:22}] et laisse les m. ensevelir leurs morts.
\item[\vref{Mt 9:18}] Ma fille est m. il y a
\item[\vref{Mt 10:8}] lépreux, ressuscitez les m., chassez les démons
\item[\vref{Mt 11:5}] sourds entendent, les m. sont ressuscités, et
\item[\vref{Mt 22:32}] le Dieu des m., mais des vivants.
\item[\vref{Mc 5:39}] fille n'est pas m., mais elle dort.
\item[\vref{Mc 9:26}] L'enfant devint com. m., de sorte que
\item[\vref{Mc 12:27}] le Dieu des m., mais le Dieu
\item[\vref{Mc 15:44}] qu'il soit déjà m. ; il fit venir
\item[\vref{Lu 7:22}] sourds entendent, les m. ressuscitent, l'Evangile est
\item[\vref{Lu 9:7}] disaient que Jean était ressuscité des m. ;
\item[\vref{Lu 15:24}] que voici était m., mais il est
\item[\vref{Lu 23:22}] soit digne de m.. Après l'avoir fait
\item[\vref{Lu 24:5}] cherchez-vs. parmi les m. celui qui est
\item[\vref{Jn 5:24}] passé de la m. à la vie.
\item[\vref{Jn 5:25}] venue, que les m. entendront la voix
\item[\vref{Jn 8:51}] parole, il ne verra jamais la m.
\item[\vref{Jn 11:4}] pas à la m., mais elle est
\item[\vref{Jn 12:9}] voir Lazare, qu'il avait ressuscité des m.
\item[\vref{Jn 12:33}] indiquer de quelle m. il devait mourir.
\item[\vref{Ac 4:2}] la résurrection des m. au Nom de
\item[\vref{Ac 8:1}] consentait à la m. d'Etienne, et en
\item[\vref{Ac 13:30}] Mais Dieu l'a ressuscité des m.
\item[\vref{Ac 22:4}] J'ai persécuté à m. cette doctrine, liant
\item[\vref{Ro 1:32}] sont dignes de m., non seulement ils
\item[\vref{Ro 5:8}] pécheurs, Christ est m. pour ns.
\item[\vref{Ro 5:10}] Dieu par la m. de son Fils,
\item[\vref{Ro 5:12}] le péché la m. ; ainsi la mort
\item[\vref{Ro 6:2}] ns. qui sommes m. au péché, comment
\item[\vref{Ro 6:4}] baptême en sa m., afin que, com.
\item[\vref{Ro 6:11}] aussi, considérez-vs. com. m. au péché, mais
\item[\vref{Ro 6:23}] péché, c'est la m. ; mais le don
\item[\vref{Ro 7:24}] me délivrera du corps de cette m. ?
\item[\vref{Ro 8:2}] loi du péché et de la m.
\item[\vref{Ro 8:6}] chair, c'est la m., mais l'affection de
\item[\vref{Ro 8:11}] Jésus d'entre les m. habite en vs.,
\item[\vref{Ro 8:38}] que ni la m., ni la vie,
\item[\vref{Ro 10:7}] C'est faire remonter Christ d'entre les m.
\item[\vref{Ro 10:9}] l'a ressuscité des m., tu seras sauvé.
\item[\vref{Ro 14:9}] que Christ est m., qu'il est ressuscité,
\item[\vref{Ro 14:15}] viande celui pour lequel Christ est m.
\item[\vref{1 Co 8:11}] lequel Christ est m., périra par ta
\item[\vref{1 Co 11:26}] vs. annoncerez la m. du Seign., jusqu'à
\item[\vref{1 Co 15:3}] que Christ est m. pour nos péchés,
\item[\vref{1 Co 15:16}] Car si les m. ne ressuscitent pas,
\item[\vref{1 Co 15:21}] Car puisque la m. est venue par
\item[\vref{1 Co 15:26}] sera détruit le dernier c'est la m.
\item[\vref{1 Co 15:56}] l'aiguillon de la m. c'est le péché ;
\item[\vref{2 Co 3:7}] service de la m., écrit avec des
\item[\vref{2 Co 4:12}] sorte que la m. opère en ns.,
\item[\vref{2 Co 5:14}] un seul est m. pour ts, alors
\item[\vref{2 Co 7:10}] la tristesse du monde produit la m.
\item[\vref{Ep 2:1}] lorsque vs. étiez m. ds vos fautes
\item[\vref{Ep 2:5}] lorsque ns. étions m. ds nos offenses,
\item[\vref{Ep 5:14}] relève-toi d'entre les m., et Christ t'éclairera.
\item[\vref{Ph 1:21}] vie, et la m. m'est un gain.
\item[\vref{Ph 2:8}] obéissant jusqu'à la m., mm jusqu'à la
\item[\vref{Ph 3:11}] puis, à la résurrection d'entre les m.
\item[\vref{Col 1:18}] premier-né d'entre les m., afin qu'il tienne
\item[\vref{Col 2:13}] lorsque vs. étiez m. ds vos offenses,
\item[\vref{Col 2:20}] dc vs. êtes m. avec Christ quant
\item[\vref{Col 3:3}] Car vs. êtes m., et votre vie
\item[\vref{1 Th 4:16}] ciel, et les m. en Christ ressusciteront
\item[\vref{1 Ti 5:6}] les plaisirs est m. quoique vivante.
\item[\vref{2 Ti 1:10}] a détruit la m. et qui a
\item[\vref{Hé 2:14}] que, par la m., il rende impuissant
\item[\vref{Hé 9:14}] conscience des œuvres m., pour servir le
\item[\vref{Hé 9:15}] afin que, la m. étant intervenue pour
\item[\vref{Hé 9:16}] nécessaire que la m. du testateur soit
\item[\vref{Hé 11:13}] Tous ceux-ci sont m. ds la foi,
\item[\vref{Hé 13:20}] ramené d'entre les m. le grand Pasteur
\item[\vref{Ja 1:15}] le péché, étant consommé, produit la m.
\item[\vref{Ja 2:26}] sans esprit est m., de mm la
\item[\vref{Ja 5:20}] âme de la m. et couvrira une
\item[\vref{1 Pi 2:24}] bois, afin qu'étant m. au péché, ns.
\item[\vref{1 Pi 3:18}] à Dieu, étant m. en la chair,
\item[\vref{1 Pi 4:5}] à juger les vivants et les m.
\item[\vref{1 Jn 3:14}] passés de la m. à la vie,
\item[\vref{1 Jn 5:17}] qui ne mène pas à la m.
\item[\vref{Jud 1:12}] fruits, deux fois m., et déracinés ;
\item[\vref{Ap 1:18}] je vis ; j'étais m., et voici, je
\item[\vref{Ap 2:10}] fidèle jusqu'à la m., et je te
\item[\vref{Ap 2:11}] n'aura pas à souffrir la seconde m.
\item[\vref{Ap 3:1}] réputation d'être vivant, mais tu es m.
\item[\vref{Ap 6:8}] se nommait la M., et le Hadès
\item[\vref{Ap 13:3}] com. blessée à m., mais sa blessure
\item[\vref{Ap 14:13}] à présent les m. qui meurent ds
\item[\vref{Ap 20:14}] Et la m. et l'enfer furent jetés ds l'étang
\item[\vref{Ap 21:8}] de soufre, qui est la seconde m.
\end{listverse}

\ConcordanceEntry{Mot}
\vspace{-2mm}
\begin{listverse}
\item[\vref{Mt 22:46}] répondre un seul m.. Et depuis ce
\item[\vref{Mt 27:37}] marquée en ces m. : CELUI-CI EST JESUS,
\item[\vref{Lu 1:67}] Saint-Esprit, et il prophétisa, en ces m. :
\item[\vref{1 Ti 6:4}] des disputes de m., d'où naissent l'envie,
\item[\vref{2 Ti 2:14}] de disputes de m., qui est une
\item[\vref{Hé 12:27}] Or ces m. : Une fois encore, marquent le changement
\item[\vref{Ap 19:16}] étaient écrits ces m. : LE ROI DES
\end{listverse}

\ConcordanceEntry{Mourir}
\vspace{-2mm}
\begin{listverse}
\item[\vref{Ge 2:17}] en mangeras, tu m., tu mourras.
\item[\vref{Ge 18:25}] que tu fasses m. le juste avec
\item[\vref{Ge 20:4}] Seign., feras-tu dc m. une nation juste ?
\item[\vref{Ex 21:17}] sa mère, il m., il mourra.
\item[\vref{1 S 20:32}] Pourquoi le ferait-on m. ? Qu'a-t-il fait ?
\item[\vref{2 S 1:16}] disant : J'ai fait m. l'oint de Yahweh !
\item[\vref{1 R 19:17}] Jéhu le fera m. ; et quiconque échappera
\item[\vref{2 R 20:1}] malade à la m.. Le prophète Esaïe,
\item[\vref{Ps 105:29}] sang et fit m. leurs poissons.
\item[\vref{Pr 19:18}] ne va pas jusqu'à le faire m.
\item[\vref{Ec 3:2}] un temps pour m. ; un temps pour
\item[\vref{Da 5:19}] car il faisait m. ceux qu'il voulait,
\item[\vref{Mt 14:5}] voulait le faire m., mais il craignait
\item[\vref{Mt 17:23}] qu'ils le feront m., mais le troisième
\item[\vref{Mt 23:31}] qui ont fait m. les prophètes.
\item[\vref{Mt 26:4}] par finesse, afin de le faire m.
\item[\vref{Mt 26:35}] s'il me fallait m. avec toi, je
\item[\vref{Jn 5:18}] à le faire m., parce que non
\item[\vref{Jn 16:2}] quiconque vs. fera m. croira rendre un
\item[\vref{Ac 2:23}] vs. l'avez fait m. par les mains
\item[\vref{Ac 26:10}] on les faisait m., je joignais mon
\item[\vref{Ro 5:7}] quelqu'un voudrait bien m. pour un bienfaiteur.
\item[\vref{Ro 6:10}] Car il est m., et c'est à
\item[\vref{Ro 7:11}] commandement, et par lui me fit m.
\item[\vref{Ro 8:13}] l'Esprit vs. faites m. les actions du
\item[\vref{Hé 9:27}] aux hommes de m. une seule fois,
\item[\vref{Hé 11:22}] la foi, Joseph m. fit mention de
\item[\vref{Ap 3:2}] reste qui va m. ; car je n'ai
\item[\vref{Ap 9:6}] hommes chercheront la m., mais ils ne
\end{listverse}

\ConcordanceEntry{Muet}
\vspace{-2mm}
\begin{listverse}
\item[\vref{Ex 4:11}] Et qui rend m. ou sourd, voyant
\item[\vref{Ps 39:3}] Je suis resté m., ds le silence ;
\item[\vref{Pr 31:8}] en faveur du m., pour le droit
\item[\vref{Es 35:6}] la langue du m. chantera en triomphe.
\item[\vref{Ez 3:26}] palais, tu seras m., et tu ne
\item[\vref{Ez 24:27}] ne seras plus m. ; ainsi tu seras
\item[\vref{Ez 33:22}] était ouverte et je n'étais plus m.
\item[\vref{Mt 9:32}] Jésus un hom. m. et démoniaque.
\item[\vref{Mt 9:33}] été chassé, le m. parla. Et les
\item[\vref{Mt 12:22}] démon, aveugle et m., et il le
\item[\vref{Mc 9:17}] fils qui est possédé d'un esprit m.
\item[\vref{Mc 9:25}] lui dit : Esprit m. et sourd, je
\item[\vref{Lu 1:20}] voici, tu seras m., et tu ne
\item[\vref{Lu 1:22}] faisait des signes, et il resta m.
\item[\vref{Lu 11:14}] démon qui était m.. Lorsque le démon
\item[\vref{Ac 8:32}] com. un agneau m. dvt celui qui
\item[\vref{1 Co 14:10}] cependant aucun de ces sons n'est m. ;
\end{listverse}

\ConcordanceEntry{Multiplier}
\vspace{-2mm}
\begin{listverse}
\item[\vref{Ge 1:22}] disant : Soyez féconds, m., et remplissez les
\item[\vref{Ex 1:7}] abondamment, et se m. et devinrent extrêmement
\item[\vref{Ex 1:12}] et plus il m. et croissait en
\item[\vref{Ex 23:29}] champs ne se m. contre toi.
\item[\vref{De 8:1}] viviez, que vs. m., et que vs.
\item[\vref{De 28:63}] et en vs. m., de mm Yahweh
\item[\vref{Esd 9:6}] iniquités se sont m. au-dessus de nos
\item[\vref{Ps 40:6}] Dieu ! tu as m. tes merveilles et
\item[\vref{Ps 49:17}] les trésors de sa maison se m.
\item[\vref{Ps 115:14}] Yahweh vs. m. ses bénédictions, à
\item[\vref{Pr 4:10}] années de ta vie te seront m.
\item[\vref{Ec 10:14}] Or l'insensé m. les paroles. L'hom.
\item[\vref{Es 1:15}] vs. ; qnd vs. m. vos prières, je
\item[\vref{Jé 30:19}] et je les m., et ils ne
\item[\vref{Mt 6:7}] vs. priez, ne m. pas de vaines
\item[\vref{Mt 24:12}] que l'iniquité sera m., la charité de
\item[\vref{Ac 6:1}] les disciples se m., il s'éleva un
\item[\vref{Ac 7:17}] s'augmenta et se m. en Egypte ;
\item[\vref{2 Co 9:10}] pain à manger, m. votre semence, et
\item[\vref{Hé 6:14}] et je te m. merveilleusement.
\item[\vref{2 Pi 1:2}] paix vs. soient m. par la connaissance
\item[\vref{Jud 1:2}] la paix et l'amour vs. soient m. !
\end{listverse}

\ConcordanceEntry{Multitude}
\vspace{-2mm}
\begin{listverse}
\item[\vref{Ge 17:4}] deviendras père d'une m. de nations.
\item[\vref{Ex 23:2}] suivras point la m. pour faire le
\item[\vref{Ec 5:2}] vient de la m. des occupations ; ainsi
\item[\vref{Lu 2:13}] à l'Ange une m. de l'armée céleste,
\item[\vref{Ac 14:1}] manière, qu'une grande m. de Juifs et
\item[\vref{Hé 12:22}] Jérus. céleste, d'une m. innombrable d'anges,
\item[\vref{Ja 5:20}] et couvrira une m. de péchés.
\item[\vref{1 Pi 4:8}] charité couvre une m. de péchés.
\item[\vref{Ap 7:9}] voici une grande m. de gens, que
\end{listverse}

\ConcordanceEntry{Mur, Muraille}
\vspace{-2mm}
\begin{listverse}
\item[\vref{Ex 14:22}] lr. servaient de m. à droite et
\item[\vref{Jos 2:15}] était sur la m. de la ville,
\item[\vref{1 S 25:16}] ont servi de m. nuit et jour,
\item[\vref{1 R 20:30}] d'Aphek, où la m. tomba sur vingt-sept
\item[\vref{Né 1:3}] l'opprobre ; et la m. de Jérus. demeure
\item[\vref{Né 6:15}] Néanmoins, la m. fut achevée le
\item[\vref{Ps 18:30}] avec mon Dieu je franchis la m.
\item[\vref{Ps 51:20}] Sion, édifie les m. de Jérus. !
\item[\vref{Ps 122:7}] soit ds tes m., et la tranquillité
\item[\vref{Pr 18:11}] com. une haute m. de retraite, selon
\item[\vref{Pr 25:28}] une brèche, et qui est sans m.
\item[\vref{Es 59:10}] le long du m., ns. tâtonnons com.
\item[\vref{Es 60:10}] étrangers rebâtiront tes m., et leurs rois
\item[\vref{Es 62:6}] gardes sur tes m. tt le jour
\item[\vref{Jé 15:20}] ce peuple une m. d'airain bien forte ;
\item[\vref{Za 2:5}] dit Yahweh, une m. de feu tt
\item[\vref{Ac 9:25}] descendirent par la m. ds une corbeille.
\item[\vref{Ac 23:3}] Dieu te frappera, m. blanchie ! Tu es
\item[\vref{2 Co 11:33}] descendit de la m. ds une corbeille,
\item[\vref{Ep 2:14}] en détruisant le m. de séparation,
\item[\vref{Hé 11:30}] la foi, les m. de Jéricho tombèrent,
\item[\vref{Ap 21:12}] grande et haute m., avec douze portes,
\end{listverse}

\ConcordanceEntry{Murmure}
\vspace{-2mm}
\begin{listverse}
\item[\vref{Ex 16:8}] aura entendu les m. que vs. avez
\item[\vref{No 17:5}] dvt moi les m. des enfants d'Israël,
\item[\vref{1 R 19:12}] feu, vint un m. doux et léger.
\item[\vref{Es 29:4}] sera com. un m. sortant de la
\item[\vref{Jn 7:12}] avait un grand m. à son sujet
\item[\vref{Ac 6:1}] il s'éleva un m. des Hellénistes contre
\item[\vref{Ph 2:14}] ttes choses sans m. et sans disputes,
\item[\vref{1 Pi 4:9}] les uns envers les autres, sans m.
\end{listverse}

\ConcordanceEntry{Murmurer}
\vspace{-2mm}
\begin{listverse}
\item[\vref{Ex 15:24}] Et le peuple m. contre Moïse en
\item[\vref{Ex 16:2}] des enfants d'Israël m. ds ce désert
\item[\vref{No 14:27}] cette méchante assemblée m. contre moi ? J'ai
\item[\vref{No 14:36}] retour avaient fait m. contre lui tte
\item[\vref{Jos 9:18}] Mais tte l'assemblée m. contre les chefs.
\item[\vref{Ps 106:25}] Et ils m. ds leurs tentes, et n'obéirent point
\item[\vref{Lu 5:30}] et les pharisiens, m. contre ses disciples,
\item[\vref{Jn 6:41}] Or les Juifs m. contre lui de
\item[\vref{1 Co 10:10}] Ne m. pas, com. quelques-uns d'entre eux murmurèrent
\item[\vref{Jud 1:16}] des gens qui m., qui se plaignent
\end{listverse}

\ConcordanceEntry{Myriade}
\vspace{-2mm}
\begin{listverse}
\item[\vref{Ge 24:60}] des milliers de m., et que ta
\item[\vref{Ps 3:7}] crains pas les m. de peuples qnd
\item[\vref{Mi 6:7}] ou à des m. de torrents d'huile ?
\end{listverse}

\ConcordanceEntry{Myrrhe}
\vspace{-2mm}
\begin{listverse}
\item[\vref{Ex 30:23}] exquises ; de la m. franche, le poids
\item[\vref{Est 2:12}] de l'huile de m., et six mois
\item[\vref{Ps 45:9}] sont parfumés de m., d'aloès et de
\item[\vref{Mt 2:11}] l'or, de l'encens et de la m.
\item[\vref{Jn 19:39}] un mélange de m. et d'aloès d'environ
\end{listverse}

\ConcordanceEntry{Mystère}
\vspace{-2mm}
\begin{listverse}
\item[\vref{Mt 13:11}] de connaître les m. du Royaume des
\item[\vref{Mc 4:11}] de connaître le m. du Royaume de
\item[\vref{Ro 11:25}] vs. ignoriez ce m., afin que vs.
\item[\vref{Ro 16:25}] la révélation du m. qui a été
\item[\vref{1 Co 2:7}] qui est un m., c'est-à-dire cachée, que
\item[\vref{1 Co 15:51}] vs. dis un m. : Nous ne dormirons
\item[\vref{Ep 1:9}] faisant connaître le m. de sa volonté,
\item[\vref{Ep 3:4}] que j'ai du m. de Christ,
\item[\vref{Ep 3:9}] été accordée du m. qui était caché
\item[\vref{Ep 5:32}] Ce m. est grand, or je parle de
\item[\vref{Ep 6:19}] faire connaître le m. de l'Evangile,
\item[\vref{Col 1:26}] à savoir le m. qui avait été
\item[\vref{Col 1:27}] gloire de ce m. parmi les Gentils,
\item[\vref{Col 2:2}] la connaissance du m. de notre Dieu
\item[\vref{Col 4:3}] afin d'annoncer le m. de Christ pour
\item[\vref{2 Th 2:7}] Car le m. de l'iniquité opère déjà, seulement celui
\item[\vref{1 Ti 3:9}] conservant le m. de la foi
\item[\vref{1 Ti 3:16}] sans contredit, le m. de la piété
\item[\vref{Ap 1:20}] Le m. des sept étoiles que tu as
\item[\vref{Ap 10:7}] la trompette, le m. de Dieu sera
\item[\vref{Ap 17:5}] nom écrit, un m. : Babylone la grande,
\item[\vref{Ap 17:7}] te dirai le m. de la fem.
\end{listverse}

\ConcordanceEntry{Naaman}
\vspace{-2mm}
\begin{listverse}
\item[\vref{2 R 5:1}] Or N., chef de l'armée du roi de
\item[\vref{Lu 4:27}] si ce n'est N., le Syrien.
\end{listverse}

\ConcordanceEntry{Nabal}
\vspace{-2mm}
\begin{listverse}
\item[\vref{1 S 25:3}] de l'hom. était N., et le nom
\end{listverse}

\ConcordanceEntry{Naboth}
\vspace{-2mm}
\begin{listverse}
\item[\vref{1 R 21:1}] ces choses, que N. de Jizreel, ayant
\item[\vref{2 R 9:21}] le champ de N. de Jizreel.
\end{listverse}

\ConcordanceEntry{Nachor}
\vspace{-2mm}
\begin{listverse}
\item[\vref{Ge 11:26}] ans, engendra Abram, N., et Haran.
\item[\vref{Ge 24:10}] en Mésopotamie, à la ville de N.
\end{listverse}

\ConcordanceEntry{Nadab}
\vspace{-2mm}
\begin{listverse}
\item[\vref{Lé 10:1}] les fils d'Aaron, N. et Abihu, prirent
\item[\vref{1 R 15:25}] Or N., fils de Jéroboam, régna sur Israël
\end{listverse}

\ConcordanceEntry{Nahum}
\vspace{-2mm}
\begin{listverse}
\item[\vref{Na 1:1}] la vision de N. d'Elkosch.
\end{listverse}

\ConcordanceEntry{Naissance}
\vspace{-2mm}
\begin{listverse}
\item[\vref{Ru 2:11}] pays de ta n., et tu es
\item[\vref{Ec 7:1}] mort que le jour de la n.
\item[\vref{Ez 16:3}] origine et ta n. du pays de
\item[\vref{Mt 1:18}] manière arriva la n. de Jésus-Christ. Comme
\item[\vref{Mt 14:6}] l'on célébra la n. d'Hérode, la fille
\item[\vref{Mc 6:21}] jour de sa n., donna un festin
\item[\vref{Lu 1:14}] et plusieurs se réjouiront de sa n.,
\item[\vref{Jn 9:1}] il vit un hom. aveugle de n.
\item[\vref{Ac 3:2}] hom. boiteux de n., qu'on portait et
\item[\vref{Ac 22:28}] dit Paul, je l'ai par ma n.
\item[\vref{Hé 11:23}] Moïse, à sa n., fut caché pendant
\end{listverse}

\ConcordanceEntry{Naît d'en haut}
\vspace{-2mm}
\begin{listverse}
\item[\vref{Jn 3:3}] Si quelqu'un ne naît d'en haut, il ne peut
\end{listverse}

\ConcordanceEntry{Naître}
\vspace{-2mm}
\begin{listverse}
\item[\vref{Ge 4:4}] côté, offrit des premiers-n. de son troupeau,
\item[\vref{Ex 4:22}] Yahweh : Israël est mon fils, mon premier-n.
\item[\vref{Ex 12:29}] frappa ts les premiers-n. du pays d'Egypte,
\item[\vref{No 3:12}] place de tt premier-n. qui ouvre la
\item[\vref{De 32:18}] oublié le Dieu qui t'a fait n.
\item[\vref{Job 14:1}] L'hom. n. de la fem. est de courte
\item[\vref{Ps 22:32}] au peuple qui n., parce qu'il aura
\item[\vref{Ps 29:9}] de Yahweh fait n. les biches, et
\item[\vref{Ps 51:7}] Voici, je suis n. ds l'iniquité, et
\item[\vref{Ps 87:4}] l'Ethiopie: C'est ds Sion qu'ils sont n.
\item[\vref{Pr 6:14}] temps, il fait n. des querelles.
\item[\vref{Ec 3:2}] Un temps pour n. et un temps
\item[\vref{Es 66:8}] Ferait-on qu'un pays n. en un jour ?
\item[\vref{Mt 2:4}] auprès d'eux où le Christ devait n.
\item[\vref{Mt 3:9}] Dieu peut faire n. de ces pierres
\item[\vref{Lu 2:7}] enfanta son fils, premier-n.. Et elle l'emmaillota,
\item[\vref{Lu 3:8}] Dieu peut faire n., mm de ces
\item[\vref{Jn 1:13}] lesquels sont n., non du sang,
\item[\vref{Jn 3:4}] un hom. peut-il n. qnd il est
\item[\vref{Ac 2:30}] reins il ferait n. selon la chair
\item[\vref{Ro 8:29}] qu'il soit le premier-n. de beaucoup de
\item[\vref{Ga 4:23}] celui de l'esclave n. selon la chair,
\item[\vref{Col 1:18}] commencement et le premier-n. d'entre les morts,
\item[\vref{Hé 1:6}] monde son Fils premier-n., il est dit :
\item[\vref{1 Jn 4:7}] son prochain est n. de Dieu et
\item[\vref{1 Jn 5:4}] ce qui est n. de Dieu est
\item[\vref{1 Jn 5:18}] que quiconque est n. de Dieu ne
\item[\vref{Ap 1:5}] témoin fidèle, le premier-n. d'entre les morts,
\end{listverse}

\ConcordanceEntry{Naomi}
\vspace{-2mm}
\begin{listverse}
\item[\vref{Ru 1:2}] de sa fem. N. et les noms
\end{listverse}

\ConcordanceEntry{Nard}
\vspace{-2mm}
\begin{listverse}
\item[\vref{Ca 1:12}] à table, mon n. exhale son parfum.
\item[\vref{Jn 12:3}] une livre de n. pur de grand
\end{listverse}

\ConcordanceEntry{Nathan}
\vspace{-2mm}
\begin{listverse}
\item[\vref{2 S 7:2}] qu'il dit à N. le prophète : Regarde
\item[\vref{2 S 12:1}] Yahweh envoya N. vers David. Nathan
\item[\vref{1 R 1:34}] le prêtre et N. le prophète l'oignent
\item[\vref{1 Ch 29:29}] le livre de N. le prophète, et
\item[\vref{2 Ch 9:29}] le livre de N. le prophète, ds
\end{listverse}

\ConcordanceEntry{Nathanaël}
\vspace{-2mm}
\begin{listverse}
\item[\vref{Jn 1:45}] Philippe rencontra N., et lui dit :
\end{listverse}

\ConcordanceEntry{Nation}
\vspace{-2mm}
\begin{listverse}
\item[\vref{Ge 12:2}] devenir une grande n., et je te
\item[\vref{Ge 18:18}] deviendra certainement une n. grande et puissante,
\item[\vref{Ex 19:6}] prêtres, et une n. sainte ; ce sont
\item[\vref{De 28:36}] toi, vers une n. que tu n'auras
\item[\vref{Job 34:29}] de tte une n. soit d'un seul
\item[\vref{Ps 33:12}] Bénie est la n. dont Yahweh est
\item[\vref{Ps 43:1}] cause contre une n. infidèle ! Délivre-moi de
\item[\vref{Es 2:4}] jugement parmi les n., et reprendra plusieurs
\item[\vref{Es 26:2}] portes, et la n. juste, celle qui
\item[\vref{Es 66:8}] jour ? Ou une n. naîtrait-elle d'un seul
\item[\vref{Jé 33:24}] eux il ne sera plus une n.
\item[\vref{Ez 37:22}] d'eux une seule n. ds le pays,
\item[\vref{So 2:1}] avec soin ô n. non désirée !
\item[\vref{Mt 24:7}] Car une n. s'élèvera contre une autre nation, et
\item[\vref{1 Pi 2:9}] prêtrise royale, la n. sainte, le peuple
\item[\vref{Ap 5:9}] de tt peuple, et de tte n. ;
\end{listverse}

\ConcordanceEntry{Nature}
\vspace{-2mm}
\begin{listverse}
\item[\vref{Ro 1:26}] en celui qui est contre la n. ;
\item[\vref{Ro 2:27}] L'incirconcis de n., qui accomplit la
\item[\vref{Ro 11:24}] sauvage selon sa n., et greffé contrairement
\item[\vref{1 Co 11:14}] La n. elle-mm ne vs. enseigne-t-elle pas que
\item[\vref{Ga 4:8}] ne le sont pas de lr. n.
\item[\vref{Ep 2:3}] ns. étions par n. des enfants de
\item[\vref{Ja 3:7}] domptés par la n. humaine ;
\item[\vref{2 Pi 1:4}] participants de la n. divine, en fuyant
\item[\vref{Jud 1:7}] les péchés contre n., ont été mises
\end{listverse}

\ConcordanceEntry{Naturel}
\vspace{-2mm}
\begin{listverse}
\item[\vref{Ro 1:26}] ont changé l'usage n. en celui qui
\item[\vref{Ro 1:27}] de mm l'usage n. de la fem.,
\item[\vref{Ja 1:23}] regarde ds un miroir son visage n.,
\end{listverse}

\ConcordanceEntry{Naufrage}
\vspace{-2mm}
\begin{listverse}
\item[\vref{2 Co 11:25}] fois, j'ai fait n. trois fois, j'ai
\item[\vref{1 Ti 1:19}] ils ont fait n. quant à la
\end{listverse}

\ConcordanceEntry{Nazaréen}
\vspace{-2mm}
\begin{listverse}
\item[\vref{Ge 49:26}] la tête du N. d'entre ses frères.
\item[\vref{Mt 2:23}] par les prophètes : Il sera appelé N.
\item[\vref{Ac 6:14}] que Jésus, ce N., détruira ce lieu-ci,
\end{listverse}

\ConcordanceEntry{Nazareth}
\vspace{-2mm}
\begin{listverse}
\item[\vref{Mt 2:23}] la ville appelée N., afin que s'accomplisse
\item[\vref{Mt 21:11}] le prophète de N., en Galilée.
\item[\vref{Lu 4:16}] se rendit à N., où il avait
\item[\vref{Jn 1:46}] de bon de N. ? Philippe lui dit :
\item[\vref{Jn 19:19}] mots : JESUS DE N., LE ROI DES
\end{listverse}

\ConcordanceEntry{Naziréat}
\vspace{-2mm}
\begin{listverse}
\item[\vref{No 6:2}] un vœu de n. pour se consacrer
\end{listverse}

\ConcordanceEntry{Nebucadnetsar}
\vspace{-2mm}
\begin{listverse}
\item[\vref{2 R 24:1}] De son temps, N., roi de Babylone,
\item[\vref{2 R 25:8}] année du roi N., roi de Babylone,
\item[\vref{Jé 25:1}] première année de N., roi de Babylone,
\item[\vref{Da 2:1}] du règne de N., Nebucadnetsar eut des
\item[\vref{Da 3:1}] Le roi N. fit une statue d'or, dont la
\item[\vref{Da 4:33}] parole s'accomplit sur N.. Il fut chassé
\end{listverse}

\ConcordanceEntry{Négligence}
\vspace{-2mm}
\begin{listverse}
\item[\vref{Esd 4:22}] cela de la n., de peur que
\item[\vref{Esd 6:9}] de Jérus., jour après jour, sans n.,
\end{listverse}

\ConcordanceEntry{Néhémie}
\vspace{-2mm}
\begin{listverse}
\item[\vref{Né 1:1}] Paroles de N., fils de Hacalia.
\end{listverse}
\begin{legend}
\NoAutoSpaceBeforeFDP{
\item Gouverneur au pays de Juda : Né 5:14; 8:9
\item Reconstruction de la muraille : Né 4:6; 6:15
\item Purification des chambres du temple : Né 13:4-9
}
\end{legend}

\ConcordanceEntry{Neige}
\vspace{-2mm}
\begin{listverse}
\item[\vref{Ex 4:6}] était blanche de lèpre com. la n.
\item[\vref{No 12:10}] blanche com. la n. ; et Aaron se
\item[\vref{2 S 23:20}] frappa un lion, un jour de n.
\item[\vref{Job 9:30}] de l'eau de n., et que je
\item[\vref{Job 38:22}] trésors de la n. ? As-tu vu les
\item[\vref{Ps 51:9}] je serai plus blanc que la n.
\item[\vref{Pr 25:13}] fraîcheur de la n. au temps de
\item[\vref{Pr 26:1}] Comme la n. ne convient pas en été, ni
\item[\vref{Pr 31:21}] craint point la n. pour sa maison,
\item[\vref{Es 1:18}] blanchis com. la n. ; s'ils sont rouges
\item[\vref{Es 55:10}] pluie et la n. descendent des cieux
\item[\vref{Da 7:9}] blanc com. la n., et les cheveux
\item[\vref{Mt 28:3}] son vêt. blanc com. de la n.
\item[\vref{Mc 9:3}] com. de la n., tels qu'il n'est
\item[\vref{Ap 1:14}] com. de la n., et ses yeux
\end{listverse}

\ConcordanceEntry{Nephthali}
\vspace{-2mm}
\begin{listverse}
\item[\vref{Ge 30:8}] pourquoi elle l'appela du nom de N.
\item[\vref{No 26:50}] les familles de N., selon leurs familles,
\item[\vref{De 33:23}] Il dit de N. : Nephthali, rassasié de
\item[\vref{Jos 19:32}] aux fils de N., selon leurs familles.
\item[\vref{Jg 5:18}] la mort ; et N. de mm, sur
\item[\vref{1 R 15:20}] Kinneroth, et tt le pays de N.
\end{listverse}

\ConcordanceEntry{Nettoyer}
\vspace{-2mm}
\begin{listverse}
\item[\vref{Jé 4:11}] non pas pour vanner ni pour n.
\item[\vref{Mt 23:26}] Pharisien aveugle ! n. premièrement l'intérieur de
\item[\vref{Lu 11:39}] autres pharisiens, vs. n. le dehors de
\item[\vref{Ja 4:8}] de vs.. Pécheurs, n. vos mains ; et
\end{listverse}

\ConcordanceEntry{Nicodème}
\vspace{-2mm}
\begin{listverse}
\item[\vref{Jn 3:1}] les pharisiens, nommé N., qui était un
\item[\vref{Jn 7:50}] N., celui qui était venu vers Jésus
\end{listverse}

\ConcordanceEntry{Nicolaïtes}
\vspace{-2mm}
\begin{listverse}
\item[\vref{Ap 2:6}] les œuvres des N., œuvres que je
\item[\vref{Ap 2:15}] la doctrine des N. ; ce que je
\end{listverse}

\ConcordanceEntry{Nimrod}
\vspace{-2mm}
\begin{listverse}
\item[\vref{Ge 10:8}] Cusch engendra aussi N., c'est lui qui
\end{listverse}

\ConcordanceEntry{Ninive}
\vspace{-2mm}
\begin{listverse}
\item[\vref{Ge 10:11}] et il bâtit N. et les rues
\item[\vref{Jon 1:2}] Lève-toi, va à N., la grande ville,
\item[\vref{Jon 3:5}] Les hommes de N. crurent à Dieu,
\item[\vref{Na 1:1}] Prophétie sur N., qui est le
\item[\vref{Lu 11:32}] Les gens de N. se lèveront, au
\end{listverse}

\ConcordanceEntry{Noce}
\vspace{-2mm}
\begin{listverse}
\item[\vref{Ca 3:11}] jour de ses n., le jour de
\item[\vref{Mt 22:2}] qui fit des n. pour son fils.
\item[\vref{Mt 22:12}] un habit de n. ? Cet hom. eut
\item[\vref{Mt 25:10}] la salle des n., puis la porte
\item[\vref{Lu 12:36}] maître revienne des n., afin de lui
\item[\vref{Lu 14:8}] quelqu'un à des n., ne te mets
\item[\vref{Jn 2:1}] on faisait des n. à Cana en
\item[\vref{Ap 19:7}] gloire, car les n. de l'Agneau sont
\end{listverse}

\ConcordanceEntry{Noé}
\vspace{-2mm}
\begin{listverse}
\item[\vref{Ge 5:29}] Et il l'appela N., en disant : Celui-ci
\item[\vref{Ge 6:8}] Mais N. trouva grâce aux yeux de Yahweh.
\item[\vref{Ge 9:8}] parla aussi à N. et à ses
\item[\vref{Mt 24:37}] aux jours de N., il en sera
\item[\vref{Hé 11:7}] Par la foi, N., ayant été divinement
\item[\vref{1 Pi 3:20}] les jours de N., tandis que l'arche
\item[\vref{2 Pi 2:5}] s'il a préservé N., lui huitième, qui
\end{listverse}

\ConcordanceEntry{Nom}
\vspace{-2mm}
\begin{listverse}
\item[\vref{Ge 32:29}] fais-moi connaître ton N.. Et il répondit :
\item[\vref{Ex 3:15}] C'est ici mon N. éternellement, et c'est
\item[\vref{Ex 6:3}] point été connu d'eux par mon N. YAHWEH.
\item[\vref{Ru 4:5}] pour maintenir le n. du défunt ds
\item[\vref{1 S 17:45}] contre toi au N. de Yahweh des
\item[\vref{Job 1:21}] enlevé ; que le n. de Yahweh soit
\item[\vref{Ps 8:2}] Seign., que ton N. est magnifique sur
\item[\vref{Ps 20:8}] ns. glorifierons le N. de Yahweh, notre
\item[\vref{Ps 102:22}] qu'on annonce le N. de Yahweh ds
\item[\vref{Ps 115:1}] mais à ton N. donne gloire, à
\item[\vref{Ps 119:55}] souviens de ton N. pendant la nuit
\item[\vref{Pr 18:10}] Le N. de Yahweh est une tour forte,
\item[\vref{Ca 1:3}] excellents parfums, ton n. est com. un
\item[\vref{Es 42:8}] c'est là mon N. ; et je ne
\item[\vref{Ez 20:9}] l'amour de mon N., afin qu'il ne
\item[\vref{Za 14:9}] jour-là, Yahweh sera Un, et son n. sera Un.
\item[\vref{Mt 6:9}] cieux ! Que ton N. soit sanctifié ;
\item[\vref{Mt 7:22}] prophétisé en ton N. ? N'avons-ns. pas chassé
\item[\vref{Mt 24:5}] viendront sous mon N., disant : Je suis
\item[\vref{Mc 5:9}] Quel est ton n. ? Légion est mon
\item[\vref{Jn 1:12}] croient en son N., il lr. a
\item[\vref{Jn 17:11}] garde-les en ton N., ceux, dis-je, que
\item[\vref{Ac 4:10}] que c'est au N. de Jésus-Christ de
\item[\vref{Ac 9:15}] pour porter mon N. dvt les Gentils,
\item[\vref{Ro 2:24}] Car le N. de Dieu est blasphémé parmi les
\item[\vref{Ep 1:21}] au-dessus de tt n. qui se nomme,
\item[\vref{Ph 2:9}] a donné le N. qui est au-dessus
\item[\vref{Col 3:17}] faites tt au N. du Seign. Jésus,
\item[\vref{2 Ti 2:19}] Quiconque invoque le N. du Seign., qu'il
\item[\vref{Ja 2:7}] blasphèment le beau N. qui a été
\item[\vref{Ap 2:17}] écrit un nouveau n., que nul ne
\item[\vref{Ap 3:5}] n'effacerai pas son n. du Livre de
\item[\vref{Ap 3:12}] sur lui le N. de mon Dieu,
\item[\vref{Ap 19:12}] il avait un n. écrit que personne
\end{listverse}

\ConcordanceEntry{Nombre}
\vspace{-2mm}
\begin{listverse}
\item[\vref{De 4:27}] resterez qu'un petit n. parmi les nations,
\item[\vref{2 R 6:16}] en plus grand n. que ceux qui
\item[\vref{Ps 40:6}] déclarer, mais lr. n. est trop grand
\item[\vref{Ps 104:24}] sont en grand n. ! Tu les as
\item[\vref{1 Co 12:12}] corps, malgré lr. n., ne forment qu'un
\item[\vref{1 Pi 3:20}] laquelle un petit n., à savoir huit
\item[\vref{Ap 13:18}] l'intelligence compte le n. de la bête,
\end{listverse}

\ConcordanceEntry{Nourrir}
\vspace{-2mm}
\begin{listverse}
\item[\vref{Ge 42:19}] du blé pour n. vos familles.
\item[\vref{Ge 42:33}] prenez de quoi n. vos familles et
\item[\vref{Ex 16:21}] fallait pour se n., et lorsque la
\item[\vref{2 S 16:2}] d'été sont pour n. les jeunes gens,
\item[\vref{1 R 13:15}] et tu prendras de quoi te n.
\item[\vref{1 R 17:4}] j'ai commandé aux corbeaux de t'y n.
\item[\vref{1 R 17:9}] à une fem. veuve de t'y n.
\item[\vref{1 R 18:4}] il les y n. de pain et
\item[\vref{1 R 22:27}] hom. en prison, n.-le de pain
\item[\vref{Es 1:2}] Yahweh parle. J'ai n. des enfants, je
\item[\vref{Es 55:2}] ce qui ne n. pas ? Pourquoi travaillez-vs.
\item[\vref{Mt 6:26}] Père céleste les n.. N'êtes vs. pas
\item[\vref{1 Ti 4:6}] serviteur de Jésus-Christ, n. des paroles de
\item[\vref{Ap 12:6}] afin d'y être n. pendant mille deux
\end{listverse}

\ConcordanceEntry{Nourriture}
\vspace{-2mm}
\begin{listverse}
\item[\vref{Ge 1:30}] herbe verte pour n.. Et cela fut
\item[\vref{Ge 9:3}] vie sera votre n. ; je vs. donne
\item[\vref{De 2:28}] vendras de la n. à prix d'argent,
\item[\vref{1 R 13:7}] tu prendras qq n. et je te
\item[\vref{1 R 19:8}] lui donna cette n., il marcha quarante
\item[\vref{Job 36:31}] il donne la n. en abondance.
\item[\vref{Job 39:3}] qui apprête la n. au corbeau, qnd
\item[\vref{Ps 42:4}] larmes sont ma n. jour et nuit,
\item[\vref{Ps 59:16}] là, cherchant lr. n., et qu'ils passent
\item[\vref{Ps 104:27}] lr. donnes la n. en lr. temps.
\item[\vref{Ps 111:5}] donné de la n. à ceux qui
\item[\vref{Pr 30:25}] qui néanmoins préparent durant l'été lr. n. ;
\item[\vref{Pr 31:14}] navires d'un marchand, elle amène sa n. de loin.
\item[\vref{Mt 6:25}] plus que la n., et le corps
\item[\vref{Mt 10:10}] ni bâton ; car l'ouvrier mérite sa n.
\item[\vref{Mt 24:45}] lr. donner la n. au temps opportun ?
\item[\vref{Lu 12:23}] plus que la n., et le corps
\item[\vref{Jn 4:34}] lr. dit : Ma n. est de faire
\item[\vref{Jn 6:55}] est une véritable n., et mon sang
\item[\vref{Ac 14:17}] ns. donnant la n. avec abondance, et
\item[\vref{1 Co 13:3}] biens pour la n. des pauvres, qnd
\item[\vref{1 Ti 6:8}] ns. avons la n. et le vêt.,
\item[\vref{Hé 5:14}] Mais la n. solide est pour ceux qui sont
\end{listverse}

\ConcordanceEntry{Nouveau, Nouvelle}
\vspace{-2mm}
\begin{listverse}
\item[\vref{Lé 23:16}] vs. offrirez à Yahweh un gâteau n.
\item[\vref{Ps 33:3}] Chantez-lui un cantique n. ! Jouez de vos
\item[\vref{Pr 3:10}] et tes cuves regorgeront de vin n.
\item[\vref{Ec 1:9}] a rien de n. sous le soleil.
\item[\vref{Es 42:10}] Yahweh un cantique n., et que sa
\item[\vref{Jé 31:31}] je traiterai une n. alliance avec la
\item[\vref{Ez 11:19}] eux un esprit n. ; j'ôterai de lr.
\item[\vref{Ez 36:26}] vs. donnerai un n. cœur, je mettrai
\item[\vref{Os 10:12}] défrichez-vs. un champ n. ! Car il est
\item[\vref{Joë 1:5}] cause du vin n., parce qu'il est
\item[\vref{Mt 26:28}] sang de la N. Alliance, qui est
\item[\vref{Mc 2:22}] met du vin n. ds de vieilles
\item[\vref{Jn 13:34}] vs. donne un n. commandement : Aimez-vs. les
\item[\vref{Ac 17:19}] quelle est cette n. doctrine que tu
\item[\vref{Ro 7:6}] ds un esprit n., et non selon
\item[\vref{Ro 14:9}] a repris une n. vie, afin qu'il
\item[\vref{1 Co 5:7}] vs. soyez une n. pâte, puisque vs.
\item[\vref{1 Co 11:25}] coupe est la N. Alliance en mon
\item[\vref{1 Co 15:58}] toujours avec un n. zèle à l'œuvre
\item[\vref{2 Co 3:6}] serviteurs de la N. Alliance, non de
\item[\vref{2 Co 5:17}] il est une n. créature ; les choses
\item[\vref{Ga 6:15}] efficacité, mais la n. créature.
\item[\vref{Ep 2:15}] être un hom. n., en faisant la
\item[\vref{Col 2:16}] d'un jour de n. lune, ou de
\item[\vref{1 Ti 3:6}] qu'il soit un n. converti, de peur
\item[\vref{Hé 9:15}] Médiateur de la N. Alliance, afin que,
\item[\vref{Hé 10:20}] est le chemin n. et vivant qu'il
\item[\vref{1 Pi 2:2}] com. des enfants n.-nés, le lait
\item[\vref{2 Pi 3:13}] sa promesse, de n. cieux et une
\item[\vref{1 Jn 2:7}] point un commandement n., mais un commandement
\item[\vref{Ap 2:17}] sera écrit un n. nom, que nul
\item[\vref{Ap 3:12}] qui est la n. Jérus. qui descend
\item[\vref{Ap 5:9}] chantaient un cantique n., en disant : Tu
\item[\vref{Ap 14:3}] com. un cantique n. dvt le trône,
\item[\vref{Ap 21:1}] je vis un n. ciel et une
\item[\vref{Ap 21:2}] ville sainte, la n. Jérus., qui descendait
\end{listverse}

\ConcordanceEntry{Nouvelle (une)}
\vspace{-2mm}
\begin{listverse}
\item[\vref{2 S 1:20}] publiez point la n. ds les rues
\item[\vref{2 S 18:19}] roi la bonne n. que Yahweh lui
\item[\vref{1 R 2:28}] Cette n. arriva à Joab, qui avait suivi
\item[\vref{Lu 1:19}] parler, et pour t'annoncer cette bonne n.
\item[\vref{Lu 2:10}] annonce une bonne n., qui sera un
\item[\vref{Ac 13:32}] annonçons cette bonne n. que la promesse
\end{listverse}

\ConcordanceEntry{Nu, Nue}
\vspace{-2mm}
\begin{listverse}
\item[\vref{Ge 3:10}] que je suis n., et je me
\item[\vref{Job 1:21}] Je suis sorti n. du ventre de
\item[\vref{Ec 5:14}] il s'en retournera n., s'en allant com.
\item[\vref{Mt 25:36}] j'étais n., et vs. m'avez vêtu ; j'étais malade,
\item[\vref{Mt 25:43}] recueilli ; j'ai été n., et vs. ne
\item[\vref{Mc 14:52}] son linceul, et se sauva tt n.
\item[\vref{Jn 21:7}] parce qu'il était n., et se jeta
\item[\vref{Ro 1:20}] com. à l'œil n., depuis la création
\item[\vref{Ap 3:17}] es malheureux, misérable, pauvre, aveugle et n.
\item[\vref{Ap 16:15}] ne pas marcher n., et qu'on ne
\end{listverse}

\ConcordanceEntry{Nuage}
\vspace{-2mm}
\begin{listverse}
\item[\vref{2 S 23:4}] un matin sans n. ; son éclat fait
\item[\vref{1 R 18:44}] Voici un petit n. qui s'élève de
\item[\vref{Job 37:16}] le balancement des n., les merveilles de
\item[\vref{Ez 30:18}] force cessera ; un n. la couvrira, et
\item[\vref{Lu 12:54}] vs. voyez un n. se lever à
\end{listverse}

\ConcordanceEntry{Nudité}
\vspace{-2mm}
\begin{listverse}
\item[\vref{Ge 9:23}] ils couvrirent la n. de lr. père ;
\item[\vref{Ex 20:26}] peur que ta n. ne soit découverte
\item[\vref{Lé 18:6}] pour découvrir sa n.. Je suis Yahweh.
\item[\vref{Ro 8:35}] famine, ou la n., ou le péril,
\item[\vref{1 Co 4:11}] la soif, la n. ; on ns. frappe
\item[\vref{2 Co 11:27}] ds le froid et ds la n.
\item[\vref{Ap 3:18}] honte de ta n. ne paraisse pas ;
\end{listverse}

\ConcordanceEntry{Nue, Nuée}
\vspace{-2mm}
\begin{listverse}
\item[\vref{Ex 13:21}] une colonne de n. pour les conduire
\item[\vref{Ex 16:10}] gloire de Yahweh apparut ds la n.
\item[\vref{Ex 40:34}] Et la n. couvrit la tente d'assignation, et la
\item[\vref{No 9:21}] Et qnd la n. y était depuis
\item[\vref{No 10:11}] année, que la n. se leva de
\item[\vref{No 12:5}] la colonne de n. et se tint
\item[\vref{2 S 22:10}] avait une épaisse n. sous ses pieds.
\item[\vref{1 R 8:10}] lieu saint, la n. remplit la maison
\item[\vref{Job 7:9}] La n. se dissipe et s'en va, ainsi
\item[\vref{Job 38:9}] lui donnai la n. pour vêt., et
\item[\vref{Ps 18:10}] avait une épaisse n. sous ses pieds.
\item[\vref{Ps 36:6}] atteint jusqu'aux cieux, ta fidélité jusqu'aux n.
\item[\vref{Ps 78:14}] jour par la n., et tte la
\item[\vref{Ps 97:2}] La n. et l'obscurité sont autour de lui ;
\item[\vref{Ps 99:7}] la colonne de n. ; ils ont gardé
\item[\vref{Ps 105:39}] Il étendit la n. pour couverture, et
\item[\vref{Pr 16:15}] est com. la n. portant la pluie
\item[\vref{Es 44:22}] transgressions com. une n. épaisse, et tes
\item[\vref{Os 6:4}] est com. la n. du matin, com.
\item[\vref{Mt 17:5}] encore, voici une n. resplendissante les couvrit
\item[\vref{Mt 24:30}] venant sur les n. du ciel, avec
\item[\vref{Lu 9:34}] parlait ainsi, une n. vint les couvrir
\item[\vref{Lu 21:27}] venant sur une n. avec puissance et
\item[\vref{Ac 1:9}] le regardaient, une n. le prit et
\item[\vref{1 Co 10:1}] été sous la n., et qu'ils ont
\item[\vref{1 Co 10:2}] Moïse ds la n. et ds la
\item[\vref{Hé 12:1}] d'une si grande n. de témoins, rejetons
\item[\vref{Hé 12:18}] ni de la n. épaisse, ni des
\item[\vref{Jud 1:12}] Ce sont des n. sans eau, emportées
\item[\vref{Ap 1:7}] vient avec les n., et tt œil
\end{listverse}

\ConcordanceEntry{Nuire}
\vspace{-2mm}
\begin{listverse}
\item[\vref{1 S 25:26}] qui cherchent à n. à mon seigneur
\item[\vref{Job 35:8}] ta méchanceté peut n., et c'est au
\item[\vref{Ps 56:6}] ttes leurs pensées tendent à me n.
\item[\vref{Jé 44:11}] vs. pour vs. n. et pour retrancher
\item[\vref{Da 11:27}] cœur à se n., et à la
\item[\vref{Lu 10:19}] l'ennemi ; et rien ne pourra vs. n.
\end{listverse}

\ConcordanceEntry{Nuit}
\vspace{-2mm}
\begin{listverse}
\item[\vref{Ge 1:5}] appela les ténèbres n.. Ainsi fut le
\item[\vref{Ex 12:30}] se leva de n., lui et ses
\item[\vref{Jg 6:27}] il l'exécuta de n. et non de
\item[\vref{Ps 19:3}] jour, et une n. fait connaître sa
\item[\vref{Ps 91:5}] terreurs de la n., ni la flèche
\item[\vref{Ps 119:55}] Nom pendant la n. et je garde
\item[\vref{Ps 139:11}] me couvriront, la n. mm sera une
\item[\vref{Lu 5:5}] travaillé tte la n., et ns. n'avons
\item[\vref{Lu 12:20}] Insensé ! Cette mm n. ton âme te
\item[\vref{Jn 9:4}] m'a envoyé. La n. vient, où personne
\item[\vref{Ro 13:12}] La n. est avancée et le jour approche.
\item[\vref{1 Co 11:23}] Seign. Jésus, la n. où il fut
\item[\vref{1 Th 5:7}] dorment, dorment la n., et ceux qui
\item[\vref{1 Ti 5:5}] Dieu, et persévère n. et jour ds
\item[\vref{Ap 4:8}] dire jour et n. : Saint ! Saint ! Saint
\item[\vref{Ap 22:5}] aura plus de n. ; et ils n'auront
\end{listverse}

\ConcordanceEntry{Obed}
\vspace{-2mm}
\begin{listverse}
\item[\vref{Ru 4:17}] l'appelèrent du nom d'O.. Ce fut le
\item[\vref{Mt 1:5}] Rahab ; Boaz engendra O., de Ruth ; Obed
\end{listverse}

\ConcordanceEntry{Obed-Edom}
\vspace{-2mm}
\begin{listverse}
\item[\vref{2 S 6:10}] la fit conduire ds la maison d'O. de Gath.
\end{listverse}

\ConcordanceEntry{Obéir}
\vspace{-2mm}
\begin{listverse}
\item[\vref{Ge 22:18}] que tu as o. à ma voix.
\item[\vref{Ge 41:40}] tt mon peuple o. à ta bouche ;
\item[\vref{Ex 5:2}] Yahweh pour que j'o. à sa voix
\item[\vref{De 11:27}] bénédiction, si vs. o. aux commandements de
\item[\vref{De 28:15}] Mais si tu n'o. point à la
\item[\vref{Jos 24:24}] Dieu et ns. o. à sa voix.
\item[\vref{Ps 81:12}] voix, et Israël ne m'a point o.
\item[\vref{Ps 106:25}] leurs tentes, et n'o. point à la
\item[\vref{Es 1:20}] si vs. refusez d'o. et si vs.
\item[\vref{Jé 26:5}] pour o. aux paroles des prophètes, mes serviteurs,
\item[\vref{Jé 35:13}] pas d'instruction pour o. à mes paroles ?
\item[\vref{Da 7:27}] royaumes lui seront assujettis et lui o.
\item[\vref{Mt 8:27}] celui-ci à qui o. mm les vents
\item[\vref{Mc 1:27}] aux esprits impurs, et ils lui o.
\item[\vref{Ac 4:19}] Dieu de vs. o. plutôt qu'à Dieu ;
\item[\vref{Ac 5:29}] Il faut plutôt o. à Dieu qu'aux
\item[\vref{Ac 6:7}] aussi de prêtres o. à la foi.
\item[\vref{Ac 7:39}] voulurent pas lui o., mais ils le
\item[\vref{Ro 2:8}] la vérité, et o. à l'injustice.
\item[\vref{Ro 6:12}] pour que vs. o. à ses convoitises.
\item[\vref{Ro 6:16}] esclaves pour lui o., vs. êtes esclaves
\item[\vref{Ga 5:7}] pour vs. empêcher d'o. à la vérité ?
\item[\vref{Ep 6:1}] Enfants, o. à vos pères et à vos
\item[\vref{Tit 3:1}] et aux autorités, d'o. aux gouverneurs, d'être
\item[\vref{1 Pi 1:2}] de l'Esprit afin d'o. à Jésus-Christ, et
\end{listverse}

\ConcordanceEntry{Obéissance}
\vspace{-2mm}
\begin{listverse}
\item[\vref{1 S 15:22}] sacrifices, autant qu'à l'o. à sa voix ?
\item[\vref{Ro 5:19}] de mm par l'o. d'un seul, plusieurs
\item[\vref{Ro 6:16}] mort, soit de l'o. qui conduit à
\item[\vref{Ro 15:18}] les Gentils à l'o., par la parole,
\item[\vref{Ro 16:19}] Car votre o. est venue à
\item[\vref{2 Co 7:15}] souvient de votre o. à ts, et
\item[\vref{2 Co 10:5}] pensée captive à l'o. de Christ.
\item[\vref{2 Co 10:6}] désobéissance, lorsque votre o. sera complète.
\item[\vref{Phm 1:21}] persuadé de ton o., je t'écris sachant
\item[\vref{Hé 5:8}] a pourtant appris l'o. par les choses
\end{listverse}

\ConcordanceEntry{Obéissant}
\vspace{-2mm}
\begin{listverse}
\item[\vref{De 21:18}] indocile et rebelle, n'o. point à la
\item[\vref{De 30:20}] ton Dieu, en o. à sa voix,
\item[\vref{Ps 103:20}] ses affaires, en o. à la voix
\item[\vref{2 Co 2:9}] si vs. êtes o. en ttes choses.
\item[\vref{Ph 2:8}] en se rendant o. jusqu'à la mort,
\item[\vref{1 Pi 1:14}] Comme des enfants o., ne vs. conformez
\item[\vref{1 Pi 1:22}] vos âmes en o. à la vérité
\end{listverse}

\ConcordanceEntry{Obscurité}
\vspace{-2mm}
\begin{listverse}
\item[\vref{Ge 15:12}] frayeur d'une grande o. tomba sur lui.
\item[\vref{Ex 14:20}] nuée et une o. ; et pour les
\item[\vref{1 R 8:12}] Salomon dit : Yahweh veut habiter ds l'o. !
\item[\vref{Ps 97:2}] La nuée et l'o. sont autour de
\item[\vref{Es 8:22}] sombres angoisses : Il sera enfoncé ds l'o.
\item[\vref{Es 29:18}] étant délivrés de l'o. et des ténèbres,
\item[\vref{Joë 2:2}] de ténèbres et d'o., jour de nuées
\item[\vref{2 Pi 2:17}] gens à qui l'o. des ténèbres est
\item[\vref{Jud 1:6}] a réservé sous l'o., ds des liens
\end{listverse}

\ConcordanceEntry{Observation}
\vspace{-2mm}
\begin{listverse}
\item[\vref{1 S 15:22}] les sacrifices, et l'o. de sa parole
\item[\vref{1 Co 7:19}] n'est rien, mais l'o. des commandements de
\end{listverse}

\ConcordanceEntry{Observer}
\vspace{-2mm}
\begin{listverse}
\item[\vref{Ge 24:9}] et lui jura d'o. ces choses.
\item[\vref{Ex 12:42}] doit être soigneusement o. en l'honneur de
\item[\vref{De 4:13}] qu'il vs. ordonna d'o., les dix paroles,
\item[\vref{De 28:58}] prends pas garde d'o. ttes les paroles
\item[\vref{Jos 22:5}] seulement bien garde d'o. les ordonnances et
\item[\vref{Esd 7:10}] de Yahweh, à l'o. et à enseigner
\item[\vref{Ps 119:29}] du mensonge et accorde-moi la grâce d'o. ta loi.
\item[\vref{Jé 11:8}] avais donné l'ordre d'o., et qu'ils n'ont
\item[\vref{Mal 3:14}] qu'avons-ns. gagné à o. ses ordonnances, et
\item[\vref{Mt 5:19}] celui qui les o., et qui enseignera
\item[\vref{Mt 23:3}] qu'ils vs. diront d'o., observez-les et faites-les,
\item[\vref{Ga 4:10}] Vous o. les jours, les mois, les temps
\item[\vref{1 Ti 5:21}] les anges élus, d'o. ces choses sans
\end{listverse}

\ConcordanceEntry{Obtenir}
\vspace{-2mm}
\begin{listverse}
\item[\vref{Mc 1:4}] de repentance, pour o. la rémission des
\item[\vref{Ac 2:38}] de Jésus-Christ, pour o. le pardon de
\item[\vref{1 Co 9:25}] le font pour o. une couronne corruptible ;
\item[\vref{Hé 4:16}] la grâce, afin d'o. miséricorde et de
\item[\vref{Hé 6:18}] notre refuge à o. l'espérance qui ns.
\item[\vref{Hé 11:35}] d'être délivrés, afin d'o. une meilleure résurrection.
\item[\vref{Ja 4:2}] vs. ne pouvez o. ce que vs.
\item[\vref{Jud 1:21}] Seign. Jésus-Christ pour o. la vie éternelle.
\end{listverse}

\ConcordanceEntry{Occasion}
\vspace{-2mm}
\begin{listverse}
\item[\vref{Jg 14:4}] Samson cherchait une o. de dispute de
\item[\vref{Da 6:4}] satrapes cherchèrent une o. d'accuser Daniel en
\item[\vref{Mt 26:16}] il cherchait une o. favorable pour le
\item[\vref{Mc 9:43}] pour toi une o. de chute, coupe-la ;
\item[\vref{Lu 22:6}] il cherchait une o. favorable pour le
\item[\vref{Ro 7:8}] le péché, saisissant l'o., produisit en moi
\item[\vref{Ro 7:11}] le péché saisissant l'o., me séduisit par
\item[\vref{Ro 14:21}] ton frère une o. de chute, ou
\item[\vref{2 Co 5:12}] ns. vs. donnons l'o. de vs. glorifier
\item[\vref{Ga 5:13}] cette liberté une o. de vivre selon
\item[\vref{Ph 4:10}] pensiez bien, mais l'o. vs. manquait.
\item[\vref{1 Ti 5:14}] à l'adversaire aucune o. de médire.
\item[\vref{2 Ti 4:2}] insiste en tte o., favorable ou non.
\end{listverse}

\ConcordanceEntry{Occuper}
\vspace{-2mm}
\begin{listverse}
\item[\vref{1 R 18:27}] ou il est o., ou il est
\item[\vref{Ps 131:1}] hautain ; je ne m'o. pas de choses
\item[\vref{Lu 2:49}] faut que je m'o. des affaires de
\item[\vref{Lu 13:7}] pas. Coupe-le ; pourquoi o.-t-il inutilement la
\item[\vref{Lu 14:9}] aurais honte d'aller o. la dernière place.
\item[\vref{Tit 2:5}] être modérées, pures, o. aux soins domestiques,
\end{listverse}

\ConcordanceEntry{Odeur}
\vspace{-2mm}
\begin{listverse}
\item[\vref{Ge 8:21}] Yahweh respira une o. d'apaisement, et dit
\item[\vref{Ge 27:27}] l'embrassa. Isaac sentit l'o. de ses habits,
\item[\vref{Ex 29:18}] feu d'une agréable o. à Yahweh.
\item[\vref{Lé 2:2}] offrande d'une bonne o. à Yahweh.
\item[\vref{Esd 6:10}] sacrifices de bonne o. au Dieu des
\item[\vref{Ca 1:3}] à cause de l'o. de tes excellents
\item[\vref{Ez 20:41}] parfums d'une agréable o., qnd je vs.
\item[\vref{Da 3:27}] point endommagés, et l'o. du feu n'avait
\item[\vref{2 Co 2:15}] sommes la bonne o. de Christ de
\item[\vref{Ep 5:2}] offrande et un sacrifice de bonne o. à Dieu.
\item[\vref{Ph 4:18}] parfum de bonne o., com. un sacrifice
\end{listverse}

\ConcordanceEntry{Oeil, Yeux}
\vspace{-2mm}
\begin{listverse}
\item[\vref{Ge 3:5}] en mangerez, vos y. seront ouverts, et
\item[\vref{Ge 21:19}] lui ouvrit les y. et elle vit
\item[\vref{Ex 21:24}] œ. pour œil, dent pour dent, main
\item[\vref{De 11:12}] a continuellement ses y., du commencement de
\item[\vref{Jg 16:21}] lui crevèrent les y. ; ils le descendirent
\item[\vref{1 S 16:7}] que voient ses y., mais Yahweh regarde
\item[\vref{2 R 9:30}] fard à ses y., orna sa tête
\item[\vref{2 Ch 7:16}] à toujours ; mes y. et mon cœur
\item[\vref{Job 42:5}] parler de toi ; mais mntnt mon œ. t'a vu.
\item[\vref{Ps 16:8}] Yahweh sous mes y. ; et puisqu'il est
\item[\vref{Ps 94:9}] qui a formé l'œ., ne verrait-il point ?
\item[\vref{Ps 115:5}] elles ont des y. et ne voient
\item[\vref{Pr 17:24}] prudent ; mais les y. du fou sont
\item[\vref{Ec 1:8}] peut en parler ; l'œ. n'est jamais rassasié
\item[\vref{Es 43:8}] qui a des y., et les sourds
\item[\vref{Da 10:6}] com. l'éclair, ses y. étaient com. des
\item[\vref{Za 4:10}] sept sont les y. de Yahweh qui
\item[\vref{Mt 5:29}] Si ton œ. droit est pour toi une occasion
\item[\vref{Mt 6:22}] L'œ. est la lampe du corps. Si
\item[\vref{Lu 24:16}] Mais leurs y. étaient retenus de
\item[\vref{Ro 1:20}] voient com. à l'œ. nu, depuis la
\item[\vref{1 Co 2:9}] des choses que l'œ. n'a pas vues,
\item[\vref{1 Co 12:21}] Et l'œ. ne peut pas dire à la
\item[\vref{Ga 3:1}] vérité, vs., aux y. de qui Jésus-Christ
\item[\vref{Ep 1:18}] Qu'il illumine les y. de votre esprit,
\item[\vref{Hé 4:13}] entièrement découvertes aux y. de celui dvt
\item[\vref{Hé 12:2}] portant les y. sur Jésus, le
\item[\vref{2 Pi 2:14}] Ils ont les y. pleins d'adultère, ils
\item[\vref{1 Jn 2:11}] car les ténèbres ont aveuglé ses y.
\item[\vref{Ap 1:7}] nuées, et tt œ. le verra, et
\item[\vref{Ap 5:6}] cornes, et sept y., qui sont les
\item[\vref{Ap 7:17}] essuiera ttes les larmes de leurs y.
\item[\vref{Ap 19:12}] Et ses y. étaient com. une flamme de feu ;
\end{listverse}

\ConcordanceEntry{Oeuvre}
\vspace{-2mm}
\begin{listverse}
\item[\vref{Ex 23:24}] point selon leurs œ., mais tu les
\item[\vref{Jos 24:31}] connu ttes les œ. que Yahweh avait
\item[\vref{Esd 9:13}] de nos mauvaises œ. et du grand
\item[\vref{Ps 46:9}] Venez, contemplez les œ. de Yahweh, et
\item[\vref{Ps 77:13}] méditerai ttes tes œ., et je parlerai
\item[\vref{Ps 92:6}] Yahweh ! que tes œ. sont magnifiques ! Tes
\item[\vref{Ps 95:9}] et éprouvé bien qu'ils virent mes œ.
\item[\vref{Ps 106:13}] oublièrent vite ses œ., et ne s'attendirent
\item[\vref{Ps 111:7}] [Mem.] Les œ. de ses mains
\item[\vref{Ps 118:17}] je raconterai les œ. de Yahweh.
\item[\vref{Ps 139:14}] admirable manière ; tes œ. sont merveilleuses, et
\item[\vref{Pr 31:31}] et que ses œ. la louent aux
\item[\vref{Es 66:18}] moi, voyant leurs œ. et leurs pensées,
\item[\vref{Mt 11:2}] sa prison des œ. du Christ, envoya
\item[\vref{Lu 24:19}] prophète puissant en œ. et en paroles
\item[\vref{Jn 14:12}] moi fera les œ. que je fais,
\item[\vref{Jn 15:24}] parmi eux les œ. qu'aucun autre n'a
\item[\vref{Ac 7:22}] était puissant en paroles et en œ.
\item[\vref{Ro 13:12}] Rejetons dc les œ. des ténèbres, et
\item[\vref{1 Co 9:1}] N'êtes-vs. pas mon œ. ds le Seign. ?
\item[\vref{Ga 5:19}] Car les œ. de la chair sont évidentes : Ce
\item[\vref{Ep 2:10}] pour les bonnes œ. que Dieu a
\item[\vref{Ep 5:11}] participez pas aux œ. infructueuses des ténèbres,
\item[\vref{Hé 6:1}] la repentance des œ. mortes, et de
\item[\vref{Ja 2:26}] foi sans les œ. est morte.
\item[\vref{1 Pi 2:12}] remarquent vos bonnes œ. et glorifient Dieu,
\item[\vref{Ap 3:15}] Je connais tes œ.. Je sais que
\item[\vref{Ap 15:3}] en disant : Tes œ. sont grandes et
\item[\vref{Ap 20:13}] ils furent jugés chacun selon ses œ.
\end{listverse}

\ConcordanceEntry{Offense}
\vspace{-2mm}
\begin{listverse}
\item[\vref{Mt 6:14}] aux hommes leurs o., votre Père céleste
\item[\vref{Ro 4:25}] livré pour nos o., et est ressuscité
\item[\vref{Ro 5:15}] grâce com. de l'o. ; car, si par
\item[\vref{Ro 5:16}] après une seule o. que le jugement
\item[\vref{Ro 5:17}] Car si par l'o. d'un seul, la
\item[\vref{Ro 5:20}] intervenue afin que l'o. abonde, mais là
\item[\vref{Ep 2:5}] morts ds nos o., il ns. a
\item[\vref{Col 2:13}] morts ds vos o., et ds l'incirconcision
\end{listverse}

\ConcordanceEntry{Offrande}
\vspace{-2mm}
\begin{listverse}
\item[\vref{Ge 4:3}] à Yahweh une o. des fruits de
\item[\vref{Ex 36:3}] dvt Moïse tte l'o. que les enfants
\item[\vref{Lé 2:1}] Lorsque quelqu'un offrira l'o. de gâteau à
\item[\vref{Ps 20:4}] de ttes tes o., qu'il réduise en
\item[\vref{Ps 119:108}] prie, agrée les o. volontaires de ma
\item[\vref{Es 1:13}] m'apporter de vaines o. : Le parfum m'est
\item[\vref{Jé 14:12}] holocaustes et des o., je n'y prendrai
\item[\vref{Mal 1:11}] Nom, et des o. pures ; car mon
\item[\vref{Mal 3:8}] ds les dîmes et ds les o.
\item[\vref{Mt 5:24}] laisse là ton o. dvt l'autel, et
\item[\vref{Lu 21:1}] qui mettaient leurs o. ds le tronc.
\item[\vref{Ep 5:2}] ns. com. une o. et un sacrifice
\item[\vref{Hé 10:10}] à savoir par l'o. du corps de
\item[\vref{Hé 10:18}] n'y a plus d'o. pour le péché.
\end{listverse}

\ConcordanceEntry{Offrir}
\vspace{-2mm}
\begin{listverse}
\item[\vref{Ge 22:2}] Morija ; et là, o.-le en holoc.
\item[\vref{1 S 13:10}] Comme il achevait d'o. l'holoc., Samuel arriva,
\item[\vref{1 Ch 29:14}] assez pour pouvoir t'o. ces choses volontairement ?
\item[\vref{Ro 12:1}] de Dieu, à o. vos corps en
\item[\vref{Hé 9:25}] n'est pas pour s'o. lui-mm plusieurs fois
\item[\vref{Hé 9:28}] Christ, qui s'est o. une seule fois
\item[\vref{1 Pi 2:5}] sainte prêtrise, afin d'o. des sacrifices spirituels,
\item[\vref{Ap 8:3}] donnés pour les o., avec les prières
\end{listverse}

\ConcordanceEntry{Og}
\vspace{-2mm}
\begin{listverse}
\item[\vref{No 21:33}] chemin de Basan. O., roi de Basan,
\item[\vref{De 3:1}] de Basan. Et O., roi de Basan,
\end{listverse}

\ConcordanceEntry{Oindre}
\vspace{-2mm}
\begin{listverse}
\item[\vref{Ex 28:41}] lui ; tu les o., tu les consacreras
\item[\vref{Jg 9:8}] arbres allèrent pour o. un roi et
\item[\vref{1 S 15:1}] m'a envoyé pour t'o. afin que tu
\item[\vref{2 S 1:21}] Saül ; l'huile a cessé de les o.
\item[\vref{Ps 45:8}] ton Dieu t'a o. d'une huile de
\item[\vref{Ps 92:11}] buffle ; je serai o. d'une huile fraîche.
\item[\vref{Es 61:1}] car Yahweh m'a o. pour évangéliser les
\item[\vref{Da 9:24}] prophétie et pour o. le Saint des
\item[\vref{Lu 4:18}] parce qu'il m'a o. pour évangéliser les
\item[\vref{Jn 12:3}] grand prix, en o. les pieds de
\item[\vref{Ac 10:38}] comment Dieu a o. du Saint-Esprit et
\item[\vref{Hé 1:9}] ton Dieu t'a o. d'une huile de
\item[\vref{Ap 3:18}] paraisse pas ; et d'o. tes yeux de
\end{listverse}

\ConcordanceEntry{Oint}
\vspace{-2mm}
\begin{listverse}
\item[\vref{1 S 26:9}] sa main sur l'o. de Yahweh ?
\item[\vref{2 S 1:14}] main pour tuer l'o. de Yahweh ?
\item[\vref{Ps 18:51}] miséricorde à son o., à David, et
\item[\vref{Ps 20:7}] Yahweh sauve son O. ; il l'exaucera des
\item[\vref{Ps 28:8}] le refuge des délivrances de son o.
\item[\vref{Ps 89:52}] outrages contre les pas de ton o.
\item[\vref{Ps 105:15}] pas à mes o., et ne faites
\item[\vref{Es 45:1}] Yahweh à son o., à Cyrus,
\item[\vref{Ez 28:14}] étais un chérubin, o. pour servir de
\item[\vref{Ha 3:13}] sors avec ton O. pour la délivrance ;
\item[\vref{Za 4:14}] les deux fils o., qui se tiennent
\end{listverse}

\ConcordanceEntry{Oiseau}
\vspace{-2mm}
\begin{listverse}
\item[\vref{Ge 1:21}] créa aussi tt o. ayant des ailes
\item[\vref{Ge 8:20}] et de tt o. pur, et il
\item[\vref{Lé 14:6}] Puis il prendra l'o. vivant, le bois
\item[\vref{Ps 124:7}] s'est échappée com. l'o. du filet des
\item[\vref{Pr 6:5}] chasseur, et com. l'o. de la main
\item[\vref{Pr 7:23}] le foie ; com. l'o. qui se hâte
\item[\vref{Pr 26:2}] Comme l'o. est prompt à s'échapper et l'hirondelle
\item[\vref{Ec 10:20}] tu couches ; car l'o. du ciel emporterait
\item[\vref{Ec 12:6}] au chant de l'o., et que ttes
\item[\vref{Mt 6:26}] Considérez les o. du ciel ; car
\item[\vref{Mt 8:20}] tanières, et les o. du ciel ont
\item[\vref{Mc 4:4}] chemin, et les o. du ciel vinrent,
\item[\vref{Ac 10:12}] bêtes sauvages, les reptiles et les o. du ciel.
\item[\vref{Ro 1:23}] corruptible, et des o., et des quadrupèdes,
\item[\vref{Ap 18:2}] repaire de tt o. impur et détestable.
\item[\vref{Ap 19:17}] à ts les o. qui volaient au
\end{listverse}

\ConcordanceEntry{Olivier (un)}
\vspace{-2mm}
\begin{listverse}
\item[\vref{Ge 8:11}] bec une feuille d'o. qu'elle avait arrachée ;
\item[\vref{Jg 9:8}] ils dirent à l'o. : Règne sur ns.
\item[\vref{Ps 52:10}] Dieu com. un o. verdoyant. Je me
\item[\vref{Es 17:6}] qnd on secoue l'o., et qu'il reste
\item[\vref{Os 14:6}] com. celle de l'o., avec un parfum
\item[\vref{Za 4:12}] ces deux branches d'o. qui sont près
\item[\vref{Ro 11:17}] qui étais un o. sauvage, tu as
\item[\vref{Ro 11:24}] été coupé de l'o. sauvage selon sa
\end{listverse}

\ConcordanceEntry{Ombre}
\vspace{-2mm}
\begin{listverse}
\item[\vref{2 R 20:9}] qu'il a prononcée : L'o. s'avancera-t-elle de dix
\item[\vref{Ps 23:4}] la vallée de l'o. de la mort,
\item[\vref{Ps 39:7}] promène com. une o., certainement on s'agite
\item[\vref{Ps 91:1}] Très-Haut, repose à l'o. du Tout-Puissant.
\item[\vref{Ps 102:12}] jours sont com. l'o. qui décline, et
\item[\vref{Ps 107:10}] ténèbres et ds l'o. de la mort,
\item[\vref{Ps 121:5}] Yahweh est ton o. à ta main
\item[\vref{Ec 7:12}] à couvert à l'o. de la sagesse,
\item[\vref{Es 4:6}] pour donner de l'o. contre la chaleur
\item[\vref{Es 49:2}] m'a caché ds l'o. de sa main,
\item[\vref{Os 14:7}] s'asseoir à son o., et ils redonneront
\item[\vref{Jon 4:5}] y resta à l'o., jusqu'à ce qu'il
\item[\vref{Lu 1:35}] couvrira de son o.. C'est pourquoi, le
\item[\vref{Ac 5:15}] au moins son o. passe sur quelqu'un
\item[\vref{Col 2:17}] qui sont l'o. des choses qui
\item[\vref{Hé 8:5}] que l'image et l'o. des choses célestes,
\item[\vref{Hé 10:1}] loi qui possède l'o. des biens à
\item[\vref{Ja 1:17}] ni changement ni o. de variation.
\end{listverse}

\ConcordanceEntry{Onction}
\vspace{-2mm}
\begin{listverse}
\item[\vref{Ex 25:6}] aromates pour l'huile d'o. et pour le
\item[\vref{Ex 30:25}] de l'huile pour l'o. sainte, un onguent
\item[\vref{Ex 40:15}] prêtrise, et lr. o. lr. sera pour
\item[\vref{No 18:8}] par ordon. perpétuelle, à cause de l'o.
\item[\vref{1 Jn 2:27}] Mais l'o. que vs. avez reçue de lui
\end{listverse}

\ConcordanceEntry{Onésime}
\vspace{-2mm}
\begin{listverse}
\item[\vref{Col 4:9}] avec O., notre fidèle et bien-aimé frère, qui
\item[\vref{Phm 1:10}] pour mon fils O., que j'ai engendré
\end{listverse}

\ConcordanceEntry{Onésiphore}
\vspace{-2mm}
\begin{listverse}
\item[\vref{2 Ti 1:16}] à la maison d'O., car il m'a
\item[\vref{2 Ti 4:19}] Priscille et Aquilas, et la famille d'O.
\end{listverse}

\ConcordanceEntry{Onyx}
\vspace{-2mm}
\begin{listverse}
\item[\vref{Ge 2:12}] trouvent le bdellium et la pierre d'o.
\item[\vref{Ex 28:20}] un chrysolithe, un o. et un béryl,
\item[\vref{Ez 28:13}] diamant, de chrysolithe, d'o., de jaspe, de
\end{listverse}

\ConcordanceEntry{Opérer}
\vspace{-2mm}
\begin{listverse}
\item[\vref{1 S 12:16}] que Yahweh va o. sous vos yeux.
\item[\vref{1 S 19:5}] et Yahweh a o. une grande délivrance
\item[\vref{2 S 23:12}] Philistins. Et Yahweh o. une grande délivrance.
\item[\vref{Jé 8:22}] mon peuple ne s'o.-t-elle pas ?
\item[\vref{Da 4:2}] Dieu Très-Haut a o. à mon égard.
\item[\vref{Lu 13:14}] que Jésus avait o. cette guérison un
\item[\vref{Jn 6:2}] les miracles qu'il o. sur les malades.
\item[\vref{1 Co 12:6}] mm Dieu qui o. ttes choses en
\item[\vref{Ap 13:14}] lui était donné d'o. en présence de
\end{listverse}

\ConcordanceEntry{Opinion}
\vspace{-2mm}
\begin{listverse}
\item[\vref{Pr 12:15}] droite à son o., mais celui qui
\end{listverse}

\ConcordanceEntry{Opposer}
\vspace{-2mm}
\begin{listverse}
\item[\vref{No 22:32}] suis sorti pour m'o. à toi ; car
\item[\vref{Ac 11:17}] cru au Seign. Jésus-Christ, pouvais-je, moi, m'o. à Dieu ?
\item[\vref{Ac 13:45}] jalousie, et ils s'o. à ce que
\item[\vref{Ga 5:17}] et ils sont o. entre eux, afin
\item[\vref{2 Ti 4:15}] il s'est fortement o. à nos paroles.
\end{listverse}

\ConcordanceEntry{Opposition}
\vspace{-2mm}
\begin{listverse}
\item[\vref{Hé 12:3}] personne une telle o. de la part
\end{listverse}

\ConcordanceEntry{Oppresseur}
\vspace{-2mm}
\begin{listverse}
\item[\vref{Ex 3:7}] cause de leurs o., car je connais
\item[\vref{Est 7:6}] Esther répondit : L'o. et l'ennemi est
\item[\vref{Ps 107:2}] a rachetés de la main de l'o.,
\item[\vref{Pr 29:13}] Le pauvre et l'o. se rencontrent, c'est
\item[\vref{Es 3:12}] il a pour o. des enfants, et
\item[\vref{Za 9:8}] les venants, et l'o. ne passera plus
\end{listverse}

\ConcordanceEntry{Oppression}
\vspace{-2mm}
\begin{listverse}
\item[\vref{Ex 3:9}] j'ai vu aussi l'o. dont les Egyptiens
\item[\vref{De 26:7}] notre souffrance, notre travail, et notre o.
\item[\vref{Ps 119:134}] Délivre-moi de l'o. des hommes afin
\item[\vref{Ec 7:7}] Certainement l'o. fait perdre le
\item[\vref{Es 30:12}] vs. confiez ds l'o. et ds les
\item[\vref{Es 54:14}] seras loin de l'o., et tu ne
\item[\vref{Es 59:13}] Dieu, de proférer l'o. et la révolte,
\item[\vref{Ro 8:35}] de Christ ? Sera-ce l'o., ou l'angoisse, ou
\end{listverse}

\ConcordanceEntry{Opprimé}
\vspace{-2mm}
\begin{listverse}
\item[\vref{Ps 9:10}] un refuge pour l'o., un refuge au
\item[\vref{Ps 10:18}] l'orphelin et à l'o., afin que l'hom.
\item[\vref{Ps 119:122}] que je sois o. par les orgueilleux.
\item[\vref{Es 38:14}] Yahweh, je suis o., sois mon garant !
\item[\vref{Es 53:7}] O. et humilié, il n'a point ouvert
\item[\vref{Es 58:6}] aller libres les o., et que l'on
\item[\vref{Lu 4:19}] en liberté les o., pour publier une
\end{listverse}

\ConcordanceEntry{Opprimer}
\vspace{-2mm}
\begin{listverse}
\item[\vref{Ge 15:13}] aux habitants du pays qui les o.
\item[\vref{Ex 22:21}] ne fouleras ni n'o. point l'étranger ; car
\item[\vref{Jg 2:18}] ceux qui les o. et les tourmentaient.
\item[\vref{Ps 73:8}] et parlent méchamment d'o. ; ils parlent d'une
\item[\vref{Ps 105:14}] personne de les o., et il châtia
\item[\vref{Ps 129:2}] Ils m'ont assez o. dès ma jeunesse,
\item[\vref{Pr 22:22}] est pauvre ; et n'o. pas le malheureux
\item[\vref{Ja 2:6}] riches ne vs. o.-ils pas, et
\end{listverse}

\ConcordanceEntry{Opprobre}
\vspace{-2mm}
\begin{listverse}
\item[\vref{Jos 5:9}] de dessus vs. l'o. de l'Egypte. Et
\item[\vref{Ps 22:7}] non un hom., l'o. des hommes et
\item[\vref{Ps 69:8}] toi, j'ai souffert l'o., la honte a
\item[\vref{Ps 89:51}] Seign., souviens-toi de l'o. de tes serviteurs,
\item[\vref{Ps 119:22}] Décharge-moi de l'o. et du mépris,
\item[\vref{Ps 119:39}] Eloigne de moi l'o. que je redoute,
\item[\vref{Pr 25:10}] n'en reçoives un o. qui ne s'efface
\item[\vref{Es 25:8}] et il ôte l'o. de son peuple
\item[\vref{Es 51:7}] Ne craignez point l'o. des hommes et
\item[\vref{Jé 6:10}] lr. est en o., ils n'y prennent
\item[\vref{Jé 23:40}] sur vs. un o. éternel et une
\item[\vref{Jé 31:19}] car je porte l'o. de ma jeunesse.
\item[\vref{Mi 6:16}] et vs. porterez l'o. de mon peuple.
\item[\vref{Lu 1:25}] pour ôter mon o. d'entre les hommes.
\item[\vref{1 Ti 3:7}] pas tomber ds l'o. et ds les
\item[\vref{Hé 11:26}] ayant estimé que l'o. de Christ était
\item[\vref{Hé 13:13}] hors du camp, en portant son o.
\end{listverse}

\ConcordanceEntry{Or}
\vspace{-2mm}
\begin{listverse}
\item[\vref{Ex 25:11}] tu la couvriras d'o. pur, tu l'en
\item[\vref{Job 22:24}] Jette l'o. ds la poussière, et l'or d'ophir
\item[\vref{Job 31:24}] mon espérance en l'o., et si j'ai
\item[\vref{Ps 19:11}] plus précieux que l'o., que beaucoup d'or
\item[\vref{Ps 119:127}] commandements, plus que l'o. et l'or fin.
\item[\vref{Pr 11:22}] com. un anneau d'o. au groin d'un
\item[\vref{Pr 25:12}] com. une bague d'o. ou com. un
\item[\vref{Da 2:38}] c'est toi qui es la tête d'o.
\item[\vref{Ag 2:8}] à moi, et l'o. est à moi,
\item[\vref{Mt 10:9}] Ne prenez ni o., ni argent, ni
\item[\vref{Ac 3:6}] ni argent, ni o. ; mais ce que
\item[\vref{1 Co 3:12}] fondement avec de l'o., de l'argent, des
\item[\vref{Ja 5:3}] votre o. et votre argent sont rouillés ; et
\item[\vref{1 Pi 1:18}] des choses corruptibles, com. l'argent ou l'o.,
\item[\vref{Ap 1:12}] m'être retourné, je vis sept chandeliers d'o.,
\item[\vref{Ap 3:18}] de moi de l'o. éprouvé par le
\end{listverse}

\ConcordanceEntry{Oracle}
\vspace{-2mm}
\begin{listverse}
\item[\vref{La 2:14}] t'ont prophétisé des o. mensongers et trompeurs.
\item[\vref{Ez 22:28}] fausses, et des o. menteurs, en disant :
\item[\vref{Ro 3:2}] ce que les o. de Dieu lr.
\item[\vref{Hé 5:12}] premiers rudiments des o. de Dieu ; et
\end{listverse}

\ConcordanceEntry{Ordonnance}
\vspace{-2mm}
\begin{listverse}
\item[\vref{Ex 15:25}] proposa là une o. et une loi,
\item[\vref{De 4:8}] lois et des o. justes, com. tte
\item[\vref{2 Ch 7:17}] tu gardes mes lois et mes o.,
\item[\vref{Ps 19:9}] Les o. de Yahweh sont droites, elles réjouissent
\item[\vref{Ps 105:10}] pour être une o. à Jacob, et
\item[\vref{Ps 119:149}] ô Yahweh ! Fais-moi revivre selon ton o.
\item[\vref{Pr 31:5}] bu, ils n'oublient l'o., et qu'ils n'altèrent
\item[\vref{Jé 5:22}] mer, par une o. perpétuelle et qui
\item[\vref{Da 6:7}] d'avis d'établir une o. royale et un
\item[\vref{Da 9:5}] de tes commandements et de tes o.
\item[\vref{Lu 1:6}] ds ttes les o. du Seign., sans
\item[\vref{Ac 7:53}] loi par une o. des anges, et
\item[\vref{Ro 9:4}] gloire, les alliances, l'o. de la loi,
\item[\vref{Ep 2:15}] qui consiste en o., afin de créer
\item[\vref{Col 2:14}] consistait en des o., et qui ns.
\item[\vref{Hé 9:1}] avait aussi des o. concernant le service
\end{listverse}

\ConcordanceEntry{Ordonner}
\vspace{-2mm}
\begin{listverse}
\item[\vref{Ge 6:22}] Dieu lui avait o. ; il le fit
\item[\vref{Ex 34:34}] d'Israël ce qui lui avait été o.
\item[\vref{De 32:46}] vs. conjure aujourd'hui d'o. à vos fils,
\item[\vref{Jos 4:16}] O. aux prêtres qui portent l'arche du
\item[\vref{Ru 3:6}] ce que sa belle-mère lui avait o.
\item[\vref{1 S 2:29}] offrandes, que j'ai o. de faire ds
\item[\vref{Da 4:6}] J'o. qu'on fasse venir dvt moi ts
\item[\vref{Mt 4:3}] Fils de Dieu, o. que ces pierres
\item[\vref{Mt 14:28}] si c'est toi, o. que j'aille vers
\item[\vref{Lu 8:31}] ne pas lr. o. d'aller ds l'abîme.
\item[\vref{Jn 8:5}] Moïse ns. a o. ds la loi
\item[\vref{Ac 13:47}] ainsi ns. l'a o. le Seign. : Je
\item[\vref{1 Co 7:10}] mariés, je lr. o., non pas moi,
\item[\vref{1 Ti 1:3}] te prie encore d'o. à certaines personnes
\item[\vref{Phm 1:8}] en Christ de t'o. ce qui est
\end{listverse}

\ConcordanceEntry{Ordre}
\vspace{-2mm}
\begin{listverse}
\item[\vref{No 22:18}] pourrais point transgresser l'o. de Yahweh, mon
\item[\vref{Ac 25:23}] fut amené sur l'o. de Festus.
\item[\vref{Ro 13:2}] l'autorité résiste à l'o. de Dieu, et
\item[\vref{Hé 6:20}] Grand-Prêtre éternellement, selon l'o. de Melchisédek.
\item[\vref{Hé 11:23}] beau, et ils ne craignirent pas l'o. du roi.
\end{listverse}

\ConcordanceEntry{Oreille}
\vspace{-2mm}
\begin{listverse}
\item[\vref{Ge 35:4}] étaient à leurs o., et il les
\item[\vref{Ex 21:6}] maître lui percera l'o. avec un poinçon ;
\item[\vref{Né 1:11}] Seign., que ton o. soit mntnt attentive
\item[\vref{Job 12:11}] L'o. ne discerne-t-elle pas les discours, ainsi
\item[\vref{Job 33:16}] Alors il ouvre l'o. aux hommes, et
\item[\vref{Ps 5:2}] Prête l'o. à mes paroles, ô Yahweh ! Ecoute
\item[\vref{Ps 10:17}] lr. cœur ; ton o. les écoute attentivement
\item[\vref{Ps 17:6}] Dieu ! Incline ton o. vers moi, écoute
\item[\vref{Ps 94:9}] qui a planté l'o., n'entendrait-il point ? Celui
\item[\vref{Pr 2:2}] tu rends ton o. attentive à la
\item[\vref{Pr 4:20}] paroles, incline ton o. à mes discours.
\item[\vref{Pr 15:31}] L'o. qui écoute la correction qui donne
\item[\vref{Pr 20:12}] L'o. qui entend et l'œil qui voit,
\item[\vref{Pr 21:13}] qui bouche son o. pour ne pas
\item[\vref{Pr 28:9}] qui détourne son o. pour ne pas
\item[\vref{Es 1:10}] de Sodome, prêtez l'o. à la loi
\item[\vref{Es 59:1}] sauver, ni son o. trop pesante pour
\item[\vref{Jé 2:2}] et crie aux o. de Jérus., et
\item[\vref{Mc 7:35}] Aussitôt ses o. s'ouvrirent, et le
\item[\vref{Jn 18:10}] et lui coupa l'o. droite. Ce serviteur
\item[\vref{Ac 7:51}] de cœur et d'o., vs. vs. obstinez
\item[\vref{1 Co 2:9}] pas vues, que l'o. n'a pas entendues,
\item[\vref{1 Co 12:16}] Et si l'o. dit : Parce que je ne suis
\item[\vref{1 Pi 3:12}] justes, et ses o. sont attentives à
\end{listverse}

\ConcordanceEntry{Orfèvre}
\vspace{-2mm}
\begin{listverse}
\item[\vref{Pr 25:4}] il en sortira un vase pour l'o.
\item[\vref{Es 40:19}] fond l'image, et l'o. la couvre d'or,
\item[\vref{Es 46:6}] ils engagent un o. pour en faire
\item[\vref{Os 8:6}] d'Israël, c'est un o. qui l'a fait,
\item[\vref{Ac 19:24}] hom., nommé Démétrius, o., fabriquait de petits
\end{listverse}

\ConcordanceEntry{Orgueil}
\vspace{-2mm}
\begin{listverse}
\item[\vref{Lé 26:19}] Je briserai l'o. de votre force
\item[\vref{Ps 90:10}] à quatre-vingts ans ; l'o. qu'ils en tirent
\item[\vref{Pr 11:2}] Quand l'o. vient, la honte vient aussi ; mais
\item[\vref{Pr 13:10}] L'o. ne produit que querelle, mais la
\item[\vref{Pr 16:18}] L'o. va dvt l'écrasement, et la fierté
\item[\vref{Pr 29:23}] L'o. de l'hom. l'abaisse, mais celui qui
\item[\vref{Jé 49:16}] ta présomption et l'o. de ton cœur
\item[\vref{Os 5:5}] Aussi l'o. d'Israël témoignera contre lui ; Israël et
\item[\vref{Ro 11:20}] dc pas par o., mais crains ;
\item[\vref{1 Co 5:2}] vs. êtes enflés d'o. ! Et vs. n'avez
\item[\vref{1 Co 13:4}] pas d'insolence, elle ne s'enfle pas d'o.,
\item[\vref{2 Co 12:20}] des rapports, de l'o. et des séditions.
\item[\vref{1 Ti 3:6}] de peur qu'enflé d'o., il ne tombe
\item[\vref{1 Ti 6:4}] il est enflé d'o., il ne sait
\item[\vref{2 Ti 3:4}] traîtres, emportés, enflés d'o., amis des voluptés
\item[\vref{1 Jn 2:16}] des yeux et l'o. de la vie,
\item[\vref{Ap 13:5}] des discours pleins d'o., et des blasphèmes ;
\end{listverse}

\ConcordanceEntry{Orgueilleux}
\vspace{-2mm}
\begin{listverse}
\item[\vref{2 S 22:28}] de ton regard, tu abaisses les o.
\item[\vref{Ps 31:24}] fidèles, et il punit sévèrement les o.
\item[\vref{Ps 94:2}] élève-toi ! Rends aux o. selon leurs œuvres !
\item[\vref{Ps 138:6}] et il reconnaît de loin les o.
\item[\vref{Pr 15:25}] la maison des o., mais il affermit
\item[\vref{Lu 1:51}] desseins que les o. formaient ds leurs
\item[\vref{Ro 1:30}] haïssant Dieu, outrageux, o., vains, inventeurs de
\item[\vref{2 Ti 3:2}] de l'argent, fanfarons, o., blasphémateurs, rebelles à
\item[\vref{Ja 4:6}] Dieu résiste aux o., mais il fait
\end{listverse}

\ConcordanceEntry{Orient}
\vspace{-2mm}
\begin{listverse}
\item[\vref{Ge 13:14}] nord, le midi, l'o., et l'occident.
\item[\vref{Lé 16:14}] du propitiatoire vers l'o. ; il fera l'aspersion
\item[\vref{No 3:38}] le tabernacle, à l'o., dvt la tente
\item[\vref{Ps 103:12}] transgressions, autant que l'o. est éloigné de
\item[\vref{Ps 107:3}] les pays, de l'o. et de l'occident,
\item[\vref{Es 2:6}] se sont remplis d'o. et adonnés à
\item[\vref{Ez 8:16}] visages tournés vers l'o. ; et ils se
\item[\vref{Mt 8:11}] plusieurs viendront de l'o. et de l'occident,
\item[\vref{Mt 24:27}] l'éclair part de l'o. et se montre
\item[\vref{Ap 21:13}] A l'o., trois portes, au nord, trois portes,
\end{listverse}

\ConcordanceEntry{Ornement}
\vspace{-2mm}
\begin{listverse}
\item[\vref{Ex 28:2}] vêtements pour la gloire et pour l'o.
\item[\vref{1 Ch 16:29}] lui. Prosternez-vs. dvt Yahweh avec des o. saints !
\item[\vref{Ps 29:2}] Nom ! Prosternez-vs. dvt Yahweh avec des o. sacrés !
\item[\vref{Ps 96:9}] Yahweh avec des o. sacrés. Tremblez dvt
\item[\vref{Pr 3:22}] ton âme et l'o. de ton cou.
\item[\vref{Es 13:19}] Ainsi Babylone, l'o. des royaumes, la
\item[\vref{Jé 2:32}] vierge oublie-t-elle ses o., l'épouse sa ceinture ?
\item[\vref{Ez 23:26}] et t'enlèveront les o. dont tu te
\item[\vref{Ez 23:40}] fardé ton visage, et t'es parée d'o.
\item[\vref{Da 11:20}] un exacteur ds l'o. du royaume, et
\item[\vref{1 Pi 3:3}] cheveux tressés, les o. d'or ou la
\end{listverse}

\ConcordanceEntry{Orphelin}
\vspace{-2mm}
\begin{listverse}
\item[\vref{De 10:18}] fait justice à l'o. et à la
\item[\vref{De 24:17}] l'étranger ni à l'o., et tu ne
\item[\vref{Ps 10:14}] malheureux, tu es le secours de l'o.
\item[\vref{Ps 10:18}] rendre justice à l'o. et à l'opprimé,
\item[\vref{Ps 146:9}] étrangers, il soutient l'o. et la veuve,
\item[\vref{Es 1:17}] faites justice à l'o., défendez la cause
\item[\vref{Jé 22:3}] ne maltraitez pas l'o., ni l'étranger, ni
\item[\vref{Os 14:3}] de toi que l'o. trouve de la
\end{listverse}

\ConcordanceEntry{Os, Ossements}
\vspace{-2mm}
\begin{listverse}
\item[\vref{Ge 2:23}] celle qui est o. de mes os
\item[\vref{Ge 50:25}] visiter, et alors vs. transporterez mes o. d'ici.
\item[\vref{Ex 12:46}] maison, et vs. n'en briserez aucun o.
\item[\vref{No 19:16}] mort, ou des o. humains, ou un
\item[\vref{2 R 13:21}] alla toucher les o. d'Elisée, il reprit
\item[\vref{Ps 22:18}] compter ts mes o. un par un.
\item[\vref{Ps 32:3}] suis tu, mes o. se sont consumés,
\item[\vref{Ps 51:10}] l'allégresse, et les o. que tu as
\item[\vref{Ec 11:5}] se forment les o. ds le ventre
\item[\vref{Ez 37:4}] Prophétise sur ces o., et dis-lr. : Ossements
\item[\vref{Mt 23:27}] au-dedans sont pleins d'o. de morts, et
\item[\vref{Lu 24:39}] ni chair ni o., com. vs. voyez
\item[\vref{Jn 19:36}] Aucun de ses o. ne sera brisé.
\item[\vref{Ep 5:30}] de sa chair et de ses o.
\item[\vref{Hé 11:22}] des ordres au sujet de ses o.
\end{listverse}

\ConcordanceEntry{Osée}
\vspace{-2mm}
\begin{listverse}
\item[\vref{2 R 17:1}] roi de Juda, O., fils d'Ela, régna
\item[\vref{Os 1:1}] qui vint à O., fils de Béeri,
\end{listverse}

\ConcordanceEntry{Othniel}
\vspace{-2mm}
\begin{listverse}
\item[\vref{Jg 3:9}] qui les délivra, O., fils de Kenaz,
\end{listverse}

\ConcordanceEntry{Oublier}
\vspace{-2mm}
\begin{listverse}
\item[\vref{Ge 41:51}] Dieu m'a fait o. tte ma peine
\item[\vref{De 4:23}] Gardez-vs. d'o. l'alliance de Yahweh,
\item[\vref{De 8:19}] Mais si tu o. Yahweh, ton Dieu,
\item[\vref{Jg 3:7}] de Yahweh, ils o. Yahweh et servirent
\item[\vref{1 S 1:11}] de moi, et n'o. pas ta servante,
\item[\vref{Ps 44:18}] ne t'avons point o., et ns. n'avons
\item[\vref{Ps 78:7}] Dieu, et qu'ils n'o. point les œuvres
\item[\vref{Ps 103:2}] bénis Yahweh, et n'o. pas un de
\item[\vref{Ps 119:93}] Je n'o. jamais tes commandements car c'est par
\item[\vref{Pr 2:17}] jeunesse et qui o. l'alliance de son
\item[\vref{Es 49:15}] Une fem. peut-elle o. son enfant qu'elle
\item[\vref{Jé 2:32}] La vierge o.-t-elle ses ornements,
\item[\vref{Jé 23:27}] comment ils feront o. mon Nom à
\item[\vref{La 2:6}] Yahweh a fait o. ds Sion la
\item[\vref{Os 4:6}] puisque tu as o. la loi de
\item[\vref{Lu 12:6}] aucun d'eux n'est o. dvt Dieu.
\item[\vref{Ph 3:14}] fais une chose : O. les choses qui
\item[\vref{Hé 6:10}] pas injuste, pour o. votre œuvre, et
\item[\vref{Hé 13:2}] N'o. pas l'hospitalité ; car, par elle, quelques-uns
\item[\vref{Ja 1:24}] s'en va, et o. aussitôt comment il
\end{listverse}

\ConcordanceEntry{Oui}
\vspace{-2mm}
\begin{listverse}
\item[\vref{Mt 5:37}] votre parole soit : O., oui, non, non ;
\item[\vref{2 Co 1:17}] en moi le o. et le non ?
\item[\vref{Ja 5:12}] Mais que votre o. soit oui, et
\item[\vref{Ap 22:20}] tte vitesse. Amen ! O., Seign. Jésus, viens !
\end{listverse}

\ConcordanceEntry{Ours}
\vspace{-2mm}
\begin{listverse}
\item[\vref{1 S 17:37}] la patte de l'o., me délivrera de
\item[\vref{2 R 2:24}] Yahweh. Alors deux o. sortirent de la
\item[\vref{Da 7:5}] semblable à un o., et se tenait
\item[\vref{Ap 13:2}] com. ceux d'un o. ; sa gueule était
\end{listverse}

\ConcordanceEntry{Outrage}
\vspace{-2mm}
\begin{listverse}
\item[\vref{Ge 16:5}] dit à Abram : L'o. qui m'est fait
\item[\vref{Né 9:26}] et ils te firent de grands o.
\item[\vref{Job 4:8}] et qui sèment l'o. les moissonnent ;
\item[\vref{Ps 69:10}] dévore, et les o. de ceux qui
\item[\vref{Pr 12:21}] Il n'arrivera aucun o. aux justes, mais
\item[\vref{Ac 5:41}] de subir des o. pour le Nom
\item[\vref{1 Th 2:2}] et reçu des o. à Philippes, com.
\end{listverse}

\ConcordanceEntry{Outrager}
\vspace{-2mm}
\begin{listverse}
\item[\vref{No 15:30}] l'étranger, il a o. Yahweh, cette personne-là
\item[\vref{Job 16:10}] la joue pour m'o. ; ils se réunissent
\item[\vref{Ps 42:11}] os, mes ennemis m'o., tandis qu'ils me
\item[\vref{Mt 5:11}] serez-vs., lorsqu'on vs. o., qu'on vs. persécutera
\item[\vref{Mt 22:6}] ses serviteurs, les o., et les tuèrent.
\item[\vref{Lu 11:45}] en disant ces choses, tu ns. o. aussi.
\item[\vref{Jn 8:49}] mais j'honore mon Père, et vs. m'o.
\item[\vref{Ac 14:5}] principaux chefs, pour o. et lapider les
\item[\vref{Hé 10:29}] et qui aura o. l'Esprit de grâce ?
\end{listverse}

\ConcordanceEntry{Ouvertement}
\vspace{-2mm}
\begin{listverse}
\item[\vref{Jn 16:25}] je vs. parlerai o. de mon Père.
\item[\vref{Jn 18:20}] lui répondit : J'ai o. parlé au monde ;
\end{listverse}

\ConcordanceEntry{Ouvrage}
\vspace{-2mm}
\begin{listverse}
\item[\vref{Ex 5:4}] peuple de son o. ? Allez mntnt à
\item[\vref{1 Ch 29:1}] et délicat, et l'o. est considérable, car
\item[\vref{Né 4:11}] les tuerons et ferons ainsi cesser l'o.
\item[\vref{Job 10:3}] et à dédaigner l'o. de tes mains,
\item[\vref{Ep 2:10}] ns. sommes son o., ayant été créés
\end{listverse}

\ConcordanceEntry{Ouvrier}
\vspace{-2mm}
\begin{listverse}
\item[\vref{Mt 9:37}] grande, mais il y a peu d'o.
\item[\vref{Mt 10:10}] ni bâton ; car l'o. mérite sa nourriture.
\item[\vref{Mt 20:1}] de louer des o. pour sa vigne.
\item[\vref{Ac 19:24}] de profit aux o. du métier.
\item[\vref{1 Co 3:9}] Car ns. sommes o. avec Dieu. Vous
\item[\vref{2 Co 11:13}] faux apôtres, des o. trompeurs qui se
\item[\vref{Ph 3:2}] garde aux mauvais o., prenez garde aux
\item[\vref{1 Ti 5:18}] le grain. Et l'o. mérite son salaire.
\item[\vref{2 Ti 2:15}] Dieu, com. un o. sans reproche, enseignant
\item[\vref{Ja 5:4}] avez frustrés les o. qui ont moissonné
\item[\vref{3 Jn 1:8}] hommes, afin d'être o. avec eux pour
\end{listverse}

\ConcordanceEntry{Ouvrir}
\vspace{-2mm}
\begin{listverse}
\item[\vref{Ge 3:5}] vos yeux seront o., et vs. serez
\item[\vref{Ge 8:6}] quarante jours, Noé o. la fenêtre qu'il
\item[\vref{Jos 8:17}] laissèrent la ville o., et ils poursuivirent
\item[\vref{Jg 16:17}] alors il lui o. tt son cœur,
\item[\vref{Ps 104:28}] et qnd tu o. ta main, ils
\item[\vref{Ps 105:41}] Il o. le rocher, et les eaux en
\item[\vref{Ps 106:17}] La terre s'o. et engloutit Dathan ;
\item[\vref{Ps 118:19}] O.-moi les portes de la justice ;
\item[\vref{Ps 119:18}] O. mes yeux afin que je regarde
\item[\vref{Ps 146:8}] Yahweh o. les yeux des aveugles ; Yahweh redresse
\item[\vref{Pr 13:3}] mais celui qui o. à tt propos
\item[\vref{Pr 31:8}] O. ta bouche en faveur du muet,
\item[\vref{Ca 5:2}] propos de Salomon :] O.-moi, ma sœur,
\item[\vref{Ez 3:2}] J'o. dc ma bouche, et il me
\item[\vref{Da 9:22}] venu mntnt pour o. ton intelligence.
\item[\vref{Mt 7:7}] vs. trouverez ; frappez, et l'on vs. o.
\item[\vref{Mt 25:11}] vinrent aussi, et dirent : Seign., Seign. ! o.-ns. !
\item[\vref{Mc 1:10}] vit les cieux s'o., et le Saint-Esprit
\item[\vref{Mc 7:34}] soupira, et lui dit : Ephphatha. C'est-à-dire : O.-toi.
\item[\vref{Lu 24:45}] Alors il lr. o. l'esprit afin qu'ils
\item[\vref{Jn 10:21}] un démon peut-il o. les yeux des
\item[\vref{Ac 16:14}] Le Seign. lui o. le cœur, afin
\item[\vref{Ac 26:18}] pour o. leurs yeux afin qu'ils passent des
\item[\vref{Ap 3:7}] de David, qui o. et nul ne
\item[\vref{Ap 3:20}] ma voix et m'o. la porte, j'entrerai
\item[\vref{Ap 5:2}] Qui est digne d'o. le livre, et
\end{listverse}

\ConcordanceEntry{Ozias}
\vspace{-2mm}
\begin{listverse}
\item[\vref{2 Ch 26:1}] de Juda prit O., âgé de seize
\end{listverse}

\ConcordanceEntry{Paille}
\vspace{-2mm}
\begin{listverse}
\item[\vref{Ge 24:32}] donna de la p. et du fourrage
\item[\vref{Ex 5:7}] donnerez plus de p. à ce peuple
\item[\vref{Es 11:7}] com. le bœuf, mangera de la p.
\item[\vref{Es 33:11}] enfanterez de la p. ; votre souffle vs.
\item[\vref{Es 65:25}] mangeront de la p., et la poussière
\item[\vref{Jé 23:28}] a-t-il entre la p. et le froment ?
\item[\vref{Da 2:35}] devinrent com. la p. de l'aire en
\item[\vref{Mt 3:12}] il brûlera la p. ds un feu
\item[\vref{Mt 7:3}] pourquoi vois-tu la p. qui est ds
\end{listverse}

\ConcordanceEntry{Pain}
\vspace{-2mm}
\begin{listverse}
\item[\vref{Ge 3:19}] Tu mangeras le p. à la sueur
\item[\vref{Ge 14:18}] fit apporter du p. et du vin ;
\item[\vref{Ge 19:3}] fit cuire des p. sans levain, et
\item[\vref{Ge 47:12}] Joseph fournit du p. à son père
\item[\vref{Ex 12:8}] mangeront avec des p. sans levain, et
\item[\vref{Ex 12:15}] sept jours des p. sans levain, et
\item[\vref{Ex 16:4}] des cieux du p., et le peuple
\item[\vref{Ex 16:15}] dit : C'est le p. que Yahweh vs.
\item[\vref{Ex 25:30}] cette table le p. de proposition continuellement
\item[\vref{Lé 8:2}] une corbeille de p. sans levain ;
\item[\vref{Lé 23:6}] fête solennelle des p. sans levain à
\item[\vref{Lé 23:17}] vos demeures deux p. pour en faire
\item[\vref{No 4:7}] la table des p. de proposition, et
\item[\vref{No 9:11}] mangeront avec du p. sans levain et
\item[\vref{De 9:9}] sans manger de p. et sans boire
\item[\vref{De 9:18}] sans manger de p. et sans boire
\item[\vref{De 23:4}] rencontre avec du p. et de l'eau,
\item[\vref{De 29:6}] point mangé de p., ni bu de
\item[\vref{Jos 5:11}] pays, savoir, des p. sans levain et
\item[\vref{Jos 9:5}] et tt le p. qu'ils avaient pour
\item[\vref{Jg 7:13}] qu'un gâteau de p. d'orge roulait ds
\item[\vref{Ru 1:6}] son peuple en lui donnant du p.
\item[\vref{1 S 2:5}] louent pour du p., mais les affamés
\item[\vref{1 S 21:4}] n'ai pas de p. ordinaire sous la
\item[\vref{1 R 7:48}] lesquelles étaient les p. de proposition ;
\item[\vref{1 R 22:27}] prison, nourrissez-le de p. et de l'eau
\item[\vref{2 R 4:42}] de Dieu du p. des prémices, à
\item[\vref{1 Ch 9:32}] la charge des p. de proposition pour
\item[\vref{Esd 10:6}] mangea point de p., ne but point
\item[\vref{Né 9:15}] des cieux, du p. qnd ils avaient
\item[\vref{Né 10:33}] pour les p. de proposition, pour l'offrande perpétuelle et
\item[\vref{Ps 37:25}] abandonné, ni sa postérité mendiant son p.
\item[\vref{Ps 41:10}] qui mangeait mon p., a levé le
\item[\vref{Ps 78:20}] ns. donner du p. ? Fournirait-il de la
\item[\vref{Ps 78:25}] mangèrent ts le p. des grands. Il
\item[\vref{Ps 104:15}] le cœur de l'hom. avec le p.
\item[\vref{Ps 127:2}] vs. mangez le p. de douleurs ; certes
\item[\vref{Ps 132:15}] je rassasierai de p. ses pauvres.
\item[\vref{Ps 146:7}] il donne du p. aux affamés ; Yahweh
\item[\vref{Pr 6:8}] en été son p. et amasse durant
\item[\vref{Pr 17:1}] un morceau de p. sec là où
\item[\vref{Pr 22:9}] donne de son p. au pauvre.
\item[\vref{Pr 25:21}] à manger du p. ; et s'il a
\item[\vref{Pr 30:8}] richesse, nourris-moi du p. qui m'est nécessaire.
\item[\vref{Pr 31:27}] mange pas le p. de la paresse.
\item[\vref{Ec 9:7}] dc, mange ton p. avec joie, et
\item[\vref{Es 55:10}] semeur, et du p. à celui qui
\item[\vref{Es 58:7}] tu partages ton p. avec celui qui
\item[\vref{La 5:9}] Nous amenons notre p. au péril de
\item[\vref{Ez 4:16}] le bâton du p. ds Jérus. ; et
\item[\vref{Ez 12:18}] l'hom., mange ton p. ds l'agitation, et
\item[\vref{Am 8:11}] une famine de p., ni une soif
\item[\vref{Mal 1:7}] mon autel du p. souillé, et vs.
\item[\vref{Mt 4:3}] ordonne que ces pierres deviennent des p.
\item[\vref{Mt 4:4}] vivra pas de p. seulement, mais de
\item[\vref{Mt 6:11}] Donne-ns. aujourd'hui notre p. quotidien ;
\item[\vref{Mt 12:4}] et mangea les p. de proposition, qu'il
\item[\vref{Mt 14:17}] ici que cinq p. et deux poissons.
\item[\vref{Mt 15:26}] de prendre le p. des enfants, et
\item[\vref{Mt 26:17}] premier jour des p. sans levain, les
\item[\vref{Mc 6:37}] Irions-ns. acheter des p. pour deux cents
\item[\vref{Lu 11:3}] chaque jour notre p. quotidien.
\item[\vref{Lu 11:5}] lui dire : Mon ami, prête-moi trois p.,
\item[\vref{Lu 22:19}] il prit du p., et après avoir
\item[\vref{Lu 24:35}] été reconnu d'eux en rompant le p.
\item[\vref{Jn 6:26}] avez mangé des p. et que vs.
\item[\vref{Jn 6:35}] Je suis le p. de vie. Celui
\item[\vref{Jn 6:48}] Je suis le p. de vie.
\item[\vref{Jn 6:51}] Je suis le p. vivant qui est
\item[\vref{Ac 2:46}] et rompant le p. de maison en
\item[\vref{Ac 20:7}] pour rompre le p., Paul, qui devait
\item[\vref{1 Co 5:8}] mais avec les p. sans levain de
\item[\vref{1 Co 10:16}] Christ ? Et le p. que ns. rompons,
\item[\vref{1 Co 11:26}] mangerez de ce p., et que vs.
\item[\vref{1 Co 11:27}] mangera de ce p. ou boira de
\item[\vref{2 Th 3:12}] manger lr. propre p. en travaillant paisiblement.
\end{listverse}

\ConcordanceEntry{Paître}
\vspace{-2mm}
\begin{listverse}
\item[\vref{Ge 30:36}] et Jacob fit p. le reste des
\item[\vref{Ge 37:12}] Joseph s'en allèrent p. les troupeaux de
\item[\vref{1 S 17:34}] Ton serviteur faisait p. les brebis de
\item[\vref{2 S 5:2}] t'a dit : Tu p. mon peuple d'Israël,
\item[\vref{Ps 78:71}] et l'amena pour p. Jacob, son peuple,
\item[\vref{Ca 1:7}] où tu fais p. ton troupeau, et
\item[\vref{Es 11:7}] La jeune vache p. avec l'ourse, leurs
\item[\vref{Jé 23:4}] pasteurs qui les p., et elles n'auront
\item[\vref{Ez 34:15}] Moi-mm je p. mes brebis et
\item[\vref{Za 11:9}] Je ne vs. p. plus ; que celle
\item[\vref{Mt 2:6}] le Chef qui p. mon peuple d'Israël.
\item[\vref{Lu 8:34}] qui les faisaient p., voyant ce qui
\item[\vref{Lu 15:15}] ses possessions pour p. les pourceaux.
\item[\vref{Ac 20:28}] établis évêques, pour p. l'Eglise de Dieu,
\item[\vref{Ap 7:17}] du trône les p., et les conduira
\end{listverse}

\ConcordanceEntry{Paix}
\vspace{-2mm}
\begin{listverse}
\item[\vref{Ge 15:15}] tes pères en p., et tu seras
\item[\vref{Ge 28:21}] je retourne en p. à la maison
\item[\vref{Ex 18:23}] peuple parviendra en p. à destination.
\item[\vref{Lé 3:1}] sacrifice d'offrande de p., et qu'il offre
\item[\vref{Lé 26:6}] Je donnerai la p. au pays, vs.
\item[\vref{No 6:26}] vers toi, et te donne la p. !
\item[\vref{No 25:12}] je lui donne mon alliance de p.
\item[\vref{De 20:10}] la guerre, tu l'inviteras à la p.
\item[\vref{Jos 9:15}] Josué fit la p. avec eux, et
\item[\vref{Jos 11:19}] qui fit la p. avec les enfants
\item[\vref{Jg 6:23}] dit : Sois en p., ne crains pas,
\item[\vref{1 S 16:4}] lui dirent : Ton arrivée annonce-t-elle la p. ?
\item[\vref{1 R 4:24}] Il était en p. avec ts les
\item[\vref{1 Ch 22:9}] je donnerai la p. et le repos
\item[\vref{Ps 4:9}] je m'endors en p., car toi seul,
\item[\vref{Ps 28:3}] qui parlent de p. avec lr. prochain
\item[\vref{Ps 34:15}] bien ; cherche la p. et poursuis-la.
\item[\vref{Ps 35:20}] parlent point de p., mais ils préméditent
\item[\vref{Ps 35:27}] qui désire la p. de son serviteur !
\item[\vref{Ps 55:21}] qui vivaient en p. avec lui, et
\item[\vref{Ps 72:7}] aura abondance de p. jusqu'à ce qu'il
\item[\vref{Ps 85:9}] il parlera de p. à son peuple
\item[\vref{Ps 119:165}] a une grande p. pour ceux qui
\item[\vref{Ps 120:7}] cherche que la p. ; mais lorsque j'en
\item[\vref{Ps 122:6}] Priez pour la p. de Jérus.. Que
\item[\vref{Pr 12:20}] joie pour ceux qui conseillent la p.
\item[\vref{Es 9:5}] le Père d'éternité, le Prince de p.,
\item[\vref{Es 9:6}] l'empire, et une p. sans fin au
\item[\vref{Es 26:3}] gardes ds la p., ds la paix,
\item[\vref{Es 26:12}] tu ordonnes la p. pour ns., car
\item[\vref{Es 27:5}] qu'il fasse la p. avec moi, qu'il
\item[\vref{Es 32:17}] justice sera la p., et le fruit
\item[\vref{Es 48:22}] a point de p. pour les méchants,
\item[\vref{Es 53:5}] ns. apporte la p. est tombé sur
\item[\vref{Es 57:19}] fruits des lèvres. P., paix à celui
\item[\vref{Es 60:17}] ferai régner la p. et dominer la
\item[\vref{Jé 4:10}] Vous aurez la p. ! Et cependant l'épée
\item[\vref{Jé 6:14}] mon peuple, disant : P. ! Paix ! Et il
\item[\vref{Jé 8:15}] On attendait la p., et il n'y
\item[\vref{Jé 14:13}] vs. donnerai ds ce lieu-ci une p. assurée.
\item[\vref{Jé 29:7}] Et cherchez la p. de la ville
\item[\vref{Ez 34:25}] une alliance de p. ; et je détruirai
\item[\vref{Da 4:1}] terre : Que votre p. soit multipliée !
\item[\vref{Ag 2:9}] je mettrai la p. en ce lieu,
\item[\vref{Za 9:10}] Roi parlera de p. aux nations ; et
\item[\vref{Mt 5:9}] qui procurent la p., car ils seront
\item[\vref{Mt 10:13}] digne, que votre p. vienne sur elle ;
\item[\vref{Mt 10:34}] venu apporter la p. sur la terre ;
\item[\vref{Lu 2:14}] très-hauts, que la p. soit sur la
\item[\vref{Lu 19:38}] Nom du Seign. ! P. ds le ciel,
\item[\vref{Lu 19:42}] appartiennent à ta p. ! Mais mntnt elles
\item[\vref{Lu 24:36}] dit : Que la p. soit avec vs. !
\item[\vref{Jn 14:27}] vs. laisse la p., je vs. donne
\item[\vref{Jn 16:33}] vs. ayez la p. en moi. Vous
\item[\vref{Jn 20:19}] dit : Que la p. soit avec vs. !
\item[\vref{Ac 9:31}] églises étaient en p. ds tte la
\item[\vref{Ac 10:36}] lr. annonçant la p. par Jésus-Christ, qui
\item[\vref{Ro 1:7}] grâce et la p. vs. soient données
\item[\vref{Ro 3:17}] pas connu la voie de la p. ;
\item[\vref{Ro 5:1}] ns. avons la p. avec Dieu, par
\item[\vref{Ro 10:15}] qui annoncent la p., de ceux qui
\item[\vref{Ro 12:18}] vs., soyez en p. avec ts les
\item[\vref{Ro 14:19}] tendent à la p. et à l'édification
\item[\vref{Ro 16:20}] Le Dieu de p. brisera bientôt Satan
\item[\vref{1 Co 7:15}] Dieu ns. a appelés à la p.
\item[\vref{1 Co 14:33}] confusion, mais de p., com. on le
\item[\vref{2 Co 13:11}] sentiment, vivez en p. ; et le Dieu
\item[\vref{Ga 5:22}] la joie, la p., la patience, la
\item[\vref{Ga 6:16}] Que la p. et la miséricorde soient sur ts
\item[\vref{Ep 2:14}] il est notre p., lui qui des
\item[\vref{Ep 6:15}] pieds chaussés, prêts pour l'Evangile de p. ;
\item[\vref{Ph 4:7}] Et la p. de Dieu, qui surpasse tte intelligence,
\item[\vref{Col 1:20}] ayant fait la p. par le sang
\item[\vref{Col 3:15}] Et que la p. de Dieu, à
\item[\vref{1 Th 5:3}] Nous sommes en p. et en sûreté.
\item[\vref{1 Th 5:13}] font. Soyez en p. entre vs.
\item[\vref{2 Th 3:16}] le Seign. de p. vs. donne toujours
\item[\vref{Hé 7:2}] Roi de Salem, c'est-à-dire, Roi de p.
\item[\vref{Hé 12:14}] Recherchez la p. avec ts, et
\item[\vref{Ja 3:18}] semé ds la p. pour ceux qui
\item[\vref{1 Pi 3:11}] qu'il recherche la p., et qu'il tâche
\item[\vref{Ap 6:4}] pouvoir ôter la p. de la terre,
\end{listverse}

\ConcordanceEntry{Palais}
\vspace{-2mm}
\begin{listverse}
\item[\vref{2 S 22:7}] Dieu ; de son p., il a entendu
\item[\vref{2 R 20:18}] eunuques ds le p. du roi de
\item[\vref{1 Ch 29:1}] considérable, car ce p. n'est point pour
\item[\vref{Est 4:2}] d'entrer ds le p. du roi revêtu
\item[\vref{Job 12:11}] ainsi que le p. savoure les mets ?
\item[\vref{Ps 5:8}] prosternerai ds le p. de ta sainteté
\item[\vref{Ps 29:9}] forêts. Dans son p. tt s'écrie : Gloire !
\item[\vref{Ps 45:16}] et allégresse, et elles entreront au p. du roi.
\item[\vref{Ps 48:4}] connu ds ses p. pour une haute
\item[\vref{Ps 119:103}] douce à mon p. ! plus douce que
\item[\vref{Ps 122:7}] murs, et la tranquillité ds tes p. !
\item[\vref{Pr 30:28}] pourtant ds les p. des rois.
\item[\vref{Jé 30:18}] ruines et le p. sera rétabli com.
\item[\vref{Da 1:4}] tenir ds le p. du roi ; et
\item[\vref{Os 8:14}] a bâti des p. ; et Juda a
\item[\vref{Jon 2:8}] à toi, jusqu'au p. de ta sainteté.
\item[\vref{Mi 1:2}] Seign., sortant du p. de sa sainteté.
\end{listverse}

\ConcordanceEntry{Palmier}
\vspace{-2mm}
\begin{listverse}
\item[\vref{Ex 15:27}] d'eau, et soixante-dix p.. Et ils campèrent
\item[\vref{Lé 23:40}] des branches de p., des rameaux d'arbres
\item[\vref{No 33:9}] d'eaux et soixante-dix p., et ils y
\item[\vref{De 34:3}] la ville des p., jusqu'à Tsoar.
\item[\vref{Jg 3:13}] ils s'emparèrent de la ville des p.
\item[\vref{Né 8:15}] des rameaux de p., et des rameaux
\item[\vref{Ps 92:13}] fleurit com. le p., il croît com.
\item[\vref{Jé 10:5}] un tronc de p., et ils ne
\end{listverse}

\ConcordanceEntry{Pamphylie}
\vspace{-2mm}
\begin{listverse}
\item[\vref{Ac 13:13}] Perge, ville de P.. Mais Jean se
\item[\vref{Ac 15:38}] quittés depuis la P., et qui ne
\end{listverse}

\ConcordanceEntry{Panetier}
\vspace{-2mm}
\begin{listverse}
\item[\vref{Ge 40:1}] l'échanson et le p. du roi d'Egypte
\item[\vref{Ge 40:22}] le chef des p., selon l'explication que
\item[\vref{Ge 41:13}] et fit pendre le chef des p.
\end{listverse}

\ConcordanceEntry{Panser}
\vspace{-2mm}
\begin{listverse}
\item[\vref{Jé 6:14}] Et ils p. à la légère la plaie de
\item[\vref{Jé 30:13}] ta cause, pour p. ta plaie ; il
\item[\vref{Jé 51:9}] Nous avons p. Babylone, mais elle
\item[\vref{Os 5:13}] vs. guérir, ni p. vos plaies.
\end{listverse}

\ConcordanceEntry{Paphos}
\vspace{-2mm}
\begin{listverse}
\item[\vref{Ac 13:6}] traversé l'île jusqu'à P., ils trouvèrent là
\item[\vref{Ac 13:13}] furent partis de P., ils vinrent à
\end{listverse}

\ConcordanceEntry{Pâque}
\vspace{-2mm}
\begin{listverse}
\item[\vref{Ex 12:11}] hâte. C'est la P. de Yahweh.
\item[\vref{Ex 34:25}] solennelle de la P. jusqu'au matin.
\item[\vref{Lé 23:5}] soirs, sera la P. à Yahweh.
\item[\vref{No 9:14}] vs. célèbre la P. de Yahweh, il
\item[\vref{De 16:2}] tu sacrifieras la P. à Yahweh, ton
\item[\vref{2 R 23:22}] Aucune P. pareille à celle-ci n'avait été célébrée
\item[\vref{Esd 6:19}] captivité célébrèrent la P. le quatorzième jour
\item[\vref{Ez 45:21}] vs. aurez la P., fête solennelle qui
\item[\vref{Mt 26:18}] je ferai la P. chez toi avec
\item[\vref{Mc 14:12}] sacrifiait l'agneau de P., ses disciples lui
\item[\vref{Lu 2:41}] à Jérus., à la fête de P.
\item[\vref{Lu 22:15}] cet agneau de P. avec vs., avant
\item[\vref{Jn 2:13}] Car la P. des Juifs était proche, c'est pourquoi
\item[\vref{Jn 2:23}] la fête de P., plusieurs crurent en
\item[\vref{Ac 12:4}] le peuple après la fête de P.
\item[\vref{1 Co 5:7}] car Christ, notre P., a été sacrifié
\item[\vref{Hé 11:28}] il fit la P. et l'aspersion du
\end{listverse}

\ConcordanceEntry{Parabole}
\vspace{-2mm}
\begin{listverse}
\item[\vref{1 R 4:32}] prononcé trois mille p. et composa cinq
\item[\vref{Ps 78:2}] bouche en une p., je proférerai les
\item[\vref{Ez 21:5}] fait que mettre en avant des p. ?
\item[\vref{Ez 24:3}] Propose une p. à la famille
\item[\vref{Os 12:11}] par les prophètes, je proposerai des p.
\item[\vref{Mt 13:3}] lr. parla en p. sur beaucoup de
\item[\vref{Mt 13:34}] ces choses en p., et il ne
\item[\vref{Mt 13:35}] ma bouche en p., je déclarerai les
\item[\vref{Mt 22:1}] de nouveau en p., et il dit :
\item[\vref{Mc 4:34}] parlait pas sans p. ; mais en particulier,
\item[\vref{Mc 12:12}] avait dit cette p.. C'est pourquoi, ils
\item[\vref{Lu 8:11}] que signifie cette p. : La semence, c'est
\item[\vref{Lu 19:11}] et proposa une p., parce qu'il était
\item[\vref{Jn 16:25}] ces choses en p.. Mais l'heure vient
\end{listverse}

\ConcordanceEntry{Paradis}
\vspace{-2mm}
\begin{listverse}
\item[\vref{Lu 23:43}] tu seras avec moi ds le p.
\item[\vref{2 Co 12:4}] ravi ds le p., et a entendu
\item[\vref{Ap 2:7}] vie, qui est au milieu du p. de Dieu.
\end{listverse}

\ConcordanceEntry{Paraître}
\vspace{-2mm}
\begin{listverse}
\item[\vref{Ge 9:14}] de nuées, l'arc p. ds la nuée ;
\item[\vref{Ps 107:25}] et il fait p. la tempête qui
\item[\vref{Pr 18:17}] plaide le premier p. juste ; mais sa
\item[\vref{Es 43:19}] chose nouvelle, qui p. bientôt, ne la
\item[\vref{Es 51:5}] mon salut va p., et mes bras
\item[\vref{Mal 3:2}] subsister qnd il p. ? Car il sera
\item[\vref{Mt 24:30}] Fils de l'hom. p. ds le ciel,
\item[\vref{Lu 16:15}] vs. cherchez à p. justes dvt les
\item[\vref{Lu 17:30}] jour où le Fils de l'hom. p.
\item[\vref{Lu 19:11}] le Royaume de Dieu allait immédiatement p.
\item[\vref{Ep 5:27}] afin de faire p. dvt lui cette
\item[\vref{Col 3:4}] apparaîtra, alors vs. p. aussi avec lui
\item[\vref{Hé 10:35}] vs. avez fait p., et qui sera
\item[\vref{Ja 4:14}] qu'une vapeur qui p. pour un peu
\item[\vref{2 Pi 1:19}] jour vienne à p. et que l'Etoile
\item[\vref{1 Jn 2:8}] passées, et que la véritable lumière p. déjà.
\item[\vref{Jud 1:24}] et vs. faire p. dvt sa gloire,
\end{listverse}

\ConcordanceEntry{Paralytique}
\vspace{-2mm}
\begin{listverse}
\item[\vref{Mt 4:24}] des lunatiques, des p. ; et il les
\item[\vref{Mt 8:6}] serviteur qui est p. est couché à
\item[\vref{Mt 9:2}] lui présenta un p. couché sur un
\item[\vref{Mc 2:9}] de dire au p. : Tes péchés te
\item[\vref{Lu 5:18}] hom. qui était p., cherchaient le moyen
\item[\vref{Jn 5:3}] des boiteux, des p., attendant le mouvement
\item[\vref{Ac 8:7}] et beaucoup de p. et de boiteux
\item[\vref{Ac 9:33}] depuis huit ans, car il était p.
\end{listverse}

\ConcordanceEntry{Paran}
\vspace{-2mm}
\begin{listverse}
\item[\vref{Ge 21:21}] le désert de P. ; et sa mère
\item[\vref{No 13:3}] du désert de P., d'après l'ordre de
\item[\vref{1 S 25:1}] leva, et descendit au désert de P.
\item[\vref{Ha 3:3}] du mont de P. ; Sélah. Sa majesté
\end{listverse}

\ConcordanceEntry{Parchemin}
\vspace{-2mm}
\begin{listverse}
\item[\vref{2 Ti 4:13}] les livres aussi ; mais principalement mes p.
\end{listverse}

\ConcordanceEntry{Parcourir}
\vspace{-2mm}
\begin{listverse}
\item[\vref{Jos 24:3}] je lui fis p. tt le pays
\item[\vref{1 R 18:6}] pays pour le p. ; Achab allait seul
\item[\vref{Ez 39:14}] autre chose que p. le pays, et
\item[\vref{Za 1:10}] a envoyés pour p. la terre.
\item[\vref{Za 6:7}] demandèrent à aller p. la terre. L'Ange
\item[\vref{Mt 10:23}] pas achevé de p. ttes les villes
\item[\vref{Jn 7:1}] ne voulait pas p. la Judée, parce
\end{listverse}

\ConcordanceEntry{Pardon}
\vspace{-2mm}
\begin{listverse}
\item[\vref{Ge 43:20}] Ils dirent : P. ! Mon seigneur, ns.
\item[\vref{1 S 1:26}] Elle dit : P., mon seigneur ! Aussi
\item[\vref{Ps 130:4}] Mais le p. se trouve auprès de toi, afin
\item[\vref{Es 33:24}] elle reçoit le p. de ses iniquités.
\item[\vref{Da 9:9}] miséricorde et le p., car ns. avons
\item[\vref{Mc 3:29}] n'obtiendra jamais de p. : Il est coupable
\item[\vref{Lu 24:47}] repentance et le p. des péchés seraient
\item[\vref{Ac 2:38}] pour obtenir le p. de vos péchés,
\end{listverse}

\ConcordanceEntry{Pardonner}
\vspace{-2mm}
\begin{listverse}
\item[\vref{Ge 18:26}] la ville, je p. à tte la
\item[\vref{Ge 50:17}] Je te prie, p. mntnt l'iniquité de
\item[\vref{Ex 10:17}] Mais p., je te prie, mon péché pour
\item[\vref{Ex 23:21}] car il ne p. point votre péché ;
\item[\vref{Ex 32:32}] Maintenant p. lr. péché ! Sinon,
\item[\vref{Lé 4:20}] pour eux, et il lr. sera p.
\item[\vref{No 14:18}] ôte l'iniquité et p. la rébellion, mais
\item[\vref{Jos 24:19}] jaloux, il ne p. point votre rébellion
\item[\vref{1 S 15:25}] je te prie, p.-moi mon péché,
\item[\vref{2 S 24:10}] ô Yahweh, de p. l'iniquité de ton
\item[\vref{1 R 8:36}] exauce-le des cieux, p. le péché de
\item[\vref{2 R 24:4}] pourquoi Yahweh ne voulut point lui p.
\item[\vref{Né 9:17}] un Dieu qui p., miséricordieux, compatissant, lent
\item[\vref{Job 7:21}] Et pourquoi ne p.-tu pas mon
\item[\vref{Ps 25:18}] ma peine, et p. ts mes péchés.
\item[\vref{Ps 32:1}] la transgression est p., et dont le
\item[\vref{Ps 99:8}] un Dieu qui p., et qui fait
\item[\vref{Ps 103:3}] C'est lui qui p. ttes tes iniquités,
\item[\vref{Es 55:7}] notre Dieu qui p. abondamment.
\item[\vref{Jé 33:8}] moi ; et je p. ttes leurs iniquités
\item[\vref{Da 9:19}] Seign., exauce ! Seign. p. ! Seign. sois attentif,
\item[\vref{Os 14:2}] à Yahweh. Dites-lui : P. ttes nos iniquités,
\item[\vref{Am 7:2}] dis : Seign. Yahweh, p., je te prie !
\item[\vref{Mal 3:17}] et je lr. p. com. un hom.
\item[\vref{Mt 6:14}] Car si vs. p. aux hommes leurs
\item[\vref{Mt 9:2}] mon enfant, tes péchés te sont p.
\item[\vref{Mt 9:6}] la terre de p. les péchés : Lève-toi,
\item[\vref{Mt 12:31}] tt blasphème sera p. aux hommes, mais
\item[\vref{Mt 18:21}] combien de fois p.-je à mon
\item[\vref{Mt 18:35}] si vs. ne p. de tt votre
\item[\vref{Mc 2:7}] blasphèmes ? Qui peut p. les péchés, si
\item[\vref{Lu 7:47}] péchés ont été p., car elle a
\item[\vref{Lu 17:3}] toi, reprends-le ; et s'il se repent, p.-lui.
\item[\vref{Lu 23:34}] Jésus dit : Père, p.-lr., car ils
\item[\vref{Jn 20:23}] à qui vs. p. les péchés, ils
\item[\vref{Ac 8:22}] pensée de ton cœur te soit p.
\item[\vref{Ro 4:7}] les iniquités sont p., et dont les
\item[\vref{Ep 4:32}] compassion, et vs. p. les uns aux
\item[\vref{Col 2:13}] vs. ayant gratuitement p. ttes vos offenses.
\item[\vref{Col 3:13}] autres, et vs. p. les uns aux
\item[\vref{Hé 10:18}] les péchés sont p., il n'y a
\item[\vref{Ja 5:15}] commis des péchés, ils lui seront p.
\item[\vref{1 Jn 1:9}] pour ns. les p., et pour ns.
\item[\vref{1 Jn 2:12}] péchés vs. sont p. à cause de
\end{listverse}

\ConcordanceEntry{Parent}
\vspace{-2mm}
\begin{listverse}
\item[\vref{Ge 24:4}] et vers mes p., et tu y
\item[\vref{Lé 18:17}] sont tes proches p. : C'est un crime.
\item[\vref{Lé 25:25}] son plus proche p., viendra et rachètera
\item[\vref{No 27:11}] héritage à son p. le plus proche
\item[\vref{Ru 2:20}] est un proche p. et il est
\item[\vref{Pr 7:4}] ma sœur ; et appelle l'intelligence, ta p.
\item[\vref{Mc 3:21}] Quand ses p. apprirent cela, ils
\item[\vref{Mc 6:4}] patrie, parmi ses p. et ds sa
\item[\vref{Lu 1:58}] voisins et ses p. ayant appris que
\item[\vref{Lu 2:41}] Ses p. allaient ts les ans à Jérus.,
\item[\vref{Lu 8:56}] Les p. de la fille furent ds l'étonnement,
\item[\vref{Lu 14:12}] frères, ni tes p., ni tes riches
\item[\vref{Lu 18:29}] maison, ou ses p., ou ses frères,
\item[\vref{Lu 21:16}] livrés par vos p., par vos frères,
\item[\vref{Ac 10:24}] avait invité ses p. et ses amis.
\item[\vref{Ro 1:30}] inventeurs de maux, rebelles à leurs p.,
\item[\vref{Ro 9:3}] mes frères, mes p. selon la chair,
\item[\vref{1 Ti 5:4}] rendre à leurs p. ce qu'ils ont
\item[\vref{2 Ti 3:2}] rebelles à leurs p., ingrats, irréligieux,
\end{listverse}

\ConcordanceEntry{Parer}
\vspace{-2mm}
\begin{listverse}
\item[\vref{Job 40:5}] P.-toi mntnt de magnificence et de
\item[\vref{Es 61:10}] époux qui se p. de magnificence, et
\item[\vref{Es 63:1}] habits rouges, magnifiquement p. en son vêt.,
\item[\vref{Jé 4:30}] que tu te p. d'ornements d'or, et
\item[\vref{Jé 31:4}] d'Israël ! Tu te p. encore de tes
\item[\vref{Jé 43:12}] et il se p. des richesses du
\item[\vref{Ez 16:11}] Je te p. d'ornements, je mis des bracelets sur
\item[\vref{Ez 23:26}] t'enlèveront les ornements dont tu te p.
\item[\vref{Os 2:15}] où elle se p. de ses anneaux
\item[\vref{1 Ti 2:9}] modestie, ne se p. ni de tresses,
\item[\vref{1 Ti 2:10}] mais qu'elles se p. de bonnes œuvres,
\item[\vref{1 Pi 3:5}] ainsi que se p. aussi autrefois les
\item[\vref{Ap 17:4}] et d'écarlate, et p. d'or, de pierres
\item[\vref{Ap 18:16}] d'écarlate, qui était p. d'or, ornée de
\item[\vref{Ap 21:2}] d'auprès de Dieu, p. com. une épouse
\end{listverse}

\ConcordanceEntry{Paresse}
\vspace{-2mm}
\begin{listverse}
\item[\vref{Pr 19:15}] La p. fait venir le sommeil, et l'âme
\item[\vref{Pr 23:21}] s'appauvrissent, et la p. fait porter des
\item[\vref{Pr 31:27}] mange pas le pain de la p.
\item[\vref{Jé 48:10}] de Yahweh avec p., maudit soit celui
\end{listverse}

\ConcordanceEntry{Paresseux}
\vspace{-2mm}
\begin{listverse}
\item[\vref{Ex 5:8}] car ils sont p., et c'est pour
\item[\vref{Jg 18:9}] Ne soyez pas p. à marcher pour
\item[\vref{Esd 4:4}] du pays rendit p. les mains du
\item[\vref{Pr 6:6}] Va, p., vers la fourmi, regarde ses voies
\item[\vref{Pr 6:9}] P., jusqu'à qnd resteras-tu couché ? Quand te
\item[\vref{Pr 10:4}] La main p. fait devenir pauvre, mais la main
\item[\vref{Pr 10:26}] tel est le p. à ceux qui
\item[\vref{Pr 12:24}] mais la main p. sera tributaire.
\item[\vref{Pr 12:27}] L'hom. p. ne rôtit point son gibier ; mais
\item[\vref{Pr 13:4}] L'âme du p. a des désirs qu'il ne peut
\item[\vref{Pr 15:19}] La voie du p. est com. une
\item[\vref{Pr 19:15}] fait venir le sommeil, et l'âme p. a faim.
\item[\vref{Pr 19:24}] Le p. cache sa main ds le sein,
\item[\vref{Pr 20:4}] Le p. ne laboure pas à cause du
\item[\vref{Pr 21:25}] Le désir du p. le tue, parce
\item[\vref{Pr 22:13}] Le p. dit : Le lion est dehors ! Je
\item[\vref{Pr 24:30}] champ de l'hom. p., et près de
\item[\vref{Pr 26:13}] Le p. dit : Il y a un lion
\item[\vref{Pr 26:14}] ainsi fait le p. sur son lit.
\item[\vref{Pr 26:15}] Le p. plonge sa main ds le plat,
\item[\vref{Pr 26:16}] Le p. se croit plus sage que sept
\item[\vref{Ec 10:18}] cause des mains p., la charpente s'affaisse ;
\item[\vref{Ro 12:11}] ne soyez pas p. à vs. employer
\item[\vref{Tit 1:12}] menteurs, de mauvaises bêtes, des ventres p.
\item[\vref{Hé 5:11}] vs. êtes devenus  p. à écouter.
\end{listverse}

\ConcordanceEntry{Parfait}
\vspace{-2mm}
\begin{listverse}
\item[\vref{De 32:4}] du rocher est p., car ttes ses
\item[\vref{2 S 22:31}] de Dieu est p., la parole de
\item[\vref{Job 37:16}] celui qui est p. en science ?
\item[\vref{Ps 19:8}] de Yahweh est p., elle restaure l'âme ;
\item[\vref{Ps 50:2}] sa splendeur qui est d'une beauté p.
\item[\vref{Ps 119:96}] ce qui est p., mais tes commandements
\item[\vref{Ca 5:2}] ma colombe, ma p. ! Car ma tête
\item[\vref{Ez 16:7}] parvins à une p. beauté, tes seins
\item[\vref{Ez 27:3}] disais : Je suis p. en beauté !
\item[\vref{Ez 28:15}] Tu étais p. ds tes voies dès le jour
\item[\vref{Mt 5:48}] Soyez dc p., com. votre Père
\item[\vref{Mt 19:21}] tu veux être p., va, vends ce
\item[\vref{Jn 17:13}] aient ma joie p. en eux-mêmes.
\item[\vref{Ro 12:2}] ce qui est bon, agréable et p.
\item[\vref{1 Co 2:6}] sagesse parmi les p., une sagesse, dis-je,
\item[\vref{Ep 4:13}] à l'état d'hom. p., à la mesure
\item[\vref{Ph 3:15}] ts qui sommes p., ayons ce mm
\item[\vref{Col 1:28}] présenter tt hom. p. en Jésus-Christ.
\item[\vref{Hé 7:28}] Fils, qui est p. pour toujours.
\item[\vref{Hé 9:11}] excellent et plus p., qui n'est pas
\item[\vref{Hé 10:14}] il a rendu p. pour toujours ceux
\item[\vref{Hé 12:23}] des justes qui ont été rendus p.,
\item[\vref{Ja 1:4}] que vs. soyez p. et accomplis, en
\item[\vref{Ja 1:17}] et tt don p. viennent d'en haut
\item[\vref{Ja 1:25}] ds la loi p., la loi de
\item[\vref{1 Pi 5:10}] temps, vs. rende p., vs. affermisse, vs.
\item[\vref{1 Jn 2:5}] Dieu est véritablement p. en lui, et
\item[\vref{1 Jn 4:18}] charité, mais la p. charité bannit la
\item[\vref{Ap 3:2}] trouvé tes œuvres p. dvt Dieu.
\end{listverse}

\ConcordanceEntry{Parfum}
\vspace{-2mm}
\begin{listverse}
\item[\vref{Ex 30:8}] fera aussi le p., à savoir le
\item[\vref{Ex 30:35}] en feras un p. aromatique selon l'art
\item[\vref{Ex 37:29}] sainte, et le p. pur odoriférant, d'ouvrage
\item[\vref{Ex 40:27}] sur lui le p. odoriférant, com. Yahweh
\item[\vref{No 16:47}] il mit du p. et fit propitiation
\item[\vref{1 S 2:28}] faire brûler les p., et porter l'éphod
\item[\vref{2 R 22:17}] ont offert des p. à d'autres dieux,
\item[\vref{2 Ch 26:18}] Ozias, d'offrir le p. à Yahweh, mais
\item[\vref{Est 2:3}] lr. donne les p. nécessaires pour lr.
\item[\vref{Pr 27:9}] L'huile et les p. réjouissent le cœur,
\item[\vref{Ec 9:8}] et que le p. ne manque point
\item[\vref{Ec 10:1}] et fermenter les p. du parfumeur ; et
\item[\vref{Ca 1:3}] de tes excellents p., ton nom est
\item[\vref{Es 1:13}] vaines offrandes : Le p. m'est en abomination,
\item[\vref{Jé 1:16}] ont fait des p. à d'autres dieux,
\item[\vref{Os 2:15}] elle brûlait des p. aux Baals, où
\item[\vref{Am 5:21}] l'odeur de vos p. ds vos assemblées
\item[\vref{Mt 26:7}] d'albâtre, plein d'un p. de grand prix,
\item[\vref{Mc 14:4}] quoi sert la perte de ce p. ?
\item[\vref{Lu 23:56}] aromates et des p.. Et le jour
\item[\vref{Jn 12:3}] maison fut remplie de l'odeur du p.
\item[\vref{Ph 4:18}] vs., com. un p. de bonne odeur,
\item[\vref{Ap 5:8}] d'or pleines de p., qui sont les
\item[\vref{Ap 8:3}] d'or, et plusieurs p. lui furent donnés
\end{listverse}

\ConcordanceEntry{Parler}
\vspace{-2mm}
\begin{listverse}
\item[\vref{Ge 8:15}] Puis Dieu p. à Noé, en
\item[\vref{Ge 16:13}] qui lui avait p. ; car elle dit :
\item[\vref{Ge 18:27}] la hardiesse de p. au Seign., moi
\item[\vref{Ge 46:2}] Et Dieu p. à Israël ds les visions de
\item[\vref{Ex 4:14}] Je sais qu'il p. très bien, et
\item[\vref{Ex 29:42}] me trouverai avec vs. pour te p.
\item[\vref{Ex 34:29}] rayonnante pendant qu'il p. avec Dieu.
\item[\vref{No 12:1}] Marie et Aaron p. contre Moïse au
\item[\vref{No 20:8}] ton frère. Vous p. en lr. présence
\item[\vref{De 4:33}] voix de Dieu p. du milieu du
\item[\vref{De 5:24}] que Dieu a p. avec l'hom., et
\item[\vref{De 6:7}] et tu en p. qnd tu te
\item[\vref{Jg 6:39}] et je ne p. plus que cette
\item[\vref{1 R 14:11}] ciel le mangeront ; car Yahweh a p.
\item[\vref{Né 13:24}] de leurs fils p. en partie asdodien
\item[\vref{Job 11:5}] souhaitable que Dieu p., et qu'il ouvre
\item[\vref{Job 33:14}] Bien que Dieu p. une première fois
\item[\vref{Job 39:38}] J'ai p. une fois, mais je ne répondrai
\item[\vref{Job 42:4}] mntnt, et je p. ; je t'interrogerai et
\item[\vref{Job 42:5}] de mes oreilles p. de toi ; mais
\item[\vref{Ps 22:31}] le servira, on p. du Seign. de
\item[\vref{Ps 50:1}] Dieu, Yahweh a p. et il a
\item[\vref{Ps 62:12}] Dieu a p. une fois, j'ai entendu cela deux
\item[\vref{Ps 78:19}] Ils p. contre Dieu, disant : Dieu pourrait-il dresser
\item[\vref{Ps 115:5}] bouche et ne p. point, elles ont
\item[\vref{Pr 6:13}] ses yeux, il p. de ses pieds,
\item[\vref{Pr 21:28}] l'hom. qui écoute p. avec gain de
\item[\vref{Pr 23:9}] Ne p. pas aux oreilles de l'insensé, car
\item[\vref{Ec 3:7}] être silencieux et un temps pour p. ;
\item[\vref{Ec 5:1}] précipite point à p., et que ton
\item[\vref{Es 38:1}] lui dit : Ainsi p. Yahweh : Donne tes
\item[\vref{Es 45:19}] Je n'ai point p. en secret ni
\item[\vref{Es 48:16}] je n'ai point p. en secret, depuis
\item[\vref{Es 52:6}] que JE SUIS p. : Voici JE SUIS !
\item[\vref{Es 65:12}] point répondu ; j'ai p., et vs. n'avez
\item[\vref{Es 66:4}] parce que j'ai p., et qu'ils n'ont
\item[\vref{Jé 1:6}] ne sais pas p., car je suis
\item[\vref{Jé 20:9}] lui, je ne p. plus en son
\item[\vref{Jé 42:18}] Car ainsi p. Yahweh des armées,
\item[\vref{Ez 2:1}] pieds, et je p. avec toi.
\item[\vref{Ez 24:18}] J'avais p. au peuple le matin, et ma
\item[\vref{Da 9:20}] Je p. encore, je priais, je confessais mon
\item[\vref{Os 2:16}] désert, et je p. à son cœur.
\item[\vref{Mal 3:16}] craignent Yahweh se p. l'un à l'autre ;
\item[\vref{Mt 9:33}] chassé, le muet p.. Et les foules
\item[\vref{Mt 10:19}] ni comment vs. p., car ce que
\item[\vref{Mt 12:32}] Quiconque p. contre le Fils
\item[\vref{Mt 12:34}] l'abondance du cœur que la bouche p.
\item[\vref{Mt 13:10}] dirent : Pourquoi lr. p.-tu en paraboles ?
\item[\vref{Mc 9:39}] puisse aussitôt après p. mal de moi.
\item[\vref{Mc 13:11}] pas vs. qui p., mais le Saint-Esprit.
\item[\vref{Mc 16:17}] mon Nom ; ils p. de nouvelles langues ;
\item[\vref{Lu 6:45}] l'abondance du cœur que la bouche p.
\item[\vref{Lu 9:30}] Moïse et Elie, p. avec lui,
\item[\vref{Lu 21:9}] Quand vs. entendrez p. des guerres et
\item[\vref{Jn 3:12}] je vs. ai p. des choses terrestres,
\item[\vref{Jn 7:46}] Jamais hom. n'a p. com. cet hom.
\item[\vref{Jn 9:29}] que Dieu a p. à Moïse ; mais
\item[\vref{Jn 9:37}] vu, et c'est celui qui te p.
\item[\vref{Jn 16:13}] car il ne p. pas de lui-mm,
\item[\vref{Ac 1:3}] jours, et lr. p. des choses qui
\item[\vref{Ac 2:4}] et commencèrent à p. des langues étrangères
\item[\vref{Ac 4:20}] pas ne pas p. de ce que
\item[\vref{Ac 18:9}] crains pas, mais p. et ne te
\item[\vref{Ro 10:14}] n'ont pas entendu p. ? Et comment en
\item[\vref{Ro 11:13}] Car je vs. p. à vs., Gentils,
\item[\vref{Ro 15:21}] avaient pas entendu p. l'entendront.
\item[\vref{1 Co 12:3}] que personne, s'il p. par l'Esprit de
\item[\vref{1 Co 14:27}] Et si quelqu'un p. une langue, que
\item[\vref{1 Co 14:29}] ou trois prophètes p., et que les
\item[\vref{1 Co 14:35}] une fem. de p. ds l'église.
\item[\vref{2 Co 4:13}] c'est pourquoi j'ai p. ! Nous croyons et
\item[\vref{Ep 4:25}] dépouillé le mensonge, p. en vérité chacun
\item[\vref{1 Th 2:4}] l'Evangile, ainsi ns. p. non com. pour
\item[\vref{Hé 1:2}] ns. a p. ds ces derniers jours par son
\item[\vref{Hé 11:4}] par elle qu'il p. encore, quoique mort.
\item[\vref{Ja 1:19}] écouter, lent à p. et lent à
\item[\vref{Ja 5:10}] prophètes qui ont p. au Nom du
\item[\vref{2 Pi 1:21}] les saints hommes de Dieu ont p.
\item[\vref{2 Pi 2:12}] et détruites, ils p. d'une manière blasphématoire
\item[\vref{2 Pi 2:16}] une ânesse muette, p. d'une voix humaine,
\item[\vref{2 Pi 3:16}] lettres, où il p. de ces choses,
\item[\vref{Ap 13:11}] l'Agneau ; mais elle p. com. le dragon.
\item[\vref{Ap 21:9}] moi et me p., en disant : Viens,
\end{listverse}

\ConcordanceEntry{Parole}
\vspace{-2mm}
\begin{listverse}
\item[\vref{Ge 3:17}] obéi à la p. de ta fem.,
\item[\vref{De 11:18}] votre âme ces p.. Liez-les com. un
\item[\vref{Jos 8:34}] haut ttes les p. de la loi,
\item[\vref{1 S 3:1}] présence d'Eli. La p. de Yahweh était
\item[\vref{Ps 45:2}] Des p. agréables bouillonnent ds mon cœur. Et
\item[\vref{Ps 77:9}] pour toujours ? Sa p. a-t-elle pris fin
\item[\vref{Ps 119:103}] Que ta p. est douce à mon palais ! plus
\item[\vref{Ps 119:160}] fondement de ta p. est la vérité,
\item[\vref{Ps 139:4}] Avant que la p. soit sur ma
\item[\vref{Ps 147:19}] Il déclare ses p. à Jacob, ses
\item[\vref{Pr 12:25}] mais la bonne p. le réjouit.
\item[\vref{Pr 13:13}] qui méprise la p. périra à cause
\item[\vref{Pr 15:1}] fureur ; mais la p. douloureuse excite la
\item[\vref{Pr 15:23}] est bonne une p. dite en son
\item[\vref{Pr 30:6}] rien à ses p., de peur qu'il
\item[\vref{Ec 5:1}] c'est pourquoi use de peu de p.
\item[\vref{Ec 8:4}] lieu qu'est la p. du roi, là
\item[\vref{Es 40:8}] tombe, mais la p. de notre Dieu
\item[\vref{Es 50:4}] soutenir par la p. celui qui est
\item[\vref{Es 55:11}] est-il de ma p. qui sort de
\item[\vref{Es 59:21}] toi, et mes p. que j'ai mises
\item[\vref{Jé 8:9}] ont rejeté la p. de Yahweh, et
\item[\vref{Jé 15:16}] J'ai trouvé tes p., je les ai
\item[\vref{Jé 20:8}] dévastation ! Et la p. de Yahweh est
\item[\vref{Da 10:12}] ton Dieu, tes p. ont été exaucées,
\item[\vref{Am 7:10}] pays ne pourrait supporter ttes ses p.
\item[\vref{Am 8:12}] pour chercher la p. de Yahweh, et
\item[\vref{Mal 3:13}] Vos p. sont rudes contre moi, a dit
\item[\vref{Mt 4:4}] mais de tte p. qui sort de
\item[\vref{Mt 6:7}] pas de vaines p., com. font les
\item[\vref{Mt 12:36}] jugement, de tte p. vaine qu'ils auront
\item[\vref{Mt 22:15}] de le surprendre par ses propres p.
\item[\vref{Mt 24:35}] passeront, mais mes p. ne passeront pas.
\item[\vref{Lu 4:22}] et s'étonnaient des p. pleines de grâce
\item[\vref{Lu 5:5}] toutefois à ta p. je jetterai les
\item[\vref{Lu 8:11}] cette parabole : La semence, c'est la p. de Dieu.
\item[\vref{Lu 11:28}] qui écoutent la p. de Dieu, et
\item[\vref{Jn 1:1}] commencement était la P., et la Parole
\item[\vref{Jn 1:14}] Et la P. a été faite chair, elle a
\item[\vref{Jn 6:60}] entendu, dirent : Cette p. est dure, qui
\item[\vref{Jn 6:63}] de rien. Les p. que je vs.
\item[\vref{Jn 6:68}] Tu as les p. de la vie
\item[\vref{Jn 12:47}] quelqu'un entend mes p. et ne les
\item[\vref{Jn 14:24}] garde pas mes p.. Et la parole
\item[\vref{Jn 17:8}] ai donné les p. que tu m'as
\item[\vref{Ac 6:7}] Et la p. de Dieu croissait, et le nombre
\item[\vref{Ac 17:11}] ils reçurent la p. avec beaucoup de
\item[\vref{Ac 19:20}] Ainsi la p. du Seign. se répandait sensiblement, et
\item[\vref{Ro 9:6}] faire que la p. de Dieu soit
\item[\vref{Ro 10:8}] que dit-elle ? La p. est près de
\item[\vref{Ro 10:17}] vient de la p. de Christ.
\item[\vref{1 Co 1:5}] tt don de p., et de tte
\item[\vref{2 Co 2:17}] falsifions pas la p. de Dieu, com.
\item[\vref{Ep 1:13}] avoir entendu la p. de la vérité,
\item[\vref{Ep 6:17}] qui est la p. de Dieu ;
\item[\vref{Ph 2:15}] dvt d'eux la p. de la vie ;
\item[\vref{Col 1:5}] connaissance par la p. de la vérité,
\item[\vref{Col 3:16}] Que la p. de Christ habite abondamment en vs.
\item[\vref{Col 4:6}] Que votre p. soit toujours assaisonnée
\item[\vref{1 Th 2:13}] de ns. la p. de la prédication
\item[\vref{1 Ti 1:15}] Cette p. est certaine et entièrement digne d'être
\item[\vref{2 Ti 1:13}] modèle des saines p. que tu as
\item[\vref{2 Ti 2:9}] malfaiteur ; cependant, la p. de Dieu n'est
\item[\vref{2 Ti 4:2}] prêche la p., insiste en tte
\item[\vref{Tit 1:3}] temps par sa p., ds la prédication
\item[\vref{Hé 2:2}] Car, si la p. prononcée par les
\item[\vref{Hé 4:12}] Car la P. de Dieu est vivante et efficace,
\item[\vref{Hé 5:13}] c'est que la p. de la justice,
\item[\vref{Ja 1:21}] avec douceur la p. qui a été
\item[\vref{Ja 3:2}] pèche pas en p., c'est un hom.
\item[\vref{1 Pi 1:23}] incorruptible, par la p. vivante de Dieu
\item[\vref{2 Pi 1:19}] avons aussi la p. des prophètes qui
\item[\vref{1 Jn 1:1}] propres mains ont touché concernant la P. de vie,
\item[\vref{1 Jn 2:7}] ancien c'est la p. que vs. avez
\item[\vref{1 Jn 2:14}] et que la p. de Dieu demeure
\item[\vref{1 Jn 3:18}] n'aimons pas en p. et avec la
\item[\vref{Ap 6:9}] tués pour la p. de Dieu, et
\item[\vref{Ap 19:13}] sang, et son Nom s'appelle LA P. DE DIEU.
\end{listverse}

\ConcordanceEntry{Part}
\vspace{-2mm}
\begin{listverse}
\item[\vref{Ge 14:24}] mangé, et la p. des hommes qui
\item[\vref{Ex 29:26}] agitée dvt Yahweh. Ce sera ta p.
\item[\vref{No 18:20}] aura point de p. pour toi au
\item[\vref{De 12:12}] il n'a ni p. ni héritage avec
\item[\vref{Jos 14:4}] donna point de p. aux Lévites ds
\item[\vref{1 S 30:24}] avoir autant de p. que celui qui
\item[\vref{Né 2:20}] vs. n'avez ni p., ni droit, ni
\item[\vref{Job 31:2}] Quelle p. Dieu m'aurait-il réservée d'en haut ? Quel
\item[\vref{Ps 16:5}] Yahweh est la p. de mon héritage
\item[\vref{Ps 119:57}] conclus que ma p. est de garder
\item[\vref{Ps 142:6}] mon refuge, ma p. sur la terre
\item[\vref{Ec 5:17}] a donnés ; car c'est là sa p.
\item[\vref{Ec 9:6}] n'auront plus aucune p. à tt ce
\item[\vref{Ec 11:2}] Donnes-en une p. à sept et
\item[\vref{Es 53:12}] lui donnerai sa p. parmi les grands ;
\item[\vref{Za 12:12}] chaque famille à p. : La famille de
\item[\vref{Mal 3:17}] je mettrai à p. mes plus précieux
\item[\vref{Mt 20:17}] il prit à p. ses douze disciples,
\item[\vref{Mt 25:23}] beaucoup ; viens prendre p. à la joie
\item[\vref{Mc 8:32}] l'ayant pris à p., se mit à
\item[\vref{Lu 10:42}] choisi la bonne p., qui ne lui
\item[\vref{Lu 12:46}] lui donnera sa p. avec les infidèles.
\item[\vref{Lu 15:12}] père, donne-moi la p. de bien qui
\item[\vref{Jn 13:8}] n'auras pas de p. avec moi.
\item[\vref{Ro 1:1}] apôtre, mis à p. pour annoncer l'Evangile
\item[\vref{1 Co 9:10}] le foule avec l'espérance d'y avoir p.
\item[\vref{1 Co 16:2}] vs. mette à p. chez lui ce
\item[\vref{2 Co 6:15}] Bélial ? Ou quelle p. a le fidèle
\item[\vref{Hé 7:2}] donna pour sa p. la dîme de
\item[\vref{Ap 18:4}] n'ayez pas de p. à ses fléaux.
\item[\vref{Ap 20:6}] ceux qui ont p. à la première
\item[\vref{Ap 21:8}] les menteurs, lr. p. sera ds l'étang
\item[\vref{Ap 22:19}] Dieu retranchera sa p. du livre de
\end{listverse}

\ConcordanceEntry{Partage}
\vspace{-2mm}
\begin{listverse}
\item[\vref{No 34:18}] de chaque tribu pour faire le p. du pays.
\item[\vref{Jos 18:10}] Josué fit le p. du pays entre
\item[\vref{2 Ch 31:2}] Lévites, selon lr. p., chacun suivant sa
\item[\vref{Ps 11:6}] c'est le calice qu'ils ont en p.
\item[\vref{Ps 73:26}] cœur, et mon p. pour toujours.
\item[\vref{Da 4:15}] bêtes, l'herbe de la terre pour p.
\item[\vref{Lu 12:14}] votre juge, et pour faire vos p. ?
\item[\vref{2 Pi 1:1}] avez reçu en p. une foi du
\end{listverse}

\ConcordanceEntry{Partager}
\vspace{-2mm}
\begin{listverse}
\item[\vref{Ge 10:25}] la terre fut p. ; et le nom
\item[\vref{Ge 15:17}] entre les animaux qui avaient été p.
\item[\vref{Ge 49:27}] le soir il p. le butin.
\item[\vref{No 34:29}] donna l'ordre de p. l'héritage aux enfants
\item[\vref{Ps 22:19}] Ils se p. mes vêtements et tirent au sort
\item[\vref{Pr 16:19}] débonnaires, que de p. le butin avec
\item[\vref{Es 53:12}] les grands ; il p. le butin avec
\item[\vref{Es 58:7}] pas que tu p. ton pain avec
\item[\vref{Ez 47:21}] Vous p. ce pays entre vs., selon les
\item[\vref{Da 11:39}] plusieurs, il lr. p. des terres à
\item[\vref{Os 10:2}] Leur cœur est p. ; ils vont être
\item[\vref{Joë 3:2}] parce qu'ils ont p. entre eux mon
\item[\vref{Mt 27:35}] l'avoir crucifié, ils p. ses vêtements, en
\item[\vref{Lu 3:11}] a deux tuniques p. avec celui qui
\item[\vref{Lu 12:13}] mon frère qu'il p. avec moi notre
\item[\vref{Lu 15:12}] Et il lr. p. ses biens.
\item[\vref{Ac 14:4}] la ville fut p. en deux, et
\end{listverse}

\ConcordanceEntry{Participer}
\vspace{-2mm}
\begin{listverse}
\item[\vref{Mt 25:21}] sur beaucoup ; viens p. à la joie
\item[\vref{Ro 11:17}] place, et rendu p. de la racine
\item[\vref{Ro 12:13}] p. aux nécessités des saints ; exercez l'hospitalité.
\item[\vref{1 Co 10:21}] ne pouvez pas p. à la table
\item[\vref{Ep 3:6}] corps, et qu'ils p. ensemble à sa
\item[\vref{Ep 5:11}] et ne p. pas aux œuvres infructueuses des ténèbres,
\item[\vref{1 Ti 5:22}] précipitation, et ne p. pas aux péchés
\item[\vref{Hé 2:14}] puisque les enfants p. à la chair
\item[\vref{Hé 10:33}] de l'autre, ayant p. aux maux de
\item[\vref{1 Pi 1:2}] Jésus-Christ, et qui p. à l'aspersion de
\item[\vref{1 Pi 4:13}] ce que vs. p. aux souffrances de
\item[\vref{1 Pi 5:1}] de Christ et p. de la gloire
\item[\vref{2 Jn 1:11}] qui le salue p. à ses mauvaises
\item[\vref{Ap 1:9}] frère et qui p. à la tribulation,
\item[\vref{Ap 18:4}] que vs. ne p. pas à ses
\end{listverse}

\ConcordanceEntry{Partir}
\vspace{-2mm}
\begin{listverse}
\item[\vref{Ge 11:2}] il arriva qu'étant p. de l'orient, ils
\item[\vref{Ge 12:4}] Abram dc p., com. Yahweh le
\item[\vref{Ps 78:52}] Il fit p. son peuple com. des brebis, il
\item[\vref{Jé 29:12}] m'invoquerez, et vs. p. ; vs. me prierez,
\item[\vref{Da 10:20}] et qnd je p., voici, le chef
\item[\vref{Mt 25:14}] d'un hom. qui, p. pour un voyage,
\item[\vref{Ac 13:3}] imposèrent les mains, et les laissèrent p.
\item[\vref{Ga 1:17}] moi, mais je p. pour l'Arabie, puis
\item[\vref{Hé 11:8}] héritage, et il p. sans savoir où
\item[\vref{Ja 2:25}] qu'elle les fit p. par un autre
\item[\vref{3 Jn 1:7}] Car ils sont p. pour son Nom,
\end{listverse}

\ConcordanceEntry{Parure}
\vspace{-2mm}
\begin{listverse}
\item[\vref{Es 3:20}] les p. de la tête, les chaînettes des
\item[\vref{Ez 16:39}] emporteront ta magnifique p., et ils te
\item[\vref{1 Pi 3:3}] Que votre p. ne soit pas
\item[\vref{1 Pi 3:4}] mais que votre p. consiste ds l'être
\end{listverse}

\ConcordanceEntry{Parvis}
\vspace{-2mm}
\begin{listverse}
\item[\vref{Ex 27:9}] feras aussi le p. du tabernacle, du
\item[\vref{Ps 65:5}] habite ds tes p. ! Nous serons rassasiés
\item[\vref{Ps 84:11}] jour ds tes p., que mille ailleurs.
\item[\vref{Ps 92:14}] fleurissent ds les p. de notre Dieu ;
\item[\vref{Ps 100:4}] et ds ses p., avec des cantiques.
\item[\vref{Es 1:12}] vs. fouliez de vos pieds mes p. ?
\item[\vref{Es 62:9}] boiront ds les p. de ma sainteté.
\item[\vref{Ez 10:4}] nuée, et le p. fut rempli de
\item[\vref{Ez 40:17}] mena ds le p. extérieur, où se
\item[\vref{Za 3:7}] tu garderas mes p., et je ferai
\item[\vref{Ap 11:2}] de côté le p. extérieur du temple,
\end{listverse}

\ConcordanceEntry{Pas (le)}
\vspace{-2mm}
\begin{listverse}
\item[\vref{Ge 33:14}] tt doucement, au p. de ce bétail
\item[\vref{1 S 20:3}] n'y a qu'un p. entre moi et
\item[\vref{Job 14:16}] tu comptes mes p., et tu veilles
\item[\vref{Job 23:11}] attaché à ses p. ; j'ai pris garde
\item[\vref{Job 31:4}] voies ? Ne compte-t-il pas ts mes p. ?
\item[\vref{Job 34:21}] de l'hom., il regarde ts ses p.
\item[\vref{Ps 17:5}] Mes p. sont fermes ds tes sentiers, mes
\item[\vref{Ps 18:37}] chemin sous mes p., et mes pieds
\item[\vref{Ps 37:31}] aucun de ses p. ne chancellera.
\item[\vref{Ps 56:7}] ils observent mes p., s'attendant à m'ôter
\item[\vref{Ps 89:52}] outrages contre les p. de ton oint.
\item[\vref{Pr 14:15}] mais l'hom. bien avisé considère ses p.
\item[\vref{Pr 16:9}] sa voie, mais Yahweh conduit ses p.
\item[\vref{Pr 20:24}] Les p. de l'hom. sont dirigés par Yahweh,
\item[\vref{Lu 17:15}] revint sur ses p., glorifiant Dieu à
\end{listverse}

\ConcordanceEntry{Passer}
\vspace{-2mm}
\begin{listverse}
\item[\vref{Ex 12:12}] Car je p. cette nuit-là par le pays d'Egypte,
\item[\vref{Ex 15:16}] ton peuple soit p., ô Yahweh ! jusqu'à
\item[\vref{Ex 33:22}] qnd ma gloire p., je te mettrai
\item[\vref{Ex 34:6}] Comme dc Yahweh p. par dvt lui,
\item[\vref{No 20:19}] payerai le prix ; je veux seulement p. à pied.
\item[\vref{No 32:5}] ns. fais point p. le Jourdain.
\item[\vref{Jos 3:16}] coupées. Le peuple p. vis-à-vis de Jéricho.
\item[\vref{Jg 16:17}] rasoir n'est jamais p. sur ma tête
\item[\vref{2 S 12:13}] à David : Yahweh p. par-dessus ton péché,
\item[\vref{Ps 26:2}] éprouve-moi, Yahweh ! Fais p. au creuset mes
\item[\vref{Ec 1:4}] Une génération p. et une autre
\item[\vref{Ec 3:15}] et Dieu rappelle ce qui est p.
\item[\vref{Es 51:9}] anciens, aux siècles p.. N'es-tu pas celui
\item[\vref{Jé 46:17}] il a laissé p. le temps fixé.
\item[\vref{Ez 20:37}] Je vs. ferai p. sous la verge,
\item[\vref{Mt 19:24}] un chameau de p. par le trou
\item[\vref{Lu 4:30}] Mais il p. au milieu d'eux, et s'en alla.
\item[\vref{Lu 21:32}] cette génération ne p. pas, que ttes
\item[\vref{Lu 21:33}] et la terre p., mais mes paroles
\item[\vref{2 Co 5:17}] choses anciennes sont p. ; voici, ttes choses
\item[\vref{2 Co 11:25}] trois fois, j'ai p. un jour et
\item[\vref{Ga 1:6}] de Christ, pour p. à un autre
\item[\vref{Ja 4:13}] ville, ns. y p. une année, ns.
\item[\vref{1 Pi 4:3}] de notre vie p., qnd ns. ns.
\item[\vref{2 Pi 3:10}] jour-là, les cieux p. avec le bruit
\item[\vref{1 Jn 2:17}] Et le monde p. avec sa convoitise ;
\end{listverse}

\ConcordanceEntry{Passion}
\vspace{-2mm}
\begin{listverse}
\item[\vref{Pr 29:11}] au-dehors tte sa p., mais le sage
\item[\vref{Ez 23:11}] qu'elle ds ses p. ; ses prostitutions dépassèrent
\item[\vref{Ro 7:5}] la chair, les p. des péchés excitées
\item[\vref{Ga 5:24}] chair avec ses p. et ses désirs.
\item[\vref{Col 3:5}] fornication, l'impureté, les p., les mauvais désirs,
\item[\vref{Tit 2:12}] l'impiété et aux p. mondaines, et à
\item[\vref{Hé 2:9}] d'honneur par la p. de sa mort,
\item[\vref{2 Pi 2:10}] chair ds la p. de l'impureté, et
\end{listverse}

\ConcordanceEntry{Pasteur}
\vspace{-2mm}
\begin{listverse}
\item[\vref{Ge 49:24}] ainsi devenu le p., le rocher d'Israël.
\item[\vref{Es 56:11}] ce sont des p. qui ne savent
\item[\vref{Jé 2:8}] pas connu, les p. se sont révoltés
\item[\vref{Jé 3:15}] vs. donnerai des p. selon mon cœur,
\item[\vref{Jé 17:16}] avancé plus qu'un p. après toi, je
\item[\vref{Jé 49:19}] qui sera le p. qui tiendra ferme
\item[\vref{Jé 50:6}] brebis perdues ; leurs p. les égaraient, et
\item[\vref{Ez 34:2}] prophétise contre les p. d'Israël ! Prophétise, et
\item[\vref{Ez 37:24}] ts un seul p. ; ils suivront mes
\item[\vref{Mi 5:4}] contre lui sept p. et huit princes
\item[\vref{Za 10:2}] parce qu'il n'y a point de p.
\item[\vref{Za 11:15}] me dit : Prends-toi encore l'équipage d'un p. insensé.
\item[\vref{Za 11:17}] Malheur au p. inutile qui abandonne
\item[\vref{Mt 9:36}] des brebis qui n'ont pas de p.
\item[\vref{Ep 4:11}] autres pour être p. et docteurs,
\item[\vref{Hé 13:20}] morts le grand P. des brebis, par
\item[\vref{1 Pi 2:25}] retournés vers le P. et l'Evêque de
\item[\vref{1 Pi 5:4}] qnd le souv. P. apparaîtra, vs. obtiendrez
\end{listverse}

\ConcordanceEntry{Pâte}
\vspace{-2mm}
\begin{listverse}
\item[\vref{Ex 12:34}] dc prit sa p. avant qu'elle soit
\item[\vref{Jé 7:18}] femmes pétrissent la p. pour faire des
\item[\vref{Ez 44:30}] prémices de votre p. aux prêtres, afin
\item[\vref{Os 7:4}] a pétri la p., jusqu'à ce qu'elle
\item[\vref{Mt 13:33}] que tte la p. soit levée.
\item[\vref{1 Co 5:6}] de levain fait lever tte la p. ?
\item[\vref{1 Co 5:7}] soyez une nouvelle p., puisque vs. êtes
\item[\vref{Ga 5:9}] de levain fait lever tte la p.
\end{listverse}

\ConcordanceEntry{Pathros}
\vspace{-2mm}
\begin{listverse}
\item[\vref{Es 11:11}] en Egypte, à P., en Ethiopie, à
\item[\vref{Jé 44:1}] au pays de P., en disant :
\item[\vref{Ez 29:14}] au pays de P., au pays de
\end{listverse}

\ConcordanceEntry{Patience}
\vspace{-2mm}
\begin{listverse}
\item[\vref{Mt 18:26}] disant : Seign., aie p. envers moi et
\item[\vref{Lu 18:7}] quoiqu'il use de p. avant d'intervenir pour
\item[\vref{Ro 2:4}] et de sa p., et de sa
\item[\vref{Ro 8:25}] pas, c'est que ns. l'attendons avec p.
\item[\vref{Ro 9:22}] avec une grande p. les vases de
\item[\vref{Ro 15:4}] que, par la p., et la consolation
\item[\vref{Ep 4:2}] et douceur, avec p., vs. supportant les
\item[\vref{Col 3:12}] de bonté, d'humilité, de douceur, de p. ;
\item[\vref{1 Ti 6:11}] la charité, la p., la douceur.
\item[\vref{Hé 10:36}] avez besoin de p., afin qu'après avoir
\item[\vref{Ja 1:3}] l'épreuve de votre foi produit la p.
\item[\vref{Ja 5:10}] pour exemple de p. ds les afflictions
\item[\vref{Ja 5:11}] enduré l'épreuve avec p.. Vous avez appris
\item[\vref{1 Pi 3:20}] incrédules, qnd la p. de Dieu les
\item[\vref{2 Pi 1:6}] la tempérance la p., à la patience
\item[\vref{2 Pi 3:15}] Et considérez la p. du Seign. com.
\item[\vref{Ap 2:19}] ta foi, ta p., et que tes
\end{listverse}

\ConcordanceEntry{Patient}
\vspace{-2mm}
\begin{listverse}
\item[\vref{Ec 7:8}] est d'un esprit p. que l'hom. qui
\item[\vref{Ro 12:12}] joyeux ds l'espérance ; p. ds la tribulation ;
\item[\vref{1 Co 13:4}] La charité est p., la charité est
\item[\vref{2 Co 6:6}] par un esprit p., par la douceur,
\item[\vref{1 Th 5:14}] faibles, et soyez p. envers ts.
\item[\vref{2 Pi 3:9}] mais il est p. envers ns., ne
\end{listverse}

\ConcordanceEntry{Patmos}
\vspace{-2mm}
\begin{listverse}
\item[\vref{Ap 1:9}] sur l'île appelée P. à cause de
\end{listverse}

\ConcordanceEntry{Patriarche}
\vspace{-2mm}
\begin{listverse}
\item[\vref{Ac 2:29}] au sujet du p. David, qu'il est
\item[\vref{Ac 7:8}] engendra Jacob, et Jacob les douze p.
\item[\vref{Hé 7:4}] mm Abraham, le p., donna la dîme
\end{listverse}

\ConcordanceEntry{Patrie}
\vspace{-2mm}
\begin{listverse}
\item[\vref{Ge 12:1}] terre, de ta p., et de la
\item[\vref{Ge 24:7}] et de ma p., qui m'a parlé
\item[\vref{Mt 13:54}] rendu ds sa p., il enseignait ds
\item[\vref{Mc 6:4}] que ds sa p., parmi ses parents
\item[\vref{Hé 11:14}] montrent clairement qu'ils cherchent encore lr. p.
\end{listverse}

\ConcordanceEntry{Pâturage}
\vspace{-2mm}
\begin{listverse}
\item[\vref{Ge 47:4}] a plus de p. pour les troupeaux
\item[\vref{Ps 23:2}] ds de verts p., il me dirige
\item[\vref{Ps 65:14}] Les p. se couvrent de brebis, et les
\item[\vref{Ps 79:13}] troupeau de ton p., ns. te louerons
\item[\vref{Ps 95:7}] peuple de son p., et les brebis
\item[\vref{Es 65:10}] Saron servira de p. au menu bétail,
\item[\vref{Jé 23:1}] troupeau de mon p. ! dit Yahweh.
\item[\vref{Ez 34:14}] ds de bons p., et lr. demeure
\item[\vref{Joë 1:18}] n'ont point de p. ! Aussi les troupeaux
\item[\vref{Am 1:2}] Jérus., et les p. des bergers se
\item[\vref{Mi 2:12}] milieu de son p. ; il y aura
\item[\vref{Jn 10:9}] il sortira, et il trouvera des p.
\end{listverse}

\ConcordanceEntry{Paul}
\vspace{-2mm}
\begin{listverse}
\item[\vref{Ac 13:9}] Saul, appelé aussi P., rempli du Saint-Esprit,
\end{listverse}
\begin{legend}
\NoAutoSpaceBeforeFDP{
\item De la tribu de Benjamin : Ph 3:5
\item Né à Tarse en Cilicie : Ac 22:3
\item Saul, respirant encore la menace et le carnage : Ac 9:1-2; 22:4; 26:9; 1 Co 15:9; 1 Ti 1:13
\item Jésus se révèle,la conversion de Paul : Ac 9:3-9; Ga 1:14-16; 1 Ti 1:14-16
\item L'appel dans le service : Ac 13:1-2; 1 Ti 1:12
\item Apôtre des gentils : Ac 9:15-16; Ga 2:8
\item Confirmation de l'apostolat : 1 Co 9:1; Ga 1:1
\item Les missions : Ac 13:4; 15:36; 18:23; 27:1
\item P. chez les anciens : Ac 15:1-2
\item Séparation de P et Barnabas : Ac 15:36-39
\item Paul reprend Pierre à Antioche : Ga 2:11-14
\item P. chassé du temple et brutalisé par les Juifs : Ac 21: 32-33
\item P. emprisonné : Ac 23:23; 24:27; 25:12; 28:14
\item Les afflictions : 2 Co 1:8-10; 11:23-29
\item L'excellence des révélations : 2 Co 12:1-7; Ep 3:3-4
\item Pierre témoigne de Paul : 2 Pi 3:15-16
\item Annonce sa mort : 2 Ti 4:6-8
\item Autres : Ph 2:24; Phm 1:22
}
\end{legend}

\ConcordanceEntry{Pauvre}
\vspace{-2mm}
\begin{listverse}
\item[\vref{Ex 23:3}] n'honoreras point le p. ds son procès.
\item[\vref{Ex 30:15}] rien, et le p. ne diminuera rien
\item[\vref{Lé 25:35}] frère sera devenu p., et qu'il tendra
\item[\vref{De 24:14}] le mercenaire, le p. et l'indigent, d'entre
\item[\vref{Jg 6:15}] est le plus p. en Manassé et
\item[\vref{1 S 2:8}] Il élève le p. de la poussière,
\item[\vref{2 S 12:4}] la brebis du p. hom. et l'a
\item[\vref{Job 29:16}] le père des p., et je m'informais
\item[\vref{Job 34:28}] le cri du p. jusqu'à lui, et
\item[\vref{Ps 41:2}] qui s'intéresse au p. ! Yahweh le délivrera
\item[\vref{Ps 72:12}] il délivrera le p. qui crie vers
\item[\vref{Ps 107:41}] il relève le p. et le délivre
\item[\vref{Pr 13:7}] tel fait le p. et a de
\item[\vref{Pr 14:20}] Le p. est haï mm par son ami,
\item[\vref{Pr 14:31}] fait tort au p. déshonore celui qui
\item[\vref{Pr 18:23}] Le p. ne prononce que des supplications, mais
\item[\vref{Pr 19:4}] celui qui est p. est abandonné mm
\item[\vref{Pr 19:17}] a pitié du p. prête à Yahweh,
\item[\vref{Pr 22:2}] riche et le p. se rencontrent ; celui
\item[\vref{Pr 22:22}] dépouille pas le p., parce qu'il est
\item[\vref{Pr 31:20}] sa main au p., et avance ses
\item[\vref{Ec 9:15}] trouvait un hom. p. et sage qui
\item[\vref{Es 11:4}] il jugera les p. avec justice, et
\item[\vref{Am 2:6}] l'argent, et le p. pour une paire
\item[\vref{Mt 5:3}] Bénis sont les p. en esprit, car
\item[\vref{Mt 11:5}] ressuscités, et l'Evangile est annoncé aux p.
\item[\vref{Mt 26:11}] aurez toujours des p. avec vs. ; mais
\item[\vref{Mc 10:21}] et donne-le aux p., et tu auras
\item[\vref{Mc 12:42}] Et une p. veuve vint, elle y mit deux
\item[\vref{Lu 19:8}] mes biens aux p. ; et si j'ai
\item[\vref{Ro 15:26}] contribution pour les p. parmi les saints
\item[\vref{1 Co 13:3}] la nourriture des p., qnd je livrerais
\item[\vref{2 Co 6:10}] toujours joyeux ; com. p. et toutefois enrichissant
\item[\vref{2 Co 8:9}] riche, s'est fait p. pour vs., afin
\item[\vref{Ja 2:2}] entre aussi un p. misérablement vêtu ;
\item[\vref{Ap 3:17}] es malheureux, misérable, p., aveugle et nu.
\item[\vref{Ap 13:16}] grands, riches et p., libres et esclaves,
\end{listverse}

\ConcordanceEntry{Pauvreté}
\vspace{-2mm}
\begin{listverse}
\item[\vref{Pr 6:11}] et ta p. viendra com. un voyageur, et ta
\item[\vref{Pr 13:18}] La p. et l'ignominie arrivent à celui qui
\item[\vref{Pr 30:8}] me donne ni p. ni richesse, nourris-moi
\item[\vref{Pr 31:7}] qu'il oublie sa p., et ne se
\item[\vref{2 Co 8:2}] et lr. profonde p. s'est répandue en
\item[\vref{2 Co 8:9}] que par sa p. vs. soyez enrichis.
\item[\vref{Ap 2:9}] affliction et ta p., quoique tu sois
\end{listverse}

\ConcordanceEntry{Payer}
\vspace{-2mm}
\begin{listverse}
\item[\vref{Ex 21:2}] sortira pour être libre, sans rien p.
\item[\vref{Mt 5:26}] que tu aies p. le dernier quart
\item[\vref{Mt 17:24}] Votre Maître ne p.-t-il pas les
\item[\vref{Mt 18:25}] pas de quoi p., son maître ordonna
\item[\vref{Mt 20:8}] les ouvriers et p.-lr. le salaire,
\item[\vref{Mt 22:17}] Est-il permis de p. le tribut à
\item[\vref{Ro 13:6}] cela que vs. p. les impôts, parce
\item[\vref{Phm 1:19}] je te le p. ; pour ne pas
\item[\vref{Hé 7:9}] dîmes, les a p. en Abraham ;
\item[\vref{Ap 18:6}] a fait, et p.-lui au double
\end{listverse}

\ConcordanceEntry{Pays}
\vspace{-2mm}
\begin{listverse}
\item[\vref{Ge 12:7}] Je donnerai ce p. à ta postérité.
\item[\vref{Ge 13:9}] Tout le p. n'est-il pas dvt toi ? Sépare-toi je
\item[\vref{Ex 3:8}] remonter de ce p.-là, ds un
\item[\vref{Lé 25:23}] perpétuité ; car le p. est à moi,
\item[\vref{No 14:23}] verront point le p. que j'ai juré
\item[\vref{De 11:12}] c'est un p. dont Yahweh, ton Dieu, prend soin,
\item[\vref{Jos 13:1}] un très grand p. à posséder.
\item[\vref{Jos 18:6}] un plan du p. en sept parts,
\item[\vref{Pr 31:23}] est assis avec les anciens du p.
\item[\vref{Ec 10:16}] Malheur à toi, p. dont le roi
\item[\vref{Da 11:16}] s'arrêtera ds le p. de noblesse, exterminant
\item[\vref{Jon 1:8}] Quel est ton p. et de quel
\item[\vref{Mal 3:12}] vs. serez un p. de délices, dit
\item[\vref{Ac 7:29}] étranger ds le p. de Madian, où
\item[\vref{Hé 8:9}] les tirer du p. d'Egypte ; car ils
\end{listverse}

\ConcordanceEntry{Péager}
\vspace{-2mm}
\begin{listverse}
\item[\vref{Mt 9:9}] au bureau du p., et il lui
\item[\vref{Mt 10:3}] et Matthieu, le p. ; Jacques, fils d'Alphée,
\item[\vref{Ro 13:7}] le tribut, le p. à qui vs.
\end{listverse}

\ConcordanceEntry{Peau}
\vspace{-2mm}
\begin{listverse}
\item[\vref{Ge 3:21}] des tuniques de p., et il les
\item[\vref{Job 2:4}] Yahweh, en disant : P. pour peau, tt
\item[\vref{Job 19:26}] Après que ma p., ce reste aura
\item[\vref{Jé 13:23}] peut-il changer sa p. et le léopard
\item[\vref{La 4:8}] ils ont la p. collée sur les
\item[\vref{Ez 37:6}] et j'étendrai la p. sur vs. ; puis
\item[\vref{Ez 37:8}] la chair, la p. fut étendue par
\item[\vref{Hé 11:37}] là, vêtus de p. de brebis et
\end{listverse}

\ConcordanceEntry{Péché}
\vspace{-2mm}
\begin{listverse}
\item[\vref{Ge 4:7}] agis mal, le p. est couché à
\item[\vref{Ge 18:20}] accru, et lr. p. s'est fort aggravé.
\item[\vref{Lé 19:17}] chargeras point d'un p. à cause de
\item[\vref{No 27:3}] mort ds son p., et il n'avait
\item[\vref{No 32:23}] sachez que votre p. vs. atteindra.
\item[\vref{De 24:16}] on fera mourir chacun pour son p.
\item[\vref{2 S 24:10}] commis un grand p. en faisant cela !
\item[\vref{Job 13:23}] d'iniquités et de p. ? Montre-moi mon crime
\item[\vref{Ps 19:14}] je sois nettoyé de mes grands p. !
\item[\vref{Ps 51:4}] mon iniquité et purifie-moi de mon p.
\item[\vref{Ps 51:5}] transgressions, et mon p. est continuellement dvt
\item[\vref{Ps 103:10}] pas selon nos p., et ne ns.
\item[\vref{Pr 14:21}] prochain commet un p., mais celui qui
\item[\vref{Pr 14:34}] nation, mais le p. est l'ignominie des
\item[\vref{Pr 24:9}] folie n'est que p., et le moqueur
\item[\vref{Es 1:18}] droits. Si vos p. sont com. l'écarlate,
\item[\vref{Es 6:7}] la propitiation est faite pour ton p.
\item[\vref{Es 30:1}] point, afin d'ajouter p. sur péché.
\item[\vref{Es 38:17}] jeté ts mes p. derrière ton dos.
\item[\vref{Es 44:22}] épaisse, et tes p. com. une nuée ;
\item[\vref{Es 53:5}] transpercé pour nos p., brisé pour nos
\item[\vref{Es 53:10}] sacrifice pour le p., il verra une
\item[\vref{Es 53:12}] a porté les p. de plusieurs, et
\item[\vref{Es 59:2}] ce sont vos p. qui vs. cachent
\item[\vref{Es 59:20}] convertiront de lr. p., dit Yahweh.
\item[\vref{Ez 23:49}] vs. porterez les p. de vos idoles ;
\item[\vref{Da 8:13}] et sur le p. qui cause la
\item[\vref{Mi 7:19}] jettera ts nos p. au fond de
\item[\vref{Za 13:1}] Jérus., pour le p. et pour la
\item[\vref{Mt 1:21}] qui sauvera son peuple de ses p.
\item[\vref{Lu 7:49}] est celui-ci qui pardonne mm les p. ?
\item[\vref{Jn 1:29}] qui ôte le p. du monde.
\item[\vref{Jn 8:7}] qui est sans p. jette le premier
\item[\vref{Jn 8:34}] se livre au p. est esclave du
\item[\vref{Jn 8:46}] me convaincra de p. ? Et si je
\item[\vref{Jn 9:34}] né ds le p., et tu ns.
\item[\vref{Jn 9:41}] n'auriez pas de p. ; mais mntnt vs.
\item[\vref{Jn 15:22}] n'auraient pas de p., mais mntnt ils
\item[\vref{Jn 16:8}] le monde de p., de justice, et
\item[\vref{Jn 19:11}] à toi commet un plus grand p.
\item[\vref{Ac 7:60}] impute pas ce p. ! Et qnd il
\item[\vref{Ac 22:16}] purifié de tes p., en invoquant le
\item[\vref{Ro 3:9}] Juifs que Grecs, sont assujettis au p.,
\item[\vref{Ro 3:20}] loi est donnée la connaissance du p.
\item[\vref{Ro 4:7}] et dont les p. sont couverts !
\item[\vref{Ro 5:12}] seul hom. le p. est entré ds
\item[\vref{Ro 5:18}] par un seul p. ts les hommes
\item[\vref{Ro 5:21}] que, com. le p. a régné par
\item[\vref{Ro 6:1}] Demeurerions-ns. ds le p., afin que la
\item[\vref{Ro 6:12}] Que le p. ne règne dc pas ds votre
\item[\vref{Ro 6:14}] Car le p. ne dominera pas sur vs., parce
\item[\vref{Ro 6:20}] étiez esclaves du p., vs. étiez libres
\item[\vref{Ro 6:23}] le salaire du p., c'est la mort ;
\item[\vref{Ro 7:9}] commandement vint, le p. commença à revivre,
\item[\vref{Ro 7:13}] Nullement ! Mais le p., afin qu'il parût
\item[\vref{Ro 7:17}] mais c'est le p. qui habite en
\item[\vref{Ro 8:3}] à celle du p. ; et pour le
\item[\vref{Ro 14:23}] pas de la foi est un p.
\item[\vref{1 Co 6:18}] fornication. Quelque autre p. qu'un hom. commette,
\item[\vref{2 Co 5:21}] pas connu le p., il l'a fait
\item[\vref{Ga 3:22}] hommes sous le p., afin que ce
\item[\vref{2 Th 2:3}] que l'hom. de p., le fils de
\item[\vref{1 Ti 5:24}] Les p. de certains hommes sont manifestes, mm
\item[\vref{Hé 3:13}] ne s'endurcisse par la séduction du p.
\item[\vref{Hé 9:26}] pour l'abolition du p. par son sacrifice.
\item[\vref{Hé 9:28}] pour ôter les p. de plusieurs, apparaîtra
\item[\vref{Hé 10:11}] qui ne peuvent jamais ôter les p.,
\item[\vref{Hé 11:25}] peu de temps des délices du p.
\item[\vref{Hé 12:1}] fardeau, et le p. qui ns. enveloppe
\item[\vref{Ja 1:15}] elle enfante le p. ; et le péché,
\item[\vref{Ja 2:9}] vs. commettez un p., et vs. êtes
\item[\vref{Ja 4:17}] a dc le p. en celui qui
\item[\vref{1 Pi 2:24}] porté lui-mm nos p. ds son corps
\item[\vref{1 Jn 1:8}] n'avons point de p., ns. ns. séduisons
\item[\vref{1 Jn 3:4}] loi, car le p. est la transgression
\item[\vref{1 Jn 3:5}] pour ôter nos p., et il n'y
\item[\vref{1 Jn 3:8}] vit ds le p. est du diable,
\item[\vref{1 Jn 3:9}] pas ds le p., car la semence
\item[\vref{1 Jn 5:16}] frère commettre un p. qui ne mène
\item[\vref{Ap 1:6}] lavés de nos p. ds son sang,
\item[\vref{Ap 18:5}] Car ses p. sont montés jusqu'au ciel, et Dieu
\end{listverse}

\ConcordanceEntry{Pécher}
\vspace{-2mm}
\begin{listverse}
\item[\vref{Ex 9:27}] lr. dit : J'ai p. cette fois ; Yahweh
\item[\vref{Jos 7:20}] et dit : J'ai p. il est vrai,
\item[\vref{1 S 2:24}] bon ; vs. faites p. le peuple de
\item[\vref{1 S 15:24}] à Samuel : J'ai p. parce que j'ai
\item[\vref{2 S 12:13}] à Nathan : J'ai p. contre Yahweh ! Et
\item[\vref{1 R 8:46}] Quand ils p. contre toi, car
\item[\vref{Né 1:8}] de dire : Vous p. et je vs.
\item[\vref{Job 1:22}] cela, Job ne p. pas et n'attribua
\item[\vref{Ps 4:5}] Tremblez, et ne p. point ; parlez en
\item[\vref{Ps 51:6}] J'ai p. contre toi, contre toi seul, et
\item[\vref{Ps 78:17}] ils continuèrent à p. contre lui, irritant
\item[\vref{Ps 106:6}] Nous avons p. avec nos pères,
\item[\vref{Ps 119:11}] de ne pas p. contre toi.
\item[\vref{Ec 5:5}] bouche de faire p. ta chair, et
\item[\vref{Jé 2:35}] tu as dit : Je n'ai point p.
\item[\vref{Ez 37:23}] lesquelles ils ont p., et je les
\item[\vref{Da 9:15}] aujourd'hui, ns. avons p., ns. avons été
\item[\vref{Am 4:4}] ds Béthel, et p. ! Commettez-y vos crimes ;
\item[\vref{Ha 2:10}] contre ton âme que tu as p.
\item[\vref{So 3:11}] lesquelles tu as p. contre moi ; parce
\item[\vref{Mt 18:15}] ton frère a p. contre toi, va,
\item[\vref{Mt 18:21}] frère lorsqu'il aura p. contre moi ? Sera-ce
\item[\vref{Lu 15:18}] Mon père, j'ai p. contre le ciel
\item[\vref{Lu 17:4}] Et s'il a p. contre toi sept
\item[\vref{Jn 9:2}] Maître, qui a p. ? Celui-ci, ou son
\item[\vref{Jn 9:3}] sa mère n'ont p. ; mais c'est afin
\item[\vref{Ro 2:12}] ceux qui auront p. sans la loi
\item[\vref{Ro 3:23}] Car ts ont p. et n’atteignent
\item[\vref{Ro 5:14}] qui n'avaient pas p. par une transgression
\item[\vref{1 Co 8:12}] Or qnd vs. p. ainsi contre vos
\item[\vref{Hé 10:26}] Car, si ns. p. volontairement après avoir
\item[\vref{Ja 3:2}] Car ns. p. ts en plusieurs choses. Si quelqu'un
\item[\vref{2 Pi 2:4}] anges qui ont p., mais s'il les
\item[\vref{1 Jn 2:1}] que vs. ne p. point. Et si
\end{listverse}

\ConcordanceEntry{Pêcher}
\vspace{-2mm}
\begin{listverse}
\item[\vref{Jé 16:16}] et ils les p. ; et ensuite, j'enverrai
\item[\vref{Lu 5:4}] eau, et jetez vos filets pour p.
\item[\vref{Jn 21:3}] dit : Je vais p.. Ils lui dirent :
\end{listverse}

\ConcordanceEntry{Pécheresse}
\vspace{-2mm}
\begin{listverse}
\item[\vref{Es 1:4}] Ah ! Nation p., peuple chargé d'iniquités,
\item[\vref{Mc 8:38}] génération adultère et p., le Fils de
\item[\vref{Lu 7:37}] ville une fem. p., qui, ayant su
\end{listverse}

\ConcordanceEntry{Pécheur}
\vspace{-2mm}
\begin{listverse}
\item[\vref{Ge 13:13}] et de grands p. contre Yahweh.
\item[\vref{Ps 1:1}] la voie des p., qui ne s'assied
\item[\vref{Ps 25:8}] il enseigne aux p. le chemin qu'ils
\item[\vref{Ps 26:9}] âme avec les p., ma vie avec
\item[\vref{Ps 51:15}] transgresseurs et les p. reviendront à toi.
\item[\vref{Pr 1:10}] fils, si les p. veulent t'attirer, ne
\item[\vref{Pr 13:21}] mal poursuit les p., mais le bien
\item[\vref{Pr 23:17}] pas d'envie aux p., mais sois tt
\item[\vref{Ec 2:26}] il donne au p. de l'occupation à
\item[\vref{Ec 9:18}] un seul hom. p. détruit beaucoup de
\item[\vref{Es 65:20}] jeune ; mais le p. âgé de cent
\item[\vref{Am 9:10}] Tous les p. de mon peuple mourront par l'épée,
\item[\vref{Mt 9:13}] la repentance les justes, mais les p.
\item[\vref{Lu 5:8}] moi, car je suis un hom. p.
\item[\vref{Lu 6:32}] en saura-t-on ? Les p. aussi aiment ceux
\item[\vref{Lu 13:2}] de plus grands p. que ts les
\item[\vref{Lu 15:7}] pour un seul p. qui se repent,
\item[\vref{Lu 18:13}] sois apaisé envers moi qui suis p. !
\item[\vref{Jn 9:16}] Comment un hom. p. peut-il faire de
\item[\vref{Jn 9:25}] si c'est un p. ; je sais une
\item[\vref{Jn 9:31}] n'exauce pas les p. ; mais si quelqu'un
\item[\vref{Ro 5:8}] ns. étions encore p., Christ est mort
\item[\vref{Ro 5:19}] ont été rendus p., de mm par
\item[\vref{1 Ti 1:9}] impies et les p., pour les irréligieux
\item[\vref{1 Ti 1:15}] pour sauver les p., dont je suis
\item[\vref{Hé 7:26}] tache, séparé des p., et élevé au-dessus
\item[\vref{Hé 12:3}] la part des p., afin que vs.
\item[\vref{Ja 4:8}] s'approchera de vs.. P., nettoyez vos mains ;
\item[\vref{Ja 5:20}] qui ramènera un p. de son égarement,
\item[\vref{1 Pi 4:18}] sauvé, que deviendront l'impie et le p. ?
\end{listverse}

\ConcordanceEntry{Pêcheur}
\vspace{-2mm}
\begin{listverse}
\item[\vref{Es 19:8}] Et les p. gémiront, ts ceux qui jettent l'hameçon
\item[\vref{Jé 16:16}] Voici, j'envoie plusieurs p., dit Yahweh, et
\item[\vref{Ez 47:10}] arrivera que des p. se tiendront le
\item[\vref{Mt 4:19}] je vs. ferai p. d'hommes.
\item[\vref{Mt 13:48}] est rempli, les p. le tirent en
\end{listverse}

\ConcordanceEntry{Peine}
\vspace{-2mm}
\begin{listverse}
\item[\vref{Ge 3:17}] fruits ds la p., ts les jours
\item[\vref{Ge 4:13}] à Yahweh : Ma p. est plus grande
\item[\vref{No 14:34}] vs. porterez la p. de vos iniquités
\item[\vref{Ps 32:5}] as ôté la p. de mon péché.
\item[\vref{Ps 73:5}] de part aux p. des humains, et
\item[\vref{Ps 90:10}] tirent n'est que p. et misère ; car
\item[\vref{Es 65:6}] ferai porter la p., oui je lr.
\item[\vref{Za 14:19}] Ce sera la p. du péché de
\item[\vref{Mt 6:34}] lui-mm. A chaque jour suffit sa p.
\item[\vref{Mc 14:6}] faites-vs. de la p. ? Elle a fait
\item[\vref{Ro 5:7}] Or à grande p. arrive-t-il que quelqu'un
\item[\vref{2 Co 11:27}] ds les p. et ds le travail, ds de
\item[\vref{2 Th 3:8}] et ds la p., ns. avons travaillé
\end{listverse}

\ConcordanceEntry{Pékach}
\vspace{-2mm}
\begin{listverse}
\item[\vref{2 R 15:25}] P., fils de Remalia, son officier, conspira
\item[\vref{2 R 15:31}] des actions de P., tt ce qu'il
\item[\vref{2 Ch 28:6}] Car P., fils de Remalia, tua en un
\item[\vref{Es 7:1}] de Syrie, et P., fils de Remalia,
\end{listverse}

\ConcordanceEntry{Pékachia}
\vspace{-2mm}
\begin{listverse}
\item[\vref{2 R 15:22}] ses pères, et P., son fils, régna
\item[\vref{2 R 15:26}] des actions de P. tt ce qu'il
\end{listverse}

\ConcordanceEntry{Penchant}
\vspace{-2mm}
\begin{listverse}
\item[\vref{De 29:19}] marche ds les p. de mon cœur,
\item[\vref{Ps 81:13}] ai abandonnés aux p. de lr. cœur,
\item[\vref{Jé 3:17}] plus suivant les p. de lr. mauvais
\item[\vref{Jé 7:24}] d'autres conseils, les p. de lr. mauvais
\item[\vref{Jé 18:12}] fera selon les p. de son mauvais
\item[\vref{Jé 23:17}] marchent suivant les p. de lr. cœur :
\item[\vref{2 Pi 2:12}] s'abandonnent à leurs p. naturels, et qui
\end{listverse}

\ConcordanceEntry{Pendre}
\vspace{-2mm}
\begin{listverse}
\item[\vref{Ge 40:19}] et te fera p. à un bois,
\item[\vref{De 21:23}] celui qui est p. est malédiction de
\item[\vref{Jos 8:29}] Puis il fit p. le roi d'Aï
\item[\vref{Jos 10:26}] il les fit p. à cinq arbres,
\item[\vref{2 S 21:6}] et ns. les p. dvt Yahweh à
\item[\vref{2 S 21:12}] Philistins les avaient p. lorsqu'ils tuèrent Saül
\item[\vref{Est 2:23}] deux eunuques furent p. à un bois
\item[\vref{Est 8:7}] il a été p. au bois pour
\item[\vref{La 5:12}] chefs ont été p. par leurs mains ;
\item[\vref{Mt 27:5}] il se retira et alla se p.
\item[\vref{Ac 5:30}] vs. avez fait mourir en le p. au bois.
\item[\vref{Ga 3:13}] est écrit : Maudit est quiconque est p. au bois,
\end{listverse}

\ConcordanceEntry{Peniel}
\vspace{-2mm}
\begin{listverse}
\item[\vref{Ge 32:30}] du nom de P. ; car, dit-il, j'ai
\end{listverse}

\ConcordanceEntry{Pensée}
\vspace{-2mm}
\begin{listverse}
\item[\vref{Ge 6:5}] tte l'imagination des p. de lr. cœur
\item[\vref{1 Ch 28:9}] les dispositions des p.. Si tu le
\item[\vref{Job 4:13}] nuit agitent la p., qnd un profond
\item[\vref{Ps 10:4}] point de Dieu : Voilà ttes ses p.
\item[\vref{Ps 17:3}] rien trouvé : Ma p. ne va point
\item[\vref{Ps 56:6}] et ttes leurs p. tendent à me
\item[\vref{Ps 92:6}] sont magnifiques ! Tes p. sont merveilleusement profondes.
\item[\vref{Ps 94:11}] Yahweh connaît les p. des hommes qui
\item[\vref{Ps 139:2}] lève ; tu discernes de loin ma p.
\item[\vref{Ps 139:17}] Dieu ! que tes p. me sont précieuses !
\item[\vref{Pr 12:2}] condamne l'hom. qui a des mauvaises p.
\item[\vref{Pr 12:5}] Les p. des justes ne sont que jugement,
\item[\vref{Pr 15:26}] Les p. du malin sont en abomination à
\item[\vref{Ec 9:10}] ni œuvre, ni p., ni connaissance, ni
\item[\vref{Ec 10:20}] mm ds ta p., et ne maudis
\item[\vref{Es 55:7}] l'hom. injuste ses p. ; et qu'il retourne
\item[\vref{Es 55:8}] Car mes p. ne sont pas vos pensées, et
\item[\vref{Jé 18:12}] ns. suivrons nos p., chacun de ns.
\item[\vref{Mi 4:12}] connaissent point les p. de Yahweh, et
\item[\vref{Mt 9:4}] Jésus, connaissant leurs p., lr. dit : Pourquoi
\item[\vref{Mt 15:19}] sortent les mauvaises p., les meurtres, les
\item[\vref{Lu 2:35}] sorte que les p. de beaucoup de
\item[\vref{Ac 4:25}] et ces vaines p. parmi les peuples ?
\item[\vref{Ac 8:22}] est possible, la p. de ton cœur
\item[\vref{Ro 2:15}] témoignage, et leurs p. s'accusant entre elles
\item[\vref{Ro 8:27}] quelle est la p. de l'Esprit, car
\item[\vref{Ro 11:34}] a connu la p. du Seign., ou
\item[\vref{1 Co 2:16}] a connu la p. du Seign. pour
\item[\vref{2 Co 10:5}] et amenant tte p. captive à l'obéissance
\item[\vref{2 Co 11:3}] sa ruse, vos p. aussi ne se
\item[\vref{Ep 4:17}] qui suivent la vanité de leurs p.
\item[\vref{Hé 4:12}] elle juge les p. et les intentions
\item[\vref{Ja 4:16}] glorifiez ds vos p. orgueilleuses. Toute fierté
\item[\vref{1 Pi 4:1}] de la mm p.. Car celui qui
\end{listverse}

\ConcordanceEntry{Penser}
\vspace{-2mm}
\begin{listverse}
\item[\vref{Ge 40:23}] des échansons ne p. plus à Joseph ;
\item[\vref{Ge 48:11}] Joseph : Je ne p. pas revoir ton
\item[\vref{Jos 20:3}] involontairement sans y p., s'y enfuie ; et
\item[\vref{Ps 77:6}] Je p. aux jours d'autrefois et aux années
\item[\vref{Es 41:20}] qu'on sache, qu'on p., et qu'on comprenne
\item[\vref{Es 43:18}] Ne p. plus aux choses passées, et ne
\item[\vref{Es 46:8}] rappelez-le à votre p., ô vs. transgresseurs !
\item[\vref{Jon 1:6}] Peut-être ton Dieu p. à ns. et
\item[\vref{Mal 3:16}] Yahweh et qui p. à son Nom.
\item[\vref{Mt 1:20}] com. il y p., voici, l'Ange du
\item[\vref{Lu 9:7}] ne savait que p.. Car quelques-uns disaient
\item[\vref{Lu 12:40}] viendra à l'heure où vs. n'y p. pas.
\item[\vref{1 Co 4:9}] Car je p. que Dieu ns. a exposés publiquement,
\item[\vref{1 Co 10:12}] celui dc qui p. demeurer debout prenne
\item[\vref{1 Co 13:11}] un enfant, je p. com. un enfant ;
\item[\vref{Ga 6:3}] Car si quelqu'un p. être qq chose,
\item[\vref{Ep 3:20}] tt ce que ns. demandons et p.,
\item[\vref{Ph 4:8}] qq louange ; soient l'objet de vos p.
\item[\vref{Col 3:2}] P. aux choses d'en haut, et non
\item[\vref{Ja 1:26}] quelqu'un parmi vs. p. être religieux, et
\item[\vref{Ja 4:5}] P.-vs. que l'Ecriture parle en vain ?
\end{listverse}

\ConcordanceEntry{Penuel}
\vspace{-2mm}
\begin{listverse}
\item[\vref{Jg 8:8}] il monta à P. et il fit
\item[\vref{Jg 8:17}] la tour de P. et tua les
\end{listverse}

\ConcordanceEntry{Percer}
\vspace{-2mm}
\begin{listverse}
\item[\vref{Ex 19:13}] sera lapidé, ou p. de flèches ; soit
\item[\vref{De 15:17}] et tu lui p. l'oreille contre la
\item[\vref{Ps 22:17}] m'entoure, ils ont p. mes mains et
\item[\vref{Ps 40:7}] offrande ; tu m'as p. les oreilles ; tu
\item[\vref{Ez 8:8}] Fils de l'hom., p. mntnt le mur ;
\item[\vref{Ag 1:6}] mettre son salaire ds un sac p.
\item[\vref{Za 12:10}] celui qu'ils ont p., et ils pleureront
\item[\vref{Mt 6:19}] où les voleurs p. et dérobent ;
\item[\vref{Mt 6:20}] les voleurs ne p. ni ne dérobent.
\item[\vref{Mt 24:43}] ne laisserait pas p. sa maison.
\item[\vref{Mc 2:4}] était, et l'ayant p., ils descendirent le
\item[\vref{Jn 19:37}] dit : Ils verront celui qu'ils ont p.
\item[\vref{Ap 1:7}] ceux qui l'ont p. ; et ttes les
\end{listverse}

\ConcordanceEntry{Perdition}
\vspace{-2mm}
\begin{listverse}
\item[\vref{No 22:32}] est dvt moi une voie de p.
\item[\vref{2 R 23:13}] la montagne de p., que Salomon, roi
\item[\vref{Job 31:3}] La p. n'est-elle pas pour l'injuste, et les
\item[\vref{Ps 55:24}] puits de la p. ; les hommes sanguinaires
\item[\vref{Mt 7:13}] mènent à la p., et il y
\item[\vref{Jn 17:12}] le fils de p., afin que l'Ecriture
\item[\vref{Ro 9:22}] vases de colère, préparés pour la p.,
\item[\vref{Ph 1:28}] une preuve de p., mais pour vs.
\item[\vref{Ph 3:19}] fin est la p., qui ont pour
\item[\vref{2 Th 2:3}] fils de la p., soit révélé,
\item[\vref{1 Ti 6:9}] hommes ds la ruine et la p.
\item[\vref{2 Pi 2:2}] leurs sectes de p.; et à cause
\item[\vref{2 Pi 3:16}] les autres Ecritures, à lr. propre p.
\item[\vref{Ap 17:8}] aller à la p.. Et les habitants
\end{listverse}

\ConcordanceEntry{Perdre}
\vspace{-2mm}
\begin{listverse}
\item[\vref{De 32:28}] nation qui se p. par ses conseils,
\item[\vref{Jos 2:9}] habitants du pays p. courage à cause
\item[\vref{Pr 1:32}] et la prospérité des insensés les p.
\item[\vref{Pr 24:10}] Si tu p. courage au jour de la détresse,
\item[\vref{Ec 3:6}] un temps pour p. ; un temps pour
\item[\vref{Es 6:5}] moi ! Je suis p., car je suis
\item[\vref{Jé 31:28}] pour détruire, pour p. et pour faire
\item[\vref{Mt 5:13}] si le sel p. sa saveur, avec
\item[\vref{Mt 16:25}] son âme la p. ; mais quiconque perdra
\item[\vref{Mt 16:26}] le monde, s'il p. son âme ? Ou,
\item[\vref{Lu 9:56}] pas venu pour p. les âmes des
\item[\vref{Lu 15:24}] ressuscité ; il était p., mais il est
\item[\vref{Jn 6:39}] que je ne p. rien de tt
\item[\vref{Jn 12:25}] sa vie la p., et celui qui
\item[\vref{Jn 17:12}] d'eux ne s'est p., sinon le fils
\item[\vref{Ac 27:34}] aucun de vos cheveux ne se p.
\item[\vref{Ep 4:27}] pas lieu au diable de vs. p.
\item[\vref{Hé 10:39}] retirent pour se p., mais de ceux
\item[\vref{Hé 12:3}] que vs. ne succombiez pas, en p. courage.
\item[\vref{2 Jn 1:8}] que vs. ne p. pas le fruit
\end{listverse}

\ConcordanceEntry{Père}
\vspace{-2mm}
\begin{listverse}
\item[\vref{Ge 2:24}] l'hom. quittera son p. et sa mère
\item[\vref{Ge 12:1}] maison de ton p., vers la terre
\item[\vref{Ge 15:15}] iras vers tes p. en paix, et
\item[\vref{Ge 17:4}] et tu deviendras p. d'une multitude de
\item[\vref{Ge 45:8}] il m'a établi p. de Pharaon, et
\item[\vref{Ex 20:12}] Honore ton p. et ta mère,
\item[\vref{Ex 21:15}] aura frappé son p. ou sa mère,
\item[\vref{De 4:37}] a aimé tes p., il a choisi
\item[\vref{De 32:6}] N'est-il pas ton p., celui qui t'a
\item[\vref{De 33:9}] dit de son p. et de sa
\item[\vref{Jos 24:2}] Dieu d'Israël : Vos p., Térach, père d'Abraham
\item[\vref{Ps 27:10}] Car mon p. et ma mère m'abandonnent, mais Yahweh
\item[\vref{Ps 103:13}] Comme un p. a compassion de ses fils, Yahweh
\item[\vref{Ps 106:7}] Nos p. n'ont point été attentifs à tes
\item[\vref{Pr 1:8}] l'instruction de ton p., et n'abandonne pas
\item[\vref{Pr 17:6}] vieillards, et les p. sont la gloire
\item[\vref{Pr 17:21}] l'ennui, et le p. du sot ne
\item[\vref{Es 9:5}] Dieu Puissant, le P. d'éternité, le Prince
\item[\vref{Es 43:27}] Ton premier p. a péché, et
\item[\vref{Es 63:16}] tu es notre P., car Abraham ne
\item[\vref{Jé 3:19}] Tu m'appelleras : Mon p. ! Et tu ne
\item[\vref{Jé 31:9}] j'ai été un p. pour Israël, et
\item[\vref{Mal 1:6}] fils honore son p., et un serviteur
\item[\vref{Mal 2:10}] ts un seul P. ? N'est-ce pas un
\item[\vref{Mal 4:6}] le cœur des p. à leurs enfants,
\item[\vref{Mt 3:9}] avons Abraham pour p. ! Car je vs.
\item[\vref{Mt 6:9}] devez prier : Notre P. qui es aux
\item[\vref{Mt 6:32}] choses. Car votre P. céleste sait que
\item[\vref{Mt 10:21}] mort, et le p. son enfant ; et
\item[\vref{Mt 10:37}] qui aime son p. ou sa mère
\item[\vref{Mt 11:27}] données par mon P., et personne ne
\item[\vref{Mt 15:5}] dira à son p. ou à sa
\item[\vref{Mt 19:29}] ou sœurs, ou p., ou mère, ou
\item[\vref{Mt 23:9}] la terre votre p. ; car un seul
\item[\vref{Lu 11:11}] parmi vs. le p. qui donnera une
\item[\vref{Lu 15:18}] j'irai vers mon p., et je lui
\item[\vref{Lu 24:49}] promesse de mon P., mais vs. dc
\item[\vref{Jn 3:35}] Le P. aime le Fils, et il a
\item[\vref{Jn 5:17}] lr. répondit : Mon P. travaille jusqu'à présent ;
\item[\vref{Jn 5:18}] était son propre P., se faisant égal
\item[\vref{Jn 6:44}] moi, si le P. qui m'a envoyé
\item[\vref{Jn 6:46}] n'a vu le P., sinon celui qui
\item[\vref{Jn 8:19}] Où est ton P. ? Jésus répondit : Vous
\item[\vref{Jn 8:41}] œuvres de votre p.. Et ils lui
\item[\vref{Jn 8:44}] Le p. dont vs. êtes issus c'est le
\item[\vref{Jn 10:15}] com. le P. me connaît, je connais aussi le
\item[\vref{Jn 14:6}] ne vient au P. que par moi.
\item[\vref{Jn 14:10}] SUIS en mon P., et que le
\item[\vref{Jn 14:28}] m'en vais au P., car le Père
\item[\vref{Jn 16:27}] car le P. lui-mm vs. aime, parce que vs.
\item[\vref{Jn 17:11}] vais à toi. P. saint, garde-les en
\item[\vref{Jn 20:17}] monté vers mon P.. Mais va vers
\item[\vref{Ac 7:51}] faites com. vos p. ont fait.
\item[\vref{Ro 4:11}] afin d'être le p. de ts les
\item[\vref{Ro 8:15}] par lequel ns. crions: Abba ! C'est-à-dire P.
\item[\vref{Ro 11:28}] sont aimés à cause de leurs p.
\item[\vref{1 Co 4:15}] pourtant pas plusieurs p., car c'est moi
\item[\vref{1 Co 5:1}] vs. a la fem. de son p.
\item[\vref{1 Co 8:6}] qui est le P., de qui viennent
\item[\vref{2 Co 6:18}] pour vs. un P. et vs. serez
\item[\vref{Ep 4:6}] seul Dieu et P. de ts, qui
\item[\vref{Ep 6:1}] obéissez à vos p. et à vos
\item[\vref{Hé 7:3}] Il est sans p., sans mère, sans
\item[\vref{Hé 12:9}] Et puisque nos p. selon la chair
\item[\vref{Ja 1:17}] et descendent du P. des lumières, en
\item[\vref{1 Pi 1:17}] invoquez com. votre P. celui qui juge
\item[\vref{1 Jn 2:13}] P., je vs. écris parce que vs.
\end{listverse}

\ConcordanceEntry{Pérets}
\vspace{-2mm}
\begin{listverse}
\item[\vref{Ge 38:29}] elle lui donna le nom de P.
\item[\vref{Ru 4:12}] la maison de P., que Tamar enfanta
\item[\vref{Ru 4:18}] la généalogie de P.. Pérets engendra Hetsron ;
\end{listverse}

\ConcordanceEntry{Pérets-Uzza}
\vspace{-2mm}
\begin{listverse}
\item[\vref{2 S 6:8}] appelé ce lieu jusqu'à ce jour P.
\item[\vref{1 Ch 13:11}] jour ce lieu-là P., brèche d'Uzza.
\end{listverse}

\ConcordanceEntry{Perfection}
\vspace{-2mm}
\begin{listverse}
\item[\vref{Job 28:3}] qu'on découvre la p. de ttes choses,
\item[\vref{Es 18:5}] vient en sa p., et que la
\item[\vref{Es 47:9}] toi ds lr. p., pour le grand
\item[\vref{Ro 1:20}] En effet, les p. invisibles de Dieu,
\item[\vref{1 Co 13:10}] Mais qnd la p. sera venue, alors
\item[\vref{Col 3:14}] qui est le lien de la p.
\item[\vref{Hé 6:1}] tendons à la p., ne posant pas
\item[\vref{Hé 7:11}] Si dc la p. s'était trouvée ds
\item[\vref{Hé 7:19}] amené à la p., mais ce qui
\item[\vref{Hé 11:40}] pas à la p. sans ns.
\end{listverse}

\ConcordanceEntry{Perfectionnement}
\vspace{-2mm}
\begin{listverse}
\item[\vref{2 Co 13:9}] ce que ns. demandons, c'est votre p.
\item[\vref{Ep 4:12}] pour travailler au p. des saints, pour
\end{listverse}

\ConcordanceEntry{Perfectionner}
\vspace{-2mm}
\begin{listverse}
\item[\vref{2 Co 7:1}] et de l'esprit, p. la sanctification ds
\item[\vref{2 Co 13:11}] mes frères, réjouissez-vs., p.-vs., consolez-vs., ayez
\item[\vref{1 Th 4:10}] prions de vs. p. ts les jours
\end{listverse}

\ConcordanceEntry{Pergame}
\vspace{-2mm}
\begin{listverse}
\item[\vref{Ap 1:11}] à Smyrne, à P., à Thyatire, à
\item[\vref{Ap 2:12}] de l'église de P. : Voici ce que
\end{listverse}

\ConcordanceEntry{Perge}
\vspace{-2mm}
\begin{listverse}
\item[\vref{Ac 13:13}] ils vinrent à P., ville de Pamphylie.
\item[\vref{Ac 14:25}] la parole à P., et descendirent à
\end{listverse}

\ConcordanceEntry{Périr}
\vspace{-2mm}
\begin{listverse}
\item[\vref{Ge 18:23}] et dit : Feras-tu p. le juste avec
\item[\vref{No 17:12}] ns. expirons, ns. p., ns. périssons ts !
\item[\vref{1 S 27:1}] cœur : Certes je p. un jour par
\item[\vref{Est 4:16}] si je dois p., je périrai.
\item[\vref{Ps 73:27}] s'éloignent de toi p. ; tu retrancheras ts
\item[\vref{Ps 102:27}] Ils p., mais tu subsisteras ; ils s'useront ts
\item[\vref{Ez 33:10}] et que ns. p. à cause d'eux,
\item[\vref{Jon 1:6}] pensera à ns. et ns. ne p. pas.
\item[\vref{Mt 8:25}] en lui disant : Seign., sauve-ns., ns. p. !
\item[\vref{Mt 10:28}] qui peut faire p. et l'âme et
\item[\vref{Mt 12:14}] sur les moyens de le faire p.
\item[\vref{Mt 18:14}] cieux qu'un seul de ces petits p.
\item[\vref{Mt 27:20}] de demander Barabbas et de faire p. Jésus.
\item[\vref{Mc 4:38}] t'inquiètes-tu pas de ce que ns. p. ?
\item[\vref{Lu 13:5}] repentez pas, vs. p. ts de la
\item[\vref{Jn 3:16}] en lui ne p. pas, mais qu'il
\item[\vref{Jn 6:27}] la nourriture qui p., mais pour celle
\item[\vref{Jn 10:28}] éternelle, elles ne p. jamais, et personne
\item[\vref{Jn 11:50}] et que tte la nation ne p. pas.
\item[\vref{Ac 7:19}] enfants à l'abandon, afin d'en faire p. la race.
\item[\vref{Ac 8:20}] Que ton argent p. avec toi, puisque
\item[\vref{Ro 2:12}] sans la loi p. aussi sans la
\item[\vref{1 Co 1:18}] pour ceux qui p., mais pour ns.
\item[\vref{1 Co 10:10}] eux murmurèrent et p. par le destructeur.
\item[\vref{1 Co 13:8}] La charité ne p. jamais. Les prophéties
\item[\vref{2 Co 2:15}] sont sauvés et parmi ceux qui p.
\item[\vref{2 Th 2:10}] pour ceux qui p. parce qu'ils n'ont
\item[\vref{2 Pi 2:12}] ignorent, et ils p. par lr. propre
\item[\vref{2 Pi 3:6}] le monde d'alors p., étant submergé par
\item[\vref{2 Pi 3:9}] voulant qu'aucun ne p., mais que ts
\end{listverse}

\ConcordanceEntry{Perle}
\vspace{-2mm}
\begin{listverse}
\item[\vref{Job 28:18}] la sagesse vaut plus que les p.
\item[\vref{Pr 8:11}] mieux que les p., et tt ce
\item[\vref{Pr 31:10}] son prix surpasse de beaucoup les p.
\item[\vref{Mt 7:6}] jetez pas vos p. dvt les pourceaux,
\item[\vref{Mt 13:46}] a trouvé une p. de grand prix
\item[\vref{1 Ti 2:9}] d'or, ni de p., ni d'habits somptueux,
\item[\vref{Ap 18:12}] pierres précieuses, de p., de fin lin,
\item[\vref{Ap 21:21}] portes étaient douze p. ; chacune des portes
\end{listverse}

\ConcordanceEntry{Permettre}
\vspace{-2mm}
\begin{listverse}
\item[\vref{Esd 4:2}] et lr. dirent : P. que ns. bâtissions
\item[\vref{Ps 16:10}] scheol, tu ne p. point que ton
\item[\vref{Ps 121:3}] Il ne p. point que ton pied chancelle, celui
\item[\vref{Os 5:4}] habitudes ne lr. p. pas de revenir
\item[\vref{Mt 14:36}] prièrent de lr. p. de toucher seulement
\item[\vref{Lu 1:74}] de ns. p., après que ns. serions délivrés de
\item[\vref{Lu 4:41}] et ne lr. p. pas de dire
\item[\vref{Lu 8:32}] Jésus de lr. p. d'entrer ds ces
\item[\vref{Ac 13:35}] endroit : Tu ne p. pas que ton
\item[\vref{Ac 27:7}] vent ne ns. p. pas d'avancer, ns.
\item[\vref{1 Co 10:13}] est fidèle ne p. pas que vs.
\item[\vref{Ap 11:9}] et ils ne p. pas que leurs
\end{listverse}

\ConcordanceEntry{Pernicieux}
\vspace{-2mm}
\begin{listverse}
\item[\vref{Ps 52:6}] ts les discours p., le langage trompeur !
\item[\vref{Ps 73:8}] Ils sont p., et parlent méchamment d'opprimer ; ils parlent
\item[\vref{Pr 12:13}] lèvres un piège p., mais le juste
\item[\vref{Es 32:7}] de l'avare sont p. ; il prend des
\item[\vref{Ez 38:10}] et que tu formeras un dessein p.
\item[\vref{1 Ti 6:9}] désirs insensés et p. qui plongent les
\end{listverse}

\ConcordanceEntry{Perpétuité}
\vspace{-2mm}
\begin{listverse}
\item[\vref{Ex 15:18}] Yahweh régnera à jamais et à p.
\item[\vref{No 18:19}] sel et à p. dvt Yahweh, pour
\item[\vref{Jos 14:9}] ton héritage à p., pour toi et
\item[\vref{Ps 119:44}] ta loi, à toujours et à p.
\item[\vref{Ps 132:12}] seront assis à p. sur ton trône.
\item[\vref{Ps 145:21}] sa sainteté à toujours, et à p. !
\item[\vref{Ps 148:6}] a établis à p. et à toujours ;
\item[\vref{Es 26:4}] en Yahweh à p., car le Rocher
\item[\vref{Es 30:8}] à venir, à p., à jamais ;
\item[\vref{Da 12:3}] les étoiles, à toujours et à p.
\item[\vref{Mi 4:5}] notre Dieu, à toujours et à p.
\end{listverse}

\ConcordanceEntry{Perse, Perses}
\vspace{-2mm}
\begin{listverse}
\item[\vref{Da 5:28}] et donné aux Mèdes et aux P.
\item[\vref{Da 6:8}] Mèdes et des P., qui est immuable.
\item[\vref{Da 8:20}] les rois des Mèdes et des P. ;
\item[\vref{Da 10:1}] Cyrus, roi de P., une parole fut
\end{listverse}

\ConcordanceEntry{Persécuter}
\vspace{-2mm}
\begin{listverse}
\item[\vref{Ps 69:27}] Car ils p. celui que tu avais frappé, et
\item[\vref{Ps 119:86}] fidélité ; on me p. sans cause, aide-moi !
\item[\vref{Jé 15:15}] ceux qui me p. ! Ne m'enlève pas,
\item[\vref{Mt 5:10}] ceux qui sont p. pour la justice,
\item[\vref{Mt 5:44}] ceux qui vs. maltraitent et vs. p.,
\item[\vref{Lu 11:49}] les uns, et p. les autres,
\item[\vref{Lu 21:12}] vs., et vs. p., vs. livrant aux
\item[\vref{Jn 15:20}] maître. S'ils m'ont p., ils vs. persécuteront
\item[\vref{Ac 9:4}] lui disait : Saul, Saul, pourquoi me p.-tu ?
\item[\vref{Ac 22:4}] J'ai p. à mort cette doctrine, liant et
\item[\vref{1 Co 4:12}] bénissons ; ns. sommes p., et ns. le
\item[\vref{2 Co 4:9}] ns. sommes p., mais non abandonnés ;
\item[\vref{Ga 1:13}] et comment je p. à outrance l'Eglise
\item[\vref{Ga 4:29}] selon la chair p. celui qui était
\item[\vref{Ga 6:12}] ne pas être p. pour la croix
\item[\vref{2 Ti 3:12}] veulent vivre pieusement en Jésus-Christ seront p.
\item[\vref{Ap 12:13}] la terre, il p. la fem. qui
\end{listverse}

\ConcordanceEntry{Persécuteur}
\vspace{-2mm}
\begin{listverse}
\item[\vref{Ps 7:2}] de ts mes p., et délivre-moi,
\item[\vref{Ps 35:3}] javelot contre mes p. ! Dis à mon
\item[\vref{La 1:3}] repos ; ts ses p. l'ont attrapée ds
\item[\vref{La 4:19}] [Qof.] Nos p. étaient plus légers
\item[\vref{1 Ti 1:13}] un blasphémateur, un p., et un hom.
\end{listverse}

\ConcordanceEntry{Persécution}
\vspace{-2mm}
\begin{listverse}
\item[\vref{Mt 13:21}] tribulation ou une p. à cause de
\item[\vref{Mc 10:30}] terres, avec des p. ; et ds le
\item[\vref{Ac 8:1}] eut une grande p. contre l'Eglise de
\item[\vref{Ac 11:19}] dispersés par la p. survenue à cause
\item[\vref{Ac 13:50}] ils provoquèrent une p. contre Paul et
\item[\vref{Ro 8:35}] l'angoisse, ou la p., ou la famine,
\item[\vref{2 Co 12:10}] nécessités, ds les p., et ds les
\item[\vref{2 Th 1:4}] de ttes vos p., et des afflictions
\item[\vref{2 Ti 3:11}] tu sais les p. et les afflictions
\end{listverse}

\ConcordanceEntry{Persévérance}
\vspace{-2mm}
\begin{listverse}
\item[\vref{Da 6:20}] tu sers avec p., a-t-il pu te
\item[\vref{Lu 8:15}] bon, et portent du fruit avec p.
\item[\vref{Ro 5:3}] afflictions, sachant que l'affliction produit la p.,
\item[\vref{Ro 5:4}] et la p. l'épreuve, et l'épreuve l'espérance.
\item[\vref{Ep 6:18}] avec une entière p., et priez pour
\item[\vref{2 Th 1:4}] cause de votre p. et de votre
\item[\vref{2 Ti 3:10}] foi, ma douceur, ma charité, ma p.
\item[\vref{Ap 3:10}] parole de ma p., je te garderai
\item[\vref{Ap 13:10}] C'est ici la p. et la foi
\item[\vref{Ap 14:12}] Ici est la p. des saints ; ici
\end{listverse}

\ConcordanceEntry{Persévérer}
\vspace{-2mm}
\begin{listverse}
\item[\vref{No 32:12}] car ils ont p. à suivre Yahweh.
\item[\vref{Jos 14:8}] mais moi je p. à suivre Yahweh,
\item[\vref{Job 2:9}] fem. lui dit : P.-tu encore ton
\item[\vref{Mt 10:22}] mais celui qui p. jusqu'à la fin
\item[\vref{Mc 13:13}] mais celui qui p. jusqu'à la fin,
\item[\vref{Lu 22:28}] ceux qui avez p. avec moi ds
\item[\vref{Jn 8:44}] il n'a pas p. ds la vérité,
\item[\vref{Ac 1:14}] d'un commun accord, p. ds la prière
\item[\vref{Ac 2:42}] Et ils p. ts ds la doctrine des apôtres,
\item[\vref{Ac 13:43}] les exhortèrent à p. ds la grâce
\item[\vref{Ac 14:22}] les exhortant à p. ds la foi,
\item[\vref{Ro 2:7}] ceux qui, en p. ds les bonnes
\item[\vref{Col 4:2}] P. ds la prière, veillant ds cet
\item[\vref{Hé 8:9}] ils n'ont pas p. ds mon alliance,
\item[\vref{Ja 1:25}] et qui aura p., n'étant pas un
\end{listverse}

\ConcordanceEntry{Personne}
\vspace{-2mm}
\begin{listverse}
\item[\vref{Ge 14:21}] Abram : Donne-moi les p., et prends pour
\item[\vref{Ps 53:2}] il n'y a p. qui fasse le
\item[\vref{Ec 4:10}] tombé, il n'aura p. pour le relever.
\item[\vref{Es 50:2}] ne s'est-il trouvé p. ? J'ai appelé : Pourquoi
\item[\vref{Es 59:16}] il s'étonne que p. ne se tienne
\item[\vref{Es 63:5}] il n'y avait p. pour m'aider ; et
\item[\vref{Es 64:6}] Il n'y a p. qui invoque ton Nom, qui se
\item[\vref{Es 66:4}] j'ai appelé, et p. n'a répondu, parce
\item[\vref{Mt 24:4}] Prenez garde que p. ne vs. séduise.
\item[\vref{Mt 24:36}] et de l'heure, p. ne le sait,
\item[\vref{Mc 3:27}] P. ne peut entrer ds la maison
\item[\vref{Lu 8:51}] ne permit à p. d'entrer avec lui,
\item[\vref{Lu 10:22}] mon Père, et p. ne connaît qui
\item[\vref{Lu 13:4}] bien, ces dix-huit p. sur qui est
\item[\vref{Jn 1:18}] P. n'a jamais vu Dieu ; le Fils
\item[\vref{Jn 3:2}] de Dieu, car p. ne peut faire
\item[\vref{Jn 3:13}] Car p. n'est monté au ciel, si ce
\item[\vref{Jn 7:27}] le Christ viendra, p. ne saura d'où
\item[\vref{Jn 10:18}] P. ne me l'ôte, mais je la
\item[\vref{Ro 13:1}] Que tte p. soit soumise aux
\item[\vref{Ro 13:8}] devez rien à p., si ce n'est
\item[\vref{1 Co 3:18}] Que p. ne s'abuse lui-mm : Si quelqu'un d'entre
\item[\vref{Col 2:8}] Prenez garde que p. ne fasse de
\item[\vref{Ap 5:3}] il n'y avait p., ni ds le
\item[\vref{Ap 13:17}] et que p. ne puisse acheter ni vendre, sans
\item[\vref{Ap 14:1}] cent quarante-quatre mille p. qui avaient le
\item[\vref{Ap 14:3}] les anciens. Et p. ne pouvait apprendre
\item[\vref{Ap 19:12}] nom écrit que p. ne connaît, si
\end{listverse}

\ConcordanceEntry{Persuader}
\vspace{-2mm}
\begin{listverse}
\item[\vref{Jé 20:7}] Yahweh ! Tu m'as p., et je me
\item[\vref{Lu 16:31}] pas non plus p. qnd quelqu'un des
\item[\vref{Ac 18:4}] sabbat, et il p. des Juifs et
\item[\vref{Ac 21:14}] se laissait pas p., ns. n'insistâmes pas,
\item[\vref{Ac 26:28}] vas bientôt me p. de devenir chrétien !
\item[\vref{Ac 28:24}] les uns furent p. par les choses
\item[\vref{Ro 4:21}] étant pleinement p. que celui qui
\item[\vref{Ro 14:5}] chacun soit pleinement p. en son esprit.
\item[\vref{Ph 1:25}] Et je suis p., je sais que
\item[\vref{Hé 11:11}] parce qu'elle fut p. que celui qui
\end{listverse}

\ConcordanceEntry{Perte}
\vspace{-2mm}
\begin{listverse}
\item[\vref{Mc 14:4}] quoi sert la p. de ce parfum ?
\item[\vref{Lu 8:44}] mm instant la p. de sang s'arrêta.
\item[\vref{Ac 27:22}] n'y aura de p. que celle du
\item[\vref{1 Co 3:15}] en fera la p. ; mais pour lui,
\item[\vref{Ph 3:7}] regardées com. une p. à cause de
\end{listverse}

\ConcordanceEntry{Pervers}
\vspace{-2mm}
\begin{listverse}
\item[\vref{De 32:20}] sont une génération p., des fils infidèles.
\item[\vref{Jg 19:22}] ville, fils d'hommes p., environnèrent la maison,
\item[\vref{2 S 22:27}] mais avec le p. tu agis selon
\item[\vref{Pr 12:8}] a le cœur p. est l'objet du
\item[\vref{Pr 16:30}] méditer des choses p., et remuant ses
\item[\vref{Mt 17:17}] race incrédule et p., jusqu'à qnd serai-je
\item[\vref{Ac 2:40}] en disant : Sauvez-vs. de cette génération p.
\item[\vref{Ph 2:15}] génération corrompue et p., parmi laquelle vs.
\item[\vref{2 Th 3:2}] hommes méchants et p., car ts n'ont
\end{listverse}

\ConcordanceEntry{Pervertir}
\vspace{-2mm}
\begin{listverse}
\item[\vref{Ex 23:2}] grand nombre pour p. le droit.
\item[\vref{Pr 10:9}] mais celui qui p. ses voies sera
\item[\vref{Pr 17:23}] en secret, pour p. les voies du
\item[\vref{Pr 19:1}] que celui qui p. ses lèvres et
\item[\vref{Es 47:10}] ta science t'ont p., et tu disais
\item[\vref{Es 59:8}] ils se sont p. ds leurs sentiers,
\item[\vref{Jé 3:21}] car ils ont p. lr. voie, ils
\item[\vref{La 3:35}] lorsqu'on p. la justice humaine
\item[\vref{Am 2:7}] faibles ; et ils p. la voie des
\item[\vref{Am 5:12}] rançon, et vs. p. à la porte
\item[\vref{Mi 3:9}] abomination, et qui p. tt ce qui
\item[\vref{Tit 3:11}] tel hom. est p., et qu'il pèche
\end{listverse}

\ConcordanceEntry{Pesant}
\vspace{-2mm}
\begin{listverse}
\item[\vref{Ex 17:12}] Moïse étant devenues p., ils prirent une
\item[\vref{Ex 18:18}] cela est trop p. pour toi, tu
\item[\vref{No 11:14}] il est trop p. pour moi.
\item[\vref{Job 6:3}] elle serait plus p. que le sable
\item[\vref{Ps 38:5}] appesanties com. un p. fardeau, au-delà de
\item[\vref{Ps 66:11}] as mis sur nos reins un p. fardeau.
\item[\vref{Pr 27:3}] La pierre est p., et le sable
\item[\vref{Ec 12:7}] les cigales deviennent p., et que l'appétit
\item[\vref{Es 59:1}] son oreille trop p. pour pouvoir entendre.
\item[\vref{Za 12:3}] Jérus. une pierre p. pour ts les
\item[\vref{Mt 23:4}] ensemble des fardeaux p. et insupportables et
\end{listverse}

\ConcordanceEntry{Peser}
\vspace{-2mm}
\begin{listverse}
\item[\vref{No 11:14}] il est trop p. pour moi.
\item[\vref{1 S 2:3}] c'est lui qui p. ttes les actions.
\item[\vref{1 S 17:5}] cuirasse à écailles p. cinq mille sicles
\item[\vref{Esd 8:33}] quatrième jour, ns. p. l'argent, l'or, et
\item[\vref{Job 31:6}] qu'on me p. ds des balances justes, et Dieu
\item[\vref{Pr 16:2}] yeux ; mais Yahweh p. les esprits.
\item[\vref{Es 40:12}] et qui a p. au crochet les
\item[\vref{Jé 32:9}] et je lui p. l'argent, qui fut
\item[\vref{Ez 5:1}] une balance à p., et tu partageras
\item[\vref{Da 5:27}] P. : Tu as été pesé ds la
\item[\vref{Za 11:12}] pas. Alors ils p. pour mon salaire
\item[\vref{Ap 16:21}] dont les grêlons p. un talent, tomba
\end{listverse}

\ConcordanceEntry{Peste}
\vspace{-2mm}
\begin{listverse}
\item[\vref{Ex 5:3}] frappe par la p. ou par l'épée.
\item[\vref{No 14:12}] frapperai par la p. et je le
\item[\vref{De 28:21}] fera que la p. s'attachera à toi,
\item[\vref{2 S 24:13}] trois jours la p. soit ds ton
\item[\vref{Ps 78:50}] il livra lr. vie à la p.
\item[\vref{Ps 91:3}] l'oiseleur, de la p. et de ses
\item[\vref{Ps 91:6}] ni la p. qui marche ds les ténèbres, ni
\item[\vref{Jé 29:17}] famine, et la p., et je les
\item[\vref{Ez 5:12}] mourra de la p., et sera consumé
\item[\vref{Os 13:14}] où est ta p. ? Scheol, où est
\item[\vref{Am 4:10}] parmi vs. la p. de la mm
\item[\vref{Ha 3:5}] La p. marche dvt lui, et une flamme
\item[\vref{Lu 21:11}] famines et des p. ; il y aura
\item[\vref{Ac 24:5}] qui est une p., qui sème des
\end{listverse}

\ConcordanceEntry{Petit}
\vspace{-2mm}
\begin{listverse}
\item[\vref{Ge 1:16}] et le plus p. luminaire pour présider
\item[\vref{Ge 32:10}] Je suis trop p. pour ttes les
\item[\vref{Lé 22:28}] brebis, ou la chèvre avec son p.
\item[\vref{De 22:6}] d'oiseaux, ayant des p. ou des œufs,
\item[\vref{De 32:11}] nichée, couve ses p., étend ses ailes,
\item[\vref{Jg 6:15}] suis le plus p. de la maison
\item[\vref{Job 36:27}] les eaux en p. gouttes, elles répandent
\item[\vref{Job 38:32}] conduis-tu la Grande Ourse avec ses p. ?
\item[\vref{Job 39:3}] corbeau, qnd ses p. crient à Dieu,
\item[\vref{Ps 8:3}] la bouche des p. enfants et de
\item[\vref{Ps 39:5}] sache de combien p. durée je suis !
\item[\vref{Ps 119:141}] Je suis p. et méprisé, toutefois je n'oublie point
\item[\vref{Es 54:7}] délaissée pour un p. moment, mais je
\item[\vref{Es 60:22}] La p. famille deviendra un millier de personnes,
\item[\vref{Jé 49:15}] je te rendrai p. entre les nations,
\item[\vref{Mt 2:20}] Lève-toi, prends le p. enfant et sa
\item[\vref{Mt 5:19}] l'un de ces p. commandements, et qui
\item[\vref{Mt 18:14}] aux cieux qu'un seul de ces p. périsse.
\item[\vref{Mt 25:40}] de ces plus p. de mes frères,
\item[\vref{Mc 9:42}] un de ces p. qui croient en
\item[\vref{Lu 9:48}] Quiconque reçoit ce p. enfant en mon
\item[\vref{Lu 10:21}] as révélées aux p. enfants. Oui, Père,
\item[\vref{Ac 8:10}] depuis le plus p. jusqu'au plus grand
\item[\vref{Ja 3:5}] langue, c'est un p. membre, et cependant
\item[\vref{1 Pi 3:20}] ds laquelle un p. nombre, à savoir
\item[\vref{1 Jn 2:1}] Mes p.-enfants, je vs. écris ces choses
\item[\vref{Ap 20:12}] grands et les p., qui se tenaient
\end{listverse}

\ConcordanceEntry{Peu}
\vspace{-2mm}
\begin{listverse}
\item[\vref{No 16:13}] Est-ce p. de chose que tu ns. aies
\item[\vref{De 7:22}] ton Dieu, chassera p. à peu ces
\item[\vref{Ps 37:16}] au juste le p. qu'il a, que
\item[\vref{Pr 15:16}] Vaut mieux un p. de bien avec
\item[\vref{Mt 7:14}] y en a p. qui les trouvent.
\item[\vref{Lu 7:47}] qui on pardonne p. aime peu.
\item[\vref{Lu 10:2}] il y a p. d'ouvriers. Priez dc
\item[\vref{Lu 13:23}] n'y a-t-il que p. de gens qui
\item[\vref{Ac 1:5}] du Saint-Esprit ds p. de jours.
\item[\vref{2 Co 8:15}] celui qui avait p. n'en a pas
\item[\vref{Ga 5:9}] Un p. de levain fait lever tte la
\item[\vref{Hé 11:25}] jouir pour un p. de temps des
\item[\vref{1 Pi 5:10}] aurez souffert un p. de temps, vs.
\end{listverse}

\ConcordanceEntry{Peuple}
\vspace{-2mm}
\begin{listverse}
\item[\vref{Ge 11:6}] seul et mm p., ils ont un
\item[\vref{Ex 1:9}] dit à son p. : Voici, le peuple
\item[\vref{Ex 5:1}] Laisse aller mon p., afin qu'il me
\item[\vref{Ex 17:3}] Le p. dc eut soif en ce lieu-là,
\item[\vref{Ex 32:1}] Mais le p., voyant que Moïse tardait tant à
\item[\vref{Lé 26:12}] votre Dieu, et vs. serez mon p.
\item[\vref{De 1:28}] en disant : Le p. est plus grand
\item[\vref{De 7:6}] tu es un p. saint pour Yahweh,
\item[\vref{Jg 7:4}] à Gédéon : Le p. est encore trop
\item[\vref{Ru 1:16}] je demeurerai ; ton p. sera mon peuple
\item[\vref{2 Ch 7:14}] si mon p., sur lequel mon Nom est invoqué,
\item[\vref{Est 3:8}] a un certain p. dispersé ds ttes
\item[\vref{Ps 33:12}] Dieu et le p. qu'il s'est choisi
\item[\vref{Ps 50:7}] Ecoute, ô mon p. ! et je parlerai.
\item[\vref{Ps 67:4}] Les p. te célébreront, ô Dieu ! Tous les
\item[\vref{Ps 81:14}] Ô si mon p. m'écoutait ! Si Israël marchait ds mes
\item[\vref{Ps 95:7}] ns. sommes le p. de son pâturage,
\item[\vref{Ps 102:19}] à venir, le p. qui sera créé
\item[\vref{Ps 110:3}] Ton p. est plein d'ardeur qnd tu rassembles
\item[\vref{Ps 116:14}] vœux à Yahweh, dvt tt son p.
\item[\vref{Pr 14:28}] la multitude du p., mais qnd le
\item[\vref{Es 53:8}] faite pour les péchés de mon p.
\item[\vref{Es 63:8}] ils sont mon p., des enfants qui
\item[\vref{Es 65:2}] jours vers un p. rebelle, à celui
\item[\vref{Os 1:9}] n'êtes pas mon p., et je ne
\item[\vref{Os 4:6}] Mon p. est détruit, parce qu'il est sans
\item[\vref{So 3:12}] de toi un p. humble et faible,
\item[\vref{Mt 1:21}] qui sauvera son p. de ses péchés.
\item[\vref{Mt 26:5}] se fasse qq tumulte parmi le p.
\item[\vref{Mc 7:6}] est écrit : Ce p. m'honore des lèvres,
\item[\vref{Lu 1:17}] au Seign. un p. bien disposé.
\item[\vref{Jn 18:14}] qu'un seul hom. meure pour le p.
\item[\vref{Ac 2:47}] à tt le p.. Et le Seign.
\item[\vref{Ac 6:8}] et de grands prodiges parmi le p.
\item[\vref{Ac 18:10}] car j'ai un p. nombreux ds cette
\item[\vref{Ac 28:27}] cœur de ce p. est devenu insensible ;
\item[\vref{Ro 9:25}] Osée : J'appellerai mon p. celui qui n'était
\item[\vref{Tit 2:14}] lui être un p. qui lui appartienne
\item[\vref{Hé 4:9}] Il reste dc un repos au p. de Dieu.
\item[\vref{1 Pi 2:9}] nation sainte, le p. acquis, afin que
\item[\vref{Ap 5:9}] langue, de tt p., et de tte
\item[\vref{Ap 18:4}] de Babylone, mon p., afin que vs.
\item[\vref{Ap 21:3}] ils seront son p., et Dieu lui-mm
\end{listverse}

\ConcordanceEntry{Peur}
\vspace{-2mm}
\begin{listverse}
\item[\vref{Ge 3:3}] toucherez point, de p. que vs. ne
\item[\vref{Ge 3:10}] et j'ai eu p. parce que je
\item[\vref{Jg 7:3}] et qui a p. s'en retourne et
\item[\vref{Ps 27:1}] De qui aurai-je p. ? Yahweh est le
\item[\vref{Es 41:14}] N’aie pas p., toi vermisseau de
\item[\vref{Es 51:12}] es-tu pour avoir p. de l'hom. mortel
\item[\vref{Mt 8:26}] dit : Pourquoi avez-vs. p., gens de peu
\item[\vref{Mt 14:27}] Rassurez-vs., c'est moi, n'ayez pas de p. !
\item[\vref{Jn 3:20}] la lumière, de p. que ses œuvres
\item[\vref{1 Co 9:27}] tiens assujetti, de p. d'être moi-mm désapprouvé
\item[\vref{1 Th 3:5}] votre foi, de p. que le tentateur
\item[\vref{Hé 4:11}] ce repos-là, de p. que quelqu'un ne
\item[\vref{1 Pi 3:14}] veulent vs. faire p., et n'en soyez
\item[\vref{1 Jn 4:18}] a point de p. ds la charité,
\end{listverse}

\ConcordanceEntry{Pharaon}
\vspace{-2mm}
\begin{listverse}
\item[\vref{Ge 12:15}] la cour de P. la virent aussi
\end{listverse}
\begin{legend}
\NoAutoSpaceBeforeFDP{
\item Appellation du roi en Egypte.
\item Pharaon et Saraï : Ge 12:14-20
\item Le Pharaon du temps de Joseph
\item Le songe de Pharaon : Ge 41:1-32
\item Pharaon élève Joseph : Ge 41:39-41; 45:8
\item Pharaon honore la famille de Joseph : Ge 47:5-6
\item Un nouveau pharaon : Ex 1:8-11; 5:5-9
\item Face à Moïse : Ex 7:22; 8:11
\item Israël sort d'Egypte : Ex 12:31
\item P. et son armée à la poursuite d'Israël : Ex 14: 1-28
\item Salomon s'allie à Pharaon qui devient son beau-père : 1 R 3:1
\item Pharaon Schischak : 1 R 11:40
\item Schischak emporte les trésors de Juda : 1 R 14:25-26
\item Pharaon Neco : Jé 46:2
\item Blessure et mort de Josias : 2 Ch 35:20-24
\item Joachaz mis en prison par Neco : 2 R 23: 31-35
\item Pharaon Hophra : Jé 44:30
}
\end{legend}

\ConcordanceEntry{Pharisien}
\vspace{-2mm}
\begin{listverse}
\item[\vref{Mt 3:7}] baptême beaucoup de p. et de sadducéens,
\item[\vref{Mt 16:6}] du levain des p. et des sadducéens.
\item[\vref{Mt 19:3}] Alors les p. vinrent à lui
\item[\vref{Mt 23:26}] P. aveugle ! nettoie premièrement l'intérieur de la
\item[\vref{Mc 3:6}] Alors les p. sortirent, et aussitôt,
\item[\vref{Mc 7:3}] Car les p. et ts les Juifs ne mangent
\item[\vref{Mc 7:5}] Sur cela, les p. et les scribes
\item[\vref{Lu 7:30}] mais les p. et les docteurs de la loi,
\item[\vref{Lu 7:36}] Un des p. pria Jésus de manger chez lui.
\item[\vref{Lu 11:42}] malheur à vs., p. ! Car vs. payez
\item[\vref{Lu 16:14}] Or les p. aussi, qui étaient avares, entendaient ttes
\item[\vref{Lu 18:10}] prier, l'un était p., et l'autre publicain.
\item[\vref{Jn 7:48}] chefs ou des p. qui ait cru
\item[\vref{Ac 5:34}] Mais un p. nommé Gamaliel, docteur de la loi,
\item[\vref{Ac 15:5}] la secte des p. qui avaient cru,
\item[\vref{Ac 23:6}] et l'autre de p., s'écria ds le
\item[\vref{Ac 23:7}] débat entre les p. et les sadducéens ;
\item[\vref{Ph 3:5}] Hébreu né d'Hébreux, p. en ce qui
\end{listverse}

\ConcordanceEntry{Phénicie}
\vspace{-2mm}
\begin{listverse}
\item[\vref{Ac 11:19}] d'Etienne allèrent jusqu'en P., ds l'île de
\item[\vref{Ac 15:3}] ils traversèrent la P. et la Samarie,
\item[\vref{Ac 21:2}] traversée vers la P., ns. montâmes et
\end{listverse}

\ConcordanceEntry{Phéréziens}
\vspace{-2mm}
\begin{listverse}
\item[\vref{Ge 13:7}] Cananéens et les P. habitaient ds le
\item[\vref{Ex 33:2}] les Héthiens, les P., les Héviens et
\item[\vref{Jos 9:1}] les Cananéens, les P., les Héviens et
\end{listverse}

\ConcordanceEntry{Philadelphie}
\vspace{-2mm}
\begin{listverse}
\item[\vref{Ap 1:11}] à Sardes, à P., et à Laodicée.
\item[\vref{Ap 3:7}] de l'église de P. : Voici ce que
\end{listverse}

\ConcordanceEntry{Philémon}
\vspace{-2mm}
\begin{listverse}
\item[\vref{Phm 1:1}] frère Timothée, à P. notre bien-aimé et
\end{listverse}

\ConcordanceEntry{Philippe}
\vspace{-2mm}
\begin{listverse}
\item[\vref{Mt 10:3}] P., et Barthélemy ; Thomas, et Matthieu, le
\end{listverse}
\begin{legend}
\NoAutoSpaceBeforeFDP{
\item Apôtre : Mt 10:3; Jn 1:43-46; Jn 6:5-7; 12:20-22; 14:8-9
\item Diacre et évangeliste : Ac 6:5; 21:8-9
\item En Samarie : Ac 8:5-13
\item Conversion et baptême de l'eunuque Ethiopien : Ac 8:26-38
}
\end{legend}

\ConcordanceEntry{Philippes}
\vspace{-2mm}
\begin{listverse}
\item[\vref{Ac 16:12}] ns. allâmes à P., qui est la
\item[\vref{1 Th 2:2}] des outrages à P., com. vs. le
\end{listverse}

\ConcordanceEntry{Philistins}
\vspace{-2mm}
\begin{listverse}
\item[\vref{Ge 10:14}] sont sortis les P., et les Caphtorim.
\end{listverse}
\begin{legend}
\NoAutoSpaceBeforeFDP{
\item Les ascendants des Philistins : Ge 10:14
\item Les cinq princes : Jos 13:3
\item Les Philistins et les puits d'Isaac : Ge 26:15-17
\item Les guerres entre les philistins et Israël : 1 S 7:13; 31:1; 1 Ch 18:1
\item David dans le pays des philistins : 1 S 27:1-7
\item Goliath, un Philistin : Jg 13-16; 1 S 17
\item Prophéties et jugements : Es 14:31; Jé 47; Ez 25:15-17; Am 1:8; Za 9:6
\item Autres : Ps 60:10; 87:4-5
}
\end{legend}

\ConcordanceEntry{Philosophie}
\vspace{-2mm}
\begin{listverse}
\item[\vref{Col 2:8}] proie par la p., et par de
\end{listverse}

\ConcordanceEntry{Phinées}
\vspace{-2mm}
\begin{listverse}
\item[\vref{Ex 6:25}] qui lui enfanta P.. Ce sont là
\end{listverse}
\begin{legend}
\NoAutoSpaceBeforeFDP{
\item Fils d'Eléazar, petit-fils d'Aaron et prêtre : Ex 6:25; No 31:6;
\item Phinées détourne la colère de Yahweh : No 25:1-14; Ps 106:30; Jos 22:13; Jg 20:27-28
\item Fils du prêtre Eli et prêtre : 1 S:1-3; 2:12-17; 4:11
}
\end{legend}

\ConcordanceEntry{Phrygie}
\vspace{-2mm}
\begin{listverse}
\item[\vref{Ac 16:6}] Ayant traversé la P. et le pays
\item[\vref{Ac 18:23}] Galatie et de P., fortifiant ts les
\end{listverse}

\ConcordanceEntry{Pièce (une)}
\vspace{-2mm}
\begin{listverse}
\item[\vref{Za 11:12}] mon salaire trente p. d'argent.
\item[\vref{Za 11:13}] pris les trente p. d'argent, et les
\item[\vref{Mt 26:15}] lui comptèrent trente p. d'argent.
\item[\vref{Lu 21:2}] mettait deux petites p. de monnaie.
\item[\vref{Ac 19:19}] à cinquante mille p. d'argent.
\end{listverse}

\ConcordanceEntry{Pied}
\vspace{-2mm}
\begin{listverse}
\item[\vref{Ge 8:9}] plante de son p., retourna à lui
\item[\vref{Ge 18:4}] et lavez vos p., et reposez-vs. sous
\item[\vref{No 22:25}] elle serra le p. de Balaam contre
\item[\vref{De 8:4}] toi, et ton p. ne s'est point
\item[\vref{De 11:24}] plante de votre p. sera à vs. :
\item[\vref{De 33:24}] il trempera son p. ds l'huile.
\item[\vref{Jos 1:3}] plante de votre p., je vs. l'ai
\item[\vref{Jos 3:17}] Yahweh, s'arrêtèrent de p. ferme sur le
\item[\vref{2 S 22:34}] a rendu mes p. semblables à ceux
\item[\vref{1 R 15:23}] malade de ses p. au temps de
\item[\vref{Job 39:18}] oublie que le p. peut les écraser,
\item[\vref{Ps 8:7}] tu as tt mis sous ses p.,
\item[\vref{Ps 22:17}] ont percé mes mains et mes p.
\item[\vref{Ps 26:12}] Mon p. se tient ds la droiture ; je
\item[\vref{Ps 40:3}] a mis mes p. sur un roc
\item[\vref{Ps 91:12}] peur que ton p. ne heurte contre
\item[\vref{Pr 1:16}] parce que leurs p. courent au mal,
\item[\vref{Pr 3:26}] il gardera ton p. de tte embûche.
\item[\vref{Ec 4:17}] garde à ton p., et approche-toi pour
\item[\vref{Es 1:6}] la plante du p. jusqu'à la tête,
\item[\vref{Es 58:13}] tu détournes ton p. pendant le sabbat
\item[\vref{Ez 34:18}] fouliez de vos p. le reste de
\item[\vref{Da 2:33}] fer et ses p. étaient en partie
\item[\vref{Za 14:4}] Ses p. se poseront en ce jour sur
\item[\vref{Mt 4:6}] peur que ton p. ne heurte contre
\item[\vref{Mt 18:8}] main ou ton p. est pour toi
\item[\vref{Mt 28:9}] s'approchèrent, embrassèrent ses p. et l'adorèrent.
\item[\vref{Mc 6:33}] y accoururent à p. de ttes les
\item[\vref{Lu 7:44}] pour laver mes p. ; mais elle, elle
\item[\vref{Jn 11:2}] qui essuya ses p. avec ses cheveux ;
\item[\vref{Jn 12:3}] en oignit les p. de Jésus, et
\item[\vref{Jn 13:5}] à laver les p. de ses disciples,
\item[\vref{Ac 3:7}] chevilles de ses p. devinrent fermes.
\item[\vref{Ac 5:2}] le déposa aux p. des apôtres.
\item[\vref{Ro 3:15}] leurs p. sont légers pour répandre le sang ;
\item[\vref{Ro 10:15}] sont beaux les p. de ceux qui
\item[\vref{1 Co 12:15}] Si le p. dit : Parce que je ne suis
\item[\vref{1 Co 15:27}] choses sous ses p.. Or qnd il
\item[\vref{Ep 6:15}] et ayant vos p. chaussés, prêts pour
\item[\vref{1 Ti 5:10}] d'avoir lavé les p. des saints, d'avoir
\item[\vref{Ap 1:15}] Ses p. étaient semblables à de l'airain ardent,
\item[\vref{Ap 10:2}] il posa son p. droit sur la
\item[\vref{Ap 11:2}] ils fouleront aux p. la ville sainte
\end{listverse}

\ConcordanceEntry{Piège}
\vspace{-2mm}
\begin{listverse}
\item[\vref{Ex 10:7}] hom. sera-t-il un p. pour ns. ? Laisse
\item[\vref{Ex 23:33}] ce serait un p. pour toi.
\item[\vref{Jos 23:13}] pour vs. un p. et un filet,
\item[\vref{1 S 18:21}] pour lui un p., et que par
\item[\vref{Ps 69:23}] pour eux un p. et un appât
\item[\vref{Ps 141:9}] Garde-moi du p. qu'ils m'ont tendu
\item[\vref{Pr 12:13}] des lèvres un p. pernicieux, mais le
\item[\vref{Pr 18:7}] lèvres sont un p. pour son âme.
\item[\vref{Pr 20:25}] C'est un p. à l'hom. que de dévorer les
\item[\vref{Pr 29:25}] hommes tend un p., mais celui qui
\item[\vref{Ec 7:26}] cœur est un p. et un filet,
\item[\vref{Ec 9:12}] sont pris au p. ; com. eux, les
\item[\vref{Os 5:1}] avez été un p. à Mitspa, et
\item[\vref{Lu 20:20}] lui tendre des p. et saisir de
\item[\vref{1 Co 7:35}] vs. tendre un p., mais pour vs.
\item[\vref{1 Ti 3:7}] et ds les p. du diable.
\item[\vref{1 Ti 6:9}] tentation, ds le p., et ds beaucoup
\item[\vref{2 Ti 2:26}] pour sortir des p. du diable par
\end{listverse}

\ConcordanceEntry{Pierre}
\vspace{-2mm}
\begin{listverse}
\item[\vref{Mt 4:18}] frères, Simon, appelé P., et André, son
\end{listverse}
\begin{legend}
\NoAutoSpaceBeforeFDP{
\item Apôtre, l'un des douze : Mt 10:2; Ac 1:13; 1 Pi 1:1
\item L'appel de Jésus : Mt 4:18-19; Jn 1:40-43
\item P. marche sur les eaux : Mt 14:29-30
\item Reconnaît  Jésus comme le messie : Mt 16:16
\item Jésus bâtit son Eglise : Mt 16:18
\item P. face à l'annonce de la mort et de la résurrection de Jésus : Mt 16:22-23
\item Témoin de la transfiguration : Mt 17:1-7; 2 Pi 1:16
\item Préparation de la dernière Pâque : Lu 22:8
\item Jésus lave les pieds des disciples : Jn 13:6-8
\item P. face à la crucifixion : Mt 26:31-35,58; 69-75
\item Reniement et repentance : Luc 22:61-62
\item P. au sépulcre : Jn 20: 3-7
\item Prédication de P. durant la pentecôte : Ac 2:14-40
\item Prédication de l'Evangile aux circoncis confiée à P : Jn 21:16; Ga 2:7-8
\item P. une colonne : Ga 1:18; 2:6-9
\item Prodiges et miracles : Ac 3:6-8; 5:12-15; 9:32-43
\item Persécutions de l'Eglise et P emprisonné : Ac 4:1-4,21; 12:1-17
\item P. et les Gentils : Ac 10:1,44-46
\item P. en Samarie : Ac 8:14-17
\item P. est la circoncision : Ac 15:1-11
\item Paul reprend Pierre à Antioche : Ga 2:11-14
\item P. témoigne de Paul : 2 Pi 3:15-16
\item Le Maître révèle à Pierre sa mort : Jn 21:18-19; 2 Pi 1:13-14
}
\end{legend}

\ConcordanceEntry{Pierre, Cailloux}
\vspace{-2mm}
\begin{listverse}
\item[\vref{Ge 11:3}] lr. servit de p., et le bitume
\item[\vref{Ge 28:11}] prit dc une p. et en fit
\item[\vref{Ex 20:25}] un autel de p., ne les taille
\item[\vref{De 8:9}] pays dont les p. sont du fer,
\item[\vref{Jos 4:9}] dressa aussi douze p. au milieu du
\item[\vref{Jos 8:32}] là, sur les p. une copie de
\item[\vref{1 S 17:40}] le torrent cinq p. bien polies, et
\item[\vref{1 Ch 29:2}] de bois, des p. d'onyx, et des
\item[\vref{Esd 5:8}] bâtit avec des p. de taille, et
\item[\vref{Né 4:2}] la vie les p. des monceaux de
\item[\vref{Ps 102:15}] serviteurs aiment ses p. et chérissent sa
\item[\vref{Ps 118:22}] La P. que les architectes avaient rejetée, est
\item[\vref{Es 8:14}] mais aussi une p. d'achoppement, un rocher
\item[\vref{Es 28:16}] en Sion une p., une pierre éprouvée,
\item[\vref{Jé 2:27}] Et à la p. : Tu m'as engendré !
\item[\vref{Jé 3:9}] adultère avec la p. et le bois.
\item[\vref{Jé 6:21}] ce peuple des p. d'achoppement, auxquelles les
\item[\vref{Ez 11:19}] le cœur de p., et je lr.
\item[\vref{Da 2:34}] jusqu'à ce qu'une p. se détacha sans
\item[\vref{Da 5:4}] de fer, de bois et de p.
\item[\vref{Za 3:9}] quant à la p. que j'ai mise
\item[\vref{Za 4:7}] fera sortir la p. principale ; il y
\item[\vref{Mt 3:9}] naître de ces p. mêmes des enfants
\item[\vref{Mt 4:3}] ordonne que ces p. deviennent des pains.
\item[\vref{Mt 7:9}] vs. donnera une p. à son fils,
\item[\vref{Mt 21:44}] tombera sur cette p. s'y brisera, et
\item[\vref{Mt 24:2}] restera pas ici p. sur pierre qui
\item[\vref{Mt 27:66}] d'une garde, après avoir scellé la p.
\item[\vref{Mt 28:2}] vint rouler la p. à côté de
\item[\vref{Lu 19:40}] se taisent, les p. crieront.
\item[\vref{Jn 8:7}] le premier la p. contre elle.
\item[\vref{Jn 8:59}] ils prirent des p. pour les jeter
\item[\vref{Ac 4:11}] C'est cette p. rejetée, par vs.
\item[\vref{Ro 9:32}] heurtés contre la p. d'achoppement,
\item[\vref{Ro 9:33}] en Sion la p. d'achoppement, et un
\item[\vref{Ro 14:13}] pas mettre une p. d'achoppement ou de
\item[\vref{1 Co 3:12}] de l'argent, des p. précieuses, du bois,
\item[\vref{Ep 2:20}] lui-mm étant la p. angulaire ;
\item[\vref{1 Pi 2:4}] approchant de lui, p. vivante, rejetée par
\item[\vref{1 Pi 2:5}] aussi, com. des p. vivantes, vs. êtes
\item[\vref{1 Pi 2:7}] est dit : La p. que ceux qui
\end{listverse}

\ConcordanceEntry{Piété}
\vspace{-2mm}
\begin{listverse}
\item[\vref{Os 6:4}] ferai-je, Juda ? Votre p. est com. la
\item[\vref{Ac 3:12}] ou par notre p., ns. avions fait
\item[\vref{1 Ti 2:2}] tranquille, en tte p. et honnêteté.
\item[\vref{1 Ti 3:16}] mystère de la p. est grand : Dieu
\item[\vref{1 Ti 4:8}] Exerce-toi à la p. ; car l'exercice corporel
\item[\vref{1 Ti 5:4}] à exercer la p. envers lr. propre
\item[\vref{1 Ti 6:3}] la doctrine qui est selon la p.,
\item[\vref{2 Ti 3:5}] l'apparence de la p., mais en ayant
\item[\vref{Tit 1:1}] la vérité qui est selon la p.,
\item[\vref{Tit 2:12}] la sagesse, la justice et la p.,
\item[\vref{Hé 5:7}] été exaucé à cause de sa p.
\item[\vref{2 Pi 1:3}] et à la p., par la connaissance
\item[\vref{2 Pi 1:7}] à la p. l'amour fraternel, et à l'amour fraternel
\end{listverse}

\ConcordanceEntry{Pieux}
\vspace{-2mm}
\begin{listverse}
\item[\vref{Ps 12:2}] car les hommes p. n'existent plus, les
\item[\vref{Ps 16:3}] pays, les hommes p. sont l'objet de
\item[\vref{Lu 2:25}] était juste et p., il attendait la
\item[\vref{Jn 9:31}] si quelqu'un est p. envers Dieu, et
\item[\vref{Ac 2:5}] y séjournaient, hommes p. de ttes les
\item[\vref{Ac 8:2}] Et quelques hommes p. emportèrent Etienne pour
\item[\vref{Ac 10:2}] Cet hom. était p. et craignait Dieu,
\item[\vref{Ac 22:12}] nommé Ananias, hom. p. selon la loi,
\end{listverse}

\ConcordanceEntry{Pilate}
\vspace{-2mm}
\begin{listverse}
\item[\vref{Mt 27:2}] livrèrent à Ponce P., qui était le
\item[\vref{Mt 27:24}] Alors P. voyant qu'il ne gagnait rien, mais
\item[\vref{Mc 15:15}] P., voulant satisfaire la foule, lr. relâcha
\item[\vref{Lu 3:1}] César, lorsque Ponce P. était gouverneur de
\item[\vref{Lu 13:1}] des Galiléens, dont P. avait mêlé le
\item[\vref{Lu 23:12}] ce mm jour, P. et Hérode devinrent
\item[\vref{Ac 3:13}] et renié dvt P., quoiqu'il jugeât qu'il
\item[\vref{Ac 4:27}] Hérode et Ponce P., avec les Gentils,
\item[\vref{1 Ti 6:13}] fait cette belle confession dvt Ponce P.,
\end{listverse}

\ConcordanceEntry{Pillage}
\vspace{-2mm}
\begin{listverse}
\item[\vref{No 31:11}] et tt le p., tant des hommes
\item[\vref{Esd 9:7}] la captivité, au p., et à la
\item[\vref{Est 9:10}] ne mirent pas leurs mains au p.
\item[\vref{Es 59:15}] est exposé au p. ; Yahweh voit, et
\item[\vref{Jé 15:13}] Je livre au p., sans en faire
\item[\vref{Jé 30:16}] je livrerai au p. ts ceux qui
\item[\vref{Da 11:33}] flamme, à la captivité et au p.
\item[\vref{So 1:13}] biens seront au p. et leurs maisons
\end{listverse}

\ConcordanceEntry{Piller}
\vspace{-2mm}
\begin{listverse}
\item[\vref{Lé 19:13}] tu ne le p. point. Le salaire
\item[\vref{De 3:7}] Mais ns. p. pour ns. ttes
\item[\vref{Jos 8:2}] roi : Seulement vs. p. pour vs. le
\item[\vref{1 S 17:53}] des Philistins, et p. leurs camps.
\item[\vref{Pr 28:24}] Celui qui p. son père ou
\item[\vref{Es 8:1}] de butiner, qu'on se hâte de p.
\item[\vref{Es 42:22}] ici un peuple p. et dépouillé ! Ils
\item[\vref{Jé 20:5}] ennemis, qui les p., les enlèveront et
\item[\vref{Os 13:15}] ses fontaines. On p. le trésor de
\item[\vref{Na 2:10}] P. l'argent ! Pillez l'or ! Il y a
\item[\vref{Ha 2:8}] que tu as p. beaucoup de nations,
\item[\vref{So 2:9}] mon peuple les p., et les restes
\item[\vref{Za 2:8}] qui vs. ont p. ; car celui qui
\item[\vref{Za 14:2}] prise, les maisons p., et les femmes
\item[\vref{Mt 12:29}] hom. fort et p. ses biens, sans
\item[\vref{2 Co 7:2}] personne, ns. n'avons p. personne.
\end{listverse}

\ConcordanceEntry{Pire}
\vspace{-2mm}
\begin{listverse}
\item[\vref{2 S 19:7}] ce mal sera p. que ts ceux
\item[\vref{1 R 14:9}] Tu as fait p. que ts ceux
\item[\vref{2 Ch 33:9}] Jérus., jusqu'à faire p. que les nations
\item[\vref{Mi 7:4}] plus juste est p. qu'une haie d'épines.
\item[\vref{Mt 9:16}] de l'habit, et la déchirure serait p.
\item[\vref{Mt 12:45}] cet hom. est p. que la première.
\item[\vref{Mt 27:64}] dernière imposture serait p. que la première.
\item[\vref{Jn 5:14}] qu'il ne t'arrive qq chose de p.
\item[\vref{1 Ti 5:8}] et il est p. qu'un infidèle.
\item[\vref{2 Pi 2:20}] dernière condition est p. que la première.
\end{listverse}

\ConcordanceEntry{Pischon}
\vspace{-2mm}
\begin{listverse}
\item[\vref{Ge 2:11}] du premier est P. ; c'est le fleuve
\end{listverse}

\ConcordanceEntry{Pisga}
\vspace{-2mm}
\begin{listverse}
\item[\vref{No 23:14}] le sommet de P. ; il bâtit sept
\item[\vref{De 3:27}] au sommet du P., et lève tes
\item[\vref{De 34:1}] au sommet du P., vis-à-vis de Jéricho.
\end{listverse}

\ConcordanceEntry{Pisidie}
\vspace{-2mm}
\begin{listverse}
\item[\vref{Ac 13:14}] Antioche, ville de P., et étant entrés
\item[\vref{Ac 14:24}] Traversant ensuite la P., ils allèrent en
\end{listverse}

\ConcordanceEntry{Pithom}
\vspace{-2mm}
\begin{listverse}
\item[\vref{Ex 1:11}] Pharaon ; à savoir P. et Ramsès.
\end{listverse}

\ConcordanceEntry{Pitié}
\vspace{-2mm}
\begin{listverse}
\item[\vref{2 S 12:22}] Yahweh n'aura pas p. de moi et
\item[\vref{Job 33:24}] alors Dieu aura p. de lui, et
\item[\vref{Ps 56:2}] Dieu, aie p. de moi ! Car
\item[\vref{Ps 102:14}] temps d'en avoir p., parce que le
\item[\vref{Ps 106:46}] qui les avaient emmenés captifs eurent p. d'eux.
\item[\vref{Ps 119:58}] mon cœur : Aie p. de moi selon
\item[\vref{Ps 123:2}] notre Dieu, jusqu'à ce qu'il ait p. de ns.
\item[\vref{Pr 19:17}] Celui qui a p. du pauvre prête
\item[\vref{Es 55:7}] Yahweh, qui aura p. de lui, et
\item[\vref{Ez 16:5}] qui ait eu p. de toi pour
\item[\vref{Jon 4:10}] dit : Tu as p. du ricin pour
\item[\vref{Mal 1:9}] pour qu'il ait p. de ns. ! C'est
\item[\vref{Mt 9:27}] de David, aie p. de ns. !
\item[\vref{Mt 18:33}] pas aussi avoir p. de ton compagnon
\item[\vref{Mc 5:19}] faites, et comment il a eu p. de toi.
\item[\vref{Mc 10:47}] dire : Jésus, Fils de David, aie p. de moi !
\item[\vref{Lu 16:24}] Père Abraham aie p. de moi, et
\item[\vref{Ph 2:27}] Dieu a eu p. de lui, et
\item[\vref{Jud 1:22}] Et ayez p. des uns en usant de discrétion ;
\end{listverse}

\ConcordanceEntry{Place}
\vspace{-2mm}
\begin{listverse}
\item[\vref{Ge 2:21}] chair à la p. de cette côte.
\item[\vref{Ge 4:25}] fils à la p. d'Abel, que Caïn
\item[\vref{Mt 11:16}] assis sur les p. publiques, et qui
\item[\vref{Lu 2:7}] avait pas de p. pour eux ds
\item[\vref{Lu 14:7}] choisissaient les premières p. ; et il lr.
\item[\vref{Lu 14:21}] promptement ds les p. et ds les
\item[\vref{Jn 14:2}] dit. Je vais vs. préparer une p.
\item[\vref{Ap 2:5}] chandelier de sa p. si tu ne
\item[\vref{Ap 11:8}] étendus sur les p. de la grande
\item[\vref{Ap 20:11}] plus trouvé de p. pour eux.
\item[\vref{Ap 21:21}] perle. Et la p. de la ville
\item[\vref{Ap 22:2}] milieu de la p. de la ville,
\end{listverse}

\ConcordanceEntry{Plaider}
\vspace{-2mm}
\begin{listverse}
\item[\vref{1 S 24:16}] il regardera et p. ma cause, il
\item[\vref{Job 9:3}] Si Dieu veut p. avec lui de
\item[\vref{Pr 18:17}] Celui qui p. le premier paraît
\item[\vref{Ec 6:10}] qu'il ne pourrait p. avec celui qui
\item[\vref{Es 43:26}] ma mémoire, et p. ensemble ; toi, déclare
\item[\vref{Es 59:4}] justice, nul ne p. pour la vérité ;
\item[\vref{Jé 7:5}] un hom. qui p. contre son prochain,
\item[\vref{Jé 25:31}] terre ; car Yahweh p. avec les nations,
\item[\vref{Os 4:4}] soit, qu'on ne p. avec personne, et
\item[\vref{Mi 6:2}] peuple, et il p. avec Israël.
\item[\vref{Mt 5:40}] Si quelqu'un veut p. contre toi, et
\end{listverse}

\ConcordanceEntry{Plaie}
\vspace{-2mm}
\begin{listverse}
\item[\vref{Ge 12:17}] frappa de grandes p. Pharaon et sa
\item[\vref{Ex 9:14}] venir ttes mes p. contre ton cœur,
\item[\vref{Ex 12:13}] aura point de p. de destruction qnd
\item[\vref{No 16:48}] les vivants, la p. fut arrêtée.
\item[\vref{De 28:59}] rendra difficile tes p. et les plaies
\item[\vref{Job 5:18}] qui fait la p. et qui la
\item[\vref{Ps 106:29}] au point qu'une p. fit une brèche
\item[\vref{Ps 106:30}] justice ; et la p. fut arrêtée.
\item[\vref{Ps 147:3}] cœur brisé, et il bande leurs p.
\item[\vref{Pr 27:6}] Les p. faites par un ami sont fidèles,
\item[\vref{Es 1:6}] blessures, meurtrissures et p. pourries, qui n'ont
\item[\vref{Jé 6:14}] la légère la p. de la fille
\item[\vref{Jé 17:18}] jour du malheur, frappe-les d'une double p. !
\item[\vref{Na 3:19}] ta blessure, ta p. est douloureuse ; ts
\item[\vref{Za 14:12}] Voici la p. dont Yahweh frappera ts les peuples
\item[\vref{Lu 10:34}] s'approcha, banda ses p., en y versant
\item[\vref{Ac 16:33}] il lava leurs p., et aussitôt après
\item[\vref{Ap 11:6}] ttes sortes de p., ttes les fois
\end{listverse}

\ConcordanceEntry{Plaindre}
\vspace{-2mm}
\begin{listverse}
\item[\vref{Jg 21:22}] frères viennent se p. auprès de ns.,
\item[\vref{2 S 19:28}] avoir, pour me p. encore au roi ?
\item[\vref{Jé 15:5}] ou qui te p. ? Ou qui se
\item[\vref{Jé 16:5}] lamenter ni te p. avec eux ; car
\item[\vref{La 3:39}] hom. vivant se p.-il, un hom.,
\item[\vref{Ac 19:38}] ont à se p. de quelqu'un, il
\item[\vref{Ja 5:9}] frères, ne vs. p. pas les uns
\end{listverse}

\ConcordanceEntry{Plaine}
\vspace{-2mm}
\begin{listverse}
\item[\vref{Ge 13:10}] que tte la p. du Jourdain était
\item[\vref{No 22:1}] campèrent ds les p. de Moab, au-delà
\item[\vref{Jos 4:13}] dis-je, ds les p. de Jérico environ
\item[\vref{Jg 1:9}] la contrée du midi et la p.
\item[\vref{Ps 65:13}] Les p. du désert sont abreuvées, et les
\item[\vref{Za 7:7}] midi et la p. étaient habités ?
\item[\vref{Lu 6:17}] s'arrêta sur une p. avec la foule
\end{listverse}

\ConcordanceEntry{Plainte}
\vspace{-2mm}
\begin{listverse}
\item[\vref{Job 23:2}] Encore aujourd'hui ma p. est pleine d'amertume,
\item[\vref{Ps 102:1}] et répandant sa p. dvt Yahweh.
\item[\vref{Ha 2:1}] ce que je répondrais après ma p.
\item[\vref{Mal 2:13}] de larmes, de p. et de gémissements,
\item[\vref{Mt 2:18}] des lamentations, des p., et des grands
\item[\vref{Ac 25:2}] les Juifs portèrent p. contre Paul dvt
\end{listverse}

\ConcordanceEntry{Plaire}
\vspace{-2mm}
\begin{listverse}
\item[\vref{Ge 16:6}] com. il te p.. Saraï dc la
\item[\vref{1 S 18:5}] armée, et il p. à tt le
\item[\vref{1 R 3:10}] demande de Salomon p. à Yahweh.
\item[\vref{Est 2:4}] jeune fille qui p. au roi régnera
\item[\vref{Job 34:9}] sert à rien à l'hom. de p. à Dieu.
\item[\vref{Da 4:35}] com. il lui p. avec l'armée des
\item[\vref{Ro 8:8}] selon la chair ne peuvent pas p. à Dieu.
\item[\vref{Ro 15:1}] ne pas ns. p. en ns.-mêmes.
\item[\vref{1 Co 7:32}] Seign., comment il p. au Seign.
\item[\vref{Ga 1:10}] je cherche à p. aux hommes ? Certes,
\item[\vref{Col 1:10}] Seign., pour lui p. en ttes choses,
\item[\vref{1 Th 2:4}] non com. pour p. aux hommes, mais
\item[\vref{1 Th 4:1}] se conduire, et p. à Dieu, vs.
\item[\vref{2 Ti 2:4}] vie s'il veut p. à celui qui
\end{listverse}

\ConcordanceEntry{Plaisir}
\vspace{-2mm}
\begin{listverse}
\item[\vref{1 S 15:22}] répondit : Yahweh prend-il p. aux holocaustes et
\item[\vref{2 S 1:26}] faisais tt mon p. ; l'amour que j'avais
\item[\vref{Ps 1:2}] mais qui prend p. ds la loi
\item[\vref{Ps 22:9}] te sauve, puisqu'il prend son bon p. en toi !
\item[\vref{Ps 51:8}] Mais tu prends p. à la vérité
\item[\vref{Ps 73:25}] je ne prends p. qu'en toi seul.
\item[\vref{Ps 149:4}] Car Yahweh prend p. à son peuple,
\item[\vref{Pr 16:7}] Quand Yahweh prend p. aux voies d'un
\item[\vref{Pr 18:1}] qui lui fait p., et se mêle
\item[\vref{Ec 9:7}] longtemps Dieu prend p. à tes œuvres.
\item[\vref{Ec 12:3}] diras : Je n'y prends point de p. ;
\item[\vref{Es 53:10}] et le bon p. de Yahweh prospérera
\item[\vref{Es 55:11}] quoi je prends p., et prospérera ds
\item[\vref{Ez 20:40}] Là, je prendrai p. en eux, et
\item[\vref{Ez 33:11}] ne prends point p. ds la mort
\item[\vref{Os 6:6}] Car je prends p. à la miséricorde
\item[\vref{Mi 7:18}] parce qu'il prend p. à la miséricorde.
\item[\vref{Ha 1:13}] ne saurais prendre p. à regarder le
\item[\vref{Ag 1:8}] j'y prendrai mon p. et je serai
\item[\vref{Mc 12:37}] Et une grande foule l'écoutait avec p.
\item[\vref{Mc 12:38}] scribes qui prennent p. à se promener
\item[\vref{Lu 8:14}] richesses, et les p. de la vie ;
\item[\vref{Ro 7:22}] je prends bien p. à la loi
\item[\vref{2 Co 12:10}] cela je prends p. ds les faiblesses,
\item[\vref{Ep 1:5}] selon le bon p. de sa volonté,
\item[\vref{2 Th 2:12}] qui ont pris p. à l'iniquité soient
\item[\vref{1 Ti 5:6}] vit ds les p. est morte quoique
\item[\vref{Hé 10:38}] mon âme ne prend pas de p. en lui.
\end{listverse}

\ConcordanceEntry{Plant}
\vspace{-2mm}
\begin{listverse}
\item[\vref{De 32:32}] vigne est du p. de Sodome, et
\item[\vref{Ps 80:16}] et le p. que ta droite avait planté, et
\item[\vref{Ps 128:3}] table com. des p. d'oliviers.
\item[\vref{Jé 2:21}] dont tt le p. était franc ; comment
\end{listverse}

\ConcordanceEntry{Plante}
\vspace{-2mm}
\begin{listverse}
\item[\vref{Ge 2:5}] la terre aucune p. des champs, et
\item[\vref{Es 53:2}] com. une jeune p., com. un rejeton
\item[\vref{Es 60:21}] germe de mes p., l'œuvre de mes
\item[\vref{Mt 13:32}] que les autres p. et devient un
\item[\vref{Mt 15:13}] et dit : Toute p. que mon Père
\item[\vref{Ro 6:5}] devenus une mm p. avec lui par
\end{listverse}

\ConcordanceEntry{Planter}
\vspace{-2mm}
\begin{listverse}
\item[\vref{Ge 2:8}] Aussi Yahweh Dieu p. un jardin en
\item[\vref{Ge 9:20}] terre, commença à p. de la vigne.
\item[\vref{Ge 21:33}] Abraham p. des tamaris à Beer-Schéba ; et là
\item[\vref{De 16:21}] Tu ne p. point d'arbre d'Asherah, près de l'autel
\item[\vref{2 S 7:10}] et je l'ai p. pour qu'il y
\item[\vref{Ps 1:3}] com. un arbre p. près des ruisseaux
\item[\vref{Ps 80:16}] ta droite avait p., et le fils
\item[\vref{Ps 92:14}] Etant p. ds la maison de Yahweh, ils
\item[\vref{Ps 94:9}] Celui qui a p. l'oreille, n'entendrait-il point ?
\item[\vref{Ec 3:2}] un temps pour p. et un temps
\item[\vref{Jé 2:21}] je t'avais moi-mm p. com. une vigne
\item[\vref{Jé 18:9}] royaume, pour l'édifier et pour le p. ;
\item[\vref{Jé 24:6}] plus, je les p. et je ne
\item[\vref{Mt 21:33}] de famille qui p. une vigne, et
\item[\vref{Lu 17:6}] sycomore : Déracine-toi, et p.-toi ds la
\item[\vref{Lu 17:28}] on vendait, on p. et on bâtissait.
\item[\vref{Ro 6:5}] devenus une mm p. avec lui par
\item[\vref{1 Co 3:6}] J'ai p. ; Apollos a arrosé ; mais c'est Dieu
\item[\vref{Ja 1:21}] qui a été p. en vs. et
\end{listverse}

\ConcordanceEntry{Pléiades}
\vspace{-2mm}
\begin{listverse}
\item[\vref{Job 9:9}] ourse, l'orion, les p., et les étoiles
\item[\vref{Job 38:31}] les liens des p. ou détacher les
\item[\vref{Am 5:8}] a fait les P. et l'Orion, qui
\end{listverse}

\ConcordanceEntry{Plein}
\vspace{-2mm}
\begin{listverse}
\item[\vref{Ex 16:33}] et mets-y un p. d'omer de manne,
\item[\vref{Ps 48:11}] ta droite est p. de justice.
\item[\vref{Ps 65:10}] de Dieu est p. d'eau ; tu prépares
\item[\vref{Ps 91:6}] ni la destruction qui frappe en p. midi.
\item[\vref{Ps 104:24}] La terre est p. de tes richesses.
\item[\vref{Pr 17:1}] paix, qu'une maison p. de viandes, là
\item[\vref{Es 58:1}] Crie à p. gosier, ne te retiens pas, élève
\item[\vref{Mt 14:20}] emporta douze paniers p. des morceaux qui
\item[\vref{Mt 23:27}] qui, au-dedans sont p. d'ossements de morts,
\item[\vref{Lu 5:4}] Simon : Avance en p. eau, et jetez
\item[\vref{Ac 2:13}] C'est qu'ils sont p. de vin doux.
\item[\vref{Ro 3:14}] lr. bouche est p. de malédictions et
\item[\vref{1 Co 15:43}] faiblesse, il ressuscite p. de force ;
\item[\vref{2 Ti 3:7}] parvenir à la p. connaissance de la
\item[\vref{Ja 5:11}] le Seign. est p. de compassion et
\item[\vref{Ap 4:6}] trône quatre animaux, p. d'yeux dvt et
\item[\vref{Ap 5:8}] des coupes d'or p. de parfums, qui
\end{listverse}

\ConcordanceEntry{Plénitude}
\vspace{-2mm}
\begin{listverse}
\item[\vref{Es 66:11}] plaisir de la p. de sa gloire.
\item[\vref{Jn 1:16}] reçu de sa p., et grâce pour
\item[\vref{Ep 1:23}] corps, et la p. de celui qui
\item[\vref{Ep 3:19}] vs. soyez remplis de tte la p. de Dieu.
\item[\vref{Col 1:19}] été que tte p. habitât en lui.
\item[\vref{Col 2:9}] corporellement tte la p. de la divinité.
\end{listverse}

\ConcordanceEntry{Pleurer}
\vspace{-2mm}
\begin{listverse}
\item[\vref{Ge 21:16}] de lui, éleva la voix et p.
\item[\vref{Ge 37:35}] disait : C'est en p. que je descendrai
\item[\vref{Ge 50:17}] père. Et Joseph p. qnd on lui
\item[\vref{Ex 2:6}] et voici l'enfant p.. Elle en fut
\item[\vref{Jg 20:26}] à Béthel ; ils p., et restèrent là
\item[\vref{2 S 15:30}] Il montait en p., la tête couverte,
\item[\vref{Esd 3:12}] représentant cette maison-là, p. à haute voix ;
\item[\vref{Esd 10:1}] faisait cette confession, p. et étant prosterné
\item[\vref{Né 8:9}] lamentations, et ne p. point ! Car tt
\item[\vref{Est 4:3}] Juifs ; ils jeûnaient, p., gémissaient et beaucoup
\item[\vref{Ps 137:1}] assis et ns. p. en ns. souvenant
\item[\vref{Ec 3:4}] un temps pour p. et un temps
\item[\vref{Es 22:4}] regards, que je p. amèrement. Ne vs.
\item[\vref{Jé 9:1}] larmes, et je p. jour et nuit
\item[\vref{Jé 13:17}] ceci, mon âme p. en secret, à
\item[\vref{Za 12:10}] percé, et ils p. sur lui, com.
\item[\vref{Mt 2:18}] grands gémissements : Rachel p. ses enfants, et
\item[\vref{Mt 5:4}] sont ceux qui p., car ils seront
\item[\vref{Mt 26:75}] sorti dehors, il p. amèrement.
\item[\vref{Mc 5:38}] c'est-à-dire ceux qui p. et qui poussaient
\item[\vref{Lu 6:21}] êtes-vs., vs. qui p. mntnt, car vs.
\item[\vref{Lu 7:13}] elle, et il lui dit : Ne p. pas !
\item[\vref{Lu 19:41}] en la voyant, p. sur elle, et
\item[\vref{Jn 11:35}] Et Jésus p.
\item[\vref{Jn 16:20}] vs. dis : Vous p. et vs. vs.
\item[\vref{Jn 20:11}] sépulcre dehors, et p.. Et com. elle
\item[\vref{Ro 12:15}] se réjouissent ; et p. avec ceux qui
\end{listverse}

\ConcordanceEntry{Pleurs}
\vspace{-2mm}
\begin{listverse}
\item[\vref{Ge 35:8}] donna le nom d'Allon-Bacuth (chêne des p.).
\item[\vref{Esd 10:1}] peuple se lamenta abondamment par des p.
\item[\vref{Ps 6:9}] a entendu la voix de mes p.
\item[\vref{Ps 30:6}] soir arrivent les p., et le matin
\item[\vref{Ps 35:14}] j'étais abattu, en p., com. pour le
\item[\vref{Es 65:19}] le bruit des p. et le bruit
\item[\vref{Jé 48:5}] P. sur pleurs s'élèveront à la montée
\item[\vref{Joë 2:12}] jeûnes, avec des p. et des lamentations !
\item[\vref{Mt 8:12}] y aura des p. et des grincements
\end{listverse}

\ConcordanceEntry{Pleuvoir}
\vspace{-2mm}
\begin{listverse}
\item[\vref{Ge 2:5}] n'avait pas fait p. sur la terre
\item[\vref{Ge 7:4}] jours, je ferai p. sur la terre
\item[\vref{Ge 19:24}] Alors Yahweh fit p. du ciel, sur
\item[\vref{Ex 9:23}] terre. Yahweh fit p. de la grêle
\item[\vref{Ps 11:6}] Il fait p. sur les méchants des charbons, du
\item[\vref{Ps 78:24}] il fit p. la manne sur eux pour lr.
\item[\vref{Jé 14:22}] a-t-il qui fassent p., et les cieux
\item[\vref{Ja 5:17}] pour qu'il ne p. pas, et il
\item[\vref{Ap 11:6}] afin qu'il ne p. pas pendant les
\end{listverse}

\ConcordanceEntry{Plomb}
\vspace{-2mm}
\begin{listverse}
\item[\vref{No 31:22}] l'argent, l'airain, le fer, l'étain, le p. ;
\item[\vref{Job 19:24}] et sur du p., et qu'ils soient
\item[\vref{Jé 6:29}] est brûlant, le p. est consumé par
\item[\vref{Ez 22:18}] fer et du p. ds un creuset ;
\item[\vref{Za 5:7}] une masse de p., et une fem.
\item[\vref{Za 5:8}] la masse de p. sur l'ouverture.
\end{listverse}

\ConcordanceEntry{Pluie}
\vspace{-2mm}
\begin{listverse}
\item[\vref{Ge 7:12}] Et la p. tomba sur la terre pendant quarante
\item[\vref{Lé 26:4}] vs. donnerai les p. en lr. temps,
\item[\vref{De 11:14}] votre pays la p. en son temps,
\item[\vref{1 S 12:17}] et de la p.. Sachez alors et
\item[\vref{1 R 17:1}] ni rosée ni p., sinon à ma
\item[\vref{Job 38:28}] La p. a-t-elle un père qui enfante les
\item[\vref{Ps 68:10}] fait tomber une p. abondante sur ton
\item[\vref{Ps 84:7}] en fontaine ; la p. la couvre de
\item[\vref{Ps 147:8}] il prépare la p. pour la terre ;
\item[\vref{Pr 16:15}] nuée portant la p. de la dernière
\item[\vref{Es 55:10}] Car com. la p. et la neige
\item[\vref{Jé 5:24}] ns. donne la p. en son temps,
\item[\vref{Am 4:7}] aussi privés de p., qnd il restait
\item[\vref{Za 10:1}] à Yahweh la p., la pluie au
\item[\vref{Mt 5:45}] il envoie sa p. sur les justes
\item[\vref{Lu 17:29}] de Sodome, une p. de feu et
\item[\vref{Ac 14:17}] du ciel les p. et les saisons
\item[\vref{Hé 6:7}] abreuvée par la p. qui tombe souvent
\item[\vref{Ja 5:17}] tomba pas de p. sur la terre
\end{listverse}

\ConcordanceEntry{Plume}
\vspace{-2mm}
\begin{listverse}
\item[\vref{Jg 5:14}] qui manient la p. du scribe.
\item[\vref{Ps 45:2}] sera com. la p. d'un habile écrivain !
\item[\vref{Ps 91:4}] couvrira de ses p., et tu trouveras
\item[\vref{Jé 8:8}] faussement, et la p. des scribes est
\item[\vref{Ez 17:3}] déployées, couvert de p. de ttes les
\end{listverse}

\ConcordanceEntry{Poids}
\vspace{-2mm}
\begin{listverse}
\item[\vref{Pr 11:1}] Yahweh, mais le p. juste lui est
\item[\vref{Pr 20:10}] Le double p. et la double
\item[\vref{2 Co 4:17}] en ns. un p. éternel d'une gloire
\end{listverse}

\ConcordanceEntry{Poing}
\vspace{-2mm}
\begin{listverse}
\item[\vref{Ex 21:18}] ou avec le p., sans causer sa
\item[\vref{Es 58:4}] pour frapper du p. méchamment ; vs. ne
\item[\vref{Mt 26:67}] des coups de p. et des soufflets,
\end{listverse}

\ConcordanceEntry{Poisson}
\vspace{-2mm}
\begin{listverse}
\item[\vref{Ge 1:21}] créa les grands p. et ts les
\item[\vref{Ex 7:21}] Et le p. qui était ds le fleuve mourut,
\item[\vref{No 11:5}] ns. souvenons des p. que ns. mangions
\item[\vref{Ps 8:9}] ciel et les p. de la mer,
\item[\vref{Ez 47:9}] grande quantité de p. ; car là où
\item[\vref{Jon 2:1}] à un grand p. d'engloutir Jonas, et
\item[\vref{Mt 7:10}] lui demande un p., lui donnera-t-il un
\item[\vref{Mt 12:40}] ventre d'un grand p., de mm le
\item[\vref{Mt 14:17}] ici que cinq pains et deux p.
\item[\vref{Mt 15:34}] lui dirent : Sept, et quelques petits p.
\item[\vref{Mt 17:27}] prends le premier p. qui viendra ; ouvre-lui
\item[\vref{Lu 24:42}] un morceau de p. rôti, et un
\item[\vref{Jn 21:11}] cent cinquante-trois grands p. ; et quoiqu'il y
\end{listverse}

\ConcordanceEntry{Poitrine}
\vspace{-2mm}
\begin{listverse}
\item[\vref{Lé 7:31}] l'autel, mais la p. sera pour Aaron
\item[\vref{Da 2:32}] très fin, sa p. et ses bras
\item[\vref{Na 2:8}] colombes, frappant leurs p. com. un tambour.
\item[\vref{Mt 24:30}] se frappant la p., et verront le
\item[\vref{Jn 13:25}] penché sur la p. de Jésus, lui
\item[\vref{Ap 1:13}] ayant une ceinture d'or sur la p.
\end{listverse}

\ConcordanceEntry{Poix}
\vspace{-2mm}
\begin{listverse}
\item[\vref{Ge 6:14}] tu l'enduiras de p. en dedans et
\item[\vref{Ex 2:3}] bitume et de p., mit l'enfant dedans,
\item[\vref{Es 34:9}] seront changés en p., et sa poussière
\end{listverse}

\ConcordanceEntry{Porc}
\vspace{-2mm}
\begin{listverse}
\item[\vref{Lé 11:7}] Le p., car il a bien le sabot
\item[\vref{Es 65:4}] la chair de p., et ayant ds
\item[\vref{Es 66:17}] la chair de p. et des choses
\end{listverse}

\ConcordanceEntry{Port}
\vspace{-2mm}
\begin{listverse}
\item[\vref{Ps 107:30}] les conduit au p. qu'ils désiraient.
\item[\vref{Ac 27:12}] Et com. le p. n'était pas bon
\end{listverse}

\ConcordanceEntry{Porte}
\vspace{-2mm}
\begin{listverse}
\item[\vref{Ge 18:1}] assis à la p. de sa tente,
\item[\vref{Ge 22:17}] postérité possédera la p. de ses ennemis.
\item[\vref{Ge 28:17}] c'est ici la p. des cieux !
\item[\vref{Ex 12:7}] linteau de la p. des maisons où
\item[\vref{Ex 32:26}] tenant à la p. du camp, dit :
\item[\vref{De 6:9}] de ta maison et sur tes p.
\item[\vref{Jos 6:26}] il posera ses p. sur son puîné.
\item[\vref{Ru 4:11}] était à la p. et les anciens
\item[\vref{1 R 6:31}] il fit une p. à deux battants
\item[\vref{Né 3:1}] ils rebâtirent la p. des brebis. Ils
\item[\vref{Ps 24:7}] P., élevez vos linteaux ; élevez-vs. portes éternelles,
\item[\vref{Ps 78:23}] il ouvrit les p. des cieux ;
\item[\vref{Ps 87:2}] Yahweh aime les p. de Sion, plus
\item[\vref{Ps 107:18}] ils touchent aux p. de la mort.
\item[\vref{Ps 118:19}] Ouvrez-moi les p. de la justice ;
\item[\vref{Ps 122:2}] s'arrêtent ds tes p., ô Jérus. !
\item[\vref{Pr 8:34}] jour à mes p., et qui monte
\item[\vref{Es 60:18}] Salut ; et tes p. : Louange.
\item[\vref{Ez 46:1}] Seign. Yahweh : La p. du parvis intérieur,
\item[\vref{Za 8:16}] vue de la paix ds vos p. ;
\item[\vref{Mal 1:10}] vs. fermera les p. pour que vs.
\item[\vref{Mt 6:6}] ayant fermé ta p., prie ton Père,
\item[\vref{Mt 16:18}] Eglise, et les p. de l'enfer ne
\item[\vref{Mt 25:10}] noces, puis la p. fut fermée.
\item[\vref{Lu 11:7}] pas, car ma p. est déjà fermée,
\item[\vref{Lu 13:25}] aura fermé la p., et que vs.,
\item[\vref{Jn 10:7}] Je suis la p. par où entrent
\item[\vref{Jn 10:9}] Je suis la p.. Si quelqu'un entre
\item[\vref{Jn 18:16}] dehors à la p.. Et l'autre disciple,
\item[\vref{Jn 20:19}] la semaine, les p. du lieu où
\item[\vref{Ac 5:19}] la nuit les p. de la prison,
\item[\vref{Ac 14:27}] aux Gentils la p. de la foi.
\item[\vref{Col 4:3}] ns. ouvre une p. pour la parole,
\item[\vref{Hé 13:12}] sang, a souffert hors de la p.
\item[\vref{Ja 5:9}] le Juge se tient à la p.
\item[\vref{Ap 3:8}] j'ai ouvert une p. dvt toi, et
\item[\vref{Ap 3:20}] tiens à la p., et je frappe.
\item[\vref{Ap 21:12}] muraille, avec douze p., et aux portes
\item[\vref{Ap 21:21}] Et les douze p. étaient douze perles ;
\item[\vref{Ap 22:14}] d'entrer par les p. ds la ville.
\end{listverse}

\ConcordanceEntry{Porter}
\vspace{-2mm}
\begin{listverse}
\item[\vref{Ex 19:4}] je vs. ai p. com. sur des
\item[\vref{No 11:14}] à moi seul, p. tt ce peuple,
\item[\vref{No 24:18}] et Israël se p. vaillamment.
\item[\vref{De 1:31}] ton Dieu, t'a p. com. un hom.
\item[\vref{Jos 3:6}] prêtres, en disant : P. l'arche de l'alliance,
\item[\vref{Ps 89:51}] et comment je p. ds mon sein
\item[\vref{Ps 91:12}] ils te p. sur les mains, de peur que
\item[\vref{Es 53:4}] vérité, il a p. nos maladies, et
\item[\vref{Es 53:12}] que lui-mm a p. les péchés de
\item[\vref{Ez 16:58}] Tu p. sur toi tes méchancetés et tes
\item[\vref{Ez 18:20}] Le fils ne p. point l'iniquité du
\item[\vref{Mt 13:23}] la comprend. Il p. du fruit, et
\item[\vref{Mt 22:20}] demanda : De qui p.-t-il l'image et
\item[\vref{Jn 15:8}] Si vs. p. beaucoup de fruit, mon Père sera
\item[\vref{Ro 1:5}] l'apostolat, afin de p. ts les Gentils
\item[\vref{2 Co 6:14}] Ne p. pas un mm joug avec les
\item[\vref{Ga 6:5}] Car chacun p. son propre fardeau.
\item[\vref{Ph 2:15}] le monde, qui p. au dvt d'eux
\item[\vref{Ph 3:14}] arrière, et me p. vers celles qui
\item[\vref{1 Pi 2:24}] lui qui a p. lui-mm nos péchés
\end{listverse}

\ConcordanceEntry{Portier}
\vspace{-2mm}
\begin{listverse}
\item[\vref{1 Ch 9:24}] y avait des p. aux quatre vents,
\item[\vref{1 Ch 26:1}] aux classes des p., il y eut
\item[\vref{Esd 2:42}] Des fils des p. : Les fils de
\item[\vref{Né 7:45}] P. : Les fils de Schallum, les fils
\item[\vref{Mc 13:34}] et ordonne au p. de veiller.
\item[\vref{Jn 10:3}] Le p. lui ouvre, les brebis entendent sa
\end{listverse}

\ConcordanceEntry{Portion}
\vspace{-2mm}
\begin{listverse}
\item[\vref{Ge 31:14}] Avons-ns. encore qq p. et qq héritage
\item[\vref{Ge 33:19}] Il acheta une p. du champ où
\item[\vref{De 32:9}] car la p. de Yahweh, c'est son peuple, Jacob
\item[\vref{1 S 1:5}] à Anne une p. double ; car il
\item[\vref{2 R 2:9}] j'aie, une double p. de ton esprit !
\item[\vref{Né 8:10}] et envoyez-en des p. à ceux qui
\item[\vref{Pr 31:15}] elle donne à ses servantes lr. p.
\item[\vref{Es 61:7}] confusion est lr. p. ; c'est pourquoi ils
\item[\vref{Jé 10:16}] La p. de Jacob n'est pas com. ces
\item[\vref{Ez 47:13}] douze tribus d'Israël. Joseph aura deux p.
\end{listverse}

\ConcordanceEntry{Possédé (un)}
\vspace{-2mm}
\begin{listverse}
\item[\vref{Mt 10:8}] démons hors des p.. Vous l'avez reçu
\item[\vref{Mc 1:39}] et chassait les démons hors des p.
\item[\vref{Mc 3:22}] disaient : Il est p. par Béelzébul ; c'est
\item[\vref{Mc 3:30}] disaient : Il est p. d'un esprit impur.
\item[\vref{Mc 5:2}] barque, un hom. p. d'un esprit impur,
\item[\vref{Lu 13:11}] fem. qui était p. d'un démon qui
\item[\vref{Ac 8:7}] qui en étaient p., et beaucoup de
\item[\vref{Ac 19:13}] ceux qui étaient p. d'esprits malins, en
\item[\vref{1 Ti 6:10}] quelques-uns en étant p., se sont détournés
\end{listverse}

\ConcordanceEntry{Posséder}
\vspace{-2mm}
\begin{listverse}
\item[\vref{Ge 15:8}] à quoi connaîtrai-je que je le p. ?
\item[\vref{Ge 22:17}] et ta postérité p. la porte de
\item[\vref{No 13:30}] il dit : Montons, p. ce pays, car
\item[\vref{De 4:22}] passerez, et vs. p. ce bon pays.
\item[\vref{Jos 1:11}] vs. donne afin que vs. le p.
\item[\vref{Job 42:10}] double de tt ce qu'il avait p.
\item[\vref{Ps 25:13}] et sa postérité p. la terre en
\item[\vref{Es 61:7}] c'est pourquoi ils p. le double ds
\item[\vref{Es 63:18}] peuple saint n'a p. le pays que
\item[\vref{Ab 1:17}] maison de Jacob p. ses possessions.
\item[\vref{Ha 1:6}] la terre, pour p. des demeures qui
\item[\vref{So 2:9}] les restes de ma nation les p.
\item[\vref{Za 2:12}] Et Yahweh p. Juda com. sa
\item[\vref{Ac 4:34}] ts ceux qui p. des champs ou
\item[\vref{2 Co 6:10}] rien et toutefois p. ttes choses.
\item[\vref{1 Th 4:4}] de vs. sache p. son corps ds
\item[\vref{1 Jn 5:15}] savons que ns. p. la chose que
\end{listverse}

\ConcordanceEntry{Possession}
\vspace{-2mm}
\begin{listverse}
\item[\vref{Ge 17:8}] de Canaan, en p. perpétuelle, et je
\item[\vref{No 18:20}] part et ta p., au milieu des
\item[\vref{Ps 16:6}] échu, une belle p. m'est accordée.
\item[\vref{Ps 135:4}] choisi Jacob et Israël pour sa p.
\item[\vref{Mi 2:2}] Ils convoitent des p. et s'en emparent,
\item[\vref{Lu 15:15}] l'envoya ds ses p. pour paître les
\item[\vref{Lu 19:12}] éloigné, pour prendre p. d'un royaume, et
\item[\vref{Ac 2:45}] ils vendaient leurs p. et leurs biens,
\item[\vref{Ac 5:1}] et Saphira, sa fem., vendit une p.,
\item[\vref{Ap 19:6}] Tout-Puissant a pris p. de son Royaume.
\end{listverse}

\ConcordanceEntry{Possible}
\vspace{-2mm}
\begin{listverse}
\item[\vref{Ps 145:3}] il n'est pas p. de sonder sa
\item[\vref{Mt 19:26}] quant à Dieu ttes choses sont p.
\item[\vref{Mt 24:24}] séduire mm les élus, s'il était p.
\item[\vref{Mt 26:39}] Père, s'il est p., fais que cette
\item[\vref{Mt 26:42}] s'il n'est pas p. que cette coupe
\item[\vref{Mc 9:23}] ttes choses sont p. au croyant.
\item[\vref{Mc 10:27}] à Dieu ; car ttes choses sont p. à Dieu.
\item[\vref{Mc 13:22}] séduire mm les élus, s'il était p.
\item[\vref{Mc 14:36}] choses te sont p., éloigne de moi
\item[\vref{Lu 18:27}] qui est impossible aux hommes est p. à Dieu.
\item[\vref{Ac 2:24}] qu'il n'était pas p. qu'il soit retenu
\item[\vref{Ac 8:22}] que, s'il est p., la pensée de
\item[\vref{Ro 12:18}] S'il est p., autant que cela dépend de vs.,
\end{listverse}

\ConcordanceEntry{Poste}
\vspace{-2mm}
\begin{listverse}
\item[\vref{2 Ch 30:16}] tinrent à lr. p., selon lr. charge,
\item[\vref{Né 7:3}] chacun à son p., et chacun dvt
\item[\vref{Né 12:44}] que les Lévites étaient à lr. p.,
\item[\vref{Es 21:8}] suis à mon p. ttes les nuits ;
\end{listverse}

\ConcordanceEntry{Postérité}
\vspace{-2mm}
\begin{listverse}
\item[\vref{Ge 3:15}] et entre ta p. et sa postérité ;
\item[\vref{Ge 12:7}] pays à ta p.. Et Abram bâtit
\item[\vref{Ge 21:12}] en Isaac te sera donnée une p.
\item[\vref{Ps 22:31}] La p. le servira, on parlera du Seign.
\item[\vref{Ps 89:5}] J'affermirai ta p. pour toujours, et
\item[\vref{Ps 89:30}] rendrai éternelle sa p., et son trône
\item[\vref{Ps 105:6}] La p. d'Abraham sont ses serviteurs ; les enfants
\item[\vref{Es 53:10}] il verra une p. et prolongera ses
\item[\vref{Es 54:3}] gauche, et ta p. possédera les nations
\item[\vref{Mal 2:15}] qu'il cherchait une p. de Dieu. Gardez-vs.
\item[\vref{Mt 22:24}] et suscitera une p. à son frère.
\item[\vref{Jn 7:42}] venir de la p. de David, et
\item[\vref{Jn 8:33}] Nous sommes la p. d'Abraham, et ns.
\item[\vref{Ro 4:18}] avait été dit : Ainsi sera ta p.
\item[\vref{Ro 9:7}] sont de la p. d'Abraham, ils ne
\item[\vref{Ga 3:29}] êtes dc la p. d'Abraham, et héritiers
\item[\vref{Hé 11:18}] Les descendants d'Isaac seront ta véritable p.
\item[\vref{Ap 22:16}] rejeton et la p. de David, l'étoile
\end{listverse}

\ConcordanceEntry{Potier}
\vspace{-2mm}
\begin{listverse}
\item[\vref{Ps 2:9}] en pièces com. un vase de p.
\item[\vref{Es 29:16}] com. l'argile d'un p. ? Même l'ouvrage dira-t-il
\item[\vref{Es 41:25}] foule com. le p. foule la boue.
\item[\vref{Jé 18:2}] la maison d'un p. ; et là, je
\item[\vref{Da 2:41}] partie d'argile de p. et en partie
\item[\vref{Za 11:13}] dit : Jette-le au p., ce prix honorable
\item[\vref{Mt 27:7}] le champ d'un p. pour la sépulture
\item[\vref{Mt 27:10}] le champ d'un p., selon ce que
\item[\vref{Ro 9:21}] Le p. de terre n'a-t-il pas le pouvoir
\item[\vref{Ap 2:27}] les vases d'un p., ainsi que j'en
\end{listverse}

\ConcordanceEntry{Potiphar}
\vspace{-2mm}
\begin{listverse}
\item[\vref{Ge 37:36}] en Egypte à P., eunuque de Pharaon,
\item[\vref{Ge 39:1}] Joseph en Egypte, P., eunuque de Pharaon,
\end{listverse}

\ConcordanceEntry{Poule}
\vspace{-2mm}
\begin{listverse}
\item[\vref{Mt 23:37}] enfants, com. la p. rassemble ses poussins
\item[\vref{Lu 13:34}] enfants, com. la p. rassemble ses poussins
\end{listverse}

\ConcordanceEntry{Pourceau}
\vspace{-2mm}
\begin{listverse}
\item[\vref{Pr 11:22}] un anneau d'or au groin d'un p.
\item[\vref{Es 66:3}] le sang d'un p. ; celui qui fait
\item[\vref{Mt 7:6}] perles dvt les p., de peur qu'ils
\item[\vref{Mt 8:31}] permets-ns. d'entrer ds ce troupeau de p.
\item[\vref{Mc 5:13}] entrèrent ds les p., qui étaient environ
\item[\vref{Lu 15:16}] carouges que les p. mangeaient, mais personne
\end{listverse}

\ConcordanceEntry{Pourpre}
\vspace{-2mm}
\begin{listverse}
\item[\vref{Ex 26:36}] un rideau de p., d'écarlate, de cramoisi
\item[\vref{Ex 28:6}] l'éphod d'or, de p., d'écarlate, de cramoisi,
\item[\vref{Da 5:29}] revêtit Daniel de p., on lui mit
\item[\vref{Na 2:4}] sont teints de p. ; le fer des
\item[\vref{Mc 15:17}] d'une robe de p., et posèrent sur
\item[\vref{Lu 16:19}] était vêtu de p. et de fin
\item[\vref{Ac 16:14}] Lydie, marchande de p., de la ville
\item[\vref{Ap 17:4}] était vêtue de p. et d'écarlate, et
\end{listverse}

\ConcordanceEntry{Pourriture}
\vspace{-2mm}
\begin{listverse}
\item[\vref{Pr 12:4}] est com. la p. ds ses os.
\item[\vref{Pr 14:30}] la chair, mais l'envie est la p. des os.
\item[\vref{Es 5:24}] sera com. la p., et lr. fleur
\item[\vref{Es 38:17}] fosse de la p., car tu as
\item[\vref{Os 5:12}] com. de la p. pour la maison
\item[\vref{Ha 3:16}] mes lèvres ; la p. entre ds mes
\item[\vref{Za 14:12}] chacun tombera en p. tandis qu'ils seront
\end{listverse}

\ConcordanceEntry{Poursuivre}
\vspace{-2mm}
\begin{listverse}
\item[\vref{Ge 49:23}] les archers l'ont p. de lr. haine.
\item[\vref{Ex 14:8}] roi d'Egypte, qui p. les enfants d'Israël.
\item[\vref{Ex 15:9}] L'ennemi disait : Je p., j'atteindrai, je partagerai
\item[\vref{Lé 26:8}] d'entre vs. en p. cent, et cent
\item[\vref{2 R 9:27}] mais Jéhu le p. et dit : Frappez-le
\item[\vref{Ez 35:6}] le sang te p. ; parce que tu
\item[\vref{Os 8:3}] a rejeté le bien ; l'ennemi le p.
\item[\vref{Am 1:11}] parce qu'il a p. son frère avec
\item[\vref{Jn 5:16}] pourquoi les Juifs p. Jésus et cherchaient
\item[\vref{Ac 13:14}] De Perge, ils p. lr. route, et
\item[\vref{Hé 12:1}] si aisément, et p. constamment la course
\end{listverse}

\ConcordanceEntry{Pourvoir}
\vspace{-2mm}
\begin{listverse}
\item[\vref{Ge 22:8}] fils, Dieu se p. lui-mm de l'agneau
\item[\vref{Ge 22:14}] de Yahweh-Jiré (Yahweh p.) ; c'est pourquoi on
\item[\vref{1 R 12:16}] Et toi David, p. mntnt à ta
\item[\vref{Ac 21:24}] avec eux, et p. à leurs besoins,
\item[\vref{Ph 2:25}] envoyé de quoi p. à mes besoins.
\item[\vref{Ph 4:19}] Aussi mon Dieu p. à tt ce
\end{listverse}

\ConcordanceEntry{Poussière}
\vspace{-2mm}
\begin{listverse}
\item[\vref{Ge 2:7}] l'hom. de la p. de la terre,
\item[\vref{Ge 3:14}] tu mangeras la p. ts les jours
\item[\vref{Ge 3:19}] que tu es p., tu retourneras aussi
\item[\vref{Ge 13:16}] postérité com. la p. de la terre ;
\item[\vref{Ex 8:12}] et frappe la p. de la terre,
\item[\vref{1 S 2:8}] pauvre de la p., et il tire
\item[\vref{Job 7:5}] de monceaux de p., ma peau se
\item[\vref{Job 42:6}] repens sur la p. et sur la
\item[\vref{Ps 22:16}] réduis à la p. de la mort.
\item[\vref{Ps 22:30}] descendent ds la p. s'inclineront, mm celui
\item[\vref{Ps 90:3}] l'hom. à la p., et tu dis :
\item[\vref{Ps 102:15}] aiment ses pierres et chérissent sa p.
\item[\vref{Ps 103:14}] souvenant que ns. ne sommes que p.
\item[\vref{Ps 119:25}] attachée à la p., fais-moi revivre selon
\item[\vref{Pr 8:26}] commencement de la p. du monde habitable.
\item[\vref{Ec 3:20}] fait de la p., et tt retourne
\item[\vref{Ec 12:9}] avant que la p. retourne ds la
\item[\vref{Da 12:2}] dorment ds la p. de la terre
\item[\vref{Mt 10:14}] cette ville, la p. de vos pieds.
\item[\vref{1 Co 15:47}] tiré de la p., mais le second
\item[\vref{Ap 18:19}] jetteront de la p. sur leurs têtes,
\end{listverse}

\ConcordanceEntry{Poutre}
\vspace{-2mm}
\begin{listverse}
\item[\vref{Ca 1:17}] Les p. de nos maisons sont de cèdre,
\item[\vref{Ha 2:11}] la charpente la p. lui répond.
\item[\vref{Mt 7:3}] n'aperçois-tu pas la p. qui est ds
\item[\vref{Lu 6:42}] vois pas la p. qui est ds
\end{listverse}

\ConcordanceEntry{Pouvoir (le)}
\vspace{-2mm}
\begin{listverse}
\item[\vref{No 14:16}] n'avait pas le p. de faire entrer
\item[\vref{De 28:32}] tu n'auras aucun p. en ta main.
\item[\vref{Job 1:12}] est en ton p. ; seulement ne porte
\item[\vref{Ps 22:21}] mon unique du p. des chiens !
\item[\vref{Ps 49:16}] mon âme du p. du scheol, qnd
\item[\vref{Pr 18:21}] vie sont au p. de la langue,
\item[\vref{Ec 8:8}] son souffle pour p. le retenir, il
\item[\vref{Es 59:1}] trop courte pour p. sauver, ni son
\item[\vref{Es 63:1}] qui ai tt p. de sauver.
\item[\vref{Jé 10:23}] n'est pas au p. de l'hom. qui
\item[\vref{Da 3:27}] n'avait eu aucun p. sur leurs corps,
\item[\vref{Mt 9:6}] l'hom. a le p. sur la terre
\item[\vref{Jn 1:12}] a donné le p. de devenir enfants
\item[\vref{Jn 10:18}] moi-mm. J'ai le p. de la donner,
\item[\vref{Jn 19:10}] que j'ai le p. de te crucifier,
\item[\vref{Jn 19:11}] Tu n'aurais aucun p. sur moi, s'il
\item[\vref{Ac 8:19}] Donnez-moi aussi ce p., afin que ts
\item[\vref{Ro 6:9}] plus, la mort n'a plus de p. sur lui.
\item[\vref{Ro 7:1}] loi exerce son p. sur l'hom. durant
\item[\vref{2 Co 13:8}] ns. n'avons aucun p. contre la vérité,
\item[\vref{Ep 6:13}] Dieu, afin de p. résister ds le
\item[\vref{Hé 2:14}] qui avait le p. de la mort,
\item[\vref{Ap 11:6}] Ils ont le p. de fermer le
\item[\vref{Ap 13:5}] aussi donné le p. d'agir pendant quarante-deux
\end{listverse}

\ConcordanceEntry{Pratique}
\vspace{-2mm}
\begin{listverse}
\item[\vref{De 5:1}] et veillez à les mettre en p.
\item[\vref{Ez 33:31}] mettent point en p. ; ils les répètent
\item[\vref{Mt 7:24}] les met en p., je le comparerai
\item[\vref{Ac 19:19}] adonnés à des p. magiques, apportèrent leurs
\item[\vref{Ja 1:22}] Et mettez en p. la parole, et
\end{listverse}

\ConcordanceEntry{Pratiquer}
\vspace{-2mm}
\begin{listverse}
\item[\vref{Lé 18:5}] l'hom. qui les p. vivra par elles.
\item[\vref{Né 9:34}] pères n'ont point p. ta loi et
\item[\vref{Ps 119:40}] Voici, je désire p. tes commandements, fais-moi
\item[\vref{Ez 36:27}] vs. observiez et p. mes lois.
\item[\vref{Os 7:1}] car ils ont p. la fausseté ; le
\item[\vref{Mt 6:1}] Gardez-vs. de p. votre justice dvt
\item[\vref{Jn 13:17}] êtes bénis, pourvu que vs. les p.
\item[\vref{Ga 5:3}] est tenu de p. la loi tt
\item[\vref{1 Ti 4:15}] P. ces choses et donne-toi tt entier
\item[\vref{Tit 3:8}] de s'appliquer à p. les bonnes œuvres.
\item[\vref{Ja 1:25}] auditeur oublieux, mais p. les œuvres qui
\item[\vref{Ap 22:11}] qui est juste p. encore la justice ;
\item[\vref{Ap 22:15}] quiconque aime et p. le mensonge.
\end{listverse}

\ConcordanceEntry{Précéder}
\vspace{-2mm}
\begin{listverse}
\item[\vref{Ps 18:13}] splendeur qui le p., s'échappaient les nuées,
\item[\vref{Mt 26:32}] ressuscité, je vs. p. en Galilée.
\item[\vref{Jn 1:15}] après moi m'a p., car il était
\item[\vref{1 Th 4:15}] du Seign., ne p. pas ceux qui
\item[\vref{Hé 7:18}] commandement qui a p., à cause de
\end{listverse}

\ConcordanceEntry{Précepte}
\vspace{-2mm}
\begin{listverse}
\item[\vref{De 6:20}] veulent dire ces p., ces lois, et
\item[\vref{Ps 93:5}] Tes p. sont entièrement fidèles. Yahweh ! La sainteté
\item[\vref{Ps 119:36}] cœur à tes p. et non point
\item[\vref{Ps 119:129}] [Pe.] Tes p. sont merveilleux, c'est
\item[\vref{Es 28:10}] faut lr. donner p. après précepte, précepte
\item[\vref{Es 28:13}] sera pour eux p. sur précepte, précepte
\item[\vref{1 Th 4:2}] vs. savez quels p. ns. vs. avons
\end{listverse}

\ConcordanceEntry{Prêcher}
\vspace{-2mm}
\begin{listverse}
\item[\vref{Ps 40:10}] Je p. ta justice ds la grande assemblée ;
\item[\vref{Mt 3:1}] temps-là arriva Jean-Baptiste, p. ds le désert
\item[\vref{Mt 4:17}] Jésus commença à p., et à dire :
\item[\vref{Mc 1:38}] afin que j'y p. aussi ; car c'est
\item[\vref{Mc 3:15}] pour les envoyer p., avec la puissance
\item[\vref{Mc 16:20}] ils s'en allèrent p. partout. Le Seign.
\item[\vref{Ac 8:25}] dc après avoir p. et annoncé la
\item[\vref{Ac 9:20}] Et aussitôt il p. ds les synagogues
\item[\vref{Ac 15:21}] gens qui le p., puisqu'on le lit
\item[\vref{Ac 21:28}] Voici l'hom. qui p. partout et à
\item[\vref{Ro 10:14}] s'il n'y a personne qui lr. p. ?
\item[\vref{Ro 15:20}] pas encore été p., afin de ne
\item[\vref{1 Co 9:27}] désapprouvé après avoir p. aux autres.
\item[\vref{2 Co 4:5}] ns. ne ns. p. pas ns.-mêmes, mais
\item[\vref{2 Co 11:4}] quelqu'un vient vs. p. un autre Jésus
\item[\vref{Ph 1:15}] vrai que quelques-uns p. Christ par envie
\item[\vref{1 Ti 3:16}] vu des anges, p. aux Gentils, cru
\item[\vref{2 Ti 4:2}] p. la parole, insiste en tte occasion,
\item[\vref{1 Pi 1:12}] qui vs. ont p. l'Evangile par le
\item[\vref{1 Pi 3:19}] il est allé p. aux esprits qui
\end{listverse}

\ConcordanceEntry{Précieux}
\vspace{-2mm}
\begin{listverse}
\item[\vref{1 S 26:21}] vie t'a été p.. Voici, j'ai agi
\item[\vref{Ps 36:8}] Dieu ! Combien est p. ta bonté ! Aussi
\item[\vref{Ps 72:14}] lr. sang sera p. dvt ses yeux.
\item[\vref{Pr 3:15}] Elle est plus p. que les perles,
\item[\vref{Pr 6:26}] chasse après l'âme p. de l'hom.
\item[\vref{Pr 16:16}] Combien est-il plus p. que l'or fin,
\item[\vref{Jé 15:19}] sépares la chose p. de la méprisable,
\item[\vref{1 Pi 1:7}] foi, beaucoup plus p. que l'or périssable,
\item[\vref{1 Pi 1:19}] par le sang p. de Christ, com.
\end{listverse}

\ConcordanceEntry{Précipitation}
\vspace{-2mm}
\begin{listverse}
\item[\vref{Ps 31:23}] disais ds ma p. : Je suis retranché
\item[\vref{Ps 116:11}] disais ds ma p. : Tout hom. est
\item[\vref{Jé 46:5}] ils s'enfuient avec p. sans regarder derrière
\item[\vref{Ac 19:36}] apaiser et ne rien faire avec p.
\item[\vref{1 Ti 5:22}] à personne avec p., et ne participe
\end{listverse}

\ConcordanceEntry{Précipiter}
\vspace{-2mm}
\begin{listverse}
\item[\vref{Ps 18:30}] moyen, je me p. sur une troupe,
\item[\vref{2 Pi 2:4}] s'il les a p. ds l'abîme, et
\item[\vref{Ap 12:9}] Et il fut p. le grand dragon,
\item[\vref{Ap 12:10}] Dieu jour et nuit, a été p.
\item[\vref{Ap 18:21}] disant : Ainsi sera p. avec impétuosité Babylone,
\end{listverse}

\ConcordanceEntry{Précurseur}
\vspace{-2mm}
\begin{listverse}
\item[\vref{Hé 6:20}] entré com. notre p., ayant été fait
\end{listverse}

\ConcordanceEntry{Prédestiner}
\vspace{-2mm}
\begin{listverse}
\item[\vref{Ro 8:29}] les a aussi p. à être conformes
\item[\vref{1 Co 2:7}] les siècles, avait p. pour notre gloire,
\item[\vref{Ep 1:5}] ns. ayant p. pour ns. adopter
\item[\vref{Ep 1:11}] héritiers, ayant été p., suivant la résolution
\item[\vref{1 Pi 1:20}] p. avant la fondation du monde, et
\end{listverse}

\ConcordanceEntry{Prédicateur}
\vspace{-2mm}
\begin{listverse}
\item[\vref{Ro 10:15}] y aura-t-il des p., s'ils ne sont
\item[\vref{1 Ti 2:7}] j'ai été établi p., apôtre (je dis
\item[\vref{2 Ti 1:11}] j'ai été établi p., apôtre et docteur
\item[\vref{2 Pi 2:5}] qui était le p. de la justice ;
\end{listverse}

\ConcordanceEntry{Prédication}
\vspace{-2mm}
\begin{listverse}
\item[\vref{Es 53:1}] cru à notre p. ? Et à qui
\item[\vref{Mt 12:41}] repentirent à la p. de Jonas ; et
\item[\vref{1 Co 1:18}] Car la p. de la croix est une folie
\item[\vref{1 Th 2:3}] eu ds notre p. ni séduction, ni
\item[\vref{1 Ti 5:17}] travaillent à la p. et à l'enseignement.
\item[\vref{2 Ti 4:17}] afin que ma p. soit pleinement approuvée
\end{listverse}

\ConcordanceEntry{Prédiction}
\vspace{-2mm}
\begin{listverse}
\item[\vref{Es 47:13}] qui font leurs p. selon les lunes,
\end{listverse}

\ConcordanceEntry{Prédire}
\vspace{-2mm}
\begin{listverse}
\item[\vref{Ge 41:54}] com. Joseph l'avait p.. Et la famine
\item[\vref{1 S 25:30}] bien qu'il t'a p., et qu'il t'établira
\item[\vref{Es 43:12}] moi qui ai p. ce qui devait
\item[\vref{Mt 24:25}] Voici, je vs. l'ai p.
\item[\vref{Ac 3:24}] parlé, ont aussi p. ces jours.
\item[\vref{Ac 26:22}] et Moïse ont p. devoir arriver,
\end{listverse}

\ConcordanceEntry{Préférer}
\vspace{-2mm}
\begin{listverse}
\item[\vref{Job 34:19}] riches pour les p. aux pauvres, parce
\item[\vref{Ps 132:13}] Sion, il l'a p. pour être son
\item[\vref{Es 41:9}] appelé en te p. aux plus excellents
\item[\vref{1 Ti 5:21}] ces choses sans p. l'un à l'autre,
\end{listverse}

\ConcordanceEntry{Prémices}
\vspace{-2mm}
\begin{listverse}
\item[\vref{De 26:10}] voici, j'apporte les p. des fruits de
\item[\vref{Pr 3:9}] biens et les p. de tt ton
\item[\vref{Jé 2:3}] il était les p. de son revenu ;
\item[\vref{Ro 8:23}] qui avons les p. de l'Esprit, ns.-mêmes,
\item[\vref{Ro 11:16}] Or si les p. sont saintes, la
\item[\vref{Ro 16:5}] pour Christ les p. d'Achaïe.
\item[\vref{1 Co 15:20}] il est les p. de ceux qui
\item[\vref{1 Co 15:23}] rang, Christ com. p., puis ceux qui
\item[\vref{1 Co 16:15}] qu'elle est les p. de l'Achaïe, et
\item[\vref{Ja 1:18}] soyons com. les p. de ses créatures.
\item[\vref{Ap 14:4}] pour être des p. pour Dieu et
\end{listverse}

\ConcordanceEntry{Premier}
\vspace{-2mm}
\begin{listverse}
\item[\vref{Ge 4:4}] côté, offrit des p.-nés de son
\item[\vref{Esd 3:12}] avaient vu la p. maison sur son
\item[\vref{Esd 9:2}] ont été les p. à commettre ce
\item[\vref{Ps 18:45}] Ils m'obéissent au p. ordre, les fils
\item[\vref{Ps 135:8}] a frappé les p.-nés d'Egypte, tant
\item[\vref{Pr 18:17}] qui plaide le p. paraît juste ; mais
\item[\vref{Es 41:4}] JE SUIS le p., et JE SUIS
\item[\vref{Es 43:27}] Ton p. père a péché, et tes docteurs
\item[\vref{Es 44:6}] Je suis le p., et je suis
\item[\vref{Es 48:12}] JE SUIS le p., JE SUIS aussi
\item[\vref{Mt 19:30}] qui sont les p. seront les derniers,
\item[\vref{Mt 20:27}] veut être le p. parmi vs., qu'il
\item[\vref{Mt 23:6}] Ils aiment les p. places ds les
\item[\vref{Mt 28:1}] à l'aube du p. jour de la
\item[\vref{Jn 2:11}] Jésus fit ce p. miracle à Cana
\item[\vref{Jn 5:4}] et alors le p. qui y descendait
\item[\vref{Jn 8:7}] péché jette le p. la pierre contre
\item[\vref{Ac 20:7}] Le p. jour de la semaine, les disciples
\item[\vref{Ro 8:29}] qu'il soit le p.-né de beaucoup
\item[\vref{Ro 11:35}] a donné le p., pour qu'il ait
\item[\vref{1 Co 15:46}] n'est pas le p., mais ce qui
\item[\vref{1 Ti 1:15}] les pécheurs, dont je suis le p.
\item[\vref{1 Ti 2:13}] été formé le p., Eve ensuite.
\item[\vref{Hé 10:9}] abolit ainsi le p. afin d'établir le
\item[\vref{1 Jn 4:19}] parce qu'il ns. a aimés le p.
\item[\vref{Ap 1:5}] témoin fidèle, le p.-né d'entre les
\item[\vref{Ap 1:11}] et l'Oméga, le p. et le dernier.
\item[\vref{Ap 22:13}] et l'Oméga, le p. et le dernier,
\end{listverse}

\ConcordanceEntry{Première alliance}
\vspace{-2mm}
\begin{listverse}
\item[\vref{Lé 26:45}] faveur de la Première Alliance, par laquelle je
\item[\vref{Hé 8:7}] effet, si la Première Alliance avait été irréprochable,
\item[\vref{Hé 9:1}] En vérité, la Première Alliance avait aussi des
\end{listverse}

\ConcordanceEntry{Premièrement}
\vspace{-2mm}
\begin{listverse}
\item[\vref{Za 12:7}] Yahweh sauvera p. les tentes de
\item[\vref{Mt 6:33}] Mais cherchez p. le Royaume de
\item[\vref{Mt 13:30}] aux moissonneurs : Arrachez p. l'ivraie, et liez-la
\item[\vref{Mt 17:10}] scribes disent-ils qu'il faut qu'Elie vienne p. ?
\item[\vref{Mt 23:26}] Pharisien aveugle ! nettoie p. l'intérieur de la
\item[\vref{Lu 6:42}] œil ? Hypocrite ! Ôte p. la poutre de
\item[\vref{Ac 3:26}] C'est à vs. p. que Dieu, ayant
\item[\vref{Ro 1:16}] croient, du Juif p., puis aussi du
\item[\vref{1 Th 4:16}] et les morts en Christ ressusciteront p.
\item[\vref{1 Ti 3:10}] doivent aussi être p. éprouvés, et qu'ensuite
\item[\vref{Hé 7:2}] Son nom signifie p. Roi de justice,
\item[\vref{Ja 3:17}] d'en haut est p. pure, ensuite pacifique,
\item[\vref{1 Pi 4:17}] Or s'il commence p. par ns., quelle
\end{listverse}

\ConcordanceEntry{Prendre}
\vspace{-2mm}
\begin{listverse}
\item[\vref{Ex 6:7}] Et je vs. p. pour être mon
\item[\vref{Ex 20:7}] Tu ne p. point le Nom de Yahweh, ton
\item[\vref{1 R 19:4}] assez, ô Yahweh ! P. mon âme, car
\item[\vref{Mt 20:14}] P. ce qui est à toi, et
\item[\vref{Mt 24:40}] champ ; l'un sera p., et l'autre laissé ;
\item[\vref{Mc 6:8}] de ne rien p. pour le chemin,
\item[\vref{Jn 14:3}] et je vs. p. avec moi, afin
\item[\vref{Jn 16:14}] glorifiera, car il p. ce qui est
\item[\vref{Jn 21:3}] mais ils ne p. rien cette nuit-là.
\item[\vref{Ap 5:9}] es digne de p. le livre, et
\end{listverse}

\ConcordanceEntry{Préparer}
\vspace{-2mm}
\begin{listverse}
\item[\vref{Es 25:6}] Yahweh des armées p. à ts les
\item[\vref{Joë 3:9}] parmi les nations ! P. la guerre ! Réveillez
\item[\vref{Am 4:12}] te traiterai ainsi, p.-toi à la
\item[\vref{Mal 3:1}] mon messager ; il p. le chemin dvt
\item[\vref{Ep 2:10}] que Dieu a p. d'avance, afin que
\item[\vref{Hé 11:16}] qu'il lr. a p. une cité.
\item[\vref{Ap 8:6}] sept trompettes se p. à en sonner.
\item[\vref{Ap 12:6}] avait un lieu p. par Dieu, afin
\item[\vref{Ap 19:7}] sont venues, et son Epouse s'est p.
\end{listverse}

\ConcordanceEntry{Près}
\vspace{-2mm}
\begin{listverse}
\item[\vref{Ge 12:11}] com. il était p. d'entrer en Egypte,
\item[\vref{No 24:17}] non pas de p. ; une Etoile est
\item[\vref{De 30:14}] parole est fort p. de toi, ds
\item[\vref{Ps 22:12}] la détresse est p. de moi, et
\item[\vref{Ps 23:2}] il me dirige p. des eaux paisibles.
\item[\vref{Ps 145:18}] Qof.] Yahweh est p. de ts ceux
\item[\vref{Es 55:6}] se trouve, invoquez-le tandis qu'il est p.
\item[\vref{Ro 13:11}] salut est plus p. de ns. que
\end{listverse}

\ConcordanceEntry{Prescience}
\vspace{-2mm}
\begin{listverse}
\item[\vref{Ac 2:23}] et selon la p. de Dieu, vs.
\item[\vref{1 Pi 1:2}] élus selon la p. de Dieu le
\end{listverse}

\ConcordanceEntry{Présent}
\vspace{-2mm}
\begin{listverse}
\item[\vref{Ge 43:11}] en porter un p. à cet hom.,
\item[\vref{Ex 23:8}] prendras point de p. ; car le présent
\item[\vref{De 10:17}] et qui ne prend point de p. ;
\item[\vref{De 16:19}] recevras point de p., car les présents
\item[\vref{Jos 15:19}] répondit : Donne-moi un p., puisque tu m'as
\item[\vref{Jg 3:18}] achevé d'offrir le p., il renvoya le
\item[\vref{1 S 8:3}] ils recevaient des p. et violaient la
\item[\vref{Ps 15:5}] n'accepte point de p. contre l'innocent. Celui
\item[\vref{Ps 26:10}] la main droite est pleine de p.
\item[\vref{Ps 45:13}] des peuples te supplieront avec des p.
\item[\vref{Ps 72:10}] et de Séba lui apporteront des p.
\item[\vref{Pr 18:16}] Le p. d'un hom. lui fait faire place,
\item[\vref{Es 5:23}] méchant pour des p., et qui ôtent
\item[\vref{Es 33:15}] pas accepter un p. ; celui qui bouche
\item[\vref{Es 39:1}] lettres avec un p. à Ezéchias, parce
\item[\vref{Es 45:13}] sans rançon ni p., dit Yahweh des
\item[\vref{Da 5:17}] et donne tes p. à un autre ;
\item[\vref{Mi 3:11}] jugent pour des p., ses prêtres enseignent
\item[\vref{Ph 4:17}] je recherche des p., mais je cherche
\item[\vref{Ap 11:10}] ils s'enverront des p. les uns aux
\end{listverse}

\ConcordanceEntry{Présenter}
\vspace{-2mm}
\begin{listverse}
\item[\vref{Ex 23:15}] nul ne se p. dvt ma face
\item[\vref{1 S 10:19}] établis-ns. un roi ! P.-vs. dc mntnt,
\item[\vref{2 S 7:18}] David alla se p. dvt Yahweh, et
\item[\vref{Job 1:6}] Dieu vinrent se p. dvt Yahweh, et
\item[\vref{Job 2:1}] vinrent pour se p. dvt Yahweh, Satan
\item[\vref{Ps 33:9}] il ordonne, et la chose se p.
\item[\vref{Ps 106:30}] Mais Phinées se p., et fit justice ;
\item[\vref{Mi 6:6}] Avec quoi me p.-je dvt Yahweh,
\item[\vref{Mal 2:12}] retranchera celui qui p. une offrande à
\item[\vref{Mt 5:39}] ta joue droite, p.-lui aussi l'autre.
\item[\vref{Mt 28:9}] voici, Jésus se p. dvt elles, et
\item[\vref{Mc 13:9}] et vs. serez p. dvt les gouverneurs
\item[\vref{Lu 1:80}] où il se p. à Israël.
\item[\vref{Lu 2:22}] Jérus., pour le p. au Seign.,
\item[\vref{Lu 24:36}] discours, Jésus se p. lui-mm au milieu
\item[\vref{Ac 2:14}] Alors Pierre, se p. avec les onze,
\item[\vref{Ac 6:6}] Ils les p. aux apôtres ; qui, après avoir prié,
\item[\vref{1 Co 3:22}] soit les choses p., soit les choses
\item[\vref{2 Co 11:2}] époux, pour vs. p. à Christ com.
\item[\vref{Ph 4:6}] en ttes choses p. vos demandes à
\item[\vref{Col 1:22}] mort, pour vs. p. saints, et sans
\item[\vref{Col 1:28}] que ns. puissions p. tt hom. parfait
\item[\vref{Hé 11:1}] la foi rend p. les choses qu'on
\end{listverse}

\ConcordanceEntry{Préserver}
\vspace{-2mm}
\begin{listverse}
\item[\vref{Ge 19:19}] mon égard en p. ma vie, mais
\item[\vref{Ex 12:27}] l'Egypte, et qu'il p. nos maisons. Alors
\item[\vref{De 6:24}] heureux, et qu'il p. notre vie, com.
\item[\vref{1 S 25:39}] et qui a p. son serviteur de
\item[\vref{Esd 9:9}] ns. accorder de p. nos vies afin
\item[\vref{Job 5:20}] guerre il te p. de l'épée.
\item[\vref{Ps 12:8}] Yahweh ! garde-les, et p. à jamais chacun
\item[\vref{Ps 140:4}] main du méchant ! P.-moi de l'hom.
\item[\vref{Pr 7:5}] Afin qu'elles te p. de la fem.
\item[\vref{Ez 12:16}] eux, quelques hommes, p. de l'épée, de
\item[\vref{Mi 2:3}] ne pourrez point p. votre cou, et
\item[\vref{Jn 17:15}] ôter du monde, mais de les p. du mal.
\item[\vref{1 Pi 3:10}] jours heureux, qu'il p. sa langue du
\end{listverse}

\ConcordanceEntry{Présider}
\vspace{-2mm}
\begin{listverse}
\item[\vref{Ge 1:16}] grand luminaire pour p. au jour, et
\item[\vref{1 Ch 27:2}] fils de Zabdiel, p. sur la première
\item[\vref{Ro 12:8}] que celui qui p. le fasse soigneusement ;
\end{listverse}

\ConcordanceEntry{Presser}
\vspace{-2mm}
\begin{listverse}
\item[\vref{Ge 19:15}] jour, les anges p. Lot, en disant :
\item[\vref{Ex 5:13}] les oppresseurs les p., en disant : Achevez
\item[\vref{Ex 12:33}] Et les Egyptiens p. le peuple et
\item[\vref{Ps 119:157}] et qui me p. sont en grand
\item[\vref{Pr 30:33}] com. celui qui p. le nez en
\item[\vref{Joë 2:8}] Ils ne se p. point les uns
\item[\vref{Mc 3:9}] ne pas être p. par la foule.
\item[\vref{Mc 5:24}] de gens le suivaient et le p.
\item[\vref{Ac 16:15}] Et elle ns. p. par ses instances.
\item[\vref{2 Co 8:4}] ns. p. avec de grandes prières de recevoir
\item[\vref{Ph 1:23}] Car je suis p. des deux côtés :
\end{listverse}

\ConcordanceEntry{Pressoir}
\vspace{-2mm}
\begin{listverse}
\item[\vref{Jg 6:11}] du froment au p. pour le mettre
\item[\vref{Jg 7:25}] tuèrent Zeeb au p. de Zeeb. Ils
\item[\vref{Né 13:15}] Juda fouler aux p. le jour du
\item[\vref{Es 63:3}] à fouler au p., et nul hom.
\item[\vref{La 1:15}] a foulé au p. la vierge, fille
\item[\vref{Joë 3:13}] descendez, car le p. est plein, les
\item[\vref{Ag 2:16}] on  venait au p., au lieu de
\item[\vref{Mt 21:33}] y creusa un p., et bâtit une
\end{listverse}

\ConcordanceEntry{Prêt, Prête}
\vspace{-2mm}
\begin{listverse}
\item[\vref{Est 3:14}] à se tenir p. pour ce jour-là.
\item[\vref{Ez 36:8}] mon peuple d'Israël ; car ils sont p. à venir.
\item[\vref{Mi 7:3}] leurs mains sont p. : Le gouverneur exige,
\item[\vref{Za 5:11}] qnd elle sera p., il sera déposé
\item[\vref{Mt 24:44}] vs. aussi tenez-vs. p. ; car le Fils
\item[\vref{Mt 25:10}] Celles qui étaient p. entrèrent avec lui
\item[\vref{Mc 4:29}] faucille, parce que la moisson est p.
\item[\vref{Lu 14:17}] conviés : Venez, car tt est déjà p.
\item[\vref{Lu 22:33}] Seign., je suis p. à aller avec
\item[\vref{Ro 1:15}] moi, je suis p. à vs. annoncer
\item[\vref{2 Co 9:2}] que l'Achaïe est p. depuis l'année dernière ;
\item[\vref{2 Co 9:3}] que vs. soyez p., com. j'ai dit.
\item[\vref{2 Co 10:6}] Et étant p. à tirer vengeance de tte désobéissance,
\item[\vref{Ep 6:15}] vos pieds chaussés, p. pour l'Evangile de
\item[\vref{Tit 3:1}] aux gouverneurs, d'être p. à faire ttes
\item[\vref{1 Pi 1:5}] salut, qui est p. à être révélé
\item[\vref{1 Pi 3:15}] et soyez toujours p. à répondre avec
\item[\vref{1 Pi 4:5}] celui qui est p. à juger les
\item[\vref{Ap 9:15}] anges qui étaient p. pour l'heure, le
\end{listverse}

\ConcordanceEntry{Prêter}
\vspace{-2mm}
\begin{listverse}
\item[\vref{Ex 22:25}] Si tu p. de l'argent à mon peuple, au
\item[\vref{De 15:6}] l'a promis, tu p. sur gage à
\item[\vref{1 S 1:28}] pourquoi je le p. à Yahweh : Il
\item[\vref{1 S 16:7}] à Samuel : Ne p. pas attention à
\item[\vref{Ps 37:26}] temps, et il p. ; et sa postérité
\item[\vref{Ps 49:5}] Je p. l'oreille aux sentences qui me sont
\item[\vref{Ps 140:6}] mon Dieu, Yahweh, p. l'oreille à la
\item[\vref{Pr 19:17}] pitié du pauvre p. à Yahweh, qui
\item[\vref{Pr 22:7}] emprunte est l'esclave de celui qui p.
\item[\vref{Es 51:4}] mon peuple, et p.-moi l'oreille, vs.
\item[\vref{Ez 18:13}] s'il p. à intérêt, et tire une usure ;
\item[\vref{Lu 6:35}] du bien, et p. sans rien espérer,
\item[\vref{Lu 11:5}] dire : Mon ami, p.-moi trois pains,
\end{listverse}

\ConcordanceEntry{Prétoire}
\vspace{-2mm}
\begin{listverse}
\item[\vref{Mt 27:27}] Jésus ds le p. et assemblèrent dvt
\item[\vref{Jn 18:28}] chez Caïphe au p. ; et c'était le
\item[\vref{Jn 19:9}] rentra ds le p., et dit à
\item[\vref{Ac 23:35}] gardé ds le p. d'Hérode.
\item[\vref{Ph 1:13}] ds tt le p., et partout ailleurs.
\end{listverse}

\ConcordanceEntry{Prêtre}
\vspace{-2mm}
\begin{listverse}
\item[\vref{Ge 14:18}] or il était p. du Dieu Très-Haut.
\item[\vref{Ge 47:22}] les terres des p., parce qu'il y
\item[\vref{Jos 22:31}] Phinées, fils du p. Eléazar, dit aux
\item[\vref{Jg 17:13}] parce que j'ai un Lévite pour p.
\item[\vref{1 S 2:35}] je m'établirai un p. fidèle, qui agira
\item[\vref{1 R 13:33}] de nouveau des p. des hauts lieux
\item[\vref{2 R 17:32}] ils établirent des p. des hauts lieux
\item[\vref{Esd 2:63}] en attendant qu'un p. ait consulté l'urim
\item[\vref{Esd 6:20}] Car les p. et les Lévites s'étaient purifiés com.
\item[\vref{Ps 99:6}] étaient parmi ses p., et Samuel parmi
\item[\vref{Ps 110:4}] que tu es p. éternellement, à la
\item[\vref{Ps 132:9}] Que tes p. soient revêtus de justice, et que
\item[\vref{Ps 132:16}] de salut ses p., et ses bien-aimés
\item[\vref{Mi 3:11}] des présents, ses p. enseignent pour un
\item[\vref{Mal 2:7}] les lèvres du p. doivent garder la
\item[\vref{Mt 12:5}] du sabbat, les p. violent le sabbat
\item[\vref{Mc 14:53}] Jésus chez le grand-p., où s'assemblèrent ts
\item[\vref{Lu 10:31}] Un p., qui par hasard descendait par le
\item[\vref{Lu 17:14}] Allez, montrez-vs. aux p.. Et, pendant qu'ils
\item[\vref{Ac 5:27}] sanhédrin. Et le grand-p. les interrogea, disant :
\item[\vref{Ac 14:13}] Le p. de Jupiter, qui était à l'entrée
\item[\vref{Ac 23:4}] Tu insultes le grand-p. de Dieu ?
\item[\vref{Hé 2:17}] qu'il soit un Grand-P. miséricordieux et fidèle
\item[\vref{Hé 3:1}] l'Apôtre et le Grand-P. que ns. confessons,
\item[\vref{Hé 4:14}] avons un Souverain Grand-P., Jésus, le Fils
\item[\vref{Hé 4:15}] n'avons pas un Grand-P. qui ne puisse
\item[\vref{Hé 5:5}] lui-mm d'être fait Grand-P., mais il a
\item[\vref{Hé 7:1}] de Salem et P. du Dieu Très-Haut.
\item[\vref{Hé 8:6}] Mais mntnt, notre Grand-P. a obtenu un
\item[\vref{Hé 10:21}] Et ayant un Grand-P. établi sur la
\item[\vref{Hé 13:11}] sanctuaire par le grand-p. pour le péché,
\item[\vref{Ap 1:6}] rois et des p. pour Dieu, son
\item[\vref{Ap 5:10}] rois et des p. pour notre Dieu ;
\item[\vref{Ap 20:6}] mais ils seront p. de Dieu, et
\end{listverse}

\ConcordanceEntry{Prêtrise}
\vspace{-2mm}
\begin{listverse}
\item[\vref{Ex 29:9}] ils posséderont la p. par ordon. perpétuelle.
\item[\vref{No 3:3}] oints et consacrés pour exercer la p.
\item[\vref{No 3:10}] ils exerceront lr. p.. Que si qq
\item[\vref{No 16:10}] Lévi, et vs. recherchez encore la p. !
\item[\vref{No 25:13}] Et l'alliance de p. perpétuelle sera tant
\item[\vref{De 10:6}] fils, exerça la p. à sa place.
\item[\vref{Jos 18:7}] parce que la p. de Yahweh est
\item[\vref{1 S 2:36}] charges de la p. pour manger un
\item[\vref{Esd 2:62}] rejetés pour ne pas souiller la p.,
\item[\vref{Né 13:29}] ont souillé la p. et l'alliance contractée
\item[\vref{Lu 1:8}] Zacharie exerçait la p. dvt Dieu, selon
\item[\vref{Hé 7:12}] Or la p. étant changée, il est nécessaire qu'il
\item[\vref{Hé 7:14}] Moïse n'a rien dit de la p.
\item[\vref{Hé 7:24}] éternellement, possède une p. qui n'est pas
\item[\vref{1 Pi 2:5}] et une sainte p., afin d'offrir des
\item[\vref{1 Pi 2:9}] vs. êtes la p. royale, la nation
\end{listverse}

\ConcordanceEntry{Prier}
\vspace{-2mm}
\begin{listverse}
\item[\vref{Ge 20:7}] prophète ; et il p. pour toi et
\item[\vref{No 21:7}] serpents et Moïse p. pour le peuple.
\item[\vref{1 S 1:27}] J'ai p. pour avoir cet enfant, et Yahweh
\item[\vref{1 S 2:1}] Alors Anne p., et dit : Mon
\item[\vref{1 S 12:23}] de cesser de p. pour vs. ! Je
\item[\vref{Job 42:8}] Job, mon serviteur, p. pour vs., et
\item[\vref{Ps 32:6}] tt fidèle te p. au temps où
\item[\vref{Jé 29:12}] partirez ; vs. me p., et je vs.
\item[\vref{Da 9:20}] parlais encore, je p., je confessais mon
\item[\vref{Mt 6:6}] toi, qnd tu p., entre ds ta
\item[\vref{Mt 14:23}] part, afin de p. ; et le soir
\item[\vref{Mt 26:36}] ce que j'aie p. ds le lieu
\item[\vref{Mc 1:35}] ds un lieu désert, où il p.
\item[\vref{Mc 5:23}] et le p. instamment, en disant : Ma petite fille
\item[\vref{Mc 13:33}] tt, veillez et p. ; car vs. ne
\item[\vref{Mc 14:32}] Asseyez-vs. ici jusqu'à ce que j'aie p.
\item[\vref{Lu 3:21}] et pendant qu'il p., le ciel s'ouvrit,
\item[\vref{Lu 5:16}] tenait retiré ds les déserts, et p.
\item[\vref{Lu 6:12}] une montagne pour p., et qu'il passa
\item[\vref{Lu 18:1}] qu'il faut toujours p., et ne pas
\item[\vref{Lu 21:36}] Veillez dc, et p. en tt temps,
\item[\vref{Lu 22:32}] mais j'ai p. pour toi afin
\item[\vref{Lu 22:40}] il lr. dit : P. afin que vs.
\item[\vref{Lu 22:44}] en agonie, il p. plus instamment, et
\item[\vref{Jn 17:9}] Je p. pour eux. Je ne prie pas
\item[\vref{Jn 17:20}] Or je ne p. pas seulement pour eux, mais aussi
\item[\vref{Ac 4:31}] qnd ils eurent p., le lieu où
\item[\vref{1 Co 14:14}] Car si je p. ds une langue
\item[\vref{1 Co 14:15}] faire dc ? Je p. par l'esprit, mais
\item[\vref{Ep 6:18}] P. en tt temps ds l'Esprit par
\item[\vref{1 Th 5:17}] P. sans cesse.
\item[\vref{1 Ti 2:8}] que les hommes p. en tt lieu,
\item[\vref{Ja 5:17}] et cependant il p. avec instance pour
\item[\vref{1 Pi 4:7}] Soyez dc sobres et vigilants pour p.
\end{listverse}

\ConcordanceEntry{Prière}
\vspace{-2mm}
\begin{listverse}
\item[\vref{1 R 8:28}] attentif à la p. que t'adresse ton
\item[\vref{1 R 9:3}] dit : J'exauce ta p., et la supplication
\item[\vref{2 Ch 30:27}] fut entendue, lr. p. parvint jusqu'aux cieux,
\item[\vref{Job 16:17}] et que ma p. soit pure.
\item[\vref{Job 42:9}] avait commandé ; et Yahweh exauça la p. de Job.
\item[\vref{Ps 65:3}] Tu entends nos p.. Toute chair viendra
\item[\vref{Ps 72:20}] Fin des p. de David, fils d'Isaï.
\item[\vref{Ps 90:1}] P. de Moïse, hom. de Dieu. Seign. !
\item[\vref{Ps 102:1}] P. de l'affligé étant ds l'angoisse et
\item[\vref{Ps 102:18}] égard à la p. du désolé, et
\item[\vref{Es 56:7}] ma maison de p. ; leurs holocaustes et
\item[\vref{Da 6:13}] il fait sa p. trois fois par
\item[\vref{Mt 17:21}] que par la p. et par le
\item[\vref{Mt 23:14}] faire de longues p., c'est pourquoi vs.
\item[\vref{Lu 1:10}] était dehors en p., à l'heure du
\item[\vref{Lu 1:13}] pas, car ta p. est exaucée, et
\item[\vref{Ac 6:4}] vaquer à la p. et au service
\item[\vref{Ac 12:5}] sans cesse des p. à Dieu pour
\item[\vref{Ac 16:13}] de faire la p., près du fleuve.
\item[\vref{Ro 8:26}] demander ds nos p.. Mais l'Esprit lui-mm
\item[\vref{Ep 6:18}] ttes sortes de p. et de supplications,
\item[\vref{Ph 4:6}] Dieu par des p. et des supplications,
\item[\vref{Col 4:12}] vs. ds ses p., afin que vs.
\item[\vref{1 Ti 2:1}] des requêtes, des p., des supplications, et
\item[\vref{Hé 5:7}] avec larmes des p. et des supplications
\item[\vref{Ja 5:16}] guéris. Car la p. du juste faite
\item[\vref{1 Pi 3:7}] afin que vos p. ne soient pas
\item[\vref{Ap 5:8}] qui sont les p. des saints.
\end{listverse}

\ConcordanceEntry{Prince}
\vspace{-2mm}
\begin{listverse}
\item[\vref{Ge 23:6}] Tu es un p. de Dieu parmi
\item[\vref{1 S 29:6}] mais tu ne plais pas aux p.
\item[\vref{1 R 11:34}] je le maintiendrai p., pour l'amour de
\item[\vref{Ps 2:2}] personne, et les p. se liguent-ils avec
\item[\vref{Ps 45:17}] les établiras pour p. sur tte la
\item[\vref{Ps 76:13}] le souffle des p. ; il est redoutable
\item[\vref{Ps 148:11}] ts les peuples, p. et ts les
\item[\vref{Pr 8:15}] par moi les p. décrètent ce qui
\item[\vref{Ec 10:16}] et dont les p. mangent dès le
\item[\vref{Ca 7:2}] Jérus. :] Fille de p. combien sont belles
\item[\vref{Es 9:5}] Dieu Puissant, le Père d'éternité, le P. de paix,
\item[\vref{Ez 28:2}] l'hom., dis au p. de Tyr : Ainsi
\item[\vref{Ez 34:24}] serviteur David sera p. au milieu d'elles ;
\item[\vref{Ez 45:7}] Pour le p. vs. réserverez un espace aux deux
\item[\vref{Ez 45:17}] Mais le p. sera tenu de fournir les holocaustes,
\item[\vref{Os 13:10}] dit : Donne-moi un roi et des p. ?
\item[\vref{Na 3:17}] Tes p. et leurs scribes sont com. des
\item[\vref{Ha 1:10}] rois, et les p. sont l'objet de
\item[\vref{Mt 9:34}] démons par le p. des démons.
\item[\vref{Jn 12:31}] monde ; mntnt le p. de ce monde
\item[\vref{Ac 3:15}] fait mourir le P. de la vie,
\item[\vref{Ac 5:31}] puissance pour être P. et Sauveur, afin
\item[\vref{Ac 7:27}] Qui t'a établi p. et juge sur
\item[\vref{Ep 2:2}] monde, selon le p. de la puissance
\item[\vref{1 Ti 6:15}] Béni et seul P., le Roi des
\item[\vref{Hé 2:10}] gloire, consacre le P. de lr. salut
\item[\vref{Ap 1:5}] morts, et le P. des rois de
\end{listverse}

\ConcordanceEntry{Principauté}
\vspace{-2mm}
\begin{listverse}
\item[\vref{Ro 8:38}] anges, ni les p., ni les puissances,
\item[\vref{Ep 1:21}] au-dessus de tte p., de tte puissance,
\item[\vref{Ep 3:10}] afin que les p. et les puissances
\item[\vref{Ep 6:12}] mais contre les p., contre les puissances,
\item[\vref{Col 1:16}] dominations, ou les p., ou les puissances,
\item[\vref{Col 2:10}] Chef de tte p. et puissance.
\item[\vref{Col 2:15}] a dépouillé les p. et les puissances,
\end{listverse}

\ConcordanceEntry{Principe}
\vspace{-2mm}
\begin{listverse}
\item[\vref{Hé 6:1}] que les premiers p. de Christ, tendons
\item[\vref{1 Pi 5:2}] mais par un p. d'affection.
\end{listverse}

\ConcordanceEntry{Priscille, Prisca}
\vspace{-2mm}
\begin{listverse}
\item[\vref{Ro 16:3}] Saluez P. et Aquilas, mes compagnons d'œuvre en
\item[\vref{2 Ti 4:19}] Salue P. et Aquilas, et la famille d'Onésiphore.
\end{listverse}

\ConcordanceEntry{Prison}
\vspace{-2mm}
\begin{listverse}
\item[\vref{Ge 39:20}] ds une étroite p. ; ds l'endroit où
\item[\vref{Ge 41:14}] hâte de la p. ; on le rasa,
\item[\vref{Es 42:7}] ds les ténèbres hors de la p.
\item[\vref{Es 42:22}] cachés ds des p. ; ils sont un
\item[\vref{Jé 20:2}] mit ds la p. qui était à
\item[\vref{Mt 4:12}] été mis en p., se retira ds
\item[\vref{Mt 5:25}] que tu ne sois mis en p.
\item[\vref{Mt 25:36}] visité ; j'étais en p., et vs. êtes
\item[\vref{Lu 12:58}] que celui-ci ne te mette en p.
\item[\vref{Lu 21:12}] vs. mettant en p. ; et ils vs.
\item[\vref{Ac 5:18}] jetèrent ds la p. publique.
\item[\vref{Ac 8:3}] et femmes, il les mettait en p.
\item[\vref{Ac 12:4}] et jeté en p., il le mit
\item[\vref{Ac 16:23}] les mirent en p., en recommandant au
\item[\vref{Ac 24:27}] plaisir aux Juifs, laissa Paul en p.
\item[\vref{Ac 26:10}] J'ai mis en p. plusieurs des saints,
\item[\vref{2 Co 6:5}] coups, ds les p., ds les troubles,
\item[\vref{Hé 11:36}] le fouet, les chaînes et la p. ;
\item[\vref{1 Pi 3:19}] prêcher aux esprits qui sont en p.,
\item[\vref{Ap 2:10}] d'entre vs. en p., afin que vs.
\item[\vref{Ap 20:7}] accomplis, Satan sera délié de sa p.
\end{listverse}

\ConcordanceEntry{Prisonnier}
\vspace{-2mm}
\begin{listverse}
\item[\vref{Ge 14:14}] avait été emmené p., il arma trois
\item[\vref{Ge 39:22}] Joseph ts les p. qui étaient ds
\item[\vref{No 31:9}] fils d'Israël emmenèrent p. les femmes de
\item[\vref{2 Ch 28:17}] battu Juda et avaient emmené des p.
\item[\vref{Est 2:6}] Jérus., avec les p. qui avaient été
\item[\vref{Ps 85:2}] en repos les p. de Jacob.
\item[\vref{Es 61:1}] liberté, et aux p. l'ouverture de la
\item[\vref{Mi 1:16}] ils sont emmenés p. loin de toi.
\item[\vref{Mt 27:15}] de relâcher un p. à chaque fête,
\item[\vref{Ac 16:25}] Dieu, et les p. les entendaient.
\item[\vref{Ac 28:16}] centenier mit les p. entre les mains
\item[\vref{Ro 7:23}] qui me rend p. à la loi
\item[\vref{Ro 16:7}] qui ont été p. avec moi, et
\item[\vref{Ep 3:1}] Paul, je suis p. de Jésus-Christ pour
\item[\vref{2 Ti 1:8}] qui suis son p. ; mais souffre avec
\item[\vref{Phm 1:9}] et mm mntnt p. de Jésus-Christ ;
\item[\vref{Hé 13:3}] Souvenez-vs. des p., com. si vs.
\end{listverse}

\ConcordanceEntry{Priver}
\vspace{-2mm}
\begin{listverse}
\item[\vref{Ge 27:45}] fait. Pourquoi serais-je p. de vs. deux
\item[\vref{Lé 20:20}] et ils mourront p. d'enfants.
\item[\vref{Ru 1:5}] fem. resta seule, p. de ses deux
\item[\vref{Job 39:20}] Car Dieu l'a p. de sagesse et
\item[\vref{Es 38:10}] scheol, je suis p. de ce qui
\item[\vref{Es 47:8}] c'est que d'être p. d'enfants.
\item[\vref{Jé 33:10}] qui sont désolées, p. d'hommes, d'habitants, de
\item[\vref{Ez 14:15}] et qu'elles le p. d'enfants, tellement qu'il
\item[\vref{Os 9:12}] je les en p. tellement, que pas
\item[\vref{Am 4:7}] vs. ai aussi p. de pluie, qnd
\item[\vref{1 Co 7:5}] Ne vs. p. pas l'un de l'autre, si ce
\item[\vref{Ep 2:12}] temps-là sans Christ, p. du droit de
\item[\vref{Ph 3:8}] je me suis p. de ttes ces
\item[\vref{1 Ti 6:5}] corrompus d'entendement et p. de la vérité,
\item[\vref{Hé 4:1}] ds son repos, ne s'en trouve p.
\item[\vref{Hé 12:15}] personne ne se p. de la grâce
\item[\vref{Ap 8:12}] le jour fut p. d'un tiers de
\end{listverse}

\ConcordanceEntry{Prix}
\vspace{-2mm}
\begin{listverse}
\item[\vref{1 S 26:24}] aujourd'hui de grand p. à mes yeux,
\item[\vref{2 S 24:24}] pour un certain p., et je n'offrirai
\item[\vref{Ps 49:8}] à Dieu le p. du rachat.
\item[\vref{Pr 7:23}] que c'est au p. de sa vie.
\item[\vref{Pr 31:10}] vertueuse ? Car son p. surpasse de beaucoup
\item[\vref{Jon 1:3}] il paya le p. du transport et
\item[\vref{Za 11:13}] au potier, ce p. honorable auquel ils
\item[\vref{Mt 13:46}] perle de grand p. et il est
\item[\vref{Mt 27:6}] ds le trésor, car c'est le p. du sang.
\item[\vref{Ac 4:34}] ils apportaient le p. des choses vendues,
\item[\vref{Ac 5:2}] une partie du p., sa fem. le
\item[\vref{Ac 7:16}] avait acheté à p. d'argent des fils
\item[\vref{1 Co 6:20}] à un grand p.. Glorifiez dc Dieu
\item[\vref{1 Co 7:23}] à un grand p., ne devenez pas
\item[\vref{1 Co 9:24}] seul remporte le p. ? Courez de manière
\item[\vref{Ph 3:14}] pour remporter le p. de la vocation
\item[\vref{Col 2:18}] son gré le p. de la course,
\item[\vref{1 Pi 3:4}] est d'un grand p. dvt Dieu.
\end{listverse}

\ConcordanceEntry{Procès}
\vspace{-2mm}
\begin{listverse}
\item[\vref{Ez 44:24}] il surviendra qq p., ils assisteront au
\item[\vref{Os 4:1}] Yahweh a un p. avec les habitants
\item[\vref{Os 12:3}] a aussi un p. avec Juda, et
\item[\vref{Mi 6:2}] Ecoutez, montagnes, le p. de Yahweh, vs.
\item[\vref{Ha 1:3}] qui excitent des p. et des querelles ?
\item[\vref{Ac 12:19}] avoir fait le p. aux gardes, il
\item[\vref{1 Co 6:4}] vs. avez des p. pour les affaires
\end{listverse}

\ConcordanceEntry{Prochain}
\vspace{-2mm}
\begin{listverse}
\item[\vref{Ex 2:13}] qui avait tort : Pourquoi frappes-tu ton p. ?
\item[\vref{Lé 19:18}] tu aimeras ton p. com. toi-mm. Je
\item[\vref{Jos 20:5}] a tué son p., et qu'il ne
\item[\vref{1 R 8:31}] pèche contre son p. et qu'on lui
\item[\vref{Ps 28:3}] paix avec lr. p. pendant que la
\item[\vref{Pr 3:28}] pas à ton p. : Va, et reviens,
\item[\vref{Pr 6:29}] fem. de son p. ; quiconque la touchera
\item[\vref{Pr 14:21}] qui méprise son p. commet un péché,
\item[\vref{Pr 24:28}] pas contre ton p. sans cause ; car
\item[\vref{Pr 25:17}] maison de ton p., de peur qu'étant
\item[\vref{Jé 22:13}] fait travailler son p. pour rien, sans
\item[\vref{Jé 31:34}] n'enseignera plus son p., ni personne son
\item[\vref{Ha 1:13}] méchant dévore son p. qui est plus
\item[\vref{Za 3:10}] vs. appellera son p. sous la vigne
\item[\vref{Mc 12:33}] et d'aimer son p. com. soi-mm, c'est
\item[\vref{Lu 10:29}] à Jésus : Et qui est mon p. ?
\item[\vref{Ro 15:2}] plaise à son p. pour son bien,
\item[\vref{Hé 8:11}] n'enseignera plus son p., ni personne son
\item[\vref{1 Jn 4:7}] quiconque aime son p. est né de
\end{listverse}

\ConcordanceEntry{Proche}
\vspace{-2mm}
\begin{listverse}
\item[\vref{Ps 85:10}] sa délivrance est p. de ceux qui
\item[\vref{Es 13:6}] de Yahweh est p., il vient com.
\item[\vref{Es 51:5}] Ma justice est p., mon salut va
\item[\vref{Ez 30:3}] le jour est p., oui le jour
\item[\vref{Joë 1:15}] de Yahweh est p. : Il vient com.
\item[\vref{Ab 1:15}] de Yahweh est p. sur ttes les
\item[\vref{So 1:7}] de Yahweh est p. ; Yahweh a préparé
\item[\vref{Mt 4:17}] car le Royaume des cieux est p.
\item[\vref{Mt 24:33}] de l'hom. est p., à la porte.
\item[\vref{Mt 26:18}] Mon temps est p. ; je ferai la
\item[\vref{Ph 4:5}] ts les hommes. Le Seign. est p.
\item[\vref{Ja 5:8}] cœurs, car l'avènement du Seign. est p.
\item[\vref{1 Pi 4:7}] ttes choses est p. : Soyez dc sobres
\item[\vref{Ap 1:3}] sont écrites ! Car le temps est p.
\item[\vref{Ap 22:10}] ce livre. Car le temps est p.
\end{listverse}

\ConcordanceEntry{Proclamer}
\vspace{-2mm}
\begin{listverse}
\item[\vref{De 15:2}] qnd on aura p. le relâche, en
\item[\vref{Né 6:7}] prophètes pour te p. à Jérus. et
\item[\vref{Ps 145:11}] règne, et ils p. ta puissance
\item[\vref{Pr 8:7}] que ma bouche p. la vérité, et
\item[\vref{Es 61:1}] cœur brisé, pour p. aux captifs la
\item[\vref{Es 62:11}] ce que Yahweh p. aux extrémités de
\item[\vref{Jé 34:8}] de Jérus., pour p. la liberté,
\item[\vref{Jon 3:2}] grande ville, et p.-y à haute
\item[\vref{Mi 3:5}] mon peuple, qui p. la paix qnd
\item[\vref{Lu 4:19}] pour p. aux captifs la délivrance, et aux
\item[\vref{Ap 5:2}] sa force, qui p. d'une voix forte :
\end{listverse}

\ConcordanceEntry{Prodige}
\vspace{-2mm}
\begin{listverse}
\item[\vref{No 14:22}] gloire, et les p. que j'ai faits
\item[\vref{Jg 6:13}] sont ts ces p. que nos pères
\item[\vref{Ps 105:27}] milieu d'eux des p. et des miracles
\item[\vref{Ps 135:9}] a envoyé des p. et des miracles
\item[\vref{Jé 32:20}] miracles et des p. qui sont connus
\item[\vref{Joë 2:30}] je ferai des p. ds les cieux
\item[\vref{Mt 24:24}] feront de grands p. et des miracles,
\item[\vref{Jn 4:48}] voyez pas des p. et des miracles,
\item[\vref{Ac 2:43}] miracles et de p. se faisaient par
\item[\vref{Ac 19:11}] Dieu faisait des p. extraordinaires par les
\item[\vref{2 Co 12:12}] des signes, des p. et des miracles.
\item[\vref{2 Th 2:9}] signes, et de p. mensongers,
\item[\vref{Ap 13:13}] opérait de grands p., mm jusqu'à faire
\item[\vref{Ap 16:14}] qui font des p., et qui vont
\end{listverse}

\ConcordanceEntry{Produit}
\vspace{-2mm}
\begin{listverse}
\item[\vref{Lé 25:3}] vigne ; et tu en recueilleras le p.
\item[\vref{De 11:17}] donnerait plus son p., et vs. péririez
\item[\vref{Ha 3:17}] ce que l'olivier p. mentira, et aucun
\item[\vref{Ag 1:11}] que la terre p., et sur les
\item[\vref{Mc 4:28}] Car la terre p. d'elle-mm, premièrement l'herbe,
\item[\vref{Ro 4:15}] car la loi p. la colère, et
\item[\vref{Ro 5:3}] sachant que l'affliction p. la persévérance,
\item[\vref{2 Co 7:10}] est selon Dieu, p. une repentance à
\item[\vref{Ga 3:5}] l'Esprit, et qui p. en vs. les
\item[\vref{Ph 2:13}] c'est Dieu qui p. en vs. avec
\item[\vref{Hé 12:11}] mais ensuite il p. un fruit paisible
\item[\vref{Ja 1:15}] péché ; et le péché, étant consommé, p. la mort.
\end{listverse}

\ConcordanceEntry{Profane}
\vspace{-2mm}
\begin{listverse}
\item[\vref{Lé 10:10}] ce qui est p., entre ce qui
\item[\vref{Ez 22:26}] chose sainte et p. ; ils ne donnent
\item[\vref{Ez 42:20}] entre le lieu saint et le p.
\item[\vref{1 Ti 4:7}] rejette les fables p., et semblables aux
\item[\vref{2 Ti 2:16}] discours vains et p. ; car ceux qui
\item[\vref{Hé 10:29}] pour une chose p. le sang de
\item[\vref{Hé 12:16}] ni fornicateur, ni p. com. Esaü, qui
\end{listverse}

\ConcordanceEntry{Profaner}
\vspace{-2mm}
\begin{listverse}
\item[\vref{Lé 18:21}] et tu ne p. point le Nom
\item[\vref{Ps 79:1}] héritage ; on a p. ton saint temple,
\item[\vref{Es 24:5}] Le pays était p. par ses habitants
\item[\vref{Es 47:6}] mon peuple, j'ai p. mon héritage, c'est
\item[\vref{Jé 19:4}] et qu'ils ont p. ce lieu, et
\item[\vref{Ez 22:26}] loi et ont p. mes choses saintes ;
\item[\vref{Ez 24:21}] Voici, je vais p. mon lieu saint,
\item[\vref{Da 11:31}] côté, et on p. le sanctuaire qui
\item[\vref{Am 2:7}] jeune fille, pour p. mon saint Nom.
\item[\vref{Mi 4:11}] disent : Qu'elle soit p. ! Et que notre
\item[\vref{Mal 1:7}] En quoi t'avons-ns. p. ? C'est en disant :
\item[\vref{Mal 1:12}] vs., vs. le p., en disant : La
\item[\vref{Ac 24:6}] mm tenté de p. le temple ; et
\end{listverse}

\ConcordanceEntry{Profit}
\vspace{-2mm}
\begin{listverse}
\item[\vref{1 S 12:21}] vs. apportent ni p. ni délivrance, puisque
\item[\vref{Job 22:2}] L'hom. apportera-t-il qq p. à Dieu ? C'est
\item[\vref{Ps 119:36}] tes préceptes et non point au p.
\item[\vref{Mi 4:13}] par interdit leurs p. à Yahweh, et
\item[\vref{Ac 16:16}] apportait un grand p. à ses maîtres,
\item[\vref{2 Ti 2:14}] ne revient aucun p., mais elle est
\item[\vref{Jud 1:16}] personnes pour le p. qui lr. en
\end{listverse}

\ConcordanceEntry{Profond}
\vspace{-2mm}
\begin{listverse}
\item[\vref{Ge 2:21}] fit tomber un p. sommeil sur Adam,
\item[\vref{Ge 15:12}] se couchait, qu'un p. sommeil tomba sur
\item[\vref{Job 11:8}] feras-tu ? C'est plus p. que le scheol :
\item[\vref{Ps 92:6}] sont magnifiques ! Tes pensées sont merveilleusement p.
\item[\vref{Ec 7:24}] ce qui est p., qui le trouvera ?
\item[\vref{Es 29:10}] un esprit d'un p. sommeil ; il a
\item[\vref{Da 2:22}] révèle les choses p. et cachées, il
\item[\vref{Am 5:8}] qui change les p. ténèbres en aube
\item[\vref{Jn 4:11}] le puits est p. ; d'où aurais-tu dc
\item[\vref{1 Co 2:10}] sonde ttes choses, mm les choses p. de Dieu.
\end{listverse}

\ConcordanceEntry{Profondeur}
\vspace{-2mm}
\begin{listverse}
\item[\vref{Ps 139:15}] brodé ds les p. de la terre.
\item[\vref{Pr 25:3}] cause de sa p. ; ni le cœur
\item[\vref{Ez 31:18}] d'Eden ds les p. de la terre,
\item[\vref{Jon 2:4}] jeté ds les p., au cœur de
\item[\vref{Ro 8:39}] hauteur, ni la p., ni aucune autre
\item[\vref{Ro 11:33}] Ô p. de la richesse, et de la
\item[\vref{Ep 3:18}] la longueur, la p. et la hauteur,
\item[\vref{Ap 2:24}] pas connu les p. de Satan, com.
\end{listverse}

\ConcordanceEntry{Progrès}
\vspace{-2mm}
\begin{listverse}
\item[\vref{1 Th 4:1}] fassiez ts les jours de nouveaux p.
\item[\vref{2 Th 1:3}] que votre charité mutuelle fait des p.
\item[\vref{1 Ti 4:15}] afin que tes p. soient évidents pour
\item[\vref{2 Ti 3:9}] de plus grands p., car lr. folie
\end{listverse}

\ConcordanceEntry{Proie}
\vspace{-2mm}
\begin{listverse}
\item[\vref{Ge 15:11}] les oiseaux de p. descendirent sur les
\item[\vref{Ge 49:27}] il dévorera la p., et sur le
\item[\vref{Ps 124:6}] point livrés en p. à leurs dents !
\item[\vref{Pr 23:28}] aguets pour une p., et elle augmente
\item[\vref{Os 5:14}] irai, j'emporterai la p., et il n'y
\item[\vref{Na 2:13}] leurs tanières de p., et leurs repaires
\item[\vref{Ha 2:7}] te tourmenter ? Et tu deviendras lr. p.
\item[\vref{Za 2:9}] elles seront la p. de leurs serviteurs.
\item[\vref{Col 2:8}] de vs. sa p. par la philosophie,
\end{listverse}

\ConcordanceEntry{Projet}
\vspace{-2mm}
\begin{listverse}
\item[\vref{Esd 4:5}] faire échouer lr. p., pendant tt le
\item[\vref{Job 5:12}] il anéantit les p. des hommes rusés,
\item[\vref{Job 27:11}] cacherai pas les p. du Tout-Puissant.
\item[\vref{Pr 16:3}] Yahweh, et tes p. seront bien ordonnés.
\item[\vref{Pr 20:18}] Les p. s'affermissent par le conseil ; fais dc
\item[\vref{Ez 16:43}] plus de mauvais p. avec ttes tes
\end{listverse}

\ConcordanceEntry{Promesse}
\vspace{-2mm}
\begin{listverse}
\item[\vref{Ps 119:81}] attendant ta délivrance ; j'espère en ta p.
\item[\vref{Ac 2:39}] est faite la p., et à ts
\item[\vref{Ac 7:17}] temps de la p., pour laquelle Dieu
\item[\vref{Ac 13:23}] Dieu, selon sa p., a suscité Jésus
\item[\vref{Ro 4:14}] anéantie, et la p. est abolie,
\item[\vref{Ro 9:8}] enfants de la p. qui sont réputés
\item[\vref{2 Co 1:20}] y a de p. de Dieu, ttes
\item[\vref{Ga 3:16}] Or les p. ont été faites à Abraham et
\item[\vref{Ga 4:23}] libre naquit en vertu de la p.
\item[\vref{Ep 3:6}] ensemble à sa p. en Christ par
\item[\vref{Ep 6:2}] c'est le premier commandement avec une p.,
\item[\vref{1 Ti 4:8}] choses, ayant les p. de la vie
\item[\vref{Hé 6:13}] Dieu fit la p. à Abraham, ne
\item[\vref{Hé 7:6}] et bénit celui qui avait les p.
\item[\vref{Hé 8:6}] a été établie sur de meilleures p.
\item[\vref{Hé 10:23}] a fait la p. est fidèle.
\item[\vref{2 Pi 1:4}] grandes et précieuses p., afin que par
\item[\vref{2 Pi 3:4}] Où est la p. de son avènement ?
\item[\vref{2 Pi 3:9}] l'exécution de sa p., com. quelques-uns croient
\item[\vref{1 Jn 2:25}] c'est ici la p. qu'il ns. a
\end{listverse}

\ConcordanceEntry{Promettre}
\vspace{-2mm}
\begin{listverse}
\item[\vref{De 26:17}] Tu as fait p. aujourd'hui à Yahweh
\item[\vref{De 26:18}] Yahweh t'a fait p. que tu seras
\item[\vref{2 Pi 2:19}] ils lr. p. la liberté, alors qu'ils sont eux-mêmes
\end{listverse}

\ConcordanceEntry{Prophète}
\vspace{-2mm}
\begin{listverse}
\item[\vref{Ge 20:7}] car il est p. ; et il priera
\item[\vref{Ex 7:1}] et Aaron, ton frère, sera ton p.
\item[\vref{No 11:29}] de Yahweh fût p., et que Yahweh
\item[\vref{De 18:15}] tes frères, un p. com. moi: Vous
\item[\vref{De 34:10}] en Israël de p. com. Moïse, que
\item[\vref{1 S 3:20}] Samuel était établi p. de Yahweh.
\item[\vref{1 S 9:9}] voyant ! Car le p. s'appelait autrefois le
\item[\vref{1 S 10:11}] était avec les p., et qu'il prophétisait,
\item[\vref{1 S 19:20}] une assemblée de p. qui prophétisaient, et
\item[\vref{1 R 13:11}] avait un vieux p. qui demeurait à
\item[\vref{1 R 18:4}] Jézabel exterminait les p. de Yahweh, Abdias
\item[\vref{1 R 18:19}] quatre cent cinquante p. de Baal et
\item[\vref{1 R 18:22}] suis demeuré seul p. de Yahweh ; et
\item[\vref{2 R 9:1}] Alors Elisée, le p., appela l'un des
\item[\vref{2 R 19:2}] vers Esaïe, le p., fils d'Amots.
\item[\vref{2 Ch 18:5}] d'Israël assembla les p., au nombre de
\item[\vref{2 Ch 20:20}] croyez en ses p., et vs. réussirez.
\item[\vref{Esd 5:2}] avec eux les p. de Dieu qui
\item[\vref{Ps 105:15}] faites pas de mal à mes p. !
\item[\vref{Jé 1:5}] je t'avais établi p. pour les nations.
\item[\vref{Jé 5:31}] C'est que les p. prophétisent le mensonge,
\item[\vref{Jé 23:11}] Car le p. et le prêtre sont corrompus ; j'ai
\item[\vref{La 4:13}] péchés de ses p., et des iniquités
\item[\vref{Ez 2:5}] aura eu un p. parmi eux.
\item[\vref{Ez 7:26}] la vision aux p. ; la loi périra
\item[\vref{Ez 13:2}] prophétise contre les p. d'Israël qui prophétisent,
\item[\vref{Ez 14:9}] arrive que le p. soit séduit, et
\item[\vref{Os 9:7}] le saura ! Les p. sont fous, les
\item[\vref{Ha 1:1}] Prophétie qu'Habakuk le p. a vue.
\item[\vref{Mt 5:12}] a persécuté les p. qui ont été
\item[\vref{Mt 5:17}] loi ou les p. ; je ne suis
\item[\vref{Mt 10:41}] qui reçoit un p. en qualité de
\item[\vref{Mt 11:9}] allés voir ? Un p. ? Oui, vs. dis-je,
\item[\vref{Mt 13:17}] vérité, beaucoup de p. et de justes
\item[\vref{Mt 13:57}] lr. dit : Un p. n'est méprisé que
\item[\vref{Mt 14:5}] parce qu'elle regardait Jean com. un p.
\item[\vref{Mt 21:11}] C'est Jésus, le p. de Nazareth, en
\item[\vref{Mt 24:24}] et de faux p., ils feront de
\item[\vref{Lu 1:76}] tu seras appelé p. du Très-Haut ; car
\item[\vref{Lu 6:26}] en faisaient de mm aux faux p. !
\item[\vref{Lu 7:39}] cet hom. était p., certes il saurait
\item[\vref{Lu 13:33}] convient pas qu'un p. meure hors de
\item[\vref{Lu 16:29}] Moïse et les p. ; qu'ils les écoutent.
\item[\vref{Lu 24:19}] qui était un p. puissant en œuvres
\item[\vref{Lu 24:25}] ce que les p. ont annoncé !
\item[\vref{Jn 4:19}] je vois que tu es un p.
\item[\vref{Jn 6:14}] est véritablement le P. qui devait venir
\item[\vref{Ac 2:30}] com. il était p., et qu'il savait
\item[\vref{Ac 3:24}] mm ts les p. depuis Samuel, et
\item[\vref{Ac 10:43}] Tous les p. rendent de lui le témoignage que
\item[\vref{Ac 13:1}] à Antioche des p. et des docteurs :
\item[\vref{Ac 13:6}] certain magicien, faux p. Juif, nommé Bar-Jésus,
\item[\vref{1 Co 12:28}] apôtres, deuxièmement des p., troisièmement des docteurs,
\item[\vref{1 Co 14:32}] les esprits des p. sont soumis aux
\item[\vref{Ep 2:20}] apôtres et des p., et Jésus-Christ lui-mm
\item[\vref{Hé 1:1}] pères par les p., à plusieurs reprises
\item[\vref{1 Pi 1:10}] salut que les p., qui ont prophétisé
\item[\vref{Ap 11:10}] que ces deux p. ont tourmenté les
\item[\vref{Ap 20:10}] et le faux p.. Et ils seront
\item[\vref{Ap 22:6}] Dieu des saints p., a envoyé son
\end{listverse}

\ConcordanceEntry{Prophétesse}
\vspace{-2mm}
\begin{listverse}
\item[\vref{Ex 15:20}] Et Marie, la p., sœur d'Aaron, prit
\item[\vref{Jg 4:4}] ce temps-là, Débora, p., fem. de Lappidoth,
\item[\vref{2 R 22:14}] auprès de la p. Hulda, fem. de
\item[\vref{Né 6:14}] de Noadia, la p., et du reste
\item[\vref{Es 8:3}] approché de la p. ; elle conçut et
\item[\vref{Lu 2:36}] aussi Anne, la p., fille de Phanuel,
\item[\vref{Ap 2:20}] qui se dit p., enseigner et séduire
\end{listverse}

\ConcordanceEntry{Prophétie}
\vspace{-2mm}
\begin{listverse}
\item[\vref{Esd 6:14}] succès, selon les p. d'Aggée, le prophète,
\item[\vref{Da 9:24}] et à la p. et pour oindre
\item[\vref{Ha 1:1}] P. qu'Habakuk le prophète a vue.
\item[\vref{Za 12:1}] P., parole de Yahweh, sur Israël. Ainsi
\item[\vref{Ro 12:6}] donnée, soit la p., prophétisons selon la
\item[\vref{1 Co 12:10}] un autre, la p. ; à un autre,
\item[\vref{1 Co 13:2}] le don de p. et que je
\item[\vref{1 Co 13:8}] périt jamais. Les p. seront abolies et
\item[\vref{1 Co 14:6}] science, ou par p., ou par doctrine ?
\item[\vref{1 Co 14:22}] les non-croyants ; la p., au contraire, est
\item[\vref{1 Th 5:20}] Ne méprisez pas les p.
\item[\vref{1 Ti 1:18}] que conformément aux p. qui auparavant ont
\item[\vref{1 Ti 4:14}] été donné par p., par l'imposition des
\item[\vref{2 Pi 1:20}] d'abord ceci: Qu'aucune p. de l'Ecriture ne
\item[\vref{2 Pi 1:21}] car la p. n'a jamais été autrefois apportée par
\item[\vref{Ap 1:3}] paroles de cette p., et qui gardent
\item[\vref{Ap 11:6}] jours de lr. p. ; ils ont aussi
\item[\vref{Ap 19:10}] de Jésus est l'Esprit de la p.
\item[\vref{Ap 22:7}] paroles de la p. de ce livre !
\item[\vref{Ap 22:10}] paroles de la p. de ce livre.
\end{listverse}

\ConcordanceEntry{Prophétiser}
\vspace{-2mm}
\begin{listverse}
\item[\vref{No 11:25}] sur eux, ils p. ; mais ils ne
\item[\vref{1 S 10:10}] saisit, et il p. au milieu d'eux.
\item[\vref{1 S 19:24}] ses vêtements et p. dvt Samuel ; et
\item[\vref{1 R 22:8}] car il ne p. rien de bon,
\item[\vref{Esd 5:1}] d'Iddo, le prophète, p. aux Juifs qui
\item[\vref{Es 30:10}] prophètes : Ne ns. p. pas des choses
\item[\vref{Jé 2:8}] les prophètes ont p. par Baal, et
\item[\vref{Jé 5:31}] que les prophètes p. le mensonge, et
\item[\vref{Jé 29:27}] Jérémie d'Anathoth, qui p. parmi vs. ?
\item[\vref{Ez 12:27}] longtemps, et il p. pour des temps
\item[\vref{Ez 37:4}] il me dit : P. sur ces os,
\item[\vref{Ez 37:9}] il me dit : P. à l'Esprit ! Prophétise,
\item[\vref{Joë 2:28}] et vos filles p. ; vos vieillards auront
\item[\vref{Am 3:8}] Le Seign. Yahweh parle, qui ne p. ?
\item[\vref{Mi 2:6}] Ne p. point, disent-ils, ne prophétisez point de
\item[\vref{Mi 2:11}] disant : Je te p. sur du vin
\item[\vref{Za 13:3}] que qnd quelqu'un p. dorénavant, son père
\item[\vref{Mt 7:22}] Seign. ! N'avons-ns. pas p. en ton Nom ?
\item[\vref{Mt 11:13}] la loi ont p. jusqu'à Jean.
\item[\vref{Mt 26:68}] en disant : Christ, p.-ns. qui est
\item[\vref{Ac 2:17}] et vos filles p., vos jeunes gens
\item[\vref{Ac 19:6}] et ils parlaient diverses langues et p.
\item[\vref{Ac 21:9}] Il avait quatre filles vierges qui p.
\item[\vref{Ro 12:6}] soit la prophétie, p. selon la proportion
\item[\vref{1 Co 11:4}] prie ou qui p., ayant qq chose
\item[\vref{1 Co 14:4}] mais celui qui p. édifie l'Eglise.
\item[\vref{1 Co 14:31}] vs. pouvez ts p. l'un après l'autre,
\item[\vref{1 Pi 1:10}] prophètes, qui ont p. concernant la grâce
\item[\vref{Jud 1:14}] après Adam, a p., en disant :
\item[\vref{Ap 10:11}] faut que tu p. de nouveau sur
\item[\vref{Ap 11:3}] deux témoins de p. pendant mille deux
\end{listverse}

\ConcordanceEntry{Propitiation}
\vspace{-2mm}
\begin{listverse}
\item[\vref{Ex 30:10}] fois l'an la p. sur les cornes
\item[\vref{Ex 32:30}] peut-être je ferai p. pour votre péché.
\item[\vref{Lé 14:19}] l'expiation et fera p. pour celui qui
\item[\vref{Lé 16:10}] Yahweh pour faire p. par lui, et
\item[\vref{Lé 17:11}] afin de faire p. pour vos âmes,
\item[\vref{No 8:19}] et pour faire p. pour les enfants
\item[\vref{No 31:50}] afin de faire p. pour nos personnes
\item[\vref{No 35:33}] fera point de p. pour le pays,
\item[\vref{De 32:43}] ennemis, et fait p. pour sa terre
\item[\vref{1 S 3:14}] il se fait p. pour l'iniquité de
\item[\vref{1 Ch 6:49}] et pour faire p. pour Israël ; com.
\item[\vref{2 Ch 30:18}] bon, tienne la p. pour faite,
\item[\vref{Né 10:33}] afin de faire p. pour Israël ; et
\item[\vref{Job 33:24}] ds la fosse ; j'ai trouvé la p. !
\item[\vref{Ps 65:4}] tu feras la p. de nos transgressions.
\item[\vref{Pr 16:6}] Il y aura p. de l'iniquité par
\item[\vref{Es 6:7}] ôtée, et la p. est faite pour
\item[\vref{Ez 45:15}] afin de faire p. pour vs., dit
\item[\vref{Da 9:24}] péchés, faire la p. pour l'iniquité, pour
\item[\vref{Ro 3:25}] une victime de p. par la foi,
\item[\vref{Hé 2:17}] pour faire la p. pour les péchés
\item[\vref{1 Jn 2:2}] la victime de p. pour nos péchés,
\item[\vref{1 Jn 4:10}] pour être la p. pour nos péchés.
\end{listverse}

\ConcordanceEntry{Propitiatoire}
\vspace{-2mm}
\begin{listverse}
\item[\vref{Ex 25:17}] feras aussi un p. d'or pur, dont
\item[\vref{Ex 25:21}] tu poseras le p. au-dessus de l'arche,
\item[\vref{Ex 25:22}] de dessus le p., d'entre les deux
\item[\vref{Ex 40:20}] mit aussi le p. au-dessus de l'arche.
\item[\vref{Lé 16:2}] voile, dvt le p. qui est sur
\item[\vref{Lé 16:13}] parfum couvre le p. qui est sur
\item[\vref{Lé 16:14}] doigt au-dvt du p. vers l'orient ; il
\item[\vref{No 7:89}] du haut du p. placé sur l'arche
\item[\vref{Hé 9:5}] lr. ombre le p.. Ce n'est pas
\end{listverse}

\ConcordanceEntry{Propos}
\vspace{-2mm}
\begin{listverse}
\item[\vref{Ge 37:2}] rapportait à lr. père leurs mauvais p.
\item[\vref{Ex 21:14}] quelqu'un s'élève de p. délibéré contre son
\item[\vref{Ps 19:15}] Que les p. de ma bouche et la méditation
\item[\vref{Ps 56:6}] ils tordent mes p., et ttes leurs
\item[\vref{Pr 13:3}] ouvre à tt p. ses lèvres, tombera
\item[\vref{Pr 18:13}] répond à qq p. avant de l'avoir
\item[\vref{Pr 25:11}] telle est une parole dite à p.
\item[\vref{Ca 5:2}] Sulamithe rapporte les p. de Salomon :] Ouvre-moi,
\item[\vref{Es 33:15}] pas entendre des p. sanguinaires, et qui
\item[\vref{Lu 20:37}] connaître qnd, à p. du buisson, il
\item[\vref{Ep 5:4}] parole grossière, ni p. insensés, ni plaisanterie,
\end{listverse}

\ConcordanceEntry{Propriété}
\vspace{-2mm}
\begin{listverse}
\item[\vref{Ge 23:18}] fut acquis com. p. à Abraham, en
\item[\vref{Ge 49:30}] le Héthien, com. p. sépulcrale.
\item[\vref{Jos 1:15}] qui est votre p., et que vs.
\item[\vref{1 R 12:16}] n'avons point de p. avec le fils
\item[\vref{Né 11:3}] chacun ds sa p., selon sa ville,
\item[\vref{Ez 36:3}] vs. soyez la p. des autres nations,
\item[\vref{Ez 45:8}] sa terre, sa p. en Israël ; et
\item[\vref{Ez 48:20}] carré pour la p. de la ville.
\end{listverse}

\ConcordanceEntry{Prosélyte}
\vspace{-2mm}
\begin{listverse}
\item[\vref{Mt 23:15}] pour faire un p., et qnd il
\item[\vref{Ac 2:10}] sont venus de Rome ? Juifs et P.,
\item[\vref{Ac 6:5}] Parménas, et Nicolas, p. d'Antioche.
\item[\vref{Ac 13:43}] Juifs et de p. craignant Dieu, suivirent
\end{listverse}

\ConcordanceEntry{Prospérer}
\vspace{-2mm}
\begin{listverse}
\item[\vref{Ge 39:3}] que Yahweh faisait p. entre ses mains
\item[\vref{1 Ch 22:11}] toi, et tu p., et tu bâtiras
\item[\vref{2 Ch 26:5}] il rechercha Yahweh, Dieu le fit p.
\item[\vref{Job 8:6}] toi, et fera p. la demeure de
\item[\vref{Ps 128:2}] mains ; tu es heureux et tu p.
\item[\vref{Pr 28:13}] ses transgressions ne p. pas, mais celui
\item[\vref{Es 52:13}] Voici, mon serviteur p., il sera fort
\item[\vref{Es 53:10}] plaisir de Yahweh p. en sa main.
\item[\vref{Es 55:11}] prends plaisir, et p. ds l'œuvre pour
\item[\vref{Jé 23:5}] en Roi ; il p., et exercera le
\item[\vref{Da 6:28}] Ainsi Daniel p. sous le règne
\item[\vref{Za 8:12}] Car les semailles p., la semence de
\item[\vref{3 Jn 1:2}] souhaite que tu p. en ttes choses,
\end{listverse}

\ConcordanceEntry{Prospérité}
\vspace{-2mm}
\begin{listverse}
\item[\vref{1 R 10:7}] sagesse et ta p. surpassent tt ce
\item[\vref{Job 5:24}] paix de la p. sous ta tente,
\item[\vref{Ps 118:25}] je te prie, donne mntnt la p. !
\item[\vref{Pr 1:32}] tue, et la p. des insensés les
\item[\vref{Jé 2:37}] ta confiance, et tu n'auras aucune p. par eux.
\item[\vref{Jé 22:21}] durant ta grande p., mais tu disais :
\item[\vref{Jé 33:9}] de tte la p. que je vais
\item[\vref{1 Co 16:2}] assembler, selon la p. que Dieu lui
\item[\vref{3 Jn 1:2}] santé, com. ton âme est en p.
\end{listverse}

\ConcordanceEntry{Prosterner}
\vspace{-2mm}
\begin{listverse}
\item[\vref{Ge 47:31}] Et Israël se p. sur le chevet
\item[\vref{Ex 20:5}] Tu ne te p. point dvt elles, et ne les
\item[\vref{De 17:3}] dieux et se p. dvt eux, dvt
\item[\vref{Ps 72:11}] rois aussi se p. dvt lui, ttes
\item[\vref{Ps 95:6}] Venez, p.-ns., inclinons-ns., et mettons-ns. à genoux
\item[\vref{Ps 99:9}] Yahweh, notre Dieu, p.-vs. sur la
\item[\vref{Es 44:17}] taillée ; il se p. dvt elle, il
\item[\vref{Jé 1:16}] et se sont p. dvt l'ouvrage de
\item[\vref{Ez 8:16}] et ils se p. vers l'orient, dvt
\item[\vref{So 1:5}] ceux qui se p. sur les toits
\item[\vref{Mt 4:9}] si tu te p. et m'adores.
\item[\vref{Mt 20:20}] fils, et se p. pour lui demander
\item[\vref{Ap 3:9}] venir et se p. à tes pieds,
\item[\vref{Ap 4:10}] vingt-quatre anciens se p. dvt celui qui
\end{listverse}

\ConcordanceEntry{Prostitué, prostituée}
\vspace{-2mm}
\begin{listverse}
\item[\vref{Ge 38:15}] que c'était une p., car elle avait
\item[\vref{Lé 21:7}] point une fem. p. ou déshonorée ; ils
\item[\vref{Lé 21:14}] fem. déshonorée ou p. ; mais il prendra
\item[\vref{De 23:17}] filles d'Israël, aucune p., et il n'y
\item[\vref{De 23:18}] le salaire d'une p., ni le prix
\item[\vref{Jos 2:1}] maison d'une fem. p., nommée Rahab, et
\item[\vref{Jg 11:1}] fils d'une fem. p. ; et c'est Galaad
\item[\vref{Jg 16:1}] vit une fem. p. et il entra
\item[\vref{1 R 3:16}] Alors deux femmes p. vinrent au roi
\item[\vref{1 R 14:24}] pays des hommes p. à la paillardise,
\item[\vref{2 R 23:7}] les maisons des p. qui étaient ds
\item[\vref{Pr 6:26}] de la fem. p. on est réduit
\item[\vref{Pr 29:3}] avec les femmes p. dissipe ses richesses.
\item[\vref{Es 1:21}] est-elle devenue une p. ? Elle était pleine
\item[\vref{Jé 3:13}] Dieu, tu t'es p. aux étrangers sous
\item[\vref{Ez 23:30}] que tu t'es p. aux nations, avec
\item[\vref{Os 1:2}] prends une fem. p. et des enfants
\item[\vref{Joë 3:3}] l'enfant pour une p., ils ont vendu
\item[\vref{Mi 1:7}] ils serviront de salaire à une p.
\item[\vref{Na 3:4}] prostitutions de cette p., pleine de charmes,
\item[\vref{Mt 21:31}] publicains et les p. vs. devanceront ds
\item[\vref{1 Co 6:16}] s'unit à la p., devient un mm
\item[\vref{Hé 11:31}] foi, Rahab, la p., ne périt pas
\item[\vref{Ja 2:25}] Pareillement, Rahab, la p., ne fut-elle pas
\item[\vref{Ap 17:1}] de la grande p., qui est assise
\item[\vref{Ap 19:2}] jugé la grande p. qui a corrompu
\end{listverse}

\ConcordanceEntry{Prostituer}
\vspace{-2mm}
\begin{listverse}
\item[\vref{Ex 34:15}] viendront à se p. après leurs dieux
\item[\vref{Lé 18:23}] fem. ne se p. point à une
\item[\vref{De 31:16}] lèvera et se p. après les dieux
\item[\vref{Jg 8:33}] détournèrent et se p. aux Baals, et
\item[\vref{2 Ch 21:13}] Jérus., com. s'est p. la maison d'Achab,
\item[\vref{Ps 106:39}] œuvres, et se p. par leurs actions.
\item[\vref{Es 23:17}] et elle se p. avec ts les
\item[\vref{Jé 3:1}] toi, tu t'es p. à plusieurs amoureux,
\item[\vref{Ez 16:16}] aura jamais, et tu t'y es p.
\item[\vref{Os 2:7}] lr. mère s'est p., celle qui les
\item[\vref{Os 9:1}] que tu t'es p. en abandonnant ton
\item[\vref{Am 7:17}] Ta fem. se p. ds la ville,
\end{listverse}

\ConcordanceEntry{Prostitution}
\vspace{-2mm}
\begin{listverse}
\item[\vref{No 15:39}] pour vs. laisser entraîner à la p.
\item[\vref{Jg 8:27}] un objet de p. pour tt Israël ;
\item[\vref{2 R 9:22}] que durent les p. de Jézabel, ta
\item[\vref{Es 23:17}] salaire de sa p., et elle se
\item[\vref{Jé 3:2}] pays par tes p. et par ta
\item[\vref{Ez 6:9}] livrés à la p. après leurs idoles ;
\item[\vref{Ez 23:17}] lit de ses p., et la souillèrent
\item[\vref{Os 1:2}] des enfants de p. ; car le pays
\item[\vref{Os 3:3}] plus à la p., tu ne seras
\item[\vref{Mi 1:7}] ses salaires de p. seront brûlés au
\item[\vref{Na 3:4}] la multitude des p. de cette prostituée,
\item[\vref{Ap 2:21}] repente de sa p., mais elle ne
\item[\vref{Ap 17:2}] été enivrés du vin de sa p.
\item[\vref{Ap 18:3}] vin de sa p. effrénée, et les
\end{listverse}

\ConcordanceEntry{Protecteur}
\vspace{-2mm}
\begin{listverse}
\item[\vref{Ps 5:12}] tu sois lr. p. ; et que ceux
\item[\vref{Ps 31:3}] moi un Rocher p., une forteresse, afin
\item[\vref{Ps 43:2}] Toi, mon Dieu p., pourquoi me repousses-tu ?
\item[\vref{Ez 28:16}] d'entre les pierres éclatantes, ô chérubin p. !
\item[\vref{Os 4:18}] Apportez ; ce n'est qu'ignominie que ses p.
\item[\vref{Za 9:15}] armées sera lr. p. ; ils dévoreront après
\item[\vref{Za 12:8}] Yahweh sera le p. des habitants de
\end{listverse}

\ConcordanceEntry{Protection}
\vspace{-2mm}
\begin{listverse}
\item[\vref{No 14:9}] notre pain, lr. p. s'est retirée de
\item[\vref{Ps 52:9}] Dieu pour sa p., mais qui se
\item[\vref{Ps 60:9}] Ephraïm est la p. de ma tête,
\item[\vref{Ez 28:14}] pour servir de p. ; je t'avais établi,
\item[\vref{Lu 1:54}] pris sous sa p. Israël, son serviteur,
\end{listverse}

\ConcordanceEntry{Protéger}
\vspace{-2mm}
\begin{listverse}
\item[\vref{De 33:12}] lui ; il le p. toujours, et demeurera
\item[\vref{2 S 8:6}] un tribut. Yahweh p. David partout où
\item[\vref{2 R 19:34}] Car je p. cette ville, afin de la délivrer,
\item[\vref{2 R 20:6}] d'Assyrie ; et je p. cette ville, par
\item[\vref{2 Ch 32:22}] et il les p. de ttes parts.
\item[\vref{Ps 20:2}] Nom du Dieu de Jacob te p. !
\item[\vref{Pr 4:6}] te gardera ; aime-la, et elle te p.
\item[\vref{Es 37:35}] Car je p. cette ville pour la délivrer pour
\item[\vref{So 2:3}] l'humilité ; peut-être serez-vs. p. au jour de
\end{listverse}

\ConcordanceEntry{Prouver}
\vspace{-2mm}
\begin{listverse}
\item[\vref{Né 7:61}] lr. race, pour p. qu'ils étaient d'Israël.
\item[\vref{Ac 9:22}] habitaient à Damas, p. que Jésus était
\item[\vref{Ac 25:7}] graves accusations, qu'ils ne pouvaient pas p.
\item[\vref{Ro 3:9}] ns. avons déjà p. que ts, tant
\item[\vref{Ro 5:8}] Mais Dieu p. son amour envers
\end{listverse}

\ConcordanceEntry{Prudence}
\vspace{-2mm}
\begin{listverse}
\item[\vref{Ps 119:99}] J'ai surpassé en p. ts ceux qui
\item[\vref{Pr 1:5}] et l'hom. intelligent acquerra de la p. ;
\item[\vref{Pr 3:4}] grâce et la p. au yeux de
\item[\vref{Pr 11:14}] par faute de p., mais la délivrance
\item[\vref{Pr 12:8}] raison de sa p., mais celui qui
\item[\vref{Pr 15:5}] garde à la réprimande agit avec p.
\item[\vref{Pr 16:16}] excellent que l'argent, d'acquérir de la p. !
\item[\vref{Pr 19:11}] La p. de l'hom. retient sa colère ; c'est
\item[\vref{Pr 20:18}] conseil ; fais dc la guerre avec p.
\item[\vref{Pr 24:6}] car par la p. tu feras la
\item[\vref{Es 36:5}] guerre de la p. et de la
\item[\vref{Ac 24:3}] faits pour ce peuple, selon ta p.
\end{listverse}

\ConcordanceEntry{Prudent}
\vspace{-2mm}
\begin{listverse}
\item[\vref{2 Ch 2:12}] un fils sage, p. et intelligent, qui
\item[\vref{Pr 10:5}] L'enfant p. amasse en été, mais celui qui
\item[\vref{Pr 10:19}] celui qui retient ses lèvres est p.
\item[\vref{Pr 11:12}] est dépourvu de sens, mais l'hom. p. se tait.
\item[\vref{Pr 14:18}] folie ; mais les p. seront couronnés de
\item[\vref{Pr 14:35}] plaisir au serviteur p., mais son indignation
\item[\vref{Pr 15:14}] cœur de l'hom. p. cherche la science ;
\item[\vref{Pr 16:21}] On appellera p. le sage de
\item[\vref{Pr 17:2}] Le serviteur p. sera maître sur
\item[\vref{Pr 19:14}] mais la fem. p. est un don
\item[\vref{Pr 22:3}] L'hom. p. prévoit le mal et se cache,
\item[\vref{Pr 28:7}] est un fils p., mais celui qui
\item[\vref{Os 14:9}] celui qui est p. ? Qu'il les connaisse !
\item[\vref{Am 5:13}] C'est pourquoi, l'hom. p. se tient ds
\item[\vref{Mt 7:24}] à un hom. p. qui a bâti
\item[\vref{Mt 10:16}] loups. Soyez dc p. com. des serpents,
\item[\vref{Mt 24:45}] serviteur fidèle et p., que son maître
\item[\vref{Lu 16:8}] siècle sont plus p. ds lr. génération,
\item[\vref{Ro 16:19}] que vs. soyez p. quant au bien,
\item[\vref{Tit 2:2}] soient sobres, honnêtes, p., sains ds la
\end{listverse}

\ConcordanceEntry{Prunelle}
\vspace{-2mm}
\begin{listverse}
\item[\vref{De 32:10}] gardé com. la p. de son œil,
\item[\vref{Ps 17:8}] Garde-moi com. la p. de l'œil, cache-moi
\item[\vref{Pr 7:2}] enseignements com. la p. de tes yeux.
\item[\vref{La 2:18}] et que la p. de tes yeux
\item[\vref{Za 2:8}] touche à la p. de son œil.
\end{listverse}

\ConcordanceEntry{Publicain}
\vspace{-2mm}
\begin{listverse}
\item[\vref{Mt 5:46}] en aurez-vs. ? Les p. aussi n'en font-ils
\item[\vref{Mt 9:11}] mange-t-il avec les p. et les gens
\item[\vref{Mt 11:19}] un ami des p. et des gens
\item[\vref{Mt 18:17}] com. un Gentil et com. un p.
\item[\vref{Mt 21:31}] en vérité, les p. et les prostituées
\item[\vref{Lu 3:12}] à lui des p. pour être baptisés,
\item[\vref{Lu 7:29}] cela, et les p. justifiaient Dieu, ayant
\item[\vref{Lu 15:1}] Or ts les p. et les pécheurs
\item[\vref{Lu 18:10}] prier, l'un était pharisien, et l'autre p.
\item[\vref{Lu 18:13}] Mais le p. se tenant loin, n'osait mm pas
\item[\vref{Lu 19:2}] Zachée, chef des p., cherchait à voir
\end{listverse}

\ConcordanceEntry{Publier}
\vspace{-2mm}
\begin{listverse}
\item[\vref{Lé 23:2}] Yahweh, que vs. p., seront de saintes
\item[\vref{No 29:1}] Ce jour sera p. parmi vs. au
\item[\vref{Jg 7:3}] Maintenant dc fais p. ceci aux oreilles
\item[\vref{1 R 21:9}] ds ces lettres : P. un jeûne et
\item[\vref{2 Ch 20:3}] rechercher Yahweh, et p. un jeûne pour
\item[\vref{Esd 8:21}] Et je p. là un jeûne près de la
\item[\vref{Est 1:22}] que cela soit p. selon la langue
\item[\vref{Ps 22:32}] viendront et ils p. sa justice au
\item[\vref{Ps 79:13}] en génération ns. p. tes louanges.
\item[\vref{Ps 145:4}] tes œuvres, et p. tes hauts faits !
\item[\vref{Pr 12:23}] cœur des insensés p. la folie.
\item[\vref{Es 3:9}] contre eux, ils p. lr. péché com.
\item[\vref{Es 48:3}] je les ai p. ; je les ai
\item[\vref{Es 52:7}] bonnes nouvelles, qui p. la paix, qui
\item[\vref{Es 61:2}] pour p. une année de grâce de Yahweh,
\item[\vref{Jé 34:15}] dvt moi, en p. la liberté chacun
\item[\vref{Da 2:13}] sentence fut dc p. ; on mettait à
\item[\vref{Joë 3:9}] P. ceci parmi les nations ! Préparez la
\item[\vref{Am 4:5}] remerciement ; proclamez et p. les offrandes volontaires !
\item[\vref{Jon 3:5}] à Dieu, ils p. un jeûne et
\item[\vref{Na 2:1}] bonnes nouvelles, qui p. la paix ! Toi
\item[\vref{Mc 1:45}] allé, commença à p. ouvertement plusieurs choses
\item[\vref{Lu 2:1}] ces jours-là fut p. un édit par
\item[\vref{Lu 4:19}] les opprimés, pour p. une année de
\item[\vref{Lu 8:39}] alla dc, et p. par tte la
\item[\vref{Ro 9:17}] mon Nom soit p. par tte la
\item[\vref{1 Co 14:25}] adorera Dieu et p. que Dieu est
\end{listverse}

\ConcordanceEntry{Puiser}
\vspace{-2mm}
\begin{listverse}
\item[\vref{Ge 24:11}] celles qui vont p. de l'eau.
\item[\vref{Ex 2:19}] il ns. a p. abondamment de l'eau
\item[\vref{De 29:11}] jusqu'à celui qui p. ton eau ;
\item[\vref{Jos 9:21}] bois et à p. l'eau pour tte
\item[\vref{Ru 2:9}] de ce que les garçons auront p.
\item[\vref{1 S 7:6}] à Mitspa. Ils p. de l'eau qu'ils
\item[\vref{1 S 9:11}] qui sortaient pour p. de l'eau, et
\item[\vref{Ps 78:15}] com. s'il eût p. des abîmes.
\item[\vref{Pr 20:5}] profondes, et l'hom. intelligent sait y p.
\item[\vref{Es 12:3}] Et vs. p. de l'eau avec joie aux sources
\item[\vref{Es 30:14}] foyer, ou pour p. de l'eau à
\item[\vref{Na 3:14}] P.-toi de l'eau pour le siège !
\item[\vref{Ag 2:16}] au lieu de p. de la cuve
\item[\vref{Jn 2:8}] il lr. dit : P.-en mntnt, et
\item[\vref{Jn 4:7}] fem. samaritaine vint p. de l'eau, Jésus
\end{listverse}

\ConcordanceEntry{Puissance}
\vspace{-2mm}
\begin{listverse}
\item[\vref{Ge 45:19}] tu as la p. d'ordonner: Faites ceci :
\item[\vref{Ex 9:16}] en toi ma p., et afin que
\item[\vref{De 8:17}] force et la p. de ma main
\item[\vref{1 Ch 29:11}] la magnificence, la p., la gloire, l'éternité,
\item[\vref{Job 37:23}] Tout-Puissant, grand en p., en jugement et
\item[\vref{Ps 105:4}] Yahweh et sa p., cherchez continuellement sa
\item[\vref{Ps 110:2}] sceptre de ta p., en disant : Domine
\item[\vref{Da 8:7}] pour délivrer le bélier de sa p.
\item[\vref{Da 11:4}] pas la mm p. qu'il a exercée,
\item[\vref{Os 13:14}] rachèterai de la p. du scheol, je
\item[\vref{Am 6:13}] que ns. avons acquis de la p. ?
\item[\vref{Za 4:6}] point par la p. ni par la
\item[\vref{Za 9:4}] il jettera sa p. ds la mer,
\item[\vref{Mt 6:13}] le règne, la p. et la gloire.
\item[\vref{Mc 13:26}] avec une grande p. et une grande
\item[\vref{Lu 1:35}] toi, et la p. du Très-Haut te
\item[\vref{Lu 4:14}] Galilée ds la p. de l'Esprit, et
\item[\vref{Lu 24:49}] revêtus de la p. d'en haut.
\item[\vref{Ac 1:8}] vs. recevrez la p. du Saint-Esprit qui
\item[\vref{Ac 8:10}] et disaient : Celui-ci est la grande p. de Dieu.
\item[\vref{Ac 26:18}] et de la p. de Satan à
\item[\vref{Ro 1:16}] qu'il est la p. de Dieu pour
\item[\vref{Ro 1:20}] à savoir sa p. éternelle et sa
\item[\vref{1 Co 1:18}] qui sommes sauvés, elle est la p. de Dieu.
\item[\vref{1 Co 1:24}] prêchons Christ, la p. de Dieu et
\item[\vref{1 Co 2:5}] sagesse des hommes, mais de la p. de Dieu.
\item[\vref{1 Co 4:19}] paroles, mais la p. de ceux qui
\item[\vref{2 Co 4:7}] l'excellence de cette p. soit de Dieu
\item[\vref{2 Co 12:9}] suffit, car ma p. s'accomplit ds la
\item[\vref{2 Co 13:4}] vivant par la p. de Dieu ; et
\item[\vref{Ep 1:19}] grandeur de sa p. envers ns. qui
\item[\vref{Ep 3:20}] qui par la p. qui agit en
\item[\vref{Hé 6:5}] Dieu, et les p. du siècle à
\item[\vref{2 Pi 1:3}] Puisque sa divine p. ns. a donné
\item[\vref{Ap 3:8}] as peu de p., que tu as
\item[\vref{Ap 4:11}] gloire, honneur et p. ; car tu as
\item[\vref{Ap 11:17}] éclater ta grande p., et de ce
\end{listverse}

\ConcordanceEntry{Puissant}
\vspace{-2mm}
\begin{listverse}
\item[\vref{Ge 17:1}] suis le Dieu Tout-P.. Marche dvt ma
\item[\vref{Ge 28:3}] Que le Dieu Tout-P. te bénisse, te
\item[\vref{Ge 49:24}] les mains du P. de Jacob : Il
\item[\vref{Ex 6:3}] com. le Dieu Tout-P., mais je n'ai
\item[\vref{Ru 1:20}] Mara, car le Tout-P. m'a remplie d'amertume.
\item[\vref{2 S 3:39}] Tseruja, sont trop p. pour moi. Que
\item[\vref{Est 9:4}] Car Mardochée était p. ds la maison
\item[\vref{Job 11:7}] en le sondant ? Connaîtras-tu parfaitement le Tout-P. ?
\item[\vref{Job 21:15}] Qui est le Tout-P. pour que ns.
\item[\vref{Job 22:25}] et le Tout-P. sera ton or,
\item[\vref{Ps 12:5}] disent : Nous sommes p. par nos langues,
\item[\vref{Ps 24:8}] Yahweh fort et p., Yahweh puissant ds
\item[\vref{Ps 45:4}] Ô Très-p., ceins ton épée sur ta cuisse,
\item[\vref{Ps 89:9}] semblable à toi, p. Yahweh ? Aussi ta
\item[\vref{Ps 103:20}] vs., ses anges p. en force, qui
\item[\vref{Es 9:5}] Conseiller, le Dieu P., le Père d'éternité,
\item[\vref{Mt 3:11}] moi est plus p. que moi, et
\item[\vref{Lu 1:49}] parce que le Tout-P. a fait pour
\item[\vref{Lu 1:52}] leurs trônes les p., et il a
\item[\vref{Ac 7:22}] et il était p. en paroles et
\item[\vref{2 Co 10:4}] mais elles sont p. par la vertu
\item[\vref{Ap 1:8}] QUI ETAIT, et QUI VIENT, le Tout-P.
\item[\vref{Ap 19:6}] Seign. notre Dieu Tout-P. a pris possession
\item[\vref{Ap 21:22}] le Seign. Dieu Tout-P. et l'Agneau en
\end{listverse}

\ConcordanceEntry{Puits}
\vspace{-2mm}
\begin{listverse}
\item[\vref{Ge 16:14}] a appelé ce p. le puits du
\item[\vref{Ge 21:19}] elle vit un p. d'eau ; elle alla
\item[\vref{Ge 26:15}] et ts les p. que les serviteurs
\item[\vref{Ge 26:19}] y trouvèrent un p. d'eau vive.
\item[\vref{Ex 2:15}] de Madian et s'assit près d'un p.
\item[\vref{Lé 11:36}] la source, le p. ou tel autre
\item[\vref{No 20:17}] boirons l'eau d'aucun p. ; ns. marcherons par
\item[\vref{No 21:16}] C'est là le p. où Yahweh dit
\item[\vref{Né 9:25}] de biens, les p. qu'on avait creusés,
\item[\vref{Ps 55:24}] les précipiteras au p. de la perdition ;
\item[\vref{Ps 69:16}] et que le p. ne ferme point
\item[\vref{Pr 5:15}] qui coule du milieu de ton p. ;
\item[\vref{Pr 23:27}] et l'étrangère un p. de détresse.
\item[\vref{Jé 6:7}] Comme le p. fait jaillir ses eaux, ainsi elle
\item[\vref{Lu 14:5}] tombe ds un p., ne l'en retirera
\item[\vref{Jn 4:11}] puiser, et le p. est profond ; d'où
\item[\vref{Ap 9:1}] la clef du p. de l'abîme fut
\end{listverse}

\ConcordanceEntry{Punir}
\vspace{-2mm}
\begin{listverse}
\item[\vref{Ex 19:12}] extrémités. Quiconque touchera la montagne sera p. de mort.
\item[\vref{Ex 20:5}] est jaloux, qui p. l'iniquité des pères
\item[\vref{Ex 32:34}] ferai punition, je p. sur eux lr.
\item[\vref{Esd 9:13}] ns. as moins p. que ne méritaient
\item[\vref{Job 31:11}] crime, une iniquité p. par les juges ;
\item[\vref{Ps 6:2}] Yahweh, ne me p. pas ds ta
\item[\vref{Ps 89:33}] je p. de la verge lr. transgression, et
\item[\vref{Ps 106:32}] et Moïse fut p. à cause d'eux.
\item[\vref{Pr 21:11}] Quand on p. le moqueur, le sot devient sage ;
\item[\vref{Es 1:24}] me satisferai en p. mes adversaires, et
\item[\vref{Es 13:11}] Je p. le monde habitable à cause de
\item[\vref{Jé 9:25}] Yahweh, où je p. tt circoncis ayant
\item[\vref{Za 5:3}] qui vole, sera p. com. elle ; et
\item[\vref{Mt 5:22}] son frère sera p. par les juges ;
\item[\vref{Ac 4:21}] ils pourraient les p., à cause du
\item[\vref{Ro 13:4}] faire justice en p. celui qui fait
\item[\vref{Hé 12:25}] échappé, ns. serons p. beaucoup plus, si
\item[\vref{1 Pi 2:14}] sa part pour p. les méchants et
\item[\vref{2 Pi 2:9}] injustes pour être p. au jour du
\end{listverse}

\ConcordanceEntry{Pur}
\vspace{-2mm}
\begin{listverse}
\item[\vref{Lé 4:12}] ds un lieu p., où l'on répand
\item[\vref{Lé 10:10}] est impur et ce qui est p.,
\item[\vref{Lé 13:13}] alors il jugera p. celui qui a
\item[\vref{Lé 20:25}] séparez les bêtes p. de celles qui
\item[\vref{Job 4:17}] Dieu ? L'hom. serait-il p. dvt celui qui
\item[\vref{Job 11:4}] Ma doctrine est p., et je suis
\item[\vref{Job 15:15}] ne sont pas p. à ses yeux,
\item[\vref{Job 23:10}] m'éprouvait, j'en sortirai p. com. l'or.
\item[\vref{Job 33:9}] Je suis p., sans péché, je suis net, il
\item[\vref{Ps 12:7}] sont des paroles p., c'est un argent
\item[\vref{Ps 18:27}] celui qui est p., tu te montres
\item[\vref{Ps 119:9}] jeune hom. rendra-t-il p. sa voie ? Ce
\item[\vref{Pr 30:12}] qui croit être p., et qui toutefois
\item[\vref{Ha 1:13}] les yeux trop p. pour voir le
\item[\vref{Mt 8:3}] le veux, sois p.. Et à l'instant
\item[\vref{Jn 15:3}] Vous êtes déjà p., à cause de
\item[\vref{Ac 24:16}] toujours une conscience p. dvt Dieu, et
\item[\vref{Ro 14:20}] ttes choses sont p., mais il est
\item[\vref{2 Co 7:11}] de tte manière p. ds cette affaire.
\item[\vref{Ph 1:10}] que vs. soyez p. et irréprochables pour
\item[\vref{1 Ti 5:22}] pas aux péchés d'autrui ; toi-mm, conserve-toi p.
\item[\vref{Hé 10:22}] et le corps lavé d'une eau p.
\end{listverse}

\ConcordanceEntry{Pureté}
\vspace{-2mm}
\begin{listverse}
\item[\vref{2 S 22:21}] rendu selon la p. de mes mains ;
\item[\vref{Job 22:30}] cause de la p. de tes mains.
\item[\vref{Ps 18:21}] traité selon la p. de mes mains,
\item[\vref{Pr 22:11}] qui aime la p. de cœur, et
\item[\vref{2 Co 6:6}] par la p., par la connaissance, par un esprit
\item[\vref{1 Ti 4:12}] charité, en esprit, en foi, en p.
\item[\vref{1 Ti 5:2}] jeunes com. des sœurs, en tte p.
\item[\vref{Tit 2:7}] tte altération, en p., en intégrité,
\item[\vref{Hé 9:13}] et procurent la p. de la chair,
\item[\vref{1 Pi 3:2}] lorsqu'ils verront la p. de votre conduite,
\end{listverse}

\ConcordanceEntry{Purification}
\vspace{-2mm}
\begin{listverse}
\item[\vref{Lé 12:4}] jours de sa p. soient accomplis.
\item[\vref{No 19:9}] faire l'eau de p.. C'est une purification
\item[\vref{1 Ch 23:28}] chambres, pour la p. de ttes les
\item[\vref{2 Ch 4:6}] servir à la p.. On y lavait
\item[\vref{Né 12:45}] charge de la p.. Les chantres et
\item[\vref{Est 2:9}] nécessaires pour sa p., les choses dont
\item[\vref{Mc 1:44}] présente pour ta p. les choses que
\item[\vref{Lu 2:22}] jours de la p. de Marie furent
\item[\vref{Lu 5:14}] offre pour ta p. ce que Moïse
\item[\vref{Jn 2:6}] pierre destinés aux p. des Juifs, dont
\item[\vref{Jn 3:25}] Jean et les Juifs concernant la p.
\item[\vref{Ac 21:26}] quel jour lr. p. devait s'achever, et
\item[\vref{Hé 1:3}] par lui-mm la p. de nos péchés,
\item[\vref{1 Pi 3:21}] n'est pas la p. des souillures de
\item[\vref{2 Pi 1:9}] ayant oublié la p. de ses anciens
\end{listverse}

\ConcordanceEntry{Purifier}
\vspace{-2mm}
\begin{listverse}
\item[\vref{No 19:13}] qui ne se p. pas, souille le
\item[\vref{Ps 51:4}] mon iniquité et p.-moi de mon
\item[\vref{Ps 73:13}] vain que j'ai p. mon cœur et
\item[\vref{Pr 20:9}] peut dire : J'ai p. mon cœur, je
\item[\vref{Jé 13:27}] Ne seras-tu pas p. ? Jusqu'à qnd cela
\item[\vref{Ez 24:13}] car je t'avais p., et tu n'as
\item[\vref{Da 12:10}] Plusieurs seront p., blanchis et éprouvés ;
\item[\vref{Za 13:9}] et je le p. com. on purifie
\item[\vref{Mt 11:5}] les lépreux sont p., les sourds entendent,
\item[\vref{Lu 4:27}] d'eux ne fut p., si ce n'est
\item[\vref{Jn 11:55}] avant la Pâque, afin de se p.
\item[\vref{Ac 10:15}] que Dieu a p., ne les tiens
\item[\vref{Ac 15:9}] et eux, ayant p. leurs cœurs par
\item[\vref{2 Co 7:1}] de telles promesses, p.-ns. de tte
\item[\vref{Tit 2:14}] et de ns. p., pour lui être
\item[\vref{Hé 9:22}] la loi, sont p. par le sang,
\item[\vref{Ja 4:8}] doubles de cœur, p. vos cœurs.
\item[\vref{1 Jn 1:7}] Fils Jésus-Christ ns. p. de tt péché.
\item[\vref{1 Jn 3:3}] en lui, se p., com. lui aussi
\end{listverse}

\ConcordanceEntry{Python}
\vspace{-2mm}
\begin{listverse}
\item[\vref{Es 29:4}] d'un esprit de P., et ta parole
\item[\vref{Ac 16:16}] un esprit de p., et qui, en
\end{listverse}

\ConcordanceEntry{Quarante}
\vspace{-2mm}
\begin{listverse}
\item[\vref{Ex 24:18}] sur la montagne q. jours et quarante
\item[\vref{No 14:33}] ds ce désert q. ans et ils
\item[\vref{No 14:34}] qui ont été q. jours, un jour
\item[\vref{De 2:7}] toi pendant ces q. années, et tu
\item[\vref{De 9:18}] Yahweh, com. auparavant, q. jours et quarante
\item[\vref{2 S 5:4}] lorsqu'il commença à régner ; il régna q. ans.
\item[\vref{1 R 19:8}] nourriture, il marcha q. jours et quarante
\item[\vref{Ps 95:10}] en dégoût durant q. ans, et j'ai
\item[\vref{Jon 3:4}] et disait : Encore q. jours, et Ninive
\item[\vref{Mt 4:2}] Après avoir jeûné q. jours et quarante
\item[\vref{Ac 1:3}] à eux  pendant q. jours, et lr.
\end{listverse}

\ConcordanceEntry{Quart de sou}
\vspace{-2mm}
\begin{listverse}
\item[\vref{Mt 5:26}] que tu aies payé le dernier quart de sou.
\end{listverse}

\ConcordanceEntry{Querelle}
\vspace{-2mm}
\begin{listverse}
\item[\vref{Ge 13:7}] qu'il y eut q. entre les bergers
\item[\vref{Ge 26:20}] Guérar eurent une q. avec les bergers
\item[\vref{Ex 17:7}] cause de la q. des enfants d'Israël,
\item[\vref{Ex 21:18}] hommes ont une q., et que l'un
\item[\vref{Pr 6:14}] tt temps, il fait naître des q.
\item[\vref{Pr 13:10}] ne produit que q., mais la sagesse
\item[\vref{Pr 15:18}] furieux excite la q., mais l'hom. lent
\item[\vref{Pr 17:14}] Le commencement d'une q. est com. qnd
\item[\vref{Pr 18:6}] l'insensé entrent en q., et sa bouche
\item[\vref{Pr 19:13}] père, et les q. d'une fem. sont
\item[\vref{Pr 22:10}] prendra fin ; la q. et l'ignominie cesseront.
\item[\vref{Pr 26:20}] de rapporteurs les q. s'apaisent.
\item[\vref{Pr 30:33}] qui provoque la colère excite la q.
\item[\vref{Es 41:11}] hommes qui ont q. avec toi périront.
\item[\vref{Ro 1:29}] de meurtre, de q., de fraude, de
\item[\vref{Col 3:13}] si l'un a q. contre l'autre, com.
\item[\vref{2 Ti 2:23}] qu'elles ne font que produire des q.
\item[\vref{Ja 4:1}] disputes et les q. ? N'est-ce pas de
\end{listverse}

\ConcordanceEntry{Question}
\vspace{-2mm}
\begin{listverse}
\item[\vref{1 S 17:30}] posa les mêmes q.. Et le peuple
\item[\vref{1 R 10:3}] à ttes ses q., et il n'y
\item[\vref{Da 1:20}] Sur ttes les q. savantes qui réclamaient
\item[\vref{Mt 22:46}] personne n'osa plus lui poser des q.
\item[\vref{Ac 15:2}] et les anciens, pour traiter cette q.
\item[\vref{2 Ti 2:23}] Et rejette les q. folles et qui
\end{listverse}

\ConcordanceEntry{Quitter}
\vspace{-2mm}
\begin{listverse}
\item[\vref{Ge 2:24}] C'est pourquoi l'hom. q. son père et
\item[\vref{Pr 27:10}] Ne q. point ton ami ni l'ami de
\item[\vref{Mt 8:15}] la fièvre la q. ; puis elle se
\item[\vref{Mt 19:27}] ns. avons tt q. et ns. t'avons
\item[\vref{Mt 19:29}] Et quiconque aura q. ou maisons, ou
\item[\vref{Mc 5:17}] supplier Jésus de q. lr. territoire.
\item[\vref{Ac 12:10}] ds une rue. Et subitement, l'ange q. Pierre.
\item[\vref{Ac 16:39}] les priant de q. la ville.
\item[\vref{Hé 11:27}] la foi, il q. l'Egypte, sans craindre
\item[\vref{2 Pi 1:14}] je dois la q., com. notre Seign.
\end{listverse}

\ConcordanceEntry{Rabbath, Rabba}
\vspace{-2mm}
\begin{listverse}
\item[\vref{De 3:11}] n'est-il pas ds R., ville des fils
\item[\vref{2 S 12:26}] Joab combattait contre R. qui appartenait aux
\item[\vref{Jé 49:2}] de guerre contre R. des fils d'Ammon ;
\item[\vref{Ez 21:25}] doit venir contre R. des fils d'Ammon,
\end{listverse}

\ConcordanceEntry{Rabbi}
\vspace{-2mm}
\begin{listverse}
\item[\vref{Mc 9:5}] dit à Jésus : R., il est bon
\item[\vref{Jn 1:38}] Ils lui répondirent : R., c'est-à-dire Maître, où
\end{listverse}

\ConcordanceEntry{Rabbouni}
\vspace{-2mm}
\begin{listverse}
\item[\vref{Jn 20:16}] et lui dit : R. ! C'est-à-dire, mon Maître !
\end{listverse}

\ConcordanceEntry{Rabschaké}
\vspace{-2mm}
\begin{listverse}
\item[\vref{2 R 19:4}] les paroles de R., que le roi
\item[\vref{Es 36:12}] Et R. répondit : Mon maître m'a-t-il envoyé vers
\end{listverse}

\ConcordanceEntry{Raca}
\vspace{-2mm}
\begin{listverse}
\item[\vref{Mt 5:22}] à son frère : R. ! sera puni par
\end{listverse}

\ConcordanceEntry{Race}
\vspace{-2mm}
\begin{listverse}
\item[\vref{Ge 7:3}] d'en conserver la r. sur tte la
\item[\vref{Ge 19:32}] ns. conservions la r. de notre père.
\item[\vref{Pr 30:11}] y a une r. de gens qui
\item[\vref{Mt 17:17}] et dit : Ô r. incrédule et perverse,
\item[\vref{Mt 23:33}] Serpents, r. de vipères ! Comment
\item[\vref{Lu 3:7}] baptisés par lui : R. de vipères, qui
\item[\vref{Ac 17:28}] poètes : De lui ns. sommes la r.
\item[\vref{Ac 17:29}] étant de la r. de Dieu, ns.
\item[\vref{Ph 3:5}] jour, de la r. d'Israël, de la
\item[\vref{1 Pi 2:9}] vs. êtes la r. élue, vs. êtes
\end{listverse}

\ConcordanceEntry{Rachat}
\vspace{-2mm}
\begin{listverse}
\item[\vref{Ex 30:12}] à Yahweh le r. de sa personne,
\item[\vref{Ru 2:20}] ont sur ns. le droit de r.
\item[\vref{Ru 3:9}] car tu as le droit de r.
\item[\vref{Ps 49:9}] Car le r. de lr. âme est trop considérable,
\item[\vref{Jé 32:7}] le droit de r. pour l'acquérir.
\end{listverse}

\ConcordanceEntry{Rachel}
\vspace{-2mm}
\begin{listverse}
\item[\vref{Ge 29:16}] L'aînée s'appelait Léa, et la cadette R.
\item[\vref{Ge 29:18}] Jacob aimait R., et il dit :
\item[\vref{Ge 30:22}] se souvint de R., il l'exauça et
\item[\vref{Ge 30:25}] arriva qu'après que R. eut enfanté Joseph,
\item[\vref{Ge 35:24}] Les fils de R. : Joseph et Benjamin.
\item[\vref{Jé 31:15}] des larmes amères ; R. pleure ses fils ;
\item[\vref{Mt 2:18}] des grands gémissements : R. pleure ses enfants,
\end{listverse}

\ConcordanceEntry{Racheté (un)}
\vspace{-2mm}
\begin{listverse}
\item[\vref{Ps 107:2}] Qu'ainsi disent les r. de Yahweh, ceux
\item[\vref{Es 35:9}] trouvera ; mais les r. y marcheront.
\item[\vref{Es 51:10}] afin que les r. y passent ?
\item[\vref{Es 62:12}] peuple saint, les r. de Yahweh ; et
\item[\vref{Es 63:4}] l'année de mes r. est venue.
\end{listverse}

\ConcordanceEntry{Racheter}
\vspace{-2mm}
\begin{listverse}
\item[\vref{De 7:8}] et vs. a r. de la maison
\item[\vref{Ru 4:4}] tu veux la r. par droit de
\item[\vref{Né 1:10}] que tu as r. par ta grande
\item[\vref{Job 33:28}] Mais Dieu a r. mon âme afin
\item[\vref{Ps 49:8}] ne peuvent se r. l'un l'autre ni
\item[\vref{Es 1:27}] Sion sera r. par la droiture
\item[\vref{Es 44:22}] reviens à moi, car je t'ai r.
\item[\vref{Es 52:3}] vs. serez aussi r. sans argent.
\item[\vref{Es 63:9}] lui-mm les a r. ds son amour
\item[\vref{1 Co 7:23}] Vous avez été r. à un grand
\item[\vref{Ga 3:13}] Christ ns. a r. de la malédiction
\item[\vref{Ep 5:16}] r. le temps, car les jours sont
\item[\vref{Col 4:5}] du dehors, et r. le temps.
\item[\vref{Tit 2:14}] afin de ns. r. de tte iniquité,
\item[\vref{1 Pi 1:18}] vs. avez été r. de votre vaine
\item[\vref{2 Pi 2:1}] qui les a r., attireront sur eux-mêmes
\item[\vref{Ap 5:9}] tu ns. as r. pour Dieu par
\item[\vref{Ap 14:4}] Ils ont été r. d'entre les hommes
\end{listverse}

\ConcordanceEntry{Racine}
\vspace{-2mm}
\begin{listverse}
\item[\vref{De 29:18}] parmi vs. de r. qui produise du
\item[\vref{Ps 80:10}] as fait prendre r., et elle a
\item[\vref{Pr 12:3}] méchanceté, mais la r. des justes ne
\item[\vref{Es 11:1}] et un rejeton naîtra de ses r.
\item[\vref{Es 27:6}] que Jacob prendra r., Israël fleurira, et
\item[\vref{Ez 31:7}] parce que sa r. était sur de
\item[\vref{Mt 3:10}] mise à la r. des arbres ; c'est
\item[\vref{Mt 13:6}] sécha parce qu'elle n'avait pas de r.
\item[\vref{Mc 4:17}] n'ont pas de r. en eux-mêmes, ils
\item[\vref{Mc 11:20}] disciples virent le figuier séché jusqu'aux r.
\item[\vref{Lu 3:9}] mise à la r. des arbres ; tt
\item[\vref{Lu 8:13}] n'ont pas de r., ils croient pour
\item[\vref{Ro 11:18}] qui portes la r., mais c'est la
\item[\vref{1 Ti 6:10}] l'argent est la r. de ts les
\item[\vref{Hé 12:15}] à ce qu'aucune r. d'amertume, poussant des
\end{listverse}

\ConcordanceEntry{Raconter}
\vspace{-2mm}
\begin{listverse}
\item[\vref{Ge 24:66}] Le serviteur r. à Isaac ttes
\item[\vref{Ex 4:28}] Et Moïse r. à Aaron ttes les paroles de
\item[\vref{Ru 2:19}] reconnue ! Et elle r. à sa belle-mère
\item[\vref{Ps 9:2}] cœur Yahweh, je r. ttes tes merveilles.
\item[\vref{Ps 19:2}] Les cieux r. la gloire de
\item[\vref{Ps 26:7}] reconnaissance, et pour r. ttes tes merveilles.
\item[\vref{Ps 48:14}] palais pour le r. à la génération
\item[\vref{Ps 66:16}] écoutez, et je r. ce qu'il a
\item[\vref{Ps 71:16}] Seign. Yahweh ; je r. ta seule justice.
\item[\vref{Ps 118:17}] vivrai et je r. les œuvres de
\item[\vref{Ps 119:13}] mes lèvres je r. ttes les ordonnances
\item[\vref{Ps 145:6}] redoutable, et je r. ta grandeur.
\item[\vref{Mc 14:9}] monde entier, on r. aussi en mémoire
\item[\vref{Ac 14:27}] l'église, et ils r. ttes les choses
\end{listverse}

\ConcordanceEntry{Rafraîchir}
\vspace{-2mm}
\begin{listverse}
\item[\vref{Ca 2:17}] le jour se r. et que les
\item[\vref{Lu 16:24}] l'eau et me r. la langue ; car
\end{listverse}

\ConcordanceEntry{Rafraîchissement}
\vspace{-2mm}
\begin{listverse}
\item[\vref{Pr 3:8}] nombril et un r. pour tes os.
\item[\vref{Ac 3:20}] des temps de r. viennent par la
\item[\vref{Ac 9:31}] s'accroissaient par le r. du Saint-Esprit.
\end{listverse}

\ConcordanceEntry{Rahab}
\vspace{-2mm}
\begin{listverse}
\item[\vref{Jos 2:1}] fem. prostituée, nommée R., et ils y
\item[\vref{Jos 6:17}] y sont ; seulement R., la prostituée, vivra,
\item[\vref{Mt 1:5}] engendra Boaz, de R. ; Boaz engendra Obed,
\item[\vref{Hé 11:31}] Par la foi, R., la prostituée, ne
\item[\vref{Ja 2:25}] Pareillement, R., la prostituée, ne
\end{listverse}

\ConcordanceEntry{Raidir}
\vspace{-2mm}
\begin{listverse}
\item[\vref{De 2:30}] son esprit, et r. son cœur afin
\item[\vref{De 10:16}] et vs. ne r. plus votre cou.
\item[\vref{2 R 17:14}] n'écoutèrent point et r. lr. cou, com.
\item[\vref{2 Ch 30:8}] Maintenant, ne r. pas votre cou
\item[\vref{2 Ch 36:13}] de Dieu. Il r. son cou, et
\item[\vref{Né 9:16}] s'élevèrent orgueilleusement et r. lr. cou. Ils
\item[\vref{Pr 29:1}] qui étant repris, r. son cou, sera
\item[\vref{Jé 7:26}] mais ils ont r. lr. cou, ils
\end{listverse}

\ConcordanceEntry{Raillerie}
\vspace{-2mm}
\begin{listverse}
\item[\vref{De 28:37}] de proverbes, de r., parmi ts les
\item[\vref{Jé 24:9}] en proverbe, en r., et en malédiction,
\item[\vref{Joë 2:17}] un sujet de r. ! Pourquoi dirait-on parmi
\item[\vref{Mi 6:16}] un objet de r., et vs. porterez
\end{listverse}

\ConcordanceEntry{Raisin}
\vspace{-2mm}
\begin{listverse}
\item[\vref{Ge 40:10}] développa et ses grappes donnèrent des r. mûrs.
\item[\vref{No 6:3}] d'aucune liqueur de r., et il ne
\item[\vref{De 32:32}] de Gomorrhe ; leurs r. sont des raisins
\item[\vref{Ca 2:5}] des gâteaux de r., fortifiez-moi avec des
\item[\vref{Es 5:2}] qu'elle produirait des r., mais elle a
\item[\vref{Es 18:5}] fleur devient un r. qui mûrit, il
\item[\vref{Jé 8:13}] aura plus de r. à la vigne,
\item[\vref{Jé 31:29}] ont mangé des r. verts, et les
\item[\vref{Ez 18:2}] ont mangé des r. verts et les
\item[\vref{Os 9:10}] Israël com. des r. ds le désert ;
\item[\vref{Am 9:13}] qui foule les r. atteindra celui qui
\item[\vref{Mt 7:16}] fruits. Cueille-t-on des r. sur des épines,
\item[\vref{Ap 14:18}] terre, car ses r. sont mûrs.
\end{listverse}

\ConcordanceEntry{Raison}
\vspace{-2mm}
\begin{listverse}
\item[\vref{Ge 27:36}] N'est-ce pas avec r. qu'on a appelé
\item[\vref{2 S 13:16}] Tu n'as aucune r. de me faire
\item[\vref{Job 36:3}] de loin mes r., et je défendrai
\item[\vref{Pr 11:22}] détourne de la r. est com. un
\item[\vref{Pr 12:8}] est estimé en r. de sa prudence,
\item[\vref{Da 4:36}] ce temps, la r. me revint, et
\item[\vref{Mt 6:30}] à plus forte r., ô gens de
\item[\vref{Mt 7:11}] combien plus forte r. votre Père qui
\item[\vref{Mt 10:25}] combien plus forte r. appelleront-ils ainsi ses
\item[\vref{Lu 8:47}] peuple pour quelle r. elle l'avait touché,
\item[\vref{Lu 12:28}] combien plus forte r. vs. vêtira-t-il, ô
\item[\vref{Jn 8:48}] répondirent : N'avons-ns. pas r. de dire que
\item[\vref{Jn 9:23}] Pour cette r. son père et
\item[\vref{Ro 5:9}] A plus forte r. dc, étant mntnt
\item[\vref{Ro 11:24}] à plus forte r. eux seront-ils greffés
\end{listverse}

\ConcordanceEntry{Raisonnement}
\vspace{-2mm}
\begin{listverse}
\item[\vref{Job 32:11}] l'oreille à vos r. jusqu'à ce que
\item[\vref{1 Co 3:20}] Seign. connaît les r. des sages, ils
\item[\vref{2 Co 10:5}] détruisant les r. et tte hauteur
\end{listverse}

\ConcordanceEntry{Rajeunir}
\vspace{-2mm}
\begin{listverse}
\item[\vref{Job 33:25}] ds son enfance ; et il sera r.
\end{listverse}

\ConcordanceEntry{Rama}
\vspace{-2mm}
\begin{listverse}
\item[\vref{1 S 1:19}] lr. maison à R.. Elkana connut Anne,
\item[\vref{Jé 31:15}] des cris à R., des lamentations, des
\item[\vref{Mt 2:18}] a entendu à R. des cris, des
\end{listverse}

\ConcordanceEntry{Ramasser}
\vspace{-2mm}
\begin{listverse}
\item[\vref{Ru 2:7}] et que je r. quelques gerbes, après
\item[\vref{1 R 17:10}] était là, qui r. du bois. Et
\item[\vref{Ec 3:5}] un temps pour r. des pierres ; un
\item[\vref{Mi 2:12}] Jacob ! Et je r. entièrement le reste
\item[\vref{Mt 13:47}] la mer et r. ttes sortes de
\item[\vref{Lu 15:13}] fils, ayant tt r., partit pour un
\item[\vref{Jn 6:13}] Ils les r. dc, et ils remplirent douze paniers
\item[\vref{Ac 28:3}] Paul ayant r. un tas de
\end{listverse}

\ConcordanceEntry{Rameau}
\vspace{-2mm}
\begin{listverse}
\item[\vref{Es 11:1}] il sortira un r. du tronc d'Isaï,
\item[\vref{Mal 4:1}] ne lr. laissera ni racine ni r.
\item[\vref{Mt 21:8}] autres coupaient des r. des arbres, et
\end{listverse}

\ConcordanceEntry{Ramener}
\vspace{-2mm}
\begin{listverse}
\item[\vref{Ge 14:16}] Et il r. ts les biens qu'ils avaient pris ;
\item[\vref{Ge 24:6}] Garde-toi bien d'y r. mon fils !
\item[\vref{No 22:23}] l'ânesse pour la r. ds le chemin.
\item[\vref{1 S 6:21}] Les Philistins ont r. l'arche de Yahweh ;
\item[\vref{1 R 13:20}] Yahweh vint au prophète qui l'avait r.
\item[\vref{Né 9:26}] avertissaient pour les r. à toi, et
\item[\vref{Ps 85:2}] terre, tu as r. et mis en
\item[\vref{Ps 126:1}] degrés. Quand Yahweh r. les captifs de
\item[\vref{Pr 26:15}] fatigant de la r. à sa bouche.
\item[\vref{Es 49:6}] Jacob et pour r. les restes d'Israël ;
\item[\vref{Ez 34:4}] vs. n'avez point r. celle qui était
\item[\vref{Ez 47:1}] Puis il me r. vers l'entrée de
\item[\vref{Lu 1:17}] puissance d'Elie, pour r. les cœurs des
\item[\vref{Lu 5:11}] qnd ils eurent r. les barques à
\item[\vref{Ac 20:12}] Ils r. le jeune hom. vivant, et ce
\item[\vref{Ga 2:4}] afin de ns. r. ds la servitude.
\item[\vref{Hé 13:20}] paix, qui a r. d'entre les morts
\end{listverse}

\ConcordanceEntry{Ramer}
\vspace{-2mm}
\begin{listverse}
\item[\vref{Jon 1:13}] Et ces hommes r. pour revenir sur
\item[\vref{Mc 6:48}] de peine à r. parce que le
\end{listverse}

\ConcordanceEntry{Rançon}
\vspace{-2mm}
\begin{listverse}
\item[\vref{Job 36:18}] pas alors de r. si grande qui
\item[\vref{Pr 21:18}] méchant sert de r. pour le juste,
\item[\vref{Mc 10:45}] sa vie en r. pour plusieurs.
\item[\vref{1 Ti 2:6}] donné lui-mm en r. pour ts. C'est
\item[\vref{Hé 9:15}] intervenue pour la r. des transgressions commises
\end{listverse}

\ConcordanceEntry{Ranimer}
\vspace{-2mm}
\begin{listverse}
\item[\vref{Ca 2:5}] R.-moi avec des gâteaux de raisins,
\item[\vref{2 Ti 1:6}] je t'exhorte à r. le don de
\end{listverse}

\ConcordanceEntry{Rapine}
\vspace{-2mm}
\begin{listverse}
\item[\vref{Es 61:8}] qui hait la r. et l'iniquité ; j'établirai
\item[\vref{Mt 23:25}] sont pleins de r. et d'intempérance.
\item[\vref{Lu 11:39}] êtes pleins de r. et de méchanceté.
\end{listverse}

\ConcordanceEntry{Rappeler}
\vspace{-2mm}
\begin{listverse}
\item[\vref{Ps 42:5}] Je r. ces choses ds mon souvenir, en
\item[\vref{Es 12:4}] parmi les peuples, r. que son Nom
\item[\vref{Mt 16:9}] et ne vs. r.-vs. plus les
\item[\vref{Ac 20:35}] faibles, et se r. les paroles du
\item[\vref{Hé 10:32}] Or r.-vs. des premiers jours, où, après
\item[\vref{2 Pi 1:12}] pas de vs. r. sans cesse ces
\item[\vref{Jud 1:5}] je veux vs. r. une chose, que
\end{listverse}

\ConcordanceEntry{Rapporter}
\vspace{-2mm}
\begin{listverse}
\item[\vref{Ge 3:16}] tes désirs se r. vers ton mari,
\item[\vref{Ex 14:5}] Or on avait r. au roi d'Egypte
\item[\vref{No 11:24}] dc sortit et r. au peuple les
\item[\vref{1 S 3:15}] Samuel craignait de r. cette vision à
\item[\vref{Job 1:15}] échappé moi seul, pour te le r.
\item[\vref{Pr 11:13}] Celui qui va r., révèle les secrets,
\item[\vref{Pr 17:9}] mais celui qui r. la chose divise
\item[\vref{Ec 10:20}] le Baal ailé r. tes paroles.
\item[\vref{Es 21:6}] sentinelle, et qu'elle r. ce qu'elle verra.
\item[\vref{Mt 11:4}] dit : Allez et r. à Jean les
\item[\vref{Mt 27:3}] se repentit et r. les trente pièces
\item[\vref{Mc 4:8}] qu'un grain en r. trente, un autre
\item[\vref{Mc 14:70}] es Galiléen, et ton langage s'y r.
\item[\vref{Lu 14:21}] serviteur, de retour, r. ces choses à
\item[\vref{Jn 5:15}] s'en alla, et r. aux Juifs que
\item[\vref{Ac 16:36}] Et le geôlier r. ces paroles à
\item[\vref{1 Jn 5:8}] le sang, et ces trois-là se r. à un.
\end{listverse}

\ConcordanceEntry{Rapporteur}
\vspace{-2mm}
\begin{listverse}
\item[\vref{Pr 16:28}] querelles, et le r. divise les grands
\item[\vref{Pr 18:8}] Les paroles du r. sont com. des
\item[\vref{Pr 26:20}] a plus de r. les querelles s'apaisent.
\item[\vref{Pr 26:22}] Les paroles du r. sont com. des
\item[\vref{Ro 1:30}] r., médisants, haïssant Dieu, outrageux, orgueilleux, vains,
\end{listverse}

\ConcordanceEntry{Rare}
\vspace{-2mm}
\begin{listverse}
\item[\vref{1 S 3:1}] de Yahweh était r. en ce temps-là,
\end{listverse}

\ConcordanceEntry{Raser}
\vspace{-2mm}
\begin{listverse}
\item[\vref{Lé 14:8}] lavera ses vêtements, r. tt son poil,
\item[\vref{Jg 16:17}] mère. Si j'étais r., ma force partirait,
\item[\vref{Jg 16:19}] un hom., elle r. les sept tresses
\item[\vref{Ac 18:18}] après s'être fait r. la tête à
\item[\vref{1 Co 11:6}] coupés, ou d'être r., qu'elle se voile !
\end{listverse}

\ConcordanceEntry{Rassasier}
\vspace{-2mm}
\begin{listverse}
\item[\vref{Ge 25:8}] fort âgé et r. de jours, et
\item[\vref{De 11:15}] bétail, tu mangeras et tu seras r.
\item[\vref{Job 42:17}] mourut âgé et r. de jours.
\item[\vref{Ps 22:27}] mangeront et seront r., ceux qui cherchent
\item[\vref{Ps 36:9}] Ils seront abondamment r. de la graisse
\item[\vref{Ps 37:19}] mais ils sont r. au jour de
\item[\vref{Ps 63:6}] Mon âme est r. com. de mets
\item[\vref{Ps 65:5}] parvis ! Nous serons r. des biens de
\item[\vref{Ps 90:14}] R.-ns. chaque matin de ta bonté,
\item[\vref{Ps 91:16}] Je le r. de jours, et je lui ferai
\item[\vref{Ps 103:5}] qui r. ta bouche de biens ; ta jeunesse
\item[\vref{Pr 12:11}] son champ sera r. de pain, mais
\item[\vref{Pr 28:25}] qui se confie en Yahweh sera r.
\item[\vref{Ec 1:8}] l'œil n'est jamais r. de voir et
\item[\vref{Ec 5:9}] l'argent n'est point r. par l'argent, et
\item[\vref{Es 1:11}] sacrifices ? Je suis r. des holocaustes de
\item[\vref{Es 53:11}] et en sera r. ; mon serviteur juste
\item[\vref{Es 55:2}] ce qui ne r. pas ? Ecoutez-moi attentivement,
\item[\vref{Es 58:10}] faim, si tu r. l'âme affligée ; ta
\item[\vref{Joë 2:19}] vs. en serez r. ; et je ne
\item[\vref{Mi 6:14}] ne seras pas r., et la faim
\item[\vref{Ha 2:16}] Tu seras r. de honte plutôt que de gloire ;
\item[\vref{Ag 1:6}] pas jusqu'à être r.. Vous avez bu,
\item[\vref{Mt 5:6}] de la justice, car ils seront r. !
\item[\vref{Mt 14:20}] mangèrent et furent r., et l'on emporta
\item[\vref{Lu 1:53}] Il a r. de biens les affamés, et il
\item[\vref{Lu 6:25}] vs. qui êtes r., car vs. aurez
\item[\vref{1 Co 4:8}] Vous êtes déjà r., vs. êtes déjà
\item[\vref{Ja 2:16}] paix, chauffez-vs., et r.-vs. ! Et que
\item[\vref{Ap 19:21}] les oiseaux furent r. de lr. chair.
\end{listverse}

\ConcordanceEntry{Rassembler}
\vspace{-2mm}
\begin{listverse}
\item[\vref{Ge 1:9}] des cieux soient r. en un lieu,
\item[\vref{Ge 49:2}] R.-vs., et écoutez, fils de Jacob ;
\item[\vref{De 30:3}] toi ; il te r. encore du milieu
\item[\vref{Ps 50:5}] R.-moi mes bien-aimés, qui ont traité
\item[\vref{Es 11:12}] les nations, il r. les exilés d'Israël
\item[\vref{Es 34:16}] ordonné, et son Esprit qui les r.
\item[\vref{Jé 32:37}] je vais les r. de ts les
\item[\vref{Mi 4:6}] dit Yahweh, je r. les boiteux, je
\item[\vref{So 3:18}] Je r. ceux qui sont tristes à cause
\item[\vref{Mt 23:37}] fois ai-je voulu r. tes enfants, com.
\item[\vref{Mt 24:31}] trompette, et ils r. ses élus, des
\item[\vref{Mc 13:27}] anges, et il r. ses élus des
\item[\vref{Ap 20:8}] afin de les r. pour la guerre,
\end{listverse}

\ConcordanceEntry{Ravage}
\vspace{-2mm}
\begin{listverse}
\item[\vref{Ps 46:9}] et voyez quels r. il a fait
\item[\vref{Ps 91:3}] de la peste et de ses r.
\item[\vref{Es 13:6}] vient com. un r. du Tout-Puissant.
\item[\vref{Es 59:7}] pensées d'iniquité ; le r. et la ruine
\item[\vref{Es 60:18}] pays ni de r. et de ruine
\item[\vref{Da 8:24}] il fera d'incroyables r., il réussira ds
\end{listverse}

\ConcordanceEntry{Ravager}
\vspace{-2mm}
\begin{listverse}
\item[\vref{Ge 49:19}] l'attaquer, mais il r. lr. arrière-garde.
\item[\vref{Ps 91:3}] de la peste et de ses r.
\item[\vref{Es 14:20}] car tu as r. ta terre, tu
\item[\vref{Jé 12:10}] Plusieurs pasteurs r. ma vigne, ils
\item[\vref{Am 7:9}] lieux d'Isaac seront r., et les sanctuaires
\item[\vref{Mi 5:5}] Ils r. le pays d'Assyrie avec l'épée, et
\item[\vref{Mi 6:13}] frappant, et te r. à cause de
\item[\vref{Na 2:11}] on dévaste, on r. ! Et les cœurs
\item[\vref{Na 3:16}] répandues, ont tt r., et puis se
\item[\vref{Ha 2:17}] toi ; et les r. des bêtes t'effrayeront,
\item[\vref{Za 11:6}] lr. roi ; ils r. le pays, et
\item[\vref{Mal 3:11}] il ne vs. r. pas les fruits
\item[\vref{Ac 8:3}] Mais Saul r. l'église, entrant ds
\end{listverse}

\ConcordanceEntry{Ravir}
\vspace{-2mm}
\begin{listverse}
\item[\vref{Ca 4:9}] Tu me r. le cœur, ma sœur, mon épouse,
\item[\vref{Mt 13:19}] malin vient et r. ce qui est
\item[\vref{Jn 10:28}] personne ne les r. de ma main.
\item[\vref{Ac 22:17}] temple, je fus r. en extase,
\item[\vref{2 Co 12:2}] qui a été r. jusqu'au troisième ciel.
\item[\vref{Ap 1:10}] Je fus r. en esprit au jour du Seign.,
\item[\vref{Ap 4:2}] Aussitôt, je fus r. en esprit. Et
\end{listverse}

\ConcordanceEntry{Rayonner}
\vspace{-2mm}
\begin{listverse}
\item[\vref{Ex 34:35}] de son visage r.. C'est pourquoi Moïse
\end{listverse}

\ConcordanceEntry{Rebâtir}
\vspace{-2mm}
\begin{listverse}
\item[\vref{Es 44:28}] Jérus. : Tu seras r. ! Et au temple :
\item[\vref{Es 58:12}] sortiront de toi r. les lieux déserts
\item[\vref{Es 61:4}] Et ils r. les ruines antiques, ils relèveront les
\item[\vref{Da 9:25}] que Jérus. sera r. jusqu'au Messie, le
\item[\vref{Am 9:11}] et je le r. com. il était
\item[\vref{Ag 1:2}] le temps de r. la maison de
\item[\vref{Za 1:16}] maison y sera r., dit Yahweh des
\item[\vref{Mt 26:61}] Dieu et le r. en trois jours.
\item[\vref{Mc 14:58}] trois jours, j'en r. un autre qui
\item[\vref{Ga 2:18}] Car si je r. les choses que
\end{listverse}

\ConcordanceEntry{Rebecca}
\vspace{-2mm}
\begin{listverse}
\item[\vref{Ge 22:23}] Bethuel a engendré R.. Milca enfanta ces
\item[\vref{Ge 25:20}] qnd il épousa R., fille de Bethuel,
\item[\vref{Ge 26:8}] qui plaisantait avec R., sa fem.
\item[\vref{Ge 27:6}] Et R. parla à Jacob, son fils, et
\item[\vref{Ge 27:46}] R. dit à Isaac : Je suis dégoûtée
\end{listverse}

\ConcordanceEntry{Rebelle}
\vspace{-2mm}
\begin{listverse}
\item[\vref{No 14:9}] ne soyez point r. contre Yahweh, et
\item[\vref{No 20:10}] dit : Ecoutez dc, r. ! Est-ce de ce
\item[\vref{Esd 4:12}] bâtissent la ville r. et méchante, et
\item[\vref{Ps 68:7}] enchaînés, mais les r. habitent sur une
\item[\vref{Ps 68:19}] mm parmi les r., afin qu'ils habitent
\item[\vref{Ps 105:28}] ne furent point r. à sa parole.
\item[\vref{Ps 106:7}] mais ils furent r. près de la
\item[\vref{Es 57:17}] colère ; et le r. a suivi la
\item[\vref{Ez 2:5}] sont une maison r., ils sauront pourtant
\item[\vref{Da 9:9}] ns. avons été r. envers lui.
\item[\vref{Os 9:15}] plus ; ts leurs chefs sont des r.
\item[\vref{Ac 17:5}] Mais les Juifs, r. et jaloux, prirent
\item[\vref{Ac 19:9}] restaient endurcis et r., décriant dvt la
\item[\vref{Ro 1:30}] inventeurs de maux, r. à leurs parents,
\item[\vref{Ep 2:2}] mntnt avec efficacité ds les fils r. à Dieu,
\item[\vref{2 Ti 3:2}] fanfarons, orgueilleux, blasphémateurs, r. à leurs parents,
\item[\vref{Tit 1:16}] ils sont abominables, r., et réprouvés à
\item[\vref{Hé 3:18}] repos, sinon à ceux qui furent r. ?
\item[\vref{1 Pi 2:7}] par rapport aux r., il est dit :
\end{listverse}

\ConcordanceEntry{Rébellion}
\vspace{-2mm}
\begin{listverse}
\item[\vref{No 14:18}] et pardonne la r., mais il ne
\item[\vref{De 31:27}] je connais ta r. et ton cou
\item[\vref{Jos 24:19}] pardonnera point votre r. et vos péchés.
\item[\vref{1 S 15:23}] Car la r. est un péché autant que la
\item[\vref{Os 14:4}] Je guérirai lr. r., je les aimerai
\item[\vref{Ro 11:32}] renfermés sous la r., afin de faire
\item[\vref{Ep 5:6}] vient sur les fils de la r.
\item[\vref{Col 3:6}] vient sur les fils de la r.,
\item[\vref{Hé 4:6}] pas entrés à cause de lr. r.,
\item[\vref{Jud 1:11}] péri par une r. semblable à celle
\end{listverse}

\ConcordanceEntry{Réca, Récabites}
\vspace{-2mm}
\begin{listverse}
\item[\vref{1 Ch 4:12}] ce sont là les gens de R.
\item[\vref{Jé 35:18}] la maison des R.: Ainsi parle Yahweh
\end{listverse}

\ConcordanceEntry{Recensement}
\vspace{-2mm}
\begin{listverse}
\item[\vref{Lu 2:2}] Ce premier r. eut lieu pendant
\item[\vref{Ac 5:37}] au temps du r., et il attira
\end{listverse}

\ConcordanceEntry{Recevoir}
\vspace{-2mm}
\begin{listverse}
\item[\vref{Ge 4:11}] sa bouche pour r. de ta main
\item[\vref{Ge 40:5}] le sien, pouvant r. une explication distincte.
\item[\vref{De 33:3}] tes pieds pour r. tes paroles.
\item[\vref{Ps 73:24}] et tu me r. ds la gloire.
\item[\vref{Pr 1:3}] pour r. une leçon de bon sens, de
\item[\vref{Jé 17:23}] et ne pas r. d'instruction.
\item[\vref{Os 14:2}] nos iniquités, et r. le bien, pour
\item[\vref{Mt 10:40}] Celui qui vs. r. me reçoit, et
\item[\vref{Mt 18:5}] Et quiconque r. en mon Nom
\item[\vref{Mt 20:10}] venus, ils croyaient r. davantage, mais ils
\item[\vref{Mt 21:22}] Dieu, si vs. croyez, vs. le r.
\item[\vref{Mt 21:34}] les vignerons pour r. les fruits.
\item[\vref{Lu 6:34}] qui vs. espérez r., quel gré vs.
\item[\vref{Jn 1:12}] ceux qui l'ont r., à ceux qui
\item[\vref{Jn 3:27}] L'hom. ne peut r. aucune chose si
\item[\vref{Jn 7:39}] l'Esprit que devaient r. ceux qui croiraient
\item[\vref{Jn 12:48}] et qui ne r. pas mes paroles
\item[\vref{Jn 14:17}] monde ne peut r., parce qu'il ne
\item[\vref{Ac 3:5}] attentivement, s'attendant à r. qq chose d'eux.
\item[\vref{Ac 8:17}] mains, et ils r. le Saint-Esprit.
\item[\vref{Ac 20:35}] plus de bénédiction à donner qu'à r.
\item[\vref{Ro 11:35}] qu'il ait à r. en retour ?
\item[\vref{1 Co 4:7}] que tu n'aies r. ? Et si tu
\item[\vref{1 Co 11:23}] Car j'ai r. du Seign. ce qu'aussi je vs.
\item[\vref{2 Co 5:10}] afin que chacun r. selon le bien
\item[\vref{2 Co 6:1}] de ne pas r. la grâce de
\item[\vref{2 Co 8:4}] grandes prières de r. la grâce et
\item[\vref{2 Co 11:4}] ou si vs. r. un autre esprit
\item[\vref{Col 2:6}] com. vs. avez r. le Seign. Jésus-Christ,
\item[\vref{Hé 1:14}] ceux qui doivent r. l'héritage du salut ?
\item[\vref{Ja 1:7}] s'attende pas à r. qq chose du
\item[\vref{Ja 1:21}] résidu de méchanceté, r. avec douceur la
\item[\vref{2 Jn 1:10}] doctrine, ne le r. pas ds votre
\item[\vref{3 Jn 1:8}] Nous devons dc r. de tels hommes,
\item[\vref{Ap 5:12}] est digne de r. puissance, richesses, sagesse,
\end{listverse}

\ConcordanceEntry{Recherche}
\vspace{-2mm}
\begin{listverse}
\item[\vref{De 19:18}] juges feront des r. avec soin. Si
\item[\vref{Jg 6:29}] et firent des r.. On lr. dit :
\item[\vref{Lu 15:4}] aller à la r. de celle qui
\item[\vref{1 Pi 1:10}] ont fait leurs r. et leurs investigations.
\end{listverse}

\ConcordanceEntry{Rechercher}
\vspace{-2mm}
\begin{listverse}
\item[\vref{2 Ch 14:6}] que ns. avons r. Yahweh, notre Dieu.
\item[\vref{2 Ch 26:5}] Il s'appliqua à r. Dieu pendant les
\item[\vref{2 Ch 34:3}] il commença à r. le Dieu de
\item[\vref{Ps 78:34}] alors ils le r. ; ils se repentaient
\item[\vref{Ps 111:2}] Daleth.] elles sont r. par ts ceux
\item[\vref{Ps 119:10}] Je te r. de tt mon cœur, ne me
\item[\vref{Pr 11:27}] Qui r. le bien cherche la faveur, mais
\item[\vref{Es 1:17}] à bien faire, r. la droiture, redressez
\item[\vref{Ez 36:37}] Je me laisserai r. par la maison
\item[\vref{Os 2:9}] ceux dont elle r. l'amitié, mais ne
\item[\vref{Os 3:5}] se repentiront ; et r. Yahweh, lr. Dieu,
\item[\vref{Am 5:14}] R. le bien et non le mal,
\item[\vref{Za 8:22}] puissantes nations viendront r. Yahweh des armées
\item[\vref{Mt 6:32}] que les Gentils r. ttes ces choses.
\item[\vref{Ro 3:11}] a personne qui r. Dieu ; ils se
\item[\vref{Ro 12:17}] pour le mal. R. les choses honnêtes
\item[\vref{1 Co 14:1}] R. la charité. Désirez avec ardeur les
\item[\vref{1 Th 5:15}] le mal ; mais r. toujours ce qui
\item[\vref{1 Ti 6:11}] ces choses, et r. la justice, la
\item[\vref{2 Ti 2:22}] la jeunesse, et r. la justice, la
\item[\vref{Hé 12:14}] R. la paix avec ts, et la
\item[\vref{1 Pi 3:11}] le bien, qu'il r. la paix, et
\end{listverse}

\ConcordanceEntry{Récolte}
\vspace{-2mm}
\begin{listverse}
\item[\vref{Ge 47:24}] temps de la r. viendra, vs. donnerez
\item[\vref{Ex 23:16}] fête de la r., après la fin
\item[\vref{Lé 25:16}] on te vend le nombre des r.
\item[\vref{Lu 20:10}] saison de la r., il envoya un
\end{listverse}

\ConcordanceEntry{Recommandation}
\vspace{-2mm}
\begin{listverse}
\item[\vref{Mc 8:15}] lr. fit cette r. : Gardez-vs. avec soin
\item[\vref{2 Co 3:1}] de lettres de r. auprès de vs.,
\end{listverse}

\ConcordanceEntry{Recommander}
\vspace{-2mm}
\begin{listverse}
\item[\vref{Est 4:8}] et il lui r. d'entrer chez le
\item[\vref{Ps 37:5}] [Guimel.] R. tes voies à
\item[\vref{Mc 7:36}] Et Jésus lr. r. de ne le
\item[\vref{Ac 15:40}] après avoir été r. à la grâce
\item[\vref{2 Co 3:1}] nouveau à ns. r. ns.-mêmes ? Ou avons-ns.
\item[\vref{2 Co 5:12}] ns. ne ns. r. pas de nouveau
\item[\vref{2 Co 10:18}] celui qui se r. lui-mm qui est
\item[\vref{Ga 2:10}] Ils ns. r. seulement de ns. souvenir des pauvres,
\item[\vref{1 Pi 4:19}] est bon, lui r. leurs âmes, com.
\end{listverse}

\ConcordanceEntry{Récompense}
\vspace{-2mm}
\begin{listverse}
\item[\vref{Ge 15:1}] ton bouclier, ta grande et infinie r.
\item[\vref{Ru 2:12}] Que Yahweh r. ton œuvre, et
\item[\vref{Ps 19:12}] les observe la r. est grande.
\item[\vref{Ps 58:12}] y a une r. pour le juste ;
\item[\vref{Ps 127:3}] le fruit du ventre est une r. de Dieu.
\item[\vref{Pr 22:4}] La r. de l'humilité et de la crainte
\item[\vref{Pr 24:14}] elle sera ta r., et ton espérance
\item[\vref{Es 1:23}] courent après les r. ; ils ne font
\item[\vref{Es 62:11}] lui, et sa r. marche dvt lui.
\item[\vref{Mt 5:12}] parce que votre r. sera grande ds
\item[\vref{Mt 6:2}] dis en vérité, ils reçoivent lr. r.
\item[\vref{Mt 10:41}] prophète recevra la r. d'un prophète, et
\item[\vref{Mt 10:42}] vérité qu'il ne perdra pas sa r.
\item[\vref{Ro 1:27}] en eux-mêmes la r. de lr. égarement
\item[\vref{1 Co 3:8}] chacun recevra sa r. selon son travail.
\item[\vref{Col 3:24}] Seign. l'héritage pour r.. Car vs. servez
\item[\vref{2 Pi 2:13}] recevront alors la r. de lr. iniquité.
\item[\vref{Jud 1:11}] Balaam, après la r., et ont péri
\item[\vref{Ap 11:18}] de donner la r. à tes serviteurs
\end{listverse}

\ConcordanceEntry{Récompenser}
\vspace{-2mm}
\begin{listverse}
\item[\vref{Ge 15:1}] ton bouclier, ta grande et infinie r.
\item[\vref{De 32:6}] ainsi que tu r. Yahweh, peuple insensé
\item[\vref{Ru 2:12}] Que Yahweh r. ton œuvre, et
\item[\vref{1 S 24:20}] Yahweh dc te r. pour la grâce
\item[\vref{2 S 5:8}] de David, sera r.… C'est pourquoi l'on
\item[\vref{2 Ch 15:7}] y a une r. pour vos œuvres.
\item[\vref{Ps 19:12}] les observe la r. est grande.
\item[\vref{Ps 58:12}] y a une r. pour le juste ;
\item[\vref{Ps 127:3}] le fruit du ventre est une r. de Dieu.
\item[\vref{Pr 13:13}] qui craint le commandement en sera r.
\item[\vref{Pr 22:4}] La r. de l'humilité et de la crainte
\item[\vref{Es 1:23}] courent après les r. ; ils ne font
\item[\vref{Es 62:11}] lui, et sa r. marche dvt lui.
\item[\vref{Mt 6:4}] le secret, te r. publiquement.
\item[\vref{Mt 10:41}] prophète recevra la r. d'un prophète, et
\item[\vref{Lu 6:23}] parce que votre r. sera grande ds
\item[\vref{Ro 11:9}] de chute, et cela pour lr. r. !
\item[\vref{1 Co 3:8}] chacun recevra sa r. selon son travail.
\item[\vref{1 Co 9:17}] j'en aurai la r. ; mais si je
\item[\vref{Col 3:24}] Seign. l'héritage pour r.. Car vs. servez
\item[\vref{Hé 10:35}] fait paraître, et qui sera bien r.
\item[\vref{2 Pi 2:13}] recevront alors la r. de lr. iniquité.
\item[\vref{Jud 1:11}] Balaam, après la r., et ont péri
\item[\vref{Ap 11:18}] de donner la r. à tes serviteurs
\end{listverse}

\ConcordanceEntry{Réconciliation}
\vspace{-2mm}
\begin{listverse}
\item[\vref{Ro 5:11}] lequel ns. avons mntnt obtenu la r.
\item[\vref{Ro 11:15}] a été la r. du monde, quelle
\item[\vref{2 Co 5:18}] a donné le service de la r.
\item[\vref{2 Co 5:19}] en ns. la parole de la r.
\end{listverse}

\ConcordanceEntry{Réconcilier}
\vspace{-2mm}
\begin{listverse}
\item[\vref{Mt 5:24}] et va te r. d'abord avec ton
\item[\vref{Ro 5:10}] ns. avons été r. avec Dieu par
\item[\vref{1 Co 7:11}] ou qu'elle se r. avec son mari ;
\item[\vref{2 Co 5:18}] qui ns. a r. avec lui par
\item[\vref{2 Co 5:19}] était en Christ, r. le monde avec
\item[\vref{2 Co 5:20}] Christ de vs. r. avec Dieu !
\item[\vref{Ep 2:16}] et de r. les uns et les autres avec
\item[\vref{Col 1:20}] Et de r. par lui ttes choses avec lui
\item[\vref{Col 1:21}] mauvaises œuvres, il vs. a mntnt r.
\end{listverse}

\ConcordanceEntry{Reconnaître}
\vspace{-2mm}
\begin{listverse}
\item[\vref{Ge 41:31}] forte qu'on ne r. plus de cette
\item[\vref{De 4:35}] afin que tu r. que Yahweh est
\item[\vref{Jos 3:10}] Josué dit : Vous r. à ceci que
\item[\vref{Ps 139:14}] mon âme le r. très bien.
\item[\vref{Es 61:9}] qui les verront r. qu'ils sont la
\item[\vref{Os 2:10}] elle n'a pas r. que c'était moi
\item[\vref{Mi 6:9}] et le sage r. son Nom. Ecoutez
\item[\vref{Mt 7:20}] Vous les r. dc à leurs fruits.
\item[\vref{Mt 17:12}] ne l'ont pas r., et ils lui
\item[\vref{Lu 1:18}] l'ange : A quoi r.-je cela ? Car
\item[\vref{Lu 2:12}] signe vs. le r. : Vous trouverez le
\item[\vref{Lu 6:44}] chaque arbre se r. à son fruit.
\item[\vref{Lu 24:31}] et ils le r. ; mais il disparut
\item[\vref{Ac 4:13}] étonnaient, et ils r. bien qu'ils avaient
\item[\vref{Ro 7:16}] veux pas, je r. par cela mm
\item[\vref{1 Co 14:7}] sons distincts, comment r.-t-on ce qui
\item[\vref{2 Ti 2:25}] la repentance pour r. la vérité,
\item[\vref{Hé 12:6}] ts ceux qu'il r. pour ses fils.
\end{listverse}

\ConcordanceEntry{Reculer}
\vspace{-2mm}
\begin{listverse}
\item[\vref{Ps 44:11}] Tu ns. fais r. dvt l'adversaire, et
\item[\vref{Jn 18:6}] JE SUIS, ils r. et tombèrent par
\end{listverse}

\ConcordanceEntry{Rédempteur}
\vspace{-2mm}
\begin{listverse}
\item[\vref{Job 19:25}] sais que mon R. est vivant, et
\item[\vref{Ps 19:15}] ô Yahweh ! mon Rocher et mon R. !
\item[\vref{Pr 23:11}] car lr. r. est puissant, il défendra lr. cause
\item[\vref{Es 47:4}] Quant à notre R., son Nom est
\item[\vref{Es 49:26}] ton Sauveur, ton R., le Puissant de
\item[\vref{Es 54:5}] Nom ; et ton R. est le Saint
\item[\vref{Es 59:20}] Et le R. viendra en Sion, et vers ceux
\item[\vref{Es 63:16}] Nom est notre R. de tt temps.
\item[\vref{Jé 50:34}] Leur R. est fort, son Nom est Yahweh
\end{listverse}

\ConcordanceEntry{Rédemption}
\vspace{-2mm}
\begin{listverse}
\item[\vref{Ps 111:9}] a envoyé la r. à son peuple ; [
\item[\vref{Ps 130:7}] miséricordieux et la r. est auprès de
\item[\vref{Ro 3:24}] grâce, par la r. qui est en
\item[\vref{Ro 8:23}] l'adoption, c'est-à-dire la r. de notre corps.
\item[\vref{1 Co 1:30}] de Dieu, sagesse, justice, sanctification et r.,
\item[\vref{Ep 1:7}] ns. avons la r. par son sang,
\item[\vref{Ep 1:14}] héritage jusqu'à la r. de ceux qu'il
\item[\vref{Ep 4:30}] scellés pour le jour de la r.
\item[\vref{Col 1:14}] ns. avons la r. par son sang,
\item[\vref{Hé 9:12}] avoir obtenu une r. éternelle.
\end{listverse}

\ConcordanceEntry{Redoutable}
\vspace{-2mm}
\begin{listverse}
\item[\vref{Ex 34:10}] faire avec toi sera une chose r.
\item[\vref{De 10:17}] Puissant et le R., qui n'a point
\item[\vref{De 28:58}] Nom glorieux et r. de Yahweh, ton
\item[\vref{Jg 13:6}] un aspect fort r.. Je ne lui
\item[\vref{Job 37:22}] y a en Dieu une majesté r.
\item[\vref{Ps 66:3}] tes œuvres sont r. ! Tes ennemis te
\item[\vref{Ps 66:5}] Dieu ! Il est r. qnd il agit
\item[\vref{Ps 76:8}] Tu es r., toi ! Qui peut se tenir dvt
\item[\vref{Ps 111:9}] Qof.] son Nom est saint et r.
\item[\vref{Ps 145:6}] de ta puissance r., et je raconterai
\item[\vref{Da 9:4}] Dieu grand et r., toi qui gardes
\item[\vref{Ha 1:7}] Il est r. et terrible, son gouvernement et son
\item[\vref{Mal 4:5}] jour grand et r. de Yahweh vienne.
\end{listverse}

\ConcordanceEntry{Redresser}
\vspace{-2mm}
\begin{listverse}
\item[\vref{Ps 146:8}] des aveugles ; Yahweh r. ceux qui sont
\item[\vref{Ec 1:15}] ne peut se r., et ce qui
\item[\vref{Ec 7:13}] Dieu : Qui pourra r. ce qu'il a
\item[\vref{Es 1:17}] recherchez la droiture, r. celui qui est
\item[\vref{Lu 3:5}] est tortueux sera r., et les chemins
\item[\vref{Lu 13:13}] l'instant elle se r., et glorifia Dieu.
\item[\vref{Ga 6:1}] qui êtes spirituels, r.-le avec un
\end{listverse}

\ConcordanceEntry{Réfléchir}
\vspace{-2mm}
\begin{listverse}
\item[\vref{Jn 11:50}] et vs. ne r. pas qu'il est
\item[\vref{Ac 12:12}] Après avoir r., il alla à
\end{listverse}

\ConcordanceEntry{Réforme}
\vspace{-2mm}
\begin{listverse}
\item[\vref{Hé 9:10}] cérémonies charnelles, jusqu'au temps de la r.
\end{listverse}

\ConcordanceEntry{Refroidir}
\vspace{-2mm}
\begin{listverse}
\item[\vref{Mt 24:12}] multipliée, la charité de plusieurs se r.
\end{listverse}

\ConcordanceEntry{Refuge}
\vspace{-2mm}
\begin{listverse}
\item[\vref{No 35:6}] six villes de r. que vs. donnerez
\item[\vref{Ps 7:2}] en toi mon r.. Sauve-moi de ts
\item[\vref{Ps 9:10}] Yahweh est un r. pour l'opprimé, un
\item[\vref{Ps 16:1}] car je cherche en toi mon r.
\item[\vref{Ps 18:3}] je trouve un r. ! Il est mon
\item[\vref{Ps 31:2}] Tu es mon r. : Fais que je
\item[\vref{Ps 71:7}] miracle, mais tu es mon puissant r.
\item[\vref{Ps 90:1}] pour ns. un r. de génération en
\item[\vref{Ps 91:4}] tu trouveras un r. sous ses ailes ;
\item[\vref{Ps 94:22}] Dieu est le Rocher de mon r.
\item[\vref{Ps 119:114}] Tu es mon r. et mon bouclier,
\item[\vref{Ps 142:5}] me reconnaît ; tt r. s'évanouit dvt moi,
\item[\vref{Pr 30:5}] bouclier pour ceux qui ont lr. r. en lui.
\item[\vref{Es 4:6}] pour servir de r. et d'asile contre
\item[\vref{Joë 3:16}] Yahweh est le r. pour son peuple,
\item[\vref{Hé 6:18}] qui avons notre r. à obtenir l'espérance
\end{listverse}

\ConcordanceEntry{Refuser}
\vspace{-2mm}
\begin{listverse}
\item[\vref{Ge 22:12}] ne m'as point r. ton fils, ton
\item[\vref{Ge 39:8}] Mais il le r., et dit à
\item[\vref{Ex 4:23}] Mais tu as r. de le laisser
\item[\vref{No 22:13}] pays, car Yahweh r. de me laisser
\item[\vref{1 S 8:19}] Mais le peuple r. d'écouter la voix
\item[\vref{Est 1:12}] la reine Vasthi r. de venir au
\item[\vref{Ps 21:3}] tu n'as point r. ce que demandaient
\item[\vref{Ps 78:10}] de Dieu et r. de marcher selon
\item[\vref{Ps 84:12}] et il ne r. aucun bien à
\item[\vref{Pr 1:24}] et que vs. r. d'entendre ; parce que
\item[\vref{Pr 21:7}] abat, parce qu'ils r. de faire ce
\item[\vref{Es 1:20}] Mais si vs. r. d'obéir et si
\item[\vref{Mt 18:17}] S'il r. de les écouter, dis-le à l'église ;
\item[\vref{Lu 18:4}] Pendant longtemps il r.. Mais après cela
\item[\vref{1 Ti 5:11}] Mais r. les veuves qui sont plus jeunes ;
\item[\vref{Hé 11:24}] Moïse, devenu grand, r. d'être nommé fils
\end{listverse}

\ConcordanceEntry{Réfuter}
\vspace{-2mm}
\begin{listverse}
\item[\vref{Ac 18:28}] car il r. publiquement les Juifs avec une grande
\item[\vref{Tit 1:9}] doctrine, que de r. les contredisants.
\end{listverse}

\ConcordanceEntry{Regard}
\vspace{-2mm}
\begin{listverse}
\item[\vref{Ex 25:20}] l'autre ; et le r. des chérubins sera
\item[\vref{Ps 31:23}] loin de ton r. ! Mais tu as
\item[\vref{Ps 34:6}] Quand on le r., on est illuminé,
\item[\vref{Ps 104:32}] Il jette son r. sur la terre,
\item[\vref{Pr 21:4}] Les r. hautains et le cœur enflé, cette
\item[\vref{Mc 3:34}] Et, jetant les r. sur ceux qui
\item[\vref{Mc 7:22}] fraude, l'impudicité, le r. envieux, les discours
\item[\vref{Lu 9:38}] prie, porte les r. sur mon fils,
\item[\vref{Jn 6:40}] quiconque pose son r. sur le Fils
\item[\vref{Ac 3:12}] pourquoi avez-vs. les r. fixés sur ns.,
\item[\vref{Ac 15:14}] premièrement jeté les r. sur les nations
\item[\vref{2 Co 3:13}] fixent pas les r. sur la fin
\item[\vref{Ja 1:25}] aura plongé les r. ds la loi
\item[\vref{1 Pi 1:12}] lesquelles les anges désirent plonger leurs r.
\end{listverse}

\ConcordanceEntry{Regarder}
\vspace{-2mm}
\begin{listverse}
\item[\vref{Ge 6:12}] Dieu dc r. la terre, et
\item[\vref{Ge 13:14}] tes yeux, et r. du lieu où
\item[\vref{Ge 19:26}] fem. de Lot r. en arrière, et
\item[\vref{Ex 3:4}] s'était détourné pour r. ; et Dieu l'appela
\item[\vref{Ex 3:6}] qu'il craignait de r. vers Dieu.
\item[\vref{Ex 14:24}] ds la nuée, r. le camp des
\item[\vref{1 S 1:11}] armées ! Si tu r. attentivement l'affliction de
\item[\vref{Ps 8:4}] Quand je r. tes cieux, l'ouvrage de tes doigts,
\item[\vref{Ps 10:14}] maltraite quelqu'un, tu r. pour le mettre
\item[\vref{Pr 4:25}] Que tes yeux r. droit et que
\item[\vref{Pr 6:6}] vers la fourmi, r. ses voies et
\item[\vref{Ec 12:5}] qnd ceux qui r. par les fenêtres
\item[\vref{Es 17:7}] ce jour-là, l'hom. r. vers celui qui
\item[\vref{Es 22:11}] Mais vs. ne r. pas à celui
\item[\vref{Es 53:2}] qnd ns. le r., ni apparence qui
\item[\vref{Jé 1:10}] R., je t'établis aujourd'hui sur les nations
\item[\vref{Ez 37:8}] Puis je r., et voici, il vint des nerfs
\item[\vref{Da 2:31}] Ô roi, tu r. et tu voyais
\item[\vref{Jon 3:10}] Et Dieu r. à ce qu'ils avaient fait, comment
\item[\vref{Mi 7:7}] Mais moi, je r. vers Yahweh, je
\item[\vref{Mt 5:28}] dis que quiconque r. une fem. pour
\item[\vref{Mc 10:21}] Jésus, l'ayant r., l'aima, et lui
\item[\vref{Lu 9:62}] la charrue, et r. en arrière, n'est
\item[\vref{Lu 22:61}] Seign., s'étant retourné, r. Pierre. Et Pierre
\item[\vref{Jn 1:36}] et, r. Jésus qui marchait, il dit : Voici
\item[\vref{Jn 4:35}] vos yeux, et r. les champs qui
\item[\vref{Jn 20:27}] doigt ici, et r. mes mains, avance
\item[\vref{Ac 1:11}] vs. arrêtez-vs. à r. au ciel ? Ce
\item[\vref{Ac 3:4}] yeux sur lui, et lui dit : R.-ns.
\item[\vref{Ac 17:23}] passant et en r. vos divinités, j'ai
\item[\vref{Ro 11:25}] vs. ne vs. r. pas com. sages,
\item[\vref{1 Co 4:1}] Que chacun ns. r. com. des serviteurs
\item[\vref{Ep 1:17}] ds ce qui r. sa connaissance.
\item[\vref{Ph 2:3}] cœur vs. fasse r. les autres com.
\item[\vref{Ph 3:8}] Et certes, je r. ttes les autres
\item[\vref{2 Th 3:15}] Toutefois, ne le r. pas com. un
\item[\vref{Ja 1:2}] Mes frères, r. com. un sujet
\item[\vref{Ja 1:23}] un hom. qui r. ds un miroir
\end{listverse}

\ConcordanceEntry{Régénération}
\vspace{-2mm}
\begin{listverse}
\item[\vref{Tit 3:5}] bain de la r. et le renouvellement
\end{listverse}

\ConcordanceEntry{Régénérer}
\vspace{-2mm}
\begin{listverse}
\item[\vref{1 Pi 1:3}] miséricorde ns. a r. pour une espérance
\item[\vref{1 Pi 1:23}] vs. avez été r., non par une
\end{listverse}

\ConcordanceEntry{Règne}
\vspace{-2mm}
\begin{listverse}
\item[\vref{Ge 10:10}] commencement de son r. fut Babel, Erec,
\item[\vref{2 S 7:16}] maison et ton r. seront assurés à
\item[\vref{Job 25:2}] Le r. et la terreur appartiennent à Dieu ;
\item[\vref{Ps 22:29}] Car le r. appartient à Yahweh : Il domine sur
\item[\vref{Ps 45:7}] sceptre de ton r. est un sceptre
\item[\vref{Ps 103:19}] cieux, et son r. domine sur tt.
\item[\vref{Ps 145:11}] gloire de ton r., et ils proclameront
\item[\vref{Ps 145:12}] et la splendeur glorieuse de ton r.
\item[\vref{Da 4:34}] et dont le r. subsiste de génération
\item[\vref{Da 5:26}] a compté ton r., et y a
\item[\vref{Mt 6:10}] que ton r. vienne ; que ta volonté soit faite
\item[\vref{Mt 6:13}] les siècles, le r., la puissance et
\item[\vref{Mt 16:28}] Fils de l'hom. venir ds son r.
\item[\vref{Mc 11:10}] Béni soit le r. de David, notre
\item[\vref{Lu 1:33}] éternellement, et son r. n'aura pas de
\item[\vref{Lu 23:42}] moi qnd tu viendras ds ton r.
\item[\vref{Ap 12:10}] la force, le r. de notre Dieu,
\end{listverse}

\ConcordanceEntry{Régner}
\vspace{-2mm}
\begin{listverse}
\item[\vref{Ge 37:8}] frères lui dirent : R.-tu sur ns. ?
\item[\vref{Ge 41:56}] La famine r. ds tt le
\item[\vref{Ex 15:18}] Yahweh r. à jamais et à perpétuité.
\item[\vref{Es 32:1}] Voici, un roi r. selon la justice,
\item[\vref{Jé 23:5}] Germe juste, qui r. en Roi ; il
\item[\vref{Lu 1:33}] Il r. sur la maison de Jacob éternellement,
\item[\vref{Ro 5:14}] la mort a r. depuis Adam jusqu'à
\item[\vref{Ro 5:17}] la mort a r. par lui seul,
\item[\vref{Ro 15:12}] se lèvera pour r. sur les nations ;
\item[\vref{2 Ti 2:12}] avec lui, ns. r. aussi avec lui.
\item[\vref{Ap 5:10}] Dieu ; et ns. r. sur la terre.
\item[\vref{Ap 11:15}] Christ, et il r. aux siècles des
\item[\vref{Ap 20:6}] Christ, et ils r. avec lui mille
\item[\vref{Ap 22:5}] éclairera, et ils r. aux siècles des
\end{listverse}

\ConcordanceEntry{Rein}
\vspace{-2mm}
\begin{listverse}
\item[\vref{Ex 12:11}] mangerez ainsi : Vos r. seront ceints, vs.
\item[\vref{Ps 7:10}] cœurs et les r., ô Dieu juste !
\item[\vref{Ps 16:7}] ds lesquelles mes r. m'enseignent.
\item[\vref{Ps 26:2}] au creuset mes r. et mon cœur ;
\item[\vref{Ps 139:13}] as créé mes r., tu me couvres
\item[\vref{Pr 31:17}] Elle ceint ses r. de force, et
\item[\vref{Es 11:5}] ceinture de ses r., et la fidélité,
\item[\vref{Jé 17:10}] qui éprouve les r. ; mm pour rendre
\item[\vref{Mt 3:4}] autour de ses r.. Et il se
\item[\vref{Lu 12:35}] Que vos r. soient ceints, et vos lampes allumées.
\item[\vref{Ac 2:30}] fruit de ses r. il ferait naître
\item[\vref{Ep 6:14}] ayant à vos r. la vérité pour
\item[\vref{Hé 7:10}] encore ds les r. de son père,
\item[\vref{1 Pi 1:13}] pourquoi, ceignez les r. de votre entendement,
\item[\vref{Ap 2:23}] qui sonde les r. et les cœurs,
\end{listverse}

\ConcordanceEntry{Reine}
\vspace{-2mm}
\begin{listverse}
\item[\vref{1 R 10:1}] Or la r. de Séba ayant appris la renommée
\item[\vref{Est 2:17}] tête, et l'établit r. à la place
\item[\vref{Ps 45:10}] tes bien-aimées ; la r. est à ta
\item[\vref{Jé 7:18}] gâteaux à la r. des cieux, et
\item[\vref{Mt 12:42}] La r. du Midi se lèvera au jour
\item[\vref{Ap 18:7}] Je siège en r., je ne suis
\end{listverse}

\ConcordanceEntry{Reine des cieux}
\vspace{-2mm}
\begin{listverse}
\item[\vref{Jé 7:18}] gâteaux à la reine des cieux, et pour faire
\item[\vref{Jé 44:17}] l'encens à la reine des cieux, et lui faire
\end{listverse}

\ConcordanceEntry{Rejeter}
\vspace{-2mm}
\begin{listverse}
\item[\vref{Ge 37:35}] consoler, mais il r. tte consolation. Il
\item[\vref{No 11:20}] que vs. avez r. Yahweh qui est
\item[\vref{1 S 8:7}] toi qu'ils ont r., mais c'est moi
\item[\vref{1 S 10:19}] aujourd'hui, vs. avez r. votre Dieu, celui
\item[\vref{1 S 15:23}] Puisque tu as r. la parole de
\item[\vref{Ps 77:8}] Le Seign. m'a-t-il r. pour toujours ? Ne
\item[\vref{Ps 89:39}] Néanmoins, tu l'as r. et dédaigné ! Tu
\item[\vref{Pr 4:2}] bonne doctrine, ne r. dc pas mon
\item[\vref{Pr 8:33}] et soyez sages, et ne la r. point.
\item[\vref{Os 4:6}] que tu as r. la connaissance, je
\item[\vref{Mc 7:9}] dit aussi : Vous r. bien le commandement
\item[\vref{Mc 8:31}] et qu'il soit r. par les anciens,
\item[\vref{Lu 20:17}] La pierre qu'ont r. ceux qui bâtissaient
\item[\vref{Ro 11:1}] dc : Dieu a-t-il r. son peuple ? A
\item[\vref{Ro 13:12}] le jour approche. R. dc les œuvres
\item[\vref{2 Co 4:2}] ns. avons entièrement r. les choses honteuses
\item[\vref{Col 3:8}] mntnt, vs. aussi, r. ttes ces choses :
\item[\vref{1 Ti 4:4}] ne doit être r., pourvu qu'on le
\item[\vref{Hé 12:1}] nuée de témoins, r. tt fardeau, et
\item[\vref{Hé 12:17}] bénédiction, il fut r., car il ne
\item[\vref{Ja 1:21}] C'est pourquoi, r. tte souillure et
\item[\vref{1 Pi 2:4}] lui, pierre vivante, r. par les hommes,
\item[\vref{Jud 1:13}] de la mer, r. l'écume de leurs
\end{listverse}

\ConcordanceEntry{Rejeton}
\vspace{-2mm}
\begin{listverse}
\item[\vref{Es 11:1}] d'Isaï, et un r. naîtra de ses
\item[\vref{Es 53:2}] plante, com. un r. qui sort d'une
\item[\vref{Hé 12:15}] d'amertume, poussant des r., ne vs. trouble,
\item[\vref{Ap 22:16}] Je suis le r. et la postérité
\end{listverse}

\ConcordanceEntry{Réjouir}
\vspace{-2mm}
\begin{listverse}
\item[\vref{Ex 4:14}] verra, il se r. ds son cœur.
\item[\vref{Lé 23:40}] et vs. vs. r. pendant sept jours,
\item[\vref{1 S 2:1}] Mon cœur se r. en Yahweh, ma
\item[\vref{Esd 6:22}] Yahweh les avait r. en disposant le
\item[\vref{Ps 2:11}] avec crainte, et r.-vs. avec tremblement.
\item[\vref{Ps 5:12}] en toi se r., qu'ils soient ds
\item[\vref{Ps 13:5}] adversaires ne se r., si je venais
\item[\vref{Ps 13:6}] mon cœur se r. de la délivrance
\item[\vref{Ps 19:9}] sont droites, elles r. le cœur ; les
\item[\vref{Ps 30:2}] mes ennemis se r. à mon sujet.
\item[\vref{Ps 32:11}] Vous justes, r.-vs. en Yahweh,
\item[\vref{Ps 51:10}] os que tu as brisés se r.
\item[\vref{Ps 70:5}] exultent et se r. en toi ! Et
\item[\vref{Ps 86:4}] R. l'âme de ton serviteur, car j'élève
\item[\vref{Ps 90:15}] R.-ns. autant de jours que tu
\item[\vref{Ps 92:5}] Yahweh ! tu me r. par tes œuvres,
\item[\vref{Ps 119:14}] Je me r. ds le chemin de tes préceptes
\item[\vref{Ps 119:162}] Je me r. de ta parole com. ferait celui
\item[\vref{Ps 149:2}] Qu'Israël se r. en celui qui
\item[\vref{Pr 17:21}] le père du sot ne se r. pas.
\item[\vref{Pr 24:17}] tombe, ne t'en r. pas, et qnd
\item[\vref{Ec 12:1}] Jeune hom., r.-toi ds ton
\item[\vref{Es 41:16}] toi, tu te r. en Yahweh, tu
\item[\vref{Es 49:13}] Ô cieux, r.-vs. avec des chants de triomphe !
\item[\vref{Es 54:1}] R.-toi avec chants de triomphe, stérile,
\item[\vref{Es 56:7}] et je les r. ds ma maison
\item[\vref{Es 61:10}] Je me r., je me réjouirai en Yahweh, et
\item[\vref{Mi 7:8}] ennemie, ne te r. pas à mon
\item[\vref{Mt 5:12}] R.-vs. et soyez ds l'allégresse, parce
\item[\vref{Lu 1:47}] mon esprit se r. en Dieu, mon
\item[\vref{Lu 15:6}] il lr. dit : R.-vs. avec moi ;
\item[\vref{Jn 8:56}] il l'a vu, et il s'est r.
\item[\vref{Ro 12:15}] R.-vs. avec ceux qui se réjouissent ;
\item[\vref{1 Co 12:26}] membres ensemble se r. avec lui.
\item[\vref{1 Co 13:6}] elle ne se r. pas de l'injustice,
\item[\vref{Ga 4:27}] il est écrit : R.-toi, stérile, toi
\item[\vref{Ph 2:17}] foi, je m'en r., et je m'en
\item[\vref{Ph 3:1}] reste, mes frères, r.-vs. ds le
\item[\vref{Ph 4:4}] R.-vs. toujours ds le Seign. ; je
\item[\vref{Col 1:24}] Je me r. dc mntnt ds mes souffrances pour
\item[\vref{Col 2:5}] en esprit, me r., et voyant votre
\item[\vref{Phm 1:20}] en notre Seign. ; r. mes entrailles en
\item[\vref{1 Pi 1:8}] et vs. vs. r. d'une joie ineffable
\item[\vref{1 Pi 4:13}] Mais r.-vs. de ce que vs. participez
\item[\vref{2 Jn 1:4}] me suis fort r. d'avoir trouvé quelques-uns
\item[\vref{Ap 11:10}] la terre se r., ils seront ds
\item[\vref{Ap 12:12}] C'est pourquoi r.-vs. cieux, et
\item[\vref{Ap 19:7}] R.-ns. et tressaillons de joie, et
\end{listverse}

\ConcordanceEntry{Réjouissance}
\vspace{-2mm}
\begin{listverse}
\item[\vref{2 S 6:12}] cité de David, au milieu des r.
\item[\vref{Ps 33:3}] vos instruments avec un cri de r. !
\item[\vref{Ps 100:1}] des cris de r. à Yahweh !
\item[\vref{Ps 137:6}] de Jérus. le sujet de ma r. !
\item[\vref{So 3:14}] des cris de r., ô Israël ! Réjouis-toi
\end{listverse}

\ConcordanceEntry{Relever}
\vspace{-2mm}
\begin{listverse}
\item[\vref{De 25:7}] beau-frère refuse de r. le nom de
\item[\vref{Ps 30:2}] que tu m'as r., tu n'as pas
\item[\vref{Ps 148:14}] Il a r. la force de son peuple, sujet
\item[\vref{Pr 24:16}] fois, et sera r. ; mais les méchants
\item[\vref{Ec 4:10}] tombé, il n'aura personne pour le r.
\item[\vref{Jn 8:10}] Alors Jésus s'étant r., et ne voyant
\item[\vref{Ac 10:26}] Mais Pierre le r., en lui disant :
\item[\vref{Ac 20:9}] on voulut le r., il était mort.
\end{listverse}

\ConcordanceEntry{Religieux}
\vspace{-2mm}
\begin{listverse}
\item[\vref{Ac 17:22}] vs. trouve à ts égards extrêmement r.
\item[\vref{Ja 1:26}] vs. pense être r., et ne tient
\end{listverse}

\ConcordanceEntry{Religion}
\vspace{-2mm}
\begin{listverse}
\item[\vref{Ja 1:27}] La r. pure et sans tache dvt notre
\end{listverse}

\ConcordanceEntry{Remettre}
\vspace{-2mm}
\begin{listverse}
\item[\vref{Mt 6:12}] ns. aussi ns. r. les dettes à
\item[\vref{Mt 25:27}] te fallait dc r. mon argent aux
\item[\vref{1 Pi 2:23}] mais il se r. à celui qui
\item[\vref{1 Pi 5:7}] R.-lui tt ce qui peut vs.
\end{listverse}

\ConcordanceEntry{Rémission}
\vspace{-2mm}
\begin{listverse}
\item[\vref{Mt 26:28}] beaucoup, pour la r. des péchés.
\item[\vref{Mc 1:4}] pour obtenir la r. des péchés.
\item[\vref{Lu 1:77}] salut, par la r. de leurs péchés,
\item[\vref{Lu 3:3}] repentance, pour la r. des péchés,
\item[\vref{Ac 5:31}] repentance et la r. des péchés.
\item[\vref{Ac 10:43}] lui reçoit la r. de ses péchés
\item[\vref{Ac 13:38}] lui que la r. des péchés vs.
\item[\vref{Ac 26:18}] ils reçoivent la r. de leurs péchés
\item[\vref{Ro 3:25}] justice, par la r. des péchés précédents,
\item[\vref{Ep 1:7}] à savoir la r. des offenses, selon
\item[\vref{Col 1:14}] à savoir la r. des péchés.
\item[\vref{Hé 9:22}] a pas de r. des péchés.
\end{listverse}

\ConcordanceEntry{Remonter}
\vspace{-2mm}
\begin{listverse}
\item[\vref{Ge 46:4}] aussi très certainement r. ; et Joseph te
\item[\vref{Ex 3:17}] Je vs. ferai r. de l'Egypte où
\item[\vref{Job 7:9}] celui qui descend au scheol ne r. pas ;
\item[\vref{Ps 30:4}] tu as fait r. mon âme du
\item[\vref{Ps 71:20}] tu me feras r. hors des abîmes
\item[\vref{Mi 5:1}] Israël, dont l'origine r. aux temps anciens,
\item[\vref{Ro 10:7}] l'abîme ? C'est faire r. Christ d'entre les
\end{listverse}

\ConcordanceEntry{Rempart}
\vspace{-2mm}
\begin{listverse}
\item[\vref{Job 1:10}] pas mis un r. tt autour de
\item[\vref{Ps 48:14}] Observez son r., examinez ses palais
\item[\vref{Es 26:1}] sera mis pour muraille et pour r.
\end{listverse}

\ConcordanceEntry{Remplir}
\vspace{-2mm}
\begin{listverse}
\item[\vref{Ge 1:22}] féconds, multipliez, et r. les eaux des
\item[\vref{Ge 6:11}] dvt Dieu, et r. de violence.
\item[\vref{Ex 1:7}] sorte que le pays en fut r.
\item[\vref{De 34:9}] de Nun, fut r. de l'Esprit de
\item[\vref{1 R 8:11}] gloire de Yahweh r. la maison de
\item[\vref{1 R 18:34}] Puis il dit : R. quatre cruches d'eau,
\item[\vref{Ps 81:11}] Ouvre ta bouche et je la r.
\item[\vref{Ps 90:12}] avoir un cœur r. de sagesse.
\item[\vref{Ps 126:2}] notre bouche était r. de joie, et
\item[\vref{Ps 127:5}] qui en a r. son carquois ! Ils
\item[\vref{Es 6:1}] de sa robe r. le temple.
\item[\vref{Jé 23:24}] dit Yahweh. Ne r.-je pas, moi,
\item[\vref{Ez 37:1}] milieu d'une vallée r. d'ossements.
\item[\vref{Ag 2:7}] viendront, et je r. de gloire cette
\item[\vref{Lu 4:1}] Jésus, r. du Saint-Esprit, revint du Jourdain, et
\item[\vref{Lu 8:23}] la barque se r. d'eau, et ils
\item[\vref{Lu 14:23}] d'entrer, afin que ma maison soit r.
\item[\vref{Jn 2:7}] Jésus lr. dit : R. d'eau ces vases.
\item[\vref{Jn 16:6}] la tristesse a r. votre cœur.
\item[\vref{Ac 7:55}] Mais Etienne, r. du Saint-Esprit, et
\item[\vref{Ac 9:17}] que tu sois r. du Saint-Esprit.
\item[\vref{Ro 1:29}] étant r. de tte espèce d'injustice, d'impureté, de
\item[\vref{1 Co 2:1}] des discours pompeux, r. de la sagesse
\item[\vref{2 Co 7:4}] vs. ; je suis r. de consolation, je
\item[\vref{Ep 1:23}] de celui qui r. tt en ts.
\item[\vref{Ep 3:19}] que vs. soyez r. de tte la
\item[\vref{Ep 4:10}] cieux, afin de r. ttes choses.
\item[\vref{Ep 5:18}] dissolution, mais soyez r. de l'Esprit.
\item[\vref{Ph 1:11}] étant r. de fruits de justice, qui sont
\item[\vref{Col 1:9}] que vs. soyez r. de la connaissance
\item[\vref{2 Ti 1:4}] te voir, afin que je sois r. de joie,
\item[\vref{1 Pi 3:8}] d'un mm sentiment, r. de compassion les
\item[\vref{Ap 13:3}] mortelle fut guérie. R. d'admiration, la terre
\end{listverse}

\ConcordanceEntry{Rémunération}
\vspace{-2mm}
\begin{listverse}
\item[\vref{Hé 11:26}] parce qu'il avait égard à la r.
\end{listverse}

\ConcordanceEntry{Renard}
\vspace{-2mm}
\begin{listverse}
\item[\vref{Jg 15:4}] prit trois cents r., il prit aussi
\item[\vref{Ca 2:15}] Prenez-ns. les r., et les petits
\item[\vref{Mt 8:20}] lui dit : Les r. ont des tanières,
\item[\vref{Lu 13:32}] dites à ce r. : Voici, je chasse
\end{listverse}

\ConcordanceEntry{Rencontre}
\vspace{-2mm}
\begin{listverse}
\item[\vref{Ge 14:17}] sortit à la r. d'Abram qui revenait
\item[\vref{Am 4:12}] prépare-toi à la r. de ton Dieu,
\item[\vref{Mt 25:6}] Voici, l'époux vient, allez à sa r. !
\item[\vref{1 Th 4:17}] nuées, à la r. du Seign., ds
\end{listverse}

\ConcordanceEntry{Rendre}
\vspace{-2mm}
\begin{listverse}
\item[\vref{Ge 12:2}] te bénirai, je r. ton nom grand,
\item[\vref{Ex 1:14}] Tellement qu'ils lr. r. la vie amère
\item[\vref{Job 32:8}] Tout-Puissant qui le r. intelligent ;
\item[\vref{Job 34:11}] Car il r. à l'hom. selon son œuvre, il
\item[\vref{Ps 18:34}] Il r. mes pieds semblables à ceux des
\item[\vref{Ps 29:2}] R. à Yahweh la gloire due à
\item[\vref{Ps 32:8}] Je te r. intelligent, je t'enseignerai la voie ds
\item[\vref{Ps 45:18}] Je r. ton Nom mémorable ds ts les
\item[\vref{Ps 51:14}] R.-moi la joie de ton salut
\item[\vref{Ps 116:12}] Que r.-je à Yahweh ? Tous ses bienfaits
\item[\vref{Ps 119:9}] le jeune hom. r.-t-il pure sa
\item[\vref{Ps 143:11}] Yahweh, r.-moi la vie pour l'amour de
\item[\vref{Pr 2:2}] si tu r. ton oreille attentive à la sagesse,
\item[\vref{Pr 11:5}] de l'hom. intègre r. droite sa voie,
\item[\vref{Pr 15:13}] Le cœur joyeux r. le visage beau,
\item[\vref{Mt 22:21}] il lr. dit : R. dc à César
\item[\vref{Lu 18:11}] Dieu, je te r. grâces de ce
\item[\vref{Lu 19:8}] quelqu'un, je lui r. le quadruple.
\item[\vref{Jn 8:32}] la vérité, et la vérité vs. r. libres.
\item[\vref{Ro 8:11}] d'entre les morts r. aussi la vie
\item[\vref{Ro 12:17}] Ne r. à personne le mal pour le
\item[\vref{Ph 3:3}] circoncis, ns. qui r. à Dieu notre
\item[\vref{Hé 13:17}] com. dvt en r. compte ; afin que
\item[\vref{Ja 4:4}] du monde, se r. ennemi de Dieu.
\item[\vref{1 Pi 3:9}] Ne r. pas mal pour mal, ou injure
\item[\vref{1 Pi 4:5}] Mais ils r. compte à celui qui est prêt
\item[\vref{Ap 2:23}] cœurs, et je r. à chacun de
\item[\vref{Ap 22:2}] douze fruits, et r. son fruit chaque
\item[\vref{Ap 22:12}] avec moi pour r. à chacun selon
\end{listverse}

\ConcordanceEntry{Renier}
\vspace{-2mm}
\begin{listverse}
\item[\vref{Jos 24:27}] que vs. ne r. pas votre Dieu.
\item[\vref{Pr 30:9}] je ne te r. et que je
\item[\vref{Mt 10:33}] Mais quiconque me r. dvt les hommes,
\item[\vref{Mt 26:34}] chanté, tu me r. trois fois.
\item[\vref{Ac 3:13}] avez livré et r. dvt Pilate, quoiqu'il
\item[\vref{1 Ti 5:8}] famille, il a r. la foi, et
\item[\vref{2 Ti 2:12}] Si ns. le r., il ns. reniera
\item[\vref{2 Pi 2:1}] pernicieuses, et qui r. le Seign. qui
\item[\vref{Jud 1:4}] dissolution, et qui r. le seul dominateur,
\item[\vref{Ap 2:13}] tu n'as pas r. ma foi, mm
\item[\vref{Ap 3:8}] parole, et que tu n'as pas r. mon Nom.
\end{listverse}

\ConcordanceEntry{Renommée}
\vspace{-2mm}
\begin{listverse}
\item[\vref{Jos 6:27}] Josué, et sa r. se répandit ds
\item[\vref{1 R 4:31}] Machol ; et sa r. était répandue parmi
\item[\vref{1 R 10:1}] ayant appris la r. de Salomon, à
\item[\vref{Ps 138:2}] ta parole au-dessus de tte ta r.
\item[\vref{Pr 22:1}] La r. est préférable aux grandes richesses, et
\item[\vref{Lu 4:14}] l'Esprit, et sa r. se répandit ds
\end{listverse}

\ConcordanceEntry{Renoncer}
\vspace{-2mm}
\begin{listverse}
\item[\vref{Jg 9:9}] l'olivier lr. répondit : R.-je à mon
\item[\vref{Pr 23:4}] pas à t'enrichir, r. à ta propre
\item[\vref{Mt 16:24}] après moi, qu'il r. à lui-mm, et
\item[\vref{Mc 8:34}] après moi, qu'il r. à lui-mm, qu'il
\item[\vref{Lu 9:23}] après moi, qu'il r. à lui-mm, et
\item[\vref{Lu 14:33}] d'entre vs. ne r. pas à tt
\item[\vref{Ac 14:15}] vs. exhortons à r. à ces choses
\item[\vref{Ep 6:9}] mm chose et r. aux menaces, sachant
\item[\vref{Tit 2:12}] ns. enseigne à r. à l'impiété et
\item[\vref{1 Pi 2:1}] Ayant dc r. à tte sorte
\end{listverse}

\ConcordanceEntry{Renouveler}
\vspace{-2mm}
\begin{listverse}
\item[\vref{2 Ch 24:4}] la pensée de r. la maison de
\item[\vref{Ps 103:5}] ta jeunesse est r. com. celle de
\item[\vref{2 Co 4:16}] toutefois l'intérieur est r. de jour en
\item[\vref{Ep 4:23}] que vs. soyez r. ds l'esprit de
\end{listverse}

\ConcordanceEntry{Renouvellement}
\vspace{-2mm}
\begin{listverse}
\item[\vref{Mt 19:28}] de l'hom., au r. de ttes choses,
\item[\vref{Ro 12:2}] transformés par le r. de votre entendement,
\item[\vref{Tit 3:5}] régénération et le r. du Saint-Esprit,
\end{listverse}

\ConcordanceEntry{Renverser}
\vspace{-2mm}
\begin{listverse}
\item[\vref{De 7:5}] cette manière : Vous r. leurs autels, vs.
\item[\vref{Né 1:3}] de Jérus. demeure r. et ses portes
\item[\vref{Job 5:13}] et le conseil des méchants est r. :
\item[\vref{Job 8:3}] Dieu r.-il le droit, et le Tout-Puissant
\item[\vref{Mt 21:12}] le temple ; il r. les tables des
\item[\vref{Lu 1:52}] Il a r. de dessus leurs trônes les puissants,
\item[\vref{Jn 2:15}] des changeurs, et r. les tables.
\item[\vref{Ac 13:10}] cesseras-tu pas de r. les voies droites
\item[\vref{Ga 1:7}] et qui veulent r. l'Evangile de Christ.
\item[\vref{2 Ti 2:18}] arrivée, et qui r. la foi de
\item[\vref{Tit 1:11}] bouche, et qui r. les maisons tt
\end{listverse}

\ConcordanceEntry{Renvoyer}
\vspace{-2mm}
\begin{listverse}
\item[\vref{Ge 21:14}] l'enfant et la r.. Elle se mit
\item[\vref{Esd 10:19}] donnant leurs mains, r. leurs femmes ; et
\item[\vref{Mt 15:32}] veux pas les r. à jeun, de
\item[\vref{Lu 1:53}] et il a r. les riches à
\item[\vref{Ac 23:22}] Le tribun dc r. le jeune hom.,
\item[\vref{Hé 11:31}] reçu les espions et les avait r. en paix.
\end{listverse}

\ConcordanceEntry{Répandre}
\vspace{-2mm}
\begin{listverse}
\item[\vref{Ge 8:17}] terre ; qu'ils se r. sur la terre,
\item[\vref{Ex 9:10}] Pharaon ; Moïse la r. vers les cieux
\item[\vref{Ex 29:7}] d'onction et la r. sur sa tête ;
\item[\vref{1 S 1:15}] forte ; mais je r. mon âme dvt
\item[\vref{Ps 51:20}] R. par ta grâce tes bienfaits sur
\item[\vref{Ps 119:136}] Mes yeux r. des torrents d'eau parce qu'on n'observe
\item[\vref{Pr 1:16}] au mal, et se hâtent pour r. le sang.
\item[\vref{Es 54:3}] Car tu te r. à droite et
\item[\vref{Mc 5:14}] pourceaux s'enfuirent, et r. la nouvelle ds
\item[\vref{Ro 3:15}] sont légers pour r. le sang ;
\item[\vref{Hé 12:4}] encore résisté jusqu'à r. votre sang en
\end{listverse}

\ConcordanceEntry{Réparer}
\vspace{-2mm}
\begin{listverse}
\item[\vref{1 R 18:30}] lui et il r. l'autel de Yahweh,
\item[\vref{Ps 60:4}] mise en pièces: R. ses brèches, car
\item[\vref{Am 9:11}] est tombé, j'en r. les brèches, j'en
\item[\vref{Mt 4:21}] lr. père, qui r. leurs filets, et
\item[\vref{Ac 15:16}] est tombé, je r. ses ruines et
\end{listverse}

\ConcordanceEntry{Repas}
\vspace{-2mm}
\begin{listverse}
\item[\vref{Pr 15:17}] Mieux vaut un r. d'herbes où il
\item[\vref{Mt 15:2}] les mains qnd ils prennent lr. r.
\item[\vref{Mc 3:20}] ne pouvaient mm pas prendre lr. r.
\item[\vref{1 Co 11:20}] pour manger le r. du Seign. ;
\item[\vref{2 Pi 2:13}] tromperies ds les r. qu'ils font avec
\item[\vref{Jud 1:12}] lorsqu'ils prennent leurs r. avec vs. sans
\end{listverse}

\ConcordanceEntry{Repentance}
\vspace{-2mm}
\begin{listverse}
\item[\vref{Mt 3:8}] dc des fruits convenables à la r.
\item[\vref{Mc 1:4}] le baptême de r., pour obtenir la
\item[\vref{Mc 6:12}] partirent dc, et ils prêchèrent la r.
\item[\vref{Lu 15:7}] justes qui n'ont pas besoin de r.
\item[\vref{Lu 24:47}] et que la r. et le pardon
\item[\vref{Ac 5:31}] à Israël la r. et la rémission
\item[\vref{Ac 11:18}] dc accordé la r. aussi aux Gentils,
\item[\vref{Ac 20:21}] qu'aux Grecs la r. envers Dieu, et
\item[\vref{Ro 2:4}] de Dieu te convie à la r. ?
\item[\vref{2 Co 7:10}] Dieu, produit une r. à salut dont
\item[\vref{Hé 6:6}] nouveau par la r., vu que, quant
\item[\vref{Hé 12:17}] lieu à la r., quoiqu'il l'ait demandée
\end{listverse}

\ConcordanceEntry{Repentir (se)}
\vspace{-2mm}
\begin{listverse}
\item[\vref{Ge 6:6}] Yahweh se r. d'avoir fait l'hom.
\item[\vref{Ex 32:14}] Et Yahweh se r. du mal qu'il
\item[\vref{1 S 15:29}] n'est pas un hom. pour se r.
\item[\vref{2 S 24:16}] ravager, Yahweh se r. de ce mal,
\item[\vref{Jé 18:8}] méchanceté, je me r. aussi du mal
\item[\vref{Jé 18:10}] voix, je me r. aussi du bien
\item[\vref{Joë 2:14}] et ne se r. pas, et s'il
\item[\vref{Am 7:3}] Yahweh se r. de cela. Cela
\item[\vref{Mt 11:20}] miracles, parce qu'elles ne s'étaient pas r.
\item[\vref{Mt 12:41}] parce qu'ils se r. à la prédication
\item[\vref{Ap 16:9}] ils ne se r. pas pour lui
\end{listverse}

\ConcordanceEntry{Repentir (le)}
\vspace{-2mm}
\begin{listverse}
\item[\vref{Os 13:14}] destruction ? Mais le r. se cache à
\end{listverse}

\ConcordanceEntry{Rephaïm}
\vspace{-2mm}
\begin{listverse}
\item[\vref{Ge 14:5}] ils battirent les R. à Aschteroth-Karnaïm, les
\item[\vref{De 2:11}] considérés com. des R., de mm que
\item[\vref{De 3:11}] du reste des R.. Voici, son lit,
\item[\vref{2 S 23:13}] était campée ds la vallée des R.
\end{listverse}

\ConcordanceEntry{Répondre}
\vspace{-2mm}
\begin{listverse}
\item[\vref{Ge 3:10}] Il r. : J'ai entendu ta voix ds le
\item[\vref{Ge 4:9}] Et il lui r. : Je ne sais,
\item[\vref{Jé 1:6}] Je r. : Ah ! Seign. Yahweh ! Voici, je ne
\item[\vref{Mt 8:8}] le centenier lui r., et dit : Seign.,
\item[\vref{Mt 19:8}] Il lr. r. : C'est à cause de la dureté
\item[\vref{Mt 22:46}] ne pouvait lui r. un seul mot.
\item[\vref{Jn 8:48}] Alors les Juifs r. : N'avons-ns. pas raison
\item[\vref{Ac 19:15}] l'esprit malin lr. r. : Je connais Jésus,
\item[\vref{Ac 26:28}] Et Agrippa r. à Paul : Tu
\item[\vref{Col 4:6}] vs. avez à r. à chacun.
\item[\vref{1 Pi 3:15}] toujours prêts à r. avec douceur et
\end{listverse}

\ConcordanceEntry{Réponse}
\vspace{-2mm}
\begin{listverse}
\item[\vref{Ge 41:16}] qui donnera une r. concernant la paix
\item[\vref{Pr 15:1}] La r. douce apaise la fureur ; mais la
\item[\vref{Pr 16:1}] l'hom., mais la r. de la langue
\item[\vref{Mc 15:5}] donna plus aucune r., ce qui étonna
\item[\vref{Lu 2:47}] de sa sagesse et de ses r.
\item[\vref{Lu 20:26}] étonnés de sa r., ils gardèrent le
\end{listverse}

\ConcordanceEntry{Repos}
\vspace{-2mm}
\begin{listverse}
\item[\vref{Ex 20:8}] du jour du r. pour le sanctifier.
\item[\vref{Ex 33:14}] ira, et je te donnerai du r.
\item[\vref{No 10:33}] pour lr. chercher un lieu de r.
\item[\vref{Jos 21:44}] accorda un parfait r. tt autour, selon
\item[\vref{Ru 3:1}] chercherai-je pas du r., afin que tu
\item[\vref{1 R 5:4}] m'a donné du r. de ttes parts
\item[\vref{1 Ch 22:9}] un hom. de r., et à qui
\item[\vref{Est 9:16}] ils eurent du r. et furent délivrés
\item[\vref{Job 34:29}] S'il donne le r., qui est-ce qui
\item[\vref{Ps 95:11}] colère, ils n'entreront pas ds mon r. !
\item[\vref{Ps 116:7}] retourne ds ton r., car Yahweh t'a
\item[\vref{Ps 132:14}] mon lieu de r. à perpétuité ; j'y
\item[\vref{Es 28:12}] disait : Voici le r., donnez du repos
\item[\vref{Es 30:15}] et ds le r. que vs. serez
\item[\vref{Es 63:14}] a menés au r. com. on mène
\item[\vref{Es 66:1}] quel serait le lieu de mon r. ?
\item[\vref{Jé 46:27}] il sera en r. et en paix,
\item[\vref{Mi 2:10}] un lieu de r. ; à cause de
\item[\vref{Za 1:11}] terre est en r. et tranquille.
\item[\vref{Mt 11:28}] chargés, et je vs. donnerai du r.
\item[\vref{Mt 12:43}] arides, cherchant du r., mais il n'en
\item[\vref{Hé 4:9}] reste dc un r. au peuple de
\item[\vref{Hé 4:11}] d'entrer ds ce r.-là, de peur
\item[\vref{Ap 6:11}] se tenir en r. encore un peu
\item[\vref{Ap 14:11}] ils n'auront de r. ni jour ni
\end{listverse}

\ConcordanceEntry{Reposer}
\vspace{-2mm}
\begin{listverse}
\item[\vref{Ge 2:2}] et il se r. au septième jour
\item[\vref{Ex 16:30}] peuple dc se r. le septième jour.
\item[\vref{Ps 4:9}] tu me fais r. en sécurité.
\item[\vref{Ps 91:1}] couverture du Très-Haut, r. à l'ombre du
\item[\vref{Ec 7:9}] t'irriter, car l'irritation r. ds le sein
\item[\vref{Ec 11:6}] ne laisse pas r. tes mains le
\item[\vref{Es 11:2}] L'Esprit de Yahweh r. sur lui, Esprit
\item[\vref{Da 12:13}] car tu te r., et tu demeureras
\item[\vref{Mt 8:20}] l'hom. n'a pas de place pour r. sa tête.
\item[\vref{Mc 6:31}] lieu désert, et r.-vs. un peu ;
\item[\vref{Mc 14:41}] Dormez mntnt, et r.-vs. ! C'est assez !
\item[\vref{Hé 4:10}] son repos, se r. aussi de ses
\item[\vref{1 Pi 1:21}] et votre espérance r. sur Dieu.
\item[\vref{1 Pi 4:14}] l'Esprit de Dieu r. sur vs., lequel
\item[\vref{Ap 14:13}] afin qu'ils se r. de leurs travaux,
\end{listverse}

\ConcordanceEntry{Repousser}
\vspace{-2mm}
\begin{listverse}
\item[\vref{Jg 1:34}] Les Amoréens r. les fils de
\item[\vref{2 Ch 6:42}] Yahweh Dieu, ne r. pas la face
\item[\vref{Jé 30:17}] ils t'appellent la r., cette Sion que
\item[\vref{Ac 7:27}] son prochain le r., en disant : Qui
\end{listverse}

\ConcordanceEntry{Reprendre}
\vspace{-2mm}
\begin{listverse}
\item[\vref{Mt 16:22}] mit à le r., en lui disant :
\item[\vref{Mt 18:15}] toi, va, et r.-le entre toi
\item[\vref{Mc 8:32}] à part, se mit à le r.
\item[\vref{Lu 17:3}] péché contre toi, r.-le ; et s'il
\item[\vref{1 Co 15:36}] tu sèmes ne r. pas vie s'il
\item[\vref{1 Ti 5:1}] Ne r. pas rudement le vieillard, mais exhorte-le
\item[\vref{1 Ti 5:20}] R. publiquement ceux qui pèchent, afin que
\item[\vref{2 Ti 4:2}] favorable ou non. R., censure, exhorte avec
\item[\vref{Tit 1:13}] véritable. C'est pourquoi r.-les vivement, afin
\item[\vref{Hé 12:5}] ne perds pas courage lorsqu'il te r. ;
\item[\vref{Ap 3:19}] Moi, je r. et je châtie ts ceux que
\end{listverse}

\ConcordanceEntry{Représentation}
\vspace{-2mm}
\begin{listverse}
\item[\vref{No 12:8}] ni ds aucune r. de Yahweh. Pourquoi
\item[\vref{Ez 1:28}] vision de la r. de la gloire
\end{listverse}

\ConcordanceEntry{Réprimande}
\vspace{-2mm}
\begin{listverse}
\item[\vref{Pr 1:30}] conseil ; ils ont rejeté ttes mes r.
\item[\vref{Pr 13:18}] qui garde la r. est honoré.
\item[\vref{Pr 15:5}] garde à la r. agit avec prudence.
\item[\vref{Pr 15:32}] qui écoute la r. s'acquiert du sens.
\item[\vref{Pr 27:5}] Une r. ouverte vaut mieux qu'un amour caché.
\item[\vref{Ec 7:5}] mieux entendre la r. du sage, que
\item[\vref{Na 1:4}] Il r. la mer et la fait tarir,
\end{listverse}

\ConcordanceEntry{Réprimander}
\vspace{-2mm}
\begin{listverse}
\item[\vref{Ge 37:10}] Son père le r., et lui dit :
\item[\vref{Mc 8:33}] regardant ses disciples, r. Pierre en lui
\item[\vref{Lu 9:55}] eux et les r. fortement, en lr.
\end{listverse}

\ConcordanceEntry{Reproche}
\vspace{-2mm}
\begin{listverse}
\item[\vref{Ps 51:6}] ta sentence, sans r. ds ton jugement.
\item[\vref{Mt 11:20}] à faire des r. aux villes où
\item[\vref{Lu 1:6}] ttes les ordonnances du Seign., sans r.
\item[\vref{Ph 2:15}] vs. soyez sans r. et purs, des
\item[\vref{Ph 3:6}] l'égard de la loi, étant sans r.
\item[\vref{1 Th 5:23}] soient conservés sans r. lors de la
\item[\vref{Ja 1:5}] libéralement, et sans r., et elle lui
\end{listverse}

\ConcordanceEntry{Reptile}
\vspace{-2mm}
\begin{listverse}
\item[\vref{Ge 1:20}] tte abondance des r. vivants ; et qu'il
\item[\vref{Lé 11:20}] et tt r. volant qui marche sur quatre pattes
\item[\vref{Lé 11:41}] Tout r. dc qui rampe sur la terre
\item[\vref{Ez 8:10}] de figures de r. et de bêtes
\item[\vref{Ac 10:12}] bêtes sauvages, les r. et les oiseaux
\item[\vref{Ro 1:23}] oiseaux, et des quadrupèdes, et des r.
\end{listverse}

\ConcordanceEntry{Répudier}
\vspace{-2mm}
\begin{listverse}
\item[\vref{Lé 21:7}] point une fem. r. par son mari,
\item[\vref{Es 54:6}] qui a été r., dit ton Dieu.
\item[\vref{Mt 1:19}] proposa de la r. secrètement.
\item[\vref{Mt 5:32}] que celui qui r. sa fem., si
\item[\vref{Mt 19:3}] un hom. de r. sa fem. pour
\end{listverse}

\ConcordanceEntry{Réputation}
\vspace{-2mm}
\begin{listverse}
\item[\vref{De 22:19}] répandu une mauvaise r. sur une vierge
\item[\vref{Pr 10:7}] bénédiction, mais la r. des méchants tombe
\item[\vref{Ec 7:1}] Une bonne r. vaut mieux que
\item[\vref{Ez 22:5}] toi, infâme de r., et remplie de
\item[\vref{Ap 3:1}] Tu as la r. d'être vivant, mais
\end{listverse}

\ConcordanceEntry{Résister}
\vspace{-2mm}
\begin{listverse}
\item[\vref{Jos 23:9}] n'a pu vs. r. jusqu'à ce jour.
\item[\vref{Ps 1:5}] les méchants ne r. pas ds le
\item[\vref{Ec 4:12}] deux peuvent lui r. ; et la corde
\item[\vref{Da 10:13}] de Perse m'a r. vingt et un
\item[\vref{Mt 5:39}] vs. dis : Ne r. pas au méchant.
\item[\vref{Ac 6:10}] ne pouvaient pas r. à la sagesse
\item[\vref{Ro 9:19}] celui qui peut r. à sa volonté ?
\item[\vref{Ro 13:2}] pourquoi celui qui r. à l'autorité résiste
\item[\vref{Ga 2:11}] Antioche, je lui r. en face parce
\item[\vref{Ep 6:13}] afin de pouvoir r. ds le mauvais
\item[\vref{2 Ti 3:8}] et Jambrès ont r. à Moïse, ceux-ci
\item[\vref{Hé 12:4}] n'avez pas encore r. jusqu'à répandre votre
\item[\vref{Ja 4:7}] dc à Dieu ; r. au diable, et
\item[\vref{1 Pi 5:5}] parce que Dieu r. aux orgueilleux, mais
\item[\vref{1 Pi 5:9}] R.-lui dc en demeurant fermes ds
\end{listverse}

\ConcordanceEntry{Résolution}
\vspace{-2mm}
\begin{listverse}
\item[\vref{Jg 5:16}] grandes furent les r. du cœur !
\item[\vref{Ep 1:11}] prédestinés, suivant la r. de celui qui
\item[\vref{Hé 6:17}] immuable de sa r., il y a
\end{listverse}

\ConcordanceEntry{Respect}
\vspace{-2mm}
\begin{listverse}
\item[\vref{Mt 21:37}] Ils auront du r. pour mon fils.
\item[\vref{Hé 12:28}] soyons agréables avec r. et avec crainte,
\item[\vref{1 Pi 3:15}] douceur et avec r., à quiconque vs.
\end{listverse}

\ConcordanceEntry{Respecter}
\vspace{-2mm}
\begin{listverse}
\item[\vref{Lu 18:2}] et qui ne r. personne.
\item[\vref{Ep 5:33}] que la fem. r. son mari.
\item[\vref{Hé 12:9}] ns. les avons r., ne serons-ns. pas
\end{listverse}

\ConcordanceEntry{Respirer}
\vspace{-2mm}
\begin{listverse}
\item[\vref{Ge 8:21}] Et Yahweh r. une odeur d'apaisement,
\item[\vref{Ps 150:6}] tt ce qui r. loue Yahweh ! Louez
\item[\vref{Es 11:3}] Il r. la crainte de Yahweh, il ne
\item[\vref{Ac 9:1}] Or Saul, r. encore la menace et le carnage
\end{listverse}

\ConcordanceEntry{Resplendir}
\vspace{-2mm}
\begin{listverse}
\item[\vref{De 33:2}] Séir, il a r. de la montagne
\item[\vref{Es 9:1}] et la lumière r. sur ceux qui
\item[\vref{Ez 43:2}] et la terre r. de sa gloire.
\item[\vref{Mt 13:43}] Alors les justes r. com. le soleil
\item[\vref{Ac 9:3}] coup une lumière r. du ciel com.
\item[\vref{2 Co 4:6}] que la lumière r. des ténèbres, est
\end{listverse}

\ConcordanceEntry{Ressemblance}
\vspace{-2mm}
\begin{listverse}
\item[\vref{Ge 1:26}] image, selon notre r., et qu'il domine
\item[\vref{Ge 5:3}] fils à sa r., selon son image,
\item[\vref{Ja 3:9}] maudissons les hommes faits à la r. de Dieu.
\end{listverse}

\ConcordanceEntry{Ressembler}
\vspace{-2mm}
\begin{listverse}
\item[\vref{Ps 115:8}] Ils lr. r., ceux qui les fabriquent, ts ceux
\item[\vref{Mt 6:8}] Ne lr. r. dc pas ; car votre Père sait
\item[\vref{Mt 19:14}] cieux est pour ceux qui lr. r.
\item[\vref{Ap 1:13}] d'or, quelqu'un qui r. à un fils
\end{listverse}

\ConcordanceEntry{Ressusciter}
\vspace{-2mm}
\begin{listverse}
\item[\vref{Mt 10:8}] purs les lépreux, r. les morts, chassez
\item[\vref{Mt 16:21}] mort, et qu'il r. le troisième jour.
\item[\vref{Mt 27:52}] corps des saints qui étaient morts r.
\item[\vref{Mc 9:10}] que c'était que r. des morts.
\item[\vref{Lu 24:34}] Seign. est véritablement r., et il est
\item[\vref{Jn 2:22}] pourquoi, lorsqu'il fut r. des morts, ses
\item[\vref{Jn 5:21}] com. le Père r. les morts et
\item[\vref{Jn 5:29}] fait le bien r. pour la vie,
\item[\vref{Jn 6:40}] pourquoi je le r. au dernier jour.
\item[\vref{Ac 2:24}] Mais Dieu l'a r., ayant brisé les
\item[\vref{Ac 17:31}] certaine en le r. des morts.
\item[\vref{Ac 26:8}] incroyable que Dieu r. les morts ?
\item[\vref{Ac 26:23}] souffrirait, et que r. le premier d'entre
\item[\vref{Ro 6:9}] sachant que Christ r. des morts ne
\item[\vref{Ro 8:11}] celui qui a r. Jésus d'entre les
\item[\vref{1 Co 6:14}] Dieu qui a r. le Seign., ns.
\item[\vref{1 Co 15:17}] Christ n'est pas r., votre foi est
\item[\vref{1 Co 15:20}] mntnt Christ est r. des morts, il
\item[\vref{1 Co 15:35}] Comment les morts r.-ils, et avec
\item[\vref{1 Co 15:42}] semé corruptible, il r. incorruptible ;
\item[\vref{2 Co 1:9}] en Dieu qui r. les morts ;
\item[\vref{2 Co 5:15}] est mort et r. pour eux.
\item[\vref{Ga 1:1}] Père, qui l'a r. des morts,
\item[\vref{Ep 1:20}] qnd il l'a r. des morts et
\item[\vref{Col 2:12}] aussi vs. êtes r. ensemble par la
\item[\vref{1 Th 4:16}] morts en Christ r. premièrement.
\item[\vref{Hé 11:19}] pouvait mm le r. d'entre les morts ;
\end{listverse}

\ConcordanceEntry{Restaurer}
\vspace{-2mm}
\begin{listverse}
\item[\vref{Ru 4:15}] Et il r. ton âme, et sera le soutien
\item[\vref{Ps 19:8}] est parfaite, elle r. l'âme ; le témoignage
\item[\vref{Ps 23:3}] Il r. mon âme, et me conduit ds
\item[\vref{Pr 25:13}] la moisson, il r. l'âme de son
\item[\vref{Es 57:18}] et je le r., lui et ceux
\end{listverse}

\ConcordanceEntry{Reste}
\vspace{-2mm}
\begin{listverse}
\item[\vref{2 Ch 31:10}] et cette grande quantité est le r.
\item[\vref{Es 1:9}] laissé un petit r., qui est mm
\item[\vref{Es 10:22}] la mer, un r. seulement se convertira ;
\item[\vref{Es 11:11}] pour acquérir le r. de son peuple
\item[\vref{Jé 23:3}] je rassemblerai le r. de mes brebis
\item[\vref{Jé 50:20}] je pardonnerai au r. que j'aurai fait
\item[\vref{Mi 2:12}] ramasserai entièrement le r. d'Israël, et le
\item[\vref{Ac 15:17}] afin que le r. des hommes recherche
\item[\vref{Ac 28:19}] César, n'ayant du r. aucun dessein d'accuser
\item[\vref{Ro 9:27}] mer, un petit r. seulement sera sauvé.
\item[\vref{Ro 11:5}] temps présent un r., selon l'élection de
\item[\vref{Ap 3:2}] et affermis le r. qui va mourir ;
\end{listverse}

\ConcordanceEntry{Rester}
\vspace{-2mm}
\begin{listverse}
\item[\vref{Ge 22:5}] à ses serviteurs : R. ici avec l'âne ;
\item[\vref{Ge 24:55}] la jeune fille r. avec ns. quelques
\item[\vref{Ex 14:14}] vs. et vs. r. tranquilles.
\item[\vref{1 R 19:10}] l'épée ; je suis r., moi seul et
\item[\vref{Mt 24:2}] vérité, il ne r. pas ici pierre
\item[\vref{Lu 24:29}] en lui disant : R. avec ns., car
\item[\vref{Ph 1:25}] et que je r. avec vs. ts,
\item[\vref{1 Th 4:15}] qui vivrons et r. pour l'avènement du
\item[\vref{Hé 4:9}] Il r. dc un repos au peuple de
\item[\vref{Hé 10:26}] vérité, il ne r. plus de sacrifice
\item[\vref{1 Pi 4:2}] temps qui lui r. à vivre ds
\end{listverse}

\ConcordanceEntry{Restituer}
\vspace{-2mm}
\begin{listverse}
\item[\vref{Lé 5:24}] faussement. Il le r. totalement, et il
\item[\vref{Es 42:22}] il n'y a personne qui dise : R. !
\end{listverse}

\ConcordanceEntry{Résurrection}
\vspace{-2mm}
\begin{listverse}
\item[\vref{Mt 22:23}] a pas de r., vinrent auprès de
\item[\vref{Mc 12:25}] Car à la r. des morts, ils ne prendront pas
\item[\vref{Lu 20:35}] et à la r. des morts, ne
\item[\vref{Jn 11:25}] Je suis la r. et la vie.
\item[\vref{Ac 1:22}] soit témoin avec ns. de sa r.
\item[\vref{Ac 2:31}] c'est la r. du Christ qu'il a prévue et
\item[\vref{Ac 4:33}] force à la r. du Seign. Jésus ;
\item[\vref{Ac 23:6}] et de la r. des morts que
\item[\vref{Ac 24:15}] y aura une r. des justes et
\item[\vref{Ro 1:4}] sainteté, par sa r. d'entre les morts,
\item[\vref{Ro 6:5}] aussi par la conformité à sa r.,
\item[\vref{1 Co 15:12}] a pas de r. des morts ?
\item[\vref{1 Co 15:42}] mm à la r. des morts. Le
\item[\vref{Ph 3:10}] puissance de sa r., et la communion
\item[\vref{2 Ti 2:18}] disant que la r. est déjà arrivée,
\item[\vref{Hé 11:19}] le recouvra par une espèce de r.
\item[\vref{Hé 11:35}] moyen de la r. ; et d'autres furent
\item[\vref{Ap 20:5}] ans soient accomplis. C'est la première r.
\item[\vref{Ap 20:6}] à la première r. ! La seconde mort
\end{listverse}

\ConcordanceEntry{Rétablir}
\vspace{-2mm}
\begin{listverse}
\item[\vref{Job 22:23}] Tout-Puissant, tu seras r.. Chasse l'iniquité loin
\item[\vref{Es 14:1}] et il les r. ds lr. pays ;
\item[\vref{Mt 17:11}] viendra premièrement et r. ttes choses.
\item[\vref{Ac 1:6}] temps-ci que tu r. le royaume d'Israël ?
\end{listverse}

\ConcordanceEntry{Retenir}
\vspace{-2mm}
\begin{listverse}
\item[\vref{Pr 4:4}] Que ton cœur r. mes paroles ; garde
\item[\vref{Pr 27:16}] qui veut la r., retient le vent,
\item[\vref{Ec 8:8}] pour pouvoir le r., il n'a aucune
\item[\vref{Jé 2:13}] citernes crevassées qui ne peuvent pas r. l'eau.
\item[\vref{Lu 4:42}] ils voulaient le r., afin qu'il ne
\item[\vref{Jn 20:23}] qui vs. les r., ils lr. seront
\item[\vref{Ac 2:24}] possible qu'il soit r. par elle.
\item[\vref{1 Th 5:21}] Eprouvez ttes choses ; r. ce qui est
\item[\vref{2 Th 2:6}] ce qui le r., afin qu'il soit
\item[\vref{Phm 1:13}] Je voulais le r. auprès de moi,
\item[\vref{Hé 3:6}] pourvu que ns. r. fermement jusqu'à la
\item[\vref{Hé 10:23}] R. fermement la profession de notre espérance,
\item[\vref{Ap 2:25}] Mais r. ce que vs. avez, jusqu'à ce
\end{listverse}

\ConcordanceEntry{Retirer}
\vspace{-2mm}
\begin{listverse}
\item[\vref{Ge 8:3}] les eaux se r. sans interruption de
\item[\vref{Ex 4:6}] puis il la r. ; et voici, sa
\item[\vref{Jg 16:20}] ne savait pas que Yahweh s'était r. de lui.
\item[\vref{1 S 16:14}] de Yahweh se r. de Saül, et
\item[\vref{Job 21:14}] disent à Dieu : R.-toi de ns. ;
\item[\vref{Ps 113:7}] la poussière, et r. le pauvre de
\item[\vref{Es 1:4}] ils se sont r. en arrière.
\item[\vref{Mt 4:10}] Jésus lui dit : R.-toi, Satan ! Car
\item[\vref{Mt 4:12}] en prison, se r. ds la Galilée.
\item[\vref{Mt 7:23}] ai jamais connus, r.-vs. de moi,
\item[\vref{Lu 5:8}] lui disant : Seign., r.-toi de moi,
\item[\vref{Hé 10:38}] si quelqu'un se r., mon âme ne
\end{listverse}

\ConcordanceEntry{Retomber}
\vspace{-2mm}
\begin{listverse}
\item[\vref{Lé 20:9}] père ou sa mère. Son sang r. sur lui.
\item[\vref{Ac 5:28}] vs. voulez faire r. sur ns. le
\item[\vref{Ac 18:6}] Que votre sang r. sur votre tête !
\end{listverse}

\ConcordanceEntry{Retourner}
\vspace{-2mm}
\begin{listverse}
\item[\vref{Ge 3:19}] ce que tu r. ds la terre,
\item[\vref{Ge 8:12}] colombe qui ne r. plus à lui.
\item[\vref{Ge 16:9}] Yahweh lui dit : R. vers ta maîtresse
\item[\vref{No 35:28}] le meurtrier pourra r. ds sa possession.
\item[\vref{De 4:30}] derniers jours, tu r. à Yahweh, ton
\item[\vref{Ps 114:5}] toi, Jourdain, pour r. en arrière ?
\item[\vref{Es 55:7}] pensées ; et qu'il r. à Yahweh, qui
\item[\vref{Es 55:11}] bouche, elle ne r. point vers moi
\item[\vref{Jé 3:1}] le premier mari r.-t-il encore vers
\item[\vref{Os 2:9}] irai, et je r. vers mon premier
\item[\vref{Mt 12:44}] il dit : Je r. ds ma maison,
\item[\vref{Mt 21:18}] matin, com. il r. à la ville,
\item[\vref{1 Co 7:5}] mais après cela r. ensemble, de peur
\item[\vref{Ga 4:9}] de Dieu, comment r.-vs. encore à
\item[\vref{Hé 11:15}] ils auraient eu le temps d'y r.
\item[\vref{1 Pi 2:25}] mntnt vs. êtes r. vers le Pasteur
\item[\vref{2 Pi 2:22}] Le chien est r. à ce qu'il
\end{listverse}

\ConcordanceEntry{Retraite}
\vspace{-2mm}
\begin{listverse}
\item[\vref{2 S 22:3}] sauve, ma haute r. et mon refuge.
\item[\vref{Ps 46:8}] Jacob est pour ns. une haute r.. Sélah.
\item[\vref{Ps 48:4}] ds ses palais pour une haute r.
\item[\vref{Pr 18:10}] court et y trouve une haute r.
\item[\vref{Ap 18:2}] démons, et la r. de tt esprit
\end{listverse}

\ConcordanceEntry{Retrancher}
\vspace{-2mm}
\begin{listverse}
\item[\vref{Ex 12:15}] cette personne-là sera r. d'Israël.
\item[\vref{Pr 2:22}] les méchants seront r. de la terre,
\item[\vref{Ec 3:14}] rien à en r., et Dieu le
\item[\vref{Es 53:8}] il a été r. de la terre
\item[\vref{Da 9:26}] le Messie sera r., mais non pas
\item[\vref{Jn 15:2}] Il r. tt sarment qui est en moi
\item[\vref{Ro 11:22}] bonté ; car autrement, tu seras aussi r.
\item[\vref{1 Co 5:2}] cette action soit r. du milieu de
\item[\vref{Ap 22:19}] et si quelqu'un r. qq chose des
\end{listverse}

\ConcordanceEntry{Rétribution}
\vspace{-2mm}
\begin{listverse}
\item[\vref{De 32:35}] vengeance et la r., le temps où
\item[\vref{Ps 91:8}] tu verras la r. des méchants.
\item[\vref{Pr 11:31}] la terre sa r., combien plus le
\item[\vref{Es 35:4}] vengeance viendra, la r. de Dieu ; il
\item[\vref{Es 40:10}] lui, et ses r. sont dvt lui.
\item[\vref{Hé 2:2}] tte désobéissance a reçu une juste r.,
\item[\vref{Ap 22:12}] vitesse, et ma r. est avec moi
\end{listverse}

\ConcordanceEntry{Retrouver}
\vspace{-2mm}
\begin{listverse}
\item[\vref{Ec 11:1}] car avec le temps tu le r.
\item[\vref{Mt 10:39}] vie pour l'amour de moi la r.
\item[\vref{Lu 15:24}] mais il est r.. Et ils commencèrent
\end{listverse}

\ConcordanceEntry{Retsin}
\vspace{-2mm}
\begin{listverse}
\item[\vref{2 R 15:37}] envoyer contre Juda, R., roi de Syrie,
\item[\vref{Es 7:4}] la colère de R. et de la
\end{listverse}

\ConcordanceEntry{Réunion}
\vspace{-2mm}
\begin{listverse}
\item[\vref{2 Th 2:1}] Jésus-Christ et notre r. en lui, mes
\end{listverse}

\ConcordanceEntry{Réussir}
\vspace{-2mm}
\begin{listverse}
\item[\vref{Ge 24:21}] si Yahweh faisait r. son voyage ou
\item[\vref{Ge 24:56}] Yahweh a fait r. mon voyage ; laissez-moi
\item[\vref{No 14:41}] le commandement de Yahweh ? Cela ne r. point.
\item[\vref{Jos 1:8}] tes entreprises, c'est alors que tu r.
\item[\vref{1 S 18:5}] envoyé par Saül, r. partout où il
\item[\vref{Ps 1:3}] Et ainsi tt ce qu'il fera r.
\end{listverse}

\ConcordanceEntry{Rêve}
\vspace{-2mm}
\begin{listverse}
\item[\vref{Ps 126:1}] étions com. ceux qui font un r.
\end{listverse}

\ConcordanceEntry{Réveiller (se)}
\vspace{-2mm}
\begin{listverse}
\item[\vref{Ge 9:24}] qnd Noé se r. de son vin,
\item[\vref{2 R 4:31}] en disant : L'enfant ne s'est pas r.
\item[\vref{Esd 1:5}] ceux dont Dieu r. l'esprit, afin de
\item[\vref{Ps 3:6}] m'endors, je me r., car Yahweh me
\item[\vref{Ps 44:24}] pourquoi dors-tu Seign. ? R.-toi ! Ne ns.
\item[\vref{Da 12:2}] la terre se r., les uns pour
\item[\vref{Ag 1:14}] Et Yahweh r. l'esprit de Zorobabel,
\item[\vref{Za 2:13}] Car il s'est r. de sa demeure
\item[\vref{Mt 25:7}] ces vierges se r. et préparèrent leurs
\item[\vref{Mc 4:39}] S'étant r., il menaça le vent, et dit
\item[\vref{Jn 11:11}] Lazare dort, mais je vais le r.
\item[\vref{Ac 12:7}] la prison. L'Ange r. Pierre en le
\item[\vref{Ac 16:27}] Le geôlier se r., et, voyant les
\item[\vref{Ro 13:11}] heure de ns. r. du sommeil ; car
\item[\vref{1 Co 15:34}] R.-vs. pour vivre justement, et ne
\item[\vref{Ep 5:14}] il est dit : R.-toi, toi qui
\item[\vref{2 Pi 1:13}] que je vs. r. par des avertissements,
\end{listverse}

\ConcordanceEntry{Révélation}
\vspace{-2mm}
\begin{listverse}
\item[\vref{1 S 9:15}] avait fait une r. à Samuel, en
\item[\vref{Ps 119:130}] La r. de tes paroles éclaire, elle donne
\item[\vref{1 Co 14:6}] parle pas par r., ou par science,
\item[\vref{1 Co 14:26}] langue, ou une r., ou une interprétation,
\item[\vref{2 Co 12:1}] visions et aux r. du Seign.
\item[\vref{Ga 1:12}] mais par la r. de Jésus-Christ.
\item[\vref{Ga 2:2}] fut d'après une r. que j'y montai.
\item[\vref{Ep 1:17}] sagesse et de r., ds ce qui
\item[\vref{Ep 3:3}] comment par r. il m'a fait
\item[\vref{Ap 1:1}] La r. de Jésus-Christ, que Dieu lui a
\end{listverse}

\ConcordanceEntry{Révéler}
\vspace{-2mm}
\begin{listverse}
\item[\vref{Ge 35:7}] que Dieu s'était r. à lui lorsqu'il
\item[\vref{De 29:29}] Dieu ; les choses r. sont à ns.
\item[\vref{1 S 3:7}] ne lui avait pas encore été r.
\item[\vref{Es 53:1}] le bras de Yahweh a-t-il été r. ?
\item[\vref{Mt 11:25}] tu les as r. aux petits enfants.
\item[\vref{Mt 16:17}] sang qui t'ont r. cela, mais mon
\item[\vref{Ro 1:17}] qu'en lui est r. la justice de
\item[\vref{Ro 8:18}] gloire à venir qui doit être r. en ns.
\item[\vref{1 Co 2:10}] ns. les a r. par son Esprit.
\item[\vref{2 Co 12:4}] n'est pas permis à l'hom. de r.
\item[\vref{Ga 1:16}] de r. en moi son Fils afin que
\item[\vref{1 Pi 1:5}] prêt à être r. ds les derniers
\end{listverse}

\ConcordanceEntry{Revenir}
\vspace{-2mm}
\begin{listverse}
\item[\vref{No 10:36}] posait, il disait : R. Yahweh, aux dix
\item[\vref{De 30:2}] si tu r. à Yahweh, ton Dieu, et si
\item[\vref{1 S 7:3}] disant : Si vs. r. à Yahweh de
\item[\vref{2 R 23:25}] qui, com. lui, r. à Yahweh de
\item[\vref{Ps 90:3}] Tu fais r. l'hom. à la poussière, et tu
\item[\vref{Pr 3:28}] prochain : Va, et r., demain je te
\item[\vref{Ec 1:7}] étaient partis, pour r. à la mer.
\item[\vref{Es 9:12}] le peuple ne r. pas à celui
\item[\vref{Jé 3:12}] nord, et dis : R., Israël, l'infidèle, dit
\item[\vref{Jé 31:18}] pas dompté. Fais-moi r. et je reviendrai,
\item[\vref{Os 5:4}] permettront pas de r. à lr. Dieu,
\item[\vref{Mal 3:7}] avez pas gardés. R. à moi, et
\item[\vref{Mt 25:19}] de ces serviteurs r. et lr. fit
\item[\vref{Jn 14:3}] une place, je r., et je vs.
\end{listverse}

\ConcordanceEntry{Revêtir}
\vspace{-2mm}
\begin{listverse}
\item[\vref{Ge 3:21}] de peaux, et il les en r.
\item[\vref{Ge 41:42}] il le fit r. d'habits de fin
\item[\vref{Es 59:17}] Car il se r. de la justice
\item[\vref{Mc 15:17}] Ils le r. d'une robe de pourpre, et posèrent
\item[\vref{Lu 24:49}] que vs. soyez r. de la puissance
\item[\vref{Ro 13:14}] Mais soyez r. du Seign. Jésus-Christ,
\item[\vref{2 Co 5:4}] dépouillés, mais d'être r., afin que ce
\item[\vref{Ga 3:27}] été baptisés en Christ, vs. avez r. Christ.
\item[\vref{1 Pi 5:5}] pour les autres, r.-vs. d'humilité ; parce
\item[\vref{Ap 19:8}] donné de se r. d'un fin lin
\end{listverse}

\ConcordanceEntry{Revivre}
\vspace{-2mm}
\begin{listverse}
\item[\vref{Job 14:14}] Si l'hom. meurt, r.-t-il ? J'attendrai dc
\item[\vref{Es 26:14}] morts, ils ne r. plus, ils sont
\item[\vref{Ro 7:9}] péché commença à r., et moi je
\end{listverse}

\ConcordanceEntry{Révolte}
\vspace{-2mm}
\begin{listverse}
\item[\vref{De 13:5}] a parlé de r. contre Yahweh, votre
\item[\vref{Jé 28:16}] as parlé de r. contre Yahweh.
\end{listverse}

\ConcordanceEntry{Révolter}
\vspace{-2mm}
\begin{listverse}
\item[\vref{Ge 14:4}] et la treizième année, ils s'étaient r.
\item[\vref{2 R 18:7}] entreprises. Il se r. contre le roi
\item[\vref{Hé 3:12}] point de se r. contre le Dieu
\end{listverse}

\ConcordanceEntry{Riche}
\vspace{-2mm}
\begin{listverse}
\item[\vref{Ge 13:2}] Abram était très r. en bétail, en
\item[\vref{Ex 30:15}] Le r. n'augmentera rien, et le pauvre ne
\item[\vref{Job 34:19}] regarde pas les r. pour les préférer
\item[\vref{Ps 45:13}] et les plus r. des peuples te
\item[\vref{Pr 13:7}] Tel fait le r. et n'a rien
\item[\vref{Pr 22:2}] Le r. et le pauvre se rencontrent ; celui
\item[\vref{Pr 28:11}] L'hom. r. pense être sage, mais le pauvre
\item[\vref{Es 53:9}] été avec le r., quoiqu'il n'ait point
\item[\vref{Jé 9:23}] et que le r. ne se glorifie
\item[\vref{Mt 19:23}] en vérité, un r. entrera difficilement ds
\item[\vref{Lu 1:53}] affamés, et il a renvoyé les r. à vide.
\item[\vref{Lu 18:23}] tt triste, car il était extrêmement r.
\item[\vref{Lu 21:1}] il vit des r. qui mettaient leurs
\item[\vref{Ro 10:12}] Seign., qui est r. pour ts ceux
\item[\vref{2 Co 8:9}] Jésus-Christ qui, étant r., s'est fait pauvre
\item[\vref{Ep 2:4}] Dieu, qui est r. en miséricorde, à
\item[\vref{1 Ti 6:18}] bien, qu'ils soient r. en bonnes œuvres,
\item[\vref{Ja 2:5}] pour qu'ils soient r. ds la foi,
\item[\vref{Ja 5:1}] A vs. mntnt, r. ! Pleurez et gémissez
\item[\vref{Ap 2:9}] quoique tu sois r., et le blasphème
\item[\vref{Ap 3:17}] dis : Je suis r., je suis ds
\item[\vref{Ap 3:18}] que tu deviennes r.; et des vêtements
\end{listverse}

\ConcordanceEntry{Richesse}
\vspace{-2mm}
\begin{listverse}
\item[\vref{Ge 14:11}] dc ttes les r. de Sodome et
\item[\vref{Ge 14:21}] personnes, et prends pour toi les r.
\item[\vref{1 R 3:11}] vie, ni les r., ni la mort
\item[\vref{Job 22:25}] ton or, et l'argent de tes r.
\item[\vref{Ps 49:7}] se glorifient de l'abondance de leurs r.
\item[\vref{Ps 62:11}] vains ; qnd les r. abonderont, n'y mettez
\item[\vref{Pr 11:4}] Les r. ne servent à rien au jour
\item[\vref{Pr 11:28}] confie ds ses r. tombera, mais les
\item[\vref{Pr 13:11}] Les r. provenues de la fraude seront diminuées,
\item[\vref{Pr 13:22}] fils, mais les r. du pécheur sont
\item[\vref{Pr 23:5}] Car certainement, la r. se fera des
\item[\vref{Ec 5:12}] c'est que des r. sont conservées à
\item[\vref{Jé 17:11}] qui acquiert des r. sans observer la
\item[\vref{Mt 13:22}] la séduction des r. étouffent la parole
\item[\vref{Lu 16:9}] amis avec les r. injustes, afin que
\item[\vref{Ro 11:33}] profondeur de la r., et de la
\item[\vref{Ep 1:7}] offenses, selon les r. de sa grâce,
\item[\vref{Ep 2:7}] venir les immenses r. de sa grâce
\item[\vref{Ep 3:8}] les Gentils les r. incompréhensibles de Christ,
\item[\vref{Ph 4:19}] besoin selon ses r., avec gloire en
\item[\vref{1 Ti 6:17}] ds l'incertitude des r., mais ds le
\item[\vref{Ja 5:2}] Vos r. sont pourries, et vos vêtements sont
\item[\vref{Ap 5:12}] de recevoir puissance, r., sagesse, force, honneur,
\end{listverse}

\ConcordanceEntry{Rideau}
\vspace{-2mm}
\begin{listverse}
\item[\vref{Ex 26:36}] tu feras un r. de pourpre, d'écarlate,
\end{listverse}

\ConcordanceEntry{Rien}
\vspace{-2mm}
\begin{listverse}
\item[\vref{Ge 11:6}] travailler ; et mntnt r. ne les empêchera
\item[\vref{2 S 24:24}] ne me coûtent r.. Ainsi, David acheta
\item[\vref{Es 41:24}] Voici, vs. n'êtes r., et votre œuvre
\item[\vref{Ag 2:3}] pas com. un r. dvt vos yeux,
\item[\vref{1 Co 13:2}] pas la charité, je ne suis r.
\item[\vref{2 Co 6:10}] plusieurs ; com. n'ayant r. et toutefois possédant
\item[\vref{Ph 2:3}] Ne faites r. par esprit de
\item[\vref{Ph 4:6}] vs. inquiétez de r., mais en ttes
\item[\vref{1 Ti 6:7}] Car ns. n'avons r. apporté ds le
\item[\vref{Ap 2:10}] Ne crains r. des choses que
\item[\vref{Ap 3:17}] n'ai besoin de r. ; mais tu ne
\item[\vref{Ap 21:27}] n'entrera chez elle r. de souillé, ni
\end{listverse}

\ConcordanceEntry{Rimmon}
\vspace{-2mm}
\begin{listverse}
\item[\vref{2 R 5:18}] la maison de R. pour s'y prosterner
\end{listverse}

\ConcordanceEntry{Rire (le)}
\vspace{-2mm}
\begin{listverse}
\item[\vref{Ge 21:6}] donné de quoi r. ; ts ceux qui
\item[\vref{Ec 2:2}] dit concernant le r. : Il est insensé !
\item[\vref{Ja 4:9}] larmes ; que votre r. se change en
\end{listverse}

\ConcordanceEntry{Rire}
\vspace{-2mm}
\begin{listverse}
\item[\vref{Ge 18:12}] Et Sara r. en elle-mm, et dit : Maintenant que
\item[\vref{Ge 21:9}] Sara vit r. le fils qu'Agar, l'Egyptienne, avait enfanté
\item[\vref{Ps 2:4}] les cieux se r. d'eux, le Seign.
\item[\vref{Ps 37:13}] Le Seign. se r. de lui, car
\item[\vref{Ps 59:9}] Yahweh, tu te r. d'eux, tu te
\item[\vref{Pr 1:26}] moi aussi je r. qnd vs. serez
\item[\vref{Ec 3:4}] un temps pour r. ; un temps pour
\end{listverse}

\ConcordanceEntry{Rivalité}
\vspace{-2mm}
\begin{listverse}
\item[\vref{Lé 18:18}] pour exciter une r. en découvrant sa
\end{listverse}

\ConcordanceEntry{Robe}
\vspace{-2mm}
\begin{listverse}
\item[\vref{Ex 28:31}] feras aussi la r. de l'éphod entièrement
\item[\vref{Es 6:1}] pans de sa r. remplissaient le temple.
\item[\vref{Lu 15:22}] la plus belle r. et revêtez-le, et
\item[\vref{Jud 1:23}] haïssez mm la r. souillée par la
\item[\vref{Ap 1:13}] vêtu d'une longue r., et ayant une
\item[\vref{Ap 6:11}] à chacun des r. blanches, et il
\item[\vref{Ap 7:14}] blanchi leurs longues r. ds le sang
\item[\vref{Ap 22:14}] qui lavent leurs r. afin d'avoir droit
\end{listverse}

\ConcordanceEntry{Roboam}
\vspace{-2mm}
\begin{listverse}
\item[\vref{1 R 11:43}] son père. Et R., son fils, régna
\item[\vref{1 R 14:21}] R., fils de Salomon, régna en Juda.
\end{listverse}

\ConcordanceEntry{Roc}
\vspace{-2mm}
\begin{listverse}
\item[\vref{Ps 40:3}] pieds sur un r. et a assuré
\item[\vref{Mt 7:24}] a bâti sa maison sur le r.
\item[\vref{Lu 6:48}] fondement sur le r.. Mais une inondation
\end{listverse}

\ConcordanceEntry{Rocher}
\vspace{-2mm}
\begin{listverse}
\item[\vref{Ge 49:24}] le pasteur, le r. d'Israël.
\item[\vref{Ex 17:6}] toi sur le r. d'Horeb ; et tu
\item[\vref{Ex 33:22}] un creux du r., et te couvrirai
\item[\vref{De 8:15}] de l'eau du r. le plus dur,
\item[\vref{De 32:4}] L'œuvre du r. est parfaite, car
\item[\vref{1 S 2:2}] a pas de r. tel que notre
\item[\vref{2 S 22:2}] Yahweh est mon r., ma forteresse, mon
\item[\vref{Ps 18:3}] Yahweh est mon R., ma forteresse et
\item[\vref{Ps 18:32}] qui est un r., si ce n'est
\item[\vref{Ps 28:1}] ô Yahweh ! Mon r. ! ne te rends
\item[\vref{Ps 31:3}] pour moi un R. protecteur, une forteresse,
\item[\vref{Ps 61:3}] conduis-moi sur le r. qui est trop
\item[\vref{Ps 78:35}] Dieu était lr. r., et Dieu, le
\item[\vref{Ps 89:27}] Dieu, et le R. de ma délivrance.
\item[\vref{Es 17:10}] pas souvenue du r. de ta force,
\item[\vref{Es 26:4}] perpétuité, car le R.  des siècles est
\item[\vref{Es 44:8}] a pas d'autre R., je n'en connais
\item[\vref{Es 51:1}] Yahweh ! Regardez au r. d'où vs. avez
\item[\vref{1 Co 10:4}] de l'eau du r. spirituel qui les
\item[\vref{1 Pi 2:8}] d'achoppement, et un r. de scandale ; ils
\item[\vref{Ap 6:16}] montagnes et aux r. : Tombez sur ns.,
\end{listverse}

\ConcordanceEntry{Roi}
\vspace{-2mm}
\begin{listverse}
\item[\vref{Ge 17:16}] des nations ; des r., chefs de peuples,
\item[\vref{Ex 1:8}] s'éleva un nouveau r. sur l'Egypte, qui
\item[\vref{No 23:7}] et dit : Balak, r. de Moab, m'a
\item[\vref{De 17:15}] de t'établir pour r. celui que Yahweh,
\item[\vref{Jg 17:6}] avait pas de r. en Israël. Chacun
\item[\vref{1 S 8:5}] sur ns. un r. pour ns. juger,
\item[\vref{1 S 12:12}] voyant que Nachasch, r. des fils d'Ammon,
\item[\vref{Ps 2:2}] Pourquoi les r. de la terre
\item[\vref{Ps 5:3}] Mon R. et mon Dieu ! Sois attentif à
\item[\vref{Ps 10:16}] Yahweh est R. à toujours et
\item[\vref{Ps 24:7}] éternelles, et le R. de gloire entrera !
\item[\vref{Ps 47:3}] est un grand R. sur tte la
\item[\vref{Pr 8:15}] moi règnent les r., et par moi
\item[\vref{Pr 20:8}] Le r. assis sur le trône de justice
\item[\vref{Pr 21:1}] Le cœur du r. est ds la
\item[\vref{Pr 31:4}] n'est point aux r., ce n'est point
\item[\vref{Ec 4:13}] vaut mieux qu'un r. vieux et insensé
\item[\vref{Ec 10:16}] pays dont le r. est un enfant,
\item[\vref{Ec 10:20}] maudis point le r., mm ds ta
\item[\vref{Es 32:1}] Voici, un r. régnera selon la
\item[\vref{Es 33:22}] Yahweh est notre R. ; c'est lui qui
\item[\vref{Jé 10:7}] ne te craindrait, R. des nations ? Car
\item[\vref{Jé 46:18}] vivant ! dit le R., dont le Nom
\item[\vref{Da 2:21}] qui établit les r., qui donne la
\item[\vref{Da 2:37}] Ô r., tu es le roi des rois,
\item[\vref{Os 3:4}] plusieurs jours sans r., sans chef, sans
\item[\vref{Za 9:9}] Jérus. ! Voici, ton R. vient à toi ;
\item[\vref{Mal 1:14}] suis un grand R., dit Yahweh des
\item[\vref{Mt 2:2}] Où est le R. des Juifs qui
\item[\vref{Mt 27:29}] Nous te saluons, R. des Juifs !
\item[\vref{Lu 23:2}] César, et se disant lui-mm Christ, R.
\item[\vref{Jn 6:15}] pour le faire r., se retira encore,
\item[\vref{Jn 12:13}] Béni soit le R. d'Israël qui vient
\item[\vref{Jn 19:14}] Pilate dit aux Juifs : Voilà votre R.
\item[\vref{1 Ti 1:17}] Or au R. des siècles, immortel, invisible, à Dieu
\item[\vref{1 Ti 6:15}] seul Prince, le R. des rois, et
\item[\vref{Ap 6:15}] Et les r. de la terre, et les princes,
\item[\vref{Ap 19:16}] ces mots : LE R. DES ROIS ET
\end{listverse}

\ConcordanceEntry{Romains}
\vspace{-2mm}
\begin{listverse}
\item[\vref{Jn 11:48}] lui, et les R. viendront et ils
\item[\vref{Ac 16:38}] furent effrayés en apprenant qu'ils étaient R.
\end{listverse}

\ConcordanceEntry{Rome}
\vspace{-2mm}
\begin{listverse}
\item[\vref{Ac 18:2}] de sortir de R.. Il s'approcha d'eux,
\item[\vref{Ac 19:21}] il faut aussi que je voie R.
\item[\vref{Ac 23:11}] aussi que tu rendes témoignage à R.
\item[\vref{Ac 28:16}] fûmes arrivés à R., le centenier mit
\end{listverse}

\ConcordanceEntry{Ronce}
\vspace{-2mm}
\begin{listverse}
\item[\vref{Es 5:6}] ni cultivée ; les r. et les épines
\item[\vref{Es 27:4}] me donne des r., des épines pour
\item[\vref{Es 55:13}] lieu de la r. croîtra le myrte ;
\item[\vref{Lu 6:44}] vendange pas des raisins sur des r.
\end{listverse}

\ConcordanceEntry{Roseau}
\vspace{-2mm}
\begin{listverse}
\item[\vref{Ex 2:3}] posa parmi des r. sur le bord
\item[\vref{2 R 18:21}] l'Egypte, ds ce r. cassé, qui pénètre
\item[\vref{Job 8:11}] Le r. croît-il sans marais ? Le jonc pousse-t-il
\item[\vref{Ps 68:31}] bêtes sauvages des r., la troupe des
\item[\vref{Es 42:3}] brisera point le r. cassé, et il
\item[\vref{Ez 29:6}] un soutien de r. pour la maison
\item[\vref{Mt 11:7}] le désert ? Un r. agité par le
\item[\vref{Ap 21:15}] moi avait un r. d'or pour mesurer
\end{listverse}

\ConcordanceEntry{Rosée}
\vspace{-2mm}
\begin{listverse}
\item[\vref{Ge 27:28}] donne de la r. du ciel, et
\item[\vref{De 32:2}] répande com. la r., com. une pluie
\item[\vref{De 33:13}] ciel, de la r., et de l'abîme
\item[\vref{Jg 6:40}] tt le terrain se couvrit de r.
\item[\vref{Job 38:28}] qui enfante les gouttes de la r. ?
\item[\vref{Ps 133:3}] com. la r. de l'Hermon, celle qui descend sur
\item[\vref{Pr 3:20}] et que les nuages distillent la r.
\item[\vref{Pr 19:12}] est com. la r. sur l'herbe.
\item[\vref{Es 26:19}] poussière ; car ta r. est com. la
\item[\vref{Da 4:15}] trempé de la r. des cieux, et
\item[\vref{Os 6:4}] matin, com. la r. qui se dissipe
\item[\vref{Mi 5:6}] nombreux, com. une r. qui vient de
\end{listverse}

\ConcordanceEntry{Roue}
\vspace{-2mm}
\begin{listverse}
\item[\vref{Ex 14:25}] Il ôta les r. de leurs chars
\item[\vref{Ez 1:18}] pleines d'yeux tt autour des quatre r.
\end{listverse}

\ConcordanceEntry{Rouge}
\vspace{-2mm}
\begin{listverse}
\item[\vref{Ex 15:4}] a été submergée ds la Mer R.
\item[\vref{Pr 23:31}] il se montre r., qnd il donne
\item[\vref{Es 1:18}] neige ; s'ils sont r. com. le vermillon
\item[\vref{Es 63:1}] Botsra, en habits r., magnifiquement paré en
\item[\vref{Mt 16:2}] beau temps, car le ciel est r.
\item[\vref{1 Ti 4:2}] lr. propre conscience marquée au fer r. ;
\item[\vref{Ap 12:3}] un grand dragon r. feu ayant sept
\end{listverse}

\ConcordanceEntry{Rougir}
\vspace{-2mm}
\begin{listverse}
\item[\vref{Ps 35:4}] qui méditent ma perte reculent et r. !
\item[\vref{Ps 40:15}] mon malheur retournent en arrière et r. !
\item[\vref{Ps 75:9}] et le vin r. dedans ; il est
\item[\vref{Ps 119:6}] Et je ne r. point de honte qnd je regarderai
\item[\vref{Es 1:29}] désirés, et vs. r. à cause des
\item[\vref{Es 24:23}] La lune r. et le soleil sera honteux qnd
\item[\vref{Es 54:4}] et tu ne r. pas ; mais tu
\item[\vref{Jé 8:12}] c'est que de r. ; c'est pourquoi ils
\item[\vref{Jé 50:12}] a enfantés a r. ; voici, elle sera
\item[\vref{Ez 43:11}] S'ils r. de tt ce qu'ils ont fait,
\end{listverse}

\ConcordanceEntry{Rouille}
\vspace{-2mm}
\begin{listverse}
\item[\vref{Ez 24:6}] chaudière pleine de r., et de laquelle
\item[\vref{Ez 24:12}] sont inutiles, sa r. dont elle est
\item[\vref{Mt 6:19}] vers et la r. détruisent, et où
\item[\vref{Ja 5:3}] rouillés ; et lr. r. s'élèvera en témoignage
\end{listverse}

\ConcordanceEntry{Rouleau}
\vspace{-2mm}
\begin{listverse}
\item[\vref{Ps 40:8}] moi ds le r. du livre.
\item[\vref{Ez 3:1}] trouveras, mange ce r., et va, parle
\item[\vref{Za 5:1}] et voici, un r. qui volait.
\end{listverse}

\ConcordanceEntry{Rouler}
\vspace{-2mm}
\begin{listverse}
\item[\vref{Jos 5:9}] Josué : Aujourd'hui j'ai r. de dessus vs.
\item[\vref{2 R 2:8}] son manteau, le r. et en frappa
\item[\vref{Pr 26:27}] pierre retourne sur celui qui la r.
\item[\vref{Es 34:4}] les cieux sont r. com. un livre,
\item[\vref{Mt 27:60}] roc. Puis il r. une grande pierre
\item[\vref{Mt 28:2}] du ciel, vint r. la pierre à
\item[\vref{Mc 15:46}] roc. Puis il r. une pierre sur
\item[\vref{Lu 4:20}] Ensuite, il r. le livre, le
\item[\vref{Ap 6:14}] un livre qu'on r. ; et ttes les
\end{listverse}

\ConcordanceEntry{Route}
\vspace{-2mm}
\begin{listverse}
\item[\vref{Ge 45:23}] vivres à son père pour la r.
\item[\vref{No 21:4}] Le cœur du peuple s'impatienta en r.,
\item[\vref{2 S 20:12}] milieu de la r. ; et cet hom.-là,
\item[\vref{Job 22:15}] garde à l'ancienne r. ds laquelle ont
\item[\vref{Job 38:25}] et tracé la r. de l'éclair et
\item[\vref{Ps 80:13}] passants sur la r. cueillent ses raisins ?
\item[\vref{Es 62:10}] Frayez, frayez la r., et ôtez-en les
\item[\vref{Jé 28:11}] prophète, alla au loin par la r.
\item[\vref{Os 13:7}] épierai sur la r. com. un léopard.
\item[\vref{Na 2:2}] Veille sur la r. ! Affermis tes reins !
\item[\vref{Lu 14:25}] grandes foules faisaient r. avec Jésus. Il
\item[\vref{Ac 20:13}] Paul, parce qu'il devait faire la r. à pied.
\end{listverse}

\ConcordanceEntry{Roux}
\vspace{-2mm}
\begin{listverse}
\item[\vref{Ge 25:30}] manger de ce r., de ce roux-là ;
\item[\vref{Za 1:8}] sur un cheval r., et il se
\item[\vref{Ap 6:4}] cheval qui était r. ; il fut donné
\end{listverse}

\ConcordanceEntry{Royaume}
\vspace{-2mm}
\begin{listverse}
\item[\vref{Ge 20:9}] et sur mon r. un grand péché ?
\item[\vref{Ex 19:6}] me serez un r. de prêtres, et
\item[\vref{Est 5:3}] la moitié du r., elle te serait
\item[\vref{Ps 46:7}] nations murmurent, les r. s'ébranlent ; il a
\item[\vref{Ps 68:33}] R. de la terre, chantez à Dieu,
\item[\vref{Jé 1:10}] et sur les r., pour que tu
\item[\vref{Da 2:44}] cieux suscitera un R. qui ne sera
\item[\vref{Da 7:18}] Très-Haut recevront le R., et ils posséderont
\item[\vref{Mt 3:2}] Repentez-vs., car le R. des cieux est
\item[\vref{Mt 4:8}] montra ts les r. du monde et
\item[\vref{Mt 4:17}] Repentez-vs., car le R. des cieux est
\item[\vref{Mt 4:23}] prêchant l'Evangile du R., et guérissant ttes
\item[\vref{Mt 5:19}] plus petit au R. des cieux ; mais
\item[\vref{Mt 6:33}] cherchez premièrement le R. de Dieu et
\item[\vref{Mt 10:7}] en disant : Le R. des cieux est
\item[\vref{Mt 11:12}] jusqu'à mntnt, le R. des cieux est
\item[\vref{Mt 12:25}] lr. dit : Tout r. divisé contre lui-mm
\item[\vref{Mt 12:28}] Dieu, certes le R. de Dieu est
\item[\vref{Mt 13:11}] les mystères du R. des cieux, et
\item[\vref{Mt 13:24}] il dit : Le R. des cieux est
\item[\vref{Mt 16:19}] les clefs du R. des cieux ; et
\item[\vref{Mt 19:12}] eunuques pour le R. des cieux. Que
\item[\vref{Mt 19:14}] pas ; car le R. des cieux est
\item[\vref{Mt 19:24}] à un riche d'entrer ds le R. de Dieu.
\item[\vref{Mt 20:1}] Car le R. des cieux est semblable à un
\item[\vref{Mt 22:2}] Le R. des cieux est semblable à un
\item[\vref{Mt 23:13}] qui fermez le R. des cieux aux
\item[\vref{Mt 24:7}] nation, et un r. contre un autre
\item[\vref{Mt 25:1}] Alors le R. des cieux sera semblable à dix
\item[\vref{Mt 25:34}] en héritage le R. qui vs. a
\item[\vref{Mc 1:14}] ds la Galilée, prêchant l'Evangile du R. de Dieu.
\item[\vref{Mc 1:15}] accompli et le R. de Dieu est
\item[\vref{Mc 10:15}] petit enfant le R. de Dieu, il
\item[\vref{Lu 4:5}] instant ts les r. de la terre,
\item[\vref{Lu 6:20}] pauvres, car le R. de Dieu vs.
\item[\vref{Lu 9:2}] envoya prêcher le R. de Dieu et
\item[\vref{Lu 12:32}] votre Père de vs. donner le R.
\item[\vref{Lu 19:11}] pensaient que le R. de Dieu allait
\item[\vref{Jn 3:3}] haut, il ne peut voir le R. de Dieu.
\item[\vref{Jn 18:36}] Jésus répondit : Mon R. n'est pas de
\item[\vref{Ac 1:6}] tu rétabliras le r. d'Israël ?
\item[\vref{Ro 14:17}] Car le R. de Dieu, ce n'est pas la
\item[\vref{1 Co 4:20}] Car le R. de Dieu ne consiste pas en
\item[\vref{1 Co 6:9}] n'hériteront pas le R. de Dieu ? Ne
\item[\vref{1 Co 15:24}] aura remis le R. à Dieu le
\item[\vref{Ga 5:21}] de telles choses n'hériteront pas le R. de Dieu.
\item[\vref{Ep 5:5}] d'héritage ds le R. de Christ et
\item[\vref{Col 1:13}] transportés ds le R. du Fils de
\item[\vref{1 Th 2:12}] appelle à son R. et à sa
\item[\vref{2 Ti 4:18}] sauvera ds son R. céleste. A lui
\item[\vref{Hé 11:33}] foi combattirent des r., exercèrent la justice,
\item[\vref{Hé 12:28}] pourquoi, saisissant le R. qui ne peut
\item[\vref{Ja 2:5}] et héritiers du R. qu'il a promis
\item[\vref{2 Pi 1:11}] moyen, l'entrée au R. éternel de notre
\item[\vref{Ap 11:15}] qui disaient : Les r. du monde sont
\item[\vref{Ap 17:17}] de donner lr. r. à la bête,
\item[\vref{Ap 19:6}] Tout-Puissant a pris possession de son R.
\end{listverse}

\ConcordanceEntry{Royauté}
\vspace{-2mm}
\begin{listverse}
\item[\vref{1 S 10:16}] rien concernant la r. dont Samuel lui
\item[\vref{2 Ch 13:5}] pour toujours la r. sur Israël à
\item[\vref{Est 4:14}] arrivée à la r. pour un temps
\end{listverse}

\ConcordanceEntry{Ruben, Rubénites}
\vspace{-2mm}
\begin{listverse}
\item[\vref{Ge 29:32}] le nom de R., car elle dit :
\item[\vref{No 32:1}] Les fils de R. et les fils
\item[\vref{1 Ch 5:1}] Les fils de R., le premier-né d'Israël,
\end{listverse}

\ConcordanceEntry{Rude}
\vspace{-2mm}
\begin{listverse}
\item[\vref{Ex 1:13}] d'Israël à une r. servitude.
\item[\vref{Mal 3:13}] Vos paroles sont r. contre moi, a
\item[\vref{Hé 11:37}] sciés, subirent de r. épreuves, ils furent
\end{listverse}

\ConcordanceEntry{Rue}
\vspace{-2mm}
\begin{listverse}
\item[\vref{Ge 19:2}] ns. passerons la nuit ds la r.
\item[\vref{Pr 1:20}] fait retentir sa voix ds les r.
\item[\vref{Es 51:23}] terre, com. une r. pour les passants.
\item[\vref{Es 59:14}] tombée par les r., et la droiture
\item[\vref{Mt 6:2}] et ds les r., afin d'être glorifiés
\item[\vref{Mt 12:19}] personne n'entendra sa voix ds les r.
\item[\vref{Lu 10:10}] sortez ds ses r., et dites :
\item[\vref{Lu 13:26}] et tu as enseigné ds nos r.
\item[\vref{Lu 14:21}] et ds les r. de la ville,
\item[\vref{Ac 5:15}] malades ds les r., et on les
\item[\vref{Ac 9:11}] va ds la r. appelée la droite,
\item[\vref{Ac 12:10}] s'avancèrent ds une r.. Et subitement, l'ange
\end{listverse}

\ConcordanceEntry{Rufus}
\vspace{-2mm}
\begin{listverse}
\item[\vref{Mc 15:21}] d'Alexandre et de R., passant par là
\item[\vref{Ro 16:13}] Saluez R., l'élu du Seign., et sa mère,
\end{listverse}

\ConcordanceEntry{Rugir}
\vspace{-2mm}
\begin{listverse}
\item[\vref{Jg 14:5}] un jeune lion r. vint à sa
\item[\vref{Jé 25:30}] lr. diras : Yahweh r. d'en haut ; il
\item[\vref{Am 3:8}] Le lion r., qui ne craindrait ? Le Seign. Yahweh
\item[\vref{1 Pi 5:8}] com. un lion r., cherchant qui il
\end{listverse}

\ConcordanceEntry{Ruine}
\vspace{-2mm}
\begin{listverse}
\item[\vref{Jos 8:28}] monceau perpétuel de r., jusqu'à aujourd'hui.
\item[\vref{Né 2:13}] qui étaient en r., et ses portes
\item[\vref{Job 21:20}] yeux verront sa r., et il boira
\item[\vref{Pr 3:25}] soudaine, ni la r. des méchants, qnd
\item[\vref{Pr 10:14}] l'insensé est une r. prochaine.
\item[\vref{Pr 10:29}] elle est la r. pour les ouvriers
\item[\vref{Pr 13:3}] tt propos ses lèvres, tombera en r.
\item[\vref{Pr 14:1}] la folle la r. de ses mains.
\item[\vref{Pr 16:18}] et la fierté d'esprit dvt la r.
\item[\vref{Pr 18:12}] avant que la r. arrive, mais l'humilité
\item[\vref{Pr 29:16}] s'accroissent, mais les justes verront lr. r.
\item[\vref{Ec 7:16}] ne faut : Pourquoi t'exposer à la r. ?
\item[\vref{Es 59:7}] ravage et la r. sont sur leurs
\item[\vref{Es 61:4}] ils rebâtiront les r. antiques, ils relèveront
\item[\vref{Jé 1:10}] pour que tu r. et que tu
\item[\vref{Jé 49:13}] villes deviendront des r. éternelles.
\item[\vref{Mt 7:27}] tombée, et sa r. a été grande.
\item[\vref{Lu 6:49}] tombée, et la r. de cette maison
\item[\vref{Lu 11:17}] maison divisée contre elle-mm tombe en r.
\item[\vref{Ac 15:16}] je réparerai ses r. et je le
\item[\vref{2 Th 1:9}] pour châtiment une r. éternelle, loin de
\item[\vref{1 Ti 6:9}] hommes ds la r. et la perdition.
\item[\vref{2 Ti 2:14}] elle est la r. des auditeurs.
\item[\vref{2 Pi 2:1}] sur eux-mêmes une r. soudaine.
\item[\vref{Ap 17:11}] sept, mais elle tend à sa r.
\end{listverse}

\ConcordanceEntry{Ruisseau}
\vspace{-2mm}
\begin{listverse}
\item[\vref{Ex 7:19}] rivières, sur leurs r., et sur leurs
\item[\vref{Ps 1:3}] planté près des r. d'eaux, qui rend
\item[\vref{Ps 46:5}] fleuve et ses r. réjouissent la cité
\item[\vref{Ps 65:10}] de richesses ; le r. de Dieu est
\item[\vref{Ps 126:4}] captifs, com. des r. ds le midi !
\item[\vref{Pr 5:16}] dehors, et les r. d'eau sur les
\item[\vref{Pr 21:1}] Yahweh com. des r. d'eaux ; il l'incline
\end{listverse}

\ConcordanceEntry{Ruse}
\vspace{-2mm}
\begin{listverse}
\item[\vref{Ge 34:13}] Jacob répondirent avec r. à Sichem et
\item[\vref{Ex 21:14}] le tuer par r., tu l'arracheras de
\item[\vref{Jos 9:4}] usèrent de r., car ils se
\item[\vref{Pr 14:17}] agit follement, et l'hom. plein de r. est haï.
\item[\vref{Mc 14:1}] de Jésus par r., et de le
\item[\vref{Lu 20:23}] Jésus, apercevant lr. r., lr. dit : Pourquoi
\item[\vref{Ac 13:10}] et de tte r., fils du diable,
\item[\vref{1 Co 3:19}] Il surprend les sages ds lr. r.
\item[\vref{2 Co 4:2}] marchant pas avec r. et ne falsifiant
\item[\vref{2 Co 11:3}] Eve par sa r., vos pensées aussi
\item[\vref{Ep 4:14}] et par lr. r. à séduire artificieusement.
\end{listverse}

\ConcordanceEntry{Rusé}
\vspace{-2mm}
\begin{listverse}
\item[\vref{Ge 3:1}] était le plus r. de ts les
\item[\vref{Job 5:12}] projets des hommes r., de sorte qu'ils
\item[\vref{Jé 17:9}] Le cœur est r. et désespérément malin
\end{listverse}

\ConcordanceEntry{Ruth}
\vspace{-2mm}
\begin{listverse}
\item[\vref{Ru 1:4}] Orpa et l'autre R., et ils demeurèrent
\item[\vref{Mt 1:5}] engendra Obed, de R. ; Obed engendra Isaï ;
\end{listverse}

\ConcordanceEntry{Saba, Séba}
\vspace{-2mm}
\begin{listverse}
\item[\vref{Ge 10:7}] fils de Cusch : S., Havila, Sabta, Raema,
\item[\vref{1 R 10:1}] la reine de S. ayant appris la
\item[\vref{1 R 10:13}] la reine de S. tt ce qu'elle
\item[\vref{Ps 72:10}] les rois de S. et de Séba
\end{listverse}

\ConcordanceEntry{Sabbat}
\vspace{-2mm}
\begin{listverse}
\item[\vref{Ex 16:29}] a ordonné le s., c'est pourquoi il
\item[\vref{Ex 31:15}] jour est le s. du repos, consacré
\item[\vref{Es 58:13}] pied pendant le s. pour ne pas
\item[\vref{Mt 12:5}] qu'aux jours du s., les prêtres violent
\item[\vref{Mt 12:8}] de l'hom. est Maître mm du s.
\item[\vref{Mc 2:27}] lr. dit : Le s. a été fait
\item[\vref{Lu 4:31}] il les enseignait les jours de s.
\item[\vref{Jn 5:16}] fait ces choses le jour du s.
\item[\vref{Jn 7:22}] bien un hom. le jour du s.
\item[\vref{Ac 1:12}] près de Jérus., le chemin d'un s.
\item[\vref{Ac 13:27}] des prophètes qui se lisent chaque s.
\item[\vref{Col 2:16}] jour de nouvelle lune, ou de s.,
\end{listverse}

\ConcordanceEntry{Sable}
\vspace{-2mm}
\begin{listverse}
\item[\vref{Ge 22:17}] et com. le s. qui est sur
\item[\vref{Pr 27:3}] pesante, et le s. est lourd ; mais
\item[\vref{Es 10:22}] serait com. le s. de la mer,
\item[\vref{Os 2:1}] sera com. le s. de la mer,
\item[\vref{Mt 7:26}] a bâti sa maison sur le s.
\item[\vref{Hé 11:12}] et que le s. du bord de
\item[\vref{Ap 20:8}] est com. le s. de la mer.
\end{listverse}

\ConcordanceEntry{Sac}
\vspace{-2mm}
\begin{listverse}
\item[\vref{Ge 42:25}] qu'on remplisse leurs s. de blé, et
\item[\vref{Est 4:1}] se couvrit d'un s. et de cendre.
\item[\vref{Ps 30:12}] as détaché mon s., et tu m'as
\item[\vref{Jé 4:8}] pourquoi ceignez-vs. de s., lamentez-vs. et gémissez ;
\item[\vref{Ag 1:6}] pour mettre son salaire ds un s. percé.
\item[\vref{Mt 10:10}] ni de s. pour le voyage, ni deux tuniques,
\item[\vref{Mt 11:21}] en prenant le s. et la cendre.
\item[\vref{Ap 6:12}] noir com. un s. de crin, et
\end{listverse}

\ConcordanceEntry{Sacré}
\vspace{-2mm}
\begin{listverse}
\item[\vref{Ps 29:2}] Prosternez-vs. dvt Yahweh avec des ornements s. !
\item[\vref{1 Co 9:13}] s'emploient aux choses s. mangent les choses
\end{listverse}

\ConcordanceEntry{Sacrifice}
\vspace{-2mm}
\begin{listverse}
\item[\vref{Ge 31:54}] Jacob offrit un s. sur la montagne
\item[\vref{No 25:2}] le peuple aux s. de leurs dieux ;
\item[\vref{De 12:6}] vos holocaustes, vos s., vos dîmes, vos
\item[\vref{1 S 2:29}] aux pieds mes s. et mes offrandes,
\item[\vref{1 S 15:22}] holocaustes et aux s., autant qu'à l'obéissance
\item[\vref{1 S 20:29}] famille fait un s. ds la ville,
\item[\vref{Ps 50:5}] traité alliance avec moi par le s. !
\item[\vref{Ps 50:8}] pas pour tes s. ; tes holocaustes sont
\item[\vref{Ps 51:19}] Les s. à Dieu, c'est un esprit brisé.
\item[\vref{Ps 54:8}] bon cœur des s.. Yahweh, je célébrerai
\item[\vref{Ps 107:22}] Qu'ils offrent des s. de remerciements, et
\item[\vref{Pr 15:8}] Le s. des méchants est en abomination à
\item[\vref{Pr 21:3}] une chose que Yahweh préfère aux s.
\item[\vref{Es 1:11}] multitude de vos s. ? Je suis rassasié
\item[\vref{Es 53:10}] son âme en s. pour le péché,
\item[\vref{Jé 6:20}] pas, et vos s. ne me sont
\item[\vref{Da 8:11}] lui enleva le s. perpétuel, et renversa
\item[\vref{Da 9:27}] fera cesser le s., et l'offrande ; puis
\item[\vref{Os 3:4}] sans chef, sans s., sans statue, sans
\item[\vref{Os 6:6}] et non aux s., et à la
\item[\vref{Am 5:25}] avez offert des s. et des gâteaux
\item[\vref{Mt 9:13}] et non aux s.. Car je ne
\item[\vref{Ro 12:1}] vos corps en s. vivant, saint, agréable
\item[\vref{Ep 5:2}] offrande et un s. de bonne odeur
\item[\vref{Hé 5:3}] doit offrir des s. pour ses propres
\item[\vref{Hé 9:26}] pour l'abolition du péché par son s.
\item[\vref{Hé 10:5}] pas voulu de s., ni d'offrande, mais
\item[\vref{Hé 10:26}] reste plus de s. pour les péchés,
\item[\vref{Hé 11:4}] à Dieu un s. plus excellent que
\item[\vref{Hé 13:15}] à Dieu un s. de louange, c'est-à-dire,
\item[\vref{1 Pi 2:5}] afin d'offrir des s. spirituels, agréables à
\end{listverse}

\ConcordanceEntry{Sacrifier}
\vspace{-2mm}
\begin{listverse}
\item[\vref{Ex 3:18}] que ns. puissions s. à Yahweh, notre
\item[\vref{De 32:17}] Ils ont s. à des démons, qui ne sont
\item[\vref{Né 4:2}] Les laissera-t-on faire ? S.-ils ? Et achèveront-ils
\item[\vref{1 Co 5:7}] Pâque, a été s. pour ns.
\item[\vref{1 Co 8:1}] choses qui sont s. aux idoles, ns.
\item[\vref{1 Co 10:19}] ce qui est s. à l'idole soit
\item[\vref{Ap 2:14}] mangent des viandes s. aux idoles, et
\end{listverse}

\ConcordanceEntry{Sacrilège}
\vspace{-2mm}
\begin{listverse}
\item[\vref{Ac 19:37}] ne sont ni s. ni blasphémateurs de
\item[\vref{Ro 2:22}] abomination les idoles, tu commets des s. !
\end{listverse}

\ConcordanceEntry{Sadducéens}
\vspace{-2mm}
\begin{listverse}
\item[\vref{Mt 3:7}] pharisiens et de s., il lr. dit :
\item[\vref{Mt 16:1}] pharisiens et les s. vinrent à lui,
\item[\vref{Mt 22:23}] mm jour, les s., qui disent qu'il
\item[\vref{Ac 4:1}] le commandant du temple et les s.,
\item[\vref{Ac 5:17}] la secte des s., et ils furent
\item[\vref{Ac 23:8}] Car les s. disent qu'il n'y a pas de
\end{listverse}

\ConcordanceEntry{Sage}
\vspace{-2mm}
\begin{listverse}
\item[\vref{De 32:29}] Ô s'ils étaient s., ils comprendraient ceci,
\item[\vref{1 R 4:31}] Il était plus s. qu'aucun hom., plus
\item[\vref{Né 9:20}] pour les rendre s. ; tu ne retiras
\item[\vref{Job 5:13}] il surprend les s. ds lr. ruse,
\item[\vref{Job 37:24}] il ne les voit pas ts s. de cœur.
\item[\vref{Ps 119:98}] m'as rendu plus s. que mes ennemis,
\item[\vref{Pr 3:7}] Ne sois point s. à tes yeux ;
\item[\vref{Pr 3:35}] Les s. hériteront la gloire ; mais la honte
\item[\vref{Pr 9:8}] haïsse ; reprends le s. et il t'aimera.
\item[\vref{Pr 11:30}] celui qui gagne les âmes est s.
\item[\vref{Pr 13:20}] marche avec les s. deviendra sage, mais
\item[\vref{Pr 14:1}] Toute fem. s. bâtit sa maison,
\item[\vref{Pr 14:3}] les lèvres des s. les garderont.
\item[\vref{Pr 14:16}] Le s. craint et se retire du mal,
\item[\vref{Pr 16:23}] Celui qui est s. de cœur conduit
\item[\vref{Pr 26:12}] qui croit être s. ? Il y a
\item[\vref{Ec 2:14}] Le s. a ses yeux à sa tête,
\item[\vref{Ec 9:1}] les justes, les s. et leurs actions
\item[\vref{Es 5:21}] ceux qui sont s. à leurs yeux,
\item[\vref{Da 2:21}] la sagesse aux s. et la connaissance
\item[\vref{Ab 1:8}] ferai périr les s. au milieu d'Edom,
\item[\vref{Mt 11:25}] ces choses aux s. et aux intelligents,
\item[\vref{Mt 25:4}] mais les s. prirent de l'huile ds leurs vases
\item[\vref{Ro 1:22}] Se disant être s., ils sont devenus
\item[\vref{Ro 12:16}] Ne soyez pas s. à votre propre
\item[\vref{Ro 16:27}] Dieu, dis-je, seul s., soit la gloire
\item[\vref{1 Co 1:20}] Où est le s. ? Où est le
\item[\vref{1 Co 1:25}] Dieu est plus s. que les hommes,
\item[\vref{1 Co 1:26}] pas beaucoup de s. selon la chair,
\item[\vref{1 Co 3:18}] vs. croit être s. selon ce monde,
\item[\vref{1 Co 4:10}] mais vs. êtes s. en Christ ; ns.
\item[\vref{2 Co 11:19}] volontiers les insensés, vs. qui êtes s.
\item[\vref{Ep 5:15}] dépourvus de sagesse, mais com. étant s.,
\item[\vref{2 Ti 3:15}] peuvent te rendre s. pour le salut
\item[\vref{Tit 1:8}] gens de bien, s., juste, saint, tempérant,
\item[\vref{Ja 3:13}] vs. qq hom. s. et intelligent ? Qu'il
\item[\vref{Jud 1:25}] à Dieu, seul s., notre Sauveur, par
\end{listverse}

\ConcordanceEntry{Sagesse}
\vspace{-2mm}
\begin{listverse}
\item[\vref{De 4:6}] c'est là votre s. et votre intelligence
\item[\vref{1 R 2:6}] agiras selon ta s., en sorte que
\item[\vref{1 R 4:29}] Salomon de la s., une très grande
\item[\vref{1 R 10:8}] dvt toi, et qui entendent ta s. !
\item[\vref{Job 12:2}] peuple ; et la s. mourra-t-elle avec vs. ?
\item[\vref{Ps 19:8}] il donne la s. au simple.
\item[\vref{Ps 51:8}] fais connaître la s. au-dedans de moi.
\item[\vref{Ps 90:12}] puissions avoir un cœur rempli de s.
\item[\vref{Pr 1:20}] La souveraine s. crie hautement au-dehors,
\item[\vref{Pr 2:6}] Yahweh donne la s., et de sa
\item[\vref{Pr 3:13}] a trouvé la s., et l'hom. qui
\item[\vref{Pr 16:16}] d'acquérir de la s. ! Et combien est-il
\item[\vref{Pr 21:30}] n'y a ni s., ni intelligence, ni
\item[\vref{Pr 24:3}] établie par la s., et affermie par
\item[\vref{Ec 2:9}] Jérus.. Et ma s. est demeurée avec
\item[\vref{Ec 7:11}] La s. est bonne avec un héritage, elle
\item[\vref{Ec 9:16}] j'ai dit : La s. vaut mieux que
\item[\vref{Es 29:14}] étranges ; et la s. de ses sages
\item[\vref{Jé 8:9}] Yahweh, et quelle s. ont-ils ?
\item[\vref{Jé 51:15}] habitable par sa s., et qui a
\item[\vref{Mt 11:19}] vie. Mais la s. a été justifiée
\item[\vref{Mt 13:54}] lui viennent cette s. et ces miracles ?
\item[\vref{Lu 2:52}] Jésus croissait en s., en stature, et
\item[\vref{Ac 6:3}] Saint-Esprit et de s., auxquels ns. confierons
\item[\vref{Ac 7:22}] ds tte la s. des Egyptiens ; et
\item[\vref{Ro 11:33}] et de la s. et de la
\item[\vref{1 Co 1:19}] Je détruirai la s. des sages et
\item[\vref{1 Co 1:21}] monde, avec sa s., n'a pas connu
\item[\vref{1 Co 1:30}] part de Dieu, s., justice, sanctification et
\item[\vref{1 Co 2:7}] ns. prêchons la s. de Dieu, qui
\item[\vref{1 Co 3:19}] Parce que la s. de ce monde
\item[\vref{1 Co 12:8}] la parole de s. ; et à l'autre
\item[\vref{Ep 1:17}] donne l'Esprit de s. et de révélation,
\item[\vref{Ep 3:10}] par l'Eglise la s. infiniment variée de
\item[\vref{Ep 5:15}] étant dépourvus de s., mais com. étant
\item[\vref{Col 1:9}] volonté, en tte s. et intelligence spirituelle,
\item[\vref{Tit 2:12}] siècle, selon la s., la justice et
\item[\vref{Ja 1:5}] vs. manque de s., qu'il la demande
\item[\vref{Ja 3:15}] pas là la s. qui descend d'en
\item[\vref{Ja 3:17}] Mais la s. d'en haut est premièrement pure, ensuite
\item[\vref{Ap 13:18}] Ici est la s. : Que celui qui
\item[\vref{Ap 17:9}] ait de la s.. Les sept têtes
\end{listverse}

\ConcordanceEntry{Sain}
\vspace{-2mm}
\begin{listverse}
\item[\vref{2 R 5:10}] ta chair redeviendra s., et tu seras
\item[\vref{Ps 38:4}] a rien de s. ds ma chair,
\item[\vref{Pr 14:30}] Un cœur s. est la vie de la chair,
\item[\vref{Ez 47:8}] ds la mer, les eaux deviendront s.
\item[\vref{Mt 12:13}] et elle devint s. com. l'autre.
\item[\vref{Lu 15:27}] veau gras, parce qu'il l'a recouvré s. et sauf.
\item[\vref{Ac 27:44}] parvinrent à terre s. et saufs.
\item[\vref{1 Ti 1:10}] contraire à la s. doctrine,
\item[\vref{1 Ti 6:3}] soumet pas aux s. paroles de notre
\item[\vref{2 Ti 1:13}] le modèle des s. paroles que tu
\item[\vref{2 Ti 4:3}] supporteront pas la s. doctrine, mais aimant
\item[\vref{Tit 1:9}] d'exhorter par la s. doctrine, que de
\item[\vref{Tit 1:13}] afin qu'ils soient s. ds la foi,
\item[\vref{Tit 2:2}] sobres, honnêtes, prudents, s. ds la foi,
\item[\vref{Tit 2:8}] en paroles s., que l'on ne
\end{listverse}

\ConcordanceEntry{Saint}
\vspace{-2mm}
\begin{listverse}
\item[\vref{Ex 3:5}] tu es arrêté est une terre s.
\item[\vref{Ex 12:16}] y aura une s. convocation, et il
\item[\vref{Ex 15:17}] Yahweh ! au lieu s., ô Seign., que
\item[\vref{Ex 19:6}] et une nation s. ; ce sont là
\item[\vref{Ex 39:1}] ils firent les s. vêtements pour Aaron,
\item[\vref{Lé 10:10}] ce qui est s. et ce qui
\item[\vref{Lé 19:2}] et dis-lr. : Soyez s., car je suis
\item[\vref{1 S 2:2}] Nul n'est s. com. Yahweh ; car
\item[\vref{2 R 4:9}] ns. est un s. hom. de Dieu.
\item[\vref{2 R 19:22}] haut, vers le S. d'Israël !
\item[\vref{Job 15:15}] pas à ses s. et les cieux
\item[\vref{Ps 16:3}] Les s. qui sont ds le pays, les
\item[\vref{Ps 22:4}] tu es le S., tu habites au
\item[\vref{Ps 89:19}] Roi est le S. d'Israël.
\item[\vref{Es 1:4}] lr. mépris le S. d'Israël, ils se
\item[\vref{Es 12:6}] triomphe ! Car le S. d'Israël est grand
\item[\vref{Es 47:4}] des armées, le S. d'Israël.
\item[\vref{Es 55:5}] Dieu, et du S. d'Israël, qui t'aura
\item[\vref{Es 65:5}] je suis plus s. que toi ! Ceux-là
\item[\vref{Da 4:13}] et qui sont s. descendit des cieux.
\item[\vref{Da 7:18}] Mais les s. du Très-Haut recevront le Royaume, et
\item[\vref{Da 9:24}] sur ta ville s., pour abolir la
\item[\vref{Ha 1:12}] Mon Dieu ! Mon S. ? Nous ne mourrons
\item[\vref{Mc 1:24}] qui tu es : Tu es le S. de Dieu.
\item[\vref{Lu 1:35}] C'est pourquoi, le S. qui naîtra de
\item[\vref{Ac 3:14}] avez renié le S. et le Juste,
\item[\vref{Ro 1:7}] appelés à être s. : Que la grâce
\item[\vref{Ro 12:13}] aux nécessités des s. ; exercez l'hospitalité.
\item[\vref{Ro 15:25}] vais à Jérus. pour servir les s.
\item[\vref{1 Co 7:14}] seraient impurs, or mntnt ils sont s.
\item[\vref{Ep 1:4}] que ns. soyons s. et irrépréhensibles dvt
\item[\vref{Ep 5:27}] de semblable, mais s. et irréprochable.
\item[\vref{Ep 6:18}] persévérance, et priez pour ts les s.,
\item[\vref{Col 1:22}] pour vs. présenter s., et sans tache,
\item[\vref{Col 3:12}] élus de Dieu, s. et bien-aimés, revêtez-vs.
\item[\vref{1 Th 3:13}] Seign. Jésus-Christ, accompagné de ts ses s.
\item[\vref{2 Th 1:10}] jour-là ds ses s., et pour être
\item[\vref{Hé 9:3}] qui était appelé le Saint des s.,
\item[\vref{1 Pi 1:15}] a appelés est s., vs. aussi de
\item[\vref{1 Pi 1:16}] est écrit : Soyez s., car je suis
\item[\vref{1 Pi 2:5}] spirituelle, et une s. prêtrise, afin d'offrir
\item[\vref{1 Pi 2:9}] royale, la nation s., le peuple acquis,
\item[\vref{1 Pi 3:5}] aussi autrefois les s. femmes qui espéraient
\item[\vref{Jud 1:20}] sur votre très s. foi, et priez
\item[\vref{Ap 3:7}] que dit le S. et le Véritable,
\item[\vref{Ap 4:8}] jour et nuit : S. ! Saint ! Saint est
\item[\vref{Ap 15:4}] seul tu es S., c'est pourquoi ttes
\end{listverse}

\ConcordanceEntry{Saint des saints}
\vspace{-2mm}
\begin{listverse}
\item[\vref{Ex 26:33}] entre le lieu saint et le Saint des saints.
\item[\vref{No 4:4}] à la tente d'assignation, c'est-à-dire, le Saint des saints.
\item[\vref{Ez 41:4}] il me dit : C'est ici le Saint des saints.
\item[\vref{Da 9:24}] la prophétie et pour oindre le Saint des saints.
\item[\vref{Hé 9:8}] le chemin du Saint des saints n'était pas encore
\item[\vref{Hé 10:19}] d'entrer ds le Saint des saints au moyen du
\end{listverse}

\ConcordanceEntry{Saint-Esprit, Esprit Saint}
\vspace{-2mm}
\begin{listverse}
\item[\vref{Ps 51:13}] face, et ne m'ôte pas ton Esprit Saint.
\item[\vref{Es 63:10}] ont attristé son Esprit saint, c'est pourquoi il
\item[\vref{Mt 1:20}] l'enfant qu'elle a conçu est du S.
\item[\vref{Mt 3:11}] vs. baptisera du S. et de feu.
\item[\vref{Mt 28:19}] du Père, du Fils et du S.
\item[\vref{Mc 3:29}] blasphémera contre le S. n'obtiendra jamais de
\item[\vref{Mc 12:36}] dit par le S. : Le Seign. a
\item[\vref{Mc 13:11}] pas vs. qui parlerez, mais le S.
\item[\vref{Lu 1:15}] sera rempli du S. dès le ventre
\item[\vref{Lu 3:22}] et le S. descendit sur lui sous une forme
\item[\vref{Lu 12:10}] blasphémé contre le S., il ne lui
\item[\vref{Lu 12:12}] car le S. vs. enseignera à l'heure mm ce
\item[\vref{Jn 1:33}] s'arrêter, c'est celui qui baptise du S.
\item[\vref{Jn 20:22}] eux, et lr. dit : Recevez le S.
\item[\vref{Ac 2:4}] ts remplis du S., et commencèrent à
\item[\vref{Ac 4:31}] ts remplis du S., et ils annonçaient
\item[\vref{Ac 5:3}] à mentir au S., et à soustraire
\item[\vref{Ac 8:17}] les mains, et ils reçurent le S.
\item[\vref{Ac 10:38}] a oint du S. et de force
\item[\vref{Ac 13:2}] et jeûnaient, le S. dit : Séparez-moi mntnt
\item[\vref{Ac 15:28}] paru bon au S. et à ns.,
\item[\vref{Ac 16:6}] de Galatie, le S. lr. défendit d'annoncer
\item[\vref{Ac 20:28}] sur lequel le S. vs. a établis
\item[\vref{Ro 5:5}] cœurs par le S. qui ns. a
\item[\vref{Ro 15:16}] soit agréable, étant sanctifiée par le S.
\item[\vref{1 Co 6:19}] le temple du S. qui est en
\item[\vref{1 Co 12:3}] Seign. ! Si ce n'est par le S.
\item[\vref{Ep 1:13}] été scellés du S. qui avait été
\item[\vref{Ep 4:30}] n'attristez pas le S. de Dieu, par
\item[\vref{1 Th 4:8}] Dieu qui a aussi donné son S.
\item[\vref{Tit 3:5}] la régénération et le renouvellement du S.,
\item[\vref{Hé 2:4}] les dons du S., selon sa volonté.
\item[\vref{Hé 6:4}] qui ont été faits participants au S.,
\item[\vref{1 Pi 1:12}] l'Evangile par le S. envoyé du ciel,
\item[\vref{2 Pi 1:21}] poussés par le S. que les saints
\item[\vref{1 Jn 2:20}] oints par le S., et vs. connaissez
\item[\vref{1 Jn 5:7}] Parole, et le S. ; et ces trois-là
\item[\vref{Jud 1:20}] sainte foi, et priez par le S.,
\end{listverse}

\ConcordanceEntry{Sainteté}
\vspace{-2mm}
\begin{listverse}
\item[\vref{Ex 15:11}] toi, magnifique en s., digne d'être révéré
\item[\vref{Ex 28:36}] un cachet : La s. à Yahweh.
\item[\vref{Ps 2:6}] sur Sion, la montagne de ma s. !
\item[\vref{Ps 15:1}] habitera sur la montagne de ta s. ?
\item[\vref{Ps 30:5}] et célébrez la mémoire de sa s. !
\item[\vref{Ps 93:5}] fidèles. Yahweh ! La s. orne ta maison
\item[\vref{Ps 102:20}] élevé de sa s. ; du haut des
\item[\vref{Ps 150:1}] cause de sa s. ! Louez-le à cause
\item[\vref{Jon 2:8}] à toi, jusqu'au palais de ta s.
\item[\vref{Ha 2:20}] temple de sa s.. Toute la terre,
\item[\vref{Lu 1:75}] lui ds la s. et ds la
\item[\vref{Ro 1:4}] selon l'Esprit de s., par sa résurrection
\item[\vref{Ro 6:19}] servir à la justice, pour la s.
\item[\vref{Ep 4:24}] justice et une s. véritables.
\item[\vref{1 Th 3:13}] irréprochables ds la s., dvt Dieu qui
\item[\vref{Tit 2:3}] convenable à la s. ; qu'elles ne soient
\item[\vref{Hé 12:10}] que ns. soyons participants de sa s.
\item[\vref{2 Pi 3:11}] pas être la s. de votre conduite
\end{listverse}

\ConcordanceEntry{Saisir}
\vspace{-2mm}
\begin{listverse}
\item[\vref{Ge 20:8}] et ils furent s. de crainte.
\item[\vref{Ex 4:4}] ta main et s. sa queue ; et
\item[\vref{1 S 16:13}] l'Esprit de Yahweh s. David. Et Samuel
\item[\vref{Ps 139:10}] me conduira, et ta droite me s.
\item[\vref{Jé 20:7}] persuader ; tu m'as s. et tu m'as
\item[\vref{Mt 2:10}] l'étoile, ils furent s. d'une très grande
\item[\vref{Mt 21:46}] cherchaient à se s. de lui, mais
\item[\vref{Mc 5:33}] Alors la fem. s. de crainte et
\item[\vref{Ro 7:8}] Et le péché, s. l'occasion, produisit en
\item[\vref{Ph 3:12}] aussi j'ai été s. par Jésus-Christ.
\item[\vref{Hé 12:28}] C'est pourquoi, s. le Royaume qui
\end{listverse}

\ConcordanceEntry{Salaire}
\vspace{-2mm}
\begin{listverse}
\item[\vref{Ge 29:15}] mon frère ? Dis-moi quel sera ton s. ?
\item[\vref{Ge 31:7}] dix fois mon s. ; mais Dieu ne
\item[\vref{Ex 2:9}] te donnerai ton s.. Et la fem.
\item[\vref{Lé 19:13}] pilleras point. Le s. de ton mercenaire
\item[\vref{Jé 22:13}] lui donner le s. de son travail ;
\item[\vref{Os 9:1}] as obtenu un s. de tes amants
\item[\vref{Za 11:12}] bon, donnez-moi mon s. ; sinon, ne me
\item[\vref{Lu 10:7}] l'ouvrier mérite son s.. N'allez pas de
\item[\vref{Ac 1:18}] champ avec le s. du crime qui
\item[\vref{Ro 4:4}] les œuvres, le s. ne lui est
\item[\vref{Ro 6:23}] Car le s. du péché, c'est la mort ; mais
\item[\vref{2 Pi 2:15}] qui aima le s. de l'iniquité, mais
\end{listverse}

\ConcordanceEntry{Salem}
\vspace{-2mm}
\begin{listverse}
\item[\vref{Ge 14:18}] Melchisédek, roi de S., fit apporter du
\item[\vref{Ps 76:3}] tente est à S., et sa demeure
\item[\vref{Hé 7:1}] était Roi de S. et Prêtre du
\end{listverse}

\ConcordanceEntry{Salive}
\vspace{-2mm}
\begin{listverse}
\item[\vref{1 S 21:13}] laissait couler sa s. sur sa barbe.
\item[\vref{Mc 7:33}] toucha la langue avec sa propre s.
\item[\vref{Mc 8:23}] mit de la s. sur les yeux,
\item[\vref{Jn 9:6}] boue avec sa s., et mit de
\end{listverse}

\ConcordanceEntry{Salomon}
\vspace{-2mm}
\begin{listverse}
\item[\vref{Mt 1:6}] roi David engendra S., de la fem.
\end{listverse}
\begin{legend}
\NoAutoSpaceBeforeFDP{
\item Naissance de Salomon : 2 S 12:24-25
\item Paroles prophétiques sur S : 2 S 7:12-16; 1 Ch 22:9-10
\item S. oint roi d'Israël : 1 R 1:34-40; 1 Ch 29:1
\item Salomon demande la sagesse à Yahweh : 1 R 3: 7-9
\item Yahweh exauce Salomon : 1 R 3:12-18; 4:29-34
\item Construction du temple de Yahweh : 1 R 6:1,37-38
\item La dédicace du temple : 1 R 8 et 9
\item Apparitions de Yahweh à S : 1 R 3:5; 9:2
\item La reine de Seba 1 R 10:1-13; Mt 12-42
\item S. détourne son cœur de Yahweh : 1 R 11:1-8; Né 13:26
\item La colère de Yahweh et son jugement : 1 R 11:9-13
\item Israël livré aux ennemis : 1 R 11: 14-26
}
\end{legend}

\ConcordanceEntry{Salut}
\vspace{-2mm}
\begin{listverse}
\item[\vref{Ge 49:18}] Ô Yahweh ! J'espère en ton s. !
\item[\vref{1 S 2:1}] je me suis réjouie de ton s.
\item[\vref{1 Ch 16:35}] Dieu de notre s., sauve-ns., et rassemble-ns.,
\item[\vref{Ps 18:47}] Dieu de mon s. soit exalté !
\item[\vref{Ps 50:23}] sa voie, je lui montrerai le s. de Dieu.
\item[\vref{Ps 51:14}] joie de ton s. et qu'un esprit
\item[\vref{Ps 98:3}] ont vu le s. de notre Dieu.
\item[\vref{Ps 132:16}] Je revêtirai de s. ses prêtres, et
\item[\vref{Pr 2:7}] Il réserve le s. pour ceux qui
\item[\vref{Es 45:17}] par Yahweh, d'un s. éternel ; vs. ne
\item[\vref{Es 46:13}] loin ; et mon s., il ne tardera
\item[\vref{Es 51:6}] pareillement ; mais mon s. demeurera éternellement, et
\item[\vref{Es 59:17}] le casque du s. est sur sa
\item[\vref{Es 60:18}] appelleras tes murailles : S. ; et tes portes :
\item[\vref{Es 61:10}] des vêtements du s., il m'a couvert
\item[\vref{Joë 2:32}] sauvé ; car le s. sera sur la
\item[\vref{Jon 2:10}] faits : Car le s. vient de Yahweh.
\item[\vref{Mi 7:7}] Dieu de mon s. ; mon Dieu m'exaucera.
\item[\vref{Lu 1:77}] la connaissance du s., par la rémission
\item[\vref{Lu 2:30}] Car mes yeux ont vu ton s.,
\item[\vref{Lu 3:6}] Et tte chair verra le s. de Dieu.
\item[\vref{Lu 19:9}] dit : Aujourd'hui le s. est entré ds
\item[\vref{Jn 4:22}] connaissons ; car le s. vient des Juifs.
\item[\vref{Ac 4:12}] n'y a de s. en aucun autre ;
\item[\vref{Ac 16:17}] ils vs. annoncent la voie du s. !
\item[\vref{Ro 1:16}] Dieu pour le s. de ts ceux
\item[\vref{Ro 10:10}] qu'on parvient au s., selon ce que
\item[\vref{Ro 13:11}] car mntnt le s. est plus près
\item[\vref{2 Co 6:2}] au jour du s.. Voici mntnt le
\item[\vref{2 Co 7:10}] une repentance à s. dont on ne
\item[\vref{Ep 1:13}] l'Evangile de votre s., en lui vs.
\item[\vref{Ep 6:17}] le casque du s., et l'épée de
\item[\vref{Ph 1:19}] tournera à mon s. par vos prières
\item[\vref{Ph 2:12}] œuvre votre propre s. avec crainte et
\item[\vref{1 Th 5:9}] à l'acquisition du s. par notre Seign.
\item[\vref{Hé 1:14}] ceux qui doivent recevoir l'héritage du s. ?
\item[\vref{Hé 2:3}] un si grand s., qui a commencé
\item[\vref{Hé 2:10}] Prince de lr. s. par les afflictions.
\item[\vref{Hé 5:9}] devenu l'auteur du s. éternel pour ts
\item[\vref{Hé 6:9}] et convenables au s., quoique ns. parlions
\item[\vref{1 Pi 1:5}] ns. obtenions le s., qui est prêt
\item[\vref{1 Pi 1:9}] à savoir le s. de vos âmes.
\item[\vref{1 Pi 1:10}] sujet de ce s. que les prophètes,
\item[\vref{Jud 1:3}] sujet de notre s. commun, j'ai jugé
\item[\vref{Ap 7:10}] en disant : Le s. est à notre
\item[\vref{Ap 12:10}] disait : Maintenant le s. est arrivé, ainsi
\item[\vref{Ap 19:1}] disant : Alléluia ! Le s., la gloire, l'honneur
\end{listverse}

\ConcordanceEntry{Samarie}
\vspace{-2mm}
\begin{listverse}
\item[\vref{1 R 16:24}] la montagne de S., deux talents d'argent ;
\item[\vref{2 R 17:6}] roi d'Assyrie prit S. et emmena captifs
\item[\vref{Es 7:9}] d'Ephraïm c'est la S., et le chef
\item[\vref{Mi 1:1}] au sujet de S. et de Jérus.
\item[\vref{Jn 4:4}] Or il fallait qu'il traverse la S.
\item[\vref{Ac 1:8}] Judée, et la S., et jusqu'aux extrémités
\item[\vref{Ac 8:1}] de la Judée et de la S.
\item[\vref{Ac 9:31}] Galilée, et la S., étant édifiées et
\end{listverse}

\ConcordanceEntry{Samaritain}
\vspace{-2mm}
\begin{listverse}
\item[\vref{2 R 17:29}] des hauts lieux bâties par les S.
\item[\vref{Lu 10:33}] Mais un S., qui voyageait, étant venu là, fut
\item[\vref{Jn 4:7}] Et une fem. s. vint puiser de
\item[\vref{Jn 8:48}] tu es un S., et que tu
\item[\vref{Ac 8:25}] annoncèrent l'Evangile ds plusieurs villages des S.
\end{listverse}

\ConcordanceEntry{Samson}
\vspace{-2mm}
\begin{listverse}
\item[\vref{Jg 13:24}] du nom de S.. L'enfant devint grand,
\end{listverse}
\begin{legend}
\NoAutoSpaceBeforeFDP{
\item Juge en Israël : Jg 13:24; 16:31
\item Samson victoire sur les Philistins : Jg 15:15-16; 16:29-30
\item Les exploits de S : g 14:6-19; 15:4; 16:3
\item Sa mort : Jg 16:30-31
}
\end{legend}

\ConcordanceEntry{Samuel}
\vspace{-2mm}
\begin{listverse}
\item[\vref{1 S 1:20}] elle le nomma S., parce que, dit-elle,
\end{listverse}
\begin{legend}
\NoAutoSpaceBeforeFDP{
\item Naissance de S : 1 S 1:20-28
\item Yahweh appelle S : 1 S 3:10
\item Un appel de Prophète : 1 S 3:19-21; 4:1; 1 Ch 9:22
\item Juge en Israël : 1 S 7:15-16; Ac 13:20
\item S. oint Saül comme roi : 1 S 10:1; 12:1-2
\item Prophétie et jugement sur Saül : 1 S 13:13-14; 15:28
\item S. oint David roi : 1 S 16:12-13
\item Autres : 1 S 16:1; 19:18; 25:1; 28:11; Ps 99:6
}
\end{legend}

\ConcordanceEntry{Sanballat}
\vspace{-2mm}
\begin{listverse}
\item[\vref{Né 2:10}] Quand S., le Horonite, et Tobija, le serviteur
\item[\vref{Né 6:2}] Alors S. et Guéschem m'envoyèrent dire : Viens, et
\end{listverse}

\ConcordanceEntry{Sanchérib}
\vspace{-2mm}
\begin{listverse}
\item[\vref{2 R 18:13}] du roi Ezéchias, S., roi d'Assyrie, monta
\item[\vref{Es 36:1}] du roi Ezéchias, S., roi d'Assyrie, monta
\end{listverse}

\ConcordanceEntry{Sanctification}
\vspace{-2mm}
\begin{listverse}
\item[\vref{Ro 6:22}] pour fruit la s. et pour fin
\item[\vref{1 Co 1:30}] Dieu, sagesse, justice, s. et rédemption,
\item[\vref{2 Co 7:1}] l'esprit, perfectionnant la s. ds la crainte
\item[\vref{1 Th 4:3}] à savoir votre s., et que vs.
\item[\vref{2 Th 2:13}] salut par la s. de l'Esprit, et
\item[\vref{1 Ti 2:15}] et ds la s., avec modestie.
\item[\vref{Hé 12:14}] ts, et la s., sans laquelle nul
\item[\vref{1 Pi 1:2}] Père, par la s. de l'Esprit afin
\end{listverse}

\ConcordanceEntry{Sanctifier}
\vspace{-2mm}
\begin{listverse}
\item[\vref{Ge 2:3}] jour, et le s., parce qu'en ce
\item[\vref{Ex 13:2}] S.-moi tt premier-né, tt premier-né issu
\item[\vref{Ex 20:8}] du jour du repos pour le s.
\item[\vref{Ex 30:29}] Ainsi, tu les s., et ils seront
\item[\vref{Lé 8:12}] tête d'Aaron, et l'oignit pour le s.
\item[\vref{Lé 10:3}] disant : Je serai s. par ceux qui
\item[\vref{Lé 11:44}] Dieu ; vs. vs. s. dc et vs.
\item[\vref{No 20:12}] moi, pour me s. aux yeux des
\item[\vref{Jos 3:5}] dit au peuple : S.-vs., car Yahweh
\item[\vref{Job 1:5}] enfants pour les s., et se levant
\item[\vref{Es 5:16}] Dieu saint sera s. ds la justice.
\item[\vref{Es 8:13}] S. Yahweh des armées, lui-mm, c'est lui
\item[\vref{Jn 10:36}] le Père a s., et qu’il
\item[\vref{Jn 17:17}] S.-les par ta vérité ; ta parole
\item[\vref{Jn 17:19}] Et je me s. moi-mm pour eux, afin qu'eux aussi
\item[\vref{Ro 15:16}] soit agréable, étant s. par le Saint-Esprit.
\item[\vref{1 Co 1:2}] ceux qui sont s. en Jésus-Christ, appelés
\item[\vref{1 Co 6:11}] vs. avez été s., mais vs. avez
\item[\vref{1 Co 7:14}] mari incrédule est s. par la fem.,
\item[\vref{Ep 5:26}] afin de la s. en la purifiant
\item[\vref{1 Th 5:23}] paix veuille vs. s. entièrement, et faire
\item[\vref{1 Ti 4:5}] que tt est s. par la parole
\item[\vref{2 Ti 2:21}] un vase d'honneur, s. et utile au
\item[\vref{Hé 2:11}] et celui qui s. et ceux qui
\item[\vref{Hé 9:9}] ne pouvaient pas s. la conscience de
\item[\vref{Hé 9:13}] qui sont souillés, s. et procurent la
\item[\vref{Hé 10:1}] continuellement chaque année, s. ceux qui s'y
\item[\vref{Hé 10:10}] que ns. sommes s., à savoir par
\item[\vref{Hé 10:14}] parfaits pour toujours ceux qui sont s.
\item[\vref{Hé 10:29}] il avait été s., et qui aura
\item[\vref{Hé 13:12}] Jésus, afin de s. le peuple par
\item[\vref{1 Pi 3:15}] mais s. le Seign. ds vos cœurs, et
\item[\vref{Jud 1:1}] que Dieu a s. et gardés pour
\item[\vref{Ap 22:11}] que celui qui est saint se s. encore !
\end{listverse}

\ConcordanceEntry{Sanctuaire}
\vspace{-2mm}
\begin{listverse}
\item[\vref{Ex 25:8}] me feront un s., et j'habiterai au
\item[\vref{Lé 16:33}] pour le saint s. et il fera
\item[\vref{No 18:1}] porterez l'iniquité du s. ; et toi, et
\item[\vref{1 Ch 22:19}] et bâtissez le s. de Yahweh Dieu,
\item[\vref{Ps 68:18}] d'eux, le Sinaï est ds le s.
\item[\vref{Ps 73:17}] entré ds le s. de Dieu, et
\item[\vref{Ez 28:18}] as profané tes s. par la multitude
\item[\vref{Da 8:13}] Jusqu'à qnd le s. et l'armée seront-ils
\item[\vref{Da 9:17}] face sur ton s. dévasté !
\item[\vref{Hé 9:24}] entré ds un s. fait de main
\end{listverse}

\ConcordanceEntry{Sang}
\vspace{-2mm}
\begin{listverse}
\item[\vref{Ge 4:10}] La voix du s. de ton frère
\item[\vref{Ge 9:6}] aura versé le s. de l'hom., par
\item[\vref{Ex 4:9}] fleuve deviendront du s. sur la terre.
\item[\vref{Ex 12:7}] prendront de son s., et le mettront
\item[\vref{Ex 12:13}] Et le s. sera pour vs. un signe sur
\item[\vref{Ex 24:8}] dc prit le s., et le répandit
\item[\vref{Ex 30:10}] générations, avec le s. de l'offrande pour
\item[\vref{Lé 7:27}] mangé de qq s. que ce soit,
\item[\vref{Lé 17:11}] est ds le s.. C'est pourquoi je
\item[\vref{2 S 23:17}] N'est-ce pas le s. de ces hommes
\item[\vref{1 Ch 22:8}] répandu beaucoup de s., et tu as
\item[\vref{Ps 51:16}] de tant de s., et ma langue
\item[\vref{Ps 72:14}] violence, et lr. s. sera précieux dvt
\item[\vref{Jé 2:34}] se trouve le s. des pauvres innocents,
\item[\vref{Joë 2:30}] la terre, du s. et du feu,
\item[\vref{Mt 9:20}] d'une perte de s. depuis douze ans,
\item[\vref{Mt 16:17}] chair et le s. qui t'ont révélé
\item[\vref{Mt 23:35}] vs. tt le s. innocent qui a
\item[\vref{Mt 26:28}] ceci est mon s., le sang de
\item[\vref{Mt 27:4}] en trahissant le s. innocent. Mais ils
\item[\vref{Mt 27:24}] suis innocent du s. de ce juste.
\item[\vref{Mt 27:25}] répondit : Que son s. retombe sur ns.
\item[\vref{Lu 13:1}] avait mêlé le s. avec celui de
\item[\vref{Jn 1:13}] nés, non du s., ni de la
\item[\vref{Jn 19:34}] il sortit du s. et de l'eau.
\item[\vref{Ac 15:20}] fornication, des animaux étouffés et du s.
\item[\vref{Ac 20:28}] qu'il a acquise par son propre s.
\item[\vref{Ro 3:15}] pieds sont légers pour répandre le s. ;
\item[\vref{Ro 5:9}] justifiés par son s., serons-ns. sauvés de
\item[\vref{1 Co 15:50}] chair et le s. ne peuvent hériter
\item[\vref{Ga 1:16}] consultai ni la chair ni le s.,
\item[\vref{Ep 1:7}] rédemption par son s., à savoir la
\item[\vref{Ep 2:13}] rapprochés par le s. de Christ.
\item[\vref{Ep 6:12}] chair et le s., mais contre les
\item[\vref{Col 1:20}] paix par le s. de sa croix,
\item[\vref{Hé 9:14}] combien plus le s. de Christ, qui,
\item[\vref{Hé 9:22}] purifiées par le s., et sans effusion
\item[\vref{Hé 9:25}] saints, chaque année, avec un autre s. ;
\item[\vref{Hé 10:4}] impossible que le s. des taureaux et
\item[\vref{Hé 10:29}] chose profane le s. de l'Alliance, par
\item[\vref{Hé 12:4}] jusqu'à répandre votre s. en combattant contre
\item[\vref{Hé 12:24}] Alliance, et du s. de l'aspersion, qui
\item[\vref{Hé 13:12}] par son propre s., a souffert hors
\item[\vref{1 Pi 1:2}] l'aspersion de son s. : Que la grâce
\item[\vref{1 Pi 1:19}] mais par le s. précieux de Christ,
\item[\vref{1 Jn 1:7}] autres, et le s. de son Fils
\item[\vref{1 Jn 5:6}] l'eau et le s. ; et pas seulement
\item[\vref{Ap 8:8}] tiers de la mer devint du s.,
\item[\vref{Ap 16:6}] ont répandu le s. des saints et
\item[\vref{Ap 17:6}] fem. ivre du s. des saints, et
\end{listverse}

\ConcordanceEntry{Sanhédrin}
\vspace{-2mm}
\begin{listverse}
\item[\vref{Mt 26:59}] et tt le s. cherchaient des faux
\item[\vref{Lu 22:66}] et firent amener Jésus ds le s.
\item[\vref{Jn 11:47}] pharisiens assemblèrent le s., et ils dirent :
\item[\vref{Ac 5:34}] leva ds le s., et ordonna de
\item[\vref{Ac 22:30}] à tt le s. de se réunir ;
\item[\vref{Ac 23:1}] regardant fixement le s., dit : Hommes frères !
\end{listverse}

\ConcordanceEntry{Santé}
\vspace{-2mm}
\begin{listverse}
\item[\vref{Ex 18:7}] l'autre, de lr. s., puis ils entrèrent
\item[\vref{Pr 4:22}] trouvent, et la s. de tt le
\item[\vref{Pr 16:24}] à l'âme et s. pour les os.
\item[\vref{Es 6:10}] convertisse, et qu'il ne recouvre la s.
\item[\vref{Mt 9:12}] sont en bonne s. qui ont besoin
\item[\vref{Lu 5:31}] qui sont en s. n'ont pas besoin
\item[\vref{3 Jn 1:2}] sois en bonne s., com. ton âme
\end{listverse}

\ConcordanceEntry{Saphir}
\vspace{-2mm}
\begin{listverse}
\item[\vref{Ex 24:10}] un ouvrage de s. transparent, com. le
\item[\vref{Ex 28:18}] une escarboucle, un s., et un jaspe.
\item[\vref{Es 54:11}] et je te fonderai sur des s. ;
\item[\vref{Ez 1:26}] une pierre de s., en forme de
\item[\vref{Ap 21:19}] le second de s., le troisième de
\end{listverse}

\ConcordanceEntry{Sardes}
\vspace{-2mm}
\begin{listverse}
\item[\vref{Ap 1:11}] à Thyatire, à S., à Philadelphie, et
\item[\vref{Ap 3:1}] de l'église de S. : Voici ce que
\item[\vref{Ap 3:4}] de personnes à S. qui n'ont pas
\end{listverse}

\ConcordanceEntry{Sardoine}
\vspace{-2mm}
\begin{listverse}
\item[\vref{Ex 28:17}] on mettra une s., une topaze, et
\item[\vref{Ap 4:3}] jaspe et de s. ; et le trône
\item[\vref{Ap 21:20}] le sixième de s., le septième de
\end{listverse}

\ConcordanceEntry{Sarepta}
\vspace{-2mm}
\begin{listverse}
\item[\vref{1 R 17:9}] Lève-toi, va à S., qui appartient à
\item[\vref{Lu 4:26}] fem. veuve à S., ds le pays
\end{listverse}

\ConcordanceEntry{Sarment}
\vspace{-2mm}
\begin{listverse}
\item[\vref{Ge 40:10}] cep avait trois s.. Quand il eut
\item[\vref{Ge 40:12}] interprétation : Les trois s. sont trois jours.
\item[\vref{Jn 15:2}] Il retranche tt s. qui est en
\item[\vref{Jn 15:5}] en êtes les s.. Celui qui demeure
\end{listverse}

\ConcordanceEntry{Saron}
\vspace{-2mm}
\begin{listverse}
\item[\vref{1 Ch 27:29}] Schithraï de S. était commis sur
\item[\vref{Es 33:9}] et flétri. Le S. est com. un
\item[\vref{Ac 9:35}] Lydde et à S. le virent, et
\end{listverse}

\ConcordanceEntry{Satan}
\vspace{-2mm}
\begin{listverse}
\item[\vref{1 Ch 21:1}] Mais S. s'éleva contre Israël, et il incita
\item[\vref{Job 1:6}] dvt Yahweh, et S. entra parmi eux.
\item[\vref{Za 3:1}] de Yahweh, et S. qui se tenait
\item[\vref{Mt 4:10}] lui dit : Retire-toi, S. ! Car il est
\item[\vref{Mt 12:26}] Or si S. chasse Satan, il est divisé contre
\item[\vref{Mc 4:15}] l'ont entendue, aussitôt S. vient et enlève
\item[\vref{Lu 10:18}] dit : Je voyais S. tomber du ciel
\item[\vref{Lu 22:3}] Mais S. entra ds Judas, surnommé Iscariot, qui
\item[\vref{Ac 5:3}] dit : Ananias, comment S. s'est-il emparé de
\item[\vref{Ac 26:18}] la puissance de S. à Dieu ; afin
\item[\vref{Ro 16:20}] paix brisera bientôt S. sous vos pieds.
\item[\vref{1 Co 5:5}] soit livré à S. pour la destruction
\item[\vref{2 Co 2:11}] afin que S. n'ait pas le dessus sur ns.,
\item[\vref{2 Co 11:14}] pas étonnant car S. lui-mm se déguise
\item[\vref{2 Th 2:9}] la puissance de S., avec ttes sortes
\item[\vref{1 Ti 5:15}] se sont déjà détournées pour suivre S.
\item[\vref{Ap 2:13}] le trône de S.. Et que cependant
\item[\vref{Ap 2:24}] les profondeurs de S., com. ils disent,
\item[\vref{Ap 12:9}] le diable et S., celui qui séduit
\item[\vref{Ap 20:7}] ans seront accomplis, S. sera délié de
\end{listverse}

\ConcordanceEntry{Satisfaire}
\vspace{-2mm}
\begin{listverse}
\item[\vref{Pr 13:4}] qu'il ne peut s., mais l'âme des
\item[\vref{Mc 15:15}] Pilate, voulant s. la foule, lr.
\end{listverse}

\ConcordanceEntry{Saul}
\vspace{-2mm}
\begin{listverse}
\item[\vref{Ac 8:1}] Or S. consentait à la mort d'Etienne, et
\item[\vref{Ac 13:9}] Alors S., appelé aussi Paul, rempli du Saint-Esprit,
\end{listverse}

\ConcordanceEntry{Saül}
\vspace{-2mm}
\begin{listverse}
\item[\vref{1 S 10:1}] la tête de S.. Il l'embrassa, et
\end{listverse}
\begin{legend}
\NoAutoSpaceBeforeFDP{
\item Premier roi d'Israël : 1 S 10:1; 11:14-15
\item Les guerres sous le règne de S : 1 S 14:47-48
\item Impatience et désobéissance de S : 1 S 13:13-14; 15:11-28
\item David entre au service de Saül : 1 S 16:14-23
\item S. cherche à tuer David : 1 S 18:9; 19:1; 23:14-15
\item S. fait tuer les prêtres : 1 S 22: 18-19
\item David épargne la vie de S : 1 S 24: 19; 26:9-10
\item Saül chez la femme qui évoque les morts : 1 S 28:7
\item Sa mort : 1 S 28:19; 31:1-4
\item Autres : 2 S 9: 1; 21:1; 1 Ch 10:13; Ac 13:21
}
\end{legend}

\ConcordanceEntry{Sauterelle}
\vspace{-2mm}
\begin{listverse}
\item[\vref{Ex 10:14}] fit monter les s. sur tt le
\item[\vref{No 13:33}] et à leurs yeux com. des s.
\item[\vref{Ps 78:46}] leurs récoltes aux s., le produit de
\item[\vref{Pr 30:27}] les s., qui n'ont point de roi, et
\item[\vref{Joë 1:4}] La s. a dévoré les restes du gazam,
\item[\vref{Ap 9:3}] Des s. sortirent de la fumée du puits
\end{listverse}

\ConcordanceEntry{Sauver}
\vspace{-2mm}
\begin{listverse}
\item[\vref{Ge 19:17}] l'un d'eux dit : S. ta vie, ne
\item[\vref{2 S 22:3}] force qui me s., ma haute retraite
\item[\vref{Ps 18:28}] Car tu s. le peuple affligé, et tu abaisses
\item[\vref{Ps 20:7}] déjà que Yahweh s. son Oint ; il
\item[\vref{Ps 28:9}] S. ton peuple et bénis ton héritage !
\item[\vref{Ps 33:16}] roi n'est point s. par une grande
\item[\vref{Ps 116:6}] j'étais devenu misérable et il m'a s.
\item[\vref{Ps 119:94}] suis à toi, s.-moi ; car je
\item[\vref{Es 45:20}] et invoquent un dieu qui ne s. pas.
\item[\vref{Es 45:22}] moi, et soyez s. ; car JE SUIS
\item[\vref{Es 59:1}] courte pour pouvoir s., ni son oreille
\item[\vref{Jé 8:20}] fini, et ns. ne sommes pas s. !
\item[\vref{Jé 17:14}] je serai guéri ; s.-moi et je
\item[\vref{Os 7:13}] Je voudrais les s., mais ils profèrent
\item[\vref{So 3:17}] le Puissant qui s. ; il se réjouira
\item[\vref{Mt 1:21}] c'est lui qui s. son peuple de
\item[\vref{Mt 10:22}] qui persévérera jusqu'à la fin sera s.
\item[\vref{Mt 14:30}] commençait à enfoncer, il s'écria : Seign. ! S.-moi !
\item[\vref{Mt 16:25}] Car quiconque voudra s. son âme la
\item[\vref{Mt 18:11}] est venu pour s. ce qui était
\item[\vref{Mt 19:25}] et dirent : Qui peut dc être s. ?
\item[\vref{Mt 24:22}] personne ne serait s. ; mais à cause
\item[\vref{Mt 27:42}] Il a s. les autres et il ne peut
\item[\vref{Lu 8:12}] qu'ils ne croient et ne soient s.
\item[\vref{Lu 13:23}] gens qui soient s. ? Il lr. répondit :
\item[\vref{Lu 23:39}] es le Christ, s.-toi toi-mm, et
\item[\vref{Jn 3:17}] mais afin que le monde soit s. par lui.
\item[\vref{Jn 5:34}] ces choses, afin que vs. soyez s.
\item[\vref{Jn 12:47}] monde, mais pour s. le monde.
\item[\vref{Ac 2:40}] exhortait, en disant : S.-vs. de cette
\item[\vref{Ac 4:12}] hommes par lequel ns. devions être s.
\item[\vref{Ac 15:11}] que ns. serons s. par la grâce
\item[\vref{Ac 16:31}] et tu seras s., toi et ta
\item[\vref{Ro 11:26}] tt Israël sera s., selon qu'il est
\item[\vref{1 Co 1:21}] à Dieu de s. les croyants par
\item[\vref{1 Co 3:15}] lui, il sera s., toutefois com. par
\item[\vref{1 Co 7:16}] fem., si tu s. ton mari ? Ou
\item[\vref{1 Co 9:22}] ts, afin d'en s. au moins quelques-uns.
\item[\vref{Ep 2:8}] Car vs. êtes s. par la grâce,
\item[\vref{2 Th 2:10}] l'amour de la vérité pour être s.
\item[\vref{1 Ti 1:15}] le monde pour s. les pécheurs, dont
\item[\vref{1 Ti 2:4}] les hommes soient s. et qu'ils viennent
\item[\vref{1 Ti 2:15}] Elle sera néanmoins s. en mettant des
\item[\vref{1 Ti 4:16}] ainsi, tu te s. toi-mm et tu
\item[\vref{Hé 5:7}] qui pouvait le s. de la mort,
\item[\vref{Hé 7:25}] aussi il peut s. parfaitement ceux qui
\item[\vref{Ja 1:21}] et qui peut s. vos âmes.
\item[\vref{Ja 2:14}] les œuvres ? Cette foi peut-elle le s. ?
\item[\vref{Ja 4:12}] Législateur, qui peut s. et qui peut
\item[\vref{Ja 5:15}] faite avec foi s. le malade, et
\item[\vref{Ja 5:20}] de son égarement, s. une âme de
\item[\vref{1 Pi 3:20}] huit personnes, furent s. par l'eau.
\item[\vref{1 Pi 4:18}] juste est difficilement s., que deviendront l'impie
\item[\vref{Jud 1:23}] et s. les autres par la frayeur, les
\item[\vref{Ap 21:24}] qui auront été s. marcheront à la
\end{listverse}

\ConcordanceEntry{Sauveur}
\vspace{-2mm}
\begin{listverse}
\item[\vref{Ex 15:2}] a été mon S., mon Dieu. Je
\item[\vref{2 S 22:3}] refuge. Ô mon S. ! Tu me délivres
\item[\vref{Es 19:20}] lr. enverra un s., quelqu'un de grand,
\item[\vref{Es 43:11}] moi il n'y a point de S.
\item[\vref{Es 62:11}] Sion : Voici, ton S. vient ; voici, son
\item[\vref{Es 63:8}] il a été pour eux un S.
\item[\vref{Lu 1:69}] suscité un puissant S. ds la maison
\item[\vref{Lu 2:11}] est né le S., qui est le
\item[\vref{Jn 4:42}] le Christ, le S. du monde.
\item[\vref{Ac 5:31}] être Prince et S., afin de donner
\item[\vref{Ac 13:23}] pour être le S. d'Israël.
\item[\vref{Ep 5:23}] corps, et dont il est le S.
\item[\vref{Ph 3:20}] attendons aussi le S., le Seign. Jésus-Christ,
\item[\vref{1 Ti 2:3}] bon et agréable dvt Dieu, notre S.,
\item[\vref{1 Ti 4:10}] qui est le S. de ts les
\item[\vref{Tit 2:10}] choses la doctrine de Dieu, notre S.
\item[\vref{Tit 2:13}] Dieu et notre S. Jésus-Christ,
\item[\vref{2 Pi 2:20}] du Seign. et S. Jésus-Christ, ils s'y
\item[\vref{1 Jn 4:14}] pour être le S. du monde.
\item[\vref{Jud 1:25}] seul sage, notre S., par Jésus-Christ, notre
\end{listverse}

\ConcordanceEntry{Savoir (le)}
\vspace{-2mm}
\begin{listverse}
\item[\vref{Pr 1:5}] il augmentera son s., et l'hom. intelligent
\item[\vref{Ac 26:24}] Paul ! Ton grand s. ds les lettres
\end{listverse}

\ConcordanceEntry{Savoir}
\vspace{-2mm}
\begin{listverse}
\item[\vref{Pr 19:25}] intelligent, il discernera ce qu'il faut s.
\item[\vref{Ec 9:1}] pour l'éclaircir à s. que les justes,
\item[\vref{Mt 2:8}] l'aurez trouvé, faites-le-moi s., afin que j'aille
\item[\vref{Lu 1:62}] son père pour s. comment il voulait
\item[\vref{Jn 14:5}] dc pouvons-ns. en s. le chemin ?
\item[\vref{1 Co 2:2}] la pensée de s. parmi vs. autre
\item[\vref{1 Co 8:2}] si quelqu'un croit s. qq chose, il
\item[\vref{Hé 11:8}] il partit sans s. où il allait.
\end{listverse}

\ConcordanceEntry{Scandale}
\vspace{-2mm}
\begin{listverse}
\item[\vref{Es 8:14}] un rocher de s. pour les deux
\item[\vref{Mt 13:41}] Royaume ts les s. et ceux qui
\item[\vref{Mt 16:23}] Tu m'es en s., car tu ne
\item[\vref{Lu 17:1}] n'arrive pas de s. ; mais malheur à
\item[\vref{Ro 14:21}] chute, ou de s. ou de faiblesse.
\item[\vref{Ro 16:17}] divisions et des s. contre la doctrine
\item[\vref{1 Co 1:23}] qui est un s. pour les Juifs
\item[\vref{Ga 5:11}] encore persécuté ? Le s. de la croix
\item[\vref{1 Pi 2:8}] un rocher de s. ; ils se heurtent
\end{listverse}

\ConcordanceEntry{Scandaliser}
\vspace{-2mm}
\begin{listverse}
\item[\vref{Mt 15:12}] pharisiens ont été s. qnd ils ont
\item[\vref{Mt 17:27}] ns. ne les s. pas, va à
\item[\vref{Mt 18:6}] Mais, quiconque s. un de ces
\item[\vref{Mc 14:27}] ts cette nuit s. en moi ; car
\item[\vref{Jn 6:61}] ce sujet, lr. dit : Cela vs. s.-t-il ?
\item[\vref{1 Co 8:13}] si la viande s. mon frère, je
\end{listverse}

\ConcordanceEntry{Sceau, Marque}
\vspace{-2mm}
\begin{listverse}
\item[\vref{Ge 4:15}] Yahweh mit une m. sur Caïn afin
\item[\vref{Ca 8:6}] Place-moi com. un s. sur ton cœur,
\item[\vref{Jn 6:27}] que Dieu, a marqué de son s.
\item[\vref{Ro 4:11}] la circoncision, com. s. de la justice,
\item[\vref{1 Co 9:2}] vs. êtes le s. de mon apostolat
\item[\vref{2 Ti 2:19}] ferme, ayant ce s. : Le Seign. connaît
\item[\vref{1 Jn 4:2}] Reconnaissez à cette m. l'Esprit de Dieu :
\item[\vref{Ap 5:1}] et en dehors, scellé de sept s.
\item[\vref{Ap 9:4}] n'avaient pas la m. de Dieu sur
\item[\vref{Ap 13:16}] esclaves, reçoivent une m. sur lr. main
\end{listverse}

\ConcordanceEntry{Sceller}
\vspace{-2mm}
\begin{listverse}
\item[\vref{Da 12:4}] ces paroles, et s. le livre jusqu'au
\item[\vref{Ep 1:13}] vs. avez été s. du Saint-Esprit qui
\item[\vref{Ep 4:30}] vs. avez été s. pour le jour
\item[\vref{Ap 22:10}] dit aussi : Ne s. pas les paroles
\end{listverse}

\ConcordanceEntry{Sceptre}
\vspace{-2mm}
\begin{listverse}
\item[\vref{Ge 49:10}] Le s. ne s'éloignera point de Juda, ni
\item[\vref{No 24:17}] Jacob, et un S. s'est élevé d'Israël.
\item[\vref{Est 5:2}] à Esther le s. d'or qui était
\item[\vref{Hé 1:8}] siècles ; et le s. de ton Royaume
\item[\vref{Ap 2:27}] gouvernera avec un s. de fer, et
\end{listverse}

\ConcordanceEntry{Schamgar}
\vspace{-2mm}
\begin{listverse}
\item[\vref{Jg 3:31}] il y eut S., fils d'Anath. Il
\item[\vref{Jg 5:6}] Aux jours de S., fils d'Anath, aux
\end{listverse}

\ConcordanceEntry{Scheol}
\vspace{-2mm}
\begin{listverse}
\item[\vref{1 S 2:6}] fait descendre au s. et qui en
\item[\vref{Job 11:8}] profond que le s. : Qu'y connaîtras-tu ?
\item[\vref{Job 17:13}] attendre que le s., qui va être
\item[\vref{Ps 6:6}] toi ; qui te célébrera ds le s. ?
\item[\vref{Ps 16:10}] mon âme au s., tu ne permettras
\item[\vref{Ps 30:4}] mon âme du s., tu m'as rendu
\item[\vref{Ps 86:13}] as retiré mon âme du profond s.
\item[\vref{Ps 88:4}] maux, et ma vie atteint le s.
\item[\vref{Ps 139:8}] couche ds le s., t'y voilà.
\item[\vref{Pr 15:11}] Le s. et le gouffre sont dvt Yahweh ;
\item[\vref{Pr 27:20}] Le s. et le gouffre ne sont jamais
\item[\vref{Ec 9:10}] car ds le s., où tu vas,
\item[\vref{Es 14:11}] descendue ds le s., avec le son
\item[\vref{Es 14:15}] précipité ds le s., ds les profondeurs
\item[\vref{Es 28:15}] accord avec le s. ; qnd le fléau
\item[\vref{Ez 31:16}] descendre ds le s., avec ceux qui
\item[\vref{Os 13:14}] la puissance du s., je les délivrerai
\end{listverse}

\ConcordanceEntry{Schilo}
\vspace{-2mm}
\begin{listverse}
\item[\vref{Ge 49:10}] ce que le S. vienne, et que
\end{listverse}

\ConcordanceEntry{Schimeï}
\vspace{-2mm}
\begin{listverse}
\item[\vref{2 S 16:5}] de Saül, nommé S., fils de Guéra.
\item[\vref{1 R 2:8}] as avec toi S., fils de Guéra,
\end{listverse}

\ConcordanceEntry{Schinear}
\vspace{-2mm}
\begin{listverse}
\item[\vref{Ge 10:10}] Accad, et Calné, au pays de S.
\item[\vref{Da 1:2}] au pays de S., ds la maison
\item[\vref{Za 5:11}] le pays de S. ; et qnd elle
\end{listverse}

\ConcordanceEntry{Science}
\vspace{-2mm}
\begin{listverse}
\item[\vref{Ex 28:3}] de l'esprit de s., afin qu'ils fassent
\item[\vref{Job 37:16}] de celui qui est parfait en s. ?
\item[\vref{Ps 19:3}] fait connaître sa s. à l'autre nuit.
\item[\vref{Ps 139:6}] Ta s. est trop merveilleuse pour moi, elle
\item[\vref{Pr 9:9}] le juste et il croîtra en s.
\item[\vref{Pr 15:14}] prudent cherche la s. ; mais la bouche
\item[\vref{Ec 1:16}] vu beaucoup de sagesse et de s.
\item[\vref{Da 1:17}] gens de la s. et de l'intelligence
\item[\vref{Da 1:20}] fois plus de s. en eux que
\item[\vref{Mal 2:7}] doivent garder la s., et c'est de
\item[\vref{1 Co 13:2}] mystères et la s. de ttes choses ;
\item[\vref{1 Co 14:6}] révélation, ou par s., ou par prophétie,
\item[\vref{1 Ti 6:20}] les contradictions d'une s. faussement ainsi nommée,
\end{listverse}

\ConcordanceEntry{Scorpion}
\vspace{-2mm}
\begin{listverse}
\item[\vref{Lu 10:19}] et sur les s., et sur tte
\item[\vref{Lu 11:12}] demande un œuf, lui donnera-t-il un s. ?
\item[\vref{Ap 9:3}] pouvoir qu'ont les s. de la terre.
\end{listverse}

\ConcordanceEntry{Scribe}
\vspace{-2mm}
\begin{listverse}
\item[\vref{Esd 7:6}] qui était un s. bien versé ds
\item[\vref{Mt 7:29}] de l'autorité, et non com. les s.
\item[\vref{Mt 13:52}] C'est pourquoi, tt s. instruit de ce
\item[\vref{Mt 17:10}] Pourquoi dc les s. disent-ils qu'il faut
\item[\vref{Mt 23:13}] malheur à vs., s. et pharisiens hypocrites !
\item[\vref{Mt 23:34}] sages et des s.. Vous tuerez et
\item[\vref{Mc 1:22}] autorité et non pas com. les s.
\item[\vref{Lu 5:21}] Alors les s. et les pharisiens
\item[\vref{Lu 20:39}] Quelques-uns des s. prenant la parole,
\item[\vref{1 Co 1:20}] Où est le s. ? Où est le
\end{listverse}

\ConcordanceEntry{Sec}
\vspace{-2mm}
\begin{listverse}
\item[\vref{Ge 1:9}] et que le s. paraisse ; et il
\item[\vref{Ex 14:21}] la mer à s., et les eaux
\item[\vref{Jos 4:22}] Israël a passé ce Jourdain à s.
\item[\vref{Mt 21:20}] figuier est-il devenu s. en un instant ?
\item[\vref{Hé 11:29}] com. un lieu s., ce que les
\end{listverse}

\ConcordanceEntry{Sécheresse}
\vspace{-2mm}
\begin{listverse}
\item[\vref{Job 24:19}] Comme la s. et la chaleur consument les eaux
\item[\vref{Ps 32:4}] changée en une s. d'été. Sélah.
\item[\vref{Ps 107:33}] désert, et les sources d'eaux en s. ;
\item[\vref{Es 58:11}] ds les grandes s., il fortifiera tes
\item[\vref{Ag 1:11}] j'ai appelé la s. sur la terre,
\end{listverse}

\ConcordanceEntry{Secourir}
\vspace{-2mm}
\begin{listverse}
\item[\vref{1 S 7:12}] Yahweh ns. a s. jusqu'en ce lieu-ci.
\item[\vref{2 Ch 18:31}] et Yahweh le s., et Dieu les
\item[\vref{Job 29:12}] qui criait au s., et l'orphelin qui
\item[\vref{Ps 22:20}] point ! Ma force, hâte-toi de me s. !
\item[\vref{Ac 16:9}] pria, disant : Passe en Macédoine et s.-ns. !
\item[\vref{1 Co 12:28}] de guérir, de s., de gouverner, de
\item[\vref{2 Co 6:2}] favorable et t'ai s. au jour du
\item[\vref{Hé 2:18}] est puissant pour s. ceux qui sont
\end{listverse}

\ConcordanceEntry{Secours}
\vspace{-2mm}
\begin{listverse}
\item[\vref{Ps 38:23}] venir à mon s. ! Seign., tu es
\item[\vref{Ps 46:2}] force, et notre s. qui ne manque
\item[\vref{Ps 60:13}] Donne-ns. du s. pour sortir de
\item[\vref{Ps 89:20}] J'ai ordonné mon s. en faveur d'un
\item[\vref{Ps 115:9}] Il est le s. et le bouclier
\item[\vref{Ps 121:2}] Mon s. vient de Yahweh qui a fait
\item[\vref{Ps 124:8}] Notre s. est ds le Nom de Yahweh,
\item[\vref{Ac 11:29}] ses moyens, qq s. pour subvenir aux
\item[\vref{2 Co 4:8}] ds la perplexité, mais non sans s. ;
\item[\vref{Ph 1:19}] et par le s. de l'Esprit de
\end{listverse}

\ConcordanceEntry{Secret}
\vspace{-2mm}
\begin{listverse}
\item[\vref{Ge 49:6}] ds lr. conseil s., que ma gloire
\item[\vref{2 S 12:12}] as agi en s. ; mais moi, je
\item[\vref{Job 11:6}] qu'il t'explique les s. de sa sagesse,
\item[\vref{Pr 20:19}] médit révèle les s. ; ne te mêle
\item[\vref{Pr 21:14}] don fait en s. apaise la colère,
\item[\vref{Da 2:19}] Et le s. fut révélé à Daniel ds une
\item[\vref{Am 3:7}] avoir révélé son s. à ses serviteurs
\item[\vref{Mt 2:7}] ayant appelé en s. les mages, s'informa
\item[\vref{Mt 10:26}] ni rien de s. qui ne doive
\item[\vref{Jn 18:20}] et je n'ai rien dit en s.
\item[\vref{Jn 19:38}] Jésus, mais en s. parce qu'il craignait
\item[\vref{Ro 2:16}] Dieu jugera les s. des hommes par
\item[\vref{1 Co 14:25}] ainsi les s. de son cœur
\item[\vref{2 Co 12:4}] a entendu des s. qu'il n'est pas
\item[\vref{Ep 5:12}] dire les choses qu'ils font en s. ;
\end{listverse}

\ConcordanceEntry{Secte}
\vspace{-2mm}
\begin{listverse}
\item[\vref{Ac 5:17}] à savoir la s. des sadducéens, et
\item[\vref{Ac 9:2}] quelques-uns de cette s., hommes ou femmes,
\item[\vref{Ac 15:5}] quelques-uns, de la s. des pharisiens qui
\item[\vref{Ac 24:5}] chef de la s. des Nazaréens.
\item[\vref{Ac 24:14}] voie qu'ils appellent s., je sers ainsi
\item[\vref{Ac 24:22}] qui concerne cette s., les ajourna, en
\item[\vref{Ac 26:5}] pharisien, selon la s. la plus rigide
\item[\vref{Ac 28:22}] quant à cette s., il ns. est
\item[\vref{Ga 5:20}] animosités, les disputes, les divisions, les s.,
\item[\vref{2 Pi 2:1}] introduiront secrètement des s. pernicieuses, et qui
\item[\vref{2 Pi 2:2}] plusieurs suivront leurs s. de perdition; et
\end{listverse}

\ConcordanceEntry{Sécurité}
\vspace{-2mm}
\begin{listverse}
\item[\vref{Ps 4:9}] Yahweh ! tu me fais reposer en s.
\item[\vref{Ps 16:9}] réjouit et mon corps repose en s.
\item[\vref{Ps 30:7}] Dans ma s., je disais : Je ne serai jamais
\item[\vref{Pr 1:33}] m'écoute habitera en s. et sera tranquille,
\item[\vref{Es 32:17}] justice, repos et s. pour toujours.
\item[\vref{Jé 23:6}] Israël demeurera en s. ; et c'est ici
\item[\vref{Jé 32:37}] je les y ferai habiter en s.
\item[\vref{Ph 3:1}] choses, mais pour vs. c'est une s.
\end{listverse}

\ConcordanceEntry{Sédécias}
\vspace{-2mm}
\begin{listverse}
\item[\vref{1 R 22:11}] S., fils de Kenaana, s'était fait des
\item[\vref{2 R 24:17}] changea son nom en celui de S.
\item[\vref{2 R 25:1}] du règne de S., le dixième jour
\item[\vref{Jé 29:21}] Kolaja, et sur S., fils de Maaséja,
\end{listverse}

\ConcordanceEntry{Sédition}
\vspace{-2mm}
\begin{listverse}
\item[\vref{Ps 18:44}] me délivres des s. du peuple, tu
\item[\vref{Mc 15:7}] complices pour une s., ds laquelle ils
\item[\vref{Lu 23:19}] prison pour une s. qui avait eu
\item[\vref{Ac 19:40}] d'être accusés de s. pour ce qui
\item[\vref{2 Co 12:20}] des rapports, de l'orgueil et des s.
\end{listverse}

\ConcordanceEntry{Séducteur}
\vspace{-2mm}
\begin{listverse}
\item[\vref{2 Co 6:8}] bonne réputation ; com. s. et toutefois vrais ;
\item[\vref{1 Ti 4:1}] à des esprits s. et à des
\item[\vref{Tit 1:10}] vains discoureurs, et s. d'esprits, principalement ceux
\item[\vref{2 Jn 1:7}] Car plusieurs s. sont venus ds
\end{listverse}

\ConcordanceEntry{Séduction}
\vspace{-2mm}
\begin{listverse}
\item[\vref{Mt 13:22}] siècle et la s. des richesses étouffent
\item[\vref{Mc 4:19}] ce monde, la s. des richesses et
\item[\vref{2 Th 2:10}] avec ttes les s. de l'iniquité, pour
\item[\vref{Hé 3:13}] s'endurcisse par la s. du péché.
\item[\vref{2 Pi 3:17}] autres par la s. des abominables, vs.
\end{listverse}

\ConcordanceEntry{Séduire}
\vspace{-2mm}
\begin{listverse}
\item[\vref{Ge 3:13}] Le serpent m'a s., et j'en ai
\item[\vref{2 Ch 18:20}] Moi, je le s.. Yahweh lui dit :
\item[\vref{Mt 24:5}] Et ils en s. plusieurs.
\item[\vref{Mt 24:24}] des miracles, pour s. mm les élus,
\item[\vref{Mc 13:22}] des miracles pour s. mm les élus,
\item[\vref{Ro 16:18}] des flatteries, ils s. les cœurs des
\item[\vref{2 Co 11:3}] com. le serpent s. Eve par sa
\item[\vref{Ep 4:14}] lr. ruse à s. artificieusement.
\item[\vref{1 Ti 2:14}] qui a été s., mais la fem.,
\item[\vref{Ja 1:26}] en bride, mais s. son cœur, la
\item[\vref{Ap 2:20}] prophétesse, enseigner et s. mes serviteurs pour
\item[\vref{Ap 12:9}] Satan, celui qui s. tte la terre,
\item[\vref{Ap 13:14}] Et elle s. les habitants de la terre, à
\item[\vref{Ap 20:8}] il sortira pour s. les nations qui
\end{listverse}

\ConcordanceEntry{Seigneur}
\vspace{-2mm}
\begin{listverse}
\item[\vref{Ge 15:2}] Abram répondit : S. Yahweh, que me
\item[\vref{De 10:17}] des dieux, le S. des seigneurs, le
\item[\vref{Mal 3:1}] son temple le S. que vs. cherchez ;
\item[\vref{Mt 4:10}] Tu adoreras le S., ton Dieu, et
\item[\vref{Mt 7:21}] qui me disent : S. ! Seign. ! n'entreront pas
\item[\vref{Mt 22:43}] l'Esprit, l'appelle-t-il son S., disant :
\item[\vref{Mc 12:29}] Ecoute Israël, le S., notre Dieu, le
\item[\vref{Lu 1:28}] été faite ; le S. est avec toi ;
\item[\vref{Lu 6:46}] Mais pourquoi m'appelez-vs. S., Seign. ! et ne
\item[\vref{Jn 20:28}] lui dit : Mon S., et mon Dieu !
\item[\vref{Jn 21:7}] Pierre : C'est le S.. Et qnd Simon
\item[\vref{Ac 2:34}] lui-mm dit : Le S. a dit à
\item[\vref{Ac 2:36}] Dieu a fait S. et Christ, ce
\item[\vref{Ac 10:36}] paix par Jésus-Christ, qui est le S. de ts.
\item[\vref{Ro 10:12}] ont un mm S., qui est riche
\item[\vref{1 Co 8:6}] et un seul S. : Jésus-Christ, par qui
\item[\vref{1 Co 12:3}] Jésus est le S. ! Si ce n'est
\item[\vref{2 Co 3:17}] Or le S. est l'Esprit et là où est
\item[\vref{Ep 4:5}] a un seul S., une seule foi,
\item[\vref{Ph 4:5}] les hommes. Le S. est proche.
\item[\vref{1 Th 4:17}] la rencontre du S., ds les airs
\item[\vref{2 Ti 2:19}] ce sceau : Le S. connaît ceux qui
\item[\vref{1 Pi 2:3}] toutefois vs. avez goûté combien le S. est bon.
\item[\vref{Ap 4:8}] Saint est le S. Dieu Tout-puissant, QUI
\item[\vref{Ap 19:16}] ROIS ET LE S. DES SEIGNEURS.
\end{listverse}

\ConcordanceEntry{Sein}
\vspace{-2mm}
\begin{listverse}
\item[\vref{Ex 4:6}] main ds ton s., et il mit
\item[\vref{2 R 19:3}] sont près du s. maternel, mais il
\item[\vref{Lu 6:38}] versera ds votre s. une bonne mesure,
\item[\vref{Lu 16:22}] anges ds le s. d'Abraham. Le riche
\item[\vref{Jn 1:18}] est ds le s. du Père, est
\item[\vref{Jn 3:4}] rentrer ds le s. de sa mère
\item[\vref{Jn 13:23}] couché sur le s. de Jésus.
\end{listverse}

\ConcordanceEntry{Séir}
\vspace{-2mm}
\begin{listverse}
\item[\vref{No 24:18}] sera sa possession, S. sera possédé par
\item[\vref{De 2:5}] la montagne de S. en héritage.
\item[\vref{De 33:2}] sur eux de S., il a resplendi
\item[\vref{2 Ch 25:11}] dix mille hommes des fils de S.
\item[\vref{Ez 35:2}] la montagne de S., et prophétise contre
\end{listverse}

\ConcordanceEntry{Sel}
\vspace{-2mm}
\begin{listverse}
\item[\vref{Ge 19:26}] et elle devint une statue de s.
\item[\vref{Ex 30:35}] y mettras du s. ; vs. le ferez
\item[\vref{Lé 2:13}] Tu mettras du s. sur ttes tes
\item[\vref{2 R 2:21}] y jeta le s., et dit : Ainsi
\item[\vref{Mt 5:13}] Vous êtes le s. de la terre.
\item[\vref{Mc 9:50}] Le s. est une bonne chose ; mais si
\item[\vref{Col 4:6}] toujours assaisonnée de s., avec grâce, afin
\end{listverse}

\ConcordanceEntry{Sem}
\vspace{-2mm}
\begin{listverse}
\item[\vref{Ge 5:32}] cents ans, engendra S., Cham, et Japhet.
\item[\vref{Ge 9:23}] Alors S. et Japhet prirent un manteau qu'ils
\item[\vref{Ge 10:21}] des fils à S., père de ts
\end{listverse}

\ConcordanceEntry{Semaine}
\vspace{-2mm}
\begin{listverse}
\item[\vref{Ex 34:22}] fête solennelle des s. au temps des
\item[\vref{Lé 23:15}] doit agiter, sept s. entières.
\item[\vref{Da 9:24}] y a soixante-dix s. fixées sur ton
\item[\vref{Da 9:27}] plusieurs pour une s., et à la
\item[\vref{Mt 28:1}] jour de la s., Marie de Magdala
\item[\vref{Jn 20:19}] premier de la s., les portes du
\item[\vref{Ac 20:7}] jour de la s., les disciples étant
\item[\vref{1 Co 16:2}] jour de la s., chacun de vs.
\end{listverse}

\ConcordanceEntry{Semblable}
\vspace{-2mm}
\begin{listverse}
\item[\vref{Ge 2:18}] seul ; je lui ferai une aide s. à lui.
\item[\vref{Ex 8:6}] que nul n'est s. à Yahweh, notre
\item[\vref{Ps 18:34}] rend mes pieds s. à ceux des
\item[\vref{Es 14:14}] nuées, je serai s. au Très-Haut.
\item[\vref{Mt 7:26}] en pratique, sera s. à un hom.
\item[\vref{Mc 7:13}] vs. faites encore beaucoup d'autres choses s.
\item[\vref{Lu 20:36}] parce qu'ils seront s. aux anges, et
\item[\vref{Ro 5:14}] par une transgression s. à celle d'Adam,
\item[\vref{Ro 8:3}] ds une chair s. à celle du
\item[\vref{Ga 5:21}] et les choses s. à celles-là, au
\item[\vref{Ep 5:27}] ni rien de s., mais sainte et
\item[\vref{Hé 2:17}] fallu qu'il soit s. en ttes choses
\item[\vref{Hé 4:11}] en imitant une s. rébellion.
\item[\vref{Ja 1:23}] pratique, il est s. à un hom.
\item[\vref{2 Pi 2:12}] Mais eux, s. à des bêtes
\item[\vref{1 Jn 3:2}] apparaîtra, ns. serons s. à lui, car
\item[\vref{Ap 13:4}] disant : Qui est s. à la bête,
\end{listverse}

\ConcordanceEntry{Semence}
\vspace{-2mm}
\begin{listverse}
\item[\vref{Ge 1:11}] portant de la s., et des arbres
\item[\vref{Ps 126:6}] il porte la s. pour la mettre
\item[\vref{Ec 11:6}] Sème ta s. dès le matin, et ne laisse
\item[\vref{Es 55:10}] donner de la s. au semeur, et
\item[\vref{Es 61:11}] fait germer ses s., ainsi le Seign.
\item[\vref{Ag 2:19}] encore de la s. ds les greniers ?
\item[\vref{Mt 13:19}] a reçu la s. le long du
\item[\vref{Mt 13:24}] de la bonne s. ds son champ.
\item[\vref{Mt 13:38}] monde ; la bonne s. ce sont les
\item[\vref{Mc 4:27}] et jour, la s. germe et croît,
\item[\vref{Mc 4:31}] de ttes les s. qui sont jetées
\item[\vref{Lu 8:5}] pour semer sa s.. Et en semant,
\item[\vref{Lu 8:11}] cette parabole : La s., c'est la parole
\item[\vref{1 Co 15:38}] à chacune des s. son propre corps.
\item[\vref{2 Co 9:10}] fournit de la s. au semeur, veuille
\item[\vref{1 Pi 1:23}] non par une s. corruptible, mais par
\item[\vref{1 Jn 3:9}] péché, car la s. de Dieu demeure
\end{listverse}

\ConcordanceEntry{Semer}
\vspace{-2mm}
\begin{listverse}
\item[\vref{No 20:5}] où l'on puisse s., ni un lieu
\item[\vref{Es 32:20}] Heureux vs. qui s. sur ttes les
\item[\vref{Jé 4:3}] arable, et ne s. pas parmi les
\item[\vref{Os 10:12}] S. selon la justice, moissonnez selon la
\item[\vref{Ag 1:6}] Vous avez s. beaucoup, mais vs.
\item[\vref{1 Co 15:42}] Le corps est s. corruptible, il ressuscitera
\item[\vref{Ga 6:7}] qu'un hom. aura s., il le moissonnera
\item[\vref{Ja 3:18}] la justice est s. ds la paix
\end{listverse}

\ConcordanceEntry{Semeur}
\vspace{-2mm}
\begin{listverse}
\item[\vref{Es 55:10}] la semence au s., et du pain
\item[\vref{Mt 13:3}] il dit : Un s. sortit pour semer.
\item[\vref{Mc 4:14}] Le s. c'est celui qui sème la parole.
\item[\vref{Lu 8:5}] Un s. sortit pour semer sa semence. Et
\item[\vref{2 Co 9:10}] la semence au s., veuille aussi vs.
\end{listverse}

\ConcordanceEntry{Sénevé}
\vspace{-2mm}
\begin{listverse}
\item[\vref{Mt 13:31}] au grain de s. qu'un hom. a
\item[\vref{Mt 17:20}] un grain de s., vs. diriez à
\end{listverse}

\ConcordanceEntry{Sens}
\vspace{-2mm}
\begin{listverse}
\item[\vref{1 S 25:3}] fem. de bon s. et belle de
\item[\vref{Ps 119:66}] Enseigne-moi le bon s. et la connaissance
\item[\vref{Pr 1:3}] leçon de bon s., de justice, de
\item[\vref{Pr 9:4}] à ceux qui sont dépourvus de s. :
\item[\vref{Pr 19:8}] qui acquiert du s. aime son âme,
\item[\vref{Mc 3:21}] ils disaient : Il est hors de s.
\item[\vref{Mc 5:15}] ds son bon s. ; et ils furent
\item[\vref{2 Co 5:13}] sommes hors de s., c'est pour Dieu,
\item[\vref{Hé 5:14}] l'habitude, ont leurs s. exercés à discerner
\end{listverse}

\ConcordanceEntry{Sentier}
\vspace{-2mm}
\begin{listverse}
\item[\vref{Ps 23:3}] conduit ds les s. de la justice,
\item[\vref{Ps 25:4}] Fais-moi connaître tes voies, enseigne-moi tes s.
\item[\vref{Ps 25:10}] Kaf.] Tous les s. de Yahweh sont
\item[\vref{Ps 27:11}] conduis-moi ds le s. de la droiture,
\item[\vref{Ps 119:35}] marcher sur le s. de tes commandements
\item[\vref{Ps 119:105}] pieds et une lumière sur mon s.
\item[\vref{Ps 142:4}] tu connais mon s.. Ils me tendent
\item[\vref{Pr 2:8}] pour garder les s. de la justice ;
\item[\vref{Pr 3:6}] tes voies et il dirigera tes s.
\item[\vref{Pr 3:17}] et ts ses s. ne sont que
\item[\vref{Pr 4:18}] Mais le s. des justes est com. la lumière
\item[\vref{Es 42:16}] marcher par des s. qu'ils ne connaissent
\item[\vref{Es 59:8}] pervertis ds leurs s., ts ceux qui
\item[\vref{Jé 6:16}] et enquérez-vs. des s. des siècles passés,
\item[\vref{Mt 3:3}] le chemin du Seign., aplanissez ses s.
\end{listverse}

\ConcordanceEntry{Sentiment}
\vspace{-2mm}
\begin{listverse}
\item[\vref{Ps 5:8}] sainteté avec les s. d'une crainte respectueuse.
\item[\vref{Ac 17:11}] Juifs avaient des s. plus nobles que
\item[\vref{Ro 12:3}] chacun ait des s. modestes, selon la
\item[\vref{Ro 12:16}] Ayez les mêmes s. les uns envers
\item[\vref{Ep 4:19}] ont perdu tt s., et se sont
\item[\vref{Ph 2:5}] vs. le mm s. qui a été
\item[\vref{Ph 4:2}] être d'un mm s. ds le Seign.
\item[\vref{1 Pi 3:8}] ts d'un mm s., remplis de compassion
\item[\vref{2 Pi 3:1}] mes avertissements, les s. purs que vs.
\end{listverse}

\ConcordanceEntry{Sentinelle}
\vspace{-2mm}
\begin{listverse}
\item[\vref{Es 21:6}] Va, place la s., et qu'elle rapporte
\item[\vref{Es 21:11}] de Séir : Ô s. ! Qu'en est-il de
\item[\vref{Es 52:8}] Tes s. élèvent leurs voix, elles se réjouissent
\item[\vref{Ez 3:17}] t'établis pour être s. sur la maison
\item[\vref{Ez 33:6}] Si la s. voit venir l'épée, et qu'elle ne
\item[\vref{Mi 7:4}] annoncé par tes s., ton châtiment arrive.
\item[\vref{Ha 2:1}] me tenais en s., j'étais debout ds
\end{listverse}

\ConcordanceEntry{Sentir}
\vspace{-2mm}
\begin{listverse}
\item[\vref{Ge 27:27}] et l'embrassa. Isaac s. l'odeur de ses
\item[\vref{Mc 5:29}] s'arrêta ; et elle s. en son corps
\item[\vref{Ac 17:16}] à Athènes, il s. au-dedans son esprit
\end{listverse}

\ConcordanceEntry{Séparer}
\vspace{-2mm}
\begin{listverse}
\item[\vref{Ge 1:4}] bonne ; et Dieu s. la lumière des
\item[\vref{Lé 20:24}] qui vs. ai s. des autres peuples.
\item[\vref{Jé 15:19}] et si tu s. la chose précieuse
\item[\vref{Mt 13:49}] les anges viendront s. les méchants d'avec
\item[\vref{Mt 19:6}] l'hom. dc ne s. pas ce que
\item[\vref{Ac 19:9}] se retira d'eux, s. les disciples, et
\item[\vref{Ro 8:39}] ne pourra ns. s. de l'amour de
\item[\vref{1 Co 7:10}] fem. ne se s. pas de son
\item[\vref{1 Co 7:15}] si l'incrédule se s., qu'il se sépare ;
\item[\vref{Ga 5:4}] Vous êtes s. de Christ, vs.
\item[\vref{1 Th 2:17}] été qq temps s. de vs. de
\item[\vref{Hé 7:26}] innocent, sans tache, s. des pécheurs, et
\end{listverse}

\ConcordanceEntry{Séphora}
\vspace{-2mm}
\begin{listverse}
\item[\vref{Ex 2:21}] hom.-là, qui donna S., sa fille, à
\item[\vref{Ex 4:25}] Et S. prit un couteau tranchant, coupa le
\end{listverse}

\ConcordanceEntry{Sépulcre}
\vspace{-2mm}
\begin{listverse}
\item[\vref{Ge 23:4}] une possession de s. parmi vs., afin
\item[\vref{Ex 14:11}] avait pas des s. en Egypte pour
\item[\vref{No 19:16}] humains, ou un s., sera impur durant
\item[\vref{2 R 23:17}] répondirent : C'est le s. de l'hom. de
\item[\vref{Né 2:3}] où sont les s. de mes pères
\item[\vref{Ps 5:10}] gosier est un s. ouvert, ils flattent
\item[\vref{Es 53:9}] a mis son s. parmi les méchants,
\item[\vref{Mt 27:52}] Et les s. s'ouvrirent et plusieurs corps des saints
\item[\vref{Mt 27:60}] mit ds un s. neuf, qu'il s'était
\item[\vref{Mc 16:2}] se rendirent au s., com. le soleil
\item[\vref{Jn 5:28}] sont ds les s. entendront sa voix,
\item[\vref{Jn 11:17}] déjà depuis quatre jours ds le s.
\item[\vref{Ac 2:29}] et que son s. existe encore parmi
\item[\vref{Ro 3:13}] gosier est un s. ouvert ; ils ont
\end{listverse}

\ConcordanceEntry{Séraphins}
\vspace{-2mm}
\begin{listverse}
\item[\vref{Es 6:2}] Les s. se tenaient au-dessus de lui ; et
\item[\vref{Es 6:6}] Mais l'un des s. vola vers moi,
\end{listverse}

\ConcordanceEntry{Serment}
\vspace{-2mm}
\begin{listverse}
\item[\vref{Ex 22:11}] le s. de Yahweh interviendra entre les deux
\item[\vref{Lé 5:1}] la parole du s., aura péché en
\item[\vref{De 7:8}] qu'il garde le s. qu'il a juré
\item[\vref{De 29:12}] Dieu, ds ce s., que Yahweh, ton
\item[\vref{1 S 14:26}] bouche, car le peuple craignait le s.
\item[\vref{Né 6:18}] à lui par s., parce qu'il était
\item[\vref{Ps 15:4}] s'il fait un s. à son préjudice ;
\item[\vref{Ec 8:2}] à cause du s. fait à Dieu.
\item[\vref{Jé 11:5}] que j'accomplisse le s. que j'ai juré
\item[\vref{Mt 5:33}] ce que tu auras promis par s.
\item[\vref{Mt 14:7}] lui promit avec s. de lui donner
\item[\vref{Mt 26:72}] nia encore avec s., disant : Je ne
\item[\vref{Lu 1:73}] selon le s. par lequel il avait juré à
\item[\vref{Hé 6:16}] qu'eux, et le s. qu'ils font pour
\end{listverse}

\ConcordanceEntry{Serpent}
\vspace{-2mm}
\begin{listverse}
\item[\vref{Ge 3:1}] Or le s. était le plus rusé de ts
\item[\vref{Ge 49:17}] Dan sera un s. sur le chemin,
\item[\vref{Ex 4:3}] elle devint un s.. Et Moïse s'enfuyait
\item[\vref{No 21:6}] le peuple des s. brûlants qui mordaient
\item[\vref{No 21:9}] Moïse fit un s. d'airain, et le
\item[\vref{2 R 18:4}] il brisa le s. d'airain que Moïse
\item[\vref{Ps 140:3}] langue com. un s., il y a
\item[\vref{Es 27:1}] le Léviathan, le s. fuyard, le Léviathan,
\item[\vref{Jé 8:17}] contre vs. des s., des vipères, contre
\item[\vref{Mt 7:10}] demande un poisson, lui donnera-t-il un s. ?
\item[\vref{Mt 10:16}] prudents com. des s., et simples com.
\item[\vref{Lu 10:19}] marcher sur les s. et sur les
\item[\vref{Ap 12:9}] grand dragon, le s. ancien, appelé le
\item[\vref{Ap 12:15}] sa gueule, le s. lança de l'eau
\end{listverse}

\ConcordanceEntry{Serrer}
\vspace{-2mm}
\begin{listverse}
\item[\vref{Ps 119:11}] Je s. ta parole ds mon cœur afin
\end{listverse}

\ConcordanceEntry{Servante}
\vspace{-2mm}
\begin{listverse}
\item[\vref{Ge 21:10}] Abraham : Chasse cette s. et son fils,
\item[\vref{Ps 123:2}] yeux de la s. regardent la main
\item[\vref{Pr 31:15}] donne à ses s. lr. portion.
\item[\vref{Joë 2:29}] sur les serviteurs et sur les s.
\item[\vref{Lu 1:38}] dit : Voici la s. du Seign. ; qu'il
\item[\vref{Ac 2:18}] et sur mes s., et ils prophétiseront.
\item[\vref{Ac 12:13}] vestibule, et une s., appelée Rhode, vint
\item[\vref{Ac 16:16}] la prière, une s. qui avait un
\item[\vref{Ac 16:19}] maîtres de la s. voyant disparaître l'espoir
\end{listverse}

\ConcordanceEntry{Service}
\vspace{-2mm}
\begin{listverse}
\item[\vref{1 S 2:11}] garçon vaquait au s. de Yahweh, en
\item[\vref{1 R 8:11}] pour faire le s., à cause de
\item[\vref{2 R 5:2}] qui était au s. de la fem.
\item[\vref{1 Ch 29:7}] donnèrent pour le s. de la maison
\item[\vref{Ac 6:4}] prière et au s. de la parole.
\item[\vref{Ac 20:24}] joie, et le s. que j'ai reçu
\item[\vref{Ro 11:13}] qu'apôtre des Gentils, je glorifie mon s.,
\item[\vref{Ro 12:7}] soit le s., appliquons-ns. au service ; si quelqu'un est
\item[\vref{Ro 15:31}] Judée, que mon s. que j'ai à
\item[\vref{1 Co 12:5}] aussi diversité de s., mais il n'y
\item[\vref{1 Co 16:15}] entièrement appliqués au s. des saints.
\item[\vref{2 Co 3:8}] comment le s. de l'Esprit ne
\item[\vref{2 Co 4:1}] pourquoi, ayant ce s. selon la miséricorde
\item[\vref{2 Co 5:18}] a donné le s. de la réconciliation.
\item[\vref{2 Co 5:20}] exhortait par notre s. ; ns. vs. supplions
\item[\vref{2 Co 6:3}] afin que notre s. ne soit pas
\item[\vref{Ep 4:12}] pour l'œuvre du s., pour l'édification du
\item[\vref{1 Ti 1:12}] estimé fidèle en m'établissant ds le s.,
\item[\vref{2 Ti 1:18}] personne combien  de s. il m'a rendu
\item[\vref{2 Ti 4:5}] évangéliste, rends ton s. pleinement approuvé.
\item[\vref{2 Ti 4:11}] il m'est fort utile pour le s.
\item[\vref{Hé 8:6}] a obtenu un s. d'autant supérieur qu'il
\item[\vref{1 Pi 4:10}] de vs. rende s. aux autres selon
\item[\vref{Ap 2:19}] ta charité, ton s., ta foi, ta
\item[\vref{Ap 19:10}] ton compagnon de s., et celui de
\end{listverse}

\ConcordanceEntry{Servir}
\vspace{-2mm}
\begin{listverse}
\item[\vref{Ge 11:3}] la brique lr. s. de pierre, et
\item[\vref{Ex 3:12}] peuple d'Egypte, vs. s. Dieu près de
\item[\vref{Ex 23:33}] moi ; car tu s. leurs dieux, et
\item[\vref{De 10:12}] d'aimer et de s. Yahweh, ton Dieu,
\item[\vref{Jos 24:15}] vs. déplaît de s. Yahweh, choisissez aujourd'hui
\item[\vref{Es 61:3}] de Yahweh, pour s. à sa gloire.
\item[\vref{Da 3:18}] que ns. ne s. pas tes dieux,
\item[\vref{Mt 4:10}] et tu le s. lui seul.
\item[\vref{Mt 6:24}] Nul ne peut s. deux maîtres. Car,
\item[\vref{Mt 16:26}] Et que s.-il à un hom. de gagner
\item[\vref{Mt 20:28}] venu pour être s., mais pour servir,
\item[\vref{Lu 1:74}] ennemis, de le s. sans crainte,
\item[\vref{Ro 1:25}] ont adoré et s. la créature, au
\item[\vref{Ro 6:19}] vos membres pour s. à l'impureté et
\item[\vref{1 Co 14:6}] inconnues, que vs. s. cela si je
\item[\vref{Ga 5:2}] vs. faites circoncire, Christ ne vs. s. à rien.
\item[\vref{Col 3:22}] la chair, ne s. pas seulement sous
\item[\vref{1 Th 1:9}] des idoles, pour s. le Dieu vivant
\item[\vref{1 Ti 2:10}] des femmes qui font profession de s. Dieu.
\item[\vref{Hé 4:2}] entendirent ne lr. s. de rien, parce
\item[\vref{Hé 9:14}] œuvres mortes, pour s. le Dieu vivant ?
\end{listverse}

\ConcordanceEntry{Serviteur}
\vspace{-2mm}
\begin{listverse}
\item[\vref{Ge 9:25}] il sera le s. des serviteurs de
\item[\vref{Ge 14:14}] ses plus braves s., nés ds sa
\item[\vref{1 R 10:8}] gens ! Heureux tes s. qui se tiennent
\item[\vref{Job 4:18}] pas à ses s., il impute des
\item[\vref{Ps 103:21}] qui êtes ses s., faisant sa volonté !
\item[\vref{Ps 104:4}] et des flammes de feu ses s.
\item[\vref{Ps 116:16}] je suis ton s., je suis ton
\item[\vref{Ps 134:1}] vs. ts les s. de Yahweh, qui
\item[\vref{Pr 17:2}] Le s. prudent sera maître sur l'enfant qui
\item[\vref{Pr 29:21}] Le s. devient enfin fils de celui qui
\item[\vref{Pr 30:10}] calomnie pas un s. dvt son maître,
\item[\vref{Ec 7:21}] n'entendes pas ton s. médire de toi.
\item[\vref{Es 42:1}] Voici mon s., que je soutiens,
\item[\vref{Es 43:10}] Yahweh, et mon s. que j'ai élu,
\item[\vref{Es 52:13}] Voici, mon s. prospérera, il sera
\item[\vref{Es 53:11}] sera rassasié ; mon s. juste justifiera beaucoup
\item[\vref{Es 54:17}] est l'héritage des s. de Yahweh, et
\item[\vref{Es 61:6}] on vs. nommera s. de notre Dieu ;
\item[\vref{Mt 8:13}] l'heure mm son s. fut guéri.
\item[\vref{Mt 20:26}] grand entre vs., qu'il soit votre s. ;
\item[\vref{Mt 21:34}] il envoya ses s. vers les vignerons
\item[\vref{Mt 24:45}] est dc le s. fidèle et prudent,
\item[\vref{Mt 25:21}] bon et fidèle s. ; tu as été
\item[\vref{Lu 12:38}] bénis sont ces s., s'il les trouve
\item[\vref{Lu 17:10}] Nous sommes des s. inutiles ; et ce
\item[\vref{Jn 2:5}] mère dit aux s. : Faites tt ce
\item[\vref{Jn 13:16}] vs. dis : Le s. n'est pas plus
\item[\vref{Jn 15:15}] vs. appelle plus s., car le serviteur
\item[\vref{Ro 13:4}] magistrat est un s. de Dieu pour
\item[\vref{2 Co 6:4}] ttes choses, com. s. de Dieu, par
\item[\vref{2 Co 11:15}] d'étonnement si ses s. aussi se déguisent
\item[\vref{Ga 1:10}] serais pas un s. de Christ.
\item[\vref{Ep 6:5}] S., obéissez à vos maîtres selon la
\item[\vref{1 Ti 4:6}] seras un bon s. de Jésus-Christ, nourri
\item[\vref{Hé 1:7}] et de ses s. des flammes de
\item[\vref{1 Pi 2:16}] la méchanceté, mais agissant com. des s. de Dieu.
\item[\vref{1 Pi 2:18}] S., soyez soumis en tte crainte à
\end{listverse}

\ConcordanceEntry{Servitude}
\vspace{-2mm}
\begin{listverse}
\item[\vref{Ex 1:13}] les enfants d'Israël à une rude s.
\item[\vref{Ex 13:3}] la maison de s. ; car Yahweh vs.
\item[\vref{1 R 12:4}] mntnt cette rude s. de ton père
\item[\vref{Ro 8:15}] un esprit de s., pour être encore
\item[\vref{Ro 8:21}] affranchie de la s. de la corruption,
\item[\vref{Ga 5:1}] plus sous le joug de la s.
\item[\vref{Hé 2:15}] assujettis tte lr. vie à la s.
\end{listverse}

\ConcordanceEntry{Seth}
\vspace{-2mm}
\begin{listverse}
\item[\vref{Ge 4:25}] du nom de S., car dit-il, Dieu
\end{listverse}

\ConcordanceEntry{Seul}
\vspace{-2mm}
\begin{listverse}
\item[\vref{Ge 2:18}] que l'hom. soit s. ; je lui ferai
\item[\vref{Ge 2:24}] sa fem., et ils deviendront une s. chair.
\item[\vref{Ge 11:6}] ce n'est qu'un s. et mm peuple,
\item[\vref{Ge 32:24}] Jacob demeura s.. Alors un hom.
\item[\vref{Lé 13:46}] impur. Il demeurera s. ; sa demeure sera
\item[\vref{Jos 23:10}] Un s. hom. d'entre vs. en poursuivait mille ;
\item[\vref{1 R 19:10}] suis resté, moi s. et ils me
\item[\vref{Ps 73:25}] je ne prends plaisir qu'en toi s.
\item[\vref{Mal 2:10}] pas ts un s. Père ? N'est-ce pas
\item[\vref{Mt 5:18}] la loi un s. iota ou un
\item[\vref{Mt 5:29}] pour toi qu'un s. de tes membres
\item[\vref{Jn 12:24}] meurt, il reste s. ; mais s'il meurt,
\item[\vref{Jn 17:3}] connaissent, toi, le s. vrai Dieu, et
\item[\vref{Ac 17:26}] hommes, sortis d'un s. sang, habitent sur
\item[\vref{Ro 3:12}] fasse le bien, pas mm un s. ;
\item[\vref{Ro 5:12}] com. par un s. hom. le péché
\item[\vref{Ro 5:19}] la désobéissance d'un s. hom., plusieurs ont
\item[\vref{Ro 12:5}] ns. formons un s. corps en Christ,
\item[\vref{1 Co 12:11}] Un s. et mm Esprit opère ttes ces
\item[\vref{Ga 3:20}] pas médiateur d'un s., mais Dieu est
\item[\vref{Ep 2:16}] pour former un s. corps par sa
\item[\vref{1 Ti 3:2}] irrépréhensible, mari d'une s. fem., vigilant, modéré,
\item[\vref{Hé 9:27}] de mourir une s. fois, et après
\item[\vref{Hé 10:12}] avoir offert un s. sacrifice pour les
\item[\vref{Jud 1:4}] qui renient le s. dominateur, Jésus-Christ, notre
\item[\vref{Jud 1:25}] à Dieu, s. sage, notre Sauveur, par Jésus-Christ, notre
\end{listverse}

\ConcordanceEntry{Shalom}
\vspace{-2mm}
\begin{listverse}
\item[\vref{Jg 6:24}] donna pour nom Yahweh-S.. Cet autel, qui
\end{listverse}

\ConcordanceEntry{Shofar}
\vspace{-2mm}
\begin{listverse}
\item[\vref{Ex 19:16}] fort son de s., et tt le
\item[\vref{Jos 6:4}] prêtres porteront sept s. retentissants dvt l'arche.
\item[\vref{Ps 150:3}] au son du s. ! Louez-le avec le
\item[\vref{Es 18:3}] et qnd le s. sonnera, écoutez !
\item[\vref{Es 58:1}] voix com. un s., et annonce à
\item[\vref{Ez 33:3}] il sonne du s. et avertit le
\item[\vref{Os 8:1}] tu avais un s. ds ta bouche !
\item[\vref{So 1:16}] un jour de s. et de cris
\item[\vref{Za 9:14}] Yahweh, sonnera du s., il s'avancera ds
\end{listverse}

\ConcordanceEntry{Sichem}
\vspace{-2mm}
\begin{listverse}
\item[\vref{Ge 33:18}] la ville de S., ds le pays
\item[\vref{Ge 34:2}] fut aperçue de S., fils de Hamor,
\item[\vref{Ge 37:12}] les troupeaux de lr. père à S.
\item[\vref{Jos 20:7}] montagne de Nephthali ; S. ds la montagne
\item[\vref{Jos 24:1}] tribus d'Israël à S., et il convoqua
\end{listverse}

\ConcordanceEntry{Sidon, Sidoniens}
\vspace{-2mm}
\begin{listverse}
\item[\vref{Ge 10:15}] Et Canaan engendra S., son premier-né, et
\item[\vref{1 R 5:6}] sache couper le bois com. les S.
\item[\vref{Es 23:2}] de marchands de S., et de ceux
\item[\vref{Jé 25:22}] les rois de S., et aux rois
\item[\vref{Ez 28:21}] ta face vers S., et prophétise contre
\item[\vref{Mt 11:21}] Tyr et ds S., il y a
\item[\vref{Ac 12:20}] Tyriens et aux S.. Mais ils vinrent
\end{listverse}

\ConcordanceEntry{Siècle}
\vspace{-2mm}
\begin{listverse}
\item[\vref{Mt 6:13}] ds ts les s., le règne, la
\item[\vref{Lu 16:8}] enfants de ce s. sont plus prudents
\item[\vref{Ro 12:2}] conformez pas au s. présent, mais soyez
\item[\vref{2 Co 4:4}] dieu de ce s. a aveuglé l'entendement,
\item[\vref{Ga 1:4}] arracher du présent s. mauvais, selon la
\item[\vref{2 Ti 4:10}] aimé le présent s., et il s'en
\item[\vref{Hé 6:5}] de Dieu, et les puissances du s. à venir,
\item[\vref{1 Pi 4:11}] la force, aux s. des siècles. Amen !
\item[\vref{Ap 22:5}] ils régneront aux s. des siècles.
\end{listverse}

\ConcordanceEntry{Signe}
\vspace{-2mm}
\begin{listverse}
\item[\vref{Ge 1:14}] qui servent de s. pour les saisons,
\item[\vref{Ex 3:12}] tu auras ce s. que c'est moi
\item[\vref{Ex 31:13}] car c'est un s. entre moi et
\item[\vref{2 R 20:8}] Esaïe : A quel s. connaîtrai-je que Yahweh
\item[\vref{Es 7:14}] vs. donnera un s. : Voici, une vierge
\item[\vref{Es 8:18}] pour être un s. et un miracle
\item[\vref{Es 44:25}] qui dissipe les s. des menteurs, qui
\item[\vref{Ez 12:6}] pour être un s. pour la maison
\item[\vref{Za 3:8}] qui serviront de s.. Certainement voici, je
\item[\vref{Mt 3:11}] baptise d'eau en s. de repentance ; mais
\item[\vref{Mt 16:1}] faire voir un s. venant du ciel.
\item[\vref{Mt 24:30}] Alors le s. du Fils de l'hom. paraîtra ds
\item[\vref{Lu 21:11}] et de grands s. ds le ciel.
\item[\vref{Ac 2:22}] prodiges et les s. que Dieu a
\item[\vref{Ap 12:1}] Et un grand s. parut ds le
\end{listverse}

\ConcordanceEntry{Silas, Silvain}
\vspace{-2mm}
\begin{listverse}
\item[\vref{Ac 15:22}] appelé Barsabas, et S., hommes considérés entre
\item[\vref{Ac 15:40}] Paul, ayant choisi S. pour l'accompagner, partit
\item[\vref{2 Co 1:19}] moi, et par S., et par Timothée,
\item[\vref{1 Pi 5:12}] écrit brièvement par S., notre frère, que
\end{listverse}

\ConcordanceEntry{Silence}
\vspace{-2mm}
\begin{listverse}
\item[\vref{2 R 7:9}] ns. gardons le s. et si ns.
\item[\vref{Job 13:5}] vs. gardiez le s., que vs. gardiez
\item[\vref{Ps 37:7}] Daleth.] Garde le s. dvt Yahweh, et
\item[\vref{Ps 39:3}] muet, ds le s. ; je me suis
\item[\vref{La 3:26}] et d'attendre en s. la délivrance de
\item[\vref{Mc 4:39}] à la mer : S. ! Tais-toi ! Et le
\item[\vref{Mc 14:61}] il garda le s., et ne répondit
\item[\vref{Ap 8:1}] y eut un s. ds le ciel
\end{listverse}

\ConcordanceEntry{Silo}
\vspace{-2mm}
\begin{listverse}
\item[\vref{Jos 18:1}] d'Israël s'assembla à S., et ils y
\item[\vref{Jos 22:9}] et partirent de S., ds le pays
\item[\vref{1 S 1:9}] et bu à S.. Et le prêtre
\item[\vref{1 S 3:21}] se manifester ds S. ; car Yahweh se
\item[\vref{Ps 78:60}] la demeure de S., la tente où
\item[\vref{Jé 7:14}] vos pères, com. j'ai fait à S. ;
\end{listverse}

\ConcordanceEntry{Siloé}
\vspace{-2mm}
\begin{listverse}
\item[\vref{Es 8:6}] les eaux de S. qui coulent doucement,
\item[\vref{Jn 9:7}] au réservoir de S. (nom qui veut
\end{listverse}

\ConcordanceEntry{Siméon}
\vspace{-2mm}
\begin{listverse}
\item[\vref{Ge 29:33}] elle lui donna le nom de S.
\item[\vref{Ge 34:25}] fils de Jacob, S. et Lévi, frères
\item[\vref{No 1:23}] la tribu de S., qui furent dénombrés,
\item[\vref{Lu 2:25}] un hom. appelé S.. Et cet hom.
\end{listverse}

\ConcordanceEntry{Simon}
\vspace{-2mm}
\begin{listverse}
\item[\vref{Mt 10:4}] S. le Cananite, et Judas Iscariot, celui
\item[\vref{Mt 13:55}] pas Jacques, Joseph, S. et Jude ?
\end{listverse}
\begin{legend}
\NoAutoSpaceBeforeFDP{
\item S. frère du Seigneur : Mt 13:55; Mc 6:3
\item S. le Cananite ou le Zélote : Mt 10:4
\item S. hôte de Jésus : Lu 7:36-40
\item S. le lépreux : Mt 26: 6-13
\item S. de Cyrène forcé à porter la croix : Mc 15:21
\item S. le magicien : Ac 8:9-13; 18-24
\item S. le corroyeur : Ac 9:43; 10:17
\item Simon Pierre : Cf Pierre
}
\end{legend}

\ConcordanceEntry{Simple}
\vspace{-2mm}
\begin{listverse}
\item[\vref{Ps 19:8}] fidèle, il donne la sagesse au s.
\item[\vref{Ps 119:130}] éclaire, elle donne de l'intelligence aux s.
\item[\vref{Pr 1:4}] du discernement aux s., aux jeunes gens
\item[\vref{Pr 14:15}] Le s. croit à tte parole ; mais l'hom.
\item[\vref{Mt 10:16}] des serpents, et s. com. des colombes.
\item[\vref{Ro 16:18}] flatteries, ils séduisent les cœurs des s.
\item[\vref{Ro 16:19}] au bien, et s. quant au mal.
\item[\vref{1 Co 14:16}] qui est du s. peuple dira-t-il : Amen !
\end{listverse}

\ConcordanceEntry{Simplicité}
\vspace{-2mm}
\begin{listverse}
\item[\vref{Ac 2:46}] avec joie et s. de cœur ;
\item[\vref{Ro 12:8}] fasse ds la s. ; que celui qui
\item[\vref{2 Co 11:3}] détournant de la s. qui est en
\item[\vref{Ep 6:5}] tremblement, ds la s. de votre cœur,
\item[\vref{Col 3:22}] hommes, mais en s. de cœur, craignant
\end{listverse}

\ConcordanceEntry{Sinaï, Sina}
\vspace{-2mm}
\begin{listverse}
\item[\vref{Ex 19:1}] jour-là, ils vinrent au désert de S.
\item[\vref{Ex 19:18}] Or le mont S. était tt couvert
\item[\vref{Lé 7:38}] la montagne de S., le jour où
\item[\vref{De 33:2}] est venu de S., il s'est levé
\item[\vref{Jg 5:5}] dvt Yahweh, ce S. dvt Yahweh, le
\item[\vref{Ps 68:18}] milieu d'eux, le S. est ds le
\item[\vref{Ga 4:24}] L'une du Mont S., qui n'enfante que
\end{listverse}

\ConcordanceEntry{Sion}
\vspace{-2mm}
\begin{listverse}
\item[\vref{De 4:48}] la montagne de S., qui est l'Hermon,
\item[\vref{1 R 8:1}] la cité de David, qui est S.
\item[\vref{1 Ch 11:5}] la forteresse de S., qui est la
\item[\vref{Ps 2:6}] mon Roi sur S., la montagne de
\item[\vref{Ps 48:3}] la montagne de S. ; le côté nord,
\item[\vref{Ps 78:68}] la montagne de S., celle qu'il aime.
\item[\vref{Ps 125:1}] la montagne de S. : Elle ne chancelle
\item[\vref{Ps 126:1}] les captifs de S., ns. étions com.
\item[\vref{Ps 132:13}] Yahweh a choisi S., il l'a préférée
\item[\vref{Ps 133:3}] les montagnes de S. ; car c'est là
\item[\vref{Es 2:3}] loi sortira de S., et la parole
\item[\vref{Es 52:1}] Réveille-toi, réveille-toi, S. ! Revêts-toi de ta
\item[\vref{Jn 12:15}] pas, fille de S. ; voici, ton Roi
\item[\vref{Hé 12:22}] la montagne de S., de la Cité
\item[\vref{1 Pi 2:6}] je mets en S. la principale pierre
\item[\vref{Ap 14:1}] la montagne de S., et il y
\end{listverse}

\ConcordanceEntry{Sisera}
\vspace{-2mm}
\begin{listverse}
\item[\vref{Jg 4:2}] son armée était S., qui habitait à
\item[\vref{Jg 5:26}] elle a frappé S. et lui a
\end{listverse}

\ConcordanceEntry{Sittim}
\vspace{-2mm}
\begin{listverse}
\item[\vref{No 25:1}] Israël demeurait à S. ; et le peuple
\item[\vref{Jos 3:1}] d'Israël partirent de S., ils vinrent jusqu'au
\end{listverse}

\ConcordanceEntry{Smyrne}
\vspace{-2mm}
\begin{listverse}
\item[\vref{Ap 1:11}] à Ephèse, à S., à Pergame, à
\item[\vref{Ap 2:8}] de l'église de S. : Voici ce que
\end{listverse}

\ConcordanceEntry{Sodome}
\vspace{-2mm}
\begin{listverse}
\item[\vref{Ge 13:10}] Yahweh ait détruit S. et Gomorrhe, c'était,
\item[\vref{Ge 18:20}] Le cri contre S. et Gomorrhe s'est
\item[\vref{Ge 19:1}] anges arrivèrent à S. ; et Lot était
\item[\vref{De 29:23}] la subversion de S., et de Gomorrhe,
\item[\vref{Es 1:9}] ns. serions com. S., ns. ressemblerions à
\item[\vref{Mt 11:23}] été faits ds S., elle subsisterait encore.
\item[\vref{Ap 11:8}] est appelée spirituellement S. et Egypte, où
\end{listverse}

\ConcordanceEntry{Sœur}
\vspace{-2mm}
\begin{listverse}
\item[\vref{Ge 12:13}] tu es ma s., afin que je
\item[\vref{Ge 20:2}] fem. : C'est ma s.. Et Abimélec, roi
\item[\vref{Ex 2:4}] Et la s. de cet enfant se tenait loin
\item[\vref{Lé 20:17}] hom. prend sa s., fille de son
\item[\vref{Ca 4:9}] le cœur, ma s., mon épouse, tu
\item[\vref{Mt 13:56}] Et ses s. ne sont-elles pas ttes parmi ns. ?
\item[\vref{1 Co 9:5}] avec ns. une s., une fem., com.
\item[\vref{1 Ti 5:2}] jeunes com. des s., en tte pureté.
\item[\vref{Ja 2:15}] frère ou une s. sont nus et
\end{listverse}

\ConcordanceEntry{Soif}
\vspace{-2mm}
\begin{listverse}
\item[\vref{Ex 17:3}] peuple dc eut s. en ce lieu-là,
\item[\vref{Jg 15:18}] Il eut extrêmement s., et invoqua Yahweh,
\item[\vref{Ps 42:3}] Mon âme a s. de Dieu, du
\item[\vref{Ps 63:2}] mon âme a s. de toi, mon
\item[\vref{Pr 25:21}] et s'il a s., donne-lui à boire
\item[\vref{Es 21:14}] ceux qui ont s. ; ils viennent au-dvt
\item[\vref{Es 48:21}] ils n'auront pas s. qnd il les
\item[\vref{Es 55:1}] ts qui avez s., venez aux eaux,
\item[\vref{Mt 5:6}] ont faim et s. de la justice,
\item[\vref{Mt 25:35}] manger ; j'ai eu s., et vs. m'avez
\item[\vref{Jn 4:13}] boit de cette eau-ci aura encore s. ;
\item[\vref{Jn 6:35}] qui croit en moi n'aura jamais s.
\item[\vref{Jn 7:37}] Si quelqu'un a s., qu'il vienne à
\item[\vref{Jn 19:28}] afin que l'Ecriture soit accomplie : J'ai s.
\item[\vref{Ro 12:20}] manger ; s'il a s., donne-lui à boire,
\item[\vref{Ap 21:6}] celui qui a s., je lui donnerai
\item[\vref{Ap 22:17}] celui qui a s. vienne ; que celui
\end{listverse}

\ConcordanceEntry{Soin}
\vspace{-2mm}
\begin{listverse}
\item[\vref{De 6:3}] et tu auras s. de les mettre
\item[\vref{So 2:1}] Examinez-vs., examinez-vs. avec s. ô nation non
\item[\vref{Mt 16:6}] dit : Gardez-vs. avec s. du levain des
\item[\vref{Ro 13:14}] et n'ayez pas s. de la chair
\item[\vref{1 Co 7:32}] pas marié a s. des choses qui
\item[\vref{1 Ti 5:8}] quelqu'un n'a pas s. des siens, et
\item[\vref{Tit 2:5}] pures, occupées aux s. domestiques, bonnes, soumises
\item[\vref{1 Pi 5:7}] peut vs. inquiéter, car il prend s. de vs.
\end{listverse}

\ConcordanceEntry{Soir}
\vspace{-2mm}
\begin{listverse}
\item[\vref{Ge 1:5}] Ainsi fut le s., ainsi fut le
\item[\vref{Ex 12:6}] l'assemblée d'Israël l'égorgera entre les deux s.
\item[\vref{Lé 11:24}] touchera lr. cadavre sera impur jusqu'au s.,
\item[\vref{De 28:67}] fera voir le s. ? Et le soir
\item[\vref{Ps 55:18}] Le s., le matin, et à midi je
\item[\vref{Da 8:14}] mille trois cents s. et matins ; puis
\item[\vref{Za 14:7}] au temps du s. il y aura
\item[\vref{Mt 16:2}] répondit : Quand le s. est venu, vs.
\item[\vref{Lu 24:29}] ns., car le s. approche et le
\item[\vref{Jn 20:19}] Le s. de ce jour, qui était le
\end{listverse}

\ConcordanceEntry{Soldat}
\vspace{-2mm}
\begin{listverse}
\item[\vref{Mt 8:9}] autre, j'ai des s. sous mes ordres ;
\item[\vref{Mt 28:12}] donnèrent une forte somme d'argent aux s.,
\item[\vref{Ac 12:4}] bandes de quatre s. chacune, avec l'intention
\item[\vref{Ac 27:32}] Alors les s. coupèrent les cordes
\item[\vref{Ac 28:16}] particulier avec un s. qui le gardait.
\item[\vref{2 Ti 2:3}] com. un bon s. de Jésus-Christ.
\end{listverse}

\ConcordanceEntry{Soleil}
\vspace{-2mm}
\begin{listverse}
\item[\vref{Ge 15:12}] arriva, com. le s. se couchait, qu'un
\item[\vref{Ge 32:31}] Le s. se levait lorsqu'il passa Peniel. Jacob
\item[\vref{Jos 10:12}] en présence d'Israël : S., arrête-toi sur Gabaon,
\item[\vref{Job 9:7}] Il parle au s., et le soleil
\item[\vref{Ps 84:12}] Dieu est un s. et un bouclier ;
\item[\vref{Ps 121:6}] le jour, le s. ne te frappera
\item[\vref{Ec 1:5}] Le s. aussi se lève et le soleil
\item[\vref{Ec 12:4}] avant que le s. et la lumière,
\item[\vref{Ca 1:6}] noire : Car le s. m'a regardée. Les
\item[\vref{Joë 2:31}] le s. se changera en ténèbres, et la
\item[\vref{Mal 4:2}] se lèvera le S. de justice, et
\item[\vref{Mt 5:45}] fait lever son s. sur les méchants
\item[\vref{Mt 13:43}] resplendiront com. le s. ds le Royaume
\item[\vref{Mt 17:2}] resplendit com. le s., et ses vêtements
\item[\vref{Mt 24:29}] de détresse, le s. s'obscurcira, la lune
\item[\vref{Mc 4:6}] mais, qnd le s. parut, elle fut
\item[\vref{Lu 1:78}] de laquelle le S. Levant ns. a
\item[\vref{Ac 2:20}] Le s. se changera en ténèbres, et la
\item[\vref{1 Co 15:41}] est l'éclat du s., autre l'éclat de
\item[\vref{Ep 4:26}] pas, que le s. ne se couche
\item[\vref{Ja 1:11}] Car com. le s. ardent n'est pas
\item[\vref{Ap 1:16}] était semblable au s. lorsqu'il brille ds
\item[\vref{Ap 9:2}] fournaise ; et le s. et l'air furent
\item[\vref{Ap 12:1}] fem. revêtue du s., la lune sous
\item[\vref{Ap 16:8}] coupe sur le s., et le pouvoir
\item[\vref{Ap 21:23}] pas besoin du s. ni de la
\end{listverse}

\ConcordanceEntry{Solitude}
\vspace{-2mm}
\begin{listverse}
\item[\vref{De 32:10}] des hurlements d'une s., il l'a entouré,
\item[\vref{Es 43:20}] fleuves ds la s., pour abreuver mon
\end{listverse}

\ConcordanceEntry{Sommeil}
\vspace{-2mm}
\begin{listverse}
\item[\vref{Ge 2:21}] tomber un profond s. sur Adam, qui
\item[\vref{Ge 15:12}] couchait, qu'un profond s. tomba sur Abram,
\item[\vref{1 S 26:12}] ts d'un profond s. ds lequel Yahweh
\item[\vref{Ps 13:4}] ne dorme du s. de la mort,
\item[\vref{Pr 3:24}] seras couché ton s. sera doux.
\item[\vref{Pr 6:9}] couché ? Quand te lèveras-tu de ton s. ?
\item[\vref{Pr 20:13}] N'aime point le s., de peur que
\item[\vref{Ec 5:11}] Le s. de celui qui travaille est doux,
\item[\vref{Lu 9:32}] étaient accablés de s. ; et qnd ils
\item[\vref{Jn 11:13}] pensaient qu'il parlait du repos du s.
\item[\vref{Ac 20:9}] entraîné par le s., il tomba du
\item[\vref{Ro 13:11}] ns. réveiller du s. ; car mntnt le
\end{listverse}

\ConcordanceEntry{Sonder}
\vspace{-2mm}
\begin{listverse}
\item[\vref{1 S 20:12}] Dieu d'Israël ! Je s. mon père demain,
\item[\vref{Job 5:9}] ne peut les s., et tant de
\item[\vref{Job 13:9}] plaisant qu'il vs. s. ? Vous jouerez-vs. de
\item[\vref{Ps 7:10}] juste, toi qui s. les cœurs et
\item[\vref{Ps 11:5}] Yahweh s. le juste et le méchant ; et
\item[\vref{Ps 17:3}] Tu as s. mon cœur, tu l'as visité de
\item[\vref{Ps 26:2}] S.-moi et éprouve-moi, Yahweh ! Fais passer
\item[\vref{Ps 139:1}] Yahweh, tu me s. et tu me
\item[\vref{Ps 145:3}] pas possible de s. sa grandeur.
\item[\vref{Jé 17:10}] suis Yahweh, qui s. le cœur, et
\item[\vref{Jn 5:39}] Vous s. les Ecritures, car vs. pensez avoir
\item[\vref{Ro 8:27}] et celui qui s. les cœurs connaît
\item[\vref{1 Co 2:10}] Esprit. Car l'Esprit s. ttes choses, mm
\item[\vref{1 Pi 1:11}] Ils voulaient s. l'époque et les
\item[\vref{Ap 2:23}] suis celui qui s. les reins et
\end{listverse}

\ConcordanceEntry{Songe}
\vspace{-2mm}
\begin{listverse}
\item[\vref{Ge 20:3}] nuit ds un s. à Abimélec, et
\item[\vref{Ge 28:12}] Il eut un s. ; et voici, une
\item[\vref{Ge 37:5}] Joseph eut un s. et il le
\item[\vref{Ge 40:8}] avons eu des s., et il n'y
\item[\vref{No 12:6}] vision, et je lui parlerai en s.
\item[\vref{Jg 7:13}] son compagnon un s.. Il lui disait :
\item[\vref{Job 33:15}] par des s., par des visions nocturnes, qnd les
\item[\vref{Ps 73:20}] sont com. un s. lorsqu'on s'est réveillé.
\item[\vref{Ec 5:2}] Car com. le s. vient de la
\item[\vref{Ec 5:6}] la multitude des s. il y a
\item[\vref{Jé 23:32}] qui prophétisent des s. faux, et qui
\item[\vref{Da 1:17}] ttes les visions et ts les s.
\item[\vref{Da 2:1}] Nebucadnetsar eut des s., et son esprit
\item[\vref{Joë 2:28}] vieillards auront des s., et vos jeunes
\item[\vref{Mt 1:20}] lui apparut en s., et lui dit :
\item[\vref{Mt 27:19}] souffert aujourd'hui en s. à cause de
\item[\vref{Ac 2:17}] visions, et vos vieillards auront des s.
\end{listverse}

\ConcordanceEntry{Sophonie}
\vspace{-2mm}
\begin{listverse}
\item[\vref{2 R 25:18}] premier prêtre, et S., le second prêtre,
\item[\vref{Jé 29:25}] de Jérus., à S., fils de Maaséja,
\item[\vref{So 1:1}] qui vint à S., fils de Cuschi,
\end{listverse}

\ConcordanceEntry{Sorcier}
\vspace{-2mm}
\begin{listverse}
\item[\vref{Ex 22:18}] Tu ne laisseras point vivre la s.
\item[\vref{Ap 21:8}] les fornicateurs, les s., les idolâtres et
\end{listverse}

\ConcordanceEntry{Sort}
\vspace{-2mm}
\begin{listverse}
\item[\vref{No 26:55}] partagé par le s. ; et qu'ils prennent
\item[\vref{Ab 1:11}] et jetaient le s. sur Jérus., toi
\item[\vref{Jon 1:7}] Venez, tirons au s. pour savoir qui
\item[\vref{Mt 27:35}] en tirant au s., afin que s'accomplisse
\item[\vref{Lu 1:8}] classe, il fut appelé par le s.,
\item[\vref{Ac 1:26}] les tirèrent au s., et le sort
\end{listverse}

\ConcordanceEntry{Sortir}
\vspace{-2mm}
\begin{listverse}
\item[\vref{Ge 8:7}] le corbeau, qui s., allant et revenant,
\item[\vref{Ge 8:18}] Noé dc s., et avec lui ses fils, sa
\item[\vref{Ex 3:10}] et tu feras s. mon peuple, les
\item[\vref{Ex 14:8}] fils d'Israël étaient s. à main levée.
\item[\vref{Job 1:12}] lui. Et Satan s. de dvt la
\item[\vref{Job 2:7}] Ainsi Satan s. de dvt Yahweh,
\item[\vref{Ps 19:6}] à un époux s. de sa chambre
\item[\vref{Ps 68:8}] Dieu ! Quand tu s. dvt ton peuple,
\item[\vref{Ec 5:14}] com. il est s. du ventre de
\item[\vref{Es 2:3}] car la loi s. de Sion, et
\item[\vref{Es 11:1}] Mais il s. un rameau du tronc d'Isaï, et
\item[\vref{Es 52:12}] Car vs. ne s. pas en hâte,
\item[\vref{Jé 1:5}] que tu sois s. de son sein,
\item[\vref{Jé 7:22}] les ai fait s. du pays d'Egypte.
\item[\vref{Mt 3:16}] été baptisé, il s. aussitôt hors de
\item[\vref{Mt 12:43}] l'esprit impur est s. d'un hom., il
\item[\vref{Mc 1:26}] Alors l'esprit impur s. de cet hom.,
\item[\vref{Mc 4:3}] Ecoutez ! Un semeur s. pour semer.
\item[\vref{Jn 4:30}] Ils s. dc de la ville, et vinrent
\item[\vref{Jn 5:28}] sépulcres entendront sa voix, et en s.
\item[\vref{Col 3:8}] déshonnêtes qui pourraient s. de votre bouche.
\item[\vref{1 Jn 2:19}] Ils sont s. du milieu de ns., mais ils
\item[\vref{Ap 20:8}] Et il s. pour séduire les nations qui sont
\end{listverse}

\ConcordanceEntry{Sosthène}
\vspace{-2mm}
\begin{listverse}
\item[\vref{Ac 18:17}] se saisirent de S., le chef de
\item[\vref{1 Co 1:1}] apôtre de Jésus-Christ, et le frère S.,
\end{listverse}

\ConcordanceEntry{Souci}
\vspace{-2mm}
\begin{listverse}
\item[\vref{Mc 4:19}] en qui les s. de ce monde,
\item[\vref{Lu 21:34}] et par les s. de cette vie ;
\item[\vref{2 Co 11:28}] jours, c'est le s. que j'ai de
\end{listverse}

\ConcordanceEntry{Souffle}
\vspace{-2mm}
\begin{listverse}
\item[\vref{Ge 1:30}] en soi un s. de vie, je
\item[\vref{Ge 2:7}] ses narines un s. de vie ; et
\item[\vref{Ps 18:16}] Yahweh ! par le s. du vent de
\item[\vref{Ec 8:8}] maître de son s. pour pouvoir le
\item[\vref{Es 2:22}] n'y a qu'un s. : Car quel cas
\item[\vref{Jn 3:8}] Le vent s. où il veut, et tu en
\item[\vref{Ac 2:2}] d'un vent qui s. avec violence, et
\item[\vref{2 Th 2:8}] détruira par le s. de sa bouche
\end{listverse}

\ConcordanceEntry{Souffrance}
\vspace{-2mm}
\begin{listverse}
\item[\vref{Ge 3:16}] J'augmenterai beaucoup la s. de tes grossesses ;
\item[\vref{Job 36:21}] l'iniquité, car la s. t'y dispose.
\item[\vref{Ps 69:27}] et racontent les s. de ceux que
\item[\vref{Ps 88:10}] consument ds la s. ; Yahweh ! Je crie
\item[\vref{Es 53:10}] mis ds la s.. Après avoir mis
\item[\vref{Jn 16:21}] plus de la s., à cause de
\item[\vref{Ro 8:18}] j'estime que les s. du temps présent
\item[\vref{2 Co 1:5}] Car com. les s. de Christ abondent
\item[\vref{Ph 3:10}] communion de ses s., en devenant conforme
\item[\vref{Col 1:24}] mntnt ds mes s. pour vs. ; et
\item[\vref{Hé 10:32}] avez soutenu un grand combat de s.,
\item[\vref{Ja 5:13}] est-il ds la s. ? Qu'il prie. Quelqu'un
\item[\vref{1 Pi 4:13}] vs. participez aux s. de Christ, afin
\item[\vref{1 Pi 5:9}] que les mêmes s. s'accomplissent ds la
\end{listverse}

\ConcordanceEntry{Souffrir}
\vspace{-2mm}
\begin{listverse}
\item[\vref{Pr 11:15}] un étranger en s., et celui qui
\item[\vref{Mt 16:21}] à Jérus., qu'il s. beaucoup de la
\item[\vref{Mt 17:12}] de l'hom. doit s. aussi de lr.
\item[\vref{Lu 22:15}] de Pâque avec vs., avant de s. ;
\item[\vref{Lu 24:26}] que le Christ s. et qu'il entre
\item[\vref{Ac 3:18}] ses prophètes, que le Christ devait s.
\item[\vref{Ac 9:16}] il aura à s. pour mon Nom.
\item[\vref{Ro 8:17}] Christ, si ns. s. avec lui, afin
\item[\vref{1 Co 12:26}] l'un des membres s. qq chose, ts
\item[\vref{Ga 3:4}] Avez-vs. tant s. en vain ? Si
\item[\vref{Ph 1:29}] mais aussi de s. pour lui,
\item[\vref{1 Th 3:4}] ns. aurions à s. des afflictions, com.
\item[\vref{Hé 9:26}] fallu qu'il ait s. plusieurs fois depuis
\item[\vref{1 Pi 2:19}] des afflictions en s. injustement.
\item[\vref{1 Pi 3:18}] Christ aussi a s. une fois pour
\item[\vref{1 Pi 4:19}] ceux dc qui s. selon la volonté
\item[\vref{1 Pi 5:10}] que vs. aurez s. un peu de
\item[\vref{Ap 2:10}] tu as à s.. Voici, il arrivera
\end{listverse}

\ConcordanceEntry{Soufre}
\vspace{-2mm}
\begin{listverse}
\item[\vref{Ge 19:24}] sur Gomorrhe, du s. et du feu,
\item[\vref{Ps 11:6}] feu et du s. ; un vent brûlant,
\item[\vref{Ez 38:22}] de grêle, du feu et du s.
\item[\vref{Lu 17:29}] feu et de s. tomba du ciel,
\item[\vref{Ap 9:18}] et par le s. qui sortaient de
\item[\vref{Ap 14:10}] feu et le s. dvt les saints
\item[\vref{Ap 19:20}] l'étang ardent de feu et de s.
\end{listverse}

\ConcordanceEntry{Souiller}
\vspace{-2mm}
\begin{listverse}
\item[\vref{Ge 49:4}] et tu as s. mon lit en
\item[\vref{Lé 18:20}] prochain pour te s. avec elle.
\item[\vref{Lé 18:23}] bête pour te s. avec elle ; et
\item[\vref{Ps 106:39}] Ils se s. par leurs œuvres, et se prostituèrent
\item[\vref{Ec 9:2}] celui qui est s., à celui qui
\item[\vref{Es 59:3}] vos mains sont s. de sang, et
\item[\vref{Es 64:5}] com. une chose s., et tte notre
\item[\vref{Ez 4:14}] n'a point été s., et je n'ai
\item[\vref{Ez 5:11}] que tu as s. mon lieu saint
\item[\vref{Da 1:8}] ne pas se s. par la portion
\item[\vref{Mt 15:11}] la bouche qui s. l'hom. ; mais ce
\item[\vref{Jn 18:28}] ne pas se s., et de pouvoir
\item[\vref{Ac 10:14}] rien mangé de s. ni d'impur.
\item[\vref{Ro 14:14}] que rien n'est s. par soi-mm, mais
\item[\vref{1 Co 8:7}] et lr. conscience, étant faible, est s.
\item[\vref{Tit 1:15}] lr. entendement et lr. conscience sont s.
\item[\vref{Hé 12:15}] plusieurs n'en soient s. par elles ;
\item[\vref{Ja 3:6}] parmi nos membres, s. tt le corps,
\item[\vref{1 Pi 1:4}] peut ni se s., ni se flétrir,
\item[\vref{Jud 1:8}] ds leurs rêveries, s. lr. chair, méprisent
\item[\vref{Ap 21:27}] elle rien de s., ni personne qui
\item[\vref{Ap 22:11}] celui qui est s. se souille encore ;
\end{listverse}

\ConcordanceEntry{Souillure}
\vspace{-2mm}
\begin{listverse}
\item[\vref{Ez 36:25}] de ttes vos s. et de ttes
\item[\vref{Ac 15:20}] de s'abstenir des s. des idoles et
\item[\vref{2 Co 7:1}] purifions-ns. de tte s. de la chair
\item[\vref{Hé 13:4}] le lit sans s. ; mais Dieu jugera
\item[\vref{Ja 1:21}] pourquoi, rejetant tte s. et tt résidu
\item[\vref{Ja 1:27}] conserver pur des s. de ce monde.
\item[\vref{1 Pi 3:21}] la purification des s. de la chair,
\item[\vref{2 Pi 2:20}] s'être retirés des s. du monde par
\end{listverse}

\ConcordanceEntry{Soulier}
\vspace{-2mm}
\begin{listverse}
\item[\vref{Ex 3:5}] d'ici ; déchausse tes s. de tes pieds,
\item[\vref{Ex 12:11}] vs. aurez vos s. à vos pieds,
\item[\vref{De 25:9}] lui ôtera son s. du pied, et
\item[\vref{Jos 5:15}] Josué : Délie tes s. de tes pieds ;
\item[\vref{Ru 4:7}] l'hom. ôtait son s. et le donnait
\item[\vref{Mt 3:11}] de porter ses s.. Celui-là vs. baptisera
\item[\vref{Mt 10:10}] deux tuniques, ni s., ni bâton ; car
\item[\vref{Lu 15:22}] doigt, et des s. aux pieds.
\item[\vref{Jn 1:27}] de délier la courroie de ses s.
\end{listverse}

\ConcordanceEntry{Soumettre}
\vspace{-2mm}
\begin{listverse}
\item[\vref{Ps 105:22}] pour s. les princes à ses désirs, et
\item[\vref{Lu 2:51}] il lr. était s.. Et sa mère
\item[\vref{Lu 10:17}] mêmes ns. sont s. en ton Nom.
\item[\vref{Ro 8:7}] elle ne se s. pas à la
\item[\vref{Ro 13:1}] tte personne soit s. aux autorités supérieures ;
\item[\vref{1 Co 14:32}] des prophètes sont s. aux prophètes.
\item[\vref{1 Co 16:16}] prie de vs. s. à de tels
\item[\vref{2 Co 9:13}] que vs. vs. s. à l'Evangile de
\item[\vref{Ep 5:21}] s.-vs. les uns aux autres ds
\item[\vref{Tit 1:10}] veulent pas se s., vains discoureurs, et
\item[\vref{Tit 3:1}] Rappelle-lr. d'être s. aux magistrats et
\item[\vref{Hé 12:9}] pas beaucoup plus s. au Père des
\item[\vref{Ja 4:7}] S.-vs. dc à Dieu ; résistez au
\end{listverse}

\ConcordanceEntry{Souper}
\vspace{-2mm}
\begin{listverse}
\item[\vref{Lu 14:12}] dîner ou un s., n'invite pas tes
\item[\vref{1 Co 11:21}] par prendre son s. particulier, et l'un
\item[\vref{Ap 3:20}] lui, et je s. avec lui, et
\end{listverse}

\ConcordanceEntry{Soupir}
\vspace{-2mm}
\begin{listverse}
\item[\vref{Ps 38:10}] toi, et mon s. ne t'est point
\item[\vref{Ps 90:9}] nos années se consument ds un s.
\item[\vref{Ro 8:26}] ns. par des s. inexprimables ;
\end{listverse}

\ConcordanceEntry{Soupirer}
\vspace{-2mm}
\begin{listverse}
\item[\vref{Ex 2:23}] les enfants d'Israël s. à cause de
\item[\vref{1 S 7:2}] la maison d'Israël s. après Yahweh.
\item[\vref{Ps 42:2}] Comme une biche s. après des courants
\item[\vref{Ps 63:2}] toi, mon corps s. après toi sur
\item[\vref{Ps 84:3}] Mon âme s. et languit après les parvis de
\item[\vref{Ec 1:5}] se couche ; il s. après le lieu
\item[\vref{Ez 9:4}] gémissent et qui s. à cause de
\item[\vref{Mc 8:12}] Alors, Jésus s. profondément en son
\item[\vref{Ro 8:22}] tte la création s. et souffre les
\end{listverse}

\ConcordanceEntry{Source}
\vspace{-2mm}
\begin{listverse}
\item[\vref{Ge 7:11}] jour-là ttes les s. du grand abîme
\item[\vref{Jos 15:19}] donne-moi aussi des s. d'eau. Et il
\item[\vref{Job 38:16}] As-tu pénétré jusqu'aux s. de la mer ?
\item[\vref{Ps 36:10}] Car la s. de la vie est auprès de
\item[\vref{Ps 87:7}] s'écrient : Toutes mes s. sont en toi.
\item[\vref{Ps 107:33}] désert, et les s. d'eaux en sécheresse ;
\item[\vref{Pr 4:23}] lui procèdent les s. de la vie.
\item[\vref{Pr 5:16}] que tes s. se répandent dehors, et les ruisseaux
\item[\vref{Pr 14:27}] Yahweh est une s. de vie pour
\item[\vref{Ec 12:8}] rompe sur la s., que la roue
\item[\vref{Ca 4:15}] des jardins ! Ô s. d'eaux vives ! Ruisseaux
\item[\vref{Es 12:3}] avec joie aux s. du salut,
\item[\vref{Es 58:11}] et com. une s. dont les eaux
\item[\vref{Jé 2:13}] qui suis la s. d'eaux vives, pour
\item[\vref{Jé 15:18}] moi com. une s. trompeuse, com. des
\item[\vref{Za 13:1}] y aura une s. ouverte en faveur
\item[\vref{Ap 7:17}] les conduira aux s. des eaux de
\item[\vref{Ap 21:6}] donnerai de la s. d'eau vive gratuitement.
\end{listverse}

\ConcordanceEntry{Sourd}
\vspace{-2mm}
\begin{listverse}
\item[\vref{Ex 4:11}] rend muet ou s., voyant ou aveugle ?
\item[\vref{Lé 19:14}] maudiras point le s., et tu ne
\item[\vref{Ps 28:1}] te rends point s. envers moi, de
\item[\vref{Ps 39:13}] Ne sois point s. à mes larmes !
\item[\vref{Es 29:18}] ce jour-là, les s. entendront les paroles
\item[\vref{Es 35:5}] les oreilles des s. seront débouchées.
\item[\vref{Es 42:18}] S., écoutez ! Et vs. aveugles, regardez et
\item[\vref{Mt 11:5}] sont purifiés, les s. entendent, les morts
\item[\vref{Mc 7:32}] lui amena un s. qui avait la
\item[\vref{Mc 7:37}] fait entendre les s., et parler les
\item[\vref{Mc 9:25}] Esprit muet et s., je te l'ordonne,
\item[\vref{Lu 7:22}] sont purifiés, les s. entendent, les morts
\end{listverse}

\ConcordanceEntry{Soutenir}
\vspace{-2mm}
\begin{listverse}
\item[\vref{Ex 17:12}] Aaron et Hur s. ses mains, l'un
\item[\vref{2 Ch 16:9}] la terre, pour s. ceux dont le
\item[\vref{Ps 9:5}] Car tu s. mon droit et ma cause, tu
\item[\vref{Ps 37:17}] brisés, mais Yahweh s. les justes.
\item[\vref{Ps 119:116}] S.-moi suivant ta parole, et je
\item[\vref{Ps 119:154}] S. ma cause et rachète-moi ; fais-moi revivre
\item[\vref{Ps 145:14}] [Samech.] Yahweh s. ts ceux qui
\item[\vref{Es 9:6}] l'affermir et le s. par le droit
\item[\vref{Es 41:10}] mm je te s. par la droite
\item[\vref{Es 50:4}] que je sache s. par la parole
\item[\vref{Es 63:5}] personne pour me s. ; mais mon bras
\item[\vref{Mal 3:2}] Mais qui pourra s. le jour de
\item[\vref{Ac 20:35}] ainsi qu'il faut s. les faibles, et
\item[\vref{Ac 24:13}] ils ne sauraient s. les choses dont
\item[\vref{Ph 1:30}] en s. le mm combat ds lequel vs.
\item[\vref{1 Th 3:1}] ne pouvant plus s. la privation de
\item[\vref{1 Th 3:5}] ne pouvant plus s. cette inquiétude, j'ai
\item[\vref{Hé 1:3}] son être, et s. ttes choses par
\end{listverse}

\ConcordanceEntry{Soutien}
\vspace{-2mm}
\begin{listverse}
\item[\vref{Ru 4:15}] et sera le s. de ta vieillesse ;
\item[\vref{Ps 27:1}] Yahweh est le s. de ma vie :
\end{listverse}

\ConcordanceEntry{Souvenir (se)}
\vspace{-2mm}
\begin{listverse}
\item[\vref{Ge 8:1}] Or Dieu se s. de Noé, de
\item[\vref{Ex 2:24}] et Dieu se s. de l'alliance qu'il
\item[\vref{Ex 13:3}] dit au peuple : S.-vs. de ce
\item[\vref{Ps 8:5}] que tu te s. de lui ? Et
\item[\vref{Ps 78:42}] Ils ne se s. point de sa
\item[\vref{Lu 1:54}] et il s'est s. de sa miséricorde,
\item[\vref{Lu 23:42}] à Jésus : Seign., s.-toi de moi
\item[\vref{Ac 10:4}] dvt Dieu, et il s'en est s.
\item[\vref{Ac 20:31}] pourquoi veillez, vs. s. que durant l'espace
\item[\vref{Ga 2:10}] seulement de ns. s. des pauvres, ce
\item[\vref{2 Ti 1:5}] et me s. de la foi sincère qui est
\item[\vref{2 Pi 1:15}] puissiez toujours vs. s. de ces choses.
\end{listverse}

\ConcordanceEntry{Souvenir (le)}
\vspace{-2mm}
\begin{listverse}
\item[\vref{Ge 41:9}] rappellerai aujourd'hui le s. de mes fautes.
\item[\vref{Ex 3:15}] c'est ici le s. que vs. aurez
\item[\vref{Ex 12:14}] vs. conserverez le s. de ce jour,
\item[\vref{No 16:40}] C'est un s. pour les enfants d'Israël, afin qu'aucun
\item[\vref{Jos 4:7}] à jamais un s. pour les enfants
\item[\vref{1 R 17:18}] pour rappeler le s. de mon iniquité,
\item[\vref{Ps 145:7}] Ils proclameront le s. de ton immense
\item[\vref{Ec 1:11}] ceux qui viendront n'en auront aucun s.
\item[\vref{Mal 3:16}] un livre de s. fut écrit dvt
\item[\vref{Hé 10:3}] Or le s. des péchés est réitéré ds ces
\end{listverse}

\ConcordanceEntry{Souverain}
\vspace{-2mm}
\begin{listverse}
\item[\vref{1 Ch 29:11}] tu t'élèves en s. au-dessus de ttes
\item[\vref{Ps 77:11}] qui m'affaiblit ; mais la droite du S. change.
\item[\vref{Pr 9:1}] La S. Sagesse a bâti sa maison, elle
\item[\vref{Hé 4:14}] ns. avons un S. Grand-Prêtre, Jésus, le
\item[\vref{1 Pi 5:4}] Et qnd le s. Pasteur apparaîtra, vs.
\end{listverse}

\ConcordanceEntry{Spectacle}
\vspace{-2mm}
\begin{listverse}
\item[\vref{Ez 28:17}] te donne en s. aux rois, afin
\item[\vref{1 Co 4:9}] avons été en s. au monde, aux
\item[\vref{Col 2:15}] exposées publiquement en s., en triomphant d'elles
\item[\vref{Hé 12:21}] Et ce s. était si terrible que Moïse dit :
\end{listverse}

\ConcordanceEntry{Spirituel}
\vspace{-2mm}
\begin{listverse}
\item[\vref{Ro 1:11}] de qq don s., afin que vs.
\item[\vref{1 Co 2:15}] Mais l'hom. s. discerne ttes choses
\item[\vref{1 Co 10:4}] le mm breuvage s., car ils buvaient
\item[\vref{1 Co 12:1}] concerne les dons s., je ne veux
\item[\vref{1 Co 14:1}] ardeur les dons s., mais surtout celui
\item[\vref{1 Co 15:44}] il ressuscitera corps s.. S'il y a
\item[\vref{1 Co 15:46}] ce qui est s. n'est pas le
\item[\vref{Ga 6:1}] vs. qui êtes s., redressez-le avec un
\item[\vref{Ep 5:19}] et des cantiques s., chantant et psalmodiant
\item[\vref{Col 3:16}] et des cantiques s., chantant ds votre
\item[\vref{1 Pi 2:2}] nouveau-nés, le lait s. et pur, afin
\item[\vref{1 Pi 2:5}] d'offrir des sacrifices s., agréables à Dieu
\end{listverse}

\ConcordanceEntry{Splendeur}
\vspace{-2mm}
\begin{listverse}
\item[\vref{2 S 22:13}] embrasés de la s. qui le précédait.
\item[\vref{Ps 18:13}] De la s. qui le précédait, s'échappaient les nuées,
\item[\vref{Ps 80:2}] entre les chérubins, fais briller ta s. !
\item[\vref{Ps 145:5}] Je dirai la s. glorieuse de ta
\item[\vref{Ps 145:12}] puissance, et la s. glorieuse de ton
\item[\vref{Da 12:3}] brilleront com. la s. du ciel, et
\item[\vref{Hé 1:3}] qui étant la s. de sa gloire,
\end{listverse}

\ConcordanceEntry{Stade}
\vspace{-2mm}
\begin{listverse}
\item[\vref{Lu 24:13}] Emmaüs, éloigné de Jérus. de soixante s.
\item[\vref{1 Co 9:24}] courent ds le s., courent ts, mais
\end{listverse}

\ConcordanceEntry{Statue}
\vspace{-2mm}
\begin{listverse}
\item[\vref{Ge 19:26}] en arrière, et elle devint une s. de sel.
\item[\vref{Da 2:31}] voyais une grande s. ; cette grande statue,
\item[\vref{Da 3:1}] Nebucadnetsar fit une s. d'or, dont la
\end{listverse}

\ConcordanceEntry{Stérile}
\vspace{-2mm}
\begin{listverse}
\item[\vref{Ge 11:30}] Et Saraï était s. et n'avait point
\item[\vref{Ge 29:31}] sa matrice, tandis que Rachel était s.
\item[\vref{De 7:14}] tes bêtes, ni mâle ni femelle s.
\item[\vref{Jg 13:2}] Sa fem. était s., et n'enfantait pas.
\item[\vref{1 S 2:5}] l'être ; mm la s. en a enfanté
\item[\vref{Ps 113:9}] à la fem. s., il en fait
\item[\vref{Lu 1:7}] parce qu'Elisabeth était s. et qu'ils étaient
\item[\vref{Ga 4:27}] est écrit : Réjouis-toi, s., toi qui n'enfantes
\item[\vref{2 Pi 1:8}] pas oisifs ni s. pour la connaissance
\end{listverse}

\ConcordanceEntry{Stupide}
\vspace{-2mm}
\begin{listverse}
\item[\vref{Ps 49:11}] l'insensé et le s. périssent également, et
\item[\vref{Ps 73:22}] j'étais alors s., et je n'avais
\item[\vref{Ps 92:7}] L'hom. s. n'y connaît rien, et le fou
\item[\vref{Pr 1:32}] Car l'apostasie des s. les tue, et
\item[\vref{Pr 9:4}] celui qui est s., entre ici ! Et
\item[\vref{Pr 9:13}] folle est bruyante, s. et elle ne
\item[\vref{Pr 12:1}] qui hait la réprimande est un s.
\item[\vref{Pr 19:25}] le moqueur, le s. en prend garde ;
\item[\vref{Es 19:11}] forment un conseil s.. Comment osez-vs. dire
\item[\vref{Jé 10:14}] Tout hom. devient s. par sa connaissance,
\item[\vref{Jé 51:17}] Tout hom. devient s. par sa connaissance,
\end{listverse}

\ConcordanceEntry{Subsister}
\vspace{-2mm}
\begin{listverse}
\item[\vref{1 S 6:20}] dirent : Qui pourrait s. en présence de
\item[\vref{Ps 5:6}] Les orgueilleux ne s. pas dvt tes
\item[\vref{Ps 19:10}] est pure, elle s. à toujours ; les
\item[\vref{Ps 102:27}] périront, mais tu s. ; ils s'useront ts
\item[\vref{Ps 119:89}] Yahweh ! ta parole s. à toujours ds
\item[\vref{Ps 146:10}] Sion ! ton Dieu s. d'âge en âge !
\item[\vref{Ec 3:14}] que Dieu fait s. à toujours, il
\item[\vref{Mt 12:26}] contre lui-mm ; comment dc son royaume s.-t-il ?
\item[\vref{Lu 11:18}] comment son royaume s.-t-il ? Car vs.
\item[\vref{Col 1:17}] avant ttes choses, et ttes choses s. par lui.
\item[\vref{Ap 6:17}] colère est venu, et qui peut s. ?
\end{listverse}

\ConcordanceEntry{Succès}
\vspace{-2mm}
\begin{listverse}
\item[\vref{Jos 1:8}] tu auras du s. ds tes entreprises,
\item[\vref{Esd 6:14}] Juifs bâtirent avec s., selon les prophéties
\item[\vref{Né 1:11}] donne aujourd'hui du s. à ton serviteur,
\item[\vref{Né 2:20}] ns. donnera le s. ! Nous, qui sommes
\item[\vref{Pr 8:14}] conseil et le s. ; je suis l'intelligence,
\end{listverse}

\ConcordanceEntry{Successeur}
\vspace{-2mm}
\begin{listverse}
\item[\vref{1 R 2:4}] manqueras jamais de s. sur le trône
\item[\vref{2 Ch 7:18}] manquera point de s. qui règne en
\end{listverse}

\ConcordanceEntry{Succomber}
\vspace{-2mm}
\begin{listverse}
\item[\vref{2 S 1:19}] L'élite d'Israël a s. sur tes collines !
\item[\vref{Hé 12:3}] que vs. ne s. pas, en perdant
\end{listverse}

\ConcordanceEntry{Succoth}
\vspace{-2mm}
\begin{listverse}
\item[\vref{Ge 33:17}] Jacob partit pour S.. Il bâtit une
\item[\vref{Ex 12:37}] Ramsès, vinrent à S., environ six cent
\item[\vref{Ex 13:20}] ils partirent de S., et campèrent à
\item[\vref{Jg 8:5}] aux gens de S. : Donnez, je vs.
\end{listverse}

\ConcordanceEntry{Sueur}
\vspace{-2mm}
\begin{listverse}
\item[\vref{Ge 3:19}] pain à la s. de ton visage,
\item[\vref{Lu 22:44}] instamment, et sa s. devint com. des
\end{listverse}

\ConcordanceEntry{Suivre}
\vspace{-2mm}
\begin{listverse}
\item[\vref{Ge 24:5}] voudra-t-elle pas me s. ds ce pays ;
\item[\vref{Ex 23:2}] Tu ne s. point la multitude pour faire le
\item[\vref{De 16:20}] Tu s. fermement la justice, afin que tu
\item[\vref{Ez 36:27}] sorte que vs. s. mes ordonnances, et
\item[\vref{Mt 4:19}] il lr. dit : S.-moi et je
\item[\vref{Mt 8:19}] Maître, je te s. partout où tu
\item[\vref{Mt 19:27}] et ns. t'avons s. ; que ns. en
\item[\vref{Mt 26:58}] Pierre le s. de loin jusqu'à
\item[\vref{Mc 2:15}] là beaucoup de gens qui l'avaient s.
\item[\vref{Mc 10:28}] avons tt quitté et ns. t'avons s.
\item[\vref{Mc 10:53}] la vue, et s. Jésus ds le
\item[\vref{Jn 10:4}] les brebis le s., parce qu'elles connaissent
\item[\vref{Jn 13:36}] peux pas me s. mntnt, mais tu
\item[\vref{1 Co 10:4}] spirituel qui les s., et ce rocher
\item[\vref{Ga 6:16}] ts ceux qui s. cette règle, et
\item[\vref{Ep 2:2}] vs. marchiez autrefois, s. le train de
\item[\vref{1 Pi 1:11}] et la gloire dont elles seraient s.
\item[\vref{1 Pi 2:21}] afin que vs. s. ses traces,
\item[\vref{Ap 14:4}] sont ceux qui s. l'Agneau partout où
\item[\vref{Ap 14:13}] leurs travaux, car leurs œuvres les s.
\end{listverse}

\ConcordanceEntry{Sunamite, Sulamite}
\vspace{-2mm}
\begin{listverse}
\item[\vref{1 R 1:3}] trouva Abischag, la S., que l'on amena
\item[\vref{2 R 4:12}] serviteur : Appelle cette S.. Guéhazi l'appela, et
\end{listverse}

\ConcordanceEntry{Supérieur}
\vspace{-2mm}
\begin{listverse}
\item[\vref{Da 5:12}] Beltschatsar, un esprit s., de la connaissance
\item[\vref{Ro 13:1}] soumise aux autorités s. ; car il n'y
\item[\vref{Hé 1:4}] Etant devenu d'autant s. aux anges, qu'il
\item[\vref{Hé 8:6}] un service d'autant s. qu'il est le
\item[\vref{2 Pi 2:11}] anges qui sont s. en force et
\end{listverse}

\ConcordanceEntry{Supplication}
\vspace{-2mm}
\begin{listverse}
\item[\vref{1 R 8:45}] prières et leurs s., et fais lr.
\item[\vref{Né 1:11}] et à la s. de tes serviteurs
\item[\vref{Ps 86:6}] attentif à la voix de mes s. !
\item[\vref{Ps 119:170}] Que ma s. vienne dvt toi, délivre-moi selon ta
\item[\vref{Da 9:3}] prière et des s., avec le jeûne,
\item[\vref{Za 12:10}] grâce et de s., et ils regarderont
\item[\vref{Ac 1:14}] et ds la s. avec les femmes,
\item[\vref{Ep 6:18}] prières et de s., veillez à cela
\item[\vref{Ph 4:6}] prières et des s., avec des actions
\item[\vref{1 Ti 2:1}] des prières, des s., et des actions
\end{listverse}

\ConcordanceEntry{Supporter}
\vspace{-2mm}
\begin{listverse}
\item[\vref{Né 9:30}] Tu les s. patiemment plusieurs années, et tu les
\item[\vref{Es 1:13}] ne puis plus s. votre méchanceté ni
\item[\vref{Es 1:14}] moi, je suis las de les s.
\item[\vref{Joë 2:11}] grand et terrible. Qui pourra le s. ?
\item[\vref{Am 7:10}] pays ne pourrait s. ttes ses paroles.
\item[\vref{Mi 7:9}] Je s. la colère de Yahweh, car j'ai
\item[\vref{Mt 17:17}] Jusqu'à qnd vs. s.-je ? Amenez-le-moi ici.
\item[\vref{1 Co 10:13}] sortir, afin que vs. puissiez la s.
\item[\vref{1 Co 13:7}] croit tt, elle espère tt, elle s. tt.
\item[\vref{Ep 4:2}] avec patience, vs. s. les uns les
\item[\vref{2 Ti 4:3}] les hommes ne s. pas la saine
\item[\vref{Hé 12:3}] celui qui a s. contre sa personne
\item[\vref{Hé 12:20}] ne pouvaient pas s. ce qui était
\item[\vref{Hé 13:22}] vs. prie de s. la parole d'exhortation,
\item[\vref{1 Pi 2:20}] y a-t-il à s. patiemment des coups
\item[\vref{Ap 2:2}] ne peux pas s. les méchants, et
\end{listverse}

\ConcordanceEntry{Sûreté}
\vspace{-2mm}
\begin{listverse}
\item[\vref{Ps 119:117}] je serai en s. ; et j'aurai continuellement
\item[\vref{Lu 11:21}] les biens qu'il a sont en s.
\item[\vref{Ac 5:23}] fermée avec tte s., et les gardes
\item[\vref{1 Th 5:3}] paix et en s.. Alors une destruction
\end{listverse}

\ConcordanceEntry{Surpasser}
\vspace{-2mm}
\begin{listverse}
\item[\vref{1 R 4:30}] sagesse de Salomon s. la sagesse de
\item[\vref{Pr 31:10}] Car son prix s. de beaucoup les
\item[\vref{Pr 31:29}] agissent vertueusement, mais toi, tu les s. ttes.
\item[\vref{Ec 1:16}] grand et j'ai s. en sagesse ts
\item[\vref{Da 6:3}] Mais Daniel s. les autres chefs
\item[\vref{Mt 5:20}] votre justice ne s. celle des scribes
\item[\vref{Ep 3:19}] de Christ qui s. tte connaissance, afin
\item[\vref{Ph 4:7}] de Dieu, qui s. tte intelligence, gardera
\item[\vref{Ap 2:19}] tes dernières œuvres s. les premières.
\end{listverse}

\ConcordanceEntry{Surprendre}
\vspace{-2mm}
\begin{listverse}
\item[\vref{Mt 22:15}] moyens de le s. par ses propres
\item[\vref{1 Co 3:19}] est écrit : Il s. les sages ds
\item[\vref{1 Th 5:3}] destruction soudaine les s., com. les douleurs
\end{listverse}

\ConcordanceEntry{Suse}
\vspace{-2mm}
\begin{listverse}
\item[\vref{Né 1:1}] com. j'étais à S., la capitale,
\item[\vref{Da 8:2}] que j'étais à S., la capitale, ds
\end{listverse}

\ConcordanceEntry{Synagogue}
\vspace{-2mm}
\begin{listverse}
\item[\vref{Mt 4:23}] enseignant ds leurs s., prêchant l'Evangile du
\item[\vref{Mt 6:5}] debout ds les s. et aux coins
\item[\vref{Mt 10:17}] vs. battront de verges ds leurs s.
\item[\vref{Lu 7:5}] c'est lui qui a bâti notre s.
\item[\vref{Jn 9:22}] Christ, il serait exclu de la s.
\item[\vref{Jn 12:42}] de peur d'être exclus de la s.
\item[\vref{Jn 16:2}] vs. chasseront des s. ; mm l'heure vient
\item[\vref{Jn 18:20}] enseigné ds la s. et ds le
\item[\vref{Ac 15:21}] les jours de sabbat ds les s.
\item[\vref{Ac 17:1}] Thessalonique, où les Juifs avaient une s.
\item[\vref{Ac 18:4}] discourait ds la s. chaque sabbat, et
\item[\vref{Ap 2:9}] qui sont la s. de Satan.
\item[\vref{Ap 3:9}] ceux de la s. de Satan qui
\end{listverse}

\ConcordanceEntry{Syrie, Syriens}
\vspace{-2mm}
\begin{listverse}
\item[\vref{2 S 8:5}] Les S. de Damas vinrent au secours d'Hadadézer,
\item[\vref{1 R 20:1}] Ben-Hadad, roi de S. rassembla tte son
\item[\vref{2 R 6:8}] Le roi de S. était en guerre
\item[\vref{Es 7:2}] de David : La S. s'est reposée sur
\item[\vref{Am 1:5}] le peuple de S. sera mené captif
\item[\vref{Ac 18:18}] s'embarqua pour la S., avec Priscille et
\item[\vref{Ga 1:21}] les pays de S. et de Cilicie.
\end{listverse}

\ConcordanceEntry{Tabeéra}
\vspace{-2mm}
\begin{listverse}
\item[\vref{No 11:3}] nomma ce lieu-là T., parce que le
\item[\vref{De 9:22}] de Yahweh à T., à Massa, et
\end{listverse}

\ConcordanceEntry{Tabernacle}
\vspace{-2mm}
\begin{listverse}
\item[\vref{Ex 15:2}] lui dresserai un t., c'est le Dieu
\item[\vref{Ex 26:1}] feras aussi le t. de dix tapis
\item[\vref{Lé 23:34}] fête solennelle des t. pendant sept jours,
\item[\vref{1 Ch 23:26}] à porter le t. ni ts les
\item[\vref{Ps 27:5}] cachera ds son t. au jour du
\item[\vref{Am 9:11}] je relèverai le t. de David qui
\item[\vref{Lu 16:9}] reçoivent ds les t. éternels.
\item[\vref{Jn 7:2}] la fête des t., était proche.
\item[\vref{Ac 7:44}] au désert le t. du témoignage, com.
\item[\vref{Ac 7:45}] avaient reçu ce t., ils le portèrent
\item[\vref{Hé 9:2}] construit un premier t., appelé le lieu
\item[\vref{Hé 9:11}] a traversé un t. plus excellent et
\item[\vref{Hé 9:21}] sang sur le t. et sur ts
\item[\vref{Ap 13:6}] Nom et son t., et ceux qui
\item[\vref{Ap 15:5}] le temple du t. du témoignage fut
\item[\vref{Ap 21:3}] disait : Voici le t. de Dieu avec
\end{listverse}

\ConcordanceEntry{Tabitha}
\vspace{-2mm}
\begin{listverse}
\item[\vref{Ac 9:36}] fem. disciple, appelée T., qui signifie en
\end{listverse}

\ConcordanceEntry{Table}
\vspace{-2mm}
\begin{listverse}
\item[\vref{Ex 24:12}] te donnerai des t. de pierre, la
\item[\vref{Ex 25:23}] feras aussi une t. de bois d'acacia.
\item[\vref{Ex 26:35}] tu mettras la t. au dehors de
\item[\vref{Ex 32:15}] main les deux t. du témoignage, et
\item[\vref{Ex 32:16}] Et les t. étaient l'ouvrage de Dieu, et l'écriture
\item[\vref{De 10:5}] je mis les t. ds l'arche que
\item[\vref{Jg 1:7}] pain sous ma t. ; Dieu me rend
\item[\vref{1 S 20:29}] qu'il n'est pas venu à la t. du roi.
\item[\vref{2 S 9:7}] mangeras toujours du pain à ma t.
\item[\vref{1 R 7:48}] d'or, et les t. d'or, sur lesquelles
\item[\vref{1 R 18:19}] mangent à la t. de Jézabel.
\item[\vref{Né 5:17}] aussi à ma t. les Juifs et
\item[\vref{Ps 23:5}] dvt moi une t., en face de
\item[\vref{Ps 78:19}] pourrait-il dresser une t. ds ce désert ?
\item[\vref{Ps 128:3}] autour de ta t. com. des plants
\item[\vref{Pr 3:3}] écris-les sur la t. de ton cœur ;
\item[\vref{Pr 7:3}] écris-les sur la t. de ton cœur.
\item[\vref{Jé 17:1}] gravé sur la t. de lr. cœur
\item[\vref{Da 11:26}] mets de sa t. le mettront en
\item[\vref{Mal 1:7}] en disant : La t. de Yahweh est
\item[\vref{Mal 1:12}] en disant : La t. de Yahweh est
\item[\vref{Mt 8:11}] et seront à t. ds le Royaume
\item[\vref{Mt 9:10}] Jésus était à t. ds la maison
\item[\vref{Mt 14:9}] qui étaient à t. avec lui, il
\item[\vref{Mt 15:27}] tombent de la t. de leurs maîtres.
\item[\vref{Mt 21:12}] il renversa les t. des changeurs et
\item[\vref{Mt 22:10}] remplie de conviés qui étaient à t.
\item[\vref{Mt 26:20}] se mit à t. avec les douze.
\item[\vref{Ac 6:2}] parole de Dieu pour servir aux t.
\item[\vref{Ro 11:9}] Que lr. t. soit pour eux un filet, un
\item[\vref{1 Co 8:10}] connaissance, être à t. ds le temple
\item[\vref{1 Co 10:21}] participer à la t. du Seign. et
\item[\vref{1 Co 11:21}] se met à t., chacun commence par
\item[\vref{2 Co 3:3}] non sur des t. de pierre, mais
\item[\vref{Ga 5:21}] les excès de t., et les choses
\item[\vref{Hé 9:2}] chandelier, et la t., et les pains
\end{listverse}

\ConcordanceEntry{Tache}
\vspace{-2mm}
\begin{listverse}
\item[\vref{Lé 13:4}] Mais si la t. est blanche sur
\item[\vref{Job 11:15}] ton visage sans t.. Tu seras ferme
\item[\vref{Pr 9:7}] reprend le méchant en reçoit une t.
\item[\vref{Ca 4:7}] et il n'y a point de t. en toi.
\item[\vref{Jé 13:23}] le léopard ses t. ? Pourriez-vs. aussi faire
\item[\vref{Ep 5:27}] Eglise glorieuse, sans t., ni ride, ni
\item[\vref{Col 1:22}] saints, et sans t., et irrépréhensibles dvt
\item[\vref{1 Ti 6:14}] te conservant sans t. et irrépréhensible, jusqu'à
\item[\vref{Hé 7:26}] saint, innocent, sans t., séparé des pécheurs,
\item[\vref{Hé 9:14}] Dieu sans nulle t., purifiera-t-il votre conscience
\item[\vref{Ja 1:27}] pure et sans t. dvt notre Dieu
\item[\vref{1 Pi 1:19}] d'un agneau sans défaut et sans t.,
\item[\vref{2 Pi 2:13}] Ce sont des t. et des souillures,
\item[\vref{2 Pi 3:14}] par lui sans t. et sans reproche
\item[\vref{Ap 14:5}] ils sont sans t. dvt le trône
\end{listverse}

\ConcordanceEntry{Taille}
\vspace{-2mm}
\begin{listverse}
\item[\vref{Ge 39:6}] était beau de t. et beau de
\item[\vref{No 13:32}] vus sont des gens de grande t.
\item[\vref{De 9:2}] et de haute t., les fils d'Anak,
\item[\vref{2 S 21:20}] hom. de haute t., qui avait six
\item[\vref{1 R 6:36}] de pierres de t. et d'une rangée
\item[\vref{Est 2:7}] était belle de t. et très belle
\item[\vref{Ez 13:18}] personnes de tte t., pour séduire les
\item[\vref{Lu 19:3}] foule, car il était de petite t.
\end{listverse}

\ConcordanceEntry{Taire}
\vspace{-2mm}
\begin{listverse}
\item[\vref{No 13:30}] Caleb fit t. le peuple dvt
\item[\vref{1 S 2:9}] les méchants se t. ds les ténèbres ;
\item[\vref{Est 4:14}] Mais si t. te tais, si tu te tais
\item[\vref{Ps 32:3}] je me suis t., mes os se
\item[\vref{Ps 35:22}] Yahweh, t. le vois ! Ne te tais point !
\item[\vref{Ps 50:3}] il ne se t. point ; il y
\item[\vref{Ps 83:2}] silence, ne te t. point, et ne
\item[\vref{Ps 131:2}] soumis et fait t. mon cœur, com.
\item[\vref{Pr 11:12}] de sens, mais l'hom. prudent se t.
\item[\vref{Es 42:14}] Je me suis t. dès longtemps ; me
\item[\vref{Es 62:6}] ils ne se t. point. Vous qui
\item[\vref{Es 65:6}] je ne me t. point, mais je
\item[\vref{Jé 4:19}] ne puis me t. ; car, ô mon
\item[\vref{Mt 20:31}] pour les faire t. ; mais ils criaient
\item[\vref{Mc 1:25}] le menaça, disant : T.-toi ! et sors
\item[\vref{Lu 19:40}] dis, s'ils se t., les pierres crieront.
\item[\vref{Ac 18:9}] pas, mais parle et ne te t. pas,
\item[\vref{1 Co 14:34}] parmi vs. se t. ds les églises ;
\end{listverse}

\ConcordanceEntry{Talent}
\vspace{-2mm}
\begin{listverse}
\item[\vref{Mt 18:24}] un qui lui devait dix mille t.
\item[\vref{Mt 25:15}] à l'un cinq t., à l'autre deux,
\item[\vref{Ap 16:21}] grêlons pesaient un t., tomba du ciel
\end{listverse}

\ConcordanceEntry{Talitha-Koumi}
\vspace{-2mm}
\begin{listverse}
\item[\vref{Mc 5:41}] et lui dit : Talitha koumi, qui étant expliqué,
\end{listverse}

\ConcordanceEntry{Talon}
\vspace{-2mm}
\begin{listverse}
\item[\vref{Ge 3:15}] tête, et tu lui écraseras le t.
\item[\vref{Ge 25:26}] sa main le t. d'Esaü ; c'est pourquoi
\item[\vref{Ps 41:10}] a levé le t. contre moi.
\item[\vref{Jn 13:18}] a levé son t. contre moi.
\end{listverse}

\ConcordanceEntry{Tamar}
\vspace{-2mm}
\begin{listverse}
\item[\vref{Ge 38:6}] Er, son premier-né, une fem. nommée T.
\item[\vref{Ru 4:12}] de Pérets, que T. enfanta à Juda !
\item[\vref{2 S 13:1}] qui se nommait T. ; et Amnon, fils
\end{listverse}

\ConcordanceEntry{Tambourin}
\vspace{-2mm}
\begin{listverse}
\item[\vref{Jg 11:34}] lui avec des t. et des danses.
\item[\vref{1 S 10:5}] du luth, du t., de la flûte,
\item[\vref{1 S 18:6}] Saül, avec des t., des triangles et
\item[\vref{2 S 6:5}] de luths, de t., de sistres et
\item[\vref{1 Ch 13:8}] des luths, des t., des cymbales, et
\item[\vref{Job 21:12}] au son du t. et de la
\item[\vref{Ps 149:3}] chantent avec le t. et la harpe !
\item[\vref{Es 5:12}] le luth, le t., la flûte et
\item[\vref{Es 30:32}] on entendra les t. et les harpes.
\item[\vref{Ez 28:13}] et d'or ; tes t. et tes flûtes
\end{listverse}

\ConcordanceEntry{Tapis}
\vspace{-2mm}
\begin{listverse}
\item[\vref{Ex 26:1}] tabernacle de dix t. de fin lin
\item[\vref{No 4:25}] porteront dc les t. du tabernacle, et
\item[\vref{Jg 5:10}] pour sièges des t. et vs. qui
\item[\vref{2 S 7:2}] l'arche de Dieu habite sous des t.
\item[\vref{Jé 49:29}] on prendra leurs t., ts leurs bagages
\end{listverse}

\ConcordanceEntry{Tard}
\vspace{-2mm}
\begin{listverse}
\item[\vref{Jg 19:9}] il se fait t., je vs. prie
\item[\vref{Job 24:6}] sera que fort t. qu'ils iront ravager
\item[\vref{Ps 127:2}] vs. vs. couchez t., et que vs.
\item[\vref{Mt 14:15}] il se faisait t., ses disciples vinrent
\item[\vref{Mc 6:35}] il était déjà t., ses disciples s'approchèrent
\item[\vref{Mc 11:11}] il était déjà t., il sortit pour
\item[\vref{Lu 22:59}] une heure plus t., un autre affirmait,
\item[\vref{Jn 13:36}] mntnt, mais tu me suivras plus t.
\item[\vref{Ac 4:3}] jusqu'au lendemain, parce qu'il était déjà t.
\item[\vref{Ga 3:17}] survenue quatre cent trente ans plus t.
\item[\vref{Hé 12:17}] savez que plus t., désirant hériter la
\end{listverse}

\ConcordanceEntry{Tarder}
\vspace{-2mm}
\begin{listverse}
\item[\vref{Ge 19:16}] Et com. il t., ces hommes le
\item[\vref{Ex 32:1}] voyant que Moïse t. tant à descendre
\item[\vref{Ps 40:18}] et mon libérateur : Mon Dieu, ne t. point !
\item[\vref{Es 46:13}] salut, il ne t. pas. Je mettrai
\item[\vref{Es 56:1}] mon salut ne t. pas à venir,
\item[\vref{Da 9:19}] et opère ! Ne t. pas, par amour
\item[\vref{Ha 2:3}] mentira pas. S'il t., attends-le ; car il
\item[\vref{Mt 24:48}] qui dit en lui-mm : Mon maître t. à venir.
\item[\vref{Mt 25:5}] Et com. l'époux t. à venir, elles
\item[\vref{Ac 9:38}] prier de venir chez eux sans t.
\item[\vref{Ac 22:16}] Et mntnt, pourquoi t.-tu ? Lève-toi, et
\item[\vref{Hé 10:37}] doit venir, viendra, et il ne t. pas.
\item[\vref{2 Pi 2:3}] depuis longtemps ne t. pas, et lr.
\end{listverse}

\ConcordanceEntry{Tarse}
\vspace{-2mm}
\begin{listverse}
\item[\vref{Ac 9:11}] Judas un hom. appelé Saul de T.,
\item[\vref{Ac 11:25}] s'en alla à T. pour chercher Saul ;
\item[\vref{Ac 22:3}] Juif, né à T. en Cilicie ; mais
\end{listverse}

\ConcordanceEntry{Tarsis}
\vspace{-2mm}
\begin{listverse}
\item[\vref{Ge 10:4}] de Javan : Elischa, T., Kittim, et Dodanim.
\end{listverse}
\begin{legend}
\NoAutoSpaceBeforeFDP{
\item Fils de Javan, origines de la division des îles des nations, ville : Ge 10:4; Jé 10:9
\item Désobéissance et fuite de Jonas : Jon 1:3; 4:2
\item La flotte de T : 1 R 10:22; 22:49; Ps 48:8
\item Paroles prophétiques : Ps 72:10; Es 23:1; 60:9
}
\end{legend}

\ConcordanceEntry{Tâtonner}
\vspace{-2mm}
\begin{listverse}
\item[\vref{De 28:29}] et tu t. en plein midi com. tâtonne un
\item[\vref{Job 12:25}] ils t. ds les ténèbres, sans aucune clarté,
\item[\vref{Es 59:10}] Nous t. com. des aveugles le long du
\item[\vref{Ac 13:11}] il cherchait, en t., des personnes pour
\item[\vref{Ac 17:27}] le trouver en t., quoiqu'il ne soit
\end{listverse}

\ConcordanceEntry{Taureau}
\vspace{-2mm}
\begin{listverse}
\item[\vref{Lé 1:5}] égorgera le jeune t. dvt Yahweh ; et
\item[\vref{Jg 6:25}] Prends un jeune t. d'entre les bœufs
\item[\vref{Job 42:8}] prenez mntnt sept t. et sept béliers,
\item[\vref{Ps 22:13}] Plusieurs t. sont autour de
\item[\vref{Ps 50:9}] prendrai point de t. de ta maison,
\item[\vref{Ps 51:21}] on offrira des t. sur ton autel.
\item[\vref{Ac 14:13}] ayant amené des t. et des couronnes
\item[\vref{Hé 9:13}] le sang des t. et des boucs,
\end{listverse}

\ConcordanceEntry{Tav}
\vspace{-2mm}
\begin{listverse}
\item[\vref{Ez 9:4}] marque la lettre T. sur les fronts
\item[\vref{Ez 9:6}] ont la lettre T., et commencez par
\end{listverse}

\ConcordanceEntry{Teigne}
\vspace{-2mm}
\begin{listverse}
\item[\vref{Lé 13:31}] plaie de la t. ne paraît pas
\item[\vref{De 28:27}] gale, et de t., dont tu ne
\item[\vref{Job 13:28}] robe que la t. a rongée.
\item[\vref{Job 27:18}] celle de la t., com. la cabane
\item[\vref{Ps 39:12}] détruis com. la t. ce qu'il a
\item[\vref{Es 50:9}] un vêt., la t. les dévorera.
\item[\vref{Es 51:8}] Car la t. les rongera com. un vêt., et
\item[\vref{Os 5:12}] dc com. une t. pour Ephraïm, com.
\item[\vref{Lu 12:33}] et où la t. ne gâte rien.
\end{listverse}

\ConcordanceEntry{Tékoa}
\vspace{-2mm}
\begin{listverse}
\item[\vref{2 S 14:2}] envoya chercher à T. une fem. habile,
\item[\vref{Am 1:1}] les bergers de T., visions qu'il eut
\end{listverse}

\ConcordanceEntry{Témoignage}
\vspace{-2mm}
\begin{listverse}
\item[\vref{Ex 20:16}] pas de faux t. contre ton prochain.
\item[\vref{De 31:19}] me serve de t. contre les fils
\item[\vref{Jos 22:27}] qu'il serve de t. entre ns. et
\item[\vref{Ru 4:7}] en Israël, un t. qu'on cédait son
\item[\vref{Job 29:11}] l'œil qui me voyait me rendait t. ;
\item[\vref{Ps 19:8}] restaure l'âme ; le t. de Yahweh est
\item[\vref{Ps 78:5}] a établi le t. en Jacob, et
\item[\vref{Ps 119:24}] Tes t. font mes délices, ce sont mes
\item[\vref{Ps 132:12}] alliance et mon t. que je lr.
\item[\vref{Pr 12:17}] véritables rend un t. juste, mais le
\item[\vref{Pr 25:18}] porte un faux t. contre son prochain
\item[\vref{Es 8:20}] loi et au t. ! Si l'on ne
\item[\vref{Es 19:20}] signe et un t. pour Yahweh des
\item[\vref{Jé 44:23}] ni ds ses t. ; c'est pour cela
\item[\vref{Os 7:10}] d'Israël dc rendra t. contre lui ; car
\item[\vref{Mt 8:4}] afin que cela lr. serve de t.
\item[\vref{Mt 10:18}] moi, pour rendre t. de moi dvt
\item[\vref{Mt 15:19}] vols, les faux t., les médisances.
\item[\vref{Mt 24:14}] pour servir de t. à ttes les
\item[\vref{Mt 26:59}] cherchaient des faux t. contre Jésus pour
\item[\vref{Jn 1:7}] vint pour rendre t., pour rendre, dis-je,
\item[\vref{Jn 2:25}] qu'on lui rende t. d'aucun hom. ; car
\item[\vref{Jn 3:32}] témoigne ; mais personne ne reçoit son t.
\item[\vref{Jn 4:44}] Jésus avait rendu t. qu'un prophète n'est
\item[\vref{Jn 5:36}] moi, j'ai un t. plus grand que
\item[\vref{Jn 5:39}] et ce sont elles qui rendent t. de moi.
\item[\vref{Jn 10:25}] au Nom de mon Père rendent t. de moi.
\item[\vref{Jn 15:26}] procède de mon Père, il rendra t. de moi ;
\item[\vref{Jn 18:37}] monde, pour rendre t. à la vérité.
\item[\vref{Jn 21:24}] disciple qui rend t. de ces choses,
\item[\vref{Ac 4:33}] les apôtres rendaient t. avec une grande
\item[\vref{Ac 6:3}] on ait bon t., pleins du Saint-Esprit
\item[\vref{Ac 14:3}] Seign., qui rendait t. à la parole
\item[\vref{Ac 20:24}] Jésus, pour rendre t. à l'Evangile de
\item[\vref{Ac 22:18}] recevront pas le t. que tu lr.
\item[\vref{Ac 23:11}] tu as rendu t. de moi ds
\item[\vref{Ro 2:15}] conscience lr. rendant t., et leurs pensées
\item[\vref{Ro 8:16}] L'Esprit lui-mm rend t. à notre esprit
\item[\vref{1 Co 1:6}] selon que le t. de Jésus-Christ a
\item[\vref{1 Ti 3:7}] reçoive un bon t. de ceux du
\item[\vref{2 Ti 1:8}] pas honte du t. à rendre à
\item[\vref{Hé 2:4}] confirmant aussi lr. t. par des prodiges,
\item[\vref{Hé 7:17}] lui rend ce t. : Tu es prêtre
\item[\vref{Hé 11:2}] les anciens ont obtenu un bon t.
\item[\vref{Ja 5:3}] rouille s'élèvera en t. contre vs. et
\item[\vref{1 Pi 1:11}] rendait à l'avance t., lr. faisant connaître
\item[\vref{1 Jn 5:7}] ciel qui rendent t. : Le Père, la
\item[\vref{1 Jn 5:9}] ns. recevons le t. des hommes, le
\item[\vref{Ap 6:9}] et pour le t. qu'ils avaient gardé.
\item[\vref{Ap 11:7}] de rendre lr. t., la bête qui
\item[\vref{Ap 12:11}] parole de lr. t., et ils n'ont
\item[\vref{Ap 12:17}] qui ont le t. de Jésus-Christ.
\item[\vref{Ap 15:5}] du tabernacle du t. fut ouvert ds
\item[\vref{Ap 19:10}] qui ont le t. de Jésus. Adore
\item[\vref{Ap 20:4}] décapités pour le t. de Jésus, et
\item[\vref{Ap 22:20}] Celui qui rend t. de ces choses,
\end{listverse}

\ConcordanceEntry{Témoin}
\vspace{-2mm}
\begin{listverse}
\item[\vref{Ge 31:52}] ce monceau soit t. et que ce
\item[\vref{No 35:30}] parole de deux t. ; mais un seul
\item[\vref{De 4:26}] j'appelle aujourd'hui à t. les cieux et
\item[\vref{De 17:6}] parole de deux t. ou de trois
\item[\vref{De 19:15}] Un seul t. ne sera point valable contre un
\item[\vref{De 19:16}] Quand un faux t. s'élèvera contre un
\item[\vref{Jos 24:27}] pierre servira de t. contre ns., car
\item[\vref{Ru 4:9}] Vous êtes aujourd'hui t. que j'ai acquis
\item[\vref{Job 16:19}] mntnt voilà, mon t. est aux cieux,
\item[\vref{Ps 89:38}] la lune. Le t. qui est ds
\item[\vref{Pr 14:5}] Le t. véritable ne ment jamais, mais le
\item[\vref{Pr 14:25}] Le t. fidèle délivre les âmes, mais celui
\item[\vref{Pr 19:9}] Le faux t. ne restera pas impuni, et celui
\item[\vref{Es 43:10}] Vous êtes mes t., dit Yahweh, et
\item[\vref{Es 44:8}] Vous êtes mes t. ; y a-t-il un
\item[\vref{Jé 42:5}] entre ns. un t. véritable et fidèle,
\item[\vref{Mal 2:14}] est intervenu com. t. entre toi et
\item[\vref{Mt 18:16}] deux ou trois t. tte parole soit
\item[\vref{Mt 26:60}] que plusieurs faux t. se soient présentés,
\item[\vref{Mt 26:65}] encore besoin de t. ? Voici, vs. avez
\item[\vref{Jn 3:28}] Vous-mêmes m'êtes t. que j'ai dit :
\item[\vref{Ac 1:8}] vs. serez mes t., tant à Jérus.
\item[\vref{Ac 5:32}] Nous sommes t. de ce que
\item[\vref{Ac 10:41}] peuple, mais aux t. choisis d'avance par
\item[\vref{Ac 20:26}] prends aujourd'hui à t. que je suis
\item[\vref{Ac 22:15}] lui serviras de t. auprès de ts
\item[\vref{1 Co 15:15}] sommes de faux t. de la part
\item[\vref{2 Co 13:1}] ou de trois t., tte parole sera
\item[\vref{1 Ti 5:19}] déposition de deux ou de trois t.
\item[\vref{Hé 12:1}] grande nuée de t., rejetons tt fardeau,
\item[\vref{1 Pi 5:1}] avec eux, et t. des souffrances de
\item[\vref{Ap 1:5}] qui est le t. fidèle, le premier-né
\item[\vref{Ap 3:14}] dit l'Amen, le t. fidèle et véritable,
\item[\vref{Ap 11:3}] à mes deux t. de prophétiser pendant
\end{listverse}

\ConcordanceEntry{Tempête}
\vspace{-2mm}
\begin{listverse}
\item[\vref{Ps 50:3}] tt autour de lui une grosse t.
\item[\vref{Ps 107:25}] fait paraître la t. qui élève les
\item[\vref{Es 29:6}] bruit ; avec la t., le tourbillon, et
\item[\vref{Es 54:11}] agitée de la t., dénuée de consolation,
\item[\vref{Es 66:15}] seront com. la t. ; afin qu'il tourne
\item[\vref{Jé 23:19}] Voici la t. de Yahweh, sa fureur va se
\item[\vref{Os 8:7}] ils moissonneront la t. ; ils n'auront pas
\item[\vref{Jon 1:12}] moi la cause de cette grande t.
\item[\vref{Na 1:3}] tourbillons et les t., les nuées sont
\item[\vref{Mt 8:24}] une si grande t. que la barque
\item[\vref{Ac 27:18}] battus par la t., le jour suivant,
\item[\vref{Hé 12:18}] ni des ténèbres, ni de la t.,
\item[\vref{Ja 3:4}] agités par la t., ils sont dirigés
\item[\vref{2 Pi 3:10}] bruit d'une effroyable t., et les éléments
\end{listverse}

\ConcordanceEntry{Temple}
\vspace{-2mm}
\begin{listverse}
\item[\vref{1 S 3:3}] couché ds le t. de Yahweh, où
\item[\vref{2 R 24:13}] faits pour le t. de Yahweh, com.
\item[\vref{Esd 3:6}] bien que le t. de Yahweh n'était
\item[\vref{Esd 3:8}] à fonder le t. ; et ils établirent
\item[\vref{Esd 4:1}] captivité rebâtissaient un t. à Yahweh, le
\item[\vref{Ps 11:4}] ds son saint t., Yahweh a son
\item[\vref{Ps 27:4}] de Yahweh et pour admirer son t.
\item[\vref{Ps 68:30}] Dans ton t., à Jérus., les rois t'amèneront des
\item[\vref{Ps 79:1}] profané ton saint t., on a mis
\item[\vref{Es 6:1}] pans de sa robe remplissaient le t.
\item[\vref{Es 44:28}] rebâtie ! Et au t. : Tu seras fondé.
\item[\vref{Es 66:6}] son sort du t., le son de
\item[\vref{Jé 50:28}] notre Dieu, la vengeance de son t. !
\item[\vref{Ez 8:16}] à l'entrée du t. de Yahweh, entre
\item[\vref{Ez 41:1}] entrer ds le t., et il mesura
\item[\vref{Joë 3:5}] emporté ds vos t. mes belles choses
\item[\vref{Jon 2:5}] verrai encore le t. de ta sainteté.
\item[\vref{Ha 2:20}] est ds le t. de sa sainteté.
\item[\vref{Ag 2:15}] sur pierre au t. de Yahweh !
\item[\vref{Za 6:12}] qui bâtira le t. de Yahweh.
\item[\vref{Mal 3:1}] entrera ds son t. le Seign. que
\item[\vref{Mt 4:5}] le mit sur le haut du t.,
\item[\vref{Mt 12:6}] quelqu'un de plus grand que le t.
\item[\vref{Mt 21:12}] entra ds le t. de Dieu. Il
\item[\vref{Mt 23:17}] l'or, ou le t. qui sanctifie l'or ?
\item[\vref{Mt 23:21}] jure par le t., jure par le
\item[\vref{Mt 23:35}] tué entre le t. et l'autel.
\item[\vref{Mt 24:1}] s'en allait du t., ses disciples s'approchèrent
\item[\vref{Mt 26:55}] enseignant ds le t., et vs. ne
\item[\vref{Mt 26:61}] puis détruire le t. de Dieu et
\item[\vref{Mt 27:5}] d'argent ds le t., il se retira
\item[\vref{Mt 27:51}] le voile du t. se déchira en
\item[\vref{Lu 2:27}] Il vint au t., poussé par l'Esprit.
\item[\vref{Lu 2:46}] trouvèrent ds le t., assis au milieu
\item[\vref{Lu 11:51}] l'autel et le t.. Oui, je vs.
\item[\vref{Lu 18:10}] hommes montèrent au t. pour prier, l'un
\item[\vref{Lu 22:53}] vs. ds le t., et vs. n'avez
\item[\vref{Jn 2:21}] il parlait du t. de son corps.
\item[\vref{Ac 2:46}] accord ds le t. ; et rompant le
\item[\vref{Ac 3:1}] montaient ensemble au t., à l'heure de
\item[\vref{Ac 5:20}] présentez-vs. ds le t., annoncez au peuple
\item[\vref{Ac 17:24}] pas ds des t. faits de main
\item[\vref{Ac 19:24}] fabriquait de petits t. d'argent de Diane,
\item[\vref{1 Co 3:16}] vs. êtes le t. de Dieu et
\item[\vref{1 Co 6:19}] corps est le t. du Saint-Esprit qui
\item[\vref{1 Co 8:10}] table ds le t. des idoles, sa
\item[\vref{2 Co 6:16}] a-t-il entre le t. de Dieu et
\item[\vref{Ep 2:21}] pour être un t. saint ds le
\item[\vref{2 Th 2:4}] Dieu ds le t. de Dieu se
\item[\vref{Ap 11:19}] Et le t. de Dieu fut ouvert ds le
\item[\vref{Ap 15:5}] et voici le t. du tabernacle du
\item[\vref{Ap 16:1}] Et j'entendis du t. une voix éclatante
\item[\vref{Ap 21:22}] vis pas de t. ds la ville,
\end{listverse}

\ConcordanceEntry{Temps}
\vspace{-2mm}
\begin{listverse}
\item[\vref{Ge 6:4}] avait en ce t.-là des géants
\item[\vref{Ge 21:2}] sa vieillesse, au t. précis que Dieu
\item[\vref{2 S 7:19}] serviteur pour les t. éloignés. Est-ce là
\item[\vref{1 Ch 12:32}] la connaissance des t., pour savoir ce
\item[\vref{Job 27:10}] le Tout-Puissant ? Invoque-t-il Dieu en tt t. ?
\item[\vref{Job 38:32}] sortir en lr. t. les signes du
\item[\vref{Ps 34:2}] Yahweh en tt t. ; sa louange sera
\item[\vref{Ps 72:7}] En son t., le juste fleurira, et il y
\item[\vref{Ps 75:3}] Au t. que j'aurai fixé, je jugerai avec
\item[\vref{Ps 102:14}] car il est t. d'en avoir pitié,
\item[\vref{Ps 104:27}] lr. donnes la nourriture en lr. t.
\item[\vref{Ec 3:1}] tte affaire sous les cieux son t.
\item[\vref{Ec 7:17}] point insensé : Pourquoi mourrais-tu avant ton t. ?
\item[\vref{Ec 9:11}] mais que le t. et les circonstances
\item[\vref{Es 44:8}] déclaré dès ce t.-là ? Vous êtes
\item[\vref{Es 60:22}] je hâterai ces choses en lr. t.
\item[\vref{Es 63:16}] Nom est notre Rédempteur de tt t.
\item[\vref{Ez 12:27}] prophétise pour des t. qui sont encore
\item[\vref{Ez 30:3}] ce sera le t. des nations.
\item[\vref{Da 2:21}] qui change les t. et les saisons,
\item[\vref{Da 7:25}] de changer les t. et la loi ;
\item[\vref{Da 8:17}] est pour le t. de la fin.
\item[\vref{Da 9:25}] mais en des t. d'angoisse.
\item[\vref{Za 12:4}] En ce t.-là, dit Yahweh, je frapperai d'étourdissement
\item[\vref{Mt 8:29}] venu ici ns. tourmenter avant le t. ?
\item[\vref{Mt 11:12}] Or depuis le t. de Jean-Baptiste jusqu'à
\item[\vref{Mt 13:21}] croit pour un t., et dès que
\item[\vref{Mt 16:3}] ne pouvez discerner les signes des t.
\item[\vref{Mt 26:18}] Maître dit : Mon t. est proche ; je
\item[\vref{Mc 2:26}] de Dieu, au t. du grand-prêtre Abiathar,
\item[\vref{Lu 4:13}] diable s'éloigna de lui pour un t.
\item[\vref{Lu 8:13}] croient pour un t., mais au moment
\item[\vref{Lu 12:42}] la nourriture au t. convenable ?
\item[\vref{Lu 19:44}] pas connu le t. de ta visitation.
\item[\vref{Lu 21:36}] priez en tt t., afin que vs.
\item[\vref{Jn 7:6}] lr. dit : Mon t. n'est pas encore
\item[\vref{Ac 1:6}] est-ce en ce t.-ci que tu
\item[\vref{Ac 3:20}] afin que des t. de rafraîchissement viennent
\item[\vref{Ac 17:26}] la durée des t. et les bornes
\item[\vref{Ac 17:30}] tenir compte des t. d'ignorance, annonce mntnt
\item[\vref{Ro 8:18}] les souffrances du t. présent ne sont
\item[\vref{1 Co 7:29}] mes frères : Le t. est court, que
\item[\vref{2 Co 4:18}] que pour un t., mais les invisibles
\item[\vref{2 Co 6:2}] t'ai exaucé au t. favorable et t'ai
\item[\vref{Ga 4:4}] Mais lorsque les t. ont été accomplis,
\item[\vref{Ep 3:9}] caché de tt t. en Dieu, qui
\item[\vref{Ep 5:16}] rachetant le t., car les jours
\item[\vref{Col 4:5}] ceux du dehors, et rachetez le t.
\item[\vref{1 Th 5:1}] qui est des t. et des moments,
\item[\vref{2 Th 2:6}] afin qu'il soit révélé en son t.
\item[\vref{1 Ti 4:1}] ds les derniers t., quelques-uns se détourneront
\item[\vref{1 Ti 6:15}] manifesté en son t., qui est le
\item[\vref{2 Ti 1:9}] Jésus-Christ avant les t. éternels,
\item[\vref{2 Ti 3:1}] il surviendra des t. difficiles.
\item[\vref{2 Ti 4:3}] il viendra un t. où les hommes
\item[\vref{Hé 4:16}] être secourus en t. de besoin.
\item[\vref{Hé 9:10}] cérémonies charnelles, jusqu'au t. de la réforme.
\item[\vref{1 Pi 1:6}] un peu de t. par diverses épreuves,
\item[\vref{1 Pi 4:17}] Car il est t. que le jugement
\item[\vref{Ap 1:3}] écrites ! Car le t. est proche.
\item[\vref{Ap 2:21}] ai donné du t., afin qu'elle se
\item[\vref{Ap 12:12}] fureur, sachant qu'il a peu de t.
\item[\vref{Ap 12:14}] est nourrie un t., des temps, et
\item[\vref{Ap 22:10}] livre. Car le t. est proche.
\end{listverse}

\ConcordanceEntry{Ténèbres}
\vspace{-2mm}
\begin{listverse}
\item[\vref{Ge 1:2}] et vide ; les t. étaient à la
\item[\vref{Ge 1:5}] il appela les t. nuit. Ainsi fut
\item[\vref{Ex 10:22}] y eut d'épaisses t. ds tt le
\item[\vref{De 4:11}] y avait des t., une nuée, et
\item[\vref{2 S 22:12}] de lui les t. pour tabernacle, des
\item[\vref{Job 10:21}] la terre de t. et de l'ombre
\item[\vref{Job 38:19}] et où est le lieu des t.,
\item[\vref{Ps 18:12}] Il faisait des t. sa demeure secrète,
\item[\vref{Ps 18:29}] lumière ; Yahweh, mon Dieu, éclaire mes t.
\item[\vref{Ps 91:6}] marche ds les t., ni la destruction
\item[\vref{Es 8:22}] aura que détresse, t. et de sombres
\item[\vref{Es 9:1}] marchait ds les t. voit une grande
\item[\vref{Es 60:2}] Car voici, les t. couvrent la terre,
\item[\vref{Jé 2:31}] un pays de t. ? Pourquoi mon peuple
\item[\vref{Joë 2:2}] jour de t. et d'obscurité, jour de nuées et
\item[\vref{Am 5:8}] change les profonds t. en aube du
\item[\vref{So 1:15}] un jour de t. et d'obscurité, un
\item[\vref{Mt 4:16}] assis ds les t., a vu une
\item[\vref{Mt 6:23}] toi n'est que t., combien seront grandes
\item[\vref{Mt 8:12}] jetés ds les t. du dehors, où
\item[\vref{Mt 10:27}] dis ds les t., dites-le ds la
\item[\vref{Mt 22:13}] jetez-le ds les t. de dehors, où
\item[\vref{Mc 15:33}] y eut des t. sur tte la
\item[\vref{Lu 1:79}] assis ds les t. et ds l'ombre
\item[\vref{Lu 22:53}] votre heure, et la puissance des t.
\item[\vref{Lu 23:44}] y eut des t. sur tte la
\item[\vref{Jn 1:5}] luit ds les t., mais les ténèbres
\item[\vref{Jn 8:12}] pas ds les t., mais il aura
\item[\vref{Ac 13:11}] l'obscurité et les t. tombèrent sur lui,
\item[\vref{Ac 26:18}] qu'ils passent des t. à la lumière,
\item[\vref{Ro 13:12}] les œuvres des t., et soyons revêtus
\item[\vref{1 Co 4:5}] cachées ds les t. et manifestera les
\item[\vref{Ep 5:8}] vs. étiez autrefois t., mais mntnt vs.
\item[\vref{Ep 5:11}] œuvres infructueuses des t., mais au contraire
\item[\vref{Col 1:13}] la puissance des t., et ns. a
\item[\vref{1 Pi 2:9}] a appelés des t. à sa merveilleuse
\item[\vref{2 Pi 2:17}] qui l'obscurité des t. est réservée éternellement.
\item[\vref{1 Jn 1:5}] n'y a point en lui de t.
\item[\vref{1 Jn 2:11}] est ds les t., et il marche
\item[\vref{Jud 1:13}] qui l'obscurité des t. est réservée éternellement.
\item[\vref{Ap 16:10}] fut couvert de t., et les hommes
\end{listverse}

\ConcordanceEntry{Ténébreux}
\vspace{-2mm}
\begin{listverse}
\item[\vref{Ps 35:6}] lr. chemin soit t. et glissant, et
\item[\vref{Ps 74:20}] car les lieux t. de la terre
\item[\vref{Pr 2:13}] droiture pour marcher ds les chemins t.,
\item[\vref{Es 45:19}] ds qq lieu t. de la terre ;
\item[\vref{La 3:6}] ds les lieux t., com. ceux qui
\item[\vref{Ez 30:3}] c'est un jour t. ; ce sera le
\item[\vref{Mi 3:6}] prophètes-là, et le jour lr. sera t.
\item[\vref{Mt 6:23}] ton corps sera t.. Si dc la
\end{listverse}

\ConcordanceEntry{Tenir}
\vspace{-2mm}
\begin{listverse}
\item[\vref{Ge 18:22}] mais Abraham se t. encore dvt Yahweh.
\item[\vref{Ge 25:26}] sortit son frère, t. de sa main
\item[\vref{Ex 9:11}] ne purent se t. dvt Moïse, à
\item[\vref{Ex 17:6}] je vais me t. là dvt toi
\item[\vref{Ex 32:26}] Et Moïse se t. à la porte
\item[\vref{Ex 34:29}] montagne de Sinaï, t. ds sa main
\item[\vref{De 10:8}] Yahweh, de se t. dvt Yahweh, de
\item[\vref{Jos 10:8}] aucun d'eux ne t. dvt toi.
\item[\vref{2 S 22:34}] il me fait t. debout sur mes
\item[\vref{1 R 22:19}] des cieux se t. dvt lui, à
\item[\vref{Job 41:1}] capable de se t. debout dvt moi ?
\item[\vref{Ps 25:8}] aux pécheurs le chemin qu'ils doivent t.
\item[\vref{Ps 76:8}] Qui peut se t. dvt toi qnd
\item[\vref{Es 6:6}] vola vers moi, t. à la main
\item[\vref{Da 8:18}] et me fit t. debout à la
\item[\vref{Am 9:1}] le Seign. se t. debout sur l'autel,
\item[\vref{Mt 6:5}] prier en se t. debout ds les
\item[\vref{Mt 26:7}] s'approcha de lui t. un vase d'albâtre,
\item[\vref{Lu 1:19}] Gabriel, je me t. dvt Dieu et
\item[\vref{Lu 10:39}] Marie, qui se t. assise aux pieds
\item[\vref{Ac 17:30}] Mais Dieu, sans t. compte des temps
\item[\vref{Ep 6:13}] mauvais jour, et t. ferme après avoir
\item[\vref{Ja 3:2}] il peut mm t. en bride tt
\item[\vref{3 Jn 1:10}] qu'il commet, en t. contre ns. de
\item[\vref{Ap 6:11}] dit de se t. en repos encore
\end{listverse}

\ConcordanceEntry{Tentateur}
\vspace{-2mm}
\begin{listverse}
\item[\vref{Mt 4:3}] Et le t., s'étant approché, lui dit : Si tu
\item[\vref{1 Th 3:5}] peur que le t. ne vs. ait
\end{listverse}

\ConcordanceEntry{Tentation}
\vspace{-2mm}
\begin{listverse}
\item[\vref{Mt 6:13}] induis pas en t., mais délivre-ns. du
\item[\vref{Mt 26:41}] tombiez pas en t., car l'esprit est
\item[\vref{Lu 8:13}] moment de la t. ils se retirent.
\item[\vref{1 Co 10:13}] Aucune t. ne vs. a éprouvés, qui n'ait
\item[\vref{1 Ti 6:9}] tombent ds la t., ds le piège,
\item[\vref{Hé 3:8}] jour de la t. ds le désert,
\item[\vref{Ja 1:12}] qui endure la t. ; car après avoir
\item[\vref{Ap 3:10}] l'heure de la t. qui doit arriver
\end{listverse}

\ConcordanceEntry{Tente}
\vspace{-2mm}
\begin{listverse}
\item[\vref{Ge 4:20}] habitent ds les t. et des pasteurs.
\item[\vref{Ge 9:21}] se découvrit au milieu de sa t.
\item[\vref{Ge 9:27}] habite ds les t. de Sem ; et
\item[\vref{Ge 12:8}] il dressa ses t., ayant Béthel à
\item[\vref{Ex 29:44}] sanctifierai dc la t. d'assignation et l'autel.
\item[\vref{Ex 33:7}] Moïse prit une t. et la dressa
\item[\vref{Ex 40:34}] nuée couvrit la t. d'assignation, et la
\item[\vref{Lé 16:20}] sanctuaire, pour la t. d'assignation et pour
\item[\vref{Lé 23:42}] jours sous des t. ; ts ceux qui
\item[\vref{No 1:1}] Sinaï, ds la t. d'assignation, le premier
\item[\vref{No 2:2}] autour de la t. d'assignation, vis-à-vis de
\item[\vref{2 S 7:6}] là sous une t. et ds un
\item[\vref{Né 8:16}] se firent des t., chacun sur son
\item[\vref{Job 36:29}] et le son éclatant de sa t. ?
\item[\vref{Ps 15:1}] séjournera ds ta t. ? Qui habitera sur
\item[\vref{Ps 27:5}] l'abri de sa t. ; il m'élèvera sur
\item[\vref{Ps 27:6}] sacrifices ds sa t., au son de
\item[\vref{Ps 76:3}] sa t. est à Salem, et sa demeure
\item[\vref{Es 54:2}] l'espace de ta t., et qu'on étende
\item[\vref{Os 12:10}] habiter ds des t., com. aux jours
\item[\vref{Am 5:26}] avez porté la t. de votre Moloc,
\item[\vref{Za 12:7}] sauvera premièrement les t. de Juda, afin
\item[\vref{Mal 2:12}] le retranchera des t. de Jacob, et
\item[\vref{Mt 17:4}] le veux, trois t., une pour toi,
\item[\vref{Ac 7:46}] pouvoir dresser une t. pour le Dieu
\item[\vref{Ac 18:3}] lr. métier était de faire des t.
\item[\vref{2 Co 5:1}] qui n'est qu'une t., est détruite, ns.
\item[\vref{2 Co 5:4}] sommes ds cette t., ns. gémissons, étant
\item[\vref{Hé 11:9}] demeurant sous des t. avec Isaac et
\item[\vref{2 Pi 1:13}] pendant que je suis ds cette t.,
\end{listverse}

\ConcordanceEntry{Tenter}
\vspace{-2mm}
\begin{listverse}
\item[\vref{Ex 17:2}] contre moi ? Pourquoi t.-vs. Yahweh ?
\item[\vref{No 14:22}] qui m'ont déjà t. par dix fois,
\item[\vref{De 6:16}] Vous ne t. point Yahweh, votre Dieu, com. vs.
\item[\vref{Ps 78:18}] Ils t. Dieu ds leurs cœurs, en demandant
\item[\vref{Ps 78:41}] ne cessèrent de t. Dieu et de
\item[\vref{Ps 95:9}] vos pères m'ont t. et éprouvé bien
\item[\vref{Ps 106:14}] désert et ils t. Dieu ds le
\item[\vref{Mal 3:15}] établis ; mm ils t. Dieu et sont
\item[\vref{Mt 4:1}] désert, pour être t. par le diable.
\item[\vref{Mt 22:18}] lr. malice, dit : Hypocrites ! Pourquoi me t.-vs. ?
\item[\vref{Lu 4:12}] écrit : Tu ne t. pas le Seign.,
\item[\vref{Ac 5:9}] entre vs. pour t. l'Esprit du Seign. ?
\item[\vref{Ac 15:10}] Maintenant dc pourquoi t.-vs. Dieu en
\item[\vref{Ac 24:6}] Il a mm t. de profaner le
\item[\vref{1 Co 7:5}] Satan ne vs. t. par votre manque
\item[\vref{1 Co 10:9}] Ne t. pas Christ, com. quelques-uns d'entre eux
\item[\vref{1 Co 10:13}] que vs. soyez t. au-delà de vos
\item[\vref{Ga 6:1}] peur que tu ne sois aussi t.
\item[\vref{1 Th 3:5}] ne vs. ait t. en qq sorte,
\item[\vref{Hé 2:18}] souffert lui-mm, étant t., il est puissant
\item[\vref{Hé 3:9}] vos pères me t. et m'éprouvèrent, et
\item[\vref{Hé 4:15}] il a été t. com. ns. en
\item[\vref{Hé 11:29}] Egyptiens essayèrent de t., ils furent engloutis
\item[\vref{Ja 1:13}] personne, lorsqu'il est t., ne dise : Je
\item[\vref{Ja 1:14}] Mais chacun est t. qnd il est
\end{listverse}

\ConcordanceEntry{Térach}
\vspace{-2mm}
\begin{listverse}
\item[\vref{Ge 11:26}] Et T., ayant vécu soixante-dix ans, engendra Abram,
\item[\vref{Jos 24:2}] d'Israël : Vos pères, T., père d'Abraham et
\end{listverse}

\ConcordanceEntry{Térébinthe}
\vspace{-2mm}
\begin{listverse}
\item[\vref{Ge 35:4}] cacha sous un t. qui est près
\item[\vref{Jg 6:11}] s'assit sous le t. d'Ophra, qui appartenait
\item[\vref{Es 1:29}] à cause des t. que vs. avez
\item[\vref{Es 6:13}] Mais com. le t. et le chêne
\item[\vref{Es 61:3}] les appelle des t. de la justice,
\item[\vref{Os 4:13}] peupliers, et les t., parce que lr.
\end{listverse}

\ConcordanceEntry{Terre}
\vspace{-2mm}
\begin{listverse}
\item[\vref{Ge 1:2}] Et la t. devint informe et vide ; les ténèbres
\item[\vref{Ge 1:10}] appela le sec t. ; et il appela
\item[\vref{Ge 8:22}] jours de la t., les semailles et
\item[\vref{Ex 9:29}] saches que la t. est à Yahweh.
\item[\vref{De 32:1}] et je parlerai. T. ! écoute les paroles
\item[\vref{2 S 22:8}] Alors la t. fut ébranlée et trembla, les fondements
\item[\vref{1 R 8:27}] véritablement sur la t. ? Voilà, les cieux,
\item[\vref{1 R 19:11}] un tremblement de t. ; mais Yahweh n'était
\item[\vref{Job 9:6}] Il remue la t. de sa place,
\item[\vref{Job 9:24}] lui que la t. est livrée entre
\item[\vref{Job 10:21}] plus, en la t. de ténèbres et
\item[\vref{Job 10:22}] t. d'une grande obscurité, com. étant les
\item[\vref{Job 12:8}] parle à la t., et elle t'enseignera ;
\item[\vref{Job 19:25}] se lèvera le dernier sur la t.
\item[\vref{Job 20:4}] Dieu a mis l'hom. sur la t.,
\item[\vref{Job 26:7}] il suspend la t. sur le néant.
\item[\vref{Job 33:6}] formé de la t. tt com. toi.
\item[\vref{Ps 18:8}] La t. fut ébranlée et trembla, les fondements
\item[\vref{Ps 24:1}] de David. La t. appartient à Yahweh,
\item[\vref{Ps 33:8}] Que tte la t. craigne Yahweh ! Que
\item[\vref{Ps 46:3}] point qnd la t. est bouleversée, et
\item[\vref{Ps 63:2}] toi sur cette t. aride, desséchée, et
\item[\vref{Ps 65:10}] Tu visites la t., tu lui donnes
\item[\vref{Ps 68:9}] La t. trembla et les cieux répandirent leurs
\item[\vref{Ps 72:19}] que tte la t. soit remplie de
\item[\vref{Ps 75:4}] La t. se dissout avec ts ceux qui
\item[\vref{Ps 90:2}] aies formé la t. et le monde,
\item[\vref{Ps 104:24}] avec sagesse. La t. est pleine de
\item[\vref{Ps 114:7}] Ô t. ! tremble dvt la présence du Seign.,
\item[\vref{Ps 115:16}] a donné la t. aux fils des
\item[\vref{Pr 2:21}] droits habiteront la t., les hommes intègres
\item[\vref{Pr 2:22}] retranchés de la t., et ceux qui
\item[\vref{Pr 3:19}] a fondé la t. par la sagesse,
\item[\vref{Pr 8:16}] et ts les juges de la t.
\item[\vref{Pr 8:23}] le commencement, avant l'origine de la t.
\item[\vref{Pr 8:26}] encore fait la t. et les campagnes,
\item[\vref{Ec 1:4}] vient, mais la t. demeure toujours ferme.
\item[\vref{Ec 12:9}] retourne ds la t., com. elle y
\item[\vref{Es 11:9}] sainte, car la t. sera remplie de
\item[\vref{Es 14:7}] Toute la t. jouit du repos et de la
\item[\vref{Es 14:16}] faisait trembler la t., qui ébranlait les
\item[\vref{Es 24:20}] La t. chancelle, elle chancelle com. un hom.
\item[\vref{Es 42:5}] a aplani la t. avec ce qu'elle
\item[\vref{Es 45:18}] a formé la t., qui l'a faite
\item[\vref{Es 51:13}] et fondé la t. ! Et chaque jour
\item[\vref{Es 54:5}] appelé le Dieu de tte la t.
\item[\vref{Es 55:9}] au-dessus de la t., autant mes voies
\item[\vref{Es 61:11}] Car com. la t. fait éclore son
\item[\vref{Es 66:1}] trône, et la t. est le marchepied
\item[\vref{Es 66:22}] et la nouvelle t. que je vais
\item[\vref{Jé 4:23}] Je regarde la t., et voici, elle
\item[\vref{Jé 27:5}] J'ai fait la t., les hommes et
\item[\vref{Ez 11:17}] vs. donnerai la t. d'Israël.
\item[\vref{Os 2:24}] Et la t. répondra au blé, au bon vin
\item[\vref{Joë 2:10}] La t. tremble dvt eux, les cieux sont
\item[\vref{Am 1:1}] deux ans avant le tremblement de t.
\item[\vref{Am 9:5}] qui touche la t., et elle se
\item[\vref{Ha 2:14}] Car la t. sera remplie de la connaissance de
\item[\vref{Ag 1:10}] rosée, et la t. a retenu ses
\item[\vref{Za 12:1}] et fondé la t., et qui a
\item[\vref{Mt 5:18}] ciel et la t. ne passeront pas,
\item[\vref{Mt 13:8}] ds la bonne t. ; et elle donna
\item[\vref{Mt 24:14}] ds tte la t. habitée, pour servir
\item[\vref{Mt 27:54}] le tremblement de t. et tt ce
\item[\vref{Mc 13:8}] des tremblements de t. en divers lieux,
\item[\vref{Jn 8:6}] écrivait avec son doigt sur la t.
\item[\vref{Jn 9:6}] il cracha à t. et fit de
\item[\vref{Jn 12:24}] est tombé en t. ne meurt, il
\item[\vref{Ro 10:18}] par tte la t., et lr. parole
\item[\vref{1 Co 10:26}] car la t., avec tt ce qu'elle contient, est
\item[\vref{Ep 4:9}] parties les plus basses de la t. ?
\item[\vref{Ph 3:19}] que pour les choses de la t.
\item[\vref{Col 1:16}] et sur la t., les visibles et
\item[\vref{Hé 1:10}] as fondé la t. dès le commencement,
\item[\vref{Hé 6:7}] Car la t. qui est abreuvée par la pluie
\item[\vref{Hé 11:13}] étaient étrangers et voyageurs sur la t.
\item[\vref{Hé 12:26}] ébranla alors la t., mais à l'égard
\item[\vref{2 Pi 3:5}] et que la t. est sortie de
\item[\vref{2 Pi 3:7}] cieux et la t. d'à présent sont
\item[\vref{2 Pi 3:13}] et une nouvelle t., où la justice
\item[\vref{Ap 6:12}] grand tremblement de t., et le soleil
\item[\vref{Ap 7:2}] mal à la t. et à la
\item[\vref{Ap 11:6}] de frapper la t. de ttes sortes
\item[\vref{Ap 12:16}] Mais la t. secourut la fem., elle ouvrit sa
\item[\vref{Ap 13:12}] elle obligeait la t. et ses habitants
\item[\vref{Ap 16:1}] versez sur la t. les coupes de
\item[\vref{Ap 16:18}] grand tremblement de t., dis-je, tel qu'il
\item[\vref{Ap 18:1}] autorité, et la t. fut illuminée de
\item[\vref{Ap 19:2}] a corrompu la t. par son impudicité,
\item[\vref{Ap 20:11}] assis dessus. La t. et le ciel
\item[\vref{Ap 21:1}] et une nouvelle t. ; car le premier
\end{listverse}

\ConcordanceEntry{Terrestre}
\vspace{-2mm}
\begin{listverse}
\item[\vref{Jn 3:12}] parlé des choses t., et que vs.
\item[\vref{1 Co 15:40}] et des corps t. ; mais autre est
\item[\vref{2 Co 5:1}] si notre habitation t., qui n'est qu'une
\item[\vref{Hé 9:1}] le service divin, et un sanctuaire t.
\item[\vref{Ja 3:15}] c'est une sagesse t., animale et diabolique.
\end{listverse}

\ConcordanceEntry{Terreur}
\vspace{-2mm}
\begin{listverse}
\item[\vref{Ge 35:5}] Dieu frappa de t. les villes qui
\item[\vref{Ex 23:27}] J'enverrai la t. de mon Nom
\item[\vref{Jos 2:9}] et que la t. de votre nom
\item[\vref{1 S 5:9}] une très grande t. ; et il frappa
\item[\vref{Est 9:3}] cause de la t. que lr. inspirait
\item[\vref{Job 25:2}] règne et la t. appartiennent à Dieu ;
\item[\vref{Job 33:7}] Voici ma t. ne te troublera pas, et ma
\item[\vref{Ps 91:5}] craindras ni les t. de la nuit,
\item[\vref{Jé 8:15}] temps de guérison, et voici la t. !
\item[\vref{Ez 32:32}] je mettrai ma t. ds la terre
\end{listverse}

\ConcordanceEntry{Terrible}
\vspace{-2mm}
\begin{listverse}
\item[\vref{De 7:21}] Dieu grand et t. est au milieu
\item[\vref{De 34:12}] et ts ces t. prodiges, que Moïse
\item[\vref{2 S 7:23}] choses grandes et t. contre les nations
\item[\vref{Ps 45:5}] et ta droite lancera des choses t. !
\item[\vref{Ps 47:3}] le Très-Haut, est t.. Il est un
\item[\vref{Ps 99:3}] Nom, grand et t., car il est
\item[\vref{Es 21:2}] Une vision t. m'a été révélée.
\item[\vref{Ez 30:11}] lui, les plus t. d'entre les nations,
\item[\vref{Ez 32:12}] sont les plus t. d'entre les nations ;
\item[\vref{Da 7:7}] une quatrième bête, t., épouvantable et extraordinairement
\item[\vref{Joë 2:31}] le grand et t. jour de Yahweh
\item[\vref{Ha 1:7}] est redoutable et t., son gouvernement et
\item[\vref{So 2:11}] Yahweh sera t. contre eux, car
\item[\vref{Lu 21:11}] aura des choses t., et de grands
\item[\vref{Hé 10:31}] C'est une chose t. que de tomber
\item[\vref{Hé 12:21}] spectacle était si t. que Moïse dit :
\end{listverse}

\ConcordanceEntry{Tesson}
\vspace{-2mm}
\begin{listverse}
\item[\vref{Job 2:8}] Job prit un t. pour se gratter
\item[\vref{Es 30:14}] trouve pas un t. pour prendre du
\end{listverse}

\ConcordanceEntry{Testament}
\vspace{-2mm}
\begin{listverse}
\item[\vref{Ga 3:15}] des hommes, un t. en bonne forme,
\item[\vref{Hé 9:16}] y a un t., il est nécessaire
\item[\vref{Hé 9:17}] du testateur qu'un t. est valable, puisqu'il
\end{listverse}

\ConcordanceEntry{Tête}
\vspace{-2mm}
\begin{listverse}
\item[\vref{Ge 3:15}] celle-ci t'écrasera la t., et tu lui
\item[\vref{No 6:18}] naziréen rasera la t. de son naziréat
\item[\vref{Jg 16:22}] cheveux de sa t. commencèrent à repousser,
\item[\vref{1 S 1:11}] aucun rasoir ne passera sur sa t.
\item[\vref{1 S 17:51}] lui coupa la t.. Les Philistins, voyant
\item[\vref{1 S 31:9}] Ils coupèrent la t. de Saül et
\item[\vref{2 S 4:8}] Ils apportèrent la t. d'Isch-Boscheth à David
\item[\vref{2 R 4:19}] son père : Ma t. ! Ma tête ! Et
\item[\vref{2 R 6:25}] l'assiégèrent tellement qu'une t. d'âne se vendait
\item[\vref{Job 1:20}] et rasa sa t. ; et se jetant
\item[\vref{Ps 66:12}] hommes sur notre t., et ns. avons
\item[\vref{Ps 68:22}] Dieu écrasera la t. de ses ennemis,
\item[\vref{Ps 75:6}] si haut votre t., et ne parlez
\item[\vref{Ps 110:7}] C'est pourquoi il lève haut la t.
\item[\vref{Jé 9:1}] Dieu que ma t. soit com. un
\item[\vref{Ez 29:18}] contre Tyr ; tte t. en est devenue
\item[\vref{Ez 33:4}] prendre, son sang sera sur sa t.
\item[\vref{Da 2:38}] ts ; c'est toi qui es la t. d'or.
\item[\vref{Jon 4:8}] soleil frappa la t. de Jonas, au
\item[\vref{Mt 5:36}] plus par ta t., car tu ne
\item[\vref{Mt 14:8}] un plat, la t. de Jean-Baptiste.
\item[\vref{Mt 26:7}] elle répandit le parfum sur sa t.
\item[\vref{Mt 27:30}] le roseau et frappaient sur sa t.
\item[\vref{Mt 27:37}] au-dessus de sa t. un écriteau, où
\item[\vref{Lu 12:7}] cheveux de votre t. sont ts comptés.
\item[\vref{Lu 18:5}] vienne sans cesse me casser la t.
\item[\vref{Lu 21:18}] perdra pas un cheveu de votre t.
\item[\vref{Jn 19:30}] ayant baissé la t., il rendit l'esprit.
\item[\vref{Ac 18:18}] fait raser la t. à Cenchrées, car
\item[\vref{1 Co 11:4}] chose sur la t., déshonore son chef.
\item[\vref{1 Co 12:21}] toi. Ni la t. dire aux pieds :
\item[\vref{Ap 1:14}] Sa t. et ses cheveux étaient blancs com.
\item[\vref{Ap 12:3}] feu ayant sept t. et dix cornes,
\item[\vref{Ap 13:3}] l'une de ses t. com. blessée à
\item[\vref{Ap 19:12}] avait sur sa t. plusieurs diadèmes, et
\end{listverse}

\ConcordanceEntry{Tétrarque}
\vspace{-2mm}
\begin{listverse}
\item[\vref{Mt 14:1}] temps-là, Hérode, le t., entendit parler de
\item[\vref{Lu 3:1}] la Judée, Hérode, t. de la Galilée,
\end{listverse}

\ConcordanceEntry{Thabor}
\vspace{-2mm}
\begin{listverse}
\item[\vref{Jg 4:6}] la montagne de T. et prends avec
\item[\vref{Jg 4:14}] la montagne de T., ayant dix mille
\item[\vref{Jg 8:18}] avez tués à T. ? Ils répondirent : Ils
\item[\vref{Jé 46:18}] armées ; com. le T. entre les montagnes,
\item[\vref{Os 5:1}] et un filet tendu sur le T.
\end{listverse}

\ConcordanceEntry{Thaddée, Jude}
\vspace{-2mm}
\begin{listverse}
\item[\vref{Mt 10:3}] Jacques, fils d'Alphée, et Lebbée, surnommé T. ;
\item[\vref{Lu 6:16}] J., frère de Jacques, et Judas Iscariot,
\end{listverse}

\ConcordanceEntry{Thammuz}
\vspace{-2mm}
\begin{listverse}
\item[\vref{Ez 8:14}] là des femmes assises qui pleuraient T.
\end{listverse}

\ConcordanceEntry{Tharthan}
\vspace{-2mm}
\begin{listverse}
\item[\vref{2 R 18:17}] le roi Ezéchias, T., Rab-Saris et Rabschaké
\item[\vref{Es 20:1}] L'année où T., envoyé par Sargon,
\end{listverse}

\ConcordanceEntry{Thébets}
\vspace{-2mm}
\begin{listverse}
\item[\vref{Jg 9:50}] Abimélec marcha contre T., y mit son
\item[\vref{2 S 11:21}] pas mort à T. ? Pourquoi vs. êtes-vs.
\end{listverse}

\ConcordanceEntry{Théman}
\vspace{-2mm}
\begin{listverse}
\item[\vref{Ge 36:11}] fils d'Eliphaz furent : T., Omar, Tsepho, Gaetham
\item[\vref{Job 2:11}] Job, Eliphaz de T., Bildad de Schuach,
\item[\vref{Jé 49:7}] de sagesse ds T. ? Le conseil a-t-il
\item[\vref{Ab 1:9}] seront effrayés, ô T. ! afin qu'ils soient
\item[\vref{Ha 3:3}] Dieu vient de T., et le Saint
\end{listverse}

\ConcordanceEntry{Théophile}
\vspace{-2mm}
\begin{listverse}
\item[\vref{Lu 1:3}] fin, très excellent T., de te les
\item[\vref{Ac 1:1}] premier traité, ô T. ! De ttes les
\end{listverse}

\ConcordanceEntry{Théraphim}
\vspace{-2mm}
\begin{listverse}
\item[\vref{Ge 31:19}] Rachel déroba les t. de son père.
\item[\vref{Jg 18:14}] un éphod, des t., une image taillée
\item[\vref{1 S 15:23}] l'idolâtrie et les t.. Puisque tu as
\item[\vref{1 S 19:13}] Mical prit un t., qu'elle plaça ds
\item[\vref{2 R 23:24}] les devins, les t., les idoles, et
\item[\vref{Ez 21:26}] il interroge les t., il examine le
\item[\vref{Os 3:4}] sans statue, sans éphod, et sans t.
\item[\vref{Za 10:2}] Car les t. ont des paroles vaines, et les
\end{listverse}

\ConcordanceEntry{Thessalonique}
\vspace{-2mm}
\begin{listverse}
\item[\vref{Ac 17:1}] ils arrivèrent à T., où les Juifs
\item[\vref{Ac 27:2}] un Macédonien de la ville de T.
\item[\vref{Ph 4:16}] lorsque j'étais à T., vs. m'avez envoyé
\item[\vref{2 Ti 4:10}] est allé à T. ; Crescens est allé
\end{listverse}

\ConcordanceEntry{Thola}
\vspace{-2mm}
\begin{listverse}
\item[\vref{Jg 10:1}] Après Abimélec, T. fils de Pua,
\end{listverse}

\ConcordanceEntry{Thomas}
\vspace{-2mm}
\begin{listverse}
\item[\vref{Mt 10:3}] Philippe, et Barthélemy ; T., et Matthieu, le
\end{listverse}
\begin{legend}
\NoAutoSpaceBeforeFDP{
\item Apôtre de Jésus : Mt 10:3; Jn 11:16
\item Voit pour croire : Jn 20:24-29
}
\end{legend}

\ConcordanceEntry{Thummim, Urim}
\vspace{-2mm}
\begin{listverse}
\item[\vref{Ex 28:30}] pectoral de jugement l'u. et le thummim,
\item[\vref{No 27:21}] les jugements de l'u. dvt Yahweh ; et
\item[\vref{De 33:8}] concernant Lévi : Tes t. et tes urim
\item[\vref{1 S 28:6}] songes, ni par l'u., ni par les
\item[\vref{Esd 2:63}] prêtre ait consulté l'u. et le thummim.
\item[\vref{Né 7:65}] prêtre eût consulté l'u. et le thummim.
\end{listverse}

\ConcordanceEntry{Thyatire}
\vspace{-2mm}
\begin{listverse}
\item[\vref{Ac 16:14}] la ville de T., était une fem.
\item[\vref{Ap 1:11}] à Pergame, à T., à Sardes, à
\item[\vref{Ap 2:18}] de l'église de T. : Voici ce que
\end{listverse}

\ConcordanceEntry{Tiare}
\vspace{-2mm}
\begin{listverse}
\item[\vref{Ex 28:37}] pourpre sur la t., sur le dvt
\item[\vref{Ex 39:28}] Et la t. de fin lin, et les ornements
\item[\vref{Lé 8:9}] mit aussi la t. sur la tête,
\item[\vref{Lé 16:4}] tête de la t. de lin, qui
\item[\vref{Es 3:23}] chemises fines, les t. et les voiles
\item[\vref{Ez 21:31}] Qu'on ôte cette t., et qu'on enlève
\item[\vref{Da 3:21}] leurs chaussures, leurs t., et leurs vêtements,
\end{listverse}

\ConcordanceEntry{Tibériade}
\vspace{-2mm}
\begin{listverse}
\item[\vref{Jn 6:1}] Galilée, qui est la Mer de T.
\item[\vref{Jn 6:23}] étaient arrivées de T. près du lieu
\item[\vref{Jn 21:1}] la Mer de T.. Et il s'y
\end{listverse}

\ConcordanceEntry{Tiède}
\vspace{-2mm}
\begin{listverse}
\item[\vref{Ap 3:16}] que tu es t., et que tu
\end{listverse}

\ConcordanceEntry{Tiers}
\vspace{-2mm}
\begin{listverse}
\item[\vref{No 15:6}] pétrie ds un t. de hin d'huile,
\item[\vref{Né 10:32}] chaque année le t. d'un sicle, pour
\item[\vref{Ez 5:2}] Brûles-en un t. ds le feu,
\item[\vref{Ez 5:12}] Un t. d'entre vs. mourra de la peste,
\item[\vref{Za 13:8}] que les deux t. seront retranchés et
\item[\vref{Za 13:9}] Je mettrai ce t. ds le feu,
\item[\vref{Ap 8:9}] et le t. des créatures vivantes qui étaient ds
\item[\vref{Ap 8:12}] trompette, et le t. du soleil fut
\item[\vref{Ap 9:15}] de tuer le t. des hommes.
\item[\vref{Ap 9:18}] Le t. des hommes fut tué par ces
\item[\vref{Ap 12:4}] queue entraînait le t. des étoiles du
\end{listverse}

\ConcordanceEntry{Tiglath-Piléser, Tilgath-Pilnéser}
\vspace{-2mm}
\begin{listverse}
\item[\vref{2 R 16:10}] la rencontre de T., roi d'Assyrie, à
\end{listverse}
\begin{legend}
\NoAutoSpaceBeforeFDP{
\item Roi d'Assyrie : 2 R 16:10
\item Trahison de T-P à l'égard d'Achaz : 2 Ch 28:16-21
\item Invasion d'Israël par T-P : 2 R15:29; 1 Ch 5:6
}
\end{legend}

\ConcordanceEntry{Timidité}
\vspace{-2mm}
\begin{listverse}
\item[\vref{2 Ti 1:7}] un esprit de t., mais de force,
\end{listverse}

\ConcordanceEntry{Timothée}
\vspace{-2mm}
\begin{listverse}
\item[\vref{Ac 16:1}] un disciple nommé T., fils d'une fem.
\end{listverse}
\begin{legend}
\NoAutoSpaceBeforeFDP{
\item Disciple de Jésus et collaborateur de Paul : Ac 16:1-3; 17:14; 19:22; Ro 16:21; Ph 2:19
\item Instructions de Paul : 1 Tim 1:2-3; 3:14-15
\item Autres : 1 Co 16:10; 1 Th 3:2-6; Hé13:23
}
\end{legend}

\ConcordanceEntry{Tison}
\vspace{-2mm}
\begin{listverse}
\item[\vref{Es 7:4}] queues de ces t. fumants, à cause
\item[\vref{Am 4:11}] été com. un t. arraché du feu,
\item[\vref{Za 3:2}] pas là un t. qui a été
\end{listverse}

\ConcordanceEntry{Tite}
\vspace{-2mm}
\begin{listverse}
\item[\vref{2 Co 2:12}] n'ai pas trouvé T., mon frère,
\item[\vref{2 Co 7:6}] ns. a consolés par l'arrivée de T.,
\item[\vref{2 Co 8:16}] le cœur de T. le mm empressement
\item[\vref{2 Co 8:23}] dc, quant à T., il est mon
\item[\vref{Ga 2:1}] et je pris aussi avec moi T.
\item[\vref{2 Ti 4:10}] en Galatie ; et T. en Dalmatie.
\item[\vref{Tit 1:4}] à T., mon vrai fils, selon la foi
\end{listverse}

\ConcordanceEntry{Tobija}
\vspace{-2mm}
\begin{listverse}
\item[\vref{Né 2:10}] le Horonite, et T., le serviteur Ammonite,
\item[\vref{Né 4:3}] Et T., l'Ammonite, qui était auprès de lui,
\item[\vref{Né 6:12}] que Sanballat et T. lui avaient donné
\item[\vref{Né 13:4}] de notre Dieu, s'était allié à T. ;
\end{listverse}

\ConcordanceEntry{Toile}
\vspace{-2mm}
\begin{listverse}
\item[\vref{Job 8:14}] est com. une t. d'araignée.
\item[\vref{Es 38:12}] retranché com. la t. que le tisserand
\item[\vref{Es 59:5}] ils tissent des t. d'araignée ; celui qui
\end{listverse}

\ConcordanceEntry{Toison}
\vspace{-2mm}
\begin{listverse}
\item[\vref{De 18:4}] prémices de la t. de tes brebis.
\item[\vref{Jg 6:37}] vais mettre une t. de laine ds
\end{listverse}

\ConcordanceEntry{Toit}
\vspace{-2mm}
\begin{listverse}
\item[\vref{Ge 19:8}] sont venus à l'ombre de mon t.
\item[\vref{De 22:8}] autour de ton t., afin que tu
\item[\vref{Jg 9:51}] montèrent sur le t. de la tour.
\item[\vref{Jg 16:27}] mm sur le t. près de trois
\item[\vref{2 S 11:2}] promenait sur le t. de la maison
\item[\vref{Pr 21:9}] au coin d'un t., que ds une
\item[\vref{Pr 25:24}] à l'angle d'un t. que de partager
\item[\vref{Jé 19:13}] maisons sur les t. desquelles ils brûlaient
\item[\vref{Ez 40:13}] portail depuis le t. d'une chambre jusqu'au
\item[\vref{So 1:5}] prosternent sur les t. dvt l'armée des
\item[\vref{Mt 8:8}] entres sous mon t. ; mais dis seulement
\item[\vref{Mt 10:27}] dis à l'oreille, prêchez-le sur les t.
\item[\vref{Mt 24:17}] sera sur le t. ne descende pas
\item[\vref{Mc 2:4}] ils découvrirent le t. du lieu où
\item[\vref{Lu 17:31}] sera sur le t., et qui aura
\item[\vref{Ac 10:9}] monta sur le t., vers la sixième
\end{listverse}

\ConcordanceEntry{Tombeau}
\vspace{-2mm}
\begin{listverse}
\item[\vref{Job 10:19}] du ventre de ma mère au t.
\item[\vref{Job 21:32}] sépulcre, et il demeurera ds le t.
\item[\vref{Ps 88:12}] sépulcre, de ta fidélité ds le t. ?
\item[\vref{Ps 107:20}] et il les délivre de leurs t.
\item[\vref{Mt 23:29}] vs. bâtissez les t. des prophètes et
\end{listverse}

\ConcordanceEntry{Tomber}
\vspace{-2mm}
\begin{listverse}
\item[\vref{Ge 7:12}] Et la pluie t. sur la terre
\item[\vref{Ge 17:3}] Alors Abram t. sur sa face,
\item[\vref{Ge 49:17}] que le cavalier t. à la renverse.
\item[\vref{Lé 16:10}] le sort sera t. pour être Azazel,
\item[\vref{Lé 26:8}] et vos ennemis t. par l'épée dvt
\item[\vref{Lé 26:36}] l'épée, et ils t. sans que personne
\item[\vref{Jg 20:44}] Il t. dix-huit mille hommes de Benjamin, ts
\item[\vref{1 S 3:19}] ne laissa pas t. à terre une
\item[\vref{1 S 4:18}] de Dieu, Eli t. à la renverse,
\item[\vref{1 S 5:4}] voici, Dagon était t. le visage contre
\item[\vref{Job 33:15}] qnd les hommes t. ds un profond
\item[\vref{Job 37:6}] à la neige : T. sur la terre !
\item[\vref{Ps 7:16}] creuse, et il t. ds la fosse
\item[\vref{Ps 9:16}] Les nations t. ds la fosse
\item[\vref{Ps 20:9}] plient, et ils t. ; ns., ns. tenons
\item[\vref{Ps 37:14}] arc, pour faire t. le malheureux et
\item[\vref{Ps 91:7}] Que mille t. à ton côté,
\item[\vref{Ps 106:26}] de les faire t. ds le désert,
\item[\vref{Ps 118:13}] pour me faire t., mais Yahweh m'a
\item[\vref{Pr 4:19}] n'aperçoivent pas ce qui les fera t.
\item[\vref{Pr 7:26}] elle a fait t. plusieurs blessés à
\item[\vref{Pr 11:14}] Le peuple t. par faute de
\item[\vref{Pr 11:28}] ds ses richesses t., mais les justes
\item[\vref{Pr 24:16}] Car le juste t. sept fois, et
\item[\vref{Pr 24:17}] Quand ton ennemi t., ne t'en réjouis
\item[\vref{Pr 28:10}] le mauvais chemin t. ds la fosse
\item[\vref{Ec 4:10}] l'un des deux t., l'autre relèvera son
\item[\vref{Ec 11:3}] qnd un arbre t., au sud ou
\item[\vref{Es 1:30}] dont le feuillage t., et com. un
\item[\vref{Es 14:12}] Comment es-tu t. du ciel, astre
\item[\vref{Es 21:9}] dit : Elle est t., elle est tombée,
\item[\vref{Es 40:7}] et la fleur t., parce que le
\item[\vref{Es 54:15}] complotera contre toi t. pour l'amour de
\item[\vref{Es 57:16}] car dvt moi t. en défaillance les
\item[\vref{Jé 8:4}] Yahweh : Si on t., ne se relève-t-on
\item[\vref{La 2:11}] des nourrissons qui t. en défaillance ds
\item[\vref{Ez 1:28}] sa vue, je t. sur ma face,
\item[\vref{Da 3:23}] Méschac, et Abed-Négo, t. ts liés au
\item[\vref{Da 11:26}] beaucoup de gens t. blessés à mort.
\item[\vref{Os 14:1}] car tu es t. par ton iniquité.
\item[\vref{Jon 1:7}] et le sort t. sur Jonas.
\item[\vref{Mi 7:8}] si je suis t., je me relèverai ;
\item[\vref{Za 14:12}] chair de chacun t. en pourriture tandis
\item[\vref{Mt 10:29}] Cependant, il n'en t. pas un à
\item[\vref{Mt 13:4}] de la semence t. le long du
\item[\vref{Mt 13:5}] une autre partie t. ds les endroits
\item[\vref{Mt 15:14}] autre aveugle, ils t. ts deux ds
\item[\vref{Mt 15:27}] des miettes qui t. de la table
\item[\vref{Mt 21:44}] Celui qui t. sur cette pierre
\item[\vref{Lu 10:18}] Je voyais Satan t. du ciel com.
\item[\vref{Ac 1:18}] donné, il est t., s'est rompu par
\item[\vref{Ro 11:11}] bronché afin de t. ? Nullement ! Mais, par
\item[\vref{1 Co 10:8}] sorte qu'il en t. vingt-trois mille en
\item[\vref{1 Co 10:12}] demeurer debout prenne garde qu'il ne t. !
\item[\vref{1 Ti 3:6}] d'orgueil, il ne t. sous le jugement
\item[\vref{Hé 4:11}] que quelqu'un ne t. en imitant une
\item[\vref{Hé 11:30}] murs de Jéricho t., après qu'on en
\item[\vref{Ja 5:17}] et il ne t. pas de pluie
\item[\vref{Ap 1:17}] le vis, je t. à ses pieds
\item[\vref{Ap 2:5}] d'où tu es t., repens-toi, et fais
\item[\vref{Ap 6:13}] étoiles du ciel t. sur la terre,
\item[\vref{Ap 6:16}] et aux rochers : T. sur ns., et
\item[\vref{Ap 9:1}] une étoile qui t. du ciel sur
\item[\vref{Ap 14:8}] disant : Elle est t., elle est tombée
\item[\vref{Ap 17:10}] les cinq sont t. ; l'un est, et
\item[\vref{Ap 22:8}] et vues, je t. à terre aux
\end{listverse}

\ConcordanceEntry{Tonnerre}
\vspace{-2mm}
\begin{listverse}
\item[\vref{Ex 9:23}] Yahweh envoya des t. et de la
\item[\vref{Ex 19:16}] y eut des t., et des éclairs,
\item[\vref{1 S 2:10}] il lancera son t. sur chacun d'eux ;
\item[\vref{1 S 7:10}] jour-là, un grand t. sur les Philistins,
\item[\vref{1 S 12:17}] il enverra des t. et de la
\item[\vref{Job 28:26}] un chemin à l'éclair et au t.,
\item[\vref{Job 38:25}] la route de l'éclair et du t.,
\item[\vref{Ps 77:19}] voix de ton t. était ds le
\item[\vref{Ps 81:8}] lieu caché du t. ; je t'ai éprouvé
\item[\vref{Ps 104:7}] en fuite au son de ton t.
\item[\vref{Es 29:6}] armées avec des t., des tremblements de
\item[\vref{Mc 3:17}] ce qui veut dire fils de t. ;
\item[\vref{Jn 12:29}] un coup de t. ; les autres disaient :
\item[\vref{Ap 6:1}] une voix de t. : Viens, et vois.
\item[\vref{Ap 10:3}] crié, les sept t. firent entendre leurs
\item[\vref{Ap 19:6}] l'éclat de grands t., disant : Alléluia ! Car
\end{listverse}

\ConcordanceEntry{Topheth}
\vspace{-2mm}
\begin{listverse}
\item[\vref{2 R 23:10}] roi profana aussi T., ds la vallée
\item[\vref{Jé 7:31}] hauts lieux de T., qui est ds
\item[\vref{Jé 7:32}] sera plus appelée T., ni la vallée
\item[\vref{Jé 19:11}] seront enterrés à T. parce qu'il n'y
\end{listverse}

\ConcordanceEntry{Torrent}
\vspace{-2mm}
\begin{listverse}
\item[\vref{Ge 32:23}] fit passer le t. ; il fit aussi
\item[\vref{Ps 110:7}] Il boit au t. pendant la marche :
\item[\vref{Ps 124:4}] auraient submergés, les t. auraient passé sur
\item[\vref{Jé 31:9}] les conduirai aux t. d'eaux, et par
\item[\vref{Ez 47:5}] étaient déjà un t. que je ne
\item[\vref{Ha 3:10}] elles tremblent ; des t. d'eau se précipitent,
\item[\vref{Mt 7:25}] est tombée, les t. sont venus, les
\item[\vref{Jn 18:1}] disciples au-delà du t. de Cédron, où
\end{listverse}

\ConcordanceEntry{Tort}
\vspace{-2mm}
\begin{listverse}
\item[\vref{Ex 2:13}] celui qui avait t. : Pourquoi frappes-tu ton
\item[\vref{Ps 69:5}] mes ennemis à t., se sont renforcés.
\item[\vref{Pr 14:31}] Celui qui fait t. au pauvre déshonore
\item[\vref{Ec 5:7}] Province qu'on fasse t. au pauvre, et
\item[\vref{Es 11:9}] se fera ni t. ni dommage sur
\item[\vref{Mal 3:5}] l'orphelin, qui font t. à l'étranger, et
\item[\vref{Mt 20:13}] fais pas de t. ; n'es-tu pas tombé
\item[\vref{Mc 10:19}] ne fais aucun t. à personne ; honore
\item[\vref{Lu 19:8}] si j'ai fait t. de qq chose
\item[\vref{Ac 25:10}] n'ai fait aucun t. aux Juifs, com.
\item[\vref{1 Co 6:7}] vs. fasse du t. ? Pourquoi ne souffrez-vs.
\item[\vref{2 Co 7:2}] Nous n'avons fait t. à personne, ns.
\item[\vref{2 Co 12:13}] été à votre charge ? Pardonnez-moi ce t. !
\item[\vref{Ga 4:13}] m'avez fait aucun t.. Et vs. savez
\item[\vref{Phm 1:18}] t'a fait qq t., ou s'il te
\end{listverse}

\ConcordanceEntry{Tortueux}
\vspace{-2mm}
\begin{listverse}
\item[\vref{Pr 2:15}] les sentiers sont t., et qui ds
\item[\vref{Es 27:1}] dis-je, le serpent t., et il tuera
\item[\vref{Es 40:4}] et les lieux t. seront redressés, et
\item[\vref{Es 45:3}] j'aplanirai les lieux t. ; je romprai les
\item[\vref{Lu 3:5}] ce qui est t. sera redressé, et
\end{listverse}

\ConcordanceEntry{Toucher}
\vspace{-2mm}
\begin{listverse}
\item[\vref{Ge 3:3}] vs. ne le t. point, de peur
\item[\vref{Ex 19:12}] montagne et de t. aucune de ses
\item[\vref{Ex 29:37}] tt ce qui t. l'autel sera saint.
\item[\vref{Ex 30:29}] ce qui les t. sera saint.
\item[\vref{Lé 5:2}] son insu, aura t. une chose souillée,
\item[\vref{Lé 5:3}] qnd il aura t. à l'impureté d'un
\item[\vref{Lé 6:11}] Yahweh. Quiconque les t. sera sanctifié.
\item[\vref{No 19:22}] que l'hom. impur t. sera souillé, et
\item[\vref{Jg 16:26}] que je puisse t. les colonnes sur
\item[\vref{1 S 10:26}] des vaillants hommes dont Dieu avait t. le cœur.
\item[\vref{1 R 19:5}] un ange le t. et lui dit :
\item[\vref{2 R 13:21}] d'Elisée. L'hom. alla t. les os d'Elisée,
\item[\vref{Job 2:6}] main : Seulement ne t. pas à sa
\item[\vref{Ps 105:15}] disant : Ne t. pas à mes
\item[\vref{Pr 6:29}] prochain ; quiconque la t. ne restera pas
\item[\vref{Es 6:7}] Il en t. ma bouche, et dit : Voici, ceci
\item[\vref{Es 52:11}] de là ! Ne t. rien d'impur ! Sortez
\item[\vref{Jé 1:9}] sa main, et t. ma bouche ; et
\item[\vref{Da 8:18}] terre. Il me t., et me fit
\item[\vref{Za 2:8}] celui qui vs. t., touche à la
\item[\vref{Mt 8:15}] Il t. sa main, et la fièvre la
\item[\vref{Mt 9:21}] je puis seulement t. son vêt., je
\item[\vref{Mt 14:36}] lr. permettre de t. seulement le bord
\item[\vref{Mt 20:34}] ému de compassion, t. leurs yeux, et
\item[\vref{Lu 6:19}] cherchait à le t., parce qu'une force
\item[\vref{Lu 7:14}] Il s'approcha, et t. le cercueil. Et
\item[\vref{Lu 8:45}] dit : Qui m'a t. ? Comme ts le
\item[\vref{Lu 24:39}] c'est bien moi. T.-moi, et voyez :
\item[\vref{Jn 20:17}] dit : Ne me t. pas ; car je
\item[\vref{Ac 2:37}] eurent le cœur t. de componction, et
\item[\vref{Ac 19:12}] linges qui avaient t. son corps, et
\item[\vref{1 Co 7:1}] de ne pas t. de fem.
\item[\vref{2 Co 6:17}] le Seign. ; ne t. à aucune chose
\item[\vref{Col 2:21}] prends pas ! ne goûte pas ! ne t. pas !
\item[\vref{Hé 12:18}] montagne qu'on pouvait t. avec la main,
\item[\vref{1 Jn 1:1}] propres mains ont t. concernant la Parole
\item[\vref{1 Jn 5:18}] lui-mm, et le malin ne le t. point.
\end{listverse}

\ConcordanceEntry{Toujours}
\vspace{-2mm}
\begin{listverse}
\item[\vref{Ge 6:3}] contestera point à t. avec les hommes
\item[\vref{Ex 17:16}] Yahweh, Yahweh aura t. la guerre contre
\item[\vref{De 11:1}] et tu garderas t. ses lois, ses
\item[\vref{1 Ch 23:25}] établira sa demeure ds Jérus. à t.
\item[\vref{2 Ch 5:13}] miséricorde demeure à t. ! Il arriva que
\item[\vref{Ps 10:16}] est Roi à t. et à perpétuité ;
\item[\vref{Ps 18:51}] David, et à sa postérité, pour t.
\item[\vref{Ps 73:23}] Je serai dc t. avec toi ; tu
\item[\vref{Ps 103:9}] garde point à t. sa colère.
\item[\vref{Jé 3:12}] ne garde pas ma colère à t.
\item[\vref{Da 12:3}] les étoiles, à t. et à perpétuité.
\item[\vref{Mt 26:11}] car vs. aurez t. des pauvres avec
\item[\vref{Lu 15:31}] enfant, tu es t. avec moi, et
\item[\vref{Jn 6:34}] dc : Seign., donne-ns. t. ce pain-là.
\item[\vref{Jn 8:29}] que je fais t. les choses qui
\item[\vref{Jn 8:35}] ne demeure pas t. ds la maison ;
\item[\vref{Jn 11:42}] que tu m'exauces t. ; mais je l'ai
\item[\vref{Ac 7:51}] vs. vs. obstinez t. contre le Saint-Esprit ;
\item[\vref{2 Co 6:10}] attristés et toutefois t. joyeux ; com. pauvres
\item[\vref{Ph 4:4}] Réjouissez-vs. t. ds le Seign. ;
\item[\vref{Col 4:6}] votre parole soit t. assaisonnée de sel,
\item[\vref{1 Th 4:17}] ainsi ns. serons t. avec le Seign.
\item[\vref{1 Th 5:16}] Soyez t. joyeux.
\item[\vref{Tit 1:12}] Les Crétois sont t. menteurs, de mauvaises
\item[\vref{Hé 7:25}] par lui, étant t. vivant pour intercéder
\item[\vref{Hé 7:28}] le Fils, qui est parfait pour t.
\item[\vref{Hé 10:12}] s'est assis pour t. à la droite
\item[\vref{2 Pi 1:15}] départ, vs. puissiez t. vs. souvenir de
\item[\vref{Jud 1:16}] qui se plaignent t., qui marchent selon
\end{listverse}

\ConcordanceEntry{Tour (la)}
\vspace{-2mm}
\begin{listverse}
\item[\vref{Ge 11:4}] ville, et une t. dont le sommet
\item[\vref{Ps 61:4}] mon refuge, une t. forte au-dvt de
\item[\vref{Pr 18:10}] Yahweh est une t. forte, le juste
\item[\vref{Ca 4:4}] est com. la t. de David, bâtie
\item[\vref{Ca 7:5}] est com. une t. d'ivoire, tes yeux
\item[\vref{Es 5:2}] il bâtit une t. au milieu d'elle,
\item[\vref{Jé 31:38}] Yahweh, depuis la t. de Hananeel, jusqu'à
\item[\vref{Mt 21:33}] et bâtit une t. ; puis il l'afferma
\item[\vref{Lu 13:4}] est tombée la t. de Siloé et
\end{listverse}

\ConcordanceEntry{Tourment}
\vspace{-2mm}
\begin{listverse}
\item[\vref{Job 3:10}] pas caché le t. loin de mes
\item[\vref{Pr 22:8}] l'injustice moissonne le t., et la verge
\item[\vref{Ec 1:14}] est vanité et t. d'esprit.
\item[\vref{Ec 4:6}] avec travail et t. d'esprit.
\item[\vref{Mt 24:9}] vs. livreront aux t., et vs. tueront ;
\item[\vref{Lu 16:23}] était ds les t., il vit de
\item[\vref{1 Ti 6:10}] sont jetés eux-mêmes ds bien des t.
\item[\vref{Hé 10:29}] de combien pires t. pensez-vs. dc que
\item[\vref{Hé 11:35}] furent livrés aux t. et n'acceptèrent pas
\item[\vref{Ap 9:5}] mois ; et le t. qu'elles causaient était
\item[\vref{Ap 14:11}] fumée de lr. t. montera aux siècles
\item[\vref{Ap 18:7}] autant donnez-lui de t. et de deuil ;
\end{listverse}

\ConcordanceEntry{Tourmenter}
\vspace{-2mm}
\begin{listverse}
\item[\vref{Jg 14:17}] parce qu'elle le t.. Puis elle l'expliqua
\item[\vref{1 S 16:15}] mauvais esprit envoyé de Dieu te t.
\item[\vref{Ps 73:4}] rien ne les t. jusqu'à lr. mort,
\item[\vref{Ps 129:1}] Ils m'ont souvent t. dès ma jeunesse.
\item[\vref{Mt 4:24}] se portaient mal, t. de diverses maladies,
\item[\vref{Mt 8:29}] venu ici ns. t. avant le temps ?
\item[\vref{Mt 12:22}] amena un hom. t. d'un démon, aveugle
\item[\vref{Mt 15:22}] fille est cruellement t. par le démon.
\item[\vref{Lu 6:18}] aussi qui étaient t. par des esprits
\item[\vref{Ac 5:16}] ceux qui étaient t. par des esprits
\item[\vref{1 Ti 6:10}] sont jetés eux-mêmes ds bien des t.
\item[\vref{Hé 10:29}] de combien pires t. pensez-vs. dc que
\item[\vref{Hé 11:35}] furent livrés aux t. et n'acceptèrent pas
\item[\vref{Ap 9:5}] mais de les t. pendant cinq mois ;
\item[\vref{Ap 11:10}] deux prophètes ont t. les habitants de
\item[\vref{Ap 14:10}] et il sera t. ds le feu
\item[\vref{Ap 20:10}] Et ils seront t. jour et nuit,
\end{listverse}

\ConcordanceEntry{Tourner}
\vspace{-2mm}
\begin{listverse}
\item[\vref{Ge 3:24}] des chérubins qui t. ça et là
\item[\vref{Ge 9:23}] leurs visages étaient t. en arrière, de
\item[\vref{Ex 13:18}] Mais Dieu fit t. le peuple par
\item[\vref{Lé 19:31}] Ne vs. t. point vers ceux qui évoquent les
\item[\vref{Lé 20:6}] personne qui se t. vers ceux qui
\item[\vref{No 6:26}] Yahweh t. sa face vers toi, et te
\item[\vref{De 31:20}] puis il se t. vers d'autres dieux,
\item[\vref{Jos 24:23}] de vs., et t. votre cœur vers
\item[\vref{Ps 5:4}] matin je me t. vers toi, et
\item[\vref{Ps 18:41}] Tu fais t. le dos à mes ennemis dvt
\item[\vref{Pr 4:27}] Ne te t. ni à droite ni à gauche ;
\item[\vref{Ec 9:11}] Je me suis t. ailleurs, et j'ai
\item[\vref{Es 8:21}] son Dieu, et t. les yeux en
\item[\vref{Ez 1:12}] ils ne se t. point lorsqu'ils marchaient.
\item[\vref{Lu 10:23}] Puis, se t. vers ses disciples, il lr. dit
\item[\vref{Lu 23:28}] Mais Jésus se t. vers elles, lr.
\item[\vref{Ac 13:46}] voici, ns. ns. t. vers les Gentils.
\item[\vref{2 Ti 4:4}] vérité et se t. vers les fables.
\item[\vref{1 Pi 5:8}] diable, votre adversaire, t. autour de vs.
\end{listverse}

\ConcordanceEntry{Tradition}
\vspace{-2mm}
\begin{listverse}
\item[\vref{Mt 15:2}] disciples transgressent-ils la t. des anciens ? Car
\item[\vref{Mc 7:3}] conformément à la t. des anciens.
\item[\vref{Mc 7:9}] de Dieu, afin de garder votre t.
\item[\vref{Mc 7:13}] Dieu par votre t. que vs. avez
\item[\vref{Ga 1:14}] ardent zélateur des t. de mes pères.
\item[\vref{Col 2:8}] conformes à la t. des hommes et
\end{listverse}

\ConcordanceEntry{Trafic}
\vspace{-2mm}
\begin{listverse}
\item[\vref{Pr 3:14}] Car le t. qu'on peut faire d'elle est meilleur
\item[\vref{Pr 31:18}] sent que son t. est bon ; sa
\item[\vref{Es 23:18}] Mais son t. et son salaire seront sanctifiés à
\item[\vref{Es 45:14}] l'Egypte, et le t. de l'Ethiopie, et
\item[\vref{Ez 28:16}] grandeur de ton t., tu as été
\item[\vref{Mt 22:5}] son champ, et l'autre à son t.
\end{listverse}

\ConcordanceEntry{Trait}
\vspace{-2mm}
\begin{listverse}
\item[\vref{Ge 49:23}] ont lancé des t. ; les archers l'ont
\item[\vref{Ps 7:14}] sur lui des t. meurtriers, il rend
\item[\vref{Mt 5:18}] ou un seul t. de lettre jusqu'à
\item[\vref{Lu 16:17}] ne l'est qu'un t. de la lettre
\end{listverse}

\ConcordanceEntry{Traiter}
\vspace{-2mm}
\begin{listverse}
\item[\vref{Ge 12:13}] je sois bien t. à cause de
\item[\vref{Ge 48:20}] Que Dieu te t. com. Ephraïm et
\item[\vref{Ex 34:12}] Garde-toi de t. alliance avec les
\item[\vref{Lé 24:22}] mm jugement. Vous t. l'étranger com. celui
\item[\vref{1 S 25:14}] qui les a t. rudement.
\item[\vref{1 R 2:7}] Tu t. avec bienveillance les fils de Barzillaï,
\item[\vref{Ps 28:4}] T.-les selon leurs œuvres et selon
\item[\vref{Ps 103:10}] Il ne ns. t. pas selon nos
\item[\vref{Es 55:3}] vivra ; et je t. avec vs. une
\item[\vref{Es 61:8}] vérité et je t. avec eux une
\item[\vref{Mt 10:15}] de Gomorrhe seront t. moins rigoureusement au
\item[\vref{Mt 10:25}] au disciple d'être t. com. son maître,
\item[\vref{Lu 15:19}] appelé ton fils ; t.-moi com. l'un
\item[\vref{2 Co 6:13}] Or pour ns. t. de la mm
\item[\vref{Hé 8:8}] Seign., où je t. avec la maison
\item[\vref{1 Pi 3:7}] c'est-à-dire féminin ; les t. avec honneur com.
\end{listverse}

\ConcordanceEntry{Traître}
\vspace{-2mm}
\begin{listverse}
\item[\vref{Es 21:2}] été révélée. Le t. demeure traître, celui
\item[\vref{Lu 6:16}] Jacques, et Judas Iscariot, qui devint t.
\item[\vref{Ac 7:52}] avez été les t. et les meurtriers,
\item[\vref{2 Ti 3:4}] t., emportés, enflés d'orgueil, amis des voluptés
\end{listverse}

\ConcordanceEntry{Tranquille}
\vspace{-2mm}
\begin{listverse}
\item[\vref{Ex 14:14}] combattra pour vs. et vs. resterez t.
\item[\vref{2 Ch 20:30}] de Josaphat fut t., et son Dieu
\item[\vref{Ps 35:20}] contre les gens t. de la terre.
\item[\vref{Ps 62:6}] mon âme, demeure t., regarde à Dieu,
\item[\vref{Pr 1:33}] sécurité et sera t., sans être effrayé
\item[\vref{Es 7:4}] toi, et demeure t., ne crains point,
\item[\vref{Es 62:1}] me tiendrai pas t., et pour l'amour
\item[\vref{Jé 48:11}] Moab était t. depuis sa jeunesse,
\item[\vref{Jé 49:31}] vers la nation t. qui habite en
\item[\vref{Ez 23:42}] bruit d'une multitude t. ; et parmi cette
\item[\vref{Ha 2:5}] se tenant pas t. chez lui ; il
\item[\vref{Za 1:11}] la terre est en repos et t.
\item[\vref{1 Ti 2:2}] vie paisible et t., en tte piété
\end{listverse}

\ConcordanceEntry{Transfigurer}
\vspace{-2mm}
\begin{listverse}
\item[\vref{Mt 17:2}] Et il fut t. en lr. présence,
\end{listverse}

\ConcordanceEntry{Transformer}
\vspace{-2mm}
\begin{listverse}
\item[\vref{Ps 107:35}] Il t. le désert en étangs d'eaux, et
\item[\vref{Es 27:9}] qnd il aura t. ttes les pierres
\item[\vref{So 3:9}] Alors je t. les langues des nations en des
\item[\vref{Ro 12:2}] présent, mais soyez t. par le renouvellement
\item[\vref{2 Co 3:18}] découvert, ns. sommes t. en la mm
\item[\vref{Ph 3:21}] qui t. notre corps vil pour le rendre
\end{listverse}

\ConcordanceEntry{Transgresser}
\vspace{-2mm}
\begin{listverse}
\item[\vref{No 14:41}] lr. dit : Pourquoi t.-vs. le commandement
\item[\vref{De 26:13}] ordonnés ; je n'ai t. ni oublié aucun
\item[\vref{1 S 15:24}] parce que j'ai t. le commandement de
\item[\vref{Os 6:7}] Mais ils ont t. l'alliance, com. si
\item[\vref{Mt 15:2}] Pourquoi tes disciples t.-ils la tradition
\item[\vref{Mt 15:3}] Et vs., pourquoi t.-vs. le commandement
\item[\vref{Lu 15:29}] jamais je n'ai t. ton commandement, et
\item[\vref{Ro 2:27}] toi qui la t., tt en ayant
\item[\vref{1 Jn 3:4}] Quiconque pèche t. la loi, car
\item[\vref{2 Jn 1:9}] Quiconque t. la doctrine de
\end{listverse}

\ConcordanceEntry{Transgresseur}
\vspace{-2mm}
\begin{listverse}
\item[\vref{Ps 51:15}] tes voies aux t. et les pécheurs
\item[\vref{Pr 26:10}] à gage les insensés et les t.
\item[\vref{Es 46:8}] rappelez-le à votre pensée, ô vs. t. !
\item[\vref{Es 48:8}] as été appelé t. dès le ventre.
\item[\vref{Es 53:12}] au rang des t., et que lui-mm
\item[\vref{Ro 2:25}] si tu es t. de la loi,
\item[\vref{Ga 2:18}] montre que je suis moi-mm un t.
\item[\vref{Ja 2:9}] convaincus par la loi com. des t.
\item[\vref{Ja 2:11}] tues, tu deviens t. de la loi.
\end{listverse}

\ConcordanceEntry{Transgression}
\vspace{-2mm}
\begin{listverse}
\item[\vref{Lé 5:15}] aura commis une t. et péchera involontairement,
\item[\vref{Lé 5:21}] aura commis une t. contre Yahweh, en
\item[\vref{Ps 25:7}] ni de mes t. ; souviens-toi de moi
\item[\vref{Ps 32:1}] à qui la t. est pardonnée, et
\item[\vref{Ps 32:5}] dit : J'avouerai mes t. à Yahweh ! Et
\item[\vref{Ps 39:9}] de ttes mes t. ! Ne permets pas
\item[\vref{Ps 51:5}] je reconnais mes t., et mon péché
\item[\vref{Ps 59:4}] en moi de t. ni de péché,
\item[\vref{Ps 65:4}] tu feras la propitiation de nos t.
\item[\vref{Ps 103:12}] de ns. nos t., autant que l'orient
\item[\vref{Ps 107:17}] cause de lr. t. et à cause
\item[\vref{Pr 10:12}] mais la charité couvre ttes les t.
\item[\vref{Pr 28:13}] qui cache ses t. ne prospère pas,
\item[\vref{Es 43:25}] qui efface tes t. pour l'amour de
\item[\vref{Es 44:22}] J'efface tes t. com. une nuée
\item[\vref{Es 50:1}] été répudiée à cause de vos t.
\item[\vref{Da 9:24}] pour abolir la t. et mettre fin
\item[\vref{Ro 2:23}] Dieu par la t. de la loi !
\item[\vref{Ro 4:15}] n'y a pas non plus de t.
\item[\vref{Ro 5:14}] péché par une t. semblable à celle
\item[\vref{Ga 3:19}] à cause des t., jusqu'à ce que
\item[\vref{1 Ti 2:14}] a été la cause de la t.
\item[\vref{Hé 2:2}] et si tte t. et tte désobéissance
\item[\vref{Hé 9:15}] la rançon des t. commises sous la
\item[\vref{2 Pi 2:15}] mais il fut repris pour sa t.;
\item[\vref{1 Jn 3:4}] péché est la t. de la loi.
\end{listverse}

\ConcordanceEntry{Travail}
\vspace{-2mm}
\begin{listverse}
\item[\vref{Ge 5:29}] œuvre, et du t. pénible de nos
\item[\vref{Ge 31:42}] affliction et le t. de mes mains,
\item[\vref{Ex 1:14}] par un rude t., en les employant
\item[\vref{De 2:7}] ds tt le t. de tes mains,
\item[\vref{Esd 5:8}] les murs ; ce t. se réalise complètement
\item[\vref{Né 4:6}] le peuple avait le cœur au t.
\item[\vref{Ps 104:23}] et à son t., jusqu'au soir.
\item[\vref{Ps 128:2}] Tu jouis du t. de tes mains ;
\item[\vref{Pr 14:23}] En tt t. il y a qq profit, mais
\item[\vref{Ec 1:3}] de tt son t. auquel il s'occupe
\item[\vref{Ec 2:10}] de tt mon t. et c'est là
\item[\vref{Ec 2:18}] haï tt mon t. auquel je me
\item[\vref{Ec 4:4}] vu que tt t. et tt succès
\item[\vref{Es 53:11}] Il jouira du t. de son âme
\item[\vref{Ag 2:17}] ds tt le t. de vos mains.
\item[\vref{Jn 4:38}] et vs. êtes entrés ds lr. t.
\item[\vref{1 Co 3:8}] chacun recevra sa récompense selon son t.
\item[\vref{1 Co 15:58}] sachant que votre t. ne sera pas
\item[\vref{2 Co 11:8}] pas relâché du t., afin de n'être
\item[\vref{1 Th 1:3}] votre foi, le t. de votre charité,
\item[\vref{1 Th 3:5}] et que notre t. ne soit devenu
\item[\vref{Hé 6:10}] œuvre, et le t. de la charité
\item[\vref{2 Jn 1:8}] le fruit du t. que vs. avez
\item[\vref{Ap 2:2}] œuvres, et ton t., et ta patience,
\item[\vref{Ap 14:13}] reposent de leurs t., car leurs œuvres
\end{listverse}

\ConcordanceEntry{Travailler}
\vspace{-2mm}
\begin{listverse}
\item[\vref{Ge 11:6}] ils commencent à t. ; et mntnt rien
\item[\vref{Ex 20:9}] Tu t. six jours, et tu feras tte
\item[\vref{Né 4:17}] chargeaient les fardeaux, t. chacun d'une main,
\item[\vref{Né 4:22}] la garde la nuit et de t. le jour.
\item[\vref{Ps 127:1}] qui la bâtissent t. en vain ; si
\item[\vref{Pr 16:26}] Celui qui t., travaille pour lui-mm,
\item[\vref{Pr 21:25}] parce que ses mains refusent de t.
\item[\vref{Pr 23:4}] Ne t. pas à t'enrichir, renonce à ta
\item[\vref{Pr 31:13}] lin, et elle t. de bon cœur
\item[\vref{Es 55:2}] nourrit pas ? Pourquoi t.-vs. pour ce
\item[\vref{Jé 22:13}] droiture ; qui fait t. son prochain pour
\item[\vref{Jé 51:58}] les peuples auront t. en vain, et
\item[\vref{Jon 4:10}] tu n'as point t. et que tu
\item[\vref{Ha 2:13}] que les peuples t. pour le feu,
\item[\vref{Ag 2:4}] dit Yahweh. Et t., car je suis
\item[\vref{Mt 6:28}] champs : Ils ne t. ni ne filent ;
\item[\vref{Mt 20:12}] Ces derniers n'ont t. qu'une heure, et
\item[\vref{Mt 21:28}] Mon fils, va t. aujourd'hui ds ma
\item[\vref{Lu 5:5}] Maître, ns. avons t. tte la nuit,
\item[\vref{Jn 4:38}] vs. n'avez pas t. ; d'autres ont travaillé,
\item[\vref{Jn 6:27}] T., non pour la nourriture qui périt,
\item[\vref{Jn 9:4}] nuit vient, où personne ne peut t.
\item[\vref{Ac 18:3}] eux et y t.. Et lr. métier
\item[\vref{Ro 16:12}] et Tryphose, qui t. pour le Seign.
\item[\vref{1 Co 4:12}] ns. fatiguons à t. de nos propres
\item[\vref{1 Co 9:6}] pas le droit de ne pas t. ?
\item[\vref{1 Co 15:10}] vaine, mais j'ai t. plus qu'eux ts,
\item[\vref{1 Co 16:10}] vs., car il t. à l'œuvre du
\item[\vref{2 Co 12:14}] m'épargnerai pas à t., pour ne pas
\item[\vref{Ga 4:11}] Je crains d'avoir t. inutilement pour vs.
\item[\vref{Ph 2:16}] n'avoir pas couru en vain, ni t. en vain.
\item[\vref{1 Th 2:9}] de Dieu, en t. nuit et jour,
\item[\vref{1 Th 4:11}] affaires, et de t. de vos propres
\item[\vref{1 Th 5:12}] pour ceux qui t. parmi vs., qui
\item[\vref{2 Th 3:10}] ne veut pas t., qu'il ne mange
\item[\vref{1 Ti 5:17}] spécialement ceux qui t. à la prédication
\item[\vref{2 Ti 2:6}] que le laboureur t. premièrement, et ensuite
\item[\vref{Ap 2:3}] que tu as t. pour mon Nom,
\end{listverse}

\ConcordanceEntry{Tremblement}
\vspace{-2mm}
\begin{listverse}
\item[\vref{1 R 19:11}] ce fut un t. de terre ; mais
\item[\vref{Ps 2:11}] Yahweh avec crainte, et réjouissez-vs. avec t.
\item[\vref{Es 29:6}] des tonnerres, des t. de terre, et
\item[\vref{Ez 3:12}] bruit d'un grand t., disant : Bénie soit
\item[\vref{Za 14:5}] fui dvt le t. de terre, aux
\item[\vref{Mt 24:7}] pestes, et des t. de terre en
\item[\vref{Mt 28:2}] eut un grand t. de terre ; car
\item[\vref{Ac 16:26}] fit un grand t. de terre, en
\item[\vref{1 Co 2:3}] la crainte, et ds un grand t. ;
\item[\vref{2 Co 7:15}] avec crainte et t., son affection pour
\item[\vref{Ep 6:5}] avec crainte et t., ds la simplicité
\item[\vref{Ap 6:12}] fit un grand t. de terre, et
\item[\vref{Ap 8:5}] éclairs, et un t. de terre.
\item[\vref{Ap 16:18}] fit un grand t. de terre, dis-je,
\end{listverse}

\ConcordanceEntry{Trembler}
\vspace{-2mm}
\begin{listverse}
\item[\vref{Ex 19:18}] d'une fournaise, et tte la montagne t. fort.
\item[\vref{De 2:25}] de toi, ils t. et seront ds
\item[\vref{1 Ch 16:30}] T., vs. ts habitants de la terre
\item[\vref{Esd 9:4}] ts ceux qui t. aux paroles du
\item[\vref{Job 37:2}] attentivement et en t. le bruit de
\item[\vref{Ps 4:5}] T., et ne péchez point ; parlez en
\item[\vref{Ps 99:1}] Que les peuples t. ! Il est assis
\item[\vref{Ps 114:7}] Ô terre ! t. dvt la présence
\item[\vref{Ps 119:161}] mais mon cœur t. à cause de
\item[\vref{Ec 12:5}] de la maison t., et que les
\item[\vref{Es 64:2}] et les montagnes t. dvt toi.
\item[\vref{Es 66:2}] abattu, et qui t. à ma parole.
\item[\vref{Es 66:5}] Yahweh, vs. qui t. à sa parole ;
\item[\vref{Jé 5:22}] dit Yahweh, ne t.-vs. pas dvt
\item[\vref{Da 5:19}] de ttes langues t. dvt lui et
\item[\vref{Da 10:11}] parlé, je me tins debout en t.
\item[\vref{Os 13:1}] qu'Ephraïm parlait, on t. ; il s'élevait en
\item[\vref{Joë 2:1}] habitants du pays t. ! Car le jour
\item[\vref{Mal 2:5}] et il a t. dvt mon Nom.
\item[\vref{Mt 27:51}] et la terre t., et les pierres
\item[\vref{Ac 4:31}] ils étaient assemblés t. ; et ils furent
\item[\vref{Hé 12:21}] dit : Je suis épouvanté et tt t. !
\item[\vref{Ja 2:19}] démons le croient aussi, et ils t.
\end{listverse}

\ConcordanceEntry{Trésor}
\vspace{-2mm}
\begin{listverse}
\item[\vref{Ge 43:23}] a donné un t. ds vos sacs ;
\item[\vref{De 28:12}] t'ouvrira son bon t., les cieux, pour
\item[\vref{Jos 6:19}] entreront ds le t. de Yahweh.
\item[\vref{Ps 49:18}] n'emportera rien, ses t. ne descendront point
\item[\vref{Ps 135:7}] tire le vent hors de ses t.
\item[\vref{Pr 2:4}] tu la recherches soigneusement com. des t.,
\item[\vref{Pr 10:2}] Les t. de méchanceté ne profitent pas, mais
\item[\vref{Pr 15:16}] Yahweh, qu'un grand t. avec lequel il
\item[\vref{Pr 27:24}] Car le t. ne dure pas à toujours, et
\item[\vref{Es 45:3}] te donnerai des t. cachés, et des
\item[\vref{Ag 2:7}] nations ; et les t. de ttes les
\item[\vref{Mt 6:19}] amassez pas des t. sur la terre,
\item[\vref{Mt 6:20}] Mais amassez-vs. des t. ds le ciel,
\item[\vref{Mt 6:21}] où est ton t., là aussi sera
\item[\vref{Mt 19:21}] tu auras un t. ds le ciel.
\item[\vref{Lu 6:45}] choses du bon t. de son cœur,
\item[\vref{2 Co 4:7}] ns. avons ce t. ds des vases
\item[\vref{Col 2:3}] cachés ts les t. de la sagesse
\item[\vref{1 Ti 6:19}] pour l'avenir un t. placé sur un
\item[\vref{Hé 11:26}] un plus grand t. que les richesses
\item[\vref{Ja 5:3}] avez amassé des t. pour les derniers
\end{listverse}

\ConcordanceEntry{Tribu}
\vspace{-2mm}
\begin{listverse}
\item[\vref{Ge 49:28}] forment les douze t. d'Israël. Et c'est
\item[\vref{1 R 11:13}] j'en donnerai une t. à ton fils,
\item[\vref{Ps 78:55}] fit habiter les t. d'Israël ds les
\item[\vref{Ps 78:68}] il choisit la t. de Juda, la
\item[\vref{Ps 122:4}] laquelle montent les t., les tribus de
\item[\vref{Es 49:6}] pour relever les t. de Jacob et
\item[\vref{Jé 10:16}] Israël est la t. de son héritage.
\item[\vref{Ez 47:13}] selon les douze t. d'Israël. Joseph aura
\item[\vref{Os 5:9}] savoir parmi les t. d'Israël com. une
\item[\vref{Ha 3:9}] serment fait aux t., à savoir ta
\item[\vref{Za 9:1}] sur ttes les t. d'Israël.
\item[\vref{Mt 19:28}] jugerez les douze t. d'Israël.
\item[\vref{Mt 24:30}] ciel, ttes les t. de la terre
\item[\vref{Ph 3:5}] d'Israël, de la t. de Benjamin, Hébreu
\item[\vref{Hé 7:14}] descendu de la t. de Juda, à
\item[\vref{Ja 1:1}] Jésus-Christ, aux douze t. qui sont dispersées,
\item[\vref{Ap 1:7}] et ttes les t. de la terre
\item[\vref{Ap 7:4}] de ttes les t. des enfants d'Israël.
\item[\vref{Ap 7:9}] nation, de tte t., de tt peuple
\item[\vref{Ap 21:12}] noms des douze t. des fils d'Israël.
\end{listverse}

\ConcordanceEntry{Tribulation}
\vspace{-2mm}
\begin{listverse}
\item[\vref{Mt 13:21}] que survient une t. ou une persécution
\item[\vref{Jn 16:33}] Vous aurez des t. ds le monde ;
\item[\vref{Ac 14:22}] par beaucoup de t. qu'il ns. faut
\item[\vref{Ac 20:23}] liens et des t. m'attendent.
\item[\vref{Ro 2:9}] Il y aura t. et angoisse sur
\item[\vref{Ro 12:12}] patients ds la t. ; persévérants ds la
\item[\vref{Ap 1:9}] participe à la t., au règne, et
\item[\vref{Ap 7:14}] de la grande t., et qui ont
\end{listverse}

\ConcordanceEntry{Tribunal}
\vspace{-2mm}
\begin{listverse}
\item[\vref{Mt 10:17}] vs. livreront aux t. et vs. battront
\item[\vref{Mt 27:19}] qu'il siégeait au t., sa fem. envoya
\item[\vref{Ac 18:12}] Paul, et le menèrent dvt le t.,
\item[\vref{Ac 25:10}] comparais dvt le t. de César, où
\item[\vref{Ro 14:10}] ts dvt le t. de Christ.
\item[\vref{2 Co 5:10}] comparaître dvt le t. de Christ, afin
\end{listverse}

\ConcordanceEntry{Tribut}
\vspace{-2mm}
\begin{listverse}
\item[\vref{Ge 49:15}] le fardeau, il s'assujettit à un t.
\item[\vref{Jos 16:10}] la servitude et assujettis à un t.
\item[\vref{2 S 8:6}] lui payèrent un t.. Yahweh protégeait David
\item[\vref{Mt 17:25}] qui perçoivent-ils des t. ou des impôts ?
\item[\vref{Mt 22:17}] de payer le t. à César, ou
\item[\vref{Ro 13:7}] devez l'impôt, le t. à qui vs.
\end{listverse}

\ConcordanceEntry{Triomphe}
\vspace{-2mm}
\begin{listverse}
\item[\vref{No 23:21}] a en lui un chant de t. royal.
\item[\vref{Ps 42:5}] une voix de t. et de louange,
\item[\vref{Ps 66:1}] la terre, poussez des cris de t. à Dieu,
\item[\vref{Ps 118:15}] de chant de t. et de délivrance
\item[\vref{Ps 126:2}] de chants de t., alors on disait
\item[\vref{Ps 145:7}] avec chants de t. ta justice !
\item[\vref{Pr 11:10}] qnd les méchants périssent, c'est un t.
\item[\vref{Es 12:6}] avec chant de t. ! Car le Saint
\item[\vref{Jé 31:7}] avec chant de t., et avec allégresse
\item[\vref{So 3:14}] avec chant de t., fille de Sion !
\end{listverse}

\ConcordanceEntry{Triompher}
\vspace{-2mm}
\begin{listverse}
\item[\vref{Ps 9:20}] l'hom. mortel ne t. point ! Que les
\item[\vref{Ps 20:6}] Nous t. ds ton salut, ns. lèverons la
\item[\vref{Da 11:12}] tomber des milliers, mais il ne t. pas.
\item[\vref{Mt 12:20}] qu'il ait fait t. la justice.
\item[\vref{2 Co 2:14}] ns. fait toujours t. en Christ, et
\item[\vref{Col 2:15}] en spectacle, en t. d'elles par la
\item[\vref{Ja 2:13}] mais la miséricorde t. du jugement.
\end{listverse}

\ConcordanceEntry{Triste}
\vspace{-2mm}
\begin{listverse}
\item[\vref{Ge 40:6}] regarda ; et voici, ils étaient fort t.
\item[\vref{Ex 33:4}] peuple entendit ces t. nouvelles, et en
\item[\vref{Né 8:10}] soyez dc point t., car la joie
\item[\vref{Pr 14:13}] le cœur sera t., et la joie
\item[\vref{So 3:18}] ceux qui sont t. à cause de
\item[\vref{Mt 6:16}] pas un air t., com. font les
\item[\vref{Lu 18:23}] il devint tt t., car il était
\item[\vref{Lu 24:17}] en marchant ? Et pourquoi êtes-vs. tt t. ?
\end{listverse}

\ConcordanceEntry{Tristesse}
\vspace{-2mm}
\begin{listverse}
\item[\vref{Né 2:2}] peut être qu'une t. de cœur. Je
\item[\vref{Ps 42:10}] marcherai-je ds la t. à cause de
\item[\vref{Ps 119:158}] suis rempli de t. car ils n'observent
\item[\vref{Ec 7:3}] car par la t. du visage le
\item[\vref{Mt 26:38}] parts saisie de t. jusqu'à la mort ;
\item[\vref{Lu 22:45}] ses disciples, qu'il trouva endormis de t.,
\item[\vref{Jn 16:20}] attristés ; mais votre t. sera changée en
\item[\vref{Ro 9:2}] j'ai une grande t., et un continuel
\item[\vref{2 Co 2:7}] pas accablé par une trop grande t.
\item[\vref{2 Co 7:10}] Puisque la t. qui est selon
\item[\vref{Ph 2:27}] je n'aie pas t. sur tristesse.
\item[\vref{Ph 2:28}] joie, et que j'aie moins de t.
\item[\vref{Hé 12:11}] joie, mais de t. ; mais ensuite il
\item[\vref{Ja 4:9}] en pleurs, et votre joie en t.
\end{listverse}

\ConcordanceEntry{Troas}
\vspace{-2mm}
\begin{listverse}
\item[\vref{Ac 16:8}] ensuite la Mysie, et descendirent à T.
\item[\vref{Ac 20:5}] les devants et ns. attendirent à T.
\item[\vref{2 Co 2:12}] étant venu à T. pour l'Evangile de
\item[\vref{2 Ti 4:13}] j'ai laissé à T., chez Carpus, et
\end{listverse}

\ConcordanceEntry{Tromper}
\vspace{-2mm}
\begin{listverse}
\item[\vref{Ge 31:20}] Et Jacob t. Laban, le Syrien, en ne l'avertissant
\item[\vref{Lé 25:14}] de vs. ne t. son frère.
\item[\vref{Jos 9:22}] Pourquoi ns. avez-vs. t., en ns. disant :
\item[\vref{Ps 78:36}] Mais ils le t. de lr. bouche
\item[\vref{Ps 119:118}] moyen dont ils se servent pour t.
\item[\vref{Pr 11:18}] œuvre qui le t., mais la récompense
\item[\vref{Pr 24:28}] cause ; car voudrais-tu t. de tes lèvres ?
\item[\vref{Pr 26:19}] l'hom. qui a t. son ami, et
\item[\vref{La 1:19}] mais ils m'ont t.. Mes prêtres et
\item[\vref{Am 8:5}] sicle, et falsifiant la balance pour t. ?
\item[\vref{Ab 1:7}] avec toi t'ont t. et ont eu
\item[\vref{Mal 3:8}] L'hom. t.-t-il Dieu ? Car vs. me trompez,
\item[\vref{1 Co 6:9}] Dieu ? Ne vs. t. pas vs.-mêmes : Ni
\item[\vref{Col 2:4}] personne ne vs. t. par des discours
\item[\vref{Ja 1:16}] Mes frères bien-aimés, ne vs. y t. pas :
\item[\vref{Ja 1:22}] seulement, en vs. t. vs.-mêmes par de
\end{listverse}

\ConcordanceEntry{Tromperie}
\vspace{-2mm}
\begin{listverse}
\item[\vref{Ge 27:35}] est venu avec t., et il a
\item[\vref{Ps 17:1}] prière faite avec des lèvres sans t. !
\item[\vref{Ps 25:3}] qui agissent avec t. sans cause seront
\item[\vref{Ps 55:12}] d'elle, et la t. et la fraude
\item[\vref{Ps 101:7}] qui usera de t. ne demeurera pas
\item[\vref{Pr 12:20}] a de la t. ds le cœur
\item[\vref{Pr 14:8}] la folie des insensés est la t.
\item[\vref{Pr 26:24}] au-dedans de lui il maintient la t.
\item[\vref{Es 66:4}] attention à leurs t., et je ferai
\item[\vref{Jé 8:5}] ferme à la t., et ils refusent
\item[\vref{Ez 22:7}] on use de t. à l'égard de
\item[\vref{Da 11:23}] il usera de t., et il montera,
\item[\vref{Mal 2:14}] tu agis avec t. ; et toutefois elle
\item[\vref{Ep 4:14}] doctrine, par la t. des hommes et
\item[\vref{Col 2:8}] par de vaines t. conformes à la
\item[\vref{2 Pi 2:13}] délices de leurs t. ds les repas
\end{listverse}

\ConcordanceEntry{Trompette}
\vspace{-2mm}
\begin{listverse}
\item[\vref{Ex 19:13}] point. Quand la t. sonnera longuement, ils
\item[\vref{No 10:2}] Fais-toi deux t. d'argent, battues au
\item[\vref{1 Ch 15:24}] prêtres, sonnaient des t. dvt l'arche de
\item[\vref{Esd 3:10}] habits, avec leurs t., et les Lévites,
\item[\vref{Né 12:35}] prêtres avec les t., Zacharie, fils de
\item[\vref{Job 39:27}] son de la t., et il ne
\item[\vref{Ps 27:6}] son de la t. ; je chanterai et
\item[\vref{Ps 89:16}] son de la t. ! Il marche, ô
\item[\vref{Ez 7:14}] sonne de la t., tt est prêt,
\item[\vref{Mt 6:2}] pas sonner la t. dvt toi, com.
\item[\vref{Mt 24:31}] grand son de t., et ils rassembleront
\item[\vref{1 Co 14:8}] Et si la t. rend un son incertain, qui se
\item[\vref{1 Co 15:52}] à la dernière t.. Car la trompette
\item[\vref{1 Th 4:16}] et avec la t. de Dieu, descendra
\item[\vref{Hé 12:19}] retentissement de la t., ni du son
\item[\vref{Ap 8:2}] Dieu, et sept t. lr. furent données.
\item[\vref{Ap 8:6}] avaient les sept t. se préparèrent à
\item[\vref{Ap 18:22}] sonnent de la t. ; et on ne
\end{listverse}

\ConcordanceEntry{Trompeur}
\vspace{-2mm}
\begin{listverse}
\item[\vref{Ps 5:7}] en abomination l'hom. sanguinaire et le t.
\item[\vref{Pr 12:17}] le faux témoin fait des rapports t.
\item[\vref{Pr 14:25}] celui qui prononce des mensonges est t.
\item[\vref{Es 32:5}] libéral, et l'avare t. ne sera plus
\item[\vref{La 2:14}] t'ont prophétisé des oracles mensongers et t.
\item[\vref{Mal 1:14}] maudit soit l'hom. t., qui a ds
\item[\vref{2 Co 11:13}] apôtres, des ouvriers t. qui se déguisent
\end{listverse}

\ConcordanceEntry{Trône}
\vspace{-2mm}
\begin{listverse}
\item[\vref{Ge 41:40}] plus grand que toi quant au t.
\item[\vref{Ex 17:16}] levée contre le t. de Yahweh, Yahweh
\item[\vref{2 S 7:16}] yeux, et ton t. sera pour toujours
\item[\vref{2 Ch 6:10}] assis sur le t. d'Israël, com. Yahweh
\item[\vref{Job 23:3}] comment le trouver, j'irais jusqu'à son t.,
\item[\vref{Ps 9:5}] sièges sur ton t. en juste juge.
\item[\vref{Ps 11:4}] Yahweh a son t. ds les cieux ;
\item[\vref{Ps 89:15}] base de ton t. ; la bonté et
\item[\vref{Ps 93:2}] Ton t. est établi dès lors, tu es
\item[\vref{Ps 122:5}] été posés les t. pour juger, les
\item[\vref{Ps 132:11}] fruit de tes entrailles sur ton t.
\item[\vref{Pr 16:12}] parce que le t. est affermi par
\item[\vref{Pr 20:28}] il soutient son t. par la grâce.
\item[\vref{Es 6:1}] assis sur un t. haut et élevé,
\item[\vref{Es 9:6}] sans fin au t. de David et
\item[\vref{La 5:19}] éternellement, et ton t. subsiste de génération
\item[\vref{Ez 1:26}] en forme de t. ; et sur cette
\item[\vref{Da 7:9}] ce que les t. soient placés. Et
\item[\vref{Jon 3:6}] leva de son t., ôta de dessus
\item[\vref{Za 6:13}] dominera sur son t., il sera prêtre,
\item[\vref{Mt 19:28}] assis sur le t. de sa gloire,
\item[\vref{Mt 25:31}] s'assiéra sur le t. de sa gloire.
\item[\vref{Lu 1:32}] lui donnera le t. de David, son
\item[\vref{Col 1:16}] invisibles, soit les t., ou les dominations,
\item[\vref{Hé 4:16}] avec assurance du t. de la grâce,
\item[\vref{Ap 2:13}] où est le t. de Satan. Et
\item[\vref{Ap 4:2}] Et voici, un t. était dressé ds
\item[\vref{Ap 20:4}] Je vis des t., sur lesquels des
\item[\vref{Ap 20:11}] vis un grand t. blanc, et celui
\item[\vref{Ap 21:5}] assis sur le t. dit : Voici, je
\item[\vref{Ap 22:3}] plus d'anathème. Le t. de Dieu et
\end{listverse}

\ConcordanceEntry{Trophime}
\vspace{-2mm}
\begin{listverse}
\item[\vref{Ac 20:4}] que Tychique et T., originaires d'Asie.
\item[\vref{2 Ti 4:20}] et j'ai laissé T. malade à Milet.
\end{listverse}

\ConcordanceEntry{Trouble}
\vspace{-2mm}
\begin{listverse}
\item[\vref{1 R 18:17}] qui jettes le t. en Israël ?
\item[\vref{Job 34:29}] qui causera du t. ? S'il cache sa
\item[\vref{Pr 15:6}] y a du t. ds les revenus
\item[\vref{Es 22:5}] le jour de t., d'oppression et de
\item[\vref{Jé 50:34}] mettre ds le t. les habitants de
\item[\vref{Za 14:13}] produira un grand t. parmi eux ; car
\item[\vref{Mc 13:8}] famines et des t.. Ces choses ne
\item[\vref{Mc 16:8}] peur et le t. les avaient saisies ;
\item[\vref{Ac 19:23}] arriva un grand t., à cause de
\item[\vref{2 Co 6:5}] prisons, ds les t., ds les travaux,
\item[\vref{Hé 12:15}] rejetons, ne vs. t., et que plusieurs
\end{listverse}

\ConcordanceEntry{Troubler}
\vspace{-2mm}
\begin{listverse}
\item[\vref{Jos 7:25}] Pourquoi ns. as-tu t. ? Yahweh te troublera
\item[\vref{1 S 16:14}] mauvais esprit envoyé par Yahweh le t.
\item[\vref{1 S 28:15}] Saül : Pourquoi m'as-tu t. en me faisant
\item[\vref{Job 13:21}] et que ta frayeur ne me t. pas.
\item[\vref{Ps 90:7}] et ns. sommes t. par ta fureur.
\item[\vref{Pr 29:8}] Les hommes moqueurs t. la ville, mais
\item[\vref{Es 35:4}] ont le cœur t. : Prenez courage et
\item[\vref{Mt 24:6}] guerres ; gardez-vs. d'être t., car il faut
\item[\vref{Mc 10:32}] Les disciples étaient t., et le suivaient
\item[\vref{Lu 1:29}] T. par cette parole, Marie se demandait
\item[\vref{Jn 12:27}] mon âme est t.. Et que dirai-je ?
\item[\vref{Jn 14:1}] cœur ne se t. pas. Vous croyez
\item[\vref{Ac 16:20}] qui sont Juifs, t. notre ville,
\item[\vref{Ga 1:7}] gens qui vs. t., et qui veulent
\item[\vref{Ga 5:10}] celui qui vs. t., quel qu'il soit,
\item[\vref{1 Th 3:3}] nul ne soit t. ds ces afflictions,
\item[\vref{2 Th 2:2}] votre entendement, ni t. par une inspiration,
\item[\vref{Hé 12:15}] rejetons, ne vs. t., et que plusieurs
\item[\vref{1 Pi 3:6}] sans vs. laisser t. par aucune crainte.
\end{listverse}

\ConcordanceEntry{Troupeau}
\vspace{-2mm}
\begin{listverse}
\item[\vref{Ge 4:4}] premiers-nés de son t., et de lr.
\item[\vref{Ex 3:1}] fut berger du t. de Jéthro, son
\item[\vref{Ps 74:1}] fume-t-elle contre le t. de ton pâturage ?
\item[\vref{Ca 1:7}] fais paître ton t., et où tu
\item[\vref{Es 40:11}] Il paîtra son t. com. un berger,
\item[\vref{Ez 34:2}] Les pasteurs ne paissent-ils pas le t. ?
\item[\vref{Ez 34:3}] gras, vs. ne paissez point le t. !
\item[\vref{Mi 7:14}] ta houlette, le t. de ton héritage,
\item[\vref{Mal 1:14}] a ds son t. un mâle, et
\item[\vref{Mt 8:30}] d'eux un grand t. de pourceaux qui
\item[\vref{Mt 26:31}] les brebis du t. seront dispersées.
\item[\vref{Lu 2:8}] qui gardaient lr. t. pendant les veilles
\item[\vref{Lu 12:32}] crains pas petit t., car il a
\item[\vref{Lu 17:7}] ou paît les t., lui dira, qnd
\item[\vref{Jn 10:16}] aura un seul t., et un seul
\item[\vref{Ac 20:28}] à tt le t. sur lequel le
\item[\vref{Ac 20:29}] très dangereux, qui n'épargneront pas le t.,
\item[\vref{1 Co 9:7}] fait paître un t. et ne se
\item[\vref{1 Pi 5:2}] Paissez le t. de Dieu qui
\end{listverse}

\ConcordanceEntry{Trouver}
\vspace{-2mm}
\begin{listverse}
\item[\vref{Ge 2:20}] Adam, il ne t. point d'aide semblable
\item[\vref{Ge 4:14}] que quiconque me t. me tuera.
\item[\vref{Ge 6:8}] Mais Noé t. grâce aux yeux de Yahweh.
\item[\vref{De 18:10}] Qu'on ne t. au milieu de toi personne qui
\item[\vref{1 R 21:20}] à Elie : M'as-tu t. mon ennemi ? Mais
\item[\vref{Job 11:7}] T.-tu le fond en Dieu en
\item[\vref{Job 23:3}] savais comment le t., j'irais jusqu'à son
\item[\vref{Ps 17:3}] tu n'as rien t. : Ma pensée ne
\item[\vref{Ps 89:21}] J'ai t. David, mon serviteur, je l'ai oint
\item[\vref{Pr 2:5}] Yahweh, et tu t. la connaissance de
\item[\vref{Pr 3:13}] l'hom. qui a t. la sagesse, et
\item[\vref{Pr 8:17}] ceux qui me cherchent soigneusement me t.
\item[\vref{Pr 8:31}] de sa terre, t. mes délices avec
\item[\vref{Pr 8:35}] celui qui me t., a trouvé la
\item[\vref{Pr 18:22}] Celui qui t. une fem. trouve
\item[\vref{Pr 19:8}] garde à l'intelligence t. le bonheur.
\item[\vref{Pr 20:6}] bonté ; mais qui t. un hom. fidèle ?
\item[\vref{Pr 31:10}] [Aleph.] Qui t. une fem. vertueuse ?
\item[\vref{Ec 7:14}] que l'hom. ne t. rien à redire
\item[\vref{Ec 7:24}] ce qui est profond, qui le t. ?
\item[\vref{Ec 7:29}] ce que j'ai t. ; c'est que Dieu
\item[\vref{Ec 8:17}] ne peut pas t. l'œuvre qui se
\item[\vref{Ec 12:12}] a cherché pour t. des discours agréables ;
\item[\vref{Ca 3:1}] cherché, mais je ne l'ai point t.
\item[\vref{Ca 3:2}] cherché, mais je ne l'ai point t.
\item[\vref{Es 65:1}] me suis laissé t. par ceux qui
\item[\vref{Jé 15:16}] J'ai t. tes paroles, je les ai aussitôt
\item[\vref{Jé 29:13}] et vs. me t., après que vs.
\item[\vref{Jé 29:14}] je me laisserai t. par vs., dit
\item[\vref{Ez 3:1}] ce que tu t., mange ce rouleau,
\item[\vref{Ez 14:14}] et Job s'y t., ils sauveraient leurs
\item[\vref{Ez 14:20}] et Job, s'y t., je suis vivant,
\item[\vref{Ez 20:9}] lesquelles ils se t., et aux yeux
\item[\vref{Ez 22:30}] pas ; mais je n'en ai pas t.
\item[\vref{Da 1:9}] Et Dieu fit t. à Daniel faveur
\item[\vref{Da 1:19}] il ne s'en t. pas de tels
\item[\vref{Da 6:11}] s'assemblèrent, et ils t. Daniel qui priait
\item[\vref{Da 6:22}] que j'ai été t. innocent dvt lui ;
\item[\vref{Os 5:6}] ils ne le t. point, il s'est
\item[\vref{Mt 7:7}] cherchez, et vs. t. ; frappez, et l'on
\item[\vref{Mt 7:8}] celui qui cherche t., et l'on ouvre
\item[\vref{Mt 7:14}] y en a peu qui les t.
\item[\vref{Mt 11:29}] cœur ; et vs. t. le repos pour
\item[\vref{Mc 11:2}] serez entrés, vs. t. un ânon attaché,
\item[\vref{Mc 14:16}] ville, et ils t. les choses com.
\item[\vref{Lu 4:17}] l'ayant déroulé, il t. le passage où
\item[\vref{Lu 15:6}] moi ; car j'ai t. ma brebis qui
\item[\vref{Lu 18:8}] viendra, pensez-vs. qu'il t. la foi sur
\item[\vref{Lu 23:4}] foule : Je ne t. aucun crime en
\item[\vref{Lu 24:3}] entrées, elles ne t. pas le corps
\item[\vref{Jn 1:41}] dit : Nous avons t. le Messie, c'est-à-dire
\item[\vref{Ro 7:21}] Je t. dc cette loi au-dedans de moi :
\item[\vref{Ro 10:20}] dire : J'ai été t. par ceux qui
\item[\vref{Ph 3:9}] que je sois t. en lui, ayant
\item[\vref{1 Pi 4:12}] Mes bien-aimés, ne t. pas étrange qnd
\item[\vref{2 Pi 3:14}] appliquez-vs. à être t. par lui sans
\item[\vref{Ap 2:2}] tu les as t. menteurs ;
\item[\vref{Ap 3:2}] je n'ai pas t. tes œuvres parfaites
\item[\vref{Ap 5:4}] que personne n'était t. digne d'ouvrir le
\item[\vref{Ap 9:6}] ils ne la t. pas ; et ils
\item[\vref{Ap 18:24}] et l'on a t. chez elle le
\item[\vref{Ap 20:15}] ne fut pas t. écrit ds le
\end{listverse}

\ConcordanceEntry{Truie}
\vspace{-2mm}
\begin{listverse}
\item[\vref{2 Pi 2:22}] vomi, et la t. lavée est retournée
\end{listverse}

\ConcordanceEntry{Tsadok}
\vspace{-2mm}
\begin{listverse}
\item[\vref{2 S 8:17}] T., fils d'Achithub, et Achimélec, fils d'Abiathar,
\item[\vref{1 R 1:8}] Mais le prêtre T., Benaja fils de
\item[\vref{1 R 1:39}] T., le prêtre, prit du tabernacle une
\item[\vref{1 R 2:35}] roi établit aussi T. prêtre à la
\item[\vref{1 R 4:2}] son service : Azaria, fils du prêtre T. ;
\end{listverse}

\ConcordanceEntry{Tselophchad}
\vspace{-2mm}
\begin{listverse}
\item[\vref{No 26:33}] T., fils de Hépher, n'eut point de
\item[\vref{No 27:1}] les filles de T., fils de Hépher,
\item[\vref{No 36:11}] Noa, filles de T., se marièrent aux
\end{listverse}

\ConcordanceEntry{Tseruja}
\vspace{-2mm}
\begin{listverse}
\item[\vref{2 S 3:39}] les fils de T., sont trop puissants
\item[\vref{1 Ch 2:16}] T. et Abigaïl furent leurs sœurs. Tseruja
\end{listverse}

\ConcordanceEntry{Tsiba}
\vspace{-2mm}
\begin{listverse}
\item[\vref{2 S 9:2}] un serviteur nommé T., que l'on fit
\item[\vref{2 S 16:1}] le sommet, voici, T., serviteur de Mephiboscheth,
\item[\vref{2 S 19:29}] dit : Toi et T., vs. partagerez les
\end{listverse}

\ConcordanceEntry{Tsiklag}
\vspace{-2mm}
\begin{listverse}
\item[\vref{1 S 27:6}] ce mm jour, T.. C'est pourquoi Tsiklag
\item[\vref{1 S 30:1}] étant revenus à T., trouvèrent que les
\item[\vref{2 S 1:1}] des Amalécites, resta deux jours à T.
\end{listverse}

\ConcordanceEntry{Tsin}
\vspace{-2mm}
\begin{listverse}
\item[\vref{No 13:21}] le désert de T. jusqu'à Rehob, à
\item[\vref{No 20:1}] le désert de T. au premier mois,
\item[\vref{No 27:14}] le désert de T., lors de la
\end{listverse}

\ConcordanceEntry{Tsoan}
\vspace{-2mm}
\begin{listverse}
\item[\vref{No 13:22}] sept ans avant T. en Egypte.
\item[\vref{Ps 78:12}] pays d'Egypte, ds le champ de T.
\end{listverse}

\ConcordanceEntry{Tsoar}
\vspace{-2mm}
\begin{listverse}
\item[\vref{Ge 14:2}] le roi de Béla, qui est T.
\item[\vref{Ge 19:22}] c'est pourquoi cette ville fut appelée T.
\item[\vref{Ge 19:30}] Lot monta de T. et habita sur
\item[\vref{De 34:3}] Jéricho, la ville des palmiers, jusqu'à T.
\item[\vref{Es 15:5}] fugitifs s'enfuient jusqu'à T., com. une génisse
\end{listverse}

\ConcordanceEntry{Tsophar}
\vspace{-2mm}
\begin{listverse}
\item[\vref{Job 2:11}] de Schuach, et T. de Naama, ayant
\item[\vref{Job 20:1}] Alors T. de Naama prit la parole et
\item[\vref{Job 42:9}] de Schuach, et T. de Naama vinrent
\end{listverse}

\ConcordanceEntry{Tsoréa}
\vspace{-2mm}
\begin{listverse}
\item[\vref{Jos 19:41}] lr. héritage fut, T., Eschthaol, Ir-Schémesch,
\item[\vref{Jg 13:2}] un hom. de T., de la famille
\item[\vref{Jg 13:25}] à Machané-Dan, entre T. et Eschthaol.
\end{listverse}

\ConcordanceEntry{Tubal}
\vspace{-2mm}
\begin{listverse}
\item[\vref{Ge 10:2}] Magog, Madaï, Javan, T., Méschec, et Tiras.
\item[\vref{Es 66:19}] de l'arc, à T. et à Javan,
\item[\vref{Ez 38:2}] Méschec et de T., et prophétise contre
\end{listverse}

\ConcordanceEntry{Tubal-Caïn}
\vspace{-2mm}
\begin{listverse}
\item[\vref{Ge 4:22}] Tsilla aussi enfanta T., qui fut forgeur
\end{listverse}

\ConcordanceEntry{Tuer}
\vspace{-2mm}
\begin{listverse}
\item[\vref{Ge 4:8}] sur Abel, son frère, et le t.
\item[\vref{Ge 4:23}] ma parole ! Je t. un hom. pour
\item[\vref{Ge 27:41}] approchent, et je t. Jacob, mon frère.
\item[\vref{Ge 34:25}] la ville et t. ts les mâles.
\item[\vref{Ge 37:26}] Que gagnerons-ns. à t. notre frère et
\item[\vref{Ex 2:14}] ns. ? Veux-tu me t. com. tu as
\item[\vref{Ex 13:15}] laisser aller, Yahweh t. ts les premiers-nés
\item[\vref{Ex 32:27}] chacun de vs. t. son frère, son
\item[\vref{No 11:15}] à mon égard, t.-moi, je te
\item[\vref{No 22:33}] t'aurais mm déjà t., et je lui
\item[\vref{De 4:42}] meurtrier qui aurait t. son prochain involontairement,
\item[\vref{De 5:17}] Tu ne t. point.
\item[\vref{De 12:15}] âme, tu pourras t. et manger de
\item[\vref{De 12:21}] toi, alors tu t. de ton gros
\item[\vref{De 21:1}] posséder, un hom. t., étendu ds un
\item[\vref{Jg 9:56}] son père, en t. ses soixante-dix frères,
\item[\vref{Jg 15:16}] mâchoire d'âne, j'ai t. mille hommes.
\item[\vref{1 S 24:12}] ne t'ai pas t.. Sache et reconnais
\item[\vref{1 R 18:13}] fis qnd Jézabel t. les prophètes de
\item[\vref{Job 13:15}] Voici, qu'il me t., je ne cesserai
\item[\vref{Pr 7:26}] ceux qu'elle a t. étaient forts.
\item[\vref{Ec 3:3}] un temps pour t. et un temps
\item[\vref{Ez 32:20}] ceux qui seront t. par l'épée. L'épée
\item[\vref{Os 6:5}] je les ai t. par les paroles
\item[\vref{Mt 2:16}] et il envoya t. ts les enfants
\item[\vref{Mt 5:21}] anciens : Tu ne t. pas ; et celui
\item[\vref{Mt 10:28}] pas ceux qui t. le corps et
\item[\vref{Mt 23:37}] Jérus., Jérus., qui t. les prophètes, et
\item[\vref{Ac 23:12}] ne boiraient jusqu'à ce qu'ils aient t. Paul.
\item[\vref{Ro 11:3}] Seign., ils ont t. tes prophètes, et
\item[\vref{2 Co 3:6}] car la lettre t., mais l'Esprit vivifie.
\item[\vref{Hé 11:28}] le destructeur qui t. les premiers-nés, ne
\item[\vref{1 Jn 3:12}] malin et qui t. son frère. Et
\item[\vref{Ap 6:4}] les hommes se t. les uns les
\item[\vref{Ap 6:9}] qui avaient été t. pour la parole
\item[\vref{Ap 9:15}] l'année, afin de t. le tiers des
\item[\vref{Ap 13:10}] captivité ; si quelqu'un t. avec l'épée, il
\end{listverse}

\ConcordanceEntry{Tumulte}
\vspace{-2mm}
\begin{listverse}
\item[\vref{Ps 65:8}] flots, et le t. des peuples.
\item[\vref{Es 22:2}] bruyante, pleine de t., ville joyeuse ! Tes
\item[\vref{Jé 51:16}] y a un t. d'eaux ds les
\item[\vref{Ez 23:46}] soient abandonnées au t. et au pillage.
\item[\vref{Os 10:14}] C'est pourquoi un t. s'élèvera parmi ton
\item[\vref{Am 2:2}] mourra ds le t., au milieu des
\item[\vref{Mt 26:5}] se fasse qq t. parmi le peuple.
\item[\vref{Mt 27:24}] mais que le t. s'augmentait, prit de
\item[\vref{Mc 5:38}] Jésus vit le t., c'est-à-dire ceux qui
\item[\vref{Ac 4:25}] serviteur : Pourquoi ce t. parmi les nations
\item[\vref{Ac 20:1}] Lorsque le t. eut cessé, Paul
\item[\vref{Ac 24:18}] ds le temple, sans attroupement ni t.
\end{listverse}

\ConcordanceEntry{Tunique}
\vspace{-2mm}
\begin{listverse}
\item[\vref{Ge 3:21}] sa fem. des t. de peaux, et
\item[\vref{Ge 37:3}] lui fit une t. de plusieurs couleurs.
\item[\vref{Ex 28:39}] feras aussi une t. de fin lin
\item[\vref{1 S 2:19}] faisait une petite t., qu'elle lui apportait
\item[\vref{2 S 13:18}] était habillée d'une t. de couleurs ; car
\item[\vref{Né 7:70}] cinq cent trente t. de prêtres.
\item[\vref{Job 30:18}] près, com. fait l'ouverture de ma t.
\item[\vref{Ps 22:19}] vêtements et tirent au sort ma t.
\item[\vref{Ca 5:3}] J'ai enlevé ma t., lui dis-je, comment
\item[\vref{Es 3:22}] et les larges t., les manteaux et
\item[\vref{Es 22:21}] revêtirai de ta t., je le ceindrai
\item[\vref{Mt 10:10}] voyage, ni deux t., ni souliers, ni
\item[\vref{Lu 3:11}] qui a deux t. partage avec celui
\item[\vref{Lu 6:29}] l'empêche pas de prendre aussi ta t.
\item[\vref{Jn 19:23}] prirent aussi sa t., qui était sans
\item[\vref{Jn 19:24}] au sort ma t.. Ainsi firent les
\item[\vref{Jn 21:7}] il ceignit sa t., parce qu'il était
\item[\vref{Ac 9:39}] lui montrèrent les t. et les vêtements
\end{listverse}

\ConcordanceEntry{Turban}
\vspace{-2mm}
\begin{listverse}
\item[\vref{Es 62:3}] de Yahweh, un t. royal ds la
\item[\vref{Ez 24:17}] morts ; attache ton t. sur ta tête,
\item[\vref{Za 3:5}] sa tête un t. pur ! Et ils
\end{listverse}

\ConcordanceEntry{Tychique}
\vspace{-2mm}
\begin{listverse}
\item[\vref{Ac 20:4}] Timothée, ainsi que T. et Trophime, originaires
\item[\vref{Ep 6:21}] que je fais, T., notre frère bien-aimé
\item[\vref{2 Ti 4:12}] J'ai aussi envoyé T. à Ephèse.
\end{listverse}

\ConcordanceEntry{Tyr}
\vspace{-2mm}
\begin{listverse}
\item[\vref{Jos 19:29}] ville forte de T., et cette frontière
\item[\vref{Ez 27:3}] Tu diras à T. : Toi qui demeures
\end{listverse}
\begin{legend}
\NoAutoSpaceBeforeFDP{
\item Une ville et sa forteresse : Jos 19:29; 2 S 24:7; Joë 3:4
\item Prophéties et jugements : Es 23:1-5,15; Jé 25:22-30; Ez 27:2
\item Jésus à Tyr et à Sidon : Mt 15:21; Mc 7:24
\item Autres : 1 R 5:1; Ps 45:13; Am 1:9; Mt 11:21; Ac 21:3
}
\end{legend}

\ConcordanceEntry{Ulcère}
\vspace{-2mm}
\begin{listverse}
\item[\vref{Ex 9:11}] à cause des u. ; car les magiciens
\item[\vref{Lé 22:22}] qui ait un u., une gale sèche
\item[\vref{De 28:27}] te frappera de l'u. d'Egypte, d'hémorroïdes, de
\item[\vref{2 R 20:7}] et l'appliquèrent sur l'u.. Et Ezéchias fut
\item[\vref{Job 2:7}] frappa Job d'un u. malin, depuis la
\item[\vref{Es 38:21}] un emplâtre sur l'u. ; et Ezéchias guérira.
\item[\vref{Lu 16:21}] les chiens venaient encore lécher ses u.
\item[\vref{Ap 16:2}] terre. Et un u. malin et dangereux
\end{listverse}

\ConcordanceEntry{Un, une}
\vspace{-2mm}
\begin{listverse}
\item[\vref{Ge 1:5}] le matin ; ce fut le jour u.
\item[\vref{Ge 2:24}] et ils deviendront u. seule chair.
\item[\vref{De 6:4}] Israël ! Yahweh, notre Dieu, Yahweh est U.
\item[\vref{Es 34:16}] n'en manquera pas u. seul point ; ni
\item[\vref{Mal 2:10}] N'avons-ns. pas ts u. seul Père ? N'est-ce
\item[\vref{Mt 23:9}] votre père ; car u. seul est votre
\item[\vref{Jn 8:9}] ils se retirèrent u. à un, depuis
\item[\vref{Jn 10:16}] il y aura u. seul troupeau, et
\item[\vref{Jn 11:52}] pour rassembler en u. seul corps les
\item[\vref{Jn 17:11}] afin qu'ils soient u., com. ns. sommes
\item[\vref{Ro 3:30}] puisqu'il y a u. seul Dieu, qui
\item[\vref{1 Co 6:17}] est avec lui u. seul esprit.
\item[\vref{1 Co 8:6}] ns. n'avons pourtant qu'u. seul Dieu, qui
\item[\vref{1 Co 10:17}] Puisqu'il y a u. seul pain, ns.
\item[\vref{1 Co 12:11}] U. seul et mm Esprit opère ttes
\item[\vref{2 Co 11:2}] jaloux de vs. d'u. jalousie de Dieu,
\item[\vref{Ga 3:20}] n'est pas médiateur d'u. seul, mais Dieu
\item[\vref{Ga 3:28}] vs. êtes ts u. en Jésus-Christ.
\item[\vref{Ep 2:16}] de réconcilier les u. et les autres
\item[\vref{Ep 4:4}] Il y a u. seul corps, un seul Esprit, com.
\item[\vref{Ep 4:5}] il y a u. seul Seign., une seule foi, un
\item[\vref{Ep 4:6}] u. seul Dieu et Père de ts,
\item[\vref{Col 3:15}] appelés pour être u. seul corps, tienne
\item[\vref{Ja 2:19}] que Dieu est u., tu fais bien ;
\end{listverse}

\ConcordanceEntry{Unique}
\vspace{-2mm}
\begin{listverse}
\item[\vref{Ge 22:2}] ton fils, ton u., celui que tu
\item[\vref{Jg 11:34}] était seule et u., sans qu'il eût
\item[\vref{Ps 35:17}] leurs tempêtes, mon u. des lionceaux.
\item[\vref{Pr 4:3}] fils tendre et u. auprès de ma
\item[\vref{Ca 6:9}] ma parfaite est u. ; elle est l'unique
\item[\vref{Am 8:10}] pour un fils u., et sa fin
\item[\vref{Za 12:10}] sur un fils u., et ils pleureront
\item[\vref{Za 14:7}] sera un jour u., connu de Yahweh,
\item[\vref{Lu 7:12}] un mort, fils u. de sa mère,
\item[\vref{Lu 8:42}] avait une fille u., âgée d'environ douze
\item[\vref{Lu 9:38}] mon fils, car c'est mon fils u.
\item[\vref{Jn 1:14}] gloire com. la gloire du Fils u. du Père.
\item[\vref{Jn 1:18}] Dieu ; le Fils u., qui est ds
\item[\vref{Jn 3:16}] donné son Fils u., afin que quiconque
\item[\vref{Jn 3:18}] pas cru au Nom du Fils u. de Dieu.
\item[\vref{Hé 11:17}] les promesses offrit mm son fils u.,
\item[\vref{1 Jn 4:9}] envoyé son Fils u. ds le monde,
\end{listverse}

\ConcordanceEntry{Unir}
\vspace{-2mm}
\begin{listverse}
\item[\vref{Jg 20:11}] contre la ville, u. com. un seul
\item[\vref{Ps 31:14}] ils se concertent u. contre moi : Ils
\item[\vref{Ps 133:1}] des frères demeurent u. ensemble !
\item[\vref{Mi 7:3}] déclare ce qu'il convoite, et ils s'u.
\item[\vref{1 Co 1:10}] vs. soyez bien u. ds un mm
\item[\vref{1 Co 6:16}] que celui qui s'u. à la prostituée,
\item[\vref{1 Co 6:17}] Mais celui qui s'u. au Seign. est
\item[\vref{1 Co 7:35}] propre à vs. u. au Seign. sans
\item[\vref{2 Co 5:14}] de Christ ns. u. étroitement, car ns.
\item[\vref{Col 2:2}] soient consolés, étant u. ensemble ds la
\end{listverse}

\ConcordanceEntry{Unité}
\vspace{-2mm}
\begin{listverse}
\item[\vref{Ep 4:3}] efforçant de garder l'u. de l'Esprit par
\item[\vref{Ep 4:13}] ts parvenus à l'u. de la foi
\end{listverse}

\ConcordanceEntry{Ur}
\vspace{-2mm}
\begin{listverse}
\item[\vref{Ge 11:28}] sa naissance, à U. en Chaldée.
\item[\vref{Ge 11:31}] ils sortirent ensemble d'U. en Chaldée pour
\item[\vref{Ge 15:7}] t'ai fait sortir d'U. en Chaldée, afin
\item[\vref{Né 9:7}] l'as fait sortir d'U. en Chaldée, et
\end{listverse}

\ConcordanceEntry{Urie}
\vspace{-2mm}
\begin{listverse}
\item[\vref{2 S 11:3}] fille d'Eliam, fem. d'U., le Héthien ?
\item[\vref{2 R 16:16}] Le prêtre U., exécuta tt ce
\end{listverse}
\begin{legend}
\NoAutoSpaceBeforeFDP{
\item Bath-Schéba son épouse : 2 S 11:3
\item Autres : 2 S 11:14-17; 12:9; 23:39; 1 R 15:5
\item Le prêtre : 2 R 16:10-16
\item Le prophète, fils de Schemaeja : Jé 26:20-23
}
\end{legend}

\ConcordanceEntry{Urim}
\vspace{-2mm}
\begin{listverse}
\item[\vref{Ex 28:30}] pectoral de jugement l'u. et le thummim,
\item[\vref{Lé 8:8}] mis au pectoral l'u. et le thummim.
\item[\vref{No 27:21}] les jugements de l'u. dvt Yahweh ; et
\item[\vref{De 33:8}] thummim et tes u. sont à l'hom.
\item[\vref{1 S 28:6}] songes, ni par l'u., ni par les
\item[\vref{Esd 2:63}] prêtre ait consulté l'u. et le thummim.
\item[\vref{Né 7:65}] prêtre eût consulté l'u. et le thummim.
\end{listverse}

\ConcordanceEntry{Usage}
\vspace{-2mm}
\begin{listverse}
\item[\vref{No 4:9}] dont on fait u. pour son service ;
\item[\vref{De 26:14}] disparaître pour un u. impur, et je
\item[\vref{1 Ch 23:31}] nombre et les u. prescrits.
\item[\vref{1 Ch 28:15}] ses lampes, selon l'u. de chaque chandelier.
\item[\vref{Am 2:14}] pourra pas faire u. de sa vigueur,
\item[\vref{Ro 1:26}] eux ont changé l'u. naturel en celui
\item[\vref{Ro 9:21}] vase d'honneur et un vase d'un u. vil ?
\item[\vref{Col 2:22}] ttes périssables par l'u., et établies suivant
\item[\vref{1 Ti 1:8}] en fait un u. légitime,
\item[\vref{2 Ti 2:20}] d'honneur et les autres sont d'un u. vil.
\item[\vref{Tit 3:14}] œuvres, pour les u. nécessaires, afin qu'ils
\end{listverse}

\ConcordanceEntry{Utile}
\vspace{-2mm}
\begin{listverse}
\item[\vref{Job 35:8}] hom. que ta justice peut être u.
\item[\vref{Ac 20:20}] qui vs. était u., et que je
\item[\vref{1 Co 7:35}] qui vs. est u., non pas pour
\item[\vref{1 Co 10:23}] ne sont pas u. ; ttes choses me
\item[\vref{Ph 1:22}] Mais s'il est u. pour mon œuvre
\item[\vref{1 Ti 4:8}] l'exercice corporel est u. à peu de
\item[\vref{2 Ti 2:21}] d'honneur, sanctifié et u. au Seign., et
\item[\vref{2 Ti 3:16}] de Dieu et u. pour enseigner, pour
\item[\vref{2 Ti 4:11}] il m'est fort u. pour le service.
\item[\vref{Phm 1:11}] mntnt est bien u. à toi et
\end{listverse}

\ConcordanceEntry{Uts}
\vspace{-2mm}
\begin{listverse}
\item[\vref{Ge 10:23}] les fils d'Aram : U., Hul, Guéter et
\item[\vref{Job 1:1}] ds le pays d'U. un hom. appelé
\item[\vref{Jé 25:20}] rois du pays d'U., à ts les
\end{listverse}

\ConcordanceEntry{Uzza}
\vspace{-2mm}
\begin{listverse}
\item[\vref{2 S 6:3}] sur la colline ; U. et Achjo, fils
\item[\vref{1 Ch 13:7}] maison d'Abinadab ; et U. et Achjo conduisaient
\end{listverse}

\ConcordanceEntry{Vache}
\vspace{-2mm}
\begin{listverse}
\item[\vref{Ge 41:2}] voici, sept jeunes v. belles à voir,
\item[\vref{No 19:2}] t'amènent une jeune v. rousse, entière, sans
\item[\vref{Es 7:21}] nourrira une jeune v. et deux brebis.
\item[\vref{Es 11:7}] La jeune v. paîtra avec l'ourse, leurs petits auront
\end{listverse}

\ConcordanceEntry{Vague (une)}
\vspace{-2mm}
\begin{listverse}
\item[\vref{Ps 42:8}] ondées ; ttes tes v. et tes flots
\item[\vref{Ps 89:10}] mer ; qnd ses v. s'élèvent, tu les
\item[\vref{Ps 93:4}] que les fortes v. de la mer.
\item[\vref{Ps 107:25}] qui élève les v. de la mer.
\item[\vref{Jé 5:22}] passera pas ; ses v. s'agitent, mais elles
\item[\vref{Jon 2:4}] et ttes tes v. ont passé sur
\item[\vref{Ac 27:41}] se brisait par la violence des v.
\item[\vref{Jud 1:13}] des v. impétueuses de la mer, rejetant l'écume
\end{listverse}

\ConcordanceEntry{Vaillant}
\vspace{-2mm}
\begin{listverse}
\item[\vref{Ex 15:3}] Yahweh est un v. guerrier, son Nom
\item[\vref{Jos 6:2}] Jéricho et son roi, ses hommes v.
\item[\vref{Jg 6:12}] Très fort et v. héros, Yahweh est
\item[\vref{2 S 23:8}] les noms des v. hommes qui étaient
\item[\vref{Job 38:3}] reins com. un v. hom. ; je t'interrogerai,
\item[\vref{Ps 19:6}] sentier avec la joie d'un hom. v. ;
\item[\vref{Ps 89:20}] faveur d'un hom. v. ; j'ai élevé l'élu
\item[\vref{Ca 3:7}] y a soixante v. hommes, des plus
\item[\vref{Es 42:13}] com. un hom. v., il réveille sa
\item[\vref{Da 11:3}] il s'élèvera un v. roi, qui dominera
\item[\vref{Joë 2:7}] com. des hommes v., et montent sur
\item[\vref{Joë 3:11}] ô Yahweh, fais descendre tes hommes v. !
\item[\vref{Am 2:16}] d'entre les hommes v. s'enfuira tt nu
\item[\vref{So 1:14}] sa force. Là sont les hommes v.
\item[\vref{Za 10:5}] seront com. des v. hommes foulant la
\end{listverse}

\ConcordanceEntry{Vain}
\vspace{-2mm}
\begin{listverse}
\item[\vref{Ex 20:7}] ton Dieu, en v. ; car Yahweh ne
\item[\vref{Job 35:13}] que c'est en v. ; que Dieu n'écoute
\item[\vref{Ps 2:1}] pourquoi les peuples projettent-ils des choses v. ?
\item[\vref{Ps 73:13}] c'est dc en v. que j'ai purifié
\item[\vref{Ps 119:37}] vue des choses v. ; fais-moi vivre ds
\item[\vref{Ps 127:2}] C'est en v. que vs. vs. levez de grand
\item[\vref{Pr 1:17}] Car c'est en v. qu'on jette le
\item[\vref{Pr 30:9}] ne prenne en v. le nom de
\item[\vref{Ec 6:4}] est venu en v., et s'en va
\item[\vref{Es 49:4}] J'ai travaillé en v., j'ai consumé ma
\item[\vref{Es 65:23}] travailleront plus en v., et ils n'engendreront
\item[\vref{Mal 1:10}] n'allumiez pas en v. le feu sur
\item[\vref{Mal 3:14}] dit : C'est en v. que l'on sert
\item[\vref{Mt 15:9}] ils m'honorent en v., en enseignant des
\item[\vref{Mc 7:7}] C'est en v. qu'ils m'honorent, en enseignant des doctrines
\item[\vref{Ro 13:4}] n'est pas en v. qu'il porte l'épée,
\item[\vref{1 Co 15:2}] moins que vs. n'ayez cru en v.
\item[\vref{1 Co 15:10}] n'a pas été v., mais j'ai travaillé
\item[\vref{1 Co 15:14}] prédication est dc v., et votre foi
\item[\vref{Ga 2:2}] pas courir ou avoir couru en v.
\item[\vref{Ga 3:4}] tant souffert en v. ? Si toutefois c'est
\item[\vref{Ph 2:16}] pas couru en v., ni travaillé en
\item[\vref{1 Ti 1:6}] écartés ds de v. discours,
\item[\vref{Ja 1:26}] la religion d'un tel hom. est v.
\item[\vref{Ja 4:5}] l'Ecriture parle en v. ? L'Esprit qui habite
\item[\vref{1 Pi 1:18}] rachetés de votre v. manière de vivre,
\end{listverse}

\ConcordanceEntry{Vaincre}
\vspace{-2mm}
\begin{listverse}
\item[\vref{Ge 32:25}] pouvait pas le v., il frappa à
\item[\vref{Jg 16:5}] comment pourrions-ns. le v. ; et ns. le
\item[\vref{Ps 129:2}] jeunesse, mais ils ne m'ont pas v.
\item[\vref{Jé 20:7}] et tu m'as v.. Je suis un
\item[\vref{Jé 20:10}] et ns. le v., ns. tirerons vengeance
\item[\vref{Lu 11:22}] survient et le v., il lui enlève
\item[\vref{Jn 16:33}] bon courage, j'ai v. le monde.
\item[\vref{2 Pi 2:20}] nouveau et sont v. par elles, lr.
\item[\vref{1 Jn 2:13}] que vs. avez v. l'esprit du malin.
\item[\vref{Ap 2:7}] A celui qui v., je lui donnerai
\item[\vref{Ap 2:26}] celui qui aura v., et qui aura
\item[\vref{Ap 11:7}] la guerre, les v., et les tuera.
\item[\vref{Ap 12:11}] Et ils l'ont v. à cause du
\item[\vref{Ap 15:2}] ceux qui avaient v. la bête et
\item[\vref{Ap 17:14}] et l'Agneau les v., parce qu'il est
\item[\vref{Ap 21:7}] Celui qui v. héritera ttes choses ;
\end{listverse}

\ConcordanceEntry{Vainqueur}
\vspace{-2mm}
\begin{listverse}
\item[\vref{Ge 14:17}] d'Abram qui revenait v. de Kedorlaomer, et
\item[\vref{Ge 32:28}] as été le v. en luttant avec
\item[\vref{No 13:30}] ce pays, car ns. y serons v. !
\item[\vref{1 S 14:47}] où il se tournait, il était v.
\item[\vref{2 S 1:1}] qui était revenu v. des Amalécites, resta
\item[\vref{Os 12:4}] force, il fut v. en luttant avec
\item[\vref{Za 9:9}] est juste et v., il est monté
\item[\vref{Ro 8:37}] sommes plus que v. par celui qui
\item[\vref{Ap 6:2}] est sortit en v. pour vaincre.
\end{listverse}

\ConcordanceEntry{Valeur}
\vspace{-2mm}
\begin{listverse}
\item[\vref{Ge 23:9}] cède contre sa v. en argent, afin
\item[\vref{2 Ch 3:8}] fin, pour une v. de six cents
\item[\vref{2 Ch 13:7}] des gens sans v., des fils de
\item[\vref{Job 20:18}] selon sa juste v., et ne s'en
\item[\vref{Job 28:13}] connaît pas sa v., et elle ne
\item[\vref{Ps 44:13}] tu ne l'estimes pas d'une grande v.
\item[\vref{Ac 19:19}] en estima la v. à cinquante mille
\item[\vref{Ga 4:24}] faits ont une v. allégorique, car ces
\item[\vref{Ga 5:6}] prépuce n'ont de v., mais seulement la
\end{listverse}

\ConcordanceEntry{Valoir}
\vspace{-2mm}
\begin{listverse}
\item[\vref{Pr 3:15}] ttes tes choses désirables ne la v. point.
\item[\vref{Ec 4:9}] Deux v. mieux qu'un, car ils ont un
\item[\vref{Es 7:23}] aura mille vignes, v. mille sicles d'argent,
\item[\vref{Ez 27:17}] avec toi, faisant v. ton commerce en
\item[\vref{Mt 25:16}] et les fit v., et gagna cinq
\item[\vref{Lu 12:7}] dc pas ; vs. v. plus que beaucoup
\item[\vref{Lu 12:24}] nourrit. Combien ne v.-vs. pas plus
\item[\vref{Lu 19:13}] lr. dit : Faites-les v. jusqu'à ce que
\item[\vref{Lu 19:15}] de connaître comment chacun l'avait fait v.
\item[\vref{2 Pi 2:21}] Car mieux v. pour eux n'avoir
\end{listverse}

\ConcordanceEntry{Van}
\vspace{-2mm}
\begin{listverse}
\item[\vref{Es 30:24}] vanné avec la pelle et le v.
\item[\vref{Jé 15:7}] vanne avec un v. aux portes du
\item[\vref{Mt 3:12}] Il a son v. à la main, et il nettoiera
\end{listverse}

\ConcordanceEntry{Vanité}
\vspace{-2mm}
\begin{listverse}
\item[\vref{Job 7:16}] car mes jours ne sont que v.
\item[\vref{Ps 4:3}] qnd aimerez-vs. la v. et chercherez-vs. le
\item[\vref{Ps 39:6}] hom. quoiqu'il soit debout n'est que v.. Sélah.
\item[\vref{Ps 39:12}] cher. Certainement, tt hom. n'est que v.. Sélah.
\item[\vref{Ps 60:13}] délivrance qu'on attend de l'hom. est v.
\item[\vref{Ps 62:10}] Oui, v., les fils de l'hom. ! Mensonge, les
\item[\vref{Ps 94:11}] des hommes qui ne sont que v.
\item[\vref{Pr 30:8}] de moi la v. et la parole
\item[\vref{Ec 1:2}] V. des vanités, dit l'Ecclésiaste, vanité des
\item[\vref{Ec 6:2}] Cela est une v. et un mal
\item[\vref{Ec 6:11}] a beaucoup de v.. Quel avantage en
\item[\vref{Es 30:7}] l'Egypte n'est que v. et néant ; c'est
\item[\vref{Es 41:29}] ne sont que v., leurs idoles de
\item[\vref{Es 44:9}] sont ts que v., et leurs choses
\item[\vref{Es 57:13}] emmènera ts, la v. les enlèvera ; mais
\item[\vref{Jé 2:5}] marché après la v. et soient devenus
\item[\vref{Jé 10:3}] ne sont que v.. On coupe le
\item[\vref{Ez 13:7}] des visions de v., et prononcé des
\item[\vref{Za 10:2}] consolent par la v.. C'est pourquoi ils
\item[\vref{Ro 8:20}] soumise à la v., non de son
\item[\vref{1 Co 5:6}] Votre v. est mal fondée. Ne savez-vs. pas
\item[\vref{Ep 4:17}] qui suivent la v. de leurs pensées.
\item[\vref{2 Pi 2:18}] fort enflés de v., ils amorcent par
\end{listverse}

\ConcordanceEntry{Vanter}
\vspace{-2mm}
\begin{listverse}
\item[\vref{Pr 20:14}] puis il s'en va, et se v.
\item[\vref{Pr 25:14}] Celui qui se v. d'une fausse libéralité,
\item[\vref{Pr 27:1}] Ne te v. pas du lendemain, car tu ne
\item[\vref{Ja 3:5}] elle peut se v. de grandes choses.
\end{listverse}

\ConcordanceEntry{Vapeur}
\vspace{-2mm}
\begin{listverse}
\item[\vref{Ge 2:6}] il monta une v. de la terre
\item[\vref{Job 36:27}] pluie selon la v. d'eau qui la
\item[\vref{Job 36:33}] bétail ressent la v. qui monte.
\item[\vref{Es 18:4}] et par la v. de la rosée,
\item[\vref{Ac 2:19}] feu, et une v. de fumée.
\item[\vref{Ja 4:14}] n'est certes qu'une v. qui paraît pour
\end{listverse}

\ConcordanceEntry{Vase}
\vspace{-2mm}
\begin{listverse}
\item[\vref{Ex 16:33}] Aaron : Prends un v., et mets-y un
\item[\vref{Lé 6:21}] Et le v. de terre ds lequel on l'aura
\item[\vref{2 R 4:3}] Va, demande des v. ds la rue
\item[\vref{Job 28:17}] échange pour un v. de fin or.
\item[\vref{Ps 2:9}] pièces com. un v. de potier.
\item[\vref{Pr 26:23}] sont com. un v. de terre recouvert
\item[\vref{Ec 12:8}] détache, que le v. d'or se brise,
\item[\vref{Es 45:9}] façonné ! Que le v. plaide contre les
\item[\vref{Es 52:11}] qui portez les v. de Yahweh.
\item[\vref{Jé 18:4}] Et le v. qu'il faisait avec l'argile qu'il tenait
\item[\vref{Da 5:3}] furent apportés les v. d'or qui avaient
\item[\vref{Mt 25:4}] l'huile ds leurs v. avec leurs lampes.
\item[\vref{Mt 26:7}] lui tenant un v. d'albâtre, plein d'un
\item[\vref{Jn 19:29}] avait là un v. plein de vinaigre.
\item[\vref{Ac 9:15}] il m'est un v. que j'ai choisi,
\item[\vref{Ac 10:11}] ouvert, et un v. descendant sur lui
\item[\vref{Ac 11:5}] une vision. Un v. semblable à une
\item[\vref{Ro 9:20}] contre Dieu ? Le v. d'argile dira-t-il à
\item[\vref{Ro 9:21}] de terre un v. d'honneur et un
\item[\vref{Ro 9:22}] grande patience les v. de colère, préparés
\item[\vref{Ro 9:23}] gloire ds les v. de miséricorde, qu'il
\item[\vref{2 Co 4:7}] trésor ds des v. de terre, afin
\item[\vref{2 Ti 2:21}] il sera un v. d'honneur, sanctifié et
\item[\vref{Hé 9:4}] laquelle était le v. d'or où était
\item[\vref{1 Pi 3:7}] com. avec un v. plus fragile, c'est-à-dire
\end{listverse}

\ConcordanceEntry{Vasthi}
\vspace{-2mm}
\begin{listverse}
\item[\vref{Est 1:9}] La reine V. fit aussi un festin aux femmes
\item[\vref{Est 1:12}] Mais la reine V. refusa de venir
\item[\vref{Est 1:19}] à savoir que V. ne vienne plus
\end{listverse}

\ConcordanceEntry{Veau}
\vspace{-2mm}
\begin{listverse}
\item[\vref{Ex 32:4}] en fit un v. en métal fondu.
\item[\vref{1 R 12:28}] conseil, fit deux v. d'or, et dit
\item[\vref{Ez 1:7}] d'un pied de v., ils étincelaient com.
\item[\vref{Os 8:5}] Samarie, ton v. t'a chassée loin !
\item[\vref{Lu 15:27}] a tué le v. gras, parce qu'il
\item[\vref{Ac 7:41}] ces jours-là un v., et ils offrirent
\item[\vref{Ap 4:7}] semblable à un v. ; le troisième animal
\end{listverse}

\ConcordanceEntry{Veille}
\vspace{-2mm}
\begin{listverse}
\item[\vref{Ps 63:7}] toi durant les v. de la nuit.
\item[\vref{Ps 90:4}] et com. une v. de la nuit.
\item[\vref{Ps 119:148}] ont devancé les v. de la nuit
\item[\vref{Mt 14:25}] vers la quatrième v. de la nuit,
\item[\vref{Mt 24:43}] savait à quelle v. de la nuit
\item[\vref{Lu 2:8}] troupeau pendant les v. de la nuit.
\item[\vref{Lu 12:38}] à la seconde v. ou à la
\item[\vref{2 Co 6:5}] travaux, ds les v., ds les jeûnes,
\item[\vref{2 Co 11:27}] ds de fréquentes v., ds la faim,
\end{listverse}

\ConcordanceEntry{Veiller}
\vspace{-2mm}
\begin{listverse}
\item[\vref{Ge 31:49}] dit : Que Yahweh v. sur moi et
\item[\vref{Ps 5:4}] me tourne vers toi, et je v.
\item[\vref{Pr 8:34}] qui m'écoute, qui v. chaque jour à
\item[\vref{Jé 31:28}] que com. j'ai v. sur eux pour
\item[\vref{Jé 44:27}] Voici, je v. sur eux pour
\item[\vref{Da 4:17}] de ceux qui v., cette résolution est
\item[\vref{Na 2:2}] garde la forteresse ! V. sur la route !
\item[\vref{Mal 2:12}] cela, celui qui v. et qui répond,
\item[\vref{Mt 24:42}] V. dc, car vs. ne savez pas
\item[\vref{Mt 26:41}] V. et priez, afin que vs. ne
\item[\vref{Mc 13:37}] dis, je le dis à ts : V. !
\item[\vref{Ac 20:31}] C'est pourquoi v., vs. souvenant que
\item[\vref{1 Co 16:13}] V., soyez fermes ds la foi, agissez
\item[\vref{Ep 6:18}] et de supplications, v. à cela avec
\item[\vref{1 Ti 4:16}] V. sur toi-mm et sur la doctrine ;
\item[\vref{2 Ti 4:5}] Mais toi, v. en ttes choses,
\item[\vref{Hé 12:15}] V. à ce que personne ne se
\item[\vref{Hé 13:17}] soumis, car ils v. pour vos âmes,
\item[\vref{1 Pi 5:8}] Soyez sobres et v. : Car le diable,
\item[\vref{Ap 3:3}] Si tu ne v. pas, je viendrai
\item[\vref{Ap 16:15}] est celui qui v. et qui garde
\end{listverse}

\ConcordanceEntry{Vendre}
\vspace{-2mm}
\begin{listverse}
\item[\vref{Ge 31:15}] il ns. a v., et mm il
\item[\vref{Ge 37:28}] citerne, et le v. pour vingt pièces
\item[\vref{Ge 45:5}] que vs. m'avez v. pour être mené
\item[\vref{Lé 25:39}] qu'il se sera v. à toi, tu
\item[\vref{De 32:30}] Rocher les avait v., et que Yahweh
\item[\vref{Jg 4:2}] pourquoi Yahweh les v. entre les mains
\item[\vref{1 R 21:20}] que tu t'es v. pour faire ce
\item[\vref{Né 5:8}] qui avaient été v. aux nations, et
\item[\vref{Né 10:31}] sabbat, pour les v. ; d'abandonner la septième
\item[\vref{Est 7:4}] ns. avons été v., mon peuple et
\item[\vref{Ps 44:13}] Tu as v. ton peuple pour rien, et tu
\item[\vref{Es 52:3}] Vous avez été v. pour rien, et
\item[\vref{Joë 3:3}] prostituée, ils ont v. la jeune fille
\item[\vref{Mt 13:44}] joie, il va v. tt ce qu'il
\item[\vref{Mt 19:21}] être parfait, va, v. ce que tu
\item[\vref{Ac 2:45}] ils v. leurs possessions et leurs biens, et
\item[\vref{Ac 5:8}] Dis-moi, avez-vs. autant v. le champ ? Et
\item[\vref{Ro 7:14}] je suis charnel, v. au péché.
\item[\vref{1 Co 10:25}] ce qui se v. au marché, sans
\item[\vref{Ap 13:17}] puisse acheter ni v., sans avoir la
\end{listverse}

\ConcordanceEntry{Vengeance}
\vspace{-2mm}
\begin{listverse}
\item[\vref{Lé 19:18}] n'useras point de v., et tu ne
\item[\vref{De 32:35}] A moi la v. et la rétribution, le temps où
\item[\vref{Jg 5:2}] fait de telles v. en Israël, et
\item[\vref{2 S 22:48}] qui me donne v., qui m'assujettit les
\item[\vref{Ps 79:10}] Dieu ? Que la v. du sang de
\item[\vref{Ps 94:1}] Yahweh, Dieu des v. ! Dieu des vengeances,
\item[\vref{Pr 6:34}] pas l'adultère au jour de la v.
\item[\vref{Es 35:4}] votre Dieu, la v. viendra, la rétribution
\item[\vref{Es 63:4}] jour de la v. était ds mon
\item[\vref{Jé 46:10}] un jour de v., où il se
\item[\vref{Jé 50:15}] c'est ici la v. de Yahweh. Vengez-vs.
\item[\vref{Ez 25:17}] eux de grandes v. par des châtiments
\item[\vref{Mi 5:14}] Et j'exercerai ma v. avec colère et
\item[\vref{Lu 21:22}] des jours de v., afin que ttes
\item[\vref{2 Co 10:6}] prêts à tirer v. de tte désobéissance,
\item[\vref{1 Th 4:6}] le Seign. tire v. de ttes ces
\item[\vref{2 Th 1:8}] pour exercer la v. contre ceux qui
\item[\vref{Hé 10:30}] moi que la v. appartient, et je
\end{listverse}

\ConcordanceEntry{Venger}
\vspace{-2mm}
\begin{listverse}
\item[\vref{Ge 4:24}] si Caïn est v. sept fois davantage,
\item[\vref{Ex 21:20}] main, on le v., on le vengera.
\item[\vref{De 32:43}] peuple ! Car il v. le sang de
\item[\vref{Jg 16:28}] coup, je me v. des Philistins pour
\item[\vref{Ps 18:48}] moyens de me v. et qui m'assujettit
\item[\vref{Ps 149:7}] pour se v. des nations, pour châtier les peuples,
\item[\vref{Es 1:24}] et je me v. de mes ennemis.
\item[\vref{Jé 5:9}] âme ne se v.-elle pas d'une
\item[\vref{Na 1:2}] jaloux, il se v., Yahweh se venge,
\item[\vref{Lu 18:8}] bientôt il les v.. Mais qnd le
\item[\vref{Ro 12:19}] Ne vs. v. pas vs.-mêmes, mes bien-aimés, mais laissez
\item[\vref{Ap 6:10}] pas et ne v.-tu pas notre
\item[\vref{Ap 19:2}] parce qu'il a v. le sang de
\end{listverse}

\ConcordanceEntry{Vengeur}
\vspace{-2mm}
\begin{listverse}
\item[\vref{No 35:12}] refuge contre le v., afin que le
\item[\vref{No 35:27}] et si le v. du sang le rencontre hors des
\item[\vref{De 19:12}] les mains du v. de sang, afin
\item[\vref{2 S 14:11}] afin que le v. de sang n'augmente
\end{listverse}

\ConcordanceEntry{Venin}
\vspace{-2mm}
\begin{listverse}
\item[\vref{De 32:24}] bêtes et le v. des serpents qui
\item[\vref{De 32:33}] vin est un v. de dragon, et
\item[\vref{Job 6:4}] en suce le v. ; les terreurs de
\item[\vref{Job 20:16}] Il sucera du v. d'aspic, la langue
\item[\vref{Ps 58:5}] Ils ont un v. semblable au venin
\item[\vref{Ps 140:3}] y a du v. de vipère sous
\item[\vref{La 3:5}] m'a environné de v. et de peine.
\item[\vref{Ez 16:36}] Parce que ton v. s'est répandu, et
\item[\vref{Ro 3:13}] y a du v. d'aspic sous leurs
\item[\vref{Ja 3:8}] peut réprimer ; elle est pleine d'un v. mortel.
\end{listverse}

\ConcordanceEntry{Venir}
\vspace{-2mm}
\begin{listverse}
\item[\vref{Ge 2:19}] il les fit v. vers Adam pour
\item[\vref{Ge 6:17}] voici, je ferai v. un déluge d'eau
\item[\vref{Ex 33:15}] ta présence ne v. avec ns., ne
\item[\vref{No 10:29}] vs. le donnerai. V. avec ns., et
\item[\vref{1 S 16:12}] le fit dc v.. Il était roux,
\item[\vref{1 Ch 29:14}] Car ttes choses v. de toi, et
\item[\vref{Ps 38:23}] Hâte-toi de v. à mon secours !
\item[\vref{Ps 105:28}] ténèbres et fit v. l'obscurité ; et ils
\item[\vref{Pr 19:15}] La paresse fait v. le sommeil, et
\item[\vref{Es 1:18}] V. mntnt, dit Yahweh, et débattons nos
\item[\vref{Es 28:29}] Cela aussi v. de Yahweh des
\item[\vref{Es 53:6}] Yahweh a fait v. sur lui l'iniquité
\item[\vref{Da 9:12}] il a fait v. sur ns. un
\item[\vref{Za 3:8}] voici, je ferai v. mon serviteur, le
\item[\vref{Mt 10:23}] que le Fils de l'hom. sera v.
\item[\vref{Mt 11:3}] celui qui devait v., ou devons-ns. en
\item[\vref{Mt 11:14}] lui qui est l'Elie qui devait v.
\item[\vref{Mt 11:28}] V. à moi, vs. ts qui êtes
\item[\vref{Mt 16:24}] Si quelqu'un veut v. après moi, qu'il
\item[\vref{Mt 17:12}] qu'Elie est déjà v., et ils ne
\item[\vref{Mt 19:14}] lr. dit : Laissez v. à moi les
\item[\vref{Mt 25:5}] l'époux tardait à v., elles s'assoupirent et
\item[\vref{Mc 1:45}] déserts, et l'on v. à lui de
\item[\vref{Mc 10:14}] lr. dit : Laissez v. à moi les
\item[\vref{Jn 2:4}] fem. ? Mon heure n'est pas encore v.
\item[\vref{Jn 4:23}] Mais l'heure v., et elle est
\item[\vref{Jn 6:37}] ne mettrai pas dehors celui qui v. à moi ;
\item[\vref{Jn 6:44}] Nul ne peut v. à moi, si
\item[\vref{Jn 8:21}] ne pouvez pas v. là où je
\item[\vref{Jn 10:12}] les brebis, voit v. le loup, abandonne
\item[\vref{Ac 10:5}] Joppé, et fais v. Simon, surnommé Pierre.
\item[\vref{Ro 5:14}] la figure de celui qui devait v.
\item[\vref{2 Th 2:2}] lettre qu'on dirait v. de ns., com.
\item[\vref{Hé 10:37}] celui qui doit v., viendra, et il
\item[\vref{2 Pi 2:5}] et a fait v. le déluge sur
\item[\vref{Ap 3:9}] Voici, je ferai v. ceux de la
\item[\vref{Ap 22:12}] Voici, je v. à tte vitesse,
\end{listverse}

\ConcordanceEntry{Vent}
\vspace{-2mm}
\begin{listverse}
\item[\vref{Ge 3:8}] ils entendirent au v. du jour la
\item[\vref{Ge 8:1}] fit passer un v. sur la terre,
\item[\vref{1 R 19:11}] eut un grand v. impétueux qui déchirait
\item[\vref{Ps 11:6}] du soufre ; un v. brûlant, c'est le
\item[\vref{Ps 78:26}] les cieux le v. d'orient et il
\item[\vref{Ps 104:4}] Il fait des v. ses messagers, et
\item[\vref{Ps 135:7}] il tire le v. hors de ses
\item[\vref{Pr 27:16}] retenir, retient le v., et elle se
\item[\vref{Pr 30:4}] a recueilli le v. ds le creux
\item[\vref{Ec 5:15}] avantage a-t-il d'avoir travaillé pour du v. ?
\item[\vref{Ec 11:4}] prend garde au v., ne sèmera point ;
\item[\vref{Ec 11:5}] le chemin du v., ni comment se
\item[\vref{Es 26:18}] com. enfanté du v.. Nous ne saurions
\item[\vref{Ez 1:4}] et voici, un v. impétueux vint du
\item[\vref{Da 7:2}] voici, les quatre v. des cieux se
\item[\vref{Os 8:7}] qu'ils sèment du v., ils moissonneront la
\item[\vref{Am 4:13}] et créé le v., et qui déclare
\item[\vref{Jon 1:4}] lever un grand v. sur la mer,
\item[\vref{Jon 4:8}] Dieu prépara un v. chaud d'orient qu'on
\item[\vref{Mt 7:25}] sont venus, les v. ont soufflé contre
\item[\vref{Mt 8:27}] obéissent mm les v. et la mer ?
\item[\vref{Mt 14:24}] flots ; car le v. était contraire.
\item[\vref{Jn 3:8}] Le v. souffle où il veut, et tu
\item[\vref{Ac 2:2}] com. celui d'un v. qui souffle avec
\item[\vref{Ac 27:7}] parce que le v. ne ns. permettait
\item[\vref{Ja 1:6}] poussé çà et là par le v.
\item[\vref{Ap 7:1}] retenaient les quatre v. de la terre,
\end{listverse}

\ConcordanceEntry{Ventre}
\vspace{-2mm}
\begin{listverse}
\item[\vref{Ge 3:14}] marcheras sur ton v., et tu mangeras
\item[\vref{Ge 25:22}] heurtaient ds son v., et elle dit :
\item[\vref{Ge 25:24}] y avait deux jumeaux ds son v.
\item[\vref{Job 1:21}] sorti nu du v. de ma mère,
\item[\vref{Ps 17:14}] tu remplis lr. v. de tes biens ;
\item[\vref{Ps 22:10}] tiré hors du v. de ma mère,
\item[\vref{Ps 127:3}] le fruit du v. est une récompense
\item[\vref{Ec 5:14}] est sorti du v. de sa mère,
\item[\vref{Ec 11:5}] os ds le v. de celle qui
\item[\vref{Es 49:1}] appelé dès le v., il a fait
\item[\vref{Jé 1:5}] formé ds le v. de ta mère,
\item[\vref{Am 1:13}] ont fendu le v. des femmes enceintes
\item[\vref{Jon 2:1}] fut ds le v. du poisson trois
\item[\vref{Mt 12:40}] nuits ds le v. d'un grand poisson,
\item[\vref{Mt 15:17}] va ds le v., puis est jeté
\item[\vref{Mt 19:12}] nés dès le v. de lr. mère ;
\item[\vref{Lu 1:31}] concevras ds ton v., et tu enfanteras
\item[\vref{Lu 11:27}] Béni est le v. qui t'a porté,
\item[\vref{Ro 16:18}] mais lr. propre v. ; et, par de
\item[\vref{1 Co 6:13}] sont pour le v., et le ventre
\item[\vref{Ga 1:15}] choisi dès le v. de ma mère,
\item[\vref{Ph 3:19}] pour dieu lr. v., et dont la
\end{listverse}

\ConcordanceEntry{Venue}
\vspace{-2mm}
\begin{listverse}
\item[\vref{Os 6:3}] le connaître ; sa v. est aussi certaine
\item[\vref{Ac 13:24}] Avant la v. de Jésus, Jean avait prêché le
\item[\vref{2 Pi 3:12}] en hâtant la v. du jour de
\item[\vref{1 Jn 4:3}] avez appris la v., et qui mntnt
\end{listverse}

\ConcordanceEntry{Ver}
\vspace{-2mm}
\begin{listverse}
\item[\vref{Job 25:6}] qui n'est qu'un v. ; et le fils
\item[\vref{Ps 22:7}] je suis un v. et non un
\item[\vref{Es 66:24}] moi ; car lr. v. ne mourra point,
\item[\vref{Jon 4:7}] jour monta, un v. qui frappa le
\item[\vref{Mc 9:48}] là où lr. v. ne meurt pas,
\item[\vref{Ja 5:2}] vos vêtements sont rongés par les v.,
\end{listverse}

\ConcordanceEntry{Verge}
\vspace{-2mm}
\begin{listverse}
\item[\vref{Ex 4:2}] ds ta main ? Il répondit : Une v.
\item[\vref{Ex 4:20}] prit aussi la v. de Dieu ds
\item[\vref{No 17:8}] témoignage, voici, la v. d'Aaron, avait fleuri,
\item[\vref{No 20:11}] rocher avec sa v. et il en
\item[\vref{2 S 7:14}] châtierai avec une v. d'hommes et avec
\item[\vref{Job 9:34}] ôte dc sa v. de dessus moi,
\item[\vref{Ps 89:33}] punirai de la v. lr. transgression, et
\item[\vref{Pr 13:24}] qui épargne sa v. hait son fils,
\item[\vref{Pr 22:15}] enfant, mais la v. de la correction
\item[\vref{Pr 23:14}] frapperas avec la v., et tu sauveras
\item[\vref{Es 10:5}] Malheur à l'Assyrie, v. de ma colère !
\item[\vref{Ez 21:15}] réjouirons-ns. ? C'est la v. de mon fils,
\item[\vref{Za 11:10}] je pris ma v., appelée Grâce, et
\item[\vref{Mt 10:17}] vs. battront de v. ds leurs synagogues.
\item[\vref{Jn 18:22}] coup de sa v. à Jésus, en
\item[\vref{Ac 5:40}] firent battre de v., ils lr. défendirent
\item[\vref{1 Co 4:21}] vs. avec la v., ou avec charité
\item[\vref{2 Co 11:25}] été battu de v. trois fois, j'ai
\item[\vref{Hé 9:4}] manne, et la v. d'Aaron qui avait
\item[\vref{Hé 12:6}] frappe de la v. ts ceux qu'il
\end{listverse}

\ConcordanceEntry{Véritable}
\vspace{-2mm}
\begin{listverse}
\item[\vref{Ps 111:7}] équité. [Nun.] Tous ses commandements sont v.,
\item[\vref{Pr 12:17}] prononce des choses v. rend un témoignage
\item[\vref{Pr 14:5}] Le témoin v. ne ment jamais,
\item[\vref{Jé 42:5}] ns. un témoin v. et fidèle, si
\item[\vref{Da 4:37}] les œuvres sont v. et les voies
\item[\vref{Mt 22:16}] que tu es v., que tu enseignes
\item[\vref{Jn 1:9}] lumière était la v. lumière, qui, en
\item[\vref{Jn 7:18}] l'a envoyé est v., et il n'y
\item[\vref{Jn 7:28}] m'a envoyé est v., et vs. ne
\item[\vref{Hé 9:24}] la figure du v., mais il est
\item[\vref{Hé 11:18}] d'Isaac seront ta v. postérité.
\item[\vref{1 Pi 5:12}] ds laquelle vs. êtes est la v.
\item[\vref{1 Jn 2:27}] choses, qu'elle est v. et n'est pas
\item[\vref{1 Jn 5:20}] pour connaître le V. ; et ns. sommes
\item[\vref{3 Jn 1:12}] vs. savez que notre témoignage est v.
\item[\vref{Ap 3:7}] Saint et le V., qui a la
\item[\vref{Ap 3:14}] témoin fidèle et v., le commencement de
\item[\vref{Ap 6:10}] es saint et v., ne jugeras-tu pas
\item[\vref{Ap 15:3}] sont justes et v., ô Roi des
\item[\vref{Ap 16:7}] tes jugements sont v. et justes.
\item[\vref{Ap 19:9}] aussi : Ces paroles de Dieu sont v.
\item[\vref{Ap 21:5}] ces paroles sont v. et certaines.
\end{listverse}

\ConcordanceEntry{Vérité}
\vspace{-2mm}
\begin{listverse}
\item[\vref{2 Ch 18:15}] dire que la v. au Nom de
\item[\vref{Né 9:13}] des lois de v., des statuts et
\item[\vref{Ps 25:5}] marcher selon la v., et instruis-moi, car
\item[\vref{Ps 51:8}] plaisir à la v. au fond du
\item[\vref{Ps 57:11}] cieux, et ta v. jusqu'aux nues.
\item[\vref{Ps 119:160}] parole est la v., et ttes les
\item[\vref{Ps 132:11}] a juré la v. à David, et
\item[\vref{Pr 23:23}] Achète la v., et ne la
\item[\vref{Es 53:4}] En v., il a porté nos maladies, et
\item[\vref{Es 59:14}] éloignée ; car la v. est tombée par
\item[\vref{Es 59:15}] Même la v. a disparu, et quiconque se retire
\item[\vref{Da 8:12}] corne jeta la v. par terre, et
\item[\vref{Da 10:21}] le livre de v.. Et il n'y
\item[\vref{Za 8:16}] chacun dise la v. à son prochain ;
\item[\vref{Mal 2:6}] loi de la v. était ds sa
\item[\vref{Mt 5:18}] le dis en v., tant que le
\item[\vref{Mt 19:28}] le dis en v., qnd le Fils
\item[\vref{Mt 21:21}] le dis en v., si vs. aviez
\item[\vref{Jn 1:17}] grâce et la v. sont venues par
\item[\vref{Jn 8:32}] Vous connaîtrez la v., et la vérité
\item[\vref{Jn 14:6}] le chemin, la v., et la vie.
\item[\vref{Jn 16:13}] lui, l'Esprit de v., il vs. conduira
\item[\vref{Jn 17:17}] Sanctifie-les par ta v. ; ta parole est
\item[\vref{Jn 17:19}] qu'eux aussi soient sanctifiés par la v.
\item[\vref{Jn 18:37}] témoignage à la v.. Quiconque est de
\item[\vref{Jn 18:38}] Qu'est-ce que la v. ? Et qnd il
\item[\vref{Ro 1:18}] des hommes qui retiennent injustement la v. captive,
\item[\vref{2 Co 13:8}] pouvoir contre la v., mais ns. n'en
\item[\vref{Ga 2:5}] afin que la v. de l'Evangile soit
\item[\vref{Ep 4:15}] que, suivant la v. avec la charité,
\item[\vref{Ep 4:25}] mensonge, parlez en v. chacun avec son
\item[\vref{2 Th 2:10}] l'amour de la v. pour être sauvés.
\item[\vref{1 Ti 2:4}] viennent à la connaissance de la v.
\item[\vref{1 Ti 3:15}] la colonne et l'appui de la v.
\item[\vref{Ja 1:18}] parole de la v., afin que ns.
\item[\vref{Ja 5:19}] loin de la v., et qu'un autre
\item[\vref{1 Pi 1:22}] obéissant à la v. par le Saint-Esprit,
\item[\vref{2 Pi 1:12}] fondés ds la v. présente.
\item[\vref{2 Pi 2:2}] voie de la v. sera blasphémée.
\item[\vref{1 Jn 1:8}] ns.-mêmes, et la v. n'est point en
\item[\vref{2 Jn 1:4}] marchent ds la v., selon le commandement
\end{listverse}

\ConcordanceEntry{Vermisseau}
\vspace{-2mm}
\begin{listverse}
\item[\vref{Job 4:19}] sont consumés à la rencontre d'un v. !
\item[\vref{Job 25:6}] fils d'un hom., qui n'est qu'un v.  !
\item[\vref{Es 41:14}] pas peur, toi v. de Jacob, et
\end{listverse}

\ConcordanceEntry{Verre}
\vspace{-2mm}
\begin{listverse}
\item[\vref{Mt 10:42}] boire seulement un v. d'eau froide à
\item[\vref{Ap 4:6}] une mer de v. semblable à du
\item[\vref{Ap 15:2}] une mer de v. mêlée de feu,
\item[\vref{Ap 21:18}] semblable à du v. fort transparent.
\item[\vref{Ap 21:21}] pur, com. du v. transparent.
\end{listverse}

\ConcordanceEntry{Verrou}
\vspace{-2mm}
\begin{listverse}
\item[\vref{De 33:25}] Tes v. seront de fer et d'airain, et
\item[\vref{Jg 3:23}] de la chambre et tira le v.
\item[\vref{Pr 18:19}] sont com. les v. d'un palais.
\item[\vref{Ca 5:5}] mes doigts sur la poignée du v.
\item[\vref{Na 3:13}] tes ennemis ; le feu consumera tes v.
\end{listverse}

\ConcordanceEntry{Vertu}
\vspace{-2mm}
\begin{listverse}
\item[\vref{Ge 17:7}] alliance éternelle en v. de laquelle je
\item[\vref{Lu 1:78}] notre Dieu, en v. de laquelle le
\item[\vref{2 Co 10:4}] puissantes par la v. de Dieu, pour
\item[\vref{Ga 4:23}] libre naquit en v. de la promesse.
\item[\vref{Ph 4:8}] y a qq v. et qq louange ;
\item[\vref{1 Pi 2:9}] vs. annonciez les v. de celui qui
\item[\vref{2 Pi 1:3}] par sa gloire et par sa v.,
\end{listverse}

\ConcordanceEntry{Vertueux}
\vspace{-2mm}
\begin{listverse}
\item[\vref{Ex 18:21}] peuple des hommes v., craignant Dieu ; des
\item[\vref{Ex 18:25}] Israël des hommes v., et les établit
\item[\vref{Ru 3:11}] sait que tu es une fem. v.
\item[\vref{Pr 12:4}] La fem. v. est la couronne de son mari,
\item[\vref{Pr 31:10}] trouvera une fem. v. ? Car son prix
\end{listverse}

\ConcordanceEntry{Vêtement}
\vspace{-2mm}
\begin{listverse}
\item[\vref{Ge 39:12}] saisit par son v. et lui dit :
\item[\vref{Ex 28:2}] frère, de saints v. pour la gloire
\item[\vref{De 8:4}] Ton v. ne s'est point usé sur toi,
\item[\vref{De 24:17}] en gage le v. de la veuve.
\item[\vref{Job 37:17}] Pourquoi tes v. sont chauds, qnd
\item[\vref{Ps 22:19}] se partagent mes v. et tirent au
\item[\vref{Es 61:10}] m'a revêtu des v. du salut, il
\item[\vref{Da 7:9}] jours s'assit. Son v. était blanc com.
\item[\vref{Za 3:3}] était vêtu de v. sales, et il
\item[\vref{Mt 3:4}] Jean avait un v. de poils de
\item[\vref{Mt 6:28}] au sujet du v. ? Apprenez comment croissent
\item[\vref{Mt 9:21}] seulement toucher son v., je serai guérie.
\item[\vref{Mt 14:36}] bord de son v.. Et ts ceux
\item[\vref{Mt 21:8}] foules étendirent leurs v. sur le chemin,
\item[\vref{Mt 28:3}] éclair, et son v. blanc com. de
\item[\vref{1 Ti 6:8}] nourriture et le v., cela ns. suffira.
\item[\vref{Hé 1:11}] et ils vieilliront ts com. un v.,
\item[\vref{Ap 3:4}] pas souillé leurs v., et qui marcheront
\item[\vref{Ap 3:18}] riche; et des v. blancs, afin que
\item[\vref{Ap 16:15}] qui garde ses v., afin de ne
\item[\vref{Ap 19:13}] était revêtu d'un v. teint de sang,
\end{listverse}

\ConcordanceEntry{Vêtir}
\vspace{-2mm}
\begin{listverse}
\item[\vref{Ex 29:5}] et tu feras v. à Aaron la
\item[\vref{Lé 21:10}] sera consacré pour v. les saints vêtements,
\item[\vref{Pr 27:26}] sont pour te v., et les boucs
\item[\vref{Da 10:5}] avait un hom. v. de lin, et
\item[\vref{Ag 1:6}] Vous avez été v., mais non pas
\item[\vref{Mt 6:25}] quoi vs. serez v.. La vie n'est-elle
\item[\vref{Mt 25:36}] et vs. m'avez v. ; j'étais malade, et
\item[\vref{Mc 1:6}] Or Jean était v. de poils de
\item[\vref{Mc 5:15}] légion, assis et v., et ds son
\item[\vref{2 Co 5:3}] ns. sommes trouvés v. et non pas
\item[\vref{1 Ti 2:9}] que les femmes, v. d'une manière décente,
\item[\vref{Ap 1:13}] un fils d'hom., v. d'une longue robe,
\item[\vref{Ap 3:5}] qui vaincra sera v. de vêtements blancs,
\end{listverse}

\ConcordanceEntry{Veuve}
\vspace{-2mm}
\begin{listverse}
\item[\vref{Ex 22:22}] n'affligerez point la v. ni l'orphelin.
\item[\vref{Lé 21:14}] prendra point une v., ni une répudiée,
\item[\vref{De 10:18}] et à la v., qui aime l'étranger
\item[\vref{De 27:19}] et à la v. ! Et tt le
\item[\vref{1 R 17:9}] à une fem. v. de t'y nourrir.
\item[\vref{Ps 68:6}] le juge des v.. Dieu est ds
\item[\vref{Mal 3:5}] qui oppriment la v. et l'orphelin, qui
\item[\vref{Mc 12:19}] frère épousera sa v. et suscitera une
\item[\vref{Lu 4:25}] y avait plusieurs v. en Israël du
\item[\vref{Lu 18:3}] y avait une v., qui venait souvent
\item[\vref{Ac 6:1}] parce que leurs v. étaient méprisées ds
\item[\vref{1 Ti 5:3}] Honore les v. qui sont véritablement
\item[\vref{Ap 18:7}] ne suis pas v., et je ne
\end{listverse}

\ConcordanceEntry{Viande}
\vspace{-2mm}
\begin{listverse}
\item[\vref{Ex 16:3}] des pots de v., et que ns.
\item[\vref{No 11:4}] donnera de la v. à manger ?
\item[\vref{Ps 78:18}] demandant de la v. selon lr. désir.
\item[\vref{Pr 9:2}] a apprêté sa v., elle a mêlé
\item[\vref{Da 1:5}] portion de la v. royale et du
\item[\vref{Ac 15:29}] vs. abstenir des v. sacrifiées aux idoles,
\item[\vref{Ro 14:15}] manger de la v., tu ne te
\item[\vref{1 Co 3:2}] pas de la v., parce que vs.
\item[\vref{1 Co 8:8}] n'est pas une v. qui ns. rend
\item[\vref{1 Co 10:3}] mangé la mm v. spirituelle,
\end{listverse}

\ConcordanceEntry{Victime}
\vspace{-2mm}
\begin{listverse}
\item[\vref{Ps 40:7}] ni holoc. ni v. expiatoire pour le
\item[\vref{Ez 43:25}] un bouc com. v. expiatoire, et les
\item[\vref{Mal 2:3}] excréments de vos v. sur vos visages,
\item[\vref{Ac 7:42}] sacrifices et des v. pendant quarante ans
\item[\vref{Ro 3:25}] pour être une v. de propitiation par
\item[\vref{1 Jn 2:2}] qui est la v. de propitiation pour
\end{listverse}

\ConcordanceEntry{Victoire}
\vspace{-2mm}
\begin{listverse}
\item[\vref{1 S 17:47}] lance. Car la v. est à Yahweh,
\item[\vref{2 S 19:2}] Ainsi, la v. fut en ce
\item[\vref{Es 25:8}] mort par sa v. ; et le Seign.
\item[\vref{1 Co 15:54}] mort a été détruite ds la v.
\item[\vref{1 Co 15:57}] a donné la v. par notre Seign.
\item[\vref{1 Jn 5:4}] fait remporter la v. sur le monde,
\end{listverse}

\ConcordanceEntry{Vie}
\vspace{-2mm}
\begin{listverse}
\item[\vref{Ge 1:30}] un souffle de v., je donne tte
\item[\vref{Ge 19:17}] dit : Sauve ta v., ne regarde point
\item[\vref{Ge 50:20}] pour sauver la v. à un peuple
\item[\vref{Ex 1:14}] lr. rendirent la v. amère par un
\item[\vref{Ex 21:23}] mort, tu donneras v. pour vie,
\item[\vref{De 30:15}] dvt toi la v. et le bien,
\item[\vref{De 30:19}] dvt toi la v. et la mort,
\item[\vref{Jos 6:25}] Josué sauva la v. à Rahab la
\item[\vref{Jos 9:15}] lr. laisser la v., et les chefs
\item[\vref{1 R 19:10}] ils me cherchent pour m'ôter la v.
\item[\vref{Job 2:6}] Seulement ne touche pas à sa v.
\item[\vref{Job 14:1}] est de courte v., et rassasié d'agitations.
\item[\vref{Job 33:18}] fosse, et sa v. de l'épée.
\item[\vref{Ps 22:21}] Délivre ma v. de l'épée, mon
\item[\vref{Ps 30:6}] grâce tte la v.. Le soir arrivent
\item[\vref{Ps 34:13}] plaisir à la v., qui aime la
\item[\vref{Ps 41:3}] lui conserve la v.. Il est heureux
\item[\vref{Ps 143:3}] il foule ma v. par terre ; il
\item[\vref{Es 38:16}] on a la v., et ds ttes
\item[\vref{Da 12:2}] uns pour la v. éternelle, et les
\item[\vref{Os 6:2}] ns. rendra la v. ds deux jours ;
\item[\vref{Mt 7:14}] mènent à la v., et il y
\item[\vref{Mt 10:39}] aura conservé sa v. la perdra, mais
\item[\vref{Mt 19:16}] pour avoir la v. éternelle ?
\item[\vref{Mt 20:28}] de donner sa v. en rançon pour
\item[\vref{Mc 10:30}] à venir, la v. éternelle.
\item[\vref{Jn 1:4}] elle était la v., et la vie
\item[\vref{Jn 3:16}] qu'il ait la v. éternelle.
\item[\vref{Jn 4:36}] fruit pour la v. éternelle, afin que
\item[\vref{Jn 5:24}] envoyé, a la v. éternelle et ne
\item[\vref{Jn 5:26}] Père a la v. en lui-mm, ainsi
\item[\vref{Jn 5:40}] venir à moi pour avoir la v.
\item[\vref{Jn 6:27}] permanente jusqu'à la v. éternelle, laquelle le
\item[\vref{Jn 6:47}] moi a la v. éternelle.
\item[\vref{Jn 6:63}] vs. ai dites sont Esprit et v.
\item[\vref{Jn 6:68}] paroles de la v. éternelle.
\item[\vref{Jn 10:10}] brebis aient la v., et qu'elles l'aient
\item[\vref{Jn 10:17}] je donne ma v., afin de la
\item[\vref{Jn 10:28}] lr. donne la v. éternelle, elles ne
\item[\vref{Jn 12:25}] qui aime sa v. la perdra, et
\item[\vref{Jn 13:37}] mntnt ? J'exposerai ma v. pour toi.
\item[\vref{Jn 20:31}] vs. ayez la v. par son Nom.
\item[\vref{Ac 13:46}] indignes de la v. éternelle, voici, ns.
\item[\vref{Ac 17:28}] ns. avons la v., le mouvement et
\item[\vref{Ac 20:24}] cas de ma v., com. si elle
\item[\vref{Ro 2:7}] à savoir la v. éternelle à ceux
\item[\vref{Ro 6:22}] pour fin la v. éternelle.
\item[\vref{Ro 6:23}] Dieu, c'est la v. éternelle par Jésus-Christ,
\item[\vref{Ro 8:6}] l'Esprit c'est la v. et la paix ;
\item[\vref{Ro 8:10}] mais l'Esprit est v. à cause de
\item[\vref{Ro 11:15}] réception, sinon une v. d'entre les morts ?
\item[\vref{1 Co 15:19}] que pour cette v. seulement, ns. sommes
\item[\vref{Ga 6:8}] de l'Esprit la v. éternelle.
\item[\vref{Ep 4:18}] étrangers à la v. de Dieu, à
\item[\vref{Ph 1:21}] Christ est ma v., et la mort
\item[\vref{Col 3:3}] morts, et votre v. est cachée avec
\item[\vref{1 Ti 2:2}] ns. menions une v. paisible et tranquille,
\item[\vref{1 Ti 4:8}] promesses de la v. présente et de
\item[\vref{1 Ti 6:12}] foi, saisis la v. éternelle, à laquelle
\item[\vref{1 Ti 6:19}] qu'ils obtiennent la v. éternelle.
\item[\vref{2 Ti 1:10}] en lumière la v. et l'immortalité par
\item[\vref{Tit 1:2}] l'espérance de la v. éternelle, que Dieu,
\item[\vref{Tit 3:7}] héritiers de la v. éternelle selon notre
\item[\vref{Hé 7:3}] ni fin de v., mais il est
\item[\vref{Hé 7:16}] puissance de la v. impérissable.
\item[\vref{1 Pi 3:10}] veut aimer sa v. et voir des
\item[\vref{2 Pi 1:3}] appartient à la v. et à la
\item[\vref{1 Jn 1:2}] car la v. a été manifestée, et ns. l'avons
\item[\vref{1 Jn 3:15}] ne possède la v. éternelle.
\item[\vref{1 Jn 3:16}] a donné sa v. pour ns. ; ns.
\item[\vref{1 Jn 5:12}] Fils a la v. ; celui qui n'a
\item[\vref{1 Jn 5:13}] vs. avez la v. éternelle, et afin
\item[\vref{1 Jn 5:20}] vraiDieu, et la v. éternelle.
\item[\vref{Jud 1:21}] pour obtenir la v. éternelle.
\item[\vref{Ap 2:8}] et qui est revenu à la v. :
\item[\vref{Ap 12:11}] pas aimé leurs v., mais les ont
\item[\vref{Ap 22:1}] d'eau de la v., transparent com. du
\end{listverse}

\ConcordanceEntry{Vieil}
\vspace{-2mm}
\begin{listverse}
\item[\vref{Mt 9:16}] neuf à un v. habit ; car la
\item[\vref{Ro 6:6}] sachant que notre v. hom. a été
\item[\vref{Ep 4:22}] vs. dépouilliez le v. hom., pour ce
\item[\vref{Col 3:9}] étant dépouillés du v. hom. et de
\end{listverse}

\ConcordanceEntry{Vieillard}
\vspace{-2mm}
\begin{listverse}
\item[\vref{Lé 19:32}] la personne du v.. Tu craindras ton
\item[\vref{De 28:50}] soutiendra point le v. et n'aura point
\item[\vref{Jos 6:21}] depuis l'enfant jusqu'au v., mm jusqu'aux bœufs,
\item[\vref{1 S 28:14}] répondit : C'est un v. qui monte, et
\item[\vref{2 Ch 36:17}] vierge, ni le v., ni l'hom. à
\item[\vref{Es 3:5}] arrogamment contre le v., et l'hom. de
\item[\vref{Es 47:6}] durement appesanti ton joug sur le v.
\item[\vref{Es 65:20}] ni nourrisson, ni v. qui n'accomplisse leurs
\item[\vref{Jé 6:11}] seront pris, le v. et celui qui
\item[\vref{Jé 51:22}] en pièces le v. et le jeune
\item[\vref{Joë 2:28}] filles prophétiseront ; vos v. auront des songes,
\item[\vref{1 Ti 5:1}] pas rudement le v., mais exhorte-le com.
\item[\vref{Phm 1:9}] savoir Paul, un v., et mm mntnt
\end{listverse}

\ConcordanceEntry{Vieillesse}
\vspace{-2mm}
\begin{listverse}
\item[\vref{Ge 21:2}] Abraham ds sa v., au temps précis
\item[\vref{Ge 25:8}] après une heureuse v., fort âgé et
\item[\vref{Ge 37:3}] eu ds sa v., et il lui
\item[\vref{1 R 11:4}] temps de la v. de Salomon, que
\item[\vref{Job 41:23}] l'abîme pour une tête blanchie de v.
\item[\vref{Ps 71:9}] temps de ma v. ; ne m'abandonne point
\item[\vref{Ps 71:18}] ds la blanche v., afin que j'annonce
\item[\vref{Ps 92:15}] ds la blanche v., ils sont gras
\item[\vref{Es 46:4}] Jusqu'à votre v., JE SUIS ; et
\item[\vref{Lu 1:36}] fils en sa v., et celle qui
\end{listverse}

\ConcordanceEntry{Vieillir}
\vspace{-2mm}
\begin{listverse}
\item[\vref{Ps 6:8}] le chagrin ; il v. à cause de
\item[\vref{Ps 37:25}] jeune et j'ai v. ; et je n'ai
\item[\vref{La 3:4}] Il a fait v. ma chair et
\item[\vref{Ro 7:6}] non selon la lettre qui a v.
\item[\vref{Hé 1:11}] permanent ; et ils v. ts com. un
\end{listverse}

\ConcordanceEntry{Vierge}
\vspace{-2mm}
\begin{listverse}
\item[\vref{Ex 22:16}] hom. séduit une v. non fiancée, et
\item[\vref{De 22:17}] trouvé ta fille v.. Cependant, voici les
\item[\vref{Jg 19:24}] j'ai une fille v., et cet hom.
\item[\vref{2 S 13:2}] car elle était v. ; et il paraissait
\item[\vref{1 R 1:2}] une jeune fille v. ; qu'elle se tienne
\item[\vref{2 R 19:21}] toi, la fille, v. de Sion ; elle
\item[\vref{Job 31:1}] dc arrêté mes regards sur une v. ?
\item[\vref{Pr 30:19}] la trace de l'hom. chez la v.
\item[\vref{Es 7:14}] signe : Voici, une v. sera enceinte, et
\item[\vref{Mt 1:23}] Voici, la v. deviendra enceinte, elle
\item[\vref{Mt 25:1}] semblable à dix v. qui, ayant pris
\item[\vref{1 Co 7:25}] qui concerne les v., je n'ai pas
\item[\vref{2 Co 11:2}] vs. présenter à Christ com. une v. pure.
\item[\vref{Ap 14:4}] car ils sont v. ; ce sont ceux
\end{listverse}

\ConcordanceEntry{Vieux}
\vspace{-2mm}
\begin{listverse}
\item[\vref{Ge 18:11}] et Sara étaient v., fort avancés en
\item[\vref{Ge 19:31}] Notre père est v., et il n'y
\item[\vref{Ge 43:27}] lr. dit : Votre v. père, dont vs.
\item[\vref{Jos 23:1}] que Josué était v., fort avancé en
\item[\vref{1 S 2:22}] Eli était très v., il apprit tt
\item[\vref{1 S 8:1}] Samuel était devenu v., il établit ses
\item[\vref{1 R 1:1}] roi David était v. et avancé en
\item[\vref{1 R 13:11}] y avait un v. prophète qui demeurait
\item[\vref{Pr 22:6}] lorsqu'il sera devenu v., il ne s'en
\item[\vref{Mc 2:21}] une partie du v. et la déchirure
\item[\vref{Lu 1:18}] Car je suis v. et ma fem.
\item[\vref{Lu 5:39}] bu du vin v., ne veut du
\item[\vref{Jn 3:4}] qnd il est v. ? Peut-il rentrer ds
\item[\vref{Jn 21:18}] qnd tu seras v., tu étendras tes
\item[\vref{1 Co 5:8}] non avec du v. levain, non avec
\item[\vref{Hé 8:13}] ce qui devient v. et ancien, est
\end{listverse}

\ConcordanceEntry{Vigilant}
\vspace{-2mm}
\begin{listverse}
\item[\vref{1 Ti 3:2}] d'une seule fem., v., modéré, honorable, hospitalier,
\item[\vref{Ap 3:2}] Sois v., et affermis le reste qui va
\end{listverse}

\ConcordanceEntry{Vigne}
\vspace{-2mm}
\begin{listverse}
\item[\vref{Ge 9:20}] terre, commença à planter de la v.
\item[\vref{Ge 49:11}] attache à la v. son ânon, et
\item[\vref{De 6:11}] point creusés, des v. et des oliviers
\item[\vref{Jg 9:12}] dirent à la v. : Viens, toi, et
\item[\vref{1 R 4:25}] chacun sous sa v. et sous son
\item[\vref{1 R 21:1}] Jizreel, ayant une v. à Jizreel, près
\item[\vref{Ps 80:9}] avais retiré une v. hors d'Egypte, tu
\item[\vref{Ps 128:3}] maison com. une v. qui porte du
\item[\vref{Pr 24:30}] près de la v. d'un hom. dépourvu
\item[\vref{Ca 1:6}] faite gardienne des v. et je n'ai
\item[\vref{Es 5:7}] d'Israël est la v. de Yahweh des
\item[\vref{Jé 2:21}] plantée com. une v. exquise, dont tt
\item[\vref{Ez 15:2}] bois de la v. de plus que
\item[\vref{Mi 4:4}] s'assiéra sous sa v. et sous son
\item[\vref{Mt 20:1}] de louer des ouvriers pour sa v.
\item[\vref{Mt 21:33}] qui planta une v., et l'entoura d'une
\item[\vref{Mt 26:29}] ce fruit de v., jusqu'au jour que
\item[\vref{Mc 12:9}] maître de la v. ? Il viendra, et
\item[\vref{1 Co 9:7}] qui plante une v. et n'en mange
\item[\vref{Ap 14:18}] grappes de la v. de la terre,
\end{listverse}

\ConcordanceEntry{Vigneron}
\vspace{-2mm}
\begin{listverse}
\item[\vref{Mc 12:9}] fera périr ces v., et donnera la
\item[\vref{Lu 13:7}] il dit au v. : Voilà trois ans
\item[\vref{Jn 15:1}] cep, et mon Père est le v.
\end{listverse}

\ConcordanceEntry{Vigoureux}
\vspace{-2mm}
\begin{listverse}
\item[\vref{Jos 14:11}] encore aujourd'hui aussi v. que j'étais le
\item[\vref{Es 18:7}] peuple robuste et v., de la part,
\item[\vref{Es 22:17}] loin d'un jet v. ; il t'enveloppera, il
\item[\vref{Hé 11:34}] des malades devinrent v., se montrèrent fort
\end{listverse}

\ConcordanceEntry{Vigueur}
\vspace{-2mm}
\begin{listverse}
\item[\vref{Ge 49:3}] commencement de ma v., qui excelle en
\item[\vref{No 23:22}] eux com. la v. du buffle.
\item[\vref{No 24:8}] lui com. la v. du buffle ; il
\item[\vref{De 21:17}] commencement de sa v., le droit d'aînesse
\item[\vref{De 34:7}] affaiblie, et sa v. n'était point passée.
\item[\vref{Job 8:16}] est plein de v. étant exposé au
\item[\vref{Job 30:2}] de leurs mains ? En eux la v. a péri.
\item[\vref{Ps 31:5}] en secret, car tu es ma v.
\item[\vref{Ps 105:36}] pays, les prémices de tte lr. v.
\item[\vref{Pr 31:3}] donne pas ta v. aux femmes, ni
\item[\vref{Es 40:29}] force de celui qui n'a aucune v.
\item[\vref{Da 10:8}] défait, et je ne conservai aucune v.
\item[\vref{Da 10:16}] rempli d'effroi, et j'ai perdu tte v.
\item[\vref{Am 2:14}] usage de sa v., et le vaillant
\end{listverse}

\ConcordanceEntry{Ville}
\vspace{-2mm}
\begin{listverse}
\item[\vref{Ge 4:17}] Il bâtit une v., et il donna
\item[\vref{Ge 11:4}] Allons ! Bâtissons-ns. une v., et une tour
\item[\vref{No 35:7}] Toutes les v. que vs. donnerez
\item[\vref{De 1:28}] haute taille ; les v. sont grandes et
\item[\vref{Jos 11:13}] brûla aucune des v. situées sur des
\item[\vref{Né 11:1}] à Jérus., la v. sainte, et que
\item[\vref{Ps 48:3}] nord, c'est la v. du grand Roi.
\item[\vref{Ps 127:1}] ne garde la v., celui qui la
\item[\vref{Pr 10:15}] riche sont la v. de sa force,
\item[\vref{Es 22:2}] Toi v. bruyante, pleine de tumulte, ville joyeuse !
\item[\vref{Es 45:13}] il rebâtira ma v., et libérera mes
\item[\vref{Es 52:1}] ta force ! Jérus., v. sainte ! Revêts-toi de
\item[\vref{Jé 1:18}] pays com. une v. forte, une colonne
\item[\vref{Jé 29:7}] paix de la v. où je vs.
\item[\vref{Jé 30:18}] ses demeures ; la v. sera rebâtie sur
\item[\vref{Ez 24:6}] Malheur à la v. sanguinaire, à la
\item[\vref{Ez 48:35}] circuit de la v. sera de dix-huit
\item[\vref{Da 9:26}] viendra, détruira la v. et le sanctuaire,
\item[\vref{Za 8:3}] Jérus. sera appelée v. fidèle ; et la
\item[\vref{Mt 2:1}] né à Bethléhem, v. de Juda, au
\item[\vref{Mt 9:1}] la mer et vint ds sa v.
\item[\vref{Mt 10:23}] persécuteront ds une v., fuyez ds une
\item[\vref{Lu 8:1}] qu'il allait de v. en ville, et
\item[\vref{Lu 19:17}] choses, reçois le gouvernement de dix v.
\item[\vref{Ac 27:2}] Macédonien de la v. de Thessalonique.
\item[\vref{Ap 16:19}] La grande v. fut divisée en
\item[\vref{Ap 18:16}] Malheur ! La grande v. qui était vêtue
\item[\vref{Ap 21:2}] je vis la v. sainte, la nouvelle
\item[\vref{Ap 21:10}] montra la grande v., la sainte Jérus.,
\item[\vref{Ap 21:18}] jaspe, mais la v. était d'or pur,
\item[\vref{Ap 22:3}] sera ds la v., et ses serviteurs
\end{listverse}

\ConcordanceEntry{Vin}
\vspace{-2mm}
\begin{listverse}
\item[\vref{Ge 9:21}] il but du v., s'enivra, et se
\item[\vref{Ge 14:18}] pain et du v. ; or il était
\item[\vref{Lé 10:9}] boirez point de v., ni de boisson
\item[\vref{No 6:3}] il s'abstiendra de v. et de boisson
\item[\vref{Jg 9:13}] Renoncerais-je à mon v. qui réjouit Dieu
\item[\vref{Jg 13:4}] ne boire ni v. ni liqueur forte,
\item[\vref{Ps 75:9}] Yahweh, et le v. rougit dedans ; il
\item[\vref{Ps 104:15}] et le v. qui réjouit le cœur de l'hom.,
\item[\vref{Pr 20:1}] Le v. est moqueur et la boisson forte
\item[\vref{Pr 23:30}] s'arrêtent auprès du v., pour ceux qui
\item[\vref{Pr 31:6}] périt, et du v. à celui qui
\item[\vref{Ec 10:19}] réjouir, et le v. réjouit les vivants,
\item[\vref{Es 25:6}] un banquet de v. vieux, un banquet,
\item[\vref{Es 55:1}] dis-je, achetez du v. et du lait
\item[\vref{Da 1:5}] royale et du v. dont il buvait,
\item[\vref{Os 4:11}] La luxure, le v. et le moût,
\item[\vref{Mi 2:11}] prophétiserai sur du v. et sur les
\item[\vref{Mt 9:17}] non plus du v. nouveau ds de
\item[\vref{Mc 15:23}] à boire du v. mêlé de myrrhe,
\item[\vref{Lu 1:15}] ne boira ni v., ni boisson forte,
\item[\vref{Jn 2:3}] Et le v. ayant manqué, la mère de Jésus
\item[\vref{Ac 2:13}] disaient : C'est qu'ils sont pleins de v. doux.
\item[\vref{Ro 14:21}] pas boire de v., et de s'abstenir
\item[\vref{Ep 5:18}] enivrez pas du v. ds lequel il
\item[\vref{1 Ti 3:3}] ni adonné au v., ni violent, ni
\item[\vref{1 Ti 5:23}] d'un peu de v., à cause de
\item[\vref{Ap 6:6}] de mal au v. et à l'huile.
\item[\vref{Ap 18:3}] ont bu du v. de sa prostitution
\item[\vref{Ap 19:15}] la cuve du v. de l'indignation et
\end{listverse}

\ConcordanceEntry{Vinaigre}
\vspace{-2mm}
\begin{listverse}
\item[\vref{No 6:3}] ne boira ni v. fait de vin,
\item[\vref{Ru 2:14}] morceau ds le v.. Et elle s'assit
\item[\vref{Ps 69:22}] ma soif, ils m'ont abreuvé de v.
\item[\vref{Pr 10:26}] Ce qu'est le v. aux dents et
\item[\vref{Pr 25:20}] et com. du v. répandu sur le
\item[\vref{Mt 27:34}] à boire du v. mêlé avec du
\item[\vref{Mt 27:48}] qu'il remplit de v., et l'ayant fixée
\item[\vref{Jn 19:30}] eut pris le v., il dit : Tout
\end{listverse}

\ConcordanceEntry{Violence}
\vspace{-2mm}
\begin{listverse}
\item[\vref{Ge 19:9}] eux. Et faisant v. à Lot, ils
\item[\vref{Ge 49:5}] des instruments de v. ds leurs demeures.
\item[\vref{De 22:25}] l'hom. lui fait v. et couche avec
\item[\vref{Jg 9:24}] Afin que la v. faite aux soixante-dix
\item[\vref{2 S 13:14}] qu'elle, lui fit v. et coucha avec
\item[\vref{2 S 22:3}] Sauveur ! Tu me délivres de la v.
\item[\vref{Esd 4:23}] firent cesser par force et par v.
\item[\vref{Ps 7:17}] tête, et sa v. redescend sur son
\item[\vref{Ps 11:5}] âme hait celui qui aime la v.
\item[\vref{Ps 25:19}] me haïssent d'une haine pleine de v.
\item[\vref{Ps 55:10}] j'ai vu la v. et les querelles
\item[\vref{Pr 10:6}] juste, mais la v. couvrira la bouche
\item[\vref{Es 53:9}] point commis de v., et qu'il n'y
\item[\vref{Jé 22:3}] et n'usez d'aucune v., et ne répandez
\item[\vref{Ab 1:10}] cause de la v. que tu as
\item[\vref{Ha 1:3}] et de la v. dvt moi, et
\item[\vref{So 3:4}] ils ont fait v. à la loi.
\item[\vref{Mc 1:26}] en l'agitant avec v. et en poussant
\item[\vref{Lu 16:16}] est prêché, et chacun y fait v.
\end{listverse}

\ConcordanceEntry{Violent}
\vspace{-2mm}
\begin{listverse}
\item[\vref{Ps 18:49}] adversaires, tu me sauves de l'hom. v.
\item[\vref{Ps 140:4}] Préserve-moi de l'hom. v., de ceux qui
\item[\vref{Pr 3:31}] envie à l'hom. v., et ne choisis
\item[\vref{Pr 6:12}] est un hom. v. et ses discours
\item[\vref{Pr 16:29}] L'hom. v. attire son compagnon et le fait
\item[\vref{Pr 22:24}] et ne va pas avec l'hom. v. ;
\item[\vref{Da 11:14}] et des hommes v. parmi ton peuple
\item[\vref{Mt 11:12}] ce sont les v. qui s'en emparent.
\item[\vref{1 Ti 1:13}] et un hom. v. ; mais j'ai obtenu
\item[\vref{1 Ti 3:3}] au vin, ni v., ni porté au
\end{listverse}

\ConcordanceEntry{Violer}
\vspace{-2mm}
\begin{listverse}
\item[\vref{No 30:3}] vœu, il ne v. pas sa parole ;
\item[\vref{1 S 8:3}] des présents et v. la justice.
\item[\vref{Esd 9:14}] retournerions-ns. à v. tes commandements, et
\item[\vref{Ps 89:32}] s'ils v. mes statuts, et qu'ils ne gardent
\item[\vref{Mt 5:19}] dc qui aura v. l'un de ces
\item[\vref{Mt 12:5}] sabbat, les prêtres v. le sabbat ds
\item[\vref{Jn 5:18}] seulement il avait v. le sabbat, mais
\item[\vref{Jn 7:23}] ne soit pas v., pourquoi êtes-vs. irrités
\item[\vref{Ac 23:3}] loi, et tu v. la loi en
\item[\vref{1 Ti 5:12}] ce qu'elles ont v. lr. première foi.
\end{listverse}

\ConcordanceEntry{Vipère}
\vspace{-2mm}
\begin{listverse}
\item[\vref{Ge 49:17}] le chemin, une v. sur le sentier,
\item[\vref{Es 11:8}] main ds la caverne de la v.
\item[\vref{Mt 3:7}] dit : Race de v., qui vs. a
\item[\vref{Ac 28:3}] au feu, une v. en sortit à
\end{listverse}

\ConcordanceEntry{Virginité}
\vspace{-2mm}
\begin{listverse}
\item[\vref{De 22:15}] signes de la v. de la jeune
\item[\vref{Jg 11:37}] je pleurerai ma v. avec mes compagnes.
\item[\vref{Ez 23:3}] lr. sein fut déshonoré et lr. v. touchée.
\item[\vref{Lu 2:36}] son mari sept ans depuis sa v.
\end{listverse}

\ConcordanceEntry{Visage}
\vspace{-2mm}
\begin{listverse}
\item[\vref{Ge 3:19}] sueur de ton v., jusqu'à ce que
\item[\vref{Ge 4:5}] irrité, et son v. fut abattu.
\item[\vref{Ex 3:6}] Moïse cacha son v., parce qu'il craignait
\item[\vref{Ex 34:29}] peau de son v. était devenue rayonnante
\item[\vref{1 S 1:18}] mangea, et son v. ne fut plus
\item[\vref{1 S 5:3}] trouvèrent Dagon le v. contre terre, dvt
\item[\vref{1 R 19:13}] il s'enveloppa le v. de son manteau,
\item[\vref{2 R 4:29}] bâton sur le v. de l'enfant.
\item[\vref{2 R 20:2}] Ezéchias tourna son v. contre le mur
\item[\vref{Job 11:15}] pourras élever ton v. sans tache. Tu
\item[\vref{Ps 6:8}] J'ai le v. usé par le chagrin ; il vieillit
\item[\vref{Ps 69:8}] l'opprobre, la honte a couvert mon v.
\item[\vref{Ps 104:15}] fait resplendir son v. avec l'huile, et
\item[\vref{Pr 15:13}] joyeux rend le v. beau, mais l'esprit
\item[\vref{Pr 27:19}] ds l'eau le v. répond au visage,
\item[\vref{Ec 7:3}] la tristesse du v. le cœur devient
\item[\vref{Ec 8:1}] fait briller son v., et son regard
\item[\vref{Ca 7:5}] Bath-Rabbim et ton v. est com. la
\item[\vref{Es 3:9}] L'aspect de lr. v. témoigne contre eux,
\item[\vref{Es 50:6}] pas caché mon v. aux opprobres et
\item[\vref{Es 50:7}] j'ai rendu mon v. semblable à un
\item[\vref{Es 52:14}] te voyant, son v. était défiguré plus
\item[\vref{Es 53:3}] com. caché notre v. arrière de lui,
\item[\vref{Ez 12:6}] couvriras aussi ton v., afin que tu
\item[\vref{Da 1:15}] dix jours, leurs v. parurent en meilleur
\item[\vref{Mt 6:17}] oins ta tête et lave ton v.,
\item[\vref{Mt 17:2}] lr. présence, son v. resplendit com. le
\item[\vref{Mt 26:67}] lui crachèrent au v., et lui donnèrent
\item[\vref{Mt 28:3}] Son v. était com. un éclair, et son
\item[\vref{Ac 20:25}] verrez plus mon v., vs. ts au
\item[\vref{1 Co 4:11}] ns. frappe au v., et ns. sommes
\item[\vref{2 Co 3:7}] yeux sur le v. de Moïse, à
\item[\vref{2 Co 11:20}] vs. frappe au v., vs. le supportez.
\item[\vref{Ja 1:23}] qui regarde ds un miroir son v. naturel,
\item[\vref{Ap 1:16}] tranchants, et son v. était semblable au
\end{listverse}

\ConcordanceEntry{Visible}
\vspace{-2mm}
\begin{listverse}
\item[\vref{Ro 2:28}] celle qui est v. ds la chair.
\item[\vref{2 Co 4:18}] pas aux choses v., mais aux invisibles ;
\item[\vref{Col 1:16}] la terre, les v. et les invisibles,
\item[\vref{Hé 11:3}] n'ont pas été faites des choses v.
\end{listverse}

\ConcordanceEntry{Vision}
\vspace{-2mm}
\begin{listverse}
\item[\vref{Ge 15:1}] Abram ds une v., en disant : Abram,
\item[\vref{Ex 3:3}] regarderai cette grande v., pourquoi le buisson
\item[\vref{No 12:6}] à lui en v., et je lui
\item[\vref{1 S 3:15}] Or Samuel craignait de rapporter cette v. à Eli.
\item[\vref{1 Ch 17:15}] ttes ces paroles, et tte cette v.
\item[\vref{2 Ch 26:5}] intelligence ds les v. de Dieu et
\item[\vref{2 Ch 32:32}] écrites ds la v. d'Esaïe, le prophète,
\item[\vref{Job 20:8}] s'enfuira com. une v. de nuit ;
\item[\vref{Ps 89:20}] autrefois parlé en v. concernant ton bien-aimé,
\item[\vref{Pr 29:18}] a pas de v., le peuple périt,
\item[\vref{La 2:9}] reçoivent plus aucune v. de Yahweh.
\item[\vref{Ez 1:28}] C'est là la v. de la représentation
\item[\vref{Ez 12:23}] et ttes les v. s'accompliront.
\item[\vref{Da 2:19}] Daniel ds une v. pendant la nuit.
\item[\vref{Da 8:17}] l'hom., car la v. est pour le
\item[\vref{Joë 2:28}] et vos jeunes gens verront des v.
\item[\vref{Ha 2:2}] dit : Ecris la v., et grave-la sur
\item[\vref{Mt 17:9}] personne de cette v., jusqu'à ce que
\item[\vref{Lu 1:22}] avait eu une v. ds le temple ;
\item[\vref{Ac 9:10}] Seign. dit en v. : Ananias ! Et il
\item[\vref{Ac 11:5}] et j'eus une v.. Un vase semblable
\item[\vref{Ac 16:9}] Paul eut une v. d'un hom. macédonien
\item[\vref{Ac 18:9}] Paul ds une v. pendant la nuit :
\item[\vref{Ac 26:19}] n'ai pas été désobéissant à la v. céleste.
\item[\vref{2 Co 12:1}] j'en viendrai jusqu'aux v. et aux révélations
\end{listverse}

\ConcordanceEntry{Visiter}
\vspace{-2mm}
\begin{listverse}
\item[\vref{Ge 50:24}] pas de vs. v., et il vs.
\item[\vref{Ru 1:6}] que Yahweh avait v. son peuple en
\item[\vref{Ps 59:6}] d'Israël, réveille-toi pour v. ttes les nations !
\item[\vref{Jé 15:15}] souviens-toi de moi, v.-moi, venge-moi de
\item[\vref{Jé 29:10}] Babylone, je vs. v., et j'accomplirai ma
\item[\vref{Za 10:3}] Yahweh des armées v. son troupeau, la
\item[\vref{Mt 25:36}] et vs. m'avez v. ; j'étais en prison,
\item[\vref{Lu 1:68}] ce qu'il a v. et délivré son
\item[\vref{Lu 7:16}] et Dieu a v. son peuple.
\item[\vref{Ac 7:23}] le dessein d'aller v. ses frères, les
\item[\vref{Ja 1:27}] Père, c'est de v. les orphelins et
\end{listverse}

\ConcordanceEntry{Vivant}
\vspace{-2mm}
\begin{listverse}
\item[\vref{Ge 2:7}] vie ; et l'hom. devint une âme v.
\item[\vref{Lé 16:10}] Azazel, sera présenté v. dvt Yahweh pour
\item[\vref{No 14:28}] Dis-lr. : Je suis v., dit Yahweh, je
\item[\vref{No 16:30}] et qu'ils descendent v. ds le scheol,
\item[\vref{De 4:4}] vs. êtes ts v. aujourd'hui.
\item[\vref{Jos 3:10}] que le Dieu v. est au milieu
\item[\vref{1 R 17:23}] lui disant : Regarde, ton fils est v.
\item[\vref{Job 19:25}] mon Rédempteur est v., et qu'il se
\item[\vref{Job 30:23}] la maison assignée à ts les v.
\item[\vref{Ps 27:13}] de Yahweh sur la terre des v.…
\item[\vref{Ps 56:14}] dvt Dieu, à la lumière des v.
\item[\vref{Ps 116:9}] de Yahweh, sur la terre des v.
\item[\vref{Ps 124:3}] auraient engloutis tt v., qnd lr. colère
\item[\vref{Ps 143:2}] Car aucun hom. v. n'est juste dvt
\item[\vref{Ec 7:2}] hom., et le v. met cela ds
\item[\vref{Ec 9:5}] Certainement les v. savent qu'ils mourront,
\item[\vref{Es 38:19}] Mais le v., le vivant est celui qui te
\item[\vref{Es 49:18}] toi. JE SUIS v., dit Yahweh, tu
\item[\vref{La 3:39}] Pourquoi un hom. v. se plaindrait-il, un
\item[\vref{Da 6:26}] c'est le Dieu v., et il subsiste
\item[\vref{Jon 2:7}] m'as fait remonter v. de la fosse,
\item[\vref{Mt 22:32}] le Dieu des morts, mais des v.
\item[\vref{Lu 24:5}] parmi les morts celui qui est v. ?
\item[\vref{Ac 1:3}] se présenta lui-mm v., avec plusieurs preuves
\item[\vref{Ac 10:42}] Dieu, juge des v. et des morts.
\item[\vref{Ac 20:12}] le jeune hom. v., et ce fut
\item[\vref{Ro 6:11}] au péché, mais v. pour Dieu en
\item[\vref{Ro 14:9}] sur les morts que sur les v.
\item[\vref{1 Co 15:45}] fait en âme v.. Le dernier Adam
\item[\vref{1 Th 1:9}] des idoles, pour servir le Dieu v. et vrai,
\item[\vref{Hé 7:8}] il est rendu témoignage qu'il est v.
\item[\vref{Hé 7:25}] lui, étant toujours v. pour intercéder pour
\item[\vref{Ap 1:18}] voici, je suis v. aux siècles des
\item[\vref{Ap 3:1}] la réputation d'être v., mais tu es
\item[\vref{Ap 10:6}] celui qui est v. aux siècles des
\end{listverse}

\ConcordanceEntry{Vivifier}
\vspace{-2mm}
\begin{listverse}
\item[\vref{Es 57:15}] d'esprit, afin de v. l'esprit des humbles,
\item[\vref{Jn 6:63}] C'est l'Esprit qui v. ; la chair ne
\item[\vref{1 Co 15:22}] aussi ts seront v. en Christ.
\item[\vref{2 Co 3:6}] car la lettre tue, mais l'Esprit v.
\item[\vref{Ep 2:5}] il ns. a v. ensemble avec Christ ;
\item[\vref{Col 2:13}] il vs. a v. ensemble avec lui,
\item[\vref{1 Pi 3:18}] la chair, mais v. par l'Esprit ;
\end{listverse}

\ConcordanceEntry{Vivre}
\vspace{-2mm}
\begin{listverse}
\item[\vref{Ex 1:18}] et avez-vs. laissé v. les fils ?
\item[\vref{Lé 18:5}] qui les pratiquera v. par elles. Je
\item[\vref{De 8:3}] que l'hom. ne v. pas de pain
\item[\vref{De 32:39}] et je fais v., je blesse et
\item[\vref{Ps 33:19}] et les fasse v. durant la famine.
\item[\vref{Ps 118:17}] mourrai pas, je v. et je raconterai
\item[\vref{Ps 119:37}] choses vaines ; fais-moi v. ds ta voie.
\item[\vref{Es 55:3}] et votre âme v. ; et je traiterai
\item[\vref{Ez 3:21}] pèche point, il v. ; il vivra parce
\item[\vref{Ez 33:11}] voie et qu'il v.. Détournez-vs., détournez-vs. de
\item[\vref{Ez 47:9}] qui se meut v. partout où les
\item[\vref{Am 5:4}] la maison d'Israël : Cherchez-moi, et vs. v. !
\item[\vref{Ha 2:4}] mais le juste v. de sa foi.
\item[\vref{Mc 12:44}] possédait, tt ce qu'elle avait pour v.
\item[\vref{Lu 10:28}] bien répondu. Fais cela, et tu v.
\item[\vref{Jn 6:51}] ce pain, il v. éternellement ; et le
\item[\vref{Jn 14:19}] parce que je v., vs. aussi vs.
\item[\vref{Ro 6:2}] au péché, comment v.-ns. encore ds
\item[\vref{Ro 8:5}] Car ceux qui v. selon la chair,
\item[\vref{Ro 14:7}] de ns. ne v. pour lui-mm, et
\item[\vref{Ro 14:8}] Car si ns. v., ns. vivons pour
\item[\vref{1 Co 15:34}] Réveillez-vs. pour v. justement, et ne
\item[\vref{2 Co 6:9}] et voici ns. v. ; com. châtiés et
\item[\vref{Ga 2:20}] et si je v., ce n'est plus
\item[\vref{Ep 2:3}] parmi lesquels ns. v. ts autrefois, selon
\item[\vref{Ep 6:3}] et que tu v. longtemps sur la
\item[\vref{1 Th 5:10}] ns. dormions, ns. v. avec lui.
\item[\vref{2 Ti 2:11}] avec lui, ns. v. aussi avec lui.
\item[\vref{2 Ti 3:12}] aussi qui veulent v. pieusement en Jésus-Christ
\item[\vref{Tit 2:12}] mondaines, et à v. ds le présent
\item[\vref{Ja 4:15}] et si ns. v., ns. ferons aussi
\item[\vref{1 Pi 1:18}] vaine manière de v., qui vs. avait
\item[\vref{1 Pi 4:6}] chair, et qu'ils v. selon Dieu ds
\item[\vref{1 Jn 2:6}] lui doit aussi v. com. Jésus-Christ lui-mm
\item[\vref{1 Jn 4:9}] ds le monde, afin que ns. v. par lui.
\end{listverse}

\ConcordanceEntry{Vocation}
\vspace{-2mm}
\begin{listverse}
\item[\vref{Ro 11:29}] dons et la v. de Dieu sont
\item[\vref{Ep 1:18}] l'espérance de sa v., et quelles sont
\item[\vref{Ep 4:1}] digne de la v. à laquelle vs.
\item[\vref{Ep 4:4}] à une seule espérance par votre v. ;
\item[\vref{Ph 3:14}] prix de la v. céleste de Dieu
\item[\vref{2 Th 1:11}] dignes de la v., et qu'il accomplisse
\item[\vref{2 Ti 1:9}] par une sainte v., non selon nos
\item[\vref{Hé 3:1}] participants de la v. céleste, considérez attentivement
\item[\vref{2 Pi 1:10}] à affermir votre v. et votre élection ;
\end{listverse}

\ConcordanceEntry{Vœu}
\vspace{-2mm}
\begin{listverse}
\item[\vref{Ge 28:20}] Jacob fit un v., en disant : Si
\item[\vref{No 6:2}] en faisant un v. de naziréat pour
\item[\vref{De 23:21}] tu fais un v. à Yahweh, ton
\item[\vref{Jg 11:30}] Jephthé fit un v. à Yahweh, et
\item[\vref{1 S 1:11}] elle fit un v., en disant : Yahweh
\item[\vref{Ps 10:17}] Tu entends les v. de ceux qui
\item[\vref{Ps 20:6}] notre Dieu ; Yahweh exaucera ts tes v.
\item[\vref{Ps 22:26}] assemblée ; j'accomplirai mes v. en présence de
\item[\vref{Ps 50:14}] et accomplis tes v. envers le Très-Haut !
\item[\vref{Ps 56:13}] Ô Dieu ! Les v. que je t'ai
\item[\vref{Ps 76:12}] Faites vos v. à Yahweh votre
\item[\vref{Ps 132:2}] et fait ce v. au puissant de
\item[\vref{Pr 7:14}] de paix ; j'ai aujourd'hui accompli mes v.
\item[\vref{Ec 5:3}] as fais qq v. à Dieu, ne
\item[\vref{Ec 5:4}] fasses point de v. que d'en faire
\item[\vref{Jon 1:16}] sacrifices à Yahweh, et firent des v.
\item[\vref{Jon 2:10}] louange, j'accomplirai les v. que j'ai faits :
\item[\vref{Ac 18:18}] Cenchrées, car il avait fait un v.
\item[\vref{Ac 21:23}] quatre hommes qui ont fait un v.,
\item[\vref{Ac 23:21}] ont fait un v. avec exécration de
\end{listverse}

\ConcordanceEntry{Voie}
\vspace{-2mm}
\begin{listverse}
\item[\vref{Ge 6:12}] avait corrompu sa v. sur la terre.
\item[\vref{Ge 18:19}] de garder la v. de Yahweh, pour
\item[\vref{Ex 18:20}] fais-lr. connaître la v. par laquelle ils
\item[\vref{2 Ch 17:6}] grandit ds les v. de Yahweh, et
\item[\vref{2 Ch 27:6}] avait affermi ses v. dvt Yahweh, son
\item[\vref{Job 34:24}] puissants par des v. incompréhensibles, et il
\item[\vref{Ps 1:1}] pas sur la v. des pécheurs, qui
\item[\vref{Ps 5:9}] ennemis, aplanis ta v. sous mes pas.
\item[\vref{Ps 18:31}] Les v. de Dieu sont sans défaut ; la
\item[\vref{Ps 25:9}] justice, et il lr. enseigne sa v.
\item[\vref{Ps 32:8}] je t'enseignerai la v. ds laquelle tu
\item[\vref{Ps 37:5}] Guimel.] Recommande tes v. à Yahweh, confie-toi
\item[\vref{Ps 39:2}] garde à mes v., de peur de
\item[\vref{Ps 95:10}] et ils n'ont point connu mes v. ;
\item[\vref{Ps 103:7}] fait connaître ses v. à Moïse, et
\item[\vref{Ps 119:30}] Je choisis la v. de la vérité
\item[\vref{Ps 139:3}] couche ; tu connais parfaitement ttes mes v.
\item[\vref{Ps 139:24}] sur une mauvaise v. ; conduis-moi sur la
\item[\vref{Pr 3:6}] ds ttes tes v. et il dirigera
\item[\vref{Pr 3:17}] Ses v. sont des voies agréables, et ts
\item[\vref{Pr 4:26}] que ttes tes v. soient bien stables.
\item[\vref{Pr 8:32}] Heureux sont ceux qui gardent mes v.
\item[\vref{Pr 10:9}] qui pervertit ses v. sera connu.
\item[\vref{Pr 14:12}] y a telle v. qui semble droite
\item[\vref{Pr 16:7}] prend plaisir aux v. d'un hom., il
\item[\vref{Pr 20:24}] comment dc l'hom. peut-il comprendre sa v. ?
\item[\vref{Ec 2:3}] ce que je v. ce qu'il est
\item[\vref{Es 45:13}] j'aplanirai ttes ses v. ; il rebâtira ma
\item[\vref{Es 55:8}] pensées, et mes v. ne sont pas
\item[\vref{Es 66:3}] ont choisi leurs v., et lr. âme
\item[\vref{Jé 3:21}] ont perverti lr. v., ils ont oublié
\item[\vref{Jé 10:23}] sais que la v. de l'hom. ne
\item[\vref{La 3:40}] Nun.] Recherchons nos v., sondons-les, et retournons
\item[\vref{Ez 18:25}] vs. dites : La v. du Seign. n'est
\item[\vref{Ez 33:11}] détourne de sa v. et qu'il vive.
\item[\vref{Da 5:23}] est ton souffle, et ttes tes v.
\item[\vref{Mi 4:2}] enseignera sur ses v., et ns. marcherons
\item[\vref{Mt 22:16}] tu enseignes la v. de Dieu selon
\item[\vref{Mc 1:2}] lequel préparera ta v. dvt toi.
\item[\vref{Mc 12:14}] tu enseignes la v. de Dieu selon
\item[\vref{Ac 16:17}] vs. annoncent la v. du salut !
\item[\vref{Ac 18:26}] et lui exposèrent plus exactement la v. de Dieu.
\item[\vref{Ac 24:14}] que selon la v. qu'ils appellent secte,
\item[\vref{Ro 3:17}] pas connu la v. de la paix ;
\item[\vref{Ro 11:33}] insondables, et ses v. incompréhensibles !
\item[\vref{1 Co 12:31}] vs. montrer la v. la plus excellente.
\item[\vref{2 Pi 2:15}] ont suivi la v. de Balaam, fils
\item[\vref{Jud 1:11}] ont suivi la v. de Caïn, et
\item[\vref{Ap 15:3}] Dieu Tout-Puissant ! Tes v. sont justes et
\end{listverse}

\ConcordanceEntry{Voile}
\vspace{-2mm}
\begin{listverse}
\item[\vref{Ge 24:65}] elle prit son v. et se couvrit.
\item[\vref{Ex 26:33}] tu mettras le v. sous les crochets,
\item[\vref{Ex 34:33}] avait mis un v. sur son visage.
\item[\vref{Lé 4:17}] en face du v., par sept fois.
\item[\vref{Ps 104:2}] il étend les cieux com. un v.
\item[\vref{Ez 13:21}] déchirerai aussi vos v., et je délivrerai
\item[\vref{Mt 27:51}] Et voici, le v. du temple se
\item[\vref{1 Co 11:15}] été donnée pour lui servir de v. ?
\item[\vref{2 Co 3:13}] qui mettait un v. sur son visage,
\item[\vref{2 Co 3:15}] lit Moïse, le v. demeure sur lr.
\item[\vref{2 Co 3:16}] cœur se convertit au Seign., le v. est ôté.
\item[\vref{Hé 6:19}] l'âme, et qui pénètre jusqu'au-delà du v.,
\item[\vref{Hé 9:3}] après le second v. était le tabernacle,
\item[\vref{Hé 10:20}] à travers le v., c'est-à-dire sa propre
\item[\vref{1 Pi 2:16}] liberté com. un v. qui couvre la
\end{listverse}

\ConcordanceEntry{Voiler}
\vspace{-2mm}
\begin{listverse}
\item[\vref{Mc 14:65}] lui, à lui v. le visage, et
\item[\vref{Lu 9:45}] et elle était v. pour eux, afin
\item[\vref{1 Co 11:6}] coupés, ou d'être rasée, qu'elle se v. !
\item[\vref{2 Co 4:3}] Evangile est encore v., il ne l'est
\end{listverse}

\ConcordanceEntry{Voir}
\vspace{-2mm}
\begin{listverse}
\item[\vref{Ge 2:19}] vers Adam pour v. comment il les
\item[\vref{Ge 8:8}] une colombe pour v. si les eaux
\item[\vref{Ge 11:5}] Yahweh descendit pour v. la ville et
\item[\vref{Ge 32:30}] car, dit-il, j'ai v. Dieu face à
\item[\vref{Ex 24:10}] Et ils v. le Dieu d'Israël, et sous ses
\item[\vref{Ex 33:20}] ne pourras pas v. ma face, car
\item[\vref{No 14:23}] ts ceux-là ne v. point le pays
\item[\vref{No 24:17}] Je le v., mais non pas mntnt ; je le
\item[\vref{De 32:19}] Yahweh l'a v., et a été
\item[\vref{Jos 5:6}] lr. laisserait point v. le pays qu'il
\item[\vref{1 R 10:7}] yeux ne l'aient v.. Et voici, on
\item[\vref{Job 19:26}] été détruit, sans ma chair, je v. Dieu !
\item[\vref{Job 19:27}] Je le v. moi-mm, et mes yeux le verront,
\item[\vref{Ps 16:10}] que ton bien-aimé v. la corruption.
\item[\vref{Ps 27:13}] pas sûr de v. la bonté de
\item[\vref{Ps 31:8}] joie ; car tu v. mon affliction, tu
\item[\vref{Ps 60:5}] Tu as fait v. à ton peuple
\item[\vref{Ps 90:16}] ton œuvre se v. sur tes serviteurs,
\item[\vref{Ps 91:16}] je lui ferai v. ma délivrance.
\item[\vref{Ps 94:9}] formé l'œil, ne v.-il point ?
\item[\vref{Es 6:9}] comprendrez point ; et v., voyez, mais vs.
\item[\vref{Ha 1:13}] trop purs pour v. le mal, et
\item[\vref{Ha 2:15}] l'enivre afin qu'on v. sa nudité.
\item[\vref{Za 3:1}] Yahweh me fit v. Josué, le grand-prêtre,
\item[\vref{Mt 5:8}] sont purs de cœur, car ils v. Dieu !
\item[\vref{Mt 6:4}] ton Père, qui v. ce qui se
\item[\vref{Mt 7:3}] Et pourquoi v.-tu la paille
\item[\vref{Mt 11:7}] Mais qu'êtes-vs. allés v. ds le désert ?
\item[\vref{Mt 13:17}] justes ont désiré v. les choses que
\item[\vref{Mc 2:12}] n'avons jamais rien v. de pareil.
\item[\vref{Mc 8:25}] rétabli, et les v. ts clairement.
\item[\vref{Lu 23:8}] il désirait le v., à cause de
\item[\vref{Jn 1:18}] Personne n'a jamais v. Dieu ; le Fils
\item[\vref{Jn 1:39}] dit : Venez et v.. Ils y allèrent,
\item[\vref{Jn 3:3}] il ne peut v. le Royaume de
\item[\vref{Jn 8:56}] de ce qu'il v. mon jour ; il
\item[\vref{Jn 9:25}] j'étais aveugle et que mntnt je v.
\item[\vref{Jn 12:21}] le prièrent, disant : Seign., ns. désirons v. Jésus.
\item[\vref{Jn 12:45}] celui qui me v., voit celui qui
\item[\vref{Jn 16:16}] vs. ne me v. plus ; et après
\item[\vref{Jn 20:29}] que tu m'as v., Thomas, tu as
\item[\vref{Jn 21:1}] Jésus se fit v. encore à ses
\item[\vref{Ac 2:27}] que ton Saint v. la corruption.
\item[\vref{Ac 4:20}] que ns. avons v. et entendu.
\item[\vref{Ac 8:23}] Car je v. que tu es ds un fiel
\item[\vref{Ac 9:27}] chemin, Saul avait v. le Seign., qui
\item[\vref{Ac 22:11}] Comme je ne v. rien, à cause
\item[\vref{Ac 28:6}] s'attendaient à le v. enfler ou tomber
\item[\vref{1 Co 9:1}] libre ? N'ai-je pas v. notre Seign. Jésus-Christ ?
\item[\vref{1 Ti 6:16}] nul hom. n'a v. ni ne peut
\item[\vref{Hé 11:1}] une démonstration de celles qu'on ne v. pas.
\item[\vref{Hé 11:13}] ils les ont v. de loin, crues,
\item[\vref{2 Pi 1:9}] aveugle, et ne v. pas de loin,
\item[\vref{1 Jn 1:1}] que ns. avons v. de nos propres
\item[\vref{1 Jn 3:2}] car ns. le v. tel qu'il est.
\item[\vref{1 Jn 4:20}] son frère qu'il v., peut-il aimer Dieu
\item[\vref{3 Jn 1:11}] qui fait le mal n'a pas v. Dieu.
\item[\vref{Ap 3:18}] yeux de collyre, afin que tu v.
\item[\vref{Ap 16:15}] et qu'on ne v. pas sa honte !
\end{listverse}

\ConcordanceEntry{Voisin}
\vspace{-2mm}
\begin{listverse}
\item[\vref{Ex 12:4}] prendra avec le v. le plus proche
\item[\vref{Ex 32:27}] son frère, son ami, et son v.
\item[\vref{Ru 4:17}] Et les v. lui donnèrent un nom, en disant :
\item[\vref{2 R 4:3}] à ts tes v., des vases vides,
\item[\vref{Ps 79:4}] d'opprobre à nos v., de moquerie et
\item[\vref{Pr 27:10}] calamité ; car le v. qui est proche
\item[\vref{Jé 12:14}] ts mes mauvais v., qui mettent la
\item[\vref{Lu 1:58}] Ses v. et ses parents ayant appris que
\item[\vref{Lu 14:12}] ni tes riches v., de peur qu'ils
\item[\vref{Lu 15:6}] amis et ses v., et il lr.
\item[\vref{Jn 9:8}] Or ses v., et ceux qui auparavant l'avaient connu
\end{listverse}

\ConcordanceEntry{Voix}
\vspace{-2mm}
\begin{listverse}
\item[\vref{Ge 3:8}] du jour la v. de Yahweh Dieu
\item[\vref{Ge 21:17}] Dieu entendit la v. de l'enfant, et
\item[\vref{Ge 22:18}] que tu as obéi à ma v.
\item[\vref{Ge 27:22}] et dit : Cette v. est la voix
\item[\vref{Ex 15:26}] écoutes attentivement la v. de Yahweh, ton
\item[\vref{2 S 22:14}] et le Très-Haut fit retentir sa v. ;
\item[\vref{1 R 18:27}] Criez à haute v., puisqu'il est dieu ;
\item[\vref{2 R 4:31}] n'y eut ni v. ni signe d'attention.
\item[\vref{2 Ch 20:19}] Dieu d'Israël, d'une v. haute et forte.
\item[\vref{Job 37:4}] tonne de sa v. majestueuse ; et il
\item[\vref{Job 37:5}] terriblement par sa v. ; et il fait
\item[\vref{Job 40:4}] tonnes-tu de la v. com. lui ?
\item[\vref{Ps 29:4}] La v. de Yahweh est forte, la voix
\item[\vref{Ps 95:7}] conduit. Si vs. entendez aujourd'hui sa v.,
\item[\vref{Pr 8:1}] l'intelligence ne fait-elle pas entendre sa v. ?
\item[\vref{Ec 10:20}] ciel emporterait ta v., le Baal ailé
\item[\vref{Ca 2:8}] C'est ici la v. de mon bien-aimé !
\item[\vref{Es 6:8}] Puis j'entendis la v. du Seign., disant :
\item[\vref{Es 40:3}] La v. de celui qui crie au désert
\item[\vref{Es 40:6}] La v. dit : Crie ! Et on a répondu :
\item[\vref{Es 40:9}] nouvelles, élève ta v. avec force ; élève-la,
\item[\vref{Es 42:2}] fera entendre sa v. ds les rues.
\item[\vref{Es 42:11}] villes élèvent la v. ! Que les villages
\item[\vref{Jé 7:28}] n'écoute pas la v. de Yahweh, son
\item[\vref{Ez 1:28}] et j'entendis une v. qui parlait.
\item[\vref{Da 3:4}] cria à haute v., en disant : On
\item[\vref{Joë 2:11}] fait entendre sa v. dvt son armée ;
\item[\vref{Am 1:2}] fait entendre sa v. de Jérus., et
\item[\vref{Jon 3:2}] proclames-y à haute v. ce que je
\item[\vref{Mt 3:3}] C'est ici la v. de celui qui
\item[\vref{Mt 17:5}] Et voici, une v. fit entendre de
\item[\vref{Jn 5:28}] sépulcres entendront sa v., et en sortiront.
\item[\vref{Jn 10:3}] brebis entendent sa v., il appelle ses
\item[\vref{Jn 10:4}] le suivent, parce qu'elles connaissent sa v.
\item[\vref{Ac 12:14}] Elle reconnut la v. de Pierre; et,
\item[\vref{Ac 12:22}] Le peuple s'écria : V. d'un dieu et
\item[\vref{Ro 10:18}] Au contraire ! Leur v. est allée par
\item[\vref{1 Th 4:16}] commandement, et une v. d'archange, et avec
\item[\vref{Hé 3:7}] Saint-Esprit : Aujourd'hui, si vs. entendez sa v.,
\item[\vref{Ap 1:10}] derrière moi une v. forte, com. le
\item[\vref{Ap 1:15}] fournaise ; et sa v. était com. le
\item[\vref{Ap 8:5}] des tonnerres, des v., des éclairs, et
\item[\vref{Ap 21:3}] du trône une v. forte qui disait :
\end{listverse}

\ConcordanceEntry{Voleur}
\vspace{-2mm}
\begin{listverse}
\item[\vref{Mt 6:19}] et où les v. percent et dérobent ;
\item[\vref{Mt 21:13}] en avez fait une caverne de v.
\item[\vref{Mt 24:43}] la nuit le v. doit venir, il
\item[\vref{Jn 10:1}] ailleurs, est un v. et un brigand.
\item[\vref{Jn 10:10}] Le v. ne vient que pour dérober, tuer
\item[\vref{Jn 12:6}] parce qu'il était v., et que, tenant
\item[\vref{1 Co 6:10}] homosexuels, ni les v., ni les avares,
\item[\vref{1 Th 5:2}] viendra com. un v. ds la nuit.
\item[\vref{1 Th 5:4}] ce jour-là vs. surprenne com. un v. ;
\item[\vref{1 Ti 1:10}] homosexuels, pour les v. d'hommes, pour les
\item[\vref{1 Pi 4:15}] com. meurtrier, ou v., ou malfaiteur ou
\item[\vref{2 Pi 3:10}] viendra com. un v. ds la nuit,
\item[\vref{Ap 3:3}] toi com. un v., et tu ne
\item[\vref{Ap 16:15}] viens com. un v.. Béni est celui
\end{listverse}

\ConcordanceEntry{Volontaire}
\vspace{-2mm}
\begin{listverse}
\item[\vref{Ex 36:3}] apportait encore chaque matin quelques offrandes v.
\item[\vref{Lé 7:16}] ou une offrande v., son sacrifice sera
\item[\vref{1 Ch 29:9}] cœur leurs offrandes v. à Yahweh ; et
\item[\vref{2 Ch 31:14}] charge des offrandes v. offertes à Dieu,
\item[\vref{Esd 2:68}] firent des offrandes v. pour la maison
\item[\vref{Esd 7:16}] avec les offrandes v. du peuple et
\item[\vref{Ez 46:12}] offre un sacrifice v., qq holoc., soit
\item[\vref{Am 4:5}] publiez les offrandes v. ! Car c'est là
\item[\vref{Col 2:23}] sagesse en dévotion v., et en humilité
\end{listverse}

\ConcordanceEntry{Volontairement}
\vspace{-2mm}
\begin{listverse}
\item[\vref{Ex 25:2}] hom. dont le cœur me l'offrira v.
\item[\vref{Ex 35:29}] dis-je, d'Israël apportèrent v. des présents à
\item[\vref{2 R 12:4}] que chacun apporte v. à la maison
\item[\vref{1 Ch 29:5}] qui se disposera v. à offrir aujourd'hui
\item[\vref{Esd 1:4}] ce qu'on offrira v. pour la maison
\item[\vref{Esd 1:6}] précieuses, outre tt ce qu'on offrit v.
\item[\vref{Né 11:2}] qui se présentèrent v. pour habiter à
\item[\vref{Es 1:19}] Si vs. obéissez v., vs. mangerez le
\item[\vref{Jé 38:17}] Si tu sors v. pour aller vers
\item[\vref{Os 14:4}] je les aimerai v. ; parce que ma
\item[\vref{2 Co 8:3}] qu'ils ont donné v. selon leurs moyens,
\item[\vref{2 Co 8:17}] zélé, il est allé chez vs. v.
\item[\vref{Phm 1:14}] par contrainte, mais v., que tu me
\item[\vref{Hé 10:26}] si ns. péchons v. après avoir reçu
\item[\vref{1 Pi 5:2}] par contrainte, mais v. ; non pour un
\item[\vref{2 Pi 3:5}] Car ils ignorent v. ceci: C'est que
\end{listverse}

\ConcordanceEntry{Volonté}
\vspace{-2mm}
\begin{listverse}
\item[\vref{Esd 7:18}] fassiez, selon la v. de votre Dieu,
\item[\vref{Est 1:8}] faire selon la v. de chacun.
\item[\vref{Job 23:12}] fait plier ma v. aux paroles de
\item[\vref{Ps 103:21}] qui êtes ses serviteurs, faisant sa v. !
\item[\vref{Mt 6:10}] vienne ; que ta v. soit faite sur
\item[\vref{Mt 7:21}] qui fait la v. de mon Père
\item[\vref{Mt 11:26}] que telle a été ta bonne v.
\item[\vref{Mt 21:31}] a fait la v. du père ? Ils
\item[\vref{Lu 12:47}] a connu la v. de son maître,
\item[\vref{Jn 1:13}] ni de la v. de la chair,
\item[\vref{Jn 4:34}] de faire la v. de celui qui
\item[\vref{Jn 5:30}] cherche pas ma v., mais la volonté
\item[\vref{Jn 7:17}] veut faire sa v., il connaîtra si
\item[\vref{Ac 13:22}] mon cœur, qui exécutera tte ma v.
\item[\vref{Ro 2:18}] tu connais sa v., et tu sais
\item[\vref{Ro 9:19}] celui qui peut résister à sa v. ?
\item[\vref{1 Co 7:37}] de sa propre v., et qui décide
\item[\vref{2 Co 8:11}] en avoir la v., vs. l'accomplissiez aussi
\item[\vref{Ep 1:5}] selon le bon plaisir de sa v.,
\item[\vref{Ep 1:9}] mystère de sa v., qu'il avait premièrement
\item[\vref{Ep 5:17}] quelle est la v. du Seign.
\item[\vref{Col 1:9}] connaissance de sa v., en tte sagesse
\item[\vref{1 Th 4:3}] c'est ici la v. de Dieu, à
\item[\vref{2 Ti 2:26}] ont été pris pour faire sa v.
\item[\vref{Hé 10:7}] Que je fasse, ô Dieu, ta v. !
\item[\vref{1 Pi 4:2}] mais selon la v. de Dieu, pendant
\item[\vref{1 Pi 4:3}] d'avoir accompli la v. des Gentils, pendant
\item[\vref{2 Pi 1:21}] apportée par la v. humaine, mais c'est
\item[\vref{1 Jn 5:14}] chose selon sa v., il ns. exauce.
\item[\vref{Ap 4:11}] c'est par ta v. qu'elles existent et
\end{listverse}

\ConcordanceEntry{Vomir}
\vspace{-2mm}
\begin{listverse}
\item[\vref{Lé 18:25}] et le pays v. ses habitants.
\item[\vref{Job 20:15}] mais il les v. ; Dieu les jettera
\item[\vref{Ps 57:5}] de gens qui v. la flamme, parmi
\item[\vref{Pr 23:8}] Tu voudrais v. le morceau que
\item[\vref{Pr 25:16}] qu'en étant rassasié, tu ne le v.
\item[\vref{Pr 26:11}] ce qu'il a v., ainsi l'insensé réitère
\item[\vref{Jé 25:27}] soyez enivrés, mm v., et tombez sans
\item[\vref{Jon 2:11}] et le poisson v. Jonas sur la
\item[\vref{2 Pi 2:22}] ce qu'il avait v., et la truie
\item[\vref{Ap 3:16}] bouillant, je te v. de ma bouche.
\end{listverse}

\ConcordanceEntry{Vouloir}
\vspace{-2mm}
\begin{listverse}
\item[\vref{Ps 115:3}] ciel, il fait tt ce qu'il v.
\item[\vref{Mt 8:2}] si tu le v., tu peux me
\item[\vref{Mt 26:39}] com. je le v., mais com. tu
\item[\vref{Mc 1:41}] disant : Je le v., sois pur.
\item[\vref{Lu 13:34}] de fois ai-je v. rassembler tes enfants,
\item[\vref{Jn 5:40}] Mais vs. ne v. pas venir à
\item[\vref{Ac 9:6}] dit : Seign., que v.-tu que je
\item[\vref{Ro 7:16}] que je ne v. pas, je reconnais
\item[\vref{Ro 7:19}] bien que je v., mais je fais
\item[\vref{Ro 9:16}] de celui qui v., ni de celui
\item[\vref{Ga 5:17}] ne fassiez pas ce que vs. v.
\item[\vref{Ph 2:13}] avec efficacité le v. et le faire,
\item[\vref{Ja 4:4}] Celui dc qui v. être ami du
\item[\vref{Ja 4:15}] le Seign. le v., et si ns.
\item[\vref{Ap 11:6}] plaies, ttes les fois qu'ils le v.
\item[\vref{Ap 22:17}] que celui qui v., prenne gratuitement de
\end{listverse}

\ConcordanceEntry{Voyage}
\vspace{-2mm}
\begin{listverse}
\item[\vref{Ge 24:21}] voir si Yahweh faisait réussir son v. ou non.
\item[\vref{Ge 28:20}] garde pendant le v. que je fais,
\item[\vref{De 6:7}] tu iras en v., qnd tu te
\item[\vref{Jos 5:7}] les avait pas circoncis pendant le v.
\item[\vref{Esd 8:21}] donner un heureux v., pour nos enfants,
\item[\vref{Né 2:6}] lui : Combien ton v. durera-t-il, et qnd
\item[\vref{Pr 7:19}] parti pour un v. lointain.
\item[\vref{Mt 10:10}] sac pour le v., ni deux tuniques,
\item[\vref{Mt 25:14}] partant pour un v., appela ses serviteurs
\item[\vref{Mc 13:34}] partant pour un v., laisse sa maison,
\item[\vref{Jn 4:6}] Jésus, fatigué du v., se tenait là
\item[\vref{Ac 19:29}] Gaïus et Aristarque, Macédoniens, compagnons de v. de Paul.
\item[\vref{2 Co 11:26}] été souvent en v., en péril sur
\end{listverse}

\ConcordanceEntry{Voyant (un)}
\vspace{-2mm}
\begin{listverse}
\item[\vref{1 S 9:9}] allons vers le v. ! Car le prophète
\item[\vref{2 S 24:11}] qui était le v. de David :
\item[\vref{2 R 17:13}] prophètes, ts les v., en disant : Détournez-vs.
\item[\vref{2 Ch 12:15}] et d'Iddo le v., parmi les registres
\item[\vref{2 Ch 16:7}] temps-là, Hanani le v., vint vers Asa,
\item[\vref{2 Ch 29:25}] David, Gad, le v. du roi, et
\item[\vref{2 Ch 29:30}] et d'Asaph le v. ; et ils le
\item[\vref{2 Ch 35:15}] de Jeduthun, le v. du roi. Les
\item[\vref{Es 30:10}] qui disent aux v. : Ne voyez pas !
\end{listverse}

\ConcordanceEntry{Vrai}
\vspace{-2mm}
\begin{listverse}
\item[\vref{1 R 10:6}] sujet de ta sagesse était dc v. !
\item[\vref{Esd 5:17}] voir s'il est v. qu'il y a
\item[\vref{Ps 19:10}] de Yahweh sont v., et ils sont
\item[\vref{Es 43:9}] les entende et qu'on dise : C'est v. !
\item[\vref{Mt 15:27}] dit : Cela est v., Seign. ! Cependant les
\item[\vref{Mt 17:11}] dit : Il est v. qu'Elie viendra premièrement
\item[\vref{Mt 20:23}] dit : Il est v. que vs. boirez
\item[\vref{Jn 4:23}] venue, où les v. adorateurs adoreront le
\item[\vref{Jn 4:37}] dit d'ordinaire est v. en ceci : L'un
\item[\vref{Jn 6:32}] vs. donne le v. pain du ciel ;
\item[\vref{Jn 10:41}] a dit de cet hom. était v.
\item[\vref{Jn 15:1}] JE SUIS le v. cep, et mon
\item[\vref{Jn 17:3}] toi, le seul v. Dieu, et celui
\item[\vref{Ac 12:15}] affirma que ce qu'elle disait était v.
\item[\vref{Ro 2:25}] Or il est v. que la circoncision
\item[\vref{Ro 3:4}] soit reconnu pour v., et tt hom.
\item[\vref{2 Co 6:8}] bonne réputation ; com. séducteurs et toutefois v. ;
\item[\vref{Ph 4:3}] toi aussi, mon v. compagnon, oui je
\item[\vref{1 Th 1:9}] pour servir le Dieu vivant et v.,
\item[\vref{Tit 1:4}] à Tite, mon v. fils, selon la
\item[\vref{Ap 14:13}] Seign. ! Oui, c'est v. ! dit l'Esprit, afin
\end{listverse}

\ConcordanceEntry{Vue}
\vspace{-2mm}
\begin{listverse}
\item[\vref{Ge 2:9}] désirable à la v. et bon à
\item[\vref{De 34:7}] il mourut ; sa v. n'était point affaiblie,
\item[\vref{Ha 1:8}] ils ont la v. plus aiguë que
\item[\vref{Mt 11:5}] aveugles recouvrent la v., les boiteux marchent,
\item[\vref{Mt 20:34}] ils recouvrèrent la v., et ils le
\item[\vref{Mc 10:51}] dit : Maître, que je recouvre la v.
\item[\vref{Lu 4:19}] recouvrement de la v., pour mettre en
\item[\vref{Lu 7:21}] il rendit la v. à plusieurs aveugles.
\item[\vref{Lu 18:43}] il recouvra la v. et suivit Jésus,
\item[\vref{Jn 9:18}] avait recouvré la v., jusqu'à ce qu'ils
\item[\vref{Ac 9:17}] tu recouvres la v. et que tu
\item[\vref{Ac 22:13}] frère, recouvre la v.. Au mm instant,
\item[\vref{2 Co 5:7}] la foi et non par la v.
\end{listverse}

\ConcordanceEntry{Yahweh}
\vspace{-2mm}
\begin{listverse}
\item[\vref{Ge 2:5}] encore germé ; car Y. Dieu n'avait pas
\item[\vref{Ge 2:7}] Or Y. Dieu forma l'hom. de la poussière
\item[\vref{Ge 3:23}] Et Y. Dieu le chassa du jardin d'Eden
\item[\vref{Ge 4:26}] commença à invoquer le Nom de Y.
\item[\vref{Ge 6:3}] Et Y. dit : Mon Esprit ne contestera point
\item[\vref{Ge 7:1}] Et Y. dit à Noé : Entre, toi et
\item[\vref{Ge 8:21}] Et Y. respira une odeur d'apaisement, et dit
\item[\vref{Ge 11:5}] Alors Y. descendit pour voir la ville et
\item[\vref{Ge 12:7}] Et Y. apparut à Abram, et lui dit :
\item[\vref{Ge 15:18}] En ce jour-là, Y. traita alliance avec
\item[\vref{Ge 21:1}] Et Y. visita Sara, com. il avait dit ;
\item[\vref{Ge 24:26}] Alors l'hom. s'inclina et adora Y.,
\item[\vref{Ge 24:44}] la fem. que Y. a destinée au
\item[\vref{Ge 25:21}] Isaac pria instamment Y. au sujet de
\item[\vref{Ge 25:22}] suis-je enceinte ? Et elle alla consulter Y.
\item[\vref{Ge 28:21}] mon père, alors Y. sera mon Dieu.
\item[\vref{Ge 31:3}] Alors Y. dit à Jacob : Retourne au pays
\item[\vref{Ge 39:2}] Et Y. était avec Joseph ; et il prospéra
\item[\vref{Ge 49:18}] Ô Y. ! J'espère en ton salut !
\item[\vref{Ex 3:2}] Et l'Ange de Y. lui apparut ds
\item[\vref{Ex 5:2}] dit : Qui est Y. pour que j'obéisse
\item[\vref{Ex 6:3}] été connu d'eux par mon Nom Y.
\item[\vref{Ex 12:23}] Car Y. passera pour frapper l'Egypte et il
\item[\vref{Ex 13:12}] tu consacreras à Y. tt premier-né issu
\item[\vref{Ex 14:14}] Y. combattra pour vs. et vs. resterez
\item[\vref{Ex 15:1}] ce cantique à Y., et dirent : Je
\item[\vref{Ex 16:9}] la présence de Y., car il a
\item[\vref{Ex 17:15}] et le nomma Y., ma bannière.
\item[\vref{Ex 19:11}] au troisième jour, Y. descendra sur la
\item[\vref{Ex 31:13}] que je suis Y. qui vs. sanctifie.
\item[\vref{Ex 33:11}] Et Y. parlait à Moïse face à face,
\item[\vref{Ex 34:28}] demeura là avec Y. quarante jours et
\item[\vref{Lé 18:1}] Y. parla encore à Moïse, en disant :
\item[\vref{Lé 19:2}] suis saint, moi, Y., votre Dieu.
\item[\vref{Lé 20:7}] car je suis Y., votre Dieu.
\item[\vref{Lé 22:32}] d'Israël. Je suis Y., qui vs. sanctifie,
\item[\vref{Lé 24:16}] le Nom de Y., il mourra, il
\item[\vref{No 10:35}] disait : Lève-toi, ô Y., et tes ennemis
\item[\vref{No 15:22}] ces commandements que Y. a fait connaître
\item[\vref{De 4:23}] d'oublier l'alliance de Y., votre Dieu, qu'il
\item[\vref{De 4:35}] tu reconnaisses que Y. est Dieu et
\item[\vref{De 5:11}] le Nom de Y., ton Dieu, en
\item[\vref{De 6:4}] Ecoute Israël ! Y., notre Dieu, Yahweh
\item[\vref{De 6:5}] Tu aimeras dc Y., ton Dieu, de
\item[\vref{De 23:14}] Car Y., ton Dieu, marche au milieu de
\item[\vref{De 26:1}] le pays que Y., ton Dieu, te
\item[\vref{De 29:29}] cachées sont à Y., notre Dieu ; les
\item[\vref{De 30:2}] tu reviens à Y., ton Dieu, et
\item[\vref{De 32:36}] Mais Y. jugera son peuple ; et il se
\item[\vref{Jos 22:31}] reconnaissons aujourd'hui que Y. est au milieu
\item[\vref{Jos 23:8}] Mais attachez-vs. à Y., votre Dieu, com.
\item[\vref{Jos 24:24}] Josué : Nous servirons Y., notre Dieu et
\item[\vref{Jg 2:7}] Le peuple servit Y. tt le temps
\item[\vref{Jg 2:13}] Ils abandonnèrent dc Y., et servirent Baal
\item[\vref{Jg 5:3}] je chanterai à Y., je chanterai un
\item[\vref{Jg 6:24}] un autel à Y., et lui donna
\item[\vref{1 S 15:22}] Samuel répondit : Y. prend-il plaisir aux
\item[\vref{2 S 22:31}] la parole de Y. est éprouvée ; il
\item[\vref{2 S 24:10}] David dit à Y. : J'ai commis un
\item[\vref{1 R 3:5}] Et Y. apparut de nuit à Salomon à
\item[\vref{1 R 18:39}] et dirent : C'est Y. qui est Dieu !
\item[\vref{2 R 2:1}] il arriva lorsque Y. enleva Elie au
\item[\vref{2 R 17:36}] Mais vs. craindrez Y., qui vs. a
\item[\vref{2 Ch 16:12}] ne chercha point Y. ds sa maladie,
\item[\vref{Né 8:10}] la joie de Y. est votre force.
\item[\vref{Job 40:1}] Et Y. répondit à Job du milieu d'un
\item[\vref{Ps 1:2}] la loi de Y., et qui médite
\item[\vref{Ps 3:9}] délivrance vient de Y. ! Que ta bénédiction
\item[\vref{Ps 4:6}] sacrifices de justice, et confiez-vs. en Y.
\item[\vref{Ps 18:3}] Y. est mon Rocher, ma forteresse et
\item[\vref{Ps 19:8}] La loi de Y. est parfaite, elle
\item[\vref{Ps 22:9}] s'abandonne, disent-ils, à Y. ! Qu'il te délivre,
\item[\vref{Ps 23:1}] Psaume de David. Y. est mon berger:
\item[\vref{Ps 25:12}] l'hom. qui craint Y. ? Yahweh lui enseignera
\item[\vref{Ps 26:2}] Sonde-moi et éprouve-moi, Y. ! Fais passer au
\item[\vref{Ps 31:6}] me rachèteras, ô Y., le Dieu de
\item[\vref{Ps 33:13}] Y. regarde des cieux, il voit ts
\item[\vref{Ps 34:8}] Heth.] L'Ange de Y. campe tt autour
\item[\vref{Ps 47:3}] car Y., le Très-Haut, est terrible. Il est
\item[\vref{Ps 72:18}] Béni soit Y. Dieu, le Dieu
\item[\vref{Ps 94:11}] Y. connaît les pensées des hommes qui
\item[\vref{Ps 96:10}] parmi les nations : Y. règne ; mm le
\item[\vref{Ps 100:3}] Sachez que Y. est Dieu ! C'est
\item[\vref{Ps 105:1}] Célébrez Y., invoquez son Nom ! Faites connaître parmi
\item[\vref{Ps 111:2}] Les œuvres de Y. sont grandes, [Daleth.]
\item[\vref{Ps 118:16}] La droite de Y. est élevée ! La
\item[\vref{Ps 150:6}] qui respire loue Y. ! Louez Yahweh !
\item[\vref{Pr 1:7}] La crainte de Y. est la principale
\item[\vref{Pr 3:9}] Honore Y. avec tes biens et les prémices
\item[\vref{Pr 10:22}] La bénédiction de Y. est celle qui
\item[\vref{Pr 15:33}] La crainte de Y. enseigne la sagesse,
\item[\vref{Pr 16:3}] tes affaires à Y., et tes projets
\item[\vref{Pr 16:4}] Y. a fait ttes choses pour lui-mm ;
\item[\vref{Pr 21:31}] bataille, mais la délivrance vient de Y.
\item[\vref{Pr 22:12}] Les yeux de Y. veillent sur la
\item[\vref{Pr 31:30}] fem. qui craint Y. est celle qui
\item[\vref{Es 2:11}] s'élèvent seront humiliés, Y. sera seul haut
\item[\vref{Es 2:18}] Y. seul sera élevé en ce jour-là.
\item[\vref{Es 4:2}] le germe de Y. sera plein de
\item[\vref{Es 6:3}] saint, saint est Y. des armées ! Toute
\item[\vref{Es 9:6}] le zèle de Y. des armées.
\item[\vref{Es 11:2}] L'Esprit de Y. reposera sur lui,
\item[\vref{Es 34:16}] le livre de Y. et lisez : Il
\item[\vref{Es 37:20}] Maintenant dc, ô Y., notre Dieu ! Délivre-ns.
\item[\vref{Es 41:4}] le commencement. Moi, Y., JE SUIS le
\item[\vref{Es 42:8}] Je suis Y., c'est là mon Nom ; et je
\item[\vref{Es 43:10}] mes témoins, dit Y., et mon serviteur
\item[\vref{Es 43:11}] Moi, Je suis Y., et à part
\item[\vref{Es 45:6}] Dieu. Je suis Y., et il n'y
\item[\vref{Es 45:7}] crée l'adversité ; moi, Y., je fais ttes
\item[\vref{Es 49:7}] Ainsi parle Y., le Rédempteur, le
\item[\vref{Es 51:1}] et qui cherchez Y. ! Regardez au rocher
\item[\vref{Es 51:9}] force, bras de Y. ! Réveille-toi com. aux
\item[\vref{Es 53:1}] le bras de Y. a-t-il été révélé ?
\item[\vref{Es 54:5}] est ton époux : Y. des armées est
\item[\vref{Es 60:2}] les peuples ; mais Y. se lève sur
\item[\vref{Es 60:16}] que je suis Y., ton Sauveur, ton
\item[\vref{Es 60:19}] t'éclairera plus, mais Y. sera pour toi
\item[\vref{Es 61:1}] L'Esprit du Seign. Y. est sur moi,
\item[\vref{Es 66:15}] Car voici, Y. viendra avec le
\item[\vref{Jé 3:14}] rebelles, convertissez-vs., dit Y., car j'ai droit
\item[\vref{Jé 4:4}] Jérus., circoncisez-vs. pour Y., circoncisez vos cœurs,
\item[\vref{Jé 17:5}] Ainsi parle Y. : Maudit soit l'hom.
\item[\vref{Jé 17:10}] Je suis Y., qui sonde le cœur, et qui
\item[\vref{Jé 23:6}] dont on l'appellera : Y. notre justice.
\item[\vref{Jé 29:10}] Car ainsi parle Y. : Lorsque les soixante-dix
\item[\vref{Jé 51:50}] pas ; souvenez-vs. de Y. ds ces pays
\item[\vref{La 1:18}] [Tsade.] Y. est juste car j'ai été rebelle
\item[\vref{Ez 3:14}] la main de Y. me fortifia.
\item[\vref{Ez 11:5}] L'Esprit de Y. tomba sur moi.
\item[\vref{Ez 37:28}] que je suis Y., qui sanctifie Israël,
\item[\vref{Ez 48:10}] le sanctuaire de Y. sera au milieu.
\item[\vref{Ez 48:35}] la ville depuis ce jour-là sera : Y. est ici.
\item[\vref{Da 9:4}] Je priai Y., mon Dieu, et je lui fis
\item[\vref{Os 6:3}] Alors ns. connaîtrons Y., et ns. poursuivrons
\item[\vref{Joë 2:12}] Maintenant encore, dit Y., revenez à moi
\item[\vref{Joë 2:32}] le Nom de Y. sera sauvé ; car
\item[\vref{Am 4:13}] de la terre ; Y., LE DIEU DES
\item[\vref{Am 5:6}] Cherchez Y., et vs. vivrez, de peur qu'il
\item[\vref{Jon 1:14}] Alors ils invoquèrent Y., et dirent : Ô
\item[\vref{Jon 2:10}] faits : Car le salut vient de Y.
\item[\vref{Mi 1:3}] Car voici, Y. sortira de son
\item[\vref{Mi 4:6}] ce jour-là, dit Y., je rassemblerai les
\item[\vref{Mi 7:8}] ds les ténèbres, Y. m'éclairera.
\item[\vref{Na 1:2}] Y. est un Dieu jaloux, il se
\item[\vref{Na 1:7}] Y. est bon, il est une forteresse
\item[\vref{Ha 2:16}] la droite de Y. fera le tour
\item[\vref{Ha 2:20}] Mais Y. est ds le temple de sa
\item[\vref{Ha 3:18}] me réjouis en Y., et je me
\item[\vref{So 2:3}] qu'il ordonne, cherchez Y., cherchez la justice,
\item[\vref{Ag 2:4}] Zorobabel, fortifie-toi ! dit Y.. Toi aussi, Josué,
\item[\vref{Za 4:10}] les yeux de Y. qui parcourent tte
\item[\vref{Za 8:22}] nations viendront rechercher Y. des armées à
\item[\vref{Za 10:1}] Demandez à Y. la pluie, la
\item[\vref{Za 12:7}] Y. sauvera premièrement les tentes de Juda,
\item[\vref{Za 14:1}] le jour de Y. arrive, et tes
\item[\vref{Za 14:7}] unique, connu de Y., et qui ne
\item[\vref{Za 14:9}] Y. sera Roi sur tte la terre ;
\item[\vref{Mal 2:2}] mon Nom, dit Y. des armées, j'enverrai
\item[\vref{Mal 3:1}] il vient, dit Y. des armées.
\end{listverse}

\ConcordanceEntry{Zabulon}
\vspace{-2mm}
\begin{listverse}
\item[\vref{Ge 30:20}] pourquoi elle l'appela du nom de Z.
\item[\vref{Ap 7:8}] la tribu de Z., douze mille marqués
\end{listverse}
\begin{legend}
\NoAutoSpaceBeforeFDP{
\item Fils de Jacob et Léa : Ge 30:20; De 33:18
\item Territoire de Z et prophétie de Jacob : Ge 49:13; Jos 19: 10-16
\item Autres: No 1:30; 26:26; 2 Ch 30:11; Ps 68:28; Mt 4 :15
}
\end{legend}

\ConcordanceEntry{Zacharie}
\vspace{-2mm}
\begin{listverse}
\item[\vref{2 R 15:8}] roi de Juda, Z., fils de Jéroboam,
\item[\vref{Za 1:1}] Yahweh vint à Z., le prophète, fils
\item[\vref{Lu 1:5}] un prêtre nommé Z., de la classe
\end{listverse}
\begin{legend}
\NoAutoSpaceBeforeFDP{
\item Fils de Jéroboam et roi d'Israël : 2 R 15:8-12
\item Fils de Jéhojada, le prêtre : 2 Ch 24:20-22
\item Prophète et don de vision 2 Ch 26:5
\item Zacharie fils de Barachie : Mt 23:35
\item Fils d'Iddo, le Prophète : Esd 5:1; Za 1:1
\item Prêtre et père de Jean-Baptiste : Lu 1:5-6,13-20,63
}
\end{legend}

\ConcordanceEntry{Zachée}
\vspace{-2mm}
\begin{listverse}
\item[\vref{Lu 19:2}] hom. riche, appelé Z., chef des publicains,
\item[\vref{Lu 19:8}] Et Z., se présentant dvt le Seign., lui
\end{listverse}

\ConcordanceEntry{Zébach}
\vspace{-2mm}
\begin{listverse}
\item[\vref{Jg 8:5}] ainsi, je poursuivrai Z. et Tsalmunna, rois
\item[\vref{Ps 83:12}] princes soient com. Z. et Tsalmunna !
\end{listverse}

\ConcordanceEntry{Zébédée}
\vspace{-2mm}
\begin{listverse}
\item[\vref{Mt 20:20}] des fils de Z. s'approcha de lui
\item[\vref{Mc 1:20}] laissant lr. père Z. ds la barque
\end{listverse}

\ConcordanceEntry{Zélateur}
\vspace{-2mm}
\begin{listverse}
\item[\vref{Ga 1:14}] le plus ardent z. des traditions de
\end{listverse}

\ConcordanceEntry{Zèle}
\vspace{-2mm}
\begin{listverse}
\item[\vref{No 25:11}] animé de mon z. au milieu d'eux ;
\item[\vref{2 S 21:2}] Saül ds son z. pour les enfants
\item[\vref{1 R 19:10}] J'ai déployé mon z. pour Yahweh, le
\item[\vref{2 R 10:16}] tu verras le z. que j'ai pour
\item[\vref{2 R 19:31}] que fera le z. de Yahweh des
\item[\vref{Esd 7:21}] cieux, vs. demandera, soit fait avec z.,
\item[\vref{Ps 69:10}] Car le z. de ta maison me dévore, et
\item[\vref{Ps 119:139}] Mon z. me consume parce que mes adversaires
\item[\vref{Es 9:6}] que fera le z. de Yahweh des
\item[\vref{Es 63:15}] Où sont ton z. et ta puissance ?
\item[\vref{Jn 2:17}] est écrit : Le z. de ta maison
\item[\vref{Ac 22:3}] étant plein de z. pour la loi
\item[\vref{Ro 10:2}] qu'ils ont du z. pour Dieu, mais
\item[\vref{1 Co 15:58}] avec un nouveau z. à l'œuvre du
\item[\vref{2 Co 8:7}] en connaissance, en z., et ds la
\item[\vref{Ga 4:17}] Ils ont du z. pour vs., mais
\item[\vref{Ph 3:6}] quant au z., persécutant l'Eglise ; et quant à la
\item[\vref{Col 4:13}] a un grand z. pour vs., et
\item[\vref{Ap 3:19}] Aie dc du z. et repens-toi.
\end{listverse}

\ConcordanceEntry{Zélé}
\vspace{-2mm}
\begin{listverse}
\item[\vref{Ac 21:20}] ils sont ts z. pour la loi.
\item[\vref{2 Co 8:17}] étant lui-mm très z., il est allé
\item[\vref{Ga 4:18}] est bon d'être z. pour le bien
\item[\vref{Tit 2:14}] et qui soit z. pour les bonnes
\end{listverse}

\ConcordanceEntry{Zélote}
\vspace{-2mm}
\begin{listverse}
\item[\vref{Lu 6:15}] Jacques, fils d'Alphée, et Simon, surnommé z. ;
\item[\vref{Ac 1:13}] et Simon le z., et Jude, frère
\end{listverse}

\ConcordanceEntry{Zilpa}
\vspace{-2mm}
\begin{listverse}
\item[\vref{Ge 29:24}] Et Laban donna Z., sa servante, à
\item[\vref{Ge 35:26}] Les fils de Z., servante de Léa :
\end{listverse}

\ConcordanceEntry{Zimri}
\vspace{-2mm}
\begin{listverse}
\item[\vref{1 R 16:9}] Son serviteur, Z., capitaine de la
\item[\vref{1 R 16:20}] des actions de Z. et la conspiration
\end{listverse}

\ConcordanceEntry{Ziph}
\vspace{-2mm}
\begin{listverse}
\item[\vref{1 S 23:15}] au désert de Z., ds la forêt.
\item[\vref{1 S 26:2}] au désert de Z., avec trois mille
\end{listverse}

\ConcordanceEntry{Ziph (fils)}
\vspace{-2mm}
\begin{listverse}
\item[\vref{1 Ch 2:42}] le père de Z., et les fils
\item[\vref{1 Ch 4:16}] de Jehalléleel furent Z., Zipha, Thirja, et
\end{listverse}

\ConcordanceEntry{Zodiaque}
\vspace{-2mm}
\begin{listverse}
\item[\vref{2 R 23:5}] la lune, au z. et à tte
\item[\vref{Job 38:32}] les signes du z., et conduis-tu la
\end{listverse}

\ConcordanceEntry{Zorobabel}
\vspace{-2mm}
\begin{listverse}
\item[\vref{Ag 1:14}] réveilla l'esprit de Z., fils de Schealthiel,
\end{listverse}
\begin{legend}
\NoAutoSpaceBeforeFDP{
\item Fils de Schealthiel et gouverneur de Juda : Ag 2:2,21
\item Reprise des travaux du temple : Esd 3:8; 5:2
\item Autres : Esd 2:2; Ag 1:1; 2:4; Za 4:6
}
\end{legend}

}

\end{multicols}
