\ShortTitle{2 Ch.}\BookTitle{2 Chroniques}\BFont
\noindent\hrulefill
{\footnotesize
\textit{
\bigskip
{\centering{}
\\Auteur~: Inconnu
\\(Heb.~: Hayyamim dibre)
\\Signification~: Actes des journées
\\Thème~: La grandeur de Juda
\\Date de rédaction~: 5\up{ème} siècle av. J.-C.\\}
}
\textit{
\\Initialement, 1 et 2 Chroniques ne constituaient qu'un seul ouvrage. Ce livre raconte le règne de Salomon, la construction de la maison de Dieu et du palais. Il reprend ensuite l'histoire des royaumes d'Israël et de Juda, du schisme à la captivité babylonienne, mettant en exergue l'instabilité du peuple dont le cœur balançait entre Yahweh et les idoles.\bigskip
}
}
\par\nobreak\noindent\hrulefill
\begin{multicols}{2}
\Chap{1}
\TextTitle{Yahweh élève Salomon qui demande la sagesse\FTNTT{1 R. 2:12~; 3:4-9~; 1 Ch. 29:23-25.}}
\VerseOne{}Or Salomon, fils de David, se fortifia dans son royaume~; Yahweh, son Dieu, fut avec lui, et l'éleva au plus haut.
\VS{2}Salomon parla à tout Israël, aux chefs de milliers et de centaines, aux juges et à tous les principaux de tout Israël, chefs des pères.
\VS{3}Salomon et toute l'assemblée avec lui allèrent au haut lieu qui était à Gabaon~; car là était la tente d'assignation de Dieu, que Moïse, serviteur de Yahweh, avait faite dans le désert.
\VS{4}Mais David avait fait monter l'arche de Dieu de Kirjath-Jearim au lieu qu'il avait préparé~; car il lui avait dressé une tente à Jérusalem.
\VS{5}L'autel d'airain que Betsaleel, fils d'Uri, fils de Hur, avait fait, était là devant le tabernacle de Yahweh. Et Salomon et l'assemblée y cherchèrent Yahweh\FTNT{Ex. 27:1-8~; Ex. 36:1-2.}.
\VS{6}Salomon offrit là, devant Yahweh, mille holocaustes, sur l'autel d'airain qui était devant la tente d'assignation.
\VS{7}En cette nuit-là, Dieu apparut à Salomon, et lui dit~: Demande ce que tu veux que je te donne.
\VS{8}Et Salomon répondit à Dieu~: Tu as usé d'une grande bienveillance envers David, mon père, et tu m'as établi roi à sa place.
\VS{9}Maintenant, ô Yahweh Dieu~! que ta parole à David, mon père, se confirme~; car tu m'as établi roi sur un peuple nombreux comme la poussière de la terre.
\VS{10}Donne-moi donc maintenant de la sagesse et de l'intelligence, afin que je sache me conduire devant ce peuple~; car qui pourrait juger ton peuple, ce peuple si grand~?
\TextTitle{Yahweh agrée la prière de Salomon et l'exauce\FTNTT{1 R. 3:10-28.}}
\VS{11}Et Dieu dit à Salomon~: Puisque c'est là ce qui est dans ton cœur, et que tu n'as demandé ni des richesses, ni des biens, ni de la gloire, ni la mort de ceux qui te haïssent, ni même des jours nombreux, mais que tu as demandé pour toi de la sagesse et de l'intelligence, afin de pouvoir juger mon peuple, sur lequel je t'ai établi roi,
\VS{12}la sagesse et l'intelligence te sont données. Je te donnerai aussi des richesses, des biens et de la gloire, comme n'en ont pas eu les rois qui ont été avant toi, et comme il n'en aura aucun après toi.
\VS{13}Puis Salomon s'en retourna à Jérusalem, du haut lieu qui était à Gabaon devant la tente d'assignation~; et il régna sur Israël.
\VS{14}Salomon rassembla des chars et des cavaliers~; il avait quatorze cents chars et douze mille cavaliers~; et il les plaça dans les villes où il tenait ses chars, et auprès du roi, à Jérusalem.
\VS{15}Et le roi fit que l'argent et l'or étaient aussi communs à Jérusalem que les pierres, et les cèdres, que les sycomores de la plaine.
\VS{16}Le lieu d'où étaient issus les chevaux de Salomon était l'Egypte~; une caravane de marchands du roi allait les prendre par troupe à un prix convenu.
\VS{17}On faisait monter et sortir d'Egypte un char pour six cents sicles d'argent, et un cheval pour cent cinquante. On en amenait de même par eux pour tous les rois des Héthiens, et pour les rois de Syrie.
\Chap{2}
\TextTitle{La prière de Salomon exaucée\FTNTT{1 R. 5:1-18~; 7:13,14.}}
\VerseOne{}Or Salomon ordonna de bâtir une maison au Nom de Yahweh, ainsi qu'une maison royale.
\VS{2}Et il fit un dénombrement de soixante-dix mille hommes qui portaient les fardeaux, et de quatre vingt mille qui coupaient le bois sur la montagne, et de trois mille six cents qui étaient commis sur eux.
\VS{3}Puis Salomon envoya vers Huram, roi de Tyr, pour lui dire~: Fais pour moi comme tu as fait pour David, mon père, à qui tu as envoyé des cèdres, pour se bâtir une maison afin d'y habiter.
\VS{4}Voici, je vais bâtir une maison au Nom de Yahweh, mon Dieu, pour la lui consacrer, pour faire brûler devant lui le parfum des aromates, pour présenter continuellement devant lui les pains de proposition, et pour offrir les holocaustes du matin et du soir, des sabbats, des nouvelles lunes, et des fêtes de Yahweh, notre Dieu, ce qui est perpétuel en Israël.
\VS{5}La maison que je vais bâtir sera grande~; car notre Dieu est plus grand que tous les dieux.
\VS{6}Mais qui aurait le pouvoir de lui bâtir une maison, puisque les cieux et les cieux des cieux ne sauraient le contenir~? Et qui suis-je pour lui bâtir une maison, si ce n'est pour faire brûler des parfums devant sa face~?
\VS{7}Maintenant, envoie-moi un homme habile pour travailler l'or, l'argent, l'airain et le fer, en écarlate, en cramoisi et en pourpre, sachant faire des sculptures, pour travailler avec les hommes habiles que j'ai avec moi en Juda et à Jérusalem, et que David, mon père, a préparés.
\VS{8}Envoie-moi aussi du Liban du bois de cèdre, de cyprès et de santal~; car je sais que tes serviteurs savent couper les bois du Liban. Voici, mes serviteurs seront avec les tiens.
\VS{9}Qu'on me prépare du bois en grande quantité~; car la maison que je vais bâtir sera grande et magnifique.
\VS{10}Et je donnerai à tes serviteurs qui couperont, qui abattront les bois, vingt mille cors de froment foulé, vingt mille cors d'orge, vingt mille baths de vin, et vingt mille baths d'huile.
\VS{11}Huram, roi de Tyr, répondit dans un écrit qu'il envoya à Salomon~: C'est parce que Yahweh aime son peuple qu'il t'a établi roi sur eux.
\VS{12}Et Huram dit~: Béni soit Yahweh, le Dieu d'Israël, qui a fait les cieux et la terre, de ce qu'il a donné au roi David un fils sage, prudent et intelligent, qui va bâtir une maison à Yahweh, et une maison royale~!
\VS{13}Je t'envoie donc un homme habile et intelligent, Huram-Abi,
\VS{14}fils d'une femme d'entre les filles de Dan, et d'un père tyrien. Il sait travailler l'or, l'argent, l'airain et le fer, les pierres et le bois, en écarlate, en pourpre, en fin lin et en cramoisi~; il sait faire toutes sortes de sculptures et imaginer toutes sortes d'objets d'art qu'on lui donne à faire. Il travaillera avec tes hommes habiles et avec les hommes habiles de mon seigneur David, ton père.
\VS{15}Et maintenant, que mon seigneur envoie à ses serviteurs le froment, l'orge, l'huile et le vin comme il l'a dit.
\VS{16}Et nous couperons des bois du Liban autant que tu en auras besoin, et nous te les amènerons en radeaux, par la mer, jusqu'à Japho, et tu les feras monter à Jérusalem.
\VS{17}Alors Salomon compta tous les hommes étrangers qui étaient au pays d'Israël, d'après le dénombrement que David, son père, en avait fait. On en trouva cent cinquante-trois mille six cents.
\VS{18}Et il en établit soixante-dix mille qui portaient des fardeaux, quatre-vingt mille qui taillaient les pierres dans la montagne, et trois mille six cents surveillants pour faire travailler le peuple.
\Chap{3}
\TextTitle{Salomon commence la construction du temple\FTNTT{1 R. 6:1.}}
\VerseOne{}Salomon commença donc à bâtir la maison de Yahweh à Jérusalem, sur la montagne de Morija, qui avait été indiquée à David, son père, au lieu même que David avait préparé dans l'aire d'Ornan, le Jébusien.
\VS{2}Il commença à bâtir, le second jour du second mois, la quatrième année de son règne.
\TextTitle{Les matériaux du temple et les dimensions\FTNTT{1 R. 6:2-38~; 7:13-22.}}
\VS{3}Or voici les fondements fixés par Salomon pour bâtir la maison de Dieu~: La longueur, en coudées de l'ancienne mesure, était de soixante coudées, et la largeur de vingt coudées.
\VS{4}Le portique qui était sur le devant, et dont la longueur répondait à la largeur de la maison, avait vingt coudées, et cent vingt de hauteur. Il le revêtit intérieurement d'or pur.
\VS{5}Et il recouvrit la grande maison de bois de cyprès~; il la revêtit d'or fin, et y fit mettre des palmes et des chaînettes.
\VS{6}Il revêtit la maison de pierres précieuses, pour l'ornement~; et l'or était de l'or de Parvaïm.
\VS{7}Il revêtit d'or la maison, les poutres, les seuils, les parois et les portes~; et il fit sculpter des chérubins sur les parois.
\VS{8}Il fit le Saint des saints, dont la longueur était de vingt coudées, selon la largeur de la maison, et la largeur de vingt coudées~; et il le couvrit d'or fin, pour une valeur de six cents talents.
\VS{9}Et le poids des clous montait à cinquante sicles d'or. Il revêtit aussi d'or les chambres hautes.
\VS{10}Il fit dans le Saint des saints deux chérubins sculptés, et on les couvrit d'or~;
\VS{11}La longueur des ailes des chérubins était de vingt coudées. L'aile du premier, longue de cinq coudées, touchait la paroi de la maison, et l'autre aile, longue de cinq coudées, touchait une aile de l'autre chérubin.
\VS{12}Et une aile de l'autre chérubin, longue de cinq coudées, touchait la paroi de la maison~; et l'autre aile longue de cinq coudées, joignait l'aile de l'autre chérubin.
\VS{13}Les ailes étendues de ces chérubins faisaient vingt coudées. Ils se tenaient debout sur leurs pieds, leurs faces tournées vers la maison.
\VS{14}Il fit le voile de pourpre, d'écarlate, de cramoisi et de fin lin: Et il y représenta par dessus des chérubins.
\VS{15}Devant la maison, il fit deux colonnes de trente-cinq coudées de hauteur, et le chapiteau sur leur sommet était de cinq coudées.
\VS{16}Il fit des chaînes dans le sanctuaire~; et il en mit sur le sommet des colonnes~; et il fit cent grenades qu'il mit aux chaînes.
\VS{17}Il dressa les colonnes sur le devant du temple, l'une à droite, et l'autre à gauche~; il appela celle de droite Jakin, et celle de gauche Boaz.
\Chap{4}
\TextTitle{L'autel d'airain, la mer de fonte et les ustensiles du temple\FTNTT{1 R. 7:23-50.}}
\VerseOne{}Il fit aussi un autel d'airain\FTNT{Voir l'annexe «~Le temple de Salomon - extérieur~».} long de vingt coudées, large de vingt coudées, et haut de dix coudées.
\VS{2}Il fit la mer de fonte de dix coudées d'un bord à l'autre, ronde tout autour, et haute de cinq coudées, et une circonférence que mesurait un cordon de trente coudées.
\VS{3}Des figures de bœufs l'entouraient en dessous, dix par coudée, faisant tout le tour de la mer~; il y avait deux rangées de bœufs fondus avec elle en une seule pièce.
\VS{4}Elle était posée sur douze bœufs, dont trois tournés vers le nord, trois tournés vers l'occident, trois tournés vers le sud, et trois tournés vers l'orient. La mer était sur eux, et toute la partie postérieure de leur corps était en dedans.
\VS{5}Son épaisseur était d'une paume~; et son bord était comme le bord d'une coupe en fleur de lis. Elle avait une contenance de trois mille baths\FTNT{Ex 25~; Ex 27.}.
\VS{6}Il fit aussi dix cuves, et en mit cinq à droite et cinq à gauche, pour servir à la purification. On y lavait ce qui appartenait aux holocaustes, et la mer servait aux prêtres pour s'y laver.
\VS{7}Il fit dix chandeliers d'or, d'après l'ordonnance, et les mit dans le temple, cinq à droite et cinq à gauche.
\VS{8}Il fit aussi dix tables, et il les mit dans le temple, cinq à droite et cinq à gauche. Il fit cent coupes d'or.
\VS{9}Il fit encore le parvis des prêtres, le grand parvis et des portes pour ce parvis, et couvrit d'airain ces portes.
\VS{10}Il mit la mer du côté droit, vers l'orient, face au sud-est.
\VS{11}Et Huram fit les cuves, les pelles et les bassins. Huram acheva de faire l'ouvrage qu'il faisait pour le roi Salomon dans la maison de Dieu~:
\VS{12}Deux colonnes, les bourrelets et les deux chapiteaux sur le sommet des colonnes~; les deux maillages pour couvrir les deux bourrelets des chapiteaux sur le sommet des colonnes~;
\VS{13}et les quatre cents grenades pour les deux maillages, deux rangs de grenades à chaque maille, pour couvrir les deux bourrelets des chapiteaux sur le sommet des colonnes.
\VS{14}Il fit aussi les bases, et il fit les cuves sur les bases~;
\VS{15}la mer et les douze bœufs sous elle~;
\VS{16}les pots, les pelles et les fourchettes et tous leurs ustensiles~; Huram-Abi les fit au roi Salomon, pour la maison de Yahweh, en airain poli.
\VS{17}Le roi les fit fondre dans la plaine du Jourdain, dans une terre grasse, entre Succoth et Tseréda.
\VS{18}Et Salomon fit tous ces ustensiles en si grand nombre qu'on ne rechercha point le poids de l'airain.
\VS{19}Salomon fit encore tous les ustensiles\FTNT{Voir l'annexe «~Le temple de Salomon - intérieur~».} qui étaient dans la maison de Yahweh~: L'autel d'or, et les tables sur lesquelles on mettait le pain de proposition~;
\VS{20}les chandeliers et leurs lampes d'or fin, qu'on devait allumer devant le sanctuaire, selon l'ordonnance~;
\VS{21}les fleurs, les lampes, et les mouchettes d'or, d'un or parfaitement pur~;
\VS{22}et les mouchettes, les bassins, les tasses et les encensoirs d'or fin. Quant à l'entrée de la maison, les portes intérieures conduisant dans le Saint des saints, et les portes de la maison pour entrer au temple étaient d'or.
\Chap{5}
\TextTitle{L'arche de l'alliance dans le sanctuaire, Yahweh manifeste sa gloire\FTNTT{1 R. 7:51-8:11.}}
\VerseOne{}Ainsi fut achevé tout l'ouvrage que Salomon fit pour la maison de Yahweh. Puis Salomon fit apporter ce que David, son père, avait consacré~: L'argent, l'or et tous les ustensiles~; et il les mit dans les trésors de la maison de Dieu.
\VS{2}Alors Salomon assembla à Jérusalem les anciens d'Israël, et tous les chefs des tribus, les chefs des pères des fils d'Israël, pour transporter de la ville de David, qui est Sion, l'arche de l'alliance de Yahweh.
\VS{3}Et tous les hommes d'Israël s'assemblèrent auprès du roi pour la fête~; c'était le septième mois.
\VS{4}Tous les anciens d'Israël vinrent, et les Lévites portèrent l'arche.
\VS{5}Ils transportèrent l'arche, la tente d'assignation, et tous les ustensiles sacrés qui étaient dans la tente~; les prêtres et les Lévites les emportèrent.
\VS{6}Or le roi Salomon, et toute l'assemblée d'Israël réunie auprès de lui étaient devant l'arche, sacrifiant du menu et du gros bétail en si grand nombre qu'on ne pouvait ni dénombrer ni compter.
\VS{7}Les prêtres portèrent l'arche de l'alliance de Yahweh à sa place, dans le sanctuaire de la maison, dans le Saint des saints, sous les ailes des chérubins.
\VS{8}Les chérubins étendaient les ailes sur l'endroit où devait être l'arche, et les chérubins couvraient l'arche et ses barres par-dessus.
\VS{9}Les barres avaient une longueur telle que leurs extrémités se voyaient en avant de l'arche, devant le sanctuaire~; mais elles ne se voyaient point du dehors. Et l'arche a été là jusqu'à ce jour.
\VS{10}Il n'y avait dans l'arche que les deux tables que Moïse y avait mises en Horeb, quand Yahweh traita alliance avec les enfants d'Israël à leur sortie d'Egypte.
\VS{11}Or il arriva que comme les prêtres sortaient du lieu saint (car tous les prêtres présents s'étaient sanctifiés, sans observer l'ordre des classes),
\VS{12}et que tous les Lévites qui étaient chantres, Asaph, Héman, Jeduthun, leurs fils et leurs frères, vêtus de fin lin, avec des cymbales, des luths et des harpes, se tenaient à l'orient de l'autel~; et il y avait avec eux cent vingt prêtres sonnant des trompettes~;
\VS{13}il arriva, dis-je, que comme un seul homme, ceux qui sonnaient des trompettes et ceux qui chantaient firent entendre leur voix d'un même accord, pour célébrer et pour louer Yahweh, et firent retentir le son des trompettes, des cymbales et d'autres instruments de musique, et ils célébrèrent Yahweh, en disant~: Car il est bon, car sa miséricorde demeure à toujours\FTNT{Jé. 33:11~; Ps. 118:29~; Ps. 136.}~! Il arriva que la maison de Yahweh fut remplie d'une nuée.
\VS{14}Les prêtres ne purent s'y tenir pour faire le service, à cause de la nuée~; car la gloire de Yahweh remplissait la maison de Dieu.
\Chap{6}
\TextTitle{Salomon s'adresse à l'assemblée d'Israël\FTNTT{1 R. 8:12-21.}}
\VerseOne{}Alors Salomon dit~: Yahweh a dit qu'il habiterait dans l'obscurité\FTNT{Nous avons ici une prophétie concernant la venue du Messie. Dieu, qui est lumière, a accepté d'habiter dans les ténèbres afin de nous sauver (Mt. 4:16~; Jn. 1:5).}.
\VS{2}Et moi, j'ai bâti une maison qui sera ta demeure, et un domicile afin que tu y résides à toujours~!
\VS{3}Puis le roi tourna son visage, et bénit toute l'assemblée d'Israël~; et toute l'assemblée d'Israël était debout.
\VS{4}Et il dit~: Béni soit Yahweh, le Dieu d'Israël, qui de sa bouche a parlé à David, mon père, et qui par sa main puissante accomplit ce qu'il avait déclaré en disant~:
\VS{5}Depuis le jour où j'ai fait sortir mon peuple du pays d'Egypte, je n'ai point choisi de ville entre toutes les tribus d'Israël pour y bâtir une maison afin que mon Nom y réside, et je n'ai point choisi d'homme pour être chef de mon peuple d'Israël.
\VS{6}Mais j'ai choisi Jérusalem pour que mon Nom y réside, et j'ai choisi David pour qu'il règne sur mon peuple d'Israël.
\VS{7}Or David, mon père, avait à cœur de bâtir une maison au Nom de Yahweh, le Dieu d'Israël.
\VS{8}Mais Yahweh parla à David, mon père~: Puisque tu as eu à cœur de bâtir une maison à mon Nom, tu as bien fait d'avoir eu cette intention.
\VS{9}Seulement, ce n'est pas toi qui bâtiras cette maison~; mais ce sera ton fils, qui sortira de tes entrailles, qui bâtira cette maison à mon Nom.
\VS{10}Yahweh a accompli la parole qu'il avait déclarée~; j'ai succédé à David, mon père, et je me suis assis sur le trône d'Israël, comme Yahweh l'avait dit, et j'ai bâti cette maison au Nom de Yahweh, le Dieu d'Israël.
\VS{11}J'y ai mis l'arche où est l'alliance de Yahweh, qu'il traita avec les enfants d'Israël.
\TextTitle{Prière de Salomon\FTNTT{1 R. 8:22-61.}}
\VS{12}Puis il se plaça devant l'autel de Yahweh, en face de toute l'assemblée d'Israël, et il étendit ses mains.
\VS{13}Car Salomon avait fait une tribune d'airain, et il l'avait mise au milieu du grand parvis~; elle était longue de cinq coudées, large de cinq coudées, et haute de trois coudées. Il s'y plaça, se mit à genoux en face de toute l'assemblée d'Israël, et étendant ses mains vers les cieux, il dit~:
\VS{14}Ô Yahweh, Dieu d'Israël~! Il n'y a ni dans les cieux ni sur la terre de Dieu semblable à toi, qui gardes l'alliance et la miséricorde envers tes serviteurs qui marchent de tout leur cœur devant ta face.
\VS{15}Toi qui as tenu parole à ton serviteur David, mon père. Ce que tu lui avais promis, et ce que tu as déclaré de ta bouche, tu l'as accompli de ta main puissante, comme il paraît aujourd'hui.
\VS{16}Maintenant, ô Yahweh, Dieu d'Israël~! tiens la parole que tu as faite à ton serviteur David, mon père, en disant~: Tu ne manqueras jamais devant moi d'un successeur assis sur le trône d'Israël, pourvu que tes fils prennent garde à leur voie pour marcher dans ma loi, comme tu as marché devant ma face.
\VS{17}Et maintenant, ô Yahweh, Dieu d'Israël~! que ta parole, que tu as déclarée à David, ton serviteur, soit confirmée~!
\VS{18}Mais Dieu habiterait-il véritablement sur la terre avec les hommes~? Voici, les cieux, même les cieux des cieux, ne peuvent te contenir, combien moins cette maison que j'ai bâtie~!
\VS{19}Toutefois, ô Yahweh, mon Dieu, aie égard à la prière de ton serviteur et à sa supplication, pour écouter le cri et la prière que ton serviteur t'adresse.
\VS{20}Que tes yeux soient ouverts jour et nuit sur cette maison, sur le lieu où tu as promis de mettre ton Nom~! Ecoute la prière que ton serviteur te fait en ce lieu.
\VS{21}Exauce les supplications de ton serviteur et de ton peuple d'Israël, quand ils prieront en ce lieu. Exauce des cieux, du lieu de ta demeure~; exauce et pardonne~!
\VS{22}Si quelqu'un pèche contre son prochain, et qu'on lui impose un serment pour le faire jurer, et qu'il vient prêter serment devant ton autel, dans cette maison~;
\VS{23}écoute-le des cieux, agis et juge tes serviteurs, en donnant au méchant son salaire, et fais retomber sa conduite sur sa tête, en justifiant le juste, et lui rendant selon sa justice.
\VS{24}Quand ton peuple d'Israël sera battu par l'ennemi, pour avoir péché contre toi~; s'ils retournent à toi, s'ils donnent gloire à ton Nom, s'ils t'adressent dans cette maison des prières et des supplications~;
\VS{25}toi, exauce-les des cieux, et pardonne le péché de ton peuple d'Israël, et ramène-les dans la terre que tu leur as donnée à eux et à leurs pères.
\VS{26}Quand les cieux seront fermés, et qu'il n'y aura point de pluie, parce qu'ils auront péché contre toi~; s'ils prient en ce lieu, s'ils donnent gloire à ton Nom, et s'ils se détournent de leurs péchés, parce que tu les auras affligés~;
\VS{27}toi, exauce-les des cieux, et pardonne le péché de tes serviteurs et de ton peuple d'Israël, après que tu leur auras enseigné le bon chemin, par lequel ils doivent marcher~; et envoie de la pluie sur la terre que tu as donnée en héritage à ton peuple.
\VS{28}Quand il y aura dans le pays la famine ou la peste, quand il y aura la rouille, la nielle, les sauterelles d'une espèce ou d'une autre, quand les ennemis les assiégeront dans leur pays, dans leurs portes, ou qu'il y aura un fléau, une maladie quelconque~;
\VS{29}si un homme, si tout ton peuple d'Israël fait entendre des prières et des supplications, et que chacun reconnaît sa plaie et sa douleur, et étend ses mains vers cette maison~;
\VS{30}exauce-le des cieux, du lieu de ta demeure, et pardonne. Rends à chacun selon toutes ses voies, toi qui connais leur cœur~; car seul tu connais le cœur des fils des hommes~;
\VS{31}afin qu'ils te craignent, pour marcher dans tes voies, tout le temps qu'ils vivront sur la terre que tu as donnée à nos pères.
\VS{32}Et l'étranger, qui ne sera pas de ton peuple d'Israël, mais qui viendra d'un pays éloigné, à cause de ton grand Nom, de ta main puissante, et de ton bras étendu~; quand il viendra prier dans cette maison,
\VS{33}exauce-le des cieux, du lieu de ta demeure, et accorde tout ce que cet étranger réclamera de toi~; afin que tous les peuples de la terre connaissent ton Nom pour te craindre comme ton peuple d'Israël, et sachent que ton Nom est invoqué sur cette maison que j'ai bâtie.
\VS{34}Quand ton peuple sortira en guerre contre ses ennemis, par la voie par laquelle tu l'auras envoyé~; s'ils te prient, en regardant vers cette ville que tu as choisie, et vers cette maison que j'ai bâtie à ton Nom,
\VS{35}exauce des cieux leur prière et leur supplication, et fais-leur droit.
\VS{36}Quand ils pécheront contre toi, car il n'y a point d'homme qui ne pèche, et qu'irrité contre eux, tu les auras livrés à leurs ennemis, et que ceux qui les auront pris les auront emmenés captifs en quelque pays, soit éloigné soit proche~;
\VS{37}si dans le pays où ils seront captifs, ils rentrent en eux-mêmes et s'ils se repentent, s'ils t'adressent des supplications dans le pays de leur captivité, en disant~: Nous avons péché, nous avons commis l'iniquité, nous avons agi méchamment~!
\VS{38}S'ils retournent à toi de tout leur cœur et de toute leur âme, dans le pays de leur captivité où ils ont été emmenés captifs, et s'ils t'adressent des prières, les regards tournés vers leur pays que tu as donné à leurs pères, vers cette ville que tu as choisie, et vers cette maison que j'ai bâtie à ton Nom~;
\VS{39}exauce des cieux, du lieu de ta demeure, leurs prières et leurs supplications, et fais-leur droit~; pardonne à ton peuple qui aura péché contre toi~!
\VS{40}Maintenant, ô mon Dieu, que tes yeux soient ouverts et que tes oreilles soient attentives à la prière qu'on te fera en ce lieu~!
\VS{41}Et maintenant, Yahweh Dieu~! Lève-toi, viens au lieu de ton repos, toi et l'arche de ta puissance. Yahweh Dieu, que tes prêtres soient revêtus du salut, et que tes bien-aimés se réjouissent du bien que tu leur fais~!
\VS{42}Yahweh Dieu, ne repousse pas la face de ton oint~; souviens-toi des grâces accordées à David, ton serviteur.
\Chap{7}
\TextTitle{Yahweh répond par le feu~: Sa gloire remplit la maison}
\VerseOne{}Lorsque Salomon eut achevé de prier, le feu descendit du ciel et consuma l'holocauste et les sacrifices\FTNT{Lé. 9:24~; 1 R. 18:38.}~; et la gloire de Yahweh remplit la maison.
\VS{2}Les prêtres ne pouvaient entrer dans la maison de Yahweh, parce que la gloire de Yahweh avait rempli la maison de Yahweh.
\VS{3}Tous les enfants d'Israël virent descendre le feu et la gloire de Yahweh sur la maison~; et ils se courbèrent, le visage contre terre, sur le pavé, se prosternèrent et louèrent Yahweh, en disant~: Car il est bon, car sa miséricorde demeure éternellement~!
\TextTitle{Salomon et le peuple offrent des sacrifices à Yahweh\FTNTT{1 R. 8:62-66.}}
\VS{4}Or le roi et tout le peuple offraient des sacrifices devant Yahweh.
\VS{5}Le roi Salomon offrit un sacrifice de vingt-deux mille bœufs, et cent vingt mille brebis. Ainsi, le roi et tout le peuple firent la dédicace de la maison de Dieu.
\VS{6}Les prêtres se tenaient à leurs fonctions, ainsi que les Lévites, avec les instruments de musique de Yahweh, que le roi David avait faits pour louer Yahweh en disant~: Car sa miséricorde demeure éternellement~; ayant les Psaumes de David entre leurs mains. Et les prêtres sonnaient des trompettes vis-à-vis d'eux, et tout Israël se tenait debout.
\VS{7}Salomon consacra le milieu du parvis, qui est devant la maison de Yahweh~; car il offrit là les holocaustes et les graisses des sacrifices d'offrande de paix\FTNT{Voir commentaire en Lé. 3:1.}, parce que l'autel d'airain que Salomon avait fait ne pouvait contenir les holocaustes, les offrandes et les graisses.
\VS{8}Ainsi Salomon célébra, en ce temps-là, la fête pendant sept jours, avec tout Israël. Il y avait une grande multitude, venue depuis l'entrée d'Hamath jusqu'au torrent d'Egypte.
\VS{9}Le huitième jour, ils firent une assemblée solennelle~; car ils firent la dédicace de l'autel pendant sept jours, et la fête pendant sept jours.
\VS{10}Le vingt-troisième jour du septième mois, il laissa aller le peuple dans ses tentes, se réjouissant et ayant le cœur plein de joie, à cause du bien que Yahweh avait fait à David, à Salomon, et à Israël, son peuple.
\TextTitle{Yahweh apparaît à Salomon\FTNTT{1 R. 9:1-9.}}
\VS{11}Salomon acheva donc la maison de Yahweh et la maison du roi~; et Salomon réussit dans tout ce qui lui vint à cœur de faire dans la maison de Yahweh et dans sa maison.
\VS{12}Yahweh apparut à Salomon pendant la nuit, et lui dit~: J'exauce ta prière, et je choisis ce lieu comme une maison de sacrifices.
\VS{13}Quand je fermerai les cieux, et qu'il n'y aura point de pluie, et quand j'ordonnerai aux sauterelles de consumer le pays, et quand j'enverrai la peste parmi mon peuple~;
\VS{14}si mon peuple, sur lequel mon Nom est invoqué, s'humilie, prie, et cherche ma face, et s'il se détourne de ses mauvaises voies, alors je l'exaucerai des cieux, je pardonnerai ses péchés, et je guérirai son pays.
\VS{15}Mes yeux seront désormais ouverts, et mes oreilles seront attentives à la prière faite en ce lieu.
\VS{16}Maintenant je choisis et je sanctifie cette maison, afin que mon Nom y soit à toujours~; mes yeux et mon cœur seront toujours là.
\VS{17}Et toi, si tu marches devant moi comme David, ton père, a marché, faisant tout ce que je t'ai ordonné, et si tu gardes mes lois et mes ordonnances,
\VS{18}j'affermirai le trône de ton royaume, comme je l'ai déclaré à David, ton père, en disant~: Il ne te manquera point de successeur qui règne en Israël.
\VS{19}Mais si vous vous détournez, et si vous abandonnez mes lois et mes commandements que je vous ai prescrits, et si vous allez servir d'autres dieux et vous prosterner devant eux,
\VS{20}je vous arracherai de mon pays que je vous ai donné, je rejetterai loin de moi cette maison que j'ai consacrée à mon Nom, et j'en ferai un sujet de sarcasmes et de moqueries parmi tous les peuples.
\VS{21}Et quiconque passera près de cette maison qui aura été élevée, sera dans l'étonnement et dira~: Pourquoi Yahweh a-t-il ainsi traité ce pays et cette maison~?
\VS{22}Et on répondra~: Parce qu'ils ont abandonné Yahweh, le Dieu de leurs pères, qui les a fait sortir du pays d'Egypte, et qu'ils se sont attachés à d'autres dieux, et qu'ils se sont prosternés devant eux, et les ont servis~; à cause de cela, il a fait venir sur eux tous ces maux.
\Chap{8}
\TextTitle{Les réalisations de Salomon\FTNTT{1 R. 9:15-28~; 10:26-29.}}
\VerseOne{}Au bout de vingt ans, pendant lesquels Salomon bâtit la maison de Yahweh et sa propre maison,
\VS{2}il bâtit les villes que Huram lui avait données et y fit habiter les enfants d'Israël.
\VS{3}Puis Salomon marcha contre Hamath de Tsoba, et la conquit.
\VS{4}Il bâtit Thadmor au désert, et toutes les villes servant de magasins qu'il bâtit dans le pays de Hamath.
\VS{5}Il bâtit Beth-Horon la haute, et Beth-Horon la basse, villes fortes de murailles, de portes et de barres~;
\VS{6}Baalath, et toutes les villes servant de magasins qu'avait Salomon, toutes les villes pour les chars, les villes pour la cavalerie, et tout ce que Salomon prit plaisir à bâtir à Jérusalem, au Liban, et dans tout le pays de sa domination.
\VS{7}Tout le peuple qui était resté des Héthiens, des Amoréens, des Phéréziens, des Héviens et des Jébusiens, qui n'étaient point d'Israël~;
\VS{8}leurs descendants, qui étaient restés après eux dans le pays, et que les enfants d'Israël n'avaient pas détruits, Salomon les leva comme des gens de corvée jusqu'à ce jour.
\VS{9}Salomon n'employa comme esclave pour ses travaux aucun des fils d'Israël~; car ils étaient des hommes de guerre, les chefs de ses officiers, les chefs de ses chars et de ses hommes d'armes.
\VS{10}Voici le nombre des chefs de ceux qui étaient préposés aux travaux du roi Salomon~: Ils étaient deux cent cinquante, ayant autorité sur le peuple.
\VS{11}Salomon fit monter la fille de Pharaon de la cité de David dans la maison qu'il lui avait bâtie~; car il dit~: Ma femme n'habitera point dans la maison de David, roi d'Israël, parce que les lieux où l'arche de Yahweh est entrée sont saints.
\VS{12}Alors Salomon offrit des holocaustes à Yahweh, sur l'autel de Yahweh qu'il avait bâti devant le portique.
\VS{13}Il offrait chaque jour ce qui était prescrit par Moïse pour les sabbats, pour les nouvelles lunes, et pour les fêtes, trois fois par année, à la fête des pains sans levain, à la fête des semaines, et à la fête des tabernacles\FTNT{Ex. 14:17~; Lé. 23:1-44.}.
\VS{14}Il établit, selon l'ordonnance de David, son père, les classes des prêtres selon leur fonction, et les Lévites selon leurs charges, pour célébrer Yahweh et pour faire, jour par jour, le service en présence des prêtres~; et les portiers, selon leurs classes, à chaque porte~; car tel était le commandement de David, homme de Dieu.
\VS{15}Et on ne s'écarta pas du commandement du roi à l'égard des prêtres et des Lévites, en aucune chose, ni à l'égard les trésors.
\VS{16}Ainsi fut préparé tout l'ouvrage de Salomon, jusqu'au jour de la fondation de la maison de Yahweh et jusqu'à ce qu'elle fut terminée. La maison de Yahweh fut donc achevée.
\VS{17}Alors Salomon alla à Etsjon-Guéber et à Eloth, sur le rivage de la mer, dans le pays d'Edom.
\VS{18}Et Huram lui envoya, sous la conduite de ses serviteurs, des navires et des serviteurs connaissant la mer. Ils allèrent avec les serviteurs de Salomon à Ophir, et ils y prirent quatre cent cinquante talents d'or, qu'ils apportèrent au roi Salomon.
\Chap{9}
\TextTitle{La reine de Séba chez Salomon\FTNTT{1 R. 10:1-13.}}
\VerseOne{}Or la reine de Séba, ayant appris la renommée de Salomon, vint à Jérusalem pour éprouver Salomon par des énigmes. Elle avait une suite très nombreuse, et des chameaux portant des aromates, de l'or en grande quantité et des pierres précieuses. Elle vint auprès de Salomon, et elle lui parla de tout ce qu'elle avait dans le cœur.
\VS{2}Salomon lui expliqua tout ce qu'elle lui proposa~; il n'y eut rien que Salomon n'entendît et qu'il ne sût lui expliquer.
\VS{3}Alors, la reine de Séba vit toute la sagesse de Salomon, et la maison qu'il avait bâtie,
\VS{4}les mets de sa table, la demeure de ses serviteurs, l'ordre de service et les vêtements de ceux qui le servaient, ses échansons et leurs vêtements, et les marches par où l'on montait à la maison de Yahweh, et elle fut toute ravie hors d'elle-même.
\VS{5}Elle parla ainsi au roi~: Ce que j'ai entendu dire dans mon pays de tes actions et de ta sagesse était donc vrai~!
\VS{6}Je ne croyais pas ce qu'on en disait avant d'être venue et que mes yeux ne l'aient vu~; et voici, on ne m'avait pas rapporté la moitié de la grandeur de ta sagesse~; tu surpasses la rumeur que j'avais entendue.
\VS{7}Heureux tes gens~! Heureux tes serviteurs qui se tiennent continuellement devant toi, et qui entendent ta sagesse~!
\VS{8}Béni soit Yahweh, ton Dieu, qui a pris plaisir en toi pour te placer sur son trône comme roi pour Yahweh, ton Dieu~! C'est parce que ton Dieu aime Israël et veut le faire subsister à jamais, qu'il t'a établi roi sur eux pour faire droit et justice.
\VS{9}Puis elle donna au roi cent vingt talents d'or, une très grande quantité d'aromates, et des pierres précieuses~; et il n'y eut plus d'aromates tels que ceux que la reine de Séba donna au roi Salomon.
\VS{10}Les serviteurs de Huram et les serviteurs de Salomon, qui amenèrent de l'or d'Ophir, amenèrent aussi du bois de santal et des pierres précieuses.
\VS{11}Le roi fit de ce bois de santal les chemins qui allaient à la maison de Yahweh et à la maison du roi, et des harpes et des luths pour les chantres. On n'en avait point vu auparavant de semblable dans le pays de Juda.
\VS{12}Le roi Salomon donna à la reine de Séba tout ce qu'elle désira, ce qu'elle demanda, plus qu'elle n'avait apporté au roi~; et elle s'en retourna, revint dans son pays, elle et ses serviteurs.
\TextTitle{Les richesses de Salomon\FTNTT{cp. 1 R. 4:1-34.}}
\VS{13}Le poids de l'or qui arrivait à Salomon chaque année était de six cent soixante-six talents d'or,
\VS{14}outre ce qu'il retirait des négociants et des marchands qui en apportaient, et de tous les rois d'Arabie et des gouverneurs de ces pays-là, qui apportaient de l'or et de l'argent à Salomon.
\VS{15}Le roi Salomon fit deux cents grands boucliers d'or battu, employant six cents sicles d'or battu pour chaque bouclier~;
\VS{16}et trois cents autres boucliers plus petits d'or battu, employant trois cents sicles d'or pour chaque bouclier~; et le roi les mit dans la maison de la forêt du Liban.
\VS{17}Le roi fit aussi un grand trône d'ivoire, qu'il couvrit d'or pur.
\VS{18}Ce trône avait six marches et un marchepied d'or qui était accolé au trône~; et il avait des accoudoirs de l'un et de l'autre côté du siège~; et deux lions se tenaient auprès des accoudoirs.
\VS{19}Douze lions se tenaient là sur les six marches de part et d'autre. Rien de pareil n'avait été fait pour aucun royaume.
\VS{20}Et toutes les coupes à boire du roi Salomon étaient d'or, et toute la vaisselle de la maison de la forêt du Liban était d'or pur~; rien n'était d'argent~; on n'en faisait aucun cas du temps de Salomon.
\VS{21}Car les navires du roi allaient à Tarsis avec les serviteurs de Huram~; et une fois tous les trois ans arrivaient les navires de Tarsis, apportant de l'or, de l'argent, des dents d'éléphants, des singes et des paons.
\VS{22}Le roi Salomon fut plus grand que tous les rois de la terre, tant en richesses qu'en sagesse.
\VS{23}Tous les rois de la terre cherchaient à voir la face de Salomon, pour écouter la sagesse que Dieu avait mise dans son cœur.
\VS{24}Et chacun d'eux apportait son présent~: Des ustensiles d'argent, des ustensiles d'or, des vêtements, des armes, des aromates, des chevaux et des mulets, et il en était ainsi année après année.
\VS{25}Salomon avait quatre mille écuries pour ses chevaux, avec des chars~; et douze mille cavaliers qu'il plaça dans les villes où il avait des chars et auprès du roi à Jérusalem.
\VS{26}Il dominait sur tous les rois depuis le fleuve jusqu'au pays des Philistins, et jusqu'à la frontière d'Egypte.
\VS{27}Et le roi fit que l'argent était aussi commun à Jérusalem que les pierres, et les cèdres aussi nombreux que les sycomores qui sont dans les plaines.
\VS{28}On tirait des chevaux pour Salomon de l'Egypte et de tous les pays.
\TextTitle{Mort de Salomon\FTNTT{1 R. 11:1-40.}}
\VS{29}Le reste des actions de Salomon, les premières et les dernières, cela n'est-il pas écrit dans le livre de Nathan le prophète, dans la prophétie d'Achija de Silo, et dans la vision de Jéedo le voyant, concernant Jéroboam, fils de Nebath~?
\VS{30}Salomon régna quarante ans à Jérusalem sur tout Israël.
\VS{31}Puis Salomon s'endormit avec ses pères, et on l'ensevelit dans la cité de David, son père~; et Roboam, son fils, régna à sa place.
\Chap{10}
\TextTitle{Roboam règne sur Israël\FTNTT{1 R. 12:1-15.}}
\VerseOne{}Roboam se rendit à Sichem, car tout Israël était venu à Sichem pour l'établir roi.
\VS{2}Et il arriva que Jéroboam, fils de Nebath, qui était en Egypte, où il s'était enfui de devant le roi Salomon, l'eut appris, il revint d'Egypte.
\VS{3}Car on envoya l'appeler. Ainsi Jéroboam et tout Israël vinrent et parlèrent à Roboam, en disant~:
\VS{4}Ton père a mis sur nous un joug pesant. Allège maintenant cette rude servitude de ton père, et ce joug pesant qu'il a mis sur nous, et nous te servirons.
\VS{5}Alors il leur dit~: Revenez vers moi dans trois jours. Et le peuple s'en alla.
\VS{6}Le roi Roboam demanda conseil aux vieillards qui avaient été auprès de Salomon, son père, pendant sa vie, et il leur parla ainsi~: Comment, et quelle chose me conseillez-vous de répondre à ce peuple~?
\VS{7} Et ils lui répondirent en ces termes~: Si tu es bon envers ce peuple, si tu es bienveillant envers eux, et que tu leur dises de bonnes paroles, ils seront tes serviteurs à toujours.
\VS{8}Mais il laissa le conseil que les vieillards lui avaient donné, et il demanda conseil aux jeunes gens qui avaient grandi avec lui, et qui se tenaient auprès de lui.
\VS{9}Et il leur dit~: Que me conseillez-vous de répondre à ce peuple qui m'a parlé en disant~: Allège le joug que ton père a mis sur nous~?
\VS{10}Et les jeunes gens qui avaient grandi avec lui, lui parlèrent en disant~: Tu répondras en disant à ce peuple qui t'a parlé et t'a dit~: Ton père a mis sur nous un joug pesant, mais toi, allège-le~; tu leur répondras donc~: Mon petit doigt est plus gros que les reins de mon père.
\VS{11}Or mon père a mis sur vous un joug pesant, mais moi, je rendrai votre joug encore plus pesant. Mon père vous a châtiés avec des fouets, mais moi, je vous châtierai avec des scorpions.
\TextTitle{Roboam délaisse le conseil des anciens}
\VS{12}Trois jours après, Jéroboam, avec tout le peuple, vint vers Roboam, suivant ce qu'avait dit le roi~: Revenez vers moi dans trois jours.
\VS{13}Mais le roi leur répondit durement. Le roi Roboam délaissa le conseil des anciens,
\VS{14}et leur parla suivant le conseil des jeunes gens, en disant~: Mon père a mis sur vous un joug pesant~; mais moi, j'y ajouterai encore. Mon père vous a châtiés avec des fouets~; mais moi, je vous châtierai avec des scorpions.
\VS{15}Le roi n'écouta donc point le peuple~; cela était conduit par Dieu, afin que Yahweh accomplisse la parole qu'il avait déclarée par Achija de Silo, à Jéroboam, fils de Nebath.
\TextTitle{Israël se détache de la maison de David\FTNTT{1 R. 12:16-19.}}
\VS{16}Quand tout Israël vit que le roi ne les écoutait pas, le peuple répondit au roi, en disant~: Quelle part avons-nous avec David~? Nous n'avons point d'héritage avec le fils d'Isaï. Israël, chacun à ses tentes~! Et toi David, pourvois maintenant à ta maison. Ainsi, tout Israël s'en alla dans ses tentes.
\VS{17}Mais quant aux enfants d'Israël qui habitaient les villes de Juda, Roboam régna sur eux.
\VS{18}Alors le roi Roboam envoya Hadoram, qui était préposé aux impôts~; mais les enfants d'Israël le lapidèrent à coups de pierres et il mourut. Et le roi Roboam se hâta de monter sur un char pour s'enfuir à Jérusalem.
\VS{19}C'est ainsi qu'Israël s'est rebellé contre la maison de David, jusqu'à ce jour.
\Chap{11}
\TextTitle{Yahweh interdit la guerre entre Juda et Israël\FTNTT{1 R. 12:21-24.}}
\VerseOne{}Roboam, étant arrivé à Jérusalem, assembla la maison de Juda et de Benjamin, cent quatre-vingt mille hommes d'élite et de guerre, afin de combattre contre Israël, pour le ramener sous le règne de Roboam.
\VS{2}Mais la parole de Yahweh vint à Schemaeja, homme de Dieu, en ces termes~:
\VS{3}Parle à Roboam, fils de Salomon, roi de Juda, et à ceux d'Israël qui sont en Juda et en Benjamin, et dis-leur~:
\VS{4}Ainsi parle Yahweh~: Ne montez point, et ne combattez point contre vos frères. Retournez chacun dans sa maison~; c'est par moi que cette chose est arrivée. Et ils obéirent aux paroles de Yahweh, et ils s'en retournèrent sans aller contre Jéroboam\FTNT{1 R. 12:21-24.}.
\VS{5}Roboam demeura donc à Jérusalem, et il bâtit des villes fortes en Juda.
\VS{6}Il bâtit Bethléhem, Etham, Tekoa,
\VS{7}Beth-Tsur, Soco, Adullam,
\VS{8}Gath, Maréscha, Ziph,
\VS{9}Adoraïm, Lakis, Azéka,
\VS{10}Tsorea, Ajalon et Hébron, qui étaient en Juda et en Benjamin, et en fit des villes fortes.
\VS{11}Il les fortifia et y mit des gouverneurs, des provisions de vivres, d'huile et de vin.
\VS{12}Dans chacune de ces villes, il mit des boucliers et des lances, et il les rendit puissantes. Ainsi Juda et Benjamin lui furent soumis.
\TextTitle{Les prêtres et les Lévites soutiennent Roboam}
\VS{13}Les prêtres et les Lévites, qui étaient dans tout Israël, vinrent de toutes leurs contrées se joindre à lui.
\TextTitle{Jéroboam abandonne Yahweh\FTNTT{1 R. 12:26-30~; 14:7-8.}}
\VS{14}Car les Lévites abandonnèrent leurs faubourgs et leurs possessions et vinrent en Juda et à Jérusalem, parce que Jéroboam et ses fils les avaient rejetés des fonctions de prêtres pour Yahweh.
\VS{15}Car il s'était établi des prêtres pour les hauts lieux, pour les boucs, et pour les veaux qu'il avait faits.
\VS{16}Et à leur suite, ceux d'entre toutes les tribus d'Israël qui avaient appliqué leur cœur à chercher Yahweh, le Dieu d'Israël, vinrent à Jérusalem pour sacrifier à Yahweh, le Dieu de leurs pères.
\VS{17}Ils fortifièrent le royaume de Juda et affermirent Roboam, fils de Salomon, pendant trois ans~; car on suivit les voies de David et de Salomon pendant trois ans.
\TextTitle{Les femmes et les enfants de Roboam}
\VS{18}Or Roboam prit pour femme~: Mahalath, fille de Jerimoth, fils de David et d'Abichaïl, fille d'Eliab, fils d'Isaï.
\VS{19}Elle lui enfanta des fils~: Jeusch, Schemaria et Zaham.
\VS{20}Après elle, il prit Maaca, fille d'Absalom, qui lui enfanta Abija, Attaï, Ziza et Schelomith.
\VS{21}Roboam aima Maaca, fille d'Absalom, plus que toutes ses femmes et ses concubines. Car il prit dix-huit femmes et soixante concubines, et il engendra vingt-huit fils et soixante filles.
\VS{22}Roboam établit pour chef Abija, fils de Maaca, comme prince entre ses frères~; car il voulait le faire roi.
\VS{23}Il agit prudemment et dispersa tous ses fils dans toutes les contrées de Juda et de Benjamin, dans toutes les villes fortes~; il leur donna de quoi vivre en abondance, et demanda pour eux une multitude de femmes.
\Chap{12}
\TextTitle{Roboam affermi, il abandonne Yahweh\FTNTT{1 R. 14:21-24.}}
\VerseOne{}Et il arriva lorsque la royauté de Roboam fut affermie et qu'il eut acquis de la force, il abandonna la loi de Yahweh, et tout Israël avec lui.
\TextTitle{Yahweh veut livrer Juda à Schischak\FTNTT{1 R. 14:25-28.}}
\VS{2}C'est pourquoi il arriva que la cinquième année du Roi Roboam, Schischak, roi d'Egypte, monta contre Jérusalem, parce qu'ils avaient péché contre Yahweh.
\VS{3}Il avait mille deux cents chars et soixante mille cavaliers, et le peuple qui vint avec lui d'Egypte, des Libyens, des Sukkiens et des Ethiopiens, était innombrable.
\VS{4}Il prit les villes fortes qui appartenaient à Juda, et vint jusqu'à Jérusalem.
\VS{5}Alors Schemaeja, le prophète, vint vers Roboam et les chefs de Juda, qui s'étaient assemblés à Jérusalem à cause de Schischak, et leur dit~: Ainsi parle Yahweh~: Vous m'avez abandonné~; moi aussi je vous abandonne aux mains de Schischak.
\VS{6}Alors les chefs d'Israël et le roi s'humilièrent, et dirent~: Yahweh est juste~!
\VS{7}Et quand Yahweh vit qu'ils s'humiliaient, la parole de Yahweh vint à Schemaeja, et il lui dit~: Ils se sont humiliés~; je ne les détruirai pas, mais je leur donnerai dans peu de temps un moyen d'échapper, et ma fureur ne se répandra point sur Jérusalem par la main de Schischak.
\VS{8}Toutefois, ils lui seront asservis, afin qu'ils sachent ce que c'est que de me servir ou de servir les royaumes de la terre.
\VS{9}Schischak, roi d'Egypte, monta donc contre Jérusalem, et prit les trésors de la maison de Yahweh et les trésors de la maison du roi~; il prit tout. Il prit les boucliers d'or que Salomon avait faits.
\VS{10}Le roi Roboam fit des boucliers d'airain à leur place, et il les mit entre les mains des chefs des coureurs qui gardaient la porte de la maison du roi.
\VS{11}Et toutes les fois que le roi entrait dans la maison de Yahweh, les coureurs venaient et les portaient~; puis ils les rapportaient dans la chambre des coureurs.
\VS{12}Ainsi comme il s'était humilié, la colère de Yahweh se détourna de lui, et ne le détruisit pas entièrement~; car il y avait encore de bonnes choses en Juda.
\TextTitle{Mort de Roboam\FTNTT{1 R. 14:21,29,31.}}
\VS{13}Le roi Roboam se fortifia donc dans Jérusalem, et régna. Il avait quarante et un ans quand il devint roi, et il régna dix-sept ans à Jérusalem, la ville que Yahweh avait choisie de toutes les tribus d'Israël, pour y mettre son Nom. Sa mère s'appelait Naama, l'Ammonite.
\VS{14}Il fit le mal, car il ne disposa point son cœur pour chercher Yahweh.
\VS{15}Or les actions de Roboam, les premières et les dernières, ne sont-elles pas écrites dans les livres de Schemaeja le prophète, et d'Iddo le voyant, parmi les registres généalogiques~? Les guerres entre Roboam et Jéroboam furent continuelles.
\VS{16}Roboam s'endormit avec ses pères, et il fut enseveli dans la cité de David~; et Abija, son fils, régna à sa place.
\Chap{13}
\TextTitle{Abija règne sur Juda~; guerre entre Israël et Juda\FTNTT{1 R. 15:1-8.}}
\VerseOne{}La dix-huitième année du roi Jéroboam, Abija commença à régner sur Juda.
\VS{2}Il régna trois ans à Jérusalem. Sa mère s'appelait Micaja, fille d'Uriel, de Guibea. Or il y eut guerre entre Abija et Jéroboam.
\VS{3}Abija engagea la guerre avec une armée de vaillants guerriers, quatre cent mille hommes d'élite~; et Jéroboam se rangea en bataille contre lui avec huit cent mille hommes d'élite, forts et vaillants.
\VS{4}Et Abija se leva du haut de la montagne de Tsemaraïm, parmi les montagnes d'Ephraïm, et dit~: Jéroboam et tout Israël, écoutez-moi~!
\VS{5}Ne savez-vous pas que Yahweh, le Dieu d'Israël, a donné pour toujours la royauté sur Israël à David, à lui et à ses fils, par une alliance de sel\FTNT{Sel~: Voir commentaire en Lé. 2:13.}~!
\VS{6}Mais Jéroboam, fils de Nebath, serviteur de Salomon, fils de David, s'est élevé et s'est rebellé contre son seigneur.
\VS{7}Et des gens sans valeur, des fils de Belial, se sont assemblés avec lui et se sont fortifiés contre Roboam, fils de Salomon. Or Roboam était un jeune homme craintif et sans force devant eux.
\VS{8}Et maintenant, vous vous dites être forts devant la royauté de Yahweh, qui est aux mains des fils de David~; vous êtes une multitude, et vous avez avec vous les veaux d'or que Jéroboam vous a faits pour dieux.
\VS{9}N'avez-vous pas rejeté les prêtres de Yahweh, les fils d'Aaron, et les Lévites~? Et ne vous êtes-vous pas faits des prêtres comme les peuples des autres pays~? Quiconque venait, avec un jeune taureau et sept béliers, pour être consacré, devenait prêtre de ce qui n'est pas Dieu.
\VS{10}Mais quant à nous, Yahweh est notre Dieu, et nous ne l'avons pas abandonné~; les prêtres qui font le service de Yahweh sont fils d'Aaron, et ce sont les Lévites qui tiennent cette fonction.
\VS{11}Nous faisons brûler pour Yahweh, chaque matin et chaque soir, les holocaustes et le parfum d'aromates. Les pains de proposition sont rangés sur la table pure, et on allume le chandelier d'or avec ses lampes, chaque soir. Car nous gardons ce que Yahweh, notre Dieu, veut qu'on garde~; mais vous, vous l'avez abandonné.
\VS{12}Et voici, nous avons avec nous, à notre tête, Dieu et ses prêtres, et les trompettes retentissantes, pour sonner avec éclat contre vous. Fils d'Israël, ne combattez pas contre Yahweh, le Dieu de vos pères~; car cela ne vous réussira pas.
\VS{13}Mais Jéroboam fit une embuscade par un détour, et arriva derrière eux~; de sorte que les Israélites étaient en face de Juda, qui avait l'embuscade par-derrière.
\VS{14}Ceux de Juda se retournèrent et voici ils avaient la bataille par-devant et par-derrière. Alors ils crièrent à Yahweh, et les prêtres sonnèrent des trompettes.
\TextTitle{Victoire de Juda sur Israël}
\VS{15}Les hommes de Juda poussèrent un cri, et au cri de guerre des hommes de Juda, Yahweh frappa Jéroboam et tout Israël devant Abija et Juda.
\VS{16}Les fils d'Israël s'enfuirent devant ceux de Juda, parce que Dieu les livra entre leurs mains.
\VS{17}Abija et son peuple leur firent un grand carnage, et il tomba d'Israël cinq cent mille hommes d'élite blessés à mort.
\VS{18}Ainsi, les enfants d'Israël furent humiliés en ce temps-là~; et les enfants de Juda devinrent plus forts, parce qu'ils s'étaient appuyés sur Yahweh, le Dieu de leurs pères.
\VS{19} Abija poursuivit Jéroboam, et lui prit ces villes~: Béthel et les villes de son ressort, Jeschana et les villes de son ressort, Ephron et les villes de son ressort.
\TextTitle{Mort de Jéroboam\FTNTT{1 R. 14:19,20.}}
\VS{20}Et Jéroboam n'eut plus de force durant le temps d'Abija~; et Yahweh le frappa, et il mourut.
\TextTitle{Les femmes et les fils d'Abija\FTNTT{1 R. 15:7-8.}}
\VS{21}Mais Abija se fortifia~; il prit quatorze femmes, et engendra vingt-deux fils et seize filles.
\VS{22}Le reste des actions d'Abija, sa conduite et ses paroles sont écrites dans les mémoires du prophète Iddo.
\VS{23}Abija s'endormit avec ses pères, et on l'ensevelit dans la cité de David~; et Asa, son fils, régna à sa place. De son temps, le pays fut en repos pendant dix ans.
\Chap{14}
\TextTitle{Asa règne sur Juda, il rétablit l'ordre de Yahweh\FTNTT{1 R. 15:11.}}
\VerseOne{}Asa fit ce qui est bon et droit aux yeux de Yahweh, son Dieu.
\VS{2}Il ôta les autels étrangers et les hauts lieux~; il brisa les statues et mit en pièces les Asherah.
\VS{3}Et il recommanda à Juda de rechercher Yahweh, le Dieu de leurs pères, et de pratiquer la loi et les commandements.
\VS{4}Il ôta de toutes les villes de Juda les hauts lieux et les colonnes consacrées au soleil. Et le royaume fut en repos devant lui.
\VS{5}Il bâtit des villes fortes en Juda, car le pays fut en repos. Et pendant ces années-là, il n'y eut point de guerre contre lui, parce que Yahweh lui donna du repos.
\VS{6}Et il dit à Juda~: Bâtissons ces villes, et entourons-les de murailles, de tours, de portes et de barres~; le pays est encore devant nous, parce que nous avons recherché Yahweh, notre Dieu. Nous l'avons recherché, et il nous a donné du repos de toutes parts. Ainsi, ils bâtirent et prospérèrent.
\TextTitle{Asa s'appuie sur Yahweh et triomphe de Zérach\FTNTT{2 Ch. 16:1-10.}}
\VS{7}Or Asa avait dans son armée trois cent mille hommes de Juda, portant le grand bouclier et la lance, et deux cent quatre-vingt mille de Benjamin, portant le bouclier et tirant de l'arc, tous vaillants guerriers.
\VS{8}Mais Zérach, l'Ethiopien, sortit contre eux avec une armée d'un million d'hommes, et de trois cents chars~; et il vint jusqu'à Maréscha.
\VS{9}Asa alla au-devant de lui, et ils se rangèrent en bataille dans la vallée de Tsephata, près de Maréscha.
\VS{10}Alors Asa cria à Yahweh, son Dieu, et dit~: Yahweh~! Toi seul peux nous secourir, que l'on soit nombreux ou sans force~! Aide-nous, Yahweh, notre Dieu~! Car nous nous appuyons sur toi, et nous sommes venus en ton Nom contre cette multitude. Tu es Yahweh, notre Dieu~: Que l'homme ne prévale pas contre toi~!
\VS{11}Et Yahweh frappa les Ethiopiens devant Asa et devant Juda~; et les Ethiopiens s'enfuirent.
\VS{12}Asa et le peuple qui était avec lui les poursuivirent jusqu'à Guérar, et tant d'Ethiopiens tombèrent sans pouvoir sauver leur vie~; car ils furent brisés devant Yahweh et son armée, et on emporta un très grand butin.
\VS{13}Ils frappèrent aussi toutes les villes autour de Guérar, car la terreur de Yahweh était sur eux~; et ils pillèrent toutes ces villes, car il s'y trouvait un grand butin.
\VS{14}Ils frappèrent aussi les tentes des troupeaux, et emmenèrent des brebis et des chameaux en abondance~; puis ils retournèrent à Jérusalem.
\Chap{15}
\TextTitle{Azaria le prophète avertit Asa}
\VerseOne{}Alors l'Esprit de Dieu fut sur Azaria, fils d'Oded.
\VS{2}Et il sortit au-devant d'Asa, et lui dit~: Asa, et tout Juda et Benjamin, écoutez-moi~! Yahweh est avec vous quand vous êtes avec lui. Si vous le cherchez, vous le trouverez~; mais si vous l'abandonnez, il vous abandonnera.
\VS{3}Pendant longtemps Israël a été sans vrai Dieu, sans prêtre qui l'enseignait, et sans loi.
\VS{4}Mais dans leur détresse, ils sont revenus vers Yahweh, le Dieu d'Israël~; ils l'ont cherché, et ils l'ont trouvé\FTNT{Ps. 107:19-20.}.
\VS{5}Dans ces temps-là, il n'y avait point de sûreté pour ceux qui allaient et venaient, car il y avait de grands troubles parmi tous les habitants du pays.
\VS{6}Une nation était écrasée par une autre nation, et une ville par une autre ville~; car Dieu les agitait par toutes sortes d'angoisses.
\VS{7}Mais vous, fortifiez-vous, et que vos mains ne se relâchent pas~; car il y a une récompense pour vos œuvres.
\TextTitle{Asa écoute les paroles d'Azaria\FTNTT{1 R. 15:12-15.}}
\VS{8}Or dès qu'Asa eut entendu ces paroles et la prophétie d'Oded le prophète, il se fortifia~; et fit disparaître les abominations de tout le pays de Juda et de Benjamin, et des villes qu'il avait prises dans les montagnes d'Ephraïm~; et il rétablit l'autel de Yahweh, qui était devant le portique de Yahweh.
\VS{9}Puis il assembla tout Juda et Benjamin, et ceux d'Ephraïm, de Manassé et de Siméon, qui habitaient avec eux~; car un grand nombre de gens d'Israël passaient à lui, voyant que Yahweh, son Dieu, était avec lui.
\VS{10}Ils s'assemblèrent donc à Jérusalem, le troisième mois de la quinzième année du règne d'Asa~;
\VS{11}et ils sacrifièrent ce jour-là à Yahweh sept cents bœufs et sept mille brebis, du butin qu'ils avaient amené.
\VS{12}Et ils rentrèrent dans l'alliance pour chercher Yahweh, le Dieu de leurs pères, de tout leur cœur et de toute leur âme~;
\VS{13}de sorte qu'on devait faire mourir quiconque ne rechercherait pas Yahweh, le Dieu d'Israël, petit ou grand, homme ou femme.
\VS{14}Et ils jurèrent à Yahweh, à haute voix, avec des cris de joie, et au son des shofars et des cors.
\VS{15}Tout Juda se réjouit de ce serment, parce qu'ils avaient juré de tout leur cœur et qu'ils avaient recherché Yahweh de leur plein gré, et qu'ils l'avaient trouvé. Et Yahweh leur donna du repos de toutes parts.
\VS{16}Le roi Asa destitua même sa mère, Maaca, de son rang de reine, parce qu'elle avait fait une idole pour Asherah. Asa abattit l'idole, l'écrasa et la brûla près du torrent de Cédron.
\VS{17}Mais les hauts lieux ne furent point ôtés du milieu d'Israël. Néanmoins, le cœur d'Asa fut intègre tout le long de ses jours.
\VS{18}Il remit dans la maison de Dieu les choses que son père avait consacrées, avec ce qu'il avait lui-même consacré, l'argent, l'or et les ustensiles.
\VS{19}Et il n'y eut point de guerre jusqu'à la trente-cinquième année du règne d'Asa.
\Chap{16}
\TextTitle{Alliance d'Asa et du roi de Syrie contre Israël\FTNTT{1 R. 15:16-22~; cp. 1 R. 15:27~; 16:7.}}
\VerseOne{}La trente-sixième année du règne d'Asa, Baescha, roi d'Israël, monta contre Juda, et il bâtit Rama, pour empêcher quiconque de sortir et d'entrer vers Asa, roi de Juda.
\VS{2}Alors Asa sortit de l'argent et de l'or des trésors de la maison de Yahweh et de la maison royale, et il envoya dire à Ben-Hadad, roi de Syrie, qui habitait à Damas~:
\VS{3}Il y a alliance entre nous, et entre mon père et ton père~; voici, je t'envoie de l'argent et de l'or~; va, romps l'alliance que tu as avec Baescha, roi d'Israël, afin qu'il s'éloigne de moi.
\VS{4}Ben-Hadad écouta le roi Asa, et il envoya les chefs de son armée contre les villes d'Israël, et ils frappèrent Ijjon, Dan, Abel-Maïm, et tous les magasins des villes de Nephthali.
\VS{5}Et aussitôt que Baescha l'apprit, il cessa de bâtir Rama et suspendit ses travaux.
\VS{6}Alors le roi Asa prit avec lui tout Juda, et ils emportèrent les pierres et le bois de Rama, que Baescha faisait bâtir~; et il en bâtit Guéba et Mitspa.
\TextTitle{Hanani condamne l'alliance d'Asa}
\VS{7}En ce temps-là, Hanani le voyant, vint vers Asa, roi de Juda, et lui dit~: Parce que tu t'es appuyé sur le roi de Syrie, et que tu ne t'es point appuyé sur Yahweh, ton Dieu, l'armée du roi de Syrie a échappé de ta main.
\VS{8}Les Ethiopiens et les Libyens n'étaient-ils pas une grande armée, ayant des chars et une multitude de cavaliers~? Mais parce que tu t'étais appuyé sur Yahweh, il les livra entre tes mains.
\VS{9}Car les yeux de Yahweh parcourent toute la terre, pour soutenir ceux dont le cœur est tout entier à lui. Tu as agi follement en cela~; car désormais tu auras des guerres.
\VS{10}Asa fut irrité contre le voyant, et le mit en prison, car il était indigné contre lui à ce sujet. Asa opprima aussi, en ce temps-là, quelques-uns du peuple.
\TextTitle{Mort d'Asa\FTNTT{1 R. 15:23-24.}}
\VS{11}Or voici, les actions d'Asa, les premières et les dernières, sont écrites dans le livre des rois de Juda et d'Israël.
\VS{12}Asa fut malade des pieds la trente-neuvième année de son règne, et sa maladie fut très grave. Toutefois, il ne chercha point Yahweh dans sa maladie, mais les médecins.
\VS{13}Puis Asa s'endormit avec ses pères, et il mourut la quarante et unième année de son règne.
\VS{14}On l'ensevelit dans le sépulcre qu'il s'était creusé dans la cité de David. On le coucha dans un lit qui était rempli de parfums et d'aromates, composés par le travail d'un parfumeur~; et on lui en brûla une quantité considérable.
\Chap{17}
\TextTitle{Josaphat règne sur Juda, il recherche Yahweh\FTNTT{1 R. 15:24.}}
\VerseOne{}Josaphat son fils régna à sa place et se fortifia contre Israël.
\VS{2}Il mit des troupes dans toutes les villes fortes de Juda, et des garnisons dans le pays de Juda, et dans les villes d'Ephraïm qu'Asa, son père, avait prises.
\VS{3}Yahweh fut avec Josaphat, parce qu'il suivit les premières voies de David, son père, et qu'il ne rechercha point les Baals~;
\VS{4}car il rechercha le Dieu de son père, et il marcha dans ses commandements, et non pas selon ce que faisait Israël.
\VS{5}Yahweh affermit donc le royaume entre ses mains~; et tout Juda apportait des présents à Josaphat, et il eut en abondance des richesses et de la gloire.
\VS{6}Son cœur grandit dans les voies de Yahweh, et il ôta encore de Juda les hauts lieux et les Asherah.
\VS{7}Puis, la troisième année de son règne, il envoya ses chefs Ben-Haïl, Abdias, Zacharie, Nethaneel et Michée, pour enseigner dans les villes de Juda~;
\VS{8}et avec eux les Lévites Schemaeja, Nethania, Zebadia, Asaël, Schemiramoth, Jonathan, Adonija, Tobija et Tob-Adonija, Lévites, et avec eux Elischama et Joram, les prêtres.
\VS{9}Ils enseignèrent dans Juda, ayant avec eux le livre de la loi de Yahweh. Ils firent le tour de toutes les villes de Juda, et enseignèrent parmi le peuple.
\TextTitle{Affermissement du règne de Josaphat}
\VS{10}La terreur de Yahweh fut sur tous les royaumes des pays qui entouraient Juda, et ils ne firent point la guerre à Josaphat.
\VS{11}On apporta aussi à Josaphat des présents de la part des Philistins, et un impôt en argent~; et les Arabes lui amenèrent aussi du bétail, sept mille sept cents béliers et sept mille sept cents boucs.
\VS{12}Ainsi Josaphat s'élevait jusqu'au plus haut degré de gloire. Et il bâtit en Juda des châteaux et des villes pour servir de magasins.
\VS{13}Il fit de grands travaux dans les villes de Juda~; et il avait à Jérusalem des gens de guerre puissants et vaillants.
\VS{14}Voici leur dénombrement, selon les maisons de leurs pères. Les chefs de milliers de Juda furent Adna le chef, avec trois cent mille vaillants guerriers.
\VS{15}Et après lui, Jochanan le chef, avec deux cent quatre-vingt mille hommes.
\VS{16}A ses côtés, Amasia, fils de Zicri, qui s'était volontairement offert à Yahweh, avec deux cent mille vaillants guerriers.
\VS{17}De Benjamin, Eliada, vaillant guerrier, avec deux cent mille hommes, armés d'arcs et de boucliers,
\VS{18}à côté de lui Zozabad, avec cent quatre-vingt mille hommes équipés pour le combat.
\VS{19}Tels sont ceux qui étaient au service du roi, outre ceux que le roi avait placés dans toutes les villes fortes de Juda.
\Chap{18}
\TextTitle{Josaphat s'allie à Achab contre les Syriens\FTNTT{1 R. 22:2-4.}}
\VerseOne{}Or Josaphat, ayant beaucoup de richesses et de gloire, s'allia par mariage avec Achab.
\VS{2}Et au bout de quelques années, il descendit vers Achab, à Samarie. Achab tua pour lui, et pour le peuple qui était avec lui, un grand nombre de brebis et de bœufs, et l'incita à monter contre Ramoth de Galaad.
\VS{3}Achab, roi d'Israël, dit à Josaphat, roi de Juda~: Viendras-tu avec moi contre Ramoth de Galaad~? Et il lui répondit~: Compte sur moi comme sur toi, et sur mon peuple comme sur ton peuple, nous irons avec toi à la guerre.
\TextTitle{Les prophètes de mensonge encouragent Achab\FTNTT{1 R. 22:5-12.}}
\VS{4}Puis Josaphat dit au roi d'Israël~: Consulte aujourd'hui, je te prie, la parole de Yahweh.
\VS{5}Le roi d'Israël assembla les prophètes, au nombre de quatre cents, et leur dit~: Irons-nous à la guerre contre Ramoth de Galaad, ou dois-je y renoncer~? Ils répondirent~: Monte, et Dieu la livrera entre les mains du roi.
\VS{6}Mais Josaphat dit~: N'y a-t-il point encore ici quelque prophète de Yahweh, afin que nous l'interrogions~?
\VS{7}Le roi d'Israël dit à Josaphat~: Il y a encore un homme par qui on peut consulter Yahweh~; mais je le hais parce qu'il ne me prophétise rien de bon, mais du mal~; c'est Michée, fils de Jimla. Josaphat dit~: Que le roi ne parle pas ainsi~!
\VS{8}Alors le roi d'Israël appela un eunuque, et dit~: Fais promptement venir Michée, fils de Jimla.
\VS{9}Or le roi d'Israël et Josaphat, roi de Juda, étaient assis, chacun sur son trône, revêtus de leurs habits, et ils étaient assis dans la place, à l'entrée de la porte de Samarie~; et tous les prophètes prophétisaient en leur présence.
\VS{10}Alors Sédécias, fils de Kenaana, s'étant fait des cornes de fer, dit~: Ainsi parle Yahweh~: Avec ces cornes tu heurteras les Syriens jusqu'à les détruire.
\VS{11}Tous les prophètes prophétisaient de même, en disant~: Monte à Ramoth de Galaad, et tu prospéreras~; Yahweh la livrera entre les mains du roi.
\TextTitle{Michée annonce la défaite et la mort d'Achab\FTNTT{1 R. 22:13-28~; 1 R. 22:29-40.}}
\VS{12}Or le messager qui était allé appeler Michée, lui parla et lui dit~: Voici, tous les prophètes disent d'une même bouche du bien au roi~; je te prie que ta parole soit semblable à celle de chacun d'eux~! Annonce du bien~!
\VS{13}Mais Michée répondit~: Yahweh est vivant~! Je dirai ce que mon Dieu dira.
\VS{14}Il vint donc vers le roi, et le roi lui dit~: Michée, irons-nous à la guerre contre Ramoth de Galaad, devons-nous y renoncer~? Et il répondit~: Montez, vous prospérerez, et ils seront livrés entre vos mains.
\VS{15}Et le roi lui dit~: Combien de fois devrais-je te faire jurer de ne me dire que la vérité au Nom de Yahweh~?
\VS{16}Et il répondit~: J'ai vu tout Israël dispersé par les montagnes, comme un troupeau de brebis qui n'a point de berger~; et Yahweh a dit~: Ces gens n'ont point de seigneur~; que chacun retourne en paix dans sa maison~!
\VS{17}Alors le roi d'Israël dit à Josaphat~: Ne t'ai-je pas dit qu'il ne prophétise rien de bon quand il s'agit de moi, mais seulement du mal~?
\VS{18}Et Michée dit~: Ecoute la parole de Yahweh~! J'ai vu Yahweh assis sur son trône, et toute l'armée des cieux se tenant à sa droite et à sa gauche.
\VS{19}Et Yahweh dit~: Qui est-ce qui séduira Achab, roi d'Israël, afin qu'il monte et qu'il tombe à Ramoth de Galaad~? Et l'un répondait d'une façon et l'autre d'une autre.
\VS{20}Alors un esprit s'avança et se tint devant Yahweh, et dit~: Moi, je le séduirai. Yahweh lui dit~: Comment~?
\VS{21}Il répondit~: Je sortirai, dit-il, et je serai un esprit de mensonge\FTNT{Achab a été frappé de l'esprit d'égarement (2 Th. 2:9-11). Voir commentaires en Ge. 6:3~; Mt. 12:31.} dans la bouche de tous ses prophètes. Et Yahweh dit~: Tu le séduiras, et même tu en viendras à bout. Sors, et fais ainsi.
\VS{22}Maintenant voici, Yahweh a mis un esprit de mensonge dans la bouche de tes prophètes que voilà~; et Yahweh a prononcé du mal contre toi.
\VS{23}Alors Sédécias, fils de Kenaana, s'étant approché, frappa Michée sur la joue, et dit~: Par quel chemin l'Esprit de Yahweh s'est-il retiré de moi pour te parler~?
\VS{24}Et Michée répondit~: Voici, tu le verras au jour où tu iras de chambre en chambre pour te cacher~!
\VS{25}Alors le roi d'Israël dit~: Prenez Michée, et emmenez-le vers Amon, chef de la ville, et vers Joas, fils du roi.
\VS{26}Et vous direz~: Ainsi parle le roi~: Mettez cet homme en prison, et nourrissez-le du pain et de l'eau de l'affliction, jusqu'à ce que je revienne en paix.
\VS{27}Et Michée dit~: Si jamais tu retournes et reviens en paix, Yahweh n'aura point parlé par moi. Et il dit~: Entendez cela peuples, vous tous qui êtes ici~!
\VS{28}Le roi d'Israël monta donc avec Josaphat, roi de Juda, à Ramoth de Galaad.
\VS{29}Le roi d'Israël dit à Josaphat~: Je vais me déguiser pour aller au combat~; mais toi, revêts-toi de tes habits. Ainsi le roi d'Israël se déguisa~; et ils allèrent au combat.
\VS{30}Or le roi des Syriens avait donné cet ordre aux chefs de ses chars, disant~: Vous ne combattrez ni petit ni grand, mais seulement le roi d'Israël.
\VS{31}Les chefs des chars aperçurent Josaphat, et dirent~: C'est le roi d'Israël~! Et ils se tournèrent vers lui pour le combattre~; mais Josaphat poussa un cri, et Yahweh le secourut, et Dieu les éloigna de lui.
\VS{32}Quand les chefs des chars virent que ce n'était pas le roi d'Israël, ils se détournèrent de lui.
\VS{33}Alors quelqu'un tira de son arc au hasard, et frappa le roi d'Israël entre les jointures de la cuirasse~; et le roi dit à son conducteur de char~: Tourne-toi, et sors-moi du camp~; car je suis blessé.
\VS{34}Or en ce jour-là, le combat fut très rude. Le roi d'Israël se posa dans son char, en face des Syriens, jusqu'au soir~; et il mourut vers le coucher du soleil.
\Chap{19}
\TextTitle{Jéhu dénonce l'alliance de Josaphat avec Achab}
\VerseOne{}Josaphat roi de Juda, revint en paix dans sa maison, à Jérusalem.
\VS{2}Mais Jéhu, fils de Hanani, le voyant, sortit au-devant du roi Josaphat, et lui dit~: Faut-il donner du secours au méchant, ou aimer ceux qui haïssent Yahweh~? A cause de cela, Yahweh est irrité contre toi.
\VS{3}Mais il s'est trouvé de bonnes choses en toi, puisque tu as ôté du pays les Asherah, et tu as appliqué ton cœur à rechercher Dieu.
\VS{4}Josaphat demeura à Jérusalem. Puis, il ressortit de nouveau parmi le peuple, depuis Beer-Schéba jusqu'à la montagne d'Ephraïm, et il les ramena à Yahweh, le Dieu de leurs pères.
\TextTitle{Josaphat organise la justice}
\VS{5}Il établit aussi des juges dans le pays, dans toutes les villes fortes de Juda, de ville en ville.
\VS{6}Et il dit aux juges~: Veillez sur ce que vous ferez~; car vous n'exercez pas la justice de la part d'un homme, mais de la part de Yahweh, qui sera avec vous quand vous prononcerez les jugements.
\VS{7}Maintenant, que la crainte de Yahweh soit sur vous~; prenez garde à ce que vous ferez~; car il n'y a point d'iniquité chez Yahweh, notre Dieu, ni d'acception de personnes, ni d'acceptation de présents.
\VS{8}Josaphat établit aussi à Jérusalem des Lévites, des prêtres, et des chefs des pères d'Israël, pour le jugement de Yahweh, et pour les contestations~; car on revenait à Jérusalem.
\VS{9}Il leur donna des ordres, en disant~: Vous agirez ainsi dans la crainte de Yahweh, avec fidélité et avec intégrité de cœur.
\VS{10}Dans toute contestation qui viendra devant vous, de la part de vos frères qui habitent dans leurs villes, soit d'un meurtre, d'une loi, d'un commandement, d'un statut ou d'une ordonnance, vous les instruirez, afin qu'ils ne se rendent pas coupables envers Yahweh, et que sa colère ne vienne pas sur vous et sur vos frères. Vous agirez ainsi afin de ne pas être coupables.
\VS{11}Et voici, Amaria, le grand-prêtre, sera au-dessus de vous pour toutes les affaires de Yahweh~; et Zebadia, fils d'Ismaël, prince de la maison de Juda, pour toutes les affaires du roi~; et pour secrétaires, vous avez devant vous les Lévites. Fortifiez-vous et faites ainsi~; et que Yahweh soit avec l'homme de bien~!
\Chap{20}
\TextTitle{Menaces des ennemis de Juda, prière de Josaphat}
\VerseOne{}Après ces choses, les fils de Moab et les fils d'Ammon, et avec eux les Maonites, vinrent contre Josaphat pour lui faire la guerre.
\VS{2}On vint le rapporter à Josaphat, en disant~: Il vient contre toi une grande multitude depuis l'autre bord de la mer, de Syrie~; et les voici à Hatsatson-Thamar, qui est En-Guédi.
\VS{3}Alors Josaphat craignit~; mais il se disposa à rechercher Yahweh, et publia un jeûne pour tout Juda.
\VS{4}Juda s'assembla donc pour rechercher Yahweh~; on vint même de toutes les villes de Juda pour chercher Yahweh.
\VS{5}Et Josaphat se tint au milieu de l'assemblée de Juda et de Jérusalem, dans la maison de Yahweh, devant le nouveau parvis.
\VS{6}Il dit~: Yahweh, Dieu de nos pères~! N'es-tu pas Dieu dans les cieux, toi qui domines sur tous les royaumes des nations~? Ne tiens-tu pas dans ta main la force et la puissance, de sorte que nul ne peut résister~?
\VS{7}N'est-ce pas toi, ô notre Dieu, qui as dépossédé les habitants de ce pays devant ton peuple d'Israël, et qui l'as donné pour toujours à la postérité d'Abraham, qui t'aimait~?
\VS{8}Ils y ont habité et t'y ont bâti un sanctuaire pour ton Nom, en disant~:
\VS{9}S'il nous arrive quelque malheur, l'épée, le jugement, la peste, ou la famine, nous nous tiendrons devant cette maison, et en ta présence~; car ton Nom est en cette maison~; et nous crierons à toi dans notre détresse, et tu exauceras et tu délivreras~!
\VS{10}Maintenant, voici les enfants d'Ammon et de Moab, et ceux de la montagne de Séir, chez lesquels tu ne permis pas à Israël d'entrer quand il venait du pays d'Egypte, car il se détourna d'eux, et ne les détruisit pas.
\VS{11}Voici, pour nous récompenser, ils viennent nous chasser de ton héritage, que tu nous as fait posséder.
\VS{12}Ô notre Dieu~! Ne seras-tu pas juge contre eux~? Car nous sommes sans force devant cette grande multitude qui vient contre nous, et nous ne savons que faire~; mais nos yeux sont sur toi.
\VS{13}Or tout Juda se tenait devant Yahweh, même avec leurs petits enfants, leurs femmes et leurs fils.
\TextTitle{Yahweh répond à Josaphat}
\VS{14}Alors l'Esprit de Yahweh saisit au milieu de l'assemblée Jachaziel, fils de Zacharie, fils de Benaja, fils de Jeïel, fils de Matthania, Lévite, d'entre les fils d'Asaph,
\VS{15}et il dit~: Soyez attentifs, tout Juda et habitants de Jérusalem, et toi, roi Josaphat~! Ainsi parle Yahweh~: Ne craignez point, et ne soyez point effrayés en face de cette grande multitude~; car ce ne sera pas à vous de combattre, mais à Dieu.
\VS{16}Descendez demain vers eux~; les voici qui montent par la montée de Tsits, et vous les trouverez à l'extrémité de la vallée, en face du désert de Jeruel.
\VS{17}Ce ne sera point à vous de combattre dans cette bataille~; présentez-vous, tenez-vous là, et voyez la délivrance que Yahweh va vous donner. Juda et Jérusalem, ne craignez point, et ne soyez point effrayés~! Demain, sortez au-devant d'eux, et Yahweh sera avec vous.
\VS{18}Alors Josaphat s'inclina le visage contre terre, et tout Juda et les habitants de Jérusalem se jetèrent devant Yahweh, se prosternant devant Yahweh.
\VS{19}Et les Lévites, d'entre les fils des Kehathites et d'entre les fils des Koréites, se levèrent pour célébrer Yahweh, le Dieu d'Israël, d'une voix haute et forte.
\TextTitle{Yahweh délivre Juda des armées ennemies}
\VS{20}Puis, le matin, ils se levèrent de bonne heure et sortirent vers le désert de Tekoa. Et comme ils sortaient, Josaphat se tint debout et dit~: Ecoutez-moi Juda et vous, habitants de Jérusalem~! Croyez en Yahweh, votre Dieu, et vous serez en sûreté~; croyez en ses prophètes, et vous réussirez.
\VS{21}Puis, ayant consulté le peuple, il établit des chantres de Yahweh, qui célébraient sa sainte magnificence~; et marchant devant l'armée, ils disaient~: Louez Yahweh, car sa miséricorde dure à toujours\FTNT{Ps. 136.}~!
\VS{22}Et au moment où ils commencèrent le chant et la louange, Yahweh mit des embuscades contre les fils d'Ammon, de Moab, et ceux de la montagne de Séir, qui venaient contre Juda. Et ils furent battus.
\VS{23}Les fils d'Ammon et de Moab se levèrent contre les habitants de la montagne de Séir pour les dévouer par interdit et les exterminer~; et quand ils en eurent fini avec les habitants de Séir, ils s'aidèrent l'un l'autre à se détruire mutuellement.
\VS{24}Et quand Juda fut arrivé sur la hauteur d'où l'on voit le désert, ils regardèrent vers cette multitude, et voici, c'étaient des cadavres gisant à terre, et personne n'avait échappé.
\VS{25}Ainsi Josaphat et son peuple vinrent pour piller leurs dépouilles, et ils trouvèrent parmi les cadavres des biens en abondance, et des objets précieux~; et ils en saisirent tant, qu'ils ne pouvaient tout porter~; et ils pillèrent le butin pendant trois jours, car il était considérable.
\VS{26}Le quatrième jour, ils s'assemblèrent dans la vallée de Beraca~; car ils bénirent là Yahweh~; c'est pourquoi on a appelé ce lieu, jusqu'à ce jour, la vallée de Beraca.
\VS{27}Et tous les hommes de Juda et de Jérusalem, et Josaphat à leur tête, s'en retournèrent, revenant à Jérusalem avec joie~; car Yahweh les avait réjouis au sujet de leurs ennemis.
\VS{28}Ils entrèrent donc à Jérusalem, dans la maison de Yahweh, avec des luths, des harpes et des trompettes.
\VS{29}Et la crainte de Dieu fut sur tous les royaumes des autres pays, lorsqu'ils apprirent que Yahweh avait combattu contre les ennemis d'Israël.
\VS{30}Ainsi le royaume de Josaphat fut tranquille, et son Dieu lui donna du repos de toutes parts.
\TextTitle{Règne de Josaphat, son alliance coupable\FTNTT{1 R. 22:41-49.}}
\VS{31}Josaphat régna donc sur Juda. Il était âgé de trente-cinq ans quand il devint roi, et il régna vingt-cinq ans à Jérusalem. Sa mère s'appelait Azuba, fille de Schilchi.
\VS{32}Il suivit les traces d'Asa, son père, et il ne s'en détourna point, faisant ce qui est droit aux yeux de Yahweh.
\VS{33}Seulement les hauts lieux ne furent pas ôtés, et le peuple n'avait pas encore le cœur fermement attaché au Dieu de ses pères.
\VS{34}Or le reste des actions de Josaphat, les premières et les dernières, voici, elles sont écrites dans les mémoires de Jéhu, fils de Hanani, insérées dans le livre des rois d'Israël.
\VS{35}Après cela, Josaphat, roi de Juda, s'associa avec Achazia, roi d'Israël, dont la conduite était impie.
\VS{36}Il s'associa avec lui pour faire des navires, afin d'aller à Tarsis~; et ils firent des navires à Etsjon-Guéber.
\VS{37}Alors Eliézer, fils de Dodava, de Maréscha, prophétisa contre Josaphat, en disant~: Parce que tu t'es associé avec Achazia, Yahweh a détruit ton œuvre. Et les navires furent brisés, et ne purent aller à Tarsis.
\Chap{21}
\TextTitle{Joram règne sur Juda\FTNTT{1 R. 22:50~; 2 R. 8:16-19.}}
\VerseOne{}Puis Josaphat s'endormit avec ses pères, et il fut enseveli avec eux dans la cité de David. Et Joram, son fils, régna à sa place.
\VS{2}Il avait des frères, fils de Josaphat~: Azaria, Jehiel, Zacharie, Azaria, Micaël et Schephathia. Tous ceux-là étaient fils de Josaphat, roi d'Israël.
\VS{3}Leur père leur avait fait de grands dons d'argent, d'or et de choses précieuses, avec des villes fortes en Juda~; mais il avait donné le royaume à Joram, parce qu'il était le premier-né.
\VS{4}Quand Joram fut élevé sur le royaume de son père, et s'y fut fortifié, il tua avec l'épée tous ses frères, et quelques-uns aussi des chefs d'Israël.
\VS{5}Joram était âgé de trente-deux ans quand il devint roi, et il régna huit ans à Jérusalem.
\VS{6}Il marcha dans la voie des rois d'Israël, comme avait fait la maison d'Achab~; car la fille d'Achab était sa femme, et il fit ce qui est mal aux yeux de Yahweh.
\VS{7}Toutefois, Yahweh, à cause de l'alliance qu'il avait traitée avec David, ne voulut pas détruire la maison de David, selon qu'il avait dit qu'il lui donnerait une lampe, à lui et à ses fils, pour toujours.
\TextTitle{Rébellion d'Edom et de Libna\FTNTT{1 R. 8:20-23.}}
\VS{8}De son temps, Edom se rebella de l'autorité de Juda, et établit un roi sur lui.
\VS{9}Joram se mit donc en marche avec ses chefs et tous ses chars~; et s'étant levé de nuit, il battit les Edomites qui l'entouraient, et tous les chefs des chars.
\VS{10}Néanmoins, Edom se rebella contre l'autorité de Juda jusqu'à ce jour. En ce même temps, Libna se rebella aussi contre son autorité, parce qu'il avait abandonné Yahweh, le Dieu de ses pères.
\VS{11}Il fit aussi des hauts lieux dans les montagnes de Juda~; il fit que les habitants de Jérusalem se prostituèrent, et il y entraîna ceux de Juda.
\TextTitle{Elie prononce un jugement sur Joram}
\VS{12}Alors il lui vint un écrit de la part d'Elie, le prophète, disant~: Ainsi parle Yahweh, le Dieu de David, ton père~: Parce que tu n'as point suivi le chemin de Josaphat, ton père, ni celui d'Asa, roi de Juda,
\VS{13}mais que tu as suivi les voies des rois d'Israël, et que tu as poussé à la prostitution Juda et les habitants de Jérusalem, comme s'est prostituée la maison d'Achab, et que tu as tué tes frères, meilleurs que toi, la maison même de ton père~;
\VS{14}voici, Yahweh frappera d'une grande plaie ton peuple, tes fils, tes femmes et tous tes biens.
\VS{15}Et toi, tu auras une grosse maladie, une maladie d'entrailles~; jusqu'à ce que, de jour en jour, tes entrailles sortent par la force de la maladie.
\TextTitle{Yahweh excite les Philistins et les Arabes contre Joram}
\VS{16}Yahweh souleva contre Joram l'esprit des Philistins et des Arabes, qui habitent près des Ethiopiens.
\VS{17}Ils montèrent donc contre Juda, et firent une brèche pour piller toutes les richesses qui furent trouvées dans la maison du roi~; et même, ils emmenèrent captifs ses fils et ses femmes, de sorte qu'il ne lui resta d'autre fils que Joachaz, le plus jeune de ses fils.
\TextTitle{Mort de Joram}
\VS{18}Après tout cela, Yahweh frappa ses entrailles d'une maladie sans remède.
\VS{19}Elle s'avança chaque jour, et vers la fin de la seconde année, ses entrailles sortirent par la force de son mal, et il mourut dans de grandes souffrances. Son peuple ne brûla pas sur lui de parfums, comme il l'avait fait pour ses pères.
\VS{20}Il était âgé de trente-deux ans quand il devint roi, et il régna huit ans à Jérusalem. Il s'en alla sans être regretté, et on l'ensevelit dans la cité de David, mais non dans les sépulcres des rois.
\Chap{22}
\TextTitle{Achazia règne sur Juda\FTNTT{2 R. 8:24-29.}}
\VerseOne{}Les habitants de Jérusalem firent régner à sa place Achazia, le plus jeune de ses fils, parce que les troupes qui étaient venues au camp avec les Arabes avaient tué tous les plus âgés~; et Achazia, fils de Joram, roi de Juda, régna.
\VS{2}Achazia était âgé de quarante-deux ans quand il devint roi, et il régna un an à Jérusalem. Sa mère avait pour nom Athalie, fille d'Omri.
\VS{3}Il suivit aussi les voies de la maison d'Achab, car sa mère lui donnait des conseils impies.
\VS{4}Il fit donc ce qui est mal aux yeux de Yahweh, comme la maison d'Achab~; parce qu'ils furent ses conseillers après la mort de son père, pour sa ruine.
\TextTitle{Achazia livré aux mains de Jéhu\FTNTT{2 R. 8:28-29~; 9:1-30.}}
\VS{5}Conduit par leurs conseils, il alla avec Joram, fils d'Achab, roi d'Israël, à la guerre à Ramoth de Galaad, contre Hazaël, roi de Syrie. Et les Syriens frappèrent Joram,
\VS{6}qui s'en retourna à Jizreel, pour guérir des blessures que les Syriens lui avaient faites à Rama, lorsqu'il faisait la guerre contre Hazaël, roi de Syrie. Azaria, fils de Joram, roi de Juda, descendit pour voir Joram, le fils d'Achab, à Jizreel, parce qu'il était malade.
\VS{7}Et ce fut pour son entière ruine, qui procédait de Dieu, qu'Achazia vint auprès de Joram. En effet, quand il fut arrivé, il sortit avec Joram pour aller au-devant de Jéhu, fils de Nimschi, que Yahweh avait oint pour retrancher la maison d'Achab.
\VS{8}Et comme Jéhu faisait justice de la maison d'Achab\FTNT{2 R. 10:12-30.}, il trouva les chefs de Juda et les fils des frères d'Achazia, qui servaient Achazia, et il les tua.
\VS{9}Il chercha ensuite Achazia, qui s'était caché en Samarie. On le prit, et on l'amena vers Jéhu qui le fit mourir. Puis on l'ensevelit, car on dit~: C'est le fils de Josaphat, qui cherchait Yahweh de tout son cœur. Et il n'y eut plus personne dans la maison d'Achazia qui fut capable de régner.
\TextTitle{Joas échappe au massacre de sa famille\FTNTT{2 R. 11:1-3.}}
\VS{10}Or Athalie, mère d'Achazia, voyant que son fils était mort, se leva et fit périr toute la race royale de la maison de Juda.
\VS{11}Mais Joschéba, fille du roi Joram, prit Joas, fils d'Achazia, en le dérobant d'entre les fils du roi qu'on faisait mourir. Elle le mit avec sa nourrice dans la chambre des lits. Ainsi Joschabeath, fille du roi Joram et femme de Jehojada le prêtre, étant la sœur d'Achazia, le cacha de la vue d'Athalie, qui ne put le faire mourir.
\VS{12}Il fut ainsi caché avec eux dans la maison de Dieu six ans~; et c'est Athalie qui régnait sur le pays.
\Chap{23}
\TextTitle{Joas devient roi grâce à Jehojada\FTNTT{2 R. 11:4-12.}}
\VerseOne{}Mais la septième année, Jehojada prit courage et traita alliance avec les chefs de centaines, Azaria, fils de Jerocham, Ismaël, fils de Jochanan, Azaria, fils d'Obed, Maaséja, fils d'Adaja, et Elischaphath, fils de Zicri.
\VS{2}Ils firent le tour de Juda, pour rassembler de toutes les villes de Juda les Lévites et les chefs des pères d'Israël~; puis ils vinrent à Jérusalem.
\VS{3}Et toute cette assemblée traita alliance avec le roi dans la maison de Dieu. Jehojada leur dit~: Voici, c'est le fils du roi qui régnera, selon la parole de Yahweh au sujet des fils de David.
\VS{4}Vous ferez donc ceci~: Le tiers qui parmi vous entre en service au sabbat, prêtres et Lévites, fera la garde des seuils.
\VS{5}Un autre tiers se tiendra dans la maison du roi, et un tiers à la porte de Jesod~; et tout le peuple sera dans les parvis de la maison de Yahweh.
\VS{6}Que personne n'entre dans la maison de Yahweh, sauf les prêtres et les Lévites de service~: Ils entreront car ils sont sanctifiés~; et tout le reste du peuple gardera les ordres de Yahweh.
\VS{7}Les Lévites environneront le roi de toutes parts, tenant chacun leurs armes à la main, et donneront la mort à quiconque voudra entrer dans la maison~; vous serez avec le roi quand il entrera et quand il sortira.
\VS{8}Les Lévites et tout Juda firent tout ce que Jehojada le prêtre, avait ordonné. Ils prirent chacun leurs gens, tant ceux qui entraient en service que ceux qui en sortaient au sabbat~; car Jehojada, le prêtre, n'avait exempté aucune classe.
\VS{9}Et Jehojada le prêtre, donna aux chefs de centaines les lances, les grands et les petits boucliers qui provenaient du roi David, et qui étaient dans la maison de Dieu.
\VS{10}Puis il rangea tout le peuple autour du roi, chacun tenant ses armes à la main, du côté droit du temple jusqu'au côté gauche de la maison, près de l'autel et de la maison.
\VS{11}Alors ils firent sortir le fils du roi, et mirent sur lui la couronne et le témoignage. Ils l'établirent roi, et Jehojada et ses fils l'oignirent et dirent~: Vive le roi~!
\TextTitle{Mort d'Athalie\FTNTT{2 R. 11:13-16.}}
\VS{12}Mais Athalie, entendant le bruit du peuple qui courait et célébrait le roi, vint vers le peuple, dans la maison de Yahweh.
\VS{13}Elle regarda, et voici, le roi se tenait près de la colonne, à l'entrée~; les chefs et les trompettes étaient près du roi, et tout le peuple du pays était dans la joie, et l'on sonnait des trompettes~; les chantres, avec des instruments de musique, dirigeaient les chants de louanges. Alors Athalie déchira ses vêtements et dit~: Conspiration~! Conspiration~!
\VS{14}Le prêtre Jehojada fit sortir les chefs de centaines qui étaient à la tête de l'armée, et leur dit~: Faites-la sortir hors des rangs, et que celui qui la suivra soit mis à mort par l'épée~! Car le prêtre avait dit~: Ne la mettez pas à mort dans la maison de Yahweh.
\VS{15}Ils mirent donc la main sur elle pour la faire entrer dans la maison du roi, par l'entrée de la porte des chevaux~; et là ils la firent mourir.
\TextTitle{Jehojada fait asseoir Joas sur le trône de Juda\FTNTT{2 R. 11:17-20.}}
\VS{16}Puis Jehojada traita, avec tout le peuple et le roi, une alliance pour être le peuple de Yahweh.
\VS{17}Et tout le peuple entra dans la maison de Baal pour la détruire. Ils brisèrent ses autels et ses images et ils tuèrent devant les autels Matthan, prêtre de Baal.
\VS{18}Jehojada remit aussi les fonctions de la maison de Yahweh entre les mains des prêtres, des Lévites, comme David les avait répartis dans la maison de Yahweh, afin qu'ils élèvent des holocaustes à Yahweh, comme cela est écrit dans la loi de Moïse, avec joie et avec des chants, selon les ordonnances de David.
\VS{19}Il établit aussi les portiers aux portes de la maison de Yahweh, afin qu'il n'y entrât aucune personne souillée de quelque manière que ce fût.
\VS{20}Il prit les chefs de centaines, hommes considérés, qui avaient de l'autorité parmi le peuple, et tout le peuple du pays. Il fit descendre le roi, de la maison de Yahweh à la maison du roi, en entrant par la porte supérieure~; et ils firent asseoir le roi sur le trône royal.
\VS{21}Alors tout le peuple du pays se réjouit, et la ville fut tranquille, bien qu'on eût mis à mort Athalie par l'épée.
\Chap{24}
\TextTitle{Joas règne sur Juda~; ses travaux sur le temple\FTNTT{2 R. 11:21-12:8.}}
\VerseOne{}Joas était âgé de sept ans quand il devint roi, et il régna quarante ans à Jérusalem. Sa mère avait pour nom Tsibja, de Beer-Schéba.
\VS{2}Joas fit ce qui est droit aux yeux de Yahweh, pendant toute la vie de Jehojada, le prêtre.
\VS{3}Et Jehojada prit pour lui deux femmes, dont il eut des fils et des filles.
\VS{4}Après cela Joas eut la pensée de renouveler la maison de Yahweh.
\VS{5}Il assembla donc les prêtres et les Lévites, et leur dit~: Allez vers les villes de Juda, et recueillez de l'argent par tout Israël, suffisamment pour réparer la maison de votre Dieu d'année en année, et hâtez-vous pour cette affaire. Mais les Lévites ne se hâtèrent point.
\VS{6}Alors le roi appela Jehojada, leur chef, et lui dit~: Pourquoi n'as-tu pas veillé à ce que les Lévites aient apporté de Juda et de Jérusalem, l'impôt sur l'assemblée d'Israël, selon Moïse, serviteur de Yahweh, pour la tente du témoignage~?
\VS{7}Car l'impie Athalie et ses fils ont ravagé la maison de Dieu~; et même ils ont employé pour les Baals toutes les choses consacrées à la maison de Yahweh.
\TextTitle{Offrandes volontaires pour la réparation du temple\FTNTT{2 R. 12:9-16.}}
\VS{8}Et le roi ordonna qu'on fasse un seul coffre, et qu'on le mette à la porte de la maison de Yahweh, à l'extérieur.
\VS{9}Puis on publia dans Juda et dans Jérusalem, pour qu'on apportât à Yahweh l'impôt mis par Moïse, serviteur de Dieu, sur Israël dans le désert.
\VS{10}Tous les chefs et tout le peuple s'en réjouirent, et l'on apporta et jeta le tribut dans le coffre, jusqu'à ce qu'il fût plein.
\VS{11}Au moment venu, les Lévites apportaient le coffre aux inspecteurs du roi, car ceux-ci voyaient qu'il y avait beaucoup d'argent. Un secrétaire du roi et un commissaire du grand-prêtre venaient et vidaient le coffre~; puis ils le rapportaient et le remettaient à sa place. Ils faisaient ainsi jour après jour, et ils recueillaient de l'argent en abondance.
\VS{12}Le roi et Jehojada le donnaient à ceux qui étaient chargés de l'ouvrage pour le service de la maison de Yahweh, et ceux-ci engageaient des tailleurs de pierres et des charpentiers pour réparer la maison de Yahweh, et aussi des ouvriers pour le fer et l'airain, afin de réparer la maison de Yahweh.
\VS{13}Ceux qui étaient chargés de l'ouvrage travaillèrent donc~; et par leurs mains, les travaux s'exécutèrent, de sorte qu'ils rétablirent la maison de Dieu en son état, et l'affermirent.
\VS{14}Lorsqu'ils eurent achevé, ils apportèrent devant le roi et devant Jehojada le reste de l'argent~; et on fit faire des ustensiles pour la maison de Yahweh, des ustensiles pour le service et pour les holocaustes, des coupes et d'autres ustensiles d'or et d'argent. Et on offrit continuellement des holocaustes dans la maison de Yahweh, tant que vécut Jehojada.
\TextTitle{Mort de Jehojada, Joas abandonne Yahweh\FTNTT{2 R. 12:9-16.}}
\VS{15}Or Jehojada devint vieux et rassasié de jours, et il mourut~; il était âgé de cent trente ans quand il mourut.
\VS{16}On l'ensevelit dans la cité de David avec les rois~; car il avait fait du bien à Israël, et à l'égard de Dieu et de sa maison.
\VS{17}Mais, après la mort de Jehojada, les chefs de Juda vinrent et se prosternèrent devant le roi~; et le roi les écouta.
\VS{18}Ils abandonnèrent la maison de Yahweh, le Dieu de leurs pères, et ils servirent les Asherah et les faux dieux~; et la colère de Yahweh fut sur Juda et sur Jérusalem, parce qu'ils s'étaient ainsi rendus coupables.
\VS{19}Yahweh envoya parmi eux des prophètes, pour les faire retourner à lui par leurs avertissements~; mais ils ne voulurent point les écouter.
\VS{20}Alors l'Esprit de Dieu revêtit Zacharie, fils de Jehojada, le prêtre, et se tenant devant le peuple, il leur dit~: Dieu m'a parlé ainsi~: Pourquoi transgressez-vous les commandements de Yahweh~? Vous ne prospérerez point~; car vous avez abandonné Yahweh, et il vous abandonnera aussi.
\VS{21}Mais ils se liguèrent contre lui et le lapidèrent, par ordre du roi, dans le parvis de la maison de Yahweh.
\VS{22}Ainsi le roi Joas ne se souvint point de la bonté dont Jehojada, père de Zacharie, avait usé envers lui~; et il tua son fils, qui dit en mourant~: Yahweh le voit, et il en demandera compte~!
\TextTitle{Invasion des Syriens, conspiration et mort de Joas\FTNTT{2 R. 12:17-21~; cp. 2 R. 13:7.}}
\VS{23}A la fin de cette année-là, l'armée de Syrie monta contre Joas, et vint en Juda et à Jérusalem. Ils détruisirent, d'entre le peuple, tous les chefs du peuple, et ils envoyèrent au roi de Damas tout leur butin.
\VS{24}Et quoique l'armée venue de Syrie fût peu nombreuse, Yahweh livra entre leurs mains une armée très nombreuse, parce qu'ils avaient abandonné Yahweh, le Dieu de leurs pères. Ainsi les Syriens firent justice de Joas.
\VS{25}Quand ils s'éloignèrent de lui, après l'avoir laissé dans de grandes souffrances, ses serviteurs conspirèrent contre lui, à cause du sang des fils de Jehojada, le prêtre~; ils le tuèrent sur son lit, et il mourut. On l'ensevelit dans la cité de David, mais on ne l'ensevelit pas dans les sépulcres des rois.
\VS{26}Ce sont ici ceux qui conspirèrent contre lui~: Zabad, fils de Schimeath, femme Ammonite, et Jozabad, fils de Schimrith, femme Moabite.
\VS{27}Quant à ses fils et à la grande charge qui reposa sur lui, et à la réparation de la maison de Dieu, voici, ces choses sont écrites dans les mémoires du livre des rois. Amatsia, son fils, régna à sa place.
\Chap{25}
\TextTitle{Amatsia règne sur Juda\FTNTT{2 R. 12:21~; 14:1-6.}}
\VerseOne{}Amatsia devint roi à l'âge de vingt-cinq ans, et il régna vingt-neuf ans à Jérusalem. Sa mère avait pour nom Joaddan, de Jérusalem.
\VS{2}Il fit ce qui est droit aux yeux de Yahweh, mais non d'un cœur entier.
\VS{3}Après qu'il fut affermi dans son règne, il fit mourir ses serviteurs qui avaient tué le roi, son père.
\VS{4}Mais il ne fit pas mourir leurs fils~; car il fit selon ce qui est écrit dans la loi, dans le livre de Moïse, où Yahweh a donné ce commandement~: Les pères ne mourront point pour les fils, et les fils ne mourront point pour les pères~; mais chacun mourra pour son péché.
\TextTitle{Amatsia en guerre contre les Edomites, sa victoire\FTNTT{2 R. 14:7.}}
\VS{5}Puis Amatsia rassembla ceux de Juda, et il les rangea selon les familles des pères, par chefs de milliers et par chefs de centaines, pour tout Juda et Benjamin~; il en fit le dénombrement depuis l'âge de vingt ans et au-dessus~; et il trouva trois cent mille hommes d'élite, propres à l'armée, maniant la lance et le bouclier.
\VS{6}Il prit aussi à sa solde, pour cent talents d'argent, cent mille vaillants hommes de guerre d'Israël.
\VS{7}Mais un homme de Dieu vint à lui, et lui dit~: Ô roi~! Que l'armée d'Israël ne marche point avec toi~; car Yahweh n'est point avec Israël ni avec tous ces fils d'Ephraïm.
\VS{8}Si tu vas avec eux, quand bien même tu ferais de vaillants combats, Dieu te fera tomber devant l'ennemi~; car Dieu a la puissance d'aider et de faire tomber.
\VS{9}Amatsia dit à l'homme de Dieu~: Mais que faire des cent talents que j'ai donnés à la troupe d'Israël~? L'homme de Dieu dit~: Yahweh peut t'en donner beaucoup plus.
\VS{10}Ainsi Amatsia sépara les troupes qui lui étaient venues d'Ephraïm, et les fit retourner chez elles~; mais leur colère s'enflamma très ardemment contre Juda, et ces gens retournèrent chez eux dans une grande colère.
\VS{11}Alors Amatsia prit courage, conduisit son peuple et s'en alla dans la vallée du sel, où il battit dix mille hommes des fils de Séir.
\VS{12}Les fils de Juda prirent dix mille hommes vivants, et les ayant amenés sur le sommet d'une roche, ils les jetèrent du haut de la roche, de sorte qu'ils furent tous brisés.
\VS{13}Mais les gens de la troupe qu'Amatsia avait renvoyée, afin qu'ils n'aillent pas avec lui à la guerre, firent une incursion dans les villes de Juda, depuis Samarie jusqu'à Beth-Horon. Ils y tuèrent trois mille personnes et emportèrent un gros butin.
\TextTitle{Idolâtrie d'Amatsia\FTNTT{2 R. 14:7.}}
\VS{14}Lorsqu'Amatsia fut de retour de la défaite des Edomites, et ayant apporté les dieux des fils de Séir, il se les établit pour dieux~; il se prosterna devant eux et leur brûla de l'encens.
\VS{15}Et la colère de Yahweh s'enflamma contre Amatsia, et il envoya vers lui un prophète qui lui dit~: Pourquoi as-tu recherché les dieux d'un peuple qui n'ont point délivré leur peuple de ta main~?
\VS{16}Et comme il parlait au roi, il lui répondit~: T'a-t-on établi conseiller du roi~? Cesse maintenant~! Pourquoi veux-tu qu'on te tue~? Et le prophète se retira, mais en disant~: Je sais que Dieu a résolu de te détruire, parce que tu as fait cela, et que tu n'as point écouté mon conseil.
\TextTitle{Défaite d'Amatsia contre Israël\FTNTT{2 R. 14:8-14.}}
\VS{17}Puis Amatsia, roi de Juda, ayant tenu conseil, envoya vers Joas, fils de Joachaz, fils de Jéhu, roi d'Israël, pour lui dire~: Viens, voyons-nous en face~!
\VS{18}Mais Joas, roi d'Israël, envoya dire à Amatsia, roi de Juda~: L'épine du Liban envoya dire au cèdre du Liban~: Donne ta fille pour femme à mon fils~! Et les bêtes sauvages qui sont au Liban passèrent et foulèrent l'épine.
\VS{19}Voici, tu dis que tu as frappé les Edomites, et ton cœur s'est élevé pour te glorifier. Maintenant, reste dans ta maison~! Pourquoi t'engagerais-tu dans un combat où tu tomberais, et Juda avec toi~?
\VS{20}Mais Amatsia ne l'écouta point~; Dieu avait résolu de le livrer aux mains de Joas parce qu'il eût recours aux dieux d'Edom.
\VS{21}Joas, roi d'Israël, monta~; et ils se virent en face, lui et Amatsia, roi de Juda, à Beth-Schémesch, qui est de Juda.
\VS{22}Juda fut battu en face d'Israël, et chacun s'enfuit dans sa tente.
\VS{23}Joas, roi d'Israël, prit Amatsia, roi de Juda, fils de Joas, fils de Joachaz, à Beth-Schémesch. Il l'emmena à Jérusalem et fit une brèche de quatre cents coudées dans la muraille de Jérusalem, depuis la porte d'Ephraïm jusqu'à la porte de l'angle.
\VS{24}Il prit l'or, l'argent, tous les vases qui se trouvaient dans la maison de Dieu sous la garde d'Obed-Edom, les trésors de la maison du roi~; il fit des otages et il retourna à Samarie.
\TextTitle{Assassinat d'Amatsia\FTNTT{2 R. 14:17-20.}}
\VS{25}Amatsia, fils de Joas, roi de Juda, vécut quinze ans, après que Joas, fils de Joachaz, roi d'Israël, mourut.
\VS{26}Le reste des actions d'Amatsia, les premières et les dernières, voici cela n'est-il pas écrit dans le livre des rois de Juda et d'Israël~?
\VS{27}Or depuis le moment où Amatsia se détourna de Yahweh, on fit une conspiration contre lui à Jérusalem, et il s'enfuit à Lakis~; mais on le poursuivit à Lakis, et on le fit mourir.
\VS{28}Puis on le transporta sur des chevaux, et on l'ensevelit avec ses pères dans la ville de Juda.
\Chap{26}
\TextTitle{Ozias règne sur Juda~; il est fidèle à Yahweh\FTNTT{2 R. 14:21-15:4.}}
\VerseOne{}Alors, tout le peuple de Juda prit Ozias, âgé de seize ans, et l'établit roi à la place de son père Amatsia.
\VS{2}Ce fut lui qui bâtit Eloth, et la ramena sous la puissance de Juda, après que le roi se fut endormi avec ses pères.
\VS{3}Ozias était âgé de seize ans quand il devint roi, et il régna cinquante-deux ans à Jérusalem. Sa mère avait pour nom Jecolia, de Jérusalem.
\VS{4}Il fit ce qui est droit aux yeux de Yahweh, comme avait fait Amatsia, son père.
\VS{5}Il s'appliqua à rechercher Dieu pendant les jours de Zacharie, qui avait une intelligence dans les visions de Dieu et pendant les jours où il rechercha Yahweh, Dieu le fit prospérer.
\VS{6}Il sortit et fit la guerre contre les Philistins. Il brisa la muraille de Gath, la muraille de Jabné, et la muraille d'Asdod~; et il bâtit des villes dans le pays d'Asdod et chez les Philistins.
\VS{7}Dieu le secourut contre les Philistins et contre les Arabes qui habitaient à Gur-Baal, et contre les Maonites.
\VS{8}Même les Ammonites faisaient des présents à Ozias, et sa renommée parvint jusqu'à l'entrée de l'Egypte~; car il était devenu très puissant.
\VS{9}Ozias bâtit des tours à Jérusalem, sur la porte de l'angle, sur la porte de la vallée, sur l'angle, et il les fortifia.
\VS{10}Il bâtit des tours dans le désert, et il creusa de nombreux puits, parce qu'il avait de nombreux troupeaux dans la plaine et dans la campagne, des laboureurs et des vignerons sur les montagnes, et au Carmel~; car il aimait l'agriculture.
\VS{11}Ozias avait une armée de gens de guerre, allant à la guerre par bandes, selon le compte de leur dénombrement fait par Jeïel le scribe, et Maaséja le commissaire, et sous la conduite de Hanania l'un des chefs du roi.
\VS{12}Le nombre total des chefs des pères, des vaillants guerriers, était de deux mille six cents.
\VS{13}Il y avait sous leur conduite une armée de trois cent sept mille cinq cents combattants, tous gens de guerre, puissants et vaillants, capables de soutenir le roi contre l'ennemi.
\VS{14}Ozias leur procura, pour toute l'armée, des boucliers, des lances, des casques, des cuirasses, des arcs et des pierres de fronde.
\VS{15}Il fit faire à Jérusalem des machines inventées par un ingénieur, pour être placées sur les tours et sur les angles, pour lancer des flèches et de grosses pierres. Et sa renommée se répandit au loin~; car il fut extraordinairement soutenu, jusqu'à ce qu'il devienne fort puissant.
\TextTitle{Ozias pèche et est frappé de lèpre\FTNTT{2 R. 15:5-7,32.}}
\VS{16}Mais dès qu'il fut puissant, son cœur s'éleva pour le corrompre. Et il pécha contre Yahweh, son Dieu~: Il entra dans le temple de Yahweh pour brûler des parfums sur l'autel des parfums.
\VS{17}Mais Azaria le prêtre, entra après lui, et avec lui quatre-vingts prêtres de Yahweh, hommes vaillants,
\VS{18}qui s'opposèrent au roi Ozias, et lui dirent~: Ce n'est pas à toi, Ozias, d'offrir le parfum à Yahweh, mais aux prêtres, fils d'Aaron, qui sont consacrés pour cela. Sors du sanctuaire, car tu as péché~! Et cela ne sera pas à ta gloire devant Yahweh Dieu.
\VS{19}Alors Ozias, qui avait à la main un encensoir pour faire brûler le parfum, se mit en colère. Et comme il s'irritait contre les prêtres, la lèpre parut sur son front, en présence des prêtres, dans la maison de Yahweh, près de l'autel des parfums.
\VS{20}Azaria, le principal prêtre, le regarda ainsi que tous les prêtres. Et voici, il avait de la lèpre sur le front. Ils le pressèrent et lui-même se hâta de sortir, parce que Yahweh l'avait frappé.
\VS{21}Le roi Ozias fut ainsi lépreux jusqu'au jour de sa mort~; il habita seul comme lépreux dans une maison écartée, car il était exclu de la maison de Yahweh. Et Jotham, son fils, avait la charge de la maison du roi, jugeant le peuple du pays.
\VS{22}Esaïe, fils d'Amots, le prophète, a écrit le reste des actions d'Ozias, les premières et les dernières.
\VS{23}Ozias s'endormit avec ses pères, et on l'ensevelit avec ses pères dans le champ de la sépulture des rois~; car on dit~: Il est lépreux. Et Jotham, son fils, régna à sa place.
\Chap{27}
\TextTitle{Jotham règne sur Juda~; sa mort\FTNTT{2 R. 15:7,32-38.}}
\VerseOne{}Jotham était âgé de vingt-cinq ans quand il devint roi, et il régna seize ans à Jérusalem. Sa mère avait le nom de Jeruscha, fille de Tsadok.
\VS{2}Il fit ce qui est droit aux yeux de Yahweh, tout comme Ozias, son père, avait fait~; mais il n'entra pas dans le temple de Yahweh. Néanmoins, le peuple se corrompait encore.
\VS{3}Ce fut lui qui bâtit la porte supérieure de la maison de Yahweh, et il fit beaucoup de constructions sur les murs de la colline.
\VS{4}Il bâtit des villes sur les montagnes de Juda, des châteaux et des tours dans les forêts.
\VS{5}Il fut en guerre avec le roi des fils d'Ammon, et fut le plus fort. Cette année-là, les fils d'Ammon lui donnèrent cent talents d'argent, dix mille cors de froment, et dix mille d'orge. Les fils d'Ammon lui en donnèrent autant la seconde et la troisième année.
\VS{6}Jotham devint donc très puissant, parce qu'il avait affermi ses voies devant Yahweh, son Dieu.
\VS{7}Le reste des actions de Jotham, tous ses combats et sa conduite, voici, toutes ces choses sont écrites dans le livre des rois d'Israël et de Juda.
\VS{8}Il était âgé de vingt-cinq ans quand il devint roi, et il régna seize ans à Jérusalem.
\VS{9}Puis Jotham s'endormit avec ses pères, et on l'ensevelit dans la cité de David. Et Achaz, son fils, régna à sa place.
\Chap{28}
\TextTitle{Achaz règne sur Juda\FTNTT{2 R. 15:38-16:4.}}
\VerseOne{}Achaz était âgé de vingt ans quand il devint roi, et il régna seize ans à Jérusalem. Il ne fit point ce qui est droit aux yeux de Yahweh, comme David, son père.
\VS{2}Il suivit la voie des rois d'Israël~; et il fit même des images de fonte pour les Baals.
\VS{3}Il brûla des parfums dans la vallée du fils de Hinnom, et il brûla ses fils au feu, suivant les abominations des nations que Yahweh avait chassées devant les enfants d'Israël.
\VS{4}Il offrait aussi des sacrifices et brûlait des parfums dans les hauts lieux, sur les collines, et sous tout arbre vert.
\TextTitle{La Syrie et Israël envahissent Juda\FTNTT{2 R. 16:5-6.}}
\VS{5}C'est pourquoi Yahweh, son Dieu, le livra entre les mains du roi de Syrie. Les Syriens le battirent et lui prirent un grand nombre de prisonniers, qu'ils emmenèrent à Damas. Il fut livré aussi entre les mains du roi d'Israël, qui lui fit endurer une grande défaite.
\VS{6}Car Pékach, fils de Remalia, tua en un seul jour en Juda cent vingt mille hommes, tous vaillants, parce qu'ils avaient abandonné Yahweh, le Dieu de leurs pères.
\VS{7}Zicri, homme vaillant d'Ephraïm, tua Maaséja, fils du roi, et Azrikam, chef de la maison, et Elkana, le second après le roi.
\VS{8}Les fils d'Israël emmenèrent prisonniers deux cent mille de leurs frères, tant femmes que fils et filles~; ils firent aussi sur eux un gros butin. Ils emmenèrent le butin à Samarie.
\TextTitle{Les captifs de Juda libérés grâce à Oded}
\VS{9}Or il y avait un prophète de Yahweh nommé Oded. Il sortit au-devant de cette armée qui revenait à Samarie, et leur dit~: Voici, Yahweh, le Dieu de vos pères, étant indigné contre Juda, les a livrés entre vos mains, et vous les avez tués avec une colère telle qu'elle est parvenue aux cieux.
\VS{10}Et maintenant, vous pensez assujettir les fils de Juda et de Jérusalem pour serviteurs et pour servantes~! Mais n'êtes-vous pas également coupables envers Yahweh, votre Dieu~?
\VS{11}Maintenant écoutez-moi, et ramenez les prisonniers que vous vous êtes faits parmi vos frères~; car la colère ardente de Yahweh est sur vous.
\VS{12}Alors quelques-uns des chefs des fils d'Ephraïm, Azaria, fils de Jochanan, Bérékia, fils de Meschillémoth, Ezéchias, fils de Schallum, et Amasa, fils de Hadlaï, s'élevèrent contre ceux qui retournaient de la guerre,
\VS{13}et leur dirent~: Vous ne ferez point entrer ici ces captifs. C'est pour nous rendre coupables devant Yahweh, voulez-vous en rajouter à nos péchés et à notre culpabilité~; car nous sommes déjà grandement coupables, et une colère ardente est sur Israël.
\VS{14}Alors les soldats abandonnèrent les captifs et le butin devant les chefs et toute l'assemblée.
\VS{15}Et des hommes, désignés par leurs noms, se levèrent, prirent les captifs, utilisèrent le butin pour revêtir tous ceux d'entre eux qui étaient nus avec des vêtements et des chaussures. Ils leur donnèrent à manger et à boire, les oignirent et ils conduisirent sur des ânes tous ceux qui étaient affaiblis pour les emmener à Jéricho, la ville des palmiers, auprès de leurs frères~; puis ils s'en retournèrent à Samarie.
\TextTitle{Achaz fait appel aux Assyriens\FTNTT{2 R. 15:29~; 16:7-18.}}
\VS{16}En ce temps-là, le roi Achaz envoya demander du secours aux rois d'Assyrie.
\VS{17}Les Edomites étaient revenus, avaient battu Juda et avaient emmené des prisonniers.
\VS{18}Les Philistins s'étaient aussi jetés sur les villes de la plaine et du sud de Juda~; et ils avaient pris Beth-Schémesch, Ajalon, Guedéroth, Soco et les villes de son ressort, Thimna et les villes de son ressort, Guimzo et les villes de son ressort, et ils y demeurèrent.
\VS{19}Car Yahweh humilia Juda, à cause d'Achaz, roi d'Israël, parce qu'il avait mis le désordre en Juda, et qu'il avait commis des transgressions contre Yahweh.
\VS{20}Tilgath-Pilnéser, roi d'Assyrie, vint vers lui~; mais il l'assiégea, et ne le fortifia pas.
\VS{21}Or Achaz dépouilla la maison de Yahweh, la maison du roi et celle des chefs, pour faire des dons au roi d'Assyrie, mais sans avoir du secours.
\TextTitle{Achaz irrite Yahweh par ses péchés}
\VS{22}Dans le temps de sa détresse, il continua à pécher contre Yahweh, lui, le roi Achaz.
\VS{23}Il sacrifia aux dieux de Damas qui l'avaient battu, et il dit~: Puisque les dieux des rois de Syrie leur viennent en aide, je leur sacrifierai, afin qu'ils me viennent en aide. Mais ils furent la cause de sa chute et de celle de tout Israël.
\VS{24}Or Achaz rassembla les ustensiles de la maison de Dieu, et il mit en pièces les ustensiles de la maison de Dieu. Il ferma les portes de la maison de Yahweh, et se fit des autels dans tous les coins de Jérusalem.
\VS{25}Il fit des hauts lieux dans chaque ville de Juda, pour offrir des parfums à d'autres dieux~; et il irrita Yahweh, le Dieu de ses pères.
\TextTitle{Mort d'Achaz\FTNTT{2 R. 16:19-20.}}
\VS{26}Quant au reste de ses actions et de toutes ses voies, les premières et les dernières, voici, elles sont écrites dans le livre des rois de Juda et d'Israël.
\VS{27}Puis Achaz s'endormit avec ses pères, et on l'ensevelit dans la ville de Jérusalem~; car on ne le mit point dans les sépulcres des rois d'Israël. Et Ezéchias, son fils, régna à sa place.
\Chap{29}
\TextTitle{Ezéchias règne sur Juda~; le réveil du peuple\FTNTT{2 R. 18:1-7~; cp. Es. 36-39.}}
\VerseOne{}Ezéchias devint roi à l'âge de vingt-cinq ans, et il régna vingt-neuf ans à Jérusalem. Sa mère avait pour nom Abija, fille de Zacharie.
\VS{2}Il fit ce qui est droit aux yeux de Yahweh, tout comme avait fait David, son père.
\VS{3}La première année de son règne, au premier mois, il ouvrit les portes de la maison de Yahweh, et il les répara.
\VS{4}Il fit venir les prêtres et les Lévites, et les rassembla dans la place orientale.
\VS{5}Et il leur dit~: Ecoutez-moi, Lévites~! Sanctifiez-vous et sanctifiez la maison de Yahweh, le Dieu de vos pères, et ôtez du sanctuaire tout ce qui est impur.
\VS{6}Car nos pères ont péché, ils ont fait ce qui est mal aux yeux de Yahweh, notre Dieu. Ils l'ont abandonné, ils ont détourné leurs faces du tabernacle de Yahweh et lui ont tourné le dos.
\VS{7}Ils ont même fermé les portes du portique et ont éteint les lampes, ils n'ont fait ni monter d'offrande, ni brûler du parfum et des holocaustes au Dieu d'Israël dans le sanctuaire.
\VS{8}C'est pourquoi la colère de Yahweh a été sur Juda et sur Jérusalem~; et il les a livrés à de grands troubles, à la ruine et à la moquerie, comme vous le voyez de vos yeux.
\VS{9}Car voici, nos pères sont tombés par l'épée, et nos fils, nos filles et nos femmes sont en captivité.
\VS{10}Maintenant donc j'ai à cœur de traiter alliance avec Yahweh, le Dieu d'Israël, pour que son ardente colère se détourne de nous.
\VS{11}Or mes fils, cessez d'être négligents~; car Yahweh vous a choisis, afin que vous vous teniez devant lui à son service, comme ses serviteurs, pour lui brûler des parfums.
\VS{12}Les Lévites se levèrent~: Machath, fils d'Amasaï, Joël, fils d'Azaria, des fils des Kehathites~; et des fils des Merarites, Kis, fils d'Abdi, Azaria, fils de Jehalléleel~; et des Guerschonites, Joach, fils de Zimma, et Eden, fils de Joach~;
\VS{13}et des fils d'Elitsaphan, Schimri et Jeïel~; et des fils d'Asaph, Zacharie et Matthania~;
\VS{14}et des fils d'Héman, Jehiel et Schimeï, et des fils de Jeduthun, Schemaeja et Uzziel.
\VS{15}Ils assemblèrent leurs frères, et ils se sanctifièrent~; puis ils entrèrent selon l'ordre du roi, et d'après la parole de Yahweh, pour purifier la maison de Yahweh.
\VS{16}Ainsi les prêtres entrèrent à l'intérieur de la maison de Yahweh pour la purifier. Ils firent sortir dans le parvis de la maison de Yahweh toutes les impuretés qu'ils trouvèrent dans le temple de Yahweh. Les Lévites les prirent pour les emporter dehors, au torrent de Cédron.
\VS{17}Ils commencèrent à sanctifier le temple le premier jour du premier mois. Le huitième jour du mois, ils entrèrent au portique de Yahweh, et ils sanctifièrent la maison de Yahweh pendant huit jours. Le seizième jour du premier mois, ils avaient achevé.
\VS{18}Puis ils se rendirent chez le roi Ezéchias, et dirent~: Nous avons purifié toute la maison de Yahweh, l'autel des holocaustes et ses ustensiles, la table des pains de proposition et ses ustensiles\FTNT{Ex. 29.}.
\VS{19}Nous avons remis en état et sanctifié tous les ustensiles que le roi Achaz avait rendus odieux pendant son règne, par ses transgressions~; ils sont maintenant devant l'autel de Yahweh.
\TextTitle{Nouvelle consécration du temple}
\VS{20}Alors le roi Ezéchias se leva de bonne heure, rassembla les chefs de la ville, et monta à la maison de Yahweh.
\VS{21}Ils amenèrent sept taureaux, sept béliers, sept agneaux et sept boucs sans défaut, en sacrifice pour le péché, pour le royaume, pour le sanctuaire et pour Juda\FTNT{Lé 4:3-26.}. Puis le roi dit aux prêtres, fils d'Aaron, de les faire monter en offrande sur l'autel de Yahweh.
\VS{22}Ils égorgèrent donc les bœufs, et les prêtres recueillirent le sang et en aspergèrent l'autel~; ils égorgèrent les béliers et aspergèrent le sang sur l'autel~; ils égorgèrent les agneaux et aspergèrent le sang sur l'autel.
\VS{23}Puis on fit approcher les boucs pour le sacrifice du péché, devant le roi et devant l'assemblée, et ils posèrent leurs mains sur eux\FTNT{Lé 8:14.}.
\VS{24}Alors les prêtres les égorgèrent, et offrirent en expiation leur sang vers l'autel, afin de faire propitiation pour tout Israël~; car le roi avait ordonné cet holocauste et ce sacrifice d'expiation pour tout Israël.
\VS{25}Il plaça aussi les Lévites dans la maison de Yahweh, avec des cymbales, des luths et des harpes, comme l'avait ordonné David, Gad, le voyant du roi, et Nathan le prophète~; car c'était un commandement de Yahweh, par ses prophètes.
\VS{26}Les Lévites se tinrent donc là avec les instruments de David, et les prêtres avec les trompettes.
\VS{27}Alors Ezéchias ordonna de faire monter en offrande l'holocauste sur l'autel~; et au moment où commença l'holocauste, le cantique de Yahweh commença aussi, avec les trompettes et les instruments de David, roi d'Israël.
\VS{28}Toute l'assemblée se prosterna en chantant le cantique, et les trompettes sonnèrent~; et cela continua jusqu'à ce que l'holocauste fût achevé.
\VS{29}Et quand on eut achevé de faire monter l'holocauste, le roi et tous ceux qui se trouvaient avec lui fléchirent les genoux et se prosternèrent.
\VS{30}Puis le roi Ezéchias et les chefs dirent aux Lévites de célébrer Yahweh par les paroles de David et d'Asaph le voyant~; et ils le célébrèrent dans des réjouissances, et s'inclinèrent pour se prosterner.
\VS{31}Alors Ezéchias prit la parole, et dit~: Vous avez maintenant consacré vos mains à Yahweh. Approchez-vous, amenez des sacrifices et faites des sacrifices de reconnaissance dans la maison de Yahweh. Et l'assemblée amena des sacrifices et firent des sacrifices de reconnaissance, et tous ceux qui étaient d'un cœur volontaire offrirent des holocaustes.
\VS{32}Le nombre des holocaustes que l'assemblée offrit fut de soixante-dix taureaux, cent béliers, deux cents agneaux, le tout en holocauste à Yahweh.
\VS{33}Et les autres choses consacrées furent, six cents bœufs, et trois mille brebis.
\VS{34}Mais ils étaient peu de prêtres et ne purent pas dépouiller tous les holocaustes~; les Lévites, leurs frères, les aidèrent jusqu'à ce que cette œuvre fut achevée, et jusqu'à ce que les autres prêtres se fussent sanctifiés~; car les Lévites avaient eu plus à cœur de se sanctifier que les prêtres.
\VS{35}Il y eut aussi un grand nombre d'holocaustes, avec les graisses des offrandes de paix et avec les libations des holocaustes. Ainsi, le service de la maison de Yahweh fut rétabli.
\VS{36}Ezéchias et tout le peuple se réjouirent de ce que Dieu avait ainsi disposé le peuple~; car les choses se firent instantanément.
\Chap{30}
\TextTitle{Rétablissement de la Pâque}
\VerseOne{}Puis Ezéchias envoya dire à tout Israël et à Juda~; et il écrivit aussi des lettres à Ephraïm et à Manassé, pour les faire venir à la maison de Yahweh à Jérusalem, pour célébrer la Pâque en l'honneur de Yahweh, le Dieu d'Israël.
\VS{2}Le roi, ses chefs et toute l'assemblée avaient tenu un conseil à Jérusalem afin de célébrer la Pâque au second mois\FTNT{No. 9:10-11.}~;
\VS{3}car on ne pouvait la célébrer au temps ordinaire, parce qu'il n'y avait pas un nombre suffisant de prêtres sanctifiés, et que le peuple n'était pas rassemblé à Jérusalem.
\VS{4}Le roi vit cela d'un bon œil ainsi que toute l'assemblée.
\VS{5}Ils décidèrent de faire une publication dans tout Israël, depuis Beer-Schéba jusqu'à Dan, pour que l'on vienne à Jérusalem célébrer la Pâque à Yahweh, le Dieu d'Israël. Car elle n'était pas célébrée par la multitude depuis longtemps conformément à ce qui était écrit.
\VS{6}Les coureurs allèrent donc avec des lettres de la part du roi et de ses chefs, partout en Israël et en Juda. Selon que le roi l'avait ordonné, ils disaient~: Enfants d'Israël, retournez à Yahweh, le Dieu d'Abraham, d'Isaac et d'Israël, et afin qu'il revienne vers vous, qui êtes le reste échappé de la main des rois d'Assyrie.
\VS{7}Ne soyez pas comme vos pères ni comme vos frères, qui ont péché contre Yahweh, le Dieu de leurs pères, c'est pourquoi il les a livrés à la désolation, comme vous le voyez.
\VS{8}Maintenant, ne raidissez pas votre cou comme vos pères. Tendez les mains vers Yahweh, venez à son sanctuaire consacré pour toujours, servez Yahweh, votre Dieu, et son ardente colère se détournera de vous.
\VS{9}Car si vous revenez à Yahweh, vos frères et vos fils trouveront grâce auprès de ceux qui les ont emmenés captifs, et ils reviendront en ce pays parce que Yahweh, votre Dieu, est compatissant et miséricordieux~; et il ne détournera point sa face de vous, si vous revenez à lui.
\VS{10}Les coureurs passaient ainsi de ville en ville, par le pays d'Ephraïm et de Manassé jusqu'à Zabulon~; mais on riait et on se moquait d'eux.
\VS{11}Toutefois, quelques-uns d'Aser, de Manassé et de Zabulon s'humilièrent, et vinrent à Jérusalem.
\VS{12}La main de Dieu fut aussi sur Juda, pour leur donner un même cœur, afin d'exécuter l'ordre du roi et des chefs, selon la parole de Yahweh.
\VS{13}C'est pourquoi il s'assembla un grand peuple à Jérusalem pour célébrer la fête des pains sans levain\FTNT{Ex. 12:15~; Lé. 23:6.}, au second mois. Ce fut une très grande assemblée.
\VS{14}Ils se levèrent et ôtèrent les autels qui étaient à Jérusalem~; ils ôtèrent aussi tous ceux où l'on brûlait de l'encens, et ils les jetèrent dans le torrent de Cédron.
\VS{15}Puis on immola la Pâque, au quatorzième jour du second mois~; car les prêtres et les Lévites avaient eu honte et s'étaient sanctifiés, et ils amenèrent les holocaustes dans la maison de Yahweh.
\VS{16}Ils se tinrent à leur poste, selon leur charge, d'après la loi de Moïse, homme de Dieu. Et les prêtres répandaient le sang qu'ils recevaient des mains des Lévites.
\VS{17}Car il y en avait un grand nombre dans cette assemblée qui ne s'étaient pas sanctifiés~; c'est pourquoi les Lévites eurent la charge d'immoler la Pâque pour tous ceux qui n'étaient pas purs, afin de les consacrer à Yahweh.
\VS{18}Car une grande partie du peuple, à savoir la plupart de ceux d'Ephraïm, de Manassé, d'Issacar et de Zabulon, ne s'étaient pas purifiés et mangèrent la Pâque contrairement à ce qui est écrit. Mais Ezéchias pria pour eux, en disant~: Que Yahweh, qui est bon, tienne la propitiation pour faite,
\VS{19}pour quiconque a disposé son cœur à rechercher Dieu, Yahweh, le Dieu de leurs pères, bien qu'il ne soit pas purifié conformément au sanctuaire~!
\VS{20}Yahweh exauça Ezéchias, et guérit le peuple.
\VS{21}Les enfants d'Israël qui se trouvèrent à Jérusalem célébrèrent donc la fête des pains sans levain, pendant sept jours, dans une grande réjouissance~; les Lévites et les prêtres célébraient Yahweh chaque jour, avec les instruments qui retentissaient à la louange de Yahweh.
\VS{22}Ezéchias parla au cœur de tous les Lévites, qui prêtaient une grande attention et de l'intelligence au service de Yahweh. Ils mangèrent pendant la fête, sept jours durant, offrant des sacrifices d'offrande de paix, et louant Yahweh, le Dieu de leurs pères.
\TextTitle{Sept jours supplémentaires pour la Pâque}
\VS{23}Puis toute l'assemblée fut d'avis de célébrer sept autres jours. Et ils célébrèrent ces sept jours dans la joie.
\VS{24}Car Ezéchias, roi de Juda, offrit à l'assemblée mille taureaux et sept mille brebis~; et les chefs donnèrent à l'assemblée mille taureaux et dix mille brebis~; et des prêtres en grand nombre s'étaient sanctifiés.
\VS{25}Toute l'assemblée de Juda, avec les prêtres et les Lévites, et toute l'assemblée venue d'Israël, ainsi que les étrangers venus du pays d'Israël, et ceux qui habitaient en Juda, se réjouirent.
\VS{26}Il y eut une grande joie à Jérusalem~; car depuis le temps de Salomon, fils de David, roi d'Israël, il ne s'était pas fait une telle chose dans Jérusalem.
\VS{27}Puis les prêtres et les Lévites se levèrent et bénirent le peuple, et leur voix fut entendue, leur prière parvint jusqu'aux cieux, jusqu'à la sainte demeure de Yahweh.
\Chap{31}
\TextTitle{Destruction des idoles et organisation des services du temple}
\VerseOne{}Lorsque tout cela fut achevé, tous ceux d'Israël qui s'étaient retrouvés là, allèrent dans les villes de Juda, et brisèrent les statues, abattirent les Asherah et renversèrent les hauts lieux et les autels, dans tout Juda et Benjamin, dans Ephraïm et Manassé, jusqu'à détruire tout\FTNT{2 R. 18:4.}. Puis tous les enfants d'Israël retournèrent dans leurs villes, chacun dans sa possession.
\VS{2}Et Ezéchias rétablit les classes des prêtres et des Lévites, selon leur partage, chacun suivant sa charge, tant les prêtres que les Lévites, pour les holocaustes et les offrandes de paix, pour faire le service, célébrer et chanter les louanges aux portes du camp de Yahweh.
\VS{3}Le roi donna une portion de ses biens pour les holocaustes, pour les holocaustes du matin et du soir, pour les holocaustes des sabbats, des nouvelles lunes et des fêtes, comme cela est écrit dans la loi de Yahweh.
\VS{4}Il dit au peuple, aux habitants de Jérusalem, de donner la portion des prêtres et des Lévites, afin de s'appliquer à la loi de Yahweh.
\VS{5}Dès que la chose fut publiée, les enfants d'Israël amenèrent en abondance les prémices du blé, du moût, de l'huile, du miel et de tous les produits des champs~; ils apportèrent les dîmes de tout, en abondance.
\VS{6}Les enfants d'Israël et de Juda, qui demeuraient dans les villes de Juda, apportèrent aussi les dîmes du gros et du menu bétail, et les dîmes des choses saintes, qui étaient consacrées à Yahweh, leur Dieu~; et ils les mirent par tas.
\VS{7}Ils commencèrent à former les tas au troisième mois, et ils les achevèrent au septième mois.
\VS{8}Alors Ezéchias et les chefs vinrent voir les tas, et ils bénirent Yahweh et son peuple d'Israël.
\VS{9}Ezéchias interrogea les prêtres et les Lévites au sujet de ces tas.
\VS{10}Le grand-prêtre Azaria, de la maison de Tsadok, lui répondit, et parla ainsi~: Depuis qu'on a commencé à apporter des offrandes à la maison de Yahweh, nous avons mangé et avons été rassasiés, et il est resté cette grande quantité~; car Yahweh a béni son peuple, et cette grande quantité est le reste.
\VS{11}Alors Ezéchias leur dit de préparer des chambres dans la maison de Yahweh~; et ils les préparèrent.
\VS{12}On y apporta fidèlement les offrandes et les dîmes, les choses consacrées. Conania, le Lévite, en eut l'intendance, et Schimeï, son frère, était son second.
\VS{13}Jehiel, Azazia, Nachath, Asaël, Jerimoth, Jozabad, Eliel, Jismakia, Machath, et Benaja, étaient commis sous l'autorité de Conania et de Schimeï, son frère, d'après l'indication du roi Ezéchias, et d'Azaria, chef de la maison de Dieu.
\VS{14}Koré, le Lévite, fils de Jimna, portier de l'orient, avait la charge des offrandes volontaires offertes à Dieu, pour distribuer l'offrande élevée à Yahweh, et les choses consacrées et saintes.
\VS{15}Il avait sous sa direction Eden, Minjamin, Josué, Schemaeja, Amaria, et Schecania, dans les villes des prêtres, pour distribuer fidèlement les portions à leurs frères, grands et petits, suivant leurs divisions,
\VS{16}à ceux qui étaient enregistrés comme mâles, depuis l'âge de trois ans et au-delà~; à tous ceux qui entraient dans la maison de Yahweh, pour le service quotidien, pour servir dans leurs charges et suivant leurs divisions~;
\VS{17}aux prêtres et aux Lévites enregistrés selon la maison de leurs pères, depuis ceux de vingt ans et au-delà, selon leurs charges et selon leurs divisions~;
\VS{18}à ceux de toute l'assemblée enregistrés avec leurs petits enfants, leurs femmes, leurs fils et leurs filles~; car ils se consacraient avec fidélité aux choses saintes~;
\VS{19}et pour les enfants d'Aaron, les prêtres, qui étaient à la campagne et dans les faubourgs de leurs villes, dans chaque ville, il y avait des gens désignés par leur nom, pour distribuer les portions à tous les mâles des prêtres, et à tous les Lévites enregistrés.
\VS{20}Ezéchias en fit ainsi dans tout Juda~; et il fit ce qui est bon, droit et véritable, devant Yahweh, son Dieu.
\VS{21}Il travailla de tout son cœur et il réussit dans tout l'ouvrage qu'il entreprit pour le service de la maison de Dieu, et pour la loi, et pour les commandements, en recherchant son Dieu.
\Chap{32}
\TextTitle{Menaces de Sanchérib, roi d'Assyrie\FTNTT{2 R. 19:17-37~; 19:8-13~; Es. 36:2-20.}}
\VerseOne{}Après que ces choses furent bien établies, Sanchérib, roi d'Assyrie, vint et entra en Juda, et campa contre les villes fortes, dans l'intention de faire une brèche.
\VS{2}Ezéchias, voyant que Sanchérib était venu, et qu'il se tournait vers Jérusalem pour lui faire la guerre,
\VS{3}tint conseil avec ses chefs et ses vaillants hommes pour boucher les sources d'eau qui étaient hors de la ville, et ils l'aidèrent.
\VS{4}Un peuple nombreux s'assembla, et ils bouchèrent toutes les sources et le torrent qui coule par le milieu de la contrée, en disant~: Pourquoi les rois d'Assyrie trouveraient-ils à leur venue de l'eau en abondance~?
\VS{5}Il se fortifia et rebâtit toute la muraille où il y avait une brèche, et l'éleva jusqu'aux tours~; il bâtit une autre muraille en dehors~; il répara Millo, dans la cité de David, et il fit faire beaucoup d'armes et de boucliers.
\VS{6}Il donna des chefs de guerre au peuple, les assembla auprès de lui sur la place de la porte de la ville, et parla à leur cœur, en disant~:
\VS{7}Fortifiez-vous, soyez forts~! Ne craignez point et ne soyez pas effrayés devant le roi d'Assyrie et toute la multitude qui est avec lui~; car avec nous il y a quelqu'un de plus puissant.
\VS{8}Avec lui est le bras de la chair, mais avec nous est Yahweh, notre Dieu, pour nous aider et pour combattre dans nos combats. Et le peuple s'appuya sur les paroles d'Ezéchias, roi de Juda.
\VS{9}Après cela, Sanchérib, roi d'Assyrie, pendant qu'il était devant Lakis, ayant avec lui toutes les forces de son royaume, envoya ses serviteurs à Jérusalem vers Ezéchias, roi de Juda, et vers tous ceux de Juda qui étaient à Jérusalem, pour leur dire~:
\VS{10}Ainsi parle Sanchérib, roi d'Assyrie~: Sur qui vous confiez-vous pour que vous restiez à Jérusalem pour y être assiégés~?
\VS{11}Ezéchias ne vous incite-t-il pas pour vous livrer à la mort, par la famine et par la soif, en vous disant~: Yahweh, notre Dieu, nous délivrera de la main du roi d'Assyrie~?
\VS{12}Cet Ezéchias n'a-t-il pas ôté les hauts lieux et les autels, et n'a-t-il pas ordonné à Juda et à Jérusalem~: Vous vous prosternerez devant un seul autel pour y brûler le parfum~?
\VS{13}Ne savez-vous pas ce que nous avons fait, moi et mes pères, à tous les peuples des autres pays~? Les dieux des nations de ces pays ont-ils pu de quelque manière que ce soit délivrer leur pays de ma main~?
\VS{14}Quel est celui de tous les dieux de ces nations, que mes pères ont entièrement détruites, qui ait pu délivrer son peuple de ma main, pour que votre Dieu puisse vous délivrer de ma main~?
\VS{15}Maintenant donc, qu'Ezéchias ne vous abuse point, et qu'il ne vous incite plus de cette manière, et ne le croyez pas~; car aucun dieu d'aucune nation ni d'aucun royaume n'a pu délivrer son peuple de ma main ni de la main de mes pères~; combien moins votre Dieu vous délivrerait-il de ma main~?
\VS{16}Ses serviteurs parlèrent encore contre Yahweh Dieu, et contre Ezéchias, son serviteur.
\VS{17}Il écrivit aussi une lettre pour blasphémer contre Yahweh, le Dieu d'Israël, en parlant ainsi~: Comme les dieux des nations des autres pays n'ont pu délivrer leur peuple de ma main, ainsi le Dieu d'Ezéchias ne pourra délivrer son peuple de ma main.
\VS{18}Et ses serviteurs crièrent à haute voix en langue judaïque, au peuple de Jérusalem qui était sur la muraille, pour les effrayer et les épouvanter, afin de prendre la ville.
\VS{19}Ils parlèrent du Dieu de Jérusalem comme des dieux des peuples de la terre, qui ne sont qu'un ouvrage de mains d'homme.
\TextTitle{Prière d'Ezéchias et exaucement de Yahweh\FTNTT{2 R. 19:14-37~; Es. 36:21-37:35.}}
\VS{20}Alors le roi Ezéchias, et Esaïe, le prophète, fils d'Amots, prièrent à ce sujet et crièrent vers les cieux.
\VS{21}Et Yahweh envoya un ange, dans le camp du roi d'Assyrie, qui extermina tous les vaillants hommes, les princes et les chefs, en sorte qu'il retourna dans son pays, dans la honte. Il entra dans la maison de son dieu~; et là, ceux qui étaient sortis de ses entrailles le firent tomber par l'épée.
\VS{22}C'est ainsi que Yahweh sauva Ezéchias et les habitants de Jérusalem de la main de Sanchérib, roi d'Assyrie, et de la main de tout homme, et il les protégea de toutes parts.
\VS{23}Plusieurs apportèrent des offrandes à Yahweh, à Jérusalem, et des choses précieuses à Ezéchias, roi de Juda, qui après cela fut élevé aux yeux de toutes les nations.
\TextTitle{Maladie et guérison d'Ezéchias\FTNTT{2 R. 20:1-11.}}
\VS{24}En ces jours-là, Ezéchias fut malade à en mourir, et il pria Yahweh, qui l'exauça et lui accorda un prodige.
\VS{25}Mais Ezéchias ne fut pas reconnaissant du bienfait qu'il avait reçu~; car son cœur s'éleva, et il y eut des maux contre lui, et contre Juda et Jérusalem.
\VS{26}Mais Ezéchias s'humilia de l'élévation de son cœur, lui et les habitants de Jérusalem, et la colère de Yahweh ne vint plus sur eux durant les jours d'Ezéchias.
\TextTitle{Fin du règne d'Ezéchias, sa mort\FTNTT{2 R. 20:12-21~; cp. Es. 39.}}
\VS{27}Ezéchias eut de très grandes richesses et de la gloire, et il se fit des trésors d'argent, d'or, de pierres précieuses, d'aromates, de boucliers, et de toutes sortes d'objets précieux~;
\VS{28}des magasins pour les récoltes de blé, de moût et d'huile, des étables pour toutes sortes de bétail, avec des rangées dans les étables.
\VS{29}Il se fit aussi des villes, et il acquit des troupeaux du gros et du menu bétail en abondance~; car Dieu lui avait donné de très grandes richesses.
\VS{30}Ce fut Ezéchias, qui boucha le canal du haut des eaux de Guihon, et les conduisit directement en bas, vers l'occident de la cité de David. Ainsi Ezéchias réussit dans tout ce qu'il fit.
\VS{31}Toutefois, lorsque les princes de Babylone envoyèrent des messagers vers lui pour s'informer du prodige qui s'était produit dans le pays, Dieu l'abandonna pour le mettre à l'épreuve, afin de connaître tout ce qui était dans son cœur\FTNT{Es. 29.}.
\VS{32}Le reste des actions d'Ezéchias, ses bonnes œuvres, voici, elles sont écrites dans la vision d'Esaïe, le prophète, fils d'Amots, dans le livre des rois de Juda et d'Israël.
\VS{33}Puis Ezéchias s'endormit avec ses pères, et on l'ensevelit au plus haut des sépulcres des fils de David~; et tout Juda, et Jérusalem lui firent honneur à sa mort, et Manassé, son fils régna à sa place.
\Chap{33}
\TextTitle{Manassé, roi impie de Juda\FTNTT{2 R. 21:1-9.}}
\VerseOne{}Manassé était âgé de douze ans quand il devint roi, et il régna cinquante-cinq ans à Jérusalem.
\VS{2}Il fit ce qui est mal aux yeux de Yahweh, suivant les abominations des nations que Yahweh avait chassées devant les enfants d'Israël.
\VS{3}Il rebâtit les hauts lieux qu'Ezéchias, son père, avait démolis, il redressa les autels aux Baals, il fit des idoles d'Asherah, et se prosterna devant toute l'armée des cieux et la servit.
\VS{4}Il bâtit aussi des autels dans la maison de Yahweh, de laquelle Yahweh avait parlé ainsi~: Mon Nom sera dans Jérusalem à jamais.
\VS{5}Il bâtit des autels à toute l'armée des cieux, dans les deux parvis de la maison de Yahweh.
\VS{6}Il fit passer ses fils par le feu dans la vallée du fils de Hinnom~; il pratiquait la magie, les sorcelleries et la voyance~; il établit des gens qui évoquaient les esprits et des devins. Il s'adonna à faire à l'extrême ce qui est mal aux yeux de Yahweh, pour l'irriter.
\VS{7}Il posa aussi une image taillée, une idole qu'il avait faite, dans la maison de Dieu, de laquelle Dieu avait dit à David, et à Salomon, son fils~: Je mettrai à perpétuité mon Nom dans cette maison et dans Jérusalem, que j'ai choisie entre toutes les tribus d'Israël~;
\VS{8}et je ne ferai plus sortir Israël de la terre que j'ai assignée à leurs pères, pourvu seulement qu'ils prennent garde à faire tout ce que je leur ai ordonné, selon toute la loi, les préceptes et les ordonnances prescrites par Moïse.
\VS{9}Manassé donc fit s'égarer Juda et les habitants de Jérusalem, jusqu'à faire pire que les nations que Yahweh avait exterminées de devant les enfants d'Israël.
\TextTitle{Yahweh avertit Manassé\FTNTT{2 R. 21:10-16.}}
\VS{10}Yahweh parla à Manassé et à son peuple~; mais ils ne furent pas attentifs.
\TextTitle{Manassé emmené captif se repent\FTNTT{2 R. 21:17-18.}}
\VS{11}Alors Yahweh fit venir contre eux les chefs de l'armée du roi d'Assyrie, qui mirent Manassé dans les fers~; ils le lièrent d'une double chaîne d'airain, et l'emmenèrent à Babylone.
\VS{12}Et dès qu'il fut dans l'angoisse, il supplia Yahweh, son Dieu, et il s'humilia profondément devant le Dieu de ses pères.
\VS{13}Il lui adressa ses supplications, et Dieu se laissa fléchir par sa prière, et exauça sa supplication. Il le fit retourner à Jérusalem, dans son royaume. Manassé reconnut alors que c'est Yahweh qui est Dieu.
\VS{14}Après cela, il bâtit une muraille extérieure à la cité de David, vers l'occident de Guihon, dans la vallée, jusqu'à l'entrée de la porte des poissons~; il environna la colline et l'éleva à une grande hauteur~; il établit aussi des chefs d'armée dans toutes les villes fortes de Juda.
\VS{15}Il ôta de la maison de Yahweh l'idole, et les dieux étrangers, et tous les autels qu'il avait bâtis sur la montagne de la maison de Yahweh et à Jérusalem, et les jeta hors de la ville.
\VS{16}Puis il rebâtit l'autel de Yahweh et y offrit des sacrifices d'offrande de paix et de reconnaissance~; et il ordonna à Juda de servir Yahweh, le Dieu d'Israël.
\VS{17}Toutefois, le peuple sacrifiait encore dans les hauts lieux, mais seulement à Yahweh, son Dieu.
\VS{18}Le reste des actions de Manassé, et la prière qu'il fit à son Dieu, et les paroles des voyants qui lui parlaient, au Nom de Yahweh, le Dieu d'Israël, voilà, toutes ces choses sont écrites dans les actes des rois d'Israël.
\VS{19}Sa prière, et comment Dieu se laissa fléchir par sa prière, ses péchés et ses infidélités, les lieux sur lesquels il bâtit des hauts lieux, et dressa des idoles d'Astarté et des images taillées, avant de s'être humilié, voici cela est écrit dans le livre de Hozaï.
\VS{20}Puis Manassé s'endormit avec ses pères, et on l'ensevelit dans sa maison. Et Amon, son fils, régna à sa place.
\TextTitle{Amon règne brièvement sur Juda\FTNTT{2 R. 21:18-26.}}
\VS{21}Amon était âgé de vingt-deux ans quand il devint roi, et il régna deux ans à Jérusalem.
\VS{22}Il fit ce qui est mal aux yeux de Yahweh, comme avait fait Manassé, son père. Il sacrifia à toutes les images taillées que Manassé, son père, avait faites, et il les servit.
\VS{23}Mais il ne s'humilia point devant Yahweh, comme s'était humilié Manassé, son père, mais se rendit de plus en plus coupable.
\VS{24}Et ses serviteurs ayant fait une conspiration contre lui, le firent mourir dans sa maison.
\VS{25}Mais le peuple du pays frappa tous ceux qui avaient conspiré contre le roi Amon. Et le peuple du pays établit pour roi, à sa place, Josias, son fils.
\Chap{34}
\TextTitle{Josias règne sur Juda~; ses réformes\FTNTT{2 R. 22:1-2.}}
\VerseOne{}Josias était âgé de huit ans quand il devint roi, et il régna trente et un ans à Jérusalem.
\VS{2}Il fit ce qui est droit aux yeux de Yahweh. Il marcha dans les voies de David, son père~; et ne s'en détourna ni à droite ni à gauche.
\VS{3}La huitième année de son règne, lorsqu'il était jeune, il commença à rechercher le Dieu de David, son père~; et à la douzième année, il commença à purifier Juda et Jérusalem des hauts lieux, des idoles d'Asherah, et des images taillées, et des images de fonte.
\VS{4}On démolit dans sa présence les autels des Baals, et il abattit les tentes solaires\FTNT{Tentes solaires~: lieux d'idolâtrie.} qui étaient par-dessus. Il brisa les Asherah, les images taillées et les images de fonte~; et les ayant réduites en poudre, il la répandit sur les sépulcres de ceux qui leur avaient sacrifié.
\VS{5}Puis il brûla les os des prêtres sur leurs autels, et il purifia ainsi Juda et Jérusalem.
\VS{6}Il fit la même chose dans les villes de Manassé, d'Ephraïm et de Siméon, et jusqu'à Nephthali, dans leurs ruines et tout autour.
\VS{7}Il démolit les autels et mit en pièces les Asherah et les images taillées, et il les réduisit en poussière~; il abattit toutes les tentes solaires dans tout le pays d'Israël. Puis il revint à Jérusalem.
\TextTitle{Restauration du temple\FTNTT{2 R. 22:3-7.}}
\VS{8}La dix-huitième année de son règne, après avoir purifié le pays et le temple, il envoya Schaphan, fils d'Atsalia, et Maaséja, chefs de la ville, et Joach, fils de Joachaz, commis sur les registres, pour réparer la maison de Yahweh, son Dieu.
\VS{9}Ils vinrent vers Hilkija, le grand-prêtre~; et on livra l'argent qui avait été apporté dans la maison de Dieu et que les Lévites, gardes du seuil, avaient amassé des mains de Manassé, d'Ephraïm et de tout le reste d'Israël, et aussi de tout Juda et Benjamin~; puis ils s'en retournèrent à Jérusalem.
\VS{10}On le remit entre les mains de ceux qui avaient la charge de l'ouvrage, qui étaient préposés sur la maison de Yahweh. Et ceux qui avaient la charge de l'ouvrage et qui travaillaient dans la maison de Yahweh le distribuèrent pour restaurer et réparer la maison de Yahweh.
\VS{11}Ils le donnèrent aux charpentiers et aux maçons, pour acheter des pierres de taille et du bois pour les poutres et pour la charpente des maisons que les rois de Juda avaient détruites.
\VS{12}Ces hommes s'employaient fidèlement à cet ouvrage. Jachath et Abdias, Lévites d'entre les fils de Merari, étaient préposés sur eux, et Zacharie et Meschullam, d'entre les fils des Kehathites, pour les diriger. Ces Lévites avaient tous de l'intelligence pour les instruments de musique.
\VS{13}Ils surveillaient ceux qui portaient les fardeaux, et dirigeaient tous ceux qui faisaient l'ouvrage, dans quelque service que ce soit~; les scribes, les administrateurs et les portiers, d'entre les Lévites.
\TextTitle{Le livre de la loi redécouvert\FTNTT{2 R. 22:8-10.}}
\VS{14}Au moment où l'on sortit l'argent qui avait été apporté dans la maison de Yahweh, Hilkija, le prêtre, trouva le livre de la loi de Yahweh, donné par Moïse.
\VS{15}Alors Hilkija, prenant la parole, dit à Schaphan, le secrétaire~: J'ai trouvé le livre de la loi dans la maison de Yahweh. Et Hilkija donna le livre à Schaphan.
\VS{16}Schaphan apporta le livre au roi, et rapporta tout au roi, en disant~: Les mains de tes serviteurs ont fait tout ce qui leur a été donné à faire.
\VS{17}Ils ont amassé l'argent qui se trouvait dans la maison de Yahweh, et l'ont livré entre les mains des administrateurs, et entre les mains de ceux qui ont la charge de l'ouvrage.
\VS{18}Schaphan, le secrétaire, raconta en disant au roi~: Hilkija, le prêtre, m'a donné un livre~; et Schaphan le lut devant le roi.
\TextTitle{Lecture du livre de la loi\FTNTT{2 R. 22:11-13.}}
\VS{19}Lorsque le roi entendit les paroles de la loi, il déchira ses vêtements.
\VS{20}Il ordonna à Hilkija, à Achikam, fils de Schaphan, à Abdon, fils de Michée, à Schaphan, le secrétaire, et à Asaja, serviteur du roi, en disant~:
\VS{21}Allez, consultez Yahweh pour moi et pour ce qui reste en Israël et en Juda, concernant les paroles du livre qui a été trouvé~; car la colère de Yahweh est grande, et elle s'est déversée sur nous, parce que nos pères n'ont point gardé la parole de Yahweh, pour faire selon tout ce qui est écrit dans ce livre.
\TextTitle{Instruction d'Hulda, la prophétesse\FTNTT{2 R. 22:14-20.}}
\VS{22}Hilkija et les gens du roi allèrent vers Hulda, la prophétesse, femme de Schallum, fils de Thokehath, fils de Hasra, garde des vêtements, laquelle demeurait à Jérusalem, dans un autre quartier, et lui en parlèrent.
\VS{23}Alors elle leur répondit~: Ainsi parle Yahweh, le Dieu d'Israël~: Dites à l'homme qui vous a envoyés vers moi~:
\VS{24}Ainsi parle Yahweh~: Voici, je vais faire venir le malheur sur ce lieu et sur ses habitants, à savoir toutes les malédictions du serment qui sont écrites dans le livre qu'on a lu devant le roi de Juda.
\VS{25}Parce qu'ils m'ont abandonné, et qu'ils ont fait brûler des parfums aux autres dieux, pour m'irriter par toutes les œuvres de leurs mains, ma colère s'est déversée sur ce lieu, et elle ne sera point éteinte.
\VS{26}Mais quant au roi de Juda, qui vous a envoyés pour consulter Yahweh, vous lui direz~: Ainsi parle Yahweh, le Dieu d'Israël, au sujet des paroles que tu as entendues~:
\VS{27}Parce que ton cœur a été touché, et que tu t'es humilié devant Dieu, quand tu as entendu ses paroles contre ce lieu et contre ses habitants, et que t'étant humilié devant moi, tu as déchiré tes vêtements et pleuré devant moi, je t'ai aussi entendu, dit Yahweh.
\VS{28}Voici, je vais te recueillir avec tes pères, et tu seras recueilli dans tes sépulcres en paix, et tes yeux ne verront point tout ce mal que je vais faire venir sur ce lieu et sur ses habitants. Et ils rapportèrent cette parole au roi.
\TextTitle{Renouvellement de l'alliance avec Yahweh\FTNTT{2 R. 23:1-3.}}
\VS{29}Alors le roi envoya assembler tous les anciens de Juda et de Jérusalem.
\VS{30}Le roi monta à la maison de Yahweh avec tous les hommes de Juda et les habitants de Jérusalem, les prêtres et les Lévites, et tout le peuple, depuis le plus grand jusqu'au plus petit~; et on lut devant eux toutes les paroles du livre de l'alliance, qui avait été trouvé dans la maison de Yahweh.
\VS{31}Et le roi se tint debout à sa place~; et traita devant Yahweh cette alliance qu'ils suivraient Yahweh, et qu'ils garderaient ses commandements, ses témoignages et ses lois, chacun de tout son cœur et de toute son âme, en pratiquant les paroles de l'alliance écrites dans ce livre.
\VS{32}Et il fit tenir debout tous ceux qui se trouvèrent à Jérusalem et en Benjamin~; et les habitants de Jérusalem firent selon l'alliance de Dieu, le Dieu de leurs pères.
\VS{33}Josias ôta de tous les pays qui appartenaient aux enfants d'Israël, toutes les abominations~; et il obligea tous ceux qui se trouvaient en Israël à servir Yahweh, leur Dieu. Pendant toute sa vie, ils ne se détournèrent point de Yahweh, le Dieu de leurs pères.
\Chap{35}
\TextTitle{Josias rétablit la Pâque\FTNTT{2 R. 23:21-27.}}
\VerseOne{}Or Josias célébra la Pâque à Yahweh à Jérusalem, et on immola la Pâque le quatorzième jour du premier mois.
\VS{2}Il rétablit les prêtres dans leurs charges, et les encouragea au service de la maison de Yahweh.
\VS{3}Il dit aussi aux Lévites qui enseignaient tout Israël et qui étaient consacrés à Yahweh~: Mettez l'arche sainte dans la maison que Salomon, fils de David, roi d'Israël, a bâtie. Qu'elle ne soit plus une charge sur vos épaules. Maintenant, servez Yahweh, votre Dieu, et son peuple d'Israël.
\VS{4}Préparez-vous, selon les maisons de vos pères, selon vos divisions suivant l'écrit de David, roi d'Israël, et suivant l'écrit de Salomon, son fils.
\VS{5}Tenez-vous dans le sanctuaire pour vos frères, les fils du peuple, selon les classes des maisons des pères, et selon que chaque famille des Lévites est partagée.
\VS{6}Immolez la Pâque, sanctifiez-vous, et préparez-la pour vos frères, afin qu'ils puissent la faire selon la parole que Yahweh a donnée par Moïse.
\VS{7}Josias éleva une offrande pour les gens du peuple et pour tous ceux qui se trouvaient là, des troupeaux d'agneaux et de chevreaux, au nombre de trente mille, et trois mille bœufs, le tout pour la Pâque~; cela fut pris sur les biens du roi.
\VS{8}Ses chefs élevèrent une offrande de bon gré pour le peuple, aux prêtres et aux Lévites. Hilkija, Zacharie et Jehiel, princes de la maison de Dieu, donnèrent aux prêtres, pour la Pâque, deux mille six cents agneaux, et trois cents bœufs.
\VS{9}Conania, Schemaeja et Nethaneel, ses frères, et Haschabia, Jeïel et Jozabad, qui étaient les princes des Lévites, élevèrent une offrande de cinq mille agneaux aux Lévites pour faire la Pâque, et cinq cents bœufs.
\VS{10}Le service étant préparé, les prêtres se tinrent à leurs postes, et les Lévites suivant leurs divisions, selon l'ordre du roi.
\VS{11}Puis on immola la Pâque~; et les prêtres répandaient le sang reçu de leurs mains, et les Lévites les dépouillaient.
\VS{12}Ils mirent à part les holocaustes, pour les donner aux gens du peuple, suivant les divisions des maisons de leurs pères, afin de les offrir à Yahweh, selon ce qui est écrit au livre de Moïse~; ils firent de même pour les bœufs.
\VS{13}Ils firent cuire la Pâque au feu, selon l'ordonnance~; et ils firent cuire dans des chaudières, des chaudrons et des poêles, les choses consacrées~; et ils les apportèrent rapidement à tous les gens du peuple.
\VS{14}Puis ils apprêtèrent ce qui était pour eux et pour les prêtres, car les prêtres, fils d'Aaron, furent occupés jusqu'à la nuit à élever en offrande les holocaustes et les graisses~; c'est pourquoi les Lévites apprêtèrent ce qui était pour eux et pour les prêtres, fils d'Aaron.
\VS{15}Les chantres, fils d'Asaph, étaient à leur place, selon l'ordre de David, d'Asaph, d'Héman et de Jeduthun, le voyant du roi. Les portiers étaient à chaque porte, ils n'eurent pas à se détourner de leur service, car leurs frères les Lévites apprêtaient ce qui était pour eux.
\VS{16}Ainsi, tout le service de Yahweh, en ce jour-là, fut réglé pour faire la Pâque et pour élever en offrande les holocaustes sur l'autel de Yahweh, selon l'ordre du roi Josias.
\VS{17}Les fils d'Israël qui s'y trouvèrent célébrèrent donc la Pâque en ce temps-là, et la fête des pains sans levain pendant sept jours.
\VS{18}Or on n'avait point célébré en Israël de Pâque semblable à celle-là depuis les jours de Samuel le prophète~; et aucun des rois d'Israël n'avait célébré une Pâque pareille comme le fit Josias, avec les prêtres et les Lévites, et tout Juda et Israël, qui s'y étaient trouvés avec les habitants de Jérusalem.
\VS{19}Cette Pâque fut célébrée la dix-huitième année du règne de Josias.
\TextTitle{Blessure et mort de Josias\FTNTT{2 R. 23:28-30.}}
\VS{20}Après tout cela, quand Josias eut réparé la maison de Yahweh, Néco, roi d'Egypte, monta pour faire la guerre à Carkemisch, sur l'Euphrate. Josias sortit à sa rencontre.
\VS{21}Mais Néco envoya vers lui des messagers pour lui dire~: Qu'y a-t-il entre nous, roi de Juda~? Ce n'est pas à toi que j'en veux aujourd'hui, mais à une maison qui me fait la guerre~; et Dieu m'a dit de me hâter. Désiste-toi donc de venir contre Dieu, qui est avec moi, de peur qu'il ne te détruise.
\VS{22}Cependant Josias ne se détourna point de lui, mais se déguisa pour combattre contre lui et il n'écouta pas les paroles de Néco, qui venaient de la bouche de Dieu. Il vint donc pour combattre dans la vallée de Meguiddo.
\VS{23}Les archers tirèrent sur le roi Josias~; et le roi dit à ses serviteurs~: Emportez-moi, car je suis très blessé.
\VS{24}Ses serviteurs l'ôtèrent du char, le mirent sur un second char qu'il avait, et le menèrent à Jérusalem. Il mourut et il fut enseveli dans les sépulcres de ses pères, et tous ceux de Juda et de Jérusalem menèrent le deuil de Josias.
\VS{25}Jérémie fit aussi des lamentations sur Josias~; et tous les chanteurs et toutes les chanteuses parlèrent dans leurs complaintes sur Josias jusqu'à ce jour~; et on en a fait une coutume en Israël. Voici, ces choses sont écrites dans les lamentations.
\VS{26}Le reste des actions de Josias, et ses œuvres de piété, selon ce qui est écrit dans la loi de Yahweh,
\VS{27}ses premières et ses dernières actions, sont écrites dans le livre des rois d'Israël et de Juda.
\Chap{36}
\TextTitle{Joachaz règne brièvement sur Juda\FTNTT{2 R. 23:31-33.}}
\VerseOne{}Alors le peuple du pays prit Joachaz, fils de Josias, et on l'établit roi à Jérusalem, à la place de son père.
\VS{2}Joachaz était âgé de vingt-trois ans quand il devint roi, et il régna trois mois à Jérusalem.
\VS{3}Le roi d'Egypte le destitua à Jérusalem, et condamna le pays à une amende de cent talents d'argent et d'un talent d'or.
\TextTitle{Règne de Jojakim, déportation à Babylone\FTNTT{2 R. 23:34-24:4-9.}}
\VS{4}Le roi d'Egypte établit pour roi sur Juda et Jérusalem Eliakim, frère de Joachaz~; et changea son nom en celui de Jojakim. Puis Néco prit Joachaz, son frère, et l'emmena en Egypte.
\VS{5}Jojakim était âgé de vingt-cinq ans quand il devint roi, et il régna onze ans à Jérusalem. Il fit ce qui est mal aux yeux de Yahweh, son Dieu.
\VS{6}Nebucadnetsar, roi de Babylone, monta contre lui et le lia de doubles chaînes d'airain pour le mener à Babylone.
\VS{7}Nebucadnetsar emporta aussi à Babylone des ustensiles de la maison de Yahweh, et il les mit dans son temple à Babylone.
\VS{8}Le reste des actions de Jojakim, et les abominations qu'il commit, et ce qui fut trouvé en lui, cela est écrit dans le livre des rois d'Israël et de Juda. Et Jojakin, son fils, régna à sa place.
\VS{9}Jojakin était âgé de huit ans quand il devint roi, et il régna trois mois et dix jours à Jérusalem. Il fit ce qui est mal aux yeux de Yahweh.
\TextTitle{Sédécias le dernier roi de Juda, autres déportations à Babylone\FTNTT{2 R. 24:10-20~; cp. 2 R. 25:1-21~; Jé. 39:8-10.}}
\VS{10}Et l'année suivante, le roi Nebucadnetsar envoya, et le fit emmener à Babylone avec les ustensiles précieux de la maison de Yahweh~; et il établit roi sur Juda et Jérusalem, Sédécias, son frère.
\VS{11}Sédécias était âgé de vingt et un ans quand il devint roi, et il régna onze ans à Jérusalem.
\VS{12}Il fit ce qui est mal aux yeux de Yahweh, son Dieu~; et il ne s'humilia point devant Jérémie le prophète, qui lui parlait de la part de Yahweh.
\VS{13}Et même il se rebella contre le roi Nebucadnetsar, qui l'avait fait prêter serment par le Nom de Dieu. Il raidit son cou, et il obstina son cœur pour ne point retourner à Yahweh, le Dieu d'Israël.
\VS{14}Pareillement, tous les chefs des prêtres et le peuple furent infidèles et continuèrent de plus en plus à pécher, selon toutes les abominations des nations~; et ils souillèrent la maison que Yahweh avait sanctifiée dans Jérusalem.
\VS{15}Or Yahweh, le Dieu de leurs pères, les avait sommés par ses messagers qu'il envoya de bonne heure, car il voulait épargner son peuple et sa propre demeure.
\VS{16}Mais ils se moquèrent des messagers de Dieu, ils méprisèrent ses paroles et traitèrent ses prophètes de séducteurs, jusqu'à ce que la fureur de Yahweh monta contre son peuple au point qu'il n'y eut plus de remède.
\VS{17}C'est pourquoi il fit monter contre eux le roi des Chaldéens, qui tua par l'épée leurs jeunes gens dans la maison de leur sanctuaire~; il n'épargna ni le jeune homme, ni la vierge, ni le vieillard, ni l'homme à cheveux blancs~; il les livra tous entre ses mains.
\VS{18}Il fit apporter à Babylone tous les ustensiles de la maison de Dieu, grands et petits, les trésors de la maison de Yahweh, et les trésors du roi et ceux de ses chefs.
\VS{19}Ils brûlèrent la maison de Dieu, ils démolirent les murailles de Jérusalem, ils livrèrent au feu tous ses palais et détruisirent tout ce qu'il y avait comme objets précieux.
\VS{20}Puis le roi de Babylone transporta à Babylone le reste qui échappa à l'épée, et ils furent ses esclaves et ceux de ses fils, jusqu'à la domination du royaume de Perse,
\VS{21}afin que la parole de Yahweh, prononcée par la bouche de Jérémie, soit accomplie~; jusqu'à ce que la terre ait pris plaisir à ses sabbats et durant tous les jours qu'elle demeura dévastée~; elle se reposa pour accomplir les soixante-dix années.
\TextTitle{L'édit de Cyrus autorise les juifs à retourner dans leurs villes}
\VS{22}Or la première année de Cyrus, roi de Perse, afin que la parole de Yahweh prononcée par Jérémie soit accomplie, Yahweh réveilla l'esprit de Cyrus, roi de Perse, qui fit publier dans tout son royaume, et même par écrit, en disant~:
\VS{23}Ainsi parle Cyrus, roi de Perse~: Yahweh, le Dieu des cieux, m'a donné tous les royaumes de la terre, et lui-même m'a ordonné de lui bâtir une maison à Jérusalem, qui est en Juda. Qui d'entre vous est de son peuple~? Que Yahweh, son Dieu, soit avec lui, et qu'il monte~!
\PPE{}
\end{multicols}
