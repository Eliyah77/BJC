\ShortTitle{Juges}\BookTitle{Juges}\BFont
\begin{multicols}{2}
\Chap{1}
\VerseOne{}Or il arriva qu'après la mort de Josué les enfants d'Israël consultèrent l'Eternel, en disant : Qui de nous montera le premier contre les Cananéens pour leur faire la guerre ?
\VS{2}Et l'Eternel répondit : Juda montera ; voici, j'ai livré le pays entre ses mains.
\VS{3}Et Juda dit à Siméon son frère : Monte avec moi en mon partage, et nous ferons la guerre aux Cananéens ; et j'irai aussi avec toi en ton partage. Ainsi Siméon alla avec lui.
\VS{4}Juda donc monta, et l'Eternel livra les Cananéens et les Phérésiens entre leurs mains ; et ils battirent en Bézec dix mille hommes.
\VS{5}Or ayant trouvé Adoni-bézec en Bézec, ils combattirent contre lui, et frappèrent les Cananéens et les Phérésiens.
\VS{6}Et Adoni-bézec s'enfuit, mais ils le poursuivirent ; et l'ayant pris, ils lui coupèrent les pouces de ses mains et de ses pieds.
\VS{7}Alors Adoni-bézec dit : Soixante et dix Rois, dont les pouces des mains et des pieds avaient été coupés, ont recueilli [du pain] sous ma table ; comme j'ai fait, Dieu m'a ainsi rendu ; et ayant été amené à Jérusalem, il y mourut.
\VS{8}Or les enfants de Juda avaient fait la guerre contre Jérusalem, et l'avaient prise, et ils avaient fait passer [ses habitants] au tranchant de l'épée, et mis la ville en feu.
\VS{9}Puis les enfants de Juda étaient descendus pour faire la guerre aux Cananéens, qui habitaient dans les montagnes, et au Midi, et dans la plaine.
\VS{10}Juda donc s'en était allé contre les Cananéens qui habitaient à Hébron ; or le nom d'Hébron était auparavant Kirjath-Arbah ; et il avait frappé Sesaï, Ahiman et Talmaï.
\VS{11}Et de là il était allé contre les habitants de Débir, le nom de laquelle était auparavant Kirjath-sépher.
\VS{12}Et Caleb avait dit : Qui frappera Kirjath-sépher et la prendra, je lui donnerai ma fille Hacsa pour femme.
\VS{13}Et Hothniël, fils de Kénas, frère puîné de Caleb, la prit ; et [Caleb] lui donna sa fille Hacsa pour femme.
\VS{14}Et il arriva que comme elle s'en allait, elle l'incita à demander à son père un champ ; puis elle descendit impétueusement de dessus son âne ; et Caleb lui dit : Qu'as-tu ?
\VS{15}Et elle lui répondit : Donne-moi un présent ; puisque tu m'as donné une terre sèche, donne-moi aussi des sources d'eaux. Et Caleb lui donna les fontaines [du quartier] de dessus, et les fontaines [du quartier] de dessous.
\VS{16}Or les enfants du Kénien, beau-père de Moïse, étaient montés de la ville des palmes avec les enfants de Juda, au désert de Juda, qui est au Midi de Harad ; parce qu'ils avaient marché et demeuré avec le peuple.
\VS{17}Puis Juda s'en alla avec Siméon son frère, et ils frappèrent les Cananéens qui habitaient à Tsephath, et la détruisirent à la façon de l'interdit, c'est pourquoi on appela la ville du nom de Horma.
\VS{18}Juda prit aussi Gaza avec ses confins ; Askelon avec ses confins ; et Hékron avec ses confins.
\VS{19}Et l'Eternel fut avec Juda, et ils dépossédèrent [les habitants] de la montagne : mais ils ne dépossédèrent point les habitants de la vallée, parce qu'ils avaient des chariots de fer.
\VS{20}Et on donna, selon que Moïse l'avait dit, Hébron à Caleb ; qui en déposséda les trois fils de Hanak.
\VS{21}Quant aux enfants de Benjamin, ils ne dépossédèrent point le Jébusien qui habitait à Jérusalem ; c'est pourquoi le Jébusien a habité avec les enfants de Benjamin à Jérusalem jusqu'à ce jour.
\VS{22}Ceux aussi de la maison de Joseph montèrent contre Bethel, et l'Eternel fut avec eux.
\VS{23}Et ceux de la maison de Joseph firent reconnaître Bethel, dont le nom était auparavant, Luz.
\VS{24}Et les espions virent un homme qui sortait de la ville, auquel ils dirent : Nous te prions de nous montrer un endroit par où l'on puisse entrer dans la ville, et nous te ferons grâce.
\VS{25}Il leur montra donc un endroit par où l'on pouvait entrer dans la ville, et ils la firent passer au tranchant de l'épée ; mais ils laissèrent aller cet homme-là, et toute sa famille.
\VS{26}Puis cet homme s'en étant allé au pays des Héthiens, y bâtit une ville, et l'appela Luz, qui est son nom jusqu'à ce jour.
\VS{27}Manassé aussi ne déposséda point [les habitants] de Beth-séan, ni des villes de son ressort, ni [les habitants] de Tahanac, ni des villes de son ressort ; ni les habitants de Dor, ni des villes de son ressort ; ni les habitants de Jibléham, ni des villes de son ressort, ni les habitants de Meguiddo, ni des villes de son ressort ; et les Cananéens osèrent encore habiter en ce pays-là.
\VS{28}Il est vrai qu'il arriva que quand Israël fut devenu plus fort, il rendît les Cananéens tributaires ; mais il ne les déposséda pas entièrement.
\VS{29}Ephraïm aussi ne déposséda point les Cananéens qui habitaient à Guézer ; mais les Cananéens habitèrent avec lui à Guézer.
\VS{30}Zabulon ne déposséda point les habitants de Kitron, ni les habitants de Nahalol ; mais les Cananéens habitèrent avec lui, et lui furent tributaires.
\VS{31}Aser ne déposséda point les habitants de Hacco, ni les habitants de Sidon, ni d'Ahlab, ni d'Aczib, ni d'Helba, ni d'Aphik, ni de Rehob.
\VS{32}Mais ceux d'Aser habitèrent parmi les Cananéens habitants du pays ; car ils ne les dépossédèrent point.
\VS{33}Nephthali ne déposséda point les habitants de Beth-sémes, ni les habitants de Beth-hanath, mais il habita parmi les Cananéens habitants du pays ; et les habitants de Beth-sémes, et de Beth-hanath lui furent tributaires.
\VS{34}Et les Amorrhéens tinrent les enfants de Dan fort resserrés dans la montagne, et ils ne souffraient point qu'ils descendissent dans la vallée.
\VS{35}Et ces Amorrhéens-là osèrent encore habiter à Har-Hérés, à Ajalon, et à Sahalbim ; mais la main de la maison de Joseph étant devenue plus forte, ils furent rendus tributaires.
\VS{36}Or la contrée des Amorrhéens [était] depuis la montée de Hakrabbim, depuis la roche, et au-dessus.
\Chap{2}
\VerseOne{}Or l'Ange de l'Eternel monta de Guilgal à Bokim, et dit : Je vous ai fait monter hors d'Egypte, et je vous ai fait entrer au pays dont j'avais juré à vos pères, et j'ai dit : Je n'enfreindrai jamais mon alliance que j'ai traitée avec vous.
\VS{2}Et vous aussi vous ne traiterez point alliance avec les habitants de ce pays ; vous démolirez leurs autels ; mais vous n'avez point obéi à ma voix ; qu'est-ce que vous avez fait ?
\VS{3}Et j'ai dit aussi : Je ne les chasserai point de devant vous, mais ils seront à vos côtés, et leurs dieux vous seront en piège.
\VS{4}Et il arriva qu'aussitôt que l'Ange de l'Eternel eut dit ces paroles à tous les enfants d'Israël, le peuple éleva sa voix, et pleura.
\VS{5}C'est pourquoi ils appelèrent ce lieu-là Bokim ; et ils sacrifièrent là à l'Eternel.
\VS{6}Or Josué avait renvoyé le peuple, et les enfants d'Israël s'en étaient allés chacun à son héritage, pour posséder le pays.
\VS{7}Et le peuple avait servi l'Eternel tout le temps de Josué, et tout le temps des Anciens qui avaient survécu à Josué, [et] qui avaient vu toutes les grandes œuvres de l'Eternel, lesquelles il avait faites pour Israël.
\VS{8}Puis Josué, fils de Nun, serviteur de l'Eternel, était mort, âgé de cent dix ans.
\VS{9}Et on l'avait enseveli dans les bornes de son héritage à Timnath-Hérés, en la montagne d'Ephraïm, du côté du Septentrion de la montagne de Gahas.
\VS{10}Et toute cette génération aussi avait été recueillie avec ses pères ; puis une autre génération s'était levée après eux, laquelle n'avait point connu l'Eternel, ni les œuvres qu'il avait faites pour Israël.
\VS{11}Les enfants d'Israël donc firent ce qui déplaît à l'Eternel, et servirent les Bahalins.
\VS{12}Et ayant abandonné l'Eternel, le Dieu de leurs pères, qui les avait fait sortir du pays d'Egypte, ils allèrent après d'autres dieux, d'entre les dieux des peuples qui [étaient] autour d'eux, et ils se prosternèrent devant eux ; ainsi ils irritèrent l'Eternel.
\VS{13}Ils abandonnèrent donc l'Eternel, et servirent Bahal et Hastaroth.
\VS{14}Et la colère de l'Eternel s'enflamma contre Israël, et il les livra entre les mains de gens qui les pillèrent ; et il les vendit en la main de leurs ennemis d'alentour, de sorte qu'ils ne purent plus se maintenir devant leurs ennemis.
\VS{15}Partout où ils allaient, la main de l'Eternel était contr'eux en mal, comme l'Eternel en avait parlé, et comme l'Eternel le leur avait juré, et ils furent dans de [grandes] angoisses.
\VS{16}Et l'Eternel leur suscitait des Juges, qui les délivraient de la main de ceux qui les pillaient.
\VS{17}Mais ils ne voulaient pas même écouter leurs Juges, ils paillardaient après d'autres dieux ; ils se prosternaient devant eux ; ils se détournaient aussitôt du chemin par lequel leurs pères avaient marché, obéissant aux commandements de l'Eternel ; mais eux ne faisaient pas ainsi.
\VS{18}Or quand l'Eternel leur suscitait des Juges, l'Eternel était aussi avec le Juge, et il les délivrait de la main de leurs ennemis pendant tout le temps du Juge ; car l'Eternel se repentait pour les sanglots qu'ils jetaient à cause de ceux qui les opprimaient et qui les accablaient.
\VS{19}Puis il arrivait que quand le Juge mourait, ils se corrompaient de nouveau plus que leurs pères, allant après d'autres dieux pour les servir, et se prosterner devant eux ; ils ne diminuaient rien de leur mauvaise conduite ni de leur train obstiné.
\VS{20}C'est pourquoi la colère de l'Eternel s'enflamma contre Israël, et il dit : Parce que cette nation a transgressé mon alliance que j'avais commandée à leurs pères, et qu'ils n ont point obéi à ma voix ;
\VS{21}Aussi je ne déposséderai plus de devant eux aucune des nations que Josué laissa quand il mourut ;
\VS{22}Afin d'éprouver par elles Israël, [et voir] s'ils garderont la voie de l'Eternel pour y marcher, comme leurs pères l'ont gardée, ou non.
\VS{23}L'Eternel donc laissa ces nations-là sans les déposséder sitôt, et il ne les livra point entre les mains de Josué.
\Chap{3}
\VerseOne{}Or ce sont ici les nations que l'Eternel laissa pour éprouver par elles Israël, [savoir] tous ceux qui n'avaient point eu connaissance de toutes les guerres de Canaan ;
\VS{2}Afin qu'au moins les générations des enfants d'Israël sussent et apprissent ce que c'est que la guerre ; au moins ceux qui auparavant n'en avaient rien connu.
\VS{3}[Ces nations donc furent] les cinq Gouvernements des Philistins, et tous les Cananéens, les Sidoniens, et les Héviens, qui habitaient en la montagne du Liban, depuis la montagne de Bahal-hermon, jusqu'à l'entrée de Hamath.
\VS{4}[Ces nations], dis-je, servirent à éprouver Israël, pour voir s'ils obéiraient aux commandements de l'Eternel, lesquels il avait donnés à leurs pères par le moyen de Moïse.
\VS{5}Ainsi les enfants d'Israël habitèrent parmi les Cananéens, les Héthiens, les Amorrhéens, les Phérésiens, les Héviens, et les Jébusiens.
\VS{6}Et ils prirent leurs filles pour femmes, et ils donnèrent leurs filles à leurs fils, et servirent leurs dieux.
\VS{7}Les enfants d'Israël firent donc ce qui déplaît à l'Eternel ; ils oublièrent l'Eternel leur Dieu, et servirent les Bahalins, et les bocages.
\VS{8}C'est pourquoi la colère de l'Eternel s'enflamma contre Israël, et il les vendit en la main de Cusan-rischathajim, Roi de Mésopotamie. Et les enfants d'Israël furent asservis à Cusan-rischathajim huit ans.
\VS{9}Puis les enfants d'Israël crièrent à l'Eternel, et l'Eternel leur suscita un Libérateur qui les délivra, [savoir] Hothniël, fils de Kénas, frère puîné de Caleb.
\VS{10}Et l'Esprit de l'Eternel fut sur lui, et il jugea Israël, et sortit en bataille, et l'Eternel livra entre ses mains Cusan-rischathajim, Roi d'Aram ; et sa main fut fortifiée contre Cusan-rischathajim.
\VS{11}Et le pays fut en repos quarante ans. Puis Hothniel, fils de Kénas, mourut.
\VS{12}Et les enfants d'Israël se mirent encore à faire ce qui déplaît à l'Eternel ; et l'Eternel fortifia Héglon, Roi de Moab, contre Israël, parce qu'ils avaient fait ce qui déplaît à l'Eternel.
\VS{13}Et [Héglon] assembla auprès de lui les enfants d'Hammon et d'Hamalec, et il alla, et frappa Israël, et ils s'emparèrent de la ville des palmes.
\VS{14}Et les enfants d'Israël furent asservis à Héglon, Roi de Moab, dix-huit ans.
\VS{15}Puis les enfants d'Israël crièrent à l'Eternel, et l'Eternel leur suscita un Libérateur, [savoir] Ehud, fils de Guéra Benjamite, duquel la main droite était serrée. Et les enfants d'Israël envoyèrent par lui un présent à Héglon, Roi de Moab.
\VS{16}Or Ehud s'était fait une épée à deux tranchants, de la longueur d'une coudée, qu'il avait ceinte sous ses vêtements sur sa cuisse droite.
\VS{17}Et il présenta le don à Héglon, Roi de Moab ; et Héglon était un homme fort gras.
\VS{18}Or il arriva que quand il eut achevé de présenter le don, il reconduisit le peuple qui avait apporté le don.
\VS{19}Mais [Ehud] s'en étant retourné depuis les carrières de pierre, qui [étaient] vers Guilgal, il dit : Ô Roi ! j'ai à te dire quelque chose en secret. Et il lui répondit : Tais-toi, et tous ceux qui étaient auprès de lui, sortirent de là.
\VS{20}Et Ehud s'approchant de lui, qui était assis seul dans sa chambre d'Eté, il lui dit : J'ai un mot à te dire de la part de Dieu. Alors [le Roi] se leva du trône.
\VS{21}Et Ehud avançant sa main gauche, prit l'épée de dessus sa cuisse droite, et la lui enfonça dans le ventre ;
\VS{22}Et la poignée entra après la lame, et la graisse serra tellement la lame, qu'il ne pouvait tirer l'épée du ventre ; et il en sortit de la fiente.
\VS{23}Après cela Ehud sortit par le porche, fermant après soi les portes de la chambre, laquelle il ferma à la clef.
\VS{24}Ainsi il sortit ; et les serviteurs d'Héglon vinrent, et regardèrent, et voilà, les portes de la chambre étaient fermées à la clef ; et ils dirent : Sans doute il est à ses affaires dans sa chambre d'Eté.
\VS{25}Et ils attendirent tant qu'ils en furent honteux ; et voyant qu'il n'ouvrait point les portes de la chambre, ils prirent la clef, et l'ouvrirent ; et voici, leur Seigneur était étendu mort à terre.
\VS{26}Mais Ehud échappa tandis qu'ils s'amusaient, et passa les carrières de pierre, et se sauva à Séhira.
\VS{27}Et quand il y fut entré, il sonna de la trompette en la montagne d'Ephraïm, et les enfants d'Israël descendirent avec lui en la montagne ; et il marchait devant eux.
\VS{28}Et il leur dit : Suivez-moi, car l'Eternel a livré entre vos mains les Moabites vos ennemis. Ainsi ils descendirent après lui, et se saisissant des passages du Jourdain contre les Moabites, ils ne laissèrent passer personne.
\VS{29}Et en ce temps-là ils frappèrent des Moabites environ dix mille hommes, tous en bon état, et tous vaillants, et il n'en échappa aucun.
\VS{30}En ce jour-là donc Moab fut humilié sous la main d'Israël ; et le pays fut en repos quatre-vingts ans.
\VS{31}Et après lui fut [en sa place] Samgar, fils d'Hanath, qui frappa six cents Philistins avec un aiguillon à bœufs, et qui délivra Israël.
\Chap{4}
\VerseOne{}Mais les enfants d'Israël se mirent encore à faire ce qui déplaît à l'Eternel, après qu'Ehud fut mort.
\VS{2}C'est pourquoi l'Eternel les vendit en la main de Jabin, Roi de Canaan, qui régnait en Hatsor, [et] de l'armée duquel Sisera était le chef, qui demeurait à Haroseth des Nations.
\VS{3}Et les enfants d'Israël crièrent à l'Eternel ; car [Jabin] avait neuf cents chariots de fer, et il avait violemment opprimé les enfants d'Israël durant vingt ans.
\VS{4}En ce temps-là Débora Prophétesse, femme de Lappidoth, jugeait Israël.
\VS{5}Et Débora se tenait sous un palmier entre Rama et Bethel, en la montagne d'Ephraïm ; et les enfants d'Israël montaient vers elle pour être jugés.
\VS{6}Or elle envoya appeler Barac, fils d'Abinoham de Kédés de Nephthali, et lui dit : L'Eternel le Dieu d'Israël n'a-t-il pas commandé, [et dit ?] Va, [et] fais amas de gens en la montagne de Tabor, et prends avec toi dix mille hommes des enfants de Nephthali, et des enfants de Zabulon.
\VS{7}Et j'attirerai à toi au torrent de Kison, Sisera, chef de l'armée de Jabin, avec ses chariots et la multitude de ses gens, et je le livrerai entre tes mains.
\VS{8}Et Barac lui dit : Si tu viens avec moi, j'y irai ; mais si tu ne viens pas avec moi, je n'y irai point.
\VS{9}Et elle répondit : Je ne manquerai pas d'aller avec toi ; mais tu n'auras pas d'honneur dans le chemin dans lequel tu iras ; car l'Eternel vendra Sisera en la main d'une femme. Débora donc se levant, s'en alla avec Barac à Kédés.
\VS{10}Et Barac ayant assemblé Zabulon et Nephthali en Kédés, fit monter après soi dix mille hommes ; et Débora monta avec lui.
\VS{11}Or Héber Kénien, des enfants de Hobab, beau-père de Moïse, s'étant séparé de Kaïn, avait tendu ses tentes jusques au bois de chênes de Tsahanajim, qui est près de Kédés.
\VS{12}Et on rapporta à Sisera que Barac, fils d'Abinoham, était monté en la montagne de Tabor.
\VS{13}Et Sisera assembla tous ses chariots, [savoir] neuf cents chariots de fer, et tout le peuple qui était avec lui, depuis Haroseth des Nations, jusqu'au torrent de Kison.
\VS{14}Et Débora dit à Barac : Lève-toi, car c'est ici le jour auquel l'Eternel a livré Sisera en ta main. L'Eternel n'est-il pas sorti devant toi ? Barac donc descendit de la montagne de Tabor, et dix mille hommes après lui.
\VS{15}Et l'Eternel frappa Sisera, et tous ses chariots, et toute l'armée, au tranchant de l'épée, devant Barac ; et Sisera descendit du chariot, et s'enfuit à pied.
\VS{16}Et Barac poursuivit les chariots et l'armée jusqu'à Haroseth des Nations ; et toute l'armée de Sisera fut passée au fil de l'épée ; il n'en demeura pas un seul.
\VS{17}Et Sisera s'enfuit à pied dans la tente de Jahel, femme de Héber Kénien ; car il y avait paix entre Jabin, Roi de Hatsor, et la maison de Héber Kénien.
\VS{18}Et Jahel étant sortie au-devant de Sisera, lui dit : Mon seigneur ! retire-toi, retire-toi chez moi, ne crains point. Il se retira donc chez elle dans la tente, et elle le couvrit d'une couverture.
\VS{19}Puis il lui dit : Je te prie, donne-moi un peu d'eau à boire, car j'ai soif. Et elle ouvrant un baril de lait lui donna à boire, et le couvrit.
\VS{20}Il lui dit aussi : Demeure à l'entrée de la tente, et au cas que quelqu'un vienne, et t'interroge, disant : Y a-t-il ici quelqu'un ? alors tu répondras : Non.
\VS{21}Et Jahel, femme de Héber, prit un clou de la tente, et prenant un marteau en sa main, elle vint à lui doucement ; et lui enfonça un clou dans sa tempe, lequel entra dans la terre pendant qu'il dormait profondément, car il était fort las ; et ainsi il mourut.
\VS{22}Et voici, Barac poursuivait Sisera, et Jahel sortit au-devant de lui, et lui dit : Viens, et je te montrerai l'homme que tu cherches. Et Barac entra chez elle, et voici, Sisera était étendu mort, et le clou était dans sa tempe.
\VS{23}En ce jour-là donc Dieu humilia Jabin, Roi de Canaan, devant les enfants d'Israël.
\VS{24}Et la puissance des enfants d'Israël s'avançait et se renforçait de plus en plus contre Jabin, Roi de Canaan, jusqu'à ce qu'ils l'eurent exterminé.
\Chap{5}
\VerseOne{}En ce jour-là Débora, avec Barac fils d'Abinoham, chanta en disant :
\VS{2}Bénissez l'Eternel de ce qu'il a fait de telles vengeances en Israël, [et] de ce que le peuple a été porté de bonne volonté.
\VS{3}Vous Rois écoutez, vous Princes prêtez l'oreille ; moi, moi, je chanterai à l'Eternel, je psalmodierai à l'Eternel le Dieu d'Israël.
\VS{4}Ô Eternel ! quand tu sortis de Séhir, quand tu marchas du territoire d'Edom, la terre fut ébranlée, même les cieux fondirent, les nuées, dis-je, fondirent en eaux.
\VS{5}Les montagnes s'écoulèrent de devant l'Eternel, ce Sinaï [s'écoula] de devant l'Eternel le Dieu d'Israël.
\VS{6}Aux jours de Samgar, fils de Hanath, aux jours de Jahel, les grands chemins n'étaient plus battus, et ceux qui allaient par les chemins allaient par des routes détournées.
\VS{7}Les villes non murées n'étaient plus habitées en Israël, elles n'étaient point habitées, jusqu'à ce que je me suis levée, moi Débora, jusqu'à ce que je me suis levée pour être mère en Israël.
\VS{8}[Israël] choisissait-il des dieux nouveaux ? alors la guerre était aux portes. A-t-il été vu bouclier ou lance en quarante mille d'Israël ?
\VS{9}J'ai mon cœur vers les Gouverneurs d'Israël, qui se sont portés volontairement d'entre le peuple. Bénissez l'Eternel.
\VS{10}Vous qui montez sur les ânesses blanches, [et] qui êtes assis dans le siège de la justice, et vous qui allez dans les chemins, parlez.
\VS{11}Le bruit des archers [ayant cessé] dans les lieux où l'on puisait l'eau, qu'on s'y entretienne des justices de l'Eternel, [et] des justices de ses villes non murées en Israël ; alors le peuple de Dieu descendra aux portes.
\VS{12}Réveille-toi, réveille-toi, Débora ; réveille-toi, réveille-toi, dit le Cantique, lève-toi Barac, et emmène en captivité ceux que tu as faits captifs, toi fils d'Abinoham.
\VS{13}[L'Eternel] a fait alors dominer le réchappé, le peuple sur les magnifiques ; l'Eternel m'a fait dominer sur les forts.
\VS{14}Leur racine est depuis Ephraïm jusqu'à Hamalek ; Benjamin [a été] après toi parmi tes peuples ; de Makir sont descendus les Gouverneurs ; et de Zabulon ceux qui manient la plume du Scribe.
\VS{15}Et les principaux d'Issacar ont été avec Débora, et Issacar ainsi que Barac ; il a été envoyé avec sa suite dans la vallée ; il y a eu aux séparations de Ruben, de grandes considérations dans leur cœur.
\VS{16}Pourquoi t'es-tu tenu entre les barres des étables, afin d'entendre les cris des troupeaux ? Il y a eu aux séparations de Ruben de grandes consultations dans leur cœur.
\VS{17}Galaad est demeuré au delà du Jourdain ; et pourquoi Dan s'est-il tenu aux navires ? Aser s'est tenu aux ports de la mer, et il est demeuré dans ses havres.
\VS{18}Mais pour Zabulon, c'est un peuple qui a exposé son âme à la mort ; et Nephthali aussi, sur les hauteurs de la campagne.
\VS{19}Les Rois sont venus, ils ont combattu ; les Rois de Canaan ont alors combattu à Tahanac, près des eaux de Méguiddo ; mais ils n'ont point fait de gain d'argent.
\VS{20}On a combattu des cieux, les étoiles, [dis-je], ont combattu du lieu de leur cours contre Sisera.
\VS{21}Le torrent de Kison les a emportés, le torrent de Kédummim, le torrent de Kison ; mon âme tu as foulé aux pieds la force.
\VS{22}Alors a été rompue la corne des pieds des chevaux par le battement des pieds, par le battement, [dis-je], des pieds de ses puissants [chevaux].
\VS{23}Maudissez Meroz, a dit l'Ange de l'Eternel ; maudissez, maudissez ses habitants, car ils ne sont point venus au secours de l'Eternel, au secours de l'Eternel, avec les forts.
\VS{24}Bénie soit par-dessus toutes les femmes Jahel, femme de Héber Kénien, qu'elle soit bénie par-dessus les femmes [qui se tiennent] dans les tentes.
\VS{25}Il a demandé de l'eau, elle lui a donné du lait ; elle lui a présenté de la crème dans la coupe des magnifiques.
\VS{26}Elle a avancé sa main gauche au clou et sa main droite au marteau des ouvriers ; elle a frappé Sisera, et lui a fendu la tête ; elle a transpercé et traversé ses tempes.
\VS{27}Il s'est courbé entre les pieds de [Jahel], il est tombé, il a été étendu entre les pieds de Jahel, il s'est courbé, il est tombé ; [et] au lieu où il s'est courbé, il est tombé là tout défiguré.
\VS{28}La mère de Sisera regardait par la fenêtre, et s'écriait [en regardant] par les treillis : Pourquoi son char tarde-t-il à venir ? Pourquoi ses chariots vont-ils si lentement ?
\VS{29}Et les plus sages de ses dames lui ont répondu ; et elle aussi se répondait à soi-même :
\VS{30}N'ont-ils pas trouvé ? ils partagent le butin ; une fille, deux filles à chacun par tête. Le butin [des vêtements] de couleurs est à Sisera, le butin de couleurs de broderie ; couleur de broderie à deux endroits, autour du cou de ceux du butin.
\VS{31}Qu'ainsi périssent, ô Eternel ! tous tes ennemis ; et que ceux qui t'aiment soient comme le soleil quand il sort en sa force. Or le pays fut en repos quarante ans.
\Chap{6}
\VerseOne{}Or les enfants d'Israël firent ce qui déplaît à l'Eternel ; et l'Eternel les livra entre les mains de Madian pendant sept ans.
\VS{2}Et la main de Madian se renforça contre Israël, [et] à cause des Madianites les enfants d'Israël se firent des creux qui sont dans les montagnes, et des cavernes, et des forts.
\VS{3}Car il arrivait que quand Israël avait semé, Madian montait avec Hamalec et les Orientaux, et ils montaient contre lui.
\VS{4}Et faisant un camp contr'eux ils ravageaient les fruits du pays jusqu'à Gaza, et ne laissaient rien de reste en Israël, ni vivres, ni menu bétail, ni bœufs, ni ânes.
\VS{5}Car eux et leurs troupeaux montaient, et ils venaient avec leurs tentes en aussi grand nombre que des sauterelles, tellement qu'eux et leurs chameaux étaient sans nombre ; et ils venaient au pays pour le ravager.
\VS{6}Israël donc fut fort appauvri par Madian, et les enfants d'Israël crièrent à l'Eternel.
\VS{7}Et il arriva que quand les enfants d'Israël eurent crié à l'Eternel à l'occasion de Madian,
\VS{8}L'Eternel envoya un Prophète vers les enfants d'Israël, qui leur dit : Ainsi a dit l'Eternel le Dieu d'Israël : Je vous ai fait monter hors d'Egypte, et je vous ai retirés de la maison de servitude ;
\VS{9}Et vous ai délivrés de la main des Egyptiens, et de la main de tous ceux qui vous opprimaient, et les ai chassés de devant vous, et vous ai donné leur pays.
\VS{10}Je vous ai dit aussi : Je suis l'Eternel votre Dieu, vous ne craindrez point les dieux des Amorrhéens, au pays desquels vous habitez ; mais vous n'avez point obéi à ma voix.
\VS{11}Puis l'Ange de l'Eternel vint, et s'assit sous un chêne qui était à Hophra, appartenant à Joas Abihézérite. Et Gédeon son fils battait le froment dans le pressoir, pour le sauver de devant Madian.
\VS{12}Alors l'Ange de l'Eternel lui apparut, et lui dit : Très-fort et vaillant homme, l'Eternel est avec toi.
\VS{13}Et Gédeon lui répondit : Hélas mon Seigneur ! [est-il possible] que l'Eternel soit avec nous ? et pourquoi donc toutes ces choses nous sont-elles arrivées ? Et où sont toutes ces merveilles que nos pères nous ont récitées, en disant : L'Eternel ne nous a-t-il pas fait monter hors d'Egypte ? car maintenant l'Eternel nous a abandonnés, et nous a livrés entre les mains des Madianites.
\VS{14}Et l'Eternel le regardant lui dit : Va avec cette force que tu as, et tu délivreras Israël de la main des Madianites ; ne t'ai-je pas envoyé ?
\VS{15}Et il lui répondit : Hélas, mon Seigneur ! par quel moyen délivrerai-je Israël ? Voici, mon millier est le plus pauvre qui soit en Manassé, et je suis le plus petit de la maison de mon père.
\VS{16}Et l'Eternel lui dit : Parce que je serai avec toi, tu frapperas les Madianites comme s'ils n'étaient qu'un seul homme.
\VS{17}Et il lui répondit : Je te prie, si j'ai trouvé grâce devant toi, de me donner un signe [pour montrer] que c'est toi qui parles avec moi.
\VS{18}Je te prie, ne t'en va point d'ici jusqu'à ce que je revienne à toi, et que j'apporte mon présent, et que je le mette devant toi. Et il dit : J'y demeurerai jusqu'à ce que tu reviennes.
\VS{19}Alors Gédeon rentra, et apprêta un chevreau de lait, et des gâteaux sans levain d'un Epha de farine ; mit la chair dans un panier, le bouillon dans un pot, et il les lui apporta sous le chêne, et les lui présenta.
\VS{20}Et l'Ange de Dieu lui dit : Prends cette chair et ces gâteaux sans levain, et mets-les sur ce rocher, et répands le bouillon ; et il le fit ainsi.
\VS{21}Alors l'Ange de l'Eternel ayant étendu le bâton qu'il avait en sa main, toucha la chair et les gâteaux sans levain, et le feu monta du rocher, et consuma la chair et les gâteaux sans levain ; puis l'Ange de l'Eternel s'en alla de devant lui.
\VS{22}Et Gédeon vit que c'était l'Ange de l'Eternel, et il dit : ha, Seigneur Eternel ; est-ce pour cela que j'ai vu l'Ange de l'Eternel face à face ?
\VS{23}Et l'Eternel lui dit : Il va bien pour toi ; ne crains point, tu ne mourras point.
\VS{24}Et Gédeon bâtit là un autel à l'Eternel, et l'appela L'ETERNEL DE PAIX. Et cet autel est demeuré jusqu'à aujourd'hui à Hophra des Abihézérites.
\VS{25}Or il arriva en cette nuit-là que l'Eternel lui dit : Prends un taureau d'entre les bœufs qui sont à ton père, savoir le deuxième taureau, de sept ans ; et démolis l'autel de Bahal qui est à ton père, et coupe le bocage qui est auprès ;
\VS{26}Et bâtis un autel à l'Eternel ton Dieu sur le haut de ce fort, en un lieu convenable. Tu prendras ce deuxième taureau, et tu l'offriras en holocauste avec les arbres du bocage que tu couperas.
\VS{27}Gédeon donc ayant pris dix hommes d'entre ses serviteurs, fit comme l'Eternel lui avait dit ; et parce qu'il craignait la maison de son père et les gens de la ville, s'il l'eût fait de jour, il le fit de nuit.
\VS{28}Et les gens de la ville se levèrent de bon matin, et voici, l'autel de Bahal avait été démoli, et le bocage qui était auprès, était coupé, et le deuxième taureau était offert en holocauste sur l'autel qu'on avait bâti.
\VS{29}Et ils se disaient les uns aux autres : Qui a fait ceci ? Et s'en étant informés, et ayant cherché, ils dirent : Gédeon le fils de Joas a fait ceci.
\VS{30}Puis les gens de la ville dirent à Joas : Fais sortir ton fils, et qu'il meure ; car il a démoli l'autel de Bahal, et a coupé le bocage qui était auprès.
\VS{31}Et Joas répondit à tous ceux qui s'adressèrent à lui : Est-ce vous qui prendrez la cause de Bahal ? Est-ce vous qui le sauverez ? Quiconque aura pris sa cause, sera mis à mort d'ici au matin. S'il est Dieu, qu'il défende sa cause, de ce qu'on a démoli son autel.
\VS{32}Et en ce jour-là il appela Gédeon Jérubbahal, et dit : Que Bahal défende sa cause de ce que [Gédeon] a démoli son autel.
\VS{33}Or tous les Madianites, les Hamalécites, et les Orientaux s'assemblèrent tous, et ayant passé le Jourdain ils se campèrent en la vallée de Jizréhel.
\VS{34}Et l'Esprit de l'Eternel revêtit Gédeon ; lequel sonna de la trompette, et les Abihézérites s'assemblèrent auprès de lui.
\VS{35}Il envoya aussi des messagers par toute [la Tribu de] Manassé, qui s'assembla aussi auprès de lui ; puis il envoya des messagers en Aser, en Zabulon, et en Nephthali, lesquels montèrent pour aller au-devant d'eux.
\VS{36}Et Gédeon dit à Dieu : Si tu dois délivrer Israël par mon moyen, comme tu l'as dit,
\VS{37}Voici, je m'en vais mettre une toison dans l'aire ; si la rosée est sur la toison seule, et que le sec soit dans toute la place, je connaîtrai que tu délivreras Israël par mon moyen, selon que tu m'en as parlé.
\VS{38}Et la chose arriva ainsi, car s'étant levé de bon matin le lendemain, et ayant pressé cette toison, il en fit sortir pleine une tasse d'eau de rosée.
\VS{39}Gédeon dit encore à Dieu : Que ta colère ne s'enflamme point contre moi, et je parlerai seulement cette fois ; je te prie, que je fasse un essai en la toison encore cette fois seulement ; je te prie qu'il n'y ait rien de sec que la toison, et fais que la rosée soit sur toute la place [de l'aire].
\VS{40}Et Dieu fit ainsi cette nuit-là ; car il n'y eut rien de sec que la toison, et la rosée fut sur toute la place de l'aire.
\Chap{7}
\VerseOne{}Jérubbahal donc, qui [est] Gédeon, se levant dès le matin, et tout le peuple qui était avec lui, ils se campèrent près de la fontaine de Harod, et ils avaient le camp de Madian du côté du Septentrion, vers le coteau de Moreh dans la vallée.
\VS{2}Or l'Eternel dit à Gédeon : Le peuple qui est avec toi est en trop grand nombre, pour que je livre Madian en leur main, de peur qu'Israël ne se glorifie contre moi, en disant : Ma main m'a délivré.
\VS{3}Maintenant donc fais publier, le peuple l'entendant, et qu'on dise : Quiconque est timide et a peur, qu'il s'en retourne, et s'en aille dès le matin du côté de la montagne de Galaad ; et vingt-deux mille du peuple s'en retournèrent ; et il en resta dix mille.
\VS{4}Et l'Eternel dit à Gédeon : Il y a encore du peuple en trop grand nombre ; fais-les descendre vers l'eau, et là je te les choisirai ; et celui dont je te dirai, celui-ci ira avec toi, il ira avec toi ; et celui duquel je te dirai, celui-ci n'ira point avec toi ; il n'y ira point.
\VS{5}Il fit donc descendre le peuple vers l'eau ; et l'Eternel dit à Gédeon : Quiconque lapera l'eau de sa langue, comme le chien lape, tu le mettras à part ; et [tu mettras aussi à part] tous ceux qui se courberont sur leurs genoux pour boire.
\VS{6}Et le nombre de ceux qui lapaient l'eau dans leur main, [la] portant à leur bouche, fut de trois cents hommes ; mais tout le reste du peuple se courba sur ses genoux, pour boire de l'eau.
\VS{7}Alors l'Eternel dit à Gédeon : Je vous délivrerai par le moyen de ces trois cents hommes qui ont lapé [l'eau], et je livrerai Madian en ta main. Que tout le peuple donc s'en aille, chacun en son lieu.
\VS{8}Ainsi le peuple prit en sa main des provisions, et leurs trompettes. Et Gédeon renvoya tous les hommes d'Israël chacun en sa tente, et retint les trois cents hommes. Or le camp de Madian était au-dessous, dans la vallée.
\VS{9}Et il arriva cette nuit-là, que l'Eternel lui dit : Lève-toi, descends au camp, car je l'ai livré en ta main.
\VS{10}Et si tu crains d'y descendre, descends vers le camp toi et Purah ton serviteur.
\VS{11}Et tu entendras ce qu'ils diront, et tes mains seront fortifiées, puis tu descendras au camp. Il descendit donc avec Purah son serviteur, jusqu'au premier corps de garde du camp.
\VS{12}Or Madian, et Hamalec, et tous les Orientaux, étaient répandus dans la vallée comme des sauterelles, tant il y en avait, et leurs chameaux étaient sans nombre, comme le sable qui est sur le bord de la mer, tant il y en avait.
\VS{13}Gédeon donc y étant arrivé, voilà, un homme récitait à son compagnon un songe, et lui disait : Voici, j'ai songé un songe ; il me semblait qu'un gâteau de pain d'orge se roulait vers le camp de Madian, et qu'étant venu jusqu'aux tentes, il les a frappées, de sorte qu'elles en sont tombées, et il les a renversées, [en roulant] du haut [de la montagne], et elles sont tombées.
\VS{14}Alors son compagnon répondit, et dit : Cela n'est autre chose que l'épée de Gédeon, fils de Joas, homme d'Israël. Dieu a livré Madian et tout ce camp en sa main.
\VS{15}Et quand Gédeon eut entendu le récit du songe, et son interprétation, il se prosterna ; et étant retourné au camp d'Israël, il dit : Levez-vous, car l'Eternel a livré le camp de Madian en vos mains.
\VS{16}Puis il divisa les trois cents hommes en trois bandes, et leur donna à chacun des trompettes à la main, et des cruches vides, et des flambeaux dans les cruches.
\VS{17}Et il leur dit : Prenez garde à moi et faites comme je ferai ; lorsque je serai arrivé au bout du camp, vous ferez comme je ferai.
\VS{18}Quand donc je sonnerai de la trompette, et tous ceux aussi qui sont avec moi, alors vous sonnerez aussi des trompettes autour de tout le camp, et vous direz : L'EPEE DE L'ETERNEL, ET DE GEDEON.
\VS{19}Gédeon donc, et les cent hommes qui étaient avec lui, arrivèrent au bout du camp, comme on venait de poser la seconde garde ; on ne faisait que poser les gardes lorsqu'ils sonnèrent des trompettes ; et qu'ils cassèrent les cruches qu'ils avaient en leurs mains.
\VS{20}Ainsi les trois bandes sonnèrent des trompettes, et cassèrent les cruches, tenant en leur main gauche les flambeaux, et en leur main droite les trompettes pour sonner, et ils criaient : L'EPEE DE L'ETERNEL, ET DE GEDEON.
\VS{21}Et ils se tinrent chacun en sa place autour du camp ; et toute l'armée courait ça et là, s'écriant et fuyant.
\VS{22}Car comme les trois cents hommes sonnaient des trompettes, l'Eternel tourna l'épée d'un chacun contre son compagnon, même par tout le camp. Et l'armée s'enfuit jusqu'à Beth-sittah, vers Tserera, jusqu'au bord d'Abelmeholah, vers Tabbat.
\VS{23}Et les hommes d'Israël, [savoir] de Nephthali et d'Aser, et de tout Manassé s'assemblèrent, et poursuivirent Madian.
\VS{24}Alors Gédeon envoya des messagers par toute la montagne d'Ephraïm, pour leur dire : Descendez pour aller à la rencontre de Madian, et saisissez-vous les premiers des eaux du Jourdain jusqu'à Beth-bara. Les hommes d'Ephraïm donc s'étant assemblés, se saisirent des eaux du Jourdain jusqu'à Beth-bara.
\VS{25}Et ils prirent deux des chefs de Madian, Horeb et Zééb, et ils tuèrent Horeb au rocher de Horeb ; mais ils tuèrent Zééb au pressoir de Zééb ; et ils poursuivirent Madian, et apportèrent les têtes de Horeb et de Zééb à Gédeon, au deçà du Jourdain.
\Chap{8}
\VerseOne{}Alors les hommes d'Ephraïm lui dirent : Que veut dire ce que tu nous as fait, de ne nous avoir pas appelés quand tu es allé à la guerre contre Madian ; et ils s'emportèrent fortement contre lui.
\VS{2}Et il leur répondit : Qu'ai-je fait maintenant au prix de ce que vous avez fait ? Les grappillages d'Ephraïm ne sont-ils pas meilleurs que la vendange d'Abihézer ?
\VS{3}Dieu a livré entra vos mains les chefs de Madian, Horeb et Zééb ; or qu'ai-je pu faire au prix de ce que vous avez fait ? et leur esprit fut apaisé envers lui quand il leur eut ainsi parlé.
\VS{4}Or Gédeon étant arrivé au Jourdain le passa, lui et les trois cents hommes qui étaient avec lui, lesquels tout las qu'ils étaient, poursuivaient [l'ennemi].
\VS{5}C'est pourquoi il dit aux gens de Succoth : Donnez, je vous prie, au peuple qui me suit, quelques pains car ils sont las ; et ainsi je poursuivrai Zébah et Tsalmunah, Rois de Madian.
\VS{6}Mais les principaux de Succoth répondirent : La paume de Zébah et celle de Tsalmunah sont-elles maintenant en ta main, que nous donnions du pain à ton armée ?
\VS{7}Et Gédeon dit : Quand donc l'Eternel aura livré Zébah et Tsalmunah en ma main, je froisserai alors votre chair avec des épines du désert, et avec des chardons.
\VS{8}Puis de là il monta à Pénuël, et il tint les mêmes discours à ceux de Pénuël. Et les gens de Pénuël lui répondirent comme les gens de Succoth avaient répondu.
\VS{9}Il parla donc aussi aux hommes de Pénuël, en disant : Quand je retournerai en paix, je démolirai cette tour.
\VS{10}Or Zébah et Tsalmunah étaient à Karkor, et leurs armées avec eux, environ quinze mille hommes, qui étaient tous ceux qui étaient restés de toute l'armée des Orientaux ; car il en était tombé morts six vingt mille hommes tirant l'épée.
\VS{11}Et Gédeon monta par le chemin de ceux qui habitent dans les tentes, du côté oriental de Nobah et de Jogbeha, et défit l'armée, qui se croyait assurée.
\VS{12}Et comme Zébah et Tsalmunah s'enfuyaient, il les poursuivit, et prit les deux Rois de Madian, Zébah et Tsalmunah, et mit en déroute toute l'armée.
\VS{13}Puis Gédeon fils de Joas retourna de la bataille de la montée de Hèrés.
\VS{14}Et prenant un garçon de Succoth, il l'interrogea ; [et] ce garçon lui donna par écrit les principaux de Succoth, et ses Anciens, [au nombre] de soixante dix-sept hommes.
\VS{15}Et il s'en vint aux gens de Succoth, et leur dit : Voici Zébah et Tsalmunah, au sujet desquels vous m'avez insulté, en disant : La paume de Zébah et celle de Tsalmunah sont-elles maintenant en ta main, que nous donnions du pain à tes gens fatigués ?
\VS{16}Il prit donc les Anciens de la ville, et des épines du désert, et des chardons, et il en froissa les hommes de Succoth.
\VS{17}Or il avait démoli la tour de Pénuël, et mis à mort les gens de la ville.
\VS{18}Puis il dit à Zébah et à Tsalmunah : Comment [étaient faits] ces hommes que vous avez tués sur le Tabor ? Ils répondirent : Ils étaient entièrement comme toi ; chacun d'eux avait l'air d'un fils de Roi.
\VS{19}Et il leur dit : C'étaient mes frères, enfants de ma mère ; l'Eternel est vivant, si vous leur eussiez sauvé la vie, je ne vous tuerais pas.
\VS{20}Puis il dit à Jéther son premier-né : Lève-toi, tue-les ; mais le jeune garçon ne tira point son épée, car il avait peur, parce qu'il était encore jeune garçon.
\VS{21}Et Zébah et Tsalmunah dirent : Lève-toi toi-même, et te jette sur nous ; car tel qu'est l'homme, telle est sa force. Et Gédeon se leva, et tua Zébah et Tsalmunah, et prit les colliers qui étaient aux cous de leurs chameaux.
\VS{22}Et ceux d'Israël dirent tous d'un accord à Gédeon : Domine sur nous, tant toi que ton fils, et le fils de ton fils ; car tu nous as délivrés de la main de Madian.
\VS{23}Et Gédeon leur répondit : Je ne dominerai point sur vous, ni mon fils ne dominera point sur vous ; l'Eternel dominera sur vous.
\VS{24}Mais Gédeon leur dit : Je vous ferai une prière, qui est que vous me donniez chacun de vous les bagues qu'il a eues du butin ; car les ennemis avaient des bagues d'or, parce qu'ils étaient Ismaélites.
\VS{25}Et ils répondirent : Nous les donnerons très-volontiers ; et étendant un manteau ils y jetèrent chacun les bagues qu'ils avaient eues du butin.
\VS{26}Et le poids des bagues d'or qu'il avait demandées, fut de mille sept cents [sicles] d'or, sans les colliers, les bœttes de senteur, et les vêtements d'écarlate qui étaient sur les Rois de Madian, et sans les chaînes qui étaient aux cous de leurs chameaux.
\VS{27}Puis Gédeon en fit un Ephod ; et le mit en sa ville, qui était Hophra ; et tout Israël paillarda après lui en ce lieu-là ; ce qui tourna en piège à Gédeon, et à sa maison.
\VS{28}Ainsi Madian fut humilié devant les enfants d'Israël, et n'éleva plus sa tête ; et le pays fut en repos quarante ans aux jours de Gédeon.
\VS{29}Jérubbahal donc fils de Joas s'en vint [en sa ville], et se tint en sa maison.
\VS{30}Or Gédeon eut soixante et dix fils, sortis de sa hanche, parce qu'il eut plusieurs femmes.
\VS{31}Et sa concubine, qui était à Sichem, lui enfanta aussi un fils, et il le nomma Abimélec.
\VS{32}Puis Gédeon, fils de Joas, mourut en bonne vieillesse, et fut enseveli au sépulcre de Joas son père à Hophra des Abihézérites.
\VS{33}Et il arriva après que Gédeon fut mort, que les enfants d'Israël se détournèrent, et paillardèrent après les Bahalins, et s'établirent Bahal-bérith pour Dieu.
\VS{34}Ainsi les enfants d'Israël ne se souvinrent point de l'Eternel leur Dieu, qui les avait délivrés de la main de tous leurs ennemis qui les environnaient.
\VS{35}Et ils n'usèrent d'aucune gratuité envers la maison de Jérubbahal Gédeon, selon tout le bien qu'il avait fait à Israël.
\Chap{9}
\VerseOne{}Et Abimélec, fils de Jérubbahal, s'en alla à Sichem vers les frères de sa mère, et leur parla, et à toute la famille de la maison du père de sa mère, en disant :
\VS{2}Je vous prie, faites entendre ces paroles à tous les Seigneurs de Sichem ; Lequel vous semble le meilleur, ou que soixante-dix hommes, tous enfants de Jérubbahal, dominent sur vous ; ou qu'un seul homme domine sur vous ? et souvenez-vous que je suis votre os et votre chair.
\VS{3}Les frères donc de sa mère dirent de sa part toutes ces paroles, les Seigneurs de Sichem l'entendant ; et leur cœur fut incliné vers Abimélec ; car ils dirent, c'est notre frère.
\VS{4}Et ils lui donnèrent soixante-dix [pièces] d'argent [prises] de la maison de Bahal-bérith, avec lesquelles Abimélec leva des hommes n'ayant rien, et vagabonds, qui le suivirent.
\VS{5}Et il vint en la maison de son père à Hophra, et tua sur une même pierre ses frères, enfants de Jérubbahal, qui étaient soixante-dix hommes ; mais Jotham, le plus petit fils de Jérubbahal, demeura de reste, parce qu'il s'était caché.
\VS{6}Et tous les Seigneurs de Sichem s'assemblèrent, avec toute la maison de Millo, et ils vinrent, et établirent Abimélec pour Roi auprès du bois de chênes qui est en Sichem.
\VS{7}Et on le rapporta à Jotham, qui s'en alla, et se tint au sommet de la montagne de Guérizim, et élevant sa voix, il cria, et leur dit : Ecoutez-moi, Seigneurs de Sichem, et que Dieu vous entende.
\VS{8}Les arbres allèrent [un jour] en toute diligence pour oindre sur eux un Roi, et ils dirent à l'olivier : Règne sur nous.
\VS{9}Mais l'olivier leur répondit : Me ferait-on quitter ma graisse, par laquelle Dieu et les hommes sont honorés, afin que j'aille m'agiter pour les [autres] arbres ?
\VS{10}Puis les arbres dirent au figuier : Viens toi, [et] règne sur nous.
\VS{11}Et le figuier leur répondit : Me ferait-on quitter ma douceur, et mon bon fruit, afin que j'aille m'agiter pour les [autres] arbres ?
\VS{12}Puis les arbres dirent à la vigne : Viens toi, et règne sur nous.
\VS{13}Et la vigne répondit : Me ferait-on quitter mon bon vin, qui réjouit Dieu et les hommes, afin que j'aille m'agiter pour les [autres] arbres ?
\VS{14}Alors tous les arbres dirent à l'épine : Viens toi, et règne sur nous.
\VS{15}Et l'épine répondit aux arbres : Si c'est en sincérité que vous m'oignez pour Roi sur vous, venez, et retirez-vous sous mon ombre ; sinon, que le feu sorte de l'épine, et qu'il dévore les cèdres du Liban.
\VS{16}Maintenant donc, si vous avez agi avec sincérité et avec intégrité, en établissant Abimélec pour Roi, et si vous en avez bien usé envers Jérubbahal et sa maison, et si vous lui avez fait selon qu'il vous y a obligés par ses actions.
\VS{17}(Car mon père a combattu pour vous, et a exposé sa vie, et vous a délivrés de la main de Madian.
\VS{18}Mais vous vous êtes élevés aujourd'hui contre la maison de mon père, et avez tué sur une pierre ses enfants, [qui étaient] soixante-dix hommes, et avez établi pour Roi Abimélec fils de sa servante, sur les Seigneurs de Sichem, parce qu'il est votre frère.)
\VS{19}Si, dis-je, vous avez agi aujourd'hui avec sincérité et avec intégrité envers Jérubbahal, et envers sa maison, réjouissez-vous d'Abimélec, et qu'il se réjouisse aussi de vous.
\VS{20}Sinon, que le feu sorte d'Abimélec, et qu'il dévore les Seigneurs de Sichem, et la maison de Millo ; et que le feu sorte des Seigneurs de Sichem, et de la maison de Millo, et qu'il dévore Abimélec.
\VS{21}Puis Jotham s'enfuit en diligence, et s'en alla à Béer, et y demeura, à cause d'Abimélec son frère.
\VS{22}Et Abimélec domina sur Israël trois ans.
\VS{23}Mais Dieu envoya un mauvais esprit entre Abimélec et les Seigneurs de Sichem ; et les Seigneurs de Sichem furent infidèles à Abimélec.
\VS{24}Afin que la violence faite aux soixante-dix fils de Jérubbahal, et leur sang, retournât sur Abimélec leur frère, qui les avait tués ; et sur les Seigneurs de Sichem qui lui avaient tenu la main pour tuer ses frères.
\VS{25}Les Seigneurs de Sichem donc lui mirent des embûches sur le sommet des montagnes, et ils pillaient tous ceux qui passaient près d'eux par le chemin. Ce qui fut rapporté à Abimélec.
\VS{26}Alors Gahal, fils de Hébed, vint avec ses frères, et ils entrèrent dans Sichem ; et les Seigneurs de Sichem eurent confiance en lui.
\VS{27}Puis étant sortis aux champs ils vendangèrent leurs vignes, et en foulèrent [les raisins], et firent bonne chère ; et ils entrèrent dans la maison de leur Dieu, et ils mangèrent, et burent, et maudirent Abimélec.
\VS{28}Alors Gahal, fils de Hébed, dit : Qui est Abimélec, et quelle est Sichem, que nous servions Abimélec ? [N'est-il pas] fils de Jérubbahal ? Et Zébul [n'est-il pas] son prévôt ? Servez [plutôt] les hommes de Hémor, père de Sichem. Mais pour quelle raison servirions-nous celui-ci ?
\VS{29}Plût à Dieu qu'on me donnât ce peuple sous ma conduite, et je chasserais Abimélec. Et il dit à Abimélec : Multiplie ton armée, et sors.
\VS{30}Et Zébul, capitaine de la ville, entendit les paroles de Gahal, fils de Hébed, et sa colère s'enflamma.
\VS{31}Puis il envoya adroitement des messagers vers Abimélec, pour lui dire : Voici, Gahal fils de Hébed, et ses frères, sont entrés dans Sichem ; et voici, ils arment la ville contre toi.
\VS{32}Maintenant donc lève-toi de nuit, toi, et le peuple qui est avec toi, et mets des embûches aux champs.
\VS{33}Et au matin, environ le soleil levant, tu te lèveras de matin, et te jetteras sur la ville ; et voici, [Gahal] et le peuple qui est avec lui, sortiront contre toi, et tu lui feras selon que tu en trouveras le moyen.
\VS{34}Abimélec donc se leva de nuit, et tout le peuple qui était avec lui, et ils mirent des embûches contre Sichem, et [les partagèrent] en quatre bandes.
\VS{35}Alors Gahal, fils de Hébed, sortit, et s'arrêta à l'entrée de la porte de la ville ; et Abimélec, et tout le peuple qui était avec lui se levèrent de l'embuscade.
\VS{36}Et Gahal ayant aperçu ce peuple-là, dit à Zébul : Voici du peuple qui descend du sommet des montagnes. Et Zébul lui dit : Tu vois l'ombre des montagnes, comme si c'étaient des hommes.
\VS{37}Et Gahal parla encore, et dit : Voilà du peuple qui descend du milieu du pays, et une bande vient du chemin du bois de chênes des devins.
\VS{38}Et Zébul lui dit : Où est maintenant ta vanterie, quand tu disais : Qui est Abimélec, que nous le servions ? N'est-ce pas ici ce peuple que tu as méprisé ? Sors maintenant, je te prie, et combats contr'eux.
\VS{39}Alors Gahal sortit conduisant les Seigneurs de Sichem, et combattit contre Abimélec.
\VS{40}Et Abimélec le poursuivit, comme il s'enfuyait de devant lui, et plusieurs tombèrent morts jusqu'à l'entrée de la porte.
\VS{41}Et Abimélec s'arrêta à Aruma ; et Zébul repoussa Gahal et ses frères, afin qu'ils ne demeurassent plus dans Sichem.
\VS{42}Et il arriva dès le lendemain que le peuple sortit aux champs ; ce qui fut rapporté à Abimélec ;
\VS{43}Lequel prit du peuple, et le divisa en trois bandes, et les mit en embuscade dans les champs, et ayant aperçu que le peuple sortait de la ville, il se leva contr'eux, et les défit.
\VS{44}car Abimélec, et la bande qui était avec lui se répandirent, et se tinrent à l'entrée de la porte de la ville, mais les deux autres bandes se jetèrent sur tous ceux qui étaient aux champs, et les défirent.
\VS{45}Ainsi Abimélec combattit tout ce jour-là contre la ville, et prit la ville, et tua le peuple qui y était, et ayant rasé la ville, y sema du sel.
\VS{46}Et tous les Seigneurs de la Tour de Sichem, ayant appris cela, se retirèrent dans le fort, qui était la maison du Dieu Bérith.
\VS{47}Et on rapporta à Abimélec que tous les Seigneurs de la Tour de Sichem s'étaient assemblés [dans le fort].
\VS{48}Alors Abimélec monta sur le mont Tsalmon, lui et tout le peuple qui était avec lui ; et Abimélec prit une hache et coupa une branche d'arbre, et l'ayant mise sur son épaule, la porta, et dit au peuple qui était avec lui : Avez-vous vu ce que j'ai fait ? dépêchez-vous, faites comme moi.
\VS{49}Chacun donc de tout le peuple coupa une branche, et ils suivirent Abimélec, et mirent [ces branches] tout autour du fort, et y ayant mis le feu, ils brûlèrent le fort. Et toutes les personnes de la Tour de Sichem moururent, au nombre d'environ mille, tant hommes que femmes.
\VS{50}Puis Abimélec s'en allant à Tébets, y mit son camp, et la prit.
\VS{51}Or il y avait au milieu de la ville une Tour forte, où s'enfuirent tous les hommes et toutes les femmes, et tous les Seigneurs de la ville, et ayant fermé les portes après eux, ils montèrent sur le toit de la Tour.
\VS{52}Alors Abimélec venant jusqu'à la Tour, l'attaqua, et s'approcha jusqu'à la porte de la Tour pour la brûler par feu.
\VS{53}Mais une femme jeta une pièce de meule sur la tête d'Abimélec, et lui cassa le crâne.
\VS{54}Lequel ayant appelé incessamment le garçon qui portait ses armes, lui dit : Tire ton épée, et me tue, de peur qu'on ne dise de moi, une femme l'a tué. Son garçon donc le transperça, et il mourut.
\VS{55}Et ceux d'Israël voyant qu'Abimélec était mort, s'en allèrent chacun en son lieu.
\VS{56}Ainsi Dieu rendit à Abimélec le mal qu'il avait commis contre son père, en tuant ses soixante-dix frères ;
\VS{57}Et toute la méchanceté des hommes de Sichem ; Dieu, [dis-je], la fit retourner sur leurs têtes ; et ainsi la malédiction de Jotham, fils de Jérubbahal, vint sur eux.
\Chap{10}
\VerseOne{}Après Abimélec, Tolah fils de Puah, fils de Dodo, homme d'Issacar, fut suscité pour délivrer Israël, et il habitait à Samir en la montagne d'Ephraïm.
\VS{2}Et il jugea Israël vingt-trois ans, puis il mourut, et fut enseveli à Samir.
\VS{3}Après lui fut suscité Jaïr Galaadite, qui jugea Israël vingt-deux ans.
\VS{4}Et il eut trente fils, qui montaient sur trente ânons, et qui avaient trente villes, qu'on appelle les villes de Jaïr jusqu'à ce jour, lesquelles sont au pays de Galaad.
\VS{5}Et Jaïr mourut, et fut enseveli à Kamon.
\VS{6}Puis les enfants d'Israël recommencèrent à faire ce qui déplaît à l'Eternel, et servirent les Bahalins, et Hastaroth, savoir, les dieux de Syrie, les dieux de Sidon, les dieux de Moab, les dieux des enfants de Hammon, et les dieux des Philistins ; et ils abandonnèrent l'Eternel, et ne le servaient plus.
\VS{7}Alors la colère de l'Eternel s'enflamma contre Israël, et il les vendit en la main des Philistins, et en la main des enfants de Hammon ;
\VS{8}Qui opprimèrent et foulèrent les enfants d'Israël cette année-là, qui était la dix-huitième ; [savoir] tous les enfants d'Israël, qui étaient au-delà du Jourdain au pays des Amorrhéens, qui est en Galaad.
\VS{9}Même les enfants de Hammon passèrent le Jourdain pour combattre aussi contre Juda, et contre Benjamin, et contre la maison d'Ephraïm ; et Israël fut fort serré.
\VS{10}Alors les enfants d'Israël crièrent à l'Eternel, en disant : Nous avons péché contre toi, et certes nous avons abandonné notre Dieu, et nous avons servi les Bahalins.
\VS{11}Mais l'Eternel répondit aux enfants d'Israël : N'avez-vous pas été opprimés par les Egyptiens, les Amorrhéens, les enfants de Hammon, les Philistins,
\VS{12}Les Sidoniens, les Hamalécites, et les Mahonites ? cependant quand vous avez crié vers moi, je vous ai délivrés de leurs mains.
\VS{13}Mais vous m'avez abandonné, et vous avez servi d'autres dieux ; c'est pourquoi je ne vous délivrerai plus.
\VS{14}Allez, et criez aux dieux que vous avez choisis ; qu'ils vous délivrent au temps de votre détresse.
\VS{15}Mais les enfants d'Israël répondirent à l'Eternel : Nous avons péché, fais-nous, comme il te semblera bon ; nous te prions seulement que tu nous délivres aujourd'hui.
\VS{16}Alors ils ôtèrent du milieu d'eux les dieux des étrangers, et servirent l'Eternel, qui fut touché en son cœur de l'affliction d'Israël.
\VS{17}Or les enfants de Hammon s'assemblèrent, et se campèrent en Galaad ; et les enfants d'Israël aussi s'assemblèrent, et se campèrent à Mitspa.
\VS{18}Et le peuple, [et] les principaux de Galaad dirent l'un à l'autre : Qui sera l'homme qui commencera à combattre contre les enfants de Hammon ? il sera pour chef à tous les habitants de Galaad.
\Chap{11}
\VerseOne{}Or Jephthé Galaadite était un fort et vaillant homme, mais fils d'une paillarde, toutefois Galaad l'avait engendré.
\VS{2}Et la femme de Galaad lui avait enfanté des fils ; et quand les fils de cette femme-là furent grands, ils chassèrent Jephthé, en lui disant : Tu n'auras point d'héritage dans la maison de notre père ; car tu es fils d'une femme étrangère.
\VS{3}Jephthé donc s'enfuit de devant ses frères, et habita au pays de Tob ; et des gens qui n'avaient rien se ramassèrent auprès de Jephthé, et ils allaient et venaient avec lui.
\VS{4}Or il arriva, quelque temps après, que les enfants de Hammon firent la guerre à Israël.
\VS{5}Et comme les enfants de Hammon faisaient la guerre à Israël, les Anciens de Galaad s'en allèrent pour ramener Jephthé du pays de Tob.
\VS{6}Et ils dirent à Jephthé : Viens, et sois notre capitaine, afin que nous combattions contre les enfants de Hammon.
\VS{7}Et Jephthé répondit aux Anciens de Galaad : N'est-ce pas vous qui m'avez haï, et chassé de la maison de mon père ? et pourquoi êtes vous venus à moi maintenant que vous êtes dans l'affliction ?
\VS{8}Alors les Anciens de Galaad dirent à Jephthé : La raison pourquoi nous sommes maintenant retournés à toi, c'est afin que tu viennes avec nous, et que tu combattes contre les enfants de Hammon, et que tu sois notre chef, [savoir] de nous tous qui habitons à Galaad.
\VS{9}Et Jephthé répondit aux Anciens de Galaad : Si vous me ramenez pour combattre contre les enfants de Hammon, et que l'Eternel les livre entre mes mains, je serai votre chef.
\VS{10}Et les Anciens de Galaad dirent à Jephthé : Que l'Eternel écoute entre nous, si nous ne faisons selon tout ce que tu as dit.
\VS{11}Jephthé donc s'en alla avec les Anciens de Galaad, et le peuple l'établit sur soi pour chef, et pour capitaine ; et Jephthé prononça devant l'Eternel à Mitspa toutes les paroles qu'il avait dites.
\VS{12}Puis Jephthé envoya des messagers au Roi des enfants de Hammon pour lui dire : Qu'y a-t-il entre toi et moi, que tu sois venu contre moi pour faire la guerre en mon pays ?
\VS{13}Et le Roi des enfants de Hammon répondit aux messagers de Jephthé : C'est parce qu'Israël a pris mon pays quand il montait d'Egypte, depuis Arnon jusqu'à Jabbok, même jusqu'au Jourdain ; maintenant donc rends-moi ces contrées-là à l'amiable.
\VS{14}Mais Jephthé envoya encore des messagers au Roi des enfants de Hammon ;
\VS{15}Qui lui dirent : Ainsi a dit Jephthé : Israël n'a rien pris du pays de Moab, ni du pays des enfants de Hammon.
\VS{16}Mais après qu'Israël, étant monté d'Egypte, fut venu par le désert jusqu'à la mer Rouge et fut parvenu à Kadès ;
\VS{17}Et qu'il eut envoyé des messagers au Roi d'Edom, pour lui dire : Que je passe, je te prie, par ton pays ; à quoi le Roi d'Edom ne voulut point entendre ; et qu'il eut aussi envoyé au Roi de Moab, qui ne le voulut point non plus [entendre] ; et après qu'Israël ayant demeuré à Kadès,
\VS{18}Et ayant marché par le désert, eut fait le tour du pays d'Edom, et du pays de Moab, et fut arrivé au pays de Moab du côté d'Orient, il se campa au delà d'Arnon, et n'entra point dans les frontières de Moab ; parce qu'Arnon était la frontière de Moab.
\VS{19}Mais Israël envoya des messagers à Sihon, Roi des Amorrhéens, qui était Roi de Hesbon, auquel Israël fit dire : Nous te prions, que nous passions par ton pays, jusqu'à notre lieu.
\VS{20}Mais Sihon ne se fiant point à Israël pour le laisser passer par son pays, assembla tout son peuple, et ils campèrent vers Jahats, et combattirent contre Israël.
\VS{21}Et l'Eternel le Dieu d'Israël livra Sihon et tout son peuple entre les mains d'Israël, et Israël les défit, et conquit tout le pays des Amorrhéens qui habitaient en ce pays-là.
\VS{22}Ils conquirent donc tout le pays des Amorrhéens depuis Arnon jusqu'à Jabbok, et depuis le désert jusqu'au Jourdain.
\VS{23}Or maintenant que l'Eternel le Dieu d'Israël a dépossédé les Amorrhéens de devant son peuple d'Israël, en aurais-tu la possession ?
\VS{24}N'aurais-tu pas la possession de ce que Kémos ton dieu t'aurait donné à posséder ? Ainsi nous posséderons le pays de tous ceux que l'Eternel notre Dieu aura chassés de devant nous.
\VS{25}Or maintenant vaux-tu mieux en quelque sorte que ce soit que Balac, fils de Tsippor, Roi de Moab ? Et lui n'a-t-il pas contesté et combattu autant qu'il a pu contre Israël ?
\VS{26}Pendant qu'Israël a demeuré à Hesbon, et dans les villes de son ressort, et à Haroher, et dans les villes de son ressort, et dans toutes les villes qui sont le long d'Arnon, l'espace de trois cents ans, pourquoi ne les avez-vous pas recouvrées pendant ce temps-là ?
\VS{27}Je ne t'ai donc point offensé, mais tu fais une méchante action de me faire la guerre. Que l'Eternel, qui est le Juge, juge aujourd'hui entre les enfants d'Israël et les enfants de Hammon.
\VS{28}Mais le Roi des enfants de Hammon ne voulut point écouter les paroles que Jephthé lui avait fait dire.
\VS{29}L'Esprit de l'Eternel fut donc sur Jephthé, qui passa au travers de Galaad et de Manassé ; et il passa jusqu'à Mitspé de Galaad, et de Mitspé de Galaad il passa jusqu'aux enfants de Hammon.
\VS{30}Et Jephthé voua un vœu à l'Eternel, et dit : Si tu livres les enfants de Hammon en ma main ;
\VS{31}Alors tout ce qui sortira des portes de ma maison au devant de moi, quand je retournerai en paix [du pays] des enfants de Hammon, sera à l'Eternel, et je l'offrirai en holocauste.
\VS{32}Jephthé donc passa jusques où étaient les enfants de Hammon pour combattre contr'eux ; et l'Eternel les livra en sa main.
\VS{33}Et il en fit un très-grand carnage, depuis Haroher jusqu'à Minnith, en vingt villes, et jusqu'à la plaine des vignes ; et les enfants de Hammon furent humiliés devant les enfants d'Israël.
\VS{34}Puis comme Jephthé venait à Mitspa en sa maison, voici sa fille, qui était seule et unique, sans qu'il eût d'autre fils, ou fille, sortit au devant de lui avec tambours et flûtes.
\VS{35}Et il arriva qu'aussitôt qu'il l'eut aperçue, il déchira ses vêtements, et dit : Ha ! ma fille, tu m'as entièrement abaissé, et tu es du nombre de ceux qui me troublent ; car j'ai ouvert ma bouche à l'Eternel, et je ne m'en pourrai point rétracter.
\VS{36}Et elle répondit : Mon père, as-tu ouvert ta bouche à l'Eternel, fais-moi selon ce qui est sorti de ta bouche, puisque l'Eternel t'a vengé de tes ennemis, les enfants de Hammon.
\VS{37}Toutefois elle dit à son père : Que ceci me soit accordé ; laisse-moi pour deux mois, afin que je m'en aille, et que je descende par les montagnes, et que je pleure ma virginité, moi et mes compagnes.
\VS{38}Et il dit : Va, et il la laissa aller pour deux mois. Elle s'en alla donc avec ses compagnes, et pleura sa virginité dans les montagnes.
\VS{39}Et au bout de deux mois elle retourna vers son père ; et il lui fit selon le vœu qu'il avait voué. Or elle n'avait point connu d'homme. Et ce fut une coutume en Israël,
\VS{40}Que d'an en an les filles d'Israël allaient pour lamenter la fille de Jephthé Galaadite, quatre jours en l'année.
\Chap{12}
\VerseOne{}Or les hommes d'Ephraïm s'étant assemblés, passèrent vers le Septentrion, et dirent à Jephthé : Pourquoi es-tu passé pour combattre contre les enfants de Hammon, et que tu ne nous as point appelés pour aller avec toi ? Nous brûlerons au feu ta maison, et te [brûlerons] aussi.
\VS{2}Et Jephthé leur dit : J'ai eu un grand différend avec les enfants de Hammon, moi et mon peuple, et quand je vous ai appelés, vous ne m'avez point délivré de leurs mains.
\VS{3}Et voyant que vous ne me délivriez pas, j'ai exposé ma vie, et je suis passé jusqu'où étaient les enfants de Hammon, et l'Eternel les a livrés en ma main ; pourquoi donc êtes-vous montés aujourd'hui vers moi pour me faire la guerre ?
\VS{4}Puis Jephthé ayant assemblé tous les gens de Galaad, combattit contre Ephraïm ; et ceux de Galaad battirent Ephraïm, parce qu'ils avaient dit : Vous [êtes] des échappés d'Ephraïm, Galaad [est] au milieu d'Ephraïm, au milieu de Manassé.
\VS{5}Et les Galaadites se saisirent des passages du Jourdain avant que ceux d'Ephraïm y arrivassent ; et quand quelqu'un de ceux d'Ephraïm qui étaient échappés, disait : Que je passe ; les gens de Galaad lui disaient : Es-tu Ephratien ? et il répondait : Non.
\VS{6}Alors ils lui disaient : Dis un peu Schibboleth, et il disait Sibboleth, et ne pouvait point prononcer [Schibboleth] ; sur quoi se saisissant de lui ils le mettaient à mort au passage du Jourdain. Et en ce temps-là il y eut quarante-deux mille hommes d'Ephraïm qui furent tués.
\VS{7}Et Jephthé jugea Israël six ans ; puis Jephthé Galaadite mourut, et fut enseveli en [une] des villes de Galaad.
\VS{8}Après lui Ibtsan de Bethléhem jugea Israël.
\VS{9}Et il eut trente fils et trente filles, lesquelles il mit hors [de sa maison, en les mariant], et il prit de dehors trente filles pour ses fils ; et il jugea Israël sept ans.
\VS{10}Puis Ibtsan mourut, et fut enseveli à Bethléhem.
\VS{11}Après lui Elon Zabulonite jugea Israël, dix ans.
\VS{12}Puis Elon Zabulonite mourut et fut enseveli à Ajalon, dans la terre de Zabulon.
\VS{13}Après lui Habdon, fils d'Hillel, Pirhathonite jugea Israël.
\VS{14}Il eut quarante fils, et trente petits-fils, qui montaient sur soixante-dix ânons ; et il jugea Israël huit ans.
\VS{15}Puis Habdon, fils d'Hillel, Pirhathonite mourut, et fut enseveli à Pirhathon, en la terre d'Ephraïm, sur la montagne de l'Hamalécite.
\Chap{13}
\VerseOne{}Et les enfants d'Israël recommencèrent à faire ce qui déplaît à l'Eternel, et l'Eternel les livra entre les mains des Philistins pendant quarante ans.
\VS{2}Or il y avait un homme de Tsorah, d'une famille de ceux de Dan, dont le nom était Manoah, et sa femme était stérile, et n'avait [jamais] eu d'enfant.
\VS{3}Et l'Ange de l'Eternel apparut à cette femme-là, et lui dit : Voici, tu es stérile, et tu n'as [jamais] eu d'enfant ; mais tu concevras, et enfanteras un fils.
\VS{4}Prends donc bien garde dès maintenant de ne point boire de vin ni de cervoise, et de ne manger aucune chose souillée.
\VS{5}Car voici, tu vas être enceinte, et tu enfanteras un fils, et le rasoir ne passera point sur sa tête ; parce que l'enfant sera Nazarien de Dieu dès le ventre [de sa mère] ; et ce sera lui qui commencera à délivrer Israël de la main des Philistins.
\VS{6}Et la femme vint, et parla à son mari, en disant : Il est venu auprès de moi un homme de Dieu, dont la face est semblable à la face d'un Ange de Dieu, fort vénérable, mais je ne l'ai point interrogé d'où il était et il ne m'a point déclaré son nom.
\VS{7}Mais il m'a dit : Voici, tu vas être enceinte, et tu enfanteras un fils. Maintenant donc ne bois point de vin ni de cervoise, et ne mange aucune chose souillée ; car cet enfant sera Nazarien de Dieu dès le ventre [de sa mère] jusqu'au jour de sa mort.
\VS{8}Et Manoah pria instamment l'Eternel, et dit : Hélas, Seigneur ! que l'homme de Dieu que tu as envoyé, vienne encore, je te prie, vers nous, et qu'il nous enseigne ce que nous devons faire à l'enfant, quand il sera né.
\VS{9}Et Dieu exauça la prière de Manoah. Ainsi l'Ange de Dieu vint encore à la femme comme elle était assise dans un champ ; mais Manoah son mari n'était point avec elle.
\VS{10}Et la femme courut vite le rapporter à son mari, en lui disant : Voici, l'homme qui était venu l'autre jour vers moi, m'est apparu.
\VS{11}Et Manoah se leva, et suivit sa femme ; et venant vers l'homme, il lui dit : Es-tu cet homme qui a parlé à cette femme-ci ? Et il répondit : C'est moi.
\VS{12}Et Manoah dit : Tout ce que tu as dit arrivera ; [mais] quel ordre faudra-t-il tenir envers l'enfant, et que lui faudra-t-il faire ?
\VS{13}Et l'Ange de l'Eternel répondit à Manoah : La femme se gardera de toutes les choses dont je l'ai avertie.
\VS{14}Elle ne mangera rien qui sorte de la vigne, [rien en quoi il y ait] du vin ; et elle ne boira ni vin ni cervoise, et ne mangera aucune chose souillée. Elle prendra garde à tout ce que je lui ai commandé.
\VS{15}Alors Manoah dit à l'Ange de l'Eternel : Je te prie, que nous te retenions, et nous t'apprêterons un chevreau de lait.
\VS{16}Et l'Ange de l'Eternel répondit à Manoah : Quand tu me retiendrais je ne mangerais point de ton pain ; mais si tu fais un holocauste, tu l'offriras à l'Eternel. Or Manoah ne savait point que ce fût l'Ange de l'Eternel.
\VS{17}Et Manoah dit à l'Ange de l'Eternel : Quel est ton nom, afin que nous te fessions un présent lorsque ce que tu as dit sera arrivé ?
\VS{18}Et l'Ange de l'Eternel lui dit : Pourquoi t'enquiers-tu ainsi de mon nom ? car il est admirable.
\VS{19}Alors Manoah prit un chevreau de lait, et un gâteau, et les offrit à l'Eternel sur le rocher. Et l'Ange fit une chose merveilleuse à la vue de Manoah et de sa femme.
\VS{20}C'est, que la flamme montant de dessus l'autel vers les cieux, l'Ange de l'Eternel monta aussi avec la flamme de l'autel ; ce que Manoah et sa femme ayant vu ils se prosternèrent le visage contre terre.
\VS{21}Et l'Ange de l'Eternel n'apparut plus à Manoah ni à sa femme. Alors Manoah connut que c'était l'Ange de l'Eternel.
\VS{22}Et Manoah dit à sa femme : Certainement nous mourrons, parce que nous avons vu Dieu.
\VS{23}Mais sa femme lui répondit : Si l'Eternel nous eût voulu faire mourir, il n'aurait pas pris de notre main l'holocauste ni le gâteau, et il ne nous aurait pas fait voir toutes ces choses, en un temps comme celui-ci, ni fait entendre les choses que nous avons entendues.
\VS{24}Puis cette femme enfanta un fils, et elle l'appela Samson ; et l'enfant devint grand, et l'Eternel le bénit.
\VS{25}Et l'Esprit de l'Eternel commença de le saisir à Mahané-dan, entre Tsorah et Estaol.
\Chap{14}
\VerseOne{}Or Samson étant descendu à Timna, y vit une femme d'entre les filles des Philistins.
\VS{2}Et étant remonté [en sa maison] il le déclara à son père et à sa mère, en disant : J'ai vu une femme à Timna d'entre les filles des Philistins ; maintenant donc prenez-la, afin qu'elle soit ma femme.
\VS{3}Et son père et sa mère lui dirent : N'y a-t-il point de femme parmi les filles de tes frères, et parmi tout mon peuple, que tu ailles prendre une femme d'entre les Philistins, ces incirconcis ? Et Samson dit à son père : Prenez-la moi, car elle plaît à mes yeux.
\VS{4}Mais son père et sa mère ne savaient pas que cela [venait] de l'Eternel ; car [Samson] cherchait que les Philistins lui donnassent quelque occasion. Or en ce temps-là les Philistins dominaient sur Israël.
\VS{5}Samson donc descendit avec son père et sa mère à Timna, et ils vinrent jusqu'aux vignes de Timna, et voici, un jeune lion rugissant [venait] contre lui.
\VS{6}Et l'Esprit de l'Eternel ayant saisi Samson, il déchira le lion comme s'il eût déchiré un chevreau, sans avoir rien en sa main ; mais il ne déclara point à son père ni à sa mère ce qu'il avait fait.
\VS{7}Il descendit donc, et parla à la femme, et elle lui plut.
\VS{8}Puis retournant quelques jours après pour la prendre, il se détourna pour voir la charogne du lion, et voici il y avait dans la charogne du lion un essaim d'abeilles, et du miel.
\VS{9}Et il en prit en sa main, et s'en alla son chemin, en mangeant ; et étant arrivé vers son père, et vers sa mère, il leur en donna, et ils [en] mangèrent ; mais il ne leur déclara pas qu'il avait pris ce miel dans la charogne du lion.
\VS{10}Son père donc descendit vers cette femme, et Samson fit là un festin ; car c'est ainsi que les jeunes gens avaient accoutumé de faire.
\VS{11}Et sitôt qu'on l'eut vu, on prit trente compagnons, qui furent avec lui.
\VS{12}Et Samson leur dit : Je vous proposerai maintenant une énigme ; et si vous me l'expliquez pendant les sept jours du festin, et la trouvez, je vous donnerai trente linges, [savoir] trente robes de rechange.
\VS{13}Mais si vous ne me l'expliquez pas, vous me donnerez trente linges, [savoir] trente robes de rechange. Et ils lui répondirent : Propose ton énigme, et nous l'entendrons.
\VS{14}Et il leur dit : De celui qui dévorait est procédée la viande, et du fort est procédée la douceur. Mais ils ne purent en trois jours expliquer l'énigme.
\VS{15}Et au septième jour, ils dirent à la femme de Samson : Persuade à ton mari de nous déclarer l'énigme, de peur que nous ne te brûlions au feu, toi et la maison de ton père. Nous avez-vous appelés ici pour avoir notre bien n'[est-il pas ainsi ?]
\VS{16}La femme de Samson donc pleura auprès de lui, et dit : Certainement tu me hais, et tu ne m'aimes point ; n'as-tu pas proposé une énigme aux enfants de mon peuple, et tu ne me l'as point déclarée ? Et il lui répondit : Voici, je ne l'ai point déclarée à mon père ni à ma mère, et te la déclarerais-je ?
\VS{17}Elle pleurait ainsi auprès de lui durant les sept jours du festin, mais au septième jour il la lui déclara, parce qu'elle le tourmentait ; puis elle la déclara aux enfants de son peuple.
\VS{18}Les gens de la ville donc lui dirent au septième jour, avant que le soleil se couchât : Qu y a-t-il de plus doux que le miel, et qu'y a-t-il de plus fort que le lion ? Et il leur dit : Si vous n'eussiez point labouré avec ma génisse vous n'eussiez point trouvé mon énigme.
\VS{19}Et l'Esprit de l'Eternel le saisit, et il descendit à Askelon, et ayant tué trente hommes de ceux d'Askelon, il prit leurs dépouilles, et donna les robes de rechange à ceux qui avaient expliqué l'énigme ; et sa colère s'enflamma, et il monta en la maison de son père.
\VS{20}Et la femme de Samson fut [mariée] à son compagnon, qui était son intime ami.
\Chap{15}
\VerseOne{}Or il arriva quelques jours après, au temps de la moisson des bleds, que Samson alla visiter sa femme, [lui] portant un chevreau de lait, et il dit : J'entrerai vers ma femme en sa chambre ; mais son père ne lui permit point d'y entrer ;
\VS{2}Car il lui dit : J'ai cru que tu avais certainement de l'aversion pour elle, c'est pourquoi je l'ai donnée à ton compagnon. Sa sœur puînée n'est-elle pas plus belle qu'elle ? je te prie donc qu'elle soit ta [femme] au lieu d'elle.
\VS{3}Et Samson leur dit : A présent je serai innocent à l'égard des Philistins quand je leur ferai du mal.
\VS{4}Samson donc s'en alla, et prit trois cents renards ; il prit aussi des flambeaux, et il tourna les renards queue contre queue, et mit un flambeau entre les deux queues, tout au milieu.
\VS{5}Puis il mit le feu aux flambeaux, et lâcha les [renards] aux bleds des Philistins qui étaient sur le pied ; et il brûla tant le bled qui était en gerbes, que celui qui était sur le pied, même jusqu'aux vignes et aux oliviers.
\VS{6}Et les Philistins dirent : Qui a fait cela ? Et on répondit : Samson, le beau-fils du Timnien, parce qu'il lui a pris sa femme, et qu'il l'a donnée à son compagnon. Les Philistins donc montèrent, et la brûlèrent au feu, avec son père.
\VS{7}Alors Samson leur dit : Est-ce donc ainsi que vous faites ? Cependant je me vengerai de vous avant que je cesse.
\VS{8}Et il les battit entièrement, et en fit un grand carnage ; puis il descendit, et s'arrêta dans un quartier du rocher de Hétam.
\VS{9}Alors les Philistins montèrent, et se campèrent en Juda, et se répandirent en Léhi.
\VS{10}Et les hommes de Juda dirent : Pourquoi êtes-vous montés contre nous ? Ils répondirent : Nous sommes montés pour lier Samson, afin que nous lui fassions comme il nous a fait.
\VS{11}Alors trois mille hommes de Juda descendirent vers le quartier du rocher de Hétam, et dirent à Samson : Ne sais-tu pas que les Philistins dominent sur nous ; pourquoi donc nous as-tu fait ceci ? Il leur répondit : Je leur ai fait comme ils m'ont fait.
\VS{12}Ils lui dirent encore : Nous sommes descendus pour te lier, afin de te livrer entre les mains des Philistins. Et Samson leur dit : Jurez-moi que vous ne vous jetterez point sur moi.
\VS{13}Et ils répondirent, et dirent : Non, mais nous te lierons très-bien, afin de te livrer entre leurs mains ; mais nous ne te tuerons point. Ils le lièrent donc de deux cordes neuves, et le firent monter hors du rocher.
\VS{14}Or quand il fut venu jusqu'à Léhi, les Philistins jetèrent des cris de joie à sa rencontre, et l'esprit de l'Eternel le saisit, et les cordes qui étaient sur ses bras, devinrent comme du lin où l'on a mis le feu, et ses liens s'écoulèrent de dessus ses mains.
\VS{15}Et ayant trouvé une mâchoire d'âne qui n'était pas encore desséchée, il avança sa main, la prit, et il en tua mille hommes.
\VS{16}Puis Samson dit : Avec une mâchoire d'âne, un monceau, deux monceaux ; avec une mâchoire d'âne j'ai tué mille hommes.
\VS{17}Et quand il eut achevé de parler, il jeta de sa main la mâchoire, et nomma ce lieu-là Ramath-léhi.
\VS{18}Et il eut une fort grande soif, et il cria à l'Eternel en disant : Tu as mis en la main de ton serviteur cette grande délivrance, et maintenant mourrais-je de soif, et tomberais-je entre les mains des incirconcis ?
\VS{19}Alors Dieu fendit une des grosses dents de cette mâchoire d'âne, et il en sortit de l'eau ; et quand [Samson] eut bu, l'esprit lui revint, et il reprit ses forces ; : c'est pourquoi ce lieu-là a été appelé jusqu'à ce jour Hen-hakkoré, qui est à Léhi.
\VS{20}Or [Samson] jugea Israël au temps des Philistins, vingt ans.
\Chap{16}
\VerseOne{}Or Samson s'en alla à Gaza, et vit là une femme paillarde, et alla vers elle.
\VS{2}Et on dit à ceux de Gaza : Samson est venu ici ; ils l'environnèrent, et lui dressèrent une embuscade toute la nuit à la porte de la ville, et ils se tinrent tranquilles toute la nuit, en disant : [Qu'on ne bouge point] jusqu'au point du jour, et nous le tuerons.
\VS{3}Mais Samson après avoir dormi jusqu'à la minuit, se leva, et se saisit des portes de la ville, et des deux poteaux, et les ayant enlevés avec la barre, il les mit sur ses épaules, et les porta sur le sommet de la montagne qui est vis-à-vis de Hébron.
\VS{4}Après cela il aima une femme [qui se tenait] près du torrent de Sorek, le nom de laquelle était Délila.
\VS{5}Et les Gouverneurs des Philistins montèrent vers elle, et lui dirent : Persuade-le jusqu'à ce que tu saches de lui en quoi consiste sa grande force, et comment nous le surmonterions, afin que nous le liions pour l'abattre ; et nous te donnerons chacun onze cents [pièces] d'argent.
\VS{6}Délila donc dit à Samson : Déclare-moi, je te prie, en quoi consiste ta grande force, et avec quoi tu serais bien lié, pour t'abattre.
\VS{7}Et Samson lui répondit : Si on me liait de sept cordes fraîches, qui ne fussent point encore sèches, je deviendrais sans force, et je serais comme un autre homme.
\VS{8}Les Gouverneurs donc des Philistins lui envoyèrent sept cordes fraîches qui n'étaient point encore sèches, et elle l'en lia.
\VS{9}Or il y avait chez elle dans une chambre des gens qui étaient en embûches, et elle lui dit : Les Philistins sont sur toi, Samson. Alors il rompit les cordes, comme se romprait un filet d'étoupes dès qu'il sent le feu, et sa force ne fut point connue.
\VS{10}Puis Délila dit à Samson : Voici tu t'es moqué de moi, car tu m'as dit des mensonges ; je te prie, déclare-moi maintenant avec quoi tu pourrais être bien lié.
\VS{11}Et il lui répondit : Si on me liait serré de courroies neuves, dont on ne se serait jamais servi, je deviendrais sans force, et je serais comme un autre homme.
\VS{12}Délila donc prit des courroies neuves, et elle l'en lia ; puis elle lui dit : Les Philistins sont sur toi, Samson. Or il y avait des gens en embûches dans la chambre : et il rompit les courroies de dessus ses bras comme un filet.
\VS{13}Puis Délila dit à Samson : Tu t'es moqué de moi jusqu'ici, et tu m'as dit des mensonges ; déclare-moi avec quoi tu serais bien lié. Et il dit : [Ce serait] si tu avais tissu sept tresses de ma tête autour d'une ensuble.
\VS{14}Et elle les mit [dans l'ensuble] avec l'attache, puis elle dit : Les Philistins sont sur toi, Samson. Alors il se réveilla de son sommeil, et enleva l'attache de la tissure avec l'ensuble.
\VS{15}Alors elle lui dit : Comment dis-tu : Je t'aime, puisque ton cœur n'est point avec moi ? Tu t'es moqué de moi trois fois, et tu ne m'as point déclaré en quoi consiste ta grande force.
\VS{16}Et elle le tourmentait tous les jours par ses paroles, et le pressait vivement, tellement que son âme en fut affligée jusqu'à la mort.
\VS{17}Alors il lui ouvrit tout son cœur, et lui dit : Le rasoir n'a jamais passé sur ma tête ; car je suis Nazarien de Dieu dès le ventre de ma mère ; si je suis rasé, ma force m'abandonnera, je me trouverai sans force, et je serai comme tous les [autres] hommes.
\VS{18}Délila donc voyant qu'il lui avait ouvert tout son cœur, envoya appeler les Gouverneurs des Philistins, et leur dit : Montez à cette fois ; car il m'a ouvert tout son cœur. Les Gouverneurs donc des Philistins montèrent vers elle, portant l'argent en leurs mains.
\VS{19}Et elle l'endormit sur ses genoux, et ayant appelé un homme, elle lui fit raser sept tresses des cheveux de sa tête, et commença à l'abattre, et sa force l'abandonna.
\VS{20}Alors elle dit : Les Philistins sont sur toi, Samson. Et il s'éveilla de son sommeil, disant [en lui-même] : J'en sortirai comme les autres fois, et je me tirerai [de leurs mains] ; mais il ne savait pas que l'Eternel s'était retiré de lui.
\VS{21}Les Philistins donc le saisirent, et lui crevèrent les yeux, et le menèrent à Gaza, et le lièrent de deux chaînes d'airain ; et il tournait la meule dans la prison.
\VS{22}Et les cheveux de sa tête commencèrent à revenir comme ils étaient lorsqu'il fut rasé.
\VS{23}Or les Gouverneurs des Philistins s'assemblèrent pour offrir un grand sacrifice à Dagon leur dieu, et pour se réjouir, et ils dirent : Notre dieu a livré en nos mains Samson notre ennemi.
\VS{24}Le peuple aussi l'ayant vu, loua son dieu, en disant : Notre dieu a livré entre nos mains notre ennemi, et le destructeur de notre pays, et celui qui en a tant tué d'entre nous.
\VS{25}Or comme ils avaient le cœur joyeux, ils dirent : Faites venir Samson, afin qu'il nous fasse rire. Ils appelèrent donc Samson, et ils le tirèrent de la prison ; il se jouait devant eux ; et ils le firent tenir entre les piliers.
\VS{26}Alors Samson dit au garçon qui le tenait par la main : Mets-moi en une telle place que je puisse toucher les piliers sur lesquels la maison est appuyée, afin que je m'y appuie.
\VS{27}Or la maison était pleine d'hommes et de femmes, et tous les Gouverneurs des Philistins y étaient. Il y avait même sur le toit près de trois mille personnes tant d'hommes, que de femmes, qui regardaient Samson se jouer.
\VS{28}Alors Samson invoqua l'Eternel, et dit : Seigneur Eternel, je te prie, souviens-toi de moi ; ô Dieu ! je te prie, fortifie-moi seulement cette fois, et que pour un coup je me venge des Philistins pour mes deux yeux.
\VS{29}Samson donc embrassa les deux piliers du milieu, sur lesquels la maison était appuyée, et se tint à eux, l'un desquels était à sa main droite, et l'autre à sa gauche.
\VS{30}Et il dit : Que je meure avec les Philistins. Il s'étendit donc de toute sa force ; et la maison tomba sur les Gouverneurs et sur tout le peuple qui y était. Et il fit mourir beaucoup plus de gens en sa mort, qu'il n'en avait fait mourir en sa vie.
\VS{31}Ensuite ses frères, et toute la maison de son père, descendirent, et l'emportèrent ; et étant remontés ils l'ensevelirent entre Tsorah et Estaol, dans le sépulcre de Manoah son père. Or il jugea Israël vingt ans.
\Chap{17}
\VerseOne{}Or il y avait un homme de la montagne d'Ephraïm, duquel le nom était Mica ;
\VS{2}Qui dit à sa mère : Les onze cents [pièces] d'argent qui te furent prises, pour lesquelles tu fis des imprécations, en ma présence, voici, j'ai cet argent-là par-devers moi ; je l'avais pris. Alors sa mère dit : Béni soit mon fils par l'Eternel.
\VS{3}Et quand il rendit à sa mère les onze cents [pièces] d'argent, sa mère dit : J'avais entièrement dédié de ma main cet argent à l'Eternel pour mon fils, afin d'en faire une image taillée, et une de fonte, ainsi je te le rendrai maintenant.
\VS{4}Après donc qu'il eut rendu cet argent à sa mère, elle en prit deux cents [pièces], et les donna au fondeur, qui en fit une image taillée, et une de fonte ; et elles furent dans la maison de Mica.
\VS{5}Ainsi cet homme, [savoir] Mica, eut une maison de dieux, et fit un Ephod et des Théraphims, et consacra l'un de ses fils, qui lui servit de Sacrificateur.
\VS{6}En ce temps-là il n'y avait point de Roi en Israël ; chacun faisait ce qui lui semblait être droit.
\VS{7}Or il y eut un jeune homme de Bethléhem de Juda, [ville] de la famille de Juda, qui était Lévite, et qui avait fait là son séjour ;
\VS{8}Lequel partit de cette ville-là, [savoir] de Bethléhem de Juda, pour aller demeurer où il trouverait [sa commodité], et continuant son chemin, il vint en la montagne d'Ephraïm jusqu'à la maison de Mica.
\VS{9}Et Mica lui dit : D'où viens-tu ? Le Lévite lui répondit : Je suis de Bethléhem de Juda, et je m'en vais pour demeurer où je trouverai [ma commodité].
\VS{10}Et Mica lui dit : Demeure avec moi, et sois-moi pour père et pour Sacrificateur, et je te donnerai dix [pièces] d'argent par an, et ce que tes habits coûteront, et ta nourriture. Et le Lévite y alla.
\VS{11}Ainsi le Lévite convint de demeurer avec cet homme-là, et ce jeune homme lui fut comme l'un de ses enfants.
\VS{12}Et Mica consacra le Lévite, et ce jeune homme lui servit de Sacrificateur, et demeura en sa maison.
\VS{13}Alors Mica dit : Maintenant je connais que l'Eternel me fera du bien, parce que j'ai un Lévite pour Sacrificateur.
\Chap{18}
\VerseOne{}En ce temps-là il n'y avait point de Roi en Israël, et en ce même temps la tribu de Dan cherchait un héritage pour soi afin d'y demeurer ; car jusqu'à ce temps-là il ne lui en était point échu entre les Tribus d'Israël pour le posséder.
\VS{2}C'est pourquoi les enfants de Dan envoyèrent de leur famille cinq hommes, d'une et d'autre qualité, gens vaillants, de Tsorah et d'Estaol, pour reconnaître le pays, et le reconnaître exactement ; et leur dirent : Allez [et] reconnaissez exactement le pays. Ils vinrent donc en la montagne d'Ephraïm jusqu'à la maison de Mica, et y passèrent la nuit.
\VS{3}Et quand ils furent auprès de la maison de Mica, ils reconnurent la voix du jeune homme Lévite ; et s'étant détournés vers cette maison-là, ils lui dirent : Qui t'a amené ici, qu'y fais-tu ? et qu'as-tu ici ?
\VS{4}Et il répondit : Mica a fait pour moi telle et telle chose ; il m'a donné des gages et je lui sers de Sacrificateur.
\VS{5}Ils dirent encore : Nous te prions de consulter Dieu, afin que nous sachions si le voyage que nous entreprenons prospérera.
\VS{6}Et le Sacrificateur leur dit : Allez en paix ; l'Eternel a devant ses yeux le voyage que vous entreprenez.
\VS{7}Ces cinq hommes donc s'en allèrent, et arrivèrent à Laïs, et ils virent que le peuple de cette ville habitait en assurance, et vivait en repos, et se croyait en sûreté, à la façon des Sidoniens ; et qu'il n'y avait personne au pays qui leur fît de la peine en aucune chose, parce qu'ils étaient libres de toute ancienneté ; et aussi ils étaient éloignés des Sidoniens, et n'avaient commerce avec personne.
\VS{8}Puis étant revenus à leurs frères à Tsorah [et] à Estaol, leurs frères leur dirent : Que [rapportez]-vous ?
\VS{9}Et ils répondirent : Allons, montons contr'eux ; car nous avons vu le pays, et nous l'avons trouvé très-bon ; et vous êtes sans rien faire ? ne soyez point paresseux à partir pour aller posséder le pays.
\VS{10}Quand vous y entrerez, vous viendrez vers un peuple qui se tient assuré, et en un pays de grande étendue, car Dieu l'a livré entre vos mains ; c'est un lieu où il ne manque rien de tout ce qui est sur la terre.
\VS{11}Il partit donc de là, [savoir] de Tsorah et d'Estaol, six cents hommes armés de la famille de Dan.
\VS{12}Et montant, ils campèrent à Kirjath-jéharim, qui est en Juda ; c'est pourquoi on a appelé ce lieu-là Mahané-dan, jusqu'à ce jour, et il est derrière Kirjath-jéharim.
\VS{13}Puis de là ils passèrent à la montagne d'Ephraïm, et arrivèrent à la maison de Mica.
\VS{14}Alors les cinq hommes qui étaient allés pour reconnaître le pays de Laïs, prenant la parole, dirent à leurs frères : Savez-vous bien qu'en ces maisons il y a un Ephod et des Théraphims, une image de taille et une de fonte ? Voyez donc maintenant ce que vous aurez à faire.
\VS{15}Alors ils se détournèrent vers ce lieu-là, et vinrent en la maison où était le jeune homme Lévite, savoir en la maison de Mica, et le saluèrent.
\VS{16}Or les six cents hommes des enfants de Dan, qui étaient sous les armes, s'arrêtèrent à l'entrée de la porte ;
\VS{17}Mais les cinq hommes qui étaient allés pour reconnaître le pays, montèrent et entrèrent dans la maison, et prirent l'imagé taillée, l'Ephod, les Théraphims, et l'image de fonte, pendant que le Sacrificateur était à l'entrée de la porte, avec les six cents hommes armés.
\VS{18}Etant donc entrés dans la maison de Mica, ils prirent l'image taillée, l'Ephod, les Théraphims, et l'image de fonte. Et le Sacrificateur leur dit : Que faites-vous ?
\VS{19}Et ils lui dirent : Tais-toi, et mets ta main sur ta bouche, et viens avec nous, et sois-nous pour père et pour Sacrificateur. Lequel te vaut-il mieux, d'être Sacrificateur de la maison d'un homme seul, ou d'être Sacrificateur d'une Tribu et d'une famille en Israël ?
\VS{20}Et le Sacrificateur en eut de la joie en son cœur, et ayant pris l'Ephod, les Théraphims, et l'image taillée, il se mit au milieu du peuple.
\VS{21}Après quoi ils retournèrent et reprirent leur chemin, et mirent devant eux les petits enfants, le bétail, et le bagage.
\VS{22}Et quand ils furent loin de la maison de Mica, ceux qui [demeuraient] dans les maisons voisines de celle de Mica furent assemblés à grand cri ; et ils atteignirent les enfants de Dan.
\VS{23}Et ils crièrent après eux ; mais eux tournant visage dirent à Mica : Qu'as-tu, que tu te sois ainsi écrié pour amasser des gens ?
\VS{24}Il répondit : Vous avez enlevé mes dieux que j'avais faits, vous [avez pris] le Sacrificateur, et vous en êtes allés. Et que me reste-t-il ? Comment donc me dites-vous : Qu'as-tu ?
\VS{25}Et les enfants de Dan lui dirent : Ne fais point entendre ta voix après nous, de peur que ces gens en colère ne se jettent sur vous, et que vous n'y laissiez la vie, toi, et tous ceux de ta famille.
\VS{26}Les enfants donc de Dan continuèrent leur chemin, mais Mica ayant vu qu'ils étaient plus forts que lui, tourna visage, et s'en revint en sa maison.
\VS{27}Ainsi ayant pris les choses que Mica avait faites, et le Sacrificateur qu'il avait, ils arrivèrent à Laïs, vers un peuple qui était en repos, et qui se tenait assuré, et ils les firent passer au fil de l'épée, et ayant mis le feu dans la ville, ils la brûlèrent.
\VS{28}Et il n'y eut personne qui la délivrât ; car elle était loin de Sidon, et n'avait commerce avec personne, et elle était située en la vallée qui appartenait au [pays de] Beth-réhob : puis ils bâtirent [là] une ville, et y habitèrent.
\VS{29}Et ils nommèrent cette ville-là, Dan ; selon le nom de Dan leur père qui était né à Israël, au lieu que la ville avait nom auparavant Laïs.
\VS{30}Et les enfants de Dan se dressèrent cette image taillée, et Jonathan fils de Guerson, fils de Manassé, lui et ses enfants furent Sacrificateurs pour la Tribu de Dan jusqu'au jour qu'elle partit du pays.
\VS{31}Ils y dressèrent donc l'image taillée que Mica avait faite, tout le temps que la maison de Dieu fut à Silo.
\Chap{19}
\VerseOne{}Il arriva aussi en ce temps-là, n'y ayant point de Roi en Israël, qu'il y eut un Lévite, demeurant aux côtés de la montagne d'Ephraïm, qui prit une femme concubine de Bethléhem de Juda.
\VS{2}Mais sa concubine paillarda chez lui, et s'en alla d'avec lui en la maison de son père en Bethléhem de Juda, et elle y fut quelques jours, savoir l'espace de quatre mois.
\VS{3}Puis son mari se leva, et s'en alla après elle, pour lui parler selon son cœur, [et] pour la ramener. Il avait aussi avec soi son serviteur, et deux ânes ; et elle le fit entrer dans la maison de son père ; et le père de la jeune femme le voyant se réjouit de son arrivée.
\VS{4}Son beau-père donc, père de la jeune femme, le retint [à grande instance] ; de sorte qu'il demeura avec lui trois jours ; et ils mangèrent et burent, et logèrent là.
\VS{5}Et au quatrième jour s'étant levé de bon matin, il se mit en chemin pour s'en aller ; mais le père de la jeune femme dit à son beau-fils : Fortifie ton cœur avec une bouchée de pain, et puis vous vous en irez.
\VS{6}Ils s'assirent donc, et mangèrent et burent eux deux ensemble ; et le père de la jeune femme dit au mari : Je te prie qu'il te plaise de passer encore ici cette nuit, et que ton cœur se réjouisse.
\VS{7}Et comme le mari se fut mis en chemin pour s'en aller, son beau-père le pressa tellement, qu'il s'en retourna ; et il y passa encore la nuit.
\VS{8}Et au cinquième jour il se leva de bon matin pour s'en aller, et le père de la jeune femme dit : Je te prie, fortifie ton cœur ; et ils tardèrent tant, que le jour baissa pendant qu'ils mangeaient eux deux [ensemble].
\VS{9}Puis le mari se mit en chemin pour s'en aller, lui et sa concubine, avec son serviteur. Et son beau-père le père de la jeune femme, lui dit : Voici maintenant le jour baisse, il se fait tard, je vous prie passez ici la nuit, voici le jour finit, passe ici la nuit, et que ton cœur se réjouisse ; et demain au matin vous vous lèverez pour aller votre chemin, et tu t'en iras en ta maison.
\VS{10}Mais le mari ne voulut point y passer la nuit ; mais il se leva, et s'en alla et vint jusque vis-à-vis de Jébus, qui est Jérusalem, ayant avec soi ses deux ânes embâtés, et sa concubine.
\VS{11}Or comme ils étaient près de Jébus, et que le jour était fort avancé, le serviteur dit à son maître : Marchez, je vous prie, et détournons-nous vers cette ville des Jébusiens, afin que nous y passions la nuit.
\VS{12}Et son maître lui répondit : Nous ne nous détournerons point vers aucune ville des étrangers, où il n'y a point d'enfants d'Israël ; mais nous passerons jusqu'à Guibha.
\VS{13}Il dit aussi à son serviteur : Marche, et nous gagnerons l'un de ces lieux-là, et nous passerons la nuit à Guibha, ou à Rama.
\VS{14}Ils passèrent donc plus avant et marchèrent, et le soleil se coucha comme ils furent près de Guibha, qui appartient à Benjamin.
\VS{15}Alors ils se détournèrent vers Guibha pour y entrer et y passer la nuit ; et y étant entrés, ils demeurèrent en la place de la ville, car il n'y avait personne qui les retirât chez soi afin qu'ils y passassent la nuit.
\VS{16}Et voici sur le soir un vieux homme venait des champs de son travail, et cet homme était de la montagne d'Ephraïm, mais il demeurait à Guibha, dont les habitants étaient enfants de Jémini.
\VS{17}Et levant ses yeux il vit dans la place de la ville ce passant ; et cet homme vieux lui dit : Où vas-tu, et d'où viens-tu ?
\VS{18}Et il lui répondit : Nous passons de Bethléhem de Juda vers les côtés de la montagne d'Ephraïm, d'où je suis, parce que j'étais allé jusqu'à Bethléhem de Juda, mais maintenant je m'en vais à la maison de l'Eternel, et il n'y a ici personne qui me retire chez lui.
\VS{19}Nous avons pourtant de la paille et du fourrage pour nos ânes, et du pain et du vin pour moi et pour ta servante, et pour le garçon qui est avec tes serviteurs ; nous n'avons besoin d'aucune chose.
\VS{20}Et le vieillard lui dit : Paix te soit, quoi qu'il en soit je me charge de tout ce dont tu as besoin ; [je te prie] seulement de ne passer point la nuit dans la place.
\VS{21}Alors il le fit entrer en sa maison, et il donna du fourrage aux ânes ; ils lavèrent leurs pieds, mangèrent et burent.
\VS{22}Comme ils faisaient bonne chère, voici les gens de la ville, hommes fort corrompus, environnèrent la maison, heurtant à la porte, et ils parlèrent au vieux homme, maître de la maison, en disant : Fais sortir cet homme qui est entré en ta maison, afin que nous le connaissions.
\VS{23}Mais cet homme, [savoir] le maître de la maison, sortit vers eux, et leur dit : Non, mes frères, ne lui faites point de mal, je vous prie : puisque cet homme est entré en ma maison, ne faites point une telle infamie.
\VS{24}Voici, j'ai une fille vierge, et [cet homme] a sa concubine, je vous les amènerai dehors maintenant, et vous les violerez, et vous ferez d'elles comme il vous semblera bon ; mais ne commettez point cette action infâme à l'égard de cet homme.
\VS{25}Mais ces gens-là ne voulurent point l'écouter ; c'est pourquoi cet homme prit sa concubine, et la leur amena dehors ; et ils la connurent, et abusèrent d'elle toute la nuit jusques au matin, puis ils la renvoyèrent comme l'aube du jour se levait.
\VS{26}Cette femme donc, comme le jour approchait, s'en revint, et étant tombée à la porte de la maison de l'homme où était son Seigneur, elle y demeura jusqu'au jour.
\VS{27}Et son Seigneur se leva de bon matin, et ayant ouvert la porte il sortait pour continuer son chemin ; mais voici, sa femme concubine était tombée à la porte de la maison, et avait les mains sur le seuil.
\VS{28}Et il lui dit : Lève-toi, et allons-nous-en ; mais elle ne répondait point. Alors il la chargea sur un âne, et se mit en chemin, et s'en alla en son lieu.
\VS{29}Et étant venu en sa maison, il prit un couteau, et empoignant sa concubine il la partagea avec ses os en douze parts, et en envoya dans tous les Cantons d'Israël.
\VS{30}Et il arriva que tous ceux qui virent cela dirent : Une telle chose n'a été faite ni vue depuis le jour que les enfants d'Israël sont montés hors du pays d'Egypte, jusqu'à ce jour. Pensez à cela, consultez, et prononcez.
\Chap{20}
\VerseOne{}Alors tous les enfants d'Israël sortirent, et tout le peuple fut assemblé comme si ce n'eût été qu'un seul homme, depuis Dan jusqu'à Beersebah, et jusqu'au pays de Galaad, vers l'Eternel, en Mitspa.
\VS{2}Et les Cantons de tout le peuple, toutes les Tribus d'Israël, se trouvèrent à l'assemblée du peuple de Dieu, [au nombre de] quatre cent mille hommes de pied dégainant l'épée.
\VS{3}(Or les enfants de Benjamin apprirent que les enfants d'Israël étaient montés en Mitspa.) Les enfants donc d'Israël dirent : Qu'on nous récite comment ce mal est arrivé.
\VS{4}Et le Lévite, mari de la femme tuée, répondit et dit : Etant arrivés à Guibha, qui est de Benjamin, moi et ma concubine, pour y passer la nuit ;
\VS{5}Les Seigneurs de Guibha se sont élevés contre moi, et ont environné de nuit la maison contre moi, prétendant me tuer ; et ils ont tellement violé ma concubine qu'elle en est morte.
\VS{6}C'est pourquoi ayant pris ma concubine, je l'ai mise en pièces, et je les ai envoyées par tous les quartiers de l'héritage d'Israël ; car ils ont fait un crime énorme, et une action infâme en Israël.
\VS{7}Vous voici tous, enfants d'Israël, délibérez-en ici entre vous, et donnez-en votre avis.
\VS{8}Et tout le peuple se leva, comme si ce n'eût été qu'un seul homme, et ils dirent : Aucun de nous n'ira en sa tente, ni aucun de nous ne se retirera dans sa maison.
\VS{9}Mais maintenant voici ce que nous ferons à Guibha, en procédant contr'elle par sort.
\VS{10}Nous prendrons dix hommes de cent dans toutes les Tribus d'Israël, et cent de mille, et mille de dix mille, qui prendront de la provision pour le peuple, afin qu'étant entrés à Guibha de Benjamin, ils la traitent selon toute la turpitude qu'elle a commise en Israël.
\VS{11}Ainsi tous ceux d'Israël furent assemblés contre cette ville-là, étant unis comme s'ils n'eussent été qu'un seul homme.
\VS{12}Alors les Tribus d'Israël envoyèrent des hommes par toute la tribu de Benjamin pour lui dire : Quelle méchante action a-t-on [commise] parmi vous ?
\VS{13}Maintenant donc livrez-nous ces méchants hommes qui sont à Guibha, afin que nous les fassions mourir, et que nous ôtions le mal du milieu d'Israël. Mais les Benjamites ne voulurent point écouter la voix de leurs frères les enfants d'Israël.
\VS{14}Mais les Benjamites sortant de leurs villes s'assemblèrent à Guibha, pour sortir en bataille contre les enfants d'Israël.
\VS{15}Et en ce jour-là on fit le dénombrement des enfants de Benjamin qui étaient dans ces villes, [et il se trouva] vingt-six mille hommes, tirant l'épée, sans les habitants de Guibha, dont on fit aussi le dénombrement, qui furent sept cents hommes d'élite.
\VS{16}De tout ce peuple-là il y avait sept cents hommes d'élite, desquels la main droite était serrée, tous tirant la pierre avec la fronde à un cheveu près, et ils n'y manquaient point.
\VS{17}Et les hommes d'Israël furent [tous] dénombrés, excepté ceux de Benjamin, et il s'en trouva quatre cent mille hommes tirant l'épée, tous gens de guerre.
\VS{18}Or ils partirent, et montèrent à la maison du [Dieu] Fort, et consultèrent Dieu. Les enfants donc d'Israël dirent : Qui est-ce d'entre nous qui montera le premier pour faire la guerre aux enfants de Benjamin ? Et l'Eternel répondit : Juda [montera] le premier.
\VS{19}Puis les enfants d'Israël se levèrent de bon matin, et campèrent près de Guibha.
\VS{20}Et ceux d'Israël sortirent en bataille contre Benjamin, et rangèrent contr'eux [leur] armée près de Guibha.
\VS{21}Et les enfants de Benjamin sortirent de Guibha, et en ce jour-là ils mirent par terre de ceux d'Israël vingt-deux mille hommes.
\VS{22}Toutefois le peuple de ceux d'Israël reprit courage, et se rangea de nouveau en bataille au lieu où il s'était rangé le premier jour.
\VS{23}Parce que les enfants d'Israël étaient montés, et avaient pleuré devant l'Eternel jusqu'au soir, et avaient consulté l'Eternel en disant : M'approcherai-je encore pour combattre contre les enfants de Benjamin, mon frère ? Et l'Eternel avait répondu : Montez contre lui.
\VS{24}Les enfants d'Israël s'approchèrent des enfants de Benjamin pour la seconde journée.
\VS{25}Benjamin aussi sortit de Guibha contr'eux en cette seconde journée, et ils mirent encore par terre dix-huit mille hommes des enfants d'Israël, tous tirant l'épée.
\VS{26}Alors tous les enfants d'Israël, et tout le peuple montèrent, et vinrent à la maison du [Dieu] Fort, et y pleurèrent, et se tinrent là devant l'Eternel, et jeûnèrent ce jour-là jusqu'au soir, et offrirent des holocaustes, et des sacrifices de prospérités devant l'Eternel.
\VS{27}Ensuite les enfants d'Israël consultèrent l'Eternel, (or là était l'Arche de l'alliance de Dieu en ces jours-là.
\VS{28}Et Phinées fils d'Eléazar, fils d'Aaron, se tenait devant l'Eternel en ces jours-là :) [ils consultèrent donc l'Eternel] en disant : Sortirai-je encore une autre fois en bataille contre les enfants de Benjamin, mon frère, ou m'en désisterai-je ? Et l'Eternel répondit : Montez, car demain je les livrerai entre vos mains.
\VS{29}Et Israël mit une embuscade à l'entour de Guibha.
\VS{30}Et les enfants d'Israël montèrent pour la troisième journée contre les enfants de Benjamin, et ils se rangèrent contre Guibha, comme les autres fois.
\VS{31}Alors les enfants de Benjamin étant sortis à la rencontre du peuple, furent attirés hors de la ville, et commencèrent à frapper quelques-uns du peuple, environ trente hommes d'Israël qui furent blessés à mort comme les autres fois, dans les chemins, dont l'un monte à la maison du [Dieu] Fort ; et l'autre à Guibha, parmi les champs.
\VS{32}Et les enfants de Benjamin dirent : Ils tombent battus devant nous, comme la première fois. Mais les enfants d'Israël disaient : Fuyons, attirons-les hors de la ville, dans les chemins.
\VS{33}Tous ceux d'Israël donc se levant de leur lieu, se rangèrent à Bahal-tamar ; et les gens de l'embuscade aussi sortirent de leur lieu, [savoir de] la prairie de Guibha.
\VS{34}Et dix mille hommes d'élite de tout Israël vinrent contre Guibha, et la mêlée fut rude ; et ceux de [Benjamin] ne s'aperçurent point que le mal les atteignait.
\VS{35}L'Eternel donc battit Benjamin devant les Israélites ; et les enfants d'Israël mirent ce jour-là par terre vingt-cinq mille et cent hommes de Benjamin, tous tirant l'épée.
\VS{36}Les enfants de Benjamin donc virent qu'ils étaient battus. Or ceux d'Israël avaient fait place à ceux de Benjamin ; car ils s'assuraient sur l'embuscade qu'ils avaient mise près de Guibha.
\VS{37}Et ceux qui étaient en embuscade se jetèrent incontinent sur Guibha ; ainsi ceux qui étaient en embuscade marchèrent à la file, et frappèrent toute la ville au tranchant de l'épée.
\VS{38}Or ceux d'Israël avaient donné pour signal à ceux qui étaient en embuscade, qu'ils fissent monter beaucoup de fumée de la ville.
\VS{39}Ceux d'Israël donc avaient tourné le dos en la bataille, et Benjamin avait commencé de frapper et de blesser à mort environ trente hommes de ceux d'Israël ; car ils disaient : Quoi qu'il en soit, certainement ils tombent battus devant nous, comme à la première bataille.
\VS{40}Mais quand la fumée qui avait été élevée, commença à monter de la ville comme une colonne de fumée, Benjamin regarda derrière soi, et voici toute la ville montait en feu vers le ciel.
\VS{41}Alors ceux d'Israël tournèrent visage et ceux de Benjamin furent épouvantés ; car ils virent que le mal les avait atteints.
\VS{42}Et ils tournèrent le dos devant ceux d'Israël vers le chemin du désert ; mais l'armée [d'Israël] les serra de près. Et quant à ceux des villes, ils les mirent par terre, les ayant enfermés au milieu d'eux.
\VS{43}Ils environnèrent [donc] ceux de Benjamin, [et] les poursuivirent, et les foulèrent aux pieds depuis Menuha jusqu'à l'opposite de Guibha, vers le soleil levant.
\VS{44}Et il y eut de Benjamin dix-huit mille hommes tués, tous vaillants hommes.
\VS{45}Alors [ceux de Benjamin] tournant le dos fuirent vers le désert au rocher de Rimmon, et [ceux d'Israël] en grappillèrent par les chemins cinq mille hommes, et les poursuivant de près jusqu'à Guidhom, ils en frappèrent deux mille hommes.
\VS{46}Tous ceux donc qui tombèrent [morts] en ce jour-là de Benjamin, furent vingt-cinq mille hommes, tirant l'épée, [et] tous vaillants hommes.
\VS{47}Et il y eut six cents hommes de ceux qui avaient tourné le dos, qui échappèrent vers le désert au rocher de Rimmon, et qui demeurèrent au rocher de Rimmon quatre mois.
\VS{48}Et ceux d'Israël retournèrent vers les enfants de Benjamin, et les frappèrent au tranchant de l'épée, tant les hommes de chaque ville que les bêtes, et tout ce qui s'y trouva. Ils brûlèrent aussi toutes les villes qui s'y trouvèrent.
\Chap{21}
\VerseOne{}Or ceux d'Israël avaient juré en Mitspa, en disant : Nul de nous ne donnera sa fille pour femme aux Benjamites.
\VS{2}Puis le peuple vint à la maison du [Dieu] Fort, et ils demeurèrent là jusqu'au soir en la présence de Dieu ; et ils élevèrent leurs voix, et pleurèrent amèrement,
\VS{3}Et dirent : Ô Eternel, Dieu d'Israël ! pourquoi ceci est-il arrivé en Israël, qu'une Tribu d'Israël ait été aujourd'hui retranchée ?
\VS{4}Et le lendemain le peuple se leva de bon matin, et bâtit là un autel, et ils offrirent des holocaustes, et des sacrifices de prospérités.
\VS{5}Alors les enfants d'Israël dirent : Qui est celui d'entre toutes les Tribus d'Israël qui n'est point monté à l'assemblée vers l'Eternel ? Car on avait fait un grand serment contre tout homme qui ne monterait point vers l'Eternel à Mitspa, en disant : Un tel sera puni de mort.
\VS{6}Car les enfants d'Israël se repentaient de ce qui était arrivé à Benjamin leur frère, et disaient aujourd'hui une Tribu a été retranchée d'Israël.
\VS{7}Comment ferons-nous pour donner des femmes à ceux qui sont demeurés de reste, puisque nous avons juré par l'Eternel que nous ne leur donnerions point de nos filles pour femmes ?
\VS{8}Ils dirent donc : Y a-t-il quelqu'un d'entre les Tribus d'Israël qui ne soit point monté vers l'Eternel à Mitspa ? Or voici, aucun homme de Jabès de Galaad n'était venu au camp, à l'assemblée.
\VS{9}Car quand on fit le dénombrement du peuple, voici, il ne s'était trouvé aucun des habitants de Jabès de Galaad.
\VS{10}C'est pourquoi l'assemblée y envoya douze mille hommes des plus vaillants, et leur commanda en disant : Allez, et frappez les habitants de Jabès de Galaad au tranchant de l'épée, tant les femmes que les petits enfants.
\VS{11}Voici donc ce que vous ferez : Vous exterminerez à la façon de l'interdit tout mâle, et toute femme qui aura eu compagnie d'homme.
\VS{12}Et ils trouvèrent entre les habitants de Jabès de Galaad quatre cents filles vierges, qui n'avaient point eu compagnie d'homme ; et ils les amenèrent au camp à Silo, qui est au pays de Canaan.
\VS{13}Alors toute l'assemblée envoya pour parler aux enfants de Benjamin qui étaient au rocher de Rimmon, et pour leur offrir la paix.
\VS{14}En ce temps-là donc les Benjamites retournèrent, et on leur donna pour femmes celles qui avaient été conservées en vie d'entre les femmes de Jabès de Galaad ; mais il ne s'en trouva pas [assez pour eux].
\VS{15}Et le peuple se repentit de ce qui avait été fait à Benjamin ; car l'Eternel avait fait une brèche aux Tribus d'Israël.
\VS{16}Et les Anciens de l'assemblée dirent : Comment ferons-nous pour donner des femmes à ceux qui sont demeurés de reste ; car les femmes ont été exterminées d'entre les Benjamites.
\VS{17}Puis ils dirent : Ceux qui sont réchappés posséderont ce qui appartenait à Benjamin, afin qu'une Tribu d'Israël ne soit point effacée.
\VS{18}Cependant nous ne leur pourrons point donner des femmes d'entre nos filles ; car les enfants d'Israël ont juré, en disant : Maudit soit celui qui donnera une femme à ceux de Benjamin.
\VS{19}Et ils dirent : Voici la solennité ordinaire de l'Eternel est à Silo, qui est vers l'Aquilon de Bethel, et au soleil levant du chemin qui monte de Bethel à Sichem, et au midi de Lebona.
\VS{20}Et ils commandèrent aux enfants de Benjamin, en disant : Allez, et mettez des gens en embuscade aux vignes.
\VS{21}Et quand vous verrez que les filles de Silo sortiront pour danser avec des flûtes, alors vous sortirez des vignes, et vous ravirez pour vous chacun sa femme d'entre les filles de Silo, et vous en irez au pays de Benjamin.
\VS{22}Et quand leurs pères, ou leurs frères viendront vers nous pour plaider, nous leur dirons : Ayez pitié d'eux pour l'amour de nous, puisque nous n'avons point pris femme pour [chacun d'eux] en cette guerre, et vous serez coupables si vous ne leur en donnez point en un temps comme celui-ci.
\VS{23}Les enfants de Benjamin firent ainsi, et enlevèrent des femmes selon leur nombre, d'entre celles qui dansaient, lesquelles ils ravirent ; puis s'en allant ils retournèrent à leur héritage, et rebâtirent des villes, et y habitèrent.
\VS{24}Ainsi en ce temps-là chacun des enfants d'Israël s'en alla de là en sa Tribu, et dans sa famille, et ils se retirèrent de là chacun dans son héritage.
\VS{25}En ces jours-là il n'y avait point de Roi en Israël ; mais chacun faisait ce qui lui semblait être droit.
\PPE{}
\end{multicols}
