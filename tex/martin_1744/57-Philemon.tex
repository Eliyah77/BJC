\ShortTitle{Philemon}\BookTitle{Philemon}\BFont
\begin{multicols}{2}
\Chap{1}
\VerseOne{}Paul, prisonnier de Jésus-Christ, et le frère Timothée, à Philémon notre bien-aimé, et Compagnon d'œuvre ;
\VS{2}Et à Apphie notre bien-aimée, et à Archippe notre Compagnon d'armes, et à l'Eglise qui est en ta maison.
\VS{3}Que la grâce et la paix vous soient données de la part de Dieu notre Père, et de la part du Seigneur Jésus-Christ.
\VS{4}Je rends grâces à mon Dieu, faisant toujours mention de toi dans mes prières ;
\VS{5}Apprenant la foi que tu as au Seigneur Jésus, et ta charité envers tous les Saints.
\VS{6}Afin que la communication de ta foi montre son efficace, en se faisant connaître par tout le bien qui est en vous par Jésus-Christ.
\VS{7}Car, mon frère, nous avons une grande joie et une grande consolation de ta charité, en ce que tu as réjoui les entrailles des Saints.
\VS{8}C'est pourquoi bien que j'aie une grande liberté en Christ de te commander ce qui est de ton devoir,
\VS{9}Cependant je te prie plutôt par la charité, bien que je suis ce que je suis, savoir Paul, Ancien, et même maintenant prisonnier de Jésus-Christ ;
\VS{10}Je te prie [donc] pour mon fils Onésime, que j'ai engendré dans mes liens ;
\VS{11}Qui t'a été autrefois inutile, mais qui maintenant est bien utile et à toi et à moi, et lequel je te renvoie.
\VS{12}Reçois-le donc, comme mes propres entrailles.
\VS{13}Je voulais le retenir auprès de moi, afin qu'il me servît à ta place, dans les liens de l'Evangile.
\VS{14}Mais je n'ai rien voulu faire sans ton avis, afin que ce ne fût point comme par contrainte, mais volontairement, que tu me laissasses un bien qui est à toi.
\VS{15}Car peut-être n'a-t-il été séparé de toi pour un temps, qu'afin que tu le recouvrasses pour toujours.
\VS{16}Non plus comme un esclave, mais comme étant au-dessus d'un esclave, [savoir], comme un frère bien-aimé, principalement de moi ; et combien plus de toi, soit selon la chair, soit selon le Seigneur ?
\VS{17}Si donc tu me tiens pour ton compagnon, reçois-le comme moi-même.
\VS{18}Que s'il t'a fait quelque tort, ou s'il te doit quelque chose, mets-le-moi en compte.
\VS{19}Moi Paul j'ai écrit ceci de ma propre main, je te le payerai ; pour ne pas te dire que tu te dois toi-même à moi.
\VS{20}Oui, mon frère, que je reçoive ce plaisir de toi en [notre] Seigneur ; réjouis mes entrailles en [notre] Seigneur.
\VS{21}Je t'ai écrit m'assurant de ton obéissance, et sachant que tu feras même plus que je ne te dis.
\VS{22}Mais aussi en même temps prépare-moi un logement ; car j'espère que je vous serai donné par vos prières.
\VS{23}Epaphras, qui est prisonnier avec moi en Jésus-Christ, te salue ;
\VS{24}Marc [aussi], et Aristarque, et Démas, et Luc, mes compagnons d'œuvre.
\VS{25}Que la grâce de notre Seigneur Jésus-Christ soit avec votre esprit, Amen !
\PPE{}
\end{multicols}
