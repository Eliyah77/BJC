\ShortTitle{1Corinthiens}\BookTitle{1Corinthiens}\BFont
\begin{multicols}{2}
\Chap{1}
\VerseOne{}Paul appelé à être Apôtre de Jésus-Christ, par la volonté de Dieu, et le frère Sosthènes,
\VS{2}A l'Eglise de Dieu qui est à Corinthe, aux sanctifiés en Jésus-Christ, qui êtes appelés à être saints, avec tous ceux qui en quelque lieu que ce soit invoquent le Nom de notre Seigneur Jésus-Christ, leur [Seigneur] et le nôtre ;
\VS{3}Que la grâce et la paix vous soient données par Dieu notre Père, et par le Seigneur Jésus-Christ.
\VS{4}Je rends toujours grâces à mon Dieu à cause de vous, pour la grâce de Dieu qui vous a été donnée en Jésus-Christ ;
\VS{5}De ce qu'en toutes choses vous êtes enrichis en lui de tout don de parole, et de toute connaissance ;
\VS{6}Selon que le témoignage de Jésus-Christ a été confirmé en vous ;
\VS{7}Tellement qu'il ne vous manque aucun don, pendant que vous attendez la manifestation de notre Seigneur Jésus-Christ,
\VS{8}Qui aussi vous affermira jusqu’à la fin, pour être irrépréhensibles en la journée de notre Seigneur Jésus-Christ.
\VS{9}[Et] Dieu, par qui vous avez été appelés à la communion de son Fils Jésus-Christ notre Seigneur, est fidèle.
\VS{10}Or je vous prie, mes frères, par le Nom de notre Seigneur Jésus-Christ, que vous parliez tous un même langage, et qu'il n'y ait point de divisions entre vous, mais que vous soyez bien unis dans un même sentiment, et dans un même avis.
\VS{11}Car, mes frères, il m'a été dit de vous par ceux [qui sont de chez] Chloé, qu'il y a des dissensions parmi vous.
\VS{12}Voici donc ce que je dis, c'est que chacun de vous dit : pour moi, je suis de Paul ; et moi je suis d'Apollos ; et moi, de Céphas ; et moi, de Christ.
\VS{13}Christ est-il divisé ? Paul a-t-il été crucifié pour vous ? ou avez-vous été baptisés au nom de Paul ?
\VS{14}Je rends grâces à Dieu que je n'ai baptisé aucun de vous, sinon Crispus et Gaïus ;
\VS{15}Afin que personne ne dise que j'ai baptisé en mon nom.
\VS{16}J'ai bien aussi baptisé la famille de Stéphanas ; du reste, je ne sais pas si j'ai baptisé quelque autre.
\VS{17}Car Christ ne m'a pas envoyé pour baptiser, mais pour évangéliser, non point avec les discours de la sagesse [humaine], afin que la croix de Christ ne soit point anéantie.
\VS{18}Car la parole de la croix est une folie à ceux qui périssent ; mais à nous qui obtenons le salut, elle est la vertu de Dieu.
\VS{19}Vu qu'il est écrit : j'abolirai la sagesse des sages, et j'anéantirai l'intelligence des hommes intelligents.
\VS{20}Où est le sage ? où est le Scribe ? où est le Disputeur de ce Siècle ? Dieu n'a-t-il pas manifesté la folie de la sagesse de ce monde ?
\VS{21}Car puisqu'en la sapience de Dieu, le monde n'a point connu Dieu par la sagesse, le bon plaisir de Dieu a été de sauver les croyants par la folie de la prédication.
\VS{22}Car les Juifs demandent des miracles, et les Grecs cherchent la sagesse.
\VS{23}Mais pour nous, nous prêchons Christ crucifié, qui est un scandale pour les Juifs, et une folie pour les Grecs.
\VS{24}A ceux, dis-je, qui sont appelés tant Juifs que Grecs, [nous leur prêchons] Christ, la puissance de Dieu, et la sagesse de Dieu.
\VS{25}Parce que la folie de Dieu est plus sage que les hommes, et la faiblesse de Dieu est plus forte que les hommes.
\VS{26}Car, mes frères, vous voyez votre vocation, que vous n'êtes pas beaucoup de sages selon la chair, ni beaucoup de puissants, ni beaucoup de nobles.
\VS{27}Mais Dieu a choisi les choses folles de ce monde, pour rendre confuses les sages ; et Dieu a choisi les choses faibles de ce monde, pour rendre confuses les fortes ;
\VS{28}Et Dieu a choisi les choses viles de ce monde, et les méprisées, même celles qui ne sont point, pour abolir celles qui sont.
\VS{29}Afin que nulle chair ne se glorifie devant lui.
\VS{30}Or c'est par lui que vous êtes en Jésus-Christ, qui vous a été fait de la part de Dieu sagesse, justice, sanctification, et rédemption ;
\VS{31}Afin que comme il est écrit, celui qui se glorifie, se glorifie au Seigneur.
\Chap{2}
\VerseOne{}Pour moi donc, mes frères, quand je suis venu vers vous, je n'y suis point venu avec des discours pompeux, remplis de la sagesse [humaine], en vous annonçant le témoignage de Dieu.
\VS{2}Parce que je ne me suis proposé de savoir autre chose parmi vous, que Jésus-Christ, et Jésus–Christ crucifié.
\VS{3}Et j'ai même été parmi vous dans la faiblesse, dans la crainte, et dans un grand tremblement.
\VS{4}Et ma parole et ma prédication n'a point été en paroles persuasives de la sagesse humaine : mais en évidence d'Esprit et de puissance ;
\VS{5}Afin que votre foi ne soit point l'effet de la sagesse des hommes, mais de la puissance de Dieu.
\VS{6}Or nous proposons une sagesse entre les parfaits, une sagesse, dis-je, qui n'est point de ce monde, ni des princes de ce siècle, qui vont être anéantis.
\VS{7}Mais nous proposons la sagesse de Dieu, [qui est] en mystère, [c'est-à-dire], cachée ; laquelle Dieu avait, dès avant les siècles, déterminée à notre gloire.
\VS{8}[Et] laquelle aucun des princes de ce siècle n'a connue ; car s'ils l'eussent connue, jamais ils n'eussent crucifié le Seigneur de gloire.
\VS{9}Mais ainsi qu'il est écrit : ce sont des choses que l'œil n'a point vues ; que l'oreille n'a point ouïes, et qui ne sont point montées au cœur de l'homme, lesquelles Dieu a préparées à ceux qui l’aiment.
\VS{10}Mais Dieu nous les a révélées par son Esprit. Car l'Esprit sonde toutes choses, même les choses profondes de Dieu.
\VS{11}Car qui est-ce des hommes qui sache les choses de l'homme, sinon l’esprit de l'homme qui est en lui ? De même aussi nul n'a connu les choses de Dieu, sinon l'Esprit de Dieu.
\VS{12}Or nous avons reçu non point l'esprit de ce monde, mais l'Esprit qui est de Dieu ; afin que nous connaissions les choses qui nous ont été données de Dieu ;
\VS{13}Lesquelles aussi nous proposons, non point avec les paroles que la sagesse humaine enseigne, mais avec celles qu'enseigne le Saint-Esprit, appropriant les choses spirituelles à ceux qui sont spirituels.
\VS{14}Or l'homme animal ne comprend point les choses qui sont de l'Esprit de Dieu, car elles lui sont une folie ; et il ne peut même les entendre, parce qu'elles se discernent spirituellement.
\VS{15}Mais [l'homme] spirituel discerne toutes choses, et il n'est jugé de personne.
\VS{16}Car qui a connu la pensée du Seigneur pour le pouvoir instruire ? mais nous, nous avons l'intention de Christ.
\Chap{3}
\VerseOne{}Et pour moi, mes frères, je n'ai pu vous parler comme à des [hommes] spirituels, mais comme à des [hommes] charnels, [c'est-à-dire], comme à des enfants en Christ.
\VS{2}Je vous ai donné du lait à boire, et non pas de la viande, parce que vous ne la pouviez pas encore [porter] ; même maintenant vous ne le pouvez pas encore ; parce que vous êtes encore charnels.
\VS{3}Car puisqu'il y a parmi vous de l'envie, et des dissensions, et des divisions, n'êtes-vous pas charnels, et ne vous conduisez-vous pas à la manière des hommes ?
\VS{4}Car quand l'un dit : pour moi, je suis de Paul ; et l'autre : pour moi, je suis d'Apollos ; n'êtes-vous pas charnels ?
\VS{5}Qui est donc Paul, et qui est Apollos, sinon des Ministres, par lesquels vous avez cru, selon que le Seigneur a donné à chacun ?
\VS{6}J'ai planté ; Apollos a arrosé ; mais c'est Dieu qui a donné l'accroissement.
\VS{7}Or ni celui qui plante, ni celui qui arrose, ne sont rien ; mais Dieu, qui donne l'accroissement.
\VS{8}Et tant celui qui plante, que celui qui arrose, ne sont qu'une même chose ; mais chacun recevra sa récompense selon son travail.
\VS{9}Car nous sommes ouvriers avec Dieu ; [et] vous êtes le labourage de Dieu, [et] l'édifice de Dieu.
\VS{10}Selon la grâce de Dieu qui m'a été donnée, j'ai posé le fondement comme un sage architecte, et un autre édifie dessus ; mais que chacun examine comment il édifie dessus.
\VS{11}Car personne ne peut poser d'autre fondement que celui qui est posé, lequel est Jésus-Christ.
\VS{12}Que si quelqu'un édifie sur ce fondement, de l'or, de l'argent, des pierres précieuses, du bois, du foin, du chaume ;
\VS{13}L'œuvre de chacun sera manifestée ; car le jour la fera connaître, parce qu'elle sera manifestée par le feu ; et le feu éprouvera quelle sera l'œuvre de chacun.
\VS{14}Si l'œuvre de quelqu'un qui aura édifié dessus, demeure, il en recevra la récompense.
\VS{15}Si l'œuvre de quelqu'un brûle, il en fera la perte ; mais pour lui, il sera sauvé, toutefois comme par le feu.
\VS{16}Ne savez-vous pas que vous êtes le Temple de Dieu, et que l’Esprit de Dieu habite en vous ?
\VS{17}Si quelqu'un détruit le Temple de Dieu, Dieu le détruira ; car le Temple de Dieu est saint, et vous êtes ce [Temple].
\VS{18}Que personne ne s'abuse lui-même ; si quelqu'un d'entre vous croit être sage en ce monde, qu'il se rende fou, afin de devenir sage.
\VS{19}Parce que la sagesse de ce monde est une folie devant Dieu ; car il est écrit : il surprend les sages en leur ruse.
\VS{20}Et encore : le Seigneur connaît que les discours des sages sont vains.
\VS{21}Que personne donc ne se glorifie dans les hommes ; car toutes choses sont à vous ;
\VS{22}Soit Paul, soit Apollos, soit Céphas, soit le monde, soit la vie, soit la mort, soit les choses présentes, soit les choses à venir, toutes choses sont à vous,
\VS{23}Et vous à Christ, et Christ à Dieu.
\Chap{4}
\VerseOne{}Que chacun nous tienne pour Ministres de Christ, et pour dispensateurs des mystères de Dieu.
\VS{2}Mais, au reste, il est exigé des dispensateurs que chacun soit trouvé fidèle.
\VS{3}Pour moi, je me soucie fort peu d'être jugé de vous, ou de jugement d'homme ; et aussi je ne me juge point moi-même.
\VS{4}Car je ne me sens coupable de rien ; mais pour cela je ne suis pas justifié ; mais celui qui me juge, c'est le Seigneur.
\VS{5}C'est pourquoi ne jugez de rien avant le temps, jusqu'à ce que le Seigneur vienne, qui aussi mettra en lumière les choses cachées dans les ténèbres, et qui manifestera les conseils des cœurs ; et alors Dieu rendra à chacun [sa] louange.
\VS{6}Or, mes frères, j'ai tourné [par une façon de parler], ce discours sur moi et sur Apollos, à cause de vous ; afin que vous appreniez de nous à ne point présumer au delà de ce qui est écrit, de peur que l'un pour l'autre vous ne vous enfliez contre autrui.
\VS{7}Car qui est-ce qui met de la différence entre toi et un autre ? et qu'est-ce que tu as, que tu ne l'aies reçu ? et si tu l'as reçu, pourquoi t'en glorifies-tu comme si tu ne l'avais point reçu ?
\VS{8}Vous êtes déjà rassasiés, vous êtes déjà enrichis, vous êtes faits Rois sans nous ; et plût à Dieu que vous régnassiez, afin que nous régnassions aussi avec vous !
\VS{9}Car je pense que Dieu nous a exposés publiquement, [nous] qui sommes les derniers Apôtres, comme des gens condamnés à la mort, vu que nous sommes rendus le spectacle du monde, des Anges et des hommes.
\VS{10}Nous [sommes] fous pour l'amour de Christ, mais vous [êtes] sages en Christ ; nous [sommes] faibles, et vous [êtes] forts ; vous êtes dans l'estime, et nous sommes dans le mépris.
\VS{11}Jusqu'à cette heure nous souffrons la faim et la soif, et nous sommes nus ; nous sommes souffletés, et nous sommes errants çà et là.
\VS{12}Et nous nous fatiguons en travaillant de nos propres mains ; on dit du mal de nous, et nous bénissons ; nous sommes persécutés, et nous le souffrons.
\VS{13}Nous sommes blâmés, et nous prions ; nous sommes faits comme les balayures du monde, et comme le rebut de tous, jusqu'à maintenant.
\VS{14}Je n'écris point ces choses pour vous faire honte ; mais je vous donne des avis comme à mes chers enfants.
\VS{15}Car quand vous auriez dix mille maîtres en Christ, vous n'avez pourtant pas plusieurs pères : car c'est moi qui vous ai engendrés en Jésus-Christ par l'Evangile.
\VS{16}Je vous prie donc d'être mes imitateurs.
\VS{17}C'est pour cela que je vous ai envoyé Timothée, qui est mon fils bien-aimé, et qui est fidèle en [notre] Seigneur ; afin qu'il vous fasse souvenir de mes voies en Christ, et comment j'enseigne partout dans chaque Eglise.
\VS{18}Or quelques-uns se sont glorifiés comme si je ne devais point aller vers vous.
\VS{19}Mais j'irai bientôt vers vous, si le Seigneur le veut ; et je connaîtrai non point la parole de ceux qui se sont glorifiés, mais l'efficace.
\VS{20}Car le Royaume de Dieu ne consiste point en paroles, mais en efficace.
\VS{21}Que voulez-vous ? irai-je à vous avec la verge, ou avec charité, et un esprit de douceur ?
\Chap{5}
\VerseOne{}On entend dire de toutes parts qu'il y a parmi vous de l'impudicité, et même une telle impudicité, qu'entre les Gentils il n'est point fait mention de semblable ; c'est que quelqu'un entretient la femme de son père.
\VS{2}Et cependant vous êtes enflés [d'orgueil], et vous n'avez pas plutôt mené deuil, afin que celui qui a commis cette action fût retranché du milieu de vous.
\VS{3}Mais moi, étant absent de corps, mais présent en esprit, j'ai déjà ordonné comme si j'étais présent, touchant celui qui a ainsi commis une telle action.
\VS{4}Vous et mon esprit étant assemblés au Nom de notre Seigneur Jésus-Christ, [j'ai, dis-je, ordonné], par la puissance de notre Seigneur Jésus-Christ,
\VS{5}Qu'un tel homme soit livré à satan, pour la destruction de la chair ; afin que l'esprit soit sauvé au jour du Seigneur Jésus.
\VS{6}Votre vanité est mal fondée ; ne savez-vous pas qu'un peu de levain fait lever toute la pâte ?
\VS{7}Otez donc le vieux levain, afin que vous soyez une nouvelle pâte, comme vous êtes sans levain ; car Christ, notre Pâque, a été sacrifié pour nous.
\VS{8}C'est pourquoi faisons la fête, non point avec le vieux levain, ni avec un levain de méchanceté et de malice, mais avec les pains sans levain de la sincérité et de la vérité.
\VS{9}Je vous ai écrit dans [ma] Lettre, que vous ne vous mêliez point avec les fornicateurs.
\VS{10}Mais non pas absolument avec les fornicateurs de ce monde, ou avec les avares, ou les ravisseurs, ou les idolâtres ; car autrement certes, il vous faudrait sortir du monde.
\VS{11}Or maintenant, je vous écris que vous ne vous mêliez point avec eux ; c'est-à-dire, que si quelqu'un qui se nomme frère, est fornicateur, ou avare, ou idolâtre, ou médisant, ou ivrogne, ou ravisseur, vous ne mangiez pas même avec un tel homme.
\VS{12}Car aussi qu'ai-je affaire de juger de ceux qui sont de dehors ? ne jugez-vous pas de ceux qui sont de dedans ?
\VS{13}Mais Dieu juge ceux qui sont de dehors. Otez donc d'entre vous-mêmes le méchant.
\Chap{6}
\VerseOne{}Quand quelqu'un d'entre vous a une affaire contre un autre, ose-t-il bien aller en jugement devant les iniques, et il ne va pas devant les Saints ?
\VS{2}Ne savez-vous pas que les Saints jugeront le monde ? or si le monde doit être jugé par vous, êtes-vous indignes de juger des plus petites choses ?
\VS{3}Ne savez-vous pas que nous jugerons les Anges ? combien plus [donc devons-nous juger] des choses qui concernent cette vie ?
\VS{4}Si donc vous avez des procès pour les affaires de cette vie, prenez pour juges ceux qui sont des moins estimés dans l'Eglise.
\VS{5}Je le dis à votre honte : n'y a-t-il donc point de sages parmi vous, non pas même un seul qui puisse juger entre ses frères ?
\VS{6}Mais un frère a des procès contre son frère, et cela devant les infidèles.
\VS{7}C'est même déjà un grand défaut en vous, que vous ayez des procès entre vous. Pourquoi n'endurez-vous pas plutôt qu'on vous fasse tort ? Pourquoi ne souffrez-vous pas plutôt du dommage ?
\VS{8}Mais, au contraire, vous faites tort, et vous causez du dommage, et même à vos frères.
\VS{9}Ne savez-vous pas que les injustes n'hériteront point le Royaume de Dieu ?
\VS{10}Ne vous trompez point vous-mêmes : ni les fornicateurs, ni les idolâtres, ni les adultères, ni les efféminés, ni ceux qui commettent des péchés contre nature, ni les larrons, ni les avares, ni les ivrognes, ni les médisants, ni les ravisseurs, n'hériteront point le royaume de Dieu.
\VS{11}Et quelques-uns de vous étiez tels ; mais vous avez été lavés, mais vous avez été sanctifiés, mais vous avez été justifiés au Nom du Seigneur Jésus, et par l'Esprit de notre Dieu.
\VS{12}Toutes choses me sont permises, mais toutes choses ne conviennent pas ; toutes choses me sont permises, mais je ne serai point assujetti sous la puissance d'aucune chose.
\VS{13}Les viandes sont pour l'estomac, et l'estomac est pour les viandes : mais Dieu détruira l'un et l'autre. Or le corps n'est point pour la fornication, mais pour le Seigneur, et le Seigneur pour le corps.
\VS{14}Et Dieu qui a ressuscité le Seigneur, nous ressuscitera aussi par sa puissance.
\VS{15}Ne savez-vous pas que vos corps sont les membres de Christ ? Oterai-je donc les membres de Christ, pour en faire les membres d'une prostituée ? à Dieu ne plaise !
\VS{16}Ne savez-vous pas que celui qui s'unit avec une prostituée, devient un même corps avec elle ? car deux, est-il dit, seront une même chair.
\VS{17}Mais celui qui est uni au Seigneur, est un même esprit [avec lui].
\VS{18}Fuyez la fornication ; quelque [autre] péché que l'homme commette, il est hors du corps ; mais le fornicateur pèche contre son propre corps.
\VS{19}Ne savez-vous pas que votre corps est le Temple du Saint-Esprit, qui est en vous, et que vous avez de Dieu ? Et vous n'êtes point à vous-mêmes ;
\VS{20}Car vous avez été achetés par prix ; glorifiez donc Dieu en votre corps, et en votre esprit, qui appartiennent à Dieu.
\Chap{7}
\VerseOne{}Or quant aux choses dont vous m'avez écrit : [Je vous dis] qu'il est bon à l'homme de ne pas se marier.
\VS{2}Toutefois pour éviter l'impureté, que chacun ait sa femme, et que chaque femme ait son mari.
\VS{3}Que le mari rende à sa femme la bienveillance qui lui est due ; et que la femme de même la rende à son mari.
\VS{4}Car la femme n'a pas son propre corps en sa puissance, mais [il est en celle] du mari ; et le mari tout de même n'a pas en sa puissance son propre corps, mais [il est en celle] de la femme.
\VS{5}Ne vous privez point l'un de l'autre, si ce n'est par un consentement mutuel, pour un temps, afin que vous vaquiez au jeûne et à la prière, mais après cela retournez ensemble, de peur que satan ne vous tente, par votre incontinence.
\VS{6}Or je dis ceci par permission, et non par commandement.
\VS{7}Car je voudrais que tous les hommes fussent comme moi ; mais chacun a son propre don [lequel il a reçu] de Dieu, l'un en une manière, et l'autre en une autre.
\VS{8}Or je dis à ceux qui ne sont point mariés, et aux veuves, qu'il leur est bon de demeurer comme moi.
\VS{9}Mais s'ils ne sont pas continents, qu'ils se marient ; car il vaut mieux se marier que de brûler.
\VS{10}Et quant à ceux qui sont mariés, je leur commande, non pas moi, mais le Seigneur, que la femme ne se sépare point du mari.
\VS{11}Et si elle s'en sépare, qu'elle demeure sans être mariée, ou qu'elle se réconcilie avec son mari ; que le mari aussi ne quitte point sa femme.
\VS{12}Mais aux autres je leur dis, [et] non pas le Seigneur : Si quelque frère a une femme infidèle, et qu'elle consente d'habiter avec lui, qu'il ne la quitte point.
\VS{13}Et si quelque femme a un mari infidèle, et qu'il consente d'habiter avec elle, qu'elle ne le quitte point.
\VS{14}Car le mari infidèle est sanctifié en la femme, et la femme infidèle est sanctifiée dans le mari ; autrement vos enfants seraient impurs ; or maintenant ils sont saints.
\VS{15}Que si l'infidèle se sépare, qu'il se sépare ; le frère ou la sœur ne sont point asservis dans ce cas-là ; mais Dieu nous a appelés à la paix.
\VS{16}Car que sais-tu, femme, si tu ne sauveras point ton mari ? ou que sais-tu, mari, si tu ne sauveras point ta femme ?
\VS{17}Toutefois que chacun se conduise selon le don qu'il a reçu de Dieu, chacun selon que le Seigneur l'y a appelé ; et c'est ainsi que j'en ordonne dans toutes les Eglises.
\VS{18}Quelqu'un est-il appelé étant circoncis ? qu'il ne ramène point le prépuce. Quelqu'un est-il appelé étant dans le prépuce ? qu'il ne se fasse point circoncire.
\VS{19}La Circoncision n'est rien, et le prépuce aussi n'est rien, mais l'observation des commandements de Dieu.
\VS{20}Que chacun demeure dans la condition où il était quand il a été appelé.
\VS{21}Es-tu appelé étant esclave ? ne t'en mets point en peine ; mais aussi si tu peux être mis en liberté, uses en plutôt :
\VS{22}Car celui qui étant esclave est appelé à notre Seigneur, il est l'affranchi du Seigneur ; et de même celui qui est appelé étant libre, il est l'esclave de Christ.
\VS{23}Vous avez été achetés par prix ; ne devenez point les esclaves des hommes.
\VS{24}[Mes] frères, que chacun demeure envers Dieu dans l'état où il était quand il a été appelé.
\VS{25}Pour ce qui concerne les vierges, je n'ai point de commandement du Seigneur, mais j'en donne avis comme ayant obtenu miséricorde du Seigneur, pour être fidèle.
\VS{26}J'estime donc que cela est bon pour la nécessité présente, en tant qu'il est bon à l'homme d'être ainsi.
\VS{27}Es-tu lié à une femme ? ne cherche point d'en être séparé. Es-tu détaché de ta femme ? ne cherche point de femme.
\VS{28}Que si tu te maries, tu ne pèches point ; et si la vierge se marie, elle ne pèche point aussi ; mais ceux qui sont mariés auront des afflictions en la chair ; or je vous épargne.
\VS{29}Mais je vous dis ceci, mes frères, que le temps est court ; et ainsi que ceux qui ont une femme, soient comme s'ils n'en avaient point ;
\VS{30}Et ceux qui sont dans les pleurs, comme s'ils n'étaient point dans les pleurs ; et ceux qui sont dans la joie, comme s'ils n'étaient point dans la joie ; et ceux qui achètent, comme s'ils ne possédaient point.
\VS{31}Et ceux qui usent de ce monde, comme n'en abusant point : car la figure de ce monde passe.
\VS{32}Or je voudrais que vous fussiez sans inquiétude. Celui qui n'est point marié a soin des choses qui sont du Seigneur, comment il plaira au Seigneur.
\VS{33}Mais celui qui est marié, a soin des choses de ce monde, et comment il plaira à sa femme, [et ainsi] il est divisé.
\VS{34}La femme qui n'est point mariée, et la vierge, a soin des choses qui sont du Seigneur, pour être sainte de corps et d'esprit ; mais celle qui est mariée a soin des choses qui sont du monde, comment elle plaira à son mari.
\VS{35}Or je dis ceci ayant égard à ce qui vous est utile, non point pour vous tendre un piége, mais pour vous porter à ce qui est bienséant, et propre [à vous unir] au Seigneur sans aucune distraction.
\VS{36}Mais si quelqu'un croit que ce soit un déshonneur à sa fille de passer la fleur de son âge, et qu'il faille la marier, qu'il fasse ce qu'il voudra, il ne pèche point ; qu'elle soit mariée.
\VS{37}Mais celui qui demeure ferme en son cœur, n'y ayant point de nécessité [qu'il marie sa fille], mais étant le maître de sa propre volonté a arrêté en son cœur de garder sa fille, il fait bien.
\VS{38}Celui donc qui la marie fait bien, mais celui qui ne la marie pas, fait mieux.
\VS{39}La femme est liée par la Loi pendant tout le temps que son mari est en vie, mais si son mari meurt, elle est en liberté de se marier à qui elle veut ; seulement [que ce soit] en [notre] Seigneur.
\VS{40}Elle est néanmoins plus heureuse si elle demeure ainsi, selon mon avis ; or j'estime que j'ai aussi l'Esprit de Dieu.
\Chap{8}
\VerseOne{}Pour ce qui regarde les choses qui sont sacrifiées aux idoles : nous savons que nous avons tous de la connaissance. La science enfle, mais la charité édifie.
\VS{2}Et si quelqu'un croit savoir quelque chose, il n'a encore rien connu comme il faut connaître ;
\VS{3}Mais si quelqu'un aime Dieu, il est connu de lui.
\VS{4}Pour ce qui regarde donc de manger des choses sacrifiées aux idoles, nous savons que l'idole n'est rien au monde, et qu'il n'y a aucun autre Dieu qu'un seul ;
\VS{5}Car encore qu'il y en ait qui soient appelés dieux, soit au ciel, soit en la terre (comme il y a plusieurs dieux, et plusieurs Seigneurs,)
\VS{6}Nous n'avons pourtant qu'un seul Dieu, [qui est] le Père ; duquel [sont] toutes choses, et nous en lui ; et un seul Seigneur Jésus-Christ, par lequel [sont] toutes choses, et nous par lui.
\VS{7}Mais il n'y a pas en tous la [même] connaissance ; car quelques-uns qui jusqu'à présent font conscience à cause de l'idole, de manger des choses qui ont été sacrifiées à l'idole, en mangent pourtant ; c'est pourquoi leur conscience étant faible, elle en est souillée.
\VS{8}Or la viande ne nous rend pas agréables à Dieu ; car si nous mangeons, nous n'en avons rien davantage ; et si nous ne mangeons point, nous n'en avons pas moins.
\VS{9}Mais prenez garde que cette liberté que vous avez, ne soit en quelque sorte en scandale aux faibles.
\VS{10}Car si quelqu'un te voit, toi qui as de la connaissance, être à table au temple des idoles, la conscience de celui qui est faible, ne sera-t-elle pas induite à manger des choses sacrifiées à l'idole ?
\VS{11}Et ainsi ton frère, qui est faible, pour lequel Christ est mort, périra par ta connaissance.
\VS{12}Or quand vous péchez ainsi contre vos frères, et que vous blessez leur conscience qui est faible, vous péchez contre Christ.
\VS{13}C'est pourquoi, si la viande scandalise mon frère, je ne mangerai jamais de chair, pour ne point scandaliser mon frère.
\Chap{9}
\VerseOne{}Ne suis-je pas Apôtre ? ne suis-je pas libre ? n’ai-je pas vu notre Seigneur Jésus-Christ ? n'êtes-vous pas mon ouvrage au Seigneur ?
\VS{2}Si je ne suis pas Apôtre pour les autres, je le suis au moins pour vous ; car vous êtes le sceau de mon Apostolat au Seigneur.
\VS{3}C'est là mon apologie envers ceux qui me condamnent.
\VS{4}N'avons-nous pas le pouvoir de manger et de boire ?
\VS{5}N'avons-nous pas le pouvoir de mener avec nous une sœur femme, ainsi que les autres Apôtres, et les frères du Seigneur, et Céphas ?
\VS{6}N'y a-t-il que Barnabas et moi qui n'ayons pas le pouvoir de ne point travailler ?
\VS{7}Qui est-ce qui va jamais à la guerre à ses dépens ? qui est-ce qui plante la vigne, et ne mange point de son fruit ? qui est-ce qui paît le troupeau, et ne mange pas du lait du troupeau ?
\VS{8}Dis-je ces choses selon l'homme ? la loi ne dit-elle pas aussi la même chose ?
\VS{9}Car il est écrit dans la Loi de Moise : tu n'emmuselleras point le bœuf qui foule le grain. [Or] Dieu a-t-il soin des bœufs ?
\VS{10}Et n'est-ce pas entièrement pour nous qu'il a dit ces choses ; certes elles sont écrites pour nous ; car celui qui laboure, doit labourer avec espérance ; et celui qui foule le blé, [le foule] avec espérance d'en être participant.
\VS{11}Si nous avons semé des biens spirituels, est-ce une grande chose que nous recueillions de vos biens charnels ?
\VS{12}Et si d'autres usent de ce pouvoir à votre égard, pourquoi n'en userions-nous pas plutôt qu'eux ? cependant nous n'avons point usé de ce pouvoir, mais au contraire nous supportons toutes sortes d'incommodités, afin de ne donner aucun empêchement à l'Evangile de Christ.
\VS{13}Ne savez-vous pas que ceux qui s'emploient aux choses sacrées, mangent de ce qui est sacré ; et que ceux qui servent à l'autel, participent à l'autel ?
\VS{14}Le Seigneur a ordonné tout de même que ceux qui annoncent l'Evangile, vivent de l'Evangile.
\VS{15}Cependant je ne me suis point prévalu d'aucune de ces choses, et je n'écris pas même ceci afin qu'on en use de cette manière envers moi, car j’aimerais mieux mourir, que de voir que quelqu'un anéantît ma gloire.
\VS{16}Car encore que j'évangélise, je n'ai pas de quoi m'en glorifier ; parce que la nécessité m'en est imposée ; et malheur à moi, si je n'évangélise pas !
\VS{17}Mais si je le fais de bon cœur, j’en aurai la récompense ; mais si c'est à regret, je ne fais que m'acquitter de la commission qui m’en a été donnée.
\VS{18}Quelle récompense en ai-je donc ? c'est qu'en prêchant l'Evangile, je prêche l'Evangile de Christ sans apporter aucune dépense, afin que je n'abuse pas de mon pouvoir dans l’Evangile.
\VS{19}Car bien que je sois en liberté à l'égard de tous, je me suis pourtant asservi à tous, afin de gagner plus de personnes.
\VS{20}Et je me suis fait aux Juifs comme Juif, afin de gagner les Juifs ; à ceux qui sont sous la Loi, comme si j'étais sous la Loi, afin de gagner ceux qui sont sous la Loi ;
\VS{21}A ceux qui sont sans Loi, comme si j'étais sans Loi (quoique je ne sois point sans Loi quant à Dieu, mais je suis sous la Loi de Christ,) afin de gagner ceux qui sont sans Loi.
\VS{22}Je me suis fait comme faible aux faibles, afin de gagner les faibles ; je me suis fait toutes choses à tous, afin qu'absolument j'en sauve quelques-uns.
\VS{23}Et je fais cela à cause de l'Evangile, afin que j'en sois fait aussi participant avec les autres.
\VS{24}Ne savez-vous pas que quand on court dans la lice, tous courent bien, mais un seul remporte le prix ? courez [donc] tellement que vous le remportiez.
\VS{25}Or quiconque lutte, vit entièrement de régime ; et quant à ceux-là, ils le font pour avoir une couronne corruptible ; mais nous, pour en avoir une incorruptible.
\VS{26}Je cours donc, [mais] non pas sans savoir comment ; je combats, [mais] non pas comme battant l'air.
\VS{27}Mais je mortifie mon corps, et je me le soumets ; de peur qu'après avoir prêché aux autres, je ne sois trouvé moi-même en quelque sorte non recevable.
\Chap{10}
\VerseOne{}Or mes frères, je ne veux pas que vous ignoriez que nos pères ont tous été sous la nuée, et qu'ils ont tous passé par la mer ;
\VS{2}Et qu'ils ont tous été baptisés par Moïse en la nuée et en la mer ;
\VS{3}Et qu'ils ont tous mangé d'une même viande spirituelle ;
\VS{4}Et qu'ils ont tous bu d'un même breuvage spirituel : car ils buvaient [de l'eau] de la pierre spirituelle qui [les] suivait ; et la pierre était Christ.
\VS{5}Mais Dieu n'a point pris plaisir en plusieurs d'eux ; car ils ont été accablés au désert.
\VS{6}Or ces choses ont été des exemples pour vous, afin que nous ne convoitions point des choses mauvaises, comme eux-mêmes les ont convoitées ;
\VS{7}Et que vous ne deveniez point idolâtres, comme quelques-uns d'eux ; ainsi qu'il est écrit : le peuple s'est assis pour manger et pour boire ; et puis ils se sont levés pour jouer.
\VS{8}Et afin que nous ne nous laissions point aller à la fornication, comme quelques-uns d'eux s'y sont abandonnés, et il en est tombé en un jour vingt-trois mille.
\VS{9}Et que nous ne tentions point Christ, comme quelques-uns d'eux [l'] ont tenté, et ont été détruits par les serpents.
\VS{10}Et que vous ne murmuriez point, comme quelques-uns d'eux ont murmurés, et sont péris par le destructeur.
\VS{11}Or toutes ces choses leur arrivaient en exemple, et elles sont écrites pour notre instruction, comme [étant] ceux [auxquels] les derniers temps sont parvenus.
\VS{12}Que celui donc qui croit demeurer debout, prenne garde qu'il ne tombe.
\VS{13}[Aucune] tentation ne vous a éprouvés, qui n'ait été une [tentation] humaine ; et Dieu est fidèle, qui ne permettra point que vous soyez tentés au-delà de vos forces, mais avec la tentation il vous en fera trouver l'issue, afin que vous la puissiez soutenir.
\VS{14}C'est pourquoi, mes bien-aimés, fuyez l'idolâtrie.
\VS{15}Je [vous] parle comme à des personnes intelligentes ; jugez vous-mêmes de ce que je dis.
\VS{16}La coupe de bénédiction, laquelle nous bénissons, n'est-elle pas la communion du sang de Christ ? et le pain que nous rompons, n'est-il pas la communion du corps de Christ ?
\VS{17}Parce qu'il n'y a qu'un seul pain, nous qui sommes plusieurs, sommes un seul corps ; car nous sommes tous participants du même pain.
\VS{18}Voyez l'Israël selon la chair, ceux qui mangent les sacrifices, ne sont-ils pas participants de l'autel ?
\VS{19}Que dis-je donc ? que l'idole soit quelque chose ? ou que ce qui est sacrifié à l'idole, soit quelque chose ? [Non].
\VS{20}Mais je dis que les choses que les Gentils sacrifient, ils les sacrifient aux démons, et non pas à Dieu ; or je ne veux pas que vous soyez participants des démons.
\VS{21}Vous ne pouvez boire la coupe du Seigneur, et la coupe des démons ; vous ne pouvez être participants de la table du Seigneur, et de la table des démons.
\VS{22}Voulons-nous inciter le Seigneur à la jalousie ? sommes-nous plus forts que lui ?
\VS{23}Toutes choses me sont permises, mais toutes choses ne sont pas convenables ; toutes choses me sont permises, mais toutes choses n'édifient pas.
\VS{24}Que personne ne cherche ce qui lui est propre, mais que chacun [cherche] ce qui est pour autrui.
\VS{25}Mangez de tout ce qui se vend à la boucherie, sans vous en enquérir pour la conscience :
\VS{26}Car la terre [est] au Seigneur, avec tout ce qu'elle contient.
\VS{27}Que si quelqu'un des infidèles vous convie, et que vous y vouliez aller, mangez de tout ce qui est mis devant vous, sans vous en enquérir pour la conscience.
\VS{28}Mais si quelqu'un vous dit : cela est sacrifié aux idoles, n'en mangez point à cause de celui qui vous en a avertis, et à cause de la conscience ; car la terre [est] au Seigneur, avec tout ce qu'elle contient.
\VS{29}Or je dis la conscience, non pas la tienne, mais celle de l'autre ; car pourquoi ma liberté serait-elle condamnée par la conscience d'un autre ?
\VS{30}Et si par la grâce j'en suis participant, pourquoi suis-je blâmé [pour une chose] dont je rends grâces ?
\VS{31}Soit donc que vous mangiez, soit que vous buviez, ou que vous fassiez quelque autre chose, faites tout à la gloire de Dieu.
\VS{32}Soyez tels que vous ne donniez aucun scandale ni aux Juifs, ni aux Grecs, ni à l'Eglise de Dieu.
\VS{33}Comme aussi je complais à tous en toutes choses, ne cherchant point ma commodité propre, mais celle de plusieurs, afin qu'ils soient sauvés.
\Chap{11}
\VerseOne{}Soyez mes imitateurs, comme je [le suis] moi-même de Christ.
\VS{2}Or, mes frères, je vous loue de ce que vous vous souvenez de tout ce qui me concerne, et de ce que vous gardez mes ordonnances, comme je vous les ai données.
\VS{3}Mais je veux que vous sachiez que le Chef de tout homme, c'est Christ ; et que le Chef de la femme, c'est l'homme ; et que le Chef de Christ, c'est Dieu.
\VS{4}Tout homme qui prie, ou qui prophétise, ayant [quelque chose] sur la tête, déshonore sa tête.
\VS{5}Mais toute femme qui prie, ou qui prophétise sans avoir la tête couverte, déshonore sa tête : car c'est la même chose que si elle était rasée.
\VS{6}Si donc la femme n'est pas couverte, qu'on lui coupe les cheveux. Or s'il est déshonnête à la femme d'avoir les cheveux coupés, ou d'être rasée, qu'elle soit couverte.
\VS{7}Car pour ce qui est de l'homme, il ne doit point couvrir sa tête, vu qu'il est l'image et la gloire de Dieu ; mais la femme est la gloire de l'homme.
\VS{8}Parce que l'homme n'a point [été tiré] de la femme, mais la femme [a été tirée] de l'homme.
\VS{9}Et aussi l'homme n'a pas été créé pour la femme, mais la femme pour l'homme.
\VS{10}C'est pourquoi la femme à cause des Anges doit avoir sur la tête une marque qu'elle est sous la puissance [de son mari].
\VS{11}Toutefois ni l'homme n'est point sans la femme, ni la femme sans l'homme en notre Seigneur.
\VS{12}Car comme la femme [est] par l'homme, aussi l'homme est par la femme ; mais toutes choses procèdent de Dieu.
\VS{13}Jugez-en entre vous-mêmes : est-il convenable que la femme prie Dieu sans être couverte ?
\VS{14}La nature même ne vous enseigne-t-elle pas que si l'homme nourrit sa chevelure, ce lui est du déshonneur ;
\VS{15}Mais que si la femme nourrit sa chevelure, ce lui est de la gloire, parce que la chevelure lui est donnée pour couverture.
\VS{16}Que si quelqu'un aime à contester, nous n'avons pas une telle coutume, ni aussi les Eglises de Dieu.
\VS{17}Or en ce que je vais vous dire, je ne vous loue point : c'est que vos assemblées ne sont pas mieux réglées qu'elles l'étaient ; elles le sont moins.
\VS{18}Car premièrement, quand vous vous assemblez dans l'Eglise, j'apprends qu'il y a des divisions parmi vous ; et j'en crois une partie :
\VS{19}Car il faut qu'il y ait même des hérésies parmi vous, afin que ceux qui sont dignes d'approbation, soient manifestés parmi vous.
\VS{20}Quand donc vous vous assemblez [ainsi] tous ensemble, ce n'est pas manger la Cène du Seigneur.
\VS{21}Car lorsqu'il s'agit de prendre le repas, chacun prend par avance son souper particulier, en sorte que l'un a faim, et l'autre fait bonne chère.
\VS{22}N'avez-vous donc pas de maisons pour manger et pour boire ? ou méprisez-vous l'Eglise de Dieu ? et faites-vous honte à ceux qui n'ont rien ? que vous dirai-je ? vous louerai-je ? je ne vous loue point en ceci.
\VS{23}Car j'ai reçu du Seigneur ce qu'aussi je vous ai donné ; c’est que le Seigneur Jésus la nuit qu'il fut trahi, prit du pain ;
\VS{24}Et après avoir rendu grâces il le rompit, et dit : prenez, mangez : ceci est mon corps [qui est] rompu pour vous ; faites ceci en mémoire de moi.
\VS{25}De même aussi après le souper, il prit la coupe, en disant : cette coupe est la nouvelle alliance en mon sang ; faites ceci toutes les fois que vous en boirez, en mémoire de moi.
\VS{26}Car toutes les fois que vous mangerez de ce pain, et que vous boirez de cette coupe, vous annoncerez la mort du Seigneur jusques à ce qu'il vienne.
\VS{27}C'est pourquoi quiconque mangera de ce pain, ou boira de la coupe du Seigneur indignement, sera coupable du corps et du sang du Seigneur.
\VS{28}Que chacun donc s'éprouve soi-même, et ainsi qu'il mange de ce pain, et qu'il boive de cette coupe ;
\VS{29}Car celui qui [en] mange et qui [en] boit indignement, mange et boit sa condamnation, ne distinguant point le corps du Seigneur.
\VS{30}Et c'est pour cela que plusieurs sont faibles et malades parmi vous, et que plusieurs dorment.
\VS{31}Car si nous nous jugions nous-mêmes, nous ne serions point jugés.
\VS{32}Mais quand nous sommes jugés, nous sommes enseignés par le Seigneur, afin que nous ne soyons point condamnés avec le monde.
\VS{33}C'est pourquoi, mes frères, quand vous vous assemblez pour manger, attendez-vous l'un l'autre.
\VS{34}Et si quelqu'un a faim, qu'il mange en sa maison, afin que vous ne vous assembliez pas pour votre condamnation. Touchant les autres points, j'en ordonnerai quand je serai arrivé.
\Chap{12}
\VerseOne{}Or pour ce qui regarde les dons spirituels, je ne veux point, mes frères, que vous [en] soyez ignorants.
\VS{2}Vous savez que vous étiez Gentils, entraînés après les idoles muettes, selon que vous étiez menés.
\VS{3}C'est pourquoi je vous fais savoir que nul homme parlant par l'Esprit de Dieu, ne dit que Jésus doit être rejeté ; et que nul ne peut dire que par le Saint-Esprit, que Jésus est le Seigneur.
\VS{4}Or il y a diversité de dons, mais il n'y a qu'un même Esprit.
\VS{5}Il y a aussi diversité de ministères, mais il n'y a qu'un même Seigneur.
\VS{6}Il y a aussi diversité d'opérations ; mais il n'y a qu'un même Dieu, qui opère toutes choses en tous.
\VS{7}Or à chacun est donnée la lumière de l'Esprit pour procurer l'utilité [commune].
\VS{8}Car à l'un est donnée par l'Esprit, la parole de sagesse ; et à l'autre par le même Esprit, la parole de connaissance ;
\VS{9}Et à un autre, la foi par ce même Esprit ; à un autre, les dons de guérison par ce même Esprit ;
\VS{10}Et à un autre, les opérations des miracles ; à un autre, la prophétie ; à un autre, le don de discerner les esprits ; à un autre, la diversité de Langues ; et à un autre, le don d'interpréter les Langues.
\VS{11}Mais un seul et même Esprit fait toutes ces choses, distribuant à chacun ses dons comme il le trouve à propos.
\VS{12}Car comme le corps n'est qu'un, et cependant il a plusieurs membres, mais tous les membres de ce corps, qui n'est qu'un, quoiqu'ils soient plusieurs, ne sont qu'un corps, il en est de même de Christ.
\VS{13}Car nous avons tous été baptisés d'un même Esprit, pour être un même corps, soit Juifs, soit Grecs, soit esclaves, soit libres, nous avons tous, dis-je, été abreuvés d'un même Esprit.
\VS{14}Car aussi le corps n'est pas un seul membre, mais plusieurs.
\VS{15}Si le pied dit : parce que je ne suis pas la main, je ne suis point du corps ; n'est-il pas pourtant du corps ?
\VS{16}Et si l'oreille dit : parce que je ne suis pas l'œil, je ne suis point du corps ; n'est-elle pas pourtant du corps ?
\VS{17}Si tout le corps est l'œil, où sera l'ouïe ? si tout est l'ouïe, où sera l'odorat ?
\VS{18}Mais maintenant Dieu a placé chaque membre dans le corps, comme il a voulu.
\VS{19}Et si tous étaient un seul membre, où serait le corps ?
\VS{20}Mais maintenant il y a plusieurs membres, toutefois il n'y a qu'un seul corps.
\VS{21}Et l'œil ne peut pas dire à la main : je n'ai que faire de toi ; ni aussi la tête aux pieds : je n'ai que faire de vous.
\VS{22}Et qui plus est, les membres du corps qui semblent être les plus faibles, sont beaucoup plus nécessaires.
\VS{23}Et ceux que nous estimons être les moins honorables au corps, nous les ornons avec plus de soin, et les parties qui sont en nous les moins belles à voir, sont les plus parées.
\VS{24}Car les parties qui sont belles en nous, n'en ont pas besoin ; mais Dieu a apporté ce tempérament dans notre corps, qu'il a donné plus d'honneur à ce qui en manquait ;
\VS{25}Afin qu'il n'y ait point de division dans le corps, mais que les membres aient un soin mutuel les uns des autres.
\VS{26}Et soit que l'un des membres souffre quelque chose, tous les membres souffrent avec lui ; ou soit que l'un des membres soit honoré, tous les membres ensemble s'en réjouissent.
\VS{27}Or vous êtes le corps de Christ, et vous êtes chacun un de ses membres.
\VS{28}Et Dieu a mis dans l'Eglise, d'abord des Apôtres, ensuite des Prophètes, en troisième lieu des Docteurs, ensuite les miracles, puis les dons de guérisons, les secours, les gouvernements, les diversités de Langues.
\VS{29}Tous sont-ils Apôtres ? tous sont-ils Prophètes ? tous sont-ils Docteurs ? tous ont-ils le don des miracles ?
\VS{30}Tous ont-ils les dons de guérisons ? tous parlent-ils [diverses] Langues ? tous interprètent-ils ?
\VS{31}Or désirez avec ardeur des dons plus excellents, et je vais vous en montrer un chemin qui surpasse encore de beaucoup.
\Chap{13}
\VerseOne{}Quand je parlerais toutes les langues des hommes, et même des Anges, si je n'ai pas la charité, je suis [comme] l'airain qui résonne, ou [comme] la cymbale retentissante.
\VS{2}Et quand j'aurais le don de prophétie, que je connaîtrais tous les mystères, [et que j'aurais] toute sorte de science ; et quand j'aurais toute la foi [qu'on puisse avoir], en sorte que je transportasse les montagnes, si je n'ai pas la charité, je ne suis rien.
\VS{3}Et quand je distribuerais tout mon bien pour la nourriture des pauvres, et que je livrerais mon corps pour être brûlé, si je n'ai pas la charité : cela ne me sert de rien.
\VS{4}La charité est patience ; elle est douce ; la charité n'est point envieuse ; la charité n'use point d'insolence ; elle ne s'enorgueillit point ;
\VS{5}Elle ne se porte point déshonnêtement ; elle ne cherche point son propre profit ; elle ne s'aigrit point ; elle ne pense point à mal ;
\VS{6}Elle ne se réjouit point de l'injustice ; mais elle se réjouit de la vérité ;
\VS{7}Elle endure tout, elle croit tout, elle espère tout, elle supporte tout.
\VS{8}La charité ne périt jamais, au lieu que quant aux prophéties, elles seront abolies ; et quant aux Langues, elles cesseront ; et quant à la connaissance, elle sera abolie.
\VS{9}Car nous connaissons en partie, et nous prophétisons en partie.
\VS{10}Mais quand la perfection sera venue, alors ce qui est en partie sera aboli.
\VS{11}Quand j'étais enfant, je parlais comme un enfant, je jugeais comme un enfant, je pensais comme un enfant ; mais quand je suis devenu homme, j'ai aboli ce qui était de l'enfance.
\VS{12}Car nous voyons maintenant par un miroir obscurément, mais alors nous verrons face à face ; maintenant je connais en partie, mais alors je connaîtrai selon que j'ai été aussi connu.
\VS{13}Or maintenant ces trois choses demeurent, la foi, l'espérance, et la charité ; mais la plus excellente de ces [vertus] c'est la charité.
\Chap{14}
\VerseOne{}Recherchez la charité. Désirez avec ardeur les dons spirituels, mais surtout de prophétiser.
\VS{2}Parce que celui qui parle une Langue [inconnue], ne parle point aux hommes, mais à Dieu, car personne ne l'entend, et les mystères qu'il prononce ne sont que pour lui.
\VS{3}Mais celui qui prophétise, édifie, exhorte et console les hommes [qui l'entendent].
\VS{4}Celui qui parle une Langue [inconnue], s'édifie lui-même ; mais celui qui prophétise, édifie l'Eglise.
\VS{5}Je désire bien que vous parliez tous [diverses] Langues, mais beaucoup plus que vous prophétisiez, car celui qui prophétise est plus grand que celui qui parle [diverses] Langues, si ce n'est qu'il interprète, afin que l'Eglise en reçoive de l'édification.
\VS{6}Maintenant donc, mes frères, si je viens à vous, et que je parle des Langues [inconnues], que vous servira cela, si je ne vous parle par révélation, ou par science, ou par prophétie, ou par doctrine ?
\VS{7}De même, si les choses inanimées qui rendent leur son, soit un hautbois, soit une harpe, ne forment des tons différents, comment connaîtra-t-on ce qui est sonné sur le hautbois, ou sur la harpe ?
\VS{8}Et si la trompette rend un son qu'on n'entend pas, qui est-ce qui se préparera à la bataille ?
\VS{9}De même si vous ne prononcez dans votre langage une parole qui puisse être entendue, comment entendra-t-on ce qui se dit ? car vous parlerez en l'air.
\VS{10}Il y a, selon qu'il se rencontre, tant de divers sons dans le monde, et cependant aucun de ces sons n'est muet.
\VS{11}Mais si je ne sais point ce qu'on veut signifier par la parole, je serai barbare à celui qui parle ; et celui qui parle me sera barbare.
\VS{12}Ainsi puisque vous désirez avec ardeur des dons spirituels, cherchez d'en avoir abondamment pour l'édification de l'Eglise.
\VS{13}C'est pourquoi que celui qui parle une Langue [inconnue], prie de telle sorte qu'il interprète.
\VS{14}Car si je prie en une Langue [inconnue], mon esprit prie, mais l'intelligence que j'en ai, est sans fruit.
\VS{15}Quoi donc ? je prierai d'esprit, mais je prierai aussi d'une manière à être entendu ; je chanterai d'esprit, mais je chanterai aussi d'une manière à être entendu.
\VS{16}Autrement si tu bénis d'esprit, comment celui qui est du simple peuple, dira-t-il Amen à ton action de grâces, puisqu'il ne sait ce que tu dis ?
\VS{17}Il est bien vrai que tu rends grâces ; mais un autre n'en est pas édifié.
\VS{18}Je rends grâces à mon Dieu que je parle plus de Langues que vous tous.
\VS{19}Mais j'aime mieux prononcer dans l'Eglise cinq paroles d'une manière à être entendu, afin que j'instruise aussi les autres, que dix mille paroles en une Langue [inconnue].
\VS{20}Mes frères, ne soyez point des enfants en prudence, mais soyez de petits enfants en malice ; et par rapport à la prudence, soyez des hommes faits.
\VS{21}Il est écrit dans la Loi : je parlerai à ce peuple par des gens d'une autre Langue, et par des lèvres étrangères ; et ainsi ils ne m'entendront point, dit le Seigneur.
\VS{22}C’est pourquoi les Langues sont pour un signe, non point aux croyants, mais aux infidèles ; la prophétie, au contraire, [est un signe] non point aux infidèles, mais aux croyants.
\VS{23}Si donc toute l'Eglise s'assemble en un [corps], et que tous parlent des Langues [étrangères], et qu'il entre des gens du commun, ou des infidèles, ne diront-ils pas que vous êtes hors du sens ?
\VS{24}Mais si tous prophétisent, et qu'il entre quelque infidèle, ou quelqu'un du commun, il est convaincu par tous, et il est jugé de tous.
\VS{25}Et ainsi les secrets de son cœur sont manifestés, de sorte qu'il se jettera sur sa face, et adorera Dieu, et il publiera que Dieu est véritablement parmi vous.
\VS{26}Que sera-ce donc, mes frères ? c'est que toutes les fois que vous vous assemblerez, selon que chacun de vous aura ou un Psaume, ou une instruction, ou une Langue [étrangère], ou une révélation, ou une interprétation, que tout se fasse pour l'édification.
\VS{27}Et si quelqu'un parle une Langue [inconnue], que cela se fasse par deux, ou tout au plus par trois, et cela par tour ; mais qu'il y en ait un qui interprète.
\VS{28}Que s'il n'y a point d'interprète, que [cet homme] se taise dans l'Eglise, et qu'il parle à soi-même, et à Dieu.
\VS{29}Et que deux ou trois prophètes parlent, et que les autres en jugent.
\VS{30}Et si quelque chose est révélée à un autre qui est assis, que le premier se taise.
\VS{31}Car vous pouvez tous prophétiser l'un après l'autre, afin que tous apprennent, et que tous soient consolés.
\VS{32}Et les Esprits des Prophètes sont sujets aux Prophètes.
\VS{33}Car Dieu n'est point un [Dieu] de confusion, mais de paix, comme [on le voit] dans toutes les Eglises des Saints.
\VS{34}Que les femmes qui sont parmi vous se taisent dans les Eglises ; car il ne leur est point permis de parler, mais [elles doivent] être soumises, comme aussi la Loi le dit.
\VS{35}Et si elles veulent apprendre quelque chose, qu'elles interrogent leurs maris dans la maison ; car il est malhonnête que les femmes parlent dans l'Eglise.
\VS{36}La parole de Dieu est-elle procédée de vous ? ou est-elle parvenue seulement à vous ?
\VS{37}Si quelqu'un croit être Prophète, ou spirituel, qu'il reconnaisse que les choses que je vous écris sont des commandements du Seigneur.
\VS{38}Et si quelqu'un est ignorant, qu'il soit ignorant.
\VS{39}C'est pourquoi, mes frères, désirez avec ardeur de prophétiser, et n'empêchez point de parler [diverses] Langues.
\VS{40}Que toutes choses se fassent avec bienséance, et avec ordre.
\Chap{15}
\VerseOne{}Or, mes frères, je vous fais savoir l'Evangile que je vous ai annoncé, et que vous avez reçu, et auquel vous vous tenez fermes ;
\VS{2}Et par lequel vous êtes sauvés, si vous le retenez en quelle manière je vous l'ai annoncé ; à moins que vous n'ayez cru en vain.
\VS{3}Car avant toutes choses, je vous ai donné ce que j'avais aussi reçu, [savoir], que Christ est mort pour nos péchés, selon les Ecritures ;
\VS{4}Et qu'il a été enseveli, et qu'il est ressuscité le troisième jour, selon les Ecritures ;
\VS{5}Et qu'il a été vu de Céphas, et ensuite des Douze.
\VS{6}Depuis il a été vu de plus de cinq cents frères à une fois, dont plusieurs sont encore vivants, et quelques-uns sont morts.
\VS{7}Ensuite il a été vu de Jacques, et puis de tous les Apôtres.
\VS{8}Et après tous, il a été vu aussi de moi, comme d'un avorton.
\VS{9}Car je suis le moindre des Apôtres, qui ne suis pas digne d'être appelé Apôtre, parce que j'ai persécuté l'Eglise de Dieu.
\VS{10}Mais par la grâce de Dieu je suis ce que je suis ; et sa grâce envers moi n'a point été vaine, mais j'ai travaillé beaucoup plus qu'eux tous ; toutefois non point moi, mais la grâce de Dieu qui est avec moi.
\VS{11}Soit donc moi, soit eux, nous prêchons ainsi, et vous l'avez cru ainsi.
\VS{12}Or si on prêche que Christ est ressuscité des morts, comment disent quelques-uns d'entre vous qu'il n'y a point de résurrection des morts ?
\VS{13}Car s'il n'y a point de résurrection des morts, Christ aussi n'est point ressuscité.
\VS{14}Et si Christ n'est point ressuscité, notre prédication est donc vaine, et votre foi aussi est vaine.
\VS{15}Et même nous sommes de faux témoins de la part de Dieu : car nous avons rendu témoignage de la part de Dieu qu'il a ressuscité Christ ; lequel pourtant il n'a pas ressuscité, si les morts ne ressuscitent point.
\VS{16}Car si les morts ne ressuscitent point, Christ aussi n'est point ressuscité.
\VS{17}Et si Christ n'est point ressuscité, votre foi est vaine, et vous êtes encore dans vos péchés.
\VS{18}Ceux donc aussi qui dorment en Christ, sont péris.
\VS{19}Si nous n'avons d'espérance en Christ que pour cette vie seulement, nous sommes les plus misérables de tous les hommes.
\VS{20}Mais maintenant Christ est ressuscité des morts, et il a été fait les prémices de ceux qui dorment.
\VS{21}Car puisque la mort est par un seul homme, la résurrection des morts est aussi par un seul homme.
\VS{22}Car comme tous meurent en Adam, de même aussi tous seront vivifiés en Christ.
\VS{23}Mais chacun en son rang, les prémices, c'est Christ ; puis ceux qui sont de Christ [seront vivifiés] en son avénement.
\VS{24}Et après viendra la fin, quand il aura remis le Royaume à Dieu le Père, et quand il aura aboli tout empire, et toute puissance, et toute force.
\VS{25}Car il faut qu'il règne jusqu'à ce qu'il ait mis tous ses ennemis sous ses pieds.
\VS{26}L'ennemi qui sera détruit le dernier, c'est la mort.
\VS{27}Car [Dieu] a assujetti toutes choses sous ses pieds ; or quand il est dit que toutes choses lui sont assujetties, il est évident que celui qui lui a assujetti toutes choses est excepté.
\VS{28}Et après que toutes choses lui auront été assujetties, alors aussi le Fils lui-même sera assujetti à celui qui lui a assujetti toutes choses ; afin que Dieu soit tout en tous.
\VS{29}Autrement que feront ceux qui sont baptisés pour les morts, si absolument les morts ne ressuscitent point ? pourquoi donc sont-ils baptisés pour les morts ?
\VS{30}Pourquoi aussi sommes-nous en danger à toute heure ?
\VS{31}Par notre gloire que j'ai en notre Seigneur Jésus-Christ, je meurs de jour en jour.
\VS{32}Si j'ai combattu contre les bêtes à Ephèse, par des vues humaines, quel profit en ai-je si les morts ne ressuscitent point ? mangeons et buvons, car demain nous mourrons.
\VS{33}Ne soyez point séduits : les mauvaises compagnies corrompent les bonnes mœurs.
\VS{34}Réveillez-vous [pour vivre] justement, et ne péchez point ; car quelques-uns sont sans la connaissance de Dieu ; je vous [le] dis à votre honte.
\VS{35}Mais quelqu'un dira : comment ressuscitent les morts, et en quel corps viendront-ils ?
\VS{36}Ô fou ! ce que tu sèmes n'est point vivifié, s'il ne meurt.
\VS{37}Et quant à ce que tu sèmes, tu ne sèmes point le corps qui naîtra, mais le grain nu, selon qu'il se rencontre, de blé, ou de quelque autre grain.
\VS{38}Mais Dieu lui donne le corps comme il veut, et à chacune des semences son propre corps.
\VS{39}Toute chair n'est pas une même sorte de chair ; mais autre est la chair des hommes, et autre la chair des bêtes, et autre celle des poissons, et autre celle des oiseaux.
\VS{40}[Il y a] aussi des corps célestes, et des corps terrestres ; mais autre est la gloire des célestes, et autre celle des terrestres.
\VS{41}Autre est la gloire du soleil, et autre la gloire de la lune, et autre la gloire des étoiles ; car une étoile est différente d'une autre étoile en gloire.
\VS{42}Il en sera aussi de même en la résurrection des morts ; [le corps] est semé en corruption, il ressuscitera incorruptible.
\VS{43}Il est semé en déshonneur, il ressuscitera en gloire ; il est semé en faiblesse, il ressuscitera en force.
\VS{44}Il est semé corps animal, il ressuscitera corps spirituel : il y a un corps animal, et il y a un corps spirituel.
\VS{45}Comme aussi il est écrit : le premier homme Adam a été fait en âme vivante ; et le dernier Adam en esprit vivifiant.
\VS{46}Or ce qui est spirituel, n'est pas le premier : mais ce qui est animal ; et puis ce qui est spirituel.
\VS{47}Le premier homme étant de la terre, est tiré de la poussière ; mais le second homme [savoir] le Seigneur, [est] du Ciel.
\VS{48}Tel qu'est celui qui est tiré de la poussière, tels aussi sont ceux qui sont tirés de la poussière ; et tel qu'est le céleste, tels aussi sont les célestes.
\VS{49}Et comme nous avons porté l'image de celui qui est tiré de la poussière, nous porterons aussi l'image du céleste.
\VS{50}Voici donc ce que je dis, mes frères, c'est que la chair et le sang ne peuvent point hériter le Royaume de Dieu, et que la corruption n'hérite point l'incorruptibilité.
\VS{51}Voici, je vous dis un mystère : nous ne dormirons pas tous, mais nous serons tous transmués ;
\VS{52}En un moment, et en un clin d'œil, à la dernière trompette, car la trompette sonnera, et les morts ressusciteront incorruptibles, et nous serons transmués.
\VS{53}Car il faut que ce corruptible revête l'incorruptibilité, et que ce mortel revête l'immortalité.
\VS{54}Or quand ce corruptible aura revêtu l'incorruptibilité, et que ce mortel aura revêtu l'immortalité, alors cette parole de l'Ecriture sera accomplie : la mort est détruite par la victoire.
\VS{55}Où [est], ô mort, ton aiguillon ? où [est], ô sépulcre, ta victoire ?
\VS{56}Or l'aiguillon de la mort, c'est le péché ; et la puissance du péché, c'est la Loi.
\VS{57}Mais grâces à Dieu, qui nous a donné la victoire par notre Seigneur Jésus-Christ.
\VS{58}C'est pourquoi, mes frères bien-aimés, soyez fermes, inébranlables, vous appliquant toujours avec un nouveau zèle à l’œuvre du Seigneur ; sachant que votre travail ne vous sera pas inutile auprès du Seigneur.
\Chap{16}
\VerseOne{}Touchant la collecte qui se fait pour les Saints, faites comme j'en ai ordonné aux Eglises de Galatie.
\VS{2}C'est que chaque premier jour de la semaine, chacun de vous mette à part chez soi, ce qu'il pourra assembler suivant la prospérité [que Dieu lui accordera], afin que lorsque je viendrai, les collectes ne soient point à faire.
\VS{3}Puis quand je serai arrivé, j'enverrai ceux que vous approuverez par vos Lettres pour porter votre libéralité à Jérusalem.
\VS{4}Et s'il est à propos que j'y aille moi-même, ils viendront aussi avec moi.
\VS{5}J'irai donc vers vous, ayant passé par la Macédoine ; car je passerai par la Macédoine.
\VS{6}Et peut-être que je séjournerai parmi vous, ou même que j'y passerai l'hiver ; afin que vous me conduisiez partout où j’irai.
\VS{7}Car je ne vous veux point voir maintenant en passant, mais j'espère que je demeurerai avec vous quelque temps, si le Seigneur le permet.
\VS{8}Toutefois je demeurerai à Ephèse jusqu'à la Pentecôte.
\VS{9}Car une grande porte, [et de grande] efficace m'y est ouverte, mais il y a plusieurs adversaires.
\VS{10}Que si Timothée vient, prenez garde qu'il soit en sûreté parmi vous ; car il s'emploie à l'œuvre du Seigneur comme moi-même.
\VS{11}Que personne donc ne le méprise ; mais conduisez-le en toute sûreté, afin qu'il vienne me trouver ; car je l'attends avec les frères.
\VS{12}Quant à Apollos notre frère, je l'ai beaucoup prié d'aller vers vous avec les frères, mais il n'a nullement eu la volonté d'y aller maintenant : toutefois il y ira quand il en aura la commodité.
\VS{13}Veillez, soyez fermes en la foi, portez-vous vaillamment, fortifiez-vous.
\VS{14}Que toutes vos affaires se fassent en charité.
\VS{15}Or, mes frères, vous connaissez la famille de Stéphanas, [et vous savez] qu'elle est les prémices de l'Achaïe, et qu'ils se sont entièrement appliqués au service des Saints ;
\VS{16}Je vous prie de vous soumettre à eux, et à chacun de ceux qui s'emploient à l'œuvre [du Seigneur], et qui travaillent avec nous.
\VS{17}Or je me réjouis de la venue de Stéphanas, de Fortunat, et d'Achaïque ; parce qu'ils ont suppléé à ce que vous ne pouvez pas faire pour moi.
\VS{18}Car ils ont recréé mon esprit et le vôtre : ayez donc de la considération pour de telles personnes.
\VS{19}Les Eglises d'Asie vous saluent. Aquilas et Priscille, avec l'Eglise qui [est] en leur maison, vous saluent affectueusement en [notre] Seigneur.
\VS{20}Tous les frères vous saluent. Saluez-vous l'un l'autre par un saint baiser.
\VS{21}La salutation [est] de la propre main de moi Paul.
\VS{22}S'il y a quelqu'un qui n'aime point le Seigneur Jésus-Christ, qu'il soit anathème, Maranatha.
\VS{23}Que la grâce de notre Seigneur Jésus-Christ soit avec vous !
\VS{24}Mon amour s'étend à vous tous en Jésus-Christ. Amen.
\PPE{}
\end{multicols}
