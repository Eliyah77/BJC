\ShortTitle{1Pierre}\BookTitle{1Pierre}\BFont
\begin{multicols}{2}
\Chap{1}
\VerseOne{}Pierre, Apôtre de Jésus-Christ, aux étrangers qui êtes dispersés dans le pays du Pont, en Galatie, en Cappadoce, en Asie, et en Bithynie,
\VS{2}Elus selon la prescience de Dieu le Père, par l'Esprit sanctifiant, pour obéir à Jésus-Christ et pour obtenir l'aspersion de son sang : que la grâce et la paix vous soient multipliées.
\VS{3}Béni [soit] Dieu, le Père de notre Seigneur Jésus-Christ, qui par sa grande miséricorde nous a régénérés pour avoir une espérance vive, par la résurrection de Jésus-Christ d'entre les morts ;
\VS{4}D'obtenir l'héritage incorruptible, qui ne se peut souiller, ni flétrir, conservé dans les cieux pour nous,
\VS{5}Qui sommes gardés par la puissance de Dieu, par la foi, afin que nous obtenions le salut, qui est prêt d'être révélé au dernier temps.
\VS{6}En quoi vous vous réjouissez, quoique vous soyez maintenant affligés pour un peu de temps par diverses tentations, vu que cela est convenable ;
\VS{7}Afin que l'épreuve de votre foi, beaucoup plus précieuse que l'or, qui périt, et qui toutefois est éprouvé par le feu, vous tourne à louange, à honneur, et à gloire, quand Jésus-Christ sera révélé ;
\VS{8}Lequel, quoique vous ne l'ayez point vu, vous aimez ; en qui, quoique maintenant vous ne le voyiez point, vous croyez, et vous vous réjouissez d'une joie ineffable et glorieuse ;
\VS{9}Remportant la fin de votre foi, [savoir] le salut des âmes.
\VS{10}Duquel salut les Prophètes qui ont prophétisé de la grâce qui était réservée pour vous, se sont enquis, et l'ont diligemment recherché ;
\VS{11}Recherchant soigneusement quand, et en quel temps, l'Esprit [prophétique] de Christ qui [était] en eux, rendant par avance témoignage, déclarait les souffrances qui devaient arriver à Christ, et la gloire qui les devait suivre.
\VS{12}Et il leur fut révélé que ce n'était pas pour eux-mêmes, mais pour nous, qu'ils administraient ces choses, lesquelles ceux qui vous ont prêché l'Evangile, par le Saint-Esprit envoyé du Ciel, vous ont maintenant annoncées, et dans lesquelles les Anges désirent de regarder jusqu'au fond.
\VS{13}Vous donc, ayant les reins de votre entendement ceints, et étant sobres, espérez parfaitement en la grâce qui vous est présentée, jusqu'à ce que Jésus-Christ soit révélé ;
\VS{14}Comme des enfants obéissants, ne vous conformant point à vos convoitises d'autrefois, pendant votre ignorance.
\VS{15}Mais comme celui qui vous a appelés est saint, vous aussi de même soyez saints dans toute [votre] conversation ;
\VS{16}Parce qu'il est écrit : soyez saints, car je suis saint.
\VS{17}Et si vous invoquez comme votre Père celui qui sans avoir égard à l'apparence des personnes, juge selon l'œuvre d'un chacun, conduisez-vous avec crainte durant le temps de votre séjour temporel ;
\VS{18}Sachant que vous avez été rachetés de votre vaine conduite, qui vous avait été enseignée par vos pères, non point par des choses corruptibles, comme par argent, ou par or ;
\VS{19}Mais par le précieux sang de Christ, comme de l'agneau sans défaut et sans tache,
\VS{20}Déjà ordonné avant la fondation du monde, mais manifesté dans les derniers temps pour vous ;
\VS{21}Qui par lui croyez en Dieu qui l'a ressuscité des morts, et qui lui a donné la gloire, afin que votre foi et votre espérance fussent en Dieu.
\VS{22}Ayant donc purifié vos âmes en obéissant à la vérité par le Saint-Esprit, afin que vous ayez une amitié fraternelle qui soit sans hypocrisie, aimez-vous l'un l'autre tendrement d'un cœur pur.
\VS{23}Vu que vous avez été régénérés, non par une semence corruptible, mais [par une semence] incorruptible, [savoir] par la parole de Dieu, vivante, et permanente à toujours.
\VS{24}Parce que toute chair est comme l'herbe, et toute la gloire de l'homme comme la fleur de l'herbe ; l'herbe est séchée, et sa fleur est tombée ;
\VS{25}Mais la parole du Seigneur demeure éternellement ; et c'est cette parole qui vous a été évangélisée.
\Chap{2}
\VerseOne{}Vous étant donc dépouillés de toute malice, et de toute fraude, de dissimulations, d'envies et de toutes médisances,
\VS{2}Désirez ardemment, comme des enfants nouvellement nés, [de vous nourrir] du lait spirituel et pur afin que vous croissiez par lui.
\VS{3}Si toutefois vous avez goûté combien le Seigneur est bon.
\VS{4}Et vous approchant de lui, qui est la Pierre vive, rejetée des hommes, mais choisie de Dieu, et précieuse,
\VS{5}Vous aussi comme des pierres vives êtes édifiés pour être une maison spirituelle, et une sainte Sacrificature, afin d'offrir des sacrifices spirituels, agréables à Dieu par Jésus-Christ.
\VS{6}C'est pourquoi il est dit dans l'Ecriture : voici, je mets en Sion la maîtresse pierre du coin, élue et précieuse ; et celui qui croira en elle, ne sera point confus.
\VS{7}Elle est donc précieuse pour vous qui croyez ; mais par rapport aux rebelles, [il est dit] : la pierre que ceux qui bâtissaient ont rejetée est devenue la maîtresse pierre du coin, une pierre d'achoppement, une pierre de scandale.
\VS{8}Lesquels heurtent contre la parole, et sont rebelles ; à quoi aussi ils ont été destinés.
\VS{9}Mais vous êtes la race élue, la Sacrificature royale, la nation sainte, le peuple acquis, afin que vous annonciez les vertus de celui qui vous a appelés des ténèbres à sa merveilleuse lumière ;
\VS{10}[Vous] qui autrefois n'[étiez] point [son] peuple, mais qui maintenant êtes le peuple de Dieu ; vous qui n'aviez point obtenu miséricorde, mais qui maintenant avez obtenu miséricorde.
\VS{11}Mes bien-aimés, je vous exhorte, que comme étrangers et voyageurs, vous vous absteniez des convoitises charnelles, qui font la guerre à l'âme ;
\VS{12}Ayant une conduite honnête avec les Gentils, afin qu'au lieu qu'ils médisent de vous comme de malfaiteurs, ils glorifient Dieu au jour de la visitation, pour vos bonnes œuvres qu'ils auront vues.
\VS{13}Soyez donc soumis à tout établissement humain, pour l'amour de Dieu : soit au Roi, comme à celui qui est par-dessus les autres ;
\VS{14}Soit aux Gouverneurs, comme à ceux qui sont envoyés de sa part, pour punir les méchants et pour honorer les gens de bien.
\VS{15}Car c'est là la volonté de Dieu, qu'en faisant bien, vous fermiez la bouche à l'ignorance des hommes fous.
\VS{16}Comme libres, et non pas comme ayant la liberté pour servir de voile à la méchanceté, mais comme serviteurs de Dieu.
\VS{17}Portez honneur à tous. Aimez tous vos frères. Craignez Dieu. Honorez le Roi.
\VS{18}Serviteurs, soyez soumis en toute crainte à vos maîtres, non seulement à ceux qui sont bons et équitables, mais aussi à ceux qui sont fâcheux :
\VS{19}Car c'est une chose agréable à Dieu si quelqu'un à cause de la conscience qu'il a envers Dieu, endure des afflictions, souffrant injustement.
\VS{20}Autrement, quel honneur en aurez-vous, si recevant des soufflets pour avoir mal fait, vous le souffrez patiemment ? mais si en faisant bien vous êtes pourtant affligés, et que vous le souffriez patiemment, voilà où Dieu prend plaisir.
\VS{21}Car aussi vous êtes appelés à cela ; vu même que Christ a souffert pour nous, nous laissant un modèle, afin que vous suiviez ses traces ;
\VS{22}Lui qui n'a point commis de péché, et dans la bouche duquel il n'a point été trouvé de fraude.
\VS{23}Qui lorsqu'on lui disait des outrages, n'en rendait point, et quand on lui faisait du mal, n'usait point de menaces ; mais il se remettait à celui qui juge justement.
\VS{24}Lequel même a porté nos péchés en son corps sur le bois ; afin qu'étant morts au péché, nous vivions à la justice ; [et] par la meurtrissure duquel même vous avez été guéris.
\VS{25}Car vous étiez comme des brebis errantes, mais maintenant vous êtes convertis au Pasteur et à l'Evêque de vos âmes.
\Chap{3}
\VerseOne{}Que les femmes aussi soient soumises à leurs maris, afin que même s'il y en a qui n'obéissent point à la parole, ils soient gagnés sans la parole, par la conduite de [leurs] femmes ;
\VS{2}Lorsqu'ils auront vu la pureté de votre conduite, accompagnée de crainte.
\VS{3}Et que leur ornement ne soit point celui de dehors, qui consiste dans la frisure des cheveux, dans une parure d'or, et dans la magnificence des habits ;
\VS{4}Mais que leur ornement consiste dans l'homme caché dans le cœur, [c'est-à-dire] dans l'incorruptibilité d'un esprit doux et paisible, qui est d'un grand prix devant Dieu ;
\VS{5}Car c'est ainsi que se paraient autrefois les saintes femmes qui espéraient en Dieu, et qui demeuraient soumises à leurs maris ;
\VS{6}Comme Sara, qui obéissait à Abraham, l'appelant [son] Seigneur ; de laquelle vous êtes les filles en faisant bien, lors même que vous ne craignez rien de ce que vous pourriez avoir à craindre.
\VS{7}Vous maris aussi, comportez-vous discrètement avec elles, comme avec un vaisseau plus fragile, [c'est-à-dire], féminin, leur portant du respect, comme ceux qui êtes aussi avec elles héritiers de la grâce de vie, afin que vos prières ne soient point interrompues.
\VS{8}Enfin soyez tous d'un même sentiment, remplis de compassion l'un envers l'autre, vous entr'aimant fraternellement, miséricordieux, [et] doux.
\VS{9}Ne rendant point mal pour mal, ni outrage pour outrage ; mais, au contraire, bénissant ; sachant que vous êtes appelés à cela, afin que vous héritiez la bénédiction.
\VS{10}Car celui qui veut aimer sa vie et voir [ses] jours bienheureux, qu'il garde sa langue de mal, et ses lèvres de prononcer aucune fraude ;
\VS{11}Qu'il se détourne du mal, et qu'il fasse le bien ; qu'il recherche la paix, et qu'il tâche de se la procurer.
\VS{12}Car les yeux du Seigneur sont sur les justes, et ses oreilles [sont attentives] à leurs prières ; mais la face du Seigneur est contre ceux qui se conduisent mal.
\VS{13}Or qui est-ce qui vous fera du mal, si vous êtes les imitateurs de celui qui est bon ?
\VS{14}Que si toutefois vous souffrez quelque chose pour la justice, vous êtes bienheureux ; mais ne craignez point les maux dont ils veulent vous faire peur, et [n'en] soyez point troublés ;
\VS{15}Mais sanctifiez le Seigneur dans vos cœurs, et soyez toujours prêts à répondre avec douceur et avec respect à chacun qui vous demande raison de l'espérance qui est en vous.
\VS{16}Ayant une bonne conscience, afin que ceux qui blâment votre bonne conduite en Christ, soient confus en ce qu'ils médisent de vous comme de malfaiteurs.
\VS{17}Car il vaut mieux que vous souffriez en faisant bien, si la volonté de Dieu est que vous souffriez, qu'en faisant mal.
\VS{18}Car aussi Christ a souffert une fois pour les péchés, lui juste pour les injustes, afin de nous amener à Dieu ; étant mort en la chair mais vivifié par l'Esprit.
\VS{19}Par lequel aussi étant allé, il a prêché aux esprits qui sont dans la prison ;
\VS{20}[Et] qui avaient été autrefois incrédules, quand la patience de Dieu les attendait une fois, durant les jours de Noé, tandis que l'Arche se préparait dans laquelle un petit nombre, savoir huit personnes furent sauvées par l'eau.
\VS{21}A quoi aussi maintenant répond la figure qui nous sauve, [c'est-à-dire], le Baptême ; non point celui par lequel les ordures de la chair sont nettoyées, mais la promesse faite à Dieu d'une conscience pure, par la résurrection de Jésus-Christ ;
\VS{22}Qui est à la droite de Dieu, étant allé au Ciel ; [et] auquel sont assujettis les Anges, et les dominations, et les puissances.
\Chap{4}
\VerseOne{}Puis donc que Christ a souffert pour nous en la chair, vous aussi soyez armés de cette même pensée, que celui qui a souffert en la chair, a désisté du péché ;
\VS{2}Afin que durant le temps qui reste en la chair, vous ne viviez plus selon les convoitises des hommes, mais selon la volonté de Dieu.
\VS{3}Car il nous doit suffire d'avoir accompli la volonté des Gentils, durant le temps de notre vie passée, quand nous nous abandonnions aux impudicités, aux convoitises, à l'ivrognerie, aux excès dans le manger et dans le boire, et aux idolâtries abominables ;
\VS{4}Ce que [ces Gentils] trouvant fort étrange, ils vous blâment de ce que vous ne courez pas avec eux dans un même abandonnement de dissolution.
\VS{5}Mais ils rendront compte à celui qui est prêt à juger les vivants et les morts.
\VS{6}Car c'est aussi pour cela qu'il a été évangélisé aux morts, afin qu'ils fussent jugés selon les hommes en la chair, et qu'ils vécussent selon Dieu dans l'esprit.
\VS{7}Or la fin de toutes choses est proche : soyez donc sobres, et vigilants à prier.
\VS{8}Mais surtout, ayez entre vous une ardente charité : car la charité couvrira une multitude de péchés.
\VS{9}Soyez hospitaliers les uns envers les autres, sans murmures.
\VS{10}Que chacun selon le don qu'il a reçu, l'emploie pour le service des autres, comme bons dispensateurs de la différente grâce de Dieu.
\VS{11}Si quelqu'un parle, [qu'il parle] comme [annonçant] les paroles de Dieu, si quelqu'un administre, [qu'il administre] comme par la puissance que Dieu lui en a fournie ; afin qu'en toutes choses Dieu soit glorifié par Jésus-Christ, auquel appartient la gloire et la force aux siècles des siècles, Amen !
\VS{12}Mes bien-aimés, ne trouvez point étrange quand vous êtes [comme] dans une fournaise pour votre épreuve, comme s'il vous arrivait quelque chose d'extraordinaire.
\VS{13}Mais en ce que vous participez aux souffrances de Christ, réjouissez vous ; afin qu'aussi à la révélation de sa gloire, vous vous réjouissiez avec allégresse.
\VS{14}Si on vous dit des injures pour le Nom de Christ, vous êtes bienheureux : car l'Esprit de gloire et de Dieu repose sur vous, lequel est blasphémé par ceux qui [vous noircissent] mais pour vous vous le glorifiez.
\VS{15}Que nul de vous ne souffre comme meurtrier, ou larron, ou malfaiteur, ou curieux des affaires d'autrui.
\VS{16}Mais si quelqu'un souffre comme Chrétien, qu'il n'en ait point de honte, mais qu'il glorifie Dieu en cela.
\VS{17}Car il est temps que le jugement commence par la maison de Dieu ; or [s'il commence] premièrement par nous, quelle sera la fin de ceux qui n'obéissent point à l'Evangile de Dieu ?
\VS{18}Et si le juste est difficilement sauvé, où comparaîtra le méchant et le pécheur ?
\VS{19}Que ceux-là donc aussi qui souffrent par la volonté de Dieu, puisqu'ils font ce qui est bon lui recommandent leurs âmes, comme au fidèle Créateur.
\Chap{5}
\VerseOne{}Je prie les Anciens qui sont parmi vous, moi qui suis Ancien avec eux, et témoin des souffrances de Christ, et participant de la gloire qui doit être révélée, [et je leur dis] :
\VS{2}Paissez le Troupeau de Christ qui vous est commis, en prenant garde sur lui, non point par contrainte, mais volontairement ; non point pour un gain déshonnête, mais par un principe d'affection.
\VS{3}Et non point comme ayant domination sur les héritages [du Seigneur], mais en telle manière que vous soyez pour modèle au Troupeau.
\VS{4}Et quand le souverain Pasteur apparaîtra, vous recevrez la couronne incorruptible de gloire.
\VS{5}De même, vous jeunes gens, soyez soumis aux Anciens, et ayant tous de la soumission l'un pour l'autre, soyez parés par-dedans d'humilité, parce que Dieu résiste aux orgueilleux, mais il fait grâce aux humbles.
\VS{6}Humiliez-vous donc sous la puissante main de Dieu, afin qu'il vous élève quand il en sera temps ;
\VS{7}Lui remettant tout ce qui peut vous inquiéter : car il a soin de vous.
\VS{8}Soyez sobres, [et] veillez : car le diable, votre adversaire, tourne autour de vous comme un lion rugissant, cherchant qui il pourra dévorer.
\VS{9}Résistez-lui [donc] en [demeurant] fermes dans la foi, sachant que les mêmes souffrances s'accomplissent en la compagnie de vos frères, qui sont dans le monde.
\VS{10}Or le Dieu de toute grâce, qui nous a appelés à sa gloire éternelle en Jésus-Christ, après que vous aurez souffert un peu de temps, vous rende accomplis, vous affermisse, vous fortifie, [et] vous établisse.
\VS{11}A lui soit la gloire et la force, aux siècles des siècles, Amen !
\VS{12}Je vous ai écrit brièvement par Silvain notre frère, que je crois vous être fidèle, vous déclarant et vous protestant que la grâce de Dieu dans laquelle vous êtes est la véritable.
\VS{13}[L'Eglise] qui est à Babylone, élue avec vous, et Marc mon fils, vous saluent.
\VS{14}Saluez-vous l'un l'autre par un baiser de charité. Que la paix soit à vous tous, qui êtes en Jésus-Christ, Amen !
\PPE{}
\end{multicols}
