\ShortTitle{Galates}\BookTitle{Galates}\BFont
\begin{multicols}{2}
\Chap{1}
\VerseOne{}Paul Apôtre, non de la part des hommes, ni de la part d'aucun homme, mais de la part de Jésus-Christ, et de la part de Dieu le Père, qui l'a ressuscité des morts ;
\VS{2}Et tous les frères qui sont avec moi, aux Eglises de Galatie.
\VS{3}Que la grâce et la paix vous soient données de la part de Dieu le Père, et de la part de notre Seigneur Jésus-Christ :
\VS{4}Qui s'est donné lui-même pour nos péchés, afin que selon la volonté de Dieu notre Père, il nous retirât du présent siècle mauvais.
\VS{5}A lui soit gloire aux siècles des siècles ; Amen !
\VS{6}Je m'étonne qu'abandonnant [Jésus-] Christ, qui vous avait appelés par sa grâce, vous ayez été si promptement transportés à un autre Evangile.
\VS{7}Qui n'est pas un autre [Evangile], mais il y a des gens qui vous troublent, et qui veulent renverser l'Evangile de Christ.
\VS{8}Mais quand nous-mêmes [vous évangéliserions], ou quand un Ange du Ciel vous évangéliserait outre ce que nous vous avons évangélisé, qu'il soit anathème.
\VS{9}Comme nous l'avons déjà dit, je le dis encore maintenant : si quelqu'un vous évangélise outre ce que vous avez reçu, qu'il soit anathème.
\VS{10}Car maintenant prêché-je les hommes, ou Dieu ? ou cherché-je à complaire aux hommes ? Certes si je complaisais encore aux hommes, je ne serais pas le serviteur de Christ.
\VS{11}Or mes frères, je vous déclare que l'Evangile que j'ai annoncé, n'est point selon l'homme.
\VS{12}Parce que je ne l'ai point reçu ni appris d'aucun homme, mais par la révélation de Jésus-Christ.
\VS{13}Car vous avez appris quelle a été autrefois ma conduite dans le Judaïsme, [et] comment je persécutais à outrance l'Eglise de Dieu, et la ravageais ;
\VS{14}Et j'avançais dans le Judaïsme plus que plusieurs de mon âge dans ma nation ; étant le plus ardent zélateur des traditions de mes pères.
\VS{15}Mais quand ç'a été le bon plaisir de Dieu, qui m'avait choisi dès le ventre de ma mère, et qui m'a appelé par sa grâce,
\VS{16}De révéler son Fils en moi, afin que je l'évangélisasse parmi les Gentils, je ne commençai pas d'abord par prendre conseil de la chair et du sang ;
\VS{17}Et je ne retournai point à Jérusalem vers ceux qui avaient été Apôtres avant moi, mais je m'en allai en Arabie, et je repassai à Damas.
\VS{18}Puis je retournai trois ans après à Jérusalem pour visiter Pierre, et je demeurai chez lui quinze jours.
\VS{19}Et je ne vis aucun des autres Apôtres, sinon Jacques, le frère du Seigneur.
\VS{20}Or dans les choses que je vous écris, voici, [je vous dis] devant Dieu, que je ne mens point.
\VS{21}J'allai ensuite dans les pays de Syrie et de Cilicie.
\VS{22}Or j'étais inconnu de visage aux Eglises de Judée qui étaient en Christ ;
\VS{23}Mais elles avaient seulement ouï dire : celui qui autrefois nous persécutait, annonce maintenant la foi qu'il détruisait autrefois.
\VS{24}Et elles glorifiaient Dieu à cause de moi.
\Chap{2}
\VerseOne{}Depuis je montai encore à Jérusalem quatorze ans après, avec Barnabas, et je pris aussi avec moi Tite.
\VS{2}Or j'y montai par révélation, et je conférai avec ceux [de Jérusalem] touchant l'Evangile que je prêche parmi les Gentils, même en particulier avec ceux qui sont en estime, afin qu'en quelque sorte je ne courusse, ou n'eusse couru en vain.
\VS{3}Et même on n'obligea point Tite, qui était avec moi, à se faire circoncire, quoiqu'il fût Grec.
\VS{4}Et ce fut à cause des faux frères qui s'étaient introduits dans [l'Eglise], et qui y étaient entrés couvertement pour épier notre liberté, que nous avons en Jésus-Christ, afin de nous ramener dans la servitude.
\VS{5}Et nous ne leur avons point cédé par aucune sorte de soumission, non pas même un moment ; afin que la vérité de l'Evangile demeurât parmi vous.
\VS{6}Et je ne suis en rien différent de ceux qui semblent être quelque chose, quels qu'ils aient été autrefois, (Dieu n'ayant point d'égard à l'apparence extérieure de l'homme) car ceux qui sont en estime ne m'ont rien communiqué [de plus].
\VS{7}Mais, au contraire, quand ils virent que la Prédication de l'Evangile du Prépuce m'était commise, comme celle de la Circoncision l'était à Pierre :
\VS{8}(Car celui qui a opéré avec efficace par Pierre en la charge d'Apôtre envers la Circoncision, a aussi opéré avec efficace par moi envers les Gentils.)
\VS{9}Jacques, dis-je, Céphas, et Jean (qui sont estimés être les Colonnes) ayant reconnu la grâce que j'avais reçue, me donnèrent, à moi et à Barnabas, la main d'association, afin que nous allassions vers les Gentils, et qu'ils allassent eux vers ceux de la Circoncision ;
\VS{10}[Nous recommandant] seulement de nous souvenir des pauvres ; ce que je me suis aussi étudié de faire.
\VS{11}Mais quand Pierre fut venu à Antioche, je lui résistai en face, parce qu'il méritait d'être repris.
\VS{12}Car avant que quelques-uns fussent venus de la part de Jacques, il mangeait avec les Gentils ; mais quand ceux-là furent venus, il s'en retira, et s'en sépara, craignant ceux qui étaient de la Circoncision.
\VS{13}Les autres Juifs usaient aussi de dissimulation comme lui, tellement que Barnabas lui-même se laissait entraîner par leur dissimulation.
\VS{14}Mais quand je vis qu'ils ne marchaient pas de droit pied selon la vérité de l'Evangile, je dis à Pierre devant tous : si toi qui es Juif, vis comme les Gentils, et non pas comme les Juifs, pourquoi contrains-tu les Gentils à Judaïser ?
\VS{15}Nous qui sommes Juifs de naissance, et non point pécheurs d'entre les Gentils ;
\VS{16}Sachant que l'homme n'est pas justifié par les œuvres de la Loi, mais seulement par la foi en [Jésus-]Christ, nous, dis-je, nous avons cru en Jésus-Christ, afin que nous fussions justifiés par la foi de Christ, et non point par les œuvres de la Loi ; parce que personne ne sera justifié par les œuvres de la Loi.
\VS{17}Or si en cherchant d'être justifiés par Christ, nous sommes aussi trouvés pécheurs, Christ est-il pourtant ministre du péché ? à Dieu ne plaise !
\VS{18}Car si je rebâtissais les choses que j'ai renversées, je montrerais que j'ai été moi-même un prévaricateur.
\VS{19}Mais par la Loi je suis mort à la Loi, afin que je vive à Dieu.
\VS{20}Je suis crucifié avec Christ, et je vis, non pas maintenant moi, mais Christ vit en moi ; et ce que je vis maintenant en la chair, je le vis en la foi du Fils de Dieu, qui m'a aimé, et qui s'est donné lui-même pour moi.
\VS{21}Je n'anéantis point la grâce de Dieu : car si la justice est par la Loi, Christ est donc mort inutilement.
\Chap{3}
\VerseOne{}Ô Galates insensés ! qui est-ce qui vous a ensorcelés pour faire que vous n'obéissiez point à la vérité, vous à qui Jésus-Christ a été auparavant portrait devant les yeux, et crucifié entre vous ?
\VS{2}Je voudrais seulement entendre ceci de vous : avez-vous reçu l'Esprit par les œuvres de la Loi, ou par la prédication de la foi ?
\VS{3}Etes-vous si insensés, qu'en ayant commencé par l'Esprit, maintenant vous finissiez par la chair ?
\VS{4}Avez-vous tant souffert en vain ? si toutefois c'est en vain.
\VS{5}Celui donc qui vous donne l'Esprit, et qui produit en vous les dons miraculeux, [le fait-il] par les œuvres de la Loi, ou par la prédication de la foi ?
\VS{6}Comme Abraham a cru à Dieu, et il lui a été imputé à justice ;
\VS{7}Sachez aussi que ceux qui sont de la foi, sont enfants d'Abraham.
\VS{8}Aussi l'Ecriture prévoyant que Dieu justifierait les Gentils par la foi, a auparavant évangélisé à Abraham, en lui [disant] : toutes les nations seront bénies en toi.
\VS{9}C'est pourquoi ceux qui sont de la foi, sont bénis avec le fidèle Abraham.
\VS{10}Mais tous ceux qui sont des œuvres de la Loi, sont sous la malédiction ; car il est écrit : maudit est quiconque ne persévère pas dans toutes les choses qui sont écrites au Livre de la Loi pour les faire.
\VS{11}Or que par la Loi personne ne soit justifié devant Dieu, cela paraît [par ce qui est dit] : que le juste vivra de la foi.
\VS{12}Or la Loi n'est pas de la foi ; mais l'homme qui aura fait ces choses, vivra par elles.
\VS{13}Christ nous a rachetés de la malédiction de la Loi, quand il a été fait malédiction pour nous ; (car il est écrit : maudit est quiconque pend au bois.)
\VS{14}Afin que la bénédiction d'Abraham parvînt aux Gentils par Jésus-Christ, et que nous reçussions par la foi l'Esprit qui avait été promis.
\VS{15}Mes frères, je vais vous parler à la manière des hommes. Si une alliance faite par un homme, est confirmée, nul ne la casse, ni n'y ajoute.
\VS{16}Or les promesses ont été faites à Abraham, et à sa semence ; il n'est pas dit, et aux semences, comme s'il avait parlé de plusieurs, mais comme parlant d'une seule, et à sa semence : qui est Christ.
\VS{17}Voici donc ce que je dis : c'est que quant à l'alliance qui a été auparavant confirmée par Dieu en Christ, la Loi qui est venue quatre cent-trente ans après, ne peut point l'annuler, pour abolir la promesse.
\VS{18}Car si l'héritage est par la Loi, il n'est point par la promesse ; or Dieu l'a donné à Abraham par la promesse.
\VS{19}A quoi donc [sert] la Loi ? elle a été ajoutée à cause des transgressions, jusqu'à ce que vînt la semence à [l'égard de] laquelle la promesse avait été faite ; et elle a été ordonnée par les Anges, par le ministère d'un Médiateur.
\VS{20}Or le Médiateur n'est pas d'un seul : mais Dieu est un seul.
\VS{21}La Loi donc [a-t-elle été ajoutée] contre les promesses de Dieu ? nullement. Car si la Loi eût été donnée pour pouvoir vivifier, véritablement la justice serait de la Loi.
\VS{22}Mais l'Ecriture a montré que tous les hommes étaient pécheurs, afin que la promesse par la foi en Jésus-Christ fût donnée à ceux qui croient.
\VS{23}Or avant que la foi vînt, nous étions gardés sous la Loi, étant renfermés [sous l'attente] de la foi qui devait être révélée.
\VS{24}La Loi a donc été notre Pédagogue [pour nous amener] à Christ, afin que nous soyons justifiés par la foi.
\VS{25}Mais la foi étant venue, nous ne sommes plus sous le Pédagogue.
\VS{26}Parce que vous êtes tous enfants de Dieu par la foi en Jésus-Christ.
\VS{27}Car vous tous qui avez été baptisés en Christ, vous avez revêtu Christ ;
\VS{28}[Où] il n'y a ni Juif ni Grec ; [où] il n'y a ni esclave ni libre ; [où] il n'y a ni mâle ni femelle ; car vous êtes tous un en Jésus-Christ.
\VS{29}Or si vous êtes de Christ, vous êtes donc la semence d'Abraham, et héritiers selon la promesse.
\Chap{4}
\VerseOne{}Or je dis que pendant tout le temps que l'héritier est un enfant, il n'est en rien différent du serviteur, quoiqu'il soit Seigneur de tout.
\VS{2}Mais il est sous des tuteurs et des curateurs jusqu'au temps déterminé par le père.
\VS{3}Nous aussi, lorsque nous étions des enfants, nous étions asservis sous les rudiments du monde.
\VS{4}Mais quand l'accomplissement du temps est venu, Dieu a envoyé son Fils, né d'une femme, [et] soumis à la Loi.
\VS{5}Afin qu'il rachetât ceux qui étaient sous la Loi, et que nous reçussions l'adoption des enfants.
\VS{6}Et parce que vous êtes enfants, Dieu a envoyé l'Esprit de son Fils dans vos cœurs, criant Abba, [c'est-à-dire] Père.
\VS{7}Maintenant donc tu n'es plus serviteur, mais fils ; or si tu es fils, tu es aussi héritier de Dieu par Christ.
\VS{8}Mais lorsque vous ne connaissiez point Dieu, vous serviez ceux qui de [leur] nature ne sont point Dieux.
\VS{9}Et maintenant que vous avez connu Dieu, ou plutôt que vous avez été connus de Dieu, comment retournez-vous encore à ces faibles et misérables éléments, auxquels vous voulez encore servir comme auparavant ?
\VS{10}Vous observez les jours, les mois, les temps et les années.
\VS{11}Je crains pour vous que peut-être je n'aie travaillé en vain parmi vous.
\VS{12}Soyez comme moi ; car je [suis] aussi comme vous ; je vous [en] prie, mes frères ; vous ne m'avez fait aucun tort.
\VS{13}Et vous savez comment je vous ai ci-devant évangélisé dans l'infirmité de la chair.
\VS{14}Et vous n'avez point méprisé ni rejeté mon épreuve, telle qu'elle était en ma chair ; mais vous m'avez reçu comme un Ange de Dieu, et comme Jésus-Christ même.
\VS{15}Quelle était donc la déclaration [que vous faisiez] de votre bonheur ? car je vous rends témoignage que, s'il eût été possible, vous eussiez arraché vos yeux, et vous me les eussiez donnés.
\VS{16}Suis-je donc devenu votre ennemi, en vous disant la vérité ?
\VS{17}Ils sont jaloux de vous, [mais] ce n'est pas comme il faut ; au contraire, ils vous veulent exclure, afin que vous soyez jaloux d'eux.
\VS{18}Mais il est bon d'être toujours zélé pour le bien, et de ne l'être pas seulement quand je suis présent avec vous.
\VS{19}Mes petits enfants, pour lesquels enfanter je travaille de nouveau, jusqu'à ce que Christ soit formé en vous :
\VS{20}Je voudrais être maintenant avec vous, et changer de langage, car je suis en perplexité sur votre sujet.
\VS{21}Dites-moi, vous qui voulez être sous la Loi, n'entendez-vous point la Loi ?
\VS{22}Car il est écrit qu'Abraham a eu deux fils, l'un de la servante, et l'autre de la [femme] libre.
\VS{23}Mais celui qui était de la servante, naquit selon la chair ; et celui qui était de la [femme] libre, naquit par la promesse.
\VS{24}Or ces choses doivent être entendues par allégorie : car ce sont les deux alliances ; l'une du mont de Sinaï, qui ne produit que des esclaves, et c'est Agar.
\VS{25}Car ce nom d'Agar veut dire Sinaï ; qui est une montagne en Arabie, et correspondante à la Jérusalem de maintenant, laquelle sert avec ses enfants.
\VS{26}Mais la Jérusalem d'en haut est [la femme] libre, et c'est la mère de nous tous.
\VS{27}Car il est écrit : réjouis-toi, stérile, qui n'enfantais point ; efforce-toi, et t'écrie, toi qui n'étais point en travail d'enfant ; car il y a beaucoup plus d'enfants de [celle qui avait été] laissée, que de celle qui avait un mari.
\VS{28}Or pour nous, mes frères, nous sommes enfants de la promesse, ainsi qu'Isaac.
\VS{29}Mais comme alors celui qui était né selon la chair, persécutait celui [qui était né] selon l'Esprit, il [en est] de même aussi maintenant.
\VS{30}Mais que dit l'Ecriture ? chasse la servante et son fils : car le fils de la servante ne sera point héritier avec le fils de la [femme] libre.
\VS{31}Or mes frères, nous ne sommes point enfants de la servante, mais de la [femme] libre.
\Chap{5}
\VerseOne{}Tenez-vous donc fermes dans la liberté à l'égard de laquelle Christ nous a affranchis, et ne vous soumettez plus au joug de la servitude.
\VS{2}Voici, je vous dis moi Paul, que si vous êtes circoncis, Christ ne vous profitera de rien.
\VS{3}Et de plus je proteste à tout homme qui se circoncit, qu'il est obligé d'accomplir toute la Loi.
\VS{4}Christ devient inutile à l'égard de vous tous qui [voulez] être justifiés par la Loi ; et vous êtes déchus de la grâce.
\VS{5}Mais pour nous, nous espérons par l'esprit d'être justifiés par la foi.
\VS{6}Car en Jésus-Christ ni la Circoncision ni le prépuce n'ont aucune efficace, mais la foi opérante par la charité.
\VS{7}Vous couriez bien : qui est-ce [donc] qui vous a empêchés d'obéir à la vérité ?
\VS{8}Cette persuasion ne vient pas de celui qui vous appelle.
\VS{9}Un peu de levain fait lever toute la pâte.
\VS{10}Je m'assure de vous en [notre] Seigneur, que vous n'aurez point d'autre sentiment ; mais celui qui vous trouble en portera la condamnation, quel qu'il soit.
\VS{11}Et pour moi, mes frères, si je prêche encore la Circoncision, pourquoi est-ce que je souffre encore la persécution ? le scandale de la croix est donc aboli.
\VS{12}Plût à Dieu que ceux qui vous troublent fussent retranchés !
\VS{13}Car, mes frères, vous avez été appelés à la liberté ; seulement ne [prenez] pas une telle liberté pour une occasion de vivre selon la chair ; mais servez-vous l'un l'autre avec charité.
\VS{14}Car toute la Loi est accomplie dans cette seule parole : tu aimeras ton Prochain comme toi-même.
\VS{15}Mais si vous vous mordez et vous dévorez les uns les autres, prenez garde que vous ne soyez consumés l'un par l'autre.
\VS{16}Je [vous] dis donc : marchez selon l'Esprit ; et vous n'accomplirez point les convoitises de la chair.
\VS{17}Car la chair convoite contre l'esprit, et l'esprit contre la chair ; et ces choses sont opposées l'une à l'autre ; tellement que vous ne faites point les choses que vous voudriez.
\VS{18}Or si vous êtes conduits par l'Esprit, vous n'êtes point sous la Loi.
\VS{19}Car les œuvres de la chair sont évidentes, lesquelles sont l'adultère, la fornication, la souillure, l'impudicité,
\VS{20}L'idolâtrie, l'empoisonnement, les inimitiés, les querelles, les jalousies, les colères, les disputes, les divisions, les sectes,
\VS{21}Les envies, les meurtres, les ivrogneries, les gourmandises, et les choses semblables à celles-là ; au sujet desquelles je vous prédis, comme je vous l'ai déjà dit, que ceux qui commettent de telles choses n'hériteront point le Royaume de Dieu.
\VS{22}Mais le fruit de l'Esprit est la charité, la joie, la paix, un esprit patient, la bonté, la bénéficence, la fidélité, la douceur, la tempérance.
\VS{23}Or la Loi ne condamne point de telles choses.
\VS{24}Or ceux qui sont de Christ, ont crucifié la chair avec ses affections et ses convoitises.
\VS{25}Si nous vivons par l'Esprit, conduisons-nous aussi par l'Esprit.
\VS{26}Ne désirons point la vaine gloire, en nous provoquant l'un l'autre, et en nous portant envie l'un à l'autre.
\Chap{6}
\VerseOne{}Mes frères, lorsqu'un homme est surpris en quelque faute, vous qui êtes spirituels, redressez un tel homme avec un esprit de douceur ; et toi, prends garde à toi-même, de peur que tu ne sois aussi tenté.
\VS{2}Portez les charges les uns des autres, et accomplissez ainsi la Loi de Christ.
\VS{3}Car si quelqu'un s'estime être quelque chose, quoiqu'il ne soit rien, il se séduit lui-même.
\VS{4}Or que chacun examine ses actions, et alors il aura de quoi se glorifier en lui-même seulement, et non dans les autres.
\VS{5}Car chacun portera son propre fardeau.
\VS{6}Que celui qui est enseigné dans la parole, fasse participant de tous ses biens celui qui l'enseigne.
\VS{7}Ne vous abusez point, Dieu ne peut être moqué ; car ce que l'homme aura semé, il le moissonnera aussi.
\VS{8}C'est pourquoi celui qui sème à sa chair, moissonnera aussi de la chair la corruption ; mais celui qui sème à l'Esprit, moissonnera de l'Esprit la vie éternelle.
\VS{9}Or ne nous relâchons point en faisant le bien ; car nous moissonnerons en la propre saison, si nous ne devenons point lâches.
\VS{10}C'est pourquoi pendant que nous en avons le temps, faisons du bien à tous ; mais principalement aux domestiques de la foi.
\VS{11}Vous voyez quelle grande Lettre je vous ai écrite de ma propre main.
\VS{12}Tous ceux qui cherchent à se rendre agréables dans ce qui regarde la chair, sont ceux qui vous contraignent d'être circoncis ; afin seulement qu'ils ne souffrent point de persécution pour la croix de Christ.
\VS{13}Car ceux-là même qui sont circoncis ne gardent point la Loi ; mais ils veulent que vous soyez circoncis, afin de se glorifier en votre chair.
\VS{14}Mais pour moi, à Dieu ne plaise que je me glorifie sinon en la croix de notre Seigneur Jésus-Christ, par lequel le monde m'est crucifié, et moi au monde !
\VS{15}Car en Jésus-Christ ni la Circoncision, ni le prépuce n'ont aucune efficace, mais la nouvelle créature.
\VS{16}Et à l'égard de tous ceux qui marcheront selon cette règle, que la paix et la miséricorde soient sur eux, et sur l'Israël de Dieu.
\VS{17}Au reste, que personne ne me donne du chagrin ; car je porte en mon corps les flétrissures du Seigneur Jésus.
\VS{18}Mes frères, que la grâce de notre Seigneur Jésus-Christ soit avec votre esprit ; Amen !
\PPE{}
\end{multicols}
