\ShortTitle{Philippiens}\BookTitle{Philippiens}\BFont
\begin{multicols}{2}
\Chap{1}
\VerseOne{}Paul et Timothée, Serviteurs de Jésus-Christ, à tous les Saints en Jésus-Christ qui sont à Philippes, avec les Evêques et les Diacres.
\VS{2}Que la grâce et la paix vous soient données de par Dieu, notre Père, et de par le Seigneur Jésus-Christ.
\VS{3}Je rends grâces à mon Dieu toutes les fois que je fais mention de vous.
\VS{4}En priant toujours pour vous tous avec joie dans toutes mes prières.
\VS{5}A cause de votre attachement à l'Evangile, depuis le premier jour jusqu'à maintenant.
\VS{6}Etant assuré de cela même, que celui qui a commencé cette bonne œuvre en vous, l'achèvera jusqu'à la journée de Jésus-Christ :
\VS{7}Comme il est juste que je pense ainsi de vous tous, parce que je retiens dans mon cœur que vous avez tous été participants de la grâce avec moi dans mes liens, et dans la défense et la confirmation de l'Evangile.
\VS{8}Car Dieu m'est témoin que je vous aime tous tendrement, conformément à la charité de Jésus-Christ.
\VS{9}Et je lui demande cette grâce, que votre charité abonde encore de plus en plus avec connaissance et toute intelligence.
\VS{10}Afin que vous discerniez les choses contraires, pour être purs et sans achoppement jusqu'à la journée de Christ ;
\VS{11}Etant remplis de fruits de justice, qui [sont] par Jésus-Christ, à la gloire et à la louange de Dieu.
\VS{12}Or mes frères, je veux bien que vous sachiez que les choses qui me sont arrivées, sont arrivées pour un plus grand avancement de l'Evangile.
\VS{13}De sorte que mes liens en Christ ont été rendus célèbres dans tout le Prétoire, et partout ailleurs ;
\VS{14}Et que plusieurs de nos frères en [notre] Seigneur étant rassurés par mes liens, osent annoncer la parole plus hardiment, et sans crainte.
\VS{15}Il est vrai que quelques-uns prêchent Christ par envie et par un esprit de dispute ; et que les autres le font, au contraire, par une bonne volonté.
\VS{16}Les uns, dis-je, annoncent Christ par un esprit de dispute, et non pas purement ; croyant ajouter de l'affliction à mes liens.
\VS{17}Mais les autres le font par charité, sachant que je suis établi pour la défense de l'Evangile.
\VS{18}Quoi donc ? toutefois en quelque manière que ce soit, par ostentation, ou par amour de la vérité, Christ est annoncé ; et c'est de quoi je me réjouis, et je me réjouirai.
\VS{19}Or je sais que ceci me tournera à salut par votre prière, et par le secours de l'Esprit de Jésus-Christ :
\VS{20}Selon ma ferme attente et mon espérance, que je ne serai confus en rien ; mais qu'en toute assurance, Christ sera maintenant, comme il l'a toujours été, glorifié en mon corps, soit par la vie, soit par la mort.
\VS{21}Car Christ m'est gain à vivre et à mourir.
\VS{22}Mais s'il m'est utile de vivre en la chair, et ce que je dois choisir, je n'en sais rien.
\VS{23}Car je suis pressé des deux [côtés] : mon désir tendant bien à déloger, et à être avec Christ, ce qui m'est beaucoup meilleur ;
\VS{24}Mais il est plus nécessaire pour vous que je demeure en la chair.
\VS{25}Et je sais cela comme tout assuré, que je demeurerai, et que je continuerai d'être avec vous tous pour votre avancement, et pour la joie de [votre] foi ;
\VS{26}Afin que vous ayez en moi un sujet de vous glorifier de plus en plus en Jésus-Christ, par mon retour au milieu de vous.
\VS{27}Seulement conduisez-vous dignement comme il est séant selon l'Evangile de Christ ; afin que soit que je vienne, et que je vous voie ; soit que je sois absent, j'entende quant à votre état, que vous persistez en un même esprit, combattant ensemble d'un même courage par la foi de l'Evangile, et n'étant en rien épouvantés par les adversaires.
\VS{28}Ce qui leur est une démonstration de perdition, mais à vous, de salut ; et cela de la part de Dieu.
\VS{29}Parce qu'il vous a été gratuitement donné dans ce qui a du rapport à Christ, non seulement de croire en lui, mais aussi de souffrir pour lui ;
\VS{30}Ayant [à soutenir] le même combat que vous avez vu en moi, et que vous apprenez être maintenant en moi.
\Chap{2}
\VerseOne{}Si donc il y a quelque consolation en Christ, s’il y a quelque soulagement dans la charité, s'il y a quelque communion d'esprit, s'il y a quelques cordiales affections et quelques compassions,
\VS{2}Rendez ma joie parfaite, étant d'un même sentiment, ayant un même amour, n'étant qu'une même âme, et consentant [tous] à une même chose.
\VS{3}Que rien ne se fasse par un esprit de dispute, ou par vaine gloire ; mais que par humilité de cœur l’un estime l'autre plus excellent que soi-même.
\VS{4}Ne regardez point chacun, à votre intérêt particulier, mais [que chacun ait égard] aussi à ce qui concerne les autres.
\VS{5}Qu'il y ait donc en vous un même sentiment qui a été en Jésus-Christ.
\VS{6}Lequel étant en forme de Dieu, n'a point regardé comme une usurpation d'être égal à Dieu.
\VS{7}Cependant il s'est anéanti lui-même, ayant pris la forme de serviteur, fait à la ressemblance des hommes ;
\VS{8}Et étant trouvé en figure comme un homme, il s'est abaissé lui-même, et a été obéissant jusques à la mort, à la mort même de la croix.
\VS{9}C'est pourquoi aussi Dieu l'a souverainement élevé, et lui a donné un Nom, qui est au-dessus de tout Nom ;
\VS{10}Afin qu'au Nom de Jésus tout genou se ploie, tant de ceux qui sont aux cieux, que de ceux qui sont en la terre, et au-dessous de la terre,
\VS{11}Et que toute Langue confesse que Jésus-Christ est le Seigneur, à la gloire de Dieu le Père.
\VS{12}C'est pourquoi, mes bien-aimés, ainsi que vous avez toujours obéi, non seulement comme en ma présence, mais beaucoup plus maintenant en mon absence, employez-vous à votre propre salut avec crainte et tremblement.
\VS{13}Car c'est Dieu qui produit en vous avec efficace le vouloir, et l'exécution, selon son bon plaisir.
\VS{14}Faites toutes choses sans murmures, et sans disputes ;
\VS{15}afin que vous soyez sans reproche, et purs, des enfants de Dieu, irrépréhensibles au milieu de la génération corrompue et perverse, parmi lesquels vous reluisez comme des flambeaux au monde, qui portent au devant d'eux la parole de la vie.
\VS{16}Pour me glorifier en la journée de Christ de n'avoir point couru en vain, ni travaillé en vain.
\VS{17}Que si même je sers d'aspersion sur le sacrifice et le service de votre foi, j'en suis joyeux ; et je m'en réjouis avec vous tous.
\VS{18}Vous aussi pareillement soyez-en joyeux, et réjouissez-vous-en avec moi.
\VS{19}Or j'espère [avec la grâce] du Seigneur Jésus de vous envoyer bientôt Timothée, afin que j'aie aussi plus de courage quand j'aurai connu votre état.
\VS{20}Car je n'ai personne d'un pareil courage, et qui soit vraiment soigneux de ce qui vous concerne.
\VS{21}Parce que tous cherchent leur intérêt particulier, et non les intérêts de Jésus-Christ.
\VS{22}Mais vous savez l'épreuve [que j'ai faite] de lui, puisqu'il a servi avec moi en l'Evangile, comme l'enfant sert son père.
\VS{23}J'espère donc de l'envoyer dès que j'aurai pourvu à mes affaires.
\VS{24}Et je m'assure en [notre] Seigneur que moi-même aussi je vous irai voir bientôt.
\VS{25}Mais j'ai cru nécessaire de vous envoyer Epaphrodite mon frère, mon compagnon d'œuvre et mon compagnon d'armes, qui aussi m'a été envoyé de votre part pour me fournir ce dont j'ai eu besoin.
\VS{26}Car aussi il désirait ardemment de vous voir tous, et il était fort affligé de ce que vous aviez appris qu'il avait été malade.
\VS{27}En effet il a été malade, et fort proche de la mort ; mais Dieu a eu pitié de lui, et non seulement de lui, mais aussi de moi, afin que je n'eusse pas tristesse sur tristesse.
\VS{28}Je l'ai donc envoyé à cause de cela avec plus de soin, afin qu'en le revoyant vous ayez de la joie, et que j'aie moins de tristesse.
\VS{29}Recevez-le donc en [notre] Seigneur, avec toute [sorte de] joie ; et ayez de l'estime pour ceux qui sont tels que lui.
\VS{30}Car il a été proche de la mort pour l'œuvre de Christ, n'ayant eu aucun égard à sa propre vie, afin de suppléer au défaut de votre service envers moi.
\Chap{3}
\VerseOne{}Au reste, mes frères, réjouissez-vous en [notre] Seigneur. Il ne m'est point fâcheux, et c'est votre sûreté, que je vous écrive les mêmes choses.
\VS{2}Prenez garde aux Chiens ; prenez garde aux mauvais Ouvriers ; prenez garde à la Circoncision.
\VS{3}Car c'est nous qui sommes la Circoncision, [nous] qui servons Dieu en esprit, et qui nous glorifions en Jésus-Christ, et qui n'avons point de confiance en la chair ;
\VS{4}Quoi que je ne pourrais bien aussi avoir confiance en la chair ; même si quelqu'un estime qu'il a de quoi se confier en la chair, j'en ai encore davantage ;
\VS{5}[Moi] qui ai été circoncis le huitième jour, qui suis de la race d'Israël, de la Tribu de Benjamin, Hébreu, né d'Hébreux, Pharisien de religion :
\VS{6}Quant au zèle, persécutant l'Eglise ; et quant à la justice qui est de la Loi, étant sans reproche,
\VS{7}Mais ce qui m'était un gain, je l'ai regardé comme m'étant nuisible et cela pour l'amour de Christ.
\VS{8}Et certes, je regarde toutes les autres choses comme m'étant nuisibles en comparaison de l'excellence de la connaissance de Jésus-Christ mon Seigneur, pour l'amour duquel je me suis privé de toutes ces choses, et je les estime comme du fumier, afin que je gagne Christ ;
\VS{9}Et que je sois trouvé en lui, ayant non point ma justice qui est de la Loi, mais celle qui est par la foi en Christ, [c'est à dire], la justice qui est de Dieu par la foi ;
\VS{10}[Pour] connaître Jésus-Christ, et la vertu de sa résurrection, et la communion de ses afflictions, étant rendu conforme à sa mort ;
\VS{11}[Essayant] si en quelque manière je puis parvenir à la résurrection des morts.
\VS{12}Non que j'aie déjà atteint [le but], ou que je sois déjà rendu accompli : mais je poursuis [ce but] pour tâcher d'y parvenir, c'est pourquoi aussi j'ai été pris par Jésus-Christ.
\VS{13}Mes frères, pour moi, je ne me persuade pas d'avoir atteint [le but] ;
\VS{14}Mais [je fais] une chose, [c'est qu'en] oubliant les choses qui sont derrière [moi], et m'avançant vers celles qui sont devant [moi], je cours vers le but, [savoir] au prix de la céleste vocation, [qui est] de Dieu en Jésus-Christ ;
\VS{15}C'est pourquoi, nous tous qui sommes parfaits ayons ce même sentiment ; et si en quelque chose vous avez un autre sentiment, Dieu vous le révélera aussi.
\VS{16}Cependant marchons suivant une même règle pour les choses auxquelles nous sommes parvenus, et ayons un même sentiment.
\VS{17}Soyez tous ensemble mes imitateurs, mes frères, et considérez ceux qui marchent comme vous nous avez pour modèle.
\VS{18}Car il y en a plusieurs qui marchent d'une [telle manière], que je vous ai souvent dit, et maintenant je vous le dis encore en pleurant, qu'ils sont ennemis de la croix de Christ ;
\VS{19}Desquels la fin est la perdition, desquels le Dieu est le ventre, et desquels la gloire est dans leur confusion, n'ayant d'affection que pour les choses de la terre.
\VS{20}Mais pour nous, notre bourgeoisie est dans les Cieux, d'où aussi nous attendons le Sauveur, le Seigneur Jésus-Christ ;
\VS{21}Qui transformera notre corps vil, afin qu'il soit rendu conforme à son corps glorieux, selon cette efficace par laquelle il peut même s'assujettir toutes choses.
\Chap{4}
\VerseOne{}C'est pourquoi, mes très chers frères que j'aime tendrement, vous qui êtes ma joie et ma couronne, demeurez ainsi fermes en notre Seigneur, mes bien-aimés.
\VS{2}Je prie Evodie, et je prie aussi Syntiche, d'avoir un même sentiment au Seigneur.
\VS{3}Je te prie aussi, toi mon vrai compagnon, aide-leur, comme à celles qui ont combattu avec moi dans l'Evangile, avec Clément, et mes autres compagnons d'œuvre, dont les noms [sont écrits] au Livre de vie.
\VS{4}Réjouissez-vous en [notre] Seigneur ; je vous le dis encore, réjouissez-vous.
\VS{5}Que votre douceur soit connue de tous les hommes. Le Seigneur est près.
\VS{6}Ne vous inquiétez de rien, mais en toutes choses présentez vos demandes à Dieu par des prières et des supplications, avec des actions de grâces.
\VS{7}Et la paix de Dieu, laquelle surpasse toute intelligence, gardera vos cœurs et vos sentiments en Jésus-Christ.
\VS{8}Au reste, mes frères, que toutes les choses qui sont véritables, toutes les choses qui sont vénérables, toutes les choses qui sont justes, toutes les choses qui sont pures, toutes les choses qui sont aimables, toutes les choses qui sont de bonne renommée, [toutes] celles où il y a quelque vertu et quelque louange, pensez à ces choses ;
\VS{9}[Car] aussi vous les avez apprises, reçues, entendues et vues en moi. Faites ces choses, et le Dieu de paix sera avec vous.
\VS{10}Or je me suis fort réjoui en [notre] Seigneur, de ce qu'à la fin vous avez fait revivre le soin que vous avez de moi ; à quoi aussi vous pensiez, mais vous n’en aviez pas l'occasion.
\VS{11}Je ne dis pas ceci ayant égard à quelque indigence : car j'ai appris à être content des choses selon que je me trouve.
\VS{12}Je sais être abaissé, je sais aussi être dans l'abondance ; partout et en toutes choses je suis instruit tant à être rassasié, qu'à avoir faim ; tant à être dans l'abondance, que dans la disette.
\VS{13}Je puis toutes choses en Christ qui me fortifie.
\VS{14}Néanmoins vous avez bien fait de prendre part à mon affliction.
\VS{15}Vous savez aussi, vous Philippiens, qu'au commencement [de la prédication] de l'Evangile, quand je partis de Macédoine, aucune Eglise ne me communiqua rien en matière de donner et de recevoir, excepté vous seuls.
\VS{16}Et même lorsque j'étais à Thessalonique, vous m'avez envoyé une fois, et même deux fois, ce dont j'avais besoin.
\VS{17}Ce n'est pas que je recherche des présents, mais je cherche le fruit qui abonde pour votre compte.
\VS{18}J'ai tout reçu, et je suis dans l'abondance, et j'ai été comblé de biens en recevant d'Epaphrodite ce qui m'a été envoyé de votre part, [comme] un parfum de bonne odeur, [comme] un sacrifice que Dieu accepte, et qui lui est agréable.
\VS{19}Aussi mon Dieu suppléera selon ses richesses à tout ce dont vous aurez besoin, et [vous donnera sa] gloire en Jésus-Christ.
\VS{20}Or à notre Dieu et [notre] Père, [soit] gloire aux siècles des siècles ; Amen !
\VS{21}Saluez chacun des Saints en Jésus-Christ. Les frères qui sont avec moi vous saluent.
\VS{22}Tous les Saints vous saluent, et principalement ceux qui sont de la maison de César.
\VS{23}La grâce de notre Seigneur Jésus-Christ [soit] avec vous tous, Amen.
\PPE{}
\end{multicols}
