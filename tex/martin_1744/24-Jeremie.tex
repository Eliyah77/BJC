\ShortTitle{Jeremie}\BookTitle{Jeremie}\BFont
\begin{multicols}{2}
\Chap{1}
\VerseOne{}Les paroles de Jérémie, fils de Hilkija, d'entre les Sacrificateurs qui étaient à Hanathoth, au pays de Benjamin ;
\VS{2}Auquel fut [adressée] la parole de l'Eternel, aux jours de Josias fils d'Amon Roi de Juda, la treizième année de son règne ;
\VS{3}Laquelle lui fut aussi [adressée] aux jours de Jéhojakim, fils de Josias, Roi de Juda, jusqu’à la fin de la onzième année de Sédécias, fils de Josias, Roi de Juda, [savoir] jusqu’au temps que Jérusalem fut transportée, ce qui arriva au cinquième mois.
\VS{4}La parole donc de l'Eternel me fut [adressée], en disant :
\VS{5}Avant que je te formasse dans le ventre [de ta mère], je t'ai connu ; et avant que tu fusses sorti de [son] sein, je t'ai sanctifié, je t'ai établi Prophète pour les nations.
\VS{6}Et je répondis : Ha ! ha ! Seigneur Eternel ! voici, je ne sais pas parler ; car je suis un enfant.
\VS{7}Et l'Eternel me dit : ne dis point : je suis un enfant ; car tu iras partout où je t'enverrai, et tu diras tout ce que je te commanderai.
\VS{8}Ne crains point [de te montrer] devant eux, car je suis avec toi pour te délivrer, dit l'Eternel.
\VS{9}Et l'Eternel avança sa main, et en toucha ma bouche, puis l'Eternel me dit : voici, j'ai mis mes paroles en ta bouche.
\VS{10}Regarde, je t'ai établi aujourd'hui sur les nations, et sur les Royaumes, afin que tu arraches et que tu démolisses, que tu ruines, et que tu détruises ; que tu bâtisses, et que tu plantes.
\VS{11}Puis la parole de l'Eternel me fut [adressée], en disant : que vois-tu, Jérémie ? Et je répondis : je vois une branche d'amandier.
\VS{12}Et l'Eternel me dit : tu as bien vu ; car je me hâte d'exécuter ma parole.
\VS{13}Alors la parole de l'Eternel me fut [adressée] pour la seconde fois, en disant : que vois-tu ? Et je répondis : je vois un pot bouillant, dont le devant est tourné vers l'Aquilon.
\VS{14}Et l'Eternel me dit : le mal se découvrira du côté de l'Aquilon sur tous les habitants de ce pays-ci.
\VS{15}Car voici, je m'en vais appeler toutes les familles des Royaumes de l'Aquilon, dit l'Eternel ; et elles viendront, et mettront chacune son trône à l'entrée des portes de Jérusalem, et près de toutes ses murailles à l'environ, et près de toutes les villes de Juda.
\VS{16}Et je leur prononcerai mes jugements, à cause de toute leur malice, par laquelle ils m'ont délaissé, et ont fait des parfums à d'autres dieux, et se sont prosternés devant l'ouvrage de leurs mains.
\VS{17}Toi donc, trousse tes reins, et te lève, et dis-leur toutes les choses que je te commanderai ; ne crains point [de te montrer] devant eux, de peur que je ne te fasse mettre en pièces en leur présence.
\VS{18}Car voici, je t'ai aujourd'hui établi comme une ville forte ; et comme une colonne de fer, et comme des murailles d'airain, contre tout ce pays-ci, c'est-à-dire, contre les Rois de Juda, contre les principaux du pays, contre ses Sacrificateurs, et contre le peuple du pays.
\VS{19}Et ils combattront contre toi, mais ils ne seront pas plus forts que toi ; car je suis avec toi, dit l'Eternel, pour te délivrer.
\Chap{2}
\VerseOne{}Et la parole de l'Eternel me fut [adressée], en disant :
\VS{2}Va, et crie, ceux de Jérusalem l'entendant, et dis : Ainsi a dit l'Eternel : il me souvient [pour l'amour] de toi de la compassion que j'ai eue pour toi en ta jeunesse, et de l'amour de tes épousailles, quand tu venais après moi dans le désert, en un pays qu'on ne sème point.
\VS{3}Israël était une chose sainte à l'Eternel, c'étaient les prémices de son revenu ; tous ceux qui le dévoraient étaient [trouvés] coupables, il leur en arrivait du mal, dit l'Eternel.
\VS{4}Ecoutez la parole de l'Eternel, maison de Jacob, et vous toutes les familles de la maison d'Israël :
\VS{5}Ainsi a dit l'Eternel : quelle injustice ont trouvée vos pères en moi, qu'ils se soient éloignés de moi, qu'ils aient marché après la vanité, et qu'ils soient devenus vains ?
\VS{6}Et ils n'ont point dit : où est l'Eternel qui nous a fait remonter du pays d'Egypte, qui nous a conduits par un désert, par un pays de landes et montagneux, par un pays aride et d'ombre de mort, par un pays où aucun homme n'avait passé, et où personne n'avait habité ?
\VS{7}Car je vous ai fait entrer dans un pays de Carmel, afin que vous mangeassiez ses fruits, et de ses biens ; mais sitôt que vous [y] êtes entrés, vous avez souillé mon pays, et avez rendu abominable mon héritage.
\VS{8}Les Sacrificateurs n'ont point dit : où est l'Eternel ? Et ceux qui expliquaient la Loi, ne m'ont point connu ; et les pasteurs ont prévariqué contre moi, et les prophètes ont prophétisé de par Bahal, et ont marché après des choses qui ne profitent de rien.
\VS{9}Pour cette cause encore je plaiderai avec vous, dit l'Eternel, et je plaiderai avec les enfants de vos enfants.
\VS{10}Car passez par les Îles de Kittim, et voyez ; envoyez en Kédar, et considérez bien, et regardez s'il y a eu rien de tel.
\VS{11}Y a-t-il aucune nation qui ait changé de dieux, lesquels toutefois ne [sont] pas dieux ? Mais mon peuple a changé sa gloire en ce qui ne profite de rien.
\VS{12}Cieux soyez étonnés de ceci ; ayez-[en] de l'horreur, et soyez extrêmement asséchés, dit l'Eternel.
\VS{13}Car mon peuple a fait deux maux ; ils m'ont abandonné, moi [qui suis] la source des eaux vives, pour se creuser des citernes, des citernes crevassées qui ne peuvent point contenir d'eau.
\VS{14}Israël est-il un esclave, ou un esclave né dans la maison ? pourquoi donc a-t-il été mis au pillage ?
\VS{15}Les lionceaux ont rugi, et ont jeté leur cri sur lui ; et on a mis leur pays en désolation, ses villes ont été brûlées, de sorte qu'il n'y a personne qui y habite.
\VS{16}Même les enfants de Noph, et de Taphnés te casseront le sommet de la tête.
\VS{17}Ne t'es-tu pas fait cela parce que tu as abandonné l'Eternel ton Dieu, dans le temps qu'il te menait par le chemin ?
\VS{18}Et maintenant, qu'as-tu affaire d'aller en Egypte pour y boire de l'eau de Sihor ? et qu'as-tu affaire d'aller en Assyrie, pour y boire de l'eau du fleuve ?
\VS{19}Ta malice te châtiera, et tes débauches te réprimanderont, afin que tu saches et que tu voies que c'est une chose mauvaise et amère, que tu aies abandonné l'Eternel ton Dieu, et que tu ne sois point rempli de ma frayeur, dit le Seigneur l'Eternel des armées.
\VS{20}Parce que depuis longtemps j'ai brisé ton joug, et rompu tes liens, tu as dit : je ne serai plus dans la servitude ; c'est pourquoi tu as erré en te prostituant sur toute haute colline, et sous tout arbre vert.
\VS{21}Or je t'avais moi-même plantée [comme] une vigne exquise, de laquelle tout le plant était franc ; comment donc t'es-tu changée en sarments d'une vigne abâtardie ?
\VS{22}Quand tu te laverais avec du nitre, et que tu prendrais beaucoup de savon, ton iniquité [demeurerait] encore marquée devant moi, dit le Seigneur l'Eternel.
\VS{23}Comment dis-tu : je ne me suis point souillée, je ne suis point allée après les Bahalins ? Regarde ton train dans la vallée, reconnais ce que tu as fait, dromadaire légère, qui ne tiens point de route certaine.
\VS{24}Anesse sauvage, accoutumée au désert, humant le vent à son plaisir ; et qui est-ce qui lui pourrait faire rebrousser sa course ? nul de ceux qui la cherchent ne se lassera après elle, on la trouvera en son mois.
\VS{25}Retiens ton pied, que tu ne marches déchaussée, et ton gosier, que tu ne sois altérée. Mais tu as dit : c'en est fait. Non ; car j'aime les étrangers, et j'irai après eux.
\VS{26}Comme le larron est confus, quand il est surpris, ainsi sont confus ceux de la maison d'Israël, eux, leurs Rois, les principaux d'entre eux, leurs Sacrificateurs, et leurs Prophètes ;
\VS{27}Qui disent au bois : tu es mon père ; et à la pierre : tu m'as engendré. Car ils m'ont tourné le dos, et non pas la face ; puis ils disent dans le temps de leur calamité : lève-toi, et nous délivre.
\VS{28}Et où sont tes dieux que tu t'es faits ? qu'ils se lèvent pour voir s'ils te délivreront au temps de la calamité ; car, ô Juda ! tu as eu autant de dieux que de villes.
\VS{29}Pourquoi plaideriez-vous contre moi ? vous avez tous péché contre moi, dit l'Eternel.
\VS{30}J'ai frappé en vain vos enfants, ils n'ont point reçu d'instruction ; votre épée a dévoré vos Prophètes, comme un lion qui ravage [tout].
\VS{31}Ô race ! considérez vous-mêmes la parole de l'Eternel, [qui dit] : Ai-je été un désert à Israël ? ai-je été une terre toute ténébreuse ? Pourquoi mon peuple a-t-il dit : nous sommes les maîtres ; nous ne viendrons plus à toi ?
\VS{32}La vierge oubliera-t-elle son ornement ? l'épouse ses atours ? mais mon peuple m'a oublié durant des jours sans nombre.
\VS{33}Pourquoi rends-tu ainsi affectée ta contenance pour chercher des amoureux, en sorte que tu as même enseigné tes manières de faire aux femmes de mauvaise vie ?
\VS{34}Même dans les pans de ta robe a été trouvé le sang des âmes des pauvres innocents, que tu n'avais point surpris en fracture, mais il y a été trouvé pour toutes ces choses-là.
\VS{35}Et tu dis : je suis innocente ; quoi qu'il en soit, sa colère s'est détournée de moi. Voici, je m'en vais contester contre toi, sur ce que tu as dit : je n'ai point péché.
\VS{36}Pourquoi te donnes-tu tant de mouvement, changeant de chemin ? tu seras aussi confuse d'Egypte, que tu as été confuse d'Assyrie.
\VS{37}Tu sortiras même d'ici, ayant tes mains sur ta tête, parce que l'Eternel a rejeté les fondements de ta confiance, et tu n'auras aucune prospérité par eux.
\Chap{3}
\VerseOne{}On dit : si quelqu'un délaisse sa femme, et qu'elle se séparant de lui se joigne à un autre mari, [le premier mari] retournera-t-il encore vers elle ? Le pays même n'en serait-il pas entièrement souillé ? Or toi, tu t'es prostituée à plusieurs amoureux, toutefois retourne-toi vers moi, dit l'Eternel.
\VS{2}Lève tes yeux vers les lieux élevés, et regarde quel est le lieu où tu ne te sois point abandonnée ; tu te tenais par les chemins, comme un Arabe au désert ; et tu as souillé le pays par tes débauches, et par ta malice.
\VS{3}C'est pourquoi les pluies ont été retenues, et il n'y a point eu de pluie de la dernière saison, et tu as un front de femme débauchée ; tu n'as point voulu avoir de honte.
\VS{4}Ne crieras-tu point désormais vers moi : mon Père, tu es le conducteur de ma jeunesse ?
\VS{5}Tiendra-t-il [sa colère] à toujours, et me la gardera-t-il à jamais ? Voici, tu as [ainsi] parlé, et tu as fait ces maux-là, autant que tu as pu.
\VS{6}Aussi l'Eternel me dit aux jours du Roi Josias : n'as-tu point vu ce qu'Israël la revêche a fait ? elle s'en est allée sur toute haute montagne, et sous tout arbre vert, et elle s'y est prostituée.
\VS{7}Et quand elle a eu fait toutes ces choses, j'ai dit : retourne-toi vers moi ; mais elle n'est point retournée ; ce que sa sœur Juda la perfide a vu ;
\VS{8}Et j'ai vu que pour toutes les occasions par lesquelles Israël la revêche avait commis adultère, je l'ai renvoyée, et lui ai donné ses lettres de divorce ; toutefois Juda sa sœur l'infidèle n'a point eu de crainte, mais s'en est allée, et elle aussi s'est prostituée.
\VS{9}Et il est arrivé que par la facilité qu'elle a à s'abandonner elle a souillé le pays, et a commis adultère avec la pierre et le bois.
\VS{10}Et néanmoins pour tout ceci, Juda sa sœur la perfide n'est point retournée à moi de tout son cœur, mais avec mensonge, dit l'Eternel.
\VS{11}L'Eternel donc m'a dit : Israël la revêche s'est montrée plus juste que Juda la perfide.
\VS{12}Va donc, et crie ces paroles-ci vers l'Aquilon ; et dis : retourne-toi, Israël la revêche, dit l'Eternel ; je ne ferai point tomber ma colère sur vous ; car je suis miséricordieux, dit l'Eternel, et je ne vous la garderai point à toujours.
\VS{13}Mais reconnais ton iniquité ; car tu as péché contre l'Eternel ton Dieu, et tu t'es prostituée aux étrangers sous tout arbre vert, et n'as point écouté ma voix, dit l'Eternel.
\VS{14}Enfants revêches, convertissez-vous, dit l'Eternel ; car j'ai droit de mari sur vous ; et je vous prendrai l'un d'une ville, et deux d'une lignée, et je vous ferai entrer en Sion.
\VS{15}Et je vous donnerai des pasteurs selon mon cœur, qui vous paîtront de science et d'intelligence.
\VS{16}Et il arrivera que quand vous serez multipliés et accrus sur la terre, en ces jours-là, dit l'Eternel, on ne dira plus : L'Arche de l'alliance de l'Eternel ; elle ne leur montera plus au cœur, ils n'en feront point mention, ils ne la visiteront plus, et cela ne se fera plus.
\VS{17}En ce temps-là on appellera Jérusalem, le Trône de l'Eternel, et toutes les nations s'assembleront vers elle, au Nom de l'Eternel, à Jérusalem, et elles ne marcheront plus après la dureté de leur cœur mauvais.
\VS{18}En ces jours-là la maison de Juda marchera avec la maison d'Israël, et ils viendront ensemble du pays d'Aquilon au pays que j'ai donné en héritage à vos pères.
\VS{19}Car j'ai dit : comment te mettrai-je entre [mes] fils, et te donnerai-je la terre désirable, l'héritage de la noblesse des armées des nations ? et j'ai dit : tu me crieras, mon Père, et tu ne te détourneras point de moi.
\VS{20}Certainement comme une femme pèche contre son ami, ainsi avez-vous péché contre moi, maison d'Israël, dit l'Eternel.
\VS{21}Une voix a été ouïe sur les lieux élevés, des pleurs de supplications des enfants d'Israël, parce qu'ils ont perverti leur voie, et qu'ils ont mis en oubli l'Eternel, leur Dieu.
\VS{22}Enfants rebelles, convertissez-vous, je remédierai à vos rébellions. Voici, nous venons vers toi ; car tu [es] l'Eternel notre Dieu.
\VS{23}Certainement [on s'attend] en vain aux collines, [et] à la multitude des montagnes ; mais c'est en l'Eternel notre Dieu qu'est la délivrance d'Israël.
\VS{24}Car la honte a consumé dès notre jeunesse le travail de nos pères, leurs brebis et leurs bœufs, leurs fils et leurs filles.
\VS{25}Nous serons gisants dans notre honte, et notre ignominie nous couvrira, parce que nous avons péché contre l'Eternel notre Dieu, nous et nos pères, dès notre jeunesse, et jusqu’à aujourd'hui ; et nous n'avons point obéi à la voix de l'Eternel notre Dieu.
\Chap{4}
\VerseOne{}Israël, si tu te retournes, dit l'Eternel, retourne-toi à moi ; si tu ôtes tes abominations de devant moi, tu ne seras plus errant ça et là.
\VS{2}Alors tu jureras en vérité, et en jugement, et en justice, l'Eternel est vivant ; et les nations se béniront en lui, et se glorifieront en lui.
\VS{3}Car ainsi a dit l'Eternel à ceux de Juda et de Jérusalem : défrichez-vous les terres, et ne semez point sur les épines.
\VS{4}Hommes de Juda, et vous habitants de Jérusalem, soyez circoncis à l'Eternel, et ôtez les prépuces de vos cœurs ; de peur que ma fureur ne sorte comme un feu, et qu'elle ne s'embrase, sans qu'il y ait personne qui l'éteigne, à cause de la méchanceté de vos actions.
\VS{5}Faites savoir en Juda, et publiez dans Jérusalem, et dites : sonnez du cor par le pays, criez, [et] vous amassez ; et dites : assemblez-vous, et nous entrerons dans les villes fortes.
\VS{6}Dressez l'enseigne vers Sion, retirez-vous en troupe, et ne vous arrêtez point ; car je m'en vais faire venir de l'Aquilon le mal et une grande calamité.
\VS{7}Le lion est sorti de la caverne, et le destructeur des nations est parti ; il est sorti de son lieu pour réduire ton pays en désolation, tes villes seront ruinées, tellement qu'il n'y aura personne qui y habite.
\VS{8}C'est pourquoi ceignez-vous de sacs, lamentez, et hurlez ; car l'ardeur de la colère de l'Eternel n'est point détournée de nous.
\VS{9}Et il arrivera en ce jour-là, dit l'Eternel, que le cœur du Roi, et le cœur des principaux sera épouvanté, et que les Sacrificateurs seront étonnés, et que les Prophètes seront tout confus.
\VS{10}C'est pourquoi j'ai dit : ha ! ha ! Seigneur Eternel ! oui certainement tu as abusé ce peuple et Jérusalem, en disant : vous aurez la paix ; et l'épée est venue jusqu’à l'âme.
\VS{11}En ce temps-là on dira à ce peuple, et à Jérusalem : un vent éclaircissant les lieux élevés [souffle] au désert, dans le chemin de la fille de mon peuple, non pas pour vanner ni pour nettoyer :
\VS{12}Un vent plus véhément que ceux-là viendra à moi, et je leur ferai maintenant leur procès.
\VS{13}Voici, il montera comme des nuées, et ses chariots seront semblables à un tourbillon, ses chevaux seront plus légers que des aigles ; malheur à nous ! car nous sommes détruits.
\VS{14}Jérusalem, nettoie ton cœur de ta malice, afin que tu sois délivrée ; jusques à quand séjourneront au dedans de toi les pensées de ton injustice ?
\VS{15}Car le cri apporte des nouvelles de Dan, et publie du mont d'Ephraïm le tourment.
\VS{16}Faites l'entendre aux nations, voici, publiez contre Jérusalem, [et dites] : les assiégeants viennent d'un pays éloigné, et ils ont jeté leur cri contre les villes de Juda.
\VS{17}Ils se sont mis tout autour d'elle comme les gardes des champs, parce qu'elle m'a été rebelle, dit l'Eternel.
\VS{18}Ta conduite et tes actions t'ont produit ces choses ; telle a été ta malice ; parce que ç'a été une chose amère ; certainement elle te touchera jusques au cœur.
\VS{19}Mon ventre ! mon ventre ! je suis dans la douleur ; le dedans de mon cœur, mon cœur me bat, je ne me puis taire ; car ô mon âme ! tu as ouï le son du cor, et le retentissement bruyant de l'alarme.
\VS{20}Une ruine est appelée par l'autre, car toute la terre est détruite ; mes tentes ont été incontinent détruites, [et] mes pavillons en un moment.
\VS{21}Jusques à quand verrai-je l'enseigne, et entendrai-je le son du cor ?
\VS{22}Car mon peuple est insensé, ils ne m'ont point reconnu ; ce sont des enfants insensés, et qui n'ont point d'entendement ; ils sont habiles à faire le mal, mais ils ne savent pas faire le bien.
\VS{23}J'ai regardé la terre, et voici, elle est sans forme et vide ; et les cieux, et il n'y a point de clarté.
\VS{24}J'ai regardé les montagnes, et voici, elles branlent ; et toutes les collines sont renversées.
\VS{25}J'ai regardé, et voici, il n'y a pas un seul homme, et tous les oiseaux des cieux s'en sont fuis.
\VS{26}J'ai regardé, et voici, Carmel est un désert, et toutes ses villes ont été ruinées par l'Eternel, et par l'ardeur de sa colère.
\VS{27}Car ainsi a dit l'Eternel : toute la terre ne sera que désolation ; néanmoins je ne l'achèverai pas entièrement.
\VS{28}C'est pourquoi la terre mènera deuil, et les cieux seront obscurcis au-dessus, parce que je l'ai prononcé ; je l'ai pensé, je ne m'en repentirai point, et je ne le révoquerai point.
\VS{29}Toute ville s'enfuit à cause du bruit des gens de cheval, et de ceux qui tirent de l'arc ; ils sont entrés dans les bois épais, et sont montés sur les rochers ; toute ville est abandonnée et personne n'y habite.
\VS{30}Et quand tu auras été détruite, que feras-tu ? quoique tu te vêtes de cramoisi, et que tu te pares d'ornements d'or, et que tu couvres ton visage de fard, tu t'embellis en vain ; tes amoureux t'ont rebutée, ils chercheront ta vie.
\VS{31}Car j'ai ouï un cri comme celui d'une [femme] qui est en travail, et une angoisse comme celle d'une femme qui est en travail de son premier-né ; c'est le cri de la fille de Sion ; elle soupire, elle étend ses mains, [en disant] : Malheur maintenant à moi, car mon âme est défaillie à cause des meurtriers.
\Chap{5}
\VerseOne{}Promenez-vous par les rues de Jérusalem, et regardez maintenant, et sachez, et vous enquérez par ses places, si vous y trouverez un homme de bien, s'il y a quelqu'un qui fasse ce qui est droit, et qui cherche la fidélité ; et je pardonnerai à la [ville].
\VS{2}Que s'ils disent : l'Eternel est vivant ; ils jurent en cela plus faussement.
\VS{3}Eternel, tes yeux [ne regardent-ils pas] à la fidélité ? Tu les as frappés, et ils n'en ont point senti de douleur ; tu les as consumés, [et] ils ont refusé de recevoir instruction ; ils ont endurci leurs faces plus qu'une roche, ils ont refusé de se convertir.
\VS{4}Et j'ai dit : certainement ce ne sont que les plus abjects, qui se sont montrés fous, parce qu'ils ne connaissent point la voie de l'Eternel, le droit de leur Dieu.
\VS{5}Je m'en irai donc aux plus grands, et je leur parlerai ; car ceux-là connaissent la voie de l'Eternel, le droit de leur Dieu ; mais ceux-là mêmes ont aussi brisé le joug, [et] ont rompu les liens.
\VS{6}C'est pourquoi le lion de la forêt les a tués, le loup du soir les a ravagés, [et] le léopard est au guet contre leurs villes ; quiconque en sortira sera déchiré, car leurs péchés sont multipliés, [et] leurs rébellions sont renforcées.
\VS{7}Comment te pardonnerai-je en cela ? tes fils m'ont abandonné, et ils jurent par ceux qui ne sont point dieux ; je les ai rassasiés, et ils ont commis adultère et sont allés en foule dans la maison de la prostituée.
\VS{8}Ils sont comme des chevaux bien repus, quand ils se lèvent le matin, chacun hennit après la femme de son prochain.
\VS{9}Ne punirais-je point ces choses-là, dit l'Eternel ? et mon âme ne se vengerait-elle pas d'une nation qui est telle ?
\VS{10}Montez sur ses murailles, et les rompez ; mais ne les achevez pas entièrement ; ôtez ses créneaux ; car ils ne sont point à l'Eternel.
\VS{11}Parce que la maison d'Israël et la maison de Juda se sont portées fort infidèlement contre moi, dit l'Eternel.
\VS{12}Ils ont démenti l'Eternel, et ont dit : cela n'arrivera pas, et le mal ne viendra pas sur nous, nous ne verrons pas l'épée ni la famine.
\VS{13}Et les Prophètes sont légers comme le vent, et la parole n'est point en eux ; ainsi leur sera-t-il fait.
\VS{14}C'est pourquoi, ainsi a dit l'Eternel, le Dieu des armées : parce que vous avez proféré cette parole-là, voici, je m'en vais mettre mes paroles en ta bouche pour y être comme un feu, et ce peuple sera comme le bois, et ce feu les consumera.
\VS{15}Maison d'Israël, voici, je m'en vais faire venir contre vous une nation d'un pays éloigné, dit l'Eternel, une nation robuste, une nation ancienne, une nation de laquelle tu ne sauras point la Langue, et dont tu n'entendras point ce qu'elle dira.
\VS{16}Son carquois est comme un sépulcre ouvert, [et] ils sont tous vaillants.
\VS{17}Et elle mangera ta moisson et ton pain, que tes fils et tes filles doivent manger ; elle mangera tes brebis, et tes bœufs ; elle mangera [les fruits] de tes vignes, et de tes figuiers, et réduira à la pauvreté par l'épée tes villes fortes, sur lesquelles tu te confiais.
\VS{18}Toutefois en ces jours-là, dit l'Eternel, je ne vous achèverai pas entièrement.
\VS{19}Et il arrivera que vous direz : pourquoi l'Eternel notre Dieu nous a-t-il fait toutes ces choses ? et tu leur diras ainsi : comme vous m'avez abandonné, [et] avez servi les dieux de l'étranger dans votre pays, ainsi serez-vous asservis aux étrangers en un pays qui n'est point à vous.
\VS{20}Faites savoir ceci dans la maison de Jacob, et le publiez dans Juda, en disant :
\VS{21}Ecoutez maintenant ceci, peuple fou, et qui n'avez point d'intelligence, qui avez des yeux, et ne voyez point ; et qui avez des oreilles, et n'entendez point.
\VS{22}Ne me craindrez-vous point, dit l'Eternel, et ne serez-vous point épouvantés devant ma face ? moi qui ai mis le sable pour la borne de la mer, par une ordonnance perpétuelle, et qui ne passera point ; ses vagues s'émeuvent, mais elles ne seront pas les plus fortes ; et elles bruient, mais elles ne la passeront point.
\VS{23}Mais ce peuple-ci a un cœur rétif et rebelle ; ils se sont reculés en arrière, et s'en sont allés.
\VS{24}Et ils n'ont point dit en leur cœur : craignons maintenant l'Eternel notre Dieu, qui nous donne la pluie de la première et de la dernière saison, [et qui] nous garde les semaines ordonnées pour la moisson.
\VS{25}Vos iniquités ont détourné ces choses, et vos péchés ont empêché qu'il ne vous arrivât du bien.
\VS{26}Car il s'est trouvé dans mon peuple des méchants qui sont aux aguets, comme celui qui tend des pièges, ils posent une machine de perdition pour y prendre les hommes.
\VS{27}Comme la cage est remplie d'oiseaux, ainsi leurs maisons [sont] remplies de fraude, et par ce moyen ils se sont agrandis et enrichis.
\VS{28}Ils sont engraissés et parés ; même ils ont surpassé les actions des méchants ; ils ne font justice à personne, non pas même à l'orphelin, et ils prospèrent, et ne font point droit aux pauvres.
\VS{29}Ne punirais-je point ces choses-là, dit l'Eternel ? et mon âme ne se vengerait-elle pas d'une nation qui est telle ?
\VS{30}Il est arrivé en la terre une chose étonnante, et qui fait horreur :
\VS{31}C'est que les Prophètes prophétisent le mensonge, et les Sacrificateurs dominent par leur moyen ; et mon peuple a aimé cela. Que ferez-vous donc quand elle prendra fin ?
\Chap{6}
\VerseOne{}Enfants de Benjamin, fuyez vous-en [par troupes] du milieu de Jérusalem, et sonnez du cor à Tékoah, et élevez un signal de feu à Bethkérem ; car le mal et une grande ruine a paru de l'Aquilon.
\VS{2}J'avais rendu la fille de Sion semblable à une femme qui ne bouge point de la maison, et qui est délicate.
\VS{3}Les pasteurs avec leurs troupeaux viendront contre elles, ils planteront leurs tentes autour d'elle, chacun paîtra en son quartier.
\VS{4}Préparez le combat contre elle, levez-vous, et montons en plein midi. Malheur à nous, car le jour décline, et les ombres du soir s'accroissent.
\VS{5}Levez-vous, montons de nuit, et ruinons ses palais.
\VS{6}Car ainsi a dit l'Eternel des armées : coupez des arbres, et dressez des terrasses contre Jérusalem ; c'est la ville qui doit être visitée, elle n'est tout entière que rapine au dedans.
\VS{7}Comme le puits fait bouillonner ses eaux, ainsi elle fait bouillonner sa malice ; on n'entend en elle devant moi continuellement que violence et que dégât, avec des maladies et des plaies.
\VS{8}Jérusalem, reçois instruction, de peur que mon affection ne se retire de toi, et que je ne fasse de toi un désert, [et] une terre inhabitable.
\VS{9}Ainsi a dit l'Eternel des armées : on grappillera entièrement comme une vigne les restes d'Israël ; remets ta main dans les paniers comme un vendangeur.
\VS{10}A qui parlerai-je, [et] qui sommerai-je, afin qu'ils écoutent ? voici, leur oreille est incirconcise, et ils ne peuvent entendre ; voici, la parole de l'Eternel leur est en opprobre, ils n'y prennent point de plaisir.
\VS{11}C'est pourquoi je suis plein de la fureur de l'Eternel, et je suis las de la retenir, de sorte que je la répandrai sur les enfants par la rue, et sur l'assemblée des jeunes gens, même le mari sera pris avec la femme, et l'homme âgé avec celui qui est rassasié de jours.
\VS{12}Et leurs maisons retourneront aux étrangers, les champs, et les femmes aussi ; car j'étendrai ma main sur les habitants du pays, dit l'Eternel.
\VS{13}Parce que depuis le plus petit d'entre eux jusques au plus grand, chacun s'adonne au gain déshonnête, tant le Prophète que le Sacrificateur, tous se portent faussement.
\VS{14}Et ils ont pansé la froissure de la fille de mon peuple à la légère, disant : paix, paix ; et il n'y avait point de paix.
\VS{15}Ont-ils été confus de ce qu'ils ont commis des abominations ? Ils n'en ont même eu aucune honte, et ils ne savent ce que c'est que de rougir ; c'est pourquoi ils tomberont sur ceux qui seront tombés, ils tomberont au temps que je les visiterai, a dit l'Eternel.
\VS{16}Ainsi a dit l'Eternel : tenez-vous sur les chemins, et regardez, et vous enquérez touchant les sentiers des siècles passés, quel est le bon chemin, et marchez-y, et vous trouverez le repos de vos âmes ; et ils ont répondu : nous n'[y] marcherons point.
\VS{17}J'avais aussi établi sur vous des sentinelles [qui disent] : soyez attentifs au son de la trompette ; mais on a répondu : nous n'[y] serons point attentifs.
\VS{18}Vous donc, nations, écoutez, et toi Assemblée, connais ce qui est entre eux.
\VS{19}Ecoute, terre : voici, je m'en vais faire venir un mal sur ce peuple, [savoir] le fruit de leurs pensées, parce qu'ils n'ont point été attentifs à mes paroles, et qu'ils ont rejeté ma Loi.
\VS{20}Pourquoi m'offrir de l'encens venu de Seba, et le bon roseau aromatique du pays éloigné ? vos holocaustes ne [me] plaisent point, et vos sacrifices ne me sont point agréables.
\VS{21}C'est pourquoi ainsi a dit l'Eternel : voici, je m'en vais mettre en ce peuple des achoppements, auxquels les pères et les enfants, le voisin et son compagnon heurteront ensemble, et ils périront.
\VS{22}Ainsi a dit l'Eternel : voici, un peuple vient du pays de l'Aquilon, et une grande nation se réveillera du fond de la terre.
\VS{23}Ils prendront l'arc et l'étendard ; ils [seront] cruels, et n'auront point de compassion ; leur voix bruira comme la mer, et ils seront montés sur des chevaux, chacun d'eux est rangé en homme de guerre contre toi, fille de Sion !
\VS{24}En aurons-nous ouï le bruit ? nos mains [en] deviendront lâches ; l'angoisse nous saisira, [et] un travail comme de celle qui enfante.
\VS{25}Ne sortez point aux champs, et n'allez point par le chemin, car l'épée de l'ennemi [et] la frayeur est tout à l'entour.
\VS{26}Fille de mon peuple, ceins-toi d'un sac, et te vautre dans la cendre ; mène deuil [comme] sur un fils unique, et [fais] une lamentation très-amère, car le destructeur viendra subitement sur nous.
\VS{27}Je t'ai établi [comme] une place forte et une forteresse au milieu de mon peuple, afin que tu connaisses et sondes leur voie.
\VS{28}Ils sont tous revêches, et plus que revêches, et ils vont médisant ; ils sont [comme] de l'airain, et du fer ; ils sont tous des gens qui se perdent l'un l'autre.
\VS{29}Le soufflet est brûlé, le plomb est consumé par le feu, le fondeur a fondu en vain, car les mauvais n'ont point été séparés.
\VS{30}On les appellera : Argent réprouvé ; car l'Eternel les a réprouvés.
\Chap{7}
\VerseOne{}La parole fut [adressée] à Jérémie par l'Eternel, en disant :
\VS{2}Tiens-toi debout à la porte de la maison de l'Eternel, et y crie cette parole, et dis : vous tous hommes de Juda qui entrez par ces portes, pour vous prosterner devant l'Eternel, écoutez la parole de l'Eternel ;
\VS{3}Ainsi a dit l'Eternel des armées, le Dieu d'Israël : corrigez votre vie et vos actions, et je vous ferai habiter en ce lieu-ci.
\VS{4}Ne vous fiez point sur des paroles trompeuses, en disant : c'est ici le Temple de l'Eternel, le Temple de l'Eternel, le Temple de l'Eternel.
\VS{5}Mais corrigez sérieusement votre conduite et vos actions, et appliquez-vous à faire droit à ceux qui plaident l'un contre l'autre.
\VS{6}Et ne faites point de tort à l'étranger, ni à l'orphelin, ni à la veuve ; et ne répandez point en ce lieu-ci le sang innocent, et ne marchez point après les dieux étrangers, à votre ruine.
\VS{7}Et je vous ferai habiter depuis un siècle jusqu’à l'autre siècle en ce lieu-ci, au pays que j'ai donné à vos pères.
\VS{8}Voici, vous vous fiez sur des paroles trompeuses, sans aucun profit.
\VS{9}Ne dérobez-vous pas ? ne tuez-vous pas ? ne commettez-vous pas adultère ? ne jurez-vous pas faussement ? ne faites-vous pas des encensements à Bahal ? n'allez-vous pas après les dieux étrangers, que vous ne connaissez point ?
\VS{10}Toutefois vous venez, et vous vous présentez devant moi dans cette maison, sur laquelle mon Nom est réclamé, et vous dites : nous avons été délivrés pour faire toutes ces abominations.
\VS{11}Cette maison, sur laquelle mon Nom est réclamé devant vos yeux, n'est-elle pas devenue une caverne de brigands ? et voici, moi-même je l'ai vu, dit l'Eternel.
\VS{12}Mais allez maintenant à mon lieu, qui était à Silo, là où j'avais placé mon Nom dès le commencement, et regardez ce que je lui ai fait, à cause de la malice de mon peuple d'Israël.
\VS{13}Maintenant donc, parce que vous faites toutes ces choses, dit l'Eternel, et que je vous ai parlé, me levant dès le matin, et parlant, et que vous n'avez point écouté ; je vous ai appelés, et vous n'avez point répondu :
\VS{14}Je ferai à cette maison, sur laquelle mon Nom est réclamé, et sur laquelle vous vous fiez, et à ce lieu que je vous ai donné à vous et à vos pères, comme j'ai fait à Silo.
\VS{15}Et je vous chasserai de devant ma face, comme j'ai chassé tous vos frères, avec toute la postérité d'Ephraïm.
\VS{16}Toi donc ne prie point pour ce peuple, et ne jette point de cri, et ne fais point de requête pour eux, et n'intercède point envers moi ; car je ne t'exaucerai point.
\VS{17}Ne vois-tu pas ce qu'ils font dans les villes de Juda, et dans les rues de Jérusalem ?
\VS{18}Les fils amassent le bois, et les pères allument le feu, et les femmes pétrissent la pâte pour faire des gâteaux à la Reine des cieux ; et pour faire des aspersions aux dieux étrangers, afin de m'irriter.
\VS{19}Ce qu'ils m'irritent est-il contre moi ? dit l'Eternel. N'est-ce pas contre eux-mêmes, à la confusion de leurs faces ?
\VS{20}C'est pourquoi ainsi a dit le Seigneur l'Eternel : voici, ma colère et ma fureur vont fondre sur ce lieu-ci, sur les hommes et sur les bêtes, sur les arbres des champs, et sur le fruit de la terre, ma colère s'embrasera, et ne s'éteindra point.
\VS{21}Ainsi a dit l'Eternel des armées, le Dieu d'Israël : ajoutez vos holocaustes à vos sacrifices, et mangez-en la chair.
\VS{22}Car je n'ai point parlé avec vos pères, et je ne leur ai point donné de commandement touchant les holocaustes et les sacrifices, au jour que je les fis sortir du pays d'Egypte.
\VS{23}Mais voici ce que je leur ai commandé, disant : écoutez ma voix, et je serai votre Dieu, et vous serez mon peuple ; et marchez dans toutes les voies que je vous ai ordonnées, afin que vous soyez heureux.
\VS{24}Mais ils n'ont point écouté, et n'ont point incliné leur oreille ; mais ils ont suivi [d'autres] conseils, et la dureté de leur cœur mauvais, ils se sont éloignés et ne sont point retournés jusques à moi.
\VS{25}Depuis le jour que vos pères sont sortis du pays d'Egypte, jusqu’à aujourd'hui, je vous ai envoyé tous mes serviteurs Prophètes, me levant [chaque] jour dès le matin, et les envoyant.
\VS{26}Mais ils ne m'ont point écouté, et ils n'ont point incliné leur oreille, mais ils ont raidi leur cou ; ils ont fait pis que leurs pères.
\VS{27}Tu leur diras donc toutes ces paroles, mais ils ne t'écouteront point ; et tu crieras après eux, mais ils ne te répondront point.
\VS{28}C'est pourquoi tu leur diras : [c'est] ici la nation qui n'a point écouté la voix de l'Eternel son Dieu, et qui n'a point reçu d'instruction ; la fidélité est périe, et elle a été retranchée de leur bouche.
\VS{29}Tonds ta chevelure, [ô Jérusalem !] et [la] jette au loin, et prononce à haute voix ta complainte sur les lieux élevés ; car l'Eternel a rejeté et abandonné la génération contre laquelle il est fort courroucé.
\VS{30}Parce que les enfants de Juda ont fait ce qui me déplaît, dit l'Eternel, ils ont mis leurs abominations dans cette maison, sur laquelle mon Nom est invoqué, afin de la souiller.
\VS{31}Et ils ont bâti les hauts lieux de Topheth, qui [est] dans la vallée du fils de Hinnom, pour brûler leurs fils et leurs filles au feu, ce que je n'ai pas commandé, et à quoi je n'ai jamais pensé.
\VS{32}C'est pourquoi voici, les jours viennent, dit l'Eternel, qu'elle ne sera plus appelée Topheth, ni la vallée du fils de Hinnom, mais la vallée de la tuerie, et on ensevelira [les morts] à Topheth, à cause qu'il n'y aura plus [d'autre] lieu.
\VS{33}Et les corps morts de ce peuple seront pour viande aux oiseaux des cieux, et aux bêtes de la terre, sans qu'il y ait personne qui les effarouche.
\VS{34}Je ferai aussi cesser des villes de Juda, et des rues de Jérusalem, la voix de joie, et la voix d'allégresse, la voix de l'époux, et la voix de l'épouse, car le pays sera mis en désolation.
\Chap{8}
\VerseOne{}En ce temps-là, dit l'Eternel, on jettera les os des Rois de Juda, et les os de ses Princes, les os des Sacrificateurs, et les os des Prophètes, et les os des habitants de Jérusalem, hors de leurs sépulcres.
\VS{2}Et on les étendra devant le soleil, et devant la lune, et devant toute l'armée des cieux, qui sont des choses qu'ils ont aimées, qu'ils ont servies et après lesquelles ils ont marché ; des choses qu'ils ont recherchées, et devant lesquelles ils se sont prosternés ; ils ne seront point recueillis ni ensevelis, ils seront comme du fumier sur le dessus de la terre.
\VS{3}Et la mort sera plus désirable que la vie à tous ceux qui seront restés de cette méchante race, ceux, [dis-je], qui seront restés dans tous les lieux où je les aurai chassés, dit l'Eternel des armées.
\VS{4}Tu leur diras donc : ainsi a dit l'Eternel : si on tombe, ne se relèvera-t-on pas ? et si on se détourne, ne retournera-t-on pas [au chemin] ?
\VS{5}Pourquoi [donc] s'est égaré ce peuple, [les habitants de] Jérusalem, d'un égarement continuel ? ils se sont adonnés opiniâtrement à la tromperie, et ont refusé de se convertir.
\VS{6}Je me suis rendu attentif, et j'ai écouté, mais nul ne parle selon la justice, il n'y a personne qui se repente de son péché, disant : qu'ai-je fait ? Ils sont tous retournés vers les objets qui les entraînent, comme le cheval qui se jette avec impétuosité parmi la bataille.
\VS{7}Même la cigogne a connu dans les cieux ses saisons, la tourterelle et l'hirondelle, et la grue, ont pris garde au temps qu'elles doivent venir ; mais mon peuple n'a point connu le droit de l'Eternel.
\VS{8}Comment dites-vous : nous sommes les sages, et la Loi de l'Eternel est avec nous ? voilà, certes on a agi faussement, et la plume des Scribes [est une plume] de fausseté.
\VS{9}Les sages ont été confus, ils ont été épouvantés et pris, car ils ont rejeté la parole de l'Eternel ? et en quoi seraient-ils sages ?
\VS{10}C'est pourquoi je donnerai leurs femmes à d'autres, et leurs champs à des gens qui les posséderont en héritage ; car depuis le plus petit jusqu’au plus grand, chacun s'adonne au gain déshonnête, tant le Prophète que le Sacrificateur, tous agissent faussement.
\VS{11}Et ils ont pansé la plaie de la fille de mon peuple à la légère, en disant : paix, paix ; et il n'y avait point de paix.
\VS{12}Ont-ils été confus de ce qu'ils ont commis abomination ? Ils n'en ont eu même aucune honte, et ils ne savent ce que c'est que de rougir ; c'est pourquoi ils tomberont sur ceux qui seront tombés ; ils tomberont au temps que je les visiterai, a dit l'Eternel.
\VS{13}En les ramassant je les consumerai entièrement dit l'Eternel ; il n'y a pas une grappe dans les vignes ; et il n'y a pas une figue au figuier, la feuille est flétrie, et ce que je leur ai donné sera transporté avec eux.
\VS{14}Pour quelle raison nous arrêtons-nous ? Assemblez-vous, et entrons dans les villes fortes, et nous serons là en repos, car l'Eternel notre Dieu nous a fait taire, et nous a donné à boire de l'eau de fiel, parce que nous avons péché contre l'Eternel.
\VS{15}On attend la paix, et il n'y a rien de bon ; [on attend] le temps de la guérison, et voici le trouble.
\VS{16}Le ronflement de ses chevaux a été ouï de Dan, et tout le pays a été ému du bruit des hennissements de ses puissants chevaux ; ils sont venus et ont dévoré le pays et tout ce qui y était, la ville et ceux qui habitaient en elle.
\VS{17}Qui plus est, voici, je m'en vais envoyer contre vous des serpents, des basilics, contre lesquels il n'y a point d'enchantement, et ils vous mordront, dit l'Eternel.
\VS{18}J'ai voulu prendre des forces pour soutenir la douleur, mais mon cœur est languissant au dedans de moi.
\VS{19}Voici la voix du cri de la fille de mon peuple [qui crie] d'un pays éloigné ; l'Eternel n'est-il point en Sion ? son Roi n'est-il point au milieu d'elle ? Mais pourquoi m'ont-ils irrité par leurs images taillées, par les vanités de l'étranger ?
\VS{20}La moisson est passée ; l'Eté est fini, et nous n'avons point été délivrés.
\VS{21}Je suis amèrement affligé à cause de la calamité de la fille de mon peuple, j'en suis en deuil, j'en suis tout désolé.
\VS{22}N'y a-t-il point de baume en Galaad ? n'y a-t-il point là de médecin ? pourquoi donc la plaie de la fille de mon peuple n'est-elle pas consolidée ?
\Chap{9}
\VerseOne{}Plût à Dieu que ma tête fût comme un réservoir d'eau, et que mes yeux fussent une vive fontaine de larmes, et je pleurerais jour et nuit les blessés à mort de la fille de mon peuple !
\VS{2}Plût à Dieu que j'eusse au désert une cabane de voyageurs, j'abandonnerais mon peuple, et me retirerais d'avec eux ; car ils sont tous des adultères, et une troupe de perfides.
\VS{3}Ils ont tendu leur langue, [qui leur a été comme] leur arc pour décocher le mensonge, et ils se sont renforcés dans la terre contre la fidélité, parce qu'ils sont allés de malice en malice, et ne m'ont point reconnu, dit l'Eternel.
\VS{4}Gardez-vous chacun de son intime ami, et ne vous fiez à aucun frère ; car tout frère fait métier de supplanter, et tout intime ami va médisant.
\VS{5}Et chacun se moque de son intime ami, et on ne parle point en vérité ; ils ont instruit leur langue à dire le mensonge, ils se tourmentent extrêmement pour mal faire.
\VS{6}Ta demeure est au milieu de la tromperie ; ils refusent à cause de la tromperie de me reconnaître, dit l'Eternel.
\VS{7}C'est pourquoi ainsi a dit l'Eternel des armées : voici, je m'en vais les fondre, et je les éprouverai ; car comment en agirais-je autrement à l'égard de la fille de mon peuple ?
\VS{8}Leur langue est un trait décoché, elle profère des fraudes ; chacun a la paix dans sa bouche avec son intime ami, mais dans son intérieur il lui dresse des embûches.
\VS{9}Ne punirais-je point en eux ces choses-là ? dit l'Eternel. Mon âme ne se vengerait-elle pas d'une nation qui est telle ?
\VS{10}J'élèverai ma voix avec larmes, et je prononcerai à haute voix une lamentation à cause des montagnes, et une complainte à cause des cabanes du désert, parce qu'elles ont été brûlées, de sorte qu'il n'y a personne qui y passe, et qu'on n'y entend plus le cri des troupeaux ; les oiseaux des cieux et le bétail s'en sont fuis, ils s'en sont allés.
\VS{11}Et je réduirai Jérusalem en monceaux de ruines, elle sera une retraite de dragons, et je détruirai les villes de Juda, tellement qu'il n'y aura personne qui y habite.
\VS{12}Qui est l'homme sage qui entende ceci, et qui est celui à qui la bouche de l'Eternel ait parlé, qui en fasse le rapport, [et qui dise] pourquoi le pays est-il désolé, et brûlé comme un désert, sans que personne y passe ?
\VS{13}L'Eternel donc a dit : parce qu'ils ont abandonné ma Loi, laquelle je leur avais proposée, et qu'ils n'ont point écouté ma voix, et n'ont point marché selon elle ;
\VS{14}Mais parce qu'ils ont marché après la dureté de leur cœur, et après les Bahalins ; ce que leurs pères leur ont enseigné ;
\VS{15}C'est pourquoi, ainsi a dit l'Eternel des armées, le Dieu d'Israël : voici, je m'en vais donner à manger à ce peuple-ci de l'absinthe, et je leur donnerai à boire de l'eau de fiel.
\VS{16}Je les disperserai parmi les nations que ni eux ni leurs pères n'ont point connues ; et j'enverrai après eux l'épée, jusqu’à ce que je les aie consumés.
\VS{17}Ainsi a dit l'Eternel des armées : recherchez, et appelez des pleureuses, afin qu'elles viennent, et mandez les femmes sages, et qu'elles viennent ;
\VS{18}Qu'elles se hâtent, et qu'elles prononcent à haute voix une lamentation sur nous, et que nos yeux se fondent en pleurs et que nos paupières fassent ruisseler des larmes.
\VS{19}Car une voix de lamentation a été ouïe de Sion, [disant] : comment avons-nous été détruits ? Nous sommes fort confus, parce que nous avons abandonné le pays, parce que nos tentes nous ont jetés dehors.
\VS{20}C'est pourquoi, vous femmes, écoutez la parole de l'Eternel, et que votre oreille reçoive la parole de sa bouche ; et enseignez vos filles à lamenter, et chacune sa compagne à faire des complaintes.
\VS{21}Car la mort est montée par nos fenêtres, elle est entrée dans nos palais, pour exterminer les enfants, [tellement qu'il n'y en a plus] dans les rues ; et les jeunes gens, [tellement qu'il n'y en a plus] par les places.
\VS{22}Dis : ainsi a dit l'Eternel : même les corps morts des hommes seront étendus comme du fumier sur le dessus des champs, et comme une poignée d'épis après les moissonneurs, lesquels personne ne recueille.
\VS{23}Ainsi a dit l'Eternel : que le sage ne se glorifie point en sa sagesse ; que le fort ne se glorifie point en sa force, et que le riche ne se glorifie point en ses richesses ;
\VS{24}Mais que celui qui se glorifie, se glorifie en ce qu'il a de l'intelligence, et qu'il me connaît ; car je suis l'Eternel, qui fais miséricorde, et jugement, et justice sur la terre ; parce que je prends plaisir en ces choses-là, dit l'Eternel.
\VS{25}Voici, les jours viennent, dit l'Eternel, que je punirai tout circoncis ayant [encore] le prépuce.
\VS{26}L'Egypte, et Juda, et Edom, et les enfants de Hammon, et Moab, et tous ceux qui sont aux bouts des coins, habitant dans le désert ; car toutes les nations ont le prépuce, et toute la maison d'Israël a le prépuce du cœur.
\Chap{10}
\VerseOne{}Maison d'Israël, écoutez la parole que l'Eternel a prononcée sur vous.
\VS{2}Ainsi a dit l'Eternel : n'apprenez point les façons de faire des nations, et ne soyez point épouvantés des signes des cieux, sous ombre que les nations en sont épouvantées.
\VS{3}Car les statuts des peuples [ne] sont [que] vanité, parce qu'on coupe du bois de la forêt pour le mettre en œuvre avec la hache ;
\VS{4}Puis on l'embellit avec de l'argent et de l'or, et on le fait tenir avec des clous et à coups de marteau, afin qu'il ne remue point.
\VS{5}Ils sont façonnés tout droits comme un palmier, et ils ne parlent point ; on les porte par nécessité, à cause qu'ils ne peuvent pas marcher ; ne les craignez point, car ils ne font point de mal, et aussi il n'est pas en leur pouvoir de faire du bien.
\VS{6}Il n'y a point de [dieu] semblable à toi, ô Eternel ! tu es grand, et ton Nom est grand en force.
\VS{7}Qui ne te craindrait, Roi des nations, car cela t'est dû ? parce qu'entre tous les plus sages des nations, et dans tous leurs Royaumes il n'y en a point de semblable à toi.
\VS{8}Et ils sont tous ensemble abrutis, et devenus fous ; le bois [ne leur] apprend que des vanités.
\VS{9}L'argent qui est étendu en plaques, est apporté de Tarsis, et l'or est apporté d'Uphaz, pour être mis en œuvre par l'ouvrier, et par les mains du fondeur ; et la pourpre et l'écarlate est leur vêtement ; toutes ces choses sont l'ouvrage de gens adroits.
\VS{10}Mais l'Eternel est le Dieu de vérité, [c'est] le Dieu vivant, et le Roi éternel ; la terre sera ébranlée par sa colère, et les nations ne pourront soutenir son indignation.
\VS{11}Vous leur direz ainsi : les dieux qui n'ont point fait les cieux et la terre périront de la terre, et de dessous les cieux.
\VS{12}[Mais l'Eternel] est celui qui a fait la terre par sa vertu, qui a formé le monde habitable par sa sagesse, et qui a étendu les cieux par son intelligence ;
\VS{13}Sitôt qu'il a fait éclater sa voix [il y a] un grand bruit d'eaux dans les cieux ; après qu'il a fait monter du bout de la terre les vapeurs, il fait briller l'éclair avant la pluie, et il tire le vent hors de ses trésors.
\VS{14}Tout homme se montre abruti dans sa science ; tout fondeur est rendu honteux par les images taillées ; car ce qu'ils font est une fausseté, et il n'y a point de respiration en elles.
\VS{15}Elles ne sont que vanité, et un ouvrage propre à abuser ; elles périront au temps de leur visitation.
\VS{16}La portion de Jacob n'est point comme ces choses-là ; car c'est celui qui a tout formé, et Israël est le lot de son héritage ; son nom est l'Eternel des armées.
\VS{17}Toi qui habites en un lieu fort, retire du pays ta marchandise.
\VS{18}Car ainsi a dit l'Eternel : voici, je m'en vais à cette fois jeter [au loin, comme] avec une fronde, les habitants du pays, et je les mettrai à l'étroit, tellement qu'ils le trouveront.
\VS{19}Malheur à moi, [diront-ils], à cause de ma plaie, ma plaie est douloureuse. Mais moi j'ai dit : quoi qu'il en soit, c'est une maladie qu'il faut que je souffre.
\VS{20}Ma tente est gâtée ; tous mes cordages sont rompus ; mes enfants sont sortis d'auprès de moi, et ne sont plus ; il n'y a plus personne qui dresse ma tente, et qui élève mes pavillons.
\VS{21}Car les pasteurs sont abrutis, et n'ont point recherché l'Eternel ; c'est pourquoi ils ne se sont point conduits sagement, et tous leurs pâturages ont été dissipés.
\VS{22}Voici, un bruit de certaines nouvelles est venu, avec une grande émotion de devers le pays d'Aquilon, pour ravager les villes de Juda, et en faire une retraite de dragons.
\VS{23}Eternel, je connais que la voie de l'homme ne dépend pas de lui, et qu'il n'est pas au pouvoir de l'homme qui marche, de diriger ses pas.
\VS{24}Ô Eternel ! châtie-moi, mais [que ce soit] par mesure, [et] non en ta colère, de peur que tu ne me réduises à rien.
\VS{25}Répands ta fureur sur les nations qui ne te connaissent point, et sur les familles qui n'invoquent point ton Nom ; car ils ont dévoré Jacob, ils l'ont, dis-je, dévoré et consumé, et ils ont mis en désolation son agréable demeure.
\Chap{11}
\VerseOne{}La parole fut [adressée] à Jérémie par l'Éternel, en disant :
\VS{2}Ecoutez les paroles de cette alliance, et prononcez[-les] aux hommes de Juda, et aux habitants de Jérusalem.
\VS{3}Tu leur diras donc : ainsi a dit l'Eternel le Dieu d'Israël : maudit soit l'homme qui n'écoutera point les paroles de cette alliance ;
\VS{4}Laquelle j'ai ordonnée à vos pères, le jour que je les ai retirés du pays d'Egypte, du fourneau de fer, en disant : Ecoutez ma voix, et faites toutes les choses que je vous ai commandées, et vous serez mon peuple, et je serai votre Dieu.
\VS{5}Afin que je ratifie le serment que j'ai fait à vos pères, de leur donner un pays découlant de lait et de miel, comme [il paraît] aujourd'hui. Et je répondis, et dis : Ainsi soit-il ! ô Eternel !
\VS{6}Puis l'Eternel me dit : crie toutes ces paroles par les villes de Juda, et par les rues de Jérusalem, en disant : écoutez les paroles de cette alliance, et les faites ;
\VS{7}Car j'ai sommé expressément vos pères au jour que je les ai fait monter du pays d'Egypte jusqu’à aujourd'hui, me levant dès le matin, et les sommant, en disant : écoutez ma voix.
\VS{8}Mais ils ne l'ont pas écoutée, et n'y ont point incliné leur oreille, mais ils ont marché chacun suivant la dureté de leur cœur mauvais ; c'est pourquoi j'ai fait venir sur eux toutes les paroles de cette alliance, laquelle je leur avais commandé de garder, et qu'ils n'ont point gardée.
\VS{9}Et l'Eternel me dit : il y a une conjuration parmi les hommes de Juda, et parmi les habitants de Jérusalem.
\VS{10}Ils sont retournés aux iniquités de leurs ancêtres, qui ont refusé d'écouter mes paroles, et qui sont allés après d'autres dieux pour les servir ; la maison d'Israël et la maison de Juda ont enfreint mon alliance, que j'avais traitée avec leurs pères.
\VS{11}C'est pourquoi ainsi a dit l'Eternel : voici, je m'en vais faire venir sur eux un mal duquel ils ne pourront sortir ; ils crieront vers moi, mais je ne les exaucerai point.
\VS{12}Et les villes de Juda et les habitants de Jérusalem s'en iront, et crieront vers les dieux auxquels ils font leurs parfums ; mais ces dieux-là ne les délivreront nullement au temps de leur affliction.
\VS{13}Car, ô Juda ! tu as eu autant de dieux que de villes ; et toi, Jérusalem, tu as dressé autant d'autels aux choses honteuses, que tu as de rues, des autels, [dis-je], pour faire des parfums à Bahal.
\VS{14}Toi donc ne fais point de requête pour ce peuple, et ne jette point de cri, ni ne fais point de prière pour eux ; car je ne les exaucerai point au temps qu'ils crieront vers moi dans leur calamité.
\VS{15}Qu'est-ce que mon bien-aimé a à faire dans ma maison, que tant de gens se servent d'elle pour y faire leurs complots ? la chair sainte est transportée loin de toi, et encore quand tu fais mal, tu t'égayes.
\VS{16}L'Eternel avait appelé ton nom, Olivier vert [et] beau, à cause du beau fruit ; [mais] au son d'un grand bruit il y a allumé le feu, et ses branches ont été rompues.
\VS{17}Car l'Eternel des armées, qui t'a plantée, a prononcé du mal contre toi, à cause de la malice de la maison d'Israël, et de la maison de Juda, qu'ils ont commise contre eux-mêmes, jusqu’à m'irriter en faisant des parfums à Bahal.
\VS{18}Et l'Eternel me l'a donné à connaître, et je l'ai connu ; et tu m'as fait voir leurs actions.
\VS{19}Mais moi, comme un agneau, [ou comme] un bœuf qui est mené pour être égorgé, je n'ai point su qu'ils eussent fait contre moi quelque machination, [en disant] : détruisons l'arbre avec son fruit, et l'exterminons de la terre des vivants, et qu'on ne se souvienne plus de son nom.
\VS{20}Mais toi, Eternel des armées, qui juges justement, et qui sondes les reins et le cœur, fais que je voie la vengeance que tu feras d'eux ; car je t'ai découvert ma cause.
\VS{21}C'est pourquoi ainsi a dit l'Eternel, touchant les gens de Hanathoth, qui cherchent ta vie, et qui disent : ne prophétise plus au Nom de l'Eternel, et tu ne mourras point par nos mains.
\VS{22}C'est pourquoi [donc] ainsi a dit l'Eternel des armées : voici, je vais les punir ; les jeunes gens mourront par l'épée, leurs fils et leurs filles mourront par la famine ;
\VS{23}Et il ne restera rien d'eux, car je ferai venir le mal sur les gens de Hanathoth, en l'année de leur visitation.
\Chap{12}
\VerseOne{}Eternel, quand je contesterai avec toi, tu [seras trouvé] juste ; mais toutefois j'entrerai en contestation avec toi. Pourquoi la voie des méchants a-t-elle prospéré ; et pourquoi tous les perfides vivent-ils en paix ?
\VS{2}Tu les as plantés, et ils ont pris racine ; ils s'avancent, et fructifient. Tu es près de leur bouche, mais [tu es] loin de leurs reins.
\VS{3}Mais, ô Eternel ! tu m'as connu, tu m'as vu, et tu as sondé quel est mon cœur envers toi. Traîne-les comme des brebis qu'on mène pour être égorgées, et prépare-les pour le jour de la tuerie.
\VS{4}Jusques à quand la terre mènera-t-elle deuil, et l'herbe de tous les champs séchera-t-elle à cause de la malice des habitants qui sont en la terre ? Les bêtes et les oiseaux ont été consumés [par la disette], parce que [ces méchants] ont dit : on ne verra point notre dernière fin.
\VS{5}Si tu as couru avec les gens de pied, et qu'ils t'aient lassé, comment te mêleras-tu parmi les chevaux ? et si tu t'es cru en sûreté dans une terre de paix, que feras-tu lorsque le Jourdain sera enflé ?
\VS{6}Certainement tes frères mêmes, et la maison de ton père, ceux-là mêmes ont agi perfidement contre toi, eux-mêmes ont crié après toi à plein gosier ; ne les crois point, quoiqu'ils te parlent aimablement.
\VS{7}J'ai abandonné ma maison, j'ai quitté mon héritage, ce que mon âme aimait le plus je l'ai livré en la main de ses ennemis.
\VS{8}Mon héritage m'a été comme un lion dans la forêt ; il a jeté son cri contre moi, c'est pourquoi je l'ai en haine.
\VS{9}Mon héritage me sera-t-il donc [comme] l'oiseau peint ? les oiseaux ne sont-ils pas autour de lui ? Venez, assemblez-vous, vous tous les animaux des champs, venez pour le dévorer.
\VS{10}Plusieurs pasteurs ont gâté ma vigne, ils ont foulé mon partage, ils ont réduit mon partage désirable en un désert affreux.
\VS{11}On l'a ravagé, et lui, tout désolé, a été en deuil devant moi ; toute la terre a été ravagée, parce qu'il n'y a personne qui y fasse attention.
\VS{12}Les destructeurs sont venus sur toutes les hautes places qui sont au désert, car l'épée de l'Eternel dévore depuis un bout du pays jusqu’à l'autre ; il n'y a point de paix pour aucune chair.
\VS{13}Ils ont semé du froment, et ils moissonneront des épines ; ils se sont peinés, [et] ils n'y profiteront rien ; vous serez confus en vos revenus par l'ardeur de la colère de l'Eternel.
\VS{14}Ainsi a dit l'Eternel contre tous mes mauvais voisins, qui mettent la main sur l'héritage que j'ai fait hériter à mon peuple d'Israël : voici, je vais les arracher de leur pays, et j'arracherai la maison de Juda du milieu d'eux.
\VS{15}Mais il arrivera qu'après les avoir arrachés, j'aurai encore compassion d'eux, et je les ferai retourner chacun à son héritage, et chacun en son quartier.
\VS{16}Et il arrivera que s'ils apprennent bien les voies de mon peuple, pour jurer en mon Nom, l'Eternel est vivant, ainsi qu'ils ont enseigné à mon peuple à jurer par Bahal, ils seront édifiés parmi mon peuple.
\VS{17}Mais s'ils n'écoutent point, j'arracherai entièrement une telle nation, et [la] ferai périr, dit l'Eternel.
\Chap{13}
\VerseOne{}Ainsi m'a dit l'Eternel : Va, et achète-toi une ceinture de lin, et mets-la sur tes reins, et ne la mets point dans l'eau.
\VS{2}J'achetai donc une ceinture selon la parole de l'Eternel, et la mis sur mes reins.
\VS{3}Et la parole de l'Eternel me fut [adressée] pour la seconde fois, en disant :
\VS{4}Prends la ceinture que tu as achetée, et qui est sur tes reins, et te lève, et t'en va vers l'Euphrate, et la cache là dans le trou d'un rocher.
\VS{5}Je m'en allai donc, et la cachai dans l'Euphrate, comme l'Eternel m'avait commandé.
\VS{6}Et il arriva que plusieurs jours après l'Eternel me dit : lève-toi, et t'en va vers l'Euphrate, et reprends de là la ceinture que je t'avais commandé d'y cacher.
\VS{7}Et je m'en allai vers l'Euphrate, je creusai ; et je repris la ceinture du lieu où je l'avais cachée, et voici, la ceinture était pourrie, et n'était plus bonne à rien.
\VS{8}Alors la parole de l'Eternel me fut [adressée], en disant :
\VS{9}Ainsi a dit l'Eternel : je ferai ainsi pourrir l'orgueil de Juda, et le grand orgueil de Jérusalem.
\VS{10}[L'orgueil] de ce peuple très méchant, qui refusent d'écouter mes paroles, et qui marchent selon la dureté de leur cœur, et vont après d'autres dieux pour les servir, et pour se prosterner devant eux, tellement qu'il sera comme cette ceinture, qui n'est bonne à aucune chose.
\VS{11}Car comme une ceinture est jointe sur les reins d'un homme, je m'étais attaché toute la maison d'Israël, et toute la maison de Juda, dit l'Eternel, afin qu'ils fussent mon peuple, ma renommée, ma louange, et ma gloire ; mais ils n'ont point écouté.
\VS{12}Tu leur diras donc cette parole-ci : ainsi a dit l'Eternel, le Dieu d'Israël : tout vaisseau sera rempli de vin ; et ils te diront : ne savons-nous pas bien que tout vaisseau sera rempli de vin ?
\VS{13}Mais tu leur diras : ainsi a dit l'Eternel : voici, je m'en vais remplir d'ivresse tous les habitants de ce pays, et les Rois qui sont assis sur le trône de David pour l'amour de lui, et les Sacrificateurs, et les Prophètes, et tous les habitants de Jérusalem.
\VS{14}Et je les briserai l'un contre l'autre, les pères et les enfants ensemble, dit l'Eternel, je n'en aurai point de compassion, je ne les épargnerai point, et je n'en aurai point de pitié pour ne les détruire point.
\VS{15}Ecoutez et prêtez l'oreille, ne vous élevez point, car l'Eternel a parlé.
\VS{16}Donnez gloire à l'Eternel votre Dieu, avant qu'il fasse venir les ténèbres, et avant que vos pieds bronchent sur les montagnes dans lesquelles on ne voit pas clair ; vous attendrez la lumière, et il la changera en une ombre de mort, et la réduira en obscurité.
\VS{17}Que si vous n'écoutez ceci, mon âme pleurera en secret à cause de [votre] orgueil, et mon œil versera des larmes en abondance, même il se fondra en larmes, parce que le troupeau de l'Eternel aura été emmené prisonnier.
\VS{18}Dis au Roi et à la Régente : humiliez-vous, et vous asseyez [sur la cendre], car votre couronne magnifique tombera de dessus vos têtes.
\VS{19}Les villes du Midi sont fermées, et il n'y a personne qui les ouvre ; tout Juda est transporté en captivité, il est universellement transporté.
\VS{20}Levez vos yeux, et voyez ceux qui viennent de l'Aquilon. Où est le parc qui t'a été donné, et ton magnifique troupeau ?
\VS{21}Que diras-tu quand il te punira ? car tu les as enseignés contre toi, pour être supérieurs sur ta tête ; les douleurs ne te saisiront-elles point, comme elles saisissent la femme qui enfante ?
\VS{22}Que si tu dis en ton cœur : pourquoi me sont arrivées ces choses ? C'est pour la grandeur de ton iniquité que tes habits ont été retroussés, [et] que tes talons ont été serrés de près.
\VS{23}Le More changerait-il sa peau, et le léopard ses taches ? pourriez-vous aussi faire quelque bien, vous qui n'êtes appris qu'à mal faire ?
\VS{24}C'est pourquoi je les disperserai comme du chaume, qui est emporté ça et là par le vent du désert.
\VS{25}C'est ici ton lot, et la portion que je t'ai mesurée dit l'Eternel ; parce que tu m'as oublié, et que tu as mis ta confiance au mensonge,
\VS{26}A cause de cela j'ai retroussé tes habits sur ton visage, et ton ignominie paraîtra.
\VS{27}Tes adultères, et tes hennissements, l'énormité de ta prostitution est sur les collines, par les champs, j'ai vu tes abominations ; malheur à toi, Jérusalem, ne seras-tu point nettoyée ? jusques à quand cela durera-t-il ?
\Chap{14}
\VerseOne{}La parole de l'Eternel, qui fut [adressée] à Jérémie, sur ce que [les pluies avaient été] retenues.
\VS{2}La Judée a mené deuil, et ses portes sont en un pitoyable état. Ils sont tous en deuil [gisant] par terre, et le cri de Jérusalem est monté [au ciel].
\VS{3}Et les personnes distinguées ont envoyé à l'eau les moindres d'entre eux ; ils sont venus aux lieux creux, ils n'y ont point trouvé d'eau, et ils s'en sont retournés leurs vaisseaux vides ; ils ont été rendus honteux et confus, et ils ont couvert leur tête.
\VS{4}Parce que la terre s'est crevassée à cause qu'il n'y a point eu de pluie au pays ; les laboureurs ont été rendus honteux, [et] ils ont couvert leur tête.
\VS{5}Même la biche a fait son faon au champ et l'a abandonné, parce qu'il n'y a point d'herbe.
\VS{6}Et les ânes sauvages se sont tenus sur les lieux élevés, ils ont attiré l'air comme des dragons ; leurs yeux sont consumés, parce qu'il n'y a point d'herbe.
\VS{7}Eternel, si nos iniquités rendent témoignage contre nous, agis à cause de ton Nom, car nos rébellions sont multipliées ; c'est contre toi que nous avons péché.
\VS{8}Toi qui es l'attente d'Israël, [et] son libérateur au temps de la détresse, pourquoi serais-tu en la terre comme un étranger, et comme un voyageur qui se détourne pour passer la nuit ?
\VS{9}Pourquoi serais-tu comme un homme étonné, et comme un homme fort qui ne peut délivrer ? Or tu es au milieu de nous, ô Eternel ! et ton Nom est réclamé sur nous ; ne nous abandonne point.
\VS{10}L'Eternel a dit ainsi à ce peuple, parce qu'ils ont aimé à aller ainsi çà et là, et qu'ils n'ont point retenu leurs pieds, l'Eternel n'a point pris plaisir en eux, il se souviendra maintenant de leurs iniquités, et il punira leurs péchés.
\VS{11}Puis l'Eternel me dit : ne fais point de requête en faveur de ce peuple.
\VS{12}Quand ils jeûneront, je n'exaucerai point leur cri, et quand ils offriront des holocaustes et des oblations, je n'y prendrai point de plaisir ; mais je les consumerai par l'épée, et par la famine, et par la mortalité.
\VS{13}Et je dis : ah ! ah ! Seigneur Eternel ! voici, les Prophètes leur disent : vous ne verrez point l'épée, et vous n'aurez point de famine, mais je vous donnerai une paix assurée en ce lieu-ci.
\VS{14}Et l'Eternel me dit : ce n'est que mensonge ce que ces prophètes prophétisent en mon Nom ; je ne les ai point envoyés, je ne leur ai point donné de charge, et je ne leur ai point parlé ; ils vous prophétisent des visions de mensonge, des divinations de néant, et des tromperies de leur cœur.
\VS{15}C'est pourquoi, ainsi a dit l'Eternel touchant les prophètes qui prophétisent en mon Nom, et que je n'ai point envoyés, et qui disent : l'épée ni la famine ne sera point en ce pays ; ces prophètes-là seront consumés par l'épée et par la famine.
\VS{16}Et le peuple auquel ils ont prophétisé sera jeté par les rues de Jérusalem à cause de la famine et de l'épée ; et il n'y aura personne qui les ensevelisse, tant eux que leurs femmes, leurs fils et leurs filles, et je répandrai sur eux leur méchanceté.
\VS{17}Tu leur diras donc cette parole-ci : que mes yeux se fondent en larmes nuit et jour, et qu'ils ne cessent point ; car la vierge, fille de mon peuple, a été fort maltraitée, la plaie [en] est fort douloureuse.
\VS{18}Si je sors aux champs, voici les gens morts par l'épée ; et si j'entre en la ville, voici les langueurs de la faim ; même le Prophète et le Sacrificateur ont couru par le pays, et ils ne savent où ils en sont.
\VS{19}Aurais-tu entièrement rejeté Juda ? et ton âme aurait-elle Sion en dédain ? Pourquoi nous as-tu frappés tellement qu'il n'y a point de guérison ? on attend la paix, et il n'y a rien de bon ; et le temps de la guérison, et voici le trouble.
\VS{20}Eternel, nous reconnaissons notre méchanceté, [et] l'iniquité de nos pères ; car nous avons péché contre toi.
\VS{21}Ne nous rejette point à cause de ton Nom, et n'expose point à opprobre le trône de ta gloire ; souviens-toi de ton alliance avec nous, [et] ne la romps point.
\VS{22}Parmi les vanités des nations y en a-t-il qui fassent pleuvoir, et les cieux donnent-ils la menue pluie ? N'est-ce pas toi qui le fais, ô Eternel notre Dieu ? C'est pourquoi nous nous attendrons à toi ; car c'est toi qui as fait toutes ces choses.
\Chap{15}
\VerseOne{}Et l'Eternel me dit : quand Moïse et Samuel se tiendraient devant moi, je n'aurais pourtant point d'affection pour ce peuple ; chasse-les de devant ma face, et qu'ils sortent.
\VS{2}Que s'ils te disent : où sortirons-nous ? Tu leur répondras : ainsi a dit l'Eternel : ceux qui [sont destinés] à la mort [iront] à la mort ; et ceux qui [sont destinés] à l’épée [iront] à l’épée ; et ceux qui [sont destinés] à la famine, [iront] à la famine ; et ceux qui [sont destinés] à la captivité [iront] en captivité.
\VS{3}J'établirai aussi sur eux quatre espèces de [punitions], dit l'Eternel, l'épée pour tuer, et les chiens pour traîner, et les oiseaux des cieux, et les bêtes de la terre pour dévorer et pour détruire.
\VS{4}Et je les livrerai à être agités par tous les Royaumes de la terre, à cause de Manassé fils d'Ezéchias, Roi de Juda, pour les choses qu'il a faites dans Jérusalem.
\VS{5}Car qui serait ému de compassion envers toi, Jérusalem ? ou qui viendrait se condouloir avec toi ? ou qui se détournerait pour s'enquérir de ta prospérité ?
\VS{6}Tu m'as abandonné, dit l'Eternel, et tu t'en es allée en arrière ; c'est pourquoi j'étendrai ma main sur toi, et je te détruirai ; je suis las de me repentir.
\VS{7}Je les vannerai avec un van aux portes du pays ; j'ai désolé [et] fait périr mon peuple, et ils ne se sont point détournés de leur voie.
\VS{8}J'ai multiplié ses veuves plus que le sable de la mer, je leur ai amené sur la mère de leur jeunesse un destructeur en plein midi, j'ai fait tomber subitement sur elle l'ennemi et les frayeurs.
\VS{9}Celle qui en avait enfanté sept est devenue languissante, elle a rendu l'esprit, son soleil lui est couché pendant qu'il était encore jour, elle a été rendue honteuse et confuse ; et je livrerai son reste à l'épée devant leurs ennemis, dit l'Eternel.
\VS{10}Malheur à moi, ô ma mère ! de ce que tu m'as enfanté pour être un homme de débat, et un homme de querelle à tout le pays ; je ne me suis obligé à personne, et personne ne s'est obligé à moi, et néanmoins chacun me maudit et me méprise.
\VS{11}[Alors l'Eternel dit :] ceux qui seront restés de toi ne viendront-ils pas à bien ? et ne ferai-je pas que l'ennemi viendra au devant de toi au temps de la calamité, et au temps de la détresse ?
\VS{12}Le fer usera-t-il le fer de l'Aquilon, et l'acier ?
\VS{13}Je livrerai au pillage, sans en faire le prix, tes richesses et tes trésors ; et cela à cause de tous tes péchés, et même par toutes tes contrées.
\VS{14}Et je ferai passer tes ennemis par un pays que tu ne connais point, car le feu a été allumé en ma colère, il sera allumé sur vous.
\VS{15}Eternel, tu le connais, souviens-toi de moi, et me visite, et me venge de ceux qui me persécutent ; ne m'enlève point, quand tu auras longtemps différé ta colère ; connais que j'ai souffert opprobre pour l'amour de toi.
\VS{16}Tes paroles se sont-elles trouvées ? je les ai [aussitôt] mangées ; et ta parole m'a été en joie, et elle a été l'allégresse de mon cœur ; car ton Nom est réclamé sur moi, ô Eternel ! Dieu des armées.
\VS{17}Je ne me suis point assis au conventicule des railleurs, et je ne m'y suis point réjoui ; mais je me suis tenu assis tout seul, à cause de ta main, parce que tu m'as rempli d'indignation.
\VS{18}Pourquoi ma douleur est-elle [rendue] continuelle, et ma plaie est-elle sans espérance ? elle a refusé d'être guérie ; me serais-tu bien comme une chose qui trompe ? [comme] des eaux qui ne durent pas ?
\VS{19}C'est pourquoi ainsi a dit l'Eternel : si tu te retournes, je te ramènerai, et tu te tiendras devant moi ; et si tu sépares la chose précieuse de la méprisable, tu seras comme ma bouche ; qu'ils se retournent vers toi, mais toi ne retourne pas vers eux.
\VS{20}Et je te ferai être à l'égard de ce peuple une muraille d'acier bien forte ; ils combattront contre toi, mais ils n'auront point le dessus contre toi, car je [suis] avec toi pour te garantir, et pour te délivrer, dit l'Eternel.
\VS{21}Et je te délivrerai de la main des malins, et te rachèterai de la main des terribles.
\Chap{16}
\VerseOne{}Puis la parole de l'Eternel me fut adressée, en disant :
\VS{2}Tu ne prendras point de femme, et tu n'auras point de fils ni de filles en ce lieu-ci.
\VS{3}Car ainsi a dit l'Eternel touchant les fils et les filles qui naîtront en ce lieu-ci, et touchant leurs mères qui les auront enfantés, et touchant les pères qui les auront engendrés en ce pays ;
\VS{4}Ils mourront de maladies très douloureuses ; ils ne seront point lamentés, ni ensevelis, mais ils seront sur le dessus de la terre comme du fumier, et ils seront consumés par l'épée et par la famine, et leurs corps morts seront pour viande aux oiseaux des cieux, et aux bêtes de la terre.
\VS{5}Même, ainsi a dit l'Eternel : n'entre point en aucune maison de deuil, et ne va point lamenter, ni te condouloir pour eux ; car j'ai retiré de ce peuple, dit l'Eternel, ma paix, ma miséricorde, et mes compassions.
\VS{6}Et les grands et les petits mourront en ce pays ; ils ne seront point ensevelis, et on ne les pleurera point, et personne ne se fera aucune incision, ni ne se rasera pour eux.
\VS{7}On ne leur distribuera point de [pain] dans le deuil pour consoler quelqu'un d'eux au sujet d'un mort ; et on ne leur donnera point à boire de la coupe de consolation pour leur père ou pour leur mère.
\VS{8}Aussi tu n'entreras point en aucune maison de festin afin de t'asseoir avec eux pour manger, ou pour boire.
\VS{9}Car ainsi a dit l'Eternel des armées, le Dieu d'Israël : voici, je m'en vais faire cesser de ce lieu-ci devant vos yeux et en vos jours, la voix de joie et la voix d'allégresse, la voix de l'époux et la voix de l'épouse.
\VS{10}Et il arrivera que quand tu auras prononcé à ce peuple toutes ces paroles-là, ils te diront : pourquoi l'Eternel a-t-il prononcé tout ce grand mal contre nous ? et quelle est notre iniquité, et quel [est] notre péché que nous avons commis contre l'Eternel notre Dieu ?
\VS{11}Et tu leur diras : parce que vos pères m'ont abandonné, dit l'Eternel, et sont allés après d'autres dieux, et les ont servis, et se sont prosternés devant eux, et m'ont abandonné, et n'ont point gardé ma Loi,
\VS{12}Et que vous avez encore fait pis que vos pères ; car voici, chacun de vous marche après la dureté de son cœur mauvais, afin de ne m'écouter point ;
\VS{13}A cause de cela je vous transporterai de ce pays en un pays que vous n'avez point connu, ni vous ni vos pères ; et là vous serez asservis jour et nuit à d'autres dieux, parce que je ne vous aurai point fait de grâce.
\VS{14}Néanmoins voici, les jours viennent, dit l'Eternel, qu'on ne dira plus : l'Eternel [est] vivant qui a fait remonter les enfants d'Israël du pays d'Egypte ;
\VS{15}Mais, l'Eternel est vivant qui a fait remonter les enfants d'Israël du pays de l'Aquilon, et de tous les pays auxquels il les avait chassés ; après que je les aurai ramenés dans leur pays, lequel j'ai donné à leurs pères.
\VS{16}Voici, je m'en vais mander à plusieurs Pêcheurs, dit l'Eternel, et ils les pêcheront ; et ensuite je m'en vais mander à plusieurs Veneurs, qui les chasseront par toutes les montagnes, et par tous les coteaux, et par tous les trous des rochers.
\VS{17}Car mes yeux sont sur tout leur train, lequel n'est point caché devant moi ; ni leur iniquité n'est point celée devant mes yeux.
\VS{18}Mais premièrement je leur rendrai le double de leur iniquité et de leur péché, à cause qu'ils ont souillé mon pays par leurs victimes abominables, et qu'ils ont rempli mon héritage de leurs abominations.
\VS{19}Eternel, qui es ma force et ma puissance, et mon refuge au jour de la détresse, les nations viendront à toi des bouts de la terre, et diront : certes nos pères ont hérité le mensonge et la vanité, et les choses auxquelles il n'y a point de profit.
\VS{20}L'homme se fera-t-il bien des dieux ? qui toutefois ne sont point dieux.
\VS{21}C'est pourquoi, voici, je vais leur faire connaître à cette fois, je vais leur faire connaître ma main et ma force, et ils sauront que mon Nom est l'Eternel.
\Chap{17}
\VerseOne{}Le péché de Juda est écrit avec un burin d’acier de fer, [et] avec une pointe de diamant ; il est gravé sur la table de leur cœur, et aux cornes de leurs autels.
\VS{2}De sorte que leurs fils se souviendront de leurs autels, et de leurs bocages, auprès des arbres verts sur les hautes collines.
\VS{3}Montagnard, je livrerai par les champs tes richesses [et] tous tes trésors au pillage ; tes hauts lieux [sont pleins] de péché dans toutes tes contrées.
\VS{4}Et toi, et [ceux qui sont] avec toi, vous laisserez vacant l'héritage que je t'avais donné, et je ferai que tu seras asservi à tes ennemis, dans un pays que tu ne connais point, parce que vous avez allumé le feu en ma colère ; [et] il brûlera à toujours.
\VS{5}Ainsi a dit l'Eternel : maudit soit l'homme qui se confie en l'homme, et qui fait de la chair son bras, et dont le cœur se retire de l'Eternel.
\VS{6}Car il sera comme la bruyère en une lande, et il ne s'apercevra point quand le bien sera venu ; mais il demeurera au désert en des lieux secs, en une terre salée et inhabitable.
\VS{7}Béni soit l'homme qui se confie en l'Eternel, et duquel l'Eternel est la confiance.
\VS{8}Car il sera comme un arbre planté près des eaux, et qui étend ses racines le long d'une eau courante ; quand la chaleur viendra, il ne s'en apercevra point ; et sa feuille sera verte, il ne sera point en peine en l'année de la sécheresse, et ne cessera point de porter du fruit.
\VS{9}Le cœur est rusé, et désespérément malin par dessus toutes choses ; qui le connaîtra ?
\VS{10}Je suis l'Eternel, qui sonde le cœur, et qui éprouve les reins ; même pour rendre à chacun selon sa voie, [et] selon le fruit de ses actions.
\VS{11}Celui qui acquiert des richesses, sans observer la justice, est une perdrix [qui] couve ce qu'elle n'a point pondu ; il les laissera au milieu de ses jours, et à la fin il sera trouvé insensé.
\VS{12}Le lieu de notre Sanctuaire est un trône de gloire, un lieu haut élevé dès le commencement.
\VS{13}Eternel, qui es l'attente d'Israël, tous ceux qui t'abandonnent seront honteux ; ceux qui se détournent de moi, seront écrits en la terre, parce qu'ils ont délaissé la source des eaux vives, l'Eternel.
\VS{14}Eternel, guéris-moi, et je serai guéri ; sauve-moi, et je serai sauvé ; car tu es ma louange.
\VS{15}Voici, ceux-ci me disent : où est la parole de l'Eternel ? qu'elle vienne présentement !
\VS{16}Mais je ne me suis point avancé plus qu'un pasteur après toi, et je n'ai point désiré le jour de l'extrême affliction, tu le sais ; et ce qui est sorti de mes lèvres a été devant toi.
\VS{17}Ne me sois point en effroi, tu es ma retraite au jour du mal.
\VS{18}Que ceux qui me persécutent soient honteux, mais que je ne sois point honteux ; qu'ils soient épouvantés, mais que je ne sois point épouvanté ; amène sur eux le jour du mal, et les accable d'une double plaie.
\VS{19}Ainsi m'a dit l'Eternel : va, et tiens-toi debout à la porte des enfants du peuple, par laquelle les Rois de Juda entrent, et par laquelle ils sortent ; et à toutes les portes de Jérusalem.
\VS{20}Et leur dis : écoutez la parole de l'Eternel, Rois de Juda, et vous tous [hommes] de Juda, et vous tous habitants de Jérusalem qui entrez par ces portes ;
\VS{21}Ainsi a dit l'Eternel : prenez garde à vos âmes, et ne portez aucun fardeau le jour du Sabbat, et ne les faites point passer par les portes de Jérusalem.
\VS{22}Et ne tirez point hors de vos maisons aucun fardeau le jour du Sabbat, et ne faites aucune œuvre, mais sanctifiez le jour du Sabbat, comme j'ai commandé à vos pères.
\VS{23}Mais ils n'ont point écouté, et n'ont point incliné leur oreille, mais ils ont raidi leur cou, pour n'écouter point, et pour ne recevoir point d'instruction.
\VS{24}Il arrivera donc si vous m'écoutez attentivement, dit l'Eternel pour ne faire passer aucun fardeau par les portes de cette ville le jour du Sabbat, et si vous sanctifiez le jour du Sabbat, tellement que vous ne fassiez aucune œuvre en ce jour-là ;
\VS{25}Que les Rois et les principaux, ceux qui sont assis sur le trône de David, montés sur des chariots et sur des chevaux, eux et les principaux d'entre eux, les hommes de Juda, et les habitants de Jérusalem, entreront par les portes de cette ville ; et cette ville sera habitée à toujours.
\VS{26}On viendra aussi des villes de Juda, et des environs de Jérusalem et du pays de Benjamin, et de la campagne, et des montagnes, et de devers le Midi, et on apportera des holocaustes, des sacrifices, des oblations et de l'encens ; on apportera aussi [des sacrifices] d'action de grâces en la maison de l'Eternel.
\VS{27}Mais si vous ne m'écoutez point pour sanctifier le jour du Sabbat, et pour ne porter aucun fardeau, et n'en faire entrer aucun par les portes de Jérusalem le jour du Sabbat, je mettrai le feu à ses portes, il consumera les palais de Jérusalem, et ne sera point éteint.
\Chap{18}
\VerseOne{}Cette parole fut [adressée] de par l'Eternel à Jérémie, disant :
\VS{2}Lève-toi, et descends dans la maison d'un potier, et là je te ferai entendre mes paroles.
\VS{3}Je descendis donc dans la maison d'un potier, et voici, il faisait son ouvrage, assis sur sa selle.
\VS{4}Et le vase qu'il faisait de l'argile qui [était] en sa main, fut gâté, et il en fit encore un autre vase, comme il lui sembla bon de [le] faire.
\VS{5}Alors la parole de l'Eternel me fut [adressée], en disant :
\VS{6}Maison d'Israël, ne vous pourrai-je pas faire comme a fait ce potier ; dit l'Eternel ? voici, comme l'argile est dans la main d'un potier, ainsi êtes-vous dans ma main, maison d'Israël.
\VS{7}En un instant je parlerai contre une nation, et contre un Royaume, pour arracher, pour démolir, et pour détruire ;
\VS{8}Mais si cette nation contre laquelle j'aurai parlé se détourne du mal qu'elle aura fait, je me repentirai aussi du mal que j'avais pensé de lui faire.
\VS{9}Et si en un instant je parle d'une nation et d'un Royaume, pour l'édifier et pour le planter ;
\VS{10}Et que cette nation fasse ce qui me déplaît, en sorte qu'elle n'écoute point ma voix, je me repentirai aussi du bien que j'avais dit que je lui ferais.
\VS{11}Or donc parle maintenant aux hommes de Juda, et aux habitants de Jérusalem, en disant : ainsi a dit l'Eternel : voici, je projette du mal contre vous, et je forme un dessein contre vous ; abandonnez donc maintenant chacun sa mauvaise voie, et changez votre voie, et vos actions.
\VS{12}Et ils répondirent : il n'y a plus d'espérance ; c'est pourquoi nous suivrons nos pensées, et chacun de nous fera selon la dureté de son cœur mauvais.
\VS{13}C'est pourquoi ainsi a dit l'Eternel : Demandez maintenant aux nations qui a entendu de telles choses ? la vierge d'Israël a fait une chose très-énorme.
\VS{14}N'abandonnera-t-on pas la neige du Liban pour la roche du champ ? et ne laissera-t-on pas les eaux qui ne sont point naturelles, et qui sont froides, [encore] qu'elles coulent ?
\VS{15}Mais mon peuple m'a oublié, et il a fait des parfums à ce qui n'est que vanité, et qui les a fait broncher dans leurs voies, pour [les faire retirer] des sentiers anciens, afin de marcher dans les sentiers d'un chemin qui n'est point battu ;
\VS{16}Pour faire venir sur leur pays une désolation et un opprobre perpétuel ; [tellement que] quiconque passera par là en sera étonné, et branlera la tête.
\VS{17}Je les disperserai devant l'ennemi, comme par le vent d'Orient ; je leur tournerai le dos, au jour de leur calamité.
\VS{18}Et ils ont dit : venez, et faisons quelques machinations contre Jérémie ; car la Loi ne se perdra pas chez le Sacrificateur, ni le conseil chez le Sage, ni la parole chez le Prophète ; venez, et frappons-le de la langue, et ne soyons point attentifs à aucun de ses discours.
\VS{19}Eternel, entends-moi, et écoute la voix de ceux qui plaident contre moi.
\VS{20}Le mal sera-t-il rendu pour le bien ? car ils ont creusé une fosse pour mon âme. Souviens-toi que je me suis présenté devant toi afin de parler pour leur bien, [et] afin de détourner d'eux ta grande colère.
\VS{21}C'est pourquoi livre leurs enfants à la famine, et fais couler leur sang à coups d'épée ; que leurs femmes soient privées d'enfants, et veuves, et que leurs maris soient mis à mort ; et leurs jeunes gens tués avec l'épée dans la bataille.
\VS{22}Que le cri soit ouï de leurs maisons, quand tu auras fait venir subitement des troupes contre eux ; parce qu'ils ont creusé une fosse pour me prendre, et qu'ils ont tendu des filets à mes pieds.
\VS{23}Or tu sais, ô Eternel ! que tout leur conseil est contre moi pour me mettre à mort ; ne sois point apaisé à l'égard de leur iniquité, et n'efface point leur péché de devant ta face, mais qu'on les fasse tomber en ta présence ; agis contre eux au temps de ta colère.
\Chap{19}
\VerseOne{}Ainsi a dit l'Eternel : va, et achète une bouteille de terre d'un potier, et [prends] des anciens du peuple, et des anciens des Sacrificateurs ;
\VS{2}Et sors à la vallée du fils de Hinnom, qui est auprès de l'entrée de la porte Orientale, et crie là les paroles que je te dirai.
\VS{3}Dis donc : Rois de Juda, et vous habitants de Jérusalem, écoutez la parole de l'Eternel : ainsi a dit l'Eternel des armées, le Dieu d'Israël : voici, je m'en vais faire venir sur ce lieu-ci un mal, tel que quiconque l'entendra, les oreilles lui en tinteront.
\VS{4}Parce qu'ils m'ont abandonné, et qu'ils ont profané ce lieu, et y ont fait des encensements à d'autres dieux, que ni eux, ni leurs pères, ni les Rois de Juda n'ont point connus, et parce qu'ils ont rempli ce lieu du sang des innocents ;
\VS{5}Et qu'ils ont bâti des hauts lieux de Bahal, afin de brûler au feu leurs fils pour en faire des holocaustes à Bahal, ce que je n'ai point commandé, et dont je n'ai point parlé, et à quoi je n'ai jamais pensé ;
\VS{6}A cause de cela voici, les jours viennent, dit l'Eternel, que ce lieu-ci ne sera plus appelé Topheth, ni la vallée du fils de Hinnom, mais la vallée de la tuerie.
\VS{7}Et j'anéantirai le conseil de Juda et de Jérusalem en ce lieu-ci, et je les ferai tomber par l'épée en la présence de leurs ennemis, et en la main de ceux qui cherchent leur vie ; et je donnerai leurs corps morts à manger aux oiseaux des cieux et aux bêtes de la terre.
\VS{8}Je détruirai cette ville, et je la couvrirai d'opprobre ; quiconque passera près d'elle sera étonné, et lui insultera à cause de toutes ses plaies.
\VS{9}Et je leur ferai manger la chair de leurs fils, et la chair de leurs filles ; et chacun mangera la chair de son compagnon durant le siège, et dans l'extrémité où les réduiront leurs ennemis, et ceux qui cherchent leur vie.
\VS{10}Puis tu casseras la bouteille en présence de ceux qui seront allés avec toi ;
\VS{11}Et tu leur diras : ainsi a dit l'Eternel des armées : je briserai ce peuple et cette ville, de même qu'on brise un vaisseau de potier, qui ne peut être soudé, et ils seront ensevelis à Topheth, parce qu'il [n'y aura] plus d'autre lieu pour [les] ensevelir.
\VS{12}Je ferai ainsi à ce lieu-ci, dit l'Eternel, et à ses habitants ; tellement que je réduirai cette ville au même état que Topheth ;
\VS{13}Et les maisons de Jérusalem, et les maisons des Rois de Juda seront souillées, comme le lieu de Topheth, à cause de toutes les maisons sur les toits desquelles ils ont fait des parfums à toute l'armée des cieux, et des aspersions à d'autres dieux.
\VS{14}Puis Jérémie s'en vint de Topheth là où l'Eternel l'avait envoyé pour prophétiser, et il se tint debout au parvis de la maison de l'Eternel, et dit à tout le peuple :
\VS{15}Ainsi a dit l'Eternel des armées, le Dieu d'Israël : voici, je m'en vais faire venir sur cette ville, et sur toutes ses villes, tout le mal que j'ai prononcé contre elle, parce qu'ils ont raidi leur cou, pour ne point écouter mes paroles.
\Chap{20}
\VerseOne{}Alors Pashur, fils d'Immer Sacrificateur, qui était prévôt [et] conducteur dans la maison de l'Eternel, entendit Jérémie qui prophétisait ces choses.
\VS{2}Et Pashur frappa le Prophète Jérémie, et le mit dans la prison qui est à la haute porte de Benjamin, en la maison de l'Eternel.
\VS{3}Et il arriva dès le lendemain, que Pashur tira Jérémie hors de la prison, et Jérémie lui dit : l'Eternel n'a pas appelé ton Nom Pashur, mais Magor-missabib.
\VS{4}Car ainsi a dit l'Eternel : voici, je vais te livrer à la frayeur, toi, et tous tes amis, qui tomberont par l'épée de leurs ennemis, et tes yeux le verront ; je livrerai tous ceux de Juda entre les mains du Roi de Babylone, qui les transportera à Babylone, et les frappera avec l'épée.
\VS{5}Et je livrerai toutes les richesses de cette ville, et tout son travail, et tout ce qu'elle a de précieux, je livrerai, dis-je, tous les trésors des Rois de Juda, entre les mains de leurs ennemis, qui les pilleront, les enlèveront, et les emporteront à Babylone.
\VS{6}Et toi, Pashur, et tous les habitants de ta maison, vous irez en captivité, tu iras à Babylone, tu y mourras, et y seras enseveli, toi et tous tes amis, auxquels tu as prophétisé le mensonge.
\VS{7}Ô Eternel ! tu m'as sollicité, et j'ai été attiré ; tu as été plus fort que moi, et tu as eu le dessus ; je suis un objet de moquerie tout le jour, chacun se moque de moi.
\VS{8}Car depuis que je parle je n'ai fait que jeter des cris, que crier violence et pillerie, parce que la parole de l'Eternel m'est tournée en opprobre et en moquerie tout le jour.
\VS{9}C'est pourquoi j'ai dit : je ne ferai plus mention de lui, et je ne parlerai plus en son Nom ; mais il y a eu dans mon cœur [comme] un feu ardent, renfermé dans mes os ; je suis las de le porter, et je n'en puis plus.
\VS{10}Car j'ai ouï les insultes de plusieurs, la frayeur [m'a saisi] de tous côtés. Rapportez, [disent-ils], et nous le rapporterons. Tous ceux qui ont paix avec moi épient si je bronche, [et disent] : peut-être qu'il sera abusé ; alors nous aurons le dessus, et nous nous vengerons de lui.
\VS{11}Mais l'Eternel est avec moi comme un homme fort et terrible ; c'est pourquoi ceux qui me persécutent seront renversés, ils n'auront point le dessus, ils seront honteux ; car ils n'ont pas été prudents ; ce sera une confusion éternelle, [qui] ne s'oubliera jamais.
\VS{12}C'est pourquoi, Eternel des armées, qui sondes les justes, qui vois les reins et le cœur, fais que je voie la vengeance que tu en feras ; car je t'ai découvert ma cause.
\VS{13}Chantez à l'Eternel, louez l'Eternel ; car il a délivré l'âme des pauvres de la main des méchants.
\VS{14}Maudit soit le jour auquel je naquis ; que le jour auquel ma mère m'enfanta ne soit point béni.
\VS{15}Maudit soit l'homme qui apporta de bonnes nouvelles à mon père, en lui disant : un enfant mâle t'est né ; et qui le réjouit si bien.
\VS{16}Que cet homme-là soit comme les villes que l'Eternel a renversées sans s'en repentir, qu'il entende le cri au matin, et le retentissement bruyant au temps du Midi.
\VS{17}Que ne m'a-t-on fait mourir dans le sein de ma mère ? pourquoi ma mère ne m'a-t-elle été mon sépulcre, et pourquoi son sein n'a-t-il conçu [sans] jamais enfanter.
\VS{18}Pourquoi suis-je né pour ne voir que peine et qu'ennui, et afin que mes jours fussent consumés avec honte ?
\Chap{21}
\VerseOne{}La parole qui fut [adressée] à Jérémie de par l'Eternel, lorsque le Roi Sédécias envoya vers lui Pashur fils de Malkija, et Sophonie fils de Mahaséja Sacrificateur, pour [lui] dire :
\VS{2}Enquiers-toi maintenant de l'Eternel pour nous ; car Nébucadnetsar Roi de Babylone combat contre nous ; peut-être que l'Eternel agira pour nous selon toutes ses merveilles, et le fera retirer de nous.
\VS{3}Et Jérémie leur dit : vous direz ainsi à Sédécias :
\VS{4}Ainsi a dit l'Eternel, le Dieu d'Israël : voici, je m'en vais faire retourner de dehors la muraille les instruments de guerre qui sont en vos mains, avec lesquels vous combattez contre le Roi de Babylone et contre les Caldéens qui vous assiègent, et je les ramasserai au milieu de cette ville.
\VS{5}Et je combattrai contre vous avec une main étendue, et avec un bras puissant, avec colère, avec ardeur, et avec une grande indignation.
\VS{6}Et je frapperai les habitants de cette ville, les hommes, et les bêtes ; [et] ils mourront d'une grande mortalité.
\VS{7}Et après cela, dit l'Eternel, je livrerai Sédécias Roi de Juda, et ses serviteurs, et le peuple, et ceux qui seront demeurés de reste en cette ville de la mortalité, de l'épée, et de la famine, en la main de Nébucadnetsar Roi de Babylone, et en la main de leurs ennemis, et en la main de ceux qui cherchent leur vie ; et il les frappera au tranchant de l'épée, il ne les épargnera point, il n'en aura point de compassion, il n'en aura point de pitié.
\VS{8}Tu diras aussi à ce peuple : ainsi a dit l'Eternel : voici, je mets devant vous le chemin de la vie, et le chemin de la mort.
\VS{9}Quiconque se tiendra dans cette ville mourra par l'épée, ou par la famine, ou par la mortalité, mais celui qui en sortira, et se rendra aux Caldéens qui vous assiègent, vivra, et sa vie lui sera pour butin.
\VS{10}Car j'ai dressé ma face en mal et non en bien contre cette ville, dit l'Eternel, elle sera livrée en la main du Roi de Babylone, et il la brûlera par le feu.
\VS{11}Et quant à la maison du Roi de Juda, écoutez la parole de l'Eternel.
\VS{12}Maison de David, ainsi a dit l'Eternel : faites justice dès le matin, et délivrez celui qui aura été pillé, d'entre les mains de celui qui lui fait tort, de peur que ma fureur ne sorte comme un feu, et qu'elle ne s'embrase, et qu'il n'y ait personne qui l'éteigne, à cause de la malice de vos actions.
\VS{13}Voici, j'en veux à toi, qui habites dans la vallée, [et qui es] le rocher du plat pays, dit l'Eternel ; à vous qui dites : qui est-ce qui descendra contre nous, et qui entrera dans nos demeures ?
\VS{14}Et je vous punirai selon le fruit de vos actions, dit l'Eternel ; et j'allumerai le feu dans sa forêt, lequel consumera tout ce qui est à l'entour d'elle.
\Chap{22}
\VerseOne{}Ainsi a dit l'Eternel : descends en la maison du Roi de Juda, et y prononce cette parole.
\VS{2}Tu diras donc : écoute la parole de l'Eternel, ô Roi de Juda ! qui es assis sur le trône de David, toi et tes serviteurs, et ton peuple, qui entrez par ces portes.
\VS{3}Ainsi a dit l'Eternel : faites jugement et justice, et délivrez celui qui aura été pillé, d'entre les mains de celui qui lui fait tort ; ne foulez point l'orphelin, ni l'étranger, ni la veuve ; et n'usez d'aucune violence, et ne répandez point le sang innocent dans ce lieu-ci.
\VS{4}Car si vous mettez exactement en effet cette parole, alors les Rois qui sont assis en la place de David sur son trône, montés sur des chariots et sur des chevaux, entreront par les portes de cette maison, eux et leurs serviteurs, et leur peuple.
\VS{5}Mais si vous n'écoutez point ces paroles, j'ai juré par moi-même, dit l'Eternel, que cette maison sera réduite en désolation.
\VS{6}Car ainsi a dit l'Eternel touchant la maison du Roi de Juda : tu m'es un Galaad, [et] le sommet du Liban, [mais] si je ne te réduis en désert, et en villes qui ne sont point habitées.
\VS{7}Je préparerai contre toi des destructeurs chacun avec ses armes, qui couperont tes cèdres exquis, et les jetteront au feu ;
\VS{8}Et plusieurs nations passeront près de cette ville, et chacun dira à son compagnon : pourquoi l'Eternel a-t-il fait ainsi à cette grande ville ?
\VS{9}Et on dira : c'est parce qu'ils ont abandonné l'alliance de l'Eternel leur Dieu, et qu'ils se sont prosternés devant d'autres dieux, et les ont servis.
\VS{10}Ne pleurez point celui qui est mort, et n'en faites point de lamentation ; mais pleurez amèrement celui qui s'en va, car il ne retournera plus, et ne verra plus le pays de sa naissance.
\VS{11}Car ainsi a dit l'Eternel touchant Sallum fils de Josias, Roi de Juda, qui a régné en la place de Josias son père, [et] qui est sorti de ce lieu, iI n'y retournera plus.
\VS{12}Mais il mourra au lieu auquel on l'a transporté, et ne verra plus ce pays.
\VS{13}Malheur à celui qui bâtit sa maison par l'injustice, et ses étages sans droiture, qui se sert pour rien de son prochain, et ne lui donne point le salaire de son travail.
\VS{14}Qui dit : je me bâtirai une grande maison et des étages bien aérés, et qui se perce des fenêtrages ; elle est lambrissée de cèdre, et peinte de vermillon.
\VS{15}Régneras-tu, que tu te mêles parmi les cèdres ? Ton père n'a-t-il pas mangé et bu ? quand il a fait jugement et justice, alors il a prospéré.
\VS{16}Il a jugé la cause de l'affligé, et du pauvre, et alors il a prospéré ; cela n'était-il pas me connaître ? dit l'Eternel.
\VS{17}Mais tes yeux et ton cœur ne [sont adonnés] qu'à ton gain déshonnête, qu'à répandre le sang innocent, qu'à faire tort, et qu'à opprimer.
\VS{18}C'est pourquoi, ainsi a dit l'Eternel touchant Jéhojakim fils de Josias Roi de Juda : on ne le plaindra point, [en disant] : hélas mon frère ! et hélas ma sœur ! on ne le plaindra point, [en disant] : hélas Sire ! et, hélas sa magnificence !
\VS{19}Il sera enseveli de la sépulture d'un âne, étant traîné, et jeté au delà des portes de Jérusalem.
\VS{20}Monte au Liban, et crie, jette ta voix en Basan, et crie par les passages, à cause que tous tes amoureux ont été mis en pièces.
\VS{21}Je t'ai parlé durant ta grande prospérité, [mais] tu as dit : je n'écouterai point ; tel [est] ton train dès ta jeunesse, que tu n'as point écouté ma voix.
\VS{22}Le vent remplira tous tes pasteurs, et tes amoureux iront en captivité ; certainement tu seras alors honteuse et confuse à cause de toute ta malice.
\VS{23}Tu te tiens au Liban, et tu fais ton nid dans les cèdres, ô que tu seras un objet de compassion quand les tranchées te viendront, [et] ta douleur, comme de la femme qui est en travail d'enfant.
\VS{24}Je suis vivant, dit l'Eternel, que quand Chonja, fils de Jéhojakim, Roi de Juda, serait un cachet en ma main droite, je l'arracherai de là.
\VS{25}Et je le livrerai en la main de ceux qui cherchent la vie, et en la main de ceux de la présence desquels tu as peur, et en la main de Nébucadnetsar Roi de Babylone, et en la main des Caldéens.
\VS{26}Et je te jetterai, toi et ta mère qui t'a enfanté, en un autre pays, auquel vous n'êtes point nés ; et vous y mourrez.
\VS{27}Et quant au pays qu’ils désirent pour y retourner, ils n'y retourneront point.
\VS{28}Ce personnage Chonja serait-ce une idole méprisée [et] rompue ? serait-ce un vaisseau qui n’a rien d’aimable ? pourquoi ont-ils été jetés là, lui et sa postérité, jetés, dis-je, en un pays qu'ils ne connaissent point ?
\VS{29}Ô terre ! terre ! terre ! écoute la parole de l'Eternel.
\VS{30}Ainsi a dit l'Eternel : écrivez que ce personnage-là est privé d'enfant, que c'est un homme qui ne prospérera point pendant ses jours, et que même il n'y aura point d'homme de sa postérité qui prospère, et qui soit assis sur le trône de David, ni qui domine plus en Juda.
\Chap{23}
\VerseOne{}Malheur aux pasteurs qui détruisent et dissipent le troupeau de ma pâture, dit l'Eternel !
\VS{2}C'est pourquoi ainsi a dit l'Eternel, le Dieu d'Israël, touchant les pasteurs qui paissent mon peuple : vous avez dissipé mes brebis, et vous les avez chassées, et ne les avez point visitées ; voici, je m'en vais visiter sur vous la malice de vos actions, dit l'Eternel.
\VS{3}Mais je rassemblerai le reste de mes brebis de tous les pays auxquels je les aurai chassées, et les ferai retourner à leurs parcs, et elles fructifieront et multiplieront.
\VS{4}J'établirai aussi sur elles des pasteurs qui les paîtront, et elles n'auront plus de peur, et ne s'épouvanteront point, et il n'en manquera aucune, dit l'Eternel.
\VS{5}Voici, les jours viennent, dit l'Eternel, que je ferai lever à David un Germe juste, qui régnera [comme] Roi ; il prospérera, et exercera le jugement et la justice sur la terre.
\VS{6}En ses jours Juda sera sauvé, et Israël habitera en assurance ; et c'est ici le nom, duquel on l'appellera : l'Eternel notre justice.
\VS{7}C'est pourquoi voici, les jours viennent, dit l'Eternel, qu'on ne dira plus : l'Eternel est vivant, qui a fait remonter les enfants d'Israël du pays d'Egypte ;
\VS{8}Mais, l'Eternel est vivant, qui a fait remonter, et qui a ramené la postérité de la maison d'Israël, du pays de devers l'Aquilon, et de tous les pays auxquels je les avais chassés, et ils habiteront en leur terre.
\VS{9}A cause des Prophètes mon cœur est brisé au dedans de moi, tous mes os en tremblent, je suis comme un homme ivre, et comme un homme que le vin a surmonté, à cause de l'Eternel, et à cause des paroles de sa sainteté.
\VS{10}Car le pays est rempli [d'hommes] adultères, et le pays mène deuil à cause des exécrations : les pâturages du désert sont devenus tous secs, l'oppression de ces gens est mauvaise, et leur force n'est pas en faveur de l'équité.
\VS{11}Car le Prophète et le Sacrificateur se contrefont ; j'ai même trouvé dans ma maison leur méchanceté, dit l'Eternel.
\VS{12}C'est pourquoi leur voie sera comme des lieux glissants dans les ténèbres, ils y seront poussés, et y tomberont ; car je ferai venir du mal sur eux, [en] l'année de leur visitation, dit l'Eternel.
\VS{13}Or j'avais bien vu des choses mal convenables dans les Prophètes de Samarie, [car] ils prophétisaient de par Bahal, et faisaient égarer mon peuple Israël.
\VS{14}Mais j'ai vu des choses énormes dans les Prophètes de Jérusalem ; car ils commettent des adultères, et ils marchent dans le mensonge ; ils ont donné main forte aux hommes injustes et pas un ne s'est détourné de sa malice ; ils me sont tous comme Sodome, et les habitants de la ville, comme Gomorrhe.
\VS{15}C'est pourquoi, ainsi a dit l'Eternel des armées touchant ces Prophètes : voici, je m'en vais leur faire manger de l'absinthe, et leur faire boire de l'eau de fiel ; parce que la profanation s'est répandue des Prophètes de Jérusalem par tout le pays.
\VS{16}Ainsi a dit l'Eternel des armées : n'écoutez point les paroles des Prophètes qui vous prophétisent ; ils vous font devenir vains, ils prononcent la vision de leur cœur, [et] ils ne [la tiennent] pas de la bouche de l'Eternel.
\VS{17}Ils ne cessent de dire à ceux qui me méprisent : l'Eternel a dit : vous aurez la paix ; et ils disent à tous ceux qui marchent dans la dureté de leur cœur : il ne vous arrivera point de mal.
\VS{18}Car qui s'est trouvé au conseil secret de l'Eternel ? et qui a aperçu et ouï sa parole ? qui a été attentif à sa parole, et l'a ouïe ?
\VS{19}Voici la tempête de l'Eternel, sa fureur va se montrer, et le tourbillon prêt à fondre tombera sur la tête des méchants.
\VS{20}La colère de l'Eternel ne sera point détournée qu'il n'ait exécuté et mis en effet les pensées de son cœur ; vous aurez une claire intelligence de ceci sur la fin des jours.
\VS{21}Je n'ai point envoyé ces Prophètes-là, et ils ont couru ; je ne leur ai point parlé, et ils ont prophétisé.
\VS{22}S'ils s'étaient trouvés dans mon conseil secret, ils auraient aussi fait entendre mes paroles à mon peuple, et ils les auraient détournés de leur mauvais train, et de la malice de leurs actions.
\VS{23}Suis-je un Dieu de près, dit l'Eternel, et ne suis-je point aussi un Dieu de loin ?
\VS{24}Quelqu'un se pourra-t-il cacher dans quelque retraite, que je ne le voie point ? dit l'Eternel. Ne remplis-je pas, moi, les cieux et la terre ? dit l'Eternel.
\VS{25}J'ai ouï ce que les Prophètes ont dit, prophétisant le mensonge en mon Nom, [et] disant : j'ai eu un songe, j'ai eu un songe.
\VS{26}Jusques à quand ceci sera-t-il au cœur des Prophètes qui prophétisent le mensonge, et qui prophétisent la tromperie de leur cœur ?
\VS{27}Qui pensent comment ils feront oublier mon Nom à mon peuple, par les songes qu'un chacun d'eux récite à son compagnon, comme leurs pères ont oublié mon Nom pour Bahal.
\VS{28}Que le Prophète par devers lequel est le songe, récite le songe ; et que celui par devers lequel est ma parole, profère ma parole en vérité. Quelle [convenance y a-t-il] de la paille avec le froment ? dit l'Eternel.
\VS{29}Ma parole n'est-elle pas comme un feu, dit l'Eternel ; et comme un marteau qui brise la pierre ?
\VS{30}C'est pourquoi voici, j'en veux aux Prophètes, dit l'Eternel, qui dérobent mes paroles, chacun de son prochain.
\VS{31}Voici, j'en veux aux Prophètes, dit l'Eternel, qui accommodent leurs langues, et qui disent : il dit.
\VS{32}Voici, j'en veux à ceux qui prophétisent des songes de fausseté, dit l'Eternel, et qui les récitent, et font égarer mon peuple par leurs mensonges, et par leur témérité, quoique je ne les aie point envoyés, et que je ne leur aie point donné de charge ; c'est pourquoi ils ne profiteront de rien à ce peuple, dit l'Eternel.
\VS{33}Si donc ce peuple t'interroge, ou qu'il interroge le Prophète, ou le Sacrificateur, en disant : quelle est la charge de l'Eternel ? tu leur diras : quelle charge ? Je vous abandonnerai, dit l'Eternel.
\VS{34}Et quant au Prophète, et au Sacrificateur, et au peuple qui aura dit : la charge de l'Eternel ; je punirai cet homme-là, et sa maison.
\VS{35}Vous direz ainsi chacun à son compagnon, et chacun à son frère : qu'a répondu l'Eternel ; et qu'a prononcé l'Eternel ?
\VS{36}Et vous ne ferez plus mention de la charge de l'Eternel ; car la parole de chacun lui sera pour charge ; parce que vous avez perverti les paroles du Dieu vivant, [les paroles] de l'Eternel des armées, notre Dieu.
\VS{37}Tu diras ainsi au Prophète : que t'a répondu l'Eternel, et que t'a prononcé l'Eternel ?
\VS{38}Et si vous dites : la charge de l'Eternel ; à cause de cela, a dit l'Eternel, parce que vous avez dit cette parole, la charge de l'Eternel ; et que j'ai envoyé vers vous, pour vous dire : ne dites plus : la charge de l'Eternel.
\VS{39}A cause de cela me voici, et je vous oublierai entièrement, et j'arracherai de ma présence, vous et la ville que j'ai donnée à vous et à vos pères.
\VS{40}Et je mettrai sur vous un opprobre éternel, et une confusion éternelle, qui ne sera point mise en oubli.
\Chap{24}
\VerseOne{}L'Eternel me fit voir [une vision], et voici deux paniers de figues, posés devant le Temple de l'Eternel, après que Nébucadnetsar Roi de Babylone eut transporté de Jérusalem Jéchonias fils de Jéhojakim, Roi de Juda, et les principaux de Juda, avec les charpentiers et les serruriers, et les eut emmenés à Babylone.
\VS{2}L'un des paniers avait de fort bonnes figues, comme sont d'ordinaire les figues qui sont les premières mûres ; et l'autre panier avait de fort mauvaises figues, lesquelles on n'aurait pu manger, tant elles étaient mauvaises.
\VS{3}Et l'Eternel me dit : que vois-tu, Jérémie ? Et je répondis : des figues, de bonnes figues, fort bonnes ; et de mauvaises, fort mauvaises, lesquelles on ne saurait manger, tant elles sont mauvaises.
\VS{4}Alors la parole de l'Eternel me fut [adressée], en disant :
\VS{5}Ainsi a dit l'Eternel, le Dieu d'Israël : comme ces figues sont bonnes, ainsi je me souviendrai, pour leur faire du bien, de ceux qui ont été transportés de Juda, lesquels j'ai envoyés hors de ce lieu au pays des Caldéens.
\VS{6}Et je mettrai mes yeux sur eux pour leur faire du bien, et je les ferai retourner en ce pays, je les y rétablirai, et je ne les ruinerai plus, je les planterai, et je ne les arracherai point.
\VS{7}Et je leur donnerai un cœur pour me connaître, [pour connaître, dis-je], que je suis l'Eternel ; et ils seront mon peuple, et je serai leur Dieu : car ils se retourneront à moi de tout leur cœur.
\VS{8}Et comme ces figues sont si mauvaises qu'on n'en peut manger, tant elles sont mauvaises ; ainsi certainement, a dit l'Eternel, je rendrai tel Sédécias le Roi de Juda, et les principaux de sa Cour, et le reste de Jérusalem qui sont demeurés dans ce pays, et ceux qui s'habitueront au pays d'Egypte.
\VS{9}Et je les livrerai pour être agités pour leur malheur par tous les Royaumes de la terre, et pour être en opprobre, en proverbe, en raillerie, et en malédiction par tous les lieux où je les aurai chassés.
\VS{10}Et j'enverrai sur eux l'épée, la famine, et la mortalité, jusqu’à ce qu'ils soient consumés de dessus la terre que je leur avais donnée, à eux, et à leurs pères.
\Chap{25}
\VerseOne{}La parole qui fut [adressée] à Jérémie touchant tout le peuple de Juda, la quatrième année de Jéhojakim fils de Josias Roi de Juda, qui est la première année de Nébucadnetsar Roi de Babylone.
\VS{2}Laquelle Jérémie le Prophète prononça à tout le peuple de Juda, et à tous les habitants de Jérusalem, en disant :
\VS{3}Depuis la treizième année de Josias fils d'Amon Roi de Juda, jusqu’à ce jour, qui est la vingt-troisième année, la parole de l'Eternel m'a été [adressée], et je vous ai parlé, me levant dès le matin, et parlant ; mais vous n'avez point écouté.
\VS{4}Et l'Eternel vous a envoyé tous ses serviteurs Prophètes, se levant dès le matin, et les envoyant ; mais vous ne les avez point écoutés, et vous n'avez point incliné vos oreilles pour écouter.
\VS{5}Lorsqu'ils disaient : détournez-vous maintenant chacun de son mauvais train, et de la malice de vos actions, et vous habiterez d'un siècle à l'autre sur la terre que l'Eternel vous a donnée, à vous et à vos pères.
\VS{6}Et n'allez point après d'autres dieux, pour les servir, et pour vous prosterner devant eux, et ne m'irritez point par les œuvres de vos mains ; et je ne vous ferai aucun mal.
\VS{7}Mais vous m'avez désobéi, dit l'Eternel, pour m'irriter par les œuvres de vos mains, à votre dommage.
\VS{8}C'est pourquoi ainsi a dit l'Eternel des armées : parce que vous n'avez point écouté mes paroles,
\VS{9}Voici, j'enverrai, et j'assemblerai toutes les familles de l'Aquilon, dit l'Eternel, [j'enverrai], dis-je, vers Nébucadnetsar Roi de Babylone mon serviteur ; et je les ferai venir contre ce pays et contre ses habitants, et contre toutes ces nations d'alentour ; je les détruirai à la façon de l'interdit, je les mettrai en désolation, et en opprobre, et en déserts éternels.
\VS{10}Et je ferai cesser parmi eux la voix de joie, et la voix d'allégresse, la voix de l'époux et la voix de l'épouse, le bruit des meules, et la lumière des lampes.
\VS{11}Et tout ce pays sera un désert, jusqu’à s'en étonner, et ces nations seront asservies au Roi de Babylone soixante-dix ans.
\VS{12}Et il arrivera que quand les soixante-dix ans seront accomplis, je punirai, dit l'Eternel, le Roi de Babylone, et cette nation-là, de leurs iniquités, et le pays des Caldéens, que je mettrai en désolations éternelles.
\VS{13}Et je ferai venir sur ce pays-là toutes mes paroles que j'ai prononcées contre lui, toutes les choses qui sont écrites dans ce livre, lesquelles Jérémie a prophétisées contre toutes ces nations.
\VS{14}Car de grands Rois aussi et de grandes nations se serviront d'eux, et je leur rendrai selon leurs actions, et selon l'œuvre de leurs mains.
\VS{15}Car ainsi m'a dit l'Eternel, le Dieu d'Israël : prends de ma main la coupe de ce vin, [savoir] de cette fureur-ci, et en fais boire à tous les peuples auxquels je t'envoie.
\VS{16}Ils [en] boiront, et ils [en] seront troublés, et ils en perdront l'esprit, à cause de l'épée que j'enverrai parmi eux.
\VS{17}Je pris donc la coupe de la main de l'Eternel, et j'en fis boire à toutes les nations auxquelles l'Eternel m'envoyait.
\VS{18}[Savoir] à Jérusalem, et aux villes de Juda, et à ses Rois, et à ses principaux, pour les mettre en désolation, en étonnement, en opprobre, et en malédiction, comme [il paraît] aujourd'hui.
\VS{19}A Pharaon Roi d'Egypte, et à ses serviteurs, et aux principaux [de sa Cour], et à tout son peuple.
\VS{20}Et à tout le mélange [d'Arabie], et à tous les Rois du pays de Huts ; et à tous les Rois du pays des Philistins, à Askélon, Gaza, et Hékron, et au reste d'Asdod.
\VS{21}A Edom ; et à Moab ; et aux enfants de Hammon ;
\VS{22}A tous les Rois de Tyr ; et à tous les Rois de Sidon ; et aux Rois des Iles qui sont au delà de la mer ;
\VS{23}A Dédan ; Téma ; et Buz ; et à tous ceux qui se sont coupés les cheveux ;
\VS{24}A tous les Rois d'Arabie, et à tous les Rois du mélange qui habitent au désert.
\VS{25}Et à tous les Rois de Zimri ; et à tous les Rois de Hélam ; et à tous les Rois de Mède ;
\VS{26}Et à tous les Rois de l'Aquilon, tant proches qu'éloignés l'un de l'autre ; et à tous les Royaumes de la terre, qui sont sur le dessus de la terre ; et le Roi de Sésac [en] boira après eux.
\VS{27}Et tu leur diras : ainsi a dit l'Eternel des armées, le Dieu d'Israël : buvez et soyez enivrés, même rendez le vin que vous avez bu et soyez renversés sans vous relever, à cause de l'épée que j'enverrai parmi vous.
\VS{28}Or il arrivera qu'ils refuseront de prendre la coupe de ta main pour [en] boire ; mais tu leur diras : ainsi a dit l'Eternel des armées : vous en boirez certainement.
\VS{29}Car voici, je commence d'envoyer du mal sur la ville sur laquelle mon Nom est réclamé, et vous, en seriez-vous exempts en quelque sorte ? vous n'en serez point exempts ; car je m'en vais appeler l'épée sur tous les habitants de la terre, dit l'Eternel des armées.
\VS{30}Tu prophétiseras donc contre eux toutes ces paroles-là, et tu leur diras : l'Eternel rugira d'en haut, et fera entendre sa voix de la demeure de sa Sainteté ; il rugira d'une façon épouvantable contre son agréable demeure ; il redoublera vers tous les habitants de la terre un cri d'encouragement, comme quand on presse au pressoir.
\VS{31}Le son éclatant en est venu jusques au bout de la terre ; car l'Eternel plaide avec les nations, et il contestera contre toute chair ; on livrera les méchants à l'épée, dit l'Eternel.
\VS{32}Ainsi a dit l'Eternel des armées : voici, le mal s'en va sortir d'une nation à l'autre, et un grand tourbillon se lèvera du fond de la terre.
\VS{33}Et en ce jour-là ceux qui auront été mis à mort par l'Eternel seront [étendus] depuis un bout de la terre, jusques à son autre bout, ils ne seront point pleurés, et ils ne seront point recueillis, ni ensevelis ; mais ils seront comme du fumier sur le dessus de la terre.
\VS{34}[Vous] pasteurs, hurlez et criez ; et vous magnifiques du troupeau, vautrez-vous [dans la poudre] ; car les jours [déterminés] pour vous massacrer, et [les jours] de votre mort sont accomplis ; et vous tomberez comme un vaisseau désirable.
\VS{35}Et les pasteurs n'auront aucun moyen de s'enfuir, ni les magnifiques du troupeau, d'échapper.
\VS{36}Il y aura une voix du cri des pasteurs, et un hurlement des plus puissants du troupeau, à cause que l'Eternel s'en va ravager leurs pâturages.
\VS{37}Et les cabanes paisibles seront abattues, à cause de l'ardeur de la colère de l'Eternel.
\VS{38}Il a abandonné son Tabernacle, comme le lionceau, car leur pays va être mis en désolation, à cause de l'ardeur de la fourrageuse, à cause, dis-je, de l'ardeur de sa colère.
\Chap{26}
\VerseOne{}Au commencement du règne de Jéhojakim, fils de Josias, Roi de Juda, cette parole fut [adressée] à Jérémie par l'Eternel, en disant :
\VS{2}Ainsi a dit l'Eternel : tiens-toi debout au parvis de la maison de l'Eternel, et prononce à toutes les villes de Juda qui viennent pour se prosterner dans la maison de l'Eternel, toutes les paroles que je t'ai commandé de leur prononcer ; n'en retranche pas une parole.
\VS{3}Peut-être qu'ils écouteront, et qu'ils se détourneront chacun de son mauvais train ; et je me repentirai du mal que je pense de leur faire à cause de la malice de leurs actions.
\VS{4}Tu leur diras donc : ainsi a dit l'Eternel : si vous ne m'écoutez point pour marcher dans ma Loi, laquelle je vous ai proposée,
\VS{5}Pour obéir aux paroles des Prophètes mes serviteurs que je vous envoie, me levant dès le matin, et [les] envoyant, lesquels vous n'avez point écoutés :
\VS{6}Je mettrai cette maison en même état que Silo, et je livrerai cette ville en malédiction à toutes les nations de la terre.
\VS{7}Or les Sacrificateurs et les Prophètes, et tout le peuple entendirent Jérémie prononçant ces paroles dans la maison de l'Eternel.
\VS{8}Et il arriva qu'aussitôt que Jérémie eut achevé de prononcer tout ce que l'Eternel lui avait commandé de prononcer à tout le peuple, les Sacrificateurs et les Prophètes et tout le peuple le saisirent, en disant : tu mourras de mort.
\VS{9}Pourquoi as-tu prophétisé au Nom de l'Eternel, disant : cette maison sera comme Silo, et cette ville sera déserte, tellement que personne n'y habitera ? Et tout le peuple s'assembla vers Jérémie dans la maison de l'Eternel.
\VS{10}Et les principaux de Juda ayant ouï toutes ces choses montèrent de la maison du Roi à la maison de l'Eternel, et s'assirent à l'entrée de la porte neuve de la maison de l'Eternel.
\VS{11}Et les Sacrificateurs et les Prophètes parlèrent aux principaux, et à tout le peuple, en disant : cet homme mérite d'être condamné à la mort, car il a prophétisé contre cette ville, comme vous l'avez entendu de vos oreilles.
\VS{12}Et Jérémie parla à tous les principaux, et à tout le peuple, en disant : l'Eternel m'a envoyé pour prophétiser contre cette maison, et contre cette ville, [selon] toutes les paroles que vous avez ouïes.
\VS{13}Maintenant donc, corrigez votre conduite et vos actions ; et écoutez la voix de l'Eternel votre Dieu, et l'Eternel se repentira du mal qu'il a prononcé contre vous.
\VS{14}Pour moi, me voici entre vos mains, faites de moi comme il vous semblera bon et juste.
\VS{15}Mais sachez comme une chose certaine, que si vous me faites mourir, vous mettrez du sang innocent sur vous, et sur cette ville, et sur ses habitants ; car en vérité l'Eternel m'a envoyé vers vous, afin de prononcer toutes ces paroles, vous l'entendant.
\VS{16}Alors les principaux et tout le peuple dirent aux Sacrificateurs et aux Prophètes : cet homme ne mérite pas d'être condamné à la mort ; car il nous a parlé au Nom de l'Eternel notre Dieu.
\VS{17}Et quelques-uns des Anciens du pays se levèrent, et parlèrent à toute l'assemblée du peuple, en disant :
\VS{18}Michée Morastite a prophétisé aux jours d'Ezéchias Roi de Juda, et a parlé à tout le peuple de Juda, en disant : ainsi a dit l'Eternel des armées, Sion sera labourée [comme] un champ, et Jérusalem sera réduite en monceaux de pierres, et la montagne du Temple en de hauts lieux d'une forêt.
\VS{19}Ezéchias le Roi de Juda, et tous ceux de Juda le firent-ils mourir ? Ne craignit-il pas l'Eternel, et ne supplia-t-il pas l'Eternel ? et l'Eternel se repentit du mal qu'il avait prononcé contre eux ; nous faisons donc un grand mal contre nos âmes.
\VS{20}Mais aussi, [dirent les autres], il y eut un homme qui prophétisa au Nom de l'Eternel, [savoir] Urie, fils de Sémahia, de Kiriath-jéharim, lequel ayant prophétisé contre cette même ville, et contre ce même pays, en la même manière que Jérémie ;
\VS{21}Et le Roi Jéhojakim, avec tous ses officiers, et les principaux ayant entendu ses paroles, le Roi chercha à le faire mourir ; mais Urie l'ayant appris, et ayant craint, s'enfuit, et se retira en Egypte.
\VS{22}Et le Roi Jéhojakim envoya des hommes en Egypte, [savoir] Elnathan, fils de Hacbor, et quelques gens avec lui, qui allèrent en Egypte,
\VS{23}Et qui firent revenir Urie d'Egypte, et l'amenèrent au Roi Jéhojakim, qui le frappa avec l'épée, et jeta son corps mort aux sépulcres du peuple.
\VS{24}Toutefois la main d'Ahikam fils de Saphan fut pour Jérémie, afin qu'on ne le livrât point entre les mains du peuple, pour le faire mourir.
\Chap{27}
\VerseOne{}Au commencement du règne de Jéhojakim fils de Josias, Roi de Juda, cette parole fut [adressée] par l'Eternel à Jérémie, pour dire :
\VS{2}Ainsi m'a dit l'Eternel : fais-toi des liens, et des jougs, et les mets sur ton cou ;
\VS{3}Et les envoie au Roi d'Edom, et au Roi de Moab, et au Roi des enfants de Hammon, et au Roi de Tyr, et au Roi de Sidon, par les mains des messagers qui doivent venir à Jérusalem vers Sédécias Roi de Juda.
\VS{4}Et commande-leur de dire à leurs maîtres : ainsi a dit l'Eternel des armées, le Dieu d'Israël : vous direz ainsi à vos maîtres :
\VS{5}J'ai fait la terre, les hommes, et les bêtes qui sont sur la terre par ma grande force, et par mon bras étendu, et je l'ai donnée à qui bon m'a semblé.
\VS{6}Et maintenant j'ai livré tous ces pays en la main de Nébucadnetsar Roi de Babylone, mon serviteur ; et même je lui ai donné les bêtes de la campagne, afin qu'elles lui soient asservies.
\VS{7}Et toutes les nations lui seront asservies, et à son fils, et au fils de son fils, jusqu’à ce que le temps de son pays même vienne aussi, et que plusieurs nations et de grands Rois l'asservissent.
\VS{8}Et il arrivera que la nation et le Royaume qui ne se sera pas soumis à Nébucadnetsar, Roi de Babylone, et qui n'aura pas soumis son cou au joug du Roi de Babylone, je punirai cette nation-là, dit l'Eternel, par l'épée, et par la famine, et par la mortalité, jusqu’à ce que je les aie consumés par sa main.
\VS{9}Vous donc n'écoutez point vos Prophètes, ni vos devins, ni vos songeurs, ni vos augures, ni vos magiciens, qui vous parlent, en disant : vous ne serez point asservis au Roi de Babylone.
\VS{10}Car ils vous prophétisent le mensonge pour vous faire aller loin de votre terre, afin que je vous en jette dehors, et que vous périssiez.
\VS{11}Mais la nation qui soumettra son cou au joug du Roi de Babylone, et qui se soumettra à lui, je la laisserai dans sa terre, dit l'Eternel, et elle la labourera, et y demeurera.
\VS{12}Puis je parlai à Sédécias Roi de Juda, selon toutes ces paroles-là, en disant : soumettez votre cou au joug du Roi de Babylone, et rendez-vous sujets à lui, et à son peuple, et vous vivrez.
\VS{13}Pourquoi mourriez-vous, toi et ton peuple, par l'épée, et par la famine, et par la mortalité, selon que l'Eternel a parlé touchant la nation qui ne se sera point soumise au Roi de Babylone ?
\VS{14}N'écoutez donc point les paroles des Prophètes qui vous parlent, en disant : vous ne serez point asservis au Roi de Babylone ; car ils vous prophétisent le mensonge.
\VS{15}Même je ne les ai point envoyés, dit l'Eternel, et ils vous prophétisent faussement en mon Nom, afin que je vous rejette, et que vous périssiez, vous et les Prophètes qui vous prophétisent.
\VS{16}Je parlai aussi aux Sacrificateurs, et à tout le peuple, en disant : ainsi a dit l'Eternel : n'écoutez point les paroles de vos Prophètes qui vous prophétisent, en disant : voici, les vaisseaux de la maison de l'Eternel retourneront bientôt de Babylone ; car ils vous prophétisent le mensonge.
\VS{17}Ne les écoutez donc point, rendez-vous sujets au Roi de Babylone, et vous vivrez ; pourquoi cette ville serait-elle réduite en un désert ?
\VS{18}Et s'ils sont Prophètes, et que la parole de l'Eternel soit en eux, qu'ils intercèdent maintenant envers l'Eternel des armées, afin que les vaisseaux qui sont restés dans la maison de l'Eternel, et dans la maison du Roi de Juda, et à Jérusalem, n'aillent point à Babylone.
\VS{19}Car ainsi a dit l'Eternel des armées touchant les colonnes, et la mer, et les soubassements, et les autres vaisseaux qui sont restés en cette ville.
\VS{20}Lesquels Nébucadnetsar Roi de Babylone n'a point emportés, quand il a transporté de Jérusalem à Babylone Jéchonias fils de Jéhojakim Roi de Juda, et tous les Magistrats de Juda, et de Jérusalem ;
\VS{21}L'Eternel, [dis-je], des armées, le Dieu d'Israël a dit ainsi, touchant les vaisseaux qui sont restés dans la maison de l'Eternel, et dans la maison du Roi de Juda, et à Jérusalem :
\VS{22}Ils seront emportés à Babylone, et ils y demeureront jusqu’au jour que je les visiterai, dit l'Eternel ; puis je les ferai remonter, et revenir en ce lieu-ci.
\Chap{28}
\VerseOne{}Il arriva aussi en cette même année, au commencement du règne de Sédécias Roi de Juda, [savoir] en la quatrième année, au cinquième mois, que Hanania fils de Hazur Prophète, qui était de Gabaon, me parla dans la maison de l'Eternel, en la présence des Sacrificateurs et de tout le peuple, en disant :
\VS{2}Ainsi a dit l'Eternel des armées, le Dieu d'Israël : j'ai rompu le joug du Roi de Babylone.
\VS{3}Dans deux ans accomplis je ferai rapporter en ce lieu-ci tous les vaisseaux de la maison de l'Eternel, que Nébucadnetsar Roi de Babylone a emportés de ce lieu, et transportés à Babylone.
\VS{4}Et je ferai revenir en ce lieu-ci, dit l'Eternel, Jéchonias fils de Jéhojakim Roi de Juda, et tous ceux qui ont été transportés de Juda en Babylone ; car je romprai le joug du Roi de Babylone.
\VS{5}Alors Jérémie le Prophète parla à Hanania le Prophète, en la présence des Sacrificateurs, et en la présence de tout le peuple qui assistaient dans la maison de l'Eternel.
\VS{6}Et Jérémie le Prophète dit : ainsi soit-il ; qu'ainsi fasse l'Eternel ; que l'Eternel mette en effet tes paroles que tu as prophétisées, afin qu'il fasse revenir de Babylone en ce lieu-ci les vaisseaux de la maison de l'Eternel, et tous ceux qui ont été transportés à Babylone.
\VS{7}Toutefois écoute maintenant cette parole que je prononce, toi et tout le peuple l'entendant.
\VS{8}Les Prophètes qui ont été avant moi et avant toi dès longtemps, ont prophétisé contre plusieurs pays, et contre de grands Royaumes, la guerre, et l'affliction, et la mortalité.
\VS{9}Le Prophète qui aura prophétisé la paix, quand la parole de ce Prophète sera accomplie, ce Prophète-là sera reconnu pour avoir été véritablement envoyé par l'Eternel.
\VS{10}Alors Hanania le Prophète prit le joug de dessus le cou de Jérémie le Prophète, et le rompit.
\VS{11}Puis Hanania parla en la présence de tout le peuple, en disant : ainsi a dit l'Eternel : entre ci et deux ans accomplis, je romprai ainsi le joug de Nébucadnetsar Roi de Babylone de dessus le cou de toutes les nations. Et Jérémie le Prophète s'en alla son chemin.
\VS{12}Mais la parole de l'Eternel fut adressée à Jérémie, après que Hanania le Prophète eut rompu le joug de dessus le cou de Jérémie le Prophète, en disant :
\VS{13}Va, et parle à Hanania, en disant : ainsi a dit l'Eternel : tu as rompu les jougs qui étaient de bois, mais au lieu de ceux-là, fais-en qui soient de fer.
\VS{14}Car ainsi a dit l'Eternel des armées, le Dieu d'Israël : j'ai mis un joug de fer sur le cou de toutes ces nations, afin qu'elles soient asservies à Nébucadnetsar Roi de Babylone, car elles lui seront asservies, et je lui ai aussi donné les bêtes des champs.
\VS{15}Puis Jérémie le Prophète dit à Hanania le prophète : écoute maintenant, ô Hanania ! l'Eternel ne t'a point envoyé, mais tu as fait que ce peuple s'est confié au mensonge.
\VS{16}C'est pourquoi ainsi a dit l'Eternel, voici, je te chasserai de dessus la terre, [et] tu mourras cette année ; car tu as parlé de révolte contre l'Eternel.
\VS{17}Et Hanania le prophète mourut cette année-là au septième mois.
\Chap{29}
\VerseOne{}Or ce sont ici les paroles des Lettres que Jérémie le Prophète envoya de Jérusalem au reste des Anciens de ceux qui avaient été transportés, et aux Sacrificateurs, et aux Prophètes, et à tout le peuple que Nébucadnetsar avait transporté de Jérusalem à Babylone ;
\VS{2}Après que le Roi Jéchonias fut sorti de Jérusalem, avec la Régente, et les Eunuques, et les principaux de Juda et de Jérusalem, et les charpentiers, et les serruriers.
\VS{3}Par Elhasa fils de Saphan, et Guémarja fils de Hilkija, lesquels Sédécias Roi de Juda envoyait à Babylone vers Nébucadnetsar Roi de Babylone ; [et ces Lettres] étaient de telle teneur :
\VS{4}Ainsi a dit l'Eternel des armées, le Dieu d'Israël, à tous ceux qui ont été transportés, [et] que j'ai fait transporter de Jérusalem à Babylone.
\VS{5}Bâtissez des maisons, et [y] demeurez ; plantez des jardins, et en mangez les fruits.
\VS{6}Prenez des femmes, et engendrez des fils et des filles ; prenez aussi des femmes pour vos fils, et donnez vos filles à des hommes, et qu'elles enfantent des fils et des filles, et multipliez là, et ne soyez point diminués.
\VS{7}Et cherchez la paix de la ville dans laquelle je vous ai fait transporter, et priez l'Eternel pour elle ; parce qu'en sa paix vous aurez la paix.
\VS{8}Car ainsi a dit l'Eternel des armées, le Dieu d'Israël : que vos prophètes qui sont parmi vous, et vos devins, ne vous séduisent point, et ne croyez point à vos songes que vous songez.
\VS{9}Parce qu'ils vous prophétisent faussement en mon Nom ; je ne les ai point envoyés, dit l'Eternel.
\VS{10}Car ainsi a dit l'Eternel : lorsque les soixante-dix ans seront accomplis à Babylone, je vous visiterai, et je mettrai en exécution ma bonne parole sur vous, pour vous faire retourner en ce lieu-ci.
\VS{11}Car je sais que les pensées que j'ai sur vous, dit l'Eternel, sont des pensées de paix, et non pas d'adversité, pour vous donner une fin telle que vous attendez.
\VS{12}Alors vous m'invoquerez pour vous en retourner ; et vous me prierez, et je vous exaucerai.
\VS{13}Vous me chercherez, et vous me trouverez, après que vous m'aurez recherché de tout votre cœur.
\VS{14}Car je me ferai trouver à vous, dit l'Eternel, je ramènerai vos captifs, et je vous rassemblerai d'entre toutes les nations, et de tous les lieux où je vous aurai dispersés, dit l'Eternel, et je vous ferai retourner au lieu dont je vous aurai transportés.
\VS{15}Parce que vous aurez dit : l'Eternel nous a suscité des Prophètes qui ont [prophétisé que nous viendrions] à Babylone.
\VS{16}C'est pourquoi ainsi a dit l'Eternel, touchant le Roi qui est assis sur le trône de David, et touchant tout le peuple qui habite dans cette ville, [c'est-à-dire], touchant vos frères qui ne sont point allés avec vous en captivité ;
\VS{17}Ainsi a dit l'Eternel des armées : voici, je m'en vais envoyer sur eux l'épée, la famine, et la mortalité, et je les ferai devenir comme des figues qui sont étrangement mauvaises, [et] qu'on ne peut manger, tant elles sont mauvaises.
\VS{18}Et je les poursuivrai avec l'épée, par la famine, et par la mortalité ; et je les abandonnerai à être agités par tous les Royaumes de la terre, et pour être en exécration, en étonnement, en raillerie, et en opprobre à toutes les nations parmi lesquelles je les aurai dispersés.
\VS{19}Parce qu'ils n'ont point écouté mes paroles, dit l'Eternel, qui leur ai envoyé mes serviteurs Prophètes, en me levant dès le matin, et les envoyant ; et vous n'avez point écouté, dit l'Eternel.
\VS{20}Vous tous donc qui avez été transportés, [et] que j'ai renvoyés de Jérusalem à Babylone, écoutez la parole de l'Eternel.
\VS{21}Ainsi a dit l'Eternel des armées, le Dieu d'Israël, touchant Achab fils de Kolaja, et touchant Sédécias fils de Mahaséja, qui vous prophétisent faussement en mon Nom : voici, je m'en vais les livrer en la main de Nébucadnetsar Roi de Babylone, et il les frappera devant vos yeux.
\VS{22}Et on prendra d'eux un formulaire de malédiction parmi tous ceux qui ont été transportés de Juda, qui [sont] à Babylone, en disant : l'Eternel te mette en tel état qu'il a mis Sédécias et Achab, lesquels le Roi de Babylone a grillés au feu.
\VS{23}Parce qu'ils ont commis des impuretés en Israël, et qu'ils ont commis adultère avec les femmes de leurs prochains, et qu'ils ont dit en mon Nom des paroles fausses, que je ne leur avais pas commandées, et je le sais, et j'en suis témoin, dit l'Eternel.
\VS{24}Parle aussi à Sémahia Néhélamite, en disant :
\VS{25}Ainsi a dit l'Eternel des armées, le Dieu d'Israël : parce que tu as envoyé en ton Nom des Lettres à tout le peuple qui [est] à Jérusalem, et à Sophonie fils de Mahaséja Sacrificateur, et à tous les Sacrificateurs, en disant :
\VS{26}L'Eternel t'a établi pour Sacrificateur en la place de Jéhojadah le Sacrificateur, afin que vous ayez la charge de la maison de l'Eternel sur tout homme agité par l'esprit, et faisant du Prophète, pour les mettre en prison et aux fers :
\VS{27}Et maintenant, pourquoi n'as-tu pas réprimé Jérémie de Hanathot, qui vous prophétise ?
\VS{28}Car à cause de cela il nous a envoyé dire à Babylone : la [captivité] sera longue : bâtissez des maisons, et [y] demeurez ; plantez des jardins, et mangez-en les fruits.
\VS{29}(Or Sophonie le Sacrificateur avait lu ces Lettres-là, Jérémie le Prophète l'entendant,)
\VS{30}C'est pourquoi la parole de l'Eternel fut [adressée] à Jérémie, en disant :
\VS{31}Mande à tous ceux qui ont été transportés, et leur dis : ainsi a dit l'Eternel touchant Sémahia Néhélamite : parce que Sémahia vous a prophétisé, quoique je ne l'aie point envoyé, et vous a fait confier au mensonge ;
\VS{32}A cause de cela l'Eternel a dit ainsi : voici, je m'en vais punir Sémahia Néhélamite et sa postérité, [et] il n'y aura personne de sa race qui habite parmi ce peuple, et il ne verra point le bien que je m'en vais faire à mon peuple, dit l'Eternel ; parce qu'il a parlé de révolte contre l'Eternel.
\Chap{30}
\VerseOne{}La parole qui fut [adressée] à Jérémie par l'Eternel, en disant :
\VS{2}Ainsi a parlé l'Eternel, le Dieu d'Israël, en disant : écris-toi dans un livre toutes les paroles que je t'ai dites.
\VS{3}Car voici, les jours viennent, dit l'Eternel, que je ramènerai les captifs de mon peuple d'Israël et de Juda, a dit l'Eternel, et je les ferai retourner au pays que j'ai donné à leurs pères, et ils le posséderont.
\VS{4}Et ce sont ici les paroles que l'Eternel a prononcées touchant Israël, et Juda :
\VS{5}Ainsi a donc dit l'Eternel : nous avons ouï un bruit d'épouvantement et de frayeur, et il n'y a point de paix.
\VS{6}Informez-vous, je vous prie, et considérez si un mâle enfante ; pourquoi donc ai-je vu tout homme tenant ses mains sur ses reins comme une femme qui enfante ? et [pourquoi] tous les visages sont-ils jaunes ?
\VS{7}Hélas ! que cette journée-là est grande, il n'y en a point eu de semblable, et elle sera un temps de détresse à Jacob ; [mais] il en sera pourtant délivré.
\VS{8}Et il arrivera en ce jour-là, dit l'Eternel des armées, que je briserai son joug de dessus ton cou, et que je romprai tes liens ; et les étrangers ne t'asserviront plus ;
\VS{9}Mais ils serviront l'Eternel leur Dieu, et David leur Roi, lequel je leur susciterai.
\VS{10}Toi donc, mon serviteur Jacob, ne crains point, dit l'Eternel, et ne t'épouvante point, ô Israël ! car voici, je m'en vais te délivrer du pays éloigné ; et ta postérité du pays de leur captivité ; et Jacob retournera, il sera en repos et à son aise, et il n'y aura personne qui lui fasse peur.
\VS{11}Car je suis avec toi, dit l'Eternel, pour te délivrer ; et même je consumerai entièrement toutes les nations parmi lesquelles je t'aurai dispersé ; mais quant à toi, je ne te consumerai point entièrement, mais je te châtierai par mesure, et je ne te tiendrai pas entièrement pour innocent.
\VS{12}Car ainsi a dit l'Eternel : ta blessure est hors d'espérance, [et] ta plaie est fort douloureuse.
\VS{13}Il n'y a personne qui défende ta cause pour nettoyer [ta plaie] ; il n'y a point pour toi de remède qui y fasse revenir la chair.
\VS{14}Tous tes amoureux t'ont oubliée, ils ne te cherchent point ; car je t'ai frappée d'une plaie d'ennemi, d'un châtiment [d'homme] cruel, à cause de la grandeur de tes iniquités ; tes péchés se sont renforcés.
\VS{15}Pourquoi cries-tu à cause de ta plaie ? ta douleur est hors d'espérance ; je t'ai fait ces choses à cause de la grandeur de ton iniquité, tes péchés se sont renforcés.
\VS{16}Néanmoins tous ceux qui te dévorent, seront dévorés, et tous ceux qui te mettent dans la détresse iront en captivité ; et tous ceux qui te fourragent seront fourragés ; et j'abandonnerai au pillage tous ceux qui te pillent.
\VS{17}Même je consoliderai tes plaies, et te guérirai de tes blessures, dit l'Eternel. Parce qu'ils t'ont appelée la déchassée, [et qu'ils ont dit :] c'est Sion, personne ne la recherche :
\VS{18}Ainsi a dit l'Eternel : voici, je m'en vais ramener les captifs des tentes de Jacob, et j'aurai pitié de ses pavillons ; la ville sera rétablie sur son sol, et le palais sera assis en sa place.
\VS{19}Et il sortira d'eux actions de grâces et voix de gens qui rient, et je les multiplierai, et ils ne seront plus diminués ; et je les agrandirai, et ils ne seront point rendus petits.
\VS{20}Et ses enfants seront comme auparavant, et son assemblée sera affermie devant moi, et je punirai tous ceux qui l'oppriment.
\VS{21}Et celui qui aura autorité sur lui sera de lui, et son dominateur sortira du milieu de lui, je le ferai approcher, et il viendra vers moi ; car qui est celui qui ait disposé son cœur pour venir vers moi ? dit l'Eternel.
\VS{22}Et vous serez mon peuple, et je serai votre Dieu.
\VS{23}Voici, la tempête de l'Eternel, la fureur est sortie, un tourbillon qui s'entasse ; il se posera sur la tête des méchants.
\VS{24}L'ardeur de la colère de l'Eternel ne se détournera point, jusqu’à ce qu'il ait exécuté et mis en effet les desseins de son cœur ; vous entendrez ceci aux derniers jours.
\Chap{31}
\VerseOne{}En ce temps-là, dit l'Eternel, je serai le Dieu de toutes les familles d'Israël, et ils seront mon peuple.
\VS{2}Ainsi a dit l'Eternel : le peuple réchappé de l'épée a trouvé grâce dans le désert ; on va pour faire trouver du repos à Israël.
\VS{3}L'Eternel m'est apparu de loin, [et m'a dit] : je t'ai aimée d'un amour éternel, c'est pourquoi j'ai prolongé envers toi ma gratuité.
\VS{4}Je t'établirai encore, et tu seras établie, ô vierge d'Israël ! Tu te pareras encore de tes tambours, et tu sortiras avec la danse des joueurs.
\VS{5}Tu planteras encore des vignes sur les montagnes de Samarie ; les vignerons [les] planteront, et ils en recueilleront les fruits pour leur usage.
\VS{6}Car il y a un jour auquel les gardes crieront en la montagne d'Ephraïm : levez-vous, et montons en Sion vers l'Eternel notre Dieu.
\VS{7}Car ainsi a dit l'Eternel : réjouissez-vous avec chant de triomphe, et avec allégresse à cause de Jacob, et vous égayez à cause du chef des nations ; faites-le entendre, chantez des louanges, et dites : Eternel, délivre ton peuple, le reste d'Israël.
\VS{8}Voici, je m'en vais les faire venir du pays d'Aquilon, et je les rassemblerai du fond de la terre ; l'aveugle et le boiteux, la femme enceinte et celle qui enfante seront ensemble parmi eux ; une grande assemblée retournera ici.
\VS{9}Ils y seront allés en pleurant, mais je les ferai retourner avec des supplications, et je les conduirai aux torrents d'eaux, et par un droit chemin, auquel ils ne broncheront point ; car j'ai été pour père à Israël, et Ephraïm est mon premier-né.
\VS{10}Nations, écoutez la parole de l'Eternel, et l'annoncez aux Iles éloignées, et dites : celui qui a dispersé Israël le rassemblera, et le gardera comme un berger [garde] son troupeau.
\VS{11}Car l'Eternel a racheté Jacob, et l'a retiré de la main d'un [ennemi] plus fort que lui.
\VS{12}Ils viendront donc, et se réjouiront avec chant de triomphe au [lieu] le plus haut de Sion, et ils accourront aux biens de l'Eternel, au froment, au vin, et à l'huile, et au fruit du gros et du menu bétail ; et leur âme sera comme un jardin plein de fontaines, et ils ne seront plus dans l'ennui.
\VS{13}Alors la vierge se réjouira en la danse, et les jeunes gens et les anciens ensemble, et je tournerai leur deuil en joie, et je les consolerai, et les réjouirai [les délivrant] de leur douleur.
\VS{14}Je rassasierai aussi de graisse l'âme des Sacrificateurs, et mon peuple sera rassasié de mon bien, dit l'Eternel.
\VS{15}Ainsi a dit l'Eternel : une voix très amère de lamentation [et] de pleurs a été ouïe à Rama, Rachel pleurant ses enfants, a refusé d'être consolée touchant ses enfants, de ce qu'il n'y en a plus.
\VS{16}Ainsi a dit l'Eternel : empêche ta voix de lamenter, et tes yeux de verser des larmes, car ton œuvre aura son salaire, dit l'Eternel, et on retournera du pays de l'ennemi.
\VS{17}Et il y a de l'espérance pour tes derniers jours, dit l'Eternel, et tes enfants retourneront en leurs quartiers.
\VS{18}J'ai très bien ouï Ephraïm se plaignant, [et disant] : tu m'as châtié, et j'ai été châtié comme un taureau indompté ; convertis-moi, et je serai converti ; car tu es l'Eternel mon Dieu.
\VS{19}Certes après que j'aurai été converti, je me repentirai ; et après que je me serai reconnu, je frapperai sur ma cuisse. J'ai été honteux et confus, parce que j'ai porté l'opprobre de ma jeunesse.
\VS{20}Ephraïm ne m'a-t-il pas été un cher enfant ? ne m'a-t-il pas été un enfant que j'ai aimé ? car toutes les fois que j'ai parlé de lui, je n'ai pas manqué de m'en souvenir [avec tendresse] : c'est pourquoi mes entrailles se sont émues à cause de lui, et j'aurai certainement pitié de lui, dit l'Eternel.
\VS{21}Dresse-toi des indices sur les chemins, et fais des monceaux de pierres ; prends garde aux chemins, [et] par quelle voie tu es venue. Retourne-t-en, vierge d'Israël, retourne à tes villes.
\VS{22}Jusques à quand seras-tu agitée, fille rebelle ? Car l'Eternel a créé une chose nouvelle sur la terre, la femme environnera l'homme.
\VS{23}Ainsi a dit l'Eternel des armées, le Dieu d'Israël : on dira encore cette parole-ci dans le pays de Juda et dans ses villes, quand j'aurai ramené leurs captifs : l'Eternel te bénisse, ô agréable demeure de la justice, montagne de sainteté.
\VS{24}Et Juda et toutes ses villes ensemble, les laboureurs, et ceux qui marchent avec les troupeaux, habiteront en elle.
\VS{25}Car j'ai enivré l'âme altérée par le travail, et j'ai rempli toute âme qui languissait.
\VS{26}C'est pourquoi je me suis réveillé, et j'ai regardé, et mon sommeil m'a été doux.
\VS{27}Voici, les jours viennent, dit l'Eternel, que j'ensemencerai la maison d'Israël, et la maison de Juda, de semence d'hommes, et de semence de bêtes.
\VS{28}Et il arrivera que comme j'ai veillé sur eux pour arracher et démolir, pour détruire, pour perdre, et pour faire du mal ; ainsi je veillerai sur eux pour bâtir et pour planter, dit l'Eternel.
\VS{29}En ces jours-là on ne dira plus : les pères ont mangé le verjus, et les dents des enfants en sont agacées ;
\VS{30}Mais chacun mourra pour son iniquité ; tout homme qui mangera le verjus, ses dents en seront agacées.
\VS{31}Voici, les jours viennent, dit l'Eternel, que je traiterai une nouvelle alliance avec la maison d'Israël, et avec la maison de Juda.
\VS{32}Non selon l'alliance que je traitai avec leurs pères, au jour que je les pris par la main pour les faire sortir du pays d'Egypte, laquelle alliance ils ont enfreinte ; et toutefois je leur avais été pour mari, dit l'Eternel.
\VS{33}Car c'est ici l'alliance que je traiterai avec la maison d'Israël après ces jours-là, dit l'Eternel : Je mettrai ma Loi au dedans d'eux, je l'écrirai dans leur cœur ; et je serai leur Dieu, et ils seront mon peuple.
\VS{34}Chacun d'eux n'enseignera plus son prochain, ni chacun son frère, en disant : connaissez l'Eternel ; car ils me connaîtront tous, depuis le plus petit d'entre eux jusques au plus grand, dit l'Eternel ; parce que je pardonnerai leur iniquité, et que je ne me souviendrai plus de leur péché.
\VS{35}Ainsi a dit l'Eternel, qui donne le soleil pour être la lumière du jour, et le règlement de la lune et des étoiles pour être la lumière de la nuit ; qui fend la mer, et les flots en bruient ; duquel le Nom est l'Eternel des armées ;
\VS{36}Si jamais ces règlements disparaissent de devant moi, dit l'Eternel, aussi la race d'Israël cessera d'être à jamais une nation devant moi.
\VS{37}Ainsi a dit l'Eternel : si les cieux se peuvent mesurer par dessus, et les fondements de la terre sonder par dessous, aussi rejetterai-je toute la race d'Israël ; à cause de toutes les choses qu'ils ont faites, dit l'Eternel.
\VS{38}Voici, les jours viennent, dit l'Eternel, que cette ville sera rebâtie à l'Eternel, depuis la tour d'Hananéel, jusqu'à la porte du coin.
\VS{39}Et encore le cordeau à mesurer sera tiré vis-à-vis d'elle sur la colline de Gareb, et fera le tour vers Goha.
\VS{40}Et toute la vallée de la voirie et des cendres, et tout le quartier jusqu’au torrent de Cédron, jusqu’au coin de la porte des chevaux vers l'Orient, sera une sainteté à l'Eternel, et ne sera plus démoli ni détruit à jamais.
\Chap{32}
\VerseOne{}La parole qui fut [adressée] de par l'Eternel à Jérémie, la dixième année de Sédécias Roi de Juda, qui est l'an dix-huitième de Nébucadnetsar.
\VS{2}(Or l'armée du Roi de Babylone assiégeait alors Jérusalem, et Jérémie le Prophète était enfermé dans la cour de la prison, qui était dans la maison du Roi de Juda ;
\VS{3}Car Sédécias Roi de Juda l'avait enfermé, et lui avait dit : pourquoi prophétises-tu ? en disant : ainsi a dit l'Eternel : voici, je m'en vais livrer cette ville en la main du Roi de Babylone, et il la prendra.
\VS{4}Et Sédécias Roi de Juda n'échappera point de la main des Caldéens, mais il sera certainement livré en la main du Roi de Babylone, et lui parlera bouche à bouche, et ses yeux verront les yeux de ce Roi.
\VS{5}Et il emmènera Sédécias à Babylone, qui y demeurera jusqu’à ce que je le visite, dit l'Eternel ; si vous combattez contre les Caldéens, vous ne prospérerez point ;)
\VS{6}Jérémie donc dit : la parole de l'Eternel m'a été [adressée], en disant :
\VS{7}Voici Hanaméel fils de Sallum ton oncle, qui vient vers toi, pour te dire : achète-toi mon champ, qui est à Hanathoth ; car tu as le droit de retrait lignager pour le racheter.
\VS{8}Hanaméel donc, fils de mon oncle, vint à moi, selon la parole de l'Eternel, dans la cour de la prison, et me dit : achète, je te prie, mon champ qui [est] à Hanathoth, dans le territoire de Benjamin : car tu as le droit de possession héréditaire, et de retrait lignager, achète-le [donc] pour toi ; et je connus alors que c'était la parole de l'Eternel.
\VS{9}Ainsi j'achetai le champ de Hanaméel, fils de mon oncle, qui [est] à Hanathoth ; et je lui pesai l'argent, [qui fut] dix-sept sicles d'argent.
\VS{10}Puis j'en écrivis le contrat, et le cachetai, et je pris des témoins après avoir pesé l'argent dans la balance ;
\VS{11}Et je pris le contrat d'acquisition, tant celui qui était cacheté, selon l'ordonnance et les statuts, que celui qui était ouvert.
\VS{12}Et je donnai le contrat d'acquisition à Baruc fils de Nérija, fils de Mahaséja, en présence d'Hanaméel fils de mon oncle, et des témoins qui s'étaient souscrits dans le contrat de l'acquisition, [et] en présence de tous les Juifs qui étaient assis dans la cour de la prison.
\VS{13}Puis je commandai en leur présence à Baruc, en lui disant :
\VS{14}Ainsi a dit l'Eternel des armées, le Dieu d'Israël : prends ces contrats ici, savoir ce contrat d'acquisition, qui est cacheté, et ce contrat qui est ouvert, et mets-les dans un pot de terre, afin qu'ils puissent se conserver longtemps.
\VS{15}Car ainsi a dit l'Eternel des armées, le Dieu d'Israël : on achètera encore des maisons, des champs, et des vignes en ce pays.
\VS{16}Et après que j'eus donné à Baruc fils de Nerija le contrat d'acquisition, je fis requête à l'Eternel, en disant :
\VS{17}Ah ! ah ! Seigneur Eternel, voici, tu as fait le ciel et la terre par ta grande puissance, et par ton bras étendu ; aucune chose ne te sera difficile ;
\VS{18}Tu fais miséricorde jusqu'en mille générations, et tu rends l'iniquité des pères dans le sein de leurs enfants après eux ; tu es le [Dieu] Fort, le Grand, le Puissant, le Nom duquel est l'Eternel des armées ;
\VS{19}Grand en conseil, et abondant en moyens ; car tes yeux sont ouverts sur toutes les voies des enfants des hommes, pour rendre à chacun selon ses voies, et selon le fruit de ses œuvres.
\VS{20}Tu as fait au pays d'Egypte des signes et des miracles [qui sont connus] jusques à ce jour, et dans Israël, et parmi les hommes, et tu t'es acquis un nom tel qu'[il paraît] aujourd'hui.
\VS{21}Car tu as retiré Israël ton peuple du pays d'Egypte, avec des signes et des miracles, et avec une main forte, et avec un bras étendu, et en répandant partout la frayeur.
\VS{22}Et tu leur as donné ce pays que tu avais juré à leurs pères de leur donner, qui est un pays découlant de lait et de miel ;
\VS{23}Et ils [y] sont entrés et l'ont possédé ; mais ils n'ont point obéi à ta voix, et n'ont point marché en ta Loi, [et] n'ont pas fait toutes les choses que tu leur avais commandé de faire ; c'est pourquoi tu as fait que tout ce mal ici les a rencontrés.
\VS{24}Voilà, les terrasses [sont élevées], on est venu contre la ville pour la prendre, et à cause de l'épée, de la famine, et de la mortalité, la ville est livrée en la main des Caldéens qui combattent contre elle ; et ce que tu as dit est arrivé, et voici, tu le vois.
\VS{25}Et cependant, Seigneur Eternel ! tu m'as dit : achète-toi ce champ à prix d'argent, et prends-en des témoins, quoique la ville soit livrée en la main des Caldéens.
\VS{26}Mais la parole de l'Eternel fut [adressée] à Jérémie, en disant :
\VS{27}Voici, je suis l'Eternel, le Dieu de toute chair ; y aura-t-il quelque chose qui me soit difficile ?
\VS{28}C'est pourquoi ainsi a dit l'Eternel : voici, je m'en vais livrer cette ville entre les mains des Caldéens, et entre les mains de Nébucadnetsar Roi de Babylone, qui la prendra.
\VS{29}Et les Caldéens qui combattent contre cette ville, y entreront, [et] mettront le feu à cette ville, et la brûleront, avec les maisons sur les toits desquelles on a fait des parfums à Bahal, et [où] l'on a fait des aspersions à d'autres dieux pour m'irriter.
\VS{30}Car les enfants d'Israël et les enfants de Juda n'ont fait dès leur jeunesse que ce qui déplaît à mes yeux ; et les enfants d'Israël ne font que m'irriter par les œuvres de leurs mains, dit l'Eternel.
\VS{31}Car cette ville a été portée à provoquer ma colère et ma fureur, depuis le jour qu'ils l'ont bâtie, jusqu’à aujourd'hui, afin que je l'abolisse de devant ma face ;
\VS{32}A cause de toute la malice des enfants d'Israël, et des enfants de Juda, qu'ils ont commise pour m'irriter, eux, leurs Rois, les principaux d'entre eux, leurs Sacrificateurs, et leurs Prophètes, les hommes de Juda, et les habitants de Jérusalem.
\VS{33}Ils m'ont abandonné ; et quand je les ai enseignés, me levant dès le matin et les enseignant, ils n'ont point été obéissants pour recevoir instruction.
\VS{34}Mais ils ont mis leurs abominations dans la maison sur laquelle mon Nom est réclamé, pour la souiller.
\VS{35}Et ils ont bâti les hauts lieux de Bahal, qui sont en la vallée du fils de Hinnom, pour faire passer par le feu leurs fils et leurs filles à Molec ; ce que je ne leur avais point commandé, et je n'avais jamais pensé qu'ils fissent cette abomination pour faire pécher Juda.
\VS{36}Et maintenant, à cause de cela l'Eternel, le Dieu d'Israël, dit ainsi touchant cette ville de laquelle vous dites qu'elle est livrée entre les mains du Roi de Babylone, à cause que l'épée, la famine, et la mortalité [sont en elle] :
\VS{37}Voici, je m'en vais les rassembler de tous les pays dans lesquels je les aurai dispersés par ma colère et par ma fureur, et par ma grande indignation, et je les ferai retourner en ce lieu-ci, et je les [y] ferai demeurer en sûreté.
\VS{38}Et ils me seront pour peuple, et je leur serai pour Dieu.
\VS{39}Et je leur donnerai un même cœur, et un même chemin, afin qu'ils me craignent à toujours, pour leur bien et le bien de leurs enfants après eux.
\VS{40}Et je traiterai avec eux une alliance éternelle, [savoir] que je ne me retirerai point d'eux pour leur faire du bien ; et je mettrai ma crainte dans leur cœur, afin qu'ils ne se retirent point de moi.
\VS{41}Et je prendrai plaisir à leur faire du bien, et je les planterai dans ce pays-ci solidement, de tout mon cœur, et de toute mon âme.
\VS{42}Car ainsi a dit l'Eternel : comme j'ai fait venir tout ce grand mal sur ce peuple, ainsi je m'en vais faire venir sur eux tout le bien que je prononce en leur faveur.
\VS{43}Et on achètera des champs dans ce pays, duquel vous dites que ce n'est que désolation, n'y ayant ni homme ni bête, [et qui] est livré entre les mains des Caldéens.
\VS{44}On achètera, [dis-je], des champs à prix d'argent, et on en écrira les contrats, et on les cachettera, et on en prendra des témoins au pays de Benjamin, et aux environs de Jérusalem, dans les villes de Juda, tant dans les villes des montagnes, que dans les villes de la plaine, et dans les villes du Midi. Car je ferai retourner leurs captifs, dit l'Eternel.
\Chap{33}
\VerseOne{}Et la parole de l'Eternel fut [adressée] une seconde fois à Jérémie, quand il était encore enfermé dans la cour de la prison, en disant :
\VS{2}Ainsi a dit l'Eternel qui s'en va faire ceci, l'Eternel qui s'en va le former pour l'établir, le Nom duquel est l'Eternel.
\VS{3}Crie vers moi, je te répondrai, et je te déclarerai des choses grandes et cachées, lesquelles tu ne sais point.
\VS{4}Car ainsi a dit l'Eternel, le Dieu d'Israël, touchant les maisons de cette ville-ci, et les maisons des Rois de Juda ; elles s'en vont être démolies par le moyen des terrasses, et par l'épée.
\VS{5}Ils sont venus à combattre contre les Caldéens, mais ç'a été pour remplir leurs maisons des corps morts des hommes que j'ai fait frapper en ma colère et en ma fureur, et parce que j'ai caché ma face de cette ville à cause de toute leur malice.
\VS{6}Voici, je m'en vais lui donner la santé et la guérison, je les guérirai, et je leur ferai voir abondance de paix et de vérité.
\VS{7}Et je ferai retourner les captifs de Juda, et les captifs d'Israël, et je les rétablirai comme auparavant.
\VS{8}Et je les purifierai de toute leur iniquité, par laquelle ils ont péché contre moi ; et je pardonnerai toutes leurs iniquités par lesquelles ils ont péché contre moi, et par lesquelles ils ont péché grièvement contre moi.
\VS{9}Et [cette ville] me sera un sujet de réjouissance, de louange et de gloire, chez toutes les nations de la terre qui entendront parler de tout le bien que je vais leur faire, et elles seront effrayées et épouvantées à cause de tout le bien, et de toute la prospérité que je vais lui donner.
\VS{10}Ainsi a dit l'Eternel : dans ce lieu-ci duquel vous dites : il est désert, n'y ayant ni homme ni bête, dans les villes de Juda, et dans les rues de Jérusalem, qui sont désolées, n'y ayant ni homme, ni habitant, ni aucune bête, on y entendra encore,
\VS{11}La voix de joie, et la voix d'allégresse, la voix de l'époux, et la voix de l'épouse, [et] la voix de ceux qui disent : célébrez l'Eternel des armées ; car l'Eternel est bon, parce que sa miséricorde demeure à toujours, lorsqu'ils apportent des oblations d'action de grâces à la maison de l'Eternel ; car je ferai retourner les captifs de ce pays, [et je les mettrai] au même état qu'auparavant, a dit l'Eternel.
\VS{12}Ainsi a dit l'Eternel des armées : en ce lieu désert, où il n'y a ni homme ni bête, et dans toutes ses villes, il y aura encore des cabanes de bergers qui y feront reposer leurs troupeaux ;
\VS{13}Dans les villes des montagnes, et dans les villes de la plaine, dans les villes du Midi, dans le pays de Benjamin, dans les environs de Jérusalem, et dans les villes de Juda ; et les troupeaux passeront encore sous les mains de celui qui les compte, a dit l'Eternel.
\VS{14}Voici, les jours viennent, dit l'Eternel, que je mettrai en effet la bonne parole que j'ai prononcée touchant la maison d'Israël, et la maison de Juda.
\VS{15}En ces jours-là, et en ce temps-là je ferai germer à David le Germe de justice, qui exercera le jugement et la justice en la terre.
\VS{16}En ces jours-là Juda sera délivré, et Jérusalem habitera en assurance, et c'est ici le nom dont elle sera appelée : l'Eternel notre justice.
\VS{17}Car ainsi a dit l'Eternel, il ne manquera jamais à David d'homme assis sur le trône de la maison d'Israël ;
\VS{18}Et d'entre les Sacrificateurs Lévites, il ne manquera jamais d'y avoir devant moi d'homme offrant des holocaustes, faisant les parfums de gâteau, et faisant des sacrifices tous les jours.
\VS{19}Davantage la parole de l'Eternel fut [adressée] à Jérémie, en disant :
\VS{20}Ainsi a dit l'Eternel : Si vous pouvez abolir mon alliance touchant le jour, et mon alliance touchant la nuit, tellement que le jour et la nuit ne soient plus en leur temps ;
\VS{21}Aussi mon alliance avec David mon serviteur sera abolie ; tellement qu'il n'ait plus de fils régnant sur son trône ; et avec les Lévites Sacrificateurs, faisant mon service.
\VS{22}Car [comme] on ne peut compter l'armée des cieux, ni mesurer le sable de la mer, ainsi je multiplierai la postérité de David mon serviteur, et les Lévites qui font mon service.
\VS{23}La parole de l'Eternel fut encore [adressée] à Jérémie, en disant :
\VS{24}N'as-tu pas vu ce que ce peuple a prononcé, disant : l'Eternel a rejeté les deux familles qu'il avait élues ; car [par là] ils méprisent mon peuple ; tellement qu'à leur compte il ne sera plus une nation ?
\VS{25}Ainsi a dit l'Eternel : si [je n'ai point établi] mon alliance touchant le jour et la nuit, et si je n'ai point établi les ordonnances des cieux et de la terre ;
\VS{26}Aussi rejetterai-je la postérité de Jacob, et celle de David mon serviteur, pour ne prendre plus de sa postérité des gens qui dominent sur la postérité d'Abraham, d'Isaac, et de Jacob ; car je ferai retourner leurs captifs, et j'aurai compassion d'eux.
\Chap{34}
\VerseOne{}La parole qui fut [adressée] par l'Eternel à Jérémie, lorsque Nébucadnetsar Roi de Babylone, et toute son armée, tous les Royaumes de la terre, et tous les peuples qui étaient sous la puissance de sa main combattaient contre Jérusalem, et contre toutes ses villes, disant :
\VS{2}Ainsi a dit l'Eternel, le Dieu d'Israël : va, et parle à Sédécias Roi de Juda, et lui dis : ainsi a dit l'Eternel, voici, je m'en vais livrer cette ville en la main du Roi de Babylone, et il la brûlera au feu ;
\VS{3}Et tu n'échapperas point de sa main ; car certainement tu seras pris ; et tu seras livré entre ses mains, et tes yeux verront les yeux du Roi de Babylone ; et il parlera à toi bouche à bouche, et tu viendras dans Babylone.
\VS{4}Toutefois, ô Sédécias Roi de Juda, écoute la parole de l'Eternel : l'Eternel a parlé ainsi de toi : tu ne mourras point par l'épée.
\VS{5}Mais tu mourras en paix, et on fera brûler sur toi des choses aromatiques, comme on en a brûlé sur tes pères, les Rois précédents qui ont été devant toi ; et on te plaindra, [en disant] : hélas, Seigneur ! car j'ai prononcé cette parole, dit l'Eternel.
\VS{6}Jérémie donc le Prophète prononça toutes ces paroles-là à Sédécias Roi de Juda, dans Jérusalem.
\VS{7}Et l'armée du Roi de Babylone combattait contre Jérusalem, et contre toutes les villes de Juda qui étaient demeurées de reste, [savoir] contre Lakis, et contre Hazéka, car c'étaient [les seules] villes fortes qui restaient entre les villes de Juda.
\VS{8}La parole qui fut [adressée] par l'Eternel à Jérémie, après que le Roi Sédécias eut traité alliance avec tout le peuple qui était à Jérusalem, pour leur publier la liberté ;
\VS{9}Afin que chacun renvoyât libre son serviteur, et chacun sa servante, Hébreu ou Hébreue, et qu'aucun Juif ne fût l'esclave de son frère.
\VS{10}Tous les principaux donc, et tout le peuple qui étaient entrés dans cette alliance, entendirent que chacun devait renvoyer libre son serviteur, et chacun sa servante, sans plus les asservir ; et ils obéirent, et les renvoyèrent.
\VS{11}Mais ensuite ils changèrent d'avis, et firent revenir leurs serviteurs et leurs servantes, qu'ils avaient renvoyés libres, et les assujettirent pour leur être serviteurs et servantes.
\VS{12}Et la parole de l'Eternel fut [adressée] à Jérémie par l'Eternel, en disant :
\VS{13}Ainsi a dit l'Eternel, le Dieu d'Israël : je traitai alliance avec vos pères, le jour que je les tirai hors du pays d'Egypte, de la maison de servitude, en disant :
\VS{14}Dans la septième année vous renverrez chacun votre frère Hébreu, qui vous aura été vendu ; il te servira six ans, puis tu le renverras libre d'avec toi ; mais vos pères ne m'ont point écouté, et n'ont point incliné leur oreille.
\VS{15}Et vous vous étiez convertis aujourd'hui, et vous aviez fait ce qui [était] juste devant moi, en publiant la liberté chacun à son prochain, et vous aviez traité alliance en ma présence, dans la maison sur laquelle mon Nom est réclamé ;
\VS{16}Mais vous avez changé d'avis, et avez souillé mon Nom ; car vous avez fait revenir chacun son serviteur, et chacun sa servante, que vous aviez renvoyés libres pour être à eux-mêmes, et vous les avez assujettis, afin qu'ils vous soient serviteurs et servantes.
\VS{17}C'est pourquoi ainsi a dit l'Eternel : vous ne m'avez point écouté pour publier la liberté chacun à son frère, et chacun à son prochain ; voici, je m'en vais publier, dit l'Eternel, la liberté contre vous à l'épée, à la peste, et à la famine ; et je vous livrerai pour être transportés par tous les Royaumes de la terre.
\VS{18}Et je livrerai les hommes qui ont transgressé mon alliance, et qui n'ont point effectué les paroles de l'alliance qu'ils ont traitée devant moi, lorsqu'ils sont passés entre les deux moitiés du veau qu'ils ont coupé en deux ;
\VS{19}Les principaux de Juda, et les principaux de Jérusalem, les Eunuques, et les Sacrificateurs et tout le peuple du pays, qui ont passé entre les deux moitiés du veau ;
\VS{20}Je les livrerai, dis-je, entre les mains de leurs ennemis, et entre les mains de ceux qui cherchent leur vie ; et leurs corps morts seront pour viande aux oiseaux des cieux, et aux bêtes de la terre.
\VS{21}Je livrerai aussi Sédécias Roi de Juda, et les principaux de sa Cour entre les mains de leurs ennemis, et entre les mains de ceux qui cherchent leur vie ; savoir entre les mains de l'armée du Roi de Babylone, qui s'est retiré de devant vous.
\VS{22}Voici, je m'en vais leur donner ordre, dit l'Eternel, et je les ferai retourner vers cette ville-ci, et ils combattront contre elle, et la prendront, et la brûleront au feu, et je mettrai les villes de Juda en désolation, tellement qu'il n'y aura personne qui y habite.
\Chap{35}
\VerseOne{}C'est ici la parole qui fut [adressée] par l'Eternel à Jérémie, aux jours de Jéhojakim, fils de Josias Roi de Juda, en disant :
\VS{2}Va à la maison des Récabites, et leur parle, et les fais venir en la maison de l'Eternel, dans l'une des chambres, et présente-leur du vin à boire.
\VS{3}Je pris donc Jaazanja fils de Jérémie, fils de Habatsinja, et ses frères, et tous ses fils, et toute la maison des Récabites ;
\VS{4}Et je les fis venir dans la maison de l'Eternel, en la chambre des fils de Hanan, fils de Jigdalia, homme de Dieu, laquelle était près de la chambre des principaux ; qui était sur la chambre de Mahaséja, fils de Sallum, garde des vaisseaux.
\VS{5}Et je mis devant les enfants de la maison des Récabites, des gobelets pleins de vin, et des tasses, et je leur dis : buvez du vin.
\VS{6}Et ils répondirent : nous ne boirons point de vin ; car Jéhonadab, fils de Récab notre père nous a donné un commandement, en disant : vous ne boirez point de vin, ni vous, ni vos enfants, à jamais ;
\VS{7}Vous ne bâtirez aucune maison, vous ne sèmerez aucune semence, vous ne planterez aucune vigne, et vous n'en aurez point ; mais vous habiterez en des tentes tous les jours de votre vie, afin que vous viviez longtemps sur la terre dans laquelle vous séjournez comme étrangers.
\VS{8}Nous avons donc obéi à la voix de Jéhonadab, fils de Récab notre père, dans toutes les choses qu'il nous a commandées, de sorte que nous n'avons point bu de vin tous les jours de notre vie, ni nous, [ni] nos femmes, ni nos fils, ni nos filles.
\VS{9}Nous n'avons bâti aucune maison pour notre demeure, et nous n'avons eu ni vigne, ni champ, ni semence.
\VS{10}Mais nous avons demeuré dans des tentes, et nous avons obéi, et avons fait selon toutes les choses que Jéhonadab notre père nous a commandées.
\VS{11}Mais il est arrivé que quand Nébucadnetsar Roi de Babylone est monté au pays, nous avons dit : venez, et entrons dans Jérusalem pour fuir de devant l'armée des Caldéens, et de devant l'armée de Syrie ; et nous sommes demeurés dans Jérusalem.
\VS{12}Alors la parole de l'Eternel fut [adressée] à Jérémie, en disant :
\VS{13}Ainsi a dit l'Eternel des armées, le Dieu d'Israël : va, et dis aux hommes de Juda, et aux habitants de Jérusalem : ne recevrez-vous point d'instruction pour obéir à mes paroles ? dit l'Eternel.
\VS{14}Toutes les paroles de Jéhonadab, fils de Récab, qu'il a commandées à ses enfants de ne boire point de vin, ont été observées, et ils n'[en] ont point bu jusques à ce jour ; mais ils ont obéi au commandement de leur père ; mais moi je vous ai parlé, me levant dès le matin, et parlant, et vous ne m'avez point obéi.
\VS{15}Car je vous ai envoyé tous les Prophètes, mes serviteurs, me levant dès le matin, et les envoyant, pour vous dire : détournez-vous maintenant chacun de son mauvais train, et corrigez vos actions, et ne suivez point d'autres dieux pour les servir, afin que vous demeuriez en la terre que j'ai donnée à vous et à vos pères ; mais vous n'avez point incliné vos oreilles, et ne m'avez point écouté.
\VS{16}Parce que les enfants de Jéhonadab fils de Récab ont observé le commandement de leur père, lequel il leur avait fait, et que ce peuple ne m'a point écouté ;
\VS{17}A cause de cela l'Eternel le Dieu des armées, le Dieu d'Israël, dit ainsi : voici, je m'en vais faire venir sur Juda et sur tous les habitants de Jérusalem tout le mal que j'ai prononcé contre eux ; parce que je leur ai parlé, et ils n'ont point écouté ; et que je les ai appelés, et ils n'ont point répondu.
\VS{18}Et Jérémie dit à la maison des Récabites : ainsi a dit l'Eternel des armées, le Dieu d'Israël : parce que vous avez obéi au commandement de Jéhonadab votre père, et que vous avez gardé tous ses commandements, et avez fait selon tout ce qu'il vous a commandé ;
\VS{19}C'est pourquoi ainsi a dit l'Eternel des armées, le Dieu d'Israël : il n'arrivera jamais qu'il n'y ait quelqu'un appartenant à Jéhonadab fils de Récab, qui assiste devant moi tous les jours.
\Chap{36}
\VerseOne{}Or il arriva, en la quatrième année de Jéhojakim fils de Josias Roi de Juda, que cette parole fut [adressée] par l'Eternel à Jérémie, en disant :
\VS{2}Prends-toi un rouleau de livre, et y écris toutes les paroles que je t'ai dites contre Israël, et contre Juda, et contre toutes les nations, depuis le jour que je t'ai parlé, [c'est-à-dire], depuis les jours de Josias, jusqu’à aujourd'hui.
\VS{3}Peut-être que la maison de Juda fera attention à tout le mal que je pense de leur faire, afin que chacun se détourne de sa mauvaise voie, et que je leur pardonne leur iniquité, et leur péché.
\VS{4}Jérémie donc appela Baruc, fils de Nérija, et Baruc écrivit de la bouche de Jérémie dans le rouleau de livre toutes les paroles de l'Eternel, lesquelles il lui dicta.
\VS{5}Puis Jérémie donna charge à Baruc, en disant : je suis retenu, [et] je ne puis entrer dans la maison de l'Eternel.
\VS{6}Tu y entreras donc, et tu liras dans le rouleau que tu as écrit, [et que je t'ai dicté] de ma bouche, les paroles de l'Eternel, le peuple l'entendant, en la maison de l'Eternel, au jour du jeûne ; tu les liras, dis-je, tous ceux de Juda, qui seront venus de leurs villes, l'entendant.
\VS{7}Peut-être que leur supplication sera reçue devant l'Eternel, et que chacun se détournera de sa mauvaise voie ; car la colère et la fureur que l'Eternel a déclarée contre ce peuple [est] grande.
\VS{8}Baruc donc fils de Nérija fit selon tout ce que Jérémie le Prophète lui avait commandé, lisant dans le livre les paroles de l'Eternel, en la maison de l'Eternel.
\VS{9}Or il arriva en la cinquième année de Jéhojakim fils de Josias Roi de Juda, au neuvième mois, qu'on publia le jeûne en la présence de l'Eternel à tout le peuple de Jérusalem et à tout le peuple qui était venu des villes de Juda à Jérusalem.
\VS{10}Et Baruc lut dans le livre les paroles de Jérémie, en la maison de l'Eternel, dans la chambre de Guémaria fils de Saphan, Secrétaire, dans le haut parvis, à l'entrée de la porte neuve de la maison de l'Eternel, tout le peuple l'entendant.
\VS{11}Et quand Michée fils de Guémaria, fils de Saphan, eut ouï [par la lecture] du livre toutes les paroles de l'Eternel ;
\VS{12}Il descendit en la maison du Roi vers la chambre du Secrétaire, et voici tous les principaux y étaient assis, [savoir] Elisamah le Secrétaire, et Délaia fils de Sémahia, Elnathan fils de Hacbor, et Guémaria fils de Saphan, et Sédécias fils de Hanania, et tous les principaux.
\VS{13}Et Michée leur rapporta toutes les paroles qu'il avait ouïes quand Baruc lisait au livre le peuple l'entendant.
\VS{14}C'est pourquoi tous les principaux envoyèrent vers Baruc, Jéhudi, fils de Néthania, fils de Sélemja, fils de Cusci, pour lui dire : prends en ta main le rouleau dans lequel tu as lu, le peuple l'entendant, et viens ici. Baruch donc fils de Nérija prit le rouleau en sa main, et vint vers eux.
\VS{15}Et ils lui dirent : assieds-toi maintenant, et y lis, nous l'entendant ; et Baruc lut, eux l'écoutant.
\VS{16}Et il arriva que sitôt qu'ils eurent ouï toutes ces paroles, ils furent effrayés entre eux, et dirent à Baruc : nous ne manquerons point de rapporter au Roi toutes ces paroles.
\VS{17}Et ils interrogèrent Baruc, en disant : déclare-nous maintenant comment tu as écrit toutes ces paroles-là de sa bouche.
\VS{18}Et Baruc leur dit : il me dictait de sa bouche toutes ces paroles, et je les écrivais avec de l'encre dans le livre.
\VS{19}Alors les principaux dirent à Baruc : va, et te cache, toi et Jérémie, et que personne ne sache où vous serez.
\VS{20}Puis ils s'en allèrent vers le Roi au parvis, mais ils mirent en garde le rouleau dans la chambre d'Elisamah le Secrétaire ; et ils racontèrent toutes ces paroles, le Roi l'entendant.
\VS{21}Et le Roi envoya Jéhudi pour prendre le rouleau ; et quand Jéhudi l'eut pris de la chambre d'Elisamah le Secrétaire, il le lut, le Roi et tous les principaux qui assistaient autour de lui l'entendant.
\VS{22}Or le Roi était assis dans l'appartement d'hiver, au neuvième mois, et [il y avait] devant lui un brasier ardent.
\VS{23}Et il arriva qu'aussitôt que Jéhudi en eut lu trois ou quatre pages, le Roi le coupa avec le canif du Secrétaire, et le jeta au feu du brasier, jusqu'à ce que tout le rouleau fût consumé au feu qui [était] dans le brasier.
\VS{24}Et ni le Roi ni tous ses serviteurs qui entendirent toutes ces paroles n'en furent point effrayés, et ne déchirèrent point leurs vêtements.
\VS{25}Toutefois Elnathan, et Délaia, et Guémaria intercédèrent envers le Roi, afin qu'il ne brûlât point le rouleau, mais il ne les écouta point.
\VS{26}Même le Roi commanda à Jérahméel fils de Hammélec, et à Séraja fils de Hazriel, et à Sélemja fils de Habdéel, de saisir Baruc le secrétaire, et Jérémie le Prophète ; mais l'Eternel les cacha.
\VS{27}Et la parole de l'Eternel fut [adressée] à Jérémie, après que le Roi eut brûlé le rouleau, et les paroles que Baruc avait écrites de la bouche de Jérémie, en disant :
\VS{28}Prends encore un autre rouleau, et y écris toutes les premières paroles qui étaient dans le premier rouleau que Jéhojakim Roi de Juda a brûlé.
\VS{29}Et tu diras à Jéhojakim Roi de Juda : ainsi a dit l'Eternel : tu as brûlé ce rouleau, et tu as dit : pourquoi y as-tu écrit, en disant que le Roi de Babylone viendra certainement, et qu'il ravagera ce pays, et en exterminera les hommes et les bêtes ?
\VS{30}C'est pourquoi ainsi a dit l'Eternel touchant Jéhojakim Roi de Juda : il n'aura personne qui soit assis sur le trône de David, et son corps mort sera jeté de jour à la chaleur, et de nuit à la gelée.
\VS{31}Je visiterai donc sur lui, et sur sa postérité, et sur ses serviteurs, leur iniquité ; et je ferai venir sur eux, et sur les habitants de Jérusalem, et sur les hommes de Juda, tout le mal que je leur ai prononcé, et qu'ils n'ont point écouté.
\VS{32}Jérémie donc prit un autre rouleau, et le donna à Baruc fils de Nérija Secrétaire, lequel y écrivit de la bouche de Jérémie toutes les paroles du livre que Jéhojakim Roi de Juda avait brûlé au feu, et plusieurs paroles semblables y furent encore ajoutées.
\Chap{37}
\VerseOne{}Or le Roi Sédécias, fils de Josias, régna en la place de Chonja fils de Jéhojakim, et il fut établi pour Roi sur le pays de Juda par Nébucadnetsar Roi de Babylone.
\VS{2}Mais il n'obéit point, ni lui, ni ses serviteurs, ni le peuple du pays, aux paroles de l'Eternel, qu'il avait prononcées par le moyen de Jérémie le Prophète.
\VS{3}Toutefois le Roi Sédécias envoya Jéhucal fils de Sélemja, et Sophonie fils de Mahaséja Sacrificateur, vers Jérémie le Prophète, pour lui dire : fais, je te prie, requête pour nous à l'Eternel notre Dieu ;
\VS{4}(Car Jérémie allait et venait parmi le peuple, parce qu'on ne l'avait pas encore mis en prison.)
\VS{5}Alors l'armée de Pharaon sortit d'Egypte, et quand les Caldéens, qui assiégeaient Jérusalem, en ouïrent les nouvelles, ils se retirèrent de devant Jérusalem.
\VS{6}Et la parole de l'Eternel fut [adressée] à Jérémie le Prophète, en disant :
\VS{7}Ainsi a dit l'Eternel, le Dieu d'Israël : vous direz ainsi au Roi de Juda, qui vous a envoyés pour m'interroger : voici, l'armée de Pharaon, qui est sortie à votre secours, s'en va retourner en son pays d'Egypte.
\VS{8}Et les Caldéens reviendront, et combattront contre cette ville, et la prendront, et la brûleront au feu.
\VS{9}Ainsi a dit l'Eternel : ne vous abusez point vous-mêmes, en disant : les Caldéens se retireront certainement de nous ; car ils ne s'en retireront point.
\VS{10}Même quand vous auriez battu toute l'armée des Caldéens qui combattent contre vous, et qu'il n'y aurait de reste entre eux que des gens percés de blessures, ils se relèveront pourtant chacun dans sa tente, et brûleront cette ville au feu.
\VS{11}Or il arriva que quand l'armée des Caldéens se fut retirée de devant Jérusalem, à cause de l'armée de Pharaon ;
\VS{12}Jérémie sortit de Jérusalem pour s'en aller au pays de Benjamin, se glissant hors de là à travers le peuple.
\VS{13}Mais quand il fut à la porte de Benjamin, il y avait là un capitaine de la garde, duquel le nom était Jireija, fils de Sélemja, fils de Hanania, qui saisit Jérémie le Prophète, en [lui] disant : Tu te vas rendre aux Caldéens.
\VS{14}Et Jérémie répondit : cela n'est point ; je ne vais point me rendre aux Caldéens ; mais il ne l'écouta point, et Jireija prit Jérémie, et l'amena vers les principaux.
\VS{15}Et les principaux se mirent en colère contre Jérémie, et le battirent, et le mirent en prison dans la maison de Jéhonathan le Secrétaire, car ils en avaient fait un lieu de prison.
\VS{16}Et ainsi Jérémie entra dans la fosse, et dans les cachots ; et Jérémie y demeura plusieurs jours.
\VS{17}Mais le Roi Sédécias y envoya, et l'en tira, et il l'interrogea en secret dans sa maison, et lui dit : y a-t-il quelque parole de par l'Eternel ? Et Jérémie répondit : il y en a ; et lui dit : tu seras livré entre les mains du Roi de Babylone.
\VS{18}Puis Jérémie dit au Roi Sédécias : quelle faute ai-je commise contre toi, et envers tes serviteurs, et envers ce peuple, pour m'avoir mis en prison ?
\VS{19}Mais où sont vos Prophètes qui vous prophétisaient, en disant : le Roi de Babylone ne reviendra point contre vous, ni contre ce pays ?
\VS{20}Or écoute maintenant, je te prie, ô Roi mon Seigneur ! et que maintenant ma supplication soit reçue devant ta face, et ne me renvoie point dans la maison de Jéhonathan le Secrétaire, de peur que je n'y meure.
\VS{21}C'est pourquoi le Roi Sédécias commanda qu'on gardât Jérémie dans la cour de la prison, et qu'on lui donnât tous les jours un pain de la place des boulangers, jusqu’à ce que tout le pain de la ville fût consumé. Ainsi Jérémie demeura dans la cour de la prison.
\Chap{38}
\VerseOne{}Mais Séphatia fils de Mattan, et Guédalia fils de Pashur, et Jucal fils de Sélemja, et Pashur fils de Malkija, entendirent les paroles que Jérémie prononçait à tout le peuple, en disant :
\VS{2}Ainsi a dit l'Eternel : celui qui demeurera dans cette ville mourra par l'épée, par la famine, ou par la mortalité, mais celui qui sortira vers les Caldéens vivra, et son âme lui sera pour butin, et il vivra.
\VS{3}Ainsi a dit l'Eternel : cette ville sera livrée certainement à l'armée du Roi de Babylone, et il la prendra.
\VS{4}Et les principaux dirent au Roi : qu'on fasse mourir cet homme ; car par ce moyen il rend lâches les mains des hommes de guerre qui sont demeurés de reste dans cette ville, et les mains de tout le peuple, en leur disant de telles paroles ; parce que cet homme ne cherche point la prospérité de ce peuple, mais le mal.
\VS{5}Et le Roi Sédécias dit : voici, il est entre vos mains ; car le Roi ne peut rien par dessus vous.
\VS{6}Ils prirent donc Jérémie, et le jetèrent dans la fosse de Malkija, fils de Hammélec, laquelle était dans la cour de la prison, et ils descendirent Jérémie avec des cordes dans cette fosse où il n'y avait point d'eau, mais de la boue ; et ainsi Jérémie enfonça dans la boue.
\VS{7}Mais Hebed-mélec Cusien, Eunuque, qui était dans la maison du Roi, apprit qu'ils avaient mis Jérémie dans cette fosse ; et le Roi était assis à la porte de Benjamin.
\VS{8}Et Hebed-mélec sortit de la maison du Roi, et parla au Roi, en disant :
\VS{9}Ô Roi mon Seigneur ! ces hommes-là ont mal fait dans tout ce qu'ils ont fait contre Jérémie le Prophète, en le jetant dans la fosse, car il serait déjà mort de faim dans le lieu où il était, parce qu'il n'[y a] plus de pain dans la ville.
\VS{10}C'est pourquoi le Roi commanda à Hebed-mélec Cusien, en disant : prends d'ici trente hommes sous ta conduite, et fais remonter hors de la fosse Jérémie le Prophète, avant qu'il meure.
\VS{11}Hebed-mélec donc prit ces hommes sous sa conduite, et entra dans la maison du Roi au lieu qui est sous la Trésorerie, d'où il prit de vieux lambeaux et de vieux haillons, et les descendit avec des cordes à Jérémie dans la fosse ;
\VS{12}Et Hebed-mélec Cusien dit à Jérémie : mets ces vieux lambeaux et ces haillons sous les aisselles de tes bras, au dessous des cordes ; et Jérémie fit ainsi.
\VS{13}Ainsi ils tirèrent Jérémie dehors avec les cordes, et le firent remonter hors de la fosse ; et Jérémie demeura dans la cour de la prison.
\VS{14}Et le Roi Sédécias envoya, et fit amener vers lui Jérémie le Prophète à la troisième entrée qui était dans la maison de l'Eternel. Et le Roi dit à Jérémie : je vais te demander une chose, ne m'en cèle rien.
\VS{15}Et Jérémie répondit à Sédécias : quand je te l'aurai déclarée, n'est-il pas vrai que tu me feras mourir ? et quand je t'aurai donné conseil, tu ne m'écouteras point.
\VS{16}Alors le Roi Sédécias jura secrètement à Jérémie, en disant : l'Eternel [est] vivant, qui nous a fait cette âme-ci, que je ne te ferai point mourir, et que je ne te livrerai point entre les mains de ces gens-là qui cherchent ta vie.
\VS{17}Alors Jérémie dit à Sédécias : ainsi a dit l'Eternel, le Dieu des armées, le Dieu d'Israël : si tu sors volontairement pour aller vers les principaux du Roi de Babylone, ta vie te sera conservée, et cette ville ne sera point brûlée au feu, et tu vivras toi et ta maison.
\VS{18}Mais si tu ne sors pas vers les principaux du Roi de Babylone, cette ville sera livrée entre les mains des Caldéens, qui la brûleront au feu ; et tu n'échapperas point de leurs mains.
\VS{19}Et le Roi Sédécias dit à Jérémie : j'appréhende à cause des Juifs qui se sont rendus aux Caldéens, qu'on ne me livre entre leurs mains, et qu'ils ne se moquent de moi.
\VS{20}Et Jérémie lui répondit : on ne te livrera point à eux ; je te prie, écoute la voix de l'Eternel dans ce que je te dis, afin que tu t'en trouves bien, et que tu vives.
\VS{21}Que si tu refuses de sortir, voici ce que l'Eternel m'a fait voir,
\VS{22}C'est que, voici, toutes les femmes qui sont demeurées de reste dans la maison du Roi de Juda, seront menées dehors aux principaux du Roi de Babylone, et elles diront que ceux qui ne te prédisaient que paix, t'ont incité, et t'ont gagné ; tellement que tes pieds sont enfoncés dans la boue, s'étant reculés en arrière.
\VS{23}Ils s'en vont donc mener dehors aux Caldéens toutes tes femmes et tes enfants, et tu n'échapperas point de leurs mains, mais tu seras pris, pour être livré entre les mains du Roi de Babylone, et tu seras cause que cette ville sera brûlée au feu.
\VS{24}Alors Sédécias dit à Jérémie : que personne ne sache rien de ces paroles, et tu ne mourras point.
\VS{25}Et si les principaux entendent que je t'ai parlé, et qu'ils viennent vers toi, et te disent : déclare-nous maintenant ce que tu as dit au Roi, et ce que le Roi a dit, ne nous en cèle rien, et nous ne te ferons point mourir ;
\VS{26}Tu leur diras : j'ai présenté ma supplication devant le Roi, afin qu'il ne me fît point ramener dans la maison de Jéhonathan pour y mourir.
\VS{27}Tous les principaux donc vinrent vers Jérémie, et l'interrogèrent ; mais il leur fit un rapport conforme à toutes les paroles que le Roi [lui] avait commandé de dire ; et ils cessèrent de lui parler, car on n'avait rien su de cette affaire.
\VS{28}Ainsi Jérémie demeura dans la cour de la prison, jusqu’au jour que Jérusalem fut prise, et il y était lorsque Jérusalem fut prise.
\Chap{39}
\VerseOne{}La neuvième année de Sédécias Roi de Juda, au dixième mois, Nébucadnetsar Roi de Babylone vint avec toute son armée contre Jérusalem, et ils l'assiégèrent.
\VS{2}Et la onzième année de Sédécias, au quatrième mois, le neuvième jour du mois, il y eut une brèche faite à la ville.
\VS{3}Et tous les principaux [Capitaines] du Roi de Babylone [y entrèrent], et s'assirent à la porte du milieu, [savoir] Nergal-saréetser, Samgar-nebu, Sar-sekim, Rabsaris, Nergal, Saréetser, Rabmag, et tout le reste des principaux [Capitaines] du Roi de Babylone.
\VS{4}Or il arriva qu'aussitôt que Sédécias Roi de Juda, et tous les hommes de guerre les eurent vus, ils s'enfuirent, et sortirent de nuit hors de la ville, par le chemin du jardin du Roi, par la porte [qui était] entre les deux murailles, et ils s'en allaient par le chemin de la campagne.
\VS{5}Mais l'armée des Caldéens les poursuivit, et ils atteignirent Sédécias dans les campagnes de Jérico ; et l'ayant pris, ils l'amenèrent vers Nébucadnetsar Roi de Babylone à Ribla, qui est au pays de Hamath, où on lui fit son procès.
\VS{6}Et le Roi de Babylone fit égorger à Ribla les fils de Sédécias en sa présence ; le Roi de Babylone fit aussi égorger tous les magistrats de Juda.
\VS{7}Puis il fit crever les yeux à Sédécias, et le fit lier de doubles chaînes d'airain, pour l'emmener à Babylone.
\VS{8}Les Caldéens brûlèrent aussi les maisons royales, et les maisons du peuple, et démolirent les murailles de Jérusalem.
\VS{9}Et Nébuzar-adan, prévôt de l'hôtel, transporta à Babylone le reste du peuple qui était demeuré dans la ville, et ceux qui s'étaient allés rendre à lui, le résidu, dis-je, du peuple qui était demeuré de reste.
\VS{10}Mais Nébuzar-adan, prévôt de l'hôtel, laissa d'entre le peuple les plus pauvres qui n'avaient rien dans le pays de Juda, et en ce jour-là il leur donna des vignes et des champs.
\VS{11}Or Nébucadnetsar Roi de Babylone avait donné ordre et commission à Nébuzar-adan prévôt de l'hôtel, touchant Jérémie, en disant :
\VS{12}Retire cet homme-là, et aie les yeux sur lui, et ne lui fais aucun mal ; mais fais pour lui tout ce qu'il te dira.
\VS{13}Nébuzar-adan donc, prévôt de l'hôtel, envoya, et aussi Nébusazban, Rabsaris, Nergal, Saréetser, Rabmag, et tous les principaux [Capitaines] du Roi de Babylone ;
\VS{14}Ils envoyèrent, [dis-je], retirer Jérémie de la cour de la prison, et le donnèrent à Guédalia fils d'Ahikam, fils de Saphan, pour le conduire à la maison ; ainsi il demeura parmi le peuple.
\VS{15}Or la parole de l'Eternel avait été [adressée] à Jérémie, du temps qu'il était enfermé dans la cour de la prison, en disant :
\VS{16}Va, et parle à Hebed-mélec Cusien, et lui dis : ainsi a dit l'Eternel des armées, le Dieu d'Israël : voici, je m'en vais faire venir mes paroles sur cette ville pour son malheur, et non point pour son bien, et elles seront accomplies ce jour-là, en ta présence.
\VS{17}Mais je te délivrerai en ce jour-là, dit l'Eternel, et tu ne seras point livré entre les mains des hommes dont tu as peur.
\VS{18}Car certainement je te délivrerai, tellement que tu ne tomberas point par l'épée ; mais ta vie te sera pour butin, parce que tu as eu confiance en moi, dit l'Eternel.
\Chap{40}
\VerseOne{}La parole qui fut [adressée] par l'Eternel à Jérémie, quand Nébuzar-adan prévôt de l'hôtel l'eut renvoyé de Rama, après l'avoir pris lorsqu'il était lié de chaînes parmi tous ceux qu'on transportait de Jérusalem et de Juda, [et] qu'on menait captifs à Babylone.
\VS{2}Quand donc le prévôt de l'hôtel eut retiré Jérémie, il lui dit : l'Eternel ton Dieu a prononcé ce mal sur ce lieu-ci.
\VS{3}Et l'Eternel l'a fait venir, et a fait ainsi qu'il avait dit ; parce que vous avez péché contre l'Eternel, et que vous n'avez point écouté sa voix, à cause de cela ceci vous est arrivé.
\VS{4}Maintenant donc voici, aujourd'hui je t'ai délié des chaînes que tu avais aux mains, s'il te plaît de venir avec moi à Babylone, viens et je prendrai soin de toi ; mais s'il ne te plaît pas de venir avec moi à Babylone, demeure ; regarde, toute la terre [est] à ta disposition ; va où il te semblera bon et convenable d'aller.
\VS{5}Or [Guédalia] ne retournera plus [ici] ; retourne-t'en donc vers Guédalia fils d'Ahikam, fils de Saphan, que le Roi de Babylone a commis sur les villes de Juda ; et demeure avec lui parmi le peuple, ou va partout où il te plaira d'aller ; et le prévôt de l'hôtel lui donna des vivres et quelques présents, et le renvoya.
\VS{6}Jérémie donc alla vers Guédalia fils d'Ahikam à Mitspa, et demeura avec lui parmi le peuple qui était resté dans le pays.
\VS{7}Et tous les Capitaines des gens de guerre qui étaient à la campagne, eux et leurs gens entendirent que le Roi de Babylone avait commis Guédalia fils d'Ahikam sur le pays, et qu'il lui avait commis les hommes, et les femmes, et les enfants, et cela d'entre les plus pauvres du pays, [savoir] de ceux qui n'avaient pas été transportés à Babylone.
\VS{8}Alors ils allèrent vers Guédalia à Mitspa ; savoir, Ismaël fils de Néthania, et Johanan et Jonathan enfants de Karéah, et Séraja fils de Tanhumet, et les enfants de Héphai Nétophathite, et Jézania fils d'un Mahacathite, eux et leurs gens.
\VS{9}Et Guédalia fils d'Ahikam, fils de Saphan, leur jura, à eux et à leurs gens, en disant : ne faites pas difficulté d'être assujettis aux Caldéens, demeurez dans le pays, et soyez sujets du Roi de Babylone, et vous vous en trouverez bien.
\VS{10}Et pour moi, voici, je demeurerai à Mitspa pour me tenir prêt à recevoir les ordres des Caldéens qui viendront vers nous ; mais vous, recueillez le vin, les fruits d'Eté, et l'huile, et mettez-les dans vos vaisseaux, et demeurez dans vos villes que vous avez prises [pour votre demeure].
\VS{11}Pareillement aussi tous les Juifs qui [étaient] en Moab, et parmi les enfants de Hammon, et dans l'Idumée, et dans tous ces pays-là, quand ils eurent entendu que le Roi de Babylone avait laissé quelque reste à Juda, et qu'il avait commis sur eux Guédalia fils d'Ahikam, fils de Saphan ;
\VS{12}Tous ces Juifs-là retournèrent de tous les lieux auxquels ils avaient été chassés, et vinrent au pays de Juda, vers Guédalia à Mitspa, et recueillirent du vin, et des fruits d'Eté en grande abondance.
\VS{13}Mais Johanan fils de Karéah, et tous les Capitaines des gens de guerre qui [étaient] à la campagne vinrent vers Guédalia à Mitspa ;
\VS{14}Et lui dirent : ne sais-tu pas certainement que Bahalis, Roi des enfants de Hammon, a envoyé Ismaël le fils de Néthania pour t'ôter la vie ? mais Guédalia fils d'Ahikam ne les crut point.
\VS{15}Et Johanan fils de Karéah parla en secret à Guédalia à Mitspa, en disant : je m'en irai maintenant, et je frapperai Ismaël fils de Néthania, sans que personne le sache ; pourquoi t'ôterait-il la vie, afin que tous les Juifs qui se sont rassemblés vers toi soient dissipés, et que les restes de Juda périssent ?
\VS{16}Mais Guédalia, fils d'Ahikam, dit à Johanan fils de Karéah : ne fais point cela ; car tu parles faussement d'Ismaël.
\Chap{41}
\VerseOne{}Or il arriva au septième mois qu'Ismaël fils de Néthania, fils d'Elisamah, de la race royale, et l'un des principaux de chez le Roi, et dix hommes avec lui, vinrent vers Guédalia fils d'Ahikam à Mitspa, et mangèrent là du pain ensemble à Mitspa.
\VS{2}Mais Ismaël fils de Néthania se leva, et les dix hommes qui étaient avec lui, et ils frappèrent avec l'épée Guédalia fils d'Ahikam, fils de Saphan, et on le fit mourir, lui que le Roi de Babylone avait commis sur le pays.
\VS{3}Ismaël frappa aussi tous les Juifs qui étaient avec lui, [c'est-à-dire], avec Guédalia à Mitspa, et les Caldéens, gens de guerre, qui furent trouvés là.
\VS{4}Et il arriva que le jour après qu'on eut fait mourir Guédalia, avant que personne le sût,
\VS{5}Quelques hommes de Sichem, de Silo, et de Samarie ; en tout quatre-vingts hommes, ayant la barbe rasée, les vêtements déchirés, et s'étant fait des incisions, vinrent avec des dons et de l'encens dans leurs mains pour les apporter en la maison de l'Eternel.
\VS{6}Alors Ismaël fils de Néthania sortit de Mitspa au devant d'eux, et il marchait en pleurant, et quand il les eut rencontrés, il leur dit : venez vers Guédalia fils d'Ahikam.
\VS{7}Mais sitôt qu'ils furent venus au milieu de la ville, Ismaël fils de Néthania, accompagné des hommes qui étaient avec lui, les égorgea, [et les jeta] dans une fosse.
\VS{8}Mais il se trouva dix hommes entre eux qui dirent à Ismaël : ne nous fais point mourir, car nous avons dans les champs des amas secrets de froment, d'orge, d'huile, et de miel ; et il les laissa, et ne les fit point mourir avec leurs frères.
\VS{9}Et la fosse dans laquelle Ismaël jeta les corps morts des hommes qu'il tua à l'occasion de Guédalia est celle que le Roi Asa avait fait faire lorsqu'il eut peur de Bahasa Roi d'Israël ; et Ismaël fils de Néthania la remplit de gens tués.
\VS{10}Et Ismaël emmena prisonniers tous ceux du peuple qui étaient demeurés de reste à Mitspa, [savoir] les filles du Roi, et tout le peuple qui était demeuré de reste à Mitspa, que Nébuzar-adan prévôt de l'hôtel avait commis à Guédalia fils d'Ahikam ; Ismaël, dis-je, fils de Néthania, les emmenait prisonniers, et s'en allait pour passer vers les enfants de Hammon.
\VS{11}Mais Johanan fils de Karéah, et tous les Capitaines des gens de guerre qui étaient avec lui, ayant entendu tout le mal qu'Ismaël fils de Néthania avait fait ;
\VS{12}Et ayant pris tous leurs gens ils s'en allèrent pour combattre contre Ismaël fils de Néthania, lequel ils rencontrèrent près des grosses eaux qui sont à Gabaon.
\VS{13}Et il arriva qu'aussitôt que tout le peuple qui était avec Ismaël eut vu Johanan fils de Karéah, et tous les Capitaines des gens de guerre qui [étaient] avec lui, ils s'en réjouirent.
\VS{14}Et tout le peuple qu'Ismaël emmenait prisonnier de Mitspa tourna visage, et se retournant ils s'en allèrent vers Johanan fils de Karéah.
\VS{15}Mais Ismaël fils de Néthania échappa avec huit hommes de devant Johanan, et s'en alla vers les enfants de Hammon.
\VS{16}Et Johanan fils de Karéah, et tous les Capitaines des gens de guerre qui étaient avec lui prirent tout le reste du peuple qu'ils avaient retiré des mains d'Ismaël fils de Néthania, [qu'il emmenait prisonnier] de Mitspa, après qu'il eut frappé Guédalia fils d'Ahikam, [savoir] les vaillants hommes de guerre, et les femmes, et les enfants, et les Eunuques ; et les ramenèrent de Gabaon.
\VS{17}Et ils s'en allèrent et demeurèrent à Guéruth-kimham, auprès de Bethléhem, pour s'en aller et se retirer en Egypte,
\VS{18}A cause des Caldéens ; car ils avaient peur d'eux, parce qu'Ismaël fils de Néthania avait tué Guédalia fils d'Ahikam, qui avait été commis sur le pays par le Roi de Babylone.
\Chap{42}
\VerseOne{}Alors tous les Capitaines des gens de guerre, et Johanan fils de Karéah, et Jézania fils de Hosahia, et tout le peuple, depuis le plus petit jusqu’au plus grand, s'approchèrent,
\VS{2}Et dirent à Jérémie le Prophète : que notre supplication soit reçue de toi, et fais requête à l'Eternel ton Dieu pour nous, [savoir] pour tout ce reste-ci ; car de beaucoup [de monde que nous étions] nous sommes restés peu, comme tes yeux nous voient ;
\VS{3}Et que l'Eternel ton Dieu nous déclare le chemin par lequel nous aurons à marcher, et ce que nous avons à faire.
\VS{4}Et Jérémie le Prophète leur répondit : J'ai entendu [votre demande] ; voici, je m'en vais faire requête à l'Eternel votre Dieu selon vos paroles ; et il arrivera que je vous déclarerai tout ce que l'Eternel vous répondra, et je ne vous en cacherai pas un mot.
\VS{5}Et ils dirent à Jérémie : l'Eternel soit pour témoin véritable et fidèle entre nous, si nous ne faisons selon toutes les paroles, pour lesquelles l'Eternel ton Dieu t'aura envoyé vers nous.
\VS{6}Soit bien, soit mal, nous obéirons à la voix de l'Eternel notre Dieu, vers lequel nous t'envoyons ; afin qu'il nous arrive du bien quand nous aurons obéi à la voix de l'Eternel notre Dieu.
\VS{7}Et il arriva au bout de dix jours que la parole de l'Eternel fut [adressée] à Jérémie.
\VS{8}Et il appela Johanan fils de Karéah, et tous les Capitaines des gens de guerre qui étaient avec lui, et tout le peuple, depuis le plus petit jusqu'au plus grand ;
\VS{9}Et leur dit : Ainsi a dit l'Eternel, le Dieu d'Israël, vers lequel vous m'avez envoyé pour présenter votre supplication devant lui.
\VS{10}Si vous continuez à demeurer dans ce pays, je vous rétablirai, et je ne vous détruirai point ; je vous [y] planterai, et je ne vous arracherai point, car je me suis repenti du mal que je vous ai fait.
\VS{11}N'ayez point peur du Roi de Babylone, duquel vous avez peur ; n'en ayez point peur, dit l'Eternel, car je suis avec vous pour vous sauver, et pour vous délivrer de sa main.
\VS{12}Même je vous ferai obtenir miséricorde, tellement qu'il aura pitié de vous, et vous fera retourner en votre pays.
\VS{13}Que si vous dites : nous ne demeurerons point en ce pays, et nous n'écouterons point la voix de l'Eternel notre Dieu ;
\VS{14}En disant : non ; mais nous irons au pays d'Egypte, afin que nous ne voyions point de guerre, et que nous n'entendions point le son de la trompette, et que nous n'ayons point disette de pain ; et nous demeurerons là.
\VS{15}A cause de cela écoutez maintenant la parole de l'Eternel, [vous] les restes de Juda. Ainsi a dit l'Eternel des armées, le Dieu d'Israël : si vous vous préparez pour aller en Egypte, et que vous y entriez pour y séjourner ;
\VS{16}Il arrivera que l'épée dont vous avez peur vous attrapera là au pays d'Egypte ; et la famine que vous craignez si fort vous suivra en Egypte, tellement que vous y mourrez.
\VS{17}Et il arrivera que tous les hommes qui auront fait les démarches nécessaires pour entrer en Egypte afin d'y séjourner, mourront par l'épée, par la famine, et par la mortalité ; nul d'eux ne restera, ni échappera de devant le mal que je m'en vais faire venir sur eux.
\VS{18}Car ainsi a dit l'Eternel des armées, le Dieu d'Israël : comme ma colère et ma fureur a fondu sur les habitants de Jérusalem, ainsi ma fureur fondra sur vous, quand vous serez entrés en Egypte ; et vous serez en exécration, et en étonnement, et en malédiction, et en opprobre ; et vous ne verrez plus ce lieu-ci.
\VS{19}Vous, les restes de Juda, l'Eternel a parlé contre vous : n'entrez point en Egypte, vous sentirez certainement que je vous en ai sommés aujourd'hui.
\VS{20}Car vous avez usé de fraude contre vous-mêmes quand vous m'avez envoyé vers l'Eternel votre Dieu, en me disant : fais requête pour nous envers l'Eternel notre Dieu, et nous déclare tout ce que l'Eternel notre Dieu te dira, et nous [le] ferons.
\VS{21}Et je vous [l']ai déclaré aujourd'hui ; mais vous n'avez point écouté la voix de l'Eternel votre Dieu, ni rien de tout ce pour quoi il m'a envoyé vers vous.
\VS{22}Maintenant donc sachez certainement que vous mourrez par l'épée, par la famine, et par la mortalité, au lieu auquel vous avez désiré d'entrer, pour y séjourner.
\Chap{43}
\VerseOne{}Or il arriva qu'aussitôt que Jérémie eut achevé de prononcer à tout le peuple toutes les paroles de l'Eternel leur Dieu, pour lesquelles l'Eternel leur Dieu l'avait envoyé vers eux, [savoir] toutes ces choses-là ;
\VS{2}Hazaria fils de Hosahja, et Johanan fils de Karéah, et tous ces hommes fiers, dirent à Jérémie : tu profères des mensonges ; l'Eternel notre Dieu ne t'a pas envoyé nous dire : n'entrez point en Egypte pour y séjourner.
\VS{3}Mais Baruc fils de Nérija t'incite contre nous, afin de nous livrer entre les mains des Caldéens, pour nous faire mourir, et pour nous faire transporter à Babylone.
\VS{4}Ainsi Johanan fils de Karéah, et tous les Capitaines des gens de guerre, et tout le peuple n'écoutèrent point la voix de l'Eternel, pour demeurer au pays de Juda.
\VS{5}Car Johanan fils de Karéah, et tous les Capitaines des gens de guerre prirent tout le résidu de ceux de Juda qui étaient retournés de toutes les nations, parmi lesquelles ils avaient été chassés, pour demeurer dans le pays de Juda ;
\VS{6}Les hommes, et les femmes, et les enfants, et les filles du Roi, et toutes les personnes que Nébuzar-adan prévôt de l'hôtel avait laissées avec Guédalia fils d'Ahikam, fils de Saphan ; [ils prirent] aussi Jérémie le Prophète, et Baruc fils de Nérija ;
\VS{7}Et ils entrèrent au pays d'Egypte, car ils n'obéirent point à la voix de l'Eternel ; et ils vinrent jusqu'à Taphnés.
\VS{8}Alors la parole de l'Eternel fut [adressée] à Jérémie dans Taphnés, en disant :
\VS{9}Prends en ta main de grosses pierres, et les cache dans l'argile, en la tuilerie qui est à l'entrée de la maison de Pharaon à Taphnés, les Juifs le voyant ;
\VS{10}Et leur dis : Ainsi a dit l'Eternel des armées, le Dieu d'Israël : voici, je m'en vais mander et faire venir Nébucadnetsar Roi de Babylone mon serviteur, et je mettrai son trône sur ces pierres que j'ai cachées, et il étendra son pavillon sur elles ;
\VS{11}Et il viendra et frappera le pays d'Egypte. Ceux qui [sont destinés] à la mort, [iront] à la mort ; et ceux qui [sont destinés] à la captivité, [iront] en captivité ; et ceux qui [sont destinés] à l'épée, [seront livrés] à l'épée.
\VS{12}Et j'allumerai le feu dans les maisons des dieux d'Egypte, et [Nébucadnetsar] les brûlera, et il emmènera captifs ceux d'Egypte, et il se parera des richesses du pays d'Egypte, comme le pasteur s'enveloppe de son vêtement, et il en sortira en paix.
\VS{13}Il brisera aussi les statues de la maison du soleil qui est au pays d'Egypte, et brûlera au feu les maisons des dieux d'Egypte.
\Chap{44}
\VerseOne{}La parole qui fut [adressée] à Jérémie touchant tous les Juifs qui demeuraient au pays d'Egypte, et qui habitaient à Migdol, à Taphnés, à Noph, et au pays de Patros, en disant :
\VS{2}Ainsi a dit l'Eternel des armées, le Dieu d'Israël : vous avez vu tout le mal que j'ai fait venir sur Jérusalem, et sur toutes les villes de Juda ; et voici elles [sont] aujourd'hui un désert, et personne n'y demeure ;
\VS{3}A cause des maux qu'ils ont faits pour m'irriter, en allant faire des encensements pour servir d'autres dieux, lesquels ils n'ont point connus, ni eux, ni vous, ni vos pères.
\VS{4}Et je vous ai envoyé tous mes serviteurs les Prophètes, me levant dès le matin, et les envoyant pour vous dire : ne commettez point maintenant cette chose abominable, laquelle je hais.
\VS{5}Mais ils n'ont point écouté, et n'ont point incliné leur oreille pour se détourner de leur malice, afin de ne faire point d'encensements à d'autres dieux.
\VS{6}C'est pourquoi ma fureur et ma colère s'est répandue sur eux et s'est allumée dans les villes de Juda, et dans les rues de Jérusalem, qui sont réduites en désert, [et] en désolation, comme [il paraît] aujourd'hui.
\VS{7}Maintenant donc, ainsi a dit l'Eternel, le Dieu des armées, le Dieu d'Israël : pourquoi faites-vous ce grand mal contre vous-mêmes, pour vous faire retrancher du milieu de Juda, hommes et femmes, petits enfants, et ceux qui tètent, afin qu'on ne vous laisse aucun de reste ?
\VS{8}En m'irritant par les œuvres de vos mains, en faisant des encensements à d'autres dieux dans le pays d'Egypte, auquel vous venez d'entrer pour y séjourner, afin que vous soyez retranchés, et que vous soyez en malédiction et en opprobre parmi toutes les nations de la terre ?
\VS{9}Avez-vous oublié les crimes de vos pères, et les crimes des Rois de Juda, et les crimes de leurs femmes, et vos propres crimes, et les crimes de vos femmes, qu'elles ont commis dans le pays de Juda, et dans les rues de Jérusalem ?
\VS{10}Jusques à ce jour ils n'ont point été affligés, ils n'ont point eu de crainte, et ils n'ont point marché en ma Loi, ni en mes ordonnances, que je vous ai proposées, et à vos pères aussi.
\VS{11}C'est pourquoi ainsi a dit l'Eternel des armées, le Dieu d'Israël : voici, je m'en vais tourner ma face contre vous pour vous nuire, et pour retrancher tout Juda.
\VS{12}Et je prendrai le reste de ceux de Juda qui se sont préparés pour entrer au pays d'Egypte, et y séjourner, et ils seront tous consumés ; ils tomberont dans le pays d'Egypte, ils seront consumés par l'épée, et par la famine, depuis le plus petit jusqu’au plus grand ; ils mourront par l'épée, et par la famine ; et ils seront en exécration, en étonnement, en malédiction, et en opprobre.
\VS{13}Et je punirai ceux qui demeurent au pays d'Egypte, comme j'ai puni Jérusalem, par l'épée, par la famine, et par la mortalité.
\VS{14}Et il n'y aura personne des restes de Juda d'entre ceux qui sont venus pour demeurer là, [c'est-à-dire], au pays d'Egypte, pour retourner au pays de Juda, auquel ils se promettent de retourner pour y demeurer, qui échappe, et qui y reste ; car pas un ne retournera, sinon ceux qui se seront échappés [des autres].
\VS{15}Mais tous ceux qui savaient que leurs femmes faisaient des encensements à d'autres dieux, et toutes les femmes qui étaient là en grande compagnie ; et tout le peuple qui demeurait dans le pays d'Egypte, à Patros, répondirent à Jérémie, en disant :
\VS{16}Quant à la parole que tu nous as dite au Nom de l'Eternel, nous ne t'écouterons point ;
\VS{17}Mais nous ferons assurément tout ce qui est sorti de notre bouche en faisant des encensements à la Reine des cieux, et lui faisant des aspersions, comme nous et nos pères, nos Rois, et les principaux d'entre nous avons fait dans les villes de Juda, et dans les rues de Jérusalem, et nous avons eu alors abondamment de pain, nous avons été à notre aise, et nous n'avons point vu de mal.
\VS{18}Mais depuis le temps que nous avons cessé de faire des encensements à la Reine des cieux, et de lui faire des aspersions, nous avons eu faute de tout, et nous avons été consumés par l'épée et par la famine.
\VS{19}Quand nous faisions des encensements à la Reine des cieux, et quand nous lui faisions des aspersions, lui avons-nous offert à l'insu de nos maris des gâteaux sur lesquels elle était représentée, ou lui avons-nous répandu des aspersions ?
\VS{20}Alors Jérémie parla à tout le peuple, contre les hommes, et contre les femmes, et contre tout le peuple qui avait fait cette réponse, et leur dit :
\VS{21}L'Eternel ne s'est-il pas souvenu des encensements que vous avez faits dans les villes de Juda, et dans les rues de Jérusalem, vous et vos pères, vos Rois et les principaux d'entre vous, et le peuple du pays, et son cœur n'en a-t-il pas été touché ?
\VS{22}En sorte que l'Eternel ne l'a pu supporter davantage, à cause de la malice de vos actions, et à cause des abominations que vous avez commises ; tellement que votre pays a été réduit en désert, et en étonnement, et en malédiction, sans que personne y habite, comme [il paraît] aujourd'hui.
\VS{23}Parce donc que vous avez fait ces encensements, et que vous avez péché contre l'Eternel, et que vous n'avez point écouté la voix de l'Eternel, et n'avez point marché en sa Loi, ni en ses ordonnances, ni en ses témoignages, à cause de cela ce mal vous est arrivé, comme [il paraît] aujourd'hui.
\VS{24}Puis Jérémie dit à tout le peuple, et à toutes les femmes : vous tous ceux de Juda, qui êtes au pays d'Egypte, écoutez la parole de l'Eternel.
\VS{25}Ainsi a parlé l'Eternel des armées le Dieu d'Israël, en disant : c'est vous, et vos femmes qui ont parlé par votre bouche touchant ce que vous avez accompli de vos mains, en disant : certainement nous accomplirons nos vœux que nous avons voués, en faisant des encensements à la Reine des cieux ; et lui faisant des aspersions. Vous avez entièrement accompli vos vœux, et vous les avez effectués très exactement.
\VS{26}C'est pourquoi écoutez la parole de l'Eternel, vous tous ceux de Juda qui demeurez au pays d'Egypte : voici, j'ai juré par mon grand Nom, a dit l'Eternel, que mon Nom ne sera plus réclamé par la bouche d'aucun de Juda, qui dise en tout le pays d'Egypte : le Seigneur l'Eternel est vivant.
\VS{27}Voici, je veille contre eux pour [leur] mal, et non pour [leur] bien ; et tous les hommes de Juda qui sont au pays d'Egypte seront consumés par l'épée, et par la famine, jusqu’à ce qu'il n'y en ait plus aucun.
\VS{28}Et ceux qui seront échappés de l'épée retourneront du pays d'Egypte au pays de Juda en fort petit nombre, et tout le reste de ceux de Juda qui seront entrés au pays d'Egypte pour y séjourner, saura quelle est la parole qui s'accomplira, la mienne, ou la leur.
\VS{29}Et ceci vous sera pour signe, dit l'Eternel, que je vous punirai en ce lieu-ci, afin que vous sachiez que mes paroles seront infailliblement accomplies contre vous en mal.
\VS{30}Ainsi a dit l'Eternel : voici, je m'en vais livrer Pharaon-Hophrah Roi d'Egypte en la main de ses ennemis, et en la main de ceux qui cherchent sa vie, comme j'ai livré Sédécias Roi de Juda en la main de Nébucadnetsar Roi de Babylone son ennemi, et qui cherchait sa vie.
\Chap{45}
\VerseOne{}La parole que Jérémie le Prophète dit à Baruc fils de Nérija, quand il écrivait dans un livre ces paroles-là, de la bouche de Jérémie, en la quatrième année de Jéhojakim fils de Josias Roi de Juda, disant :
\VS{2}Ainsi a dit l'Eternel, le Dieu d'Israël, touchant toi, Baruc.
\VS{3}Tu as dit : malheur à moi ! car l'Eternel a ajouté la tristesse à ma douleur ; je me suis lassé dans mon gémissement, et je n'ai point trouvé de repos.
\VS{4}Tu lui diras ainsi : ainsi a dit l'Eternel : voici, je m'en vais détruire ce que j'ai bâti, et arracher ce que j'ai planté, [savoir] tout ce pays-ci.
\VS{5}Et toi te chercherais-tu des grandeurs ? Ne les cherche point ; car voici, je m'en vais faire venir du mal sur toute chair, dit l'Eternel ; mais je te donnerai ta vie pour butin, dans tous les lieux où tu iras.
\Chap{46}
\VerseOne{}La parole de l'Eternel qui fut [adressée] à Jérémie le Prophète contre les nations.
\VS{2}A l'égard de l'Egypte, contre l'armée de Pharaon-Neco Roi d'Egypte, qui était auprès du fleuve d'Euphrate, à Carkémis, laquelle Nébucadnetsar Roi de Babylone défit en la quatrième année de Jéhojakim, fils de Josias Roi de Juda.
\VS{3}Préparez le bouclier et l'écu, et approchez-vous pour la bataille.
\VS{4}Attelez les chevaux, et [vous] cavaliers, montez ; présentez-vous avec les casques, fourbissez les lances, revêtez les cuirasses.
\VS{5}D'où vient que je vois [ceci] ? Ils sont effrayés ; ils tournent en arrière ; leurs hommes forts ont été défaits, et s'enfuient avec précipitation, sans regarder derrière eux ; la frayeur les environne, dit l'Eternel.
\VS{6}Que l'homme léger à la course ne s'enfuie point, et que le fort ne se sauve point ; ils sont renversés et tombés vers l'Aquilon, auprès du rivage du fleuve d'Euphrate.
\VS{7}Qui est celui-ci qui s'élève comme une rivière, et duquel les eaux sont émues comme les fleuves ?
\VS{8}C'est l'Egypte ; elle s'élève comme une rivière, et [ses] eaux s'émeuvent comme les fleuves ; et elle dit : je m'élèverai, je couvrirai la terre, je détruirai les villes, et ceux qui y habitent.
\VS{9}Montez chevaux, agissez en furieux, [venez] chariots, et que les hommes forts sortent : ceux de Cus et de Put qui manient le bouclier, et les Ludiens qui manient [et] bandent l'arc.
\VS{10}Car c'est ici la journée du Seigneur l'Eternel des armées, journée de vengeance, pour se venger de ses adversaires. L'épée dévorera, et elle sera rassasiée [et] enivrée de leur sang ; car il y a un sacrifice au Seigneur l'Eternel des armées dans le pays de l'Aquilon, auprès du fleuve d'Euphrate.
\VS{11}Monte en Galaad, et prends du baume, vierge fille d'Egypte. En vain emploies-tu remède sur remède ; car il n'y a point de guérison pour toi.
\VS{12}Les nations ont appris ton ignominie, et ton cri a rempli la terre ; car le fort est tombé sur le fort, et ils sont tombés tous deux ensemble.
\VS{13}La parole que l'Eternel prononça à Jérémie le Prophète touchant la venue de Nébucadnetsar Roi de Babylone, pour frapper le pays d'Egypte :
\VS{14}Faites savoir en Egypte, et publiez à Migdol, à Noph, et à Taphnés ; [et] dites : présente-toi, et te tiens prêt ; car l'épée a dévoré ce qui est autour de toi.
\VS{15}Pourquoi chacun de tes vaillants hommes a-t-il été emporté ? il n'a pu tenir ferme, parce que l'Eternel l'a poussé.
\VS{16}Il en a terrassé un grand nombre, et même chacun est tombé sur son compagnon, et ils ont dit : lève-toi, retournons à notre peuple, et au pays de notre naissance, loin de l'épée de l'oppresseur.
\VS{17}Ils ont crié là, Pharaon Roi d'Egypte n'est que bruit ; il a laissé passer le temps assigné.
\VS{18}Je suis vivant, dit le Roi dont le Nom est l'Eternel des armées, que comme Tabor [est] entre les montagnes, et comme Carmel [est] dans la mer, [ainsi] viendra-t-il.
\VS{19}Ô fille habitante de l'Egypte, équipe-toi pour déloger, car Noph sera désolée, et rendue déserte, sans qu'[il y ait] plus d'habitants.
\VS{20}L'Egypte est une très belle génisse ; [mais] la destruction vient, elle vient de l'Aquilon.
\VS{21}Même les gens de guerre qu'elle entretient chez elle à ses gages, sont comme des veaux engraissés, car aussi ont-ils tourné le dos ; ils s'en sont fuis ensemble, ils n'ont point tenu ferme, parce que le jour de leur calamité, le temps de leur punition est venu sur eux.
\VS{22}Elle sifflera comme un serpent, car ils marcheront avec une puissante [armée], et ils viendront contre elle avec des cognées, comme des bûcherons.
\VS{23}Ils couperont sa forêt, dit l'Eternel, quoiqu'on n'en pût compter [les arbres] ; parce que [leur armée sera] en plus grand nombre que les sauterelles, et on ne saurait la compter.
\VS{24}La fille d'Egypte est rendue honteuse, elle est livrée entre les mains du peuple de l'Aquilon.
\VS{25}L'Eternel des armées, le Dieu d'Israël, a dit : voici, je m'en vais punir le grand peuple de No, et Pharaon, et l'Egypte, et ses dieux, et ses Rois, tant Pharaon, que ceux qui se confient en lui.
\VS{26}Et je les livrerai entre les mains de ceux qui cherchent leur vie, entre les mains, dis-je, de Nébucadnetsar Roi de Babylone, et entre les mains de ses serviteurs ; mais après cela elle sera habitée comme aux temps passés, dit l'Eternel.
\VS{27}Et toi Jacob mon serviteur, ne crains point, et ne t'épouvante point, toi Israël ; car voici, je m'en vais te délivrer du pays éloigné ; et ta postérité, du pays de leur captivité ; et Jacob retournera, et sera en repos et à son aise, et il n'y aura personne qui lui fasse peur.
\VS{28}Toi donc, Jacob mon serviteur, ne crains point, dit l'Eternel ; car je suis avec toi ; et même je consumerai entièrement toutes les nations parmi lesquelles je t'aurai chassé ; mais je ne te consumerai point entièrement, et je te châtierai par mesure ; toutefois je ne te tiendrai pas tout à fait pour innocent.
\Chap{47}
\VerseOne{}La parole de l'Eternel, qui fut [adressée] à Jérémie le Prophète contre les Philistins, avant que Pharaon frappât Gaza.
\VS{2}Ainsi a dit l'Eternel : voici des eaux qui montent de l'Aquilon ; elles seront comme un torrent débordé ; elles se déborderont sur la terre, et sur tout ce qui est en elle, sur la ville, et sur ses habitants ; les hommes crieront, et tous les habitants du pays hurleront ;
\VS{3}A cause du bruit éclatant de la corne des pieds de ses puissants chevaux, à cause de l'impétuosité de ses chariots, et à cause du bruit de ses roues ; les pères n'ont pas regardé les enfants, tant ils ont eu les mains lâches.
\VS{4}A cause du jour qui vient pour ravager tous les Philistins, [et] pour retrancher à Tyr et à Sidon quiconque restera pour les secourir ; car l'Eternel s'en va saccager les Philistins, qui sont les restes de l'Ile de Caphtor.
\VS{5}Gaza est devenue chauve ; Askélon ne dit plus mot avec le reste de leur vallée ; jusques à quand feras-tu des incisions sur toi ?
\VS{6}Ha ! épée de l'Eternel, jusques à quand ne te reposeras-tu point ? rentre en ton fourreau, apaise-toi, et te tiens en repos.
\VS{7}Mais comment te reposerais-tu ? car l'Eternel lui a donné charge, il l'a assignée contre Askélon, et contre le rivage de la mer.
\Chap{48}
\VerseOne{}Quant à Moab ; ainsi a dit l'Eternel des armées, le Dieu d'Israël : malheur à Nébo, car elle a été saccagée ! Kiriathajim a été rendue honteuse, et a été prise ; la haute retraite a été rendue honteuse et effrayée.
\VS{2}Moab ne se glorifiera plus de Hesbon ; car on a machiné du mal contre elle, [en disant] : venez, et exterminons-la, [et] qu'elle ne soit plus nation ; toi aussi Madmen tu seras détruite, et l'épée te poursuivra.
\VS{3}Il y a un bruit de crierie de devers Horonajim, pillage et une grande défaite.
\VS{4}Moab est brisé, on a fait ouïr le cri de ses petits enfants.
\VS{5}Pleurs sur pleurs monteront par la montée de Luhith, car on entendra dans la descente de Horonajim ceux qui crieront à cause des plaies que les ennemis leur auront faites.
\VS{6}Fuyez, [dira-t-on], sauvez vos vies ; et vous serez comme de la bruyère dans un désert.
\VS{7}Car parce que tu as eu confiance en tes ouvrages, et en tes trésors, tu seras prise, et Kémos sortira pour être transporté avec ses Sacrificateurs, et ses principaux.
\VS{8}Et celui qui fait le dégât entrera dans toutes les villes, et pas une ville n'échappera ; la vallée périra, et le plat pays sera détruit, suivant ce que l'Eternel a dit ;
\VS{9}Donnez des ailes à Moab ; car certainement il s'envolera, et ses villes seront réduites en désolation, sans qu'il y ait personne qui [y] habite.
\VS{10}Maudit soit celui qui fera l'œuvre de l'Eternel frauduleusement, et maudit soit celui qui gardera son épée de répandre le sang !
\VS{11}Moab a été à son aise depuis sa jeunesse ; il a reposé sur sa lie ; il n'a point été vidé de vaisseau en vaisseau, et n'a point été transporté, c'est pourquoi sa saveur lui est toujours demeurée, et son odeur ne s'est point changée ;
\VS{12}Mais voici, les jours viennent, dit l'Eternel, que je lui enverrai des gens qui l'enlèveront, qui videront ses vaisseaux, et qui mettront ses outres en pièces.
\VS{13}Et Moab sera honteux à cause de Kémos, comme la maison d'Israël est devenue honteuse à cause de Béthel, qui était sa confiance.
\VS{14}Comment dites-vous : nous sommes forts et vaillants dans le combat ?
\VS{15}Moab va être saccagé, et chacune de ses villes s'en va en fumée, et l'élite de ses jeunes gens va descendre pour être égorgée, dit le Roi dont le Nom est l'Eternel des armées.
\VS{16}La calamité de Moab est proche, et sa ruine s'avance à grands pas.
\VS{17}Vous tous qui êtes autour de lui, soyez-en émus à compassion, et [vous] tous qui connaissez son nom, dites : comment a été rompue cette forte verge, et ce sceptre d'honneur ?
\VS{18}Toi qui te tiens chez la fille de Dibon, descends de ta gloire, et t'assieds dans un lieu de sécheresse ; car celui qui a saccagé Moab est monté contre toi, [et] a détruit tes forteresses.
\VS{19}Habitante d'Haroher, tiens-toi sur le chemin, et contemple ; interroge celui qui s'enfuit, et celle qui est échappée, [et] dis : qu'est-il arrivé ?
\VS{20}Moab est rendu honteux ; car il a été mis en pièces ; hurlez et criez, rapportez dans Arnon que Moab a été saccagé ;
\VS{21}Et que le jugement est venu sur le plat pays, sur Holon, et sur Jathsa, et sur Mephahat,
\VS{22}Et sur Dibon, et sur Nebo, et sur Bethdiblathajim,
\VS{23}Et sur Kiriathajim, et sur Beth-gamul, et sur Beth-méhon,
\VS{24}Et sur Kérijoth, et sur Botsra, et sur toutes les villes du pays de Moab éloignées et proches.
\VS{25}La force de Moab a été rompue, et son bras a été cassé, dit l'Eternel.
\VS{26}Enivrez-le, car il s'est élevé contre l'Eternel. Moab se vautrera dans le vin qu'il aura rendu et il deviendra aussi un sujet de moquerie.
\VS{27}Car, [ô Moab !] Israël ne t'a-t-il pas été en dérision, [comme] un homme qui aurait été surpris entre les larrons ? chaque fois que tu as parlé de lui, tu en as tressailli de joie.
\VS{28}Habitants de Moab quittez les villes, et demeurez dans les rochers, et soyez comme le pigeon qui fait son nid aux côtés de l'entrée des cavernes.
\VS{29}Nous avons appris l'orgueil de Moab le très-superbe, son arrogance et son orgueil, et sa fierté, et son cœur altier.
\VS{30}J'ai connu sa fureur, dit l'Eternel ; mais il n'en sera pas ainsi ; [j'ai connu] ceux sur lesquels il s'appuie ; ils n'ont rien fait de droit.
\VS{31}Je hurlerai donc à cause de Moab, même je crierai à cause de Moab tout entier ; on gémira sur ceux de Kir-hérès.
\VS{32}Ô vignoble de Sibmah je pleurerai sur toi du pleur de Jahzer ; tes provins ont passé au delà de la mer, ils ont atteint jusqu’à la mer de Jahzer ; celui qui fait le dégât s'est jeté sur tes fruits d'Eté, et sur ta vendange.
\VS{33}L'allégresse aussi, et la gaieté s'est retirée loin du champ fertile, et du pays de Moab, et j'ai fait cesser le vin des cuves ; on n'y foulera plus en chantant, et la chanson de la vendange n'[y] sera plus chantée.
\VS{34}A cause du cri de Hesbon qui est parvenu jusqu’à Elhalé, ils ont jeté leurs cris jusqu’à Jahats ; même depuis Tsohar jusqu'à Horonajim, [comme] une génisse de trois ans ; car aussi les eaux de Nimrim seront réduites en désolation.
\VS{35}Et je ferai qu'il n'y aura plus en Moab, dit l'Eternel, aucun qui offre sur les hauts lieux, ni aucun qui fasse des encensements à ses dieux.
\VS{36}C'est pourquoi mon cœur mènera un bruit sur Moab comme des flûtes ; mon cœur mènera un bruit comme des flûtes sur ceux de Kir-hérès, parce que toute l'abondance de ce qu'il a acquis est périe.
\VS{37}Car toute tête sera chauve, et toute barbe sera rasée ; et il y aura des incisions sur toutes les mains, et le sac sera sur les reins.
\VS{38}Il y aura des lamentations sur tous les toits de Moab, et dans ses places, parce que j'aurai brisé Moab comme un vaisseau auquel on ne prend nul plaisir, dit l'Eternel.
\VS{39}Hurlez, [en disant] : comment a-t-il été mis en pièces ? Comment Moab a-t-il tourné le dos tout honteux ? car Moab sera un objet de moquerie et de frayeur à tous ceux qui sont autour de lui.
\VS{40}Car ainsi a dit l'Eternel : voici il volera comme un aigle, et il étendra ses ailes sur Moab.
\VS{41}Kérijoth a été prise, et on s'est saisi des forteresses, et le cœur des hommes forts de Moab sera en ce jour-là comme le cœur d'une femme qui est en travail.
\VS{42}Et Moab sera exterminé, tellement qu'il ne sera plus peuple, parce qu'il s'est élevé contre l'Eternel.
\VS{43}Habitant de Moab, la frayeur, la fosse, et le filet sont sur toi, dit l'Eternel.
\VS{44}Celui qui s'enfuira à cause de la frayeur, tombera dans la fosse ; et celui qui remontera de la fosse, sera pris au filet ; car je ferai venir sur lui, [c'est-à-savoir] sur Moab, l'année de leur punition, dit l'Eternel.
\VS{45}Ils se sont arrêtés en l'ombre de Hesbon, voulant éviter la force ; mais le feu est sorti de Hesbon, et la flamme du milieu de Sihon, qui dévorera un canton de Moab, et le sommet de la tête des gens bruyants.
\VS{46}Malheur à toi, Moab ! le peuple de Kémos est perdu ; car tes fils ont été enlevés pour être emmenés captifs, et tes filles pour être emmenées captives.
\VS{47}Toutefois je ramènerai et mettrai en repos les captifs de Moab, aux derniers jours, dit l'Eternel. Jusqu'ici est le jugement de Moab.
\Chap{49}
\VerseOne{}Quant aux enfants de Hammon, ainsi a dit l'Eternel : Israël n'a-t-il point d'enfants, ou n'a-t-il point d'héritier ? pourquoi donc Malcam a-t-il hérité [du pays] de Gad, et pourquoi son peuple demeure-t-il dans les villes [de Gad] ?
\VS{2}C'est pourquoi voici, les jours viennent, dit l'Eternel, que je ferai ouïr l'alarme dans Rabba des enfants de Hammon, et elle sera réduite en un monceau de ruines, et les villes de son ressort seront brûlées au feu, et Israël possédera ceux qui l'auront possédé, a dit l'Eternel.
\VS{3}Hurle, ô Hesbon ! car Haï a été ravagée. Villes du ressort de Rabba, criez ; ceignez le sac sur vous, lamentez, courez le long des haies ; car Malcam ira en captivité avec ses Sacrificateurs et ses principaux.
\VS{4}Pourquoi te glorifies-tu de tes vallées ? ta vallée est écoulée, fille revêche. Elle se confiait en ses trésors, et [disait] : qui viendra contre moi ?
\VS{5}Voici, je m'en vais faire venir de tous tes environs la frayeur sur toi, dit le Seigneur l'Eternel des armées, et vous serez chassés chacun çà et là, et il n'y aura personne qui rassemble les dispersés.
\VS{6}Mais après cela je ferai retourner les captifs des enfants de Hammon, dit l'Etemel.
\VS{7}Quant à Edom, ainsi a dit l'Eternel des armées : n'est-il pas vrai qu'[il] n'[y a] plus de sagesse dans Téman ? le conseil a manqué à ses habitants, leur sagesse s'est évanouie.
\VS{8}Fuyez, tournez [le dos], vous habitants de Dédan, qui avez fait des creux pour y habiter ; car j'ai fait venir sur Esaü sa calamité, le temps auquel je l'ai visité.
\VS{9}Est-il entré chez toi des vendangeurs ? ils ne t'auraient point laissé de grappillage. Sont-ce des larrons de nuit ? ils auraient fait du dégât autant qu'il leur aurait suffi.
\VS{10}Mais j'ai fouillé Esaü, j'ai découvert ses lieux secrets, tellement qu'il ne se pourra cacher ; sa postérité est ravagée, ses frères aussi, et ses voisins ; et ce n'est plus rien.
\VS{11}Laisse tes orphelins, et je leur donnerai de quoi vivre, et que tes veuves s'assurent sur moi.
\VS{12}Car ainsi a dit l'Eternel : voici, ceux qui ne devaient point boire de la coupe, en boiront certainement ; et toi en serais-tu exempt en quelque manière ? tu n'en seras point exempt ; mais tu en boiras certainement.
\VS{13}Car j'ai juré par moi-même, dit l'Eternel, que Botsra sera réduite en désolation, en opprobre, en désert, et en malédiction, et que toutes ses villes seront réduites en déserts perpétuels.
\VS{14}J'ai ouï une publication de par l'Eternel, et il y a un ambassadeur envoyé parmi les nations, [pour leur dire] : assemblez-vous, et venez contre elle, et levez-vous pour combattre.
\VS{15}Car voici, je t'avais fait petit entre les nations, et contemptible entre les hommes.
\VS{16}Mais ta présomption, [et] la fierté de ton cœur t'ont séduit, toi qui habites dans les creux des rochers, et qui occupes la hauteur des coteaux. Quand tu aurais élevé ton nid comme l'aigle, je t'en ferai descendre, dit l'Eternel.
\VS{17}Et l'Idumée sera réduite en désolation, tellement que quiconque passera près d'elle en sera étonné, et lui insultera à cause de toutes ses plaies.
\VS{18}Il n'y demeurera personne, a dit l'Eternel, et aucun fils d'homme n'y séjournera, comme dans la subversion de Sodome et de Gomorrhe, et de leurs lieux circonvoisins.
\VS{19}Voici, il montera comme un lion à cause de l'enflure du Jourdain, vers la demeure du pays rude, et après l'avoir fait reposer je le ferai courir hors de l'Idumée. Et qui [est] d'élite, que je lui donne commission contre elle ? car qui est semblable à moi ? et qui me déterminera le temps ? et qui sera le Pasteur qui tiendra ferme contre moi ?
\VS{20}C'est pourquoi écoutez la résolution que l'Eternel a prise contre Edom, et les desseins qu'il a formés contre les habitants de Téman ; si les plus petits du troupeau ne les traînent par terre, et si l'on ne réduit en désolation leurs cabanes sur eux.
\VS{21}La terre a été ébranlée du bruit de leur ruine ; il y a eu un cri, le son en a été ouï en la mer Rouge.
\VS{22}Voici, il montera comme un aigle, et il volera, et étendra ses ailes sur Botsra ; et le cœur des forts d'Edom en ce jour-là sera comme le cœur d'une femme qui est en travail.
\VS{23}Quant à Damas ; Hamath et Arpad ont été rendues honteuses, parce qu'elles ont appris des nouvelles très mauvaises, ils sont fondus, il y a une tourmente en la mer, elle ne se peut apaiser.
\VS{24}Damas est toute lâche, elle est mise en fuite, la peur l'a surprise, l'angoisse et les douleurs l'ont saisie comme d'une femme qui enfante.
\VS{25}Comment n'a été réservée la ville renommée, ma ville de plaisance ?
\VS{26}Car ses gens d'élite tomberont dans ses rues, et on fera perdre la parole à tous ses hommes de guerre en ce jour-là, dit l'Eternel des armées.
\VS{27}Et je mettrai le feu à la muraille de Damas, qui dévorera les palais de Ben-hadad.
\VS{28}Quant à Kédar, et aux Royaumes de Hatsor, lesquels Nébucanetsar Roi de Babylone frappera, ainsi a dit l'Eternel : levez-vous, montez vers Kédar, et détruisez les enfants d'Orient.
\VS{29}Ils enlèveront leurs tentes et leurs troupeaux, et prendront pour eux leurs tentes, et tout leur équipage, et leurs chameaux, et on criera : frayeur tout autour.
\VS{30}Fuyez, écartez-vous tant que vous pourrez, vous habitants de Hatsor, qui avez fait des creux pour y habiter, dit l'Eternel ; car Nébucadnetsar Roi de Babylone a formé un dessein contre vous, il a pris une résolution contre vous.
\VS{31}Levez-vous, montez vers la nation qui est en repos, qui habite en assurance, dit l'Eternel ; qui n'ont ni portes, ni barres, et qui habitent seuls.
\VS{32}Et leurs chameaux seront au pillage, et la multitude de leur bétail sera en proie ; et je les disperserai à tout vent, vers ceux qui se coupent l'extrémité des cheveux, et je ferai venir de tous les côtés leur calamité, dit l'Eternel.
\VS{33}Et Hatsor deviendra un repaire de dragons, et un désert à toujours ; il n'y demeurera personne, et aucun fils d'homme n'y séjournera.
\VS{34}La parole de l'Eternel qui fut [adressée] à Jérémie le Prophète, contre Hélam, au commencement du règne de Sédécias Roi de Juda, en disant :
\VS{35}Ainsi a dit l'Eternel des armées : voici, je m'en vais rompre l'arc d'Hélam, qui est leur principale force.
\VS{36}Et je ferai venir contre Hélam les quatre vents, des quatre bouts des cieux ; et je les disperserai par tous ces vents-là ; et il n'y aura point de nation chez laquelle ne viennent ceux qui seront chassés d'Hélam.
\VS{37}Et je ferai que ceux d'Hélam seront épouvantés devant leurs ennemis, et devant ceux qui cherchent leur vie ; et je ferai venir du mal sur eux, l'ardeur de ma colère, dit l'Eternel ; et j'enverrai l'épée après eux, jusqu’à ce que je les aie consumés.
\VS{38}Et je mettrai mon trône en Hélam, et j'en détruirai les Rois et les principaux, dit l'Eternel.
\VS{39}Mais il arrivera qu'aux derniers jours je ferai retourner d'Hélam les captifs, dit l'Eternel.
\Chap{50}
\VerseOne{}La parole que l'Eternel prononça contre Babylone, [et] contre le pays des Caldéens, par le moyen de Jérémie le Prophète.
\VS{2}Faites savoir parmi les nations, et publiez-le, et levez l'enseigne ; publiez-le, ne le cachez point ; dites : Babylone a été prise ; Bel est rendu honteux ; Mérodac est brisé, ses idoles sont rendues honteuses, et leurs dieux de fiente sont brisés.
\VS{3}Car une nation est montée contre elle de devers l'Aquilon, qui mettra son pays en désolation, et il n'y aura personne qui y habite ; les hommes et les bêtes s'en sont fuis, ils s'en sont allés.
\VS{4}En ces jours-là, et en ce temps-là, dit l'Eternel, les enfants d'Israël viendront, eux et les enfants de Juda ensemble ; ils marcheront allant et pleurant, et cherchant l'Eternel leur Dieu.
\VS{5}Ceux de Sion s'enquerront du chemin vers lequel [ils devront dresser] leurs faces, [et ils diront] : venez, et vous joignez à l'Eternel. Il y a une alliance éternelle, elle ne sera jamais mise en oubli.
\VS{6}Mon peuple a été comme des brebis perdues ; leurs pasteurs les ont fait égarer, et les ont fait errer par les montagnes ; ils sont allés de montagne en colline, et ils ont mis en oubli leur gîte.
\VS{7}Tous ceux qui les ont trouvées les ont mangées, et leurs ennemis ont dit : nous ne serons coupables d'aucun mal, parce qu'ils ont péché contre l'Eternel, contre le séjour de la justice ; et l'Eternel a été l'attente de leurs pères.
\VS{8}Fuyez hors de Babylone, et sortez du pays des Caldéens, et soyez comme les boucs qui vont devant le troupeau.
\VS{9}Car voici, je m'en vais susciter et faire venir contre Babylone une assemblée de grandes nations du pays de l'Aquilon, qui se rangeront en bataille contre elle, de sorte qu'elle sera prise. Leurs flèches seront comme celles d'un homme puissant, qui ne fait que détruire, et qui ne retourne point à vide.
\VS{10}Et la Caldée sera abandonnée au pillage, et tous ceux qui la pilleront seront assouvis, dit l'Eternel.
\VS{11}Parce que vous vous êtes réjouis, parce que vous vous êtes égayés, en ravageant mon héritage, parce que vous vous êtes engraissés comme une génisse qui est à l'herbe, et que vous avez henni comme de puissants chevaux.
\VS{12}Votre mère est devenue fort honteuse, et celle qui vous a enfantés a rougi ; voici, elle sera toute la dernière entre les nations, elle sera un désert, un pays sec, une lande.
\VS{13}Elle ne sera plus habitée à cause de l'indignation de l'Eternel, elle ne sera tout entière que désolation ; quiconque passera près de Babylone sera étonné, et lui insultera à cause de toutes ses plaies.
\VS{14}Rangez-vous en bataille contre Babylone, mettez-vous tout alentour ; vous tous qui tendez l'arc, tirez contre elle, et n'épargnez point les traits ; car elle a péché contre l'Eternel.
\VS{15}Jetez des cris de joie contre elle tout alentour ; elle a tendu sa main ; ses fondements sont tombés, ses murailles sont renversées ; car c'est ici la vengeance de l'Eternel ; vengez-vous d'elle ; faites-lui comme elle a fait.
\VS{16}Retranchez de Babylone le semeur, et celui qui tient la faucille au temps de la moisson ; que chacun s'en retourne vers son peuple, et que chacun s'enfuie vers son pays, à cause de l'épée de l'oppresseur.
\VS{17}Israël est comme une brebis égarée que les lions ont effarouchée. Le Roi d'Assur l'a dévorée le premier, mais ce dernier-ci, Nébucadnetsar Roi de Babylone, lui a brisé les os.
\VS{18}C'est pourquoi ainsi a dit l'Eternel des armées, le Dieu d'Israël : voici, je m'en vais visiter le Roi de Babylone et son pays, comme j'ai visité le Roi d'Assyrie.
\VS{19}Et je ferai retourner Israël en ses cabanes ; il paîtra en Carmel et en Basan, et son âme sera rassasiée en la montagne d'Ephraïm, et de Galaad.
\VS{20}En ces jours-là, et en ce temps-là, dit l'Eternel, on cherchera l'iniquité d'Israël, mais il n'y en aura point ; et les péchés de Juda, mais ils ne seront point trouvés ; car je pardonnerai à ceux que j'aurai fait demeurer de reste.
\VS{21}[Venez] contre ce pays-là, vous [deux] rebelles ; monte contre lui, et contre les habitants destinés à la visitation ; taris, et détruis à la façon de l'interdit après eux, dit l'Eternel, et fais selon toutes les choses que je t'ai commandées.
\VS{22}L'alarme est au pays, et une grande calamité.
\VS{23}Comment est mis en pièces et est rompu le marteau de toute la terre ! Comment Babylone est-elle réduite en sujet d'étonnement parmi les nations !
\VS{24}Je t'ai tendu des filets, et aussi as-tu été prise, ô Babylone ! et tu n'en savais rien ; tu as été trouvée, et même attrapée, parce que tu t'en es prise à l'Eternel.
\VS{25}L'Eternel a ouvert son arsenal, et en a tiré les armes de son indignation ; parce que le Seigneur l'Eternel des armées a une entreprise à exécuter dans le pays des Caldéens.
\VS{26}Venez contre elle des bouts de la terre, ouvrez ses granges, foulez-la comme des javelles ; détruisez-la à la façon de l'interdit, et qu'elle n'ait rien de reste.
\VS{27}Coupez la gorge à tous ses veaux, et qu'ils descendent à la tuerie ; malheur à eux ! car le jour est venu, le temps de leur visitation.
\VS{28}[On entend] la voix de ceux qui s'enfuient, et qui sont échappés du pays de Babylone, pour annoncer dans Sion la vengeance de l'Eternel notre Dieu, la vengeance de son Temple.
\VS{29}Assemblez à cri public les archers contre Babylone ; vous tous qui tirez de l'arc, campez-vous contre elle tout alentour ; que personne n'échappe ; rendez-lui selon ses œuvres ; faites-lui selon tout ce qu'elle a fait ; car elle s'est fièrement portée contre l'Eternel, contre le Saint d'Israël.
\VS{30}C'est pourquoi ses gens d'élite tomberont dans les places, et on fera perdre la parole à tous ses gens de guerre en ce jour-là, dit l'Eternel.
\VS{31}Voici, j'en veux à toi, qui es la fierté même, dit le Seigneur l'Eternel des armées ; car ton jour est venu, le temps auquel je te visiterai.
\VS{32}La fierté bronchera et tombera, et il n'y aura personne qui la relève ; j'allumerai aussi le feu en ses villes, et il dévorera tous ses environs.
\VS{33}Ainsi a dit l'Eternel des armées : les enfants d'Israël et les enfants de Juda ont été ensemble opprimés ; tous ceux qui les ont pris les retiennent, et ont refusé de les laisser aller.
\VS{34}Leur Rédempteur est fort, son Nom [est] l'Eternel des armées ; il plaidera avec chaleur leur cause, pour donner du repos au pays, et mettre dans le trouble les habitants de Babylone.
\VS{35}L'épée est sur les Caldéens, dit l'Eternel, et sur les habitants de Babylone, sur ses principaux, et sur ses sages.
\VS{36}L'épée est tirée contre ses Devins, et ils en perdront l'esprit ; l'épée est sur ses hommes forts, et ils [en] seront épouvantés.
\VS{37}L'épée est sur ses chevaux, et sur ses chariots, et sur tout l'amas de diverses sortes de gens lequel [est] au milieu d'elle, et ils deviendront [comme] des femmes ; l'épée est sur ses trésors, et ils seront pillés.
\VS{38}La sécheresse sera sur ses eaux, et elles tariront ; parce que c'est un pays d'images taillées, et ils agiront en insensés à l'égard de leurs dieux qui les épouvantent.
\VS{39}C'est pourquoi les bêtes sauvages des déserts avec celles des Iles y habiteront, et les chats-huants y habiteront aussi ; et elle ne sera plus habitée à jamais, et on n'y demeurera point en quelque temps que ce soit.
\VS{40}Il n'y demeurera personne, a dit l'Eternel, et aucun fils d'homme n'y habitera, comme dans la subversion que Dieu a faite de Sodome et de Gomorrhe, et de leurs lieux circonvoisins.
\VS{41}Voici, un peuple et une grande nation vient de l'Aquilon, et plusieurs Rois se réveilleront du fond de la terre.
\VS{42}Ils prendront l'arc et l'étendard ; ils sont cruels, et ils n'auront point de compassion ; leur voix bruira comme la mer, et ils seront montés sur des chevaux ; chacun d'eux est rangé en homme de guerre contre toi, fille de Babylone.
\VS{43}Le Roi de Babylone en a ouï le bruit, et ses mains en sont devenues lâches ; l'angoisse l'a saisi, [et] un travail comme de celle qui enfante.
\VS{44}Voici, il montera comme un lion à cause de l'enflure du Jourdain, vers la demeure du pays rude, et après que je les aurai fait reposer je les ferai courir hors de la Caldée, et qui est d'élite, que je lui donne commission contre elle ? Car qui est semblable à moi ? et qui me déterminera le temps ? et qui sera le Pasteur qui tiendra ferme contre moi ?
\VS{45}C'est pourquoi écoutez la résolution que l'Eternel a prise contre Babylone, et les desseins qu'il a faits contre le pays des Caldéens : si les plus petits du troupeau ne les traînent par terre, et si on ne réduit en désolation leurs cabanes sur eux.
\VS{46}La terre a été ébranlée du bruit de la prise de Babylone, et le cri en a été ouï parmi les nations.
\Chap{51}
\VerseOne{}Ainsi a dit l'Eternel : voici, je m'en vais faire lever un vent de destruction contre Babylone, et contre ceux qui habitent au cœur [du Royaume] de ceux qui s'élèvent contre moi.
\VS{2}Et j'enverrai contre Babylone des vanneurs qui la vanneront, et qui videront son pays ; car de tous côtés ils seront venus contre elle au jour de son mal.
\VS{3}Qu'on bande l'arc contre celui qui bande son arc, et contre celui qui se confie en sa cuirasse ; et n'épargnez point ses gens d'élite, exterminez à la façon de l'interdit toute son armée ;
\VS{4}Et les blessés à mort tomberont au pays des Caldéens ; et les transpercés [tomberont] dans ses places ;
\VS{5}Car Israël et Juda n'est point privé de son Dieu, de l'Eternel des armées ; quoique leur pays ait été trouvé par le Saint d'Israël plein de crimes.
\VS{6}Fuyez hors de Babylone, et sauvez chacun sa vie, ne soyez point exterminés dans son iniquité ; car c'est le temps de la vengeance de l'Eternel ; il lui rend ce qu'elle a mérité.
\VS{7}Babylone a été comme une coupe d'or en la main de l'Eternel, enivrant toute la terre ; les nations ont bu de son vin ; c'est pourquoi les nations en ont perdu l'esprit.
\VS{8}Babylone est tombée en un instant, et a été brisée ; hurlez sur elle, prenez du baume pour sa douleur, peut-être qu'elle guérira.
\VS{9}Nous avons traité Babylone, et elle n'est point guérie ; laissez-la et allons-nous-en chacun en son pays ; car son procès est parvenu jusqu’aux cieux, et s'est élevé jusqu’aux nues.
\VS{10}L'Eternel a mis en évidence notre justice. Venez, et racontons en Sion l'œuvre de l'Eternel notre Dieu.
\VS{11}Fourbissez les flèches, et empoignez à pleines mains les boucliers ; l'Eternel a réveillé l'esprit des Rois de Méde ; car sa pensée est contre Babylone pour la détruire, parce que c'est ici la vengeance de l'Eternel, et la vengeance de son Temple.
\VS{12}Elevez l'enseigne sur les murailles de Babylone, renforcez la garnison, posez les gardes, préparez des embûches ; car l'Eternel a formé un dessein, même il a fait ce qu'il a dit contre les habitants de Babylone.
\VS{13}Tu étais assise sur plusieurs eaux, abondante en trésors ; ta fin est venue, et le comble de ton gain déshonnête.
\VS{14}L'Eternel des armées a juré par soi-même, en disant : si je ne te remplis d'hommes comme de hurebecs, et s'ils ne s'entre-répondent pour s'encourager contre toi.
\VS{15}C'est lui qui a fait la terre par sa vertu, et qui a rangé le monde habitable par sa sagesse, et qui a étendu les cieux par son intelligence.
\VS{16}Sitôt qu'il fait ouïr sa voix il y a un grand bruit d'eaux dans les cieux ; après qu'il a fait monter du bout de la terre les vapeurs, ses éclairs annoncent la pluie, et il tire le vent hors de ses trésors.
\VS{17}Tout homme paraît abruti dans sa science ; tout fondeur est rendu honteux par les images taillées ; car ce qu'ils fondent est une fausseté, et il n'y a point de respiration en elles.
\VS{18}Elles ne sont que vanité, et un ouvrage propre à abuser ; elles périront au temps de leur visitation.
\VS{19}La portion de Jacob n'est point comme ces choses-là ; car c'est celui qui a tout formé, et il est le lot de son héritage ; son Nom est l'Eternel des armées.
\VS{20}Tu m'as été un marteau [et] des instruments de guerre ; par toi j'ai mis en pièces les nations, et par toi j'ai détruit les Royaumes.
\VS{21}Et par toi j'ai mis en pièces le cheval et celui qui le monte ; et par toi j'ai mis en pièces le chariot et celui qui était monté dessus.
\VS{22}Et par toi j'ai mis en pièces l'homme et la femme ; et par toi j'ai mis en pièces le vieillard et le jeune garçon ; et par toi j'ai mis en pièces le jeune homme et la vierge.
\VS{23}Et par toi j'ai mis en pièces le pasteur et son troupeau ; et par toi j'ai mis en pièces le laboureur et ses bœufs accouplés ; et par toi j'ai mis en pièces les gouverneurs et les magistrats.
\VS{24}Mais je rendrai à Babylone, et à tous les habitants de la Caldée, tout le mal qu'ils ont fait à Sion, vous le voyant, dit l'Eternel.
\VS{25}Voici, j'en veux à toi, montagne qui détruis, dit l'Eternel, qui détruis toute la terre ; et j'étendrai ma main sur toi, et je te roulerai en bas du haut des rochers, et je te réduirai en montagne d'embrasement.
\VS{26}Et on ne pourra prendre de toi aucune pierre pour la placer à l'angle de l'édifice, ni aucune pierre pour servir de fondement, car tu seras des désolations perpétuelles, dit l'Eternel.
\VS{27}Levez l'enseigne sur la terre, sonnez de la trompette parmi les nations ; préparez les nations contre elle ; convoquez contre elle les Royaumes d'Ararat, de Minni, et d'Askenas ; établissez contre elle des Capitaines, faites monter ses chevaux comme le hurebec qui se hérisse.
\VS{28}Préparez contre elle les nations, les Rois de Méde, ses gouverneurs, et tous ses magistrats, et tout le pays de sa domination.
\VS{29}Et la terre en sera ébranlée, et en sera en travail, parce que tout ce que l'Eternel a pensé a été effectué contre Babylone, pour réduire le pays en désolation, tellement qu'il n'y ait personne qui [y] habite.
\VS{30}Les hommes forts de Babylone ont cessé de combattre, ils se sont tenus dans les forteresses, leur force est éteinte, et ils sont devenus [comme] des femmes ; on a brûlé ses demeures ; et ses barres ont été rompues.
\VS{31}Le courrier viendra à la rencontre du courrier, et le messager viendra à la rencontre du messager, pour annoncer au Roi de Babylone que sa ville est prise par un bout ;
\VS{32}Et que ses gués sont surpris, et que ses marais sont brûlés au feu, et que les hommes de guerre sont épouvantés.
\VS{33}Car ainsi a dit l'Eternel des armées, le Dieu d'Israël : la fille de Babylone est comme une aire ; il est temps qu'elle soit foulée ; encore un peu, et le temps de sa moisson viendra.
\VS{34}Nébucadnetsar Roi de Babylone, [dira Jérusalem], m'a dévorée et m'a froissée ; il m'a mise dans le même état qu'un vaisseau qui ne sert de rien ; il m'a engloutie comme un dragon ; il a rempli son ventre de mes délices, il m'a chassée au loin.
\VS{35}Ce qu'il m'a ravi par violence, et ma chair [est] à Babylone, dira l'habitante de Sion ; et mon sang est chez les habitants de la Caldée, dira Jérusalem.
\VS{36}C'est pourquoi ainsi a dit l'Eternel : voici, je m'en vais plaider ta cause, et je ferai la vengeance pour toi ; je dessécherai sa mer, et je ferai tarir sa source.
\VS{37}Et Babylone sera réduite en monceaux, en demeure de dragons, en étonnement, et en opprobre, sans que personne [y] habite.
\VS{38}Ils rugiront ensemble comme des lionceaux, et bruiront comme des faons de lions.
\VS{39}Je les ferai échauffer dans leurs festins, et les enivrerai, afin qu'ils se réjouissent, et qu'ils dorment d'un sommeil perpétuel, et qu'ils ne se réveillent plus, dit l'Eternel.
\VS{40}Je les ferai descendre comme des agneaux à la tuerie, et comme [on y mène] les moutons avec les boucs.
\VS{41}Comment a été prise Sésac ? et [comment] a été saisie celle qui était la louange de toute la terre ? comment Babylone a-t-elle été réduite en désolation parmi les nations ?
\VS{42}La mer est montée sur Babylone, elle a été couverte de la multitude de ses flots.
\VS{43}Ses villes ont été un sujet d'étonnement, une terre sèche et de landes, un pays où personne ne demeure, et où il ne passe pas un fils d'homme.
\VS{44}Je punirai aussi Bel à Babylone, et je tirerai hors de sa bouche ce qu'il avait englouti, et les nations n'aborderont plus vers lui ; la muraille même de Babylone est tombée.
\VS{45}Mon peuple, sortez du milieu d'elle, et sauvez chacun sa vie de l'ardeur de la colère de l'Eternel.
\VS{46}De peur que votre cœur ne s'amollisse, et que vous n'ayez peur des nouvelles qu'on entendra dans tout le pays ; car des nouvelles viendront une année, et après cela [d'autres] nouvelles une [autre] année, et il y aura violence dans la terre, et dominateur sur dominateur.
\VS{47}C'est pourquoi voici, les jours viennent que je punirai les images taillées de Babylone, et tout son pays sera rendu honteux, et tous ses blessés à mort tomberont au milieu d'elle.
\VS{48}Les cieux, et la terre, et tout ce qui y est, se réjouiront avec chant de triomphe contre Babylone, parce qu'il viendra de l'Aquilon des destructeurs contre elle, dit l'Eternel.
\VS{49}Et comme Babylone a fait tomber les blessés à mort d'Israël, ainsi les blessés à mort de tout le pays tomberont à Babylone.
\VS{50}Vous qui êtes échappés de l'épée, marchez, ne vous arrêtez point ; souvenez-vous de l'Eternel dans ces pays éloignés où vous êtes, et que Jérusalem vous revienne au cœur.
\VS{51}[Mais vous direz] : nous sommes honteux des reproches que nous avons entendus ; la confusion a couvert nos faces, en ce que les étrangers sont venus contre les Sanctuaires de la maison de l'Eternel.
\VS{52}C'est pourquoi voici, les jours viennent, dit l'Eternel, que je ferai justice de ses images taillées, et les blessés à mort gémiront par tout son pays.
\VS{53}Quand Babylone serait montée jusqu'aux cieux, et qu'elle aurait fortifié le plus haut de sa forteresse, toutefois les destructeurs y entreront de par moi, dit l'Eternel.
\VS{54}Un grand cri s'entend de Babylone, et un grand débris du pays des Caldéens.
\VS{55}Parce que l'Eternel s'en va détruire Babylone, et il abolira du milieu d'elle la voix magnifique, et leurs flots bruiront comme de grosses eaux, l'éclat de leur bruit retentira.
\VS{56}Car le destructeur est venu contre elle, contre Babylone ; ses hommes forts ont été pris, et leurs arcs ont été brisés ; car le [Dieu] Fort des rétributions, l'Eternel, ne manque jamais à rendre la pareille.
\VS{57}J'enivrerai donc ses principaux et ses sages, ses gouverneurs et ses magistrats, et ses hommes forts ; ils dormiront d'un sommeil perpétuel, et ils ne se réveilleront plus, dit le Roi dont le Nom est l'Eternel des armées.
\VS{58}Ainsi a dit l'Eternel des armées : Il n'y aura aucune muraille de Babylone, quelque large qu'elle soit, qui ne soit entièrement rasée ; et ses portes, qui sont si hautes, seront brûlées au feu ; ainsi les peuples auront travaillé inutilement, et les nations pour le feu, et elles s'y seront lassées.
\VS{59}C'est ici l'ordre que Jérémie le Prophète donna à Séraja, fils de Nérija, fils de Mahaséja, quand il alla de la part de Sédécias Roi de Juda en Babylone, la quatrième année de son Règne ; or Séraja était principal Chambellan.
\VS{60}Car Jérémie écrivit dans un livre tout le mal qui devait venir sur Babylone ; savoir toutes ces paroles qui sont écrites contre Babylone.
\VS{61}Jérémie donc dit à Séraja : Sitôt que tu seras venu à Babylone, et que tu l'auras vue, tu liras toutes ces paroles-là ;
\VS{62}Et tu diras : Eternel, tu as parlé contre ce lieu-ci pour l'exterminer, en sorte qu'il n'y ait aucun habitant, depuis l'homme jusqu'à la bête, mais qu'il soit réduit en désolations perpétuelles.
\VS{63}Et sitôt que tu auras achevé de lire ce livre, tu le lieras à une pierre, et le jetteras dans l'Euphrate ;
\VS{64}Et tu diras : Babylone sera ainsi plongée, et elle ne se relèvera point du mal que je m'en vais faire venir sur elle, et ils en seront accablés. Jusques ici sont les paroles de Jérémie.
\Chap{52}
\VerseOne{}Sédécias était âgé de vingt-et-un ans quand il commença à régner, et il régna onze ans à Jérusalem, sa mère avait nom Hamutal, [et] elle était fille de Jérémie de Libna.
\VS{2}Il fit ce qui déplaît à l'Eternel, comme avait fait Jéhojakim.
\VS{3}Car il [arriva] à cause de la colère de l'Eternel contre Jérusalem et Juda, jusqu'à les rejeter de devant soi, que Sédécias se rebella contre le Roi de Babylone.
\VS{4}Il arriva donc l'an neuvième de son Règne ; le dixième jour du dixième mois, que Nébucadnetsar Roi de Babylone vint contre Jérusalem, lui et toute son armée, et ils se campèrent contre elle, et firent des terrasses tout alentour.
\VS{5}Et la ville fut assiégée jusqu’à l'onzième année du Roi Sédécias.
\VS{6}Et le neuvième jour du quatrième mois la famine se renforça dans la ville, tellement qu'il n'y avait point de pain pour le peuple du pays.
\VS{7}Alors la brèche fut faite à la ville, et tous les gens de guerre s'enfuirent, et sortirent de nuit hors de la ville, par le chemin de la porte entre les deux murailles, qui était près du jardin du Roi (or les Caldéens étaient tout autour de la ville) et s'en allèrent par le chemin de la campagne.
\VS{8}Mais l'armée des Caldéens poursuivit le Roi, et quand ils eurent atteint Sédécias dans les campagnes de Jéricho toute son armée se dispersa d'avec lui.
\VS{9}Ils prirent donc le Roi, et le firent monter vers le Roi de Babylone à Riblatha au pays de Hamath, où on lui fit son procès.
\VS{10}Et le Roi de Babylone fit égorger les fils de Sédécias en sa présence ; il fit égorger aussi tous les principaux de Juda à Riblatha.
\VS{11}Puis il fit crever les yeux à Sédécias, et le fit lier de doubles chaînes d'airain, et le Roi de Babylone le mena à Babylone, et le mit en prison jusqu'au jour de sa mort.
\VS{12}Et au dixième jour du cinquième mois, en l'an dix-neuvième de Nébucadnetsar Roi de Babylone, Nébuzar-adan, prévôt de l'hôtel, serviteur ordinaire du Roi de Babylone, entra dans Jérusalem ;
\VS{13}Et brûla la maison de l'Eternel, et la maison Royale, et toutes les maisons de Jérusalem, et mit le feu dans toutes les maisons des Grands.
\VS{14}Et toute l'armée des Caldéens, qui était avec le prévôt de l'hôtel, démolit toutes les murailles qui étaient autour de Jérusalem.
\VS{15}Et Nébuzar-adan, prévôt de l'hôtel, transporta [à Babylone] des plus pauvres du peuple, le reste du peuple, [savoir] ceux qui étaient demeurés de reste dans la ville, et ceux qui étaient allés rendre au Roi de Babylone, avec le reste de la multitude.
\VS{16}Toutefois Nébuzar-adan, prévôt de l'hôlel, laissa quelques-uns des plus pauvres du pays pour être vignerons et laboureurs.
\VS{17}Et les Caldéens mirent en pièces les colonnes d'airain qui étaient dans la maison de l'Eternel, avec les soubassements ; et la mer d'airain qui était dans la maison de l'Eternel, et en emportèrent tout l'airain à Babylone.
\VS{18}Ils emportèrent aussi les chaudrons, et les racloirs, et les serpes, et les bassins, et les tasses, et tous les ustensiles d'airain dont on faisait le service.
\VS{19}Le prévôt de l'hôtel emporta aussi les coupes, et les encensoirs, et les bassins, et les chaudrons, et les chandeliers, et les tasses, et les gobelets ; ce qui était d'or, et ce qui était d'argent.
\VS{20}Quant aux deux colonnes, à la mer, et aux douze bœufs d'airain qui servaient de soubassements, lesquels le Roi Salomon avait faits pour la maison de l'Eternel, on ne pesa point l'airain de tous ces vaisseaux-là.
\VS{21}Or quant aux colonnes chaque colonne avait dix-huit coudées de haut, et un cordon de douze coudées l'environnait ; et elle était épaisse de quatre doigts, et était creuse ;
\VS{22}et il y avait par-dessus un chapiteau d'airain ; et la hauteur d'un des chapiteaux [était] de cinq coudées, il y avait aussi un rets et des grenades tout autour du chapiteau, le tout d'airain ; et la seconde colonne était de même façon, et aussi les grenades.
\VS{23}Il y avait aussi quatre-vingt-seize grenades au côté, [et] les grenades qui étaient sur le rets à l'entour, étaient cent en tout.
\VS{24}Davantage le prévôt de l'hôtel emmena Séraja, qui était le premier Sacrificateur, et Sophonie, qui était le second Sacrificateur, et les trois gardes des vaisseaux.
\VS{25}Il emmena aussi de la ville un Eunuque qui avait la charge des hommes de guerre, et sept hommes de ceux qui étaient près de la personne du Roi, lesquels furent trouvés dans la ville ; et le Secrétaire du Capitaine de l'armée qui enrôlait le peuple du pays ; et soixante hommes d'entre le peuple du pays, qui furent trouvés dans la ville.
\VS{26}Nébuzar-adan donc, prévôt de l'hôtel, les prit, et les emmena vers le Roi de Babylone à Ribla.
\VS{27}Et le Roi de Babylone les frappa, et les fit mourir à Ribla au pays de Hamath. Ainsi Juda fut transporté hors de sa terre.
\VS{28}Et c'est ici le peuple que Nébucadnetsar transporta ; la septième année, trois mille vingt-trois Juifs.
\VS{29}La dix-huitième année de Nébucadnetsar, on transporta de Jérusalem huit cent trente-deux personnes.
\VS{30}La vingt-troisième année de Nébucadnetsar, Nébuzar-adan, prévôt de l'hôtel, transporta sept cent quarante-cinq personnes des Juifs ; toutes les personnes donc furent quatre mille six cents.
\VS{31}Or il arriva l'an trente-septième de la captivité de Jéhojachin, Roi de Juda, au vingt-cinquième jour du douzième mois, qu'Evilmérodac, Roi de Babylone, l'année qu'il commença à régner, tira de prison Jéhojachin Roi de Juda, et le mit en liberté.
\VS{32}Et lui parla avec bonté, et mit son trône au dessus du trône des [autres] Rois qui étaient avec lui à Babylone.
\VS{33}Et après qu'il lui eut changé ses vêtements de prison, il mangea du pain ordinairement tous les jours de sa vie en la présence du Roi.
\VS{34}Et quant à son ordinaire, un ordinaire continuel lui fut établi de par le Roi de Babylone pour chaque jour, jusques au jour de sa mort, tout le temps de sa vie.
\PPE{}
\end{multicols}
