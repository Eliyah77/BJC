\ShortTitle{Levitique}\BookTitle{Levitique}\BFont
\begin{multicols}{2}
\Chap{1}
\VerseOne{}Or l'Eternel appela Moïse, et lui parla du Tabernacle d'assignation, en disant :
\VS{2}Parle aux enfants d'Israël, et leur dis : Quand quelqu'un d'entre vous offrira à l'Eternel une offrande d'une bête à quatre pieds, il fera son offrande de gros ou de menu bétail.
\VS{3}Si son offrande pour un holocauste est de gros bétail, il offrira un mâle sans tare ; il l'offrira de son bon gré, à l'entrée du Tabernacle d'assignation, devant l'Eternel.
\VS{4}Et il posera sa main sur la tête de l'holocauste, et il sera agréé pour lui, afin de faire propitiation pour lui.
\VS{5}Puis on égorgera le veau devant l'Eternel, et les fils d'Aaron Sacrificateurs en offriront le sang, et ils répandront le sang tout autour sur l'autel, qui est à l'entrée du Tabernacle d'assignation.
\VS{6}Et on égorgera l'holocauste, et on le coupera par pièces.
\VS{7}Et les fils d'Aaron Sacrificateurs mettront le feu sur l'autel, et arrangeront le bois sur le feu.
\VS{8}Et les fils d'Aaron Sacrificateurs arrangeront les pièces, la tête, et la fressure au dessus du bois qui sera au feu sur l'autel.
\VS{9}Mais il lavera avec de l'eau le ventre et les jambes, et le Sacrificateur fera fumer toutes ces choses sur l'autel ; c'est un holocauste, un sacrifice fait par feu, en bonne odeur à l'Eternel.
\VS{10}Que si son offrande pour l'holocauste est de menu bétail, d'entre les brebis ou d'entre les chèvres, il offrira un mâle sans tare.
\VS{11}Et on l'égorgera à côté de l'autel vers le Septentrion devant l'Eternel, et les fils d'Aaron Sacrificateurs en répandront le sang sur l'autel tout autour.
\VS{12}Puis on le coupera par pièces, avec sa tête, et sa fressure ; et le Sacrificateur les arrangera sur le bois qui sera au feu qui est sur l'autel.
\VS{13}Mais il lavera avec de l'eau le ventre et les jambes. Puis le Sacrificateur offrira toutes ces choses, et les fera fumer sur l'autel ; c'est un holocauste, un sacrifice fait par feu d'agréable odeur à l'Eternel.
\VS{14}Que si son offrande pour l'holocauste à l'Eternel est d'oiseaux, il fera son offrande de tourterelles, ou de pigeonneaux.
\VS{15}Et le Sacrificateur l'offrira sur l'autel, et lui entamera la tête avec l'ongle, afin de la faire fumer sur l'autel, et il en épreindra le sang au côté de l'autel.
\VS{16}Et il ôtera son jabot avec sa plume, et les jettera près de l'autel vers l'Orient, où seront les cendres.
\VS{17}Il l'entamera donc avec ses ailes sans le diviser ; et le Sacrificateur le fera fumer sur l'autel, au dessus du bois qui sera au feu ; c'est un holocauste, un sacrifice fait par feu d'agréable odeur à l'Eternel.
\Chap{2}
\VerseOne{}Et quand quelque personne offrira l'offrande du gâteau à l'Eternel, son offrande sera de fleur de farine, et il versera de l'huile sur le gâteau, et mettra de l'encens par dessus.
\VS{2}Et il l'apportera aux fils d'Aaron Sacrificateurs, et le [Sacrificateur] prendra une poignée de la fleur de farine, et de l'huile dont le gâteau aura été fait, avec tout l'encens qui était sur le gâteau, et il fera fumer son mémorial sur l'autel ; c'est une offrande faite par feu en bonne odeur à l'Eternel.
\VS{3}Mais ce qui restera du gâteau sera pour Aaron et ses fils ; c'est une chose très-sainte d'entre les offrandes faites par feu à l'Eternel.
\VS{4}Et quand tu offriras une offrande de gâteaux cuits au four, ce seront des tourteaux sans levain, de fine farine, pétris avec de l'huile, et des beignets sans levain, oints d'huile.
\VS{5}Et si ton offrande est de gâteau cuit sur la plaque, elle sera de fine farine pétrie dans l'huile, sans levain.
\VS{6}Tu la mettras par morceaux, et tu verseras de l'huile sur elle ; car c'est une offrande de gâteau.
\VS{7}Et si ton offrande est un gâteau de poêle, elle sera faite de fine farine avec de l'huile.
\VS{8}Puis tu apporteras à l'Eternel le gâteau qui sera fait de ces choses-là, et on le présentera au Sacrificateur, qui l'apportera vers l'autel.
\VS{9}Et le Sacrificateur en lèvera son mémorial, et le fera fumer sur l'autel ; c'est une offrande faite par feu en bonne odeur à l'Eternel.
\VS{10}Et ce qui restera du gâteau sera pour Aaron et ses fils ; c'est une chose très-sainte, d'entre les offrandes faites par feu à l'Eternel.
\VS{11}Quelque gâteau que vous offriez à l'Eternel, il ne sera point fait avec du levain ; car vous ne ferez point fumer de levain, ni de miel, dans aucune offrande faite par feu à l'Eternel.
\VS{12}Vous pourrez bien les offrir à l'Eternel dans l'offrande des prémices, [mais] ils ne seront point mis sur l'autel pour être [une oblation] de bonne odeur.
\VS{13}Tu saleras aussi de sel toute offrande de ton gâteau, et tu ne laisseras point manquer sur ton gâteau le sel de l'alliance de ton Dieu ; mais dans toutes tes oblations tu offriras du sel.
\VS{14}Et si tu offres à l'Eternel le gâteau des premiers fruits, tu offriras, pour le gâteau de tes premiers fruits, des épis qui commencent à mûrir, rôtis au feu, [savoir] les grains de quelques épis bien grenés, broyés entre les mains.
\VS{15}Puis tu mettras de l'huile sur le gâteau, et tu mettras aussi de l'encens par dessus ; c'est une offrande de gâteau.
\VS{16}Et le Sacrificateur fera fumer son mémorial, [pris] de ses grains broyés, et de son huile avec tout l'encens ; c'est une offrande faite par feu à l'Eternel.
\Chap{3}
\VerseOne{}Et si l'offrande de quelqu'un [est] un sacrifice de prospérités, [et] qu'il l'offre de gros bétail, soit mâle soit femelle, il l'offrira sans tare, devant l'Eternel.
\VS{2}Et il posera sa main sur la tête de son offrande, et on l'égorgera à l'entrée du Tabernacle d'assignation, et les fils d'Aaron Sacrificateurs répandront le sang sur l'autel à l'entour.
\VS{3}Puis on offrira, du sacrifice de prospérités une offrande faite par feu à l'Eternel, [savoir] la graisse qui couvre les entrailles, et toute la graisse qui est sur les entrailles ;
\VS{4}Et les deux rognons avec la graisse qui est sur eux, jusque sur les flancs, et on ôtera la taie qui est sur le foie [pour la mettre] avec les rognons.
\VS{5}Et les fils d'Aaron feront fumer tout cela sur l'autel, par dessus l'holocauste qui sera sur le bois [qu'on aura mis] sur le feu ; c'est une offrande faite par feu de bonne odeur à l'Eternel.
\VS{6}Que si son offrande est de menu bétail pour le sacrifice de prospérités à l'Eternel, soit mâle soit femelle, il l'offrira sans tare.
\VS{7}S'il offre un agneau pour son offrande, il l'offrira devant l'Eternel.
\VS{8}Et il posera sa main sur la tête de son offrande, et on l'égorgera devant le Tabernacle d'assignation, et les fils d'Aaron répandront son sang sur l'autel à l'entour.
\VS{9}Et il offrira du sacrifice de prospérités une offrande faite par feu à l'Eternel, en ôtant sa graisse, et sa queue entière jusque contre l'échine, avec la graisse qui couvre les entrailles, et toute la graisse qui est sur les entrailles ;
\VS{10}Et les deux rognons avec la graisse qui est sur eux, jusque sur les flancs, et il ôtera la taie qui est sur le foie [pour la mettre] sur les rognons.
\VS{11}Et le Sacrificateur fera fumer [tout] cela sur l'autel ; c'est une viande d'offrande faite par feu à l'Eternel.
\VS{12}Que si son offrande [est] d'entre les chèvres, il l'offrira devant l'Eternel.
\VS{13}Et il posera sa main sur la tête de son [offrande], et on l'égorgera devant le Tabernacle d'assignation ; et les enfants d'Aaron répandront son sang sur l'autel à l'entour.
\VS{14}Puis il offrira son offrande pour sacrifice fait par feu à l'Eternel, [savoir], la graisse qui couvre les entrailles, et toute la graisse qui est sur les entrailles.
\VS{15}Et les deux rognons, et la graisse qui est sur eux, jusque sur les flancs, et il ôtera la taie qui est sur le foie [pour la mettre] sur les rognons.
\VS{16}Puis le Sacrificateur fera fumer [toutes] ces choses-là sur l'autel ; c'est une viande d'offrande faite par feu en bonne odeur. Toute graisse appartient à l'Eternel.
\VS{17}C'est une ordonnance perpétuelle en vos âges, et dans toutes vos demeures, que vous ne mangerez aucune graisse, ni aucun sang.
\Chap{4}
\VerseOne{}L'Eternel parla encore à Moïse, en disant :
\VS{2}Parle aux enfants d'Israël, et leur dis : Quand une personne aura péché par erreur contre quelqu'un des commandements de l'Eternel, en commettant des choses qui ne se doivent point faire, et qu'il aura fait quelqu'une de ces choses ;
\VS{3}Si c'est le Sacrificateur oint qui ait commis un péché semblable à quelque faute du peuple, il offrira à l'Eternel pour son péché qu'il aura fait, un veau sans tare, pris du troupeau, en offrande pour le péché.
\VS{4}Il amènera le veau à l'entrée du Tabernacle d'assignation devant l'Eternel, il posera sa main sur la tête du veau, et l'égorgera devant l'Eternel.
\VS{5}Et le Sacrificateur oint prendra du sang du veau, et l'apportera dans le Tabernacle d'assignation.
\VS{6}Et le Sacrificateur trempera son doigt dans le sang, et fera aspersion du sang par sept fois devant l'Eternel, au devant du voile du Sanctuaire.
\VS{7}Le Sacrificateur mettra aussi devant l'Eternel du sang sur les cornes de l'autel du parfum des drogues, qui est dans le Tabernacle d'assignation ; mais il répandra tout le reste du sang du veau au pied de l'autel de l'holocauste, qui est à l'entrée du Tabernacle d'assignation.
\VS{8}Et il lèvera toute la graisse du veau de l'offrande pour le péché, [savoir], la graisse qui couvre les entrailles, et toute la graisse qui est sur les entrailles.
\VS{9}Et les deux rognons avec la graisse qui est sur eux, jusque sur les flancs, et il ôtera la taie qui est sur le foie [pour la mettre] sur les rognons ;
\VS{10}Comme on les ôte du bœuf du sacrifice de prospérités, et le Sacrificateur fera fumer [toutes] ces choses-là sur l'autel de l'holocauste.
\VS{11}Mais quant à la peau du veau et toute sa chair, avec sa tête, ses jambes, ses entrailles, et sa fiente,
\VS{12}Et [même] tout le veau, il le tirera hors du camp dans un lieu net, où l'on répand les cendres, et il le brûlera sur du bois au feu, il sera brûlé au lieu où l'on répand les cendres.
\VS{13}Et si toute l'assemblée d'Israël a péché par erreur, et que la chose n'ait pas été aperçue par l'assemblée, et qu'ils aient violé quelque commandement de l'Eternel, en commettant des choses qui ne se doivent point faire, et se soient rendus coupables ;
\VS{14}Et que le péché qu'ils ont fait vienne en évidence, l'assemblée offrira en offrande pour le péché un veau pris du troupeau, et on l'amènera devant le Tabernacle d'assignation.
\VS{15}Et les Anciens de l'assemblée poseront leurs mains sur la tête du veau devant l'Eternel.
\VS{16}Et le Sacrificateur oint portera du sang du veau dans le Tabernacle d'assignation.
\VS{17}Ensuite le Sacrificateur trempera son doigt dans le sang, et en fera aspersion devant l'Eternel au devant du voile, par sept fois.
\VS{18}Et il mettra du sang sur les cornes de l'autel qui est devant l'Eternel dans le Tabernacle d'assignation, et il répandra tout le reste du sang au pied de l'autel de l'holocauste, qui est à l'entrée du Tabernacle d'assignation.
\VS{19}Et il lèvera toute sa graisse, et la fera fumer sur l'autel ;
\VS{20}Et il fera de ce veau, comme il a fait du veau de l'offrande pour son péché. Le Sacrificateur fera ainsi ; il fera propitiation pour eux ; et il leur sera pardonné.
\VS{21}Puis il tirera hors du camp le veau, et le brûlera comme il a brûlé le premier veau ; car c'est l'offrande pour le péché de l'assemblée.
\VS{22}Que si quelqu'un des principaux a péché, ayant violé par erreur quelqu'un des commandements de l'Eternel son Dieu, en commettant des choses qui ne se doivent point faire, et s'est rendu coupable ;
\VS{23}Et qu'on l'avertisse de son péché, qu'il a commis, il amènera pour sacrifice un jeune bouc mâle sans tare ;
\VS{24}Et il posera sa main sur la tête du bouc, et on l'égorgera au lieu où l'on égorge l'holocauste devant l'Eternel ; [car] c'est une offrande pour le péché.
\VS{25}Puis le Sacrificateur prendra avec son doigt du sang de l'offrande pour le péché, et le mettra sur les cornes de l'autel de l'holocauste, et il répandra le reste de son sang au pied de l'autel de l'holocauste.
\VS{26}Et il fera fumer toute sa graisse sur l'autel comme la graisse du sacrifice de prospérités ; ainsi le Sacrificateur fera propitiation pour lui de son péché, et il lui sera pardonné.
\VS{27}Que si quelque personne du commun peuple a péché par erreur, en violant quelqu'un des commandements de l'Eternel, [et] en commettant des choses qui ne se doivent point faire, et s'est rendu coupable ;
\VS{28}Et qu'on l'avertisse de son péché qu'il a commis, il amènera son offrande d'une jeune chèvre, sans tare, femelle, pour son péché qu'il a commis.
\VS{29}Et il posera sa main sur la tête de l'offrande pour le péché, et on égorgera l'offrande pour le péché au lieu où l'on égorge l'holocauste.
\VS{30}Puis le Sacrificateur prendra du sang de la chèvre avec son doigt, et le mettra sur les cornes de l'autel de l'holocauste, et il répandra tout le reste de son sang au pied de l'autel.
\VS{31}Et il ôtera toute sa graisse comme on ôte la graisse de dessus le sacrifice de prospérités, et le Sacrificateur la fera fumer sur l'autel, en bonne odeur à l'Eternel, il fera propitiation pour lui, et il lui sera pardonné.
\VS{32}Que s'il amène un agneau pour l'oblation de son péché, ce sera une femelle sans tare qu'il amènera.
\VS{33}Et il posera sa main sur la tête de l'offrande pour le péché, et on l'égorgera pour le péché au lieu où l'on égorge l'holocauste.
\VS{34}Puis le Sacrificateur prendra avec son doigt du sang de l'offrande pour le péché, et le mettra sur les cornes de l'autel de l'holocauste, et il répandra tout le reste de son sang au pied de l'autel.
\VS{35}Et il ôtera toute sa graisse, comme on ôte la graisse de l'agneau du sacrifice de prospérités, et le Sacrificateur les fera fumer sur l'autel par dessus les sacrifices de l'Eternel faits par feu, et il fera propitiation pour lui, touchant son péché qu'il aura commis, et il lui sera pardonné.
\Chap{5}
\VerseOne{}Et quand quelqu'un aura péché lorsqu'il aura ouï quelqu'un proférant quelque parole exécrable, et en aura été témoin, soit qu'il l'ait vu ou qu'il l'ait su, et ne l'aura point déclaré, il portera son iniquité.
\VS{2}Ou quand quelqu'un aura touché une chose souillée, soit la charogne des bêtes sauvages immondes, soit la charogne des animaux domestiques immondes, soit la charogne des reptiles, lesquels sont immondes, quoiqu'il ne s'en soit pas aperçu, il est toutefois souillé, et coupable.
\VS{3}Ou quand il aura touché à la souillure d'un homme, à quelle que ce soit de ses souillures ; soit qu'il ne s'en soit pas aperçu, soit qu'il l'ait connu, il est coupable.
\VS{4}Ou quand quelqu'un aura juré en proférant légèrement de ses lèvres de faire du mal ou du bien, selon tout ce que l'homme profère légèrement en jurant, soit qu'il ne s'en soit pas aperçu, soit qu'il l'ait connu, il est coupable dans l'un de ces points.
\VS{5}Quand donc quelqu'un sera coupable en l'un de ces points-là, il confessera en quoi il aura péché.
\VS{6}Et il amènera [la victime] de son péché à l'Eternel pour le péché qu'il aura commis, [savoir] une femelle du menu bétail, soit une jeune brebis, soit une jeune chèvre, pour le péché ; et le Sacrificateur fera propitiation pour lui de son péché.
\VS{7}Et s'il n'a pas le moyen de trouver une brebis ou une chèvre, il apportera à l'Eternel pour offrande du péché qu'il aura commis, deux tourterelles, ou deux pigeonneaux ; l'un en offrande pour le péché ; et l'autre, pour l'holocauste.
\VS{8}Il les apportera, [dis-je], au Sacrificateur, qui offrira premièrement celui qui est pour le péché ; et il leur entamera la tête avec l'ongle vers le cou, sans la séparer.
\VS{9}Puis il fera aspersion du sang du [sacrifice pour] le péché sur un côté de l'autel ; et ce qui restera du sang on l'épreindra au pied de l'autel ; [car] c'est un [sacrifice] pour le péché.
\VS{10}Et de l'autre il en fera un holocauste, selon l'ordonnance, et le Sacrificateur fera pour lui la propitiation pour son péché qu'il aura commis ; et il lui sera pardonné.
\VS{11}Que si celui qui aura péché n'a pas le moyen de trouver deux tourterelles, ou deux pigeonneaux, il apportera pour son offrande la dixième partie d'un Epha de fine farine, [mais] il ne mettra sur elle ni huile ni encens ; car c'est une offrande pour le péché.
\VS{12}Il l'apportera au Sacrificateur, qui en prendra une poignée pour mémorial de cette offrande, et la fera fumer sur l'autel, sur les sacrifices faits par feu à l'Eternel ; [car c'est une offrande pour le] péché.
\VS{13}Ainsi le Sacrificateur fera propitiation pour lui, pour son péché qu'il aura commis en l'une de ces choses-là, et il lui sera pardonné ; et le reste sera pour le Sacrificateur, comme étant une offrande de gâteau.
\VS{14}L'Eternel parla aussi à Moïse, en disant :
\VS{15}Quand quelqu'un aura commis un crime et un péché par erreur, en [retenant] des choses sanctifiées à l'Eternel, il amènera [une victime pour] son péché à l'Eternel ; [savoir] un bélier sans tare, [pris] du troupeau, avec l'estimation que tu feras de la chose sainte, la faisant en sicles d'argent, selon le sicle du Sanctuaire, à cause de son péché.
\VS{16}Il restituera donc ce en quoi il aura péché en [retenant] de la chose sainte, et il y ajoutera un cinquième par dessus, et le donnera au Sacrificateur ; et le Sacrificateur fera propitiation pour lui, par le bélier du sacrifice [pour le péché], et il lui sera pardonné.
\VS{17}Et quand quelqu'un aura péché, et aura violé quelqu'un des commandements de l'Eternel, en commettant des choses qu'on ne doit point faire, et qu'il ne l'aura point su, il sera coupable, et portera son iniquité.
\VS{18}Il amènera donc au Sacrificateur un bélier sans tare [pris] du troupeau, avec l'estimation que tu feras de la faute ; et le Sacrificateur fera propitiation pour lui de la faute qu'il aura commise par erreur, et dont il ne se sera point aperçu ; et ainsi il lui sera pardonné.
\VS{19}Il y a du péché ; certainement il s'est rendu coupable contre l'Eternel.
\Chap{6}
\VerseOne{}L'Eternel parla aussi à Moïse, en disant :
\VS{2}Quand quelque personne aura péché, et aura commis un crime contre l'Eternel, en mentant à son prochain pour un dépôt, ou pour une chose qu'on aura mise entre ses mains, soit qu'il l'ait ravie, soit qu'il ait trompé son prochain.
\VS{3}Ou s'il a trouvé quelque chose perdue, et qu'il mente à ce sujet ; ou s'il jure faussement sur quelqu'une de toutes les choses qu'il arrive à l'homme de faire, en péchant à leur égard ;
\VS{4}S'il arrive donc qu'il ait péché, et qu'il soit trouvé coupable, il rendra la chose qu'il aura ravie, ou ce qu'il aura usurpé par tromperie, ou le dépôt qui lui aura été donné en garde, ou la chose perdue qu'il aura trouvée ;
\VS{5}Ou tout ce dont il aura juré faussement ; il restituera le principal, et il ajoutera un cinquième par dessus à celui-là qui il appartenait ; il le donnera le jour qu'il aura été déclaré coupable.
\VS{6}Et il amènera pour l'Eternel au Sacrificateur [la victime de] son péché, [savoir] un bélier sans tare, [pris] du troupeau, avec l'estimation que tu feras de la faute.
\VS{7}Et le Sacrificateur fera propitiation pour lui devant l'Eternel ; et il lui sera pardonné, pour tout ce qu'il aura fait en quoi il se sera rendu coupable.
\VS{8}L'Eternel parla aussi à Moïse, en disant :
\VS{9}Commande à Aaron et à ses fils, et leur dis : C'est ici la Loi de l'holocauste ; l'holocauste [demeurera] sur le feu qui est sur l'autel, toute la nuit jusqu'au matin, et le feu de l'autel y sera tenu allumé.
\VS{10}Et le Sacrificateur, vêtu de sa robe de lin, mettra ses caleçons de lin sur sa chair, et il lèvera les cendres après que le feu aura consumé l'holocauste sur l'autel ; puis il les mettra près de l'autel.
\VS{11}Alors il dépouillera ses vêtements, et s'étant vêtu d'autres habits, il transportera les cendres hors du camp, en un lieu net.
\VS{12}Et quant au feu qui est sur l'autel, on l'y tiendra allumé, [et] on ne le laissera point éteindre. Le Sacrificateur allumera du bois au feu tous les matins, il arrangera l'holocauste sur le bois, et y fera fumer les graisses des offrandes de prospérités.
\VS{13}On tiendra le feu continuellement allumé sur l'autel, [et] on ne le laissera point éteindre.
\VS{14}Et c'est ici la Loi de l'offrande du gâteau ; les fils d'Aaron l'offriront devant l'Eternel sur l'autel.
\VS{15}Et on lèvera une poignée de la fleur de farine du gâteau, et de son huile, avec tout l'encens qui est sur le gâteau, et on le fera fumer en bonne odeur sur l'autel pour mémorial à l'Eternel.
\VS{16}Et Aaron et ses fils mangeront ce qui en restera ; on le mangera sans levain dans un lieu saint, on le mangera dans le parvis du Tabernacle d'assignation.
\VS{17}On n'en cuira point qui soit fait avec du levain ; je leur ai donné cela pour leur portion d'entre mes offrandes faites par feu ; c'est une chose très-sainte, comme [la victime pour] le péché, et [la victime pour] le délit.
\VS{18}Tout mâle d'entre les fils d'Aaron en mangera ; c'est une ordonnance perpétuelle en vos âges touchant les offrandes faites par feu à l'Eternel ; quiconque les touchera sera sanctifié.
\VS{19}L'Eternel parla aussi à Moïse, en disant :
\VS{20}C'est ici l'offrande d'Aaron et de ses fils, laquelle ils offriront à l'Eternel le jour qu'il sera oint, [savoir] la dixième partie d'un Epha de fine farine, pour offrande perpétuelle ; une moitié le matin, [et] l'autre moitié le soir.
\VS{21}Elle sera apprêtée sur une plaque avec de l'huile, tu l'apporteras ainsi rissolée, et tu offriras les pièces cuites du gâteau en bonne odeur à l'Eternel.
\VS{22}Et le Sacrificateur d'entre ses fils qui [sera] oint en sa place, fera cela par ordonnance perpétuelle ; on le fera fumer tout entier à l'Eternel.
\VS{23}Et tout le gâteau du Sacrificateur sera consumé, sans en manger.
\VS{24}L'Eternel parla aussi à Moïse, en disant :
\VS{25}Parle à Aaron et à ses fils, et leur dis : C'est ici la Loi [de la victime pour] le péché ; [la victime] pour le péché sera égorgée devant l'Eternel, dans le [même] lieu où l'holocauste sera égorgé ; [car] c'est une chose très-sainte.
\VS{26}Le Sacrificateur qui offrira [la victime] pour le péché, la mangera ; elle se mangera dans un lieu saint, dans le parvis du Tabernacle d'assignation.
\VS{27}Quiconque touchera sa chair sera saint, et s'il en rejaillit quelque sang sur le vêtement, ce sur quoi le sang sera tombé sera lavé dans le lieu saint.
\VS{28}Et le vaisseau de terre dans lequel on l'aura fait bouillir, sera cassé ; mais si on l'a fait bouillir dans un vaisseau d'airain, il sera écuré, et lavé dans l'eau.
\VS{29}Tout mâle d'entre les Sacrificateurs en mangera ; c'est une chose très-sainte.
\VS{30}Nulle [victime pour] le péché dont on portera du sang dans le Tabernacle d'assignation, pour faire la propitiation dans le Sanctuaire, ne sera mangée, mais elle sera brûlée au feu.
\Chap{7}
\VerseOne{}Or c'est ici la loi de [l'oblation pour le] délit ; c'est une chose très-sainte.
\VS{2}Au même lieu où l'on égorgera l'holocauste, on égorgera [la victime] pour le délit ; et on en répandra le sang sur l'autel à l'entour.
\VS{3}Puis on en offrira toute la graisse, avec sa queue, et toute la graisse qui couvre les entrailles.
\VS{4}Et on en ôtera les deux rognons, et la graisse qui est sur eux jusque sur les flancs, et la taie qui est sur le foie [pour la mettre] sur les deux rognons.
\VS{5}Et le Sacrificateur fera fumer [toutes] ces choses sur l'autel en offrande faite par feu à l'Eternel ; c'est [un sacrifice pour] le délit.
\VS{6}Tout mâle d'entre les sacrificateurs en mangera ; il sera mangé dans un lieu saint, [car] c'est une chose très-sainte.
\VS{7}[L'oblation pour] le délit sera semblable [à l'oblation pour] le péché ; il y aura une même loi pour les deux : [et la victime] appartiendra au Sacrificateur qui aura fait propitiation par elle.
\VS{8}Et le Sacrificateur qui offre l'holocauste pour quelqu'un, aura la peau de l'holocauste qu'il aura offert.
\VS{9}Et tout gâteau cuit au four, et qui sera apprêté en la poêle, ou sur la plaque, appartiendra au Sacrificateur qui l'offre.
\VS{10}Mais tout gâteau pétri à l'huile, ou sec, sera également pour tous les fils d'Aaron.
\VS{11}Et c'est ici la Loi du Sacrifice de prospérités qu'on offrira à l'Eternel.
\VS{12}Si quelqu'un l'offre pour rendre grâce, il offrira avec le Sacrifice d'action de grâces des tourteaux sans levain pétris à l'huile, et des beignets sans levain oints d'huile, et de la fleur de farine rissolée en tourteaux pétris à l'huile.
\VS{13}Il offrira avec ces tourteaux du pain levé pour son offrande avec le sacrifice d'action de grâces pour ses prospérités.
\VS{14}Et il en offrira une pièce de toutes les espèces qu'il offrira pour oblation élevée à l'Eternel ; [et] cela appartiendra au Sacrificateur qui répandra le sang du sacrifice de prospérités.
\VS{15}Mais la chair du sacrifice d'action de grâces de ses prospérités sera mangée le jour qu'elle sera offerte ; on n'en laissera rien jusqu'au matin.
\VS{16}Que si le sacrifice de son offrande est un vœu, ou une offrande volontaire, il sera mangé le jour qu'on aura offert son sacrifice ; et s'il y en a quelque reste, on le mangera le lendemain.
\VS{17}Mais ce qui sera demeuré de reste de la chair du sacrifice, sera brûlé au feu le troisième jour.
\VS{18}Que si on mange de la chair du sacrifice de ses prospérités le troisième jour, celui qui l'aura offert ne sera point agréé, [et] il ne lui sera point imputé comme une bonne action, ce sera une abomination, et la personne qui en aura mangé, portera son iniquité.
\VS{19}Et la chair [de ce sacrifice] qui aura touché quelque chose de souillé, ne se mangera point, elle sera brûlée au feu ; mais quiconque sera net, mangera de cette chair.
\VS{20}Car une personne qui mangera de la chair du sacrifice de prospérités, laquelle appartient à l'Eternel, et qui aura sur soi quelque souillure, cette personne-là sera retranchée d'entre ses peuples.
\VS{21}Si une personne touche quelque chose de souillé, soit souillure d'homme, soit bête souillée, ou telle autre chose souillée, et qu'il mange de la chair du sacrifice de prospérités, laquelle appartient à l'Eternel, cette personne-là sera retranchée d'entre ses peuples.
\VS{22}L'Eternel parla aussi à Moïse, en disant :
\VS{23}Parle aux enfants d'Israël, et leur dis : Vous ne mangerez aucune graisse de bœuf ni d'agneau, ni de chèvre.
\VS{24}On pourra bien se servir pour tout autre usage de la graisse d'une bête morte, ou de la graisse d'une bête déchirée [par les bêtes sauvages], mais vous n'en mangerez point.
\VS{25}Car quiconque mangera de la graisse d'une bête dont on offre [la graisse] en sacrifice par feu à l'Eternel, la personne qui en aura mangé, sera retranchée d'entre ses peuples.
\VS{26}Dans aucune de vos demeures vous ne mangerez point de sang, soit d'oiseaux, soit d'autres bêtes.
\VS{27}Toute personne qui aura mangé de quelque sang que ce soit, sera retranchée d'entre ses peuples.
\VS{28}L'Eternel parla aussi à Moïse, en disant :
\VS{29}Parle aux enfants d'Israël, et leur dis : Celui qui offrira le sacrifice de ses prospérités à l'Eternel, apportera à l'Eternel son offrande, [qu'il prendra] du sacrifice de ses prospérités.
\VS{30}Ses mains, [dis-je], apporteront les offrandes de l'Eternel qui se font par feu, [c'est à savoir] la graisse, il l'apportera avec la poitrine, [offrant] la poitrine pour la tournoyer en offrande tournoyée devant l'Eternel.
\VS{31}Puis le Sacrificateur fera fumer la graisse sur l'autel ; mais la poitrine appartiendra à Aaron et à ses fils.
\VS{32}Vous donnerez aussi au Sacrificateur pour offrande élevée, l'épaule droite de vos sacrifices de prospérités.
\VS{33}Celui d'entre les fils d'Aaron qui offrira le sang et la graisse des sacrifices de prospérités, aura pour sa part l'épaule droite.
\VS{34}Car j'ai pris des enfants d'Israël la poitrine de tournoiement, et l'épaule d'élévation, de tous les sacrifices de leurs prospérités, et je les ai données à Aaron Sacrificateur, et à ses fils, par ordonnance perpétuelle, [les ayant prises] des enfants d'Israël.
\VS{35}C'est là [le droit] de l'onction d'Aaron et de l'onction de ses fils sur ces offrandes de l'Eternel faites par feu, depuis le jour qu'on les aura présentés pour exercer la sacrificature à l'Eternel.
\VS{36}Et ce que l'Eternel a commandé qui leur fût donné par les enfants d'Israël, depuis le jour qu'on les aura oints, pour être une ordonnance perpétuelle dans leurs âges.
\VS{37}Telle est donc la loi de l'holocauste, du gâteau, [du sacrifice pour] le péché, et du [sacrifice pour le] délit, et des consécrations, et du sacrifice de prospérités.
\VS{38}Laquelle l'Eternel ordonna à Moïse sur la montagne de Sinaï ; lorsqu'il commanda aux enfants d'Israël d'offrir leurs offrandes à l'Eternel dans le désert de Sinaï.
\Chap{8}
\VerseOne{}L'Eternel parla aussi à Moïse en disant :
\VS{2}Prends Aaron et ses fils avec lui, les vêtements, l'huile d'onction, et un veau [pour le sacrifice] pour le péché, deux béliers, et une corbeille de pains sans levain.
\VS{3}Et convoque toute l'assemblée à l'entrée du Tabernacle d'assignation.
\VS{4}Et Moïse fit comme l'Eternel lui avait commandé ; et l'assemblée fut convoquée à l'entrée du Tabernacle d'assignation.
\VS{5}Et Moïse dit à l'assemblée : C'[est] ici ce que l'Eternel a commandé de faire.
\VS{6}Et Moïse fit approcher Aaron et ses fils, et les lava avec de l'eau.
\VS{7}Et il mit sur Aaron la chemise, et le ceignit du baudrier, et le revêtit du Rochet, et mit sur lui l'Ephod, et le ceignit avec le ceinturon exquis de l'Ephod, dont il le ceignit par dessus.
\VS{8}Puis il mit sur lui le Pectoral, après avoir mis au Pectoral Urim et Thummim.
\VS{9}Il lui mit aussi la tiare sur la tête, et il mit sur le devant de la tiare la lame d'or, qui est la couronne de sainteté, comme l'Eternel l'avait commandé à Moïse.
\VS{10}Puis Moïse prit l'huile de l'onction, et oignit le Tabernacle, et toutes les choses qui y étaient, et les sanctifia.
\VS{11}Et il en fit aspersion sur l'autel par sept fois, et en oignit l'autel, tous ses ustensiles, la cuve, et son soubassement, pour les sanctifier.
\VS{12}Il versa aussi de l'huile de l'onction sur la tête d'Aaron, et l'oignit pour le sanctifier.
\VS{13}Puis Moïse ayant fait approcher les fils d'Aaron, les revêtit des chemises, et les ceignit de baudriers, et leur attacha des calottes, comme l'Eternel l'avait commandé à Moïse.
\VS{14}Alors il fit approcher le veau de l'offrande pour le péché, et Aaron et ses fils posèrent leurs mains sur la tête du veau de [l'offrande pour le] péché.
\VS{15}Et Moïse l'ayant égorgé prit de son sang, et en mit avec son doigt sur les cornes de l'autel à l'entour, et fit propitiation pour l'autel, et il répandit le reste du sang au pied de l'autel ; ainsi il le sanctifia pour faire la propitiation sur lui.
\VS{16}Puis il prit toute la graisse qui était sur les entrailles, et la taie du foie, les deux rognons avec leur graisse, et Moïse les fit fumer sur l'autel.
\VS{17}Mais il fit brûler au feu hors du camp le veau avec sa peau, sa chair, et sa fiente, comme l'Eternel l'avait commandé à Moïse.
\VS{18}Il fit aussi approcher le bélier de l'holocauste, et Aaron et ses fils posèrent leurs mains sur la tête du bélier.
\VS{19}Et Moïse l'ayant égorgé, répandit le sang sur l'autel tout à l'entour.
\VS{20}Puis il mit le [bélier] en pièces, et en fit fumer la tête, les pièces, et la fressure.
\VS{21}Et il lava dans l'eau les entrailles et les jambes, et fit fumer tout le bélier sur l'autel ; car c'était un holocauste de bonne odeur, c'était une offrande faite par feu à l'Eternel, comme l'Eternel l'avait commandé à Moïse.
\VS{22}Il fit aussi approcher l'autre bélier, [savoir] le bélier des consécrations ; et Aaron et ses fils posèrent les mains sur la tête du bélier.
\VS{23}Et Moïse l'ayant égorgé prit de son sang, et le mit sur le mol de l'oreille droite d'Aaron, et sur le pouce de sa main droite, et sur le gros orteil de son pied droit.
\VS{24}Il fit aussi approcher les fils d'Aaron, et mit du même sang sur le mol de leur oreille droite, et sur le pouce de leur main droite, et sur le gros orteil de leur pied droit, et il répandit le reste du sang sur l'autel tout à l'entour.
\VS{25}Après il prit la graisse, la queue, et toute la graisse qui [est] sur les entrailles, et la taie du foie, et les deux rognons avec leur graisse, et l'épaule droite.
\VS{26}Il prit aussi de la corbeille des pains sans levain qui étaient devant l'Eternel, un gâteau sans levain, et un gâteau de pain fait à l'huile, et un beignet, et les mit sur les graisses, et sur l'épaule droite.
\VS{27}Puis il mit toutes ces choses sur les paumes des mains d'Aaron, et sur les paumes des mains de ses fils, et les tournoya en offrande tournoyée devant l'Eternel.
\VS{28}Après Moïse les reçut d'entre leurs mains, et les fit fumer sur l'autel par dessus l'holocauste ; car ce sont les consécrations de bonne odeur, c'est un sacrifice fait par feu à l'Eternel.
\VS{29}Moïse prit aussi la poitrine du bélier des consécrations, et la tournoya devant l'Eternel, et ce fut la portion de Moïse, comme l'Eternel l'avait commandé à Moïse.
\VS{30}Et Moïse prit de l'huile de l'onction, et du sang qui était sur l'autel, et il en fit aspersion sur Aaron et sur ses vêtements ; sur ses fils, et sur les vêtements de ses fils avec lui ; ainsi il sanctifia Aaron [et] ses vêtements ; ses fils, et les vêtements de ses fils avec lui.
\VS{31}Après cela, Moïse dit à Aaron et à ses fils : Faites bouillir la chair à l'entrée du Tabernacle d'assignation, et vous la mangerez là, avec le pain qui est dans la corbeille des consécrations, comme [le Seigneur] me l'a commandé, en disant : Aaron et ses fils mangeront ces choses.
\VS{32}Mais vous brûlerez au feu ce qui sera demeuré de reste de la chair et du pain.
\VS{33}Et vous ne sortirez point pendant sept jours de l'entrée du Tabernacle d'assignation, jusqu'au temps que les jours de vos consécrations soient accomplis ; car on emploiera sept jours à vous consacrer.
\VS{34}L'Eternel a commandé de faire [en ces autres jours] comme on a fait en celui-ci, pour faire la propitiation en votre faveur.
\VS{35}Vous demeurerez donc pendant sept jours à l'entrée du Tabernacle d'assignation, jour et nuit, et vous observerez ce que l'Eternel vous a ordonné d'observer, afin que vous ne mouriez point ; car il m'a été ainsi commandé.
\VS{36}Ainsi Aaron et ses fils firent toutes les choses que l'Eternel avait commandées par le moyen de Moïse.
\Chap{9}
\VerseOne{}Et il arriva au huitième jour que Moïse appela Aaron et ses fils, et les Anciens d'Israël.
\VS{2}Et il dit à Aaron : Prends un veau du troupeau [pour l'offrande] pour le péché, et un bélier pour l'holocauste, [tous deux] sans tare, et les amène devant l'Eternel.
\VS{3}Et tu parleras aux enfants d'Israël, en disant : Prenez un jeune bouc [pour l'offrande] pour le péché, un veau et un agneau, [tous deux] d'un an, qui soient sans tare, pour l'holocauste ;
\VS{4}Et un taureau et un bélier pour le sacrifice de prospérités, pour sacrifier devant l'Eternel, et un gâteau pétri à l'huile ; car aujourd'hui l'Eternel vous apparaîtra.
\VS{5}Ils prirent donc les choses que Moïse avait commandées, [et les amenèrent] devant le Tabernacle d'assignation ; et toute l'assemblée s'approcha, et se tint devant l'Eternel.
\VS{6}Et Moïse dit : Faites ce que l'Eternel vous a commandé, et la gloire de l'Eternel vous apparaîtra.
\VS{7}Et Moïse dit à Aaron : Approche-toi de l'autel, et fais ton offrande pour le péché, et ton holocauste, et fais propitiation pour toi, et pour le peuple ; et présente l'offrande pour le peuple ; et fais propitiation pour eux, comme l'Eternel l'a commandé.
\VS{8}Alors Aaron s'approcha de l'autel, et égorgea le veau [de son offrande pour le] péché.
\VS{9}Et les fils d'Aaron lui présentèrent le sang, et il trempa son doigt dans le sang, et le mit sur les cornes de l'autel ; puis il répandit le reste du sang au pied de l'autel.
\VS{10}Mais il fit fumer sur l'autel la graisse, et les rognons, et la taie du foie de [l'offrande pour le] péché, comme l'Eternel l'avait commandé à Moïse.
\VS{11}Et il brûla au feu la chair et la peau hors du camp.
\VS{12}Il égorgea aussi l'holocauste, et les fils d'Aaron lui présentèrent le sang, lequel il répandit sur l'autel tout à l'entour.
\VS{13}Puis ils lui présentèrent l'holocauste [coupé] par pièces, et la tête, et il fit fumer [ces choses-là] sur l'autel.
\VS{14}Et il lava les entrailles et les jambes, qu'il fit fumer au dessus de l'holocauste sur l'autel.
\VS{15}Il offrit l'offrande du peuple, il prit le bouc de [l'offrande pour] le péché pour le peuple, il l'égorgea, et l'offrit pour le péché, comme la première [offrande].
\VS{16}Il l'offrit en holocauste, et agit selon l'ordonnance.
\VS{17}Ensuite il offrit l'oblation du gâteau, et il en remplit la paume de sa main, et la fit fumer sur l'autel, outre l'holocauste du matin.
\VS{18}Il égorgea aussi le taureau et le bélier pour le sacrifice de prospérités, qui était pour le peuple ; et les fils d'Aaron lui présentèrent le sang, lequel il répandit sur l'autel tout à l'entour.
\VS{19}[Ils présentèrent] les graisses du taureau et du bélier, la queue, et ce qui couvre [les entrailles], et les rognons, et la taie du foie.
\VS{20}Et ils mirent les graisses sur les poitrines, et on fit fumer les graisses sur l'autel.
\VS{21}Et Aaron tournoya en offrande tournoyée devant l'Eternel les poitrines et l'épaule droite, comme l'Eternel l'avait commandé à Moïse.
\VS{22}Aaron éleva aussi ses mains vers le peuple, et les bénit ; et il descendit après avoir offert [l'offrande pour] le péché, l'holocauste, et [le sacrifice de] prospérités.
\VS{23}Moïse donc et Aaron vinrent au Tabernacle d'assignation, et étant sortis ils bénirent le peuple ; et la gloire de l'Eternel apparut à tout le peuple.
\VS{24}Car le feu sortit de devant l'Eternel, et consuma sur l'autel l'holocauste et les graisses ; ce que tout le peuple ayant vu, ils s'écrièrent de joie, et se prosternèrent le visage contre terre.
\Chap{10}
\VerseOne{}Or les enfants d'Aaron, Nadab et Abihu, prirent chacun leur encensoir, et y ayant mis du feu, ils mirent dessus du parfum, et offrirent devant l'Eternel un feu étranger ; ce qu'il ne leur avait point commandé.
\VS{2}Et le feu sortit de devant l'Eternel ; et les dévora ; et ils moururent devant l'Eternel.
\VS{3}Et Moïse dit à Aaron : C'est ce dont l'Eternel avait parlé, en disant : Je serai sanctifié en ceux qui s'approchent de moi, et je serai glorifié en la présence de tout le peuple ; et Aaron se tut.
\VS{4}Et Moïse appela Misaël et Eltsaphan les fils de Huziel, oncle d'Aaron, et leur dit : Approchez-vous, emportez vos frères de devant le Sanctuaire hors du camp.
\VS{5}Alors ils s'approchèrent, et les emportèrent avec leurs chemises hors du camp, comme Moïse en avait parlé.
\VS{6}Puis Moïse dit à Aaron, et à Eléazar et à Ithamar ses fils : Ne découvrez point vos têtes, et ne déchirez point vos vêtements, et vous ne mourrez point, et [l'Eternel] ne se mettra point en colère contre toute l'assemblée, mais que vos frères, toute la maison d'Israël, pleurent à cause de l'embrasement que l'Eternel a fait.
\VS{7}Et ne sortez point de l'entrée du Tabernacle d'assignation, de peur que vous ne mouriez ; car l'huile de l'onction de l'Eternel [est] sur vous ; et ils firent selon la parole de Moïse.
\VS{8}Et l'Eternel parla à Aaron, en disant :
\VS{9}Vous ne boirez point de vin ni de cervoise, toi, ni tes fils avec toi, quand vous entrerez au Tabernacle d'assignation, de peur que vous ne mouriez ; c'est une ordonnance perpétuelle en vos âges.
\VS{10}Afin que vous puissiez discerner entre ce qui est saint ou profane, entre ce qui est souillé ou net ;
\VS{11}Et afin que vous enseigniez aux enfants d'Israël toutes les ordonnances que l'Eternel leur aura prononcées par le moyen de Moïse.
\VS{12}Puis Moïse parla à Aaron, et à Eléazar et à Ithamar ses fils, qui étaient demeurés de reste. Prenez, [leur dit-il,] l'offrande du gâteau qui [est] demeuré de reste des offrandes de l'Eternel faites par feu, et la mangez en pains sans levain auprès de l'autel ; car c'[est] une chose très-sainte.
\VS{13}Vous la mangerez dans un lieu saint ; parce que c'est la portion qui est assignée à toi, et à tes fils, des offrandes faites par feu à l'Eternel ; car il m'a été ainsi commandé.
\VS{14}Vous mangerez aussi la poitrine de tournoiement, et l'épaule d'élévation dans un lieu pur, toi, tes fils, et tes filles avec toi ; car ces choses-là t'ont été données des sacrifices de prospérités des enfants d'Israël, pour ta portion, et pour celle de tes enfants.
\VS{15}Ils apporteront l'épaule d'élévation, et la poitrine de tournoiement, avec les offrandes faites par feu, [qui sont] les graisses pour les faire tourner en offrande tournoyée devant l'Eternel ; et [cela] t'appartiendra et à tes fils avec toi par une ordonnance perpétuelle, comme l'Eternel [l'a] commandé.
\VS{16}Or Moïse cherchait soigneusement le bouc [de l'offrande pour] le péché, mais voici il avait été brûlé, et [Moïse] se mit en grande colère contre Eléazar et Ithamar, les fils d'Aaron qui étaient demeurés de reste, [et] leur dit :
\VS{17}Pourquoi n'avez-vous point mangé [l'offrande pour] le péché dans un lieu saint ; car c'est une chose très-sainte ; vu qu'elle vous a été donnée pour porter l'iniquité de l'assemblée, afin de faire propitiation pour eux devant l'Eternel.
\VS{18}Voici, son sang n'a point été porté dans le Sanctuaire ; ne manquez donc [plus] à la manger dans le lieu saint, comme je l'avais commandé.
\VS{19}Alors Aaron répondit à Moise : Voici, ils ont offert aujourd'hui leur [offrande pour le] péché et leur holocauste devant l'Eternel, et ces choses-ci me sont arrivées. Si j'eusse mangé aujourd'hui [l'offrande pour le] péché, cela eût-il plu à l'Eternel ?
\VS{20}Ce que Moïse ayant entendu, il l'approuva.
\Chap{11}
\VerseOne{}Et l'Eternel parla à Moïse et à Aaron, et leur dit :
\VS{2}Parlez aux enfants d'Israël, et leur dites : Ce sont ici les animaux dont vous mangerez d'entre toutes les bêtes qui sont sur la terre.
\VS{3}Vous mangerez d'entre les bêtes à quatre pieds de toutes celles qui ont l'ongle divisé, et qui ont le pied fourché, et qui ruminent.
\VS{4}Mais vous ne mangerez point de celles qui ruminent [seulement], ou qui ont l'ongle divisé [seulement] ; comme le chameau, car il rumine bien, mais il n'a point l'ongle divisé ; il vous [est] souillé.
\VS{5}Et le lapin ; car il rumine bien, mais il n'a point l'ongle divisé ; il vous est souillé.
\VS{6}Et le lièvre ; car il rumine bien, mais il n'a point l'ongle divisé ; il vous est souillé.
\VS{7}Et le pourceau ; car il a bien l'ongle divisé, et le pied fourché, mais il ne rumine nullement ; il vous est souillé.
\VS{8}Vous ne mangerez point de leur chair, même vous ne toucherez point leur chair morte ; ils vous sont souillés.
\VS{9}Vous mangerez de ceci d'entre tout ce qui est dans les eaux ; vous mangerez de tout ce qui a des nageoires et des écailles dans les eaux, soit dans la mer, soit dans les fleuves.
\VS{10}Mais vous ne mangerez de rien qui n'ait point de nageoires et d'écailles, soit dans la mer, soit dans les fleuves, tant des reptiles des eaux, que de toute chose vivante qui est dans les eaux ; cela vous sera en abomination.
\VS{11}Elles vous seront donc en abomination, vous ne mangerez point de leur chair, et vous tiendrez pour une chose abominable leur chair morte.
\VS{12}Tout ce donc qui vit dans les eaux, et qui n'a point de nageoires et d'écailles, vous sera en abomination.
\VS{13}Et d'entre les oiseaux vous tiendrez ceux-ci pour abominables, on n'en mangera point, ils vous seront en abomination : l'Aigle, l'Orfraie, le Faucon.
\VS{14}Le Vautour, et le Milan, selon leur espèce ;
\VS{15}Tout Corbeau selon son espèce ;
\VS{16}Le Chat-huant, la Hulotte, le Coucou, et l'Epervier, selon leur espèce ;
\VS{17}La Chouette, le Plongeon, le Hibou,
\VS{18}Le Cygne, le Cormoran, le Pélican,
\VS{19}La Cigogne, et le Héron, selon leur espèce, et la Huppe, et la Chauve-souris.
\VS{20}Et tout reptile volant qui marche sur quatre pieds, vous sera en abomination.
\VS{21}Mais vous mangerez de tout reptile volant qui marche à quatre pieds, ayant des jambes sur ses pieds, pour sauter avec elles sur la terre.
\VS{22}Ce sont donc ici ceux dont vous mangerez ; l'Arbé selon son espèce, le Solham selon son espèce, l'Hargol selon son espèce ; et le Kabag selon son espèce.
\VS{23}Mais tout autre reptile volant qui a quatre pieds, vous sera en abomination.
\VS{24}Vous serez donc souillés par ces bêtes ; quiconque touchera leur chair morte, sera souillé jusqu'au soir.
\VS{25}Quiconque aussi portera de leur chair morte, lavera ses vêtements, et sera souillé jusqu'au soir.
\VS{26}Toute bête qui a l'ongle divisé, et qui n'a point le pied fourché, et ne rumine point, vous sera souillée ; quiconque les touchera, sera souillé.
\VS{27}Et tout ce qui marche sur ses pattes, entre tous les animaux qui marchent à quatre pieds, vous sera souillé ; quiconque touchera leur chair morte, sera souillé jusqu'au soir.
\VS{28}Et celui qui portera de leur chair morte, lavera ses vêtements, et sera souillé jusqu'au soir ; elles vous sont souillées.
\VS{29}Ceci aussi vous sera souillé entre les reptiles, qui rampent sur la terre, la Belette, la Souris, et la Tortue, selon leur espèce.
\VS{30}Le Hérisson, le Crocodile, le Lézard, la Limace, et la Taupe.
\VS{31}Ces choses vous sont souillées entre les reptiles ; quiconque les touchera mortes, sera souillé jusqu'au soir.
\VS{32}Aussi tout ce sur quoi il en tombera quelque chose, quand elles seront mortes, sera souillé, soit vaisseau de bois, soit vêtement, soit peau, ou sac, quelque vaisseau que ce soit dont on se sert à faire quelque chose, sera mis dans l'eau, et sera souillé jusqu'au soir, puis il sera net.
\VS{33}Mais s'il en tombe quelque chose dans quelque vaisseau de terre que ce soit, tout ce qui est dedans sera souillé, et vous casserez le vaisseau.
\VS{34}Et toute viande qu'on mange, sur laquelle il y aura eu de l'eau, sera souillée ; tout breuvage qu'on boit dans quelque vaisseau que ce soit, en sera souillé.
\VS{35}Et s'il tombe quelque chose de leur chair morte sur quoi que ce soit, cela sera souillé ; le four et le foyer seront abattus ; ils sont souillés, et ils vous seront souillés.
\VS{36}Toutefois la fontaine, ou le puits, [ou tel autre] amas d'eaux seront nets. Celui donc qui touchera leur chair morte, sera souillé.
\VS{37}Et s'il est tombé de leur chair morte sur quelque semence qui se sème, elle sera nette.
\VS{38}Mais si on avait mis de l'eau sur la semence, et que quelque chose de leur chair morte tombe sur elle, elle vous sera souillée.
\VS{39}Et quand quelqu'une des bêtes qui vous sont pour viande, sera morte [d'elle-même], celui qui en touchera la chair morte, sera souillé jusqu'au soir.
\VS{40}Et celui qui aura mangé de sa chair morte, lavera ses vêtements, et sera souillé jusqu'au soir. Celui aussi qui portera la chair morte de cette bête, lavera ses vêtements, et sera souillé jusqu'au soir.
\VS{41}Tout reptile donc qui rampe sur la terre, vous sera en abomination, [et] on n'en mangera point.
\VS{42}Vous ne mangerez point de tout ce qui rampe sur la poitrine, ni de tout ce qui marche sur les quatre pieds, ni de tout ce qui a plusieurs pieds entre tous les reptiles qui se traînent sur la terre ; car ils sont en abomination.
\VS{43}Ne rendez point vos personnes abominables par aucun reptile qui se traîne, et ne vous souillez point par eux : car vous seriez souillés par eux.
\VS{44}Parce que je suis l'Eternel votre Dieu. Vous vous sanctifierez donc, et vous serez saints ; car je suis saint ; ainsi vous ne souillerez point vos personnes par aucun reptile qui se traîne sur la terre.
\VS{45}Car je suis l'Eternel, qui vous ai fait monter du pays d'Egypte, afin que je sois votre Dieu, et que vous soyez saints ; car je suis saint.
\VS{46}Telle est la Loi touchant les bêtes, et les oiseaux, et tout animal ayant vie, qui se meut dans les eaux, et toute chose ayant vie, qui se traîne sur la terre.
\VS{47}Afin de discerner entre la chose souillée et la chose nette, et entre les animaux qu'on peut manger, et les animaux dont on ne doit point manger.
\Chap{12}
\VerseOne{}L'Eternel parla aussi à Moïse, en disant :
\VS{2}Parle aux enfants d'Israël, et leur dis : Si la femme après avoir conçu, enfante un mâle, elle sera souillée pendant sept jours ; elle sera souillée comme au temps de ses mois.
\VS{3}Et au huitième jour on circoncira la chair du prépuce de l'enfant.
\VS{4}Et elle demeurera trente-trois jours au sang de sa purification, et ne touchera aucune chose sainte, et ne viendra point au Sanctuaire, jusqu'à ce que les jours de sa purification soient accomplis.
\VS{5}Que si elle enfante une fille, elle sera souillée deux semaines, comme au temps de ses mois, et elle demeurera soixante-six jours au sang de sa purification.
\VS{6}Après que le temps de sa purification sera accompli, soit pour fils, ou pour fille, elle présentera au Sacrificateur un agneau d'un an en holocauste, et un pigeonneau, ou une tourterelle, [en offrande] pour le péché, à l'entrée du Tabernacle d'assignation.
\VS{7}Et le Sacrificateur offrira ces choses devant l'Eternel, et fera propitiation pour elle, et elle sera nettoyée du flux de son sang. Telle est la Loi de celle qui enfante un fils ou une fille.
\VS{8}Que si elle n'a pas le moyen de trouver un agneau, alors elle prendra deux tourterelles, ou deux pigeonneaux, l'un pour l'holocauste, et l'autre [en offrande] pour le péché, et le Sacrificateur fera propitiation pour elle, et elle sera nettoyée.
\Chap{13}
\VerseOne{}L'Eternel parla aussi à Moïse et à Aaron, en disant :
\VS{2}L'homme qui aura dans la peau de sa chair une tumeur, ou gâle, ou bouton, et que cela paraîtra dans la peau de sa chair comme une plaie de lèpre, on l'amènera à Aaron Sacrificateur, ou à un de ses fils Sacrificateurs.
\VS{3}Et le Sacrificateur regardera la plaie qui est dans la peau de sa chair, et si le poil de la plaie est devenu blanc, et si la plaie, à la voir, est plus enfoncée que la peau de sa chair, c'est une plaie de lèpre ; le Sacrificateur donc le regardera, et le jugera souillé.
\VS{4}Mais si le bouton est blanc en la peau de sa chair, et qu'à le voir il ne soit point plus enfoncé que la peau, et si son poil n est pas devenu blanc, le Sacrificateur fera enfermer pendant sept jours celui qui a la plaie.
\VS{5}Et le Sacrificateur la regardera le septième jour, et s'il aperçoit que la plaie se soit arrêtée, et qu'elle n'ait point crû dans la peau, le Sacrificateur le fera renfermer pendant sept autres jours.
\VS{6}Et le Sacrificateur la regardera encore le septième jour suivant, et s'il aperçoit que la plaie s'est retirée, et qu'elle ne s'est point accrue sur la peau, le Sacrificateur le jugera net ; c'est de la gâle, et il lavera ses vêtements, et sera net.
\VS{7}Mais si la gâle a crû en quelque sorte que ce soit sur la peau, après qu'il aura été examiné par le Sacrificateur pour être jugé net, et qu'il aura été examiné pour la seconde fois par le Sacrificateur ;
\VS{8}Le Sacrificateur le regardera encore, et s'il aperçoit que la gâle ait crû sur la peau, le Sacrificateur le jugera souillé ; c'est de la lèpre.
\VS{9}Quand il y aura une plaie de lèpre en un homme ; on l'amènera au Sacrificateur.
\VS{10}Lequel le regardera ; et s'il aperçoit qu'il y ait une tumeur blanche en la peau, et que le poil soit devenu blanc, et qu'il paraisse de la chair vive en la tumeur ;
\VS{11}C'est une lèpre invétérée en la peau de sa chair, et le Sacrificateur le jugera souillé, et ne le fera point enfermer ; car il est jugé souillé.
\VS{12}Si la lèpre boutonne fort dans la peau, et qu'elle couvre toute la peau de la plaie, depuis la tête de cet homme jusqu'à ses pieds, autant qu'en pourra voir le Sacrificateur ;
\VS{13}Le Sacrificateur le regardera, et s'il aperçoit que la lèpre ait couvert toute la chair de cet homme, alors il jugera net [celui qui a] la plaie ; la plaie est devenue toute blanche ; il est net.
\VS{14}Mais le jour auquel on aura aperçu de la chair vive, il sera souillé.
\VS{15}Alors le Sacrificateur regardera la chair vive, et le jugera souillé ; la chair vive est souillée ; c'est de la lèpre.
\VS{16}Que si la chair vive se change, et devient blanche, alors il viendra vers le Sacrificateur.
\VS{17}Et le Sacrificateur le regardera, et s'il aperçoit que la plaie soit devenue blanche, le Sacrificateur jugera net [celui qui a] la plaie : il est net.
\VS{18}Si la chair a eu en sa peau un ulcère, qui soit guéri ;
\VS{19}Et qu'à l'endroit où était l'ulcère il y ait une tumeur blanche, ou une pustule blanche-roussâtre, il sera regardé par le Sacrificateur.
\VS{20}Le Sacrificateur donc la regardera, et s'il aperçoit qu'à la voir elle soit plus enfoncée que la peau, et que son poil soit devenu blanc, alors le Sacrificateur le jugera souillé ; c'est une plaie de lèpre, la lèpre a boutonné dans l'ulcère.
\VS{21}Que si le Sacrificateur la regardant aperçoit que le poil ne soit point devenu blanc, et qu'elle ne soit point plus enfoncée que la peau ; mais qu'elle se soit retirée, le Sacrificateur le fera enfermer pendant sept jours.
\VS{22}Que si elle s'est étendue en quelque sorte que ce soit sur la peau, le Sacrificateur le jugera souillé ; c'est une plaie.
\VS{23}Mais si le bouton s'arrête en son lieu, ne croissant point, c'est un feu d'ulcère ; ainsi le Sacrificateur le jugera net.
\VS{24}Que si la chair a en sa peau une inflammation de feu, et que la chair vive de la partie enflammée soit un bouton blanc-roussâtre, ou blanc [seulement] ;
\VS{25}Le Sacrificateur le regardera, et s'il aperçoit que le poil soit devenu blanc dans le bouton, et qu'à le voir il soit plus enfoncé que la peau, c'est de la lèpre, elle a boutonné dans l'inflammation ; le Sacrificateur donc le jugera souillé ; c'est une plaie de lèpre.
\VS{26}Mais si le Sacrificateur le regardant aperçoit qu'il n'y a point de poil blanc au bouton, et qu'il n'est point plus bas que la peau, et qu'il s'est retiré, le Sacrificateur le fera enfermer pendant sept jours.
\VS{27}Puis le Sacrificateur le regardera le septième jour, [et] si [le bouton] a crû en quelque sorte que ce soit dans la peau, le Sacrificateur le jugera souillé ; c'est une plaie de lèpre.
\VS{28}Que si le bouton s'arrête en son lieu sans croître sur la peau, et s'est retiré, c'est une tumeur d'inflammation ; et le Sacrificateur le jugera net ; c'est un feu d'inflammation.
\VS{29}Si l'homme ou la femme a une plaie en la tête, ou [l'homme] en la barbe,
\VS{30}Le Sacrificateur regardera la plaie, et si à la voir elle est plus enfoncée que la peau, ayant en soi du poil jaunâtre délié, le Sacrificateur le jugera souillé ; c'est de la teigne, c'est une lèpre de tête, ou de barbe.
\VS{31}Et si le Sacrificateur regardant la plaie de la teigne, aperçoit, qu'à la voir elle n'est point plus enfoncée que la peau, et n'a en soi aucun poil noir, le Sacrificateur fera enfermer pendant sept jours [celui qui a] la plaie de la teigne ;
\VS{32}Et le septième jour le Sacrificateur regardera la plaie, et s'il aperçoit que la teigne ne s'est point étendue, et qu'elle n'a [aucun] poil jaunâtre, et qu'à voir la teigne elle ne soit pas plus enfoncée que la peau ;
\VS{33}[Celui qui a la plaie de la teigne] se rasera, mais il ne rasera point [l'endroit] de la teigne, et le Sacrificateur fera enfermer pendant sept autres jours [celui qui a] la teigne.
\VS{34}Puis le Sacrificateur regardera la teigne au septième jour, et s'il aperçoit que la teigne ne s'est point étendue sur la peau, et qu'à la voir elle n'est point plus enfoncée que la peau, le Sacrificateur le jugera net, et cet homme lavera ses vêtements, et sera net.
\VS{35}Mais si la teigne croît en quelque sorte que ce soit dans la peau, après sa purification,
\VS{36}Le Sacrificateur la regardera ; et s'il aperçoit que la teigne ait crû dans la peau, le Sacrificateur ne cherchera point de poil jaunâtre ; il est souillé.
\VS{37}Mais s'il aperçoit que la teigne se soit arrêtée, et qu il y soit venu du poil noir, la teigne est guérie ; il est net, et le Sacrificateur le jugera net.
\VS{38}Et si l'homme ou la femme ont dans la peau de leur chair des boutons, des boutons, [dis-je], qui soient blancs,
\VS{39}Le Sacrificateur les regardera, et s'il aperçoit que dans la peau de leur chair il y ait des boutons retirés et blancs, c'est une tache blanche qui a boutonné dans la peau ; il est donc net.
\VS{40}Si l'homme a la tête pelée, il est chauve, [et néanmoins] il est net.
\VS{41}Mais si sa tête est pelée du côté de son visage, il est chauve, [et néanmoins] il est net.
\VS{42}Et si dans la partie pelée ou chauve, il y a une plaie blanche-roussâtre, c'est une lèpre qui a bourgeonné dans sa partie pelée ou chauve.
\VS{43}Et le Sacrificateur le regardera, et s'il aperçoit que la tumeur de la plaie soit blanche-roussâtre dans sa partie pelée ou chauve, semblable à la lèpre de la peau de la chair ;
\VS{44}L'homme est lépreux, il est souillé ; le Sacrificateur ne manquera pas de le juger souillé : sa plaie est en sa tête.
\VS{45}Or le lépreux en qui sera la plaie, aura ses vêtements déchirés, et sa tête nue, et il sera couvert sur la lèvre de dessus, et il criera : le souillé, le souillé.
\VS{46}Pendant tout le temps qu'il aura cette plaie, il sera jugé souillé ; il est souillé, il demeurera seul, et sa demeure sera hors du camp.
\VS{47}Et si le vêtement est infecté de la plaie de la lèpre, soit vêtement de laine, soit vêtement de lin ;
\VS{48}Ou dans la chaîne, ou dans la trame du lin, ou de la laine, ou aussi dans la peau, ou dans quelque ouvrage que ce soit de pelleterie.
\VS{49}Et si cette plaie est verte, ou roussâtre dans le vêtement, ou dans la peau, ou dans la chaîne, ou dans la trame, ou dans quelque chose que ce soit de peau, ce sera une plaie de lèpre, et elle sera montrée au Sacrificateur.
\VS{50}Et le Sacrificateur regardera la plaie, et fera enfermer pendant sept jours [celui qui a] la plaie.
\VS{51}Et au septième jour il regardera la plaie ; si la plaie est crue au vêtement, ou en la chaîne, ou en la trame, ou en la peau, ou en quelque ouvrage que ce soit de pelleterie, la plaie est une lèpre rongeante, elle est souillée.
\VS{52}Il brûlera donc le vêtement, la chaîne, ou la trame de laine, ou de lin, et toutes les choses de peau, qui auront cette plaie ; car c'est une lèpre rongeante, cela sera brûlé au feu.
\VS{53}Mais si le Sacrificateur regarde, et aperçoit que la plaie ne soit point crue au vêtement, ou en la chaîne, ou en la trame, ou en quelque [autre] chose qui soit faite de peau ;
\VS{54}Le Sacrificateur commandera qu'on lave la chose où est la plaie, et il le fera enfermer pendant sept autres jours.
\VS{55}Que si le Sacrificateur, après qu'on aura fait laver la plaie, la regarde, et s'il aperçoit que la plaie n'ait point changé sa couleur, et qu'elle ne soit point accrue, c'est une chose souillée, tu la brûleras au feu ; c'est une enfonçure en son envers, ou en son endroit pelé.
\VS{56}Que si le Sacrificateur regarde, et aperçoit que la plaie se soit retirée après qu'on l'a fait laver, il la déchirera du vêtement, ou de la peau, ou de la chaîne, ou de la trame.
\VS{57}Que si elle paraît encore au vêtement, ou dans la chaîne, ou dans la trame, ou dans quelque chose que ce soit de peau, c'est une lèpre qui a boutonné ; vous brûlerez au feu la chose où est la plaie.
\VS{58}Mais si tu as lavé le vêtement, ou la chaîne, ou la trame, ou quelque chose de peau, et que la plaie s'en soit allée, il sera encore lavé ; puis il sera net.
\VS{59}Telle est la loi de la plaie de la lèpre au vêtement de laine, ou de lin, ou en la chaîne, ou en la trame, ou en quelque chose que ce soit de peau, pour la juger nette, ou souillée.
\Chap{14}
\VerseOne{}L'Eternel parla aussi à Moïse, en disant :
\VS{2}C'est ici la loi du lépreux pour le jour de la purification ; il sera amené au Sacrificateur.
\VS{3}Et le Sacrificateur sortira hors du camp, et le regardera ; et s'il aperçoit que la plaie de la lèpre soit guérie au lépreux,
\VS{4}Le Sacrificateur commandera qu'on prenne pour celui qui doit être nettoyé, deux passereaux vivants [et] nets, avec du bois de cèdre, et du cramoisi, et de l'hysope,
\VS{5}Et le Sacrificateur commandera qu'on coupe la gorge à l'un des passereaux sur un vaisseau de terre, sur de l'eau vive.
\VS{6}Puis il prendra le passereau vivant, le bois de cèdre, le cramoisi, et l'hysope ; et il trempera [toutes ces choses] avec le passereau vivant, dans le sang de l'autre passereau qui aura été égorgé sur de l'eau vive.
\VS{7}Et il en fera aspersion par sept fois sur celui qui doit être nettoyé de la lèpre, et le nettoiera, et il laissera aller par les champs, le passereau vivant.
\VS{8}Et celui qui doit être nettoyé lavera ses vêtements, et rasera tout son poil, et se lavera dans l'eau, et il sera net, et ensuite il entrera au camp, mais il demeurera sept jours hors de sa tente.
\VS{9}Et au septième jour il rasera tout son poil, sa tête, sa barbe, les sourcils de ses yeux, tout son poil ; il rasera, [dis-je], tout son poil ; puis il lavera ses vêtements et sa chair, et il sera net.
\VS{10}Et au huitième jour il prendra deux agneaux sans tare, et une brebis d'un an sans tare, et trois dixièmes de fine farine à faire le gâteau, pétrie à l'huile, et un log d'huile.
\VS{11}Et le Sacrificateur qui fait la purification, présentera celui qui doit être nettoyé, et ces choses-là, devant l'Eternel à l'entrée du Tabernacle d'assignation.
\VS{12}Puis le Sacrificateur prendra l'un des agneaux, et l'offrira en offrande pour le délit avec un log d'huile, et tournoiera ces choses devant l'Eternel, en oblation tournoyée.
\VS{13}Et il égorgera l'agneau au lieu où l'on égorge [l'offrande] pour le péché, et l'holocauste, dans le lieu saint ; car [l'offrande pour] le délit appartient au Sacrificateur, comme [l'offrande pour] le péché ; c'est une chose très-sainte.
\VS{14}Et le Sacrificateur prendra du sang [de l'offrande pour] le délit, et le mettra sur le mol de l'oreille droite de celui qui doit être nettoyé, et sur le pouce de sa main droite, et sur le gros orteil de son pied droit.
\VS{15}Puis le Sacrificateur prendra de l'huile du log, et en versera dans la paume de sa main gauche.
\VS{16}Et le Sacrificateur trempera le doigt de sa main droite en l'huile qui est dans sa paume gauche, et fera aspersion de l'huile avec son doigt sept fois devant l'Eternel.
\VS{17}Et du reste de l'huile qui sera dans sa paume, le Sacrificateur en mettra sur le mol de l'oreille droite de celui qui doit être nettoyé, et sur le pouce de sa main droite, et sur le gros orteil de son pied droit, sur le sang pris de [l'offrande pour le] délit.
\VS{18}Mais ce qui restera de l'huile sur la paume du Sacrificateur, il le mettra sur la tête de celui qui doit être nettoyé ; et ainsi le Sacrificateur fera propitiation pour lui devant l'Eternel.
\VS{19}Ensuite le Sacrificateur offrira [l'offrande pour] le péché, et fera propitiation pour celui qui doit être nettoyé de sa souillure ; puis il égorgera l'holocauste.
\VS{20}Et le Sacrificateur offrira l'holocauste et le gâteau sur l'autel, et fera propitiation pour celui qui doit être nettoyé ; et il sera net.
\VS{21}Mais s'il est pauvre, et s'il n'a pas le moyen de fournir ces choses, il prendra un agneau en offrande tournoyée pour le délit, afin de faire propitiation pour soi, et un dixième de fine farine pétrie à l'huile, pour le gâteau, avec un log d'huile ;
\VS{22}Et deux tourterelles ou deux pigeonneaux, selon qu'il pourra fournir, dont l'un sera pour le péché, et l'autre pour l'holocauste.
\VS{23}Et le huitième jour de sa purification il les apportera au Sacrificateur à l'entrée du Tabernacle d'assignation, devant l'Eternel.
\VS{24}Et le Sacrificateur recevra l'agneau [de l'offrande pour] le délit, et le log d'huile, et les tournoiera devant l'Eternel en offrande tournoyée.
\VS{25}Et il égorgera l'agneau de [l'offrande pour le] délit ; puis le Sacrificateur prendra du sang de [l'offrande pour le] délit, et le mettra sur le mol de l'oreille droite de celui qui doit être nettoyé, et sur le pouce de sa main droite, et sur le gros orteil de son pied droit.
\VS{26}Puis le Sacrificateur versera de l'huile dans la paume de sa main gauche.
\VS{27}Et avec le doigt de sa main droite il fera aspersion de l'huile qui est dans sa main gauche, sept fois devant l'Eternel.
\VS{28}Et il mettra de cette huile qui est dans sa paume, sur le mol de l'oreille droite de celui qui doit être nettoyé, et sur le pouce de sa main droite, et sur le gros orteil de son pied droit, sur le lieu du sang pris de [l'offrande pour le] délit.
\VS{29}Après il mettra le reste de l'huile qui est dans sa paume sur la tête de celui qui doit être nettoyé, afin de faire propitiation pour lui devant l'Eternel.
\VS{30}Puis il sacrifiera l'une des tourterelles, ou l'un des pigeonneaux, selon ce qu'il aura pu fournir.
\VS{31}De ce donc qu'il aura pu fournir, l'un sera pour le péché, et l'autre pour l'holocauste, avec le gâteau ; ainsi le Sacrificateur fera propitiation devant l'Eternel pour celui qui doit être nettoyé.
\VS{32}Telle est la loi de celui auquel il y a une plaie de lèpre, et qui n'a pas le moyen de fournir à sa purification.
\VS{33}Puis l'Eternel parla à Moïse et à Aaron, en disant :
\VS{34}Quand vous serez entrés au pays de Canaan, que je vous donne en possession, si j'envoie une plaie de lèpre en quelque maison du pays que vous posséderez ;
\VS{35}Celui à qui la maison appartiendra viendra, et le fera savoir au Sacrificateur, en disant : Il me semble que j'aperçois comme une plaie en ma maison.
\VS{36}Alors le Sacrificateur commandera qu'on vide la maison avant qu'il y entre pour regarder la plaie ; afin que rien de ce qui est dans la maison ne soit souillé, puis le Sacrificateur entrera pour voir la maison.
\VS{37}Et il regardera la plaie, et s'il aperçoit que la plaie qui est aux parois de la maison, ait quelques fossettes tirant sur le vert, ou roussâtres, qui soient, à les voir, plus enfoncées que la paroi ;
\VS{38}Le Sacrificateur sortira de la maison, à l'entrée, et fera fermer la maison pendant sept jours.
\VS{39}Et au septième jour le Sacrificateur retournera, et la regardera, et s'il aperçoit que la plaie se soit étendue sur les parois de la maison ;
\VS{40}Alors il commandera d'arracher les pierres, auxquelles est la plaie, et de les jeter hors de la ville dans un lieu souillé.
\VS{41}Il fera aussi racler l'enduit de la maison par dedans tout à l'entour, et l'enduit qu'on aura raclé, on le jettera hors de la ville en un lieu souillé.
\VS{42}Puis on prendra d'autres pierres, et on les apportera au lieu des [premières] pierres, et on prendra d'autre mortier pour r'enduire la maison.
\VS{43}Mais si la plaie retourne et boutonne en la maison, après qu'on aura arraché les pierres, et après qu'on l'aura raclée, et r'enduite,
\VS{44}Le Sacrificateur y entrera, et la regardera, et s'il aperçoit que la plaie soit accrue en la maison, c'est une lèpre rongeante en la maison ; elle est souillée.
\VS{45}On démolira donc la maison, ses pierres, et son bois, avec tout son mortier, et on les transportera hors de la ville en un lieu souillé.
\VS{46}Et si quelqu'un est entré dans la maison, pendant tout le temps que le Sacrificateur l'avait faite fermer, il sera souillé jusqu'au soir.
\VS{47}Et celui qui dormira dans cette maison lavera ses vêtements ; celui aussi qui mangera dans cette maison lavera ses vêtements.
\VS{48}Mais quand le Sacrificateur y sera entré, et qu'il aura aperçu que la plaie n'a point crû en cette maison, après l'avoir faite r'enduire, il jugera la maison nette ; car sa plaie est guérie.
\VS{49}Alors il prendra pour purifier la maison, deux passereaux, du bois de cèdre, du cramoisi, et de l'hysope.
\VS{50}Et il coupera la gorge à l'un des passereaux sur un vaisseau de terre, sur de l'eau vive.
\VS{51}Et il prendra le bois de cèdre, l'hysope, le cramoisi, et le passereau vivant, et trempera le tout dans le sang du passereau qu'on aura égorgé, et dans l'eau vive ; puis il fera aspersion dans la maison par sept fois.
\VS{52}Il purifiera donc la maison avec le sang du passereau, et avec l'eau vive, et avec le passereau vivant, le bois de cèdre, l'hysope, et le cramoisi.
\VS{53}Puis il laissera aller hors de la ville par les champs le passereau vivant, et il fera propitiation pour la maison ; et elle sera nette.
\VS{54}Telle est la loi de toute plaie de lèpre, et de teigne ;
\VS{55}De lèpre de vêtement, et de maison ;
\VS{56}De tumeur, de gâle, et de bouton ;
\VS{57}Pour enseigner en quel temps une chose est souillée, et en quel temps elle est nette ; telle est la loi de la lèpre.
\Chap{15}
\VerseOne{}L'Eternel parla aussi à Moïse et à Aaron, en disant :
\VS{2}Parlez aux enfants d'Israël, et leur dîtes : Tout homme à qui la chair découle, sera souillé à cause de son flux.
\VS{3}Et telle sera la souillure de son flux ; quand sa chair laissera aller son flux, ou que sa chair retiendra son flux, c'est sa souillure.
\VS{4}Tout lit sur lequel aura couché celui qui découle, sera souillé ; et toute chose sur laquelle il se sera assis, sera souillée.
\VS{5}Quiconque aussi touchera son lit lavera ses vêtements, et se lavera avec de l'eau ; et il sera souillé jusqu'au soir.
\VS{6}Et qui s'assiéra sur quelque chose sur laquelle celui qui découle se soit assis, lavera ses vêtements, et se lavera dans l'eau ; et il sera souillé jusqu'au soir.
\VS{7}Et celui qui touchera la chair de celui qui découle, lavera ses vêtements, et se lavera dans l'eau ; et il sera souillé jusqu'au soir.
\VS{8}Et si celui qui découle crache sur celui qui est net, celui qui était net lavera ses vêtements, et se lavera dans l'eau ; et il sera souillé jusqu'au soir.
\VS{9}Toute monture aussi que celui qui découle aura montée, sera souillée.
\VS{10}Quiconque touchera quelque chose qui aura été sous lui, sera souillé jusqu'au soir ; et quiconque portera de telle chose lavera ses vêtements, et se lavera dans l'eau ; et il sera souillé jusqu'au soir.
\VS{11}Quiconque aura été touché par celui qui découle, sans qu'il ait lavé ses mains dans l'eau, lavera ses vêtements ; et il se lavera dans l'eau ; et il sera souillé jusqu'au soir.
\VS{12}Et le vaisseau de terre que celui qui découle aura touché, sera cassé ; mais tout vaisseau de bois sera lavé dans l'eau.
\VS{13}Or quand celui qui découle sera purifié de son flux, il comptera sept jours pour sa purification, il lavera ses vêtements, et sa chair avec de l'eau vive, et ainsi il sera net.
\VS{14}Et au huitième jour il prendra pour soi deux tourterelles, ou deux pigeonneaux, et il viendra devant l'Eternel à l'entrée du Tabernacle d'assignation, et les donnera au Sacrificateur.
\VS{15}Et le Sacrificateur les sacrifiera, l'un [en offrande pour] le péché, et l'autre en holocauste ; ainsi le Sacrificateur fera propitiation pour lui devant l'Eternel à cause de son flux.
\VS{16}L'homme aussi duquel sera sortie de la semence, lavera dans l'eau toute sa chair, et il sera souillé jusqu'au soir.
\VS{17}Et tout habit, ou toute peau sur laquelle il y aura de la semence, sera lavée dans l'eau, et sera souillée jusqu'au soir.
\VS{18}Même la femme dont un tel homme aura la compagnie, se lavera dans l'eau [avec son mari], et ils seront souillés jusqu'au soir.
\VS{19}Et quand la femme sera découlante, ayant son flux de sang en sa chair, elle sera séparée sept jours ; [et] quiconque la touchera sera souillé jusqu'au soir.
\VS{20}Toute chose sur laquelle elle aura couché, durant sa séparation, sera souillée ; toute chose aussi sur laquelle elle aura été assise, sera souillée.
\VS{21}Quiconque aussi touchera le lit de cette femme, lavera ses vêtements, et se lavera dans l'eau ; et il sera souillé jusqu'au soir.
\VS{22}Et quiconque touchera quelque chose sur laquelle elle se sera assise, lavera ses vêtements, et se lavera dans l'eau ; et il sera souillé jusqu'au soir.
\VS{23}Même si la chose [que quelqu'un aura touchée était] sur le lit, ou sur quelque chose sur laquelle elle était assise, quand quelqu'un aura touché cette chose-là ; il sera souillé jusqu'au soir.
\VS{24}Et si quelqu'un a habité avec elle tellement que ses fleurs soient sur lui, il sera souillé sept jours ; et toute couche sur laquelle il dormira, sera souillée.
\VS{25}Quand aussi la femme découle par flux de son sang plusieurs jours, sans que ce soit le temps de ses mois ; ou quand elle découlera plus longtemps que le temps de ses mois, tout le temps du flux de sa souillure, elle sera souillée comme au temps de sa séparation.
\VS{26}Toute couche sur laquelle elle couchera tous les jours de son flux, lui sera comme la couche de sa séparation ; et toute chose sur laquelle elle s'assied sera souillée, comme [pour] la souillure de sa séparation.
\VS{27}Et quiconque aura touché ces choses-là, lavera ses vêtements, et se lavera dans l'eau ; et il sera souillé jusqu'au soir.
\VS{28}Mais si elle est purifiée de son flux, elle comptera sept jours, et après elle sera nette.
\VS{29}Et au huitième jour elle prendra deux tourterelles ou deux pigeonneaux, et les apportera au Sacrificateur à l'entrée du Tabernacle d'assignation.
\VS{30}Et le Sacrificateur en sacrifiera l'un [en offrande] pour le péché, et l'autre en holocauste ; ainsi le Sacrificateur fera propitiation pour elle devant l'Eternel, à cause du flux de sa souillure.
\VS{31}Ainsi vous séparerez les enfants d'Israël de leurs souillures, et ils ne mourront point à cause de leurs souillures, en souillant mon pavillon qui est au milieu d'eux.
\VS{32}Telle est la loi de celui qui découle, et de celui duquel sort la semence, qui le souille.
\VS{33}Telle est aussi la loi de celle qui est malade de ses mois, et de toute personne qui découle, et qui a son flux, soit mâle, soit femelle, et de celui qui couche avec celle qui est souillée.
\Chap{16}
\VerseOne{}Or l'Eternel parla à Moïse après la mort des deux enfants d'Aaron, lorsque s'étant approchés de la présence de l'Eternel, ils moururent.
\VS{2}L'Eternel donc dit à Moïse : Parle à Aaron ton frère, et [lui dis] qu'il n'entre point en tout temps dans le Sanctuaire au dedans du voile devant le Propitiatoire, qui est sur l'Arche, afin qu'il ne meure point ; car je me montrerai dans une nuée sur le Propitiatoire.
\VS{3}Aaron entrera en cette manière dans le Sanctuaire, [après qu'il aura offert] un veau du troupeau pour le péché, et un bélier pour l'holocauste.
\VS{4}Il se revêtira de la sainte chemise de lin, ayant mis les caleçons de lin sur sa chair, et il se ceindra du baudrier de lin, et portera la tiare de lin, qui sont les saints vêtements, et il s'en vêtira après avoir lavé sa chair avec de l'eau.
\VS{5}Et il prendra de l'assemblée des enfants d'Israël deux jeunes boucs [en offrande] pour le péché, et un bélier pour l'holocauste.
\VS{6}Puis Aaron offrira son veau [en offrande] pour le péché, et fera propitiation tant pour soi que pour sa maison.
\VS{7}Et il prendra les deux boucs, et les présentera devant l'Eternel, à l'entrée du Tabernacle d'assignation.
\VS{8}Puis Aaron jettera le sort sur les deux boucs ; un sort pour l'Eternel, et un sort pour [le bouc qui doit être] Hazazel.
\VS{9}Et Aaron offrira le bouc sur lequel le sort sera échu pour l'Eternel, et le sacrifiera [en offrande] pour le péché.
\VS{10}Mais le bouc sur lequel le sort sera échu pour [être] Hazazel, sera présenté vivant devant l'Eternel pour faire propitiation par lui, [et on] l'enverra au désert pour [être] Hazazel.
\VS{11}Aaron donc offrira son veau [en offrande] pour le péché, et fera propitiation pour soi et pour sa maison, il égorgera, [dis-je], son veau qui est l'offrande pour le péché.
\VS{12}Puis il prendra plein un encensoir de la braise du feu qui est sur l'autel devant l'Eternel, et ses pleines paumes de parfum de drogues pulvérisées, et il l'apportera de la maison dans le voile ;
\VS{13}Et il mettra le parfum sur le feu devant l'Eternel ; afin que la nuée du parfum couvre le Propitiatoire qui est sur le Témoignage ; ainsi il ne mourra point.
\VS{14}Il prendra aussi du sang du veau, et il en fera aspersion avec son doigt au devant du Propitiatoire vers l'Orient ; il fera, [dis-je], aspersion de ce sang-là sept fois avec son doigt devant le Propitiatoire.
\VS{15}Il égorgera aussi le bouc du peuple, qui est [l'offrande pour] le péché, et il apportera son sang au dedans du voile, et fera de son sang comme il a fait du sang du veau, en faisant aspersion vers le Propitiatoire ; sur le devant du Propitiatoire.
\VS{16}Et il fera expiation pour le Sanctuaire, [le nettoyant] des souillures des enfants d'Israël, et de leurs fautes, selon tous leurs péchés ; et il fera la même chose au Tabernacle d'assignation, qui demeure avec eux au milieu de leurs souillures.
\VS{17}Et personne ne sera au Tabernacle d'assignation quand le Sacrificateur y entrera pour faire propitiation dans le Sanctuaire, jusqu'à ce qu'il en sorte, lorsqu'il fera propitiation pour soi et pour sa maison, et pour toute l'assemblée d'Israël.
\VS{18}Puis il sortira vers l'autel qui est devant l'Eternel, et fera expiation pour lui ; et prenant du sang du veau et du sang du bouc, il le mettra sur les cornes de l'autel tout à l'entour.
\VS{19}Et il fera par sept fois aspersion du sang avec son doigt sur l'autel, et le nettoiera et le sanctifiera des souillures des enfants d'Israël.
\VS{20}Et quand il aura achevé de faire expiation pour le Sanctuaire, et pour le Tabernacle d'assignation, et pour l'autel, alors il offrira le bouc vivant.
\VS{21}Et Aaron posant ses deux mains sur la tête du bouc vivant, confessera sur lui toutes les iniquités des enfants d'Israël, et toutes leurs fautes, selon tous leurs péchés, et il les mettra sur la tête du bouc, et l'enverra au désert par un homme exprès.
\VS{22}Et le bouc portera sur soi toutes leurs iniquités dans une terre inhabitable, puis cet homme laissera aller le bouc par le désert.
\VS{23}Et Aaron reviendra au Tabernacle d'assignation, et quittera les vêtements de lin dont il s'était vêtu quand il était entré au Sanctuaire, et les posera là.
\VS{24}Il lavera aussi sa chair avec de l'eau dans le lieu saint, et se revêtira de ses vêtements ; puis étant sorti, il offrira son holocauste, et l'holocauste du peuple, et fera propitiation pour soi, et pour le peuple.
\VS{25}Il fera aussi fumer sur l'autel la graisse de [l'offrande pour le] péché.
\VS{26}Et celui qui aura conduit le bouc pour [être] Hazazel, lavera ses vêtements et sa chair avec de l'eau ; puis il rentrera au camp.
\VS{27}Mais on tirera hors du camp le veau et le bouc qui auront été offerts [en offrande pour] le péché, et desquels le sang aura été porté au Sanctuaire pour y faire propitiation, et on brûlera au feu leur peau, leur chair, et leur fiente.
\VS{28}Et celui qui les aura brûlés lavera ses vêtements et sa chair avec de l'eau ; après quoi il rentrera au camp.
\VS{29}Et ceci vous sera pour une ordonnance perpétuelle. Le dixième jour du septième mois vous affligerez vos âmes, et vous ne ferez aucune œuvre, tant celui qui est du pays, que l'étranger qui fait son séjour parmi vous.
\VS{30}Car en ce jour-là [le Sacrificateur] fera propitiation pour vous, afin de vous nettoyer ; [ainsi] vous serez nettoyés de tous vos péchés en la présence de l'Eternel.
\VS{31}Ce vous sera donc un Sabbat de repos, et vous affligerez vos âmes ; c'est une ordonnance perpétuelle.
\VS{32}Et le Sacrificateur qu'on aura oint, et qu'on aura consacré pour exercer la sacrificature en la place de son père, fera propitiation, s'étant revêtu des vêtements de lin, qui sont les saints vêtements.
\VS{33}Et il fera expiation pour le saint Sanctuaire, pour le Tabernacle d'assignation, et pour l'autel, et pour les Sacrificateurs, et pour tout le peuple de l'assemblée.
\VS{34}Ceci donc vous sera pour une ordonnance perpétuelle, afin de faire propitiation pour les enfants d'Israël de tous leurs péchés une fois l'an ; et on fit comme l'Eternel l'avait commandé à Moïse.
\Chap{17}
\VerseOne{}L'Eternel parla aussi à Moïse, en disant :
\VS{2}Parle à Aaron et à ses fils, et à tous les enfants d'Israël, et leur dis : C'est ici ce que l'Eternel a commandé, en disant :
\VS{3}Quiconque de la maison d'Israël aura égorgé un bœuf, ou un agneau, ou une chèvre dans le camp, ou qui l'aura égorgé hors du camp,
\VS{4}Et ne l'aura point amené à l'entrée du Tabernacle d'assignation pour en faire une offrande à l'Eternel devant le pavillon de l'Eternel, le sang sera imputé à cet homme-là ; il a répandu du sang ; c'est pourquoi cet homme-là sera retranché du milieu de son peuple.
\VS{5}Afin que les enfants d'Israël amènent leurs sacrifices, lesquels ils sacrifient dans les champs, qu'ils les amènent, dis-je, à l'Eternel, à l'entrée du Tabernacle d'assignation, vers le Sacrificateur, et qu'ils les sacrifient en sacrifices de prospérités à l'Eternel ;
\VS{6}Et que le Sacrificateur en répande le sang sur l'autel de l'Eternel à l'entrée du Tabernacle d'assignation, et en fasse fumer la graisse en bonne odeur à l'Eternel ;
\VS{7}Et qu'ils n'offrent plus leurs sacrifices aux diables, avec lesquels ils ont paillardé. Que ce leur soit une ordonnance perpétuelle en leurs âges.
\VS{8}Tu leur diras donc : Quiconque des enfants d'Israël, ou des étrangers qui font leur séjour parmi eux, aura offert un holocauste ou un sacrifice,
\VS{9}Et qui ne l'aura point amené à l'entrée du Tabernacle d'assignation, pour le sacrifier à l'Eternel, cet homme-là sera retranché d'entre ses peuples.
\VS{10}Quiconque de la famille d'Israël ou des étrangers qui font leur séjour parmi eux, aura mangé de quelque sang que ce soit, je mettrai ma face contre cette personne qui aura mangé du sang, et je la retrancherai du milieu de son peuple.
\VS{11}Car l'âme de la chair est dans le sang ; c'est pourquoi je vous ai ordonné qu'il soit mis sur l'autel afin de faire propitiation pour vos âmes ; car c'est le sang qui fera propitiation pour l'âme.
\VS{12}C'est pourquoi j'ai dit aux enfants d'Israël : Que personne d'entre vous ne mange du sang ; que l'étranger même qui fait son séjour parmi vous, ne mange point de sang.
\VS{13}Et quiconque des enfants d'Israël, et des étrangers qui font leur séjour parmi eux, aura pris à la chasse une bête sauvage, ou un oiseau que l'on mange, il répandra leur sang, et le couvrira de poussière.
\VS{14}Car l'âme de toute chair est dans son sang, c'est son âme ; c'est pourquoi j'ai dit aux enfants d'Israël : Vous ne mangerez point le sang d'aucune chair ; car l'âme de toute chair est son sang ; quiconque en mangera sera retranché.
\VS{15}Et toute personne qui aura mangé de la chair de quelque bête morte d'elle-même, ou déchirée [par les bêtes sauvages], tant celui qui est né au pays que l'étranger, lavera ses vêtements, et se lavera avec de l'eau, et il sera souillé jusqu'au soir ; puis il sera net.
\VS{16}Que s'il ne lave pas [ses vêtements], et sa chair, il portera son iniquité.
\Chap{18}
\VerseOne{}L'Eternel parla encore à Moïse, en disant :
\VS{2}Parle aux enfants d'Israël, et leur dis : Je suis l'Eternel votre Dieu.
\VS{3}Vous ne ferez point ce qui se fait au pays d'Egypte où vous avez habité, ni ce qui se fait au pays de Canaan, auquel je vous amène ; et vous ne vivrez point selon leurs statuts.
\VS{4}[Mais] vous ferez selon mes statuts, et vous garderez mes ordonnances pour marcher en elles ; je suis l'Eternel votre Dieu.
\VS{5}Vous garderez donc mes statuts, et mes ordonnances, lesquelles [si] l'homme accomplit, il vivra par elles ; je suis l'Eternel.
\VS{6}Que nul ne s'approche de celle qui [est] sa proche parente pour découvrir sa nudité ; je suis l'Eternel.
\VS{7}Tu ne découvriras point la nudité de ton père, ni la nudité de ta mère ; c'[est] ta mère, tu ne découvriras point sa nudité.
\VS{8}Tu ne découvriras point la nudité de la femme de ton père ; c'est la nudité de ton père.
\VS{9}Tu ne découvriras point la nudité de ta sœur, fille de ton père, ou fille de ta mère, née dans la maison, ou hors [de la maison] ; tu ne découvriras point leur nudité.
\VS{10}Quant à la nudité de la fille de ton fils, ou de la fille de ta fille, tu ne découvriras point leur nudité, car elles sont ta nudité.
\VS{11}Tu ne découvriras point la nudité de la fille de la femme de ton père, née de ton père, c'est ta sœur.
\VS{12}Tu ne découvriras point la nudité de la sœur de ton père ; elle est proche parente de ton père.
\VS{13}Tu ne découvriras point la nudité de la sœur de ta mère ; car elle est proche parente de ta mère.
\VS{14}Tu ne découvriras point la nudité du frère de ton père, [et] ne t'approcheras point de sa femme ; elle est ta tante.
\VS{15}Tu ne découvriras point la nudité de ta belle-fille ; elle est la femme de ton fils, tu ne découvriras point sa nudité.
\VS{16}Tu ne découvriras point la nudité de la femme de ton frère, c'est la nudité de ton frère.
\VS{17}Tu ne découvriras point la nudité d'une femme et de sa fille, et ne prendras point la fille de son fils, ni la fille de sa fille pour découvrir leur nudité, elles sont tes proches parentes ; c'est une énormité.
\VS{18}Tu ne prendras point aussi une femme avec sa sœur pour l'affliger en découvrant sa nudité sur elle, pendant sa vie.
\VS{19}Tu n'approcheras point de la femme durant la séparation de sa souillure, pour découvrir sa nudité.
\VS{20}Tu n'auras point aussi la compagnie de la femme de ton prochain, te souillant avec elle.
\VS{21}Tu ne donneras point de tes enfants pour les faire passer [par le feu] devant Molec, et tu ne profaneras point le Nom de ton Dieu ; je suis l'Eternel.
\VS{22}Tu n'auras point aussi la compagnie d'un mâle ; c'est une abomination.
\VS{23}Tu ne t'approcheras point aussi d'aucune bête pour te souiller avec elle ; et la femme ne se prostituera point à une bête ; c'est une confusion.
\VS{24}Ne vous souillez point en aucune de ces choses ; car les nations que je m'en vais chasser de devant vous, se sont souillées en toutes ces choses ;
\VS{25}Dont la terre a été souillée, et je m'en vais punir sur elle son iniquité, et la terre vomira ses habitants.
\VS{26}Mais quant à vous, vous garderez mes ordonnances et mes jugements, et vous ne ferez aucune de ces abominations, tant celui qui est né au pays, que l'étranger qui fait son séjour parmi vous.
\VS{27}Car les gens de ce pays-là qui ont été avant vous, ont fait toutes ces abominations, et la terre en a été souillée.
\VS{28}La terre ne vous vomirait-elle point, si vous la souilliez, comme elle aura vomi les gens qui y ont été avant vous ?
\VS{29}Car quiconque fera aucune de toutes ces abominations, les personnes qui les auront faites seront retranchées du milieu de leur peuple.
\VS{30}Vous garderez donc ce que j'ai ordonné de garder, et vous ne pratiquerez aucune de ces coutumes abominables qui ont été pratiquées avant vous, et vous ne vous souillerez point par elles ; je suis l'Eternel votre Dieu.
\Chap{19}
\VerseOne{}L'Eternel parla aussi à Moïse, en disant :
\VS{2}Parle à toute l'assemblée des enfants d'Israël, et leur dis : Soyez saints ; car je suis saint, moi l'Eternel votre Dieu.
\VS{3}Vous craindrez chacun sa mère et son père, et vous garderez mes Sabbats ; je suis l'Eternel votre Dieu.
\VS{4}Vous ne vous tournerez point vers les idoles, et ne vous ferez aucuns dieux de fonte ; je suis l'Eternel votre Dieu.
\VS{5}Si vous offrez un sacrifice de prospérités à l'Eternel, vous le sacrifierez de votre bon gré.
\VS{6}II se mangera au jour que vous l'aurez sacrifié, et le lendemain, mais ce qui restera jusqu'au troisième jour, sera brûlé au feu.
\VS{7}Que si on en mange au troisième jour, ce sera une abomination ; il ne sera point agréé.
\VS{8}Quiconque aussi en mangera, portera son iniquité ; car il aura profané la chose sainte de l'Eternel ; et cette personne-là sera retranchée d'entre ses peuples.
\VS{9}Et quand vous ferez la moisson de votre terre, tu n'achèveras point de moissonner le bout de ton champ, et tu ne glaneras point ce qui restera à cueillir de ta moisson.
\VS{10}Et tu ne grappilleras point ta vigne, ni ne recueilleras point les grains [tombés] de ta vigne, mais tu les laisseras au pauvre et à l'étranger ; je suis l'Eternel votre Dieu.
\VS{11}Vous ne déroberez point, ni ne dénierez point [la chose à qui elle appartient] ; et aucun de vous ne mentira à son prochain.
\VS{12}Vous ne jurerez point par mon Nom en mentant ; car tu profanerais le Nom de ton Dieu ; je suis l'Eternel.
\VS{13}Tu n'opprimeras point ton prochain, et tu ne le pilleras point. Le salaire de ton mercenaire ne demeurera point par devers toi jusqu'au matin.
\VS{14}Tu ne maudiras point le sourd, et tu ne mettras point d'achoppement devant l'aveugle, mais tu craindras ton Dieu ; je suis l'Eternel.
\VS{15}Vous ne ferez point d'iniquité en jugement ; tu n'auras point d'égard à la personne du pauvre, et tu n'honoreras point la personne du grand, [mais] tu jugeras justement ton prochain.
\VS{16}Tu n'iras point médisant parmi ton peuple. Tu ne t'élèveras point contre le sang de ton prochain ; je suis l'Eternel.
\VS{17}Tu ne haïras point ton frère en ton cœur. Tu reprendras soigneusement ton prochain, et tu ne souffriras point de péché en lui.
\VS{18}Tu n'useras point de vengeance, et tu ne la garderas point aux enfants de ton peuple ; mais tu aimeras ton prochain comme toi-même ; je suis l'Eternel.
\VS{19}Vous garderez mes ordonnances. Tu n'accoupleras point tes bêtes avec d'autres de diverse espèce. Tu ne sèmeras point ton champ de diverses sortes de grains, et tu ne mettras point sur toi de vêtements de diverses espèces, [comme] de laine et de lin.
\VS{20}Si un homme a couché avec une femme, laquelle étant esclave fut fiancée à un homme, et qu'elle n'ait pas été rachetée, et que la liberté ne lui ait pas été donnée, ils auront le fouet ; [mais] on ne les fera point mourir ; parce qu'elle n'avait pas été affranchie.
\VS{21}Et l'homme amènera son [offrande pour le] délit à l'Eternel à l'entrée du Tabernacle d'assignation, [savoir] un bélier pour le délit.
\VS{22}Et le Sacrificateur fera propitiation pour lui devant l'Eternel par le bélier de [l'offrande pour le] délit, à cause de son péché qu'il aura commis ; et son péché qu'il aura commis lui sera pardonné.
\VS{23}Et quand vous serez entrés au pays, et que vous y aurez planté quelque arbre fruitier, vous tiendrez son fruit pour son prépuce ; il vous sera incirconcis pendant trois ans, et on n'en mangera point.
\VS{24}Mais en la quatrième année tout son fruit sera une chose sainte, pour en louer l'Eternel.
\VS{25}Et en la cinquième année vous mangerez son fruit, afin qu'il vous multiplie son rapport ; je suis l'Eternel votre Dieu.
\VS{26}Vous ne mangerez rien avec le sang. Vous n'userez point de divinations, et vous ne pronostiquerez point le temps.
\VS{27}Vous ne tondrez point en rond les coins de votre tête, et vous ne gâterez point les coins de votre barbe.
\VS{28}Vous ne ferez point d'incisions dans votre chair pour un mort, et vous n'imprimerez point de caractère sur vous ; je suis l'Eternel.
\VS{29}Tu ne souilleras point ta fille en la prostituant pour la faire paillarder ; afin que la terre ne soit point souillée par la paillardise, et ne soit point remplie d'énormité.
\VS{30}Vous garderez mes Sabbats, et vous aurez en révérence mon Sanctuaire ; je suis l'Eternel.
\VS{31}Ne vous détournez point après ceux qui ont l'esprit de Python, ni après les devins ; ne cherchez point de vous souiller par eux ; je suis l'Eternel votre Dieu.
\VS{32}Lève-toi devant les cheveux blancs, et honore la personne du vieillard, et crains ton Dieu ; je suis l'Eternel.
\VS{33}Si quelque étranger habite en votre pays, vous ne lui ferez point de tort.
\VS{34}L'étranger qui habite parmi vous, vous sera comme celui qui est né parmi vous, et vous l'aimerez comme vous-mêmes ; car vous avez été étrangers au pays d'Egypte. Je suis l'Eternel votre Dieu.
\VS{35}Vous ne ferez point d'iniquité en jugement, ni en règle, ni en poids, ni en mesure.
\VS{36}Vous aurez les balances justes, les pierres [à peser] justes, l'Epha juste, et le Hin juste. Je suis l'Eternel votre Dieu qui vous ai retirés du pays d'Egypte.
\VS{37}Gardez donc toutes mes ordonnances, et mes jugements, et les faites ; je suis l'Eternel.
\Chap{20}
\VerseOne{}L'Eternel parla aussi à Moïse, en disant :
\VS{2}Tu diras aux enfants d'Israël : Quiconque des enfants d'Israël, ou des étrangers qui demeurent en Israël, donnera de sa postérité à Molec, sera puni de mort ; le peuple du pays l'assommera de pierres.
\VS{3}Et je mettrai ma face contre un tel homme, et je le retrancherai du milieu de son peuple, parce qu'il aura donné de sa postérité à Molec, pour souiller mon Sanctuaire, et profaner le Nom de ma Sainteté.
\VS{4}Que si le peuple du pays ferme les yeux en quelque manière que se soit, pour ne point voir quand cet homme-là aura donné de sa postérité à Molec, [et] ne le point faire mourir ;
\VS{5}Je mettrai ma face contre cet homme-là, et contre sa famille, et je le retrancherai du milieu de mon peuple, avec tous ceux qui paillardent à son exemple, en paillardant après Molec.
\VS{6}Quant à la personne qui se détournera après ceux qui ont l'esprit de Python, et après les devins, en paillardant après eux, je mettrai ma face contre cette personne-là, et je la retrancherai du milieu de son peuple.
\VS{7}Sanctifiez-vous donc, et soyez saints ; car je suis l'Eternel votre Dieu.
\VS{8}Gardez aussi mes ordonnances, et les faites ; je suis l'Eternel qui vous sanctifie.
\VS{9}Quand quelqu'un aura maudit son père ou sa mère, on le fera mourir de mort ; il a maudit son père ou sa mère, son sang est sur lui.
\VS{10}Quant à l'homme qui aura commis adultère avec la femme d'un autre, parce qu'il a commis adultère avec la femme de son prochain, on fera mourir de mort l'homme et la femme adultères.
\VS{11}L'homme qui aura couché avec la femme de son père, a découvert la nudité de son père ; on les fera mourir de mort tous deux, leur sang est sur eux.
\VS{12}Et quand un homme aura couché avec sa belle-fille, on les fera mourir de mort tous deux ; ils ont fait une confusion ; leur sang est sur eux.
\VS{13}Quand un homme aura eu la compagnie d'un mâle, ils ont tous deux fait une chose abominable ; on les fera mourir de mort, leur sang est sur eux.
\VS{14}Et quand un homme aura pris une femme, et la mère de cette femme, c'est une énormité, il sera brûlé au feu avec elles, afin qu'il n'y ait point d'énormité au milieu de vous.
\VS{15}L'homme qui se sera souillé avec une bête, sera puni de mort ; vous tuerez aussi la bête.
\VS{16}Et quand quelque femme se sera prostituée à quelque bête, tu tueras cette femme et la bête ; on les fera mourir de mort, leur sang est sur eux.
\VS{17}Quand un homme aura pris sa sœur, fille de son père, ou fille de sa mère, et aura vu sa nudité, et qu'elle aura vu la nudité de cet homme ; c'est une chose infâme ; ils seront donc retranchés en la présence des enfants de leur peuple ; il a découvert la nudité de sa sœur, il portera son iniquité.
\VS{18}Quand un homme aura couché avec une femme qui a ses mois, et qu'il aura découvert la nudité de cette [femme], en découvrant son flux, et qu'elle aura découvert le flux de son sang ; ils seront tous deux retranchés du milieu de leur peuple.
\VS{19}Tu ne découvriras point la nudité de la sœur de ta mère, ni de la sœur de ton père ; parce qu'il aura découvert sa chair, ils porteront [tous deux] leur iniquité.
\VS{20}Et quand un homme aura couché avec sa tante, il a découvert la nudité de son oncle ; ils porteront leur péché, et ils mourront sans en laisser d'enfants.
\VS{21}Et quand un homme aura pris la femme de son frère, c'est une ordure ; il a découvert la honte de son frère, ils n'[en] auront point d'enfants.
\VS{22}Ainsi gardez toutes mes ordonnances, et mes jugements, et observez-les ; et le pays auquel je vous fais entrer pour y habiter ne vous vomira point.
\VS{23}Vous ne suivrez point aussi les ordonnances des nations que je m'en vais chasser de devant vous ; car elles ont fait toutes ces choses-là, et je les ai eues en abomination.
\VS{24}Et je vous ai dit : Vous posséderez leur pays, et je vous le donnerai pour le posséder ; c'est un pays découlant de lait et de miel. Je suis l'Eternel votre Dieu, qui vous ai séparés des [autres] peuples.
\VS{25}C'est pourquoi séparez la bête nette de la souillée, l'oiseau net d'avec le souillé, et ne rendez point abominables vos personnes [en mangeant] des bêtes et des oiseaux [immondes], ni rien qui rampe sur la terre, rien de ce que je vous ai défendu comme une chose immonde.
\VS{26}Vous me serez donc saints ; car je suis saint, moi l'Eternel, et je vous ai séparés des [autres] peuples, afin que vous soyez à moi.
\VS{27}Quand un homme ou une femme aura un esprit de Python, ou sera devin, on les fera mourir de mort ; on les assommera de pierres ; leur sang est sur eux.
\Chap{21}
\VerseOne{}L'Eternel dit aussi à Moïse : Parle aux Sacrificateurs, fils d'Aaron, et leur dis : [Qu'aucun d'eux] ne se souille entre ses peuples pour un mort.
\VS{2}Sinon pour son proche parent, qui le touche de près, [savoir] pour sa mère, pour son père, pour son fils, pour sa fille, et pour son frère.
\VS{3}Et quant à sa sœur vierge, qui le touche de près, et qui n'aura point eu de mari, il se souillera pour elle.
\VS{4}S'il est marié il ne se souillera point [pour sa femme] parmi son peuple, en se rendant impur.
\VS{5}Ils n'arracheront point les cheveux de leur tête pour la rendre chauve, et ils ne raseront point les coins de leur barbe, ni ne feront d'incision en leur chair.
\VS{6}Ils seront saints à leur Dieu, et ils ne profaneront point le nom de leur Dieu ; car ils offrent les sacrifices de l'Eternel faits par feu, qui est la viande de leur Dieu ; c'est pourquoi ils seront très-saints.
\VS{7}Ils ne prendront point une femme paillarde, ou déshonorée ; ils ne prendront point aussi une femme répudiée par son mari ; car ils sont saints à leur Dieu.
\VS{8}Tu feras donc que chacun d'eux soit saint, parce qu'ils offrent la viande de ton Dieu. Ils te seront donc saints, car je suis saint, moi l'Eternel qui vous sanctifie.
\VS{9}Si la fille du Sacrificateur se souille en commettant paillardise, elle souille son père ; qu'elle soit [donc] brûlée au feu.
\VS{10}Et le souverain Sacrificateur d'entre ses frères, sur la tête duquel l'huile de l'onction aura été répandue, et qui se sera consacré pour vêtir les [saints] vêtements, ne découvrira point sa tête, et ne déchirera point ses vêtements.
\VS{11}Il n'ira point vers aucune personne morte ; il ne se rendra point impur pour son père, ni pour sa mère ;
\VS{12}Et il ne sortira point du Sanctuaire, et ne souillera point le Sanctuaire de son Dieu ; parce que la couronne, [et] l'huile de l'onction de son Dieu est sur lui. Je [suis] l'Eternel.
\VS{13}Il prendra pour femme une vierge.
\VS{14}Il ne prendra point une veuve ; ni une répudiée, ni une femme déshonorée, [ni] une paillarde ; mais il prendra pour femme une vierge d'entre ses peuples.
\VS{15}Il ne souillera point sa postérité entre ses peuples ; car je suis l'Eternel qui le sanctifie.
\VS{16}L'Eternel parla aussi à Moïse, en disant :
\VS{17}Parle à Aaron, et lui dis : Si quelqu'un de ta postérité dans ses âges a quelque défaut [en son corps], il ne s'approchera point pour offrir la viande de son Dieu.
\VS{18}Car aucun homme en qui il y aura quelque défaut n'en approchera ; [savoir] l'homme aveugle, ou boiteux, ou camus, ou qui aura quelque superfluité dans ses membres.
\VS{19}Ou l'homme qui aura quelque fracture aux pieds, ou aux mains.
\VS{20}Ou qui sera bossu, ou grêle, ou qui aura quelque suffusion en l'œil, ou qui aura une gâle sèche, ou une gâle d'ulcère, ou qui sera rompu.
\VS{21}Nul homme donc de la postérité d'Aaron Sacrificateur en qui il y aura quelque défaut, ne s'approchera pour offrir les offrandes faites par feu à l'Eternel ; il y a un défaut en lui, il ne s'approchera donc point pour offrir la viande de son Dieu.
\VS{22}Il pourra bien manger de la viande de son Dieu, [savoir] des choses très-saintes ; et des choses saintes.
\VS{23}Mais il n'entrera point vers le voile, ni ne s'approchera point de l'autel, parce qu'il y a en lui une défectuosité, de peur de souiller mes Sanctuaires ; car je suis l'Eternel qui les sanctifie.
\VS{24}Moïse donc parla ainsi à Aaron et à ses fils, et à tous les enfants d'Israël.
\Chap{22}
\VerseOne{}Puis l'Eternel parla à Moïse, en disant :
\VS{2}Dis à Aaron et à ses fils, quand ils auront à s'abstenir des choses saintes des enfants d'Israël, afin qu'ils ne profanent point le nom de ma sainteté dans les choses qu'eux-mêmes me sanctifient ; je suis l'Eternel ;
\VS{3}Dis-leur donc : Tout homme de toute votre postérité en vos âges qui étant souillé s'approchera des choses saintes que les enfants d'Israël auront sanctifiées à l'Eternel, cette personne-là sera retranchée de ma présence ; je suis l'Eternel.
\VS{4}Tout homme de la postérité d'Aaron étant lépreux, ou découlant, ne mangera point des choses saintes jusqu'à ce qu'il soit nettoyé ; et celui aussi qui aura touché quelque homme souillé pour avoir touché un mort, et celui qui aura un flux de semence.
\VS{5}Et celui qui aura touché quelque reptile dont il soit souillé, ou quelque homme par lequel il soit souillé, quelque souillure qui puisse être en lui.
\VS{6}La personne qui aura touché ces choses sera souillée jusqu'au soir, et ne mangera point des choses saintes, si elle n'a lavé sa chair avec de l'eau.
\VS{7}Ensuite, elle sera nette après le soleil couché ; et elle mangera des choses saintes ; car c'est sa viande.
\VS{8}Il ne mangera point de la chair d'aucune bête morte d'elle-même, ou déchirée par [les bêtes sauvages], pour se souiller par elle ; je suis l'Eternel.
\VS{9}Qu'ils gardent donc ce que j'ai ordonné de garder, et qu'ils ne commettent point de péché au sujet de la viande [sainte], afin qu'ils ne meurent point, pour l'avoir souillée ; je suis l'Eternel qui les sanctifie.
\VS{10}Nul étranger aussi ne mangera des choses saintes ; le forain logé chez le Sacrificateur, et le mercenaire, ne mangeront point des choses saintes.
\VS{11}Mais quand le Sacrificateur aura acheté quelque personne de son argent, elle en mangera ; pareillement celui qui sera né dans sa maison ; ceux-ci mangeront de sa viande.
\VS{12}Que si la fille du Sacrificateur est mariée à un étranger, elle ne mangera point des choses saintes, présentées en offrande élevée.
\VS{13}Mais si la fille du Sacrificateur est veuve, ou répudiée, et si elle n'a point d'enfants, étant retournée en la maison de son père, comme [elle y demeurait en] sa jeunesse, elle mangera de la viande de son père ; mais nul étranger n'en mangera.
\VS{14}Que si quelqu'un par ignorance mange d'une chose sainte, il ajoutera un cinquième par dessus, et le donnera au Sacrificateur avec la chose sainte.
\VS{15}Et ils ne souilleront point les choses sanctifiées des enfants d'Israël, qu'ils auront offertes à l'Eternel.
\VS{16}Mais on leur fera porter la peine du péché, parce qu'ils auront mangé de leurs choses saintes ; car je suis l'Eternel qui les sanctifie.
\VS{17}L'Eternel parla encore à Moïse, en disant :
\VS{18}Parle à Aaron et à ses fils, et à tous les enfants d'Israël, et leur dis : Quiconque de la maison d'Israël, ou des étrangers qui sont en Israël, offrira son offrande, selon tous ses vœux, ou selon toutes ses offrandes volontaires, lesquelles on offre en holocauste à l'Eternel ;
\VS{19}Il offrira de son bon gré, un mâle sans tare, d'entre les vaches, [ou] d'entre les brebis, ou d'entre les chèvres.
\VS{20}Vous n'offrirez aucune chose qui ait quelque tare, car elle ne serait point agréée pour vous.
\VS{21}Que si un homme offre à l'Eternel un sacrifice de prospérités en s'acquittant de quelque vœu, ou en faisant quelque offrande volontaire, soit de bœufs, ou de brebis, ce qui sera sans tare sera agréé ; il n'y doit avoir aucune tare.
\VS{22}Vous n'offrirez point à l'Eternel ce qui sera aveugle, ou rompu, ou mutilé, ou qui aura un porreau, ou une gâle sèche, ou une gâle d'ulcère, et vous n'en donnerez point pour le sacrifice qui se fait par feu sur l'autel à l'Eternel.
\VS{23}Tu pourras bien faire une offrande volontaire, d'un bœuf, ou d'une brebis, ou d'une chèvre ayant quelque superfluité, ou quelque défaut dans ses membres, mais ils ne seront point agréés pour le vœu.
\VS{24}Vous n'offrirez point à l'Eternel, et ne sacrifierez point en votre pays [une bête] qui ait les génitoires froissés, ou cassés, ou arrachés, ou taillés.
\VS{25}Vous ne prendrez point aussi de la main de l'étranger aucune de toutes ces choses pour les offrir en viande à votre Dieu, car la corruption qui est en eux est une tare en elles ; elles ne seront point agréées pour vous.
\VS{26}L'Eternel parla encore à Moïse, en disant :
\VS{27}Quand un veau, ou un agneau, ou une chèvre seront nés, et qu'ils auront été sept jours sous leur mère, depuis le huitième jour et les suivants, ils seront agréables pour l'offrande du sacrifice qui se fait par feu à l'Eternel.
\VS{28}Vous n'égorgerez point aussi en un même jour la vache, ou la brebis, ou la chèvre, avec son petit.
\VS{29}Quand vous offrirez un sacrifice d'action de grâces à l'Eternel, vous le sacrifierez de votre bon gré.
\VS{30}Il sera mangé le jour même, [et] vous n'en réserverez rien jusqu'au matin ; je suis l'Eternel.
\VS{31}Gardez donc mes commandements, et les faites. ; je suis l'Eternel.
\VS{32}Et ne profanez point le nom de ma sainteté, car je serai sanctifié entre les enfants d'Israël ; je suis l'Eternel qui vous sanctifie ;
\VS{33}Et qui vous ai retirés du pays d'Egypte, pour vous être Dieu ; je suis l'Eternel.
\Chap{23}
\VerseOne{}L'Eternel parla aussi à Moïse en disant :
\VS{2}Parle aux enfants d'Israël, et leur dis : Les fêtes solennelles de l'Eternel, que vous publierez, seront de saintes convocations ; et ce sont ici mes fêtes solennelles.
\VS{3}On travaillera six jours, mais au septième jour, qui est le Sabbat du repos, il y aura une sainte convocation ; vous ne ferez aucune œuvre, [car] c'est le Sabbat à l'Eternel, dans toutes vos demeures.
\VS{4}Et ce sont ici les fêtes solennelles de l'Eternel, qui seront de saintes convocations, que vous publierez en leur saison.
\VS{5}Au premier mois, le quatorzième jour du mois entre les deux vêpres, est la Pâque à l'Eternel.
\VS{6}Et le quinzième jour de ce même mois est la fête solennelle des pains sans levain à l'Eternel ; vous mangerez des pains sans levain pendant sept jours.
\VS{7}Le premier jour vous aurez une sainte convocation ; vous ne ferez aucune œuvre servile.
\VS{8}Mais vous offrirez à l'Eternel pendant sept jours des offrandes faites par feu, [et] au septième jour il y aura une sainte convocation ; vous ne ferez aucune œuvre servile.
\VS{9}L'Eternel parla aussi à Moïse, en disant :
\VS{10}Parle aux enfants d'Israël, et leur dis : Quand vous serez entrés au pays que je vous donne, et que vous en aurez fait la moisson, vous apporterez alors au Sacrificateur une poignée des premiers fruits de votre moisson.
\VS{11}Et il tournoiera cette poignée-là devant l'Eternel, afin qu'elle soit agréée pour vous ; le Sacrificateur la tournoiera le lendemain du Sabbat.
\VS{12}Et le jour que vous ferez tournoyer cette poignée, vous sacrifierez un agneau sans tare, et d'un an en holocauste à l'Eternel.
\VS{13}Et le gâteau de cet holocauste sera de deux dixièmes de fine farine, pétrie à l'huile, pour offrande faite par feu à l'Eternel en bonne odeur ; et son aspersion de vin sera la quatrième partie d'un Hin.
\VS{14}Et vous ne mangerez ni pain, ni grain rôti, ni grain en épi, jusqu'à ce propre jour-là, même jusqu'à ce que vous ayez apporté l'offrande à votre Dieu ; c'est une ordonnance perpétuelle en vos âges, dans toutes vos demeures.
\VS{15}Vous compterez aussi dès le lendemain du Sabbat, [savoir] dès le jour que vous aurez apporté la poignée qu'on doit tournoyer, sept semaines entières.
\VS{16}Vous compterez donc cinquante jours jusqu'au lendemain de la septième semaine ; et vous offrirez à l'Eternel un gâteau nouveau.
\VS{17}Vous apporterez de vos demeures deux pains, pour en faire une offrande tournoyée, ils [seront] de deux dixièmes, [et] de fine farine, pétris avec du levain ; [ce sont] les premiers fruits à l'Eternel.
\VS{18}Vous offrirez aussi avec ce pain-là sept agneaux sans tare, et d'un an, et un veau pris du troupeau, et deux béliers, qui seront un holocauste à l'Eternel, avec leurs gâteaux, et leurs aspersions, des sacrifices faits par feu en bonne odeur à l'Eternel.
\VS{19}Vous sacrifierez aussi un jeune bouc [en offrande] pour le péché, et deux agneaux d'un an pour le sacrifice de prospérités.
\VS{20}Et le Sacrificateur les tournoiera avec le pain des premiers fruits, et avec les deux agneaux, en offrande tournoyée devant l'Éternel ; ils seront saints à l'Eternel pour le Sacrificateur.
\VS{21}Vous publierez donc en ce même jour-là que vous avez une sainte convocation ; vous ne ferez aucune œuvre servile ; c'est une ordonnance perpétuelle dans toutes vos demeures en vos âges.
\VS{22}Et quand vous ferez la moisson de votre terre tu n'achèveras point de moissonner le bout de ton champ, et tu ne glaneras point les épis qui resteront de ta moisson, mais tu les laisseras pour le pauvre, et pour l'étranger ; je suis l'Eternel votre Dieu.
\VS{23}L'Eternel parla aussi à Moïse, en disant :
\VS{24}Parle aux enfants d'Israël, et leur dis : Au septième mois, le premier jour du mois il y aura repos pour vous, un mémorial de jubilation, et une sainte convocation.
\VS{25}Vous ne ferez aucune œuvre servile, et vous offrirez à l'Eternel des offrandes faites par feu.
\VS{26}L'Eternel parla aussi à Moïse, en disant :
\VS{27}Pareillement en ce même mois, qui est le septième, le dixième jour sera le jour des propitiations ; vous aurez une sainte convocation, et vous affligerez vos âmes, et vous offrirez à l'Eternel des sacrifices faits par feu.
\VS{28}En ce jour-là vous ne ferez aucune œuvre ; car c'est le jour des propitiations, afin de faire propitiation pour vous devant l'Eternel votre Dieu.
\VS{29}Car toute personne qui n'aura pas été affligée en ce propre jour-là sera retranchée d'entre ses peuples.
\VS{30}Et toute personne qui aura fait quelque œuvre en ce jour-là, je ferai périr cette personne-là du milieu de son peuple.
\VS{31}Vous ne ferez [donc] aucune œuvre ; c'est une ordonnance perpétuelle en vos âges dans toutes vos demeures.
\VS{32}Ce vous est un Sabbat de repos : vous affligerez donc vos âmes. Le neuvième jour du mois, au soir, depuis un soir jusqu'à l'autre soir, vous célébrerez votre repos.
\VS{33}L'Eternel parla aussi à Moïse, en disant :
\VS{34}Parle aux enfants d'Israël, et leur dis : Au quinzième jour de ce septième mois [sera] la fête solennelle des Tabernacles pendant sept jours, à l'Eternel.
\VS{35}Au premier jour il y aura une sainte convocation ; vous ne ferez aucune œuvre servile.
\VS{36}Pendant sept jours vous offrirez à l'Eternel des offrandes faites par feu ; et au huitième jour vous aurez une sainte convocation, et vous offrirez à l'Eternel des offrandes faites, par feu ; c'est une assemblée solennelle ; vous ne ferez aucune œuvre servile.
\VS{37}Ce sont là les fêtes solennelles de l'Eternel, que vous publierez, pour être des convocations saintes, afin d'offrir à l'Eternel des offrandes faites par feu ; [savoir] un holocauste, un gâteau, un sacrifice, et une aspersion ; chacune de ces choses en son jour ;
\VS{38}Outre les Sabbats de l'Eternel, et outre vos dons, et outre tous vos vœux, et outre toutes les offrandes volontaires que vous présenterez à l'Eternel.
\VS{39}Et aussi au quinzième jour du septième mois, quand vous aurez recueilli le rapport de la terre, vous célébrerez la fête solennelle de l'Eternel pendant sept jours. Le premier jour sera jour de repos ; le huitième aussi sera jour de repos.
\VS{40}Et au premier jour vous prendrez du fruit d'un bel arbre, des branches de palmier, et des rameaux d'arbres branchus, et des saules de rivière, et vous vous réjouirez pendant sept jours devant l'Eternel votre Dieu.
\VS{41}Et vous célébrerez à l'Eternel cette fête solennelle pendant sept jours en l'année ; c'est une ordonnance perpétuelle en vos âges ; vous la célébrerez le septième mois.
\VS{42}Vous demeurerez sept jours dans des tentes ; tous ceux qui seront nés entre les Israélites demeureront dans des tentes.
\VS{43}Afin que votre postérité sache que j'ai fait demeurer les enfants d'Israël dans des tentes, quand je les retirai du pays d'Egypte ; je suis l'Eternel votre Dieu.
\VS{44}Moïse déclara ainsi aux enfants d'Israël les fêtes solennelles de l'Eternel.
\Chap{24}
\VerseOne{}L'Eternel parla aussi à Moïse, en disant :
\VS{2}Commande aux enfants d'Israël qu'ils t'apportent de l'huile vierge pour le luminaire, afin de faire brûler les lampes continuellement.
\VS{3}Aaron les arrangera devant l'Eternel continuellement, depuis le soir jusqu'au matin hors du voile du Témoignage dans le Tabernacle d'assignation ; c'est une ordonnance perpétuelle en vos âges.
\VS{4}Il arrangera, [dis-je], continuellement les lampes sur le chandelier pur, devant l'Eternel.
\VS{5}Tu prendras aussi de la fine farine, et tu en feras cuire douze gâteaux, chaque gâteau sera de deux dixièmes.
\VS{6}Et tu les exposeras devant l'Eternel par deux rangées sur la Table pure, six à chaque rangée.
\VS{7}Et tu mettras de l'encens pur sur chaque rangée, qui sera un mémorial pour le pain ; c'est une offrande faite par feu à l'Eternel.
\VS{8}On les arrangera chaque jour de Sabbat continuellement devant l'Eternel, de la part des enfants d'Israël ; c'est une alliance perpétuelle.
\VS{9}Et ils appartiendront à Aaron, et à ses fils, qui les mangeront dans un lieu saint ; car ils lui seront une chose très-sainte d'entre les offrandes de l'Eternel faites par feu ; c'[est] une ordonnance perpétuelle.
\VS{10}Or le fils d'une femme israélite, qui aussi était fils d'un homme Egyptien, sortit parmi les enfants d'Israël, et ce fils de la femme Israélite, et un homme Israélite se querellèrent dans le camp.
\VS{11}Et le fils de la femme Israélite blasphéma le nom [de l'Eternel], et le maudit ; et on l'amena à Moïse. Or sa mère s'appelait Sélomith, fille de Dibri, de la Tribu de Dan.
\VS{12}Et on le mit en garde jusqu'à ce qu'on leur eût déclaré [ce qu'ils en devraient faire] selon la parole de l'Eternel.
\VS{13}Et l'Eternel parla à Moïse, en disant :
\VS{14}Tire hors du camp celui qui a maudit ; et que tous ceux qui l'ont entendu mettent les mains sur sa tête, et que toute l'assemblée le lapide.
\VS{15}Et parle aux enfants d'Israël, et leur dis : Quiconque aura maudit son Dieu, portera son péché.
\VS{16}Et celui qui aura blasphémé le Nom de l'Eternel, sera puni de mort ; toute l'assemblée ne manquera pas de le lapider, on fera mourir tant l'étranger, que celui qui est né au pays, lequel aura blasphémé le Nom [de l'Eternel].
\VS{17}On punira aussi de mort celui qui aura frappé à mort quelque personne que ce soit.
\VS{18}Celui qui aura frappé une bête à mort, la rendra vie pour vie.
\VS{19}Et quand quelque homme aura fait un outrage à son prochain, on lui fera comme il a fait.
\VS{20}Fracture pour fracture, œil pour œil, dent pour dent ; selon le mal qu'il aura fait à un homme, il lui sera aussi fait.
\VS{21}Celui qui frappera une bête à [mort], la rendra ; mais on fera mourir celui qui aura frappé un homme à [mort].
\VS{22}Vous rendrez un même jugement. [Vous traiterez] l'étranger comme celui qui est né au pays ; car je suis l'Eternel votre Dieu.
\VS{23}Moïse donc parla aux enfants d'Israël, qui firent sortir hors du camp celui qui avait maudit, et l'assommèrent de pierres ; ainsi les enfants d'Israël firent comme l'Eternel l'avait commandé à Moïse.
\Chap{25}
\VerseOne{}L'Eternel parla aussi à Moïse sur la montagne de Sinaï, en disant :
\VS{2}Parle aux enfants d'Israël, et leur dis : Quand vous serez entrés au pays que je vous donne, la terre se reposera ; ce sera un Sabbat à l'Eternel.
\VS{3}Pendant six ans tu sèmeras ton champ, et pendant six ans tu tailleras ta vigne, et en recueilleras le rapport.
\VS{4}Mais la septième année il y aura un Sabbat de repos pour la terre, ce sera un Sabbat à l'Eternel ; tu ne sèmeras point ton champ, et ne tailleras point ta vigne.
\VS{5}Tu ne moissonneras point ce qui sera provenu de soi-même de ce qui sera tombé en moissonnant, et tu ne vendangeras point les raisins de ta vigne non taillée ; [ce] sera l'année du repos de la terre.
\VS{6}Mais ce qui proviendra de la terre l'année du Sabbat vous servira d'aliment, à toi, et à ton serviteur, et à ta servante, à ton mercenaire, et à ton étranger qui demeurent avec toi ;
\VS{7}Et à tes bêtes, et aux animaux qui sont en ton pays ; tout son rapport sera pour manger.
\VS{8}Tu compteras aussi sept semaines d'années, [savoir] sept fois sept ans, et les jours de sept semaines feront quarante-neuf ans.
\VS{9}Puis tu feras sonner la trompette de jubilation le dixième jour du septième mois, le jour, [dis-je], des propitiations, vous ferez sonner la trompette par tout votre pays.
\VS{10}Et vous sanctifierez l'an cinquantième, et publierez la liberté dans le pays à tous ses habitants ; ce vous sera l'année du Jubilé, et vous retournerez chacun en sa possession, et chacun en sa famille.
\VS{11}Cette cinquantième année vous sera [l'année] du Jubilé, vous ne sèmerez point et ne moissonnerez point ce que la terre rapportera d'elle-même, et vous ne vendangerez point les fruits de la vigne non taillée.
\VS{12}Car c'est [l'année] du Jubilé, elle vous sera sainte ; vous mangerez ce que les champs rapporteront cette année-là.
\VS{13}En cette année du Jubilé vous retournerez chacun en sa possession.
\VS{14}Et si tu fais quelque vente à ton prochain, ou si tu achètes [quelque chose] de ton prochain, que nul de vous ne foule son frère.
\VS{15}Mais tu achèteras de ton prochain selon le nombre des années après le Jubilé. Pareillement on te fera les ventes selon le nombre des années du rapport.
\VS{16}Selon qu'il y aura plus d'années, tu augmenteras le prix de ce que tu achètes ; et selon qu'il y aura moins d'années, tu le diminueras ; car on te vend le nombre des récoltes.
\VS{17}Que donc nul de vous ne foule son prochain ; mais craignez votre Dieu, car je suis l'Eternel votre Dieu.
\VS{18}Faites selon mes ordonnances, gardez mes jugements, observez-les, et vous habiterez sûrement au pays.
\VS{19}Et la terre vous donnera ses fruits, vous en mangerez, vous en serez rassasiés, et vous habiterez sûrement en elle.
\VS{20}Et si vous dites : Que mangerons-nous en la septième année si nous ne semons point, et si nous ne recueillons point notre récolte ?
\VS{21}Je commanderai à ma bénédiction [de se répandre] sur vous en la sixième année, et [la terre] rapportera pour trois ans.
\VS{22}Puis vous sèmerez en la huitième année, et vous mangerez du rapport du passé jusqu'à la neuvième année ; jusqu'à ce [donc] que son rapport sera venu, vous mangerez celui du passé.
\VS{23}La terre ne sera point vendue absolument, car la terre est à moi ; et vous êtes étrangers et forains chez moi.
\VS{24}C'est pourquoi dans tout le pays de votre possession vous donnerez le droit de rachat pour la terre.
\VS{25}Si ton frère est devenu pauvre, et vend quelque chose de ce qu'il possède, celui qui a le droit de rachat, [savoir] son [plus] proche parent, viendra et rachètera la chose vendue par son frère.
\VS{26}Que si cet homme n'a personne qui ait le droit de rachat, et qu'il ait trouvé de soi-même suffisamment de quoi faire le rachat de ce qu'il a vendu ;
\VS{27}Il comptera les années du temps qu'il a fait la vente, et il restituera le surplus à l'homme auquel il l'avait faite, et ainsi il retournera dans sa possession.
\VS{28}Mais s'il n'a pas trouvé suffisamment de quoi lui rendre, la chose qu'il aura vendue sera en la main de celui qui l'aura achetée, jusqu'à l'année du Jubilé ; puis [l'acheteur] en sortira au Jubilé, et [le vendeur] retournera dans sa possession.
\VS{29}Et si quelqu'un a vendu une maison à habiter dans quelque ville fermée de murailles, il aura le droit de rachat jusqu'à la fin de l'année de sa vente ; son droit de rachat sera d'une année.
\VS{30}Mais si elle n'est point rachetée dans l'année accomplie, la maison qui [est] dans la ville fermée de murailles, demeurera à l'acheteur absolument et en ses âges ; il n'en sortira point au Jubilé.
\VS{31}Mais les maisons des villages, qui ne sont point entourés de murailles, seront réputées comme un fonds de terre ; le vendeur aura droit de rachat, et [l'acheteur] sortira au Jubilé.
\VS{32}Et quant aux villes des Lévites, les Lévites auront un droit de rachat perpétuel des maisons des villes de leur possession.
\VS{33}Et celui qui aura acheté [quelque maison] des Lévites, sortira au Jubilé de la maison vendue, qui est en la ville de sa possession ; car les maisons des villes des Lévites [sont] leur possession parmi les enfants d'Israël.
\VS{34}Mais le champ des faubourgs de leurs villes ne sera point vendu ; car c'[est] leur possession perpétuelle.
\VS{35}Quand ton frère sera devenu pauvre, et qu'il tendra vers toi ses mains tremblantes, tu le soutiendras, [tu soutiendras] aussi l'étranger, et le forain, afin qu'il vive avec toi.
\VS{36}Tu ne prendras point de lui d'usure, ni d'intérêt, mais tu craindras ton Dieu ; et ton frère vivra avec toi.
\VS{37}Tu ne lui donneras point ton argent à usure, ni ne lui donneras de tes vivres à surcroît.
\VS{38}Je suis l'Eternel votre Dieu qui vous ai retirés du pays d'Egypte, pour vous donner le pays de Canaan, afin de vous être Dieu.
\VS{39}Pareillement quand ton frère sera devenu pauvre auprès de toi, et qu'il se sera vendu à toi, tu ne te serviras point de lui comme on se sert des esclaves.
\VS{40}[Mais] il sera chez toi comme serait le mercenaire, [et] l'étranger, [et] il te servira jusqu'à l'année du Jubilé.
\VS{41}Alors il sortira de chez toi avec ses enfants, il s'en retournera dans sa famille, et rentrera dans la possession de ses pères.
\VS{42}Car ils sont mes serviteurs, parce que je les ai retirés du pays d'Egypte ; c'est pourquoi ils ne seront point vendus comme on vend les esclaves.
\VS{43}Tu ne domineras point sur lui rigoureusement, mais tu craindras ton Dieu.
\VS{44}Et quant à ton esclave et à ta servante qui seront à toi, ils seront d'entre les nations qui sont autour de vous ; vous achèterez d'elles le serviteur et la servante.
\VS{45}Vous en achèterez aussi d'entre les enfants des étrangers qui demeurent avec vous, même de leurs familles qui seront parmi vous, lesquelles ils auront engendrées en votre pays, et vous les posséderez.
\VS{46}Vous les aurez comme un héritage pour les laisser à vos enfants après vous, afin qu'ils en héritent la possession, [et] vous vous servirez d'eux à perpétuité ; mais quant à vos frères, les enfants d'Israël, nul ne dominera rigoureusement sur son frère.
\VS{47}Et lorsque l'étranger ou le forain qui est avec toi se sera enrichi, et que ton frère qui est avec lui sera devenu si pauvre qu'il se soit vendu à l'étranger, [ou] au forain qui est avec toi, ou à quelqu'un de la postérité de la famille de l'étranger.
\VS{48}Après s'être vendu il y aura droit de rachat pour lui, [et] un de ses frères le rachètera.
\VS{49}Ou son oncle, ou le fils de son oncle, ou quelque autre proche parent de son sang d'entre ceux de sa famille, le rachètera ; ou lui-même, s'il en trouve le moyen, se rachètera.
\VS{50}Et il comptera avec son acheteur depuis l'année qu'il s'est vendu à lui, jusqu'à l'année du Jubilé ; de sorte que l'argent du prix pour lequel il s'est vendu, se comptera à raison du nombre des années ; le temps qu'il aura servi lui sera compté comme les journées d'un mercenaire.
\VS{51}S'il y a encore plusieurs années, il restituera le prix de son achat à raison de ces [années], selon le prix pour lequel il a été acheté.
\VS{52}Et s'il reste peu d'années jusqu'à l'an du Jubilé, il comptera avec lui, et restituera le prix de son achat à raison des années qu'il a servi.
\VS{53}Il aura été avec lui comme un mercenaire qui se loue d'année en année ; [et cet étranger] ne dominera point sur lui rigoureusement en ta présence.
\VS{54}Que s'il n'est pas racheté par quelqu'un de ces moyens, il sortira l'année du Jubilé, lui et ses fils avec lui.
\VS{55}Car les enfants d'Israël me sont serviteurs ; ce [sont] mes serviteurs que j'ai retirés du pays d'Egypte ; je suis l'Eternel votre Dieu.
\Chap{26}
\VerseOne{}Vous ne vous ferez point d'idoles, et vous ne vous dresserez point d'image taillée, ni de statue, et vous ne mettrez point de pierre peinte dans votre pays pour vous prosterner devant elles ; car je suis l'Eternel votre Dieu.
\VS{2}Vous garderez mes Sabbats, et vous révérerez mon Sanctuaire ; je suis l'Eternel.
\VS{3}Si vous marchez dans mes ordonnances, et si vous gardez mes commandements et les faites ;
\VS{4}Je vous donnerai les pluies qu'il vous faut en leur temps, la terre donnera son fruit, et les arbres des champs donneront leur fruit.
\VS{5}La foulure des grains atteindra la vendange chez vous, et la vendange atteindra les semailles ; vous mangerez votre pain, vous en serez rassasiés, et vous habiterez sûrement en votre pays.
\VS{6}Je donnerai la paix au pays, vous dormirez sans qu'aucun vous épouvante ; je ferai qu'il n'y aura plus de mauvaises bêtes au pays ; et l'épée ne passera point par votre pays.
\VS{7}Mais vous poursuivrez vos ennemis, et ils tomberont par l'épée devant vous.
\VS{8}Cinq d'entre vous en poursuivront cent, et cent en poursuivront dix mille ; et vos ennemis tomberont par l'épée devant vous.
\VS{9}Et je me tournerai vers vous, je vous ferai croître et multiplier, et j'établirai mon alliance avec vous.
\VS{10}Vous mangerez aussi des provisions fort vieilles, et vous tirerez dehors le vieux pour y loger le nouveau.
\VS{11}Même je mettrai mon Tabernacle au milieu de vous, et mon âme ne vous aura point à contrecœur.
\VS{12}Mais je marcherai au milieu de vous, je vous serai Dieu, et vous serez mon peuple.
\VS{13}Je [suis] l'Eternel votre Dieu qui vous ai retirés du pays d'Egypte, afin que vous ne fussiez point leurs esclaves ; j'ai rompu les bois de votre joug, et vous ai fait marcher la tête levée.
\VS{14}Mais si vous ne m'écoutez point, et que vous ne fassiez pas tous ces commandements ;
\VS{15}Et que vous rejetiez mes ordonnances, et que votre âme ait mes jugements à contrecœur, pour ne point faire tous mes commandements, et pour enfreindre mon alliance ;
\VS{16}Aussi je vous ferai ceci ; je répandrai sur vous la frayeur, la langueur, et l'ardeur, qui [vous] consumeront les yeux, et vous tourmenteront l'âme ; et vous sèmerez en vain votre semence ; car vos ennemis la mangeront.
\VS{17}Et je mettrai ma face contre vous ; vous serez battus devant vos ennemis ; ceux qui vous haïssent domineront sur vous ; et vous fuirez, sans qu'aucun vous poursuive.
\VS{18}Que si encore après ces choses vous ne m'écoutez point, j'en ajouterai sept fois autant pour vous châtier, à cause de vos péchés.
\VS{19}Et j'abattrai l'orgueil de votre force, et je ferai que le ciel sera pour vous comme de fer, et votre terre comme d'airain.
\VS{20}Votre force se consumera inutilement, car votre terre ne donnera point son rapport, et les arbres de la terre ne donneront point leur fruit.
\VS{21}Que si vous marchez de front contre moi, et que vous refusiez de m'écouter, j'ajouterai sur vous sept fois autant de plaies, selon vos péchés.
\VS{22}J'enverrai contre vous les bêtes des champs, qui vous priveront de vos enfants, qui tueront votre bétail, et vous réduiront à un petit nombre, et vos chemins seront déserts.
\VS{23}Que si vous ne vous corrigez pas après ces choses [pour vous convertir] à moi, mais que vous marchiez de front contre moi ;
\VS{24}Je marcherai aussi de front contre vous, et je vous frapperai encore sept fois autant, selon vos péchés.
\VS{25}Et je ferai venir sur vous l'épée qui fera la vengeance de mon alliance ; et quand vous vous retirerez dans vos villes j'enverrai la mortalité parmi vous, et vous serez livrés entre les mains de l'ennemi.
\VS{26}Lorsque je vous aurai rompu le bâton du pain, dix femmes cuiront votre pain dans un four, et vous rendront votre pain au poids ; vous en mangerez, et vous n'en serez point rassasiés.
\VS{27}Que si avec cela vous ne m'écoutez point, mais que vous marchiez de front contre moi,
\VS{28}Je marcherai de front contre vous en ma fureur, et je vous châtierai aussi sept fois autant selon vos péchés ;
\VS{29}Et vous mangerez la chair de vos fils, et vous mangerez aussi la chair de vos filles.
\VS{30}Je détruirai vos hauts lieux ; je ruinerai vos Tabernacles ; je mettrai vos charognes sur les charognes de vos dieux de fiente, et mon âme vous aura en haine.
\VS{31}Je réduirai aussi vos villes en désert, je ruinerai vos Sanctuaires, et je ne flairerai point votre odeur agréable.
\VS{32}Et je désolerai le pays, tellement que vos ennemis qui s'y habitueront, en seront étonnés.
\VS{33}Et je vous disperserai parmi les nations, et je tirerai l'épée après vous, et votre pays sera en désolation, et vos villes en désert.
\VS{34}Alors la terre prendra plaisir à ses Sabbats, tout le temps qu'elle sera désolée, ; et lorsque vous serez au pays de vos ennemis la terre se reposera, et prendra plaisir à ses Sabbats.
\VS{35}Tout le temps qu'elle demeurera désolée, elle se reposera ; au lieu qu'elle ne s'était point reposée en vos Sabbats, lorsque vous y habitiez.
\VS{36}Et quant à ceux qui demeureront de reste d'entre vous, je rendrai leur cœur lâche lorsqu'ils seront au pays de leurs ennemis, de sorte que le bruit d'une feuille émue les poursuivra, et ils fuiront comme s'ils fuyaient de devant l'épée, et ils tomberont sans qu'aucun les poursuive.
\VS{37}Et ils s'entreheurteront l'un l'autre comme s'ils fuyaient de devant l'épée, sans que personne les poursuive ; et vous ne pourrez point subsister devant vos ennemis.
\VS{38}Et vous périrez entre les nations, et la terre de vos ennemis vous consumera.
\VS{39}Et ceux qui demeureront de reste d'entre vous se fondront à cause de leurs iniquités, au pays de vos ennemis ; et ils se fondront aussi à cause des iniquités de leurs pères, avec eux.
\VS{40}Alors ils confesseront leur iniquité, et l'iniquité de leurs pères, selon les péchés qu'ils auront commis contre moi ; et même selon qu'ils auront marché de front contre moi.
\VS{41}J'aurai aussi marché de front contre eux, et je les aurai amenés au pays de leurs ennemis ; et alors leur cœur incirconcis s'humiliera, et ils recevront alors avec soumission, [la punition de] leur iniquité.
\VS{42}Et alors je me souviendrai de mon alliance avec Jacob, et de mon alliance avec Isaac, et je me souviendrai aussi de mon alliance avec Abraham, et je me souviendrai de la terre.
\VS{43}Quand [donc] la terre aura été abandonnée par eux, et qu'elle aura pris plaisir à ses Sabbats, ayant demeure désolée à cause d'eux ; lors donc qu'ils auront reçu avec soumission [la punition de] leur iniquité, à cause qu'ils ont rejeté mes jugements, et que leur âme a dédaigné mes ordonnances.
\VS{44}[Je m'en souviendrai, dis-je], lorsqu'ils seront au pays de leurs ennemis ; parce que je ne les ai point rejetés, ni eus en haine pour les consumer entièrement, et pour rompre l'alliance que j'ai faite avec eux ; car je [suis] l'Eternel leur Dieu.
\VS{45}Et je me souviendrai pour leur bien de l'alliance faite avec leurs ancêtres, lesquels j'ai retirés du pays d'Egypte, à la vue des nations, pour être leur Dieu ; je [suis] l'Eternel.
\VS{46}Ce sont là les ordonnances, les jugements, et les lois que l'Eternel établit entre lui et les enfants d'Israël sur la montagne de Sinaï, par le moyen de Moïse.
\Chap{27}
\VerseOne{}L'Eternel parla aussi à Moïse, en disant :
\VS{2}Parle aux enfants d'Israël, et leur dis : Quand quelqu'un aura fait un vœu important, les personnes [vouées] à l'Eternel [seront mises] à ton estimation.
\VS{3}Et l'estimation que tu feras d'un mâle, depuis l'âge de vingt ans jusqu'à l'âge de soixante ans, sera du prix de cinquante sicles d'argent, selon le sicle du Sanctuaire.
\VS{4}Mais si c'est une femme, alors ton estimation sera de trente sicles.
\VS{5}Que si c'est d'une personne de l'âge de cinq ans jusqu'à l'âge de vingt ans, alors l'estimation que tu feras d'un mâle sera de vingt sicles ; et quant à la femme, [l'estimation sera] de dix sicles.
\VS{6}Et si c'est d'une personne de l'âge d'un mois jusqu'à l'âge de cinq ans, l'estimation que tu feras d'un mâle sera de cinq sicles d'argent ; et l'estimation que tu feras d'une fille, sera de trois sicles d'argent.
\VS{7}Et lorsque c'est d'une personne âgée de soixante ans et au-dessus, si c'est un mâle, ton estimation sera de quinze sicles ; et si c'est une femme, [l'estimation sera] de dix sicles.
\VS{8}Et s'il est plus pauvre que [ne monte] ton estimation, il se présentera devant le Sacrificateur qui en fera l'estimation, et le Sacrificateur en fera l'estimation selon ce que pourra fournir celui qui a fait le vœu.
\VS{9}Et si c'est d'une [de ces sortes] de bêtes dont on fait offrande à l'Eternel, tout ce qui aura été donné à l'Eternel de cette sorte [de bêtes], sera saint.
\VS{10}Il ne la changera point, et n'en mettra point une autre en sa place, une bonne pour une mauvaise, ou une mauvaise pour une bonne : et s'il met en quelque sorte que ce soit une bête pour une autre bête, tant celle-là que l'autre qui aura été mise en sa place, sera sainte.
\VS{11}Et si c'est d'une bête souillée, dont on ne fait point offrande à l'Eternel, il présentera la bête devant le Sacrificateur,
\VS{12}Qui en fera l'estimation selon qu'elle sera bonne ou mauvaise ; et il en sera fait ainsi, selon que toi, Sacrificateur, en auras fait l'estimation.
\VS{13}Mais s'il la veut racheter en quelque sorte, il ajoutera un cinquième par dessus ton estimation.
\VS{14}Et quand quelqu'un aura sanctifié sa maison pour être sainte à l'Eternel, le Sacrificateur l'estimera selon qu'elle sera bonne ou mauvaise ; [et] on se tiendra à l'estimation que le Sacrificateur en aura faite.
\VS{15}Mais si celui qui l'a sanctifiée veut racheter sa maison, il ajoutera par dessus le cinquième de l'argent de ton estimation, et elle lui demeurera.
\VS{16}Et si l'homme sanctifie à l'Eternel [quelque partie] du champ de sa possession, ton estimation sera selon ce qu'on y sème ; le Homer de semence d'orge sera estimé cinquante sicles d'argent.
\VS{17}Que s'il a sanctifié son champ dès l'année du Jubilé, on se tiendra à ton estimation.
\VS{18}Mais s'il sanctifie son champ après le Jubilé, le Sacrificateur lui mettra en compte l'argent selon le nombre des années qui restent jusqu'à l'année du Jubilé, et cela sera rabattu de ton estimation.
\VS{19}Et si celui qui a sanctifié le champ, le veut racheter en quelque sorte que ce soit, il ajoutera par dessus le cinquième de l'argent de ton estimation, et il lui demeurera.
\VS{20}Mais s'il ne rachète point le champ, et que le champ se vende à un autre homme, il ne se rachètera plus.
\VS{21}Et ce champ-là ayant passé le Jubilé, sera saint à l'Eternel, comme un champ d'interdit, la possession en sera au Sacrificateur.
\VS{22}Et s'il sanctifie à l'Eternel un champ qu'il ait acheté, n'étant point des champs de sa possession ;
\VS{23}Le Sacrificateur lui comptera la somme de ton estimation jusqu'à l'année du Jubilé, et il donnera en ce jour-là ton estimation, [afin que ce soit] une chose sainte à l'Eternel.
\VS{24}Mais en l'année du Jubilé le champ retournera à celui duquel il l'avait acheté, [et] auquel était la possession du fond.
\VS{25}Et toute estimation que tu auras faite, sera selon le sicle du Sanctuaire ; le sicle est de vingt oboles.
\VS{26}Toutefois nul ne [pourra] sanctifier le premier-né d'entre les bêtes, car il appartient à l'Eternel par droit de primogéniture, soit de vache, soit de brebis, ou de chèvre, il est à l'Eternel.
\VS{27}Mais s'il est de bêtes souillées, il le rachètera selon ton estimation, et il ajoutera à ton estimation un cinquième ; et s'il n'est point racheté, il sera vendu selon ton estimation.
\VS{28}Or nul interdit que quelqu'un aura dévoué à l'Eternel par interdit, de tout ce qui est sien, soit homme, ou bête, ou champ de sa possession, ne se vendra, ni ne se rachètera ; tout interdit sera très-saint à l'Eternel.
\VS{29}Nul interdit dévoué par interdit d'entre les hommes, ne se rachètera, mais on le fera mourir de mort.
\VS{30}Or toute dîme de la terre, tant du grain de la terre que du fruit des arbres, est à l'Eternel ; c'est une sainteté à l'Eternel.
\VS{31}Mais si quelqu'un veut racheter en quelque sorte que ce soit quelque chose de sa dîme, il y ajoutera le cinquième par dessus.
\VS{32}Mais toute dîme de bœufs, de brebis et de chèvres, [savoir] tout ce qui passe sous la verge, qui est le dixième, sera sanctifié à l'Eternel.
\VS{33}n ne choisira point le bon ou le mauvais, et on n'en mettra point d'autre en sa place ; que si on le fait en quelque sorte que ce soit, la bête changée et l'autre qui aura été mise en sa place, sera sanctifiée, [et] ne sera point rachetée.
\VS{34}Ce sont là les commandements que l'Eternel donna à Moïse sur la montagne de Sinaï, pour les enfants d'Israël.
\PPE{}
\end{multicols}
