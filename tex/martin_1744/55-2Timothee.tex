\ShortTitle{2Timothee}\BookTitle{2Timothee}\BFont
\begin{multicols}{2}
\Chap{1}
\VerseOne{}Paul Apôtre de Jésus-Christ, par la volonté de Dieu, selon la promesse de la vie qui est en Jésus-Christ :
\VS{2}A Timothée, mon fils bien-aimé, que la grâce, la miséricorde et la paix te soient données de la part de Dieu le Père, et de la part de Jésus-Christ notre Seigneur.
\VS{3}Je rends grâces à Dieu, lequel je sers dès mes ancêtres avec une pure conscience, faisant sans cesse mention de toi dans mes prières nuit et jour.
\VS{4}Me souvenant de tes larmes, je désire fort de te voir afin que je sois rempli de joie ;
\VS{5}Et me souvenant de la foi sincère qui est en toi, et qui a premièrement habité en Loïs, ta grand-mère, et en Eunice, ta mère, et je suis persuadé qu'elle [habite] aussi en toi.
\VS{6}C'est pourquoi je t'exhorte de ranimer le don de Dieu, qui est en toi par l'imposition de mes mains.
\VS{7}Car Dieu ne nous a pas donné un esprit de timidité, mais de force, de charité et de prudence.
\VS{8}Ne prends donc point à honte le témoignage de notre Seigneur, ni moi, qui suis son prisonnier ; mais prends part aux afflictions de l'Evangile, selon la puissance de Dieu ;
\VS{9}Qui nous a sauvés, et qui nous a appelés par une sainte vocation, non selon nos œuvres, mais selon son propre dessein, et selon la grâce qui nous a été donnée en Jésus-Christ avant les temps éternels ;
\VS{10}Et qui maintenant a été manifestée par l'apparition de notre Sauveur Jésus-Christ, qui a détruit la mort, et qui a mis en lumière la vie et l'immortalité par l'Evangile ;
\VS{11}Pour lequel j'ai été établi Prédicateur, Apôtre, et Docteur des Gentils.
\VS{12}C'est pourquoi aussi je souffre ces choses ; mais je n'en ai point de honte ; car je connais celui en qui j'ai cru, et je suis persuadé qu'il est puissant pour garder mon dépôt jusqu'à cette journée-là.
\VS{13}Retiens le vrai patron des saines paroles que tu as entendues de moi, dans la foi et dans la charité qui est en Jésus-Christ.
\VS{14}Garde le bon dépôt par le Saint-Esprit qui habite en nous.
\VS{15}Tu sais ceci, que tous ceux qui [sont] en Asie, se sont éloignés de moi ; entre lesquels sont Phygelle et Hermogène.
\VS{16}Le Seigneur fasse miséricorde à la maison d'Onésiphore : car souvent il m'a consolé, et il n'a point eu honte de ma chaîne ;
\VS{17}Au contraire, quand il a été à Rome, il m'a cherché très soigneusement, et il m'a trouvé.
\VS{18}Le Seigneur lui fasse trouver miséricorde envers le Seigneur en cette journée-là ; et tu sais mieux [que personne] combien il m'a rendu de services à Ephèse.
\Chap{2}
\VerseOne{}Toi donc, mon fils, sois fortifié dans la grâce qui est en Jésus- Christ.
\VS{2}Et les choses que tu as entendues de moi devant plusieurs témoins, commets-les à des personnes fidèles, qui soient capables de les enseigner aussi à d'autres.
\VS{3}Toi donc, endure les travaux, comme un bon soldat de Jésus-Christ.
\VS{4}Nul qui va à la guerre ne s'embarrasse des affaires de cette vie, afin qu'il plaise à celui qui l'a enrôlé pour la guerre.
\VS{5}De même, si quelqu'un combat dans la lice, il n'est point couronné s'il n'a pas combattu selon les lois.
\VS{6}Il faut [aussi] que le laboureur travaille premièrement, et ensuite il recueille les fruits.
\VS{7}Considère ce que je dis ; or le Seigneur te donne de l’intelligence en toutes choses.
\VS{8}Souviens-toi que Jésus-Christ, qui est de la semence de David, est ressuscité des morts, selon mon Evangile.
\VS{9}Pour lequel je souffre beaucoup de maux, jusqu'à être mis dans les chaînes, comme un malfaiteur ; mais cependant la parole de Dieu n'est point liée.
\VS{10}C'est pourquoi je souffre tout pour l'amour des élus, afin qu'eux aussi obtiennent le salut qui est en Jésus-Christ, avec la gloire éternelle.
\VS{11}Cette parole est certaine, que si nous mourons avec lui, nous vivrons aussi avec lui.
\VS{12}Si nous souffrons avec lui, nous régnerons aussi avec lui ; si nous le renions, il nous reniera aussi.
\VS{13}Si nous sommes des perfides, il demeure fidèle : il ne se peut renier soi-même.
\VS{14}Remets ces choses en mémoire, protestant devant le Seigneur qu'on ne dispute point de paroles, qui est une chose dont il ne revient aucun profit, [mais] elle est la ruine des auditeurs.
\VS{15}Etudie-toi de te rendre approuvé à Dieu, ouvrier sans reproche, enseignant purement la parole de la vérité.
\VS{16}Mais réprime les disputes vaines et profanes, car elles passeront plus avant dans l'impiété ;
\VS{17}Et leur parole rongera comme une gangrène, et entre ceux-là sont Hyménée et Philète ;
\VS{18}Qui se sont écartés de la vérité, en disant que la résurrection est déjà arrivée, et qui renversent la foi de quelques-uns.
\VS{19}Toutefois le fondement de Dieu demeure ferme, ayant ce sceau : le Seigneur connaît ceux qui sont siens ; et, quiconque invoque le nom de Christ, qu'il se retire de l'iniquité.
\VS{20}Or dans une grande maison il n'y a pas seulement des vaisseaux d'or et d'argent, mais il y en a aussi de bois et de terre : les uns à honneur, et les autres à déshonneur.
\VS{21}Si quelqu'un donc se purifie de ces choses, il sera un vaisseau sanctifié à honneur, et utile au Seigneur, et préparé à toute bonne œuvre.
\VS{22}Fuis aussi les désirs de la jeunesse, et recherche la justice, la foi, la charité, et la paix avec ceux qui invoquent d'un cœur pur le Seigneur.
\VS{23}Et rejette les questions folles, et qui sont sans instruction, sachant qu'elles ne font que produire des querelles.
\VS{24}Or il ne faut pas que le serviteur du Seigneur soit querelleur, mais doux envers tout le monde, propre à enseigner, supportant patiemment les mauvais.
\VS{25}Enseignant avec douceur ceux qui ont un sentiment contraire, [afin d'essayer] si quelque jour Dieu leur donnera la repentance pour reconnaître la vérité ;
\VS{26}Et afin qu'ils se réveillent [pour sortir] des pièges du Démon, par lequel ils ont été pris pour faire sa volonté.
\Chap{3}
\VerseOne{}Or sache ceci, qu'aux derniers jours il surviendra des temps fâcheux.
\VS{2}Car les hommes seront idolâtres d'eux-mêmes, avares, vains, orgueilleux, blasphémateurs, désobéissants à leurs pères et à leurs mères, ingrats, profanes ;
\VS{3}Sans affection naturelle, sans fidélité, calomniateurs, incontinents, cruels, haïssant les gens de bien ;
\VS{4}Traîtres, téméraires, enflés [d'orgueil], amateurs des voluptés, plutôt que de Dieu.
\VS{5}Ayant l'apparence de la piété, mais en ayant renié la force : éloigne-toi donc de telles gens.
\VS{6}Or d'entre ceux-ci sont ceux qui se glissent dans les maisons, et qui tiennent captives les femmes chargées de péchés, et agitées de diverses convoitises ;
\VS{7}Qui apprennent toujours, mais qui ne peuvent jamais parvenir à la pleine connaissance de la vérité.
\VS{8}Et comme Jannès et Jambrès ont résisté à Moise, ceux-ci de même résistent à la vérité ; [étant des] gens qui ont l'esprit corrompu, et qui sont réprouvés quant à la foi.
\VS{9}Mais ils n'avanceront pas plus avant : car leur folie sera manifestée à tous, comme le fut celle de ceux-là.
\VS{10}Mais pour toi, tu as pleinement compris ma doctrine, ma conduite, mon intention, ma foi, ma douceur, ma charité, ma patience.
\VS{11}Et tu [sais] les persécutions et les afflictions qui me sont arrivées à Antioche, à Iconie, et à Lystre, quelles persécutions, [dis-je], j'ai soutenues, et [comment] le Seigneur m'a délivré de toutes.
\VS{12}Or tous ceux aussi qui veulent vivre selon la piété en Jésus-Christ, souffriront persécution.
\VS{13}Mais les hommes méchants et séducteurs iront en empirant, séduisant, et étant séduits.
\VS{14}Mais toi, demeure ferme dans les choses que tu as apprises, et qui t'ont été confiées, sachant de qui tu les as apprises ;
\VS{15}Vu même que dès ton enfance tu as la connaissance des saintes Lettres, qui te peuvent rendre sage à salut, par la foi en Jésus-Christ.
\VS{16}Toute l'Ecriture est divinement inspirée, et utile pour enseigner, pour convaincre, pour corriger, et pour instruire selon la justice ;
\VS{17}Afin que l'homme de Dieu soit accompli, et parfaitement instruit pour toute bonne œuvre.
\Chap{4}
\VerseOne{}Je te somme devant Dieu, et devant le Seigneur Jésus-Christ, qui doit juger les vivants et les morts, en son apparition et en son règne.
\VS{2}Prêche la parole, insiste dans toutes les occasions ; reprends, censure, exhorte avec toute douceur d'esprit, et avec doctrine.
\VS{3}Car le temps viendra auquel ils ne souffriront point la saine doctrine, mais aimant qu'on leur chatouille les oreilles, [par des discours agréables] ils chercheront des Docteurs qui répondent à leurs désirs.
\VS{4}Et ils détourneront leurs oreilles de la vérité, et se tourneront aux fables.
\VS{5}Mais toi, veille en toutes choses, souffre les afflictions, fais l'œuvre d'un Evangéliste, rends ton Ministère pleinement approuvé.
\VS{6}Car pour moi, je m'en vais maintenant être mis pour l'aspersion du sacrifice, et le temps de mon départ est proche.
\VS{7}J'ai combattu le bon combat, j'ai achevé la course, j'ai gardé la foi.
\VS{8}Au reste, la couronne de justice m'est réservée, et le Seigneur, juste juge, me la rendra en cette journée-là, et non seulement à moi, mais aussi à tous ceux qui auront aimé son apparition.
\VS{9}Hâte-toi de venir bientôt vers moi.
\VS{10}Car Démas m'a abandonné, ayant aimé le présent siècle, et il s'en est allé à Thessalonique ; Crescens est allé en Galatie ; [et] Tite en Dalmatie.
\VS{11}Luc est seul avec moi ; prends Marc, et amène-le avec toi : car il m'est fort utile pour le Ministère.
\VS{12}J'ai aussi envoyé Tychique à Ephèse.
\VS{13}Quand tu viendras apporte avec toi le manteau que j'ai laissé à Troas, chez Carpus, et les Livres aussi ; mais principalement mes parchemins.
\VS{14}Alexandre le forgeron m'a fait beaucoup de maux, le Seigneur lui rendra selon ses œuvres.
\VS{15}Garde-toi donc de lui, car il s'est fort opposé à nos paroles.
\VS{16}Personne ne m'a assisté dans ma première défense, mais tous m'ont abandonné ; [toutefois] que cela ne leur soit point imputé !
\VS{17}Mais le Seigneur m'a assisté, et fortifié, afin que ma prédication fût rendue pleinement approuvée, et que tous les Gentils l'ouïssent ; et j'ai été délivré de la gueule du Lion.
\VS{18}Le Seigneur aussi me délivrera de toute mauvaise œuvre, et me sauvera dans son Royaume céleste. A lui [soit] gloire aux siècles des siècles, Amen !
\VS{19}Salue Prisce et Aquilas, et la famille d'Onésiphore.
\VS{20}Eraste est demeuré à Corinthe, et j'ai laissé Trophime malade à Milet.
\VS{21}Hâte-toi de venir avant l'hiver. Eubulus et Pudens, et Linus, et Claudia, et tous les frères, te saluent.
\VS{22}Le Seigneur Jésus-Christ soit avec ton esprit. Que la grâce soit avec vous, Amen !
\PPE{}
\end{multicols}
