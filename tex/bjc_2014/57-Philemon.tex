\ShortTitle{Philémon}\BookTitle{Philémon}\BFont
\noindent\hrulefill
{\footnotesize
\textit{
\bigskip
{\centering{}
\\Signifie : Attentionné, qui embrasse
\\Thème : Un exemple d'amour
\\Auteur : Paul
\\Date de rédaction : Env. 60\\}
}
%\bigskip
\textit{
\\Paul écrivit cette lettre en prison lors de sa deuxième captivité à Rome vers l’été 62 en même temps que l’épître aux Colossiens. Elle s’adressait à Philémon, chrétien fortuné de Colosses ainsi qu’à sa femme Appia, son fils Archippe et l’église qui se réunissait dans leur maison. Paul demanda à Philémon de pardonner Onésime, son esclave, de s’être échappé d’auprès de lui. Paul assura à Philémon que désormais une nouvelle relation le lierait à Onésime qui avait accepté Jésus-Christ dans sa vie. Il alla même jusqu’à proposer de payer personnellement ce qu’Onésime lui devait tout en exprimant l’espoir que Philémon ferait plus que ce qu’il lui demandait. Ainsi Paul plaida pour Onésime comme Christ le fit en notre faveur. Nous retrouvons dans cette lettre la manière dont la Parole doit être appliquée en cas de différend.\bigskip
}
}
\par\nobreak\noindent\hrulefill
\begin{multicols}{2}
\Chap{1}
\TextTitle{Introduction}
\VerseOne{}Paul, prisonnier de Jésus-Christ, et le frère Timothée, à Philémon notre bien-aimé et compagnon d'œuvre ;
\VS{2}et à Apphia, notre bien-aimée, et à Archippe, notre compagnon de combat, et à l'Eglise qui est dans ta maison.
\VS{3}Que la grâce et la paix vous soient données de la part de Dieu notre Père, et de la part du Seigneur Jésus-Christ.
\TextTitle{Philémon}
\VS{4}Je rends grâces à mon Dieu, faisant toujours mention de toi dans mes prières ;
\VS{5}Apprenant la foi que tu as au Seigneur Jésus, et ta charité envers tous les saints.
\VS{6}Afin que la communication de ta foi devienne efficace, en se faisant connaître par tout le bien qui est en vous, au nom de Jésus-Christ.
\VS{7}Car, mon frère, nous avons une grande joie et une grande consolation de ta charité, en ce que tu as réjoui les entrailles des saints.
\TextTitle{Paul plaide en faveur d'Onésime}
\VS{8}C'est pourquoi, bien que j'aie une grande liberté en Christ de t’ordonner ce qui est bienséant,
\VS{9}cependant je te prie plutôt par la charité, bien que je suis ce que je suis, à savoir Paul, un vieillard, et même maintenant prisonnier de Jésus-Christ ;
\VS{10}je te prie donc pour mon fils Onésime, que j'ai engendré dans mes liens ;
\VS{11}qui t'a été autrefois inutile, mais qui maintenant est bien utile et à toi et à moi, et que je te renvoie.
\VS{12}Reçois-le donc comme mes propres entrailles.
\VS{13}Je voulais le retenir auprès de moi, afin qu'il me serve à ta place, dans les liens de l'Evangile.
\VS{14}Mais je n'ai rien voulu faire sans ton avis, afin que ce ne soit point comme par contrainte, mais volontairement, que tu me laisses un bien qui est à toi.
\VS{15}Car peut-être n'a-t-il été séparé de toi que pour un temps, afin que tu le recouvres\FTNT{Synonymes : retrouver, reconquérir, regagner.} pour toujours.
\VS{16}Non plus comme un esclave, mais comme étant au-dessus d'un esclave, à savoir, comme un frère bien-aimé, principalement de moi ; et combien plus de toi, soit selon la chair, soit selon le Seigneur ?
\VS{17}Si donc tu me tiens pour ton compagnon, reçois-le comme moi-même.
\VS{18}Et s'il t'a fait quelque tort, ou s'il te doit quelque chose, mets-le sur mon compte.
\VS{19}Moi Paul, j'ai écrit ceci de ma propre main, je payerai ; pour ne pas te dire que tu te dois toi-même à moi.
\VS{20}Oui, mon frère, que je reçoive ce plaisir de toi en notre Seigneur ; réjouis mes entrailles en notre Seigneur.
\VS{21}Je t'ai écrit m'assurant de ton obéissance, et sachant que tu feras même plus que ce que je te dis.
\TextTitle{Conclusion}
\VS{22}Mais aussi, en même temps prépare-moi un logement ; car j'espère que je vous serai remis par vos prières.
\VS{23}Epaphras, qui est prisonnier avec moi en Jésus-Christ, te salue ;
\VS{24}Marc aussi, Aristarque, Démas, et Luc, mes compagnons d'œuvre.
\VS{25}Que la grâce de notre Seigneur Jésus-Christ soit avec votre esprit, Amen !
\PPE{}
\end{multicols}
