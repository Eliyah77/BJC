\ShortTitle{2 Corinthiens}\BookTitle{2 Corinthiens}\BFont
\noindent\hrulefill
{\footnotesize
\textit{
\bigskip
{\centering{}
\\Auteur : Paul
\\Thème : L'autorité de Paul
\\Date de rédaction : Env. 57 ap. J.-C.\\}
}
%\bigskip
\textit{
\\Dans l'antiquité, Corinthe, capitale de l'Achaïe, était la ville la plus prospère et la plus puissante de Grèce. Située sur
un isthme séparant la mer Egée de la mer Ionienne, Corinthe était au carrefour de l'Asie et de l'Italie et constituait un
véritable centre commercial où les produits orientaux et occidentaux se croisaient.
%\bigskip
\\Rédigée quelques mois après la première, la seconde lettre de Paul aux Corinthiens fait état d'une vague de méfiance à l'égard de Paul et exprime les souffrances qui furent les siennes et qui somme toute authentifient son apostolat.\bigskip
}
}
\par\nobreak\noindent\hrulefill
\begin{multicols}{2}
\Chap{1}
\TextTitle{Introduction}
\VerseOne{}Paul, apôtre de Jésus-Christ par la volonté de Dieu, et le frère Timothée, à l'église de Dieu qui est à Corinthe, avec tous les saints qui sont dans toute l'Achaïe.
\VS{2}Que la grâce et la paix vous soient données par Dieu notre Père et par le Seigneur Jésus-Christ.
\TextTitle{Consolation de Paul dans ses afflictions}
\VS{3}Béni soit Dieu, qui est le Père de notre Seigneur Jésus-Christ, le Père des miséricordes et le Dieu de toute consolation,
\VS{4}qui nous console dans toute notre affliction, afin que par la consolation dont nous sommes nous-mêmes 
consolés de Dieu, nous puissions consoler ceux qui sont en quelque affliction que ce soit.
\VS{5}Car comme les souffrances de Christ abondent en nous, de même notre consolation abonde aussi par Christ.
\VS{6}Et si nous sommes affligés, c'est pour votre consolation et votre salut, qui s'effectue en endurant les mêmes souffrances dont nous aussi souffrons ; et si nous sommes réconfortés, c'est pour votre consolation et votre salut.
\VS{7}Et l'espérance que nous avons de vous est ferme, sachant que comme vous êtes participants des souffrances, de même aussi vous le serez de la consolation.
\VS{8}Car mes frères, nous ne voulons pas que vous ignoriez l'affliction qui nous est arrivée en Asie, c'est que nous avons été accablés excessivement, au-delà de nos forces, de telle sorte que nous avions perdu l'espérance de conserver notre vie.
\VS{9}Car nous nous sommes vus comme si nous avions reçu en nous même la sentence de mort, afin que nous n'ayons point de confiance en nous-mêmes, mais en Dieu qui ressuscite les morts.
\VS{10}Et qui nous a délivré d'une si grande mort et qui nous en délivre, et en qui nous espérons qu'il nous délivrera aussi à l'avenir.
\VS{11}Etant aussi aidés par la prière que vous faites pour nous, afin que des actions de grâces soient rendues pour nous par plusieurs personnes à cause du don qui nous aura été fait en faveur de plusieurs.
\TextTitle{La gloire d'un témoignage pur}
\VS{12}Car c'est ici notre gloire, savoir le témoignage de notre conscience, que nous nous sommes conduits dans le monde, et particulièrement avec vous, avec simplicité et sincérité de Dieu, non point avec une sagesse charnelle, mais selon la grâce de Dieu.
\VS{13}Car nous ne vous écrivons pas d'autres choses que celles que vous lisez et que mêmes vous connaissez. Et j'espère que vous les reconnaîtrez aussi jusqu'à la fin,
\VS{14}selon que vous avez reconnu en partie que nous sommes votre gloire, comme vous êtes aussi la nôtre pour le jour du Seigneur Jésus.
\TextTitle{L'attitude de Paul}
\VS{15}Et dans une telle confiance je voulais premièrement aller vers vous, afin que vous ayez une seconde grâce ;
\VS{16}et passer de chez vous en Macédoine, puis de Macédoine revenir vers vous, et être accompagné par vous en Judée.
\VS{17}Or quand je me proposais cela, ai-je usé de légèreté ? Ou les choses que je propose, sont-elles proposées selon la chair, de sorte qu'il y ait eu en moi le oui et le non ?
\VS{18}Mais Dieu est fidèle, notre parole, que nous vous avons adressée, n'a pas été oui et non.
\VS{19}Car le Fils de Dieu, Jésus-Christ, qui a été prêché par nous au milieu de vous, savoir par moi, et par Silvain, et par Timothée, n'a pas été oui et non, mais il a été oui en lui.
\VS{20}Car autant qu'il y a de promesses de Dieu, toutes sont oui en lui, et amen en lui, à la gloire de Dieu par nous.
\VS{21}Or celui qui nous affermit avec vous en Christ, et qui nous a oints, c'est Dieu,
\VS{22}qui nous a aussi marqués d'un sceau, et nous a donné les arrhes\FTNT{Du grec « arrhabon » : arrhes ; monnaie donnée en gage d'un futur paiement en attendant que le solde soit payé.} de l'Esprit dans nos cœurs.
\VS{23}Or je prends Dieu à témoin sur mon âme, que c'est pour vous épargner que je ne suis plus allé à Corinthe.
\VS{24}Non que nous dominions sur votre foi, mais nous contribuons à votre joie, puisque vous demeurez fermes dans la foi.
\Chap{2}
\TextTitle{L'œuvre de la repentance}
\VerseOne{}Mais j'avais résolu en moi-même de ne pas retourner chez vous avec tristesse.
\VS{2}Car si je vous attriste, qui est-ce qui me réjouira, à moins que ce ne soit celui que j'aurai moi-même affligé ?
\VS{3}Et je vous ai même écrit ceci, afin que quand j'arriverai, je n'ai point de tristesse de la part de ceux de qui je devais recevoir de la joie, ayant cette confiance à l'égard de vous tous que ma joie est celle de vous tous.
\VS{4}Car je vous ai écrit dans une grande affliction et angoisse de cœur, avec beaucoup de larmes, non afin que vous soyez attristés, mais afin que vous connaissiez la charité\FTNT{Littéralement « agape » : amour fraternel.} toute particulière que j'ai pour vous.
\VS{5}Si quelqu'un a été la cause de cette tristesse, ce n'est pas moi seul qu'il a attristé, mais en quelque sorte (afin que je ne vous surcharge pas) c'est vous tous.
\VS{6}C'est assez pour cet homme, de la correction qui lui a été faite par plusieurs,
\VS{7}de sorte que vous devez bien plutôt lui faire grâce et le consoler, afin qu'un tel homme ne soit pas accablé par une trop grande tristesse.
\VS{8}C'est pourquoi je vous prie de confirmer publiquement envers lui votre charité.
\VS{9}Car c'est aussi pour cela que je vous ai écrit, afin de vous éprouver, et de connaître si vous êtes obéissants en toutes choses.
\VS{10}Or à celui à qui vous pardonnez quelque chose, je pardonne aussi ; car de ma part aussi si j'ai pardonné quelque chose à celui à qui j'ai pardonné, je l'ai fait à cause de vous, devant la face de Christ,
\VS{11}afin que Satan n'ait pas le dessus sur nous, car nous n'ignorons pas ses machinations.
\VS{12}Au reste, étant venu à Troas pour l'Evangile de Christ, quoique la porte m'y fût ouverte par le Seigneur, je n'ai pourtant point eu de repos en mon esprit, parce que je n'ai pas trouvé Tite, mon frère ;
\VS{13}mais ayant pris congé d'eux, je partis pour la Macédoine.
\TextTitle{Le parfum de Christ est répandu}
\VS{14}Or grâces soient rendues à Dieu, qui nous fait toujours triompher en Christ, et qui manifeste par nous l'odeur de sa connaissance en tout lieu.
\VS{15}Car nous sommes la bonne odeur de Christ de la part de Dieu, parmi ceux qui sont sauvés et parmi ceux qui périssent :
\VS{16}Aux uns, une odeur mortelle qui les tue ; aux autres, une odeur vivifiante qui les conduit à la vie. Mais qui est suffisant pour ces choses ?
\VS{17}Car nous ne falsifions pas la parole de Dieu, comme font plusieurs, mais nous parlons de Christ avec sincérité, comme de la part de Dieu et devant Dieu.
\Chap{3}
\TextTitle{Les Corinthiens : Lettres de Christ}
\VerseOne{}Commençons-nous de nouveau à nous recommander nous-mêmes ? Ou avons-nous besoin, comme quelques-uns, de lettres de recommandation auprès de vous, ou de lettres de recommandation envers vous ?
\VS{2}Vous êtes vous-mêmes notre lettre, écrite dans nos cœurs, connue et lue de tous les hommes.
\VS{3}Car il paraît en vous que vous êtes la lettre de Christ, gravée par notre ministère, non avec de l'encre, mais avec l'Esprit du Dieu vivant, non sur des tables de pierre, mais sur les tables de chair, qui sont vos cœurs.
\VS{4}Or nous avons une telle confiance en Dieu par Christ.
\VS{5}Non que nous soyons capables par nous-mêmes de penser quelque chose, comme de nous-mêmes, mais notre capacité vient de Dieu,
\TextTitle{Paul ministre de la Nouvelle Alliance}
\VS{6}qui nous a aussi rendus capables d'être ministres de la Nouvelle Alliance\FTNT{Dans la plupart des versions, le mot grec « diatheke » a été traduit par « testament » alors que ce mot signifie aussi « alliance ». On le retrouve notamment dans les passages suivants : Mt. 26:28 ; Mc. 14:24 ; Lu. 1:72 ; 22:20 ; Ac. 3:25 ; 7:8 ; Ro. 9:4 ; 11:27 ; 1 Co. 11:25 ; Ga. 3:15,17 ; 4:24 ; Ep. 2:12 ; Hé. 7:22 ; 8:6,8-9 ; 9:4,15-17,20 ; 10:16,29 ; 12:24 ; 13:20 ; Ap. 11:19. Le fait d'avoir regroupé les écrits de Genèse à Malachie sous l'appellatif « Ancien Testament » a induit beaucoup de chrétiens en erreur. L'Ancienne Alliance correspond uniquement à la loi cérémonielle de Moïse qui a été accomplie par Christ à la croix (Jn. 19:30). Ainsi, avant la mort du Seigneur, on ne peut pas parler de testament puisqu'il faut qu'il y ait au préalable la mort du testateur. Or il est évident que les animaux sacrifiés sous la loi ne nous ont rien légué (Hé. 9:1-16).}, non de la lettre, mais de l'Esprit ; car la lettre tue, mais l'Esprit vivifie.
\VS{7}Or si le ministère de mort, écrit avec des lettres, et gravé sur des pierres, était glorieux au point que les enfants d'Israël ne pouvaient fixer les yeux sur le visage de Moïse, à cause de la gloire de son visage, laquelle devait prendre fin,
\VS{8}comment le ministère de l'Esprit ne sera-t-il pas plus glorieux ?
\VS{9}Car si le ministère de la condamnation a été glorieux, le ministère de la justice le surpasse de beaucoup en gloire.
\VS{10}Et même le premier ministère qui a été si glorieux, ne l'a pas été autant que le second qui le surpasse de beaucoup en gloire.
\VS{11}Car si ce qui devait prendre fin a été glorieux, ce qui est permanent est beaucoup plus glorieux.
\VS{12}Ayant donc une telle espérance, nous usons d'une grande liberté dans les paroles ;
\VS{13}et nous ne sommes pas comme Moïse qui mettait un voile sur son visage, afin que les enfants d'Israël ne fixent pas les regards sur la fin de ce qui devait être anéanti\FTNT{Ce mot grec signifie aussi : rendre vain, inemployé, inactif, cessé, abolir (voir Ep. 2:15, Romain 3:3; 3:31; 4:14}.
\VS{14}Mais leurs entendements sont endurcis, car jusqu'à aujourd'hui ce même voile qui est aboli\FTNT{Voir commentaire du verset 13} par Christ, demeure dans la lecture de l'Ancienne Alliance, sans être ôté.
\VS{15}Mais jusqu'à ce jour, quand on lit Moïse, un voile demeure sur leur cœur.
\VS{16}Mais quand le cœur se convertit au Seigneur, le voile est ôté.
\VS{17}Or le Seigneur est l'Esprit ; et là où est l'Esprit du Seigneur, là est la liberté.
\VS{18}Ainsi nous tous qui contemplons, comme dans un miroir la gloire du Seigneur à visage découvert, nous sommes transformés en la même image, de gloire en gloire, comme par l'Esprit du Seigneur.
\Chap{4}
\TextTitle{La vérité pratique du ministère de Paul}
\VerseOne{}C'est pourquoi, ayant ce ministère selon la miséricorde que nous avons reçue, nous ne nous relâchons point.
\VS{2}Mais nous avons entièrement rejeté les choses honteuses que l'on cache, ne marchant point avec ruse et ne falsifiant point la parole de Dieu, mais nous rendant approuvés à toute conscience d'homme devant Dieu, par la manifestation de la vérité.
\VS{3}Que si notre Evangile est encore voilé, il ne l'est que pour ceux qui périssent ;
\VS{4}pour les incrédules dont le dieu de ce siècle a aveuglé l'entendement, afin qu'ils ne soient pas éclairés par la lumière de l'Evangile de la gloire de Christ, lequel est l'image de Dieu.
\VS{5}Car nous ne nous prêchons pas nous-mêmes, mais nous prêchons Jésus-Christ le Seigneur, et nous déclarons que nous sommes vos serviteurs pour l'amour de Jésus.
\VS{6}Car Dieu qui a dit que la lumière resplendisse des ténèbres\FTNT{Ge. 1:3.}, est celui qui a resplendi dans nos coeurs, pour manifester la connaissance de la gloire de Dieu qui se trouve en Jésus-Christ.
\VS{7}Mais nous avons ce trésor dans des vases de terre, afin que l'excellence de cette puissance soit de Dieu et non pas de nous.
\TextTitle{Les souffrances de Paul}
\VS{8}Etant affligés à tous égards, mais non réduits entièrement à l'extrémité ; étant en perplexité, mais non sans secours ;
\VS{9}étant persécutés, mais non abandonnés ; étant abattus, mais non perdus ;
\VS{10}portant toujours partout dans notre corps la mort du Seigneur Jésus, afin que la vie de Jésus soit aussi manifestée dans notre corps.
\VS{11}Car nous qui vivons, nous sommes sans cesse livrés à la mort pour l'amour de Jésus, afin que la vie de Jésus soit aussi manifestée dans notre chair mortelle ;
\VS{12}de sorte que la mort opère en nous, mais la vie en vous.
\VS{13}Or ayant un même esprit de foi, selon qu'il est écrit : J'ai cru, c'est pourquoi j'ai parlé\FTNT{Ps. 116:10.} ! Nous croyons aussi et c'est aussi pourquoi nous parlons,
\VS{14}sachant que celui qui a ressuscité le Seigneur Jésus nous ressuscitera aussi par Jésus, et nous fera comparaître en sa présence avec vous.
\VS{15}Car toutes ces choses sont pour vous, afin que cette grâce surabonde, à la gloire de Dieu, par le remerciement de plusieurs.
\VS{16}C'est pourquoi nous ne nous relâchons pas. Mais quoique notre homme extérieur se détruit, toutefois l'intérieur est renouvelé de jour en jour.
\VS{17}Car notre légère affliction, qui ne fait que passer, produit en nous un poids éternel d'une gloire souverainement excellente,
\VS{18}puisque nous ne regardons point aux choses visibles, mais aux invisibles ; car les choses visibles ne sont que pour un temps, mais les invisibles sont éternelles.
\Chap{5}
\TextTitle{L'héritage céleste, espoir dans les souffrances}
\VerseOne{}Car nous savons que si notre habitation terrestre, qui n'est qu'une tente, est détruite, nous avons un édifice de Dieu qui n'a pas été fait de main d'homme, une maison éternelle dans les cieux.
\VS{2}Car c'est aussi pour cela que nous gémissons, désirant avec ardeur d'être revêtus de notre domicile céleste,
\VS{3}si toutefois nous sommes trouvés vêtus et non pas nus.
\VS{4}Car nous qui sommes dans cette tente, nous gémissons étant accablés, vu que nous désirons, non pas d'être dépouillés, mais d'être revêtus, afin que ce qui est mortel soit absorbé par la vie.
\VS{5}Or celui qui nous a formés à cela même, c'est Dieu qui nous a donné les arrhes de l'Esprit.
\VS{6}Nous avons donc toujours confiance ; et nous savons qu'en demeurant dans ce corps, nous sommes loin du Seigneur,
\VS{7}car nous marchons par la foi et non par la vue.
\VS{8}Nous avons, dis-je, de la confiance, et nous aimons mieux être absents de ce corps et être avec le Seigneur.
\VS{9}C'est pourquoi aussi nous nous efforçons de lui être agréables, et présents et absents.
\VS{10}Car il nous faut tous comparaître devant le tribunal\FTNT{Le Tribunal de Christ n'a pas vocation à déterminer le salut des enfants de Dieu. Les chrétiens y seront jugés en fonction des œuvres produites sur la terre. En effet, chacun devra rendre compte de ce qu'il aura fait et de la gestion des dons et ministères reçus. Voir Ro. 14:10 ; 1 Co. 4:4 ; 2 Co. 3:10-14 ; 2 Tim. 4:8.} de Christ, afin que chacun reçoive selon le bien ou le mal qu'il aura fait, étant dans son corps.
\TextTitle{Le ministère de la réconciliation}
\VS{11}Connaissant donc combien le Seigneur doit être craint, nous persuadons les hommes ; et nous sommes connus de Dieu, et j'espère que dans vos consciences, vous nous connaissez aussi.
\VS{12}Car nous ne nous recommandons pas de nouveau à vous, mais nous vous donnons l'occasion de vous glorifier de nous, afin que vous ayez de quoi répondre à ceux qui se glorifient de l'apparence, et non du cœur.
\VS{13}Car soit que nous soyons hors de sens, c'est pour Dieu ; soit que nous soyons de bon sens c'est pour vous.
\VS{14}Car la charité de Christ nous unit étroitement, tenant ceci pour certain, que si un est mort pour tous, tous aussi sont morts ;
\VS{15}et qu'il est mort pour tous, afin que ceux qui vivent, ne vivent point dorénavant pour eux-mêmes, mais pour celui qui est mort et ressuscité pour eux.
\VS{16}C'est pourquoi dès maintenant nous ne connaissons personne selon la chair ; et si même nous avons connu Christ selon la chair, toutefois nous ne le connaissons plus ainsi maintenant.
\VS{17}Si donc quelqu'un est en Christ, il est une nouvelle créature ; les choses anciennes sont passées ; voici, toutes choses sont faites nouvelles.
\VS{18}Or tout cela vient de Dieu, qui nous a réconciliés avec lui par Jésus-Christ et qui nous a donné le ministère de la réconciliation.
\VS{19}Car Dieu était en Christ, réconciliant le monde avec lui-même, en ne leur imputant point leurs péchés, et il a mis en nous la parole de la réconciliation.
\VS{20}Nous sommes donc ambassadeurs pour Christ, et c'est comme si Dieu vous exhortait par notre ministère ; nous vous supplions donc pour l'amour de Christ de vous réconcilier avec Dieu !
\VS{21}Car il a fait celui qui n'a point connu de péché, être péché pour nous, afin que nous devenions juste devant Dieu par lui.
\Chap{6}
\TextTitle{Le meilleur et le pire du ministère}
\VerseOne{}Ainsi donc étant ouvriers avec lui, nous vous prions aussi à ne pas recevoir la grâce de Dieu en vain.
\VS{2}Car il dit : Je t'ai exaucé au temps favorable et t'ai secouru au jour du salut\FTNT{Es. 49:8.} ; voici maintenant le temps favorable, voici maintenant le jour du salut.
\VS{3}Ne donnons aucun scandale en quoi que ce soit, afin que notre ministère ne soit point blâmé.
\VS{4}Mais nous nous rendons recommandables, en toutes choses, comme ministres de Dieu, par beaucoup de patience, dans les afflictions, dans les nécessités, dans les détresses,
\VS{5}sous les coups, dans les prisons, dans les troubles, dans les travaux, dans les veilles, dans les jeûnes,
\VS{6}par la pureté, par la connaissance, par un esprit patient, par la douceur, par le Saint-Esprit, par une charité sincère,
\VS{7}par la parole de la vérité, par la puissance de Dieu, par les armes de justice que l'on porte à la main droite et à la main gauche ;
\VS{8}au milieu de la gloire et de l'ignominie, au milieu de la calomnie et de la bonne réputation ; comme séducteurs, et toutefois vrais,
\VS{9}comme inconnus, et toutefois connus, comme mourants, et voici nous vivons, comme châtiés, et toutefois non mis à mort ;
\VS{10}comme attristés, et toutefois toujours joyeux ; comme pauvres, et toutefois enrichissant plusieurs ; comme n'ayant rien, et toutefois possédant toutes choses.
\TextTitle{Exhortation à la séparation et la sanctification}
\VS{11}Ô Corinthiens ! Notre bouche est ouverte pour vous, notre cœur s'est élargi.
\VS{12}Vous n'êtes point à l'étroit au-dedans de nous, mais vous êtes à l'étroit dans vos entrailles.
\VS{13}Or pour nous traiter de la même manière, je vous parle comme à mes enfants, élargissez-vous aussi à notre égard !
\VS{14}Ne portez pas un même joug avec les infidèles ; car quelle communion y a-t-il entre la justice et l'iniquité ? Ou quelle communion y a-t-il entre la lumière et les ténèbres ?
\VS{15}Et quel accord y a-t-il entre Christ et Bélial\FTNT{Bélial : De l'hébreu « beliya'al » : méchants, pervers, pervertis, vil, destruction, dangereusement. L'un des noms de Satan qui signifie « indignité, méchanceté, impiété ».} ? Ou quelle part a le fidèle avec l'infidèle ?
\VS{16}Et quel rapport y a-t-il entre le temple de Dieu et les idoles ? Car vous êtes le temple du Dieu vivant, selon ce que Dieu a dit : J'habiterai au milieu d'eux et j'y marcherai ; je serai leur Dieu et ils seront mon peuple\FTNT{Lé. 26:12 ; Ez. 37:26.}.
\VS{17}C'est pourquoi, sortez du milieu d'eux et séparez-vous, dit le Seigneur ; ne touchez à aucune chose souillée et je vous recevrai\FTNT{Es. 52:11 ; Ap. 18:4.}.
\VS{18}Je serai pour vous un Père et vous serez pour moi des fils et des filles, dit le Seigneur Tout-Puissant\FTNT{Jn. 1:12 ; Ap. 21:7.}.
\Chap{7}
\VerseOne{}Or donc mes bien-aimés, puisque nous avons de telles promesses, purifions-nous de toute souillure de la chair et de l'esprit, perfectionnant la sanctification dans la crainte de Dieu.
\TextTitle{Le cœur de Paul pour les Corinthiens}
\VS{2}Recevez-nous! Nous n'avons fait tort à personne, nous n'avons corrompu personne, nous n'avons pillé personne.
\VS{3}Je ne dis pas ceci pour vous condamner, car je vous ai déjà dit que vous êtes dans nos cœurs à la vie et à la mort.
\VS{4}J'ai une grande liberté envers vous, j'ai grand sujet de me glorifier de vous ; je suis rempli de consolation, je suis comblé de joie au milieu de toutes nos afflictions.
\VS{5}Car depuis notre arrivée en Macédoine, notre chair n'eut aucun repos, mais nous avons été affligés de toute manière, ayant eu des combats au dehors et des craintes au dedans.
\VS{6}Mais Dieu qui console les abattus nous a consolés par l'arrivée de Tite,
\VS{7}et non seulement par son arrivée, mais aussi par la consolation qu'il a reçue de vous ; car il nous a raconté votre grand désir, vos larmes, votre affection ardente pour moi, de sorte que je m'en suis extrêmement réjoui.
\VS{8}Car bien que je vous aie attristés par ma lettre, je ne m'en repens pas, quoique je m'en sois déjà repenti, parce que je vois que si cette lettre vous a affligés, cela n'a été que pour peu de temps.
\VS{9}Je me réjouis donc maintenant, non de ce que vous avez été affligés, mais de ce que vous avez été portés à la repentance ; car vous avez été attristés selon Dieu, de sorte que vous n'avez reçu aucun dommage de notre part.
\VS{10}Puisque la tristesse qui est selon Dieu produit une repentance à salut dont on ne se repent jamais, mais que la tristesse du monde produit la mort.
\VS{11}Car voici, cette même tristesse qui est selon Dieu, quel empressement n'a-t-elle pas produit en vous ! Quelle justification, quelle indignation, quelle crainte, quel grand désir, quel zèle, quelle vengeance! Vous vous êtes montrés de toutes manières purs dans cette affaire.
\VS{12}Quoi que je vous aie donc écrit, ce n'était ni à cause de celui qui a commis la faute, ni à cause de celui envers qui elle a été commise, mais pour faire voir parmi vous l'empressement que j'ai de vous devant Dieu.
\VS{13}C'est pourquoi nous avons été consolés de ce que vous avez fait pour notre consolation. Mais nous nous sommes encore plus réjouis de la joie qu'a eu Tite, en ce que son esprit a été tranquillisé par vous tous.
\VS{14}Parce que si en quelque chose je me suis glorifié de vous auprès de lui, je n'en ai pas été confus ; mais comme nous avons toujours dit toutes choses selon la vérité, ainsi ce dont je m'étais glorifié auprès de Tite s'est trouvé être la vérité même.
\VS{15}C'est pourquoi quand il se souvient de l'obéissance de vous tous, et comment vous l'avez reçu avec crainte et tremblement, son affection pour vous en est plus grande.
\VS{16}Je me réjouis donc de ce qu'en toutes choses, j'ai confiance en vous.
\Chap{8}
\TextTitle{Collecte des Macédoniens en faveur des pauvres}
\VerseOne{}Au reste, mes frères, nous voulons vous faire connaître la grâce que Dieu a faite aux églises de la Macédoine.
\VS{2}Au travers de leur grande épreuve d'affliction, leur joie a été augmentée et leur profonde pauvreté s'est répandue en richesses par leur prompte libéralité.
\VS{3}Car je suis témoin qu'ils ont donné volontairement selon leurs moyens, et même au-delà de leurs moyens,
\VS{4}nous pressant avec de grandes prières de recevoir la grâce et de prendre part à cette contribution en faveur des saints.
\VS{5}Et ils n'ont pas fait seulement comme nous l'espérions, mais ils se sont donnés premièrement eux-mêmes au Seigneur, et puis à nous, par la volonté de Dieu ;
\VS{6}de sorte que nous avons exhorté Tite, comme il avait commencé auparavant, qu'il achève aussi cette grâce\FTNT{Autre traduction : « Œuvre de charité »} envers vous.
\TextTitle{Exemple du Messie}
\VS{7}C'est pourquoi, comme vous excellez en toutes choses, en foi, en parole, en connaissance, en zèle, et dans la charité que vous avez pour nous, faites en sorte d'exceller aussi dans cette grâce.
\VS{8}Je ne dis pas cela pour vous donner un ordre, mais pour éprouver par le zèle des autres la sincérité de votre charité.
\VS{9}Car vous connaissez la grâce de notre Seigneur Jésus-Christ qui, étant riche, s'est fait pauvre pour vous, afin que par sa pauvreté vous soyez enrichis.
\VS{10}Et en cela je vous donne mon avis, parce qu'il vous est profitable, à vous qui avez déjà commencé dès l'année passée, non seulement de faire, mais aussi de vouloir.
\VS{11}Maintenant donc, achevez de faire, afin que comme vous avez été prompts à en avoir la volonté ; vous l'accomplissiez aussi selon vos moyens.
\VS{12}Un esprit bien disposé, quand il existe, est agréable selon ce qu'il a, et non selon ce qu'il n'a pas.
\VS{13}Or ce n'est pas afin que les autres soient soulagés et que vous soyez dans la détresse, mais afin que ce soit par égalité. Que dans le temps présent, votre abondance supplée à leurs insuffisances
\VS{14}afin que leur abondance serve également à votre insuffisance, et qu'ainsi il y ait de l'égalité,
\VS{15}selon ce qui est écrit : Celui qui avait beaucoup n'a rien eu de superflu, et celui qui avait peu n'en a pas eu moins\FTNT{Ex. 16:18.}.
\TextTitle{Exemple des églises}
\VS{16}Et grâces soient rendues à Dieu qui a mis dans le cœur de Tite le même empressement pour vous ;
\VS{17}car il a reçu mon exhortation, et étant lui-même très zélé, il est allé chez vous volontairement.
\VS{18}Et nous avons aussi envoyé avec lui le frère dont la louange, qu'il s'est acquise dans la prédication de l'Evangile, est répandue par toutes les églises ;
\VS{19}Et non seulement cela, mais aussi il a été désigné par vote par les églises notre compagnon de voyage, pour cette grâce\FTNT{ également traduit par « les aumônes »} qui est administrée par nous à la gloire du Seigneur même, et selon l'ardeur de votre zèle.
\VS{20}Nous évitons ainsi que quelqu'un nous blâme au sujet de cette abondante collecte, qui est administrée par nous;
\VS{21}et prenant soin de faire ce qui est bon, non seulement devant le Seigneur, mais aussi devant les hommes.
\VS{22}Nous avons envoyé aussi avec eux notre autre frère, dont nous avons souvent éprouvé le zèle à plusieurs occasions, qui est maintenant encore plus zélé, à cause de la grande confiance qu'il a en vous.
\VS{23}Ainsi donc, quant à Tite, il est mon associé et mon compagnon d'œuvre auprès de vous ; et quant à nos frères, ils sont les envoyés des églises et la gloire de Christ.
\VS{24}Montrez donc envers eux et devant les églises une preuve de votre charité et du sujet que nous avons de nous glorifier de vous.
\Chap{9}
\TextTitle{Encouragement aux dons}
\VerseOne{}Car pour ce qui est de la collecte qui se fait pour les saints, il est superflu que je vous en écrive.
\VS{2}Car je connais votre esprit bien disposé, dont je me glorifie de vous devant ceux de Macédoine, leur disant que l'Achaïe est prête dès l'année dernière ; et votre zèle en a excité plusieurs.
\VS{3}J'ai envoyé ces frères, afin que ce en quoi je me suis glorifié de vous, ne soit pas vain en cette occasion, et que vous soyez prêts, comme j'ai dit.
\VS{4}De peur que ceux de Macédoine venant avec moi, et ne vous trouvant pas prêts, nous, pour ne pas dire vous-même, n'ayons de la honte de l'assurance avec laquelle nous nous sommes glorifiés de vous.
\VS{5}C'est pourquoi j'ai estimé qu'il était nécessaire de prier les frères d'aller premièrement vers vous, et d'achever de préparer votre libéralité que vous avez déjà promise ; afin qu'elle soit prête comme une libéralité et non pas comme un fruit de l'avarice.
\VS{6}Or je vous dis ceci, celui qui sème peu moissonnera aussi peu, et celui qui sème abondamment moissonnera aussi abondamment.
\VS{7}Mais que chacun contribue selon qu'il a résolu dans son cœur, non pas avec chagrin, non par contrainte; car Dieu aime celui qui donne avec joie.
\VS{8}Et Dieu est Tout-Puissant pour vous combler de toutes sortes de grâces, afin qu'ayant toujours tout ce qui suffit en toute chose, vous soyez abondants en toute bonne œuvre,
\VS{9}selon ce qui est écrit : Il a fait des largesses, il a donné aux pauvres ; sa justice demeure éternellement\FTNT{Ps. 112:9.}.
\VS{10}Que celui qui fournit de la semence au semeur veuille aussi vous donner du pain à manger, multiplier votre semence, et augmenter les revenus de votre justice ;
\VS{11}étant pleinement enrichis pour exercer une parfaite libéralité, laquelle fait que nous en rendons grâces à Dieu.
\VS{12}Car le service de ce don est non seulement suffisant pour subvenir aux nécessités des saints, mais il abonde aussi de telle sorte, que plusieurs ont de quoi en rendre grâces à Dieu.
\VS{13}Glorifiant Dieu pour l'épreuve qu'ils font de ce service, en ce que vous vous soumettez à l'Evangile de Christ ; et de votre prompte et libérale communication envers eux et envers tous.
\VS{14}Ils prient Dieu pour vous et ils vous aiment\FTNT{Aimer : Du grec « epipotheo » : désirer, chérir.} très affectueusement à cause de la grâce excellente que Dieu vous a accordée.
\VS{15}Grâces soient rendues à Dieu pour son don inexprimable.
\Chap{10}
\TextTitle{L'autorité apostolique de Paul lui vient de Dieu}
\VerseOne{}Au reste, je vous prie, moi Paul, par la douceur et la bonté de Christ, moi qui m'humilie lorsque je suis en votre présence, mais qui étant absent use de hardiesse à votre égard ;
\VS{2}je vous prie, dis-je, que lorsque je serai présent, je ne sois pas obligé de me servir avec confiance de cette hardiesse, avec laquelle j'ai dessein d'agir contre quelques-uns qui nous regardent comme marchant selon la chair.
\VS{3}Mais en marchant dans la chair, nous ne combattons pas selon la chair.
\VS{4}Car les armes de notre guerre ne sont pas charnelles, mais elles sont puissantes par la vertu de Dieu, pour la destruction des forteresses ;
\VS{5}détruisant les raisonnements et toute hauteur qui s'élèvent contre la connaissance de Dieu et amenant toute pensée captive à l'obéissance de Christ.
\VS{6}Et étant prêts à tirer vengeance de toute désobéissance, lorsque votre obéissance sera complète.
\VS{7}Considérez-vous les choses selon l'apparence ? Si quelqu'un se persuade qu'il est de Christ, qu'il se dise bien en lui-même que, comme il est de Christ, nous aussi nous sommes de Christ.
\VS{8}Car si même je veux me glorifier davantage de l'autorité que le Seigneur nous a donnée pour votre édification et non votre destruction, je ne saurais en avoir honte,
\VS{9}afin que je ne paraisse pas vouloir vous épouvanter par mes lettres.
\VS{10}Car mes lettres, disent-ils, sont sévères et puissantes, mais la présence de corps est faible, et la parole est méprisable.
\VS{11}Que celui qui est tel, considère que tels nous sommes en paroles dans nos lettres, étant absents, tels aussi nous sommes dans nos actes, étant présents.
\VS{12}Car nous n'osons pas nous joindre, ni nous comparer à quelques-uns de ceux qui se recommandent eux-mêmes. Mais en se mesurant à leur propre mesure et en se comparant à eux-mêmes, ils manquent de compréhension.
\VS{13}Mais pour nous, nous ne voulons pas nous glorifier de ce qui n'est pas de notre mesure, mais seulement dans la limite du champ d'action que Dieu nous a assigné en nous amenant jusqu'à vous.
\VS{14}Car nous ne nous étendons pas nous-même au delà des limites prescrites, comme si nous n'étions pas parvenus jusqu'à vous ; vu que nous sommes parvenus même jusqu'à vous par la prédication de l'Evangile de Christ.
\VS{15}Nous ne nous glorifions pas dans ce qui n'est pas de notre mesure, c'est-à-dire, dans les travaux d'autrui ; mais nous avons l'espérance que votre foi venant à croître en vous, nous serons amplement accrus dans ce qui nous a été départi selon la mesure réglée ;
\VS{16}jusqu'à évangéliser dans les lieux qui sont au-delà de chez vous, sans nous glorifier de ce qui a déjà été fait dans le domaine des autres.
\VS{17}Que celui qui se glorifie se glorifie dans le Seigneur.
\VS{18}Car ce n'est pas celui qui se recommande lui-même qui est approuvé, mais celui que le Seigneur recommande\FTNT{Le Seigneur recommande ses serviteurs, il témoigne d'eux auprès des autres (Ac. 10:1-48). Un véritable serviteur de Dieu laisse au Seigneur le soin de témoigner de lui auprès des autres alors que les faux ouvriers se recommandent eux-mêmes (2 Co. 3:1).}.
\Chap{11}
\VerseOne{}Plaise à Dieu que vous me supportiez un peu dans ma folie ! Je vous prie supportez-moi !
\VS{2}Car je suis jaloux de vous d'une jalousie de Dieu, parce que je vous ai fiancés à un seul époux, pour vous présenter à Christ comme une vierge pure.
\TextTitle{Mise en garde contre les ministres de Satan}
\VS{3}Mais je crains que comme le serpent séduisit Eve\FTNT{Ge. 3:1-6.} par sa ruse, vos pensées aussi ne se corrompent en se détournant de la simplicité qui est en Christ.
\VS{4}Car si quelqu'un vient vous prêcher un autre Jésus que nous n'avons pas prêché, ou si vous recevez un autre esprit que celui que vous avez reçu, ou un autre évangile que celui que vous avez embrassé, vous le supportez fort bien.
\VS{5}Mais j'estime que je n'ai été en rien moindre que les plus excellents apôtres.
\VS{6}Mais si je suis comme quelqu'un d'ignorant par rapport au langage, je ne le suis pourtant pas en connaissance, mais nous avons été entièrement manifestés en toutes choses envers vous.
\VS{7}Ai-je commis une faute en ce que je me suis abaissé moi-même, afin que vous soyez élevés, parce que je vous ai annoncé l'Evangile de Dieu sans rien prendre ? 
\VS{8}J'ai dépouillé les autres églises prenant de quoi m'entretenir pour vous servir. Et lorsque j'étais chez vous et que j'ai été dans la nécessité, je ne me suis point relâché du travail afin de n'être à charge à personne,
\VS{9}car les frères venus de la Macédoine ont pourvu à ce qui me manquait. Et en toutes choses, je me suis gardé d'être à votre charge et je m'en garderai encore.
\VS{10}Par la vérité de Christ qui est en moi, j'atteste que ce sujet de gloire ne me sera point ravi dans les contrées de l'Achaïe.
\VS{11}Pourquoi ? Est-ce parce que je ne vous aime point ? Dieu le sait !
\VS{12}Mais ce que je fais, je le ferai encore, pour ôter ce prétexte à ceux qui cherchent un prétexte, afin qu'ils soient trouvés tels que nous dans les choses dont ils se glorifient.
\VS{13}Car de tels hommes, sont de faux apôtres, des ouvriers trompeurs qui se déguisent en apôtres de Christ.
\VS{14}Et cela n'est pas étonnant car Satan lui-même se déguise en ange de lumière\FTNT{Satan est maître en matière de déguisement et d'imitation.}.
\VS{15}Ce n'est donc pas un grand sujet d'étonnement si ses ministres aussi se déguisent en ministres de justice ; mais leur fin sera conforme à leurs œuvres.
\TextTitle{Paul se glorifie de sa faiblesse\FTNTT{2 Co. 11:16-12:18}}
\VS{16}Je le dis encore afin que personne ne me regarde comme un insensé ; ou bien, supportez-moi comme un insensé afin que je me glorifie aussi un peu.
\VS{17}Ce que je vais dire, en rapportant les sujets que j'aurais de me glorifier, je ne le dirai pas selon le Seigneur, mais comme par folie.
\VS{18}Puisque plusieurs se glorifient selon la chair, moi aussi je me glorifierai.
\VS{19}Car vous supportez volontiers les insensés, vous qui êtes sages.
\VS{20}Même si quelqu'un vous asservit, si quelqu'un vous dévore, si quelqu'un prend votre bien, si quelqu'un s'élève sur vous, si quelqu'un vous frappe au visage, vous le supportez.
\VS{21}Je le dis avec honte, nous avons été faible. Mais si en quelque chose quelqu'un ose se glorifier, je parle en insensé, j'ai la même hardiesse !
\VS{22}Sont-ils Hébreux ? Je le suis aussi. Sont-ils Israélites ? Je le suis aussi. Sont-ils de la postérité d'Abraham ? Je le suis aussi.
\VS{23}Sont-ils ministres de Christ ? Je parle comme un insensé, je le suis plus qu'eux ; par les travaux, bien plus ; par les blessures, plus qu'eux ; par les emprisonnements, bien plus. Plusieurs fois en danger de mort,
\VS{24}cinq fois j'ai reçu des Juifs quarante coups moins un ;
\VS{25}j'ai été battu de verges trois fois, j'ai été lapidé une fois, j'ai fait naufrage trois fois, j'ai passé un jour et une nuit dans la mer profonde.
\VS{26}J'ai été souvent en voyage, en péril sur les fleuves, en péril de la part des brigands, en péril de la part de ceux de ma nation, en péril parmi les Gentils, en péril dans les villes, en péril dans les déserts, en péril sur la mer, en péril parmi de faux frères.
\VS{27}Dans les peines et dans le travail, dans de fréquentes veilles, dans la faim, dans la soif, souvent dans les jeûnes, dans le froid et dans la nudité.
\VS{28}Outre les choses de dehors, ce qui me tient assiégé tous les jours, c'est le souci que j'ai de toutes les églises.
\VS{29}Qui est affaibli, que je ne sois aussi affaibli ? Qui est scandalisé, que je n'en sois aussi brûlé ?
\VS{30}S'il faut se glorifier, je me glorifierai des choses qui sont de mon infirmité.
\VS{31}Dieu, qui est le Père de notre Seigneur Jésus-Christ et qui est béni éternellement, sait que je ne mens point.
\VS{32}A Damas, le gouverneur du roi Arétas faisait garder la ville des Damascéniens voulant se saisir de moi,
\VS{33}mais on me descendit de la muraille dans une corbeille, par une fenêtre et ainsi j'échappai de ses mains.
\Chap{12}
\VerseOne{}Certes, il ne me convient pas de me glorifier, car j'en viendrai jusqu'aux visions et aux révélations du Seigneur.
\VS{2}Je connais un homme en Christ il y a quatorze ans passés, si ce fut en corps je ne sais ; si ce fut hors du corps, je ne sais ; Dieu le sait, qui a été ravi jusqu'au troisième ciel.
\VS{3}Et je sais que cet homme, si ce fut dans son corps ou si ce fut hors du corps, je ne sais pas ; Dieu le sait,
\VS{4}a été ravi dans le paradis, et a entendu des secrets qu'il n'est pas permis à l'homme de révéler.
\TextTitle{L'écharde de Paul}
\VS{5}Je me glorifierai d'un tel homme, mais je ne me glorifierai pas de moi-même, si ce n'est dans mes infirmités. 
\VS{6}Or quand je voudrais me glorifier, je ne serais point imprudent, car je dirais la vérité ; mais je m'en abstiens, afin que personne ne m'estime au-dessus de ce qu'il me voit être, ou de ce qu'il entend dire de moi.
\VS{7}Mais de peur que je ne m'élève à cause de l'excellence des révélations, il m'a été mis une écharde\FTNT{La nature exacte de l'écharde de Paul ne nous est pas détaillée. Elle lui avait été infligée par un « ange de Satan », par la volonté de Dieu. Nous constatons une chose qui est commune à tous les enfants de Dieu : Paul avait un adversaire constamment aux aguets pour essayer de le décourager, le détruire ou l'intimider en s'opposant par tous les moyens à la mission que le Seigneur lui avait confiée. Cette écharde était aussi un moyen utilisé par Dieu pour garder Paul dans l'humilité.} dans la chair, un ange de Satan pour me souffleter, afin que je ne m'élève point.
\VS{8}C'est pourquoi j'ai prié trois fois le Seigneur de faire que cet ange de Satan se retire de moi.
\VS{9}Mais le Seigneur m'a dit : Ma grâce te suffit car ma puissance s'accomplit dans la faiblesse. Je me glorifierai donc bien volontiers plutôt dans mes faiblesses afin que la puissance de Christ habite en moi.
\VS{10}Et à cause de cela je prends plaisir dans les faiblesses, dans les injures, dans les nécessités, dans les persécutions, et dans les angoisses pour Christ ; car quand je suis faible, c'est alors que je suis fort.
\TextTitle{Mise en garde de Paul}
\VS{11}J'ai été insensé en me glorifiant, mais vous m'y avez contraint ; car je devais être recommandé par vous, vu que je n'ai été inférieur en aucune chose aux plus excellents apôtres, quoique je ne sois rien.
\VS{12}Certainement les marques de mon apostolat se sont accomplies parmi vous par une patience entière, par des signes, des prodiges et des miracles.
\VS{13}Car en quoi avez-vous été inférieurs aux autres églises, sinon en ce que je n'ai point été à votre charge ? Pardonnez-moi ce tort.
\VS{14}Voici pour la troisième fois que je suis prêt d'aller vers vous, et je ne m'épargnerai pas à travailler, pour ne pas être à charge ; car je ne demande pas vos biens, mais c'est vous-mêmes que je demande. Car ce ne sont pas les enfants qui doivent amasser pour leurs pères, mais les pères pour leurs enfants\FTNT{Un bon père amasse dans le but de préparer l'avenir de ses enfants et non l'inverse.}.
\VS{15}Et quant à moi, je dépenserai très volontiers et je me dépenserai entièrement pour vos âmes ; bien que vous aimant beaucoup plus, je sois moins aimé. 
\VS{16}Mais soit ! Dira-t-on, que je ne vous ai point été à charge, mais, étant un homme rusé, je vous ai pris par finesse !
\VS{17}Ai-je donc tiré profit de vous par quelqu'un de ceux que je vous ai envoyés ?
\VS{18}J'ai exhorté Tite d'aller, et j'ai envoyé un de nos frères avec lui. Mais Tite a-t-il retiré du profit de vous ? N'avons-nous pas lui et moi marché d'un même esprit ? N'avons-nous pas marché sur les mêmes traces ?
\VS{19}Avez-vous encore la pensée que nous voulions nous justifier envers vous ? Nous parlons devant Dieu en Christ, et tout cela, mes très chers frères, pour votre édification.
\VS{20}Car je crains qu'il n'arrive que quand je viendrai, je ne vous trouve pas tels que je voudrais, et que vous ne me trouviez pas tel que vous voudriez, et qu'il n'y ait des contestations, des jalousies, des animosités, des dissensions, des médisances, des rapports, de l'orgueil et des séditions.
\VS{21}Et qu'étant revenu chez vous, mon Dieu ne m'humilie à votre sujet, en sorte que je sois affligé à l'occasion de plusieurs de ceux qui ont péché auparavant, et qui ne se sont point repentis de l'impureté, de la fornication et de l'impudicité dont ils se sont rendus coupables.
\Chap{13}
\TextTitle{Invitation à l'examen de soi}
\VerseOne{}C'est ici la troisième fois que je viens à vous. Par la bouche de deux ou de trois témoins,\FTNT{De. 19:15.} toute parole sera confirmée.
\VS{2}Je l'ai déjà dit, et je le dis encore comme lorsque j'étais présent pour la seconde fois, et maintenant étant absent, j'écris à ceux qui ont péché auparavant et à tous les autres, que si je viens encore une fois, je n'épargnerai personne.
\VS{3}Puisque vous cherchez la preuve que Christ parle par moi, lui qui n'est point faible envers vous, mais qui est puissant en vous.
\VS{4}Car quoiqu'il ait été crucifié par faiblesse, il est néanmoins vivant par la puissance de Dieu ; et nous aussi nous souffrons les mêmes faiblesses à cause de lui, mais nous vivrons avec lui par la puissance que Dieu a déployée envers vous.
\VS{5}Examinez-vous vous-mêmes pour savoir si vous êtes dans la foi ; éprouvez-vous vous-mêmes. Ne reconnaissez-vous pas que Jésus-Christ est en vous ? A moins peut-être que vous ne soyez réprouvés.
\VS{6}Mais j'espère que vous reconnaîtrez que pour nous, nous ne sommes pas réprouvés.
\VS{7}Et je prie Dieu que vous ne fassiez aucun mal, non afin que nous soyons trouvés approuvés, mais afin que vous fassiez ce qui est bon et que nous soyons comme réprouvés.
\VS{8}Car nous n'avons aucun pouvoir contre la vérité, mais nous n'en avons que pour la vérité.
\VS{9}Et nous nous réjouissons si nous sommes faibles, et que vous, vous êtes forts ; et ce que nous demandons, c'est votre perfectionnement.
\VS{10}C'est pourquoi j'écris ces choses étant absent, afin que quand je serai présent, je n'use point de rigueur, selon l'autorité que le Seigneur m'a donnée pour l'édification et non pour la destruction.
\TextTitle{Conclusion}
\VS{11}Au reste, mes frères, réjouissez-vous, perfectionnez-vous, consolez-vous, ayez un même sentiment, vivez en paix ; et le Dieu de charité et de paix sera avec vous.
\VS{12}Saluez-vous les uns les autres par un saint baiser. Tous les saints vous saluent.
\VS{13}Que la grâce du Seigneur Jésus-Christ, la charité de Dieu et la communication du Saint-Esprit soient avec vous tous. Amen !
\PPE{}
\end{multicols}
