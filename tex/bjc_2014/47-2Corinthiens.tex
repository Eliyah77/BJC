\ShortTitle{2 Corinthiens}\BookTitle{2 Corinthiens}\BFont
\noindent\hrulefill
{\footnotesize
\textit{
\bigskip
{\centering{}
\\Auteur : Paul avec Tite et Luc
\\Thème : L'autorité de Paul
\\Date de rédaction : Env. 57 ap. J.-C.\\}
}
%\bigskip
\textit{
\\Dans l’antiquité, Corinthe, capitale de l’Achaïe, était la ville la plus prospère et la plus puissante de Grèce. Située sur
un isthme séparant la mer Egée de la mer Ionienne, Corinthe était au carrefour de l’Asie et de l’Italie et constituait un
véritable centre commercial où les produits orientaux et occidentaux se croisaient.
%\bigskip
\\Rédigée quelques mois après la première, la seconde lettre de Paul aux Corinthiens fait état d’une vague de méfiance à l’égard de Paul et exprime les souffrances qui furent les siennes et qui somme toute authentifient son apostolat.\bigskip
}
}
\par\nobreak\noindent\hrulefill
\begin{multicols}{2}
\Chap{1}
\TextTitle{Introduction}
\VerseOne{}Paul, apôtre de Jésus-Christ par la volonté de Dieu, et le frère Timothée, à l'église de Dieu qui est à Corinthe, et à tous les saints qui sont dans toute l'Achaïe.
\VS{2}Que la grâce et la paix vous soient données de la part de Dieu notre Père et du Seigneur Jésus-Christ.
\TextTitle{Consolation de Paul dans ses afflictions}
\VS{3}Béni soit Dieu, le Père de notre Seigneur Jésus-Christ, le Père des miséricordes et le Dieu de toute consolation,
\VS{4}qui nous console dans toutes nos afflictions, afin que par la consolation dont nous sommes l’objet de la part de Dieu, nous puissions consoler ceux qui se trouvent dans l’affliction.
\VS{5}Car de même que les souffrances de Christ abondent en nous, de même notre consolation abonde par Christ.
\VS{6}Et si nous sommes affligés, c'est pour votre consolation et pour votre salut ; si nous sommes consolés, c'est pour votre consolation et pour votre salut qui se réalise par la patience à supporter les mêmes souffrances que nous endurons aussi.
\VS{7}Et l'espérance que nous avons de vous est ferme, sachant que comme vous êtes participants des souffrances, de même aussi vous le serez de la consolation.
\VS{8}Car mes frères, nous ne voulons pas que vous ignoriez l’affliction qui nous est survenue en Asie, que nous avons été excessivement accablés, au-delà de nos forces, de telle sorte que nous avions perdu l'espérance de conserver notre vie.
\VS{9}Et nous regardions comme certain notre arrêt de mort, afin de ne pas placer notre confiance en nous-mêmes, mais en Dieu qui ressuscite les morts.
\VS{10}C’est lui qui nous a délivrés et qui nous délivrera d'une si grande mort, et en qui nous espérons qu'il nous délivrera aussi à l'avenir.
\VS{11}Etant aussi aidés par la prière que vous faites pour nous, afin que la grâce obtenue pour nous par plusieurs soit pour plusieurs une occasion de rendre grâces à notre sujet.
\TextTitle{La sincérité de Paul dans son ministère}
\VS{12}Car ce qui fait notre gloire c’est le témoignage de notre conscience, que nous nous sommes conduits dans le monde, et surtout à votre égard, avec simplicité et sincérité de Dieu, non point avec une sagesse charnelle, mais avec la grâce de Dieu.
\VS{13}Nous ne vous écrivons pas autre chose que ce que vous lisez, et vous-mêmes le reconnaissez. Et j'espère que vous les reconnaîtrez aussi jusqu'à la fin,
\VS{14}de même que vous avez reconnu en partie que nous sommes votre gloire, comme vous serez aussi la nôtre au jour du Seigneur Jésus.
\TextTitle{Sa manière d'agir}
\VS{15}C’est dans une telle confiance que je voulais premièrement aller vers vous, afin que vous ayez une seconde grâce ;
\VS{16}et passer de chez vous en Macédoine, puis de Macédoine revenir vers vous, et être accompagné par vous en Judée.
\VS{17}Or quand je me proposais cela, ai-je usé de légèreté ? ou les choses  que je propose, sont-elles proposées selon la chair, de sorte qu'il y ait eu en moi le oui et le non ?
\VS{18}Mais Dieu est fidèle, la parole que nous vous avons adressée n’a pas été oui et non.
\VS{19}Car le Fils de Dieu, Jésus-Christ, qui a été prêché par nous au milieu de vous, par moi, par Silvain, et par Timothée, n'a pas été oui et non, mais il a été oui en lui.
\VS{20}Car autant qu’il y a de promesses de Dieu, elles sont oui en lui, et amen en lui, afin que Dieu soit glorifié par nous.
\VS{21}Or celui qui nous affermit avec vous en Christ, et qui nous a oints, c'est Dieu,
\VS{22}lequel nous a aussi marqués d’un sceau, et a mis dans nos cœurs les arrhes\FTNT{Du grec «~arrhabon~»~: arrhes~; monnaie donnée en gage d’un futur paiement, en attendant que le solde soit payé.} de l'Esprit.
\VS{23}Or j'appelle Dieu à témoin sur mon âme, que c’est pour vous épargner que je ne suis plus allé à Corinthe.
\VS{24}Non que nous dominions sur votre foi, mais nous contribuons à votre joie, puisque vous demeurez fermes dans la foi.
\Chap{2}
\TextTitle{Les fruits de la repentance}
\VerseOne{}Je résolus en moi-même de ne pas retourner chez vous avec tristesse.
\VS{2}Car si je vous attriste, qui peut me réjouir, sinon celui que j'aurai moi-même affligé ?
\VS{3}Je vous ai écrit ceci, pour ne pas éprouver à mon arrivée de la tristesse de la part de ceux de qui je devais recevoir de la joie, ayant en vous tous cette confiance que ma joie est la vôtre à tous.
\VS{4}Je vous ai écrit dans une grande affliction et angoisse de cœur, avec beaucoup de larmes, non pas afin que vous soyez attristés, mais afin que vous connaissiez la charité\FTNT{Littéralement «~agape~»~: amour fraternel.} toute particulière que j'ai pour vous.
\VS{5}Si quelqu'un a été la cause de cette tristesse, ce n'est pas moi seul qu'il a attristé, afin que je ne le surcharge point, mais en quelque sorte c'est vous tous.
\VS{6}C'est assez pour cet homme, de la correction qui lui a été faite par plusieurs,
\VS{7}en sorte que vous devez bien plutôt lui pardonner et le consoler, de peur qu’il ne soit accablé par une trop grande tristesse.
\VS{8}C'est pourquoi je vous prie de confirmer envers lui votre charité.
\VS{9}C’est aussi pour cela que je vous ai écrit, afin de vous éprouver, et de connaître si vous êtes obéissants en toutes choses.
\VS{10}Or à celui à qui vous pardonnez quelque chose, je pardonne aussi ; si j’ai pardonné à celui à qui j'ai pardonné, c’est à cause de vous, en présence de Christ,
\VS{11}afin que Satan n'ait pas le dessus sur nous, car nous n'ignorons pas ses machinations.
\VS{12}Au reste, lorsque je fus arrivé à Troas pour l'Evangile de Christ, quoique la porte m'y fût ouverte par le Seigneur, je n’eus point de repos en mon esprit, parce que je ne trouvai pas Tite, mon frère ;
\VS{13}mais ayant pris congé d'eux, je partis pour la Macédoine.
\VS{14}Grâces soient rendues à Dieu, qui nous fait toujours triompher en Christ, et qui manifeste par nous l'odeur de sa connaissance en tout lieu.
\VS{15}Nous sommes, en effet, pour Dieu le parfum de Christ, parmi ceux qui sont sauvés, et parmi ceux qui périssent :
\VS{16}Aux uns, une odeur mortelle, pour la mort ; aux autres, une odeur vivifiante, pour la vie. Mais qui est suffisant pour ces choses ?
\VS{17}Car nous ne falsifions pas la parole de Dieu, comme font plusieurs, mais nous parlons de Christ avec sincérité, comme de la part de Dieu, et devant Dieu.
\Chap{3}
\TextTitle{Les corinthiens : lettre de Christ écrite avec l'Esprit du Dieu vivant}
\VerseOne{}Commençons-nous de nouveau à nous recommander nous-mêmes ? Ou avons-nous besoin, comme quelques-uns, de lettres de recommandation auprès de vous, ou de lettres de recommandation de votre part ?
\VS{2}Vous êtes vous-mêmes notre lettre de recommandation, écrite dans nos cœurs, connue et lue de tous les hommes.
\VS{3}Car il est évident que vous êtes la lettre de Christ, écrite par notre ministère, non avec de l'encre, mais avec l'Esprit du Dieu vivant, non sur des tables de pierre, mais sur les tables de chair, qui sont vos cœurs.
\VS{4}Cette assurance-là, nous l’avons par Christ auprès de Dieu.
\VS{5}Ce n’est pas à dire que nous soyons par nous-mêmes capables de penser quelque chose comme venant de nous-mêmes. Notre capacité, au contraire, vient de Dieu.
\TextTitle{Paul ministre de la nouvelle alliance}
\VS{6}Il nous a rendus capables d'être ministres de la nouvelle alliance\FTNT{Dans la plupart des versions, le mot grec «~diatheke~» a été traduit par «~testament~» alors que ce mot signifie aussi «~alliance~». On le retrouve notamment dans les passages suivants~: Mt. 26:28~; Mc. 14:24~; Lu. 1:72~; Lu. 22:20~; Ac. 3:25~; Ac. 7:8~; Ro. 9:4~; Ro. 11:27~; 1 Co. 11:25~; Ga. 3:15~; Ga. 3:17~; Ga. 4:24~; Ep. 2:12~; Hé. 7:22~; Hé. 8:6~; Hé. 8:8-9~; Hé. 9:4~; Hé. 9:15-17~; Hé. 9:20~; Hé. 10:16~; Hé. 10:29~; Hé. 12:24~; Hé. 13:20~; Ap. 11:19. Le fait d’avoir regroupé les écrits de Genèse à Malachie sous l’appellatif «~Ancien Testament~» a induit beaucoup de chrétiens en erreur. L’Ancienne Alliance correspond uniquement à la loi cérémonielle de Moïse qui a été accomplie par Christ à la croix (Jn. 19:30). Ainsi, avant la mort du Seigneur, on ne peut pas parler de testament puisqu’il faut qu’il y ait au préalable la mort du testateur. Or il est évident que les animaux sacrifiés sous la loi ne nous ont rien légué (Hé. 9:1-16).}, non de la lettre, mais de l'Esprit ; car la lettre tue, mais l'Esprit vivifie.
\VS{7}Or si le ministère de la mort, écrit et gravé avec des lettres, sur des pierres, a été glorieux au point que les enfants d'Israël ne pouvaient regarder fixement le visage de Moïse, à cause de la gloire de son visage, bien que cette gloire devait disparaître,
\VS{8}comment le ministère de l'Esprit ne sera-t-il pas plus glorieux ?
\VS{9}Car si le ministère de la condamnation a été glorieux, le ministère de la justice le surpasse de beaucoup en gloire.
\VS{10}Et même ce premier ministère qui a été si glorieux, ne l'a pas été en comparaison du second qui le surpasse de beaucoup en gloire.
\VS{11}Car si ce qui devait disparaître a été glorieux, ce qui est permanent est beaucoup plus glorieux.
\VS{12}Ayant donc une telle espérance, nous parlons avec une grande liberté,
\VS{13}et nous ne faisons pas comme Moïse qui mettait un voile sur son visage, afin que les enfants d'Israël ne fixent pas les regards sur la fin d’un éclat qui devait disparaître.
\VS{14}Mais ils sont devenus durs d’entendement. Car jusqu'à ce jour, ce même voile qui n’est ôté que par Christ, demeure sur leur cœur quand ils font la lecture de l'ancienne alliance.
\VS{15}Jusqu'à ce jour, quand on lit Moïse, un voile est jeté sur leur cœur.
\VS{16}Mais lorsque les cœurs se convertissent au Seigneur, le voile est ôté.
\VS{17}Or le Seigneur c’est l’Esprit ; et là où est l'Esprit du Seigneur, là est la liberté.
\VS{18}Nous tous qui contemplons comme dans un miroir la gloire du Seigneur à visage découvert, nous sommes transformés en la même image, de gloire en gloire, par l'Esprit du Seigneur.
\Chap{4}
\TextTitle{La vérité pratique de son ministère}
\VerseOne{}C'est pourquoi, ayant ce ministère selon la miséricorde qui nous été faite, nous ne perdons pas courage.
\VS{2}Mais nous avons entièrement rejeté les choses honteuses que l'on cache, ne marchant point avec ruse, et ne falsifiant point la parole de Dieu, mais nous rendant approuvés à toute conscience des hommes devant Dieu, par la manifestation de la vérité.
\VS{3}Que si notre Evangile est encore voilé, il ne l'est que pour ceux qui périssent ;
\VS{4}pour les incrédules dont le dieu de ce siècle a aveuglé l’esprit, afin qu’ils ne soient pas éclairés par la lumière de l'Evangile de la gloire de Christ, lequel est l'image de Dieu.
\VS{5}Car nous ne nous prêchons pas nous-mêmes mais nous prêchons Jésus-Christ le Seigneur, et nous déclarons que nous sommes vos serviteurs pour l'amour de Jésus.
\VS{6}Car Dieu qui a dit que la lumière resplendisse des ténèbres\FTNT{Ge. 1:3.}, est celui qui a resplendi dans nos coeurs, pour manifester la connaissance de la gloire de Dieu en la présence de Jésus-Christ.
\VS{7}Mais nous avons ce trésor dans des vases de terre, afin que l'excellence de cette puissance soit attribuée à Dieu, et non pas à nous.
\TextTitle{Les souffrances de Paul}
\VS{8}Etant affligés à tous égards, mais non réduits entièrement à l’extrémité ; étant en perplexité, mais non sans secours ;
\VS{9}étant persécutés, mais non abandonnés ; étant abattus, mais non perdus ;
\VS{10}portant toujours partout dans notre corps la mort du Seigneur Jésus, afin que la vie de Jésus soit aussi manifestée dans notre corps.
\VS{11}Car nous qui vivons, nous sommes sans cesse livrés à la mort pour l'amour de Jésus, afin que la vie de Jésus soit aussi manifestée dans notre chair mortelle.
\VS{12}De sorte que la mort agit en nous, et la vie agit en vous.
\VS{13}Or ayant un même esprit de foi, selon qu'il est écrit : J'ai cru, c'est pourquoi j'ai parlé\FTNT{Ps. 116:10.} ! Nous aussi nous croyons, et c'est aussi pourquoi nous parlons,
\VS{14}sachant que celui qui a ressuscité le Seigneur Jésus nous ressuscitera aussi par Jésus, et nous fera comparaître avec vous en sa présence.
\VS{15}Car toutes ces choses sont pour vous, afin que cette grâce en se multipliant, fasse abonder, à la gloire de Dieu, les actions de grâces d’un plus grand nombre.
\VS{16}C'est pourquoi nous ne nous relâchons pas. Mais quoique notre homme extérieur se détruit, toutefois l'intérieur est renouvelé de jour en jour.
\VS{17}Car nos légères afflictions du moment produisent pour nous, au-delà de toute mesure, un poids éternel d'une gloire souverainement excellente,
\VS{18}quand nous ne regardons point aux choses visibles, mais aux invisibles ; car les choses visibles ne sont que pour un temps, mais les invisibles sont éternelles.
\Chap{5}
\TextTitle{Ses ambitions}
\VerseOne{}Nous savons, en effet, que si cette tente où nous habitons sur la terre est détruite, nous avons un édifice qui est l’ouvrage de Dieu, une maison éternelle dans les cieux, qui n’a pas été faite de main d’homme.
\VS{2}Et c'est à cause de cela que nous gémissons dans cette tente désirant avec ardeur revêtir notre domicile céleste,
\VS{3}si toutefois nous sommes trouvés vêtus, et non pas nus.
\VS{4}Car tandis que nous sommes dans cette tente, nous gémissons, accablés, parce que nous désirons, non pas nous dépouiller, mais nous revêtir, afin que ce qui est mortel soit absorbé par la vie.
\VS{5}Et celui qui nous a formés pour cela c'est Dieu qui nous a donné les arrhes de l'Esprit.
\VS{6}Nous sommes donc toujours pleins de confiance, et nous savons qu’en demeurant dans ce corps, nous demeurons loin du Seigneur,
\VS{7}car nous marchons par la foi, et non par la vue.
\VS{8}Nous sommes pleins de confiance, et nous aimons mieux quitter ce corps pour être avec le Seigneur.
\VS{9}C'est pourquoi aussi nous nous efforçons de lui être agréables, soit que nous demeurions dans ce corps, soit que nous le quittions.
\VS{10}Car il nous faut tous comparaître devant le tribunal\FTNT{Le Tribunal de Christ n’a pas vocation à déterminer le salut des enfants de Dieu. Les chrétiens y seront jugés en fonction des œuvres produites sur la terre. En effet, chacun devra rendre compte de ce qu’il aura fait et de la gestion des dons et ministères reçus. Voir Ro. 14:10~; 1 Co. 4:4~; 2 Co. 3:10-14~; 2 Tim. 4:8.} de Christ, afin que chacun reçoive selon le bien ou le mal qu’il aura fait étant dans son corps.
\TextTitle{Ses motifs d'action}
\VS{11}Connaissant donc la crainte du Seigneur, nous cherchons à convaincre les hommes ; et Dieu nous connaît, et j’espère que dans vos consciences vous nous connaissez aussi.
\VS{12}Car nous ne nous recommandons pas de nouveau nous-mêmes auprès de vous, mais nous vous donnons l'occasion de vous glorifier à notre sujet, afin que vous ayez de quoi répondre à ceux qui se glorifient de l'apparence, et non pas de ce qui est dans le cœur.
\VS{13}En effet, si je suis hors de sens c’est pour Dieu ; si je suis de bon sens c’est pour vous.
\VS{14}Parce que la charité de Christ nous unit étroitement, parce que nous estimons que, si un est mort pour tous, tous aussi sont morts ;
\VS{15}et qu'il est mort pour tous, afin que ceux qui vivent, ne vivent plus pour eux-mêmes, mais pour celui qui est mort et ressuscité pour eux.
\VS{16}Ainsi, dès maintenant nous ne connaissons personne selon la chair ; et si nous avons connu Christ selon la chair, nous ne le connaissons plus de cette manière.
\VS{17}Si donc quelqu'un est en Christ, il est une nouvelle créature ; les choses anciennes sont passées ; voici, toutes choses sont devenues nouvelles.
\VS{18}Et tout cela vient de Dieu, qui nous a réconciliés avec lui par Jésus-Christ, et qui nous a donné le ministère de la réconciliation.
\VS{19}Car Dieu était en Christ, réconciliant le monde avec lui-même, en n’imputant point aux hommes leurs péchés, et il a mis en nous la parole de la réconciliation.
\VS{20}Nous faisons donc la fonction d’ambassadeurs pour Christ, comme si Dieu vous exhortait par notre ministère ; nous vous en supplions pour l’amour de Christ : Soyez réconciliés avec Dieu !
\VS{21}Celui qui n’a point connu de péché, il l’a fait devenir péché pour nous, afin que nous devenions en lui justice de Dieu.
\Chap{6}
\TextTitle{Son humilité}
\VerseOne{}Puisque nous travaillons avec le Seigneur, nous vous prions de ne pas recevoir la grâce de Dieu en vain.
\VS{2}Car il dit : Je t'ai exaucé au temps favorable et t'ai secouru au jour du salut\FTNT{Es. 49:8.} ; voici maintenant le temps favorable, voici maintenant le jour du salut.
\VS{3}Nous ne voulons scandaliser personne en quoi que ce soit, afin que notre ministère ne soit point blâmé.
\VS{4}Mais nous nous rendons recommandables à tous égards, comme ministres de Dieu, par beaucoup de patience dans les afflictions, dans les calamités, dans les détresses,
\VS{5}sous les coups, dans les prisons, dans les séditions, dans les travaux, dans les veilles, dans les jeûnes,
\VS{6}par la pureté, par la connaissance, par la persévérance, par la douceur, par le Saint-Esprit, par une charité sincère,
\VS{7}par la parole de vérité, par la puissance de Dieu, par les armes offensives et défensives de la justice ;
\VS{8}au milieu de la gloire et de l'ignominie, au milieu de la mauvaise et de la bonne réputation ; étant regardés comme imposteurs, quoique véridiques,
\VS{9}comme inconnus, quoique bien connus, comme mourants, et voici nous vivons, comme châtiés, quoique non mis à mort,
\VS{10}comme attristés, et nous sommes toujours joyeux, comme pauvres, et nous en enrichissons plusieurs, comme n'ayant rien, et nous possédons toutes choses.
\TextTitle{Appel à la séparation et à la purification}
\VS{11}Ô Corinthiens ! Notre bouche s’est ouverte pour vous, notre cœur s'est élargi.
\VS{12}Vous n’y êtes point à l'étroit, mais c’est votre cœur qui s’est rétréci pour nous.
\VS{13}Rendez-nous la pareille (je vous parle comme à mes enfants) élargissez aussi votre cœur !
\VS{14}Ne vous mettez pas sous un joug étranger avec les infidèles. Car quel rapport y a-t-il entre la justice et l'iniquité ? Ou qu’y a-t-il de commun entre la lumière et les ténèbres ?
\VS{15}Et quel accord y a-t-il entre Christ et Bélial\FTNT{Bélial~: De l’hébreu «~beliya’al~»~: méchants, pervers, pervertis, vil, destruction, dangereusement. L’un des noms de Satan qui signifie «~indignité, méchanceté, impiété~».} ? Ou quelle part a le fidèle avec l'infidèle ?
\VS{16}Et quel rapport y a-t-il entre le temple de Dieu et les idoles ? Car vous êtes le temple du Dieu vivant, selon ce que Dieu a dit : J'habiterai et je marcherai au milieu d'eux ; je serai leur Dieu, et ils seront mon peuple\FTNT{Lé. 26:12~; Ez. 37:26.}.
\VS{17}C'est pourquoi sortez du milieu d'eux, et séparez-vous, dit le Seigneur ; ne touchez pas à ce qui est impur, et je vous accueillerai\FTNT{Es. 52:11~; Ap. 18:4.}.
\VS{18}Je serai pour vous un Père, et vous serez pour moi des fils et des filles, dit le Seigneur Tout-Puissant\FTNT{Jn. 1:12~; Ap. 21:7.}.
\Chap{7}
\VerseOne{}Ayant donc mes bien-aimés de telles promesses, purifions-nous de toute souillure de la chair et de l'esprit, en achevant notre sanctification dans la crainte de Dieu.
\TextTitle{Paul ouvre son coeur aux Corinthiens}
\VS{2}Recevez-nous, nous n'avons fait tort à personne, nous n'avons corrompu personne, nous n'avons pillé personne.
\VS{3}Je ne dis pas ceci pour vous condamner, car je vous ai déjà dit que vous êtes dans nos cœurs à la vie et à la mort.
\VS{4}J'ai une grande confiance en vous, j'ai tout sujet de me glorifier de vous ; je suis rempli de consolation, je suis comblé de joie au milieu de toutes nos afflictions.
\VS{5}Car depuis notre arrivée en Macédoine, notre chair n’eut aucun repos, mais nous avons été affligés de toute manière, ayant eu des combats au dehors, et des craintes au dedans.
\VS{6}Mais Dieu qui console les abattus nous a consolés par l’arrivée de Tite,
\VS{7}et non seulement par son arrivée, mais aussi par la consolation qu'il a reçue de vous ; car il nous a raconté votre grand désir, vos larmes, votre affection ardente pour moi, en sorte que je m'en suis extrêmement réjoui.
\VS{8}Quoique je vous aie attristés par ma lettre, je ne m'en repens pas. Et si je m'en suis repenti, car je vois que cette lettre vous a affligés, bien que momentanément,
\VS{9}je me réjouis donc maintenant, non pas de ce que vous avez été affligés, mais de ce que votre tristesse vous a portés à la repentance ; car vous avez été attristés selon Dieu, afin de ne recevoir de notre part aucun dommage.
\VS{10}En effet, la tristesse selon Dieu produit une repentance à salut dont on ne se repent jamais, tandis que la tristesse du monde produit la mort.
\VS{11}En effet, cette tristesse qui est selon Dieu, quel empressement n’a-t-elle pas produit en vous ? Quelle justification ? Quelle indignation ? Quelle crainte ? Quel désir ardent ? Quel zèle ? Quelle punition ? Vous avez montré à tous égards que vous étiez purs dans cette affaire.
\VS{12}Si donc je vous ai écrit, ce n’était ni à cause de celui qui a fait l’injure, ni à cause de celui qui l’a reçue, mais pour vous faire connaître à tous le soin que nous prenons de vous devant Dieu.
\VS{13}C'est pourquoi nous avons été consolés. Mais outre notre consolation, nous nous sommes réjouis beaucoup plus encore par la joie de Tite, dont l’esprit a été tranquillisé par vous tous.
\VS{14}Et si je me suis glorifié de vous devant lui en quelque chose, je n’en ai point eu de confusion ; mais comme nous avons toujours parlé selon la vérité, ce dont nous nous sommes glorifiés auprès de Tite s’est trouvé être aussi la vérité.
\VS{15}C'est pourquoi quand il se souvient de votre obéissance à tous, et de l’accueil que vous lui avez réservé avec crainte et tremblement, son affection pour vous devient plus grande.
\VS{16}Je me réjouis de pouvoir en toutes choses me confier en vous.
\Chap{8}
\TextTitle{Exemple des Macédoniens concernant la collecte en faveur des pauvres de Jérusalem}
\VerseOne{}Au reste, mes frères, nous voulons vous faire connaître la grâce que Dieu a faite aux églises de la Macédoine.
\VS{2}A travers la grande épreuve de leur affliction, leur joie débordante et leur profonde pauvreté ont produit avec abondance de riches libéralités.
\VS{3}Car je suis témoin qu'ils ont donné volontairement selon leurs moyens, et même au-delà de leurs moyens,
\VS{4}nous demandant avec de grandes instances la grâce de prendre part aux aumônes et à la contribution en faveur des saints.
\VS{5}Et non seulement ils ont contribué comme nous l’espérions, mais ils se sont donnés premièrement eux-mêmes au Seigneur, et puis à nous, par la volonté de Dieu.
\VS{6}Nous avons engagé Tite à achever chez vous cette œuvre de charité, comme il l’avait commencée.
\TextTitle{Exemple du Messie}
\VS{7}C'est pourquoi, comme vous excellez en toutes choses, en foi, en parole, en connaissance, en toutes sortes de soins, et dans la charité que vous avez pour nous, faites en sorte d’exceller aussi dans cette œuvre de charité.
\VS{8}Je ne dis pas cela pour vous donner un ordre, mais pour éprouver par l’empressement des autres la sincérité de votre charité.
\VS{9}Car vous connaissez la grâce de notre Seigneur Jésus-Christ qui, étant riche, s'est fait pauvre pour vous, afin que par sa pauvreté vous soyez enrichis.
\VS{10}C’est un avis que je donne là-dessus, car cela vous convient, à vous qui non seulement avez commencé à agir pour cette collecte, mais qui en avez eu la volonté dès l’année dernière.
\VS{11}Achevez donc maintenant ce que vous avez commencé, afin que l’accomplissement selon vos moyens réponde à l’empressement que vous avez mis à vouloir.
\VS{12}La bonne volonté, quand elle existe, est agréable en raison de ce qu’elle peut avoir à sa disposition, et non de ce qu’elle n’a pas.
\VS{13}Je ne veux pas vous exposer à la détresse pour soulager les autres, mais suivre une règle d’égalité. Dans la circonstance présente, votre superflu pourvoira à leurs besoins,
\VS{14}afin que leur superflu pourvoie pareillement aux vôtres, en sorte qu’il y ait égalité,
\VS{15}selon ce qui est écrit : Celui qui avait ramassé beaucoup, n'avait rien de trop ; et celui qui avait ramassé peu, n'en manquait pas\FTNT{Ex. 16:18.}.
\TextTitle{Exemple des églises}
\VS{16}Grâces soient rendues à Dieu qui a mis dans le cœur de Tite le même empressement pour vous ;
\VS{17}car il a bien accueilli notre demande, et c’est avec un nouveau zèle et de son plein gré qu’il part pour aller chez vous.
\VS{18}Et nous avons aussi envoyé avec lui le frère dont la louange en ce qui concerne la prédication de l'Evangile est répandue dans toutes les églises ;
\VS{19}de plus, il a été choisi par les suffrages des églises pour être notre compagnon de voyage et pour porter les aumônes que nous accomplissons à la gloire du Seigneur même, et afin de répondre à l’ardeur de votre zèle.
\VS{20}Nous agissons ainsi afin que personne ne nous blâme au sujet de ces aumônes dont nous avons la charge ;
\VS{21}car nous recherchons ce qui est bien, non seulement devant le Seigneur, mais aussi devant les hommes.
\VS{22}Nous avons envoyé aussi avec eux notre autre frère, dont nous avons souvent éprouvé le zèle dans beaucoup d’occasions, et qui en montre plus encore cette fois à cause de sa grande confiance en vous.
\VS{23}Ainsi donc, pour ce qui est de Tite, il est mon associé et mon compagnon d’œuvre auprès de vous ; et pour ce qui est de nos frères, ils sont les envoyés des églises, la gloire de Christ.
\VS{24}Donnez-leur donc, à la face des églises, une preuve de votre charité, et montrez-leur que nous avons sujet de nous glorifier de vous.
\Chap{9}
\TextTitle{Encouragement par rapport aux dons}
\VerseOne{}Il est superflu que je vous écrive touchant la collecte destinée aux saints.
\VS{2}Je connais, en effet, votre bonne volonté, dont je me glorifie pour vous auprès des Macédoniens, à qui j’ai dit que l’Achaïe est prête dès l'année dernière ; et votre zèle a stimulé le plus grand nombre.
\VS{3}J’ai envoyé ces frères, afin que ce en quoi je me suis glorifié de vous, ne soit pas vain en cette occasion, et que vous soyez prêts, comme j'ai dit.
\VS{4}Je ne voudrais pas, si les Macédoniens m’accompagnent et ne vous trouvent pas prêts, que cette assurance tourne à notre confusion, pour ne pas dire à la vôtre.
\VS{5}C'est pourquoi j'ai estimé nécessaire d’inviter les frères à se rendre auparavant chez vous, et de s’occuper de votre libéralité déjà promise, afin qu'elle soit prête, de manière à être une libéralité, et non un acte d’avarice.
\VS{6}Au reste, je vous avertis que celui qui sème peu moissonnera peu, et celui qui sème abondamment moissonnera abondamment.
\VS{7}Mais que chacun contribue comme il l’a résolu en son cœur, sans tristesse ni contrainte ; car Dieu aime celui qui donne avec joie.
\VS{8}Et Dieu est Tout-Puissant pour vous combler de toutes sortes de grâces, afin que possédant toujours en toutes choses de quoi satisfaire à tous vos besoins, vous ayez encore en abondance pour toute bonne œuvre,
\VS{9}selon ce qui est écrit : Il a fait des largesses, il a donné aux pauvres ; sa justice demeure éternellement\FTNT{Ps. 112:9.}.
\VS{10}Que celui qui fournit de la semence au semeur veuille aussi vous donner du pain à manger, et multiplier votre semence, et augmenter les revenus de votre justice ;
\VS{11}afin que vous soyez pleinement enrichis pour exercer une parfaite libéralité, et qu’ainsi nous ayons sujet de rendre des actions de grâces à Dieu.
\VS{12}Car le secours de cette assistance non seulement pourvoit aux besoins des saints, mais il est encore une source abondante de nombreuses actions de grâces envers Dieu.
\VS{13}Glorifiant Dieu pour l’épreuve qu’ils font de cette assistance, en ce que vous vous soumettez à l’Évangile de Christ ; et de votre prompte et libérale communication envers eux, et envers tous.
\VS{14}ils prient Dieu pour vous, et ils vous aiment\FTNT{Aimer~: Du grec «~epipotheo~»~: désirer, chérir.} très affectueusement à cause de la grâce excellente que Dieu vous a accordée.
\VS{15}Grâces soient rendues à Dieu pour son don merveilleux.
\Chap{10}
\TextTitle{Paul défend son autorité apostolique}
\VerseOne{}Au reste, je vous prie, moi Paul, par la douceur et la bonté de Christ - moi qui parais méprisable lorsque je suis en votre présence, et plein de hardiesse quand je suis éloigné
\VS{2}je vous prie, lorsque je serai présent, de ne pas me forcer à recourir avec assurance à cette hardiesse, dont je me propose d’user contre quelques-uns qui nous regardent comme marchant selon la chair.
\VS{3}Mais en marchant dans la chair, nous ne combattons pas selon la chair.
\VS{4}Car les armes de notre guerre ne sont pas charnelles, mais elles sont puissantes par la vertu de Dieu, pour la destruction des forteresses.
\VS{5}Détruisant les raisonnements et toute hauteur qui s'élèvent contre la connaissance de Dieu, et amenant toute pensée captive à l'obéissance de Christ.
\VS{6}Et étant prêts à tirer vengeance de toute désobéissance, lorsque votre obéissance sera complète.
\VS{7}Considérez-vous les choses selon l'apparence ? Si quelqu'un se persuade qu’il appartient à Christ, qu'il se dise bien en lui-même que, comme il est de Christ, nous aussi nous sommes de Christ.
\VS{8}Et si même je me glorifiais un peu trop de l’autorité que le Seigneur nous a donnée pour votre édification et non votre destruction, je ne saurais en avoir honte,
\VS{9}afin que je ne paraisse pas vouloir vous intimider par mes lettres.
\VS{10}Car ses lettres, disent-ils, sont sévères et fortes, mais présent en personne, il est faible, et sa parole est méprisable.
\VS{11}Que celui qui parle ainsi considère que tels nous sommes en paroles dans nos lettres, étant absents, tels aussi nous sommes dans nos actes, étant présents.
\VS{12}Car nous n'osons pas nous égaler ni nous comparer à quelques-uns de ceux qui se recommandent eux-mêmes. Mais en se mesurant à leur propre mesure et en se comparant à eux-mêmes, ils manquent d’intelligence.
\VS{13}Mais pour nous, nous ne voulons pas nous glorifier outre mesure, mais seulement dans la limite du champ d’action que Dieu nous assigné en nous amenant jusqu’à vous.
\VS{14}Car nous ne nous étendons pas nous même au delà des limites prescrites, comme si nous n'étions pas parvenus jusqu'à vous ; vu que nous sommes parvenus même jusqu’à vous par la prédication de l'Evangile de Christ.
\VS{15}Nous ne nous glorifions pas des travaux d’autrui qui sont hors de nos limites. Mais nous avons l’espérance, si votre foi augmente, de devenir encore plus grands parmi vous, selon les limites qui nous sont assignées,
\VS{16}jusqu'à évangéliser dans les lieux qui sont au-delà de chez vous, sans nous glorifier de ce qui a déjà été fait dans le domaine des autres.
\VS{17}Que celui qui se glorifie se glorifie dans le Seigneur.
\VS{18}Car ce n'est pas celui qui se recommande lui-même qui est approuvé, c'est celui que le Seigneur recommande\FTNT{Le Seigneur recommande ses serviteurs, il témoigne d’eux auprès des autres (Ac. 10:1-48). Un véritable serviteur de Dieu laisse au Seigneur le soin de témoigner de lui auprès des autres alors que les faux ouvriers se recommandent eux-mêmes (2 Co. 3:1).}.
\Chap{11}
\VerseOne{}Oh ! Si vous pouviez supportez de ma part un peu de folie ! Mais vous me supportez !
\VS{2}Car je suis jaloux de vous d'une jalousie de Dieu, parce que je vous ai fiancés à un seul Epoux, pour vous présenter à Christ comme une vierge pure.
\TextTitle{Les faux docteurs}
\VS{3}Mais je crains que comme le serpent séduisit Eve\FTNT{Ge. 3:1-6.} par sa ruse, vos pensées aussi ne se corrompent en se détournant de la simplicité à l’égard de Christ.
\VS{4}Car si quelqu'un vient vous prêcher un autre Jésus que nous n'avons pas prêché, ou si vous recevez un autre esprit que celui que vous avez reçu, ou un autre évangile que celui que vous avez embrassé, vous le supportez fort bien.
\VS{5}Or j'estime que je n'ai été inférieur en rien à ces apôtres par excellence.
\VS{6}Si je suis un ignorant sous le rapport du langage, je ne le suis pourtant point sous celui de la connaissance, et nous l’avons montré parmi vous à tous égards et en toutes choses.
\VS{7}Ai-je commis une faute, en m’abaissant moi-même afin que vous soyez élevés, quand je vous ai annoncé gratuitement l’Evangile de Dieu ?
\VS{8}J'ai dépouillé les autres églises en recevant d’elles un salaire pour vous servir. Et lorsque j’étais chez vous et que je me suis trouvé dans le besoin, je n’ai été à la charge de personne,
\VS{9}car les frères venus de la Macédoine ont pourvu à ce qui me manquait. Et en toutes choses, je me suis gardé d’être à votre charge, et je m'en garderai encore.
\VS{10}Par la vérité de Christ qui est en moi, j’atteste que ce sujet de gloire ne me sera point ravi dans les contrées de l'Achaïe.
\VS{11}Pourquoi ? Est-ce parce que je ne vous aime point ? Dieu le sait !
\VS{12}Mais ce que je fais, je le ferai encore, pour ôter ce prétexte à ceux qui cherchent un prétexte, afin qu’ils soient trouvés tels que nous dans les choses dont ils se glorifient.
\VS{13}Car ces hommes-là sont de faux apôtres, des ouvriers trompeurs qui se déguisent en apôtres de Christ.
\VS{14}Et cela n'est pas étonnant puisque Satan lui-même se déguise en ange de lumière\FTNT{Satan est maître en matière de déguisement et d’imitation.}.
\VS{15}Ce n'est donc pas un grand sujet d'étonnement si ses ministres aussi se déguisent en ministres de justice ; mais leur fin sera conforme à leurs œuvres.
\TextTitle{Sujets de gloire de Paul\FTNTT{2 Co. 11:16-12:18}}
\VS{16}Je le dis encore, afin que personne ne me regarde comme un insensé ; sinon, supportez-moi comme un insensé, afin que je me glorifie aussi un peu.
\VS{17}Ce que je dis, avec l’assurance d’avoir sujet de me glorifier, je ne le dis pas selon le Seigneur, mais comme par folie.
\VS{18}Puisqu’il en est plusieurs qui se glorifient selon la chair, je me glorifierai aussi.
\VS{19}Car vous supportez bien volontiers les insensés, vous qui êtes sages.
\VS{20}Si quelqu'un vous asservit, si quelqu'un vous dévore, si quelqu'un prend votre bien, si quelqu'un est arrogant, si quelqu'un vous frappe au visage, vous le supportez.
\VS{21}Je le dis avec honte, nous avons montré de la faiblesse. Cependant, tout ce que peut oser quelqu’un - je parle en insensé -, moi aussi je l’ose !
\VS{22}Sont-ils Hébreux ? Moi aussi. Sont-ils Israélites ? Moi aussi. Sont-ils de la postérité d'Abraham ? Moi aussi.
\VS{23}Sont-ils ministres de Christ ? - je parle comme un insensé - je le suis plus qu'eux ; par les travaux, bien plus ; par les blessures, bien plus ; par les emprisonnements, bien plus. Plusieurs fois en danger de mort,
\VS{24}cinq fois j’ai reçu des Juifs quarante coups moins un,
\VS{25}j'ai été battu de verges trois fois, j'ai été lapidé une fois, j'ai fait naufrage trois fois, j'ai passé un jour et une nuit dans l’abîme.
\VS{26}Fréquemment en voyage, j’ai été en péril sur les fleuves, en péril de la part des brigands, en péril de la part de ceux de ma nation, en péril de la part des gentils, en péril dans les villes, en péril dans les déserts, en péril sur la mer, en péril parmi de faux frères.
\VS{27}J’ai été dans le travail et dans la peine, exposé à de nombreuses veilles, à la faim et à la soif, à des jeûnes multipliés, au froid et à la nudité.
\VS{28}Et sans parler d’autres choses, je suis assiégé tous les jours par les soucis que me donnent toutes les églises.
\VS{29}Qui est affaibli, que je ne sois faible ? Qui est scandalisé, que je n'en sois aussi brûlé ?
\VS{30}S'il faut se glorifier, c’est de ma faiblesse que je me glorifierai.
\VS{31}Le Dieu et Père de notre Seigneur Jésus-Christ, lui qui est béni éternellement, sait que je ne mens point.
\VS{32}A Damas, le gouverneur du roi Arétas avait mis des gardes dans la ville des Damascéniens pour se saisir de moi,
\VS{33}mais on me descendit par une fenêtre, dans une corbeille, le long de la muraille, et ainsi j'échappai de ses mains.
\Chap{12}
\VerseOne{}Certainement, il ne me convient pas de me glorifier, car j’en viendrai jusqu’aux visions et aux révélations du Seigneur.
\VS{2}Je connais un homme en Christ, qui fut ravi jusqu’au troisième ciel, il y a quatorze ans passés, (si ce fut dans son corps, je ne sais pas ; si ce fut hors du corps, je ne sais pas ; Dieu le sait).
\VS{3}Et je sais que cet homme (si ce fut dans son corps, ou si ce fut hors du corps, je ne sais pas ; Dieu le sait),
\VS{4}fut ravi dans le paradis, et qu’il entendit des paroles merveilleuses qu'il n'est pas permis à l'homme d’exprimer.
\TextTitle{Paul et son écharde dans la chair}
\VS{5}Je me glorifierai d'un tel homme, mais je ne me glorifierai point de moi-même, sinon de mes infirmités.
\VS{6}Si je voulais me glorifier, je ne serais pas un insensé, car je dirais la vérité ; mais je m'en abstiens, afin que personne n’ait à mon sujet une opinion supérieure à ce qu’il voit en moi ou à ce qu’il entend de moi.
\VS{7}Mais pour que je ne sois pas enflé d’orgueil, à cause de l'excellence de ces révélations, il m'a été mis une écharde\FTNT{La nature exacte de l'écharde de Paul ne nous est pas détaillée. Elle lui avait été infligée par un «~ange de Satan~», par la volonté de Dieu. Nous constatons une chose qui est commune à tous les enfants de Dieu~: Paul avait un adversaire constamment aux aguets pour essayer de le décourager, le détruire ou l’intimider en s'opposant par tous les moyens à la mission que le Seigneur lui avait confiée. Cette écharde était aussi un moyen utilisé par Dieu pour garder Paul dans l’humilité.} dans la chair, un ange de Satan pour me souffleter et m’empêcher de m’enorgueillir.
\VS{8}Trois fois j'ai prié le Seigneur de faire que cet ange de Satan se retire de moi.
\VS{9}Mais le Seigneur m'a dit : Ma grâce te suffit, car ma puissance s’accomplit dans la faiblesse. Je me glorifierai donc bien volontiers de mes faiblesses, afin que la puissance de Christ habite en moi.
\VS{10}C’est pourquoi je me plais dans les faiblesses, dans les outrages, dans les calamités, dans les persécutions, et dans les angoisses pour Christ ; car quand je suis faible, c'est alors que je suis fort.
\TextTitle{Avertissements}
\VS{11}J'ai été insensé en me glorifiant, mais vous m'y avez contraint ; c’est par vous que je devais être recommandé, car je n'ai été inférieur en rien aux apôtres par excellence, quoique je ne sois rien.
\VS{12}Certainement les preuves de mon apostolat ont éclaté au milieu de vous par une patience à toute épreuve, par des signes, des prodiges et des miracles.
\VS{13}Car en quoi avez-vous été inférieurs aux autres églises, sinon en ce que je n’ai point été à votre charge ? Pardonnez-moi ce tort.
\VS{14}Voici pour la troisième fois que je suis prêt à aller vers vous, et je ne serai point à votre charge ; car ce ne sont pas vos biens que je cherche, c’est vous-mêmes. Ce n’est pas, en effet, aux enfants d’amasser pour les parents, mais aux parents pour les enfants\FTNT{Un bon père amasse dans le but de préparer l’avenir de ses enfants et non l’inverse.}.
\VS{15}Pour moi, je dépenserai très volontiers pour vous tout ce que j’ai, et je me donnerai encore moi-même pour vos âmes. En vous aimant davantage, serais-je moins aimé de vous ?
\VS{16}Soit ! Dira-t-on, que je ne vous ai point été à charge, c'est qu'étant un homme intelligent, je vous ai pris par ruse !
\VS{17}Ai-je donc tiré profit de vous par quelqu’un de ceux que je vous ai envoyés ?
\VS{18}J'ai engagé Tite à aller chez vous, et avec lui j’ai envoyé le frère. Tite a-t-il tiré profit de vous ? Et n'avons-nous pas lui et moi marché dans le même esprit ? N'avons-nous pas marché sur les mêmes traces ?
\VS{19}Pensez-vous encore que nous voulions nous justifier auprès de vous ? Nous parlons devant Dieu en Christ, et tout cela, mes très chers frères, pour votre édification.
\VS{20}Car je crains de ne pas vous trouver, à mon arrivée, tels que je voudrais, et d’être moi-même trouvé par vous tel que vous ne voudriez pas. Je crains de trouver des querelles, de la jalousie, des animosités, des rivalités, des médisances, des calomnies, de l’orgueil, des troubles.
\VS{21}Je crains qu’à mon arrivée, mon Dieu ne m’humilie de nouveau à votre sujet, et que je n’aie à pleurer sur plusieurs de ceux qui ont péché précédemment, et qui ne se sont pas repentis de l’impureté, de la débauche et des dérèglements dont ils se sont rendus coupables.
\Chap{13}
\TextTitle{S'examiner}
\VerseOne{}Je vais chez vous pour la troisième fois. Toute affaire se réglera sur la déclaration de deux ou de trois témoins\FTNT{De. 19:15.}.
\VS{2}Lorsque j’étais présent pour la deuxième fois, j’ai déjà dit, et aujourd’hui que je suis absent je dis encore d’avance à ceux qui ont péché précédemment et à tous les autres, que si je retourne chez vous, je n'épargnerai personne,
\VS{3}puisque vous cherchez la preuve que Christ parle par moi, lui qui n'est point faible envers vous, mais qui est puissant parmi vous.
\VS{4}Car il a été crucifié à cause de sa faiblesse, mais il vit par la puissance de Dieu ; et nous de même, nous sommes aussi faibles comme lui, mais nous vivrons avec lui par la puissance que Dieu a déployée envers vous.
\VS{5}Examinez-vous vous-mêmes pour savoir si vous êtes dans la foi ; éprouvez-vous vous-mêmes. Ne reconnaissez-vous pas que Jésus-Christ est en vous ? A moins peut-être que vous ne soyez désapprouvés.
\VS{6}Mais j'espère que vous reconnaîtrez que nous, nous ne sommes pas désapprouvés.
\VS{7}Et je prie Dieu que vous ne fassiez rien de mal, non pour paraître nous-mêmes approuvés, mais afin que vous pratiquiez ce qui est bien et que nous, nous soyons comme désapprouvés.
\VS{8}Car nous n’avons pas de pouvoir contre la vérité, nous n’en avons que pour la vérité.
\VS{9}Nous nous réjouissons lorsque nous sommes faibles, tandis que vous êtes forts ; et ce que nous demandons à Dieu, c’est votre perfectionnement.
\VS{10}C'est pourquoi j'écris ces choses étant absent, afin que présent, je n’aie pas à user de rigueur, selon l’autorité que le Seigneur m'a donnée pour l'édification et non point pour la destruction.
\TextTitle{Conclusion}
\VS{11}Au reste, mes frères, réjouissez-vous, perfectionnez-vous, consolez-vous, ayez un même sentiment, vivez en paix ; et le Dieu de charité et de paix sera avec vous.
\VS{12}Saluez-vous les uns les autres par un saint baiser. Tous les saints vous saluent.
\VS{13}Que la grâce du Seigneur Jésus-Christ, la charité de Dieu, et la communion du Saint-Esprit soient avec vous tous. Amen !
\PPE{}
\end{multicols}
