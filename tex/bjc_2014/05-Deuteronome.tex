\ShortTitle{Deutéronome}\BookTitle{Deutéronome}\BFont
\noindent\hrulefill
{\footnotesize
\textit{
\bigskip
{\centering{}
\\Auteur : Probablement Moïse
\\(Heb. : Devarim)
\\Signification : Paroles
\\Thème : Rappel de la loi
\\Date de rédaction : 1450-1410 av. J.-C.\\}
}
%\bigskip
\textit{
\\Ce livre est un rappel de la loi de Yahweh. Après quarante années d'errance dans le désert, Moïse s'adresse à la nouvelle génération par des discours et des exhortations, depuis les plaines de Moab. Au travers de son serviteur, Dieu rappelle ainsi la loi donnée sur le mont Sinaï, les expériences vécues par la génération passée et par conséquent, l'importance de la soumission à Dieu. De leur obéissance dépendraient les bénédictions ou les malédictions contenues dans ce livre.\bigskip
}
}
\par\nobreak\noindent\hrulefill
\begin{multicols}{2}
\Chap{1}
\TextTitle{Rappel de l'infidélité d'Israël\FTNTT{No. 14}}
\VerseOne{}Ce sont ici les paroles que Moïse déclara à tout Israël de l'autre côté du Jourdain, dans le désert, dans la plaine, qui est vis-à-vis de Suph, entre Paran, Tophel, Laban, Hatséroth, et Di-Zahab.
\VS{2}Il y a onze journées depuis Horeb, par le chemin de la montagne de Séir, jusqu'à Kadès-Barnéa.
\VS{3}Or, il arriva dans la quarantième année, au onzième mois, le premier jour du mois, que Moïse parla aux fils d'Israël selon tout ce que Yahweh lui avait ordonné de leur dire,
\VS{4}après qu'il eut battu Sihon, roi des Amoréens, qui habitait à Hesbon, et Og, roi de Basan, qui demeurait à Aschtaroth et à Edréi\FTNT{No. 21:23-24.}.
\VS{5}Moïse donc commença à expliquer cette loi, de l'autre côté du Jourdain, dans le pays de Moab, en disant :
\VS{6}Yahweh, notre Dieu, nous a parlé à Horeb, en disant : Vous avez assez demeuré dans cette montagne.
\VS{7}Tournez-vous, partez, et allez à la montagne des Amoréens et dans tous les lieux voisins, dans la plaine, dans la montagne, dans la vallée, vers le sud, sur le rivage de la mer, au pays des Cananéens et au Liban, jusqu'au grand fleuve, le fleuve d'Euphrate.
\VS{8}Regardez, j'ai mis devant vous le pays ; entrez et prenez possession du pays que Yahweh a juré de donner à vos pères, Abraham, Isaac et Jacob, et à leur postérité après eux.
\VS{9}Et je vous parlai en ce temps-là, et je vous dis : Je ne puis pas, à moi seul, vous porter.
\VS{10}Yahweh, votre Dieu, vous a multipliés, et vous voici aujourd'hui comme les étoiles du ciel par le nombre.
\VS{11}Que Yahweh, le Dieu de vos pères, vous fasse croître mille fois au delà de ce que vous êtes et vous bénisse, comme il vous l'a dit !
\VS{12}Comment porterais-je moi seul vos chagrins, vos charges, et vos procès ?
\VS{13}Prenez dans vos tribus des hommes sages, intelligents et connus, et je les établirai chefs sur vous. 
\VS{14}Et vous me répondîtes et dîtes : Il est bon de faire ce que tu as dit.
\VS{15}Alors je pris les chefs de vos tribus, des hommes sages et connus, et je les établis chefs sur vous, chefs de milliers, et chefs de centaines, et chefs de cinquantaines, et chefs de dizaines, et officiers selon vos tribus. 
\VS{16}Puis j'ordonnai en ce temps-là à vos juges, en disant : Ecoutez les différends qui seront entre vos frères, et jugez droitement entre l'homme et son frère, et entre l'étranger qui est avec lui\FTNT{Lé. 19:15 ; De. 16:19 ; Pr. 24:23.}.
\VS{17}Vous n'aurez point d'égard à l'apparence de la personne en jugement ; vous entendrez autant le petit que le grand ; vous ne craindrez personne, car le jugement est à Dieu ; et vous ferez venir devant moi la cause qui sera trop difficile pour vous, et je l'entendrai. 
\VS{18}Et en ce temps-là, je vous ordonnai toutes les choses que vous deviez faire.
\VS{19}Puis nous partîmes d'Horeb, et nous marchâmes dans tout ce grand et affreux désert que vous avez vu ; par le chemin de la montagne des Amoréens, ainsi que Yahweh, notre Dieu, nous l'avait ordonné, et nous vînmes jusqu'à Kadès-Barnéa.
\VS{20}Alors je vous dis : Vous êtes arrivés jusqu'à la montagne des Amoréens, que Yahweh, notre Dieu, nous donne.
\VS{21}Regarde, Yahweh, ton Dieu, met le pays devant toi ; monte et prends-en possession, comme Yahweh, le Dieu de tes pères, te l'a dit ; ne crains point et ne t'effraie point.
\VS{22}Et vous vous approchâtes tous de moi, et dîtes : Envoyons devant nous des hommes pour explorer le pays, et qui nous rapportent des nouvelles du chemin par lequel nous devrons monter, et des villes où nous devrons aller\FTNT{No. 13:2.}.
\VS{23}Et ce discours sembla bon à mes yeux ; et je pris douze hommes parmi vous, un homme par tribu.
\VS{24}Et ils se mirent en chemin et montèrent dans la montagne, et vinrent jusqu'au torrent d'Eschcol et explorèrent le pays.
\VS{25}Et ils prirent dans leurs mains des fruits du pays, et ils nous les apportèrent ; ils nous donnèrent des nouvelles, et nous dirent : Le pays que Yahweh, notre Dieu nous donne est bon. 
\VS{26}Mais vous ne refusâtes d'y monter, et vous fûtes rebelles à l'ordre de Yahweh, votre Dieu.
\VS{27}Et vous murmurâtes dans vos tentes, en disant : C'est parce que Yahweh nous hait qu'il nous a fait sortir du pays d'Egypte, afin de nous livrer entre les mains des Amoréens pour nous exterminer.
\VS{28}Où monterions-nous ? Nos frères nous ont fait fondre le cœur, en disant : Le peuple est plus grand que nous, et de plus haute taille ; les villes sont grandes et closes jusqu'au ciel ; et même nous avons vu là les fils des d'Anakim.
\VS{29}Mais je vous dis : Ne tremblez point et ne les craignez point.
\VS{30}Yahweh, votre Dieu, qui marche devant vous, lui-même combattra pour vous, selon tout ce que vous avez vu qu'il a fait pour vous en Egypte;
\VS{31}et au désert, où tu as vu de quelle manière  Yahweh, ton Dieu, t'a porté comme un homme porterait son fils, sur tout le chemin où vous avez marché, jusqu'à ce que vous soyez arrivés dans ce lieu-ci.
\VS{32}Mais malgré cela, vous ne crûtes point encore en Yahweh, votre Dieu,
\VS{33}qui marchait devant vous sur le chemin afin de vous chercher un lieu pour camper, marchant de nuit dans la colonne de feu pour vous éclairer dans le chemin par lequel vous deviez marcher et de jour dans la nuée.
\VS{34}Et Yahweh entendit la voix de vos paroles et se mit en grande colère et jura, disant :
\VS{35}Aucun des hommes de cette méchante génération ne verra ce bon pays que j'ai juré de donner à vos pères,
\VS{36}à l'exception de Caleb, fils de Jephunné ; lui le verra, et je donnerai à lui et à ses fils le pays sur lequel il a marché, parce qu'il a persévéré à suivre Yahweh\FTNT{No. 14:22-24.}.
\VS{37}Même Yahweh s'est mis en colère contre moi à cause de vous, disant : Et toi aussi tu n'y entreras pas.
\VS{38}Josué, fils de Nun qui te sert y entrera ; fortifie-le, car c'est lui qui mettra les enfants d'Israël en possession de ce pays\FTNT{De. 34:4.}.
\VS{39}Et vos petits-enfants, dont vous avez dit qu'ils seront en proie, vos enfants, dis-je, qui aujourd'hui ne savent pas ce que c'est le bien ou le mal, eux y entreront et je leur donnerai ce pays et ils le posséderont.
\VS{40}Mais vous, retournez vous-en en arrière, et allez dans le désert par le chemin de la mer Rouge.
\VS{41}Et vous répondîtes et me dîtes : Nous avons péché contre Yahweh, nous monterons et nous combattrons, comme Yahweh, notre Dieu, nous l'a ordonné. Et vous ceignîtes chacun vos armes de guerre, et légèrement vous entreprîtes de monter à la montagne.
\VS{42}Et Yahweh me dit : Dis-leur : Ne montez point et ne combattez point, car je ne suis point au milieu de vous ; afin que vous ne soyez point battus par vos ennemis.
\VS{43}Je vous parlai, mais vous ne m'écoutâtes point et vous vous rebelâtes contre l'ordre de Yahweh, et vous fûtes orgueilleux et vous montâtes sur la montagne.
\VS{44}Et les Amoréens, qui demeurent sur cette montagne, sortirent contre vous et vous poursuivirent comme font les abeilles ; et ils vous battirent depuis Séir jusqu'à Horma.
\VS{45}Et étant retournés vous pleurâtes devant Yahweh ; mais Yahweh n'écouta point votre voix, et ne vous prêta point l'oreille.
\VS{46}Ainsi Vous demeurâtes à Kadès plusieurs jours, autant de temps que vous y aviez demeuré.
\Chap{2}
\TextTitle{Périple du peuple dans le désert}
\VerseOne{}Alors nous retournâmes en arrière, et nous partîmes pour le désert, par le chemin de la Mer Rouge, comme Yahweh me l'avait dit, et nous tournâmes autour de la montagne de Séir plusieurs jours.
\VS{2}Et Yahweh me parla, en disant :
\VS{3}Vous avez assez tourné autour de cette montagne. Tournez-vous vers le nord.
\VS{4}Ordonne au peuple, en disant : Vous allez passer la frontière de vos frères, les fils d'Esaü, qui demeurent en Séir. Ils auront peur de vous ; mais soyez bien sur vos gardes.
\VS{5}N'ayez pas de démêlé avec eux ; car je ne vous donnerai rien dans leur pays, pas même de quoi poser la plante du pied : J'ai donné à Esaü la montagne de Séir en héritage.
\VS{6}Vous achèterez d'eux la nourriture à prix d'argent et vous en mangerez, et vous achèterez d'eux l'eau à prix d'argent et vous en boirez.
\VS{7}Car Yahweh, ton Dieu, t'a béni dans tout le travail de tes mains, il a connu ta marche dans ce grand désert. Yahweh, ton Dieu, a été avec toi pendant ces quarante années, et tu n'as manqué de rien.
\VS{8}Nous passâmes à distance de nos frères, les fils d'Esaü, qui demeurent en Séir, à distance du chemin de la plaine, d'Elath et d'Etsjon-Guéber, et nous nous tournâmes, et nous passâmes par le chemin du désert de Moab.
\VS{9}Yahweh me dit : N'assiège point Moab, et ne t'engage pas dans un combat avec lui ; car je ne te donnerai rien en héritage dans son pays : J'ai donné Ar en héritage aux fils de Lot\FTNT{Ge. 19:36-38.}.
\VS{10}Les Emim y habitaient auparavant ; c'était un peuple grand, nombreux et de haute taille comme les Anakim.
\VS{11}Ils étaient considérés comme des Rephaïm, de même que les Anakim ; mais les Moabites les appelaient Emim.
\VS{12}Séir était habité autrefois par les Horiens ; mais les fils d'Esaü les en dépossédèrent, les détruisirent devant eux, et y habitèrent à leur place, comme l'a fait Israël dans le pays de son héritage que Yahweh lui a donné.
\VS{13}Mais maintenant, levez-vous, et passez le torrent de Zéred. Et nous passâmes le torrent de Zéred.
\VS{14}Or le temps que nous avons marché de Kadès-Barnéa, jusqu'à ce que nous ayons passé le torrent de Zéred, fut de trente-huit ans, jusqu'à ce que toute la génération des hommes de guerre eût été consumée du milieu du camp, comme Yahweh le leur avait juré.
\VS{15}La main de Yahweh fut aussi sur eux pour les détruire du milieu du camp, jusqu'à ce qu'ils eussent été consumés.
\VS{16}Or il est arrivé qu'après que tous les hommes de guerre eurent été consumés par la mort du milieu du peuple,
\VS{17}Yahweh me parla et dit :
\VS{18}Tu vas passer aujourd'hui la frontière de Moab, à savoir Har.
\VS{19}Tu t'approcheras en face des fils d'Ammon. Ne les assiège point, et ne t'engage point dans un combat avec eux ; car je ne te donnerai rien en possession dans le pays des fils d'Ammon : Je l'ai donné en héritage aux fils de Lot.
\VS{20} Ce pays, aussi considéré comme un pays de Rephaïm ; car les Rephaïm y habitaient auparavant, et les Ammonites les appelaient Zamzummim ;
\VS{21}c'était un peuple grand, nombreux, et de haute taille, comme les Anakim, Yahweh les détruisit devant eux, et ils les dépossédèrent, et habitèrent à leur place.
\VS{22}Comme il fit pour les fils d'Esaü qui demeurent en Séir, quand il détruisit les Horiens devant eux ; ils les dépossédèrent et habitèrent à leur place jusqu'à ce jour.
\VS{23}Or, quant aux Avviens, qui demeuraient en Hatserim jusqu'à Gaza, furent détruits par les Caphtorim, sortis de Caphtor, qui demeurèrent à leur place.
\VS{24}Levez-vous, partez, et passez le torrent d'Arnon. Regarde, je livre entre tes mains Sihon, roi de Hesbon, l'Amoréen, et son pays. Commence à en prendre possession, et fais-lui la guerre !
\VS{25}Aujourd'hui, je vais commencer à mettre la frayeur et la crainte de toi sur les peuples qui sont sous les cieux ; et ayant entendu parler de toi, ils trembleront et seront dans l'angoisse à cause de ta présence.
\VS{26}J'envoyai, du désert de Kedémoth, des messagers à Sihon, roi de Hesbon, avec des paroles de paix, disant\FTNT{No. 21:21.} :
\VS{27}Permets que je passe par ton pays ; et j'irai par le grand chemin, sans me détourner ni à droite ni à gauche.
\VS{28}Tu me vendras de la nourriture à prix d'argent, afin que je mange, et tu me donneras de l'eau à prix d'argent, afin que je boive ; seulement que j'y passe de mes pieds.
\VS{29}C'est ce qu'ont fait les fils d'Esaü qui demeurent en Séir, et les Moabites qui demeurent à Ar, jusqu'à ce que je passe le Jourdain pour entrer au pays que Yahweh, notre Dieu, nous donne.
\VS{30}Mais Sihon, roi de Hesbon, ne voulut point nous laisser passer par son pays ; car Yahweh, ton Dieu, avait endurci son esprit, et raidit son cœur, afin de le livrer entre tes mains, comme tu le vois aujourd'hui.
\VS{31}Yahweh me dit : Regarde, j'ai commencé à te livrer Sihon et son pays ; commence à posséder son pays, pour le tenir en héritage.
\VS{32}Sihon donc, sortit nous rencontrer avec tout son peuple pour nous combattre en Jahats.
\VS{33}Mais Yahweh, notre Dieu, nous le livra en face, et nous le battîmes, lui, ses fils, et tout son peuple.
\VS{34}Et en ce temps-là, nous prîmes toutes ses villes, et nous détruisîmes à la façon de l'interdit les villes, les hommes, les femmes, et les petits enfants, sans laisser de survivants.
\VS{35}Seulement nous pillâmes les bêtes pour nous, et le butin des villes que nous avions prises.
\VS{36}Depuis Aroër, qui est sur le bord du torrent de l'Arnon, et la ville qui est dans la vallée, jusqu'à Galaad, il n'y eut pas une ville qui fût trop haute pour nous : Yahweh, notre Dieu, nous livra tout.
\VS{37}Seulement tu n'approchas point du pays des fils d'Ammon, de tous les bords du torrent de Jabbok, des villes de la montagne, ni d'aucun lieu que Yahweh, notre Dieu, t'avait ordonné de ne point attaquer.
\Chap{3}
\TextTitle{Yawheh livre Og, roi de Basan, entre les mains d'Israël}
\VerseOne{}Alors nous nous tournâmes, et nous montâmes par le chemin de Basan. Et Og, roi de Basan, sortit nous rencontrer, avec tout son peuple, pour nous combattre à Edréi.
\VS{2}Et Yahweh me dit : Ne le crains point ; car je le livre entre tes mains, lui, tout son peuple, et son pays ; et tu lui feras comme tu as fait à Sihon, roi des Amoréens, qui demeurait à Hesbon.
\VS{3}Ainsi Yahweh, notre Dieu, livra aussi entre nos mains Og, roi de Basan, avec tout son peuple ; nous le battîmes sans laisser de survivants.
\VS{4}En ce même temps, nous prîmes aussi toutes ses villes, et il n'y eut point de ville que nous ne prîmes : Soixante villes, toute la contrée d'Argob, le royaume d'Og en Basan.
\VS{5}Toutes ces villes-là étaient fortifiées, avec de hautes murailles, des portes et des barres ; il y avait aussi des villes sans murailles en fort grand nombre.
\VS{6}Et nous les détruisîmes à la façon de l'interdit, comme nous l'avions fait à Sihon, roi de Hesbon ; nous dévouâmes à la façon de l'interdit toutes les villes, les hommes, les femmes, et les petits enfants.
\VS{7}Mais nous pillâmes pour nous toutes les bêtes et le butin des villes.
\VS{8}Nous prîmes donc, en ce temps-là, le pays de la main des deux rois des Amoréens, qui étaient de l'autre côté du Jourdain, depuis le torrent d'Arnon jusqu'à la montagne de l'Hermon ;
\VS{9}or les Sidoniens donnent à l'Hermon le nom de Sirion, mais les Amoréens le nomment Senir ;
\VS{10}toutes les villes de la plaine, tout Galaad, et tout Basan jusqu'à Salca et Edréi, les villes du royaume d'Og en Basan.
\VS{11}Og, roi de Basan, avait survécu seul du reste des Rephaïm. Voici, son lit, un lit de fer, n'est-il pas dans Rabbath, ville des fils d'Ammon ? Sa longueur est de neuf coudées, et sa largeur de quatre coudées, en coudées d'homme.
\TextTitle{Premières terres attribuées à Ruben, Gad et la demi-tribu de Manassé}
\VS{12}En ce temps-là donc, nous prîmes possession de ce pays. Je donnai aux Rubénites et aux Gadites le territoire à partir d'Aroër, sur le torrent d'Arnon, et la moitié de la montagne de Galaad, avec ses villes\FTNT{Jos 13:23-32.}.
\VS{13}Je donnai à la demi-tribu de Manassé le reste de Galaad et tout le royaume d'Og, en Basan : Toute la contrée d'Argob avec tout le Basan, c'est ce qu'on appelait le pays des Réphaïm.
\VS{14}Jaïr, fils de Manassé, prit toute la contrée d'Argob jusqu'à la frontière des Gueschuriens et des Maacathiens, et il donna son nom à Basan, appelé bourgs de Jaïr jusqu'à aujourd'hui.
\VS{15}Je donnai aussi Galaad à Makir.
\VS{16}Mais aux Rubénites et aux Gadites, je donnai de Galaad jusqu'au torrent de l'Arnon, dont le milieu du torrent sert de frontière, et jusqu'au torrent de Jabbok, frontière des fils d'Ammon ;
\VS{17}la plaine, et le Jourdain, de la frontière de Kinnéreth jusqu'à la mer de la plaine, la Mer Salée, aux pieds de Pisga vers l'orient.
\VS{18}Or, en ce temps-là, je vous ordonnai, en disant : Yahweh, votre Dieu, vous donne ce pays pour le posséder. Vous tous, qui êtes vaillants, vous passerez armés devant vos frères, les fils d'Israël.
\VS{19}Seulement vos femmes, vos petits-enfants, et vos troupeaux, car je sais que vous avez beaucoup de troupeaux, resteront dans les villes que je vous ai données,
\VS{20}jusqu'à ce que Yahweh ait accordé du repos à vos frères comme à vous, et qu'eux aussi possèdent le pays que Yahweh, votre Dieu, leur donne de l'autre côté du Jourdain. Puis vous retournerez chacun dans l'héritage que je vous ai donné.
\VS{21}En ce temps-là, j'ordonnai à Josué, en disant : Tes yeux ont vu tout ce que Yahweh, votre Dieu, a fait à ces deux rois : Yahweh en fera de même à tous les royaumes vers lesquels tu vas passer.
\VS{22}Ne les craignez point ; car Yahweh, votre Dieu, combattra lui-même pour vous.
\TextTitle{Moïse n'entrera pas dans la terre promise}
\VS{23}En ce même temps, j'implorai la grâce de Yahweh, en disant :
\VS{24}Seigneur Yahweh, tu as commencé à montrer à ton serviteur ta grandeur et ta main puissante ; car quel est le dieu dans le ciel et sur la terre qui puisse faire selon tes œuvres et selon ta puissance ?
\VS{25}Que je passe, je te prie, et que je voie ce bon pays de l'autre côté du Jourdain, ces bonnes montagnes et le Liban.
\VS{26}Mais Yahweh s'irrita contre moi, à cause de vous, et ne m'écouta point. Yahweh me dit : C'est assez, ne me parle plus de cette affaire.
\VS{27}Monte au sommet du Pisga, et lève tes yeux à l'occident, au nord, au sud, et à l'orient, et regarde de tes yeux ; car tu ne passeras point ce Jourdain.
\VS{28}Donnes-en la charge à Josué, et fortifie-le et affermis-le ; car c'est lui qui passera devant ce peuple et qui le mettra en possession du pays que tu verras.
\VS{29}Ainsi nous demeurâmes dans la vallée, vis-à-vis de Beth-Peor.
\Chap{4}
\TextTitle{Encouragement à garder la loi de Yahweh}
\VerseOne{}Et maintenant Israël, écoute les lois et les ordonnances que je vous enseigne, pour les pratiquer afin que vous viviez, que vous entriez et possédiez le pays que Yahweh, le Dieu de vos pères, vous donne.
\VS{2}Vous n'ajouterez\FTNT{De. 12:32 ; Pr. 30:6 ; Ap. 22:18-19.} rien à la parole que je vous ordonne, et vous n'en retrancherez rien ; afin de garder les commandements de Yahweh, votre Dieu, que je vous ordonne.
\VS{3}Vos yeux ont vu ce que Yahweh a fait à cause de Baal-Peor : Yahweh, ton Dieu, a détruit du milieu de toi tout homme qui était allé après Baal-Peor\FTNT{No 25:4-9.}.
\VS{4}Mais vous, qui vous êtes attachés à Yahweh, votre Dieu, vous êtes tous vivants aujourd'hui.
\VS{5}Regardez, je vous ai enseigné des lois et des ordonnances, comme Yahweh, mon Dieu, me l'a ordonné, afin que vous les pratiquiez au milieu du pays où vous allez pour le posséder.
\VS{6}Vous les garderez et vous les pratiquerez, car c'est là votre sagesse et votre intelligence aux yeux de tous les peuples, qui entendant ces lois, et qui diront : Cette grande nation est un peuple sage et intelligent !
\TextTitle{Israël, privilégié parmi tous les peuples}
\VS{7}car quelle est la grande nation qui ait ses dieux proches d'elle, comme nous avons Yahweh, notre Dieu, toutes les fois que nous l'invoquons ?
\VS{8}Et quelle est la grande nation qui ait des lois et des ordonnances justes, comme toute cette loi que je mets aujourd'hui devant vous ?
\VS{9}Seulement, prends garde à toi et garde soigneusement ton âme, afin que tous les jours de ta vie tu n'oublies point les choses que tes yeux ont vues, et qu'elles ne sortent de ton cœur\FTNT{Pr. 4:23.} ; enseigne-les à tes fils, et aux fils de tes fils.
\VS{10}Rappelle-toi du jour où tu te tins face à Yahweh, ton Dieu, à Horeb, après que Yahweh me dit : Convoque le peuple ! Je veux leur faire entendre mes paroles, pour qu'ils apprennent à me craindre tout le temps qu'ils seront vivants sur la terre ; et pour qu'ils les enseignent à leurs fils.
\VS{11}Et que vous vous approchâtes, et vous vous tîntes au pied de la montagne. Or la montagne était embrasée de feu jusqu'au milieu du ciel. Il y avait des ténèbres, des nuées, et de l'obscurité.
\VS{12}Et que Yahweh vous parla du milieu du feu ; vous entendîtes le son de ses paroles, mais vous ne vîtes aucune ressemblance, vous entendîtes seulement la voix\FTNT{Ex. 19:17-19.}.
\VS{13}Et il déclara son alliance, qu'il vous ordonna d'observer, les dix paroles, qu'il écrivit sur deux tables de pierre.
\VS{14}Yahweh m'ordonnaaussi, en ce temps-là, de vous enseigner les lois et les ordonnances, afin que vous les pratiquiez dans le pays que vous allez posséder.
\VS{15}Prenez bien garde à vos âmes, puisque vous n'avez vu aucune ressemblance le jour où Yahweh, votre Dieu, vous parla du milieu du feu à Horeb,
\VS{16}de peur que vous ne vous corrompiez et que vous ne vous fassiez une image taillée, une représentation d'idole ayant la forme d'un mâle ou d'une femelle,
\VS{17}ou la forme d'un animal qui soit sur la terre, ou la forme d'un oiseau ailé qui vole dans les cieux,
\VS{18}ou la forme d'un animal qui rampe sur la terre, ou la forme d'un poisson qui soit dans les eaux au-dessous de la terre.
\VS{19}De peur aussi qu'élevant tes yeux vers les cieux, et voyant le soleil, la lune, et les étoiles, toute l'armée des cieux, tu ne sois poussé à te prosterner devant elles, et que tu ne les serves : C'est ce que Yahweh, ton Dieu, a donné en partage à tous les peuples, sous tous les cieux.
\VS{20}Mais vous, Yahweh vous a pris, et vous a fait sortir d'Egypte, du fourneau de fer, afin que vous fussiez un peuple de son héritage, comme vous l'êtes aujourd'hui.
\TextTitle{Conséquences de la désobéissance et de l'idolâtrie}
\VS{21}Or Yahweh s'irrita contre moi, à cause de vos paroles, et il jura que je ne passerais point le Jourdain, et que je n'entrerais point dans ce bon pays que Yahweh, ton Dieu, te donne en héritage.
\VS{22}Je mourrai dans ce pays-ci, je ne passerai point le Jourdain ; mais vous le passerez, et vous posséderez ce bon pays.
\VS{23}Gardez-vous d'oublier l'alliance de Yahweh, votre Dieu, qu'il a traitée avec vous, et que vous ne vous fassiez d'image taillée, de représentation quelconque, que Yahweh, votre Dieu, vous a défendues.
\VS{24}Car Yahweh, ton Dieu, est un feu dévorant\FTNT{Hé. 12:29.}, un Dieu jaloux.
\VS{25}Quand tu auras engendré des fils, et des fils de tes fils, et que vous serez depuis longtemps dans le pays, si vous vous corrompez, et que vous faites des images taillées, ou des représentations de quelque chose que ce soit, si vous faites ce qui est mal aux yeux de Yahweh, votre Dieu, afin de l'irriter,
\VS{26}j'appelle aujourd'hui à témoin les cieux et la terre contre vous, certainement vous périrez promptement dans ce pays que vous allez posséder au-delà du Jourdain, vous n'y prolongerez point vos jours, car vous serez entièrement détruits.
\VS{27}Yahweh vous dispersera parmi les peuples, et vous ne resterez qu'un petit nombre parmi les nations, chez lesquelles Yahweh vous emmènera.
\VS{28}Et là, vous servirez des dieux, œuvres de main d'homme, du bois et de la pierre, qui ne peuvent voir, ni entendre, ni manger, ni sentir\FTNT{Es. 44:9 ; Es. 46:7 ; Ps. 115:4-7}.
\TextTitle{Yahweh, puissant, miséricordieux et fidèle à son alliance}
\VS{29}Mais de là, tu chercheras Yahweh, ton Dieu, et tu le trouveras, si tu le cherches de tout ton cœur et de toute ton âme.
\VS{30}Quand tu seras dans la détresse, et que toutes ces choses te seront arrivées, alors, dans les derniers jours, tu retourneras à Yahweh, ton Dieu, et tu écouteras sa voix ;
\VS{31}parce que Yahweh, ton Dieu, est le Dieu puissant et miséricordieux, il ne t'abandonnera point et ne te détruira point, il n'oubliera point l'alliance de tes pères, qu'il leur a jurée.
\VS{32}Interroge les premiers temps, qui ont été avant toi, depuis le jour où Dieu créa l'homme sur la terre, et d'une extrémité des cieux à l'autre, y eut-il jamais si grand événement, et n'a-t-on jamais entendu de chose semblable ?
\VS{33}Est-ce qu'un peuple a entendu la voix de Dieu parlant du milieu du feu, comme tu l'as entendue, et qui soit demeuré en vie ?
\VS{34}Où Dieu a-t-il essayé de venir prendre pour lui une nation du milieu d'une nation, par des épreuves, des signes, des miracles, et des batailles, à main forte, et à bras étendu, et par des choses grandes et terribles, comme tout ce que Yahweh, notre Dieu, a fait pour vous en Egypte, sous vos yeux ?
\VS{35}Cela t'a été montré afin que tu reconnaisses que Yahweh est Dieu et qu'il n'y en a point d'autre.
\VS{36}Il t'a fait entendre sa voix des cieux pour t'instruire ; et il t'a montré son grand feu sur la terre, et tu as entendu ses paroles du milieu du feu.
\VS{37}Et parce qu'il a aimé tes pères, il a choisi leur postérité après eux et il t'a retiré d'Egypte en sa présence, par sa grande puissance ;
\VS{38}pour chasser de devant toi des nations plus grandes et plus puissantes que toi, pour te faire entrer dans leur pays, et pour te le donner en héritage, comme tu le vois aujourd'hui.
\VS{39}Sache donc aujourd'hui, et rappelle dans ton cœur que Yahweh est Dieu, en haut dans les cieux et sur la terre, et qu'il n'y en a point d'autre.
\VS{40}Garde donc ses lois et ses commandements que je t'ordonne aujourd'hui, afin que tu sois heureux, toi et tes fils après toi, et que tu prolonges tes jours sur la terre que Yahweh, ton Dieu, te donne\FTNT{Ex 20.}.
\TextTitle{Trois villes de refuge à l'est du Jourdain}
\VS{41}Alors Moïse sépara trois villes de l'autre côté du Jourdain vers le soleil levant,
\VS{42}afin que le meurtrier qui aurait tué son prochain involontairement, sans l'avoir haï auparavant, s'y enfuie ; et que s'enfuyant dans l'une de ces villes-là, il eût sa vie sauve.
\VS{43}C'étaient : Betser dans le désert, dans la plaine du pays, chez les Rubénites ; Ramoth en Galaad, chez les Gadites, Golan en Basan, chez les Manassites.
\TextTitle{Convocation pour rappel de la loi}
\VS{44}C'est ici la loi que Moïse plaça face aux enfants d'Israël.
\VS{45}Voici les témoignages, les lois, et les ordonnances que Moïse déclara aux enfants d'Israël, après qu'ils furent sortis d'Egypte.
\VS{46}C'était de l'autre côté du Jourdain, dans la vallée, vis-à-vis de Beth-Peor, au pays de Sihon, roi des Amoréens, qui demeurait à Hesbon, et qui fut battu par Moïse et les fils d'Israël après être sortis d'Egypte.
\VS{47}Et ils s'emparèrent de son pays avec le pays d'Og, roi de Basan, deux rois des Amoréens qui étaient de l'autre côté du Jourdain, vers le soleil levant.
\VS{48}Depuis Aroër, sur le bord du torrent de l'Arnon, jusqu'à la montagne de Sion, qui est l'Hermon,
\VS{49}et toute la plaine de l'autre côté du Jourdain vers l'orient, jusqu'à la mer de la plaine, au pied du Pisga.
\Chap{5}
\TextTitle{L'alliance établie à Horeb rappelée à la nouvelle génération}
\VerseOne{}Moïse appela tout Israël, et leur dit : Ecoute, Israël, les lois et les ordonnances que je prononce aujourd'hui à vos oreilles. Apprenez-les, prenez garde de les pratiquer.
\VS{2}Yahweh, notre Dieu, a traité avec nous une alliance en Horeb\FTNT{Ex. 19:5.}.
\VS{3}Dieu n'a point traité cette alliance avec nos pères ; mais avec nous, qui sommes ici aujourd'hui, tous vivants.
\VS{4}Yahweh vous parla face à face sur la montagne, du milieu du feu.
\VS{5}Je me tenais en ce temps-là entre Yahweh et vous, pour vous rapporter la parole de Yahweh ; parce que vous aviez peur face à ce feu, et vous ne montâtes point sur la montagne. Il dit\FTNT{Les dix paroles (Ex. 20).} :
\VS{6}Je suis Yahweh, ton Dieu, qui t'a fait sortir du pays d'Egypte, de la maison de servitude.
\VS{7}Tu n'auras point d'autres dieux devant ma face.
\VS{8}Tu ne te feras point d'image taillée, ni de représentation des choses qui sont en haut dans les cieux, ni sur la terre, ni dans les eaux sous la terre.
\VS{9}Tu ne te prosterneras point devant elles, et tu ne les serviras point ; car je suis Yahweh, ton Dieu, un Dieu jaloux, qui punis l'iniquité des pères sur les enfants jusqu'à la troisième et à la quatrième génération de ceux qui me haïssent,
\VS{10}et qui fais miséricorde jusqu'à mille générations à ceux qui m'aiment et qui gardent mes commandements.
\VS{11}Tu ne prendras point le Nom de Yahweh, ton Dieu, en vain ; car Yahweh ne tiendra pas pour innocent celui qui prendra son Nom en vain.
\VS{12}Garde le jour du sabbat pour le sanctifier, comme Yahweh, ton Dieu, te l'a ordonné.
\VS{13}Tu travailleras six jours, et tu feras toute ton œuvre.
\VS{14}Mais le septième jour est le sabbat de Yahweh, ton Dieu : Tu ne feras aucune œuvre, ni ton fils, ni ta fille, ni ton serviteur, ni ta servante, ni ton bœuf, ni ton âne, ni aucune de tes bêtes, ni l'étranger qui est dans tes portes, afin que ton serviteur et ta servante se reposent comme toi.
\VS{15}Et tu te souviendras que tu as été esclave au pays d'Egypte, et que Yahweh, ton Dieu, t'en a fait sortir à main forte et à bras étendu : C'est pourquoi Yahweh, ton Dieu, t'a ordonné d'observer le jour du sabbat.
\VS{16}Honore ton père et ta mère, comme Yahweh, ton Dieu, te l'a ordonné, afin que tes jours se prolongent et que tu sois heureux sur la terre que Yahweh, ton Dieu, te donne.
\VS{17}Tu ne tueras point.
\VS{18}Tu ne commettras point d'adultère.
\VS{19}Tu ne déroberas point.
\VS{20}Tu ne diras point de faux témoignage contre ton prochain.
\VS{21}Tu ne convoiteras point la femme de ton prochain ; tu ne désireras point la maison de ton prochain, ni son champ, ni son serviteur, ni sa servante, ni son bœuf, ni son âne, ni aucune chose qui soit à ton prochain.
\VS{22}Yahweh déclara ces paroles à toute votre assemblée sur la montagne, du milieu du feu, des nuées et de l'obscurité, à voix forte sans rien ajouter. Il les écrivit sur deux tables de pierre qu'il me donna.
\TextTitle{Moïse, intermédiaire entre Yahweh et le peuple}
\VS{23}Or il arriva qu'aussitôt que vous eûtes entendu la voix du milieu de l'obscurité, parce que la montagne était embrasée par le feu, vos chefs de tribus et vos anciens s'approchèrent de moi,
\VS{24}et vous dîtes : Voici, Yahweh, notre Dieu, nous a fait voir sa gloire et sa grandeur, et nous avons entendu sa voix du milieu du feu ; aujourd'hui, nous avons vu que Dieu a parlé avec l'homme, et qu'il est resté en vie.
\VS{25}Et maintenant pourquoi mourrions-nous ? Car ce grand feu là nous dévorera ; si nous entendons encore la voix de Yahweh, notre Dieu, nous mourrons.
\VS{26}Car qui, de toute chair, a entendu comme nous la voix du Dieu vivant parlant du milieu du feu, et qui soit resté en vie ?
\VS{27}Approche-toi et écoute tout ce que Yahweh, notre Dieu, dira ; puis tu nous diras tout ce que Yahweh, notre Dieu, t'aura dit ; nous l'entendrons, et nous le ferons.
\VS{28}Yahweh entendit la voix de vos paroles pendant que vous me parliez. Et Yahweh me dit : J'ai entendu les paroles que ce peuple t'ont adressées : Tout ce qu'ils ont dit est bien.
\VS{29}Ô ! S'ils avaient toujours ce même cœur pour me craindre et pour garder tous mes commandements, afin qu'ils fussent heureux, eux et leurs enfants, pour toujours !
\VS{30}Va, dis-leur : Retournez dans vos tentes.
\VS{31}Mais toi, reste ici avec moi, et je te dirai tous les commandements, les lois, et les ordonnances que tu leur enseigneras, afin qu'ils les pratiquent dans le pays que je leur donne en possession.
\VS{32}Vous prendrez donc garde de faire ce que Yahweh, votre Dieu, vous a ordonné ; vous ne vous en détournerez ni à droite ni à gauche.
\VS{33}Vous marcherez dans toute la voie que Yahweh, votre Dieu, vous a ordonnée, afin que vous viviez et que vous soyez heureux, et que vous prolongiez vos jours sur la terre que vous posséderez.
\Chap{6}
\TextTitle{Obéissance à la loi, source de bénédictions}
\VerseOne{}Voici les commandements, les lois et les ordonnances que Yahweh, votre Dieu, m'a ordonné de vous enseigner, afin que vous les pratiquiez dans le pays dans lequel vous allez passer pour le posséder ;
\VS{2}afin que tu craignes Yahweh, ton Dieu, en gardant durant tous les jours de ta vie, toi, ton fils, et le fils de ton fils, toutes ses lois et ses commandements que je t'ordonne, pour que tes jours soient prolongés.
\VS{3}Tu les écouteras donc, ô Israël, et tu auras soin de les mettre en pratique, afin que tu sois heureux, et que vous vous multipliiez beaucoup sur la terre où coulent le lait et le miel, comme Yahweh, le Dieu de tes pères, l'a dit\FTNT{Ex. 3:8.}.
\VS{4}Ecoute Israël ! Yahweh, notre Dieu Yahweh est Un\FTNT{Jacob fut le premier à faire cette prière qui affirme l'unicité de Dieu. Dieu est UN (en hébreu « Echad »). Loin de l'infirmer ou de la contredire, Jésus a confirmé cette prière et l'enseignement capital qu'elle contient (Mc. 12:29). Dieu n'est pas trois personnes en une, mais UN. Cette parole annonce un monothéisme absolu. Elle s'oppose catégoriquement au polythéisme des Cananéens qui adoraient de multiples dieux, les étoiles, la lune, le soleil, les arbres, les rois etc. Aussi les Hébreux avaient reçu l'ordre de la part de Yahweh de détruire toutes les idoles qu'ils trouveraient sur la terre promise (De. 16:21). Voir également commentaire en Ge. 1:5.}.
\VS{5}Tu aimeras donc Yahweh, ton Dieu, de tout ton cœur, de toute ton âme, et de toute ta force\FTNT{Mt 22:37 ; Mc 12:30.}.
\TextTitle{La loi de Yahweh transmise de père en fils}
\VS{6}Et ces paroles, que je t'ordonne aujourd'hui, seront dans ton cœur.
\VS{7}Tu les enseigneras soigneusement à tes fils, et tu en parleras quand tu te tiendras dans ta maison, quand tu iras en voyage, quand tu te coucheras et quand tu te lèveras.
\VS{8}Et tu les lieras comme un signe sur tes mains, et elles seront comme des fronteaux entre tes yeux.
\VS{9}Tu les écriras aussi sur les poteaux de ta maison et sur tes portes.
\VS{10}Yahweh, ton Dieu, te fera entrer dans le pays qu'il a juré à tes pères, Abraham, Isaac, et Jacob, de te donner. Tu posséderas de grandes et bonnes villes que tu n'as point bâties,
\VS{11}des maisons pleines de toutes sortes de biens que tu n'as point remplies, des puits creusés que tu n'as point creusés, des vignes et des oliviers que tu n'as point plantés ; tu mangeras, et tu te rassasieras,
\VS{12}prends garde à toi, de peur que tu n'oublies Yahweh, qui t'a fait sortir du pays d'Egypte, de la maison de servitude.
\VS{13}Tu craindras Yahweh, ton Dieu, tu le serviras et tu jureras par son Nom.
\VS{14}Vous n'irez point après d'autres dieux, d'entre les dieux des peuples qui sont autour de vous ;
\VS{15}car Yahweh, ton Dieu, est un Dieu jaloux au milieu de toi ; de peur que la colère de Yahweh, ton Dieu, ne s'enflamme contre toi, et qu'il ne t'extermine de dessus la terre.
\VS{16}Vous ne tenterez point Yahweh, votre Dieu, comme vous l'avez tenté à Massa.
\VS{17}Vous garderez soigneusement les commandements de Yahweh, votre Dieu, ses ordonnances et ses lois qu'il vous a ordonnées.
\VS{18}Tu feras ce qui est droit et bon aux yeux de Yahweh, afin que tu sois heureux, que tu entres et que tu possèdes le bon pays que Yahweh a juré à tes pères,
\VS{19}après qu'il aura chassé tous tes ennemis de devant toi, comme Yahweh l'a dit.
\VS{20}Quand ton fils t'interrogera à l'avenir, en disant : Que veulent dire ces préceptes, ces lois, et ces ordonnances que Yahweh, notre Dieu, vous a ordonnées ?
\VS{21}Tu diras à ton fils : Nous étions esclaves de Pharaon en Egypte, et Yahweh nous a fait sortir de l'Egypte par sa main puissante.
\VS{22}Yahweh a fait sous nos yeux des signes et des miracles, grands et désastreux contre l'Egypte, contre Pharaon et contre toute sa maison ;
\VS{23}et il nous a fait sortir de là pour nous conduire dans le pays qu'il avait juré à nos pères de nous donner.
\VS{24}Yahweh nous a ordonné de pratiquer toutes ces lois, et de craindre Yahweh, notre Dieu, afin que nous soyons toujours heureux, et qu'il préserve notre vie, comme aujourd'hui.
\VS{25}Et ceci sera notre justice, que nous prenions garde de faire tous ces commandements devant Yahweh, notre Dieu, comme il nous l'a ordonné.
\Chap{7}
\TextTitle{Yahweh interdit le mélange avec les peuples païens}
\VerseOne{}Quand Yahweh, ton Dieu, t'aura fait entrer dans le pays où tu vas entrer pour le posséder, et qu'il aura chassé de devant toi beaucoup de nations, les Héthiens, les Guirgasiens, les Amoréens, les Cananéens, les Phéréziens, les Héviens, et les Jébusiens, sept nations plus grandes et plus puissantes que toi ;
\VS{2}et que Yahweh, ton Dieu, te les aura livrées en face et que tu les auras battues, tu les dévoueras complètement par interdit, tu ne traiteras point d'alliance avec elles, et tu ne leur feras point de grâce.
\VS{3}Tu ne t'allieras point par mariage avec elles, tu ne donneras point tes filles à leurs fils, et tu ne prendras point leurs filles pour tes fils\FTNT{Jos. 23:12-13.} ;
\VS{4}car elles détourneraient de moi tes fils, et ils serviraient d'autres dieux, et la colère de Yahweh s'enflammerait contre vous : Il te détruirait promptement.
\VS{5}Mais vous les traiterez de cette manière : Vous renverserez leurs autels, vous briserez leurs statues, vous abattrez leurs Asherah\FTNT{Le mot idole vient de l'hébreu « Asherah ». Il est cité au moins quarante fois dans le Tanakh. Il fait référence à un objet en bois utilisé dans le culte d'Astarté, l'épouse de Baal. Voir De. 19:21.}, et vous brûlerez au feu leurs images taillées.
\VS{6}Car tu es un peuple saint pour Yahweh, ton Dieu. Yahweh, ton Dieu, t'a choisi pour que tu lui sois un peuple précieux entre tous les peuples qui sont sur la face de la terre.
\VS{7}Ce n'est pas parce que vous êtes plus nombreux que tous les peuples que Yahweh vous a aimé et qu'il vous a choisis ; car vous êtes le plus petit de tous les peuples.
\VS{8}Mais c'est parce que Yahweh vous aime, et qu'il garde le serment qu'il a juré à vos pères, Yahweh vous a fait sortir par sa main puissante, et vous a rachetés de la maison de servitude, de la main de Pharaon, roi d'Egypte.
\VS{9}Sache que c'est Yahweh, ton Dieu, qui est Dieu. Ce Dieu fidèle garde son alliance et sa miséricorde jusqu'à mille générations envers ceux qui l'aiment et qui gardent ses commandements,
\VS{10}et qui rend la pareille en face à ceux qui le haïssent, et les fait périr ; il ne diffère point envers celui qui le hait, il lui rend la pareille en face.
\VS{11}Garde les commandements, les lois, et les ordonnances que je t'ordonne aujourd'hui, et mets-les en pratique.
\TextTitle{Obéissance à Yahweh, source de victoire}
\VS{12}Et il arrivera que si vous écoutez ces ordonnances, si vous les gardez et les mettez en pratique, Yahweh, ton Dieu, gardera l'alliance et la bonté qu'il a jurées à tes pères.
\VS{13}Et il t'aimera, te bénira, et te multipliera ; il bénira le fruit de tes entrailles, et le fruit de ta terre, ton blé, ton vin, et ton huile, les portées de ton gros et de ton menu bétail, sur la terre qu'il a juré de donner à tes pères.
\VS{14}Tu seras béni plus que tous les peuples ; il n'y aura, parmi toi et parmi tes bêtes, ni mâle ni femelle stérile\FTNT{Ex. 23:26.}.
\VS{15}Yahweh détournera de toi toute maladie ; il ne t'enverra aucun de ces mauvais maux d'Egypte qui te sont connus, mais il les fera venir sur tous ceux qui te haïssent.
\VS{16}Tu détruiras donc tous les peuples que Yahweh, ton Dieu, va te livrer, ton œil n'aura point de pitié, et tu ne serviras point leurs dieux, car cela te serait un piège.
\VS{17}Si tu dis dans ton cœur : Ces nations sont plus nombreuses que moi, comment pourrai-je les déposséder ?
\VS{18}Ne les crains point. Rappelle-toi bien ce que Yahweh, ton Dieu, a fait à Pharaon, et à tous les Egyptiens,
\VS{19}de ces grandes épreuves que tes yeux ont vues, les signes et les miracles, la main forte et le bras étendu par lesquels Yahweh, ton Dieu, t'a fait sortir : Ainsi fera Yahweh, ton Dieu, à tous ces peuples que tu crains.
\VS{20}Yahweh, ton Dieu, enverra contre eux les frelons, jusqu'à ce que périssent ceux qui resteront, et ceux qui se seront cachés de devant toi.
\VS{21}Ne t'effraie point devant eux, car Yahweh, ton Dieu, le Dieu grand et terrible est au milieu de toi.
\VS{22}Or Yahweh, ton Dieu, chassera peu à peu ces nations de devant toi ; tu ne pourras pas les exterminer promptement, de peur que les bêtes des champs ne se multiplient contre toi.
\VS{23}Mais Yahweh, ton Dieu, les livrera devant toi ; et il les troublera par de grandes confusions, jusqu'à ce qu'elles soient détruites.
\VS{24}Et il livrera leurs rois entre tes mains, et tu feras disparaître leurs noms de dessous les cieux ; aucun homme ne tiendra face à toi, jusqu'à ce que tu les aies détruits.
\VS{25}Tu brûleras au feu les images taillées de leurs dieux. Tu ne convoiteras point et tu ne prendras point pour toi l'argent et l'or qui seront sur elles, de peur que tu en sois pris au piège ; car c'est une abomination pour Yahweh, ton Dieu.
\VS{26}Ainsi tu n'introduiras point de choses abominables dans ta maison, afin que tu ne sois pas, comme cette chose, dévoué par interdit ; tu la détesteras fortement, et tu l'auras en abomination, car c'est une chose dévouée par interdit.
\Chap{8}
\TextTitle{Le désert, formation des fils}
\VerseOne{}Vous observerez et vous mettrez en pratique tous les commandements que je vous ordonne aujourd'hui, afin que vous viviez, que vous multipliiez, et que vous entriez en possession du pays que Yahweh a juré de donner à vos pères.
\VS{2}Et souviens-toi de tout le chemin par lequel Yahweh, ton Dieu, t'a fait marcher pendant ces quarante ans dans ce désert, afin de t'humilier et de t'éprouver, pour connaître ce qui était dans ton cœur, et si tu garderais ses commandements ou non.
\VS{3}Il t'a donc humilié, il t'a laissé avoir faim, mais il t'a donné à manger de la manne, que tu ne connaissais pas et que n'avaient point connue tes pères, afin de te faire connaître que l'homme ne vit pas de pain seulement, mais que l'homme vit de tout ce qui sort de la bouche de Yahweh\FTNT{Mt. 4:4 ; Lu. 4:4.}.
\VS{4}Ton vêtement ne s'est point usé sur toi, et ton pied ne s'est point enflé durant ces quarante années\FTNT{Né. 9:21.}.
\VS{5}Reconnais dans ton cœur que Yahweh, ton Dieu, te châtie comme un homme châtie son fils\FTNT{Hé. 12:5-12.}.
\TextTitle{Invitation à demeurer humblement dans les voies de Yahweh}
\VS{6}Et garde les commandements de Yahweh, ton Dieu, pour marcher dans ses voies, et pour le craindre.
\VS{7}Car Yahweh, ton Dieu, va te faire entrer dans un bon pays, un pays de torrents d'eaux, de fontaines et d'abîmes, qui jaillissent des vallées et des montagnes ;
\VS{8}un pays de blé, d'orge, de vignes, de figuiers, et de grenadiers ; un pays d'oliviers donnant de l'huile et du miel ;
\VS{9}un pays où tu ne mangeras point le pain avec disette, où tu ne manqueras de rien ; un pays dont les pierres sont du fer, et des montagnes desquelles tu tailleras l'airain.
\VS{10}Tu mangeras et tu te rassasieras, tu béniras Yahweh, ton Dieu, pour le bon pays qu'il t'a donné.
\VS{11}Prends garde à toi de peur que tu n'oublies Yahweh, ton Dieu, en ne gardant point ses commandements, ses ordonnances, et ses lois que je t'ordonne aujourd'hui.
\VS{12}Et quand tu mangeras, et te rassasieras, lorsque tu bâtiras et habiteras de belles maisons,
\VS{13}et quand ton gros et menu bétail se multipliera, et que ton argent et ton or augmentera, et que tout ce qui est à toi se multipliera,
\VS{14}que ton cœur ne s'élève point, et que tu n'oublies point Yahweh, ton Dieu, qui t'a fait sortir du pays d'Egypte, de la maison de servitude,
\VS{15}qui t'a fait marcher dans ce grand et affreux désert de serpents brûlants et de scorpions, dans des lieux arides et sans eau, et qui a fait jaillir pour toi de l'eau du rocher le plus dur,
\VS{16}qui t'as fait manger dans ce désert la manne que tes pères n'avaient point connue, afin de t'humilier et de t'éprouver, pour te faire ensuite du bien,
\VS{17}et que tu ne dises dans ton cœur : Ma force et la puissance de ma main m'ont acquis ces richesses.
\VS{18}Mais souviens-toi de Yahweh, ton Dieu, car c'est lui qui te donnera de la force pour acquérir ces richesses, afin de confirmer son alliance qu'il a jurée à tes pères, comme tu le vois aujourd'hui.
\VS{19}Mais si tu oublies Yahweh, ton Dieu, et que tu vas après d'autres dieux, si tu les sers, et que tu te prosternes devant eux, je vous avertis aujourd'hui que vous périrez certainement.
\VS{20}Vous périrez comme les nations que Yahweh fait périr devant vous, parce que vous n'aurez obéi à la voix de Yahweh, votre Dieu.
\Chap{9}
\TextTitle{Yahweh fidèle à son alliance malgré la rébellion du peuple}
\VerseOne{}Ecoute, Israël ! Tu vas passer aujourd'hui le Jourdain, pour aller posséder des nations plus grandes et plus puissantes que toi, des villes grandes et fortifiées jusqu'au ciel,
\VS{2}un peuple grand et de haute taille, les fils d'Anak, que tu connais, et dont tu as entendu dire : Qui tiendra face aux fils d'Anak ?
\VS{3}Sache donc aujourd'hui que Yahweh, ton Dieu, passera devant toi, comme un feu dévorant, c'est lui qui les détruira, qui les humiliera devant toi ; tu les chasseras, et tu les feras périr promptement, comme Yahweh te l'a dit.
\VS{4}Ne parle pas en ton cœur, quand Yahweh, ton Dieu, les chassera devant toi, en disant : C'est à cause de ma justice que Yahweh me fait entrer en possession de ce pays. Car c'est à cause de la méchanceté de ces nations-là que Yahweh les chasse devant toi.
\VS{5}Ce n'est point pour ta justice ni pour la droiture de ton cœur que tu entres en possession de leur pays, mais c'est pour la méchanceté de ces nations-là que Yahweh, ton Dieu, les chasse devant toi, et pour confirmer la parole que Yahweh a jurée à tes pères, Abraham, Isaac, et Jacob.
\VS{6}Sache donc que ce n'est point pour ta justice que Yahweh, ton Dieu, te donne ce bon pays pour que tu le possèdes ; car tu es un peuple au cou raide.
\VS{7}Souviens-toi, n'oublie pas que tu as excité la colère de Yahweh, ton Dieu, dans le désert. Depuis le jour où tu es sorti du pays d'Egypte jusqu'à ce que vous arriviez dans ce lieu, vous avez été rebelles contre Yahweh.
\VS{8}Même à Horeb vous avez excité la colère de Yahweh ; et Yahweh s'irrita contre vous, pour vous détruire.
\VS{9}Quand je montai sur la montagne, pour prendre les tables de pierre, les tables de l'alliance que Yahweh a traitée avec vous, je demeurai sur la montagne quarante jours et quarante nuits, sans manger de pain et sans boire d'eau ;
\VS{10}et Yahweh me donna les deux tables de pierre écrites du doigt de Dieu, et contenant toutes les paroles que Yahweh avait déclarées sur la montagne, du milieu du feu, le jour de l'assemblée.
\VS{11}Et il arriva qu'au bout de quarante jours et quarante nuits, Yahweh me donna les deux tables de pierre, qui sont les tables de l'alliance.
\VS{12}Puis Yahweh me dit : Lève-toi, descends promptement d'ici ; car ton peuple, que tu as fait sortir d'Egypte, s'est corrompu. Ils se sont détournés promptement de la voie que je leur avais commandé, ils se sont fait une image en métal fondu.
\VS{13}Yahweh me parla, en disant : Je vois que ce peuple est un peuple au cou raide.
\VS{14}Laisse-moi les détruire et effacer leur nom de dessous les cieux ; et je te ferai devenir une nation plus puissante et plus grande que celle-ci.
\VS{15}Je retournai et je descendis de la montagne ; or la montagne était toute en feu, et j'avais les deux tables de l'alliance dans mes deux mains.
\VS{16}Puis je regardai, et voici, vous aviez péché contre Yahweh, votre Dieu, vous vous étiez fait un veau en métal fondu, vous vous étiez détournés promptement de la voie que vous avait ordonnée Yahweh.
\VS{17}Alors je saisis les deux tables, je les jetai de mes deux mains, et je les brisai devant vos yeux.
\TextTitle{Moïse, intercesseur auprès de Yahweh pour Israël}
\VS{18}Puis je me prosternai devant Yahweh, comme auparavant, quarante jours et quarante nuits, sans manger de pain et sans boire d'eau, à cause de tout votre péché, que vous aviez commis en faisant ce qui est mal aux yeux de Yahweh, afin de l'irriter.
\VS{19}Car je craignais face à la colère et à la fureur dont Yahweh était enflammé contre vous pour vous détruire. Et Yahweh m'exauça encore cette fois.
\VS{20}Yahweh était très irrité contre Aaron, voulant le faire périr, mais j'intercédai pour Aaron en ce temps-là.
\VS{21}Puis je pris le veau\FTNT{Le veau d'or (Ex. 32).} que vous aviez fait, votre péché, et je le brûlai au feu, je le brisai en le broyant, jusqu'à ce qu'il soit réduit en poudre, et je jetai cette poudre dans le torrent qui descend de la montagne.
\VS{22}Vous avez fort irrité la colère de Yahweh à Tabeéra, à Massa, et à Kibroth-Hattaava, .
\VS{23}Et quand Yahweh vous envoya à Kadès-Barnéa, en disant : Montez, et prenez possession du pays que je vous donne ! Vous fûtes rebelles à la parole de Yahweh, votre Dieu, vous n'eûtes point confiance, et vous n'obéîtes point à sa voix.
\VS{24}Vous avez été rebelles à Yahweh depuis le jour que je vous connais.
\VS{25}Je me prosternai donc devant Yahweh, je me prosternai quarante jours et quarante nuits, parce que Yahweh avait dit qu'il vous détruirait.
\VS{26}Et je priai Yahweh, et je dis : Ô Seigneur, Yahweh, ne détruis point ton peuple, ton héritage que tu as racheté par ta grandeur, et que tu as fait sortir d'Egypte par ta main puissante.
\VS{27}Souviens-toi de tes serviteurs Abraham, Isaac, et Jacob. Ne regarde point à l'obstination de ce peuple, ni à sa méchanceté, ni à son péché,
\VS{28}de peur que le pays d'où tu nous as fait sortir ne dise : Parce que Yahweh n'était pas capable de les conduire dans le pays qu'il leur avait promis, et parce qu'il les haïssait, il les a fait sortir pour les faire mourir dans le désert.
\VS{29}Cependant ils sont ton peuple et ton héritage, que tu as fait sortir par ta grande puissance et par ton bras étendu.
\Chap{10}
\TextTitle{Rappel du remplacement des tables de la loi}
\VerseOne{}En ce temps-là Yahweh me dit : Taille deux tables de pierre comme les premières, et monte vers moi sur la montagne ; tu feras une arche de bois\FTNT{Ex. 34:1-4.}.
\VS{2}Et j'écrirai sur ces tables les paroles qui étaient sur les premières tables que tu as brisées, et tu les mettras dans l'arche.
\VS{3}Ainsi je fis une arche de bois d'acacia, je taillai deux tables de pierre comme les premières, et je montai sur la montagne, les deux tables dans ma main\FTNT{Ex. 34:1-4.}.
\VS{4}Et Yahweh écrivit sur ces tables ce qui avait été écrit sur les premières, les dix paroles qu'il avait dites sur la montagne, du milieu du feu, le jour de l'assemblée ; puis Yahweh me les donna.
\VS{5}Je me retournai et je descendis de la montagne ; je mis les tables dans l'arche que j'avais faite, et elles y sont demeurées, comme Yahweh me l'avait ordonné.
\VS{6}Or les fils d'Israël partirent de Beéroth-Bené Jaakan pour Moséra. Là mourut Aaron, et il fut enseveli ; Eléazar, son fils, exerça la sacrificature à sa place.
\VS{7}De là ils partirent pour Gudgoda, et de Gudgoda pour Jothbatha, qui est un pays de torrents d'eau.
\VS{8}Or en ce temps-là, Yahweh sépara la tribu de Lévi afin de porter l'arche de l'alliance de Yahweh, de se tenir devant Yahweh, de le servir, et de bénir en son Nom, jusqu'à ce jour.
\VS{9}C'est pourquoi Lévi n'a ni portion ni d'héritage avec ses frères : Yahweh est son héritage, comme Yahweh, ton Dieu, lui a dit.
\VS{10}Je restai sur la montagne, comme la première fois, quarante jours et quarante nuits. Yahweh m'exauça encore cette fois ; Yahweh ne voulut point te détruire.
\VS{11}Mais Yahweh me dit : Lève-toi, va, marche devant ce peuple. Qu'ils aillent prendre possession du pays que j'ai juré à leurs pères de leur donner.
\TextTitle{Une alliance basée sur l'amour de Yahweh}
\VS{12}Maintenant donc, ô Israël, que demande de toi Yahweh, ton Dieu, sinon que tu craignes Yahweh, ton Dieu, afin de marcher dans toutes ses voies, d'aimer et de servir Yahweh, ton Dieu, de tout ton cœur, et de toute ton âme ;
\VS{13}de garder les commandements de Yahweh et ses lois que je t'ordonne aujourd'hui, afin que tu sois heureux ?
\VS{14}Voici, les cieux, et les cieux des cieux appartiennent à Yahweh, ton Dieu, la terre et tout ce qu'elle renferme.
\VS{15}Et Yahweh s'est attaché à tes pères, pour les aimer ; et après eux, il vous a choisis, vous, leur postérité, entre tous les peuples, comme vous le voyez aujourd'hui.
\VS{16}Circoncisez donc le prépuce de votre cœur, et vous ne roidirez plus votre cou.
\VS{17}Car Yahweh, votre Dieu, est le Dieu des dieux, le Seigneur des seigneurs\FTNT{Yahweh, le Dieu des dieux et le Seigneur des seigneurs n'est autre que Jésus-Christ, notre Seigneur qui s'est révélé à Jean comme le Seigneur des seigneurs et le Roi des rois (Ap. 19:16).}, le Dieu grand, fort, et redoutable, qui ne fait acception de personnes et qui ne reçoit point de présents ;
\VS{18}qui fait justice à l'orphelin et à la veuve, qui aime l'étranger et lui donne le pain et le vêtement.
\VS{19}Vous aimerez donc l'étranger ; car vous avez été étrangers dans le pays d'Egypte.
\VS{20}Tu craindras Yahweh, ton Dieu, tu le serviras, tu t'attacheras à lui, et tu jureras par son Nom.
\VS{21}Il est ta louange, il est ton Dieu, qui a fait pour toi des choses grandes et redoutables que tes yeux ont vues.
\VS{22}Tes pères descendirent en Egypte au nombre de soixante-dix âmes ; et maintenant Yahweh, ton Dieu, t'a fait devenir comme les étoiles des cieux, en multitude.
\Chap{11}
\TextTitle{Exhortation à la reconnaissance et à l'obéissance}
\VerseOne{}Tu aimeras donc Yahweh, ton Dieu, et tu garderas toujours ses lois, ses ordonnances, et ses commandements.
\VS{2}et reconnaissez aujourd'hui, ce que n'ont point connu ni vu vos fils, le châtiment de Yahweh, votre Dieu, sa grandeur, sa main puissante, et son bras étendu,
\VS{3}et ses signes, et les œuvres qu'il a accomplies au milieu de l'Egypte contre Pharaon, roi d'Egypte, et contre tout son pays.
\VS{4}Et ce que Yahweh a fait à l'armée d'Egypte, à ses chevaux et à ses chars, quand il a fait déborder sur eux les eaux de la Mer Rouge, lorsqu'ils vous poursuivaient et il les a détruits jusqu'à ce jour\FTNT{Ex. 14:28.} ;
\VS{5}ce qu'il a fait dans le désert, jusqu'à votre arrivée en ce lieu-ci ;
\VS{6}ce qu'il a fait à Dathan et à Abiram, fils d'Eliab, fils de Ruben, comment la terre ouvrit sa bouche et les engloutit, avec leurs maisons et leurs tentes, et tous les êtres qui les suivaient, au milieu de tout Israël\FTNT{No. 16:1-33.}.
\VS{7}Car ce sont vos yeux qui ont vu toutes les grandes œuvres que Yahweh a faites.
\TextTitle{Les bienfaits de la terre promise sont pour un peuple fidèle}
\VS{8}Vous garderez donc tous les commandements que je vous ordonne aujourd'hui, afin que vous ayez la force d'entrer et de vous emparer du pays où vous allez passer pour en prendre possession,
\VS{9}et afin que vous prolongiez vos jours sur la terre que Yahweh a juré à vos pères de leur donner, ainsi qu'à leur postérité, pays où coulent le lait et le miel.
\VS{10}Car le pays où tu vas entrer afin de le posséder n'est pas comme le pays d'Egypte, d'où vous êtes sortis, où tu semais ta semence, et l'arrosais avec ton pied, comme un jardin potager.
\VS{11}Mais le pays où vous allez passer pour le posséder est un pays de montagnes et de vallées, qui boit les eaux de la pluie du ciel ;
\VS{12}c'est un pays dont Yahweh, ton Dieu, prend soin, et sur lequel Yahweh, ton Dieu, a continuellement ses yeux, du commencement de l'année jusqu'à la fin de l'année.
\VS{13}Il arrivera donc que si vous obéissez attentivement à mes commandements que je vous ordonne aujourd'hui, si vous aimez Yahweh, votre Dieu, et que vous le servez de tout votre cœur et de toute votre âme,
\VS{14}alors je donnerai à votre pays la pluie en son temps, la pluie de la première et de l'arrière-saison, et tu recueilleras ton blé, ton vin, et ton huile.
\VS{15}Je mettrai aussi dans ton champ de l'herbe pour ton bétail, tu mangeras et tu seras rassasié.
\VS{16}Prenez garde à vous, de peur que votre cœur ne soit trompé, et que vous ne vous détourniez, et ne serviez d'autres dieux, et ne vous prosterniez devant eux.
\VS{17}Et que la colère de Yahweh s'enflammerait contre vous ; il fermerait les cieux, et il n'y aurait point de pluie ; la terre ne donnerait plus son produit, et vous péririez promptement dans ce bon pays que Yahweh vous donne.
\VS{18}Mettez donc dans votre cœur et dans votre âme ces paroles. Liez-les comme un signe sur vos mains, et qu'elles soient comme des fronteaux entre vos yeux.
\VS{19}Et enseignez-les à vos fils, en leur en parlant, quand tu seras dans ta maison, quand tu partiras en voyage, quand tu te coucheras et quand tu te lèveras.
\VS{20}Tu les écriras aussi sur les poteaux de ta maison, et sur tes portes.
\VS{21}Afin que vos jours et les jours de vos fils, sur la terre que Yahweh a juré à vos pères de leur donner, soient aussi nombreux que les jours des cieux sur la terre.
\VS{22}Car si vous gardez soigneusement tous ces commandements que je vous ordonne de faire, aimant Yahweh, votre Dieu, marchant dans toutes ses voies, et vous attachant à lui,
\VS{23}Alors Yahweh chassera devant vous toutes ces nations et vous prendrez possession de nations plus grandes et plus puissantes que vous.
\VS{24}Tout lieu que foulera la plante de votre pied sera à vous\FTNT{Jos. 1:3 ; Jos 14:9.} : Votre territoire s'étendra du désert au Liban, et du fleuve, le fleuve de l'Euphrate jusqu'à la Mer Occidentale.
\VS{25}Aucun homme ne tiendra face à vous. Yahweh, votre Dieu, mettra, comme il vous l'a dit, la frayeur et la crainte de vous sur tout le pays où vous marcherez.
\TextTitle{La malédiction ou la bénédiction}
\VS{26}Regardez, je mets aujourd'hui devant vous la bénédiction et la malédiction :
\VS{27}La bénédiction, si vous obéissez aux commandements de Yahweh, votre Dieu, que je vous ordonne aujourd'hui ;
\VS{28}la malédiction, si vous n'obéissez point aux commandements de Yahweh, votre Dieu, et si vous vous détournez du chemin que je vous ordonne aujourd'hui, pour aller après d'autres dieux que vous ne connaissez point.
\VS{29}Et quand Yahweh, ton Dieu, t'aura fait entrer dans le pays dont tu vas prendre possession, tu prononceras alors les bénédictions, étant sur la montagne de Garizim, et les malédictions, étant sur la montagne d'Ebal.
\VS{30}Ces montagnes ne sont-elles pas de l'autre côté du Jourdain, derrière le chemin du soleil couchant, au pays des Cananéens qui demeurent dans la plaine, vis-à-vis de Guilgal, près des chênes de Moré ?
\VS{31}Car vous allez passer le Jourdain, pour entrer et prendre possession du pays que Yahweh, votre Dieu, vous donne ; vous le posséderez, et vous y habiterez.
\VS{32}Vous garderez et pratiquerez toutes les lois et les ordonnances que je mets aujourd'hui devant vous.
\Chap{12}
\TextTitle{Lois sur les sacrifices offerts au lieu où résidera le Nom de Yahweh}
\VerseOne{}Ce sont ici les lois et les ordonnances que vous garderez et pratiquerez, dans le pays que Yahweh, le Dieu de vos pères, vous a donné à posséder, tout le temps que vous vivrez sur cette terre.
\VS{2}Vous détruirez entièrement tous les lieux où les nations que vous allez déposséder servent leurs dieux, sur les hautes montagnes et sur les collines, et sous tout arbre verdoyant.
\VS{3}Vous renverserez leurs autels, vous briserez leurs statues, vous brûlerez au feu leurs idoles, vous abattrez les images taillées de leurs dieux, et vous ferez périr leurs noms de ces lieux-là.
\VS{4}Vous ne ferez pas ainsi à Yahweh, votre Dieu.
\VS{5}Mais vous le chercherez dans sa demeure, et vous irez au lieu que Yahweh, votre Dieu, aura choisi d'entre toutes vos tribus, pour y mettre son Nom.
\VS{6}Et vous y apporterez vos holocaustes, vos sacrifices, vos dîmes, vos offrandes élevées, vos vœux, vos offrandes volontaires de vos mains, et les premiers-nés de votre gros et de votre menu bétail\FTNT{Lé. 17:3-4.}.
\VS{7}Et là, vous mangerez devant Yahweh, votre Dieu, et vous vous réjouirez, vous et vos familles, de toutes les choses auxquelles vous aurez mis la main, et dans lesquelles Yahweh, votre Dieu, vous aura bénis.
\VS{8}Vous ne ferez pas comme nous faisons ici aujourd'hui, où chacun fait ce qui lui semble juste à ses yeux,
\VS{9}car vous n'êtes point encore entrés dans le lieu de repos, et dans l'héritage que Yahweh, votre Dieu, vous donne.
\VS{10}Vous passerez le Jourdain, et vous habiterez dans le pays que Yahweh, votre Dieu, vous donne en héritage ; il vous donnera du repos de tous vos ennemis qui vous entourent, et vous y habiterez en sécurité.
\VS{11}Et il y aura un lieu que Yahweh, votre Dieu, choisira pour y faire habiter son Nom. Vous y apporterez tout ce que je vous ordonne, vos holocaustes, vos sacrifices, vos dîmes, vos offrandes élevées de vos mains, et toutes offrandes de choix pour les vœux que vous aurez vouer à Yahweh.
\VS{12}Et là, vous vous réjouirez devant Yahweh, votre Dieu, vous, vos fils et vos filles, vos serviteurs et vos servantes, et le Lévite qui sera dans vos portes ; car il n'a ni part ni héritage avec vous.
\VS{13}Garde-toi d'offrir tes holocaustes dans tous les lieux que tu verras ;
\VS{14}mais tu offriras tes holocaustes dans le lieu que Yahweh choisira dans l'une de tes tribus, et tu y feras tout ce que je t'ordonne.
\VS{15}Toutefois, selon le désir de ton âme, tu pourras tuer et manger de la viande dans toutes tes portes, selon la bénédiction que t'accordera Yahweh, ton Dieu ; celui qui sera impur et celui qui sera pur en mangeront, comme on mange de la gazelle et du cerf.
\VS{16}Seulement, vous ne mangerez point de sang : Tu le répandras sur la terre, comme de l'eau.
\VS{17}Tu ne pourras pas manger dans tes portes la dîme de ton blé, de ton vin, de ton huile, ni les premiers-nés de ton gros et menu bétail, ni aucune de tes offrandes en accomplissement d'un vœu, ni tes offrandes volontaires, ni les offrandes élevées de tes mains.
\VS{18}Mais tu les mangeras devant Yahweh, ton Dieu, au lieu que Yahweh, ton Dieu, choisira ; toi, ton fils, ta fille, ton serviteur et ta servante, et le Lévite qui sera dans tes portes ; et tu te réjouiras devant Yahweh, ton Dieu, de tout ce à quoi tu auras mis la main.
\VS{19}Garde-toi, tout le temps que tu vivras sur la terre, d'abandonner le Lévite.
\VS{20}Quand Yahweh, ton Dieu, aura élargi tes frontières, comme il te l'a promis, et que tu diras : Je mangerai de la chair, parce que ton âme désirera manger de la chair, tu en mangeras selon tous les désirs de ton âme.
\VS{21}Si le lieu que Yahweh, ton Dieu, aura choisi pour y mettre son Nom, est loin de toi, alors tu tueras de ton gros et menu bétail, comme je te l'ai ordonné, et tu en mangeras dans tes portes selon tous les désirs de ton âme.
\VS{22}Tu en mangeras comme on mange de la gazelle et du cerf ; celui qui sera impur et celui qui sera pur en mangeront également.
\VS{23}Seulement, garde-toi de manger le sang, car le sang c'est l'âme ; et tu ne mangeras point l'âme avec la chair\FTNT{Lé 7:26.}.
\VS{24}Tu n'en mangeras point : Tu le répandras sur la terre comme de l'eau.
\VS{25}Tu n'en mangeras point, afin que tu sois heureux, toi et tes fils après toi, parce que tu auras fait ce qui est droit aux yeux de Yahweh.
\VS{26}Mais tu prendras les choses que tu auras consacrées, qui seront à toi, et ce que tu auras voué, tu les prendras et tu viendras au lieu que Yahweh aura choisi.
\VS{27}Et tu offriras tes holocaustes, la chair et le sang, sur l'autel de Yahweh, ton Dieu ; mais le sang de tes autres sacrifices sera versé sur l'autel de Yahweh, ton Dieu, et tu en mangeras la chair.
\VS{28}Garde et écoute toutes ces paroles que je t'ordonne, afin que tu sois heureux, toi et tes fils après toi, à jamais, en faisant ce qui est bon et droit aux yeux de Yahweh, ton Dieu.
\TextTitle{Mise en garde contre la séduction et les dieux étrangers}
\VS{29}Quand Yahweh, ton Dieu, aura exterminé de devant toi les nations que tu vas prendre en possession, que tu les auras possédées, et que tu habiteras dans leur pays,
\VS{30} prends garde à toi, de peur que tu ne sois pris au piège après elles, quand elles auront été détruites de devant toi ; et que tu ne recherches leurs dieux, en disant : Comme ces nations-là servaient leurs dieux, je le ferai aussi tout de même.
\VS{31}Tu ne feras point ainsi à Yahweh, ton Dieu ; car elles ont fait à leurs dieux tout ce qui est en abomination et qui est odieux à Yahweh, et même ils brûlaient au feu leurs fils et leurs filles à leurs dieux.
\VS{32}Vous prendrez garde de faire tout ce que je vous commande. Vous n'y ajouterez rien, et vous n'en retrancherez rien.
\Chap{13}
\TextTitle{Eprouver les faux prophètes, ôter le méchant du milieu de l'assemblée}
\VerseOne{}S'il s'élève au milieu de toi un prophète ou un songeur de songes, qui te donne un signe ou miracle,
\VS{2}et que ce signe ou ce miracle dont il t'a parlé, arrive, et qu'il te dise : Allons après d'autres dieux que tu ne connais point, et servons-les !
\VS{3}Tu n'écouteras point les paroles de ce prophète ni de ce songeur de songes, car Yahweh, votre Dieu, vous met à l'épreuve pour savoir si vous aimez Yahweh, votre Dieu, de tout votre cœur et de toute votre âme.
\VS{4}Vous marcherez après Yahweh, votre Dieu, vous le craindrez ; vous garderez ses commandements, vous obéirez à sa voix, vous le servirez, et vous vous attacherez à lui.
\VS{5}Mais on fera mourir ce prophète-là ou ce songeur de songes, parce qu'il a parlé de révolte\FTNT{Révolte de l'hébreu « carah » qui signifie « apostasie ». L'apostasie est une déviation progressive. C'est tout d'abord l'abandon d'une vérité reçue. L'apôtre Paul enseigne que deux événements doivent avoir lieu avant le retour du Seigneur sur la terre : L'apostasie et la révélation de l'homme du péché, le fils de perdition, c'est-à-dire l'Antichrist (2 Th. 2:1-3 ; 2 Ti. 4:1).} contre Yahweh, votre Dieu, qui vous a fait sortir du pays d'Egypte et vous a délivrés de la maison de servitude, pour vous conduire loin de la voie que Yahweh, votre Dieu, vous a ordonné de marcher. Tu ôteras le méchant du milieu de toi.
\VS{6}Quand ton frère, fils de ta mère, ou ton fils, ou ta fille, ou ta femme bien-aimée, ou ton intime ami, qui est comme ton âme, t'incitera, en te disant en secret : Allons, et servons d'autres dieux, que tu n'as point connus, ni tes pères,
\VS{7}d'entre les dieux des peuples qui sont autour de vous, près ou loin de toi, d'une extrémité de la terre jusqu'à l'autre,
\VS{8}tu ne t'accorderas pas avec lui, et tu ne l'écouteras point. Ton œil ne le regardera pas avec pitié, tu ne l'épargneras point, et tu ne le cacheras point.
\VS{9}Mais tu le feras mourir, tu le feras mourir\FTNT{Répétition du mot « mourir », voir commentaire en Ge. 2:17} ; ta main sera la première sur lui pour le mettre à mort, et ensuite la main de tout le peuple.
\VS{10}Tu le lapideras avec des pierres, et il mourra, parce qu'il a cherché à t'éloigner loin de Yahweh, ton Dieu, qui t'a fait sortir du pays d'Egypte, de la maison de servitude.
\VS{11}Afin que tout Israël entende et craigne, et que l'on ne fasse plus une action aussi méchante au milieu de toi.
\TextTitle{Jugement des villes idolâtres}
\VS{12}Si tu entends dire dans l'une des villes que Yahweh, ton Dieu, t'a données pour y habiter :
\VS{13}Des hommes, fils de Bélial, sont sortis du milieu de toi, et ont chassé les habitants de leur ville, en disant : Allons et servons d'autres dieux, des dieux que tu ne connais point !
\VS{14}Tu chercheras, tu examineras, tu t'enquerras bien. Et si c'est la vérité, si la chose est établie, si cette abomination a été faite au milieu de toi,
\VS{15}tu frapperas du tranchant de l'épée les habitants de cette ville, tu la dévoueras par interdit, et tu passeras le bétail au fil de l'épée.
\VS{16}Tu assembleras tout son butin au milieu de la place, et tu brûleras entièrement au feu cette ville et tout son butin, devant Yahweh, ton Dieu : Elle sera pour toujours un monceau de ruines, sans être jamais rebâtie.
\VS{17}Rien de ce qui sera dévoué ne s'attachera à ta main, afin que Yahweh revienne de l'ardeur de sa colère, qu'il te fasse miséricorde et grâce, et qu'il te multiplie, comme il a juré à tes pères,
\VS{18}si tu obéis à la voix de Yahweh, ton Dieu, en gardant tous ses commandements que je t'ordonne aujourd'hui, et en faisant ce qui est droit aux yeux de Yahweh, ton Dieu.
\Chap{14}
\TextTitle{Israël, peuple mis en part}
\VerseOne{}Vous êtes les fils de Yahweh, votre Dieu. Vous ne vous ferez aucune incision, et ne vous rasez point entre les yeux pour aucun mort.
\VS{2}Car tu es un peuple saint pour Yahweh, ton Dieu ; et Yahweh t'a choisi pour que tu lui sois un peuple qui lui appartienne entre tous les peuples qui sont sur la face de la terre.
\TextTitle{Lois sur l'alimentation}
\VS{3}Tu ne mangeras d'aucune chose abominable.
\VS{4}Ces ont ici les bêtes que vous mangerez : Le bœuf, la brebis et la chèvre ;
\VS{5}le cerf, la gazelle et le daim ; le bouquetin, le chevreuil, la chèvre sauvage, et le mouflon.
\VS{6}Vous mangerez donc toute bête qui a le sabot divisé, le pied fendu, et qui rumine.
\VS{7}Mais vous ne mangerez point de ceux qui ruminent seulement, ou qui ont le sabot divisé et le pied fendu seulement, comme le chameau, le lièvre, et le lapin, car ils ruminent bien, mais ils n'ont pas le sabot qui est fendu : Ils vous seront impurs.
\VS{8}Le porc aussi, car il a le sabot fendu, mais il ne rumine point : Il vous sera impur. Vous ne mangerez point de leur chair, et vous ne toucherez point à leur cadavre.
\VS{9}Voici ce que vous mangerez de tout ce qui est dans les eaux : Vous mangerez de tout ce qui a des nageoires et des écailles.
\VS{10}Mais vous ne mangerez point de ce qui n'a ni nageoires ni écailles : Cela vous sera impur.
\VS{11}Vous mangerez tout oiseau pur.
\VS{12}Mais voici ceux dont vous ne mangerez point : L'aigle, l'orfraie, l'aigle de mer ;
\VS{13}le vautour, le milan, et l'autour, selon leur espèce ;
\VS{14}le corbeau, selon son espèce ;
\VS{15}l'autruche, le hibou, la mouette, l'épervier, selon son espèce ;
\VS{16}le chat-huant, la chouette et le cygne ;
\VS{17}le cormoran, le pélican, le plongeon ;
\VS{18}la cigogne, le héron, selon leur espèce, la huppe et la chauve-souris.
\VS{19}Et tout reptile qui vole sera impur pour vous ; on n'en mangera point.
\VS{20}Mais vous mangerez de tout ce qui vole et qui est pur.
\VS{21}Vous ne mangerez aucun cadavre ; tu le donneras à l'étranger qui sera dans tes portes, et il le mangera, ou tu le vendras à un étranger ; car tu es un peuple saint pour Yahweh, ton Dieu. Tu ne feras point cuire le chevreau dans le lait de sa mère.
\TextTitle{Lois sur les dîmes (No. 18:21-32)}
\VS{22}Tu ne manqueras point de donner la dîme\FTNT{Il est question ici de la dîme que les Hébreux consommaient chaque année.} de tout le produit de ta semence, de ce qui sortira de ton champ, chaque année.
\VS{23}Et tu mangeras devant Yahweh, ton Dieu, au lieu qu'il aura choisi pour y faire habiter son Nom, la dîme de ton blé, de ton vin et de ton huile, et les premiers-nés de ton gros et menu bétail, afin que tu apprennes à toujours craindre Yahweh, ton Dieu.
\VS{24}Mais quand le chemin sera trop long pour que tu puisses les transporter, parce que le lieu que Yahweh, ton Dieu, aura choisi pour y mettre son Nom, sera trop loin de toi, lorsque Yahweh, ton Dieu, t'aura béni,
\VS{25}alors tu l'échangeras contre de l'argent, tu serreras l'argent dans ta main, et tu iras au lieu que Yahweh, ton Dieu, aura choisi.
\VS{26}Et tu donneras l'argent contre tout ce que ton âme désirera, des bœufs, des brebis, du vin et des liqueurs fortes, tout ce que ton âme demandera, tu le mangeras devant Yahweh, ton Dieu, et tu te réjouiras, toi et ta famille.
\VS{27}Tu n'abandonneras point le Lévite qui sera dans tes portes, parce qu'il n'a ni portion ni héritage avec toi\FTNT{Ce verset fait référence à la première dîme qui devait être donnée aux Lévites (Voir commentaire en No. 18:21 et Mal. 3:1).}.
\VS{28}Au bout de trois ans, tu feras sortir toutes les dîmes de tes produits de cette année-là, et tu les déposeras dans tes portes.
\VS{29}Alors le Lévite, qui n'a ni portion ni héritage avec toi, l'étranger, l'orphelin, et la veuve qui seront dans tes portes, viendront, mangeront et se rassasieront, afin que Yahweh, ton Dieu, te bénisse dans toute l'œuvre que tu feras de tes mains.
\Chap{15}
\TextTitle{Lois sur l'année sabbatique : la justice et la bonté de Yahweh}
\VerseOne{}Tous les sept ans, tu célébreras l'année de la relâche\FTNT{Ex. 21:2, Jé 34:14.}.
\VS{2}Et c'est ici la manière de célébrer l'année de relâche. Que tout homme ayant droit d'exiger quelque chose que ce soit, qu'il puisse exiger de son prochain, donnera relâche, et ne l'exigera point de son prochain ni de son frère, quand on aura proclamé le relâche, en l'honneur de Yahweh.
\VS{3}Tu l'exigeras de l'étranger ; mais ta main relâchera tout ce qui t'appartiendra chez ton frère.
\VS{4}Afin qu'il n'y ait point d'indigent chez toi, car Yahweh te bénira abondamment dans le pays que Yahweh, ton Dieu, te donnera à posséder pour héritage ;
\VS{5}pourvu seulement que tu obéisses bien à la voix de Yahweh, ton Dieu, en prenant garde de faire tous ces commandements que je t'ordonne aujourd'hui.
\VS{6}Parce que Yahweh, ton Dieu, te bénira comme il te l'a promis, tu prêteras sur gage à beaucoup de nations, et tu n'emprunteras point sur gage ; tu domineras sur beaucoup de nations, et elles ne domineront point sur toi.
\VS{7}Quand un de tes frères sera indigent au milieu de toi, dans l'une de tes portes, dans le pays que Yahweh, ton Dieu, te donne, tu n'endurciras point ton cœur, et tu ne fermeras point ta main à ton frère indigent.
\VS{8}Mais tu ne manqueras pas de lui ouvrir ta main, et tu ne manqueras pas de lui prêter sur gage autant qu'il en aura besoin pour son indigence, dans laquelle il se trouvera.
\VS{9}Prends garde à toi, de peur que tu n'aies dans ton cœur quelque chose de Bélial, et que tu ne dises : La septième année, l'année de la relâche approche ! Et que ton œil soit méchant envers ton frère indigent, afin de ne rien lui donner et qu'il ne crie à Yahweh contre toi, et qu'il n'y ait du péché en toi.
\VS{10}Tu lui donneras, et que ton cœur ne lui donne point à regret ; car à cause de cela, Yahweh, ton Dieu, te bénira dans toutes tes œuvres, et dans tout ce à quoi tu mettras tes mains.
\VS{11}Car il y aura toujours des indigents dans le pays ; c'est pourquoi je t'ordonne, et je te dis : Tu ne manqueras point d'ouvrir ta main à ton frère, à l'affligé, et à l'indigent dans ton pays.
\TextTitle{Loi sur les esclaves\FTNTT{Ps. 40:6-8}}
\VS{12}Quand l'un de tes frères Hébreux, homme ou femme, te sera vendu, il te servira six ans ; mais la septième année, tu le renverras libre de chez toi.
\VS{13}Et quand tu le renverras libre de chez toi, tu ne le renverras point à vide.
\VS{14}Tu ne manqueras pas de le charger de quelque chose de ton menu bétail, de ton aire, de ton pressoir, et tu lui donneras de ce que Yahweh, ton Dieu, t'aura béni.
\VS{15}Et tu te souviendras que tu as été esclave au pays d'Egypte, et que Yahweh, ton Dieu, t'en a racheté ; et c'est pour cela que je t'ordonne ceci aujourd'hui.
\VS{16}Mais s'il arrive qu'il te dise : Je ne sortirai point de chez toi ; parce qu'il t'aime, toi et ta maison, et qu'il se trouve bien chez toi,
\VS{17}alors tu prendras un poinçon et tu lui perceras l'oreille contre la porte, et il sera ton serviteur pour toujours. Tu en feras de même à ta servante.
\VS{18}Ce ne sera point, à tes yeux, dur de le renvoyer libre de chez toi, car il t'a servi six ans, double salaire d'un mercenaire ; et Yahweh, ton Dieu, te bénira en tout ce que tu feras.
\TextTitle{Loi sur les premiers-nés des animaux}
\VS{19}Tu consacreras à Yahweh, ton Dieu, tout premier-né mâle qui naîtra parmi ton gros et ton menu bétail. Tu ne travailleras point avec le premier-né de ton bœuf, et tu ne tondras point le premier-né de tes brebis\FTNT{Ex. 13:2.}.
\VS{20}Tu le mangeras, toi et ta famille, chaque année devant Yahweh, ton Dieu, dans le lieu que Yahweh aura choisi.
\VS{21}Mais s'il a quelque défaut, boiteux ou aveugle, ou qu'il ait quelque autre mauvais défaut, tu ne le sacrifieras point à Yahweh, ton Dieu.
\VS{22}Mais tu le mangeras dans tes portes ; celui qui sera impur et celui qui sera pur en mangeront également, comme on mange de la gazelle et du cerf.
\VS{23}Seulement, tu n'en mangeras point le sang ; mais tu le répandras sur la terre comme de l'eau.
\Chap{16}
\TextTitle{La Pâque et la fête des pains sans levain}
\VerseOne{}Observe le mois des épis, et fais la Pâque à Yahweh, ton Dieu ; car c'est au mois des épis que Yahweh, ton Dieu, t'a fait sortir, de nuit, d'Egypte\FTNT{Ex. 12:2-29.}.
\VS{2}Et tu sacrifieras la Pâque à Yahweh, ton Dieu, du gros et du menu bétail, au lieu que Yahweh choisira pour y faire habiter son Nom.
\VS{3}Tu ne mangeras point de pain levé, mais tu mangeras sept jours des pains sans levain, du pain d'affliction, parce que tu es sorti précipitamment du pays d'Egypte, afin que tous les jours de ta vie tu te souviennes du jour où tu es sorti du pays d'Egypte.
\VS{4}Et il ne se verra point de levain chez toi, sur tout le territoire de ton pays pendant sept jours\FTNT{1 Co. 5:7.} ; et aucune chair que tu sacrifieras le soir du premier jour ne restera jusqu'au matin.
\VS{5}Tu ne pourras point sacrifier la Pâque dans l'une de tes portes que Yahweh, ton Dieu, te donne ;
\VS{6}mais c'est au lieu que Yahweh, ton Dieu, choisira pour y faire habiter son Nom, que tu sacrifieras la Pâque, le soir, au coucher du soleil, moment où tu es sorti d'Egypte.
\VS{7}Tu la cuiras et tu la mangeras dans le lieu que Yahweh, ton Dieu, aura choisi. Et le matin, tu t'en retourneras et tu t'en iras dans tes tentes.
\VS{8}Pendant six jours, tu mangeras des pains sans levain ; et le septième jour, il y aura une assemblée solennelle à Yahweh, ton Dieu : Tu ne feras aucune œuvre.
\TextTitle{La fête des semaines}
\VS{9}Tu te compteras sept semaines ; tu commenceras à compter ces sept semaines dès que la faucille sera mise dans les blés.
\VS{10}Puis tu feras la fête des semaines à Yahweh, ton Dieu, en présentant l'offrande volontaire de ta main, que tu donneras, selon que Yahweh, ton Dieu, t'aura béni.
\VS{11}Et tu te réjouiras devant Yahweh, ton Dieu, toi, ton fils et ta fille, ton serviteur et ta servante, le Lévite qui sera dans tes portes, l'étranger, l'orphelin et la veuve qui seront au milieu de toi, dans le lieu que Yahweh, ton Dieu, aura choisi pour y faire habiter son Nom.
\VS{12}Et tu te souviendras que tu as été esclave en Egypte, et tu garderas et pratiqueras ces lois.
\TextTitle{La fête des tabernacles}
\VS{13}Tu feras la fête des tabernacles pendant sept jours, après que tu auras recueilli le produit de ton aire et de ton pressoir.
\VS{14}Et tu te réjouiras à cette fête, toi, ton fils et ta fille, ton serviteur et ta servante, le Lévite, l'étranger, l'orphelin, et la veuve qui seront dans tes portes.
\VS{15}Tu célébreras la fête pendant sept jours à Yahweh, ton Dieu, dans le lieu que Yahweh aura choisi ; car Yahweh, ton Dieu, te bénira dans toute ta récolte, et dans tout le travail de tes mains, et tu vivras dans la joie.
\TextTitle{Offrandes à Yahweh selon ses moyens}
\VS{16}Trois fois l'an, tout mâle d'entre vous se présentera devant Yahweh, ton Dieu, dans le lieu qu'il aura choisi, à la fête des pains sans levain, à la fête des semaines, et à la fête des tabernacles. On ne se présentera point devant Yahweh à vide.
\VS{17}Et chacun, selon ce que sa main peut donner, sera en proportion de la bénédiction que Yahweh, ton Dieu, lui aura accordées.
\TextTitle{Des juges établis pour faire respecter la justice de Yahweh}
\VS{18}Tu t'établiras des juges et des officiers dans toutes les villes que Yahweh, ton Dieu, te donne, selon tes tribus ; et ils jugeront le peuple d'un juste jugement.
\VS{19}Tu ne te détourneras point de la justice, tu ne prêteras point attention à l'apparence des personnes, et tu ne recevras point de présents, car les présents aveuglent les yeux des sages et corrompent les paroles des justes.
\VS{20}Tu suivras fermement la justice, afin que tu vives et que tu possèdes le pays que Yahweh, ton Dieu, te donne.
\TextTitle{Prescriptions sur les cultes}
\VS{21}Tu ne planteras point d'arbre d'Asherah\FTNT{Bois ou arbre d'Asherah : Il est question d'un objet en bois, pieu sacré ou arbre utilisé dans le culte d'Astarté, l'épouse de Baal (Ex. 34:13 ; De. 7:5 ; De. 12:3 ; Jg. 3:7 ; Jg. 6:25-30 ; 1 R. 14:15-23).}, près de l'autel que tu feras à Yahweh, ton Dieu.
\VS{22}Tu ne dresseras point non plus de statue ; Yahweh, ton Dieu, hait ces choses.
\Chap{17}
\VerseOne{}Tu ne sacrifieras à Yahweh, ton Dieu, ni boeuf, ni agneau qui ait quelque défaut ou quelque chose de mauvais ; car c'est une abomination à Yahweh, ton Dieu.
\TextTitle{Punition de l'idolâtrie}
\VS{2}S'il se trouve au milieu de toi dans l'une des villes que Yahweh, ton Dieu, te donne, un homme ou une femme faisant ce qui est mal aux yeux de Yahweh, ton Dieu, en transgressant son alliance,
\VS{3}et allant servir d'autres dieux et se prosterner devant eux, devant le soleil, devant la lune, ou devant toute l'armée des cieux, ce que je n'ai pas ordonné ;
\VS{4}et que cela t'aura été rapporté, et que tu l'auras entendu, alors tu feras des recherches avec soin. Si la chose est vraie, que le fait est établi, et que cette abomination a été commise en Israël,
\VS{5}alors tu feras sortir vers tes portes cet homme ou cette femme, qui aura fait cette mauvaise action, cet homme, dis-je, ou cette femme, et tu les lapideras avec des pierres, et ils mourront.
\VS{6}On fera mourir sur la parole de deux témoins ou de trois témoins\FTNT{Mt. 18:15-17.}, celui qui doit être mis à mort ; il ne sera pas mis à mort sur la parole d'un seul témoin.
\VS{7}La main des témoins sera la première sur lui pour le faire mourir, et ensuite la main de tout le peuple. Tu ôteras le mal du milieu de toi.
\TextTitle{Soumission aux autorités}
\VS{8}Quand une affaire te paraîtra trop difficile à juger entre meurtre et meurtre, entre cause et cause, entre plaie et plaie, qui sont des affaires de procès dans tes portes, alors tu te lèveras et tu monteras au lieu que Yahweh, ton Dieu, aura choisi.
\VS{9}Et tu iras vers les sacrificateurs, les Lévites, et vers le juge qu'il y aura en ce temps-là, tu les consulteras, et ils te feront connaître et te déclareront la sentence du jugement.
\VS{10}Tu feras conformément à la sentence qu'ils t'auront déclarée de leur bouche dans le lieu que Yahweh aura choisi, et tu prendras garde de faire tout ce qu'ils t'enseigneront.
\VS{11}Tu feras conformément à la loi qu'ils t'auront enseignée de leur bouche et selon la sentence qu'ils t'auront prononcée ; tu ne te détourneras ni à droite ni à gauche de ce qu'ils t'auront déclaré.
\VS{12}Mais l'homme qui agira par orgueil et n'obéira pas au sacrificateur qui se tient là pour servir Yahweh, ton Dieu, ou au juge, cet homme mourra. Tu ôteras le mal d'Israël,
\VS{13}et tout le peuple l'entendra et craindra, et n'agira plus par orgueil.
\TextTitle{Instructions sur la royauté}
\VS{14}Quand tu seras entré dans le pays que Yahweh, ton Dieu, te donne, que tu le posséderas, que tu y demeureras, et que tu diras : J'établirai un roi sur moi, comme toutes les nations qui sont autour de moi,
\VS{15}tu ne manqueras pas de t'établir pour roi celui que Yahweh, ton Dieu, aura choisi, tu établiras un roi du milieu de tes frères, tu ne pourras point désigner un homme étranger qui ne soit pas ton frère\FTNT{Dans sa prescience, Yahweh savait que le peuple se détournerait de ses voies et réclamerait un roi, à l'identique des nations alentour. (1 S. 8). Or depuis leur sortie d'Egypte, seul Yahweh était leur Dieu et leur Roi.}.
\VS{16}Seulement, il n'aura pas de nombreux chevaux, et il ne ramènera point le peuple en Egypte pour augmenter le nombre de chevaux ; car Yahweh vous a dit : Vous ne retournerez plus par ce chemin.
\VS{17}Il n'aura point un grand nombre de femmes, afin que son cœur ne se détourne point ; et qu'il n'accumule point beaucoup d'argent et d'or.
\VS{18}Et dès qu'il sera assis sur le trône de son royaume, il écrira pour lui, dans un livre, une copie de cette loi, qu'il prendra des sacrificateurs, les Lévites.
\VS{19}Il l'aura auprès de lui et la lira tous les jours de sa vie, afin qu'il apprenne à craindre Yahweh, son Dieu, à prendre garde à toutes les paroles de cette loi, et à ces ordonnances, afin de les pratiquer ;
\VS{20}afin que son cœur ne s'élève point au-dessus de ses frères, et qu'il ne se détourne point de ce commandement ni à droite ni à gauche ; afin qu'il prolonge ses jours dans son royaume, lui et ses fils, au milieu d'Israël.
\Chap{18}
\TextTitle{Héritage des Lévites et des sacrificateurs}
\VerseOne{}Les sacrificateurs, les Lévites, et même toute la tribu de Lévi, n'auront ni part ni héritage avec Israël ; ils mangeront les sacrifices consumés par le feu de Yahweh, et de son héritage.
\VS{2}Ils n'auront point d'héritage parmi leurs frères : Yahweh sera leur héritage, comme il leur a dit.
\VS{3}Or c'est ici le droit que les sacrificateurs prendront du peuple, sur ceux qui offriront un sacrifice, un bœuf ou un agneau : On donnera au sacrificateur l'épaule, les mâchoires et l'estomac.
\VS{4}Tu lui donneras les prémices de ton blé, de ton vin et de ton huile, et les prémices de la toison de tes brebis.
\VS{5}Car Yahweh, ton Dieu, l'a choisi d'entre toutes les tribus, afin qu'il se tienne devant lui, et qu'il fasse le service au Nom de Yahweh, lui et ses fils, à toujours.
\VS{6}Or quand le Lévite viendra de l'une de tes portes, de tout lieu où il habite en Israël, et qu'il viendra selon tout le désir de son âme, au lieu que Yahweh aura choisi,
\VS{7}et qu'il fera le service au Nom de Yahweh, son Dieu, comme tous ses frères Lévites qui se tiennent là devant Yahweh,
\VS{8}il mangera une portion égale à la leur, outre ce qu'il aura vendu de son patrimoine.
\TextTitle{Les abominations des nations interdites en Israël}
\VS{9}Quand tu seras entré dans le pays que Yahweh, ton Dieu, te donne, tu n'apprendras point à faire les abominations de ces nations-là.
\VS{10}Qu'on ne trouve au milieu de toi personne qui fasse passer par le feu son fils ou sa fille, personne qui pratique la divination, l'astrologie, l'augure, la sorcellerie,
\VS{11}ni d'enchanteur qui use d'enchantements, personne qui consulte les médiums ou disent la bonne aventure, personne qui interroge les morts\FTNT{Yahweh interdit tout contact avec le monde des esprits et des démons. Le croyant qui accepte l'Evangile comprendra sans peine et simplement en obéissant à la Parole que ce domaine est interdit. Voir Ex. 22:18 ; Lé. 19:26 ; Lé. 19:31 ; Lé. 20:6 ; Lé. 20:27 ; Es. 8:19 ; 2 Ch. 33:6 ; Ac. 19:13-20.}.
\VS{12}Car quiconque fait ces choses est en abomination à Yahweh ; et à cause de ces abominations, Yahweh, ton Dieu, va chasser ces nations-là devant toi.
\VS{13}Tu seras intègre avec Yahweh, ton Dieu.
\VS{14}Car ces nations, que tu vas déposséder, écoutent les pronostiqueurs et les devins ; mais à toi, Yahweh, ton Dieu, ne le permet point.
\TextTitle{Annonce du Messie}
\VS{15}Yahweh, ton Dieu, te suscitera du milieu de toi, d'entre tes frères, un prophète comme moi\FTNT{Moïse a annoncé la venue d'un prophète comme lui, c'est-à-dire un prophète de la délivrance et de l'exode. Ce prophète n'est autre que Jésus-Christ qui nous délivre de l'emprise de Satan et nous sort du monde pour nous amener dans la nouvelle Jérusalem (Jn. 14:2 ; Col. 1:13). Notons qu'au moment de la transfiguration, Elie et Moïse parlaient avec Jésus de son départ (« exodus » en grec ; Lu. 9:31).}: Vous l'écouterez.
\VS{16}Selon tout ce que tu as demandé à Yahweh, ton Dieu, à Horeb, le jour de l'assemblée, quand tu disais : Que je n'entende plus la voix de Yahweh, mon Dieu, et que je ne voie plus ce grand feu, de peur de mourir.
\VS{17}Alors Yahweh me dit : Ce qu'ils ont dit est bien.
\VS{18}Je leur susciterai un prophète comme toi du milieu de leurs frères, je mettrai mes paroles dans sa bouche, et il leur dira tout ce que je lui ordonnerai.
\VS{19}Et il arrivera que si un homme n'écoute pas mes paroles qu'il dira en mon Nom, je lui en demanderai compte.
\TextTitle{Comment éprouver les prophètes ?}
\VS{20}Mais le prophète qui agira de manière orgueilleuse pour dire en mon Nom une parole que je ne lui aurai point ordonné de dire, ou qui parlera au nom des autres dieux, ce prophète-là mourra.
\VS{21}Et si tu dis dans ton cœur : Comment connaîtrons-nous la parole que Yahweh n'aura point dite ?
\VS{22}Quand le prophète parlera au Nom de Yahweh, et que ce qu'il aura dit n'arrivera pas, ce sera une parole que Yahweh ne lui aura point dite. C'est par orgueil que le prophète l'a dite : N'aie point peur de lui.
\Chap{19}
\TextTitle{Les villes de refuge\FTNTT{No. 35:1-34}}
\VerseOne{}Quand Yahweh, ton Dieu, aura exterminé les nations dont Yahweh, ton Dieu, te donne le pays, et que tu les auras dépossédées et que tu demeureras dans leurs villes, et dans leurs maisons,
\VS{2}alors tu sépareras trois villes au milieu du pays que Yahweh, ton Dieu, te donne à posséder.
\VS{3}Tu établiras des chemins, et tu diviseras en trois le territoire de ton pays, que Yahweh, ton Dieu, te donnera en héritage. Ce sera afin que tout meurtrier s'y enfuie.
\VS{4}Or voici comment on procédera envers le meurtrier qui s'enfuira pour sauver sa vie. Celui qui aura frappé son prochain involontairement, et sans l'avoir haï dans le passé ;
\VS{5}ainsi, si quelqu'un va couper du bois dans la forêt avec une autre personne, la hache à la main pour couper du bois, si le fer glisse du manche, trouve son compagnon, et s'il en meurt ; il s'enfuira alors dans une de ces villes, afin qu'il vive.
\VS{6}De peur que celui qui venge le sang ne poursuive le meurtrier, parce que son cœur est échauffé, et qu'il ne le rattrape, si le chemin est trop long, et ne le frappe à mort, alors qu'il ne mérite pas la mort, parce qu'il ne le haïssait pas auparavant\FTNT{No. 35:1-34.}.
\VS{7}C'est pourquoi je t'ordonne, en disant : Sépare-toi trois villes.
\VS{8}Lorsque Yahweh, ton Dieu, aura élargi tes frontières, comme il l'a juré à tes pères, et qu'il t'aura donné tout le pays qu'il a promis à tes pères de te donner,
\VS{9}parce que tu auras gardé et mis en pratique tous ces commandements que je t'ordonne aujourd'hui, en aimant Yahweh, ton Dieu, et en marchant toujours dans ses voies, alors tu ajouteras encore trois villes à ces trois-là,
\VS{10}afin que le sang innocent ne soit versé au milieu du pays que Yahweh, ton Dieu, te donne en héritage, et que tu ne sois pas coupable de meurtre.
\VS{11}Mais si un homme hait son prochain, lui dresse un piège, se lève contre lui et frappe cette personne, de sorte qu'il meure, et qu'il s'enfuit dans l'une de ces villes,
\VS{12}alors les anciens de sa ville l'enverront saisir, et le livreront entre les mains du vengeur de sang, afin qu'il meure.
\VS{13}Ton œil ne l'épargnera point, mais tu feras disparaître d'Israël le sang innocent, et tu seras heureux.
\VS{14}Tu ne déplaceras point les bornes de ton prochain, fixées par tes ancêtres, dans l'héritage que tu posséderas, dans le pays que Yahweh, ton Dieu, te donne à posséder.
\TextTitle{Résoudre des différends}
\VS{15}Un seul témoin ne sera point valable contre un homme pour constater un crime ou un péché, quel que soit le péché ; mais sur la parole de deux témoins ou de trois témoins la chose sera valable.
\VS{16}Quand un faux témoin s'élèvera contre un homme pour témoigner contre lui d'un crime,
\VS{17}ces deux hommes en contestation comparaîtront devant Yahweh, en présence des sacrificateurs et des juges qui seront là en ce temps-là.
\VS{18}Et les juges feront des recherches avec soin. Si le témoin est un faux témoin, s'il a donné un faux témoignage contre son frère,
\VS{19}tu lui feras comme il avait pensé faire à son frère. Tu ôteras ainsi le mal du milieu de toi.
\VS{20}Et les autres entendront et craindront, et ne feront plus une chose aussi méchante au milieu de toi.
\VS{21}Ton œil ne l'épargnera point : Vie pour vie, œil pour œil, dent pour dent, main pour main, pied pour pied.
\Chap{20}
\TextTitle{Instructions diverses pour la guerre}
\VerseOne{}Quand tu iras à la guerre contre tes ennemis, et que tu verras des chevaux et des chars, et un peuple plus grand que toi, tu ne les craindras point, car Yahweh, ton Dieu, qui t'a fait monter du pays d'Egypte, est avec toi.
\VS{2}Et quand vous vous approcherez du combat, le sacrificateur s'avancera et parlera au peuple.
\VS{3}Et leur dira : Ecoute Israël : Vous vous approchez aujourd'hui pour combattre vos ennemis. Que votre cœur ne faiblisse pas ; ne craignez point, ne soyez point effrayés et ne soyez point terrifiés face à eux.
\VS{4}Car Yahweh, votre Dieu, marche avec vous, pour combattre vos ennemis, pour vous sauver.
\VS{5}Les officiers parleront au peuple, en disant : Qui est l'homme qui a bâti une maison neuve et ne l'a pas inaugurée ? Qu'il s'en aille et retourne dans sa maison, de peur qu'il ne meure dans la bataille et qu'un autre homme ne l'inaugure.
\VS{6}Qui est celui qui a planté une vigne et n'en a point encore cueilli le fruit ? Qu'il s'en aille et retourne dans sa maison, de peur qu'il ne meure dans la bataille et qu'un autre homme n'en cueille le fruit.
\VS{7}Qui est celui qui a fiancé une femme et ne l'a point prise en mariage ? Qu'il s'en aille et retourne dans sa maison, de peur qu'il ne meure dans la bataille et qu'un autre homme ne la prenne en mariage.
\VS{8}Et les officiers continueront à parler au peuple, et diront : Si un homme a peur et est timide, qu'il s'en aille et retourne dans sa maison, de peur que le cœur de ses frères ne devienne craintif comme le sien.
\VS{9}Quand les officiers auront fini de parler au peuple, ils désigneront les chefs des armées à la tête du peuple.
\VS{10}Quand tu t'approcheras d'une ville pour lui faire la guerre, tu l'inviteras à la paix.
\VS{11}Et si elle te donne une réponse de paix et s'ouvre à toi, tout le peuple qui s'y trouvera te sera tributaire et te servira.
\VS{12}Si elle ne fait pas la paix avec toi et qu'elle te fait la guerre, alors tu l'assiègeras.
\VS{13}Et quand Yahweh, ton Dieu, l'aura livrée entre tes mains, tu frapperas tous les mâles au fil de l'épée.
\VS{14}Mais les femmes, les enfants, le bétail, tout ce qui sera dans la ville, et tout son butin, tu le prendras pour toi et tu mangeras le butin de tes ennemis, que Yahweh, ton Dieu, t'aura donné.
\VS{15}Tu feras ainsi à toutes les villes qui sont très éloignées de toi, et qui ne sont point des villes de ces nations.
\VS{16}Mais dans les villes de ces peuples que Yahweh, ton Dieu, te donne en héritage, tu ne laisseras vivre personne qui respire.
\VS{17}Car tu ne manqueras point de les dévouer par interdit : Héthiens, Amoréens, Cananéens, Phéréziens, Héviens, et Jébusiens, comme Yahweh, ton Dieu, te l'a ordonné.
\VS{18}Afin qu'ils ne vous enseignent point à faire toutes les abominations qu'ils font pour leurs dieux, et que vous ne péchiez point contre Yahweh, votre Dieu.
\VS{19}Quand tu assiégeras une ville durant plusieurs jours, en lui faisant la guerre pour la saisir, tu ne détruiras point les arbres à coups de hache, tu t'en nourriras et tu ne les couperas point, car l'arbre des champs est-il un homme pour être assiégé par toi ?
\VS{20}Mais seulement tu détruiras et tu couperas les arbres que tu sauras ne point être des arbres fruitiers, et tu construiras des retranchements contre la ville qui te fait la guerre, jusqu'à ce qu'elle tombe.
\Chap{21}
\TextTitle{Lois sur le meurtre anonyme}
\VerseOne{}S'il se trouve sur la terre que Yahweh, ton Dieu, te donne à posséder, un homme tué, étendu dans un champ, sans que l'on sache qui l'a frappé,
\VS{2}tes anciens et tes juges sortiront, et ils mesureront de l'homme tué jusqu'aux villes qui sont autour.
\VS{3}Puis les anciens de la ville la plus proche de l'homme tué prendront une génisse du troupeau qui n'a pas travaillé et qui n'a point tiré au joug.
\VS{4}Et les anciens de cette ville feront descendre cette génisse vers un torrent intarissable, où on ne travaille ni ne sème ; et là, ils briseront la nuque à la génisse dans le torrent.
\VS{5}Et les sacrificateurs, fils de Lévi, s'approcheront ; car Yahweh, ton Dieu, les a choisis pour qu'ils le servent, et qu'ils bénissent au Nom de Yahweh ; et leur bouche doit décider de toute contestation et toute blessure.
\VS{6}Et tous les anciens de cette ville, qui seront les plus proche de l'homme qui aura été tué, laveront leurs mains sur la génisse à laquelle on aura brisé la nuque dans le torrent.
\VS{7}Et prenant la parole, ils diront : Nos mains n'ont point répandu ce sang et nos yeux ne l'ont point vu.
\VS{8}Ô Yahweh ! Sois propice à ton peuple d'Israël que tu as racheté ; ne lui impute point le sang innocent qui a été répandu au milieu de ton peuple d'Israël ; et le meurtre sera expié pour eux.
\VS{9}Et tu ôteras le sang innocent du milieu de toi, en faisant ce qui est droit aux yeux de Yahweh.
\TextTitle{Lois sur le mariage et l'héritage}
\VS{10}Quand tu iras en guerre contre tes ennemis, que Yahweh, ton Dieu, les aura livrés entre tes mains, et que tu en auras emmené des captifs,
\VS{11}si tu vois parmi les captifs une femme belle de figure, et que tu désires la prendre pour femme,
\VS{12}alors tu la conduiras dans l'intérieur de ta maison, et elle rasera sa tête et fera ses ongles,
\VS{13}elle ôtera les vêtements de sa captivité, elle demeurera dans ta maison, et pleurera son père et sa mère durant un mois. Puis tu iras vers elle, tu l'épouseras, et elle sera ta femme.
\VS{14}Si il arrive qu'elle ne te plaise plus, tu la renverras où elle voudra, mais tu ne la vendras certainement pas pour de l'argent ni la traiteras en esclave, parce que tu l'auras humiliée.
\VS{15}Quand un homme, qui a deux femmes, aime l'une et hait l'autre, si celle qu'il aime et celle qu'il hait enfantent des fils, et que le fils aîné est de celle qui est haïe,
\VS{16}alors, le jour où il laissera en héritage ce qu'il aura, il ne pourra pas reconnaître comme premier-né le fils de celle qu'il aime, à la place du fils de celle qui est haïe, et qui est le premier-né.
\VS{17}Mais il reconnaîtra pour premier-né le fils de celle qui est haïe, et il lui donnera la double portion de tout ce qui s'y trouvera être à lui ; car il est le commencement de sa vigueur, le droit d'aînesse lui appartient.
\TextTitle{Le fils indocile sous la loi\FTNTT{cp. Lu. 15:11-23}}
\VS{18}Si un homme a un fils indocile et rebelle, n'obéissant point à la voix de son père, ni à la voix de sa mère, et qui, bien qu'ils l'aient châtié, ne les écoute point,
\VS{19}alors le père et la mère le prendront et le mèneront aux anciens de sa ville, et à la porte du lieu de sa demeure.
\VS{20}Et ils diront aux anciens de sa ville : Voici notre fils qui est indocile et rebelle, qui n'obéit point à notre voix, et qui se livre à l'excès et à l'ivrognerie.
\VS{21}Et tous les gens de la ville le lapideront avec des pierres, et il mourra. Tu ôteras le mal du milieu de toi, afin que tout Israël entende et craigne.
\VS{22}Si un homme a commis un péché digne de mort, et qu'on le fait mourir, et que tu l'aies pendu à un bois,
\VS{23}son cadavre ne passera point la nuit sur le bois ; mais tu ne manqueras point de l'ensevelir le même jour, car celui qui est pendu est malédiction de Dieu\FTNT{Ga. 3:13.}, et tu ne souilleras point la terre que Yahweh, ton Dieu, te donne en héritage.
\Chap{22}
\TextTitle{Lois sur la vie en société}
\VerseOne{}Si tu vois le bœuf ou la brebis de ton frère s'égarer, tu ne te cacheras point, tu ne manqueras point de les ramener à ton frère.
\VS{2}Si ton frère ne demeure point près de toi, et que tu ne le connais point, tu les recueilleras dans ta maison et il sera chez toi jusqu'à ce que ton frère les cherche ; et alors tu les lui rendras.
\VS{3}Tu feras de même pour son âne, tu feras de même pour son vêtement, et tu feras de même pour tout ce que ton frère aura perdu et que tu trouveras ; tu ne devras point t'en détourner.
\VS{4}Si tu vois l'âne de ton frère ou son bœuf tombé dans le chemin, tu ne t'en détourneras point, et tu ne manqueras point de le relever.
\VS{5}La femme ne portera point l'habit d'un homme ni l'homme ne se vêtira point d'un habit de femme ; car celui qui fait ces choses est en abomination à Yahweh, ton Dieu\FTNT{Dans ce passage, Yahweh condamne le travestisme. Cette pratique était répandue chez les Cananéens. Le travestisme consiste à adopter le comportement, les habitudes sociales et la tenue vestimentaire du sexe opposé dans le but de lui ressembler.}.
\VS{6}Si tu rencontres sur le chemin, sur un arbre ou sur la terre, un nid d'oiseaux, ayant des petits ou des œufs, et la mère couchée sur les petits ou les œufs, tu ne prendras point la mère et les petits,
\VS{7}mais tu ne manqueras point de laisser aller la mère et tu ne prendras que les petits, afin que tu sois heureux et que tu prolonges tes jours.
\VS{8}Si tu bâtis une maison neuve, tu feras un parapet tout autour de ton toit, afin que tu ne mettes point de sang sur ta maison, si quelqu'un tombait de là.
\TextTitle{Lois sur les mélanges}
\VS{9}Tu ne sèmeras point dans ta vigne deux sortes de semences, de peur que tout le produit de la semence que tu auras semé et le produit de ta vigne ne soient mis à part.
\VS{10}Tu ne laboureras point avec un âne et un bœuf ensemble.
\VS{11}Tu ne te vêtiras point d'un tissu mélangé de laine et de lin ensemble.
\VS{12}Tu te feras des franges aux quatre pans du vêtement dont tu te couvriras.
\TextTitle{Lois sur la virginité, l'adultère et la fidélité}
\VS{13}Si un homme a pris une femme et est allé vers elle, et qu'il la haïsse,
\VS{14}et qui lui impute des choses qui donnent l'occasion de parler d'elle et de la diffamer, en disant : J'ai pris cette femme, et quand je me suis approché d'elle, je ne l'ai point trouvé vierge,
\VS{15}alors le père et la mère de la jeune femme prendront et produiront les signes de la virginité de la jeune femme devant les anciens de la ville, à la porte.
\VS{16}Et le père de la jeune femme dira aux anciens : J'ai donné ma fille à cet homme pour femme, et il l'a haïe ;
\VS{17}et voici, il lui impute des choses qui lui donnent l'occasion de parler d'elle, disant : Je n'ai point trouvé ta fille vierge. Cependant, voici les signes de la virginité de ma fille. Et ils étendront le drap devant les anciens de la ville.
\VS{18}Alors les anciens de la ville prendront le mari, et le châtieront ;
\VS{19}et parce qu'il aura répandu une mauvaise réputation sur une vierge d'Israël, ils le condamneront à une amende de cent sicles d'argent, qu'ils donneront au père de la jeune femme. Elle sera sa femme, et il ne pourra pas la répudier, tant qu'il vivra.
\VS{20}Mais si la chose est vraie, si la jeune femme ne s'est point trouvée vierge,
\VS{21}alors ils feront sortir la jeune femme à l'entrée de la maison de son père ; les gens de sa ville la lapideront de pierres et elle mourra, car elle a commis une infamie en Israël, en se prostituant dans la maison de son père. Tu ôteras le mal du milieu de toi.
\VS{22}Si l'on trouve un homme couché avec une femme mariée, ils mourront tous les deux, l'homme qui a couché avec la femme, et la femme aussi. Tu ôteras ainsi le mal d'Israël.
\VS{23}Si une jeune fille vierge est fiancée à un homme, et qu'un homme la rencontre dans la ville, et couche avec elle,
\VS{24}vous les conduirez tous deux à la porte de la ville, vous les lapiderez de pierres, et ils mourront ; la jeune fille, parce qu'elle n'a point crié étant dans la ville, et l'homme parce qu'il a humilié la femme de son prochain. Tu ôteras le mal du milieu de toi.
\VS{25}Si l'homme rencontre dans les champs la jeune fille fiancée, et que l'homme lui fait violence et couche avec elle, alors l'homme qui aura couché avec elle mourra lui seul.
\VS{26}Mais tu ne feras rien à la jeune fille ; la jeune fille n'a point commis de péché digne de mort, car c'est comme si un homme s'élevait contre son prochain et lui ôtait la vie.
\VS{27}Parce que l'ayant trouvée dans les champs, la jeune fille fiancée a pu crier, sans que personne ne l'ait délivrée.
\VS{28}Si un homme rencontre une jeune fille vierge non fiancée, lui fait violence et couche avec elle, et qu'ils soient découverts,
\VS{29}l'homme qui aura couché avec elle donnera au père de la jeune fille cinquante sicles d'argent ; et il la prendra pour femme, parce qu'il l'a humiliée, et il ne pourra point la répudier, tant qu'il vivra.
\VS{30}Un homme ne prendra point la femme de son père ni ne découvrira le pan de la robe de son père.
\Chap{23}
\TextTitle{Lois traitant de cas particuliers}
\VerseOne{}Celui dont les testicules ont été écrasés ou l'urètre coupé n'entrera point dans l'assemblée de Yahweh.
\VS{2}Le bâtard n'entrera point dans l'assemblée de Yahweh ; même sa dixième génération n'entrera point dans l'assemblée de Yahweh.
\VS{3}L'Ammonite et le Moabite n'entreront point dans l'assemblée de Yahweh, même leur dixième génération, à jamais,
\VS{4}parce qu'ils ne sont point venus à votre rencontre avec du pain et de l'eau, sur le chemin, lorsque vous sortiez d'Egypte, et parce qu'ils ont engagé à prix d'argent contre vous Balaam, fils de Beor, de Pethor en Mésopotamie, pour qu'il vous maudisse.
\VS{5}Mais Yahweh, ton Dieu, n'a point voulu écouter Balaam ; et Yahweh, ton Dieu, a changé la malédiction en bénédiction, parce que Yahweh, ton Dieu, t'aime.
\VS{6}Tu ne chercheras jamais, tant que tu vivras, leur paix ni leur bien.
\VS{7}Tu n'auras point en abomination l'Edomite, car il est ton frère ; tu n'auras point en abomination l'Egyptien, car tu as été étranger dans son pays :
\VS{8}Les fils qui leur naîtront à la troisième génération entreront dans l'assemblée de Yahweh.
\TextTitle{La sainteté et la justice dans le camp de Yahweh}
\VS{9}Quand le camp sortira contre tes ennemis, garde-toi de toute chose mauvaise.
\VS{10}S'il y a parmi vous un homme qui ne soit point pur, par suite d'un accident nocturne, il sortira hors du camp, et n'entrera point dans le camp.
\VS{11}Et sur le soir, il se lavera dans l'eau, et dès que le soleil sera couché, il rentrera dans le camp.
\VS{12}Tu auras un endroit hors du camp, et tu sortiras là dehors.
\VS{13}Tu auras un pieu parmi tes bagages, et quand tu voudras aller dehors, tu creuseras, puis tu recouvriras tes excréments.
\VS{14}Car Yahweh, ton Dieu, marche au milieu de ton camp pour te délivrer et pour livrer tes ennemis devant toi ; que tout ton camp soit saint, afin qu'il ne voie chez toi aucune chose honteuse, et qu'il ne se détourne point de toi.
\VS{15}Tu ne livreras point à son maître l'esclave qui se sera sauvé chez toi d'auprès de son maître.
\VS{16}Il demeurera avec toi, au milieu de toi, dans le lieu qu'il choisira, dans l'une de tes villes, là où bon lui semblera : Tu ne l'opprimeras point.
\VS{17}Il n'y aura, parmi les filles d'Israël, aucune prostituée, et il n'y aura, parmi les fils d'Israël, aucun qui se prostitue.
\VS{18}Tu n'apporteras point dans la maison de Yahweh, ton Dieu, le salaire d'une prostituée, ni le prix d'un chien, pour quelque voeu que ce soit ; car tous les deux sont en abomination devant Yahweh, ton Dieu.
\VS{19}Tu n'exigeras aucun intérêt à ton frère, ni intérêt pour de l'argent, ni intérêt pour des vivres, ni intérêt pour quelque chose que ce soit que l'on prête avec intérêt.
\VS{20}Tu prêteras avec intérêt à l'étranger, mais tu ne prêteras point avec intérêt à ton frère, afin que Yahweh, ton Dieu, te bénisse dans tout ce que ta main entreprendra dans le pays où tu vas entrer en possession.
\TextTitle{Voeux faits à Yahweh}
\VS{21}Si tu fais un vœu à Yahweh, ton Dieu, tu ne tarderas point à l'accomplir, car Yahweh, ton Dieu, ne manquerait point de te le redemander, ainsi il y aurait du péché en toi.
\VS{22}Mais si tu t'abstiens de faire un vœu, il n'y aura pas de péché en toi.
\VS{23}Mais tu prendras garde de faire ce qui sortira de tes lèvres, l'offrande volontaire que tu auras vouée à Yahweh, ton Dieu, et que ta bouche aura prononcée.
\TextTitle{Lois diverses}
\VS{24}Si tu entres dans la vigne de ton prochain, tu pourras manger des raisins selon ton appétit, jusqu'à en être rassasié ; mais tu n'en mettras point dans ton vase.
\VS{25}Si tu entres dans les blés de ton prochain, tu pourras arracher des épis avec ta main ; mais tu n'agiteras point la faucille sur les blés de ton prochain.
\Chap{24}
\TextTitle{Loi sur le divorce}
\VerseOne{}Quand un homme aura pris et épousé une femme, s'il arrive qu'elle ne trouve pas grâce à ses yeux, parce qu'il a trouvé en elle quelque chose de honteux, il lui écrira une lettre de divorce, et après la lui avoir remise en main, il la renverra de sa maison.
\VS{2}Elle sortira de sa maison, s'en ira, et elle pourra devenir la femme d'un autre homme.
\VS{3}Si ce dernier homme la hait, écrit une lettre de divorce, la lui donne dans sa main, et la renvoie de sa maison, ou que ce dernier homme qui l'a prise pour femme, meure,
\VS{4}alors son premier mari qui l'avait renvoyée ne pourra pas la reprendre pour femme après avoir été souillée, car c'est une abomination devant Yahweh, ainsi tu ne feras point pécher le pays que Yahweh, ton Dieu, te donne en héritage.
\TextTitle{Lois diverses sur l'organisation de la société}
\VS{5}Quand un homme aura nouvellement épousé une femme, il n'ira point à l'armée, et on ne lui imposera aucune charge ; il en sera libre pour sa maison pendant un an, et il réjouira la femme qu'il a prise.
\VS{6}On ne prendra point pour gage les deux meules, pas même la meule de dessus ; parce qu'on prendrait pour gage la vie.
\VS{7}Si l'on trouve un homme qui ait dérobé l'un de ses frères, l'un des fils d'Israël, qui en ait fait son esclave ou qui l'ait vendu, ce voleur mourra. Tu ôteras le mal du milieu de toi.
\VS{8}Prends garde à la plaie de la lèpre, afin de bien observer et de faire tout ce que les sacrificateurs, les Lévites, vous enseigneront ; vous prendrez garde de faire selon ce que je leur ai ordonné.
\VS{9}Souviens-toi de ce que Yahweh, ton Dieu, fit à Marie, en chemin, après votre sortie d'Egypte.
\TextTitle{Lois en faveur des nécessiteux}
\VS{10}Lorsque tu feras à ton prochain un prêt quelconque, tu n'entreras point dans sa maison pour prendre son gage ;
\VS{11}mais tu te tiendras dehors, et l'homme à qui tu feras le prêt t'apportera le gage dehors.
\VS{12}Si cet homme est pauvre, tu ne te coucheras point ayant encore son gage ;
\VS{13}tu ne manqueras point de lui rendre le gage dès que le soleil sera couché, afin qu'il se couche dans son vêtement et qu'il te bénisse ; et cela te sera imputé à justice devant Yahweh, ton Dieu.
\VS{14}Tu n'opprimeras point le mercenaire, le pauvre et l'indigent, d'entre tes frères, ou d'entre les étrangers qui demeurent dans ton pays, dans tes portes.
\VS{15}Tu lui donneras son salaire le jour même avant que le soleil se couche ; car il est pauvre, et son désir s'y porte. Afin qu'il ne crie point contre toi à Yahweh, et que tu ne pèches point.
\VS{16}On ne fera point mourir les pères pour les fils, et on ne fera point mourir les fils pour les pères ; mais on fera mourir chacun pour son péché.
\VS{17}Tu ne feras pas d'injustice à l'étranger ni àl'orphelin, et tu ne prendras point en gage le vêtement de la veuve.
\VS{18}Et tu te souviendras que tu as été esclave en Egypte, et que Yahweh, ton Dieu, t'a racheté de là ; c'est pourquoi je t'ordonne de faire ces choses.
\VS{19}Quand tu moissonneras dans ton champ, et que tu auras oublié une gerbe dans ton champ, tu ne retourneras point la prendre : Elle sera pour l'étranger, pour l'orphelin et pour la veuve, afin que Yahweh, ton Dieu, te bénisse dans toute l'œuvre de tes mains.
\VS{20}Quand tu secoueras tes oliviers, tu n'y retourneras point pour cueillir ce qui reste aux branches : Ce sera pour l'étranger, pour l'orphelin et pour la veuve.
\VS{21}Quand tu vendangeras ta vigne, tu ne grappilleras point après : Ce sera pour l'étranger, pour l'orphelin et pour la veuve.
\VS{22}Et tu te souviendras que tu as été esclave dans le pays d'Egypte ; c'est pourquoi je t'ordonne de faire ces choses.
\Chap{25}
\TextTitle{Le juste justifié et le méchant condamné}
\VerseOne{}Quand il y aura un différend entre des hommes et qu'ils viendront en jugement afin qu'on les juge, on justifiera le juste, et on condamnera le méchant.
\VS{2}Si le méchant mérite d'être battu, le juge le fera jeter par terre et frapper en sa présence par un certain nombre de coups, selon l'exigence de son crime.
\VS{3}Il le fera battre de quarante coups, pas plus, de peur que si l'on continuait à le frapper avec plus de coups, ton frère ne soit méprisé à tes yeux.
\VS{4}Tu n'emmuselleras point ton bœuf lorsqu'il foulera le grain.
\TextTitle{Loi sur la continuité de la postérité}
\VS{5}Quand des frères demeureront ensemble, et que l'un d'entre eux mourra sans fils, alors la femme du défunt ne se mariera point dehors avec un homme qui est étranger, mais son beau-frère viendra vers elle, la prendra pour femme, et l'épousera comme son beau-frère.
\VS{6}Et le premier-né qu'elle enfantera succédera au frère mort et portera son nom, afin que son nom ne soit point effacé d'Israël.
\VS{7}Et s'il ne plaît pas à cet homme-là de prendre sa belle-sœur, alors sa belle-sœur montera à la porte vers les anciens\FTNT{Ru. 4:1-10}, et dira : Mon beau-frère refuse de relever le nom de son frère en Israël, et ne veut point m'épouser par droit de beau-frère.
\VS{8}Alors les anciens de la ville l'appelleront et lui parleront. S'il demeure ferme, et qu'il dit : Il ne me plaît point de la prendre,
\VS{9}alors sa belle-sœur s'approchera de lui à la vue des anciens, lui ôtera son soulier du pied, et lui crachera au visage. Et prenant la parole, elle dira : C'est ainsi qu'on fera à l'homme qui ne bâtit point la maison de son frère.
\VS{10}Et son nom sera appelé en Israël la maison du déchaussé.
\TextTitle{L'abomination sévèrement et justement punie}
\VS{11}Quand des hommes se querelleront ensemble, l'un contre l'autre, si la femme de l'un s'approche pour délivrer son mari de la main de celui qui le frappe, et qu'étendant sa main elle saisisse ses parties intimes,
\VS{12}tu lui couperas la main, et ton œil ne l'épargnera point.
\VS{13}Tu n'auras point dans ton sac deux poids différents, un grand et un petit.
\VS{14}Il n'y aura point dans ta maison deux épha différents, un grand et un petit\FTNT{Lé. 19:35-37.}.
\VS{15}Mais tu auras un poids exact et juste, tu auras un épha exact et juste, afin que tes jours se prolongent sur la terre que Yahweh, ton Dieu, te donne.
\VS{16}Car celui qui fait ces choses, celui qui commet une injustice, est en abomination à Yahweh, ton Dieu.
\TextTitle{Yahweh confirme le sort d'Amalek}
\VS{17}Souviens-toi ce que te fit Amalek en chemin, quand vous sortiez d'Egypte\FTNT{Ex. 17:8.},
\VS{18}comment il est venu te rencontrer sur le chemin, et, sans aucune crainte de Dieu, attaqua par derrière ceux qui étaient fatigués, quand toi-même tu étais épuisé.
\VS{19}Quand Yahweh, ton Dieu, t'aura accordé du repos de tous tes ennemis qui t'entourent, dans le pays que Yahweh, ton Dieu, te donne en héritage afin que tu le possèdes, alors tu effaceras la mémoire d'Amalek de dessous les cieux : Ne l'oublie point.
\Chap{26}
\TextTitle{La loi des prémices\FTNTT{cp. Ex. 23:16-19}}
\VerseOne{}Lorsque tu seras entré dans le pays que Yahweh, ton Dieu, te donne en héritage, et quand tu le posséderas et y habiteras,
\VS{2}alors tu prendras des prémices de tous les fruits que tu retireras de la terre dans le pays que Yahweh, ton Dieu, te donne ; tu les mettras dans une corbeille, et tu iras au lieu que Yahweh, ton Dieu, choisira pour y faire habiter son Nom\FTNT{Ex. 23:16-19.}.
\VS{3}Tu viendras vers le sacrificateur qui sera là en ce temps-là, et tu lui diras : Je déclare aujourd'hui à Yahweh, ton Dieu, que je suis entré dans le pays que Yahweh a juré à nos pères de nous donner.
\VS{4}Et le sacrificateur prendra la corbeille de ta main, et la posera devant l'autel de Yahweh, ton Dieu.
\VS{5}Puis tu prendras la parole, et tu diras devant Yahweh, ton Dieu : Mon père était un Araméen qui périssait, il descendit en Egypte avec un petit nombre de gens, il y séjourna et il y devint une nation grande, puissante, et nombreuse.
\VS{6}Puis les Egyptiens nous maltraitèrent, nous humilièrent, et nous imposèrent une dure servitude.
\VS{7}Nous criâmes à Yahweh, le Dieu de nos pères. Yahweh entendit notre voix, et il vit notre souffrance, notre travail, et notre oppression.
\VS{8}Yahweh nous fit sortir d'Egypte, à main forte et à bras étendu, avec une grande frayeur, avec des signes et des miracles.
\VS{9}Et il nous a conduits dans ce lieu, et nous a donné ce pays où coulent le lait et le miel.
\VS{10}Maintenant donc voici, j'apporte les prémices des fruits de la terre que tu m'as donnée, ô Yahweh ! Tu les poseras devant Yahweh, ton Dieu, et tu te prosterneras devant Yahweh, ton Dieu.
\VS{11}Et tu te réjouiras de tout le bien que Yahweh ton Dieu t'aura donné, et à ta maison, toi et le Lévite, et l'étranger qui sera au milieu de toi.
\VS{12}Quand tu auras achever de lever toute la dîme de ta récolte, la troisième année, l'année de la dîme, tu la donneras au Lévite, à l'étranger, à l'orphelin, et à la veuve ; ils en mangeront dans tes portes, et ils en seront rassasiés.
\VS{13}Tu diras en la présence de Yahweh, ton Dieu : J'ai fait disparaître de ma maison ce qui est consacré, et je l'ai donné au Lévite, à l'étranger, à l'orphelin, et à la veuve, selon tous tes commandements que tu m'as ordonnés ; je n'ai transgressé ni oublié aucun de tes commandements.
\VS{14}Je n'en ai point mangé dans mon affliction, et je n'en ai rien fait disparaître pour un usage impur, et je n'en ai point donné pour un mort ; j'ai obéi à la voix de Yahweh, mon Dieu ; j'ai fait selon tout ce que tu m'avais ordonné.
\VS{15}Regarde de ta sainte demeure, des cieux, et bénis ton peuple d'Israël et la terre que tu nous as donnée, comme tu l'avais juré à nos pères, pays où coulent le lait et le miel.
\VS{16}Aujourd'hui, Yahweh, ton Dieu, t'ordonne de faire ces lois et ces ordonnances ; prends garde de les faire de tout ton cœur et de toute ton âme.
\VS{17}Aujourd'hui, tu as fait promettre à Yahweh qu'il sera ton Dieu, pour que tu marches dans ses voies, que tu observes ses lois, ses commandements et ses ordonnances, et que tu obéisses à sa voix.
\VS{18}Et aujourd'hui, Yahweh t'a fait promettre que tu seras un peuple précieux, comme il te l'a dit, et que tu observeras tous ses commandements,
\VS{19}pour qu'il te donne sur toutes les nations qu'il a créées la supériorité en louange, en renommée, et en beauté, et pour que tu sois un peuple saint à Yahweh, ton Dieu, comme il te l'a dit.
\Chap{27}
\TextTitle{La loi gravée sur des pierres au mont Ebal}
\VerseOne{}Moïse et les anciens d'Israël ordonnèrent au peuple, en disant : Gardez tous les commandements que je vous ordonne aujourd'hui.
\VS{2}Le jour où vous aurez traversé le Jourdain, pour entrer dans le pays que Yahweh, ton Dieu, te donne, tu dresseras de grandes pierres, et tu les enduiras de chaux.
\VS{3}Tu y écriras toutes les paroles de cette loi, quand tu auras traversé le Jourdain, pour entrer dans le pays que Yahweh, ton Dieu, te donne, pays où coulent le lait et le miel, comme te l'a dit Yahweh, le Dieu de tes pères.
\VS{4}Lorsque vous aurez traversé le Jourdain, vous dresserez ces pierres sur le mont Ebal, selon ce que je vous ordonne aujourd'hui, et tu les enduiras de chaux.
\VS{5}Là, tu y bâtiras un autel à Yahweh, ton Dieu ; un autel de pierres, sur lesquelles tu ne lèveras point le fer.
\VS{6}Tu bâtiras l'autel de Yahweh, ton Dieu, de pierres entières. Tu y offriras des holocaustes à Yahweh, ton Dieu ;
\VS{7}tu y offriras aussi des offrandes paix, et tu mangeras là et te réjouiras devant Yahweh, ton Dieu.
\VS{8}Tu écriras sur ces pierres toutes les paroles de cette loi, en les gravant bien distinctement.
\TextTitle{Mise en garde de Yahweh à son peuple: Bénédictions et malédictions}
\VS{9}Moïse et les sacrificateurs, les Lévites, parlèrent à tout Israël, en disant : Ecoute et garde le silence, Israël ! Aujourd'hui, tu es devenu le peuple de Yahweh, ton Dieu.
\VS{10}Tu obéiras à la voix de Yahweh, ton Dieu, et tu feras ses commandements et ses lois que je t'ordonne aujourd'hui.
\VS{11}Moïse ordonna au peuple ce jour-là, disant :
\VS{12}Lorsque vous aurez traversé le Jourdain, Siméon, Lévi, Juda, Issacar, Joseph, et Benjamin, se tiendront sur le mont Garizim, pour bénir le peuple ;
\VS{13}et Ruben, Gad, Aser, Zabulon, Dan et Nephthali, se tiendront sur le mont Ebal, pour maudire.
\VS{14}Et les Lévites prendront la parole, et diront à haute voix à tous les hommes d'Israël :
\VS{15}Maudit soit l'homme qui fait une image taillée ou une image en métal fondu, car c'est une abomination à Yahweh, œuvre des mains d'un artisan, et qui la met dans un lieu secret ! Et tout le peuple répondra et dira : Amen !
\VS{16}Maudit soit celui qui méprise son père et sa mère ! Et tout le peuple dira : Amen !
\VS{17}Maudit soit celui qui déplace les bornes de son prochain ! Et tout le peuple dira : Amen !
\VS{18}Maudit soit celui qui égare un aveugle dans le chemin ! Et tout le peuple dira : Amen !
\VS{19}Maudit soit celui qui porte atteinte au droit de l'étranger, de l'orphelin, et de la veuve ! Et tout le peuple dira : Amen !
\VS{20}Maudit soit celui qui couche avec la femme de son père, car il découvre le pan de la robe de son père ! Et tout le peuple dira : Amen !
\VS{21}Maudit soit celui qui couche avec une bête ! Et tout le peuple dira : Amen !
\VS{22}Maudit soit celui qui couche avec sa sœur, fille de son père, ou fille de sa mère ! Et tout le peuple dira : Amen !
\VS{23}Maudit soit celui qui couche avec sa belle-mère ! Et tout le peuple dira : Amen !
\VS{24}Maudit soit celui qui frappe son prochain en secret ! Et tout le peuple dira : Amen !
\VS{25}Maudit soit celui qui reçoit un présent pour mettre à mort un homme, en versant le sang innocent ! Et tout le peuple dira : Amen !
\VS{26}Maudit soit celui qui n'accomplit point les paroles de cette loi et ne les met pas en pratique ! Et tout le peuple dira : Amen !
\Chap{28}
\TextTitle{Les bénédictions qui accompagnent l'obéissance}
\VerseOne{}Si tu écoutes attentivement la voix de Yahweh, ton Dieu, et que tu prennes garde de pratiquer tous ses commandements que je t'ordonne aujourd'hui, Yahweh, ton Dieu, te donnera la supériorité sur toutes les nations de la terre.
\VS{2}Voici toutes les bénédictions qui viendront sur toi, et qui t'atteindront, quand tu obéiras à la voix de Yahweh, ton Dieu :
\VS{3}Tu seras béni dans la ville, et tu seras aussi béni aux champs.
\VS{4}Le fruit de tes entrailles, le fruit de ta terre, le fruit de tes troupeaux, les portées de ton gros et de ton menu bétail seront bénis.
\VS{5}Ta corbeille et ta huche seront bénies.
\VS{6}Tu seras béni en entrant, et tu seras béni en sortant.
\VS{7}Yahweh te donnera la victoire sur tes ennemis qui s'élèveront contre toi, ils sortiront contre toi par un chemin, et ils s'enfuiront devant toi par sept chemins.
\VS{8}Yahweh ordonnera à la bénédiction d'être avec toi dans tes greniers et dans tout ce que tes mains entreprendront. Il te bénira dans le pays que Yahweh, ton Dieu, te donne.
\VS{9}Yahweh t'établira pour lui être un peuple saint, comme il te l'a juré, lorsque tu garderas les commandements de Yahweh, ton Dieu, et que tu marcheras dans ses voies.
\VS{10}Tous les peuples de la terre verront que tu es appelé du Nom de Yahweh, et ils te craindront.
\VS{11}Yahweh te fera abonder de biens dans le fruit de tes entrailles, le fruit de tes troupeaux, et le fruit de ton sol, sur la terre que Yahweh a juré à tes pères de te donner.
\VS{12}Yahweh t'ouvrira son bon trésor, les cieux, pour donner à ton pays la pluie en sa saison et pour bénir tout le travail de tes mains ; tu prêteras à beaucoup de nations, et tu n'emprunteras point.
\VS{13}Yahweh te mettra à la tête et non à la queue, tu seras toujours en haut et jamais en bas, lorsque tu obéiras aux commandements de Yahweh, ton Dieu, que je t'ordonne aujourd'hui, afin que tu prennes garde de les faire,
\VS{14}et que tu ne te détournes ni à droite ni à gauche de toutes les paroles que je t'ordonne aujourd'hui, pour aller après d'autres dieux et pour les servir.
\TextTitle{Les malédictions qui suivent la désobéissance}
\VS{15}Mais si tu n'obéis point à la voix de Yahweh, ton Dieu, pour prendre garde de faire tous ses commandements et ses lois que je t'ordonne aujourd'hui, voici toutes les malédictions qui viendront sur toi, et qui t'atteindront :
\VS{16}Tu seras maudit dans la ville, et tu seras maudit dans les champs.
\VS{17}Ta corbeille et ta huche seront maudites.
\VS{18}Le fruit de tes entrailles, le fruit de ta terre, les portées de ton gros et de ton menu bétail seront maudits.
\VS{19}Tu seras maudit à ton entrée, et tu seras maudit à ta sortie.
\VS{20}Yahweh enverra sur toi la malédiction, la confusion, et la ruine dans tout ce que tes mains entreprendront et feront, jusqu'à ce que tu sois détruit, et que tu périsses promptement, à cause de la méchanceté de tes pratiques, qui t'auront amené à m'abandonner.
\VS{21}Yahweh attachera à toi la peste, jusqu'à ce qu'elle te consume sur la terre où tu vas entrer pour en prendre possession.
\VS{22}Yahweh te frappera de consomption, de fièvre, d'inflammation, de chaleur brûlante, de l'épée, de sécheresse et de nielle, qui te poursuivront jusqu'à ce que tu périsses.
\VS{23}Les cieux sur ta tête seront d'airain, et la terre sous toi sera de fer.
\VS{24}Yahweh te donnera pour pluie à ton pays de la poussière et de la poudre, qui descendra des cieux sur toi jusqu'à ce que tu sois détruit.
\VS{25}Yahweh fera que tu seras battu devant tes ennemis ; tu sortiras par un chemin contre eux, et tu t'enfuiras devant eux par sept chemins ; et tu seras tremblant face à tous les royaumes de la terre.
\VS{26}Ton cadavre sera la nourriture de tous les oiseaux des cieux et des bêtes de la terre ; et il n'y aura personne pour les effrayer.
\VS{27}Yahweh te frappera de l'ulcère d'Egypte, d'hémorroïdes, de gale, et de teigne, dont tu ne pourras guérir.
\VS{28}Yahweh te frappera de folie, d'aveuglement, et d'égarement d'esprit ;
\VS{29}et tu tâtonneras en plein midi comme tâtonne un aveugle dans l'obscurité, tu ne prospèreras pas dans tes voies, et tu seras opprimé et dépouillé tous les jours, et il n'y aura personne pour venir te sauver.
\VS{30}Tu fianceras une femme, mais un autre homme couchera avec elle et la violera ; tu bâtiras une maison, mais tu ne l'habiteras point ; tu planteras une vigne, mais tu n'en jouiras point.
\VS{31}Ton bœuf sera tué sous tes yeux, et tu n'en mangeras point ; ton âne sera enlevé devant toi, et on ne te le rendra point ; tes brebis seront données à tes ennemis, et il n'y aura personne pour te sauver.
\VS{32}Tes fils et tes filles seront livrés à un autre peuple, tes yeux le verront, et languiront tout le jour après eux, et ta main sera sans puissance.
\VS{33}Un peuple que tu n'auras point connu mangera le fruit de ta terre et tout ton travail, et tu seras opprimé et écrasé tous les jours.
\VS{34}Tu deviendras fou par ce que tu verras de tes yeux.
\VS{35}Yahweh te frappera d'un ulcère malin sur les genoux et sur les jambes dont tu ne pourras guérir, il t'en frappera depuis la plante du pied jusqu'au sommet de ta tête.
\VS{36}Yahweh te fera marcher, toi et ton roi que tu auras établi sur toi, vers une nation que tu n'auras point connue, ni toi ni tes pères. Et là, tu serviras d'autres dieux, du bois et de la pierre.
\VS{37}Et tu seras un sujet d'étonnement, de proverbes, de railleries, parmi tous les peuples vers lesquels Yahweh t'aura emmené.
\VS{38}Tu emmèneras beaucoup de semence dans ton champ, et tu recueilleras peu, car les sauterelles la consumeront.
\VS{39}Tu planteras des vignes et tu les cultiveras ; mais tu n'en boiras point le vin et tu n'en recueilleras rien, car les vers la mangeront.
\VS{40}Tu auras des oliviers sur tout le territoire ; mais tu ne t'oindras point d'huile, car tes olives tomberont.
\VS{41}Tu engendreras des fils et des filles ; mais ils ne seront pas à toi, car ils iront en captivité.
\VS{42}Les insectes possèderont tous tes arbres et le fruit de ta terre.
\VS{43}L'étranger qui sera au milieu de toi montera toujours plus au-dessus de toi, et toi, tu descendras toujours plus bas.
\VS{44}Il te prêtera, et tu ne lui prêteras point ; il sera la tête, et tu seras la queue.
\VS{45}Toutes ces malédictions viendront sur toi, elles te poursuivront et t'atteindront jusqu'à ce que tu sois détruit, parce que tu n'auras pas obéi à la voix de Yahweh, ton Dieu, pour garder ses commandements et ses lois qu'il t'a ordonnés.
\VS{46}Elles seront à jamais pour toi et ta postérité comme des signes et des prodiges.
\VS{47}Et parce que tu n'auras pas servi Yahweh, ton Dieu, avec joie, et de bon cœur, malgré l'abondance de toutes choses,
\VS{48}tu serviras, dans la faim, dans la soif, dans la nudité, et dans la disette de toutes choses, ton ennemi que Yahweh enverra contre toi. Il mettra un joug de fer sur ton cou, jusqu'à ce qu'il t'ait détruit.
\TextTitle{Prophétie sur l'invasion babylonienne et la dispersion d'Israël}
\VS{49}Yahweh fera lever de loin, des extrémités de la terre, une nation qui volera comme l'aigle, une nation dont tu ne comprendras pas la langue,
\VS{50}une nation au visage féroce, et qui ne soutiendra point le vieillard et n'aura point pitié pour l'enfant\FTNT{Cette prophétie fut merveilleusement accomplie en 587 av. J.-C.. Voir 2 R. 24 et 25.}.
\VS{51}Elle mangera le fruit de tes troupeaux et les fruits de ta terre, jusqu'à ce que tu sois détruit ; elle n'épargnera ni blé, ni vin, ni huile, ni portée de ton gros et de ton menu bétail, jusqu'à ce qu'elle t'ait fait périr.
\VS{52}Elle t'assiégera dans toutes tes portes, jusqu'à ce que tombent ces hautes et fortes murailles dans lesquelles tu auras mis ta confiance dans tout ton pays ; elle t'assiégera dans toutes tes portes, dans tout le pays que Yahweh, ton Dieu, te donne.
\VS{53}Tu mangeras le fruit de tes entrailles, la chair de tes fils et de tes filles que Yahweh, ton Dieu, t'aura donnés, durant le siège et la détresse dont ton ennemi te serrera.
\VS{54}L'homme le plus tendre et le plus délicat d'entre vous regardera d'un œil malin son frère, sa femme bien-aimée, et le reste de ses fils qu'il a épargnés ;
\VS{55}et ne donnera à aucun d'eux de la chair de ses fils, qu'il mangera, parce qu'il ne lui restera rien du tout, à cause du siège et de la détresse dont ton ennemi te serrera dans toutes tes portes.
\VS{56}La femme la plus tendre et la plus délicate d'entre vous, qui n'a point osé mettre la plante de son pied sur la terre, par délicatesse et par mollesse, regardera d'un œil malin son mari bien-aimé, son fils, et sa fille ;
\VS{57}et le placenta qui sortira d'entre ses jambes, et les fils qu'elle enfantera ; car manquant de tout, elle les mangera secrètement, à cause du siège et de la détresse, dont ton ennemi te serrera dans toutes les villes.
\VS{58}Si tu ne prends garde de faire toutes les paroles de cette loi, qui sont écrites dans ce livre, en craignant le Nom glorieux et terrible de Yahweh, ton Dieu,
\VS{59}alors Yahweh rendra difficile tes plaies et les plaies de ta postérité, par des plaies grandes et des plaies de durée, des maladies malignes et longues.
\VS{60}Et il fera retourner sur toi toutes les maladies d'Egypte, face auxquelles tu avais peur ; et elles s'attacheront à toi.
\VS{61}Même Yahweh fera venir sur toi toutes maladies et toutes plaies, qui ne sont point écrites dans le livre de cette loi, jusqu'à ce que tu sois détruit.
\VS{62}Et vous resterez en petit nombre, après avoir été aussi nombreux comme les étoiles des cieux, parce que tu n'auras point obéi à la voix de Yahweh, ton Dieu.
\VS{63}Comme Yahweh s'est réjoui sur vous, en vous faisant du bien et en vous multipliant, de même Yahweh se réjouira sur vous en vous faisant périr et en vous détruisant ; et vous serez arrachés de la terre dans laquelle vous allez entrer en possession.
\VS{64}Yahweh te dispersera parmi tous les peuples, d'un bout de la terre jusqu'à l'autre ; et là, tu serviras d'autres dieux que ni toi ni tes pères n'avez connus, le bois et la pierre.
\VS{65}Tu n'auras aucun repos parmi ces nations, même la plante de ton pied n'aura aucun repos. Car Yahweh te donnera un cœur tremblant, des yeux languissants, et une âme souffrante.
\VS{66}Et ta vie sera en suspens devant toi, tu trembleras la nuit et le jour, et tu douteras de ta vie.
\VS{67}Tu diras le matin : Qui donnera le soir ? Et le soir tu diras : Qui donnera le matin ? À cause de l'effroi dont ton cœur sera effrayé, et à cause des choses que tu verras de tes yeux.
\VS{68}Et Yahweh te fera retourner en Egypte sur des navires, pour faire le chemin dont je t'ai dit : Tu ne le verras plus ; et là, vous vous vendrez à vos ennemis, comme esclaves et servantes ; et il n'y aura personne pour vous acheter.
\Chap{29}
\TextTitle{Yahweh rappelle sa fidélité à Israël}
\VerseOne{}Voici les paroles de l'alliance que Yahweh ordonna à Moïse de traiter avec les fils d'Israël au pays de Moab, outre l'alliance qu'il avait traitée avec eux à Horeb.
\VS{2}Moïse appela tout Israël, et leur dit : Vous avez vu tout ce que Yahweh a fait sous vos yeux, dans le pays d'Egypte, à Pharaon, à tous ses serviteurs, et à tout son pays,
\VS{3}les grandes épreuves que tes yeux ont vues, ces signes et ces grands miracles.
\VS{4}Mais, jusqu'à ce jour, Yahweh ne vous a point donné un cœur pour connaître, ni des yeux pour voir, ni des oreilles pour entendre.
\VS{5}Je t'ai conduit pendant quarante ans par le désert ; tes vêtements ne se sont point usés, et ton soulier ne s'est point usé à ton pied.
\VS{6}Vous n'avez point mangé de pain, ni bu de vin ni de liqueur forte, afin que vous connussiez que je suis Yahweh, votre Dieu.
\VS{7}Vous êtes parvenus dans ce lieu ; Sihon, roi de Hesbon, et Og, roi de Basan, sont sortis à notre rencontre, pour nous combattre, et nous les avons battus.
\VS{8}Nous avons pris leur pays, et nous l'avons donné en héritage aux Rubénites, aux Gadites, et à la demi-tribu des Manassites.
\TextTitle{Béni celui qui reste fidèle à l'alliance}
\VS{9}Vous garderez donc les paroles de cette alliance, et vous les pratiquerez, afin de réussir dans tout ce que vous ferez.
\VS{10}Vous vous tiendrez aujourd'hui devant Yahweh, votre Dieu, vos chefs de tribus, vos anciens, vos officiers, tous les hommes d'Israël,
\VS{11}vos enfants, vos femmes, et l'étranger qui est au milieu de ton camp, depuis celui qui coupe ton bois jusqu'à celui qui puise ton eau ;
\VS{12}afin que tu entres dans l'alliance de Yahweh, ton Dieu, dans ce serment, que Yahweh, ton Dieu, traite aujourd'hui avec toi,
\VS{13}afin de t'établir aujourd'hui pour son peuple et qu'il soit ton Dieu, comme il te l'a dit, et comme il l'a juré à tes pères, Abraham, Isaac et Jacob.
\VS{14}Ce n'est pas seulement avec vous que je traite cette alliance, ce serment.
\VS{15}Mais c'est avec ceux qui sont ici, avec nous aujourd'hui devant Yahweh, notre Dieu, et avec ceux qui ne sont point ici, avec nous aujourd'hui.
\TextTitle{Mise en garde de celui qui abandonne l'alliance}
\VS{16}Vous savez comment nous avons habité dans le pays d'Egypte, et comment nous sommes passés au milieu des nations, que vous avez traversées.
\VS{17}Vous avez vu leurs abominations et leurs idoles, le bois et la pierre, l'argent et l'or qui sont parmi eux.
\VS{18}Qu'il n'y ait parmi vous ni homme, ni femme, ni famille, ni tribu qui détourne son cœur aujourd'hui de Yahweh, notre Dieu, pour aller servir les dieux de ces nations. Qu'il n'y ait parmi vous de racine qui produise du poison et de l'absinthe.
\VS{19}Que personne, après avoir entendu les paroles de ce serment, ne se glorifie dans son cœur, en disant : J'aurai la paix, même si je marcherai dans les penchants de mon cœur, et que j'ajouterai l'ivresse à la soif.
\VS{20}Yahweh ne voudra point lui pardonner. Mais la colère et la jalousie s'enflammeront contre cet homme, et toutes les malédictions écrites dans ce livre reposeront sur lui, et Yahweh effacera son nom de dessous les cieux.
\VS{21}Yahweh le séparera de toutes les tribus d'Israël, pour son malheur, selon toutes les malédictions de l'alliance écrite dans ce livre de la loi.
\VS{22}Et la génération suivante, vos fils qui se lèveront après vous, et l'étranger qui viendra d'un pays lointain, quand ils verront les plaies et les maladies, dont Yahweh aura frappé ce pays ;
\VS{23}quand ils verront le soufre, le sel, et l'embrasement de toute la contrée, où il n'y aura ni semence, ni produit, et où aucune herbe ne poussera, ainsi que la destruction de Sodome, de Gomorrhe, d'Adma et de Tseboïm, que Yahweh détruisit dans sa colère et dans sa fureur ;
\VS{24}Toutes les nations diront : Pourquoi Yahweh a-t-il traité ainsi ce pays ? D'où vient l'ardeur de cette grande colère ?
\VS{25}Et on répondra : C'est parce qu'ils ont abandonné l'alliance de Yahweh, le Dieu de leurs pères, qu'il a traitée avec eux quand il les fit sortir du pays d'Egypte ;
\VS{26}car ils sont allés servir d'autres dieux et se sont prosternés devant eux ; des dieux qu'ils ne connaissaient point et qu'il ne leur avait point donnés en partage.
\VS{27}Alors la colère de Yahweh s'est enflammée contre ce pays, et il a fait venir sur lui toutes les malédictions écrites dans ce livre.
\VS{28}Yahweh les a arrachés de leur terre avec colère, avec indignation, avec une grande colère, et il les a jetés sur un autre pays, comme on le voit aujourd'hui.
\VS{29}Les choses cachées sont à Yahweh, notre Dieu ; les choses révélées sont à nous et à nos fils, à jamais, afin que nous fassions toutes les paroles de cette loi.
\Chap{30}
\TextTitle{Yahweh bénira et restaurera le peuple repentant}
\VerseOne{}Lorsque toutes ces choses seront venues sur toi, la bénédiction et la malédiction, que je mets devant toi, si tu les rappelles dans ton cœur, parmi toutes les nations vers lesquelles Yahweh, ton Dieu, t'aura chassé ;
\VS{2}si tu reviens à Yahweh, ton Dieu, et si tu obéis à sa voix de tout ton cœur, de toute ton âme, toi et tes fils, selon tout ce que je t'ordonne aujourd'hui,
\VS{3}Yahweh, ton Dieu, ramènera tes captifs et aura compassion de toi ; il te rassemblera encore du milieu de tous les peuples parmi lesquels Yahweh, ton Dieu, t'aura dispersé.
\VS{4}Quand tu seras dispersé à l'extrémité des cieux, Yahweh, ton Dieu, te rassemblera de là, et de là, il te prendra.
\VS{5}Yahweh, ton Dieu, te ramènera dans le pays que tes pères possédaient, et tu le posséderas ; il te fera du bien, et te rendra plus nombreux que tes pères.
\VS{6}Yahweh, ton Dieu, circoncira ton cœur, et le cœur de ta postérité, pour que tu aimes Yahweh, ton Dieu, de tout ton cœur, et de toute ton âme, afin que tu vives\FTNT{Ro. 2:29.}.
\VS{7}Yahweh, ton Dieu, mettra toutes ces malédictions sur tes ennemis, et sur ceux qui te haïront et te persécuteront.
\VS{8}Ainsi tu retourneras à Yahweh, tu obéiras à sa voix, et tu feras tous ses commandements que je t'ordonne aujourd'hui.
\TextTitle{Faire connaître la loi aux futures générations}
\VS{9}Yahweh, ton Dieu, te fera abonder en bien dans toute l'œuvre de ta main, dans le fruit de tes entrailles, dans le fruit de tes troupeaux et dans le fruit de ta terre ; car Yahweh se réjouira de nouveau de ton bonheur, comme il s'est réjoui de celui de tes pères,
\VS{10}lorsque tu obéiras à la voix de Yahweh, ton Dieu, en gardant ses commandements et ses ordonnances écrites dans ce livre de la loi, quand tu reviendras à Yahweh, ton Dieu, de tout ton cœur et de toute ton âme.
\TextTitle{Le peuple devant un choix}
\VS{11}Ce commandement que je t'ordonne aujourd'hui n'est pas au-dessus de tes forces et hors de ta portée.
\VS{12}Il n'est pas aux cieux pour que tu dises : Qui montera pour nous aux cieux, nous l'apportera et nous le fera entendre, pour que nous le fassions ?
\VS{13}Il n'est point de l'autre côté de la mer pour que tu dises : Qui passera de l'autre côté de la mer pour nous, nous l'apportera, et nous le fera entendre pour que nous le fassions ?
\VS{14}Cette parole est tout près de toi, dans ta bouche et dans ton cœur, afin que tu la pratiques.
\VS{15}Regarde, je mets aujourd'hui devant toi la vie et le bien, la mort et le mal.
\VS{16}Car je t'ordonne aujourd'hui d'aimer Yahweh, ton Dieu, de marcher dans ses voies, de garder ses commandements, ses lois, et ses ordonnances, afin que tu vives, que tu multiplies, et que Yahweh, ton Dieu, te bénisse dans le pays où tu vas entrer en possession.
\VS{17}Mais si ton cœur se détourne, si tu n'obéis point, et si tu te laisses entraîner à te prosterner devant d'autres dieux et à les servir,
\VS{18}je vous déclare aujourd'hui que vous périrez certainement, et que vous ne prolongerez point vos jours sur la terre dont vous allez entrer en possession, après avoir passé le Jourdain.
\VS{19}J'en prends aujourd'hui à témoin les cieux et la terre contre vous : J'ai mis devant toi la vie et la mort, la bénédiction et la malédiction. Choisis la vie\FTNT{Mt. 7:13-14.}, afin que tu vives, toi et ta postérité.
\VS{20}en aimant Yahweh, ton Dieu, en obéissant à sa voix, et en t'attachant à lui : Car c'est lui qui est ta vie et la longueur de tes jours, afin que tu demeures sur la terre que Yahweh a juré à tes pères, Abraham, Isaac, et Jacob, de leur donner.
\Chap{31}
\TextTitle{Moïse encourage et affermit le peuple}
\VerseOne{}Moïse s'en alla, et dit ces paroles à tout Israël :
\VS{2}Aujourd'hui, leur dit-il, je suis âgé de cent vingt ans, je ne pourrai plus sortir ni entrer, et Yahweh m'a dit : Tu ne passeras point ce Jourdain.
\VS{3}Yahweh, ton Dieu, passera lui-même devant toi, il détruira ces nations devant toi, et tu les posséderas. Josué passera aussi devant toi, comme Yahweh l'a dit.
\VS{4}Yahweh les traitera comme il a fait à Sihon et à Og, rois des Amoréens, qu'il a détruits avec leurs pays.
\VS{5}Yahweh les livrera devant vous, et vous leur ferez selon tout le commandement que je vous ai ordonné.
\VS{6}Fortifiez-vous et prenez courage ! Ne craignez point et ne soyez point effrayés devant eux ; car Yahweh, ton Dieu, marchera avec toi, il ne te délaissera point et ne t'abandonnera point.
\VS{7}Moïse appela Josué, et lui dit en présence de tout Israël : Fortifie-toi et prends courage, car tu entreras avec ce peuple dans le pays que Yahweh a juré à leurs pères de leur donner, et c'est toi qui les en mettras en possession.
\VS{8}Yahweh est celui qui marchera devant toi, il sera lui-même avec toi, il ne te délaissera point, il ne t'abandonnera point ; ne crains point, et ne t'effraie point.
\VS{9}Moïse écrivit cette loi, et il la donna aux sacrificateurs, fils de Lévi, qui portaient l'arche de l'alliance de Yahweh, et à tous les anciens d'Israël.
\VS{10}Moïse leur ordonna, en disant : Tous les sept ans, au temps fixé de l'année du relâche, à la fête des tabernacles,
\VS{11}quand tout Israël viendra se présenter devant Yahweh, ton Dieu, dans le lieu qu'il aura choisi, tu liras alors cette loi devant tout Israël, à leurs oreilles.
\VS{12}Tu rassembleras le peuple, les hommes, les femmes, les enfants et l'étranger qui sera dans tes portes, pour qu'ils t'entendent, et qu'ils apprennent à craindre Yahweh, votre Dieu, et qu'ils prennent garde de faire toutes les paroles de cette loi.
\VS{13}Et leurs fils qui ne la connaîtront point l'entendront, et ils apprendront à craindre Yahweh, votre Dieu, tous les jours que vous vivrez sur cette terre que vous allez posséder après avoir passé le Jourdain.
\TextTitle{Yahweh annonce les évènements à venir}
\VS{14}Yahweh dit à Moïse : Voici, le jour où tu vas mourir est proche. Appelle Josué, et tenez-vous dans la tente d'assignation. Je lui donnerai mes ordres. Moïse et Josué allèrent et se présentèrent dans la tente d'assignation.
\VS{15}Et Yahweh apparut dans la tente, dans une colonne de nuée ; et la colonne de nuée s'arrêta à l'entrée de la tente.
\VS{16}Yahweh dit à Moïse : Voici, tu vas te coucher avec tes pères. Et ce peuple se lèvera et se prostituera après les dieux étrangers du pays au milieu duquel il va entrer. Il m'abandonnera et violera mon alliance que j'ai traitée avec lui.
\VS{17}En ce jour-là ma colère s'enflammera contre lui. Je les abandonnerai, et je leur cacherai ma face. Il sera dévoré, une multitude de maux et d'angoisses l'atteindront, et il dira en ce jour-là : N'est-ce pas parce que mon Dieu n'est point au milieu de moi, que ces maux m'ont atteint ?
\VS{18}En ce jour-là, je cacherai entièrement ma face, à cause de tout le mal qu'il aura fait, parce qu'il se sera tourné vers d'autres dieux.
\VS{19}Maintenant, écrivez ce cantique. Enseigne-le aux fils d'Israël, mets-le dans leur bouche, afin que ce cantique me serve de témoignage contre les fils d'Israël.
\VS{20}Car je le conduirai sur la terre que j'ai juré à ses pères, où coulent le lait et le miel ; il mangera, se rassasiera, et s'engraissera ; puis il se tournera vers d'autres dieux, et il les servira, il me méprisera et violera mon alliance ;
\VS{21}et quand il sera atteint par une multitude de maux et d'angoisses, ce cantique, qui ne sera point oublié et qui sera dans la bouche de la postérité, répondra comme témoin contre eux. Je connais ses desseins, qu'il a déjà préparés aujourd'hui, avant même que je l'aie fait entrer dans le pays que j'ai juré.
\VS{22}Moïse écrivit ce cantique en ce jour-là, et l'enseigna aux fils d'Israël.
\VS{23}Yahweh commanda à Josué, fils de Nun, en disant : Fortifie-toi et prends courage, car c'est toi qui feras entrer les fils d'Israël dans le pays que je leur ai juré ; et je serai avec toi.
\VS{24}Et quand Moïse eut complètement achevé d'écrire les paroles de cette loi dans un livre,
\VS{25}Moïse commanda aux Lévites qui portaient l'arche de l'alliance de Yahweh, en disant :
\VS{26}Prenez ce livre de la loi, et mettez-le à côté de l'arche de l'alliance de Yahweh, votre Dieu, et il sera là comme témoin contre toi.
\VS{27}Car je connais ta rébellion et ton cou raide. Voici, déjà aujourd'hui étant en vie avec vous, vous avez été rebelles contre Yahweh, combien plus le serez-vous après ma mort ?
\VS{28}Faites assembler devant moi tous les anciens de vos tribus, et vos officiers, et je dirai ces paroles en leur présence, et j'appellerai à témoin contre eux les cieux et la terre.
\VS{29}Car je sais qu'après ma mort vous vous corromprez, et que vous vous détournerez de la voie que je vous ai ordonnée ; mais à la fin, le malheur vous atteindra, parce que vous aurez fait ce qui déplaît aux yeux de Yahweh, en l'irritant par les œuvres de vos mains.
\VS{30}Ainsi Moïse prononça entièrement les paroles de ce cantique-ci, en présence de toute l'assemblée d'Israël.
\Chap{32}
\TextTitle{Cantique de Moïse}
\VerseOne{}Cieux ! Prêtez l'oreille, et je parlerai. Terre ! écoute les paroles de ma bouche.
\VS{2}Que mon enseignement tombe comme la pluie, que ma parole se répande comme la rosée, comme une pluie fine sur l'herbe, et comme une averse sur la verdure !
\VS{3}Car je proclamerai le Nom de Yahweh. Donnez gloire à notre Dieu.
\VS{4}L'œuvre du rocher\FTNT{Voir commentaire en Es. 8:13-17.} est parfaite, car toutes ses voies sont justes. C'est un Dieu fidèle et sans iniquité, il est juste et droit.
\VS{5}Ils se sont corrompus, à lui n'est la faute ; la faute est à ses fils, c'est une génération fausse et tortueuse.
\TextTitle{Israël, le choix de Yahweh}
\VS{6}Est-ce ainsi que tu récompenses Yahweh, peuple insensé et dépourvu de sagesse ? N'est-il pas ton père, celui qui t'a acquis ? Il t'a fait et t'a établi.
\VS{7}Souviens-toi des anciens jours, considère les années, de génération en génération, interroge ton père, et il te l'apprendra, et tes anciens, et ils te le diront.
\VS{8}Quand le Très-Haut laissa un héritage aux nations, quand il sépara les fils des hommes, il fixa les limites des peuples selon le nombre des fils d'Israël ;
\VS{9}car la part de Yahweh, c'est son peuple, Jacob est le lot de son héritage.
\VS{10}Il l'a trouvé dans un pays désert, dans une solitude aux hurlements chaotiques, il l'a entouré, il l'a dirigé, il l'a gardé comme la prunelle de son œil,
\VS{11}comme l'aigle qui réveille sa nichée, couve ses petits, étend ses ailes, les prend, les porte sur ses ailes.
\VS{12}Yahweh seul l'a conduit, et il n'y a point eu avec lui de dieu étranger.
\VS{13}Il l'a fait monter à cheval sur les hauteurs du pays, et il a mangé les fruits des champs ; il lui a donné à sucer le miel du rocher, l'huile du rocher le plus dur,
\VS{14}la crème des vaches, le lait des brebis, et la graisse des agneaux, des béliers de Basan, et des boucs, et la fleur du froment ; et tu as bu le vin qui était le sang de la grappe.
\TextTitle{Condamnation de l'apostasie d'Israël}
\VS{15}Jeshurun\FTNT{Littéralement « Jeshurun » en hébreu : « celui qui est droit ». Nom symbolique donné à Israël pour décrire son caractère idéal.} s'est engraissé, et a regimbé ; tu es devenu gras, gros, et épais ! Et il a abandonné Dieu qui l'a fait, et il a méprisé le rocher de son salut.
\VS{16}Ils ont provoqué sa jalousie par des dieux étrangers, ils l'ont irrité par des abominations.
\VS{17}Ils ont sacrifié à des démons, qui ne sont point Dieu ; aux dieux qu'ils ne connaissaient point, nouveaux, venus depuis peu, et que vos pères n'ont point redoutés.
\VS{18}Tu as oublié le rocher qui t'a engendré, et tu as oublié le Dieu qui t'a fait naître.
\VS{19}Yahweh l'a vu, et a été irrité, parce que ses fils et ses filles l'ont provoqué à la colère.
\VS{20}Et il a dit : Je cacherai ma face, je verrai quelle sera leur fin ; car ils sont une génération perverse, des fils infidèles.
\VS{21}Ils ont excité ma jalousie par ce qui n'est point Dieu, ils m'ont irrité par leurs vanités ; ainsi je provoquerai leur jalousie par ce qui n'est point un peuple, et je les offenserai par une nation insensée.
\VS{22}Car le feu de ma colère s'est allumé, et brûlera jusqu'au fond du scheol, dévorera la terre et son fruit, et embrasera les fondements des montagnes.
\VS{23}Je rassemblerai sur eux des maux, et je détruirai toutes mes flèches sur eux.
\VS{24}Ils seront consumés par la famine, rongés par des charbons ardents, et par une destruction amère ; j'enverrai contre eux la dent des bêtes et le venin des serpents qui rampent sur la poussière.
\VS{25}L'épée venant de dehors les privera les uns des autres ; et au-dedans, la terreur les privera d'enfants. Il en sera du jeune homme comme de la vierge, de l'enfant à la mamelle comme de l'homme aux cheveux blancs.
\TextTitle{A Yahweh la vengeance et la rétribution}
\VS{26}Je dirais : Je les détruirai, et je ferai disparaître leur mémoire d'entre les hommes !
\VS{27}Si je ne craignais la colère de l'ennemi, de peur que leurs adversaires ne se méprennent, et ne disent : Notre main est élevée, et ce n'est pas Yahweh qui a fait tout ceci.
\VS{28}Car c'est une nation qui se perd par ses conseils, et il n'y a en eux aucune intelligence.
\VS{29}Si seulement ils étaient sages, ils comprendraient ceci, et ils considéreraient leur fin.
\VS{30}Comment un seul en poursuivrait-il mille, et deux en mettraient-ils dix mille en fuite, si ce n'était que leur Rocher les avait vendus, et que Yahweh ne les avait enserrés ?
\VS{31}Car leur rocher n'est pas comme notre Rocher, nos ennemis en sont juges.
\VS{32}Car leur vigne est du plant de Sodome, et du terroir de Gomorrhe ; leurs raisins sont des raisins empoisonnés, leurs grappes sont amères.
\VS{33}Leur vin est un venin de dragon, et du poison cruel d'aspic.
\VS{34}Cela n'est-il pas caché près de moi, scellé dans mes trésors ?
\VS{35}A moi la vengeance et la rétribution, le temps où leur pied glissera ! Car le jour de leur calamité est près, et les choses qui leur doivent arriver se hâtent.
\VS{36}Mais Yahweh jugera son peuple ; et il se repentira en faveur de ses serviteurs, quand il verra que leur force a disparu, et qu'il n'y a personne de retenu ni d'abandonné.
\VS{37}Et il dira : Où sont leurs dieux, le rocher en qui ils se confiaient,
\VS{38}qui mangeaient la graisse de leurs sacrifices, qui buvaient le vin de leurs libations ? Qu'ils se lèvent, qu'ils vous aident, et qu'ils vous protègent !
\VS{39}Voyez maintenant que c'est moi, moi-même, et il n'y a point de dieu avec moi\FTNT{Ce verset confirme que Dieu est un puisqu'il n'y a pas d'autres dieux à ses côtés.} ; je fais mourir et je fais vivre, je blesse et je guéris ; et personne ne délivre de ma main.
\VS{40}Car je lève ma main au ciel, et je dis : Je vis éternellement.
\VS{41}Si j'aiguise l'éclair de mon épée, et si ma main saisit la justice, je rendrai la vengeance à mes adversaires et je rétribuerai ceux qui me haïssent.
\VS{42}Mon épée dévorera leur chair et j'enivrerai mes flèches de sang, du sang des tués et des captifs, de la tête des chefs de l'ennemi.
\VS{43}Nations, réjouissez-vous avec son peuple ! Car il venge le sang de ses serviteurs, il tire vengeance de ses ennemis, et fait propitiation pour sa terre et pour son peuple.
\TextTitle{Fin du cantique, invitation à demeurer fidèle}
\VS{44}Moïse vint et prononça toutes les paroles de ce cantique, à l'oreille du peuple, lui et Josué, fils de Nun.
\VS{45}Et quand Moïse eut achevé de prononcer toutes ces paroles à tout Israël,
\VS{46}il leur dit : Appliquez votre cœur à toutes ces paroles que je vous conjure aujourd'hui d'ordonner à vos fils, afin qu'ils prennent garde de faire toutes les paroles de cette loi.
\VS{47}Car ce n'est pas une parole vaine pour vous, mais c'est votre vie ; et par cette parole vous prolongerez vos jours sur la terre que vous posséderez, après avoir passé le Jourdain.
\TextTitle{Moïse, invité à monter sur le mont Nebo}
\VS{48}En ce même jour, Yahweh parla à Moïse, en disant :
\VS{49}Monte sur cette montagne d'Abarim, sur le mont Nebo, au pays de Moab, vis-à-vis de Jéricho ; et regarde le pays de Canaan, que je donne en possession aux fils d'Israël.
\VS{50}Tu mourras sur la montagne où tu vas monter, et tu seras recueilli auprès de ton peuple, comme Aaron, ton frère, est mort sur la montagne d'Hor, et a été recueilli vers son peuple,
\VS{51}parce que vous avez péché contre moi au milieu des fils d'Israël, aux eaux de Meriba, à Kadès, dans le désert de Tsin ; car vous ne m'avez point sanctifié au milieu des fils d'Israël.
\VS{52}Tu verras le pays devant toi, mais tu n'entreras point dans le pays, que je donne aux fils d'Israël.
\Chap{33}
\TextTitle{Moïse bénit les tribus d'Israël}
\VerseOne{}Voici la bénédiction par laquelle Moïse, homme de Dieu, bénit les fils d'Israël avant sa mort.
\VS{2}Il dit : Yahweh est venu de Sinaï, il s'est levé sur eux de Séir, il a resplendi de la montagne de Paran, et il est sorti du milieu des myriades des saints : Il leur a de sa droite envoyé le feu de la loi.
\VS{3}En effet, il aime les peuples ; tous ses saints sont dans ta main. Ils se sont mis à tes pieds pour recevoir tes paroles.
\VS{4}Moïse nous a donné la loi, héritage de l'assemblée de Jacob.
\VS{5}Il était roi de Jeshurun\FTNT{De. 32:15.}, quand les chefs du peuple s'assemblaient ensemble, avec les tribus d'Israël.
\VS{6}Que Ruben vive et qu'il ne meure point, encore que ses hommes soient en petit nombre.
\VS{7}Et voici ce qu'il dit pour Juda : Ô Yahweh ! Ecoute la voix de Juda, et ramène-le vers son peuple. Que ses mains soient puissantes, et sois-lui en aide contre ses ennemis.
\VS{8}Il dit aussi touchant Lévi : Tes thummim et tes urim sont à l'homme fidèle que tu as éprouvé à Massa, et avec qui tu as contesté aux eaux de Meriba.
\VS{9}Il dit de son père et de sa mère : Je ne les ai point vus ! Il ne reconnaît point ses frères, et ne connaîtrait point ses fils. Car ils gardent tes paroles, et ils gardent ton alliance.
\VS{10}Ils enseignent tes ordonnances à Jacob, et ta loi à Israël ; ils mettent l'encens sous tes narines, et l'holocauste sur ton autel.
\VS{11}Ô Yahweh, bénis sa force ! Agrée l'œuvre de ses mains ! Brise les reins de ceux qui s'élèvent contre lui, et que ceux qui le haïssent ne se relèvent plus !
\VS{12}Il dit de Benjamin : Le bien-aimé de Yahweh habitera en sécurité avec lui ; il le protégera toujours, et demeurera entre ses épaules.
\VS{13}Il dit de Joseph : Son pays est béni par Yahweh, du meilleur don des cieux, de la rosée, et de l'abîme qui est en bas,
\VS{14}du meilleur des produits du soleil, du meilleur de ce qui est produit par la lune,
\VS{15}du meilleur produit du sommet des montagnes d'ancienneté, du meilleur des collines éternelles,
\VS{16}du meilleur de la terre et de sa plénitude. Que la grâce de celui qui demeura dans le buisson vienne sur la tête de Joseph, sur le sommet, sur le sommet de la tête de celui qui est consacré d'entre ses frères !
\VS{17}Sa majesté est comme le premier-né de son taureau ; et ses cornes comme les cornes du buffle ; il poussera tous les peuples ensemble jusqu'aux extrémités de la terre : Ce sont les myriades d'Ephraïm, et ce sont les myriades de Manassé.
\VS{18}Il dit de Zabulon : Réjouis-toi, Zabulon, dans ta sortie, et toi, Issacar, dans tes tentes.
\VS{19}Ils appelleront les peuples sur la montagne, ils y offriront des sacrifices de justice, car ils suceront l'abondance des mers, et les trésors cachés dans le sable.
\VS{20}Il dit aussi de Gad : Béni soit celui qui élargit Gad ! Il habite comme une lionne, et il déchire le bras et la tête.
\VS{21}Il a choisi les prémices, parce que c'était là qu'était cachée la portion du législateur, et il est venu en tête du peuple ; il a exécuté la justice de Yahweh et ses jugements envers Israël.
\VS{22}Et il dit de Dan : Dan est un jeune lion, il s'élance de Basan.
\VS{23}Il dit de Nephthali : Nephthali, rassasié de faveur, et rempli de la bénédiction de Yahweh, possède l'occident et le sud.
\VS{24}Il dit aussi d'Aser : Aser sera béni entre les fils ; il sera agréable à ses frères, et il trempera son pied dans l'huile.
\VS{25}Tes verrous seront de fer et d'airain, et ta force durera autant que tes jours.
\VS{26}Nul n'est comme le Dieu d'Israël, porté sur les cieux pour te venir en aide, et sur les nuées dans sa majesté.
\VS{27}Le Dieu d'éternité est un refuge, et au-dessous de toi sont ses bras éternels ; car il a chassé de devant toi tes ennemis, et il a dit : Extermine.
\VS{28}Israël habitera en sécurité, la source de Jacob est à part dans un pays de blé et de vin, et ses cieux distilleront la rosée.
\VS{29}Que tu es heureux, Israël ! Qui est le peuple semblable à toi, qui ait été sauvé par Yahweh, le bouclier de ton secours et l'épée de ta majesté ? Tes ennemis dissimuleront devant toi, et tu fouleras de tes pieds leurs lieux élevés.
\Chap{34}
\TextTitle{Moïse voit le pays mais n'y entrera pas}
\VerseOne{}Moïse monta des plaines de Moab sur le mont Nebo, au sommet du Pisga, vis-à-vis de Jéricho. Et Yahweh lui fit voir tout le pays : De Galaad jusqu'à Dan,
\VS{2}tout Nephthali, le pays d'Ephraïm et de Manassé, tout le pays de Juda, jusqu'à la Mer Occidentale,
\VS{3}le sud, les environs du Jourdain, la plaine de Jéricho, la ville des palmiers, jusqu'à Tsoar.
\VS{4}Yahweh lui dit : C'est ici le pays que j'ai juré à Abraham, à Isaac, et à Jacob, en disant : Je le donnerai à ta postérité. Je te l'ai fait voir de tes yeux ; mais tu n'y entreras point.
\TextTitle{Mort de Moïse}
\VS{5}Moïse, serviteur de Yahweh, mourut là, dans le pays de Moab, selon la parole de Yahweh.
\VS{6}Et il l'ensevelit dans la vallée, au pays de Moab, vis-à-vis de Beth-Peor. Personne n'a connu son sépulcre jusqu'à aujourd'hui\FTNT{Jud. 1:9.}.
\VS{7}Moïse était âgé de cent vingt ans quand il mourut ; sa vue n'était point affaiblie, et sa vigueur n'était point passée.
\VS{8}Les fils d'Israël pleurèrent Moïse trente jours dans les plaines de Moab ; et ces jours de pleurs et de deuil sur Moïse furent accomplis.
\TextTitle{Josué, successeur de Moïse}
\VS{9}Josué, fils de Nun, fut rempli de l'Esprit de sagesse, parce que Moïse lui avait imposé les mains\FTNT{Jos. 1:9.}. Les fils d'Israël lui obéirent, et firent ce que Yahweh avait ordonné à Moïse.
\VS{10}Il ne s'est plus levé en Israël de prophète comme Moïse, que Yahweh connaissait face à face.
\VS{11}Selon tous les signes et les miracles que Yahweh l'envoya faire au pays d'Egypte, devant Pharaon, et tous ses serviteurs, et tout son pays,
\VS{12}et selon toute cette main forte, et tous ces terribles prodiges, que Moïse fit sous les yeux de tout Israël.
\PPE{}
\end{multicols}
