\ShortTitle{2 Pi.}\BookTitle{2 Pierre}\BFont
\noindent\hrulefill
{\footnotesize
\textit{
\bigskip
{\centering{}
\\Auteur~: Pierre
\\(Gr.~: Petro)
\\Signification~: Roc, pierre
\\Thème~: Appel à la sainteté et prophétie sur l'apparition de moqueurs et de maîtres corrompus
\\Date de rédaction~: Env. 66 ap. J.-C.\\}
}
\textit{
\\Sans doute écrite à Rome, cette épître, tout comme la première, semble avoir été destinée aux églises d'Asie Mineure. Pierre y exhorte les chrétiens à être vigilants quant aux faux docteurs et aux nombreuses hérésies. Il insiste sur l'appel et l'élection des chrétiens dont le comportement doit être exemplaire. Il leur rappelle la véracité des écrits prophétiques et les invite à persévérer dans la vie chrétienne.\bigskip
}
}
\par\nobreak\noindent\hrulefill
\begin{multicols}{2}
\Chap{1}
\TextTitle{Introduction}
\VerseOne{}Simon Pierre, serviteur et apôtre de Jésus-Christ, à vous qui avez reçu en partage une foi du même prix que la notre, par la justice de notre Dieu et Sauveur Jésus-Christ\FTNT{Pierre affirma avec force la divinité de Jésus-Christ.}:
\VS{2}Que la grâce et la paix vous soient multipliées par la connaissance de Dieu et de notre Seigneur Jésus.
\TextTitle{Les grandes vertus chrétiennes}
\VS{3}Puisque sa divine puissance nous a donné tout ce qui appartient à la vie et à la piété, par la connaissance de celui qui nous a appelés par sa gloire et par sa vertu,
\VS{4}par lesquelles nous sont données les grandes et précieuses promesses, afin que par elles vous soyez faits participants de la nature divine, en fuyant la corruption qui règne dans le monde par la convoitise.
\VS{5}A cause de cela même, faites tous vos efforts pour fournir à votre foi la vertu, et à la vertu la connaissance,
\VS{6}à la connaissance la tempérance, et à la tempérance la patience, à la patience la piété,
\VS{7}à la piété l'amour fraternel, et à l'amour fraternel la charité.
\VS{8}Car si ces choses sont en vous, et y abondent, elles ne vous laisseront point oisifs ni stériles pour la connaissance de notre Seigneur Jésus-Christ.
\VS{9}Mais celui en qui ces choses ne se trouvent point est aveugle, et ne voit pas de loin, ayant oublié la purification de ses anciens péchés.
\VS{10}C'est pourquoi, mes frères, efforcez-vous plutôt à affermir votre vocation et votre élection~; car, en faisant cela, vous ne broncherez jamais.
\VS{11}Car par ce moyen, l'entrée au Royaume éternel de notre Seigneur et Sauveur Jésus-Christ vous sera abondamment accordée.
\TextTitle{Autorité du témoignage de Pierre}
\VS{12}C'est pourquoi je ne négligerai pas de vous rappeler sans cesse ces choses, quoique vous ayez de la connaissance, et que vous soyez fondés dans la vérité présente.
\VS{13}Car je crois qu'il est juste que je vous réveille par des avertissements, pendant que je suis dans cette tente\FTNT{Tente ou tabernacle. Ces termes sont utilisés comme une belle métaphore du corps humain qui est la demeure de l'esprit humain. Voir \vref{Za. 12:1}.},
\VS{14}sachant que dans peu de temps je dois la quitter, comme notre Seigneur Jésus-Christ lui-même me l'a déclaré.
\TextTitle{Souvenir de la transfiguration}
\VS{15}Mais j'aurai soin qu'après mon départ, vous puissiez toujours vous souvenir de ces choses.
\VS{16}Car ce n'est pas en suivant des fables composées avec artifice, que nous vous avons fait connaître la puissance et l'avènement\FTNT{L'avènement du Seigneur. Voir \vref{Mt. 24:3}.} de notre Seigneur Jésus-Christ, mais comme ayant vu sa majesté de nos propres yeux.
\VS{17}Car il reçut de Dieu le Père, honneur et gloire, lorsque cette voix lui fut adressée du milieu de la gloire magnifique~: Celui-ci est mon Fils bien-aimé, en qui j'ai mis toute mon affection\FTNT{\vref{Mt. 3:17}.}.
\VS{18}Et nous entendîmes cette voix envoyée du ciel, lorsque nous étions avec lui sur la sainte montagne.
\TextTitle{Témoignage sur la véracité des Ecritures prophétiques}
\VS{19}Nous avons aussi la parole des prophètes qui est très ferme, à laquelle vous faites bien d'être attentifs, comme à une lampe qui brille dans un lieu obscur, jusqu'à ce que le jour vienne à paraître et que l'Etoile du matin\FTNT{Jésus est l'Etoile du matin. C'est cette étoile qui indiqua aux mages le chemin de la maison où était l'enfant Jésus (\vref{Mt. 2}). Balaam a également parlé de cette étoile (\vref{No. 24:17}) et Jésus lui-même s'est présenté comme l'Etoile brillante du matin (\vref{Ap. 22:16}).} se lève dans vos cœurs~;
\VS{20}sachant premièrement ceci, qu'aucune prophétie de l'Ecriture ne procède d'une interprétation particulière,
\VS{21}car la prophétie n'a jamais été autrefois apportée par la volonté humaine, mais les saints hommes de Dieu, ont parlé, étant poussés par le Saint-Esprit.
\Chap{2}
\TextTitle{Avertissements contre les faux docteurs}
\VerseOne{}Mais comme il y a eu de faux prophètes parmi le peuple, il y aura aussi parmi vous de faux docteurs, qui introduiront secrètement des sectes pernicieuses, et qui reniant le Seigneur qui les a rachetés, attireront sur eux-mêmes une ruine soudaine. 
\VS{2}Et plusieurs suivront leurs sectes de perdition, et à cause d'eux, la voie de la vérité sera blasphémée.
\VS{3}Par cupidité, ils trafiqueront\FTNT{Trafic~: beaucoup de chrétiens deviennent les proies de faux pasteurs.} de vous au moyen de paroles déguisées, mais la condamnation qui leur est destinée depuis longtemps ne tarde point, et leur perdition ne sommeille point.
\VS{4}Car si Dieu n'a pas épargné les anges qui ont péché, s'il les a précipités dans l'abîme\FTNT{Abîme. Du grec «~tartaroo~», ce terme est le nom de la région souterraine, lugubre et sombre, considérée par les grecs anciens comme la demeure du méchant à sa mort, où il souffre le châtiment pour ses mauvaises actions. «~Tartaroo~» vient de «~tartaros~» qui signifie «~les plus profonds abîmes du Hadès~».}, les a liés avec des chaînes d'obscurité, les a livrés pour y être gardés jusqu'au jugement~;
\VS{5}et s'il n'a point épargné l'ancien monde, mais a gardé Noé\FTNT{Noé. Voir \vref{Ge. 7}.}, lui huitième, qui était le prédicateur de la justice~; et a fait venir le déluge sur le monde des impies~;
\VS{6}et s'il a condamné à la destruction totale les villes de Sodome et de Gomorrhe, les réduisant en cendres, et les mettant pour être un exemple à ceux qui vivraient dans l'impiété,
\VS{7}et s'il a délivré le juste Lot\FTNT{Lot. Voir \vref{Ge. 18-19}.}, qui avait eut beaucoup à souffrir de ces abominables par leur infâme conduite~;
\VS{8}car cet homme juste, qui habitait au milieu d'eux, affligeait tous les jours son âme juste, à cause de ce qu'il voyait et entendait dire de leurs méchantes actions~;
\VS{9}le Seigneur sait ainsi délivrer de l'épreuve ceux qui l'honorent, et réserver les injustes pour être punis au jour du jugement~;
\VS{10}principalement ceux qui vont après la chair dans la passion de l'impureté, et qui méprisent l'autorité. Gens audacieux et arrogants, ils ne craignent point d'injurier les gloires,
\VS{11}alors que les anges qui sont supérieurs en force et en puissance, ne prononcent point contre elles de jugement blasphématoire devant le Seigneur.
\VS{12}Mais eux, semblables à des bêtes sans intelligence, qui s'abandonnent à leurs penchants naturels, et qui sont nées pour être prises et détruites, ils parlent d'une manière blasphématoire de ce qu'ils ignorent, et ils périront par leur propre corruption,
\VS{13}et ils recevront la récompense de leur iniquité. Ils aiment à être tous les jours dans les délices. Ce sont des taches et des souillures, et ils font leurs délices de leurs tromperies dans les repas qu'ils font avec vous.
\VS{14}Ils ont les yeux pleins d'adultère, ils ne cessent jamais de pécher, ils attirent les âmes mal affermies~; ils ont le cœur exercé à la cupidité~; ce sont des enfants de malédiction,
\VS{15}qui ayant laissé le droit chemin, se sont égarés, et ont suivi la voie de Balaam\FTNT{Balaam. Voir \vref{No. 22}.}, fils de Bosor, qui aima le salaire de l'iniquité, mais il fut repris pour sa transgression:
\VS{16}Car une ânesse muette, parlant d'une voix humaine, arrêta la folie du prophète.
\VS{17}Ce sont des fontaines sans eau, des nuées agitées par le tourbillon, et des gens à qui l'obscurité des ténèbres est réservée éternellement.
\VS{18}Car en prononçant des discours fort enflés de vanité, ils amorcent par les convoitises de la chair, et par leurs impudicités, ceux qui s'étaient véritablement retirés de ceux qui vivent dans l'égarement~;
\VS{19}leur promettant la liberté, quoiqu'ils soient eux-mêmes esclaves de la corruption, car on est réduit dans la servitude de celui par qui on est vaincu.
\VS{20}Parce que, si, après s'être retirés des souillures du monde par la connaissance du Seigneur et Sauveur Jésus-Christ, toutefois étant de nouveau enveloppés par elles, ils en sont vaincus, leur dernière condition est pire que la première\FTNT{\vref{Mt. 12:43-45}.}.
\VS{21}Car mieux valait pour eux n'avoir pas connu la voie de la justice, que de l'avoir connue et se détourner du saint commandement qui leur avait été donné.
\VS{22}Mais ce qu'on dit par un proverbe véritable leur est arrivé~: Le chien est retourné à ce qu'il avait vomi, et la truie lavée est retournée se vautrer dans le bourbier.
\Chap{3}
\TextTitle{L'objectif de l'épître}
\VerseOne{}Mes bien-aimés, c'est ici la seconde lettre que je vous écris, afin de réveiller, dans l'une et dans l'autre, par mes avertissements, les sentiments purs que vous avez,
\VS{2}et afin que vous vous souveniez des paroles qui ont été dites auparavant par les saints prophètes, et du commandement que vous avez reçu de nous, qui sommes apôtres du Seigneur et Sauveur.
\TextTitle{Malgré les moqueries et l'incrédulité des hommes, Christ reviendra}
\VS{3}Sur toutes choses, sachez qu'aux derniers jours\FTNT{Derniers jours. Voir \vref{Ge. 49:1-2}.} il viendra des moqueurs, se conduisant selon leurs propres convoitises,
\VS{4}et disant~: Où est la promesse de son avènement~? Car depuis que les pères sont morts, toutes choses demeurent comme elles ont été dès le commencement de la création.
\VS{5}Car ils ignorent volontairement ceci: C'est que les cieux furent autrefois créés par la parole de Dieu, et que la terre est sortie de l'eau, et qu'elle subsiste parmi l'eau,
\VS{6}et que par ces choses-là, le monde d'alors périt, étant submergé par les eaux du déluge\FTNT{Le déluge. Voir \vref{Ge. 6}.}.
\VS{7}Mais les cieux et la terre d'à présent sont gardés par la même parole, étant réservés pour le feu au jour du jugement, et de la destruction des hommes impies.
\VS{8}Mais vous, mes bien-aimés, n'ignorez pas ceci, qu'un jour est devant le Seigneur comme mille ans, et mille ans comme un jour\FTNT{\vref{Ps. 90:4}.}.
\VS{9}Le Seigneur ne retarde point l'exécution de sa promesse, comme quelques-uns croient qu'il y ait du retard, mais il est patient envers nous, ne voulant qu'aucun ne périsse, mais que tous se repentent.
\TextTitle{La terre passera}
\VS{10}Or le jour\FTNT{Le jour du Seigneur. Voir \vref{Za. 14:1}.} du Seigneur viendra comme un voleur dans la nuit, et en ce jour-là, les cieux passeront avec le bruit d'une effroyable tempête, et les éléments seront dissous par l'ardeur du feu, et la terre avec toutes les œuvres qu'elle renferme sera brûlée entièrement.
\VS{11}Puisque toutes ces choses doivent se dissoudre, quelles ne doivent pas être la sainteté de votre conduite et votre piété~?
\VS{12}En attendant, et en hâtant la venue du jour de Dieu, par lequel les cieux étant enflammés seront dissous, et les éléments se fondront par l'ardeur du feu~!
\VS{13}Mais nous attendons, selon sa promesse, de nouveaux cieux et une nouvelle terre\FTNT{\vref{Es. 66:22}.}, où la justice habitera.
\VS{14}C'est pourquoi, mes bien-aimés, en attendant ces choses, appliquez-vous à être trouvés par lui sans tache et sans reproche dans la paix.
\VS{15}Et considérez la patience du Seigneur comme une preuve qu'il veut votre salut~; comme Paul, notre frère bien-aimé, vous en a écrit, selon la sagesse qui lui a été donnée,
\VS{16}ainsi que dans toutes ses lettres, il parle de ces points, dans lesquels il y a des choses difficiles à comprendre, que les ignorants et les mal affermis tordent\FTNT{\vref{Jé. 23:36}.}, comme ils tordent aussi les autres Ecritures, à leur propre perdition.
\TextTitle{Conclusion}
\VS{17}Vous donc, mes bien-aimés, puisque vous êtes déjà avertis, prenez garde qu'étant emportés avec les autres par la séduction des abominables, vous ne veniez à déchoir de votre fermeté.
\VS{18}Mais croissez dans la grâce et dans la connaissance de notre Seigneur et Sauveur Jésus-Christ. A lui soit la gloire maintenant et jusqu'au jour d'éternité~! Amen~!
\PPE{}
\end{multicols}
