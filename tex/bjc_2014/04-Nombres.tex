\ShortTitle{Nombres}\BookTitle{Nombres}\BFont
\noindent\hrulefill
{\footnotesize
\textit{
\bigskip
{\centering{}
\\Auteur : Probablement Moïse
\\(Heb. : Bamidbar)
\\Signification : Dans le désert
\\Thème : Pérégrination dans le désert
\\Date de rédaction : Env. 1450-1410 av. J.-C.\\}
}
%\bigskip
\textit{
\\Ce livre commence par le recensement des fils d'Israël et relate trente-huit des quarante années qu'ils passèrent dans le désert du Sinaï. Il couvre une période qui s'étend de la deuxième année après la sortie d'Egypte à la veille de l'entrée en Canaan, terre que Dieu avait promis de donner à la descendance d'Abraham. Ce pays où coulaient le lait et le miel s'étendait de Sidon jusqu'à Lesha, en passant par Gaza et Sodome. En plus des Cananéens, il accueillait en son sein des enfants d'Anak, les Amalécites, les Hétiens, les Jébusiens et les Amoréens.
%\bigskip
\\Ces écrits retracent les premières victoires d'Israël et regroupent diverses lois et instructions sur le partage de la terre promise. Ils témoignent également de la révolte et de l'incrédulité de la génération sortie d'Egypte dont la quasi-totalité périt dans le désert.\bigskip
}
}
\par\nobreak\noindent\hrulefill
\begin{multicols}{2}
\Chap{1}
\TextTitle{Dénombrement des hommes de guerre}
\VerseOne{}Or Yahweh parla à Moïse dans le désert de Sinaï, dans la tente d'assignation, le premier jour du second mois, la seconde année, après qu'ils furent sortis du pays d'Egypte, en disant :
\VS{2}Faites le dénombrement de toute l'assemblée des fils d'Israël, selon leurs familles, selon les maisons de leurs pères, en comptant nom par nom, savoir tous les mâles\FTNT{Ex. 30:12 ; Ex. 38:26.}, chacun par tête ;
\VS{3}depuis l'âge de vingt ans et au-dessus, tous ceux d'Israël qui peuvent aller à la guerre, vous les compterez selon leurs armées, toi et Aaron.
\VS{4}Il y aura avec vous un homme par tribu, celui qui est le chef de la maison de ses pères.
\VS{5}Voici les noms des hommes qui vous assisteront. Pour la tribu de Ruben : Elitsur, fils de Schedéur ;
\VS{6}pour celle de Siméon : Schelumiel, fils de Tsurischaddaï ;
\VS{7}pour celle de Juda : Nachschon, fils d'Amminadab ;
\VS{8}pour celle d'Issacar : Nethaneel, fils de Tsuar ;
\VS{9}pour celle de Zabulon : Eliab, fils de Hélon ;
\VS{10}pour les fils de Joseph, pour la tribu d'Ephraïm : Elischama, fils d'Ammihud ; pour celle de Manassé : Gamliel, fils de Pedahtsur ;
\VS{11}pour la tribu de Benjamin : Abidan, fils de Guideoni ;
\VS{12}pour celle de Dan : Ahiézer, fils d'Ammischaddaï ;
\VS{13}pour celle d'Aser : Paguiel, fils d'Ocran ;
\VS{14}pour celle de Gad : Eliasaph, fils de Déuel ;
\VS{15}pour celle de Nephthali : Ahira, fils d'Enan.
\VS{16}C'étaient là ceux qu'on appelait pour tenir l'assemblée ; ils étaient les princes des tribus de leurs pères, chefs des milliers d'Israël.
\VS{17}Alors Moïse et Aaron prirent ces hommes qui avaient été désignés par leurs noms,
\VS{18}et ils convoquèrent toute l'assemblée, le premier jour du second mois. On les enregistra selon leurs familles et selon la maison de leurs pères, en comptant les noms depuis l'âge de vingt ans et au-dessus, chacun par tête ;
\VS{19}Comme Yahweh l'avait commandé à Moïse, il les dénombra au désert de Sinaï.
\VS{20}Les fils donc de Ruben, premier-né d'Israël, selon leurs générations, leurs familles, et les maisons de leurs pères, dont on fit le dénombrement par leur nom, et par tête, savoir tous les mâles de l'âge de vingt ans, et au dessus, tous ceux qui pouvaient aller à la guerre.
\VS{21}Ceux, dis-je, de la tribu de Ruben, qui furent dénombrés, furent quarante-six mille cinq cents.
\VS{22}Des enfants de Siméon, selon leurs générations, leurs familles, et les maisons de leurs pères, ceux qui furent dénombrés par leur nom et par tête, savoir tous les mâles de l'âge de vingt ans, et au dessus, tous ceux qui pouvaient aller à la guerre ;
\VS{23}ceux, dis-je, de la tribu de Siméon, qui furent dénombrés, furent cinquante-neuf mille trois cents.
\VS{24}Des fils de Gad, selon leurs générations, leurs familles, et les maisons de leurs pères, dénombrés chacun par leur nom, depuis l'âge de vingt ans, et au dessus, tous ceux qui pouvaient aller à la guerre ;
\VS{25}ceux, dis-je, de la tribu de Gad, qui furent dénombrés, furent quarante-cinq mille six cent cinquante.
\VS{26}Des enfants de Juda, selon leurs générations, leurs familles, et les maisons de leurs pères, dénombrés chacun par leur nom, depuis l'âge de vingt ans, et au dessus, tous ceux qui pouvaient aller à la guerre ;
\VS{27}ceux, dis-je, de la tribu de Juda, qui furent dénombrés, furent soixante et quatorze mille six cents.
\VS{28}Des fils d'Issacar, selon leurs générations, leurs familles, et les maisons de leurs pères, dénombrés chacun par leur nom, depuis l'âge de vingt ans, et au dessus, tous ceux qui pouvaient aller à la guerre ;
\VS{29}ceux, dis-je, de la tribu d'Issacar, qui furent dénombrés, furent cinquante-quatre mille quatre cents.
\VS{30}Des enfants de Zabulon, selon leurs générations, leurs familles, et les maisons de leurs pères, dénombrés chacun par leur nom, depuis l'âge de vingt ans, et au dessus, tous ceux qui pouvaient aller à la guerre ;
\VS{31}ceux, dis-je, de la tribu de Zabulon, qui furent dénombrés, furent cinquante-sept mille quatre cents.
\VS{32}Quant aux fils de Joseph ; les fils d'Ephraïm, selon leurs générations, leurs familles, et les maisons de leurs pères, dénombrés chacun par leur nom, depuis l'âge de vingt ans, et au dessus, tous ceux qui pouvaient aller à la guerre ;
\VS{33}ceux, dis-je, de la tribu d'Ephraïm, qui furent dénombrés, furent quarante mille cinq cents.
\VS{34}Des fils de Manassé, selon leurs générations, leurs familles, et les maisons de leurs pères, dénombrés chacun par leur nom, depuis l'âge de vingt ans, et au dessus, tous ceux qui pouvaient aller à la guerre ;
\VS{35}ceux, dis-je, de la tribu de Manassé, qui furent dénombrés, furent trente-deux mille deux cents.
\VS{36}Des fils de Benjamin, selon leurs générations, leurs familles, et les maisons de leurs pères, dénombrés chacun par leur nom, depuis l'âge de vingt ans, et au dessus, tous ceux qui pouvaient aller à la guerre ;
\VS{37}ceux, dis-je, de la tribu de Benjamin, qui furent dénombrés, furent trente-cinq mille quatre cents.
\VS{38}Des fils de Dan, selon leurs générations, leurs familles, et les maisons de leurs pères, dénombrés chacun par leur nom, depuis l'âge de vingt ans, et au dessus, tous ceux qui pouvaient aller à la guerre ;
\VS{39}ceux, dis-je, de la tribu de Dan qui furent dénombrés, furent soixante-deux mille sept cents.
\VS{40}Des fils d'Aser, selon leurs générations, leurs familles, et les maisons de leurs pères, dénombrés chacun par leur nom, depuis l'âge de vingt ans, et au dessus, tous ceux qui pouvaient aller à la guerre ;
\VS{41}ceux, dis-je, de la tribu d'Aser, qui furent dénombrés, furent quarante et un mille cinq cents.
\VS{42}Des fils de Nephthali, selon leurs générations, leurs familles, et les maisons de leurs pères, dénombrés chacun par leur nom, depuis l'âge de vingt ans, et au dessus, tous ceux qui pouvaient aller à la guerre ;
\VS{43}ceux, dis-je, de la tribu de Nephthali, qui furent dénombrés, furent cinquante-trois mille quatre cents.
\VS{44}Ce sont là ceux dont Moïse et Aaron firent le dénombrement, les douze princes d'entre les enfants d'Israël y étant, un pour chaque maison de leurs pères.
\VS{45}Ainsi tous ceux des enfants d'Israël, dont on fit le dénombrement, selon les maisons de leurs pères, depuis l'âge de vingt ans, et au dessus, tous ceux d'entre les Israélites, qui pouvaient aller à la guerre ;
\VS{46}tous ceux, dis-je, dont on fit le dénombrement, furent six cent trois mille cinq cent cinquante.
\VS{47}Mais les Lévites ne furent point dénombrés avec eux, selon la tribu de leurs pères.
\VS{48}Car Yahweh avait parlé à Moïse, en disant :
\VS{49}Tu ne feras aucun dénombrement de la tribu de Lévi, et tu n'en lèveras point la somme avec les autres enfants d'Israël.
\VS{50}Mais tu donneras aux Lévites la charge du tabernacle du témoignage, et de tous ses ustensiles, et de tout ce qui lui appartient ; ils porteront le tabernacle, et tous ses ustensiles ; ils y serviront, et camperont autour du tabernacle.
\VS{51}Et quand le tabernacle partira, les Lévites le démonteront, et quand le tabernacle campera, les Lévites le dresseront. Que si quelque étranger en approche, on le fera mourir\FTNT{Ez. 44:8-9.}.
\VS{52}Or les enfants d'Israël camperont chacun dans son camp, et chacun sous sa bannière, selon leurs armées.
\VS{53}Mais les Lévites camperont autour du tabernacle du Témoignage, afin qu'il n'y ait point d'indignation sur l'assemblée des enfants d'Israël, et ils prendront en leur charge le tabernacle du Témoignage.
\VS{54}Et les enfants d'Israël firent selon toutes les choses que Yahweh avait commandées à Moïse ; ils le firent ainsi.
\Chap{2}
\TextTitle{Disposition du camp d'Israël par tribu}
\VerseOne{}Et Yahweh parla à Moïse et à Aaron, en disant :
\VS{2}Les enfants d'Israël camperont chacun sous sa bannière, avec les enseignes des maisons de leurs pères, tout autour de la tente d'assignation, vis-à-vis de lui.
\VS{3}Ceux de la bannière du camp de Juda camperont droit vers l'est, selon ses armées ; et Nachschon, fils d'Amminadab, sera le chef des fils de Juda ;
\VS{4}et son armée, et ses dénombrés, soixante-quatorze mille six cents.
\VS{5}Près de lui campera la tribu d'Issacar, et Nethanaël, fils de Tsuar, sera le chef des enfants d'Issacar ;
\VS{6}et son armée, et ses dénombrés, cinquante-quatre mille quatre cents.
\VS{7}Puis la tribu de Zabulon, et Eliab, fils de Hélon, sera le chef des enfants de Zabulon ;
\VS{8}et son armée, et ses dénombrés, cinquante-sept mille quatre cents.
\VS{9}Tous les dénombrés du camp de Juda, cent quatre-vingt-six mille quatre cents, selon leurs armées, partiront les premiers.
\VS{10}La bannière du camp de Ruben, selon ses armées, sera vers le sud, et Elitsur, fils de Schedéur, sera le chef des enfants de Ruben ;
\VS{11}et son armée, et ses dénombrés, quarante-six mille cinq cents.
\VS{12}Près de lui campera la tribu de Siméon, et Schelumiel, fils de Tsurischaddaï, sera le chef des enfants de Siméon ;
\VS{13}et son armée, et ses dénombrés, cinquante-neuf mille trois cents.
\VS{14}Puis la tribu de Gad, et Eliasaph, fils de Déuel, sera le chef des enfants de Gad ;
\VS{15}et son armée, et ses dénombrés, quarante-cinq mille six cent cinquante.
\VS{16}Tous les dénombrés du camp de Ruben, cent cinquante et un mille quatre cent cinquante, selon leurs armées, partiront les seconds.
\VS{17}Ensuite la tente d'assignation partira avec le camp des Lévites, au milieu des camps qui partiront comme ils auront campés, chacune en sa place, selon leurs bannières.
\VS{18}La bannière du camp d'Ephraïm, selon ses armées, sera vers l'occident ; et Elischama, fils de Ammihud, sera le chef des enfants d'Ephraïm ;
\VS{19}et son armée, et ses dénombrés, quarante mille cinq cents.
\VS{20}Près de lui campera la tribu de Manassé, et Gamliel, fils de Pedahtsur, sera le chef des fils de Manassé ;
\VS{21}et son armée, et ses dénombrés, trente-deux mille deux cents.
\VS{22}Puis la tribu de Benjamin, et Abidan, fils de Guideoni, sera le chef des fils de Benjamin ;
\VS{23}et son armée, et ses dénombrés, trente-cinq mille et quatre cents.
\VS{24}Tous les dénombrés pour le camp d'Ephraïm, cent huit mille et cent, selon leurs armées, partiront les troisièmes.
\VS{25}La bannière du camp de Dan, selon ses armées, sera vers le nord, et Ahiézer, fils de Ammischaddaaï, sera le chef des fils de Dan ;
\VS{26}et son armée, et ses dénombrés, soixante-deux mille sept cents.
\VS{27}Près de lui campera la tribu d'Aser, et Paguiel, fils de Ocran, sera le chef des fils d'Aser ;
\VS{28}et son armée, et ses dénombrés, quarante et un mille cinq cents.
\VS{29}Puis la tribu de Nephthali, et Ahira, fils d'Enan, sera le chef des fils de Nephthali ;
\VS{30}Et son armée, et ses dénombrés, cinquante-trois mille quatre cents.
\VS{31}Tous les dénombrés du camp de Dan, cent cinquante-sept mille six cents, partiront les derniers des bannières.
\VS{32}Ce sont là ceux des enfants d'Israël dont on fit le dénombrement selon les maisons de leurs pères. Tous les dénombrés des camps selon leurs armées ; furent six cent trois mille cinq cent cinquante.
\VS{33}Mais les Lévites ne furent point dénombrés avec les autres enfants d'Israël, comme Yahweh l'avait commandé à Moïse.
\VS{34}Et les enfants d'Israël firent selon toutes les choses que Yahweh avait commandées à Moïse, et campèrent ainsi selon leurs bannières, et partirent ainsi, chacun selon leurs familles, et selon la maison de leurs pères.
\Chap{3}
\TextTitle{Organisation des sacrificateurs et des Lévites}
\VerseOne{}Or ce sont ici les générations d'Aaron et de Moïse, au temps que Yahweh parla à Moïse sur la montagne de Sinaï.
\VS{2}Et ce sont ici les noms des fils d'Aaron ; Nadab, qui était l'aîné, Abihu, Eléazar, et Ithamar.
\VS{3}Ce sont là les noms des fils d'Aaron, les sacrificateurs, qui furent oints et consacrés pour exercer la sacrificature\FTNT{Ex. 40:15 ; Lé. 8:30.}.
\VS{4}Mais Nadab et Abihu moururent en la présence de Yahweh, quand ils offrirent un feu étranger devant Yahweh au désert de Sinaï, et ils n'eurent point d'enfants ; mais Eléazar et Ithamar exercèrent la sacrificature en la présence d'Aaron leur père\FTNT{Lé. 10:1-2 ; 1 Ch. 24:2.}.
\VS{5}Yahweh parla à Moïse, en disant :
\VS{6}Fais approcher la tribu de Lévi, et fais qu'elle se tienne devant Aaron, le sacrificateur, afin qu'ils le servent.
\VS{7}Et qu'ils aient la charge de ce qu'il leur ordonnera de garder, et de ce que toute l'assemblée leur ordonnera de garder, devant la tente d'assignation, en faisant le service du tabernacle.
\VS{8}Et qu'ils gardent tous les ustensiles de la tente d'assignation, et ce qui leur sera donné en charge par les enfants d'Israël, pour faire le service du tabernacle.
\VS{9}Ainsi tu donneras les Lévites à Aaron et à ses fils ; ils lui sont complétement donnés d'entre les enfants d'Israël.
\VS{10}Tu établiras donc Aaron et ses fils, et ils exerceront leur sacrificature. Que si quelque étranger en approche, on le fera mourir.
\VS{11}Et Yahweh parla à Moïse, en disant :
\VS{12}Voici, j'ai pris les Lévites d'entre les enfants d'Israël, à la place de tout premier-né qui ouvre la matrice parmi les enfants d'Israël ; c'est pourquoi les Lévites seront à moi.
\VS{13}Car tout premier-né m'appartient, depuis le jour où je frappai tout premier-né au pays d'Egypte ; je me suis sanctifié tout premier-né en Israël, depuis les hommes jusqu'aux bêtes ; ils seront à moi, je suis Yahweh\FTNT{Ex. 13:2 ; Ex. 22:29 ; Ex. 34:19 ; Lé. 27:26.}.
\TextTitle{Les familles des Lévites}
\VS{14}Yahweh parla aussi à Moïse au désert de Sinaï, en disant :
\VS{15}Dénombre les enfants de Lévi, par les maisons de leurs pères, et par leurs familles, en comptant tout mâle depuis l'âge d'un mois, et au dessus.
\VS{16}Et Moïse les dénombra, selon le commandement de Yahweh, ainsi qu'il lui avait été ordonné.
\VS{17}Or ce sont ici les fils de Lévi selon leurs noms : Guerschon, Kehath, et Merari.
\VS{18}Et ce sont ici les noms des fils de Guerschon, selon leurs familles, Libni, et Schimeï.
\VS{19}Et les fils de Kehath selon leurs familles, Amram, Jitsehar, Hébron et Uziel ;
\VS{20}et les fils de Merari, selon leurs familles, Machli et Muschi ; ce sont là les familles de Lévi, selon les maisons de leurs pères.
\VS{21}De Guerschon est sortie la famille de Libni, et la famille de Schimeï ; ce sont les familles des Guerschonites.
\VS{22}Ceux dont on fit le dénombrement, en comptant de tous les mâles depuis l'âge d'un mois et au dessus, furent au nombre de sept mille cinq cents.
\VS{23}Les familles des Guerschonites camperont derrière le tabernacle à l'occident.
\VS{24}Et Eliasaph, fils de Laël, sera le chef de la maison des pères des Guerschonites.
\TextTitle{Les fonctions des Lévites}
\VS{25}Et les fils de Guerschon auront en charge à la tente d'assignation, la tente, le tabernacle, sa couverture, le rideau de l'entrée de la tente d'assignation.
\VS{26}Et les courtines du parvis avec le rideau de l'entrée du parvis, qui servent pour tabernacle et pour l'autel, tout autour, et son cordage, pour tout son service.
\VS{27}Et de Kehath est sortie la famille des Amramites, la famille des Jitseharites, la famille des Hébronites, et la famille des Uziélites ; ce furent là les familles des Kehathites,
\VS{28}dont tous les mâles depuis l'âge d'un mois, et au dessus, furent au nombre de huit mille six cents, ayant la charge du sanctuaire.
\VS{29}Les familles des fils de Kehath camperont du côté du tabernacle vers le sud.
\VS{30}Et Elitsaphan, fils d'Uziel, sera le chef de la maison des pères des familles des Kehathites.
\VS{31}Et ils auront en charge l'arche, la table, le chandelier, les autels, et les ustensiles du sanctuaire avec lesquels on fait le service, et le rideau, avec tout ce qui y sert.
\VS{32}Et le chef des chefs des Lévites sera Eléazar, fils d'Aaron, le sacrificateur ; qui aura la surveillance sur ceux qui auront la charge du sanctuaire.
\VS{33}Et de Merari est sortie la famille des Machlites, et la famille des Muschi ; ce furent là les familles de Merari ;
\VS{34}ceux dont on fit le dénombrement, après le compte qui fut fait de tous les mâles, depuis l'âge d'un mois et au dessus, furent six mille deux cents.
\VS{35}Et Esuriel, fils d'Abihaïl, sera le chef de la maison des pères des familles des Merarites ; ils camperont du côté du tabernacle vers le nord.
\VS{36}Et on donnera aux enfants de Merari la surveillance des planches du tabernacle, de ses barres, de ses piliers, de ses bases, et de tous ses ustensiles, avec tout ce qui y sera ;
\VS{37}et des piliers du parvis tout autour, avec leurs bases, leurs pieux, et leurs cordes.
\VS{38}Et Moïse, et Aaron et ses fils campaient devant le tabernacle, à l'orient, devant la tente d'assignation, vers l'orient ; ils avaient la garde et le soin du sanctuaire, remis à la garde des enfants d'Israël ; et si quelque étranger en approche, on le fera mourir.
\VS{39}Tous ceux des Lévites dont on fit le dénombrement, lesquels Moïse et Aaron comptèrent par leurs familles, suivant le commandement de Yahweh, tous les mâles de l'âge d'un mois et au dessus, furent de vingt-deux mille.
\TextTitle{Le rachat des premiers-nés}
\VS{40}Yahweh dit à Moïse : Fais le dénombrement de tous les premiers-nés mâles des enfants d'Israël, depuis l'âge d'un mois, et au dessus, et relève le nombre de leurs noms.
\VS{41}Et tu prendras pour moi, je suis Yahweh, les Lévites, à la place de tous les premiers-nés qui sont entre les enfants d'Israël ; tu prendras aussi les bêtes des Lévites, à la place de tous les premiers-nés des bêtes des enfants d'Israël.
\VS{42}Moïse fit le dénombrement, comme Yahweh lui avait commandé, de tous les premiers-nés qui étaient parmi les enfants d'Israël.
\VS{43}Et tous les premiers-nés des mâles, selon le nombre des noms, depuis l'âge d'un mois et au dessus, selon leur dénombrement, furent vingt-deux mille deux cent soixante-treize.
\VS{44}Et Yahweh parla à Moïse, en disant :
\VS{45}Prends les Lévites à la place de tous les premiers-nés qui sont parmi les enfants d'Israël, et les bêtes des Lévites, à la place de leurs bêtes ; et les Lévites seront à moi ; je suis Yahweh.
\VS{46}Et quant à ceux qu'il faut racheter, les deux cent soixante-treize parmi les premiers-nés des fils d'Israël, qui sont de plus que les Lévites,
\VS{47}tu prendras cinq sicles par tête, tu les prendras selon le sicle du sanctuaire ; le sicle est de vingt guéras\FTNT{Ex. 30:13; Lé. 27:6 ; Lé. 27:25 ; Ez. 45:12.}.
\VS{48}Et tu donneras à Aaron et à ses fils l'argent de ceux qui auront été rachetés, dépassant le nombre des Lévites.
\VS{49}Moïse donc prit l'argent du rachat de ceux qui étaient de plus, outre ceux qui avaient été rachetés par l'échange des Lévites.
\VS{50}Et il reçut l'argent des premiers-nés des enfants d'Israël, qui fut mille trois cent soixante-cinq sicles, selon le sicle du Sanctuaire.
\VS{51}Et Moïse donna l'argent des rachetés à Aaron, et à ses fils, selon le commandement de Yahweh, ainsi que Yahweh le lui avait commandé.
\Chap{4}
\TextTitle{Les fonctions des fils de Kehath}
\VerseOne{}Et Yahweh parla à Moïse et à Aaron, en disant :
\VS{2}Faites le dénombrement des fils de Kehath d'entre les enfants de Lévi par leurs familles, et par les maisons de leurs pères,
\VS{3}depuis l'âge de trente ans et au dessus, jusqu'à l'âge de cinquante ans, tous ceux qui entrent en rang, pour s'employer à la tente d'assignation.
\VS{4}C'est ici le service des fils de Kehath à la tente d'assignation, c'est-à-dire, le Saint des saints.
\VS{5}Quand le camp partira, Aaron et ses fils viendront démonter le voile\FTNT{Le voile intérieur est l'image du corps humain de Christ (Mt. 26:26). Ce voile fut déchiré de haut en bas lorsque le Seigneur est mort sur la croix (Mt. 27:50-51). Désormais, le croyant peut pénétrer dans la présence du Père (Hé. 10:19-20).} qui sert de rideau, et en couvriront l'arche du témoignage ;
\VS{6}puis ils mettront au dessus une couverture de peaux de béliers, ils étendront par dessus un drap de pourpre, et ils y mettront ses barres.
\VS{7}Et ils étendront un drap de pourpre sur la table des pains de proposition, et mettront sur elle les plats, les tasses, les bassins, et les calices de libations. Le pain continuel sera sur elle.
\VS{8}Ils étendront au dessus un drap teint de cramoisi, ils le couvriront d'une couverture de peaux de béliers, et ils y mettront ses barres.
\VS{9}Et ils prendront un drap de pourpre, et en couvriront le chandelier du luminaire avec ses lampes, ses mouchettes, ses vases à cendre, et tous ses vases à huile, dont on fait usage pour son service\FTNT{Ex. 25:30-38.} ;
\VS{10}et ils le mettront avec tous ses ustensiles, dans une couverture de peaux de béliers, et le mettront sur une perche.
\VS{11}Ils étendront sur l'autel d'or un drap de pourpre, ils le couvriront d'une couverture de peaux de béliers, et ils y mettront ses barres.
\VS{12}Ils prendront aussi tous les ustensiles du service dont on se sert dans le lieu saint, ils les mettront dans un drap de pourpre, et ils les couvriront d'une couverture de peaux de béliers, et les mettront sur des perches.
\VS{13}Ils ôteront les cendres de l'autel, et étendront dessus un drap de pourpre.
\VS{14}Et ils mettront dessus les ustensiles dont on se sert pour l'autel, les brasiers, les fourchettes, les pelles, les bassins, et tous les ustensiles de l'autel ; ils étendront dessus une couverture de peaux de béliers, et ils y mettront ses barres.
\VS{15}Le camp partira après qu'Aaron et ses fils auront achevé de couvrir le lieu saint et tous ses ustensiles, et après cela les filss de Kehath viendront pour le porter, et ils ne toucheront point les choses saintes, de peur qu'ils ne meurent ; c'est là ce que les filss de Kehath porteront de la tente d'assignation.
\TextTitle{Les fonctions d'Eléazar}
\VS{16}Et Eléazar fils d'Aaron, le sacrificateur, aura la surveillance de l'huile du luminaire, du parfum odoriférant, de l'offrande continuelle, et de l'huile de l'onction ; la charge de tout le tabernacle, et de toutes les choses qui sont dans le lieu saint, et de ses ustensiles\FTNT{Ex. 30:23-35.}.
\VS{17}Yahweh parla à Moïse et à Aaron, en disant :
\VS{18}Ne retranchez pas la tribu des familles des Kehathites d'entre les Lévites.
\VS{19}Mais faites ceci pour eux, afin qu'ils vivent et ne meurent point ; c'est que quand ils approcheront du Saint des saints, Aaron et ses fils viendront, qui les placeront chacun à son service, et à sa charge.
\VS{20}Et ils n'entreront point pour regarder quand on enveloppera les choses saintes, afin qu'ils ne meurent point.
\TextTitle{Les fonctions des fils de Guerschon}
\VS{21}Yahweh parla à Moïse, en disant :
\VS{22}Fais aussi le dénombrement des fils de Guerschon selon les maisons de leurs pères, et selon leurs familles ;
\VS{23}depuis l'âge de trente ans, et au dessus, jusqu'à l'âge de cinquante ans, dénombrant tous ceux qui entrent pour tenir leur rang, afin de s'employer à servir à la tente d'assignation.
\VS{24}C'est ici le service des familles des Guerschonites, en ce à quoi ils doivent servir, et en ce qu'ils doivent porter.
\VS{25}Ils porteront donc les tapis du tabernacle, et la tente d'assignation, sa couverture, la couverture de peaux de béliers qui est sur lui par dessus, et le rideau de l'entrée de la tente d'assignation ;
\VS{26}les courtines du parvis, et le rideau de l'entrée de la porte du parvis, qui servent pour le tabernacle et pour l'autel tout autour, leurs cordages, et tous les ustensiles de leur service, et tout ce qui est fait pour eux ; c'est ce en quoi ils serviront.
\VS{27}Tout le service des fils de Guerschonites en tout ce qu'ils doivent porter, et en tout ce à quoi ils doivent servir, sera réglé par les ordres d'Aaron et de ses fils, et vous les chargerez d'observer tout ce qu'ils doivent porter.
\VS{28}C'est là le service des familles des fils des Guerschonites dans la tente d'assignation ; et leur charge sera sous la conduite d'Ithamar, fils d'Aaron, le sacrificateur.
\TextTitle{Les fonctions des fils de Merari}
\VS{29}Tu dénombreras aussi lesfils de Mérari selon leurs familles et selon les maisons de leurs pères.
\VS{30}Tu les dénombreras depuis l'âge de trente ans et au dessus, jusqu'à l'âge de cinquante ans, tous ceux qui entrent en rang pour s'employer au service dans la tente d'assignation.
\VS{31}Or c'est ici la charge de ce qu'ils auront à porter, selon tout le service qu'ils auront à faire à la tente d'assignation, savoir les planches du tabernacle, ses barres, et ses piliers, avec ses bases\FTNT{Ex. 26:15.},
\VS{32}et les piliers du parvis tout autour, et leurs bases, leurs pieux, leurs cordages, tous leurs ustensiles, et tout ce dont on se sert en ces choses-là, et vous leur compterez, en les désignant par nom, tous les ustensiles qu'ils auront charge de porter, pièce par pièce.
\VS{33}C'est là le service des familles des fils de Merari, pour tout leur service à la tente d'assignation, sous la conduite d'Ithamar, fils d'Aaron, le sacrificateur.
\VS{34}Moïse donc et Aaron, et les princes de l'assemblée dénombrèrent les fils des Kéhathites, selon leurs familles, et selon les maisons de leurs pères.
\VS{35}Depuis l'âge de trente ans, et au dessus, jusqu'à l'âge de cinquante ans, tous ceux qui entraient en rang pour servir à la tente d'assignation.
\VS{36}Et ceux dont on fit le dénombrement selon leurs familles, étaient deux mille sept cent cinquante.
\VS{37}Ce sont là les dénombrés des familles des Kéhathites, tous servant à la tente d'assignation, que Moïse et Aaron dénombrèrent selon le commandement que Yahweh avait fait par le moyen de Moïse.
\VS{38}Or quant aux dénombrés des fils de Guerschon selon leurs familles, et selon les maisons de leurs pères,
\VS{39}depuis l'âge de trente ans, et au dessus, jusqu'à l'âge de cinquante ans, tous ceux qui entraient en rang pour servir à la tente d'assignation,
\VS{40}ceux, dis-je, qui en furent dénombrés selon leurs familles, et selon les maisons de leurs pères, étaient deux mille six cent trente.
\VS{41}Ce sont là les dénombrés des familles des fils de Guerschon, tous servant dans la tente d'assignation, que Moïse et Aaron dénombrèrent selon le commandement de Yahweh.
\VS{42}Et quant aux dénombrés des familles des fils de Merari, selon leurs familles, et selon les maisons de leurs pères,
\VS{43}depuis l'âge de trente ans, et au dessus, jusqu'à l'âge de cinquante ans, tous ceux qui entraient en rang, pour servir à la tente d'assignation ;
\VS{44}ceux, dis-je, qui en furent dénombrés selon leurs familles, étaient trois mille deux cents.
\VS{45}Ce sont là les dénombrés des familles des fils de Merari, que Moïse et Aaron dénombrèrent selon le commandement que Yahweh avait fait par le moyen de Moïse.
\VS{46}Ainsi tous ces dénombrés, que Moïse et Aaron et les princes d'Israël dénombrèrent d'entre les Lévites, selon leurs familles, et selon les maisons de leurs pères ;
\VS{47}depuis l'âge de trente ans, et au dessus, jusqu'à l'âge de cinquante ans, tous ceux qui entraient en service pour s'employer en ce à quoi il fallait servir, et à ce qu'il fallait porter de la tente d'assignation.
\VS{48}Tous ceux, dis-je, qui en furent dénombrés, étaient huit mille cinq cent quatre-vingts.
\VS{49}On les dénombra selon le commandement que Yahweh en avait fait par le moyen de Moïse, chacun selon ce en quoi il avait à servir, et ce qu'il avait à porter, et la charge de chacun fut telle que Yahweh l'avait commandé à Moïse.
\Chap{5}
\TextTitle{Mise en garde contre toute souillure ; lois diverses}
\VerseOne{}Et Yahweh parla à Moïse, en disant :
\VS{2}Commande aux enfants d'Israël qu'ils mettent hors du camp tout lépreux, tout homme ayant une gonorrhée, et tout homme souillé pour un mort\FTNT{Lé. 13 ; Lé. 15.}.
\VS{3}Vous les mettrez dehors, tant l'homme que la femme, vous les mettrez, dis-je, hors du camp, afin qu'ils ne souillent point le camp au milieu desquels j'habite.
\VS{4}Et les enfants d'Israël firent ainsi, et les envoyèrent hors du camp, comme Yahweh l'avait dit à Moïse ; les enfants d'Israël firent ainsi.
\VS{5}Et Yahweh parla à Moïse, en disant :
\VS{6}Parle aux enfants d'Israël ; quand un homme ou une femme aura commis un des péchés que l'homme commet en faisant un crime contre Yahweh, et qu'une telle personne en sera trouvée coupable ;
\VS{7}alors ils confesseront leur péché, qu'ils auront commis ; et le coupable restituera la somme totale de ce en quoi il aura été trouvé coupable, et il y ajoutera un cinquième par-dessus, et le donnera à celui contre qui il aura commis le délit.
\VS{8}Que si cet homme n'a personne à qui appartienne le droit de restituer pour retirer ce en quoi aura été commis le délit, cette chose-là sera restituée à Yahweh, et elle appartiendra au sacrificateur, outre le bélier expiatoire avec lequel on fera propitiation pour lui.
\VS{9}De même, toute offrande élevée d'entre toutes les choses sanctifiées des enfants d'Israël, qu'ils présenteront\FTNT{Ez. 44:30.} au sacrificateur, lui appartiendra.
\VS{10} Les choses donc que quelqu'un aura sanctifiées appartiendront au sacrificateur ; ce que chacun lui aura donné, lui appartiendra\FTNT{Lé. 10:12-13.}.
\VS{11}Yahweh parla à Moïse, en disant :
\VS{12}Parle aux enfants d'Israël, et dis leur : Si la femme de quelqu'un se détourne et lui devienne infidèle ;
\VS{13}et que quelqu'un aura couché avec elle, et l'aura connue, sans que son mari en ait rien su, mais qu'elle se soit cachée, et qu'elle se soit souillée, et qu'il n'y ait point de témoin contre elle, et qu'elle n'ait point été surprise ;
\VS{14}et que l'esprit de jalousie saisisse son mari, tellement qu'il soit jaloux de sa femme, parce qu'elle s'est souillée ; ou que l'esprit de jalousie le saisisse tellement, qu'il soit jaloux de sa femme, encore qu'elle ne se soit point souillée ;
\VS{15}cet homme-là fera venir sa femme devant le sacrificateur, et il apportera l'offrande de cette femme pour elle, savoir la dixième partie d'un epha de farine d'orge ; mais il ne répandra point d'huile dessus ; et il n'y mettra point d'encens ; car c'est un gâteau de jalousie, un gâteau de mémorial, pour remettre en mémoire l'iniquité\FTNT{Lé. 5:11.}.
\VS{16}Le sacrificateur la fera approcher et la fera tenir debout devant Yahweh.
\VS{17}Puis le sacrificateur prendra de l'eau sainte dans un vase de terre, et il prendra de la poussière qui sera sur le sol du tabernacle, et la mettra dans l'eau.
\VS{18}Ensuite le sacrificateur fera tenir debout la femme devant Yahweh, il découvrira la tête de cette femme, et lui posera sur les paumes des mains le gâteau de mémorial, le gâteau de jalousie ; le sacrificateur tiendra dans sa main les eaux amères, qui apportent la malédiction.
\VS{19}Et le sacrificateur fera jurer la femme et lui dira : Si aucun homme n'a couché avec toi, et si étant sous la puissance de ton mari tu ne t'es point détournée et souillée, sois exempte du mal de ces eaux amères qui apportent la malédiction.
\VS{20}Mais si, étant sous la puissance de ton mari, tu t'es détournée et souillée, et si un autre homme que ton mari a couché avec toi,
\VS{21}alors le sacrificateur fera jurer la femme avec un serment d'imprécation et lui dira : Que Yahweh te livre à la malédiction et à l'exécration au milieu de ton peuple, en faisant flétrir ta cuisse et enfler ton ventre,
\VS{22}et que ces eaux qui apportent la malédiction, entrent dans tes entrailles pour te faire enfler le ventre et flétrir ta cuisse ! Alors la femme répondra : Amen ! Amen !
\VS{23}Ensuite le sacrificateur écrira dans un livre ces imprécations, et les effacera avec les eaux amères.
\VS{24}Et il fera boire à la femme les eaux amères qui apportent la malédiction, et les eaux qui apportent la malédiction entreront en elle pour être amères.
\VS{25}Le sacrificateur donc prendra des mains de la femme le gâteau de jalousie, et l'agitera de côté et d'autre devant Yahweh, et l'offrira sur l'autel ;
\VS{26}le sacrificateur prendra une poignée de cette offrande comme mémorial, et il la brûlera sur l'autel. C'est après cela qu'il fera boire les eaux à la femme.
\VS{27}Et après qu'il lui aura fait boire les eaux, s'il est vrai qu'elle se soit souillée et qu'elle a été infidèle à son mari, les eaux qui apportent la malédiction entreront en elle et lui seront amères, et son ventre enflera, sa cuisse se flétrira, et cette femme sera assujettie à l'exécration du serment au milieu de son peuple.
\VS{28}Mais si la femme ne s'est point souillée, mais qu'elle soit pure, elle sera reconnue innocente et aura des enfants.
\VS{29}Telle est la loi sur la jalousie, quand la femme qui est sous la puissance de son mari se détourne et se souille,
\VS{30}ou quand un mari saisi d'un esprit de jalousie a des soupçons sur sa femme : Le sacrificateur la fera tenir debout devant Yahweh et fera à l'égard de cette femme tout ce qui est ordonné par cette loi.
\VS{31}Le mari sera exempt de faute, mais cette femme portera son iniquité.
\Chap{6}
\TextTitle{Le vœu de naziréat}
\VerseOne{}Yahweh parla à Moïse, en disant :
\VS{2}Parle aux enfants d'Israël, et dis-leur : Lorsqu'un homme ou une femme se consacrera en faisant un vœu de naziréat pour se consacrer à Yahweh,
\VS{3}il s'abstiendra de vin et de boisson forte, il ne boira ni vinaigre fait de vin, ni vinaigre fait avec une boisson forte ; il ne boira d'aucune liqueur de raisins, et il ne mangera point de raisins, frais ou secs.
\VS{4}Durant tous les jours de son naziréat il ne mangera d'aucun fruit de la vigne, depuis les pépins jusqu'à la peau du raisin\FTNT{Jg. 13:7 ; Lu. 1:15.}.
\VS{5}Le rasoir ne passera point sur sa tête durant tous les jours de son naziréat. Il sera saint jusqu'à ce que les jours pour lesquels il s'est consacré à Yahweh soient accomplis, et il laissera croître les cheveux de sa tête\FTNT{Jg. 13:5 ; 1 S. 1:11.}.
\VS{6}Durant tous les jours pour lesquels il s'est consacré à Yahweh il ne s'approchera d'aucune personne morte\FTNT{Lé. 21:1-4.} ;
\VS{7}il ne se souillera point à la mort de son père, ni de sa mère, ni de son frère, ni de sa sœur, car il porte sur sa tête la consécration de son Dieu.
\VS{8}Durant tous les jours de son naziréat, il sera consacré à Yahweh.
\VS{9}Que si quelqu'un vient à mourir subitement près de lui, la tête de son naziréat sera souillée, et il rasera sa tête au jour de sa purification, il la rasera le septième jour.
\VS{10}Le huitième jour, il apportera au sacrificateur deux tourterelles ou deux pigeonneaux, à l'entrée de la tente d'assignation\FTNT{Lé. 1 ; Lé. 12:6.}.
\VS{11}Et le sacrificateur en sacrifiera l'un pour le sacrifice d'expiation et l'autre en holocauste, et il fera propitiation pour lui de ce qu'il a péché à l'occasion du mort. Il sanctifiera donc ainsi sa tête en ce jour-là.
\VS{12}Et il séparera à Yahweh les jours de son naziréat, offrant un agneau d'un an pour le délit, et les premiers jours seront comptés pour rien, car son naziréat a été souillé.
\VS{13}Or c'est ici la loi du naziréen. Lorsque les jours de son naziréat seront accomplis, on le fera venir à la porte de la tente d'assignation.
\VS{14}Il présentera son offrande à Yahweh : Un agneau d'un an et sans défaut pour l'holocauste, une brebis d'un an et sans défaut pour le sacrifice d'expiation, et un bélier sans défaut pour le sacrifice d'offrande de paix\FTNT{Voir commentaire en Lé. 3:1.}.
\VS{15}Une corbeille de pains sans levain, de gâteaux de fine farine, pétrie à l'huile, et de galettes sans levain, oints d'huile, avec leur gâteau, et leurs libations ;
\VS{16}lesquels le sacrificateur offrira devant Yahweh ; il sacrifiera aussi son offrande pour le péché, et son holocauste.
\VS{17}Et il offrira le bélier en sacrifice d'offrande de paix à Yahweh, avec la corbeille des pains sans levain ; le sacrificateur offrira aussi son gâteau, et sa libation.
\VS{18}Et le naziréen rasera la tête de son naziréat à l'entrée de la tente d'assignation, et prendra les cheveux de la tête de son naziréat, et les mettra sur le feu qui est sous le sacrifice d'offrande de paix.
\VS{19}Et le sacrificateur prendra l'épaule cuite du bélier, et un gâteau sans levain de la corbeille, et une galette sans levain, et les mettra sur les paumes des mains du naziréen, après qu'il se sera fait raser son naziréat.
\VS{20}Et le sacrificateur les agitera de côté et d'autre devant Yahweh : C'est une chose sainte qui appartient au sacrificateur, avec la poitrine agitée et l'épaule offerte par élévation. Et après cela le naziréen boira du vin.
\FTNT{Lé. 7:32-34 ; Ex. 29:24-27.}.
\VS{21}Telle est la loi du naziréen qui aura voué à Yahweh son offrande pour son naziréat, outre ce qu'il aura encore moyen d'offrir ; il fera selon son vœu qu'il aura voué, suivant la loi de son naziréat.
\TextTitle{Aaron et ses fils bénissent Israël}
\VS{22}Yahweh parla à Moïse, en disant :
\VS{23}Parle à Aaron et à ses fils, et dis-leur : Vous bénirez ainsi les enfants d'Israël, en leur disant :
\VS{24}Yahweh te bénisse, et te garde !
\VS{25}Yahweh fasse luire sa face sur toi, et te fasse grâce\FTNT{Ps. 67:2 ; Ps. 119:135.} !
\VS{26}Yahweh tourne sa face vers toi, et te donne la paix !
\VS{27}Ils mettront donc mon Nom sur les enfants d'Israël, et je les bénirai.
\Chap{7}
\TextTitle{Les offrandes des princes}
\VerseOne{}Or il arriva le jour que Moïse eut achevé de dresser le tabernacle, et qu'il l'eut oint et sanctifié avec tous ses ustensiles, de même que l'autel avec tous ses ustensiles, il arriva, dis-je, après qu'il les eut oints et sanctifiés ;
\VS{2}que les princes d'Israël, et les chefs des maisons de leurs pères, qui sont les princes des tribus, et qui avaient assisté à faire les dénombrements, firent leur offrande.
\VS{3}Et ils amenèrent leur offrande devant Yahweh : Six chariots couverts et douze bœufs ; chaque chariot pour deux des princes, et chaque bœuf pour chacun d'eux ; ils les offrirent devant le tabernacle.
\VS{4}Alors Yahweh parla à Moïse, en disant :
\VS{5}Prends d'eux ces choses, et elles seront employées pour le service de la tente d'assignation ; et tu les donneras aux Lévites, à chacun selon ses fonctions.
\VS{6}Moïse donc prit les chariots et les bœufs, et il les remit aux Lévites.
\VS{7}Il donna aux fils de Guerschon deux chariots et quatre bœufs, selon leurs fonctions.
\VS{8}Mais il donna aux fils de Merari quatre chariots et huit bœufs, selon leurs fonctions, sous la conduite d'Ithamar, fils d'Aaron, le sacrificateur.
\VS{9}Or il n'en donna point aux fils de Kehath, parce que le service du sanctuaire était de leur charge ; ils portaient ces choses saintes sur les épaules.
\VS{10}Et les princes présentèrent leur offrande pour la dédicace de l'autel, le jour où on l'oignit ; les princes, dis-je, présentèrent leur offrande devant l'autel.
\VS{11}Et Yahweh dit à Moïse : Un des princes offrira un jour, et un autre l'autre jour, son offrande pour la dédicace de l'autel.
\VS{12}Le premier jour donc, Nachschon, fils d'Amminadab, présenta son offrande pour la tribu de Juda.
\VS{13}Il offrit un plat d'argent du poids de cent trente sicles, un bassin d'argent de soixante-dix sicles, selon le sicle du sanctuaire, tous deux pleins de fine farine pétrie à l'huile, pour l'offrande ;
\VS{14}une coupe d'or de dix sicles pleine de parfum ;
\VS{15}un jeune taureau, un bélier, un agneau d'un an, pour l'holocauste ;
\VS{16}un jeune bouc pour le sacrifice d'expiation;
\VS{17}et pour le sacrifice d'offrande de paix, deux bœufs, cinq béliers, cinq boucs, et cinq agneaux d'un an. Telle fut l'offrande de Nachschon, fils d'Amminadab.
\VS{18}Le second jour, Nethaneel, fils de Tsuar, chef de la tribu d'Issacar, présenta son offrande.
\VS{19}Et il offrit pour son offrande un plat d'argent du poids de cent trente sicles, un bassin d'argent de soixante-dix sicles, selon le sicle du sanctuaire, tous deux pleins de fine farine pétrie à l'huile, pour l'offrande ;
\VS{20}une coupe d'or de dix sicles pleine de parfum ;
\VS{21}un jeune taureau, un bélier, un agneau d'un an, pour l'holocauste ;
\VS{22}un jeune bouc pour le sacrifice d'expiation ;
\VS{23}et pour le sacrifice d'offrande de paix, deux bœufs, cinq béliers, cinq boucs, et cinq agneaux d'un an. Telle fut l'offrande de Nethaneel, fils de Tsuar.
\VS{24}Le troisième jour, Eliab, fils de Hélon, chef des fils de Zabulon, présenta son offrande.
\VS{25}Il offrit un plat d'argent du poids de cent trente sicles, un bassin d'argent de soixante-dix sicles, selon le sicle du sanctuaire, tous deux pleins de fine farine pétrie à l'huile, pour l'offrande ;
\VS{26}une coupe d'or de dix sicles pleine de parfum ;
\VS{27}un jeune taureau, un bélier, un agneau d'un an, pour l'holocauste ;
\VS{28}un jeune bouc pour le sacrifice d'expiation ;
\VS{29}et pour le sacrifice d'offrande de paix, deux bœufs, cinq béliers, cinq boucs, et cinq agneaux d'un an. Telle fut l'offrande d'Eliab, fils de Hélon.
\VS{30}Le quatrième jour, Elitsur, fils de Schedéur, prince des fils de Ruben, présenta son offrande.
\VS{31}Il offrit un plat d'argent du poids de cent trente sicles, un bassin d'argent de soixante-dix sicles, selon le sicle du sanctuaire, tous deux pleins de fine farine pétrie à l'huile, pour l'offrande ;
\VS{32}une coupe d'or de dix sicles pleine de parfum ;
\VS{33}un jeune taureau, un bélier, un agneau d'un an, pour l'holocauste ;
\VS{34}un jeune bouc pour le sacrifice d'expiation ;
\VS{35}et pour le sacrifice d'offrande de paix, deux bœufs, cinq béliers, cinq boucs, et cinq agneaux d'un an. Telle fut l'offrande d'Elitsur, fils de Schedéur.
\VS{36}Le cinquième jour, Schelumiel, fils de Tsurischaddaï, prince des fils de Siméon, présenta son offrande.
\VS{37}Il offrit un plat d'argent du poids de cent trente sicles, un bassin d'argent de soixante-dix sicles, selon le sicle du sanctuaire, tous deux pleins de fine farine pétrie à l'huile pour l'offrande ;
\VS{38}une coupe d'or de dix sicles pleine de parfum ;
\VS{39}un jeune taureau, un bélier, un agneau d'un an, pour l'holocauste ;
\VS{40}un jeune bouc pour le sacrifice d'expiation ;
\VS{41}et pour le sacrifice d'offrande de paix, deux bœufs, cinq béliers, cinq boucs, et cinq agneaux d'un an. Telle fut l'offrande de Schelumiel, fils de Tsurischaddaï.
\VS{42}Le sixième jour, Eliasaph, fils de Déuel, prince des fils de Gad, présenta son offrande.
\VS{43}Il offrit un plat d'argent du poids de cent trente sicles, un bassin d'argent de soixante-dix sicles, selon le sicle du sanctuaire, tous deux pleins de fine farine pétrie à l'huile pour l'offrande ;
\VS{44}une coupe d'or de dix sicles pleine de parfum ;
\VS{45}un jeune taureau, un bélier, un agneau d'un an, pour l'holocauste ;
\VS{46}un jeune bouc pour le sacrifice d'expiation ;
\VS{47}et pour le sacrifice d'offrande de paix, deux boeufs, cinq béliers, cinq boucs, et cinq agneaux d'un an. Telle fut l'offrande d'Eliasaph, fils de Déuel.
\VS{48}Le septième jour, Elischama, fils d'Ammihud, prince des fils d'Ephraïm, présenta son offrande.
\VS{49}Il offrit un plat d'argent, du poids de cent trente sicles, un bassin d'argent de soixante-dix sicles, selon le sicle du sanctuaire, tous deux pleins de fine farine pétrie à l'huile pour l'offrande ;
\VS{50}une coupe d'or de dix sicles pleine de parfum ;
\VS{51}un jeune taureau, un bélier, un agneau d'un an, pour l'holocauste ;
\VS{52}un jeune bouc pour le sacrifice d'expiation ;
\VS{53}et pour le sacrifice d'offrande de paix, deux bœufs, cinq béliers, cinq boucs, et cinq agneaux d'un an. Telle fut l'offrande d'Elischama, fils d'Ammihud.
\VS{54}Le huitième jour, Gamliel, fils de Pedahtsur, prince des fils de Manassé, présenta son offrande.
\VS{55}Il offrit un plat d'argent, du poids de cent trente sicles, un bassin d'argent de soixante-dix sicles, selon le sicle du sanctuaire, tous deux pleins de fine farine pétrie à l'huile pour l'offrande ;
\VS{56}une coupe d'or de dix sicles pleine de parfum ;
\VS{57}un jeune taureau, un bélier, un agneau d'un an, pour l'holocauste ;
\VS{58}un jeune bouc pour le sacrifice d'expiation ;
\VS{59}et pour le sacrifice d'offrande de paix, deux bœufs, cinq béliers, cinq boucs, et cinq agneaux d'un an. Telle fut l'offrande de Gamliel, fils de Pedahtsur.
\VS{60}Le neuvième jour, Abidan, fils de Guideoni, prince des fils de Benjamin, présenta son offrande.
\VS{61}Il offrit un plat d'argent, du poids de cent trente sicles, un bassin d'argent de soixante-dix sicles, selon le sicle du sanctuaire, tous deux pleins de fine farine pétrie à l'huile pour l'offrande ;
\VS{62}une coupe d'or de dix sicles pleine de parfum ;
\VS{63}un jeune taureau, un bélier, un agneau d'un an, pour l'holocauste ;
\VS{64}un jeune bouc pour le sacrifice d'expiation ;
\VS{65}et pour le sacrifice d'offrande de paix, deux bœufs, cinq béliers, cinq boucs, et cinq agneaux d'un an. Telle fut l'offrande d'Abidan, fils de Guideoni.
\VS{66}Le dixième jour, Ahiézer, fils d'Ammischaddaï, prince des fils de Dan, présenta son offrande.
\VS{67}Il offrit un plat d'argent du poids de cent trente sicles, un bassin d'argent de soixante-dix sicles, selon le sicle du sanctuaire, tous deux pleins de fine farine pétrie à l'huile pour l'offrande ;
\VS{68}une coupe d'or de dix sicles pleine de parfum ;
\VS{69}Un jeune taureau, un bélier, un agneau d'un an, pour l'holocauste ;
\VS{70}un jeune bouc pour le sacrifice d'expiation ;
\VS{71}et pour le sacrifice d'offrande de paix, deux bœufs, cinq béliers, cinq boucs, et cinq agneaux d'un an. Telle fut l'offrande d'Ahiézer, fils d'Ammischaddaï.
\VS{72}Le onzième jour, Paguiel, fils d'Ocran, prince des fils d'Aser, présenta son offrande.
\VS{73}Il offrit un plat d'argent, du poids de cent trente sicles, un bassin d'argent de soixante-dix sicles, selon le sicle du sanctuaire, tous deux pleins de fine farine pétrie à l'huile pour l'offrande ;
\VS{74}une coupe d'or de dix sicles pleine de parfum ;
\VS{75}un jeune taureau, un bélier, un agneau d'un an, pour l'holocauste ;
\VS{76}un jeune bouc pour le sacrifice d'expiation ;
\VS{77}et pour le sacrifice d'offrande de paix, deux bœufs, cinq béliers, cinq boucs, et cinq agneaux d'un an. Telle fut l'offrande de Paguiel, fils d'Ocran.
\VS{78}Le douzième jour, Ahira, fils d'Enan, prince des fils de Nephthali, présenta son offrande.
\VS{79}Il offrit un plat d'argent du poids de cent trente sicles, un bassin d'argent de soixante-dix sicles, selon le sicle du sanctuaire, tous deux pleins de fine farine pétrie à l'huile pour l'offrande ;
\VS{80}une coupe d'or de dix sicles pleine de parfum ;
\VS{81}un jeune taureau, un bélier, un agneau d'un an, pour l'holocauste ;
\VS{82}un jeune bouc pour le sacrifice d'expiation ;
\VS{83}et pour le sacrifice d'offrande de paix, deux bœufs, cinq béliers, cinq boucs, et cinq agneaux d'un an. Telle fut l'offrande d'Ahira, fils d'Enan.
\TextTitle{Les dons des princes}
\VS{84}Telle fut la dédicace de l'autel, qui fut faite par les princes d'Israël, lorsqu'il fut oint. Douze plats d'argent, douze bassins d'argent, douze tasses d'or ;
\VS{85}chaque plat d'argent était de cent trente sicles, et chaque bassin de soixante-dix ; tout l'argent de ces ustensiles montait à deux mille quatre cents sicles, selon le sicle du sanctuaire ;
\VS{86}douze coupes d'or pleines de parfum, chacune de dix sicles, selon le sicle du sanctuaire ; tout l'or des tasses montait à cent-vingt sicles.
\VS{87}Tous les animaux pour l'holocauste étaient douze veaux, douze béliers, et douze agneaux d'un an, avec leurs offrandes, et douze jeunes boucs pour le sacrifice d'expiation.
\VS{88}Tous les animaux du sacrifice d'offrande de paix étaient vingt-quatre veaux, avec soixante béliers, soixante boucs, et soixante agneaux d'un an. Telle fut donc la dédicace de l'autel, après qu'on l'eut oint.
\VS{89}Et quand Moïse entrait dans la tente d'assignation pour parler avec Yahweh, il entendait une voix qui lui parlait du haut du propitiatoire placé sur l'arche du témoignage, entre les deux chérubins. Et il lui parlait\FTNT{Ex. 25:22.}.
\Chap{8}
\TextTitle{Les lampes sur le chandelier}
\VerseOne{}Yahweh parla à Moïse, en disant :
\VS{2}Parle à Aaron, et tu lui diras : Quand tu allumeras les lampes, les sept lampes éclaireront sur le devant du chandelier\FTNT{Ex. 25:37.}.
\VS{3}Et Aaron fit ainsi ; il plaça les lampes pour éclairer sur le devant du chandelier, comme Yahweh l'avait commandé à Moïse.
\VS{4}Or le chandelier était fait de telle manière, qu'il était d'or battu au marteau, d'ouvrage fait au marteau, sa tige aussi, et ses fleurs. On fit ainsi le chandelier selon le modèle que Yahweh en avait fait voir à Moïse\FTNT{Ex. 25:31-40.}.
\TextTitle{Purification des Lévites}
\VS{5}Puis Yahweh parla à Moïse, en disant :
\VS{6}Prends les Lévites du milieu des enfants d'Israël, et purifie-les.
\VS{7} Tu leur feras ainsi pour les purifier. Tu feras aspersion sur eux de l'eau de purification ; ils feront passer le rasoir sur toute leur chair, ils laveront leurs vêtements, et ils se purifieront.
\VS{8}Puis ils prendront un jeune taureau avec son offrande de gâteau de fine farine pétrie à l'huile ; et tu prendras un autre jeune taureau pour le sacrifice d'expiation.
\VS{9}Alors tu feras approcher les Lévites devant la tente d'assignation, et tu convoqueras toute l'assemblée des enfants d'Israël.
\VS{10}Tu feras, dis-je, approcher les Lévites devant Yahweh, et les enfants d'Israël poseront leurs mains sur les Lévites.
\VS{11}Et Aaron fera tourner de côté et d'autre les Lévites devant Yahweh, comme offrande de la part des enfants d'Israël, et ils seront employés au service de Yahweh.
\VS{12}Et les Lévites poseront leurs mains sur la tête des veaux ; puis tu offriras l'un en sacrifice pour l'expiation, et l'autre en holocauste à Yahweh, afin de faire propitiation pour les Lévites.
\VS{13}Après tu feras tenir les Lévites devant Aaron et devant ses fils, et tu les présenteras en offrande à Yahweh.
\VS{14}Ainsi tu sépareras les Lévites du milieu des enfants d'Israël, et les Lévites m'appartiendront.
\VS{15}Après cela, les Lévites viendront pour servir dans la tente d'assignation quand tu les auras purifiés et présentés en offrande.
\VS{16}Car ils me sont entièrement donnés du milieu des enfants d'Israël ; je les ai pris pour moi à la place des premiers-nés ; de tous les premiers-nés des fils d'Israël.
\VS{17}Car tout premier-né des enfants d'Israël est à moi, tant des hommes que des animaux ; je me les suis consacrés le jour où j'ai frappé tous les premiers-nés dans le pays d'Egypte.
\VS{18}Or j'ai pris les Lévites au lieu de tous les premiers-nés d'entre les enfants d'Israël.
\VS{19}Et J'ai entièrement donné d'entre les enfants d'Israël, les Lévites à Aaron et à ses fils, pour faire le service des enfants d'Israël dans la tente d'assignation, et pour faire propitiation pour les enfants d'Israël ; afin qu'il n'y ait point de plaie sur les enfants d'Israël, comme il y aurait si les enfants d'Israël s'approchaient du sanctuaire.
\VS{20}Moïse et Aaron, et toute l'assemblée des enfants d'Israël firent à l'égard des Lévites tout ce que Yahweh avait ordonné à Moïse touchant les Lévites ; ainsi firent les enfants d'Israël.
\VS{21}Les Lévites donc se purifièrent, et lavèrent leurs vêtements, et Aaron les fit tourner de côté et d'autre comme une offrande devant Yahweh, et il fit propitiation pour eux afin de les purifier.
\VS{22}Cela étant fait, les Lévites vinrent faire leur service dans la tente d'assignation devant Aaron, et devant ses fils, selon ce que Yahweh avait commandé à Moïse touchant les Lévites ; ainsi fut-il fait à leur égard.
\VS{23}Puis Yahweh parla à Moïse, en disant :
\VS{24}Voici ce qui concerne les Lévites. Depuis l'âge de vingt-cinq ans et au-dessus, tout Lévite entrera en fonction dans la tente d'assignation ;
\VS{25}dès l'âge de cinquante ans, il sortira du service et ne servira plus.
\VS{26}Cependant il servira ses frères dans la tente d'assignation, pour garder ce qui leur a été commis, mais il ne fera plus de service. Tu agiras ainsi à l'égard des Lévites pour ce qui concerne leurs fonctions.
\Chap{9}
\TextTitle{La Pâque}
\VerseOne{}Yahweh avait aussi parlé à Moïse dans le désert de Sinaï, le premier mois de la seconde année, après qu'ils furent sortis du pays d'Egypte, en disant :
\VS{2}Que les enfant d'Israël célèbrent la Pâque\FTNT{Ex. 12 ; 1 Co. 5:7.} au temps fixé.
\VS{3}Vous la ferez en sa saison, le quatorzième jour de ce mois entre les deux soirs, selon toutes ses ordonnances et selon tout ce qu'il y faut faire.
\VS{4}Moïse donc parla aux enfants d'Israël afin qu'ils célèbrent la Pâque.
\VS{5}Et ils firent la Pâque le quatorzième jour du premier mois, entre les deux soirs, dans le désert de Sinaï ; selon tout ce que Yahweh avait commandé à Moïse, les enfants d'Israël le firent ainsi.
\VS{6}Or il y eut quelques-uns qui étaient impurs à cause d'un mort et qui ne purent célébrer la Pâque ce jour-là. Ils se présentèrent ce même jour devant Moïse et devant Aaron,
\VS{7}et ces hommes leur dirent : Nous sommes impurs à cause d'un mort, pourquoi serions-nous privés de présenter l'offrande à Yahweh dans sa saison au milieu des enfants d'Israël ?
\VS{8}Et Moïse leur dit : Arrêtez-vous, et j'entendrai ce que Yahweh commandera sur votre sujet.
\VS{9}Alors Yahweh parla à Moïse, en disant :
\VS{10}Parle aux enfants d'Israël, et dis-leur : Si quelqu'un d'entre vous, ou de votre postérité, est impur à cause d'un mort, ou est en voyage dans un lieu éloigné, il célébrera cependant la Pâque en l'honneur de Yahweh.
\VS{11}Ils la feront le quatorzième jour du second mois, entre les deux soirs ; et ils la mangeront avec du pain sans levain et des herbes amères\FTNT{Ex. 12:10 ; Ex. 23:18 ; Ex. 34:25 ; De. 16:4 ; Jn. 19:33-36.}.
\VS{12}Ils n'en laisseront rien jusqu'au matin, et n'en briseront point les os. Ils la feront selon toutes les ordonnances de la Pâque.
\VS{13}Mais si celui qui est pur et qui n'est pas en voyage s'abstient de célébrer la Pâque, il sera retranché d'entre ses peuples parce qu'il n'a pas présenté l'offrande de Yahweh en sa saison.
\VS{14}Et si un étranger en séjour chez vous célèbre la Pâque de Yahweh, il la fera selon l'ordonnance de la Pâque. II y aura une même ordonnance entre vous, pour l'étranger comme pour celui qui est né au pays\FTNT{Ex. 12:49.}.
\TextTitle{La nuée conduit Israël}
\VS{15}Or le jour où le tabernacle fut dressé, la nuée couvrit le tabernacle de la tente d'assignation ; et le soir jusqu'au matin, elle parut sur le tabernacle avec l'apparence d'un feu\FTNT{Ex. 13:21-22; Ex. 40:34-38 ; De. 1:33.}.
\VS{16}Il en fut ainsi continuellement ; la nuée le couvrait, mais elle paraissait la nuit comme du feu.
\VS{17}Et selon que la nuée se levait de dessus le tabernacle les enfants d'Israël partaient ; et au lieu où la nuée s'arrêtait, les enfants d'Israël y campaient.
\VS{18}Les enfants d'Israël marchaient sur le commandement de Yahweh, et ils campaient sur le commandement de Yahweh ; ils campaient aussi longtemps que la nuée se tenait sur le tabernacle.
\VS{19}Et quand la nuée restait plusieurs jours sur le tabernacle, les enfants d'Israël obéissaient au commandement de Yahweh, et ne partaient point.
\VS{20}Et pour peu de jours que la nuée fût sur le tabernacle, ils campaient sur le commandement de Yahweh, et ils partaient sur le commandement de Yahweh.
\VS{21}Et quand la nuée y était depuis le soir jusqu'au matin, et que la nuée se levait au matin, ils partaient ; fût-ce de jour ou de nuit, quand la nuée se levait, ils partaient.
\VS{22}Si la nuée s'arrêtait sur le tabernacle deux jours, ou un mois, ou plus longtemps, les enfants d'Israël restaient campés, et ne partaient point ; mais quand elle se levait, ils partaient.
\VS{23}Ils campaient donc au commandement de Yahweh, et ils partaient au commandement de Yahweh ; et ils prenaient garde à Yahweh, suivant le commandement de Yahweh, qu'il leur faisait savoir par Moïse.
\Chap{10}
\TextTitle{Les trompettes d'argent}
\VerseOne{}Puis Yahweh parla à Moïse, en disant :
\VS{2}Fais-toi deux trompettes d'argent, battues au marteau. Elles te serviront pour convoquer l'assemblée, et pour le départ des camps.
\VS{3}Quand on en sonnera, toute l'assemblée s'assemblera auprès de toi à l'entrée de la tente d'assignation.
\VS{4}Et quand on sonnera d'une seule, les princes, qui sont les chefs des milliers d'Israël, s'assembleront vers toi.
\VS{5}Mais quand vous sonnerez avec un retentissement bruyant, ceux qui campent à l'orient partiront.
\VS{6}Et quand vous sonnerez la seconde fois avec un retentissement bruyant, ceux qui campent au midi partiront, on sonnera avec un retentissement bruyant, pour leur départ.
\VS{7}Lorsque vous convoquerez l'assemblée, vous ne sonnerez pas avec un retentissement bruyant.
\VS{8}Or les fils d'Aaron, les sacrificateurs, sonneront des trompettes. Ce sera une loi perpétuelle pour vous et pour vos descendants.
\VS{9}Et lorsque, dans votre pays, vous irez à la guerre contre l'ennemi qui vous combattra, vous sonnerez des trompettes avec un retentissement bruyant, et Yahweh votre Dieu, se souviendra de vous, et vous serez délivrés de vos ennemis.
\VS{10}Aussi dans vos jours de joie, dans vos fêtes solennelles, et au commencement de vos mois, vous sonnerez des trompettes en offrant vos holocaustes, et vos sacrifices d'offrande de paix et elles vous serviront de souvenir devant votre Dieu. Je suis Yahweh, votre Dieu.
\TextTitle{La nuée se lève, reprise de la marche dans le désert}
\VS{11}Or il arriva le vingtième jour du second mois de la seconde année, que la nuée se leva de dessus le tabernacle du témoignage.
\VS{12}Et les enfants d'Israël partirent du désert de Sinaï, selon l'ordre fixé pour leur marche. La nuée se posa dans le désert de Paran.
\VS{13}Ils partirent donc pour la première fois, suivant le commandement de Yahweh, déclaré par Moïse.
\VS{14}Et la bannière du camp des fils de Juda partit la première, selon leurs armées. Nachschon, fils d'Amminadab, commandait l'armée de Juda ;
\VS{15}et Nethaneel, fils de Tsuar, commandait l'armée de la tribu des fils d'Issacar ;
\VS{16}et Eliab, fils de Hélon, commandait l'armée de la tribu des fils de Zabulon.
\VS{17}Et le tabernacle fut démonté ; et les fils de Guerschon, et les fils de Merari, qui portaient le tabernacle, partirent.
\VS{18}Puis la bannière du camp de Ruben partit, selon leurs armées. Et Elitsur, fils de Schedéur, commandait l'armée de Ruben ;
\VS{19}et Schelumiel, fils de Tsurischaddaï, commandait l'armée de la tribu des fils de Siméon ;
\VS{20}et Eliasaph, fils de Déuel, commandait l'armée des fils de Gad.
\VS{21}Alors les Kehathites, qui portaient le sanctuaire, partirent ; cependant on dressait le tabernacle, en attendant leur arrivée.
\VS{22}Puis la bannière du camp des fils d'Ephraïm partit, selon leurs armées. Elischama, fils d'Ammihud, commandait l'armée d'Ephraïm ;
\VS{23}et Gamliel, fils de Pedahtsur, commandait l'armée de la tribu des fils de Manassé ;
\VS{24}et Abidan, fils de Guideoni, commandait l'armée de la tribu des fils de Benjamin.
\VS{25}Enfin la bannière des camps des fils de Dan, qui faisait l'arrière-garde, partit, selon leurs armées ; et Ahiézer, fils d'Ammischaddaï, commandait l'armée de Dan.
\VS{26}Et Paguiel, fils d'Ocran, commandait l'armée de la tribu des fils d'Aser ;
\VS{27}et Ahira, fils d'Enan, commandait l'armée de la tribu des fils de Nephthali.
\VS{28}Tel fut l'ordre d'après lequel les enfants d'Israël se mirent en marche selon leurs armées, c'est ainsi qu'ils partirent.
\VS{29}Or Moïse dit à Hobab, fils de Réuel, le Madianite, beau-père de Moïse : Nous allons au lieu dont Yahweh a dit : Je vous le donnerai. Viens avec nous, et nous te ferons du bien, car Yahweh a promis de faire du bien à Israël.
\VS{30}Et Hobab lui répondit : Je n'irai point, mais je m'en irai dans mon pays, et vers ma parenté.
\VS{31}Et Moïse lui dit : Je te prie, ne nous quitte pas ; car tu nous serviras de guide, parce que tu connais les lieux où nous aurons à camper dans le désert.
\VS{32}Et il arrivera que, quand tu seras venu avec nous, et que le bien que Yahweh doit nous faire sera arrivé, nous te ferons aussi du bien.
\VS{33}Ainsi ils partirent de la montagne de Yahweh et ils marchèrent trois jours et l'arche de l'alliance de Yahweh alla devant eux, et fit une marche de trois jours pour leur chercher un lieu de repos.
\VS{34}Et la nuée de Yahweh était sur eux le jour, quand ils partaient du camp.
\VS{35}Or il arrivait qu'au départ de l'arche, Moïse disait : Lève-toi, ô Yahweh, et tes ennemis seront dispersés, et ceux qui te haïssent s'enfuiront de devant toi\FTNT{Ps. 68:2.} !
\VS{36}Et quand on la posait, il disait : Reviens Yahweh, aux dix mille milliers d'Israël !
\Chap{11}
\TextTitle{Jugement contre les murmures du peuple}
\VerseOne{}Après il arriva que le peuple murmura et cela déplut aux oreilles de Yahweh. Lorsque Yahweh l'entendit, sa colère s'enflamma, et le feu de Yahweh s'alluma parmi eux et en consuma l'extrémité du camp.
\VS{2}Alors le peuple cria à Moïse. Moïse pria Yahweh, et le feu s'éteignit.
\VS{3}Et on nomma ce lieu-là Tabeéra, parce que le feu de Yahweh s'était allumé parmi eux.
\TextTitle{Le peuple regrette l'Egypte}
\VS{4}Et le peuple nombreux qui se trouvaient au milieu d'Israël fut épris de convoitise ; et même, les enfants d'Israël se mirent à pleurer disant : Qui nous donnera de la viande à manger\FTNT{Ex. 16:3; Ps. 106:14 ; 1 Co. 10:6.} ?
\VS{5}Nous nous souvenons des poissons que nous mangions en Egypte, et qui ne nous coûtaient rien, des concombres, des melons, des poireaux, des oignons, et de l'ail.
\VS{6}Et maintenant nos âmes sont asséchées ; nos yeux ne voient que de la manne\FTNT{Ps. 78:24.}.
\VS{7}Or La manne était comme la graine de coriandre, et avait l'apparence du bdellium\FTNT{Ex. 16:14-31; Jn. 6:31-58.}.
\VS{8}Le peuple se dispersait et la ramassait, il la moulait aux meules, ou la pilait dans un mortier, il la cuisait au pot et en faisait des gâteaux. Elle avait le goût d'une liqueur d'huile fraîche.
\VS{9}Et quand la rosée descendait la nuit sur le camp, la manne y descendait aussi.
\TextTitle{Moïse dans l'affliction}
\VS{10}Moïse donc entendit le peuple qui pleurait, chacun dans sa famille et à l'entrée de sa tente. La colère de Yahweh s'enflamma fortement et Moïse en fut attristé.
\VS{11}Et Moïse dit à Yahweh : Pourquoi affliges-tu ton serviteur et pourquoi n'ai-je pas trouvé grâce à tes yeux, que tu aies mis sur moi la charge de tout ce peuple ?
\VS{12}Est-ce moi qui ai conçu tout ce peuple ou l'ai-je engendré pour que tu me dises : Porte-le dans ton sein comme le nourricier porte un enfant qui tète, porte-le jusqu'au pays que tu as juré à ses pères ?
\VS{13}D'où aurais-je de la viande pour en donner à tout ce peuple ? Car il pleure auprès de moi, en disant : Donne-nous de la viande à manger !
\VS{14}Je ne puis, à moi seul, porter tout ce peuple, car il est trop pesant pour moi\FTNT{De. 1:9-12.}.
\VS{15}Si tu agis ainsi à mon égard, tue-moi, je te prie donc, si j'ai trouvé grâce à tes yeux, et que je ne voie pas mon malheur.
\TextTitle{[Yahweh établit soixante-dix anciens autour de Moïse]
(Ex. 18:19)}
\VS{16}Alors Yahweh dit à Moïse : Assemble-moi soixante-dix hommes des anciens d'Israël, que tu connais être les anciens du peuple et ses officiers, et amène-les à la tente d'assignation, et qu'ils s'y présentent avec toi.
\VS{17}Puis je descendrai, et je parlerai là avec toi, je mettrai de l'Esprit qui est sur toi sur eux ; afin qu'ils portent avec toi la charge du peuple, et que tu ne la portes pas toi seul.
\VS{18}Et tu diras au peuple : Sanctifiez-vous pour demain, et vous mangerez de la viande ; puisque vous avez pleuré aux oreilles de Yahweh, en disant : Qui nous fera manger de la viande ? Car nous étions bien en Egypte. Ainsi Yahweh vous donnera de la viande, et vous en mangerez.
\VS{19}Vous n'en mangerez pas un jour, ni deux jours, ni cinq jours, ni dix jours, ni vingt jours,
\VS{20}mais jusqu'à un mois entier, jusqu'à ce qu'elle vous sorte par les narines, et que vous en ayez du dégoût, parce que vous avez rejeté Yahweh qui est au milieu de vous ; vous avez pleuré devant lui, en disant : Pourquoi sommes-nous sortis d'Egypte ?
\VS{21}Moïse dit : Six cent mille hommes de pied forment ce peuple au milieu duquel je suis, et tu as dit : Je leur donnerai de la viande afin qu'ils en mangent un mois entier !
\VS{22}Leur tuera-t-on des brebis ou des bœufs, en sorte qu'il y en ait assez pour eux ? Ou leur assemblera-t-on tous les poissons de la mer, en sorte qu'ils en aient assez ?
\VS{23}Yahweh répondit à Moïse: La main de Yahweh serait-elle trop courte ? Tu verras maintenant si ce que je t'ai dit arrivera ou non\FTNT{Es. 50:2 ; Es. 59:1-2.}.
\VS{24}Moïse donc sortit et rapporta au peuple les paroles de Yahweh. Il assembla soixante-dix hommes des anciens du peuple, et les plaça autour de la tente.
\VS{25}Yahweh descendit dans la nuée et parla à Moïse ; il prit de l'Esprit qui était sur lui et le mit sur les soixante-dix hommes anciens. Et dès que l'Esprit reposa sur eux, ils prophétisèrent ; mais ils ne continuèrent pas.
\TextTitle{Prophétie d'Eldad et de Médad}
\VS{26}Or il eut deux hommes restés au camp, l'un s'appelait Eldad, et l'autre Médad, sur lesquels l'Esprit reposa. Ils étaient de ceux qui avaient été inscrits, mais ils n'étaient pas allés à la tente, et ils prophétisaient dans le camp.
\VS{27}Alors un garçon courut le rapporter à Moïse, en disant : Eldad et Médad prophétisent dans le camp.
\VS{28}Et Josué, fils de Nun, qui servait Moïse, l’un de ses jeunes gens, répondit, en disant : Mon seigneur Moïse, empêche-les.
\VS{29}Et Moïse lui répondit : Es-tu jaloux pour moi ? Plût à Dieu que tout le peuple de Yahweh fût prophète, et que Yahweh mît son esprit sur eux !
\VS{30}Puis Moïse se retira au camp, lui et les anciens d'Israël.
\TextTitle{Les cailles et le jugement de Yahweh}
\VS{31}Alors Yahweh fit lever un vent de la mer qui amena des cailles et les répandit sur le camp environ le chemin d'une journée, de çà et de là, tout autour du camp ; et il y en avait presque la hauteur de deux coudées sur la terre\FTNT{Ex. 16:13-15 ; Ps. 78:26-29 ; Ps. 105:40.}.
\VS{32}Et le peuple se leva tout ce jour-là, et toute la nuit, et tout le jour suivant, et amassa des cailles ; celui qui en avait amassé le moins en avait dix homers ; et ils les étendirent soigneusement pour eux tout autour du camp.
\VS{33}Mais la chair était encore entre leurs dents, avant qu’elle fût mâchée, la colère de Yahweh s'embrasa contre le peuple, et il frappa le peuple d'une très grande plaie\FTNT{Ps. 78:30-31.}.
\VS{34}Et on nomma ce lieu-là Kibroth-Hattaava ; car on ensevelit là le peuple qui avait convoité.
\VS{35}Et de Kibroth-Hattaava le peuple s'en alla pour Hatséroth, et il s'arrêta à Hatséroth.
\Chap{12}
\TextTitle{Marie et Aaron murmurent contre Moïse}
\VerseOne{}Alors Marie et Aaron parlèrent contre Moïse au sujet de la femme ethiopienne qu'il avait prise, car il avait pris une femme ethiopienne.
\VS{2}Et ils dirent : Est-ce seulement par Moïse que Yahweh parle? N'est-ce pas aussi par nous qu'il parle ? Et Yahweh entendit cela. 
\VS{3}Or cet homme Moïse était un homme fort doux, plus que tous les hommes qui étaient sur la terre.
\VS{4}Et soudain Yahweh dit à Moïse, à Aaron, et à Marie : Venez vous trois à la tente d’assignation ; et ils y allèrent eux trois.
\VS{5}Alors Yahweh descendit dans la colonne de nuée et se tint à l'entrée de la tente. Puis il appela Aaron et Marie, qui s'avancèrent tous les deux.
\VS{6}Et il dit : Ecoutez maintenant mes paroles ! Lorsqu'il y aura parmi vous un prophète, moi qui suis Yahweh je me ferai bien connaître à lui en vision, et je lui parlerai en songe.
\VS{7}Il n'en est pas ainsi de mon serviteur Moïse, qui est fidèle dans toute ma maison\FTNT{Hé. 3:2.}.
\VS{8}Je parle avec lui bouche à bouche, et il me voit en effet, et non point en obscurité, ni dans aucune représentation de Yahweh. Pourquoi donc n’avez-vous pas craint de parler contre mon serviteur, contre Moïse ?
\FTNT{Ex. 33:11 ; De. 34:10.} 
\VS{9}Ainsi la colère de Yahweh s'embrasa contre eux. Et il s'en alla.
\VS{10}Car la nuée se retira de dessus la tente. Et voici, Marie était frappée d'une lèpre blanche comme la neige ; et Aaron se tourna vers Marie et la vit lépreuse.
\VS{11}Alors Aaron dit à Moïse : hélas, de grâce, mon seigneur ! Je te prie ne mets point sur nous ce péché, car nous avons fait follement, et nous avons péché.
\VS{12}Je te prie qu'elle ne soit pas comme un enfant mort-né, dont la moitié de la chair est déjà consumée quand il sort du ventre de sa mère !
\VS{13}Alors Moïse cria à Yahweh, en disant : Ô Dieu, je te prie, guéris-la, je t’en prie.
\VS{14}Et Yahweh répondit à Moïse : Si son père lui avait craché au visage, ne serait-elle pas dans l'ignominie pendant sept jours ? Qu'elle soit enfermée sept jours en dehors du camp, après quoi, elle y sera reçue\FTNT{Lé. 13:46.}.
\VS{15}Ainsi Marie fut enfermée hors du camp sept jours ; et le peuple ne partit pas de là jusqu'à ce que Marie fût rentrée.
\VS{16}Après cela le peuple partit de Hatséroth, et il campa dans le désert de Paran.
\Chap{13}
\TextTitle{Douze espions envoyés pour explorer Canaan}
\VerseOne{}Et Yahweh parla à Moïse, en disant :
\VS{2}Envoie des hommes pour explorer le pays de Canaan, que je donne aux enfants d'Israël. Tu enverras un homme de chaque tribu de leurs pères, tous seront des principaux d'entre eux.
\VS{3}Moïse donc les envoya du désert de Paran, d'après l'ordre de Yahweh ; et tous ces hommes étaient chefs des enfants d'Israël.
\VS{4}Et ce sont ici leurs noms : De la tribu de Ruben : Schammua, fils de Zaccur ;
\VS{5}de la tribu de Siméon : Schaphath, fils de Hori ;
\VS{6}de la tribu de Juda : Caleb, fils de Jephunné ;
\VS{7}de la tribu d'Issacar : Jigual, fils de Joseph ;
\VS{8}de la tribu d'Ephraïm : Hosée, fils de Nun ;
\VS{9}de la tribu de Benjamin : Palthi, fils de Raphu ;
\VS{10}de la tribu de Zabulon : Gaddiel, fils de Sodi ;
\VS{11}de l'autre tribu de Joseph : la tribu de Manassé, Gaddi, fils de Susi ;
\VS{12}de la tribu de Dan : Ammiel, fils de Guemalli ;
\VS{13}de la tribu d'Aser : Sethur, fils de Micaël ;
\VS{14}de la tribu de Nephthali : Nachbi, fils de Vophsi ;
\VS{15}de la tribu de Gad : Guéuel, fils de Maki.
\VS{16}Ce sont là les noms des hommes que Moïse envoya pour explorer le pays. Moïse donna à Hosée, fils de Nun, le nom de Josué\FTNT{Moïse changea le nom d'Hosée en y ajoutant le Nom de Yahweh. Hosée signifie « sauveur » et Josué (ou Jésus) « Yahweh est salut ». Josué préfigurait Jésus-Christ qui nous a délivrés et transportés dans le Royaume des cieux (Col. 1:12-14). Moïse avait compris prophétiquement que seul Jésus peut nous faire rentrer dans notre héritage.}.
\VS{17}Moïse les envoya pour explorer le pays de Canaan, et il leur dit : Montez de ce côté par le sud ; et vous monterez sur la montagne.
\VS{18}Et vous verrez quel est ce pays-là, et quel est le peuple qui l’habite, s’il est fort ou faible ; s’il est en petit ou en grand nombre.
\VS{19}Et quel est le pays où il habite, s’il est bon ou mauvais ; et quelles sont les villes dans lesquelles il habite, si c’est dans des camps, ou dans des villes fortifiées.
\VS{20}Et quelle est la terre, si elle est grasse ou maigre, s’il y a des arbres, ou non. Ayez bon courage, et prenez du fruit du pays. Or c’était alors le temps des premiers raisins.
\VS{21}Étant donc partis, ils examinèrent le pays, depuis le désert de Tsin jusqu'à Rehob, à l'entrée de Hamath.
\VS{22}Ils montèrent par le sud, et ils allèrent jusqu'à Hébron, où étaient Ahiman, Schéschaï, et Talmaï, enfants d'Anak. Hébron avait été bâtie sept ans avant Tsoan en Egypte.
\VS{23}Et ils vinrent jusqu'au torrent d'Eschcol, et coupèrent de là un sarment de vigne, avec une grappe de raisins ; ils étaient deux à le porter avec une perche. Ils apportèrent aussi des grenades et des figues.
\VS{24}Et on donna à ce lieu le nom de vallée d'Eschcol ; à cause de la grappe que les fils d'Israël y coupèrent.
\VS{25}Et au bout de quarante jours, ils furent de retour du pays qu'ils étaient allés explorer.
\TextTitle{Comptes rendus des envoyés}
\VS{26}Et à leur arrivée, ils se rendirent auprès de Moïse et d'Aaron, et de toute l'assemblée des enfants d'Israël, dans le désert de Padan à Kadès. Ils leur firent le rapport, ainsi qu'à toute l'assemblée, ils leur montrèrent les fruits du pays.
\VS{27}Ils firent donc leur rapport à Moïse, et lui dirent : Nous avons été dans le pays où tu nous as envoyés. Véritablement, c'est un pays où coulent le lait et le miel, et en voici les fruits.
\VS{28}Seulement, le peuple qui habite ce pays est puissant, les villes sont fortifiées, très grandes ; nous y avons vu des enfants d'Anak\FTNT{De. 1:24-28.}.
\VS{29}Les Amalécites habitent la contrée du midi ; les Héthiens, les Jébusiens et les Amoréens habitent la montagne ; les Cananéens habitent le long de la mer, et vers le rivage du Jourdain.
\VS{30}Caleb fit taire le peuple devant Moïse, et il dit : Montons, possédons ce pays, car nous y serons vainqueurs !
\VS{31}Mais les hommes qui y étaient montés avec lui dirent : Nous ne pouvons pas monter contre ce peuple-là, car il est plus fort que nous.
\VS{32}Et ils décrièrent devant les enfants d'Israël le pays qu'ils avaient exploré, en disant  : Le pays que nous avons parcouru pour l'explorer est un pays qui dévore ses habitants et  tous ceux que nous y avons vus sont des gens de grande taille.
\VS{33}Et nous y avons vu aussi des géants, des enfants d'Anak, de la race des géants et nous étions à nos yeux et à leurs yeux comme des sauterelles.
\Chap{14}
\TextTitle{Rébellion et incrédulité d'Israël\FTNTT{1 Co. 10:1-5 ; Hé. 3:7-19}}
\VerseOne{}Alors toute l'assemblée éleva la voix et se mit à pousser des cris, et le peuple pleura cette nuit-là.
\VS{2}Et tous les enfants d'Israël murmurèrent contre Moïse et Aaron, et toute l'assemblée leur dit : Oh! Si nous étions morts dans le pays d’Egypte! Ou si nous étions morts dans ce désert\FTNT{De. 1:26-27.}!
\VS{3}Et pourquoi Yahweh nous fait-il aller dans ce pays, où nous tomberons par l'épée, où nos femmes et nos petits enfants deviendront une proie ? Ne vaut-il pas mieux retourner en Egypte ?
\VS{4}Et ils se dirent l'un à l'autre : Etablissons-nous un chef, et retournons en Egypte.
\VS{5}Alors Moïse et Aaron tombèrent sur leurs visages devant toute l'assemblée des enfants d'Israël.
\VS{6}Et Josué, fils de Nun, et Caleb, fils de Jephunné, qui étaient parmi ceux qui avaient exploré le pays, déchirèrent leurs vêtements,
\VS{7}et parlèrent à toute l'assemblée des enfants d'Israël, en disant : Le pays que nous avons exploré est un très bon pays.
\VS{8}Si nous sommes agréables à Yahweh, il nous fera entrer dans ce pays, et il nous le donnera. C'est un pays où coulent le lait et le miel.
\VS{9}Seulement, ne soyez point rebelles contre Yahweh, et ne craignez point le peuple de ce pays-là, car ils seront notre pain, leur protection s’est retirée de dessus eux. Yahweh est avec nous, ne les craignez point\FTNT{De. 20:3-4.} !
\VS{10}Alors toute l'assemblée parlait de les lapider ; mais la gloire de Yahweh apparut à tous les enfants d'Israël, devant la tente d'assignation.
\TextTitle{Moïse intercède pour le pardon d'Israël}
\VS{11}Et Yahweh dit à Moïse : Jusqu'à quand ce peuple-ci m’irritera-t-il par mépris et jusqu'à quand ne croira-t-il point en moi, malgré tous les signes que j'ai faits au milieu de lui ?
\VS{12}Je le frapperai par la peste et je le détruirai, mais je ferai de toi une nation plus grande et plus puissante que lui.
\VS{13}Et Moïse dit à Yahweh : Mais les Egyptiens l'entendront, car tu as fait monter par ta puissance ce peuple-ci du milieu d'eux\FTNT{Ex. 32:10-12.},
\VS{14}et ils diront avec les habitants de ce pays qui auront entendu que tu étais, ô Yahweh ! Au milieu de ce peuple, et que tu apparaissais, ô Yahweh ! A vue d’œil, que ta nuée s’arrêtait sur eux, et que tu marchais devant eux le jour dans la colonne de nuée, et la nuit dans la colonne de feu ;
\VS{15}si tu fais mourir ce peuple comme un seul homme, les nations qui ont entendu parler de toi diront :
\VS{16}Yahweh n'avait pas le pouvoir de faire entrer ce peuple dans le pays qu'il avait juré de leur donner, il l'a égorgé dans le désert.
\VS{17}Maintenant, je te prie, que la puissance du Seigneur se montre dans sa grandeur, comme tu l'as déclaré en disant :
\VS{18}Yahweh est lent à la colère et riche en bonté, il ôte l'iniquité et pardonne la rébellion, mais il ne tient point le coupable pour innocent, et il punit l'iniquité des pères sur les fils, jusqu'à la troisième et à la quatrième génération\FTNT{Ex. 20:5; Ex. 34:6 ; Ex. 34:7; Ps. 86:15 ; Ps. 103:8 ; Ps. 145:8 ; Jon. 4:2 ; De. 5:9.}.
\VS{19}Pardonne, je te prie, l'iniquité de ce peuple, selon la grandeur de ta miséricorde, comme tu as pardonné à ce peuple depuis l'Egypte jusqu'ici.
\TextTitle{Réponse de Yahweh à Moïse}
\VS{20}Et Yahweh dit : Je pardonne selon ta parole.
\VS{21}Mais certainement je suis vivant, et la gloire de Yahweh remplira toute la terre.
\VS{22}Car tous ceux qui ont vu ma gloire, et les prodiges que j'ai faits en Egypte et dans le désert, qui m'ont déjà tenté par dix fois, et qui n'ont point écouté ma voix,
\VS{23}tous ceux-là ne verront point le pays que j'ai juré à leurs pères de leur donner, tous ceux, dis-je, qui m'ont irrité par mépris, ne le verront pas\FTNT{De. 1:35-38.}.
\VS{24}Mais parce que mon serviteur Caleb a été animé d'un autre esprit, et qu'il a persévéré à me suivre, je le ferai entrer dans le pays où il a été, et ses descendants le posséderont en héritage.
\VS{25}Or les Amalécites et les Cananéens habitent la vallée. Demain, tournez-vous et partez pour le désert, dans la direction de la Mer Rouge.
\VS{26}Yahweh parla à Moïse et à Aaron, en disant :
\VS{27}Jusqu'à quand laisserai-je cette méchante assemblée murmurer contre moi? J'ai entendu les murmures des enfants d'Israël, qui murmuraient contre moi\FTNT{Ps. 106:25.}.
\VS{28}Dis-leur : Je suis vivant, dit Yahweh, je vous ferai ainsi que vous avez parlé à mes oreilles.
\VS{29}Vos cadavres tomberont dans ce désert, et tous ceux d’entre vous qui ont été dénombrés, selon tout le compte que vous en avez fait, depuis l’âge de vingt ans, et au dessus, vous tous qui avez murmuré contre moi ;
\VS{30}vous n'entrerez pas dans le pays que j'avais juré de vous faire habiter, excepté Caleb, fils de Jephunné, et Josué, fils de Nun.
\VS{31}Et quant à vos petits enfants, dont vous avez dit : Ils deviendront une proie ! Je les y ferai entrer, et ils connaîtront le pays que vous avez méprisé.
\VS{32}Mais quant à vous, vos cadavres tomberont dans ce désert ;
\VS{33}mais vos enfants paîtront dans ce désert quarante ans et ils porteront la peine de vos prostitutions, jusqu'à ce que vos cadavres soient tous consumés dans le désert.
\VS{34}Selon le nombre des jours que vous avez mis à reconnaître le pays, qui ont été quarante jours, un jour pour une année, vous porterez la peine de vos iniquités quarante ans, et vous connaîtrez ma rupture de promesse.
\VS{35}Je suis Yahweh, j'ai parlé ! C'est ainsi que je traiterai cette méchante assemblée, qui s'est assemblée contre moi ; ils seront consumés dans ce désert, et ils y mourront.
\VS{36}Les hommes donc que Moïse avait envoyés pour épier le pays, et qui étant de retour avaient fait murmurer contre lui toute l’assemblée, en diffamant le pays ;
\VS{37}ces hommes-là, qui avaient décrié le pays, moururent frappés d'une plaie devant Yahweh.
\VS{38}Mais Josué, fils de Nun, et Caleb, fils de Jephunné, restèrent seuls vivants parmi ceux qui étaient allés pour explorer le pays.
\TextTitle{Israël battu par les Amalécites et les Cananéens}
\VS{39}Or Moïse dit ces choses à tous les enfants d'Israël, et le peuple fut dans un grand deuil.
\VS{40}Puis ils se levèrent de bon matin et montèrent au sommet de la montagne, en disant : Nous voici, et nous monterons au lieu dont Yahweh a parlé car nous avons péché.
\VS{41}Mais Moïse leur dit : Pourquoi transgressez-vous le commandement de Yahweh ? Cela ne réussira point.
\VS{42}Ne montez pas ; car Yahweh n'est pas au milieu de vous ; afin que vous ne soyez pas battus devant vos ennemis\FTNT{De. 1:41-42.}.
\VS{43}Car les Amalécites et les Cananéens sont là devant vous, et vous tomberez par l'épée ; parce que vous vous êtes détournés de Yahweh, Yahweh ne sera point avec vous.
\VS{44}Toutefois ils s'obstinèrent à monter au sommet de la montagne ; mais l'arche de l'alliance de Yahweh et Moïse ne sortirent point du milieu du camp.
\VS{45}Alors les Amalécites et les Cananéens qui habitaient sur cette montagne descendirent, les battirent, et les taillèrent en pièces jusqu'à Horma.
\Chap{15}
\TextTitle{Consignes pour le pays de Canaan}
\VerseOne{}Puis Yahweh parla à Moïse, en disant :
\VS{2}Parle aux enfants d'Israël, et dis-leur : Quand vous serez entrés au pays que je vous donne, où vous devez demeurer,
\VS{3}et que vous voudrez faire un sacrifice consumé par le feu à Yahweh, un holocauste, ou un sacrifice en accompagnement d'un vœu, ou en offrande volontaire, ou bien dans vos fêtes, pour produire avec votre gros ou votre menu bétail une agréable odeur à Yahweh\FTNT{Ex. 29:18; Lé. 22:21.},
\VS{4}celui qui offrira son offrande à Yahweh présentera en offrande un dixième de fleur de farine, pétrie dans un quart de hin d'huile\FTNT{Lé. 2:1-2.},
\VS{5}et un quart de hin de vin pour la libation que tu feras sur l'holocauste, ou sur un autre sacrifice pour chaque agneau.
\VS{6}Si c'est pour un bélier, tu feras en offrande deux dixièmes de fleur de farine, pétrie dans un tiers de hin d'huile,
\VS{7}et un tiers de hin de vin pour la libation, comme offrande d'une bonne odeur à Yahweh.
\VS{8}Et si tu sacrifies un veau, soit comme holocauste, soit comme sacrifice en accompagnement d'un vœu, ou comme sacrifice d'offrande de paix à Yahweh,
\VS{9}on présentera en offrande avec le veau trois dixièmes de fleur de farine, pétrie dans un demi-hin d'huile.
\VS{10}Et tu offriras la moitié d'un hin de vin pour la libation, en offrande consumée par le feu d'une bonne odeur à Yahweh.
\VS{11}On fera de même pour chaque boeuf, chaque bélier, et chaque petit des brebis ou des chèvres.
\VS{12}Selon le nombre que vous en sacrifierez, vous ferez ainsi à chacun, d'après leur nombre.
\VS{13}Tous ceux qui sont nés au pays feront ces choses de cette manière, en offrant un sacrifice consumé par le feu, d'une bonne odeur à Yahweh.
\TextTitle{Loi sur l'étranger vivant au milieu d'Israël}
\VS{14}Si un étranger séjournant chez vous, ou se trouvant au milieu de vous en vos générations, offre un sacrifice consumé par le feu d'une bonne odeur à Yahweh, il l'offrira de la même manière que vous.
\VS{15}Ô assemblée ! Il y aura une même ordonnance pour vous et pour l’étranger qui fait son séjour parmi vous, il y aura une même ordonnance perpétuelle en vos âges ; il en sera de l’étranger comme de vous en la présence de Yahweh.
\VS{16}Il y aura une même loi et une seule ordonnance pour vous et pour l'étranger qui séjourne au milieu de vous.
\TextTitle{Lois diverses}
\VS{17}Yahweh parla à Moïse, en disant :
\VS{18}Parle aux enfants d'Israël, et dis-leur : Quand vous serez arrivés dans le pays où je vous ferai entrer,
\VS{19}et que vous mangerez du pain de ce pays, vous en offrirez à Yahweh une offrande élevée.
\VS{20}Vous offrirez en offrande élevée un gâteau, les prémices de votre pâte ; vous l'offrirez comme ce qu'on prélève de l'aire.
\VS{21}Vous donnerez pour Yahweh une offrande des prémices de votre pâte, dans les temps à venir.
\VS{22}Et lorsque vous aurez péché involontairement\FTNT{Voir commentaire en Lé. 4:2.}, et que vous n’aurez pas fait tous ces commandements que Yahweh a fait connaître à Moïse,
\VS{23}tout ce que Yahweh vous a commandé par Moïse, depuis le jour où Yahweh a commencé de donner ses commandements, et dans la suite dans vos générations,
\VS{24}s’il arrive que la chose ait été faite involontairement, sans que l'assemblée s'en soit aperçue, toute l'assemblée sacrifiera un jeune taureau en holocauste d'une bonne odeur à Yahweh, avec l'offrande et la libation, d'après les règles établies ; elle offrira encore un jeune bouc en sacrifice pour l'expiation.
\VS{25}Ainsi le sacrificateur fera propitiation pour toute l'assemblée des enfants d'Israël, et il leur sera pardonné parce que c’est une chose arrivée involontairement, et ils ont apporté leur offrande, un sacrifice consumé par le feu à Yahweh et l'offrande pour l'expiation devant Yahweh, à cause de leur péché involontaire.
\VS{26}Alors il sera pardonné à toute l'assemblée des enfants d'Israël, et à l'étranger qui séjourne au milieu d'eux, car c'est involontairement que tout le peuple a péché.
\VS{27}Si c'est une seule personne qui a péché involontairement, elle offrira une chèvre d'un an en offrande pour le péché\FTNT{Lé. 4:27-28.}.
\VS{28}Et le sacrificateur fera propitiation pour la personne qui aura péché involontairement, de ce qu'elle aura péché involontairement devant Yahweh, et faisant propitiation pour elle, il lui sera pardonné.
\VS{29}Il y aura une même loi pour celui qui aura fait quelque chose involontairement, tant pour celui qui est né au pays des enfants d’Israël, que pour l’étranger qui fait son séjour parmi eux.
\VS{30}Mais quant à celui qui aura péché par fierté, tant celui qui est né au pays, que l’étranger, il a outragé Yahweh, cette personne-là sera retranchée du milieu de son peuple.
\VS{31}Parce qu’il a méprisé la parole de Yahweh, et qu’il a enfreint son commandement. Cette personne donc sera certainement retranchée ; son iniquité est sur elle.
\TextTitle{Un homme lapidé selon la loi\FTNTT{Ro. 3:19 ; 7:7-11 ; 2 Co. 3:7-9 ; Ga. 3:10}}
\VS{32}Or comme les enfants d'Israël étaient dans le désert, on trouva un homme qui ramassait du bois le jour du sabbat.
\VS{33}Et ceux qui l'avaient trouvé ramassant du bois, l'amenèrent à Moïse, à Aaron, et à toute l'assemblée.
\VS{34}Et on le mit sous garde, car ce qu'on devait lui faire n'avait pas été déclaré.
\VS{35}Alors Yahweh dit à Moïse : On punira de mort cet homme, et toute l'assemblée le lapidera hors du camp.
\VS{36}Toute l'assemblée donc le mena hors du camp et le lapida, et il mourut, comme Yahweh l'avait ordonné à Moïse.
\VS{37}Et Yahweh parla à Moïse, en disant :
\VS{38}Parle aux enfants d'Israël, et dis-leur : Qu'ils se fassent de génération en génération des franges aux bords de leurs vêtements, et qu'ils mettent sur les franges au bords de leurs vêtements un cordon de couleur pourpre\FTNT{De. 22:12 ; Mt. 23:5.}.
\VS{39}Quand vous aurez cette frange, vous la regarderez et vous vous souviendrez de tous les commandements de Yahweh, pour les mettre en pratique, et vous ne suivrez pas les désirs de vos cœurs et de vos yeux, pour vous laisser entraîner à la prostitution.
\VS{40}Afin que vous vous souveniez de tous mes commandements, et que vous les fassiez, et que vous soyez saints à votre Dieu.
\VS{41}Je suis Yahweh, votre Dieu, qui vous ai retiré du pays d'Egypte, pour être votre Dieu. Je suis Yahweh, votre Dieu.
\Chap{16}
\TextTitle{La révolte de Koré\FTNTT{Jud. 11}}
\VerseOne{}Or Koré\FTNT{Koré, Dathan et Abiram, s'étaient révoltés contre Aaron et Moïse, car ils voulaient s'attribuer l'honneur d'offrir à Dieu des sacrifices. Ils voulaient exercer la sacrificature (sacerdoce) alors que Yahweh ne les avait pas établis ministres du culte. Vouloir servir Dieu sans avoir reçu un appel divin est dangereux}, fils de Jitsehar, fils de Kehath, fils de Lévi, se révolta avec Dathan et Abiram, fils d'Eliab, et On, fils de Péleth, tous trois fils de Ruben.
\VS{2}Et ils s’élevèrent contre Moïse, avec deux cent cinquante hommes des fils d'Israël, qui étaient des principaux de l'assemblée, de ceux que l'on convoquait pour tenir le conseil, et qui étaient des gens de renom.
\VS{3}Et ils s'assemblèrent contre Moïse et contre Aaron, et leur dirent : C'en est assez ! Puisque tous ceux de l’assemblée sont saints, et que Yahweh est au milieu d'eux, pourquoi vous élevez-vous au-dessus de l'assemblée de Yahweh ?
\VS{4}Quand Moïse eut entendu cela, il se jeta sur son visage.
\VS{5}Et il parla à Koré et à tous ceux qui étaient assemblés avec lui, et leur dit : Demain au matin, Yahweh fera connaître celui qui lui appartient, et celui qui est saint, et il le fera approcher de lui ; il fera, dis-je, approcher de lui celui qu’il aura choisi.
\VS{6}Faites ceci, prenez des encensoirs, Koré et toute son assemblée.
\VS{7}Et demain, mettez-y du feu, et mettez-y du parfum devant Yahweh ; et celui que Yahweh choisira, c'est celui-là qui sera saint. C'en est assez, fils de Lévi !
\VS{8}Moïse dit aussi à Koré : Ecoutez maintenant, fils de Lévi :
\VS{9}Est-ce trop peu de chose pour vous, que le Dieu d'Israël vous ait séparés de l'assemblée d'Israël, pour vous faire approcher de lui, afin de faire le service du tabernacle de Yahweh, et pour vous tenir devant l'assemblée, afin de la servir ?
\VS{10}Et qu'il t'ait fait approcher de lui, toi et tous tes frères, les fils de Lévi, et vous recherchez encore la sacrificature !
\VS{11}C'est pourquoi toi et toute ton assemblée, vous vous êtes rassemblés contre Yahweh ! Car qui est Aaron pour que vous murmuriez contre lui ?
\VS{12}Et Moïse envoya appeler Dathan et Abiram, fils d'Eliab, qui répondirent : Nous n'y monterons point.
\VS{13}Est-ce peu de chose que tu nous aies fait monter hors d’un pays où coulent le lait et le miel, pour nous faire mourir dans le désert, que tu veuilles aussi dominer sur nous ?
\VS{14}Certes, tu ne nous as pas fait venir dans un pays où coulent le lait et le miel ! Et tu ne nous as pas donné un héritage de champs ni de vignes ! Veux-tu crever les yeux de ces gens ? Nous ne monterons pas.
\VS{15}Alors Moïse fut très irrité, et il dit à Yahweh : N'aie point égard à leur offrande. Je n'ai point pris d'eux un seul âne, et je n'ai fait de mal à aucun d'eux.
\VS{16}Puis Moïse dit à Koré : Toi et tous ceux qui sont assemblés avec toi, trouvez-vous demain devant Yahweh, toi et eux avec Aaron.
\VS{17}Et prenez chacun vos encensoirs, et mettez-y du parfum ; et que chacun présente devant Yahweh son encensoir : Il y aura deux cent cinquante encensoirs ; toi et Aaron aussi, chacun avec son encensoir.
\VS{18}Ils prirent donc chacun son encensoir, et y mirent du feu, et ensuite y posèrent du parfum, et ils se tinrent à l'entrée de la tente d'assignation, avec Moïse et Aaron.
\VS{19}Et Koré fit assembler contre eux toute l'assemblée à l'entrée de la tente d'assignation ; et la gloire de Yahweh apparut à toute l'assemblée.
\VS{20}Puis Yahweh parla à Moïse et à Aaron, en disant :
\VS{21}Séparez-vous du milieu de cette assemblée, et je les consumerai en un seul instant\FTNT{Ex. 32:10}.
\VS{22}Mais ils tombèrent sur leur visage et dirent : Ô Dieu ! Dieu des esprits de toute chair ! Un seul homme a péché, et tu te mettrais en colère contre toute l'assemblée\FTNT{Hé. 12:9.} ?
\VS{23}Et Yahweh parla à Moïse, en disant :
\VS{24}Parle à l'assemblée, et dis lui : Retirez-vous d'auprès de la demeure de Koré, de Dathan, et d'Abiram.
\VS{25}Moïse donc se leva, et alla vers Dathan et Abiram ; et les anciens d'Israël le suivirent.
\VS{26}Et il parla à l'assemblée, en disant : Eloignez-vous, je vous prie, d'auprès des tentes de ces méchants hommes, et ne touchez à rien qui leur appartienne, de peur que vous ne périssiez punis pour tous leurs péchés.
\VS{27}Ils se retirèrent donc d'auprès des demeures de Koré, de Dathan et d'Abiram. Et Dathan et Abiram sortirent et se tinrent debout à l'entrée de leurs tentes, avec leurs femmes, leurs fils, et leurs petits-enfants.
\VS{28}Et Moïse dit : Vous connaîtrez à ceci que Yahweh m'a envoyé pour faire toutes ces choses, et que je n'agis pas de moi-même.
\VS{29}Si ces gens meurent comme tous les hommes meurent, et s'ils subissent le sort commun à tous les hommes, Yahweh ne m'a point envoyé ;
\VS{30}mais si Yahweh fait une chose nouvelle, et si la terre ouvre sa bouche pour les engloutir avec tout ce qui leur appartient, et qu'ils descendent vivants dans le scheol, vous saurez alors que ces hommes-là ont irrité par mépris Yahweh.
\VS{31}Et il arriva qu'aussitôt qu'il eut achevé de dire toutes ces paroles, la terre qui était sous eux se fendit.
\VS{32}Et la terre ouvrit sa bouche et les engloutit, avec leurs tentes et tous les hommes qui étaient à Koré, et tous leurs biens\FTNT{De. 11:6 ; Ps. 106:17.}.
\VS{33}Ils descendirent donc vivants dans le scheol, eux et tout ceux qui leur appartenait ; la terre les recouvrit, et ils disparurent au milieu de l'assemblée.
\VS{34}Et tout Israël qui était autour d'eux s'enfuit à leurs cris ; car ils disaient : Prenons garde que la terre ne nous engloutisse !
\VS{35}Un feu sortit de part Yahweh et consuma les deux cent cinquante hommes qui offraient le parfum.
\VS{36}Puis Yahweh parla à Moïse, en disant :
\VS{37}Dis à Eléazar, fils d'Aaron, le sacrificateur, qu’il ramasse les encensoirs du milieu de l’embrasement, et d'en répandre au loin le feu, car ils sont sanctifiés.
\VS{38}Avec les encensoirs de ceux qui ont péché contre leurs âmes, que l'on fasse des lames étendues dont on couvrira l'autel. Puisqu'ils ont été offerts devant Yahweh et qu'ils sont sanctifiés, ils serviront de signe aux enfants d'Israël.
\VS{39}Ainsi Eléazar, le sacrificateur, prit les encensoirs d'airain, que ces hommes qui furent brûlés avaient présentés, et on en fit des lames pour couvrir l'autel.
\VS{40}C'est un souvenir pour les enfants d'Israël, afin qu'aucun étranger qui n’est pas de la race d'Aaron, ne s'approche pour offrir du parfum devant Yahweh, et ne soit comme Koré, et comme ceux qui ont été assemblés avec lui ; selon ce que Yahweh avait déclaré par Moïse.
\TextTitle{Le peuple frappé d'une plaie à cause de ses murmures}
\VS{41}Or dès le lendemain, toute l'assemblée des enfants d'Israël murmura contre Moïse et contre Aaron, en disant : Vous avez fait mourir le peuple de Yahweh.
\VS{42}Et il arriva comme l'assemblée s'amassait contre Moïse et contre Aaron, et comme ils tournaient les regards vers la tente d'assignation, voici la nuée la couvrit, et la gloire de Yahweh apparut.
\VS{43}Moïse donc et Aaron vinrent devant la tente d'assignation.
\VS{44}Et Yahweh parla à Moïse, en disant :
\VS{45}Retirez-vous du milieu de cette assemblée, et je les consumerai en un instant. Alors ils se prosternèrent le visage contre terre ;
\VS{46}puis Moïse dit à Aaron : Prends l'encensoir, et mets-y du feu de dessus l'autel, mets-y aussi du parfum, et va promptement à l'assemblée, et fais propitiation pour eux ; car une grande colère est sortie de devant Yahweh, la plaie a commencé.
\VS{47}Et Aaron prit l'encensoir, comme Moïse lui avait dit, et il courut au milieu de l'assemblée, et voici la plaie avait déjà commencé sur le peuple. Alors il mit du parfum et fit propitiation pour le peuple.
\VS{48}Et comme il se tenait entre les morts et les vivants, la plaie fut arrêtée.
\VS{49}Et il y en eut quatorze mille sept cents qui moururent de cette plaie, outre ceux qui étaient morts à cause de Koré.
\VS{50}Et Aaron retourna auprès de Moïse, à l'entrée de la tente d'assignation, et la plaie s’arrêta.
\Chap{17}
\TextTitle{Yahweh confirme l'appel d'Aaron, sa verge fleurit}
\VerseOne{}Après cela Yahweh parla à Moïse, en disant :
\VS{2}Parle aux enfants d’Israël, et prends une verge de chacun d’eux selon la maison de leur père, de tous ceux qui sont les princes, selon la maison de leurs pères, douze verges, puis tu écriras le nom de chacun sur sa verge,
\VS{3}mais tu écriras le nom d'Aaron sur la verge de Lévi\FTNT{La verge d'Aaron est une image du Messie ressuscité. Elle avait produit la vie tandis que celles des autres princes n'avaient produit aucun fruit. Cette histoire nous parle également de la confirmation de l'appel d'Aaron face aux critiques dont il était l'objet. On connaît l'arbre par ses fruits (Mt. 7:16-20 ; Lu. 7:17-22).}; car il y aura une verge pour chaque chef des maisons de leurs pères.
\VS{4}Et tu les déposeras dans la tente d'assignation, devant le témoignage, où je me rencontre avec vous.
\VS{5}Et la verge de l'homme que j'aurai choisi fleurira ; et je ferai cesser de devant moi les murmures des enfants d'Israël, par lesquels ils murmurent contre vous.
\VS{6}Quand Moïse parla aux enfants d'Israël, tous leurs princes lui donnèrent une verge, chaque prince une verge, selon les maisons de leurs pères, soit douze verges ; or la verge d'Aaron était au milieu des leurs.
\VS{7}Et Moïse mit les verges devant Yahweh, dans la tente du témoignage.
\VS{8}Et le lendemain, lorsque Moïse entra dans la tente du témoignage, voici, la verge d'Aaron, avait fleuri, pour la maison de Lévi, et elle avait poussé des boutons, produit des fleurs et mûri des amandes.
\VS{9}Alors Moïse ôta de devant Yahweh toutes les verges et les porta à tous les fils d'Israël, afin qu'ils les voient et qu'ils prennent chacun leurs verges.
\VS{10}Et Yahweh dit à Moïse : Reporte la verge d'Aaron devant le témoignage, pour être conservée comme un signe pour les fils de rébellion, afin que tu fasses cesser de devant moi leurs murmures et qu'ils ne meurent point\FTNT{Hé. 9:3-5.}.
\VS{11}Et Moïse fit ainsi ; il se conforma à l'ordre que Yahweh lui avait donné.
\VS{12}Les enfants d'Israël parlèrent à Moïse, en disant : Voici, nous expirons, nous périssons, nous périssons tous !
\VS{13}Quiconque s'approche du tabernacle de Yahweh, meurt. Serons-nous tous entièrement expirés ?
\Chap{18}
\TextTitle{Droits et devoirs des sacrificateurs et des Lévites}
\VerseOne{}Alors Yahweh dit à Aaron : Toi et tes fils, et la maison de ton père avec toi, vous porterez l'iniquité du sanctuaire ; et toi, et tes fils avec toi, vous porterez l'iniquité de votre sacrificature.
\VS{2}Fais aussi approcher de toi tes frères, la tribu de Lévi, qui est la tribu de ton père, afin qu'ils te soient attachés et qu'ils te servent, mais toi et tes fils avec toi, vous servirez devant la tente du témoignage.
\VS{3}Ils garderont ce que tu leur ordonneras de garder, et ce qu'il faut garder de toute la tente,  mais ils n'approcheront point des ustensiles du sanctuaire, ni de l'autel de peur qu'ils ne meurent, et que vous ne mouriez avec eux.
\VS{4}Ils te seront donc attachés, et ils garderont tout ce qu'il faut garder dans la tente d'assignation, selon tout le service du tabernacle et aucun étranger n'approchera de vous.
\VS{5}Mais vous prendrez garde à ce qu'il faut faire dans le sanctuaire, et à ce qu'il faut faire à l'autel, afin qu'il n'y ait plus d'indignation sur les enfants d'Israël.
\VS{6}Car quant à moi voici, j'ai pris vos frères, les Lévites, du milieu des enfants d'Israël, qui sont donnés en pur don pour Yahweh, afin qu'ils soient employés au service de la tente d'assignation.
\VS{7}Mais toi et tes fils avec toi, vous observerez la fonction de votre sacrificature en tout ce qui concerne l’autel et ce qui est au dedans du voile, et vous y ferez le service. J’établis votre sacrificature en office de pur don ; c’est pourquoi si un étranger en approche, on le fera mourir.
\VS{8}Yahweh dit encore à Aaron : Voici, je t'ai donné la garde de mes offrandes élevées sur toutes les choses consacrées par les enfants d'Israël ; je te les ai données, et à tes enfants, par ordonnance perpétuelle, à cause de l'onction.
\VS{9}Ceci t'appartiendra d'entre les choses très saintes qui ne sont pas brûlées, savoir toutes leurs offrandes, soit de tous leurs gâteaux, soit de tous leurs sacrifices pour l'expiation, et tous leurs sacrifices pour la culpabilité qu'ils m'apporteront ; ce sont des choses très saintes pour toi et pour tes enfants.
\VS{10}Vous les mangerez dans un lieu très saint ; tout mâle en mangera ; vous les regarderez comme saintes\FTNT{Lé. 6:17-22 ; Lé. 7:6 ; Lé. 10:13.}.
\VS{11}Voici encore ce qui t'appartiendra : Tous les dons que les enfants d'Israël présenteront par élévation et en les agitant de côté et d'autre, je te les donne à toi, à tes fils, et à tes filles avec toi, par une loi perpétuelle ; quiconque sera pur dans ta maison en mangera\FTNT{Lé. 7:34 ; Lé. 10:14.}.
\VS{12}Je te donne aussi leurs prémices qu'ils offriront à Yahweh : Tout ce qu'il y aura de meilleur en huile, et tout le meilleur du moût et du blé.
\VS{13}Les premiers fruits de toutes les choses que leur terre produira, et qu’ils apporteront à Yahweh t’appartiendront ; quiconque sera pur dans ta maison, en mangera.
\VS{14}Tout ce qui sera dévoué en Israël t'appartiendra\FTNT{Lé. 27:28 ; Ez. 44:29.}.
\VS{15}Tout premier-né de toute chair, qu'ils offriront à Yahweh, tant des hommes que des animaux t'appartiendront. Mais, tu feras racheter le premier-né de l'homme, et tu feras racheter le premier-né d'un animal impur.
\VS{16}Et ceux qui doivent être rachetés, depuis l’âge d’un mois, tu les rachèteras selon ton estimation que tu en feras, au prix de cinq sicles d'argent, selon le sicle du sanctuaire, qui est de vingt guéras.
\VS{17}Mais tu ne feras point racheter le premier-né du bœuf, ni le premier-né de la brebis, ni le premier-né de la chèvre : Ce sont des choses saintes. Tu répandras leur sang sur l'autel, et tu brûleras leur graisse : Ce sera un sacrifice consumé par le feu d'une bonne odeur à Yahweh.
\VS{18}Mais leur chair t'appartiendra, comme la poitrine qu'on agite de côté et d'autre, et comme l'épaule droite.
\VS{19}Je t'ai donné, à toi et à tes fils, et à tes filles avec toi, par une loi perpétuelle, toutes les offrandes présentées par élévation des choses sanctifiées, que les enfants d'Israël offriront à Yahweh. C'est une alliance de sel\FTNT{Le sel est un aliment pratiquement impérissable et incorruptible. Dans l'Antiquité, il symbolisait l'incorruptibilité (Lé. 2:13).} et à perpétuité devant Yahweh, pour toi et pour ta postérité avec toi.
\VS{20}Puis Yahweh dit à Aaron : Tu ne posséderas rien dans leur pays, et il n'y aura point de part pour toi au milieu d'eux ; c'est moi qui suis ta part et ta possession, au milieu des enfants d'Israël\FTNT{De. 10:9 ; De. 18:2 ; Ez. 44:28.}.
\TextTitle{Lois sur les dîmes (De. 14:22-29)}
\VS{21}Et je donne comme possession aux fils de Lévi, toutes les dîmes\FTNT{Il y avait plusieurs sortes de dîmes sous la loi Mosaïque Alliance :
— La 1ère dîme : Le peuple devait payer une dîme générale au bénéfice des Lévites (No. 18:21).
Toutes les tribus d'Israël, à l'exception des Lévites, eurent une possession géographique qu'ils reçurent comme héritage après leur entrée en Canaan. Mais les Lévites devaient accomplir une tâche particulière au sein de la nation. Ils devaient s'occuper du service dans la tente d'assignation. En compensation de ce service, ils devaient percevoir un impôt de 10\% des revenus de tous les Israélites.
— La 2ème dîme : Les Lévites devaient payer la « dîme de la dîme », au bénéfice des sacrificateurs (No. 18:25-31).
Tous les sacrificateurs étaient des Lévites, mais tous les Lévites n'étaient pas des sacrificateurs. Les sacrificateurs descendaient d'Aaron et ils exerçaient des responsabilités particulières dans la tente d'assignation, puis dans le temple. Cette seconde dîme permettait aux sacrificateurs d'être nourris et assurait donc le bon fonctionnement du service du temple.
— La 3ème dîme : Tous les Israélites devaient conserver une dîme de toute leur production en prévision de leurs pèlerinages annuels à Jérusalem (De. 14:22-26).
Trois fois par an, tout le peuple devait s'assembler à Jérusalem, l'endroit choisi par le Seigneur, à l'occasion des principales fêtes. Dieu avait prévu que chacun puisse disposer de ressources suffisantes pour leur permettre de se réjouir pleinement à ces occasions. C'est pour cela qu'ils devaient mettre de côté 10\% de leurs productions agricoles annuelles. Il est intéressant de noter que la dîme n'était jamais payée en argent, mais toujours en nature.
— La 4ème dîme : Il fallait payer une dîme spéciale à l'intention des pauvres, des orphelins et des veuves (De. 14:28-29). 
Certains affirment que la dîme existait bien avant la loi. Mais ils ignorent que la Bible parle de plusieurs sortes de lois.
— Les lois cérémonielles (Hé. 9:1)
Ces lois étaient relatives au culte et concernaient le tabernacle puis le temple, les sacrifices, les ablutions (Lé. 16 ; Hé. 9 : 1-10). Les dîmes (la dîme des sacrificateurs) devaient être amenées dans le temple (Mal. 3:10), elles faisaient donc partie des lois cérémonielles. Or les Lévites et les sacrificateurs de la Première Alliance n'existent plus sous la Nouvelle Alliance car les enfants de Dieu sont un royaume de rois et de sacrificateurs (Ap. 1:6 ; Ap. 5:10).
— Les lois morales (Ex. 20 : 1-17). Dieu est saint et il veut un peuple saint qui marche dans sa crainte, dans la sainteté et dans l'obéissance. Lé. 18 nous parle des lois morales ; elles n'ont pas été abolies, elles existent toujours. Elles sont inscrites dans la conscience de l'homme, elles sont gravées dans notre cœur (Hé. 8:10).
— Les lois sociales (Ex. 21:1-24). Ce sont des lois civiles régissant la vie sociale d'Israël, comme nous pouvons le lire dans Ex. 21 par exemple. Ces lois n'ont rien à voir avec les croyants de la Nouvelle Alliance. Les lois morales témoignent de la nature de Dieu, ce sont des lois éternelles qui existaient bien avant Abraham. Les lois cérémonielles ont commencé dès la fondation du monde (Ap. 13 : 8) car l'agneau de Dieu était immolé avant la fondation du monde (1 Pi. 1:19-20). Seules les lois sociales ont débuté avec Moïse car elles concernaient exclusivement les Israélites. Ces trois sortes de lois ont été institutionnalisées par Moïse, mais les deux premières (morales et cérémonielles) existaient avant ce dernier. Les quatre sortes de dîmes faisaient bel et bien partie des lois sociales et cérémonielles. Or, ces lois ne sont plus d'actualité sous la Nouvelle Alliance. En conclusion, nous pouvons dire que Jésus nous a rachetés en accomplissant les lois cérémonielles afin que nous pratiquions les lois morales (Ep. 2:10). Voir également commentaire en Mal 3 : 10.} d'Israël, pour le service auquel ils sont employés, le service de la tente d'assignation.
\VS{22}Et les enfants d'Israël n'approcheront plus de la tente d'assignation, afin qu'ils ne se chargent d'un péché et qu'ils ne meurent point.
\VS{23}Mais les Lévites s'emploieront au service de la tente d'assignation, et ils resteront chargés de leurs iniquités. Cette loi sera perpétuelle parmi vos descendants, et ils ne posséderont point d'héritage parmi les enfants d'Israël.
\VS{24}Car je donne comme possession aux Lévites les dîmes que les enfants d'Israël présenteront à Yahweh en offrande élevée ; c'est pourquoi je dis d'eux qu'ils n'auront point d'héritage parmi les fils d'Israël.
\VS{25}Puis Yahweh parla à Moïse, en disant :
\VS{26}Tu parleras aussi aux Lévites, et tu leur diras : Quand vous recevrez des enfants d'Israël les dîmes que je vous donne de leur part comme possession, vous en offrirez l'offrande élevée à Yahweh, la dîme de la dîme ;
\VS{27}et votre offrande élevée vous sera comptée comme le blé qu'on prélève de l'aire, et comme l'abondance qu'on prélève de la cuve.
\VS{28}C'est ainsi que vous prélèverez une offrande pour Yahweh de toutes les dîmes que vous recevrez des enfants d'Israël, et vous donnerez au sacrificateur Aaron l'offrande que vous en aurez prélevée pour Yahweh.
\VS{29}Sur tous les dons qui vous seront faits, vous prélèverez toute l'offrande élevée pour Yahweh ; sur tout ce qu'il y aura de meilleur, vous prélèverez la portion consacrée.
\VS{30}Et tu leur diras : Quand vous aurez offert en offrande élevée le meilleur de la dîme, pris de la dîme même, il sera imputé aux Lévites comme le revenu de l'aire, et comme le revenu de la cuve.
\VS{31}Et vous la mangerez en tout lieu, vous et votre maison ; car c'est votre salaire pour le service auquel vous êtes employés dans la tente d'assignation.
\VS{32}Vous ne serez point coupables de péché au sujet de la dîme, quand vous en aurez offert en offrande élevée sur ce qu'il y aura de meilleur et vous ne souillerez point les choses saintes des enfants d'Israël et vous ne mourrez point.
\Chap{19}
\TextTitle{La jeune vache rousse ; l'eau de purification}
\VerseOne{}Yahweh parla à Moïse et à Aaron, en disant :
\VS{2}Voici ce qui est ordonné par la loi que Yahweh a commandé, en disant : Parle aux enfants d'Israël, et dis-leur qu'ils t'amènent une jeune vache rousse, entière, sans défaut, et qui n'ait point porté le joug.
\VS{3}Puis vous la donnerez à Eléazar, le sacrificateur, qui la mènera hors du camp, et on l'égorgera en sa présence\FTNT{Lé. 4:12 ; Hé. 13:11-12.}.
\VS{4}Ensuite, Eléazar, le sacrificateur, prendra de son sang avec son doigt, et fera sept fois l'aspersion du sang vers le devant de la tente d'assignation.
\VS{5}Et on brûlera la jeune vache en sa présence ; on brûlera sa peau, sa chair, son sang et ses excréments\FTNT{Ex. 29:14.}.
\VS{6}Le sacrificateur prendra du bois de cèdre, de l'hysope, et du cramoisi, et les jettera dans le feu où sera brûlée la jeune vache.
\VS{7}Puis le sacrificateur lavera ses vêtements et sa chair avec de l'eau ; après cela, il rentrera au camp, et le sacrificateur sera impur jusqu'au soir.
\VS{8}Celui qui l'aura brûlé, lavera ses vêtements dans l'eau, il lavera aussi dans l'eau son corps ; et il sera impur jusqu'au soir.
\VS{9}Et un homme pur ramassera les cendres de la jeune vache, et les mettra hors du camp, dans un lieu pur ; elles seront gardées pour l'assemblée des enfants d'Israël ; afin d'en faire l'eau de purification. C'est une purification pour le péché.
\VS{10}Celui qui aura ramassé les cendres de la jeune vache, lavera ses vêtements, et sera impur jusqu'au soir ; ce sera une loi perpétuelle pour les enfants d'Israël, et pour l'étranger en séjour au milieu d'eux.
\VS{11}Celui qui touchera un mort, un corps humain quelconque, sera impur pendant sept jours\FTNT{Ag. 2:13.}.
\VS{12}Il se purifiera avec cette eau le troisième jour et le septième jour, et il sera pur ; mais s'il ne se purifie pas le troisième jour, il ne sera pas pur le septième jour.
\VS{13}Alors celui qui touchera un mort, le corps d'un homme qui sera mort et qui ne se purifiera pas, souille le tabernacle de Yahweh ; celui-là sera retranché d'Israël. Il est impur, car l'eau de purification n'a pas été répandue sur lui, son impureté demeure encore sur lui.
\VS{14}Voici la loi. Lorsqu'un homme mourra dans une tente, quiconque entrera dans la tente, et quiconque se trouvera dans la tente sera impur pendant sept jours.
\VS{15}Aussi tout vase découvert, sur lequel il n'y aura point de couvercle attaché, sera impur.
\VS{16}Et quiconque touchera, dans les champs, un homme qui aura été tué par l'épée, ou un mort, ou des ossements humains, ou un sépulcre, sera impur durant sept jours.
\VS{17}Et on prendra, pour celui qui est impur, de la poudre de la jeune vache brûlée pour faire la purification, et on la mettra dans un vase, avec de l'eau vive par-dessus.
\VS{18}Puis un homme pur prendra de l'hysope, et la trempera dans l'eau ; il en fera aspersion sur la tente, et sur tous les ustensiles, et sur toutes les personnes qui auront été là, et sur celui qui a touché des ossements ou un homme tué, ou un mort, ou un sépulcre.
\VS{19}Celui qui est pur fera l'aspersion sur celui qui est impur, le troisième jour et le septième jour, et il le purifiera le septième jour ; puis il lavera ses vêtements, et se lavera dans l'eau, et il sera pur le soir.
\VS{20}Mais l'homme qui sera impur, et qui ne se purifiera point, sera retranché du milieu de l'assemblée, parce qu'il a souillé le sanctuaire de Yahweh ; comme l'eau de purification n'a pas été répandue sur lui, il est impur.
\VS{21}Et ce sera pour eux une loi perpétuelle, et celui qui fera l'aspersion de l'eau de purification lavera ses vêtements ; et quiconque touchera l'eau de purification sera impur jusqu'au soir.
\VS{22}Et tout ce que l'homme impur touchera sera souillé, et la personne qui le touchera sera impure jusqu'au soir.
\Chap{20}
\TextTitle{Mort de Marie}
\VerseOne{}Or toute l'assemblée des enfants d'Israël arriva dans le désert de Tsin au premier mois, et le peuple s'arrêta à Kadès. Marie mourut là, et y fut ensevelie.
\TextTitle{Murmures du peuple à cause du manque d'eau\FTNTT{De. 32:51 ; cp. Ex 17:1-7}}
\VS{2}Et il n'y avait point d'eau pour l'assemblée ; et ils se soulevèrent contre Moïse et contre Aaron.
\VS{3}Et le peuple contesta contre Moïse et ils lui dirent : Pourquoi ne sommes-nous pas morts quand nos frères moururent devant Yahweh ?
\VS{4}Et pourquoi avez-vous fait venir l'assemblée de Yahweh dans ce désert, pour que nous y mourions, nous et notre bétail\FTNT{Ex. 17:3.} ?
\VS{5}Et pourquoi nous avez-vous fait monter hors d'Egypte, pour nous amener dans ce méchant lieu qui n'est pas un lieu où l'on puisse semer, ni un lieu pour des figuiers, ni pour des vignes, ni pour des grenadiers, et sans eau pour boire ?
\VS{6}Alors Moïse et Aaron se retirèrent de devant l'assemblée à l'entrée de la tente d'assignation et ils tombèrent sur leurs faces ; et la gloire de Yahweh apparut.
\TextTitle{Incrédulité de Moïse et d'Aaron à Meriba}
\VS{7}Yahweh parla à Moïse, en disant :
\VS{8}Prends la verge, et convoque l'assemblée, toi et Aaron, ton frère. Vous parlerez en leur présence au rocher\FTNT{Christ, le rocher des âges ( Es. 8:13-17 ; 1 Co. 10:1-4).}, et il donnera son eau ; ainsi tu leur feras sortir de l'eau du rocher, et tu donneras à boire à l'assemblée et à leur bétail.
\VS{9}Moïse prit la verge qui était devant Yahweh, comme il lui avait ordonné.
\VS{10}Moïse et Aaron convoquèrent l'assemblée devant le rocher. Et il leur dit : Ecoutez donc, rebelles ! Est-ce de ce rocher que nous vous ferons sortir de l'eau ?
\VS{11}Puis Moïse leva sa main, et frappa deux fois le rocher avec sa verge et  il en sortit des eaux en abondance. L'assemblée but, et leur bétail aussi.
\VS{12}Alors Yahweh dit à Moïse et à Aaron : Parce que vous n'avez pas cru en moi, pour me sanctifier aux yeux des enfants d'Israël, ainsi vous ne ferez point entrer cette assemblée dans le pays que je lui donne.
\VS{13}Ce sont là, les eaux de Meriba, où les enfants d'Israël contestèrent avec Yahweh, qui fut sanctifié en eux.
\TextTitle{La méchanceté d'Edom\FTNTT{Ge. 25:30 ; Ab. 10}}
\VS{14}De Kadès, Moïse envoya des messagers au roi d'Edom, pour lui dire : Ainsi parle ton frère Israël : Tu sais tout le travail que nous avons eu.
\VS{15}Nos pères descendirent en Egypte, et nous y demeurâmes longtemps. Mais les Egyptiens nous ont maltraités, nous et nos pères.
\VS{16}Et nous avons crié à Yahweh, et il a entendu notre voix. Il a envoyé l'ange et nous a fait sortir d'Egypte. Et voici, nous sommes à Kadès, ville qui est à l'extrémité de ton territoire\FTNT{Ex. 2:23 ; Ex. 23:20 ; Ac. 7:30-38.}.
\VS{17}Je te prie, laisse-nous passer par ton pays ; nous ne traverserons ni les champs ni les vignes, et nous ne boirons pas l'eau des puits ; nous marcherons par le chemin royal ; nous ne nous détournerons ni à droite ni à gauche, jusqu'à ce que nous ayons passé ton territoire.
\VS{18}Et Edom lui dit : Tu ne passeras point par mon pays, sinon je sortirai à ta rencontre avec l'épée.
\VS{19}Les enfants d'Israël lui répondirent : Nous monterons par le grand chemin, et si nous buvons de tes eaux, moi et mes bêtes, je t'en payerai le prix ; je ne ferai que passer avec mes pieds, pas autre chose.
\VS{20}Mais il lui répondit : Tu ne passeras pas ! Edom sortit à sa rencontre avec un peuple nombreux, et à main armée.
\VS{21}Ainsi Edom ne voulut point permettre à Israël de passer par ses frontières ; c’est pourquoi Israël se détourna de lui.
\VS{22}Et toute l'assemblée des enfants d'Israël partit de Kadès et arriva à la montagne de Hor.
\TextTitle{Mort d'Aaron}
\VS{23}Et Yahweh parla à Moïse et à Aaron à la montagne de Hor, près des frontières du pays d'Edom, en disant :
\VS{24}Aaron sera recueilli auprès de son peuple, car il n'entrera pas dans le pays que je donne aux enfants d'Israël, parce que vous avez été rebelles à mon commandement aux eaux de Meriba.
\VS{25}Prends donc Aaron et Eléazar, son fils, et fais-les monter sur la montagne de Hor.
\VS{26}Puis dépouille Aaron de ses vêtements, et fais-les revêtir à Eléazar, son fils. C'est là qu'Aaron sera recueilli et qu'il mourra.
\VS{27}Moïse fit ce que Yahweh avait ordonné ; et ils montèrent sur la montagne de Hor, aux yeux de toute l'assemblée.
\VS{28}Et Moïse dépouilla Aaron de ses vêtements et en fit revêtir Eléazar, son fils. Aaron mourut là, au sommet de la montagne. Moïse et Eléazar descendirent de la montagne\FTNT{De. 10:6.}.
\VS{29}Toute l'assemblée, toute la maison d'Israël, voyant qu'Aaron était mort, le pleurèrent trente jours.
\Chap{21}
\TextTitle{Les Cananéens livrés à Israël}
\VerseOne{}Quand le roi d'Arad, Cananéen, qui habitait le midi, eut appris qu'Israël venait par le chemin d'Atharim, il combattit Israël et emmena des prisonniers.
\VS{2}Alors Israël fit un vœu à Yahweh, en disant : Si tu livres ce peuple entre mes mains, je dévouerai ses villes par le moyen de l'interdit.
\VS{3}Et Yahweh exauça la voix d'Israël et livra entre ses mains les Cananéens. On les dévoua par interdit, avec leurs villes ; et on donna à ce lieu le nom de Horma.
\TextTitle{Le serpent d'airain\FTNTT{Jn. 3:14-15 ; 2 Co. 5:20}}
\VS{4}Puis ils partirent de la montagne de Hor, par le chemin de la Mer Rouge, pour faire le tour du pays d'Edom. Le cœur du peuple s'impatienta en route,
\VS{5}et parla contre Dieu, et contre Moïse, en disant : Pourquoi nous as-tu fait monter hors d'Egypte, pour mourir dans ce désert ? Car il n'y a point de pain ni d'eau, et notre âme est dégoûtée de cette nourriture misérable.
\VS{6}Et Yahweh envoya contre le peuple des serpents brûlants qui mordaient le peuple ; tellement qu'il en mourut un grand nombre en Israël\FTNT{1 Co. 10:9.}.
\VS{7}Alors le peuple vint vers Moïse, et dit : Nous avons péché, car nous avons parlé contre Yahweh et contre toi. Invoque Yahweh afin qu'il éloigne de nous les serpents et Moïse pria pour le peuple.
\VS{8}Et Yahweh dit à Moïse : Fais-toi un serpent brûlant, et mets-le sur une perche ; quiconque aura été mordu et le regardera conservera la vie.
\VS{9}Moïse fit un serpent d'airain\FTNT{Voir Jn. 3:14-16. Ceux qui regardent à Jésus-Christ, et non aux hommes, obtiennent la délivrance. L'airain nous parle du jugement (Job. 20:24), le serpent de la malédiction (Gn. 3:14), et la perche parle de la croix (1 Co. 1:18). Jésus a pris nos malédictions sur la croix de Golgotha (Ga. 3:13).}, et le mit sur une perche ; quiconque avait été mordu par un serpent et regardait le serpent d'airain conservait la vie.
\VS{10}Les enfants d'Israël partirent et campèrent à Oboth.
\VS{11}Et ils partirent d'Oboth et ils campèrent en Ijjé-Abarim, dans le désert qui est vis-à-vis de Moab, vers le soleil levant.
\VS{12}Puis Ils partirent de là et campèrent vers le torrent de Zéred.
\VS{13}Et ils partirent de là et campèrent de l'autre côté d'Arnon, qui est dans le désert, en sortant du territoire des Amoréens ; car Arnon est la frontière de Moab, entre les Moabites et les Amoréens\FTNT{Jg. 11:18.}.
\VS{14}C'est pourquoi il est dit dans le livre des batailles de Yahweh : Vaheb en Supha, et les torrents de l'Arnon,
\VS{15}et le cours des torrents qui s'étend du côté d'Ar et touche à la frontière de Moab.
\VS{16}De là ils allèrent à Beer. C'est ce Beer où Yahweh dit à Moïse : Rassemble le peuple, et je leur donnerai de l'eau.
\VS{17}Alors Israël chanta ce cantique : Monte, puits ! chantez-lui en vous répondant les uns aux autres.
\VS{18}Puits que des princes ont creusés. Que les grands du peuple ont creusé, avec le sceptre, avec leurs bâtons ! Du désert ils vinrent à Matthana ;
\VS{19}de Matthana à Nahaliel ; et de Nahaliel à Bamoth ;
\VS{20}de Bamoth à la vallée qui est dans le territoire de Moab, au sommet de Pisga, et qui regarde vers Jeshimon.
\TextTitle{Israël bat le roi des Amoréens et le roi de Basan}
\VS{21}Puis Israël envoya des messagers à Sihon, roi des Amoréens, pour lui dire :
\VS{22}Laisse-moi passer par ton pays ; nous ne nous détournerons ni dans les champs, ni dans les vignes, et nous ne boirons pas l'eau des puits ; mais nous marcherons par la route royale, jusqu'à ce que nous ayons passé ton territoire.
\VS{23}Mais Sihon ne permit pas à Israël de passer sur son territoire ; il rassembla tout son peuple et sortit à la rencontre d'Israël, dans le désert ; il vint à Jahats, et combattit Israël\FTNT{De. 2:26-30 ; Jg. 11:29-30.}.
\VS{24}Israël le fit passer au fil de l'épée et conquit son pays, depuis l'Arnon jusqu'à Jabbok, et jusqu'à la frontière des fils d'Ammon ; car la frontière des fils d'Ammon était forte\FTNT{De. 2:30 ; De. 29:7 ; Ps. 135:11-12.}.
\VS{25}Et Israël prit toutes les villes et s'établit dans toutes les villes des Amoréens, à Hesbon, et dans toutes les villes de son ressort.
\VS{26}Or Hesbon était la ville de Sihon, roi des Amoréens ; il avait fait la guerre au précédent roi de Moab, et lui avait enlevé tout son pays jusqu'à l'Arnon.
\VS{27}C'est pourquoi les poètes disent : Venez à Hesbon ! Que la ville de Sihon soit rebâtie et fortifiée !
\VS{28}Car le feu est sorti de Hesbon, et la flamme de la cité de Sihon ; elle a consumé Ar-Moab, les habitants des hauteurs de l'Arnon.
\VS{29}Malheur à toi, Moab ! Peuple de Kemosch, tu es perdu ! il a livré ses fils qui se sauvaient et ses filles en captivité à Sihon, roi des Amoréens\FTNT{Jé. 48:46.}.
\VS{30}Nous les avons défaits à coups de flèches : De Hesbon à Dibon tout est détruit ; nous les avons mis en déroute jusqu'à Nophach, jusqu'à Médeba.
\VS{31}Israël s'établit dans le pays des Amoréens.
\VS{32}Puis Moïse envoya des gens pour reconnaître Jaezer, ils prirent les villes de son ressort, et chassèrent les Amoréens qui y étaient.
\VS{33}Ensuite, ils se tournèrent et montèrent par le chemin de Basan. Og, roi de Basan, sortit à leur rencontre, avec tout son peuple pour les combattre à Edréï.
\VS{34}Et Yahweh dit à Moïse : Ne le crains point, car je le livre entre tes mains, lui et tout son peuple, et son pays ; tu le traiteras comme tu as traité Sihon, roi des Amoréens, qui habitait à Hesbon\FTNT{De. 3:1-2.}.
\VS{35}Ils le battirent donc, lui et ses fils, et tout son peuple, sans en laisser échapper un seul, et ils s'emparèrent de son pays.
\Chap{22}
\TextTitle{Balak cherche à maudire Israël ; Balaam\FTNTT{2 Pi. 2:15 ; Jud. 11 ; Ap. 2:14} séduit par les honneurs}
\VerseOne{}Puis les enfants d'Israël partirent, et ils campèrent dans les plaines de Moab, au-delà du Jourdain, vis-à-vis de Jéricho.
\VS{2}Balak, fils de Tsippor, vit tout ce qu'Israël avait fait aux Amoréens.
\VS{3}Et Moab eut une grande frayeur du peuple, parce qu'il était en grand nombre, il fut saisi de terreur en face des enfants d'Israël.
\VS{4}Et Moab dit aux anciens de Madian : Maintenant cette multitude va brouter tout ce qui nous entoure, comme le bœuf broute l'herbe des champs. Balak, fils de Tsippor, était alors roi de Moab.
\VS{5}Il envoya des messagers auprès de Balaam, fils de Beor, à Pethor, située sur le fleuve, dans le pays des fils de son peuple, afin de l'appeler et de lui dire : Voici, un peuple est sorti d'Egypte, il couvre la surface de la terre, et il habite vis-à-vis de moi.
\VS{6}Viens donc maintenant, je te prie, maudis-moi ce peuple, car il est plus puissant que moi ; peut-être que je serai le plus fort, et que nous le battrons, et que je le chasserai du pays ; car je sais que celui que tu bénis est béni, et que celui que tu maudis est maudit.
\VS{7}Les anciens de Moab s'en allèrent avec les anciens de Madian, ayant dans leurs mains de quoi payer le devin. Ils arrivèrent auprès de Balaam, et lui rapportèrent les paroles de Balak.
\VS{8}Il leur répondit : Demeurez ici cette nuit, et je vous répondrai d'après ce que Yahweh me dira. Et les chefs des Moabites restèrent chez Balaam.
\VS{9}Et Dieu vint à Balaam et dit : Qui sont ces hommes que tu as chez toi ?
\VS{10}Et Balaam répondit à Dieu : Balak, fils de Tsippor, roi de Moab, les a envoyés pour me dire :
\VS{11}Voici, un peuple qui est sorti d'Egypte, et qui couvre la face de la terre ; viens donc, maudis-le-moi ; peut-être qu'ainsi je pourrai le combattre, et je le chasserai.
\VS{12}Et Dieu dit à Balaam : Tu n'iras point avec eux, et tu ne maudiras point ce peuple, car il est béni.
\VS{13}Et Balaam se leva le matin, et il dit aux chefs qui avaient été envoyés par Balak : Retournez dans votre pays, car Yahweh refuse de me laisser venir avec vous.
\VS{14}Ainsi les chefs des Moabites se levèrent et retournèrent auprès de Balak, et dirent : Balaam a refusé de venir avec nous.
\VS{15}Et Balak envoya encore des chefs en plus grand nombre, et plus considérés que les premiers.
\VS{16}Ils arrivèrent auprès de Balaam, et lui dirent : Ainsi parle Balak, fils de Tsippor : Que l'on ne t'empêche donc pas de venir vers moi ;
\VS{17}car je te rendrai beaucoup d'honneurs, et je ferai tout ce que tu me diras ; je te prie donc viens, maudis-moi ce peuple.
\VS{18}Et Balaam répondit et dit aux serviteurs de Balak : Quand Balak me donnerait sa maison pleine d'or et d'argent, je ne pourrais point transgresser l'ordre de Yahweh, mon Dieu ; je ne pourrais faire aucune chose, ni petite ni grande.
\VS{19}Toutefois, je vous prie, demeurez maintenant ici encore cette nuit, et je saurai ce que Yahweh aura de plus à me dire.
\VS{20}Dieu vint, la nuit à Balaam, et lui dit : Puisque ces hommes sont venus t'appeler, lève-toi, et va avec eux ; mais quoi qu'il en soit, tu feras ce que je te dirai.
\VS{21}Ainsi Balaam se leva le matin, et sella son ânesse, et partit avec les chefs de Moab.
\VS{22}Mais la colère de Dieu s'enflamma parce qu'il était parti ; et l'Ange de Yahweh se plaça sur le chemin pour lui résister. Balaam était monté sur son ânesse, et ses deux serviteurs étaient avec lui.
\VS{23}L'ânesse vit l'Ange de Yahweh qui se tenait sur le chemin, son épée nue dans la main ; elle se détourna du chemin et alla dans les champs. Balaam frappa l'ânesse pour la ramener dans le chemin\FTNT{2 Pi. 2:16 ; Jud. 1:11.}.
\VS{24}L'Ange de Yahweh se plaça dans un sentier entre les vignes ; il y avait un mur de chaque côté.
\VS{25}Mais l'ânesse vit l'Ange de Yahweh ; elle se serra contre le mur, et elle serra le pied de Balaam contre le mur. Balaam la frappa de nouveau.
\VS{26}Et l'Ange de Yahweh passa plus loin et s'arrêta dans un lieu étroit où il n'y avait point d'espace pour se détourner à droite ou à gauche.
\VS{27}Et l'ânesse vit l'Ange de Yahweh, et elle s'abattit sous Balaam. Balaam se mit en grande colère, et il frappa l'ânesse avec son bâton.
\VS{28}Alors Yahweh ouvrit la bouche de l'ânesse, et elle dit à Balaam : Que t'ai-je fait, pour que tu m'aies déjà frappée trois fois ?
\VS{29}Et Balaam répondit à l'ânesse : C'est parce que tu t'es moquée de moi ; si j'avais une épée dans la main, je te tuerais sur le champ.
\VS{30}Et l'ânesse dit à Balaam: Ne suis-je pas ton ânesse, sur laquelle tu montes depuis que je suis à toi jusqu'à aujourd'hui? Ai-je l'habitude de te faire ainsi ? Et il répondit : Non.
\VS{31}Alors Yahweh ouvrit les yeux de Balaam, et il vit l'Ange de Yahweh qui se tenait sur le chemin, et qui avait dans sa main son épée nue ; et il s'inclina et se prosterna sur son visage.
\VS{32}Et l'Ange de l'Eternel lui dit : Pourquoi as-tu frappé ton ânesse déjà trois fois ? Voici je suis sorti pour m'opposer à toi ; car ta voie est devant moi une voie de perdition.
\VS{33}Mais l'ânesse m'a vu et elle s'est détournée de devant moi déjà trois fois ; autrement, si elle ne s'était détournée de moi, je t'aurais même déjà tué, et je lui aurais laissé la vie.
\VS{34}Alors Balaam dit à l'Ange de Yahweh : J'ai péché, car je ne savais point que tu t'étais placé au-devant de moi sur le chemin ; et maintenant, si cela est mauvais à tes yeux, je m’en retournerai.
\VS{35}L'Ange de Yahweh dit à Balaam : Va avec ces hommes ; mais tu ne feras que répéter les paroles que je te dirai. Et Balaam alla avec les chefs envoyés par Balak.
\VS{36}Et quand Balak apprit que Balaam arrivait, il sortit à sa rencontre jusqu'à la ville de Moab, qui est sur la limite de l'Arnon, à l'extrême frontière.
\VS{37}Et Balak dit à Balaam : N'ai-je pas auparavant envoyé vers toi pour t'appeler ? Pourquoi n'es-tu pas venu vers moi ? Ne puis-je donc pas te traiter avec honneur ?
\VS{38}Et Balaam répondit à Balak : Je suis venu vers toi ; mais pourrais-je maintenant dire quelque chose ? Je ne dirai que ce que Dieu m'aura mis dans la bouche.
\VS{39}Et Balaam alla avec Balak, et ils arrivèrent dans la cité de Kirjath-Hutsoth.
\VS{40}Et Balak sacrifia des bœufs et des brebis, et il en envoya à Balaam et aux chefs qui étaient venus avec lui.
\VS{41}Le matin, Balak prit Balaam, et le fit monter à Bamoth-Baal, d'où Balaam vit une partie du peuple.
\Chap{23}
\TextTitle{Balaam ne maudit pas mais bénit Israël des hauts lieux de Baal}
\VerseOne{}Et Balaam dit à Balak : Bâtis-moi ici sept autels, et prépare-moi ici sept veaux et sept béliers.
\VS{2}Balak fit ce que Balaam avait dit ; et Balak offrit avec Balaam un veau et un bélier sur chaque autel.
\VS{3}Balaam dit à Balak : Tiens-toi près de ton holocauste, et je m'éloignerai ; peut-être que Yahweh viendra à ma rencontre, et je te dirai tout ce qu'il me révélera. Et il alla sur un lieu élevé.
\VS{4}Et Dieu vint au-devant de Balaam, et Balaam lui dit : J'ai dressé sept autels, et j'ai sacrifié un veau et un bélier sur chaque autel.
\VS{5}Et Yahweh mit des paroles dans la bouche de Balaam et lui dit : Retourne vers Balak, et tu parleras ainsi.
\VS{6}Il retourna vers lui ; Balak se tenait près de son holocauste, avec tous les chefs de Moab.
\VS{7}Alors Balaam prononça son discours sentencieux et dit : Balak, roi de Moab, m'a fait descendre d'Aram\FTNT{De l'hébreu « Aram » traduit par « Aram » ou « Syrie » (1 R. 11:25)}, des montagnes d'orient, en me disant : Viens, maudis-moi Jacob ! Viens, sois irrité contre Israël !
\VS{8}Mais comment le maudirai-je ? Dieu ne l'a point maudit. Comment le détesterai-je ? Yahweh ne l'a point détesté.
\VS{9}Car je le vois du sommet des rochers, et je le contemple du haut des collines : C'est un peuple qui a sa demeure à part, et il ne sera pas compté parmi les nations\FTNT{De. 33:28.}.
\VS{10}Qui peut compter la poussière de Jacob, et dire le nombre du quart d'Israël ? Que je meure de la mort des justes, et que ma fin soit semblable à la leur !
\VS{11}Alors Balak dit à Balaam : Que m'as-tu fait ? Je t'ai pris pour maudire mon ennemi, et voici, tu le bénis très expressément !
\VS{12}Et il répondit et dit : N'aurai-je pas soin de dire ce que Yahweh met dans ma bouche ?
\TextTitle{Balaam bénit Israël au sommet de Pisga}
\VS{13}Balak lui dit : Viens, je te prie, avec moi dans un autre lieu d'où tu le verras, tu n'en verras qu'une partie, tu n'en verras pas la totalité. Et de là maudis-le-moi.
\VS{14}Puis il le mena au champ de Tsophim, sur le sommet de Pisga ; il bâtit sept autels, et offrit un taureau et un bélier sur chaque autel.
\VS{15}Alors Balaam dit à Balak : Tiens-toi ici près de ton holocauste, et j'irai à la rencontre de Dieu, comme j'ai déjà fait.
\VS{16}Yahweh vint au-devant de Balaam, il mit des paroles dans sa bouche et dit : Retourne vers Balak, et tu parleras ainsi.
\VS{17}Il retourna vers lui ; et voici, Balak se tenait près de son holocauste, et les chefs de Moab avec lui. Balak lui dit : Qu'est-ce que Yahweh a dit ?
\VS{18}Alors il prononça son discours sentencieux et dit : Lève-toi, Balak, écoute! Fils de Tsippor, prête-moi l'oreille !
\VS{19}Dieu n'est point un homme pour mentir ni fils d'un homme pour se repentir. Ce qu'il a dit, ne le fera-t-il pas? Ce qu'il a déclaré, ne l'exécutera-t-il pas\FTNT{Ja. 1:17.} ?
\VS{20}Voici, j'ai reçu l'ordre de bénir : Il a béni, je ne le révoquerai point.
\VS{21}Il n'a point aperçu d'iniquité en Jacob, il ne voit point de perversité en Israël ; Yahweh, son Dieu, est avec lui, et il y a en lui un chant de triomphe royal\FTNT{Jé. 50:20 ; Ro. 4:7.}.
\VS{22}Dieu les a fait sortir d'Egypte, il est pour eux comme la vigueur du buffle.
\VS{23}L'enchantement ne peut rien contre Jacob ni la divination contre Israël. Au temps marqué, il sera dit à Jacob et à Israël : Quelle est l'œuvre de Yahweh?
\VS{24}Voici, ce peuple se lèvera comme un vieux lion, et se dressera comme un lion qui est dans sa force ; il ne se couchera pas jusqu'à ce qu'il ait dévoré la proie, et bu le sang des blessés à mort.
\VS{25}Balak dit à Balaam : Et bien, ne le maudis pas, mais du moins ne le bénis pas.
\VS{26}Et Balaam répondit à Balak : Ne t'ai-je pas parlé ainsi : Je ferai tout ce que Yahweh dira ?
\TextTitle{Balaam bénit Israël de Peor}
\VS{27}Balak dit encore à Balaam : Viens donc, je te conduirai dans un autre lieu ; peut-être que Dieu trouvera bon que tu me le maudisses de là.
\VS{28}Balak conduisit Balaam sur le sommet de Peor, qui regarde du côté de Jeshimon.
\VS{29}Balaam lui dit : Bâtis-moi ici sept autels, et apprête-moi ici sept veaux et sept béliers.
\VS{30}Balak fit donc comme Balaam lui avait dit ; puis il offrit un taureau et un bélier sur chaque autel.
\Chap{24}
\VerseOne{}Or Balaam vit que Yahweh voulait bénir Israël, et il n'alla plus comme les autres fois à la rencontre des enchantements ; mais il tourna son visage du côté du désert.
\VS{2}Balaam leva les yeux, il vit Israël qui se tenait rangé selon ses tribus. Alors l'Esprit de Dieu fut sur lui.
\VS{3}Et il prononça à haute voix son discours sentencieux et dit : Parole de Balaam, fils de Beor, parole de l'homme qui a l'œil ouvert,
\VS{4}parole de celui qui entend les paroles de Dieu, qui voit la vision du Tout-Puissant, qui se prosterne et dont les yeux s'ouvrent.
\VS{5}Que tes tentes sont belles, ô Jacob! Tes demeures, ô Israël !
\VS{6}Elles sont étendues comme des torrents, comme des jardins près d'un fleuve, comme des arbres d'aloès que Yahweh a plantés, comme des cèdres auprès des eaux.
\VS{7}L'eau coule de ses seaux, et sa semence est parmi d'abondantes eaux. Son roi s'élève au-dessus d'Agag, et son royaume devient puissant.
\VS{8}Dieu, qui l'a fait sortir d'Égypte, est pour lui comme la vigueur du buffle; il consumera les nations qui sont ses ennemies; il brisera leurs os, et les percera de ses flèches.
\VS{9}Il s'est courbé, il s'est couché comme un lion qui est dans sa force, et comme un vieux lion ; qui le réveillera ? Quiconque te bénit, sera béni, et quiconque te maudit, sera maudit.
\VS{10}Alors Balak se mit très en colère contre Balaam, il frappa des mains et Balak parla ainsi à Balaam : Je t’ai appelé pour maudire mes ennemis, et voici, tu les as bénis, et bénis trois fois déjà.
\VS{11}Fuis maintenant, va-t'en chez toi ! J'avais dit que je te rendrais des honneurs, mais Yahweh t'empêche d’être honoré.
\VS{12}Et Balaam répondit à Balak : N'ai-je pas dit à tes messagers que tu m'as envoyés :
\VS{13}Quand Balak me donnerait sa maison pleine d'argent et d'or, je ne pourrais transgresser l'ordre de Yahweh pour faire de moi-même du bien ou du mal ; mais ce que Yahweh dira, je le dirai.
\VS{14}Maintenant donc je m'en vais vers mon peuple. Viens, je te donnerai un conseil, et je te dirai ce que ce peuple fera à ton peuple, dans les derniers jours.
\TextTitle{Prophétie sur le Roi qui sort de Jacob, le Messie}
\VS{15}Alors il prononça son discours sentencieux et dit : Parole de Balaam, fils de Beor, parole de l'homme qui a l'œil ouvert,
\VS{16}parole de celui qui entend les paroles de Dieu, qui connaît la science du Très-Haut, qui voit la vision du Tout-Puissant, qui se prosterne et dont les yeux sont ouverts.
\VS{17}Je le vois, mais pas maintenant ; je le contemple, mais pas de près ; une Etoile sort de Jacob\FTNT{L'Etoile en question est Jésus-Christ, qui se révéla à Jean comme l'Etoile brillante du matin (Ap. 22:16).}, et un Sceptre s'élève d'Israël. Il transpercera les cotés de Moab, il détruira tous les fils de Seth.
\VS{18}Edom est sa propriété, Séir est sa propriété par ses ennemis. Israël manifeste sa force.
\VS{19}Et il y en aura un de Jacob qui dominera, il fait périr le reste de la ville.
\VS{20}Il vit Amalek, il prononça son discours sentencieux et dit : Amalek est la première des nations, mais un jour il sera détruit.
\VS{21}Balaam vit aussi les Kéniens\FTNT{Il y a plusieurs sens à ce mot :\\Caïn = « possession », « artisan, forgeron », fils d'Adam \\Kéniens = «  forgerons »tribu du beau-père de Moïse qui vivait dans la région  du sud de la Palestine.}. Il prononça sa parabole et dit : Ta demeure est solide, et tu as mis ton nid dans le rocher ;
\VS{22}toutefois, le Kénien sera consumé, quand l'Assyrien t'emmènera en captivité.
\VS{23}Balaam continua à prononcer sa parabole et il dit : Malheur à celui qui vivra quand Dieu fera ces choses.
\VS{24}Et des navires viendront de Kittim, et ils humilieront l'Assyrien et l'Hébreu ; et lui aussi sera détruit.
\VS{25}Puis Balaam se leva, partit et retourna chez lui. Balak s'en alla aussi par son chemin.
\Chap{25}
\TextTitle{Prostitution d'Israël à Baal-Peor\FTNTT{No. 31:16 ; Ja. 4:4 ; Ap. 2:14}}
\VerseOne{}Alors Israël demeurait à Sittim ; et le peuple commença à se livrer la fornication avec les filles de Moab.
\VS{2}Car elles convièrent le peuple aux sacrifices de leurs dieux ; et le peuple mangea et se prosterna devant leurs dieux.
\VS{3}Et Israël s'attacha à Baal-Peor, c'est pourquoi la colère de Yahweh s'enflamma contre Israël\FTNT{Ps. 106:28 ; Os. 9:10.}.
\VS{4}Et Yahweh dit à Moïse : Prends tous les chefs du peuple, et fais-les pendre  devant Yahweh en face du soleil, afin que la colère de Yahweh se détourne d'Israël\FTNT{De. 4:3 ; Jos. 22:17.}.
\VS{5}Moïse dit aux juges d'Israël : Que chacun de vous fasse mourir les hommes qui sont à sa charge, et qui se sont joints à Baal-Peor.
\VS{6}Et voici, un homme des enfants d'Israël vint, et amena vers à ses frères une Madianite, devant Moïse et devant toute l'assemblée des fils d'Israël, tandis qu'ils pleuraient à l'entrée de la tente d'assignation.
\VS{7}Ce que Phinées, fils d'Eléazar, fils d'Aaron le sacrificateur, ayant vu, se leva du milieu de l'assemblée et prit une lance dans sa main.
\VS{8}Et il entra dans la tente de l'homme Israélite et les transperça tous deux, l'homme Israélite puis la femme, par le ventre. Et la plaie s'arrêta parmi les enfants d'Israël\FTNT{Ps. 106:30.}.
\VS{9}Or il y en eut vingt-quatre mille qui moururent de cette plaie.
\VS{10}Et Yahweh parla à Moïse, en disant :
\VS{11}Phinées, fils d'Eléazar, fils d'Aaron, le sacrificateur, a détourné ma colère de dessus les enfants d'Israël, parce qu'il a été animé de mon zèle au milieu d'eux ; et je n'ai point, dans mon ardeur, consumé les fils d'Israël.
\VS{12}C’est pourquoi, dis-lui : Voici, je lui donne mon alliance de paix.
\VS{13}Et l'alliance de sacrificature perpétuelle sera tant pour lui que pour sa postérité après lui, parce qu'il a été animé de zèle pour son Dieu, et qu'il a fait propitiation pour les enfants d'Israël.
\VS{14}Et le nom de l'homme Israélite tué, lequel fut tué avec la Madianite, était Zimri, fils de Salu, chef d'une maison de père des Siméonites.
\VS{15}Et le nom de la femme Madianite qui fut tuée était Cozbi, fille de Tsur, chef du peuple, et de maison de père en Madian.
\VS{16}Yahweh parla à Moïse, en disant :
\VS{17}Mettez en détresse les Madianites, tuez-les ;
\VS{18}car ils ont montré de l’hostilité envers vous, en vous trompant par leurs ruses ; dans l'affaire de Peor, et dans l'affaire de Cozbi, fille d'un chef d'entre les Madianites, leur sœur, qui a été tuée le jour de la plaie arrivée pour le fait de Peor.
\Chap{26}
\TextTitle{Nouveau dénombrement des hommes de guerre}
\VerseOne{} Or il arriva qu'après cette plaie-là, Yahweh parla à Moïse, et à Eléazar, fils d'Aaron, le sacrificateur, en disant :
\VS{2}Faites le dénombrement de toute l'assemblée des enfants d'Israël, depuis l'âge de vingt ans et au-dessus, selon les maisons de leurs pères, à savoir de tous ceux d'Israël qui peuvent aller à la guerre.
\VS{3}Moïse donc et Eléazar, le sacrificateur, leur parlèrent dans les plaines de Moab, près du Jourdain de Jéricho, en disant :
\VS{4}Qu’on fasse le dénombrement depuis l'âge de vingt ans et au-dessus, comme Yahweh l'avait ordonné à Moïse et aux enfants d'Israël, quand ils furent sortis du pays d'Egypte.
\VS{5}Ruben, premier-né d'Israël. Fils de Ruben : Hénoc, de qui descend la famille des Hénokites ; Pallu, de qui descend la famille des Palluites ;
\VS{6}Hetsron, de qui descend la famille des Hetsronites ; Carmi, de qui descend la famille des Carmites.
\VS{7}Ce sont là les familles des Rubénites : Ceux qui furent dénombrés étaient quarante-trois mille sept cent trente.
\VS{8}Et les fils de Pallu : Eliab.
\VS{9}Fils d'Eliab : Nemuel, Dathan et Abiram. Ce Dathan et cet Abiram, qui étaient de ceux qu’on appelait pour tenir l’assemblée, et qui se révoltèrent contre Moïse et contre Aaron dans l'assemblée de Koré, lors de leur révolte contre Yahweh.
\VS{10}Et lorsque la terre ouvrit sa bouche et les engloutit, ainsi que Koré, ceux qui s'étaient assemblés avec lui moururent. Et le feu dévora les deux cent cinquante hommes qui servirent d'avertissement.
\VS{11}Mais les fils de Koré ne moururent pas.
\VS{12}Les fils de Siméon selon leurs familles : De Nemuel descend la famille des Némuélites; de Jamin, la famille des Jaminites; de Jakin, la famille des Jakinites ;
\VS{13}de Zérach, la famille des Zérachites ; de Saül, la famille des Saülites.
\VS{14}Ce sont là les familles des Siméonites, qui furent vingt-deux mille deux cents.
\VS{15}Fils de Gad selon leurs familles. De Tsephon, descend la famille des Tsephonites ; de Haggi, la famille des Haggites ; de Schuni, la famille des Schunites ;
\VS{16}d'Ozni, la famille des Oznites ; d'Eri, la famille des Erites ;
\VS{17}D'Arod, la famille des Arodites ; d'Areéli, la famille des Areélites.
\VS{18}Ce sont là les familles des fils de Gad, d'après leur dénombrement : Quarante mille cinq cents.
\VS{19}Fils de Juda, Er, et Onan ; mais Er et Onan moururent au pays de Canaan\FTNT{Ge. 38:7-10 ; Ge. 46:12.}.
\VS{20}Voici les fils de Juda selon leurs familles : De Schéla descend la famille des Schélanites ; de Pérets, la famille des Péretsites ; de Zérach, la famille des Zérachites.
\VS{21}Les fils de Pérets furent : Hetsron, de qui descend la famille des Hetsronites ; Hamul, de qui descend la famille des Hamulites.
\VS{22}Ce sont là les familles de Juda, selon leur dénombrement: Soixante-seize mille cinq cents.
\VS{23}Fils d'Issacar, selon leurs familles : De Thola descend la famille des Tholaïtes ; de Puva, la famille des Puvites ;
\VS{24}de Jaschub, la famille des Jaschubites ; de Schimron, la famille des Schimronites.
\VS{25}Ce sont là les familles d'Issacar, d'après leur dénombrement : Soixante-quatre mille trois cents.
\VS{26}Fils de Zabulon, selon leurs familles : De Séred, descend la famille des Sardites ; d'Elon, la famille des Elonites ; de Jahleel, la famille des Jahleélites.
\VS{27}Ce sont là les familles des Zabulonites, d'après leur dénombrement : Soixante mille cinq cents.
\VS{28}Fils de Joseph, selon leurs familles : Manassé et Ephraïm.
\VS{29}Fils de Manassé. De Makir descend la famille des Makirites. Makir engendra Galaad. De Galaad descend la famille des Galaadites.
\VS{30}Voici les fils de Galaad : Jézer, de qui descend la famille des Jézerites ; Hélek, la famille des Hélekites.
\VS{31}Asriel, la famille des Asriélites ; Sichem, la famille des Sichémites ;
\VS{32}Schemida, la famille des Schemidaïtes ; Hépher, la famille des Héphrites.
\VS{33}Tselophchad, fils de Hépher, n'eut point de fils, mais des filles. Voici les noms des filles de Tselophchad : Machla, Noa, Hogla, Milca, et Thirtsa.
\VS{34}Ce sont là les familles de Manassé, d'après leur dénombrement : Cinquante-deux mille sept cents.
\VS{35}Voici les fils d'Ephraïm, selon leurs familles : De Schutélach descend la famille des Schutalchites ; de Béker, la famille des Bakrites ; de Thachan, la famille des Thachanites.
\VS{36}Voici les fils de Schutélach : d'Eran est descendue la famille des Eranites.
\VS{37}Ce sont là les familles des fils d'Ephraïm, d'après leur dénombrement : Trente-deux mille cinq cents. Ce sont là les fils de Joseph, selon leurs familles.
\VS{38}Fils de Benjamin, selon leurs familles : De Béla descend la famille des Balites ; d'Aschbel, la famille des Aschbélites ; d'Achiram, la famille des Achiramites ;
\VS{39}De Schupham, la famille des Schuphamites ; de Hupham, la famille des Huphamites.
\VS{40}Les fils de Béla furent Ard et Naaman. D'Ard descend la famille des Ardites ; et de Naaman la famille des Naamanites.
\VS{41}Ce sont là les fils de Benjamin, d'après leurs familles ; et leur dénombrement : Quarante-cinq mille six cents.
\VS{42}Voici les fils de Dan, selon leurs familles : De Schucham descend la famille des Schuchamites. Ce sont là les familles de Dan, selon leurs familles.
\VS{43}Toutes les familles des Schuchamites, selon leur dénombrement : Soixante-quatre mille quatre cents.
\VS{44}Fils d'Aser, selon leurs familles : De Jimna descend la famille des Jimnites ; de Jischvi, la famille des Jischvites ; de Beria la famille des Beriites.
\VS{45}Des fils de Beria descendent : De Héber, la famille des Hébrites ; de Malkiel, la famille des Malkiélites.
\VS{46}Et le nom de la fille d'Aser était Sérach.
\VS{47}Ce sont là les familles des fils d'Aser, d'après leur dénombrement : Cinquante-trois mille quatre cents.
\VS{48}Fils de Nephthali, selon leurs familles : De Jahtseel descend la famille des Jahtseélites ; de Guni, la famille des Gunites ;
\VS{49}de Jetser la famille des Jitsrites ; de Schillem, la famille des Schillémites.
\VS{50}Ce sont là les familles de Nephthali, selon leurs familles, et leur dénombrement : Quarante-cinq mille quatre cents.
\VS{51}Voici les dénombrés des fils d'Israël, qui furent six cent un mille sept cent trente.
\VS{52}Yahweh parla à Moïse, en disant :
\VS{53}Le pays sera partagé entre ceux-ci en héritage, selon le nombre des noms.
\VS{54}A ceux qui sont en plus grand nombre, tu donneras plus d'héritage, et à ceux qui sont en plus petit nombre tu donneras moins d'héritage; on donnera à chacun son héritage selon le nombre de ses dénombrés.
\VS{55}Toutefois, que le pays soit partagé par le sort ; et qu’ils prennent leur héritage selon les noms des tribus de leurs pères\FTNT{Jos. 11:23 ; Jos. 14:2 ; Jos. 18:6-8.}.
\VS{56}L’héritage de chacun sera selon que le sort le montrera, et on aura égard au plus grand et au plus petit nombre.
\VS{57}Et ce sont ici les dénombrés de Lévi selon leurs familles ; de Guerschon, la famille des Guerschonites ; de Kehath, la famille des Kehathites ; de Merari, la famille des Merarites.
\VS{58}Ce sont ici les familles de Lévi ; la famille des Libnites, la famille des Hébronites, la famille des Machlites, la famille des Muschites, la famille des Korites. Kehath engendra Amram.
\VS{59}Et le nom de la femme d'Amram était Jokébed, fille de Lévi, qui naquit à Lévi en Egypte ; et elle enfanta à Amram: Aaron, Moïse, et Marie, leur soeur.
\VS{60}Et il naquit à Aaron : Nadab et Abihu, Eléazar et Ithamar.
\VS{61}Nadab et Abihu moururent lorsqu'ils apportèrent du feu étranger devant Yahweh\FTNT{Lé. 10:1-2 ; 1 Ch. 24:2.}.
\VS{62}Et tous les dénombrés des Lévites furent vingt-trois mille, tous mâles, depuis l’âge d’un mois, et au dessus, qui ne furent point dénombrés avec les autres enfants d’Israël, car on ne leur donna point d’héritage entre les enfants d’Israël.
\VS{63}ce sont là ceux qui furent dénombrés par Moïse et Eléazar, le sacrificateur, qui firent le dénombrement des fils d'Israël dans les plaines de Moab, près du Jourdain de Jéricho.
\VS{64}Entre lesquels il ne s’en trouva aucun de ceux qui avaient été dénombrés par Moïse et Aaron le sacrificateur, quand ils firent le dénombrement des enfants d’Israël au désert de Sinaï.
\VS{65}Car Yahweh avait dit d'eux : ils mourront certainement dans le désert, et qu'ainsi il n'en restera pas un, excepté Caleb, fils de Jephunné, et Josué, fils de Nun\FTNT{1 Co. 10:5.}.
\Chap{27}
\TextTitle{Loi sur les héritages\FTNTT{No. 36}}
\VerseOne{}Or les filles de Tselophchad, fils de Hépher, fils de Galaad, fils de Makir, fils de Manassé, d'entre les familles de Manassé, fils de Joseph, s'approchèrent ;  et ce sont ici les noms de ses filles : Machla, Noa, Hogla, Milca, et Thirtsa.
\VS{2}Elles se présentèrent devant Moïse, devant Eléazar, le sacrificateur, et devant les princes et toute l'assemblée, à l'entrée de la tente d'assignation. Elles dirent :
\VS{3}Notre père est mort dans le désert ; il n'était toutefois pas dans la troupe de ceux qui s’assemblèrent contre Yahweh, dans l'assemblée de Koré, mais il est mort dans son péché, et il n'avait point de fils.
\VS{4}Pourquoi le nom de notre père serait-il retranché de sa famille, parce qu'il n'a point eu de fils ? Donne-nous une possession parmi les frères de notre père.
\VS{5}Moïse rapporta leur cause devant Yahweh.
\VS{6}Et Yahweh parla à Moïse, en disant :
\VS{7}Les filles de Tselophchad ont parlé droitement. Tu ne manqueras pas de leur donner un héritage à posséder parmi les frères de leur père, et tu leur feras passer l’héritage de leur père.
\VS{8}Tu parleras aussi aux enfants d'Israël, et tu leur diras : Lorsqu'un homme mourra sans avoir de fils, vous ferez passer son héritage à sa fille.
\VS{9}S'il n'a pas de fille, vous donnerez son héritage à ses frères.
\VS{10}S'il n'a pas de frères, vous donnerez son héritage aux frères de son père.
\VS{11}Et si son père n'a pas de frère, vous donnerez son héritage à son parent le plus proche de sa famille, et il le possédera. Et ce sera pour les enfants d'Israël une ordonnance de droit, comme Yahweh l'a ordonné à Moïse.
\TextTitle{Moïse voit de loin le pays promis aux fils d'Israël}
\VS{12}Yahweh dit aussi à Moïse : Monte sur cette montagne d'Abarim, et regarde le pays que je donne aux enfants d'Israël\FTNT{De. 32:48-49.}.
\VS{13}Tu le regarderas donc ; et puis tu seras toi aussi recueilli auprès de ton peuple, comme Aaron ton frère y a été recueilli ;
\VS{14}parce que vous avez été rebelles à mon ordre dans le désert de Tsin, lors de la contestation de l'assemblée, vous ne m'avez point sanctifié au sujet des eaux devant eux ; ce sont les eaux de Meriba, à Kadès, dans le désert de Tsin.
\TextTitle{Yahweh désigne Josué comme successeur de Moïse}
\VS{15}Moïse parla à Yahweh, en disant :
\VS{16}Que Yahweh, le Dieu des esprits de toute chair, établisse sur l'assemblée un homme\FTNT{Hé. 12:9.},
\VS{17}qui sorte devant eux et qui entre devant eux, et qui les fasse sortir et qui les fasse entrer, afin que l'assemblée de Yahweh ne soit pas comme des brebis qui n'ont point de berger\FTNT{1 R. 22:17 ; Mt. 9:36 ; Mc. 6:34.}.
\VS{18}Alors Yahweh dit à Moïse : Prends Josué, fils de Nun, un homme en qui est l'Esprit, et tu poseras ta main sur lui\FTNT{De. 34:9.}.
\VS{19}Tu le présenteras devant Eléazar, le sacrificateur, et devant toute l'assemblée ; et tu lui donneras des instructions sous leurs yeux.
\VS{20}Et tu mettras sur lui de ta gloire, afin que toute l'assemblée des enfants d'Israël l'écoute.
\VS{21}Et il se présentera devant Eléazar, le sacrificateur, qui consultera pour lui les jugements de l'urim\FTNT{Lé. 8:8.} devant Yahweh ; et à sa parole ils sortiront, et à sa parole ils entreront, lui, les enfants d’Israël, avec lui, et toute l’assemblée.
\VS{22}Moïse donc fit comme Yahweh lui avait ordonné. Il prit Josué et le présenta devant Eléazar, le sacrificateur, et devant toute l'assemblée.
\VS{23}Puis il posa ses mains sur lui, et lui donna des instructions, comme Yahweh l'avait dit par Moïse.
\Chap{28}
\TextTitle{Consignes relatives au temps des sacrifices}
\VerseOne{}Yahweh parla à Moïse, en disant :
\VS{2}Donne cet ordre aux enfants d'Israël, et dis-leur : Vous aurez soin de m'offrir en leur temps, mon offrande, ma nourriture, pour mes sacrifices consumés par le feu, qui me sont d'une bonne odeur\FTNT{Lé. 3:11 ; Lé. 21:6.}.
\VS{3}Tu leur diras : Voici le sacrifice consumé par le feu que vous offrirez à Yahweh : Deux agneaux d'un an sans défaut, chaque jour, en holocauste perpétuel\FTNT{Ex. 29:38.}.
\VS{4}Tu sacrifieras l'un des agneaux le matin, et l'autre agneau entre les deux soirs,
\VS{5}et la dixième partie d'épha de fine farine pour le gâteau pétrie avec le quart d'un hin d'huile vierge\FTNT{Lé. 2:1 ; Ex. 29:40 ; Ex. 16:36.}.
\VS{6}C'est l'holocauste perpétuel, qui a été offert à la montagne de Sinaï, c'est un sacrifice consumé par le feu, d'une bonne odeur à Yahweh.
\VS{7}Et sa libation sera d'un quart de hin pour chaque agneau : Et tu verseras dans le lieu saint la libation de boisson forte à Yahweh.
\VS{8}Et tu sacrifieras l’autre agneau entre les deux soirs, tu feras le même gâteau qu'au matin, et la même libation, en sacrifice consumé par le feu d'une bonne odeur à Yahweh.
\VS{9}Mais le jour du sabbat vous offrirez deux agneaux d'un an sans défaut, et deux dixièmes de fine farine pétrie à l'huile pour le gâteau, avec sa libation.
\VS{10}C'est l'holocauste du sabbat, pour chaque sabbat, outre l'holocauste perpétuel avec sa libation.
\VS{11}Et au commencement de vos mois, vous offrirez en holocauste à Yahweh deux jeunes taureaux, un bélier, et sept agneaux d'un an sans défaut ;
\VS{12}et trois dixièmes de fine farine pétrie à l'huile, pour le gâteau de chaque taureau, et deux dixièmes de fine farine pétrie à l'huile pour le gâteau du bélier.
\VS{13}Et un dixième de fine farine pétrie à l'huile, comme gâteau pour chaque agneau, en holocauste, d'une bonne odeur, et en sacrifice consumé par le feu à Yahweh.
\VS{14}Et leurs libations seront d'un demi-hin de vin pour chaque veau, d'un tiers de hin pour un bélier, et d'un quart de hin pour chaque agneau, c'est l'holocauste du commencement de chaque mois, selon tous les mois de l'année.
\VS{15}On sacrifiera aussi à Yahweh un jeune bouc en sacrifice d'expiation, outre l'holocauste perpétuel, et sa libation.
\VS{16}Au quatorzième jour du premier mois, ce sera la Pâque à Yahweh.
\VS{17}Et au quinzième jour du même mois sera un jour de fête. On mangera pendant sept jours des pains sans levain\FTNT{Ex. 12 ; Lé. 23:5-6.}.
\VS{18}Au premier jour, il y aura une sainte convocation : Vous ne ferez aucune œuvre servile.
\VS{19}Et vous offrirez un sacrifice consumé par le feu en holocauste à Yahweh : Deux jeunes taureaux, un bélier, et sept agneaux d'un an, sans défaut.
\VS{20}Leur gâteau sera de fine farine pétrie à l’huile, vous en offrirez trois dixièmes pour chaque jeune taureau, et deux dixièmes pour un bélier ;
\VS{21}tu en offriras aussi un dixième pour chacun des sept agneaux,
\VS{22}et un bouc en sacrifice pour l'expiation, afin de faire propitiation pour vous.
\VS{23}Vous offrirez ces choses là, outre l'holocauste du matin, qui est l'holocauste perpétuel.
\VS{24}Vous les offrirez ces choses-là chaque jour, pendant sept jours, comme l'aliment d'un sacrifice consumé par le feu, d'une bonne odeur à Yahweh. On offrira cela outre l'holocauste perpétuel, et sa libation.
\VS{25}Et au septième jour, vous aurez une sainte convocation : Vous ne ferez aucune œuvre servile.
\VS{26}Et au jour des prémices, quand vous offrirez à Yahweh une offrande nouvelle de gâteau à votre fête des semaines, vous aurez une sainte convocation : Vous ne ferez aucune œuvre servile.
\VS{27}Et vous offrirez en holocauste d'une bonne odeur à Yahweh, deux jeunes taureaux, un bélier, et sept agneaux d'un an.
\VS{28}Et leur gâteau sera de fine farine pétrie à l’huile, de trois dixièmes pour chaque jeune taureau, et de deux dixièmes pour le bélier,
\VS{29}et d'un dixième pour chacun des sept agneaux.
\VS{30}Et un jeune bouc, afin de faire propitiation pour vous .
\VS{31}Vous les offrirez, outre l'holocauste perpétuel et son offrande. Ils seront sans défaut, et leurs libations.
\Chap{29}
\TextTitle{Consignes relatives au temps des sacrifices - suite}
\VerseOne{}Et le premier jour du septième mois, vous aurez une sainte convocation : Vous ne ferez aucune œuvre servile. Ce jour sera publié parmi vous au son des trompettes\FTNT{Lé. 23:24-25.}.
\VS{2}Et vous offrirez en holocauste de bonne odeur à Yahweh, un jeune taureau, un bélier, et sept agneaux d'un an, sans défaut.
\VS{3}Et leur gâteau sera de fine farine pétrie à l’huile, de trois dixièmes pour le jeune taureau, de deux dixièmes pour le bélier,
\VS{4}et un dixième pour chacun des sept agneaux.
\VS{5}Et un jeune bouc en sacrifice pour l'expiation, afin de faire propitiation pour vous,
\VS{6}outre l’holocauste du commencement du mois et son gâteau, et l’holocauste continuel et son gâteau, et leurs libations selon leur ordonnance. Ce sont des sacrifices consumés par le feu en bonne odeur à Yahweh.
\VS{7}Et au dixième jour de ce septième mois, vous aurez une sainte convocation, et vous affligerez vos âmes : Vous ne ferez aucune œuvre\FTNT{Lé. 16:29-31 ; Lé. 23:27.}.
\VS{8}Et vous offrirez en holocauste, de bonne odeur à Yahweh, un jeune taureau, un bélier, et sept agneaux d'un an, qui seront sans défaut.
\VS{9}Et leur gâteau sera de fine farine pétrie à l'huile, de trois dixièmes pour le taureau, et de deux dixièmes pour le bélier,
\VS{10}et d'un dixième pour chacun des sept agneaux.
\VS{11}Un jeune bouc aussi en sacrifice d'expiation, outre le sacrifice des expiations, l'holocauste perpétuel et son gâteau, avec leurs libations.
\VS{12}Et au quinzième jour du septième mois, vous aurez une sainte convocation : Vous ne ferez aucune œuvre servile. Vous célébrerez une fête à Yahweh, pendant sept jours\FTNT{Lé. 23:34-43.}.
\VS{13}Et vous offrirez en holocauste un sacrifice consumé par le feu, d'une agréable odeur à Yahweh, treize jeunes taureaux, deux béliers, et quatorze agneaux d'un an, sans défaut.
\VS{14}Et leur gâteau sera de fine farine pétrie à l’huile, de trois dixièmes pour chacun des treize jeunes taureaux, de deux dixièmes pour chacun des deux béliers,
\VS{15}et d'un dixième pour chacun des quatorze agneaux.
\VS{16}Et un jeune bouc en sacrifice d'expiation, outre l'holocauste perpétuel, son gâteau, et sa libation.
\VS{17}Et au second jour, vous offrirez douze jeunes taureaux, deux béliers, et quatorze agneaux d'un an, sans défaut,
\VS{18}avec les gâteaux et les libations pour les jeunes taureaux, pour les béliers, et pour les agneaux, selon leur nombre, d'après les ordonnances.
\VS{19}Vous offrirez un jeune bouc en sacrifice d'expiation, outre l'holocauste perpétuel, et son offrande, avec leurs libations.
\VS{20}Et au troisième jour, vous offrirez onze taureaux, deux béliers, et quatorze agneaux d'un an, sans défaut ;
\VS{21}et les gâteaux et les libations pour les jeunes taureaux, les béliers et les agneaux, selon leur nombre, selon leur ordonnance.
\VS{22}Et un bouc en sacrifice d'expiation, outre l'holocauste continuel, son gâteau et sa libation.
\VS{23}Et au quatrième jour, vous offrirez dix jeunes taureaux, deux béliers, et quatorze agneaux d'un an, sans défaut,
\VS{24}les gâteaux et les libations pour les taureaux, les béliers, et les agneaux, selon leur nombre et leur ordonnance.
\VS{25}Et un jeune bouc en sacrifice d'expiation, outre l'holocauste perpétuel, son offrande, et sa libation.
\VS{26}Et au cinquième jour, vous offrirez neuf jeunes taureaux, deux béliers, et quatorze agneaux d'un an, sans défaut,
\VS{27}avec les gâteaux et les libations pour les taureaux, les béliers, et les agneaux, selon leur nombre et leur ordonnance.
\VS{28}Et un bouc en sacrifice d'expiation, outre l'holocauste continuel, son gâteau, et sa libation.
\VS{29}Et le sixième jour, vous offrirez huit jeunes taureaux, deux béliers et quatorze agneaux d'un an, sans défaut,
\VS{30}et les gâteaux, les libations pour les taureaux, les béliers, et les agneaux selon leur nombre leur ordonnance.
\VS{31}Et un bouc en sacrifice d'expiation, outre l'holocauste continuel, son offrande, et sa libation.
\VS{32}Et au septième jour, vous offrirez sept jeunes taureaux, deux béliers, et quatorze agneaux d'un an, sans défaut,
\VS{33}avec les gâteaux et les libations pour les jeunes taureaux, les béliers, et les agneaux, selon leur nombre et leur ordonnance.
\VS{34}Et un bouc en sacrifice d'expiation, outre l'holocauste continuel, son gâteau, et sa libation.
\VS{35}Et au huitième jour, vous aurez une assemblée solennelle : Vous ne ferez aucune œuvre servile.
\VS{36}Et vous offrirez en holocauste un sacrifice consumé par le feu, d'une agréable odeur à Yahweh : Un jeune taureau, un bélier, et sept agneaux d'un an, sans défaut,
\VS{37}avec les gâteaux et les libations pour le jeune taureau, le bélier, et les agneaux, selon leur nombre et leur ordonnance.
\VS{38}Et un bouc en sacrifice d'expiation, outre l'holocauste perpétuel, son offrande, et sa libation.
\VS{39}Vous offrirez ces choses à Yahweh dans vos fêtes solennelles, outre vos vœux, et vos offrandes volontaires, selon vos holocaustes, vos gâteaux, vos libations, et vos sacrifices d'offrande de paix.
\Chap{30}
\TextTitle{Les voeux}
\VerseOne{}Et Moïse parla aux enfants d'Israël selon toutes les choses que Yahweh lui avait ordonné.
\VS{2}Moïse parla aussi aux chefs des tribus des enfants d'Israël, en disant : Voici ce que Yahweh ordonne.
\VS{3}Quand un homme fera un vœu à Yahweh, ou aura juré par serment, pour lier son âme par un vœu, il ne violera pas sa parole ; il fera selon toutes les choses qui sont sorties de sa bouche\FTNT{De. 23:21.}.
\VS{4}Mais quand une femme fera un vœu à Yahweh, et qu'elle se liera par un serment, dans sa jeunesse, étant encore dans la maison de son père,
\VS{5}et que son père aura entendu son vœu et le serment par lequel elle a lié son âme, si son père ne lui dit rien, tous ses vœux seront valables, et tout serment par lequel elle aura lié son âme sera valable ;
\VS{6}mais si son père la désapprouve le jour où il l'a entendue, aucun de ses vœux ou de ses serments par lesquels elle a lié son âme ne sera valable, et Yahweh lui pardonnera ; parce que son père l'a désapprouvée.
\VS{7}Et si elle a un mari, et qu'elle s'est engagée par quelque vœu ou par une parole échappée de ses lèvres par laquelle elle aura lié son âme,
\VS{8}et que son mari l'aura entendue, et que le jour même où il l'a entendue, il ne lui a rien dit, ses vœux alors seront valables, et ses serments par lesquels elle aura lié son âme seront valables ;
\VS{9}Mais si son mari la désapprouve le jour où il l'a entendue, alors il annulera le vœu par lequel elle s'est engagée et la parole échappée de ses lèvres, par laquelle elle avait lié son âme ; et Yahweh lui pardonnera.
\VS{10}Mais le vœu de la veuve ou de la répudiée, tout ce par quoi elle aura lié son âme, sera valable pour elle.
\VS{11}Que si étant encore dans la maison de son mari elle a fait un vœu, ou si elle a lié son âme par serment,
\VS{12}et que son mari l'ait entendue, et ne lui en ait rien dit, et ne l'ait pas désapprouvée, alors tous ses vœux seront valables, et tout serment par lequel elle a lié son âme sera valable.
\VS{13}Mais si son mari les a entièrement annulés le jour où il les a entendus, alors rien de ce qui est sorti de ses lèvres, soit ses vœux, soit le serment par lequel elle a lié son âme ne seront valables ; parce que son mari les a annulés, et Yahweh lui pardonnera.
\VS{14}Son mari peut ratifier et son mari peut annuler tout vœu et toute obligation faite par serment, pour affliger l'âme.
\VS{15}Mais si son mari ne lui en a absolument rien dit, d’un jour à l’autre, il aura ratifié tous ses vœux ou toutes ses obligations dont elle était tenue ; il les aura, dis-je, ratifiés, parce qu'il ne lui en a rien dit le jour où il les a entendus.
\VS{16}Mais s'il les a expressément annulés après les avoir entendus, alors il portera l'iniquité de sa femme.
\VS{17}Telles sont les ordonnances que Yahweh ordonna à Moïse, entre un mari et sa femme; entre un père et sa fille, étant encore dans la maison de son père, dans sa jeunesse.
\Chap{31}
\TextTitle{Jugements sur Madian\FTNTT{No. 25:6-18}}
\VerseOne{}Yahweh parla à Moïse, en disant :
\VS{2}Fais la vengeance des enfants d'Israël sur les Madianites, puis tu seras recueilli auprès de ton peuple.
\VS{3}Moïse donc parla au peuple, en disant : Que quelques-uns d'entre vous s'équipent pour aller à la guerre, et qu'ils aillent contre Madian, pour exécuter la vengeance de Yahweh sur Madian.
\VS{4}Vous enverrez à la guerre mille hommes de chaque tribu, de toutes les tribus d'Israël.
\VS{5}On donna d'entre les milliers d'Israël mille hommes de chaque tribu, qui furent douze mille hommes équipés pour la guerre.
\VS{6}Moïse les envoya à la guerre, savoir mille de chaque tribu, et avec eux Phinées, fils d'Eléazar, le sacrificateur, qui portait les instruments sacrés et les trompettes retentissantes.
\VS{7}Ils s'avancèrent donc contre Madian, comme Yahweh l'avait donné à Moïse, et ils en tuèrent tous les mâles.
\VS{8}Ils tuèrent aussi les rois de Madian, outre les autres qui y furent tués, Evi, Rékem, Tsur, Hur, et Réba, cinq rois de Madian ; ils firent aussi passer au fil de l'épée Balaam, fils de Beor\FTNT{Jos. 13:21-22.}.
\VS{9}Et les fils d'Israël emmenèrent prisonniers les femmes de Madian, avec leurs petits enfants, et pillèrent tout leur gros et menu bétail, et tous leurs biens.
\VS{10}Ils brûlèrent par le feu toutes leurs villes, leurs demeures, et tous leurs châteaux.
\VS{11}Ils prirent tout le butin et tout le pillage, tant des hommes que du bétail\FTNT{De. 20:14.} ;
\VS{12}puis ils amenèrent les captifs, le pillage, et le butin, à Moïse, à Eléazar le sacrificateur, et à l'assemblée des enfants d'Israël, au camp, dans les plaines de Moab, qui sont près du Jourdain, vis-à-vis de Jéricho.
\VS{13}Moïse, Eléazar, le sacrificateur, et tous les princes de l'assemblée sortirent au-devant d'eux, hors du camp.
\VS{14}Et Moïse se mit en grande colère contre les officiers de l'armée, les chefs des milliers, et les chefs des centaines, qui revenaient de cet exploit de guerre.
\VS{15}Et Moïse leur dit : N'avez-vous pas gardé en vie toutes les femmes ?
\VS{16}Voici ce sont elles qui, à la parole de Balaam, ont donné l'occasion aux fils d’Israël de pécher contre Yahweh dans l'affaire de Peor ; ce qui attira la plaie sur l'assemblée de Yahweh\FTNT{2 Pi. 2:15 ; Ap. 2:14.}.
\VS{17}Or maintenant, tuez tous les mâles d'entre les petits enfants, et tuez toute femme qui a connu un homme en couchant avec lui\FTNT{Jg. 21:11.} ;
\VS{18}mais vous garderez en vie toutes les jeunes filles qui n'ont point connu la couche d'un homme.
\VS{19}Au reste, demeurez sept jours hors du camp ; quiconque aura tué quelqu'un, et quiconque aura touché quelqu'un qui aura été tué, se purifiera le troisième et le septième jour, tant vous que vos prisonniers.
\VS{20}Vous purifierez aussi tous vos vêtements, et tout ce qui sera fait de peau, et tout ouvrage de poil de chèvre, et toute vaisselle de bois.
\VS{21}Eléazar, le sacrificateur, dit aux hommes de guerre qui étaient allés au combat : Voici l'ordonnance et la loi que Yahweh a ordonné à Moïse.
\VS{22}En général l'or, l'argent, l'airain, le fer, l'étain, le plomb ;
\VS{23}tout ce qui peut passer par le feu, vous le ferez passer par le feu pour le rendre pur. Seulement on purifiera avec l'eau de purification toutes les choses qui ne peuvent aller au feu, vous les ferez passer dans l'eau.
\VS{24}Vous laverez aussi vos vêtements le septième jour, ensuite vous serez purs ; puis vous entrerez au camp.
\TextTitle{Partage du butin}
\VS{25}Et Yahweh parla à Moïse, en disant :
\VS{26}Fais le compte du butin et de tout ce qu'on a emmené, tant des personnes que des bêtes, toi et Eléazar, le sacrificateur, et les chefs des pères de l'assemblée.
\VS{27}Et partage par moitié le butin entre les combattants qui sont allés à la guerre et toute l'assemblée\FTNT{1 S. 30:24.}.
\VS{28}Tu prélèveras aussi pour Yahweh un tribut sur les hommes de guerre qui sont allés à la bataille, savoir un sur cinq cents, tant des personnes, que des bœufs, des ânes et des brebis.
\VS{29}On le prendra sur leur moitié, et tu le donneras à Eléazar, le sacrificateur, en offrande présentée par élévation à Yahweh.
\VS{30}Et sur la moitié qui appartient aux enfants d'Israël, tu prendras un sur cinquante, tant des personnes que des bœufs, des ânes, des brebis et de tous autres animaux, et tu le donneras aux Lévites qui ont la charge de garder le tabernacle de Yahweh.
\VS{31}Moïse et Eléazar, le sacrificateur, firent comme Yahweh l'avait ordonné à Moïse.
\VS{32}Or le butin qui était resté du pillage du peuple qui était allé à la guerre, était de six cent soixante-quinze mille brebis ;
\VS{33}de soixante-douze mille bœufs ;
\VS{34}de soixante et un mille ânes,
\VS{35}quant aux femmes qui n'avaient point connu la couche d' un homme, elles étaient en tout trente-deux mille âmes.
\VS{36}Et la moitié du butin, à savoir la part de ceux qui étaient allés à la guerre, montait à trois cent trente-sept mille cinq cents brebis ;
\VS{37}dont le tribut pour Yahweh, quant aux brebis, fut de six cent soixante-quinze.
\VS{38}Trente-six mille bœufs ; dont le tribut pour Yahweh, quant aux boeufs, fut de soixante-douze bœufs,
\VS{39}trente mille cinq cents ânes ; dont le tribut pour Yahweh, quant aux ânes, fut de soixante et un ânes ;
\VS{40}et à seize mille personnes, dont le tribut pour Yahweh fut de trente-deux personnes.
\VS{41}Et Moïse donna à Eléazar, le sacrificateur, le tribut de l'offrande présentée par élévation à Yahweh, comme Yahweh le lui avait ordonné.
\VS{42}Et de l'autre moitié qui appartenait aux enfants d'Israël, que Moïse avait tiré des hommes qui étaient allés à la guerre,
\VS{43}Or de cette moitié formant la part de l'assemblée, fut de trois cent trente-sept mille cinq cents brebis,
\VS{44}trente-six mille bœufs,
\VS{45}trente mille cinq cents ânes,
\VS{46}et à seize mille personnes.
\VS{47}De cette moitié, dis-je, qui appartenait aux enfants d'Israël, Moïse prit un sur cinquante, tant des personnes que des bêtes, et les donna aux Lévites qui avaient la charge de garder le tabernacle de Yahweh, comme Yahweh le lui avait ordonné.
\VS{48}Les commandants des milliers de l'armée, tant les chefs des milliers que les chefs des centaines, s'approchèrent de Moïse,
\VS{49}et lui dirent : Tes serviteurs ont fait le compte des hommes de guerre qui étaient sous nos ordres, il ne manque pas un homme d'entre nous.
\VS{50}C'est pourquoi, nous offrons l'offrande de Yahweh, chacun les objets que nous avons trouvés : Des joyaux d'or, des chaînes de cheville, des bracelets, des anneaux, des pendants d'oreilles et des colliers, afin de faire propitiation pour nos personnes devant Yahweh.
\VS{51}Moïse et Eléazar, le sacrificateur, reçurent d'eux cet or, tous ces objets travaillés.
\VS{52}Et tout l'or de l'offrande présentée par élévation à Yahweh, de la part des chefs de milliers et des chefs de centaines, montait à seize mille sept cent cinquante sicles.
\VS{53}Or les hommes de guerre gardèrent chacun pour soi ce qu'ils avaient pillé.
\VS{54}Moïse donc et Eléazar, le sacrificateur, prirent l'or des chefs des milliers et des chefs de centaines, et l'apportèrent à la tente d'assignation, comme souvenir pour les enfants d'Israël, devant Yahweh.
\Chap{32}
\TextTitle{Ruben et Gad en Galaad}
\VerseOne{}Les fils de Ruben et les fils de Gad avaient beaucoup de bétail, en très grande quantité, et ils virent que le pays de Jaezer et le pays de Galaad étaient un lieu propre pour du bétail.
\VS{2}Ainsi les fils de Gad et les fils de Ruben vinrent, et parlèrent à Moïse et à Eléazar, le sacrificateur, et aux princes de l'assemblée, en disant :
\VS{3}Atharoth, Dibon, Jaezer, Nimra, Hesbon, et Elealé, Sebam, Nebo, et Beon,
\VS{4}ce pays-là que Yahweh a frappé devant l'assemblée d'Israël, est un pays propre pour le bétail, et tes serviteurs ont des troupeaux.
\VS{5}Ils dirent donc: Si nous avons trouvé grâce à tes yeux, que ce pays soit donné en possession à tes serviteurs ; et ne nous fais point passer le Jourdain.
\VS{6}Mais Moïse répondit aux fils de Gad, et aux fils de Ruben : Vos frères iront-ils à la guerre, et vous, demeurerez-vous ici ?
\VS{7}Pourquoi voulez-vous décourager les enfants d'Israël de passer dans le pays que Yahweh leur a donné ?
\VS{8}C'est ainsi que firent vos pères quand je les envoyai de Kadès-Barnéa pour examiner le pays.
\VS{9}Car ils montèrent jusqu'à la vallée d'Eschcol, virent le pays, puis découragèrent les enfants d'Israël, afin qu'ils n'entrent point dans le pays que Yahweh leur avait donné.
\VS{10}C'est pourquoi la colère de Yahweh s'enflamma ce jour-là, et il jura en disant :
\VS{11}Les hommes qui sont montés hors d'Egypte, depuis l'âge de vingt ans et au-dessus, ne verront point le pays que j'ai juré de donner à Abraham, Isaac, et à Jacob ; car ils n’ont point persévéré à me suivre.\FTNT{De. 1:35},
\VS{12}excepté Caleb, fils de Jephunné, le Kénizien, et Josué, fils de Nun,  car ils ont persévéré à suivre Yahweh.
\VS{13}Ainsi la colère de Yahweh s'enflamma contre Israël et il les fit errer dans le désert pendant quarante ans, jusqu'à ce que toute la génération qui avait fait le mal aux yeux de Yahweh, ait été consumée.
\VS{14}Et voici, vous vous êtes levés à la place de vos pères, comme une race d'hommes pécheurs, pour augmenter encore l'ardeur de la colère de Yahweh contre Israël.
\VS{15}Si vous vous détournez de lui, il continuera encore à vous laisser au désert, et vous ferez détruire tout ce peuple.
\VS{16}Mais ils s'approchèrent de lui et lui dirent : Nous bâtirons ici des cloisons pour nos troupeaux, et des villes pour nos petits enfants ;
\VS{17}et nous nous équiperons pour marcher promptement devant les enfants d'Israël, jusqu'à ce que nous les ayons introduits en leur lieu; mais nos petits enfants demeureront dans les villes fortes, à cause des habitants du pays.
\VS{18}Nous ne retournerons point dans nos maisons avant que chacun des enfants d'Israël n'ait pris possession de son héritage ;
\VS{19}et nous ne posséderons rien en héritage avec eux au-delà du Jourdain, ni plus avant ; parce que nous aurons notre héritage de ce côté-ci du Jourdain, à l'orient.
\VS{20}Et Moïse leur dit : Si vous faites cela, si vous vous équipez devant Yahweh pour aller à la guerre,
\VS{21}si chacun de vous étant équipé passe le Jourdain devant Yahweh, jusqu'à ce qu'il ait chassé ses ennemis loin de devant lui,
\VS{22}et que le pays soit assujetti devant Yahweh, et qu’ensuite vous vous en retournez, alors vous serez innocents envers Yahweh, et envers Israël ; et ce pays-ci vous appartiendra pour le posséder devant Yahweh.
\VS{23}Mais si vous ne faites point cela, vous péchez contre Yahweh ; et sachez que votre péché vous atteindra.
\VS{24}Bâtissez donc des villes pour vos petits enfants, et des cloisons pour vos troupeaux, et faites ce que vous avez dit.
\VS{25}Alors les fils de Gad et les fils de Ruben parlèrent à Moïse, en disant : Tes serviteurs feront ce que mon seigneur a ordonné.
\VS{26}Nos petits enfants, nos femmes, nos troupeaux, et tout notre bétail demeureront ici dans les villes de Galaad ;
\VS{27}et tes serviteurs passeront chacun armés pour aller à la guerre devant Yahweh, prêts à combattre, comme mon seigneur a parlé.
\VS{28}Alors Moïse donna des ordres à leur sujet à Eléazar, le sacrificateur, à Josué, fils de Nun, et aux chefs des pères des tribus des fils d'Israël.
\VS{29}Il leur dit : Si les fils de Gad et les fils de Ruben passent avec vous le Jourdain tous armés, prêts à combattre devant Yahweh, et que le pays vous soit assujetti, vous leur donnerez le pays de Galaad en possession.
\VS{30}Mais s'ils ne marchent point en armes avec vous, qu'ils s'établissent au milieu de vous dans le pays de Canaan.
\VS{31}Les fils de Gad et les fils de Ruben répondirent, en disant : Nous ferons ce que Yahweh a dit à tes serviteurs.
\VS{32}Nous passerons en armes devant Yahweh au pays de Canaan, afin que nous possédions pour notre héritage ce qui est de ce côté-ci du Jourdain.
\VS{33}Ainsi Moïse donna aux fils de Gad et aux fils de Ruben, et à la demi-tribu de Manassé, fils de Joseph, le royaume de Sihon, roi des Amoréens ; et le royaume de Og, roi de Basan, le pays avec ses villes, selon les bornes des villes du pays tout autour.
\VS{34}Alors les fils de Gad rebâtirent Dibon, Atharoth, Aroër,
\VS{35}Athroth-Schophan, Jaezer, Jogbeha,
\VS{36}Beth-Nimra et Beth-Haran, villes fortifiées. Ils firent aussi des cloisons pour les troupeaux.
\VS{37}Et les fils de Ruben rebâtirent Hesbon, Elealé, Kirjathaim,
\VS{38}Nébo, Baal-Meon, et Sibma, dont ils changèrent les noms, et ils donnèrent des noms aux villes qu'ils rebâtirent.
\VS{39}Or les fils de Makir, fils de Manassé, allèrent en Galaad, le prirent et dépossédèrent les Amoréens qui y étaient.
\VS{40}Moïse donc  donna Galaad à Makir, fils de Manassé, qui y habita.
\FTNT{De. 3:15.}.
\VS{41}Jaïr, fils de Manassé, se mit en marche, prit leurs bourgs, et les appela bourgs de Jaïr\FTNT{De. 3:14 ; 1 Ch. 2:22.}.
\VS{42}Et Nobach se mit en marche, prit Kenath avec les villes de son ressort, et l'appela Nobach d'après son nom.
\Chap{33}
\TextTitle{Les stations de l'Egypte jusqu'au Jourdain}
\VerseOne{}Ce sont ici les étapes des enfants d'Israël, qui sortirent du pays d'Egypte, selon leurs armées, sous la main de Moïse et d'Aaron.
\VS{2}Moïse écrivit leurs départs, et leurs étapes, d'après l'ordre de Yahweh ! Et voici leurs étapes selon leurs départs.
\VS{3}Les enfants d'Israël donc partirent de Ramsès le quinzième jour du premier mois, dès le lendemain de la Pâque, et ils sortirent à main levée, à la vue de tous les Egyptiens\FTNT{Ex. 14:8.}.
\VS{4}Et les Egyptiens ensevelissaient ceux que Yahweh avait frappés parmi eux, à savoir tous les premiers-nés ; même Yahweh exerçait aussi ses jugements contre leurs dieux\FTNT{Ex. 12:12 ; Ex. 18:11.}.
\VS{5}Et les enfants d'Israël partirent de Ramsès, et campèrent à Succoth\FTNT{Ex. 12:37.}.
\VS{6}Et ils partirent de Succoth et campèrent à Etham, qui est au bout du désert\FTNT{Ex. 13:20.}.
\VS{7}Et ils partirent d'Etham et se détournèrent vers Pi-Hahiroth, qui est vis-à-vis de Baal-Tsephon, et campèrent devant Migdol\FTNT{Ex. 14:2.}.
\VS{8}Et ils partirent de devant Pi-Hahiroth et passèrent au travers de la mer vers le désert, et firent trois journées de marche par le désert d'Etham et campèrent à Mara.
\VS{9}Puis ils partirent de Mara et vinrent à Elim où il y avait douze fontaines d'eaux et soixante-dix palmiers, et ils y campèrent\FTNT{Ex. 15:27.}.
\VS{10}Et ils partirent d'Elim et campèrent près de la Mer Rouge.
\VS{11}Puis ils partirent de la Mer Rouge et campèrent au désert de Sin\FTNT{Ex. 16:1.}.
\VS{12}Ils partirent du désert de Sin et campèrent à Dophka.
\VS{13}Puis ils partirent de Dophka et campèrent à Alusch.
\VS{14}Et ils partirent d'Alusch et campèrent à Rephidim où il n'y avait point d'eau à boire pour le peuple\FTNT{Ex. 17:1.}.
\VS{15}Puis ils partirent de Rephidim et campèrent dans le désert de Sinaï\FTNT{Ex. 17:1}.
\VS{16}Ils partirent du désert de Sinaï et campèrent à Kibroth-Hattaava.
\VS{17}Et ils partirent de Kibroth-Hattaava et campèrent à Hatséroth.
\VS{18}Puis ils partirent de Hatséroth et campèrent à Rithma.
\VS{19}Et ils partirent de Rithma et campèrent à Rimmon-Pérets.
\VS{20}Ils partirent de Rimmon-Pérets et campèrent à Libna.
\VS{21}Et ils partirent de Libna et campèrent à Rissa.
\VS{22}Puis ils partirent de Rissa et campèrent vers Kehélatha.
\VS{23}Et ils partirent de Kehélatha et campèrent à la montagne de Schapher.
\VS{24}Ils partirent de la montagne de Schapher et campèrent à Harada.
\VS{25}Et ils partirent de Harada et campèrent à Makhéloth.
\VS{26}Puis ils partirent de Makhéloth et campèrent à Tahath.
\VS{27}Ils partirent de Tahath et campèrent à Tarach.
\VS{28}Et ils partirent de Tarach et campèrent à Mithka.
\VS{29}Puis ils partirent de Mithka et campèrent à Haschmona.
\VS{30}Ils partirent de Haschmona et campèrent à Moséroth.
\VS{31}Et ils partirent de Moséroth et campèrent à Bené-Jaakan.
\VS{32}Ils partirent de Bené-Jaakan et campèrent à Hor-Guidgad.
\VS{33}Puis ils partirent de Hor-Guidgad et campèrent vers Jothbatha.
\VS{34}Ils partirent de Jothbatha et campèrent à Abrona.
\VS{35}Et ils partirent d'Abrona et campèrent à Etsjon-Guéber.
\VS{36}Ils partirent d'Etsjon-Guéber et campèrent dans le désert de Tsin, qui est Kadès.
\VS{37}Puis ils partirent de Kadès et campèrent à la montagne de Hor, qui est au bout du pays d'Edom.
\VS{38}Et Aaron le sacrificateur, monta sur la montagne de Hor, suivant l'ordre de Yahweh, et mourut là, la quarantième année après que les enfants d'Israël furent sortis du pays d'Egypte, le premier jour du cinquième mois.
\VS{39}Et Aaron était âgé de cent vingt-trois ans quand il mourut sur la montagne de Hor.
\VS{40}Alors le Cananéen, roi d'Arad, qui habitait vers le midi au pays de Canaan, apprit que les enfants d'Israël venaient.
\VS{41}Et ils partirent de la montagne de Hor et campèrent à Tsalmona.
\VS{42}Puis ils partirent de Tsalmona et campèrent à Punon.
\VS{43}Et Ils partirent de Punon et campèrent à Oboth.
\VS{44}Ils partirent d'Oboth et campèrent à Ijjé-Abarim, sur les frontières de Moab.
\VS{45}Puis ils partirent d'Ijjé-Abarim et campèrent à Dibon-Gad.
\VS{46}Et ils partirent de Dibon-Gad, et campèrent à Almon-Diblathaïm.
\VS{47}Ils partirent d'Almon-Diblathaïm et campèrent aux montagnes de Abarim devant Nébo.
\VS{48}Et ils partirent des montagnes d'Abarim et campèrent aux plaines de Moab, près du Jourdain de Jéricho.
\VS{49}Pui ils campèrent près du Jourdain, depuis Beth-Jeschimoth jusqu'à Abel-Sittim, dans les plaines de Moab.
\TextTitle{Consignes pour les possessions attribuées à Israël}
\VS{50}Et Yahweh parla à Moïse dans les plaines de Moab, près du Jourdain de Jéricho, en disant :
\VS{51}Parle aux enfants d'Israël, et dis-leur : Puisque vous allez passer le Jourdain pour entrer au pays de Canaan,
\VS{52}Vous chasserez de devant vous tous les habitants du pays, vous détruirez toutes leurs peintures, et vous ruinerez toutes leurs images de fonte, et vous démolirez tous leurs hauts lieux\FTNT{De. 7:5 ; De. 12:2.}.
\VS{53}Et vous prendrez possession du pays, et vous y habiterez. Car je vous ai donné le pays pour le posséder.
\VS{54}Or vous recevrez le pays en héritage par le sort, selon vos familles. A ceux qui sont en plus grand nombre, vous donnerez plus d’héritage, et à ceux qui sont en plus petit nombre, vous donnerez moins d’héritage. Chacun aura selon ce qui lui sera échu par le sort, et vous hériterez selon les tribus de vos pères.
\VS{55}Mais si vous ne chassez pas de devant vous les habitants du pays, il arrivera que ceux d'entre eux que vous aurez laissés comme reste, seront comme des épines à vos yeux, et comme des pointes à vos côtés, et ils vous serreront de près dans le pays auquel vous habiterez\FTNT{Jos. 23:13.}.
\VS{56}Et il arrivera que je vous ferai tout comme j’ai eu dessein de leur faire.
\Chap{34}
\TextTitle{Consignes sur les limites de chaque tribu}
\VerseOne{}Yahweh parla aussi à Moïse, en disant :
\VS{2}Donne l'ordre aux enfants d'Israël, et dis-leur :  Parce que vous allez entrer au pays de Canaan, ce pays deviendra votre héritage, le pays de Canaan selon ses limites.
\VS{3}Votre frontière du côté du sud sera depuis le désert de Tsin, le long d'Edom, et votre frontière du côté du sud commencera au bout de la mer salée, vers l'orient ;
\VS{4}Et cette frontière tournera du sud vers la montée d'Akrabbim, et passera jusqu'à Tsin ; et elle aboutira du côté du sud de Kadès-Barnéa ; et sortira aussi par Hatsar-Addar, et passera jusqu'à Atsmon.
\VS{5}Et cette frontière tournera depuis Atsmon jusqu'au torrent d'Egypte ; et elle aboutira à la mer.
\VS{6}Quant à la frontière d'occident, vous aurez la grande mer et ses limites ; ce sera votre frontière occidentale.
\VS{7}Et ce sera ici votre frontière au nord ; depuis la grande mer, vous marquerez pour vos limites la montagne de Hor ;
\VS{8}et depuis la montagne de Hor, vous marquerez pour vos limites l’entrée de Hamath, et cette frontière aboutira vers Tsedad ;
\VS{9}cette frontière passera jusqu’à Ziphron, et elle aboutira à Hatsar-Enan ; telle sera votre frontière au nord.
\VS{10}Puis vous marquerez pour vos limites vers l'orient de Hatsar-Enan à Schepham.
\VS{11}Et cette frontière descendra de Schepham à Ribla, du côté de l'orient d'Aïn ; et cette frontière descendra et s'étendra le long de la mer de Kinnéreth vers l'orient.
\VS{12}Cette frontière descendra au Jourdain pour aboutir à la mer salée ; tel sera le pays que vous aurez avec ses limites tout autour.
\VS{13}Et Moïse donna l'ordre aux enfants d'Israël, en disant : C'est là le pays que vous hériterez par le sort, et que Yahweh a ordonné de donner à neuf tribus, et à la demi-tribu.
\VS{14}Car la tribu des fils de Ruben selon les familles de leurs pères, et la tribu des fils de Gad, selon les familles de leurs pères, ont pris leur héritage ; et la demi-tribu de Manassé a pris aussi son héritage.
\VS{15}Deux tribus, dis-je, et la demi-tribu ont pris leur héritage de l'autre côté du Jourdain, vis-à-vis de Jéricho, du côté du levant.
\VS{16}Et Yahweh parla à Moïse, en disant :
\VS{17}Ce sont ici les noms des hommes qui vous partageront le pays : Eléazar le sacrificateur, et Josué fils de Nun.
\VS{18}Vous prendrez aussi un prince de chaque tribu pour faire le partage du pays.
\VS{19}Et voici les noms de ces hommes. Pour la tribu de Juda : Caleb, fils de Jephunné ;
\VS{20}pour la tribu des fils de Siméon : Samuel, fils d'Ammihud ;
\VS{21}pour la tribu de Benjamin : Elidad, fils de Kislon ;
\VS{22}pour la tribu des fils de Dan : Celui qui en est le chef, Buki, fils de Jogli ;
\VS{23}pour les fils de Joseph, pour la tribu des fils de Manassé : Celui qui en est le chef, Hanniel, fils d'Ephod ;
\VS{24}et pour la tribu des fils d'Ephraïm : Celui qui en est le chef, Kemuel, fils de Schiphtan ;
\VS{25}pour la tribu des fils de Zabulon : Celui qui en est le chef, Elitsaphan, fils de Parnac ;
\VS{26}pour la tribu des fils d'Issacar : Celui qui en est le chef, Paltiel, fils d'Azzan ;
\VS{27}pour la tribu des fils d'Aser : Celui qui en est le chef, Ahihud, fils de Schelomi.
\VS{28}pour la tribu des fils de Nephthali : Celui qui en est le chef, Pedahel, fils d'Ammihud.
\VS{29}Ce sont là, ceux à qui Yahweh donna l'ordre de partager l'héritage aux enfants d'Israël dans le pays de Canaan.
\Chap{35}
\TextTitle{Quarante-huit villes pour les Lévites dont six villes de refuge}
\VerseOne{}Yahweh parla à Moïse dans les plaines de Moab, près du Jourdain, vis-à-vis de Jéricho, en disant :
\VS{2}Donne l'ordre aux enfants d'Israël qu’ils donnent aux Lévites, sur l'héritage qu'ils posséderont, des villes pour y habiter. Vous leur donnerez aussi les faubourgs qui sont autour de ces villes\FTNT{Jos. 21:2.}.
\VS{3}Ils auront donc les villes pour y habiter ; et les faubourgs de ces villes seront pour leurs bétails, pour leurs biens, et pour tous leurs animaux.
\VS{4}Les faubourgs des villes que vous donnerez aux Lévites, seront de mille coudées tout autour depuis la muraille de la ville en dehors.
\VS{5}Et vous mesurerez depuis le dehors de la ville du côté de l'orient, deux mille coudées ; et du côté du sud, deux mille coudées ; et du côté de l'occident, deux mille coudées ; et du côté du nord, deux mille coudées ; et la ville sera au milieu ; tels seront les faubourgs de leurs villes.
\VS{6}Et des villes que vous donnerez aux Lévites, il y aura six villes de refuge que vous donnerez pour que le meurtrier s’y enfuie, et outre celles-là, vous leur donnerez quarante-deux villes.
\VS{7}Toutes les villes que vous donnerez aux Lévites seront quarante-huit villes, elles et leurs faubourgs.
\VS{8}Et quant aux villes que vous leur donnerez sur la possession des enfants d'Israël, de ceux qui en auront plus vous en prendrez plus, et de ceux qui en auront moins vous en prendrez moins; chacun donnera de ses villes aux Lévites, en proportion de l'héritage qu'il possédera.
\VS{9}Puis Yahweh parla à Moïse, en disant :
\VS{10}Parle aux enfants d'Israël, et dis-leur : Quand vous aurez passé le Jourdain, pour entrer au pays de Canaan ;
\VS{11}Etablissez-vous des villes qui vous soient des villes de refuge, afin que le meurtrier qui aura frappé à mort quelqu'un involontairement, s’y enfuie\FTNT{Jos. 20:2-3 ; Ex. 21:13.}.
\VS{12}Et ces villes seront pour vous des villes de refuge contre le vengeur, afin que le meurtrier ne meure pas, jusqu'à ce qu'il ait comparu en jugement devant l'assemblée.
\VS{13}De ces villes que vous donnerez, il y en aura six de refuge pour vous.
\VS{14}Vous donnerez trois de ces villes au-delà du Jourdain, et les trois autres dans le pays de Canaan, qui seront des villes de refuge\FTNT{De. 19:2 ; De. 4:41-42.}.
\VS{15}Ces six villes serviront de refuge aux enfants d'Israël, à l'étranger et à celui qui séjourne au milieu de vous, afin que quiconque aura frappé à mort quelqu'un involontairement, s’y enfuie.
\VS{16}Mais si un homme en frappe un autre avec un instrument de fer, et qu'il en meure, il est meurtrier ; on punira de mort le meurtrier.
\VS{17}Et s'il le frappe avec une pierre qu'il tenait à la main, dont on puisse mourir, et qu’il en meure, c'est un meurtrier ; le meurtrier sera puni de mort.
\VS{18}De même s'il le frappe d'un instrument de bois qu'il tenait à la main, dont on puisse mourir, et qu’il en meure, il est un meurtrier ;  on punira de mort le meurtrier.
\VS{19}Et le vengeur du sang fera mourir le meurtrier quand il le rencontrera, il pourra le faire mourir.
\VS{20}Et s’il le pousse par haine, ou s'il jette quelque chose sur lui avec préméditation, et qu’il en meure ;
\VS{21}ou si par inimitié il le frappe de sa main, et qu’il en meure, on punira de mort celui qui l'a frappé, car il est meurtrier ; le vengeur du sang pourra le faire mourir quand il le rencontrera\FTNT{De. 19:11-12.}.
\VS{22}Mais s'il le pousse subitement, sans inimitié, ou s'il jette quelque chose  sur lui, sans préméditation,
\VS{23}ou s'il fait tomber sur lui quelque pierre sans l’avoir vu, et qu’il en meure, n’étant pas son ennemi et ne lui cherchant pas du mal,
\VS{24}Alors l'assemblée jugera entre celui qui a frappé et le vengeur du sang, selon ces ordonnances ;
\VS{25}l'assemblée délivrera le meurtrier de la main du vengeur de sang, et le fera retourner dans la ville de refuge où il s'était enfui. Il y demeurera jusqu'à la mort du souverain sacrificateur, qui aura été oint de la sainte huile.
\VS{26}Mais si le meurtrier sort de quelque manière que ce soit hors des bornes de la ville de son refuge, où il s'est enfui,
\VS{27}et si le vengeur du sang le rencontre hors des bornes de la ville de son refuge, et qu'il tue le meurtrier, il ne sera point coupable de meurtre.
\VS{28}Car il doit demeurer dans la ville de son refuge jusqu’à la mort du souverain sacrificateur ; et après la mort du souverain sacrificateur, le meurtrier pourra retourner dans sa possession.
\VS{29}Et ces choses-ci seront des ordonnances de jugement pour vous et pour vos générations, dans toutes vos demeures.
\VS{30}Celui qui fera mourir le meurtrier, le fera mourir sur la parole de deux témoins ; mais un seul témoin ne sera point reçu en témoignage contre quelqu'un, pour le faire mourir\FTNT{De. 17:6 ; De. 19:15.}.
\VS{31}Et vous ne prendrez point de rançon pour la vie du meurtrier, qui est coupable et digne de mort ; mais il doit être puni de mort.
\VS{32}Vous ne prendrez point de rançon pour le laisser s'enfuir de sa ville de refuge, pour qu’il retourne habiter dans le pays, jusqu'à la mort du sacrificateur.
\VS{33}Et vous ne souillerez point le pays où vous serez, car le sang souille le pays ; et il ne se fera point de propitiation pour le pays, du sang qui y sera répandu que par le sang de celui qui l'aura répandu.
\VS{34}Vous ne souillerez donc point le pays où vous allez demeurer, et au milieu duquel j'habiterai ; car je suis Yahweh qui habite au milieu des enfants d'Israël.
\Chap{36}
\TextTitle{Loi sur les héritages\FTNTT{No. 27:1-11}}
\VerseOne{}Or les chefs des pères de la famille des fils de Galaad, fils de Makir, fils de Manassé, d'entre les familles des fils de Joseph, s'approchèrent et parlèrent devant Moïse, et devant les princes, les chefs des pères des enfants d'Israël,
\VS{2}et ils dirent : Yahweh a donné l'ordre à mon seigneur de donner aux enfants d'Israël le pays en héritage par le sort ; et mon seigneur a reçu l'ordre de Yahweh de donner l'héritage de Tselophchad, notre frère, à ses filles.
\VS{3}Si elles se marient à l'un des fils des autres tribus d'Israël, leur héritage sera retranché de l'héritage de nos pères et sera ajouté à l'héritage de la tribu de laquelle elles seront ; ainsi sera diminué l'héritage qui nous est échu par le sort.
\VS{4}Même quand viendra le jubilé pour les enfants d'Israël, on ajoutera leur héritage à l'héritage de la tribu à laquelle elles appartiendront, ainsi leur héritage sera retranché de l'héritage de la tribu de nos pères\FTNT{Lé. 25:10-13.}.
\VS{5}Et Moïse ordonna aux enfants d'Israël, suivant l'ordre et la bouche Yahweh, en disant : Ce que la tribu des fils de Joseph dit est juste.
\VS{6}C’est ici ce que Yahweh ordonne au sujet des filles de Tselophchad : Elles se marieront à qui bon leur semblera, toutefois elles se marieront dans l'une des familles de la tribu de leurs pères.
\VS{7}Ainsi l’héritage ne sera point transporté entre les enfants d'Israël de tribu en tribu; car chacun des enfants d'Israël se tiendra à l'héritage de la tribu de ses pères.
\VS{8}Et toute fille, qui possédera un héritage d’entre les tribus des enfants d'Israël, se mariera à quelqu'un de la famille de la tribu de son père, afin que chacun des enfants d'Israël possède l'héritage de ses pères.
\VS{9}L'héritage donc ne sera point transporté d'une tribu à une autre, mais chacune des tribus des enfants d'Israël se tiendra à son héritage.
\VS{10}Les filles de Tselophchad firent comme Yahweh avait donné à Moïse.
\VS{11}Machla, Thirtsa, Hogla, Milca, et Noa, filles de Tselophchad, se marièrent aux fils de leurs oncles.
\VS{12}Ainsi elles se marièrent à ceux qui étaient des familles des fils de Manassé, fils de Joseph ; et leur héritage demeura dans la tribu de la famille de leur père.
\VS{13}Ce sont là les ordonnances et les jugements que Yahweh ordonna par Moïse aux enfants d'Israël, dans les plaines de Moab, près du Jourdain, vis-à-vis de Jéricho.
\PPE{}
\end{multicols}
