\ShortTitle{Jacques}\BookTitle{Jacques}\BFont
\noindent\hrulefill
{\footnotesize
\textit{
\bigskip
{\centering{}
\\Signification : Qui supplante
\\Thème : La vie chrétienne sous son aspect pratique
\\Auteur : Jacques
\\Date de rédaction : Env. 45-50\\}
}
%\bigskip
\textit{
\\Jacques, frère de Jésus-Christ homme et ancien au sein de la première église chrétienne située à Jérusalem, écrivit aux chrétiens d’origines juives dispersés dans l’Empire romain. Il les consola quant à l’adversité qu’ils rencontraient et les exhorta à tenir ferme, leur expliquant que la foi authentique devait être accompagnée d’œuvres. Il les mit en garde contre la convoitise, source de toutes tentations, et les prévient également quant à l’amour du monde et la confiance que certains pouvaient mettre dans l’argent. Pour terminer, il exhorta tout un chacun à être patient dans l’épreuve et à prier sans cesse jusqu’au retour du Seigneur.\bigskip
}
}
\par\nobreak\noindent\hrulefill
\begin{multicols}{2}
\Chap{1}
\TextTitle{Introduction}
\VerseOne{}Jacques, serviteur de Dieu, et du Seigneur Jésus-Christ, aux douze tribus qui sont dispersées, salut.
\TextTitle{[L'épreuve de la foi
\\Le but de l'épreuve de la foi]}
\VS{2}Mes frères, regardez comme un sujet d'une parfaite joie quand vous êtes exposés à diverses épreuves,
\VS{3}sachant que l'épreuve de votre foi produit la patience.
\VS{4}Mais il faut que la patience accomplisse parfaitement son œuvre, afin que vous soyez parfaits et accomplis, en sorte qu’il ne vous manque rien.
\VS{5}Et si quelqu'un de vous manque de sagesse, qu'il la demande à Dieu, qui la donne à tous libéralement, et sans reproche, et elle lui sera donnée.
\VS{6}Mais qu'il la demande avec foi, ne doutant nullement ; car celui qui doute est semblable au flot de la mer, agité et poussé çà et là par le vent.
\VS{7}Qu’un tel homme ne s'attende pas à recevoir quelque chose du Seigneur.
\VS{8}L'homme double de coeur est inconstant dans toutes ses voies.
\VS{9}Que le frère de basse condition se glorifie dans son élévation.
\VS{10}Que le riche, au contraire, se glorifie dans sa basse condition ; car il passera comme la fleur de l'herbe.
\VS{11}En effet, le soleil s'est levé avec sa chaleur ardente et l'herbe a séché, sa fleur est tombée, et son éclat a péri, ainsi le riche se flétrira dans ses entreprises.
\VS{12}Heureux l'homme qui endure la tentation\FTNT{Le terme grec «~peirasmos~» utilisé dans ce verset veut aussi dire «~épreuve~».} ; car après avoir été éprouvé, il recevra la couronne de vie, que le Seigneur a promise à ceux qui l'aiment.
\TextTitle{L'excitation au mal ne vient pas de Dieu}
\VS{13}Quand quelqu'un est tenté, qu'il ne dise pas : Je suis tenté par Dieu. Car Dieu ne peut être tenté par le mal, et aussi ne tente-t-il personne.
\VS{14}Mais chacun est tenté quand il est attiré et amorcé par sa propre convoitise.
\VS{15}Puis quand la convoitise a conçu, elle enfante le péché ; et le péché étant consommé, produit la mort.
\VS{16}Mes frères bien-aimés, ne vous y trompez pas :
\VS{17}Tout ce qui nous est donné d'excellent et tout don parfait viennent d'en haut, et descendent du Père des lumières, en qui il n'y a ni changement ni ombre de variation.
\VS{18}Il nous a engendrés de sa propre volonté, par la parole de la vérité, afin que nous soyons comme les prémices de ses créatures.
\VS{19}Ainsi, mes frères bien-aimés, que tout homme soit prompt à écouter, lent à parler, et lent à la colère ;
\VS{20}car la colère de l'homme n'accomplit pas la justice de Dieu.
\VS{21}C'est pourquoi, rejetant toute souillure, et tout résidu\FTNT{Le mot «~résidu~» vient du grec «~perisseia~», ce mot signifie «~abondance~», «~surabondamment~», «~tout excès~», «~reste~». Les Grecs utilisaient ce terme pour décrire l'excès de cire dans leurs oreilles. Il est question de la méchancesté qui reste dans un Chrétien et qui provient de son état antérieur à sa conversion.} de méchanceté, recevez avec douceur la parole qui a été plantée en vous, et qui peut sauver vos âmes.
\TextTitle{Le test de l'obéissance}
\VS{22}Et mettez en pratique la parole, et ne l'écoutez pas seulement, en vous trompant vous-mêmes par de vains discours.
\VS{23}Car si quelqu'un écoute la parole, et ne la met pas en pratique, il est semblable à un homme qui regarde dans un miroir son visage naturel,
\VS{24}et qui après s'être regardé, s'en va, et oublie aussitôt comment il était.
\VS{25}Mais celui qui aura plongé les regards dans la loi parfaite, la loi de la liberté, et qui aura persévéré, n'étant point un auditeur oublieux, mais pratiquant les œuvres qui lui sont prescrites, celui-là sera heureux dans son oeuvre.
\TextTitle{Le test de la vraie religion}
\VS{26}Si quelqu'un parmi vous croit être religieux alors qu’il ne tient pas sa langue en bride, mais séduit son cœur, la religion d'un tel homme est vaine.
\VS{27}La religion pure et sans tache envers notre Dieu et notre Père, c'est de visiter les orphelins et les veuves dans leurs afflictions, et de se conserver pur des souillures de ce monde.
\Chap{2}
\TextTitle{Le test de l'amour fraternel}
\VerseOne{}Mes frères, n’ayez point la foi en notre Seigneur Jésus-Christ glorieux, en ayant égard à l’apparence des personnes.
\VS{2}En effet, s'il entre dans votre assemblée un homme qui porte un anneau d'or et un habit magnifique, et qu’il y entre aussi un pauvre misérablement vêtu ;
\VS{3}et que vous ayez égard à celui qui porte l’habit magnifique, et lui disiez : Toi, assieds-toi ici honorablement ! Et que vous disiez au pauvre : Toi, tiens-toi là debout ; ou assieds-toi sur mon marchepied ;
\VS{4}n’avez-vous pas fait différence en vous-mêmes, et n’êtes-vous pas des juges qui avez des pensées injustes ?
\VS{5}Ecoutez, mes frères bien-aimés, Dieu n'a-t-il pas choisi les pauvres de ce monde, qui sont riches en la foi et héritiers du Royaume qu'il a promis à ceux qui l'aiment ?
\VS{6}Mais vous avez déshonoré le pauvre. Et cependant les riches ne vous oppriment-ils pas, et ne vous traînent-ils pas devant les tribunaux ?
\VS{7}N’est-ce pas eux qui blasphèment le beau Nom qui a été invoqué sur vous ?
\VS{8}Si vous accomplissez la loi royale, qui est selon l'Ecriture, tu aimeras ton prochain comme toi-même\FTNT{Lé. 19:18.} ; vous faites bien.
\VS{9}Mais si vous avez égard à l'apparence des personnes, vous commettez un péché, et vous êtes convaincus par la loi comme des transgresseurs.
\VS{10}Car quiconque observe toute la loi, mais pèche contre un seul commandement, devient coupable de tous.
\VS{11}En effet, celui qui a dit : Tu ne commettras point d’adultère, a aussi dit : Tu ne tueras point. Or, si donc tu ne commets point d’adultère\FTNT{Ex. 20:13-14.}, mais que tu tues, tu deviens transgresseur de la loi.
\VS{12}Parlez et agissez comme devant être jugés par la loi de la liberté.
\VS{13}Car il y aura un jugement sans miséricorde sur celui qui n’aura point usé de miséricorde\FTNT{Mt. 7:2.}. La miséricorde triomphe du jugement.
\VS{14}Mes frères, que servira-t-il à quelqu’un de dire qu'il a la foi, s’il n’a pas les œuvres ? Cette foi peut-elle le sauver ?
\VS{15}Et si un frère ou une sœur sont nus et manquent de ce qui leur est nécessaire chaque jour pour vivre,
\VS{16}et que l’un d'entre vous leur dise : Allez en paix, chauffez-vous, et rassasiez-vous ! Et que vous ne leur donniez pas les choses nécessaires pour le corps, que leur servira cela ?
\VS{17}De même aussi la foi, si elle n'a pas les œuvres, elle est morte en elle-même.
\VS{18}Mais quelqu'un dira : Tu as la foi, et moi j'ai les œuvres. Montre-moi donc ta foi sans les œuvres, et moi je te montrerai ma foi par mes œuvres.
\VS{19}Tu crois qu'il n'y a qu'un Dieu, tu fais bien ; les démons le croient aussi, et ils tremblent.
\VS{20}Mais, ô homme vain ! Veux-tu savoir que la foi qui est sans les œuvres est morte ?
\TextTitle{[L'exemple d'Abraham]
\\(Ro. 4:1-25)}
\VS{21}Abraham notre père ne fut-il pas justifié par les œuvres, quand il offrit son fils Isaac sur l'autel ?
\VS{22}Ne vois-tu donc pas que sa foi agissait avec ses œuvres, et que ce fut par ses œuvres que sa foi fut rendue parfaite ?
\VS{23}Ainsi s’accomplit ce que dit l’Ecriture : Abraham crut en Dieu, et cela lui fut imputé à justice\FTNT{Ge. 15:6.}; et il fut appelé ami de Dieu.
\VS{24}Vous voyez donc que l'homme est justifié par les œuvres, et non par la foi seulement.
\VS{25}Pareillement Rahab, la prostituée, ne fut-elle pas également justifiée par les œuvres, lorsqu’elle reçut les messagers, et qu'elle les fit partir par un autre chemin\FTNT{Jos. 2:1-21.} ?
\VS{26}Car comme le corps sans esprit est mort, de même la foi sans les œuvres est morte.
\Chap{3}
\TextTitle{La réalité de la foi, prouvée par le contrôle de la langue}
\VerseOne{}Ne soyez pas nombreux, mes frères, à devenir des enseignants\FTNT{Du grec «~didaskalos~» : «~maître~», «~professeur~», «~docteur chargé d’instruire, d’enseigner la parole~». Mt. 8:19 ; Mt. 22:16 ; 1 Co. 12:28.}, sachant que nous en recevrons un plus grand jugement.
\VS{2}Car nous péchons tous en plusieurs choses ; si quelqu’un ne pèche pas en paroles, c’est un homme parfait, et il peut même tenir en bride tout le corps.
\VS{3}Voici, nous mettons le mors dans la bouche des chevaux, afin qu’ils nous obéissent, et nous menons çà et là tout le corps.
\VS{4}Voici aussi les navires, quoiqu’ils soient si grands, et qu’ils soient agités par la tempête, ils sont dirigés partout çà et là par un petit gouvernail, selon qu’il plaît à celui qui les gouverne.
\VS{5}Il en est ainsi de la langue, c'est un petit membre, et cependant elle peut se vanter de grandes choses. Voici un petit feu, combien de bois allume-t-il?
\VS{6}La langue aussi est un feu ; c’est le monde de l’iniquité. La langue est placée parmi nos membres, souillant tout le corps, et enflammant tout le cours de la vie, étant elle-même enflammée par le feu de la géhenne.
\VS{7}Car toutes les espèces d’animaux sauvages, d'oiseaux, de reptiles, et d’animaux marins, se domptent et ont été domptés par la nature humaine.
\VS{8}Mais nul homme ne peut dompter la langue : c'est un mal qu’on ne peut réprimer, elle est pleine d'un venin mortel.
\VS{9}Par elle nous bénissons Dieu notre Père, et par elle nous maudissons les hommes faits à la ressemblance de Dieu.
\VS{10}De la même bouche sortent la bénédiction et la malédiction. Il ne faut pas qu’il en soit ainsi mes frères.
\VS{11}Une fontaine fait-elle jaillir par la même ouverture l’eau douce et l’eau amère ?
\VS{12}Mes frères, un figuier peut-il produire des olives, ou une vigne des figues ? De même, aucune fontaine ne peut produire de l'eau salée et de l'eau douce.
\VS{13}Y a-t-il parmi vous quelque homme sage et intelligent ? Qu’il fasse voir ses oeuvres par une bonne conduite avec douceur et sagesse.
\VS{14}Mais si vous avez un zèle amer et de l'irritation dans vos cœurs, ne vous glorifiez pas, et ne mentez pas en déshonorant la vérité de l'Evangile.
\VS{15}Car ce n’est pas là la sagesse qui descend d’en haut ; mais c’est une sagesse terrestre, animale\FTNT{Animale ou charnelle.} et diabolique.
\VS{16}Car là où il y a de la jalousie et un esprit de querelle, là est le désordre, et toute sorte de mal.
\VS{17}Mais la sagesse d'en haut est premièrement pure, ensuite pacifique, modérée, conciliante, pleine de miséricorde et de bons fruits, sans partialité, et sans hypocrisie.
\VS{18}Or le fruit de la justice est semé dans la paix pour ceux qui s’adonnent à la paix.
\Chap{4}
\TextTitle{Réprobation de la mondanité}
\VerseOne{}D'où viennent parmi vous les disputes et les querelles ? N’est-ce pas de vos voluptés qui combattent dans vos membres ?
\VS{2}Vous convoitez, et vous n’obtenez pas ce que vous désirez ; vous avez une envie mortelle, vous êtes jaloux, et vous ne pouvez obtenir ce que vous enviez ; vous vous querellez, vous vous disputez, et vous n'avez pas ce que vous désirez, parce que vous ne demandez pas.
\VS{3}Vous demandez, et vous ne recevez point, parce que vous demandez mal, dans le but de satisfaire vos voluptés.
\VS{4}Hommes et femmes adultères, ne savez-vous pas que l'amitié du monde est inimitié contre Dieu ? Celui donc qui veut être ami du monde, se rend ennemi de Dieu.
\VS{5}Pensez-vous que l'Ecriture parle en vain ; l’Esprit qui habite en nous, vous inspire-t-il l’envie ?
\VS{6}Il vous accorde au contraire une plus grande grâce ; c’est pourquoi l’Ecriture dit : Dieu résiste aux orgueilleux, mais il fait grâce aux humbles\FTNT{Pr. 3:34.}.
\VS{7}Soumettez-vous donc à Dieu ; résistez au diable, et il s'enfuira de vous.
\VS{8}Approchez-vous de Dieu, et il s'approchera de vous. Pécheurs, nettoyez vos mains ; et vous qui êtes doubles de cœur, purifiez vos cœurs.
\VS{9}Sentez vos misères ; soyez dans le deuil et dans les larmes ; que votre rire se change en pleurs, et votre joie en tristesse.
\VS{10}Humiliez-vous dans la présence du Seigneur, et il vous élèvera.
\VS{11}Mes frères ne médisez point les uns des autres ; celui qui médit de son frère, et qui condamne son frère, médit de la loi et juge la loi. Or si tu juges la loi, tu n'es pas observateur de la loi, mais le juge.
\VS{12}Il n'y a qu'un seul Législateur, qui peut sauver et qui peut perdre ; mais toi qui es-tu, qui juges les autres ?
\VS{13}A vous, maintenant, qui dites : Aujourd’hui ou demain nous irons dans telle ou telle ville, et nous y passerons une année, et nous trafiquerons et nous gagnerons ;
\VS{14}qui toutefois ne savez pas ce qui arrivera le lendemain car qu’est-ce que votre vie ? Ce n’est certes qu’une vapeur qui parait pour un peu de temps, et qui ensuite s’évanouit ;
\VS{15}au lieu de dire : Si le Seigneur le veut et si nous vivons, nous ferons aussi ceci ou cela.
\VS{16}Mais maintenant vous vous glorifiez dans vos pensées orgueilleuses ; toute vanterie de cette nature est mauvaise.
\VS{17}Il y a donc du péché en celui qui sait faire le bien, et qui ne le fait pas.
\Chap{5}
\TextTitle{Avertissement aux riches}
\VerseOne{}A vous maintenant, riches ! Pleurez et gémissez à cause des malheurs qui vont tomber sur vous.
\VS{2}Vos richesses sont pourries et vos vêtements sont rongés par les vers.
\VS{3}Votre or et votre argent sont rouillés ; et leur rouille s’élèvera en témoignage contre vous et dévorera vos chairs comme un feu. Vous avez amassé des trésors pour les derniers jours.
\VS{4}Voici, le salaire des ouvriers qui ont moissonné vos champs, et dont vous les avez frustrés, crie ; et les cris des moissonneurs sont parvenus aux oreilles du Seigneur des armées.
\VS{5}Vous avez vécu dans les délices sur la terre, vous vous êtes livrés aux voluptés, et vous avez rassasié vos cœurs comme au jour de sacrifices.
\VS{6}Vous avez condamné et mis à mort le juste qui ne vous a pas résisté.
\TextTitle{Exhortations en vue du retour du Seigneur}
\VS{7}Mais vous, mes frères, attendez patiemment jusqu'à l’avènement du Seigneur. Voici, le laboureur attend le précieux fruit de la terre, prenant patience à son égard, jusqu'à ce qu'il ait reçu les pluies de la première et de la dernière saison.
\VS{8}Vous aussi, attendez patiemment, et affermissez vos cœurs, car l’avènement du Seigneur est proche.
\VS{9}Mes frères, ne vous plaignez pas les uns des autres, afin que vous ne soyez pas jugés. Voici, le Juge se tient à la porte.
\VS{10}Mes frères, prenez pour exemple de patience dans les afflictions les prophètes qui ont parlé au Nom du Seigneur.
\VS{11}Voici, nous disons bienheureux ceux qui ont souffert patiemment. Vous avez entendu parler de la patience de Job, et vous avez vu la fin que le Seigneur lui accorda, car le Seigneur est plein de compassion et de miséricorde.
\VS{12}Avant toutes choses, mes frères, ne jurez ni par le ciel, ni par la terre, ni par aucun autre serment. Mais que votre oui soit oui, et que votre non soit non, afin que vous ne tombiez pas sous le jugement\FTNT{Mt. 5:37 ; Mt. 12:36.}.
\VS{13}Quelqu’un parmi vous est-il dans la souffrance ? Qu’il prie. Quelqu’un est-il dans la joie ? Qu’il chante.
\VS{14}Quelqu’un parmi vous est-il malade ? Qu’il appelle les anciens de l'église, et qu'ils prient pour lui en l’oignant d'huile au Nom du Seigneur.
\VS{15}Et la prière faite avec foi sauvera le malade, et le Seigneur le relèvera ; et s'il a commis des péchés, ils lui seront pardonnés.
\VS{16}Confessez donc vos péchés les uns les autres, et priez les uns pour les autres afin que vous soyez guéris. Car la prière du juste faite avec véhémence est de grande efficacité.
\VS{17}Elie était un homme sujet aux mêmes infirmités que nous, et cependant il pria avec instance pour qu'il ne pleuve point, et il ne tomba point de pluie sur la terre pendant trois ans et six mois\FTNT{1 R. 17:1.}.
\VS{18}Puis il pria de nouveau, et le ciel donna de la pluie, et la terre produisit son fruit.
\TextTitle{Conclusion}
\VS{19}Mes frères, si quelqu'un parmi vous s’est égaré loin de la vérité, et qu’un autre l'y ramène,
\VS{20}qu'il sache que celui qui ramènera un pécheur de son égarement, sauvera une âme de la mort, et couvrira une multitude de péchés.
\PPE{}
\end{multicols}
