\begin{center}{\LARGE Introduction}\end{center}
\begin{small}
\subsection*{Pourquoi cette Bible révisée~?}

En novembre 2013, alors que j'étais en prière, je demandais au Seigneur ce qu'il attendait de moi. Ce dernier m'a répondu à travers plusieurs songes dans lesquels il me disait de réviser la Bible. Je dois dire que j'ai eu du mal à croire que Dieu puisse me demander une telle chose. De plus, je me sentais incapable d'assumer un si grand projet, aussi je lui ai demandé à plusieurs reprises de me confirmer que c'était bien sa volonté, chose qu'il a faite. J'ai ensuite parlé de ce que j'avais reçu à des frères et sœurs qui travaillent avec moi et ces derniers m'ont confirmé que cette vision venait bien du Seigneur. Une dynamique s'est créée aussitôt et bien qu'aucun d'entre nous ne se sentît à la hauteur de la tâche qui nous était confiée, nous nous sommes rapidement organisés pour concrétiser cette vision, comptant sur le Seigneur pour qu'il nous donne les capacités et la sagesse dont nous avions besoin.\bigskip

Deux constats majeurs nous ont amenés à la conclusion qu'une révision de la Bible était plus que nécessaire. Tout d'abord, la plupart des bibles modernes les plus diffusées sont basées sur le texte minoritaire comportant une quantité importante de fautes de traduction, d'omissions et de rajouts qui altèrent la compréhension du message et induisent par conséquent le lecteur en erreur. Or il est du devoir de tout chrétien de mettre en pratique la Parole, notamment en veillant sur son authenticité.\bigskip

«~\emph{Car, je vous le dis en vérité, tant que le ciel et la terre ne passeront point, il ne disparaîtra pas de la loi un seul iota ou un seul trait de lettre jusqu'à ce que tout soit arrivé. Celui donc qui aura violé l'un de ces petits commandements, et qui aura enseigné les hommes à faire de même, sera appelé le plus petit au Royaume des cieux~; mais celui qui les observera, et qui enseignera à les observer, celui-là sera appelé grand au Royaume des cieux.}~» Matthieu 5:18-19.\bigskip

«~\emph{Je le déclare à quiconque entend les paroles de la prophétie de ce livre~: Si quelqu'un y ajoute quelque chose, Dieu le frappera des fléaux décrits dans ce livre. Et si quelqu'un retranche quelque chose des paroles du livre de cette prophétie, Dieu retranchera sa part de l'arbre de vie, et de la ville sainte, décrits dans ce livre.}~» Apocalypse 22:18-19.\bigskip

Nous ne devons pas oublier que la Bible a été initialement écrite en trois langues, à savoir l'hébreu, le grec et quelques versets en araméen. En réalisant cette révision, notre but est de restituer le sens des mots d'origine et d'expurger toute l'influence de l'ennemi. Ce travail a permis de mettre en lumière une évidence~: la personne de Jésus-Christ occupe une place centrale de Genèse à Apocalypse, ce qui ne fait que confirmer et attester sa divinité.\bigskip

«~\emph{Puis il leur dit~: C'est là ce que je disais lorsque j'étais encore avec vous, qu'il fallait que s'accomplisse tout ce qui est écrit de moi dans la loi de Moïse, dans les prophètes, et dans les psaumes.}~» Luc 24:44.\bigskip

Ensuite, nous déplorons le fait que la majorité des bibles en circulation soient vendues alors que Jésus-Christ a dit «~\emph{Vous l'avez reçu gratuitement, donnez-le gratuitement}~» (Mt. 10:8). Il est donc impensable que celui qui a chassé du temple vendeurs et changeurs puisse approuver un seul instant le commerce qui est fait avec sa Parole (Jn. 2:14-16).\bigskip

«~\emph{Vous tous qui avez soif, venez aux eaux, et vous qui n'avez pas d'argent, venez, achetez et mangez~; venez, dis-je, achetez du vin et du lait sans argent et sans rien payer~!}~» \vref{Esaïe 55:1}.\bigskip

«~\emph{Il me dit aussi~: Tout est accompli. Je suis l'Alpha et l'Oméga, le commencement et la fin. A celui qui a soif, je lui donnerai de la source d'eau vive, gratuitement.}~» Apocalypse 21:6.\bigskip

«~\emph{Et l'Esprit et l'épouse disent~: Viens. Et que celui qui entend dise~: Viens. Et que celui qui a soif vienne~; que celui qui veut prenne gratuitement de l'eau de la vie.}~» Apocalypse 22:17.\bigskip

Les apôtres ont scrupuleusement respecté l'ordre du Seigneur en Matthieu 10:8. Pierre a dénoncé avec la plus grande sévérité Simon, le magicien qui avait eu la folie de croire que le don de Dieu pouvait être monnayé. Et durant tout son service, Paul a enseigné l'Evangile gratuitement.\bigskip

«~\emph{Puis ils leur imposèrent les mains, et ils reçurent le Saint-Esprit. Lorsque Simon vit que le Saint-Esprit était donné par l’imposition des mains des apôtres, il leur présenta de l’argent, en leur disant : Donnez-moi aussi ce pouvoir, afin que tous ceux à qui j’imposerai les mains reçoivent le Saint-Esprit. Mais Pierre lui dit~: Que ton argent périsse avec toi, puisque tu as estimé que le don de Dieu s’acquérait avec de l’argent. Tu n’as point de part ni d’héritage en cette affaire ; car ton coeur n’est point droit devant Dieu. Repens-toi donc de cette méchanceté, et prie Dieu, afin que, s’il est possible, la pensée de ton coeur te soit pardonnée. Car je vois que tu es dans un fiel très amer et dans un lien d’iniquité.}~» Actes 8:17-23.\bigskip

«~\emph{Je n'ai désiré ni l'argent, ni l'or, ni les vêtements de personne.}~» Actes 20:33.\bigskip

«~\emph{Quelle récompense en ai-je donc~? C’est qu’en prêchant l’Evangile, je prêche l’Evangile de Christ sans qu’il en coûte rien, afin que je n’abuse pas de mon pouvoir dans l’Evangile.}~» 1 Corinthiens 9:18.\bigskip

Nous pensons qu'il est juste et honnête que la Bible porte le nom de son véritable auteur et qu'elle soit gratuitement diffusée selon sa volonté et l'ordre clair qu'il a donné. Cette Bible s'appelle donc La Bible de Jésus-Christ et est gratuitement mise à la disposition de ceux qui souhaitent se la procurer.\bigskip

\subsection*{Comment a été réalisée cette révision~?}

Pour réaliser cette révision, nous nous sommes appuyés sur le texte majoritaire (originaux et traductions). Ainsi, tout en essayant de conserver un vocabulaire qui soit à la portée de tous, certains mots et expressions ont été changés pour restituer pleinement leur signification initiale. A titre d'exemple, vous constaterez régulièrement que certains mots sont répétés deux fois de suite. Cela n'est pas une erreur mais la restitution littérale de certaines expressions qui insistent sur une vérité (voir commentaire en Gn. 2:15-17). En effet, Dieu parle une fois et une seconde fois pour avertir les hommes (Job. 33:14). «~Et quant à ce que le songe a été réitéré à Pharaon pour la seconde fois, c'est que la chose est arrêtée de la part de Dieu, et que Dieu se hâtera de l'exécuter~» (Genèse 41:32).\bigskip

Les Ecrits ont été classés dans l'ordre de la tradition juive pour le Tanakh et dans l'ordre chronologique de leur rédaction pour les épîtres afin de permettre au lecteur de mieux comprendre le contexte et le déroulement de la prophétie biblique. L'appellation «~Ancien Testament~» a été remplacée par l'acronyme hébreu Tanakh (voir sommaire). Quant à ce qu'on appelle communément le «~Nouveau Testament~», il sera désormais question du Testament de Jésus. En effet, l'Ancienne Alliance n'étant pas un testament, on ne peut donc pas parler de «~Nouveau Testament~» mais plutôt d'une Nouvelle Alliance (voir commentaires en Ex. 19:5~; Mt. 27:51~; Jn. 19:30).\bigskip

Je remercie tout d'abord le Seigneur pour son aide précieuse qu'il m'a apportée pour la révision de cette Bible, ainsi qu'à celles et ceux qui m’ont assisté dans ce travail.\newline

\begin{flushright}
Shora KUETU
\end{flushright}
\end{small}
