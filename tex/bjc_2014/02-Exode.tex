\ShortTitle{Exode}\BookTitle{Exode}\BFont
\noindent\hrulefill
{\footnotesize
\textit{
\bigskip
{\centering{}
\\Auteur : Probablement Moïse
\\(Heb. : Shemot)
\\Signification : Noms
\\Thème : La délivrance
\\Date de rédaction : Env. 1450-1410 av. J.-C.\\}
}
%\bigskip
\textit{
\\Les fils de Jacob s’étaient retrouvés en Egypte pour survivre à une famine qui avait frappé la terre entière pendant plusieurs années. Grâce à leur frère Joseph, alors gouverneur d’Egypte, ils bénéficièrent d’un bon traitement. Mais la mort de ce dernier et la montée au pouvoir d’un nouveau Pharaon - probablement Ramsès II – inaugurèrent une période de quatre siècles de souffrances pour le peuple élu.
%\bigskip
\\En effet, les Hébreux avaient été réduits en esclavage. En réponse aux cris de douleur de son peuple, Dieu suscita Moïse, dont le nom signifie « tiré de ». Ce descendant de Lévi fut élevé dans le palais de Pharaon, mais dut s’enfuir parce qu’il avait tué un Egyptien. Après quarante ans passés dans le pays de Madian, le Dieu qui s’appelle « Je suis » se révéla à Moïse sur la montagne d’Horeb et lui confia la mission d’aller délivrer son peuple du joug égyptien.
%\bigskip
\\Ce livre retrace la sortie d’Egypte et le début de la traversée du désert, jalonnée de prodiges exceptionnels.\bigskip
}
}
\par\nobreak\noindent\hrulefill
\begin{multicols}{2}
\Chap{1}
\TextTitle{Après la mort de Joseph}
\VerseOne{}Or ce sont ici les noms des fils d'Israël qui entrèrent en Egypte avec Jacob. Ils y entrèrent chacun avec sa famille : 
\VS{2}Ruben, Siméon, Lévi, et Juda,
\VS{3}Issacar, Zabulon, et Benjamin,
\VS{4}Dan, et Nephthali, Gad, et Aser.
\VS{5}Toutes les personnes issues des reins de Jacob étaient soixante-dix âmes. Joseph était alors en Egypte.
\VS{6}Joseph mourut ainsi que tous ses frères, et toute cette génération-là.
\VS{7}Les fils d'Israël foisonnèrent et crûrent en très grande abondance, et multiplièrent et devinrent très puissant, tellement que le pays en fut rempli\FTNT{De. 26:5 ; Ac. 7:17.}.
\TextTitle{Israël esclave en Egypte}
\VS{8}Depuis il s'éleva un nouveau roi sur l'Egypte, qui n'avait point connu Joseph.
\VS{9}Il dit à son peuple : Voici, le peuple des fils d'Israël est plus grand et plus puissant que nous.
\VS{10}Agissons donc prudemment avec lui, de peur qu'il ne se multiplie, et que s'il surviennait une guerre, il ne se joigne à nos ennemis, ne fasse la guerre contre nous, et qu'il ne s'en aille du pays.
\VS{11}Ils établirent donc sur le peuple des comissaires d'impôts, pour l'affliger en le surchargeant; car le peuple bâtit des villes à greniers pour  Pharaon; à savoir Pithom et Ramsès. Mais plus ils l'affligeaient et plus il  multipliait et croissait en toute abondance ; c'est pourquoi ils haïssaient les fils d'Israël\FTNT{Ps. 105:24.}.
\VS{13}Et les Egyptiens assujettirent les fils d’Israël à une rude servitude\FTNT{Ge. 15:13.}.
\VS{14}Tellement ils leur rendirent la vie amère par un rude travail, en les employant à faire du mortier, des briques, et toute sorte d'ouvrage qui se fait aux champs ; c’était avec cruauté qu’ils leur imposèrent toutes ces charges.
\VS{15}Le roi d'Egypte parla aussi aux sages-femmes des Hébreux, nommées l’une Schiphra et l’autre Pua.
\VS{16}Il leur dit : Quand vous accoucherez les femmes des Hébreux, et que vous les verrez sur les sièges, si c'est un garçon, mettez-le à mort ; mais si c'est une fille, qu'elle vive.
\VS{17}Mais les sages-femmes craignirent Dieu et ne firent pas ce que le roi d'Egypte leur avait dit ; car elles laissèrent vivre les fils.
\VS{18}Alors le roi d'Egypte appela les sages-femmes et leur dit : Pourquoi avez-vous fait cela, et avez-vous laissé vivre les fils ?
\VS{19}Les sages-femmes répondirent à Pharaon : Parce que les femmes des Hébreux ne sont point comme les femmes Egyptiennes ; car elles sont vigoureuses, elles ont accouché avant que la sage-femme ne soit arrivée chez elles.
\VS{20}Dieu fit du bien aux sages-femmes ; et le peuple multiplia et devint très puissant.
\VS{21}Parce que les sages-femmes craignirent Dieu, il leur édifia des maisons.
\VS{22}Alors Pharaon donna cet ordre à tout son peuple : Jetez dans le fleuve tous les fils qui naîtront, mais laissez vivre toutes les filles.
\Chap{2}
\TextTitle{Naissance de Moïse\FTNTT{Hé. 11:23-27}}
\VerseOne{}Un homme de la maison de Lévi s'en alla et prit une fille de Lévi\FTNT{No. 26:59.}.
\VS{2}Cette femme conçut et enfanta un fils. Voyant qu'il était beau, elle le cacha pendant trois mois\FTNT{Hé. 11:23}.
\VS{3}Mais ne pouvant le tenir caché plus longtemps, elle prit une arche de jonc, et l'enduisit de bitume et de poix, mit l'enfant dedans, et le posa parmi des roseaux sur le bord du fleuve.
\VS{4}Et La sœur de cet enfant se tenait loin pour savoir ce qu'il en arriverait.
\VS{5}La fille de Pharaon descendit à la rivière pour se baigner, et ses servantes se promenaient sur le bord de la rivière, ayant vu le coffret au milieu des roseaux, elle envoya une de ses servantes pour le prendre.
\VS{6}Et l'ayant ouvert, elle  vit l'enfant; et voici, l'enfant pleurait, et elle en fut touchée de compassion, et dit : C'est un des enfants de ces Hébreux.
\VS{7}Alors la sœur de l'enfant dit à la fille de Pharaon : Irai-je appeler une femme d'entre les Hébreues, qui allaite ? Et elle t'allaitera cet enfant.
\VS{8}La fille de Pharaon lui répondit : Va ; et la jeune fille s'en alla et appela la mère de l'enfant.
\VS{9}Et La fille de Pharaon lui dit : Emporte cet enfant, et allaite-le moi, je te donnerai ton salaire ; et la femme prit l'enfant et l'allaita.
\VS{10}Quand l'enfant fut devenu grand, elle l'amena à la fille de Pharaon ; il fut pour elle comme un fils. Elle lui donna le nom de Moïse ; parce que, dit-elle, je l'ai tiré des eaux.
\TextTitle{Moïse prend à cœur le sort d’Israël ; fuite à Madian}
\VS{11}Or il arriva, en ce temps-là, que Moïse, étant devenu grand, sortit vers ses frères, et vit leurs travaux ; il vit aussi un Egyptien qui frappait un Hébreu d'entre ses frères\FTNT{Hé. 11:24-25.}.
\VS{12}Et ayant regardé çà et là, et voyant qu'il n'y avait personne, il tua l'Egyptien, et le cacha dans le sable.
\VS{13}Il sortit encore le second jour ; et voici, deux hommes Hébreux se querellaient. Il dit à celui qui avait tort : Pourquoi frappes-tu ton prochain ?
\VS{14}Lequel répondit : Qui t'a établi prince et juge sur nous ? Veux-tu me tuer comme tu as tué l'Egyptien ? Et Moïse craignit, et dit : Certainement le fait est connu.
\VS{15}Or Pharaon ayant appris ce fait-là, chercha à faire mourir Moïse ; mais Moïse s'enfuit de devant Pharaon, s'arrêta au pays de Madian, et s'assit près d'un puits.
\VS{16}Or le sacrificateur de Madian avait sept filles qui vinrent puiser de l'eau, et elles remplirent les auges pour abreuver le troupeau de leur père.
\VS{17}Mais des bergers survinrent, et les chassèrent ; et Moïse se leva et les secourut, et abreuva leur troupeau.
\VS{18}Et quand elles furent revenues chez Réuel, leur père, il leur dit : Comment êtes-vous revenues si tôt aujourd'hui ?
\VS{19}Elles répondirent : Un homme Egyptien nous a délivrées de la main des bergers ; et même il nous a puisé abondamment de l'eau et a abreuvé le troupeau.
\VS{20}Il dit à ses filles : Où est-il ? Pourquoi avez-vous ainsi laissé cet homme ? Appelez-le, et qu'il mange du pain.
\VS{21}Et Moïse s'accorda de demeurer avec cet homme-là, qui donna Séphora, sa fille, à Moïse.
\VS{22}Et elle enfanta un fils, et il le nomma  Guerschom :  Car, dit-il, je séjourne dans un pays étranger.
\TextTitle{Yahweh entend les cris de son peuple et se souvient de lui}
\VS{23}Or il arriva longtemps après que le roi d'Egypte mourut, et les fils d'Israël soupirèrent à cause de la servitude, et ils crièrent. Et leur cri monta jusqu'à Dieu, à cause de la servitude\FTNT{No. 20:15-16.}.
\VS{24}Dieu entendit leurs gémissements, et se souvint de l'alliance qu'il avait traitée avec Abraham, Isaac et Jacob.
\VS{25}Ainsi Dieu regarda les fils d'Israël, et il fit attenton à leur état.
\Chap{3}
\TextTitle{Yahweh se révèle à Moïse dans le buisson ardent}
\VerseOne{}Moïse fut berger du troupeau de Jéthro, son beau-père, sacrificateur de Madian ; il mena le troupeau derrière le désert, et vint à la montagne de Dieu à Horeb.
\VS{2}Et l'Ange de Yahweh\FTNT{Voir commentaire en Ge. 16:7.} lui apparut dans une flamme de feu, du milieu d'un buisson. Il regarda, et voici, le buisson était tout en feu, et le buisson ne se consumait point.
\VS{3}Alors Moïse dit : Je me détournerai maintenant, et je regarderai cette grande vision, pourquoi le buisson ne se consume point.
\VS{4}Et Yahweh vit que Moïse s'était détourné pour regarder ; et Dieu l'appela du milieu du buisson, en disant : Moïse, Moïse ! Et il répondit : Me voici !
\VS{5}Et Dieu dit : N'approche point d'ici ; déchausse tes souliers de tes pieds, car le lieu où tu es arrêté est une terre sainte.
\VS{6}Il dit aussi : Je suis le Dieu de ton père, le Dieu d'Abraham, le Dieu d'Isaac, et le Dieu de Jacob\FTNT{Mt. 22:32 ; Mc. 12:26 ; Lu. 20:37 ; Ac. 7:32.} ; Moïse cacha son visage, parce qu'il craignait de regarder vers Dieu.
\VS{7} Et Yahweh dit : J'ai très bien vu l'affliction de mon peuple qui est en Egypte, et j'ai entendu le cri qu'ils ont jeté à cause de leurs oppresseurs, car je connais leurs douleurs.
\VS{8}C'est pourquoi je suis descendu pour le délivrer de la main des Egyptiens, et pour le faire remonter de ce pays-là, dans un pays bon et vaste, dans un pays découlant de lait et de miel ; au lieu où sont les Cananéens, les Héthiens, les Amoréens, les Phéréziens, les Héviens, et les Jébusiens.
\VS{9}Et maintenant, voici, le cri des fils d'Israël est parvenu à moi, et j'ai vu aussi l'oppression dont les Egyptiens les oppriment.
\VS{10}Maintenant donc viens, je t'enverrai vers Pharaon ; et tu retireras mon peuple, les fils d'Israël, hors d'Egypte\FTNT{Os. 12:14 ; Mi. 6:4.}.
\VS{11}Et Moïse répondit à Dieu : Qui suis-je, moi, pour aller vers Pharaon, et pour retirer de l'Egypte les fils d'Israël ?
\VS{12} Et Dieu lui dit : Va, car je serai avec toi ; et tu auras ce signe que c’est moi qui t’envoie : C'est que, quand tu auras retiré mon peuple d'Egypte, vous servirez Dieu près de cette montagne.
\TextTitle{Yahweh révèle son nom à Moïse}
\VS{13}Et Moïse dit à Dieu : Voici, quand je serai venu vers les fils d'Israël, et que je leur aurai dit : Le Dieu de vos pères m'a envoyé vers vous, s'ils me disent alors : Quel est son nom ? Que leur dirai-je ?
\VS{14} Et Dieu dit à Moïse : JE SUIS CELUI QUI SUIS. Il dit aussi : Tu diras ainsi aux fils d'Israël : Celui qui s'appelle JE SUIS\FTNT{Je suis  («~Ehyeh~» en hébreu), c’est de là que vient le nom de Yahweh. Or le Nom de Jésus signifie «~Yahweh est Salut~». Dieu révèle son nom à Moïse : «~Je suis celui qui suis~». Or Jésus-Christ s’est ouvertement attribué ce nom en Jn. 8:58. N’ayant compris ni le plan de Dieu ni qui était celui qui les visitait, les religieux juifs ont voulu le lapider car ils estimaient qu’il blasphémait. Car en déclarant être «~Je suis~», Jésus-Christ proclamait ouvertement sa divinité (Ro. 9:5), chose que les juifs ne pouvaient concevoir. Dans l’évangile de Jean, Jésus déclare clairement qu’il est le «~JE SUIS~» d’Ex. 3:14. «~Je suis le pain de vie~» (Jn. 6:35), «~Je suis la lumière du monde~» (Jn. 8:12), «~Je suis le bon berger~» (Jn. 10:11), «~Je suis la porte~» (Jn. 10:7), «~Je suis la résurrection~» (Jn. 11:25), «~Je suis le chemin, la vérité et la vie~» (Jn. 14:6), «~Je suis la vraie vigne~» (Jn. 15:1).}, m'a envoyé vers vous.
\VS{15}Dieu dit encore à Moïse : Tu diras ainsi aux fils d'Israël : Yahweh, le Dieu de vos pères, le Dieu d'Abraham, le Dieu d'Isaac, et le Dieu de Jacob m'a envoyé vers vous. Et c'est ici le mémorial que vous aurez de moi dans tous les âges.
\VS{16}Va, et rassemble les anciens d'Israël, et dis leur : Yahweh, le Dieu de vos pères, le Dieu d'Abraham, d'Isaac, et de Jacob, m'est apparu, en disant : Certainement je vous ai visités, et j'ai vu ce qu'on vous fait en Egypte.
\VS{17}J'ai dit : Je vous ferai remonter de l'Egypte où vous êtes affligés, dans le pays des Cananéens, des Héthiens, des Amoréens, des Phéréziens, des Héviens, et des Jébusiens, qui est un pays découlant de lait et de miel.
\VS{18}Et ils obéiront à ta parole; et tu iras, toi et les anciens d'Israël, vers le roi d'Egypte, et vous lui direz : Yahweh, le Dieu des Hébreux, est venu nous rencontrer. Maintenant donc, nous te priions que nous allions le chemin de trois jours au désert, et que nous sacrifiions à Yahweh, notre Dieu.
\VS{19}Or je sais que le roi d'Egypte ne vous permettra point de vous en aller, pas même contraint par une main puissante.
\VS{20}Mais j'étendrai ma main, et je frapperai l'Egypte par toutes les merveilles que je ferai au milieu d'elle ; et après cela, il vous laissera aller.
\VS{21}Je ferai que ce peuple trouve grâce envers les Egyptiens, et il arrivera que, quand vous partirez, vous ne vous en irez point à vide.
\VS{22}Mais chacune demandera à sa voisine, et à l'hôtesse de sa maison, des vases d'argent, des vases d'or, et des vêtements, que vous mettrez sur vos fils et sur vos filles : Ainsi vous dépouillerez les Egyptiens.
\Chap{4}
\TextTitle{Moïse résiste en évoquant l'incrédulité du peuple}
\VerseOne{}Et Moïse répondit et dit : Mais voici, ils ne me croiront point, et n'obéiront point à ma parole ; car ils diront : Yahweh ne t'est point apparu.
\VS{2}Et Yahweh lui dit : Qu'est-ce que tu as dans ta main ? Il répondit : Une verge.
\VS{3}Et Dieu lui dit : Jette-la par terre ; il la jeta par terre, et elle devint un serpent. Et Moïse s'enfuyait de devant lui.
\VS{4}Et Yahweh dit à Moïse : Etends ta main et saisis sa queue ; et il étendit sa main, et l'empoigna ; et il redevint une verge dans sa main.
\VS{5}C’est là ce que tu feras, afin qu'ils croient que Yahweh, le Dieu de leurs pères, le Dieu d'Abraham, le Dieu d'Isaac, et le Dieu de Jacob, t'est apparu.
\VS{6}Yahweh lui dit encore : Mets maintenant ta main dans ton sein, et il mit sa main dans son sein ; puis il la tira ; et voici, sa main était blanche de lèpre comme la neige.
\VS{7}Et Dieu lui dit : Remets ta main dans ton sein ; et il remit sa main dans son sein ; puis il la retira hors de son sein ; et voici, elle était redevenue comme son autre chair.
\VS{8}Mais s'il arrive qu'ils ne te croient point, et qu'ils n'obéissent point à la voix du premier signe, ils croiront à la voix du second signe.
\VS{9}S'il arrive qu'ils ne croient point à ces deux signes, et qu'ils n'obéissent point à ta parole, tu prendras de l'eau du fleuve, et tu la répandras sur la terre, et les eaux que tu auras prises du fleuve deviendront du sang sur la terre.
\TextTitle{Moïse résiste en évoquant son incapacité à parler}
\VS{10}Et Moïse répondit à Yahweh : Hélas! Seigneur ! Je ne suis point un homme qui ait ni d'hier ni d’avant-hier, la parole aisée, ni même depuis que tu parles à ton serviteur ; car j'ai la bouche et la langue empêchées.
\VS{11}Et Yahweh lui dit : Qui a fait la bouche de l'homme ? Ou qui a fait le muet, ou le sourd, ou le voyant, ou l'aveugle ? N’est-ce pas moi Yahweh\FTNT{Ps. 94:9.} ?
\VS{12}Va donc maintenant, je serai avec ta bouche, et je t'enseignerai ce que tu auras à dire\FTNT{Lu. 12:12 ; Mt. 10:19 ; Mc. 13:11.}.
\VS{13}Et Moïse répondit : Hélas ! Seigneur! Envoie, je te prie, celui que tu dois envoyer.
\VS{14}Et la colère de Yahweh s’enflamma contre Moïse, et il lui dit : Aaron, le Lévite, n'est-il pas ton frère ? Je sais qu'il parlera très bien, et même le voilà qui sort à ta rencontre, et quand il te verra, il se réjouira dans son cœur.
\VS{15}Tu lui parleras donc et tu mettras ces paroles dans sa bouche ; je serai avec ta bouche et avec la sienne, et je vous enseignerai ce que vous aurez à faire.
\VS{16}Et il parlera pour toi au peuple, et ainsi il te sera pour bouche, et tu lui seras pour Dieu.
\VS{17}Tu prendras aussi dans ta main cette verge, avec laquelle tu feras ces signes-là.
\TextTitle{Moïse accepte sa mission et part en Egypte}
\VS{18}Ainsi Moïse s'en alla, et retourna vers Jéthro, son beau-père, et lui dit : Je te prie, que je m'en aille, et que je retourne vers mes frères qui sont en Egypte, pour voir s'ils vivent encore. Et Jéthro lui dit : Va en paix.
\VS{19}Or Yahweh dit à Moïse au pays de Madian : Va, et retourne en Egypte ; car tous ceux qui cherchaient ta vie sont morts.
\VS{20}Moïse prit sa femme, et ses fils, les mit sur un âne, et retourna au pays d'Egypte. Moïse prit aussi la verge de Dieu dans sa main.
\VS{21}Et Yahweh avait dit à Moïse : Quand tu t'en iras pour retourner en Egypte, tu prendras garde à tous les miracles que j'ai mis dans ta main ; et tu les feras devant Pharaon ; mais j'endurcirai son cœur, et il ne laissera point aller le peuple.
\VS{22}Tu diras donc à Pharaon, ainsi parle Yahweh : Israël est mon fils, mon premier-né\FTNT{Os. 11:1.}.
\VS{23}Et je t'ai dit : Laisse aller mon fils, afin qu'il me serve. Mais tu as refusé de le laisser aller : Voici, je m'en vais tuer ton fils, ton premier-né.
\VS{24}Or il arriva que, comme Moïse était en chemin dans l'hotellerie Yahweh le rencontra, et chercha à le faire mourir.
\VS{25}Et Séphora prit un couteau tranchant, et coupa le prépuce de son fils, et le jeta à ses pieds, et dit : Certes, tu m'es un époux de sang !
\VS{26}Alors Yahweh se retira de lui ; et Séphora dit : Epoux de sang; à cause de la circoncision.
\TextTitle{Yahweh envoie Aaron vers Moïse}
\VS{27}Et Yahweh dit à Aaron : Va dans le désert, au-devant de Moïse. Il y alla donc, et le rencontra sur la montagne de Dieu et l’embrassa.
\VS{28}Et Moïse raconta à Aaron toutes les paroles de Yahweh qui l'avait envoyé, et tous les signes qu'il lui avait ordonné de faire.
\VS{29}Moïse donc poursuivit son chemin avec Aaron ; et ils assemblèrent tous les anciens des fils d'Israël.
\VS{30}Et Aaron rapporta toutes les paroles que Yahweh avait dites à Moïse, et il exécuta les signes aux yeux du peuple.
\VS{31}Et le peuple crut. Ils apprirent que Yahweh avait visité les fils d'Israël, qu'il avait vu leur affliction ; et ils s'inclinèrent et se prosternèrent.
\Chap{5}
\TextTitle{Pharaon s'oppose à Moïse\FTNTT{Ex.5-14}}
\VerseOne{}Après cela Moïse et Aaron se rendirent ensuite auprès de Pharaon et lui dirent : Ainsi parle Yahweh, le Dieu d'Israël : Laisse aller mon peuple, afin qu'il me célèbre une fête solennelle dans le désert.
\VS{2}Mais Pharaon dit : Qui est Yahweh pour que j'obéisse à sa voix et que je laisse aller Israël ? Je ne connais point Yahweh, et je ne laisserai point aller Israël.
\VS{3}Et ils dirent : Le Dieu des Hébreux est venu au-devant de nous. Permets-nous de faire trois journées de marche dans le désert, et que nous sacrifiions à Yahweh, notre Dieu ; de peur qu'il ne se jette sur nous par la peste ou par l'épée.
\VS{4}Et le roi d'Egypte leur dit : Moïse et Aaron, pourquoi détournez-vous le peuple de son ouvrage ? Allez maintenant à vos chrges.
\VS{5}Pharaon dit aussi : Voici, le peuple de ce pays est maintenant en grand nombre, et vous lui feriez cesser leur travail !
\VS{6}Et ce jour-là, Pharaon donna cet ordre aux oppresseurs établis sur le peuple et à ses commissaires, en disant :
\VS{7}Vous ne donnerez plus de paille à ce peuple pour faire des briques comme auparavant, mais qu'ils aillent s’amasser de la paille.
\VS{8}Néanmoins, vous leur imposerez la quantité de briques qu'ils faisaient auparavant, sans en rien diminuer ; car ils sont paresseux, et c'est pour cela qu'ils crient en disant : Allons et sacrifions à notre Dieu !
\VS{9}Que la servitude soit aggravée sur ces gens-là, et qu'ils s'occupent, et ne s'amusent plus à des paroles de mensonge.
\VS{10}Alors les oppresseurs du peuple et ses commissaires sortirent et dirent au peuple : Ainsi parle Pharaon : Je ne vous donnerai plus de paille.
\VS{11}Allez vous-mêmes et prenez de la paille où vous en trouverez ; mais il ne sera rien diminué de votre travail.
\VS{12}Alors le peuple se répandit par tout le pays d'Egypte pour ramasser du chaume au lieu de paille.
\VS{13}Et les oppresseurs les pressaient en disant : Achevez vos ouvrages, chaque jour sa tâche, comme lorsque la paille vous était fournie.
\VS{14}Même les commissaires des fils d'Israël, que les oppresseurs de Pharaon avaient établis sur eux, furent battus, et on leur dit : Pourquoi n'avez-vous point achevé votre tâche en faisant des briques hier et aujourd'hui, comme auparavant ?
\VS{15}Alors les commissaires des fils d'Israël vinrent crier à Pharaon, en disant : Pourquoi fais-tu ainsi à tes serviteurs ?
\VS{16}On ne donne point de paille à tes serviteurs, et toutefois on nous dit : Faites des briques. Et voici, tes serviteurs sont battus, et ton peuple est traité comme coupable.
\VS{17}Et il répondit : Vous êtes des paresseux, des paresseux ! C'est pourquoi vous dites : Allons, sacrifions à Yahweh !
\VS{18}Maintenant donc allez, travaillez ; car on ne vous donnera point de paille, et vous rendrez la même quantité de briques.
\VS{19}Les commissaires des fils d'Israël virent qu'ils souffraient, puisqu'on disait : Vous ne diminuerez rien de vos briques sur la tâche de chaque jour.
\VS{20}Et en sortant de chez Pharaon, ils rencontrèrent Moïse et Aaron, qui  se trouvèrent au-devant d'eux;
\VS{21}et ils leur dirent : Que Yahweh vous regarde, et en juge, vu que vous nous avez mis en mauvais odeur devant Pharaon et devant ses serviteurs, leur mettant l'épée à la main pour nous tuer.
\VS{22}Alors Moïse retourna vers Yahweh, et dit : Seigneur ! Pourquoi as-tu fait maltraiter ce peuple ? Pourquoi m'as-tu envoyé ?
\VS{23}Car depuis que je suis allé vers Pharaon pour parler en ton Nom, il a maltraité ce peuple, et tu n'as point délivré ton peuple.
\Chap{6}
\TextTitle{Yahweh fortifie Moïse et rappelle son alliance avec Israël}
\VerseOne{}Et Yahweh dit à Moïse : Tu verras maintenant ce que je ferai à Pharaon ; car il les laissera aller y étant contraint par une main puissante, étant, dis-je contraint par ma main puissante, il les chassera de son pays.
\VS{2}Dieu parla encore à Moïse et lui dit : Je suis Yahweh.
\VS{3}Je suis apparu à Abraham, à Isaac, et à Jacob, comme le Dieu Tout-Puissant, mais je n'ai point été connu d'eux par mon nom YAHWEH.
\VS{4}J'ai aussi fait cette alliance avec eux, que je leur donnerai le pays de Canaan, le pays de leurs pèlerinages, dans lequel ils ont demeuré comme étrangers.
\VS{5}Et j'ai entendu les sanglots des fils d'Israël, que les Egyptiens tiennent esclaves, et je me suis souvenu de mon alliance;
\VS{6}c’est pourquoi dis aux enfants d’Israël : je suis Yahweh, et je vous retirerai de dessous les charges des Egyptiens, et je vous délivrerai de leur servitude, je vous rachèterai à bras étendu, et par de grands jugements.
\VS{7}Et je vous prendrai pour être mon peuple, je vous serai Dieu; et vous connaîtrez que je suis Yahweh, votre Dieu, qui vous retire de dessous les charges des Égyptiens.
\VS{8}Et je vous ferai entrer dans le pays au sujet duquel j’ai levé ma main que je le donnerai à Abraham, à Isaac, et à Jacob, et je vous le donnerai en héritage ; Je suis Yahweh.
\VS{9}Moïse donc parla de cette manière aux fils d'Israël. Mais ils n'écoutèrent point Moïse, à cause de l'angoisse de  leur esprit, et à cause de leur dure servitude.
\VS{10}Et Yahweh parla à Moïse, en disant :
\VS{11}Va, et dis à Pharaon, roi d'Egypte, qu'il laisse sortir les fils d'Israël de son pays.
\VS{12}Alors Moïse parla devant Yahweh, en disant : Voici, les fils d'Israël ne m'ont point écouté, et comment Pharaon m'écoutera-t-il, moi, qui suis incirconcis des lèvres ?
\VS{13} Mais Yahweh parla à Moïse et à Aaron, et leur ordonna d'aller trouver les fils d'Israël, et Pharaon, roi d'Egypte, pour retirer les fils d'Israël du pays d'Egypte.
\TextTitle{Les chefs d'Israël}
\VS{14}Voici les chefs des pères : Les fils de Ruben, premier-né d'Israël : Hénoc et Pallu, Hetsron et Carmi ; ce sont là les familles de Ruben\FTNT{Ge. 46:9 ; No. 26:5 ; 1 Ch. 5:3.}.
\VS{15}Les fils de Siméon : Jemuel, Jamin, Ohad, Jakin et Tsochar, et Saül, fils d'une Cananéenne ; ce sont là les familles de Siméon.
\VS{16}Voici les noms des fils de Lévi selon leur naissance : Guerschon, Kehath et Merari. Les années de la vie de Lévi furent de cent trente-sept ans.
\VS{17}Les fils de Guerschon : Libni et Schimeï, selon leurs familles.
\VS{18}Les fils de Kehath : Amram, Jitsehar, Hébron, et Uziel. Et les années de la vie de Kehath furent de cent trente-trois ans.
\VS{19}Les fils de Merari : Machli et Muschi ; ce sont là les familles de Lévi selon leurs générations.
\VS{20}Or Amram prit Jokébed, sa tante, pour femme, qui lui enfanta Aaron et Moïse ; les années de la vie d’Amram furent de cent trente-sept ans.
\VS{21}Et les fils de Jitsehar : Koré, Népheg et Zicri.
\VS{22}Et les fils d’Uziel : Mischaël, Eltsaphan, et Sithri.
\VS{23}Aaron prit pour femme Elischéba, fille d’Amminadab, sœur de Nachschon, qui lui enfanta Nadab, Abihu, Eléazar, et Ithamar.
\VS{24}Et les fils de Koré : Assir, Elkana, et Abiasaph. Ce sont là les familles des Korites.
\VS{25}Eléazar, fils d'Aaron, prit pour femme une des filles de Puthiel, qui lui enfanta Phinées. Ce sont là les chefs des pères des Lévites selon leurs familles.
\VS{26}Or C'est là cet Aaron et ce Moïse à qui Yahweh dit : Retirez les fils d'Israël du pays d'Egypte selon leurs armées.
\VS{27}Ce sont eux qui parlèrent à Pharaon, roi d'Egypte, pour retirer d'Egypte les fils d'Israël. C'est ce Moïse et c'est cet Aaron.
\VS{28}Le jour où Yahweh parla à Moïse dans le pays d'Egypte,
\VS{29}Yahweh parla à Moïse et dit : Je suis Yahweh; dis à Pharaon, roi d'Egypte, toutes les paroles que je t'ai dites.
\VS{30}Et Moïse dit en présence de Yahweh : Voici, je suis incirconcis des lèvres, comment Pharaon m'écoutera-t-il ?
\Chap{7}
\TextTitle{L'appel de Moïse confirmé}
\VerseOne{}Et Yahweh dit à Moïse : Voici, je t'ai établi pour être Dieu à Pharaon, et Aaron, ton frère, sera ton prophète.
\VS{2}Tu diras tout ce que je t’ordonnerai, et Aaron, ton frère, parlera à Pharaon pour qu'il laisse aller les fils d'Israël hors de son pays.
\VS{3}J'endurcirai le cœur de Pharaon, et je multiplierai mes signes et mes miracles dans le pays d'Egypte.
\VS{4}Pharaon ne vous écoutera point ; je mettrai ma main sur l'Egypte, et je sortirai mes armées, mon peuple, les fils d'Israël, du pays d'Egypte, par de grands jugements.
\VS{5}Les Egyptiens connaîtront\FTNT{Les nations reconnaîtront que Jésus-Christ est le Dieu d’Israël lorsqu’il reviendra en Sion pour délivrer et restaurer son peuple (Za. 14).} que je suis Yahweh quand j'aurai étendu ma main sur l'Egypte, et que j'aurais retiré du milieu d'eux les fls d'Israël.
\VS{6}Et Moïse et Aaron firent comme Yahweh leur avait ordonné ; ils firent ainsi.
\VS{7}Or Moïse était âgé de quatre-vingts ans, et Aaron de quatre-vingt-trois ans quand ils parlèrent à Pharaon.
\TextTitle{La verge d'Aaron devient un serpent}
\VS{8}Yahweh parla à Moïse et à Aaron en disant :
\VS{9}Quand Pharaon vous parlera en disant : Faites un miracle ; tu diras alors à Aaron : Prends ta verge, jette-la devant Pharaon, et elle deviendra un serpent.
\VS{10}Moïse donc et Aaron allèrent auprès de Pharaon, et firent comme Yahweh avait ordonné ; Aaron jeta sa verge devant Pharaon, et devant ses serviteurs, et elle devint un serpent.
\VS{11}Mais Pharaon fit venir aussi les sages et les enchanteurs ; et les magiciens d'Egypte, et eux aussi firent autant par leurs enchantements.
\VS{12}Ils jetèrent donc chacun leurs verges, et elles devinrent des serpents ; mais la verge d'Aaron engloutit leurs verges.
\VS{13}Le cœur de Pharaon s'endurcit, et il ne les écouta point ; selon ce que Yahweh avait dit.
\TextTitle{Les eaux du fleuve changées en sang}
\VS{14}Yahweh dit à Moïse : Le cœur de Pharaon est endurci, il a refusé de laisser aller le peuple.
\VS{15}Va-t'en dès le matin vers Pharaon ; voici, il sortira pour aller près de l'eau ; tu te présenteras donc devant lui sur le bord du fleuve, et tu prendras dans ta main la verge qui a été changée en serpent.
\VS{16}Et tu lui diras : Yahweh, le Dieu des Hébreux, m'avait envoyé vers toi pour te dire : Laisse aller mon peuple, afin qu'il me serve au désert ; mais voici, tu ne m'as point écouté jusqu’ici.
\VS{17}Ainsi parle Yahweh : A ceci tu sauras que je suis Yahweh ; je m'en vais frapper de la verge qui est dans ma main les eaux du fleuve, et elles seront changées en sang.
\VS{18}Et le poisson qui est dans le fleuve mourra, le fleuve deviendra puant, et les Egyptiens travailleront beaucoup pour trouver à boire des eaux du fleuve.
\VS{19}Yahweh parla aussi à Moïse : Dis à Aaron : Prends ta verge, et étends ta main sur les eaux des Egyptiens, sur leurs rivières, sur leurs ruisseaux, et sur leurs marais, et sur tous les amas de leurs eaux, et elles deviendront du sang ; il y aura du sang par tout le pays d'Egypte, dans les vases de bois et de pierre.
\VS{20}Moïse donc et Aaron firent ce que Yahweh avait ordonné. Aaron, ayant levé la verge, en frappa les eaux du fleuve, sous les yeux de Pharaon et de ses serviteurs ; et toutes les eaux du fleuve furent changées en sang.
\VS{21}Et le poisson qui était dans le fleuve mourut, et le fleuve en devint puant, tellement que les Egyptiens ne pouvaient point boire les eaux du fleuve ; il y eut du sang dans tout le pays d'Egypte.
\VS{22}Et les magiciens d'Egypte en firent le semblable par leurs enchantements. Et le cœur de Pharaon s'endurcit tellement, qu'il ne les écouta point, selon ce que Yahweh avait dit.
\VS{23}Et Pharaon leur ayant tourné le dos, alla dans sa maison, et ne prit même pas à cœur ces choses qu'il avait vues.
\VS{24}Or tous les Egyptiens creusèrent autour du fleuve pour trouver de l'eau à boire, parce qu'ils ne pouvaient pas boire de l'eau du fleuve.
\VS{25}Il se passa sept jours depuis que Yahweh eut frappé le fleuve.
\TextTitle{Invasion de grenouilles}
\VS{26}Après cela, Yahweh dit à Moïse : Va vers Pharaon, et dis-lui : Ainsi parle Yahweh : Laisse aller mon peuple, afin qu'il me serve.
\VS{27}Si tu refuses de le laisser aller, voici, je m'en vais frapper de grenouilles toutes tes contrées;
\VS{28}et le fleuve fourmillera de grenouilles, qui monteront et entreront dans ta maison, et dans la chambre où tu couches, et sur ton lit, et dans les maisons de tes serviteurs, et parmi tout ton peuple, dans tes fours, et dans tes maies .
\VS{29}Ainsi, les grenouilles monteront sur toi, sur ton peuple, et sur tous tes serviteurs.
\Chap{8}
\VerseOne{}Yahweh donc dit à Moïse : Dis à Aaron : Etends ta main avec ta verge sur les fleuves, sur les rivières, et sur les marais, et fais monter les grenouilles sur le pays d'Egypte.
\VS{2}Et Aaron étendit sa main sur les eaux de l'Egypte, et les grenouilles montèrent et couvrirent le pays d'Egypte.
\VS{3}Mais les magiciens firent de même par leurs enchantements, et firent monter des grenouilles sur le pays d'Egypte.
\VS{4}Alors Pharaon appela Moïse et Aaron, et leur dit : Fléchissez Yahweh par vos prières, afin qu'il retire les grenouilles de dessus moi et de dessus mon peuple ; et je laisserai aller le peuple, afin qu'ils sacrifient à Yahweh.
\VS{5} Et Moïse dit à Pharaon : Glorifie-toi sur moi !Pour quel temps fléchirai-je par mes prières Yahweh pour toi, et pour tes serviteurs et pour ton peuple, afin que Yahweh retire les grenouilles loin de toi et de tes maisons? Il en demeurera seulement dans le fleuve.
\VS{6} Alors il répondit : Pour demain. Et Moïse dit : Il sera fait selon ta parole, afin que tu saches qu'il n'y a nul Dieu tel que Yahweh, notre Dieu.
\VS{7}Les grenouilles donc se retireront de toi, de tes maisons, de tes serviteurs, et de ton peuple ; il en demeurera seulement dans le fleuve.
\VS{8}Alors Moïse et Aaron sortirent de chez Pharaon ; Moïse cria à Yahweh au sujet des grenouilles qu'il avait fait venir sur Pharaon.
\VS{9} Et Yahweh fit selon la parole de Moïse. Ainsi les grenouilles moururent dans les maisons, dans les villages, et dans les champs.
\VS{10}On les amassa par monceaux, et la terre en fut infectée.
\VS{11}Mais Pharaon, voyant qu'il y avait du relâche, endurcit son cœur, et ne les écouta point, selon ce que Yahweh avait dit.
\TextTitle{Invasion de poux}
\VS{12}Et Yahweh dit à Moïse : Dis à Aaron : Etends ta verge et frappe la poussière de la terre, et elle deviendra des poux dans tout le pays d'Egypte.
\VS{13} Et ils firent ainsi: Et Aaron étendit sa main avec sa verge, et frappa la poussière de la terre ; et elle fut changée en poux, sur les hommes, et sur les bêtes ; toute la poussière du pays fut changée en poux dans tout le pays d'Egypte.
\VS{14}Et les magiciens voulurent faire de même par leurs enchantements, pour produire des poux, mais ils ne purent pas. Les poux furent donc tant sur les hommes que sur les bêtes.
\VS{15}Alors les magiciens dirent à Pharaon : C'est ici le doigt de Dieu\FTNT{Lu. 11:20.} ! Toutefois, le cœur de Pharaon s'endurcit et il ne les écouta point, selon que Yahweh avait dit.
\TextTitle{Invasion de mouches}
\VS{16}Puis Yahweh dit à Moïse : Lève-toi de bon matin, et présente-toi devant Pharaon ; voici, il sortira près de l'eau, et tu lui diras : Ainsi parle Yahweh : Laisse aller mon peuple, afin qu'il me serve.
\VS{17}Car si tu ne laisses pas aller mon peuple, voici, je m'en vais envoyer contre toi, contre tes serviteurs, contre ton peuple, et contre tes maisons, un mélange d'insectes ; et les maisons des Egyptiens seront remplies de ce mélange, et la terre aussi sur laquelle ils seront \FTNT{Ps. 105:31 ; Ps 78:43.}.
\VS{18}Mais je distinguerai ce jour-là le pays de Gosen, où se tient mon peuple, tellement qu'il n’y aura nul mélange d'insectes; afin que tu saches que je suis Yahweh au milieu de la terre.
\VS{19}Et je ferai la différence entre ton peuple et mon peuple ; demain, ce signe-là se fera.
\VS{20}Et Yahweh le fit ainsi ; et un grand mélange d'insectes entra dans la maison de Pharaon, et dans chaque maison de ses serviteurs, et tout le pays d'Egypte, de sorte que la terre fut gâtée par ce mélange.
\TextTitle{Pharaon tente de compromettre Moïse}
\VS{21} Et Pharaon appela Moïse et Aaron, et leur dit : Allez,  sacrifiez à votre Dieu dans ce pays.
\VS{22}Mais Moïse dit : Il n'est pas convenable de faire ainsi ; car nous sacrifierions à Yahweh, notre Dieu, l'abomination des Egyptiens. Voici, si nous sacrifions l'abomination des Egyptiens devant leurs yeux, ne nous lapideraient-ils pas ?
\VS{23}Nous irons le chemin de trois jours au désert, et nous sacrifierons à Yahweh, notre Dieu, comme il nous dira.
\VS{24}Alors Pharaon dit : Je vous laisserai aller pour sacrifier dans le désert à Yahweh, votre Dieu ; toutefois, vous ne vous éloignerez pas en y allant. Fléchissez Yahweh pour moi, par vos prières.
\VS{25}Moïse dit : Voici, je sors de chez toi, et jefléchirai Yahweh,par prières, afin que le mélange d'insectes se retire demain de Pharaon, de ses serviteurs, et de son peuple. Mais que Pharaon ne continue point à se moquer en ne laissant point aller le peuple pour sacrifier à Yahweh.
\VS{26}Alors Moïse sortit de chez Pharaon et fléchit Yahweh par la prière.
\VS{27}Et Yahweh fit selon la parole de Moïse ; et le mélange d'insectes se retira de Pharaon, de ses serviteurs, et de son peuple ; il n’en resta pas un seul insecte.
\VS{28}Mais Pharaon endurcit son cœur cette fois encore, et ne laissa point aller le peuple.
\Chap{9}
\TextTitle{La mort des troupeaux}
\VerseOne{}Alors Yahweh dit à Moïse : Va vers Pharaon, et dis-lui : Ainsi parle Yahweh, le Dieu des Hébreux : Laisse aller mon peuple, afin qu'il me serve.
\VS{2}Car si tu refuses de les laisser aller, et si tu le retiens encore,
\VS{3}voici, la main de Yahweh sera sur ton bétail qui est dans les champs, tant sur les chevaux que sur les ânes, sur les chameaux, sur les bœufs, et sur les brebis, et il y aura une très grande mortalité.
\VS{4}Et Yahweh distinguera le bétail des Israélites du bétail des Egyptiens, afin que rien de ce qui est aux fils d'Israël ne meure.
\VS{5}Et Yahweh fixa un temps, en disant : Demain, Yahweh fera ceci dans le pays.
\VS{6}Yahweh donc fit cela dès le lendemain ; et tout le bétail des Egyptiens mourut; mais du bétail des fils d'Israël, il ne mourut pas une seule bête.
\VS{7}Et Pharaon envoya examiner, et voici, il n'y avait pas une seule bête morte du bétail des fils d'Israël. Toutefois, le cœur de Pharaon s'endurcit, et il ne laissa point aller le peuple.
\TextTitle{Des ulcères sur les Egyptiens et les bêtes}
\VS{8}Alors Yahweh dit à Moïse et à Aaron : Remplissez vos mains de cendre de fournaise ; et que Moïse les répande vers les cieux en la présence de Pharaon.
\VS{9}Et elles deviendront de la poussière sur tout le pays d'Egypte, et il s'en fera des ulcères bourgeonnant en pustules tant sur les hommes que sur les bêtes, dans tout le pays d'Egypte.
\VS{10}Ils prirent donc de la cendre de fournaise et se tinrent devant Pharaon ; Moïse la répandit vers les cieux, et il se forma des ulcères bourgeonnant en pustules tant sur les hommes que sur les bêtes.
\VS{11} Et les magiciens ne purent se tenir devant Moïse, à cause des ulcères ; car les magiciens avaient des ulcères, comme tous les Egyptiens.
\VS{12} Et Yahweh endurcit le cœur de Pharaon, et il ne les écouta point selon ce que Yahweh avait dit à Moïse.
\TextTitle{L’Egypte frappée par la grêle et le feu}
\VS{13}Puis Yahweh parla à Moïse : Lève-toi de bon matin, et présente-toi devant Pharaon, et dis-lui : Ainsi parle Yahweh, le Dieu des Hébreux : Laisse aller mon peuple, afin qu'il me serve.
\VS{14}Car cette fois, je vais faire venir toutes mes plaies contre ton cœur, sur tes serviteurs, et sur ton peuple, afin que tu saches qu'il n'y a nul Dieu semblable à moi sur toute la terre.
\VS{15}Car maintenant si j'avais étendu ma main, je t'aurais frappé de la peste, toi et ton peuple, et tu serais effacé de la terre.
\VS{16}Mais certainement, je t'ai fait subsister pour te faire voir ma puissance, afin que mon Nom soit célébré sur toute la terre\FTNT{Ro. 9:17.}.
\VS{17}T'élèves-tu encore contre mon peuple, pour ne point le laisser aller ?
\VS{18}Voici, je m'en vais faire pleuvoir demain à cette même heure une grêle tellement forte qu’il n'y en a point eu de semblable en Egypte, depuis le jour où elle fut fondée jusqu’à maintenant.
\VS{19}Maintenant envoie rassembler ton bétail, et tout ce que tu as à la campagne ; car la grêle tombera sur tous les hommes, sur le bétail qui se trouvera à la campagne, et qui n’aura pas renfermé, et ils mourront.
\VS{20}Ceux des serviteurs de Pharaon, qui craignirent la parole de Yahweh, firent promptement retirer dans les maisons ses serviteurs et ses bêtes.
\VS{21}Mais celui qui n'appliqua point son cœur à la parole de Yahweh, laissa ses serviteurs et ses bêtes à la campagne.
\VS{22}Et Yahweh dit à Moïse : Etends ta main vers les cieux, et il y aura de la grêle sur tout le pays d'Egypte, sur les hommes et sur les bêtes, et sur toutes les herbes des champs au pays d'Egypte.
\VS{23}Moïse donc étendit sa verge vers les cieux, et Yahweh envoya des tonnerres et de la grêle, et le feu se promenait sur la terre. Yahweh fit pleuvoir de la grêle sur le pays d'Egypte.
\VS{24}Il y eut donc de la grêle et du feu entremêlé avec la grêle, laquelle était si grosse qu'il n'y en avait point eu de semblable sur toute la terre d'Egypte, depuis qu'elle a été habitée.
\VS{25}La grêle frappa dans tout le pays d'Egypte tout ce qui était aux champs, depuis les hommes jusqu'aux bêtes. La grêle frappa aussi toutes les herbes des champs, et brisa tous les arbres des champs.
\VS{26}Il n'y eut que la contrée de Gosen, dans laquelle étaient les fils d'Israël, où il n'y eut point de grêle.
\TextTitle{Pharaon continue d’endurcir son cœur }
\VS{27}Alors Pharaon envoya appeler Moïse et Aaron, et leur dit : J'ai péché cette fois ; Yahweh est juste, mais moi et mon peuple sommes méchants.
\VS{28}Fléchissez  par des prières Yahweh : Que ce soit assez, et que Dieu ne fasse plus tonner ni grêler, car je vous laisserai aller, et on ne vous arrêtera plus.
\VS{29}Alors Moïse dit : Aussitôt que je sortirai de la ville, j'étendrai mes mains vers Yahweh et les tonnerres cesseront. Il n'y aura plus de grêle, afin que tu saches que la terre est à Yahweh\FTNT{Ps. 24:1.}.
\VS{30}Mais quant à toi et tes serviteurs, je sais que vous ne craindrez pas encore Yahweh Dieu.
\VS{31}Or le lin et l'orge avaient été frappés, car l'orge était en épis et c’était la floraison du lin.
\VS{32} Mais le blé et l'épeautre ne furent point frappés, parce qu'ils sont tardifs.
\VS{33}Moïse donc sortit de chez Pharaon pour aller hors de la ville. Il étendit ses mains vers Yahweh, et les tonnerres cessèrent, et la grêle et la pluie ne tombèrent plus sur la terre.
\VS{34}Pharaon, voyant que la pluie, la grêle, et les tonnerres avaient cessé, continua encore à pécher, et il endurcit son cœur, lui et ses serviteurs.
\VS{35}Le cœur donc de Pharaon s'endurcit, et il ne laissa point aller les fils d'Israël, selon ce que Yahweh avait dit par l’intermédiaire de Moïse.
\Chap{10}
\TextTitle{Invasion de sauterelles}
\VerseOne{} Et Yahweh dit à Moïse : Va vers Pharaon, car j'ai endurci son cœur et le cœur de ses serviteurs, afin que je mette au-dedans de lui les signes que je m'en vais faire;
\VS{2}et afin que tu racontes à ton fils et au fils de ton fils, les signes que j’accomplirai sur les Egyptiens et les prodiges que je ferai au milieu d'eux, et que vous sachiez que je suis Yahweh.
\VS{3}Moïse donc et Aaron vinrent  vers Pharaon, et lui dirent : Ainsi parle Yahweh, le Dieu des Hébreux : Jusqu'à quand refuseras-tu de t'humilier devant moi ? Laisse aller mon peuple, afin qu'il me serve.
\VS{4}Car si tu refuses de laisser aller mon peuple, voici, je ferai venir demain des sauterelles dans ton territoire.
\VS{5}Elles couvriront la face de la terre, et l'on ne pourra plus voir la terre ; elles dévoreront le reste de ce qui a échappé, ce que la grêle vous a laissé ; et elles dévoreront tous les arbres qui poussent dans vos champs.
\VS{6}Et elles rempliront tes maisons, et les maisons de tous tes serviteurs, et les maisons de tous les Egyptiens ; ce que tes pères n'ont point vu ni les pères de tes pères, depuis qu’ils existent sur la terre jusqu'à ce jour. Puis, ayant tourné le dos à Pharaon, il sortit d'auprès de lui.
\VS{7}Et les serviteurs de Pharaon lui dirent : Jusqu'à quand celui-ci nous sera-t-il un piège ? Laisse aller ces gens, et qu'ils servent Yahweh, leur Dieu. Attendras-tu de savoir avant cela que l'Egypte est perdue ?
\VS{8}Alors on fit revenir Moïse et Aaron vers Pharaon, il leur dit : Allez, servez Yahweh, votre Dieu. Qui sont tous ceux qui iront ?
\VS{9} Et Moïse répondit : Nous irons avec nos jeunes gens et nos vieillards, avec nos fils et nos filles ; nous irons avec nos brebis et nos bœufs ; car nous avons à célébrer une fête solennelle à Yahweh.
\VS{10}Alors il leur dit : Que Yahweh soit avec vous, comme je laisserai aller vos petits enfants ! Prenez garde, car le mal est devant vous.
\VS{11}Il n'en sera donc pas ainsi que vous l'avez demandé; mais vous, hommes, allez maitenant et servez Yahweh ; car c'est ce que vous demandiez. Et on les chassa de la présence de Pharaon.
\VS{12}Alors Yahweh dit à Moïse : Etends ta main sur le pays d'Egypte, pour faire venir les sauterelles, afin qu'elles montent sur le pays d'Égypte, qu'elles dévorent toute l'herbe de la terre, tout ce que la grêle a laissé.
\VS{13}Moïse étendit donc sa verge sur le pays d'Egypte ; et Yahweh amena sur le pays, tout ce jour-là et toute la nuit, un vent d'orient ; le matin vint, et le vent d'orient enleva les sauterelles.
\VS{14}Et il fit monter les sauterelles sur tout le pays d'Egypte, et les mit dans toutes les contrées d'Egypte; elles étaient fort grosses et il y n'en  avait point eu avant elles de semblables, et il n'y en aura point de semblables après elles.
\VS{15}Et elles couvrirent la face de tout le pays, tellement que  le pays en fut obscurci ; elles dévorèrent toute l'herbe de la terre, tout le fruit des arbres que la grêle avait laissé ; il ne resta aucune verdure aux arbres ni aux herbes des champs, dans tout le pays d'Egypte.
\VS{16}Aussitôt Pharaon se hâta d'appeler Moïse et Aaron, et dit : J'ai péché contre Yahweh, votre Dieu, et contre vous.
\VS{17}Mais pardonne, je te prie, mon péché, pour cette fois seulement ; et fléchissez Yahweh, votre Dieu, par vos prières, afin  qu'il retire de moi cette mort-ci seulement.
\VS{18}Il sortit donc de chez Pharaon, et fléchit Yahweh par ses prières.
\VS{19}Et Yahweh fit lever un vent d'occident très fort qui enleva les sauterelles et les précipita dans la Mer Rouge. Il ne resta pas une seule sauterelle dans tout le territoire de l'Egypte.
\VS{20}Mais Yahweh endurcit le cœur de Pharaon, et il ne laissa point aller les fils d'Israël.
\TextTitle{Les ténèbres sur les Egyptiens}
\VS{21}Puis Yahweh dit à Moïse : Etends ta main vers les cieux, qu'il y ait sur le pays d'Egypte des ténèbres si épaisses,  qu'on puisse les toucher à la main.
\VS{22}Moïse étendit donc sa main vers les cieux, et il y eut d'épaisses ténèbres dans tout le pays d'Egypte, pendant trois jours\FTNT{Ps. 105:28.}.
\VS{23}On ne se voyait pas l'un l'autre, et nul ne se leva de sa place pendant trois jours. Mais pour tous les fils d'Israël, il y eut de la lumière dans le lieu de leurs demeures.
\TextTitle{Pharaon tente encore de compromettre Moïse}
\VS{24}Alors Pharaon appela Moïse et dit : Allez, servez Yahweh ; que vos brebis et vos bœufs seuls demeurent ; vos petits enfants iront aussi avec vous.
\VS{25}Moïse répondit : Tu mettras toi-même entre nos mains de quoi faire des sacrifices et des holocaustes, que nous ferons à Yahweh, notre Dieu.
\VS{26}Et même, nos troupeaux viendront aussi avec nous, il n'en restera pas un sabot. Car nous en prendrons pour servir Yahweh, notre Dieu ; car nous ne savons pas ce que nous choisirons pour offrir à Yahweh, jusqu’à ce que nous soyons arrivés en ce lieu là.
\VS{27}Mais Yahweh endurcit le cœur de Pharaon, et il ne voulut point les laisser aller.
\VS{28}Et Pharaon lui dit :Va-t-en !Arrière de moi ! Garde-toi de revoir ma face, car le jour où tu verras ma face, tu mourras.
\VS{29}Alors Moïse répondit : Tu as bien dit, je ne reverrai plus ta face\FTNT{Hé. 11:27.}.
\Chap{11}
\TextTitle{Pharaon méprise l’avertissement sur la mort des premiers-nés}
\VerseOne{}Or Yahweh dit à Moïse : Je ferai venir encore une plaie sur Pharaon, et sur l'Egypte, et après cela il vous laissera aller d'ici; il vous laissera entièrement aller, et vous chassera tout à fait.
\VS{2}Parle maintenant au peuple l'entendant, et leur dis : Que chacun demande à son voisin, et chacune à sa voisine, des vases d'argent et des vases d'or.
\VS{3}Or Yahweh fit trouver grâce au peuple devant les Egyptiens ; et même Moïse pasait pour un grand homme dans le pays d'Egypte, tant parmi les serviteurs de Pharaon que parmi le peuple.
\VS{4}Et Moïse dit : Ainsi parle Yahweh : Vers le milieu de la nuit, je passerai au travers de l'Egypte;
\VS{5}et tout premier-né mourra dans le pays d'Egypte, depuis le premier-né de Pharaon, qui devait être assis sur son trône, jusqu'au premier-né de la servante qui est derrière la meule, et jusqu’à tous les premiers-nés des bêtes.
\VS{6}Et il y aura un grand cri dans tout le pays d'Egypte, tel qu'il n'y en a jamais eu et qu’il n'y en aura jamais de semblable.
\VS{7}Mais contre tous les fils d'Israël, un chien même ne remuera point sa langue, depuis l'homme jusqu’aux bêtes ; afin que vous sachiez que Dieu fera la différence entre les Egyptiens et les Israélites.
\VS{8}Et tous tes serviteurs viendront vers moi, et se prosterneront devant moi, en disant : Sors, toi, et tout le peuple qui est avec toi. Après cela, je sortirai. Ainsi, Moïse sortit de chez Pharaon dans une ardente colère.
\VS{9}Yahweh donc dit à Moïse : Pharaon ne vous écoutera point, afin que mes miracles soient multipliés dans le pays d'Egypte.
\VS{10}Et Moïse et Aaron firent tous ces miracles-là devant Pharaon. Et Yahweh endurcit le cœur de Pharaon, tellement qu'il ne laissa point aller les fils d'Israël hors de son pays.
\Chap{12}
\TextTitle{La première Pâque}
\VerseOne{}Or Yahweh dit à Moïse et à Aaron dans le pays d'Egypte :
\VS{2}Ce mois-ci sera pour vous le premier des mois, il sera pour vous le premier des mois de l'année.
\VS{3}Parlez à toute l'assemblée d'Israël, en disant : Jusqu'au dixième jour de ce mois, que chacun prenne un petit d'entre les brebis ou d'entre les chèvres, selon les familles des pères; un petit,dis-je, d'entre les brebis ou d'entre les chèvres, par famille.
\VS{4}Mais si la famille est moindre qu'il ne faut pour manger un petit d'entr les brebis ou d'entre les chèvres, qu'elle le prenne avec son voisin qui est près de sa maison, selon le nombre de personnes ; vous compterez combien il en faudra pour manger d'entre les brebis ou d'entre les chèvres, ayant égard à  ce que chacun de vous peut manger.
\VS{5}Or le petit d'entre les brebis ou d'entre les chèvres  sera  sans défaut, et sera un mâle ayant un an\FTNT{La Pâque juive était célébrée le 14ème jour du premier mois de l’année juive soit le 14 du mois de Nissan (Ex. 12:2 ; No. 9:1-5).  L’agneau pascal était une préfiguration de Jésus-Christ, l’agneau de Dieu qui ôte le péché du monde (Jn. 1:29). Ses caractéristiques sont les suivantes : 
- L’agneau devait nécessairement être un mâle sans défaut (Ex. 12:5). Jésus est l’enfant mâle mis au monde par une vierge, il n’a pas été affecté par le sang corrompu d’Adam, il est donc sans défaut (Es. 7:14 ; Mt. 1:20-21). Pour être certains de la perfection de l’animal, les hébreux  devaient l’examiner pendant quatre jours avant de l’immoler (Ex. 12:3-6). Il est à noter que la loi juive exigeait que deux ou trois témoins soient présents pour constater un crime ou un péché (De. 17:6 ; De. 19:15). Ces quatre jours font donc office de quatre témoins pour attester de la pureté de l’animal. De même, les quatre auteurs de l’évangile attestent la sainteté du Seigneur. De plus, avant sa mise à mort, le Seigneur a été examiné par deux législations : juive (le sanhédrin) et  romaine (Ponce Pilate). Ces deux législations attestèrent, malgré elles, son innocence (Mt. 25:60 ;  Mt. 27:24 ; Mc. 14:55-56 ; Mc. 15:14 ; Lu. 23:4 ; Jn. 18:31 ; Jn. 19:6) et confirmèrent qu’il était sans défaut et donc digne d’être offert en sacrifice.
- Yahweh avait prescrit aux hébreux d’immoler l’agneau entre les deux soirs (Ex. 12:6), c’est-à-dire avant le crépuscule, entre  la neuvième et la onzième heure. Jésus fut arrêté la nuit de Pâque (Mc. 14:12-41). Sa crucifixion eut lieu le lendemain, à  la troisième heure (Mc. 15:25), et sa mort survint à la neuvième heure (Mt. 27:45). L’agneau devait être rôti au feu puis consommé avec du pain sans levain et des herbes amères (Ex. 12:8). Le feu symbolise le jugement que le Seigneur a pris sur lui à cause de nos péchés (Es. 53:5 ; Ro. 4:25 ; 1 Pi. 1:18-20). Le pain sans levain est une autre image de Jésus, le pain de vie (Jn. 6:35) net de tout péché (1 Co. 5:8). Les herbes amères préfigurent quant à elles l’affliction et la souffrance du Seigneur (Hé. 2:10).} ; vous le prendrez d'entre les brebis ou d'entre les chèvres;
\VS{6}Et vous le garderez jusqu'au quatorzième jour de ce mois; et toute la congrégation de l'assemblée d'Israël l’égorgera entre les deux soirs.
\VS{7} Et ils prendront de son sang, et le mettront sur les deux poteaux et sur le linteau de la porte des maisons où ils le mangeront.
\VS{8} Et ils en mangeront la chair rôtie au feu cette nuit-là ; et ils la mangeront avec des pains sans levain, et avec des herbes amères.
\VS{9}N'en mangez rien à demi cuit, ni qui ait été bouilli dans l'eau; mais qu'il soit rôti au feu, sa tête, ses jambes, et ses entrailles.
\VS{10} Et ne laissez aucun reste jusqu’au matin, mais  s'il en reste quelque chose le matin, vous le brûlerez au feu.
\VS{11}Et vous le mangerez ainsi : Vos reins seront ceints, vous aurez vos souliers à vos pieds, et votre bâton à la main, et vous le mangerez à la hâte. C'est la Pâque de Yahweh.
\TextTitle{Le sang qui sauve ; l’instauration de la fête de la Pâque}
\VS{12}Car je passerai cette nuit-là par le pays d'Egypte, et je frapperai tout premier-né au pays d'Egypte, depuis les hommes jusqu’aux bêtes; et j'exercerai des jugements sur tous les dieux de l'Egypte. Je suis Yahweh.
\VS{13}Et le sang sera pour vous un signe sur les maisons où vous serez ; car je verrai le sang et je passerai par-dessus vous, et il n'y aura point de plaie à destruction quand je frapperai le pays d'Egypte.
\VS{14}Et ce jour là, vous conserverez le souvenir de ce jour, et vous le célébrerez comme une fête solennelle à Yahweh ; vous le célébrerez comme une fête solennelle par une ordonance perpétuelle de génération en génération.
\VS{15}Vous mangerez pendant sept jours des pains sans levain, et dès le premier jour, vous ôterez le levain de vos maisons ; car quiconque mangera du pain levé, depuis le premier jour jusqu’au septième, cette personne-là sera retranchée d'Israël.
\VS{16}Au premier jour il y aura une sainte convocation, et il y aura de même au septième jour une sainte convocation ; il ne se fera aucune œuvre dans ces jours-là ; seulement, on vous apprêtera à manger ce qu'il faudra pour chaque personne.
\VS{17}Vous prendrez donc garde aux pains sans levain, parce qu'en ce même jour j'aurai retiré vos armées du pays d'Egypte ; vous observerez donc ce jour-là de génération en génération par une ordonance perpétuelle.
\VS{18}Au premier mois, le quatorzième jour du mois, au soir, vous mangerez des pains sans levain jusqu'au vingt et unième jour du mois, au soir.
\VS{19}Il ne se trouvera point de levain dans vos maisons pendant sept jours, car quiconque mangera du pain levé, cette personne-là sera retranchée de l’assemblée d'Israël, tant celui qui habite comme étranger que celui qui est né au pays.
\VS{20}Vous ne mangerez point de pain levé; mais vous mangerez dans tous les lieux où vous demeurerez des pains sans levain.
\VS{21}Moïse donc appela tous les anciens d'Israël et leur dit : Choisissez et prenez un petit d'entre les brebis pu d'entre les chèvres selon vos familles, et égorgez la Pâque.
\VS{22}Puis vous prendrez un bouquet d'hysope et le tremperez dans le sang qui sera dans un bassin, et vous arroserez du sang qui sera dans le bassin, le linteau et les deux poteaux ; et nul de vous ne sortira de la porte de sa maison jusqu'au matin.
\VS{23}Car Yahweh passera pour frapper l'Egypte et il verra le sang sur le linteau et sur les deux poteaux, et Yahweh passera par-dessus la porte, et ne permettra point que le destructeur entre dans vos maisons pour frapper.
\VS{24}Vous garderez ceci comme une ordonance perpétuelle pour toi et pour tes fils.
\VS{25}Quand donc vous serez entrés dans le pays que Yahweh vous donnera, selon qu'il en a parlé, vous observerez ce service.
\VS{26}Et quand vos fils vous diront : Que signifie pour vous ce service?
\VS{27}Alors vous répondrez : C'est le sacrifice de la Pâque à Yahweh, qui passa en Egypte par-dessus les maisons des fils d'Israël, quand il frappa l'Egypte, et qu'il préserva nos maisons. Alors le peuple s'inclina et se prosterna.
\VS{28}Ainsi les fils d'Israël s'en allèrent et firent comme Yahweh l’ordonna à Moïse et à Aaron, ils le firent ainsi.
\TextTitle{Les premiers-nés d'Egypte frappés}
\VS{29}Et il arriva qu'à minuit Yahweh frappa tous les premiers-nés du pays d'Egypte, depuis le premier-né de Pharaon, qui devait être assis sur son trône, jusqu'aux premiers-nés des captifs qui étaient dans la prison, et tous les premiers-nés des bêtes.
\VS{30}Et Pharaon se leva de nuit, lui et ses serviteurs, et tous les Egyptiens ; et il y eut un grand cri en Egypte, parce qu'il n'y avait point de maison où il n'y ait eu un mort\FTNT{Hé. 11:28 ; No. 8:17 ; Ps. 78:51 ; Ps. 105:36.}.
\TextTitle{Israël sort d'Egypte}
\VS{31}Il appela donc Moïse et Aaron de nuit, et leur dit : Levez-vous, sortez du milieu de mon peuple, tant vous que les fils d'Israël, allez et servez Yahweh, comme vousen avez parlé.
\VS{32}Prenez aussi votre menu et gros bétail, comme vous en avez parlé, et allez-vous en et bénissez-moi.
\VS{33}Et les Egyptiens pressaient le peuple et se hâtaient de les faire sortir du pays, car ils disaient : Nous sommes tous morts.
\VS{34}Le peuple donc prit sa pâte avant qu'elle fût levée, ayant leurs maies liées avec leurs vêtements, sur leurs épaules.
\VS{35}Or les fils d'Israël firent selon la parole de Moïse, et demandèrent aux Egyptiens des vases d'argent et d'or, et des vêtements.
\VS{36}Et Yahweh fit trouver grâce au peuple auprès des Egyptiens, qui les leur prêtèrent; de sorte qu'ils dépouillèrent les Egyptiens.
\VS{37}Ainsi, les fils d'Israël étant partis de Ramsès, vinrent à Succoth, environ six cent mille hommes de pied, sans les enfants.
\VS{38}Il s'en alla aussi avec eux un grand nombre de toutes sortes de gens ; et du menu et du gros bétail, en fort grands troupeaux.
\VS{39} Or parce qu'ils avaient été chassés d'Egypte, et qu'ils n'avaient pas pu tarder plus longtemps, et que même ils n'avaient fait aucune provision, ils  cuisirent par gâteaux sans levain, la pâte qu’ils avaient emportée d’Egypte; car ils ne l'avaient point fait lever. 
\VS{40}Or le séjour des fils d'Israël en Egypte fut de quatre cent trente ans\FTNT{Ge. 15:13 ; Ac. 7:6 ; Ga. 3:17.}.
\VS{41}Il arriva donc au bout de quatre cent trente ans, il arriva dis-je, en ce propre jour-là, que toutes les armées de Yahweh sortirent du pays d'Egypte.
\VS{42}C'est la nuit qui doit être soigneusement observée en l'honneur de Yahweh, parce qu'alors il les retira du pays d'Egypte ; cette nuit-là est à observer en l'honneur de Yahweh, par tous les fils d'Israël de génération en génération\FTNT{De. 16:1-6.}.
\VS{43}Yahweh dit aussi à Moïse et à Aaron : C'est ici l'ordonnance de la Pâque : Aucun étranger n'en mangera;
\VS{44}mais tout esclave qu'on aura acheté par argent sera circoncis, et alors il en mangera.
\VS{45}L'étranger et le mercenaire n'en mangeront point.
\VS{46}On la mangera dans une même maison, et vous n'emporterez point de sa chair hors de la maison, et vous n'en casserez point les os.
\VS{47}Toute l'assemblée d'Israël la fera.
\VS{48}Et si quelque étranger qui habite chez toi veut faire la Pâque à Yahweh, que tout mâle qui lui appartient soit circoncis; et alors il s'approchera pour la faire, et il sera comme celui qui est né dans le pays ; mais aucun incirconcis n'en mangera.
\VS{49}Il y aura une même loi pour celui qui est né dans le pays et pour l'étranger qui habite parmi vous.
\VS{50}Tous les fils d'Israël firent ce que Yahweh avait ordonné à Moïse et à Aaron ; ils le firent ainsi.
\VS{51}Il arriva donc en même jour que Yahweh retira les fils d'Israël du pays d'Egypte, selon leurs armées.
\Chap{13}
\TextTitle{Consécration des premiers-nés à Yahweh}
\VerseOne{}Et Yahweh parla à Moïse et dit :
\VS{2}Sanctifie-moi tout premier-né, tout premier-né issu du sein maternel parmi les fils d'Israël, tant des hommes que des bêtes, car il est à moi\FTNT{Lé. 27:26-27 ; No. 3:13 ; No. 8:17 ; Lu. 2:22-23.}.
\VS{3}Moïse donc dit au peuple : Souvenez-vous de ce jour où vous êtes sortis d'Egypte, de la maison de servitude ; car Yahweh vous en a rrtirés par sa main puissante; on ne mangera donc point de pain levé.
\VS{4}Vous sortez aujourd'hui dans le mois où les épis mûrissent.
\VS{5}Quand donc Yahweh t'aura introduit dans le pays des Cananéens, des Héthiens, des Amoréens, des Héviens et des Jébusiens, qu’il a juré à tes pères de te donner, et qui est un pays découlant de lait et de miel, alors tu feras ce service durant ce mois-ci.
\VS{6}Pendant sept jours tu mangeras des pains sans levain, et au septième jour il y aura une fête solennelle à Yahweh.
\VS{7}On mangera durant sept jours des pains sans levain ; il ne sera point vu chez toi de pain levé et même il ne sera point vu de levain dans toutes tes contrées.
\VS{8}Et ce jour-là, tu feras entendre ces choses à tes fils, en disant : C'est à cause de ce que Yahweh m'a fait  en me retirant d'Egypte.
\VS{9}Et ceci te sera pour signe sur ta main, et comme un rappel entre tes yeux, afin que la loi de Yahweh soit dans ta bouche, car Yahweh t'aura retiré d'Egypte par sa main puissante\FTNT{De. 6:8 ; De. 11:18.}.
\VS{10}Tu observeras cette ordonnance au jour fixé d’année en année.
\VS{11}Aussi, quand Yahweh t'aura introduit dans le pays des Cananéens, selon qu'il a juré à toi et à tes pères, et qu'il te l'aura donné,
\VS{12}tu consacreras à Yahweh tout premier-né issu du sein de sa mère, même tout premier-né des animaux que tu auras : Les mâles appartiendront à Yahweh.
\VS{13}Et tu rachèteras avec un petit d'entre les brebis ou d'entre les chèvres, tout premier-né de l’ânesse, et si tu ne le rachètes point, tu lui briseras la nuque. Tu rachèteras aussi tout premier-né des hommes parmi tes fils.
\VS{14}Et quand ton fils t'interrogera à l'avenir, en disant : Que veut dire ceci ? Alors tu lui diras : Yahweh nous a retirés par main forte hors d'Egypte, de la maison de servitude.
\VS{15}Car il arriva que, quand Pharaon s’obstinait à ne point nous laisser aller, Yahweh tua tous les premiers-nés au pays d'Egypte, depuis les premiers-nés des hommes jusqu’aux premiers-nés des bêtes. Voilà pourquoi je sacrifie à Yahweh tout premier-né mâle issu du sein de sa mère, et je rachète tout premier-né de mes fils.
\VS{16}Ceci te sera donc pour signe sur ta main, et pour  fronteaux entre tes yeux, que Yahweh nous a retirés d'Egypte par main puissante.
\TextTitle{Début du voyage, Yahweh dirige son peuple}
\VS{17}Or lorsque Pharaon laissa aller le peuple, Dieu ne les conduisit point par le chemin du pays des Philistins, bien qu'il fût le plus court ; car Dieu dit : C'est afin q'il n'arrive que le peuple  se repente quand il verra la guerre, et qu'il ne retourne en Egypte.
\VS{18}Mais Dieu fit tourner le peuple par le chemin du désert, vers la Mer Rouge. Ainsi, les fils d'Israël montèrent en armes hors du pays d'Egypte.
\VS{19}Et Moïse avait pris avec lui les ossements de Joseph, parce que Joseph avait expressément fait jurer les fils d'Israël, en leur disant : Dieu vous visitera très certainement, et vous transporterez donc avec vous mes ossements d'ici\FTNT{Ge. 50:25 ; Jos. 24:32.}.
\VS{20}Et ils partirent de Succoth, et campèrent à Etham, qui est à l’extrémité du désert.
\VS{21}Et Yahweh allait devant eux, de jour dans une colonne de nuée pour les conduire par le chemin ; et de nuit dans une colonne de feu pour les éclairer, afin qu'ils marchent jour et nuit\FTNT{No. 9:13-23 ; No. 10:43 ; De. 1:33 ; Né. 9:12-19 ; 1 Co. 10:1.}.
\VS{22}Et il ne retira point la colonne de nuée le jour ni la colonne de feu la nuit de devant le peuple.
\Chap{14}
\TextTitle{Pharaon et son armée à la poursuite d’Israël}
\VerseOne{}Et Yahweh parla à Moïse, et dit :
\VS{2}Parle aux fils d'Israël et dis-leur: Qu'ils se détournent, et qu'ils campent devant Pi-Hahiroth, entre Migdol et la mer, vis-à-vis de Baal-Tsephon. Vous camperez vis-à-vis de ce lieu-là près de la mer\FTNT{No. 33:7.}.
\VS{3}Pharaon dira des fils d'Israël : Ils sont confus dans le pays, le désert les a enfermés.
\VS{4}Et j'endurcirai le cœur de Pharaon, et il vous poursuivra. Ainsi je serai glorifié en Pharaon et en toute son armée , et les Egyptiens sauront que je suis Yahweh ; et ils firent ainsi.
\VS{5}Or on avait rapporté au roi d'Egypte que le peuple s'enfuyait, et le cœur de Pharaon et de ses serviteurs fut changé à l'égard du peuple, et ils dirent : Qu'est-ce que nous avons fait en laissant aller Israël, de sorte qu'il ne nous servira plus ?
\VS{6}Alors il fit atteler son char, et il prit son peuple avec lui.
\VS{7}Il prit donc six cents chars d'élite, et tous les chars de l'Egypte ; et il y avait des capitaines sur tout cela.
\VS{8}Et Yahweh endurcit le cœur de Pharaon, roi d'Egypte, qui poursuivit les fils d'Israël. Or les fils d'Israël étaient sortis à main levée\FTNT{Lé. 26:13 ; No. 33:3.}.
\VS{9}Les Egyptiens donc les poursuivirent ; et tous les chevaux des chars de Pharaon, ses cavaliers, et son armée les atteignirent comme ils étaient campés près de la mer, vers Pi-Hahiroth vis-à-vis de Baal-Tsephon.
\VS{10}Et Pharaon approchait. Les fils d'Israël levèrent leurs yeux, et voici, les Egyptiens marchaient après eux. Et les fils d'Israël eurent une grande frayeur et crièrent à Yahweh.
\VS{11}Ils dirent aussi à Moïse : Est-ce qu'il n'y avait pas des sépulcres en Egypte pour que tu nous aies emmenés pour mourir au désert ? Que nous as-tu fait en nous faisant sortir d'Egypte ?
\VS{12}N’est-ce pas ce que nous te disions en Egypte, en disant :Retire-toi de nous et que nous servions les Egyptiens? Car nous aimons mieux les servir que de mourir au désert.
\TextTitle{Délivrance miraculeuse par Yahweh}
\VS{13}Et Moïse dit au peuple : Ne craignez point, arrêtez-vous, et voyez la délivrance que Yahweh vous donnera aujourd'hui ; car les Egyptiens que vous voyez aujourd'hui, vous ne les verrez plus.
\VS{14}Yahweh combattra pour vous, et vous demeurerez tranquilles.
\VS{15}Or Yahweh avait dit à Moïse : Que cries-tu à moi ? Parle aux fils d'Israël, qu'ils marchent.
\VS{16}Et toi, élève ta verge,  étends ta main sur la mer, et fends-la; et  que les fils d'Israël entrent au milieu de la mer à sec.
\VS{17} Et quant à moi, voici, je m'en vais endurcir le cœur des Egyptiens, afin qu'ils entrent après eux ; et je serai glorifié en Pharaon, et en toute son armée, en ses chars et en ses cavaliers.
\VS{18}Et les Egyptiens sauront que je suis Yahweh, quand j'aurai été glorifié en Pharaon, en ses chars, et en ses cavaliers.
\VS{19}Et l'Ange de Dieu qui allait devant le camp d'Israël partit, et s'en alla derrière eux ; et la colonne de nuée partit de devant eux et se tint derrière eux.
\VS{20}Et elle vint entre le camp des Egyptiens et le camp d'Israël. Elle était aux uns une nuée et une obscurité; et pour les autres, elle les éclairait la nuit. L'un des camps n'approcha point de l'autre durant toute la nuit.
\VS{21}Or Moïse avait étendu sa main sur la mer, et Yahweh fit reculer la mer toute la nuit par un vent d'orient qui souffla avec puissance ; il mit la mer à sec, et les eaux se fendirent\FTNT{Jos. 4:23 ; Ps. 66:6 ; Ps. 106:9 ; Hé. 11:29.}.
\VS{22}Et les fils d'Israël entrèrent au milieu de la mer à sec, et les eaux leur servaient de mur à droite et à gauche.
\VS{23}Et les Egyptiens les poursuivirent ; et ils entrèrent après eux au milieu de la mer, à savoir tous les chevaux de Pharaon, ses chars, et ses cavaliers.
\VS{24}Mais il arriva que sur la veille du matin, Yahweh étant dans la colonne de feu et dans la nuée, regarda le camp des Egyptiens et le mit en déroute.
\VS{25}Il ôta les roues de leurs chars et alourdit leur marche. Alors les Egyptiens dirent : Fuyons  de devant les Israëlites, car Yahweh combat pour eux contre les Egyptiens.
\VS{26}Et Yahweh dit à Moïse : Etends ta main sur la mer, et les eaux retourneront sur les Egyptiens, sur leurs chars, et sur leurs cavaliers.
\VS{27}Moïse donc étendit sa main sur la mer, et la mer reprit son impétuosité vers le matin. Et les Egyptiens s'enfuyant rencntrèrent la mer qui s'était rejointe ; et ainsi Yahweh jeta les Egyptiens au milieu de la mer.
\VS{28}Car les eaux retournèrent et couvrirent les chars et les cavaliers de toute l'armée de Pharaon, qui étaient entrés après les Israélites dans la mer, et il n'en resta pas un seul.
\VS{29}Mais les fils d'Israël marchèrent au milieu de la mer à sec, et les eaux leur servaient de mur à droite et à gauche.
\VS{30}Ainsi, Yahweh délivra, en ce jour-là, Israël de la main des Egyptiens ; et Israël vit sur le bord de la mer les Egyptiens morts.
\VS{31}Israël vit donc la grande puissance que Yahweh avait déployée contre les Egyptiens; et le peuple craignit Yahweh, ils crurent en Yahweh, et en Moïse, son serviteur.
\Chap{15}
\TextTitle{Cantique de délivrance}
\VerseOne{}Alors Moïse et les fils d'Israël chantèrent ce cantique à Yahweh, et dirent : Je chanterai à Yahweh, car il est hautement élevé ; il a jeté dans la mer le cheval et celui qui le montait.
\VS{2}Yahweh est ma force et ma louange, et il a été mon sauveur, mon Dieu. Je lui dresserai un tabernacle, c'est le Dieu de mon père, je l'exalterai.
\VS{3}Yahweh est un vaillant guerrier, son nom est Yahweh.
\VS{4}Il a jeté dans la mer les chars de Pharaon et son armée ; l’élite de ses capitaines a été submergée dans la Mer Rouge.
\VS{5}Les gouffres les ont couverts, ils sont descendus au fond des eaux comme une pierre\FTNT{Né. 9:11.}.
\VS{6}Ta droite, ô Yahweh, s'est montrée magnifique en force ! Ta droite, ô Yahweh, a brisé l'ennemi\FTNT{Ps. 118:15-16 ; Ps. 77:16.} !
\VS{7}Tu as ruiné par la grandeur de ta majesté ceux qui s'élevaient contre toi ; tu as lâché ta colère, et elle les a consumés comme du chaume.
\VS{8}Par le souffle de tes narines, les eaux ont été amoncelées ; les eaux courantes se sont arrêtés comme un monceau ; les gouffres ont été gelés au milieu de la mer.
\VS{9}L'ennemi disait : Je poursuivrai, j'atteindrai, je partagerai le butin; mon âme sera assouvie d'eux, je tirerai mon épée, ma main les détruira.
\VS{10}Tu as soufflé de ton vent, la mer les a couverts ; ils ont été  enfoncés comme du plomb au plus profond des eaux.
\VS{11}Qui est comme toi parmi les dieux, ô Yahweh ! Qui est comme toi, magnifique en sainteté, digne d'être révéré et célébré, faisant des choses merveilleuses ?
\VS{12}Tu as étendu ta droite, la terre les a engloutis.
\VS{13}Tu as conduit par ta miséricorde ce peuple que tu as racheté ; tu l'as conduit par ta force à la demeure de ta sainteté.
\VS{14}Les peuples l'ont entendu, et ils en ont tremblé ; la douleur a saisi les habitants du pays des Philistins.
\VS{15}Alors les princes d'Edom seront troublés, et le tremblement saisira les puissants de Moab, tous les habitants de Canaan se fondront.
\VS{16}La frayeur et l'épouvante tomberont sur eux ; ils seront rendus muets comme une pierre par la grandeur de ton bras, jusqu'à ce que ton peuple soit passé, ô Yahweh ! Jusqu'à ce que ce peuple que tu as acquis soit passé\FTNT{De. 2:25 ; De. 11:25 ; Jos. 2:9.}.
\VS{17}Tu les introduiras et les planteras sur la montagne de ton héritage, au lieu que tu as préparé pour ta demeure, ô Yahweh ! Au lieu saint, ô Seigneur, que tes mains ont établi !
\VS{18}Yahweh régnera à jamais et à perpétuité.
\VS{19}Car les chevaux de Pharaon, ses chars et ses cavaliers sont entrés dans la mer, et Yahweh a fait retourner sur eux les eaux de la mer ; mais les fils d'Israël ont marché à sec au milieu de la mer.
\VS{20} Et Marie, la prophétesse, sœur d'Aaron, prit un tambour dans sa main, et toutes les femmes sortirent après elle, avec des tambours et des flûtes.
\VS{21}Et Marie leur répondait : Chantez à Yahweh, car il est hautement élevé ; il a jeté dans la mer le cheval et celui qui le montait.
\TextTitle{Yahweh pourvoit pour son peuple}
\VS{22}Après cela, Moïse fit partir les Israélites de la Mer Rouge, et ils partirent vers le désert de Schur ; et ayant marché trois jours dans le désert, ils ne trouvèrent point d'eau.
\VS{23}De là, ils vinrent à Mara, mais ils ne purent boire les eaux de Mara, parce qu'elles étaient amères ; c'est pourquoi ce lieu fut appelé Mara.
\VS{24}Et le peuple murmura contre Moïse en disant : Que boirons-nous ?
\VS{25}Et Moïse cria à Yahweh, et Yahweh lui montra\FTNT{«~Montra~» de l’hébreu «~yarah~» qui veut également dire «~enseigner~», «~signaler~», «~lancer~», «~instruire~», «~informer~», «~montrer~», «~jeter~» etc.} un certain bois qu'il jeta dans les eaux ; et les eaux devinrent douces. Il lui proposa là  une ordonnance et une loi, et il l'éprouva là,
\VS{26}et lui dit : Si tu écoutes attentivement la voix de Yahweh, ton Dieu, si tu fais ce qui est droit devant lui, si tu prêtes l'oreille à ses commandements, si tu gardes toutes ses ordonnances, je ne ferai venir sur toi aucune des infirmités que j'ai fait venir sur l'Egypte, car je suis Yahweh qui te guérit\FTNT{De. 7:12-15.}.
\VS{27}Puis ils vinrent à Elim, où il y avait douze fontaines d'eau, et soixante-dix palmiers. Et ils campèrent là, près des eaux.
\Chap{16}
\TextTitle{Yahweh envoie la manne}
\VerseOne{}Et toute l'assemblée des fils d'Israël étant partie d'Elim, vint au désert de Sin, qui est entre Elim et Sinaï, le quinzième jour du second mois après qu'ils furent sortis du pays d'Egypte.
\VS{2} Et toute l'assemblée des fils d'Israël murmura dans ce désert contre Moïse et Aaron.
\VS{3}Et les fils d'Israël leur dirent : Ah, que ne sommes-nous morts par la main de Yahweh dans le pays d'Egypte, quand nous étions assis près des pots de viande, et que nous mangions du pain à satiété ? Car vous nous avez amenés dans ce désert pour faire mourir de faim toute cette assemblée\FTNT{1 Co. 10:10 ; No. 11:4.}.
\VS{4}Et Yahweh dit à Moïse : Voici, je vais vous faire pleuvoir des cieux du pain, et le peuple sortira et en recueillera chaque jour la provision d'un jour, afin que je l'éprouve, pour voir s'il observera ma loi ou non.
\VS{5} Mais qu'ils apprêtent au sixième jour ce qu'ils auront apporté, et qu'il y ait  le double de ce qu’ils recueilleront chaque jour.
\VS{6}Moïse donc et Aaron dirent à tous les fils d'Israël : Ce soir vous saurez que Yahweh vous a tirés du pays d'Egypte.
\VS{7} Et au matin vous verrez la gloire de Yahweh, parce qu'il a entendu vos murmures, qui sont contre Yahweh ; car que sommes-nous pour que vous murmuriez contre nous ?
\VS{8}Moïse dit donc :  Ce sera quand Yahweh vousaura donné ce soir de la chair à manger, et qu'au matin il vous aura rassasiés de pain, parce qu'il a entendu vos murmures, par lesquels vous avez murmuré contre lui.  Car que sommes-nous ? Vos murmures ne sont pas contre nous, mais contre Yahweh.
\VS{9}Et Moïse dit à Aaron : Dis à toute l'assemblée des fils d'Israël : Approchez-vous de la présence de Yahweh, car il a entendu vos murmures.
\VS{10}Or il arriva qu'aussitôt qu'Aaron eut parlé à toute l'assemblée des fils d'Israël, ils regardèrent vers le désert, et voici, la gloire de Yahweh se montra dans la nuée.
\VS{11} Et Yahweh parla à Moïse, en disant :
\VS{12}J'ai entendu les murmures des fils d'Israël. Parle-leur et dis-leur : Entre les deux soirs, vous mangerez de la chair, et au matin vous serez rassasiés de pain ; et vous saurez que je suis Yahweh, votre Dieu.
\VS{13}Sur le soir donc, il monta des cailles qui couvrirent le camp, et au matin il y eut une couche de rosée autour du camp.
\TextTitle{Récolte de la manne}
\VS{14}Et cette couche de rosée étant évanouie,  voici, sur la surface du désert, quelque chose de menu et de rond, comme du grain sur la terre.
\VS{15}Ce que les fils d'Israël ayant vu, ils se dirent l'un à l'autre : Qu'est-ce ? Car ils ne savaient ce que c'était. Et Moïse leur dit : C'est le pain que Yahweh vous donne à manger\FTNT{Ps. 105:40.}.
\VS{16}Or ce que Yahweh a ordonné, c'est que chacun en recueille autant qu'il lui en faut pour sa nourriture, un omer par tête, selon le nombre de vos personnes ; chacun en prendra pour ceux qui sont dans sa tente.
\VS{17}Les fils d'Israël firent donc ainsi ; et les uns en recueillirent plus, les autres moins.
\VS{18}Et ils le mesuraient par homer ; et celui qui en avait recueilli beaucoup n'en avait pas plus qu'il ne lui en fallait ; ni celui qui en avait recueilli peu, n'en avait pas moins ; mais chacun en recueillait selon ce qu'il en pouvait manger.
\VS{19} Et Moïse leur avait dit : Que personne n'en laisse rien de reste jusqu’au matin.
\VS{20}Mais il y en eut qui n'obéirent point à Moïse, car quelques-uns en réservèrent jusqu'au matin ;  et il s'y engendra des vers, et cela puait. Et Moïse se mit en grande colère contre eux.
\VS{21}Ainsi, chacun en recueillait tous les matins autant qu'il lui en fallait pour se nourrir, et lorsque la chaleur du soleil était venue, elle se fondait.
\VS{22}Mais le sixième jour, ils recueillirent du pain en double , deux omers pour chacun ;et  les principaux de l'assemblée vinrent pour le rapporter à Moïse.
\TextTitle{Le sabbat\FTNTT{Né. 9:13-14 ; Mt. 12:1}}
\VS{23} Et il leur dit : C'est ce que Yahweh a dit : Demain est le repos, le sabbat sanctifié à Yahweh ; faites cuire ce que vous avez à cuire, et faites bouillir ce que vous avez à bouillir, et serrez tout ce qui sera de surplus, pour le garder jusqu'au matin.
\VS{24}Ils le serrèrent donc jusqu’au matin, comme Moïse l'avait ordonné, et il ne pua point, et il n'y eut point de vers dedans.
\VS{25}Alors Moïse dit : Mangez-le aujourd'hui, car c'est aujourd'hui le repos de Yahweh ; aujourd'hui vous n'en trouverez point dans les champs.
\VS{26}Durant six jours vous le recueillerez, mais le septième est le sabbat, il n'y en aura point ce jour-là.
\VS{27}Et au septième jour, quelques-uns du peuple sortirent pour en recueillir, mais ils n'en trouvèrent point.
\VS{28}Et Yahweh dit à Moïse : Jusqu’à quand refuserez-vous de garder mes commandements et mes lois ?
\VS{29}Considérez que Yahweh vous a ordonné le sabbat, c'est pourquoi il vous donne au sixième jour du pain pour deux jours ; que chacun demeure au lieu où il sera, et qu'aucun ne sorte du lieu où il est le septième jour.
\VS{30}Le peuple donc se reposa le septième jour.
\VS{31}Et la maison d'Israël nomma ce pain manne\FTNT{Le mot «~manne~» vient de l’hébreu «~man~» et veut dire «~Qu’est-ce que cela ?~». La manne  est une image de Jésus, le Pain de vie descendu du ciel (Jn. 6:32-52). La consommation quotidienne du Pain de vie, qui est aussi la Parole de Dieu, apporte la vie éternelle.}. Elle était comme de la semence de coriandre blanche, et ayant le goût d’un gâteau au miel.
\VS{32}Et Moïse dit : Voici ce que Yahweh a ordonné : Qu'on en remplisse un omer pour le garder pour vos générations, afin qu'on voie le pain que je vous ai fait manger au désert, après vous avoir retirés du pays d'Egypte.
\VS{33}Moïse dit à Aaron : Prends un vase, et mets-y un plein omer de manne, et pose-le devant Yahweh, afin qu’il soit conservé pour vos générations.
\VS{34}Et Aaron le posa devant le témoignage pour y être gardé, selon que le Seigneur l'avait ordonné à Moïse.
\VS{35}Et les fils d'Israël mangèrent la manne durant quarante ans, jusqu'à leur arrivée dans un pays habité ; ils mangèrent, dis-je, la manne, jusqu'à leur arrivée aux frontières du pays de Canaan.
\VS{36}Or un omer est la dixième partie d'un épha.
\Chap{17}
\TextTitle{Miracle de l'eau qui sort du rocher}
\VerseOne{}Et toute l'assemblée des fils d'Israël partit du désert de Sin, selon l’ordre de marche que Yahweh leur avait ordonné, et ils campèrent à Rephidim, où il n'y avait point d'eau à boire pour le peuple.
\VS{2} Et le peuple se souleva contre Moïse et ils lui dirent : Donnez-nous de l'eau à boire. Et Moïse leur dit : Pourquoi vous soulevez-vous contre moi ? Pourquoi tentez-vous Yahweh\FTNT{No. 20:2-5.} ?
\VS{3}Le peuple donc eut soif en ce lieu-là,par faute d'eau; et ainsi le peuple murmura contre Moïse en disant : Pourquoi nous as-tu fait monter hors d'Egypte, pour nous faire mourir de soif, nous, nos enfants, et nos troupeaux ?
\VS{4}Et Moïse cria à Yahweh en disant : Que ferai-je à ce peuple ? Encore un peu, et ils me lapideront.
\VS{5}Et Yahweh répondit à Moïse: Passe devant le peuple, et prends avec toi des anciens d'Israël, prends aussi dans ta main la verge avec laquelle tu as frappé le fleuve, et viens !
\VS{6}Voici, je vais me tenir là devant toi sur le rocher d’Horeb ; et tu frapperas le rocher, et il en sortira des eaux, et le peuple en boira. Moïse donc fit ainsi aux yeux des anciens d'Israël\FTNT{De. 9:8 ; Ps. 78:15 ; 1 Co. 10:4.}.
\VS{7}Et il nomma le lieu Massa et Meriba, à cause de la querelle des fils d'Israël, et parce qu'ils avaient tenté Yahweh, en disant : Yahweh est-il au milieu de nous ou non ?
\TextTitle{Bataille et victoire contre Amalek}
\VS{8}Alors Amalek vint et livra bataille contre Israël à Rephidim\FTNT{De. 25:17-18.}.
\VS{9}Et Moïse dit à Josué : Choisis-nous des hommes, et sors pour combattre contre Amalek ; et je me tiendrai demain sur le sommet de la colline, et la verge de Dieu sera dans ma main.
\VS{10}Et Josué fit comme Moïse lui avait ordonné en combattant contre Amalek. Mais  Moïse, Aaron et Hur montèrent au sommet de la colline.
\VS{11}Et il arrivait que lorsque Moïse élevait sa main, Israël était alors le plus fort, mais quand il reposait sa main, alors Amalek était le plus fort.
\VS{12}Et les mains de Moïse étant devenues pesantes, ils prirent une pierre et la mirent sous lui, et il s'assit dessus ; Aaron et Hur soutenaient ses mains, l'un d’un côté, et l'autre de l’autre côté ; et ainsi ses mains furent fermes jusqu'au soleil couchant.
\VS{13}Josué donc défit Amalek et son peuple au tranchant de l'épée.
\VS{14} Et Yahweh dit à Moïse : Ecris ceci pour mémoire dans un livre, et fais entendre à Josué que j'effacerai entièrement la mémoire d'Amalek de dessous les cieux.
\VS{15}Et Moïse bâtit un autel et le nomma Yahweh ma bannière.
\VS{16}Il dit aussi : Parce que la main a été levée contre le trône de Yahweh, Yahweh aura toujours la guerre contre Amalek.
\Chap{18}
\TextTitle{Jéthro conseille Moïse}
\VerseOne{}Or Jéthro, sacrificateur de Madian, beau-père de Moïse, apprit toutes les choses que Yahweh avait faites à Moïse, et à Israël, son peuple,  à savoir comment Yahweh avait retiré Israël de l'Egypte;
\VS{2} prit Séphora, la femme de Moïse, après que Moïse l'eut renvoyée,
\VS{3}et les deux fils de cette femme, dont l'un s'appelait Guerschom, car il avait dit : J’habite un pays étranger ;
\VS{4}et l'autre Eliézer, car il avait dit : Le Dieu de mon père m'a secouru et m'a délivré de l'épée de Pharaon.
\VS{5}Jéthro donc, beau-père de Moïse, vint vers Moïse avec ses fils et sa femme au désert, où il était campé, à la montagne de Dieu.
\VS{6}Il lui fit dire à Moïse : Jéthro, ton beau-père, vient vers toi, et ta femme et ses deux fils avec elle.
\VS{7}Et Moïse sortit au-devant de son beau-père, et s'étant prosterné, il l’embrassa ; et ils s’enquirent l'un de l'autre de leur santé, puis ils entrèrent dans la tente.
\VS{8}Et Moïse raconta à son beau-père toutes les choses que Yahweh avait faites à Pharaon et aux Egyptiens en faveur d'Israë, et toute la fatigue qu'ils avaient soufferte en chemin, et comment Yahweh les avait délivrés.
\VS{9}Et Jéthro se réjouit de tout le bien que Yahweh avait fait à Israël, parce qu'il les avait délivrés de la main des Egyptiens.
\VS{10}Puis Jéthro dit : Béni soit Yahweh qui vous a délivrés de la main des Egyptiens et de la main de Pharaon, qui a,dis-je, délivré le peuple de la main des Egyptiens!
\VS{11}Je reconnais maintenant que Yahweh est plus grand que tous les dieux, car en cela même en quoi ils ont agi présomptieusement, il a eu le dessus sur eux.
\VS{12}Jéthro, beau-père de Moïse, apporta aussi un holocauste et des sacrifices pour les offrir à Dieu. Et Aaron et tous les anciens d'Israël vinrent pour manger du pain avec le beau-père de Moïse dans la présence de Dieu.
\VS{13}Et il arriva, le lendemain, comme Moïse siégeait pour juger le peuple, et que le peuple se tenait devant Moïse depuis le matin jusqu'au soir,
\VS{14}que le beau-père de Moïse vit tout ce qu'il faisait au peuple, et il lui dit : Qu'est-ce que tu fais à l'égard de ce peuple ? Pourquoi es-tu assis seul, et tout le peuple se tient devant toi depuis le matin jusqu'au soir ?
\VS{15} Et Moïse répondit à son beau-père : C'est que le peuple vient à moi pour s'enquérir de Dieu.
\VS{16}Quand ils ont quelque affaire, ils viennent à moi, et je juge entre l'un et l'autre, et je leur fais entendre les ordonnances de Dieu et ses lois.
\VS{17}Mais le beau-père de Moïse lui dit : Ce que tu fais n'est pas bien.
\VS{18}Certainement, tu succomberas, toi et ce peuple qui est avec toi ; car cela est trop pesant pour toi, tu ne saurais faire cela toi seul.
\VS{19}Ecoute donc mon conseil; je te conseillerai et  Dieu sera avec toi: Sois pour ce peuple auprès de Dieu, et rapporte les causes à Dieu.
\VS{20}Et instruis-les des ordonnances et des lois ; et fais-leur connaître la voie par laquelle ils auront à marcher et ce qu'ils auront à faire.
\VS{21}Et choisis-toi d'entre tout le peuple des hommes vertueux, craignant Dieu ; des hommes véritables, haïssant le gain déshonnête, et établis-les chefs des milliers, chefs des centaines, chefs des cinquantaines et chefs des dizaines.
\VS{22}Et qu'ils jugent le peuple en tout temps, mais qu'ils te rapportent toutes les grandes affaires, et qu'ils jugent toutes les petites causes ; ainsi ils te soulageront et porteront une partie de la charge avec toi.
\VS{23}Si tu fais cela, et que Dieu te l’ordonne, tu pourras subsister, et tout le peuple parviendra en paix à destination.
\VS{24}Moïse donc obéit à la parole de son beau-père, et fit tout ce qu'il lui avait dit.
\VS{25}Ainsi, Moïse choisit de tout Israël des hommes vertueux, et les établit chefs sur le peuple, chefs des milliers, chefs des centaines, chefs des cinquantaines, et chefs des dizaines,
\VS{26}lesquels devaient juger le peuple en tout temps, mais ils devaient rapporter à Moïse les choses difficiles, et juger de toutes les petites affaires.
\VS{27}Puis Moïse laissa partir son beau-père, qui s'en alla dans son pays.
\Chap{19}
\TextTitle{Yahweh veut un royaume de sacrificateurs}
\VerseOne{}Au premier jour du troisième mois, après que les fils d'Israël furent sortis du pays d'Egypte, en ce même  jour-là, ils vinrent au désert de Sinaï.
\VS{2}Etant donc partis de Rephidim, ils vinrent au désert de Sinaï, et campèrent au désert. Et Israël campa vis-à-vis de la montagne.
\VS{3}Et Moïse monta vers Dieu, car Yahweh l'avait appelé de la montagne pour lui dire : Tu parleras ainsi à la maison de Jacob, et tu annonceras ceci aux fils d'Israël :
\VS{4}Vous avez vu ce que j'ai fait aux Egyptiens ; comment je vous ai portés comme sur des ailes d'aigle et vous ai amenés à moi.
\VS{5}Maintenant donc , si vous obéissez exactement à ma voix, et si vous gardez mon alliance, vous serez aussi d'entre tous les peuples mon plus précieux joyau, car toute la terre m'appartient.\FTNT{C’est ici que débute la période de la Loi ou Première Alliance. Le fait d’avoir réuni les textes de Genèse à Malachie sous l’appellation «~Ancien Testament~» a induit beaucoup de personnes en erreur quant à leur compréhension du plan de Dieu pour nos vies. Tout d’abord, l’emploi du mot «~testament~» est inapproprié puisqu’on ne peut parler de testament sans qu’il y ait au préalable la mort du testateur (Hé. 9:16-17). Certes, des animaux étaient tués sous la Loi pour couvrir les péchés. Toutefois, ces sacrifices étaient imparfaits et par conséquent prévus pour ne durer qu’un temps, en attendant le sacrifice parfait de Jésus-Christ (Hé. 10:1-14). De plus, il est évident que les animaux sacrifiés ne nous ont rien légué.
 Ensuite, il est à noter que tous les textes classés dans ce que l’on appelle à tort «~Ancien Testament~» ne se rapportent pas exclusivement et nécessairement à la Loi. Ainsi, des prophètes, en commençant par Moïse en personne,  ayant vécu sous la Loi, ont prophétisé et écrit sur d’autres sujets que la Loi, notamment sur la grâce et la fin des temps. N’oublions pas non plus que Jésus-Christ est né et a vécu sous la Loi (Ga. 4:4). En tant que juif, il l’a scrupuleusement respectée de telle sorte qu’elle fut totalement accomplie en Lui (Mt. 5:17-18 ; Jn. 19:30). En conséquence, la fin de la Loi mosaïque eut lieu après la mort du Seigneur, précisément au moment où le Seigneur a dit «~Tout est accompli~», et lorsque le voile du temple s’est déchiré de haut en bas (Mt. 27:50-51 ; Jn. 19:30). La Nouvelle Alliance, ou le Testament de Jésus débuta avec l’effusion de l’Esprit (Ac. 2). De la mort du Seigneur à la Pentecôte, une période de transition de cinquante jours s’est écoulée. Jésus-Christ s’est présenté pendant ce temps dans le sanctuaire céleste pour présenter son sang dans le Saint des saints. Une fois son sacrifice examiné et accepté, le Saint-Esprit qui avait été retiré de l’homme (Ge. 6:3) put de nouveau revenir habiter les cœurs des croyants. 
Mais qu’est-ce que la Loi exactement ?  Beaucoup de chrétiens sont dans la confusion à ce sujet.  En réalité, il n’y avait pas qu’une Loi mais trois sortes de lois : les  lois  morales et les lois cérémonielles  qui préexistaient depuis l’éternité ; et les lois civiles qui ont débuté avec Moïse car elles ne concernaient que son peuple.
-Les lois civiles régissaient le fonctionnement de la vie en communauté des hébreux. Elles étaient exclusivement réservées au peuple d’Israël dans le camp puis dans le pays de Canaan (Ex. 21:1-2 ; De. 23).
-Les lois morales font référence à la nature de Dieu : son amour, sa justice, sa sainteté ... Les dix commandements, à l’exception du sabbat tel que prescrit par Moïse (Ex. 16:28-29 ; Lé. 15:32), font partie des lois morales (Ex. 20:1-17). Les dix paroles ne constituent qu’une base, un résumé. Ainsi, d’autres règles morales sont énoncées tout au long des Ecritures notamment sur la sexualité (Lé. 18:1-22), l'interdiction des sacrifices humains et de l’occultisme (De. 18:10-13), le respect d’autrui et l’entraide (Lé. 19:10-18 ; Lé. 19:29-36). Comme il est impossible de consigner dans un livre tous les péchés moraux, le Seigneur a inscrit les lois morales dans le cœur de l’homme afin qu’il sache instinctivement faire la différence entre le bien et mal (Ro. 2:14-15). Jésus les a résumées en ces quelques mots : «~Tu aimeras le Seigneur ton Dieu de tout ton cœur, et de toute ton âme, et de toute ta pensée. Celui-ci est le premier et le plus grand commandement. Et le second semblable à celui-là, est : Tu aimeras ton prochain comme toi-même.~» (Mt. 22:37-39). Ces lois sont encore en vigueur aujourd’hui et le resteront pour toujours.
-Les lois cérémonielles étaient relatives au culte  et au sanctuaire terrestre, c’est-à-dire le Tabernacle puis le temple de Jérusalem (Hé. 9:1-10). Elles regroupent toutes les ordonnances concernant les sacrifices, les ablutions, les sabbats, les fêtes de Yahweh, la dîme des Lévites et des sacrificateurs (voir commentaire en No. 18:21 et Mal. 3:10). Les livres du Lévitique et des Nombres exposent en détail toutes les ordonnances reçues par Moïse d’après le modèle céleste que Yahweh lui avait montré  sur le Mont Sinaï (Ex. 26:30). Les lois cérémonielles préexistaient donc depuis l’éternité.
Les lois cérémonielles représentent la Première Alliance qui avait pour fondement la Loi morale. Or cette alliance a vieilli puis disparu car elle n’était que l’ombre des choses à venir (Hé. 8:13). En effet, elle était basée sur quatre points principaux : le temple, le culte centralisé, le sacrifice et les sacrificateurs. En Christ, nous n’avons plus besoin d’un temple physique puisque nous sommes devenus les temples vivants  de Yahweh (1 Co. 6:19 ; Ep. 2:22). Nous pouvons désormais adorer le Seigneur en Esprit et en vérité, à tout moment et en tout lieu (Jn. 4:23). Le sacerdoce lévitique ayant été aboli, chaque enfant de Dieu est devenu un sacrificateur (Ap. 5:10) qui offre en sacrifice sa propre vie consacrée au Seigneur (Ro. 12:1).
Les lois cérémonielles ont donc trouvé leur parfait accomplissement en Jésus-Christ : tous les sacrifices sanglants le préfiguraient, toutes les solennités ont été réalisées en Lui (voir note en Lé. 23). Christ est donc la fin de la Loi, non pas morale, mais cérémonielle (Ro. 10:4).
Un lien étroit existe entre les lois morales et les lois cérémonielles. La loi morale est comme un diagnostic qui révèle une pathologie incurable comme le sida : le péché (Ro. 5:13-20 ; Ro. 7:7-14). En la découvrant, l’homme se sent condamné car il réalise qu’il ne peut pas répondre aux exigences de la justice divine. La loi cérémonielle (le sang des animaux - Hé. 9:1-13 ; Hé. 10:11) a donné aux hommes une sorte de trithérapie pour les soulager provisoirement de leurs péchés mais sans pour autant les ôter (guérir, délivrer, nettoyer, laver) définitivement. Seul le sang de la Nouvelle Alliance, c’est-à-dire le sang de Jésus-Christ, a pu nous délivrer une fois pour toutes (Jn. 1:29 ; Hé. 9:11-26 ; Hé. 10:1-23 ; Ap. 1:6).}.
\VS{6}Et vous me serez un royaume de sacrificateurs, et une nation sainte ; ce sont là les discours que tu tiendras aux fils d'Israël.
\VS{7}Puis Moïse vint et appela les anciens du peuple, et proposa devant eux toutes ces choses-là que Yahweh lui avait ordonné.
\VS{8}Et tout le peuple répondit d'un commun accord, en disant : Nous ferons tout ce que Yahweh a dit. Et Moïse rapporta à Yahweh toutes les paroles du peuple.
\TextTitle{Moïse doit sanctifier le peuple pour qu'il rencontre Yahweh}
\VS{9}Et Yahweh dit à Moïse : Voici, je viendrai à toi dans une nuée épaisse, afin que le peuple entende quand je parlerai avec toi, et qu'il te croie aussi toujours ; car Moïse avait rapporté à Yahweh les paroles du peuple.
\VS{10}Yahweh dit aussi à Moïse : Va-t'en vers le peuple, et sanctifie-les aujourd'hui et demain, et qu'ils lavent leurs vêtements.
\VS{11}Et qu'ils soient tous prêts pour le troisième jour, car au troisième jour, Yahweh descendra sur la montagne de Sinaï, à la vue de tout le peuple.
\VS{12}Or tu mettras des bornes pour le peuple tout autour, et tu diras : Gardez-vous de monter sur la montagne et de toucher aucune de ses extrémités. Quiconque touchera la montagne sera puni de mort.
\VS{13}Aucune main ne la touchera, et certainement il sera lapidé, ou percé de flèches ; soit bête, soit homme, il ne vivra point. Quand la trompette sonnera longuement, ils monteront vers la montagne.
\VS{14}Et Moïse descendit de la montagne vers le peuple, et sanctifia le peuple, et ils lavèrent leurs vêtements.
\VS{15}Et il dit au peuple : Soyez tous prêts pour le troisième jour, et ne vous approchez point de vos femmes.
\VS{16}Et le troisième jour au matin, il y eut des tonnerres, et des éclairs, et une grosse nuée sur la montagne, avec un très fort son de shofar, et tout le peuple dans le camp fut effrayé.
\VS{17}Alors Moïse fit sortir le peuple du camp pour aller au-devant de Dieu ; et ils s'arrêtèrent au pied de la montagne.
\VS{18}Or le mont Sinaï était tout couvert de fumée, parce que Yahweh y était descendu en feu ; et sa fumée montait comme la fumée d'une fournaise, et toute la montagne tremblait fort.
\VS{19}Et comme le son du shofar se renforçait de plus en plus, Moïse parla, et Dieu lui répondit par une voix.
\VS{20}Yahweh donc étant descendu sur la montagne de Sinaï, au sommet de la montagne, Yahweh appela Moïse au sommet de la montagne ; et Moïse y monta.
\VS{21}Et Yahweh dit à Moïse : Descends, somme le peuple qu'il ne rompe point les barrières pour monter vers Yahweh afin de regarder ; de peur qu'un grand nombre d'entre eux ne périsse.
\VS{22}Et même, que les sacrificateurs qui s'approchent de Yahweh se sanctifient aussi, de peur qu'il n'arrive que Yahweh se jette sur eux.
\VS{23} Et Moïse dit à Yahweh : Le peuple ne pourra pas monter sur la montagne de Sinaï, parce que tu nous as sommés en me disant : Mets des bornes sur la montagne, et sanctifie-la.
\VS{24}Et Yahweh lui dit : Va, descends ; puis tu monteras, toi, et Aaron avec toi ; mais que les sacrificateurs et le peuple ne rompent point les bornes pour monter vers Yahweh, de peur qu’il n'arrive qu'il se jette sur eux.
\VS{25}Moïse descendit donc vers le peuple, et lui dit ces choses.
\Chap{20}
\TextTitle{Les dix paroles}
\VerseOne{}Alors Dieu prononça toutes ces paroles, disant :
\VS{2}Je suis Yahweh, ton Dieu, qui t'ai retiré du pays d'Egypte, de la maison de servitude.
\VS{3}Tu n'auras point d'autres dieux devant ma face.
\VS{4}Tu ne te feras point d'image taillée, ni aucune ressemblance des choses qui sont là-haut aux cieux, ni ici-bas sur la terre, ni dans les eaux sous la terre\FTNT{Lé. 26:1.}.
\VS{5}Tu ne te prosterneras point devant elles, et ne les serviras point ; car je suis Yahweh, ton Dieu ; le Dieu qui est jaloux, punissant l'iniquité des pères sur les fils, jusqu'à la troisième et à la quatrième génération de ceux qui me haïssent ;
\VS{6}et faisant miséricorde en mille générations à ceux qui m'aiment et qui gardent mes commandements.
\VS{7}Tu ne prendras point le Nom de Yahweh, ton Dieu, en vain ; car Yahweh ne tiendra point pour innocent celui qui aura pris son Nom en vain\FTNT{Lé. 19:12 ; Mt. 5:33.}.
\VS{8}Souviens-toi du jour du repos pour le sanctifier.
\VS{9}Tu travailleras six jours, et tu feras toute ton œuvre.
\VS{10}Mais le septième jour est le repos de Yahweh ton Dieu. Tu ne feras aucune oeuvre en ce jour-là, ni toi, ni ton fils, ni ta fille, ni ton serviteur, ni ta servante, ni ton bétail, ni ton étranger qui est dans tes portes.
\VS{11}Car Yahweh a fait en six jours les cieux, la terre, la mer, et tout ce qui y en eux, et s'est reposé le septième jour ; c'est pourquoi Yahweh a béni le jour du repos et l’a sanctifié\FTNT{Ge. 2:3 ; Ex. 31:14 ; Ez. 20:12.}.
\VS{12}Honore ton père et ta mère, afin que tes jours soient prolongés sur la terre que Yahweh, ton Dieu, te donne\FTNT{Lé. 19:3 ; De. 5:16 ; Mt. 15:4 ; Ep. 6:2.}.
\VS{13}Tu ne commettras pas de meurtre\FTNT{Mt. 5:21.}.
\VS{14}Tu ne commettras pas d’adultère\FTNT{Lé. 20:10 ; De. 5:18 ; Pr. 6:32 ; Mt. 5:32 ; Ro. 7:3.}.
\VS{15}Tu ne déroberas pas.
\VS{16}Tu ne diras pas de faux témoignage contre ton prochain.
\VS{17}Tu ne convoiteras pas la maison de ton prochain ; tu ne convoiteras pas la femme de ton prochain, ni son serviteur, ni sa servante, ni son bœuf, ni son âne, ni aucune chose qui soit à ton prochain.
\TextTitle{Le peuple tout tremblant devant Yahweh}
\VS{18}Or tout le peuple apercevait les tonnerres, les éclairs, le son du shofar, et la montagne fumante. Et le peuple voyant cela tremblait et se tenait loin.
\VS{19}Et ils dirent à Moïse : Parle, toi, avec nous, et nous écouterons ; mais que Dieu ne parle point avec nous, de peur que nous ne mourions\FTNT{De. 5:23-24 ; Hé. 12:18-19.}.
\VS{20}Et Moïse dit au peuple : Ne craignez point ; car Dieu est venu pour vous éprouver, et afin que sa crainte soit devant vous, et que vous ne péchiez point.
\VS{21}Le peuple donc se tint loin, mais Moïse s'approcha de l'obscurité dans laquelle Dieu était.
\VS{22}Et Yahweh dit à Moïse : Tu diras ainsi aux fils d'Israël : Vous avez vu que je vous ai parlé des cieux.
\VS{23}Vous ne vous ferez point avec moi de dieux d'argent ni de dieux d'or.
\VS{24}Tu me feras un autel de terre, sur lequel tu sacrifieras tes holocaustes, et tes offrandes de paix, ton menu et ton gros bétail. En quelque lieu que ce soit où je mettrai la mémoire de mon Nom, je viendrai là à toi, et je te bénirai.
\VS{25}Si tu me fais un autel de pierres, ne les taille point ; car si tu fais passer le fer dessus, tu le souillerais.
\VS{26}Et tu ne monteras point à mon autel par des marches, de peur que ta nudité ne soit découverte en y montant.
\Chap{21}
\TextTitle{Lois sur les maîtres et leurs esclaves}
\VerseOne{}Ce sont ici les lois que tu leur proposeras.
\VS{2}Si tu achètes un esclave Hébreu, il te servira six ans, et au septième il sortira pour être libre, sans rien payer\FTNT{De. 15:12 ; Lé. 25:39-43 ;  Jé. 34:14.}.
\VS{3}S'il est venu avec son corps seulement, il sortira avec son corps ; s'il avait une femme, sa femme sortira aussi avec lui.
\VS{4}Si son maître lui a donné une femme qui lui ait enfanté des fils ou des filles, sa femme et les enfants qu'il en aura seront à son maître, mais il sortira avec son corps.
\VS{5}Si l'esclave dit positivement: J'aime mon maître, ma femme, et mes fils, je ne sortirai point pour être libre.
\VS{6}Alors son maître le fera venir devant les juges, et le fera approcher de la porte ou du poteau, et son maître lui percera l'oreille avec un poinçon ; et il le servira pour toujours.
\VS{7}Si quelqu'un vend sa fille pour être esclave, elle ne sortira point comme les esclaves sortent.
\VS{8}Si elle déplaît à son maître, qui ne l'aura point fiancée, il la fera acheter ; mais il n'aura pas le pouvoir de la vendre à un peuple étranger, après lui avoir été infidèle.
\VS{9}Mais s'il l'a fiancé à son fils, il fera pour elle selon le droit des filles.
\VS{10}S’il en prend une pour lui, il ne retranchera rien de sa nourriture, de ses vêtements et du droit conjugal.
\VS{11}S'il ne fait pas pour elle ces trois choses-là, elle sortira sans payer aucun argent.
\TextTitle{Lois sur les dommages corporels}
\VS{12}Si quelqu'un frappe un homme et qu'il en meure,on le fera mourir de mort \FTNT{Lé. 24:17 ; No. 35:11-16 ; De. 19:2-11 ; Jos. 20:2.}.
\VS{13}S'il ne lui a point dressé d'embûches, mais que Dieu l'ait fait tomber entre ses mains, je t'établirai un lieu où il s'enfuira.
\VS{14}Mais si quelqu'un s'élève de propos délibéré contre son prochain, pour le tuer par ruse, tu le tireras de mon autel, afin qu'il meure.
\VS{15}Celui qui aura frappé son père ou sa mère sera puni de mort\FTNT{Lé. 20:9 ; De. 27:16 ; Mt. 15:4.}.
\VS{16}Si quelqu'un dérobe un homme et le vend, ou s'il est trouvé entre ses mains, on le fera mourir de mort.
\VS{17}Celui qui aura maudit son père ou sa mère sera puni de mort.
\VS{18}Si quelques uns ont une querelle, et que l'un ait frappé l'autre d'une pierre ou du poing, sans causer sa mort, mais qu'il soit obligé de se mettre au lit,
\VS{19}s'il se lève et marche dehors en s'appuyant sur son bâton, celui qui l'aura frappé sera absous ; toutefois, il le dédommagera de ce qu'il a chômé et le fera guérir entièrement.
\VS{20}Si quelqu'un a frappé du bâton son serviteur ou sa servante, et qu'il soit mort sous sa main, on ne manquera point de le venger.
\VS{21}Mais s'il survit un jour ou deux, il ne sera point vengé, car c'est son argent.
\VS{22}Si des hommes se querellent, et que l'un d'eux frappe une femme enceinte, et qu'elle en accouche, s'il n'y a pas cas de mort, il sera condamné à l'amende que le mari de la femme lui imposera, et il la donnera selon que les juges en ordonneront.
\VS{23}Mais s'il y a cas de mort, tu donneras vie pour vie,
\VS{24}oeil pour oeil, dent pour dent, main pour main, pied pour pied\FTNT{Lé. 24:20 ; De. 19:21 ; Mt. 5:38.},
\VS{25}brûlure pour brûlure, plaie pour plaie, meurtrissure pour meurtrissure.
\VS{26}Si quelqu'un frappe l'oeil de son serviteur, ou l'oeil de sa servante, et lui gâte l'oeil, il le laissera aller libre pour son œil ;
\VS{27}et s'il fait tomber une dent à son serviteur, ou à sa servante, il le laissera aller libre pour sa dent.
\VS{28}Si un bœuf heurte de sa corne un homme ou une femme, et que la personne en meure, le bœuf sera lapidé sans nulle exception, et on ne mangera point de sa chair, mais le maître du bœuf sera absous.
\VS{29}Si le bœuf était auparavant sujet à frapper de sa corne, et que son maître en ait été averti avec protestation, et qu'il ne l'ait point surveillé, s'il tue un homme ou une femme, le bœuf sera lapidé, et on fera aussi mourir son maître.
\VS{30}Si on lui impose un prix pour se racheter, il donnera la rançon de sa vie, selon tout ce qui lui sera imposé.
\VS{31}Si le bœuf heurte de sa corne un fils ou une fille, il lui sera fait selon cette même loi.
\VS{32}Si le bœuf heurte de sa corne un esclave, soit homme, soit femme, celui à qui est le bœuf donnera trente sicles d'argent au maître de l'esclave, et le bœuf sera lapidé.
\VS{33}Si quelqu'un découvre une fosse, ou si quelqu'un creuse une fosse, et ne la couvre point, et qu'il y tombe un bœuf ou un âne,
\VS{34}le maître de la fosse donnera satisfaction, et rendra l'argent au maître du bœuf, mais la bête morte lui appartiendra.
\VS{35}Et si le bœuf de quelqu'un blesse le bœuf de son prochain, et qu'il en meure, ils vendront le bœuf vivant, et en partageront l'argent par moitié, ils partageront aussi par moitié le bœuf mort.
\VS{36}Mais s'il est connu que le bœuf avait auparavant l’habitude de heurter avec sa corne, et que le maître ne l'ait point gardé, il restituera bœuf pour bœuf ; mais le bœuf mort sera pour lui.
\Chap{22}
\TextTitle{Lois sur les torts causés à autrui}
\VerseOne{}Si quelqu'un dérobe un bœuf, ou un chevreau, ou un agneau, et qu'il le tue, ou le vende, il restituera cinq bœufs pour le bœuf, et quatre agneaux ou chevreaux pour l'agneau ou pour le chevreau.
\VS{2}Si le voleur est trouvé dérobant avec effraction, et est frappé de sorte qu'il en meure, celui qui l'aura frappé ne sera point coupable de meurtre.
\VS{3}Mais si le soleil est levé sur lui, il sera coupable de meurtre. Il fera donc une entière restitution ; et s'il n'a pas de quoi, il sera vendu pour son vol.
\VS{4}Si ce qui a été dérobé est trouvé vivant entre ses mains, soit bœuf, soit âne, soit brebis ou chèvre, il rendra le double.
\VS{5}Si quelqu'un fait brouter dans un champ ou dans une vigne, en lâchant son bétail  qui aille paître dans le champ d'autrui, il rendra  le meilleur de son champ et le meilleur de sa vigne.
\VS{6}Si un feu éclate et rencontre des épines, et que le blé qui est en tas, ou sur pied, ou le champ, soit consumé, celui qui aura allumé le feu rendra entièrement ce qui en aura été  brûlé.
\VS{7}Si quelqu'un donne à son prochain de l'argent ou des vases à garder, et qu'on le dérobe de sa maison, et si l'on trouve le voleur, il rendra le double\FTNT{Lé. 5:20-26.}.
\VS{8}Mais si on ne trouve point le voleur, on fera venir le maître de la maison devant les juges pour jurer s'il n'a point mis sa main sur le bien de son prochain.
\VS{9}Dans toute affaire d'infidélité concernant un bœuf, un âne, une brebis, une chèvre, un vêtement ou tout objet perdu, dont quelqu'un dira qu'il lui appartient, la cause des deux parties viendra devant les juges ; et celui que les juges auront condamné, rendra le double à son prochain.
\VS{10}Si quelqu'un donne à garder à  son prochain un âne, un bœuf, quelque menue ou grosse bête, et qu'elle meure, ou qu'elle se soit cassé quelque  membre, ou qu'on l'ait emmenée sans que personne l'ait vue,
\VS{11}le serment de Yahweh interviendra entre les deux parties\FTNT{Hé. 6:16.}, pour savoir s'il n'a point mis sa main sur le bien de son prochain, et le maître de la bête se contentera du serment, et l'autre ne la rendra point.
\VS{12}Mais s'il est vrai qu' elle lui a été dérobée, il la rendra à son maître.
\VS{13}S'il est vrai qu'elle ait été déchirée par les bêtes sauvages, il la produira en témoignage, et il ne rendra point ce qui a été déchiré.
\VS{14}Si quelqu'un a emprunté de son prochain quelque bête, et qu'elle se casse quelque membre, ou qu'elle meure, son maître n'étant point présent, il ne manquera pas de la rendre.
\VS{15}Mais si son maître est avec lui, il ne la rendra point ; si elle a été louée, on payera seulement son louage.
\TextTitle{Lois diverses}
\VS{16}Si un homme séduit une vierge non fiancée, et couche avec elle, il faut qu'il la dote, et qu'il la prenne pour femme\FTNT{De. 22:28.}.
\VS{17}Mais si le père de la fille refuse absolument de la lui donner, il lui comptera autant d'argent qu'on en donne pour la dot des vierges.
\VS{18}Tu ne laisseras point vivre la sorcière\FTNT{De. 18:10-11 ; Lé. 20:27.}.
\VS{19}Celui qui couche avec une bête sera puni de mort\FTNT{Lé. 18:23 ; Lé. 20:15 ; De. 27:21.}.
\VS{20}Celui qui sacrifie à d'autres dieux qu'à Yahweh seul sera dévoué par interdit\FTNT{De. 13:6-16 ; De. 17:2-5 ; Lé. 17:7.}.
\VS{21}Tu ne fouleras ni n'opprimeras point l'étranger ; car vous avez été étrangers au pays d'Egypte\FTNT{Lé. 19:34.}.
\VS{22}Vous n'affligerez point la veuve ni l'orphelin\FTNT{De. 24:17-18 ;  Za. 7:10.}.
\VS{23}Si vous les affligez en quoi que ce soit, et qu'ils crient à moi, certainement j'entendrai leur cri.
\VS{24}Et ma colère s'embrasera, et je vous ferai mourir par l'épée ; et vos femmes seront veuves, et vos fils orphelins.
\VS{25}Si tu prêtes de l'argent à mon peuple, au pauvre qui est avec toi, tu ne te comporteras point avec lui en créancier, vous ne lui exigerez point d’intérêt.
\VS{26}Si tu prends en gage le vêtement de ton prochain, tu le lui rendras avant que le soleil soit couché\FTNT{De. 24:10-13.}.
\VS{27}Car c'est sa seule couverture, c'est son vêtement pour couvrir sa peau ; où coucherait-il ? S'il arrive donc qu'il crie à moi, je l'entendrai ; car je suis miséricordieux.
\VS{28}Tu ne maudiras point les juges, et tu ne maudiras point le prince de ton peuple\FTNT{Lé. 24:15-16.}.
\VS{29}Tu ne différeras point de m'offrir de ton abondance et de tes liqueurs; tu me donneras le premier-né de tes fils\FTNT{De. 26:2-11 ; Ex. 13:12-15.}.
\VS{30}Tu feras la même chose de ta vache, de ta brebis, et de ta chèvre. Il sera sept jours avec sa mère, et le huitième jour tu me le donneras.
\VS{31}Vous me serez saints, et vVous ne mangerez point de la chair déchirée dans les champs, mais vous la jetterez aux chiens.
\Chap{23}
\TextTitle{Lois diverses (suite)}
\VerseOne{}Tu ne léveras point de faux bruit, et tu ne te joindras point au méchant pour être un faux témoin, afin que violence soit faite\FTNT{De. 19:16-21 ; Ex. 20:16.}.
\VS{2}Tu ne suivras point la multitude pour faire le mal ; et tu ne témoigneras point dans un procès en sorte que tu te détournes après un grand nombre pour pervertir le droit.
\VS{3}Tu n'honoreras point le pauvre dans son procès\FTNT{De. 1:17.}.
\VS{4}Si tu rencontres le bœuf de ton ennemi, ou son âne égaré, tu ne manqueras point de le lui ramener.
\VS{5}Si tu vois l'âne de celui qui te hait, abattu sous sa charge, tu t'arrêteras pour le secourir, et tu ne manqueras pas de l'aider.
\VS{6}Tu ne pervertiras point le droit de l'indigent, qui est au milieu de toi, dans son procès.
\VS{7}Tu t'éloigneras de toute parole fausse, et tu ne feras point mourir l'innocent et le juste ; car je ne justifierai point le méchant.
\VS{8}Tu ne prendras point de présent ; car le présent aveugle les plus éclairés, et pervertit les paroles des justes.
\VS{9}Tu n'opprimeras point l'étranger ; car vous savez ce que c'est que d'être étrangers, parce que vous avez été étrangers au pays d'Egypte.
\TextTitle{Le sabbat, le repos de la terre}
\VS{10}Pendant six ans tu ensemenceras ta terre, et en recueilleras le revenu.
\VS{11}Mais la septième année, tu lui donneras du relâche, et la laisseras reposer, afin que les pauvres de ton peuple en mangent, et que les bêtes des champs mangent ce qui restera. Tu en feras de même de ta vigne et de tes oliviers.
\VS{12}Tu travailleras six jours, mais tu te reposeras au septième jour, afin que ton bœuf et ton âne se reposent, et que le fils de ta servante et l'étranger reprennent courage.
\VS{13}Vous prendrez garde à toutes les choses que je vous ai ordonnées. Vous ne ferez point mention du nom des dieux étrangers, on ne l'entendra point de ta bouche\FTNT{Jos. 23:7 ; Ps. 16:4.}.
\TextTitle{Les fêtes solennelles}
\VS{14}Trois fois l'an, tu me célébreras une fête solennelle\FTNT{Lé. 23:4-44.}.
\VS{15}Tu garderas la fête solennelle des pains sans levain\FTNT{Ex. 29:2.} ; tu mangeras des pains sans levain pendant sept jours, comme je t'ai ordonné, en la saison et au mois où les épis mûrissent ; car c’est en ce mois-là que tu es sorti d'Egypte ; et nul ne se présentera devant ma face à vide.
\VS{16}Et la fête solennellede la moisson des premiers fruits de ton travail, de ce que tu auras semé au champ; et la fête de la récolte, après la fin de l'année, quand tu auras recueilli du champ les fruits de ton travail\FTNT{Ex. 34:22.}.
\VS{17}Trois fois l'an, tous les mâles d'entre vous se présenteront devant le Seigneur Yahweh.
\VS{18}Tu ne sacrifieras point le sang de mon sacrifice avec du pain levé ; et la graisse de ma fête solennelle ne passera point la nuit jusqu’au matin\FTNT{Ex. 34:25-26.}.
\VS{19}Tu apporteras dans la maison de Yahweh, ton Dieu, les prémices des premiers fruits de ta terre. Tu ne feras point cuire le chevreau dans le lait de sa mère.
\TextTitle{Mises en garde et promesses de Yahweh}
\VS{20}Voici, j'envoie un Ange devant toi, afin qu'il te garde dans le chemin, et qu'il t'introduise dans lieu que je t'ai préparé.
\VS{21}Garde-toi de provoquer sa colère, et écoute sa voix, et ne l'irrite point, car il ne pardonnera point votre péché ; car mon Nom est en lui.
\VS{22}Mais si tu écoutes attentivement sa voix, et si tu fais tout ce que je te dirai, je serai l'ennemi de tes ennemis, et j'affligerai ceux qui t'affligeront.
\VS{23}Car mon Ange marchera devant toi, et t'introduira au pays des Amoréens, des Héthiens, des Phéréziens, des Cananéens, des Héviens, et des Jébusiens, et je les exterminerai.
\VS{24}Tu ne te prosterneras point devant leurs dieux, et tu ne les serviras point, et tu ne feras point selon leurs œuvres, mais tu les détruiras entièrement, et tu briseras entièrement leurs statues\FTNT{Ex. 20:5 ; Ex. 34:13 ; No. 33:52.}.
\VS{25}Vous servirez Yahweh, votre Dieu. Et il bénira ton pain et tes eaux ; et j'ôterai les maladies du milieu de toi\FTNT{De. 6:13 ; Mt. 4:10 ; De. 7:15-16 ; Ex. 15:26.}.
\VS{26}Il n'y aura point dans ton pays de femme qui avorte, ou qui soit stérile ; j'accomplirai le nombre de tes jours.
\VS{27}J'enverrai la terreur de mon Nom devant toi, et j'effrayerai tout peuple vers lequel tu arriveras, et je ferai que tous tes ennemis tourneront le dos devant toi\FTNT{De. 7:23.}.
\VS{28}Et j'enverrai des frelons devant toi, qui chasseront les Héviens, les Cananéens, et les Héthiens, de devant ta face\FTNT{De. 7:20 ; Jos. 24:12.}.
\VS{29}Je ne les chasserai point loin de devant ta face en une année, de peur que le pays ne devienne un désert, et que les bêtes des champs ne se multiplient contre toi.
\VS{30}Mais je les chasserai peu à peu loin de devant toi, jusqu'à ce que tu te sois accru, et que tu possèdes le pays.
\VS{31}Et je mettrai des bornes depuis la Mer Rouge jusqu'à la mer des Philistins, et depuis le désert jusqu'au fleuve ; car je livrerai entre tes mains les habitants du pays et je les chasserai de devant toi.
\VS{32}Tu ne traiteras point d'alliance avec eux ni avec leurs dieux.
\VS{33}Ils n'habiteront point dans ton pays, de peur qu'ils ne te fassent pécher contre moi ; car tu servirais leurs dieux, et ce serait un piège pour toi.
\Chap{24}
\TextTitle{La Loi de Yahweh lue au peuple ; le sang de l’Alliance}
\VerseOne{}Puis il dit à Moïse : Monte vers Yahweh, toi et Aaron, Nadab et Abihu, et soixante-dix des anciens d'Israël, et vous vous prosternerez de loin.
\VS{2}Et Moïse s'approchera seul de Yahweh, mais eux ne s'en approcheront point, et le peuple ne montera point avec lui.
\VS{3}Alors Moïse vint, et récita au peuple toutes les paroles de Yahweh, et toutes ses lois, et tout le peuple répondit d'une voix et dit : Nous ferons toutes les choses que Yahweh a dites.
\VS{4}Or Moïse écrivit toutes les paroles de Yahweh, et s'étant levé de bon matin, il bâtit un autel au bas de la montagne, et dressa pour monument douze pierres pour les douze tribus d'Israël.
\VS{5} Et il envoya des jeunes hommes, des fils d'Israël, qui offrirent des holocaustes et qui sacrifièrent des veaux à Yahweh  en sacrifice d'offrande de paix.
\VS{6}Et Moïse prit la moitié du sang, et le mit dans des bassins, et répandit l'autre moitié sur l'autel.
\VS{7}Ensuite, il prit le livre de l'alliance et le lut, et le  peuple qui l'écoutait dit : Nous ferons tout ce que Yahweh a dit, et nous obéirons.
\VS{8}Moïse donc prit le sang, et le répandit sur le peuple, en disant : Voici le sang de l'Alliance que Yahweh a traitée avec vous, selon toutes ces paroles\FTNT{Hé. 9:20 ; Mt. 26:28 ; Mc. 14:24 ; Lu. 22:20 ; 1 Co. 11:25.}.
\TextTitle{Yahweh fait monter Moïse sur la montagne}
\VS{9}Puis Moïse, Aaron, Nadab, Abihu, et les soixante-dix anciens d'Israël montèrent.
\VS{10}Et ils virent le Dieu d'Israël, et sous ses pieds comme un ouvrage de saphir transparent, comme le ciel dans toute sa pureté.
\VS{11}Et il ne mit point sa main sur ceux qui avaient été choisis d'entre les fils d'Israël; ansi, ils virent Dieu, et ils mangèrent et burent.
\VS{12}Et Yahweh dit à Moïse : Monte vers moi sur la montagne, et demeure là ; et je te donnerai des tables de pierre, la loi et les commandements que j'ai écrits pour les enseigner.
\VS{13}Alors Moïse se leva avec Josué qui le servait ; et Moïse monta sur la montagne de Dieu.
\VS{14}Et il dit aux anciens d'Israël : Demeurez ici en nous attendant jusqu'à ce que nous retournions vers vous. Et voici, Aaron et Hur seront avec vous ; quiconque aura quelque affaire, qu'il s'adresse à eux.
\VS{15}Moïse donc monta sur la montagne, et une nuée couvrit la montagne\FTNT{Ex. 19:9-16.}.
\VS{16}Et la gloire de Yahweh demeura sur la montagne de Sinaï, et la nuée la couvrit pendant six jours. Et au septième jour, il appela Moïse du milieu de la nuée.
\VS{17}Et ce qu'on voyait de la gloire de Yahweh au sommet de la montagne, était comme un feu dévorant aux yeux des fils d'Israël\FTNT{De. 4:24 ; De. 9:3 ; Hé. 12:29.}.
\VS{18}Et Moïse entra dans la nuée et monta sur la montagne. Moïse fut sur la montagne quarante jours et quarante nuits.
\Chap{25}
\TextTitle{Des offrandes volontaires pour les matériaux du tabernacle}
\VerseOne{}Et Yahweh parla à Moïse, en disant :
\VS{2}Parle aux fils d'Israël, et qu'on prenne une offrande pour moi. Vous prendrez mon offrande de tout homme dont le cœur me l'offrira volontairement.
\VS{3}Et c'est ici l'offrande que vous prendrez d'eux : De l'or, de l'argent, de l'airain,
\VS{4}de la pourpre, de l'écarlate, du cramoisi, du fin lin, du poil de chèvre,
\VS{5}des peaux de moutons teintes en rouge, des peaux de dauphins, du bois d’acacia,
\VS{6}de l'huile pour le luminaire, des aromates pour l'huile d'onction et pour le parfum odoriférant,
\VS{7}des pierres d'onyx, et d’autres pierres pour la garniture de l'éphod et pour le pectoral.
\VS{8}Et ils me feront un sanctuaire, et j'habiterai au milieu d'eux\FTNT{Ex. 29:45-46.}.
\VS{9}Ils le feront conformément à tout ce que je vais te montrer, selon le modèle du tabernacle et selon le modèle de tous ses ustensiles; vous le ferez donc ainsi.
\TextTitle{L'arche de l'alliance}
\VS{10}Et ils feront une arche de bois d’acacia ; et sa longueur sera de deux coudées et demie, et sa largeur d'une coudée et demie, et sa hauteur d'une coudée et demie.
\VS{11}Et tu la couvriras d’or pur, tu l'en couvriras en dehors et en dedans ; et tu feras sur elle un couronnement d'or tout autour\FTNT{Ex. 37:1-9.}.
\VS{12}Et tu fondras pour elle quatre anneaux d'or, que tu mettras à ses quatre coins, deux anneaux à l'un de ses côtés, et deux autres de l'autre côté.
\VS{13}Tu feras aussi des barres de bois d’acacia, et tu les couvriras d'or.
\VS{14}Puis tu feras entrer les barres dans les anneaux aux côtés de l'arche, pour porter l'arche avec elles.
\VS{15}Les barres seront dans les anneaux de l'arche, et on ne les en tirera point.
\VS{16}Et tu mettras dans l'arche le témoignage que je te donnerai\FTNT{Hé. 9:4.}.
\VS{17}Tu feras aussi un propitiatoire d’or pur, dont la longueur sera de deux coudées et demie, et la largeur d'une coudée et demie.
\VS{18}Et tu feras deux chérubins d'or ; tu les feras d'ouvrage étendu au marteau, tirés des deux extrémités du propitiatoire.
\VS{19}Fais donc un chérubin tiré des extrémités et un chérubin tiré de l'autre extrémité ; vous ferez les chérubins tirés du propitiatoire à ses deux extrémités.
\VS{20}Et les chérubins étendront les ailes en haut, couvrant de leurs ailes le propitiatoire, et leurs faces seront vis-à-vis l’une de l’autre ; et le regard des chérubins sera vers le propitiatoire\FTNT{1 R. 8:6-7 ; Hé. 9:5.}.
\VS{21}Et tu poseras le propitiatoire au-dessus de l'arche, et tu mettras dans l'arche le témoignage que je te donnerai.
\VS{22}Et je me rencontrerai là avec toi, et je te dirai de dessus le propitiatoire, d'entre les deux chérubins qui seront sur l'arche du témoignage, toutes les choses que je t’ordonnerai pour les fils d'Israël\FTNT{Ex. 29:42-43 ; No. 7:89.}.
\TextTitle{La table des pains de proposition}
\VS{23}Tu feras aussi une table de bois d’acacia. Sa longueur sera de deux coudées, et sa largeur d'une coudée, et sa hauteur d'une coudée et demie.
\VS{24}Tu la couvriras d’or pur, et tu lui feras un couronnement d'or tout autour.
\VS{25}Tu lui feras aussi à l’entour une clôture d’une largeur de main, et tout autour de sa clôture tu feras un couronnement d'or.
\VS{26}Tu lui feras aussi quatre anneaux d'or que tu mettras aux quatre coins qui seront à ses quatre pieds.
\VS{27}Les anneaux seront à l'endroit de la clôture, afin d'y mettre les barres pour porter la table.
\VS{28}Tu feras les barres de bois d’acacia, et tu les couvriras d'or, et on portera la table avec elles.
\VS{29}Tu feras aussi ses plats, ses tasses, ses gobelets, et ses bassins, avec lesquels on fera les aspersions ; tu les feras d’or pur\FTNT{Ex. 37:10-16.}.
\VS{30}Et tu mettras sur cette table le pain de proposition continuellement devant moi\FTNT{Lé. 24:5-9.}.
\TextTitle{Le chandelier d’or pur}
\VS{31}Tu feras aussi un chandelier d’or pur\FTNT{Le chandelier avait une double symbolique. D’une part, il préfigurait Jésus-Christ, notre lumière (Jn. 1:4-5 ; Jn. 8:12). Les sept lampes évoquaient l’omniscience de l’Esprit de Jésus-Christ (Za. 3:9 ; Jn. 16:29-30 ; Ap. 1:4 ; Ap. 3:1 ; Ap. 4:5 ; Ap. 5:6). Il est à noter que ce chandelier comportait des calices en forme de fleurs, de pommes et d’amandes (Ex. 25:33) qui symbolisaient les fruits de l’Esprit que nous devons nécessairement porter (Ga. 5:22). D’autre part, il est une image de l’Eglise (Ap. 1:20). Voir commentaire Es. 8:13-17. Ex. 37:17-24.}. Le chandelier sera étendu au marteau ; son pied, sa tige et ses branches, ses plats, ses pommeaux et ses fleurs seront tirés de lui.
\VS{32}Six branches sortiront de ses côtés : Trois branches d'un côté du chandelier, et trois autres de l'autre côté du chandelier.
\VS{33}Il y aura sur l’une des branches trois petits plats en forme d'amande, un pommeau et une fleur ; sur l'autre branche trois petits plats en forme d'amande, un pommeau et une fleur ; il en sera de même des six branches sortant du chandelier.
\VS{34}ll y aura aussi au chandelier quatre petits plats en forme d'amande, ses pommaux et ses fleurs.
\VS{35}Un pommeau sous deux branches tirées du chandelier, un pommeau sous deux autres branches tirées de lui, et un pommeau sous deux autres branches tirées de lui ; il en sera de même des six branches sortant du chandelier.
\VS{36}Leurs pommeaux et leurs branches seront tirés de lui, et tout le chandelier sera un seul ouvrage étendu au marteau, et d’or pur.
\VS{37}Tu feras aussi ses sept lampes, et on les allumera afin qu'elles éclairent vis-à-vis du chandelier.
\VS{38}Et ses mouchettes et ses petits plats destinés à recevoir ce qui tombe des lampes seront d’or pur.
\VS{39}On le fera avec tous ses ustensiles d'un talent d’or pur.
\VS{40}Regarde donc, et fais selon le modèle qui t'est montré sur la montagne.
\Chap{26}
\TextTitle{Les tapis de fin lin}
\VerseOne{}Tu feras aussi le tabernacle de dix tapis de fin lin retors, de pourpre, d'écarlate, et de cramoisi ; et tu les feras semés de chérubins d'un ouvrage exquis\FTNT{Ex. 36:8-38.}.
\VS{2}La longueur d'un tapis sera de vingt-huit coudées, et la largeur du même tapis de quatre coudées ; tous les tapis auront une même mesure.
\VS{3}Cinq de ces tapis seront joints l'un à l'autre, et les cinq autres seront aussi joints l'un à l'autre.
\VS{4}Fais aussi des lacets de pourpre sur le bord d'un tapis, au bord du premier assemblage ; et tu feras la même chose au bord du dernier tapis dans l'autre assemblage.
\VS{5}Tu feras donc cinquante lacets au premier tapis, et tu feras cinquante lacets au bord du tapis qui est dans le second assemblage. Les lacets seront vis-à-vis l'un de l'autre.
\VS{6}Tu feras aussi cinquante crochets d'or, et tu attacheras les tapis l'un à l'autre avec les crochets ; ainsi le tabernacle ne fera qu’un.
\TextTitle{Les tapis de poil de chèvre}
\VS{7}Tu feras aussi des tapis de poil de chèvre pour servir de tente sur le tabernacle ; tu feras onze de ces tapis.
\VS{8}La longueur d'un tapis sera de trente coudées, et la largeur du même tapis sera de quatre coudées ; les onze tapis auront une même mesure.
\VS{9}Puis tu joindras séparément cinq de ces tapis, et les six tapis à part ; mais tu redoubleras le sixième tapis sur le devant du tabernacle.
\VS{10}Tu feras aussi cinquante lacets sur le  bord de l'un des tapis, à savoir au dernier qui est assemblé, et cinquante lacets au bord du tapis du second assemblage.
\VS{11}Tu feras aussi cinquante crochets d'airain, et tu feras entrer les crochets dans les lacets ; et tu assembleras ainsi la tente qui fera un tout.
\VS{12}Mais ce qu'il y aura en surplus dans les tapis de la tente, à savoir la moitié du tapis de reste, retombera sur le derrière du tabernacle.
\VS{13}La coudée d’une part, et la coudée d’autre part, qui seront de reste sur la longueur des tapis de la tente, retomberont sur les deux côtés du tabernacle, pour le couvrir.
\TextTitle{Les couvertures de peaux de béliers et de dauphins}
\VS{14}Tu feras aussi pour ce tabernacle une couverture de peaux de béliers teintes en rouge, et une couverture de peaux de dauphins par-dessus\FTNT{Ex. 35:7 ; Ex. 35:23 ; Ex. 36:19 ; Ex. 39:34.}.
\TextTitle{Les planches et leurs bases}
\VS{15}Et tu feras pour le tabernacle, des planches de bois d’acacia, qu'on fera tenir debout\FTNT{Ex. 36:20-34.}.
\VS{16}La longueur d'une planche sera de dix coudées, et la largeur d’une même planche d'une coudée et demie.
\VS{17}Il y aura à chaque planche deux tenons joints l’un à l’autre ; et tu feras de même pour toutes les planches du tabernacle.
\VS{18}Tu feras donc les planches du tabernacle, à savoir vingt planches qui regardent vers le midi.
\VS{19}Et au-dessous des vingt planches, tu feras quarante bases d'argent ; deux bases sous une planche pour ses deux tenons, et deux bases sous l'autre planche pour ses deux tenons.
\VS{20}Et vingt planches de l'autre côté du tabernacle, du coté nord.
\VS{21}Et leurs quarante bases seront d'argent, deux bases sous une planche, et deux bases sous l'autre planche.
\VS{22}Et pour le fond du tabernacle, vers l'occident, tu feras six planches.
\VS{23}Tu feras aussi deux planches pour les angles du tabernacle, aux deux cotés du fond.
\VS{24}Et ils seront égaux par le bas, et ils seront joints et unis par le haut avec un anneau ; il en sera de même des deux planches qui seront aux deux angles.
\VS{25}Il y aura donc huit planches, et seize bases d'argent ; deux bases sous une planche et deux bases sous une autre planche.
\VS{26}Après cela, tu feras cinq barres de bois d’acacia, pour les planches d'un des côtés du tabernacle.
\VS{27}Pareillement, tu feras cinq barres pour les planches de l'autre côté du tabernacle ; et cinq barres pour les planches du côté du tabernacle, pour le fond, vers le côté de l'occident.
\VS{28}Et la barre du milieu sera au milieu des planches d’une extrémité à l’autre.
\TextTitle{Le revêtement d’or}
\VS{29}Tu couvriras aussi d'or les planches, et tu feras d'or leurs anneaux pour mettre les barres, et tu couvriras d'or les barres.
\VS{30}Tu dresseras le tabernacle selon le modèle qui t’est montré sur la montagne.
\TextTitle{Les voiles intérieur et extérieur}
\VS{31}Et tu feras un voile\FTNT{Le voile intérieur symbolisait la chair de Jésus-Christ qui a été brisée à cause de nos péchés (Es. 53:5 ; Hé. 10:20). Ex. 36:35-38 ; Mt. 27:51 ; Hé. 9:3.} de pourpre, d'écarlate, de cramoisi, et de fin lin retors ; on le fera d'ouvrage exquis, semé de chérubins.
\VS{32}Et tu le mettras sur quatre piliers de bois d’acacia couverts d'or, ayant leurs crochets d'or. Et ils seront sur quatre bases d'argent.
\VS{33}Puis tu mettras le voile sous les crochets, et tu  feras entrer là dedans, c'est-à-dire au dedans du voile, l'arche du témoignage ; et ce voile vous fera la séparation entre le lieu saint et le Saint des saints.
\VS{34}Et tu poseras le propitiatoire sur l'arche du témoignage, dans le Saint des saints.
\VS{35}Et tu mettras la table au dehors de ce voile, et le chandelier vis-à-vis de la table, au côté du tabernacle, vers le sud ; et tu placeras la table côté nord.
\VS{36}Et à l'entrée du tabernacle, tu feras une tapisserie de pourpre, d'écarlate, de cramoisi et de fin lin retors, d'ouvrage de broderie.
\VS{37}Tu feras aussi pour cette tapisserie cinq piliers de bois d’acacia, que tu couvriras d'or, et leurs crochets seront d'or ; et tu fondras pour eux cinq bases d'airain.
\Chap{27}
\TextTitle{L'autel d'airain}
\VerseOne{}Tu feras aussi un autel de bois d’acacia, ayant cinq coudées de long, et cinq coudées de large ; l'autel sera carré, et sa hauteur sera de trois coudées.
\VS{2}Tu feras ses cornes à ses quatre coins ; ses cornes seront tirées de lui, et tu le couvriras d'airain\FTNT{C’est sur l’autel d’airain que les animaux étaient sacrifiés. Il préfigurait la croix et le jugement que Jésus-Christ a pris sur lui à notre place (Es. 53:5 ; 2 Co. 13:4 ; Ph. 2:8).}.
\VS{3}Tu feras ses chaudrons pour recevoir ses cendres, et ses racloirs, ses bassins, ses fourchettes, et ses encensoirs ; tu feras tous ses ustensiles d'airain.
\VS{4}Tu lui feras une grille d'airain en forme de treillis, et tu feras au treillis quatre anneaux d'airain à ses quatre coins.
\VS{5}Et tu le mettras au-dessous de l'enceinte de l'autel en bas, et le treillis s'étendra jusqu'au milieu de l'autel.
\VS{6}Tu feras aussi des barres pour l'autel, des barres de bois d’acacia, et tu les couvriras d'airain.
\VS{7}Et on fera passer ses barres dans les anneaux ; les barres seront aux deux côtés de l'autel pour le porter.
\VS{8}Tu le feras creux avec des planches ; ils le feront ainsi qu'il t'a été montré sur la montagne\FTNT{Ex. 38:1-7.}.
\TextTitle{Le parvis}
\VS{9}Tu feras aussi le parvis du tabernacle, au coté qui regarde vers le sud; il y aura pour former le parvis, des toiles de fin lin retors ; la longueur de l'un des côtés sera de cent coudées.
\VS{10}Il y aura vingt piliers avec leurs vingt bases d'airain, mais les crochets des piliers et leurs filets seront d'argent.
\VS{11}Ainsi du côté nord, il y aura également des toiles sur une longueur de cent coudées, avec vingt piliers avec leurs vingt bases d'airain ; mais les crochets des piliers avec leurs filets seront d'argent.
\VS{12}La largeur du parvis du côté de l'occident sera de cinquante coudées de toiles, qui auront dix piliers, avec leurs dix bases.
\VS{13}Et la largeur du parvis du côté de l'orient, directement vers le levant, sera de cinquante coudées.
\VS{14}A l'un des côtés, il y aura quinze coudées de toiles, avec leurs trois piliers et leurs trois bases.
\VS{15}Et de l'autre côté, quinze coudées de toiles, avec leurs trois piliers et leurs trois bases.
\TextTitle{La porte du parvis}
\VS{16}Il y aura aussi pour la porte du parvis une tapisserie de vingt coudées, faite de pourpre, d'écarlate, de cramoisi, et de fin lin retors, ouvrage de broderie, avec quatre piliers et quatre bases.
\VS{17}Tous les piliers du parvis seront ceints d'un filet d'argent, et leurs crochets seront d'argent, mais leurs bases seront d'airain.
\VS{18}La longueur du parvis sera de cent coudées, et la largeur de cinquante, de chaque côté ; et la hauteur de cinq coudées. Il sera de fin lin retors, et les soubassements des piliers seront d'airain.
\VS{19}Que tous les ustensiles du tabernacle, pour tout son service, et tous ses pieux, avec les pieux du parvis, soient d'airain\FTNT{Ex. 38:9-20.}.
\TextTitle{L'huile d'olive vierge pour les lampes}
\VS{20}Tu ordonneras aux fils d'Israël qu'ils t'apportent de l'huile d'olive vierge pour le luminaire, afin de faire luire les lampes continuellement\FTNT{Ex. 35:8-28 ; Lé. 24:1-4.}.
\VS{21}Aaron avec ses fils les prépareront dans la présence de Yahweh, depuis le soir jusqu'au matin, dans le tabernacle d'assignation, hors du voile qui est devant le témoignage ; ce sera une ordonnance perpétuelle pour les fils d'Israël.
\Chap{28}
\TextTitle{La sacrificature}
\VerseOne{}Et toi, fais approcher de toi Aaron, ton frère, et ses fils avec lui, d'entre les fils d'Israël, pour m'exercer la sacrificature,à savoir Aaron, et Nadab, Abihu, Eléazar, et Ithamar, fils d'Aaron.
\VS{2}Et tu feras à Aaron, ton frère, de saints vêtements pour gloire et pour ornement.
\TextTitle{Les vêtements sacrés des sacrificateurs}
\VS{3}Et tu parleras à tous les hommes d'esprit, à chacun de ceux que j'ai remplis de l'esprit de science, afin qu'ils fassent des vêtements à Aaron pour le sanctifier, afin qu'il m'exerce la sacrificature.
\VS{4}Et ce sont ici les vêtements qu’ils feront : Le pectoral, l'éphod, la robe, la tunique brodée, la tiare, et la ceinture. Ils feront donc les saints vêtements à Aaron, ton frère, et à ses fils, pour m'exercer la sacrificature.
\VS{5}Et ils prendront de l'or, de la pourpre, de l'écarlate, du cramoisi, et du fin lin.
\TextTitle{L'éphod}
\VS{6}Et ils feront l’éphod d’or, de pourpre, d'écarlate et de cramoisi, et de fin lin retors ; d'un ouvrage exquis.
\VS{7}Il aura deux épaulettes qui se joindront par les deux bouts ; et c’est ainsi qu’il sera joint.
\VS{8}La ceinture exquise dont il sera ceint, et qui sera par-dessus, sera de même ouvrage, et tirée de lui, étant d'or, de  de pourpre, d'écarlate, de cramoisi, et de fin lin retors.
\VS{9}Et tu prendras deux pierres d'onyx, et tu graveras sur elles les noms des fils d'Israël :
\VS{10}Six de leurs noms sur une pierre et les six noms des autres sur l'autre pierre, selon leur naissance.
\VS{11}Tu graveras sur les deux pierres les noms des fils d’Israël, comme on grave les pierres et les cachets, tu les entoureras de montures d’or.
\VS{12 Et tu mettras les deux pierres sur les épaulettes de l'éphod, afin qu'elles soient des pierres de mémorial pour les fils d'Israël ; car Aaron portera leurs noms sur ses deux épaules devant Yahweh, pour mémorial.
\VS{13}Tu feras aussi des montures d'or,
\VS{14}et deux chaînettes d’or pur que tu tresseras en forme de cordons, et tu fixeras aux montures les chaînettes ainsi tressées.
\TextTitle{Le pectoral}
\VS{15}Tu feras aussi le pectoral du jugement d'un ouvrage exquis, comme l'ouvrage de l'éphod, d'or, de pourpre, d'écarlate, de cramoisi, et de fin lin retors.
\VS{16}Il sera carré et double ; et sa longueur sera d’un empan, et sa largeur d'un empan.
\VS{17Et tu le rempliras de garniture de pierres, à quatre rangées de pierres précieuses. A la première rangée, on mettra une sardoine, une topaze, et une émeraude.
\VS{18}Et à la seconde rangée, une escarboucle, un saphir, et un jaspe.
\VS{19}Et à  la troisième rangée, une opale, une agate, et une améthyste.
\VS{20}Et à la quatrième rangée, un chrysolithe, un onyx et un béryl, qui seront enchâssés dans de l'or, selon leur garniture.
\VS{21}Et ces pierres-là seront selon les noms des fils d'Israël, douze selon leurs noms, chacune d'elles gravées comme des cachets, selon le nom qu'elle doit porter, et elles seront pour les douze tribus.
\VS{22}Tu feras donc pour le pectoral des chaînettes d’or pur, tressées en forme de cordon.
\VS{23}Et tu feras sur le pectoral deux anneaux d'or, et tu mettras les deux anneaux aux deux bouts du pectoral.
\VS{24}Et tu mettras les deux chaînettes d'or, faites en cordon, dans les deux anneaux à l'extrémité du pectoral.
\VS{25}Et tu mettras les deux autres bouts des deux chaînettes en cordon sur les deux montures, et tu les mettras sur les épaulettes de l'éphod, sur le devant de l'éphod.
\VS{26}Tu feras aussi deux autres anneaux d'or, que tu mettras aux deux autres bouts du pectoral, sur le bord qui sera du côté de l'éphod à l’intérieur.
\VS{27}Et tu feras deux autres anneaux d'or, que tu mettras aux deux épaulettes de l'éphod par le bas, sur le devant, à l'endroit où il se joint, au-dessus de la ceinture exquise de l'éphod.
\VS{28}Et ils joindront le pectoral élevé par ses anneaux, aux anneaux de l'éphod, avec un cordon de pourpre, afin qu'il tienne au-dessus de la ceinture exquise de l'éphod, et que le pectoral ne puisse pas se séparer de l'éphod.
\VS{29}Ainsi, Aaron portera sur son cœur les noms des fils d'Israël gravés sur le pectoral du jugement, quand il entrera dans le lieu saint, pour mémorial devant Yahweh continuellement.
\TextTitle{L'urim et le thummim}
\VS{30}Et tu mettras sur le pectoral de jugement l'urim et le thummim\FTNT{L'urim («~lumières~») et le thummim («~perfections~») étaient deux pierres du pectoral que l’on utilisait ensemble pour déterminer la décision de Dieu sur certaines questions.}, qui seront sur le cœur d'Aaron, quand il viendra devant Yahweh ; et Aaron portera le jugement des fils d'Israël sur son cœur devant Yahweh continuellement.
\TextTitle{La robe de l'éphod}
\VS{31}Tu feras aussi la robe de l'éphod entièrement de pourpre.
\VS{32}Il y aura, au milieu, une ouverture pour la tête, et cette ouverture aura tout autour un bord tissé, comme l’ouverture d’une cotte de mailles, afin que la robe ne se déchire pas.
\VS{33}Tu feras à ses bords des grenades de pourpre, d'écarlate, et de cramoisi tout autour, et des clochettes d'or entre elles tout autour.
\VS{34}Une clochette d'or, puis une grenade, une clochette d'or, puis une grenade, aux bords de la robe tout autour.
\VS{35}Et Aaron en sera revêtu quand il fera le service, et on en entendra le son lorsqu'il entrera dans le lieu saint devant Yahweh, et quand il en sortira, afin qu'il ne meure pas.
\TextTitle{La lame d’or gravée : La sainteté à Yahweh}
\VS{36}Et tu feras une lame d’or pur, sur laquelle tu graveras ces mots, comme on grave un cachet : La sainteté à Yahweh.
\VS{37}Tu l’attacheras avec un cordon de pourpre sur la tiare, sur le devant de la tiare.
\VS{38}Et elle sera sur le front d'Aaron ; et Aaron portera l'iniquité commise par les fils d’Israël, en faisant leurs saintes offrandes, elle sera continuellement sur son front devant Yahweh, pour qu’il leur soit favorable.
\TextTitle{Les vêtements de service d'Aaron et ses fils}
\VS{39}Tu feras aussi une tunique de fin lin qui s'appliquera sur le corps, et tu feras aussi la tiare de fin lin ; mais tu feras la ceinture d'ouvrage de broderie\FTNT{Ex. 39:1-32.}.
\VS{40}Tu feras aussi aux fils d'Aaron des tuniques, des ceintures, et des bonnets, pour leur gloire et leur ornement.
\VS{41}Et tu en revêtiras Aaron ton frère, et ses fils avec lui ; tu les oindras, tu les consacreras et tu les sanctifieras ; puis ils m'exerceront la sacrificature\FTNT{Lé. 8:12 ; Lé. 16:32 ; No. 3:3.}.
\VS{42} Et tu leur feras des caleçons de lin, pour couvrir leur nudité, qui tiendront depuis les reins jusqu'au bas des cuisses.
\VS{43}Et Aaron et ses fils seront ainsi habillés quand ils entreront au tabernacle d'assignation, ou quand ils approcheront de l'autel pour faire le service dans le lieu saint ; et ils ne porteront point la peine d'aucune iniquité, et ne mourront point. Ce sera une ordonnance perpétuelle pour lui et pour sa postérité après lui.
\Chap{29}
\TextTitle{Les sacrificateurs consacrés au service de Yahweh}
\VerseOne{}Or c'est ici ce que tu leur feras, quand tu les sanctifieras pour m'exercer la sacrificature : Prends un veau du troupeau, et deux béliers sans tare\FTNT{Lé. 8:2 ; Lé. 9:2 ; Hé. 7:26-28.} ;
\VS{2}et des pains sans levain, et des gâteaux sans levain pétris à l'huile, et des beignets sans levain, oints d'huile ; et tu les feras de fine farine de froment\FTNT{Lé. 6:13.}.
\VS{3}Tu les mettras dans une corbeille, et tu les présenteras dans la corbeille ; tu présenteras aussi le veau et les deux moutons.
\VS{4}Puis, tu feras approcher Aaron et ses fils à l'entrée de la tente d'assignation, et tu les laveras avec de l'eau\FTNT{Ex. 40:12.}.
\VS{5}Ensuite, tu prendras les vêtements, et tu feras vêtir à Aaron la tunique et la robe de l'éphod, l'éphod et pectoral, et tu le ceindras par-dessus avec la ceinture exquise de l'éphod.
\VS{6}Puis, tu mettras sur sa tête la tiare, et la couronne de sainteté sur la tiare.
\VS{7}Et tu prendras l'huile d'onction et la répandras sur sa tête ; et tu l'oindras ainsi.
\VS{8}Puis, tu feras approcher ses fils, et tu leur feras vêtir les tuniques.
\VS{9}Et tu les ceindras des ceintures, Aaron, dis-je, et ses fils\FTNT{Es. 11:5 ; Ep. 6:14.}, et tu leur attacheras des bonnets, et ils posséderont la sacrificature par ordonnance perpétuelle. Et tu consacreras ainsi Aaron et ses fils.
\VS{10}Et tu feras approcher le veau devant la tente d'assignation, et Aaron et ses fils poseront leurs mains sur la tête du veau.
\VS{11}Et tu égorgeras le veau devant Yahweh, à l'entrée de la tente d'assignation.
\VS{12}Puis, tu prendras du sang du veau, et le mettras avec ton doigt sur les cornes de l'autel, et tu répandras tout le reste du sang au pied de l'autel.
\VS{13}Tu prendras aussi toute la graisse qui couvre les entrailles, et le lobe du foie, les deux rognons, la graisse qui les entoure, et tu les feras fumer sur l'autel.
\VS{14}Mais tu brûleras au feu la chair du veau, sa peau, et ses excréments, hors du camp. C'est un sacrifice pour le péché\FTNT{Lé. 1:3-13 ; Hé. 13:11 ; Hé. 9:11.}.
\VS{15}Puis, tu prendras l'un des béliers, et Aaron et ses fils poseront leurs mains sur la tête du bélier.
\VS{16}Puis, tu égorgeras le bélier, et prenant son sang, tu le répandras sur l'autel tout autour.
\VS{17}Après, tu couperas le bélier par pièces, et ayant lavé ses entrailles et ses jambes, tu les mettras sur ses pièces et sur sa tête.
\VS{18} Et tu feras fumer tout le bélier sur l'autel ; c'est un holocauste à Yahweh, c’est un sacrifice consumé par le feu d’une agréable odeur à Yahweh.
\VS{19}Puis, tu prendras l'autre bélier, et Aaron et ses fils mettront leurs mains sur sa tête.
\VS{20}Et tu égorgeras le bélier, et prenant de son sang, tu le mettras sur le lobe de l'oreille droite d'Aaron, et sur le lobe de l'oreille droite de ses fils, sur le pouce de leur main droite, sur le gros orteil de leur pied droit, et tu répandras le reste du sang sur l'autel tout autour.
\VS{21}Et tu prendras du sang qui sera sur l'autel, de l'huile d'onction, et tu en feras l’aspersion sur Aaron, sur ses vêtements, sur ses fils, et sur les vêtements de ses fils avec lui. Ainsi, lui, ses vêtements, ses fils, et les vêtements de ses fils, seront sanctifiés avec lui.
\VS{22}Tu prendras aussi la graisse du bélier, la queue, et la graisse qui couvre les entrailles, le grand lobe du foie, les deux rognons, la graisse qui est dessus, et l'épaule droite ; car c'est le bélier de consécration.
\VS{23}Tu prendras aussi un pain, un gâteau à l'huile, et un beignet dans la corbeille où seront ces choses sans levain, laquelle sera devant Yahweh.
\VS{24}Et tu mettras toutes ces choses sur les mains d’Aaron et sur les mains de ses fils, et tu les agiteras de côté et d’autre devant Yahweh\FTNT{No. 6:19.}.
\VS{25}Puis, les recevant de leurs mains, tu les feras fumer sur l'autel, sur l'holocauste, pour être une odeur agréable devant Yahweh ; c'est un sacrifice consumé par le feu à Yahweh.
\TextTitle{La part des sacrificateurs}
\VS{26}Tu prendras aussi la poitrine du bélier des consécrations, qui est pour Aaron, et tu l’agiteras de côté et d’autre en offrande agitée devant Yahweh. Ce sera ta part.
\VS{27}Tu sanctifieras la poitrine et l’épaule du bélier qui aura servi à la consécration d’Aaron et de ses fils, la poitrine en l’agitant de côté et d’autre, l’épaule en la présentant par élévation\FTNT{Lé. 10:14 ; No. 18:18.}.
\VS{28}Ceci sera une ordonnance perpétuelle pour Aaron et pour ses fils, qui sera offerte par les fils d'Israël ; car c'est une offrande élevée. Quand il y aura une offrande élevée de celles qui sont faites par les fils d'Israël, de leurs offrandes de paix, leur offrande élevée sera à Yahweh.
\VS{29}Les vêtements consacrés qui seront pour Aaron, seront pour ses fils après lui, afin qu'ils soient oints et consacrés dans ces vêtements.
\VS{30}Ils seront portés pendant sept jours par celui de ses fils qui lui succédera dans le sacerdoce, et qui entrera dans la tente d’assignation pour faire le service dans le lieu saint.
\VS{31}Tu prendras le bélier des consécrations, et tu feras bouillir sa chair dans un lieu saint.
\VS{32}Aaron et ses fils mangeront à l'entrée de la tente d'assignation la chair du bélier et le pain qui sera dans la corbeille.
\VS{33}Ils mangeront donc ces choses, par lesquelles la propitiation aura été faite, pour les consacrer et les sanctifier ; mais l'étranger n'en mangera point, parce qu'elles sont saintes.
\VS{34}S'il y a des restes de la chair des consécrations et du pain jusqu’au matin, tu brûleras ces restes-là au feu ; on n'en mangera point, parce que c'est une chose sainte.
\VS{35}Tu feras donc ainsi à Aaron et à ses fils, selon toutes les choses que je t'ai ordonnées ; tu les consacreras durant sept jours\FTNT{Lé. 8:31-35.}.
\VS{36}Et tu offriras comme sacrifice pour l'expiation tous les jours un veau pour faire l'expiation, et tu purifieras l'autel par cette propitiation, et tu l'oindras pour le sanctifier\FTNT{Ez. 43:19-20.}.
\VS{37}Pendant sept jours, tu feras propitiation pour l'autel, et tu le sanctifieras ; l'autel sera une chose très sainte ; tout ce qui touchera l'autel sera saint\FTNT{No. 28:3.}.
\TextTitle{L’holocauste perpétuel}
\VS{38}Or c'est ici ce que tu feras sur l'autel : Tu offriras chaque jour continuellement deux agneaux d'un an.
\VS{39}Tu sacrifieras l'un des agneaux au matin, et l'autre agneau entre les deux soirs.
\VS{40}Avec un dixième de fine farine pétrie dans la quatrième partie d'un hin d’huile vierge, et avec une aspersion de vin de la quatrième partie d'un hin pour chaque agneau,
\VS{41}tu sacrifieras l'autre agneau entre les deux soirs, avec un gâteau comme au matin, et tu lui feras la même aspersion, en bonne odeur ; c'est un sacrifice consumé par le feu devant Yahweh.
\VS{42}Voilà l’holocauste perpétuel qui sera offert par vos descendants, à l’entrée de la tente d’assignation, devant Yahweh : C’est là que je me rencontrerai avec vous, et que je te parlerai.
\VS{43}Je me trouverai là pour les fils d'Israël, et ce lieu sera sanctifié par ma gloire.
\VS{44}Je sanctifierai donc la tente d'assignation et l'autel. Je sanctifierai aussi Aaron et ses fils, afin qu'ils m'exercent la sacrificature.
\VS{45}J'habiterai au milieu des fils d'Israël, et je serai leur Dieu.
\VS{46}Ils sauront que je suis Yahweh, leur Dieu, qui les ai tirés du pays d'Egypte, pour habiter au milieu d'eux. Je suis Yahweh leur Dieu.
\Chap{30}
\TextTitle{L’autel des parfums}
\VerseOne{}Tu feras aussi un autel pour brûler les parfums, et tu le feras de bois d’acacia.
\VS{2}Sa longueur sera d'une coudée, et sa largeur d'une coudée ; il sera carré ; mais sa hauteur sera de deux coudées, et ses cornes seront tirées de lui.
\VS{3}Tu le couvriras d’or pur, tant le dessus, que ses côtés tout autour, et les cornes. Tu lui feras une bordure d’or tout autour.
\VS{4}Tu lui feras aussi deux anneaux d'or au-dessous de la bordure, à ses deux côtés, lesquels tu mettras aux deux coins, pour y faire passer les barres qui serviront à le porter.
\VS{5}Tu feras les barres de bois d’acacia, et tu les couvriras d'or.
\VS{6}Tu les mettras devant le voile, qui est au-devant de l'arche du témoignage, à l'endroit du propitiatoire qui est sur le témoignage, où je me trouverai avec toi.
\VS{7}Aaron fera brûler sur cet autel un parfum de choses aromatiques ; il y fera brûler un parfum chaque matin, quand il préparera les lampes.
\VS{8}Quand Aaron allumera les lampes entre les deux soirs, il y brûlera aussi le parfum, c’est ainsi que l’on brûlera à perpétuité du parfum devant Yahweh parmi vos descendants\FTNT{Ex. 37:25-29 ; 2 Ch. 13:11.}.
\VS{9}Vous n'offrirez point sur cet autel aucun parfum étranger, ni d'holocauste, ni d'offrande, et vous n'y répandrez aucune libation.
\VS{10}Aaron fera une fois l'an la propitiation sur les cornes de cet autel ; il fera, dis-je, la propitiation une fois l'an sur cet autel dans vos générations, avec le sang de l'offrande pour l'expiation faite pour les propitiations. C'est une chose très sainte à Yahweh.
\TextTitle{L'offrande du rachat\FTNTT{Ex. 15:1-21 ; Ps. 107:1-2}}
\VS{11}Yahweh parla aussi à Moïse, et lui dit :
\VS{12}Quand tu feras le dénombrement des fils d'Israël, selon leur nombre, ils donneront chacun à Yahweh le rachat de sa personne, quand tu en feras le dénombrement, et il n'y aura point de plaie sur eux quand tu en feras le dénombrement\FTNT{No. 1:2.}.
\VS{13}Tous ceux qui passeront par le dénombrement donneront un demi-sicle, selon le sicle du sanctuaire, qui est de vingt guéras ; le demi-sicle donc sera l'offrande que l'on donnera à Yahweh\FTNT{Lé. 27:25 ; No. 3:47 ; Ez. 45:12.}.
\VS{14}Tous ceux qui passeront par le dénombrement, depuis l'âge de vingt ans et au-dessus, donneront cette offrande à Yahweh.
\VS{15}Le riche n'augmentera rien, et le pauvre ne diminuera rien du demi-sicle, quand ils donneront à Yahweh l'offrande pour faire le rachat de vos personnes.
\VS{16}Tu prendras donc des fils d'Israël l'argent des expiations, et tu l'appliqueras l'oeuvre de la tente d'assignation. Ce sera pour les fils d'Israël, un souvenir devant Yahweh pour faire le rachat de vos personnes.
\TextTitle{Purification par l'eau de la cuve d'airain\FTNTT{Jn. 13:3-10 ; Hé. 10:22 ; 1 Jn. 1:9}}
\VS{17}Yahweh parla encore à Moïse, en disant :
\VS{18}Fais aussi une cuve d'airain, avec sa base d'airain, pour laver. Tu la mettras entre la tente d'assignation et l'autel, et tu mettras de l'eau dedans ;
\VS{19}Aaron et ses fils y laveront leurs mains et leurs pieds.
\VS{20}Quand ils entreront dans la tente d'assignation ils se laveront avec de l'eau, afin qu'ils ne meurent point, et aussi quand ils approcheront de l'autel pour faire le service, et pour offrir des holocaustes consumés par le feu devant Yahweh.
\VS{21}Ils laveront donc leurs pieds et leurs mains, afin qu'ils ne meurent point ; ce leur sera une ordonnance perpétuelle, tant pour Aaron que pour ses fils et pour ses descendants.
\TextTitle{L'huile pour l'onction sainte\FTNTT{Jn. 4:23 ; Ep. 2:18, 5:18-19}}
\VS{22}Yahweh parla aussi à Moïse, en disant :
\VS{23}Prends des meilleurs aromates ; cinq cents sicles de myrrhe, la moitié de cinnamome odoriférant, c'est-à-dire, le poids de deux cent cinquante sicles, et du roseau aromatique deux cent cinquante sicles.
\VS{24}Cinq cents sicles de casse, selon le sicle du sanctuaire, et un hin d'huile d'olive.
\VS{25}Tu en feras de l'huile pour l'onction sainte composé selon l’art du parfumeur, ce sera l'huile de l'onction sainte.
\VS{26}Puis tu en oindras la tente d'assignation, et l'arche du témoignage.
\VS{27}La table et tous ses ustensiles, le chandelier et ses ustensiles, et l'autel du parfum,
\VS{28}L'autel des holocaustes et tous ses ustensiles, la cuve et sa base.
\VS{29}Ainsi tu les sanctifieras, et ils seront une chose très-sainte ; tout ce qui les touchera, sera saint.
\VS{30}Tu oindras aussi Aaron et ses fils, et les sanctifieras pour m'exercer la sacrificature.
\VS{31}Tu parleras aussi aux fils d'Israël, en disant : Ce sera pour moi une huile d’onction sainte, parmi vos descendants.
\VS{32}On n'en oindra point la chair d'aucun homme, et vous n'en ferez point d'autre de même composition ; elle est sainte, elle vous sera sainte.
\VS{33}Quiconque en composera une semblable, et qui en mettra sur un autre, sera retranché de son peuple.
\TextTitle{L'encens pur parfumé}
\VS{34}Yahweh dit aussi à Moïse : Prends des aromates, de la gomme, de l’ongle odorant, du galbanum, le tout préparé, et de l'encens pur, le tout à poids égal.
\VS{35}Tu en feras un parfum aromatique selon l'art du parfumeur, tu y mettras du sel ; vous le ferez pur, et ce sera pour vous une chose sainte.
\VS{36}Quand tu l'auras pilé bien menu, tu en mettras dans la tente d'assignation devant le témoignage, où je me trouverai avec toi. Ce sera pour vous une chose très-sainte.
\VS{37}Quant au parfum que tu feras, vous ne ferez point pour vous de semblable composition ; ce sera une chose sainte, pour Yahweh.
\VS{38}Quiconque en fera un semblable pour le sentir, sera retranché de son peuple.
\Chap{31}
\TextTitle{Yahweh suscite des artisans}
\VerseOne{}Yahweh parla aussi à Moïse, en disant :
\VS{2}Regarde, j'ai appelé par son nom Betsaleel, fils d'Uri, fils de Hur, de la tribu de Juda.
\VS{3}Je l'ai rempli de l'Esprit de Dieu, en sagesse, en intelligence, en science, et en toutes sortes d'ouvrages,
\VS{4}afin de planifier, de travailler l’or, l’argent et l’airain.
\VS{5}De graver les pierres à enchâsser, de travailler le bois, et d’exécuter toutes sortes d’ouvrages.
\VS{6}Et voici, je lui ai donné pour compagnon Oholiab, fils d'Ahisamac, de la tribu de Dan ; et j'ai mis de l’intelligence dans l’esprit de tous ceux qui sont habiles pour qu'ils fassent tout ce que je t’ai ordonné :
\VS{7}La tente d'assignation, l'arche du témoignage, et le propitiatoire qui doit être au-dessus, et tous les ustensiles du tabernacle ;
\VS{8}la table avec tous ses ustensiles ; le chandelier d’or pur avec tous ses ustensiles ; l'autel du parfum ;
\VS{9}l'autel de l'holocauste avec tous ses ustensiles, la cuve et sa base ;
\VS{10}les vêtements du service ; les saints vêtements pour le sacrificateur Aaron, et les vêtements de ses fils pour exercer la sacrificature ;
\VS{11}l'huile d'onction, et le parfum des choses aromatiques pour le sanctuaire, et ils feront toutes les choses que je t'ai ordonnées.
\TextTitle{Le sabbat comme signe entre Yahweh et Israël}
\VS{12}Yahweh parla encore à Moïse, en disant :
\VS{13}Toi aussi parle aux fils d'Israël, en disant : Vous garderez mes sabbats, car c'est un signe entre moi et vous, et parmi vos descendants, et l’on connaîtra que je suis Yahweh qui vous sanctifie.
\VS{14}Gardez donc le sabbat, car c’est une chose sainte. Quiconque le violera sera puni de mort ; quiconque fera une œuvre en ce jour-là sera retranché du milieu de son peuple.
\VS{15}On travaillera six jours, mais le septième jour est le sabbat, le jour du repos, consacré à Yahweh ; quiconque fera une œuvre le jour du repos sera puni de mort.
\VS{16}Les fils d'Israël garderont le sabbat pour célébrer le jour du repos, eux et leurs descendants, par une alliance perpétuelle.
\VS{17}C'est un signe entre moi et les fils d'Israël à perpétuité ; car Yahweh a fait en six jours les cieux et la terre, et il a cessé son œuvre au septième jour, et s'est reposé\FTNT{Ez. 20:12 ; Ge. 2:2.}.
\VS{18}Dieu donna à Moïse, après qu'il eut achevé de parler avec lui sur la montagne de Sinaï, les deux tables du témoignage ; tables de pierre, écrites du doigt de Dieu\FTNT{De. 9:10.}.
\Chap{32}
\TextTitle{Le culte du veau d'or}
\VerseOne{}Le peuple, voyant que Moïse tardait à descendre de la montagne, s'assembla autour d’Aaron et lui dit : Lève-toi, fais-nous un dieu qui marche devant nous, car ce Moïse, cet homme qui nous a fait monter du pays d'Egypte, nous ne savons ce qui lui est arrivé\FTNT{Ac. 7:40.}.
\VS{2}Aaron leur répondit : Mettez en pièces les anneaux d'or qui sont aux oreilles de vos femmes, de vos fils, et de vos filles, et apportez-les-moi\FTNT{Ex. 35:22.}.
\VS{3}Tout le peuple mit en pièces les anneaux d'or qui étaient à leurs oreilles, et ils les apportèrent à Aaron.
\VS{4}Il les reçut de leurs mains, forma l'or avec un burin, et en fit un veau\FTNT{Le veau d’or était adoré par les Égyptiens sous le nom d’Apis ou «~taureau sacré~». Dieu de la fertilité sous forme de taureau, il était parfois représenté avec un corps d’homme et une tête de taureau. Il symbolisait la fécondité, la force, et la puissance sexuelle. La légende raconte qu’Apis est né d’une vache qui serait une manifestation de la déesse Isis, laquelle aurait été fécondée par un rayon de soleil. À une époque tardive, le taureau servait de voyant extralucide ; c’est par la bouche des enfants qui jouaient devant son temple que l’avenir se dévoilait. Un seul taureau était choisi suivant vingt-neuf critères de sélection. Après cette sélection, il était alors vénéré comme un dieu durant toute son existence (14 ans en moyenne). A sa mort, il avait droit à un embaumement digne d’un pharaon : l’animal était momifié, mis dans un sarcophage, puis placé dans un sérapéum à Saqqarah, un tombeau commun à tous les Apis précédents. Associé à Rê et à Ptah, il commença à être représenté avec un disque solaire entre les cornes sous le nouvel empire. Vénéré depuis l’époque préhistorique jusqu’à l’époque romaine, l’hellénisme en fera un dieu syncrétique, mélange de Hadès, d’Osiris et d’Apis, donnant ainsi naissance à un nouveau dieu : Sérapis. Ps. 106:19.} en métal fondu. Ils dirent : Voici ton dieu, ô Israël, qui t'a fait monter du pays d'Egypte.
\VS{5}Lorsqu’Aaron vit cela, il bâtit un autel devant le veau ; et cria en disant : Demain, il y aura une fête solennelle à Yahweh.
\VS{6}Le lendemain, ils se levèrent de bon matin, ils offrirent des holocaustes, et présentèrent des offrandes de paix. Le peuple s'assit pour manger et pour boire, puis ils se levèrent pour se divertir\FTNT{1 Co. 10:7.}.
\TextTitle{Yahweh condamne l'idolâtrie d'Israël}
\VS{7}Yahweh dit à Moïse : Va, descends ; car ton peuple que tu as fait monter du pays d'Egypte, s'est corrompu\FTNT{De. 32:5.}.
\VS{8}Ils se sont promptement détournés de la voie que je leur avais ordonnée, ils se sont fait un veau en métal fondu, et se sont prosternés devant lui, ils lui ont offert des sacrifices, puis ont dit : Voici ton dieu, ô Israël, qui t'a fait monter du pays d'Egypte\FTNT{1 R. 12:28.}.
\VS{9}Yahweh dit encore à Moïse : J'ai regardé ce peuple, et voici, c'est un peuple au cou raide.
\VS{10}Maintenant laisse-moi, et ma colère s'embrasera contre eux, et je les consumerai ; mais je te ferai devenir une grande nation.
\TextTitle{Moïse implore Yahweh pour le peuple}
\VS{11}Alors Moïse supplia Yahweh, son Dieu, et dit : Yahweh, pourquoi ta colère s'embraserait-elle contre ton peuple, que tu as retiré du pays d'Egypte par une grande puissance et par une main forte\FTNT{Ps. 106:23.} ?
\VS{12}Pourquoi les Egyptiens diraient : C’est pour leur malheur qu’il les a fait sortir, pour les tuer sur les montagnes, et pour les consumer de dessus la terre ? Reviens de l'ardeur de ta colère, et repens-toi de ce mal que tu veux faire à ton peuple\FTNT{No. 14:11-15 ; De. 9:28.}.
\VS{13}Souviens-toi d'Abraham, d'Isaac et d'Israël, tes serviteurs, auxquels tu as juré par toi-même en leur disant : Je multiplierai votre postérité comme les étoiles des cieux, et je donnerai à votre postérité tout ce pays, dont j'ai parlé, et ils l'hériteront à jamais\FTNT{De. 34:4.}.
\VS{14}Yahweh se repentit du mal qu'il avait dit qu'il ferait à son peuple.
\TextTitle{Jugement sur le peuple}
\VS{15}Moïse regarda, et descendit de la montagne, ayant dans sa main les deux tables du témoignage, et les tables étaient écrites des deux côtés, écrites de l’un et de l’autre côté.
\VS{16}Les tables étaient l'ouvrage de Dieu, et l'écriture était l'écriture de Dieu, gravée sur les tables.
\VS{17}Josué, entendant la voix du peuple qui faisait un grand bruit, dit à Moïse : Il y a un bruit de bataille au camp.
\VS{18}Moïse répondit : Ce n’est ni un cri de vainqueurs ni un cri de vaincus ; ce que j’entends, c’est la voix de gens qui chantent.
\VS{19}Et comme il approchait du camp, il vit le veau et les danses. La colère de Moïse s'embrasa ; il jeta de ses mains les tables, et les rompit au pied de la montagne.
\VS{20}Il prit ensuite le veau qu'ils avaient fait, le brûla au feu, le réduisit en poudre et répandit cette poudre dans de l'eau, et il en fit boire aux fils d'Israël\FTNT{De. 9:17-21.}.
\VS{21}Moïse dit à Aaron : Que t'a fait ce peuple pour que tu aies fait venir sur lui un si grand péché ?
\VS{22}Aaron lui répondit : Que la colère de mon Seigneur ne s'embrase point, tu sais que ce peuple est porté au mal.
\VS{23}Ils m'ont dit : Fais-nous un dieu qui marche devant nous, car ce Moïse, cet homme qui nous a fait monter du pays d'Egypte, nous ne savons ce qui lui est arrivé.
\VS{24}Je leur ai dit : Que celui qui a de l'or, le mette en pièces ! Et ils me l'ont donné ; je l'ai jeté au feu, et ce veau en est sorti.
\VS{25}Moïse vit que le peuple s’était livré au désordre, et qu’ Aaron l'avait laissé dans ce désordre, exposé à l’opprobre parmi leurs ennemis.
\VS{26}Moïse se tenant à la porte du camp, dit : Qui est pour Yahweh ? Qu'il vienne vers moi ! Et tous les fils de Lévi s'assemblèrent vers lui.
\VS{27}Il leur dit : Ainsi parle Yahweh, le Dieu d'Israël : Que chacun mette son épée à son côté, passez et repassez de porte en porte par le camp, et que chacun de vous tue son frère, son ami, et son voisin.
\VS{28}Les fils de Lévi firent selon la parole de Moïse ; et ce jour-là il tomba parmi le peuple environ trois mille hommes.
\VS{29}Car Moïse avait dit : Consacrez aujourd'hui vos mains à Yahweh, chacun même contre son fils, et contre son frère, afin que vous attiriez aujourd'hui sur vous la bénédiction.
\TextTitle{Moïse intercède pour Israël}
\VS{30}Le lendemain Moïse dit au peuple : Vous avez commis un grand péché ; mais je monterai maintenant vers Yahweh, et peut-être je ferais propitiation pour vos péchés.
\VS{31}Moïse donc retourna vers Yahweh et dit : Hélas ! Je te prie, ce peuple a commis un grand péché, en se faisant un dieu d'or.
\VS{32}Maintenant pardonne leur péché ! Sinon, efface-moi maintenant de ton livre que tu as écrit.
\VS{33}Yahweh répondit à Moïse : C’est celui qui aura péché contre moi que j’effacerai de mon livre\FTNT{Ap. 3:5 ; Ap. 20:15 ; Ap. 21:27.}.
\VS{34}Va maintenant, conduis le peuple où je t’ai dit. Voici, mon Ange ira devant toi ; mais au jour de ma vengeance, je les punirai de leur péché.
\VS{35}Yahweh frappa le peuple, parce qu'il avait fait le veau, fabriqué par Aaron.
\Chap{33}
\TextTitle{Yahweh ne veut plus marcher avec Israël}
\VerseOne{}Yahweh dit à Moïse : Va, pars d'ici, toi et le peuple que tu as fait monter du pays d'Egypte, au pays que j'ai juré de donner à Abraham, à Isaac, et à Jacob, en disant : Je le donnerai à ta postérité.
\VS{2}J'enverrai un Ange devant toi, et je chasserai les Cananéens, les Amoréens, les Héthiens, les Phéréziens, les Héviens et les Jébusiens,
\VS{3}pour vous conduire dans ce pays où coulent le lait et le miel, mais je ne monterai point au milieu de toi, parce que tu es un peuple au cou raide, de peur que je ne te consume en chemin.
\VS{4}Lorsque le peuple entendit ces tristes nouvelles, il fut dans le deuil, et aucun d'eux ne mit ses ornements sur soi.
\VS{5}Car Yahweh avait dit à Moïse : Parle aux fils d'Israël : Vous êtes un peuple au cou raide ; si je montais un seul instant au milieu de toi, je te consumerais. Maintenant donc ôte tes ornements de dessus toi, et je saurai ce que je te ferai.
\VS{6}Les fils d'Israël se dépouillèrent de leurs ornements vers la montagne d'Horeb.
\TextTitle{Moïse dresse la tente d'assignation hors du camp}
\VS{7}Moïse prit la tente, et la dressa hors du camp, l'éloignant du campement ; il l'appela tente d'assignation ; tous ceux qui cherchaient Yahweh, sortaient vers la tente d'assignation qui était hors du camp.
\VS{8}Lorsque Moïse se rendait vers la tente, tout le peuple se levait, chacun se tenait à l'entrée de sa tente, et regardait Moïse par-derrière, jusqu’à ce qu'il soit entré dans la tente.
\VS{9}Lorsque Moïse était entré dans la tente, la colonne de nuée descendait et s'arrêtait à l’entrée de la tente, et Yahweh parlait avec Moïse.
\VS{10}Tout le peuple voyait la colonne de nuée qui s’arrêtait à l’entrée de la tente, tout le peuple se levait, et chacun se prosternait à l’entrée de sa tente.
\VS{11}Yahweh parlait à Moïse face à face, comme un homme parle avec son ami. Puis Moïse retournait au camp, mais son jeune serviteur Josué, fils de Nun, ne bougeait point de la tente\FTNT{No. 12:8 ; De. 34:10 ; Jn. 15:14-15.}.
\TextTitle{Moïse demande que Yahweh marche avec Israël}
\VS{12}Moïse donc dit à Yahweh : Regarde, tu m'as dit : Fais monter ce peuple, et tu ne m'as point fait connaître celui que tu dois envoyer avec moi ; tu as même dit : Je te connais par ton nom, et aussi, tu as trouvé grâce à mes yeux.
\VS{13}Or maintenant, je te prie, si j'ai trouvé grâce à tes yeux, fais-moi connaître ton chemin, et je te connaîtrai, afin que je trouve grâce à tes yeux ; considère aussi que cette nation est ton peuple\FTNT{Ps. 25:4.}.
\VS{14}Yahweh dit : Je marcherai moi-même avec toi, et je te donnerai du repos.
\VS{15}Moïse lui dit : Si tu ne marches pas toi-même avec nous, ne nous fais point monter d'ici.
\VS{16}Comment sera-t-il donc certain que j’ai trouvé grâce à tes yeux, moi et ton peuple ? Ne sera-ce pas quand tu marcheras avec nous, et quand nous serons distingués, moi et ton peuple, de tous les peuples qui sont sur la face de la terre ?
\VS{17}Yahweh dit à Moïse : Je ferai aussi ce que tu dis ; car tu as trouvé grâce à mes yeux, et je te connais par ton nom.
\TextTitle{Moïse veut voir la gloire de Yahweh}
\VS{18}Moïse dit aussi : Je te prie, fais-moi voir ta gloire !
\VS{19}Et Dieu dit : Je ferai passer toute ma bonté devant ta face, et je crierai le nom de Yahweh devant toi ; je fais grâce à qui je fais grâce, et j'ai compassion de celui de qui j'ai compassion\FTNT{Ro. 9:15.}.
\VS{20}Puis il dit : Tu ne pourras pas voir ma face, car nul homme ne peut me voir et vivre\FTNT{Jn. 1:18 ; Jn. 14:8-11.}.
\VS{21}Yahweh dit aussi : Voici, il y a un lieu près de moi, tu te tiendras sur le rocher\FTNT{Le rocher préfigurait Jésus-Christ, le Roc sur lequel nous devons bâtir nos vies et le fondement de l’Eglise (Ps. 18:32 ; Mt. 7:24-25 ; Mt. 16:18 ; 1 Co. 3:11). Voir commentaire Es. 8 : 13-17.}.
\VS{22}Quand ma gloire passera, je te mettrai dans un creux du rocher, et te couvrirai de ma main, jusqu’à ce que je sois passé.
\VS{23}Puis je retirerai ma main, et tu me verras par-derrière, mais ma face ne se verra point.
\Chap{34}
\TextTitle{De nouvelles tables ; la gloire de Yahweh\FTNTT{Ex. 33:18-23}}
\VerseOne{}Yahweh dit à Moïse : Taille-toi deux tables de pierre comme les premières, et j'écrirai sur elles les paroles qui étaient sur les premières tables que tu as rompues\FTNT{De. 10:1.}.
\VS{2}Sois prêt au matin, et monte au matin sur la montagne de Sinaï, et présente-toi là devant moi sur le haut de la montagne.
\VS{3}Mais que personne ne monte avec toi, et même que personne ne paraisse sur toute la montagne ; et que ni menu ni gros bétail ne paisse sur cette montagne\FTNT{Ex. 19:12-13 ; Ex. 19:13.}.
\VS{4}Moïse tailla deux tables de pierre comme les premières, se leva de bon matin, et monta sur la montagne de Sinaï, comme Yahweh le lui avait ordonné, et il prit dans sa main les deux tables de pierre.
\VS{5}Yahweh descendit dans la nuée, s'arrêta là avec lui, et cria le nom de Yahweh.
\VS{6}Yahweh passa devant lui et s’écria : Yahweh, Yahweh, le Dieu miséricordieux et compatissant, lent à la colère, riche en bonté et en fidélité\FTNT{No. 14:18 ; 2 Ch. 30:9 ; Né. 9:17 ; Ps. 103:8.}.
\VS{7}Qui conserve sa miséricorde jusqu’à mille générations, qui pardonne l'iniquité, le crime, et le péché, qui ne tient point le coupable pour innocent, et qui punit l'iniquité des pères sur les fils, et sur les fils des fils, jusqu’à la troisième et à la quatrième génération\FTNT{Ex. 20:6 ; De. 5:10 ; Jé. 32:18.}.
\VS{8}Aussitôt Moïse s’inclina à terre et adora.
\VS{9}Il dit : Seigneur ! Je te prie, si j'ai trouvé grâce à tes yeux, que le Seigneur marche maintenant au milieu de nous, car c'est un peuple au cou raide. Pardonne donc nos iniquités et notre péché, et prends-nous pour ta possession.
\TextTitle{Yahweh renouvelle ses promesses\FTNTT{Ex. 33:18-23}}
\VS{10}Il répondit : Voici, moi qui traite alliance devant tout ton peuple, je ferai des merveilles qui n'ont point été faites dans aucun pays ni chez aucune nation. Tout le peuple qui t’environne verra l'œuvre de Yahweh, et c’est par toi que j’accomplirai des choses terribles.
\VS{11}Garde soigneusement ce que je t’ordonne aujourd'hui. Voici, je chasserai devant toi les Amoréens, les Cananéens, les Héthiens, les Phéréziens, les Héviens, et les Jébusiens.
\VS{12}Garde-toi de traiter alliance avec les habitants du pays où tu dois entrer, de peur qu’ils ne soient un piège pour toi\FTNT{De. 7:2 ; Jos. 23:12-13 ; 2 Co. 6:14.}.
\VS{13}Vous démolirez leurs autels, vous briserez leurs statues, et vous abattrez leurs idoles\FTNT{Littéralement «~Asherah~». Ce mot est cité au moins quarante fois dans le Tanakh. Il fait référence à un objet en bois utilisé dans le culte de la parèdre de Baal. Manassé, roi de Juda, introduisit l’emblème d’Asherah dans le temple (2 R. 21:1-7) en dépit de l’interdiction formelle de Yahweh (De. 16:21). Il n’en fut enlevé que lors des réformes de Josias et d’Ezéchias (2 R. 18:3-4 ; 2 R. 23:6-14). Asherah correspond à Astarté (la déesse Ischtar des Babyloniens, l’Aphrodite des Grecs, la Vénus des Romains, la Reine du ciel mentionnée dans Jé. 7:18 ; Jé. 44:15-30. Voir aussi  Jg. 2:13 ; Jg. 10:6 ; 1 S. 31:10 ; 1 R. 11:5-33 ; 2 R. 23:13, où elle est appelée «~l’abomination des Sidoniens~). Elle correspond aussi à la Diane des Ephésiens (Ac. 19:23-40), à la mère de Dieu de l’église romaine, à l’Isis des Egyptiens. Certains Hébreux associaient Asherah à l’épouse de Yahweh. Pourtant, Yahweh,  le Dieu d’Abraham est Un (De. 6:4). Nous comprenons alors pourquoi Dieu demandait à ce que cette idole soit détruite (De. 16:21).}.
\VS{14}Tu ne te prosterneras point devant un autre dieu, parce que Yahweh se nomme le Dieu jaloux ; c'est le Dieu qui est jaloux.
\VS{15}Garde-toi de faire alliance avec les habitants du pays, de peur que, se prostituant à leurs dieux et leur offrant des sacrifices, ils ne t’invitent, et que tu ne manges de leurs victimes ;
\VS{16}de peur que tu ne prennes leurs filles pour tes fils, lesquelles se prostituant à leurs dieux, n’entraînent tes fils à se prostituer à leurs dieux.
\VS{17}Tu ne te feras aucun dieu de métal fondu.
\TextTitle{Les fêtes et le sabbat\FTNTT{Lé. 23:4-44}}
\VS{18}Tu garderas la fête solennelle des pains sans levain ; tu mangeras les pains sans levain pendant sept jours, comme je te l'ai ordonné, dans le mois des épis ; car c’est dans le mois des épis que tu es sorti du pays d'Egypte.
\VS{19}Tout premier-né m’appartient ; même le premier mâle qui naîtra de toutes les bêtes, tant du gros que du menu bétail.
\VS{20}Mais tu rachèteras avec un agneau ou un chevreau le premier-né d'un âne. Si tu ne le rachètes pas, tu lui couperas le cou. Tu rachèteras tout premier-né de tes fils ; et nul ne se présentera devant ma face à vide.
\VS{21}Tu travailleras six jours, mais au septième jour tu te reposeras ; tu te reposeras au temps du labourage et de la moisson.
\VS{22}Tu feras la fête solennelle des semaines au temps des premiers fruits de la moisson du froment ; et la fête solennelle de la récolte à la fin de l'année.
\VS{23}Trois fois par an, tous les mâles se présenteront devant le Seigneur Yahweh, le Dieu d'Israël.
\VS{24}Car je déposséderai les nations devant toi, et j'étendrai tes limites. Nul ne convoitera ton pays lorsque tu monteras pour comparaître trois fois par an devant Yahweh, ton Dieu.
\VS{25}Tu n'offriras point le sang de mon sacrifice avec du pain levé ; on ne gardera rien du sacrifice de la fête solennelle de la Pâque jusqu’au matin.
\VS{26}Tu apporteras les prémices des premiers fruits de la terre dans la maison de Yahweh, ton Dieu. Tu ne feras point cuire le chevreau dans le lait de sa mère.
\VS{27}Yahweh dit aussi à Moïse : Ecris ces paroles ; car c’est conformément à ces paroles que j'ai traité alliance avec toi et avec Israël.
\VS{28}Moïse demeura là avec Yahweh quarante jours et quarante nuits, sans manger de pain et sans boire d'eau ; et Yahweh écrivit sur les tables les paroles de l'alliance, c'est-à-dire, les dix paroles.
\TextTitle{La gloire de Yahweh sur le visage de Moïse}
\VS{29}Moïse descendit de la montagne de Sinaï ayant les deux tables du témoignage dans sa main. En descendant de la montagne, il ne s'aperçut point que la peau de son visage était devenue resplendissante pendant qu'il parlait avec Dieu.
\VS{30}Aaron et tous les fils d'Israël ayant vu Moïse, et s'étant aperçus que la peau de son visage était resplendissante, craignirent de s’approcher de lui.
\VS{31}Moïse les appela, et Aaron et tous les principaux de l'assemblée retournèrent vers lui ; et Moïse parla avec eux.
\VS{32}Après quoi, tous les fils d'Israël s'approchèrent, et il leur donna tous les ordres qu’il avait reçus de Yahweh sur la montagne de Sinaï.
\VS{33}Lorsque Moïse acheva de leur parler, il mit un voile sur son visage.
\VS{34}Et quand Moïse entrait devant Yahweh pour parler avec lui, il ôtait le voile jusqu’à ce qu'il sortît ; et étant sorti, il disait aux fils d'Israël ce qui lui avait été ordonné.
\VS{35}Les fils d'Israël regardaient le visage de Moïse et voyaient que la peau de son visage rayonnait. C'est pourquoi Moïse remettait le voile sur son visage, jusqu’à ce qu'il entrât pour parler avec Yahweh.
\Chap{35}
\TextTitle{Rappels sur le sabbat}
\VerseOne{}Moïse donc assembla toute la congrégation des fils d'Israël, et leur dit : Ce sont ici les choses que Yahweh a ordonné de faire.
\VS{2}On travaillera six jours, mais le septième jour sera une chose sainte pour vous ; c'est le sabbat, le jour du repos consacré à Yahweh ; quiconque travaillera ce jour-là sera puni de mort.
\VS{3}Vous n'allumerez point de feu dans aucune de vos demeures le jour du repos.
\TextTitle{Les offrandes pour le tabernacle\FTNTT{Ex. 25:1-8}}
\VS{4}Puis Moïse parla à toute l'assemblée des fils d'Israël et leur dit : C'est ici ce que Yahweh vous a ordonné, en disant :
\VS{5}Prenez sur ce qui vous appartient une offrande pour Yahweh. Tout homme, dont le cœur est bien disposé, apportera cette offrande pour Yahweh, de l'or, de l'argent, de l'airain\FTNT{Ex. 25:2 ; 2 Co. 8:12.},
\VS{6}de la pourpre, de l'écarlate, du cramoisi, du fin lin, du poil de chèvre,
\VS{7}des peaux de béliers teintes en rouge, des peaux de dauphins, du bois d’acacia,
\VS{8}de l'huile pour le chandelier, des aromates pour l'huile d'onction et pour le parfum odoriférant,
\VS{9}des pierres d'onyx, et d’autres pierres pour la garniture de l'éphod et pour le pectoral.
\VS{10}Que tous ceux d’entre vous qui ont de l’habileté viennent et exécutent tout ce que Yahweh a ordonné.
\VS{11}Le tabernacle, sa tente, et sa couverture, ses agrafes, ses planches, ses barres, ses colonnes, et ses bases ;
\VS{12}l'arche et ses barres, le propitiatoire et le voile pour le couvrir ;
\VS{13}la table et ses barres, et tous ses ustensiles, et le pain de proposition ;
\VS{14}le chandelier et ses ustensiles, ses lampes et l'huile pour le chandelier ;
\VS{15}l'autel du parfum et ses barres ; l'huile d'onction, le parfum odoriférant, le rideau de la porte pour l'entrée du tabernacle ;
\VS{16}l'autel de l'holocauste, sa grille d'airain, ses barres et tous ses ustensiles ; la cuve avec sa base ;
\VS{17}les toiles du parvis, ses colonnes, ses bases, et le rideau de la porte du parvis ;
\VS{18}les pieux du tabernacle, les pieux du parvis et leur cordage ;
\VS{19}les vêtements d’office pour le service dans le sanctuaire, les vêtements sacrés pour le sacrificateur Aaron, et les vêtements de ses fils pour exercer la sacrificature.
\VS{20}Toute l'assemblée des fils d'Israël sortit de la présence de Moïse.
\VS{21}Tous ceux qui furent entraînés par le cœur et animés de bonne volonté, vinrent et apportèrent l'offrande de Yahweh pour l’œuvre de la tente d'assignation, pour tout son service, et pour les vêtements sacrés.
\VS{22}Les hommes vinrent aussi bien que les femmes. Tous ceux dont le cœur était bien disposé, apportèrent des boucles, des bagues, des anneaux, des bracelets, toutes sortes d’objets d'or ; chacun agita l’offrande d'or à Yahweh.
\VS{23}Tous ceux chez qui furent trouvé des étoffes teintes en bleu, en pourpre, en cramoisi, du fin lin et du poil de chèvre, des peaux de béliers teintes en rouge et des peaux de dauphins, les apportèrent.
\VS{24}Tous ceux qui élevèrent une offrande d'argent et d'airain, l’apportèrent pour l'offrande de Yahweh ; tous ceux aussi chez qui fut trouvé du bois d’acacia pour tout l'ouvrage du service, l'apportèrent.
\VS{25}Toute femme adroite fila de sa main et apporta ce qu'elle avait filé : De la pourpre, de l'écarlate, du cramoisi, et du fin lin\FTNT{Pr. 31:19.}.
\VS{26}Toutes les femmes dont le cœur était bien disposé, et qui avaient de l’habileté, filèrent du poil de chèvre.
\VS{27}Les principaux de l'assemblée apportèrent des pierres d'onyx, et d’autres pierres pour la garniture de l'éphod et du pectoral ;
\VS{28}des aromates, et de l'huile pour le chandelier, pour l'huile d'onction, et pour le parfum odoriférant.
\VS{29}Tous les fils d’Israël, hommes et femmes, dont le cœur était disposé à contribuer à l’œuvre que Yahweh avait ordonnée par le moyen de Moïse, apportèrent volontairement des présents à Yahweh.
\TextTitle{Betsaleel et Oholiab oints pour l’œuvre du tabernacle}
\VS{30}Moïse dit aux fils d'Israël : Sachez que Yahweh a appelé par son nom Betsaleel, fils d'Uri, fils de Hur, de la tribu de Juda.
\VS{31}Il l'a rempli de l'Esprit de Dieu, de sagesse, d’intelligence, de science, pour toutes sortes d'ouvrages.
\VS{32}Il l’a rendu capable de faire des inventions, de travailler l’or, l’argent et l’airain ;
\VS{33}de graver les pierres à enchâsser, de travailler le bois, d’exécuter toutes sortes d’ouvrages d’art.
\VS{34}Il lui a donné aussi l’intelligence pour enseigner, tant à lui qu'à Oholiab, fils d'Ahisamac, de la tribu de Dan.
\VS{35}Il les a remplis de sagesse et d’intelligence pour exécuter tous les ouvrages de sculpture et d’art, pour broder et tisser les étoffes teintes en bleu, en pourpre, en cramoisi, et en fin lin, pour faire toutes espèces de travaux et d’inventions\FTNT{Es. 28:26.}.
\Chap{36}
\TextTitle{Construction du tabernacle d'après le modèle donné par Yahweh\FTNTT{Ex. 36-39}}
\VerseOne{}Betsaleel et Oholiab, et tous les hommes d'esprit auxquels Yahweh avait donné de la sagesse et de l'intelligence pour savoir et pour faire, exécutèrent les ouvrages destinés au service du sanctuaire, selon tout ce que Yahweh avait ordonné.
\VS{2}Moïse donc appela Betsaleel et Oholiab, et tous les hommes d'esprit, dans le cœur desquels Yahweh avait mis de la sagesse, tous ceux dont le cœur était disposé à s’appliquer à l’œuvre pour l’exécuter.
\VS{3}Ils emportèrent devant Moïse toute l'offrande que les fils d'Israël avaient apportée pour faire l'ouvrage du service du sanctuaire. On apportait encore chaque matin des offrandes volontaires.
\VS{4}Alors tous les hommes d’esprit, occupés à tous les travaux du lieu saint, quittèrent chacun l’ouvrage qu’ils faisaient,
\VS{5}et parlèrent à Moïse, en disant : Le peuple ne cesse d'apporter plus qu'il ne faut pour le service et pour l'ouvrage que Yahweh a ordonné de faire.
\VS{6}Par l’ordre de Moïse, on cria dans le camp : Que personne, homme ou femme, ne s’occupe plus du service pour l'offrande pour le lieu saint ; et ainsi on empêcha le peuple d'offrir.
\VS{7}Les objets suffisaient, et au-delà, pour tous les ouvrages à faire.
\TextTitle{Les tapis de fin lin}
\VS{8}Tous les hommes habiles d'entre ceux qui faisaient l'ouvrage, firent le tabernacle avec dix tapis de fin lin retors, de pourpre, d'écarlate, et de cramoisi ; on y représenta des chérubins artistement travaillés.
\VS{9}La longueur d'un tapis était de vingt-huit coudées, et la largeur du même tapis de quatre coudées ; tous les tapis avaient une même mesure\FTNT{Ex. 26:1-6.}.
\VS{10}Ils joignirent cinq tapis l'un à l'autre, et cinq autres tapis l'un à l'autre.
\VS{11}Ils firent des lacets bleus au bord du tapis terminant le premier assemblage ; on fit de même au bord du tapis terminant le second assemblage.
\VS{12}On mit cinquante lacets au premier tapis, et l’on mit cinquante lacets au bord du tapis terminant le second assemblage ; ces lacets se correspondaient les uns aux autres.
\VS{13}On fit cinquante agrafes d'or, et on attacha les tapis l'un à l'autre avec les agrafes ; et le tabernacle forma un tout.
\TextTitle{Les tapis de poil de chèvre}
\VS{14}Puis on fit des tapis de poil de chèvre, pour servir de tente sur le tabernacle ; on fit onze de ces tapis.
\VS{15}La longueur d'un tapis était de trente coudées, et la largeur du même tapis de quatre coudées ; et les onze tapis étaient d'une même mesure.
\VS{16}On assembla cinq de ces tapis à part, et six tapis à part.
\VS{17}On fit aussi cinquante lacets sur le bord de l'un des tapis, au dernier qui était attaché, et cinquante lacets sur le bord de l'autre tapis qui était attaché.
\VS{18}On fit aussi cinquante agrafes d'airain pour assembler la tente, afin qu’elle forme un tout.
\TextTitle{Les couvertures de peaux de béliers et de dauphins}
\VS{19}On fit pour la tente une couverture de peaux de béliers teintes en rouge, et une couverture de peaux de dauphins par-dessus.
\TextTitle{Les planches et leurs bases}
\VS{20}On fit pour le tabernacle des planches de bois d’acacia, qu'on fit tenir debout.
\VS{21}La longueur d’une planche était de dix coudées, et la largeur de la même planche d'une coudée et demie.
\VS{22}Il y avait deux tenons à chaque planche, joints l'un à l'autre ; on fit la même chose pour toutes les planches du tabernacle.
\VS{23}On fit les planches pour le tabernacle ; vingt planches au côté qui regardaient directement vers le sud.
\VS{24}Au-dessous des vingt planches, on fit quarante bases d'argent, deux bases sous une planche, pour ses deux tenons, et deux bases sous l'autre planche, pour ses deux tenons.
\VS{25}On fit aussi vingt planches pour le second côté du tabernacle, du côté nord,
\VS{26}et leurs quarante bases d'argent : Deux bases sous une planche, et deux bases sous l'autre planche.
\VS{27}Pour le fond du tabernacle, vers l'occident, on fit six planches.
\VS{28}On fit deux planches pour les angles du tabernacle dans le fond ;
\VS{29}elles étaient doubles depuis le bas, et liées à leur sommet par un anneau ; on fit la même chose aux deux planches qui étaient aux deux angles.
\VS{30}Il y avait donc huit planches et seize bases d'argent, deux bases sous chaque planche.
\VS{31}Puis on fit cinq barres de bois d’acacia, pour les planches de l'un des côtés du tabernacle ;
\VS{32}et cinq barres pour les planches de l'autre côté du tabernacle ; et cinq barres pour les planches du tabernacle pour le fond, vers le côté de l'occident.
\VS{33}On fit la barre du milieu pour traverser les planches d’une extrémité à l’autre.
\TextTitle{Le revêtement d’or}
\VS{34}On couvrit d'or les planches, et on fit leurs anneaux d'or pour y faire passer les barres, et on couvrit d'or les barres.
\TextTitle{Les voiles intérieur et extérieur}
\VS{35}On fit aussi le voile de fil bleu, pourpre et cramoisi, et de fin lin retors ; on le fit artistement travaillé, et on y représenta des chérubins.
\VS{36}On lui fit quatre piliers de bois d’acacia, qu'on couvrit d'or, ayant leurs crochets d'or ; et on fondit pour eux quatre bases d'argent.
\VS{37}On fit aussi à l'entrée de la tente un rideau de fil bleu, pourpre et cramoisi, et de fin lin retors ; c’était un ouvrage de broderie.
\VS{38}On fit ses cinq piliers avec leurs crochets ; on couvrit d'or leurs chapiteaux et leurs tringles ; mais leurs cinq bases étaient d'airain.
\Chap{37}
\TextTitle{L'arche de l'alliance}
\VerseOne{}Puis Betsaleel fit l'arche de bois d’acacia. Sa longueur était de deux coudées et demie, et sa largeur d'une coudée et demie, et sa hauteur d'une coudée et demie\FTNT{Ex. 23:10-31.}.
\VS{2}Il la couvrit d’or pur en dehors et en dedans, et il fit une bordure d'or tout autour.
\VS{3}Il fondit pour elle quatre anneaux d'or pour les mettre sur ses quatre coins : Deux anneaux à l'un de ses côtés, et deux autres à l'autre côté.
\VS{4}Il fit aussi des barres de bois d’acacia, et les couvrit d'or.
\VS{5}Il fit entrer les barres dans les anneaux aux côtés de l'arche, pour porter l'arche.
\TextTitle{Le propitiatoire d’or pur}
\VS{6}Il fit aussi le propitiatoire d’or pur ; sa longueur était de deux coudées et demie, et sa largeur d'une coudée et demie.
\VS{7}Il fit deux chérubins d'or ; il les fit d’or battu aux deux extrémités du propitiatoire ;
\VS{8}un chérubin à l’une des extrémités et un chérubin à l’autre extrémité.
\VS{9}Les chérubins étendaient leurs ailes en haut, couvrant de leurs ailes le propitiatoire, et se regardant l’un l’autre ; les chérubins avaient la face tournée vers le propitiatoire.
\TextTitle{La table des pains de proposition}
\VS{10}Il fit aussi la table de bois d’acacia ; sa longueur était de deux coudées, et sa largeur d'une coudée, et sa hauteur d'une coudée et demie.
\VS{11}Il la couvrit d’or pur, et il y fit une bordure d’or tout autour.
\VS{12}Il y fit à l'entour un rebord d’une largeur d’une paume de la main, sur lequel il mit une bordure d’or tout autour.
\VS{13}Il fondit pour la table quatre anneaux d'or, et il mit les anneaux aux quatre coins, qui étaient à ses quatre pieds.
\VS{14}Les anneaux étaient près du rebord, et recevaient les barres pour porter la table.
\VS{15}Il fit les barres de bois d’acacia, et les couvrit d'or pour porter la table.
\VS{16}Il fit les ustensiles qu’on devait mettre sur la table, ses plats, ses coupes, ses calices et ses tasses pour servir aux libations ; il les fit d’or pur.
\TextTitle{Le chandelier d’or pur}
\VS{17}Il fit le chandelier d’or pur. Il fit le chandelier d’or battu. Son pied, sa tige, ses calices, ses pommes et ses fleurs, étaient d’une même pièce.
\VS{18}Six branches sortaient de ses côtés, trois branches d'un côté du chandelier, et trois de l'autre côté du chandelier.
\VS{19}Il y avait sur une branche trois calices en forme d'amande, avec pommes et fleurs ; et sur une autre branche trois calices en forme d'amande, avec pommes et fleurs ; il fit la même chose aux six branches qui sortaient du chandelier.
\VS{20}Il y avait sur le chandelier quatre calices en forme d'amande, avec ses pommes et ses fleurs.
\VS{21}Il y avait une pomme sous deux branches sortant du chandelier, une pomme sous deux autres branches, et une pomme sous deux autres branches. Il en était de même pour les six branches qui sortaient du chandelier.
\VS{22}Leurs pommes et leurs branches étaient tirées de lui, et tout le chandelier était un ouvrage d'une seule pièce d’or battu et d’or pur.
\VS{23}Il fit aussi ses sept lampes, ses mouchettes, et ses vases à cendre d’or pur.
\VS{24}Il employa un talent d’or pur pour faire le chandelier avec ses ustensiles.
\TextTitle{L'autel des parfums}
\VS{25}Il fit aussi de bois d’acacia l'autel du parfum. Sa longueur était d'une coudée, et sa largeur d'une coudée. Il était carré, et sa hauteur était de deux coudées, et des cornes sortaient de l’autel\FTNT{Ex. 30:1-10.}.
\VS{26}Il couvrit d’or pur le dessus de l'autel, les côtés tout autour, et ses cornes ; il y fit une bordure d’or tout autour.
\VS{27}Il fit aussi au-dessous de la bordure deux anneaux d'or à ses deux côtés ; il en mit aux deux côtés, pour recevoir les barres qui servaient à le porter.
\VS{28}Il fit les barres de bois d’acacia, et les couvrit d'or.
\TextTitle{L’huile d’onction et le parfum}
\VS{29}Il fit aussi l'huile pour l'onction sainte, et le parfum odoriférant, pur, composé selon l’art du parfumeur.
\Chap{38}
\TextTitle{L'autel des holocaustes}
\VerseOne{}Il fit aussi de bois d’acacia l'autel des holocaustes. Sa longueur était de cinq coudées, et sa largeur de cinq coudées. Il était carré, et sa hauteur était de trois coudées\FTNT{Ex. 27:1-8.}.
\VS{2}Il fit ses cornes à ses quatre coins. Ses cornes sortaient de l’autel, et il le couvrit d'airain.
\VS{3}Il fit aussi tous les ustensiles de l'autel : Les cendriers, les pelles, les bassins, les fourchettes, et les brasiers ; il fit tous ses ustensiles d'airain.
\VS{4}Il fit pour l'autel une grille d'airain en forme de treillis, au-dessous du rebord de l'autel, depuis le bas jusqu’ à la moitié de la hauteur de l’autel.
\VS{5}Il fondit quatre anneaux aux quatre coins de la grille d'airain, pour mettre les barres.
\VS{6}Il fit les barres de bois d’acacia, et les couvrit d'airain.
\VS{7}Il fit passer les barres dans les anneaux. Il passa dans les anneaux aux côtés de l’autel les barres qui servaient à le porter. Il le fit creux, avec des planches.
\TextTitle{La cuve d'airain}
\VS{8}Il fit aussi la cuve d'airain et sa base d'airain en employant les miroirs des femmes qui s’assemblaient à l’entrée de la tente d'assignation\FTNT{Ex. 30:14-18.}.
\TextTitle{Le parvis}
\VS{9}Il fit aussi un parvis, pour le côté sud, et des toiles de fin lin retors, de cent coudées, pour le parvis.
\VS{10}Il fit d'airain leurs vingt colonnes avec leurs vingt bases, mais les crochets des colonnes et leurs tringles étaient d'argent.
\VS{11}Pour le côté nord, il fit des toiles de cent coudées, leurs vingt colonnes et leurs vingt bases étaient d'airain, mais les crochets des colonnes et leurs tringles étaient d'argent.
\VS{12}Pour le côté de l'occident, des toiles de cinquante coudées, leurs dix colonnes, et leurs dix bases. Les crochets des colonnes et leurs tringles étaient d'argent.
\VS{13}Pour le côté de l'orient, des toiles de cinquante coudées.
\VS{14}Il fit pour l'un des côtés quinze coudées de toiles, et leurs trois colonnes avec leurs trois bases.
\VS{15}Pour l'autre côté, quinze coudées de toiles, afin qu'il y en eût autant de part et d’autre de la porte du parvis, et leurs trois colonnes avec leurs trois bases.
\VS{16}Il fit toutes les toiles du parvis qui étaient tout autour de fin lin retors.
\VS{17}Il fit aussi d'airain les bases des colonnes, mais il fit d'argent les crochets des colonnes, les tringles, et leurs chapiteaux ; toutes les colonnes du parvis étaient jointes par des tringles d’argent.
\TextTitle{La porte du parvis}
\VS{18}Il fit le rideau de la porte du parvis d'ouvrage de broderie de fil bleu, de pourpre, de cramoisi et de fin lin retors, de la longueur de vingt coudées, et sa hauteur était de cinq coudées, comme la largeur des toiles du parvis.
\VS{19}Ses quatre colonnes avec leurs bases étaient d'airain, leurs crochets d'argent et leurs tringles étaient d’argent ; et leurs chapiteaux étaient couverts d'argent.
\VS{20}Tous les pieux de l’enceinte du tabernacle et du parvis étaient d'airain.
\TextTitle{Les comptes du tabernacle}
\VS{21}Voici les comptes du tabernacle, du tabernacle d’assignation, révisés d’après l’ordre de Moïse, par les soins des Lévites, sous la direction d’Ithamar, fils du sacrificateur Aaron.
\VS{22}Betsaleel, fils d'Uri, fils de Hur, de la tribu de Juda, fit toutes les choses que Yahweh avait ordonnées à Moïse.
\VS{23}Il eut pour aide Oholiab, fils d'Ahisamac, de la tribu de Dan, habile à graver, à inventer, et à broder sur les étoffes teintes en bleu, en pourpre, en cramoisi, et sur le fin lin.
\VS{24}Le total de l’or employé à l’œuvre pour tous les travaux du lieu saint, or qui fut le produit des offrandes, montait à vingt-neuf talents et sept cent trente sicles, selon le sicle du lieu saint.
\VS{25}L'argent de ceux de l'assemblée dont on fit le dénombrement se montait à cent talents et mille sept cent soixante-quinze sicles, selon le sicle du lieu saint.
\VS{26}Un demi-sicle par tête, la moitié d'un sicle selon le sicle du lieu saint. Tous ceux qui passèrent par le dénombrement depuis l'âge de vingt ans et au-dessus, furent six cent trois mille cinq cent cinquante hommes.
\VS{27}Il y eut cent talents d'argent pour fondre les bases du lieu saint, et les bases du voile, cent bases de cent talents, un talent pour chaque base.
\VS{28}Mais des mille sept cent soixante-quinze sicles, il fit les crochets et les tringles pour les colonnes, et il couvrit leurs chapiteaux.
\VS{29}L'airain des offrandes montait à soixante-dix talents et deux mille quatre cents sicles ;
\VS{30}dont on fit les bases de l’entrée de la tente d'assignation, et l'autel d'airain avec sa grille d'airain, et tous les ustensiles de l'autel ;
\VS{31}les bases autour du parvis, les bases de l’entrée du parvis, et tous les pieux de l’enceinte du tabernacle et du parvis. Avec les étoffes teintes en bleu, en pourpre et en cramoisi, on fit les vêtements d’office pour le service dans le lieu saint, et on fit les vêtements sacrés pour Aaron, comme Yahweh l’avait ordonné à Moïse.
\Chap{39}
\TextTitle{Les vêtements sacrés d'Aaron}
\VerseOne{}Ils firent aussi de pourpre, de bleu, et de cramoisi les vêtements du service, pour faire le service du lieu saint. Ils firent les vêtements sacrés pour Aaron, comme Yahweh l'avait ordonné à Moïse\FTNT{Ex. 28.}.
\VS{2}On fit donc l'éphod d'or, de pourpre, de fil bleu, de cramoisi, et de fin lin retors.
\VS{3}On étendit des lames d'or, et on les coupa en fils que l’on entrelaça dans les étoffes teintes en bleu, en pourpre et en cramoisi, et dans le fin lin ; il était artistement travaillé.
\VS{4}On y fit des épaulettes\FTNT{Voir annexe «~Les habits du souverain sacrificateur~».} qui le joignaient, et c’est ainsi qu’il était joint par ses deux extrémités.
\VS{5}La ceinture était du même travail que l’éphod et fixée sur lui. Elle était d’or, de fil bleu, pourpre et cramoisi, et de fin lin retors, comme Yahweh l'avait ordonné à Moïse.
\VS{6}On entoura de montures d’or des pierres d’onyx, sur lesquelles on grava les noms des fils d’Israël, comme on grave les cachets.
\VS{7}On les mit sur les épaulettes de l'éphod, en souvenir des fils d'Israël, comme Yahweh l’avait ordonné à Moïse.
\VS{8}On fit aussi le pectoral\FTNT{Voir annexe «~Les habits du souverain sacrificateur~».} artistement travaillé, comme l'ouvrage de l'éphod, d'or, de pourpre, de fil bleu, de cramoisi, et de fin lin retors.
\VS{9}On fit le pectoral carré et double ; sa longueur était d'une paume de la main, et sa largeur d'une paume de la main ; il était double.
\VS{10}On le garnit de quatre rangées de pierres : A la première rangée on mit une sardoine, une topaze et une émeraude.
\VS{11}A la seconde rangée une escarboucle, un saphir, et un diamant.
\VS{12}A la troisième rangée, une opale, une agate, et une améthyste.
\VS{13}A la quatrième rangée, une chrysolithe, un onyx, et un jaspe\FTNT{Ap. 21:18-19.}, enchâssées dans leur monture d’or.
\VS{14}Ainsi il y avait autant de pierres qu'il y avait de noms des fils d'Israël, douze selon leurs noms, chacune d'elles gravées comme des cachets, selon le nom qu'elle devait porter, et elles étaient pour les douze tribus.
\VS{15}On fit sur le pectoral des chaînettes d’or pur tressées en forme de cordon.
\VS{16}On fit deux montures d'or et deux anneaux d'or, on mit les deux anneaux aux deux extrémités du pectoral.
\VS{17}On passa les deux cordons d'or dans les deux anneaux, aux deux extrémités du pectoral.
\VS{18}On mit les deux autres bouts des deux cordons aux deux montures, sur les épaulettes de l'éphod, sur le devant de l'éphod.
\VS{19}On fit aussi deux autres anneaux d'or, et on les mit aux deux autres extrémités du pectoral sur son bord, qui était du côté de l'éphod à l’intérieur.
\VS{20}On fit aussi deux autres anneaux d'or, et on les mit aux deux épaulettes de l'éphod par le bas, sur le devant de l'éphod, à l'endroit où il se joignait au-dessus de la ceinture de l'éphod.
\VS{21}On attacha le pectoral par ses anneaux aux anneaux de l'éphod, avec un cordon de pourpre, afin qu'il soit au-dessus de la ceinture de l'éphod, et que le pectoral ne puisse pas se séparer de l'éphod, comme Yahweh l'avait ordonné à Moïse.
\VS{22}On fit aussi la robe de l'éphod tissée entièrement d’étoffe bleue.
\VS{23}L'ouverture était au milieu de la robe, comme l'ouverture d’une cotte de mailles ; et il y avait un ourlet à l'ouverture de la robe tout autour, afin qu'elle ne se déchire pas.
\VS{24}Aux bordures de la robe, on fit des grenades de couleur bleue, pourpre et cramoisie, en fil retors.
\VS{25}On fit aussi des clochettes d’or pur ; on mit les clochettes entre les grenades aux bordures de la robe tout autour, parmi les grenades ;
\VS{26}une clochette et une grenade ; une clochette et une grenade, sur tout le tour de la bordure de la robe, pour faire le service, comme Yahweh l'avait ordonné à Moïse.
\VS{27}On fit aussi à Aaron et à ses fils des tuniques de fin lin tissées.
\VS{28}La tiare de fin lin, les bonnets de fin lin servant de parure, et les caleçons de lin, de fin lin retors.
\VS{29}La ceinture de fin lin retors, brodée, et de couleur bleue, pourpre et cramoisie, comme Yahweh l'avait ordonné à Moïse ;
\VS{30}et la lame du diadème sacré d’or pur, sur laquelle on écrivit comme on grave un cachet : Sainteté à Yahweh.
\VS{31}On mit sur elle un cordon bleu, pour l'appliquer à la tiare par-dessus, comme Yahweh l'avait ordonné à Moïse.
\TextTitle{Le matériel pour excercer la sacrificature est prêt}
\VS{32}Ainsi fut achevé tous les ouvrages du tabernacle, de la tente d'assignation. Les fils d'Israël firent selon toutes les choses que Yahweh avait ordonnées à Moïse ; ils les firent ainsi.
\VS{33}Ils apportèrent à Moïse le tabernacle, la tente, et tous ses ustensiles, ses crochets, ses planches, ses barres, ses colonnes, et ses bases ;
\VS{34}la couverture de peaux de béliers teintes en rouge, la couverture de peaux de dauphins, et le voile de séparation devant le Saint des saints ;
\VS{35}l'arche du témoignage et ses barres, et le propitiatoire ;
\VS{36}la table avec tous ses ustensiles, et les pains de proposition\FTNT{Ex. 31:8-10.} ;
\VS{37}le chandelier d’or pur avec toutes ses lampes préparées, et tous ses ustensiles, et l'huile pour le chandelier ;
\VS{38}l'autel d'or, l'huile d'onction, le parfum odoriférant, et le rideau de l’entrée de la tente ;
\VS{39}l'autel d'airain, avec sa grille d'airain, ses barres, et tous ses ustensiles ; la cuve, et sa base ;
\VS{40}les toiles du parvis, ses colonnes, ses bases, le rideau pour l’entrée du parvis, son cordage, ses pieux, et tous les ustensiles pour le service du tabernacle, pour la tente d'assignation ;
\VS{41}les vêtements du service pour faire le service du sanctuaire, les vêtements sacrés pour le sacrificateur Aaron, et les vêtements de ses fils pour exercer la sacrificature.
\VS{42}Les fils d'Israël firent tout l'ouvrage comme Yahweh l'avait ordonné à Moïse.
\VS{43}Moïse vit tout l'ouvrage, et voici, on l'avait fait ainsi que Yahweh l'avait ordonné, on l'avait fait ainsi. Et Moïse les bénit.
\Chap{40}
\TextTitle{Moïse dresse le tabernacle}
\VerseOne{}Yahweh parla à Moïse, en disant :
\VS{2}Au premier jour du premier mois, tu dresseras le tabernacle, la tente d'assignation.
\VS{3}Tu y mettras l'arche du témoignage, au-devant de laquelle tu tendras le voile.
\VS{4}Puis tu apporteras la table et y arrangeras ce qui doit y être arrangé. Tu apporteras aussi le chandelier et allumeras ses lampes.
\VS{5}Tu mettras aussi l'autel d'or pour le parfum au-devant de l'arche du témoignage, et tu mettras le rideau à l'entrée du tabernacle.
\VS{6}Tu mettras aussi l'autel de l'holocauste devant l'entrée du tabernacle, de la tente d'assignation.
\VS{7}Tu mettras aussi la cuve entre la tente d'assignation et l'autel, et tu y mettras de l'eau.
\VS{8}Tu placeras le parvis à l’entour, et tu mettras le rideau à l’entrée du parvis.
\VS{9}Tu prendras aussi l'huile d'onction, et tu en oindras le tabernacle, et tout ce qui y est, tu le sanctifieras, avec tous ses ustensiles ; et il sera saint.
\VS{10}Tu oindras l'autel de l'holocauste, et tous ses ustensiles, tu sanctifieras l'autel, et l'autel sera très-saint.
\VS{11}Tu oindras la cuve et sa base, et la sanctifieras.
\VS{12}Tu feras approcher Aaron et ses fils à l'entrée de la tente d'assignation, et tu les laveras avec de l'eau.
\VS{13}Tu revêtiras Aaron des vêtements sacrés, tu l'oindras et le sanctifieras ; et il m'exercera la sacrificature.
\VS{14}Tu feras aussi approcher ses fils que tu revêtiras de tuniques.
\VS{15}Tu les oindras comme tu auras oint leur père ; et ils m'exerceront la sacrificature, et leur onction leur assurera la sacrificature à toujours parmi leurs descendants.
\VS{16}Moïse fit selon toutes les choses que Yahweh lui avait ordonnées ; il le fit ainsi.
\VS{17}Au premier jour du premier mois de la deuxième année, le tabernacle fut dressé.
\VS{18}Moïse dressa le tabernacle, mit ses bases, posa ses planches, mit ses barres, et dressa ses colonnes.
\VS{19}Il étendit la tente sur le tabernacle, et mit la couverture de la tente au-dessus du tabernacle par le haut, comme Yahweh l'avait ordonné à Moïse.
\VS{20}Il prit et posa le témoignage dans l'arche, et mit les barres à l'arche ; il mit aussi le propitiatoire au-dessus de l'arche.
\VS{21}Il apporta l'arche dans le tabernacle, et posa le voile du rideau, et le mit devant l'arche du témoignage, comme Yahweh l'avait ordonné à Moïse.
\VS{22}Il mit aussi la table dans la tente d'assignation, du côté nord du tabernacle, en dehors du voile.
\VS{23}Il y déposa en ordre les pains devant Yahweh, comme Yahweh l'avait ordonné à Moïse.
\VS{24}Il mit aussi le chandelier dans la tente d'assignation, en face de la table, du côté sud du tabernacle.
\VS{25}Il alluma les lampes devant Yahweh, comme Yahweh l'avait ordonné à Moïse.
\VS{26}Il posa aussi l'autel d'or dans la tente d'assignation, devant le voile.
\VS{27}Il fit brûler sur lui le parfum odoriférant, comme Yahweh l'avait ordonné à Moïse.
\VS{28}Il plaça le rideau à l'entrée du tabernacle.
\VS{29}Il plaça l'autel de l'holocauste à l'entrée du tabernacle, de la tente d'assignation ; il y offrit l'holocauste et l’offrande, comme Yahweh l'avait ordonné à Moïse.
\VS{30}Il plaça la cuve entre la tente d'assignation et l'autel, et y mit de l'eau pour se laver.
\VS{31}Moïse et Aaron avec ses fils s’y lavèrent les mains et les pieds.
\VS{32}Quand ils entraient dans la tente d'assignation, et s’approchaient de l'autel, ils se lavaient, comme Yahweh l'avait ordonné à Moïse.
\VS{33}Il dressa aussi le parvis autour du tabernacle et de l'autel, et mit le rideau à l’entrée du parvis. Ainsi Moïse acheva l'ouvrage.
\TextTitle{La gloire de Yahweh sur le tabernacle}
\VS{34}La nuée couvrit la tente d'assignation, et la gloire de Yahweh remplit le tabernacle\FTNT{No. 9:15 ; 1 R. 8:10.}.
\VS{35}Moïse ne put entrer dans la tente d'assignation, car la nuée se tenait dessus et la gloire de Yahweh remplissait le tabernacle.
\VS{36}Aussi longtemps que durèrent leurs marches, les fils d’Israël partaient, quand la nuée s’élevait de dessus le tabernacle.
\VS{37}Quand la nuée ne se levait point, ils ne partaient point, jusqu’au jour où elle se levait.
\VS{38}Car la nuée de Yahweh était de jour sur le tabernacle, et le feu y était la nuit, devant les yeux de toute la maison d'Israël, dans toutes leurs marches.
\PPE
\end{multicols}
