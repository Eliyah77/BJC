\ShortTitle{Habakuk}\BookTitle{Habakuk}\BFont
\noindent\hrulefill
{\footnotesize
\textit{
\bigskip
{\centering{}
\\Auteur : Habakuk
\\(Heb. : Chabaqquwq)
\\Signification : Embrasser, amour
\\Thème : Du doute à la foi
\\Date de rédaction : 7\up{ème} siècle av. J.-C.\\}
}
%\bigskip
\textit{
\\Habakuk, contemporain de Nahum, Sophonie et Jérémie, exerça son ministère dans le royaume de Juda. Véritable sentinelle, il fut chargé d'annoncer le châtiment de Juda par les Chaldéens. Ce récit, qui est en partie un dialogue entre Dieu et Habakuk, témoigne de la relation qui les liait. Il est aussi une invitation à la patience et à la foi en Yahweh.\bigskip
}
}
\par\nobreak\noindent\hrulefill
\begin{multicols}{2}
\Chap{1}
\TextTitle{Quand la méchanceté semble triompher de la justice}
\VerseOne{}Oracle qu'Habakuk, le prophète a vu.
\VS{2}Ô Yahweh ! Jusqu'à quand crierai-je sans que tu m'écoutes ? Jusqu'à quand crierai-je vers toi ? On me traite avec violence sans que tu me délivres !
\VS{3}Pourquoi me fais-tu voir la méchanceté\FTNT{La perplexité d'Habakuk était la même que celle de Job (Job. 21:7), d'Asaph (Ps. 73) et de Jérémie (Jé. 12:1-2). Les méchants semblent prospérer tandis que les justes pleurent et sont persécutés (Mal. 3 : 12-15).}, et vois-tu la perversité ? Pourquoi y a-t-il de l'oppression et de la violence devant moi, et des gens qui excitent des procès et des querelles ?
\VS{4}Parce que la loi est sans force, et que la justice ne se fait jamais, à cause de cela le méchant environne le juste, et à cause de cela on rend des jugements corrompus\FTNT{Jé. 5:26 ; Am. 5:7.}.
\TextTitle{La réponse de Yahweh}
\VS{5}Regardez parmi les nations, voyez, soyez étonnés et stupéfaits ! Car je vais faire en vos jours une œuvre que vous ne croiriez pas si on vous la racontait\FTNT{Ac. 13:41.}.
\VS{6}Car voici, je vais susciter les Chaldéens, ce peuple cruel et impétueux, marchant sur l'étendue de la terre, pour posséder des demeures qui ne lui appartiennent pas.
\VS{7}Il est redoutable et terrible, son gouvernement et son autorité viennent de lui-même.
\VS{8}Ses chevaux sont plus légers que les léopards, et ils ont la vue plus aiguë que les loups du soir ; et ses cavaliers se répandront çà et là, même ses cavaliers viendront de loin ; ils voleront comme un aigle qui fond sur sa proie\FTNT{Jé. 5:6 ; So. 3:3.}.
\VS{9}Ils viendront tous pour la violence ; ce qu'ils engloutiront de leurs regards sera porté vers l'orient, et ils amasseront les prisonniers comme du sable.
\VS{10}Ce peuple se moque des rois, et les princes sont l'objet de ses railleries ; il se rit de toutes les forteresses ; il amoncelle de la terre, et il s'en empare.
\VS{11}Alors il traverse comme le vent, il passe outre et se rend coupable, car sa force est son dieu.
\TextTitle{Yahweh est souverain}
\VS{12}N'es-tu pas de toute éternité, ô Yahweh ! Mon Dieu ! Mon Saint ? Nous ne mourrons point ! Ô Yahweh, tu l'as établi pour exécuter tes jugements ; et toi, mon rocher\FTNT{Voir commentaire en Es. 8:13-17.}, tu l'as fondé pour punir.
\VS{13}Tu as les yeux trop purs pour voir le mal, et tu ne saurais prendre plaisir à regarder le mal qu'on fait à autrui. Pourquoi regarderais-tu les perfides, et te tairais-tu quand le méchant dévore son prochain qui est plus juste que lui ?
\VS{14}Or tu as fait les hommes comme les poissons de la mer, et comme le reptile qui n'a point de maître.
\VS{15}Il a tout enlevé avec l'hameçon ; il l'a amassé avec son filet, et l'a assemblé dans son rets ; c'est pourquoi il se réjouira et s'égayera\FTNT{Am. 4:2.}.
\VS{16}A cause de cela, il sacrifie à son filet, et il offre de l'encens à ses rets, parce qu'il aura eu par leur moyen une grasse portion, et que sa viande est une chose moelleuse.
\VS{17}Videra-t-il à cause de cela son filet ? Et ne cessera-t-il jamais de faire le carnage des nations ?
\Chap{2}
\TextTitle{Une sentinelle}
\VerseOne{}Je me tenais en sentinelle, j'étais debout dans la forteresse et je faisais le guet, pour voir ce qu'il me dirait, et ce que je répondrais après ma plainte\FTNT{Es. 21:1-6 ; Jé. 6:17 ; Ez. 33:1-19.}.
\TextTitle{La rétribution des justes ne tardera pas}
\VS{2}Et Yahweh m'a répondu et m'a dit : Ecris la vision, et grave-la sur des tablettes, afin qu'on la lise couramment.
\VS{3}Car la vision est encore différée jusqu'à un certain temps, et Yahweh parlera de ce qui arrivera à la fin, et il ne mentira point. S'il tarde, attends-le, car il ne manquera point de venir, et il ne tardera point\FTNT{Hé. 10:37.}.
\VS{4}Voici, l'âme de celui qui s'élève n'est point droite en lui ; mais le juste vivra de sa foi\FTNT{Ro. 1:17 ; Hé. 10:38.}.
\VS{5}Et combien plus l'homme adonné au vin est-il perfide, et l'homme puissant est-il orgueilleux, ne se tenant point tranquille chez lui ; il élargit son âme comme le scheol, et il est insatiable comme la mort, il rassemble vers lui toutes les nations, et réunit à lui tous les peuples.
\VS{6}Tous ceux-là ne feront-ils pas de lui un sujet de railleries et d'énigmes ? Et ne dira-t-on pas : Malheur à celui qui accumule ce qui ne lui appartient point ; jusqu'à quand le fera-t-il, et entassera-t-il sur lui de la boue épaisse ?
\VS{7}Ne se lèveront-ils pas soudain, ceux qui te mordront ? Ne se réveilleront-ils pas pour te tourmenter ? Et tu deviendras leur proie.
\VS{8}Parce que tu as pillé beaucoup de nations, tout le reste des peuples te pillera, et aussi à cause des meurtres des hommes, et de la violence faite dans le pays, contre la ville, et contre tous ses habitants\FTNT{Es. 33:1 ; Na. 3:1.}.
\VS{9}Malheur à celui qui amasse pour sa maison des gains injustes, afin de placer son nid dans un lieu élevé, pour échapper à l'atteinte de la calamité !
\VS{10}C'est pour la confusion de ta maison que tu as pris conseil, en détruisant beaucoup de peuples, et c'est contre ton âme que tu as péché.
\VS{11}Car la pierre crie du milieu de la muraille, et de la charpente la poutre lui répond.
\VS{12}Malheur à celui qui bâtit des villes avec le sang et qui fonde des cités sur l'iniquité.
\VS{13}Voici, n'est-ce pas la volonté de Yahweh des armées que les peuples travaillent pour le feu, et que les peuples se lassent pour le néant ?
\VS{14}Car la terre sera remplie de la connaissance de la gloire de Yahweh\FTNT{Es. 11:9.}, comme le fond de la mer par les eaux qui le couvrent.
\VS{15}Malheur à celui qui fait boire son compagnon en lui approchant sa bouteille, et qui l'enivre afin qu'on voie sa nudité\FTNT{Ge. 9:21-24 ; Es. 5:22.}.
\VS{16}Tu seras rassasié de honte plutôt que de gloire ; toi aussi bois, et découvre-toi. La coupe de la droite de Yahweh fera le tour jusqu'à toi, et l'ignominie sera répandue sur ta gloire.
\VS{17}Car la violence faite au Liban retombera sur toi ; et les ravages des bêtes t'effrayeront, parce que tu as répandu le sang des hommes, et commis des violences dans le pays, contre la ville et tous ses habitants.
\VS{18}A quoi sert l'image taillée pour qu'un ouvrier la taille ? A quoi sert l'image de fonte, docteur de mensonge, à quoi sert-elle pour que l'ouvrier qui l'a faite place en elle sa confiance en fabriquant des idoles muettes ?
\VS{19}Malheur à ceux qui disent au bois : Réveille-toi ! Et à la pierre muette : Réveille-toi ! Enseignera-t-elle ? Voici, elle est couverte d'or et d'argent, et il n'y a aucun esprit au-dedans d'elle.
\VS{20}Mais Yahweh est dans le temple de sa sainteté. Toute la terre, tais-toi, redoutant sa présence.
\Chap{3}
\TextTitle{Psaume d'Habakuk}
\VerseOne{}Prière d'Habakuk, le prophète, sur le mode des chants lyriques.
\VS{2}Yahweh, j'ai entendu ce que tu m'as fait entendre, et j'ai été saisi de crainte, ô Yahweh ! Dans le cours des années, ravive ton œuvre ; dans le cours des années, fais-la connaître; dans ta colère souviens-toi de tes compassions.
\VS{3}Dieu vient de Théman, et le Saint vient du mont de Paran ; Sélah. Sa majesté couvre les cieux, et la terre est remplie de sa louange.
\VS{4}Sa splendeur est comme la lumière même, et des rayons sortent de sa main ; c'est là où réside sa force.
\VS{5}La peste marche devant lui, et une flamme ardente sort sous ses pieds.
\VS{6}Il s'arrête et mesure la terre ; il regarde et met en déroute les nations ; les montagnes antiques sont brisées en éclats, et les collines éternelles s'affaissent. Ses voies sont les voies anciennes.
\VS{7}Je vois les tentes de Cuschan accablées sous la punition ; les pavillons du pays de Madian sont ébranlés.
\VS{8}Est-ce contre les fleuves que s'irrite Yahweh ? Ta colère est-elle contre les fleuves, et ta fureur contre la mer, que tu sois monté sur tes chevaux et sur tes chars de délivrance ?
\VS{9}Ton arc est mis à nu et tire toutes les flèches, selon le serment fait aux tribus, à savoir ta parole. Sélah. Tu fends la terre et tu en fais sortir des fleuves\FTNT{Ps. 78:15-16 ; Ps. 105:41.}.
\VS{10}Les montagnes te voient et elles tremblent\FTNT{Ps. 114:4-7.}; des torrents d'eau se précipitent, l'abîme fait retentir sa voix de la profondeur, il élève ses mains en haut.
\VS{11}Le soleil et la lune s'arrêtent dans leur habitation\FTNT{Jos. 10:12 ; Ap. 22:5.}, ils marchent à la lueur de tes flèches, et à la splendeur de l'éclat de ta lance étincelante.
\VS{12}Tu marches sur la terre avec indignation, et foules les nations avec colère.
\VS{13}Tu sors pour la délivrance de ton peuple, tu sors avec ton Oint pour la délivrance ; tu transperces le chef, afin qu'il n'y en ait plus dans la maison du méchant, tu en découvres le fondement jusqu'au fond. Sélah.
\VS{14}Tu perces avec ses flèches la tête de ses chefs, quand ils viennent comme une tempête pour me dissiper ; ils s'égaient comme pour dévorer l'affligé dans sa retraite.
\VS{15}Tu marches avec tes chevaux par la mer, dans les monceaux des grandes eaux.
\VS{16}J'ai entendu ce que tu m'as déclaré, et mes entrailles en sont émues ; à ta voix le tremblement saisit mes lèvres ; la pourriture entre dans mes os, et je tremble en moi-même, car je serai en repos au jour de la détresse, lorsque montant vers le peuple, il le mettra en pièces.
\VS{17}Car le figuier ne fleurira pas, et il n'y aura point de fruit dans les vignes ; ce que l'olivier produit mentira, et aucun champ ne produira rien à manger ; les brebis seront retranchées du parc, et il n'y aura point de bœufs dans les étables.
\VS{18}Mais moi, je me réjouis en Yahweh, et je me réjouis dans le Dieu de ma délivrance.
\VS{19}Yahweh, le Seigneur, est ma force, et il rend mes pieds semblables à ceux des biches, et me fait marcher sur mes lieux élevés\FTNT{De. 32:13 ; Ps. 18:33-34.}. Au chef des chantres avec instruments à cordes.
\PPE{}
\end{multicols}
