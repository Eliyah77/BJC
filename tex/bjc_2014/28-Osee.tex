\ShortTitle{Osée}\BookTitle{Osée}\BFont
\noindent\hrulefill
{\footnotesize
\textit{
\bigskip
{\centering{}
\\(Hoshéa)
\\Signifie : Salut, Sauve
\\Thème : Israël sera rejeté à cause de son apostasie
D’autres nations seront appelées à sa place
\\Auteur : Osée
\\Date de rédaction : 8ème  siècle av. J.-C\\}
}
%\bigskip
\textit{
\\Osée, fils de Beéri, exerça son ministère dans le royaume du Nord au temps de Joas, roi d’Israël et fut contemporain des prophètes Amos, Michée et Esaïe.
%\bigskip
\\Yahweh demanda à Osée d’épouser une prostituée pour que le prophète puisse partager plus profondément son fardeau et la tristesse qu’il subissait en raison de l’infidélité du peuple qu’il aimait tant. Malgré la rébellion et les mauvais agissements d’Israël, Yahweh manifesta une fois de plus sa patience et utilisa Osée pour avertir et inviter les fils de Jacob à la repentance.\bigskip
}
}
\par\nobreak\noindent\hrulefill
\begin{multicols}{2}
\TextTitle{[Introduction]}
\Chap{1}
\VerseOne{}La Parole de Yahweh qui fut adressée à Osée fils de Béeri, au temps d’Ozias, de Jotham, d’Achaz et d’Ezéchias, rois de Juda, et au temps de Jéroboam, fils de Joas, roi d'Israël.
\TextTitle{[Mariage du prophète ; naissance de Jizreel]}
\VS{2}La première fois que Yahweh parla à Osée, Yahweh dit à Osée : Va, prends une femme prostituée, et aie d'elle des enfants de prostitution ; car le pays s’est entièrement prostitué en abandonnant Yahweh.
\VS{3}Il alla, et il prit Gomer, fille de Diblaïm. Elle conçut et lui enfanta un fils.
\VS{4}Et Yahweh lui dit : Donne-lui le nom de Jizreel ; car encore un peu de temps, et je châtierai la maison de Jéhu pour le sang versé à Jizreel, je ferai cesser le royaume de la maison d'Israël\FTNT{Cette prophétie s’est accomplie en 722 av. J-C. Voir 2 R.17.}.
\VS{5}En ce jour-là, je briserai l'arc d'Israël dans la vallée de Jizreel.
\TextTitle{[Naissance de Lo-Ruchama]}
\VS{6}Elle conçut de nouveau, et enfanta une fille. Et Yahweh lui dit : Donne-lui le nom de Lo-Ruchama ; car je n’aurai plus de pitié envers la maison d'Israël, je ne lui pardonnerai plus.
\VS{7}Mais je ferai miséricorde à la maison de Juda, et je les délivrerai par Yahweh, leur Dieu ; et je ne les délivrerai ni par l'arc, ni par l'épée, ni par les combats, ni par les chevaux, ni par les cavaliers.
\TextTitle{[Naissance de Lo-Ammi]}
\VS{8}Elle sevra Lo-Ruchama ; puis elle conçut, et enfanta un fils.
\VS{9}Et Yahweh dit : Donne-lui le nom de Lo-Ammi ; car vous n'êtes point mon peuple, et je ne suis pas votre Dieu.
\TextTitle{[Rétablissement futur d'Israël]}
\Chap{2}
\VerseOne{}Cependant, le nombre des fils d'Israël sera comme le sable de la mer, qui ne peut ni se mesurer ni se compter ; et dans la ville où il leur est dit : Vous n’êtes pas mon peuple ! On leur dira : Vous êtes les fils du Dieu vivant !
\VS{2}Aussi les fils de Juda et les fils d'Israël se rassembleront, et ils s'établiront un chef, et monteront hors du pays ; car grande sera la journée de Jizreel.
\VS{3}Dites à vos frères : Ammi ! et à vos sœurs Ruchama !
\TextTitle{[Châtiment d'Israël] (2 R. 17:1-18)}
\VS{4}Plaidez, plaidez contre votre mère, car elle n’est point ma femme, et je ne suis point son mari ! Qu’elle ôte de sa face ses prostitutions, et de son sein ses adultères !
\VS{5}Sinon, je la dépouille à nu, je l’expose comme au jour de sa naissance, je la rends semblable à un désert, à une terre aride, et je la fais mourir de soif ;
\VS{6}et je n’aurai pas pitié de ses enfants, car ce sont des enfants de prostitution.
\VS{5}Leur mère s'est prostituée, celle qui les a conçus s'est déshonorée, car elle a dit : Je m'en irai après mes amants qui me donnent mon pain et mes eaux, ma laine, et mon lin, mon huile, et mes boissons.
\VS{8}C'est pourquoi, voici, je vais fermer son chemin avec des épines, et y élever un mur, afin qu’elle ne trouve plus ses sentiers.
\VS{9}Elle poursuivra ses amants, mais ne les atteindra pas ; elle les cherchera, mais elle ne les trouvera pas. Puis elle dira : Je m'en irai, et je retournerai vers mon premier mari, car alors j'étais plus heureuse que maintenant.
\VS{10}Mais elle n'a pas reconnu que c'était moi qui lui donnais le blé, le vin, et l'huile, et l’on a fait des offrandes à Baal\FTNT{Baal : Voir Jg. 2:13.} avec l’argent et l’or que je lui prodiguais.
\VS{11}C'est pourquoi je reprendrai mon froment en son temps, mon vin en sa saison, et je retirerai la laine et le lin qui couvraient sa nudité.
\VS{12}Et maintenant je découvrirai sa honte aux yeux de ses amants, et personne ne la délivrera de ma main.
\VS{13}Je ferai cesser toute sa joie, ses fêtes, ses nouvelles lunes, ses sabbats, et toutes ses solennités.
\VS{14}Je ravagerai ses vignes et ses figuiers, dont elle disait : Voici le salaire que mes amants m'ont donné ! Je les réduirai en une forêt, et les bêtes des champs les dévoreront.
\VS{15}Je la châtierai pour les jours où elle encensait les Baals, où elle se parait de ses anneaux et de ses colliers, et s'en allait après ses amants, et m'oubliait, dit Yahweh.
\TextTitle{[Retour d'Israël symbolisé par la femme adultère revenant au foyer]}
\VS{16}Néanmoins, voici, je veux l'attirer et la mener au désert, là je parlerai à son cœur.
\VS{17}Là, je lui accorderai ses vignes et la vallée d’Accor, telle une porte d’espérance, et là, elle chantera comme au temps de sa jeunesse, et comme au jour où elle remonta du pays d'Egypte.
\VS{18}En ce jour-là, dit Yahweh, tu m'appelleras mon Mari ! et tu ne m'appelleras plus mon Maître !\FTNT{Littéralement : Baal.}
\VS{19}Car j'ôterai de sa bouche les noms des Baals, et on ne fera plus mention de leurs noms.
\VS{20}Aussi en ce temps-là, je traiterai pour eux une alliance avec les bêtes des champs, avec les oiseaux du ciel, et avec les reptiles de la terre ; je briserai et j'ôterai du pays l'arc, l'épée et la guerre, et je les ferai se reposer en sécurité.
\VS{21}Je serai ton fiancé pour toujours ; je serai ton fiancé par la justice, la droiture, la grâce et la miséricorde.
\VS{22}Je serai ton fiancé par la fidélité, et tu reconnaîtras Yahweh.
\VS{23}En ce jour-là, j’exaucerai, dit Yahweh, je témoignerai aux cieux, et les cieux exauceront la terre.
\VS{24}La terre exaucera le blé, le bon vin et l'huile ; et ils exauceront Jizreel.
\VS{25}Je planterai pour moi Lo-Ruchama dans ce pays, et je lui ferai miséricorde ; et je dirai à Lo-Ammi : Tu es mon peuple ! Et il me répondra : Mon Dieu !
\TextTitle{[Soumission d'Israël au futur royaume de David]}
\Chap{3}
\VerseOne{}Après cela Yahweh me dit : Va encore et aime une femme aimée d'un amant et adultère, aime-la comme Yahweh aime les enfants d'Israël, qui se tournent toutefois vers d'autres dieux et aiment les gâteaux de raisin.
\VS{2}J’achetai donc cette femme pour quinze pièces d'argent, un homer et demi d'orge.
\VS{3}Et je lui dis : Reste avec moi pendant plusieurs jours, ne t'abandonne plus à la prostitution, ne sois à aucun homme, et je serai fidèle envers toi.
\VS{4}Car les enfants d'Israël resteront plusieurs jours sans roi, sans chef, sans sacrifice, sans statue, sans éphod, et sans théraphim\FTNT{Cette prophétie s’est accomplie d’une manière extraordinaire à travers l’histoire du peuple d’Israël depuis la première venue de Jésus-Christ. Les Israélites étaient dispersés, sans unité politique faute de roi, empêchés d’offrir des sacrifices depuis la destruction du temple par Titus (39-81), fils de l’empereur romain Vespasien (9-79), en l’an 70.}.
\VS{5}Mais après cela, les enfants d'Israël se repentiront\FTNT{La repentance et la conversion nationale d’Israël auront lieu lors du retour du Messie ( Es. 59:20-21 ; Ro. 11:26-27). Dieu n’a pas abandonné son peuple, il viendra lui-même le délivrer.}, et chercheront Yahweh, leur Dieu, et David leur roi, ils tressailliront à la vue de Yahweh et de sa bonté dans les derniers jours\FTNT{Dans les derniers jours :Voir Ge. 49:1.}.
\TextTitle{[Israël, la nation pécheresse]}
\Chap{4}
\VerseOne{}Ecoutez la parole de Yahweh, fils d'Israël ! Car Yahweh a un procès avec les habitants du pays ; parce qu'il n'y a point de vérité, point de miséricorde, point de connaissance de Dieu dans le pays.
\VS{2}Il n'y a que parjures et mensonges, meurtres, vols et adultères ; on use de violence, on commet meurtre sur meurtre.
\VS{3}C'est pourquoi le pays sera dans le deuil, et tous ceux qui l’habitent seront languissants, et avec eux toutes les bêtes des champs et tous les oiseaux du ciel ; même les poissons de la mer périront.
\VS{4}Mais que nul ne conteste, et que nul ne se livre aux reproches ; car ton peuple est comme ceux qui disputent avec le sacrificateur.
\VS{5}Tu tomberas donc en plein jour, et le prophète aussi tombera avec toi de nuit, et j'exterminerai ta mère.
\TextTitle{[Israël dans l'ignorance]}
\VS{6}Mon peuple est détruit parce qu’il lui manque la connaissance\FTNT{Mon peuple est détruit parce qu’il lui manque la connaissance. Le verbe «~détruire~» vient de l’hébreu «~ damah ~» qui signifie aussi «~égorger~». Satan est celui qui vient égorger, dérober et détruire, notamment avec ses faux prophètes (Jn. 10 : 10). Chaque disciple de Jésus-Christ doit avoir une vie de prière et de méditation quotidienne afin de résister aux attaques de l’ennemi. Voici un extrait d’un document, conservé à la Bibliothèque Nationale de France, contenant certains conseils que les cardinaux donnèrent au pape Jules III (1487-1555) à son élection en 1550 : «~Voilà le livre  qui,  plus  qu'aucun  autre, provoquera  contre  nous  les  rébellions,  les  tempêtes  qui  ont risqué  de  nous  perdre.  En  effet,  quiconque  examine diligemment  l'enseignement  de  la  Bible  et  le  compare  à  ce qui  se  passe  dans  nos  Eglises,  trouvera  bien  vite  les contradictions  et  verra  que  nos  enseignements  s'écartent souvent  de  celui  de  la  Bible  et,  plus  souvent  encore, s'opposent  à celle-ci.  Si le  peuple  se  rend compte  de ceci, il ous provoquera jusqu'à ce que tout soit révélé et alors nous deviendrons l'objet de la dérision et de la haine universelles. Il est donc nécessaire que la Bible soit enlevée et dérobée des mains  du  peuple  avec  zèle,  toutefois  sans  provoquer  de tumulte [...] La lecture de l'Evangile ne doit être permise que le moins possible surtout en langue moderne et dans les pays soumis à votre autorité. Le très peu qui est lu généralement à la  messe  devrait  suffire  et  il  faudrait  défendre  à  quiconque d'en lire plus. Tant que le peuple se contentera de ce peu, vos intérêts prospéreront, mais dès l'instant qu'on voudra en lire plus, vos intérêts commenceront à en souffrir.~» Beaucoup de chrétiens du XXIe enseignements qu’ils reçoivent dans leurs églises à la lumière des saintes Ecritures comme le faisaient les juifs de Bérée (Ac.17:11). Pourtant, ils ont un accès facile à la Bible. Pour ne pas tomber dans le piège de l’ennemi, il est donc primordial de méditer la Parole de Dieu chaque jour (Jos. 1:8 ; Ps. 119:11).}. Parce que tu as rejeté la connaissance, je te rejetterai afin que tu n'exerces plus la sacrificature. Puisque tu as oublié la loi de ton Dieu, moi aussi j'oublierai tes enfants.
\VS{7}Plus ils se sont multipliés, plus ils ont péché contre moi : Je changerai leur gloire en ignominie.
\VS{8}Ils se nourrissent des péchés de mon peuple, leur âme soutient leur iniquité.
\VS{9}C'est pourquoi le sacrificateur sera traité comme le peuple, je le châtierai selon ses voies et je lui rendrai selon ses œuvres.
\VS{10}Et ils mangeront mais ils ne seront point rassasiés, ils se prostitueront mais ils ne multiplieront point ; parce qu'ils ont cessé de prendre garde à Yahweh.
\VS{11}La prostitution, le vin et le moût, font perdre l'entendement.
\TextTitle{[Israël dans l'idolâtrie]}
\VS{12}Mon peuple consulte son bois, et c’est son bâton qui lui répond ; car l'esprit de prostitution égare, et ils se prostituent loin de leur Dieu.
\VS{13}Ils sacrifient sur le sommet des montagnes, ils brûlent de l’encens sur les collines, sous les chênes, sous les peupliers, et les térébinthes, parce que leur ombrage est agréable ; c'est pourquoi vos filles se prostituent, et vos belles-filles commettent l'adultère.
\VS{14}Je ne punirai pas vos filles parce qu’elles se prostituent, ni vos belles-filles parce qu’elles commettent l’adultère, car eux-mêmes se retirent avec des prostituées, et sacrifient avec des femmes débauchées. Ainsi le peuple qui est sans intelligence sera ruiné.
\VS{15}Si tu te prostitues, ô Israël ! Au moins que Juda ne se rende point coupable ; n'entrez donc point dans Guilgal, et ne montez pas à Beth-Aven, et ne jurez point : Yahweh est vivant !
\VS{16}Parce qu'Israël se révolte comme une vache indomptable, maintenant Yahweh les fera paître comme des agneaux dans de vastes plaines.
\VS{17}Ephraïm s'est associé aux idoles ; abandonne-le !
\VS{18}Leur breuvage est devenu aigre ; ils n'ont fait que se prostituer ; ils n'aiment qu'à dire : apportez ; ce n'est qu'ignominie que ses protecteurs.
\VS{19}Le vent l'a enfermé dans ses ailes, et ils auront honte de leurs sacrifices.
\TextTitle{[Yahweh abandonne Israël]}
\Chap{5}
\VerseOne{}Ecoutez ceci sacrificateurs ! Sois attentive maison d'Israël ! Prête l'oreille maison du roi ! Car c’est à vous que s’adresse le jugement parce que vous avez été un piège à Mitspa, et un filet tendu sur le Thabor.
\VS{2}Par leurs sacrifices, les infidèles s’enfoncent dans le crime. Mais je les châtierai tous.
\VS{3}Je connais Ephraïm, et Israël ne m'est point caché : Car maintenant Ephraïm, tu t’es prostitué, et Israël est souillé.
\VS{4}Leurs œuvres ne leur permettent pas de revenir à leur Dieu, parce que l'esprit de prostitution est au milieu d'eux, et parce qu’ils ne connaissent point Yahweh.
\VS{5}L’orgueil d'Israël témoigne contre lui, Israël et Ephraïm tomberont par leur iniquité ; Juda aussi tombera avec eux.
\VS{6}Ils iront avec leurs brebis et leurs bœufs chercher Yahweh, mais ils ne le trouveront point, il s'est retiré du milieu d’eux.
\VS{7}Ils se sont montrés infidèles envers Yahweh ; car ils ont engendré des enfants illégitimes ; maintenant un mois suffira pour les dévorer avec leurs biens.
\VS{8}Sonnez du shofar à Guibea. Sonnez de la trompette à Rama ! Poussez des cris de triomphe à Beth-Aven ! Derrière toi, Benjamin !
\VS{9}Ephraïm sera un sujet d’épouvante au jour du châtiment ; j’annonce aux tribus d'Israël une chose certaine.
\VS{10}Les chefs de Juda sont comme ceux qui déplacent les bornes, je répandrai sur eux ma fureur comme un torrent.
\VS{11}Ephraïm est opprimé, brisé par le jugement, car il a vécu selon les préceptes qui lui plaisaient.
\VS{12}Je serai comme une teigne pour Ephraïm, comme une carie pour la maison de Juda.
\VS{13}Ephraïm voit sa maladie, et Juda ses plaies ; Ephraïm s'en est allé vers le roi d'Assyrie, et s’est adressé au roi Jareb. Mais ce roi ne pourra ni vous guérir ni panser vos plaies.
\VS{14}Je serai comme un lion pour Ephraïm, comme un lionceau pour la maison de Juda. Moi, moi je déchirerai, puis je m'en irai ; j'emporterai la proie, et nul n’enlèvera ma proie.
\TextTitle{[Israël revient à Yahweh]}
\VS{15}Je m'en irai, je reviendrai dans ma demeure, jusqu'à ce qu'ils se reconnaissent coupables et qu'ils cherchent ma face ; quand ils seront dans la détresse, dans leur angoisse, ils auront recours à moi.
\Chap{6}
\VerseOne{}Venez, retournons à Yahweh ! Car il a déchiré, mais il nous guérira ; il a frappé, mais il bandera nos plaies.
\VS{2}Il nous rendra la vie dans deux jours, et le troisième jour il nous relèvera, et nous vivrons en sa présence.
\VS{3}Car nous connaîtrons Yahweh, et nous continuerons à le connaître ; sa venue\FTNT{Il est question ici du retour du Seigneur Jésus-Christ : Voir Za. 14:1.} est aussi certaine que celle de l’aurore. Il viendra pour nous comme la pluie, comme la pluie de l’arrière-saison\FTNT{Voir commentaires en Joë. 2:23.} qui arrose la terre.
\TextTitle{[Yahweh dénonce le péché d'Ephraïm]}
\VS{4}Que te ferai-je, Ephraïm ? Que te ferai-je, Juda ? Votre piété est comme la nuée du matin, comme la rosée qui se dissipe dès le matin.
\VS{5}C'est pourquoi je les taillerai en pièces par mes prophètes, je les tuerai par les paroles de ma bouche\FTNT{Hé. 4:12 ; Ap. 1:16 ; Ap. 19:15.}, et mes jugements apporteront la lumière.
\VS{6}Car je prends plaisir à la miséricorde et non aux sacrifices ; et à la connaissance de Dieu, plus qu’aux holocaustes.
\VS{8}Galaad est une ville où l’on pratique l’iniquité, elle porte des traces de sang.
\VS{9}La troupe des sacrificateurs est comme une bande en embuscade, commettant des assassinats sur le chemin de Sichem ; ils se livrent au crime.
\VS{10}J'ai vu des choses infâmes dans la maison d'Israël ; là Ephraïm se prostitue, Israël en est souillé.
\VS{11}A toi aussi Juda, une moisson est préparée, quand je ramènerai les captifs de mon peuple.
\TextTitle{[Iniquité d'Ephraïm]}
\Chap{7}
\VerseOne{}Lorsque je guérissais Israël, l'iniquité d'Ephraïm et la méchanceté de Samarie se sont révélées, car ils agissent frauduleusement, le voleur vient tandis que la bande dépouille au-dehors.
\VS{2}Ils n'ont point pensé dans leur cœur que je me souviens de toute leur méchanceté ; maintenant leurs œuvres les entourent, elles sont devant ma face.
\VS{3}Ils réjouissent le roi par leur méchanceté et les chefs par leurs mensonges.
\VS{4}Ils sont tous adultères, semblables à un four chauffé par le boulanger : Il cesse d’attiser le feu depuis qu’il a pétri la pâte jusqu'à ce qu'elle soit levée.
\VS{5}Au jour de notre roi, les chefs se rendent malades par les excès de vin ; il tend la main aux moqueurs.
\VS{6}Lorsqu’ils dressent des embuscades, leur cœur s’embrase comme un four ; leur boulanger dort toute la nuit ; le matin le four est embrasé comme un feu accompagné de flammes.
\VS{7}Ils sont tous ardents comme un four, et ils dévorent leurs chefs ; tous leurs rois tombent, et il n'y a aucun d'entre eux qui crie à moi.
\VS{8}Ephraïm se mêle avec les peuples ; Ephraïm est un gâteau qui n'a point été retourné.
\VS{9}Les étrangers ont dévoré sa force, et il ne s’en doute pas ; la vieillesse s’empare de lui, et il ne s’en doute pas.
\VS{10}L'orgueil d'Israël rendra témoignage contre lui ; car ils ne reviennent pas à Yahweh leur Dieu, et ne le cherchent pas malgré tout cela.
\VS{11}Ephraïm est comme une colombe troublée, sans intelligence ; car ils appellent l'Egypte, et s’en vont vers le roi d'Assyrie.
\VS{12}Quand ils s’en iront, j'étendrai mon filet sur eux ; et je les précipiterai comme les oiseaux du ciel ; je les châtierai comme ils en ont été avertis au sein de leurs assemblées.
\VS{13}Malheur à eux, parce qu'ils me fuient. Ruine sur eux, car ils se révoltent contre moi ; je voudrais les sauver, mais ils profèrent contre moi des paroles mensongères.
\VS{14}Ils ne crient pas vers moi dans leur cœur, mais ils gémissent sur leurs couches ; ils se rassemblent pour le froment et le bon vin, et ils s’éloignent de moi.
\VS{15}Je les ai châtiés, et j'ai fortifié leurs bras, mais ils méditent le mal contre moi.
\VS{16}Ce n’est pas au Très-Haut qu’ils retournent ; ils sont comme un arc trompeur. Leurs chefs tomberont par l’épée, à cause de l’insolence de leur langue. C’est ce qui en fera un objet de moquerie dans le pays d'Egypte.
\TextTitle{[Conséquences de la désobéissance]}
\Chap{8}
\VerseOne{}Crie comme si tu avais un shofar dans ta bouche. Il vient comme un aigle contre la maison de Yahweh ; parce qu’ils ont transgressé mon alliance, et qu’ils ont agi méchamment contre ma loi.
\VS{2}Ils crieront à moi : Mon Dieu, nous te connaissons, dira Israël !
\VS{3}Israël a rejeté le bien ; l'ennemi le poursuivra.
\VS{4}Ils ont fait régner, mais non pas de ma part, ils ont établi des chefs, et je n'en ai rien su ; ils se sont fait des idoles avec leur argent et leur or ; c'est pourquoi ils seront retranchés.
\VS{5}Samarie, ton veau t'a chassée loin ; ma colère s'est embrasée contre eux ; jusqu'à quand ne pourront-ils pas s'adonner à l'innoncence ?
\VS{6}Car il vient d'Israël, c'est un orfèvre qui l'a fait et il n'est pas Dieu ; c'est pourquoi le veau de Samarie sera mis en pièces.
\VS{7}Parce qu'ils ont semé du vent, ils moissonneront la tempête ; ils n’auront pas un épi de blé ; le grain qui poussera ne donnera point de farine ; et s'il y en a, les étrangers la dévoreront.
\VS{8}Israël est dévoré ! Il est maintenant parmi les nations comme un vase dont on ne se soucie pas.
\VS{9}Car ils sont montés vers le roi d'Assyrie, qui est un âne sauvage se tenant seul à part ; Ephraïm a fait des présents à ceux qui l'aimait.
\VS{10}Et parce qu'ils ont fait des présents aux nations, je les rassemblerai maintenant ; et ils commenceront à être amoindris à cause de l'impôt pour le roi des princes.
\VS{11}Parce qu'Ephraïm a fait plusieurs autels pour pécher, ils auront des autels pour pécher.
\VS{12}Je lui ai écrit les grandes choses de ma loi, mais elles sont estimées comme des lois étrangères.
\VS{13}Quant aux sacrifices qui me sont offerts, ils sacrifient de la chair, et la mangent ; mais Yahweh ne les accepte point, et maintenant il se souviendra de leur iniquité, et punira leurs péchés ; ils retourneront en Egypte.
\VS{14}Israël a oublié celui qui l'a fait, et il a bâti des palais ; et Juda a multiplié les villes fortes ; c'est pourquoi j'enverrai le feu dans les villes de celui-ci, quand il aura dévoré les palais de celui-là.
\TextTitle{[Ephraïm puni et rejeté]}
\Chap{9}
\VerseOne{}Israël ne te réjouis point, ne sois pas dans l’allégresse comme les autres peuples, de ce que tu t’es prostitué en abandonnant ton Dieu, de ce que tu as obtenu un salaire de tes amants dans toutes les aires à blé !
\VS{2}L'aire et la cuve ne les nourriront pas, et le moût leur fera défaut.
\VS{3}Ils ne resteront pas dans le pays de Yahweh, Ephraïm retournera en Egypte, et ils mangeront en Assyrie des aliments impurs.
\VS{4}Ils ne feront pas des libations de vin à Yahweh : Elles ne lui seraient point agréables. Leurs sacrifices ne lui seraient point agréables ; leurs sacrifices seront pour eux comme le pain de deuil ; tous ceux qui en mangeront se rendront impurs ; car leur pain ne sera que pour eux, il n'entrera point dans la maison de Yahweh.
\VS{5}Que ferez-vous aux jours des fêtes solennelles, aux jours des fêtes de Yahweh ?
\VS{6}Car voici, ils partent à cause de la dévastation ; l'Egypte les recueillera, Moph leur donnera des sépulcres ; ce qu’ils ont de précieux, leur argent sera la proie des ronces, et les épines croîtront dans leurs tentes.
\VS{7}Les jours du châtiment sont venus, les jours de la rétribution sont venus, et Israël le saura ! Les prophètes sont fous, les hommes de révélation sont insensés à cause de la grandeur de ton iniquité et de ta grande aversion.
\VS{8}Ephraïm est une sentinelle contre mon Dieu ; mais le prophète est un filet d'oiseleur sur toutes ses voies, en rébellion contre la maison de son Dieu.
\VS{9}Ils se sont profondément corrompus, comme aux jours de Guibea ; Yahweh se souviendra de leur iniquité, il punira leurs péchés.
\VS{10}J’ai, dira-t-il, trouvé Israël comme des raisins dans le désert ; j'ai vu vos pères comme les premiers fruits d’un figuier ; mais ils sont allés vers Baal-Peor\FTNT{Baal-Peor, «~seigneur de la brèche~», était une divinité adorée à Peor avec des rites licencieux (No. 23:28 ; No. 25:1-3 ; Ps. 106:28-29).}, ils se sont consacrés à l’infâme idole, et ils sont devenus abominables comme ce qu'ils ont aimé.
\VS{11}La gloire d'Ephraïm s'envolera comme un oiseau, plus de naissance, plus de grossesse, plus de conception.
\VS{12}Que s'ils élèvent leurs enfants, je les en priverai tellement, que pas un d'entre eux ne deviendra homme ; malheur à eux, quand je me retirerai d'eux !
\VS{13}Ephraïm, aussi loin que je porte mes regards vers Tyr, est planté dans un lieu agréable ; mais Ephraïm mènera ses fils vers celui qui les tuera.
\VS{14}Ô Yahweh ! Donne-leur… Mais que leur donnerais-tu ? Donne-leur un sein qui avorte et des mamelles desséchées.
\VS{15}Toute leur méchanceté s’est manifestée à Guilgal ; c’est là que je les ai pris en aversion. Je les chasserai de ma maison à cause de la malice de leurs actions ; je ne les aimerai plus ; tous leurs chefs sont des rebelles.
\VS{16}Ephraïm a été frappé ; sa racine est devenue sèche ; ils ne porteront plus de fruit ; et s'ils engendrent des enfants, je mettrai à mort les fruits désirable de leur ventre.
\VS{17}Mon Dieu les rejettera, parce qu'ils ne l'ont point écouté, et ils seront vagabonds parmi les nations.
\TextTitle{[Israël doit chercher Yahweh]}
\Chap{10}
\VerseOne{}Israël était une vigne féconde qui portait beaucoup de fruits. Plus ses fruits étaient abondants, plus il a multiplié les autels ; plus son pays était prospère, plus il a embelli ses statues.
\VS{2}Leur cœur est partagé, ils vont être déclarés coupables. Yahweh renversera leurs autels, il détruira leurs statues.
\VS{3}Car bientôt ils diront : Nous n'avons point de roi, parce que nous n'avons point craint Yahweh ; et le roi, que pourrait-il faire pour nous ?
\VS{4}Ils prononcent des paroles vaines, des faux serments lorsqu’ils concluent une alliance. C'est pourquoi le châtiment germera dans les sillons des champs, comme une plante vénéneuse.
\VS{5}Les habitants de Samarie seront épouvantés à cause des jeunes vaches de Beth-Aven : Car le peuple mènera deuil sur son idole ; et les prêtres de ses idoles qui s'en etaient réjouis, mèneront deuil parce que sa gloire est transportée loin d'elle.
\VS{6}Elle sera transportée en Assyrie, pour en faire un présent au roi Jareb ; Ephraïm sera dans la confusion, et Israël aura honte de ses desseins.
\VS{7}C’en est fait de Samarie et de son roi qui sera retranché comme l'écume qui est à la surface des eaux.
\VS{8}Les hauts lieux de Beth-Aven, où Israël a péché, seront détruits ; l'épine et la ronce croîtront sur leurs autels ; et on dira aux montagnes : Couvrez-nous ! Et aux collines : Tombez sur nous !
\VS{9}Israël ! Tu as péché dès les jours de Guibea. Là ils restèrent debout. La guerre contre les méchants ne les atteignit pas à Guibea.
\VS{10}Je les châtierai selon ma volonté, et les peuples s’assembleront contre eux, lorsqu’on les enchainera pour leur double iniquité.
\VS{11}Ephraïm est une génisse bien dressée, qui aime à fouler le blé, mais je m’approcherai de son superbe cou ; j’attellerai Ephraïm, Juda labourera, Jacob brisera ses mottes.
\VS{12}Semez selon la justice, moissonnerez selon la miséricorde ; défrichez-vous un champ nouveau ! Car il est temps de chercher Yahweh, jusqu'à ce qu'il vienne, et répande sur vous sa justice.
\VS{13}Vous avez cultivé la méchanceté, et vous avez moissonné l’iniquité ; vous avez mangé le fruit du mensonge, parce que vous avez eu confiance dans vos voies, dans la multitude de vos vaillants hommes.
\VS{14}Il s'élèvera un tumulte parmi ton peuple, et on détruira toutes tes forteresses, comme Schalman a détruit Beth-Arbel, au jour de la bataille, où la mère fut écrasée avec les enfants.
\VS{15}Béthel vous fera de même, à cause de votre extrême méchanceté ; le roi d'Israël sera entièrement exterminé dès l’aurore.
\TextTitle{[L'amour de Yahweh pour Israël]}
\Chap{11}
\VerseOne{}Quand Israël était jeune enfant, je l'aimais, et j'appelai mon fils hors d'Egypte\FTNT{La sortie des Hébreux de l’Egypte sous Moïse était une préfiguration de celle de Jésus-Christ lorsqu’il fuyait le massacre décrété par Hérode (Mt. 2:15).}.
\VS{2}Mais ils se sont éloignés de ceux qui les appelaient ; ils ont sacrifié aux Baals, et offert de l’encens aux idoles.
\VS{3}C’est moi qui ai appris à Ephraïm à marcher en le prenant par les bras, et ils n'ont pas vu que je les guérissais.
\VS{4}Je les tirai avec des liens d'humanité, et avec des cordages d'amour ; et je fus pour eux comme ceux qui enlèveraient le joug de dessus leur mâchoire, et je leur présentai de la nourriture.
\VS{5}Ils ne retourneront pas au pays d'Egypte ; mais le roi d'Assyrie sera leur roi, parce qu'ils n'ont point voulu revenir à moi.
\VS{6}L'épée fondra sur leurs villes, les réduira à néant, consumera leurs forces, et les dévorera, à cause des desseins qu’ils ont eus.
\VS{7}Mon peuple est enclin à s’éloigner de moi ; on le rappelle au Très-Haut, mais aucun d'eux ne l'exalte.
\VS{8}Que ferai-je de toi, Ephraïm ? Te livrerais-je, Israël ? Te traiterai-je comme Adma ? Te rendrai-je semblable à Tséboim ? Mon cœur s’agite au-dedans de moi, mes compassions sont émues.
\VS{9}Je n'exécuterai pas l'ardeur de ma colère, je ne reviendrai pas pour détruire Ephraïm ; car je suis Dieu, et non pas un homme ; je suis le Saint au milieu de toi, et je ne viendrai pas avec colère.
\VS{10}Ils marcheront après Yahweh, qui rugira comme un lion\FTNT{Yahweh rugit comme un lion : Jésus-Christ est le lion de la tribu de Juda, car selon la chair, il est issu de la postérité de Juda (Lc. 3:23-38 ; Ap. 5:5). Le lion est le roi des animaux, or Jacob fut le premier a avoir annoncé la venue du Schilo, c’est-à-dire celui à qui appartient le sceptre (Ge. 49:8-12).}, et quand il rugira, les enfants accourront en hâte de la mer.
\VS{11}Ils accourront en hâte hors d'Egypte, comme des oiseaux, et hors du pays d'Assyrie, comme des colombes. Et je les ferai habiter dans leurs maisons, dit Yahweh.
\TextTitle{[Yahweh dénonce le péché d'Ephraïm]}
\Chap{12}
\VerseOne{}Ephraïm m'entoure avec des mensonges, et la maison d'Israël avec des tromperies ; lorsque Juda erre sans frein vis-à-vis du Dieu Puissant, vis-à-vis du Saint fidèle.
\VS{2}Ephraïm se repaît de vent, et poursuit le vent d'Orient ; il multiplie le mensonge et la violence, et il traite alliance avec l'Assyrie, et l'on porte des huiles de senteur en Egypte.
\VS{3}Yahweh a aussi un procès avec Juda, et il punira Jacob pour sa conduite, il lui rendra selon ses œuvres.
\VS{4}Dans le sein maternel, Jacob saisit son frère par le talon\FTNT{Jacob saisit le talon de son frère : Ge. 25:26}, puis dans sa vigueur il lutta avec Dieu\FTNT{Jacob lutta avec Dieu : Ge. 32:24-28.}.
\VS{5}Il lutta avec l’Ange, et il fut vainqueur, il pleura, et lui demanda grâce ; Jacob l’avait rencontré à Béthel, et c’est là que Dieu nous a parlé.
\VS{6}Yahweh est le Dieu des armées ; son nom est Yahweh.
\VS{7}Et toi donc, reviens à ton Dieu, garde la miséricorde et la justice, et espère toujours en ton Dieu.
\VS{8}Ephraïm est un marchand\FTNT{Ephraïm est appelé «~marchand~», littéralement «~Canaan~». Notez que l’ange de Laodicée s’exprime comme Ephraïm : «~Je  suis riche, je me suis enrichi...~» (Ap. 3:14-19).} qui a dans sa main des balances fausses, il aime à frauder.
\VS{9}Et Ephraïm dit : Quoi qu'il en soit, je suis devenu riche ; je me suis acquis des richesses ; c’est entièrement le produit de mon travail ; on ne trouvera en moi aucune iniquité, rien qui soit un péché.
\VS{10}Et moi, je suis Yahweh, ton Dieu, dès le pays d'Egypte ; je te ferai encore habiter dans des tentes, comme aux jours des fêtes solennelles.
\VS{11}Je parlerai par les prophètes, et je multiplierai les visions, et par les prophètes, je proposerai des paraboles.
\VS{12}Certainement Galaad n'est qu'iniquité, certainement ils ne sont que vanité ; ils ont sacrifié des bœufs dans Guilgal ; même leurs autels sont comme des monceaux de pierres sur les sillons des champs.
\VS{13}Jacob s'enfuit au pays de Syrie, et Israël servit pour une femme, et pour une femme il garda les troupeaux.
\VS{14}Par un prophète, Yahweh fit monter Israël hors d'Egypte, et par un prophète, Israël fut gardé.
\VS{15}Ephraïm a irrité amèrement Yahweh ; son Seigneur rejettera sur lui le sang qu’il a répandu, il fera retomber sur lui la honte qui lui appartient.
\TextTitle{[Méchanceté constante d'Ephraïm]}
\Chap{13}
\VerseOne{}Lorsque Ephraïm parlait, c’était la terreur ; il était élevé en Israël. Mais il s'est rendu coupable par Baal, et il est mort.
\VS{2}Maintenant ils continuent à pécher, ils se font avec leur argent des images en métal fondu, des idoles de leur invention ; toutes sont l’œuvre des artisans. Mais l’on dit aux hommes qui sacrifient : Qu’on baise les veaux\FTNT{C’est une expression d’hommage.} !
\VS{3}C'est pourquoi ils seront comme la nuée du matin, et comme la rosée qui bientôt disparaît ; comme la balle qui est emportée par le vent hors de l’aire, comme la fumée sortant de la cheminée.
\VS{4}Et moi, je suis Yahweh, ton Dieu, dès le pays d’Egypte, et tu ne devrais reconnaître d'autre dieu que moi ; et il n'y a pas d’autre Sauveur que moi\FTNT{Yahweh dit qu’il n’y a pas d’autre sauveur que lui (Es. 43:11). Or les Ecrits de la nouvelle alliance nous présentent clairement notre Sauveur : Jésus-Christ (Mt.1:21; Ac.13:23 ; 2 Ti. 1:10 ; Tit : 1:4).}.
\VS{5}Je t'ai connu dans le désert, dans une terre aride.
\VS{6}Ils se sont rassasiés dans leurs pâturages ; ils se sont rassasiés, et leur cœur s'est enflé ; alors ils m'ont oublié.
\VS{7}Je serai pour eux comme un lion ; je les épierai sur la route comme un léopard.
\VS{8}Je les attaquerai comme une ourse à qui on a enlevé ses petits, et je déchirerai l’enveloppe de leur cœur ; et là, je les dévorerai comme un lion ; les bêtes des champs les mettront en pièces.
\TextTitle{[Restauration d'Israël]}
\VS{9}Ta ruine ô Israël ! C’est que tu as été contre moi, alors que moi seul pouvais te secourir.
\VS{10}Où donc est ton roi ? Qu'il te délivre dans toutes tes villes ! Où sont tes juges, au sujet desquels tu as dit : Donne-moi un roi et des princes ?
\VS{11}Je t'ai donné un roi\FTNT{Ce passage concerne Saül, premier roi d’Israël (1 S. 8, 9 et 10)} dans ma colère, et je l'ôterai dans ma fureur.
\VS{12}L'iniquité d'Ephraïm est enveloppée, et son péché est mis en réserve.
\VS{13}Les douleurs comme de celle qui enfante le surprendront ; c'est un enfant qui n'est pas sage, qui, au temps marqué, ne sort pas du sein maternel.
\VS{14}Je les rachèterai de la puissance du scheol, je les délivrerai de la mort ; ô mort, où est ta peste\FTNT{L’apôtre Paul applique ce passage à la victoire que le Seigneur Jésus-Christ a remportée face à la mort lors de sa résurrection (Hé. 2 :14:18). Depuis la chute d’Adam, les hommes ont toujours eu peur de la mort. Cette peur est d’autant plus forte de nos jours, car la plupart des gens sont angoissés par son aspect imprévisible et inévitable et par son non-sens. Et bien que beaucoup ne croient pas à l’existence de la vie après la mort (au paradis ou en enfer), la mort associée à l’annihilation, au non-être, apparaît d’autant plus monstrueuse et insupportable. Or notre Seigneur Jésus-Christ a vaincu la mort et il promet la vie éternelle à ceux qui croient en lui (Jn. 3:16 ; Jn. 5:24-29 ; Ap. 1:18). En plaçant notre foi en lui, nous avons non seulement la victoire sur la mort, mais aussi sur l’angoisse qu’elle produit dans le cœur de tout homme.} ? Scheol où est ta destruction ? Mais le repentir se dérobe à mes regards !
\VS{15}Ephraïm a beau être fertile au milieu de ses frères, le vent d’orient, le vent de Yahweh s’élèvera du désert, viendra, desséchera ses sources et tarira ses fontaines. On pillera le trésor de tous ses objets précieux.
\VS{16}Samarie sera châtiée, car elle s'est rebellée contre son Dieu. Ils tomberont par l'épée, leurs petits enfants seront écrasés, et l’on fendra le ventre de leurs femmes enceintes.
\TextTitle{[Bénédiction future d'Israël]}
\Chap{14}
\VerseOne{}Israël, reviens à Yahweh ton Dieu ; car tu es tombé par ton iniquité.
\VS{2}Apportez avec vous des paroles, et revenez à Yahweh. Dites-lui : Pardonne toutes nos iniquités, et reçois le bien, pour le mettre à sa place, et nous t’offrirons pour sacrifices la louange de nos lèvres.
\VS{3}L'Assyrie ne nous sauvera pas ; nous ne monterons pas sur des chevaux, et nous ne dirons plus à l'ouvrage de nos mains : Notre dieu ! Car c’est auprès de toi que l'orphelin trouve de la compassion.
\VS{4}Je guérirai leur rébellion, et les aimerai volontairement ; parce que ma colère s'est détournée d’eux.
\VS{5}Je serai comme la rosée pour Israël ; il fleurira comme le lis, et il poussera ses racines comme le Liban.
\VS{6}Ses branches s’étendront, et sa magnificence sera comme celle de l'olivier, avec un parfum comme celui du Liban.
\VS{7}Ils reviendront s’asseoir à son ombre, et ils redonneront la vie au froment, et ils fleuriront comme la vigne ; et l'odeur de chacun d'eux sera comme celle du vin du Liban.
\VS{8}Ephraïm dira : Qu'ai-je à faire encore avec les idoles ? Je l'exaucerai, je le regarderai, je serai pour lui comme un cyprès verdoyant. C’est de moi que tu recevras ton fruit.
\VS{9}Que celui qui est sage prenne garde à ces choses ! Que celui qui est intelligent les comprenne. Car les voies de Yahweh sont droites ; les justes y marcheront, mais les rebelles y tomberont.
\PPE{}
\end{multicols}
