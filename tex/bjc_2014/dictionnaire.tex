\begin{multicols}{2}
\NoAutoSpaceBeforeFDP{
\DicoEntry{AARON}\textit{, de l'hébreu « Aharown » : « haut placé » ou « éclairé »}\newline
Issu de la tribu de Lévi, frère aîné de Moïse dont il fut le porte-parole. Premier souverain sacrificateur* en Israël. Voir Ex. 4:14 ; Ex. 6:16-20 et Ex. 28.

\DicoEntry{ABSALOM}\textit{, de l'hébreu « 'Abiyshalowm » : « père de la paix »}\newline
Fils du roi David et de Maaca, né à Hébron. Il tua Amnon son demi-frère aîné, car ce dernier avait déshonoré sa sœur Tamar. Quelques années plus tard, il conspira contre son père et se fit proclamer roi à Hébron. Il fut finalement tué par Joab, chef de l'armée de David. Voir 2 S. 3:3 ; 2 S. 13 ; 2 S. 15-19.

\DicoEntry{ABDIAS}\textit{, de l'hébreu « Obadyah » : « adorateur » ou « serviteur de Yahweh »}\newline
Prophète de Yahweh dont le livre éponyme figure dans le Tanakh.

\DicoEntry{ABEL}\textit{, de l'hébreu « Hebel » : « souffle, vapeur »}\newline
Deuxième fils d'Adam et Eve et première victime d'homicide de l'histoire, il fut assassiné par son frère Caïn et déclaré juste par Yahweh. Voir Ge. 4:2, 8 et Mt. 23:35.

\DicoEntry{ABIRAM}\textit{, de l'hébreu « 'Abyiram » : « mon père est exalté »}\newline
Issu de la tribu de Ruben, fils d'Eliab et frère de Dathan, il conspira avec Koré contre Moïse et Aaron. Voir No. 16:1-35.

\DicoEntry{ABLUTION}\textit{, de l'hébreu « rachats » : « laver, baigner, nettoyer »}\newline
Lavage de purification prescrit par la loi mosaïque et effectué avec de l'eau. Voir Ex. 29:4 et Hé. 9:10.

\DicoEntry{ABOMINATION}\textit{, de l'hébreu « tow'ebah » : « une chose dégoûtante, abominable » et du grec « bdelugma » : « chose folle, détestable »}\newline
Pratique violant la loi de Yahweh et manifestant l'infidélité à Dieu comme l'idolâtrie* sous toutes ses formes, la magie ou l'homosexualité*. Voir Lé. 18:6-29 ; De. 29:17-18 et Ap. 21:27.

\DicoEntry{ABRAM}\textit{, de l'hébreu « Abryram » : « père élevé »}\newline
Voir ABRAHAM.

\DicoEntry{ABRAHAM}\textit{, de l'hébreu « 'Abraham » : « père d'une multitude »}\newline
Hébreu, fils de Térach, et originaire d'Ur en Chaldée. Dieu lui demanda de quitter sa terre et sa famille pour Canaan, lui promettant que sa postérité hériterait de cette terre. De sa servante Agar, lui naquit un premier fils, Ismaël, ancêtre du peuple arabe. De sa femme Sara, lui naquit Isaac qui hérita des promesses. Il mourut à cent soixante-quinze ans. Voir Ge. 12:1-7 ; Ge. 17:4-13 ; Ge. 16 ; Ge. 21:1-8 et Ge. 25:7.

\DicoEntry{ACACIA}\textit{, de l'hébreu « shittah » : « acacia, bois d'acacia »}\newline
Arbre épineux poussant en abondance dans la péninsule du Sinaï et dans la vallée du Jourdain, il est aussi appelé bois de Sittim. Il fut l'un des matériaux utilisés pour la fabrication des objets du culte lévitique, dont l'arche*. Voir Ex. 25:10, 13, 23 et 28.

\DicoEntry{ACHAB}\textit{, de l'hébreu « Ach'ab » : « un frère du père »}\newline
Fils d'Omri, il fut roi d'Israël pendant vingt-deux ans. Marié à Jézabel*, fille du roi des Sidoniens, Achab et sa femme commirent de grandes abominations* et s'opposèrent au prophète Elie*. Voir 1 R. 16:29-31 ; 1 R. 18:1-40 et 1 R. 22:29-40.

\DicoEntry{ADAM}\textit{, de l'hébreu « 'Adam » : « être humain » ou « de la terre »}\newline
Premier homme, il vécut la première partie de sa vie dans le jardin d'Eden* avec sa femme Eve. Après avoir désobéi à Dieu en goûtant le fruit de l'arbre de la connaissance du bien et du mal, ils furent chassés du jardin. Adam fut le père de Caïn, Abel et Seth. Il mourut à neuf cent trente ans. Voir Ge. 2:7-8 ; Ge. 3 ; Ge. 4:1-2, 25-26 et Ge. 5:5.

\DicoEntry{ADONIJA}\textit{, de l'hébreu « 'Adoniyah » : « Yahweh est Seigneur »}\newline
Quatrième fils de David et de Haggith. Peu avant la mort de son père, il s'autoproclama roi tentant en vain de prendre la place qui devait revenir à Salomon. Ce dernier lui laissa la vie sauve, mais le fit tuer plus tard alors qu'il semblait encore convoiter le trône d'Israël. Voir 2 S. 3:4 et 1 R. 1, 2.

\DicoEntry{ADOPTION}\textit{, du grec « huiothesia » : « adoption, adoption comme fils »}\newline
Manifestation de l'amour éternel de Dieu, l'adoption permet à tout homme de devenir par la foi enfant de Dieu. Ce privilège, autrefois réservé au peuple d'Israël, fut étendu à toutes les nations par le sacrifice de Jésus. Cette adoption est manifestée par l'Esprit de Dieu qui témoigne à l'esprit du chrétien son appartenance à Dieu ; elle inclut les avantages du fils, dont l'héritage. Voir Jn. 1:12 ; Ro. 8:15-17 ; Ro. 9:4 ; Ep. 1:5, 11 ; Ga. 4:7 et 1 Jn. 3:1.

\DicoEntry{AGAR}\textit{, de l'hébreu « Hagar » : « fuite »}\newline
Servante égyptienne de Sara que cette dernière donna à Abraham comme concubine. Elle enfanta Ismaël, fils premier-né d'Abraham. Après la naissance d'Isaac, Abraham la chassa avec son fils. Voir Ge. 16 et Ge. 21:1-18.

\DicoEntry{AGABUS}\textit{, de l'hébreu « Chagab » et du grec « Agabos » : « sauterelle »}\newline
Prophète de Yahweh suscité au temps de l'Eglise primitive. Il prophétisa une famine qui se réalisa sous le règne de l'empereur Claude. Il annonça aussi l'arrestation de Paul à Jérusalem. Voir Ac 11:27-28 et Ac. 21:10-33.

\DicoEntry{AGGEE}\textit{, de l'hébreu « Chaggay » : « en fête » ou « né un jour de fête »}\newline
Prophète de Yahweh d'après la captivité, dont le livre éponyme figure dans le Tanakh.

\DicoEntry{AGNEAU}\textit{, de l'hébreu « kebes » : « agneau, brebis, jeune bélier »}\newline
Animal sacrifié et consommé lors de la Pâque des juifs. Il préfigurait Christ, l'Agneau de Dieu qui ôte le péché du monde. Voir Ex. 12:1-28 et Jn. 1:29.

\DicoEntry{AI}\textit{, de l'hébreu « 'Ay » : « tas de ruines »}\newline
Ville située au sud-est de Béthel, à proximité de laquelle Abraham dressa sa tente à deux reprises. Il s'agit également de la deuxième ville que Dieu livra entre les mains de Josué après la prise de Jéricho. Voir Ge. 12:8 ; Ge. 13:3 et Jos. 8.

\DicoEntry{ALLELUIA}\textit{, de l'hébreu « allelouia » : « Louez Yahweh »}\newline
Retrouvé à maintes reprises dans les Psaumes sous la forme « Louez Yahweh », cette exclamation encourage à célébrer Dieu et à se réjouir en lui. Voir Ap. 19:1-6.

\DicoEntry{ALLIANCE}\textit{, de l'hébreu « beriyth » : « pacte, alliance, engagement »}\newline
Dieu a conclu plusieurs alliances avec les hommes (ex : Noé, Abraham, David). On distingue communément deux alliances majeures dans les Ecritures : l'Ancienne Alliance - conclue avec Israël au travers de Moïse - et la Nouvelle Alliance inaugurée par Jésus-Christ. Voir Ge. 9:8-17 ; Ge. 17 ; 2 S. 7:12-16 ; Ex. 19-34 et Hé. 9-13.

\DicoEntry{ALPHA ET OMEGA}\textit{}\newline
Première et dernière lettre de l'alphabet grec, la combinaison de ces deux lettres mentionnées ensemble se rapporte à l'idée que Dieu est le premier et le dernier. Jésus fut présenté plusieurs fois comme étant « l'alpha et l'oméga » soulignant ainsi son caractère éternel. Voir Ap. 1:8 ; Ap. 21:6 et Ap. 22:13.

\DicoEntry{AME}\textit{, de l'hébreu « nephesh » : « âme, une personne, la vie, être vivant », « ce qui respire », « ce qui a une vie par le sang » et du grec « psuche » : « le souffle, la vie, l'âme »}\newline
L'âme correspond au sang ; elle est le siège des émotions, de la volonté humaine et de l'intelligence. Avec l'esprit et le corps, l'âme constitue l'être humain. Voir Lé. 17:11 ; Ge. 19:20 ; Ge. 44:30 ; Mt. 10:28 ; Ac. 20:10 et 1 Th. 5:23.

\DicoEntry{AMEN}\textit{, de l'hébreu « 'amen » : « assuré, établi » ou « ainsi soit-il ! »}\newline
Se rapportant exclusivement à ce qui est sûr, avéré et certain, ce terme est souvent utilisé comme interjection. Christ est appelé « l'Amen », faisant référence à la vérité qu'il incarne. Voir 1 Ch. 16:36 ; Jé. 28:6 ; 2 Co. 1:20 et Ap. 3:14.

\DicoEntry{AMOUR}\textit{}\newline
Il existe plusieurs traductions et définitions du mot « amour » en hébreu et en grec, elles varient selon le contexte.
\\- Les termes hébreux désignant l'amour :
\\1. « 'Ahab » : « amours »
\\Amours, amis. Voir Pr. 5:19 et Os. 8:9.
\\2. « 'Ahabah » : « amour humain, amour de Dieu pour son peuple »
\\Amour, affection, aimer. Voir De. 7:8 ; 1 S. 20:17 et Pr. 10:12.
\\2. « Checed » : « bonté, miséricorde, fidélité »
\\Grâce, miséricorde, compassion, affection. Voir Ge. 40:14 ; Ex. 34:7 et Nb. 14:19.
\\3. « Yediyd » : « bien-aimé »
\\Bien-aimé, amour. Voir De. 33:12 et Es. 5:1.
\\- Les termes grecs désignant l'amour :
\\1. « Agape » : « amour, charité, affection, bienveillance »
\\Amour de Dieu, amour désintéressé que doit manifester l'homme né d'en haut. Voir Jn. 15 :13 ; Jn. 17:26 ; Ro. 5:5 ; 1 Co. 8:1 et 1 Co. 13:3 et 1 Jn. 4:8.
\\2. « Eros » : « l'amour qui prend »
\\Amour dans la dimension sexuelle.
\\3. « Phileo » : « aimer, montrer des signes d'amour »
\\Amour filial. Voir Jn. 21:17 ; 1 Co. 16:22.
\\4. « Philadelphia » : « amour fraternel »
\\Amour des frères et sœurs d'une même famille, amour des chrétiens les uns pour les autres. Voir Ro. 12:10 ; 1 Th. 4:9 et Hé. 13:1.
\\5. « Storge » : « amour filial »
\\Amour familial, affection naturelle. Voir Ro. 1:31.

\DicoEntry{AMOS}\textit{, de l'hébreu « 'Amowc » : « fardeau, porteur de fardeaux »}\newline
Originaire de Tekoa en Juda, prophète de Yahweh dont le livre éponyme figure dans le Tanakh.

\DicoEntry{AMMONITES}\textit{, de l'hébreu « 'Ammown » : « appartenant à la nation »}\newline
Peuple issu de Ben-Ammi, né de l'inceste entre Lot et sa fille cadette. Ils furent ennemis d'Israël. Voir Ge. 19:30-38 et Ez. 25:1-7.

\DicoEntry{ANAKIM}\textit{, de l'hébreu « 'Anaqiy » : « au long cou »}\newline
Descendants d'Anak, race de géants habitant Canaan avant sa conquête par le peuple d'Israël. Ils furent vaincus par Josué et Caleb qui hérita d'une partie de leur territoire. Voir No. 13:28-33 ; De. 9:1-3 ; Jos. 11:21-22 et Jos. 14:6-15.

\DicoEntry{ANANIAS}\textit{, de l'hébreu « Chananyah » : « Dieu a été miséricordieux »}\newline
1. Chrétien ayant vendu un champ avec sa femme Saphira et ayant fait croire qu'ils avaient donné la totalité du prix rapporté pour l'Eglise alors qu'ils en avaient secrètement gardé une partie. Ce mensonge les conduisit tous deux à la mort. Voir Ac. 5:1-10.
\\2. Homme pieux vivant à Damas que le Seigneur envoya imposer les mains à Saul qui venait de se convertir afin qu'il recouvre la vue. C'est également lui qui le baptisa. Voir Ac. 9:10-18 et Ac. 22:12-16.

\DicoEntry{ANATHÈME}\textit{, du grec « anathema » : « tout ce qui est livré au malheur »}\newline
Terme désignant une personne ou une chose maudite, vouée à la destruction. Voir 1 Co. 12:3 et Ga. 1:8.

\DicoEntry{ANCIENS}\textit{, de l'hébreu « zaqen » : « vieux, aîné, de ceux qui ont de l'autorité » et du grec « presbuteros » : « ayant de l'âge »}\newline
Chez les juifs, il s'agissait des chefs de famille ou de clan qui représentaient le peuple dans les affaires religieuses et civiles. Voir Ex. 3:16 ; Lé. 4:15 et De. 31:28. Sous la Nouvelle Alliance, les églises de la Galatie avaient élu des anciens (« presbuteros ») pour prendre soin des frères et sœurs. Il s'agit d'un terme relatif aux personnes ayant de l'âge et non à la fonction d'évêque*. Voir Ac. 14:23 ; 1 Ti. 5:17 ; 1 Pi. 5:1-5 et Tit. 1:5-9.

\DicoEntry{ANDRE}\textit{, du grec « Andreas » : « virilité »}\newline
Frère de Simon Pierre, originaire de Bethsaïda en Galilée, et pêcheur de métier. Il devint l'un des douze apôtres de Jésus-Christ. Voir Mc. 1:16-17 ; Jn. 1:40 et Mt. 10:2.

\DicoEntry{ANGE}\textit{, de l'hébreu « mal'ak » et du grec « aggelos » : « messager, envoyé »}\newline
Etre spirituel au service de Dieu pouvant prendre une forme humaine. Les anges sont au service de Yahweh pour des missions spécifiques au ciel ou sur la terre. Ils peuvent avoir une fonction de messager, protecteur ou combattant. Voir Da. 10:10-13 ; Lu. 1:26-38 et Ap. 12:7.

\DicoEntry{ANNE}\textit{, de l'hébreu « Channah » : « grâce, faveur »}\newline
1. Une des deux femmes d'Elkana. Stérile, elle pria Yahweh de lui accorder un fils qu'elle lui consacrerait. Elle enfanta ainsi Samuel qui entra au service de Yahweh dès son plus jeune âge. Voir 1 S. 1, 2.
\\2. Fille de Phanuel de la tribu d'Aser, prophétesse. Veuve, elle servait le Seigneur nuit et jour dans le temple. Elle rencontra Jésus nourrisson, lorsqu'il fut amené au temple pour y être présenté à Dieu. Voir Lu. 2:36-38.
\\3. Souverain sacrificateur, beau-père de Caïphe*. Il participa à la conspiration qui mena Jésus à la croix. Voir Lu. 3:2 et Jn. 18:13.

\DicoEntry{ANTECHRIST}\textit{, du grec « antichristos » : « l'adversaire du Messie »}\newline
Aussi appelé « homme impie » et « fils de la perdition », personnage dont l'apparition se fera avant le retour glorieux du Seigneur. Il dominera le monde avant d'être vaincu par Christ. Voir 2 Th. 2:1-4 ; 2 Jn. 1:7 et Ap. 19:19-21.

\DicoEntry{ANTIOCHE}\textit{, du grec « Antiocheia » : « rapide comme un char »}\newline
Capitale de la Syrie, elle fut fondée en 300 av. J.-C. par Séleucus Nicator (\ 358-281 av. J.-C.) qui la baptisa du nom de son père Antiochus. Cette ville accueillit des chrétiens en exil ; l'évangile y fut ainsi annoncé aux juifs puis aux Grecs et un grand nombre de personnes se convertirent. Barnabas et Paul y demeurèrent une année durant laquelle ils enseignèrent la parole. C'est à Antioche que les disciples furent appelés chrétiens pour la première fois. Voir Ac. 11:19-26.

\DicoEntry{APIS}\textit{}\newline
Divinité égyptienne symbolisant la force et la fertilité. Il est représenté sous la forme d'un veau d'or ou d'un homme à tête de taureau dont les cornes entourent un disque solaire. Les Hébreux se corrompirent plusieurs fois en le vénérant. Voir Ex. 32:1-6 et 1 R. 12:28-30.

\DicoEntry{APOCALYPSE}\textit{, du grec « apokalupsis » : « mettre à nu, révélation d'une vérité, action de révéler »}\newline
Dernier livre de la Bible écrit par l'apôtre Jean, ce récit comporte une révélation de la gloire de Jésus-Christ et raconte les derniers événements de l'histoire de l'humanité jusqu'à l'avènement de la Nouvelle Jérusalem.

\DicoEntry{APOLLOS}\textit{, du grec « Apollos » : « donné par Apollon »}\newline
Juif érudit d'Alexandrie ayant une très bonne connaissance des Ecritures et enseignant avec exactitude au sujet de Jésus. Sa rencontre avec Aquilas et Priscille lui permit d'aller plus en profondeur dans la Parole et d'annoncer avec plus de force, notamment aux juifs, que Jésus est le Messie en se basant sur les écrits du Tanakh. Il réalisa plusieurs voyages missionnaires notamment à Corinthe. Voir Ac. 18:24-28, 1 Co. 3:5-6 et 1 Co. 16:12.

\DicoEntry{APOSTASIE}\textit{, du grec « apostasia » : « action de s'éloigner de, désertion, défection »}\newline
Abandon de la foi en Jésus-Christ et de la saine doctrine* se manifestant sous deux formes principales. Certaines personnes abandonnent ouvertement la foi, la communion avec Dieu et l'assemblée des saints. D'autres continuent de fréquenter les assemblées chrétiennes, mais ont laissé la saine doctrine pour s'attacher à des doctrines séductrices. Voir Mt. 24:11-12 ; 2 Th. 2:3 ; 1 Ti. 4:1-3 ; 2 Ti. 3:1-8 ; 2 Pi. 2:1-3 ; 1 Jn. 4:1 et Jud. 1:17-19.

\DicoEntry{APOTRE}\textit{, du grec « apostolos » : « envoyé en avant, messager, ambassadeur »}\newline
Lors de son ministère terrestre, Jésus choisit douze apôtres qu'il forma pour continuer l'œuvre après lui. Plusieurs autres apôtres furent suscités au temps de l'Eglise primitive, notamment Paul et Jacques, frère du Seigneur, qui avec Jean et Pierre furent les principaux instruments utilisés pour poser les fondements de la doctrine de l'Eglise. Le ministère apostolique existe encore aujourd'hui, mais la mission des apôtres actuels n'est pas d'écrire des épîtres, car la fondation a déjà été posée. Leur travail aujourd'hui consiste davantage à enseigner et veiller à ce que le fondement demeure. Voir Mc. 3:14 ; Ac. 15 ; Ga. 2:19 ; Ro. 1:1 Ep. 2:20 et Ep. 4:11.

\DicoEntry{AQUILAS}\textit{, du latin « Aquilas » : « un aigle » et PRISCILLE, du latin « Priscilla » : « petite vieille »}\newline
Couple de Juifs ayant accepté l'évangile. Après avoir été chassés de Rome, ils s'installèrent à Corinthe où ils hébergèrent Paul à son arrivée et devinrent par la suite compagnons d'œuvre de ce dernier. Ils participèrent à plusieurs voyages missionnaires, notamment à Ephèse où ils enseignèrent Apollos*. Voir Ac. 18:1-3, 18, 24-26 et Ro. 16:3-5.

\DicoEntry{ARBRE}\textit{, de l'hébreu « 'ets » : « arbre, bois »}\newline
Organisme vivant porteur de semence produisant des feuilles et des fruits selon les espèces. Lors de la création, Dieu créa différents arbres dont les fruits furent donnés pour nourrir l'homme et également deux arbres spécifiques placés au milieu du jardin d'Eden :

\DicoEntry{ARBRE DE LA CONNAISSANCE DU BIEN ET DU MAL}\textit{}\newline
Arbre dont le fruit contenait la connaissance du bien et du mal. Dieu interdit la consommation de ce dernier à l'homme sous peine de mort, mais Adam et Eve transgressèrent le commandement. C'est ainsi que le péché et la mort régnèrent sur l'humanité. Voir Ge. 2:17 ; Ge. 3:1-6 et Ro. 5:12.

\DicoEntry{ARBRE DE VIE}\textit{}\newline
Arbre dont la consommation donne la vie éternelle. Après la chute d'Adam et Eve, Dieu les chassa du jardin pour les empêcher d'y accéder. L'arbre de vie se trouve dans la ville sainte, la Nouvelle Jérusalem ; ses feuilles servent à la guérison des nations. Voir Ge. 3:22-24 et Ap. 22:2, 14, 19.

\DicoEntry{ARC-EN-CIEL}\textit{, de l'hébreu « qesheth » : « arc »}\newline
Signe de l'alliance* que Dieu conclut avec Noé et les générations qui le suivraient suite au déluge*. Cette alliance stipulait que Yahweh ne détruirait plus les hommes par les eaux. Voir Ge. 9:12-17.

\DicoEntry{ARCHANGES}\textit{, du grec « archaggelos » : « chef des anges »}\newline
Catégorie d'anges* ayant un rang et une dignité plus élevés que les autres. Voir 1 Th. 4:16 et Jud. 1:9.

\DicoEntry{ARCHE DE NOE}\textit{, de l'hébreu « tebah » : « arche, vaisseau, coffre »}\newline
Embarcation construite par Noé pour le sauver lui, sa famille ainsi que les animaux, du déluge* qui allait s'abattre sur la terre. Voir Ge. 6:8-16 ; Mt. 24:37-39 et Lu. 17:26-27.

\DicoEntry{ARCHE DU TEMOIGNAGE ou DE L'ALLIANCE}\textit{, de l'hébreu « 'arown » : « arche, coffre, cercueil »}\newline
Coffre rectangulaire en bois d'acacia* recouvert d'or pur, contenant les tables de l'alliance, la verge d'Aaron et une urne contenant un échantillon de la manne. Construite selon le modèle que Moïse avait reçu au mont Sinaï, elle était couverte par le propitiatoire*. L'arche fut placée dans le Saint des saints du tabernacle*, puis du temple*. Voir Ex. 25:10-22 ; Hé. 9:4 ; 1 R. 8:6 et 2 R. 25:8-9.

\DicoEntry{ARTAXERXES}\textit{, (règne : 465 av. J.-C.- 424 av. J.-C.), du persan « Artachshashta » : « celui qui fait régner la loi sacrée »}\newline
Fils d'Assuérus*, roi de Perse. Il autorisa Esdras à retourner à Jérusalem avec des sacrificateurs et des Lévites pour effectuer le sacerdoce dans le temple et faire respecter la loi de Yahweh. Voir Esd. 7:11-28.

\DicoEntry{ASAPH}\textit{, de l'hébreu « 'Acaph » : « celui qui rassemble, collecteur »}\newline
Lévite et chef des chantres sous David, il participa au transfert de l'arche* à Jérusalem et écrivit certains psaumes. Voir 1 Ch. 15:16-19 et 1 Ch. 16:4-7.

\DicoEntry{ASER}\textit{, de l'hébreu « 'Asher » : « heureux »}\newline
Fils de Jacob et de Zilpa, servante de Léa, il est le père de la tribu d'Aser. Voir Ge. 30:13.

\DicoEntry{ASHERA}\textit{, de l'hébreu « 'Asherah » : « pieu sacré ».}\newline
Ce mot est cité au moins quarante fois dans le Tanakh. Il fait référence à un objet en bois utilisé dans le culte de la parèdre de Baal. Manassé, roi de Juda, introduisit l'emblème d'Asherah dans le temple (2 R. 21:1-7) en dépit de l'interdiction formelle de Yahweh (De. 16:21). Il n'en fut enlevé que lors des réformes de Josias et d'Ezéchias (2 R. 18:3-4 ; 2 R. 23:6-14). Asherah correspond à Astarté (la déesse Ischtar des Babyloniens, l'Aphrodite des Grecs, la Vénus des Romains, la Reine du ciel mentionnée dans Jé. 7:18 ; Jé. 44:15-30. Voir aussi Jg. 2:13 ; Jg. 10:6 ; 1 S. 31:10 ; 1 R. 11:5-33 ; 2 R. 23:13, où elle est appelée l'abomination des Sidoniens). Elle correspond aussi à la Diane des Ephésiens (Ac. 19:23-40), à la mère de Dieu de l'église romaine, à l'Isis des Egyptiens.

\DicoEntry{ASSUERUS ou XERXES Ier}\textit{, (485 av. J.-C. – 465 av. J.-C.), du persan « 'Achashverowsh » : « je serai silencieux et pauvre ».}\newline
Père d'Artaxerxès, roi de Perse et époux d'Esther*. Voir le livre d'Esther.

\DicoEntry{AUTEL}\textit{, de l'hébreu « mizbeach » : « autel »}\newline
Table généralement façonnée avec des monticules de pierres ou en terre et élevée spécialement pour offrir des holocaustes* et des sacrifices en l'honneur de Dieu. Voir Ge. 12:7 ; Ge. 35:7 ; Ex. 20:24-26 et Ex. 30:1-8.

\DicoEntry{BAAL}\textit{, de l'hébreu « Ba'al » : « maître, possesseur, seigneur »}\newline
Dieu primaire des Phéniciens et des Cananéens auquel les Israélites s'attachèrent à plusieurs reprises pour l'adorer. Voir No. 25:3 ; Jg. 2:11 et 1 R. 18:21.

\DicoEntry{BABEL ou BABYLONE}\textit{, de l'hébreu « Babel » : « confusion (par le mélange) »}\newline
Ville de Mésopotamie* située sur l'Euphrate, capitale de la Babylonie. Les hommes y entreprirent la construction de la tour de Babel. Cependant, Yahweh confondit leur langage et les dispersa sur toute la terre. Voir Ge 10:8-10 et Ge 11:1-9.

\DicoEntry{BALAAM}\textit{, de l'hébreu « Bil'am » : « sans peuple », « dévorant »}\newline
Prophète de Yahweh ayant vécu pendant la marche d'Israël dans le désert, il fut séduit par Balak, roi de Moab, qui lui proposa de maudire Israël contre de généreux présents. Son témoignage a été utilisé plusieurs fois pour avertir les enfants de Dieu des scandales* dont ils pourraient être la cause en suivant la voie de la cupidité. Voir No. 22-24 ; No 31:8 ; Jud. 1:11 et Ap. 2:14.

\DicoEntry{BALAK}\textit{, de l'hébreu « Balaq » : « gaspilleur, dévastateur »}\newline
Roi de Moab, il essaya de convaincre Balaam* de maudire Israël qu'il redoutait. Voir No. 22-24.

\DicoEntry{BANNIERE}\textit{, de l'hébreu « nec » : « quelque chose de levé, étendard, signal, enseigne »}\newline
Drapeau, étendard élevé en signe d'appartenance à ce qu'il représente. Moise bâtit un autel du nom de Yahweh-Nissi : « Yahweh ma bannière ». Voir Ps. 60:4 ; Es. 11:10, 12 et Ex. 17:15.

\DicoEntry{BAPTEME}\textit{, du grec « baptizo » : « plonger, immerger, purifier en plongeant »}\newline
On distingue trois types de baptêmes dans les Ecritures (Mt. 3:11) :
\\1. le baptême d'eau : acte suivant la conversion par lequel une personne est immergée dans l'eau - symbolisant la mort et la résurrection en Jésus-Christ. Il s'agit selon Pierre de l'« engagement d'une bonne conscience envers Dieu ». Voir Ac. 2:38 ; Ac. 16:30-33 ; Col. 2:12-13 et 1 Pi. 3:21.
\\2. le baptême du Saint-Esprit : lors de la naissance d'en haut, gage que le Seigneur donne au nouveau converti par l'envoi du Saint-Esprit. Voir Jn. 3:5-6 ; Ti. 3:4-7 et Ep. 1:13.
\\3. le baptême de feu : symbole des souffrances que Christ a endurées à la croix et par lesquelles tous les chrétiens sont appelés à passer pour être purifiés. Voir Lu. 12:50 ; Mc. 10:35-39 ; 1 Pi. 1:6-9 et 1 Pi. 4:12-13.

\DicoEntry{BARAK}\textit{, de l'hébreu « Baraq » : « éclairs, foudre »}\newline
Fils d'Abinoam, issu de la tribu de Nephtali, il vécut en Israël au temps des juges. Encouragé et accompagné par Débora, il battit l'armée de Jabin, roi de Canaan. Voir Jg. 4.

\DicoEntry{BARTHELEMY}\textit{, du grec « Bartholomaios » : « fils de Tolmaï »}\newline
Un des douze apôtres de Jésus. Voir Mt. 10:3.

\DicoEntry{BARTIMEE}\textit{, du grec « Bartimaios » : « fils de Timée »}\newline
Fils de Timée, mendiant aveugle que Jésus guérit suite à ses cris de supplications sur la route de Jéricho. Voir Mc. 10:46-52.

\DicoEntry{BATH-SCHEBA}\textit{, de l'hébreu « Bath-Sheba` » : « fille d'un serment »}\newline
Fille d'Eliam, femme d'Urie* le héthien que David fit mourir après l'avoir mise enceinte. Elle devint la femme de David et fut la mère de Salomon. Voir 2 S. 11:3-5, 26-27 et 2 S. 12:24-25.

\DicoEntry{BEELZEBUL}\textit{, de l'hébreu « Ba'al-Zebuwb » : « seigneur des mouches »}\newline
Divinité adorée par les philistins et considérée comme le prince des démons. Voir 2 R. 1:2-6 et Mc. 3:22-26.

\DicoEntry{BELIAL}\textit{, de l'hébreu « Beliya'al » : « indignité »}\newline
Symbolisant l'infidélité, la méchanceté et la perversité, il s'agit d'un autre nom de Satan. Voir De. 15:9 ; 1 S. 1:16 ; 2 Co. 6:15.

\DicoEntry{BENEDICTION}\textit{, de l'hébreu « barak », « berakah », et du grec « eulogia » : « louange »}\newline
Parole au travers de laquelle le Seigneur annonce sa grâce sur la vie d'une personne ou d'un peuple ; les bontés liées à la bénédiction sont cependant conditionnées par l'obéissance du bénéficiaire. Sous l'Ancienne Alliance, les pères avaient coutume de bénir leurs enfants ; la bénédiction se manifestait souvent par la prospérité matérielle, la fécondité et la santé. La bénédiction est la marque du chrétien qui voit avec un œil spirituel la faveur de Dieu dans sa vie et qui bénit Dieu dans toutes les circonstances. Voir Ge. 49:1-28 ; De. 28:1-14 ; Ps. 103:1-2 et Ep. 1:3.

\DicoEntry{BENJAMIN}\textit{, de l'hébreu « Binyamiyn » : « fils de ma main droite »}\newline
Dernier fils de Jacob et Rachel ; sa mère mourut en lui donnant naissance. Il est l'ancêtre de la tribu de Benjamin. Voir Ge. 35:16-18 et Ge. 49:27.

\DicoEntry{BETE}\textit{, de l'araméen « cheyva´ » et du grec « therion » : « bête, animal »}\newline
Dans les récits à caractère apocalyptique, les bêtes sont des animaux symbolisant les puissances politiques. Voir Dn. 7 et Ap. 13, 17.

\DicoEntry{BETHANIE}\textit{, du grec « Bethania » : « maison des dattes non mûres », « maison de l'affligé »}\newline
Village proche de Jérusalem, près de la montagne des oliviers, où vivaient Simon le lépreux, Marthe*, Marie* et Lazare* que Jésus ressuscita des morts. Voir Mc. 11:1 ; Mc. 14:3 et Jn. 11:1.

\DicoEntry{BETHEL}\textit{, de l'hébreu « Beyth-'El » : « maison de Dieu »}\newline
Ville cananéenne située à l'occident de Aï. Autrefois appelé Luz - mais renommé par Jacob quand il y eut la visitation de Yahweh - Béthel devient la possession de la tribu d'Ephraïm lors de la conquête de Canaan conduite par Josué. Elle était connue pour être un lieu d'adoration où on y rendait un culte à Yahweh. Malheureusement suite au schisme d'Israël - et notamment sous le règne de Jéroboam, roi de Juda - elle devient un lieu d'abomination. C'est Josias son successeur qui, désirant marcher avec Yahweh, y ôta les faux dieux, rétablissant ainsi le culte en l'honneur du Dieu d'Israël. (Ge. 28:10-22 ; Ge. 31:13 ; 1 R. 12:26-32 ; 2 R. 23:1-15)

\DicoEntry{BETHLEHEM}\textit{, de l'hébreu « Beyth Lechem » : « maison du pain »}\newline
Ville de Juda, lieu de naissance de David et de Jésus-Christ. Voir 1 S. 16 ; Mt. 2:16 et Lu. 2:4-7.

\DicoEntry{BIBLE}\textit{, du grec « biblia » : « livres »}\newline
Aussi appelée « Parole de Dieu », recueil de livres inspirés de Dieu et utiles pour enseigner, convaincre, corriger et instruire dans la justice. Voir 2 Ti. 3:16.

\DicoEntry{BLASPHEME}\textit{, de l'hébreu « na'ats » : « repousser, mépriser, rejeter » et du grec « blasphemia » : « discours impie et injurieux envers Dieu »}\newline
Parole outrageante ou insultante envers Dieu. Voir 2 S. 12:14 et Ap. 16:9.

\DicoEntry{BLASPHEME CONTRE LE SAINT-ESPRIT}\textit{}\newline
Voir commentaire Mt. 12:22-32.

\DicoEntry{BOAZ}\textit{, de l'hébreu « Bo`az » : « en lui est la force »}\newline
Fils de Salmon et arrière-grand-père du roi David, il épousa Ruth la Moabite. Voir Ru. 4:13 et Mt. 1:1-6.

\DicoEntry{BREBIS}\textit{}\newline
Femelle du bélier, c'est l'animal pour qui le berger donne sa vie. Elle est le symbole du véritable disciple qui n'obéit qu'à la voix de son Maître et qui se laisse conduire et choyer par Jésus, le bon berger. Voir Jn. 10:1-16.

\DicoEntry{CAIN}\textit{, de l'hébreu « Qayin » : « possession », « artisan, forgeron »}\newline
Fils aîné d'Adam et Eve, il fut l'auteur du premier homicide en tuant son frère Abel. Il engendra Lémec, premier polygame de l'histoire. Voir Ge. 4:1-8, 16-19.

\DicoEntry{CAIPHE}\textit{, du grec « Kaiaphas » : « avenant, pierre »}\newline
Souverain sacrificateur nommé par Valerius Gratus, gouverneur de Judée de 15 à 26 ap. J.-C. Caïphe exerça sa fonction de 18 à 36. N'ayant pas reconnu en Christ le Messie, il déclara néanmoins qu'il était avantageux qu'un seul homme meure pour le peuple et participa à la condamnation à mort de Jésus. Voir Mt. 26:3, 57-66 ; Jn. 11:47-53 et Jn. 18:12-14

\DicoEntry{CALEB}\textit{, de l'hébreu « Kaleb » : « chien »}\newline
Fils de Jephunné, issu de la tribu de Juda, il fut l'un des espions envoyés pour explorer le pays de Canaan. Avec Josué, il fut le seul, parmi la génération sortie d'Egypte, à entrer dans la terre promise. Voir No. 13:1-6 et No. 14:22-30.

\DicoEntry{CALENDRIER HEBRAIQUE}\textit{}\newline
Nisan (ou Abib) = Mars ; Iyyar (ou Ziv) = Avril ; Sivan = Mai ; Thammuz = Juin ; Ab = Juillet ; Elul = Août ; Tisri (ou Ethanim) = Septembre ; Marchesvan (ou Bul) = Octobre ; Chislev (ou Kisleu) = Novembre ; Tébeth = Décembre ; Schebat = Janvier ; Adar = Février

\DicoEntry{CAMP}\textit{, de l'hébreu « machaneh » : « campement, camp »}\newline
Lieu de stationnement temporaire d'un groupement civil ou militaire. Voir Ge. 32:2 et Ex. 14:19.

\DicoEntry{CANAAN}\textit{, de l'hébreu « Kena'an » : « terre basse », « marchand »}\newline
Fils de Cham. Ses descendants occupèrent la région éponyme qui correspond plus ou moins aujourd'hui aux territoires réunissant la Palestine, l'État d'Israël, l'ouest de la Jordanie, le sud du Liban et l'ouest de la Syrie. Ce territoire correspondait également à la terre promise par Dieu aux Israélites dont ils prirent possession sous la conduite de Josué. Voir Ge. 9:18 ; Jos. 6 à 21 et Ac. 13:19.

\DicoEntry{CESAR, Jules}\textit{, (100 av. J.-C - 44 av. J.-C.) du latin « kaisar » : « séparé », « chef »}\newline
Général romain. Son nom devint par la suite celui de certains empereurs romains. Dans les Ecritures, César symbolise également les autorités séculières. Voir Mt. 22:21.

\DicoEntry{CESAREE de Philippes}\textit{, du grec « Kaisereia » : « appartenant à César »}\newline
Située près des sources du Jourdain, territoire qui doit son nom à l'empereur Tibère. C'est dans cette contrée que Pierre reconnut en Jésus le Messie, le Fils du Dieu vivant. Voir Mt. 16:13-17.

\DicoEntry{CHAIR}\textit{, du grec « sarx » : « la chair, le corps, la nature sensuelle de l'homme, la nature animale »}\newline
Selon le contexte, désigne le corps humain, l'être humain ou la nature humaine conduite par le péché*. Voir Lu. 24:39 ; Lu. 3:6 ; Jn. 17:2 ; Ro. 8:5-9 ; Ga. 5:16-21 et Ep. 2:3.

\DicoEntry{CHALDEE}\textit{, de l'hébreu « Kasdiy » : « briseurs de mottes », « comme des démons »}\newline
Région située au sud de la Mésopotamie dont Abraham est originaire. Voir Ge. 11:28.

\DicoEntry{CHAM}\textit{, de l'hébreu « Cham » : « chaud, bouillant »}\newline
Fils de Noé et père de Canaan qui fut maudit par Noé. Voir Ge. 9:18-27.

\DicoEntry{CHARAN}\textit{, de l'hébreu « Charan » : « montagnard », « route, caravane »}\newline
Région proche d'Ur en Chaldée où Abraham séjourna jusqu'à la mort de son père Térach. Voir Ge. 11:31 et Ge. 12:4.

\DicoEntry{CHEMIN DE SABBAT}\textit{}\newline
Selon la loi de Moïse, distance maximum que les juifs peuvent parcourir de leur demeure le jour du sabbat* (cf tableau mesures.distances). Voir Ac. 1:12.

\DicoEntry{CHERUBINS}\textit{, de l'hébreu « keruwb » : « être angélique, chérubin »}\newline
Catégorie d'anges portant et.ou gardant la gloire de Dieu. Yahweh en avait placé à l'entrée du jardin d'Eden pour empêcher l'homme d'y accéder. Deux chérubins sur lesquels Dieu siégeait étaient représentés sur le propitiatoire*. Avant sa chute, Satan était un chérubin protecteur. Voir Ge. 3:24 ; Ex. 25:17-20 ; Es. 37:16 et Ez. 28:14.

\DicoEntry{CHRETIEN}\textit{, du grec « christanos » : « de Christ », « petit christ », « comme Christ »}\newline
Comme son étymologie le suggère, le chrétien appartient à Christ, dont il a la nature et à qui il ressemble. Il est donc un disciple* de Jésus-Christ qui suit son enseignement et le met en pratique. Ce terme fut employé pour la première fois à Antioche. Voir Ac. 11:26.

\DicoEntry{CHRIST}\textit{, du grec « christos » et de l'hébreu « mashiyach » : « oint »}\newline
Souvent accolé au nom de Jésus*, ce terme suggère que ce dernier est l'oint* de Dieu, le Messie tant attendu. Jésus annonça l'émergence de faux christs (= faux ouvriers de Christ) à la fin des temps. Voir Ro. 1:1 ; Mt. 16:15-16 ; Hé. 1:9 ; Mt. 24:24 et Mc. 13:22-23.

\DicoEntry{CIRCONCISION}\textit{, de l'hébreu « muwlah » : « circoncision : couper autour »}\newline
Section et ablation du prépuce. En signe d'alliance, Dieu ordonna à Abraham de circoncire tous les mâles de sa maison ; les enfants d'Israël ont perpétré cette pratique. Sous la Nouvelle Alliance, la circoncision requise est celle du cœur. Voir Ge. 17:9-14 ; Lu. 1:59 ; Ro. 2:25-29 et 1 Co. 7:19.

\DicoEntry{CLAUDE}\textit{, (10 av. J.-C. – 54 ap. J.-C.), du grec « Klaudios » : « boiteux »}\newline
Fils de Nero Claudius Drusus (38 av. J.-C. – 9 av. J.-C.). Empereur romain qui régna de 41 à 54 ap. J.-C. ; il chassa les juifs de Rome, parmi lesquels Aquilas et Priscille. Voir Ac. 18:2.

\DicoEntry{CLERGE}\textit{, du grec « klêrikos » : « homme d'église »}\newline
Au sein de l'Eglise catholique, corps séparé des fidèles ayant une fonction gouvernante ; ses membres sont appelés les clercs ou les ecclésiastiques. Ils accèdent à leur position par le sacrement de l'ordre (ou ordination*) qui comporte trois classes : les diacres, les prêtres et les évêques.

\DicoEntry{CLERICALISME}\textit{, dérivé de clérical : « dévoué aux intérêts du clergé »}\newline
Tendance en vertu de laquelle le clergé sort du domaine religieux pour se mêler des affaires publiques et politiques afin d'y exercer une influence et faire prédominer ses idées.

\DicoEntry{CŒUR}\textit{, de l'hébreu « lebab » : « homme intérieur, volonté, cœur, partie interne, pensée »}\newline
Organe permettant la circulation du sang, les Ecritures définissent le cœur comme un grand abîme. Siège des émotions et des pensées intimes, il peut être une bonne ou une mauvaise source. Voir Ge. 20:6 ; Lé. 19:17 ; De. 4:29 ; 1 S. 12:24 et Mc. 7:21.

\DicoEntry{COLOSSES}\textit{, du grec « Kolossai » : « monstruosités »}\newline
Située en Asie Mineure, ville de Phrygie se trouvant à environ deux cents kilomètres d'Ephèse. Il s'y trouvait une église à qui Paul écrivit une lettre qui figure dans le canon biblique.

\DicoEntry{COMMUNION}\textit{, du grec « koinonia » : « ce qui est commun à plusieurs personnes, association, union »}\newline
Le disciple* de Christ est appelé à vivre deux types de communion. Il doit tout d'abord être en communion intime avec Dieu puis avec d'autres membres du corps de Christ pour vivre la communion fraternelle. Voir 1 Jn. 1:3 ; 2 Co. 13:11-13 ; Ps. 133 et Ac. 2:42.

\DicoEntry{CONCILE}\textit{, du latin « concilium » : « assemblée »}\newline
Assemblée d'évêques de l'Eglise catholique (également connue sous l'appellation « pères de l'Eglise catholique ») réunis dans le but de définir les règles de la foi chrétienne. Cette pratique va à l'encontre du message de Christ puisqu'il a strictement condamné la modification du message qu'il a lui-même prêché et confié aux apôtres*. Voir Mt. 5:18 et Ga. 1:8-9.

\DicoEntry{CONFESSION}\textit{, du grec « exomologeo » : « confesser, professer, reconnaître ouvertement »}\newline
On peut confesser des péchés pour exposer les ténèbres ou le nom du Seigneur pour le louer et annoncer la vérité. Voir Mc. 1:5 ; Ac. 19:18 et Ph. 2:11.

\DicoEntry{CONVERSION}\textit{, du grec « epistrepho » : « action de se retourner, de se tourner vers »}\newline
Fruit d'une sincère repentance, la conversion est la décision de se tourner vers Christ et de se détourner des œuvres des ténèbres. Voir Ac. 26:20 ; 2 Co. 3:16 ; Ga. 4:9 et 1 Pi. 2:25.

\DicoEntry{CONVOITISE}\textit{, du grec « epithumia » : « désir, convoitise, luxure »}\newline
Précédant l'acte du péché, désir amorcé par les sens humains et lié à la soif de posséder ce qui est défendu et ce que le monde offre. Voir Ja. 1: 14-15 et 1 Jn. 2:15-17.

\DicoEntry{CORINTHE}\textit{, du grec « Korinthos » : « rassasié »}\newline
Dans l'antiquité, Corinthe, capitale de l'Achaïe, était la ville la plus prospère et la plus puissante de Grèce. Située sur un isthme séparant la mer Egée de la mer Ionienne, Corinthe était au carrefour de l'Asie et de l'Italie et constituait un véritable centre commercial où les produits orientaux et occidentaux se croisaient. Paul demeura au moins un an et six mois à Corinthe, durée pendant laquelle il enseigna la parole de Dieu. Il écrivit par la suite deux lettres aux saints de cette ville qu'on retrouve dans le canon biblique.

\DicoEntry{CORNEILLE}\textit{, du grec « Kornelios » : « d'une corne »}\newline
Centenier romain juste et craignant Dieu. Il vivait à Césarée où Simon Pierre fut envoyé pour lui annoncer la Parole. Au travers de l'expérience de Corneille, Dieu confirma que le salut était pour toutes les nations. Voir Ac. 10.

\DicoEntry{COURONNE}\textit{, du grec « stephanos » : « couronne, une marque de rang royal, récompense de la justice, ornement »}\newline
Jésus-Christ reçut une couronne d'épines lors de la crucifixion pour rappeler ironiquement son titre de « roi des juifs ». Après la résurrection, les chrétiens recevront une couronne en récompense de leur intégrité. Devant le trône de Dieu, les vingt-quatre vieillards jettent leurs couronnes pour rendre gloire à Dieu. Voir Mt. 27:29 ; 1 Co. 9:25 ; 2 Ti. 4:8 ; Ja. 1:12 ; 1 Pi. 5:4 ; Ap. 2:10 et Ap. 4:4, 10.

\DicoEntry{CROIX}\textit{, du grec « stauros » : « pieu, croix »}\newline
Châtiment romain consistant à clouer les mains et les pieds des condamnés sur des poteaux en bois en forme de croix. Symbole du sacrifice de Jésus pour le pardon des péchés, la croix est aussi l'image de la vie de souffrance et de consécration totale à laquelle est appelé tout disciple du Seigneur. Voir Es. 53 ; Lu. 9:23 et Mt. 16:24.

\DicoEntry{CUPIDITE}\textit{, du grec « pleonexia » : « désir avide d'avoir plus, avarice »}\newline
Forme d'idolâtrie*, péché consistant à désirer de manière excessive les biens de ce monde (argent, richesses, etc.) et menant à la perdition. Voir Ep. 5:3 ; Col. 3:5 et 2 Pi. 2:14.

\DicoEntry{CYRENE}\textit{, du grec « Kurene » : « suprématie de la bride », « qui gouverne, froid »}\newline
Ville prospère située dans la région fertile d'Afrique du Nord (actuelle Libye) où vivait une importante communauté juive et de laquelle était originaire Simon à qui l'on demanda de porter la croix de Jésus. Voir Mc. 15:20-22 et Ac. 2:10.

\DicoEntry{CYRUS II LE GRAND }\textit{(règne : 559-530 av. J.-C.), du persan « Kowresh » : « possède la puissance, puissance suprême »}\newline
Fils de Cambyse, il régna sur l'Empire perse. Réveillé par Yahweh, il publia un édit en faveur du retour des Juifs à Jérusalem pour la reconstruction du temple. Voir 2 Ch. 36:22-23 et Esd. 1:1-2.

\DicoEntry{DAGON}\textit{, « Dagown » : « un poisson »}\newline
Divinité païenne adorée par les philistins, il était représenté par un personnage avec des mains et une face humaine et le corps d'un poisson. Voir 1 S. 5:1-5.

\DicoEntry{DAN}\textit{, de l'hébreu « dan » : « un juge »}\newline
Fils de Jacob et de Bilha, servante de Rachel, il est le père de la tribu des Danites. Voir Ge. 30:1-6 et Ge. 49:16-18.

\DicoEntry{DANIEL}\textit{, de l'hébreu « Daniye'l » : « Dieu est mon juge »}\newline
Issu d'une famille princière de Juda, il fut déporté pendant sa jeunesse de Jérusalem à Babylone où il reçut le nom de Beltshatsar. Son histoire est racontée dans le livre éponyme.

\DicoEntry{DARIQUE}\textit{, de l'hébreu « darkemown » : « darique, drachme, unité de mesure »}\newline
Utilisée après le retour de l'exil babylonien, monnaie d'or mise en place par le roi Darius et circulant dans l'Empire perse. Voir Esd. 8:26-27 et Né. 7:71-72.

\DicoEntry{DARIUS Ier}\textit{, (règne : 522 av. J.-C. – 486 av. J.-C.), de l'hébreu « Dar`yavesh » : « seigneur » (origine : perse)}\newline
Fils d'Assuérus, d'origine mède, roi des Chaldéens. Il encouragea la reconstruction du temple de Jérusalem après la découverte des instructions laissées par Cyrus sur un rouleau retrouvé dans la province de Médie. Voir Esd. 6.

\DicoEntry{DATHAN}\textit{, de l'hébreu « Dathan » : « appartenant à une fontaine »}\newline
Issu de la tribu de Ruben, fils d'Eliab et frère d'Abiram, il participa avec Koré à la révolte contre Moïse et Aaron. Voir No. 16:1-35.

\DicoEntry{DAVID}\textit{, de l'hébreu « David » : « bien aimé »}\newline
Issu de la tribu de Juda et dernier fils d'Isaï, il entra dès son plus jeune âge au service du roi Saül avant de devenir roi d'Israël. Homme selon le cœur de Dieu, il connut de grands succès sur les champs de bataille et fut l'auteur de nombreux psaumes. Il régna quarante-quatre ans sur Israël puis son fils Salomon* lui succéda. Voir 1 S. 13:14 ; 1 S. 16:14-23 ; 1 S. 17 ; 1 R. 2:10-11 et Ac. 13:22.

\DicoEntry{DEBORA}\textit{, de l'hébreu « Debowrah » : « abeille »}\newline
Femme de Lapiddoth, elle exerça les fonctions de prophétesse et juge en Israël. Elle fut utilisée par Dieu pour prophétiser la victoire d'Israël sur Canaan par Barak qu'elle accompagna sur le champ de bataille. Voir Jg. 4 et 5.

\DicoEntry{DELUGE}\textit{, de l'hébreu « mabbuwl » : « inondation, déluge »}\newline
Pluie torrentielle s'étant abattue sur la terre pendant quarante jours et quarante nuits au temps de Noé. Le déluge symbolisait le jugement de Dieu sur une génération dont la méchanceté avait atteint un niveau sans précédent. Tous les habitants et les animaux de la terre furent emportés par les eaux du déluge hormis Noé, sa famille et les animaux qui étaient avec eux dans l'arche*. Voir Ge. 6 à 8.

\DicoEntry{DEMAS}\textit{, du grec « Demas » : « gouverneur du peuple »}\newline
Compagnon d'œuvre de Paul qui le délaissa « par amour pour le siècle présent ». Voir Col. 4:14 et 2 Ti. 4:10.

\DicoEntry{DEMETRIUS}\textit{, du grec « Demetrios » : « qui appartient à Déméter (déesse grecque de l'agriculture) »}\newline
Orfèvre qui fabriquait des statues de la déesse Diane à Ephèse. Voyant son commerce mis en danger par les prédications de Paul, il déclencha une émeute contre ce dernier. Voir Ac. 19:23-41.

\DicoEntry{DEMONS}\textit{, du grec « daimonion » : « divinité inférieure, mauvais esprit, ministres du diable »}\newline
Egalement appelés « esprits impurs », anges* déchus ayant pris part à la révolte et à la chute de Satan*. Ils peuvent posséder le corps d'une personne, mais sont soumis à la puissance de Jésus, au nom duquel les chrétiens peuvent les chasser. Voir Mt. 10:8 ; Lu. 10:17 ; Mc. 7:26 ; Mc. 16:17 ; Lu. 4:33 ; Jud. 1:6 et Ap. 12:4.

\DicoEntry{DIABLE}\textit{}\newline
Voir SATAN.

\DicoEntry{DIACRE}\textit{, du grec « diakonos » : « domestique, subordonné, messager »}\newline
Les premiers diacres étaient des hommes remplis de l'Esprit Saint et de sagesse ; ils furent nommés pour faire un travail complémentaire aux ministres de la Parole au sein de l'église de Jérusalem. Etienne* était l'un d'eux. Il existait aussi des femmes diaconesses comme Phoebe, de l'église de Cenchrées. Voir Ac. 6:1-8 ; Ro. 16:1-2 et 1 Ti. 3:8-13.

\DicoEntry{DIANE}\textit{, du grec « Artemis » : « de la lumière »}\newline
Aussi appelée « Artemis d'Ephèse », divinité révérée dans toute l'Asie. Il existait un temple en son honneur à Ephèse. Voir Ac. 19:24-37.

\DicoEntry{DIEU}\textit{}\newline
Dieu des dieux et Seigneur des seigneurs, il est le Créateur de l'univers, du ciel, de la terre et de tout ce qui s'y trouve. Architecte d'excellence, il forma l'homme à son image et lui manifesta un amour inconditionnel par son incarnation en Jésus-Christ*. Dieu se présenta à Moïse sous le nom YHWH* (=Je suis celui qui suis) montrant son caractère éternel. Il s'est révélé à différentes personnes sous divers noms et aspects, en fonction des situations traversées montrant qu'il est celui qui remplit tout en tous et qu'il est et a tout ce dont l'homme a besoin. Ainsi, on le découvre dans les Ecritures comme étant grand, unique et indivisible, omniprésent, omniscient, souverain, incorruptible, sage, patient, saint, parfait, merveilleux, tout-puissant, fidèle, juste et bon. Bien évidemment, Dieu ne peut en aucun cas être défini dans tout ce qu'il est, dans la mesure où sa nature même échappe à toute possibilité de frontière ou de limite. Toutefois, les saints auront l'éternité pour découvrir ce Père incomparable. Voir Ge. 1, 2 ; Ge. 17:1 ; Ex. 3:14 ; De. 6:14 ; De. 10:17 ; Ps. 11:7 ; Ps. 139:7-10 ; Es. 6:3 ; La. 3:22-23 ; Mal. 3:6 ; Lu. 1:49 ; 1 Co. 1:9 ; Ro. 1:23 ; Ro. 2:4 ; Ro. 11:33-36 ; 2 Ti. 4:8 ; Hé. 4:13 ; Ja. 1:17 1 Th. 4:17 et 1 Jn. 4:8.

\DicoEntry{DIME}\textit{, de l'hébreu « ma'aser » : « dîme, dixième partie »}\newline
Abraham donna à Melchisédek la dîme du butin d'une bataille remportée (Ge. 14:17-20 et Hé. 7:1-2). Yahweh instaura, au travers de Moïse, la dîme comme une loi à respecter par les enfants d'Israël. Il en existait quatre sortes :
\\1. la dîme que les Lévites prélevaient sur le peuple (No. 18:21-24)
\\2. la dîme de la dîme, que les sacrificateurs prélevaient sur les Lévites (No. 18:25-31 ; Né. 10:38)
\\3. la dîme consommée par les Juifs eux-mêmes lors des fêtes de Yahweh (De. 14:22-26)
\\4. la dîme pour l'étranger, la veuve, l'orphelin et le lévite, donnée tous les trois ans (De. 14:28-29).
\\Cette loi concernait exclusivement Israël et non l'Eglise – Jésus-Christ ayant accompli la loi (Mt. 5:17). Sous la grâce, les chrétiens sont invités à faire des offrandes * librement et sans contrainte.

\DicoEntry{DINA}\textit{, de l'hébreu « Diynah » : « jugement, justice »}\newline
Fille de Jacob et Léa. Elle fut enlevée et déshonorée par Sichem, fils de Hamor, prince du pays de Canaan. Sichem et tous les hommes de la ville furent ensuite tués par les frères de la jeune fille, Siméon et Lévi. Voir Ge. 34.

\DicoEntry{DIOTREPHE}\textit{, du grec « Diotrephes » : « nourri par Zeus »}\newline
Chrétien dont Jean dénonça l'arrogance et les mauvais agissements. Voir 3 Jn. 1:9-11.

\DicoEntry{DISCIPLE}\textit{, du grec « mathetes » : « un étudiant, un élève, un disciple »}\newline
Personne qui écoute les enseignements de son maître et les met en pratique en vue de devenir comme lui. Jésus en choisit douze qu'il forma pendant son ministère. Le disciple de Christ doit manifester le caractère de son maître, lui être pleinement consacré et être prêt à souffrir en son nom. Voir Lu. 6:12-16 ; Lu. 14:26-33 et Mt. 10.

\DicoEntry{DIVORCE}\textit{, du grec « apostasion » : « divorce, répudiation, lettre de divorce »}\newline
Brisement des liens du mariage*. Il fut autorisé sous la loi de Moïse à cause de la dureté des cœurs, mais Christ rappela l'indissolubilité du mariage au commencement. Voir De. 24:1-3 et Mt. 19:3-8.

\DicoEntry{DOCTEUR}\textit{, du grec « didaskalos » : « professeur », « maître ».}\newline
Sous la loi de Moïse, les docteurs de la loi étaient chargés d'expliquer la Torah. Certains d'entre eux s'opposèrent à Jésus. Sous la Nouvelle Alliance, le docteur est un des cinq ministères de la Parole évoqués en Ep. 4 :11. Il enseigne la Parole de Dieu qui guérit les blessures de l'âme. Selon Jc. 3 :1, nous ne sommes pas tous appelés à être des docteurs. Voir Lu. 2:46 ; Lu. 5:17 ; 1 Co. 12:28 et Ep. 4:11.

\DicoEntry{DONS SPIRITUELS}\textit{, du grec « charisma » : « faveur que reçoit quelqu'un sans aucun mérite de sa part », « dons provenant du pouvoir de la grâce divine »}\newline
Capacités distribuées par le Saint-Esprit aux chrétiens en vue de la formation et de l'édification des saints. Voir 1 Co. 12:1-11 ; 1 Co. 14:12 ; Ro. 12:6 et 1 Pi. 4:10.

\DicoEntry{EDEN}\textit{, de l'hébreu « 'Eden » : « plaisir, délices »}\newline
Appelé aussi jardin de Dieu, premier lieu de résidence d'Adam et Eve. Yahweh y avait fait pousser des arbres de toutes espèces ; il y avait également placé au milieu l'arbre de vie* ainsi que l'arbre de la connaissance du bien et du mal* dont la consommation des fruits conduirait à la mort. L'homme fut établi en tant que gardien et cultivateur de ce jardin. Cependant, il pêcha avec la femme et ils furent chassés de ce lieu des délices. Voir Ez. 28:13 ; Ge. 2 et Ge. 3:23-24.

\DicoEntry{EGLISE}\textit{, du grec « ekklesia » : « appel hors de »}\newline
Peuple mis à part dont Christ est le chef. L'Eglise est la sainte habitation de Dieu en esprit, le corps de Christ, l'épouse de l'Agneau. On distingue l'Eglise universelle - qui regroupe tous les saints du monde entier - de l'église locale - qui est composée de tous les chrétiens d'une ville. Voir 1 Co. 3:16 ; Ep. 2:20-22 ; 1 Co. 12:27 ; Ep. 5:22-32 ; Ac. 2:47 ; 1 Co. 1:2 ; 1 Th. 1:1 ; Ph. 1:1 et 1 Ti. 2:4.

\DicoEntry{ELEAZAR}\textit{, de l'hébreu « 'El'azar » : « Dieu a secouru »}\newline
Fils d'Aaron, il était chef des chefs des Lévites avant de devenir le second souverain sacrificateur d'Israël. Voir No. 3:32 et No. 20:25-28.

\DicoEntry{ELECTION, ELU}\textit{, de l'hébreu « bachiyr » et du grec « eklektos » : « choisi, élu de Dieu »}\newline
Dans le Tanakh, Israël fut présenté comme le peuple élu de Yahweh, appelé à être un exemple pour toutes les nations de la terre. Cette élection n'est pas synonyme de préférence, car la volonté de Dieu est de sauver tous les hommes. Au travers de l'œuvre de la croix, Dieu a effectivement montré que son choix se porte vers l'humanité tout entière en payant le prix des péchés de tous. Dans son omniscience, il sait toutefois d'avance qui croira en lui ou pas. Selon la parole, même après la conversion, le chrétien doit travailler son élection, c'est-à-dire se sanctifier et obéir aux commandements de Yahweh pour entrer dans son royaume. Voir Es. 45:4 ; Es. 49:6 ; Mt. 22:14 ; Ep. 1: 4-6 et 2 Pi. 1:10-11.

\DicoEntry{ELIE}\textit{, de l'hébreu « Eliyah » : « Yahweh est mon Dieu »}\newline
Prophète d'origine tschibite que Dieu suscita en Israël au temps du roi Achab*. Il ne connut point la mort, mais fut enlevé par le Seigneur. Son histoire, ses combats et ses exploits sont racontés dans les livres des Rois.

\DicoEntry{ELISEE}\textit{, de l'hébreu « 'Eliysha' » : « Dieu est sauveur »}\newline
Prophète du royaume d'Israël, il succéda à Elie après avoir reçu la double portion de l'esprit qui était sur ce dernier. Après sa mort, ses os rendirent la vie à un défunt. Voir 1 R 19:16-21 ; 2 R. 2:9-11 et 2 R. 13:20-21.

\DicoEntry{ENFER}\textit{}\newline
Voir SEJOUR DES MORTS.

\DicoEntry{ENLEVEMENT}\textit{, du grec « metathesis » : « transfert d'un lieu à un autre, changement »}\newline
Ravissement d'hommes au ciel sans que ces derniers ne connaissent la mort*. Dans le Tanakh, se trouvent deux cas d'enlèvement : Hénoch (Ge. 5:24 ; Hé. 11:5) et Elie (2 R. 2:11). L'Eglise sera de même enlevée par le Seigneur au son de la dernière trompette. Voir 1 Co. 15:51-57 et 1 Th. 4:17.

\DicoEntry{EPHESE}\textit{, du grec « Ephesos » : « permis »}\newline
Une des principales villes de l'Empire romain sous le règne de l'empereur Claude 1er (10 av. J.-C. – 54 ap. J.-C.) Ephèse possédait le plus grand port de l'Asie Mineure, ce qui lui attribuait le contrôle du trafic commercial. Richissime et prospère, elle était renommée pour son faste, sa liberté de parole et constituait donc un endroit privilégié pour les philosophes. L'église d'Ephèse naquit du ministère de Paul, qui y enseigna pendant au moins deux ans lors de son troisième voyage missionnaire. Cette église - figurant parmi les sept du livre d'Apocalypse - fit preuve de discernement et pratiquait de bonnes œuvres, mais le Seigneur avait néanmoins un reproche à lui adresser. Elle représente l'église apostate. Voir Epître aux éphésiens et Ap. 2:1-7.

\DicoEntry{EPHOD}\textit{, de l'hébreu « ephowd » : « couverture »}\newline
Vêtement que les sacrificateurs portaient par-dessus leur tunique lorsqu'ils étaient en service. L'éphod du souverain sacrificateur était de broderie ; le pectoral était posé sur son devant. Voir Ex. 28 et Lé. 8:7.

\DicoEntry{EPHRAIM}\textit{, de l'hébreu « 'Ephrayim »: « double fertilité »}\newline
Second fils de Joseph né en Egypte, il fut adopté par Jacob avant sa mort et devint ainsi l'ancêtre d'une des douze tribus d'Israël. Voir Ge. 41:52 ; Ge. 48:5 et Jos. 14:4.

\DicoEntry{EPICURIENS}\textit{, du grec « epikoureios » : « celui qui aide, le défenseur »}\newline
Fondé à Athènes en 306 av. J.-C., groupe de philosophes se réclamant de la doctrine d'Epicure (341 av. J.-C.- 270 av. J.-C.). Ce dernier fonda une des plus importantes écoles philosophiques de l'Antiquité. Il développa une théorie athée selon laquelle l'homme est encouragé à rechercher les plaisirs matériels et sensuels. Rejetant la pensée d'une vie après la mort, les épicuriens renient l'existence d'un créateur qui se préoccuperait des hommes. Des adeptes de cette philosophie se confrontèrent à la doctrine de Christ annoncée par l'apôtre Paul et cherchèrent à l'entendre. Voir Ac. 17:18-20.

\DicoEntry{ESAIE}\textit{, de l'hébreu « Yesha'yah » : « Yahweh a sauvé »}\newline
Fils d'Amotz, un prophète de Yahweh contemporain des rois Ozias, Jotham, Achaz et Ezéchias, il annonça la venue du Messie. L'ensemble de ses prophéties est contenu dans le livre portant son nom.

\DicoEntry{ESAU}\textit{, de l'hébreu « 'Esav » : « velu, poilu, chevelu »}\newline
Fils d'Isaac et Rebecca et frère jumeau de Jacob, qui lui soutira son droit d'aînesse et sa bénédiction. Il prit pour femmes Judith et Basmath, toutes deux originaires de Canaan. Egalement connu sous le nom d'Edom, il devint l'ancêtre des Edomites. Voir Ge. 25:25-34 ; Ge. 27 et Ge. 36.

\DicoEntry{ESDRAS}\textit{, de l'hébreu « `Ezra' » : « secours »}\newline
Fils de Sereja et descendant du souverain sacrificateur Aaron, Esdras était scribe et sacrificateur. Il enseigna le peuple de Dieu dans la loi et mit en place des réformes après la reconstruction du temple. Son histoire se trouve dans le livre éponyme.

\DicoEntry{ESPRIT}\textit{, de l'hébreu « ruwach » : « vent, souffle, esprit » et du grec « pneuma » : « vérité, inspiration, souffle, vent »}\newline
L'esprit humain est aussi appelé homme intérieur, il constitue la partie spirituelle de l'homme lui permettant d'agir, de prendre des décisions et d'être en contact avec Dieu ou tout autre esprit. Principe vital, il amène l'âme à la vie. Avec l'âme* et le corps, l'esprit constitue l'être humain. Voir Ge. 6:3 ; Ex. 31:3 ; Job 27:3 ; Job 32:8 ; Mt. 12:28 et 1 Th. 5:23.

\DicoEntry{ESPRIT IMPUR}\textit{}\newline
Voir DEMONS.

\DicoEntry{ESTHER}\textit{, dérivation du perse « 'Ecter » : « étoile »}\newline
Cousine de Mardochée, juif d'origine benjaminite, reine de Perse, épouse du roi Assuérus. Son nom juif était Hadassa : « myrte ». Son histoire, qui se déroula à Suse, est racontée dans le livre portant son nom.

\DicoEntry{ETANG DE FEU}\textit{}\newline
Lieu de douleur et de damnation éternelle créé initialement pour le diable et ses anges. Y seront jetés la bête et le faux prophète, le diable, la mort et le séjour des morts* ainsi que tous ceux dont le nom ne sera pas trouvé dans le livre de vie. Voir Mt. 25:41 ; Ap. 19:20 et Ap. 20:7-15.

\DicoEntry{ETIENNE}\textit{, du grec « stephanos » : « couronne »}\newline
Diacre* de l'église de Jérusalem rempli de sagesse et d'Esprit Saint. Premier martyr chrétien, sa mort marqua le début d'une grande persécution contre l'Eglise. Voir Ac. 6:1-6 ; Ac. 7 et Ac. 8:1-3.

\DicoEntry{EUNUQUE}\textit{, du grec « cariyc » : « eunuque, chambellan castré »}\newline
Homme dans l'incapacité de procréer ou émasculé. Dans l'antiquité, les rois se choisissaient des eunuques pour les servir. En les castrant, ils s'assuraient de la fidélité et l'intégrité de ces derniers. En outre, Jésus distingua trois types d'eunuques. Voir 2 R. 20:18 ; 1 Ch. 28:1 ; Da. 1:7 et Mt. 19:12).

\DicoEntry{EVANGELISTE}\textit{}\newline
Un des cinq ministères d'Ep. 4:11 dont la mission est de prêcher la repentance et la conversion à Jésus-Christ. Comme les ministères évoqués en Ep. 4 :11, il travaille également à la perfection des saints. Philippe exerça ce ministère, Timothée fut de même encouragé à faire l'œuvre d'un évangéliste. Tous les chrétiens doivent également évangéliser. Voir Ep. 4:11 ; Ac. 21:8 et 2 Ti. 4:5.

\DicoEntry{EVANGILE}\textit{}\newline
Enseignement donné par Jésus-Christ, la prédication de la croix (la mort et la résurrection de Jésus-Christ) et du royaume de Dieu qui s'est approché des hommes (voir ROYAUME DE DIEU). Ce message annonce le salut, la guérison du cœur, la joie en Jésus-Christ, la justice, la paix, la grâce et la vie éternelle accordée à l'homme repentant, mais aussi le jugement à venir. Les apôtres propagèrent l'Evangile ; de même, tous les chrétiens sont appelés à le faire. Voir Es. 61 ; Mt. 10:7 ; Mt. 28:19-20 ; Ro. 1:16 et 1 Co. 15:1-4 ; 2 Ti. 4 :1.

\DicoEntry{EVE}\textit{, de l'hébreu « Chavvah » : « vie »}\newline
Première femme et épouse d'Adam, elle fut formée à partir de la côte de son mari dans le but d'être l'aide de ce dernier. Séduite par Satan déguisé en serpent, elle mangea le fruit de la connaissance du bien et du mal et fut avec Adam chassée du jardin. Elle donna naissance à Caïn, Abel et Seth. Voir Ge. 2:18-24 ; Ge. 3:1-13 et Ge. 4:1-2, 25.

\DicoEntry{EVEQUE}\textit{, du grec « episcopos » : « investigation, inspection, visite d'inspection », « acte par lequel Dieu visite les hommes, observe leurs voies, leurs caractères, pour leur accorder en partage joie ou tristesse », « surveillance, contrôle, fonction d'un ancien », « la charge d'une église chrétienne ».}\newline
Il est question ici d'une fonction consistant à visiter les assemblées, les inspecter afin de s'assurer du bon ordre. Voir Lu. 19 :44 ; Ac. 1 :20 ; 1 Ti. 3 :1 ; 1 Pi. 2 :12.

\DicoEntry{EXPIATION}\textit{, de l'hébreu « kaphar » : « couvrir, purger, faire une expiation »}\newline
Action de couvrir les fautes et les souillures de l'homme afin qu'il soit réconcilié avec Dieu. Sous l'Ancienne Alliance, le souverain sacrificateur faisait tous les ans un sacrifice d'expiation en entrant dans le Saint des saints pour ses péchés et les péchés du peuple. Par son sacrifice, Christ est devenu la victime expiatoire pour les péchés de tous les hommes en les prenant sur lui à la croix ; il est l'Agneau de Dieu qui ôte les péchés du monde. Voir Lé. 16 ; Jn. 1:29 ; 1 Jn. 2:2 et 1 Jn. 4:10.

\DicoEntry{EZECHIAS}\textit{, de l'hébreu « Yechizqiyah » : « Yahweh est ma force »}\newline
Fils d'Osée, roi de Juda sur qui il régna vingt-neuf ans. Figurant parmi les rois les plus intègres, son règne fut caractérisé par la droiture et la fidélité à Yahweh. Voir 2 R. 18 et 19.

\DicoEntry{EZECHIEL}\textit{, de l'hébreu « Yechezqe'l » : « Dieu fortifie »}\newline
Fils de Buri, sacrificateur et prophète de Yahweh ayant été déporté à Babylone. Il reçut de nombreuses visions - sur son temps et les temps de la fin - racontées dans le livre qui porte son nom.

\DicoEntry{FELIX}\textit{, du grec « Phestos » : « joyeux, en fête »}\newline
Gouverneur de Judée de 52 à 60 ap. J.-C., il emprisonna Paul à la suite des plaintes des Juifs. S'entretenant avec lui de temps en temps et lui octroyant certaines libertés, Felix garda Paul en prison deux ans pour plaire aux Juifs. Voir Ac. 24.

\DicoEntry{FESTUS}\textit{, du grec « Phestos » : « en fête, joyeux »}\newline
Gouverneur de Judée qui succéda à Félix* de 60 à 62 ap. J.-C. Il poursuivit l'instruction du procès de Paul que les Juifs accusaient. Il permit à Paul de s'exprimer devant le roi Agrippa* et l'envoya à Rome afin qu'il comparaisse devant César*. Voir Ac. 24:27 et Ac. 25, 26.

\DicoEntry{FETES DE YAHWEH}\textit{}\newline
Selon la loi juive, sept fêtes étaient célébrées en l'honneur de Yahweh : la Pâque de Yahweh, la fête des pains sans levain ; la fête des prémices ; la Pentecôte ; la fête des trompettes ; le jour des expiations et la fête des Tabernacles. Voir Lé. 23:6-43.

\DicoEntry{FIGUIER}\textit{}\newline
Arbre fruitier sous lequel il était coutume d'étudier la Torah en Israël. Ses fruits excellents et doux servaient en médecine. Le figuier est retrouvé dans de nombreuses histoires et paraboles des Ecritures. Il symbolise la douceur et l'humilité. Voir Jg. 9:11 ; 2 R. 20:1-7 ; Jn. 1:43-51 et Lu. 13:6-9.

\DicoEntry{FILS DE DIEU}\textit{}\newline
Expression désignant selon le contexte :
\\1. les anges. Voir Ge. 6:2-4 ; Job 38:7 et Da. 3:25.
\\2. Adam. Voir Lu. 3:38.
\\3. les chrétiens. Voir Ro. 8:14 et Ga. 3:26.
\\4. Jésus-Christ, le Fils unique de Dieu, en qui habite la plénitude de la divinité. Voir Jn. 1:14, 34, 49 ; Mc. 15:39 ; Lu. 22:70 ; Col. 2:9 ; Ro. 1:4 et 1 Jn. : 4:9, 15.

\DicoEntry{FILS DE L'HOMME}\textit{}\newline
Expression désignant un être humain, elle fut attribuée au prophète Ezéchiel près de cent fois. A de nombreuses reprises, Jésus-Christ se nomma lui-même « Fils de l'homme » afin de souligner sa nature humaine. Voir Ez. 2:1 ; Ez. 3:10 ; Ez. 4:1 ; Mc. 14:62 ; Jn. 5:27 ; Ro. 8:3 et Ph. 2:5-7.

\DicoEntry{FIN DES TEMPS}\textit{, du grec « eschatos » : « extrême, dernier, fin » et « chronos » : « temps, date, siècles »}\newline
Appelée aussi derniers jours, période précédant la fin du monde*. Elle a commencé à l'effusion du Saint-Esprit selon la prophétie de Joël. La fin des temps est caractérisée d'un côté par des manifestations extraordinaires de l'Esprit de Dieu et l'annonce de l'Evangile à tous les peuples ; de l'autre par la séduction, l'apostasie* et le péché dans des dimensions jamais atteintes auparavant. Voir Jo. 2:28-29 ; Ac. 2:16-18 ; Mt. 24:3-14 ; 1 Ti. 4:1 et 2 Ti. 3:1-5.

\DicoEntry{FIN DU MONDE}\textit{, du grec « eschatos » : « extrême, dernier, fin » et « aion » : « monde, univers, période de temps »}\newline
Cet événement correspond à la fin de notre ère. Après le jugement dernier, les impies iront dans l'étang de feu*, tandis que la Nouvelle Jérusalem accueillera les saints ; la terre sera détruite. Voir Mt. 13:36-43 ; 2 Pi. 3:10-13 ; Ap. 20:11-15 et Ap. 21.

\DicoEntry{FOI}\textit{, du grec « pistis » : « conviction de la vérité »}\newline
Confiance en la véracité de Dieu, ses paroles et l'accomplissement de ses promesses. Bien qu'il n'existe qu'une seule foi, elle est présentée sous trois formes principales sous la Nouvelle Alliance :
\\1. en tant que fruit de l'esprit*, c'est la foi qui sauve (Ga. 5:22 et Ro. 10:9)
\\2. en tant que don de l'esprit,* c'est la foi accordée pour accomplir une tâche particulière (1 Co. 12:9)
\\3. en tant que parole, c'est la foi liée à la saine doctrine, la vérité (Ro. 10:17 et 2 Ti. 4:7)
\\Condition essentielle pour être agréable à Dieu et élément déclenchant les miracles, la foi est éprouvée tout au long de la vie du croyant. Voir Hé. 11 ; Lu. 7:50 et 1 Pi. 1:7.

\DicoEntry{FORNICATION ou IMPUDICITE}\textit{, du grec « poerneia » : « relation sexuelle illicite »}\newline
Tous les rapports sexuels condamnés par la Parole, voir Lé. 18 ; 1 Co. 6:13, 16-18 et 1 Co. 7:2.

\DicoEntry{FRUIT DE L'ESPRIT}\textit{}\newline
Résultat de l'action de l'Esprit Saint dans l'homme intérieur dans le but de communiquer le caractère de Yahweh au chrétien né d'en haut. Voir Ga. 5:22.

\DicoEntry{GABRIEL}\textit{, de l'hébreu « Gabriy'el » : « héros de Dieu » ou « homme de Dieu »}\newline
Archange* que Dieu envoya pour délivrer des messages, notamment à Daniel, Zacharie et Marie. Voir Da. 9:21-27 ; Lu. 1:11-20 et Lu. 1:26-38.

\DicoEntry{GAD}\textit{, de l'hébreu « Gad » : « bonheur », « heureux », « troupe »}\newline
Fils de Jacob et Zilpa, servante de Léa, il devint l'ancêtre de la tribu de Gad. Voir Ge. 30:11 et Ge. 49:16.

\DicoEntry{GALATIE}\textit{, du grec « Galatia » : « territoire des Gaulois, Gaule »}\newline
Province antique de l'Asie Mineure, la Galatie se situait en Anatolie, dans l'actuelle Turquie autour d'Ankara. Elle devait son nom aux Galates, Celtes provenant des Balkans. Lors de son premier voyage missionnaire, Paul avait traversé cette région où plusieurs assemblées émergèrent. Il y revint plus tard pour fortifier les disciples et leur écrivit une lettre suite au trouble apporté par les judaïsants. Voir Ac. 16:6 ; Ac. 18:23 et épître aux Galates.

\DicoEntry{GALILEE}\textit{, de l'hébreu « Galiyl » : « cercle, région, district »}\newline
Région située au nord de la Palestine dans laquelle se trouve la localité de Nazareth où Jésus grandit. Il y commença son ministère, c'est aussi là qu'il se montra vivant à ses disciples après sa résurrection. Les disciples de Jésus étaient originaires de Galilée. Voir Mt. 2:19-23 ; Jn. 2 ; Mc. 16:7 ; Ac. 1:11 et Ac. 2:7.

\DicoEntry{GARIZIM}\textit{, de l'hébreu « Geriziym » : « lieux arides »}\newline
Montagne située au sud de Sichem, en face du mont Ebal, de laquelle les enfants d'Israël devaient prononcer la bénédiction* une fois entrés en Canaan. Voir Jg. 9:7 ; De. 11:29 et Jos. 8:33.

\DicoEntry{GEDEON}\textit{, de l'hébreu « Gid'own » : « coupant, abattant »}\newline
Issu de la tribu de Manassé et fils de Joas. Il fut mandaté pour délivrer Israël de la main des Madianites et fut juge en Israël pendant quarante ans. Voir Jg. 6 à 8.

\DicoEntry{GEHENNE}\textit{, du grec « geena » : « vallée de Hinnom »}\newline
Initialement, vallée située au sud de Jérusalem où des enfants étaient jetés dans le feu en sacrifice à Moloc. Le terme « géhenne » représente la destruction future des méchants et se rapporte à l'étang de feu*. Voir 2 R. 23:10 et Mt. 10:28.

\DicoEntry{GENTILS}\textit{, du grec « ethnos » : « nations », « peuples »}\newline
Dans les Ecritures, ce terme se rapportait initialement à tous ceux n'appartenant pas au peuple juif. Paul fut mandaté pour évangéliser les gentils. A partir du IIIème siècle, le terme « païen » fut introduit dans le jargon chrétien pour désigner le « non-chrétien ». Voir Mt. 18:17 et Ac. 26:17.

\DicoEntry{GERME}\textit{, de l'hébreu « tsemach » : « pousse, croissance, branche »}\newline
Terme désignant le Messie dans certains écrits prophétiques. Voir Es. 4:2 ; Jé. 23:5 et Za. 3:8.

\DicoEntry{GLOIRE}\textit{, de l'hébreu « kabhod » : « poids » ou « kabowd » : « gloire, honneur, richesse »}\newline
La gloire se rapporte à ce qui a du poids, ce qui est lourd et écrasant – il est en effet difficile pour l'homme de supporter la splendeur et la magnificence de Yahweh. Image de sa sainteté, elle s'est manifestée dans un feu dévorant sur le mont Sinaï et fut révélée à Moïse au travers de la bonté et du nom de Dieu. Cette gloire sanctifie et génère de grands miracles ; elle est racontée par les cieux et toute la création. La gloire de Yahweh sera le luminaire de la Nouvelle Jérusalem. Elle invite à la crainte, la révérence, l'humilité, la louange ; lui seul mérite la gloire. Voir Ex. 16:10 ; Ex. 24:17 ; Ex. 29:43 ; Ex : 33:18-23 ; 2 Ch. 5 :14 ; Ps. 19:1 ; Pr. 15:33 ; Ez. 44:4 ; Es. 42:8 ; Es. 48:11; 1 Th. 2:12 et Ap. 21:23.

\DicoEntry{GOG}\textit{, de l'hébreu « Gowg » : « montagne »}\newline
Très certainement le chef du pays de Magog. Voir Ez. 38 et Ap. 20:8.

\DicoEntry{GOLGOTHA}\textit{, de l'araméen « gulgoleth » : « tête, crâne »}\newline
Lieu de la crucifixion de Jésus-Christ, situé non loin de Jérusalem. Voir Jn. 19:17-20.

\DicoEntry{GOMORRHE}\textit{, de l'hébreu « Amorah » : « submersion »}\newline
Ville située dans la plaine du Jourdain. Après avoir atteint un haut degré de perversion et de débauche, elle fut détruite par Yahweh avec sa ville voisine, Sodome. Voir Ge. 13:10 ; Ge. 18:20-21 et Ge. 19:24.

\DicoEntry{GRACE}\textit{, du grec « charis » : « bonne volonté », « bonté », « faveur »}\newline
Don immérité de Dieu, elle est la source du salut* de tous les hommes et invite à la crainte de Dieu. La grâce est venue par Jésus-Christ et fut révélée au travers de l'œuvre parfaite de la croix*. Voir Jn. 1:17 ; Tit. 2:11-12 et Ro. 3:23-24.

\DicoEntry{GRANDE TRIBULATION}\textit{}\newline
Voir commentaire Ap. 7:14.

\DicoEntry{GUILGAL}\textit{, de l'hébreu « Gilgal » : « action de rouler »}\newline
Territoire situé à l'ouest du Jourdain et à l'est de Jéricho; il fut le lieu de campement des Israélites après avoir passé le Jourdain à sec. Voir Jo. 4 et 5.

\DicoEntry{HABAKUK}\textit{, de l'hébreu « Chabaqquwq » : « embrasser », « amour »}\newline
Prophète de Yahweh qui exerça son ministère dans le royaume de Juda. L'ensemble de ses prophéties se trouve dans le livre éponyme.

\DicoEntry{HARMAGUEDON}\textit{, de l'hébreu « Armageddon » : « montagne de Méguiddo »}\newline
Lieu situé au nord d'Israël dans la tribu de Zabulon. A la fin des temps, les rois et puissants de la terre s'y rassembleront pour combattre Yahweh et son armée. Voir 2 R. 23:29 et Ap. 16:13-16.

\DicoEntry{HEBREU}\textit{, de l'hébreu « `Ibriy » : « qui provient de l'autre côté, qui traverse »}\newline
Terme désignant les descendants d'Héber, fils de Schélach, de la postérité de Sem, dont est issu Abraham. Voir Ge. 11:10-32 ; Ge. 14:13-14 et Ex. 1:15-22.

\DicoEntry{HELLENISTE}\textit{, du grec « hellenistes » : « celui qui adopte les manières et coutumes des Grecs »}\newline
Israélites nés hors de la terre promise ayant adopté le mode de vie grec et parlant la langue grecque. Voir Ac. 6:1.

\DicoEntry{HENOCH}\textit{, de l'hébreu « Chanowk » : « consacré, dédié »}\newline
Fils de Jéred et père de Metuschéla. Homme pieux ayant vécu trois cent soixante-cinq ans avant d'être enlevé au ciel sans connaître la mort. Voir Ge. 5:21-24 et Hé. 11:5.

\DicoEntry{HERODE LE GRAND}\textit{, (73 av. J.-C.-4 av. J.-C), du grec « Herodes »: « héroïque »}\newline
Roi de Judée, il fut l'instigateur du massacre des enfants de la région de Bethléhem au moment de la naissance de Jésus. Il mourut quand Jésus était encore enfant. Voir Mt. 2.

\DicoEntry{HERODE ANTIPAS}\textit{, (ou le Tétrarque) (\ 21 av. J.-C. -39 ap. J.-C.)}\newline
Fils d'Hérode le Grand*, il exerça la fonction de tétrarque* de Galilée et fut contemporain à Jésus-Christ pendant presque toute la vie de ce dernier. Hérode épousa sa belle-sœur Hérodias* et fit décapiter Jean-Baptiste. Il fut qualifié de « renard » par Jésus et s'accorda avec son ennemi Pilate lors de la crucifixion du Seigneur. Voir Lu. 3:1 ; Mc. 6:14-28 ; Lu. 13:31-32 et Lu. 23:8-12.

\DicoEntry{HERODE AGRIPPA Ier}\textit{, (\ 10 av. J.-C. - \ 44 ap. J.-C)}\newline
Roi et tétrarque de Judée et petit fils du roi Hérode le Grand, il accéda au pouvoir à la genèse de l'Eglise primitive. Pour plaire aux Juifs, il fit mourir Jacques, fils de Zébédée, et emprisonna Pierre. Il mourut brusquement après avoir reçu du peuple la gloire qui devait revenir à Dieu. Voir Ac. 12.

\DicoEntry{HERODE AGRIPPA II}\textit{, (\ 27 ap. J.-C. – \ 93 ap. J.-C.)}\newline
Fils d'Agrippa Ier, il est appelé « roi Agrippa » dans les Ecritures. Il fut inspecteur du temple de Jérusalem et avait le pouvoir de choisir les grands sacrificateurs. Il rencontra Paul à Césarée lors d'une visite au gouverneur Festus*. Voir Ac. 25 et 26.

\DicoEntry{HERODIAS}\textit{, du grec « Herodias » : « héroïque »}\newline
Femme de Philippe I puis de son frère, Hérode le tétrarque. Elle commanda la décapitation de Jean-Baptiste. Voir Mc. 6:17-28.

\DicoEntry{HOLOCAUSTE}\textit{, de l'hébreu « 'olah » : « offrande entièrement consumée »}\newline
Prescrit par la loi de Moïse, sacrifice consumé par le feu d'une agréable odeur à Yahweh. Il préfigurait le sacrifice à la croix de Jésus-Christ, l'Agneau de Dieu. Voir Lé. 1:1-17 ; Hé. 9:11-22 et Hé. 10:1-19.

\DicoEntry{HOMOSEXUALITE}\textit{}\newline
Pratique abominable et fermement réprouvée par Dieu consistant en l'union de deux personnes du même sexe. Voir Lé. 18 ; Ro. 1: 24-32 et 1 Co. 6:9-10.

\DicoEntry{HOSANNA}\textit{, de l'hébreu « yasha' » : « sauve » et « na´ » : « je te prie, maintenant » et du grec « hosanna » : « sauve maintenant ! »}\newline
Cri par lequel Jésus fut accueilli par la foule quand il entra à Jérusalem. Voir Mt. 21:9, 15 ; Mc. 11:9-10 et Jn. 12:13.

\DicoEntry{HULDA}\textit{, de l'hébreu « chuldah » : « belette, taupe »}\newline
Femme de Schallum, prophétesse habitant à Jérusalem du temps de Josias, roi de Juda. Le roi chercha à consulter Yahweh au travers d'elle quand il découvrit le livre de la loi et les malheurs qui devaient suivre la désobéissance d'Israël. Voir 2 R. 22:14-20 et 2 Ch. 34:21-33.

\DicoEntry{HYSOPE}\textit{, du grec « hussopos » : « hysope, branche d'hysope »}\newline
Plante aromatique utilisée pour faire l'aspersion du sang ou d'eau sous l'Ancienne Alliance. C'est à l'aide d'une branche d'hysope qu'on présenta à Jésus une éponge trempée de vinaigre lors de sa crucifixion. Voir Ex. 12:22 ; Lé. 14:1-7 ; Hé. 9:19 ; No. 19:18-19 et Jn. 19:29.

\DicoEntry{IDOLE, IDOLATRIE}\textit{, de l'hébreu « gilluwl » : « image » et du grec « eidolon » : « image pour adorer»}\newline
Une idole peut être l'image d'un faux dieu, l'image faussée de Yawheh ou encore une personne, un objet, une activité à qui l'on donne le rang de Dieu. L'idolâtrie - culte rendu à ces idoles – est fermement réprouvée dans la Parole. Voir 1 R. 15:11-13 ; Ex. 20:3-5 ; Ex. 32 ; 1 Co. 6:9 ; Ep. 5:5 et Col 3:5.

\DicoEntry{IMPOSITION DES MAINS}\textit{}\newline
Avant leur mort, les patriarches imposaient les mains à leurs enfants pour les bénir (Ge. 48:14). Moïse imposa également les mains à Josué qui devait lui succéder (De. 34:9). Sous la Nouvelle Alliance, on peut imposer les mains à quelqu'un en vue de lui transmettre la guérison divine, l'autorité liée à une fonction particulière, les dons spirituels et même le Saint-Esprit dans certains cas. Ce geste ne doit cependant pas être fait dans la précipitation. Voir Lu. 4:40 ; Mc. 16:18 ; 1 Ti. 4:14 ; Ac. 6:6 ; Ac. 8:17 et 1 Ti. 5:22.

\DicoEntry{INCORRUPTIBILITE}\textit{, du grec « aphtharsia » : « perpétuité, pureté, sincérité »}\newline
Terme désignant ce qui ne peut ni se corrompre, ni se flétrir, ni se détruire. A l'enlèvement de l'Église, les morts en Christ ressusciteront incorruptibles et les chrétiens revêtiront de même des corps incorruptibles. Voir Mt. 24:35 et 1 Co. 15:40-57.

\DicoEntry{INCREDULITE}\textit{, du grec « apistia » : « infidélité, sans foi, faiblesse dans la foi »}\newline
Rejet, doute par rapport à la véracité de Dieu et de sa parole. L'apôtre Thomas fit preuve d'incrédulité quant à la résurrection de Christ avant de le voir vivant. Les incrédules ne peuvent pas hériter le royaume de Dieu. Voir Jn. 1:1-14 ; Jn. 14:6 ; Jn. 20:24-29 et Ap. 21:8.

\DicoEntry{INIQUITE}\textit{, du grec « adikia » : « injustice, tortuosité d'un cœur, violation volontaire de la loi »}\newline
Tout ce qui constitue une violation de la loi* et la justice de Dieu. Voir Ro. 6:13 ; 2 Pi. 2:13 et 1 Jn. 5:17.

\DicoEntry{MYSTERE DE L'INIQUITE}\textit{}\newline
Voir commentaire 2 Th. 2:7.

\DicoEntry{INTERCESSION}\textit{, de l'hébreu « palal » : « intervenir, s'interposer, prier, agir en médiateur »}\newline
Sous l'Ancienne Alliance, le souverain sacrificateur avait la mission d'intercéder pour les péchés du peuple en offrant des sacrifices. A présent, Jésus-Christ le souverain sacrificateur à perpétuité et l'avocat intercède pour ses enfants après s'être offert en sacrifice pour les péchés de l'humanité. Les hommes peuvent aussi faire des prières d'intercession comme Abraham pour Lot, Moïse pour Marie et l'Eglise pour tous les hommes. Voir Lé 16 ; Hé : 9:11-15 ; 1 Jn. 2:1 ; Ge. 18:16-33 ; No. 12:10-15 et 1 Ti. 2:1.

\DicoEntry{ISAAC}\textit{, de l'hébreu « Yitschaq » : « il rit »}\newline
Fils de la promesse qui naquit à Abraham et Sara dans leur vieillesse. Il fut épargné quand Yahweh demanda à Abraham de lui offrir son fils en sacrifice. Isaac épousa Rébecca avec qui il eut deux fils : Esaü et Jacob. Voir Ge. 17:17-21 ; Ge. 22:1-13 et Ge. 25:19-26.

\DicoEntry{ISAI}\textit{, de l'hébreu « Yishay » : « je possède »}\newline
Bethléhémite, petit-fils de Boaz et de Ruth, fils d'Obed et père de David. Voir Ru. 4:13-22.

\DicoEntry{ISMAEL}\textit{, de l'hébreu « Yishma'e'l » : « Dieu entend »}\newline
Fils d'Abraham et d'Agar, servante de Sara. Béni par Yahweh même après avoir été chassé avec sa mère par Sara, il devint le père des douze tribus ismaélites. Voir Ge. 16 et Ge. 25:12-16.

\DicoEntry{ISSACAR}\textit{, de l'hébreu « Yissaskar » : « il donnera un salaire »}\newline
Fils de Jacob et Léa, il devint l'ancêtre de la tribu d'Isaacar. Voir Ge. 30:18 et Ge. 49:14.

\DicoEntry{ISRAEL}\textit{, de l'hébreu : « Yisra'el » : « Dieu prévaut »}\newline
Nom que Dieu donna à Jacob* après avoir lutté avec lui. Il s'agit également du nom désignant le peuple issu des douze fils de Jacob et le territoire que Dieu leur donna en héritage dont Jérusalem était la capitale. Après le schisme*, Israël se rapportait au royaume du nord composé de dix tribus. Voir Ge. 32:28 ; De. 33:5 et 1 R. 12:1-24.

\DicoEntry{IVRAIE}\textit{, du grec « zizanion » : « ivraie, ressemblant au blé, mais avec des grains noirs »}\newline
Comme le blé, l'ivraie est une plante de la famille des graminées, mais c'est une mauvaise semence qui étouffe le blé. Elle représente les fils du diable qui s'introduisent discrètement parmi les enfants de Dieu et qui en seront séparés uniquement à la fin du monde* pour aller vers la damnation éternelle. Voir Mt. 13:24-30, 36-42.

\DicoEntry{JACOB}\textit{, de l'hébreu « Ya`aqob » : « celui qui prend par le talon » ou « qui supplante »}\newline
Fils d'Isaac et de Rebecca et frère jumeau d'Esaü. Il usa de stratagèmes pour ravir le droit d'aînesse ainsi que la bénédiction qui devaient revenir à son frère Esaü. Après avoir fui ce dernier, il se réfugia chez son oncle Laban dont il épousa les deux filles : Léa et Rachel. De retour en Canaan après plusieurs années, Yahweh le rencontra en chemin et changea son nom en Israël. Jacob eut douze fils qui formèrent par la suite la nation d'Israël. Voir Ge. 25:21-34 ; Ge. 27 et 28 ; Ge. 29:1-30 et Ge. 49:1-28.

\DicoEntry{JACQUES}\textit{, de l'hébreu : « Iakob » : « qui supplante » (variante de Jacob)}\newline
1. Fils de Zébédée et frère de Jean. Un des douze apôtres. Le roi Hérode Agrippa Ier* le fit mourir par l'épée. Voir Mt. 4:21-22 ; Lu. 6:12-16 ; Mc. 9:2-8 ; Mc. 14:32-33 et Ac. 12:1-2.
\\2. Fils d'Alphée, un des douze apôtres ; il était aussi appelé Jacques le mineur. Voir Lu. 6:12-16 Mc. 15:40 et Mt. 10:1-4.
\\3. Frère du Seigneur et apôtre, auteur de l'épître de Jacques. Voir Ac. 15:13-21 ; Ga. 1:19 et Mc. 6:3.
\\4. Père de l'apôtre Jude. Voir Lu. 6:16 et Ac. 1:13.

\DicoEntry{JAPHET}\textit{, de l'hébreu « Yepheth » : « ouvert », « qui s'étend »}\newline
Dernier des trois fils de Noé. Voir Ge. 10:1.

\DicoEntry{JEAN}\textit{, de l'hébreu « Yowchanan » et du grec « Ioannes » : « Yahweh a fait grace »}\newline
1. Fils de Zébédée, frère de Jacques et disciple aimé du Seigneur. Jean fut l'auteur de l'évangile éponyme, des trois épîtres qui portent son nom et de l'Apocalypse. Voir Mt. 10:2 et Jn. 13:23.
\\2. Fils de Zacharie et Elisabeth, cousin de Jésus. Plus connu sous le nom de Jean-Baptiste, il fut envoyé pour préparer le chemin du Seigneur. Il fut décapité par Hérode Antipas*. Voir Lu. 1 ; Mal. 3:1-6 ; Mt. 1:12 ; Lu. 7:28 et Mt. 14:1-12.

\DicoEntry{JELEK}\textit{, de l'hébreu « yekeq » : « jeune sauterelle »}\newline
Désignant les sauterelles, il est souvent employé dans les Ecritures pour symboliser un grand nombre ou le dévoreur que Dieu envoie. Voir Joë. 1:4 et Na. 3:15-16.

\DicoEntry{JEREMIE}\textit{, de l'hébreu « Yirmeyah » : « celui que Yahweh a désigné »}\newline
Fils de Hilkija, issu d'une famille de sacrificateurs. Prophète de Yahweh, Jérémie fut appelé dès son plus jeune âge et exerça un ministère prophétique avant et pendant les premières années de déportation. Appelé à être eunuque*, il ne se maria jamais et n'eut point d'enfant. Il fut l'auteur des livres Jérémie et Lamentations de Jérémie.

\DicoEntry{JERICHO}\textit{, de l'hébreu « Yeriychow » : « ville de la lune » ou « ville des palmiers »}\newline
Ville située à l'est de la tribu de Benjamin, près des rives du Jourdain. A la sortie du désert, les espions hébreux y furent cachés par Rahab la prostituée ; Jéricho fut ensuite détruite et livrée miraculeusement entre les mains d'Israël. C'est à Jéricho que Jésus guérit l'aveugle Bartimée et fut reçu par Zachée. Voir Jos. 2 et 6 ; Mc. 10:46-53 et Lu. 19:1-10.

\DicoEntry{JEROBOAM}\textit{, de l'hébreu « Yarob'am » : « le peuple devient nombreux »}\newline
Fils de Nebath et serviteur de Salomon, il devint plus tard son ennemi. Après le schisme, il fut le premier roi du royaume du nord sur lequel il régna vingt-deux ans. Il fut une occasion de chute pour le peuple qu'il plongea dans l'idolâtrie*. Voir 1 R. 11:26-40 ; 1 R. 12 et 13.

\DicoEntry{JERUSALEM}\textit{, de l'hébreu « Yeruwshalem » : « fondement de la paix »}\newline
Ville située en Palestine, au nord de la Judée. Lors de la conquête de Canaan, la ville fut sous le contrôle des Jébusiens. Aux environs du Xe siècle av. J.-C., David reprit la ville alors devenue forteresse jébusienne. Il en fit la capitale politique et religieuse du royaume en y faisant établir l'arche de l'alliance. Salomon y construisit le temple* sur le mont Morija. En 586 av. J.-C., bien après le schisme*, les Babyloniens la détruisirent. Elle fut rebâtie par Néhémie après le retour de la captivité babylonienne. Jésus-Christ se lamenta sur la ville à cause de son incrédulité* et y annonça sa future destruction. Jérusalem fut en effet détruite par le général romain Titus en 70 ap. J.-C puis de nouveau rebâtie. Lors de son retour glorieux, le Seigneur Jésus posera ses pieds sur le mont des Oliviers* qui est situé à Jérusalem. Le livre d'Apocalypse annonce après la fin du monde l'apparition de la Nouvelle Jérusalem, cité céleste. Voir 2 S. 5:6-9 ; 2 S. 6 ; 2 Ch. 3:1 ; Za. 14:1-4 ; Lu. 19:41-44 ; et Ap. 21:2.

\DicoEntry{JESUS}\textit{, de l'hébreu « Yehowshuwa » : « Yahweh est salut »}\newline
Fils de l'homme* et fils de Dieu*, Jésus est le Dieu vivant manifesté en chair. Il fut conçu dans le ventre de Marie par la puissance du Saint-Esprit alors que cette dernière n'avait point connu d'homme. Selon les recherches de l'historien Flavius Josèphe, sa date de naissance se situerait autour de l'an 6 av. J.-C. - l'an zéro n'étant qu'une indication approximative. Fils adoptif de Joseph le charpentier et cousin de Jean Baptiste, il vécut la plus grande partie de sa vie en Galilée, dans la ville de Nazareth. Vers l'âge de 30 ans, il se fit baptiser dans le Jourdain et commença par la suite son ministère public. Grâce à sa vie exemplaire sans péché, il put se présenter comme une offrande agréable à Dieu répondant aux exigences de la justice divine pour sauver le monde. Remplissant toutes les prophéties relatives au Messie*, il fut trahi par un de ses disciples, Judas Iscariot*. Arrêté, maltraité puis crucifié, il mourut portant le poids des péchés de l'humanité, mais il ressuscita le troisième jour. Le salut réside dans la foi en son nom. Vivant de toute éternité, Jésus est le Dieu véritable et la vie éternelle. Sa justice ne tardera pas à se manifester : il revient à toute vitesse. Voir Mc. 2:28 ; Jn. 1:34 ; 1 Ti. 3:16 ; Mt. 1:18-25 ; Lu. 1:36 ; Mt. 2:23 ; Lu. 6:16 ; 1 Pi. 2:21-25 ; 2 Co. 5:21 ; Es. 53 ; Ac. 4:12 ; Lu. 24:46 ; 1 Jn. 5:20 et Ap. 22:20.

\DicoEntry{JETHRO}\textit{, de l'hébreu « Yithrow » : « son abondance, excellence »}\newline
Sacrificateur de Madian chez qui Moise se réfugia après avoir fui l'Egypte. Il donna sa fille Séphora* pour femme à Moïse*. Voir Ex. 2:15-21.

\DicoEntry{JEUNE}\textit{, de l'hébreu « tsuwn » : « s'abstenir de nourriture » et du grec « nesteia » : « le jeûne, un exercice volontaire et religieux »}\newline
Privation totale ou partielle de nourriture dans le but d'humilier sa chair et d'adresser à Dieu des prières spécifiques. Le jeûne doit être exempt de toute hypocrisie et accompagné d'actes de justice pour être agréé de Dieu. Voir Da. 10 ; Est. 4:16 ; Es. 58 ; Lu. 2:37 et Mt. 6:16-18.

\DicoEntry{JEZABEL}\textit{, de l'hébreu « 'Iyzebel » : « Baal est l'époux » ou « l'impudique »}\newline
Fille d'Ethbaal, roi de Sidon, et femme d'Achab roi d'Israël, elle extermina les prophètes de Yahweh et accueillait huit cent cinquante faux prophètes à sa table. Elle conduisit le peuple d'Israël dans l'idolâtrie au temps d'Elie*. Jézabel est associée à l'esprit du même nom qui prolifère de faux enseignements et entraîne le peuple de Dieu dans l'impudicité. Voir 1 R. 16:31, 1 R. 18:4, 19 et Ap. 2:20.

\DicoEntry{JOB}\textit{, de l'hébreu « 'Iyowb » : « haï, ennemi » ou « Je m'exclamerai »}\newline
Originaire du pays d'Uts, homme prospère dont Yahweh témoigna l'intégrité et la droiture. Il subit en très peu de temps une succession de malheurs que Dieu permit pour se révéler à lui. Son histoire est racontée dans le livre portant son nom.

\DicoEntry{JOEL}\textit{, de l'hébreu « Yow'el » : « Yahweh est Dieu »}\newline
Fils de Pethuel, il exerça la fonction de prophète dans le royaume de Juda. Il annonça la venue du Saint-Esprit sur toute chair à la fin des temps. Le contenu de son message se trouve dans le livre éponyme.

\DicoEntry{JONAS}\textit{, de l'hébreu « Yonah » : « colombe »}\newline
Prophète de Yahweh envoyé à Ninive pour leur annoncer la destruction de leur ville. Son refus d'obéir à Dieu le conduisit dans le ventre d'un grand poisson. Son histoire est racontée dans le livre portant son nom.

\DicoEntry{JONATHAN}\textit{, de l'hébreu « Yehownathan » : « Yahweh a donné »}\newline
Fils du roi Saül, homme de guerre reconnu. Lié à David par une très forte amitié, il protégea plusieurs fois ce dernier des relents meurtriers de son père. Il mourut à la bataille de Guilboa avec son père et ses frères. Voir 1 S. 14:1-15, 1 S. 18:1-4 ; 1 S. 19:1-8 ; 1 S. 20 et 1 S. 31:1-2.

\DicoEntry{JOSAPHAT}\textit{, de l'hébreu « Yehowshaphat » : « Yahweh a jugé »}\newline
Fils d'Asa et d'Azuba, il fut roi de Juda pendant vingt-cinq ans. Il eut un règne prospère et fit ce qui est droit aux yeux de Yahweh. Voir 1 R. 15:24 ; 1 R. 22:41-46 et 2 Ch. 17.

\DicoEntry{JOSEPH}\textit{, de l'hébreu « Yowceph » : « que Yahweh ajoute » ou « il enlève »}\newline
1. Fils de Jacob et Rachel. Vendu comme esclave par ses frères, il devint, après plusieurs années de prison, gouverneur d'Egypte. Ses fils, Ephraïm et Manasée, furent adoptés par son père Jacob et furent les pères de deux des douze tribus d'Israël. Voir Ge. 30:22-24 ; Ge. 37, 39, 40, 45 et 46 ; Ge. 48:5 et Jos. 14:4.
\\2. Fils d'Héli, charpentier originaire de la tribu de Juda. Epoux de Marie, la mère de Jésus. Voir Mt. 1:18-25 et Mt. 13:55.

\DicoEntry{JOSIAS}\textit{, de l'hébreu « Yo'shiyah » : « Yahweh guérit »}\newline
Fils d'Amon, il devint roi de Juda à huit ans et y régna durant trente et un ans. Grand réformateur, à l'origine d'un grand réveil spirituel, il répara le temple, purifia le royaume des idoles et conclut une alliance de fidélité envers Yahweh. Voir 2 R. 22 et 23.

\DicoEntry{JOSUE}\textit{, de l'hébreu « Yehowshuwa`» : « Yahweh est salut »}\newline
Fils de Nun de la tribu d'Ephraïm, choisi par Dieu pour succéder à Moïse. Accompagné de la puissante main de Dieu, il conduisit Israël à entrer en possession de Canaan. Son histoire se trouve dans le livre portant son nom.

\DicoEntry{JOURDAIN}\textit{, de l'hébreu « Yarden » : « celui qui descend »}\newline
Très certainement le fleuve le plus connu des Ecritures, il est situé aux limites est de l'actuel territoire d'Israël. Josué et le peuple d'Israël passèrent le fleuve à sec. De même, Elie, puis Elisée, partagèrent les eaux du fleuve en deux. Après s'y être baigné sept fois sur les conseils d'Elisée, Naaman fut guéri de la lèpre. Jésus se fit baptiser par Jean dans le Jourdain. Voir Jos. 3 ; 2 R. 2:8, 12-14 ; 2 R. 5:10-14 et Mt. 3:13-17.

\DicoEntry{JOUR DU SEIGNEUR}\textit{}\newline
Jour où Yahweh manifestera sa justice et frappera les nations à cause de leurs péchés. Ce jour arrivera comme un voleur et surprendra beaucoup. Voir Es. 13:6-16 ; So. 1 et 2 P. 3:10.

\DicoEntry{JUDA}\textit{, de l'hébreu « Yehuwdah » : « qu'il (Dieu) soit loué »}\newline
Fils de Jacob et Léa, il est le père de la tribu du même nom installée au sud de Canaan. Sa descendance reçut la prédominance et la royauté ; David et Jésus-Christ étaient issus de cette tribu. Après le schisme*, Juda désigna aussi le nom du royaume du sud composé des tribus de Juda et Benjamin. Voir Ge. 29:35 ; Ge. 49:8-12 ; Jos. 15:1-12 ; Mt 1:1-16 et 1 R. 12:16-24.

\DicoEntry{JUDAS ISCARIOT}\textit{, de l'hébreu « Yehuwdah » : « qu'il (Dieu) soit loué »}\newline
Fils de Simon Iscariot, il fut un des douze disciples de Jésus-Christ et était chargé de la trésorerie. Il trahit le Seigneur, ce qu'il regretta amèrement et le poussa à se suicider. Voir Lu. 6:16 ; Jn. 12:4-6 ; Mt. 26:14-16 et Mt. 27:3-5.

\DicoEntry{JUDE}\textit{, de l'hébreu « Yehuwdah » : « qu'il (Dieu) soit loué »}\newline
1. Fils de Jacques, un des douze apôtres, connu également sous le nom de Thadée. Voir Lu. 6:16 et Mc. 3:18.
\\2. Prophète également appelé Barsabas, compagnon d'œuvre de Silas. Voir Ac. 15:22, 32.
\\3. Frère du Seigneur, auteur d'une épître qui porte son nom. Voir Mt. 13:55 ; Mc. 6:3 et Jud. 1:1.

\DicoEntry{JUDEE}\textit{, de l'hébreu « Yehuwdah » : « qu'il (Dieu) soit loué »}\newline
Région située au sud de la Palestine où se trouvent notamment Jérusalem et Bethléhem. Elle correspondrait approximativement au territoire de l'ancien royaume de Juda. Ce terme n'est pas utilisé dans le Tanakh. Voir Mt 2:1 ; Mc. 1:5 et Ga. 1:22.

\DicoEntry{JUGE, JUGEMENT}\textit{, de l'hébreu « shaphat » : « juger, gouverner, défendre, punir », « agir comme un législateur, juge ou gouverneur », « exécuter un jugement »}\newline
Dans toute la Parole, Yahweh est présenté comme le juge droit et incorruptible. Après la sortie d'Egypte, des juges ont été suscités par Dieu au milieu d'Israël pour délivrer le peuple de ses ennemis et le ramener vers lui (voir livre des Juges). Le Seigneur a toujours envoyé des prophètes pour annoncer ses jugements et ses décisions ainsi que des juges pour faire respecter sa loi. Sous la Nouvelle Alliance, l'homme spirituel est appelé à juger (discerner selon la Parole), mais condamner et décider du sort final d'une personne demeure la prérogative de Dieu. Yahweh est en effet le juste juge qui siège et tranche non seulement au tribunal de Christ, mais également au jugement dernier. Voir Ge. 18:25 ; Jé. 11:20 ; 2 Co. 5:10 et Ap. 20:11-15.

\DicoEntry{JUPITER}\textit{, du grec « Zeus » : « un père des secours »}\newline
Divinité romaine assimilée à Zeus chez les Grecs. Lors d'une guérison miraculeuse à Lystres, la foule pensa voir en Paul la réincarnation de Mercure et en Barnabas celle de Jupiter. Pour cela, on voulut les adorer, ce qu'ils refusèrent avec véhémence. Voir Ac.14:8-15.

\DicoEntry{JUSTE, JUSTICE}\textit{, de l'hébreu « tsedeq » : « droiture, exactitude, conforme » ou encore « tsadiq » : « juste, exact, innocent » et du grec « dikaiosune » : « la condition acceptable par Dieu » ou « intégrité, vertu, pureté de vie, droiture »}\newline
En qualité de juste juge, Yahweh a toujours recherché cette qualité chez l'homme, mais il ne l'a pas trouvé déclarant que nul n'est juste. Au travers de l'œuvre de la croix et avec l'aide du Saint-Esprit, le chrétien peut à présent marcher dans la justice de Dieu. Il est appelé à la rechercher plus que tout et à devenir esclave de la justice. Voir Mt. 5-7 ; Lu. 1:75 ; 2 Ti. 4 :8 ; Ro. 3:10 et Ro. 6:18.

\DicoEntry{JUSTIFICATION}\textit{, du grec « dikaiosis » : « état du juste »}\newline
Au travers de l'œuvre de la croix, Jésus-Christ est devenu la justification de tous ceux qui croient en lui, les rendant acceptables et libres de toute culpabilité. Voir Ro. 3:23-28 ; Ro. 4:25 et Ro. 5:18.

\DicoEntry{KORE}\textit{, de l'hébreu « Qorach » : « chauve »}\newline
Fils de Jitsehar, originaire de la tribu de Lévi, il se révolta avec Dathan* et Abiram* contre Moïse* et Aaron*. Suite à sa rébellion, il périt avec les gens de sa maison. Voir No. 16:1-35.

\DicoEntry{LAIC}\textit{, du grec « laos » : « peuple »}\newline
Notion propre à l'Eglise catholique romaine. Opposé au clergé*, les laïcs sont les autres membres de l'église, ceux qui n'ont pas de fonction dirigeante, mais qui sont tout de même appelés à honorer Dieu dans leur vie et faire connaître leur foi au milieu du monde.

\DicoEntry{LANGUES}\textit{, de l'hébreu « lashown » : « langue, langage » et du grec « glossa » : « la langue » ou « le langage d'un peuple particulier »}\newline
Selon les Ecritures, les langues sont nées à Babylone* lorsque les hommes se sont rebellés contre la volonté de Yahweh et que ce dernier a confondu leur langage dans le but de les disperser. Dans la Parole sont cités différents types de langues, chacune liée à un don ou une manifestation particulière de l'Esprit de Dieu. Lors de l'effusion du Saint-Esprit à la Pentecôte, les disciples reçurent la capacité de parler des merveilles de Dieu dans des langues étrangères. Il s'agit du don spirituel* appelé la diversité des langues et concerne uniquement les langues usuelles. Il existe également des langues angéliques ou dites inconnues que le croyant peut utiliser pour s'adresser à Dieu. Les langues étrangères tout comme les langues des anges peuvent donner lieu à une interprétation, c'est ce qu'on appelle le don d'interpréter les langues. Voir Ge. 11 ; Ac. 2:1-11 ; 1 Co. 12:10 ; 1 Co. 13:1 et 1 Co. 14:1-14, 26-27.

\DicoEntry{LAODICEE}\textit{, du grec « Laodikeia » : « justice du peuple »}\newline
Capitale de la Phrygie, l'une des provinces de l'Asie Mineure, réputée dans le domaine du commerce, notamment dans l'industrie textile. Ses vêtements et sa tapisserie principalement de couleur noire, firent sa renommée. Elle possédait une grande école de médecine qui fabriquait des remèdes réputés pour les yeux, notamment le fameux collyre. L'église de Laodicée est la dernière à qui fut adressée une lettre dans l'Apocalypse. Caractérisée par la tiédeur, l'affection aux choses terrestres et l'aveuglement spirituel, le Seigneur l'appela à la repentance*. Elle est l'image de l'église matérialiste. Voir Ap. 3:14-22.

\DicoEntry{LAZARE}\textit{, du grec « Lazaros » : « Yahweh a secouru »}\newline
1. Homme pauvre qui fut recueilli dans le sein d'Abraham après sa mort. Voir Lu. 16:19-21.
\\2. Frère de Marthe et de Marie de Béthanie, et ami de Jésus-Christ qui le ressuscita des morts. Voir Jn. 11.

\DicoEntry{LEA}\textit{, de l'hébreu « Le'ah » : « lasse »}\newline
Fille aînée de Laban et première femme de Jacob. Elle enfanta six fils, pères de six des douze tribus d'Israël (Ruben, Siméon, Lévi, Juda, Issacar et Zabulon) ainsi qu'une fille nommée Dina. Voir Ge 29:16-23 ; Ge. 30:21 et Ge 35:23.

\DicoEntry{LEMEC}\textit{, de l'hébreu « Lemek » : « puissant »}\newline
Fils de Metuschaël et descendant de Caïn, il fut le premier polygame de l'histoire en prenant deux femmes : Ada et Tsilla. Voir Ge. 4:16-24.

\DicoEntry{LEPRE}\textit{, de l'hébreu « tsara' » : « être morbide de peau »}\newline
Commune en Egypte et en orient, maladie de la peau dont le virus peut se développer dans tout le corps. Contagieuse, elle peut même souiller les vêtements et les habitations. Sous la loi mosaïque, les personnes atteintes de cette maladie étaient considérées comme impures et devaient se tenir à l'écart. Durant son ministère, Jésus guérit plusieurs lépreux. Voir Lé. 13 et 14 et Lu. 17:11-14.

\DicoEntry{LEVAIN}\textit{, de l'hébreu « chametz » : « ce qui est levé »}\newline
Symbole du mal et de la corruption, le levain était interdit dans la quasi-totalité des offrandes. Jésus a assimilé le levain des pharisiens à l'hypocrisie, à la doctrine erronée. Les chrétiens sont appelés à faire disparaître le vieux levain et à devenir le levain du monde en y faisant progresser l'évangile du royaume. Voir Lé. 2:11 ; Mt. 16:6-12 ; Mt. 13:33 et 1 Co. 5:6-7.

\DicoEntry{LEVI, LEVITES}\textit{, de l'hébreu « Leviy » : « attachement »}\newline
Fils de Jacob et Léa, Lévi participa avec son frère Siméon au massacre des hommes de la ville de Sichem après le viol de leur sœur Dina. Consacrés au service de Yahweh, ses descendants, les Lévites, n'eurent point d'héritage en Canaan, mais habitèrent différentes villes qui leur furent spécifiquement attribuées en Israël. Voir Ge. 29:34 ; Ge. 34 ; Jos. 13:14 et No. 18:20-24.

\DicoEntry{LOI}\textit{, de l'hébreu « towrah » : « loi, direction, commandement », « loi mosaïque »}\newline
L'ensemble des préceptes et ordonnances relatifs à l'alliance conclue entre Yahweh et le peuple hébreu, par l'intermédiaire de Moïse, est contenu dans les cinq premiers livres de la Bible appelée aussi « le Pentateuque ». Selon la tradition juive, il existerait 613 commandements relatifs à la moralité, la vie en société et le culte rendu à Yahweh. L'homme en étant incapable, Jésus-Christ a accompli les exigences de la loi. Il est donc possible aux hommes d'obtenir le salut par la foi et non plus par les œuvres. La loi est maintenant gravée dans les cœurs des enfants de Dieu à qui le Saint-Esprit rappelle les paroles de Jésus. Voir Ex. 18:20 ; Ex. 24:12 ; Ro. 3:19-31 et Jn. 14:26.

\DicoEntry{LOI DU PECHE}\textit{, du grec « nomos » : « toute chose établie, une coutume, un commandement »}\newline
Loi spirituelle inscrite dans la chair qui pousse l'homme charnel à se révolter contre Dieu en commettant le péché. Voir Ro. 7:13-25.

\DicoEntry{LOT}\textit{, de l'hébreu : « Lowt » : « voile, couverture »}\newline
Fils de Haran et neveu d'Abraham, Lot quitta Ur avec ce dernier avant de s'en séparer. Grâce à l'intercession d'Abraham, il fut sauvé de la destruction de Sodome avec ses deux filles. Ces dernières enivrèrent leur père et eurent des relations incestueuses avec lui de qui naquirent Moab, père des Moabites, et Amon, père des Ammonites. Voir Ge. 11:31 ; Ge. 13:1-13 ; et Ge. 19.

\DicoEntry{LUC}\textit{, du grec « Loukas » : « qui donne la lumière »}\newline
Disciple de Jésus d'origine païenne, médecin de métier, il fut un des compagnons d'œuvre de Paul et l'auteur de l'évangile qui porte son nom et du livre Actes des Apôtres. Voir Col. 4:14 et Phm. 1:24.

\DicoEntry{MACEDOINE}\textit{, du grec « Makedonia » : « terre étendue »}\newline
Province romaine située au nord de la Grèce. Paul y effectua quelques voyages missionnaires et y implanta plusieurs assemblées. Voir Ac. 16:9-12 ; Ac. 20:1-3 ; Ro. 15:23 ; 1 Co. 8:1 et 2 Co. 11:9.

\DicoEntry{MADIAN}\textit{, de l'hébreu « Midyan » : « lutte, dispute »}\newline
Un des fils issu de l'union d'Abraham* et Ketura, il devint l'ancêtre des Madianites, peuple qui habita à l'est de Canaan et au nord du désert d'Arabie. Voir Ge. 25:1-2 ; No. 31:1-12 et Jg. 6:2.

\DicoEntry{MAGOG}\textit{, de l'hébreu « Magowg » : « territoire de montagne, qui domine »}\newline
Fils de Japhet. Associé à Gog, il correspond aussi à la nation d'où vient le roi Gog qui fera la guerre à Dieu et à son peuple juste avant le jugement dernier. Voir Ge. 10:2, 9 et Ap. 20:8.

\DicoEntry{MAIN}\textit{, de l'hébreu « yad » : « main, force, pouvoir »}\newline
Partie du corps permettant de toucher, saisir ou posséder, elle représente aussi l'action, la provision, la protection ou le joug. Tout au long des Ecritures, la main de Yahweh révèle sa puissance et sa bienveillance. Voir Pr. 10:4 ; Ps. 71:4 ; Es. 40:2 ; Jé. 18:6 ; Mc. 14:58 ; Lu. 11:20 et Ac. 11:21.

\DicoEntry{MALACHIE}\textit{, de l'hébreu « Mal`akiy » : « mon messager »}\newline
Dernier prophète du Tanakh, il condamna les péchés et l'hypocrisie des enfants d'Israël et annonça la venue de Jean-Baptiste. L'ensemble de ses prophéties est contenu dans le livre portant son nom.

\DicoEntry{MALEDICTION}\textit{, de l'hébreu « arar, meerah, qelalah » et du grec « ara, katara »}\newline
Parole attirant le malheur sur un bien, une personne ou un peuple. Dieu a le pouvoir de maudire et aussi d'écarter toute malédiction. La malédiction de Dieu, contraire de la bénédiction*, fait suite à la désobéissance. A la nouvelle naissance, toutes les chaînes de malédiction qui liaient le chrétien sont brisées. Le chrétien ne doit pas maudire, mais bénir en tout temps, même ses ennemis. Voir De. 28:15-68 ; Mt. 5:44 ; Ro. 8:1 et 2 Co. 5:17.

\DicoEntry{MALFAITEUR REPENTANT}\textit{}\newline
Un des hommes coupables qui fut crucifié à côté de Jésus. Son humilité, sa sincérité et sa repentance lui permirent d'accéder au salut, Jésus-Christ lui ayant garanti l'accès au paradis. Voir Lu. 23:33-43.

\DicoEntry{MAMON}\textit{, du grec « Mammonas » : « richesses »}\newline
Dieu de l'argent. Jésus utilisa ce terme pour personnifier la richesse que beaucoup idolâtrent et qui est par conséquent en concurrence avec Yahweh dans le cœur de certains. Voir Mt. 6:24.

\DicoEntry{MANASSE}\textit{, de l'hébreu « Menashsheh » : « oublieux »}\newline
1. Fils aîné de Joseph* et d'Asnath, adopté par Jacob avant sa mort, ancêtre de la tribu de Manassé. Voir Ge. 41:51 ; Ge. 48:5 et Jos. 14:4.
\\2. Fils d'Ezéchias et de Hephtsiba, il fut l'un des pires rois du royaume de Juda qui régna 55 ans. Malgré le réveil impulsé par son père, il se détourna entièrement de Yahweh et servit des dieux étrangers. Voir 2 R. 21:1-18.

\DicoEntry{MANNE}\textit{, de l'hébreu « man » : « qu'est-ce que cela ? »}\newline
Nourriture céleste - à l'aspect de la graine de coriandre et au goût de gâteau de miel - que Dieu donna quotidiennement aux Israélites durant toute leur marche dans le désert. Ex. 16:15, 31-35.

\DicoEntry{MARANATHA}\textit{, de l'araméen « maran atha » : « le Seigneur vient » ou « Seigneur, viens »}\newline
Expression prononcée par Paul quand il s'adressa aux Corinthiens et qui doit également être le cri du cœur de tout enfant de Dieu. Voir 1 Co. 16:22 et Ap. 22:17, 20.

\DicoEntry{MARC}\textit{, du grec « Markos » : « une défense » ou « grand marteau »}\newline
Appelé aussi Jean, cousin de Barnabas, il fut la cause de la séparation de Paul et Barnabas. Il partit avec ce dernier à Chypre et devint par la suite un fidèle compagnon d'œuvre de Paul. Il écrivit l'évangile portant son nom. Voir Ac. 12:12 ; Ac. 15:36-39 ; Col. 4:10 et Phm. 24.

\DicoEntry{MARDOCHEE}\textit{, de l'hébreu « Mordekay » : « petit homme »}\newline
Fils de Jaïr de la tribu de Benjamin*, il adopta Esther*, fille de son oncle. Il sauva la vie du roi Assuérus* en déjouant les plans de Bigthan et Théresch et préserva le peuple juif des desseins meurtriers d'Haman. Il devint puissant dans la maison du roi et instaura la fête du Purim. Voir le livre d'Esther.

\DicoEntry{MARIAGE}\textit{, de l'hébreu « chathan » : « devenir un gendre, s'allier »}\newline
Bénédiction de Dieu, le mariage est une alliance en principe indissoluble entre un homme et une femme dans le but d'accomplir le plan de Dieu. Il doit être célébré dans le respect des autorités du pays dans lequel le couple se trouve et honoré de tous, particulièrement des parents dont la bénédiction est essentielle. Voir Ge. 2:22-24 ; Ge. 24:60 ; Pr. 18:22 ; Hé. 13:4 et 1 Co. 7.

\DicoEntry{MARIE}\textit{, de l'hébreu « Miryam » : « rébellion, obstination »}\newline
1. Sœur de Moïse et d'Aaron, prophètesse. Elle se rebella contre Moïse et fut frappée par la lèpre, mais en guérit grâce à l'intercession de Moïse. Voir Ex. 15:20 et No. 12.
\\2. Mère de Jésus : Elle conçut, par la vertu du Saint-Esprit, Jésus homme. Elle devint une de ses disciples et se trouvait parmi ceux qui persévéraient dans la prière dans la chambre haute lors de l'effusion du St Esprit promis. Voir Es. 7:14 ; Lu. 1:26-38 ; Mt. 1:18-25 ; Mc. 15:40-41 et Ac. 1:13-14.
\\3. Marie de Magdala : Elle fut délivrée de sept démons par Jésus qu'elle suivit pendant son ministère terrestre, et ce jusqu'à la croix. Elle fut mandatée par le Seigneur pour annoncer sa résurrection aux apôtres. Voir Lu. 8:2 ; Mt. 27:55-56 ; Mc. 16:1-11 et Jn. 20:1-18.
\\4. Marie de Béthanie : Sœur de Marthe* et de Lazare*, que Jésus ressuscita des morts. Contrairement à sa sœur, elle choisit la bonne part en restant aux pieds du Maître. Elle toucha le cœur de ce dernier en l'oignant d'un parfum de grand prix. Voir Jn 11:1-44 ; Lu. 10:38-42 et Jn. 12:1-7.

\DicoEntry{MARTHE}\textit{, du grec « Martha » : « maîtresse, dame »}\newline
Sœur de Lazare* - dont elle fut témoin de la résurrection - et de Marie* de Béthanie, elle reçut Christ dans sa maison, mais ce dernier lui reprocha son activisme au détriment de l'écoute de sa Parole. Voir Jn. 11:1-44 et Lu. 10:38-42.

\DicoEntry{MATTHIAS}\textit{, de l'hébreu « Mattithyah » : « don de Yahweh »}\newline
Disciple de Jésus et témoin oculaire de son ministère, il fut désigné pour devenir l'un des douze apôtres en remplacement de Judas Iscariot* qui avait trahi le Seigneur pour ensuite se suicider. Voir Ac. 1:15-26.

\DicoEntry{MATTHIEU}\textit{, du grec « Matthaios » : « don de Yahweh »}\newline
Collecteur d'impôts, il fut l'un des douze apôtres de Jésus et l'auteur de l'évangile qui porte son nom. Voir Mt 9:9 et Mt. 10:3.

\DicoEntry{MEDIATEUR}\textit{, du grec « mesites » : « celui qui intervient entre deux parties », « intermédiaire de communication »}\newline
Moïse a exercé cette fonction auprès du peuple d'Israël qui avait expressément demandé que Dieu ne leur parle pas directement. Christ, garant d'une Nouvelle Alliance, est à présent l'unique intermédiaire et médiateur entre Dieu et les hommes. Voir Ex. 20:19 ; Hé. 8:6 ; Hé. 9:15 et 1 Ti. 2:5.

\DicoEntry{MELCHISEDEK}\textit{, de l'hébreu « Malkiy-Tsedeq » : « roi de justice »}\newline
Roi de Salem et sacrificateur du Dieu Très-Haut, il était une apparition de Jésus-Christ avant son apparition. Il bénit Abraham après sa victoire contre Kedorlaomer. Jésus-Christ est souverain sacrificateur selon l'ordre de Melchisédek. Voir Ge. 14:14-20 ; Hé. 5:5-10 et Hé. 6:20.

\DicoEntry{MENSONGE}\textit{, de l'hébreu « sheqer » : « mensonge, déception, fausseté, tromperie, fraude » et du grec « pseudos » : « fausseté consciente et intentionnelle »}\newline
Modification de la vérité. Satan est appelé père du mensonge et les menteurs auront droit à la même sentence que lui. Voir Ex. 20:16 ; Jn. 8:44 et Ap. 21:8.

\DicoEntry{MESOPOTAMIE}\textit{, de l'hébreu « 'Aram Naharayim » : « pays entre deux fleuves »}\newline
Située entre le Tigre et l'Euphrate, région correspondant à l'actuel Irak. Avant son appel, Abraham vivait à Ur en Chaldée qui se trouvait au sud de la Mésopotamie. Voir Ge. 11:31.

\DicoEntry{MESSIE}\textit{, de l'hébreu « mashiyach » : « oint, celui qui est l'oint »}\newline
Voir CHRIST.

\DicoEntry{MICHEE}\textit{, de l'hébreu « Miykayehuw » : « qui est comme Dieu ? »}\newline
Originaire de Moréscheth, Michée exerça la fonction de prophète dans le royaume du sud au temps d'Ezéchias, roi de Juda. L'ensemble de ses prophéties se trouve dans le livre éponyme.

\DicoEntry{MICHEL ou MICAEL}\textit{, de l'hébreu « Miyka'el » : « qui est semblable à Dieu ? »}\newline
Archange* de Dieu, il est un des principaux chefs des anges*. Souvent présent dans les grandes batailles, il lutta notamment contre le roi de Perse et contre le diable. Voir Da. 10:13-21 ; Jud. 1:9 et Ap. 12:7.

\DicoEntry{MILLE}\textit{, du grec « million » : « distance de mille pas »}\newline
Unité de mesure romaine correspondant à 1480m environ. Voir Mt. 5:41.

\DicoEntry{MILLENIUM}\textit{}\newline
Période de paix de mille ans durant laquelle le Seigneur régnera sur la terre. Voir Es. 11, 12 et Ap. 20:2-7.

\DicoEntry{MINISTERE}\textit{, du grec « diakonia » : « service », dérivé du mot grec « diakonos » : « domestique »}\newline
Tâche que le chrétien exerce au service de Dieu et des hommes selon l'onction* et le mandat que Dieu lui donne. Le ministre de Dieu est donc un serviteur inutile, un simple instrument utilisé pour la gloire de Yahweh. Voir Lu. 17:10 ; 2 Co. 3:5 ; Ro. 12 ; 1 Co. 12 et 1 Pi. 4:10-11.

\DicoEntry{MISERICORDE}\textit{, de l'hébreu « checed » : « bonté, miséricorde, fidélité » et du grec « eleos » : « bonne volonté envers le misérable associée à un désir de l'aider »}\newline
Comme en témoigne le plan du salut* qu'il a déployé, Dieu est riche en miséricorde. Le disciple de Christ doit comme son maître se revêtir d'entrailles de miséricorde afin de représenter le royaume de Dieu. Ge. 24 :7 ; Nb. 24:18 ; Mt. 9:13 ; Lu. 1:78 ; Ro. 11:31 et 2 Jn. 1:3.

\DicoEntry{MOAB}\textit{, de l'hébreu « Mow'ab » : « issu d'un père »}\newline
Fils de Lot*, né de sa relation incestueuse avec sa fille aînée, il donna naissance au peuple des Moabites. Ils s'établirent au sud-est de la mer morte et s'opposèrent plusieurs fois aux enfants d'Israël. Voir Ge. 19:37 ; Jg. 3:12 ; 2 S. 8:2 ; Ez. 25:8-11.

\DicoEntry{MODALISME}\textit{}\newline
Doctrine enseignée à Rome au début du troisième siècle par Sabellius selon laquelle le Père, le Fils et le Saint-Esprit sont différents aspects au travers desquels Dieu se révèle et non trois personnes distinctes. Réfutant ainsi la doctrine de la trinité* largement acceptée par les catholiques, Sabellius fut condamné par le pape Callixte à cause de son enseignement pourtant biblique. Voir 1 Th. 3:11 ; 2 Th. 2:16-17 et 1 Jn. 5:20.

\DicoEntry{MOISE}\textit{, de l'hébreu « Mosheh » : « tiré de »}\newline
Issu de la tribu de Lévi, il fut miraculeusement sauvé du massacre des enfants de sa génération pendant la servitude d'Israël en Egypte. Il vécut les quarante premières années de sa vie dans la maison de Pharaon puis les quarante suivantes dans le désert auprès de Madian*. A l'issue de cette deuxième période, Yahweh se révéla à lui et le mandata pour délivrer le peuple d'Israël de la captivité égyptienne afin de le faire entrer dans la terre promise. Après l'avoir fait sortir au milieu des miracles et des prodiges, Moïse conduisit le peuple dans le désert pendant quarante années au cours desquelles il leur communiqua l'intégralité de la loi*. Il mourut à la porte de la terre promise à l'âge de cent vingt ans. On lui attribue l'écriture des cinq premiers livres du Tanakh. Voir Ex. 1 et 2 ; Ex. 12:40-41 ; Ex. 14:21-31 ; Ex. 24:12 ; De. 8:2 ; Ac. 7:20-43 ; Hé. 11:23-29 et De. 34:5-7.

\DicoEntry{MOISSON}\textit{, de l'hébreu « qatsiyr » : « moisson, travail de la moisson, récolte »}\newline
Sous la loi, la fête des prémices avait lieu lors de la moisson. Jésus utilise ce terme pour parler du champ missionnaire, les personnes à qui l'évangile doit être annoncé. Dans le cadre de la fin du monde, la moisson se rapporte au jugement de Dieu qui va apporter la séparation entre ses fils et les fils du diable. Voir Lé. 23:10-14 ; Mt. 9:37-38 ; Mt. 13:33-43.

\DicoEntry{MOLOC}\textit{, de l'hébreu « Molek » : « roi, conseiller »}\newline
Divinité vénérée par les Ammonites à qui il était coutume de sacrifier des enfants brûlés vifs. Les Israélites se prostituèrent plusieurs fois à Moloc. Voir 1 R. 11:5-7 et 2 R. 23:10.

\DicoEntry{MONT DES OLIVIERS}\textit{}\newline
Colline située à l'est de Jérusalem près de la vallée du Cédron. C'est du Mont des Oliviers que Jésus fut enlevé au ciel après avoir donné ses dernières recommandations aux apôtres ; c'est à ce même endroit qu'il posera les pieds lors de son glorieux retour. Voir Ac. 1:4-12 ; Za. 14:1-4.

\DicoEntry{MORT}\textit{, de l'hébreu « muwth » : « mourir, tuer, être exécuté » et du grec « thanatos » : « mort du corps »}\newline
La Bible distingue deux morts. La première entra dans le monde suite à la désobéissance de l'homme et correspond à la séparation d'avec Dieu et à la mort physique. La deuxième mort concerne uniquement ceux dont le nom n'est pas écrit dans le livre de vie et correspond à la souffrance éternelle dans l'étang de feu. Voir Ge. 3 ; Ro. 5:12 ; Ro. 6:23 et Ap. 20:11-15.

\DicoEntry{MYRRHE}\textit{, de l'hébreu « more » : « myrrhe »}\newline
Résine provenant de certains arbres d'Asie et d'Afrique, réputée pour son arôme de grand prix. Elle était utilisée sous forme d'huile pour l'onction sainte et pouvait atténuer les douleurs quand elle était mélangée au vin. Les mages offrirent de la myrrhe à Jésus lors de sa naissance. Voir Ex. 30:22-30 ; Mt. 2:11 ; Mt. 27:34 et Mc. 15:23.

\DicoEntry{NAHUM}\textit{, de l'hébreu « Nachuwm » : « consolation, qui a compassion »}\newline
Prophète de Yahweh né à Elkosch, il annonça la destruction de Ninive. L'ensemble de ses prophéties se trouve dans le livre portant son nom.

\DicoEntry{NAISSANCE D'EN HAUT}\textit{, du grec « anothen » : « depuis le haut, depuis un endroit plus élevé »}\newline
Naissance d'eau et d'esprit symbolisant respectivement la Parole qui purifie et le Saint-Esprit* qui est le gage de l'appartenance à Dieu. La naissance d'en haut est l'œuvre du Saint-Esprit qui délivre une personne du royaume des ténèbres et la transporte dans le royaume de Dieu. L'homme charnel devient alors spirituel, le cœur de pierre est ôté pour accueillir un cœur de chair, le citoyen terrestre se transforme en citoyen céleste et le vieil homme laisse place à une nouvelle créature. Voir Jn. 3:1-8 ; Ja. 1:18 ; 1 Jn. 3:9 ; Ez. 36:25-27 ; Ep. 2:6 ; Ep. 5:26 ; 1 Co. 12:13 et 2 Co. 5:17.

\DicoEntry{NATHAN}\textit{, de l'hébreu « Nathan » : « il (Yahweh) a donné »}\newline
Prophète de Yahweh au temps du roi David. Il prophétisa le règne éternel de la postérité de David et la construction du temple par son fils. Il reprit David lorsque ce dernier fit assassiner Urie* pour prendre sa femme. Voir 2 S. 7 et 12.

\DicoEntry{NAZAREEN, NAZIREEN}\textit{, de l'hébreu « naziyr » : « consacré ou voué »}\newline
Terme pouvant désigner soit un habitant de la ville de Nazareth, soit une personne qui s'est consacrée à Yahweh dans le cadre d'un voeu de naziréat. Voir No. 6.

\DicoEntry{NAZARETH}\textit{, du grec « Nazareth » : « verdoyant, germe, rejeton »}\newline
Ville située dans la région de Galilée où Jésus passa la majeure partie de sa vie. Voir Mt. 2:22-23.

\DicoEntry{NEBUCADNETSAR}\textit{, (règne : 605 av. J.-C. – 562 av. J.-C.), « Nebuwkadne'tstsar » : « que Nebo protège la couronne, les frontières » (origine inconnue)}\newline
Roi de Babylone, il mit fin au royaume de Juda en emmenant le peuple en captivité ; il détruisit le temple de Jérusalem. Il reçut l'interprétation de plusieurs songes au travers de Daniel* et reconnut le règne dominant et éternel de Yahweh. Voir 2 R. 25, Da. 1:1 et Da. 2, 4.

\DicoEntry{NEHEMIE}\textit{, de l'hébreu « Nechemyah » : « Yahweh a consolé »}\newline
Fils d'Hacalia, il fut échanson du roi Artaxerxès* à Suse, pendant la captivité de Juda. Il entreprit la réparation des murailles de Jérusalem et initia une réforme en son temps. Il devint ensuite gouverneur de Juda. Son histoire est racontée dans le livre éponyme.

\DicoEntry{NEPHILIM}\textit{, de l'hébreu « nephiyl » : « géant », racine : « naphal » : « tomber, chuter »}\newline
Etres de grande taille nés de l'union des fils de Dieu et des filles des hommes avant le déluge. On en retrouve aussi en Canaan lorsque les douze espions hébreux étaient allés observer la terre promise. Voir Ge. 6:4 et No. 13:32-33.

\DicoEntry{NEPHTHALI}\textit{, de l'hébreu « Naphtaliy » : « lutte, mon combat »}\newline
Fils de Jacob* et de Bilha, servante de Rachel*, il est l'ancêtre de la tribu de Nephthali. Voir Ge. 30:8 et Ge. 49:21.

\DicoEntry{NICODEME}\textit{, du grec « Nikodemos » : « victorieux du peuple »}\newline
Docteur de la loi, il s'approcha de Jésus de nuit, qui l'enseigna sur la naissance d'en haut. Après la crucifixion, il aida Joseph d'Arimathée pour embaumer le corps du Seigneur et pour le mettre dans un sépulcre. Voir Jn. 3:1-21 et Jn. 19:38-42.

\DicoEntry{NICOLAITES}\textit{, du grec « Nikolaites » : « destruction du peuple »}\newline
Secte suivant la doctrine de Nicolas, liée à la doctrine de Balaam, qui poussait à la consommation de viandes sacrifiées aux idoles et à l'impudicité. Voir Ap. 2:6, 14-15.

\DicoEntry{NIL}\textit{, de l'hébreu « Shiychowr » : « sombre, noir, boueux »}\newline
Principal fleuve d'Égypte situé à l'est du pays. Voir Jé. 2:18 et Es. 23:3.

\DicoEntry{NIMROD}\textit{, de l'hébreu « Nimrowd » : « rebelle »}\newline
Fils de Cush et descendant de Noé. Chasseur, il fut le premier homme puissant sur la terre et régna sur plusieurs villes dont Babel*. Voir Ge. 10:8-11 et Ge. 11:1-9.

\DicoEntry{NINIVE}\textit{, de l'hébreu « Niyneveh » : « habitation de Ninus »}\newline
Grande ville située sur la rive est du Tigre. Ses habitants se repentirent de leurs mauvaises voies suite à la prédication de Jonas, mais ils retombèrent dans le péché quelques années plus tard. Ninive fut finalement détruite sous le jugement de Dieu. Voir livres de Jonas et de Nahum.

\DicoEntry{NOCES}\textit{, du grec « gamos » : « fête du mariage »}\newline
Festivités célébrant le mariage. Dans la tradition juive, les noces duraient sept jours même si une longue période pouvait parfois s'écouler entre la conclusion du mariage (accord des familles) et la consommation du mariage (nuit de noces). Ainsi, la fiancée devait se tenir prête pour les noces à tout moment. De même, l'Eglise se prépare à être enlevée par Jésus à tout moment pour les noces de l'Agneau qui seront célébrées au ciel pendant sept ans. Voir Ge. 29:27 ; Jg. 14:12 ; 1 Th. 4:16-17 et Ap. 19:7.

\DicoEntry{NOE}\textit{, de l'hébreu « Noach » : « repos, tranquillité »}\newline
Fils de Lamech, il fut le père de trois fils : Sem, Cham et Japhet. Qualifié d'homme juste et intègre en son temps, il trouva grâce devant Yahweh qui lui ordonna de construire une arche* pour le sauver lui, sa famille et une partie des animaux de la terre du déluge qui arrivait. Son obéissance sauva la race humaine. Il vécut 950 ans. Voir Ge. 6 à 9.

\DicoEntry{NOUVELLE NAISSANCE}\textit{}\newline
Voir NAISSANCE D'EN HAUT.

\DicoEntry{OFFRANDE}\textit{, de l'hébreu « minchah » : « don, tribut, présent, oblation, sacrifice »}\newline
Sous la loi, le peuple d'Israël avait reçu des prescriptions relatives aux offrandes agréables à Yahweh ; elles consistaient essentiellement en bétail et produits naturels et étaient offertes dans le cadre de cérémonies spécifiques. Des offrandes en argent pouvaient aussi être données, notamment pour soutenir l'entretien du temple. Sous la Nouvelle Alliance, les offrandes monétaires doivent être libres et volontaires ; l'offrande la plus importante aux yeux de Dieu reste la vie consacrée de ses enfants. Voir Lé. 1 à 7 ; Mc. 12:41-42 ; 2 Co. 8:10-12 ; 2 Co. 9:7 ; Ro. 12:1 et Ro. 15:15-16.

\DicoEntry{OLIVIER}\textit{}\newline
Arbre fruitier donnant des olives avec lesquelles on produit de l'huile. Sous la loi de Moïse, elle était notamment utilisée pour alimenter les lampes qui devaient brûler continuellement dans le temple et pour oindre les personnes désignées par Dieu pour une tâche spécifique. L'olivier symbolise en outre le témoignage et la paix. Voir Jg. 9:8-9 ; Ex. 27:20-21 ; Ex. 30:22-25 ; 1 S. 16:3 et 1 R. 19:16.

\DicoEntry{OMEGA}\textit{}\newline
Dernière lettre de l'alphabet grec désignant aussi la fin d'une chose (voir ALPHA et OMEGA).

\DicoEntry{ONCTION}\textit{, de l'hébreu « mishchah » : « portion consacrée, huile d'onction, oindre » et du grec « chrisma » : « toute chose qui sert à enduire » de la racine « chrio » : « oindre, imprégner les chrétiens des dons du Saint-Esprit »}\newline
Sous l'Ancienne Alliance, l'onction était souvent accordée par l'action de verser de l'huile sur la tête de la personne ou de l'objet à consacrer. On oignait ainsi les sacrificateurs, les rois et les prophètes selon leur mandat. Sous la Nouvelle Alliance, l'onction demeure en celui qui a reçu en lui le Seigneur Jésus. Toutefois, l'onction d'huile peut être pratiquée dans le cadre de la prière pour les malades. Voir Ex. 30:22-31 ; 1 S. 16:3 et 1 R. 19:16 ; Ac. 1:8 ; 1 Jn. 2:20-27 et Ja. 5:14.

\DicoEntry{ORDINATION}\textit{, du latin « ordinatio » : « action de disposer, de mettre en œuvre »}\newline
Rite initiatique mis en place par l'Eglise catholique qu'on ne retrouve pas dans les Ecritures. Elle confère, par l'imposition des mains accompagnée d'une prière, la capacité d'exercer une fonction dirigeante au sein de l'église locale.

\DicoEntry{OTHNIEL}\textit{, de l'hébreu « `Othniy'el » : « Dieu est puissant »}\newline
Fils de Kenaz et frère cadet de Caleb, il fut le premier juge en Israël, fonction qu'il exerça pendant 40 ans. Il délivra les enfants d'Israël du joug du roi de Mésopotamie, Cuschan-Rischeathaïm. Voir Jg. 3:8-11.

\DicoEntry{OSEE}\textit{, de l'hébreu « Howshea` » : « salut, sauve »}\newline
Fils de Beéri, prophète qui, sous les ordres de Yahweh, épousa une prostituée pour illustrer l'infidélité des enfants d'Israël envers leur Dieu. L'histoire d'Osée et l'ensemble de ses prophéties se trouvent dans le livre portant son nom.

\DicoEntry{PAIEN}\textit{, du latin « paganus » qui signifie « paysan », qui provient lui-même du mot « pagus » qui signifie « campagne ».}\newline
Personne qui pratiquait une des religions polythéistes de l'Antiquité.

\DicoEntry{PAIX}\textit{, de l'hébreu « shalowm » : « état complet, perfection, bien-être, paix » et du grec « eirene » : « état de tranquillité, paix entre les individus, harmonie, sécurité »}\newline
Sous l'Ancienne Alliance, la paix était matérialisée par la prospérité, l'absence de guerre et de toutes sortes de malheurs. Sous la Nouvelle Alliance, la paix est un fruit de l'Esprit*, une promesse acquise en Jésus qui est lui-même le Prince de Paix. Différente de celle que le monde offre, la paix de Christ permet de rester confiant en toutes circonstances. Voir Lé. 26:6 ; Es. 26:12 ; Ga. 5:22 ; Jn. 14:27 et Jn. 16 :33.

\DicoEntry{PALMIER}\textit{, de l'hébreu « tamar » : « palmier, dattier »}\newline
Arbre à tronc peu ou pas ramifié, on le retrouve essentiellement dans le désert. L'image du palmier fut utilisée en décoration dans le temple. Ses branches étaient utilisées pendant la fête des tentes. Symbole de la justice et de la victoire, on le retrouve lors de l'entrée royale de Jésus à Jérusalem et devant le trône de Dieu. Voir Lé. 23:40 ; 1 R. 6:29 ; Jn. 12:12-13 et Ap. 7:9.

\DicoEntry{PAQUE}\textit{, de l'hébreu « pecach » : « passer outre, épargner », « sacrifice de la Pâque » ou « fête de la Pâque »}\newline
Première fête du calendrier hébraïque, elle fut instituée par ordonnance perpétuelle dès la sortie d'Egypte. Cette fête commémore le salut de Yahweh accordé par le sacrifice de l'agneau ; elle préfigurait Christ, l'Agneau de Dieu qui est « notre Pâque ». Voir Ex. 12 ; Lé. 23:5 ; Jn. 1:29 et 1 Co. 5:7-8.

\DicoEntry{PARADIS}\textit{, du grec « paradeisos » : « jardin »}\newline
Lieu de repos et de félicité, le paradis fut ouvert par Jésus lors de sa résurrection. Il y emmena les justes décédés qui étaient jusque-là captifs dans le séjour des morts*. Les chrétiens rejoignent ce lieu céleste à leur décès, en attendant la résurrection*. A la croix, Christ garantit l'accès à ce lieu au malfaiteur repentant. Paul fut ravi à cet endroit où il entendit des paroles merveilleuses. Voir Hé. 10:19-20 ; Ep. 4:8-10 ; Lu. 23:43 et 2 Co. 12:2-4.

\DicoEntry{PARDON}\textit{, de l'hébreu « nas a´ » : « action de lever, supporter, prendre » et du grec « aphesis » : « libérer de l'esclavage » ou « oubli des péchés, rémission des peines »}\newline
Sous l'Ancienne Alliance, le pardon était conditionné par les sacrifices d'animaux, mais Jésus-Christ a accompli cette prérogative en devenant la victime expiatoire pour nos péchés. En lui, l'homme repentant est pardonné de ses fautes et trouve également la force de pardonner à ceux qui l'offensent. Voir Lé. 4 à 6 ; Jn. 1:29 ; Mt. 6:12 + 14-15 ; Ac. 10:43 et 1 Jn. 1:9.

\DicoEntry{PARVIS}\textit{, de l'hébreu « chatser » : « cour, enclos, colonie, ville, village » (voir illustration du temple)}\newline
Première des trois parties du tabernacle et du temple ; il s'agissait d'une cour dans laquelle se trouvait l'autel d'airain où se faisaient des sacrifices et la cuve d'airain contenant de l'eau pour la purification. Voir Ex. 27:9-19.

\DicoEntry{PASTEUR}\textit{, de l'hébreu « ra`ah » : « berger »}\newline
Un des cinq ministères d'Ep. 4 :11 travaillant en collège, établi pour veiller pour le troupeau, le nourrir de la Parole et encourager les chrétiens à exercer pleinement et librement leur ministère. Toutefois, Jésus-Christ demeure le pasteur par excellence, le bon berger qui donne sa vie pour ses brebis et le gardien des âmes qui ne sommeille ni ne dort. Voir Ep. 4:11 ; Jn. 10:11-16 ; Ps. 23 et 1 Pi. 2:25.

\DicoEntry{PATMOS}\textit{, du grec « Patmos » : « mortel, fascinant »}\newline
Petite île grecque de la mer Egée sur laquelle Jean fut exilé à la fin de sa vie. Il y reçut la révélation de l'Apocalypse. Voir Ap. 1:9.

\DicoEntry{PAUL}\textit{, du grec « Paulos » : « petit »}\newline
Issu de la tribu de Benjamin et né dans la ville de Tarse, son nom était initialement Saul*. Pharisien, son zèle excessif le poussa à persécuter violemment les chrétiens à la naissance de l'Eglise. Il rencontra Christ sur la route de Damas et devint par la suite l'apôtre des gentils annonçant l'Evangile de villes en villes et de pays en pays au cours de nombreux voyages. Même en prison, il continua l'œuvre de Dieu en écrivant plusieurs lettres riches en enseignements que l'on peut retrouver dans le canon biblique. Voir Ac. 9 à 28 et les épîtres de Paul.

\DicoEntry{PEAGER ou PUBLICAIN}\textit{, du grec « telones » : « un loueur, un collecteur de taxes »}\newline
Les péagers d'origine juive étaient dépréciés de leurs compatriotes et assimilés à des pécheurs, car on les considérait comme des collaborateurs au service des romains. De plus, certains profitaient de leur fonction pour s'enrichir. Voir Mt. 9:10 ; Mt. 21:31 ; Lu. 3:12-13 et Lu. 19:2-8.

\DicoEntry{PECHE}\textit{, de l'hébreu « chatta'ah » : « ce qui manque le but » et du grec « hamartano » : « erreur, faux état d'esprit »}\newline
Le péché entra dans le monde par la transgression d'Adam et Eve et tous les hommes en furent infectés. Origine de la séparation entre Dieu et les hommes, le péché conduit à la mort*. Voir Ge. 3 ; Ro. 5:12 ; Ro. 6:23 ; Ro. 8:1-4 ; 1 Co. 15:3 et 1 Pi. 2:21-24.

\DicoEntry{PENTECOTE}\textit{, du grec « pentekoste » : « le cinquantième jour »}\newline
Fête annuelle juive célébrant la moisson des blés. La venue du Saint-Esprit* promis par Jésus eut lieu pendant la célébration de la Pentecôte. Voir Lé. 23:15-22 ; Jn. 16:7-11 ; Ac. 1:5 et Ac. 2:1-21.

\DicoEntry{PHARAON}\textit{, de l'hébreu « Par`oh » : « grand palais »}\newline
Titre donné aux rois égyptiens durant l'antiquité. Voir Ge. 37:36 et Ge. 41.

\DicoEntry{PHARISIEN}\textit{, du grec « Pharisaios » : « séparé »}\newline
Secte juive dont les membres manifestaient un attachement excessif aux coutumes et traditions religieuses. Certains d'entre eux combattirent Jésus qui dénonça ouvertement leur fausse piété et leur dévouement hypocrite envers Dieu. Désirant la mort du Seigneur, ils participèrent à la conspiration qui précéda sa crucifixion. Voir Mc. 7:1-13 ; Mt. 23:23-39 et Jn. 18:2-3.

\DicoEntry{PHILADELPHIE}\textit{, du grec « Philadelpheia » : « amour fraternel »}\newline
Ville de Lydie en Asie Mineure. Irriguée par le fleuve Hermus, Philadelphie était une contrée très fertile, propice à l'agriculture et surtout à la culture de la vigne. Elle fut construite par le roi de Pergame, et plusieurs fois sujette à des tremblements de terre. Une des sept lettres d'Apocalypse s'adressait à l'église de Philadelphie. Cette dernière - contrairement aux autres qui cumulèrent des reproches – fut très encouragée par le Seigneur. Bien que située à 45 km de Sardes à laquelle elle était rattachée, l'Eglise de Philadelphie resta ferme en retenant la Parole de Dieu et ne se laissa pas influencer par les séductions du péché. Elle incarne ainsi l'Eglise que Jésus revient chercher, l'Eglise réveillée.

\DicoEntry{PHILEMON}\textit{, du grec « Philemon » : « attentionné, qui embrasse »}\newline
Disciple de Colosses qui recevait une église dans sa maison. Il avait un esclave nommé Onésime au sujet duquel Paul lui écrivit une lettre. Voir épître de Paul à Philémon.

\DicoEntry{PHILIPPE}\textit{, du grec « Philippos » : « aimant les chevaux »}\newline
1. Homme de Bethsaïda, il fut l'un des douze apôtres* choisis par Jésus. Voir Mt. 10:3 ; Mc. 3:18 et Lu. 6:14.
\\2. Un des sept diacres élus au sein de l'église de Jérusalem. Evangéliste*, il prêcha le Christ dans la ville de Samarie, à l'eunuque éthiopien qu'il baptisa et dans différentes villes. Voir Ac. 6:5 ; Ac. 8:4-8, 26-40 et Ac. 21:8.

\DicoEntry{PHILIPPES}\textit{, du grec « Philippoi » : « appartenant à Philippe »}\newline
Fondée par Philippe II (382 av. J.-C. – 336 av. J.-C.) en 356 av. J.-C., ville grecque de Macédoine orientale. Située sur une voie romaine qui traversait les Balkans, elle est restée de taille modeste en dépit de son fort taux de fréquentation. Une église y naquit après la rencontre de Paul avec des femmes qui priaient à l'extérieur de la ville. L'apôtre leur écrivit une lettre qui figure dans le canon biblique. Voir Ac. 16:9 et l'épître aux Philippiens.

\DicoEntry{PHILISTINS}\textit{, de l'hébreu : « Pelesheth » : « immigrants »}\newline
Peuple qui habitait à l'extrême ouest de Canaan, le long de la mer Méditerranée. Ils furent plusieurs fois en conflit avec les Israélites ; Goliath était philistin. Voir Jg. 13 à 16 et 1 S. 17.

\DicoEntry{PHILOSOPHIE}\textit{, du grec « philosophia » : « amour de la sagesse »}\newline
Discipline existant depuis l'antiquité et ayant plusieurs courants de pensée en son sein comme les épicuriens* et les stoïciens*. Elle pousse ses adeptes à rechercher la sagesse par l'intelligence humaine. Paul invita les chrétiens à se garder de ces doctrines. Voir Ac. 17:16-20 et Col. 2:8.

\DicoEntry{PHINEES}\textit{, de l'hébreu « Piynechac » : « bouche de cuivre »}\newline
Fils d'Eléazar, petit-fils d'Aaron et sacrificateur. Il se démarqua par son zèle pour Dieu pour arrêter un fléau sur Israël. A cette occasion, Yahweh fit alliance perpétuelle avec Phinées et sa descendance. Voir No. 25.

\DicoEntry{PIERRE}\textit{, de l'hébreu « Cephas » et du grec « Petros » : « un roc ou une pierre »}\newline
Fils de Jonas et frère d'André*, son nom était initialement Simon*. Pêcheur de métier originaire de la ville de Bethsaïda, il fut choisi comme apôtre pour les circoncis. Il écrivit deux épîtres portant son nom. Il aurait été crucifié à Rome. Voir Jn. 1:42-44 ; Mt. 10:2 ; Ga. 2:7-8 et les deux épîtres de Pierre.

\DicoEntry{PILATE}\textit{, du grec « Pilatos » : « armé d'une lance »}\newline
Gouverneur romain de la Judée en fonction pendant le ministère de Jésus. Il s'accorda avec son ennemi Hérode lorsqu'il fallut crucifier le Seigneur. N'ayant pas trouvé de crime en Jésus, il permit finalement sa crucifixion et fit mettre l'inscription suivante sur sa croix : Jésus de Nazareth, roi des Juifs. Voir Lu. 3:1 ; Lu. 23:11 et Jn. 19:1-19.

\DicoEntry{PREDESTINATION}\textit{, du grec « proginosko » : « avoir la connaissance avant »}\newline
Révélant l'omniscience de Dieu qui connaît toutes choses à l'avance, la prédestination concerne l'œuvre de la croix prévue de toute éternité – l'Agneau ayant été immolé avant la fondation du monde. La prédestination est non pas la décision de Dieu d'envoyer certaines personnes en enfer, mais plutôt la capacité de Yahweh à connaître à l'avance ceux qui allaient devenir ses enfants d'adoption, transformés à l'image du Fils, en acceptant sa parole. Voir 1 Pi. 1:19-20 ; Ep. 1:5 ; Jn. 1:12 et Ro. 8:29-30.

\DicoEntry{PRETOIRE}\textit{, du grec « praitorion » : « quartier général dans un camp romain, la tente du commandant en chef »}\newline
Dans les évangiles, lieu de résidence des gouverneurs dans lequel se trouvaient notamment un tribunal et une prison. Voir Mt. 27:27 ; Jn. 18:28-29 et Ac. 23:35.

\DicoEntry{PRIERE}\textit{, de l'hébreu « palal » : « intervenir, s'interposer, prier » ou « 'athar » : « prier, supplier, implorer » et du grec « proseuche » : « prière adressée à Dieu » ou « parakaleo » : « appeler à, convoquer, supplier, exhorter »}\newline
Acte par lequel on s'approche de Dieu et on instaure un dialogue avec lui, en ayant foi dans sa présence et son action. Invité à prier constamment, le chrétien peut le faire pour se repentir, intercéder en faveur d'une situation particulière, demander quelque chose à Dieu, le remercier, le louer ou tout simplement lui exprimer son amour. La prière garde les enfants de Dieu dans la paix. Dieu connaissant toutes les pensées de l'homme, le plus important dans la prière reste l'écoute de la voix de Yahweh. Voir Ge. 20:17 ; 1 S. 2:1 ; Job 22:27 ; Mt. 14:36 ; Ac. 16:9 ; Ph. 4:6-7 ; 1 Th. 5:17 et 1 Pi. 4:7.

\DicoEntry{PROPHETE}\textit{, de l'hébreu « nabiy' » : « l'homme qui parle, celui qui est appelé, qui a reçu une inspiration » et du grec « prophetes » : « celui qui interprète des oracles », « quelqu'un qui déclare ce qu'il a reçu par inspiration »}\newline
Sous l'Ancienne Alliance, Dieu suscita de nombreux prophètes oints de l'Esprit afin qu'ils annoncent des messages particuliers et conduisent le peuple à l'obéissance et à la crainte de Yahweh. Sous la Nouvelle Alliance, il existe au moins trois types de prophètes. Le premier concerne ceux et celles qui prophétisent au sein des assemblées locales (Ac. 21 :8-9 ; 1 Co. 14 :29-32), ils exhortent, édifient et consolent le peuple (1 Co. 14 :1-3). Le deuxième concerne les personnes qui ont reçu la charge d'enseigner, poser les fondements, implanter des assemblées selon Ep. 4 :11. Parmi ces prophètes, on compte Barnabas, Siméon, Lucius de Cyrène, Manahen, Saul (Ac. 13 :1-5), Jude et Silas (Ac. 15 :32-33). Le troisième concerne tous les chrétiens qui sont des potentiels prophètes puisqu'ils ont l'Esprit de Christ en eux (1 Co. 14 :23-25 ; 1 Co. 14 :31). Dieu peut se servir n'importe quel chrétien pour prophétiser, c'est-à-dire communiquer une parole inspirée. Voir Ep. 2:20 et Ep. 4:11.

\DicoEntry{PROPHETIE}\textit{, du grec « propheteia » : « discours émanant de l'inspiration divine et déclarant les desseins de Dieu »}\newline
Depuis l'effusion du Saint-Esprit, tous les chrétiens nés d'en haut peuvent prophétiser sans pour autant avoir le ministère de prophète. La prophétie est en effet un don spirituel* auquel il faut aspirer et qui est attribué par le Saint-Esprit selon la volonté de Dieu. Voir Ac. 2:16-18 ; 1 Co. 12:4-10 et 1 Co. 14:1.

\DicoEntry{PROPITIATOIRE}\textit{, de l'hébreu « kapporeth » : « siège de miséricorde, lieu d'expiation »}\newline
Couvercle de l'arche* composé d'or pur, il était surmonté de deux chérubins* d'or se faisant face au milieu desquels Yahweh siégeait et se manifestait pour donner des instructions à Israël. Une fois par an, le souverain sacrificateur entrait dans le Saint des saints et aspergeait le propitiatoire du sang des animaux sacrifiés pour la purification des péchés d'Israël. Voir Ex. 25:17-22 et Lé. 16.

\DicoEntry{PROSELYTE}\textit{, du grec « proselutos » : « un nouveau venu, un étranger »}\newline
Personne issue d'une nation païenne s'étant agrégée au peuple d'Israël par le rite de la circoncision* et la pratique de la loi mosaïque. Voir Mt. 23 :15 ; Ac. 2 :10 ; Ac. 6:5 et Ac. 13:43.

\DicoEntry{PYTHON}\textit{, du grec « Puthon » : « un serpent ou un dragon}\newline
Esprit de divination auquel Paul fut confronté en Macédoine. Voir Ac. 16:16-18.

\DicoEntry{RABBI}\textit{, de l'hébreu « rab » : « capitaine, chef » et du grec « rhabbi » : « maître », « un grand monsieur, honorable » ou « un enseignant »}\newline
Les disciples appelaient Jésus « Rabbi ». Cependant, il a exhorté la foule et les conducteurs religieux à ne pas attribuer une telle marque de distinction aux hommes rappelant que seul Yahweh est maître. Voir Mc. 11:21 ; Jn. 9:2 et Mt. 23:8.

\DicoEntry{RACHEL}\textit{, de l'hébreu « Rachel » : « agnelle, brebis »}\newline
Fille de Laban, deuxième femme de Jacob pour laquelle il travailla quatorze ans. Longtemps stérile, Yahweh lui donna finalement deux garçons : Joseph et Benjamin. Elle mourut à l'accouchement du deuxième. Voir Ge. 29:10 -31 ; Ge : 30:22-24 et Ge. 35:16-19.

\DicoEntry{RAHAB}\textit{, de l'hébreu « Rachab » : « large, spacieux, tumultueux »}\newline
Prostituée habitant Jéricho, elle cacha les deux espions juifs chez elle. Grâce à son acte, Josué* lui laissa la vie sauve ainsi qu'à sa famille lorsqu'il détruisit la ville et tous ceux qui s'y trouvaient. Rahab habita ensuite au milieu d'Israël ; elle figure non seulement parmi les héros de la foi, mais aussi dans la généalogie de Jésus-Christ. Voir Jos. 2:1 ; Jos. 6:17-25 ; Mt. 1:5-16 et Hé. 11:31.

\DicoEntry{REBECCA}\textit{, de l'hébreu « Ribqah » : « ensorcelante, qui prend au piège »}\newline
Fille de Bethuel et sœur de Laban, elle fut l'épouse d'Isaac*. Yahweh mit fin à sa stérilité et elle donna naissance à des jumeaux, Esaü et Jacob, qui devinrent deux grandes nations. Voir Ge. 24 et Ge. 25:21-26.

\DicoEntry{RECONCILIATION}\textit{, du grec « katallage » : « échange, change, ajustement d'une différence »}\newline
Jésus-Christ mourut à la croix pour réconcilier l'homme avec Dieu, c'est-à-dire le faire passer de l'état de séparation (causée par le péché) à l'état d'intimité avec Dieu. L'Eglise a le ministère de réconciliation et doit en ce sens présenter à l'homme pécheur la voie de la réconciliation avec Dieu au travers de la prédication de l'Evangile*. Voir Ro. 5:11 ; Hé. 10:18-20 et 2 Co. 5:18-20.

\DicoEntry{REDEMPTION}\textit{, de l'hébreu : « peduwth » : « rachat » et du grec: « apolutrosis » : « libération effectuée suite au paiement d'une rançon »}\newline
Jésus-Christ a payé le prix nécessaire au rachat des péchés de tous les hommes par son sacrifice à la croix, leur permettant d'échapper à la mort éternelle au moyen de la foi*. Voir Ro. 3:23-24 ; Col. 1:14 ; Ep. 1:7 et Hé. 9:12.

\DicoEntry{REFORME}\textit{, de l'hébreu « yatab » : « agir bien » et du grec « diorthosis » : « remettre droit »}\newline
La plupart des prophètes du Tanakh sont des réformateurs dans la mesure où ils prônent un retour à Dieu ; le roi Josias a institué une profonde réforme pendant son règne en déployant des efforts pour revenir à l'obéissance de la Parole. Jésus-Christ est le plus grand réformateur en ce qu'il marchait à contre-courant et vint restaurer l'homme à sa condition originelle, celle d'avant la chute. Ainsi, l'homme qui reçoit Jésus entre dans un processus où il est continuellement réformé par le Saint-Esprit au travers de la Parole. Voir 2 R. 22 ; Jé. 7:5 ; Jé. 26:13 ; Os. 6:1 ; Mt. 19:8 et Jn. 16:7-15.

\DicoEntry{REPENTANCE}\textit{, du grec « metanoia » : « changement de mentalité, d'intention », « tristesse qu'on éprouve de ses péchés »}\newline
Un des points majeurs de la prédication de Jean-Baptiste* puis des apôtres*. La repentance est essentielle pour obtenir la rémission des péchés et doit être accompagnée de fruits. La repentance ne concerne pas uniquement le nouveau converti, mais tout disciple de Christ qui, jusqu'à la fin de sa vie, est dans un processus de perfectionnement. Voir Mc. 1:4 ; Lu. 3:8 ; 2 Co. 7:9-10 ; Ro. 2:4 ; Ac. 2:38 ; Ac. 13:24 ; Ac. 17:30 et Ac. 26:20.

\DicoEntry{RESURRECTION}\textit{, du grec « anastasis » : « se lever, ressusciter de la mort ».}\newline
Christ fut le premier à expérimenter la résurrection d'entre les morts. Au son de la dernière trompette*, les chrétiens décédés ressusciteront de même avec des corps incorruptibles pour les noces* de l'Agneau. Voir Mt. 28:6 ; 1 Pi. 1:3 ; Ap. 1:5 ; 1 Co. 15:52 et 1 Th. 4:16.

\DicoEntry{REVEIL}\textit{, du grec « egeiro » : « réveiller du sommeil, revenir à la vie, se lever »}\newline
Prise de conscience personnelle ou collective sur sa condition de péché et.ou l'imminence du jugement de Dieu. Il en résulte la repentance*, la véritable conversion*, la crainte de Dieu, la préparation à la rencontre de Yahweh. Une personne réveillée a les yeux focalisés sur Christ et peut accomplir la volonté du Seigneur. Voir Jon. ; Ep. 5:14 et Ro. 13:11-14.

\DicoEntry{ROBOAM}\textit{, de l'hébreu « Rhoboam » : « qui affranchit le peuple »}\newline
Fils et successeur du roi Salomon. C'est sous son règne que se produit le schisme* entre les royaumes du nord et celui du sud. Il régna sur Juda dix-sept années pendant lesquelles il fut en guerre avec le royaume du nord et fit ce qui est mal aux yeux de Yahweh. Voir 1 R. 12:1-24 et 1 R. 14:21-31.

\DicoEntry{ROMAIN}\textit{, du grec « rhome » : « force »}\newline
Pendant la vie de Jésus et pendant l'époque de l'Eglise primitive, Israël était sous la domination de l'Empire romain qui l'oppressait et lui soutirait des impôts. Paul, né à Tarse - ville romaine - put bénéficier des privilèges liés à la nationalité romaine quand il fut livré aux tribunaux. Ce dernier écrivit une lettre aux chrétiens romains - figurant dans le canon biblique - avant de les rencontrer physiquement. Voir Mt. 22 :17 ; Jn. 11:48 ; Ac. 16:35-39 ; Ac. 22:25-29 ; Ac. 23:27 et Ac. 25:16.

\DicoEntry{ROME}\textit{, du grec « rhome » : « force »}\newline
Capitale de l'Empire romain située en Italie, Rome jouissait d'une grande notoriété à l'époque de l'Eglise primitive. Bien que l'empereur Claude* ait ordonné aux Juifs de quitter la ville, Paul manifesta le désir de s'y rendre pour y annoncer l'Evangile. Il y arriva après bien des difficultés quelques années plus tard en tant que prisonnier. Voir Ac. 18:1-2 ; Ac. 19:21 ; Ac. 23:11 et Ac. 28:14-31.

\DicoEntry{ROYAUME DE DIEU}\textit{, du grec « basileia » : « pouvoir royal, royauté, domination, autorité »}\newline
Lors de son ministère terrestre, Jésus a annoncé que le royaume de Dieu était proche. Il parlait de son autorité sur toutes choses et de son règne. Ne consistant pas dans les choses terrestres, ce royaume se manifeste par la puissance de Dieu, la justice, la paix et la joie par le Saint-Esprit. Voir Lu. 9:1-2 ; Lu. 11:17-20 ; Lu. 17:20-21 et Ro. 14:17.

\DicoEntry{RUBEN}\textit{, de l'hébreu « Re'uwben » : « voici un fils »}\newline
Premier fils de Jacob et Léa, il devint le père de la tribu des Rubénites qui s'installa à l'est de la terre promise. Il perdit son droit d'aînesse après avoir eu des rapports intimes avec Bilha, concubine de son père. Voir Ge. 29:32 ; Ge. 35:22 et Ge. 49:3-4.

\DicoEntry{RUTH}\textit{, de l'hébreu « Ruwth » : « amitié, une amie »}\newline
Originaire de Moab, belle-fille de Naomi avec qui elle s'installa à Bethléhem. Elle y épousa Boaz avec qui elle eut un fils, Obed, grand-père du roi David*. Son histoire est racontée dans le livre portant son nom.

\DicoEntry{SABBAT}\textit{, de l'hébreu « shabbath » : « repos, cessation d'activité »}\newline
Septième et dernier jour de la semaine consacré à Yahweh pendant lequel aucune activité ne devait être pratiquée selon la loi. Le sabbat figure dans les dix commandements, son infraction devait être punie de mort. Suscitant de vives critiques de la part des religieux, Jésus a plusieurs fois enfreint le sabbat dont il s'est déclaré le maître. Sous la Nouvelle Alliance, le sabbat se trouve en Jésus-Christ, le chrétien n'est donc pas tenu de le respecter comme ce fut le cas sous la loi de Moïse. Ex. 20:8-11 ; Ex. 31:14-15 ; De. 5:12-15 ; Mt. 11 :28-30 ; Mc. 2:23-28 et Mc. 3:1-6.

\DicoEntry{SACERDOTALISME}\textit{}\newline
Doctrine d'origine catholique reconnaissant le prêtre ou le pasteur comme l'intermédiaire entre Dieu et les hommes. Voir 1 Ti. 2:5.

\DicoEntry{SACRIFICATURE, SACERDOCE}\textit{, de l'hébreu « kahan » : « remplir les fonctions du sacerdoce »}\newline
Sous la loi mosaïque, il était exercé par les Lévites descendants d'Aaron dans le tabernacle puis le temple et consistait notamment à accomplir les différents rituels relatifs aux sacrifices d'animaux et aux offrandes de toutes sortes. Depuis le sacrifice de Jésus à la croix, le sacerdoce concerne tous les enfants de Dieu qui sont non seulement les sacrificateurs, mais aussi les sacrifices auxquels le Seigneur prend plaisir. Voir Lé. ; Ro. 12:1 ; 1 Pi. 2:9 et Ap. 1:6.

\DicoEntry{SADDUCEENS}\textit{, du grec « saddoukaios » : « les justes »}\newline
Parti religieux juif attaché au Pentateuque de manière stricte, ils ne croyaient ni en la résurrection des morts ni aux anges. Ils s'opposèrent au ministère de Jésus qui les reprit sévèrement et échappa à leurs pièges. Ils combattirent de même les apôtres qu'ils jetèrent en prison. Voir Mt. 16:6-12 ; Mt. 22:23-33 ; Ac. 5:17-19 et Ac. 23:1-10.

\DicoEntry{SAINT}\textit{, de l'hébreu « qodesh » : « consacré, mis à part » et du grec « hagios » : « chose très sainte, consacré, un saint »}\newline
Dieu appela Israël à la sainteté, c'est-à-dire à ne pas se mélanger avec les autres peuples de peur d'être contaminés par leurs pratiques méchantes et idolâtres. Yahweh est le Saint d'Israël. Sous la Nouvelle Alliance, les chrétiens sont appelés saints, car le Saint-Esprit qui est en eux leur communique sa nature, les purifie et leur enseigne la haine du péché. Voir De. 7:6 ; Es. 49:7 ; 1 Co. 6:11, 19 ; 1 Th. 4:1-8 et Hé. 12:14.

\DicoEntry{SAINT-ESPRIT }\textit{, (voir étymologie des mots « saint » et « esprit »)}\newline
Le Saint-Esprit est l'esprit de Dieu, l'esprit de Jésus ; il est Dieu. Lors de son ministère terrestre, le Seigneur déclara qu'un consolateur viendrait habiter dans les corps des croyants. Cette parole s'accomplit lors de la Pentecôte. Le Saint-Esprit a pour mission de convaincre le monde en ce qui concerne le péché, la justice et le jugement. A la naissance d'en haut, il régénère l'esprit du chrétien sur qui il dépose son sceau, gage de l'adoption. Il enseigne et guide le chrétien tout au long de sa marche avec Dieu. Il transforme son caractère et distribue les dons spirituels pour l'édification de l'Eglise. Voir 1 S. 10: 10 ; 2 Ch. 15: 1; Jn. 14:16-17, 26 ; Jn. 16:7-15 ; Ac. 2 ; 1 Co. 6:11 ; Ro. 8:9 ; 1 Co. 3:16, 1 Co. 12:4-13 ; Ep. 1:13 et Ga. 5:16, 22.

\DicoEntry{SALOMON}\textit{, de l'hébreu « Shelomoh » : « paix, pacifique »}\newline
Fils de David, il succéda à son père et fut roi d'Israël pendant quarante ans. Il construisit le premier temple de Yahweh sur le mont Morija à Jérusalem puis un palais royal. Outre ses importantes richesses, c'est la grande sagesse que Dieu lui donna qui fit sa renommée parmi tous les peuples. Il eut sept cents femmes et trois cent concubines - dont un grand nombre de femmes étrangères - ce qui détourna son cœur de son Dieu. On lui attribue la rédaction des livres Cantique des cantiques et Ecclésiaste ; il a aussi écrit certains psaumes et plusieurs proverbes. Voir 1 R. 4:29-34 ; 1 R. 5 à 7 ; 1 R. 9:15-28 ; 1 R. 11:1-10, 42 ; Ps. 72 et 127 et Pr. 25 à 29.

\DicoEntry{SALUT}\textit{, de l'hébreu « yesha' » et du grec « soteria » : « délivrance, sûreté, sécurité »}\newline
Libération des chaines du péché, de la condamnation et de tout type d'asservissement spirituel, le salut est un don gratuit de Dieu qui s'obtient par la grâce, au moyen de la foi. C'est la manifestation de l'amour éternel de Dieu qui - ne voulant pas que l'homme périsse dans le feu de la géhenne - a payé le prix pour lui offrir la vie éternelle. Le salut réside dans le seul nom de Jésus-Christ. Voir Jn. 3:16 ; Ac. 4:12 ; Ro. 8:1 ; 1 Th. 5:9 ; Tit. 3:4-6 et Ep. 2:4-8.

\DicoEntry{SAMARIE}\textit{, de l'hébreu « Shomerown » et du grec « Samareia » : « montagne de guet »}\newline
Située dans l'actuelle Cisjordanie, ville fondée par Omri, roi d'Israël, et qui devint la capitale du royaume du nord. La ville fut prise par Salmanasar, roi d'Assyrie, sous le règne d'Osée, roi d'Israël. Au temps de Jésus, la Samarie n'était qu'une simple circonscription romaine dont la population était issue du métissage entre Israélites et des colons assyriens. Suite aux persécutions subies par l'Eglise primitive à Jérusalem, des chrétiens s'y réfugièrent et l'Evangile s'y propagea. Voir 1 R. 16:23-24 ; 2 R. 3:1 ; 2 R. 18:9 ; Os. 7 et Ac. 8:1-17.

\DicoEntry{SAMARITAINS}\textit{, du grec « samareites » : « un habitant de Samarie »}\newline
Après l'assujettissement de la Samarie par Salmanasar, roi d'Assyrie, des peuples étrangers s'y établirent et s'assemblèrent avec les Israélites. Au IVe siècle av J.-C., les samaritains construisirent un temple sur le mont Garizim, qui devint le centre religieux de Samarie, entraînant une séparation avec le reste des Juifs qui adoraient à Jérusalem. Les samaritains étaient considérés comme des étrangers et non comme de véritables juifs du fait de la mixité de leur religion. Jésus-Christ ouvrit la voie de la réconciliation avec ce peuple en racontant la parabole du bon Samaritain et en annonçant la bonne nouvelle à la femme samaritaine. Voir 2 R. 17:3, 24-29 et 2 R. 18:9 ; Jn. 4:4-26 et Lu. 10:30-37.

\DicoEntry{SAMSON}\textit{, de l'hébreu « Shimshown » : « petit soleil »}\newline
Fils de Manoach, de la tribu de Dan, il fut juge en Israël pendant vingt-ans. Consacré à Dieu dès le sein maternel et doté d'une force extraordinaire, il réalisa des prouesses qui suscitèrent la crainte de ses ennemis. Choisi pour être le libérateur d'Israël, il fut incompris par les siens qui ne le soutinrent pas. Il mourut suite à la trahison de Delila, une femme d'origine philistine. Voir Jg. 13 à 16.

\DicoEntry{SAMUEL}\textit{, de l'hébreu « Shemuw'el » : « entendu ou exaucé de Dieu »}\newline
Fils d'Elkana, de la tribu d'Ephraïm, et d'Anne, il fut consacré au service de Yahweh dès son plus jeune âge. Il exerça les fonctions de juge, sacrificateur et prophète sur Israël. Il oignit les deux premiers rois d'Israël : Saül et David. Son histoire est racontée dans les deux livres du Tanakh portant son nom.

\DicoEntry{SANCTIFICATION}\textit{, de l'hébreu « Qadash » et du grec « hagiasmos » : « consécration, purification, sainteté » ou « l'effet de la purification »}\newline
Fruit de l'action conjointe de la Parole et l'Esprit de Dieu dans la vie du croyant, la sanctification doit être recherchée par le chrétien tout au long de sa vie. Sans elle, nul ne verra Dieu. Jn. 17:17 ; 1 Th. 4:3-8 ; Hé. 12:14 et Ap. 22:11.

\DicoEntry{SANCTUAIRE}\textit{, de l'hébreu « miqdash » : « lieu sacré, lieu saint, sanctuaire de Yahweh »}\newline
Le sanctuaire terrestre, dont Moïse avait reçu le modèle, était une représentation de celui qui se trouve au ciel et où Jésus alla présenter son sang. Voir Ex. 25:8-9 et Hé. 9:1-24.

\DicoEntry{SANG}\textit{, de l'hébreu « dam » et du grec « haima » : « sang »}\newline
Déterminant le lien de famille et la lignée, le sang est, selon les Ecritures, l'âme*, la vie. Ainsi, l'effusion de sang fut nécessaire pour le pardon des péchés et le sang de Christ, qui ôte définitivement le péché, donne la vie. Voir Lé. 17:11 ; Ac. 17:26 ; Ro. 5:9 ; Hé. 9:22-28 et Ap. 5:9.

\DicoEntry{SANHEDRIN}\textit{, du grec « sunedrion » : « conseil, tribunal »}\newline
Désignait d'une part, les petits tribunaux se tenant dans chaque ville pour régler les affaires locales et d'autre part, le grand conseil de Jérusalem où étaient traitées les affaires plus importantes. Ce dernier était composé de soixante et onze membres sélectionnés parmi l'élite religieuse et les anciens d'Israël ; le souverain sacrificateur en était le président. Sous la domination romaine, ce tribunal fonctionnait de manière quasi autonome ; la sentence de la peine de mort devait néanmoins être validée par le gouverneur romain. Jésus fut jugé coupable de blasphème par le sanhédrin qui le condamna à mort. Voir No. 11:16-17 ; Mt. 5:22 ; Mt. 26:59-66 ; Jn. 11:47 ; Ac. 5:21-41 et Ac. 6:12-15.

\DicoEntry{SARA}\textit{, de l'hébreu « Sarah » : « princesse, femme noble »}\newline
Femme d'Abraham, elle enfanta Isaac à l'âge de 90 ans selon la promesse de Yahweh. Sara figure parmi les héros de la foi ; elle mourut à cent vingt-sept ans. Voir Ge. 12:5 ; 17 ; Ge. 15-16 ; Ge. 21:1-7 ; Ge. 23:1 et Hé. 11:11.

\DicoEntry{SARDES}\textit{, du grec « sardeis » : « les rouges », « prince de joie »}\newline
Capitale antique de la Lydie, Sardes se situait sur la rivière Pactole, à environ 50 km au sud de Thyatire et 75 km à l'est de Smyrne. Réputée riche et puissante en raison de ses ressources en or, ses épithètes étaient sournois car sa forteresse reposait sur un sol boueux. En effet, au VIème siècle av. J.-C., Cyrus Le Grand - vainqueur de Crésus alors roi de Lydie - s'empara de Sardes par une attaque nocturne. Par la suite, la ville subit plusieurs invasions puis un tremblement de terre en 17 ap. J.-C. L'église de Sardes fut probablement fondée par Paul au cours d'un voyage à Ephèse. Au moment où ils reçurent le message de l'ange de l'Apocalypse, il semblerait que certains chrétiens de Sardes étaient retournés au culte licencieux de Cybèle, déesse-mère et gardienne des savoirs. Ceux qui s'étaient gardés purs devaient ainsi revivifier les autres membres. Cette église symbolise l'église morte. Voir Ap. 3:1-6.

\DicoEntry{SATAN}\textit{, de l'hébreu « Satan » : « adversaire, ennemi »}\newline
Autrefois chérubin protecteur, il a péché en voulant s'approprier la gloire qui ne revient qu'à Dieu. Dans sa rébellion, il entraîna un tiers des anges qui furent précipités avec lui sur la terre. Connu également sous les noms « Prince de ce monde », « Prince des ténèbres », « Belzébul », « le malin », « l'accusateur » ou « le diable », il est l'adversaire des enfants de Dieu à qui il fait la guerre. Il a cependant été vaincu à la croix par Jésus-Christ, au nom duquel les chrétiens peuvent le chasser. Satan sera enchaîné pendant le millenium puis libéré pour un peu de temps. Il sera finalement jeté dans l'étang de feu pour l'éternité. Voir Ez. 28:14-19 ; Es. 14:12-17 ; Ap. 12:4 ; Lu. 10:18-19 ; Ja. 4:7 ; Jn. 16:11 ; Ap. 12:4 et Ap. 20:1-15.

\DicoEntry{SAUL}\textit{, de l'hébreu « Sha'uwl » : « désiré, demandé (à Dieu) »}\newline
1. Fils de Kis, israélite de la tribu de Benjamin, il fut choisi par Dieu pour être le premier roi d'Israël sur qui il régna pendant quarante ans. Il désobéit à la loi de Yahweh et tenta plusieurs fois d'assassiner David, choisi par Dieu pour lui succéder sur le trône. Saül mourut avec ses trois fils pendant la bataille de Guilboa. Voir 1 S. 10 ; 1 S. 13:1-14; 1 S. 15:10-11 ; 1 S. 18:8-16 ; 1 S. 19:8-17 et 1 S. 31.
\\2. Nom initial de l'apôtre Paul*.

\DicoEntry{SCANDALE}\textit{, de l'hébreu « mikshowl » : « trébucher » et du grec « skandalon » : « obstacle, piège »}\newline
Pierre qu'on rencontre et qui peut faire glisser sur le chemin ou encore situation ou comportement qui provoque un trouble emmenant quelqu'un à fauter. Le scandale n'est pas forcément une mauvaise action en soi ; Christ lui-même fut un scandale pour les Juifs. Toutefois, il reste souvent lié aux œuvres de la chair et peut être provoqué par un manque de discernement. Le chrétien doit veiller par rapport aux scandales. Voir Ps. 106:36 ; Mt. 13:41 ; Mt. 18:7 ; 1 Co. 1:23 et 1 Pi. 2:7-8.

\DicoEntry{SCEAU}\textit{, du grec « sphragizo » : « mettre un sceau dessus, poser une marque par l'impression d'un sceau »}\newline
Sous l'Ancienne Alliance, la circoncision était une marque de l'alliance établie entre Yahweh et son peuple. A la naissance d'en haut, le chrétien est scellé du Saint-Esprit, témoignant de son appartenance à Christ. Voir Ge. 17:10-11 Ep. 1:13 ; Ap. 7:3 et Ap. 9:4.

\DicoEntry{SCHEOL}\textit{}\newline
Voir SEJOUR DES MORTS.

\DicoEntry{SCHISME D'ISRAEL}\textit{}\newline
Le schisme est la séparation d'Israël en deux royaumes suite à la dérive de Salomon. En 931 av. J.-C., Roboam succéda à son père Salomon sur le trône royal et n'accepta pas d'alléger le joug que son père avait mis sur eux, cela entraîna la séparation du royaume en deux. On retrouva d'une part, le royaume d'Israël dirigé par Jéroboam - appelé aussi royaume du nord -, composé des dix tribus du nord et d'autre part, le royaume de Juda gouverné par le roi Roboam composé des deux tribus du sud (Benjamin et Juda). Voir 1 R. 12:1-24.

\DicoEntry{SCRIBE}\textit{, de l'hébreu « caphar » : « secrétaire, scribe », « homme instruit, qui a le savoir »}\newline
Les scribes occupaient une position importante auprès du peuple juif, ayant non seulement une mission d'enseignement de la loi, mais également une fonction au sein de la justice juive en prenant part au sanhédrin*. Voir Esd. 7:6-10 et Mt. 16:21.

\DicoEntry{SECTE}\textit{, du grec « hairesis » : « action de prendre, capturer »}\newline
Groupement de personnes adhérant à une doctrine particulière et vivant marginalement, comme les sadducéens ou les pharisiens. Les premiers disciples furent qualifiés de « secte des Nazaréens ». Pierre met en garde contre les faux prophètes qui introduisent des sectes pernicieuses pour ravir la foi des chrétiens afin de les entraîner dans la dissolution. Voir Ac. 5:17 ; Ac. 26:5 ; Ac. 24:5 et 2 Pi. 2:1.

\DicoEntry{SEDECIAS}\textit{, de l'hébreu « Tsidqiyah » : « Yahweh est justice »}\newline
Fils d'Hamoutal et oncle de Jojakin, il fut le dernier roi de Juda sur qui il régna onze ans. Son nom initial, Matthania, fut changé en Sédécias par Nebucadnetsar, roi de Babylone. Il fit ce qui est mal aux yeux de Yahweh et connut un destin tragique : ses fils furent égorgés devant lui, Nebucadnestar lui creva ensuite les yeux, Jérusalem et le temple* furent détruits et il fut emmené captif avec le peuple à Babylone. Voir 2 R. 24:17-19 ; 2 R. 25:1-21 ; Jé. 21 ; Jé. 22:1-9 ; Jé. 37, 38 et Jé. 39:6-7.

\DicoEntry{SEJOUR DES MORTS }\textit{de l'hébreu « she'owl » : « monde souterrain, tombe, enfer, fosse » et du grec « hades » : « dieu des profondeurs de la terre »}\newline
Lieu de captivité où allaient les âmes de tous les défunts avant le sacrifice de Christ. Il était scindé en deux parties séparées par un grand abîme. D'un côté se trouvait un lieu de tourments et de souffrances extrêmes accueillant tous les méchants qui ont vécu dans le péché durant leur vie terrestre et qui n'y ont pas renoncé. D'un autre côté, il y avait le sein d'Abraham où reposaient et séjournaient les âmes des justes qui avaient foi en Yahweh. Après la résurrection de Jésus, ces derniers ont été arrachés du séjour des morts par le Seigneur qui les a emmenés au paradis*. Le ciel, en tant que destination des personnes décédées, fut en effet ouvert par Christ après sa résurrection. Par conséquent, le sein d'Abraham n'a jamais accueilli de chrétiens. Le séjour des morts est à présent composé uniquement d'impies ; à la fin du monde, il sera jeté avec tous ses habitants dans l'étang de feu*. Voir Lu. 16:19 -31 ; 1 S. 28:6-20 ; Mt. 11:23 ; Ac. 2:27 ; Jn. 3:13 ; Ep. 4:8 et Ap. 20:14.
\\Note : L'histoire de Lazare et de l'homme riche racontée dans Luc 16:19-31 n'est pas une parabole. A la différence de tous les récits à caractère parabolique contés dans les Ecritures, cette histoire mentionne un nom.

\DicoEntry{SEM}\textit{, de l'hébreu « Shem » : « nom, renommée »}\newline
Fils aîné de Noé et ancêtre d'Abraham. Voir Ge. 10:1 et Ge. 11:10-27.

\DicoEntry{SENEVE}\textit{, du grec « sinapi » : « graine de moutarde »}\newline
Plante des régions orientales ayant la forme d'une petite semence pouvant grandir de manière exponentielle et atteignant jusqu'à trois mètres. Elle symbolise spirituellement la puissance de la foi*capable de déplacer les montagnes. Voir Mt. 13:32-33 et Mt. 17:20.

\DicoEntry{SEPHORA}\textit{, de l'hébreu « Tsipporah » : « petit oiseau, moineau »}\newline
Fille de Jéthro, femme de Moïse et mère d'Eliézer et de Guerschom. Elle partit dans le pays d'Egypte avec Moïse quand il répondit à l'appel de Yahweh pour aller libérer Israël. Ex. 2 :15-21 et Ex. 4:18-20.

\DicoEntry{SERAPHINS}\textit{, de l'hébreu « saraph » : « être majestueux avec six ailes au service de Dieu »}\newline
Catégorie d'anges* proclamant la sainteté de Dieu. Voir Es. 6:1-7.

\DicoEntry{SERPENT}\textit{, de l'hébreu « nachash » : « serpent, reptile »}\newline
C'est sous la forme du serpent que Satan vint séduire Eve dans le jardin d'Eden. Le serpent fut maudit d'entre tous les animaux pour son action. Le serpent ancien ou rusé désigne le diable* ; il s'oppose au serpent d'airain, Jésus, qui a donné à ses enfants le pouvoir de marcher sur les serpents. Voir Ge. 3:1-14 ; No. 21:4-9 ; 2 Co. 11:3 ; Jn. 3:14-15 ; Lu. 10:19 et Ap. 12:9, 14, 15.

\DicoEntry{SERVITEUR}\textit{, du grec « diakonos » : « domestique, subordonné, messager » ou « doulos » : « esclave »}\newline
Christ a renoncé à sa gloire et a pris la forme d'un simple serviteur. De même, le chrétien n'est pas uniquement serviteur de Dieu, il doit comme le maître servir son prochain. Voir Mc. 10:45 ; Ph. 1:1 ; Ph. 2:5-8 et 2 Co. 6:4.

\DicoEntry{SETH}\textit{, de l'hébreu « Sheth » : « compensation, mis à la place »}\newline
Troisième fils d'Adam et Eve ; il naquit après le meurtre de son frère Abel que Caïn avait tué. Seth fut l'ancêtre de Noé et de Jésus-Christ. Voir Ge. 4:25 ; Ge. 5:6-29 et Lu. 3:38.

\DicoEntry{SHOFAR}\textit{, de l'hébreu « showphar » : « corne, corne de bélier ».}\newline
Instrument de musique à vent fait à partir de la corne de bélier. Voir TROMPETTE.

\DicoEntry{SIDON}\textit{, de l'hébreu « Tsiydown » : « abondance de poisson, pêche »}\newline
Ville de l'antique Phénicie (actuel Liban) située non loin de Tyr ; on y vénérait les Baals et les Astartés. La reine Jézabel était originaire de Sidon. Voir Jg. 10:6 et 1 R. 16:31.

\DicoEntry{SILAS}\textit{, du grec « Silas » : « de la forêt, demandé »}\newline
Prophète, compagnon d'œuvre de Paul avec qui il effectua plusieurs voyages missionnaires. Voir Ac. 15-18.

\DicoEntry{SILO}\textit{, de l'hébreu « Shiyloh » : « lieu de repos »}\newline
Ville située au nord-est de la tribu d'Ephraïm où les enfants d'Israël se répartirent les territoires avant la conquête de Canaan. Avant d'être placée à Jérusalem, l'arche de l'alliance se trouvait à Silo. Voir Jos. 18:10 ; Jos. 19:51 ; 1 S. 3:19-21 et 1 S. 4:3.

\DicoEntry{SILOE}\textit{, de l'hébreu « Shiloach » : « envoyé »}\newline
Source d'eau se trouvant au sud-est de Jérusalem. Voir Né. 3:15 et Jn. 9:6-7.

\DicoEntry{SIMEON}\textit{, de l'hébreu « Shim`own » : « qui écoute, qui a été entendu »}\newline
1. Fils de Jacob et Léa. Avec Lévi, son frère, il vengea le déshonneur de sa sœur Dina, en tuant Sichem, prince de Canaan, son père Hamor, et tous leurs hommes. Il fut gardé comme otage en Egypte, lorsque Joseph voulut éprouver la sincérité de ses frères. Il devint le père de la tribu des Siméonites qui s'installèrent au sud de Canaan. Voir Ge. 29:33 ; Ge. 34 ; Ge. 42:21-38 et Jos. 19:1-9.
\\2. Homme de foi à qui le Saint-Esprit avait promis qu'il ne mourrait pas sans avoir vu le Messie. Il rencontra Jésus lorsqu'il était enfant, à Jérusalem. Voir Lu. 2:25-35.

\DicoEntry{SIMON}\textit{, de l'hébreu « Shiymown » : « désert » ou « qui entend »}\newline
1. Simon Pierre, le nom originel de l'apôtre Pierre* était Simon. Voir Jn. 1:40-42.
\\2. Simon le zélote, il faisait partie du groupuscule des zélotes avant de devenir apôtre de Christ. Voir Lu. 6:13-16.
\\3. Simon de Cyrène, il fut contraint d'aider Jésus à porter la croix jusqu'à Golgotha. Voir Mt. 27:32.
\\4. Simon le magicien, originaire de la ville de Samarie, il fut baptisé par Philippe et crut pouvoir acheter à prix d'argent la puissance du Saint-Esprit. Voir Ac. 8:9-24.

\DicoEntry{SION}\textit{, de l'hébreu « Tsiyown » : « lieu desséché »}\newline
Autre nom pour parler de Jérusalem. Sous la Nouvelle Alliance, la montagne de Sion est l'image de la Jérusalem céleste. Voir De. 4:48 ; 1 R. 8:1 ; Es. 2:3 ; 2 S. 5:6-7 et Hé. 12:22.

\DicoEntry{SISERA}\textit{, de l'hébreu « Ciycera' » : « déploiement, champ de bataille »}\newline
Chef de l'armée du roi cananéen Jabin, son armée fut vaincue par Barak et Sisera fut tué par Jaël, femme de Héber, le Kénien. Voir Jg. 4.

\DicoEntry{SMYRNE}\textit{, du grec « Smurna » : « myrrhe »}\newline
Cité de la côte occidentale de l'Asie Mineure, Smyrne (aujourd'hui Izmir) était située au nord d'Ephèse et réputée pour sa splendeur et ses richesses. Ses forteresses et ses tours de l'acropole évoquaient une couronne. Très unie à Rome, des cultes en l'honneur du dieu Zeus, de la déesse Cybèle, ou encore de l'empereur Tibère et sa mère Julie y étaient célébrés. Proche d'Ephèse, l'église de Smyrne fut probablement le fruit du travail apostolique de Paul. En proie à ces doctrines impies, l'église de Smyrne était fortement persécutée aussi bien par les Romains que par « les faux Juifs » membres « d'une synagogue de Satan ». Sa persévérance face aux afflictions lui permit de recevoir un bon témoignage du Seigneur. Elle incarne l'église persécutée. Voir Ap. 2:8-11.

\DicoEntry{SODOME}\textit{, de l'hébreu « Cedom » : « qui brûle »}\newline
Ville cananéenne située dans la plaine du Jourdain à proximité de laquelle Lot s'installa après s'être séparé d'Abraham. Ses habitants étaient de grands pêcheurs devant Yahweh à un tel point qu'il détruisit la ville - avec Gomorrhe* - en faisant tomber du ciel une pluie de feu et de soufre. Lot et ses deux filles furent épargnés grâce à l'intercession* d'Abraham. Voir Ge. 13:10-13 et Ge. 19:1-29.

\DicoEntry{SOPHONIE}\textit{, de l'hébreu « Tsephanyah » : « Yahweh a caché, protégé »}\newline
Fils de Cuschi, descendant du roi Ezéchias, prophète de Yahweh ayant vécu au temps du roi Josias. L'ensemble de ses prophéties se trouve dans le livre portant son nom.

\DicoEntry{SOUVERAIN SACRIFICATEUR}\textit{, de l'hébreu « kohen » : « prêtre, intendant principal, ministre d'état »}\newline
Sous la loi, le souverain sacrificateur descendait d'Aaron. Il servait le Seigneur dans le sanctuaire* et enseignait la loi*. Tel un médiateur* entre Yahwhe et le peuple, il portait constamment le jugement de ce dernier pour qui il consultait Dieu au moyen de l'urim et du thummim. Il devait, une fois par an, entrer dans le Saint des Saints et offrir des sacrifices d'animaux pour ses propres péchés et pour ceux du peuple. Par la suite, Jésus-Christ est devenu souverain sacrificateur à perpétuité en s'offrant comme victime expiatoire et en présentant son sang une fois pour toutes dans le Saint des saints du temple céleste. Voir Ex. 28:30 ; Ex. 29:9 ; Esd. 2:63 ; No. 35:25 ; Hé. 4:14-16 ; Hé. 7:25-28 et Hé. 9:6-12, 24-28.

\DicoEntry{STOICIENS}\textit{, du grec « stoikos » : « appartenant au portique »}\newline
Adeptes de la doctrine de Zénon de Kition (336 av. J.-C. – 264 av. J.-C.) qui fonda le stoïcisme à Chypre en 301 av. J.-C. Le stoïcisme était l'une des principales doctrines philosophiques de la Grèce antique avec l'épicurisme. Elle reposait sur la morale et la maîtrise de ses sentiments par une vie en conformité avec la nature. A Athènes, quelques stoïciens, accompagnés d'épicuriens, se confrontèrent à l'apôtre Paul, le menant à l'aréopage afin de l'interroger. Voir Ac. 17:18-20.

\DicoEntry{SYNAGOGUE}\textit{, du grec « sunagogue » : « assemblée, lieu de réunion »}\newline
Assemblée de Juifs réunis pour prier et écouter la lecture des Ecritures. Jésus y enseigna régulièrement pendant son ministère. Les apôtres annoncèrent également l'Evangile dans des synagogues. Voir Mt. 4:23 ; Mt. 9:35 ; Mc. 6:2 et Ac. 14:1.

\DicoEntry{TABERNACLE}\textit{, de l'hébreu « mishkan » : « sanctuaire, demeure, lieu d'habitation »}\newline
Appelée aussi tente d'assignation, habitation mobile de Yahweh construite selon le modèle que Dieu donna à Moïse dans le désert. Les Lévites en assuraient le service avec tous les ustensiles qui lui étaient dédiés. Une nuée s'élevait au-dessus du tabernacle pour signifier aux Israélites qu'ils devaient lever le camp* et poursuivre leur marche. Voir Ex. 25:8-9 ; Ex. 39:32 ; No. 1:50-51 ; Ex. 40:36 – 38 et 1 Ch. 6:48.

\DicoEntry{TANAKH}\textit{}\newline
Voir Introduction.

\DicoEntry{TEMPLE}\textit{, de l'hébreu « heykal » : « palais, temple, sanctuaire » (voir illustration)}\newline
David projeta de construire un temple pour Yahweh ; son fils Salomon fut mandaté pour l'ériger en remplacement du tabernacle. Il fut détruit une première fois par les Babyloniens au VIème siècle av. J.-C. Reconstruit lors du retour d'exil des Juifs, il fut de nouveau détruit en 70 par les Romains ; il n'en reste qu'un mur aujourd'hui appelé « mur des lamentations ». Sous la Nouvelle Alliance, Yahweh a choisi pour temple l'Eglise*, le corps de chaque chrétien en qui il vient résider à la naissance d'en haut. Voir 2 S. 7 ; 1 R. 6 ; 2 R. 25:8-9 ; Esd. 6:15 ; Ep. 2:21-22 et 1 Co. 6:19.

\DicoEntry{TENEBRES}\textit{, de l'hébreu « chosnek » : « obscurité, ténèbres, nuit, lieu caché »}\newline
Dès la Genèse, la lumière est séparée des ténèbres qui peuvent symboliser le péché, l'ignorance et l'absence de la vie de Dieu. Véritable prison rendant les hommes captifs, les ténèbres éternelles du séjour des morts* seront pour les anges déchus, le diable et tous les méchants. Voir Ge. 1:2-5 ; 2 S. 22:29 ; Ps. 107:10 ; Job. 17:13 ; Ro. 13:12 ; 1 Th. 5:5 ; 2 Pi. 2:4 ; 1 Jn. 2:11 et Jud. 1:6-13.

\DicoEntry{TEREBINTHE}\textit{, de l'hébreu « 'elah » : « térébinthe ou chêne »}\newline
Grand arbre robuste dont l'ombrage est agréable, il est répandu en Israël. Jacob enterra les dieux étrangers de sa maison sous un térébinthe. L'ange de Yahweh apparut sous un térébinthe à Gédéon. Des cultes idolâtres étaient célébrés à l'ombre de ces arbres. C'est à la vallée des térébinthes, située au sud-ouest de Jérusalem, que David tua Goliath. Voir Ge. 35:4 ; Jg. 6:11, 19 ; 2 S. 18:9 ; 1 Ch. 10:12 ; Os. 4:13 ; Es. 57:5 et 1 S. 17:1-50.

\DicoEntry{TETRARQUE}\textit{, du grec « tetrarches » : « tétrarque »}\newline
Titre donné au gouverneur d'un territoire sous domination romaine. Hérode Antipas* était le tétrarque de Galilée. Voir Mt. 14:1 et Lu. 3:1.

\DicoEntry{THADEE}\textit{}\newline
Voir JUDE.

\DicoEntry{THERAPHIM}\textit{, de l'hébreu « teraphiym » : « idolâtries, idoles »}\newline
Amulette utilisée dans les cultes idolâtres. Rachel déroba les théraphim de son père Laban avant de quitter sa maison. Voir Ge. 31:19, 34-35.

\DicoEntry{THESSALONIQUE}\textit{, du grec « Thessalonike » : « victoire de ce qui est faux »}\newline
Ville située au Nord de la Grèce actuelle, sur la côte de la mer Egée, elle jouissait d'une importante fréquentation puisqu'elle figurait parmi les trois ports principaux de la Méditerranée et se situait sur l'une des plus grandes routes commerciales de l'époque : la Voie Egnatienne reliant Rome à Byzance. Sur le plan religieux, les habitants étaient polythéistes et pratiquaient une variété de cultes, dont le culte impérial. Durant trois semaines, Paul enseigna dans une synagogue* à Thessalonique ; de là, il réussit à constituer un groupe de croyants. Toutefois, une violente persécution l'obligea à quitter promptement la ville, laissant la communauté nouvellement formée vulnérable et fragile. Il écrivit deux lettres aux saints de Thessalonique qui figurent dans le canon biblique.

\DicoEntry{THOMAS}\textit{, de l'hébreu « Ta'own » : « jumeau »}\newline
Surnommé Didyme, il était l'un des douze apôtres*. Dans un premier temps incrédule quant à la résurrection de Jésus, il confessa la Seigneurie de ce dernier lorsqu'il le vit ressuscité. Voir Lu. 6:12 ; Jn. 11:16 et Jn. 20:24-29.

\DicoEntry{TIMOTHEE}\textit{, du grec « Timotheos » : « qui adore, ou honore Dieu »}\newline
Fils d'une femme juive croyante et d'un père grec. Lié à Paul comme un fils à son père, il devint l'un de ses plus fidèles collaborateurs et l'accompagna à plusieurs reprises dans ses voyages missionnaires. Malgré sa jeunesse, il lui fut confié des tâches liées à la direction des églises, notamment à Ephèse. Timothée reçut de Paul deux lettres regorgeant de conseils et d'instructions pour être un bon serviteur de l'Evangile*. Voir Ac. 16:1-3 ; Ac.18:5 ; 1 Co. 16:10 et les deux épîtres de Paul à Timothée.

\DicoEntry{TITE}\textit{, du grec « Titos » : « nourrice, honorable »}\newline
D'origine grecque, Tite fut un fidèle compagnon d'œuvre de l'apôtre Paul. Il l'accompagna à Jérusalem, œuvra à Corinthe et en Dalmatie et s'occupa plus particulièrement de l'église de Crète. Il reçut une lettre de Paul qui figure dans le canon biblique. Voir 2 Co. 8:6, 23, Ga. 2:1 ; 2 Ti. 4:10 et l'épître de Paul à Tite.

\DicoEntry{TRIBULATION}\textit{, du grec « thlipsis » : « une pression, une oppression »}\newline
Persécution, tourment provoqué par l'annonce de l'Evangile. Inévitables pour entrer dans le royaume de Dieu*, les tribulations ont pour but de rendre le chrétien patient, joyeux et persévérant en toutes circonstances. Voir Mc. 4:17 ; Jn. 16: 33; Ac. 14:22 ; 2 Co. 6:4 ; 2 Co. 8:2 ; Ph. 1:29 ; 1 Th. 3:3 et 2 Th. 1:4.

\DicoEntry{TRIBUNAL}\textit{, de l'hébreu « qahal » : « assembler, convoquer » et du grec « bema » : « tribune »}\newline
Lieu où les hommes sont jugés afin de recevoir une sentence en fonction des actes qu'ils ont posés. Chaque être humain comparaîtra devant le tribunal de Christ afin de rendre compte pour lui-même. Voir Ro. 14:10-12 et 2 Co. 5:10.

\DicoEntry{TRINITE}\textit{}\newline
Doctrine selon laquelle le Dieu unique se manifesterait en trois personnes distinctes : Père, Fils et Saint-Esprit. Inspirée des triades païennes (babylonienne, égyptienne…), cette fausse doctrine d'origine catholique apparut au IIème siècle et fut fixée aux Conciles de Nicée en 325 et de Constantinople I en 381. Elle fut largement reprise par les protestants et la plupart des mouvements chrétiens alors que ni le mot ni le concept de trinité n'apparaissent dans les Ecritures. Voir Dt. 6:4 ; Es. 9:5 ; Jn. 4:23-24 ; Col. 2:8-10 ; 2 Th. 1:12 et 1 Jn. 5:20.

\DicoEntry{TROMPETTE}\textit{, plusieurs mots hébreux ont été traduits par trompette, les plus utilisés sont : « chatsotserah » : « trompette, clairon » ; « yobel » : « bélier, corne de bélier », « retentissant », « jubilé », et « showphar » : « corne de bélier ». Plusieurs mots grecs ont aussi été utilisés, notamment « salpigx » : « une trompette » et « salpizo » : « sonner de la trompette ».}\newline
Sous l'Ancienne Alliance, on l'utilisait pour donner un signal, publier une sainte convocation, fêter des moments de joie, signifier une victoire, chanter des cantiques en l'honneur de Yahweh, avertir et rassembler le peuple. Le son de la trompette est aussi l'image des voix prophétiques qui crient et appellent le peuple à revenir totalement à Dieu. Selon les Ecritures, lorsque la dernière trompette retentira, l'Eglise sera enlevée pour les noces. Dans le livre d'Apocalypse, la voix du Seigneur est comparée au son d'une trompette. Voir Ex. 19:13 ; Lé. 23:24 ; No. 10:9-10 ; 1 Ch. 16:42 ; Ez. 33:3 ; Mt. 24:31 ; 1 Co. 15:52 ; 1 Th. 4:16 et Ap. 1:10.

\DicoEntry{TYR}\textit{, de l'hébreu « Tsor » : « un rocher »}\newline
Ville de l'antique Phénicie (actuel Liban). Hiram, roi de Tyr, donna- en échange de vivres - du bois de cèdre et du bois de cyprès à Salomon pour la construction du temple. Le roi de Tyr est une image de Satan* dans une prophétie d'Ezéchiel. Voir 1 R. 5:1-12 et Ez. 28.

\DicoEntry{UR}\textit{, de l'hébreu « 'Uwr » : « flamme, éclat, feu »}\newline
Ville de Chaldée située au sud de la Babylonie et d'où Abraham était originaire. Voir Ge. 11:27-31.

\DicoEntry{URIE}\textit{, de l'hébreu « Uwriyah » : « Yahweh est ma lumière »}\newline
Héthien, mari de Bath-Schéba. Il mourut sur le champ de bataille suite à une conspiration de David qui avait connu sa femme et l'avait mise enceinte. Voir 2 S. 11.

\DicoEntry{VIE ETERNELLE}\textit{, de l'hébreu « aionios » : « sans commencement ni fin »}\newline
La vie éternelle est un don gratuit de Dieu, un héritage, une promesse qui commence dès la conversion au travers de la connaissance de Dieu. La vie éternelle est Christ lui-même. Voir Jn. 3:16, 36 ; Ro. 2:7 ; Ro. 6:23 ; Tit. 3:7 ; 1 Jn. 2:25 et 1 Jn. 5:20.

\DicoEntry{VIGNE}\textit{}\newline
Arbre cultivé pour son fruit, la vigne est assimilée à la joie à cause du vin produit par le raisin et consommé dans le cadre de festivités. Le peuple d'Israël était la première vigne de Yahweh, mais elle ne porta pas de fruits. Le royaume de Dieu est aussi associé à la vigne : Dieu est le vigneron, Jésus-Christ est le cep et tous les enfants de Dieu sont les sarments. Tout sarment qui ne porte pas de fruits est jeté au feu, c'est-à-dire en enfer. Voir Es. 5:1-7 ; Mt. 21:33-43 et Jn. 15:1-8.

\DicoEntry{VOILE}\textit{, de l'hébreu « porokhet » : « rideau, voile » et du grec « peribolaion » : « une couverture, une enveloppe »}\newline
1. Etoffe de fin lin retors qui servait de séparation entre le lieu saint et le Saint des saints. Lorsque Jésus-Christ fut crucifié, ce voile se déchira en deux, de haut en bas, ouvrant ainsi l'accès au Saint des saints. Cet événement symbolisait que, par Jésus, tout homme pouvait accéder librement à la présence du Père. Voir Ex. 26:31-33 ; Lé. 16:11-19; Mt. 27:50-51 ; Hé. 9:7-8 et Hé. 10:19-20.
\\2. Pan de tissu utilisé pour se couvrir la tête dans certaines cultures. Paul expliqua que les longs cheveux étaient une gloire pour la femme et qu'ils faisaient office de voile naturel. Voir Ge. 24:65 et 1 Co. 11:15.
\\3. Au sens figuré, le voile symbolise l'intelligence obscurcie, le cœur non converti et le manque de révélation de la parole qui sont des barrières à la compréhension de la loi. Voir 2 Co. 3:14-16.

\DicoEntry{YHWH}\textit{}\newline
Aussi appelé tétragramme (mot de quatre lettres), nom avec lequel Dieu se révéla à Moïse lorsque ce dernier le rencontra pour la première fois à Horeb. Ce nom, prononcé Yahweh, signifie « Je suis celui qui suis » et souligne le caractère éternel de Dieu. Voir Ex. 3:1-14.

\DicoEntry{ZABULON}\textit{, de l'hébreu « Zebuwluwn » : « habitation »}\newline
Fils de Jacob et Léa, il devint l'ancêtre de la tribu de Zabulon. Voir Ge. 30:19-20 et No. 2:7.

\DicoEntry{ZACHARIE}\textit{, de l'hébreu « Zekaryah » : « Yahweh se souvient »}\newline
1. Fils de Jéroboam, roi d'Israël sur qui il régna uniquement six mois. Il fit ce qui est mal devant Yahweh et fut tué suite à une conspiration contre lui. Voir 2 R. 15:8-11.
\\2. Prophète et sacrificateur, fils de Bérékia et petit-fils d'Iddo. Avec le prophète Aggée, il assista Zorobabel, gouverneur de Juda, et Josué, souverain sacrificateur, dans la restauration du temple de Yahweh au retour de la captivité des Juifs. L'ensemble de ses prophéties se trouve dans le livre portant son nom. Voir Esd. 5:1-2 et Esd. 6:14-5.
\\3. Sacrificateur et père de Jean-Baptiste qu'il eut avec sa femme Elisabeth à un âge avancé. Voir Lu. 1:5.

\DicoEntry{ZELOTE}\textit{, du grec « zelotes » : « celui qui est zélé »}\newline
Patriotes juifs fervents défenseurs de la loi et des traditions ayant pour objectif de résister à l'invasion romaine. Simon, l'un des douze apôtres, en faisait partie. Voir Lu. 6:15 et Ac. 1:13.

\DicoEntry{ZOROBABEL}\textit{, de l'hébreu « Zerubbabel » : « rejeton de Babylone »}\newline
Fils de Schealthiel, gouverneur de Juda, il participa à la restauration du temple* de Yahweh après le retour de la captivité du peuple juif. Il figure dans la généalogie de Jésus. Voir Esd. 3:2 ; Esd. 5:2 ; Ag. 1:14 ; Mt. 1:13 et Lu. 3:27.
}
\end{multicols}
