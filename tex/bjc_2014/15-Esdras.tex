\ShortTitle{Esdras}\BookTitle{Esdras}\BFont
\noindent\hrulefill
{\footnotesize
\textit{
\bigskip
{\centering{}
\\(Ezrah)
\\Signifie : Secours
\\Thème : Edit de Cyrus et reconstruction du temple 
\\Auteur : Esdras
\\Date de rédaction : 5ème siècle av. J.-C.\\}
}
%\bigskip
\textit{
\\Conformément aux prophéties reçues par Esaïe et Jérémie, Yahweh toucha le cœur du roi Cyrus afin de renvoyer les fils d’Israël sur leur terre avec la mission de reconstruire le temple détruit quelques décennies auparavant. Ce livre montre comment Dieu ramena glorieusement son peuple à Jérusalem et retrace la reconstruction du temple ainsi que les épreuves ayant accompagné ce projet. Il traite des réformes sociales et religieuses mises en place dans le cadre d’un retour total à Yahweh.\bigskip
}
}
\par\nobreak\noindent\hrulefill
\begin{multicols}{2}
\Chap{1}
\TextTitle{[Publication de Cyrus]}
\VerseOne{}La première année de Cyrus\FTNT{538 av. J.-C.}, roi de Perse, afin que la parole de Yahweh prononcée par la bouche de Jérémie\FTNT{Jé. 25:12 ; 29:10 ; 33:7-10.} soit accomplie,  Yahweh réveilla l'esprit de Cyrus, roi de Perse, qui fit publier par écrit et de vive voix dans tout son royaume, en disant :
\VS{2}Ainsi parle Cyrus, roi de Perse : Yahweh, le Dieu des cieux, m'a donné tous les royaumes de la terre, et il m'a ordonné de lui bâtir une maison à Jérusalem, en Juda.
\VS{3}Qui d'entre vous est de son peuple, qui veut s'y employer ? Que son Dieu soit avec lui, qu'il monte à Jérusalem, en Juda, et qu'il rebâtisse la maison de Yahweh, le Dieu d'Israël ! C'est le Dieu qui est à Jérusalem.
\VS{4}Dans tout lieu où séjournent des restes du peuple,  les gens du lieu leur donneront de l’argent, de l’or, des biens, et du bétail, avec des offrandes volontaires pour la maison du Dieu qui est à Jérusalem.
\TextTitle{[Cyrus rend les ustensiles]}
\VS{5}Alors les chefs des familles de Juda et de Benjamin, les sacrificateurs et les Lévites, tous ceux dont Dieu réveilla l'esprit, se levèrent afin de monter pour rebâtir la maison de Yahweh à Jérusalem.
\VS{6}Tous ceux qui étaient autour d'eux les encouragèrent, leur fournissant des objets d'argent, d'or, des biens, du bétail, et des choses précieuses, outre toutes les offrandes volontaires.
\VS{7}Le roi Cyrus prit les ustensiles de la maison de Yahweh, que Nebucadnetsar avait emportés\FTNT{2 R. 24:13 ; 2 Ch 36:7.} de Jérusalem et mis dans la maison de son dieu.
\VS{8}Cyrus, roi de Perse, les fit sortir par Mithredath, le trésorier, qui les remit à Scheschbatsar, prince de Juda.
\VS{9}Et voici leur nombre : Trente bassins d'or, mille bassins d'argent, vingt-neuf couteaux,
\VS{10}trente coupes d'or, quatre cent dix coupes d'argent de second ordre, et d'autres ustensiles par milliers.
\VS{11}Tous les ustensiles d'or et d'argent étaient de cinq mille quatre cents. Scheschbatsar emporta le tout, lorsqu’on fit remonter de Babylone à Jérusalem ceux de la captivité.
\Chap{2}
\TextTitle{[Dénombrement des Israélites revenus de captivité]}
\VerseOne{}Voici ceux de la province qui revinrent de la captivité, d'entre ceux que Nebucadnetsar, roi de Babylone, avait transportés en exil à Babylone, et qui retournèrent à Jérusalem, et en Juda ; chacun dans sa ville\FTNT{Esd. 5:8 ; Né. 1:3 ; Né. 7:6.}.
\VS{2}Ils vinrent avec Zorobabel, Josué, Néhémie, Seraja, Reélaja, Mardochée, Bilschan, Mispar, Bigvaï, Rehum, Baana. Nombre des hommes du peuple d'Israël :
\VS{3}Les fils de Pareosch, deux mille cent soixante-douze\FTNT{Né. 7:8.} ;
\VS{4}les fils de Schephathia, trois cent soixante-douze ;
\VS{5}les fils d'Arach, sept cent soixante-quinze ;
\VS{6}les fils de Pachath-Moab, des fils de Josué, de Joab, deux mille huit cent douze ;
\VS{7}les fils d'Elam, mille deux cent cinquante-quatre ;
\VS{8}les fils de Zatthu, neuf cent quarante-cinq ;
\VS{9}les fils de Zaccaï, sept cent soixante ;
\VS{10}les fils de Bani, six cent quarante-deux\FTNT{Né. 7:15.} ;
\VS{11}les fils de Bébaï, six cent vingt-trois ;
\VS{12}les fils d'Hazgad, mille deux cent vingt-deux ;
\VS{13}les fils d'Adonikam, six cent soixante-six ;
\VS{14}les fils de Bigvaï, deux mille cinquante-six ;
\VS{15}les fils d’Adin, quatre cent cinquante-quatre ;
\VS{16}les fils d'Ather, de la famille d'Ezéchias, quatre-vingt-dix-huit ;
\VS{17}les fils de Betsaï, trois cent vingt-trois ;
\VS{18}les fils de Jora, cent douze ;
\VS{19}les fils de Haschum, deux cent vingt-trois ;
\VS{20}les fils de Guibbar, quatre-vingt-quinze ;
\VS{21}les fils de Bethléhem, cent vingt-trois ;
\VS{22}les gens de Netopha, cinquante-six ;
\VS{23}les gens d'Anathoth, cent vingt-huit ;
\VS{24}les fils d'Azmaveth, quarante-deux ;
\VS{25}les fils de Kirjath-Arim, de Kephira, et de Beéroth, sept cent quarante-trois ;
\VS{26}les fils de Rama et de Guéba, six cent vingt et un ;
\VS{27}les gens de Micmas, cent vingt-deux ;
\VS{28}les gens de Béthel et d’Aï, deux cent vingt-trois ;
\VS{29}les fils de Nebo, cinquante-deux ;
\VS{30}les fils de Magbisch, cent cinquante-six.
\VS{31}les fils d'un autre Elam, mille deux cent cinquante-quatre ;
\VS{32}les fils de Harim, trois cent vingt ;
\VS{33}les fils de Lod, de Hadid, et d'Ono, sept cent vingt-cinq ;
\VS{34}les fils de Jéricho, trois cent quarante-cinq ;
\VS{35}les fils de Senaa, trois mille six cent trente.
\TextTitle{[Dénombrement des sacrificateurs revenus de captivité]}
\VS{36}Des sacrificateurs : Les fils de Jedaeja, de la maison de Josué, neuf cent soixante-treize ;
\VS{37}les fils d'Immer, mille cinquante-deux ;
\VS{38}les fils de Paschhur, mille deux cent quarante-sept ;
\VS{39}les fils de Harim, mille dix-sept.
\TextTitle{[Dénombrement des Lévites revenus de captivité]}
\VS{40}Des Lévites : Les fils de Josué et de Kadmiel, d'entre les fils d’Hodavia, soixante-quatorze.
\VS{41}Des chantres : Les fils d'Asaph, cent vingt-huit.
\VS{42}Des fils des portiers : Les fils de Schallum, les fils d'Ather, les fils de Thalmon, les fils d’Akkub, les fils de Hathitha, les fils de Schobaï, en tout cent trente-neuf.
\VS{43}Des Néthiniens : Les fils de Tsicha, les fils de Hasupha, les fils de Tabbahoth\FTNT{Esd.8:17 ; Jos. 9:23.},
\VS{44}les fils de Kéros, les fils de Siaha, les fils de Padon,
\VS{45}les fils de Lebana, les fils de Hagaba, les fils d'Akkub,
\VS{46}les fils de Hagab, les fils de Schamlaï, les fils de Hanan,
\VS{47}les fils de Guiddel, les fils de Gachar, les fils de Reaja,
\VS{48}les fils de Retsin, les fils de Nekoda, les fils de Gazzam,
\VS{49}les fils d'Uzza, les fils de Paséach, les fils de Bésaï,
\VS{50}les fils d'Asna, les fils de Mehunim, les fils de Nephusim,
\VS{51}les fils de Bakbuk, les fils de Hakupha, les fils de Harhur,
\VS{52}les fils de Batsluth, les fils de Mehida, les fils de Harscha,
\VS{53}les fils de Barkos, les fils de Sisera, les fils de Thamach,
\VS{54}les fils de Netsiach, les fils de Hathipha.
\TextTitle{[Dénombrement des serviteurs de Salomon revenus de captivité]}
\VS{55}Des fils des serviteurs de Salomon : Les fils de Sothaï, les fils de Sophéreth, les fils de Peruda,
\VS{56}les fils de Jaala, les fils de Darkon, les fils de Guiddel,
\VS{57}les fils de Schephathia, les fils de Hatthil, les fils de Pokéreth-Hatsebaïm, les fils d'Ami.
\VS{58}Total des Néthiniens et des fils des serviteurs de Salomon : Trois cent quatre-vingt-douze.
\VS{59}Voici ceux qui montèrent de Thel-Mélach, de Thel-Harscha, de Kerub-Addan et qui ne purent pas faire connaître leur maison paternelle et leur race, pour prouver qu’ils étaient d'Israël :
\VS{60}Les fils de Delaja, les fils de Tobija, les fils de Nekoda, six cent cinquante-deux.
\TextTitle{[Certains sacrificateurs rejetés de la sacrificature]}
\VS{61}Des fils des sacrificateurs : Les fils de Habaja, les fils d'Hakkots, les fils de Barzillaï, qui avait pris pour femme une des filles de Barzillaï, le Galaadite, fut appelé de leur nom.
\VS{62}Ils cherchèrent leurs registres généalogiques, mais ils ne les trouvèrent point. C'est pourquoi ils furent rejetés pour ne pas souiller le sacerdoce,
\VS{63}et le gouverneur leur dit de ne pas manger des choses très saintes, en attendant qu'un sacrificateur ait consulté l'urim et le thummim.
\TextTitle{[Nombre total des Israélites revenus de captivité]}
\VS{64}L’assemblée tout entière était de quarante-deux mille trois cent soixante,
\VS{65}sans leurs serviteurs et leurs servantes, qui étaient sept mille trois cent trente-sept. Ils avaient deux cents chantres ou chanteuses.
\VS{66}Ils avaient sept cent trente-six chevaux, et deux cent quarante-cinq mulets,
\VS{67}quatre cent trente-cinq chameaux, et six mille sept cent vingt ânes.
\VS{68}Quelques-uns d'entre les chefs des pères, quand ils vinrent à la maison de Yahweh à Jérusalem, firent des offrandes volontaires pour la maison de Dieu, afin qu'on la rétablît sur son emplacement.
\VS{69}Ils donnèrent au trésor de l'ouvrage, selon leurs moyens, soixante et un mille drachmes d'or, et cinq mille mines d'argent, et cent tuniques de sacrificateurs.
\VS{70}Ainsi, les sacrificateurs, les Lévites, quelques-uns du peuple, les chantres, les portiers et les Néthiniens habitèrent dans leurs villes. Et tous ceux d'Israël dans leurs villes aussi.
\Chap{3}
\TextTitle{[Rétablissement de l'autel et des sacrifices]}
\VerseOne{}Le septième mois approcha, et les fils d'Israël étaient dans leurs villes. Le peuple s'assembla alors comme un seul homme à Jérusalem.
\VS{2}Alors\FTNT{Ag. 1:1 ; De.12:5-6.} Josué, fils de Jotsadak, avec ses frères les sacrificateurs, et Zorobabel, fils de Schealthiel, avec ses frères, se levèrent et bâtirent l'autel du Dieu d'Israël, pour y offrir des holocaustes, comme il est écrit dans la loi de Moïse, homme de Dieu.
\VS{3}Ils rétablirent l'autel de Dieu sur ses fondements, parce qu'ils avaient peur en eux-mêmes des peuples du pays, et ils y offrirent des holocaustes à Yahweh, les holocaustes du matin et du soir\FTNT{No. 28:3.}.
\VS{4}Ils célébrèrent aussi la fête des tabernacles, comme il est écrit, et ils offrirent des holocaustes, autant qu'il en fallait  chaque jour\FTNT{Lé. 23:34 ; No. 29:12.}.
\VS{5}Après cela, ils offrirent l'holocauste perpétuel, ceux des nouvelles lunes, de toutes les fêtes solennelles consacrées à Yahweh, et ceux de quiconque faisait des offrandes volontaires à Yahweh\FTNT{No. 28:11 ; Né.10:33.}.
\VS{6}Dès le premier jour du septième mois, ils commencèrent à offrir des holocaustes à Yahweh. Cependant, les fondements du temple de Yahweh n'étaient pas encore posés.
\VS{7}Ils donnèrent de l'argent aux tailleurs de pierres et aux charpentiers, et aussi de la nourriture, des boissons, de l'huile aux Sidoniens et aux Tyriens, afin qu'ils amènent du bois de cèdre du Liban par la mer de Japho, selon la permission que Cyrus, roi de Perse, leur en avait donnée.
\TextTitle{[Les fondements du temple posés]}
\VS{8}Et la deuxième année depuis leur arrivée à la maison de Dieu à Jérusalem, au deuxième mois, Zorobabel, fils de Schealthiel,  Josué, fils de Jotsadak, et le reste de leurs frères les sacrificateurs et les Lévites, et tous ceux qui étaient revenus de la captivité à Jérusalem, débutèrent l’œuvre et désignèrent des Lévites, depuis l'âge de vingt ans et au-dessus pour surveiller l'ouvrage de la maison de Yahweh.
\VS{9}Et Josué, avec ses fils et ses frères, Kadmiel, avec ses fils, fils de Juda, les fils de Hénadad, avec leurs fils et leurs frères les Lévites, se tenaient debout pour surveiller ceux qui faisaient l'ouvrage de la maison de Dieu.
\VS{10}Et lorsque ceux qui bâtissaient posèrent les fondements du temple de Yahweh, on fit assister les sacrificateurs revêtus de leurs habits, avec leurs trompettes, et les Lévites, fils d'Asaph, avec les cymbales, pour qu’ils célèbrent Yahweh, selon l’institution de David, roi d'Israël.
\VS{11}Et en louant et célébrant Yahweh, ils s'entre-répondaient : Il est bon, parce que sa miséricorde demeure à toujours sur Israël ! Et tout le peuple poussait de grands cris de joie en louant Yahweh, parce qu'on posait les fondements de la maison de Yahweh.
\VS{12}Mais plusieurs des sacrificateurs et des Lévites, et des chefs de familles âgés, qui avaient vu la première maison, pleuraient à grand bruit pendant qu'on posait sous leurs yeux les fondements de cette maison. Et beaucoup élevaient leur voix avec des cris de joie,
\VS{13}et le peuple ne pouvait distinguer le bruit des cris de joie  d'avec le bruit des pleurs du peuple, car le peuple poussait de grands cris de joie dont le son s’entendait de très loin.
\Chap{4}
\TextTitle{[Les ennemis de Juda et de Benjamin découragent le peuple de Juda]}
\VerseOne{}Les ennemis de Juda et de Benjamin entendirent que les fils de la captivité rebâtissaient un temple à Yahweh, le Dieu d'Israël.
\VS{2}Ils vinrent vers Zorobabel et vers les chefs des familles, et leur dirent : Nous bâtirons\FTNT{On ne doit jamais s’associer avec les impies pour bâtir l’œuvre du Seigneur. Satan essaie toujours de s’infiltrer dans les assemblées afin de nous éloigner de la vérité, c’est pour cela que nous devons faire preuve de discernement (2 Co. 6:14-16).} avec vous ; car nous invoquons votre Dieu comme vous ; et nous lui avons sacrifié depuis le temps d'Esar-Haddon, roi d'Assyrie, qui nous a fait monter ici.
\VS{3}Mais Zorobabel, Josué, et les autres chefs des familles d'Israël, leur répondirent : Il ne convient pas à vous de bâtir la maison de notre Dieu ; mais nous, qui sommes ici ensemble, nous la bâtirons à Yahweh, le Dieu d'Israël, comme nous l'a ordonné le roi Cyrus, roi de Perse\FTNT{Esd. 1:1,2,5.}.
\VS{4}Alors les gens du pays rendirent paresseuses les mains du peuple de Juda ; ils l’intimidèrent pour l'empêcher de bâtir,
\VS{5}ils avaient même engagé à prix d’argent des conseillers pour faire échouer leur projet, pendant toute la durée de vie de Cyrus, roi de Perse, jusqu'au règne de Darius, roi de Perse.
\VS{6}Et sous le règne d'Assuérus, au commencement de son règne, ils écrivirent une accusation contre les habitants de Juda et de Jérusalem.
\TextTitle{[Lettre envoyé à Artaxerxès]}
\VS{7}Et du temps d'Artaxerxès, Bischlam, Mithredath, Thabeel, et le reste de leurs collègues, écrivirent à Artaxerxès, roi de Perse. La lettre était écrite en caractères araméens, et elle était traduite en araméen.
\VS{8}Rehum, le gouverneur, et Schimschaï, le secrétaire, écrivirent au roi Artaxerxès la lettre suivante concernant Jérusalem :
\VS{9}Rehum, gouverneur, Schimschaï, secrétaire, et le reste de leurs collègues, ceux de Din, d'Arpharsathac, de Tharpel, d'Apharas, d'Erec, de Babylone, de Suse, de Déha, d'Elam,
\VS{10}et les autres peuples que le grand et illustre Osnappar a transportés et fait habiter dans la ville de Samarie et les autres régions au-delà du fleuve, à cette date.
\VS{11}Voici donc ici la copie de la lettre qu'ils envoyèrent au roi Artaxerxès : Tes serviteurs, les gens de ce côté du fleuve, à cette date.
\VS{12}Que le roi sache que les Juifs qui sont montés de chez lui et arrivés vers nous à Jérusalem rebâtissent la ville rebelle et méchante, et achèvent en finissant de poser et de réparer les fondements des murs.
\VS{13}Que le roi sache donc que si cette ville est rebâtie et si ses murs sont réparés, ils ne paieront plus de tribut, ni d’impôt, ni de droit de passage, et elle causera une grande nuisance aux revenus du roi.
\VS{14}Et parce que nous mangeons le sel du palais, il ne nous parait pas convenable de voir le roi déshonoré ; c'est pourquoi nous envoyons au roi ces informations.
\VS{15}Qu'on recherche dans le livre des mémoires de tes pères, et tu trouveras et tu apprendras dans ce livre des mémoires que cette ville est une ville rebelle, nuisible aux rois et aux provinces ; et qu’on s’y est livré à la révolte depuis toujours. Donc cette ville a été détruite à cause de cela.
\VS{16}Nous faisons donc savoir au roi que si cette ville est rebâtie et si ses murs sont relevés, il n'aura plus de possession de ce côté du fleuve.
\TextTitle{[Réponse du roi Artaxerxès]}
\VS{17}Et c'est ici le décret envoyé par le roi à Rehum, le gouverneur, à Schimschaï, le secrétaire, et au reste de leurs collègues demeurant à Samarie, et aux autres de l'autre côté du fleuve : Paix sur vous, à cette date.
\VS{18}La lettre que vous nous avez envoyée a été lue exactement en ma présence.
\VS{19}J'ai donné ordre de faire des recherches et l’on a trouvé  que depuis toujours cette ville s'est soulevée contre les rois, et qu'on s’y est livré à la sédition et à la révolte.
\VS{20}Il y eut aussi à Jérusalem des rois puissants, maîtres de tout le pays de l'autre côté du fleuve, et auxquels on payait  tribut, impôt et droit de passage\FTNT{2 S. 8:2,6 ; 1 R 4:21 ; 2 Ch. 17:11 ; 32:23.}.
\VS{21}A présent, donnez l’ordre de ne pas laisser continuer ces gens-là, afin que cette ville ne se rebâtisse point, jusqu’à ce que je l'ordonne par décret.
\VS{22}Gardez-vous de mettre en cela de la négligence, de peur que le mal  n’augmente au préjudice des rois.
\VS{23}Aussitôt que la copie de la lettre du roi Artaxerxès eut été lue en présence de Rehum, de Schimschaï,  le secrétaire, et de leurs collègues, ils allèrent en hâte à Jérusalem vers les Juifs, et ils les firent cesser leurs travaux avec violence et force.
\VS{24}Alors l’ouvrage de la maison de Dieu, à Jérusalem, cessa, et elle demeura dans cet état, jusqu'à la deuxième année du règne de Darius, roi de Perse\FTNT{Esd. 5:2}.
\Chap{5}
\TextTitle{[Aggée et Zacharie prophétisent]}
\VerseOne{}Aggée, le prophète, et Zacharie, fils d'Iddo, le prophète, prophétisèrent aux Juifs qui étaient en Juda et à Jérusalem, au nom du Dieu d'Israël, qui s’adressait à eux\FTNT{Ag. 1:4 ; Za. 1:1.}.
\VS{2}Alors Zorobabel, fils de Schealthiel, et Josué, fils de Jotsadak, se levèrent et commencèrent à rebâtir la maison de Dieu à Jérusalem. Et ils avaient avec eux les prophètes de Dieu qui les soutenaient\FTNT{Ag. 1:14 ; Esd. 6:14.}.
\VS{3}En ce temps-là, Thathnaï, gouverneur de ce côté du fleuve, et Schethar-Boznaï, et leurs collègues, vinrent à eux et leur parlèrent ainsi : Qui vous a donné l’ordre de rebâtir cette maison et de relever ces murs ?\FTNT{Esd. 5:9.}
\VS{4}Ils leur dirent alors : Quels sont les noms des hommes qui construisent cet édifice ?
\VS{5}Mais l’œil de Dieu était sur les anciens des Juifs. Et on ne les laissa continuer les travaux, pendant l’envoi d’un rapport à Darius, et jusqu'à la réception d’une lettre sur cet objet.
\TextTitle{[Thathnaï, Schethar-Boznaï et leurs collègues d'Apharsac écrivent à Darius]}
\VS{6}Copie de la lettre envoyée au roi Darius par Thathnaï, gouverneur de ce côté du fleuve, Schethar-Boznaï, et leurs collègues d'Apharsac, de l’autre côté du fleuve.
\VS{7}Ils lui envoyèrent un rapport ainsi écrit : Paix parfaite soit au roi Darius !
\VS{8}Que le roi sache que nous sommes allés dans la province de Juda, vers la maison du grand Dieu. Elle se bâtit avec des pierres de taille, et le bois se pose dans les murs ; ce travail se réalise complètement et prospère entre leurs mains\FTNT{Esd. 2:1.}.
\VS{9}Nous avons interrogé ces anciens, et nous leur avons parlé ainsi : Qui vous a donné l'autorisation de rebâtir cette maison et de finir ces murs ?\FTNT{Esd. 5:3.}
\VS{10}Nous leur avons aussi demandé leurs noms pour te les faire connaître, et nous avons mis par écrit les noms des hommes à leur tête.
\VS{11}Et ils nous ont répondu de cette manière, disant : Nous sommes les serviteurs du Dieu des cieux et de la terre, et nous rebâtissons la maison qui avait été bâtie il y a de nombreuses années ; un grand roi d'Israël l’avait bâtie et finie.
\VS{12}Mais après que nos pères eurent provoqué la colère du Dieu des cieux, il les livra entre les mains de Nebucadnetsar\FTNT{Voir 2 R. 24 et 25.}, roi de Babylone, Chaldéen, qui détruisit cette maison et qui emmena le peuple en exil à Babylone\FTNT{2 Ch. 36:7}.
\VS{13}Mais la première année de Cyrus, roi de Babylone, le roi Cyrus prit un décret pour rebâtir cette maison de Dieu\FTNT{Esd. 1:1-2.}.
\VS{14}Et même le roi Cyrus ôta du temple de Babylone les ustensiles d'or et d'argent de la maison de Dieu, que Nebucadnetsar avait sortis du temple qui était à Jérusalem et transportés dans le temple de Babylone, et il les fit remettre au nommé Scheschbatsar, qu’il établit gouverneur\FTNT{Esd. 1:8.},
\VS{15}et il lui dit : Prends ces ustensiles, et va les déposer dans le temple de Jérusalem ; et que la maison de Dieu soit rebâtie sur sa place.
\VS{16}Alors ce Scheschbatsar est venu, et il a posé les fondements de la maison de Dieu à Jérusalem ; et depuis ce temps-là jusqu'à présent, on la bâtit, et elle n'est point encore achevée.
\VS{17}Maintenant, s'il semble bon au roi, que l’on fasse des recherches dans la maison des trésors du roi à Babylone, pour voir s'il est vrai qu'il y a eu un ordre donné par Cyrus de rebâtir cette  maison de Dieu à Jérusalem. Puis, que le roi nous transmette sa volonté sur cet objet.
\Chap{6}
\TextTitle{[Darius confirme l'édit de Cyrus]}
\VerseOne{}Alors le roi Darius donna un ordre de faire des recherches dans la maison des livres où l'on déposait les  trésors à Babylone.
\VS{2}Et l’on trouva à Achmetha, dans un coffre, capitale de la province de Médie, un rouleau à l’intérieur duquel était écrit  le mémoire suivant :
\VS{3}La première année du roi Cyrus, le roi Cyrus prit un décret quant à la maison de Dieu à Jérusalem : Que cette maison soit rebâtie, afin d’être un lieu où l'on offre des sacrifices, et que ses fondements soient solides pour porter sa charge. La hauteur sera de soixante coudées, et la longueur de soixante coudées,
\VS{4}trois rangées de pierres de taille et une rangée de bois neuf.  La dépense sera payée par la maison du roi.
\VS{5}Aussi, les ustensiles d'or et d'argent de la maison de Dieu, que Nebucadnetsar avait enlevés du temple de Jérusalem et apportés à Babylone, seront remis et apportés dans le temple de Jérusalem, à leur place, et déposés dans la maison de Dieu.
\VS{6}Maintenant, Thathnaï, gouverneur de l'autre côté du fleuve, Schethar-Boznaï, et vos collègues d'Apharsac de l'autre côté du fleuve, tenez-vous loin de ce lieu.
\VS{7}Laissez le travail de cette maison de Dieu ; que le gouverneur des Juifs et les anciens des Juifs rebâtissent cette maison de Dieu à sa place.
\VS{8}En raison de ce décret pris, ce que vous aurez à exécuter, avec les anciens de ces Juifs pour rebâtir cette maison de Dieu : Sur les finances du roi provenant du tribut de l’autre côté du fleuve, les frais seront complètement payés à ces hommes, afin qu'il n'y ait pas d'interruption.
\VS{9}Et ce qui sera nécessaire pour les holocaustes du Dieu des cieux, veaux, béliers et agneaux, blé, sel, vin et huile, seront livrés, sur leur demande, aux sacrificateurs de Jérusalem, jour après jour, sans négligence,
\VS{10}afin qu'ils offrent des sacrifices de bonne odeur au Dieu des cieux et qu'ils prient pour la vie du roi et de ses fils.
\VS{11}Et voici l’ordre que je donne touchant quiconque changera cette parole : On arrachera de sa maison une pièce de bois, on la dressera, afin qu'il y soit exterminé, et l’on fera de sa maison un tas de déchets\FTNT{2 R. 10:27 ; Ez. 6:11 ; Da. 3:29.}.
\VS{12}Et que Dieu, qui fait résider en ce lieu son nom, renverse tout roi et tout peuple qui étendrait sa main pour changer et détruire cette maison de Dieu à Jérusalem ! Moi, Darius, j’ai donné cet ordre. Qu'il soit donc exécuté complètement.
\TextTitle{[Achèvement et dédicace de la maison de Dieu]}
\VS{13}Alors Thathnaï, gouverneur de l'autre côté du fleuve,  Schethar-Boznaï, et leurs collègues, firent exécuter ainsi complètement ce que le roi Darius leur envoya.
\VS{14}Et les anciens des Juifs bâtirent avec succès, selon les prophéties d'Aggée, le prophète, et de Zacharie, fils d’Iddo ; ils bâtirent et finirent, d'après l'ordre du Dieu d'Israël, et d'après l'ordre de Cyrus, de Darius, et d'Artaxerxès, roi de Perse.
\VS{15}Cette maison fut achevée le troisième jour du mois d'Adar, dans la sixième année du règne du roi Darius.
\VS{16}Les fils d'Israël, les sacrificateurs, les Lévites, et le reste des fils de la captivité, célébrèrent la dédicace de cette maison de Dieu avec joie.
\VS{17}Ils offrirent pour la dédicace de cette maison de Dieu, cent taureaux, deux cents béliers, quatre cents agneaux, et douze boucs comme victimes expiatoires pour tout Israël, selon le nombre des tribus d'Israël.
\VS{18}Ils établirent les sacrificateurs selon leurs classes et les Lévites selon leurs divisions, pour le service de Dieu à Jérusalem,  selon ce qui est écrit dans le livre de Moïse\FTNT{No. 3:6,32 ; No. 8:11}.
\TextTitle{[Rétablissement de la Pâque]}
\VS{19}Puis les fils de la captivité célébrèrent la Pâque le quatorzième jour du premier mois\FTNT{Lé. 23:5 ; No. 28:16 ; De. 16:2.}.
\VS{20}Les sacrificateurs et les Lévites s'étaient purifiés comme un seul homme, tous étaient purs ; c'est pourquoi ils immolèrent la Pâque pour tous les fils de la captivité, pour leurs frères les sacrificateurs, et pour eux-mêmes\FTNT{2 Ch. 30:15,17,21}.
\VS{21}Les fils d'Israël revenus de la captivité mangèrent la Pâque, avec tous ceux qui s'étaient séparés de l’impureté des nations du pays pour chercher Yahweh, le Dieu d'Israël.
\VS{22}Ils célébrèrent avec joie la fête des pains sans levain pendant sept jours, car Yahweh les avait réjouis en disposant  le cœur du roi d'Assyrie à fortifier leurs mains dans l’œuvre de la maison de Dieu, du Dieu d'Israël.
\Chap{7}
\TextTitle{[Voyage d'Esdras jusqu'à Jérusalem]}
\VerseOne{}Après ces choses, sous le règne d'Artaxerxès, roi de Perse, Esdras, fils de Seraja, fils d'Azaria, fils de Hilkija\FTNT{Esd. 6:14.},
\VS{2}fils de Schallum, fils de Tsadok, fils d'Achithub,
\VS{3}fils d'Amaria, fils d'Azaria, fils de Merajoth,
\VS{4}fils de Zerachja, fils d'Uzzi, fils de Bukki,
\VS{5}fils d'Abischua, fils de Phinées, fils d'Eléazar, fils d'Aaron, souverain sacrificateur.
\VS{6}Esdras monta de Babylone : C’était un scribe bien exercé dans la loi de Moïse, donnée par Yahweh, le Dieu d'Israël. Et comme la main de Yahweh, son Dieu, était sur lui, le roi lui accorda toute sa requête\FTNT{Vers. 9,28.}.
\VS{7}Des fils d'Israël, des sacrificateurs, des Lévites, des chantres, des portiers, et des Néthiniens, montèrent à Jérusalem, la septième année du roi Artaxerxès.
\VS{8}Il entra à Jérusalem le cinquième mois de la septième année du roi ;
\VS{9}il était parti de Babylone au premier jour du premier mois, et il entra à Jérusalem au premier jour du cinquième mois, selon que la main de son Dieu était bonne sur lui.
\VS{10}Car Esdras avait disposé son cœur à étudier la loi de Yahweh, à l’observer et à enseigner les lois et les ordonnances parmi le peuple d'Israël.
\TextTitle{[Lettre d'Artaxerxès à Esdras]}
\VS{11}Voici la copie de la lettre que le roi Artaxerxès donna à Esdras, sacrificateur et scribe, enseignant les paroles des commandements de Yahweh et ses ordonnances concernant Israël :
\VS{12}Artaxerxès, roi des rois, à Esdras, sacrificateur et scribe de la loi du Dieu des cieux, à cette date.
\VS{13}J’ai donné ordre de laisser aller tous ceux de mon royaume qui sont du peuple d'Israël, de ses sacrificateurs et Lévites, qui se présenteront volontairement pour aller avec toi à Jérusalem.
\VS{14}Tu es envoyé de la part du roi, et de ses sept conseillers, pour inspecter Juda et Jérusalem touchant la loi de ton Dieu, laquelle est entre tes mains,
\VS{15}et pour porter l'argent et l'or que le roi et ses conseillers ont offert volontairement au Dieu d'Israël, dont la demeure est à Jérusalem\FTNT{Esd. 8:24.},
\VS{16}tout l'argent et l'or que tu trouveras dans toute la province de Babylone, avec les offrandes volontaires du peuple et des sacrificateurs, qu'ils feront volontairement à la maison de leur Dieu à Jérusalem.
\VS{17} C'est pourquoi tu achèteras avec cet argent des taureaux, des béliers, des agneaux, avec leurs offrandes et leurs libations, et tu les offriras sur l'autel de la maison de votre Dieu à Jérusalem.
\VS{18}Vous ferez, selon la volonté de votre Dieu, ce qu'il te semblera bon à toi et à tes frères de faire du reste de l'argent et de l'or.
\VS{19}Et pour ce qui est des ustensiles qui te sont remis pour le service de la maison de ton Dieu, déposes-les en présence du Dieu de Jérusalem.
\VS{20}Quand au reste de ce qui sera nécessaire pour la maison de ton Dieu, autant qu'il t'en faudra employer, tu le prendras de la maison des trésors du roi.
\VS{21}Moi, le roi Artaxerxès, je donne l’ordre à tous les trésoriers qui sont de l'autre côté du fleuve de livrer exactement à Esdras, sacrificateur et scribe de la loi du Dieu des cieux, tout ce qu’il vous demandera,
\VS{22}jusqu'à cent talents d'argent, cent cors de froment, cent baths de vin, cent baths d'huile, et du sel sans nombre.
\VS{23}Que tout ce qui est ordonné par le Dieu des cieux se fasse exactement pour la maison du Dieu des cieux, afin que sa colère ne soit pas sur le royaume, sur le roi et sur ses fils.
\VS{24}Nous vous faisons savoir qu'on ne pourra imposer ni tribut, ni impôt, ni droit de passage sur aucun des sacrificateurs, des Lévites, des chantres, des portiers, des  Néthiniens, et des serviteurs de cette maison de Dieu.
\VS{25}Et toi, Esdras, établis des magistrats et des juges selon la sagesse de ton Dieu que tu possèdes, afin qu'ils rendent justice à tout ce peuple de l'autre côté du fleuve, à tous ceux qui connaissent les lois de ton Dieu ; afin que vous enseigniez celui qui ne les connaît point.
\VS{26}Et tous ceux qui n'observeront point la loi de ton Dieu et la loi du roi seront aussitôt jugés, soit à la mort, soit au bannissement, soit à une amende pécuniaire, ou à l'emprisonnement.
\VS{27}Béni soit Yahweh, le Dieu de nos pères, qui a mis cela au cœur du roi, pour honorer la maison de Yahweh, qui est à Jérusalem ;
\VS{28}et qui a fait que j'ai trouvé grâce devant le  roi, devant ses conseillers, et devant tous les puissants chefs ! Fortifié par la main de Yahweh, mon Dieu, qui était sur moi, j'ai rassemblé les chefs d'Israël, afin qu'ils montent avec moi.
\Chap{8}
\TextTitle{[Dénombrement de ceux qui montèrent avec Esdras]}
\VerseOne{}Voici les chefs des pères, avec le dénombrement fait selon les généalogies de ceux qui montèrent avec moi de Babylone, pendant le règne du roi Artaxerxès\FTNT{1 Ch. 4:33.}.
\VS{2}Des fils de Phinées, Guerschom ; des fils d'Ithamar, Daniel ; des fils de David, Hattusch ;
\VS{3}des fils de Schecania ; des fils de Pareosch, Zacharie, et avec lui, en faisant le dénombrement par leur généalogie selon les hommes, cent cinquante hommes ;
\VS{4}des fils de Pachat Moab, Eljoénaï, fils de Zerachja, et avec lui deux cents hommes;
\VS{5}des fils de Schecania, le fils de Jachaziel, et avec lui trois cents hommes;
\VS{6}des fils d'Adin, Ebed, fils de Jonathan, et avec lui cinquante hommes ;
\VS{7}des fils d'Elam, Esaïe, fils d'Athalia, et avec lui soixante-dix hommes;
\VS{8}des fils de Schephathia, Zebadia, fils de Micaël, et avec lui quatre-vingts hommes ;
\VS{9}des fils de Joab, Abdias, fils de Jehiel, et avec lui deux cent dix-huit hommes ;
\VS{10}des fils de Schelomith, le fils de Josiphia, et avec lui cent soixante hommes ;
\VS{11}des fils de Bébaï, Zacharie, fils de Bébaï, et avec lui vingt-huit hommes ;
\VS{12}des fils d'Azgad, Jochanan, fils d'Hakkathan, et avec lui cent-dix hommes ;
\VS{13}des fils d'Adonikam, les derniers, dont voici les noms: Eliphélet, Jeïel, et Schemaeja, et avec eux soixante hommes ;
\VS{14}des fils de Bigvaï, Uthaï, Zabbud, et avec eux soixante-dix hommes.
\VS{15}Je les rassemblai près du fleuve qui coule vers Ahava, et nous campâmes là trois jours. Puis je portai mon attention sur  le peuple et les sacrificateurs, et je n'y trouvai aucun des fils de Lévi.
\VS{16}Alors j'envoyai d'entre les chefs Eliézer, Ariel, Schemaeja, Elnathan, Jarib, Elnathan, Nathan, Zacharie et Meschullam, avec les docteurs Jojarib et Elnathan.
\VS{17}Je leur donnai des ordres pour le chef Iddo, demeurant à Casiphia, et je mis dans leur bouche les paroles qu'ils devaient dire à Iddo et à ses frères les Néthiniens, qui étaient à Casiphia, afin qu'ils nous amènent des serviteurs pour la maison de notre Dieu\FTNT{Esd. 2:43.}.
\VS{18}Et comme la bonne main de notre Dieu était sur nous, ils nous amenèrent Schérébia, un homme intelligent, d'entre les fils de Machli, fils de Lévi, fils d'Israël, et avec ses fils et ses frères, au nombre dix-huit\FTNT{Esd. 7:6,9,28.} ;
\VS{19}Haschabia, et avec lui Esaïe, d'entre les fils de Merari, ses frères, et leurs fils, au nombre vingt ;
\VS{20}et des Néthiniens, que David et les chefs du peuple avaient assignés pour le service des Lévites, deux cent vingt Néthiniens, tous désignés par leurs noms\FTNT{Esd. 2:43,58.}.
\TextTitle{[Esdras publie un jeûne pour obtenir la protection de Dieu]}
\VS{21}Et je publiai là un jeûne près de la rivière d'Ahava, afin de nous humilier devant notre Dieu, le priant de nous donner un heureux voyage, pour nos enfants, et pour tous nos biens.
\VS{22}Car j'aurais eu honte de demander au roi une armée et des cavaliers pour nous soutenir contre des ennemis pendant le chemin ; car nous avions dit au roi : La main de notre Dieu est favorable sur tous ceux qui le cherchent ; mais sa force et sa colère sont contre ceux qui l'abandonnent.
\VS{23}Nous jeûnâmes donc, et nous cherchâmes notre Dieu à cause de cela. Et il se laissa fléchir par nos prières.
\TextTitle{[Trésors remis par Esdras entre les mains de douze sacrificateurs]}
\VS{24}Alors je mis à part douze chefs des sacrificateurs, Schérébia, Haschabia, et dix de leurs frères.
\VS{25}Je pesai l'argent, l'or et les ustensiles donnés en offrandes pour la maison de notre Dieu par le roi, ses conseillers, ses chefs, et tous ceux d'Israël qu'on avait trouvés\FTNT{Esd. 7:14,15.}.
\VS{26}Je pesai donc, et je remis entre leurs mains six cent cinquante talents d'argent, des ustensiles d'argent pesant cent talents, cent talents d'or,
\VS{27}vingt coupes d'or valant mille drachmes, et deux ustensiles d’un bel airain poli, aussi précieux que de l'or.
\VS{28}Et je leur dis : Vous êtes consacrés à Yahweh ; et les ustensiles sont sanctifiés, et cet argent et cet or sont une offrande volontaire faite à Yahweh, le Dieu de vos pères.
\VS{29}Soyez vigilants et gardez-les, jusqu'à ce que vous les pesiez devant les chefs des sacrificateurs et les Lévites, et devant les chefs des pères d'Israël, à Jérusalem, dans les chambres de la maison de Yahweh.
\VS{30}Les sacrificateurs et les Lévites reçurent le poids de l'argent, de l'or, et des ustensiles, pour les apporter à Jérusalem, dans la maison de notre Dieu.
\TextTitle{[Esdras arrive à Jérusalem]}
\VS{31}Nous partîmes du fleuve d'Ahava pour aller à Jérusalem, le douzième jour du premier mois. La main de notre Dieu fut sur nous et nous délivra de la main des ennemis et des  embûches sur le chemin.
\VS{32}Puis nous arrivâmes à Jérusalem, et nous nous y reposâmes trois jours.
\VS{33}Le quatrième jour, nous pesâmes l'argent, l'or, et les ustensiles dans la maison de notre Dieu, et nous les remîmes à Merémoth, fils d'Urie, le sacrificateur - il était avec Eléazar, fils de Phinées, et avec eux les Lévites Jozabad, fils de Josué, et Noadia, fils de Binnuï-
\VS{34} selon tout le nombre et le poids de toutes ces choses, et tout le poids fut mis alors par écrit.
\VS{35}Et les fils de la captivité revenus de l’exil offrirent en holocauste au Dieu d'Israël douze taureaux, quatre-vingt-seize béliers, soixante-dix-sept agneaux, et douze boucs comme victimes expiatoires pour tout Israël, le tout en holocauste à Yahweh.
\VS{36}Ils transmirent les ordres du roi entre les mains des satrapes du roi et des gouverneurs qui étaient de ce côté du fleuve, lesquels favorisèrent le peuple et la maison de Dieu.
\Chap{9}
\TextTitle{[La désobéissance]}
\VerseOne{}Après que ces choses furent terminées, les chefs du peuple s'approchèrent de moi, en disant : Le peuple d'Israël,  les sacrificateurs et les Lévites ne se sont point séparés des peuples de ces pays, quant à leurs abominations, celles des Cananéens, des Héthiens, des Phéréziens, des Jébusiens, des Ammonites, des Moabites, des Egyptiens, et des Amoréens.
\VS{2}Car ils ont pris de leurs filles pour eux et pour leurs fils, et ont mêlé la semence sainte avec les peuples de ces pays ; et des chefs et des magistrats ont été les premiers à commettre ce péché\FTNT{Né. 13:3.}.
\VS{3}Lorsque j'entendis cela, je déchirai mes vêtements et mon manteau, j'arrachai les cheveux de ma tête et ma barbe, et je m'assis tout épouvanté.
\VS{4}Et tous ceux qui tremblaient aux paroles du Dieu d'Israël, s'assemblèrent auprès de moi, à cause de l’infidélité de ceux de la captivité ; et je demeurai assis tout épouvanté jusqu'à l'offrande du soir.
\TextTitle{[Prière et confession d'Esdras]}
\VS{5}Et au temps de l'offrande du soir, je me levai du sein de mon affliction, et ayant mes vêtements et mon manteau déchirés, je me mis à genoux, et j'étendis mes mains vers Yahweh, mon Dieu,
\VS{6}et je dis : Mon Dieu ! J’ai honte, et je suis trop confus, ô mon Dieu, pour lever ma face vers toi ; car nos iniquités se sont multipliées au-dessus de nos têtes, et notre péché s'est élevé jusqu’aux cieux.
\VS{7}Depuis les jours de nos pères jusqu'à ce jour, nous sommes grandement coupables, et c’est à cause de nos iniquités que nous avons été livrés, nous, nos rois et nos sacrificateurs entre les mains des rois des pays, à l'épée, à la captivité, au pillage, et à la honte, comme il paraît aujourd'hui.
\VS{8}Et cependant Yahweh, notre Dieu, nous a maintenant fait grâce, en épargnant un reste, et il nous a donné un clou dans son saint lieu, afin d'éclaircir nos yeux et nous donner un peu de répit dans notre servitude\FTNT{Es. 22:23.}.
\VS{9}Car nous sommes esclaves, mais notre Dieu ne nous a point abandonnés dans notre servitude. Il a incliné la bienveillance des rois de Perse pour nous accorder de préserver nos vies afin que nous puissions relever la maison de notre Dieu, et rétablir ces lieux en ruines, et pour nous donner une clôture en Juda et à Jérusalem.
\VS{10}Mais maintenant, ô notre Dieu ! Que dirons-nous après ces choses ? Car nous avons abandonné tes commandements,
\VS{11}que tu as ordonnés par tes serviteurs les prophètes, en disant : Le pays dans lequel vous entrez pour le posséder est un pays souillé par les impuretés des peuples de ces pays, à cause des abominations dont ils l'ont rempli d’un bout à l'autre par leurs impuretés\FTNT{Lé. 18:25-27.};
\VS{12}maintenant donc, ne donnez point vos filles à leurs fils, et ne prenez point leurs filles pour vos fils, ne cherchez jamais ni leur bonheur, ni leur paix, ainsi vous deviendrez forts,  vous mangerez les meilleurs productions du pays, et vous le laisserez hériter à vos fils pour toujours\FTNT{De. 7:3.}.
\VS{13}Après toutes les choses qui nous sont arrivées à cause de nos mauvaises actions et des grandes offenses que nous avons commises - quoi que tu ne nous aies pas, ô notre Dieu, punis en proportion de nos péchés et maintenant que  tu nous as conservé ces réchappés ; 
\VS{14}retournerions-nous à violer tes commandements, et à faire alliance avec ces peuples abominables ? Ne serais-tu pas en colère contre nous, jusqu'à nous exterminer, sans aucun reste ni aucun réchappé ?
\VS{15}Yahweh, Dieu d'Israël ! Tu es juste, car nous sommes aujourd'hui un reste de réchappés. Voici, nous sommes devant toi avec nos fautes, ne pouvant subsister à cause d’elles devant ta face.
\Chap{10}
\TextTitle{[Confession et séparation]}
\VerseOne{}Pendant qu’Esdras priait et faisait cette confession, pleurant et étant prosterné à terre devant la maison de Dieu, une grande multitude d'hommes, de femmes, et d’enfants d'Israël, s'assembla auprès de lui ; et le peuple se lamenta abondamment par des pleurs.
\VS{2}Alors Schecania, fils de Jehiel, d'entre les fils d’Elam, prit la parole, et dit à Esdras : Nous avons péché contre notre Dieu, en nous mariant avec des femmes étrangères d'entre les peuples de ce pays. Mais Israël ne reste pas pour cela sans espérance\FTNT{De. 7:22,23.}.
\VS{3}Faisons maintenant une alliance avec notre Dieu pour le renvoi de toutes ces femmes et de leurs enfants, selon le conseil de mon seigneur et de ceux qui tremblent devant les commandements de notre Dieu. Et qu'il en soit fait selon la loi\FTNT{Esd. 9:4 ; Mal. 3:16.}.
\VS{4}Lève-toi, car cette affaire te regarde. Nous serons avec toi. Prends donc courage et agis.
\VS{5}Esdras se leva, et il fit jurer aux chefs des sacrificateurs, des Lévites, et de tout Israël, de faire selon cette parole. Et ils le jurèrent.
\VS{6}Puis Esdras se retira de devant la maison de Dieu, et s'en alla dans la chambre de Jochanan, fils d'Eliaschib ; et quand il y fut entré, il ne mangea point de pain, ne but point d'eau, parce qu'il se lamentait à cause du péché de ceux de la captivité.
\VS{7}Alors on publia dans le pays de Juda et à Jérusalem que tous ceux qui étaient retournés de la captivité aient à s'assembler à Jérusalem,
\VS{8}et que quiconque ne s'y rendrait pas dans trois jours, selon l'avis des chefs et des anciens, aurait tous ses biens complètement détruits, et que lui-même serait séparé de l'assemblée de ceux de la captivité.
\VS{9}Ainsi tous ceux de Juda et de Benjamin s'assemblèrent à Jérusalem dans les trois jours. C’était le vingtième jour du neuvième mois. Tout le peuple se tenait sur la place de la maison de Dieu, tremblant au sujet de cette affaire et à cause des pluies\FTNT{1 S. 12:18.}.
\VS{10}Esdras, le sacrificateur, se leva et leur dit : Vous avez péché en vous mariant avec des femmes étrangères, de sorte que vous avez augmenté la culpabilité d'Israël\FTNT{De. 7:3.}.
\VS{11}Prononcez maintenant votre confession à Yahweh, le Dieu de vos pères, et faites sa volonté ! Séparez-vous des peuples du pays et des femmes étrangères.
\VS{12}Et toute l'assemblée répondit à haute voix : A nous de faire ce que tu as dit !
\VS{13}Mais le peuple est nombreux, le temps est pluvieux, et il n'y a pas moyen de se tenir dehors ; d’ailleurs, ce n’est pas l’affaire d’un jour ou de deux, car il y en a beaucoup parmi nous qui ont péché dans cette affaire.
\VS{14}Que tous nos chefs se présentent donc devant toute l'assemblée, et que tous ceux qui sont dans nos villes, et qui se sont mariés avec des femmes étrangères, viennent à un temps fixé, et que les anciens de chaque ville et ses juges soient avec eux, jusqu'à ce que nous détournions de nous l'ardente colère de notre Dieu à ce sujet.
\VS{15}Il n'y eut que Jonathan, fils d'Asaël, et Jachzia, fils de Thikva, qui s'opposèrent a cet avis ; et Meschullam et Schabthaï, Lévites, les appuyèrent ;
\VS{16}mais ceux qui étaient retournés de la captivité s’y conformèrent. On choisit Esdras, le sacrificateur, et des chefs de famille selon leurs maisons paternelles, tous désignés par leurs noms ; ils siégèrent le premier jour du dixième mois, pour suivre cette affaire.
\VS{17}Le premier jour du premier mois, ils en finirent avec tous les hommes qui s’étaient mariés à des femmes étrangères.
\VS{18}Parmi les fils des sacrificateurs qui s’étaient mariés à des femmes étrangères, il se trouva d'entre les fils de Josué, fils de Jotsadak et de ses frères, Maaséja, Eliézer, Jarib et Guedalia,
\VS{19}qui, en donnant leurs mains, renvoyèrent leurs femmes ; et offrirent un bélier comme sacrifice de culpabilité ;
\VS{20}des fils d'Immer, Hanani et Zebadia ;
\VS{21}des fils de Harim, Maaséja, Elie, Schemaeja, Jehiel et Ozias ;
\VS{22}des fils de Paschhur, Eljoénaï, Maaséja, Ismaël, Nethaneel, Jozabad et Eleasa.
\VS{23}Parmi les Lévites : Jozabad, Schimeï, Kélaja (ou Kelitha) Pethachja, Juda et Eliézer.
\VS{24}Parmi les chantres : Eliaschib. Et des portiers : Schallum, Thélem et Uri.
\VS{25}Parmi ceux d'Israël : Des fils de Pareosch, Ramia, Jizzija, Malkija, Mijamin, Eléazar, Malkija et Benaja ;
\VS{26}des fils d’Elam, Matthania, Zacharie, Jehiel, Abdi, Jérémoth et Elie ;
\VS{27}des fils de Zatthu, Eljoénaï, Eliaschib, Matthania, Jérémoth, Zabad et Aziza ;
\VS{28}des fils de Bébaï, Jochanan, Hanania, Zabbaï et Athlaï ;
\VS{29}des fils de Bani, Meschullam, Malluc, Adaja, Jaschub, Scheal et Ramoth ;
\VS{30}des fils de Pachath-Moab, Adna, Kelal, Benaja, Maaséja, Matthania, Betsaleel, Binnuï et Manassé ;
\VS{31}des fils de Harim, Eliézer, Jischija, Malkija, Schemaeja, Siméon,
\VS{32}Benjamin, Malluc et Schemaria ;
\VS{33}des fils de Haschum, Matthnaï, Matthattha, Zabad, Eliphéleth, Jerémaï, Manassé et Schimeï ;
\VS{34}des fils de Bani, Maadaï, Amram, Uel,
\VS{35}Benaja, Bédia, Keluhu,
\VS{36}Vania, Merémoth, Eliaschib,
\VS{37}Matthania, Matthnaï, Jaasaï,
\VS{38}Bani, Binnuï, Schimeï,
\VS{39}Schélémia, Nathan, Adaja,
\VS{40}Macnadbaï, Schaschaï, Scharaï,
\VS{41}Azareel, Schélémia, Schemaria,
\VS{42}Schallum, Amaria et Joseph ;
\VS{43}des fils de Nebo, Jeïel, Matthithia, Zabad, Zebina, Jaddaï, Joël et Benaja.
\VS{44}Tous ceux-là avaient pris des femmes étrangères ; et  quelques-uns avaient eu des fils avec ces femmes-là.
\PPE{}
\end{multicols}
