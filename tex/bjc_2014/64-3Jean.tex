\ShortTitle{3 Jn.}\BookTitle{3 Jean}\BFont
\noindent\hrulefill
{\footnotesize
\textit{
\bigskip
{\centering{}
\\Auteur~: Jean
\\(Gr.~: Ioannes / Origine héb.~: Yohanan)
\\Signification~: Yahweh a fait grâce
\\Thème~: Sincérité, hospitalité et caractère du chrétien
\\Date de rédaction~: Env. 85 ap. J.-C.\\}
}
\textit{
\\Cette épître était destinée à Gaïus, l'un des responsables d'une église d'Asie Mineure, dont Jean loue la piété et la générosité. Il l'avertit de l'orgueil et des agissements de Diotrèphe qui étaient contraires à la Parole, mais souligne le bon témoignage de Démétrius.\bigskip
}
}
\par\nobreak\noindent\hrulefill
\begin{multicols}{2}
\Chap{1}
\TextTitle{Introduction}
\VerseOne{}L'ancien, à Gaïus le bien-aimé, que j'aime dans la vérité.
\VS{2}Bien-aimé, je souhaite que tu prospères\FTNT{La prospérité dont il est question dans ce passage n'a rien à voir avec l'évangile de prospérité qui met l'accent sur la richesse matérielle. Le mot grec «~euodoo~» signifie «~concevoir un voyage prospère et diligent~», «~mener par une voie directe et facile~», «~prospérer~», «~être heureux~».} en toutes choses, et que tu sois en bonne santé, comme ton âme est en prospérité.
\VS{3}Car j'ai été fort réjoui, quand les frères sont venus et ont rendu témoignage de ta sincérité, et comment tu marches dans la vérité.
\VS{4}Je n'ai pas de plus grande joie que d'apprendre que mes enfants marchent dans la vérité.
\TextTitle{L'hospitalité}
\VS{5}Bien-aimé, tu agis fidèlement dans tout ce que tu fais envers les frères, et envers les étrangers,
\VS{6}qui en présence de l'église ont rendu témoignage de ta charité. Et tu feras bien de les accompagner dignement, comme il est séant selon Dieu.
\VS{7}Car ils sont partis pour son Nom, ne prenant rien des Gentils.
\VS{8}Nous devons donc recevoir de tels hommes, afin d'être ouvriers avec eux pour la vérité.
\TextTitle{Les mauvais actes de Diotrèphe}
\VS{9}J'ai écrit à l'église~; mais Diotrèphe, qui aime être le premier parmi eux, ne nous reçoit pas.
\VS{10}C'est pourquoi, si je viens, je rappellerai les actions qu'il commet, en tenant contre nous de mauvais discours~; et n'étant pas content de cela, non seulement il ne reçoit pas les frères, mais il empêche même ceux qui veulent les recevoir et les chasse de l'église.
\VS{11}Bien-aimé, n'imite pas le mal, mais le bien. Celui qui fait le bien est de Dieu, mais celui qui fait le mal n'a pas vu Dieu.
\TextTitle{Témoignage de Démétrius}
\VS{12}Tous rendent témoignage à Démétrius, et la vérité même le lui rend~; et nous aussi, nous lui rendons témoignage, et vous savez que notre témoignage est véritable.
\TextTitle{Salutations}
\VS{13}J'avais plusieurs choses à écrire, mais je ne veux pas t'écrire avec l'encre et la plume.
\VS{14}Mais j'espère te voir bientôt, et nous parlerons de bouche à bouche.
\VS{15}Que la paix soit avec toi~! Les amis te saluent. Salue les amis, chacun par son nom.
\PPE{}
\end{multicols}
