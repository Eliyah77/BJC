\ShortTitle{Colossiens}\BookTitle{Colossiens}\BFont
\noindent\hrulefill
{\footnotesize
\textit{
\bigskip
{\centering{}
\\Auteur : Paul
\\Thème : La prééminence de Christ
\\Date de rédaction : Env. 60 ap. J.-C.\\}
}
%\bigskip
\textit{
\\Située en Asie Mineure, Colosses était une ville de Phrygie qui se trouvait à environ deux cents kilomètres d’Ephèse.
%\bigskip
\\Rédigée lors de la première captivité romaine de Paul, la lettre aux Colossiens a pour but de rétablir la suprématie de Christ. En effet, cette église - dont Epaphras, le probable fondateur, s’était converti à Ephèse au cours des trois années que Paul y passa - était sous l’influence d’enseignements séducteurs basés sur le gnosticisme. Cette philosophie à la fois attrayante et très dangereuse prônait entre autres le salut par la connaissance et le dualisme.\bigskip
}
}
\par\nobreak\noindent\hrulefill
\begin{multicols}{2}
\Chap{1}
\TextTitle{Introduction}
\VerseOne{}Paul, apôtre de Jésus-Christ, par la volonté de Dieu, et le frère Timothée :
\VS{2}Aux saints et frères fidèles en Christ, qui sont à Colosses : Que la grâce et la paix vous soient données de la part de Dieu notre Père et de la part du Seigneur Jésus-Christ.
\VS{3}Nous rendons grâces à Dieu, qui est le Père de notre Seigneur Jésus-Christ, et nous prions toujours pour vous.
\VS{4}Ayant été informés de votre foi en Jésus-Christ, et de votre charité envers tous les saints,
\VS{5}à cause de l'espérance des biens qui vous sont réservés dans les cieux, et dont vous avez eu précédemment connaissance par la parole de la vérité, c'est-à-dire par l'Evangile,
\VS{6}qui est parvenu jusqu'à vous, comme il l'est aussi dans le monde entier. Il porte des fruits, et il s’accroit, comme c’est aussi le cas parmi vous, depuis le jour où vous avez entendu et connu la grâce de Dieu conformément à la vérité,
\VS{7}ainsi que vous en avez été instruits par Epaphras, notre cher compagnon de service, qui est pour vous un fidèle ministre de Christ.
\VS{8}Et il nous a fait connaître votre charité selon l’Esprit.
\TextTitle{Paul désire leurs progrès spirituels}
\VS{9}C'est pourquoi depuis le jour où nous l’avons appris, nous n’avons cessé de prier pour vous, et de demander à Dieu que vous soyez remplis de la connaissance de sa volonté, en toute sagesse et intelligence spirituelle,
\VS{10}afin que vous vous conduisiez d’une manière digne du Seigneur, pour lui plaire en toutes choses, portant des fruits en toutes sortes de bonnes œuvres, et croissant dans la connaissance de Dieu,
\VS{11}fortifiés en toute force selon sa puissance glorieuse, en toute patience, et tranquillité d’esprit, avec joie.
\VS{12}Rendez grâces au Père, qui nous a rendus capables d’avoir part à l'héritage des saints dans la lumière.
\VS{13}Il nous a délivrés de la puissance des ténèbres, et nous a transportés dans le Royaume de son Fils bien-aimé,
\VS{14}en qui nous avons la rédemption par son sang, savoir la rémission des péchés.
\TextTitle{La supériorité du Messie}
\VS{15}Lequel est l'image de Dieu invisible, le premier-né\FTNT{Premier-né de toute la création : L’expression «~premier-né~» revêt au moins trois sens dans les saintes Ecritures. «~Pendant qu’ils étaient là, le temps où Marie devait accoucher arriva. Et elle enfanta son fils, premier-né...~» Lu. 2:6-7. L’expression «~fils premier-né~» est employée dans un sens littéral : Jésus est le premier enfant à qui Marie a donné naissance. Dans Ex. 4:22, l’expression est employée au sens figuré : «~Israël est mon fils, mon premier-né~». Dans ce verset, il n’est pas question de naissance réelle, mais Yahweh emploie ce mot pour décrire la place distincte qu’Israël occupe dans son plan et ses desseins. L’expression «~premier-né~» figure également dans le Ps. 89:28 et désigne un caractère de supériorité, de suprématie, de particularité et de position. Dieu dit dans ce passage qu’il fera de David un premier-né, avec un rang supérieur à celui des autres rois de la terre. Mais en termes de rang de naissance, David était le dernier-né de son père Isaï (1Sa. 16:1-15). Pourtant, Dieu était déterminé à lui accorder une place unique de suprématie, de primauté et de souveraineté. Donc la notion de «~premier-né~» ne concerne pas forcément l’ordre de la naissance physique. Concernant le passage qui nous intéresse, l'expression «~le premier-né de toute la création~» est comprise par certains chrétiens comme «~le premier créé de toute la création. Ceux qui nient la divinité de Jésus-Christ se servent de ce verset pour dire que Jésus a été créé. Toutefois, si telle avait été la pensée de Paul, il aurait employé le terme grec «~prôtoktisis~», pour «~premier-créé~», au lieu de «~prôtotokos~», c’est-à-dire «~premier-né~». Que devons-nous comprendre par le fait que Jésus soit «~le premier-né de toute la création~» ? Paul s'explique dans le verset suivant : Jésus-Christ est le premier-né de toute la création non parce qu'il a été créé le premier, mais parce que tout ce qui existe a été créé par lui. Il est le Créateur suprême et le but de la création. Il est l’Alpha, le Premier, le Commencement de toute chose (Ap. 1:8 ; Ap. 21:6 ; Ap. 22:13), l’Auteur de la vie. Jésus-Christ a toujours été là et il est la source de toute existence. En tant que Dieu, il occupe une position de suprématie par rapport à la création. Il a une position éminente et dominante. Cette expression est un titre de priorité, de hiérarchie et non pas de chronologie. Les opposants de la divinité de Jésus-Christ affirment aussi qu’avant sa venue sur terre, Jésus était l’archange Michel, ce qui signifie qu’il aurait été créé. La Parole de Dieu permet de réfuter avec force cette affirmation absurde, car elle déclare clairement que le Seigneur Jésus-Christ est le Créateur des anges et de tous les êtres, qu’ils soient visibles ou invisibles (Col. 1:15-16). Rien de ce qui a été créé n’a été créé sans lui (Jn. 1:3). Jésus est le véritable Créateur du ciel et de la terre et de tout ce qu'elle contient (Ge. 1:1 ; Ge. 2:7 ; Ps. 104:30 ; Job. 33:4 ; Es. 45:11-18 ; Jn. 1:3 ; Col. 1:12 ; 1 Co. 8:6 ; Ap. 22:3 ; Ap. 14:6). Jésus-Christ est aussi le premier-né d’entre les morts (Col. 1:18). Cela ne signifie pas que le Seigneur Jésus-Christ a été le premier à ressusciter, car il y a eu plusieurs cas de résurrection avant la sienne. Mais Christ fut le premier ressuscité des morts à ne plus mourir par la suite, le premier à ressusciter avec un corps glorieux, et il le fit en tant que Chef d’une nouvelle Création. Sa résurrection est unique, elle incarne la promesse que tous ceux qui croient en lui ressusciteront aussi (Jn. 3:16). Jésus est avant toutes choses et toutes choses subsistent par lui. Il est important de dire que Jésus est avant toutes choses, et non qu'il «~était~» avant toutes choses, car en hébreu biblique, l’auxiliaire être n’existe pas. Le temps présent est souvent utilisé dans la Bible pour décrire l’éternité de la divinité. Le Seigneur dit : «~Avant qu’Abraham fût, Je suis~» (Jn. 8:58). Non seulement le Seigneur Jésus-Christ existait avant toutes choses, mais toutes choses subsistent par lui. Sa domination est totale et s’étend du domaine physique au domaine spirituel puisqu'il est la Tête du corps de l'Eglise.} de toute la création.
\VS{16}Car par lui ont été créées toutes les choses qui sont aux Cieux et en la terre, les visibles et les invisibles, soit les trônes, ou les dominations, ou les principautés, ou les puissances, toutes choses ont été créées par lui, et pour lui.
\VS{17}Et il est avant toutes choses, et toutes choses subsistent par lui.
\VS{18}Il est le Chef du corps de l'Eglise ; il est le commencement et le premier-né d'entre les morts, afin qu'il tienne le premier rang en toutes choses.
\VS{19}Car le bon plaisir du Père a été que toute plénitude habitât en lui.
\TextTitle{L'oeuvre de réconciliation du Messie}
\VS{20}Et de réconcilier par lui toutes choses avec lui même, ayant fait la paix par le sang de sa croix, savoir, tant les choses qui sont dans les cieux, que celles qui sont sur la terre.
\VS{21}Et vous qui étiez autrefois éloignés de lui, et qui étiez ses ennemis dans votre entendement, et dans les mauvaises œuvres, il vous a maintenant réconciliés par sa mort dans le corps de sa chair,
\VS{22}pour vous faire parâitre saints, sans tache, et irrépréhensibles devant lui.
\VS{23}Si toutefois vous demeurez dans la foi, étant fondés et fermes, et n’étant point transportés hors de l’espérance de l’Evangile que vous avez entendu, lequel est prêché à toute créature qui est sous le ciel, dont moi Paul, j’ai été fait le ministre.
\TextTitle{L'Eglise : un «~mystère caché de tout temps~»\FTNTT{Ep. 3:1-11}}
\VS{24}Je me réjouis donc maintenant dans mes souffrances pour vous, et j'accomplis le reste des afflictions de Christ dans ma chair, pour son Corps, qui est l'Eglise.
\VS{25}C’est d’elle que j'ai été fait le ministre, selon la dispensation de Dieu qui m'a été donnée envers vous, afin que j’annonce pleinement la parole de Dieu,
\VS{26}savoir le mystère qui avait été caché dans tous les siècles et dans tous les âges, mais qui est maintenant manifesté à ses saints ;
\VS{27}auxquels Dieu a voulu donner à connaître quelles sont les richesses de la gloire de ce mystère parmi les Gentils, c’est à savoir Christ, qui a été prêché parmi vous, et qui est l’espérance de la gloire.
\VS{28}Lequel nous annonçons, en exhortant tout homme, et enseignant tout homme en toute sagesse, afin que nous présentions tout homme parfait en Jésus-Christ.
\VS{29}A quoi aussi je travaille, en combattant selon son efficacité\FTNTT{le terme "efficacité" vint du grec "energeia" qui signifie "action", "fonctionnement", compétence", ou encore force à l'oeuvre dans". Ce mot est utilisé seulement pour parler du pouvoir surhumain que ce soit de Dieu ou du diable. (Ep. 1:19 ; Ph. 3:21 ; 2 Th. 2:9)}, qui agit puissamment en moi.
\Chap{2}
\TextTitle{La divinité du Messie}
\VerseOne{}Or je veux en effet que vous sachiez combien est grand le combat que j'ai pour vous, et pour ceux qui sont à Laodicée, et pour tous ceux qui n'ont pas vu mon visage dans la chair,
\VS{2}afin que leurs cœurs soient consolés, étant unis ensemble dans la charité, et enrichis d'une pleine intelligence, pour la connaissance du mystère de notre Dieu et Père, et de Christ,
\VS{3}en qui sont cachés tous les trésors de la sagesse et de la connaissance.
\TextTitle{Mise en garde contre les discours séduisants\FTNTT{Ro. 16:17-18 ; 1 Co. 2:4 ; 2 Pi. 2:3}}
\VS{4}Or je dis ceci afin que personne ne vous trompe par des discours séduisants.
\VS{5}Car quoique je sois absent de corps, toutefois je suis avec vous en esprit, me réjouissant, et voyant votre ordre et la fermeté de votre foi, que vous avez en Christ.
\VS{6}Ainsi, comme vous avez reçu le Seigneur Jésus-Christ, marchez en lui,
\VS{7}étant enracinés et édifiés en lui, et fortifiés en la foi, selon que vous avez été enseignés, abondant en elle avec action de grâces.
\TextTitle{Mise en garde contre la philosophie}
\VS{8}Prenez garde que personne ne fasse de vous sa proie par la philosophie, et par des vaines tromperies, conformes à la tradition des hommes et aux rudiments du monde et non point à la doctrine de Christ.
\TextTitle{La divinité du Messie et la vraie circoncision}
\VS{9}Car en lui habite corporellement toute la plénitude de la divinité\FTNT{En Jésus-Christ habite toute la plénitude de la divinité. Il est le Dieu Tout-Puissant.}.
\VS{10}Et vous êtes rendus accomplis en lui, qui est le Chef de toute principauté et puissance,
\VS{11}en qui aussi vous êtes circoncis d’une circoncision faite sans main, qui consiste à dépouiller le corps des péchés de la chair, ce qui est la circoncision de Christ.
\VS{12}Etant ensevelis avec lui par le baptême; en qui aussi vous êtes ensemble ressuscités par la foi de l’efficace de Dieu, qui l’a ressuscité des morts.
\VS{13}Et lorsque vous étiez morts dans vos offenses, et dans l'incirconcision de votre chair, il vous a vivifiés ensemble avec lui, vous ayant gratuitement pardonné toutes vos offenses.
\TextTitle{Les observances de la loi mosaïque accomplis par le Messie\FTNTT{Mt. 5:17 ; Jn. 19:30}}
\VS{14}Il a effacé l’acte qui était contre nous, qui consistait en des ordonnances, et qui nous était contraire, et il l'a entièrement aboli en le clouant à la croix.
\VS{15}Il a dépouillé les principautés et les puissances, et les a exposées publiquement en spectacle, en triomphant d’elles par la croix.
\VS{16}Que personne donc ne vous condamne au sujet du manger ou du boire, ou au sujet d'un jour de fête, ou d'un jour de nouvelle lune, ou des sabbats.
\VS{17}C’était l’ombre des choses qui devaient venir, mais le corps en est en Christ.
\TextTitle{Mise en garde contre le mysticisme et l'ascétisme}
\VS{18}Que personne ne vous enlève à son grès le prix de la course, sous l'apparence d'humilité d'esprit et par un culte des anges, s'ingérant dans les choses qu'il n'a pas vues, étant témérairement enflé par ses pensées charnelles,
\VS{19}et sans s’attacher au Chef, dont tout le corps étant joint et ajusté ensemble par des jointures et des liens, s’accroît d'un accroissement de Dieu.
\VS{20}Si donc vous êtes morts avec Christ quant aux rudiments du monde, pourquoi, comme si vous viviez dans le monde, vous impose-t-on ces préceptes :
\VS{21}Savoir, ne prends pas ! Ne goûte pas ! Ne touche pas !
\VS{22}Choses qui sont toutes périssables par l'usage, et établies suivant les commandements et les doctrines des hommes.
\VS{23}Et qui ont pourtant quelque apparence de sagesse en devotion volontaire, et en humilité d'esprit, et en ce qu'elles n'épargnent nullement le corps, et n'ont aucun égard au rassasiement de la chair.
\Chap{3}
\TextTitle{Le croyant doit chercher les choses d'en haut}
\VerseOne{}Si donc vous êtes ressuscités avec Christ, cherchez les choses qui sont en haut, où Christ est assis à la droite de Dieu.
\VS{2}Pensez aux choses d’en haut, et non à celles qui sont sur la terre.
\VS{3}Car vous êtes morts, et votre vie est cachée avec Christ en Dieu.
\VS{4}Quand Christ, qui est votre vie, apparaîtra, vous paraîtrez aussi alors avec lui dans la gloire.
\TextTitle{La marche chrétienne}
\VS{5}Mortifiez donc vos membres qui sont sur la terre : La fornication, l’impureté, les passions, les mauvais désirs, et la cupidité, qui est une idolâtrie.
\VS{6}C’est à cause de ces choses que la colère de Dieu vient sur les fils de la rébellion.
\VS{7}C’est ainsi que vous marchiez autrefois, quand vous viviez dans ces choses.
\VS{8}Mais maintenant, rejetez toutes ces choses : La colère, l'animosité, la médisance, et les paroles déshonnêtes qui pourraient sortir de votre bouche.
\VS{9}Ne mentez point les uns aux autres, vous étant dépouillés du vieil homme et de ses œuvres,
\VS{10}et ayant revêtu le nouvel homme, qui se renouvelle dans la connaissance, selon l'image de celui qui l'a créé.
\VS{11}En qui il n'y a ni Grec, ni Juif, ni circoncis, ni incirconcis, ni barbare, ni Scythe, ni esclave, ni libre ; mais Christ y est tout et en tous.
\VS{12}Ainsi donc, comme des élus de Dieu, saints et bien-aimés, revêtez-vous des entrailles de miséricorde, de bonté, d'humilité, de douceur, de patience :
\VS{13}vous supportant les uns les autres, et vous pardonnant les uns aux autres ; et si l’un a querelle contre l’autre, comme Christ vous a pardonné, vous aussi faites-en de même.
\VS{14}Mais par-dessus toutes ces choses, revêtez-vous de la charité, qui est le lien de la perfection.
\VS{15}Et que la paix de Dieu, à laquelle vous êtes appelés pour être un seul corps, tienne le principal lieu dans vos cœurs ; et soyez reconnaissants.
\VS{16}Que la parole de Christ habite en vous abondamment en toute sagesse, vous enseignant et vous exhortant l’un l’autre par des Psaumes, des hymnes et des cantiques spirituels, avec grâce, chantant de votre cœur au Seigneur.
\VS{17}Et quoi que vous fassiez, en parole ou en œuvre, faites tout au Nom du Seigneur Jésus, rendant grâces par lui à notre Dieu et Père.
\TextTitle{Les relations familiales selon Dieu}
\VS{18}Femmes, soyez soumises à vos maris, comme il convient dans le Seigneur\FTNT{Ep. 5:22.}.
\VS{19}Maris, aimez vos femmes, et ne vous aigrissez pas contre elles\FTNT{Ep. 5:25.}.
\VS{20}Enfants, obéissez à vos pères et à vos mères en toutes choses, car cela est agréable au Seigneur\FTNT{Ep. 6:1-2}.
\VS{21}Pères, n'irritez pas vos enfants\FTNT{Ep. 6:4.}, afin qu'ils ne se découragent pas.
\TextTitle{Les rapports entre serviteurs et maîtres chrétiens}
\VS{22}Serviteurs, obéissez en toutes choses à ceux qui sont vos maîtres selon la chair, ne servant point seulement sous leurs yeux, comme voulant complaire aux hommes, mais en simplicité de cœur, craignant Dieu\FTNT{Ep. 6:5-6.}.
\VS{23}Et quoi que vous fassiez, faites tout de bon cœur, comme le faisant pour le Seigneur, et non pas pour les hommes.
\VS{24}Sachant que vous recevrez du Seigneur l’héritage pour récompense, car vous servez Christ, le Seigneur.
\VS{25}Mais celui qui agit injustement recevra ce qu'il aura fait injustement : car en Dieu il n’y a point d’égard à l’apparence des personnes.
\Chap{4}
\VerseOne{}Maîtres, accordez à vos serviteurs ce qui est juste et équitable, sachant que vous avez, vous aussi, un Maître dans les cieux.
\TextTitle{La persévérance dans la prière}
\VS{2}Persévérez dans la prière, veillant dans cet exercice avec des actions de grâces :
\VS{3}Priez aussi tous ensemble pour nous, afin que Dieu nous ouvre la porte de la parole, pour annoncer le mystère de Christ pour lequel aussi je suis prisonnier.
\VS{4}Afin que je le fasse connaître comme je dois en parler.
\VS{5}Conduisez-vous sagement envers ceux du dehors, et rachetez le temps.
\VS{6}Que votre parole soit toujours assaisonnée de sel avec grâce, afin que vous sachiez comment vous avez à répondre à chacun.
\TextTitle{Exhortations personnelles et conclusion}
\VS{7}Tychique, notre frère bien-aimé, et fidèle ministre, et mon compagnon de service en notre Seigneur, vous fera savoir tout mon état.
\VS{8}Je l’envoie vers vous expressément, afin qu’il connaisse quel est votre état, et qu’il console vos cœurs ;
\VS{9}avec Onésime, notre fidèle et bien-aimé frère, qui est des vôtres, ils vous feront connaitre t de tout ce qui se passe ici.
\VS{10}Aristarque, qui est prisonnier avec moi, vous salue aussi, et Marc qui est le cousin de Barnabas, au sujet duquel vous avez reçu un ordre : s’il vient à vous, recevez-le,
\VS{11}Et Jésus, appelé Justus, vous salue aussi. Ils sont du nombre des circoncis, et les seuls qui travaillent avec moi pour le Royaume de Dieu, et qui ont été pour moi une consolation.
\VS{12}Epaphras, qui est des vôtres, et serviteur de Jésus-Christ, vous salue ; il ne cesse de combattre pour vous dans ses prières, afin que vous demeuriez parfaits et accomplis en toute la volonté de Dieu.
\VS{13}Car je lui rends témoignage qu'il a un grand zèle pour vous, et pour ceux de Laodicée, et pour ceux d'Hiérapolis.
\VS{14}Luc, le médecin bien-aimé, vous salue ; et Démas aussi.
\VS{15}Saluez les frères qui sont à Laodicée, et Nymphas, avec l'église qui est dans sa maison.
\VS{16}Et quand cette lettre aura été lue chez vous, faites en sorte qu'elle soit aussi lue dans l'église des Laodiciens, et que vous lisiez aussi celle qui arrivera de Laodicée.
\VS{17}Et dites à Archippe : Prends garde au ministère que tu as reçu dans le Seigneur, afin de bien le remplir.
\VS{18}Je vous salue, moi Paul, de ma propre main. Souvenez-vous de mes liens. Que la grâce soit avec vous ! Amen !
\PPE{}
\end{multicols}
