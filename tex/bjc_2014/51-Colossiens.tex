\ShortTitle{Colossiens}\BookTitle{Colossiens}\BFont
\begin{multicols}{2}
\TextTitle{[Introduction]}
\Chap{1}
\VerseOne{}Paul, apôtre de Jésus-Christ, par la volonté de Dieu, et le frère Timothée :
\VS{2}Aux saints et frères fidèles en Christ, qui sont à Colosses : Que la grâce et la paix vous soient données par Dieu notre Père et Seigneur Jésus-Christ.
\VS{3}Nous rendons grâces à Dieu, qui est le Père de notre Seigneur Jésus-Christ, et nous prions continuellement pour vous.
\VS{4}Ayant été informés de votre foi en Jésus-Christ, et de votre charité envers tous les saints,
\VS{5}à cause de l'espérance qui vous est réservée dans les cieux, et dont vous avez eu précédemment connaissance par la parole de la vérité, c'est-à-dire par l'Evangile,
\VS{6}qui est parvenu jusqu'à vous, comme il l'est aussi dans le monde entier. Il porte des fruits, et il s’accroit, comme c’est aussi le cas parmi vous, depuis le jour où vous avez entendu et connu la grâce de Dieu conformément à la vérité,
\VS{7}ainsi que vous en avez été instruits par Epaphras, notre bien-aimé compagnon de service, qui est pour vous un fidèle ministre de Christ.
\VS{8}Et il nous a fait connaître votre charité selon l’Esprit.
\TextTitle{[Paul désire leurs progrès spirituels]}
\VS{9}C'est pourquoi depuis le jour où nous l’avons appris, nous n’avons cessé de prier pour vous, et de demander à Dieu que vous soyez remplis de la connaissance de sa volonté, en toute sagesse et intelligence spirituelle,
\VS{10}afin que vous vous conduisiez d’une manière digne du Seigneur, pour lui plaire en toutes choses, portant des fruits en toutes sortes de bonnes œuvres, et croissant dans la connaissance de Dieu,
\VS{11}fortifiés à tous égards par sa puissance glorieuse, en sorte que soyez toujours et avec joie persévérants et patients.
\VS{12}Rendez grâces au Père, qui nous a rendus capables d’avoir part à l'héritage des saints dans la lumière.
\VS{13}Il nous a délivrés de la puissance des ténèbres, et nous a transportés dans le Royaume de son Fils bien-aimé,
\VS{14}en qui nous avons la rédemption par son sang, la rémission des péchés.
\TextTitle{[Sept preuves de la supériorité de Christ]}
\VS{15}Le Fils est l'image du Dieu invisible, le premier-né\FTNT{Premier-né de toute la création : L’expression ~premier-né~ revêt au moins trois sens dans les saintes Ecritures. ~Pendant qu’ils étaient là, le temps où Marie devait accoucher arriva. Et elle enfanta son fils, premier-né...~ Lu. 2:6-7. L’expression ~fils premier-né~ est employée dans un sens littéral : Jésus est le premier enfant à qui Marie a donné naissance. Dans Ex. 4:22, l’expression est employée au sens figuré : ~Israël est mon fils, mon premier-né~. Dans ce verset, il n’est pas question de naissance réelle, mais Yahweh emploie ce mot pour décrire la place distincte qu’Israël occupe dans son plan et ses desseins. L’expression ~premier-né~ figure également dans le Ps. 89:28 et désigne un caractère de supériorité, de suprématie, de particularité et de position. Dieu dit dans ce passage qu’il fera de David un premier-né, avec un rang supérieur à celui des autres rois de la terre. Mais en termes de rang de naissance, David était le dernier-né de son père Isaï (1Sa. 16:1-15). Pourtant, Dieu était déterminé à lui accorder une place unique de suprématie, de primauté et de souveraineté. Donc la notion de ~premier-né~ ne concerne pas forcément l’ordre de la naissance physique. Concernant le passage qui nous intéresse, l'expression ~le premier-né de toute la création~ est comprise par certains chrétiens comme ~le premier créé de toute la création. Les Témoins de Jéhovah et d’autres sectes qui nient la divinité de Jésus-Christ se servent de ce verset pour dire que Jésus a été créé. Toutefois, si telle avait été la pensée de Paul, il aurait employé le terme grec ~prôtoktisis~, pour ~premier-créé~, au lieu de ~prôtotokos~, c’est-à-dire ~premier-né~. Que devons-nous comprendre par le fait que Jésus soit ~le premier-né de toute la création~ ? Paul s'explique dans le verset suivant : Jésus-Christ est le premier-né de toute la création non parce qu'il a été créé le premier, mais parce que tout ce qui existe a été créé par lui. Il est le Créateur suprême et le but de la création. Il est l’Alpha, le Premier, le Commencement de toute chose (Ap. 1:8 ; Ap. 21:6 ; Ap. 22:13), l’Auteur de la vie. Jésus-Christ a toujours été là et il est la source de toute existence. En tant que Dieu, il occupe une position de suprématie par rapport à la création. Il a une position éminente et dominante. Cette expression est un titre de priorité, de hiérarchie et non pas de chronologie. Les Témoins de Jéhovah affirment aussi qu’avant sa venue sur terre, Jésus était l’archange Michel, ce qui signifie qu’il aurait été créé. La Parole de Dieu permet de réfuter avec force cette affirmation absurde, car elle déclare clairement que le Seigneur Jésus-Christ est le Créateur des anges et de tous les êtres, qu’ils soient visibles ou invisibles (Col. 1:15-16). Rien de ce qui a été créé n’a été créé sans lui (Jn. 1:3). Jésus est le véritable Créateur du ciel et de la terre et de tout ce qu'elle contient (Ge. 1:1 ; Ge. 2:7 ; Ps. 104:30 ; Job. 33:4 ; Es. 45:11-18 ; Jn. 1:3 ; Col. 1:12 ; 1 Co. 8:6 ; Ap. 22:3 ; Ap. 14:6). Jésus-Christ est aussi le premier-né d’entre les morts (Col. 1:18). Cela ne signifie pas que le Seigneur Jésus-Christ a été le premier à ressusciter, car il y a eu plusieurs cas de résurrection avant la sienne. Mais Christ fut le premier ressuscité des morts à ne plus mourir par la suite, le premier à ressusciter avec un corps glorieux, et il le fit en tant que Chef d’une nouvelle Création. Sa résurrection est unique, elle incarne la promesse que tous ceux qui croient en lui ressusciteront aussi (Jn. 3:16). Jésus est avant toutes choses et toutes choses subsistent en lui. Il est important de dire que Jésus est avant toutes choses, et non qu'il ~était~ avant toutes choses, car en hébreu biblique, l’auxiliaire être n’existe pas. Le temps présent est souvent utilisé dans la Bible pour décrire l’éternité de la divinité. Le Seigneur dit : ~Avant qu’Abraham fût, Je suis~ (Jn. 8:58). Non seulement le Seigneur Jésus-Christ existait avant toutes choses, mais toutes choses subsistent en lui. Sa domination est totale et s’étend du domaine physique au domaine spirituel puisqu'il est la Tête du corps de l'Eglise.} de toute la création.
\VS{16}Car en lui ont été créées toutes les choses qui sont dans les cieux et sur la terre, les visibles et les invisibles, trônes, dominations, principautés, autorités. Tout a été créé par lui et pour lui.
\VS{17}Et il est avant toutes choses, et toutes choses subsistent par lui.
\VS{18}Il est la Tête du corps de l'Eglise ; il est le commencement, le premier-né d'entre les morts, afin qu'il tienne le premier rang en toutes choses.
\VS{19}Car Dieu a voulu que toute plénitude habitât en lui.
\TextTitle{[l'oeuvre de réconciliation de Christ]}
\VS{20}Il a voulu par lui réconcilier toutes choses avec lui-même, tant ce qui est sur la terre que ce qui est dans les cieux, en faisant la paix par le sang de sa croix.
\VS{21}Et vous qui étiez autrefois étrangers et ennemis par vos pensées et par vos mauvaises œuvres, il vous a maintenant réconciliés par sa mort dans le corps de sa chair,
\VS{22}pour vous présenter devant lui saints, sans tache, et irrépréhensibles.
\VS{23}Si toutefois vous demeurez fondés dans la foi et inébranlables, et sans vous détourner de l'espérance de l'Evangile que vous avez entendu, qui a été prêché à toute créature sous le ciel, et dont moi Paul, j'ai été fait ministre.
\TextTitle{[L'Eglise : un "mystère caché de tout temps"]
\\(Ep. 3:1-11)}
\VS{24}Je me réjouis donc maintenant dans mes souffrances pour vous, et j'accomplis le reste des afflictions de Christ dans ma chair, pour son Corps, qui est l'Eglise.
\VS{25}C’est d’elle que j'ai été fait ministre, selon la charge de Dieu qui m'a été donnée envers vous, afin que j’annonce pleinement la parole de Dieu,
\VS{26}le mystère caché de tout temps et dans tous les âges, mais manifesté maintenant à ses saints.
\VS{27}Dieu a voulu faire connaître la glorieuse richesse de ce mystère parmi les païens, à savoir Christ, qui a été prêché parmi vous, l'espérance de la gloire.
\VS{28}C’est lui que nous annonçons, exhortant tout homme, et enseignant tout homme en toute sagesse, afin de présenter à Dieu tout homme devenu parfait en Jésus-Christ.
\VS{29}A quoi aussi je travaille, en combattant selon sa force, qui agit puissamment en moi.
\TextTitle{[La divinité incarnée en Christ]
\\a. Christ, source de sagesse}
\Chap{2}
\VerseOne{}Je veux que vous sachiez combien est grand le combat que je soutiens pour vous, et pour ceux qui sont à Laodicée, et pour tous ceux qui n'ont pas vu mon visage en la chair,
\VS{2}afin que leurs cœurs soient consolés, étant unis ensemble dans la charité, et enrichis d'une pleine intelligence, pour la connaissance du mystère de notre Dieu et Père, et de Christ,
\VS{3}en qui sont cachés tous les trésors de la sagesse et de la connaissance.
\TextTitle{[b. danger des discours séduisants]
\\(Ro. 16:17-18 ; 1 Co. 2:4 ; 2 Pi. 2:3)}
\VS{4}Je dis cela afin que personne ne vous trompe par des discours séduisants.
\VS{5}Car si je suis absent de corps, je suis avec vous en esprit, me réjouissant de voir parmi vous le bon ordre et la fermeté de votre foi en Christ.
\VS{6}Ainsi, comme vous avez reçu le Seigneur Jésus-Christ, marchez en lui,
\VS{7}étant enracinés et fondés en lui, et affermis par la foi, d’après les instructions qui vous ont été données, et abondez en actions de grâces.
\TextTitle{[c. mise en garde contre la fausse philosophie et le légalisme]}
\VS{8}Prenez garde que personne ne fasse de vous sa proie par la philosophie, et par des vaines tromperies, s’appuyant sur la tradition des hommes, sur les rudiments du monde et non sur la doctrine de Christ.
\TextTitle{[d. le chrétien a tout pleinement en Christ]}
\VS{9}Car en lui habite corporellement toute la plénitude de la divinité\FTNT{En Jésus-Christ habite toute la plénitude de la divinité. Il est le Dieu Tout-Puissant.}.
\VS{10}Et vous avez toute la plénitude en lui, qui est le Chef de toute principauté et puissance,
\VS{11}en qui aussi vous êtes circoncis d'une circoncision que la main n’a pas faite, de la circoncision de Christ, qui consiste dans le dépouillement du corps des péchés de la chair.
\VS{12}Ayant été ensevelis avec lui par le baptême, vous êtes aussi ressuscités en lui et avec lui par la foi en la puissance de Dieu, qui l'a ressuscité des morts.
\VS{13}Vous qui étiez morts par vos offenses, et par l’incirconcision de votre chair, il vous a vivifiés ensemble avec lui, en nous faisant grâce pour toutes nos offenses.
\TextTitle{[e. les observances de la loi accomplis en Christ]
\\(Mt. 5:17)}
\VS{14}Il a effacé l’acte dont les ordonnances nous condamnaient et qui subsistait contre nous, et il l’a détruit en le clouant à la croix.
\VS{15}Il a dépouillé les dominations et les autorités, et les a exposées publiquement en spectacle, en triomphant d’elles par la croix.
\VS{16}Que personne donc ne vous condamne au sujet du manger ou du boire, au sujet de la distinction d'un jour de fête, d’une nouvelle lune, ou des sabbats.
\VS{17}C’était l’ombre des choses à venir, mais le corps est en Christ.
\TextTitle{[f. mise en garde contre le mysticisme]}
\VS{18}Que personne, par une fausse humilité et par un culte des anges, ne vous ravisse le prix de la course, tandis qu’il s’abandonne à ses visions et qu’il est enflé d’un vain orgueil par ses pensées charnelles,
\VS{19}et sans s’attacher au Chef, dont tout le corps étant joint et ajusté ensemble par des jointures et des liens, s’accroît d'un accroissement de Dieu.
\TextTitle{[g. mis en garde conter l'ascétisme]}
\VS{20}Si vous êtes morts avec Christ aux rudiments du monde, pourquoi, comme si vous viviez dans le monde, vous impose-t-on ces préceptes :
\VS{21}Ne prends pas ! Ne goûte pas ! Ne touche pas !
\VS{22}Préceptes qui tous deviennent pernicieux par l’abus, et qui ne sont fondés que sur les ordonnances et les doctrines des hommes ?
\VS{23}Ils ont, à la vérité, une apparence de sagesse, en ce qu’ils indiquent un culte volontaire, de l’humilité, et le mépris du corps, mais ils sont sans aucun mérite et contribuent à la satisfaction de la chair.
\TextTitle{[Quelques caractéristiques de la "vie abondante" du chrétien
\\a. l'union présente et future du croyant avec Christ]}
\Chap{3}
\VerseOne{}Si donc vous êtes ressuscités avec Christ, cherchez les choses qui sont en haut, où Christ est assis à la droite de Dieu.
\VS{2}Affectionnez-vous aux choses d’en haut, et non à celles qui sont sur la terre.
\VS{3}Car vous êtes morts, et votre vie est cachée avec Christ en Dieu.
\VS{4}Quand Christ, votre vie, apparaîtra, alors vous paraîtrez aussi avec lui dans la gloire.
\TextTitle{[b. la vie en Christ]}
\VS{5}Mortifiez donc vos membres qui sont sur la terre : La fornication, l’impureté, les passions, les mauvais désirs, et la cupidité, qui est une idolâtrie.
\VS{6}C’est à cause de ces choses que la colère de Dieu vient sur les fils de la rébellion.
\VS{7}C’est ainsi que vous marchiez autrefois, quand vous viviez dans ces choses.
\VS{8}Mais maintenant, rejetez toutes ces choses : La colère, l'animosité, la médisance, et les paroles déshonnêtes qui pourraient sortir de votre bouche.
\VS{9}Ne mentez point les uns aux autres, vous étant dépouillés du vieil homme et de ses œuvres,
\VS{10}et ayant revêtu le nouvel homme, qui se renouvelle dans la connaissance, selon l'image de celui qui l'a créé.
\VS{11}Il n'y a ni Grec, ni Juif, ni circoncis, ni incirconcis, ni barbare, ni Scythe, ni esclave, ni libre ; mais Christ est tout et en tous.
\VS{12}Ainsi donc, comme des élus de Dieu, saints et bien-aimés, revêtez-vous des sentiments de miséricorde, de bonté, d'humilité, de douceur, de patience.
\VS{13}Supportez-vous les uns les autres, et si l’un a sujet de se plaindre de l’autre, pardonnez-vous réciproquement. De même que Christ vous a pardonné, pardonnez-vous aussi.
\VS{14}Mais par-dessus toutes ces choses, revêtez-vous de la charité, qui est le lien de la perfection.
\VS{15}Et que la paix de Dieu, à laquelle vous êtes appelés pour être un seul corps, règne dans vos cœurs. Et soyez reconnaissants.
\VS{16}Que la parole de Christ demeure abondamment en vous. En toute sagesse, instruisez-vous et exhortez-vous les uns les autres par des psaumes, des hymnes et des cantiques spirituels, chantant dans vos cœurs au Seigneur avec reconnaissance.
\VS{17}Et quoi que vous fassiez, en parole ou en œuvre, faites tout au Nom du Seigneur Jésus, en rendant par lui des actions de grâces à Dieu le Père.
\TextTitle{[c. les relations familiales en Christ]}
\VS{18}Femmes, soyez soumises à vos maris, comme il convient dans le Seigneur\FTNT{Ep. 5:22.}.
\VS{19}Maris, aimez vos femmes, et ne vous aigrissez pas contre elles\FTNT{Ep. 5:25.}.
\VS{20}Enfants, obéissez en toutes choses à vos parents, car cela est agréable au Seigneur\FTNT{Ep. 6:1-2}.
\VS{21}Pères, n'irritez pas vos enfants\FTNT{Ep. 6:4.}, afin qu'ils ne se découragent pas.
\TextTitle{[d. serviteurs et maîtres chrétiens]}
\VS{22}Serviteurs, obéissez en toutes choses à vos maîtres selon la chair, non pas seulement sous leurs yeux, comme pour plaire aux hommes, mais avec simplicité de cœur, dans la crainte du Seigneur\FTNT{Ep. 6:5-6.}.
\VS{23}Et quoi que vous fassiez, faites-le de bon cœur, comme pour le Seigneur, et non pour des hommes.
\VS{24}Sachant que vous recevrez du Seigneur l’héritage pour récompense, car vous servez Christ, le Seigneur.
\VS{25}Mais celui qui agit injustement recevra selon son injustice, car en Dieu il n'y a pas d’acception de personnes.
\Chap{4}
\VerseOne{}Maîtres, accordez à vos serviteurs ce qui est juste et équitable, sachant que vous avez, vous aussi, un Maître dans les cieux.
\TextTitle{[e. persévérance dans la prière, sagesse dans les paroles]}
\VS{2}Persévérez dans la prière, veillez-y avec actions de grâces.
\VS{3}Priez en même temps pour nous, afin que Dieu nous ouvre une porte pour la parole, en sorte que je puisse annoncer le mystère de Christ, pour lequel je suis dans les chaînes,
\VS{4}afin que je le fasse connaître comme je dois en parler.
\VS{5}Conduisez-vous sagement envers ceux du dehors, et rachetez le temps.
\VS{6}Que votre parole soit toujours accompagnée de grâce, et assaisonnée de sel, afin que vous sachiez comment il faut répondre à chacun.
\TextTitle{[Exhortations personnelles]}
\VS{7}Tychique, le bien-aimé frère et le fidèle ministre, mon compagnon de service dans le Seigneur, vous communiquera tout ce qui me concerne.
\VS{8}Je l’envoie exprès vers vous, pour que vous connaissiez notre situation, et pour qu’il console vos cœurs.
\VS{9}Avec Onésime, notre fidèle et bien-aimé frère, qui est des vôtres, ils vous informeront de tout ce qui se passe ici.
\VS{10}Aristarque, mon compagnon de captivité, vous salue, ainsi que Marc, le cousin de Barnabas, au sujet duquel vous avez reçu des ordres : S'il va chez vous, accueillez-le.
\VS{11}Et Jésus, appelé Justus, vous salue aussi. Ils sont du nombre des circoncis, et les seuls qui travaillent avec moi pour le Royaume de Dieu, et qui ont été pour moi une consolation.
\VS{12}Epaphras, qui est des vôtres, et serviteur de Jésus-Christ, vous salue ; il ne cesse de combattre pour vous dans ses prières, afin que vous soyez parfaits et pleinement disposés à faire toute la volonté de Dieu.
\VS{13}Car je lui rends témoignage qu'il a un grand zèle pour vous, et pour ceux de Laodicée, et pour ceux d'Hiérapolis.
\VS{14}Luc, le médecin bien-aimé, vous salue ; et Démas aussi.
\VS{15}Saluez les frères qui sont à Laodicée, et Nymphas, avec l'église qui est dans sa maison.
\VS{16}Et quand cette lettre aura été lue chez vous, faites en sorte qu'elle soit aussi lue dans l'église des Laodiciens, et que vous lisiez aussi celle qui arrivera de Laodicée.
\VS{17}Et dites à Archippe : Prends garde au ministère que tu as reçu dans le Seigneur, afin de bien le remplir.
\VS{18}Je vous salue, moi Paul, de ma propre main. Souvenez-vous de mes liens. Que la grâce soit avec vous !
\PPE{}
\end{multicols}
