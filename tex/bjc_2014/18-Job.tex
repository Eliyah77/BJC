\ShortTitle{Job}\BookTitle{Job}\BFont
\noindent\hrulefill
{\footnotesize
\textit{
\bigskip
{\centering{}
\\(Iyov)
\\Signification : haï, ennemi et  «~je m'exclamerai~»
\\Thème : La souffrance
\\Auteur : Inconnu
\\Date de rédaction : Incertaine\\}
}
%\bigskip
\textit{
\\Job, homme prospère dont Dieu témoigna l’intégrité et la droiture, subit une succession de malheurs en très peu de temps. 
%\bigskip
\\Il subi les railleries de son épouse et les accusations de ses amis. Après avoir cherché à se justifier, Job s’humilia devant Dieu et saisit l’impuissance de sa propre justice.
%\bigskip
\\Cette histoire, pour laquelle nous n’avons aucune indication spatio-temporelle et qui pourtant parle à tous, est un encouragement pour le juste éprouvé. Rappelant que la souffrance peut être le moyen choisi par Dieu pour enseigner et se révéler à ceux qui le craignent, ce récit illustre que s’il est une chose d’entendre parler de Yahweh avec ses oreilles, il en est une toute autre de le voir de ses yeux.\bigskip
}
}
\par\nobreak\noindent\hrulefill
\begin{multicols}{2}
\Chap{1}
\TextTitle{Job et sa famille}
\VerseOne{}Il y avait dans le pays d'Uts\FTNT{Ge. 36:28} un homme dont le nom était Job\FTNT{Ez. 14:14; Ja. 5:11.}. Cet homme était intègre\FTNT{1 R. 8:61.} et droit, craignant\FTNT{Ps. 19:10; Pr. 1:7.} Dieu et se détournant du mal.
\VS{2}Il lui naquit sept fils et trois filles.
\VS{3}Et son bétail était de sept mille brebis, trois mille chameaux, cinq cents paires de bœufs, cinq cents ânesses, et un très grand nombre de serviteurs\FTNT{Job 42:12-13.}. Et cet homme était le plus grand de tous les fils de l'orient.
\VS{4}Or ses fils allaient les uns chez les autres et donnaient tour à tour un festin, et ils invitaient leurs trois sœurs à manger et à boire avec eux.
\VS{5}Quand les jours de festin étaient passés, Job envoyait chercher ses fils pour les sanctifier, et se levant de bon matin, il offrait un holocauste selon le nombre de ses enfants ; car Job disait : Peut-être mes fils ont-ils péché, et ont-ils blasphémé contre Dieu dans leurs cœurs. Job faisait toujours ainsi.\FTNT{Job 42:8.}
\VS{6}Or, il arriva un jour que les fils de Dieu\FTNT{Job 38:7; Ps. 89:7.} vinrent se présenter devant Yahweh, et Satan\FTNT{Es. 14:12; Ap. 12:9-10.} aussi vint au milieu d’eux.
\VS{7}Yahweh dit à Satan : D'où viens-tu ? Et Satan répondit à Yahweh : De courir çà et là sur la terre et de m'y promener\FTNT{1 Pi. 5:8.}.
\VS{8}Yahweh dit à Satan : N'as-tu point considéré mon serviteur Job, qui n'a point d'égal sur la terre ; homme intègre et droit, craignant Dieu, et se détournant du mal ?
\VS{9}Et Satan répondit à Yahweh : Est-ce en vain que Job craint Dieu ?
\VS{10}N'as-tu pas mis une haie \FTNT{La haie est une protection, une barrière végétale entretenue afin de protéger et de clôturer un terrain.} tout autour de lui, autour de sa maison, autour de tout ce qui lui appartient ? Tu as béni l'œuvre de ses mains, et ses troupeaux se répandent sur la terre.
\VS{11}Mais étends maintenant ta main, touche à tout ce qui lui appartient, et tu verras s’il ne te blasphème pas en face.
\VS{12}Et Yahweh dit à Satan : Voici, tout ce qui lui appartient est en ton pouvoir ; seulement ne porte pas la main sur lui. Et Satan sortit de devant la face de Yahweh\FTNT{1 R. 22:22.}.
\TextTitle{Première attaque de Satan}
\VS{13}Il arriva donc qu'un jour comme les fils et les filles de Job mangeaient et buvaient du vin dans la maison de leur frère aîné, un messager vint vers Job,
\VS{14}et lui dit : Les bœufs labouraient, et les ânesses paissaient à côté d'eux ;
\VS{15}et ceux de Séba se sont jetés dessus, les ont pris, et ont frappé les serviteurs au fil de l'épée. Et je me suis échappé, moi seul, pour te l'annoncer.
\VS{16}Cet homme parlait encore, lorsqu'un autre vint et dit : Le feu de Dieu est tombé du ciel, il a brûlé les brebis et les serviteurs, et les a consumés\FTNT{2 R. 1:10-12.}. Et je me suis échappé moi seul, pour t'en apporter la nouvelle.
\VS{17}Cet homme parlait encore, lorsqu'un autre vint et dit : Des Chaldéens\FTNT{Ge. 11:28.} ont fait trois bandes, se sont jetés sur les chameaux et les ont pris, ils ont frappé les serviteurs au fil de l'épée, et je me suis échappé moi seul, pour t'en apporter la nouvelle.
\VS{18}Cet homme parlait encore, lorsqu'un autre vint et dit : Tes fils et tes filles mangeaient et buvaient du vin dans la maison de leur frère aîné ;
\VS{19}voici, un grand vent est venu de l'autre côté du désert et a frappé contre les quatre coins de la maison ; elle est tombée sur les jeunes gens, et ils sont morts. Et je me suis échappé, moi seul, pour t'en apporter la nouvelle.
\VS{20}Alors Job se leva, déchira\FTNT{Est. 4:1 ; Job 2:12.} son manteau, et se rasa la tête ; puis se jetant par terre, il se prosterna,
\VS{21}et dit : Je suis sorti nu du ventre de ma mère, et nu je retournerai dans le sein de la terre\FTNT{Ec. 5:14. ; 1 Ti. 6:7.} ; Yahweh a donné, Yahweh a enlevé\FTNT{1 S. 2:6.} ; que le nom de Yahweh soit béni !
\VS{22}En tout cela, Job ne pécha pas et n'attribua rien d'injuste à Dieu.
\Chap{2}
\TextTitle{Deuxième attaque de Satan}
\VerseOne{}Or il arriva un jour  que les fils de Dieu vinrent un jour se présenter devant Yahweh, Satan\FTNT{Za. 3:1-2.} vint aussi au milieu d'eux se présenter devant Yahweh.
\VS{2}Yahweh dit à Satan : D'où viens-tu ? Satan répondit à Yahweh : De courir çà et là sur la terre et de m'y promener.
\VS{3}Yahweh dit à Satan: N'as-tu point considéré mon serviteur Job, qui n'a point d'égal sur la terre ; homme sincère et droit, craignant Dieu, et se détournant du mal ? Il demeure ferme dans son intégrité, quoique tu m'aies incité contre lui à le détruire sans cause\FTNT{Job 9:17.}.
\VS{4}Satan répondit à Yahweh : Peau pour peau ! Tout ce que possède un homme, il le donne pour sa vie.
\VS{5}Mais étends ta main, et touche à ses os et à sa chair\FTNT{Job 19:20.}, et je suis sûr qu'il te maudit en face !
\VS{6}Yahweh dit à Satan : Voici, il est en ta main : Seulement garde sa vie.
\VS{7}Et Satan sortit de devant la face de Yahweh. Il frappa Job d'un ulcère malin, depuis la plante des pieds jusqu'au sommet de la tête.
\VS{8}Job prit un tesson pour se gratter et s'assit au milieu de la cendre\FTNT{Jé. 6:26 ; Jon. 3:6.}.
\TextTitle{Réaction de Job et de sa femme}
\VS{9}Sa femme lui dit : Tu demeures ferme dans ton intégrité ! Bénis\FTNT{Job 1:11.} Dieu, et meurs !
\VS{10}Et il lui dit : Tu parles comme une femme insensée ! Nous recevons le bien de la part de Dieu, et nous ne recevrions pas le mal !\FTNT{Es. 45:7 ; La. 3:37 ; Am. 3:6.} En tout cela, Job ne pécha pas par ses lèvres.
\TextTitle{Job et ses trois amis}
\VS{11}Trois amis de Job, Eliphaz de Théman, Bildad de Schuach, et Tsophar de Naama, ayant appris tous les maux qui lui étaient arrivés, se concertèrent et convinrent ensemble d'un jour pour aller le plaindre et le consoler.
\VS{12}Ayant de loin levé les yeux sur lui, ils ne le reconnurent pas, alors ils élevèrent la voix et ils pleurèrent. Ils déchirèrent leurs manteaux, et jetèrent de la poussière vers le ciel au-dessus de leur tête.
\VS{13}Puis ils se tinrent assis à terre auprès de lui, sept jours et sept nuits, et aucun d'eux ne lui dit une parole, car ils voyaient que sa douleur était grande.
\Chap{3}
\TextTitle{Lamentations de Job}
\VerseOne{}Après cela, Job ouvrit la bouche et maudit le jour de sa naissance.\FTNT{Job 10:18 ; Jé. 20:14.}
\VS{2}Et prenant la parole, Job dit :
\VS{3}Périsse le jour où je suis né, et la nuit qui a dit : Un homme est conçu !
\VS{4}Que ce jour se change en ténèbres, que Dieu ne s'en soucie pas d'en haut, et qu'aucune lumière ne brille sur lui !
\VS{5}Que les ténèbres et l'ombre de la mort\FTNT{Job 10:21-22.} s'en emparent, que des nuées établissent leur demeure sur lui, et que des obscurités le terrifient !
\VS{6}Que l'obscurité prenne cette nuit, qu'elle ne se réjouisse pas au milieu des jours de l'année, qu'elle n'entre pas dans le compte des mois !
\VS{7}Voici, que cette nuit devienne stérile, et qu'aucun cri de joie n'y survienne !
\VS{8}Qu’ils la maudissent ceux qui maudissent les jours, ceux qui sont prêts à réveiller le Léviathan !
\VS{9}Que les étoiles de son crépuscule soient obscurcies, qu'elle attende en vain la lumière, et qu'elle ne voie pas les paupières de l'aurore\FTNT{Job 41:9.} !
\VS{10}Parce qu'elle n'a pas fermé le sein qui me conçut ni caché la souffrance à mes yeux.
\VS{11}Pourquoi ne suis-je pas mort dans le sein de ma mère ? Pourquoi n'ai-je pas expiré au sortir de ses entrailles ?\FTNT{Job 10:18.}
\VS{12}Pourquoi ai-je trouvé des genoux pour me recevoir, et des mamelles pour m'allaiter ?
\VS{13}Je serais couché maintenant, je serais tranquille, je dormirais, je me reposerais\FTNT{Job 17:16.},
\VS{14}avec les rois et les grands de la terre, qui se bâtissent des mausolées,
\VS{15}avec les princes qui ont de l'or, et qui remplissent d'argent leurs maisons.
\VS{16}Ou je n'existerais pas, je serais comme un avorton caché\FTNT{Ps. 58:9.}, comme les petits enfants qui n'ont pas vu la lumière.
\VS{17}Là ne s'agitent plus les méchants, et là se reposent ceux qui sont fatigués et sans force ;
\VS{18}les captifs sont tous en paix, ils n'entendent pas la voix de l'oppresseur ;
\VS{19}le petit et le grand sont là, et l'esclave est délivré de son maître.
\VS{20}Pourquoi donne-t-il la lumière à celui qui souffre, et la vie à ceux qui ont l'amertume dans l’âme,
\VS{21}qui espèrent en vain la mort, et qui la recherchent plus qu'un trésor,\FTNT{Ap. 9:6.}
\VS{22}qui seraient contents jusqu'à l'allégresse et ravis de joie s'ils trouvaient le tombeau ?
\VS{23}A l'homme qui ne sait où aller et que Dieu cerne de toutes parts ?\FTNT{Job 19:8 ; La. 3:7.}
\VS{24}Mes soupirs sont ma nourriture, et mes cris se répandent comme l'eau.
\VS{25}Ce que je crains m'arrive ; ce que je redoute, c'est ce qui m'atteint.
\VS{26}Je n'ai ni paix, ni tranquillité, ni repos, le trouble est venu.
\Chap{4}
\TextTitle{Premier discours d'Eliphaz}
\VerseOne{}Eliphaz de Théman prit la parole et dit :
\VS{2}Si l'on tente de te parler, en seras-tu peiné ? Mais qui pourrait retenir ses paroles ?
\VS{3}Voici, tu as souvent instruit les autres, et tu as fortifié les mains affaiblies\FTNT{Es. 35:3 ; Hé. 12:12.},
\VS{4}tes paroles ont relevé ceux qui chancelaient, et tu as affermi les genoux qui pliaient\FTNT{Job 16:5.}.
\VS{5}Et maintenant que le malheur t'arrive, tu faiblis ! Maintenant que tu es atteint, tu en es tout troublé !
\VS{6}Ta crainte de Dieu n'est-elle pas ton soutien ? Ton espérance, n'est-ce pas ton intégrité ?
\VS{7}Cherche dans ton souvenir : Quel est l'innocent qui a péri ? Quels sont les justes qui ont été exterminés ?\FTNT{Job 8:20.}
\VS{8}J'ai vu que ceux qui labourent l'iniquité et qui sèment la peine en moissonnent les fruits ;\FTNT{Job 15:35 ; Ga. 6:7.}
\VS{9}ils périssent par le souffle de Dieu, et ils sont consumés par le vent de ses narines.\FTNT{Ex. 15:8 ; Job 15:30 ; Es. 11:4 ; 30:33. 2 Th. 2:8.}
\VS{10}Le rugissement des lions prend fin, et les dents du lionceau sont brisées ;
\VS{11}le lion périt faute de proie, et les petits de la lionne sont dispersés.
\VS{12}Une parole m'est furtivement arrivée, et mon oreille en a saisi les sons légers.
\VS{13}Au moment où les visions de la nuit agitent la pensée, quand un profond sommeil tombe sur les hommes\FTNT{Job 33:15.},
\VS{14}une frayeur et un tremblement me saisirent, et tous mes os tremblèrent.
\VS{15}Un esprit passa devant moi... Tous mes cheveux se hérissèrent...
\VS{16}Une figure d'un aspect inconnu était devant mes yeux, et j'entendis une voix qui murmurait doucement :
\VS{17}L'homme serait-il juste devant Dieu ? L'homme serait-il pur devant celui qui l'a fait ?\FTNT{Job 25:4.}
\VS{18}Si Dieu n'a pas confiance en ses serviteurs, s'il trouve de la folie chez ses anges,\FTNT{Job 15:15 ; Job 25:5 ; 2 Pi 2:4. }
\VS{19}combien plus chez ceux qui habitent des maisons d'argile, qui ont leurs fondements dans la poussière, qu'on écrase comme des vermisseaux !\FTNT{Job 25:6.}
\VS{20}Du matin au soir ils sont brisés, sans qu'on y prenne garde, ils périssent pour toujours ;
\VS{21}le fil\FTNT{Ec. 12:8.} de leur vie est coupé, ils meurent, et ils n'ont pas acquis la sagesse.
\Chap{5}
\VerseOne{}Crie maintenant ! Y aura-t-il quelqu'un qui te réponde ? Et vers quel saint te tourneras-tu ?\FTNT{Job 15:15.}
\VS{2}La colère tue l'insensé, et le fou meurt dans ses emportements.
\VS{3}J'ai vu l'insensé prendre racine\FTNT{Jé. 12:1-2.} ; mais soudain, j'ai maudit sa demeure.
\VS{4}Ses fils sont loin de tout secours ; ils sont écrasés à la porte, et personne ne les délivre !\FTNT{Ps. 119:155.}
\VS{5}Sa moisson est dévorée par des affamés, qui viennent l'enlever jusque dans les épines, et ses biens sont engloutis par des hommes altérés.
\VS{6}Le malheur ne sort pas de la poussière, et la peine ne germe pas du sol ;
\VS{7}l'homme naît pour la peine\FTNT{Ge. 3:17-19 ; Job 14:1-5.}, comme l'étincelle pour voler et s'élever.
\VS{8}Mais moi, j'aurais recours à Dieu, et j'adresserais ma parole à Dieu.
\VS{9}Il fait de grandes choses qu'on ne peut sonder, de merveilleuses choses qu'on ne peut compter\FTNT{Job 9:10. Ps. 72:18. Ps. 92:5.}.
\VS{10}Il répand la pluie sur la terre, et envoie les eaux sur les campagnes\FTNT{De. 28:12 ; Ps. 135:7 ; Job 28:26; Job 38:25-26 ; Ac. 14:17.} ;
\VS{11}il met en haut ceux qui sont abaissés, et délivre les affligés\FTNT{1 S. 2:7; Ps. 113:7-8;  Ez. 21:31.} ;
\VS{12}il anéantit les projets des hommes rusés, et leurs mains ne peuvent les accomplir\FTNT{Né. 4:15; Ps. 33:10; Es. 8:10.} ;
\VS{13}il prend les sages dans leurs propres ruses\FTNT{1 Co. 3:19.}, et les desseins des hommes pervers sont renversés :
\VS{14}Ils rencontrent les ténèbres au milieu du jour, ils tâtonnent en plein midi comme dans la nuit\FTNT{De. 28:29.}.
\VS{15}Ainsi Dieu délivre le pauvre de l'épée de leur bouche, et le sauve de la main des puissants\FTNT{Ps. 12:3-4; Ps 52:2; Ps. 57:4.} ;
\VS{16}et l'espérance soutient le malheureux\FTNT{(12)	1S. 2:8.}, et la méchanceté a la bouche fermée\FTNT{Ps. 63: 11; Ps. 107: 42; Pr. 10:6; Es. 52:15.}.
\VS{17}Heureux l'homme que Dieu châtie ! Ne méprise donc pas la correction du Tout-Puissant\FTNT{Ps 94:12; Pr.3:11-12; Hé. 12:5-6; Ap. 3:19.}.
\VS{18}Il fait la plaie, et la bande ; il blesse et sa main guérit\FTNT{De. 32:39; 1S. 2: 6-7; Os. 6:1; Cp. Es. 30:26.}.
\VS{19}Six fois il te délivrera de l'angoisse, et sept fois le mal ne te touchera pas\FTNT{Ps 34:20; Ps. 91:3; Pr.24:16.}.
\VS{20}Il te sauvera de la mort pendant la famine, et du tranchant de l'épée pendant la guerre\FTNT{Ps. 33: 19; Ps. 37:19.}.
\VS{21}Tu seras à l'abri du fléau de la langue, tu seras sans crainte quand viendra la dévastation\FTNT{Ps. 31:21.}.
\VS{22}Tu riras de la dévastation et de la famine, et tu n'auras pas peur des bêtes de la terre\FTNT{Es. 65:25; Ez. 34:25; Os. 2:20.};
\VS{23}car tu feras une alliance avec les pierres des champs, et les bêtes des champs seront en paix avec toi\FTNT{Os. 2:20.}.
\VS{24}Tu jouiras en paix de la prospérité sous ta tente, tu pourvoiras à ta demeure et tu n'y seras point trompé ;
\VS{25}tu verras ta postérité s'accroître, et tes descendants se multiplier comme l'herbe de la terre\FTNT{Ps. 72:16; Ps. 127: 3-5; Ps. 128:6.}.
\VS{26}Tu entreras au tombeau dans ta vieillesse, comme une gerbe qu'on emporte en son temps\FTNT{Pr. 9:11; Pr. 10:27.}.
\VS{27}Voilà ce que nous avons examiné, voilà ce qui est ; à toi d'entendre et de choisir.
\Chap{6}
\TextTitle{Réponse de Job}
\VerseOne{}Job prit la parole et dit :
\VS{2}Oh ! S'il était possible de peser ma douleur, et si toutes mes calamités étaient ensemble sur la balance,
\VS{3}elles seraient plus pesantes que le sable de la mer ; voilà pourquoi mes paroles vont jusqu'à la folie\FTNT{Pr. 27:3.} !
\VS{4}Car les flèches du Tout-Puissant sont sur moi, mon âme en boit le venin ; les terreurs\FTNT{Job 30:15 ; Ps. 88:16-17.} de Dieu se rangent en bataille contre moi\FTNT{Job 19:12 ; Ps. 38:2-3.}.
\VS{5}L'âne sauvage\FTNT{Job 39:8.} crie-t-il auprès de l'herbe ? Le bœuf mugit-il auprès de son fourrage ?
\VS{6}Peut-on manger ce qui est fade et sans sel ? Trouve-t-on du goût dans un blanc d'œuf ?
\VS{7}Ce que mon âme voudrait ne pas toucher, c'est là ma nourriture, si dégoûtante soit-elle !
\VS{8}Oh ! Puisse ma prière s'accomplir et Dieu me donner ce que j'attends !
\VS{9}Qu'il plaise à Dieu de m'écraser, qu'il étende sa main pour m'achever !\FTNT{Job 7:16; 9:21; 10:1; cp. No. 11:15; 1R. 19:4; Jon. 4:3, 8.}
\VS{10}Il me restera du moins une consolation, une joie dans les maux dont il m'accable : Jamais je n'ai transgressé les paroles du Saint.
\VS{11}Pourquoi espérer quand je n'ai plus de force ? Pourquoi attendre la vie quand ma fin est certaine ?
\VS{12}Ma force est-elle une force de pierre ? Ma chair est-elle d'airain ?
\VS{13}Ne suis-je pas sans secours, et le salut n'est-il pas loin de moi ?
\VS{14}Celui qui souffre a droit à la compassion de son ami, même quand il abandonnerait la crainte\FTNT{Ps. 19:10.} du Tout-Puissant\FTNT{Pr. 17:17.}.
\VS{15}Mes frères m'ont trompé comme un torrent, comme le lit des torrents qui passent\FTNT{Ps. 38:12; Ps 41:10; Ps 69:9; Jé. 15:19.}.
\VS{16}Les glaçons en troublent le cours, la neige s'y cache ;
\VS{17}mais au temps de la sécheresse, ils tarissent, et dans les chaleurs, ils disparaissent de leur place.
\VS{18}Les caravanes se détournent de leur route, elles montent dans le désert et périssent.
\VS{19}Les caravanes de Théma\FTNT{Ge. 25:15.} fixent le regard, les voyageurs de Séba\FTNT{1R. 10:1; Ps. 72:10; Ez. 27:22-23.} s’attendent à eux ;
\VS{20}ils sont honteux d'avoir eu cette confiance, ils restent confondus quand ils arrivent.
\VS{21}Certes, vous m'êtes devenus inutiles ; vous voyez mon angoisse, et vous en avez horreur !\FTNT{Job 19:13 ; Ps. 31:12.}
\VS{22}Vous ai-je dit : Donnez-moi quelque chose, faites en ma faveur des présents avec vos biens,
\VS{23}délivrez-moi de la main de l'ennemi, et rachetez-moi de la main des violents ?
\VS{24}Instruisez-moi, et je me tairai ; faites-moi comprendre en quoi je me suis égaré.
\VS{25}Ô combien sont fortes les paroles de vérité ! Mais que veut censurer votre argumentation ?
\VS{26}Voulez-vous donc blâmer ce que j'ai dit, et ne voir que du vent dans les paroles d'un homme désespéré ?\FTNT{Ec. 9:16.}
\VS{27}Vous accablez un orphelin, vous persécutez votre ami.
\VS{28}Regardez-moi, je vous prie ! Vous mentirais-je en face ?
\VS{29}Revenez\FTNT{Job 17:10.} donc, ne soyez pas injustes ; revenez, et reconnaissez mon innocence\FTNT{Job 27:5-6 ; 34:5 ; cp. Job 23:10 ; 42:1-6.}.
\VS{30}Y a-t-il de l'iniquité sur ma langue, et ma bouche ne discerne-t-elle pas le mal ?
\Chap{7}
\VerseOne{}N'y a-t-il pas un temps de guerre limité à l'homme sur la terre ? Et ses jours ne sont-ils pas comme les jours d'un mercenaire ?
\VS{2}Comme un esclave, il soupire après l'ombre, comme un mercenaire\FTNT{Es. 16:14.}, il attend son salaire\FTNT{Ps. 39:5.}.
\VS{3}Ainsi j'ai reçu en partage des mois en vain, et l'on m'a assigné des nuits de peine\FTNT{Ps. 6:6.}.
\VS{4}Je me couche et je dis : Quand me lèverai-je ? Quand finira la nuit ? Et je suis rassasié d'agitations jusqu'au point du jour\FTNT{De. 28:67.}.
\VS{5}Ma chair se couvre de vers et d'une croûte terreuse, ma peau se crevasse et coule.
\VS{6}Mes jours sont plus rapides que la navette du tisserand, ils se consument sans espoir !\FTNT{Job 9:25 ; 17:11 ; Es. 38:12 ; Ja. 4:14.}
\VS{7}Souviens-toi que ma vie est un souffle ! Et que mes yeux ne reverront plus le bonheur\FTNT{Job 8:9; Job 14:1-2; Ps. 78:39; Ps. 89:48; Ps. 102: 12; Ps. 103:15; Es. 40:6; 1P. 1:24.}.
\VS{8}L'œil qui me regarde ne me verra plus ; ton œil me cherchera, et je ne serai plus.
\VS{9}La nuée se dissipe et s'en va, ainsi celui qui descend au scheol\FTNT{cp. Ha. 2:5 ; Lu. 16:23.} ne remontera pas\FTNT{Job 10:21-22 ; Job 14:7-14.};
\VS{10}il ne reviendra plus dans sa maison, et le lieu qu'il habitait  ne le reconnaîtra plus\FTNT{Job 10:21 ; Ps. 37:35-36 ; Ps. 103:16.}.
\VS{11}C'est pourquoi, je ne retiendrai pas ma bouche, je parlerai dans l'angoisse de mon esprit, je me plaindrai dans l'amertume de mon âme\FTNT{Job 10:1.}.
\VS{12}Suis-je une mer ? Suis-je un monstre marin, pour que tu poses autour de moi des gardes ?
\VS{13}Quand je dis : Mon lit me consolera, ma couche calmera ma plainte,
\VS{14}alors tu me terrifies par des songes, et tu m'épouvantes par des visions.
\VS{15}Ah ! Je voudrais être étranglé ! Je voudrais la mort plutôt que ces os !
\VS{16}Je les méprise !... Je ne vivrai pas toujours... Laisse-moi car mes jours sont un souffle\FTNT{Job 10:20.}.
\VS{17}Qu'est-ce que l'homme pour que tu en fasses tant de cas, pour que tu poses ta main sur son cœur,\FTNT{Ps. 8:5 ; Ps. 144:3 ; Hé. 2:6.}
\VS{18}pour que tu le visites tous les matins, pour que tu l'éprouves\FTNT{Job 23:10.} à chaque instant ?
\VS{19}Quand cesseras-tu d'avoir le regard sur moi? Quand me laisseras-tu le temps d'avaler ma salive ?\FTNT{Job 9:18.}
\VS{20}Si j'ai péché, qu'ai-je pu te faire, gardien des hommes\FTNT{1 Ti. 4:10.} ? Pourquoi me mettre en butte à tes traits? Pourquoi me rendre à charge à moi-même ?
\VS{21}Et pourquoi ne pardonnes-tu pas mon péché, et ne fais-tu pas passer mon iniquité ? Car je vais maintenant me coucher dans la poussière ; tu me chercheras, et je ne serai plus.
\Chap{8}
\TextTitle{Premier discours de Bildad}
\VerseOne{}Bildad de Schuach prit la parole et dit :
\VS{2}Jusqu'à quand parleras-tu ainsi, et les paroles de ta bouche seront-elles un vent impétueux ?\FTNT{Job 15:2.}
\VS{3}Dieu renverserait-il le droit ? Le Tout-Puissant renverserait-il la justice\FTNT{Cp. Ge. 18:25.} ?\FTNT{Job 34:12 ; Dé 32:4 ; 2 Ch. 19:7 ; Da. 9:14.}
\VS{4}Si tes fils ont péché contre lui, il les a livrés à leur crime.
\VS{5}Mais toi si tu cherches Dieu, si tu demandes grâce au Tout-Puissant ;\FTNT{Cp. Job 5:17-27.}
\VS{6}si tu es pur et droit, il veillera certainement sur toi, il rendra le bonheur à la demeure de ta justice ;
\VS{7}tes commencements\FTNT{Za. 4:10.} auront été peu de chose, et ta fin sera bien plus grande.\FTNT{Job 42:12.}
\VS{8}Interroge ceux des générations précédentes, applique-toi à l'expérience de leurs pères.\FTNT{De. 4:32 ; De. 32:7.}
\VS{9}Car nous sommes d'hier, et nous ne savons rien, car nos jours sur la terre ne sont qu'une ombre.\FTNT{1 Ch. 29:15 ; Ps. 102:12 ; Ps. 144:4.}
\VS{10}Ils t'instruiront, ils te parleront, ils tireront de leur cœur ces discours :
\VS{11}Le roseau croît-il sans marais ? Le jonc pousse-t-il sans eau ?
\VS{12}Encore vert et sans qu'on le coupe, il sèche plus vite que toutes les herbes.\FTNT{Ps. 129:6 ; Cp. Jé. 17:5-8.}
\VS{13}Ainsi est la voie de tous ceux qui oublient Dieu\FTNT{Ps. 9:18.}, et l'espérance de l'impie périra\FTNT{Job 11:20 ; Job 27:8 ; Ps. 1:4; Ps. 112:10 ; Pr. 10:28.}.
\VS{14}Sa confiance est brisée, son soutien est une toile d'araignée.
\VS{15}Il s'appuie sur sa maison, et elle ne tient pas ; il s'y cramponne, et elle ne reste pas debout.
\VS{16}Dans toute sa vigueur, en plein soleil, il étend ses rameaux sur son jardin,
\VS{17}mais ses racines s'entrelacent parmi des monceaux de pierres, il pénètre dans les rochers.
\VS{18}L'arrache-t-on du lieu qu'il occupe ? Ce lieu le renie : Je ne t'ai point connu !
\VS{19}Telle est la joie que ses voies lui procurent. Puis sur le même sol, d'autres s'élèvent après lui.
\VS{20}Dieu ne rejette pas l'homme intègre, il ne soutient pas la main des méchants.\FTNT{Job 4:7.}
\VS{21}Il remplira ta bouche de cris de joie, et tes lèvres de chants d'allégresse.\FTNT{Ps. 126:2.}
\VS{22}Tes ennemis seront couverts de honte ; la tente des méchants ne sera plus.\FTNT{Ps. 35:26 ; Ps. 109:29.}
\Chap{9}
\TextTitle{Réponse de Job}
\VerseOne{}Job prit la parole et dit :
\VS{2}Certainement, je sais qu'il en est ainsi ; et comment l'homme serait-il juste\FTNT{Ha. 2:4 ; Ga. 3:11 ; Ro. 1:17 ; Hé. 10:38.} devant Dieu ?\FTNT{Ps. 25:4 ; Ps. 143:2 ; Job 15:14-16 ; Da. 9:11 ; Ro 3:19.}
\VS{3}S'il voulait contester avec lui, sur mille choses il ne pourrait répondre à une seule.\FTNT{Es. 45:9-10.}
\VS{4}A lui la sagesse et la toute-puissance : Qui lui résisterait impunément ?\FTNT{Job 12:13 ; Job 36:5 ; Job 37:23.}
\VS{5}Il transporte les montagnes, il les renverse dans sa fureur.\FTNT{Ps. 144:5.}
\VS{6}Il fait trembler la terre sur sa base, et ses colonnes sont ébranlées.\FTNT{Ag. 2:6, 21; Hé. 12:26.}
\VS{7}Il commande au soleil, et le soleil ne se lève pas ; et il met un sceau sur les étoiles.\FTNT{Jos. 10:12.}
\VS{8}Seul, il étend les cieux\FTNT{Ge 1:6-8; Ps. 104:2; Es. 44:24; Es. 51:13.}, il marche sur les hauteurs de la mer\FTNT{Cp. Mt. 14:25.}.
\VS{9}Il a créé la Grande Ourse, l'Orion, les Pléiades, et les étoiles des régions australes.\FTNT{Ge. 1:16 ; Job 38:31-32 ; Ps. 89:12 ; Am. 5:8.}
\VS{10}Il fait de grandes choses qu'on ne peut sonder, des merveilles sans nombre.\FTNT{Job 5:9; Job 37:5 ; Ps. 86:10; Ps. 139:6, 17-18.}
\VS{11}Voici, il passe près de moi, et je ne le vois pas ; il passe encore, et je ne l'aperçois pas.\FTNT{Job 23:8-9 ; 35:14.}
\VS{12}S'il enlève, qui le lui fera rendre ? Qui lui dira : Que fais-tu ?\FTNT{Es. 45: 9-10 ; Da. 4:35 ; Ro. 11:33-35.}
\VS{13}Dieu ne revient pas sur sa colère ; sous lui s'inclinent les appuis de l'orgueil.\FTNT{Job 26:12; Cp. Es. 30:7.}
\VS{14}Et moi, comment lui répondre ? Quelles paroles choisir ?
\VS{15}Quand je serais juste, je ne répondrais pas ; je demanderais grâce à mon juge.\FTNT{Job 23:1-7.}
\VS{16}Et quand il m'exaucerait, si je l'invoque, je ne croirais pas qu'il eût écouté ma voix,
\VS{17}lui qui m'assaille comme par une tempête, qui multiplie mes plaies sans motif,\FTNT{Job 6:29.}
\VS{18}qui ne me permet pas de reprendre haleine ; qui me rassasie d'amertume.\FTNT{Job 7:19.}
\VS{19}Recourir à la force ? Il est tout-puissant. A la justice ? Qui me fera comparaître ?
\VS{20}Si je me justifie, ma propre bouche me condamnera ; si je me fais parfait, il me convaincra d'être coupable.
\VS{21}Je suis innocent ! Je ne me soucie pas de vivre, je méprise ma vie.\FTNT{Job 10:1.}
\VS{22}Qu'importe après tout ? Car j'ose le dire, il détruit l'innocent comme l'impie.\FTNT{Ec. 9:2-3 ; Cp. Ez. 21:3 ; Mt 5:45.}
\VS{23}Si du moins le fléau donnait soudainement la mort !... Mais il se rit des épreuves de l'innocent.
\VS{24}La terre est livrée aux mains de l'impie ; il couvre la face des juges. Si ce n'est pas lui, qui est-ce donc ?
\VS{25}Mes jours ont été plus légers qu'un courrier ; ils se sont enfuis sans voir le bonheur ;\FTNT{Job 7:6-7.}
\VS{26}ils passent comme les navires de roseaux, comme l'aigle qui fond sur sa proie.
\VS{27}Si je dis : Je veux oublier ma plainte, laisser ma tristesse, reprendre courage,
\VS{28}je suis effrayé de toutes mes douleurs. Je sais que tu ne me tiendras pas pour innocent.\FTNT{Cp. Ps. 130:3.}
\VS{29}Je serai jugé coupable ; pourquoi me fatiguer en vain ?
\VS{30}Quand je me laverais dans l'eau des neiges, quand je purifierais mes mains avec du savon,\FTNT{Jé. 2:22.}
\VS{31}tu me plongerais dans le fossé, et mes vêtements m'auraient en horreur.
\VS{32}Il n'est pas un homme comme moi, pour que je lui réponde, pour que nous allions ensemble en justice.\FTNT{Ec. 6:10; Es. 45:9 ; Jé 49:19 ; Ro. 9:20.}
\VS{33}Il n'y a pas d'arbitre entre nous, qui pose sa main sur nous deux.\FTNT{Cp. 1 S. 2:25.}
\VS{34}Qu'il ôte sa verge de dessus moi, et que ses terreurs ne me troublent plus ;
\VS{35}alors je parlerai sans crainte. Autrement, je ne suis pas à moi-même.
\Chap{10}
\VerseOne{}Mon âme a pris en dégoût la vie ! Je laisserai aller ma plainte, je parlerai dans l'amertume de mon âme.
\VS{2}Je dis à Dieu : Ne me condamne pas ! Fais-moi savoir pourquoi tu me prends à partie !
\VS{3}Te paraît-il bien de maltraiter, de repousser l'ouvrage de tes mains, et de faire briller ta faveur sur le conseil des méchants ?
\VS{4}As-tu des yeux de chair ? Vois-tu comme voit un homme ?
\VS{5}Tes jours sont-ils comme les jours de l'homme, et tes années comme ses années,
\VS{6}pour que tu recherches mon iniquité, pour que tu t'enquières de mon péché,
\VS{7}sachant bien que je ne suis pas coupable, et que nul ne peut me délivrer de ta main ?
\VS{8}Tes mains m'ont formé, elles m'ont fait tout entier... Et tu me détruirais !
\VS{9}Souviens-toi donc que tu m'as formé comme de l'argile ; voudrais-tu de nouveau me faire retourner à la poussière ?
\VS{10}Ne m'as-tu pas coulé comme du lait, et caillé comme du fromage ?
\VS{11}Tu m'as revêtu de peau et de chair, tu m'as tissé d'os et de nerfs ;
\VS{12}tu m'as accordé ta grâce avec la vie, tu m'as conservé le souffle par tes soins et sous ta garde.
\VS{13}Voici néanmoins ce que tu cachais dans ton cœur, voici, je le sais, ce que tu as résolu en toi-même.
\VS{14}Si je pèche, tu m'observes, tu ne pardonnes pas mon iniquité.
\VS{15}Suis-je coupable, malheur à moi ! Suis-je juste, je n'ose lever la tête plus haut, rassasié de honte et tu me vois absorbé dans ma propre misère.
\VS{16}Si j'ose redresser la tête, tu me poursuis comme un lion, tu me frappes encore par des prodiges.
\VS{17}Tu m'opposes de nouveaux témoins, tu multiplies tes fureurs contre moi, tu m'assailles d'une succession de calamités.
\VS{18}Mais pourquoi m'as-tu fait sortir du sein de ma mère ? Je serais mort, et aucun œil ne m'aurait vu ;
\VS{19}je serais comme si je n’avais pas existé, et j'aurais été porté du sein maternel au tombeau.
\VS{20}Mes jours ne sont-ils pas en petit nombre ? Qu'il arrête et me laisse ! Qu'il pose sa main, et que je respire un peu.
\VS{21}Avant que je m'en aille, pour ne plus revenir, dans la terre des ténèbres et de l'ombre de la mort,
\VS{22}terre d'une obscurité profonde, où règnent l'ombre de la mort et la confusion, et où la lumière est semblable aux ténèbres.
\Chap{11}
\TextTitle{Première accusation de Tsophar}
\VerseOne{}Tsophar de Naama prit la parole et dit :
\VS{2}Cette multitude de paroles ne trouvera-t-elle point de réponse, et suffira-t-il d'être un discoureur pour être justifié ?
\VS{3}Tes vains propos feront-ils taire les gens?  Te moqueras-tu sans que personne ne te confonde ?
\VS{4}Tu dis : Mon enseignement est pur, et je suis pur à tes yeux.
\VS{5}Oh ! Si Dieu voulait parler, s'il ouvrait les lèvres pour te répondre,
\VS{6}et s'il t'annonçait les secrets de sa sagesse, de son immense sagesse ; tu reconnaîtrais alors qu'il ne te traite pas selon ton iniquité.
\VS{7}Prétends-tu sonder Dieu ? Parviendras-tu à trouver la limite du Tout-Puissant ?
\VS{8}Ce sont les hauteurs des cieux : Qu'y feras-tu ? C'est plus profond que le scheol : Qu'y connaîtras-tu ?
\VS{9}La mesure en est plus longue que la terre, elle est plus large que la mer.
\VS{10}S'il passe, s'il emprisonne, s'il traîne à son tribunal, qui le fera revenir en arrière ?
\VS{11}Car il connaît les hommes vicieux, il discerne par le regard les coupables.
\VS{12}L'homme, au contraire, a l'intelligence d'un fou, il est né comme le petit d'un âne sauvage.
\VS{13}Dispose bien ton cœur vers Dieu, étends tes mains vers lui,
\VS{14}éloigne de tes mains l'iniquité, et ne laisse pas demeurer l'injustice sous ta tente.
\VS{15}Alors certainement tu lèveras ton front sans tache, tu seras ferme et sans crainte ;
\VS{16}tu oublieras tes peines, tu t'en souviendras comme des eaux écoulées.
\VS{17}Tes jours se lèveront comme le soleil à son midi, tes ténèbres seront comme la lumière du matin.
\VS{18}Tu seras plein de confiance, et ton attente ne sera plus vaine; tu regarderas autour de toi, et tu te coucheras en sécurité.
\VS{19}Tu t'étendras à ton aise et nul ne t'effrayera, et plusieurs caresseront ton visage.
\VS{20}Mais les yeux des méchants seront consumés ; tout refuge leur sera ôté ; la mort de l'âme, voilà leur espérance !
\Chap{12}
\TextTitle{Réplique de Job}
\VerseOne{}Job prit la parole et dit :
\VS{2}On dirait vraiment que vous êtes tout un peuple, et qu'avec vous doit mourir la sagesse.
\VS{3}J'ai pourtant du sens aussi bien que vous, je ne vous suis pas inférieur ; et qui ne sait les choses que vous dîtes ?
\VS{4}Je suis pour mes amis un objet de raillerie, quand je m'écrie à Dieu pour qu'il me réponde ; le juste, l'innocent, un objet de raillerie !
\VS{5}Mépris au malheur ! Telle est la pensée des heureux ; le mépris est réservé à ceux dont le pied chancelle !
\VS{6}Il y a la paix sous la tente des pillards, sécurité pour ceux qui offensent Dieu, pour quiconque se fait un faux dieu de sa main.
\VS{7}Mais interroge donc les bêtes, et elles t'instruiront, les oiseaux des cieux, ils te l'annonceront ;
\VS{8}parle à la terre, elle t'instruira ; et les poissons de la mer te le raconteront.
\VS{9}Qui ne reconnaît chez eux la preuve que la main de Yahweh a façonné toutes choses ?
\VS{10}Car c’est lui dans  la main duquel est l'âme de tous les vivants, le souffle de toute chair humaine.
\VS{11}L'oreille ne discerne-t-elle pas les paroles, comme le palais savoure les aliments ?
\VS{12}Dans les vieillards se trouve la sagesse, et dans une longue vie l'intelligence.
\VS{13}En Dieu résident la sagesse et la puissance. C'est à lui qu'appartiennent le conseil et l'intelligence.
\VS{14}Voici, il démolit, et on ne rebâtit pas, celui qu'il enferme ne sera pas délivré.
\VS{15}Il retient les eaux et tout se dessèche ; il les lâche, et la terre en est dévastée.
\VS{16}Il possède la puissance et la sagesse ; il maîtrise celui qui s'égare ou fait égarer les autres.
\VS{17}Il fait marcher pieds nus les conseillers ; il frappe de folie les juges.
\VS{18}Il libère du châtiment des rois, et il met une ceinture autour de leurs reins.
\VS{19}Il fait marcher pieds nus les sacrificateurs ; et il renverse les puissants.
\VS{20}Il ôte la parole à ceux qui ont de l'assurance ; il prive de jugement les vieillards.
\VS{21}Il verse le mépris sur les nobles ; il relâche la ceinture des forts.
\VS{22}Il met à découvert les profondeurs cachées dans les ténèbres, il amène à la lumière l'ombre de la mort.
\VS{23}Il agrandit les nations, et il les perd ; il étend les nations au loin, et il les ramène dans leurs limites.
\VS{24}Il ôte l'intelligence aux chefs des peuples de la terre, il les fait errer dans les déserts sans chemin ;
\VS{25}ils tâtonnent dans les ténèbres, sans aucune clarté, il les fait errer comme un homme ivre.
\Chap{13}
\VerseOne{}Voici, mon œil a vu tout cela, mon oreille l'a entendu et compris.
\VS{2}Ce que vous savez, je le sais aussi, je ne vous suis pas inférieur.
\VS{3}Mais je veux parler au Tout-Puissant, je veux plaider ma cause devant Dieu.
\VS{4}Mais vous, vous n'imaginez que des mensonges. Vous êtes tous des médecins de néant.
\VS{5}Puissiez-vous garder le silence, et que ce soit là votre sagesse.
\VS{6}Écoutez, je vous prie, ma réprimande, et soyez attentifs à la réplique de mes lèvres.
\VS{7}Direz-vous en faveur de Dieu ce qui est injuste, et pour le soutenir direz-vous des faussetés ?
\VS{8}Voulez-vous vous lever en sa présence ? Voulez-vous plaider pour Dieu?
\VS{9}S'il vous sonde, vous trouvera-t-il bons ? Ou le tromperez-vous comme on trompe un homme ?
\VS{10}Certainement, il vous reprendra, si secrètement vous faites acception de personnes.
\VS{11}Sa majesté ne vous épouvantera-t-elle pas ? Sa terreur ne tombera-t-elle pas sur vous ?
\VS{12}Vos sentences sont des sentences de cendre, vos retranchements sont des retranchements de boue !
\VS{13}Taisez-vous, laissez-moi, je veux parler ! Il m'en arrivera ce qu'il pourra.
\VS{14}Pourquoi prendrais-je ma chair entre les dents ? J'exposerai plutôt ma vie.
\VS{15}Voici, il me tuera ; je n'ai rien à espérer ; mais devant lui, je défendrai ma conduite.
\VS{16}Cela même peut servir à mon salut, car un impie ne viendrait pas devant lui.
\VS{17}Écoutez, écoutez mes paroles, prêtez l'oreille à mes déclarations.
\VS{18}Me voici prêt à préparer ma cause ; je sais que je serai justifié.
\VS{19}Quelqu'un plaidera-t-il contre moi ? Alors je me tais et je veux mourir.
\VS{20}Seulement, ne me fais pas deux choses, et je ne me cacherai pas loin de ta face :
\VS{21}Retire ta main loin de moi, et que tes terreurs ne me troublent plus.
\VS{22}Puis appelle et je répondrai, ou si je parle, réponds-moi !
\VS{23}Quel est le nombre de mes iniquités et de mes péchés ? Fais-moi connaître mes transgressions et mes péchés.
\VS{24}Pourquoi caches-tu ta face et me prends-tu pour ton ennemi ?
\VS{25}Veux-tu effrayer une feuille agitée ? Veux-tu poursuivre une paille desséchée ?
\VS{26}Pourquoi écrire contre moi d'amères souffrances, me faire périr pour des péchés de ma jeunesse ?
\VS{27}Pourquoi mettre mes pieds dans des ceps, surveiller tous mes chemins, tracer une limite autour de la plante de mes pieds,
\VS{28}quand mon corps tombe en pourriture, comme un vêtement que dévore la teigne ?
\Chap{14}
\VerseOne{}L'homme né de la femme ! Sa vie est courte, sans cesse rassasiée de troubles.
\VS{2}Comme une fleur, il éclot, et on le coupe ; il fuit comme une ombre et ne subsiste pas.
\VS{3}Et c'est sur cet être que tu ouvres les yeux ! Et tu me fais aller en justice avec toi !
\VS{4}Comment d'un être souillé sortira-t-il un homme pur ? Il n'en peut sortir personne.
\VS{5}Si ses jours sont fixés, si tu as compté ses mois, si tu en as fait des limites qu'il ne saurait passer,
\VS{6}détourne de lui les regards, et donne-lui du relâche, pour qu'il ait au moins la joie du mercenaire à la fin de sa journée.
\VS{7}Car il y a de l'espérance pour l'arbre ; si on l'a coupé, il repousse, il ne cesse pas d'avoir des rejetons,
\VS{8}quand sa racine a vieilli dans la terre, quand son tronc meurt dans la poussière,
\VS{9}dès qu'il sent l'eau, il pousse de nouveau, il produit des branches comme une jeune plante.
\VS{10}Mais l'homme meurt, et il perd sa force ; l'homme expire, et où est-il ?
\VS{11}Les eaux de la mer s'évanouissent, les fleuves tarissent et se dessèchent,
\VS{12}ainsi l'homme se couche, et il ne se relève plus, il ne se réveillera pas tant que les cieux ne se réveilleront pas, il ne sortira pas de son sommeil.
\VS{13}Oh que tu me caches dans le  scheol, que tu me gardes à l’abri jusqu’à ce que ta colère soit passée, que tu me donnes un temps arrêté, après lequel tu te souviennes de moi !
\VS{14}Si l'homme une fois mort pouvait revivre, j'aurais de l'espoir tout le temps de mes souffrances, jusqu'à ce que mon état vînt à changer.
\VS{15}Tu appellerais alors, et je répondrais, tu soupirerais après l'ouvrage de tes mains.
\VS{16}Mais aujourd'hui tu comptes mes pas, tu observes mes péchés ;
\VS{17}mes transgressions sont scellées dans un sac, et tu imagines des iniquités à ma charge.
\VS{18}La montagne s'écroule et périt, le rocher est transporté hors de sa place,
\VS{19}la pierre est broyée par les eaux, et la terre emportée par leur courant ; ainsi tu détruis l'espérance de l'homme.
\VS{20}Tu ne cesses de l'assaillir, et il s'en va ; tu le défigures, puis tu le renvoies.
\VS{21}Que ses fils soient honorés, il n'en sait rien ; qu'ils soient dans l'abaissement, il ne le verra pas.
\VS{22}C'est pour lui seul qu'il éprouve de la douleur en sa chair, c'est pour lui seul qu'il ressent de la tristesse en son âme.
\Chap{15}
\TextTitle{Deuxième discours d'Eliphaz}
\VerseOne{}Eliphaz de Théman prit la parole et dit :
\VS{2}Le sage répond-il par une science vaine ? Se remplit-il la poitrine du vent d'orient ?
\VS{3}Est-ce par des propos utiles qu'il se justifie ? Est-ce par des discours profitables ?
\VS{4}Toi, tu brises même la crainte de Dieu, tu anéantis tout mouvement de piété devant Dieu.
\VS{5}Ton iniquité dirige ta bouche, et tu as choisi le langage des rusés.
\VS{6}Ta bouche te condamne, et non pas moi ; et tes lèvres témoignent contre toi.
\VS{7}Es-tu le premier-né des hommes ? As-tu été enfanté avant les collines ?
\VS{8}As-tu entendu le conseil de Dieu ? As-tu pris la sagesse à ton profit ?
\VS{9}Que sais-tu que nous ne sachions pas ? Quelle connaissance as-tu que nous n'ayons pas ?
\VS{10}Il y a parmi nous des cheveux blancs, des vieillards, plus riches de jours que ton père.
\VS{11}Tiens-tu pour peu de choses les consolations de Dieu et les paroles qui doucement se font entendre à toi ?...
\VS{12}Où ton cœur t'emmène-t-il, et que signifie ce roulement de tes yeux ?
\VS{13}Quoi ! C'est contre Dieu que tu tournes ta colère et que tu fais sortir de ta bouche de tels discours !
\VS{14}Qu'est-ce que l'homme, pour qu'il soit pur ? Celui qui est né de la femme peut-il être juste ?
\VS{15}Si Dieu n'a pas confiance en ses saints, si les cieux ne sont pas purs à ses yeux,
\VS{16}combien moins l'être abominable et corrompu, l'homme qui boit l'iniquité comme l'eau !
\VS{17}Je vais te parler, écoute-moi ! Je raconterai ce que j'ai vu,
\VS{18}ce que les sages ont fait connaître, ce qu'ils ont révélé, l'ayant appris de leurs pères.
\VS{19}A eux seuls ce pays avait été donné, et parmi eux nul étranger n'était encore venu.
\VS{20}Le méchant passe dans l'angoisse tous les jours de sa vie, toutes les années qui sont le partage de l'impie.
\VS{21}La voix de la terreur retentit à ses oreilles ; au sein de la paix, le dévastateur vient sur lui ;
\VS{22}il ne croit pas échapper aux ténèbres, il se voit épié par l'épée qui le menace ;
\VS{23}il court çà et là pour chercher son pain, il sait que le jour des ténèbres l'attend.
\VS{24}La détresse et l'angoisse l'épouvantent, elles l'assaillent comme un roi prêt à combattre ;
\VS{25}car il a levé la main contre Dieu, il a bravé le Tout-Puissant.
\VS{26}Il a eu l'audace de courir à lui sous le dos épais de ses boucliers.
\VS{27}Il avait le visage couvert de graisse, les flancs chargés d'embonpoint ;
\VS{28}et il habite des villes détruites, des maisons que personne n’habite, sur le point de tomber en ruines.
\VS{29}Il ne s'enrichira plus, sa fortune ne s'élèvera pas, sa prospérité ne s'étendra plus sur la terre.
\VS{30}Il ne pourra pas se détourner des ténèbres, la flamme desséchera ses rejetons, et Dieu le fera disparaître par le souffle de sa bouche.
\VS{31}S'il a confiance dans la vanité, il se trompe, car la vanité sera sa récompense.
\VS{32}Elle s'accomplira avant le terme de ses jours, et ses branches ne reverdiront pas.
\VS{33}Il sera comme une vigne dépouillée de ses fruits encore verts, comme un olivier dont on fait tomber les fleurs.
\VS{34}La maison de l'impie deviendra stérile, et le feu dévorera la tente de l'homme corrompu.
\VS{35}Il conçoit le mal, et il enfante le mal, il s'établit dans son sein des fruits qui le trompent.
\Chap{16}
\TextTitle{Réponse de Job}
\VerseOne{}Job prit la parole et dit :
\VS{2}J'ai souvent entendu de pareils discours ; vous êtes tous des consolateurs fâcheux.
\VS{3}Quand finiront ces discours en l'air ? Pourquoi cette irritation dans tes réponses ?
\VS{4}Moi aussi, je pourrais parler comme vous, si vous étiez à ma place : Je vous accablerais de paroles, je hocherais la tête sur vous,
\VS{5}je vous fortifierais de ma bouche, et le mouvement de mes lèvres vous soulagerait.
\VS{6}Si je parle, ma douleur ne sera pas soulagée, si je me tais, s'en ira-t-elle ?
\VS{7}Maintenant, hélas ! Il m'a épuisé... Tu as dévasté toute ma maison ;
\VS{8}tu m'as saisi, pour témoigner contre moi ; ma maigreur se lève et m'accuse en face.
\VS{9}Il me déchire et me poursuit dans sa fureur, il grince des dents contre moi ; mon ennemi m'attaque et me perce avec ses yeux.
\VS{10}Ils ouvrent la bouche pour me dévorer, ils m'insultent et me frappent les joues, ils s'arment tous ensemble après moi.
\VS{11}Dieu me livre à la merci des impies, il me précipite entre les mains des méchants.
\VS{12}J'étais tranquille, et il m'a secoué, il m'a saisi par la nuque et m'a brisé, il m'a posé en butte à ses traits.
\VS{13}Ses traits m'environnent de toutes parts ; il me perce les reins et ne m'épargne pas, il répand ma bile sur la terre.
\VS{14}Il me fait brèche sur brèche, il court sur moi comme un guerrier.
\VS{15}J'ai cousu un sac sur ma peau, j'ai roulé ma tête dans la poussière,
\VS{16}les pleurs ont altéré mon visage, l'ombre de la mort est sur mes paupières.
\VS{17}Quoiqu'il n'y ait point de violence dans mes mains, et que ma prière fut toujours pure.
\VS{18}Ô terre, ne cache pas mon sang, et qu'il n'y ait aucun lieu où s'arrête mon cri !
\VS{19}Déjà maintenant, mon témoin est dans le ciel, mon témoin est dans les lieux élevés\FTNT{Ap. 1:5 ; Ap. 3:14.}.
\VS{20}Mes amis se moquent de moi ; c'est Dieu que j'implore avec larmes,
\VS{21}pour qu'il fasse justice entre l'homme et Dieu, entre le fils d'Adam et son semblable.
\VS{22}Car le nombre de mes années touche à son terme, et je m'en irai par un chemin d'où je ne reviendrai pas.
\Chap{17}
\VerseOne{}Mon souffle se perd, mes jours s'éteignent, le tombeau m'attend.
\VS{2}Je suis environné de moqueurs, et mon œil veille toute la nuit au milieu de leurs insultes.
\VS{3}Dépose un gage, sois ma caution auprès de toi-même ; autrement qui répondrait pour moi ?
\VS{4}Tu as fermé leur cœur à l'intelligence ; c'est pourquoi tu ne les feras pas s'élever.
\VS{5}On informe ses amis du partage du butin, et l'on a des enfants dont les yeux se consument.
\VS{6}On a fait de moi la fable des peuples, un être à qui l'on crache au visage.
\VS{7}Mon œil est obscurci par le chagrin, tous mes membres sont comme une ombre.
\VS{8}Les hommes droits en sont stupéfaits, et l'innocent se soulève contre l'impie.
\VS{9}Le juste néanmoins demeure ferme dans sa voie, celui dont les mains sont pures se fortifie de plus en plus.
\VS{10}Mais vous tous, revenez à vos mêmes discours, et je ne trouverai pas un sage parmi vous.
\VS{11}Quoi ! Mes jours sont passés, mes projets sont anéantis, les projets qui remplissaient mon cœur...
\VS{12}Et ils prétendent que la nuit c'est le jour, que la lumière est proche quand les ténèbres sont là !
\VS{13}C'est le scheol que j'attends pour demeure, c'est dans les ténèbres que je dresserai ma couche ;
\VS{14}je crie à la fosse : Tu es mon père ! Et aux vers : Vous êtes ma mère et ma sœur !
\VS{15}Où est donc mon espérance ? Mon espérance, qui pourrait la voir ?
\VS{16}Elle descendra vers les portes du scheol, quand nous irons ensemble reposer dans la poussière.
\Chap{18}
\TextTitle{Deuxième discours de Bildad}
\VerseOne{}Bildad de Schuach prit la parole et dit :
\VS{2}Quand mettrez-vous un terme à ces discours ? Ayez de l'intelligence, puis nous parlerons.
\VS{3}Pourquoi sommes-nous regardés comme des bêtes ? Pourquoi ne sommes-nous à vos yeux que des brutes ?
\VS{4}Ô toi qui te déchires toi-même dans ta fureur, faut-il, à cause de toi, que la terre devienne abandonnée ? Faut-il que les rochers disparaissent de leur place ?
\VS{5}La lumière du méchant s'éteindra, et la flamme de son feu cessera de briller.
\VS{6}La lumière s'obscurcira sous sa tente, et sa lampe s'éteindra au-dessus de lui.
\VS{7}Ses pas assurés seront à l'étroit ; malgré son conseil, il tombera.
\VS{8}Car il met les pieds sur un filet, il marche dans les mailles.
\VS{9}Il est saisi au piège par le talon, et le filet le presse ;
\VS{10}la corde est cachée dans la terre, et la trappe est sur son sentier.
\VS{11}Des terreurs l'assiègent, l'entourent, le poursuivent par derrière.
\VS{12}Sa vigueur sera affamée, la détresse est à ses côtés.
\VS{13}Les parties de sa peau sont l'une après l'autre dévorées, ses membres sont dévorés par le premier-né de la mort !
\VS{14}Il est arraché de sa tente, objet de sa confiance, il se traîne vers le roi des épouvantements.
\VS{15}Nul des siens n'habite sa tente, le soufre est répandu sur sa demeure.
\VS{16}En bas, ses racines se dessèchent ; en haut, ses branches sont coupées.
\VS{17}Sa mémoire disparaît de la terre, son nom n'est plus sur la face des champs.
\VS{18}Il est poussé de la lumière dans les ténèbres, il est chassé du monde.
\VS{19}Il ne laisse ni descendants ni postérité parmi son peuple, ni survivants dans les lieux qu'il habitait.
\VS{20}Les générations à venir seront étonnées de sa ruine, et la génération présente sera saisie d'horreur.
\VS{21}Pas d'autre demeure pour l'injuste, pas d'autre demeure pour celui qui ne connaît pas Dieu.
\Chap{19}
\TextTitle{Réponse de Job}
\VerseOne{}Job prit la parole et dit :
\VS{2}Jusqu’à quand affligerez-vous mon âme, et m'écraserez-vous de vos paroles ?
\VS{3}Voilà déjà dix fois que vous m'outragez ; n'avez-vous pas honte de m'étourdir ainsi ?
\VS{4}Si vraiment j'ai péché, seul j'en suis responsable.
\VS{5}Si vraiment vous vous élevez contre moi, si vous me reprochez l'opprobre où je me trouve,
\VS{6}sachez donc que c'est Dieu qui me poursuit, et qui m'enveloppe de son filet.
\VS{7}Voici, je crie à la violence, et on ne me répond pas ; j'implore justice, et il n'y a pas de justice !
\VS{8}Il m'a fermé tout chemin, et je ne puis passer ; il a mis des ténèbres sur mes sentiers.
\VS{9}Il m'a dépouillé de ma gloire, il a ôté la couronne de ma tête.
\VS{10}Il m'a détruit de tous côtés, et je m'en vais ; il a arraché mon espérance comme un arbre.
\VS{11}Il s'est enflammé de colère contre moi, il m'a considéré comme l'un de ses ennemis.
\VS{12}Ses troupes sont venues ensemble, elles se sont frayé leur chemin jusqu'à moi, elles ont campé autour de ma tente.
\VS{13}Il a éloigné de moi mes frères, et ceux qui me connaissaient se sont écartés comme des étrangers ;
\VS{14}mes proches m'ont abandonné, et mes connaissances m'ont oublié.
\VS{15}Je suis un étranger pour les hôtes de ma maison et mes servantes, je ne suis plus à leurs yeux qu'un inconnu.
\VS{16}J'appelle mon serviteur, et il ne répond pas ; je le supplie de ma bouche, et c'est en vain.
\VS{17}Mon humeur est étrange à ma femme, et ma plainte aux fils de mes entrailles.
\VS{18}Je suis méprisé même par des enfants ; si je me lève, ils parlent contre moi.
\VS{19}Ceux que j'avais pour confidents m'ont en horreur, ceux que j'aimais se sont tournés contre moi.
\VS{20}Mes os sont attachés à ma peau et à ma chair ; et je me suis échappé avec la peau de mes dents.
\VS{21}Ayez pitié, ayez pitié de moi, vous, mes amis ! Car la main de Dieu m'a frappé.
\VS{22}Pourquoi me poursuivre comme Dieu me poursuit ? Pourquoi vous montrer insatiables de ma chair ?
\VS{23}Oh ! Je voudrais que mes paroles fussent écrites maintenant, qu'elles fussent inscrites dans un livre ;
\VS{24}je voudrais qu'avec un burin de fer et avec du plomb elles fussent pour toujours gravées dans le roc...
\VS{25}Mais je sais que mon rédempteur est vivant, et qu'il se lèvera le dernier sur la terre.
\VS{26}Quand ma peau sera détruite, il se lèvera ; quand je n'aurai plus de chair, je verrai Dieu.
\VS{27}Je le verrai, et il me sera favorable ; mes yeux le verront, et non ceux d'un autre ; mon âme languit d'attente au dedans de moi.
\VS{28}Vous direz alors : Pourquoi le poursuivions-nous ? Car la justice de mes paroles sera reconnue.
\VS{29}Craignez l'épée pour vous-mêmes, les châtiments par le glaive sont terribles ! Et sachez qu'il y a un jugement.
\Chap{20}
\TextTitle{Dernier discours de Tsophar}
\VerseOne{}Tsophar de Naama prit la parole et dit :
\VS{2}Mes pensées me forcent à répondre, et mon agitation ne peut se contenir.
\VS{3}J'ai entendu des reproches qui m'outragent ; le souffle de mon intelligence donnera la réponse.
\VS{4}Ne sais-tu pas que, de tout temps, depuis que l'homme a été placé sur la terre,
\VS{5}le triomphe des méchants a été court, et la joie de l'impie momentanée ?
\VS{6}Quand son élévation monterait jusqu'aux cieux, et que sa tête atteindrait les nues,
\VS{7}il périra pour toujours comme son ordure, et ceux qui le voyaient diront : Où est-il ?
\VS{8}Il s'envolera comme un songe, et on ne le trouvera plus ; il disparaîtra comme une vision nocturne ;
\VS{9}l'œil qui le regardait ne le regardera plus, le lieu qu'il habitait ne l'apercevra plus.
\VS{10}Ses enfants seront assaillis par les pauvres, et ses mains restitueront ce qu'il a pris par violence.
\VS{11}La vigueur de la jeunesse, qui remplissait ses membres, se reposera avec lui dans la poussière.
\VS{12}Le mal était doux à sa bouche, il le cachait sous sa langue,
\VS{13}il le savourait sans l'abandonner, il le retenait dans son palais,
\VS{14}mais sa nourriture se changera dans ses entrailles, elle deviendra dans son corps un venin d'aspic.
\VS{15}Il a englouti des richesses, il les vomira ; Dieu les chassera de son ventre.
\VS{16}Il a sucé du venin d'aspic, la langue de la vipère le tuera.
\VS{17}Il ne reposera plus ses regards sur les ruisseaux, sur les fleuves, sur les torrents de miel et de lait.
\VS{18}Il rendra ce qu'il a gagné, et ne l'avalera plus ; il restituera tout ce qu'il a pris, et ne s'en réjouira plus.
\VS{19}Car il a opprimé, délaissé les pauvres, il a ruiné des maisons et ne les a pas rétablies.
\VS{20}Son ventre n'a pas connu de bornes ; mais il ne sauvera rien de ce qu'il a tant désiré.
\VS{21}Rien n'échappait à sa voracité, mais son bonheur ne durera pas.
\VS{22}Rempli d'abondance, il sera dans la détresse ; la main de tous les misérables viendra sur lui.
\VS{23}Et voici, pour lui remplir le ventre, Dieu enverra sur lui l'ardeur de sa colère, et le rassasiera par une pluie de traits.
\VS{24}S'il échappe aux armes de fer, l'arc d'airain le transpercera.
\VS{25}Il arrache de son corps la flèche, qui étincelle au sortir de ses entrailles, et il est en proie aux terreurs de la mort.
\VS{26}Toutes les calamités sont réservées à ses trésors ; il sera consumé par un feu que n'allumera point l'homme, et ce qui restera dans sa tente en deviendra la pâture.
\VS{27}Les cieux découvriront son iniquité, et la terre s'élèvera contre lui.
\VS{28}Le revenu de sa maison sera emporté. Tout s'écoulera au jour de la colère de Dieu.
\VS{29}Telle est la part que Dieu réserve à l'homme méchant, tel est l'héritage que Dieu lui promet.
\Chap{21}
\TextTitle{Réponse de Job}
\VerseOne{}Job prit la parole et dit :
\VS{2}Ecoutez, écoutez mes discours, donnez-moi seulement cette consolation.
\VS{3}Supportez-moi, et je parlerai ; et quand j'aurai parlé, tu pourras te moquer.
\VS{4}Mais est-ce contre un homme que s'adresse ma plainte ? Et pourquoi mon âme ne serait-elle pas impatiente ?
\VS{5}Regardez-moi, soyez étonnés, et mettez la main sur la bouche.
\VS{6}Quand j'y pense, cela m'épouvante, et un frisson saisit mon corps.
\VS{7}Pourquoi les méchants vivent-ils, vieillissent-ils, et croissent-ils en force?
\VS{8}Leur postérité s'établit avec eux et en leur présence, leurs rejetons prospèrent sous leurs yeux.
\VS{9}Dans leurs maisons règne la paix, sans mélange de crainte ; la verge de Dieu ne vient pas les frapper.
\VS{10}Leurs taureaux sont féconds, leurs génisses conçoivent et n'avortent pas.
\VS{11}Ils laissent courir leurs enfants comme un troupeau, et les enfants prennent leurs ébats.
\VS{12}Ils chantent au son du tambourin et de la harpe, ils se réjouissent au son du chalumeau.
\VS{13}Ils passent leurs jours dans le bonheur, et ils descendent en un instant au scheol.
\VS{14}Ils disaient pourtant à Dieu : Éloigne-toi de nous, nous ne voulons pas connaître tes voies.
\VS{15}Qu'est-ce que le Tout-Puissant pour que nous le servions ? Que gagnerions-nous à lui adresser nos prières ?
\VS{16}Quoi donc ! Ne sont-ils pas en possession du bonheur entre leurs mains ? Loin de moi le conseil des méchants !
\VS{17}Mais arrive-t-il que la lampe des méchants s'éteigne, que la ruine vienne sur eux, que Dieu leur distribue leur part dans sa colère,
\VS{18}qu'ils soient comme la paille face au vent, comme la balle enlevée par le tourbillon?
\VS{19}Est-ce pour les fils que Dieu réserve le châtiment du père ? Mais c'est lui que Dieu devrait punir, pour qu'il le connaisse ;
\VS{20}c'est lui qui devrait voir de ses propres yeux sa ruine, c'est lui qui devrait boire la colère du Tout-Puissant.
\VS{21}Car que lui importe sa maison après lui, quand le nombre de ses mois est achevé ?
\VS{22}Enseignerait-on la science à Dieu, lui qui juge les esprits élevés ?
\VS{23}L'un meurt au sein du bien-être, tout à son aise et en joie,
\VS{24}les flancs chargés de graisse, et ses os comme abreuvés de mœlle ;
\VS{25}l'autre meurt l'amertume dans l'âme, n'ayant jamais mangé ce qui est bon.
\VS{26}Et tous deux se couchent dans la poussière, tous deux deviennent couverts de vers.
\VS{27}Je sais bien quelles sont vos pensées, quels jugements iniques vous portez sur moi.
\VS{28}Vous dites : Où est la maison de l'homme puissant ? Où est la tente, demeure des méchants ?
\VS{29}Mais quoi ! N'avez-vous pas interrogé les voyageurs, et n'avez-vous pas reconnu ce qu'ils prouvent ?
\VS{30}Au jour du malheur, le méchant est épargné ; au jour de la  colère, il échappe.
\VS{31}Qui lui dit en face sa conduite ? Qui lui rend ce qu'il a fait ?
\VS{32}Il est porté au tombeau, et il veille encore sur sa tombe.
\VS{33}Les mottes de la vallée lui sont légères ; et tous après lui suivront la même voie, comme une multitude l'a déjà suivie.
\VS{34}Pourquoi donc m'offrir de vaines consolations ? Ce qui reste de vos réponses n'est que transgression.
\Chap{22}
\TextTitle{Dernier discours d'Eliphaz}
\VerseOne{}Eliphaz de Théman prit la parole et dit :
\VS{2}Un homme peut-il être utile à Dieu ? Mais le sage n'est utile qu'à lui-même.
\VS{3}Si tu es juste, est-ce à l'avantage du Tout-Puissant ? Si tu es intègre dans tes voies, qu'y gagne-t-il ?
\VS{4}Est-ce par crainte de toi qu'il te châtie, qu'il entre en jugement avec toi ?
\VS{5}Ta méchanceté n'est-elle pas grande ? Tes iniquités ne sont-elles pas sans fin ?
\VS{6}Tu enlevais sans motif des gages à tes frères, tu privais de leurs vêtements ceux qui étaient nus ;
\VS{7}tu ne donnais pas d'eau à boire à l'homme altéré, tu refusais du pain à l'homme affamé.
\VS{8}Le pays était à l'homme le plus fort, et le puissant s'y établissait.
\VS{9}Tu renvoyais les veuves à vide, les bras des orphelins étaient brisés.
\VS{10}C'est pour cela que tu es entouré de pièges, et que la terreur t'a saisi tout à coup.
\VS{11}Ne vois-tu donc pas ces ténèbres, ces eaux débordées qui te couvrent ?
\VS{12}Dieu n'est-il pas là-haut dans les cieux ? Regarde le sommet des étoiles, comme il est élevé !
\VS{13}Et tu dis : Qu'est-ce que Dieu connaît ? Peut-il juger à travers l'obscurité ?
\VS{14}Les nuées l'enveloppent, et il ne voit rien ; il ne parcourt que la voûte des cieux.
\VS{15}Eh quoi ! N'as-tu pas pris garde à l'ancienne route qu'ont suivie les hommes d'iniquité ?
\VS{16}Ils ont été emportés avant le temps, ils ont eu la durée d'un torrent qui s'écoule.
\VS{17}Ils disaient à Dieu : Éloigne-toi de nous ; que peut faire pour nous le Tout-Puissant ?
\VS{18}Dieu cependant avait rempli leurs maisons de biens ! Loin de moi le conseil des méchants !
\VS{19}Les justes le verront, se réjouiront, et l'innocent se moquera d'eux :
\VS{20}Voilà nos adversaires anéantis ! Voilà leurs richesses dévorées par le feu !
\VS{21}Attache-toi donc à Dieu, et tu auras la paix, tu atteindras ainsi le bonheur.
\VS{22}Reçois de sa bouche l'instruction, et mets ses paroles dans ton cœur.
\VS{23}Si tu reviens au Tout-Puissant, tu seras rétabli ; si tu éloignes l'iniquité de ta tente.
\VS{24}Jette l'or dans la poussière, l'or d'Ophir parmi les rochers des torrents ;
\VS{25}et le Tout-Puissant sera ton or, ton argent, ta richesse.
\VS{26}Alors tu feras du Tout-Puissant tes délices, tu élèveras vers Dieu ta face ;
\VS{27}tu le prieras, et il t'exaucera, et tu lui rendras tes vœux.
\VS{28}Quand tu prendras des résolutions elles s'accompliront, sur tes sentiers brillera la lumière.
\VS{29}Vienne l'humiliation, tu parleras pour ton relèvement : Dieu secourt celui dont le regard est abattu.
\VS{30}Il délivrera le coupable ; il sera délivré par la pureté de tes mains.
\Chap{23}
\TextTitle{Réponse de Job}
\VerseOne{}Job prit la parole et dit :
\VS{2}Maintenant encore ma plainte est une révolte, et pourtant ma main appesantit mes soupirs.
\VS{3}Oh ! Si je savais où le trouver, j'irais jusqu'à son trône,
\VS{4}je disposerais en ordre ma cause devant lui, je remplirais ma bouche d'arguments,
\VS{5}je saurais ce qu'il peut avoir à répondre, je comprendrais ce qu'il peut avoir à me dire.
\VS{6}Contesterait-il avec moi dans la grandeur de sa force? Ne prendrait-il pas le temps de m'écouter ?
\VS{7}Ce serait un homme juste qui argumenterait avec lui, et je serais pour toujours absous par mon juge.
\VS{8}Mais, si je vais à l'orient, il n'y est pas ; si je vais à l'occident, je ne l'aperçois pas ;
\VS{9}est-il occupé au nord, je ne le vois pas ; se cache-t-il au midi, je ne l'aperçois pas.
\VS{10}Il sait néanmoins quelle voie j'ai suivie ; et s'il m'éprouvait, j'en sortirai pur comme l'or.
\VS{11}Mon pied s'est attaché à ses pas ; j'ai gardé sa voie, et je ne m'en suis pas détourné.
\VS{12}Je n'ai pas abandonné les commandements de ses lèvres ; j'ai fait plier ma volonté aux paroles de sa bouche.
\VS{13}Mais il n'a qu'une pensée ; qui l'en fera revenir ? Ce que son âme désire, il le fait.
\VS{14}Il achèvera donc ses desseins à mon égard, et il en concevra beaucoup d'autres encore.
\VS{15}Voilà pourquoi sa présence m'épouvante ; quand j'y pense, j'ai peur de lui.
\VS{16}Dieu a brisé mon cœur, le Tout-Puissant m'a épouvanté.
\VS{17}Car ce n'est pas la présence des ténèbres qui m'anéantit, ce n'est pas l'obscurité dont ma face est couverte.
\Chap{24}
\VerseOne{}Pourquoi le Tout-Puissant ne met-il pas des temps en réserve, et pourquoi ceux qui le connaissent ne voient-ils pas ses jours ?
\VS{2}On déplace les bornes, on ravit des troupeaux, et on les fait paître ;
\VS{3}on emmène l'âne de l'orphelin, on prend pour gage le bœuf de la veuve ;
\VS{4}on fait écarter les pauvres du chemin, on force tous les affligés du pays à se cacher.
\VS{5}Et voici, comme les ânes sauvages du désert, ils sortent le matin pour chercher de la nourriture, ils n'ont que le désert pour trouver le pain de leurs enfants ;
\VS{6}ils moissonnent le fourrage qui reste dans les champs, ils grappillent dans la vigne de l'impie ;
\VS{7}ils passent la nuit nus, sans vêtements, sans couverture contre le froid ;
\VS{8}ils sont percés par la pluie des montagnes, et ils embrassent les rochers comme unique refuge.
\VS{9}On arrache l'orphelin à la mamelle, on prend des gages sur le pauvre.
\VS{10}Ils vont tout nus, sans vêtements, ils sont affamés, et ils portent les gerbes.
\VS{11}Dans les enclos de l'impie, ils font de l'huile, ils foulent le pressoir à raisin et ils ont soif.
\VS{12}Dans les villes s'exhalent les soupirs des mourants, l'âme des blessés jette des cris... Et Dieu ne prend pas garde à ces infamies !
\VS{13}En voici d'autres qui se révoltent contre la lumière, ils n'en connaissent pas les voies, ils n'en restent pas sur les sentiers.
\VS{14}Le meurtrier se lève au point du jour ; il tue le pauvre et l'indigent, et il dérobe pendant la nuit.
\VS{15}L'œil de l'adultère épie le crépuscule ; aucun œil ne me verra, dit-il, et il met un voile sur le visage.
\VS{16}Ils forcent les maisons dans les ténèbres, le jour ils se tiennent enfermés ; ils ne connaissent pas la lumière.
\VS{17}Pour eux, le matin c'est l'ombre de la mort ; si quelqu'un les reconnaît, ils ont des terreurs.
\VS{18}Eh quoi ! L'impie est d'un poids léger sur la surface de l'eau, il n'a sur la terre qu'un héritage maudit, il ne prend jamais le chemin des vignes !
\VS{19}Comme la sécheresse et la chaleur absorbent les eaux de la neige, ainsi le scheol engloutit ceux qui pèchent !
\VS{20}Quoi ! Le sein maternel l'oublie, les vers en font leurs délices, on ne se souvient plus de lui ! L'injuste est brisé comme du bois,
\VS{21}lui qui dépouille la femme stérile et sans enfants, lui qui ne répand aucun bien sur la veuve !...
\VS{22}Non ! Dieu par sa force prolonge les jours des violents, et les voilà s’élever quand ils ne croyaient plus en la vie.
\VS{23}Il leur donne de la sécurité et de la confiance, ses yeux sont sur leurs voies.
\VS{24}Ils se sont élevés ; et en un peu de temps ils ne sont plus, ils s'affaissent, ils meurent en chemin comme tous les hommes, ils sont coupés comme une tête d'épi.
\VS{25}S'il n'en est pas ainsi, qui me fera mentir, qui fera de mes paroles un rien ?
\Chap{25}
\TextTitle{Dernier discours de Bildad}
\VerseOne{}Bildad de Schuach prit la parole et dit :
\VS{2}La puissance et la terreur appartiennent à Dieu ; il fait régner la paix dans ses hautes régions.
\VS{3}Ses armées peuvent-elles se compter ? Sur qui sa lumière ne se lève-t-elle pas ?
\VS{4}Comment l'homme serait-il juste devant Dieu ?  Comment celui qui est né de la femme serait-il pur ?
\VS{5}Voici, la lune même n'est pas brillante, et les étoiles ne sont pas pures à ses yeux ;
\VS{6}combien moins l'homme qui n'est qu'un ver, le fils de l'homme qui n'est qu'un vermisseau !
\Chap{26}
\TextTitle{Réponse de Job}
\VerseOne{}Job prit la parole et dit :
\VS{2}Comme tu as aidé celui qui était sans force ! Comme tu as secouru le bras sans force !
\VS{3}Quels bons conseils tu donnes à celui qui manque d'intelligence ! Tu fais connaître l'abondance de ta sagesse !
\VS{4}A qui s'adressent tes paroles ? Et de qui est l'esprit qui est sorti de toi ?
\VS{5}Devant Dieu les ombres tremblent au-dessous des eaux et de leurs habitants ;
\VS{6}devant lui le scheol est nu, l'abîme est sans voile.
\VS{7}Il étend la direction nord sur le vide, il suspend la terre sur le néant.
\VS{8}Il renferme les eaux dans ses nuages, et la nuée n'éclate pas sous leur poids.
\VS{9}Il couvre la face de son trône, il répand sur lui sa nuée.
\VS{10}Il a tracé un cercle à la surface des eaux, comme limite entre la lumière et les ténèbres.
\VS{11}Les colonnes du ciel s'ébranlent et s'étonnent à sa menace.
\VS{12}Par sa force il soulève la mer, par son intelligence il en brise l'orgueil.
\VS{13}Son souffle donne au ciel la sérénité, sa main transperce le serpent fuyard.
\VS{14}Ce sont là les bords de ses voies, c'est le discours fait en chuchotant que nous entendons ; mais qui comprendra le tonnerre de sa puissance ?
\Chap{27}
\VerseOne{}Job prit de nouveau la parole sous forme sentencieuse et dit :
\VS{2}Dieu qui a mis de côté mon jugement est vivant ! Le Tout-Puissant qui remplit mon âme d'amertume est vivant !
\VS{3}Aussi longtemps que j'aurai mon souffle et que l'esprit de Dieu sera dans mes narines,
\VS{4}mes lèvres ne prononceront rien d'injuste, ma langue ne dira rien de faux.
\VS{5}Loin de moi la pensée de vous donner raison ! Jusqu'à mon dernier soupir, je ne me détournerai pas de mon intégrité.
\VS{6}Je tiens à ma justice, et je ne faiblirai pas ; mon cœur ne me reproche aucun de mes jours.
\VS{7}Que mon ennemi soit comme le méchant, et mon adversaire comme l'injuste !
\VS{8}Quelle espérance reste-t-il à l'impie quand Dieu coupe le fil de sa vie, quand il lui retire son âme ?
\VS{9}Est-ce que Dieu entend ses cris, quand l'angoisse vient sur lui ?
\VS{10}Trouvera-t-il son plaisir dans le Tout-Puissant ? Invoque-t-il Dieu en tout temps ?
\VS{11}Je vous enseignerai comment la main de Dieu agit, je ne vous cacherai pas les desseins du Tout-Puissant.
\VS{12}Mais vous les connaissez, et vous êtes d'accord ; pourquoi donc vous laissez-vous aller à ces vaines pensées ?
\VS{13}Voici la part que Dieu réserve à l'homme méchant, l'héritage que les violents reçoivent du Tout-Puissant.
\VS{14}S'il a des fils en grand nombre, c'est pour l'épée, et ses rejetons ne seront pas rassasiés de pain ;
\VS{15}ses survivants sont ensevelis par la peste, et leurs veuves ne les pleurent pas.
\VS{16}S'il amasse l'argent comme la poussière, s'il entasse les vêtements comme de la boue,
\VS{17}c'est lui qui entasse, mais c'est le juste qui se revêt, c'est l'innocent qui a l'argent en partage.
\VS{18}Il se bâtit une maison comme celle de la teigne, comme la cabane que fait un gardien.
\VS{19}Il se couche riche, et il périt dépouillé ; il ouvre les yeux, et tout a disparu.
\VS{20}Les terreurs l'atteignent comme des eaux ; un tourbillon le prend au milieu de la nuit.
\VS{21}Le vent d'orient l'emporte, et il s'en va ; il l'arrache de sa demeure comme un tourbillon.
\VS{22}Dieu le précipite à terre et ne l'épargne pas, et le méchant voudrait fuir devant sa main.
\VS{23}On applaudit à sa chute, et on le siffle au lieu où il se tient.
\Chap{28}
\VerseOne{}Il y a pour l'argent une mine d'où on le fait sortir, et pour l'or un lieu d'où on le purifie pour l'affiner ;
\VS{2}le fer se tire de la poussière, et la pierre se fond pour produire l'airain.
\VS{3}L'homme fait cesser les ténèbres, il explore jusqu'aux extrêmes limites les pierres cachées dans l'obscurité et dans l'ombre de la mort.
\VS{4}Il creuse un puits, loin des lieux habités ; ne se souvenant plus de ses pieds, il est suspendu, balancé, loin des humains.
\VS{5}La terre, d'où sort le pain, est bouleversée dans ses entrailles comme par le feu.
\VS{6}Ses pierres sont la demeure du saphir, et l'on y trouve de la poudre d'or.
\VS{7}L'oiseau de proie n'en connaît pas le chemin, l'œil du vautour ne l'aperçoit pas ;
\VS{8}les plus jeunes et fiers animaux n'y ont pas marché, le lion n'y a jamais passé.
\VS{9}L'homme avance sa main sur le roc, il renverse les montagnes depuis la racine ;
\VS{10}il fend des tranchées dans les rochers, et son œil voit tout ce qu'il y a de précieux ;
\VS{11}il arrête l'écoulement des eaux, et il fait sortir ce qui est caché.
\VS{12}Mais la sagesse, où se trouve-t-elle ? Où est le lieu où se tient l'intelligence ?
\VS{13}L'homme n'en connaît pas le prix, elle ne se trouve pas dans la terre des vivants.
\VS{14}L'abîme dit : Elle n'est pas en moi ; et la mer dit : Elle n'est pas avec moi.
\VS{15}Elle ne se donne pas contre de l'or pur, elle ne s'achète pas au poids de l'argent ;
\VS{16}elle ne se pèse pas contre de l'or d'Ophir, ni contre le précieux onyx, ni contre le saphir.
\VS{17}Elle ne peut se comparer à l'or ni au verre, elle ne peut s'échanger pour un vase d'or fin.
\VS{18}On ne se souvient ni du corail ni du cristal auprès d'elle : La sagesse vaut plus que les perles.
\VS{19}On ne la compare pas avec la topaze d'Ethiopie ; on ne la met pas en balance avec l'or pur.
\VS{20}D'où vient donc la sagesse ? Où est la demeure de l'intelligence ?
\VS{21}Elle est cachée aux yeux de tous les vivants, elle est cachée aux oiseaux des cieux.
\VS{22}L'abîme et la mort disent : Nous en avons entendu parler de nos oreilles.
\VS{23}C'est Dieu qui en sait le chemin, c'est lui qui en connaît la demeure ;
\VS{24}car il regarde jusqu'aux extrémités de la terre, il voit tout sous les cieux.
\VS{25}Quand il façonna le poids du vent, et qu'il estima la mesure des eaux,
\VS{26}quand il ordonna des lois à la pluie, et qu'il fit un chemin à l'éclair et au tonnerre,
\VS{27}alors il vit la sagesse et la manifesta ; il l'établit et la sonda.
\VS{28}Puis il dit à l'homme : Voici, la crainte du Seigneur, c'est la sagesse ; se détourner du mal, c'est l'intelligence.
\Chap{29}
\TextTitle{La postérité passée de Job}
\VerseOne{}Job prit de nouveau la parole sous forme sentencieuse et dit :
\VS{2}Oh ! Que ne puis-je être comme aux mois du passé, comme aux jours où Dieu me gardait,
\VS{3}quand sa lampe brillait sur ma tête, quand je marchais à sa lumière dans les ténèbres !
\VS{4}Que ne suis-je comme aux jours de mon automne, où Dieu veillait en ami sur ma tente,
\VS{5}quand le Tout-Puissant était encore avec moi, et que mes serviteurs m'entouraient ;
\VS{6}quand je lavais mes pieds dans le lait, et que le rocher répandait près de moi des torrents d'huile !
\VS{7}Si je sortais pour aller à la porte de la ville, et si je me faisais préparer un siège dans la place,
\VS{8}les jeunes gens se retiraient en me voyant, les vieillards se levaient et se tenaient debout.
\VS{9}Les princes arrêtaient leurs discours, et mettaient la main sur leur bouche ;
\VS{10}la voix des chefs se taisait, et leur langue s'attachait à leur palais.
\VS{11}L'oreille qui m'entendait me disait heureux, l'œil qui me voyait me rendait témoignage ;
\VS{12}car je délivrais l'affligé qui criait au secours, et l'orphelin qui n'avait personne pour le secourir.
\VS{13}La bénédiction de celui qui allait périr venait sur moi ; je remplissais de joie le cœur de la veuve.
\VS{14}Je me revêtais de la justice et elle se revêtait de moi, j'avais ma droiture pour manteau et pour turban.
\VS{15}J'étais les yeux de l'aveugle et les pieds du boiteux.
\VS{16}J'étais le père des pauvres, j'examinais la cause de l'inconnu ;
\VS{17}je brisais les mâchoires de l'injuste, et j'arrachais la proie d'entre ses dents.
\VS{18}Alors je disais : Je mourrai dans mon nid, mes jours seront aussi nombreux que le sable ;
\VS{19}l'eau pénétrera dans mes racines, la rosée passera la nuit sur mes branches ;
\VS{20}ma gloire se renouvellera sans cesse en moi, et mon arc se renouvellera dans ma main.
\VS{21}On m'écoutait et l'on restait dans l'attente, on gardait le silence devant mes conseils.
\VS{22}Après mes discours, nul ne répliquait, et ma parole était pour tous une bienfaisante rosée ;
\VS{23}ils s'attendaient à moi comme à la pluie, ils ouvraient la bouche comme pour une pluie de printemps.
\VS{24}Je souriais quand ils perdaient confiance, et l'on ne pouvait faire tomber la sérénité de mon visage.
\VS{25}J'aimais à aller avec eux, et je m'asseyais à leur tête ; j'étais comme un roi au milieu de ses gardes, comme un consolateur auprès des affligés.
\Chap{30}
\TextTitle{Son humiliation}
\VerseOne{}Et maintenant !... Chaque jour je suis la risée de plus jeunes que moi, de ceux dont je dédaignais de  mettre les chefs parmi les chiens de mon troupeau.
\VS{2}Mais à quoi me servirait la force de leurs mains ? En eux avait péri toute vigueur.
\VS{3}Desséchés par la disette et la faim, ils fuient dans les lieux arides, depuis longtemps abandonnés et déserts ;
\VS{4}ils arrachent près des buissons l'herbe sauvage, et la racine des genêts est leur nourriture.
\VS{5}On les chasse du milieu des hommes, on crie après eux comme après un voleur.
\VS{6}Ils habitent dans des torrents affreux, dans les cavernes de la terre et dans les rochers ;
\VS{7}ils hurlent parmi les buissons, ils se rassemblent sous les ronces.
\VS{8}Peuple insensé et sans nom, on les repousse du pays !
\VS{9}Et maintenant, je suis le sujet de leurs chansons, je suis en butte à leurs propos.
\VS{10}Ils m'ont en horreur, ils s'éloignent de moi, ils ne se retiennent pas de me cracher leur salive au visage.
\VS{11}Ils n'ont aucune retenue et ils m'humilient, ils rejettent tout frein devant moi.
\VS{12}Ces misérables se lèvent à ma droite et me poussent les pieds, ils se fraient contre moi des routes pour ma ruine ;
\VS{13}ils détruisent mon propre sentier et travaillent à ma perte, eux à qui personne ne viendrait en aide ;
\VS{14}ils arrivent comme par une large brèche, ils se précipitent sous les craquements.
\VS{15}Toutes les terreurs se tournent contre moi ; ma gloire est emportée comme par le vent, mon bonheur a passé comme un nuage.
\VS{16}Et maintenant, mon âme se répand en mon sein, les jours d'affliction m'ont saisi.
\VS{17}La nuit me perce et m'arrache les os, la douleur qui me ronge ne se donne aucun repos.
\VS{18}Par la violence du mal, mon vêtement se déforme, il se colle à mon corps comme ma tunique.
\VS{19}Dieu m'a jeté dans la boue, et je ressemble à la poussière et à la cendre.
\VS{20}Je crie vers toi, et tu ne me réponds pas ; je me tiens debout, et tu m'aperçois.
\VS{21}Tu deviens cruel contre moi, tu t'opposes à moi avec la force de ta main.
\VS{22}Tu me soulèves, tu me fais chevaucher sur le vent, et tu me fais fondre au bruit de la tempête.
\VS{23}Car, je le sais, tu me mènes à la mort, à la demeure fixée pour tous les vivants.
\VS{24}Mais celui qui va périr n'étend-il pas les mains ? Celui qui est dans le malheur n'implore-t-il pas du secours ?
\VS{25}Ne pleurais-je pas sur l'homme qui passait des jours difficiles ? Mon âme n'avait-elle pas pitié du pauvre ?
\VS{26}J'attendais le bonheur, et le malheur est arrivé ; j'espérais la lumière, et les ténèbres sont venues.
\VS{27}Mes entrailles bouillonnent sans repos, les jours d'affliction m'ont confronté.
\VS{28}Je marche noirci, mais non par le soleil ; je me lève en pleine assemblée, et je crie.
\VS{29}Je suis devenu le frère des serpents, le compagnon des autruches.
\VS{30}Ma peau noircit et tombe, mes os brûlent et se dessèchent.
\VS{31}Ma harpe n'est plus qu'un instrument de deuil, et mon chalumeau ne peut rendre que des voix en pleurs.
\Chap{31}
\TextTitle{Job se justifie}
\VerseOne{}J'avais fait une alliance avec mes yeux, et je n'aurais pas regardé une vierge.
\VS{2}Quelle part Dieu m'eût-il réservée d'en haut ? Quel héritage le Tout-Puissant m’aurait-il envoyé des cieux ?
\VS{3}La ruine n'est-elle pas pour l'injuste, et le malheur pour ceux qui commettent l'iniquité ?
\VS{4}Dieu ne voit-il pas mes voies ? Ne compte-t-il pas tous mes pas ?
\VS{5}Si j'ai marché dans le mensonge, si mon pied s'est hâté pour tromper,
\VS{6}que Dieu me pèse dans des balances justes, et il reconnaîtra mon intégrité !
\VS{7}Si mes pas se sont détournés du droit chemin, si mon cœur a suivi mes yeux, si quelque souillure s'est attachée à mes mains,
\VS{8}que je sème et qu'un autre mange, et que mes rejetons soient déracinés !
\VS{9}Si mon cœur a été séduit par une femme, si j'ai fait le guet à la porte de mon prochain,
\VS{10}que ma femme broie le grain pour un autre, et que d'autres se penchent sur elle !
\VS{11}Car c'est un crime, une iniquité punie par les juges ;
\VS{12}c'est un feu qui dévore jusqu'à la destruction, et qui aurait détruit toutes mes récoltes dans leur racine.
\VS{13}Si j'ai méprisé le droit de mon serviteur ou de ma servante, lorsqu'ils étaient en contestation avec moi,
\VS{14}qu'ai-je à faire, quand Dieu se lève ? Qu'ai-je à répondre, quand il châtie ?
\VS{15}Celui qui m'a fait dans le ventre de ma mère ne l'a-t-il pas fait aussi ? Un même Dieu ne nous a-t-il pas formés dans le sein maternel ?
\VS{16}Si j'ai refusé aux pauvres leur désir, si j'ai laissé se consumer les yeux de la veuve,
\VS{17}si j'ai mangé seul mon morceau de pain, sans que l'orphelin en ait sa part,
\VS{18}moi qui l'ai dès ma jeunesse fait grandir près de moi comme un père, et qui dès le sein de ma mère, ai été le guide de la veuve ;
\VS{19}si j'ai vu le malheureux périr faute de vêtements, le pauvre manquer de couverture,
\VS{20}sans que ses reins m'aient béni, sans qu'il ait été réchauffé par la toison de mes agneaux ;
\VS{21}si j'ai levé la main contre l'orphelin, parce que je me voyais comme un appui dans les portes ;
\VS{22}que mon épaule tombe de sa jointure, que mon bras tombe et qu'il se brise l'os !
\VS{23}Car les châtiments de Dieu m'épouvantent, et je ne pourrais pas prévaloir devant sa majesté.
\VS{24}Si j'ai mis dans l'or ma confiance, si j'ai dit à l'or fin : Tu es mon espoir ;
\VS{25}si je me suis réjoui de ma grande puissance, de la quantité des richesses que ma main a acquise ;
\VS{26}si j'ai regardé le soleil quand il brillait, la lune quand elle s'avançait de façon majestueuse,
\VS{27}et si mon cœur s'est laissé secrètement séduire, si ma main a envoyé des baisers de ma bouche ;
\VS{28}c'est encore une iniquité que doit punir le juge, et j'aurais renié le Dieu d'en haut !
\VS{29}Si je me suis réjoui du malheur de mon ennemi, si j'ai sauté d'allégresse quand le mal l'a atteint,
\VS{30}moi qui n'ai pas permis à ma langue de pécher en demandant sa mort par des malédictions ;
\VS{31}si les gens de ma tente ne disaient pas : Où est celui qui n'a pas été rassasié de sa viande ?
\VS{32}Si l'étranger passait la nuit dehors, si je n'ouvrais pas ma porte au voyageur ;
\VS{33}si, comme les hommes, j'ai caché mes transgressions et mon crime dans mon sein,
\VS{34}parce que je craignais la multitude, et je craignais le mépris des familles, en sorte que je restais tranquille et n'osais franchir ma porte...
\VS{35}Oh ! Qui me fera trouver quelqu'un qui m'écoute ? Voilà ma défense toute signée : Que le Tout-Puissant me réponde ! Qui me donnera la plainte écrite par mon adversaire ?
\VS{36}Je porterai son écrit sur mon épaule, je l'attacherai sur mon front comme une couronne ;
\VS{37}je lui déclarerai le nombre de mes pas, je m'approcherai de lui comme un prince.
\VS{38}Si ma terre crie contre moi, et que ses sillons pleurent ;
\VS{39}si j'en ai mangé le produit sans l'avoir payée, et que j'aie attristé l'âme de ses anciens maîtres ;
\VS{40}Qu'elle en produise des épines au lieu du froment, et de l'ivraie au lieu de l'orge ! C'est ici la fin des paroles de Job.
\Chap{32}
\TextTitle{Discours d'Elihu : reproches à Job et à ses amis}
\VerseOne{}Ces trois hommes-là cessèrent de répondre à Job, parce qu'il se regardait comme juste.
\VS{2}Élihu, fils de Barakeel de Buz, de la famille de Ram, s'enflamma de colère contre Job, parce qu'il disait son âme juste devant Dieu.
\VS{3}Et sa colère s'enflamma contre ses trois amis, parce qu'ils ne trouvaient rien à répondre et que néanmoins ils condamnaient Job.
\VS{4}Comme ils étaient plus âgés que lui, Elihu avait attendu jusqu'à ce moment pour parler à Job.
\VS{5}Mais, voyant que ces trois hommes n'avaient plus aucune réponse à la bouche, Elihu se mit en colère.
\VS{6}Et Elihu, fils de Barakeel de Buz, prit la parole et dit : Je suis jeune et vous êtes des vieillards ; c'est pourquoi j'ai craint, j'ai eu peur de vous faire connaître mon sentiment.
\VS{7}Je me disais en moi-même : Les jours parleront, le grand nombre des années fera connaître la sagesse.
\VS{8}Mais en réalité, dans l'homme, c'est l'esprit, le souffle du Tout-Puissant qui donne l'intelligence ;
\VS{9}ce ne sont pas les aînés qui sont sages, ce ne sont pas les vieillards qui comprennent ce qui est juste.
\VS{10}Voilà pourquoi je dis : Ecoute ! Moi aussi, je dirai ma pensée.
\VS{11}J'ai attendu la fin de vos discours, j'ai écouté vos raisonnements, jusqu'à ce que vous ayez bien examiné les discours de Job.
\VS{12}J'ai pris le soin de vous écouter ; et voici, aucun de vous n'a convaincu Job, aucun n'a répondu à ses paroles.
\VS{13}Ne dites pas cependant : En lui nous avons trouvé la sagesse ; c'est Dieu qui peut le confondre, ce n'est pas un homme !
\VS{14}Il n'a pas préparé ses discours contre moi : Aussi je ne lui répondrai pas à votre manière.
\VS{15}Ils sont épouvantés ! Ils ne répondent plus ! Ils ont la parole coupée !
\VS{16}J'ai attendu qu'ils eussent fini leurs discours, qu'ils s'arrêtassent et ne sussent que répondre.
\VS{17}A mon tour, je veux répondre aussi, je veux dire aussi ce que je pense.
\VS{18}Car je suis rempli de discours, l'esprit qui est en mon sein me presse.
\VS{19}Mon sein est comme un vin sans issue, comme des outres neuves qui vont éclater\FTNT{Mt. 9:17 ; Mc. 2:22 ; Lu. 5:38.}.
\VS{20}Je parlerai pour respirer à l'aise, j'ouvrirai mes lèvres et je répondrai.
\VS{21}Je n'aurai pas égard à l'apparence de l'homme, et je ne flatterai pas un homme.
\VS{22}Car je ne sais pas flatter : Mon Créateur m'enlèverait bien vite.
\Chap{33}
\TextTitle{Discours d'Elihu : la justice de Dieu}
\VerseOne{}Mais toi, ô Job, écoute mes discours, prête l'oreille à toutes mes paroles !
\VS{2}Voici, j'ouvre la bouche, ma langue parle dans mon palais.
\VS{3}Mes paroles exprimeront la droiture de mon cœur, mes lèvres diront la vérité pure.
\VS{4}L'Esprit de Dieu qui m'a fait, et le souffle du Tout-Puissant me donne la vie.
\VS{5}Si tu le peux, réponds-moi, prépare ta face, et tiens-toi prêt !
\VS{6}Devant Dieu je suis ta bouche, j'ai été comme toi formé de la boue ;
\VS{7}ainsi ma terreur ne te troublera pas, et mon poids ne pèsera pas sur toi.
\VS{8}Mais tu as dit à mes oreilles, et j'ai entendu le son de tes paroles :
\VS{9}Je suis pur, je suis sans péché, je suis net, il n'y a pas d'iniquité en moi.
\VS{10}Et Dieu trouve contre moi des oppositions, il me considère comme son ennemi ;
\VS{11}il met mes pieds dans les ceps, il surveille tous mes mouvements.
\VS{12}Je te répondrai qu'en cela tu n'as pas été juste, car Dieu est plus grand que l'homme.
\VS{13}Pourquoi as-tu donc plaidé contre lui ? Car il ne rend pas compte de toutes ses actions.
\VS{14}Car Dieu parle une première fois, et une seconde fois à celui qui n'aura pas pris garde à la première.
\VS{15}Par des songes, par des visions nocturnes, quand les hommes tombent dans un profond sommeil, quand ils dorment sur leur couche.
\VS{16}Alors il ouvre l'oreille de l'homme, et met le sceau à ses instructions,
\VS{17}afin de détourner l'homme de son œuvre et de le préserver de l'orgueil,
\VS{18}afin de préserver son âme de la fosse, et que sa vie ne soit ôtée par les armes.
\VS{19}L'homme est aussi châtié par des douleurs sur son lit, à cause d'une lutte perpétuelle en ses os.
\VS{20}Alors il prend en dégoût le pain, même les mets les plus désirés ;
\VS{21}sa chair se consume et disparaît, ses os qu'on ne voyait pas sont mis à nu ;
\VS{22}son âme s'approche de la fosse, et sa vie des messagers de la mort.
\VS{23}Mais s'il y a pour cet homme un messager qui interprète, un d'entre les mille, pour lui annoncer la voie de la droiture,
\VS{24}alors Dieu dans sa miséricorde dit au messager : Délivre-le, afin qu'il ne descende pas dans la fosse ; j'ai trouvé une rançon !
\VS{25}Sa chair a plus de fraîcheur qu'au premier âge, il revient aux jours de sa jeunesse.
\VS{26}Il supplie Dieu par ses prières, et Dieu lui est favorable, il lui laisse voir sa face avec joie, et lui rend sa justice.
\VS{27}Il chante devant les hommes et dit : J'ai péché, j'ai violé la justice, et je n'ai pas été puni comme je le méritais ;
\VS{28}Dieu a racheté mon âme afin qu'elle ne passât pas dans la fosse, et ma vie voit encore la lumière !
\VS{29}Voilà tout ce que Dieu fait, deux fois, trois fois, avec l'homme,
\VS{30}pour ramener son âme de la fosse, pour l'éclairer de la lumière des vivants.
\VS{31}Sois attentif, Job, écoute-moi ! Tais-toi, et je parlerai !
\VS{32}Si tu as quelque chose à dire, réponds-moi ! Parle, car je désire te justifier.
\VS{33}Si tu n'as rien à dire, écoute-moi ! Tais-toi, et je t'enseignerai la sagesse.
\Chap{34}
\TextTitle{Discours d'Elihu : il accuse Job de se révolter}
\VerseOne{}Elihu reprit et dit :
\VS{2}Sages, écoutez mes discours ! Vous qui avez la connaissance, prêtez-moi l'oreille !
\VS{3}Car l'oreille discerne les discours, comme le palais savoure ce qu'il mange.
\VS{4}Choisissons ce qui est juste, voyons entre nous ce qui est bon.
\VS{5}Job dit : Je suis juste, et Dieu cesse de me faire justice ;
\VS{6}en dépit de mon droit, je passe pour un menteur ; ma plaie est douloureuse, et je suis sans péché.
\VS{7}Y a-t-il un homme semblable à Job, buvant les moqueries comme l'eau,
\VS{8}qui marche en compagnie de ceux qui font le mal, marchant de pair avec les hommes méchants ?
\VS{9}Car il a dit : Il est utile à l'homme de plaire à Dieu.
\VS{10}Ecoutez-moi donc, hommes de sens ! Loin de Dieu la méchanceté, loin du Tout-Puissant l'injustice !
\VS{11}Il rend à l'homme selon ses œuvres, il fait trouver à chacun selon ses voies.
\VS{12}Certes, Dieu ne commet pas d'iniquité ; le Tout-Puissant ne renverse pas la justice.
\VS{13}Qui lui a donné la charge de la terre ? Qui l'a mis dans le monde ?
\VS{14}S'il ne pensait qu'à lui-même, s'il retirait à lui son esprit et son souffle,
\VS{15}toute chair périrait ensemble, et l'homme retournerait dans la poussière.
\VS{16}Si tu as de l'intelligence, écoute ceci, prête l'oreille au son de mes paroles !
\VS{17}Celui qui haïrait la justice régnerait-il ? Et condamneras-tu le juste, le puissant ?
\VS{18}Dira-t-on à un roi, qu'il est un scélérat ? Et aux princes, qu'ils sont des méchants ?
\VS{19}Combien moins le dira-t-on à celui qui n'a point d'égard à la personne des grands, et qui ne connaît point les riches pour les préférer aux pauvres, parce qu'ils sont tous l'ouvrage de ses mains ?
\VS{20}En un moment, ils mourront ; au milieu de la nuit, un peuple est ébranlé et passe ; le puissant s'en va, sans la main d'aucun homme.
\VS{21}Car les yeux de Dieu sont sur les voies de l'homme, il regarde tous ses pas.
\VS{22}Il n'y a ni ténèbres ni ombre de la mort où puissent se cacher ceux qui commettent l'iniquité.
\VS{23}Dieu n'a pas besoin de regarder longtemps pour qu'un homme aille en jugement avec lui.
\VS{24}Il brise les puissants sans enquête, et il en établit d'autres à leur place ;
\VS{25}car il connaît leurs œuvres. Il les renverse de nuit, et ils sont écrasés ;
\VS{26}il les frappe comme des impies, au lieu où se tiennent tous les regards.
\VS{27}En se détournant de lui, en abandonnant toutes ses voies,
\VS{28}ils ont fait monter à Dieu le cri du pauvre, et il a entendu le cri des affligés.
\VS{29}S'il donne le repos, qui est-ce qui le condamnera ? S'il cache sa face, qui le regardera ? Il traite à l'égal soit une nation, soit un homme,
\VS{30}afin que l'homme impie ne règne pas, et qu'il ne soit plus un piège pour le peuple.
\VS{31}Car a-t-il jamais dit à Dieu : J'ai été pardonné, je ne pécherai plus ;
\VS{32}montre-moi ce que je ne vois pas ; si j'ai fait le mal, je ne le ferai plus ?
\VS{33}Est-ce d'après toi que Dieu rendra la paix ? C'est toi qui rejettes ainsi, qui choisis, mais non pas moi ; ce que tu sais, dis-le donc !
\VS{34}Les gens de bon sens diront avec moi, et tout homme sage en conviendra,
\VS{35}que Job parle sans connaissance, et ses discours manquent d'intelligence.
\VS{36}Que Job soit constamment éprouvé, puisqu'il répond comme les hommes méchants !
\VS{37}Car il ajoute à son péché une transgression nouvelle ; il s'applaudit au milieu de nous ; il multiplie ses paroles contre Dieu.
\Chap{35}
\TextTitle{Discours d'Elihu : il reproche à Job ses propos irréfléchis}
\VerseOne{}Elihu reprit et dit :
\VS{2}Penses-tu avoir raison de dire : Je suis juste devant Dieu ?
\VS{3}Quand tu dis : Que me sert-il, et que gagnerais-je de plus sans pécher ?
\VS{4}C'est à toi que je vais répondre par mes discours, et à tes amis en même temps.
\VS{5}Regarde les cieux, et considère-les ! Vois les nuées, comme elles sont plus hautes que toi !
\VS{6}Si tu pèches, quel tort lui feras-tu ? Et quand tes péchés se multiplient, que lui fais-tu ?
\VS{7}Si tu es juste, que lui donnes-tu ? Que reçoit-il de ta main ?
\VS{8}Ta méchanceté ne peut nuire qu'à l'homme, ta justice n'est utile qu'au fils de l'homme.
\VS{9}On crie contre la multitude des oppressions, on se plaint de la violence d'un grand nombre ;
\VS{10}mais nul ne dit : Où est Dieu, mon Créateur, celui qui donne des chants d'allégresse pendant la nuit,
\VS{11}qui nous instruit plus que les bêtes de la terre, et nous donne d'être plus sages que les oiseaux des cieux ?
\VS{12}On a beau crier alors, Dieu ne répond pas, à cause de l'orgueil des méchants.
\VS{13}C'est en vain que l'on crie, Dieu n'écoute pas, le Tout-Puissant n'y a pas égard.
\VS{14}Quoique tu dises que tu ne le vois pas, ta cause est devant lui : Attends-le !
\VS{15}Mais, parce que sa colère ne punit pas encore, ce n'est pas à dire qu'il ait peu de connaissance du crime.
\VS{16}Ainsi Job ouvre vainement sa bouche, il multiplie les paroles sans intelligence.
\Chap{36}
\TextTitle{Discours d'Elihu : Dieu traite les hommes selon leurs oeuvres}
\VerseOne{}Elihu continua et dit :
\VS{2}Attends un peu, et je vais parler, car j'ai des  paroles encore pour la cause de Dieu.
\VS{3}Je prendrai mes raisons de loin, et je donnerai droit à mon Créateur.
\VS{4}Sûrement, mes discours ne sont pas des mensonges, mes sentiments devant toi sont sincères.
\VS{5}Dieu est puissant, mais il ne rejette personne ; il est puissant par la force de son intelligence.
\VS{6}Il ne laisse pas vivre le méchant, et il fait droit aux pauvres.
\VS{7}Il ne détourne pas ses yeux de dessus les justes, il les place sur le trône avec les rois, il les y fait asseoir pour toujours, afin qu'ils soient élevés.
\VS{8}S'ils sont liés de chaînes, s'ils sont pris dans les liens de l'affliction,
\VS{9}il leur fait connaître leurs œuvres, leurs transgressions, leur orgueil.
\VS{10}Alors il ouvre leur oreille pour leur discipline, il leur dit de se détourner de l'iniquité.
\VS{11}S'ils écoutent, et s'ils le servent, ils achèvent leurs jours dans le bonheur, leurs années dans la joie.
\VS{12}S'ils n'écoutent pas, ils passent par l'épée, ils expirent dans leur aveuglement.
\VS{13}Les cœurs impies se mettent en colère, ils ne crient pas à lui quand il les a liés ;
\VS{14}leur âme meurt dans leur jeunesse, leur vie s'éteint comme celle des débauchés.
\VS{15}Mais Dieu sauve le pauvre dans son affliction, et c'est par la détresse qu'il lui ouvre les oreilles.
\VS{16}Il t'écartera aussi de la bouche de l'ennemi, pour te mettre au large, en pleine liberté, et ta table sera chargée de mets succulents.
\VS{17}Mais si tu es satisfait de ta cause comme le méchant, le jugement est inséparable de ta cause.
\VS{18}Que la colère ne t'incite pas à la moquerie, et que la grandeur de la rançon ne te fasse pas te détourner !
\VS{19}Tes cris valent-ils ton or, et même toutes les forces qui se trouvent dans tes richesses ?
\VS{20}Ne soupire pas après la nuit, qui enlève les peuples de leur place.
\VS{21}Garde-toi de te tourner vers le mal, car la misère t'y dispose.
\VS{22}Dieu est élevé par sa puissance ; qui saurait enseigner comme lui ?
\VS{23}Qui lui a prescrit ses voies ? Qui lui dira : Tu fais mal ?
\VS{24}Souviens-toi d'exalter ses ouvrages, que tous les hommes chantent.
\VS{25}Tout homme les voit, chacun les contemple de loin.
\VS{26}Dieu est grand, mais nous ne connaissons pas sa grandeur, le nombre de ses années est impénétrable.
\VS{27}Il attire à lui les gouttes d'eau, il les réduit en vapeur et forme la pluie ;
\VS{28}les nuées la laissent couler, elles coulent sur beaucoup d'hommes.
\VS{29}Et qui comprendra le déchirement des nuées, le fracas de sa tente ?
\VS{30}Voici, il étend autour de lui sa lumière, et il se cache jusque dans les profondeurs de la mer.
\VS{31}Par ses moyens, il juge les peuples, et il donne la nourriture avec abondance.
\VS{32}Il tient cachée dans sa main la lumière, il l'ordonne sur ses adversaires.
\VS{33}Il s'annonce par un grondement ; les troupeaux pressentent qu'il approche.
\Chap{37}
\TextTitle{Discours d'Elihu : conclusion}
\VerseOne{}Mon cœur est tout tremblant, il bondit hors de sa place.
\VS{2}Ecoutez, écoutez le frémissement de sa voix, le grondement qui sort de sa bouche !
\VS{3}Il le conduit dans toute l'étendue des cieux, et son éclair brille jusqu'aux extrémités de la terre.
\VS{4}Puis, sa voix rugit : Il tonne de sa voix majestueuse ; il ne retient plus l'éclair, dès que sa voix retentit.
\VS{5}Dieu tonne avec sa voix d'une manière merveilleuse ; il fait de grandes choses que nous ne comprenons pas.
\VS{6}Il dit à la neige : Tombe sur la terre ! Il le dit à la pluie, même aux plus fortes pluies.
\VS{7}Il met un sceau sur la main de chaque homme, afin que tous les hommes connaissent son œuvre.
\VS{8}Les bêtes entrent dans leurs tanières, et elles demeurent dans leurs repaires.
\VS{9}L'ouragan vient du midi, et le froid, des vents du nord.
\VS{10}Par son souffle, Dieu fait la glace, il réduit l'espace où se répandaient les eaux.
\VS{11}Il charge de vapeurs les nuages, il les disperse par sa lumière ;
\VS{12}leurs évolutions se font selon ses desseins, pour l'accomplissement de tout ce qu'il leur commande, sur la face de la terre habitée ;
\VS{13}c'est comme une verge dont il frappe sa terre, ou comme un signe de sa bonté, qu'il les fait apparaître.
\VS{14}Job, arrête-toi, prête l'oreille à ces choses ! Considère encore les merveilles de Dieu !
\VS{15}Sais-tu comment Dieu les dispose, et fait briller son nuage par sa lumière ?
\VS{16}Sais-tu comment se balancent les nuages, les merveilles de celui dont la science est parfaite ?
\VS{17}Sais-tu pourquoi tes vêtements sont chauds quand la terre se repose par le vent du midi ?
\VS{18}Peux-tu comme lui étendre les cieux, aussi fermes qu'un miroir de fonte ?
\VS{19}Fais-nous connaître ce que nous devons lui dire ; nous sommes en face des ténèbres pour nous comparer à lui.
\VS{20}Lui rapportera-t-on ce que je dirai ? Quel est l'homme qui se glorifie d'être englouti ?
\VS{21}On ne peut regarder la lumière du soleil qui resplendit dans les cieux, lorsqu'un un vent passe et en ramène la pureté ;
\VS{22}le nord l'amène éclatant comme l'or. Oh ! Que la majesté de Dieu est redoutable !
\VS{23}Nous ne saurions atteindre le Tout-Puissant, grand par la force, par la justice, par le droit, par la multitude de ses bienfaits, il n'opprime pas !
\VS{24}C'est pourquoi les hommes doivent le craindre ; il ne regarde pas ceux qui sont sages dans leur cœur.
\Chap{38}
\TextTitle{Yahweh interroge Job}
\VerseOne{}Yahweh répondit à Job du milieu de la tempête et dit :
\VS{2}Qui est celui qui obscurcit mes desseins par sa connaissance ?
\VS{3}Ceins tes reins comme un vaillant homme ; je t'interrogerai, et tu m'instruiras.
\VS{4}Où étais-tu quand je fondais la terre ? Dis-le, si tu as de l'intelligence.
\VS{5}Qui en a réglé les dimensions, le sais-tu ? Ou qui a étendu sur elle le cordeau ?
\VS{6}Sur quoi ses bases sont-elles fixées ? Ou qui en a posé la pierre angulaire,
\VS{7}quand les étoiles du matin se réjouissaient ensemble, et que tous les fils de Dieu poussaient des cris de joie ?
\VS{8}Qui a fermé la mer avec des portes, quand elle sortit en s'élançant du sein maternel ;
\VS{9}quand je fis de la nuée son vêtement, et de l'obscurité ses langes ;
\VS{10}quand je lui imposai ma loi, et que je lui mis des barrières et des portes ;
\VS{11}quand je dis : Tu viendras jusqu'ici, tu n'iras pas plus loin ; ici s'arrêtera l'orgueil de tes flots ?
\VS{12}Depuis que tu es au monde, as-tu commandé au matin ? As-tu fait savoir à l'aurore quelle était sa place,
\VS{13}pour qu'elle saisisse les extrémités de la terre, et que les méchants en soient secoués ;
\VS{14}pour que la terre change de forme comme l'argile qui reçoit un sceau, et qu'elle soit parée comme d'un vêtement ;
\VS{15}pour que les méchants soient privés de leur lumière, et que le bras qui se lève soit brisé ?
\VS{16}As-tu pénétré jusqu'aux sources de la mer ? T'es-tu promené dans les profondeurs de l'abîme ?
\VS{17}Les portes de la mort se sont-elles découvertes à toi ? As-tu vu les portes de l'ombre de la mort ?
\VS{18}As-tu compris l'étendue de la terre ? Si tu sais tout cela, dis-le !
\VS{19}Où est le chemin qui conduit à la demeure de la lumière ? Et les ténèbres, où ont-elles leur demeure ?
\VS{20}Peux-tu les saisir à leur limite, et connaître les sentiers de leur maison ?
\VS{21}Tu le sais, car alors tu étais né, et le nombre de tes jours est grand !
\VS{22}Es-tu allé jusqu'aux trésors de neige ? As-tu vu les trésors de grêle,
\VS{23}que je réserve pour les temps de détresse, pour les jours de guerre et de bataille ?
\VS{24}Par quel chemin la lumière se divise-t-elle, et le vent d'orient se répand-il sur la terre ?
\VS{25}Qui a ouvert un passage à la pluie, et tracé la route de l'éclair et du tonnerre,
\VS{26}pour qu'elle pleuve sur une terre sans habitants, sur un désert sans hommes ;
\VS{27}pour qu'elle abreuve les lieux solitaires et arides, et qu'elle fasse germer et sortir l'herbe ?
\VS{28}La pluie a-t-elle un père ? Qui enfante les gouttes de la rosée ?
\VS{29}De quel sein est sortie la glace ? Et qui enfante le givre du ciel,
\VS{30}pour que les eaux se cachent comme une pierre, et que la surface de l'abîme soit enchaînée ?
\VS{31}Noues-tu les liens des Pléiades, ou détaches-tu les cordages de l'Orion?
\VS{32}Fais-tu sortir en leur temps les signes du zodiaque, et conduis-tu la Grande Ourse avec ses petits ?
\VS{33}Connais-tu les lois du ciel ? Disposes-tu de son pouvoir sur la terre ?
\VS{34}Élèves-tu la voix jusqu'aux nuées, pour que des eaux abondantes te couvrent ?
\VS{35}Envoies-tu les éclairs ? Partent-ils ? Te disent-ils : Nous voici ?
\VS{36}Qui a mis la sagesse dans le cœur, ou qui a donné l'intelligence à l'esprit ?
\VS{37}Qui peut avec sagesse compter les nuages, et verser les outres des cieux,
\VS{38}pour que la poussière ruisselle, et que les mottes de terre se collent ensemble ?
\Chap{39}
\TextTitle{Yahweh démontre son omnipotence}
\VerseOne{}Chasses-tu la proie pour le vieux lion, et rassasies-tu les lionceaux,
\VS{2}quand ils sont couchés dans leur tanières, quand ils sont en embuscade dans leur repaire ?
\VS{3}Qui prépare au corbeau sa nourriture, quand ses petites crient vers le Dieu Puissant, quand ils sont errants parce qu’ils n’ont pas de quoi manger ?
\VS{4}Sais-tu quand les chèvres sauvages de Séla mettent bas ? Observes-tu les biches quand elles mettent bas ?
\VS{5}Comptes-tu les mois pendant lesquels elles sont pleines, et connais-tu l'époque où elles enfantent ?
\VS{6}Elles se courbent, laissent échapper leur progéniture, et sont libérées de leurs douleurs.
\VS{7}Leurs petits prennent de la vigueur et grandissent en plein air, ils s'éloignent et ne reviennent plus auprès d'elles.
\VS{8}Qui libère l'âne sauvage et l'affranchit de tout lien ?
\VS{9}J'ai fait du désert sa maison, de la terre salée son lieu d'habitation.
\VS{10}Il se rit du tumulte des villes, il n'entend pas les cris d'un maître.
\VS{11}Il parcourt les montagnes pour trouver sa pâture, il cherche après tout ce qui est vert.
\VS{12}Le buffle veut-il te servir ? Passe-t-il la nuit vers ta crèche ?
\VS{13}L'attaches-tu par une corde pour qu'il trace un sillon ? Va-t-il après toi briser les mottes des vallées ?
\VS{14}Te confies-tu à lui, parce que sa force est grande ? Lui laisses-tu le soin de tes travaux ?
\VS{15}Crois-tu en lui pour qu'il ramène ta récolte ? Est-ce lui qui doit recueillir ton aire ?
\VS{16}L'aile de l'autruche se déploie joyeusement ; on dirait l'aile, le plumage de la cigogne.
\VS{17}Mais l'autruche abandonne ses œufs à la terre, et les fait chauffer sur la poussière ;
\VS{18}elle oublie que le pied peut les écraser, qu'une bête des champs peut les fouler.
\VS{19}Elle est dure envers ses petits comme s'ils n'étaient pas à elle ; elle ne s'inquiète pas de l'inutilité de son enfantement.
\VS{20}Car Dieu l'a privée de sagesse, il ne lui a pas donné l'intelligence en partage.
\VS{21}Quand elle se lève et prend son envol, elle se rit du cheval et de son cavalier.
\VS{22}Est-ce toi qui donnes la vigueur au cheval, et qui revêt son cou d'une crinière flottante ?
\VS{23}Le fais-tu bondir comme la sauterelle ? Son fier hennissement répand la terreur.
\VS{24}Il creuse le sol et se réjouit de sa force, il sort au-devant des armes ;
\VS{25}il se rit de la crainte, il n'a pas peur, il ne retourne pas en face de l'épée.
\VS{26}Sur lui retentit le carquois, brillent la lance et le javelot.
\VS{27}Bouillonnant d'ardeur, il dévore la terre, il ne peut pas supporter au son de la trompette.
\VS{28}Aussi souvent que la trompette sonne, il dit : En avant ! Et de loin il flaire la bataille, la voix tonnante des chefs et les cris de guerre.
\VS{29}Est-ce par ton intelligence que l'épervier prend son vol, et qu'il étend ses ailes vers le midi ?
\VS{30}Est-ce par ton ordre que l'aigle s'élève, et qu'il place son nid sur les hauteurs ?
\VS{31}C'est dans les rochers qu'il habite, qu'il a sa demeure, sur la cime des rochers, sur les sommets des monts.
\VS{32}De là il épie sa proie, il regarde au loin.
\VS{33}Ses petits boivent le sang ; et là où sont des cadavres, l'aigle se trouve.
\TextTitle{Yahweh lui pose une question}
\VS{34}Yahweh, s'adressant à Job, dit :
\VS{35}Celui qui conteste contre le Tout-Puissant est-il convaincu ? Celui qui argumente avec Dieu a-t-il une réponse à faire ?
\TextTitle{Réponse de Job}
\VS{36}Job répondit à Yahweh et dit :
\VS{37}Voici, je suis trop vil ; que te répondrais-je ? Je mets la main sur ma bouche.
\VS{38}J'ai parlé une fois, je ne répondrai plus ; deux fois, je n'ajouterai rien.
\Chap{40}
\TextTitle{Yahweh questionne encore Job}
\VerseOne{}Yahweh répondit à Job du milieu de la tempête et dit :
\VS{2}Ceins tes reins comme un vaillant homme ; je t'interrogerai et tu m'apprendras.
\VS{3}Anéantiras-tu jusqu'à ma justice ? Me condamneras-tu pour te donner droit ?
\VS{4}As-tu un bras comme celui de Dieu, une voix tonnante comme la sienne ?
\VS{5}Orne-toi de magnificence et de grandeur, revêts-toi de splendeur et de gloire !
\VS{6}Répands les flots de ta colère, et d'un regard abaisse les hautains !
\VS{7}D'un regard humilie les hautains, écrase sur place les méchants,
\VS{8}cache-les tous ensemble dans la poussière, enferme leur face dans les ténèbres !
\VS{9}Alors je rends hommage à mon sauveur qui me sauve par sa droite.
\VS{10}Voici le béhémoth, que j'ai façonné comme toi ! Il mange de l'herbe comme le bœuf.
\VS{11}Le voici ! Sa force est dans ses reins, et sa vigueur dans les muscles de son ventre ;
\VS{12}il plie sa queue aussi ferme qu'un cèdre ; les tendons de ses cuisses sont entrelacés ;
\VS{13}ses os sont des tubes d'airain, ses membres sont comme des barres de fer.
\VS{14}Il est la première des voies de Dieu ; celui qui l'a fait le conduit vers l'épée.
\VS{15}Il prend sa pâture dans les montagnes, où se jouent toutes les bêtes des champs.
\VS{16}Il se couche sous les lotus, caché dans les roseaux et les marécages ;
\VS{17}les lotus le couvrent de leur ombre, les saules du torrent l'enveloppent.
\VS{18}Que le fleuve vienne à déborder, il ne s'enfuit pas. Que le Jourdain se précipite dans sa bouche, il reste confiant.
\VS{19}Est-ce en face qu'on pourra le saisir ? Est-ce au moyen de filets qu'on lui percera le nez ?
\VS{20}Attireras-tu le léviathan à l'hameçon ? Saisiras-tu sa langue avec une corde ?
\VS{21}Mettras-tu un jonc dans ses narines ? Lui perceras-tu la mâchoire avec un crochet ?
\VS{22}Accumulera-t-il les supplications ? Te parlera-t-il d'une voix douce ?
\VS{23}Fera-t-il une alliance avec toi, pour te prendre pour toujours comme esclave ?
\VS{24}Joueras-tu avec lui comme avec un oiseau ? L'attacheras-tu pour amuser les jeunes filles ?
\VS{25}Les pêcheurs en trafiquent-ils ? Le partagent-ils entre les marchands ?
\VS{26}Couvriras-tu sa peau de dards, et sa tête de harpons ?
\VS{27}Mets ta main contre lui, et tu ne te souviendras plus de l'attaquer.
\VS{28}Voici, on est trompé dans son attente ; à sa vue n'est-on pas terrassé ?
\Chap{41}
\VerseOne{}Nul n'est assez féroce pour l'exciter ; qui donc me résisterait en face ?
\VS{2}De qui suis-je le débiteur ? Je le paierai. Sous le ciel tout m'appartient.
\VS{3}Je veux encore parler de ses discours, et de sa force, et de la beauté de sa structure.
\VS{4}Qui découvrira son vêtement devant ma face ? Qui viendra freiner ses mâchoires par un mors ?
\VS{5}Qui ouvrira les portes devant sa face ? Autour du lion habite la terreur.
\VS{6}Ses magnifiques et puissants boucliers sont fermés comme un sceau ;
\VS{7}ils se serrent l'un contre l'autre, et l'air n'entrerait pas entre eux ;
\VS{8}ce sont des frères qui s'embrassent, se saisissent, demeurent inséparables.
\VS{9}Ses éternuements font briller la lumière, ses yeux sont comme les paupières de l'aurore.
\VS{10}Des flammes viennent de sa bouche, des étincelles de feu s'en échappent.
\VS{11}Une fumée sort de ses narines, comme d'un chaudron qui bout, d'une chaudière ardente.
\VS{12}Son souffle allume les charbons, de sa bouche sort la flamme.
\VS{13}La force a son cou pour demeure, et l'effroi bondit devant lui.
\VS{14}Ses parties charnues sont jointes ensemble, fondues sur lui, inébranlables.
\VS{15}Son cœur est dur comme la pierre, dur comme la meule inférieure.
\VS{16}Quand il se lève, les plus vaillants ont peur, et l'épouvante les fait quitter le droit chemin.
\VS{17}C'est en vain qu'on l’attaque avec l'épée ; la lance, le javelot, la cuirasse ne servent à rien.
\VS{18}Il regarde le fer comme de la paille, l'airain comme du bois pourri.
\VS{19}La flèche de l'arc ne le met pas en fuite, les pierres de la fronde sont pour lui changés en chaume.
\VS{20}Il ne voit dans la massue qu'un brin de paille, il rit au sifflement des dards.
\VS{21}Sous son ventre sont des pointes aiguës : On dirait une herse qu'il étend sur la boue.
\VS{22}Il fait bouillir les profondeurs de la mer comme une chaudière, il la traite comme un vase rempli de parfums.
\VS{23}Il laisse après lui un sentier lumineux ; l'abîme prend la chevelure d'un vieillard.
\VS{24}Sur la terre nul n'est son maître ; il a été façonné pour ne rien craindre.
\VS{25}Il regarde avec dédain tout ce qui est élevé, il est le roi des plus fiers animaux.
\Chap{42}
\TextTitle{Job reconnaît la souveraineté de Dieu et s'humilie}
\VerseOne{}Job répondit à Yahweh et dit :
\VS{2}Je sais que tu peux tout, et que rien n'empêche tes desseins.
\VS{3}Quel est celui qui a la folie d'obscurcir mes conseils ? Oui, j'ai parlé sans les comprendre, de merveilles qui me dépassent et que je ne connais pas.
\VS{4}Ecoute-moi, et je parlerai ; je t'interrogerai et tu m'instruiras.
\VS{5}Mes oreilles avaient entendu parler de toi ; mais maintenant mon œil t'a vu.
\VS{6}C'est pourquoi je me condamne et je me repens sur la poussière et sur la cendre.
\VS{7}Après que Yahweh eut ainsi parlé à Job, il dit à Eliphaz de Théman : Ma colère est enflammée contre toi et contre tes deux amis, parce que vous n'avez pas parlé de moi avec droiture comme Job, mon serviteur.
\VS{8}Prenez maintenant sept taureaux et sept béliers, allez auprès de mon serviteur Job, et offrez un holocauste pour vous. Job, mon serviteur, priera pour vous, et c'est par égard pour lui seul que je ne vous traiterai pas selon votre folie ; car vous n'avez pas parlé de moi avec droiture, comme mon serviteur Job.
\VS{9}Eliphaz de Théman, Bildad de Schuach, et Tsophar de Naama allèrent et firent comme Yahweh leur avait dit ; et Yahweh accorda une grâce face à la prière de Job.
\VS{10}Yahweh rétablit Job de sa captivité, quand il eut prié pour ses amis ; et Yahweh lui ajouta le double de tout ce qu'il avait possédé.
\VS{11}Les frères, les sœurs, et tous ceux qui l'avaient connu auparavant vinrent tous le visiter, et ils mangèrent avec lui dans sa maison. Ils se lamentèrent et le consolèrent au sujet de tout le mal que Yahweh avait fait venir sur lui, et chacun lui donna une kesita et un anneau d'or.
\VS{12}Pendant ses dernières années, Job reçut de Yahweh plus de bénédictions qu'il n'en avait reçu dans les premières. Il posséda quatorze mille brebis, six mille chameaux, mille paires de bœufs, et mille ânesses.
\VS{13}Il eut sept fils et trois filles :
\VS{14}Il donna à la première le nom de Jemima, à la seconde celui de Ketsia, à la troisième celui de Kéren-Happuc.
\VS{15}Et il ne se trouvait pas de femmes aussi belles que les filles de Job dans tout le pays. Leur père leur donna une part de l'héritage parmi leurs frères.
\VS{16}Job vécut, après ces choses, cent quarante ans, et il vit ses fils et les fils de ses fils jusqu'à la quatrième génération.
\VS{17}Et Job mourut âgé et rassasié de jours.
\PPE{}
\end{multicols}
