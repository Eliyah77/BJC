\ShortTitle{Job}\BookTitle{Job}\BFont
\noindent\hrulefill
{\footnotesize
\textit{
\bigskip
{\centering{}
\\Auteur : Inconnu
\\(Heb. : Iyov)
\\Signification : Haï, ennemi et « je m'exclamerai »
\\Thème : La souffrance
\\Date de rédaction : Incertaine\\}
}
%\bigskip
\textit{
\\Job était un homme prospère et intègre auquel Dieu rendit témoignage. Il subit une succession de malheurs en très peu de temps en perdant tout ce qui lui était cher. Après avoir cherché à se justifier et subi les railleries de sa femme et les accusations de ses amis, Job s'humilia devant Dieu et comprit l'impuissance de sa propre justice. Cette histoire, dont on n'a aucune indication spatio-temporelle et qui pourtant parle à tous, est un encouragement pour le juste éprouvé.
%\bigskip
\\Rappelant que la souffrance peut être le moyen choisi par Dieu pour enseigner et se révéler, ce récit illustre la fidélité et la bonté de Yahweh envers ceux qui le craignent.\bigskip
}
}
\par\nobreak\noindent\hrulefill
\begin{multicols}{2}
\Chap{1}
\TextTitle{Job et sa famille}
\VerseOne{}Il y avait dans le pays d'Uts\FTNT{Ge. 36:28.} un homme appelé Job\FTNT{Ez. 14:14; Ja. 5:11.}. Cet homme était intègre\FTNT{1 R. 8:61.} et droit, craignant\FTNT{Ps. 19:10; Pr. 1:7.} Dieu et se détournant du mal.
\VS{2}Il eut sept fils et trois filles.
\VS{3}Et son bétail était de sept mille brebis, trois mille chameaux, cinq cents paires de bœufs et cinq cents ânesses, avec un très grand nombre de serviteurs\FTNT{Job 42:12-13.}; tellement que cet homme était le plus puissant de tous les Orientaux.
\VS{4}Or ses fils allaient et faisaient des festins les uns chez les autres chacun à son jour, et ils envoyaient appeler leurs trois sœurs pour manger et boire avec eux.
\VS{5}Quand les jours de festin étaient passés, Job envoyait chercher ses enfants pour les sanctifier, et se levant de bon matin, il offrait des holocaustes selon le nombre de ses enfants ; car Job disait : Peut-être mes enfants ont-ils péché, et ont-ils blasphémé contre Dieu dans leurs cœurs. Job faisait toujours ainsi\FTNT{Job 42:8.}.
\VS{6}Or, il arriva un jour que les fils de Dieu\FTNT{Ps. 89:7 ; Job 38:7.} vinrent se présenter devant Yahweh, et Satan\FTNT{Es. 14:12; Ap. 12:9-10.} entra parmi eux.
\VS{7}Yahweh dit à Satan : D'où viens-tu ? Et Satan répondit à Yahweh : Je viens de courir çà et là sur la terre et de m'y promener\FTNT{1 Pi. 5:8.}.
\VS{8}Yahweh dit à Satan : N'as-tu point considéré mon serviteur Job, qui n'a point d'égal sur la terre ; homme intègre et droit, craignant Dieu, et se détournant du mal ?
\VS{9}Et Satan répondit à Yahweh : Est-ce en vain que Job craint Dieu ?
\VS{10}N'as-tu pas mis une haie tout autour de lui, autour de sa maison, autour de tout ce qui lui appartient ? Tu as béni l'œuvre de ses mains, et son bétail a fort multiplié sur la terre.
\VS{11}Mais étends maintenant ta main, touche à tout ce qui lui appartient, et tu verras s'il ne te maudit pas en face.
\VS{12}Et Yahweh dit à Satan : Voilà, tout ce qui lui appartient est en ton pouvoir ; seulement ne porte pas la main sur lui. Et Satan sortit de devant la face de Yahweh\FTNT{1 R. 22:22.}.
\TextTitle{Première attaque de Satan}
\VS{13}Il arriva donc qu'un jour, comme les fils et les filles de Job mangeaient et buvaient du vin dans la maison de leur frère aîné, un messager vint vers Job,
\VS{14}et lui dit : Les bœufs labouraient, et les ânesses paissaient à côté d'eux;
\VS{15}et ceux de Séba se sont jetés dessus, les ont pris, et ont frappé les serviteurs au tranchant de l'épée. Et je me suis échappé moi seul, pour te le rapporter.
\VS{16}Cet homme parlait encore, lorsqu'un autre vint et dit : Le feu de Dieu est tombé du ciel, il a embrasé les brebis et les serviteurs, et les a consumés\FTNT{2 R. 1:10-12.}. Et je me suis échappé moi seul, pour te le rapporter.
\VS{17}Cet homme parlait encore, lorsqu'un autre vint et dit : Les Chaldéens\FTNT{Ge. 11:28.} se sont rangés en trois bandes, se sont jetés sur les chameaux et les ont pris, ils ont frappé les serviteurs au tranchant de l'épée, et je me suis échappé moi seul, pour te le rapporter.
\VS{18}Cet homme parlait encore, lorsqu'un autre vint et dit : Tes fils et tes filles mangeaient et buvaient du vin dans la maison de leur frère aîné ;
\VS{19}voici, un grand vent est venu de l'autre côté du désert et a frappé contre les quatre coins de la maison ; elle est tombée sur les jeunes gens, et ils sont morts. Et je me suis échappé moi seul, pour te le rapporter.
\VS{20}Alors Job se leva, déchira\FTNT{Job 2:12 ; Est. 4:1.} son manteau et  rasa sa tête ; et se jetant par terre, se prosterna,
\VS{21}et dit : Je suis sorti nu du ventre de ma mère, et nu j'y retournerai\FTNT{Ec. 5:14. ; 1 Ti. 6:7.} ; Yahweh a donné, Yahweh a enlevé\FTNT{1 S. 2:6.} ; que le nom de Yahweh soit béni !
\VS{22}En tout cela, Job ne pécha pas et n'attribua rien à Dieu d'indigne de lui.
\Chap{2}
\TextTitle{Deuxième attaque de Satan}
\VerseOne{}Or il arriva un jour que les fils de Dieu vinrent pour se présenter devant Yahweh, Satan\FTNT{Za. 3:1-2.} vint aussi au milieu d'eux se présenter devant Yahweh.
\VS{2}Yahweh dit à Satan : D'où viens-tu ? Satan répondit à Yahweh : Je viens de courir çà et là sur la terre et de m'y promener.
\VS{3}Yahweh dit à Satan: N'as-tu point considéré mon serviteur Job, qui n'a point d'égal sur la terre ; homme sincère et droit, craignant Dieu, et se détournant du mal, et qui même encore retient son intégrité, quoique tu m'aies incité contre lui pour l'engloutir sans cause\FTNT{Job 9:17.}.
\VS{4}Et Satan répondit à Yahweh, en disant : Peau pour peau, tout ce qu’un homme possède, il le donnera pour sa vie.
\VS{5}Mais étends maintenant ta main, et frappe ses os et à sa chair\FTNT{Job 19:20.}, et tu verras s'il ne te blasphème pas en face. 
\VS{6}Yahweh dit à Satan : Voici, il est en ta main : Seulement ne touche pas à sa vie.
\VS{7}Ainsi Satan sortit de devant Yahweh, et frappa Job d’un ulcère malin, depuis la plante de ses pieds jusqu’au sommet de la tête.
\VS{8}Job prit un tesson pour se gratter et s'assit sur la cendre\FTNT{Jé. 6:26 ; Jon. 3:6.}.
\TextTitle{Réaction de Job et de sa femme}
\VS{9}Et sa femme lui dit : Conserveras-tu encore ton intégrité ? Bénis\FTNT{Job 1:11.} Dieu, et meurs !
\VS{10}Et il lui dit : Tu parles comme une femme insensée ! Nous recevons de Dieu les biens, et nous n'en recevrions pas les maux\FTNT{Es. 45:7 ; Am. 3:6 ; La. 3:37.} ! En tout cela, Job ne pécha pas par ses lèvres.
\TextTitle{Job et ses trois amis}
\VS{11}Or trois des intimes amis de Job, Eliphaz de Théman, Bildad de Schuach, et Tsophar de Naama, ayant appris tous les maux qui lui étaient arrivés, vinrent chacun du lieu de leur demeure, après s'être convenus ensemble d'un jour pour venir le plaindre et le consoler.
\VS{12}Et levant leurs yeux de loin, ils ne le reconnurent pas, et élevant leur voix, ils pleurèrent. Ils déchirèrent chacun leur manteau, et répandirent de la poussière sur leur tête en la jetant vers les cieux.
\VS{13}Et ils s'assirent à terre avec lui, sept jours et sept nuits, et aucun d'eux ne lui dit une parole, car ils voyaient que sa douleur était très grande.
\Chap{3}
\TextTitle{Lamentations de Job}
\VerseOne{}Après cela, Job ouvrit la bouche et maudit son jour\FTNT{Jé. 20:14 ; Job 10:18.}.
\VS{2}Car prenant la parole, il dit :
\VS{3}Périsse le jour où je suis né, et la nuit où l'on a dit : Un enfant mâle est né !
\VS{4}Que ce jour-là ne soit que ténèbres ; que Dieu ne le recherche pas d'en haut, et qu'il ne soit point éclairé de la lumière ! 
\VS{5}Que les ténèbres et l'ombre de la mort\FTNT{Job 10:21-22.} le rendent souillé, que les nuées demeurent sur lui, qu'il soit rendu terrible comme le jour de ceux à qui la vie est amère ! 
\VS{6}Que l'obscurité couvre cette nuit-là, qu'elle ne se réjouisse pas d'être parmi les jours de l'année, qu'elle ne soit pas comptée parmi les mois !
\VS{7}Voici, que cette nuit soit stérile, et qu'aucun cri de joie n'y survienne.
\VS{8}Qu'ils la maudissent ceux qui maudissent les jours, ceux qui sont prêts à réveiller le léviathan\FTNT{Ce nom vient de la mythologie phénicienne qui en fait le monstre du chaos primitif. C'est également un monstre marin évoqué dans la Bible (Es. 27:1 ; Ps. 74:14 ; 104:26 ; Job 40:25 ; 41:1).} !
\VS{9}Que les étoiles de son crépuscule soient obscurcies ; qu'elle attende la lumière, mais qu'il n'y en ait pas, et qu'elle ne voie pas les rayons de l'aube du jour\FTNT{Job 41:9.} !
\VS{10}Parce qu'elle n'a pas fermé le ventre qui m'a porté, et qu'elle n'a pas caché le tourment loin de mes yeux.
\VS{11}Pourquoi ne suis-je pas mort dans le sein de ma mère ? Pourquoi n'ai-je pas expiré aussitôt que je suis sorti de ses entrailles\FTNT{Job 10:18.} ?
\VS{12}Pourquoi des genoux m'ont-ils reçu? Pourquoi des mamelles m'ont-elles allaité ?
\VS{13}Car maintenant je serais couché, je me reposerais, je dormirais; il y aurait eu dès lors du repos pour moi\FTNT{Job 17:16.},
\VS{14}avec les rois et les conseillers de la terre, qui se sont bâtis des mausolées,
\VS{15}avec les princes qui ont possédé de l'or, et qui ont rempli d'argent leurs maisons.
\VS{16}Ou que n'ai-je été comme un avorton caché\FTNT{Ps. 58:9.}, comme les petits enfants qui n'ont pas vu la lumière.
\VS{17}Là les méchants ne tourmentent plus personne, et là se reposent ceux qui sont fatigués. 
\VS{18}Pareillement ceux qui avaient été dans les liens, jouissent là du repos, et n'entendent plus la voix de l'oppresseur. 
\VS{19}Le petit et le grand sont là, et l'esclave n'est plus sujet à son maître.
\VS{20}Pourquoi la lumière est-elle donnée au misérable, et la vie à ceux qui ont le cœur dans l'amertume ;
\VS{21}qui attendent en vain la mort, et qui la recherchent plus que le trésor\FTNT{Ap. 9:6.},
\VS{22}qui seraient ravis de joie et seraient dans l'allégresse s'ils avaient trouvé le sépulcre ?
\VS{23}Pourquoi, dis-je, la lumière est-elle donnée à l'homme à qui le chemin est caché, et que Dieu a enfermé de tous côtés\FTNT{Job 19:8 ; La. 3:7.} ?
\VS{24}Car avant que je mange, mon soupir vient, et mes rugissements coulent comme des eaux.
\VS{25}Ce que je crains le plus m'arrive, et ce que j'appréhende le plus m'atteint. 
\VS{26}Je n'ai pas eu de paix, je n'ai pas eu de repos, ni de calme, depuis que ce trouble m'est arrivé. 
\Chap{4}
\TextTitle{Premier discours d'Eliphaz}
\VerseOne{}Alors Eliphaz de Théman prit la parole et dit :
\VS{2}Si nous entreprenons de te parler, te fâcheras-tu ? Mais qui pourrait s'empêcher de parler ?
\VS{3}Voilà, tu en as enseigné plusieurs, et tu as renforcé les mains affaiblies\FTNT{Es. 35:3 ; Hé. 12:12.},
\VS{4}tes paroles ont affermi ceux qui chancelaient, et tu as fortifié les genoux qui pliaient\FTNT{Job 16:5.}.
\VS{5}Et maintenant que le malheur t'arrive, tu faiblis ! Maintenant que tu es atteint, tu en es tout troublé !
\VS{6} Ta piété de Yahweh n'a-t-elle pas été ton espérance ? Et l'intégrité de tes voies n'a-t-elle pas été ton attente ? 
\VS{7}Rappelle, je te prie, dans ton souvenir : Quel est l'innocent qui a péri ? Quels sont les hommes droits qui ont été exterminés\FTNT{Job 8:20.} ?
\VS{8}Selon ce que j'ai vu, ceux qui labourent l'iniquité et qui sèment l'outrage les moissonnent\FTNT{Job 15:35 ; Ga. 6:7.};
\VS{9}ils périssent par le souffle de Dieu, et ils sont consumés par le vent de ses narines\FTNT{Ex. 15:8 ; Es. 11:4 ; 30:33 ; Job 15:30 ; 2 Th. 2:8.}.
\VS{10}Il étouffe le rugissement du lion, et le cri d'un grand lion, et il arrache les dents des lionceaux ;
\VS{11}le lion périt faute de proie, et les petits du vieux lion sont dissipés.
\VS{12}Mais quant à moi, une parole m'a été adressée en secret, et mon oreille en a entendu quelque peu.
\VS{13}Au moment où les visions de la nuit agitent la pensée, quand un profond sommeil saisit les hommes\FTNT{Job 33:15.},
\VS{14}une frayeur et un tremblement me saisirent, et tous mes os tremblèrent.
\VS{15}Un esprit passa devant moi, et mes cheveux en furent tout hérissés. 
\VS{16}Il se tint là et je ne reconnus pas sa face ; une représentation était devant mes yeux. Et j'entendis une voix basse qui disait :
\VS{17}L'homme serait-il juste devant Dieu ? L'homme serait-il pur devant celui qui l'a fait\FTNT{Job 25:4.} ?
\VS{18}Voici, il ne se fie pas à ses serviteurs, il trouve la folie à ses anges\FTNT{Job 15:15 ; Job 25:5 ; 2 Pi. 2:4. },
\VS{19}combien plus chez ceux qui habitent des maisons d'argile, qui ont leurs fondements dans la poussière, qui sont consumés à la rencontre d'un vermisseau \FTNT{Job 25:6.}!
\VS{20}Du matin au soir ils sont brisés, et sans qu'on s'en aperçoive, ils périssent pour toujours. 
\VS{21}L'excellence qui était en eux, n'a-t-elle pas été emportée ? Ils meurent sans être sages. 
\Chap{5}
\VerseOne{}Crie maintenant ! Y aura-t-il quelqu'un qui te réponde ? Et vers qui d'entre les saints te tourneras-tu\FTNT{Job 15:15.}?
\VS{2}Certainement la colère tue l'insensé, et le dépit fait mourir le sot.
\VS{3}J'ai vu l'insensé qui s'enracinait \FTNT{Jé. 12:1-2.}, mais j'ai aussitôt maudit sa demeure.
\VS{4}Ses fils sont loin de tout secours ; ils sont écrasés à la porte, et personne ne les délivre\FTNT{Ps. 119:155.} !
\VS{5}Sa moisson est dévorée par l'affamé, qui même la ravit d'entre les épines ; et le voleur engloutit leurs biens.
\VS{6}Le malheur ne sort pas de la poussière, et le travail ne germe pas de la terre ;
\VS{7}l'homme naît pour la peine\FTNT{Ge. 3:17-19 ; Job 14:1-5.}, comme l'étincelle pour voler et s'élever.
\VS{8}Mais moi, j'aurais recours à Dieu, et j'adresserais mes paroles à Dieu.
\VS{9}Il fait de grandes choses qu'on ne peut sonder, de merveilleuses choses qu'il est impossible de compter\FTNT{Ps. 72:18. Ps. 92:5 ; Job 9:10.}.
\VS{10}Il répand la pluie sur la face de la terre, et envoie les eaux sur les campagnes\FTNT{De. 28:12 ; Ps. 135:7 ; Job 28:26; Job 38:25-26 ; Ac. 14:17.};
\VS{11}il élève ceux qui sont abaissés, et fait que ceux qui sont en deuil soient en sûreté dans une haute retraite\FTNT{1 S. 2:7; Ez. 21:31 ; Ps. 113:7-8.} ;
\VS{12}il anéantit les projets des hommes rusés, de sorte qu'ils ne viennent pas à bout de leurs entreprises\FTNT{Es. 8:10 ; Ps. 33:10 ; Né. 4:15.} ;
\VS{13}il surprend les sages en leur ruse\FTNT{1 Co. 3:19.}, et le conseil des méchants est renversé :
\VS{14}De jour ils rencontrent les ténèbres, et ils marchent à tâtons en plein midi, comme dans la nuit\FTNT{De. 28:29.}.
\VS{15}Ainsi il délivre les pauvres de leur épée et de leur bouche, et les sauve de la main des puissants\FTNT{Ps. 12:3-4; Ps 52:2; Ps. 57:4.} ;
\VS{16}ainsi il arrive au pauvre ce qu'il a espéré\FTNT{1S. 2:8.}, mais l'iniquité a la bouche fermée\FTNT{Es. 52:15 ; Ps. 63: 11; Ps. 107: 42; Pr. 10:6.}.
\VS{17}Voici, heureux est celui que Dieu châtie ! Ne rejette donc pas le châtiment du Tout-Puissant\FTNT{Ps 94:12 ; Pr.3:11-12 ; Hé. 12:5-6; Ap. 3:19.}.
\VS{18}Car c'est lui qui fait la plaie et qui la bande ; il blesse et ses mains guérissent\FTNT{De. 32:39; 1S. 2: 6-7 ; Cp. Es. 30:26 ; Os. 6:1.}.
\VS{19}Il te délivrera dans six afflictions, et à la septième le mal ne te touchera pas\FTNT{Ps 34:20; Ps. 91:3; Pr.24:16.}.
\VS{20}En temps de famine il te garantira de la mort, et en temps de guerre il te préservera de l'épée\FTNT{Ps. 33: 19; Ps. 37:19.}.
\VS{21}Tu seras à l'abri du fléau de la langue, et tu n’auras pas peur de la dévastation quand elle arrivera\FTNT{Ps. 31:21.}.
\VS{22}Tu riras de la dévastation et de la famine, et tu n'auras pas peur des bêtes de la terre\FTNT{Es. 65:25; Ez. 34:25; Os. 2:20.};
\VS{23}car tu feras une alliance avec les pierres des champs, et tu seras en paix avec les bêtes sauvages\FTNT{Os. 2:20.}.
\VS{24}Tu jouiras en paix de la prospérité sous ta tente, tu pourvoiras à ta demeure et tu n'y seras pas trompé ;
\VS{25}tu verras ta postérité s'accroître, et tes descendants se multiplier comme l'herbe de la terre\FTNT{Ps. 72:16; Ps. 127: 3-5; Ps. 128:6.}.
\VS{26}Tu entreras au sépulcre dans la vieillesse, comme un monceau de gerbes s'entasse en sa saison\FTNT{Pr. 9:11; Pr. 10:27.}.
\VS{27}Voilà nous avons examiné cela, et il est ainsi ; écoute-le et sache-le pour ton bien.
\Chap{6}
\TextTitle{Réponse de Job}
\VerseOne{}Job prit la parole et dit :
\VS{2}Plaise à Dieu que mon indignation soit bien pesée, et qu’on mette ensemble dans une balance ma calamité!
\VS{3}Car elle serait plus pesante que le sable de la mer ; c’est pourquoi mes paroles sont englouties\FTNT{Pr. 27:3.} !
\VS{4}Parce que les flèches du Tout-Puissant sont sur moi, mon esprit en suce le venin ; les terreurs\FTNT{Job 30:15 ; Ps. 88:16-17.} de Dieu se dressent en bataille contre moi\FTNT{Job 19:12 ; Ps. 38:2-3.}.
\VS{5}L'âne sauvage\FTNT{Job 39:8.} brait-il auprès de l’herbe ? Le bœuf mugit-il auprès de son fourrage ?
\VS{6}Mange-t-on sans sel ce qui est fade ? Trouve-t-on du goût dans un blanc d’œuf ?
\VS{7}Pour moi les choses que je n'aurais pas seulement voulu toucher sont des saletés qu'il faut que je mange !
\VS{8}Oh ! Puisse ma prière s'accomplir et Dieu me donner ce que j'attends !
\VS{9}Qu’il plaise à Dieu de me réduire en poussière, qu’il laisse aller sa main pour m’achever\FTNT{Job 7:16; 9:21; 10:1; cp. No. 11:15; 1R. 19:4; Jon. 4:3, 8.} !
\VS{10}Mais j’ai encore cette consolation, quoique la douleur me consume, et qu’elle ne m’épargne point, je n'ai pas transgressé les paroles du Saint.
\VS{11}Quelle est ma force, pour je que puisse soutenir de si grands maux? Et quelle en est la fin pour que je puisse prolonger ma vie ? 
\VS{12}Ma force est-elle une force de pierre ? Ma chair est-elle d'acier ?
\VS{13}Ne suis-je pas destitué de secours, et tout appui n'est-il pas éloigné de moi ?
\VS{14}Celui qui se fond sous l'ardeur des maux a droit à la compassion de son ami ; sinon il abandonne la crainte\FTNT{Ps. 19:10.} du Tout-Puissant\FTNT{Pr. 17:17.}.
\VS{15}Mes frères m'ont trahi comme un torrent, comme le cours impétueux des torrents qui passent\FTNT{Ps. 38:12; Ps 41:10; Ps 69:9; Jé.15:19.},
\VS{16}qu'on ne voit pas à cause de la glace, et sur lesquels s'entasse la neige.
\VS{17}lesquels, au temps que la chaleur donne dessus, défaillent ; quand ils sentent la chaleur, ils disparaissent de leur lieu;
\VS{18}lesquels serpentant ça et là par les chemins, se réduisent à rien, et se perdent.
\VS{19}Les troupes des voyageurs de Théma\FTNT{Ge. 25:15.} y pensaient, ceux qui vont à Séba\FTNT{1R. 10:1; Ps. 72:10; Ez. 27:22-23.} s'y attendaient ;
\VS{20}mais ils sont honteux d'y avoir espéré; ils y sont allés, et ils ont rougi.
\VS{21}Certes, vous m'êtes devenus inutiles ; vous voyez ma calamité étonnante, et vous en avez horreur\FTNT{Job 19:13 ; Ps. 31:12.}!
\VS{22}Mais vous ai-je dit : Donnez-moi quelque chose, et de vos biens, faites des présents en ma faveur,
\VS{23}délivrez-moi de la main de l'ennemi, et rachetez-moi de la main des méchants ?
\VS{24}Instruisez-moi, et je me tairai ; faites-moi comprendre en quoi je me suis égaré.
\VS{25}Ô combien sont fortes les paroles de vérité ! Mais votre censure, à quoi tend-elle ?
\VS{26}Pensez-vous qu'il ne faille avoir que des paroles pour censurer; et que les discours de celui qui est hors d'espérance, ne soient que du vent\FTNT{Ec. 9:16.}?
\VS{27}Vous vous jetez même sur un orphelin et vous persécutez votre ami.
\VS{28}Mais maintenant je vous prie, regardez-moi bien, si je mens en votre présence ?
\VS{29}Revenez\FTNT{Job 17:10.} donc, soyez sans injustice; revenez, et reconnaissez mon innocence\FTNT{Job 27:5-6 ; 34:5 ; cp. Job 23:10 ; 42:1-6.}.
\VS{30}Y a-t-il de l'iniquité dans ma langue ? Et mon palais ne sait-il pas discerner mes calamités ? 
\Chap{7}
\VerseOne{}N'y a-t-il pas un temps de guerre limité à l'homme sur la terre ? Et ses jours ne sont-ils pas comme les jours d'un mercenaire ?
\VS{2}Comme le serviteur soupire après l'ombre, comme un ouvrier\FTNT{Es. 16:14.} attend son salaire\FTNT{Ps. 39:5.}, 
\VS{3}ainsi il m'a été donné pour mon partage des mois qui ne m'apportent rien; il m'a été assigné des nuits de travail\FTNT{Ps. 6:6.}.
\VS{4}Si je suis couché, je dis: Quand me lèverai-je ? Et quand est-ce que la nuit aura achevé sa mesure ? Et je suis plein d'inquiétudes jusqu'au point du jour\FTNT{De. 28:67.}.
\VS{5}Ma chair se couvre de vers et de monceaux de poussière, ma peau se crevasse et se dissout.
\VS{6}Mes jours passent plus légèrement que la navette d'un tisserand, et ils se consument sans espérance\FTNT{Es. 38:12 ; Job 9:25 ; 17:11; Ja. 4:14.}!
\VS{7}Souviens-toi que ma vie est un souffle ! Et que mes yeux ne reverront plus le bonheur\FTNT{Es. 40:6 ; Ps. 78:39 ; Ps. 89:48 ; Ps. 102: 12 ; Ps. 103:15 ; Job 8:9 ; Job 14:1-2 ; 1P. 1:24.}.
\VS{8}L'œil de ceux qui me regardent ne me verra plus ; tes yeux seront sur moi, et je ne serai plus.
\VS{9}La nuée se dissipe et s'en va, ainsi celui qui descend au scheol\FTNT{cp. Ha. 2:5 ; Lu. 16:23.} ne remontera pas\FTNT{Job 10:21-22 ; Job 14:7-14.} ;
\VS{10}il ne reviendra plus dans sa maison, et le lieu qu'il habitait ne le reconnaîtra plus\FTNT{Ps. 37:35-36 ; Ps. 103:16 ; Job 10:21.}.
\VS{11}C'est pourquoi je ne retiendrai pas ma bouche, je parlerai dans l'angoisse de mon esprit, je discourrai dans l'amertume de mon âme\FTNT{Job 10:1.}.
\VS{12}Suis-je une mer ? Suis-je un monstre marin pour que tu poses autour de moi des gardes ?
\VS{13}Quand je dis : Mon lit me soulagera ; le repos diminuera quelque chose de ma plainte,
\VS{14}alors tu me brises par des songes, et tu me troubles par des visions.
\VS{15}C'est pourquoi je choisirais d'être étranglé, et de mourir, plutôt que de conserver mes os.
\VS{16}Je suis ennuyé de la vie, aussi ne vivrai-je pas toujours. Retire-toi de moi, car mes jours ne sont que vanité\FTNT{Job 10:20.}.
\VS{17}Qu'est-ce que l'homme mortel pour que tu le regardes comme quelque chose de grand, pour que tu l'affectionne\FTNT{Ps. 8:5 ; Ps. 144:3 ; Hé. 2:6.},
\VS{18}pour que tu le visites tous les matins, pour que tu l'éprouves\FTNT{Job 23:10.} à chaque instant ?
\VS{19}Jusqu'à quand ne te retireras-tu pas de moi ? Ne me permettras-tu pas d'avaler ma salive\FTNT{Job 9:18.}?
\VS{20}J'ai péché ; que te ferai-je, gardien des hommes\FTNT{1 Ti. 4:10.} ? Pourquoi m'as-tu mis en butte à tes coups, et pourquoi suis-je à charge à moi-même ?
\VS{21}Et pourquoi ne pardonnes-tu pas mon péché, et ne fais-tu pas passer mon iniquité ? Car je vais maintenant dormir dans la poussière ; tu me chercheras, et je ne serai plus.
\Chap{8}
\TextTitle{Premier discours de Bildad}
\VerseOne{}Bildad de Schuach prit la parole et dit :
\VS{2}Jusqu'à quand parleras-tu ainsi, et les paroles de ta bouche seront-elles un vent impétueux\FTNT{Job 15:2.} ?
\VS{3}Dieu renverserait-il le droit\FTNT{Cp. Ge. 18:25.}, et le Tout-Puissant renverserait-il la justice\FTNT{De. 32:4 ; Job 34:12 ; Da. 9:14 ; 2 Ch. 19:7.} ?
\VS{4}Si tes fils ont péché contre lui, il les a livrés à leur crime.
\VS{5}Mais toi, si tu cherches Dieu, si tu demandes grâce au Tout-Puissant\FTNT{Cp. Job 5:17-27.} ;
\VS{6}si tu es pur et droit, certainement il se réveillera pour toi, et fera prospérer la demeure de ta justice.
\VS{7}Et ton commencement\FTNT{Za. 4:10.} aura été petit, mais ta dernière condition sera bien plus grande\FTNT{Job 42:12.}.
\VS{8}Car, je te prie, enquiers-toi des générations précédentes, et applique-toi à t'informer soigneusement de leurs pères\FTNT{De. 4:32 ; De. 32:7.}.
\VS{9}Car nous sommes d'hier, et nous ne savons rien, parce que nos jours sont sur la terre comme une ombre\FTNT{Ps. 102:12 ; Ps. 144:41 ; Ch. 29:15.}.
\VS{10}Ceux-là ne t'enseigneront-ils pas, ne te parleront-ils pas, et ne tireront-ils pas de leur cœur des discours ? :
\VS{11}Le roseau croît-il sans marais ? Le jonc pousse-t-il sans eau ?
\VS{12}Il est encore en sa verdure, sans qu'on le coupe, il sèche plus vite que toutes les herbes\FTNT{Cp. Jé. 17:5-8 ; Ps. 129:6.}.
\VS{13}Ainsi est la voie de tous ceux qui oublient Dieu\FTNT{Ps. 9:18.}, et l'espérance de l'hypocrite périra\FTNT{Ps. 1:4 ; Ps. 112:10 ; Pr. 10:28 ; Job 11:20 ; Job 27:8.}.
\VS{14}Sa confiance est frustrée, et sa confiance est comme une toile d'araignée.
\VS{15}Il s'appuie sur sa maison, et elle ne tient pas ; il s'y cramponne, et elle ne reste pas debout.
\VS{16}Mais l'homme intègre est plein de vigueur étant exposé au soleil, et ses jets poussent par dessus son jardin,
\VS{17}ses racines s'entrelacent près de la fontaine, et il embrasse le bâtiment de pierres.
\VS{18}Fera-t-on qu'il ne soit plus à sa place et que le lieu où il était le renie, en lui disant je ne t'ai pas vu! 
\VS{19}Telle est la joie que ses voies lui procurent. Puis sur le même sol, d'autres s'élèvent après lui.
\VS{20}Dieu ne rejette pas l'homme intègre, il ne soutient pas la main des méchants\FTNT{Job 4:7.}.
\VS{21}Il remplira encore ta bouche de cris de joie, et tes lèvres de chants d'allégresse\FTNT{Ps. 126:2.}.
\VS{22}Ceux qui te haïssent seront revêtus de honte, et la tente des méchants ne sera plus\FTNT{Ps. 35:26 ; Ps. 109:29.}.
\Chap{9}
\TextTitle{Réponse de Job}
\VerseOne{}Job prit la parole et dit :
\VS{2}Certainement, je sais qu'il en est ainsi ; et comment l'homme mortel se justifierait-il\FTNT{Ha. 2:4 ; Ga. 3:11 ; Ro. 1:17 ; Hé. 10:38.} devant Dieu\FTNT{Ps. 25:4 ; Ps. 143:2 ; Job 15:14-16 ; Da. 9:11 ; Ro 3:19.} ?
\VS{3}Si Dieu veut plaider avec lui de mille articles, il ne saurait lui répondre sur un seul\FTNT{Es. 45:9-10.}.
\VS{4} Dieu est sage de coeur, et puissant en force. Qui est-ce qui s'est opposé à lui, et s'en est bien trouvé\FTNT{Job 12:13 ; Job 36:5 ; Job 37:23.} ?
\VS{5}Il transporte les montagnes, et quand il les renverse dans sa fureur, elles n'en connaissent rien\FTNT{Ps. 144:5.}.
\VS{6}Il remue la terre de sa place, et ses piliers sont ébranlés\FTNT{Ag. 2:6, 21 ; Hé. 12:26.}.
\VS{7}Il parle au soleil, et le soleil ne se lève pas ; et il met un sceau sur les étoiles\FTNT{Jos. 10:12.}.
\VS{8} C'est lui seul qui étend les cieux\FTNT{Ge 1:6-8 ; Es. 44:24; Es. 51:13 ; Ps. 104:2.}, qui marche sur les hauteurs de la mer\FTNT{Cp. Mt. 14:25.}.
\VS{9}Il a fait la grande ourse, l'orion, les pléiades, et les étoiles des régions australes\FTNT{Ge. 1:16 ; Am. 5:8 ; Ps. 89:12 ; Job 38:31-32.}.
\VS{10}Il fait de grandes choses qu'on ne peut sonder, des merveilles sans nombre\FTNT{Ps. 86:10 ; Ps. 139:6, 17-18 ; Job 5:9 ; Job 37:5.}.
\VS{11}Voici, il passe près de moi, et je ne le vois pas ; il passe encore, et je ne l'aperçois pas\FTNT{Job 23:8-9 ; 35:14.}.
\VS{12}Voilà, s'il ravit, qui le lui fera rendre? Et qui lui dira : Que fais-tu\FTNT{Es. 45: 9-10 ; Da. 4:35 ; Ro. 11:33-35.} ?
\VS{13}Si Dieu ne retire pas sa colère, les orgueilleux qui viennent au secours s'inclinent sous lui\FTNT{Job 26:12; Cp. Es. 30:7.}.
\VS{14}Combien moins lui répondrais-je, moi, et comment choisirais-je mes paroles contre lui ? 
\VS{15}Moi, je ne lui répondrai pas, quand même je serais juste, je demanderai grâce à mon juge\FTNT{Job 23:1-7}.
\VS{16}Si je l'invoque et qu'il me réponde, je ne croirais pas qu'il ait écouté ma voix,
\VS{17}car il m'a écrasé du milieu d'un tourbillon, et il a ajouté plaie sur plaie, sans que je l'aie mérité\FTNT{Job 6:29.}, 
\VS{18}il ne me permet pas de reprendre haleine ; il me rassasie d'amertume\FTNT{Job 7:19.}.
\VS{19}S'il est question de savoir qui est le plus fort ; voilà, il est fort ; et s'il est question d'aller en justice, qui est-ce qui m'y fera comparaître ? 
\VS{20}Si je me justifie, ma propre bouche me condamnera ; si je me fais parfait, il me convaincra d'être coupable.
\VS{21}Je suis innocent ! Je ne me soucie pas de vivre, je méprise ma vie\FTNT{Job 10:1.}.
\VS{22}Tout revient à un ! C'est pourquoi j'ai dit: Il consume l'homme juste et  le méchant\FTNT{ Cp. Ez. 21:3 ; Ec. 9:2-3 ; Mt 5:45.}.
\VS{23}Au moins si le fléau dont il frappe faisait mourir tout aussitôt ; mais il se rit de l'épreuve des innocents. 
\VS{24}C'est par lui que la terre est livrée entre les mains du méchant ; c'est lui qui couvre la face des juges de la terre ; et si ce n'est pas lui, qui est-ce donc ? 
\VS{25}Or mes jours vont plus vite qu'un courrier ; ils s'enfuient sans avoir vu le bien\FTNT{Job 7:6-7.};
\VS{26}ils passent comme des barques de poste; comme un aigle qui se précipite sur sa proie.
\VS{27}Je dis : J'oublierai ma plainte, je renoncerai à ma colère, je me fortifierai; 
\VS{28}je suis épouvanté de tous mes tourments. Je sais que tu ne me jugeras pas innocent\FTNT{Cp. Ps. 130:3.}.
\VS{29}Je serai trouvé méchant ; pourquoi travaillerais-je en vain ?
\VS{30}Quand je me laverais dans de l'eau de neige, et que je nettoierais mes mains dans la pureté\FTNT{Jé. 2:22.}, 
\VS{31}tu me plongerais dans le fossé et mes vêtements m'auraient en horreur.
\VS{32}Car il n'est pas comme moi un homme, pour que je lui réponde, et que nous allions ensemble en jugement\FTNT{Es. 45:9 ; Jé 49:19 ; Ec. 6:10; Ro. 9:20.}.
\VS{33}Mais il n'y a personne qui prend connaissance de la cause qui serait entre nous, et qui met la main sur nous deux\FTNT{Cp. 1 S. 2:25.}.
\VS{34}Qu'il ôte donc sa verge de dessus moi, et que la frayeur que j'ai de lui ne me trouble plus.
\VS{35}Je parlerai et je ne le craindrai pas ; mais dans l'état où je suis je ne suis plus à moi-même. 
\Chap{10}
\VerseOne{}Mon âme est dégoûtée de ma vie ! Je m'abandonnerai à ma plainte, je parlerai dans l'amertume de mon âme.
\VS{2}Je dirai à Dieu : Ne me condamne pas ; montre-moi pourquoi tu plaides contre moi.
\VS{3}Te plais-tu à m'opprimer, et à dédaigner l'ouvrage de tes mains, et à bénir les desseins des méchants\FTNT{Es. 64:7-8.} ?
\VS{4}As-tu des yeux de chair ? Vois-tu comme voit un homme mortel?
\VS{5}Tes jours sont-ils comme les jours de l'homme mortel ? Tes années sont-elles comme les jours de l'homme, 
\VS{6}pour que tu recherches mon iniquité, et que tu t'informes de mon péché ?
\VS{7}Tu sais que je n'ai point commis de crime, et qu'il n'y a personne qui me délivre de ta main?
\VS{8}Tes mains m'ont formé et elles ont rangé toutes les parties de mon corps; et tu me détruirais\FTNT{Ge. 2:7 ; Ps. 119:73 ; Ps. 139:14-15.} !
\VS{9}Souviens-toi, je te prie, que tu m'as formé comme de la boue, et que tu me feras retourner à la poussière. 
\VS{10}Ne m'as-tu pas coulé comme du lait ? Et ne m'as-tu pas fait cailler comme un fromage ?
\VS{11}Tu m'as revêtu de peau et de chair, et tu m'as composé d'os et de nerfs;
\VS{12}tu m'as donné la vie, et tu as usé de miséricorde envers moi, et par tes soins continuels tu as gardé mon esprit.
\VS{13}Et cependant tu gardais ces choses en ton cœur ; mais je connais que cela était devant toi. 
\VS{14}Si je pèche, tu me remarques; et tu ne me tiens pas quitte de mon iniquité.
\VS{15}Si j'agis méchamment, malheur à moi ! Si je suis juste, je n'en lève pas la tête plus haut. Je suis rempli d'ignominie, mais regarde mon affliction. 
\VS{16}Si je redresse la tête, tu me poursuis comme un grand lion, et tu multiplies tes exploits contre moi\FTNT{Zq. 38:13 ; La. 3:10.}.
\VS{17}Tu renouvelles tes témoins contre moi, et ton indignation augmente contre moi. De nouvelles troupes toutes fraîches viennent contre moi.
\VS{18}Mais pourquoi m'as-tu fait sortir de la matrice? J'aurais expiré, et aucun œil ne m'aurait vu ;
\VS{19}et j'aurais été comme n'ayant jamais été, et j'aurais été porté du ventre de ma mère au tombeau.
\VS{20}Mes jours ne sont-ils pas en petit nombre ? Cesse donc et retire-toi de moi et permets que je me renforce un peu.
\VS{21}Avant que j'aille au lieu d'où je ne reviendrai plus, en la terre de ténèbres et de l'ombre de la mort,
\VS{22}terre d'une grande obscurité, comme étant les ténèbres de l'ombre de la mort, où il n'y a aucun ordre, et où rien ne luit que des ténèbres. 
\Chap{11}
\TextTitle{Première accusation de Tsophar}
\VerseOne{}Tsophar de Naama prit la parole et dit :
\VS{2}Ne répondra-t-on pas à tant de discours, et suffira-t-il d'être un grand parleur pour être justifié ?
\VS{3}Tes vains discours feront-ils taire les gens ? Et quand tu te seras moqué, n'y aura-t-il personne qui te fasse honte ?
\VS{4}Car tu as dit : Ma doctrine est pure, et je suis sans tache devant tes yeux. 
\VS{5}Mais certainement, il serait souhaitable que Dieu parle, et qu'il ouvre ses lèvres pour parler avec toi ;
\VS{6}qu'il t'explique les secrets de sa sagesse, à savoir qu'il devrait redoubler la conduite qu'il tient envers toi; sache donc que Dieu exige de toi beaucoup moins que ton iniquité ne mérite.
\VS{7}Trouveras-tu le fond en Dieu en le sondant ? Connaîtras-tu parfaitement le Tout-Puissant ? 
\VS{8}Ce sont les hauteurs des cieux : Qu'y feras-tu ? C'est plus profond que le scheol : Qu'y connaîtras-tu ?
\VS{9}Son étendue est plus longue que la terre et plus large que la mer.
\VS{10}S'il s'avance, emprisonne et fait comparaître, qui l'en détournera?
\VS{11}Car il connaît les hommes perfides, il discerne par le regard les coupables\FTNT{Ps. 10:11-14 ; Ps. 35:22.}.
\VS{12}Mais l'homme vide de sens devient intelligent, quoique l'homme naisse comme un ânon sauvage\FTNT{Ec. 3:18.}.
\VS{13}Si tu disposes ton cœur, et que tu étends tes mains vers lui,
\VS{14}si tu éloignes de toi l'iniquité qui est en ta main, et si tu ne permets pas que la méchanceté habite dans tes tentes, 
\VS{15}alors certainement tu pourras élever ton visage sans tache. Tu seras ferme et tu ne craindras rien;
\VS{16}tu oublieras tes peines, tu t'en souviendras comme des eaux écoulées.
\VS{17}La vie se lèvera pour toi plus brillante que le midi, et l'obscurité même sera comme le matin\FTNT{Ps. 37:6 ; Ps. 112:4.}.
\VS{18} Tu seras plein de confiance parce qu'il y aura de l'espérance pour toi; tu creuseras, et tu reposeras sûrement\FTNT{Lé. 26:6 ; Ps. 3:6 ; Pr. 3:24.}.
\VS{19}Tu te coucheras, et il n'y aura personne qui t'épouvante, et plusieurs te feront la cour. 
\VS{20}Mais les yeux des méchants seront consumés; tout refuge leur sera ôté et toute leur espérance sera de rendre l'âme !
\Chap{12}
\TextTitle{Réplique de Job}
\VerseOne{}Job reprit la parole et dit :
\VS{2}Vraiment, êtes-vous tout un peuple ; et la sagesse mourra-t-elle avec vous ?
\VS{3}J'ai du bon sens aussi bien que vous, et je ne vous suis point inférieur ; et qui est-ce qui ne sait de telles choses ?
\VS{4}Je suis un homme qui est la risée de ses amis, mais qui invoque Dieu et Dieu lui répond. On se moque d'un homme qui est juste et droit !
\VS{5} Celui dont les pieds sont tout prêts de glisser, est une lampe méprisée pour les pensées de celui qui est à son aise. 
\VS{6}Ce sont les tentes des voleurs qui prospèrent, et ceux-là qui irritent Dieu sont assurés, et ils sont ceux à qui Dieu remet tout entre les mains\FTNT{ Jé. 12:1 ; Ps. 73:12.}.
\VS{7}En effet, je te prie, interroge les bêtes, et chacune d'elles t'enseignera ; ou les oiseaux des cieux, et ils te le déclareront ;
\VS{8}ou parle à la terre, et elle t'enseignera ; même les poissons de la mer te le raconteront. 
\VS{9}Car qui est-ce qui ne sait toutes ces choses; que c'est la main de Yahweh qui a fait cela, 
\VS{10}qu'il tient en sa main l'âme de tout ce qui vit, et l'esprit de toute chair d'homme ?
\VS{11}L'oreille ne discerne-t-elle pas les discours, ainsi que le palais savoure les mets ?
\VS{12}La sagesse est dans les vieillards, et l'intelligence est le fruit d'une longue vie.
\VS{13}Mais en Dieu est la sagesse et la force ; à lui appartient le conseil et l'intelligence\FTNT{Da. 2:20.}.
\VS{14}Voici, il démolit, et on ne rebâtit pas ; il enferme un homme et on ne lui ouvre pas\FTNT{Es. 22:22 ; Ap. 3:7.}.
\VS{15}Voilà, il retient les eaux, et tout devient sec ; il les lâche, et elles renversent la terre.
\VS{16}En lui résident la force et la sagesse ; à lui est celui qui s'égare, et celui qui le fait égarer,
\VS{17}il emmène dépouillés les conseillers et il met hors de sens les juges\FTNT{2 S. 15:31 ; 2 S. 17:14-23 ; Es. 19:12 ; Es. 29:14 ; 1 Co. 1:19.}.
\VS{18}Il détache la ceinture des rois, et il serre leurs reins de cordes.
\VS{19}Il fait marcher pieds nus les sacrificateurs et il renverse les forts.
\VS{20}Il ôte la parole à ceux qui sont les plus assurés en leurs discours, et il prive de sens les anciens.
\VS{21}Il répand le mépris sur les grands ; il relâche la ceinture des forts\FTNT{Es. 40:23.}.
\VS{22}Il met en évidence les choses qui étaient cachées dans les ténèbres, et il met en lumière l'ombre de la mort\FTNT{Ps. 139:11-12 ; Ec. 12:16 ; Mt. 10:26 ; 1 Co. 4:6.}.
\VS{23} Il multiplie les nations et les fait périr ; il répand çà et là les nations, puis il les ramène. 
\VS{24}Il ôte le cœur aux chefs des peuples de la terre, et les fait errer dans les déserts où il n'y a pas de chemin;
\VS{25}ils vont à tâtons dans les ténèbres, sans aucune clarté, et il les fait chanceler comme des gens ivres. 
\Chap{13}
\VerseOne{}Voici, mon œil a vu toutes ces choses, mon oreille les a entendues et comprises.
\VS{2}Comme vous les savez, je les sais aussi ; je ne vous suis pas inférieur. 
\VS{3}Mais je veux parler au Tout-Puissant, je veux plaider auprès de Dieu.
\VS{4}Et certes vous inventez des mensonges ; vous êtes tous des médecins inutiles.
\VS{5}Plaît à Dieu que vous demeuriez entièrement dans le silence ; et cela vous sera réputé à sagesse\FTNT{Pr. 17:28.}.
\VS{6}Ecoutez donc maintenant mon raisonnement et soyez attentifs à la défense de mes lèvres.
\VS{7}Tiendrez-vous des discours injustes en faveur de Dieu, direz-vous quelque fausseté pour lui ? 
\VS{8}Ferez-vous acception de sa personne, si vous plaidez la cause de Dieu? 
\VS{9}Vous est-il plaisant qu'il vous sonde ? Vous jouerez-vous de lui, comme on se joue d'un homme mortel ? 
\VS{10}Certainement il vous censurera, si même en secret vous faites acception de personnes.
\VS{11}Sa majesté ne vous épouvantera-t-elle pas ? Et sa frayeur ne tombera-t-elle pas sur vous ? 
\VS{12}Vos discours mémorables sont des sentences de cendre, et vos éminences sont des éminences de boue. 
\VS{13}Taisez-vous devant moi, et que je parle ; et il m'arrivera ce qui pourra. 
\VS{14}Pourquoi porterais-je ma chair entre mes dents et tiendrais-je mon âme entre mes mains\FTNT{Jg. 12:3 ; 1 S. 19:5.} ?
\VS{15}Voilà, qu'il me tue, je ne cesserai pas d'espérer en lui ; et je défendrai ma conduite en sa présence.
\VS{16}Et qui plus est, il sera lui-même ma délivrance ; mais l'hypocrite ne viendra pas devant sa face\FTNT{Ps. 1:5.}.
\VS{17}Ecoutez attentivement mes paroles et prêtez l'oreille à ce que je vais vous déclarer. 
\VS{18}Voici, j'ai préparé ma cause. Je sais que je serai justifié.
\VS{19}Qui est-ce qui veut disputer contre moi ? Car maintenant si je me tais, je mourrai. 
\VS{20}Seulement ne me fais pas ces deux choses, et alors je ne me cacherai pas devant ta face :
\VS{21}Retire ta main de dessus moi et que ta frayeur ne me trouble pas.
\VS{22}Puis appelle-moi et je répondrai ; ou bien je parlerai et tu me répondras. 
\VS{23}Combien ai-je d'iniquités et de péchés ? Montre-moi mon crime et mon péché. 
\VS{24}Pourquoi caches-tu ta face, et me tiens-tu pour ton ennemi ?
\VS{25}Déploieras-tu tes forces contre une feuille que le vent emporte ? Poursuivras-tu du chaume tout sec\FTNT{1 S. 24:15.},
\VS{26}pour que tu écrives contre moi des choses amères\FTNT{Ps. 25:7.},
\VS{27}pour que tu mettes mes pieds aux ceps, et observes tous mes chemins, et que tu suives les traces de mes pieds?
\VS{28}Car celui que tu poursuis de cette manière s'en va par pièces comme du bois vermoulu et comme une robe que la teigne a rongée. 
\Chap{14}
\VerseOne{}L'homme né de la femme est de courte vie, et rassasié d'agitations\FTNT{Ps. 102:12 ; Ps. 103:15 ; Ps. 144:4 ; Ja. 4:14.}.
\VS{2}Il sort comme une fleur, puis il est coupé, et il s'enfuit comme une ombre qui ne s'arrête pas\FTNT{Es. 40:6 ; Ps. 90:61 ; 1 Pi. 1:24.}.
\VS{3}Cependant tu as ouvert tes yeux sur lui et tu me conduis en justice avec toi.
\VS{4}Qui est-ce qui tirera le pur de l'impur ? Personne\FTNT{Es. 48:8 ; Pr. 22:15.}.
\VS{5}Les jours de l'homme sont déterminés, le nombre de ses mois est entre tes mains; tu lui as prescrit ses limites et il ne passera pas au delà.
\VS{6}Retire-toi de lui, afin qu'il ait du relâche, jusqu'à ce que comme un mercenaire il ait achevé sa journée.
\VS{7}Car si un arbre est coupé, il y a de l'espérance, et il poussera encore, et ne manquera pas de rejetons ; 
\VS{8}quoique sa racine ait vieilli dans la terre, et que son tronc soit mort dans la poussière.
\VS{9}Dès qu'il sent l'eau il regerme, et produit des branches, comme un arbre nouvellement planté. 
\VS{10}Mais l'homme meurt et perd toute sa force; il expire et puis où est-il ?
\VS{11}Comme les eaux s'écoulent de la mer, et une rivière s'assèche et tarit,
\VS{12} ainsi l'homme est couché par terre et ne se relève plus ; jusqu'à ce qu'il n'y ait plus de cieux, il ne se réveillera plus, et ne sera pas réveillé de son sommeil. 
\VS{13}Oh que tu me caches dans le scheol, que tu me gardes à l'abri jusqu'à ce que ta colère soit passée, que tu me donnes un terme, après lequel tu te souviendrais de moi !
\VS{14}Si l'homme meurt, revivra-t-il ? J'attendrai donc tous les jours de mon combat, jusqu'à ce qu'il m'arrive du changement.
\VS{15} Appelle-moi, et je te répondrai ; ne dédaigne point l'ouvrage de tes mains. 
\VS{16}Mais maintenant tu comptes mes pas, et tu veilles sur mon péché\FTNT{Ps. 56:9 ; Ps. 139:2-4 ; Pr. 5:21.} ;
\VS{17}mes péchés sont cachotés comme dans une valise, et tu as cousu ensemble mes iniquités\FTNT{Os. 13:12.}.
\VS{18}Car comme une montagne s'éboule en tombant, et comme un rocher est transporté de sa place ; 
\VS{19} et comme les eaux minent les pierres, et entraînent par leur débordement la poussière de la terre, avec tout ce qu'elle a produit, tu fais ainsi périr l'attente de l'homme mortel.
\VS{20}Tu te montres toujours plus fort que lui , et il s'en va ; tu changes sa face et tu le renvoies au loin.
\VS{21}Que ses enfants soient honorés, il n'en sait rien ; et qu'ils soient abaissés, il ne s'en soucie pas.
\VS{22}Seulement sa chair, pendant qu'elle est sur lui, a de la douleur, et son âme s'afflige tandis qu'elle est en lui.
\Chap{15}
\TextTitle{Deuxième discours d'Eliphaz}
\VerseOne{}Eliphaz de Théman prit la parole et dit :
\VS{2}Un homme sage profère-t-il dans ses réponses une science aussi légère que le vent, des opinions vaines ? Remplit-il son ventre du vent d'orient ?
\VS{3}Contestant avec des discours qui ne servent de rien, et avec des paroles dont on ne peut tirer aucun profit ?
\VS{4}Certainement tu abolis la crainte de Dieu, et tu anéantis peu à peu la prière qu'on doit présenter à Dieu. 
\VS{5} Car ta bouche fait connaître ton iniquité, et tu as choisi un langage trompeur. 
\VS{6}C'est ta bouche qui te condamne, et non pas moi ; et tes lèvres témoignent contre toi. 
\VS{7}Es-tu le premier homme né ? Ou as-tu été formé avant les montagnes\FTNT{Ps. 90:2 ; Pr. 8:25.} ?
\VS{8}As-tu été instruit dans le conseil secret de Dieu, et renfermes-tu seul la sagesse\FTNT{Es. 40:13 ; Jé. 23:18 ; Ro. 11:34.} ?
\VS{9}Que sais-tu que nous ne sachions pas ? Quelle connaissance as-tu que nous n'ayons pas ?
\VS{10}Parmi nous, il y a des hommes à cheveux blancs et des gens d'une fort grande vieillesse, il y en a même de plus âgés que ton père. 
\VS{11}Les consolations du Dieu te semblent-elles trop petites ? As-tu quelque chose de caché par-devant toi ? …
\VS{12}Qu'est-ce qui t'ôte le cœur, et pourquoi clignes-tu les yeux, 
\VS{13}pour que tu pousses ton souffle contre Dieu, et que tu fasses sortir de ta bouche de tels discours ? 
\VS{14}Qu'est-ce que l'homme mortel pour qu'il soit pur, et celui qui est né de femme pour qu'il soit juste\FTNT{Ps. 14:3 ; Pr. 20:9 ; Ec. 7:20.} ?
\VS{15}Voici, Dieu ne se fie pas à ses saints et les cieux ne sont pas purs à ses yeux,
\VS{16}combien plus est abominable et corrompu l'homme qui boit l'iniquité comme de l'eau !  
\VS{17}Je t'enseignerai, écoute-moi et je te raconterai ce que j'ai vu,
\VS{18}savoir ce que les sages ont déclaré, et qu'ils n'ont point caché ; ce qu'ils avaient reçu de leurs pères.
\VS{19}Eux à qui seuls la terre a été donnée, et parmi lesquels l'étranger n'est pas passé.
\VS{20}Le méchant est comme en travail d'enfant tous les jours de sa vie, et un petit nombre d'années sont réservées à l'homme violent\FTNT{Es. 48:22 ; Es. 57:21.}.
\VS{21}Un cri de frayeur est dans ses oreilles ; au milieu de la paix il croit que le destructeur se jette sur lui\FTNT{1 Th. 5:3.} ;
\VS{22}il ne croit pas pouvoir sortir des ténèbres, car il est toujours regardé par l'épée.
\VS{23}Il court après le pain, en disant : Où y en a-t-il ? Il sait que le jour de ténèbres est tout prêt, et il le touche comme avec la main\FTNT{Ps. 109:10.}.
\VS{24}La détresse et l'angoisse l'épouvantent, elles l'assaillent comme un roi équipé pour le combat ;
\VS{25}parce qu'il a élevé sa main contre Dieu, et qu'il s'est raidi contre le Tout-Puissant ;
\VS{26}il lui a sauté au collet et sur l'épaisseur de ses gros boucliers. 
\VS{27}Parce que la graisse a couvert son visage et qu'elle a fait des replis sur son ventre;
\VS{28} il habite des villes détruites, des maisons désertes, tout près de n'être plus que des monceaux de pierres. 
\VS{29}Et il n'en sera pas plus riche, car ses biens ne subsisteront pas, et leur entassement ne se répandra pas sur la terre.
\VS{30}Il ne pourra pas sortir des ténèbres ; la flamme séchera ses branches encore tendres ; il s'en ira par le souffle de la bouche du Tout-Puissant.
\VS{31}Qu'il ne compte pas sur la vanité par laquelle il a été séduit, car la vanité sera sa récompense.
\VS{32}Ce sera fait de lui avant son temps, ses branches ne reverdiront plus. 
\VS{33}On arrachera ses fruits non mûrs comme à une vigne; on jettera sa fleur comme celle d'un olivier. 
\VS{34}Car l'assemblée des hypocrites est stérile, et le feu dévore les tentes de l'homme corrompu.
\VS{35}Ils conçoivent le travail, et ils enfantent la misère, et machinent dans le cœur des fraudes\FTNT{Es. 59:4 ; Os. 10:13.}.
\Chap{16}
\TextTitle{Réponse de Job}
\VerseOne{}Job répondit, et dit :
\VS{2}J'ai souvent entendu de pareils discours ; vous êtes tous des consolateurs fâcheux.
\VS{3}N'y aura-t-il point de fin à des paroles légères comme le vent, et de quoi te fais-tu fort pour répliquer ainsi ?
\VS{4}Parlerais-je comme vous faites, si vous étiez à ma place ; accumulerais-je des paroles contre vous, ou secouerais-je ma tête contre vous ? 
\VS{5}Je vous fortifierais par mes discours, et le mouvement de mes lèvres vous soulagerait.
\VS{6}Si je parle, ma douleur ne sera point soulagée. Si je me tais, en sera-t-elle diminuée?
\VS{7}Certes, il m'a maintenant accablé; tu as dévasté toute ma famille ; 
\VS{8}tu m'as tout couvert de rides, qui sont un témoignage des maux que je souffre; et il s'est élevé en moi une maigreur qui en rend aussi témoignage sur mon visage. 
\VS{9}Sa fureur me déchire, il se déclare mon ennemi, il grince des dents contre moi, et étant devenu mon ennemi, il aiguise ses yeux contre moi.
\VS{10}Ils ouvrent leur bouche contre moi ; ils me donnent des soufflets sur la joue pour m'outrager; ils se réunissent tous ensemble contre moi. 
\VS{11}Dieu m'enferme chez l'injuste, il me fait tomber entre les mains des méchants. 
\VS{12}J'étais en repos, et il m'a écrasé ; il m'a saisi au collet et m'a brisé, et il s'est fait de moi une bute.
\VS{13}Ses archers m'ont environné, il me perce les reins, et ne m'épargne pas ; il répand mon fiel par terre. 
\VS{14}Il m'a brisé en me faisant plaie sur plaie, il a couru sur moi comme un homme puissant.
\VS{15}J'ai cousu un sac sur ma peau; j'ai terni ma gloire dans la poussière\FTNT{Ps. 44 : 25 ; Ps. 119 : 25.}.
\VS{16}Mon visage est couvert de boue à force de pleurer, et une ombre de mort est sur mes paupières.  
\VS{17}Quoiqu'il n'y ait point d'iniquité en mes mains et que ma prière soit pure.
\VS{18}Ô terre ! Ne cache point le sang répandu par moi ; et qu'il n'y ait point de lieu pour mon cri.
\VS{19}Mais maintenant voilà, mon témoin est aux cieux, mon témoin est dans les lieux élevés\FTNT{Ap. 1:5 ; Ap. 3:14.}.
\VS{20}Mes amis sont des moqueurs: mais mon œil fond en larmes devant Dieu.
\VS{21}Ô si l'homme raisonnait avec Dieu comme un homme avec son ami intime ! 
\VS{22}Car les années de mon compte vont finir, et j'entre dans un sentier d'où je ne reviendrai plus. 
\Chap{17}
\VerseOne{}Mon souffle est corrompu, mes jours s'éteignent, le sépulcre m'attend!
\VS{2}Certes, il n'y a que des moqueurs auprès de moi, et mon œil veille toute la nuit dans les chagrins qu'ils me font.
\VS{3}Donne-moi, je te prie, donne-moi une caution auprès de toi; qui donc me frappera dans la main\FTNT{Pr. 6:1; Pr.17:18} ?
\VS{4}Tu caches à leur cœur l'intelligence, c'est pourquoi tu ne les élèveras pas\FTNT{De. 29:4 ; Mt. 11:25.}.
\VS{5}Et les yeux même des enfants de celui qui parle avec flatterie à ses intimes amis, seront consumés.
\VS{6}Il a fait de moi la fable des peuples, et je suis comme un tambour devant eux.
\VS{7}Mon œil est obscurci par le chagrin, et tous les membres de mon corps sont comme une ombre\FTNT{Ps. 6:7 ; Ps. 31:10.}.
\VS{8}Les hommes droits en sont étonnés, et l'innocent s'élève contre l'impie.
\VS{9}Mais le juste tient ferme dans sa voie, et celui qui a les mains pures croît en force. 
\VS{10}Retournez donc vous tous, et revenez, je vous prie ; car je ne trouve point de sage entre vous. 
\VS{11}Mes jours sont passés, mes desseins sont rompus, et les pensées de mon cœur sont dissipées... 
\VS{12}On me change la nuit en jour et on fait que la lumière se trouve proche des ténèbres! 
\VS{13}Certes je n'ai plus à attendre que le scheol, qui va être ma maison ; je dresserai mon lit dans les ténèbres ;
\VS{14}je crie à la fosse : Tu es mon père ; et aux vers : Vous êtes ma mère et ma sœur. 
\VS{15}Et où seront les choses que j'ai attendues, et qui est-ce qui verra ces choses qui ont été le sujet de mon attente ?
\VS{16}Elles descendront au fond du scheol, car elles reposeront ensemble avec moi dans la poussière ! 
\Chap{18}
\TextTitle{Deuxième discours de Bildad}
\VerseOne{}Alors Bildad de Schuach prit la parole et dit :
\VS{2}Quand finirez-vous ces discours ? Ecoutez, et puis nous parlerons :
\VS{3}Pourquoi sommes-nous regardés comme des bêtes, et pourquoi vous tenez-vous pour souillés à vos yeux?
\VS{4}Ô toi qui te déchires toi-même en ta fureur, la terre sera-t-elle abandonnée à cause de toi, et les rochers seront-ils transportés de leur place? 
\VS{5}Certainement, la lumière des méchants sera éteinte, et l'étincelle de leur feu ne reluira plus\FTNT{Ps. 37:9-10.}.
\VS{6}La lumière sera ténèbres dans sa tente, et la lampe qui éclaire au-dessus d'eux sera éteinte.
\VS{7}Les démarches de sa force seront resserrées, et son conseil le renversera.
\VS{8}Car il est enlacé par ses pieds dans les filets, et il marche sur des rets.
\VS{9}Le lacet lui saisit le talon, et le voleur le saisissant en a le dessus. 
\VS{10}Son piège est caché dans la terre, et sa trappe est cachée sur le sentier. 
\VS{11}De tous côtés des terreurs l'assiégeront et feront courir ses pieds çà et là\FTNT{Jé. 6:25 ; Jé. 46:5 ; Jé. 49:29.}.
\VS{12}Sa force sera affamée, la détresse est à ses côtés.
\VS{13}Le premier-né de la mort, dis-je, dévore ce qui soutient sa peau, il dévorera ce qui le soutient. 
\VS{14} Les choses en quoi il mettait sa confiance seront arrachées de sa tente, et il sera conduit vers le roi des épouvantements. 
\VS{15}On habitera dans sa tente, qui ne sera plus à lui; le soufre se répand sur sa demeure. 
\VS{16}Ses racines sécheront au-dessous, et ses branches seront coupées en haut. 
\VS{17}Sa mémoire périra sur la terre, et on ne parlera plus de son nom dans les places\FTNT{Ps. 109:13 ; Pr. 10:7.}.
\VS{18}On le chassera de la lumière dans les ténèbres, et il sera exterminé du monde. 
\VS{19}Il n'aura ni fils ni petit-fils parmi son peuple, et il n'aura personne qui lui survive dans ses demeures\FTNT{Es. 14:20-22 ; Jé. 22:30 ; Ps. 37:28 ; Ps. 109:13.}. 
\VS{20}Ceux qui seront venus après lui seront étonnés de son jour ; et ceux qui auront été avant lui en seront saisis d'horreur. 
\VS{21}Certainement, telles seront les demeures du pervers, et tel sera le lieu de celui qui n'a pas reconnu Dieu.
\Chap{19}
\TextTitle{Réponse de Job}
\VerseOne{}Job prit la parole et dit :
\VS{2}Jusqu'à quand affligerez-vous mon âme et m'accablerez-vous de paroles ? 
\VS{3}Vous avez déjà par dix fois tâché de me couvrir de confusion. N'avez-vous pas honte de vous raidir ainsi contre moi ? 
\VS{4}Mais même s'il est vrai que j'ai péché, la faute serait pour moi. 
\VS{5}Mais si absolument vous voulez parler avec hauteur contre moi, et me reprocher mon opprobre, 
\VS{6}sachez donc que c'est Dieu qui m'a renversé et qui a tendu son filet autour de moi. 
\VS{7}Voici, je crie pour la violence qui m'est faite et je ne suis pas exaucé ; je m'écrie et il n'y a point de justice !
\VS{8}Il a fermé mon chemin tellement que je ne peux passer; et il a mis des ténèbres sur mes sentiers. 
\VS{9}Il m'a dépouillé de ma gloire, il a ôté la couronne de dessus ma tête.
\VS{10}Il m'a détruit de tous côtés, et je m'en vais ; il a fait disparaître mon espérance comme celle d'un arbre que l'on arrache. 
\VS{11}Il s'est enflammé de colère contre moi, et m'a traité comme l'un de ses ennemis\FTNT{La. 2:5.}.
\VS{12}Ses troupes sont venues ensemble, et elles ont dressé leur chemin contre moi, et se sont campées autour de ma tente\FTNT{La. 2:22.}.
\VS{13}Il a fait retirer loin de moi mes frères, et ceux qui me connaissaient se sont fort éloignés de moi comme des étrangers\FTNT{Ps. 88:9.} ;
\VS{14}mes proches m'ont abandonné, et ceux que je connaissais m'ont oublié.
\VS{15}Ceux qui demeurent dans ma maison et mes servantes, me tiennent pour un inconnu, et me traitent comme un étranger. 
\VS{16}J'appelle mon serviteur, mais il ne me répond pas, quoique je l'aie supplié de ma propre bouche.
\VS{17}Mon haleine est devenue dégoûtante à ma femme ; quoique je la supplie par les enfants de mes entrailles.
\VS{18}Je suis méprisé même par des enfants ; si je me lève, ils parlent contre moi.
\VS{19}Tous ceux à qui je déclarais mes secrets, m'ont en abomination ; et tous ceux que j'aimais se sont tournés contre moi\FTNT{Ps. 55:13-14.}.
\VS{20}Mes os sont attachés à ma peau et à ma chair, et il ne me reste d'entier que la peau de mes dents\FTNT{La. 4:8.}.
\VS{21}Ayez pitié de moi, ayez pitié de moi, vous mes amis ; car la main de Dieu m'a frappé.
\VS{22}Pourquoi me poursuivez-vous comme Dieu me poursuit, sans pouvoir vous rassasier de ma chair\FTNT{Ps. 27:2.}?
\VS{23}Plût à Dieu que maintenant mes discours fussent écrits ! Plût à Dieu qu'ils fussent gravés dans un livre, 
\VS{24}avec une touche de fer, et sur du plomb, et qu'ils fussent taillés sur une pierre de roche à perpétuité !
\VS{25}Car je sais que mon Rédempteur est vivant, et qu'il demeurera le dernier sur la terre.
\VS{26}Et lorsqu'après que ma peau aura été rongée, je verrai Dieu de ma chair\FTNT{Ps. 17:15.}.
\VS{27}Je le verrai moi-même, et mes yeux le verront, et non un autre. Mes reins se consument dans mon sein. 
\VS{28}Vous devrez plutôt dire : Pourquoi le persécutons-nous? Puisque le fondement de mes paroles se trouve en moi.
\VS{29}Ayez peur de l'épée ; car la fureur avec laquelle vous me persécutez est du nombre des iniquités qui attirent l'épée ; c'est pourquoi sachez qu'il y a un jugement.
\Chap{20}
\TextTitle{Dernier discours de Tsophar}
\VerseOne{}Tsophar de Naama prit la parole et dit :
\VS{2}C'est à cause de cela que mes pensées diverses me poussent à répondre, et que cette promptitude est en moi. 
\VS{3}J'ai entendu la correction dont tu veux me faire honte, mais mon esprit tirera de mon intelligence la réponse pour moi. 
\VS{4}Ne sais-tu pas que, de tout temps, et depuis que Dieu a mis l'homme sur la terre, 
\VS{5}le triomphe des méchants est de peu de durée, et que la joie de l'hypocrite n'est que pour un moment \FTNT{Ps. 37:35-36.} ?
\VS{6}Quand sa hauteur monterait jusqu'aux cieux, et que sa tête atteindrait les nues,
\VS{7}il périra pour toujours comme ses ordures et ceux qui le voyaient diront : Où est-il ?
\VS{8}Il s'envolera comme un songe, et on ne le trouvera plus ; il s'enfuira comme une vision de nuit\FTNT{Ps. 73:19-20.} ;
\VS{9}l'œil qui le regardait ne le verra plus, le lieu qu'il habitait ne le contemplera plus.
\VS{10}Ses enfants rechercheront la faveur des pauvres, et ses mains restitueront ce que sa violence a ravi\FTNT{Ps. 109:10.}.
\VS{11}Ses os seront pleins de la punition des péchés de sa jeunesse, et elle reposera avec lui dans la poussière.
\VS{12}Si le mal est doux à sa bouche, et s'il le cache sous sa langue,
\VS{13}s'il l'épargne et ne le rejette pas, mais le retient dans son palais, 
\VS{14}ce qu'il mangera se changera dans ses entrailles en un fiel d'aspic.
\VS{15}Il a englouti les richesses, mais il les vomira ; Dieu les jettera hors de son ventre.
\VS{16}Il sucera du venin d'aspic, la langue de la vipère le tuera.
\VS{17}Il ne verra plus les ruisseaux, les fleuves, les torrents de miel et de lait.
\VS{18}Il rendra le fruit de son travail, et ne l’avalera pas ; il le rendra selon sa juste valeur, et ne s'en réjouira pas\FTNT{So. 2:10.}.
\VS{19}Parce qu'il aura foulé les pauvres et les aura abandonnés, il aura ruiné sa maison, bien loin de la bâtir. 
\VS{20}Certainement, il ne sentira pas de contentement en son ventre, et il ne sauvera rien de ce qu'il aura tant convoité\FTNT{Ec. 5:12.}.
\VS{21}Il ne lui restera rien à manger, c'est pourquoi il ne s'attendra plus à son bonheur.
\VS{22}Après que la mesure de ses biens aura été remplie, il sera dans la misère ; toutes les mains de ceux qu'il aura opprimés se jetteront sur lui.
\VS{23}S'il a eu de quoi remplir son ventre, Dieu lui fera sentir l'ardeur de sa colère, et la fera pleuvoir sur lui et sa chair.
\VS{24}S’il s’enfuit de devant les armes de fer, l’arc d’airain le transpercera.
\VS{25}La flèche lancée contre lui sortira au travers de son corps, et le fer étincelant sortira de son fiel; toutes sortes de frayeurs marcheront contre lui.
\VS{26}Toutes les ténèbres seront renfermées dans ses demeures les plus secrètes ; un feu qu'on n'aura point soufflé le consumera ; l'homme qui restera dans sa tente sera malheureux\FTNT{Ps. 12:6.}.
\VS{27}Les cieux découvriront son iniquité et la terre s'élèvera contre lui. 
\VS{28}Le revenu de sa maison sera emporté. Tout s'écoulera au jour de la colère de Dieu.
\VS{29}C'est là la portion que Dieu réserve à l'homme méchant, et l'héritage qu'il aura de Dieu pour ses discours.
\Chap{21}
\TextTitle{Réponse de Job}
\VerseOne{}Mais Job prit la parole et dit: 
\VS{2}Ecoutez attentivement mon discours, et cela me tiendra lieu de consolation de votre part. 
\VS{3}Supportez-moi, et je parlerai, et après que j'aurai parlé, moquez-vous. 
\VS{4}Pour moi, mon discours s'adresse-t-il à un homme ? Si cela était, pourquoi mon esprit ne serait-il pas affligé ?
\VS{5}Regardez-moi, soyez étonnés, et mettez la main sur la bouche.
\VS{6}Quand je pense à mon état, j'en suis tout étonné, et un tremblement saisit ma chair.
\VS{7}Pourquoi les méchants vivent-ils, vieillissent-ils, et même pourquoi sont-ils les plus puissants\FTNT{Jé. 12:1 ; Ha. 1:3 ; Mal. 3:14-15.}?
\VS{8}Leur race se maintient en leur présence avec eux, et leurs rejetons s'élèvent devant leurs yeux.
\VS{9}Leurs maisons jouissent de la paix loin de la frayeur, et la verge de Dieu n'est pas sur eux. 
\VS{10}Leur taureau engendre et n'y manque pas ; leur jeune vache se décharge de son veau et n'avorte pas\FTNT{Ps. 144:13-14.}.
\VS{11}Ils font sortir leurs jeunes enfants comme un troupeau, et leurs enfants s'ébattent. 
\VS{12}Ils sautent au son du tambourin et de la harpe, et se réjouissent au son du chalumeau. 
\VS{13}Ils passent leurs jours dans les plaisirs, et en un moment ils descendent dans le scheol.
\VS{14} Et ils disent à Dieu : Retire-toi de nous ; car nous ne nous soucions pas de la science de tes voies. 
\VS{15}Qui est le Tout-Puissant pour que nous le servions? Et quel bien nous reviendra-t-il de l'invoquer\FTNT{Ex. 5:2.}?
\VS{16}Voilà, leur bonheur n'est pas en leur puissance. Que le conseil des méchants soit loin de moi\FTNT{Ps. 1:1-2.} !
\VS{17}Combien de fois arrive-t-il que la lampe des méchants s'éteigne, et que l'orage vienne sur eux ! Dieu leur distribuera leurs portions dans sa colère\FTNT{Ps. 11:5-6 ; Pr. 13:9.}.
\VS{18}Qu'ils soient comme la paille exposée au vent, et comme la balle qui est enlevée par le tourbillon\FTNT{Ps. 1:4}.
\VS{19}Dieu réserve aux enfants du méchant la punition de ses violences, il la leur rendra, et il le saura !
\VS{20}Que ses yeux voient sa ruine, et qu' il boive le calice de la colère du Tout-Puissant\FTNT{Es. 51:17-22 ; Jé. 25:15 ; Ez. 23:31-32 ; Ap. 14:10.}.
\VS{21}Et quel plaisir aura-t-il en sa maison qu'il laisse après lui, puisque le nombre de ses mois est retranché ? 
\VS{22}Est-ce à Dieu qu'on enseignera la science, à celui qui juge ceux qui sont élevés\FTNT{Ro. 11:34 ; 1 Co. 2:16.} ?
\VS{23}L'un meurt dans toute sa vigueur, tranquille et en repos ; 
\VS{24}ses seaux os sont remplis de lait, et ses sont abreuvés de moelle; 
\VS{25}et l'autre meurt dans l'amertume de son âme, n'ayant jamais goûté le bonheur : 
\VS{26}Et néanmoins ils sont couchés ensemble dans la poussière, et les vers les couvrent. 
\VS{27}Voici, je connais vos pensées, et les jugements que vous formez contre moi. 
\VS{28}Car vous dites : Où est la maison de cet homme si puissant, et où est la tente dans laquelle les méchants demeuraient ?
\VS{29}Ne vous êtes-vous jamais informés des voyageurs, et n'avez-vous pas appris par les rapports qu'ils vous ont faits,
\VS{30}que le méchant est réservé pour le jour de la ruine, pour le jour où les fureurs sont envoyées\FTNT{Pr. 16:4 ; Ec. 9:12.} ?
\VS{31}Mais qui le reprendra en face de sa conduite ? Et qui lui rendra le mal qu'il a fait ?
\VS{32}Il sera néanmoins porté au sépulcre, et il demeurera dans le tombeau.
\VS{33}Les mottes des vallées lui sont agréables ; et tout le monde s'en va à la file après lui, et des gens sans nombre marchent au-devant de lui. 
\VS{34}Comment donc me donnez-vous des consolations vaines, puisqu'il y a toujours de la perfidie dans vos réponses ?
\Chap{22}
\TextTitle{Dernier discours d'Eliphaz}
\VerseOne{}Alors Eliphaz de Théman prit la parole et dit : 
\VS{2}L'homme apportera-t-il quelque profit à Dieu? C'est plutôt à lui-même que l'homme sage apporte du profit. 
\VS{3}Le Tout-Puissant reçoit-il quelque plaisir, si tu es juste ? Ou quelque gain, si tu marches dans l'intégrité ?
\VS{4}Te reprend-il, et entre-t-il en jugement avec toi à cause de la crainte qu'il a de toi ? 
\VS{5}Ta méchanceté n'est-elle pas grande ? Et tes injustices ne sont-elles pas sans fin ? 
\VS{6}Car tu as pris sans raison le gage de tes frères ; tu as ôté la robe à ceux qui étaient nus\FTNT{Ex. 22:21.} ;
\VS{7}tu n'as pas donné de l'eau à boire à celui qui était fatigué ; tu as refusé ton pain à celui qui avait faim.
\VS{8}La terre était à l'homme puissant, et celui qui était respecté y habitait. 
\VS{9}Tu as renvoyé les veuves à vide, et les bras des orphelins ont été cassés.
\VS{10}C'est pour cela que les filets sont tendus autour de toi, et qu'une frayeur subite t'épouvante. 
\VS{11}Et les ténèbres sont autour de toi, tellement que tu ne vois pas ; et le débordement des eaux te couvre.
\VS{12}Dieu n'habite-t-il pas au plus haut des cieux ? Regarde donc la hauteur des étoiles ; et combien elles sont élevées.
\VS{13}Mais tu dis : Qu'est-ce que Dieu connaît ? Jugera-t-il au travers des nuées obscures ?\FTNT{So. 1:12 ; Ps. 10:11-13 ; Ps. 94:7.} 
\VS{14}Les nuées nous cachent à ses yeux, et il ne voit rien, il se promène sur le tour des cieux. 
\VS{15}Eh quoi ! N'as-tu pas pris garde à l'ancienne route dans laquelle ont marché les hommes d'iniquité ?
\VS{16}Et n'as-tu pas pris garde qu'ils ont été retranchés avant le temps ; et que ce sur quoi ils se fondaient s'est écoulé comme un fleuve ?
\VS{17}Ils disaient à Dieu : Retire-toi de nous. Mais qu'est-ce que leur faisait le Tout-Puissant ?
\VS{18}Il avait rempli leur maison de biens. Que le conseil des méchants soit donc loin de moi !
\VS{19}Les justes le verront, se réjouiront, et l'innocent se moquera d'eux\FTNT{Ps. 107:42.}.
\VS{20}Certainement, notre adversaire a péri, le feu a dévoré ce qui en restait\FTNT{Ps. 37:20 ; Ec. 8:12-13.} !
\VS{21}Attache-toi donc à Dieu, je te prie, tu demeureras en repos, tu atteindras ainsi le bonheur.
\VS{22}Reçois, je te prie, la loi de sa bouche, et mets ses paroles en ton coeur\FTNT{Ps. 119:72.}.
\VS{23}Si tu retournes au Tout-Puissant, tu seras rétabli. Chasse l'iniquité loin de ta tente.
\VS{24}Et tu mettras l'or sur la poussière, et l'or d'ophir sur les rochers des torrents ;
\VS{25}et le Tout-Puissant sera ton or, et l'argent de tes richesses.
\VS{26}Car alors tu trouveras tes délices dans le Tout-Puissant, et tu élèveras ton visage vers Dieu; 
\VS{27}tu le fléchiras par tes prières, et il t'exaucera, et tu lui accompliras tes vœux\FTNT{Ps. 50:14-15.}.
\VS{28}Si tu as une résolution, elle tiendra; la lumière brillera sur tes voies\FTNT{Ps. 97:11.}.
\VS{29}Quand on aura abaissé quelqu'un et que tu auras dit qu'il soit élevé; alors Dieu délivrera celui qui tenait les yeux abaissés\FTNT{Pr. 29:23.}.
\VS{30}Il délivrera celui qui n'est pas innocent, et il sera délivré à cause de la pureté de tes mains.
\Chap{23}
\TextTitle{Réponse de Job}
\VerseOne{}Job répondit, et dit :
\VS{2}Encore aujourd'hui ma plainte est pleine d'amertume, et la main qui m'a frappé s'appesantit sur moi au delà de mon gémissement.
\VS{3}Oh ! Si je savais où le trouver, j'irais jusqu'à son trône,
\VS{4}je disposerais en ordre ma cause devant lui, je remplirais ma bouche d'arguments,
\VS{5}je saurais ce qu'il peut avoir à répondre, je comprendrais ce qu'il peut avoir à me dire.
\VS{6}Contesterait-il avec moi par la grandeur de sa force? Non, il mettrait seulement sa force en moi. 
\VS{7}Ce serait un homme juste qui argumenterait avec lui, et je serais pour toujours délivré par mon juge.
\VS{8}Mais si je vais à l'orient, il n'y est pas ; si je vais à l'occident, je ne l'aperçois pas ;
\VS{9}s'il se fait entendre au nord, je ne le vois pas ; se cache-t-il au sud, je ne l'y aperçois pas.
\VS{10}Il connaît la voie que j'ai suivie ; et s'il m'éprouvait, j'en sortirai pur comme l'or\FTNT{1 Pi. 1:7.}.
\VS{11}Mon pied s'est attaché à ses pas ; j'ai pris garde à sa voie, et je ne m'en suis pas détourné.
\VS{12}Je ne me suis pas non plus écarté du commandement de ses lèvres ; j'ai fait plier ma volonté aux paroles de sa bouche.
\VS{13}Mais s'il a une pensée, qui l'en détournera? Et ce que mon âme désire, il le fait\FTNT{Ps. 115:3 ; Ps. 135:6.}.
\VS{14}Il achèvera donc ce qu'il a résolu à mon sujet, et il y a encore en lui de telles choses.
\VS{15}C'est pourquoi je suis terrifié à cause de sa présence, et quand je le considère, je suis effrayé devant lui.
\VS{16}Dieu a brisé mon cœur, le Tout-Puissant m'a épouvanté.
\VS{17}Car ce n'est pas la présence des ténèbres qui m'anéantit, ce n'est pas l'obscurité dont ma face est couverte.
\Chap{24}
\VerseOne{}Comment les temps de la vengeance ne sont-ils pas cachés aux méchants par le Tout-Puissant, puisque ceux-là mêmes qui le connaissent n'aperçoivent pas les jours de sa punition sur eux ?
\VS{2}Ils déplacent les bornes, ils ravissent des troupeaux, et ils les font paître\FTNT{De. 19:14 ; De. 27:17 ; Pr. 13:10 ; Pr. 22:28.} ;
\VS{3}ils emmènent l'âne des orphelins, ils prennent pour gage le boeuf de la veuve.
\VS{4}Ils font retirer les pauvres du chemin, et les misérables du pays se cachent.
\VS{5}Voilà, certains sont comme des ânes sauvages dans le désert ; ils sortent pour faire leur ouvrage, se levant dès le matin pour la proie ; le désert leur fournit du pain pour leurs enfants ;
\VS{6}ils vont couper le fourrage dans les champs, mais ce ne sera que fort tard qu'ils iront ravager la vigne du méchant. 
\VS{7}Ils font passer la nuit sans vêtements à ceux qu'ils ont dépouillés, et qui n'ont pas de quoi se couvrir durant le froid \FTNT{Lé. 19:13 ; De. 24:12-13.} ;
\VS{8}ils sont tout mouillés par les grandes pluies des montagnes, et n'ayant pas de retraite, couchent dans les creux des rochers. 
\VS{9}Ils enlèvent l'orphelin à la mamelle, et prennent des gages du pauvre. 
\VS{10}Ils font aller sans habits l'homme qu'ils ont dépouillé; et ils enlèvent à ceux qui n'avaient pas de quoi manger, ce qu'ils avaient glâné.
\VS{11}Ceux qui font l'huile entre leurs murailles, et ceux qui foulent la vendange dans les cuves souffrent la soif.
\VS{12}Ils font gémir les gens dans la ville l'âme de ceux qu'ils ont fait mourir crie ; Dieu ne fait rien d'indigne de lui.
\VS{13}En voici d'autres qui se révoltent contre la lumière, ils n'en connaissent pas les voies, ils ne restent pas sur leurs sentiers.
\VS{14}Le meurtrier se lève au point du jour, et il tue le pauvre et l'indigent, et la nuit il est tel qu'un larron\FTNT{Ps. 10:8-9.}.
\VS{15}L'oeil de l'adultère épie le soir, en disant : Aucun oeil ne me verra ; et il se couvre le visage\FTNT{Ps. 64:6 ; Pr. 7:7-10.}.
\VS{16}Ils percent durant les ténèbres les maisons qu'ils avaient marquées le jour, ils haïssent la lumière\FTNT{Jn. 3:20.}.
\VS{17}La lumière leur est à tous comme l'ombre de la mort ; si quelqu'un les reconnaît, c'est pour eux une frayeur mortelle.
\VS{18}Il passera plus vite que la surface des eaux ; leur portion sera maudite sur la terre ; il ne verra point le chemin des vignes !
\VS{19}Comme la sécheresse et la chaleur consument les eaux de neige, ainsi le scheol engloutit les pécheurs\FTNT{Ps. 49:15.} !
\VS{20}Le ventre qui l'a porté l'oublie ; les vers mangent son corps qui lui a été si cher ; on ne se souvient plus de lui ; l'injuste est brisé comme du bois.
\VS{21}Il maltraite la femme stérile qui n'enfante pas ; et il ne fait pas de bien à la veuve!
\VS{22}Il attire  les puissants par sa force; lorsqu'il se lève, et on n'est plus sûr de sa vie. 
\VS{23}Dieu lui donne de quoi s'assurer, et il s'appuie sur cela ; toutefois ses yeux prennent garde à leurs voies.
\VS{24}Ils sont élevés en peu de temps, et ensuite ils ne sont plus ; ils sont abaissés, ils sont emportés comme tous les autres, et sont coupés comme le bout d'un épi. 
\VS{25}S'il n'en est pas ainsi, qui me convaincra de mensonge, qui fera de mes paroles un rien ?
\Chap{25}
\TextTitle{Dernier discours de Bildad}
\VerseOne{}Alors Bildad de Schuach prit la parole et dit: 
\VS{2}Le règne et la terreur sont au-devant de Dieu ; il maintient la paix dans ses lieux élevés.
\VS{3} Ses armées peuvent-elles se compter ? Et sur qui sa lumière ne se lève-t-elle point\FTNT{Mt. 5:45.} ?
\VS{4}Et comment l'homme se justifierait-il devant Dieu ? Et comment celui qui est né de femme serait-il pur?
\VS{5}Voilà, qu'on aille jusqu'à la lune, elle ne luit pas ; les étoiles ne sont pas pures devant ses yeux.
\VS{6}Combien moins l'homme qui n'est qu'un ver ; et le fils d'un homme, qui n'est qu'un vermisseau \FTNT{Ps. 22:7.} !
\Chap{26}
\TextTitle{Réponse de Job}
\VerseOne{}Job répondit, et dit :
\VS{2}Ô comme tu as secouru celui qui était sans force, comme tu as soutenu le bras sans force !
\VS{3}Que tu donnes de bons conseils à l'homme qui manque de la sagesse ! Tu fais connaître l'abondance de ton intelligence !
\VS{4}A qui s'adressent tes paroles ? Et de qui est l'esprit qui est sorti de toi ?
\VS{5}Devant Dieu les ombres des morts tremblent au-dessous des eaux, et de leurs habitants ;
\VS{6}devant lui le scheol est nu, l'abîme est sans voile\FTNT{Ps. 139:8-12 ; Pr. 15:11 ; Hé. 4:13.}.
\VS{7}Il étend la direction nord sur le vide, il suspend la terre sur le néant.
\VS{8}Il renferme les eaux dans ses nuages, et  la nuée n'éclate pas sous leur poids\FTNT{Ps. 104:2-3.}.
\VS{9}Il couvre la face de son trône, et il étend sa nuée par dessus.
\VS{10}Il a entouré les eaux avec des bornes jusqu'à ce qu'il n'y ait plus ni lumière ni ténèbres\FTNT{Ge. 1:9 ; Jé. 5:22 ; Ps. 33:7 ; Ps. 104:9 ; Pr. 8:29.}.
\VS{11}Les colonnes du ciel s'ébranlent et s'étonnent à sa menace.
\VS{12}Par sa force il soulève la mer, par son intelligence il en brise l'orgueil\FTNT{Ps. 89:10.}.
\VS{13}Il a orné les cieux par son Esprit, et  de sa main, il a formé le serpent fuyard.
\VS{14}Voilà, tels sont les bords de ses voies ; mais combien est petite la portion que nous en connaissons ? Et qui est ce qui pourra comprendre le bruit éclatant de sa puissance\FTNT{Ec. 3:10.} ?
\Chap{27}
\VerseOne{}Et Job continuant, reprit son discours sentencieux, et dit :
\VS{2}Dieu, qui met mon droit à l'écart, et le Tout-Puissant qui remplit mon âme d'amertume, est vivant.
\VS{3}Aussi longtemps que j'aurai ma respiration et que l'Esprit de Dieu sera dans mes narines,
\VS{4}mes lèvres ne prononceront rien d'injuste, et ma langue ne dira pas de chose fausse\FTNT{Es. 33:15 ; Ps. 15:2 ; Ps. 24:4.}.
\VS{5}Loin de moi la pensée de vous reconnaître pour justes ! Tant que je vivrai je n'abandonnerai pas mon intégrité.
\VS{6}Je conserve ma justice, et je ne l'abandonne pas ; et mon coeur ne me fait de reproche sur aucun de mes jours.
\VS{7}Qu'il en soit de mon ennemi comme du méchant ; et de celui qui se lève contre moi comme de l'injuste !
\VS{8}Quelle attente reste-t-il à l'hypocrite quand Dieu lui retire son âme\FTNT{Mt. 16:26 ; Lu. 12:20.} ?
\VS{9}Est-ce que Dieu entend ses cris quand l'angoisse vient sur lui\FTNT{ Es. 1:15 ; Jé. 14:12 ; Ez. 8:18 ; Mi. 3:4 ; Ps. 18:41 ; Pr. 1:28 ; Jn. 9:31 ; Ja. 4:3.} ?
\VS{10}Trouve-t-il son plaisir dans le Tout-Puissant ? Invoque-t-il Dieu en tout temps ?
\VS{11}Je vous enseignerai comment les voies de Dieu, et je ne vous cacherai pas les projets du Tout-Puissant.
\VS{12}Voilà, vous avez tous vu ces choses, et pourquoi vous laissez-vous aller à des pensées vaines ?
\VS{13}Voici la part que Dieu réserve à l'homme méchant, l'héritage que les violents reçoivent du Tout-Puissant.
\VS{14}Si ses enfants se sont multipliés, ce sera pour l'épée; et sa postérité n'aura pas même assez de pain ;
\VS{15}ses survivants seront bien ensevelis par la peste, et leurs veuves ne les pleureront pas\FTNT{Ps. 78:64.}.
\VS{16}Parce qu’il entasse l'argent comme la poussière, et qu'il entasse des habits comme on amasse de la boue,
\VS{17}le riche les entassera mais le juste s'en vêtira, et l'innocent partagera l'argent.
\VS{18}Il se bâtira une maison comme celle de la teigne, comme la cabane que fait un gardien\FTNT{Ps. 49:18.}.
\VS{19}Le riche tombe, et il ne se relève pas; il ouvre les yeux, et il ne trouve rien.
\VS{20}Les frayeurs l'atteignent comme des eaux ; le tourbillon l'enlève de nuit.
\VS{21}Le vent d'orient l'emporte, et il s'en va ; il l'arrache, dis-je, de sa demeure, comme un tourbillon.
\VS{22} Le Tout-Puissant se jettera sur lui, et ne l'épargnera point ; et étant poursuivi par sa main, il ne cessera de fuir.
\VS{23}On battra des mains contre lui, et on le sifflera au lieu où il se tient.
\Chap{28}
\VerseOne{}Certainement l'argent a sa mine, et l'or a un lieu d'où on le tire pour l'affiner ;
\VS{2}le fer se tire de la poussière, et la pierre étant fondue rend de l'airain.
\VS{3}Il a mis tellement fin aux ténèbres, qu'on découvre la perfection de toutes choses, même les pierres les plus cachées qui sont dans l'ombre de la mort.
\VS{4}Le torrent se déborde d'auprès d'un lieu habité, se jette dans des lieux où l'on ne met plus le pied, mais ses eaux se tarissent et s'écoulent par le travail des hommes.
\VS{5}La terre, d'où sort le pain, est bouleversée dans ses entrailles comme par le feu.
\VS{6}Ses pierres sont la demeure du saphir, et l'on y trouve de la poudre d'or.
\VS{7}L'oiseau de proie n'en connaît pas le chemin, l'œil du vautour ne l'aperçoit pas ;
\VS{8}les plus jeunes et fiers animaux n'y ont pas marché, le lion n'y a jamais passé.
\VS{9}L'homme avance sa main sur le roc, il renverse les montagnes depuis la racine ;
\VS{10}il fend des tranchées dans les rochers, et son œil voit tout ce qu'il y a de précieux ;
\VS{11}il arrête l'écoulement des eaux, et il fait sortir ce qui est caché.
\VS{12}Mais la sagesse, où se trouve-t-elle ? Où est le lieu où se tient l'intelligence ?
\VS{13}L'homme ne connaît pas sa valeur, et elle ne se trouve pas dans la terre des vivants.
\VS{14}L'abîme dit : Elle n'est pas en moi ; et la mer dit : Elle n'est pas avec moi.
\VS{15}Elle ne se donne pas contre de l'or pur, elle ne s'achète pas au poids de l'argent\FTNT{Pr. 3:14 ; Pr. 8:11 ; Pr. 16:16.}.
\VS{16}On ne l'échange point avec l'or d'Ophir, ni avec l'Onyx précieux, ni avec le Saphir.
\VS{17}L'or ni le diamant n'approchent pas de son prix, et on ne la donne pas en échange pour un vase de fin or.
\VS{18}On ne se souvient ni du corail ni du cristal auprès d'elle ; la sagesse vaut plus que les perles.
\VS{19}On ne la compare pas avec la topaze d'Ethiopie ; on ne la met pas en balance avec l'or pur.
\VS{20}D'où vient donc la sagesse ? Où est la demeure de l'intelligence ?
\VS{21}Elle est couverte aux yeux de tout homme vivant, et elle est cachée aux oiseaux des cieux.
\VS{22}Le gouffre et la mort disent : Nous avons entendu de nos oreilles parler d'elle.
\VS{23}C'est Dieu qui en sait le chemin, et qui sait où elle est ;
\VS{24}car il regarde jusqu'aux extrémités de la terre, il voit tout sous les cieux\FTNT{Ps. 14:2 ; Ps. 33:13-14 ; Ps. 102:20.}.
\VS{25}Quand il façonna le poids du vent, et qu'il estima la mesure des eaux\FTNT{Pr. 8:29.},
\VS{26}quand il ordonna des lois à la pluie, et qu'il fit un chemin à l'éclair et au tonnerre,
\VS{27}alors il vit la sagesse et la manifesta ; il l'établit et même il la sonda jusqu'au fond.
\VS{28}Puis il dit à l'homme : Voici, la crainte du Seigneur est la sagesse, et se détourner du mal c'est l'intelligence\FTNT{De. 4:6 ; Jé. 9:24 ; Ps. 111:10 ; Pr. 1:7 ; Pr. 9:10 ; Ec. 12:15.}.
\Chap{29}
\TextTitle{La prospérité passée de Job}
\VerseOne{} Et Job continuant, reprit son discours sentencieux, et dit :
\VS{2}Oh ! Qui me ferait être comme j'étais autrefois, comme j'étais en ces jours où Dieu me gardait,
\VS{3}quand il faisait luire sa lampe sur ma tête, et quand je marchais parmi les ténèbres, éclairé par sa lumière,
\VS{4}comme aux jours de mon automne, lorsque le secret de Dieu était dans ma tente ;
\VS{5}quand le Tout-Puissant était encore avec moi, et mes gens autour de moi,
\VS{6}quand je lavais mes pieds dans le lait, et que le rocher répandait près de moi des torrents d'huile\FTNT{De. 32:13.} !
\VS{7}Quand je sortais vers la porte passant par la ville, et que je me faisais préparer un siège dans la place,
\VS{8}les jeunes gens me voyant se cachaient, les vieillards se levaient, et se tenaient debout.
\VS{9}Les princes s'abstenaient de parler, et mettaient la main sur leur bouche.
\VS{10}Les conducteurs retenaient leur voix, et leur langue était attachée à leur palais.
\VS{11}L'oreille qui m'entendait disait que j'étais bienheureux, et l'oeil qui me voyait déposait en ma faveur ;
\VS{12}car je délivrais l'affligé qui criait au secours, et l'orphelin qui n'avait personne pour le secourir\FTNT{Ps. 72:12 ; Pr. 21:13.}.
\VS{13}La bénédiction de celui qui s'en allait périr venait sur moi, et je faisais que le cœur de la veuve chantait de joie.
\VS{14}J'étais revêtu de la justice, elle me servait de vêtement, et mon équité m'était comme un manteau, et comme une tiare\FTNT{Es. 59:17 ; 1 Th. 5:8 ; Ep. 6:14-17.}.
\VS{15}J'étais les yeux de l'aveugle et les pieds du boiteux.
\VS{16}J'étais le père des pauvres, et je m'informais diligemment de la cause qui ne m'était pas connue\FTNT{Pr. 29:7.}.
\VS{17}Je cassais les grosses dents de l'injuste, et je lui arrachais la proie d'entre ses dents\FTNT{Ps. 58:7.}.
\VS{18}C'est pourquoi je disais : Je mourrai dans mon  lit, et je multiplierai mes jours comme les grains de sable ;
\VS{19}ma racine sera ouverte aux eaux, et la rosée demeurera toute la nuit sur mes branches\FTNT{Jé. 17:5-8 ; Ps. 1:3.}. 
\VS{20}Ma gloire se renouvellera sans cesse en moi, et mon arc se renouvellera dans ma main.
\VS{21}On m'écoutait, et on attendait que j'eusse parlé ; et lorsque j'avais dit mon avis, on se tenait dans le silence.
\VS{22}Après mes discours, nul ne répondait, et ma parole était pour tous une bienfaisante rosée ;
\VS{23}ils m'attendaient comme on attend la pluie ; ils ouvraient leur bouche comme après la pluie de la dernière saison.
\VS{24}Riais-je avec eux ? Ils ne le croyaient point ; et ils ne faisaient point disparaître la sérénité de mon visage.
\VS{25}Je choisissais d'aller avec eux. J'étais assis à leur tête, j'étais parmi eux comme un roi dans son armée, et comme un homme qui console les affligés.
\Chap{30}
\TextTitle{Son humiliation}
\VerseOne{}Mais maintenant, ceux qui sont plus jeunes que moi, se moquent de moi ; ceux-là même dont je n'aurais pas daigné mettre les pères avec les chiens de mon troupeau.
\VS{2} En effet, à quoi m'aurait servi la force de leurs mains ? En eux la vigueur a péri. 
\VS{3}A cause de la disette et de la faim, ils sont stériles et rongent les lieux arides depuis longtemps désolés et déserts. 
\VS{4}Ils coupent des herbes sauvages auprès des arbrisseaux, et la racine des genévriers pour se chauffer. 
\VS{5}Ils sont chassés d'entre les hommes, et on crie après eux comme après un larron. 
\VS{6}Ils habitent dans les creux des torrents, dans les trous de la terre et des rochers.
\VS{7}Ils font du bruit entre les arbrisseaux, et ils s'attroupent entre les chardons. 
\VS{8}Ce sont des hommes de néant et sans nom, abaissés plus bas que la terre. 
\VS{9}Et maintenant je suis le sujet de leur chanson, et la matière de leur entretien\FTNT{Ps. 69:12 ; La. 3:14.}.
\VS{10}Ils m'ont en abomination ; ils se tiennent loin de moi ; et ils ne craignent pas de me cracher au visage. 
\VS{11}Parce que Dieu a détendu ma corde, et m'a affligé, ils secouent le frein devant moi.
\VS{12}De jeunes gens, nouvellement nés, se placent à ma droite ; ils poussent mes pieds, et je suis en butte à leur malice\FTNT{Ps. 35:15.} ;
\VS{13}ils ruinent mon sentier, ils augmentent mon affliction, sans qu'ils aient besoin que personne ne les aide.
\VS{14}Ils viennent contre moi comme par une brèche large, et ils se sont jetés sur moi à cause de ma désolation. 
\VS{15}Les frayeurs se tournent vers moi, et comme un vent elles poursuivent mon âme ; et ma délivrance s'est dissipée comme une nuée\FTNT{Os. 13:3.}.
\VS{16}C'est pourquoi maintenant mon âme se fond en moi ; les jours d'affliction m'ont atteint. 
\VS{17}Il me perce de nuit les os, et mes artères n'ont pas de relâche. 
\VS{18}Il change mon vêtement par la grandeur de sa force, et il me serre de près, comme fait l'ouverture de ma tunique.
\VS{19}Il m'a jeté dans la boue, et je ressemble à la poussière et à la cendre. 
\VS{20}Je crie à toi, et tu ne m'exauces pas ; je me tiens debout, et tu ne me regardes point. 
\VS{21}Tu es pour moi sans compassion, tu me traites en ennemi par la force de ta main. 
\VS{22}Tu m'élèves comme sur le vent, et tu m'y fais monter comme sur un chariot, et puis tu fais fondre toute ma substance. 
\VS{23}Je sais donc que tu me conduis à la mort et dans la maison assignée à tous les vivants\FTNT{Hé. 9:27.}.
\VS{24}Mais il n'étendra pas sa main jusqu'au sépulcre. Quand il les aura tués, crieront-ils ? 
\VS{25}Ne pleurais-je pas pour l'amour de celui qui passait de mauvais jours ; et mon âme n'était-elle pas affligée à cause du pauvre\FTNT{Ro. 12:15.} ?
\VS{26}Cependant lorsque j'attendais le bien, le mal m'est arrivé ; et quand j'espérais la clarté, les ténèbres sont venues. 
\VS{27}Mes entrailles sont dans une grande agitation, et ne peuvent se calmer ; les jours d'affliction m'ont prévenu. 
\VS{28}Je marche tout noirci, mais non pas du soleil ; je me lève, je crie en pleine assemblée. 
\VS{29}Je suis devenu le frère des dragons, et le compagnon des hiboux\FTNT{Ps. 102:7-8.}.
\VS{30}Ma peau est devenue noire sur moi, et mes os sont desséchés par l'ardeur qui me consume\FTNT{La. 4:8 ; La. 5:10.}.
\VS{31}C'est pourquoi ma harpe s'est changée en lamentations, et mes orgues en des sons lugubres.
\Chap{31}
\TextTitle{Job se justifie}
\VerseOne{}J'avais fait accord avec mes yeux ; comment aurais-je donc arrêté mes regards sur une vierge ? 
\VS{2}Quelle part Dieu m'eût-il réservée d'en haut ? Quel héritage le Tout-Puissant m'aurait-il envoyé des cieux ?
\VS{3}La perdition n'est-elle pas pour l'injuste, et les accidents étranges pour les ouvriers d'iniquité ?
\VS{4}Dieu ne voit-il lui-même pas mes voies ? Ne compte-t-il pas tous mes pas\FTNT{Pr. 5:21 ; Pr. 15:3 ; 2 Ch. 16:9.} ?
\VS{5}Si j'ai marché dans le mensonge, si mon pied s'est hâté pour tromper,
\VS{6}qu'on me pèse dans des balances justes, et Dieu connaîtra mon intégrité.
\VS{7}Si mes pas se sont détournés du droit chemin, et si mon cœur a marché après mes yeux, et si quelque tache s'est attachée à mes mains,
\VS{8}que je sème et qu'un autre mange, et que tout ce que j'aurai semé soit déraciné!
\VS{9}Si mon cœur a été séduit après quelque femme, et si j'ai demeuré en embûche à la porte de mon prochain\FTNT{Pr. 7.},
\VS{10}que ma femme soit déshonorée et qu'elle se prostitue à d'autres.
\VS{11}Car c'est un crime, une iniquité punie par les juges ;
\VS{12}c'est un feu qui dévore jusqu'à la destruction, et qui aurait détruit toutes mes récoltes dans leur racine.
\VS{13}Si j'ai refusé de faire droit à mon serviteur ou à ma servante, quand ils ont contesté avec moi ;
\VS{14}car qu'aurais-je fait, quand Dieu se serait levé? Et quand il m'aurait demandé des comptes, que lui aurais-je répondu ?
\VS{15}Celui qui m'a fait dans le ventre de ma mère ne l'a-t-il pas fait aussi ? Un même Dieu ne nous a-t-il pas formés dans le sein maternel\FTNT{Pr. 14:31 ; Pr. 17:5.} ?
\VS{16}Si j'ai refusé aux pauvres leur désir, si j'ai laissé se consumer les yeux de la veuve\FTNT{Es. 10:2 ; Lu. 18:2-3.},
\VS{17}si j'ai mangé seul mon morceau de pain, sans que l'orphelin en ait sa part,
\VS{18}moi qui l'ai dès ma jeunesse fait grandir près de moi comme un père, et qui dès le sein de ma mère, ai été le guide de la veuve ;
\VS{19}si j'ai vu le malheureux périr faute de vêtements, le pauvre manquer de couverture\FTNT{Mt. 25:41-45.},
\VS{20}si ses reins ne m'ont point béni, et s'il n'a pas été échauffé de la laine de mes agneaux ;
\VS{21}si j'ai levé la main contre l'orphelin, quand j'ai vu à la porte que je pouvais l'aider\FTNT{Pr. 22:22.} ;
\VS{22}que l'os de mon épaule tombe et que mon bras soit cassé et séparé de l'os auquel il est joint !
\VS{23}Car j'ai eu la crainte de l'orage de Dieu, et je ne saurais subsister devant sa majesté. 
\VS{24}Si j'ai mis mon espérance en l'or, et si j'ai dit au fin or : Tu es ma confiance\FTNT{Mc. 10:24 ; 1 Ti. 6:17.} ;
\VS{25}si je me suis réjoui de ce que mes biens étaient multiples, et de ce que ma main en avait trouvé abondamment\FTNT{Ps. 62:11.} ;
\VS{26}si j'ai regardé le soleil lorsqu'il brillait le plus, et la lune quand elle marchait noblement ;
\VS{27}et si mon cœur a été séduit en secret, et si ma bouche a embrassé ma main ; 
\VS{28}ce qui est aussi une iniquité toute jugée ; car j'aurais renié le Dieu d'en haut.
\VS{29}Si je me suis réjoui du malheur de mon ennemi, si j'ai sauté d'allégresse quand le mal l'a atteint\FTNT{Mt. 5:43-44.},
\VS{30}moi qui n'ai pas permis à ma langue de pécher en demandant sa mort par des malédictions ;
\VS{31}les gens de ma maison n'ont pas dit : Qui nous donnera de sa chair, nous n'en saurions être rassasiés\FTNT{Ps. 27:2.}?
\VS{32}L'étranger n'a pas passé la nuit dehors ; j'ai ouvert ma porte au passant\FTNT{Ge. 19:1-2 ; De. 10:19 ; 1 Pi. 4:9 ; Hé. 13:2.} ;
\VS{33}si j'ai caché mon péché comme Adam, pour couvrir mon iniquité en me flattant\FTNT{Ge. 3:10-12 ; Pr. 28:13.},
\VS{34}parce que je craignais la grande multitude, et que le mépris des familles m'inspirait de la crainte, je me tenais dans le silence, et ne sortais pas de ma porte.
\VS{35}Oh, s'il y avait quelqu'un qui veuille m'entendre ! Tout mon désir est que le Tout-Puissant me réponde, et que ma partie adverse fasse un écrit contre moi.
\VS{36}Si je ne le porte sur mon épaule, et si je ne l'attache comme une couronne,
\VS{37}je lui raconterais tous mes pas, je m'approcherais de lui comme d'un prince.
\VS{38}Si ma terre crie contre moi, et si ses sillons pleurent;
\VS{39}si j'ai mangé son fruit sans argent ; si j'ai tourmenté l'esprit de ceux qui la possédaient;
\VS{40}qu'elle me produise des épines au lieu de blé, et de l'ivraie au lieu de l'orge. C'est ici la fin des paroles de Job.
\Chap{32}
\TextTitle{Discours d'Elihu : Reproches à Job et à ses amis}
\VerseOne{}Et ces trois hommes cessèrent de répondre à Job, parce qu'il se croyait un homme juste. 
\VS{2}Alors s'enflamma la colère d'Elihu, fils de Barakeël de Buze, de la famille de Ram. Sa colère s'enflamma contre Job, parce qu'il se justifiait lui-même plus qu'il ne justifiait Dieu ; 
\VS{3}et sa colère s'enflamma contre ses trois amis, parce qu'ils ne trouvaient pas de réponse et que néanmoins ils condamnaient Job. 
\VS{4}Mais voyant qu'il n'y avait pas de réponse dans la bouche des trois hommes, la colère d'Elihu s'enflamma.
\VS{5}Mais, voyant que ces trois hommes n'avaient plus aucune réponse à la bouche, Elihu se mit en colère.
\VS{6}Élihu, fils de Barakeël, de Buze, répondit, en disant : Moi, je suis jeune, et vous êtes des vieillards ; c'est pourquoi je redoutais et je craignais de vous faire connaître ce que je sais. 
\VS{7}Je disais: Les jours parleront, et le grand nombre des années fera connaître la sagesse.
\VS{8}En vérité, il y a un esprit dans l'homme, mais c'est l'inspiration du Tout-Puissant qui le rend intelligent\FTNT{Da. 1:17 ; Da. 2:21 ; Pr. 2:6 ; Ec. 2:26.} ;
\VS{9}ce ne sont pas les aînés qui sont sages, ce ne sont pas les vieillards qui comprennent ce qui est juste.
\VS{10}C'est pourquoi je dis : Ecoute-moi ! Et je dirai aussi ma pensée.
\VS{11}J'ai attendu la fin de vos discours, j'ai écouté vos raisonnements jusqu'à ce que vous ayez bien examiné les discours de Job.
\VS{12}J'ai pris le soin de vous écouter ; et voici, aucun de vous n'a convaincu Job, aucun n'a répondu à ses paroles,
\VS{13}afin que vous ne disiez pas: Nous avons trouvé la sagesse ; c'est Dieu qui le poursuit, et non pas l'homme !
\VS{14}Il n'a pas dirigé ses discours contre moi, aussi je ne lui répondrai pas à votre manière.
\VS{15}Ils sont étonnés! Ils ne répondent plus rien! On leur a ôté la parole!
\VS{16}J'ai attendu jusqu'à ce qu'ils ne disent plus rien, car ils sont demeurés muets, et ils n'ont plus su que répondre.
\VS{17}A mon tour, je veux répondre pour moi, et je veux donner mon avis.
\VS{18}Car je suis rempli de discours, l'esprit qui est en mon sein me presse.
\VS{19}Voici, mon ventre est comme un vin qui n'a pas été ouvert ; et il éclate comme des outres neuves\FTNT{Mt. 9:17 ; Mc. 2:22 ; Lu. 5:38.}.
\VS{20}Je parlerai pour respirer à l'aise, j'ouvrirai mes lèvres et je répondrai.
\VS{21}Je ne ferai pas acception de personnes, et je flatterai aucun homme.
\VS{22}Car je ne sais pas flatter; mon Créateur m'enlèverait tout aussitôt.
\Chap{33}
\TextTitle{Discours d'Elihu sur la justice de Dieu}
\VerseOne{}C'est pourquoi Job, écoute mon discours, je te prie, et prête l'oreille à toutes mes paroles !
\VS{2}Voici, j'ouvre la bouche, ma langue parle dans mon palais.
\VS{3}Mes paroles exprimeront la droiture de mon cœur, mes lèvres diront la vérité pure.
\VS{4}L'Esprit de Dieu m'a fait, et le souffle du Tout-Puissant me donne la vie\FTNT{Ge. 2:7.}.
\VS{5}Si tu peux, réponds-moi, dresse-toi contre moi, demeure ferme!
\VS{6}Voici, je suis pour Dieu, selon que tu en as parlé; j'ai été formé de la terre tout comme toi\FTNT{Ac. 14:15.}.
\VS{7}Voici ma terreur ne te troublera pas, et ma main ne s'appesantira pas sur toi.
\VS{8}Quoi qu'il en soit, tu as dit, moi l'entendant, j'ai entendu la voix de tes discours:
\VS{9}Je suis pur, sans péché, je suis net, il n'y a pas d'iniquité en moi.
\VS{10}Voici, il cherche à rompre avec moi, il me considère comme son ennemi ;
\VS{11}il met mes pieds dans les ceps, il surveille tous mes chemins.
\VS{12}Je te répondrai qu'en cela tu n'as pas été juste, car Dieu sera toujours plus grand que l'homme.
\VS{13}Pourquoi as-tu donc plaidé contre lui ? Car il ne rend pas compte de toutes ses actions.
\VS{14}Car Dieu parle une première fois et une seconde fois à celui qui n'aura pas pris garde à la première,
\VS{15}par des songes, par des visions nocturnes, quand les hommes tombent dans un profond sommeil, quand ils dorment sur leur couche.
\VS{16}Alors il ouvre l'oreille aux hommes, et scelle leur châtiment,
\VS{17}afin de détourner l'homme de son œuvre et de le préserver de l'orgueil.
\VS{18}Il garantit son âme de la fosse, et sa vie de l'épée.
\VS{19}L'homme est aussi châtié par des douleurs sur son lit, à cause d'une lutte perpétuelle en ses os\FTNT{Ps. 38:4.}.
\VS{20}Alors sa vie prend en horreur le pain et son âme les mets les plus désirés\FTNT{Ps. 107:18.} ;
\VS{21}sa chair est tellement consumée qu'elle ne paraît plus, ses os qu'on ne voyait pas, sont tellement brisés qu'ils sont mis à nu;
\VS{22}son âme s'approche de la fosse, et sa vie des messagers de la mort.
\VS{23}Mais s'il y a pour cet homme un messager qui interprète, un d'entre les mille, pour lui annoncer la voie de la droiture,
\VS{24}alors Dieu prend pitié de lui et dit : Garantis-le, afin qu'il ne descende pas dans la fosse ; j'ai trouvé la propitiation!
\VS{25}Sa chair devient plus délicate qu'elle n'était dans son enfance; il revient aux jours de sa jeunesse.
\VS{26}Il supplie Dieu par ses prières, et Dieu lui est favorable, il lui laisse voir sa face avec joie, et lui rend sa justice\FTNT{Es. 58:9.}.
\VS{27}Il regarde vers les hommes et dit : J'ai péché, j'ai violé la justice, et je n'ai pas été puni comme je le méritais ;
\VS{28}Dieu a racheté mon âme afin qu'elle ne passe pas dans la fosse, et ma vie voit encore la lumière !
\VS{29}Voilà ce que Dieu fait, deux fois, trois fois, envers l'homme\FTNT{Ps. 62:11.},
\VS{30}pour ramener son âme de la fosse, pour l'éclairer de la lumière des vivants\FTNT{Ps. 56:14.}.
\VS{31}Sois attentif, Job, écoute-moi ! Tais-toi, et je parlerai !
\VS{32}Si tu as quelque chose à dire, réponds-moi ! Parle, car je désire te justifier.
\VS{33}Sinon, écoute-moi, tais-toi et je t'enseignerai la sagesse.
\Chap{34}
\TextTitle{Elihu accuse Job de se révolter}
\VerseOne{}Elihu reprit la parole, et dit :
\VS{2}Sages, écoutez mes discours ! Vous qui avez de l'intelligence, prêtez-moi l'oreille !
\VS{3}Car l'oreille discerne les discours comme le palais savoure ce qu'il mange.
\VS{4}Choisissons ce qui est juste, voyons entre nous ce qui est bon.
\VS{5}Job dit : Je suis juste, et Dieu a écarté ma justice;
\VS{6}mentirai-je à mon droit? Ma flèche est mortelle sans que j'aie commis de crime.
\VS{7}Où y a-t-il un homme comme Job, qui boit la moquerie comme de l'eau,
\VS{8}qui marche en la compagnie des ouvriers d'iniquité, et qui fréquente les méchants ? 
\VS{9}Car il a dit : Il est inutile à l'homme de plaire à Dieu\FTNT{Mal. 3:14.}.
\VS{10} C'est pourquoi écoutez, vous qui avez de l'intelligence, écoutez-moi ! Loin de Dieu la méchanceté, loin du Tout-Puissant l'injustice\FTNT{De. 32:4 ; Ps. 92:16 ; Ro. 9:14.} !
\VS{11}Car il rendra à l'homme selon son œuvre, il fera trouver à chacun selon sa voie\FTNT{Jé. 17:10 ; Jé. 32:19 ; Ez. 7:27 ; Pr. 24:12 ; Mt. 16:27 ; Ro. 2:6 ; 2 Co. 5:10 ; Ep. 6:8 ; Ap. 22:12.}.
\VS{12}Certes, Dieu ne commet pas l'injustice ; le Tout-Puissant ne renverse pas la justice.
\VS{13}Qui lui a donné la terre en charge ? Ou qui a établi la terre entière ?
\VS{14}S'il ne pensait qu'à lui-même, s'il retirait à lui son Esprit et son souffle\FTNT{Ps. 104:29.},
\VS{15}toute chair périrait ensemble, et l'homme retournerait dans la poussière\FTNT{Ge. 3:19 ; Ec. 3:20 ; Ec. 12:9.}.
\VS{16}Si donc tu as de l'intelligence, écoute ceci, prête l'oreille à ce que tu entendras de moi.
\VS{17}Comment celui qui n'aimerait pas à faire la justice jugerait-il le monde? Et condamneras-tu celui qui est souverainement juste ?
\VS{18}Dira-t-on à un roi qu'il est un scélérat ? Et aux princes qu'ils sont des méchants ?
\VS{19}Combien moins le dira-t-on à celui qui n'a point d'égard à la personne des grands, et qui ne connaît point les riches pour les préférer aux pauvres, parce qu'ils sont tous l'ouvrage de ses mains\FTNT{De. 10:17 ; 2 Ch. 19:7 ; Ac. 10:34 ; Ga. 2:6 ; Ro. 2:11 ; Ep. 6:9 ; Col. 3:25.} ?
\VS{20}En un moment, ils mourront ; au milieu de la nuit, un peuple sera ébranlé et passera ; le puissant s'en ira sans la main d'aucun homme.
\VS{21}Car les yeux de Dieu sont sur les voies de l'homme, il regarde tous ses pas.
\VS{22}Il n'y a ni ténèbres ni ombre de la mort où puissent se cacher les ouvriers d'iniquité.
\VS{23}Dieu ne regarde pas à deux fois un homme pour le faire aller en jugement avec lui.
\VS{24}Il brise les hommes puissants par des voies incompréhensibles, et il établit d'autres à leur place ;
\VS{25}car il connaît leurs œuvres. Il les renverse de nuit, et ils sont écrasés ;
\VS{26}il les frappe comme des impies au lieu où se tiennent tous les regards.
\VS{27}Du fait qu'ils se sont détournés de lui, et qu'ils n'ont considéré aucune de ses voies,
\VS{28}ils ont fait monter à Dieu le cri du pauvre, et il a entendu le cri des affligés\FTNT{Ja. 5:4.}.
\VS{29}S'il donne le repos, qui est-ce qui causera du trouble? S'il cache sa face à quelqu'un, qui est-ce qui le regardera, qu'il s'agisse soit de toute une nation soit d'un seul homme?
\VS{30}Afin que l'hypocrite ne règne pas, de peur qu'il soit un piège pour le peuple.
\VS{31}Car a-t-il jamais dit à Dieu : J'ai été pardonné, je ne pécherai plus ;
\VS{32}montre-moi ce que je ne vois pas ; si j'ai fait le mal, je ne le ferai plus ?
\VS{33}Mais Dieu ne te le rendra-t-il pas, puisque tu as rejeté son châtiment, quand tu as fait le choix que tu as fait? Pour moi, je ne sais que dire à cela; mais toi, si tu as quelque chose à répondre, parle.
\VS{34}Les gens de bon sens diront avec moi, et tout homme sage en conviendra,
\VS{35}que Job ne parle pas avec connaissance, et ses paroles manquent d'intelligence.
\VS{36}Ah! Mon père, que Job soit éprouvé jusqu'à ce qu'il soit vaincu, puisqu'il répond comme les impies.
\VS{37}Car il ajoute péché sur péché; il applaudit au milieu de nous ; il parle de plus en plus contre Dieu.
\Chap{35}
\TextTitle{Elihu reproche à Job ses propos irréfléchis}
\VerseOne{}Elihu reprit la parole et dit :
\VS{2}Penses-tu avoir raison de dire : Je suis juste devant Dieu ?
\VS{3}Quand tu dis : Que me sert-il, et que gagnerais-je de plus sans pécher ?
\VS{4}Je te répondrai en ces termes, et à tes amis qui sont avec toi.
\VS{5}Regarde les cieux, et considère-les ! Vois les nuées, elles sont plus hautes que toi !
\VS{6}Si tu pèches, quel mal fais-tu à Dieu ? Et si tes péchés se multipliaient, quel mal recevrait-il ?
\VS{7}Si tu es juste, que lui donnes-tu ? Que reçoit-il de ta main ?
\VS{8}C'est à un homme, comme tu es, que ta méchanceté peut nuire, et c'est au fils d'un homme que ta justice peut  être utile.
\VS{9}On fait crier les opprimés par la grandeur des maux qu'on leur inflige; ils crient à cause de la violence des grands.
\VS{10}Et nul ne dit : Où est le Dieu qui m'a fait, qui donne de quoi chanter pendant la nuit ,
\VS{11}qui nous instruit plus que les animaux de la terre, et plus intelligents que les oiseaux des cieux ?
\VS{12}On crie donc à cause de la fierté des méchants; mais Dieu ne les exauce pas.
\FTNT{Es. 1:15 ; Ez. 8:18 ; Mi. 3:4 ; Jn. 9:31.}.
\VS{13}Cependant, tu ne dois pas dire que c'est en vain ; que Dieu n'écoute point, et que le Tout-Puissant n'y a nul égard.
\VS{14}Quoique tu dises que tu ne le vois pas, le jugement est devant lui ; attends-le donc !
\VS{15}Mais maintenant, si sa colère n'as pas encore été exécutée, ce n'est pas à dire qu'il ne prend pas rigoureusement connaissance de toutes les choses que tu as faites.
\VS{16}Job ouvre donc sa bouche pour se plaindre, il multiplie les paroles sans intelligence.
\Chap{36}
\TextTitle{Discours d'Elihu : Dieu traite les hommes selon leurs oeuvres}
\VerseOne{}Elihu continua de parler, et dit :
\VS{2}Attends-moi un peu, et je te montrerai qu'il y a encore d'autres raisons pour la cause de Dieu.
\VS{3}Je tirerai de loin mes raisons, et je défendrai la justice du Créateur.
\VS{4}Car certainement il n'y aura rien de faux en tout ce que je dirai, et celui qui est avec toi est parfait dans sa connaissance.
\VS{5}Dieu est puissant, mais il ne méprise personne ; il est puissant par la force de son coeur.
\VS{6}Il ne laisse pas vivre le méchant, et il fait droit aux pauvres.
\VS{7}Il ne détourne pas ses yeux de dessus les justes, il les place sur le trône avec les rois, il les y fait asseoir pour toujours, afin qu'ils soient élevés\FTNT{Ps. 33:18 ; Ps. 34:16.}.
\VS{8}S'ils sont liés de chaînes, s'ils sont pris dans les liens de l'affliction,
\VS{9}il leur montre ce qu'ils ont fait, et il leur fait connaître que leurs œuvres se sont augmentées.
\VS{10}Alors il ouvre leur oreille pour leur discipline, il leur dit de se détourner de l'iniquité.
\VS{11}S'ils écoutent, et s'ils le servent, ils achèvent leurs jours dans le bonheur, leurs années dans la joie.
\VS{12}S'ils n'écoutent pas, ils passent par l'épée, ils expirent dans leur aveuglement.
\VS{13}Ceux qui sont hypocrites dans leur cœur, ils ne crient pas à lui quand il les a liés ;
\VS{14}leur personne meurt dans sa jeunesse, leur vie s'éteint parmi les débauchés.
\VS{15}Mais Dieu sauve celui qui est affligé de son oppression, et c'est par la détresse qu'il lui ouvre les oreilles.
\VS{16}Il t'écartera aussi de la détresse, pour te mettre au large, loin de toute angoisse, et ta table sera chargée de viandes grasses\FTNT{Ps. 50:15 ; Ps. 63:6.}.
\VS{17}Or tu remplis le jugement du méchant, mais le jugement et le droit subsisteront.
\VS{18}Certainement Dieu est irrité; prends garde qu'il ne te  plonge dans l'affliction, car il n'y aura pas alors de rançon si grande qui puisse te délivrer\FTNT{Ps. 49:8.} !
\VS{19}Tes cris valent-ils ton or, et même toutes les forces qui se trouvent dans tes richesses ?
\VS{20}Ne soupire pas après la nuit pendant laquelle les peuples s'évanouissent de leur place.
\VS{21}Garde-toi de retourner à l'iniquité, car la souffrance t'y dispose.
\VS{22}Dieu est élevé par sa puissance ; qui saurait enseigner comme lui ?
\VS{23}Qui lui a prescrit le chemin qu'il devait tenir? Qui lui a dit : Tu as fait une injustice ?
\VS{24}Souviens-toi de célébrer ses ouvrages que tous les hommes voient.
\VS{25}Tout homme les voit, chacun les contemple de loin.
\VS{26}Dieu est grand, mais nous ne le connaissons pas; quant au nombre de ses années, il est insondable\FTNT{Es. 63:16 ; Ps. 92:8 ; Ps. 93:2 ; Ps. 102:13 ; La. 5:19.}.
\VS{27}Parce qu'il met les eaux en petites gouttes, elles répandent la pluie selon la vapeur d'eau qui la contient ;
\VS{28}les nuées la font dégoutter, elles coulent sur les hommes en abondance.
\VS{29}Et qui pourra comprendre l'étendue des nuages et le son éclatant de sa tente ?
\VS{30}Voici, il étend sa lumière sur elle, et il se cache jusque dans les profondeurs de la mer.
\VS{31} Or c'est par ces choses qu'il juge les peuples, qu' il donne la nourriture en abondance.
\VS{32} Il tient caché dans les paumes de ses mains le feu étincelant, et lui ordonne de frapper ce qui se présente à sa rencontre.
\VS{33} Son bruit l'annonce, et le bétail ressent la vapeur qui monte.
\Chap{37}
\TextTitle{Conclusion d'Elihu}
\VerseOne{}Mon cœur même à cause de cela est tout tremblant, il sort de sa place.
\VS{2}Ecoutez attentivement et en tremblant le bruit de sa voix, le grondement qui sort de sa bouche\FTNT{Ps. 29:3-9.} !
\VS{3}Il le conduit dans toute l'étendue des cieux, et son éclair brille jusqu'aux extrémités de la terre\FTNT{Ps. 97:4.}.
\VS{4}Après lui s'élève un grand bruit, il tonne de sa voix majestueuse; et il ne tarde pas après que sa voix a été entendue\FTNT{Jé. 10:13.}.
\VS{5}Dieu tonne avec sa voix d'une manière étonnante ; il fait de grandes choses que nous ne comprenons pas.
\VS{6}Car il dit à la neige : Tombe sur la terre ! Il le dit à la pluie, même aux plus fortes pluies.
\VS{7}Il met un sceau sur la main de tous les hommes, afin que tous les hommes connaissent son œuvre.
\VS{8}Les bêtes entrent dans leurs tanières, et elles demeurent dans leurs repaires.
\VS{9}L'ouragan vient du fond du midi, et le froid vient des vents du nord.
\VS{10}Par son souffle, Dieu donne la glace, et il réduit l'espace où se répandaient au large les eaux\FTNT{Ps. 147:17-18.}.
\VS{11}Il lasse les nuages à force d'arroser, il écarte les nuages par sa lumière.
\VS{12}Et ceux-ci font plusieurs tours pour faire ce qu'il a commandé, sur la face de la terre, sur la face de la terre habitée.
\VS{13}Il les fait venir pour s'en servir soit comme une verge pour la terre, soit pour répandre ses bienfaits\FTNT{Ex. 9:18-23 ; 1 S. 12:18-19.}.
\VS{14}Job, arrête-toi, prête l'oreille à ces choses ! Considère encore les merveilles de Dieu !
\VS{15}Sais-tu comment Dieu les dispose, et fait briller la lumière de ses nuages?
\VS{16}Connais-tu le balancement des nuages, les merveilles de celui dont la science est parfaite ?
\VS{17}Sais-tu pourquoi tes vêtements sont chauds quand la terre se repose par le vent du midi ?
\VS{18}Peux-tu étendre avec lui les cieux, aussi fermes qu'un miroir de fonte ?
\VS{19}Montre-nous ce que nous lui dirons ; car nous ne saurions rien dire par ordre à cause de nos ténèbres. 
\VS{20}Lui racontera-t-on quand je parlerai ? S'il y a un homme qui en parle, certainement il en sera englouti ?
\VS{21}Et maintenant, on ne voit pas la lumière, quand elle resplendit dans les cieux ; après que le vent y a passé, et qu'il les a nettoyés.
\VS{22}Le temps qui la reluit comme l'or vient du nord. Il y a en Dieu une majesté redoutable.
\VS{23}Nous ne saurions comprendre le Tout-Puissant, grand en puissance, en jugement et en abondante justice, il n'opprime personne !
\VS{24}C'est pourquoi les hommes le craignent ; mais il ne les voit pas tous sages de cœur\FTNT{Ps. 92:7 ; Ro. 1:21.}.
\Chap{38}
\TextTitle{Yahweh interroge Job}
\VerseOne{}Yahweh répondit à Job du milieu du tourbillon et dit :
\VS{2}Qui est celui qui obscurcit mes décisions par des paroles sans connaissance ?
\VS{3}Ceins maintenant tes reins comme un vaillant homme ; je t'interrogerai, et tu me feras voir ta science.
\VS{4}Où étais-tu quand je fondais la terre ? Dis-le, si tu as de l'intelligence\FTNT{Pr. 8:29.}.
\VS{5}Qui en a réglé les mesures, le sais-tu ? Ou qui a appliqué sur elle le niveau ?
\VS{6}Sur quoi ses bases sont-elles plantées ? Ou qui en a posé la pierre angulaire pour la soutenir\FTNT{Ps. 104:5.}?
\VS{7}Quand les étoiles du matin se réjouissaient ensemble, et que tous les fils de Dieu poussaient des cris de joie \FTNT{Ps. 148:3.} ?
\VS{8}Qui a renfermé la mer dans ses bords, quand elle fut tirée de la matrice et qu'elle sortit? 
\VS{9}Quand je lui donnai la nuée pour vêtement, et l'obscurité pour langes ;
\VS{10}que je lui imposai ma loi, et que je lui mis des barrières et des portes;
\VS{11} et quand je dis : Tu viendras jusqu'ici, tu n'iras pas plus loin ; ici s'arrêtera l'orgueil de tes flots ?
\VS{12}Depuis que tu es au monde, as-tu commandé au matin et as-tu montré à l'aube du jour le lieu où elle doit se lever,
\VS{13}pour qu'elle saisisse les extrémités de la terre, et que les méchants en soient chassés ;
\VS{14}pour que la terre prenne une forme comme l'argile qui reçoit un sceau, et qu'elle soit parée comme d'un vêtement nouveau ;
\VS{15}pour que la lumière soit ôtée aux méchants, et que le bras qui se lève soit brisé\FTNT{Ps. 10:15.} ?
\VS{16}As-tu pénétré jusqu'aux sources de la mer ? T'es-tu promené dans les profondeurs de l'abîme ?
\VS{17}Les portes de la mort se sont-elles découvertes à toi ? As-tu vu les portes de l'ombre de la mort ?
\VS{18}As-tu compris l'étendue de la terre ? Si tu sais tout cela, dis-le !
\VS{19}Où est la demeure de la lumière, et où est le lieu des ténèbres,
\VS{20}pour que tu les prennes à leur limite, et que tu connaisses le chemin de leur maison ?
\VS{21}Tu le sais, car alors tu étais né, et le nombre de tes jours est grand !
\VS{22}Es-tu entré dans les trésors de la neige ? As-tu vu les trésors de grêle,
\VS{23}que je réserve pour les temps de détresse, pour les jours de guerre et de bataille\FTNT{Ex. 9:23 ; Jos. 10:11 ; Ap. 8 :7.} ?
\VS{24}Par quel chemin la lumière se divise-t-elle, et le vent d'orient se répand-il sur la terre\FTNT{Jn. 3:8.} ?
\VS{25}Qui a ouvert un conduit aux inondations, et tracé la route de l'éclair et du tonnerre,
\VS{26}pour qu'elle pleuve sur une terre sans habitants, sur un désert sans hommes\FTNT{Ps. 104:13-14 ; PS. 147:8 ; Ac. 14:17.} ;
\VS{27}pour qu'elle abreuve les lieux solitaires et arides, et qu'elle fasse germer et sortir l'herbe ?
\VS{28}La pluie a-t-elle un père qui enfante les gouttes de la rosée ?
\VS{29}De quel sein est sortie la glace ? Et qui enfante le givre du ciel,
\VS{30}pour que les eaux se cachent comme une pierre, et que le dessus de l'abîme soit enchaîné ?
\VS{31}Peux-tu resserer les liens des pléiades ou détacher les chaînes d'orient\FTNT{Am. 5:8.}?
\VS{32}Fais-tu sortir en leur temps les signes du zodiaque, et conduis-tu la Grande Ourse avec ses petits ?
\VS{33}Connais-tu les lois du ciel ? Disposes-tu de son pouvoir sur la terre\FTNT{Jé. 31:35-36 ; Ps. 104:4.} ?
\VS{34}Elèves-tu la voix jusqu'aux nuées, pour que des eaux abondantes te couvrent ?
\VS{35}Envoies-tu les éclairs ? Partent-ils ? Te disent-ils : Nous voici ?
\VS{36}Qui a mis la sagesse dans le cœur, ou qui a donné l'intelligence à l'esprit\FTNT{Ec. 2:26.} ?
\VS{37}Qui est-ce qui peut avec intelligence compter les nuages, et placer les outres des cieux,
\VS{38}quand la poussière est détrompée par les eaux qui l'arrosent, et que les mottes viennent à se joindre ?
\Chap{39}
\TextTitle{L'omnipotence de Yahweh}
\VerseOne{}Chasses-tu de la proie pour la lionne, et apaises-tu la faim des lionceaux\FTNT{Ps. 104:21.},
\VS{2}quand ils se tapissent dans leurs tanières et se tiennent aux aguets dans leur repaire ?
\VS{3}Qui est-ce qui apprête la nourriture au corbeau, quand ses petits crient à Dieu, et qu'ils vont errants, parce qu'ils n'ont point de quoi manger\FTNT{Ps. 104:27 ; Ps. 147:9 ; Mt. 6:26.} ?
\VS{4}Sais-tu quand les boucs de rochers mettent bas ? Observes-tu les biches de rochers quand elles font leurs petits\FTNT{Ps. 29:9.} ?
\VS{5}Comptes-tu les mois de leur gestation, et sais-tu le temps auquel elles feront leurs petits,
\VS{6}et qu'elles se courberont pour mettre bas leurs petits et se délivreront de leurs douleurs ?
\VS{7}Leurs petits se fortifient, ils croissent en plein air, ils s’en vont et ne reviennent plus vers elles.
\VS{8}Qui a laissé aller libre l’âne sauvage ? Et qui a délié les liens de l’âne farouche,
\VS{9}auquel j’ai donné le désert pour maison, et la terre inhabitée pour ses retraites\FTNT{Jé. 2:24.} ?
\VS{10}Il se rit du bruit des villes, il n'entend pas les cris d'un exacteur.
\VS{11}Les montagnes qu'il va épiant çà et là, sont ses pâturages, et il cherche toute sorte de verdure. 
\VS{12}Le buffle voudra-t-il te servir, ou demeurera-t-il à ta crèche ? 
\VS{13}Lies-tu le buffle avec une corde pour labourer ? Ou rompra-t-il les mottes des vallées après toi ? 
\VS{14}Te fies-tu à lui parce que sa force est grande, et lui abandonnes-tu ton travail? 
\VS{15}Comptes-tu sur lui pour rentrer ta semence, et pour l'amasser sur ton aire ? 
\VS{16}As-tu donné aux paons ce plumage qui est si brillant, ou à l'autruche les ailes et les plumes ? 
\VS{17}Néanmoins elle abandonne ses oeufs à terre, et les fait échauffer sur la poussière ;
\VS{18}et elle oublie que le pied peut les écraser, ou que les bêtes des champs peuvent les fouler. 
\VS{19}Elle est dure envers ses petits, comme s’ils n’étaient pas siens. Son travail est vain, elle ne s’en inquiète pas.
\VS{20}Car Dieu l'a privée de sagesse et ne lui a pas donné l'intelligence.
\VS{21}A la première occasion, elle se dresse en haut, et se moque du cheval et de celui qui le monte. 
\VS{22}As-tu donné la force au cheval, et as-tu revêtu son cou d'un hennissement éclatant comme le tonnerre ? 
\VS{23}Fais-tu bondir le cheval comme la sauterelle ? Le son magnifique de ses narines est effrayant.
\VS{24}Il creuse la terre de son pied, il s'égaie dans sa force, il va à la rencontre d'un homme armé.
\VS{25}Il se rit de la frayeur, il ne s'épouvante de rien, et il ne se détourne point de devant l'épée.
\VS{26}Il n'a point peur des flèches qui sifflent tout autour de lui, ni du fer luisant de la lance et du javelot. 
\VS{27}Il creuse la terre, plein d'émotion et d'ardeur au son de la trompette, et il ne peut se retenir. 
\VS{28}Au son bruyant de la trompette, il dit : En avant ! En avant ! Il flaire de loin la bataille, le tonnerre des capitaines, et le cri de triomphe.
\VS{29}Est-ce par ta sagesse que l'épervier prend son vol, et qu'il étend ses ailes vers le midi ?
\VS{30}Est-ce par ton commandement que l'aigle s'élève, et qu'il place son nid sur les hauteurs\FTNT{Jé. 49:16 ; Abd. 1:4.} ?
\VS{31}Il habite sur les rochers, et il s'y tient ; même sur les sommets des rochers et dans des lieux forts. 
\VS{32}De là, il découvre le gibier, ses yeux voient de loin.
\VS{33}Ses petits aussi sucent le sang ; et là où sont des cadavres, il s'y trouve aussitôt\FTNT{Mt. 24:28 ; Lu. 17:37.}.
\TextTitle{Yahweh interroge Job}
\VS{34}Yahweh prit encore la parole et dit à Job :
\VS{35}Celui qui conteste avec le Tout-Puissant, lui apprendra-t-il quelque chose ? Que celui qui dispute avec Dieu réponde à ceci.
\TextTitle{Réponse de Job}
\VS{36}Alors Job répondit à Yahweh et dit :
\VS{37}Voici, je suis un homme vil ; que te répondrais-je ? Je mets ma main sur ma bouche\FTNT{Ps. 39:10.}.
\VS{38}J'ai parlé une fois, mais je ne répondrai plus ; j'ai même parlé deux fois et je n'ajouterai rien.
\Chap{40}
\TextTitle{Yahweh questionne encore Job}
\VerseOne{}Et Yahweh répondit à Job du milieu d'un tourbillon, et lui dit :
\VS{2}Ceins maintenant tes reins comme un vaillant homme ; je t'interrogerai et tu m'enseigneras.
\VS{3}Anéantiras-tu mon jugement ? me condamneras-tu pour te justifier\FTNT{Ps. 51:6 ; Ro. 3:4.} ?
\VS{4}As-tu un bras comme celui de Dieu ; tonnes-tu de la voix comme lui ?
\VS{5}Pare-toi maintenant de magnificence et de grandeur, et revêts-toi de majesté et de gloire.
\VS{6}Répands les fureurs de ta colère, d'un regard, humilie tous les orgueilleux.
\VS{7}Regarde tout orgueilleux, abaisse-le, et écrase les méchants sur la place,
\VS{8}cache-les tous ensemble dans la poussière, et bande leur visage dans un lieu caché. 
\VS{9}Alors je te donnerai moi-même cette louange, que ta droite t'aura sauvé. 
\VS{10}Voici le Béhémoth\FTNT{ Le sens exact de ce mot est inconnu. Certains pensent qu’il s’agit peut-être d’un dinosaure disparu. La traduction par hippopotame ne semble pas correcte}, que j'ai façonné comme toi ! Il mange de l'herbe comme le bœuf.
\VS{11}Regarde donc, sa force est dans ses reins, et sa puissance dans les muscles de son ventre ;
\VS{12}il plie sa queue aussi ferme qu'un cèdre ; les tendons de ses cuisses sont entrelacés ;
\VS{13}ses os sont des tubes d'airain, ses membres sont comme des barres de fer.
\VS{14}C’est le chef-d’œuvre de Dieu ; celui qui l’a fait lui a donné son épée.
\VS{15}Car les montagnes lui apportent sa pâture, là où se jouent toutes les bêtes des champs.
\VS{16}Il se couche sous les lotus, caché dans les roseaux et les marécages ;
\VS{17}les lotus le couvrent de leur ombre, les saules du torrent l'enveloppent.
\VS{18}Voilà, il engloutit une rivière en buvant, et il ne s'en retire pas vite ; et il ne s'étonnerait pas quand le Jourdain se dégorgerait dans sa gueule. 
\VS{19}Il l'engloutit en le voyant, et son nez passe au travers des empêchements qu'il rencontre. 
\TextTitle{L'interrogatoire continue}
\VS{20}Attireras-tu le léviathan à l'hameçon ? Saisiras-tu sa langue avec une corde ?
\VS{21}Mettras-tu un jonc dans ses narines ? Lui perceras-tu la mâchoire avec un crochet ?
\VS{22}Accumulera-t-il les supplications ? Te parlera-t-il d'une voix douce ?
\VS{23}Fera-t-il un accord avec toi, et le prendras-tu pour esclave à toujours ? 
\VS{24}Joueras-tu avec lui comme avec un oiseau ? L'attacheras-tu pour amuser les jeunes filles ?
\VS{25}Les pêcheurs en trafiquent-ils ? Le partagent-ils entre les marchands ?
\VS{26}Couvriras-tu sa peau de dards, et sa tête de harpons ?
\VS{27}Mets ta main contre lui, et tu ne te souviendras plus de l'attaquer.
\VS{28}Voici, on est trompé dans son attente ; à sa vue n'est-on pas terrassé ?
\Chap{41}
\VerseOne{}Il n'y a point d'homme assez courageux pour le réveiller ; et qui est capable de se tenir debout devant moi ?
\VS{2}Qui m’a donné le premier ? Je lui rendrai, tout ce qui est sous tous les cieux est à moi\FTNT{Ex. 19:5 ; De. 10:14 ; Ps. 24:1 ; Ps. 50:12 ; 1 Co. 10:26 ; Ro. 11:35.}.
\VS{3}Je ne me tairai pas sur ses membres, sur sa force et sur la beauté de sa stature.
\VS{4}Qui découvrira son vêtement devant ma face ? Qui viendra freiner ses mâchoires par un mors ?
\VS{5}Qui ouvrira les portes devant sa face ? Autour du lion habite la terreur.
\VS{6}Les rangées de ses boucliers ne sont que magnificence ; elles sont étroitement serrées avec un sceau.
\VS{7}L'une se rapproche de l'autre, et le vent n'entre point entre-deux. 
\VS{8}Elles sont jointes l'une à l'autre, elles s'entretiennent, et ne se séparent point. 
\VS{9}Ses éternuements éclaireraient la lumière, et ses yeux sont comme les paupières de l'aube du jour. 
\VS{10}Des lampes de feu sortent de sa bouche, et il en rejaillit des étincelles de feu. 
\VS{11}Une fumée sort de ses narines comme d'un pot bouillant, ou d'une chaudière. 
\VS{12}Son souffle enflammerait des charbons, et une flamme sort de sa gueule. 
\VS{13}La force est dans son cou, et la terreur marche devant lui. 
\VS{14}Sa chair est ferme, tout est massif en lui, rien n'y branle. 
\VS{15}Son coeur est dur comme une pierre, même comme une pièce de la meule inférieure.
\VS{16}Les plus forts tremblent quand il s'élève, et ils ne savent où ils en sont, voyant comme il rompt tout. 
\VS{17}Qui s'en approchera avec l'épée ? Ni elle, ni la lance, ni le dard, ni la cuirasse, ne pourront subsister devant lui.
\VS{18}Il ne tient pas plus de compte du fer que de la paille ; et de l'airain, que du bois pourri. 
\VS{19}La flèche ne le fera point fuir, les pierres d'une fronde sont pour lui comme du chaume.
\VS{20}Il tient les machines de guerre comme des brins de chaume ; et il se moque du javelot qu'on lance sur lui. 
\VS{21}Il a sous son ventre des pointes aiguës : On dirait une herse qu'il étend sur la boue.
\VS{22}Il fait bouillonner le gouffre comme une chaudière, et rend semblable la mer à un chaudron de parfumeur. 
\VS{23}Il fait briller son sentier après lui, et on prendrait l'abîme pour une tête blanchie de vieillesse.
\VS{24}Il n'y a rien sur la terre qui puisse lui être comparé, il a été fait pour ne rien craindre.
\VS{25}Il voit au-dessous de lui tout ce qu'il y a de plus élevé ; il est roi de tous les plus fiers animaux.
\Chap{42}
\TextTitle{Job reconnaît la toute-puissance de Dieu et s'humilie}
\VerseOne{}Alors Job répondit à Yahweh et dit :
\VS{2}Je sais que tu peux tout, et qu'on ne saurait t'empêcher de faire ce que tu penses.
\VS{3}Qui est celui-ci, as-tu dit, qui étant sans science, entreprend d'obscurcir mes desseins ? J'ai donc parlé, et je n'y comprends rien ; ces choses sont trop merveilleuses pour moi, et je n'y connais rien\FTNT{Ps. 40:6 ; Ps. 131:1 ; Ps. 139:6.}.
\VS{4}Ecoute-moi maintenant, et je parlerai ; je t'interrogerai et tu m'instruiras.
\VS{5}J'avais entendu de mes oreilles parler de toi ; mais maintenant mon œil t'a vu.
\VS{6}C'est pourquoi j'ai horreur d'avoir parlé ainsi, et je m'en repens sur la poussière et sur la cendre.
\TextTitle{Le rétablissement de Job}
\VS{7}Or après que Yahweh eut ainsi parlé à Job, il dit à Eliphaz de Théman : Ma fureur est embrasée contre toi et contre tes deux amis, parce que vous n'avez pas parlé de moi avec droiture comme Job, mon serviteur.
\VS{8}C'est pourquoi prenez maintenant sept taureaux et sept béliers, allez auprès de mon serviteur Job, et offrez un holocauste pour vous. Job, mon serviteur, priera pour vous, et certainement j'exaucerai sa prière, afin que je ne vous traite pas selon votre folie ; car vous n'avez pas parlé de moi avec droiture, comme mon serviteur Job.
\VS{9} Ainsi, Eliphaz de Théman, Bildad de Schuach, et Tsophar de Naama vinrent et firent comme Yahweh leur avait commandé ; et Yahweh exauça la prière de Job.
\VS{10}Et Yahweh tira Job de sa captivité, quand il eut prié pour ses amis ; et Yahweh lui ajouta le double de tout ce qu'il avait possédé.
\VS{11}Aussi tous ses frères, toutes ses sœurs et tous ceux qui l'avaient connu auparavant, vinrent vers lui, et mangèrent avec lui dans sa maison. Et lui ayant témoigné qu'ils compatissaient à son état, ils le consolèrent de tout le mal, que Yahweh avait fait venir sur lui ; et chacun d'eux lui donna une pièce d'argent, et chacun un anneau d'or.
\VS{12}Ainsi Yahweh bénit le dernier état de Job plus que le premier, de sorte qu'il eut quatorze mille brebis, et six mille chameaux, et mille couples de bœuf et mille ânesses.
\VS{13}Il eut aussi sept fils et trois filles :
\VS{14}Il donna à la première le nom de Jemima, à la seconde celui de Ketsia, à la troisième celui de Kéren-Happuc.
\VS{15}Et il ne se trouvait pas de femmes aussi belles que les filles de Job dans tout le pays. Leur père leur donna une part de l'héritage parmi leurs frères.
\VS{16}Puis Job vécut, après ces choses, cent quarante ans, et il vit ses fils et les fils de ses fils jusqu'à la quatrième génération.
\VS{17}Puis Job mourut âgé et rassasié de jours.
\PPE{}
\end{multicols}
