\ShortTitle{Job}\BookTitle{Job}\BFont
\noindent\hrulefill
{\footnotesize
\textit{
\bigskip
{\centering{}
\\Auteur : Inconnu
\\(Heb. : Iyov)
\\Signification : haï, ennemi et « je m'exclamerai »
\\Thème : La souffrance
\\Date de rédaction : Incertaine\\}
}
%\bigskip
\textit{
\\Job était un homme prospère et intègre auquel Dieu rendit témoignage. Il subit une succession de malheurs en très peu de temps en perdant tout ce qui lui était cher. Après avoir cherché à se justifier et subi les railleries de sa femme et les accusations de ses amis, Job s'humilia devant Dieu et comprit l'impuissance de sa propre justice. Cette histoire, dont on n'a aucune indication spatio-temporelle et qui pourtant parle à tous, est un encouragement pour le juste éprouvé.
%\bigskip
\\Rappelant que la souffrance peut être le moyen choisi par Dieu pour enseigner et se révéler, ce récit illustre la fidélité et la bonté de Yahweh envers ceux qui le craignent.\bigskip
}
}
\par\nobreak\noindent\hrulefill
\begin{multicols}{2}
\Chap{1}
\TextTitle{Job et sa famille}
\VerseOne{}Il y avait dans le pays d'Uts\FTNT{Ge. 36:28} un homme dont le nom était Job\FTNT{Ez. 14:14; Ja. 5:11.}. Cet homme était intègre\FTNT{1 R. 8:61.} et droit, craignant\FTNT{Ps. 19:10; Pr. 1:7.} Dieu et se détournant du mal.
\VS{2}Il eut sept fils et trois filles.
\VS{3}Et son bétail était de sept mille brebis, trois mille chameaux, cinq cents paires de bœufs, cinq cents ânesses, avec un très grand nombre de serviteurs\FTNT{Job 42:12-13.}; tellement que cet homme était le plus puissant de tous les Orientaux.
\VS{4}Or ses fils allaient et faisaient des festins les uns chez les autres chacun à son jour, et ils envoyaient appeler leurs trois sœurs pour manger et boire avec eux.
\VS{5}Quand les jours de festin étaient passés, Job envoyait chercher ses fils pour les sanctifier, et se levant de bon matin, il offrait un holocauste selon le nombre de ses enfants ; car Job disait : Peut-être mes fils ont-ils péché, et ont-ils blasphémé contre Dieu dans leurs cœurs. Job faisait toujours ainsi.\FTNT{Job 42:8.}
\VS{6}Or, il arriva un jour que les fils de Dieu\FTNT{Ps. 89:7 ; Job 38:7.} vinrent se présenter devant Yahweh, et Satan\FTNT{Es. 14:12; Ap. 12:9-10.} aussi vint au milieu d'eux.
\VS{7}Yahweh dit à Satan : D'où viens-tu ? Et Satan répondit à Yahweh : De courir çà et là sur la terre et de m'y promener\FTNT{1 Pi. 5:8.}.
\VS{8}Yahweh dit à Satan : N'as-tu point considéré mon serviteur Job, qui n'a point d'égal sur la terre ; homme intègre et droit, craignant Dieu, et se détournant du mal ?
\VS{9}Et Satan répondit à Yahweh : Est-ce en vain que Job craint Dieu ?
\VS{10}N'as-tu pas mis une haie \FTNT{La haie est une protection, une barrière végétale entretenue afin de protéger et de clôturer un terrain.} tout autour de lui, autour de sa maison, autour de tout ce qui lui appartient ? Tu as béni l'œuvre de ses mains, et ses troupeaux se répandent sur la terre.
\VS{11}Mais étends maintenant ta main, touche à tout ce qui lui appartient, et tu verras s'il ne te maudit pas en face.
\VS{12}Et Yahweh dit à Satan : Voilà, tout ce qui lui appartient est en ton pouvoir ; seulement ne porte pas la main sur lui. Et Satan sortit de devant la face de Yahweh\FTNT{1 R. 22:22.}.
\TextTitle{Première attaque de Satan}
\VS{13}Il arriva donc qu'un jour comme les fils et les filles de Job mangeaient et buvaient du vin dans la maison de leur frère aîné, un messager vint vers Job,
\VS{14}et lui dit : Les bœufs labouraient, et les ânesses paissaient à côté d'eux;
\VS{15}et ceux de Séba se sont jetés dessus, les ont pris, et ont frappé les serviteurs au fil de l'épée. Et je me suis échappé, moi seul, pour te l'annoncer.
\VS{16}Cet homme parlait encore, lorsqu'un autre vint et dit : Le feu de Dieu est tombé du ciel, il a brûlé les brebis et les serviteurs, et les a consumés\FTNT{2 R. 1:10-12.}. Et je me suis échappé moi seul, pour te l'annoncer.
\VS{17}Cet homme parlait encore, lorsqu'un autre vint et dit : Des Chaldéens\FTNT{Ge. 11:28.} ont fait trois bandes, se sont jetés sur les chameaux et les ont pris, ils ont frappé les serviteurs au fil de l'épée, et je me suis échappé moi seul, pour te l'annoncer.
\VS{18}Cet homme parlait encore, lorsqu'un autre vint et dit : Tes fils et tes filles mangeaient et buvaient du vin dans la maison de leur frère aîné ;
\VS{19}voici, un grand vent est venu de l'autre côté du désert et a frappé contre les quatre coins de la maison ; elle est tombée sur les jeunes gens, et ils sont morts. Et je me suis échappé, moi seul, pour te l'annoncer.
\VS{20}Alors Job se leva, déchira\FTNT{Job 2:12 ; Est. 4:1.} son manteau et  rasa la tête ; et se jetant par terre, se prosterna,
\VS{21}et dit : Je suis sorti nu du ventre de ma mère, et nu je retournerai dans le sein de la terre\FTNT{Ec. 5:14. ; 1 Ti. 6:7.} ; Yahweh a donné, Yahweh a enlevé\FTNT{1 S. 2:6.} ; que le nom de Yahweh soit béni !
\VS{22}En tout cela, Job ne pécha pas et n'attribua rien d'injuste à Dieu.
\Chap{2}
\TextTitle{Deuxième attaque de Satan}
\VerseOne{}Or il arriva un jour que les fils de Dieu vinrent un jour se présenter devant Yahweh, Satan\FTNT{Za. 3:1-2.} vint aussi au milieu d'eux se présenter devant Yahweh.
\VS{2}Yahweh dit à Satan : D'où viens-tu ? Satan répondit à Yahweh : De courir çà et là sur la terre et de m'y promener.
\VS{3}Yahweh dit à Satan: N'as-tu point considéré mon serviteur Job,qui n'a point d'égal sur la terre ; homme sincère et droit, craignant Dieu, et se détournant du mal ? Il demeure ferme dans son intégrité, quoique tu m'aies incité contre lui à le détruire sans cause\FTNT{Job 9:17.}.
\VS{4}Et Satan répondit à l’Eternel, en disant : Chacun donnera peau pour peau, et tout ce qu’il a, pour sa vie.
\VS{5}Mais étends maintenant ta main, et frappe à ses os et à sa chair\FTNT{Job 19:20.}, et tu verras s'il ne te maudit pas en face. 
\VS{6}Yahweh dit à Satan : Voici, il est en ta main : Seulement garde sa vie.
\VS{7}Ainsi Satan sortit de devant l’Eternel, et frappa Job d’un ulcère malin, depuis la plante de ses pieds jusqu’au sommet de la tête.
\VS{8}Job prit un tesson pour se gratter et s'assit au milieu de la cendre\FTNT{Jé. 6:26 ; Jon. 3:6.}.
\TextTitle{Réaction de Job et de sa femme}
\VS{9}Et sa femme lui dit : Conserveras-tu encore ton intégrité ?  Bénis\FTNT{Job 1:11.} Dieu, et meurs !
\VS{10}Et il lui dit : Tu parles comme une femme insensée ! Nous recevons le bien de la part de Dieu, et nous n'en recevrions pas le mal !\FTNT{Es. 45:7 ; Am. 3:6 ; La. 3:37.} En tout cela, Job ne pécha pas par ses lèvres.
\TextTitle{Job et ses trois amis}
\VS{11}Et trois des amis de Job, Eliphaz de Théman, Bildad de Schuach, et Tsophar de Naama, ayant appris tous les maux qui lui étaient arrivés, vinrent chacun du lieu de leur demeure, après s'être convenus ensemble d'un jour pour venir le plaindre et le consoler.
\VS{12}Ayant de loin levé les yeux sur lui, ils ne le reconnurent pas, alors ils élevèrent la voix et ils pleurèrent. Ils déchirèrent leurs manteaux, et jetèrent de la poussière vers le ciel au-dessus de leur tête.
\VS{13}Et ils s'assirent à terre avec lui, sept jours et sept nuits, et aucun d'eux ne lui dit une parole, car ils voyaient que sa douleur était fort grande.
\Chap{3}
\TextTitle{Lamentations de Job}
\VerseOne{}Après cela, Job ouvrit la bouche et maudit le jour de sa naissance.\FTNT{Jé. 20:14 ; Job 10:18.}
\VS{2}Car prenant la parole, il dit :
\VS{3}Périsse le jour où je suis né, et la nuit qui a dit : Un homme est conçu !
\VS{4}Que ce jour-là ne soit que ténèbres ; que Dieu ne le recherche point d'en haut, et qu'il ne soit point éclairé de la lumière ! 
\VS{5}Que les ténèbres et l'ombre de la mort\FTNT{Job 10:21-22.} s'en emparent, que les nuées demeurent sur lui, qu'il soit rendu terrible comme le jour de ceux à qui la vie est amère ! 
\VS{6}Que l'obscurité prenne cette nuit, qu'elle ne se réjouisse pas au milieu des jours de l'année, qu'elle n'entre pas dans le compte des mois !
\VS{7}Voici, que cette nuit soit stérile, et qu'aucun cri de joie n'y survienne !
\VS{8}Qu'ils la maudissent ceux qui maudissent les jours, ceux qui sont prêts à réveiller le Léviathan !
\VS{9}Que les étoiles de son crépuscule soient obscurcies ; qu'elle attende la lumière, mais qu'il n'y en ait point, et qu'elle ne voie point les rayons de l'aube du jour ! \FTNT{Job 41:9.} !
\VS{10}Parce qu'elle n'a pas fermé le sein qui me conçut ni caché la souffrance à mes yeux.
\VS{11}Pourquoi ne suis-je pas mort dans le sein de ma mère ? Pourquoi n'ai-je pas expiré aussitôt que je suis sorti de ses entrailles ?\FTNT{Job 10:18.}
\VS{12}Pourquoi des genoux m'ont-ils reçu? Pourquoi des mamelles m'ont-elles allaité ?
\VS{13}Je serais couché maintenant, je h serais tranquille, je dormirais, je me reposerais\FTNT{Job 17:16.},
\VS{14}avec les rois et les grands de la terre, qui se bâtirent des mausolées,
\VS{15}avec les princes qui possedèrent de l'or, et qui remplirent d'argent leurs maisons.
\VS{16}Ou comme l'avorton caché, je n'existerais pas\FTNT{Ps. 58:9.}, comme les petits enfants qui n'ont pas vu la lumière.
\VS{17}Là les méchants n'agitent plus personne, et là se reposent ceux qui sont fatigués. 
\VS{18}Pareillement ceux qui avaient été dans les liens, jouissent là du repos, et n'entendent plus la voix de l'oppresseur. 
\VS{19}le petit et le grand sont là, et l'esclave est délivré de son maître.
\VS{20}Pourquoi la lumière est-elle donnée au misérable, et la vie à ceux qui ont le cœur dans l'amertume ;
\VS{21}qui désirent en vain la mort, et qui la recherchent plus que le trésor,\FTNT{Ap. 9:6.}
\VS{22}qui seraient ravis de joie et seraient dans l'allégresse s'ils avaient trouvé le tombeau ?
\VS{23}Pourquoi, dis-je, la lumière est-elle donnée à l'homme à qui le chemin est caché, et que Dieu a enfermé de toutes parts\FTNT{Job 19:8 ; La. 3:7.} ?
\VS{24}Car avant que je mange, mon soupir vient, et mes cris se répandent comme de l'eau. 
\VS{25}Ce que je crains le plus, m'arrive, et ce que je redoute le plus, m'atteint. 
\VS{26}Je n'ai point eu de paix, je n'ai point eu de repos, ni de calme, depuis que ce trouble m'est arrivé. 
\Chap{4}
\TextTitle{Premier discours d'Eliphaz}
\VerseOne{}Alors Eliphaz de Théman prit la parole et dit :
\VS{2}Si l'on tente de te parler, en seras-tu peiné ? Mais qui pourrait retenir ses paroles ?
\VS{3}Voici, tu as souvent instruit les autres, et tu as fortifié les mains affaiblies\FTNT{Es. 35:3 ; Hé. 12:12.},
\VS{4}Tes paroles ont affermi ceux qui chancelaient, et tu as fortifié les genoux qui pliaient\FTNT{Job 16:5.}.
\VS{5}Et maintenant que le malheur t'arrive, tu faiblis ! Maintenant que tu es atteint, tu en es tout troublé !
\VS{6} Ta crainte de Yahweh n'a-t-elle pas été ton espérance ? Et l'intégrité de tes voies n'a-t-elle pas été ton attente ? 
\VS{7}Rappelle, je te prie, dans ton souvenir : Quel est l'innocent qui a péri ? Quels sont les justes qui ont été exterminés ?\FTNT{Job 8:20.}
\VS{8}Selon ce que j'ai vu, ceux qui labourent l'iniquité et qui sèment la peine en moissonnent les fruits ;\FTNT{Job 15:35 ; Ga. 6:7.}
\VS{9}ils périssent par le souffle de Dieu, et ils sont consumés par le vent de ses narines.\FTNT{Ex. 15:8 ; Es. 11:4 ; 30:33 ; Job 15:30 ; 2 Th. 2:8.}
\VS{10}Il étouffe le rugissement du lion, et le cri d'un grand lion, et il arrache les dents des lionceaux ;
\VS{11}le lion périt faute de proie, et les petits de la lionne sont dispersés.
\VS{12}Une parole m'est furtivement arrivée, et mon oreille en a saisi les sons légers.
\VS{13}Au moment où les visions de la nuit agitent la pensée, quand un profond sommeil tombe sur les hommes\FTNT{Job 33:15.},
\VS{14}une frayeur et un tremblement me saisirent, et tous mes os tremblèrent.
\VS{15}Un esprit passa devant moi, et mes cheveux en furent tout hérissés. 
\VS{16}Il se tint là et je ne reconnus pas son visage ; une figure était devant mes yeux. Et j'entendis un léger murmure et une voix :
\VS{17}L'homme serait-il juste devant Dieu ? L'homme serait-il pur devant celui qui l'a fait ?\FTNT{Job 25:4.}
\VS{18}Voici, il ne se fie pas à ses serviteurs, il trouve des erreurs à ses anges\FTNT{Job 15:15 ; Job 25:5 ; 2 Pi 2:4. }
\VS{19}combien plus chez ceux qui habitent des maisons d'argile, qui ont leurs fondements dans la poussière, qu'on écrase comme des vermisseaux !\FTNT{Job 25:6.}
\VS{20}Du matin au soir ils sont brisés, et, sans qu'on s'en aperçoive, ils périssent pour toujours. 
\VS{21}L'excellence qui était en eux, n'a-t-elle pas été emportée ? Ils meurent sans être sages. 
\Chap{5}
\VerseOne{}Crie maintenant ! Y aura-t-il quelqu'un qui te réponde ? Et vers quel saint te tourneras-tu ?\FTNT{Job 15:15.}
\VS{2}La colère tue l'insensé, et le fou meurt dans ses emportements.
\VS{3}J'ai vu l'insensé qui s'enracinait \FTNT{Jé. 12:1-2.}, mais j'ai aussitôt maudit sa demeure.
\VS{4}Ses fils sont loin de tout secours ; ils sont écrasés à la porte, et personne ne les délivre !\FTNT{Ps. 119:155.}
\VS{5} Sa moisson est dévorée par l'affamé, qui même la ravit d'entre les épines ; et le voleur convoite ses biens.
\VS{6}Le malheur ne sort pas de la poussière, et le travail ne germe pas de la terre ;
\VS{7}l'homme naît pour la peine\FTNT{Ge. 3:17-19 ; Job 14:1-5.}, comme l'étincelle pour voler et s'élever.
\VS{8}Mais moi, j'aurais recours à Dieu, et j'adresserais ma parole à Dieu.
\VS{9}Il fait de grandes choses qu'on ne peut sonder, de merveilleuses choses qu'on ne peut compter\FTNT{Ps. 72:18. Ps. 92:5 ; Job 9:10.}.
\VS{10}Il répand la pluie sur la face de la terre, et envoie les eaux sur les campagnes\FTNT{De. 28:12 ; Ps. 135:7 ; Job 28:26; Job 38:25-26 ; Ac. 14:17.};
\VS{11}il met en haut ceux qui sont abaissés, et délivre les affligés\FTNT{1 S. 2:7; Ez. 21:31 ; Ps. 113:7-8.} ;
\VS{12}il anéantit les projets des hommes rusés, de sorte qu'ils ne viennent pas à bout de leurs entreprises\FTNT{Es. 8:10 ; Ps. 33:10 ; Né. 4:15.} ;
\VS{13}il prend les sages dans leur propre ruse\FTNT{1 Co. 3:19.}, et les desseins des hommes pervers sont renversés :
\VS{14}De jour ils rencontrent les ténèbres, et ils marchent à tâtons en plein midi, comme dans la nuit.
\FTNT{De. 28:29.}.
\VS{15}Ainsi Dieu délivre le pauvre de l'épée de leur bouche, et le sauve de la main des puissants\FTNT{Ps. 12:3-4; Ps 52:2; Ps. 57:4.} ;
\VS{16}et l'espérance soutient le malheureux\FTNT{1S. 2:8.}, et la méchanceté a la bouche fermée\FTNT{Es. 52:15 ; Ps. 63: 11; Ps. 107: 42; Pr. 10:6.}.
\VS{17}Voici, heureux est celui que Yahweh châtie ! Ne rejette donc point le châtiment de Yahweh.
\FTNT{Ps 94:12 ; Pr.3:11-12 ; Hé. 12:5-6; Ap. 3:19.}.
\VS{18}Car c'est lui qui fait la plaie, et la bande ; il blesse et ses mains guérissent\FTNT{De. 32:39; 1S. 2: 6-7 ; Cp. Es. 30:26 ; Os. 6:1.}.
\VS{19}Six fois il te délivrera de l'angoisse, et sept fois le mal ne te touchera pas\FTNT{Ps 34:20; Ps. 91:3; Pr.24:16.}.
\VS{20}Il te sauvera de la mort pendant la famine, et du tranchant de l'épée pendant la guerre\FTNT{Ps. 33: 19; Ps. 37:19.}.
\VS{21}Tu seras à l'abri du fléau de la langue, et tu n’auras point peur de la dévastation, quand elle arrivera.
\FTNT{Ps. 31:21.}.
\VS{22}Tu riras de la dévastation et de la famine, et tu n'auras pas peur des bêtes de la terre\FTNT{Es. 65:25; Ez. 34:25; Os. 2:20.};
\VS{23}car tu feras une alliance avec les pierres des champs, et les bêtes des champs seront en paix avec toi\FTNT{Os. 2:20.}.
\VS{24}Tu jouiras en paix de la prospérité sous ta tente, tu pourvoiras à ta demeure et tu n'y seras point trompé ;
\VS{25}tu verras ta postérité s'accroître, et tes descendants se multiplier comme l'herbe de la terre\FTNT{Ps. 72:16; Ps. 127: 3-5; Ps. 128:6.}.
\VS{26}Tu entreras au tombeau dans ta vieillesse, comme une gerbe qu'on emporte en son temps\FTNT{Pr. 9:11; Pr. 10:27.}.
\VS{27}Voilà ce que nous avons examiné, voilà ce qui est ; à toi d'entendre et de choisir.
\Chap{6}
\TextTitle{Réponse de Job}
\VerseOne{}Job prit la parole et dit :
\VS{2}Oh ! si l’on pesait ma douleur, et si l’on mettait en même temps mes calamités dans la balance !
\VS{3}Car elle serait plus pesante que le sable de la mer ; c’est pourquoi mes paroles sont englouties.
\FTNT{Pr. 27:3.} !
\VS{4}Car les flèches du Tout-Puissant sont sur moi, mon âme en boit le venin ; les terreurs\FTNT{Job 30:15 ; Ps. 88:16-17.} de Dieu se rangent en bataille contre moi\FTNT{Job 19:12 ; Ps. 38:2-3.}.
\VS{5}L'âne sauvage\FTNT{Job 39:8.} brait-il auprès de l’herbe ? Le bœuf mugit-il auprès de son fourrage ?
\VS{6}Mange-t-on sans sel ce qui est fade ? Trouve-t-on du goût dans un blanc d’œuf ?
\VS{7}Ce que mon âme voudrait ne pas toucher, c'est là ma nourriture, si dégoûtante soit-elle !
\VS{8}Oh ! Puisse ma prière s'accomplir et Dieu me donner ce que j'attends !
\VS{9}Qu’il plaise à Dieu de me réduire en poussière, qu’il laisse aller sa main pour m’achever !
!\FTNT{Job 7:16; 9:21; 10:1; cp. No. 11:15; 1R. 19:4; Jon. 4:3, 8.}
\VS{10}Mais j’ai encore cette consolation, quoique la douleur me consume, et qu’elle ne m’épargne point, je n'ai pas transgressé les paroles du Saint.
\VS{11}Quelle est ma force pour que j’espère, et quelle est ma fin pour que je prenne patience ?
\VS{12}Ma force est-elle une force de pierre ? Ma chair est-elle d'airain ?
\VS{13}Ne suis-je pas sans secours, et le salut n'est-il pas loin de moi ?
\VS{14}A celui qui souffre, est due la compassion de son ami ; mais il a abandonné la crainte. \FTNT{Ps. 19:10.} du Tout-Puissant\FTNT{Pr. 17:17.}.
\VS{15}Mes frères m'ont trompé comme un torrent, comme le lit des torrents qui passent\FTNT{Ps. 38:12; Ps 41:10; Ps 69:9; Jé. 15:19.}.
\VS{16}Les glaçons en troublent le cours, la neige s'y cache ;
\VS{17}mais au temps de la sécheresse, ils tarissent, et dans les chaleurs, ils disparaissent de leur place.
\VS{18}Les caravanes se détournent de leur route, elles montent dans le désert et périssent.
\VS{19}Les caravanes de Théma\FTNT{Ge. 25:15.} fixent le regard, les voyageurs de Séba\FTNT{1R. 10:1; Ps. 72:10; Ez. 27:22-23.} s'attendent à eux;
\VS{20}ils sont honteux d'avoir eu cette confiance, ils restent confondus quand ils arrivent.
\VS{21}Certes, vous m'êtes devenus inutiles ; vous voyez mon angoisse, et vous en avez horreur !\FTNT{Job 19:13 ; Ps. 31:12.}
\VS{22}Mais vous ai-je dit : Donnez-moi quelque chose, et de vos biens, faites des présents en ma faveur ? 
\VS{23}délivrez-moi de la main de l'ennemi, et rachetez-moi de la main des violents ?
\VS{24}Instruisez-moi, et je me tairai ; faites-moi comprendre en quoi je me suis égaré.
\VS{25}Ô combien sont fortes les paroles de vérité ! Mais que veut censurer votre argumentation ?
\VS{26}Voulez-vous donc blâmer ce que j'ai dit, et ne voir que du vent dans les paroles d'un homme désespéré ?\FTNT{Ec. 9:16.}
\VS{27}Vous vous jetez même sur un orphelin, vous persécutez votre ami.
\VS{28}Regardez-moi, je vous prie ! Et voyez si je vous mens en face ?
\VS{29}Revenez\FTNT{Job 17:10.} donc, soyez sans injustice; revenez, et reconnaissez mon innocence\FTNT{Job 27:5-6 ; 34:5 ; cp. Job 23:10 ; 42:1-6.}.
\VS{30}Y a-t-il de l'injustice dans ma langue ? et mon palais ne sait-il pas discerner le mal ? 
\Chap{7}
\VerseOne{}N'y a-t-il pas un temps de guerre limité à l'homme sur la terre ? Et ses jours ne sont-ils pas comme les jours d'un mercenaire ?
\VS{2}Comme un esclave, il soupire après l'ombre, comme un mercenaire\FTNT{Es. 16:14.}, il attend son salaire\FTNT{Ps. 39:5.}.
\VS{3}Ainsi j'ai reçu en partage des mois en vain, et l'on m'a assigné des nuits de peine\FTNT{Ps. 6:6.}.
\VS{4}Si je suis couché, je dis : Quand me lèverai-je ? Quand finira la nuit ? Et je suis rassasié d'agitations jusqu'au point du jour\FTNT{De. 28:67.}.
\VS{5}Ma chair se couvre de vers et d'une croûte terreuse, ma peau se crevasse et coule.
\VS{6}Mes jours sont plus rapides que la navette du tisserand, ils se consument sans espoir !\FTNT{Es. 38:12 ; Job 9:25 ; 17:11; Ja. 4:14.}
\VS{7}Souviens-toi que ma vie est un souffle ! Et que mes yeux ne reverront plus le bonheur\FTNT{Es. 40:6 ; Ps. 78:39 ; Ps. 89:48 ; Ps. 102: 12 ; Ps. 103:15 ; Job 8:9 ; Job 14:1-2 ; 1P. 1:24.}.
\VS{8}L'œil de ceux qui me regarndent ne me verra plus ; tes yeux seront sur moi, et je ne serai plus.
\VS{9}La nuée se dissipe et s'en va, ainsi celui qui descend au scheol\FTNT{cp. Ha. 2:5 ; Lu. 16:23.} ne remontera pas\FTNT{Job 10:21-22 ; Job 14:7-14.};
\VS{10}il ne reviendra plus dans sa maison, et le lieu qu'il habitait ne le reconnaîtra plus\FTNT{Ps. 37:35-36 ; Ps. 103:16 ; Job 10:21.}.
\VS{11}C'est pourquoi, je ne retiendrai pas ma bouche, je parlerai dans l'angoisse de mon esprit, je me plaindrai dans l'amertume de mon âme\FTNT{Job 10:1.}.
\VS{12}Suis-je une mer ? Suis-je un monstre marin, pour que tu poses autour de moi des gardes ?
\VS{13}Quand je dis : Mon lit me consolera, ma couche calmera ma plainte,
\VS{14}alors tu me terrifies par des songes, et tu m'épouvantes par des visions.
\VS{15}C'est pourquoi je choisirais d'être étranglé, et de mourir, plutôt que de conserver mes os.
\VS{16}Je les méprise !… Je ne vivrai pas toujours… Laisse-moi car mes jours sont un souffle\FTNT{Job 10:20.}.
\VS{17}Qu'est-ce que l'homme pour que tu en fasses tant de cas, pour que tu poses ta main sur son cœur,\FTNT{Ps. 8:5 ; Ps. 144:3 ; Hé. 2:6.}
\VS{18}pour que tu le visites tous les matins, pour que tu l'éprouves\FTNT{Job 23:10.} à chaque instant ?
\VS{19}Quand finiras-tu de me regarder? Ne me lâcheras-tu pas, pour que j'avale ma salive ?\FTNT{Job 9:18.}
\VS{20}J'ai péché ; que te ferai-je, gardien des hommes ?  \FTNT{1 Ti. 4:10.}  Pourquoi m'as-tu mis en butte à tes coups, et pourquoi suis-je à charge à moi-même ?
\VS{21}Et pourquoi ne pardonnes-tu pas mon péché, et ne fais-tu pas passer mon iniquité ? Car je vais maintenant me coucher dans la poussière ; tu me chercheras, et je ne serai plus.
\Chap{8}
\TextTitle{Premier discours de Bildad}
\VerseOne{}Bildad de Schuach prit la parole et dit :
\VS{2}Jusqu'à quand parleras-tu ainsi, et les paroles de ta bouche seront-elles un vent impétueux ?\FTNT{Job 15:2.}
\VS{3}Dieu renverserait-il le droit, et le Tout-puissant renverserait-il la justice ? \FTNT{Cp. Ge. 18:25.} ?\FTNT{De. 32:4 ; Job 34:12 ; Da. 9:14 ; 2 Ch. 19:7.}
\VS{4}Si tes fils ont péché contre lui, il les a livrés à leur crime.
\VS{5}Mais toi si tu cherches Dieu, si tu demandes grâce au Tout-Puissant ;\FTNT{Cp. Job 5:17-27.}
\VS{6}si tu es pur et droit, il veillera certainement sur toi, il rendra le bonheur à la demeure de ta justice ;
\VS{7}tes commencements\FTNT{Za. 4:10.} auront été peu de chose, et ta fin sera bien plus grande.\FTNT{Job 42:12.}
\VS{8}Interroge ceux des générations précédentes, applique-toi à l'expérience de leurs pères.\FTNT{De. 4:32 ; De. 32:7.}
\VS{9}Car nous sommes d'hier, et nous ne savons rien, parce que nos jours sur la terre ne sont qu'une ombre.\FTNT{Ps. 102:12 ; Ps. 144:41 ; Ch. 29:15.}
\VS{10}Ils t'instruiront, ils te parleront, ils tireront de leur cœur ces discours :
\VS{11}Le roseau croît-il sans marais ? Le jonc pousse-t-il sans eau ?
\VS{12}Il est encore en sa verdure, sans qu'on le coupe, il sèche plus vite que toutes les herbes.\FTNT{Cp. Jé. 17:5-8 ; Ps. 129:6.}
\VS{13}Ainsi est la voie de tous ceux qui oublient Dieu\FTNT{Ps. 9:18.}, et l'espérance de l'impie périra\FTNT{Ps. 1:4 ; Ps. 112:10 ; Pr. 10:28 ; Job 11:20 ; Job 27:8.}.
\VS{14}Sa confiance est brisée, son soutien est une toile d'araignée.
\VS{15}Il s'appuie sur sa maison, et elle ne tient pas ; il s'y cramponne, et elle ne reste pas debout.
\VS{16}Dans toute sa vigueur, en plein soleil, il étend ses rameaux sur son jardin,
\VS{17}mais ses racines s'entrelacent parmi des monceaux de pierres, il pénètre dans les rochers.
\VS{18}S'Il l'ôte de sa place, celle-ci le renie, disant je ne t'ai pas connu ! 
\VS{19}Telle est la joie que ses voies lui procurent. Puis sur le même sol, d'autres s'élèvent après lui.
\VS{20}Dieu ne rejette pas l'homme intègre, il ne soutient pas la main des méchants.\FTNT{Job 4:7.}
\VS{21}Il remplira encore ta bouche de cris de joie, et tes lèvres de chants d'allégresse.\FTNT{Ps. 126:2.}
\VS{22}Ceux qui te haïssent seront revêtus de honte, et la tente des méchants ne sera plus. \FTNT{Ps. 35:26 ; Ps. 109:29.}
\Chap{9}
\TextTitle{Réponse de Job}
\VerseOne{}Job prit la parole et dit :
\VS{2}Certainement, je sais qu'il en est ainsi ; et comment l'homme mortel se justifierait-il devant Dieu ? \FTNT{Ha. 2:4 ; Ga. 3:11 ; Ro. 1:17 ; Hé. 10:38.} devant Dieu ?\FTNT{Ps. 25:4 ; Ps. 143:2 ; Job 15:14-16 ; Da. 9:11 ; Ro 3:19.}
\VS{3}S'il veut plaider avec lui, il ne lui répondra pas une fois sur mille. \FTNT{Es. 45:9-10.}
\VS{4} Dieu est sage de coeur, et puissant en force. Qui est-ce qui s'est opposé à lui, et s'en est bien trouvé ? \FTNT{Job 12:13 ; Job 36:5 ; Job 37:23.}
\VS{5}Il transporte les montagnes, et quand il les renverse dans sa fureur, elles n'en connaissent rien.\FTNT{Ps. 144:5.}
\VS{6}Il remue la terre de sa place, et ses piliers sont ébranlés.\FTNT{Ag. 2:6, 21 ; Hé. 12:26.}
\VS{7}Il commande au soleil, et le soleil ne se lève pas ; et il met un sceau sur les étoiles.\FTNT{Jos. 10:12.}
\VS{8} C'est lui seul qui étend les cieux\FTNT{Ge 1:6-8 ; Es. 44:24; Es. 51:13 ; Ps. 104:2.},qui marche sur les hauteurs de la mer\FTNT{Cp. Mt. 14:25.}.
\VS{9}Il a fait la grande ourse, l'orion, les pléiades, et les étoiles des régions australes.\FTNT{Ge. 1:16 ; Am. 5:8 ; Ps. 89:12 ; Job 38:31-32.}
\VS{10}Il fait de grandes choses qu'on ne peut sonder, des merveilles sans nombre.\FTNT{Ps. 86:10 ; Ps. 139:6, 17-18 ; Job 5:9 ; Job 37:5.}
\VS{11}Voici, il passe près de moi, et je ne le vois pas ; il passe encore, et je ne l'aperçois pas.\FTNT{Job 23:8-9 ; 35:14.}
\VS{12}S'il enlève, qui l'en détournera? Qui lui dira : Que fais-tu ?\FTNT{Es. 45: 9-10 ; Da. 4:35 ; Ro. 11:33-35.}
\VS{13}Dieu ne revient pas sur sa colère ; sous lui s'inclinent les appuis de l'orgueil.\FTNT{Job 26:12; Cp. Es. 30:7.}
\VS{14}Combien moins lui répondrais-je, moi et comment choisirais-je mes paroles contre lui ? 
\VS{15}Quand je serais juste, je ne répondrais pas ; je demanderais grâce à mon juge.\FTNT{Job 23:1-7.}
\VS{16}Si je l'invoque et qu'il me réponde, ne croirais-je pas qu'il ait écouté ma voix,
\VS{17}lui qui m'assaille comme par une tempête, qui multiplie mes plaies sans motif,\FTNT{Job 6:29.}
\VS{18}qui ne me permet pas de reprendre haleine ; qui me rassasie d'amertume.\FTNT{Job 7:19.}
\VS{19}S'il est question de savoir qui est le plus fort ; voilà, il est fort ; et s'il est question d'aller en justice, qui est-ce qui m'y fera comparaître ? 
\VS{20}Si je me justifie, ma propre bouche me condamnera ; si je me fais parfait, il me convaincra d'être coupable.
\VS{21}Je suis innocent ! Je ne me soucie pas de vivre, je méprise ma vie.\FTNT{Job 10:1.}
\VS{22}Tout se vaut! C'est pourquoi j'ai dit: Il détruit l'innocent comme l'impie. \FTNT{ Cp. Ez. 21:3 ; Ec. 9:2-3 ; Mt 5:45.}
\VS{23}Au moins si le fléau dont il frappe faisait mourir tout aussitôt ; mais il se rit de l'épreuve des innocents. 
\VS{24}[C'est par lui que] la terre est livrée entre les mains du méchant ; c'est lui qui couvre la face des juges de la [terre] ; et si ce n'est pas lui, qui est-ce donc ? 
\VS{25}Or mes jours vont plus vite qu'un courrier ; ils s'en fuient sans avoir vu le bonheur ;\FTNT{Job 7:6-7.}
\VS{26}ils passent comme les navires de roseaux, comme l'aigle qui fond sur sa proie.
\VS{27}i je dis : J'oublierai ma plainte, je renoncerai à ma colère, je me fortifierai; 
\VS{28}Je suis épouvanté de tous mes tourments. Je sais que tu ne me jugeras pas innocent. .\FTNT{Cp. Ps. 130:3.}
\VS{29}Je serai jugé coupable ; pourquoi travaillerais-je en vain ?
\VS{30}Quand je me laverais dans de l'eau de neige, et que je nettoierais mes mains dans la pureté, \FTNT{Jé. 2:22.}
\VS{31}tu me plongerais dans le fossé, et mes vêtements m'auraient en horreur.
\VS{32}Car il n'est pas comme moi un homme, pour que je lui réponde, [et] que nous allions ensemble en jugement. .\FTNT{Es. 45:9 ; Jé 49:19 ; Ec. 6:10; Ro. 9:20.}
\VS{33}Mais il n'y a personne qui prend connaissance de la cause qui serait entre nous, et qui pose la main sur nous deux. \FTNT{Cp. 1 S. 2:25.}
\VS{34}Qu'il ôte donc sa verge de dessus moi, et que la frayeur que j'ai de lui ne me trouble plus. ;
\VS{35}Je parlerai, et je ne le craindrai pas ; mais dans l'état où je suis je ne suis plus à moi-même. 
\Chap{10}
\VerseOne{}Mon âme a pris en dégoût la vie ! Je laisserai aller ma plainte, je parlerai dans l'amertume de mon âme.
\VS{2}Je dirai à Dieu : Ne me condamne pas ; montre-moi pourquoi tu plaides contre moi ?
\VS{3}Te plais-tu à m'opprimer, et à dédaigner l'ouvrage de tes mains, et à bénir les desseins des méchants\FTNT{Es. 64:7-8.} ?
\VS{4}As-tu des yeux de chair ? Vois-tu comme voit un homme mortel?
\VS{5}Tes jours sont-ils comme les jours de l'homme mortel ? Tes années sont-elles comme les jours de l'homme, 
\VS{6}Que tu recherches mon iniquité, et que tu t'informes de mon péché,
\VS{7}tu sais que je n'ai point commis de crime, et qu'il n'y a personne qui me délivre de ta main?
\VS{8}Tes mains m'ont formé, et elles ont rangé toutes les parties de mon corps; et tu me détruirais !\FTNT{Ge. 2:7 ; Ps. 119:73 ; Ps. 139:14-15.} !
\VS{9}Souviens-toi, je te prie, que tu m'as formé comme de la boue, et que tu me feras retourner en poudre?
\VS{10}Ne m'as-tu pas coulé comme du lait ? et ne m'as-tu pas fait cailler comme un fromage ?
\VS{11}Tu m'as revêtu de peau et de chair, et tu m'as composé d'os et de nerfs;
\VS{12}Tu m'as donné la vie, et tu as usé de miséricorde envers moi, et [par] tes soins continuels tu as gardé mon esprit.
\VS{13}Et cependant tu gardais ces choses en ton cœur ; mais je connais que cela était devant toi. 
\VS{14}Si j'ai pèche, tu m'observes, et tu ne me tiens pas pour innocent de mon iniquité.
\VS{15}Si j'agis méchamment, malheur à moi ! si je suis juste, je n'en lève pas la tête plus haut. Je suis rempli d'ignominie ; mais regarde mon affliction. 
\VS{16}Si je redresse la tête, tu me poursuis comme à un lion, et tu multiplie tes exploits contre moi; \FTNT{Zq. 38:13 ; La. 3:10.}.
\VS{17}Tu renouvelles tes témoins contre moi, et ton indignation augmente contre moi. De nouvelles troupes toutes fraîches viennent contre moi.
\VS{18}Mais pourquoi m'as-tu fait sortir du sein de ma mère? J'aurais expiré, et aucun œil ne m'aurait vu ;
\VS{19}et j'aurais été comme n'ayant jamais été, et j'aurais été porté du ventre de ma mère au tombeau.
\VS{20}Mes jours ne sont-ils pas en petit nombre ? Cesse donc et retire-toi de moi, et que je me renforce un peu. .
\VS{21}Avant que j'aille au lieu d'où je ne reviendrai plus, en la terre de ténèbres et de l'ombre de la mort,
\VS{22}terre d'une grande obscurité, comme les ténèbres de l'ombre de la mort, où il n'y a aucun ordre, et où rien ne luit que des ténèbres. 
\Chap{11}
\TextTitle{Première accusation de Tsophar}
\VerseOne{}Tsophar de Naama prit la parole et dit :
\VS{2}Ne répondra-t-on point à tant de discours, et suffira-t-il d'être un grand parleur pour être justifié ?
\VS{3}Tes vains feront-ils taire les gens ? Et quand tu te seras moqué, n'y aura-t-il personne qui te fasse honte ?
\VS{4}Car tu as dit : Ma doctrine est pure, et je suis sans tache devant tes yeux. 
\VS{5}Mais je voudrais que Dieu parle, et qu'il ouvre sa bouche pour te répondre; ,
\VS{6}Qu'il te montre les secrets de sa sagesse, de son immense sagesse; et que tu reconnaisse que Dieu oublie une partie de ton iniquité. .
\VS{7}Trouveras-tu Dieu en le sondant ? Connaîtras-tu parfaitement le Tout-puissant ? 
\VS{8}Ce sont les hauteurs des cieux : Qu'y feras-tu ? C'est plus profond que le scheol : Qu'y connaîtras-tu ?
\VS{9}Son étendue est plus longue que la terre, et plus large que la mer.
\VS{10}S'il remue, et qu'il resserre, ou qu'il rassemble, qui l'en détournera ?
\VS{11}Car il connaît les hommes vicieux, il discerne par le regard les coupables\FTNT{Ps. 10:11-14 ; Ps. 35:22.}.
\VS{12}Mais l'homme vide de sens devient intelligent, quoique l'homme naisse comme un ânon sauvage\FTNT{Ec. 3:18.}.
\VS{13}Si tu disposes ton cœur, et que tu étendes tes mains vers lui,
\VS{14}Si tu éloignes de toi l'iniquité qui est en ta main, et si tu ne permets pas que la méchanceté habite dans tes tentes ; 
\VS{15}Alors certainement tu pourras élever ton visage sans tache ; tu seras ferme et tu ne craindras rien;
\VS{16}tu oublieras tes peines, tu t'en souviendras comme des eaux écoulées.
\VS{17}La vie se lèvera pour toi plus brillante que le midi, et l'obscurité même sera comme le matin\FTNT{Ps. 37:6 ; Ps. 112:4.}.
\VS{18} Tu seras plein de confiance, parce qu'il y aura de l'espérance pour toi; tu creuseras, et tu reposeras sûrement. \FTNT{Lé. 26:6 ; Ps. 3:6 ; Pr. 3:24.}.
\VS{19}Tu te coucheras, et il n'y aura personne qui t'épouvante, et plusieurs te feront la cour. 
\VS{20}Mais les yeux des méchants seront consumés; tout refuge leur sera ôté et toute leur espérance sera de rendre l'âme !
\Chap{12}
\TextTitle{Réplique de Job}
\VerseOne{}Job reprit la parole, et dit :
\VS{2}On dirait vraiment que vous êtes tout un peuple, et qu'avec vous doit mourir la sagesse.
\VS{3}J'ai du bon sens aussi bien que vous, et je ne vous suis point inférieur ; et qui ne sait de telles choses ?
\VS{4}Je suis pour mes amis un objet de raillerie, quand je m'écrie à Dieu pour qu'il me réponde; on se moque d'un homme qui est juste et droit.
\VS{5} Mépris au malheur! telle est la pensée des heureux; le mépris est réservé à ceux dont le pied chancelle !
\VS{6}Elles sont en paix, les tentes des pillards, et toutes les sécurités sont pour ceux qui irritent Dieu, qui se font un dieu de leur bras. \FTNT{ Jé. 12:1 ; Ps. 73:12.}.
\VS{7}Mais interroge donc les bêtes, et elles t'instruiront, ou les oiseaux des cieux, et ils te l'annonceront ;
\VS{8}Ou parle à la terre, et elle t'enseignera ; même les poissons de la mer te le raconteront ; 
\VS{9}Qui est-ce qui ne sait toutes ces choses; que c'est la main de Yahweh qui a fait cela ?
\VS{10} Qu'il tient en sa main, l'âme de tout ce qui vit, et l'esprit de toute chair humaine,
\VS{11}L'oreille ne discerne-t-elle pas les discours, ainsi que le palais savoure les aliments ?
\VS{12}La sagesse est dans les vieillards, et l'intelligence est le fruit d'une longue vie.
\VS{13}Mais en Dieu est la sagesse et la force ; à lui appartient le conseil et l'intelligence. \FTNT{Da. 2:20.}.
\VS{14}Voici, il démolit, et on ne rebâtit pas ; il enferme un homme, et on ne lui ouvre pas\FTNT{Es. 22:22 ; Ap. 3:7.}.
\VS{15}Voilà, il retient les eaux, et tout devient sec ; il les lâche, et elles bouleversent la terre.
\VS{16} En lui résident la puissance et la sagesse; de lui dépendent celui qui s'égare et celui qui égare.
\VS{17} Il emmène dépouillés les conseillers, et il met hors de sens les juges. \FTNT{2 S. 15:31 ; 2 S. 17:14-23 ; Es. 19:12 ; Es. 29:14 ; 1 Co. 1:19.}.
\VS{18}Il rend impuissant le gouvernement des rois, et lie de chaînes leurs reins. 
\VS{19}Il fait marcher pieds nus les sacrificateurs ; et il renverse les puissants.
\VS{20}Il ôte la parole à ceux qui sont les plus assurés en leurs discours, et il prive de sens les anciens.
\VS{21}Il verse le mépris sur les nobles ; il relâche la ceinture des forts\FTNT{Es. 40:23.}.
\VS{22}Il met en évidence les choses qui étaient cachées dans les ténèbres, et il produit en lumière l'ombre de la mort. \FTNT{Ps. 139:11-12 ; Ec. 12:16 ; Mt. 10:26 ; 1 Co. 4:6.}.
\VS{23} Il multiplie les nations, et les fait périr ; il répand çà et là les nations, et puis il les ramène. 
\VS{24}Il ôte la raison aux Chefs des peuples de la terre, et les fait errer dans les déserts où il n'y a point de chemin;
\VS{25}ils tâtonnent dans les ténèbres, sans aucune clarté, et il les fait chanceler comme des gens ivres. 
\Chap{13}
\VerseOne{}Voici, mon œil a vu toutes ces choses, mon oreille l'a entendu et compris.
\VS{2}Comme vous les savez, je les sais aussi ; je ne vous suis pas inférieur. 
\VS{3}Mais je veux parler au Tout-Puissant, je veux plaider auprès de Dieu. .
\VS{4}Et certes vous inventez des mensonges ; vous êtes tous des médecins inutiles.
\VS{5}Plaît à Dieu que vous demeuriez entièrement dans le silence ; et cela vous sera réputé à sagesse. \FTNT{Pr. 17:28.}.
\VS{6}Ecoutez donc maintenant ma cause, et soyez attentifs à la défense de mes lèvres.
\VS{7}Tiendrez-vous des discours injustes en faveur de Dieu, et, pour le défendre, direz-vous des mensonges?
\VS{8}Ferez-vous acception de personnes en sa faveur? Prétendrez-vous plaider pour Dieu? 
\VS{9}S'il vous sonde, vous trouvera-t-il bon ? Comme on trompe un homme, le tromperez-vous? 
\VS{10}Certainement il vous reprendra, si même en secret vous faites acception de personnes.
\VS{11}Sa majesté ne vous épouvantera-t-elle pas ? Et sa frayeur ne tombera-t-elle pas sur vous ? 
\VS{12}Vos discours mémorables sont des sentences de cendre, et vos éminences sont des éminences de boue. 
\VS{13}Taisez-vous devant moi, et que je parle ; et il m'arrivera ce qui pourra. 
\VS{14}Pourquoi porterais-je ma chair entre mes dents, et tiendrais-je mon âme entre mes mains ? \FTNT{Jg. 12:3 ; 1 S. 19:5.}.
\VS{15}Voilà, qu'il me tue, je ne cesserai pas d'espérer en lui ; et je défendrai ma conduite en sa présence.
\VS{16}Et qui plus est, il sera lui-même mon salut ; mais l'hypocrite ne viendra point devant sa face. \FTNT{Ps. 1:5.}.
\VS{17}Ecoutez attentivement mes paroles, et prêtez l'oreille à ce que je vais vous déclarer. 
\VS{18}Voici, j'ai préparé ma cause. Je sais que je serai justifié.
\VS{19}Qui est-ce qui veut disputer contre moi ? car maintenant si je me tais, je mourrai. 
\VS{20}Seulement ne me fais pas ces deux choses, et alors je ne me cacherai point devant ta face :
\VS{21}Retire ta main de dessus moi, et que tes terreurs ne me troublent pas.
\VS{22}Puis appelle-moi, et je répondrai ; ou bien je parlerai, et tu me répondras. 
\VS{23}Combien ai-je d'iniquités et de péchés ? Montre-moi mon crime et mon péché. 
\VS{24}Pourquoi caches-tu ta face, et me tiens-tu pour ton ennemi ?
\VS{25}Déploieras-tu tes forces contre une feuille que le vent emporte ? Poursuivras-tu du chaume tout sec \FTNT{1 S. 24:15.} ?
\VS{26}Que tu écrives contre moi des choses amères, et que tu me fasses porter la peine des péchés de ma jeunesse? \FTNT{Ps. 25:7.} ?
\VS{27}Que tu mettes mes pieds aux ceps, et observes tous mes chemins, et que tu suives les traces de mes pieds,
\VS{28}Quand mon corps s'en va par pièces comme du bois vermoulu, et comme une robe que la teigne a rongée? 
\Chap{14}
\VerseOne{}L'homme né de la femme est de courte vie, et rassasié d'agitations. \FTNT{Ps. 102:12 ; Ps. 103:15 ; Ps. 144:4 ; Ja. 4:14.}.
\VS{2}Il sort comme une fleur, puis il est coupé, et il s'enfuit comme une ombre qui ne s'arrête pas. \FTNT{Es. 40:6 ; Ps. 90:61 ; 1 Pi. 1:24.}.
\VS{3}Cependant tu as ouvert tes yeux sur lui, et tu me conduis en justice avec toi.
\VS{4}Qui est-ce qui tirera le pur de l'impur ? Personne. \FTNT{Es. 48:8 ; Pr. 22:15.}.
\VS{5}Les jours de l'homme sont déterminés, le nombre de ses mois est entre tes mains, tu lui as prescrit ses limites, et il ne passera point au delà.
\VS{6}Retire-toi de lui, afin qu'il ait du relâche, jusqu'à ce que comme un mercenaire il ait achevé sa journée.
\VS{7}Car si un arbre est coupé, il y a de l'espérance, et il poussera encore, et ne manquera pas de rejetons ; 
\VS{8}quoique sa racine ait vieilli dans la terre, et que son tronc soit mort dans la poussière;
\VS{9}Dès qu'il sent l'eau il regerme, et produit des branches, comme un arbre nouvellement planté. 
\VS{10}Mais l'homme meurt et perd toute sa force; il expire et puis où est-il ?
\VS{11}Les eaux s'écoulent de la mer, et une rivière s'assèche, et tarit ;
\VS{12} Ainsi l'homme est couché par terre, et ne se relève plus ; jusqu'à ce qu'il n'y ait plus de cieux, ils ne se réveillera plus, et ne sera pas réveillé de son sommeil. 
\VS{13}Oh que tu me caches dans le scheol, que tu me gardes à l'abri jusqu'à ce que ta colère soit passée, que tu me donnes un temps arrêté, après lequel tu te souviendrais de moi !
\VS{14}Si un homme meurt, revivra-t-il ? Tous les jours de ma détresse, j'attendrais jusqu'à ce que mon état vînt à changer.
\VS{15} Tu appellerais, et moi je te répondrais, tu ne dédaignerais pas l'ouvrage de tes mains.
\VS{16}Mais maintenant tu comptes mes pas, et tu n'exceptes rien de mon péché. \FTNT{Ps. 56:9 ; Ps. 139:2-4 ; Pr. 5:21.} ;
\VS{17}Mes péchés sont scellés dans un sac, et tu as cousu ensemble mes iniquités. \FTNT{Os. 13:12.}.
\VS{18}Car comme une montagne s'éboule en tombant, et comme un rocher est transporté de sa place ; 
\VS{19} et comme les eaux minent les pierres, et entraînent par leur débordement la poussière de la terre, avec tout ce qu'elle a produit, tu fais ainsi périr l'attente de l'homme. 
\VS{20}Tu te montres toujours plus fort que lui, et il s'en va, et lui ayant défiguré le visage, tu le renvoies.
\VS{21}Quand ses fils sont honorés, il n'en sait rien ; et quand ils sont abaissés, il ne s'en aperçoit pas.
\VS{22}Seulement sa chair sur lui, a de la douleur, et son âme en lui s'afflige. 
\Chap{15}
\TextTitle{Deuxième discours d'Eliphaz}
\VerseOne{}Eliphaz de Théman prit la parole et dit :
\VS{2}Un homme sage profère-t-il dans ses réponses une science aussi légère que le vent, des opinions vaines ? Remplit-il son ventre du vent d'orient ?
\VS{3}Contestant avec des discours qui ne servent de rien, et avec des paroles dont on ne peut tirer aucun profit ?
\VS{4}Certainement tu abolis la crainte de Dieu, et tu anéantis peu à peu la prière qu'on doit présenter à Dieu. 
\VS{5} Car ta bouche fait connaître ton iniquité, et tu as choisi un langage trompeur. 
\VS{6}C'est ta bouche qui te condamne, et non pas moi ; et tes lèvres témoignent contre toi. 
\VS{7}Es-tu le premier homme né ? Ou as-tu été formé avant les montagnes ? \FTNT{Ps. 90:2 ; Pr. 8:25.} ?
\VS{8}As-tu été instruit dans le conseil secret de Dieu, et renfermes-tu seul la sagesse ? \FTNT{Es. 40:13 ; Jé. 23:18 ; Ro. 11:34.} ?
\VS{9}Que sais-tu que nous ne sachions pas ? Quelle connaissance as-tu que nous n'ayons pas ?
\VS{10}Parmi nous, il y a des hommes à cheveux blancs, et des gens d'une fort grande vieillesse, il y en a même de plus âgés que ton père. 
\VS{11}Les consolations du Dieu te semblent-elles trop petites ? As-tu quelque chose de caché par-devant toi ? …
\VS{12}Pourquoi ton coeur s'emporte-il et pourquoi tes yeux clignent-ils ?
\VS{13}C'est contre Dieu que tu tournes ta colère, et que tu fais sortir de ta bouche de tels discours! 
\VS{14}Qu'est-ce que de l'homme, pour qu'il soit pur, et celui qui est né de femme, pour qu'il soit juste ? \FTNT{Ps. 14:3 ; Pr. 20:9 ; Ec. 7:20.} ?
\VS{15}Si Voici, Dieu ne se fie pas à ses saints, et les cieux ne sont pas purs à ses yeux,
\VS{16}Combien plus est abominable et corrompu, l'homme qui boit l'iniquité comme l'eau !  
\VS{17}Je t'enseignerai, écoute-moi, et je te raconterai ce que j'ai vu ,
\VS{18} savoir ce que les sages ont déclaré, et qu'ils n'ont point caché ; ce qu'ils avaient [reçu] de leurs pères.
\VS{19}Eux à qui seuls la terre a été donnée, et parmi lesquels l'étranger n'est point passé.
\VS{20}Toute sa vie, le méchant est tourmenté, et un petit nombre d'années sont réservées au malfaiteur.\FTNT{Es. 48:22 ; Es. 57:21.}.
\VS{21}Un cri de frayeur est dans ses oreilles ; au milieu de la paix [il croit] que le destructeur se jette sur lui. \FTNT{1 Th. 5:3.} ;
\VS{22}Il ne croit pas pouvoir sortir des ténèbres, car il voit la menace de   l’épée;
\VS{23}il court çà et là pour chercher son pain, il sait que le jour des ténèbres lui est préparé \FTNT{Ps. 109:10.}.
\VS{24}La détresse et l'angoisse l'épouvantent, elles l'assaillent comme un roi prêt à combattre ;
\VS{25}Parce qu'il a élevé sa main contre Dieu, et qu'il s'est levé contre le Tout-puissant ;
\VS{26}Il lui a sauté au collet, et sur l'épaisseur de ses gros boucliers. 
\VS{27}Parce que la graisse a couvert son visage, et qu'elle a fait des replis sur son ventre;
\VS{28} il habite des villes détruites, des maisons désertes, tout près de n'être plus que des monceaux de pierres. 
\VS{29}Et il ne s'enrichira plus, car ses biens ne subsisteront pas, et ses richesses ne se répandront pas sur la terre. 
\VS{30}Il ne pourra pas se détourner des ténèbres, la flamme desséchera ses rejetons, et Dieu le fera disparaître par le souffle de sa bouche.
\VS{31}S'il a confiance dans la vanité, il se trompe, car la vanité sera sa récompense.
\VS{32}Ce sera fait de lui avant son temps, ses branches ne reverdiront plus. 
\VS{33}On arrachera ses fruits non mûrs, comme à une vigne; on jettera sa fleur, comme celle d'un olivier. 
\VS{34}Car la famille des hypocrites est stérile, et le feu dévore les tentes de l'homme corrompu.
\VS{35}Ils conçoivent le travail, et ils enfantent la misère, et machinent dans le cœur des fraudes. \FTNT{Es. 59:4 ; Os. 10:13.}.
\Chap{16}
\TextTitle{Réponse de Job}
\VerseOne{}Job répondit, et dit :
\VS{2}J'ai souvent entendu de pareils discours ; vous êtes tous des consolateurs fâcheux.
\VS{3}Y aura-t-il une fin à [ces] paroles de vent ? Qu'est-ce qui t'irrite, que tu répondes ?
\VS{4}Parlerais-je comme vous faites, si vous étiez en ma place ; accumulerais-je des paroles contre vous, ou secouerais-je ma tête contre vous ? 
\VS{5}Je vous fortifierais de ma bouche, et le mouvement de mes lèvres vous soulagerait.
\VS{6}Si je parle, ma douleur ne sera point soulagée. Si je me tais, en sera-t-elle diminuée?
\VS{7}Maintenant il m'a épuisé... Tu as dévasté toute ma famille, ;
\VS{8}Tu m'as tout couvert de rides, qui sont un témoignage des maux que je souffre ; et il s'est élevé en moi une maigreur qui en rend aussi témoignage sur mon visage. 
\VS{9}Sa fureur me déchire, il se déclare mon ennemi, il grince des dents contre moi, et étant devenu mon ennemi, il étincelle des yeux contre moi.
\VS{10}Ils ouvrent contre moi leur bouche; ils me frappent à la joue pour m'outrager; ils se réunissent tous ensemble contre moi. 
\VS{11}Dieu m'a livré à l'impie, et m'a jeté entre les mains des méchants. 
\VS{12}J'étais tranquille, et il m'a secoué, il m'a saisi par la nuque et m'a brisé, il m'a posé en butte à ses traits.
\VS{13}Ses archers m'ont environné, il me perce les reins, et ne m'épargne pas ; il répand mon fiel par terre. 
\VS{14}Il m'a brisé en me faisant plaie sur plaie, il a couru sur moi comme un homme fort.
\VS{15}J'ai cousu un sac sur ma peau, j'ai souillé ma tête dans la poussière\FTNT{Ps. 44 : 25 ; Ps. 119 : 25.},
\VS{16}J'ai le visage tout enflammé, à force de pleurer, et l'ombre de la mort est sur mes paupières, 
\VS{17}Quoiqu'il n'y ait point de violence dans mes mains, et que ma prière fut toujours pure.
\VS{18}Ô terre, ne cache pas mon sang, et qu'il n'y ait aucun lieu où s'arrête mon cri !
\VS{19}Mais maintenant voilà, mon témoin est aux cieux, mon témoin est dans les lieux élevés. \FTNT{Ap. 1:5 ; Ap. 3:14.}.
\VS{20}Mes amis se moquent de moi: c'est vers Dieu que mon œil se tourne en pleurant,
\VS{21}pour qu'il fasse justice entre l'homme et Dieu, entre le fils d'Adam et son semblable.
\VS{22}Car les années de mon compte arrivent à leur terme, et j'entre dans un sentier d'où je ne reviendrai plus. 
\Chap{17}
\VerseOne{}Mon souffle se perd, mes jours s'éteignent, le sépulcre m'attend.
\VS{2}Je suis environné de moqueurs, et mon œil veille toute la nuit au milieu de leurs insultes.
\VS{3}Dépose un gage, sois ma caution auprès de toi-même; car qui voudrait répondre pour moi? 
\VS{4}C'est pourquoi tu ne les élèveras pas\FTNT{De. 29:4 ; Mt. 11:25.}.
\VS{5}Celui qui trahit ses amis pour qu'ils soient pillés, les yeux de ses fils se consument.
\VS{6}On a fait de moi la fable des peuples, un être à qui l'on crache au visage.
\VS{7}Mon œil est obscurci par le chagrin, tous mes membres sont comme une ombre\FTNT{Ps. 6:7 ; Ps. 31:10.}.
\VS{8}Les hommes droits en sont consternés, et l'innocent est irrité contre l'impie. 
\VS{9}Toutefois le juste se tient ferme dans sa voie, et celui qui a les mains pures, se renforce.
\VS{10}Retournez donc vous tous, et revenez, je vous prie ; car je ne trouve pas de sages parmi vous. 
\VS{11}Mes jours sont passés; mes desseins, chers à mon cœur, sont renversés.…
\VS{12}On me change la nuit en jour, et on fait que la lumière se trouve proche des ténèbres !
\VS{13}Certes je n'ai plus à attendre que le sépulcre, qui va être ma maison ; j'ai dressé mon lit dans les ténèbres ;
\VS{14}J'ai crié à la fosse : tu es mon père ; et aux vers : vous êtes ma mère et ma Soeur. 
\VS{15}Où est donc mon espérance? Et mon espérance, qui pourrait la voir? \VS{16}Elle descendra au fond du sépulcre ; certes elle reposera avec moi dans la poussière. 
\Chap{18}
\TextTitle{Deuxième discours de Bildad}
\VerseOne{}Bildad de Schuach prit la parole et dit :
\VS{2}Quand finirez-vous ces discours ? Ecoutez, et puis nous parlerons.
\VS{3}Pourquoi sommes-nous regardés comme des bêtes, et sommes-nous stupides à vos yeux?
\VS{4}Ô toi qui déchires ton âme dans ta colère, la terre sera-t-elle abandonnée à cause de toi, et le rocher sera-t-il transporté de sa place ? 
\VS{5}Certainement, la lumière du méchant s'éteindra, et la flamme de son feu ne brillera pas\FTNT{Ps. 37:9-10.}.
\VS{6}La lumière sera ténèbres dans sa tente, et sa lampe sera éteinte au-dessus de lui. 
\VS{7}Les pas de sa force seront resserrés, et son propre conseil le renversera.
\VS{8}Car il est poussé dans le filet par ses propres pieds ; et il marche sur les mailles du filet.
\VS{9}Le piège le prend par le talon, et le filet s'empare de lui;
\VS{10}la corde est cachée dans la terre, et la trappe est sur son sentier.
\VS{11}Les terreurs l'assiègent de tous côtés, et le font courir ses pieds çà et là.\FTNT{Jé. 6:25 ; Jé. 46:5 ; Jé. 49:29.}.
\VS{12}Sa vigueur sera affamée, la détresse est à ses côtés.
\VS{13}Il dévorera les membres de son corps, il dévorera ses membres, le premier-né de la mort ! 
\VS{14} Les choses en quoi il mettait sa confiance seront arrachées de sa tente, et il sera conduit vers le Roi des épouvantements. 
\VS{15}On habitera dans sa tente, qui ne sera plus à lui; le soufre sera répandu sur sa demeure. 
\VS{16}Ses racines sèchent au dessous, et ses branches sont coupées en haut. 
\VS{17}Sa mémoire périt sur la terre, et on ne parle plus de son nom dans les places \FTNT{Ps. 109:13 ; Pr. 10:7.}.
\VS{18}Il est chassé de la lumière dans les ténèbres, et il est exterminé du monde. 
\VS{19}Il n'a ni lignée, ni descendance au milieu de son peuple, ni survivant dans ses habitations. \FTNT{Es. 14:20-22 ; Jé. 22:30 ; Ps. 37:28 ; Ps. 109:13.}. 
\VS{20}Ceux qui seront venus après lui, seront étonnés de sa ruine ; et ceux qui auront été avant lui en seront saisis d'horreur. 
\VS{21}Tel est le sort de l'injuste. Telle est la destinée de celui qui ne connaît pas Dieu. 
\Chap{19}
\TextTitle{Réponse de Job}
\VerseOne{}Job prit la parole et dit :
\VS{2}Jusqu'à quand affligerez-vous mon âme, et m'accablerez-vous de paroles ?
\VS{3} Voilà déjà dix fois que vous m'outragez: vous n'avez pas honte de me maltraiter? 
\VS{4}Vraiment si j'ai failli, ma faute demeure avec moi. 
\VS{5}Si réellement vous voulez vous élever contre moi et faire valoir mon opprobre contre moi, 
\VS{6}Sachez donc que c'est Dieu qui me renverse, et qui tend son filet autour de moi.
\VS{7}Voici je crie pour la violence qui m'est faite, et je ne suis pas exaucé ; je m'écrie, et il n'y a point de justice !
\VS{8}Il a fermé mon chemin, et je ne puis passer; il a mis des ténèbres sur mes sentiers. 
\VS{9}Il m'a dépouillé de ma gloire, il a ôté la couronne de ma tête.
\VS{10}Il m'a détruit de tous côtés, et je m'en vais ; il a arraché mon espérance comme un arbre.
\VS{11}Il s'est enflammé de colère contre moi, et m'a traité comme un de ses ennemis\FTNT{La. 2:5.}.
\VS{12}Ses troupes sont venues ensemble, et elles ont dressé leur chemin contre moi, et se sont campées autour de ma tente\FTNT{La. 2:22.}.
\VS{13}Il a éloigné de moi mes frères, et ceux qui me connaissaient se sont écartés comme des étrangers\FTNT{Ps. 88:9.} ;
\VS{14}mes proches m'ont abandonné, et ceux que je connaissais m'ont oublié.
\VS{15}Ceux qui séjournent dans ma maison et mes servantes m'ont traité comme un étranger; je suis devenu un inconnu pour eux. 
\VS{16}J'appelle mon serviteur, il ne me répond; de ma propre bouche, je le supplie en vain. 
\VS{17}Mon haleine est devenue dégoûtante à ma femme, et ma plainte aux fils de mes entrailles.
\VS{18}Je suis méprisé même par des enfants ; si je me lève, ils parlent contre moi.
\VS{19}Ceux que j'avais pour confidents m'ont en horreur, ceux que j'aimais se sont tournés contre moi\FTNT{Ps. 55:13-14.}.
\VS{20}Mes os sont attachés à ma peau et à ma chair ; et je me suis échappé avec la peau de mes dents\FTNT{La. 4:8.}.
\VS{21}Ayez pitié, ayez pitié de moi, vous, mes amis ! Car la main de Dieu m'a frappé.
\VS{22}Pourquoi, comme Dieu, me poursuivez-vous et n'êtes-vous pas rassasiés de ma chair \FTNT{Ps. 27:2.} ?
\VS{23}Oh! je voudrais que mes paroles fussent écrites quelque part, je voudrais qu'elles fussent inscrites dans un livre; 
\VS{24}Qu'avec un burin de fer et avec du plomb, elles fussent gravées sur le roc, pour toujours... 
\VS{25}Mais je sais que mon rédempteur est vivant, il demeurera le dernier sur la terre.
\VS{26}Et après que cette peau aura été détruite, hors de ma chair, je verrai Dieu \FTNT{Ps. 17:15.}.
\VS{27}Je le verrai moi-même, et mes yeux le verront, et non un autre. Mes reins se consument dans mon sein. 
\VS{28}Vous direz : Comment le poursuivrons-nous, et trouverons-nous en lui la cause de son malheur? 
\VS{29}Ayez peur de l'épée ; car la fureur [avec laquelle vous me persécutez], est [du nombre] des iniquités qui attirent l'épée ; c'est pourquoi sachez qu'il y a un jugement. 
\Chap{20}
\TextTitle{Dernier discours de Tsophar}
\VerseOne{}Tsophar de Naama prit la parole et dit :
\VS{2}C'est à cause de cela que mes pensées diverses me poussent à répondre, et que cette promptitude est en moi. 
\VS{3}J'ai entendu la correction dont tu veux me faire honte, mais mon esprit tirera de mon intelligence la réponse pour moi. 
\VS{4}Ne sais-tu pas que, de tout temps, depuis que Dieu a mis l'homme sur la terre, 
\VS{5}Le triomphe des méchants est de peu de durée, et la joie de l'hypocrite n'est que pour un moment \FTNT{Ps. 37:35-36.} ?
\VS{6}Quand son élévation monterait jusqu'aux cieux, et que sa tête atteindrait les nues,
\VS{7}il périra pour toujours comme ses ordures, et ceux qui le voyaient diront : Où est-il ?
\VS{8}Il s'envolera comme un songe, et on ne le trouvera plus ; il se  retirera comme une vision nocturne\FTNT{Ps. 73:19-20.} ;
\VS{9}l'œil qui le regardait ne le regardera plus, le lieu qu'il habitait ne le contemplera plus.
\VS{10}Ses fils rechercheront la faveur des pauvres, et ses mains restitueront ce que sa violence a ravi\FTNT{Ps. 109:10.}.
\VS{11}Ses os seront pleins de la punition, à  cause des péchés de sa jeunesse, et elle reposera avec lui dans la poussière.
\VS{12}Le mal était doux à sa bouche, il le cachait sous sa langue,
\VS{13}s'il l'épargne, et ne le rejette point, mais le retient dans son palais ; 
\VS{14}Ce qu'il mangera se changera dans ses entrailles en un fiel d'aspic.
\VS{15}Il a englouti des richesses, il les vomira ; Dieu les arrachera de son ventre.
\VS{16}Il a sucé du venin d'aspic, la langue de la vipère le tuera.
\VS{17}Il ne verra plus les ruisseaux, les fleuves, les torrents de miel et de lait.
\VS{18}Il rendra le fruit de son travail, et ne l'avalera pas; il restituera à proportion de ce qu'il aura amassé, et ne s'en réjouira pas\FTNT{So. 2:10.}.
\VS{19}Car il a opprimé, délaissé les pauvres; il a pillé des maisons et ne les a pas rebâties.
\VS{20}Certainement il ne sentira pas dans son ventre la satisfaction de son avidité, et il ne sauvera rien de ce qu'il aura tant convoité \FTNT{Ec. 5:12.}.
\VS{21}Rien n'échappait à sa voracité, mais son bonheur ne durera pas.
\VS{22}Après que la mesure de ses biens aura été remplie, il sera dans la misère ; toutes les mains de ceux qu'il aura opprimés se jetteront sur lui.
\VS{23}Il arrivera que pour lui remplir le ventre, Dieu enverra contre lui l'ardeur de sa colère; il la fera pleuvoir sur lui et entrer dans sa chair.
\VS{24}S’il s’enfuit de devant les armes de fer, l’arc d’airain le transpercera.
\VS{25}Il arrachera la flèche, et elle sortira de son corps, et le fer étincelant, de son foie; les frayeurs de la mort viendront sur lui.
\VS{26}Toutes les ténèbres sont renfermées dans ses demeures les plus secrètes ; un feu qu'on n'aura point soufflé, le consumera ; l'homme qui restera dans sa tente sera malheureux\FTNT{Ps. 12:6.}.
\VS{27}Les cieux découvriront son iniquité, et la terre s'élèvera contre lui. 
\VS{28}Le revenu de sa maison sera emporté. Tout s'écoulera au jour de la colère de Dieu.
\VS{29}C'est là la portion que Dieu réserve à l'homme méchant, et l'héritage qu'il aura de Dieu pour ses discours.
\Chap{21}
\TextTitle{Réponse de Job}
\VerseOne{}Job répondit, et dit :
\VS{2}Ecoutez, écoutez mes discours, donnez-moi seulement cette consolation.
\VS{3}Supportez-moi, et je parlerai ; et quand j'aurai parlé, tu pourras te moquer.
\VS{4}Mais est-ce contre un homme que s'adresse ma plainte ? Et pourquoi mon âme ne serait-elle pas impatiente ?
\VS{5}Regardez-moi, soyez étonnés, et mettez la main sur la bouche.
\VS{6}Quand j'y pense, cela m'épouvante, et un frisson saisit mon corps.
\VS{7}Pourquoi les méchants vivent-ils, vieillissent-ils, et croissent-ils en puissance\FTNT{Jé. 12:1 ; Ha. 1:3 ; Mal. 3:14-15.}?
\VS{8}Leur postérité s'établit avec eux et en leur présence, leurs rejetons prospèrent sous leurs yeux.
\VS{9}Dans leurs maisons règne la paix, loin de la crainte ; la verge de Dieu ne vient pas les frapper.
\VS{10}Leurs taureaux sont féconds, leurs génisses conçoivent et n'avortent pas\FTNT{Ps. 144:13-14.}.
\VS{11}Ils laissent courir leurs enfants comme un troupeau, et les enfants prennent leurs ébats.
\VS{12}Ils chantent au son du tambourin et de la harpe, ils se réjouissent au son du chalumeau.
\VS{13}Ils passent leurs jours dans le bonheur, et ils descendent en un instant au scheol.
\VS{14}Ils disaient pourtant à Dieu : Éloigne-toi de nous, nous ne voulons pas connaître tes voies.
\VS{15}Qu'est-ce que le Tout-Puissant pour que nous le servions ? Que gagnerions-nous à lui adresser nos prières\FTNT{Ex. 5:2.} ?
\VS{16}Quoi donc ! Ne sont-ils pas en possession du bonheur entre leurs mains ? Loin de moi le conseil des méchants\FTNT{Ps. 1:1-2.} !
\VS{17}Mais arrive-t-il que la lampe des méchants s'éteigne, que la ruine vienne sur eux, que Dieu leur distribue leur part dans sa colère\FTNT{Ps. 11:5-6 ; Pr. 13:9.},
\VS{18}qu'ils soient comme la paille face au vent, comme la balle enlevée par le tourbillon\FTNT{Ps. 1:4.}?
\VS{19}Dieu réservera-t-il aux enfants du méchant la punition de ses violences? Il la leur rendra, et il le connaîtra!
\VS{20}Il verra de ses propres yeux sa ruine, c'est lui qui devrait boire la colère du Tout-Puissant\FTNT{Es. 51:17-22 ; Jé. 25:15 ; Ez. 23:31-32 ; Ap. 14:10.}.
\VS{21}Car que lui importe sa maison après lui, quand le nombre de ses mois est achevé ?
\VS{22}Enseignerait-on la science à Dieu, lui qui juge les esprits élevés\FTNT{Ro. 11:34 ; 1 Co. 2:16.} ?
\VS{23}L'un meurt au sein du bien-être, tout à son aise et en joie,
\VS{24}les flancs chargés de graisse, et ses os comme abreuvés de mœlle ;
\VS{25}l'autre meurt l'amertume dans l'âme, n'ayant jamais mangé ce qui est bon.
\VS{26}Et tous deux se couchent dans la poussière, tous deux deviennent couverts de vers.
\VS{27}Je sais bien quelles sont vos pensées, quels jugements iniques vous portez sur moi.
\VS{28}Vous dites : Où est la maison de l'homme puissant ? Où est la tente, demeure des méchants ?
\VS{29}Mais quoi ! N'avez-vous pas interrogé les voyageurs, et n'avez-vous pas appris par les rapports qu'ils vous on faits ?
\VS{30}Au jour du malheur, le méchant est épargné ; au jour de la colère, il échappe\FTNT{Pr. 16:4 ; Ec. 9:12.}.
\VS{31}Qui lui dit en face sa conduite ? Qui lui rend ce qu'il a fait ?
\VS{32}Il est porté au tombeau, et il veille encore sur sa tombe.
\VS{33}Les mottes de la vallée lui sont légères ; et tous après lui suivront la même voie, comme une multitude l'a déjà suivie.
\VS{34}Pourquoi donc m'offrir de vaines consolations ? Ce qui reste de vos réponses n'est que transgression.
\Chap{22}
\TextTitle{Dernier discours d'Eliphaz}
\VerseOne{}Eliphaz de Théman prit la parole et dit :
\VS{2}Un homme peut-il être utile à Dieu ? Mais le sage n'est utile qu'à lui-même.
\VS{3}Si tu es juste, est-ce à l'avantage du Tout-Puissant ? Si tu es intègre dans tes voies, qu'y gagne-t-il ?
\VS{4}Est-ce par crainte de toi qu'il te reprend, qu'il entre en jugement avec toi?
\VS{5}Ta méchanceté n'est-elle pas grande ? Tes iniquités ne sont-elles pas sans fin ?
\VS{6}Tu a pris sans raison le gage de tes frères, tu privais de leurs vêtements ceux qui étaient nus\FTNT{Ex. 22:21.} ;
\VS{7}tu ne donnais pas d'eau à boire à l'homme altéré, tu refusais du pain à l'homme affamé.
\VS{8}Le pays était à l'homme le plus fort, et le puissant s'y établissait.
\VS{9}Tu renvoyais les veuves à vide, les bras des orphelins étaient brisés.
\VS{10}C'est pour cela que tu es entouré de pièges, et que la terreur t'a saisi tout à coup.
\VS{11}Ne vois-tu donc pas ces ténèbres, ces eaux débordées qui te couvrent ?
\VS{12}Dieu n'est-il pas là-haut dans les cieux ? Regarde le sommet des étoiles, comme il est élevé !
\VS{13}Et tu dis : Qu'est-ce que Dieu connaît ? Peut-il juger à travers l'obscurité\FTNT{So. 1:12 ; Ps. 10:11-13 ; Ps. 94:7.} ?
\VS{14}Les nuées l'enveloppent, et il ne voit rien ; il ne parcourt que la voûte des cieux.
\VS{15}Eh quoi ! N'as-tu pas pris garde à l'ancienne route qu'ont suivie les hommes d'iniquité ?
\VS{16}Ils ont été emportés avant le temps, ils ont eu la durée d'un torrent qui s'écoule.
\VS{17}Ils disaient à Dieu : Éloigne-toi de nous ; que peut faire pour nous le Tout-Puissant ?
\VS{18}Dieu cependant avait rempli leurs maisons de biens ! Loin de moi le conseil des méchants !
\VS{19}Les justes le verront, se réjouiront, et l'innocent se moquera d'eux\FTNT{Ps. 107:42.} :
\VS{20}Certainement, notre adversaire a été détruit, le feu a dévoré ce qui en restait\FTNT{Ps. 37:20 ; Ec. 8:12-13.} !
\VS{21}Attache-toi donc à Dieu, et tu auras la paix, tu atteindras ainsi le bonheur.
\VS{22}Reçois de sa bouche l'instruction, et mets ses paroles dans ton cœur\FTNT{Ps. 119:72.}.
\VS{23}Si tu reviens au Tout-Puissant, tu seras rétabli ; si tu éloignes l'iniquité de ta tente.
\VS{24}Jette l'or dans la poussière, l'or d'Ophir parmi les rochers des torrents ;
\VS{25}et le Tout-Puissant sera ton or, ton argent, ta richesse.
\VS{26}Alors tu feras du Tout-Puissant tes délices, tu élèveras vers Dieu ta face ;
\VS{27}tu le prieras, et il t'exaucera, et tu lui rendras tes vœux\FTNT{Ps. 50:14-15.}.
\VS{28}Quand tu prendras des résolutions elles s'accompliront, sur tes sentiers brillera la lumière\FTNT{Ps. 97:11.}.
\VS{29}Quand on aura abaissé quelqu'un et que tu auras dit qu'il soit élevé; alors Dieu délivrera celui qui tenait les yeux abaissés\FTNT{Pr. 29:23.}.
\VS{30}Il délivrera le coupable ; il sera délivré par la pureté de tes mains.
\Chap{23}
\TextTitle{Réponse de Job}
\VerseOne{}Job répondit, et dit :
\VS{2}Maintenant encore ma plainte est une révolte, et pourtant ma main appesantit mes soupirs.
\VS{3}Oh ! Si je savais où le trouver, j'irais jusqu'à son trône,
\VS{4}je disposerais en ordre ma cause devant lui, je remplirais ma bouche d'arguments,
\VS{5}je saurais ce qu'il peut avoir à répondre, je comprendrais ce qu'il peut avoir à me dire.
\VS{6}Contesterait-il avec moi dans la grandeur de sa force? Ne prendrait-il pas le temps de m'écouter ?
\VS{7}Ce serait un homme juste qui argumenterait avec lui, et je serais pour toujours absous par mon juge.
\VS{8}Mais, si je vais à l'orient, il n'y est pas ; si je vais à l'occident, je ne l'aperçois pas ;
\VS{9}est-il occupé au nord, je ne le vois pas ; se cache-t-il au midi, je ne l'aperçois pas.
\VS{10}Il connaît la voie que j'ai suivie ; et s'il m'éprouvait, j'en sortirai pur comme l'or\FTNT{1 Pi. 1:7.}.
\VS{11}Mon pied s'est attaché à ses pas ; j'ai gardé sa voie, et je ne m'en suis pas détourné.
\VS{12}Je ne me suis point aussi écarté du commandement de ses lèvres ; j'ai fait plier ma volonté aux paroles de sa bouche.
\VS{13}Mais s'il a une penbsée, qui l'en détournera? Et ce que mon âme désire, il le fait.
\FTNT{Ps. 115:3 ; Ps. 135:6.}.
\VS{14}Il achèvera donc ce qu'il a résolu sur mon sujet, et il y a encore en lui de telles choses.
\VS{15}C'est pourquoi je suis terrifié à cause de sa présence, et quand je le considère, je suis effrayé devant lui.
\VS{16}Dieu a brisé mon cœur, le Tout-Puissant m'a épouvanté.
\VS{17}Car ce n'est pas la présence des ténèbres qui m'anéantit, ce n'est pas l'obscurité dont ma face est couverte.
\Chap{24}
\VerseOne{}Comment les temps de la vengeance ne seraient-ils pas cachés aux méchants par le Tout-puissant, puisque ceux-mêmes qui le connaissent n'aperçoivent pas les jours de sa punition sur eux ?
\VS{2}On déplace les bornes, on ravit des troupeaux, et on les fait paître\FTNT{De. 19:14 ; De. 27:17 ; Pr. 13:10 ; Pr. 22:28.} ;
\VS{3}Ils emmènent l'âne des orphelins, ils prennent pour gage le boeuf de la veuve;
\VS{4}Ils font retirer les pauvres du chemin, et les misérables du pays sont obligés de se cacher.
\VS{5}Voilà, certains sont comme des ânes sauvages dans le désert ; ils sortent pour faire leur ouvrage, se levant dès le matin pour la proie ; le désert leur fournit du pain pour leurs enfants ;
\VS{6}ils moissonnent le fourrage qui reste dans les champs, ils grappillent dans la vigne de l'impie ;
\VS{7}ils passent la nuit nus, sans vêtements, sans couverture contre le froid\FTNT{Lé. 19:13 ; De. 24:12-13.} ;
\VS{8}ils sont percés par la pluie des montagnes, et ils embrassent les rochers comme unique refuge.
\VS{9}On arrache l'orphelin à la mamelle, on prend des gages sur le pauvre.
\VS{10}Ils font aller sans habits l'homme qu'ils ont dépouillé; et ils enlèvent à ceux qui n'avaient pas de quoi manger, ce qu'ils avaient glâné.
\VS{11}Dans les enclos de l'impie, ils font de l'huile, ils foulent le pressoir à raisin et ils ont soif.
\VS{12}Ils font gémir les gens dans la ville, l'âme de ceux qu'ils ont fait mourir, crient; Dieu ne fait rien d'indigne de lui.
\VS{13}En voici d'autres qui se révoltent contre la lumière, ils n'en connaissent pas les voies, ils ne restent pas sur leurs sentiers.
\VS{14}Le meurtrier se lève au point du jour ; il tue le pauvre et l'indigent, et il dérobe pendant la nuit\FTNT{Ps. 10:8-9.}.
\VS{15}L'œil de l'adultère épie le crépuscule ; aucun œil ne me verra, dit-il, et il met un voile sur le visage\FTNT{Ps. 64:6 ; Pr. 7:7-10.}.
\VS{16}Ils percent durant les ténèbres les maisons, qu'ils avaient marquées le jour, ils haïssent la lumière.\FTNT{Jn. 3:20.}.
\VS{17}Pour eux, le matin c'est l'ombre de la mort ; si quelqu'un les reconnaît, ils ont des terreurs.
\VS{18}Eh quoi ! L'impie est d'un poids léger sur la surface de l'eau, il n'a sur la terre qu'un héritage maudit, il ne prend jamais le chemin des vignes !
\VS{19}Comme la sécheresse et la chaleur absorbent les eaux de la neige, ainsi le scheol engloutit ceux qui pèchent\FTNT{Ps. 49:15.} !
\VS{20}Quoi ! Le sein maternel l'oublie, les vers en font leurs délices, on ne se souvient plus de lui ! L'injuste est brisé comme du bois,
\VS{21}lui qui dépouille la femme stérile et sans enfants, lui qui ne répand aucun bien sur la veuve !…
\VS{22}Non ! Dieu par sa force prolonge les jours des violents, et les voilà s'élever quand ils ne croyaient plus en la vie.
\VS{23}Il leur donne de la sécurité et de la confiance, ses yeux sont sur leurs voies.
\VS{24}Ils se sont élevés ; et en un peu de temps ils ne sont plus, ils s'affaissent, ils meurent en chemin comme tous les hommes, ils sont coupés comme une tête d'épi.
\VS{25}S'il n'en est pas ainsi, qui me fera mentir, qui fera de mes paroles un rien ?
\Chap{25}
\TextTitle{Dernier discours de Bildad}
\VerseOne{}Alors Bildad de Schuach prit la parole et dit: 
\VS{2}Le règne et la terreur sont en la possession de Dieu ; il maintient la paix dans ses hauts lieux.
\VS{3} Ses armées peuvent-elles se compter ? Et sur qui sa lumière ne se lève-t-elle point ?\FTNT{Mt. 5:45.} ?
\VS{4}Et comment l'homme se justifierait-il devant Dieu ? Et comment celui qui est né de femme serait-il pur?
\VS{5}Voilà, qu'on aille jusqu'à la lune, elle ne luira pas ; les étoiles ne sont pas pures devant ses yeux;
\VS{6}Combien moins l'homme qui n'est qu'un ver ; et le fils d'un homme, qui n'est qu'un vermisseau \FTNT{Ps. 22:7.} !
\Chap{26}
\TextTitle{Réponse de Job}
\VerseOne{}Job répondit, et dit :
\VS{2}Comme tu as aidé celui qui était sans force ! Comme tu as secouru le bras sans force !
\VS{3}Quels bons conseils tu donnes à celui qui manque de sagesse ! Tu fais connaître l'abondance de ton intelligence !
\VS{4}A qui s'adressent tes paroles ? Et de qui est l'esprit qui est sorti de toi ?
\VS{5}Devant Dieu les ombres des morts tremblent au-dessous des eaux, et de leurs habitants ;
\VS{6}devant lui le scheol est nu, l'abîme est sans voile\FTNT{Ps. 139:8-12 ; Pr. 15:11 ; Hé. 4:13.}.
\VS{7}Il étend la direction nord sur le vide, il suspend la terre sur le néant.
\VS{8}Il renferme les eaux dans ses nuages, et la nuée n'éclate pas sous leur poids\FTNT{Ps. 104:2-3.}.
\VS{9}Il couvre la face de son trône, il répand sur lui sa nuée.
\VS{10}Il a tracé un cercle à la surface des eaux, comme limite entre la lumière et les ténèbres\FTNT{Ge. 1:9 ; Jé. 5:22 ; Ps. 33:7 ; Ps. 104:9 ; Pr. 8:29.}.
\VS{11}Les colonnes du ciel s'ébranlent et s'étonnent à sa menace.
\VS{12}Par sa force il soulève la mer, par son intelligence il en brise l'orgueil\FTNT{Ps. 89:10.}.
\VS{13}Il a orné les cieux par son Esprit, et  de sa main, il transperce le serpent fuyard.
\VS{14}Ce sont là les bords de ses voies, c'est le discours fait en chuchotant que nous entendons ; mais qui comprendra le tonnerre de sa puissance\FTNT{Ec. 3:10.} ?
\Chap{27}
\VerseOne{}Et Job continuant, reprit son discours sentencieux, et dit :
\VS{2}Dieu, qui met mon droit à l'écart, et le Tout-puissant qui remplit mon âme d'amertume, est vivant.
\VS{3}Aussi longtemps que j'aurai ma respiration et que l'esprit de Dieu sera dans mes narines,
\VS{4}mes lèvres ne prononceront rien d'injuste, et ma langue ne dira pas de chose fausse\FTNT{Es. 33:15 ; Ps. 15:2 ; Ps. 24:4.}.
\VS{5}Loin de moi la pensée de vous reconnaître pour justes ! Tant que je vivrai je n'abandonnerai pas mon intégrité.
\VS{6}Je conserve ma justice, et je ne l'abandonne pas ; et mon cœur ne me reproche rien en mes jours.
\VS{7}Qu'il en soit de mon ennemi comme du méchant ; et de celui qui se lève contre moi, comme de l'injuste !
\VS{8}Quelle espérance reste-t-il à l'hypocrite quand Dieu coupe le fil de sa vie, quand il lui retire son âme\FTNT{Mt. 16:26 ; Lu. 12:20.} ?
\VS{9}Est-ce que Dieu entend ses cris, quand l'angoisse vient sur lui\FTNT{ Es. 1:15 ; Jé. 14:12 ; Ez. 8:18 ; Mi. 3:4 ; Ps. 18:41 ; Pr. 1:28 ; Jn. 9:31 ; Ja. 4:3.} ?
\VS{10}Trouvera-t-il son plaisir dans le Tout-Puissant ? Invoque-t-il Dieu en tout temps ?
\VS{11}Je vous enseignerai comment la main de Dieu agit, je ne vous cacherai pas les desseins du Tout-Puissant.
\VS{12}Voilà, vous avez tous vu ces choses, et pourquoi vous laissez-vous aller à des pensées vaines ?
\VS{13}Voici la part que Dieu réserve à l'homme méchant, l'héritage que les violents reçoivent du Tout-Puissant.
\VS{14}S'il a des fils en grand nombre, c'est pour l'épée, et ses rejetons ne seront pas rassasiés de pain ;
\VS{15}ses survivants sont ensevelis par la peste, et leurs veuves ne les pleurent pas\FTNT{Ps. 78:64.}.
\VS{16}Parce qu’il entasse l'argent comme la poussière, et qu'il entasse des habits comme on amasse de la boue,
\VS{17}le riche tombe, et il n’est pas relevé ; il ouvre ses yeux, et il ne trouve rien.
\VS{18}Il se bâtit une maison comme celle de la teigne, comme la cabane que fait un gardien\FTNT{Ps. 49:18.}.
\VS{19}Il se couche riche, et il périt dépouillé ; il ouvre les yeux, et tout a disparu.
\VS{20}Les frayeurs l'atteignent comme des eaux ; le tourbillon l'enlève de nuit.
\VS{21}Le vent d'orient l'emporte, et il s'en va ; il l'arrache de sa demeure comme un tourbillon.
\VS{22}Dieu le précipite à terre et ne l'épargne pas, et le méchant voudrait fuir devant sa main.
\VS{23}On applaudit à sa chute, et on le siffle au lieu où il se tient.
\Chap{28}
\VerseOne{}Certainement l'argent a sa mine, et l'or a un lieu d'où on le tire pour l'affiner ;
\VS{2}le fer se tire de la poussière, et la pierre étant fondue rend de l'airain.
\VS{3}Il met fin aux ténèbres, de sorte qu'on découvre le bout de toutes choses, même les pierres les plus cachées, et qui sont dans l'ombre de la mort.
\VS{4}Le torrent se déborde d'auprès d'un lieu habité, se jette dans des lieux où l'on ne met plus le pied, mais ses eaux se tarissent et s'écoulent par le travail des hommes.
\VS{5}La terre, d'où sort le pain, est bouleversée dans ses entrailles comme par le feu.
\VS{6}Ses pierres sont la demeure du saphir, et l'on y trouve de la poudre d'or.
\VS{7}L'oiseau de proie n'en connaît pas le chemin, l'œil du vautour ne l'aperçoit pas ;
\VS{8}les plus jeunes et fiers animaux n'y ont pas marché, le lion n'y a jamais passé.
\VS{9}L'homme avance sa main sur le roc, il renverse les montagnes depuis la racine ;
\VS{10}il fend des tranchées dans les rochers, et son œil voit tout ce qu'il y a de précieux ;
\VS{11}il arrête l'écoulement des eaux, et il fait sortir ce qui est caché.
\VS{12}Mais la sagesse, où se trouve-t-elle ? Où est le lieu où se tient l'intelligence ?
\VS{13}L'homme ne connaît pas sa valeur, et elle ne se trouve pas dans la terre des vivants.
\VS{14}L'abîme dit : Elle n'est pas en moi ; et la mer dit : Elle n'est pas avec moi.
\VS{15}Elle ne se donne pas contre de l'or pur, elle ne s'achète pas au poids de l'argent\FTNT{Pr. 3:14 ; Pr. 8:11 ; Pr. 16:16.} ;
\VS{16}On ne l'échange point avec l'or d'Ophir, ni avec l'Onyx précieux, ni avec le Saphir.
\VS{17}L'or ni le diamant n'approchent pas de son prix, et on ne la donnera pas en échange pour un vase de fin or.
\VS{18}On ne se souvient ni du corail ni du cristal auprès d'elle : La sagesse vaut plus que les perles.
\VS{19}On ne la compare pas avec la topaze d'Ethiopie ; on ne la met pas en balance avec l'or pur.
\VS{20}D'où vient donc la sagesse ? Où est la demeure de l'intelligence ?
\VS{21}Elle est couverte aux yeux de tout homme vivant, et elle est cachée aux oiseaux des cieux.
\VS{22}Le gouffre et la mort disent : Nous avons entendu de nos oreilles parler d'elle.
\VS{23}C'est Dieu qui en sait le chemin, et qui sait où elle est ;
\VS{24}car il regarde jusqu'aux extrémités de la terre, il voit tout sous les cieux\FTNT{Ps. 14:2 ; Ps. 33:13-14 ; Ps. 102:20.}.
\VS{25}Quand il façonna le poids du vent, et qu'il estima la mesure des eaux\FTNT{Pr. 8:29.},
\VS{26}quand il ordonna des lois à la pluie, et qu'il fit un chemin à l'éclair et au tonnerre,
\VS{27}alors il vit la sagesse et la manifesta ; il l'établit et la sonda.
\VS{28}Puis il dit à l'homme : Voici, la crainte du Seigneur est la sagesse, et se détourner du mal c'est l'intelligence\FTNT{De. 4:6 ; Jé. 9:24 ; Ps. 111:10 ; Pr. 1:7 ; Pr. 9:10 ; Ec. 12:15.}.
\Chap{29}
\TextTitle{La postérité passée de Job}
\VerseOne{} Et Job continuant, reprit son discours sentencieux, et dit :
\VS{2}Oh ! qui me ferait être comme j'étais autrefois, comme j'étais en ces jours où Dieu me gardait,
\VS{3}Quand il faisait luire sa lampe sur ma tête, et quand je marchais parmi les ténèbres, éclairé par sa lumière,
\VS{4}Comme j'étais aux jours de mon automne, lorsque le secret de Dieu était dans ma tente ;
\VS{5}Quand le Tout-puissant était encore avec moi, et mes gens autour de moi.
\VS{6}quand je lavais mes pieds dans le lait, et que le rocher répandait près de moi des torrents d'huile\FTNT{De. 32:13.} !
\VS{7}Quand je sortais vers la porte passant par la ville, et que je me faisais préparer un siège dans la place,
\VS{8}les jeunes gens me voyant se cachaient, les vieillards se levaient, et se tenaient debout.
\VS{9}Les princes s'abstenaient de parler, et mettaient la main sur leur bouche ;
\VS{10}Les Conducteurs retenaient leur voix, et leur langue était attachée à leur palais.
\VS{11}L'oreille qui m'entendait, disait que j'étais bienheureux, et l'oeil qui me voyait, déposait en ma faveur ;
\VS{12}car je délivrais l'affligé qui criait au secours, et l'orphelin qui n'avait personne pour le secourir\FTNT{Ps. 72:12 ; Pr. 21:13.}.
\VS{13}La bénédiction de celui qui s'en allait périr, venait sur moi, et je faisais que le coeur de la veuve chantait de joie.
\VS{14}J'étais revêtu de la justice, elle me servait de vêtement, et mon équité m'était comme un manteau, et comme une tiare\FTNT{Es. 59:17 ; 1 Th. 5:8 ; Ep. 6:14-17.}.
\VS{15}J'étais les yeux de l'aveugle et les pieds du boiteux.
\VS{16}J'étais le père des pauvres, et je m'informais diligemment de la cause qui ne m'était point connue.\FTNT{Pr. 29:7.} ;
\VS{17}je cassais les grosses dents de l'injuste, et je lui arrachais la proie d'entre ses dents. \FTNT{Ps. 58:7.}.
\VS{18}C'est pourquoi je disais : Je mourrai dans mon lit, et je multiplierai mes jours comme les grains de sable ;
\VS{19}ma racine était ouverte aux eaux, et la rosée demeurait toute la nuit sur mes branches.\FTNT{Jé. 17:5-8 ; Ps. 1:3.} ;
\VS{20}ma gloire se renouvellera sans cesse en moi, et mon arc se renouvellera dans ma main.
\VS{21}On m'écoutait, et on attendait que j'eusse parlé ; et lorsque j'avais dit mon avis, on se tenait dans le silence.
\VS{22}Après mes discours, nul ne répliquait, et ma parole était pour tous une bienfaisante rosée ;
\VS{23}ils m'attendaient comme on attend la pluie ; ils ouvraient leur bouche, comme après la pluie de la dernière saison.
\VS{24}Riais-je avec eux ? ils ne le croyaient point ; et ils ne faisaient point disparaître la sérénité de mon visage.
\VS{25}Voulais-je aller avec eux ? j'étais assis au haut bout, j'étais entre eux comme un Roi dans son armée, et comme un homme qui console les affligés.
\Chap{30}
\TextTitle{Son humiliation}
\VerseOne{}Mais maintenant ceux qui sont plus jeunes que moi, se moquent de moi ; ceux-là même dont je n'aurais pas daigné mettre les pères avec les chiens de mon troupeau.
\VS{2}Et en effet, à quoi m'aurait servi la force de leurs mains ? La vieillesse était périe en eux. 
\VS{3}De disette et de faim ils se tiennent à l'écart, fuyant dans les lieux arides, ténébreux, désolés, et déserts;
\VS{4}Ils coupent des herbes sauvages auprès des arbrisseaux, et la racine des genévriers pour se chauffer. 
\VS{5}Ils sont chassés d'entre les hommes, et on crie après eux comme après un larron. 
\VS{6}Ils habitent dans les creux des torrents, dans les trous de la terre et des rochers ;
\VS{7}Ils font du bruit entre les arbrisseaux, et ils s'attroupent entre les chardons. 
\VS{8}Ce sont des hommes de néant et sans nom, abaissés plus bas que la terre. 
\VS{9}Et maintenant je suis le sujet de leur chanson, et la matière de leur entretien.\FTNT{Ps. 69:12 ; La. 3:14.}.
\VS{10}Ils m'ont en abomination ; ils se tiennent loin de moi ; et ils ne craignent pas de me cracher au visage. 
\VS{11}Parce que Dieu a détendu ma corde, et m'a affligé, ils ont secoué le frein devant moi. 
\VS{12}De jeunes gens, nouvellement nés, se placent à ma droite ; ils poussent mes pieds, et je suis en butte à leur malice\FTNT{Ps. 35:15.} ;
\VS{13}Ils ruinent mon sentier, ils augmentent mon affliction, sans qu'ils aient besoin que personne les aide;
\VS{14}Ils viennent contre moi comme par une brèche large, et ils se sont jetés sur moi à cause de ma désolation. 
\VS{15}Les frayeurs se sont tournées vers moi, [et] comme un vent elles poursuivent mon âme ; et ma délivrance s'est dissipée comme une nuée\FTNT{Os. 13:3.}.
\VS{16}C'est pourquoi maintenant mon âme se fond en moi ; les jours d'affliction m'ont atteint. 
\VS{17}Il me perce de nuit les os, et mes artères n'ont point de relâche. 
\VS{18}Il change mon vêtement par la grandeur de sa force, et il me serre de près, comme fait l'ouverture de ma tunique.
\VS{19}Il m'a jeté dans la boue, et je ressemble à la poussière et à la cendre. 
\VS{20}Je crie à toi, et tu ne m'exauces point ; je me tiens debout, et tu ne me regardes point. 
\VS{21}Tu es pour moi sans compassion, tu me traites en ennemi par la force de ta main. 
\VS{22}Tu m'élèves comme sur le vent, et tu m'y fais monter comme sur un chariot, et puis tu fais fondre toute ma substance. 
\VS{23}Je sais donc que tu me conduis à la mort et dans la maison assignée à tous les vivants\FTNT{Hé. 9:27.}.
\VS{24}Mais il n'étendra pas sa main jusqu'au sépulcre. Quand il les aura tués, crieront-ils ? 
\VS{25}Ne pleurais-je pas pour l'amour de celui qui passait de mauvais jours ; et mon âme n'était-elle pas affligée à cause du pauvre\FTNT{Ro. 12:15.} ?
\VS{26}Cependant lorsque j'attendais le bien, le mal m'est arrivé ; et quand j'espérais la clarté, les ténèbres sont venues. 
\VS{27}Mes entrailles sont dans une grande agitation, et ne peuvent se calmer ; les jours d'affliction m'ont prévenu. 
\VS{28}Je marche tout noirci, mais non pas du soleil ; je me lève, je crie en pleine assemblée. 
\VS{29}Je suis devenu le frère des dragons, et le compagnon des hiboux\FTNT{Ps. 102:7-8.}.
\VS{30}Ma peau est devenue noire sur moi, et mes os sont desséchés par l'ardeur qui me consume\FTNT{La. 4:8 ; La. 5:10.}.
\VS{31}C'est pourquoi ma harpe s'est changée en lamentations, et mes orgues en des sons lugubres.
\Chap{31}
\TextTitle{Job se justifie}
\VerseOne{}J'avais fait une alliance avec mes yeux, et je n'aurais pas regardé une vierge.
\VS{2}Quelle part Dieu m'eût-il réservée d'en haut ? Quel héritage le Tout-Puissant m'aurait-il envoyé des cieux ?
\VS{3}La perdition n'est-elle pas pour l'injuste, et les accidents étranges pour les ouvriers d'iniquité ?
\VS{4}Dieu ne voit-il pas mes voies ? Ne compte-t-il pas tous mes pas\FTNT{Pr. 5:21 ; Pr. 15:3 ; 2 Ch. 16:9.} ?
\VS{5}Si j'ai marché dans le mensonge, si mon pied s'est hâté pour tromper,
\VS{6}qu'on me pèse dans des balances justes, et Dieu connaîtra mon intégrité.
\VS{7}Si mes pas se sont détournés du droit chemin, et si mon cœur a marché après mes yeux, et si quelque tache s'est attachée à mes mains,
\VS{8}Que je sème et qu'un autre mange, et tout ce que j'aurais fait produire soit déraciné!
\VS{9}Si mon cœur a été séduit après quelque femme, et si j'ai demeuré en embûche à la porte de mon prochain\FTNT{Pr. 7.},
\VS{10}que ma femme broie le grain pour un autre, et que d'autres se penchent sur elle !
\VS{11}Car c'est un crime, une iniquité punie par les juges ;
\VS{12}c'est un feu qui dévore jusqu'à la destruction, et qui aurait détruit toutes mes récoltes dans leur racine.
\VS{13}Si j'ai refusé de faire droit à mon serviteur ou à ma servante, quand ils ont contesté avec moi ;
\VS{14}Car qu'eussé-je fait, quand le [Dieu] Fort se fût levé ? et quand il m'en eût demandé compte, que lui aurais-je répondu ?
\VS{15}Celui qui m'a fait dans le ventre de ma mère ne l'a-t-il pas fait aussi ? Un même Dieu ne nous a-t-il pas formés dans le sein maternel\FTNT{Pr. 14:31 ; Pr. 17:5.} ?
\VS{16}Si j'ai refusé aux pauvres leur désir, si j'ai laissé se consumer les yeux de la veuve\FTNT{Es. 10:2 ; Lu. 18:2-3.},
\VS{17}si j'ai mangé seul mon morceau de pain, sans que l'orphelin en ait sa part,
\VS{18}moi qui l'ai dès ma jeunesse fait grandir près de moi comme un père, et qui dès le sein de ma mère, ai été le guide de la veuve ;
\VS{19}si j'ai vu le malheureux périr faute de vêtements, le pauvre manquer de couverture\FTNT{Mt. 25:41-45.},
\VS{20}Si ses reins ne m'ont point béni, et s'il n'a pas été échauffé de la laine de mes agneaux ;
\VS{21}Si j'ai levé la main contre l'orphelin, quand j'ai vu à la porte, que je pouvais l'aider\FTNT{Pr. 22:22.} ;
\VS{22}Que l'os de mon épaule tombe et que mon bras soit cassé, et séparé de l'os auquel il est joint !
\VS{23}Car j'ai eu frayeur de l'orage du [Dieu] Fort, et je ne saurais [subsister] devant sa majesté. 
\VS{24}Si j'ai mis mon espérance en l'or, et si j'ai dit au fin or : Tu es ma confiance\FTNT{Mc. 10:24 ; 1 Ti. 6:17.} ;
\VS{25}si je me suis réjoui de ce que mes biens étaient multipliés, et de ce que ma main en avait trouvé abondamment\FTNT{Ps. 62:11.} ;
\VS{26}Si j'ai regardé le soleil lorsqu'il brillait le plus, et la lune marchant noblement ;
\VS{27}et si mon cœur a été séduit en secret, et si ma main a baisé ma bouche ; 
\VS{28}Ce qui est aussi une iniquité toute jugée ; car j'eusse renié le Dieu d'en haut.
\VS{29}Si je me suis réjoui du malheur de mon ennemi, si j'ai sauté d'allégresse quand le mal l'a atteint\FTNT{Mt. 5:43-44.},
\VS{30}moi qui n'ai pas permis à ma langue de pécher en demandant sa mort par des malédictions ;
\VS{31}Et les gens de ma maison n'ont pas dit : Qui nous donnera de sa chair ? Nous n'en saurions être rassasiés.\FTNT{Ps. 27:2.}.
\VS{32}L'étranger n'a pas passé la nuit dehors ; j'ai ouvert ma porte au passant\FTNT{Ge. 19:1-2 ; De. 10:19 ; 1 Pi. 4:9 ; Hé. 13:2.} ;
\VS{33}si j'ai caché mon péché comme Adam, pour couvrir mon iniquité en me flattant\FTNT{Ge. 3:10-12 ; Pr. 28:13.},
\VS{34}Quoique je pusse me faire craindre à une grande multitude, toutefois le moindre qui fût dans les familles m'inspirait de la crainte, et je me tenais dans le silence, et ne sortais point de la porte.
\VS{35}Ô ! S'il y avait quelqu'un qui voulût m'entendre. Tout mon désir est que le Tout-puissant me réponde, et que ma partie adverse fasse un écrit contre moi.
\VS{36}Si je ne le porte sur mon épaule, et si je ne l'attache comme une couronne ;
\VS{37}je lui raconterais tous mes pas, je m'approcherais de lui comme d'un Prince.
\VS{38}Si ma terre crie contre moi, et si ses sillons pleurent;
\VS{39}si j'ai mangé son fruit sans argent ; si j'ai tourmenté l'esprit de ceux qui la possédaient;
\VS{40}qu'elle me produise des épines au lieu de blé, et de l'ivraie au lieu d'orge. C'est ici la fin des paroles de Job.
\Chap{32}
\TextTitle{Discours d'Elihu : reproches à Job et à ses amis}
\VerseOne{}Et ces trois hommes cessèrent de répondre à Job, parce qu'il était juste à ses propres yeux. 
\VS{2}Alors s'enflamma la colère d'Élihu, fils de Barakeël, le Buzite, de la famille de Ram : sa colère s'enflamma contre Job, parce qu'il se justifiait lui-même plutôt que Dieu ; 
\VS{3}et sa colère s'enflamma contre ses trois amis, parce qu'ils ne trouvaient pas de réponse et que néanmoins ils condamnaient Job. 
\VS{4}Mais voyant qu'il n'y avait pas de réponse dans la bouche des trois hommes, la colère d'Elihu s'enflamma.
\VS{5}Mais, voyant que ces trois hommes n'avaient plus aucune réponse à la bouche, Elihu se mit en colère.
\VS{6}Élihu, fils de Barakeël, le Buzite, répondit, en disant : Moi, je suis jeune, et vous êtes des vieillards ; c'est pourquoi je redoutais et je craignais de vous faire connaître ce que je sais. 
\VS{7}Je disais: les jours parleront, et le grand nombre des années fera connaître la sagesse.
\VS{8}L'esprit dans l'homme, c'est l'esprit, le souffle du Tout-Puissant qui rend intelligent\FTNT{Da. 1:17 ; Da. 2:21 ; Pr. 2:6 ; Ec. 2:26.} ;
\VS{9}ce ne sont pas les aînés qui sont sages, ce ne sont pas les vieillards qui comprennent ce qui est juste.
\VS{10}C'est pourquoi je dis : Ecoute-moi ! Et je dirai aussi ma pensée.
\VS{11}J'ai attendu la fin de vos discours, j'ai écouté vos raisonnements, jusqu'à ce que vous ayez bien examiné les discours de Job.
\VS{12}J'ai pris le soin de vous écouter ; et voici, aucun de vous n'a convaincu Job, aucun n'a répondu à ses paroles.
\VS{13}Qu'il ne vous n'arrive pas de dire: nous avons trouvé la sagesse ; c'est Dieu qui le poursuit, et non pas l'homme !
\VS{14}Il n'a pas dirigé ses discours contre moi : Aussi je ne lui répondrai pas à votre manière.
\VS{15}Ils sont étonnés! Ils ne répondent plus rien! On leur a ôté la parole!
\VS{16}J'ai attendu jusqu'à ce qu'ils n'ont plus rien dit, car ils sont demeurés muets, et ils n'ont plus su que répondre.
\VS{17}A mon tour, je veux répondre pour moi, et je veux donner mon avis.
\VS{18}Car je suis rempli de discours, l'esprit qui est en mon sein me presse.
\VS{19}Mon sein est comme vin sans air, comme des outres neuves qui vont éclater\FTNT{Mt. 9:17 ; Mc. 2:22 ; Lu. 5:38.}.
\VS{20}Je parlerai pour respirer à l'aise, j'ouvrirai mes lèvres et je répondrai.
\VS{21}Je ne ferai pas acception de personnes, et je flatterai aucun homme.
\VS{22}Car je ne sais pas flatter : Mon Créateur m'enlèverait bien vite.
\Chap{33}
\TextTitle{Discours d'Elihu : la justice de Dieu}
\VerseOne{}C'est pourquoi Job, écoute mon discours, je te prie et prête l'oreille à toutes mes paroles !
\VS{2}Voici, j'ouvre la bouche, ma langue parle dans mon palais.
\VS{3}Mes paroles exprimeront la droiture de mon cœur, mes lèvres diront la vérité pure.
\VS{4}L'Esprit de Dieu m'a fait, et le souffle du Tout-Puissant me donne la vie\FTNT{Ge. 2:7.}.
\VS{5}Si tu peux, réponds-moi, dresse-toi contre moi, demeure ferme!
\VS{6}Voici, je suis pour le Dieu fort, selon que tu en as parlé; j'ai été formé de la terre tout comme toi.\FTNT{Ac. 14:15.} ;
\VS{7}Voici ma terreur ne te trouble pas, et ma main ne s'appesantit pas sur toi
\VS{8}Quoi qu'il en soit, tu as dit, moi l'entendant, j'ai entendu la voix de tes discours:
\VS{9}Je suis pur, sans péché, je suis net, il n'y a pas d'iniquité en moi.
\VS{10}Voici il cherche à rompre avec moi, il me considère comme son ennemi ;
\VS{11}il met mes pieds dans les ceps, il surveille tous mes chemins.
\VS{12}Je te répondrai qu'en cela tu n'as pas été juste, car Dieu sera toujours plus grand que l'homme.
\VS{13}Pourquoi as-tu donc plaidé contre lui ? Car il ne rend pas compte de toutes ses actions.
\VS{14}Car Dieu parle une première fois, et une seconde fois à celui qui n'aura pas pris garde à la première.
\VS{15}Par des songes, par des visions nocturnes, quand les hommes tombent dans un profond sommeil, quand ils dorment sur leur couche.
\VS{16}Alors il ouvre l'oreille de l'homme d'une mauvaise action et de rabaisser la fierté de l'homme.
\VS{17}afin de détourner l'homme de son œuvre et de le préserver de l'orgueil,
\VS{18}il garantit son âme de la fosse, et sa vie de l'épée.
\VS{19}L'homme est aussi châtié par des douleurs sur son lit, à cause d'une lutte perpétuelle en ses os\FTNT{Ps. 38:4.}.
\VS{20}Alors sa vie prend en horreur le pain et son âme les mets les plus désirés\FTNT{Ps. 107:18.} ;
\VS{21}Sa chair est tellement consumée qu'elle paraît plus, ses os sont tellement brisés, qu'on y connaît plus rien;
\VS{22}son âme s'approche de la fosse, et sa vie des messagers de la mort.
\VS{23}Mais s'il y a pour cet homme un messager qui interprète, un d'entre les mille, pour lui annoncer la voie de la droiture,
\VS{24}alors Dieu prend pitié de lui et dit : Garantis-le, afin qu'il ne descende pas dans la fosse ; j'ai trouvé la propitiation!
\VS{25}Sa chair devient plus délicate qu'elle n'était dan son enfance; il revient aux jours de sa jeunesse.
\VS{26}Il supplie Dieu par ses prières, et Dieu lui est favorable, il lui laisse voir sa face avec joie, et lui rend sa justice\FTNT{Es. 58:9.}.
\VS{27}Il regarde vers les hommes et dit : J'ai péché, j'ai violé la justice, et je n'ai pas été puni comme je le méritais ;
\VS{28}Dieu a racheté mon âme afin qu'elle ne passe pas dans la fosse, et ma vie voit encore la lumière !
\VS{29}Voilà ce que Dieu fait, deux fois, trois fois, envers l'homme\FTNT{Ps. 62:11.},
\VS{30}pour ramener son âme de la fosse, pour l'éclairer de la lumière des vivants\FTNT{Ps. 56:14.}.
\VS{31}Sois attentif, Job, écoute-moi ! Tais-toi, et je parlerai !
\VS{32}Si tu as quelque chose à dire, réponds-moi ! Parle, car je désire te justifier.
\VS{33}Sinon, écoute, tais-toi et je t'enseignerai la sagesse.
\Chap{34}
\TextTitle{Discours d'Elihu : il accuse Job de se révolter}
\VerseOne{}Elihu reprit la parole, et dit :
\VS{2}Sages, écoutez mes discours ! Vous qui avez la connaissance, prêtez-moi l'oreille !
\VS{3}Car l'oreille discerne les discours, comme le palais savoure ce qu'il mange.
\VS{4}Choisissons ce qui est juste, voyons entre nous ce qui est bon.
\VS{5}Job dit : Je suis juste, et Dieu a écarté ma justice;
\VS{6}mentirai-je à mon droit? Ma flèche est mortelle sans que j'aie commis de crime.
\VS{7}où y a-t-il un homme comme Job, qui boit le péché la moquerie comme l'eau,
\VS{8}qui marche en compagnie des ouvriers d'iniquité, et qui fréquente  avec les hommes marchant de pair avec les hommes méchants ?
\VS{9}Car il a dit : Il est inutile à l'homme de plaire à Dieu\FTNT{Mal. 3:14.}.
\VS{10} C'est pourquoi écoutez, vous qui avez de l'intelligence, écoutez-moi ! Loin de Dieu la méchanceté, loin du Tout-Puissant l'injustice\FTNT{De. 32:4 ; Ps. 92:16 ; Ro. 9:14.} !
\VS{11}Car il rend à l'homme selon son œuvre, il fait trouver à chacun selon sa voie\FTNT{Jé. 17:10 ; Jé. 32:19 ; Ez. 7:27 ; Pr. 24:12 ; Mt. 16:27 ; Ro. 2:6 ; 2 Co. 5:10 ; Ep. 6:8 ; Ap. 22:12.}.
\VS{12}Certes, Dieu ne commet pas l'injustice ; le Tout-Puissant ne renverse pas le droit.
\VS{13}Qui lui a donné la terre en charge ? Ou Qui a placé la terre habitable?
\VS{14}S'il ne pensait qu'à lui-même, s'il retirait à lui son esprit et son souffle\FTNT{Ps. 104:29.},
\VS{15}toute chair périrait ensemble, et l'homme retournerait dans la poussière\FTNT{Ge. 3:19 ; Ec. 3:20 ; Ec. 12:9.}.
\VS{16}Si donc tu as de l'intelligence, écoute ceci, prête l'oreille à ce que tu entendras de moi.
\VS{17}Comment celui qui n'aimerait pas à faire la justice jugerait-il le monde? Et condamneras-tu celui qui est souverainement juste ?
\VS{18}Dira-t-on à un roi, qu'il est un scélérat ? Et aux princes, qu'ils sont des méchants ?
\VS{19}Combien moins le dira-t-on à celui qui n'a point d'égard à la personne des grands, et qui ne connaît point les riches pour les préférer aux pauvres, parce qu'ils sont tous l'ouvrage de ses mains\FTNT{De. 10:17 ; 2 Ch. 19:7 ; Ac. 10:34 ; Ga. 2:6 ; Ro. 2:11 ; Ep. 6:9 ; Col. 3:25.} ?
\VS{20}En un moment, ils mourront ; au milieu de la nuit, un peuple est ébranlé et passe ; le puissant s'en va, sans la main d'aucun homme.
\VS{21}Car les yeux de Dieu sont sur les voies de l'homme, il regarde tous ses pas.
\VS{22}Il n'y a ni ténèbres ni ombre de la mort où puissent se cacher les ouvriers d'iniquité.
\VS{23}Dieu ne regarde pas à deux fois un homme, pour le faire aller en jugement avec lui.
\VS{24}Il brise les puissants par des voies incompréhensibles, et il n établit d'autres à leur place ;
\VS{25}car il connaît leurs œuvres. Il les renverse de nuit, et ils sont écrasés ;
\VS{26}il les frappe comme des impies, au lieu où se tiennent tous les regards.
\VS{27}Du fait qu'ils se sont détournés de lui, et qu'ils n'ont considéré aucune de ses voies.
\VS{28}ils ont fait monter à Dieu le cri du pauvre, et il a entendu le cri des affligés\FTNT{Ja. 5:4.}.
\VS{29}S'il donne le repos, qui est-ce qui causera du trouble? S'il cache sa face à quelqu'un, qui le regardera, qu'il s'agisse de toute une nation ou d'un seul un homme?
\VS{30}afin que l'hypocrite ne règne pas, de peur qu'il plus un piège pour le peuple.
\VS{31}Car a-t-il jamais dit à Dieu : J'ai été pardonné, je ne pécherai plus ;
\VS{32}montre-moi ce que je ne vois pas ; si j'ai fait le mal, je ne le ferai plus ?
\VS{33}Mais Dieu ne te le rendra-t-il pas, puisque tu as rejeté son châtiment, quand tu as fait le choix que tu as fait? Pour moi, je ne sais que dire à cela; mais toi, si tu as quelque chose à répondre, parle.
\VS{34}Les gens de bon sens diront avec moi, et tout homme sage en conviendra,
\VS{35}que Job ne parle pas avec connaissance, et ses paroles manquent d'intelligence.
\VS{36}Ah! Mon père, que Job soit éprouvé jusqu'à ce qu'il soit vaincu, puisqu'il vaincu, puisqu'il répond comme les impies.
\VS{37}Car il ajoute péché sur péché; il applaudit au milieu de nous ; il parle de plus en plus contre Dieu.
\Chap{35}
\TextTitle{Discours d'Elihu : il reproche à Job ses propos irréfléchis}
\VerseOne{}Elihu reprit la parole et dit :
\VS{2}Penses-tu avoir raison de dire : Je suis juste devant Dieu ?
\VS{3}Quand tu dis : Que me sert-il, et que gagnerais-je de plus sans pécher ?
\VS{4}Je te répondrai en ces termes, et à tes amis qui sont avec toi.
\VS{5}Regarde les cieux, et considère-les ! Vois les nuées, comme elles sont plus hautes que toi !
\VS{6}Si tu pèches, quel mal fais-tu à Dieu ? Et si tes péchés se multiplient, quel mal reçoit-il ?
\VS{7}Si tu es juste, que lui donnes-tu ? Que reçoit-il de ta main ?
\VS{8}C'est à un homme, comme tu es, que ta méchanceté peut seule nuire, et c'est au fils d'un homme que ta justice peut seule être utile.
\VS{9}On fait crier les opprimés par la grandeur des maux qu'on leur afflige; ils crient à cause de la violence des grands;
\VS{10}Et nul ne dit : Où est le Dieu qui m'a fait, qui donne de quoi chanter pendant la nuit ,
\VS{11}qui nous instruit plus que les animaux de la terre, et plus intelligent que les oiseaux des cieux ?
\VS{12}On crie donc à cause de la fierté des méchants; mais Dieu ne l'exauce pas.
\FTNT{Es. 1:15 ; Ez. 8:18 ; Mi. 3:4 ; Jn. 9:31.}.
\VS{13} Cependant que ce soit en vain; que Dieu n'écoute pas, et que le Tout-puissant n'y a pas égard.
\VS{14}Encore moins dois-tu lui dire, tu ne le vois pas; car le jugement est devant lui, attends-le donc!
\VS{15}Mais maintenant, ce n'est rien ce que sa colère exécute, et il n'a pas encore pris connaissance en profondeur toutes les choses que tu as faites.
\VS{16}Job ouvre donc sa bouche pour se plaindre, il multiplie les paroles sans intelligence.
\Chap{36}
\TextTitle{Discours d'Elihu : Dieu traite les hommes selon leurs oeuvres}
\VerseOne{}Elihu continua de parler, et dit :
\VS{2}Attends un peu, et je te montrerai qu'il y a encore d'autres raisons pour la cause de Dieu.
\VS{3}Je tirerai de loin mes raisons, et je défendrai la justice du Créateur.
\VS{4}Car certainement il n'y aura rien de faux en tout ce que je dirai, et celui qui est avec toi, est parfait dans sa connaissance.
\VS{5}Dieu est puissant, mais il ne méprise personne ; il est puissant par la force de son coeur.
\VS{6}Il ne laisse pas vivre le méchant, et il fait droit aux pauvres.
\VS{7}Il ne détourne pas ses yeux de dessus les justes, il les place sur le trône avec les rois, il les y fait asseoir pour toujours, afin qu'ils soient élevés\FTNT{Ps. 33:18 ; Ps. 34:16.}.
\VS{8}S'ils sont liés de chaînes, s'ils sont pris dans les liens de l'affliction,
\VS{9}il leur fait connaître leurs œuvres, leurs transgressions, leur orgueil.
\VS{10}Alors il ouvre leur oreille pour leur discipline, il leur dit de se détourner de l'iniquité.
\VS{11}S'ils écoutent, et s'ils le servent, ils achèvent leurs jours dans le bonheur, leurs années dans la joie.
\VS{12}S'ils n'écoutent pas, ils passent par l'épée, ils expirent dans leur aveuglement.
\VS{13}Ceux qui sont hypocrites dans leur cœur, ils ne crient pas à lui quand il les a liés ;
\VS{14}leur personne meurt dans sa jeunesse, leur vie s'éteint parmi les débauchés.
\VS{15}Mais Dieu sauve celui qui est affligé de son oppression, et c'est par la détresse qu'il lui ouvre les oreilles.
\VS{16}Il t'écartera aussi de la détresse, pour te mettre au large, loin de toute angoisse, et ta table sera chargée de viandes grasses\FTNT{Ps. 50:15 ; Ps. 63:6.}.
\VS{17}Or tu remplis le jugement du méchant, mais le jugement et le droit subsisteront.
\VS{18}Certainement Dieu et irrité; prends garde qu'il ne te  plonge dans l'affliction, car il n'y aura pas alors de rançon si grande pour te délivrer\FTNT{Ps. 49:8.} !
\VS{19}Tes cris valent-ils ton or, et même toutes les forces qui se trouvent dans tes richesses ?
\VS{20}Ne soupire pas après la nuit, qui enlève les peuples de leur place.
\VS{21}Garde-toi de te retourner vers l'iniquité, car la souffrance t'y dispose.
\VS{22}Dieu est élevé par sa puissance ; qui saurait enseigner comme lui ?
\VS{23}Qui lui prescrit le chemin qu'il devait tenir? Qui lui dit : Tu a fait une injustice ?
\VS{24}Souviens-toi de célébrer ses ouvrages, que tous les hommes voient.
\VS{25}Tout homme les voit, chacun les contemple de loin.
\VS{26}Dieu est grand, mais nous ne le connaissons pas, quant au nombre de ses années il est insondable\FTNT{Es. 63:16 ; Ps. 92:8 ; Ps. 93:2 ; Ps. 102:13 ; La. 5:19.}.
\VS{27}Parce qu'il met les eaux  en petites gouttes, elle répandent la pluie selon la vapeur d'eau qui la contient ;
\VS{28}les nuées la font dégoutter, elles coulent sur les hommes en abondance.
\VS{29}Et qui pourra comprendre l'étendue des nuages et le son éclatant de sa tente ?
\VS{30}Voici, il étend sa lumière sur elle, et il se cache jusque dans les profondeurs de la mer.
\VS{31} Or c'est par ces choses qu'il juge les peuples, qu' il donne la nourriture en abondance.
\VS{32} Il tient caché dans les paumes de ses mains le feu étincelant, et lui ordonne de frapper de ce qui se présente à sa rencontre.
\VS{33} Son bruit porte les nouvelles ; les troupeaux font connaître qu'il approche.
\Chap{37}
\TextTitle{Discours d'Elihu : conclusion}
\VerseOne{}Mon cœur même à cause de cela est tout tremblant, il sort de sa place.
\VS{2}Ecoutez attentivement et en tremblant le bruit de sa voix, le grondement qui sort de sa bouche\FTNT{Ps. 29:3-9.} !
\VS{3}Il le conduit dans toute l'étendue des cieux, et son éclair brille jusqu'aux extrémités de la terre\FTNT{Ps. 97:4.}.
\VS{4}Après lui de l'élève un grand bruit, il tonne de sa voix majestueuse; il ne tarde pas après que sa voix a été entendue\FTNT{Jé. 10:13.}.
\VS{5}Dieu tonne avec sa voix d'une manière étonnante ; il fait de grandes choses que nous ne comprenons pas.
\VS{6}Car il dit à la neige : Tombe sur la terre ! Il le dit à la pluie, même aux plus fortes pluies.
\VS{7}Il met un sceau à la main de tous les hommes pour reconnaître tous les hommes qui sont son ouvrage.
\VS{8}Les bêtes entrent dans leurs tanières, et elles demeurent dans leurs repaires.
\VS{9}L'ouragan vient du fond du sud, et le froid vient des vents du nord.
\VS{10}Par son souffle, Dieu donne la glace, et il réduit l'espace où se répondaient au large les eaux\FTNT{Ps. 147:17-18.}.
\VS{11}Il lasse les nuages à force d'arroser, il écarte les nuages par sa lumière.
\VS{12}Et ceux-ci font plusieurs tours pour faire ce qu'il a commandé, sur la face de la terre sur la face de la terre habitée ;
\VS{13}Il les fait venir pour s'en servir soit comme une verge pour la terre, soit pour répandre ses bienfaits\FTNT{Ex. 9:18-23 ; 1 S. 12:18-19.}.
\VS{14}Job, arrête-toi, prête l'oreille à ces choses ! Considère encore les merveilles de Dieu !
\VS{15}Sais-tu comment Dieu les dispose, et fait briller la lumière de ses nuages?
\VS{16}Connais-tu le balancement des nuages, les merveilles de celui dont la science est parfaite ?
\VS{17}Sais-tu pourquoi tes vêtements sont chauds quand la terre se repose par le vent du midi ?
\VS{18}Peux-tu étendre avec lui les cieux, aussi fermes qu'un miroir de fonte ?
\VS{19}Montre-nous ce que nous pouvons lui dire ; car nous ne saurions rien dire par ordre à cause de nos ténèbres. 
\VS{20}Lui racontera-t-on quand je parlerai ? S'il y a un homme qui en parle, certainement il en sera englouti ?
\VS{21}Et maintenant on ne voit pas la lumière du soleil qui resplendit dans les cieux, lorsque le vent passe et le nettoie ;
\VS{22}Le temps qui la reluit comme l'or vient du nord. Il y a en Dieu une majesté redoutable.
\VS{23}Nous ne saurions comprendre le Tout-Puissant, grand en puissance, en jugement et en abondante justice, il n'opprime personne !
\VS{24}C'est pourquoi les hommes le craignent ; mais il ne les voit pas tous sages de cœur\FTNT{Ps. 92:7 ; Ro. 1:21.}.
\Chap{38}
\TextTitle{Yahweh interroge Job}
\VerseOne{}Yahweh répondit à Job du milieu de le tourbillon et dit :
\VS{2}Qui est celui qui obscurcit mes décisions par des paroles sans connaissance ?
\VS{3}Ceins maintenant tes reins comme un vaillant homme ; je t'interrogerai, et tu me fera voir ta science.
\VS{4}Où étais-tu quand je fondais la terre ? Dis-le, si tu as de l'intelligence\FTNT{Pr. 8:29.}.
\VS{5}Qui en a réglé les mesures, le sais-tu ? Ou qui a appliqué sur elle le niveau ?
\VS{6}Sur quoi ses bases sont-elles plantées ? Ou qui en a posé la pierre angulaire pour la soutenir\FTNT{Ps. 104:5.}?
\VS{7}Quand les étoiles du matin se réjouissent ensemble, et que tous les fils de Dieu poussent des cris de joie \FTNT{Ps. 148:3.} ?
\VS{8}Qui a renfermé la mer dans ses bords, quand elle fut tirée de la matrice et qu'elle sortit? 
\VS{9}quand je lui donnai la nuée pour vêtement, et l'obscurité pour langes ;
\VS{10}que je lui imposai ma loi, et que je lui mis des barrières et des portes;
\VS{11} et quand je dis : Tu viendras jusqu'ici, tu n'iras pas plus loin ; ici s'arrêtera l'orgueil de tes flots ?
\VS{12}Depuis que tu es au monde, as-tu commandé au matin et as-tu montré à l'aube du jour le lieu où elle doit se lever,
\VS{13}pour qu'elle saisisse les extrémités de la terre, et que les méchants en soient chassés ;
\VS{14}pour que la terre prenne une forme comme l'argile qui reçoit un sceau, et qu'elle soit parée comme d'un vêtement nouveau ;
\VS{15}pour que la lumière soit ôtée aux méchants, et que le bras qui se lève soit brisé\FTNT{Ps. 10:15.} ?
\VS{16}As-tu pénétré jusqu'aux sources de la mer ? T'es-tu promené dans les profondeurs de l'abîme ?
\VS{17}Les portes de la mort se sont-elles découvertes à toi ? As-tu vu les portes de l'ombre de la mort ?
\VS{18}As-tu compris l'étendue de la terre ? Si tu sais tout cela, dis-le !
\VS{19}Où est la demeure de la lumière, et où est le lieu des ténèbres ?
\VS{20}Pour que tu les prennes à leur limite, et que tu connaisses le chemin de leur maison ?
\VS{21}Tu le sais, car alors tu étais né, et le nombre de tes jours est grand !
\VS{22}Es tu entré dans les trésors de la neige ? As-tu vu les trésors de grêle,
\VS{23}que je réserve pour les temps de détresse, pour les jours de guerre et de bataille\FTNT{Ex. 9:23 ; Jos. 10:11 ; Ap. 8 :7.} ?
\VS{24}Par quel chemin la lumière se partage la lumière, et le vent d'orient se répand-il sur la terre\FTNT{Jn. 3:8.} ?
\VS{25}Qui a ouvert un conduit aux inondations, et tracé la route de l'éclair et du tonnerre,
\VS{26}pour qu'elle pleuve sur une terre sans habitants, sur un désert sans hommes\FTNT{Ps. 104:13-14 ; PS. 147:8 ; Ac. 14:17.} ;
\VS{27}pour qu'elle abreuve les lieux solitaires et arides, et qu'elle fasse germer et sortir l'herbe ?
\VS{28}La pluie a-t-elle un père ? Qui enfante les gouttes de la rosée ?
\VS{29}De quel sein est sortie la glace ? Et qui enfante le givre du ciel,
\VS{30}pour que les eaux se cachent comme une pierre, et que le dessus de l'abîme soit enchaîné ?
\VS{31}Peux-tu resserer les liens des pléiades ou détacher les chaînes d'orient\FTNT{Am. 5:8.}?
\VS{32}Fais-tu sortir en leur temps les signes du zodiaque, et conduis-tu la Grande Ourse avec ses petits ?
\VS{33}Connais-tu les lois du ciel ? Disposes-tu de son pouvoir sur la terre\FTNT{Jé. 31:35-36 ; Ps. 104:4.} ?
\VS{34}Élèves-tu la voix jusqu'aux nuées, pour que des eaux abondantes te couvrent ?
\VS{35}Envoies-tu les éclairs ? Partent-ils ? Te disent-ils : Nous voici ?
\VS{36}Qui a mis la sagesse dans le cœur, ou qui a donné l'intelligence à l'esprit\FTNT{Ec. 2:26.} ?
\VS{37}Qui Est-ce qui peut avec intelligence compter les nuages, et pour placer les outres des cieux,
\VS{38}Quand la poussière est détrompée par les eaux qui l'arrosent, et que les mottes viennent à se joindre ?
\Chap{39}
\TextTitle{Yahweh démontre son omnipotence}
\VerseOne{}Chasses-tu de la proie pour la lionne, et apaises-tu la faim des lionceaux\FTNT{Ps. 104:21.},
\VS{2}Quand ils se tapissent dans leurs tanières et se tiennent aux aguets dans leur repaire ?
\VS{3}Qui est-ce qui apprête la nourriture au corbeau, quand ses petits crient à Dieu, et qu'ils vont errants, parce qu'ils n'ont point de quoi manger\FTNT{Ps. 104:27 ; Ps. 147:9 ; Mt. 6:26.} ?
\VS{4}Sais-tu quand les boucs de rochers mettent bas ? Observes-tu les biches de rochers quand elles font leurs petits\FTNT{Ps. 29:9.} ?
\VS{5}Comptes-tu les mois de leur gestation, et sais-tu le temps auquel elles font leurs petits ?
\VS{6}Et qu'elles se courbent pour mettre bas leurs petits et se délivrent de leurs douleurs ?
\VS{7}Leurs petits se fortifient, ils croissent en plein air, ils s’en vont et ne reviennent plus vers elles.
\VS{8}Qui a laissé aller libre l’âne sauvage ? et qui a délié les liens de l’âne farouche ?
\VS{9}Auquel j’ai donné le désert pour maison, et la terre inhabitée pour ses retraites\FTNT{Jé. 2:24.} ?
\VS{10}Il se rit du bruit des villes, il n'entend pas les cris d'un exacteur.
\VS{11}Les montagnes qu'il va épiant çà et là, sont ses pâturages, et il cherche toute sorte de verdure. 
\VS{12}Le buffle voudra-t-il te servir, ou demeurera-t-il à ta crèche ? 
\VS{13}Lies-tu le buffle avec son licou pour labourer ? ou rompra-t-il les mottes des vallées après toi ? 
\VS{14}Te fies-tu à lui parce que sa force est grande, et lui abandonnes-tu ton travail? 
\VS{15}Comptes-tu sur lui pour rentrer ta semence, et pour l'amasser sur ton aire ? 
\VS{16}As-tu donné aux paons ce plumage qui est si brillant, ou à l'autruche les ailes et les plumes ? 
\VS{17}Néanmoins elle abandonne ses oeufs à terre, et les fait échauffer sur la poussière ;
\VS{18}et elle oublie que le pied peut les écraser, ou que les bêtes des champs peuvent les fouler. 
\VS{19}Elle est dure envers ses petits, comme s’ils n’étaient pas siens. Son travail est vain, elle ne s’en inquiète pas.
\VS{20}Car Dieu l'a privée de sagesse et ne lui a pas donné l'intelligence.
\VS{21}A la première occasion elle se dresse en haut, et se moque du cheval et de celui qui le monte. 
\VS{22}As-tu donné la force au cheval ? et as-tu revêtu son cou d'un hennissement éclatant comme le tonnerre ? 
\VS{23}Fais-tu bondir le cheval comme la sauterelle ? le son magnifique de ses narines est effrayant.
\VS{24}Il creuse la terre de son pied, il s'égaie dans sa force, il va à la rencontre d'un homme armé ;
\VS{25}Il se rit de la frayeur, il ne s'épouvante de rien, et il ne se détourne point de devant l'épée.
\VS{26}Il n'a point peur des flèches qui sifflent tout autour de lui, ni du fer luisant de la lance et du javelot. 
\VS{27}Il creuse la terre, plein d'émotion et d'ardeur au son de la trompette, et il ne peut se retenir. 
\VS{28}Au son bruyant de la trompette, il dit : En avant ! En avant ! Il flaire de loin la bataille, le tonnerre des capitaines, et le cri de triomphe.
\VS{29}Est-ce par ta sagesse que l'épervier prend son vol, et qu'il étend ses ailes vers le midi ?
\VS{30}Est-ce par ton commandement que l'aigle s'élève, et qu'il place son nid sur les hauteurs\FTNT{Jé. 49:16 ; Abd. 1:4.} ?
\VS{31}Elle habite sur les rochers, et elle s'y tient ; même sur les sommets des rochers et dans des lieux forts. 
\VS{32}De là il découvre le gibier, ses yeux voient de loin.
\VS{33}Ses petits auss sucent le sang ; et là où sont des cadavres, il s'y trouve aussitôt\FTNT{Mt. 24:28 ; Lu. 17:37.}.
\TextTitle{Yahweh lui pose une question}
\VS{34}Yahweh prit encore la parole et dit à Job :
\VS{35}Celui qui conteste avec le Tout-puissant, lui apprendra-t-il quelque chose ? Que celui qui dispute avec Dieu réponde à ceci.
\TextTitle{Réponse de Job}
\VS{36}Alors Job répondit à Yahweh et dit :
\VS{37}Voici, je suis un homme vil ; que te répondrais-je ? Je mets ma main sur ma bouche\FTNT{Ps. 39:10.}.
\VS{38}J'ai parlé une fois, mais je ne répondrai plus ; j'ai même parlé deux fois mais je n'ajouterai plus.
\Chap{40}
\TextTitle{Yahweh questionne encore Job}
\VerseOne{}Et Yahweh répondit à Job du milieu d'un tourbillon, et lui dit :
\VS{2}Ceins maintenant tes reins comme un vaillant homme ; je t'interrogerai et tu m'enseigneras.
\VS{3}Anéantiras-tu mon jugement ? me condamneras-tu pour te justifier\FTNT{Ps. 51:6 ; Ro. 3:4.} ?
\VS{4}As-tu un bras comme celui de Dieu ; tonnes-tu de la voix, comme lui ?
\VS{5}Pare-toi maintenant de magnificence et de grandeur, et revêts-toi de majesté et de gloire.
\VS{6}Répands les fureurs de ta colère, d'un regard, humilie tous les orgueilleux
\VS{7}D'un regard humilie les orgueilleux, écrase sur place les méchants,
\VS{8}cache-les tous ensemble dans la poussière, enferme leur face dans les ténèbres !
\VS{9}Alors je rends hommage à mon sauveur qui me sauve par sa droite.
\VS{10}Voici le béhémoth, que j'ai façonné comme toi ! Il mange de l'herbe comme le bœuf.
\VS{11}Regarde donc, sa force est dans ses reins, et sa puissance dans les muscles de son ventre ;
\VS{12}il plie sa queue aussi ferme qu'un cèdre ; les tendons de ses cuisses sont entrelacés ;
\VS{13}ses os sont des tubes d'airain, ses membres sont comme des barres de fer.
\VS{14}C’est le chef-d’œuvre de Dieu ; celui qui l’a fait lui a donné son épée.
\VS{15}Car les montagnes lui apportent sa pâture, là où se jouent toutes les bêtes des champs.
\VS{16}Il se couche sous les lotus, caché dans les roseaux et les marécages ;
\VS{17}les lotus le couvrent de leur ombre, les saules du torrent l'enveloppent.
\VS{18}Voilà, il engloutit une rivière en buvant, et il ne s'en retire pas vite ; et il ne s'étonnerait pas quand le Jourdain se dégorgerait dans sa gueule. 
\VS{19}Il l'engloutit en le voyant, et son nez passe au travers des empêchements qu'il rencontre.  
\VS{20}Attireras-tu le léviathan à l'hameçon ? Saisiras-tu sa langue avec une corde ?
\VS{21}Mettras-tu un jonc dans ses narines ? Lui perceras-tu la mâchoire avec un crochet ?
\VS{22}Accumulera-t-il les supplications ? Te parlera-t-il d'une voix douce ?
\VS{23}Fera-t-il une alliance avec toi, pour te prendre pour toujours comme esclave ?
\VS{24}Joueras-tu avec lui comme avec un oiseau ? L'attacheras-tu pour amuser les jeunes filles ?
\VS{25}Les pêcheurs en trafiquent-ils ? Le partagent-ils entre les marchands ?
\VS{26}Couvriras-tu sa peau de dards, et sa tête de harpons ?
\VS{27}Mets ta main contre lui, et tu ne te souviendras plus de l'attaquer.
\VS{28}Voici, on est trompé dans son attente ; à sa vue n'est-on pas terrassé ?
\Chap{41}
\VerseOne{}Nul n'est assez féroce pour l'exciter ; qui donc me résisterait en face ?
\VS{2}De qui suis-je le débiteur ? Je le paierai. Sous le ciel tout m'appartient\FTNT{Ex. 19:5 ; De. 10:14 ; Ps. 24:1 ; Ps. 50:12 ; 1 Co. 10:26 ; Ro. 11:35.}.
\VS{3}Je veux encore parler de ses discours, et de sa force, et de la beauté de sa structure.
\VS{4}Qui découvrira son vêtement devant ma face ? Qui viendra freiner ses mâchoires par un mors ?
\VS{5}Qui ouvrira les portes devant sa face ? Autour du lion habite la terreur.
\VS{6}Ses magnifiques et puissants boucliers sont fermés comme un sceau ;
\VS{7}ils se serrent l'un contre l'autre, et l'air n'entrerait pas entre eux ;
\VS{8}ce sont des frères qui s'embrassent, se saisissent, demeurent inséparables.
\VS{9}Ses éternuements font briller la lumière, ses yeux sont comme les paupières de l'aurore.
\VS{10}Des flammes viennent de sa bouche, des étincelles de feu s'en échappent.
\VS{11}Une fumée sort de ses narines, comme d'un chaudron qui bout, d'une chaudière ardente.
\VS{12}Son souffle allume les charbons, de sa bouche sort la flamme.
\VS{13}La force a son cou pour demeure, et l'effroi bondit devant lui.
\VS{14}Ses parties charnues sont jointes ensemble, fondues sur lui, inébranlables.
\VS{15}Son cœur est dur comme la pierre, dur comme la meule inférieure.
\VS{16}Quand il se lève, les plus vaillants ont peur, et l'épouvante les fait quitter le droit chemin.
\VS{17}C'est en vain qu'on l'attaque avec l'épée ; la lance, le javelot, la cuirasse ne servent à rien.
\VS{18}Il regarde le fer comme de la paille, l'airain comme du bois pourri.
\VS{19}La flèche de l'arc ne le met pas en fuite, les pierres de la fronde sont pour lui changés en chaume.
\VS{20}Il ne voit dans la massue qu'un brin de paille, il rit au sifflement des dards.
\VS{21}Sous son ventre sont des pointes aiguës : On dirait une herse qu'il étend sur la boue.
\VS{22}Il fait bouillir les profondeurs de la mer comme une chaudière, il la traite comme un vase rempli de parfums.
\VS{23}Il laisse après lui un sentier lumineux ; l'abîme prend la chevelure d'un vieillard.
\VS{24}Sur la terre nul n'est son maître ; il a été façonné pour ne rien craindre.
\VS{25}Il regarde avec dédain tout ce qui est élevé, il est le roi des plus fiers animaux.
\Chap{42}
\TextTitle{Job reconnaît la souveraineté de Dieu et s'humilie}
\VerseOne{}Job répondit à Yahweh et dit :
\VS{2}Je sais que tu peux tout, et qu'on ne saurait t'empêcher de faire ce que tu penses.
\VS{3}Quel est celui qui a la folie d'obscurcir mes conseils ? Oui, j'ai parlé sans les comprendre, de merveilles qui me dépassent et que je ne connais pas\FTNT{Ps. 40:6 ; Ps. 131:1 ; Ps. 139:6.}.
\VS{4}Écoute-moi maintenant, et je parlerai ; je t'interrogerai et tu m'instruiras.
\VS{5}J'avais entendu parler de toi ; mais maintenant mon œil t'a vu.
\VS{6}C'est pourquoi je me condamne et je me repens d'avoir ainsi parlé et je m'en sur la poussière et sur la cendre.
\VS{7}Après que Yahweh eut ainsi parlé à Job, il dit à Eliphaz de Théman : Ma colère est embrasée contre toi et contre tes deux amis, parce que vous n'avez pas parlé de moi avec droiture comme Job, mon serviteur.
\VS{8} C'est pourquoi prenez maintenant sept taureaux et sept béliers, allez auprès de mon serviteur Job, et offrez un holocauste pour vous. Job, mon serviteur, priera pour vous, et certainement j'exaucerai sa prière, afin que je ne vous traite pas selon votre folie ; car vous n'avez pas parlé de moi avec droiture, comme mon serviteur Job.
\VS{9} Ainsi Eliphaz de Théman, Bildad de Schuach, et Tsophar de Naama allèrent et firent comme Yahweh leur avait commandé ; et Yahweh exauça la prière de Job.
\VS{10}Yahweh rétablit Job de sa captivité, quand il eut prié pour ses amis ; et Yahweh lui ajouta le double de tout ce qu'il avait possédé.
\VS{11}Ses frères, ses sœurs, et tous ceux qui l'avaient connu auparavant vinrent tous le visiter, et ils mangèrent avec lui dans sa maison. Ils compatirent et le consolèrent au sujet de tout le mal que Yahweh avait fait venir sur lui, et chacun lui donna une kesita et un anneau d'or.
\VS{12}Pendant ses dernières années, Job reçut de Yahweh plus de bénédictions qu'il n'en avait reçu dans les premières. Il posséda quatorze mille brebis, six mille chameaux, mille paires de bœufs, et mille ânesses.
\VS{13}Il eut aussi sept fils et trois filles :
\VS{14}Il donna à la première le nom de Jemima, à la seconde celui de Ketsia, à la troisième celui de Kéren-Happuc.
\VS{15}Et il ne se trouvait pas de femmes aussi belles que les filles de Job dans tout le pays. Leur père leur donna une part de l'héritage parmi leurs frères.
\VS{16}Job vécut, après ces choses, cent quarante ans, et il vit ses fils et les fils de ses fils jusqu'à la quatrième génération.
\VS{1} Job mourut âgé et rassasié de jours.
\PPE{}
\end{multicols}
