\ShortTitle{Amos}\BookTitle{Amos}\BFont
\noindent\hrulefill
{\footnotesize
\textit{
\bigskip
{\centering{}
\\Auteur : Amos
\\(Heb : Amowc)
\\Signification : Fardeau, porteur de fardeau
\\Thème : Jugement sur le péché
\\Date de rédaction : 8ème siècle av. J.-C.\\}
}
%\bigskip
\textit{
\\Originaire de Tekoa, Amos exerça son ministère dans le royaume du Nord, au temps d’Ozias, roi de Juda, et Jéroboam II, roi d’Israël. Il fut aussi le contemporain des prophètes Osée, Michée, Jonas et Esaïe.
%\bigskip
\\Alors que le peuple juif jouissait d’une certaine prospérité, l’immoralité et les sacrilèges prirent place dans le royaume. Amos avertit le peuple de son péché et du jugement qui s’y attachait. Il rappela au peuple la bonté de Dieu et l’invita à revenir à Yahweh et à lui demeurer fidèle.\bigskip
}
}
\par\nobreak\noindent\hrulefill
\begin{multicols}{2}
\Chap{1}
\TextTitle{Proclamation des jugements}
\VerseOne{}Paroles d'Amos, berger de Tekoa, qui prophétisa sur Israël, du temps d’Ozias, roi de Juda, et de Jéroboam, fils de Joas, roi d'Israël, deux ans avant le tremblement de terre\FTNT{Za. 14:5}.
\VS{2}Il dit : Yahweh rugit de Sion, et fait entendre sa voix de Jérusalem. Les habitations des bergers sont en deuil, et le sommet du Carmel est desséché\FTNT{Jé. 25:30 ; Joë 3:16}.
\TextTitle{Jugements sur les villes et les pays d'alentour}
\VS{3}Ainsi parle Yahweh : A cause de trois crimes de Damas, et même de quatre, je ne le révoquerai pas mais je le ferai\FTNT{Il est question du jugement de Dieu.}, parce qu'ils ont foulé Galaad avec des herses de fer\FTNT{Es. 17:1}.
\VS{4}J'enverrai le feu dans la maison de Hazaël, et il dévorera les palais de Ben-Hadad.
\VS{5}Je briserai aussi les verrous de Damas, j'exterminerai de Bikath-Aven ses habitants, et de Beth-Eden celui qui tient le sceptre. Le peuple de Syrie sera mené captif à Kir, dit Yahweh.
\VS{6}Ainsi parle Yahweh : A cause de trois crimes de Gaza, et même de quatre, je ne révoquerai pas cela, mais je le ferai\FTNT{Il est question du jugement de Dieu.} parce qu'ils ont emmené des captifs en grand nombre pour les livrer à Edom\FTNT{Ez. 25:13-17}.
\VS{7}J'enverrai le feu dans les murs de Gaza, et il dévorera les palais.
\VS{8}J'exterminerai d'Asdod les habitants, et d'Askalon celui qui tient le sceptre ; je tournerai ma main contre Ekron, et le reste des Philistins périra, dit le Seigneur, Yahweh.
\VS{9}Ainsi parle Yahweh : A cause de trois crimes de Tyr, et même de quatre, je ne révoquerai pas cela, mais je le ferai\FTNT{Il est question du jugement de Dieu.}, parce qu'ils ont livré à Edom des captifs en grand nombre sans se souvenir de l'alliance fraternelle\FTNT{Ez. 26:2}.
\VS{10}J'enverrai le feu dans les murs de Tyr, et il dévorera les palais.
\VS{11}Ainsi parle Yahweh : A cause de trois crimes d'Edom, et même de quatre, je ne révoquerai pas cela, mais je le ferai\FTNT{Il est question du jugement de Dieu.}, parce qu'il a poursuivi son frère avec l'épée, refoulant toute compassion, parce que sa colère déchire continuellement et qu'il garde sa fureur éternellement.
\VS{12}J'enverrai le feu dans Théman, et il dévorera les palais de Botsra\FTNT{Jé. 49:7 ; Abd. 1:9}.
\VS{13}Ainsi parle Yahweh : A cause de trois crimes des enfants d'Ammon, et même de quatre, je ne révoquerai pas cela, mais je le ferai\FTNT{Il est question du jugement de Dieu.}, parce qu’ils ont fendu le ventre des femmes enceintes de Galaad pour étendre leurs frontières\FTNT{Ez. 21:33 ; So. 2:8}.
\VS{14}J'allumerai le feu dans les murs de Rabba, et il dévorera les palais, au bruit des cris de guerre au jour du combat, et au milieu de l’ouragan au jour de la tempête.
\VS{15}Et leur roi ira en captivité, lui et ses chefs, dit Yahweh.
\Chap{2}
\TextTitle{Autres Jugements sur les nations d'alentour}
\VerseOne{}Ainsi parle Yahweh : A cause de trois crimes de Moab, et même de quatre, je ne révoquerai pas cela, mais je le ferai, parce qu'il a brûlé les os du roi d'Edom jusqu’à les calciner.
\VS{2}J'enverrai le feu dans Moab, et il dévorera les palais de Kerijoth ; et Moab périra dans le tumulte, au milieu des cris de guerre et du bruit du shofar\FTNT{Ez. 25:8-9}.
\VS{3}J'exterminerai les juges de son pays, et je tuerai tous ses chefs, dit Yahweh.
\TextTitle{Jugement sur Juda et d'Israël pour leurs crimes}
\VS{4}Ainsi parle Yahweh : A cause de trois crimes de Juda, et même de quatre, je ne révoquerai pas cela, mais je le ferai, parce qu'ils ont rejeté la loi de Yahweh et n'ont point gardé ses ordonnances ; parce qu’ils ont été égarés par les mensonges après lesquels leurs pères ont marché.
\VS{5}J'enverrai le feu dans Juda, et il dévorera les palais de Jérusalem.
\VS{6}Ainsi parle Yahweh : A cause de trois crimes d'Israël, et même de quatre, je ne révoquerai pas cela, mais je le ferai, parce qu'ils ont vendu le juste pour de l'argent, et le pauvre pour une paire de souliers.
\VS{7}Ils aspirent à voir la poussière de la terre sur la tête des misérables, et ils pervertissent la voie des malheureux. Le fils et le père vont vers la même jeune fille, pour profaner mon Saint Nom.
\VS{8}Ils se couchent près de chaque autel, sur les vêtements qu'ils ont pris en gage, et boivent dans la maison de leurs dieux le vin de ceux qu’ils châtient.
\VS{9}Pourtant j'ai détruit devant eux les Amoréens qui étaient hauts comme les cèdres et forts comme les chênes ; j'ai détruit son fruit en haut, et ses racines en bas\FTNT{No. 21:24 ; Jos. 24:8}.
\VS{10}Je vous ai fait monter du pays d'Egypte et je vous ai conduits dans le désert quarante ans pour que vous possédiez le pays des Amoréens.
\VS{11}J'ai suscité parmi vos fils des prophètes, et parmi vos jeunes hommes des nazaréens\FTNT{Le mot nazaréen vient de l'hébreu nâzîr, de la racine nâzar qui signifie séparer. Il y avait deux types de nazaréens. Premièrement ceux qui étaient appelés par Dieu. Par exemple : Samson Jg. 13:1-7 ; Samuel 1 S. 1:11 ; Jean-Baptiste Lu. 1:15. Deuxièmement les personnes qui voulaient se consacrer à Dieu. No. 6:13.}. N'en est-il pas ainsi, enfants d'Israël ? dit Yahweh.
\VS{12}Mais vous avez fait boire du vin aux nazaréens, et vous avez donné cet ordre aux prophètes disant : Ne prophétisez pas\FTNT{Es. 30:10 ; Jé. 11:21 ; Mi. 2:6} !
\VS{13}Voici, je m’en vais fouler le lieu où vous habitez, comme un chariot plein de gerbes foule tout par où il passe.
\VS{14}Tellement que l’homme agile ne pourra pas fuir, et le fort ne pourra pas se servir de sa vigueur, et l’homme vaillant ne sauvera pas sa vie\FTNT{Jé. 46:6}.
\VS{15}Celui qui manie l'arc, ne pourra pas tenir ferme, et celui qui a les pieds légers n'échappera pas, et le cavalier ne sauvera pas sa vie.
\VS{16}Le plus courageux d’entre les hommes vaillants s'enfuira tout nu en ce jour-là, dit Yahweh.
\Chap{3}
\TextTitle{La maison de Jacob coupable devant Yahweh}
\VerseOne{}Enfants d’Israël écoutez la parole que Yahweh prononce contre vous, contre toute la famille que j'ai fait monter du pays d'Egypte.
\VS{2}Je vous ai connu vous seuls d'entre toutes les familles de la terre ; c'est pourquoi je vous châtierai pour toutes vos iniquités\FTNT{Ex. 19:5-6 ; Ps. 147:19-20}.
\VS{3}Deux hommes marchent-ils ensemble s’ils ne sont pas accordés ?
\VS{4}Le lion rugit-il dans la forêt sans qu’il n'ait de proie ? Le lionceau jette-t-il son cri de sa tanière sans qu’il n'ait rien attrapé ?
\VS{5}L'oiseau tombe-t-il dans le filet posé à terre sans que ce ne soit un piège ? Le filet est-il ramassé par terre sans qu’il n’y ait rien de capturé ?
\VS{6}Le shofar sonne-t-il dans une ville sans que le peuple en étant tout effrayé s’assemble ? Arrive-t-il un malheur dans une ville sans que Yahweh ne l’ait causé\FTNT{Es. 45:7 ; La. 3:37-38} ?
\VS{7}Car le Seigneur ne fait rien sans avoir révélé son secret à ses serviteurs les prophètes.
\VS{8}Le lion rugit, qui ne serait pas effrayé ? Le Seigneur, Yahweh, parle, qui ne prophétiserait\FTNT{Lorsque Yahweh parle, des prophètes sont suscités : Jé. 20:9 ; Mi. 3:8 ; Ac. 4:20.} ?
\VS{9}Faites entendre votre voix dans les palais d'Asdod, et dans les palais du pays d'Egypte, et dites : Assemblez-vous sur les montagnes de Samarie, et voyez l’important tumulte interne et quelles oppressions dans son sein !
\VS{10}Ils ne savent pas faire ce qui est droit, dit Yahweh, ils amassent la violence et la rapine dans leurs palais.
\VS{11}C'est pourquoi ainsi parle le Seigneur, Yahweh : L'ennemi viendra, il cernera le pays, il t'ôtera ta force et tes palais seront pillés.
\VS{12}Ainsi parle Yahweh : Comme un berger arrache de la gueule d'un lion deux jambes ou un bout d'oreille, ainsi les enfants d'Israël qui habitent dans Samarie seront arrachés de l’angle d’un lit et de l’asile de damas.
\VS{13}Ecoutez et soyez mes témoins contre la maison de Jacob, dit le Seigneur, Yahweh, le Dieu des armées :
\VS{14}Le jour où je punirai Israël pour ses péchés, j’exercerai mon châtiment sur les autels de Béthel ; les cornes de l'autel seront brisées, et tomberont à terre.
\VS{15}J’abattrai la maison d'hiver et la maison d'été ; les maisons d'ivoire seront détruites, et un grand nombre de maisons disparaîtront, dit Yahweh.
\Chap{4}
\TextTitle{Yahweh condamne les sacrifices du peuple à cause de leur péché}
\VerseOne{}Ecoutez cette parole, vaches de Basan, qui êtes sur la montagne de Samarie, vous qui opprimez les faibles, qui maltraitez les pauvres, qui dites à leurs maîtres : Apportez, et que nous buvions !
\VS{2}Le Seigneur, Yahweh, l’a juré par sa sainteté : Voici, les jours viennent sur vous, où l’on vous enlèvera avec des hameçons, et votre postérité avec des crochets de pêche\FTNT{Jé. 16:16 ; Ha. 1.14-16}.
\VS{3}Vous sortirez dehors par les brèches, chacune devant soi, et vous serez jetées dans la forteresse, dit Yahweh.
\VS{4}Allez à Béthel, et péchez ; à Guilgal et péchez davantage ! Amenez vos sacrifices dès le matin, et vos dîmes tous les trois ans\FTNT{Voir commentaires en Mal. 3:10 et No. 18:21.} !
\VS{5}Brûlez de l’encens avec du pain levé pour l’offrande de remerciement ; proclamez et publiez les offrandes volontaires ; car c’est là ce que vous aimez, enfants d'Israël, dit le Seigneur, Yahweh\FTNT{Lé. 2:1}.
\TextTitle{Endurcisement du peuple malgré les châtiments de Yahweh}
\VS{6}C'est pourquoi je vous ai envoyé la famine dans toutes vos villes, et la disette de pain dans toutes vos demeures ; mais malgré cela, vous n’êtes pas revenus vers moi, dit Yahweh.
\VS{7}Je vous ai aussi privés de pluie, alors qu’il restait encore trois mois jusqu'à la moisson ; j'ai fait pleuvoir sur une ville et je n'ai pas fait pleuvoir sur une autre ville ; une parcelle a été arrosée par la pluie, et l'autre parcelle, sur laquelle il n'a pas plu, est desséchée\FTNT{1 R. 8:35 ; 1 R. 17:1 ; Ag. 1:11 ; Es. 5:6}.
\VS{8}Et deux, même trois villes sont allées vers une autre ville pour boire de l'eau et n'ont pas été désaltérées, mais vous n’êtes pas revenus vers moi, dit Yahweh.
\VS{9}Je vous ai frappés par la rouille et par la nielle, mais vous n’êtes pas revenus vers moi, dit Yahweh\FTNT{De. 28:22-39 ; 1 R. 8:37 ; 2 Ch. 6:28 ; Ag. 2:17}.
\VS{10}J’ai envoyé parmi vous la peste comme celle en Egypte ; j'ai tué par l'épée vos jeunes hommes et vos chevaux en captivité ; j'ai fait remonter, jusque dans votre nez, la puanteur de vos camps ; mais vous n’êtes pas revenus vers moi, dit Yahweh\FTNT{Ez. 14:19}.
\VS{11}Je vous ai détruits comme Dieu détruisit Sodome et Gomorrhe, et vous avez été comme un tison arraché du feu, mais vous n’êtes pas revenus vers moi, dit Yahweh\FTNT{Ge. 19:24 ; Za. 3:2 ; Jé. 49:18},
\VS{12}C'est pourquoi je te traiterai de la même manière ô Israël ; et parce que je te traiterai de la même, prépare-toi à la rencontre de ton Dieu, ô Israël !
\VS{13}Car voici celui qui a formé les montagnes et créé le vent, et qui déclare à l'homme quelle est sa pensée, qui fait l'aube et l'obscurité, et qui marche sur les hauteurs de la terre ; son nom est Yahweh, le Dieu des armées.
\Chap{5}
\TextTitle{Yahweh exhorte Israël à revenir à lui}
\VerseOne{}Ecoutez cette parole, cette complainte que je prononce sur vous, maison d'Israël !
\VS{2}Elle est tombée, elle ne se relèvera plus, la vierge d'Israël ; elle est couchée par terre, et personne ne la relève.
\VS{3}Car ainsi a parlé le Seigneur, Yahweh : La ville qui mettait en campagne mille hommes n'en aura plus que cent ; et celle qui mettait en campagne cent hommes n'en aura plus que dix dans la maison d’Israël.
\VS{4}Car ainsi a parlé Yahweh à la maison d'Israël : Cherchez-moi, et vous vivrez !
\VS{5}Ne cherchez pas Béthel, et n'allez pas à Guilgal, et ne passez point à Beer-Schéba. Car Guilgal sera transportée en captivité, et Béthel sera détruite\FTNT{Os. 4:15}.
\VS{6}Cherchez Yahweh, et vous vivrez, de peur qu'il ne saisisse comme un feu la maison de Joseph, et que ce feu ne la consume, sans qu'il y ait personne à Béthel pour l’éteindre.
\VS{7}Ils changent le droit en absinthe, et ils foulent à terre la justice\FTNT{Es. 5:26-28 ; Ha. 1:1-3}.
\VS{8}Celui qui a créé les Pléiades et l'Orion, qui change les ténèbres en aurore, et qui obscurcit le jour en nuit, qui appelle les eaux de la mer, et les répand sur la surface de la Terre, Yahweh est son nom\FTNT{Job 9:9 ; Job 38:31 ; Es. 58:8-10}.
\VS{9}Il fait éclater la ruine sur les puissants, et la ruine vient sur les forteresses.
\VS{10}Ils haïssent celui qui les reprend à la porte, et ils ont en abomination celui qui parle en intégrité.
\VS{11}C'est pourquoi, puisque vous opprimez le pauvre, et que vous prenez de lui du blé en présent, vous avez bâti des maisons en pierres de taille, mais vous n'y habiterez pas ; vous avez planté des vignes délicieuses, mais vous n'en boirez pas le vin.
\VS{12}Car je connais vos nombreux crimes, et vos péchés se sont multipliés : Vous opprimez le juste, vous recevez des présents, et vous violez à la porte le droit des pauvres.
\VS{13}C'est pourquoi, en ce temps-ci, le sage se tait, car les temps sont mauvais.
\VS{14}Recherchez le bien et non le mal, afin que vous viviez ; et qu’ainsi Yahweh, le Dieu des armées, soit avec vous, comme vous le dites.
\VS{15}Haïssez le mal, et aimez le bien, faites régner la justice à la porte ; peut-être Yahweh, le Dieu des armées, aura pitié des restes de Joseph.
\TextTitle{Le jour de Yahweh}
\VS{16}C'est pourquoi ainsi parle Yahweh, le Dieu des armées, le Seigneur, parle ainsi : Dans toutes les places on se lamentera, dans toutes les rues on dira : Hélas ! Hélas ! On appellera au deuil le laboureur, et aux lamentations ceux qui savent gémir.
\VS{17}Dans toutes les vignes on se lamentera, quand je passerai au milieu de toi, dit Yahweh.
\VS{18}Malheur à ceux qui désirent le jour de Yahweh\FTNT{L’expression «~le jour du Seigneur~» ou «~le jour de Yahweh~» est une période durant laquelle Jésus-Christ interviendra ouvertement dans les affaires des hommes. Elle est utilisé dix-neuf fois dans le Tanak (Es. 2:12 ; Es. 13:6-9 ; Ez. 13:5 ; Ez. 30:3 ; Joë. 1:15 ; Joë. 2:1, 11, 31 ; Joë. 3:14 ; Am. 5:18, 20 ; Ab. 1:15 ; So. 1:7, 14 ; Za. 14:1 ; Mal. 4:5) et quatre fois dans le testament de Jésus (Ac. 2:20 ; 2 Th. 2:2 ; 2 Pi. 3:10). On y fait également allusion dans d’autres passages (Ap. 6:17 ; Ap. 16:14).}. Qu’attendez-vous du jour de Yahweh ? Il sera ténèbres, et non lumière.
\VS{19}Vous serez comme un homme qui fuit devant un lion et que rencontre un ours, ou qui entre dans sa maison, appuie sa main sur le mur et un serpent le mord.
\TextTitle{Yahweh hais l'adoration de ceux qui méprisent la justice et la droiture}
\VS{20}Le jour de Yahweh n’est-il pas ténèbres et non lumière ? Obscurité et non clarté ?
\VS{21}Je hais, je méprise vos fêtes, je ne prends pas plaisir à vos assemblées solennelles.
\VS{22}Si vous me présentez des holocaustes, je n’agréerai pas vos offrandes, je ne regarderai pas les bêtes grasses de vos offrandes de paix.
\VS{23}Eloigne de moi le bruit de tes cantiques ; je n’écouterai pas la mélodie de tes luths.
\VS{24}Mais que la droiture soit comme un courant d’eau, et la justice comme un torrent intarissable.
\VS{25}M’avez-vous présenté des sacrifices et des offrandes pendant quarante ans dans le désert, maison d'Israël ?
\VS{26}Au contraire, vous avez porté la tente de votre roi, et le piédestal de vos idoles, l'étoile de votre dieu que vous vous êtes fabriqué.
\VS{27}C'est pourquoi je vous transporterai au-delà de Damas, dit Yahweh, dont le nom est le Dieu des armées.
\Chap{6}
\TextTitle{Malheur à ceux qui prospèrent en temps d'injustice}
\VerseOne{}Hélas vous qui êtes à votre aise en Sion, et qui vous confiez en la montagne de Samarie, lieux les plus renommés d’entre les principaux des nations, auprès desquels va la maison d'Israël.
\VS{2}Passez à Calné, et regardez ; allez de là à Hamath la grande, et descendez à Gath chez les Philistins. Ces villes sont-elles plus prospères que vos deux royaumes, ou leur territoire est-il plus grand que votre territoire ?
\VS{3}Vous qui éloignez le jour du malheur, et qui approchez le règne de la violence.
\VS{4}Vous qui vous couchez sur des lits d'ivoire, et qui sont étendus sur vos coussins ; qui mangez les agneaux du troupeau, et les veaux pris du lieu où on les engraisse ;
\VS{5}qui fredonnez au son du luth ; qui inventez des instruments de musique comme David.
\VS{6}Qui buvez le vin dans de grandes coupes, et qui parfumez des parfums les plus exquis, et qui n’êtes pas affligés pour la plaie de Joseph.
\VS{7}C’est pourquoi ils vont être emmenés à la tête des captifs, et les cris de joie des voluptueux cesseront.
\VS{8}Le Seigneur, Yahweh, l’a juré par lui-même. Yahweh, Dieu des armées, dit : J'ai en horreur l'orgueil de Jacob, et je hais ses palais ; je livrerai la ville et tout ce qu’elle contient.
\VS{9}Et s'il reste dix hommes dans une maison, ils mourront.
\VS{10}Un proche parent prendra un mort et le brûlera pour emporter les os hors de la maison ; il dira à celui qui est au fond de la maison : Y a-t-il encore quelqu'un avec toi ? Et il répondra : Il n’y a plus personne. Puis il dira : Silence ! Ce n’est pas le moment de prononcer le nom de Yahweh.
\VS{11}Car voici, Yahweh ordonne : Et il fera tomber la grande maison en ruines, et la petite maison en débris.
\VS{12}Les chevaux courent-ils sur les rochers, y laboure-t-on avec des bœufs, pour que vous ayez changé la droiture en poison, et le fruit de la justice en absinthe ?
\VS{13}Vous vous réjouissez de choses qui ne sont que néant, vous dites : N’est-ce pas par notre force que nous avons acquis de la puissance ?
\VS{14}Voici, je ferai lever contre vous, maison d’Israël, dit Yahweh, le Dieu des armées, une nation qui vous opprimera depuis l'entrée de Hamath jusqu'au torrent du désert.
\Chap{7}
\TextTitle{Avertissement par des visions\FTNTT{Am. 8:1-9:10}}
\VerseOne{}Le Seigneur, Yahweh, me fit voir cette vision : Voici, il formait des sauterelles au temps où le regain commençait à croître ; et voici le regain poussait après les récoltes du roi.
\VS{2}Et quand elles eurent achevé de dévorer l'herbe de la terre, je dis : Seigneur Yahweh, pardonne, je te prie ! Comment Jacob subsistera-t-il ? Car il est faible.
\VS{3}Yahweh se repentit de cela. Cela n'arrivera pas, dit Yahweh.
\VS{4}Le Seigneur, Yahweh, me fit voir cette vision : Voici, le Seigneur, Yahweh, proclamait le jugement par le feu. Et le feu dévorait le grand abîme et dévorait les champs.
\VS{5}Et je dis : Seigneur Yahweh ! Arrête, je te prie ! Comment Jacob subsistera-t-il ? Car il est faible.
\VS{6}Yahweh se repentit de cela. Cela non plus n'arrivera pas, dit le Seigneur, Yahweh.
\VS{7}Il me fit voir cette vision : Voici, le Seigneur se tenait debout sur un mur fait au niveau, et il avait un niveau dans la main.
\VS{8}Et Yahweh me dit : Que vois-tu, Amos ? Et je répondis : Un niveau. Et le Seigneur me dit : Je mettrai le niveau au milieu de mon peuple d'Israël, je ne lui pardonnerai plus.
\VS{9}Et les hauts lieux d'Isaac seront ravagés, et les sanctuaires d'Israël seront détruits ; et je me lèverai contre la maison de Jéroboam avec l'épée.
\TextTitle{Amatsia accuse Amos devant Jéroboam}
\VS{10}Alors Amatsia, sacrificateur de Béthel, fit dire à Jéroboam roi d'Israël : Amos conspire contre toi au milieu de la maison d'Israël ; le pays ne saurait supporter toutes ses paroles.
\VS{11}Car voici ce que dit Amos : Jéroboam mourra par l'épée, et Israël sera emmené captif hors de son pays.
\VS{12}Et Amatsia dit à Amos : Voyant\FTNT{Voyant ou prophète.}, va-t’en, fuis dans le pays de Juda, et manges-y ton pain, et là tu prophétiseras.
\VS{13}Mais ne continue pas à prophétiser à Béthel\FTNT{Bethel, qui signifie «~maison de Dieu~», était devenue le sanctuaire du roi Jéroboam. De même aujourd’hui des églises du Seigneur sont devenues la propriété des hommes et les brebis de Dieu sont devenues la propriété des pasteurs.}, car c'est le sanctuaire du roi, et c'est une maison royale.
\TextTitle{Amos répond}
\VS{14}Amos répondit à Amatsia : Je n'étais ni prophète ni fils de prophète ; j'étais un berger, et je cueillais des figues sauvages.
\VS{15}Or Yahweh m’a pris derrière le troupeau, et Yahweh m’a dit : Va, prophétise à mon peuple d'Israël.
\VS{16}Ecoute maintenant la parole de Yahweh : Tu dis : Ne prophétise pas contre Israël, et ne parle pas contre la maison d'Isaac.
\VS{17}C'est pourquoi ainsi parle Yahweh : Ta femme se prostituera dans la ville, tes fils et tes filles tomberont par l'épée, ton champ sera partagé au cordeau, et toi, tu mourras sur une terre souillée, et Israël sera emmené captif hors de son pays.
\Chap{8}
\TextTitle{Vision de la corbeille de fruit, la fin pour le peuple d'Israël}
\VerseOne{}Le Seigneur, Yahweh, me fit voir cette vision : Voici, je vis un panier de fruits.
\VS{2}Il dit : Que vois-tu, Amos ? Et je répondis : Un panier de fruits. Et Yahweh me dit : La fin est venue pour mon peuple d'Israël, je ne continuerai plus à lui pardonner.
\VS{3}En ce jour-là, les chants du palais seront des gémissements, dit le Seigneur, Yahweh ; en tout lieu, il y aura beaucoup de cadavres que l'on jettera en silence.
\VS{4}Ecoutez ceci vous qui dévorez les pauvres et qui faites périr les malheureux du pays,
\VS{5}et qui dites : Quand la nouvelle lune sera-t-elle passée pour que nous vendions du blé ? Quand finira le sabbat pour que nous ouvrions les greniers ? Nous diminuerons l’épha, nous augmenterons le sicle, nous falsifierons les balances pour tromper,
\VS{6}nous achèterons les malheureux pour de l’argent, et le pauvre pour une paire de souliers, et nous vendrons la criblure du froment.
\VS{7}Yahweh l’a juré par la gloire de Jacob : Jamais je n’oublierai toutes leurs actions !
\VS{8}Le pays, à cause d’elles, ne sera-t-il pas ébranlé, et tous ses habitants ne seront-ils pas dans le deuil ? Le pays tout entier montera comme le fleuve. Il se soulèvera et s’affaissera comme le fleuve d'Egypte.
\VS{9}Il arrivera en ce jour-là, dit le Seigneur, Yahweh, que je ferai coucher le soleil à midi, et que j’obscurcirai la terre en plein jour.
\VS{10}Je changerai vos fêtes en deuil, et tous vos chants en lamentations ; je couvrirai de sacs tous les reins, et je rendrai chauves toutes les têtes ; je mettrai le pays dans le deuil comme pour un fils unique, et sa fin sera un jour d’amertume.
\VS{11}Voici, les jours viennent, dit le Seigneur, Yahweh, où j'enverrai la famine\FTNT{Nous sommes dans une époque où la Parole de Dieu, l’Evangile véritable a presque disparu au profit de l’évangile de prospérité. L’esprit de commerce a pris place au sein de beaucoup d’églises. C’est le temps de l’église de Laodicée, une église qui fait l’apologie de la richesse matérielle.} dans le pays ; non une famine de pain, ni une soif d'eau, mais d’entendre les paroles de Yahweh.
\VS{12}Ils erreront d’une mer jusqu'à l'autre, du nord à l'orient, ils iront çà et là pour chercher la parole de Yahweh, et ils ne la trouveront pas.
\VS{13}En ce jour-là, les belles vierges et les jeunes hommes mourront de soif.
\VS{14}Ceux qui jurent par le péché de Samarie disent : Vive ton Dieu, ô Dan ! Vive la voie de Beer-Schéba ! Mais ils tomberont et ne se relèveront plus.
\Chap{9}
\TextTitle{Prophétie annonçant la destruction\FTNTT{De. 28:63-68.}}
\VerseOne{}Je vis le Seigneur qui se tenait debout sur l'autel. Et il dit : Frappe le chapiteau et que les seuils s’ébranlent ; et brise-les sur leurs têtes à tous ! Je tuerai par l'épée ce qui restera d'eux. Il ne s’enfuira pas un fugitif, il ne s’échappera pas un fuyard.
\VS{2}S’ils pénètrent dans le séjour des morts, ma main les enlèvera de là ; s’ils montent aux cieux, je les en ferai descendre.
\VS{3}S’ils se cachent au sommet du Carmel, je les y rechercherai et je les enlèverai de là ; s’ils se dérobent à mes yeux dans le fond de la mer, là j’ordonnerai au serpent de les mordre.
\VS{4}Lorsqu'ils s'en iront en captivité devant leurs ennemis, là j’ordonnerai à l’épée de les tuer ; je fixerai mon regard sur eux pour leur faire du mal et non du bien.
\VS{5}Le Seigneur, Yahweh des armées, touche la terre, et elle tremble, et tous ses habitants sont dans le deuil ; elle monte tout entière comme le fleuve, et elle s’affaisse comme le fleuve d'Egypte.
\VS{6}Il a bâti sa demeure dans les cieux, et fondé sa voûte sur la terre ; il appelle les eaux de la mer, et les répand sur la surface de la Terre. Son nom est Yahweh.
\VS{7}N'êtes-vous pas pour moi comme les enfants des Ethiopiens, enfants d'Israël ? dit Yahweh. N'ai-je pas fait monter Israël du pays d'Egypte, les Philistins de Caphtor et les Syriens de Kir ?
\VS{8}Voici, les yeux du Seigneur, Yahweh, sont sur ce royaume pécheur. Je le détruirai de dessus la surface de la terre. Cependant, je ne détruirai pas entièrement la maison de Jacob, dit Yahweh.
\VS{9}Car voici, je donnerai mes ordres, et je secouerai la maison d'Israël parmi toutes les nations, comme on secoue le blé dans le crible, sans qu'il en tombe un grain à terre.
\VS{10}Tous les pécheurs de mon peuple mourront par l'épée, ceux qui disent : Le mal n'approchera pas, il ne nous atteindra pas.
\TextTitle{Yahweh relève la maison de David}
\VS{11}En ce temps-là, je relèverai le tabernacle de David qui est tombé, j’en réparerai les brèches, j’en redresserai les ruines, et je le rebâtirai comme il était autrefois,
\VS{12}afin qu'ils possèdent le reste d’Edom et toutes les nations sur lesquelles mon nom a été invoqué, dit Yahweh, qui accomplira cela.
\TextTitle{Restauration d'Isarël}
\VS{13}Voici, les jours viennent, dit Yahweh, où le laboureur suivra de près le moissonneur, et celui qui foule les raisins atteindra celui qui répand la semence ; et le moût ruissellera des montagnes et découlera de toutes les collines.
\VS{14}Je ramènerai les captifs de mon peuple d'Israël ; ils rebâtiront les villes dévastées, et y habiteront, ils planteront des vignes, et en boiront le vin ; ils feront des jardins et en mangeront les fruits.
\VS{15}Je les planterai sur leur terre, et ils ne seront plus arrachés du pays que je leur ai donné\FTNT{Cette prophétie annonce la restauration de la maison de David. Personne ne chassera Israël de sa terre, aucune nation n’a le pouvoir de le déloger, car c’est le Seigneur qui l’a établi.}, dit Yahweh, ton Dieu.
\PPE{}
\end{multicols}
