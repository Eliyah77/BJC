\ShortTitle{1 Ch.}\BookTitle{1 Chroniques}\BFont
\noindent\hrulefill
{\footnotesize
\textit{
\bigskip
{\centering{}
\\Auteur~: Probablement Esdras
\\(Heb.~: Hayyamim dibre)
\\Signification~: Actes des journées
\\Thème~: Généalogies et Histoire
\\Date de rédaction~: 5\up{ème} siècle av. J.-C.\\}
}
\textit{
\\Les deux livres des Chroniques constituent des compléments aux livres des Rois dans la mesure où ils confirment les récits de ceux-ci.
\\Après avoir établi la généalogie d'Adam à Jacob, puis une généalogie plus détaillée de la descendance de Jacob jusqu'au retour de la captivité babylonienne, le premier livre des Chroniques reprend l'histoire du roi David et met un accent particulier sur certains combats qu'il eut à mener, les rapports avec ses serviteurs, ainsi que les préparatifs de la construction du temple. Il présente aussi l'organisation du travail des prêtres et des Lévites au service de Dieu et du peuple.\bigskip
}
}
\par\nobreak\noindent\hrulefill
\begin{multicols}{2}
\Chap{1}
\TextTitle{Généalogie d'Adam à Noé\FTNTT{Ge. 5:1-32.}}
\VerseOne{}Adam, Seth, Enosch\FTNT{Les généalogies se faisaient par les premiers-nés de chaque famille.}.
\VS{2}Kénan, Mahalaleel, Jéred~;
\VS{3}Hénoc, Metuschélah, Lémec.
\VS{4}Noé, Sem, Cham et Japhet\FTNT{Ge. 5:1-32.}.
\TextTitle{Les fils de Japhet\FTNTT{Ge. 10:2-5.}}
\VS{5}Les fils de Japhet furent~: Gomer, Magog, Madaï, Javan, Tubal, Méschec et Tiras.
\VS{6}Les fils de Gomer furent~: Aschkenaz, Diphat et Togarma.
\VS{7}Les fils de Javan furent~: Elischa, Tarsisa, Kittim et Rodanim.
\TextTitle{Les fils de Cham\FTNTT{Ge. 10:6-20.}}
\VS{8}Les fils de Cham furent~: Cusch, Mitsraïm, Puth et Canaan.
\VS{9}Les fils de Cusch furent~: Saba, Havila, Sabta, Raema et Sabteca. Les fils de Raema furent~: Séba et Dedan.
\VS{10}Cusch engendra aussi Nimrod qui commença à être puissant sur la terre.
\VS{11}Mitsraïm engendra les Ludim, les Anamim, les Lehabim, les Naphtuhim,
\VS{12}les Patrusim, les Casluhim, desquels sont issus les Philistins et les Caphtorim.
\VS{13}Canaan engendra Sidon, son fils aîné, et Heth~;
\VS{14}les Jébusiens, les Amoréens, les Guirgasiens,
\VS{15}les Héviens, les Arkiens, les Siniens,
\VS{16}les Arvadiens, les Tsemariens et les Hamathiens.
\TextTitle{Les fils de Sem\FTNTT{Ge. 10:21-31.}}
\VS{17}Les fils de Sem furent~: Elam, Assur, Arpacschad, Lud, Aram, Uts, Hul, Guéter et Méschec.
\VS{18}Arpacschad engendra Schélach, et Schélach engendra Héber.
\VS{19}A Héber naquirent deux fils~: L'un s'appelait Péleg, car en son temps la terre fut partagée~; et son frère s'appelait Jokthan.
\VS{20}Jokthan engendra Almodad, Schéleph, Hatsarmaveth, Jérach,
\VS{21}Hadoram, Uzal, Dikla,
\VS{22}Ebal, Abimaël, Séba,
\VS{23}Ophir, Havila et Jobab~; tous ceux-là furent des fils de Jokthan\FTNT{Ge. 10:2-31.}.
\TextTitle{De Sem aux fils d'Abraham\FTNTT{Ge. 11:10-26.}}
\VS{24}Sem, Arpacschad, Schélach\FTNT{Ge. 11:10-26.},
\VS{25}Héber, Péleg, Rehu,
\VS{26}Serug, Nachor, Térach,
\VS{27}et Abram, qui est Abraham.
\VS{28}Les fils d'Abraham furent Isaac et Ismaël.
\TextTitle{Les fils d'Ismaël\FTNTT{Ge. 25:12-18.}}
\VS{29}Voici leur postérité\FTNT{Ge. 25:12-18.}~: Le premier-né d'Ismaël fut Nebajoth, puis Kédar, Adbeel, Mibsam,
\VS{30}Mischma, Duma, Massa, Hadad, Téma,
\VS{31}Jethur, Naphisch et Kedma~; ce sont là les fils d'Ismaël.
\TextTitle{Les fils de Ketura\FTNTT{Ge. 25:1-4.}}
\VS{32}Quant aux fils de Ketura, concubine d'Abraham, elle enfanta Zimran, Jokschan, Medan, Madian, Jischbak et Schuach~; et les fils de Jokschan furent Séba et Dedan.
\VS{33}Les fils de Madian furent Epha, Epher, Hénoc, Abida et Eldaa. Tous ceux-là furent les fils de Ketura.
\TextTitle{Les fils d'Isaac\FTNTT{Ge. 25:19-26.}}
\VS{34}Or Abraham engendra Isaac~; et les fils d'Isaac furent Esaü et Israël.
\TextTitle{Les descendants d'Esaü \FTNTT{Ge. 36:1-14.}}
\VS{35}Les fils d'Esaü furent Eliphaz, Reuel, Jeusch, Jaelam et Koré\FTNT{Ge. 36:1-14.}.
\VS{36}Les fils d'Eliphaz furent Théman, Omar, Tsephi, Gaetham et Kenaz~; Thimna lui enfanta Amalek.
\VS{37}Les fils de Reuel furent Nahath, Zérach, Schamma et Mizza.
\VS{38}Les fils de Séir furent Lothan, Schobal, Tsibeon, Ana, Dischon, Etser et Dischan.
\VS{39}Les fils de Lothan furent Hori et Homam~; et Thimna fut la sœur de Lothan.
\VS{40}Les fils de Schobal furent Aljan, Manahath, Ebal, Schephi et Onam. Les fils de Tsibeon furent Ajja et Ana.
\VS{41}Ana eut un fils~: Dischon. Les fils de Dischon furent Hamran, Eschban, Jithran et Keran.
\VS{42}Les fils d'Etser furent Bilhan, Zaavan et Jaakan. Les fils de Dischon furent Uts et Aran.
\TextTitle{Les rois et les chefs d'Edom\FTNTT{\vref{Ge. 36:15-19,25-43}.}}
\VS{43}Voici les rois qui ont régné au pays d'Edom, avant qu'un roi ne règne sur les enfants d'Israël~: Béla, fils de Beor, et le nom de sa ville était Dinhaba.
\VS{44}Béla mourut, et Jobab, fils de Zérach de Botsra, régna à sa place.
\VS{45}Jobab mourut, et Huscham, du pays des Thémanites, régna à sa place.
\VS{46}Huscham mourut, et Hadad, fils de Bedad, régna à sa place. C'est lui qui frappa Madian dans les champs de Moab. Le nom de sa ville était Avith.
\VS{47}Hadad mourut, et Samla de Masréka, régna à sa place.
\VS{48}Samla mourut, et Saül de Rehoboth, sur le fleuve, régna à sa place.
\VS{49}Saül mourut, et Baal-Hanan, fils de Acbor, régna à sa place.
\VS{50}Baal-Hanan mourut, et Hadad régna à sa place. Le nom de sa ville était Pahi, et le nom de sa femme Mehéthabeel, qui était fille de Mathred, et petite- fille de Mézahab.
\VS{51}Enfin Hadad mourut. Ensuite vinrent les chefs d'Edom, le chef Thimna, le chef Alja, le chef Jetheth.
\VS{52}Le chef Oholibama, le chef Ela, le chef Pinon.
\VS{53}Le chef Kenaz, le chef Théman, le chef Mibtsar.
\VS{54}Le chef Magdiel, et le chef Iram. Ce sont là les chefs d'Edom.
\Chap{2}
\TextTitle{Les douze fils de Jacob (Israël)\FTNTT{Ge. 29:31-35~; 30:6-24~; 35:16-18.}}
\VerseOne{}Voici les fils d'Israël~: Ruben, Siméon, Lévi, Juda, Issacar, Zabulon,
\VS{2}Dan, Joseph, Benjamin, Nephthali, Gad et Aser.
\TextTitle{Les descendants de Juda jusqu'aux fils d'Hetsron\FTNTT{Ge. 46:12~; No. 26:19-22.}}
\VS{3}Les fils de Juda furent Er, Onan, et Schéla. Ces trois lui naquirent de la fille de Schua, la Cananéenne. Mais Er, premier-né de Juda, fut méchant aux yeux de Yahweh, qui le fit mourir.
\VS{4}Et Tamar, belle-fille de Juda, lui enfanta Pérets et Zérach. Tous les fils de Juda furent cinq.
\VS{5}Les fils de Pérets furent Hetsron et Hamul.
\VS{6}Et les fils de Zérach furent Zimri, Ethan, Héman, Calcol et Dara, cinq en tout.
\VS{7}Carmi n'eut point d'autre fils qu'Acar qui troubla Israël et qui pécha en prenant de l'interdit.
\VS{8}Ethan eut un seul fils~: Azaria.
\VS{9}Les fils qui naquirent à Hetsron furent Jerachmeel, Ram et Kelubaï.
\TextTitle{Les descendants de Ram jusqu'à David\FTNTT{Ru. 4:17-22.}}
\VS{10}Ram engendra Amminadab et Amminadab engendra Nachschon, chef des fils de Juda.
\VS{11}Nachschon engendra Salma et Salma engendra Boaz.
\VS{12}Boaz engendra Obed et Obed engendra Isaï.
\VS{13}Isaï engendra son premier-né Eliab, le second Abinadab, le troisième Schimea,
\VS{14}le quatrième Nethaneel, le cinquième Raddaï,
\VS{15}le sixième Otsem, et le septième, David.
\VS{16}Tseruja et Abigaïl furent leurs sœurs. Tseruja eut trois fils~: Abischaï, Joab, et Asaël.
\VS{17}Abigaïl enfanta Amasa, dont le père fut Jéther l'Ismaélite.
\TextTitle{Les descendants de Caleb}
\VS{18}Or Caleb, fils de Hetsron, eut des enfants d'Azuba sa femme, et aussi de Jerioth~; et ses fils furent Jéscher, Schobab et Ardon.
\VS{19}Azuba mourut, et Caleb prit pour femme Ephrath, qui lui enfanta Hur.
\VS{20}Hur engendra Uri, et Uri engendra Betsaleel.
\VS{21}Après cela, Hetsron vint vers la fille de Makir, père de Galaad, et la prit pour sa femme~; il était âgé de soixante ans, et elle lui enfanta Segub.
\VS{22}Segub engendra Jaïr, qui eut vingt-trois villes au pays de Galaad.
\VS{23}Il prit sur Gueschur et sur la Syrie les bourgades de Jaïr, et Kenath, avec les villes de son ressort, au nombre de soixante. Tous ceux-là furent fils de Makir, père de Galaad.
\VS{24}Après la mort de Hetsron, à Caleb-Ephratha, la femme de Hetsron, Abija, lui enfanta Aschchur, père de Tekoa.
\VS{25}Les fils de Jerachmeel, premier-né de Hetsron furent~: Ram, son fils aîné, puis Buna, Oren et Otsem, nés d'Achija.
\VS{26}Jerachmeel eut aussi une autre femme, dont le nom était Athara, qui fut mère d'Onam.
\VS{27}Les fils de Ram, premier-né de Jerachmeel, furent Maats, Jamin et Eker.
\VS{28}Les fils d'Onam furent Schammaï et Jada~; et les fils de Schammaï furent Nadab et Abischur.
\VS{29}Le nom de la femme d'Abischur fut Abichaïl, qui lui enfanta Achban et Molid.
\VS{30}Les fils de Nadab furent Séled et Appaïm~; mais Séled mourut sans fils.
\VS{31}Appaïm eut un seul fils~: Jischeï. Jischeï eut un seul fils~: Schéschan. Schéschan n'eut qu'Achlaï.
\VS{32}Les fils de Jada, frère de Schammaï, furent Jéther et Jonathan~; mais Jéther mourut sans fils.
\VS{33}Les fils de Jonathan furent Péleth et Zara~; ce furent là les fils de Jerachmeel.
\VS{34}Schéschan n'eut point de fils, mais des filles~; or il avait un serviteur Egyptien, dont le nom était Jarcha~;
\VS{35}Schéschan donna sa fille pour femme à Jarcha, son serviteur, et elle lui enfanta Attaï.
\VS{36}Attaï engendra Nathan, et Nathan engendra Zabad~;
\VS{37}Zabad engendra Ephlal~; et Ephlal engendra Obed~;
\VS{38} Obed engendra Jéhu~; Jéhu engendra Azaria~;
\VS{39}Azaria engendra Halets~; Halets engendra Elasa~;
\VS{40}Elasa engendra Sismaï~; Sismaï engendra Schallum~;
\VS{41}Schallum engendra Jekamja~; Jekamja engendra Elischama.
\TextTitle{Les autres fils de Caleb}
\VS{42}Les fils de Caleb, frère de Jerachmeel, furent Méscha, son premier-né, qui fut le père de Ziph, et les fils de Maréscha, père d'Hébron.
\VS{43}Les fils d'Hébron furent Koré, Thappuach, Rékem et Schéma.
\VS{44}Schéma engendra Racham, père de Jorkeam, et Rékem engendra Schammaï.
\VS{45}Le fils de Schammaï fut Maon. Maon fut père de Beth-Tsur.
\VS{46}Et Epha, concubine de Caleb, enfanta Haran, Motsa et Gazez~; Haran aussi engendra Gazez.
\VS{47}Les fils de Jahdaï furent Réguem, Jotham, Guéschan, Péleth, Epha et Schaaph.
\VS{48}Maaca, la concubine de Caleb, enfanta Schéber et Tirchana.
\VS{49}La femme de Schaaph, père de Madmanna, enfanta Scheva, père de Macbéna, et le père de Guibea, et la fille de Caleb fut Acsa.
\TextTitle{Les descendants de Hur, fils de Caleb\FTNTT{Cp. 1 Ch. 4:1.}}
\VS{50}Ceux-ci furent les fils de Caleb, fils de Hur, premier-né d'Ephrata~: Schobal, père de Kirjath-Jearim.
\VS{51}Salma, père de Bethléhem, Hareph, père de Beth-Gader.
\VS{52}Schobal, père de Kirjath-Jearim, eut des fils~: Haroé et Hatsi-Hammenuhoth.
\VS{53}Les familles de Kirjath-Jearim furent les Jéthriens, les Puthiens, les Schumathiens et les Mischraïens, desquels sont sortis les Tsoreathiens et les Eschthaoliens.
\VS{54}Les fils de Salma~: Bethléhem et les Nethophatiens, Athroth-Beth-Joab, Hatsi-Hammanachthi et les Tsoreïens.
\VS{55}Et les familles des scribes, qui habitaient à Jaebets~: Les Thireathiens, les Schimeathiens, les Sucathiens~; ce sont les Kéniens, qui sont sortis de Hamath père de Récab.
\Chap{3}
\TextTitle{Les fils de David\FTNTT{2 S. 3:2-5~; 5:13-16.}}
\VerseOne{}Voici les fils de David, qui lui naquirent à Hébron\FTNT{2 S. 3:2-5.}. Le premier-né fut Amnon, fils d' Achinoam de Jizreel~; le second Daniel, d'Abigaïl de Carmel.
\VS{2}Le troisième, Absalom, fils de Maaca, fille de Talmaï, roi de Gueschur~; le quatrième, Adonija, fils de Haggith~;
\VS{3}le cinquième, Schephatia, d'Abithal~; le sixième, Jithream, d'Egla sa femme.
\VS{4}Ces six lui naquirent à Hébron, où il régna sept ans et six mois~; puis il régna trente-trois ans à Jérusalem.
\VS{5}Ceux-ci lui naquirent à Jérusalem~: Schimea, Schobab, Nathan et Salomon, tous quatre de Bath-Schua, fille d'Ammiel~;
\VS{6}et Jibhar, Elischama, Eliphéleth,
\VS{7}Noga, Népheg, Japhia,
\VS{8}Elischama, Eliada et Eliphéleth, qui sont neuf.
\VS{9}Ce sont tous des fils de David, outre les fils de ses concubines. Et Tamar était leur sœur.
\TextTitle{De Salomon à Sédécias}
\VS{10}Le fils de Salomon fut Roboam. Abija, son fils~; Asa, son fils~; Josaphat, son fils~;
\VS{11}Joram, son fils~; Achazia, son fils~; Joas, son fils~;
\VS{12}Amatsia, son fils~; Azaria, son fils~; Jotham, son fils~;
\VS{13}Achaz, son fils~; Ezéchias, son fils~; Manassé, son fils~;
\VS{14}Amon, son fils~; Josias, son fils.
\VS{15}Les fils de Josias furent Jochanan, son premier-né~; le deuxième, Jojakim~; le troisième Sédécias~; le quatrième, Schallum.
\VS{16}Les fils de Jojakim furent Jéconias, son fils, qui eut pour fils Sédécias.
\TextTitle{Les fils de Jéconias}
\VS{17}Quant aux fils de Jéconias, Assir qui fut emmené en captivité, Schealthiel fut son fils~;
\VS{18}dont les fils furent Malkiram, Pedaja, Schénatsar, Jekamia, Hoschama et Nedabia.
\VS{19}Les fils de Pedaja furent Zorobabel et Schimeï~; et les fils de Zorobabel furent Meschullam et Hanania~; et Schelomith était leur sœur.
\VS{20}De Meschullam, Haschuba, Ohel, Bérékia, Hasadia et Juschab-Hésed, en tout cinq.
\VS{21}Les fils de Hanania furent Pelathia et Esaïe~; les fils de Rephaja, les fils d'Arnan, les fils d'Abdias et les fils de Schecania.
\VS{22}De Schecania naquit Schemaeja~; et les fils de Schemaeja, Hattusch, Jigueal, Bariach, Nearia, Schaphath, en tout six.
\VS{23}Les fils de Nearia furent trois~: Eljoénaï, Ezéchias et Azrikam.
\VS{24}Et les fils d'Eljoénaï furent sept~: Hodavia, Eliaschib, Pelaja, Akkub, Jochanan, Delaja, et Anani.
\Chap{4}
\TextTitle{Les autres fils de Hur\FTNTT{1 Ch. 2:50.}}
\VerseOne{}Les fils de Juda furent Pérets, Hetsron, Carmi, Hur et Schobal.
\VS{2}Reaja, fils de Schobal, engendra Jachath~; et Jachath engendra Achumaï et Lahad. Ce sont les familles des Tsoreathiens.
\VS{3}Voici les descendants du père d'Etham~: Jizreel, Jischma, et Jidbasch~; le nom de leur sœur était Hatselelponi.
\VS{4}Penuel, père de Guedor, et Ezer, père de Huscha, sont les fils de Hur, premier-né d'Ephrata, père de Bethléhem.
\TextTitle{Les descendants d'Aschchur\FTNTT{1 Ch. 2:24.}}
\VS{5}Aschchur, père de Tekoa, eut deux femmes~: Hélea et Naara.
\VS{6}Naara lui enfanta Achuzzam, Hépher, Thémeni et Achaschthari. Ce sont là les fils de Naara.
\VS{7}Les fils de Hélea furent Tséreth, Tsochar et Ethnan.
\VS{8}Kots engendra Anub, Hatsobéba et les familles Acharchel, fils de Harum.
\VS{9}Entre lesquels il eut Jaebets plus distingué que ses frères~; sa mère lui avait donné le nom de Jaebets, parce que, dit-elle, je l'ai enfanté avec douleur.
\TextTitle{Jaebets invoque Dieu}
\VS{10}Jaebets invoqua le Dieu d'Israël, en disant~: Ô, si tu me bénis abondamment et que tu étends mes limites, si ta main est avec moi, et si tu me mets à l'abri du mal, en sorte que je ne sois pas dans l'affliction~!… Et Dieu lui accorda ce qu'il avait demandé.
\TextTitle{Les fils de Juda et de Caleb}
\VS{11}Kelub, frère de Schucha, engendra Mechir, qui fut père d'Eschthon.
\VS{12}Et Eschthon engendra la maison de Rapha, Paséach et Thechinna, père de la ville de Nachasch~; ce sont là les gens de Réca.
\VS{13}Les fils de Kenaz furent Othniel et Seraja. Et le fils d'Othniel, Hathath.
\VS{14}Meonothaï engendra Ophra~; et Seraja engendra Joab, père de la vallée des ouvriers~; car ils étaient ouvriers.
\VS{15}Les fils de Caleb, fils de Jephunné, furent Iru, Ela et Naam, et les fils d'Ela, Kenaz.
\VS{16}Les fils de Jehalléleel furent Ziph, Zipha, Thirja, et Asareel.
\VS{17}Les fils d'Esdras furent Jéther, Méred, Epher, et Jalon~; et la femme de Méred enfanta Miriam, Schammaï, et Jischbach, père d'Eschthemoa.
\VS{18}Sa femme Jéhudija enfanta Jéred, père de Guedor~; Héber, père de Soco~; Jekuthiel, père de Zanoach. Ceux-là sont les fils de Bithja, fille de Pharaon, que Méred prit pour femme.
\VS{19}Les fils de la femme de Hodija, sœur de Nacham~: Le père de Kehila, le Garmien, et Eschthemoa, le Maacathien.
\VS{20}Et les fils de Simon furent Amnon, Rinna, Ben-Hanan et Thilon. Les fils de Jischeï furent Zocheth et Ben-Zocheth.
\TextTitle{Les fils de Juda par Schéla\FTNTT{1 Ch. 2:3.}}
\VS{21}Les fils de Schéla, fils de Juda, furent Er, père de Léca~; Laeda, père de Maréscha~; et les familles de la maison où l'on travaille le byssus, qui sont de la maison d'Aschbéa.
\VS{22}Jokim, et les gens de Cozéba, Joas et Saraph dominèrent sur Moab, avec Jaschubi-Léchem. Mais ce sont là des choses anciennes.
\VS{23}C'étaient les potiers et les habitants des plantations et des parcs. Ils cohabitaient là chez le roi et œuvraient pour lui.
\TextTitle{Les descendants de Siméon~; leurs terres et leurs conquêtes}
\VS{24}Les fils de Siméon furent Nemuel, Jamin, Jarib, Zérach et Saül.
\VS{25}Schallum son fils, Mibsam son fils, et Mischma son fils.
\VS{26}Les fils de Mischma furent Hammuel son fils, Zaccur son fils, et Schimeï son fils.
\VS{27}Schimeï eut seize fils et six filles~; mais ses frères n'eurent pas beaucoup de fils, et toute leur famille ne put être aussi nombreuse que celle des fils de Juda.
\VS{28}Ils habitèrent à Beer-Schéba, à Molada, à Hatsar-Schual,
\VS{29}à Bilha, à Etsem, à Tholad,
\VS{30}à Bethuel, à Horma, à Tsiklag,
\VS{31}à Beth-Marcaboth, à Hatsar-Susim, à Beth-Bireï, et à Schaaraïm. Ce furent là leurs villes jusqu'au temps où David devint roi.
\VS{32}Leurs villages furent Etham, Aïn, Rimmon, Thoken, et Aschan, cinq villes~;
\VS{33}et tous leurs villages, qui étaient autour de ces villes-là, jusqu'à Baal. Ce sont là leurs habitations et leur généalogie~:
\VS{34}Meschobab, Jamlec, Joscha fils d'Amatsia~;
\VS{35}Joël, Jéhu fils de Joschibia, fils de Seraja, fils d'Asiel~;
\VS{36}Eljoénaï, Jaakoba, Jeschochaja, Asaja, Adiel, Jesimiel, Benaja,
\VS{37}Ziza, fils de Schipheï, fils d'Allon, fils de Jedaja, fils de Schimri, fils de Schemaeja.
\VS{38}Ceux-là furent désignés pour être des chefs dans leurs familles, et les maisons de leurs pères s'étendirent abondamment.
\VS{39}Et ils allèrent pour entrer dans Guedor, jusqu'à l'orient de la vallée, cherchant des pâturages pour leurs troupeaux.
\VS{40}Ils trouvèrent des pâturages gras et bons, et un pays spacieux, paisible et fertile~; car ceux qui habitaient là auparavant étaient descendus de Cham.
\VS{41}Ceux-ci, dont les noms sont inscrits, vinrent du temps d'Ezéchias, roi de Juda, et abattirent leurs tentes~; et quant aux Maonites qui s'y trouvaient, ils les détruisirent à la façon de l'interdit jusqu'à ce jour, et y habitèrent à leur place, car il y avait là des pâturages pour leurs troupeaux.
\VS{42}Cinq cents hommes d'entre eux, c'est-à-dire des fils de Siméon, s'en allèrent à la montagne de Séir, et ils avaient à leur tête Pelathia, Nearia, Rephaja, et Uziel, fils de Jischeï~;
\VS{43}ils frappèrent le reste des réchappés d'Amalek, et ils demeurèrent là jusqu'à ce jour.
\Chap{5}
\TextTitle{Les descendants de Ruben jusqu'au temps des captivités}
\VerseOne{}Les fils de Ruben, le premier-né d'Israël, car il était le premier-né~; mais après qu'il eut souillé le lit de son père, son droit d'aînesse fut donné aux fils de Joseph, fils d'Israël~; cependant, Joseph ne fut pas enregistré dans la généalogie selon le droit d'aînesse.
\VS{2}Car Juda fut le plus puissant parmi ses frères, et de lui est issu un chef~; mais le droit d'aînesse est à Joseph.
\VS{3}Les fils de Ruben, premier-né d'Israël, furent donc Hénoc, Pallu, Hetsron, et Carmi.
\VS{4}Les fils de Joël furent Schemaeja, son fils~; Gog, son fils~; Schimeï, son fils~;
\VS{5}Michée, son fils~; Reaja, son fils~; Baal, son fils~;
\VS{6}Beéra, son fils, qui fut emmené captif par Tilgath-Pilnéser, roi d'Assyrie~; c'est lui qui était le principal chef des Rubénites.
\VS{7}Ses frères, selon leurs familles, d'après le registre généalogique et selon leurs générations, avaient pour chefs Jeïel et Zacharie.
\VS{8}Béla, fils d'Azaz, fils de Schéma, fils de Joël, habitait depuis Aroër jusqu'à Nebo et Baal-Meon.
\VS{9}Ensuite, il habita du côté de l'orient jusqu'à l'entrée du désert, depuis le fleuve d'Euphrate~; car son bétail s'était multiplié dans le pays de Galaad.
\VS{10}Du temps de Saül, ils firent la guerre contre les Hagaréniens, qui tombèrent par leurs mains, et ils habitèrent dans leurs tentes, dans toute la partie orientale de Galaad.
\TextTitle{Les descendants de Gad et leurs villes}
\VS{11}Les fils de Gad habitaient près d'eux, au pays de Basan, jusqu'à Salca.
\VS{12}Joël fut le premier chef, et Schapham le deuxième après lui, puis Jaenaï, puis Schaphath en Basan.
\VS{13}Et leurs frères, selon la maison de leurs pères, furent sept~: Micaël, Meschullam, Schéba, Joraï, Jaecan, Zia, et Eber.
\VS{14}Ceux-ci furent les fils d'Abichaïl, fils de Huri, fils de Jaroach, fils de Galaad, fils de Micaël, fils de Jeschischaï, fils de Jachdo, fils de Buz.
\VS{15}Achi, fils d'Abdiel, fils de Guni, fut le chef de la maison de leurs pères.
\VS{16}Ils habitèrent en Galaad, et en Basan, dans les villes de son ressort, et dans tous les faubourgs de Saron, jusqu'à leurs limites.
\VS{17}Tous ceux-ci furent inscrits dans la généalogie du temps de Jotham, roi de Juda, et du temps de Jéroboam, roi d'Israël.
\TextTitle{Captivité de Ruben, Gad et la demi-tribu de Manassé}
\VS{18}Il y eut des fils de Ruben, et de ceux de Gad, et de la demi-tribu de Manassé, d'entre les vaillants hommes, portant le bouclier et l'épée, tirant de l'arc, et exercés à la guerre, quarante-quatre mille sept cent soixante, en état d'aller à l'armée.
\VS{19}Ils firent la guerre contre les Hagaréniens, contre Jethur, Naphisch, et Nodab.
\VS{20}Et ils reçurent du secours contre eux, de sorte que les Hagaréniens, et tous ceux qui étaient avec eux furent livrés entre leurs mains, parce qu'ils crièrent à Dieu dans la bataille, et il les exauça parce qu'ils avaient mis leur confiance en lui.
\VS{21}Ainsi ils prirent leurs troupeaux, consistant en cinquante mille chameaux, deux cent cinquante mille brebis, deux mille ânes, avec cent mille personnes~;
\VS{22}car il y eut beaucoup de morts, parce que la bataille venait de Dieu. Ils habitèrent là, à leur place, jusqu'au temps de la déportation.
\VS{23}Les fils de la demi-tribu de Manassé habitèrent aussi dans ce pays-là, et s'étendirent depuis Basan jusqu'à Baal-Hermon et à Sénir, à la montagne d'Hermon~; ils étaient nombreux.
\VS{24}Et voici les chefs de la maison de leurs pères~: Epher, Jischeï, Eliel, Azriel, Jérémie, Hodavia, et Jachdiel, hommes forts et vaillants, gens de réputation, et chefs des maisons de leurs pères.
\VS{25}Mais ils péchèrent contre le Dieu de leurs pères, et se prostituèrent après les dieux des peuples du pays, que Dieu avait détruits devant eux.
\VS{26}Le Dieu d'Israël excita l'esprit de Pul, roi d'Assyrie, et l'esprit de Thilgath-Pilnéser, roi d'Assyrie, qui emmena en captivité les Rubénites, les Gadites et la demi-tribu de Manassé, et les emmena à Chalach, à Chabor, à Hara, et au fleuve de Gozan, où ils sont restés jusqu'à ce jour.
\Chap{6}
\TextTitle{Les fils de Kehath le Lévite, jusqu'à la captivité}
\VerseOne{}Les fils de Lévi furent Guerschon, Kehath et Merari.
\VS{2}Les fils de Kehath furent Amram, Jitsehar, Hébron, et Uziel.
\VS{3}Et les fils d'Amram furent Aaron, Moïse et Marie. Les fils d'Aaron furent Nadab, Abihu, Eléazar et Ithamar.
\VS{4}Eléazar engendra Phinées, et Phinées engendra Abischua.
\VS{5}Abischua engendra Bukki, et Bukki engendra Uzzi.
\VS{6}Uzzi engendra Zerachja, et Zerachja engendra Merajoth.
\VS{7}Merajoth engendra Amaria, et Amaria engendra Achithub.
\VS{8}Achithub engendra Tsadok, et Tsadok engendra Achimaats.
\VS{9}Achimaats engendra Azaria, et Azaria engendra Jochanan.
\VS{10}Jochanan engendra Azaria, qui exerça la prêtrise au temple que Salomon bâtit à Jérusalem.
\VS{11}Azaria engendra Amaria, et Amaria engendra Achithub.
\VS{12}Achithub engendra Tsadok, et Tsadok engendra Schallum.
\VS{13}Schallum engendra Hilkija, et Hilkija engendra Azaria.
\VS{14}Azaria engendra Seraja, et Seraja engendra Jehotsadak,
\VS{15}Jehotsadak s'en alla, quand Yahweh emmena en exil Juda et Jérusalem par le moyen de Nebucadnetsar.
\TextTitle{Les fils de Guerschon, Kehath et Mérari}
\VS{16}Les fils de Lévi furent donc Guerschon, Kehath et Merari.
\VS{17}Voici les noms des fils de Guerschon~: Libni et Schimeï.
\VS{18}Les fils de Kehath furent Amram, Jitsehar, Hébron et Uziel.
\VS{19}Les fils de Merari furent Machli et Muschi. Ce sont là les familles des Lévites, selon les maisons de leurs pères.
\VS{20}De Guerschon, Libni son fils, Jachath son fils, Zimma son fils,
\VS{21}Joach son fils, Iddo son fils, Zérach son fils, Jeathraï son fils.
\VS{22}Des fils de Kehath, Amminadab son fils, Koré son fils, Assir son fils,
\VS{23}Elkana son fils, Ebjasaph son fils, Assir son fils,
\VS{24}Thachath son fils, Uriel son fils, Ozias son fils, et Saül son fils.
\VS{25}Les fils d'Elkana furent Amasaï, Achimoth~;
\VS{26}Elkana, son fils~; les fils d'Elkana furent Elkana-Tsophaï, son fils, Nachath son fils,
\VS{27}Eliab son fils, Jerocham son fils, Elkana son fils.
\VS{28}Quant aux fils de Samuel, fils d'Elkana, son fils aîné fut Vaschni, puis Abija.
\VS{29}Les fils de Merari furent Machli, Libni son fils, Schimeï son fils, Uzza son fils,
\VS{30}Schimea son fils, Hagguija son fils, Asaja son fils.
\TextTitle{Les chefs des chantres}
\VS{31}Or voici ceux que David établit pour la direction de la musique dans la maison de Yahweh, depuis que l'arche fut en lieu de repos.
\VS{32}Ils faisaient le service comme chantres devant le tabernacle, devant la tente d'assignation, jusqu'à ce que Salomon eût bâti la maison de Yahweh à Jérusalem~; ils continuèrent dans leur service selon l'ordonnance qui était prescrite.
\VS{33}Voici ceux qui firent le service avec leurs fils~: D'entre les fils des Kehathites, Héman le chantre, fils de Joël, fils de Samuel,
\VS{34}fils d'Elkana, fils de Jerocham, fils d'Eliel, fils de Thoach,
\VS{35}fils de Tsuph, fils d'Elkana, fils de Machath, fils de Amasaï,
\VS{36}fils d'Elkana, fils de Joël, fils d'Azaria, fils de Sophonie,
\VS{37}fils de Thachath, fils d'Assir, fils de Ebjasaph, fils de Koré,
\VS{38}fils de Jitsehar, fils de Kehath, fils de Lévi, fils d'Israël.
\VS{39}Son frère Asaph, qui se tenait à sa droite. Asaph était fils de Bérékia, fils de Schimea,
\VS{40}fils de Micaël, fils de Baaséja, fils de Malkija,
\VS{41}fils d'Ethni, fils de Zérach, fils d'Adaja,
\VS{42}fils d'Ethan, fils de Zimma, fils de Schimeï,
\VS{43}fils de Jachath, fils de Guerschon, fils de Lévi.
\VS{44}Les fils de Merari, leurs frères étaient à la gauche~; à savoir Ethan, fils de Kischi, fils d'Abdi, fils de Malluc,
\VS{45}fils de Haschabia, fils d'Amatsia, fils de Hilkija,
\VS{46}fils d'Amtsi, fils de Bani, fils de Schémer,
\VS{47}fils de Machli, fils de Muschi, fils de Merari, fils de Lévi.
\VS{48}Et leurs autres frères Lévites furent ordonnés pour tout le service du tabernacle de la maison de Dieu.
\VS{49}Mais Aaron et ses fils offraient les parfums sur l'autel de l'holocauste et sur l'autel des parfums~; pour tout ce qu'il fallait faire dans le Saint des saints, et pour faire propitiation pour Israël~; comme Moïse, serviteur de Dieu, l'avait commandé.
\TextTitle{Les prêtres d'Aaron à Achimaats}
\VS{50}Voici les fils d'Aaron~: Eléazar son fils, Phinées son fils, Abischua son fils,
\VS{51}Bukki son fils, Uzzi son fils, Zerachja son fils,
\VS{52}Merajoth son fils, Amaria son fils, Achithub son fils,
\VS{53}Tsadok son fils, Achimaats son fils.
\TextTitle{Villes des fils d'Aaron et des Lévites}
\VS{54}Voici leurs lieux d'habitation, selon leurs demeures et leurs limites. Aux fils d'Aaron, qui appartiennent à la famille des Kehathites, désignés par le sort,
\VS{55}on leur donna Hébron dans le pays de Juda, et ses faubourgs tout autour.
\VS{56}Mais on donna à Caleb, fils de Jephunné, le territoire de la ville et ses villages.
\VS{57}On donna donc aux fils d'Aaron, d'entre les villes de refuge, Hébron, Libna et ses faubourgs, Jatthir et Eschthemoa, avec leurs faubourgs,
\VS{58}Hilen, avec ses faubourgs, Debir avec ses faubourgs,
\VS{59}Aschan avec ses faubourgs, et Beth-Schémesch avec ses faubourgs.
\VS{60}De la tribu de Benjamin, Guéba, avec ses faubourgs, Allémeth avec ses faubourgs, et Anathoth avec ses faubourgs. Toutes leurs villes, selon leurs familles, étaient treize en nombre.
\VS{61}On donna au reste des fils de Kehath, par le sort, dix villes des familles de la demi-tribu, c'est-à-dire de la demi-tribu de Manassé.
\VS{62}Et aux fils de Guerschon, selon leurs familles, de la tribu d'Issacar, de la tribu d'Aser, de la tribu de Nephthali, et de la tribu de Manassé en Basan, treize villes.
\VS{63}Aux fils de Merari, selon leurs familles, par le sort, douze villes, de la tribu de Ruben, de la tribu de Gad, et de la tribu de Zabulon.
\VS{64}Ainsi, les enfants d'Israël donnèrent aux Lévites ces villes-là, avec leurs faubourgs.
\VS{65}Et ils donnèrent, par le sort, de la tribu des fils de Juda, de la tribu des fils de Siméon, et de la tribu des fils de Benjamin, ces villes qu'ils désignèrent par leurs noms.
\VS{66}Et pour les autres familles des fils de Kehath, ils eurent pour territoire des villes de la tribu d'Ephraïm.
\VS{67}Car on leur donna entre les villes de refuge, Sichem avec ses faubourgs, dans la montagne d'Ephraïm, Guézer avec ses faubourgs,
\VS{68}Jokmeam avec ses faubourgs, Beth-Horon avec ses faubourgs,
\VS{69}Ajalon avec ses faubourgs, et Gath-Rimmon avec ses faubourgs.
\VS{70}De la demi-tribu de Manassé, Aner avec ses faubourgs, et Bileam avec ses faubourgs, on donna ces villes-là aux familles qui restaient des fils de Kehath.
\VS{71}Aux fils de Guerschon, on donna, des familles de la demi-tribu de Manassé, Golan en Basan avec ses faubourgs, et Aschtaroth, avec ses faubourgs.
\VS{72}De la tribu d'Issacar, Kédesch avec ses faubourgs, Dobrath avec ses faubourgs,
\VS{73}Ramoth avec ses faubourgs, et Anem avec ses faubourgs.
\VS{74}Et de la tribu d'Aser, Maschal, avec ses faubourgs, Abdon, avec ses faubourgs,
\VS{75}Hukok avec ses faubourgs, et Rehob avec ses faubourgs.
\VS{76}De la tribu de Nephthali, Kédesch en Galilée avec ses faubourgs, Hammon avec ses faubourgs, et Kirjathaïm avec ses faubourgs.
\VS{77}Aux fils de Merari, qui étaient le reste d'entre les Lévites, on donna, de la tribu de Zabulon, Rimmono avec ses faubourgs, et Thabor avec ses faubourgs.
\VS{78}Au-delà du Jourdain, vis-à-vis de Jéricho, vers l'orient du Jourdain, de la tribu de Ruben, Betser au désert avec ses faubourgs, Jahtsa avec ses faubourgs,
\VS{79}Kedémoth avec ses faubourgs, et Méphaath avec ses faubourgs.
\VS{80}De la tribu de Gad, Ramoth en Galaad avec ses faubourgs, Mahanaïm avec ses faubourgs,
\VS{81}Hesbon avec ses faubourgs, et Jaezer avec ses faubourgs.
\Chap{7}
\TextTitle{Les descendants d'Issacar}
\VerseOne{}Les fils d'Issacar furent Thola, Pua, Jaschub et Schimron, quatre.
\VS{2}Les fils de Thola furent Uzzi, Rephaja, Jeriel, Jachmaï, Jibsam et Samuel, chefs des maisons de leurs pères qui étaient de Thola, gens forts et vaillants dans leurs générations~; leur nombre, aux jours de David, était de vingt-deux mille six cents.
\VS{3}Le fils d'Uzzi~: Jizrachja. Et les fils de Jizrachja~: Micaël, Abdias, Joël, et Jischija, en tout cinq chefs.
\VS{4}Ils avaient avec eux, selon leurs générations, et selon les familles de leurs pères, trente-six mille hommes de troupe, armés pour la guerre, car ils eurent plusieurs femmes et plusieurs fils.
\VS{5}Leurs frères selon toutes les familles d'Issacar, hommes forts et vaillants, étant comptés tous selon leur généalogie, furent quatre-vingt-sept mille.
\TextTitle{Les descendants de Benjamin}
\VS{6}Les fils de Benjamin furent Béla, Béker et Jediaël, trois\FTNT{Benjamin avait encore d'autres fils (Ge. 46:21~; No. 26:38-41~; 1 Ch. 8:1-2).}.
\VS{7}Les fils de Béla furent Etsbon, Uzzi, Uziel, Jerimoth et Iri, cinq chefs des familles de leurs pères, hommes forts et vaillants, et leur dénombrement selon leur généalogie monta à vingt-deux mille trente-quatre.
\VS{8}Les fils de Béker furent Zemira, Joasch, Eliézer, Eljoénaï, Omri, Jerémoth, Abija, Anathoth, et Alameth, tous ceux-là furent fils de Béker,
\VS{9}et leur dénombrement selon leur généalogie, selon leurs générations, comme chefs des familles de leurs pères, hommes forts et vaillants au nombre de vingt mille deux cents.
\VS{10}Jediaël eut pour fils Bilhan. Et les fils de Bilhan furent Jeusch, Benjamin, Ehud, Kenaana, Zéthan, Tarsis, et Achischachar.
\VS{11}Tous ceux-là furent fils de Jediaël, comme chefs des familles de leurs pères, dix-sept mille deux cents hommes forts et vaillants, en état de porter les armes et d'aller à la guerre.
\VS{12}Schuppim et Huppim furent des fils d'Ir~; et Huschim fut fils d'Acher.
\TextTitle{Les descendants de Nephtali}
\VS{13}Les fils de Nephthali furent Jahtsiel, Guni, Jetser, et Schallum, fils de Bilha.
\TextTitle{Les descendants de Manassé}
\VS{14}Les fils de Manassé~: Asriel, qu'enfanta sa concubine Araméenne. Elle enfanta Makir, père de Galaad.
\VS{15}Makir prit une femme de la parenté de Huppim et de Schuppim~; car ils avaient une sœur dont le nom était Maaca. Et le nom d'un des petits-fils de Galaad fut Tselophchad~; et Tselophchad eut des filles.
\VS{16}Maaca, femme de Makir, enfanta un fils et l'appela Péresch, et le nom de son frère Schéresch, dont les fils furent Ulam et Rékem.
\VS{17}Le fils d'Ulam fut Bedan. Ce sont là les fils de Galaad, fils de Makir, fils de Manassé.
\VS{18}Mais sa sœur Hammoléketh enfanta Ischhod, Abiézer et Machla.
\VS{19}Les fils de Schemida furent Achjan, Sichem, Likchi et Aniam.
\TextTitle{Les descendants d'Ephraïm et leurs villes}
\VS{20}Or les fils d'Ephraïm furent Schutélach~; Béred son fils, Tachath son fils, Eleada son fils, Tachath son fils.
\VS{21}Zabad son fils, Schutélach son fils, Ezer, et Elead. Mais ceux de Gath, nés dans le pays, les mirent à mort, parce qu'ils étaient descendus pour prendre leur bétail.
\VS{22}Ephraïm, leur père, fut dans le deuil plusieurs jours, et ses frères vinrent pour le consoler.
\VS{23}Puis il alla vers sa femme, qui conçut et enfanta un fils~; et elle l'appela du nom de Beria, parce que le malheur était dans sa maison.
\VS{24}Il eut pour fille Schééra, qui bâtit la basse et la haute Beth-Horon, et Uzzen-Schééra.
\VS{25}Son fils fut Réphach, puis Réscheph, et Thélach son fils, Thachan son fils,
\VS{26}Laedan son fils, Ammihud son fils, Elischama son fils,
\VS{27}Nun son fils, Josué son fils.
\VS{28}Ils possédaient et habitaient Béthel ainsi que les villes de son ressort~; à l'orient Naaran, à l'occident Guézer, avec les villes de son ressort, et Sichem avec les villes de son ressort, jusqu'à Gaza avec les villes de son ressort.
\VS{29}Les lieux qui étaient aux fils de Manassé furent Beth-Schean avec les villes de son ressort, Thaanac avec les villes de son ressort, Meguiddo avec les villes de son ressort, et Dor avec les villes de son ressort. Les fils de Joseph, fils d'Israël, habitèrent dans ces villes.
\TextTitle{Les descendants d'Aser}
\VS{30}Les fils d'Aser furent Jimna, Jischva, Jischvi, Beria, et Sérach leur sœur.
\VS{31}Les fils de Beria furent Héber et Malkiel, qui fut père de Birzavith.
\VS{32}Héber engendra Japhleth, Schomer, Hotham, et Schua leur sœur.
\VS{33}Les fils de Japhleth furent Pasac, Bimhal, et Aschvath. Ce sont là les fils de Japhlet.
\VS{34}Et les fils de Schamer furent Achi, Rohega, Hubba et Aram.
\VS{35}Les fils d'Hélem, son frère, furent Tsophach, Jimna, Schélesch et Amal.
\VS{36}Les fils de Tsophach furent Suach, Harnépher, Schual, Béri, Jimra,
\VS{37}Betser, Hod, Schamma, Schilscha, Jithran, et Beéra.
\VS{38}Les fils de Jéther furent Jephunné, Pispa et Ara.
\VS{39}Les fils d'Ulla furent Arach, Hanniel et Ritsja.
\VS{40}Tous ceux-là furent fils d'Aser, chefs des maisons de leurs pères, gens d'élite, forts et vaillants, chefs des princes, et leur dénombrement selon leur généalogie, qui fut fait quand on s'assemblait pour aller à la guerre, fut de vingt-six mille hommes.
\Chap{8}
\TextTitle{Les descendants de Benjamin}
\VerseOne{}Benjamin engendra Béla, qui fut son premier-né, Aschbel le deuxième, Achrach le troisième,
\VS{2}Nocha le quatrième, et Rapha le cinquième.
\VS{3}Les fils de Béla furent Addar, Guéra, Abihud,
\VS{4}Abischua, Naaman, Achoach,
\VS{5}Guéra, Schephuphan et Huram.
\VS{6}Voici les fils d'Echud, qui étaient chefs des maisons des pères des habitants de Guéba, et qui les transportèrent à Manachath~:
\VS{7}Naaman, Achija, et Guéra. Guéra, qui les transporta et qui après engendra Uzza et Achichud.
\VS{8}Or Schacharaïm eut des enfants au pays de Moab, après avoir renvoyé Huschim et Baara, ses femmes.
\VS{9}Il engendra, de Hodesch sa femme, Jobab, Tsibja, Méscha, Malcam,
\VS{10}Jeuts, Schocja et Mirma. Ce sont là ses fils, chefs des pères.
\VS{11}Mais de Huschim, il engendra Abithub, Elpaal.
\VS{12}Les fils d'Elpaal furent Eber, Mischeam, et Schémer, qui bâtit Ono, Lod et les villes de son ressort.
\VS{13}Et Beria et Schéma furent chefs des pères des habitants d'Ajalon~; ils mirent en fuite les habitants de Gath.
\VS{14}Achjo, Schaschak, Jerémoth,
\VS{15}Zebadja, Arad, Eder,
\VS{16}Micaël, Jischpha, et Jocha, fils de Beria.
\VS{17}Zebadja, Meschullam, Hizki, Héber,
\VS{18}Jischmeraï, Jizlia, et Jobab, fils d'Elpaal.
\VS{19}Jakim, Zicri, Zabdi,
\VS{20}Eliénaï, Tsilthaï, Eliel,
\VS{21}Adaja, Beraja, et Schimrath, fils de Schimeï.
\VS{22}Jischpan, Eber, Eliel,
\VS{23}Abdon, Zicri, Hanan,
\VS{24}Hanania, Elam, Anthothija,
\VS{25}Jiphdeja et Penuel, fils de Schaschak.
\VS{26}Schamscheraï, Schecharia, Athalia,
\VS{27}Jaaréschia, Elija, et Zicri, fils de Jerocham.
\VS{28}Ce sont là les chefs des pères, selon leurs générations~; et ils habitèrent à Jérusalem.
\TextTitle{Les fils du père de Gabaon, ascendant de Saül}
\VS{29}Le père de Gabaon habita à Gabaon, sa femme avait pour nom Maaca.
\VS{30}Son fils premier-né fut Abdon, puis Tsur, Kis, Baal, Nadab,
\VS{31}Guedor, Achjo, et Zéker.
\VS{32}Mikloth engendra Schimea. Ils habitèrent aussi vis-à-vis de leurs frères à Jérusalem, avec leurs frères.
\VS{33}Ner engendra Kis, et Kis engendra Saül, et Saül engendra Jonathan, Malki-Schua, Abinadab, et Eschbaal.
\VS{34}Le fils de Jonathan fut Merib-Baal~; et Merib-Baal engendra Michée.
\VS{35}Les fils de Michée furent Pithon, Mélec, Thaeréa, et Achaz.
\VS{36}Achaz engendra Jehoadda~; et Jehoadda engendra Alémeth, Azmaveth et Zimri~; Zimri engendra Motsa.
\VS{37}Motsa engendra Binea, qui eut pour fils Rapha, qui eut pour fils Eleasa, qui eut pour fils Atsel.
\VS{38}Atsel eut six fils, dont les noms sont~: Azrikam, Bocru, Ismaël, Schearia, Abdias, et Hanan~; tous ceux-là furent fils d'Atsel.
\VS{39}Les fils d'Eschek, son frère, furent Ulam son premier-né, Jéusch le second, Eliphéleth le troisième.
\VS{40}Et les fils d'Ulam furent des hommes forts et vaillants, tirant bien de l'arc, et ils eurent beaucoup de fils et de petits-fils, jusqu'à cent cinquante~; tous des fils de Benjamin.
\Chap{9}
\TextTitle{Les habitants de Jérusalem}
\VerseOne{}Ainsi, tous ceux d'Israël furent enregistrés par généalogie et inscrits dans le livre des rois d'Israël. Et ceux de Juda furent emmenés en captivité à Babylone à cause de leurs péchés\FTNT{La captivité babylonienne voir 2 R. 24-25.}.
\VS{2}Mais ce sont ici les premiers qui habitèrent dans leurs possessions, et dans leurs villes, tant d'Israël que des prêtres, des Lévites, et des Néthiniens.
\VS{3}A Jérusalem habitaient les fils de Juda, les fils de Benjamin, et les fils d'Ephraïm et de Manassé.
\VS{4}Uthaï, fils d'Ammihud, fils d'Omri, fils d'Imri, fils de Bani, des fils de Pérets, fils de Juda.
\VS{5}Des Schilonites, Asaja le premier-né, et ses fils.
\VS{6}Des fils de Zérach, Jeuel, et ses frères, six cent quatre-vingt-dix.
\VS{7}Des fils de Benjamin, Sallu fils de Meschullam, fils de Hodavia, fils d'Assenua.
\VS{8}Jibneja, fils de Jerocham, et Ela fils d'Uzzi, fils de Micri~; et Meschullam fils de Schephathia, fils de Reuel, fils de Jibnija.
\VS{9}Leurs frères, selon leurs générations, furent neuf cent cinquante-six. Tous ces hommes-là furent chefs des pères dans les maisons de leurs pères.
\VS{10}Des prêtres~: Jedaeja, Jehojarib, et Jakin.
\VS{11}Azaria fils de Hilkija, fils de Meschullam, fils de Tsadok, fils de Merajoth, fils d'Achithub, intendant de la maison de Dieu.
\VS{12}Adaja, fils de Jerocham, fils de Paschhur, fils de Malkija~; et Maesaï, fils d'Adiel, fils de Jachzéra, fils de Meschullam, fils de Meschillémith, fils d'Immer.
\VS{13}Leurs frères, chefs de la maison de leurs pères, mille sept cent soixante hommes, forts et vaillants, occupés au service de la maison de Dieu.
\VS{14}Des Lévites~: Schemaeja, fils de Haschub, fils d'Azrikam, fils de Haschabia, des fils de Merari,
\VS{15}Bakbakkar, Héresch, et Galal~; et Matthania, fils de Michée, fils de Zicri, fils d'Asaph,
\VS{16}Abdias fils de Schemaeja, fils de Galal, fils de Jeduthun~; et Bérékia, fils d'Asa, fils d'Elkana, qui habita dans les villages des Nethophathiens.
\VS{17}Et les portiers~: Schallum, Akkub, Thalmon, et Achiman, et leurs frères~; mais Schallum était le chef.
\VS{18}Il l'a été jusqu'à maintenant, ayant la charge de la porte du roi vers l'orient. Ceux-là furent portiers pour le camp des fils de Lévi.
\VS{19}Schallum, fils de Koré, fils d'Ebiasaph, fils de Koré, et ses frères Koréites, de la maison de son père, remplissaient les fonctions de gardiens, gardant les seuils de la tente, comme leurs pères en avaient gardé l'entrée au camp de Yahweh~;
\VS{20}Phinées, fils d'Eléazar, fut établi chef sur eux en présence de Yahweh qui était avec lui.
\VS{21}Zacharie, fils de Meschélémia, était le portier de l'entrée de la tente d'assignation.
\VS{22}Ils étaient en tout deux cent douze, choisis pour être les portiers des seuils, et enregistrés selon les familles dans la généalogie, selon leurs villages~; David et Samuel, le voyant, les avaient établis dans leurs fonctions.
\VS{23}Eux, dis-je, et leurs fils furent établis sur les portes de la maison de Yahweh, qui est la maison du tabernacle, pour y faire la garde.
\VS{24}Il y avait des portiers aux quatre vents, à l'orient, à l'occident, au nord et au sud.
\VS{25}Et leurs frères, qui étaient dans leurs villages, devaient de temps à autre venir auprès d'eux pendant sept jours.
\VS{26}Car selon cette fonction, il y avait toujours quatre chefs des portiers, des Lévites, qui avaient la surveillance des chambres et des trésors de la maison de Dieu.
\VS{27}Ils se tenaient la nuit tout autour de la maison de Dieu, dont ils avaient la garde, et qu'ils devaient ouvrir tous les matins.
\VS{28}Certains d'entre eux prenaient soin des ustensiles du service~; car on en faisait le compte lorsqu'on les rentrait et qu'on les sortait.
\VS{29}D'autres veillaient sur les ustensiles, sur tous les ustensiles du sanctuaire, sur la fleur de farine, sur le vin, sur l'huile, sur l'encens et sur les aromates.
\VS{30}Mais ceux qui composaient les parfums aromatiques étaient des fils de prêtres.
\VS{31}Matthithia, d'entre les Lévites, premier-né de Schallum, Koréite, s'occupait des gâteaux cuits sur les plaques.
\VS{32}Et quelques-uns de leurs frères, parmi les fils des Kehathites, avaient la charge des pains de proposition\FTNT{Les pains de proposition sont une image du Seigneur Jésus, notre pain de vie (Jn. 6:48-59).} pour l'apprêter chaque sabbat.
\VS{33}Certains étaient des chantres, chefs des pères des Lévites, qui demeuraient dans les chambres, sans avoir d'autres charges, parce qu'ils devaient être en fonction le jour et la nuit.
\VS{34}Ce sont là les chefs des pères des Lévites, selon leurs familles~; ils furent chefs, et ils habitèrent à Jérusalem.
\TextTitle{De Jeïel au roi Saül, de Jonathan à Arsel\FTNTT{1 Ch. 10~; 1 S. 1~; 1 S. 30.}}
\VS{35}Or Jeïel, le père de Gabaon, habita à Gabaon~; et le nom de sa femme était Maaca.
\VS{36}Son fils premier-né, Abdon, puis Tsur, Kis, Baal, Ner, Nadab,
\VS{37}Guedor, Achjo, Zacharie, et Mikloth.
\VS{38}Mikloth engendra Schimeam~; et ils habitèrent vis-à-vis de leurs frères à Jérusalem, avec leurs frères.
\VS{39}Ner engendra Kis, et Kis engendra Saül, et Saül engendra Jonathan, Malki-Schua, Abinadab et Eschbaal.
\VS{40}Le fils de Jonathan fut Merib-Baal~; et Merib-Baal engendra Michée.
\VS{41}Et les fils de Michée furent Pithon, Mélec, Thachréa et Achaz.
\VS{42}Achaz engendra Jaera~; et Jaera engendra Alémeth, Azmaveth, et Zimri~; et Zimri engendra Motsa.
\VS{43}Motsa engendra Binea, qui eut pour fils Rephaja, qui eut pour fils Eleasa, qui eut pour fils Atsel.
\VS{44}Atsel eut six fils, dont les noms sont Azrikam, Bocru, Ismaël, Scheari, Abdias et Hanan. Ce furent là les fils d'Atsel.
\Chap{10}
\TextTitle{Mort de Saül\FTNTT{1 S. 31:1-10~; 2 S. 1.}}
\VerseOne{}Les Philistins combattirent contre Israël, et les hommes d'Israël s'enfuirent devant les Philistins, et tombèrent blessés à mort sur la montagne de Guilboa\FTNT{1 S. 31:1-10.}.
\VS{2}Les Philistins poursuivirent et atteignirent Saül et ses fils, et tuèrent Jonathan, Abinadab et Malki-Schua, les fils de Saül.
\VS{3}L'effort du combat se porta sur Saül~; de sorte que les archers l'atteignirent, et il eut peur de ces archers.
\VS{4}Alors Saül dit à celui qui portait ses armes~: Tire ton épée, et transperce-moi, de peur que ces incirconcis ne viennent et ne fassent de moi selon leur volonté~; mais celui qui portait ses armes ne voulut pas, parce qu'il avait très peur. Saül prit donc son épée, et se jeta dessus.
\VS{5}Alors celui qui portait les armes de Saül, ayant vu que Saül était mort, se jeta aussi sur son épée, et il mourut.
\VS{6}Ainsi mourut Saül, et ses trois fils, et toute sa maison périt avec lui.
\VS{7}Tous ceux d'Israël, qui étaient dans la vallée, ayant vu qu'on avait fui, et que Saül et ses fils étaient morts, abandonnèrent leurs villes et s'enfuirent, de sorte que les Philistins y entrèrent et y habitèrent.
\VS{8}Or il arriva que dès le lendemain, les Philistins vinrent pour dépouiller les morts, et ils trouvèrent Saül et ses fils étendus sur la montagne de Guilboa.
\VS{9}Ils le dépouillèrent et emportèrent sa tête et ses armes. Puis ils firent annoncer ces bonnes nouvelles par tout le pays des Philistins, et aux environs, pour en faire savoir les nouvelles à leurs dieux et au peuple.
\VS{10}Ils mirent ses armes dans la maison de leur dieu, et ils attachèrent sa tête dans la maison de Dagon\FTNT{1 S. 5:1-11.}.
\VS{11}Tous ceux de Jabès de Galaad, ayant appris tout ce que les Philistins avaient fait à Saül,
\VS{12}tous les vaillants hommes d'entre eux se levèrent et enlevèrent le corps de Saül et les corps de ses fils~; ils les apportèrent à Jabès, et ils ensevelirent leurs os sous un chêne à Jabès, et jeûnèrent pendant sept jours.
\VS{13}Saül mourut pour le crime qu'il avait commis contre Yahweh, en ce qu'il n'avait point gardé la parole de Yahweh, et qu'il avait même consulté ceux qui évoquent les morts\FTNT{1 S. 28:7-20.} pour savoir ce qui devait lui arriver.
\VS{14}Il ne consulta point Yahweh~; c'est pourquoi Yahweh le fit mourir, et transféra la royauté à David, fils d'Isaï.
\Chap{11}
\TextTitle{David règne sur Israël\FTNTT{2 S. 5:1-3~; 2 S. 2-4.}}
\VerseOne{}Tous ceux d'Israël s'assemblèrent auprès de David à Hébron, et lui dirent~: Voici, nous sommes tes os et ta chair.
\VS{2}Autrefois déjà, quand Saül était roi, tu étais celui qui faisais sortir et qui ramenais Israël. Yahweh, ton Dieu, t'a dit~: Tu paîtras mon peuple d'Israël, et tu seras le chef de mon peuple d'Israël.
\VS{3}Ainsi, tous les anciens d'Israël vinrent auprès du roi à Hébron~; et David traita alliance avec eux à Hébron, devant Yahweh. Ils oignirent David pour roi sur Israël, selon la parole de Yahweh, prononcée par Samuel\FTNT{\vref{2 S. 2,3,4}~; 2 S. 5:1-3.}.
\TextTitle{Jérusalem devient la cité de David\FTNTT{2 S. 5:6-10.}}
\VS{4}David et tous ceux d'Israël s'en allèrent à Jérusalem, qui est Jébus. Là étaient les Jébusiens qui habitaient le pays.
\VS{5}Ceux qui habitaient à Jébus dirent à David~: Tu n'entreras point ici. Mais David prit la forteresse de Sion, qui est la cité de David.
\VS{6}Car David avait dit: Quiconque battra le premier les Jébusiens sera chef et prince. Joab, fils de Tseruja, monta le premier, et fut fait chef.
\VS{7}David s'établit dans la forteresse~; c'est pourquoi on l'appela la cité de David\FTNT{2 S. 5:6-10.}.
\VS{8}Il bâtit aussi la ville tout autour, depuis Millo et ses environs~; et Joab répara le reste de la ville.
\VS{9}Et David allait toujours en avançant et en croissant, car Yahweh des armées était avec lui.
\TextTitle{Les vaillants hommes de David\FTNTT{2 S. 23:8-39.}}
\VS{10}Voici les chefs des hommes vaillants qui étaient au service de David, qui l'aidèrent avec tout Israël à assurer sa royauté, afin de le faire régner selon la parole de Yahweh au sujet d'Israël.
\VS{11}Ceux-ci sont du nombre des vaillants hommes que David avait. Jaschobeam, fils de Hacmoni, chef entre les trois principaux. Il brandit sa lance contre trois cents hommes et les blessa à mort en une seule fois\FTNT{2 S. 23:8-39.}.
\VS{12}Après lui était Eléazar, fils de Dodo, l'Achochite, qui fut l'un des trois vaillants hommes.
\VS{13}Il se trouvait avec David à Pas-Dammim, lorsque les Philistins s'étaient assemblés pour combattre. Il y avait là une parcelle de terre remplie d'orge~; et le peuple fuyait devant les Philistins.
\VS{14}Ils s'arrêtèrent au milieu de cette parcelle de champ, la défendirent, et battirent les Philistins. Ainsi, Yahweh accorda une grande délivrance.
\VS{15}Il en descendit encore trois des trente chefs près du rocher, auprès de David, dans la caverne d'Adullam, lorsque l'armée des Philistins campait dans la vallée des Rephaïm.
\VS{16}David était alors dans la forteresse, et la garnison des Philistins était en ce même temps-là à Bethléhem.
\VS{17}David eut un désir, et dit~: Qui est-ce qui me fera boire de l'eau du puits qui est à la porte de Bethléhem~?
\VS{18}Alors ces trois hommes passèrent au travers du camp des Philistins, et puisèrent de l'eau du puits qui était à la porte de Bethléhem~; et l'ayant apportée, la présentèrent à David, qui ne voulut point la boire, mais la répandit en l'honneur de Yahweh.
\VS{19}Car il dit~: Que mon Dieu me garde de faire une telle chose~! Boirais-je le sang de ces hommes qui ont fait un tel voyage au péril de leur vie~? Car ils m'ont apporté cette eau au péril de leur vie. Ainsi, il ne voulut point la boire. Voilà ce que firent ces trois vaillants hommes.
\VS{20}Abischaï, frère de Joab, était chef des trois. Il sortit sa lance sur trois cents hommes, les blessa à mort~; et il eut du renom entre les trois.
\VS{21}Entre les trois, il fut plus honoré que les deux autres, et il fut leur chef~; cependant, il n'égala point ces trois premiers.
\VS{22}Benaja aussi, fils de Jehojada, fils d'un vaillant homme de Kabtseel, avait fait de grands exploits. Il tua deux des plus puissants hommes de Moab. Il descendit et frappa un lion au milieu d'une fosse en un jour de neige.
\VS{23}Il tua aussi un homme Egyptien qui était haut de cinq coudées. Cet Egyptien avait à la main une lance grosse comme une ensouple de tisserand~; mais il descendit contre lui avec un bâton, et arracha la lance de la main de l'Egyptien, et le tua avec sa propre lance.
\VS{24}Benaja, fils de Jehojada, fit ces choses-là, et fut célèbre entre ces trois vaillants hommes.
\VS{25}Voilà, il était le plus honoré des trente~; cependant, il n'égala point les trois premiers. David l'établit dans son conseil privé.
\VS{26}Et les plus vaillants d'entre les gens de guerre furent Asaël, frère de Joab~; et Elchanan fils de Dodo, de Bethléhem,
\VS{27}Schammoth d'Haror, Hélets de Palon,
\VS{28}Ira, fils d'Ikkesch, de Tekoa, Abiézer d'Anathoth,
\VS{29}Sibbecaï le Huschatite, Ilaï d'Achoach,
\VS{30}Maharaï de Nethopha, Héled fils de Baana de Nethopha,
\VS{31}Ittaï fils de Ribaï, de Guibea des fils de Benjamin, Benaja de Pirathon,
\VS{32}Huraï de Nachalé-Gaasch, Abiel d'Araba,
\VS{33}Azmaveth de Bacharum, Eliachba de Schaalbon,
\VS{34}Bené-Haschem de Guizon, Jonathan fils de Schagué d'Harar,
\VS{35}Achiam fils de Sacar d'Harar, Eliphal fils d'Ur,
\VS{36}Hépher de Mekéra, Achija de Palon,
\VS{37}Hetsro de Carmel, Naaraï fils d'Ezbaï,
\VS{38}Joël frère de Nathan, Mibchar fils d'Hagri,
\VS{39}Tsélek l'Ammonite, Nachraï de Béroth, qui portait les armes de Joab fils de Tseruja,
\VS{40}Ira de Jéther, Gareb de Jéther,
\VS{41}Urie le Héthien, Zabad fils d' Achlaï,
\VS{42}Adina fils de Schiza le Rubénite, chef des Rubénites, et trente avec lui.
\VS{43}Hanan fils de Maaca, et Josaphat de Mithni,
\VS{44}Ozias d'Aschtharoth, Schama et Jehiel fils de Hotham d'Aroër,
\VS{45}Jediaël fils de Schimri, et Jocha son frère, le Thitsite,
\VS{46}Eliel de Machavim, Jeribaï, et Joschavia fils d'Elnaam, et Jithma le Moabite,
\VS{47}Eliel, et Obed, et Jaasie-Metsobaja.
\Chap{12}
\TextTitle{Les guerriers venus chez David à Tsiklag\FTNTT{2 S. 5:17~; 1 Ch. 12:8-15~; 1 Ch. 14:8.}}
\VerseOne{}Voici ceux qui allèrent trouver David à Tsiklag, lorsqu'il était encore éloigné de la présence de Saül, fils de Kis. Ils étaient parmi les vaillants hommes qui lui prêtèrent leur secours pendant la guerre.
\VS{2}Ils étaient équipés d'arcs, se servant de la main droite et de la gauche pour jeter des pierres, et pour tirer des flèches avec l'arc. Ils étaient frères de Saül, de Benjamin,
\VS{3}Achiézer, le chef, et Joas, fils de Schemaa, qui était de Guibea, Jeziel, Péleth, fils d'Azmaveth, Beraca et Jéhu d'Anathoth~;
\VS{4}Jischmaeja de Gabaon, vaillant entre les trente, et même au-dessus des trente, et Jérémie, Jachaziel, Jochanan et Jozabad de Guedéra~;
\VS{5}Eluzaï, Jerimoth, Bealia, Schemaria et Schephathia de Haroph~;
\VS{6}Elkana, Jischija, Azareel, Joézer et Jaschobeam Koréites~;
\VS{7}Joéla et Zebadia, fils de Jerocham de Guedor.
\TextTitle{Les guerriers venus chez David dans la forteresse de Moab\FTNTT{1 S. 22:2-4.}}
\VS{8}Quelques-uns aussi des Gadites se retirèrent auprès de David, dans la forteresse, au désert, hommes forts et vaillants, experts à la guerre et maniant le bouclier et la lance. Leurs visages étaient comme des faces de lion, et ils étaient aussi prompts que des gazelles sur les montagnes.
\VS{9}Ezer le premier, Abdias le second, Eliab le troisième~;
\VS{10}Mischmanna le quatrième, Jérémie le cinquième~;
\VS{11}Attaï le sixième~; Eliel le septième~;
\VS{12}Jochanan le huitième, Elzabad le neuvième~;
\VS{13}Jérémie le dixième, Macbannaï le onzième.
\VS{14}C'étaient des fils de Gath, qui furent chefs de l'armée~; le plus petit avait la charge de cent hommes, et le plus grand de mille.
\VS{15}Ce sont ceux qui passèrent le Jourdain au premier mois, quand il déborde sur tous ses rivages~; et ils chassèrent ceux qui demeuraient dans les vallées, vers l'orient et l'occident.
\VS{16}Il vint aussi des fils de Benjamin et de Juda vers David à la forteresse.
\VS{17}David sortit au-devant d'eux, et prenant la parole, il leur dit~: Si vous êtes venus en paix vers moi pour m'aider, mon cœur s'unira à vous~; mais si c'est pour me trahir et me livrer à mes ennemis, quoique je ne sois coupable d'aucune violence, que le Dieu de nos pères le voie, et qu'il fasse justice~!
\VS{18}Alors Amasaï, l'un des principaux officiers, fut revêtu de l'Esprit, et dit~: Que la paix soit avec toi, ô David~! Qu'elle soit avec toi, fils d'Isaï~! Que la paix soit à ceux qui t'aident, puisque ton Dieu t'aide~! Et David les reçut, et les établit parmi les chefs de ses troupes.
\VS{19}Des hommes de Manassé se joignirent à David, lorsqu'il alla combattre Saül avec les Philistins. Mais David et ses gens ne les aidèrent pas, parce que les princes des Philistins, après en avoir délibéré entre eux, le renvoyèrent, en disant~: Il se tournera vers son maître Saül, au péril de nos têtes.
\VS{20}Quand donc il retournait à Tsiklag, Adnach, Jozabad, Jediaël, Micaël, Jozabad, Elihu et Tsilthaï, chefs des milliers qui étaient en Manassé, se tournèrent vers lui.
\VS{21}Et ils aidèrent David contre la troupe des Amalécites, car ils étaient tous forts et vaillants, et ils furent faits chefs dans l'armée.
\VS{22}De jour en jour, il venait des gens auprès de David pour l'aider, de sorte qu'il eut une grande armée, comme une armée de Dieu\FTNT{1 S. 22:2-4.}.
\TextTitle{Les guerriers venus chez David à Hébron\FTNTT{2 S. 5:1-3.}}
\VS{23}Voici le nombre des hommes équipés pour la guerre, qui vinrent auprès de David à Hébron, afin de lui transférer la royauté de Saül, selon le commandement de Yahweh\FTNT{2 S. 5:1-3.}.
\VS{24}Des fils de Juda, qui portaient le bouclier et la lance, six mille huit cents, équipés pour la guerre.
\VS{25}Des fils de Siméon, forts et vaillants pour la guerre, sept mille cent.
\VS{26}Des fils de Lévi, quatre mille six cents.
\VS{27}Et Jehojada, prince de ceux d'Aaron, et avec lui trois mille sept cents~;
\VS{28}et Tsadok, jeune homme fort et vaillant, et vingt-deux chefs de la maison de son père.
\VS{29}Des fils de Benjamin, parents de Saül, trois mille~; car jusqu'alors la plus grande partie d'entre eux soutenaient la maison de Saül.
\VS{30}Des fils d'Ephraïm, vingt mille huit cents, forts et vaillants, et hommes de renom dans la maison de leurs pères.
\VS{31}De la demi-tribu de Manassé, dix-huit mille, qui furent désignés par leur nom pour aller établir David roi.
\VS{32}Des fils d'Issacar, fort intelligents dans la connaissance des temps, pour savoir ce que devait faire Israël, deux cents de leurs chefs, et tous leurs frères sous leurs ordres.
\VS{33}De Zabulon, cinquante mille combattants, rangés en bataille avec toutes sortes d'armes, et prêts à livrer bataille d'un cœur assuré.
\VS{34}De Nephthali, mille capitaines, et avec eux trente-sept mille, portant le bouclier et la lance.
\VS{35}Des Danites, vingt-huit mille six cents, équipés pour la guerre.
\VS{36}D'Aser, quarante mille combattants, et prêts à combattre.
\VS{37}De l'autre côté du Jourdain, à savoir des Rubénites, des Gadites, et de la demi-tribu de Manassé, cent vingt mille, avec tous les instruments de guerre pour combattre.
\VS{38}Tous ces hommes, gens de guerre, prêts a combattre, vinrent tous de bon cœur à Hébron, pour établir David roi sur tout Israël. Et tout le reste d'Israël était aussi d'un même sentiment pour établir David roi.
\VS{39}Et ils furent là avec David, mangeant et buvant pendant trois jours~; car leurs frères leur avaient préparé des vivres.
\VS{40}Et même ceux qui étaient les plus proches d'eux, jusqu'à Issacar, Zabulon et Nephthali, apportaient du pain sur des ânes, sur des chameaux, sur des mulets et sur des bœufs, de la farine, des figues sèches, des raisins secs, du vin, et de l'huile~; et ils amenaient des bœufs et des brebis en abondance, car il y avait une joie en Israël.
\Chap{13}
\TextTitle{Retour de l'arche, Uzza frappé par Yahweh\FTNTT{2 S. 6:1-11.}}
\VerseOne{}Or David tint conseil avec les chefs de milliers et de centaines, avec tous les princes du peuple.
\VS{2}Et il dit à toute l'assemblée d'Israël~: Si vous l'approuvez, et que cela vient de Yahweh, notre Dieu, envoyons partout vers nos autres frères, qui sont dans toutes les contrées d'Israël, et avec lesquels sont les prêtres et les Lévites, dans leurs villes et dans leurs faubourgs, afin qu'ils se réunissent à nous,
\VS{3}et que nous ramenions auprès de nous l'arche de notre Dieu~; car nous ne nous en sommes pas occupés du temps de Saül.
\VS{4}Et toute l'assemblée répondit qu'on le fasse ainsi~; car la chose fut approuvée par tout le peuple.
\VS{5}David donc assembla tout Israël, depuis Schichor, le torrent d'Egypte, jusqu'à l'entrée du pays de Hamath, pour ramener de Kirjath-Jearim l'arche de Dieu.
\VS{6}Et David monta avec tout Israël vers Baala à Kirjath-Jearim, qui appartient à Juda, pour faire amener de là l'arche de Dieu, devant laquelle est invoqué le Nom de Yahweh, qui habite entre les chérubins.
\VS{7}Ils mirent l'arche de Dieu sur un char neuf, et l'emmenèrent de la maison d'Abinadab~; et Uzza et Achjo conduisaient le char.
\VS{8}Et David et tout Israël dansaient en présence de Dieu de toute leur force, en chantant des cantiques et en jouant sur des violons, des luths, des tambourins, des cymbales, et des trompettes.
\VS{9}Quand ils furent arrivés à l'aire de Kidon, Uzza\FTNT{L'arche devait être transportée grâce à des barres faites spécialement à cet effet, qui ne devaient pas être enlevées (Ex. 27:6-7~; No. 1:51). Selon la Loi, seuls les Lévites devaient préparer et déplacer tout ce qui concernait le tabernacle. Et même parmi les Lévites, chaque famille avait une fonction spécifique (No. 3~; No. 4). Les Kehathites n'étaient pas autorisés à toucher l'arche, leur rôle se limitait seulement à la transporter à l'aide des barres (No. 4:15). Uzza a étendu sa main sur l'arche, alors qu'il n'était certainement pas Lévite. Il était devenu trop familier avec les choses saintes et avait pris à la légère les principes de Dieu. Il a voulu aider le Seigneur. Or, il ne faut jamais chercher à servir Dieu sans être appelé par lui.} étendit sa main pour retenir l'arche, parce que les bœufs avaient glissé.
\VS{10}Et la colère de Yahweh s'enflamma contre Uzza, et il le frappa, parce qu'il avait étendu sa main sur l'arche. Uzza mourut en présence de Dieu.
\VS{11}David fut irrité de ce que Yahweh avait fait une brèche en la personne de Uzza. On a appelé jusqu'à ce jour ce lieu-là Pérets-Uzza, brèche d'Uzza.
\VS{12}David eut peur de Dieu en ce jour-là, et il dit: Comment ferais-je entrer chez moi l'arche de Dieu~?
\VS{13}C'est pourquoi David ne la retira point chez lui, dans la cité de David, mais il la fit conduire dans la maison d'Obed-Edom de Gath.
\VS{14}Et l'arche de Dieu demeura trois mois avec la famille d'Obed-Edom, dans sa maison. Yahweh bénit la maison d'Obed-Edom, et tout ce qui lui appartenait.
\Chap{14}
\TextTitle{Rayonnement du règne de David\FTNTT{2 S. 5:11-25~; 23:13-17~; 1 Ch. 3:5-9~; 11:15-19~; 12:8-15.}}
\VerseOne{}Hiram, roi de Tyr, envoya des messagers à David, et du bois de cèdre, des tailleurs de pierres et des charpentiers, pour lui bâtir une maison.
\VS{2}Alors David reconnut que Yahweh l'affermissait comme roi sur Israël, et que son règne était fort élevé, à cause de son peuple d'Israël.
\VS{3}David prit encore des femmes à Jérusalem, et il engendra encore des fils et des filles.
\VS{4}Voici les noms des fils qu'il eut à Jérusalem~: Schammua, Schobab, Nathan, Salomon,
\VS{5}Jibhar, Elischua, Elphéleth,
\VS{6}Noga, Népheg, Japhia,
\VS{7}Elischama, Beéliada et Eliphéleth.
\VS{8}Or quand les Philistins apprirent que David avait été oint pour roi sur tout Israël, ils montèrent tous à sa recherche. David l'ayant appris, sortit au-devant d'eux.
\VS{9}Les Philistins vinrent et se répandirent dans la vallée des Rephaïm.
\VS{10}David consulta Dieu, en disant~: Monterai-je contre les Philistins, et les livreras-tu entre mes mains~? Yahweh lui répondit~: Monte, et je les livrerai entre tes mains.
\VS{11}Alors ils montèrent à Baal-Peratsim\FTNT{Baal-Peratsim signifie «~seigneur des brèches~».}, où David les battit. Puis il dit~: Dieu a fait une brèche au milieu de mes ennemis par ma main, comme une brèche faite par les eaux. C'est pourquoi on donna à ce lieu-là le nom de Baal-Peratsim.
\VS{12}Et ils laissèrent là leurs dieux, et David ordonna qu'on les brûle au feu.
\VS{13}Les Philistins se répandirent encore une autre fois dans cette même vallée.
\VS{14}David consulta encore Dieu~; et Dieu lui répondit~: Tu ne monteras point vers eux, mais tu te détourneras d'eux, et tu iras contre eux vis-à-vis des mûriers.
\VS{15}Dès que tu auras entendu au sommet des mûriers un bruit comme des gens qui marchent, tu sortiras alors pour combattre, car c'est Dieu qui marche devant toi pour frapper le camp des Philistins.
\VS{16}David fit selon ce que Dieu lui avait ordonné, et on frappa le camp des Philistins, depuis Gabaon jusqu'à Guézer.
\VS{17}Ainsi, la renommée de David se répandit par tous ces pays-là, et Yahweh remplit de frayeur toutes ces nations-là, au seul nom de David.
\Chap{15}
\TextTitle{David organise l'arrivée de l'arche à Jérusalem\FTNTT{2 S. 6:12.}}
\VerseOne{}David se bâtit des maisons dans la cité de David~; il prépara un lieu pour l'arche de Dieu, et dressa pour elle une tente.
\VS{2}Et David dit~: L'arche de Dieu ne doit être portée que par les Lévites, car Yahweh les a choisis pour porter l'arche de Dieu, et pour faire le service à toujours\FTNT{No. 4:15.}.
\VS{3}David donc assembla tous ceux d'Israël à Jérusalem, pour faire monter l'arche de Yahweh dans le lieu qu'il lui avait préparé.
\VS{4}David assembla aussi les fils d'Aaron, et les Lévites.
\VS{5}Des fils de Kehath~: Uriel, le chef, et ses frères, cent vingt.
\VS{6}Des fils de Merari~: Asaja, le chef, et ses frères, deux cent vingt.
\VS{7}Des fils de Guerschon~: Joël, le chef, et ses frères, cent trente.
\VS{8}Des fils d'Elitsaphan~: Schemaeja, le chef, et ses frères, deux cents.
\VS{9}Des fils de Hébron~: Eliel, le chef, et ses frères, quatre-vingts.
\VS{10}Des fils de Uziel~: Amminadab, le chef, et ses frères, cent douze.
\VS{11}David appela les prêtres Tsadok et Abiathar, et les Lévites, à savoir Uriel, Asaja, Joël, Schemaeja, Eliel, et Amminadab~;
\VS{12}et il leur dit: Vous qui êtes les chefs des pères des Lévites, sanctifiez-vous, vous et vos frères~; et transportez l'arche de Yahweh, le Dieu d'Israël, au lieu que je lui ai préparé.
\VS{13}Parce que vous n'y étiez pas la première fois, Yahweh, notre Dieu, a fait une brèche parmi nous~; car nous ne l'avons pas cherché selon la loi.
\VS{14}Les prêtres donc et les Lévites se sanctifièrent pour faire monter l'arche de Yahweh, le Dieu d'Israël.
\VS{15}Et les fils des Lévites portèrent l'arche de Dieu sur leurs épaules, avec les barres qu'ils avaient sur eux, comme Moïse l'avait ordonné selon la parole de Yahweh.
\VS{16}David dit aux chefs des Lévites d'établir quelques-uns de leurs frères chantres, avec des instruments de musique, des luths, des violons, et des cymbales qui feraient retentir des sons éclatants, en signe de réjouissance.
\VS{17}Les Lévites donc établirent Héman, fils de Joël, et parmi ses frères, Asaph, fils de Bérékia~; et des fils de Merari, qui étaient leurs frères, Ethan, fils de Kuschaja~;
\VS{18}avec eux leurs frères pour être du second ordre~: Zacharie, Ben, Jaaziel, Schemiramoth, Jehiel, Unni, Eliab, Benaja, Maaséja, Matthithia, Eliphelé, Miknéja, Obed-Edom, et Jeïel, les portiers.
\VS{19}Quant aux chantres~: Héman, Asaph et Ethan, ils avaient des cymbales d'airain pour les faire retentir.
\VS{20}Zacharie, Aziel, Schemiramoth, Jehiel, Unni, Eliab, Maaséja, et Benaja jouaient des luths sur alamoth~;
\VS{21}et Matthithia, Eliphelé, Miknéja, Obed-Edom, Jeïel et Azazia jouaient des harpes à huit cordes, pour conduire le chant.
\VS{22}Mais Kenania, le chef des Lévites, avait la charge de faire porter l'arche, enseignant comment il fallait la porter, car il était un homme très intelligent.
\VS{23}Bérékia et Elkana étaient portiers de l'arche.
\VS{24}Schebania, Josaphat, Nethaneel, Amasaï, Zacharie, Benaja, Eliézer, les prêtres, sonnaient des trompettes devant l'arche de Dieu, et Obed-Edom et Jechija étaient portiers de l'arche.
\TextTitle{L'arche transportée au milieu des réjouissances\FTNTT{2 S. 6:12.}}
\VS{25}David et les anciens d'Israël, avec les gouverneurs de milliers, marchaient, amenant avec joie l'arche de l'alliance de Yahweh, de la maison d'Obed-Edom.
\VS{26}Dieu aidait les Lévites qui portaient l'arche de l'alliance de Yahweh, et l'on sacrifia sept veaux et sept béliers.
\VS{27}David était vêtu d'un manteau de fin lin~; et tous les Lévites aussi qui portaient l'arche, les chantres~; et Kenania, qui avait la principale charge de faire porter l'arche, était avec les chantres~; et David avait un éphod de fin lin.
\VS{28}Ainsi tout Israël amena l'arche de l'alliance de Yahweh, avec de grands cris de joie, et au son du cor, des shofars et des cymbales, faisant retentir leur voix avec des luths et des harpes.
\VS{29}Mais il arriva, comme l'arche de l'alliance de Yahweh entrait dans la cité de David, que Mical, fille de Saül, regardant par la fenêtre, vit le roi David sautant et dansant, et elle le méprisa dans son cœur.
\Chap{16}
\TextTitle{L'arche placée dans une tente à Jérusalem~; sacrifices et cantiques pour Yahweh\FTNTT{2 S. 6:17-19.}}
\VerseOne{}Ils amenèrent donc l'arche de Dieu et la posèrent au milieu de la tente que David avait dressée pour elle~; et on offrit devant Dieu des holocaustes et des sacrifices d'offrande de paix\FTNT{Voir commentaire en Lé. 3:1.}.
\VS{2}Quand David eut achevé d'offrir les holocaustes et les sacrifices d'offrande de paix, il bénit le peuple au Nom de Yahweh.
\VS{3}Et il distribua à chacun, tant aux hommes qu'aux femmes, un pain, un morceau de viande et un gâteau de raisin.
\VS{4}Et il établit quelques-uns des Lévites pour faire le service devant l'arche de Yahweh, pour célébrer, remercier, et louer le Dieu d'Israël.
\VS{5}Asaph était le premier et Zacharie le second~; Jeïel, Schemiramoth, Jehiel, Matthithia, Eliab, Benaja, Obed-Edom, et Jeïel, qui avaient des instruments de musique, à savoir des luths et des harpes~; et Asaph faisait retentir sa voix avec des cymbales.
\VS{6}Benaja et Jachaziel, les prêtres, étaient continuellement avec des trompettes devant l'arche de l'alliance de Dieu.
\VS{7}Et en ce même jour, David remit entre les mains d'Asaph et de ses frères, les Psaumes suivants, pour commencer à célébrer Yahweh~:
\VS{8}Célébrez Yahweh, invoquez son Nom~! Faites connaître parmi les peuples ses exploits~!
\VS{9}Chantez-le, célébrez-le~! Parlez de toutes ses merveilles~!
\VS{10}Glorifiez-vous de son saint Nom~! Que le cœur de ceux qui cherchent Yahweh se réjouisse~!
\VS{11}Recherchez Yahweh et sa force, cherchez continuellement sa face~!
\VS{12}Souvenez-vous des merveilles qu'il a faites, de ses miracles et des jugements de sa bouche.
\VS{13}Postérité d'Israël, son serviteur, fils de Jacob, ses élus~!
\VS{14}Yahweh est notre Dieu~; ses jugements s'exercent sur toute la terre.
\VS{15}Souvenez-vous toujours de son alliance, de ses promesses établies pour mille générations~;
\VS{16}du traité qu'il a fait avec Abraham et du serment qu'il a fait à Isaac,
\VS{17}et qu'il a confirmé à Jacob et à Israël, pour être une loi et une alliance éternelle,
\VS{18}en disant~: Je te donnerai le pays de Canaan, comme l'héritage qui vous est échu.
\VS{19}Ils étaient alors une poignée de gens, peu nombreux, et étrangers dans le pays,
\VS{20}car ils étaient errants de nation en nation, et d'un royaume vers un autre peuple.
\VS{21}Il ne permit à personne de les opprimer~; il a même châtié des rois à cause d'eux.
\VS{22}Et il a dit~: Ne touchez point à mes oints, et ne faites point de mal à mes prophètes\FTNT{L'expression «~ne touchez pas à mes oints~» signifie qu'il ne faut pas leur porter physiquement atteinte. C'est une expression associée à des mauvais traitements physiques. Il est donc clair que ce verset, qu'on trouve également dans le Ps. 105:15, ne peut absolument pas concerner la remise en question des enseignements d'un quelconque pasteur, prophète ou apôtre. Dans le contexte de ce passage, il est question des rois, des prophètes et des prêtres, car c'est sur eux que reposait l'onction. Aujourd'hui, tous les chrétiens sont oints de Dieu (Ep. 1:13~; Ep. 4:30).}~!
\VS{23}Habitants de la terre, chantez à Yahweh~! Racontez chaque jour sa délivrance.
\VS{24}Racontez sa gloire parmi les nations, et ses merveilles parmi tous les peuples~!
\VS{25}Car Yahweh est grand et très digne de louanges, il est plus redoutable que tous les dieux.
\VS{26}Car tous les dieux des peuples sont des idoles\FTNT{Jésus-Christ est le seul et le véritable Dieu (1 Co. 8:6~; 1 Jn. 5:20).}, mais Yahweh a fait les cieux.
\VS{27}La majesté et la magnificence marchent devant lui~; la force et la joie sont dans le lieu où il habite.
\VS{28}Familles des peuples, donnez à Yahweh, donnez à Yahweh gloire et force~!
\VS{29}Donnez à Yahweh la gloire due à son Nom~! Apportez des offrandes, et présentez-vous devant lui. Prosternez-vous devant Yahweh avec des ornements saints~!
\VS{30}Tremblez, vous tous habitants de la terre tout étonnés devant sa face~! Car la terre habitable est affermie par lui, et elle ne chancelle point.
\VS{31}Que les cieux se réjouissent, que la terre soit dans l'allégresse~! Et que l'on dise parmi les nations~: Yahweh règne~!
\VS{32}Que la mer retentisse avec tout ce qu'elle contient~! Que la campagne se réjouisse avec tout ce qu'elle renferme~!
\VS{33}Que les arbres de la forêt poussent des cris de joie au devant de Yahweh, parce qu'il vient juger la terre\FTNT{Yahweh vient juger la terre. Cette prophétie confirme de façon incontestable la divinité de Jésus-Christ. Voir Za. 14:1-7.}.
\VS{34}Célébrez Yahweh, car il est bon, car sa miséricorde demeure à jamais~!
\VS{35}Et dites~: Ô Dieu de notre salut, sauve-nous, et rassemble-nous, et retire-nous d'entre les nations, pour célébrer ton saint Nom, et que nous nous glorifions de ta louange~!
\VS{36}Béni soit Yahweh, le Dieu d'Israël, de siècle en siècle~! Et tout le peuple dit~: Amen~! Louez Yahweh~!
\VS{37}On laissa donc là, devant l'arche de l'alliance de Yahweh, Asaph et ses frères, pour faire le service continuellement, remplissant leur tâche jour par jour devant l'arche.
\VS{38}On laissa Obed-Edom, et ses frères, au nombre de soixante-huit, Obed-Edom, dis-je, fils de Jeduthun, et Hosa comme portiers.
\VS{39}On établit le prêtre Tsadok, et les prêtres ses frères, devant le tabernacle de Yahweh, dans le haut lieu qui était à Gabaon,
\VS{40}pour offrir des holocaustes à Yahweh continuellement sur l'autel de l'holocauste, matin et soir, selon tout ce qui est écrit dans la loi de Yahweh, qu'il ordonna à Israël.
\VS{41}Auprès d'eux étaient Héman et Jeduthun, et les autres qui furent choisis et désignés par leur nom, pour célébrer Yahweh, parce que sa miséricorde demeure éternellement.
\VS{42}Et Héman et Jeduthun étaient avec ceux-là~; il y avait aussi des trompettes et des cymbales pour ceux qui les faisaient retentir, et des instruments pour chanter les cantiques de Dieu. Les fils de Jeduthun étaient portiers.
\VS{43}Puis tout le peuple s'en alla chacun dans sa maison, et David aussi s'en retourna pour bénir sa maison.
\Chap{17}
\TextTitle{David veut construire un temple à Yahweh\FTNTT{2 S. 7:1-3.}}
\VerseOne{}Or il arriva après que David fut établi dans sa maison, qu'il dit à Nathan, le prophète~: Voici, j'habite dans une maison de cèdres, et l'arche de l'alliance de Yahweh est sous une tente.
\VS{2}Nathan dit à David~: Fais tout ce que tu as dans le cœur, car Dieu est avec toi.
\TextTitle{Réponse de Yahweh à David\FTNTT{2 S. 7:4-17.}}
\VS{3}Mais il arriva cette nuit-là que la parole de Dieu fut adressée à Nathan, en disant~:
\VS{4}Va, et dis à David, mon serviteur~: Ainsi parle Yahweh~: Tu ne me bâtiras point de maison pour y habiter.
\VS{5}Puisque je n'ai point habité dans une maison depuis le jour où j'ai fait monter les enfants d'Israël hors d'Egypte jusqu'à ce jour~; mais j'ai été de tente en tente, et de tabernacle en tabernacle.
\VS{6}Partout où j'ai marché avec tout Israël, ai-je dit un mot à un seul des juges d'Israël, auxquels j'ai ordonné de paître mon peuple, ai-je dit~: Pourquoi ne m'avez-vous point bâti une maison de cèdres~?
\VS{7}Maintenant donc tu diras ainsi à David, mon serviteur~: Ainsi parle Yahweh des armées~: Je t'ai pris d'une cabane, d'auprès des brebis, afin que tu sois le conducteur de mon peuple d'Israël~;
\VS{8}j'ai été avec toi partout où tu as marché, j'ai exterminé devant toi tous tes ennemis, et j'ai rendu ton nom semblable au nom des grands qui sont sur la terre.
\VS{9}J'ai établi un lieu pour mon peuple d'Israël, et je l'ai planté afin qu'il habite chez lui et ne soit plus agité. Les fils d'iniquité ne le détruiront plus comme ils l'ont fait auparavant,
\VS{10}et comme à l'époque où j'ai établi des juges sur mon peuple d'Israël. J'ai humilié tous tes ennemis. Je t'informe que Yahweh te bâtira une maison.
\VS{11}Il arrivera donc que quand tes jours seront accomplis pour t'en aller avec tes pères, je ferai lever ta postérité après toi, l'un de tes fils, et j'affermirai son règne\FTNT{Cette prophétie est relative au Messie. Voir 2 S. 7:12-17.}.
\VS{12}Il me bâtira une maison, et j'affermirai son trône éternellement.
\VS{13}Je serai pour lui un père, et il sera pour moi un fils~; et je ne retirerai point de lui ma grâce, comme je l'ai retirée de celui qui a été avant toi.
\VS{14}Mais je l'établirai dans ma maison et dans mon royaume éternellement, et son trône sera affermi pour toujours.
\VS{15}Nathan récita à David toutes ces paroles, et toute cette vision.
\TextTitle{Adoration et reconnaissance de David à Yahweh\FTNTT{2 S. 7:18-29.}}
\VS{16}Alors le roi David entra, et se tint devant Yahweh, et dit~: Ô Yahweh Dieu~! Qui suis-je, et quelle est ma maison, que tu m'aies fait parvenir au point où je suis~?
\VS{17}Mais cela t'a semblé être peu de chose, ô Dieu~! Et tu as parlé de la maison de ton serviteur pour le temps à venir, et tu as porté les regards sur moi à la manière de l'homme, toi qui es élevé, ô Yahweh Dieu~!
\VS{18}Que pourrait te dire encore David de l'honneur que tu fais à ton serviteur~? Car tu connais ton serviteur.
\VS{19}Ô Yahweh~! Pour l'amour de ton serviteur, et selon ton cœur, tu as fait toutes ces grandes choses, pour lui révéler toutes ces grandeurs.
\VS{20}Ô Yahweh~! Nul n'est semblable à toi, et il n'y a point d'autre Dieu que toi selon tout ce que nous avons entendu de nos oreilles.
\VS{21}Et qui est comme ton peuple d'Israël, la seule nation sur la terre que Dieu lui-même est venu racheter pour lui, afin qu'elle soit son peuple, et pour te faire un Nom et pour accomplir des miracles et des prodiges, en chassant les nations devant ton peuple que tu as racheté d'Egypte~?
\VS{22}Et tu as établi ton peuple d'Israël afin qu'il soit ton peuple à toujours~; et toi, ô Yahweh~! Tu as été son Dieu.
\VS{23}Maintenant donc, ô Yahweh~! Que la parole que tu as prononcée sur ton serviteur et sur sa maison, soit ferme à jamais, et agis selon ta parole~!
\VS{24}Et que ton Nom subsiste et soit magnifié éternellement, de sorte qu'on dise~: Yahweh des armées, le Dieu d'Israël, est Dieu pour Israël~; et que la maison de David, ton serviteur, soit affermie devant toi.
\VS{25}Car, ô mon Dieu~! Tu as révélé à ton serviteur que tu lui bâtirais une maison. C'est pourquoi ton serviteur a pris la hardiesse de te faire cette prière.
\VS{26}Maintenant, ô Yahweh~! Tu es Dieu, et tu as parlé de ce bien à ton serviteur.
\VS{27}Veuille donc maintenant bénir la maison de ton serviteur, afin qu'elle soit éternellement devant toi~; car tu l'as bénie, ô Yahweh~! Et elle sera bénie à jamais~!
\Chap{18}
\TextTitle{Le règne de David affermi\FTNTT{2 S. 8:1-18.}}
\VerseOne{}Et il arriva que David battit les Philistins, et les humilia, et il enleva de la main des Philistins, Gath et les villes de son ressort\FTNT{2 S. 8.}.
\VS{2}Il battit aussi les Moabites, et les Moabites furent asservis à David et lui payèrent un tribut.
\VS{3}David battit aussi Hadarézer, roi de Tsoba, vers Hamath, lorsqu'il alla établir sa domination sur le fleuve de l'Euphrate.
\VS{4}David lui prit mille chars, sept mille cavaliers, et vingt mille hommes de pied~; et il coupa les jarrets des chevaux de tous les chars, mais il réserva cent chars.
\VS{5}Les Syriens de Damas vinrent au secours d'Hadarézer, roi de Tsoba, et David battit vingt-deux mille Syriens.
\VS{6}Puis David mit des garnisons dans la Syrie de Damas. Et les Syriens furent assujettis à David et lui payèrent un tribut. Yahweh sauvait David partout où il allait.
\VS{7}Et David prit les boucliers d'or qui étaient aux serviteurs de Hadarézer, et les apporta à Jérusalem.
\VS{8}Il emporta aussi de Thibchath, et de Cun, villes de Hadarézer, une grande quantité d'airain, dont Salomon fit la mer d'airain, les colonnes et les ustensiles d'airain.
\VS{9}Thohu, roi de Hamath, apprit que David avait défait toute l'armée de Hadarézer, roi de Tsoba.
\VS{10}Et il envoya Hadoram, son fils, vers le roi David pour le saluer et le féliciter de ce qu'il avait combattu Hadarézer, et qu'il l'avait défait. Car Hadarézer était dans une guerre continuelle contre Thohu. Quant à tous les vases d'or, d'argent, et d'airain,
\VS{11}le roi David les consacra aussi à Yahweh, avec l'argent et l'or qu'il avait emporté de toutes les nations, à savoir d'Edom, de Moab, des fils d'Ammon, des Philistins, et d'Amalek.
\VS{12}Et Abischaï, fils de Tseruja battit dix-huit mille Edomites dans la vallée du sel.
\VS{13}Il mit une garnison dans Edom, et tous les Edomites furent asservis à David~; et Yahweh gardait David partout où il allait.
\VS{14}Ainsi, David régna sur tout Israël, rendant jugement et justice à tout son peuple.
\VS{15}Joab, fils de Tseruja, avait la charge de l'armée, et Josaphat, fils d'Achilud, était archiviste.
\VS{16}Tsadok, fils d'Achithub, et Abimélec, fils d'Abiathar, étaient les prêtres~; et Schavscha était le secrétaire.
\VS{17}Benaja, fils de Jehojada, était sur les Kéréthiens et les Péléthiens~; mais les fils de David étaient les premiers auprès du roi.
\Chap{19}
\TextTitle{David monte contre les Ammonites et les Syriens\FTNTT{2 S. 10.}}
\VerseOne{}Or il arriva après cela que Nachasch, roi des fils d'Ammon, mourut~; et son fils régna à sa place.
\VS{2}David dit~: J'userai de bonté envers Hanun, fils de Nachasch, car son père a usé de bonté envers moi. Ainsi, David envoya des messagers pour le consoler de la mort de son père~; et les serviteurs de David vinrent au pays des fils d'Ammon vers Hanun pour le consoler.
\VS{3}Mais les chefs d'entre les fils d'Ammon dirent à Hanun~: Penses-tu que ce soit pour honorer ton père que David t'a envoyé des consolateurs~? N'est-ce pas pour examiner et épier le pays, afin de le détruire, que ses serviteurs sont venus vers toi~?
\VS{4}Alors Hanun prit les serviteurs de David, les fit raser, et les fit couper leurs habits par le milieu jusqu'aux hanches. Puis il les renvoya.
\VS{5}Ils s'en allèrent, et le firent savoir par le moyen de quelques personnes à David, qui envoya des gens à leur rencontre~; car ces hommes-là étaient fort confus. Et le roi leur fit dire~: Restez à Jéricho jusqu'à ce que votre barbe ait repoussé, et revenez ensuite.
\VS{6}Or les fils d'Ammon voyant qu'ils s'étaient rendus odieux à David, Hanun et les fils d'Ammon envoyèrent mille talents d'argent pour prendre à leur solde des chars et des cavaliers de Mésopotamie, de Syrie, de Maaca et de Tsoba.
\VS{7}Ils prirent à leur solde trente-deux mille hommes et des chars, et le roi de Maaca avec son peuple, lesquels vinrent camper devant Médeba. Les fils d'Ammon aussi s'assemblèrent de leurs villes et vinrent pour combattre.
\VS{8}David l'ayant appris, envoya Joab et ceux de toute l'armée qui étaient les plus vaillants.
\VS{9}Les fils d'Ammon sortirent et rangèrent leur armée en bataille à l'entrée de la ville~; et les rois qui étaient venus étaient à part dans la campagne.
\VS{10}Joab, voyant que l'armée était tournée contre lui devant et derrière, prit de tous les gens d'élite d'Israël et les rangea contre les Syriens.
\VS{11}Et il donna la conduite du reste du peuple à Abischaï, son frère~; et on les rangea contre les fils d'Ammon.
\VS{12}Et Joab lui dit: Si les Syriens sont plus forts que moi, tu viendras me délivrer~; et si les fils d'Ammon sont plus forts que toi, je te délivrerai.
\VS{13}Sois ferme, et montrons-nous vaillants pour notre peuple, et pour les villes de notre Dieu~; et que Yahweh fasse ce qui lui semblera bon.
\VS{14}Alors Joab et le peuple qui était avec lui s'approchèrent pour livrer bataille aux Syriens qui s'enfuirent de devant lui.
\VS{15}Et les fils d'Ammon voyant que les Syriens s'étaient enfuis, eux aussi s'enfuirent devant Abischaï, frère de Joab, et rentrèrent dans la ville, et Joab revint à Jérusalem.
\VS{16}Mais les Syriens, qui avaient été battus par ceux d'Israël, envoyèrent des messagers et firent venir les Syriens qui étaient au-delà du fleuve~; et Schophach, chef de l'armée d'Hadarézer, les conduisait.
\VS{17}On le rapporta à David, qui assembla tout Israël, passa le Jourdain, alla au-devant d'eux et se rangea en bataille contre eux. David donc rangea la bataille contre les Syriens, et ils combattirent contre lui.
\VS{18}Mais les Syriens s'enfuirent de devant Israël~; et David défit sept mille chars des Syriens et quarante mille hommes de pied~; et il tua Schophach, le chef de l'armée.
\VS{19}Alors les serviteurs d'Hadarézer, voyant qu'ils avaient été battus par ceux d'Israël, firent la paix avec David, et lui furent asservis~; et les Syriens ne voulurent plus secourir les fils d'Ammon.
\Chap{20}
\TextTitle{Conquête de Rabba\FTNTT{2 S. 11:1-12:25~; 2 S. 12:26-31.}}
\VerseOne{}Or il arriva l'année suivante, au temps où les rois se mettaient en campagne, que Joab conduisit une forte armée et ravagea le pays des fils d'Ammon~; puis il alla assiéger Rabba, tandis que David resta à Jérusalem. Joab battit Rabba et la détruisit\FTNT{2 S. 12:26-31.}.
\VS{2}David enleva la couronne de dessus la tête de son roi, et il trouva qu'elle pesait un talent d'or~: Elle était garnie de pierres précieuses. On la mit sur la tête de David qui emmena un très grand butin de la ville.
\VS{3}Il emmena aussi le peuple qui y était, et les mit aux scies, aux pics de fer et aux haches de fer~; David traita de la sorte toutes les villes des fils d'Ammon~; puis il s'en retourna avec tout le peuple à Jérusalem.
\TextTitle{Guerre contre les Philistins\FTNTT{2 S. 21:15-22.}}
\VS{4}Il arriva après cela que la guerre continua à Guézer contre les Philistins. Alors Sibbecaï, le Huschatite, frappa Sippaï, qui était des fils de Rapha, et ils furent humiliés\FTNT{2 S. 21:15-22.}.
\VS{5}Il y eut encore une autre guerre contre les Philistins. Et Elchanan, fils de Jaïr, frappa Lachmi, frère de Goliath de Gath, qui avait une lance dont le bois était comme une ensouple de tisserand.
\VS{6}Il y eut encore une autre guerre à Gath, où se trouva un homme de grande stature, qui avait six doigts à chaque main, et six orteils à chaque pied, de sorte qu'il en avait en tout vingt-quatre~; et il était aussi issu de Rapha.
\VS{7}Et il défia Israël~; mais Jonathan, fils de Schimea, frère de David, le tua.
\VS{8}Ceux-là naquirent à Gath~; ils étaient des enfants de Rapha, et ils moururent par les mains de David et par les mains de ses serviteurs.
\Chap{21}
\TextTitle{David fait le dénombrement contre la volonté de Yahweh\FTNTT{2 S. 24:1-17.}}
\VerseOne{}Mais Satan s'éleva contre Israël, et il incita David à faire le dénombrement d'Israël.
\VS{2}Et David dit à Joab et aux chefs du peuple~: Allez et faites le dénombrement d'Israël, depuis Beer-Schéba jusqu'à Dan, et rapportez-le-moi, afin que j'en connaisse le nombre.
\VS{3}Mais Joab répondit~: Que Yahweh veuille augmenter son peuple cent fois encore plus qu'il ne l'est, ô roi, mon seigneur. Tous ne sont-ils pas serviteurs de mon seigneur~? Pourquoi mon seigneur cherche-t-il cela~? Et pourquoi cela serait-il imputé comme un crime à Israël~?
\VS{4}Mais la parole du roi l'emporta sur Joab. Et Joab partit et parcourut tout Israël~; puis il revint à Jérusalem.
\VS{5}Et Joab donna à David le rôle du dénombrement du peuple, et il se trouva dans tout Israël onze cent mille hommes tirant l'épée~; et dans Juda quatre cent soixante-dix mille hommes tirant l'épée.
\VS{6}Bien qu'il n'eût pas compté entre eux ni Lévi ni Benjamin, parce que Joab exécutait la parole du roi en l'ayant en abomination,
\VS{7}cette chose déplut à Dieu, c'est pourquoi il frappa Israël.
\VS{8}Et David dit à Dieu~: J'ai commis un très grand péché d'avoir fait une telle chose~; je te prie, pardonne maintenant l'iniquité de ton serviteur, car j'ai agi en insensé.
\VS{9}Et Yahweh parla à Gad, le voyant de David, en disant~:
\VS{10}Va, parle à David, et dis-lui~: Ainsi parle Yahweh, je te propose trois choses~; choisis l'une d'elles, afin que je te la fasse.
\VS{11}Et Gad vint à David, et lui dit~: Ainsi parle Yahweh~:
\VS{12}Choisis soit la famine durant l'espace de trois ans~; soit d'être consumé durant trois mois, étant poursuivi par tes ennemis, en sorte que l'épée de tes ennemis t'atteigne~; ou que l'épée de Yahweh, la peste, soit durant trois jours sur le pays, et que l'Ange de Yahweh porte la destruction dans toutes les contrées d'Israël. Vois maintenant ce que j'aurai à répondre à celui qui m'a envoyé.
\VS{13}Alors David répondit à Gad~: Je suis dans une très grande angoisse~! Que je tombe, je te prie, entre les mains de Yahweh, parce que ses compassions sont immenses~; mais que je ne tombe point entre les mains des hommes~!
\VS{14}Yahweh envoya donc la peste sur Israël, et il tomba soixante-dix mille hommes d'Israël.
\VS{15}Dieu envoya aussi un ange à Jérusalem pour la détruire~; et comme il la détruisait, Yahweh regarda et se repentit de ce mal. Et il dit à l'ange qui détruisait~: C'est assez~! Retire à présent ta main. Et l'Ange de Yahweh se tenait près de l'aire d'Ornan, le Jébusien.
\VS{16}Or David leva les yeux et vit l'Ange de Yahweh\FTNT{Ge. 16:7.} qui était entre la terre et le ciel, ayant dans sa main son épée nue, tournée contre Jérusalem. Et David et les anciens, couverts de sacs, tombèrent sur leurs faces.
\VS{17}Et David dit à Dieu~: N'est-ce pas moi qui ai ordonné qu'on fasse le dénombrement du peuple~? C'est donc moi qui ai péché et qui ai très mal agi~; mais ces brebis qu'ont-elles fait~? Yahweh, mon Dieu~! Je te prie que ta main soit contre moi, et contre la maison de mon père, mais qu'elle ne soit pas contre ton peuple, pour le détruire.
\TextTitle{Fin de la plaie après l'offrande de David\FTNTT{2 S. 24:18-25.}}
\VS{18}Alors l'Ange de Yahweh ordonna à Gad de dire à David, qu'il monte pour dresser un autel à Yahweh, dans l'aire d'Ornan, le Jébusien.
\VS{19}David donc monta selon la parole que Gad lui avait dite au Nom de Yahweh.
\VS{20}Ornan s'étant retourné, et ayant vu l'Ange, ses quatre fils se cachèrent avec lui. Or Ornan foulait du blé.
\VS{21}David vint jusqu'à Ornan, et Ornan regarda, et ayant vu David, il sortit de l'aire et se prosterna devant lui, le visage à terre.
\VS{22}Et David dit à Ornan~: Donne-moi la place de cette aire, et j'y bâtirai un autel à Yahweh~; donne-la-moi pour le prix qu'elle vaut, afin que cette plaie soit arrêtée de dessus le peuple.
\VS{23}Et Ornan dit à David~: Prends-la, et que le roi, mon seigneur, fasse tout ce qui lui semblera bon. Voici, je donne ces bœufs pour les holocaustes, et ces instruments à fouler du blé pour le bois, et ce blé pour l'offrande~; je donne toutes ces choses.
\VS{24}Mais le roi David lui répondit~: Non, mais certainement j'achèterai tout cela au prix qu'il vaut~; car je ne présenterai point à Yahweh ce qui est à toi, et je n'offrirai point un holocauste qui ne me coûte rien.
\VS{25}David donna donc à Ornan pour cette place, six cents sicles d'or de poids.
\VS{26}Puis il bâtit là un autel à Yahweh, et il offrit des holocaustes et des sacrifices d'offrande de paix, et il invoqua Yahweh, qui l'exauça par le feu envoyé des cieux sur l'autel de l'holocauste.
\VS{27}Alors Yahweh parla à l'ange, et l'ange remit son épée dans son fourreau.
\VS{28}En ce temps-là, David, voyant que Yahweh l'avait exaucé dans l'aire d'Ornan, le Jébusien, y offrait des sacrifices.
\VS{29}Or le tabernacle de Yahweh, que Moïse avait construit au désert, et l'autel des holocaustes, étaient en ce temps-là dans le haut lieu de Gabaon.
\VS{30}Mais David ne pouvait pas aller devant cet autel pour invoquer Dieu, parce qu'il avait été épouvanté à cause de l'épée de l'Ange de Yahweh.
\Chap{22}
\TextTitle{Préparatifs de David pour la construction du temple}
\VerseOne{}Et David dit~: C'est ici la maison de Yahweh Dieu, et c'est ici l'autel pour les holocaustes d'Israël.
\VS{2}David ordonna de rassembler les étrangers qui étaient dans le pays d'Israël, et il établit des tailleurs de pierres pour tailler des pierres de taille, pour la construction de la maison de Dieu.
\VS{3}David prépara aussi du fer en abondance, afin d'en faire des clous pour les battants des portes et pour les crampons, de l'airain en quantité telle qu'il n'était pas possible de le peser,
\VS{4}et du bois de cèdre sans nombre, parce que les Sidoniens et les Tyriens amenaient à David du bois de cèdre en abondance.
\VS{5}David dit~: Salomon, mon fils, est jeune et délicat, et la maison qu'il faut bâtir à Yahweh doit être magnifique en excellence, en réputation, et en gloire, dans tous les pays. Je lui préparerai donc maintenant de quoi la bâtir. Ainsi, David prépara, avant sa mort, ces choses en abondance.
\TextTitle{Recommandation de David à Salomon}
\VS{6}Puis il appela Salomon, son fils, et lui ordonna de bâtir une maison à Yahweh, le Dieu d'Israël.
\VS{7}David donc dit à Salomon~: Mon fils, j'avais à cœur de bâtir une maison au Nom de Yahweh, mon Dieu.
\VS{8}Mais la parole de Yahweh m'a été adressée, en disant~: Tu as répandu beaucoup de sang, et tu as fait de grandes guerres~; tu ne bâtiras point de maison à mon Nom, parce que tu as répandu beaucoup de sang sur la terre devant moi.
\VS{9}Voici, il te naîtra un fils, qui sera un homme de repos, et à qui je donnerai du repos par rapport à tous ses ennemis tout autour, c'est pourquoi son nom sera Salomon. Et en son temps, je donnerai la paix et le repos à Israël.
\VS{10}Ce sera lui qui bâtira une maison à mon Nom~; et il sera un fils pour moi, et je serai un père pour lui~; et j'affermirai le trône de son règne sur Israël à jamais.
\VS{11}Maintenant donc, mon fils, Yahweh sera avec toi, et tu prospéreras, et tu bâtiras la maison de Yahweh, ton Dieu, ainsi qu'il l'a déclaré à ton égard.
\VS{12}Seulement, que Yahweh te donne de la sagesse et de l'intelligence, et qu'il t'instruise touchant le gouvernement d'Israël, et comment tu dois garder la loi de Yahweh, ton Dieu.
\VS{13}Tu prospéreras si tu as soin de mettre en pratique les lois et les ordonnances que Yahweh a prescrites à Moïse pour Israël. Fortifie-toi et prends courage~; ne crains point et ne t'effraie de rien.
\VS{14}Voici, selon ma petitesse, j'ai préparé pour la maison de Yahweh cent mille talents d'or et un million de talents d'argent. Quant à l'airain et au fer, il est d'un poids incalculable, car il est en abondance. J'ai aussi préparé le bois et les pierres~; et tu y ajouteras ce qu'il faudra.
\VS{15}Tu as avec toi beaucoup d'ouvriers, de maçons, de tailleurs de pierres, de charpentiers, et toutes sortes de gens experts dans toute espèce d'ouvrage.
\VS{16}Il y a de l'or et de l'argent, de l'airain et du fer sans nombre. Lève-toi et agis, et Yahweh sera avec toi.
\VS{17}David ordonna aussi à tous les chefs d'Israël d'aider Salomon, son fils~; et il leur dit~:
\VS{18}Yahweh,votre Dieu, n'est-il pas avec vous, et ne vous a-t-il pas donné du repos de tous côtés~? Car il a livré entre mes mains les habitants du pays, et le pays a été soumis devant Yahweh, et devant son peuple.
\VS{19}Maintenant donc, appliquez vos cœurs et vos âmes à rechercher Yahweh, votre Dieu~; levez-vous et bâtissez le sanctuaire de Yahweh Dieu, afin d'amener l'arche de l'alliance de Yahweh, et les ustensiles consacrés à Dieu dans la maison qui doit être bâtie au Nom de Yahweh.
\Chap{23}
\TextTitle{David désigne Salomon comme son successeur\FTNTT{1 Ch. 28:1.}}
\VerseOne{}David étant vieux et rassasié de jours, établit Salomon, son fils, pour roi sur Israël.
\VS{2}Et il assembla tous les principaux d'Israël, les prêtres et les Lévites.
\VS{3}On fit le dénombrement des Lévites, depuis l'âge de trente ans et au-dessus~; et les mâles d'entre-eux étant comptés, chacun par tête, il y eut trente-huit mille hommes\FTNT{No. 3:25-37.}.
\VS{4}Et David dit~: Qu'il y en ait parmi eux vingt-quatre mille pour vaquer ordinairement à l'œuvre de la maison de Yahweh, et six mille comme magistrats et juges,
\VS{5}quatre mille portiers, et quatre autres mille pour louer Yahweh avec des instruments que j'ai faits pour le louer.
\TextTitle{Dénombrement des Lévites\FTNTT{No. 3:25-37.}}
\VS{6}David les divisa en classes d'après les fils de Lévi, à savoir Guerschon, Kehath et Merari.
\VS{7}Des Guerschonites, il y eut Laedan et Schimeï.
\VS{8}Les fils de Laedan furent ces trois~: Jehiel le premier, puis Zétham, puis Joël.
\VS{9}Les fils de Schimeï furent ces trois~: Schelomith, Haziel et Haran. Ce sont là les chefs des pères de la famille de Laedan.
\VS{10}Et les fils de Schimeï furent Jachath, Zina, Jeusch et Beria. Ce sont là les quatre fils de Schimeï.
\VS{11}Jachath était le premier et Zina le second~; mais Jeusch et Beria n'eurent pas beaucoup de fils, c'est pourquoi ils furent comptés pour une seule maison de père dans le dénombrement.
\VS{12}Des fils de Kehath il y eut Amram, Jitsehar, Hébron et Uziel, en tout quatre.
\VS{13}Les fils d'Amram furent Aaron et Moïse. Aaron fut séparé lui et ses fils à toujours, pour sanctifier le Saint des saints, pour faire brûler des parfums en présence de Yahweh, pour le servir, et pour bénir en son Nom à toujours.
\VS{14}Et quant à Moïse, homme de Dieu, ses fils devaient être comptés de la tribu de Lévi.
\VS{15}Les fils de Moïse furent Guerschom et Eliézer.
\VS{16}Des fils de Guerschom, Schebuel le premier.
\VS{17}Quant aux fils d'Eliézer, Rechabia fut le premier~; et Eliézer n'eut point d'autres fils, mais les fils de Rechabia furent très nombreux.
\VS{18}Des fils de Jitsehar, Schelomith était le premier.
\VS{19}Les fils de Hébron furent Jerija le premier, Amaria le second, Jachaziel le troisième, Jekameam le quatrième.
\VS{20}Les fils d'Uziel furent Michée le premier, Jischija le second.
\VS{21}Des fils de Merari il y eut Machli et Muschi. Les fils de Machli furent Eléazar et Kis.
\VS{22}Eléazar mourut, et n'eut point de fils, mais des filles~; et les fils de Kis, leurs frères les prirent pour femmes.
\VS{23}Les fils de Muschi furent Machli, Eder et Jerémoth, eux trois.
\TextTitle{Fonctions des Lévites\FTNTT{No. 3:5-12.}}
\VS{24}Ce sont là les fils de Lévi, selon les maisons de leurs pères, chefs des pères, selon leurs dénombrements qui furent faits en comptant leurs noms, étant comptés chacun par tête~; et ils faisaient l'œuvre du service de la maison de Yahweh, depuis l'âge de vingt ans et au-dessus.
\VS{25}Car David dit~: Yahweh, le Dieu d'Israël, a donné du repos à son peuple, et il établira sa demeure dans Jérusalem à toujours.
\VS{26}Quant aux Lévites, ils n'auront plus à porter le tabernacle ni tous les ustensiles pour son service.
\VS{27}C'est pourquoi, dans les derniers registres de David, les fils de Lévi furent dénombrés depuis l'âge de vingt ans et au-dessus.
\VS{28}Car leur charge était d'assister les fils d'Aaron pour le service de la maison de Yahweh, étant établis sur le parvis et sur les chambres, pour la purification de toutes les choses saintes, pour l'œuvre du service de la maison de Dieu,
\VS{29}pour les pains de proposition, de la fleur de farine pour l'offrande, des galettes sans levain, pour tout ce qui se cuit sur la plaque, pour tout ce qui est rissolé, et pour la petite et grande mesure,
\VS{30}pour se présenter tous les matins et tous les soirs, afin de célébrer et louer Yahweh,
\VS{31}et offrir tous les holocaustes qu'il fallait offrir à Yahweh les jours de sabbat, aux nouvelles lunes, et aux fêtes solennelles, continuellement devant Yahweh, selon le nombre et les usages prescrits.
\VS{32}Ils donnaient leurs soins à la tente d'assignation, au lieu saint, et aux fils d'Aaron, leurs frères, pour le service de la maison de Yahweh.
\Chap{24}
\TextTitle{Vingt-quatre classes de prêtres}
\VerseOne{}Quant aux fils d'Aaron, voici leurs classes\FTNT{Les vingt-quatre classes de prêtres qui se tenaient devant Yahweh dans le temple de Jérusalem étaient une représentation des vingt-quatre vieillards qui se tiennent devant le trône de Dieu (Ap. 4:4).}. Les fils d'Aaron furent Nadab, Abihu, Eléazar et Ithamar.
\VS{2}Mais Nadab et Abihu\FTNT{Lé. 10:1-4.} moururent en présence de leur père, et n'eurent point de fils~; et Eléazar et Ithamar exercèrent la prêtrise.
\VS{3}Or David les sépara, à savoir Tsadok, qui était des fils d'Eléazar, et Achimélec, qui était des fils d'Ithamar, en fonction de leurs charges dans le service qu'ils avaient à faire.
\VS{4}Il se trouva parmi les fils d'Eléazar plus de chefs que parmi les fils d'Ithamar, et on en fit la division~; les fils d'Eléazar avaient seize chefs, selon leurs familles, et n’y en ayant eu que huit, des enfants d’Ithamar, selon leurs familles.
\VS{5}Et on les classa par le sort, les entremêlant les uns avec les autres, car les chefs du sanctuaire et les chefs de la maison de Dieu furent tirés tant des fils d'Eléazar que des fils d'Ithamar.
\VS{6}Schemaeja, fils de Nethaneel, le scribe, qui était de la tribu de Lévi, les mit par écrit devant le roi, les princes du peuple, devant Tsadok, le prêtre, et Achimélec, fils d'Abiathar et devant les chefs des pères des prêtres et des Lévites. On tira au sort une maison de père pour Eléazar et une autre fut tirée pour Ithamar.
\VS{7}Le premier sort échut à Jehojarib, le second à Jedaeja,
\VS{8}le troisième à Harim, le quatrième à Seorim,
\VS{9}le cinquième à Malkija, le sixième à Mijamin,
\VS{10}le septième à Hakkots, le huitième à Abija,
\VS{11}le neuvième à Josué, le dixième à Schecania,
\VS{12}le onzième à Eliaschib, le douzième à Jakim,
\VS{13}le treizième à Huppa, le quatorzième à Jeschébeab,
\VS{14}le quinzième à Bilga, le seizième à Immer,
\VS{15}le dix-septième à Hézir, le dix-huitième à Happitsets,
\VS{16}le dix-neuvième à Pethachja, le vingtième à Ezéchiel,
\VS{17}le vingt et unième à Jakim, et le vingt-deuxième à Gamul,
\VS{18}le vingt-troisième à Delaja, le vingt-quatrième à Maazia.
\VS{19}Tel fut leur classement pour le service qu'ils avaient à faire, lorsqu'ils entraient dans la maison de Yahweh, selon qu'il leur avait été ordonné par Aaron, leur père, comme Yahweh, le Dieu d'Israël, le lui avait ordonné.
\TextTitle{Les chefs des Lévites~; les fils de Kehath et de Merari}
\VS{20}Voici les chefs du reste des Lévites. Des fils de Amram~: Schubaël~; et des fils de Shubaël, Jechdia.
\VS{21}De Rechabia, des fils de Rechabia, Jischija était le premier.
\VS{22}Des Jitseharites, Schelomoth~; des fils de Schelomoth, Jachath.
\VS{23}Des fils d'Hébron, Jerija, Amaria le second~; Jachaziel le troisième, Jekameam le quatrième.
\VS{24}Des fils d'Uziel, Michée~; des fils de Michée, Schamir.
\VS{25}Le frère de Michée était Jischija~; des fils de Jischija, Zacharie.
\VS{26}Des fils de Merari, Machli et Muschi. Des fils de Jaazija, son fils.
\VS{27}Des fils de Merari, de Jaazija, son fils~: Schoham, Zaccur et Ibri.
\VS{28}De Machli, Eléazar, qui n'eut point de fils.
\VS{29}De Kis, les fils de Kis, Jerachmeel.
\VS{30}Et des fils de Muschi, Machli, Eder et Jerimoth. Ce sont là les fils des Lévites, selon les maisons de leurs pères.
\VS{31}Eux aussi, comme leurs frères, les fils d'Aaron, ils tirèrent au sort, devant le roi David, Tsadok et Achimélec, et les chefs des pères des prêtres et des Lévites. Il en fut ainsi pour chaque chef de maison comme pour le moindre de ses frères.
\Chap{25}
\TextTitle{Dénombrement des chantres}
\VerseOne{}David et les chefs de l'armée mirent à part pour le service ceux des fils d'Asaph, d'Héman et de Jeduthun qui prophétisaient avec des harpes, des luths et des cymbales. Et voici le nombre des hommes employés pour le service qu'ils avaient à faire.
\VS{2}Des fils d'Asaph~: Zaccur, Joseph, Nethania et Aschareéla, fils d'Asaph, sous la conduite d'Asaph, qui prophétisait selon les ordres du roi.
\VS{3}De Jeduthun, les six fils de Jeduthun~: Guedalia, Tseri, Esaïe, Haschabia, Matthithia et Schimeï, jouaient de la harpe, sous la conduite de leur père Jeduthun, qui prophétisait en célébrant et louant Yahweh.
\VS{4}D'Héman, les fils d'Héman~: Bukkija, Matthania, Uziel, Schebuel, Jerimoth, Hanania, Hanani, Eliatha, Guiddalthi, Romamthi-Ezer, Joschbekascha, Mallothi, Hothir, Machazioth.
\VS{5}Tous ceux-là étaient fils d'Héman, le voyant du roi, qui révélait les paroles de Dieu pour en exalter la puissance. Dieu donna à Héman quatorze fils et trois filles.
\VS{6}Tous ceux-là étaient employés, sous la conduite de leurs pères, aux cantiques de la maison de Yahweh, avec des cymbales, des luths, et des harpes, dans le service de la maison de Dieu, selon les ordres du roi donnés à Asaph, à Jeduthun et à Héman.
\VS{7}Et leur nombre avec leurs frères, auxquels on avait enseigné les cantiques de Yahweh, était de deux cent quatre-vingt-huit, tous très habiles.
\TextTitle{Répartition des chantres en vingt-quatre classes}
\VS{8}Et ils tirèrent au sort pour leurs fonctions, petits et grands, maîtres et disciples.
\VS{9}Et le premier sort échut à Asaph, à savoir à Joseph. Le second à Guedalia, lui, ses frères et ses fils étaient douze.
\VS{10}Le troisième à Zaccur, lui, ses fils et ses frères étaient douze.
\VS{11}Le quatrième à Jitseri, lui, ses fils et ses frères étaient douze.
\VS{12}Le cinquième à Nethania, lui, ses fils et ses frères étaient douze.
\VS{13}Le sixième à Bukkija, lui, ses fils et ses frères étaient douze.
\VS{14}Le septième à Jesareéla, lui, ses fils et ses frères étaient douze.
\VS{15}Le huitième à Esaïe, lui, ses fils et ses frères étaient douze.
\VS{16}Le neuvième à Matthania, lui, ses fils et ses frères étaient douze.
\VS{17}Le dixième à Schimeï, lui, ses fils et ses frères étaient douze.
\VS{18}L'onzième à Azareel, lui, ses fils et ses frères étaient douze.
\VS{19}Le douzième à Haschabia, lui, ses fils, et ses frères étaient douze.
\VS{20}Le treizième à Schubaël, lui, ses fils et ses frères étaient douze.
\VS{21}Le quatorzième à Matthithia, lui, ses fils et ses frères étaient douze.
\VS{22}Le quinzième à Jerémoth, lui, ses fils et ses frères étaient douze.
\VS{23}Le seizième à Hanania, lui, ses fils et ses frères étaient douze.
\VS{24}Le dix-septième à Joschbekascha, lui, ses fils et ses frères étaient douze.
\VS{25}Le dix-huitième à Hanani, lui, ses fils et ses frères étaient douze.
\VS{26}Le dix-neuvième à Mallothi, lui, ses fils et ses frères étaient douze.
\VS{27}Le vingtième à Elijatha, lui, ses fils et ses frères étaient douze.
\VS{28}Le vingt et unième à Hothir, lui, ses fils et ses frères étaient douze.
\VS{29}Le vingt-deuxième à Guiddalthi, lui, ses fils et ses frères étaient douze.
\VS{30}Le vingt-troisième à Machazioth, lui, ses fils et ses frères étaient douze.
\VS{31}Le vingt-quatrième à Romamthi-Ezer, lui, ses fils et ses frères étaient douze.
\Chap{26}
\TextTitle{Les classes des portiers}
\VerseOne{}Et quant aux classes des portiers, il y eut pour les Koréites~: Meschélémia, fils de Koré, d'entre les fils d'Asaph.
\VS{2}Les fils de Meschélémia furent Zacharie, le premier-né, Jediaël le second, Zebadia le troisième, Jathniel le quatrième,
\VS{3}Elam le cinquième, Jochanan le sixième et Eljoénaï le septième.
\VS{4}Les fils d'Obed-Edom furent Schemaeja le premier-né, Jozabad le second, Joach le troisième, Sacar le quatrième, Nethaneel le cinquième,
\VS{5}Ammiel le sixième, Issacar le septième, Peulthaï le huitième~; car Dieu l'avait béni.
\VS{6}A Schemaeja, son fils, naquirent des fils qui eurent le commandement sur la maison de leur père, parce qu'ils étaient des hommes forts et vaillants.
\VS{7}Les fils donc de Schemaeja furent Othni, Rephaël, Obed, Elzabad et ses frères, hommes vaillants, Elihu et Semaeja.
\VS{8}Tous ceux-là étaient des fils d'Obed-Edom, eux, leurs fils et leurs frères, étaient des hommes pleins de vigueur et de force pour le service~; ils étaient soixante-deux d'Obed-Edom.
\VS{9}Les fils de Meschélémia avec ses frères, vaillants hommes étaient au nombre de dix-huit.
\VS{10}Les fils de Hosa, d'entre les fils de Merari, furent Schimri le chef, quoiqu'il ne fût pas l'aîné, néanmoins son père l'établit pour chef~;
\VS{11}Hilkija était le second, Thebalia le troisième, Zacharie le quatrième~; tous les fils et frères de Hosa furent treize.
\VS{12}A ces classes de portiers, aux chefs de ces hommes et à leurs frères, fut remise la garde pour le service de la maison de Yahweh.
\VS{13}Ils tirèrent au sort pour chaque porte, autant pour le plus petit que pour le plus grand, selon leurs familles.
\VS{14}Et ainsi, le sort pour la porte vers l'orient échut à Schélémia. Puis on tira au sort pour Zacharie, son fils, qui était un sage conseiller, et la porte du côté du nord lui fut échue par le sort.
\VS{15}Le sort d'Obed-Edom lui échut pour la porte du côté du sud, et la maison des magasins échut à ses fils.
\VS{16}A Schuppim et à Hosa pour la porte vers l'occident, auprès de la porte de Schalléketh, au chemin montant~; une garde étant vis-à-vis de l'autre.
\VS{17}Il y avait vers l'orient six Lévites~; vers le nord, quatre par jour~; vers le sud, quatre aussi par jour~; et vers la maison des magasins, deux de chaque côté~;
\VS{18}du côté du faubourg vers l'occident, il y en avait quatre au chemin, et deux vers le faubourg.
\VS{19}Ce sont là les classes des portiers pour les fils des Koréites, et pour les fils de Merari.
\TextTitle{Les Lévites commis sur les trésors du temple}
\VS{20}Ceux-ci aussi étaient Lévites~: Achija commis sur les trésors de la maison de Dieu et les trésors des choses consacrées.
\VS{21}Des fils de Laedan, qui étaient d'entre les fils des Guerschonites, du côté de Laedan, d'entre les chefs des pères appartenant à Laedan le Guerschonite, Jehiéli.
\VS{22}D'entre les fils de Jehiéli~: Zétham et Joël, son frère, commis sur les trésors de la maison de Yahweh.
\VS{23}Pour les Amramites, les Jitseharites, les Hébronites et les Uziélites,
\VS{24}Schebuel, fils de Guerschom, fils de Moïse, était commis sur les autres trésors.
\VS{25}Et quant à ses frères issus d'Eliézer, dont Rechabia fut fils, dont le fils fut Esaïe, dont le fils fut Joram, dont le fils fut Zicri, dont le fils fut Schelomith,
\VS{26}c'étaient Schelomith et ses frères qui gardaient tous les trésors des choses saintes que le roi David, les chefs des pères, les chefs de milliers et de centaines, et les chefs de l'armée avaient consacrées.
\VS{27}C'était le butin de guerre qu'ils avaient consacré, pour l'entretien de la maison de Yahweh.
\VS{28}Tout ce qu'avait consacré Samuel, le voyant, Saül, fils de Kis, Abner, fils de Ner et Joab, fils de Tseruja, toutes les choses consacrées étaient mises sous la main de Schelomith et de ses frères.
\TextTitle{Les magistrats et juges en Israël}
\VS{29}Parmi les Jitseharites, Kenania et ses fils étaient employés aux affaires extérieures sur Israël pour être magistrats et juges.
\VS{30}Quant aux Hébronites, Haschabia et ses frères, hommes vaillants, au nombre de mille sept cents, ils avaient la surveillance d'Israël de l'autre côté du Jourdain, vers l'occident, pour toute œuvre qui concernait Yahweh, et pour le service du roi.
\VS{31}Quant aux Hébronites, selon leurs générations dans les familles des pères, Jerija fut le chef des Hébronites. On fit une recherche au sujet des Hébronites à la quarantième année du règne de David, et on trouva parmi eux à Jaezer de Galaad, des hommes forts et vaillants.
\VS{32}Les frères de Jerija, hommes vaillants, furent deux mille sept cents, issus des chefs des pères~; et le roi David les établit sur les Rubénites, sur les Gadites, et sur la demi-tribu de Manassé, pour toute œuvre qui concernait Dieu, et pour les affaires du roi.
\Chap{27}
\TextTitle{Les douze chefs de guerre de David}
\VerseOne{}Quant aux fils d'Israël, selon leur dénombrement, il y avait des chefs de pères, des chefs de milliers et de centaines, et leurs officiers, qui servaient le roi pour tout ce qui concernait les divisions, leur arrivée et leur départ, mois par mois, pendant tous les mois de l'année, et chaque division était de vingt-quatre mille hommes.
\VS{2}Et Jaschobeam, fils de Zabdiel, présidait sur la première division, pour le premier mois~; et dans sa division il y avait vingt-quatre mille hommes.
\VS{3}Il était des fils de Pérets, chef de tous les capitaines de l'armée du premier mois.
\VS{4}Dodaï, l'Achochite, présidait sur la division du deuxième mois, Mikloth, étant l'un des chefs de sa division~; et il avait une division de vingt-quatre mille hommes.
\VS{5}Le chef de la troisième armée pour le troisième mois était Benaja, fils de Jehojada, le prêtre et le capitaine en chef~; et dans sa division il y avait vingt-quatre mille hommes.
\VS{6}C'est ce Benaja qui était fort entre les trente, et par dessus les trente~; et Ammizadab, son fils, était dans sa division.
\VS{7}Le quatrième pour le quatrième mois était Asaël, frère de Joab, et Zébadia son fils, après lui~; et il y avait dans sa division vingt-quatre mille hommes.
\VS{8}Le cinquième pour le cinquième mois était le capitaine Schamehuth, le Jizrachite~; et dans sa division il y avait vingt-quatre mille hommes.
\VS{9}Le sixième pour le sixième mois était Ira, fils d'Ikkesch le Tekoïte~; et dans sa division il y avait vingt-quatre mille hommes.
\VS{10}Le septième pour le septième mois était Hélets le Pelonite, des fils d'Ephraïm~; et il y avait dans sa division vingt-quatre mille hommes.
\VS{11}Le huitième pour le huitième mois était Sibbecaï le Huschatite, de la famille des Zérachites~; et il y avait dans sa division vingt-quatre mille hommes.
\VS{12}Le neuvième pour le neuvième mois était Abiézer d'Anathoth, des Benjamites~; et il y avait dans sa division vingt-quatre mille hommes.
\VS{13}Le dixième pour le dixième mois était Maharaï de Nethopha, de la famille des Zérachites~; et il y avait dans sa division vingt-quatre mille hommes.
\VS{14}Le onzième pour le onzième mois était Benaja de Pirathon, des fils d'Ephraïm~; et il y avait dans sa division vingt-quatre mille hommes.
\VS{15}Le douzième pour le douzième mois était Heldaï de Nethopha, appartenant à Othniel~; et il y avait dans sa division vingt-quatre mille hommes.
\TextTitle{Les douze chefs des tribus d'Israël}
\VS{16}Et ceux-ci présidaient sur les tribus d'Israël~: Eliézer, fils de Zicri, était le conducteur des Rubénites. Des Siméonites~: Schephathia, fils de Maaca.
\VS{17}Des Lévites, Haschabia, fils de Kemuel. De ceux d'Aaron~: Tsadok.
\VS{18}De Juda~: Elihu, qui était des frères de David. De ceux d'Issacar~: Omri, fils de Micaël.
\VS{19}De ceux de Zabulon~: Jischemaeja, fils d'Abdias. De ceux de Nephthali~: Jerimoth, fils d'Azriel.
\VS{20}Des fils d'Ephraïm~: Hosée, fils d'Azazia. De la demi-tribu de Manassé~: Joël, fils de Pedaja.
\VS{21}De l'autre demi-tribu de Manassé en Galaad~: Jiddo, fils de Zacharie. De ceux de Benjamin~: Jaasiel, fils d'Abner.
\VS{22}De ceux de Dan~: Azareel, fils de Jerocham. Ce sont là les chefs des tribus d'Israël.
\TextTitle{Dénombrement arrêté par Yahweh}
\VS{23}Mais David ne fit point le dénombrement des Israélites, depuis l'âge de vingt ans et au-dessous~; parce que Yahweh avait dit qu'il multiplierait Israël comme les étoiles du ciel.
\VS{24}Joab, fils de Tseruja, avait bien commencé à en faire le dénombrement, mais il n'acheva pas parce que la colère de Dieu s'était répandue à cause de cela sur Israël~; c'est pourquoi ce dénombrement ne fut point mis parmi les dénombrements enregistrés dans les Chroniques du roi David.
\TextTitle{Les gestionnaires de David}
\VS{25}Or Azmaveth, fils d'Adiel, était commis sur les finances du roi~; mais Jonathan, fils d'Ozias, était commis sur les provisions dans les champs, dans les villes, les villages et les châteaux.
\VS{26}Et Ezri, fils de Kelub, était commis sur ceux qui travaillaient dans la campagne et cultivaient la terre.
\VS{27}Et Schimeï de Rama sur les vignes, et Zabdi de Schepham sur ce qui provenait des vignes, et sur les celliers du vin.
\VS{28}Et Baal-Hanan de Guéder sur les oliviers et sur les figuiers qui étaient à la campagne~; et Joasch sur les celliers à huile.
\VS{29}Schithraï de Saron était commis sur le gros bétail qui paissait en Saron~; Schaphath, fils d'Adlaï, sur le gros bétail qui paissait dans les vallées.
\VS{30}Obil, l'Ismaélite, sur les chameaux~; Jechdia de Méronoth, sur les ânesses.
\VS{31}Jaziz, l'Hagarénien, sur les troupeaux du menu bétail. Tous ceux-là avaient la charge des biens qui appartenaient au roi David.
\TextTitle{Les conseillers de David}
\VS{32}Mais Jonathan, oncle de David, était conseiller, homme très intelligent et scribe~; et Jehiel, fils de Hacmoni, était avec les fils du roi.
\VS{33}Achitophel était le conseiller du roi~; et Huschaï, l'Arkien, était l'intime ami du roi.
\VS{34}Après Achitophel était Jehojada, fils de Benaja et Abiathar~; et Joab était le chef de l'armée du roi.
\Chap{28}
\TextTitle{Dernières paroles de David, la royauté remise à Salomon\FTNTT{1 Ch. 23:2.}}
\VerseOne{}David convoqua à Jérusalem tous les chefs d'Israël, les chefs des tribus, et les chefs des divisions qui servaient le roi~; et les chefs de milliers et de centaines, et ceux qui avaient la charge de tous les biens du roi, et de tout ce qu'il possédait, ses fils avec ses eunuques, et les hommes puissants, et tous les héros et tous les hommes vaillants.
\VS{2}Puis le roi David se leva sur ses pieds, et dit~: Mes frères et mon peuple, écoutez-moi~! J'avais à cœur de bâtir une maison de repos pour l'arche de l'alliance de Yahweh, et pour le marchepied de notre Dieu, et j'ai fait les préparatifs pour la bâtir.
\VS{3}Mais Dieu m'a dit~: Tu ne bâtiras point de maison à mon Nom, parce que tu es un homme de guerre, et que tu as répandu beaucoup de sang.
\VS{4}Or comme Yahweh, le Dieu d'Israël, m'a choisi dans toute la maison de mon père pour être roi sur Israël à toujours~; car il a choisi Juda pour conducteur, et de la maison de Juda la maison de mon père, et d'entre les fils de mon père il a pris son plaisir en moi, pour me faire régner sur tout Israël.
\VS{5}Aussi, entre tous mes fils, car Yahweh m'a donné beaucoup de fils, il a choisi Salomon, mon fils, pour le faire asseoir sur le trône du royaume de Yahweh, sur Israël.
\VS{6}Et il m'a dit~: Salomon, ton fils, est celui qui bâtira ma maison et mes parvis~; car je me le suis choisi pour fils et je serai pour lui un père.
\VS{7}Et j'affermirai son règne à toujours s'il s'applique à pratiquer mes commandements et à observer mes ordonnances, comme il le fait aujourd'hui.
\VS{8}Maintenant donc, je vous somme en présence de tout Israël, qui est l'assemblée de Yahweh, et devant notre Dieu qui l'entend, que vous ayez à garder et à rechercher diligemment tous les commandements de Yahweh, votre Dieu, afin que vous possédiez ce bon pays, et que vous le fassiez hériter à vos fils après vous, à jamais.
\VS{9}Et toi, Salomon, mon fils, connais le Dieu de ton père, et sers-le avec un cœur droit et une bonne volonté~; car Yahweh sonde tous les cœurs et connaît toutes les dispositions des pensées. Si tu le cherches, il se laissera trouver par toi~; mais si tu l'abandonnes, il te rejettera pour toujours.
\VS{10}Considère maintenant que Yahweh t'a choisi pour bâtir une maison pour son sanctuaire. Fortifie-toi donc et applique-toi à y travailler.
\VS{11}David donna à Salomon, son fils, le modèle\FTNT{David donna le modèle du temple qu'il avait reçu de Dieu à Salomon. Beaucoup veulent servir Dieu sans modèle, tandis que d'autres vont chercher des modèles dans le monde (2 R. 16:10-18). Nous devons faire l'œuvre de Dieu uniquement selon le modèle biblique.} du portique, de ses maisons, des chambres du trésor, des chambres hautes, des chambres intérieures et du lieu du propitiatoire.
\VS{12}Il lui donna le modèle de toutes les choses qui lui avaient été inspirées par l'Esprit qui était avec lui, pour les parvis de la maison de Yahweh, pour les chambres d'alentour, pour les trésors de la maison de Yahweh et pour les trésors des choses saintes~;
\VS{13}pour les divisions des prêtres et des Lévites, pour toute l'œuvre du service de la maison de Yahweh, et pour tous les ustensiles du service de la maison de Yahweh.
\VS{14}Il lui donna aussi de l'or, un certain poids, pour les choses qui devaient être d'or, à savoir pour tous les ustensiles de chaque service~; et de l'argent, un certain poids, pour tous les ustensiles d'argent, à savoir pour tous les ustensiles de chaque service.
\VS{15}Le poids des chandeliers d'or, et de leurs lampes d'or, selon le poids de chaque chandelier et de ses lampes~; et le poids des chandeliers d'argent, selon le poids de chaque chandelier et de ses lampes, selon l'usage de chaque chandelier.
\VS{16}Et le poids de l'or pesant ce qu'il fallait pour chaque table des pains de proposition~; et de l'argent pour les tables d'argent.
\VS{17}Il lui donna le modèle pour les fourchettes, pour les bassins et pour les calices d'or pur~; le modèle pour les coupes d'or, selon le poids de chaque coupe, et de l'argent pour les coupes d'argent selon le poids de chaque coupe~;
\VS{18}et le modèle pour l'autel des parfums en or épuré, avec le poids. Il lui donna encore le modèle du char, des chérubins d'or qui étendent les ailes et qui couvrent l'arche de l'alliance de Yahweh.
\VS{19}Toutes ces choses, dit-il, m'ont été données par écrit, de la part de Yahweh, afin d'avoir l'intelligence de tous les ouvrages de ce modèle.
\TextTitle{David demande à Salomon de bâtir le temple}
\VS{20}C'est pourquoi David dit à Salomon, son fils~: Fortifie-toi, prends courage et travaille~; ne crains point et ne t'effraie point. Car Yahweh Dieu, mon Dieu, sera avec toi, il ne te délaissera point, et il ne t'abandonnera point, jusqu'à ce que tu aies achevé tout l'ouvrage du service de la maison de Yahweh.
\VS{21}Et voici, j'ai fait les divisions des prêtres et des Lévites pour tout le service de la maison de Dieu~; et il y a avec toi pour tout cet ouvrage toutes sortes de gens prompts et experts, pour toutes sortes de services~; et les chefs avec tout le peuple seront prêts pour exécuter tout ce que tu diras.
\Chap{29}
\TextTitle{Offrandes volontaires de David et de tout le peuple}
\VerseOne{}Puis le roi David dit à toute l'assemblée~: Dieu a choisi un seul de mes fils, à savoir Salomon, qui est encore jeune et délicat, et l'ouvrage est considérable, car ce palais n'est point pour un homme, mais pour Yahweh Dieu.
\VS{2}Et moi, j'ai préparé de toutes mes forces pour la maison de mon Dieu, de l'or pour les choses qui doivent être d'or, de l'argent pour celles qui doivent être d'argent, de l'airain pour celles d'airain, du fer pour celles de fer, du bois pour celles de bois, des pierres d'onyx, et des pierres pour être enchâssées, des pierres d'escarboucle, et des pierres de diverses couleurs, des pierres précieuses de toutes sortes, et du marbre en abondance.
\VS{3}Et outre cela, parce que j'ai une grande affection pour la maison de mon Dieu, je donne pour la maison de mon Dieu, outre toutes les choses que j'ai préparées pour la maison du sanctuaire, l'or et l'argent que j'ai entre mes plus précieux joyaux~:
\VS{4}Trois mille talents d'or, de l'or d'Ophir, et sept mille talents d'argent affiné, pour revêtir les murailles de la maison~;
\VS{5}afin qu'il y ait de l'or partout où il faut de l'or, et de l'argent partout où il faut de l'argent~; et pour tout l'ouvrage qui se fera par la main des ouvriers. Or qui est celui d'entre vous qui se disposera volontairement à offrir aujourd'hui libéralement à Yahweh~?
\VS{6}Alors les chefs des pères, les chefs des tribus d'Israël, les chefs de milliers et de centaines et les intendants du roi offrirent volontairement.
\VS{7}Ils donnèrent pour le service de la maison de Dieu cinq mille talents et dix mille drachmes d'or, dix mille talents d'argent, dix-huit mille talents d'airain, et cent mille talents de fer.
\VS{8}Ils mirent aussi les pierres que chacun avait, pour le trésor de la maison de Yahweh, entre les mains de Jehiel, le Guerschonite.
\VS{9}Et le peuple offrait avec joie volontairement, car ils offraient de tout leur cœur leurs offrandes volontaires à Yahweh~; et David en eut une très grande joie.
\TextTitle{Prière de David}
\VS{10}Puis David bénit Yahweh en présence de toute l'assemblée, et dit~: Ô Yahweh, Dieu d'Israël, notre père~! Tu es béni de tout temps et à toujours.
\VS{11}Ô Yahweh~! C'est à toi qu'appartiennent la magnificence, la puissance, la gloire, l'éternité, et la majesté~; car tout ce qui est aux cieux et sur la terre est à toi, ô Yahweh~! Le règne est à toi, et tu t'élèves en souverain au-dessus de toutes choses~!
\VS{12}Les richesses et les honneurs viennent de toi, et tu as la domination sur toutes choses~; la force et la puissance sont dans ta main, et il est aussi du pouvoir de ta main d'agrandir et de fortifier toutes choses.
\VS{13}Maintenant donc, ô notre Dieu~! Nous te célébrons et nous louons ton Nom glorieux.
\VS{14}Mais qui suis-je, et qui est mon peuple, que nous ayons assez pour pouvoir t'offrir ces choses volontairement~? Car toutes choses viennent de toi, et les ayant reçues de ta main, nous te les présentons.
\VS{15}Et même nous sommes devant toi des étrangers et des habitants, comme ont été tous nos pères~; et nos jours sont comme l'ombre sur la terre, et il n'y a point d'espérance.
\VS{16}Yahweh, notre Dieu, toute cette abondance que nous avons préparée pour bâtir une maison à ton saint Nom, est de ta main, et toutes ces choses sont à toi.
\VS{17}Et je sais, ô mon Dieu, que c'est toi qui sondes les cœurs, et que tu prends plaisir à la droiture. C'est pourquoi j'ai volontairement offert d'un cœur droit toutes ces choses, et j'ai vu maintenant avec joie que ton peuple, qui se trouve ici, t'a fait son offrande volontairement.
\VS{18}Ô Yahweh~! Dieu d'Abraham, d'Isaac et d'Israël, nos pères, conserve à toujours dans le cœur de ton peuple, ces dispositions et ces pensées, et affermis leurs cœurs en toi.
\VS{19}Donne aussi un cœur droit à Salomon, mon fils, afin qu'il garde tes commandements, tes préceptes et tes lois, et qu'il fasse tout ce qui est nécessaire et qu'il bâtisse le palais que j'ai préparé.
\TextTitle{Sacrifices en l'honneur de Yahweh~; Yahweh élève Salomon\FTNTT{1 Ch. 23:1~; 1 R. 2:12~; 1:32-37.}}
\VS{20}Après cela, David dit à toute l'assemblée~: Bénissez maintenant Yahweh, votre Dieu~! Et toute l'assemblée bénit Yahweh, le Dieu de leurs pères. Ils s'inclinèrent et se prosternèrent devant Yahweh et devant le roi.
\VS{21}Et le lendemain, ils offrirent des sacrifices à Yahweh, et des holocaustes~; à savoir mille veaux, mille moutons, et mille agneaux, avec leurs libations~; et des sacrifices en grand nombre pour tous ceux d'Israël.
\VS{22}Et ils mangèrent et burent ce jour-là devant Yahweh avec une grande joie~; et ils établirent roi pour la seconde fois Salomon, fils de David, et l'oignirent en l'honneur de Yahweh pour être leur conducteur, et Tsadok pour prêtre.
\VS{23}Salomon s'assit donc sur le trône de Yahweh pour être roi à la place de David, son père. Il prospéra, car tout Israël lui obéit.
\VS{24}Et tous les chefs et les héros, et même tous les fils du roi David consentirent d'être les sujets du roi Salomon.
\VS{25}Ainsi, Yahweh éleva souverainement Salomon, à la vue de tout Israël, et lui donna une majesté royale telle qu'aucun roi avant lui n'en avait eue en Israël.
\TextTitle{Fin du règne de David~; sa mort\FTNTT{2 S. 5:4-5~; 1 R. 2:10-12~; 1 Ch. 3:4.}}
\VS{26}David donc, fils d'Isaï, régna sur tout Israël.
\VS{27}Et les jours qu'il régna sur Israël furent quarante ans~; il régna sept ans à Hébron et trente-trois ans à Jérusalem.
\VS{28}Puis il mourut dans une heureuse vieillesse, rassasié de jours, de richesses, et de gloire. Et Salomon, son fils, régna à sa place.
\VS{29}Les actions du roi David, tant les premières que les dernières, sont écrites dans le livre de Samuel le voyant, dans le livre de Nathan le prophète, et dans le livre de Gad le prophète,
\VS{30}avec tout son règne, ses exploits et ce qui se passa de son temps, tant sur Israël que sur tous les royaumes du territoire.
\PPE{}
\end{multicols}
