\ShortTitle{Aggée}\BookTitle{Aggée}\BFont
\noindent\hrulefill
{\footnotesize
\textit{
\bigskip
{\centering{}
\\Auteur : Aggée
\\(Heb. : Haggaï)
\\Signification : En fête, né un jour de fête
\\Thème : Reconstruction du temple
\\Date de rédaction : 6ème siècle av. J.-C.\\}
}
%\bigskip
\textit{
\\Aggée, contemporain de Zacharie, exerça son ministère dans le royaume de Juda après le retour de l’exil. Alors que la reconstruction du temple était négligée, Aggée reçut un message rappelant au peuple quelles devraient être ses priorités et redéfinissant les exigences de Yahweh en matière de sainteté. Ce récit montre la bénédiction accompagnant celui qui oublie ses propres intérêts et qui prend véritablement à cœur l’œuvre de Dieu.\bigskip
}
}
\par\nobreak\noindent\hrulefill
\begin{multicols}{2}
\Chap{1}
\TextTitle{Israël coupable de négligence}
\VerseOne{}La seconde année du roi Darius, le premier jour du sixième mois, la parole de Yahweh fut adressée par le moyen d'Aggée, le prophète, à Zorobabel, fils de Schealthiel, gouverneur de Juda , et à Josué, fils de Jotsadak, le souverain sacrificateur, en ces mots\FTNT{Esd. 4:24} :
\VS{2}Ainsi parle Yahweh des armées, en ces mots : Ce peuple-ci a dit : Le temps n'est pas encore venu, le temps de rebâtir la maison de Yahweh.
\VS{3}C'est pourquoi la parole de Yahweh a été adressée par le moyen d'Aggée, le prophète, en ces mots :
\VS{4}Est-il temps pour vous d'habiter dans vos maisons lambrissées pendant que cette maison est désolée ?
\VS{5}Maintenant donc ainsi parle Yahweh des armées : Considérez attentivement votre conduite.
\VS{6}Vous avez semé beaucoup, mais vous avez recueilli peu ; vous mangez, mais non pas jusqu'à être rassasiés ; vous avez bu, mais vous n'avez pas eu de quoi boire abondamment ; vous avez été vêtus, mais non pas jusqu'à en être échauffés, et celui qui se loue, se loue pour mettre son salaire dans un sac percé\FTNT{Mi. 6:14-15}.
\VS{7}Ainsi parle Yahweh des armées : Dirigez vos cœurs vers vos voies.
\VS{8}Montez à la montagne, apportez du bois, et bâtissez cette maison ; et j'y prendrai mon plaisir et je serai glorifié, a dit Yahweh.
\VS{9}Vous comptiez sur beaucoup, et voici, il y a eu peu ; vous l'avez apporté à la maison et j'ai soufflé dessus. Pourquoi ? A cause de ma maison, dit Yahweh des armées, parce qu 'elle est en ruine pendant que vous vous empressez chacun pour sa maison.
\VS{10}A cause de cela, les cieux au-dessus de vous retiennent la rosée et la terre a retenu ses produits\FTNT{Lé. 26:19 ; Dé. 28:23}.
\VS{11}Et j'ai appelé la sécheresse sur la terre, et sur les montagnes, et sur le blé, et sur le moût, et sur l'huile, et sur tout ce que la terre produit, et sur les hommes et sur les bêtes, et sur tout le travail des mains\FTNT{Ps. 105:16 ; Am. 4:7}.
\TextTitle{Yahweh réveille son peuple}
\VS{12}Zorobabel donc, fils de Schealthiel, et Josué, fils de Jotsadak, le souverain sacrificateur, et tout le reste du peuple, entendirent la voix de Yahweh, leur Dieu, et les paroles d'Aggée, le prophète, ainsi que Yahweh, leur Dieu, l'avait envoyé ; et le peuple eut de la crainte devant Yahweh.
\VS{13}Et Aggée, messager de Yahweh, parla au peuple, suivant le message de Yahweh, en disant : Je suis avec vous, dit Yahweh.
\VS{14}Et Yahweh réveilla l'esprit de Zorobabel, fils de Schealthiel, gouverneur de Juda, et l'esprit de Josué, fils de Jotsadak, le souverain sacrificateur, et l'esprit de tout le reste du peuple ; et ils vinrent et travaillèrent à la maison de Yahweh, leur Dieu.
\VS{15}Le vingt-quatrième jour du sixième mois, de la seconde année du roi Darius.
\Chap{2}
\TextTitle{Encouragements à poursuivre la construction}
\VerseOne{}Le vingt et unième jour du septième mois, la parole de Yahweh fut adressée par le moyen d'Aggée, le prophète, en ces mots :
\VS{2}Parle maintenant à Zorobabel, fils de Schealthiel, gouverneur de Juda, et à Josué, fils de Jotsadak, le souverain sacrificateur, et au reste du peuple, et dis-leur :
\VS{3}Quel est parmi vous le survivant qui ait vu cette maison dans sa première gloire, et comment la voyez-vous maintenant ? N'est-elle pas comme un rien devant vos yeux, au prix de celle-là\FTNT{Esd. 3:12} ?
\VS{4}Maintenant Zorobabel, fortifie-toi ! dit Yahweh ;  toi aussi , Josué, fils de Jotsadak, souverain sacrificateur, fortifie toi ! Vous aussi, tout le peuple du pays, fortifiez-vous ! dit Yahweh. Et travaillez, car je suis avec vous, dit Yahweh des armées.
\VS{5}La parole de l'alliance que je traitai avec vous, quand vous sortîtes d'Egypte, et mon Esprit, demeurent au milieu de vous ; ne craignez point\FTNT{Za. 4:6}.
\VS{6}Car ainsi parle Yahweh des armées : Encore un peu de temps, et j'ébranlerai les cieux et la terre, la mer et le sec\FTNT{Hé. 12:26} ;
\VS{7}j'ébranlerai toutes les nations ; et les trésors de toutes les nations viendront, et je remplirai de gloire cette maison, dit Yahweh des armées.
\VS{8}L'argent est à moi, et l'or est à moi, dit Yahweh des armées.
\VS{9}La gloire de cette dernière maison sera plus grande que celle de la première, dit Yahweh des armées ; et je mettrai la paix en ce lieu, dit Yahweh des armées.
\TextTitle{Purification et sanctification du peuple}
\VS{10}Le vingt-quatrième jour du neuvième mois de la seconde année de Darius, la parole de Yahweh fut adressée par le moyen d'Aggée, le prophète, en disant :
\VS{11}Ainsi parle Yahweh des armées : Interroge maintenant les sacrificateurs sur la loi en ces mots :
\VS{12}Si quelqu'un porte de la chair consacrée dans le pan de son vêtement, et que ce vêtement touche du pain, ou un mets cuit, ou du vin, ou de l'huile, ou un aliment quelconque, cela devient-il sanctifié ? Et les sacrificateurs répondirent et dirent : Non.
\VS{13}Alors Aggée dit : Si celui qui est souillé pour un mort touche toutes ces choses-là, ne seront-elles pas souillées ? Et les sacrificateurs répondirent et dirent : Elles seront souillées\FTNT{Lé. 17:15 ; No. 19:22 ; Tit. 1:15}.
\VS{14}Alors Aggée répondit et dit : Tel est ce peuple et telle est cette nation devant ma face, dit Yahweh ; et telles sont toutes les œuvres de leurs mains ; même ce qu'ils offrent ici est souillé.
\VS{15}Considérez donc attentivement ce qui s'est passé depuis ce jour et en remontant avant qu'on ait mis pierre sur pierre au temple de Yahweh.
\VS{16}Avant cela, dis-je, quand on est venu à un monceau de blé ; au lieu de vingt mesures, il ne s'en est trouvé que dix ; et quand on est venu au pressoir, au lieu de puiser du pressoir cinquante mesures, il ne s'en est trouvé que vingt.
\VS{17}Je vous ai frappés de brûlure, de nielle, de grêle, dans tout le travail de vos mains ; et vous n'êtes point revenus à moi, dit Yahweh\FTNT{De. 28:22 ; 1 R. 8:37 ; 2 Ch. 6:28 ; Am. 4:9}.
\VS{18}Mettez maintenant ceci dans votre cœur ; depuis ce jour-ci et dans la suite ; depuis, dis-je, le vingt-quatrième jour du neuvième mois, depuis le jour où les fondements du temple de Yahweh ont été posés, considérez-le attentivement.
\VS{19}Ce que vous avez semé est-il encore retourné au grenier ? Même jusqu'à la vigne, au figuier, au grenadier, et à l'olivier, rien n'a rapporté ; mais depuis ce jour-ci, je donnerai la bénédiction.
\TextTitle{Destruction les royaumes des nations}
\VS{20}Et la parole de Yahweh fut adressée pour la seconde fois à Aggée, le vingt-quatrième jour du mois, en ces mots :
\VS{21}Parle à Zorobabel, gouverneur de Juda, et dis-lui : J'ébranlerai les cieux et la terre ;
\VS{22}je renverserai le trône des royaumes, je détruirai la force des royaumes des nations ! Je renverserai les chars et ceux qui les montent ; et les chevaux et ceux qui les montent seront abattus, chacun par l'épée de son frère.
\VS{23}En ce jour- là, dit Yahweh des armées, je te prendrai, ô Zorobabel, fils de Schealthiel, mon serviteur, dit Yahweh ; et je te mettrai comme un sceau\FTNT{Le sceau est un symbole d’autorité.}, car je t'ai choisi, dit Yahweh des armées.
\PPE{}
\end{multicols}
