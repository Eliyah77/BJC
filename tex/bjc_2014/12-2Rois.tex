\ShortTitle{2 Rois}\BookTitle{2 Rois}\BFont
\noindent\hrulefill
{\footnotesize
\textit{
\bigskip
{\centering{}
\\(Melakhim)
\\Signifie : Roi, Règne
\\Thème : Suite de l'histoire d'Israël et de Juda
\\Auteur : Inconnu
\\Date de rédaction : 6ème siècle av. J.-C.\\}
}
%\bigskip
\textit{
\\Suite chronologique du premier, le second livre des rois s’articule autour de la vie d’Elisée, successeur d’Elie. On y découvre le ministère prophétique d’Elisée au travers de qui Dieu se révéla comme le Tout Puissant, le Dieu compatissant, le Maître du temps et des circonstances, le Libérateur, le Dieu de la résurrection, le Puissant Guerrier et aussi le Juge.
%\bigskip
\\Ce livre relate l’histoire des derniers rois, la chute d’Israël et sa captivité puis la destruction de Jérusalem par Nebudcanetsar, roi de Babylone, en 586 avant J.-C.et la captivité de Juda.\bigskip
}
}
\par\nobreak\noindent\hrulefill
\begin{multicols}{2}
\Chap{1}
\TextTitle{Jugement de Yahweh sur Achazia, roi d'Israël}
\VerseOne{}Après la mort d’Achab, Moab se révolta contre Israël.
\VS{2}Or Achazia tomba par le treillis de sa chambre haute qui était à Samarie, et il en fut malade. Il envoya des messagers, et leur dit : Allez, consultez Baal-Zebub\FTNT{Baal-Zebub était une divinité des Philistins adorée à Ekron qui devint Béelzébul dans les Evangiles (Mt. 10:25).}, dieu d’Ekron, pour savoir si je guérirai de cette maladie.
\VS{3}Mais l’Ange de Yahweh dit à Elie\FTNT{Elie : voir 1 R. 17.}, le Thischbite : Lève-toi, monte à la rencontre des messagers du roi de Samarie, et dis-leur : N’y a-t-il point de Dieu en Israël pour que vous alliez consulter Baal-Zebub, dieu d’Ekron ?
\VS{4}C’est pourquoi ainsi parle Yahweh : Tu ne descendras pas du lit sur lequel tu es monté, mais certainement tu mourras. Et Elie s’en alla.
\VS{5}Les messagers retournèrent vers Achazia. Et il leur dit : Pourquoi revenez-vous ?
\VS{6}Ils lui répondirent : Un homme est monté à notre rencontre, et nous a dit : Allez, retournez vers le roi qui vous a envoyés, et dites-lui : Ainsi parle Yahweh : N’y a-t-il point de Dieu en Israël, pour que tu envoies consulter Baal-Zebub, dieu d’Ekron ? A cause de cela, tu ne descendras pas du lit sur lequel tu es monté, mais certainement tu mourras.
\VS{7}Achazia leur dit : Comment était cet homme qui est monté à votre rencontre et qui vous a dit ces paroles ?
\VS{8}Ils lui répondirent : C’était un homme vêtu de poil, ayant une ceinture de cuir, ceinte sur ses reins. Et Achazia dit : C’est Elie, le Thischbite.
\TextTitle{Affirmation de l'autorité d'Elie}
\VS{9}Alors il envoya vers lui un chef de cinquante avec ses cinquante hommes. Ce chef monta auprès d’Elie, qui demeurait au sommet d’une montagne, et il lui dit : Homme de Dieu, le roi a dit : Descends !
\VS{10}Mais Elie répondit, et dit au chef de cinquante : Si je suis un homme de Dieu, que le feu descende du ciel et te consume, toi et tes cinquante hommes ! Et le feu descendit du ciel et le consuma, lui et ses cinquante hommes.
\VS{11}Achazia envoya encore un autre chef de cinquante avec ses cinquante hommes. Ce chef prit la parole et dit à Elie : Homme de Dieu, ainsi parle le roi : Hâte-toi de descendre !
\VS{12}Mais Elie répondit, et leur dit : Si je suis un homme de Dieu, que le feu descende du ciel et te consume, toi et tes cinquante hommes ! Et le feu de Dieu descendit du ciel et le consuma, lui et ses cinquante hommes.
\VS{13}Achazia envoya encore un troisième chef de cinquante avec ses cinquante hommes. Ce troisième chef de cinquante hommes monta, et vint se mettre à genoux devant Elie, le suppliant, en disant : Homme de Dieu, je te prie, que ma vie et la vie de ces cinquante hommes, tes serviteurs, soit précieuse à tes yeux !
\VS{14}Voici, le feu est descendu du ciel, et a consumé les deux premiers chefs de cinquante, avec leurs cinquante hommes ; mais maintenant, je te prie, que ma vie soit précieuse à tes yeux !
\VS{15}Et l’Ange de Yahweh dit à Elie : Descends avec lui, n’aie pas peur de lui. Elie se leva donc, et descendit avec lui vers le roi.
\VS{16}Il lui dit : Ainsi parle Yahweh : Parce que tu as envoyé des messagers pour consulter Baal-Zebub, dieu d’Ekron, comme s’il n’y avait point de Dieu en Israël, pour consulter sa parole, tu ne descendras pas du lit sur lequel tu es monté, mais certainement tu mourras.
\TextTitle{Mort d'Achazia. Joram, roi d'Israël}
\VS{17}Achazia mourut, selon la parole de Yahweh prononcée par Elie. Et Joram régna à sa place, la seconde année de Joram, fils de Josaphat, roi de Juda, parce qu’Achazia n’avait point de fils.
\VS{18}Le reste des actions d’Achazia, et ce qu’il a fait, cela n’est-il pas écrit dans le livre des Chroniques des rois d’Israël ?
\Chap{2}
\TextTitle{Enlèvement d'Elie au ciel}
\VerseOne{}Lorsque Yahweh enleva Elie au ciel dans un tourbillon, Elie et Elisée partaient de Guilgal.
\VS{2}Elie dit à Elisée : Je te prie, reste ici, car Yahweh m’envoie jusqu’à Béthel. Mais Elisée répondit Yahweh est vivant et ton âme est vivante ! Je ne te quitterai pas. Ainsi ils descendirent à Béthel.
\VS{3}Les fils des prophètes qui étaient à Béthel sortirent vers Elisée, et lui dirent : Ne sais-tu pas qu’aujourd’hui Yahweh va enlever ton maître au-dessus de ta tête ? Et il répondit : Je le sais aussi ; taisez-vous.
\VS{4}Elie lui dit : Elisée, je te prie, reste ici, car Yahweh m’envoie à Jéricho. Mais Elisée lui répondit : Yahweh est vivant et ton âme est vivante ! Je ne te quitterai pas. Ainsi ils arrivèrent à Jéricho.
\VS{5}Les fils des prophètes qui étaient à Jéricho s’approchèrent d’Elisée, et lui dirent : Ne sais-tu pas qu’aujourd’hui Yahweh va enlever ton maître au-dessus de ta tête ? Et il répondit : Je le sais aussi ; taisez-vous.
\VS{6}Elie lui dit : Elisée, je te prie demeure ici, car Yahweh m’envoie jusqu’au Jourdain. Mais Elisée répondit : Yahweh est vivant et ton âme est vivante ! Je ne te quitterai pas. Ainsi ils s’en allèrent tous les deux.
\VS{7}Cinquante hommes d’entre les fils des prophètes arrivèrent et s’arrêtèrent à distance vis-à-vis d’eux, et eux deux s’arrêtèrent au bord du Jourdain.
\VS{8}Alors Elie prit son manteau, le roula, et en frappa les eaux, qui se divisèrent çà et là, et ils passèrent tous deux à sec.
\VS{9}Quand ils furent passés, Elie dit à Elisée : Demande ce que tu veux que je fasse pour toi, avant que je sois enlevé d’avec toi. Elisée répondit : Je te prie, que j’aie, une double portion\FTNT{Le fils aîné recevait une double portion par rapport aux autres fils (De. 21:15-17).} de ton esprit !
\VS{10}Elie lui dit : Tu demandes une chose difficile. Mais si tu me vois pendant que je serai enlevé d’avec toi, cela te sera accordé ; mais si tu ne me vois pas, cela ne te sera pas accordé.
\VS{11}Comme ils continuaient à marcher en parlant, voici, un char de feu et des chevaux de feu les séparèrent l’un de l’autre, et Elie monta au ciel dans un tourbillon.
\TextTitle{La double portion de l'esprit d'Elie sur Elisée}
\VS{12}Elisée le regardait et criait : Mon père ! Mon père ! Char d’Israël et sa cavalerie ! Et il ne le vit plus. Puis saisissant ses vêtements, il les déchira en deux morceaux.
\VS{13}Il releva le manteau qu’Elie avait laissé tomber. Puis il retourna, et s’arrêta sur le bord du Jourdain.
\VS{14}Il prit le manteau qu’Elie avait laissé tomber, et il en frappa les eaux, et dit : Où est Yahweh, le Dieu d’Elie, Yahweh lui-même ? Lui aussi frappa les eaux qui se divisèrent en deux ; et Elisée passa.
\TextTitle{Yahweh élève Elisée après l'enlèvement d'Elie}
\VS{15}Quand les fils des prophètes qui étaient à Jéricho, vis-à-vis, l’eurent vu, ils dirent : L’esprit d’Elie repose sur Elisée ! Ils vinrent à sa rencontre, et se prosternèrent contre terre devant lui.
\VS{16}Ils lui dirent : Voici, il y a parmi tes serviteurs cinquante hommes vaillants ; veux-tu qu’ils aillent chercher ton maître, de peur que l’Esprit de Yahweh ne l’ait enlevé, et ne l’ait jeté sur quelque montagne ou dans quelque vallée ? Elisée répondit : Ne les envoyez pas.
\VS{17}Mais ils le pressèrent tant par leurs paroles, qu’il en était embarrassé. Il leur dit donc : Envoyez-les. Ils envoyèrent cinquante hommes, qui pendant trois jours cherchèrent Elie, mais ils ne le trouvèrent point.
\VS{18}Puis ils retournèrent vers Elisée, qui était à Jéricho, et leur dit : Ne vous avais-je pas dit : N’y allez pas ?
\VS{19}Les gens de la ville dirent à Elisée : Voici, le séjour dans cette ville est bon, comme mon seigneur le voit ; mais les eaux sont mauvaises, et le pays est stérile.
\VS{20}Il dit : Apportez-moi un vase neuf, et mettez-y du sel. Et ils le lui apportèrent.
\VS{21}Puis Elisée alla vers la source des eaux, et il y jeta le sel, et dit : Ainsi parle Yahweh : J’assainis ces eaux ; elles ne causeront plus ni mort ni stérilité.
\VS{22}Les eaux furent assainies, jusqu’à ce jour, selon la parole qu’Elisée avait prononcée.
\TextTitle{Jugement des moqueurs}
\VS{23}Elisée monta de là à Béthel ; et comme il montait par le chemin, des petits garçons sortirent de la ville, et se moquèrent de lui. Ils lui disaient : Monte chauve ! Monte chauve !
\VS{24}Il se retourna pour les regarder, et il les maudit au nom de Yahweh. Alors deux ours sortirent de la forêt, et déchirèrent quarante-deux de ces enfants.
\VS{25}De là il alla sur la montagne de Carmel, d’où il retourna à Samarie.
\Chap{3}
\TextTitle{Règne de Joram sur Israël}
\VerseOne{}La dix-huitième année de Josaphat, roi de Juda, Joram, fils d’Achab, régna sur Israël à Samarie. Il régna douze ans.
\VS{2}Il fit ce qui est mal aux yeux de Yahweh, non pas toutefois comme son père et sa mère, car il ôta la statue de Baal que son père avait faite ;
\VS{3}mais il se livra aux péchés de Jéroboam, fils de Nebath, qui avait fait pécher Israël, et il ne s’en détourna point.
\TextTitle{Moab se révolte contre le roi d'Israël}
\VS{4}Or Méscha, roi de Moab, possédait des troupeaux, et il payait au roi d’Israël un tribut de cent mille agneaux et cent mille béliers avec leur laine.
\VS{5}Mais aussitôt qu’Achab mourut, le roi de Moab se révolta contre le roi d’Israël.
\VS{6}C’est pourquoi le roi Joram sortit ce jour-là de Samarie, et passa en revue tout Israël.
\VS{7}Il se mit en marche et fit dire à Josaphat, roi de Juda : Le roi de Moab s’est rebellé contre moi ; veux-tu venir avec moi faire la guerre à Moab ? Josaphat répondit : Je monterai, moi comme toi, mon peuple comme ton peuple, mes chevaux comme tes chevaux.
\VS{8}Ensuite il dit : Par quel chemin monterons-nous ? Joram répondit : Par le chemin du désert d’Edom.
\TextTitle{Position d'Elisée par rapport au roi d'Israël}
\VS{9}Ainsi, le roi d’Israël, le roi de Juda et le roi d’Edom, partirent ; ils firent un détour, et après une marche de sept jours, ils manquèrent d’eau pour l’armée et pour les bêtes qui la suivaient.
\VS{10}Alors le roi d’Israël dit : Hélas ! Yahweh a appelé ces trois rois pour les livrer entre les mains de Moab.
\VS{11}Et Josaphat dit : N’y a-t-il ici aucun prophète de Yahweh, par qui nous puissions consulter Yahweh ? Et un des serviteurs du roi d’Israël répondit, et dit : Il y a ici Elisée, fils de Schaphath, qui versait de l’eau sur les mains d’Elie.
\VS{12}Alors Josaphat dit : La parole de Yahweh est avec lui. Le roi d’Israël, Josaphat et le roi d’Edom descendirent vers lui.
\VS{13}Mais Elisée dit au roi d’Israël : Qu’y a-t-il entre moi et toi ? Va-t’en vers les prophètes de ton père et vers les prophètes de ta mère. Et le roi d’Israël lui répondit : Non ! Car Yahweh a appelé ces trois rois pour les livrer entre les mains de Moab.
\VS{14}Elisée dit : Yahweh des armées, devant lequel je me tiens, est vivant ! Si je n’avais de la considération pour Josaphat, roi de Juda, je ne ferais aucune attention à toi, et je ne te regarderais même pas.
\VS{15}Mais maintenant, amenez-moi un joueur d’instruments à cordes. Et comme le joueur jouait des instruments à cordes, la main de Yahweh fut sur Elisée.
\TextTitle{Prophétie sur la défaite de Moab}
\VS{16}Et il dit : Ainsi parle Yahweh : Faites des tranchées dans toute cette vallée.
\VS{17}Car ainsi parle Yahweh : Vous ne verrez ni vent, ni pluie, et néanmoins cette vallée sera remplie d’eaux, et vous boirez, vous et vos bêtes.
\VS{18}Mais cela est peu de chose aux yeux de Yahweh. Il livrera Moab entre vos mains ;
\VS{19}Vous frapperez toutes les villes fortes et toutes les villes d’élite, vous abattrez tous les bons arbres, vous boucherez toutes les sources d’eau, et vous ruinerez avec des pierres tous les meilleurs champs.
\VS{20}Il arriva donc au matin, environ à l’heure de l’offrande, que l’eau arriva du chemin d’Edom, en sorte que ce pays fut rempli d’eau.
\VS{21}Cependant, tous les Moabites ayant appris que ces rois étaient montés pour leur faire la guerre s’étaient assemblés. On convoqua tous ceux qui étaient en âge de porter les armes, et même au-dessus, et ils se tinrent sur la frontière.
\VS{22}Et le lendemain, ils se levèrent de bon matin, et comme le soleil se levait sur les eaux, les Moabites virent en face d’eux les eaux rouges comme du sang.
\VS{23}Ils dirent : C’est du sang ! Certainement, ces rois-là se sont entretués, et chacun a frappé son compagnon ; maintenant, Moabites, au butin !
\VS{24}Ainsi ils marchèrent contre le camp d’Israël. Mais Israël se leva, et frappa Moab, qui prit la fuite devant eux. Puis ils pénétrèrent dans le pays, et frappèrent Moab.
\VS{25}Ils détruisirent les villes, et chacun jetait des pierres dans les meilleurs champs, de sorte qu’ils les en remplirent, ils bouchèrent toutes les sources d’eaux, et abattirent tous les bons arbres ; et les frondeurs entourèrent et frappèrent Kir-Haréseth, dont on ne laissa que les pierres.
\VS{26}Le roi de Moab, voyant qu’il n’était pas le plus fort dans la bataille, prit avec lui sept cents hommes tirant l’épée pour se frayer un passage jusqu’au roi d’Edom ; mais ils ne purent pas.
\VS{27}Alors il prit son fils premier-né, qui devait régner à sa place, et l’offrit en holocauste sur la muraille. Et il y eut une grande indignation en Israël ; ainsi ils se retirèrent du roi de Moab, et retournèrent dans leur pays.
\Chap{4}
\TextTitle{Miracle avec l'huile de la veuve}
\VerseOne{}Une femme d’un des fils des prophètes cria à Elisée, en disant : Ton serviteur mon mari est mort, et tu sais que ton serviteur craignait Yahweh ; or son créancier est venu pour prendre mes deux enfants, afin qu’ils soient ses esclaves.
\VS{2}Elisée lui répondit : Que puis-je faire pour toi ? Dis-moi ce que tu as à la maison. Et elle dit : Ta servante n’a rien dans toute la maison qu’un vase d’huile.
\VS{3}Alors il lui dit : Va, demande des vases dans la rue à tous tes voisins, des vases vides, et n’en demande pas un petit nombre.
\VS{4}Puis rentre, et ferme la porte sur toi et sur tes enfants, et verse dans tous ces vases, et tu mettras de côté ceux qui seront pleins.
\VS{5}Alors elle le quitta. Ayant fermé la porte sur elle et sur ses enfants ; ils lui présentaient les vases, et elle versait.
\VS{6}Lorsqu’elle eut rempli les vases, elle dit à son fils : Présente-moi encore un vase. Mais il répondit : Il n’y a plus de vase. Et l’huile s’arrêta.
\VS{7}Elle alla le raconter à l’homme de Dieu, qui lui dit : Va, vends l’huile, et paye ta dette ; et vous vivrez, toi et tes fils, de ce qui restera.
\TextTitle{Yahweh se souvient de la Sunamite}
\VS{8}Et il arriva un jour qu’Elisée passait par Sunem, où il y avait une femme importante ; elle le retint avec grande instance à manger du pain chez elle. Et toutes les fois qu’il passait, il s’y retirait pour manger du pain.
\VS{9}Elle dit à son mari : Voilà, je sais que cet homme qui passe souvent chez nous est un saint homme de Dieu.
\VS{10}Faisons-lui, je te prie, une petite chambre haute avec des murs, et mettons-y pour lui un lit, une table, un siège, et un chandelier, afin que quand il viendra chez nous, il s’y retire.
\VS{11}Un jour, Elisée étant revenu à Sunem, il se retira dans cette chambre haute, et s’y coucha.
\VS{12}Puis il dit à Guéhazi, son serviteur : Appelle cette Sunamite. Guéhazi l’appela, et elle se présenta devant lui.
\VS{13}Et Elisée dit à Guéhazi : Dis maintenant à cette femme : Voici, tu nous as montré tout cet empressement ; que pourrait-on faire pour toi ? Faut-il parler pour toi au roi ou au chef de l’armée ? Elle répondit : J’habite au milieu de mon peuple.
\VS{14}Et il dit : Que faudrait-il faire pour elle ? Guéhazi répondit : Mais elle n’a point de fils, et son mari est vieux.
\VS{15}Et il dit : Appelle-la. Guéhazi l’appela, et elle se présenta à la porte.
\VS{16}Elisée lui dit : L’année prochaine, à cette même époque, tu embrasseras un fils. Elle répondit : Mon seigneur, homme de Dieu, ne trompe pas, ne trompe pas ta servante !
\VS{17}Cette femme devint enceinte, et enfanta un fils un an après, à la même époque, comme Elisée lui avait dit.
\TextTitle{Foi de la Sunamite, résurrection de son fils}
\VS{18}L’enfant grandit. Il sortit un jour pour aller trouver son père vers les moissonneurs.
\VS{19}Et il dit à son père : Ma tête ! Ma tête ! Et le père dit au serviteur : Porte-le à sa mère.
\VS{20}Il le porta donc et l’amena à sa mère. Et l’enfant resta sur les genoux de sa mère jusqu’à midi, puis il mourut.
\VS{21}Elle monta, et le coucha sur le lit de l’homme de Dieu ; et ayant fermé la porte sur lui, elle sortit.
\VS{22}Elle appela son mari, et dit : Je te prie envoie-moi un des serviteurs, et une ânesse ; j’irai chez l’homme de Dieu, et je reviendrai.
\VS{23}Et il dit : Pourquoi vas-tu vers lui aujourd’hui ? Ce n’est point la nouvelle lune ni le sabbat. Elle répondit : Tout va bien.
\VS{24}Elle fit donc seller l’ânesse, et dit à son serviteur : Conduis-moi et ne m’arrête pas en route sans que je te le dise.
\VS{25}Ainsi elle s’en alla, et se rendit vers l’homme de Dieu sur la montagne de Carmel. L’homme de Dieu, l’ayant aperçue, dit à Guéhazi son serviteur : Voilà la Sunamite !
\VS{26}Cours à sa rencontre et dis-lui : Te portes-tu bien ? Ton mari se porte-t-il bien ? L’enfant se porte-t-il bien ? Et elle répondit : Nous nous portons bien.
\VS{27}Dès qu’elle fut arrivée auprès de l’homme de Dieu sur la montagne, elle embrassa ses pieds. Guéhazi s’approcha pour la repousser, mais l’homme de Dieu lui dit : Laisse-la, car son âme est dans l’amertume, et Yahweh me l’a caché, et ne me l’a pas révélé.
\VS{28}Alors elle dit : Ai-je demandé un fils à mon seigneur ? N’ai-je pas dit : Ne me trompe pas ?
\VS{29}Et Elisée dit à Guéhazi : Ceins tes reins, prends mon bâton dans ta main, et pars. Si tu rencontres quelqu’un, ne le salue pas ; et si quelqu’un te salue, ne lui réponds pas. Tu mettras mon bâton sur le visage de l’enfant.
\VS{30}Mais la mère de l’enfant dit : Yahweh est vivant, et ton âme est vivante ! Je ne te quitterai point. Il se leva donc, et la suivit.
\VS{31}Or Guéhazi les avait devancés et il avait mis le bâton sur le visage de l’enfant ; mais il n’y eut ni voix ni signe d’attention. Guéhazi retourna à la rencontre d’Elisée, et l’en informa en disant : L’enfant ne s’est pas réveillé.
\VS{32}Lorsqu’Elisée entra dans la maison, l’enfant, mort, était couché sur son lit.
\VS{33}Il ferma la porte sur eux deux, et pria Yahweh.
\VS{34}Puis, il monta et se coucha sur l’enfant ; il mit sa bouche sur la bouche de l’enfant, ses yeux sur ses yeux, ses mains sur ses mains, et il s’étendit sur lui. La chair de l’enfant se réchauffa.
\VS{35}Elisée s’éloigna et marcha dans la maison, tantôt dans un lieu, tantôt dans un autre, puis il remonta et s’étendit encore sur lui. L’enfant éternua sept fois, et ouvrit ses yeux.
\VS{36}Alors Elisée appela Guéhazi, et lui dit : Appelle cette Sunamite. Guéhazi l’appela, et elle vint vers Elisée, qui lui dit : Prends ton fils !
\VS{37}Elle se jeta à ses pieds, et se prosterna contre terre. Puis elle prit son fils, et sortit.
\TextTitle{Les coloquintes sauvages}
\VS{38}Après cela Elisée revint à Guilgal. Or il y avait une famine\FTNT{ Par le passé, Israël a connu plusieurs famines, dont celle relatée en 2 R. 4:38-41. Dans ce passage, l’un des fils des prophètes trouva une vigne sauvage dans un champ et y cueillit des coloquintes sauvages. Il les ajouta au potage qui mijotait dans un pot, ne sachant pas que c’était du poison. Le pot est l’image des églises de Laodicée dans lesquelles il y a un mélange mortel de fausses doctrines et de préceptes mondains qui viennent altérer la vérité de la parole de Dieu. Ce mélange impur est absorbé par des millions de personnes ignorantes à travers le monde. Celles-ci se rendent compte qu’elles ont été empoisonnées spirituellement, une fois le mélange ingéré, elles constatent les effets pervers et dévastateurs souvent tardivement. Le champ tout comme la vigne sauvage, selon Mt. 13:38 et Ro. 11:17, symbolise le monde. Il est par ailleurs intéressant de noter que le mot herbe, « owrah » en hébreu, signifie aussi lumière (Ps. 139:12). Cette histoire n’est pas sans nous rappeler le feu étranger introduit par les fils d’Aaron dans le tabernacle, et ce, malgré l’interdiction formelle de Yahweh (Ex. 30:9 ; Lé. 10:1-5). C’est exactement ce qui se passe de nos jours. Les églises importent de plus en plus en leur sein la lumière luciférienne du monde (musique, marketing, philosophie...). Beaucoup de pasteurs et de musiciens cherchent malheureusement leur inspiration dans le monde à cause de la famine qui sévit dans les églises. Ce feu étranger représente la plupart des doctrines et pratiques promues par l’église de Laodicée.} dans le pays, et les fils des prophètes étaient assis devant lui ; et il dit à son serviteur : Mets le grand pot, et fais cuire du potage pour les fils des prophètes.
\VS{39}Mais quelqu’un étant sorti dans les champs pour cueillir des herbes, trouva de la vigne sauvage, et cueillit des coloquintes sauvages plein sa robe. Etant revenu, il les coupa en morceaux dans le pot où était le potage, car on ne savait pas ce que c’était.
\VS{40}Et on servit à manger de ce potage à quelques-uns ; mais aussitôt qu’ils eurent mangé de ce potage, ils s’écrièrent : Homme de Dieu, la mort est dans le pot ! Et ils ne purent en manger.
\VS{41}Elisée dit : Apportez-moi de la farine. Il en jeta dans le pot, et dit : Sers à ces gens, et qu’ils mangent. Et il n’y avait plus rien de mauvais dans le pot.
\TextTitle{Multiplication de pains}
\VS{42}Un homme venant de Baal-Schalischa apporta à l’homme de Dieu du pain des prémices, à savoir vingt pains d’orge, et des épis nouveaux. Elisée dit : Donne cela à ces gens, et qu’ils mangent.
\VS{43}Son serviteur répondit : Comment pourrais-je en donner à cent hommes ? Mais Elisée lui répondit : Donne-le à ces gens, et qu’ils mangent ; car ainsi parle Yahweh : Ils mangeront, et il en restera encore.
\VS{44}Il mit donc les pains devant eux. Ils mangèrent, et en eurent de reste, selon la parole de Yahweh.
\Chap{5}
\TextTitle{Guérison miraculeuse de Naaman}
\VerseOne{}Naaman\FTNT{La manière dont la guérison fut accordée à Naaman révèle la folie de Dieu et celle de l’œuvre de la croix. Les pensées de Dieu ne sont pas nos pensées, ses voies ne sont pas nos voies (Es. 55:8-10). Naaman devait se plonger sept fois dans le Jourdain, le chiffre sept étant celui de la perfection. Cet homme devait premièrement être délivré de l’orgueil. En effet, le mot « Jourdain » signifie « celui qui descend ». Naaman devait descendre de son trône. Dieu veut d’abord nous abaisser avant de nous élever (Jn. 12:24-25 ; Lu. 18:14). Ce général Syrien devait connaître également la mort par rapport à son titre, sa condition sociale, son origine, etc.  La condition de sa restauration était simple et précise. Aucun autre moyen ne convenait : Ni la main d’Elisée, ni d’autres fleuves, ni son argent. Rimmon, dieu des Syriens, ne pouvait pas le purifier non plus. Pour Naaman, les fleuves de Damas semblaient mieux qualifiés pour sa guérison, parce que leurs eaux sont claires et proviennent des montagnes enneigées, alors que l’eau du Jourdain, limpide à la sortie du lac de Galilée, ne tarde pas à se troubler et à prendre une couleur brun-sale qui provient de la nature de son lit. Les raisonnements de Naaman ont failli le faire passer à côté de sa purification. Cet homme ignorait que les voies de Dieu sont souvent une folie pour les hommes. La solution du Seigneur est toujours la bonne.}, chef de l’armée du roi de Syrie, était un homme puissant et très considéré aux yeux de son maître ; car c’était par lui que Yahweh avait délivré les Syriens. Mais cet homme fort et vaillant était lépreux.
\VS{2}Et les Syriens étaient sortis par troupes, et ils avaient emmené prisonnière une petite fille du pays d’Israël, qui était au service de la femme de Naaman.
\VS{3}Elle dit à sa maîtresse : Oh ! Si mon seigneur se présentait devant le prophète qui est à Samarie, il le guérirait de sa lèpre !
\VS{4}Naaman le rapporta à son maître, en disant : La fille qui est du pays d’Israël a dit telle et telle chose.
\VS{5}Et le roi de Syrie dit à Naaman : Va, rends-toi à Samarie, et j’enverrai une lettre au roi d’Israël. Naaman donc s’en alla, et prit avec lui dix talents d’argent, et six mille pièces d’or, et dix vêtements de rechange.
\VS{6}Il porta au roi d’Israël la lettre, où il était dit : Dès que cette lettre te sera parvenue, sache que je t’ai envoyé Naaman, mon serviteur, afin que tu le guérisses de sa lèpre.
\VS{7}Après avoir lu la lettre, le roi d’Israël déchira ses vêtements et dit : Suis-je Dieu pour faire mourir, et pour rendre la vie, pour qu’il s’adresse à moi afin que je guérisse un homme de sa lèpre ? Voyez et comprenez qu’il cherche certainement une occasion de dispute avec moi.
\VS{8}Lorsque Elisée, homme de Dieu, apprit que le roi d’Israël avait déchiré ses vêtements, il envoya dire au roi : Pourquoi as-tu déchiré tes vêtements ? Laisse-le venir vers moi, et il saura qu’il y a un prophète en Israël.
\VS{9}Naaman vint avec ses chevaux et son char, et il s’arrêta à la porte de la maison d’Elisée.
\VS{10}Elisée envoya un messager vers lui, pour lui dire : Va, et lave-toi sept fois dans le Jourdain, et ta chair redeviendra saine, et tu seras pur.
\VS{11}Mais Naaman se mit dans une grande colère, et s’en alla, en disant : Voilà, je me disais : Il sortira et viendra vers moi, il se présentera lui-même, il invoquera le nom de Yahweh, son Dieu, puis, il agitera sa main sur la plaie, et guérira le lépreux.
\VS{12}Les fleuves de Damas, l’Abana et le Parpar ne sont-ils pas meilleurs que toutes les eaux d’Israël ? Ne pourrais-je pas m’y laver et devenir pur ? Ainsi donc, il s’en retourna et s’en alla furieux.
\VS{13}Mais ses serviteurs s’approchèrent, et lui parlèrent, en disant : Mon père, si le prophète t’avait imposé quelque chose de difficile, ne l’aurais-tu pas fait ? Combien plus dois-tu faire ce qu’il t’a dit : Lave-toi, et tu deviendras pur !
\VS{14}Naaman descendit, et se plongea sept fois dans le Jourdain, selon la parole de l’homme de Dieu ; et sa chair redevint comme la chair d’un petit enfant ; et il fut pur.
\VS{15}Il retourna vers l’homme de Dieu, lui et toute sa suite, et il vint se présenter devant lui, et dit : Voici, maintenant je sais qu’il n’y a point d’autre Dieu sur toute la terre, si ce n’est en Israël. Maintenant donc, je te prie, accepte ce présent de ton serviteur.
\VS{16}Elisée répondit : Yahweh, devant lequel je me tiens, est vivant ! Je ne l’accepterai pas. Naaman le pressa fort de l’accepter, mais Elisée refusa.
\VS{17}Alors Naaman dit : Je te prie, permets que l’on donne de la terre à ton serviteur, une charge de deux mulets ; car ton serviteur ne fera plus d’holocauste ni de sacrifice à d’autres dieux, mais seulement à Yahweh.
\VS{18}Voici toutefois, que Yahweh pardonne ceci à ton serviteur. Quand mon maître entre dans la maison de Rimmon pour s’y prosterner et qu’il s’appuie sur ma main, je me prosterne aussi dans la maison de Rimmon : Que Yahweh me pardonne, quand je me prosternerai dans la maison de Rimmon.
\VS{19}Elisée lui dit : Va en paix. Lorsque Naaman eut quitté Elisée et qu’il fut à une certaine distance,
\TextTitle{Convoitise et mensonge de Guéhazi. Jugement de Dieu}
\VS{20}Guéhazi\FTNT{Guéhazi, dont le nom signifie « vallée de la vision », était un homme qui avait une vision terrestre du ministère. Sa vision n’était pas celle de la montagne (Royaume de Dieu), mais celle de la vallée (Royaume terrestre). Dans Es. 40:4, le Seigneur demanda aux juifs d’agrandir leur vallée, c’est- à-dire leur vision, car elle était étroite et terrestre. Guéhazi servait l’homme de Dieu et non Dieu. Il était préoccupé par les biens matériels : L’argent, les vêtements, les terres, les vignes, les brebis, les bœufs, les serviteurs et les servantes. Il aspirait à être un patron avec des domestiques à son service. Les présents de Naaman, que son maître Elisée avait refusés, car il avait conscience du fait que les présents aveuglent (De. 16:19), Guéhazi les a acceptés. Et pourtant, cet homme fréquentait le prophète le plus puissant de son temps, mais au lieu d’hériter de son onction, il hérita de la lèpre de Naaman. Or depuis Moïse, la lèpre était considérée comme la maladie la plus répugnante en Israël. Les lépreux étaient donc exclus de la présence de Dieu et du camp (Lé. 14). Guéhazi est l’archétype de tous les enfants dans la foi qui fréquentent les hommes et les femmes de Dieu intègres sans hériter de leur vie sanctifiée ni de leur message, à cause de leur manque de vision céleste. En effet, leur vision terrestre et « ventrale » les empêche de poursuivre le travail de leurs pères dans la foi.}, le serviteur d’Elisée, homme de Dieu, se dit en lui-même : Voici, mon maître a ménagé Naaman, ce Syrien, et n’a pas accepté de sa main ce qu’il avait apporté ; Yahweh est vivant ! Je vais courir après lui, et j’en obtiendrai quelque chose.
\VS{21}Et Guéhazi courut après Naaman. Naaman, le voyant courir après lui, descendit de son char pour aller à sa rencontre. Il dit : Tout va bien ?
\VS{22}Guéhazi répondit : Tout va bien. Mon maître m’envoie te dire : Voici, il vient d’arriver chez moi deux jeunes hommes de la montagne d’Ephraïm, d’entre les fils des prophètes. Je te prie donne-leur un talent d’argent et deux vêtements de rechange.
\VS{23}Et Naaman dit : Consens à prendre deux talents. Il insista, puis il serra deux talents d’argent dans deux sacs avec deux vêtements de rechange, et les fit porter devant Guéhazi par deux de ses serviteurs.
\VS{24}Arrivé à la colline, Guéhazi les prit de leurs mains, et les déposa dans la maison, et il renvoya ces gens qui s’en allèrent.
\VS{25}Puis il entra, et se présenta devant son maître. Elisée lui dit : D’où viens-tu, Guéhazi ? Et il répondit : Ton serviteur n’est allé nulle part.
\VS{26}Mais Elisée lui dit : Mon esprit n’est-il pas allé là, lorsque cet homme a quitté son char pour venir à ta rencontre ? Est-ce le temps de prendre de l’argent, de prendre des vêtements, des oliviers, des vignes, du menu et du gros bétail, des serviteurs et des servantes ?
\VS{27}C’est pourquoi la lèpre de Naaman s’attachera à toi et à ta postérité à jamais. Et Guéhazi sortit de la présence d’Elisée avec une lèpre comme de la neige.
\Chap{6}
\TextTitle{Miracle du fer de hache}
\VerseOne{}Les fils des prophètes dirent à Elisée : Voici, le lieu où nous sommes assis devant toi est trop étroit pour nous.
\VS{2}Allons jusqu’au Jourdain ; nous prendrons là chacun une poutre, et nous y ferons un lieu d’habitation. Elisée répondit : Allez.
\VS{3}Et l’un d’eux dit : Veuille, je te prie, venir avec tes serviteurs. Il répondit : J’irai.
\VS{4}Il partit donc avec eux. Arrivés au Jourdain, ils coupèrent du bois.
\VS{5}Mais il arriva que comme l’un d’eux abattait une poutre, le fer de sa cognée tomba dans l’eau. Il s’écria, et dit : Ah ! Mon seigneur ! Je l’avais emprunté !
\VS{6}L’homme de Dieu dit : Où est-il tombé ? Et il lui montra l’endroit. Alors Elisée coupa un morceau de bois, le jeta au même endroit, et fit surnager le fer.
\VS{7}Et il dit : Retire-le ! Et cet homme étendit sa main, et le prit.
\TextTitle{Plans militaires des Syriens dévoilés}
\VS{8}Le roi de Syrie était en guerre avec Israël, et, dans un conseil qu’il tint avec ses serviteurs, il dit : Mon camp sera dans un tel lieu.
\VS{9}L’homme de Dieu envoya dire au roi d’Israël : Garde-toi de passer dans ce lieu, car les Syriens y descendent.
\VS{10}Et le roi d’Israël envoya des gens, pour s’y tenir en observation, vers le lieu que l’homme de Dieu lui avait mentionné et signalé. Et il y était sur ses gardes. Et cela n’arriva pas seulement une fois ni deux fois.
\VS{11}Le roi de Syrie en eut le cœur troublé ; et il appela ses serviteurs, et leur dit : Ne voulez-vous pas me déclarer lequel de vous est pour le roi d’Israël ?
\VS{12}Et l’un de ses serviteurs répondit : Personne ! Ô roi, mon seigneur ! Mais Elisée, le prophète qui est en Israël, révèle au roi d’Israël les paroles même que tu déclares dans ta chambre à coucher.
\VS{13}Et il dit : Allez, et voyez où il est, et je le ferai prendre. On vint lui dire : Voici, il est à Dothan.
\VS{14}Il envoya là des chevaux, et des chars, et une grande armée, qui arrivèrent de nuit, et qui entourèrent la ville.
\TextTitle{L'armée de Yahweh veille sur Elisée}
\VS{15}Le serviteur de l’homme de Dieu se leva de grand matin, et sortit ; et voici, une armée, entourait la ville, avec des chevaux et des chars. Le serviteur dit à l’homme de Dieu : Ah ! Mon seigneur, comment ferons-nous ?
\VS{16}Il lui répondit : Ne crains point, car ceux qui sont avec nous sont en plus grand nombre que ceux qui sont avec eux.
\VS{17}Elisée pria, et dit : Je te prie, ô Yahweh ! Ouvre ses yeux, afin qu’il voie. Et Yahweh ouvrit les yeux du serviteur, et il vit : Et voici la montagne était pleine de chevaux, et de chars de feu autour d’Elisée.
\TextTitle{Les Syriens frappés d'aveuglement}
\VS{18}Les Syriens descendirent vers Elisée. Il adressa alors cette prière à Yahweh : Je te prie, frappe ces gens d’aveuglement ! Et Dieu les frappa d’aveuglement, selon la parole d’Elisée.
\VS{19}Elisée leur dit : Ce n’est pas ici le chemin, et ce n’est pas ici la ville ; suivez-moi, et je vous conduirai vers l’homme que vous cherchez. Et il les conduisit à Samarie.
\VS{20}Lorsqu’ils furent entrés dans Samarie, Elisée dit : Ô Yahweh ouvre leurs yeux afin qu’ils voient. Et Yahweh ouvrit leurs yeux, et ils virent qu’ils étaient au milieu de Samarie.
\VS{21}En les voyant, le roi d’Israël dit à Elisée : Frapperai-je, frapperai-je, mon père ?
\VS{22}Elisée répondit : Tu ne frapperas point ; frapperais-tu de ton épée et de ton arc ceux que tu as faits prisonniers ? Sers-leur du pain et de l’eau afin qu’ils mangent et boivent ; et après cela, qu’ils s’en aillent vers leur maître.
\VS{23}Le roi d’Israël leur fit servir un grand repas, et ils mangèrent et burent ; puis il les renvoya, et ils s’en allèrent vers leur maître. Alors, les armées de Syrie ne revinrent plus au pays d’Israël.
\TextTitle{Siège Syrien de Samarie}
\VS{24}Mais quelque temps après, Ben-Hadad, roi de Syrie, assembla toute son armée, monta, et assiégea Samarie.
\VS{25}Il y eut une grande famine\FTNT{Cette histoire est riche en enseignement pour notre génération. Le siège de la Samarie par les étrangers, la famine qui frappait les Hébreux, le cannibalisme de certaines femmes, la cherté des produits alimentaires, la consommation d’excréments d’animaux à cause de la famine, sont des conséquences du péché. Aujourd’hui, beaucoup d’églises sont assiégées par les choses du monde, les démons, les fausses doctrines, etc} dans Samarie ; ils l’assiégèrent tellement qu’une tête d’âne se vendait quatre-vingts pièces d’argent, et le quart d’un kab de fiente de pigeon, cinq pièces d’argent.
\VS{26}Et comme le roi d’Israël passait sur la muraille, une femme lui cria : Ô roi, mon seigneur ! Sauve-moi.
\VS{27}Il répondit : Si Yahweh ne te sauve pas, comment pourrais-je te sauver ? Serait-ce avec le produit de l’aire ou de la cuve ?
\VS{28}Il lui dit encore : Qu’as-tu ? Elle répondit : Cette femme-là m’a dit : Donne ton fils, et mangeons-le aujourd’hui, et nous mangerons mon fils demain\FTNT{Lé. 26:29 ; De. 28:53-57.}.
\VS{29}Ainsi nous avons fait bouillir mon fils, et l’avons mangé. Et le jour suivant, je lui ai dit : Donne ton fils, et nous le mangerons. Mais elle a caché son fils.
\VS{30}Dès que le roi entendit les paroles de cette femme, il déchira ses vêtements, et passa sur la muraille. Le peuple vit qu’il avait en dessous un sac sur son corps.
\VS{31}C’est pourquoi le roi dit : Que Dieu me traite dans toute sa rigueur, si aujourd’hui la tête d’Elisée, fils de Schaphath, reste sur lui.
\VS{32}Or Elisée était assis dans sa maison, et les anciens étaient assis avec lui. Le roi envoya un homme devant lui. Mais avant que le messager soit arrivé, Elisée dit aux anciens : Ne voyez-vous pas que le fils de ce meurtrier envoie quelqu’un pour m’ôter la tête ? Lorsque le messager viendra, fermez la porte et repoussez-le avec la porte. N’entendez-vous pas le bruit des pas de son maître derrière lui ?
\VS{33}Et comme il parlait encore avec eux, voici le messager descendit vers lui, et dit : Voici, ce mal vient de Yahweh ; qu’ai-je à espérer encore de Yahweh ?
\Chap{7}
\TextTitle{Yahweh nourrit abondamment Samarie grâce aux quatre lépreux}
\VerseOne{}Elisée dit : Ecoutez la parole de Yahweh ! Ainsi parle Yahweh : Demain, à cette heure, on aura une mesure de fleur de farine pour un sicle, et deux mesures d’orge pour un sicle, à la porte de Samarie.
\VS{2}L’officier sur la main duquel le roi s’appuyait répondit à l’homme de Dieu, et dit : Quand Yahweh ferait des fenêtres au ciel, cela arriverait-il ? Et Elisée dit : Tu le verras de tes yeux, mais tu n’en mangeras pas.
\VS{3}Or il y avait à l’entrée de la porte quatre hommes lépreux\FTNT{Dieu s’est servi de ces quatre lépreux comme messagers de bonnes nouvelles. Le Seigneur utilise souvent les personnes rejetées et déconsidérées (1 Co. 1:26-31).}, et ils se dirent l’un à l’autre : Pourquoi resterions-nous ici jusqu’à ce que nous mourions ?
\VS{4}Si nous pensons à entrer dans la ville, la famine est dans la ville, et nous y mourrons ; et si nous restons ici, nous mourrons également. Allons-nous jeter dans le camp des Syriens ; s’ils nous laissent vivre, nous vivrons, et s’ils nous font mourir, nous mourrons.
\VS{5}Ils se levèrent donc au crépuscule, pour entrer au camp des Syriens. Lorsqu’ils furent arrivés à l’extrémité du camp, voici, il n’y avait personne.
\VS{6}Le Seigneur avait fait entendre dans le camp des Syriens un bruit de chars, et un bruit de chevaux, et un bruit d’une grande armée ; de sorte qu’ils s’étaient dit l’un à l’autre : Voici, le roi d’Israël a payé les rois des Héthiens, et les rois des Egyptiens pour venir contre nous.
\VS{7}C’est pourquoi ils s’étaient levés au crépuscule, et s’étaient enfuis. Ils avaient abandonné leurs tentes, leurs chevaux, leurs ânes, et le camp tel qu’il était, et ils s’étaient enfuis pour sauver leur vie.
\VS{8}Les lépreux donc arrivèrent jusqu’à l’extrémité du camp. Ils entrèrent dans une tente, mangèrent, burent, emportèrent de l’argent, de l’or, des vêtements, et ils s’en allèrent, et les cachèrent. Ils revinrent et entrèrent dans une autre tente, et emportèrent de là aussi des objets, s’en allèrent, et les cachèrent.
\VS{9}Alors ils se dirent l’un à l’autre : Nous n’agissons pas bien ! Ce jour est un jour de bonnes nouvelles ; si nous gardons le silence, et si nous attendons jusqu’à lumière du matin, le châtiment nous atteindra. Venez maintenant, et allons informer la maison du roi.
\VS{10}Ils partirent, et appelèrent les portiers de la ville, et leur racontèrent, en disant : Nous sommes entrés dans le camp des Syriens, et voici, il n’y a personne. On n’y entend aucune voix d’homme ; il n’y a que des chevaux attachés, des ânes attachés, et les tentes sont comme elles étaient.
\VS{11}Alors les portiers crièrent, et transmirent ce rapport à la maison du roi.
\TextTitle{Accomplissement de la prophétie d'Elisée}
\VS{12}Le roi se leva de nuit, et dit à ses serviteurs : Je veux vous dire ce que les Syriens ont préparé contre nous. Ils savent que nous sommes affamés, et ils sont sortis du camp pour se cacher dans les champs, disant : Quand ils sortiront hors de la ville, nous les saisirons vivants, et nous entrerons dans la ville.
\VS{13}L’un des serviteurs du roi répondit : Qu’on prenne cinq des chevaux qui restent encore dans la ville ; c’est presque tout ce qui est resté du grand nombre des chevaux d’Israël ; ils sont comme toute la multitude d’Israël, qui est consumée. Envoyons voir ce qui se passe.
\VS{14}Ils prirent donc deux chars avec les chevaux, et le roi envoya des messagers après l’armée des Syriens, en disant : Allez, et voyez.
\VS{15}Et ils allèrent après eux jusqu’au Jourdain ; et voici, le chemin était plein de vêtements, et d’objets que les Syriens avaient jetés dans leur précipitation. Les messagers revinrent, et le rapportèrent au roi.
\VS{16}Alors le peuple sortit, et pilla le camp des Syriens, de sorte qu’il eut une mesure de fleur de farine pour un sicle, et deux mesures d’orge pour un sicle, selon la parole de Yahweh.
\VS{17}Le roi donna à l’officier, sur la main duquel il s’appuyait, la charge de garder la porte. Mais cet officier fut écrasé à la porte par le peuple, et il en mourut selon la parole qu’avait prononcée l’homme de Dieu, quand le roi était descendu vers lui.
\VS{18}Car lorsque l’homme de Dieu avait parlé au roi, en disant : Demain matin, à cette heure-ci, on donnera à la porte de Samarie deux mesures d’orge pour un sicle, et une mesure de fleur de farine pour un sicle ;
\VS{19}Cet officier avait répondu à l’homme de Dieu : Quand Yahweh ferait des fenêtres au ciel, ce que tu dis pourrait-il arriver ? Et l’homme de Dieu avait dit : Voici, tu le verras de tes yeux, mais tu n’en mangeras pas.
\VS{20}C’est en effet ce qui lui arriva ; car le peuple l’écrasa à la porte, et il mourut.
\Chap{8}
\TextTitle{Prophétie de sept ans de famine}
\VerseOne{}Elisée parla à la femme dont il avait fait revivre le fils, en disant : Lève-toi, et va-t’en, toi et ta famille, et séjourne où tu pourras ; car Yahweh a appelé la famine, et même elle vient sur le pays pour sept ans.
\VS{2}La femme se leva, et elle fit selon la parole de l’homme de Dieu. Elle s’en alla, elle et sa famille, et séjourna sept ans au pays des Philistins.
\TextTitle{La Sunamite retrouve ses terres}
\VS{3}Au bout des sept ans, la femme revint du pays des Philistins, et alla implorer le roi au sujet de sa maison et de ses champs.
\VS{4}Le roi parlait à Guéhazi\FTNT{Voir 2 R. 5.}, serviteur de l’homme de Dieu, en disant : Je te prie raconte-moi toutes les grandes choses qu’Elisée a faites.
\VS{5}Et pendant qu’il racontait au roi comment Elisée avait rendu la vie à un mort, la femme dont Elisée avait fait revivre le fils vint implorer le roi au sujet de sa maison et de ses champs. Guéhazi dit : Ô roi, mon seigneur, voici la femme, et voici son fils, à qui Elisée a rendu la vie.
\VS{6}Alors le roi interrogea la femme, et elle lui raconta ce qui s’était passé. Le roi lui donna un eunuque, auquel il dit : Fais restituer tout ce qui lui appartenait, même tous les revenus de ses champs, depuis le jour où elle a quitté le pays jusqu’à maintenant.
\TextTitle{Prophétie sur le règne d'Hazaël sur la Syrie}
\VS{7}Elisée se rendit à Damas. Ben-Hadad, roi de Syrie, était malade, et on lui fit ce rapport : L’homme de Dieu est venu ici.
\VS{8}Le roi dit à Hazaël : Prends avec toi un présent, et va au-devant de l’homme de Dieu, et consulte par lui Yahweh, en disant : Guérirai-je de cette maladie ?
\VS{9}Et Hazaël s’en alla au-devant d’Elisée, ayant pris avec lui un présent, à savoir quarante chameaux chargés de tout ce qu’il y avait de meilleur à Damas. Il vint se présenter devant Elisée, et dit : Ton fils, Ben-Hadad, roi de Syrie, m’a envoyé vers toi, pour te dire : Guérirai-je de cette maladie ?
\VS{10}Et Elisée lui répondit : Va, dis-lui : Certainement, tu guériras. Toutefois, Yahweh m’a révélé qu’il mourra certainement.
\VS{11}L’homme de Dieu arrêta son regard sur Hazaël, et le fixa longtemps, puis il pleura.
\VS{12}Hazaël dit : Pourquoi mon seigneur pleure-t-il ? Et il répondit : Parce que je sais le mal que tu feras aux enfants d’Israël ; tu mettras le feu à leurs villes fortes, tu tueras avec l’épée leurs jeunes gens, tu écraseras leurs petits-enfants, et tu fendras le ventre de leurs femmes enceintes
\VS{13}Hazaël dit : Mais qu’est-ce que ton serviteur, ce chien, pour faire de si grandes choses ? Et Elisée répondit : Yahweh m’a révélé que tu seras roi de Syrie.
\VS{14}Alors Hazaël quitta Elisée, et revint vers son maître, qui lui demanda : Que t’a dit Elisée ? Et il répondit : Il m’a dit que certainement tu guériras !
\VS{15}Mais le lendemain, Hazaël prit une couverture, et l’ayant plongé dans l’eau, il l’étendit sur le visage de Ben-Hadad, qui mourut. Et Hazaël régna à sa place.
\TextTitle{Joram, fils de Josaphat, roi de Juda\FTNTT{2 Ch. 21:1-7}}
\VS{16}La cinquième année de Joram, fils d’Achab, roi d’Israël, Josaphat était encore roi de Juda, Joram, fils de Josaphat, roi de Juda, commença à régner sur Juda.
\VS{17}Il était âgé de trente-deux ans lorsqu’il commença à régner. Il régna huit ans à Jérusalem.
\VS{18}Il marcha dans la voie des rois d’Israël comme avait fait la maison d’Achab, car il avait pour femme la fille d’Achab\FTNT{Le mariage de Joram, fils de Josaphat, avec Athalie, fille d’Achab, était une grande erreur. Cette union qui était contractée dans le but de favoriser la paix entre les deux royaumes entraina le déclin de Juda. Dieu est contre les alliances contre nature. Faisons attention aux associations et alliances avec des personnes que Dieu n’approuve pas. Voir Es. 30 et 31.}, et il fit ce qui est mal aux yeux de Yahweh.
\VS{19}Mais Yahweh ne voulut point détruire Juda, par amour pour David, son serviteur, selon la promesse qu’il lui avait faite de lui donner toujours une lampe parmi ses fils.
\TextTitle{Révoltes contre l'autorité de Juda}
\VS{20}De son temps, Edom se révolta contre l’autorité de Juda, et se donna un roi.
\VS{21}Joram passa à Tsaïr, avec tous ses chars ; il se leva de nuit, et frappa les Edomites qui l’entouraient, et les chefs des chars, mais le peuple s’enfuit dans ses tentes.
\VS{22}Néanmoins, les Edomites ont été rebelles à Juda jusqu’à ce jour. En ce même temps, Libna aussi se révolta.
\TextTitle{Achazia, roi de Juda\FTNTT{2 Ch. 21:18-22:4}}
\VS{23}Le reste des actions de Joram, et tout ce qu’il a fait, cela n’est-il pas écrit dans le livre des Chroniques des rois de Juda ?
\VS{24}Joram se coucha avec ses pères, et il fut enterré avec ses pères dans la cité de David. Et Achazia, son fils, régna à sa place.
\VS{25}La douzième année de Joram, fils d’Achab, roi d’Israël, Achazia, fils de Joram, roi de Juda, commença à régner.
\VS{26}Achazia était âgé de vingt-deux ans lorsqu’il commença à régner. Il régna un an à Jérusalem. Sa mère s’appelait Athalie, fille d’Omri, roi d’Israël.
\VS{27}Il marcha dans la voie de la maison d’Achab, et il fit ce qui est mal aux yeux de Yahweh, comme avait fait la maison d’Achab, car il était gendre de la maison d’Achab.
\VS{28}Il alla avec Joram, fils d’Achab, à la guerre contre Hazaël, roi de Syrie, à Ramoth en Galaad. Et les Syriens blessèrent Joram.
\VS{29}Le roi Joram s’en retourna pour se faire guérir à Jizreel des blessures que les Syriens lui avaient faites à Rama, lorsqu’il se battait contre Hazaël, roi de Syrie. Achazia, fils de Joram, roi de Juda, descendit pour voir Joram, fils d’Achab, à Jizreel, parce qu’il était malade.
\Chap{9}
\TextTitle{Jéhu oint roi d'Israël}
\VerseOne{}Alors Elisée, le prophète, appela l’un des fils des prophètes, et lui dit : Ceins tes reins, prends cette fiole d’huile dans ta main, et va à Ramoth en Galaad.
\VS{2}Quand tu y seras entré, vois Jéhu, fils de Josaphat, fils de Nimschi. Tu iras le faire lever du milieu de ses frères, et tu le conduiras dans une chambre secrète.
\VS{3}Tu prendras la fiole d’huile, tu la verseras sur sa tête, et tu diras : Ainsi parle Yahweh : Je t’ai oint pour être roi sur Israël. Après quoi tu ouvriras la porte, tu t’enfuiras, et tu ne t’arrêteras pas.
\VS{4}Le jeune homme, serviteur du prophète, s’en alla à Ramoth en Galaad.
\VS{5}Quand il arriva, voici, les chefs de l’armée étaient là assis. Il dit : Chef, j’ai à te parler. Et Jéhu répondit : Auquel de nous parles-tu ? Et il répondit : A toi, chef.
\VS{6}Alors Jéhu se leva, et entra dans la maison, et le jeune homme répandit l’huile sur la tête, et lui dit : Ainsi parle Yahweh, le Dieu d’Israël : Je t’ai oint pour être roi sur Israël, le peuple de Yahweh.
\VS{7}Tu frapperas la maison d’Achab, ton maître, et je vengerai sur Jézabel\FTNT{1 R. 16:31 ; 1 R. 17, 18, 19.} le sang de mes serviteurs les prophètes, et le sang de tous les serviteurs de Yahweh.
\VS{8}Toute la maison d’Achab périra ; j’exterminerai quiconque appartient à Achab, celui qui est esclave et celui qui est libre en Israël.
\VS{9}Je rendrai la maison d’Achab semblable à la maison de Jéroboam, fils de Nebath, et à la maison de Baescha, fils d’Achija.
\VS{10}Les chiens mangeront Jézabel dans le champ de Jizreel, et il n’y aura personne pour l’enterrer. Puis il ouvrit la porte, et s’enfuit.
\VS{11}Jéhu sortit pour rejoindre les serviteurs de son maître, et on lui dit : Tout va bien ? Pourquoi ce fou est-il venu vers toi ? Jéhu leur répondit : Vous connaissez l’homme, et ses rêveries.
\VS{12}Mais ils répliquèrent : Mensonge ! Réponds-nous donc. Et il dit : Il m’a parlé de telle et telle manière, disant : Ainsi parle Yahweh, je t’ai oint pour être roi sur Israël.
\VS{13}Alors ils se hâtèrent, et prirent chacun leurs vêtements, et les mirent sous lui au plus haut des degrés. Ils sonnèrent du shofar et dirent : Jéhu a été fait roi !
\TextTitle{Mort de Joram par Jéhu}
\VS{14}Ainsi Jéhu, fils de Josaphat, fils de Nimschi, forma une conspiration contre Joram. Or Joram et tout Israël défendaient Ramoth en Galaad contre Hazaël, roi de Syrie.
\VS{15}Le roi Joram s’en était retourné pour se faire guérir à Jizreel des blessures que les Syriens lui avaient faites, lorsqu’il se battait contre Hazaël, roi de Syrie. Jéhu dit : Si vous le trouvez bon, que personne ne sorte ni ne s’échappe de la ville pour aller porter cette nouvelle à Jizreel.
\VS{16}Alors Jéhu monta à cheval, et s’en alla à Jizreel, car Joram était là malade, et Achazia, roi de Juda, y était descendu pour le visiter.
\VS{17}Or il y avait une sentinelle sur une tour à Jizreel, qui voyant venir la troupe de Jéhu dit : Je vois une troupe de gens. Et Joram dit : Prends un cavalier, et envoie-le à leur rencontre, et qu’il dise : Est-ce la paix ?
\VS{18}Le cavalier s’en alla à sa rencontre, et dit : Ainsi parle le roi : Est-ce la paix ? Et Jéhu répondit : Qu’as-tu à faire de la paix ? Mets-toi derrière moi. La sentinelle le rapporta, en disant : Le messager est allé jusqu’à eux, et il ne revient pas.
\VS{19}Joram envoya un second cavalier, qui arriva jusqu’à eux, et dit : Ainsi parle le roi : Est-ce la paix ? Et Jéhu répondit : Qu’as-tu à faire de la paix ? Mets-toi derrière moi.
\VS{20}La sentinelle le rapporta, et dit : Il est arrivé jusqu’à eux, et il ne revient pas ; mais la manière de conduire le char est comme celle de Jéhu, fils de Nimschi ; car il le conduit avec furie.
\VS{21}Alors Joram dit : Attelle ! Et on attela son char. Ainsi Joram, roi d’Israël, sortit avec Achazia, roi de Juda, chacun dans son char, et ils allèrent à la rencontre de Jéhu, et ils le trouvèrent dans le champ de Naboth de Jizreel\FTNT{1 R. 21.}.
\VS{22}Dès que Joram vit Jéhu, il dit : Est-ce la paix, Jéhu ? Jéhu répondit : Quelle paix ! Tant que durent les prostitutions de Jézabel, ta mère, et la multitude de ses enchantements !
\VS{23}Alors Joram tourna sa main, et s’enfuit, et il dit à Achazia : Trahison, Achazia !
\VS{24}Mais Jéhu saisit l’arc de sa main, et il frappa Joram entre ses épaules, de sorte que la flèche transperça son cœur, et il tomba sur ses genoux dans son char.
\VS{25}Jéhu dit à Bidkar, son officier : Prends-le, et jette-le dans le champ de Naboth de Jizreel ; car souviens-toi, lorsque nous étions à cheval moi et toi, ensemble, derrière Achab, son père, Yahweh prononça cette sentence contre lui :
\VS{26}J’ai vu hier le sang de Naboth et le sang de ses fils, dit Yahweh, et je te rendrai la pareille dans ce champ même, dit Yahweh ! C’est pourquoi prends-le donc, et jette-le dans ce champ, selon la parole de Yahweh.
\TextTitle{Mort d'Achazia par Jéhu\FTNTT{2 Ch. 22:7,9}}
\VS{27}Achazia, roi de Juda, ayant vu cela, s’enfuit par le chemin de la maison du jardin ; mais Jéhu le poursuivit, et dit : Frappez-le sur le char ! Et on le frappa à la montée de Gur, près de Jibleam. Puis il se réfugia à Meguiddo, et il y mourut.
\VS{28}Ses serviteurs le transportèrent sur un char à Jérusalem, et ils l’enterrèrent dans son sépulcre avec ses pères, dans la cité de David.
\VS{29}Achazia avait commencé à régner sur Juda la onzième année de Joram, fils d’Achab.
\TextTitle{Mort de Jézabel par Jéhu}
\VS{30}Jéhu entra dans Jizreel. Jézabel, l’ayant appris, mit du fard à ses yeux, orna sa tête, et regarda par la fenêtre.
\VS{31}Comme Jéhu franchissait la porte, elle dit : Est-ce la paix, Zimri, assassin de son maître ?
\VS{32}Il leva sa tête vers la fenêtre, et dit : Qui est avec moi ? Qui ? Alors deux ou trois des eunuques regardèrent vers lui.
\VS{33}Et il leur dit : Jetez-la en bas ! Et ils la jetèrent, de sorte qu’il rejaillit de son sang sur la muraille et sur les chevaux. Jéhu la foula aux pieds ;
\VS{34}Puis il entra, mangea, et but, et il dit : Allez voir maintenant cette maudite, et enterrez-la, car elle est fille de roi.
\VS{35}Ils allèrent donc pour l’enterrer ; mais ils ne trouvèrent d’elle que le crâne, les pieds, et les paumes des mains.
\VS{36}Ils retournèrent l’annoncer à Jéhu, qui dit : C’est la parole que Yahweh avait déclarée par son serviteur Elie\FTNT{1 R. 21:23.}, le Thischbite, en disant : Dans le champ de Jizreel les chiens mangeront la chair de Jézabel ;
\VS{37}et le cadavre de Jézabel sera comme du fumier sur la face des champs, dans le champ de Jizreel, de sorte qu’on ne pourra dire : C’est Jézabel.
\Chap{10}
\TextTitle{Jugement de la maison d'Achab}
\VerseOne{}Achab avait soixante-dix fils dans Samarie. Jéhu écrivit des lettres qu’il envoya à Samarie aux chefs de Jizreel, aux anciens, et aux gouverneurs d’Achab. Il y était dit :
\VS{2}Dès que cette lettre vous sera parvenue, puisque vous avez avec vous les fils de votre maître, avec vous les chars et les chevaux, la ville forte et les armes,
\VS{3}choisissez qui est le plus considérable et le plus sincère parmi les fils de votre maître, mettez-le sur le trône de son père, et combattez pour la maison de votre maître.
\VS{4}Ils eurent une très grande peur, et ils dirent : Voici, deux rois n’ont point pu tenir contre lui, comment donc résisterions-nous ?
\VS{5}Et le chef de la maison, le chef de la ville, les anciens, et les gouverneurs envoyèrent dire à Jéhu : Nous sommes tes serviteurs, nous ferons tout ce que tu nous diras ; nous n’établirons personne roi, fais ce qui te semblera bon.
\VS{6}Jéhu leur écrivit une seconde lettre, où il était dit : Si vous êtes pour moi, et si vous obéissez à ma voix, prenez les têtes des fils de votre maître, et venez auprès de moi demain à cette heure-ci, à Jizreel. Or les soixante-dix hommes, fils du roi, étaient avec les plus grands de la ville qui les élevaient.
\VS{7}Aussitôt que la lettre leur fut parvenue, ils prirent les fils du roi, et ils égorgèrent ces soixante-dix hommes ; et ayant mis leurs têtes dans des corbeilles, ils les envoyèrent à Jéhu, à Jizreel.
\VS{8}Un messager vint l’en informer, en disant : Ils ont apporté les têtes des fils du roi. Et il répondit : Mettez-les en deux tas à l’entrée de la porte, jusqu’au matin.
\VS{9}Le matin, il sortit ; et se présentant à tout le peuple, il dit : Vous êtes justes ! Voici, j’ai conspiré contre mon maître, et je l’ai tué ; mais qui a frappé tous ceux-ci ?
\VS{10}Sachez maintenant qu’il ne tombera rien à terre de la parole de Yahweh\FTNT{1 R. 21:19-24.}, de la parole que Yahweh a prononcée contre la maison d’Achab ; Yahweh accomplit ce qu’il avait déclaré par son serviteur Elie.
\VS{11}Jéhu tua aussi tous ceux qui restaient de la maison d’Achab à Jizreel, tous ses grands, ses familiers et ses sacrificateurs, sans en laisser échapper un seul.
\TextTitle{Jéhu détruit la lignée d'Achab\FTNTT{2 Ch. 22:8}}
\VS{12}Puis il se leva, et partit pour aller à Samarie. Et comme il était près d’une maison de bergers sur le chemin,
\VS{13}Jéhu trouva les frères d’Achazia, roi de Juda, et leur dit : Qui êtes-vous ? Ils répondirent : Nous sommes les frères d’Achazia, et nous sommes descendus pour saluer les fils du roi, et les fils de la reine.
\VS{14}Jéhu dit : Saisissez-les vivants. Ils les saisirent vivants, et les égorgèrent, à savoir quarante-deux hommes, auprès du puits de la maison des bergers, sans en laisser échapper un seul.
\VS{15}Jéhu étant parti de là, il rencontra Jonadab, fils de Récab, qui venait au-devant de lui. Il le salua, et lui dit : Ton cœur est-il aussi droit envers moi comme mon cœur l’est à ton égard ? Et Jonadab répondit : Il l’est. Donne-moi ta main répliqua Jéhu. Et Jonadab lui donna sa main, et Jéhu le fit monter auprès de lui dans son char.
\VS{16}Puis il dit : Viens avec moi, et tu verras le zèle que j’ai pour Yahweh. Il l’emmena ainsi dans son char.
\VS{17}Et quand Jéhu fut arrivé à Samarie, il tua tous ceux qui restaient de la maison d’Achab à Samarie, et il les extermina entièrement, selon la parole que Yahweh avait dite à Elie.
\TextTitle{Jéhu extermine les prophètes de Baal}
\VS{18}Puis Jéhu assembla tout le peuple, et leur dit : Achab a peu servi Baal\FTNT{Jg. 2:13.}, mais Jéhu le servira beaucoup.
\VS{19}Maintenant donc, convoquez-moi tous les prophètes de Baal, tous ses serviteurs, et tous ses sacrificateurs, sans qu’il en manque un seul, car je veux offrir un grand sacrifice à Baal : Quiconque manquera ne vivra pas. Jéhu agissait avec ruse, pour faire périr les serviteurs de Baal.
\VS{20}Jéhu dit : Publiez une fête solennelle en l’honneur de Baal. Et ils la publièrent.
\VS{21}Jéhu envoya des messagers dans tout Israël ; et tous les serviteurs de Baal arrivèrent, il n’y en eut pas un qui ne vienne ; et ils entrèrent dans le temple de Baal, qui fut rempli d’un bout à l’autre.
\VS{22}Alors Jéhu dit à celui qui avait la charge du vestiaire : Sors des vêtements pour tous les serviteurs de Baal. Et cet homme sortit des vêtements.
\VS{23}Alors Jéhu, et Jonadab, fils de Récab, entrèrent dans le temple de Baal, et Jéhu dit aux serviteurs de Baal : Cherchez et regardez afin qu’il n’y ait pas ici de serviteurs de Yahweh. Prenez garde qu’il n’y ait seulement que les serviteurs de Baal.
\VS{24}Ils entrèrent donc pour offrir des sacrifices et des holocaustes. Or Jéhu avait placé dehors quatre-vingts hommes, et leur avait dit : Celui qui laissera échapper un de ces hommes que je remets entre vos mains, sa vie répondra de la sienne.
\VS{25}Lorsqu’on eut achevé d’offrir les holocaustes, Jéhu dit aux gardes et aux officiers : Entrez, tuez-les, et que nul n’échappe. Les gardes et les officiers les frappèrent du tranchant de l’épée, et les jetèrent là, puis ils allèrent jusqu’à la ville du temple de Baal.
\VS{26}Ils tirèrent dehors les statues de la maison de Baal, et les brûlèrent.
\VS{27}Et ils démolirent la statue de Baal. Ils démolirent aussi la maison de Baal, et ils en firent un cloaque, qui subsiste jusqu’à ce jour.
\VS{28}Ainsi Jéhu extermina Baal d’Israël.
\TextTitle{L'idolâtrie dans la vie de Jéhu}
\VS{29}Toutefois, Jéhu ne se détourna point des péchés que Jéroboam, fils de Nebath, avait fait commettre à Israël, à savoir les veaux d’or\FTNT{1 R. 12:28-29.} qui étaient à Béthel, et à Dan.
\VS{30}Yahweh dit à Jéhu : Parce que tu as fort bien exécuté ce qui était droit à mes yeux, et que tu as fait à la maison d’Achab tout ce qui était conforme à ma volonté, tes fils seront assis sur le trône d’Israël jusqu’à la quatrième génération.
\VS{31}Mais Jéhu ne prit point garde à marcher de tout son cœur dans la loi de Yahweh, le Dieu d’Israël ; il ne se détourna point des péchés que Jéroboam avait fait commettre à Israël.
\TextTitle{Hazaël, roi de Syrie}
\VS{32}Dans ce temps-là, Yahweh commença à entamer le territoire d’Israël, et Hazaël battit les Israélites sur toutes les frontières.
\VS{33}Depuis le Jourdain, jusqu’au soleil levant, il battit tout le pays de Galaad, les Gadites, les Rubénites et ceux de Manassé, depuis Aroër sur le torrent de l’Arnon, jusqu’à Galaad et à Basan.
\TextTitle{Joachaz, roi d'Israël}
\VS{34}Le reste des actions de Jéhu, tout ce qu’il a fait, et tous ses exploits, ne sont-ils pas écrits dans le livre des Chroniques des rois d’Israël ?
\VS{35}Jéhu se coucha avec ses pères, et on l’enterra à Samarie. Et Joachaz, son fils, régna à sa place.
\VS{36}Jéhu avait régné vingt-huit ans sur Israël à Samarie.
\Chap{11}
\TextTitle{Athalie fait périr la race royale de Juda\FTNTT{2 Ch. 22:9-12}}
\VerseOne{}Athalie, mère d’Achazia, ayant vu que son fils était mort, se leva, et extermina toute la race royale.
\VS{2}Mais Joschéba, fille du roi Joram, sœur d’Achazia, prit Joas, fils d’Achazia, et l’enleva du milieu des fils du roi, quand on les fit mourir : Elle le mit avec sa nourrice dans la chambre aux lits. Il fut ainsi dérobé aux regards d’Athalie, de sorte qu’on ne le fit point mourir.
\VS{3}Il resta caché six ans avec Joschéba dans la maison de Yahweh. Cependant Athalie régnait sur le pays.
\TextTitle{Joas, roi de Juda\FTNTT{2 Ch. 23:1-11}}
\VS{4}La septième année, Jehojada envoya chercher les chefs de centaines des Kéréthiens et des archers, et il les fit venir auprès de lui dans la maison de Yahweh. Il traita alliance avec eux, les fit jurer dans la maison de Yahweh, et leur montra le fils du roi.
\VS{5}Puis il leur donna cet ordre, en disant : Voici ce que vous ferez. Parmi ceux d’entre vous qui entrent en service le jour du sabbat, un tiers doit monter la garde à la maison du roi,
\VS{6}un tiers sera à la porte de Sur, et un tiers à la porte derrière les archers ; ainsi vous veillerez à la garde de la maison, afin que personne n’y entre par force.
\VS{7}Vos deux autres compagnies, tous ceux qui sortent de service le jour du sabbat feront la garde de la maison de Yahweh, auprès du roi :
\VS{8}et vous entourerez le roi de toutes parts, chacun ayant ses armes à la main, et l’on mettra à mort quiconque s’avancera dans les rangs ; vous serez avec le roi quand il sortira, et quand il entrera.
\VS{9}Les chefs de centaines firent donc tout ce que Jehojada, le sacrificateur, avait ordonné. Ils prirent chacun leurs gens, ceux qui entraient en service et ceux qui sortaient de service le jour du sabbat, et ils se rendirent vers le sacrificateur Jehojada.
\VS{10}Le sacrificateur donna aux chefs de centaine les lances et les boucliers qui provenaient du roi David, et qui étaient dans la maison de Yahweh.
\VS{11}Les archers, chacun les armes à la main, entourèrent le roi, en se plaçant depuis le côté droit de la maison, jusqu’au côté gauche, près de l’autel et près de la maison.
\VS{12}Jehojada fit amener le fils du roi, et il mit sur lui la couronne\FTNT{Couronne ou consacrer.} et le témoignage. Ils l’établirent roi et l’oignirent, et frappant des mains, ils dirent : Vive le roi !
\TextTitle{Mort d'Athalie\FTNTT{2 Ch. 23:12-15,21}}
\VS{13}Athalie entendit le bruit des archers et du peuple, et elle vint vers le peuple à la maison de Yahweh.
\VS{14}Elle regarda. Et voici, le roi se tenait sur l’estrade, selon la coutume des rois. Les chefs et les trompettes étaient près du roi : Tout le peuple du pays éclatait de joie, et on sonnait des trompettes. Alors Athalie déchira ses vêtements, et cria : Conspiration ! Conspiration !
\VS{15}Alors le sacrificateur Jehojada donna cet ordre aux chefs de centaines, qui avaient la charge de l’armée : Faites-la sortir hors des rangs, et que celui qui la suivra soit mis à mort par l’épée. Car le sacrificateur avait dit : Qu’elle ne soit pas mise à mort dans la maison de Yahweh !
\VS{16}Ils lui firent donc place, et elle retourna dans la maison du roi par le chemin de l’entrée des chevaux : c’est là qu’elle fut tuée.
\TextTitle{Le réveil sous le règne de Jehojada, le peuple revient à Yahweh\FTNTT{2 Ch. 23:16-21}}
\VS{17}Jehojada traita entre Yahweh, le roi, et le peuple l’alliance par laquelle ils devaient être le peuple de Yahweh ; il traita aussi l’alliance entre le roi et le peuple.
\VS{18}Alors tout le peuple du pays entra dans la maison de Baal, et ils la démolirent avec ses autels ; et ils brisèrent entièrement ses images ; ils tuèrent aussi Matthan, prêtre de Baal, devant les autels. Le sacrificateur Jehojada établit des gardes dans la maison de Yahweh.
\VS{19}Il prit les chefs de centaines, les Kéréthiens et les archers, et tout le peuple du pays ; et ils firent descendre le roi de la maison de Yahweh, et ils entrèrent dans la maison du roi par le chemin de la porte des archers, et Joas s’assit sur le trône des rois.
\VS{20}Tout le peuple du pays fut dans la joie, et la ville fut en repos, après qu’on eût mis à mort Athalie par l’épée dans la maison du roi.
\VS{21}Joas était âgé de sept ans lorsqu’il commença à régner.
\Chap{12}
\TextTitle{Règne de Joas\FTNTT{2 Ch. 24:2}}
\VerseOne{}La septième année de Jéhu, Joas commença à régner. Il régna quarante ans à Jérusalem. Sa mère s’appelait Tsibja, elle était de Beer-Schéba.
\VS{2}Joas fit ce qui est droit aux yeux de Yahweh pendant tout le temps qu’il suivit les instructions de Jehojada, le sacrificateur.
\VS{3}Toutefois, les hauts lieux ne disparurent point ; le peuple offrait encore des sacrifices et des parfums sur les hauts lieux.
\VS{4}Joas dit aux sacrificateurs : Tout l’argent consacré qu’on apporte dans la maison de Yahweh, l’argent ayant cours, à savoir l’argent pour l’évaluation des personnes d’après l’estimation qui en est faite, et tout l’argent que chacun apporte volontairement à la maison de Yahweh,
\VS{5}que les sacrificateurs le prennent, chacun de la part des gens de sa connaissance, et qu’ils l’emploient à réparer ce qui est à réparer dans la maison, partout où l’on trouvera quelque chose à réparer.
\VS{6}Mais il arriva que, la vingt-troisième année du roi Joas, les sacrificateurs n’avaient point encore réparé les brèches de la maison.
\VS{7}Le roi Joas appela le sacrificateur Jehojada, et les autres sacrificateurs, et il leur dit : Pourquoi n’avez-vous pas réparé ce qui était à réparer à la maison ? Maintenant, vous ne prendrez plus l’argent de vos connaissances, mais vous le livrerez pour les réparations de la maison.
\VS{8}Les sacrificateurs convinrent de ne plus prendre l’argent du peuple, et de ne pas être chargés des réparations de la maison.
\TextTitle{Offrandes volontaires pour réparer le temple\FTNTT{2 Ch. 24:8-14}}
\VS{9}Alors le sacrificateur Jehojada prit un coffre, et le perça dans son couvercle, et le plaça à côté de l’autel, à droite, à l’endroit par lequel on entrait à la maison de Yahweh. Les sacrificateurs qui avaient la garde du seuil y mettaient tout l’argent qu’on apportait à la maison de Yahweh.
\VS{10}Et dès qu’ils voyaient qu’il y avait beaucoup d’argent dans le coffre, le secrétaire du roi montait avec le souverain sacrificateur, et ils mettaient dans des sacs l’argent qui se trouvait dans la maison de Yahweh, puis ils le comptaient.
\VS{11}Ils remettaient cet argent bien compté entre les mains de ceux qui étaient chargés de faire exécuter l’ouvrage dans la maison de Yahweh. Et l’on employait cet argent pour les charpentiers et pour les architectes qui travaillaient à la maison de Yahweh,
\VS{12}pour les maçons et les tailleurs de pierres, pour acheter du bois et des pierres de taille, afin de réparer les brèches de la maison de Yahweh, et pour acheter tout ce qu’il fallait pour la réparation de la maison.
\VS{13}Mais, avec l’argent qu’on apportait dans la maison de Yahweh, on ne fit pour la maison de Yahweh ni bassins d’argent ni de couteaux, ni coupes, ni trompettes, ni aucun autre ustensile d’or, ou ustensile d’argent ;
\VS{14}Mais on le distribuait à ceux qui avaient la charge de l’ouvrage, et qui réparaient la maison de Yahweh.
\VS{15}On ne demandait pas de comptes aux hommes entre les mains desquels on remettait l’argent pour qu’ils le donnent à ceux qui faisaient l’ouvrage, car ils le faisaient fidèlement.
\VS{16}L’argent des sacrifices de culpabilité et des sacrifices d’expiation pour les péchés n’était point apporté dans la maison de Yahweh : Il était pour les sacrificateurs.
\TextTitle{Jérusalem préservé de l'invasion}
\VS{17}Alors Hazaël\FTNT{Hazaël envahit Juda à deux reprises. Ce passage fait mention de la première invasion ; la deuxième invasion est relatée en 2 Ch. 24:23.}, roi de Syrie, monta, et fit la guerre à Gath, dont il s’empara. Hazaël avait l’intention de monter contre Jérusalem.
\VS{18}Mais Joas, roi de Juda, prit tout ce qui était consacré, que Josaphat, Joram, et Achazia, ses pères, rois de Juda, avaient consacré, tout ce que lui-même avait consacré, tout l’or qui se trouva dans les trésors de la maison de Yahweh et de la maison du roi ; et il envoya le tout à Hazaël, roi de Syrie, qui ne monta pas contre Jérusalem.
\VS{19}Le reste des actions de Joas, tout ce qu’il a fait, cela n’est-il pas écrit dans le livre des Chroniques des rois de Juda ?
\VS{20}Ses serviteurs se soulevèrent et se liguèrent ; ils frappèrent Joas dans la maison de Millo, qui est à la descente de Silla.
\VS{21}Jozacar, fils de Schimeath, et Jozabad fils de Schomer, ses serviteurs, le frappèrent, et il mourut. On l’enterra avec ses pères dans la cité de David. Et Amatsia, son fils, régna à sa place.
\Chap{13}
\TextTitle{Règne de Joachaz sur Israël}
\VerseOne{}La vingt-troisième année de Joas, fils d’Achazia, roi de Juda, Joachaz, fils de Jéhu, commença à régner sur Israël à Samarie. Il régna dix-sept ans.
\VS{2}Il fit ce qui est mal aux yeux de Yahweh ; car il suivit les péchés de Jéroboam, fils de Nebath, par lesquels il avait fait pécher Israël, et il ne s’en détourna point.
\TextTitle{L'idolâtrie perdure dans le pays}
\VS{3}La colère de Yahweh s’enflamma contre Israël, et il les livra entre les mains de Hazaël, roi de Syrie, et entre les mains de Ben-Hadad, fils de Hazaël, tout le temps que ces rois vécurent.
\VS{4}Mais Joachaz implora Yahweh. Et Yahweh l’exauça, parce qu’il vit l’oppression sous laquelle le roi de Syrie tenait Israël,
\VS{5}Yahweh donna donc un libérateur à Israël, et ils échappèrent aux mains des Syriens, ainsi les enfants d’Israël habitèrent dans leurs tentes comme auparavant.
\VS{6}Mais ils ne se détournèrent point des péchés de la maison de Jéroboam, par lesquels il avait fait pécher Israël ; ils s’y livrèrent, et même l’idole d’Astarté\FTNT{Voir commentaire Jg. 2:13.} resta debout à Samarie.
\VS{7}De tout le peuple de Joachaz, Dieu ne lui avait laissé que cinquante cavaliers, dix chars, et dix mille hommes de pied ; car le roi de Syrie les avait fait périr et les avait rendus semblables à la poussière qu’on foule aux pieds.
\TextTitle{Joas règne sur Israël}
\VS{8}Le reste des actions de Joachaz, tout ce qu’il a fait, et ses exploits, cela n’est-il pas écrit dans le livre des Chroniques des rois d’Israël ?
\VS{9}Ainsi Joachaz se coucha avec ses pères, et on l’ensevelit à Samarie. Et Joas, son fils, régna à sa place.
\VS{10}La trente-septième année de Joas, roi de Juda, Joas, fils de Joachaz, commença à régner sur Israël à Samarie. Il régna seize ans.
\VS{11}Et il fit ce qui est mal aux yeux de Yahweh ; il ne se détourna d’aucun des péchés de Jéroboam, fils de Nebath, par lesquels il avait fait pécher Israël, il s’y livra comme lui.
\TextTitle{Mort de Joas}
\VS{12}Le reste des actions de Joas, tout ce qu’il a fait, ses exploits, et la guerre qu’il eut avec Amatsia, roi de Juda, tout cela n’est-il pas écrit dans le livre des Chroniques des rois d’Israël ?
\VS{13}Joas se coucha avec ses pères, et Jéroboam s’assit sur son trône. Joas fut enterré à Samarie avec les rois d’Israël.
\TextTitle{Persévérance dans l'obeissance}
\VS{14}Elisée était atteint de la maladie dont il mourut ; et Joas, roi d’Israël, descendit vers lui, pleura sur son visage, en disant : Mon père ! Mon père ! Char d’Israël et sa cavalerie !
\VS{15}Elisée lui dit : Prends un arc et des flèches. Il prit donc un arc et des flèches.
\VS{16}Puis Elisée dit au roi d’Israël : Bande l’arc avec ta main. Mets ta main sur l’arc. Et quand il y eut mis sa main, Elisée mit ses mains sur les mains du roi,
\VS{17}et il lui dit : Ouvre la fenêtre à l’orient. Et il l’ouvrit. Elisée lui dit : Tire. Après qu’il eut tiré, il lui dit : C’est la flèche de la délivrance de la part de Yahweh, la flèche de la délivrance contre les Syriens ; tu frapperas les Syriens à Aphek, jusqu’à leur extermination.
\VS{18}Elisée lui dit encore : Prends les flèches. Et il les prit. Elisée dit au roi d’Israël : Frappe contre terre. Et le roi frappa trois fois, puis il s’arrêta.
\VS{19}Et l’homme de Dieu se mit dans une très grande colère contre lui, et lui dit : Il fallait frapper cinq ou six fois ; alors tu aurais battu les Syriens jusqu’à leur extermination ; mais maintenant tu ne les frapperas que trois fois.
\TextTitle{Mort d'Elisée, ses os apportent la résurrection}
\VS{20}Elisée mourut, et on l’ensevelit. L’année suivante quelques troupes de Moabites entrèrent dans le pays.
\VS{21}Et comme on enterrait un homme, voici, on aperçut l’une des troupes de soldats, et l’on jeta l’homme dans le sépulcre d’Elisée. L’homme alla toucher les os d’Elisée, il reprit vie et se leva sur ses pieds.
\VS{22}Pendant toute la vie de Joachaz, Hazaël, roi de Syrie, avait opprimé Israël.
\VS{23}Mais Yahweh eut compassion d’eux, leur fit miséricorde, il tourna sa face vers eux par amour pour son alliance avec Abraham, Isaac, et Jacob, de sorte qu’il ne voulut point les exterminer, et il ne les rejeta pas de sa face, jusqu’à maintenant.
\VS{24}Puis Hazaël, roi de Syrie, mourut, et Ben-Hadad, son fils, régna à sa place.
\VS{25}Joas, fils de Joachaz, reprit des mains de Ben-Hadad, fils d’Hazaël, les villes enlevées par Hazaël, à Joachaz, son père, pendant la guerre. Joas le battit trois fois, et recouvra les villes d’Israël.
\Chap{14}
\TextTitle{Règne d'Amatsia sur Juda\FTNTT{2 Ch. 25:1-4}}
\VerseOne{}La deuxième année de Joas, fils de Joachaz, roi d’Israël, Amatsia, fils de Joas, roi de Juda, commença à régner.
\VS{2}Il était âgé de vingt-cinq ans lorsqu’il commença à régner, et il régna vingt-neuf ans à Jérusalem. Sa mère s’appelait Joaddan, elle était de Jérusalem.
\VS{3}Il fit ce qui est droit aux yeux de Yahweh, non pas toutefois comme David, son père ; il agit entièrement comme avait agi Joas, son père.
\VS{4}Seulement, les hauts lieux ne furent point ôtés ; le peuple offrait encore des sacrifices et des parfums sur les hauts lieux.
\VS{5}Lorsque le royaume fut affermi entre ses mains, il frappa ses serviteurs qui avaient tué le roi, son père.
\VS{6}Mais il ne fit point mourir les fils des meurtriers, suivant ce qui est écrit dans le livre de la loi de Moïse, où Yahweh donne ce commandement : On ne fera point mourir les pères pour les enfants, et l’on ne fera pas mourir les enfants pour les pères ; mais on fera mourir chacun pour son péché\FTNT{De. 24:16 ; Ez. 18:4, 20.}.
\VS{7}Il frappa dix mille hommes d’Edom dans la vallée du sel ; et il prit Séla durant la guerre, et l’appela Joktheel, nom qu’elle a conservé jusqu’à ce jour.
\VS{8}Alors Amatsia envoya des messagers vers Joas, fils de Joachaz, fils de Jéhu, roi d’Israël, pour lui dire : Viens, voyons-nous en face !
\VS{9}Et Joas, roi d’Israël, envoya dire à Amatsia, roi de Juda : L’épine du Liban envoya dire au cèdre du Liban : Donne ta fille en mariage à mon fils ! Et les bêtes sauvages qui sont au Liban passèrent et foulèrent l’épine.
\VS{10}Parce que tu as frappé et ravagé Edom, ton cœur s’est élevé. Contente-toi de ta gloire, et reste dans ta maison. Pourquoi exciterais-tu le mal par lequel tu tomberas, toi et Juda avec toi ?
\VS{11}Mais Amatsia ne l’écouta pas. Et Joas, roi d’Israël, monta : et ils s’affrontèrent, lui et Amatsia, roi de Juda, à Beth-Schémesch, qui est à Juda.
\VS{12}Juda fut battu par Israël, et ils s’enfuirent chacun dans leurs tentes.
\VS{13}Joas, roi d’Israël, prit Amatsia, roi de Juda, fils de Joas, fils d’Achazia, à Beth-Schémesch. Puis il vint à Jérusalem, et fit une brèche de quatre cents coudées dans la muraille de Jérusalem, depuis la porte d’Ephraïm, jusqu’à la porte de l’angle.
\VS{14}Il prit tout l’or et tout l’argent et tous les vases qui se trouvaient dans la maison de Yahweh, et dans les trésors de la maison royale ; il prit aussi des enfants en otages, et il retourna à Samarie.
\TextTitle{Règne de Jéroboam II sur Israël}
\VS{15}Le reste des actions de Joas, ses exploits, et comment il combattit contre Amatsia, tout cela n’est-il pas écrit dans le livre des Chroniques des rois d’Israël ?
\VS{16}Et Joas se coucha avec ses pères, et fut enseveli à Samarie avec les rois d’Israël. Et Jéroboam, son fils, régna à sa place.
\TextTitle{Mort d'Amatsia (2 Ch. 25:26-28); Azaria (Ozias), roi de Juda}
\VS{17}Amatsia, fils de Joas, roi de Juda, vécut quinze ans après la mort de Joas, fils de Joachaz, roi d’Israël.
\VS{18}Le reste des actions d’Amatsia n’est-il pas écrit dans le livre des Chroniques des rois de Juda ?
\VS{19}On forma une conspiration contre lui à Jérusalem, et il s’enfuit à Lakis ; mais on le poursuivit à Lakis, où on le fit mourir.
\VS{20}On le transporta sur des chevaux, et il fut enseveli à Jérusalem avec ses pères, dans la cité de David.
\VS{21}Alors tout le peuple de Juda prit Azaria, âgé de seize ans, et ils l’établirent roi à la place d’Amatsia, son père.
\VS{22}Azaria bâtit Elath et la fit rentrer sous la puissance de Juda, après que le roi se coucha avec ses pères.
\TextTitle{Prophétie de Jonas accomplie par Jéroboam II}
\VS{23}La quinzième année d’Amatsia, fils de Joas, roi de Juda, Jéroboam, fils de Joas, commença à régner sur Israël à Samarie, et il régna quarante et un ans.
\VS{24}Il fit ce qui est mal aux yeux de Yahweh, et ne se détourna d’aucun des péchés de Jéroboam, fils de Nebath, par lesquels il avait fait pécher Israël.
\VS{25}Il rétablit les frontières d’Israël depuis l’entrée de Hamath, jusqu’à la mer de la plaine, selon la parole de Yahweh, le Dieu d’Israël, qu’il avait prononcée par son serviteur Jonas\FTNT{Jon. 1:1.}, fils d’Amitthaï, le prophète, de Gath-Hépher.
\VS{26}Car Yahweh vit que l’affliction d’Israël était à son comble, et l’extrémité à laquelle se trouvaient réduits esclaves et homme libres, sans qu’il n’y ait personne pour venir au secours d’Israël.
\VS{27}Or Yahweh n’avait point résolu d’effacer le nom d’Israël de dessous les cieux, à cause de cela il les délivra par les mains de Jéroboam, fils de Joas.
\TextTitle{Zacharie, roi d'Israël}
\VS{28}Le reste des actions de Jéroboam, tout ce qu’il a fait, ses exploits de guerre, et comment il reconquit pour Israël Damas et Hamath qui avaient appartenu à Juda, cela n’est-il pas écrit dans le livre des Chroniques des rois d’Israël ?
\VS{29}Puis Jéroboam se coucha avec ses pères, avec les rois d’Israël. Et Zacharie, son fils, régna à sa place.
\Chap{15}
\TextTitle{Azaria (Ozias), roi de Juda. Le pays encore dans l'idolâtrie\FTNTT{2 R. 14:21-22 ; 2 Ch. 26:1-15}}
\VerseOne{}La vingt-septième année de Jéroboam, roi d’Israël, Azaria\FTNT{Azaria (Ozias, selon 2 Ch. 26:1-15 ; à ne pas confondre avec le prophète du même nom que son grand-père avait fait assassiner) fut couronné à l’âge de seize ans et mourut à l’âge de soixante-huit ans. Ce fut donc un long règne de cinquante-deux ans (2 R. 15:1-2  ; 2 Ch. 26:3), qui marqua plusieurs générations entre 790 et 730 av. J.-C. Selon Amos et Zacharie, cette époque fut ponctuée par un séisme de forte magnitude (Am. 1:1-2  ; Za. 14:5). Comme pour plusieurs de ses prédécesseurs sur le trône de Juda, le règne d’Ozias s’ouvre sous la clarté de l’approbation divine. Il fait ce qui est droit aux yeux de Yahweh. La prospérité d’Ozias fut exceptionnelle (2 Ch. 26:8 ; 2 Ch. 26:15). Tant qu’il dépendait de son Dieu, il ne lui manquait rien. Mais le diable savait comment le faire tomber dans l’orgueil, comme il sait faire tomber de nos jours beaucoup de chrétiens confortés par leurs succès, au point qu’ils ne se méfient plus d’eux-mêmes. Or lorsqu’on est élevé, la chute n’est jamais loin (2 Ch. 26:16). Ozias se crut soudain autorisé à entrer dans le temple pour y brûler le parfum, l’acte le plus sacré du sacerdoce lévitique (No. 16:36-40). Il fut frappé par Dieu, la lèpre « éclata » sur son front, l’obligeant à se hâter de sortir du temple, tandis que les sacrificateurs précipitaient sa fuite. (2 Ch. 26:19-20).}, fils d’Amatsia, roi de Juda, régna.
\VS{2}Il était âgé de seize ans lorsqu’il commença à régner, et il régna cinquante-deux ans à Jérusalem. Sa mère s’appelait Jecolia, elle était de Jérusalem.
\VS{3}Il fit ce qui est droit aux yeux de Yahweh, entièrement comme avait fait Amatsia, son père.
\VS{4}Seulement, les hauts lieux ne disparurent pas ; le peuple offrait encore des sacrifices et des parfums sur les hauts lieux.
\TextTitle{Jugement de Yahweh sur Ozias par la lèpre\FTNTT{2 Ch. 26:16-21}}
\VS{5}Alors Yahweh frappa le roi, qui fut lépreux jusqu’au jour de sa mort, et il demeura dans une maison à l’écart. Et Jotham, fils du roi, avait la charge de la maison, jugeant le peuple du pays.
\VS{6}Le reste des actions d’Azaria, tout ce qu’il a fait, cela n’est-il pas écrit dans le livre des Chroniques des rois de Juda ?
\VS{7}Azaria se coucha avec ses pères, et fut enseveli avec ses pères dans la cité de David, et Jotham, son fils, régna à sa place.
\TextTitle{Conspiration de Schallum contre Zacharie, roi d'Israël}
\VS{8}La trente-huitième année d’Azaria, roi de Juda, Zacharie, fils de Jéroboam, commença à régner sur Israël à Samarie, et il régna six mois.
\VS{9}Il fit ce qui est mal aux yeux de Yahweh, comme avaient fait ses pères ; il ne se détourna point des péchés de Jéroboam, fils de Nebath, par lesquels il avait fait pécher Israël.
\VS{10}Schallum, fils de Jabesch, fit une conspiration contre lui, et le frappa devant le peuple. Il le tua, et régna à sa place.
\VS{11}Quant au reste des actions de Zacharie, voilà, elles sont écrites dans le livre des Chroniques des rois d’Israël.
\VS{12}Ainsi s’accomplit la parole que Yahweh avait déclarée à Jéhu, en disant : Tes fils seront assis sur le trône d’Israël jusqu’à la quatrième génération, et il en fut ainsi\FTNT{2 R. 10:30.}.
\TextTitle{Schallum, roi d'Israël}
\VS{13}Schallum, fils de Jabesch, commença à régner la trente-neuvième année d’Ozias, roi de Juda. Il régna pendant un mois à Samarie.
\VS{14}Menahem, fils de Gadi, monta de Thirtsa et vint dans Samarie, et frappa dans Samarie Schallum, fils de Jabesch, et le fit mourir ; et il régna à sa place.
\VS{15}Le reste des actions de Schallum, et la conspiration qu’il forma, cela est écrit dans le livre des Chroniques des rois d’Israël.
\TextTitle{Menahem, roi d'Israël}
\VS{16}Alors Menahem frappa Thiphsach et tous ceux qui y étaient, avec son territoire depuis Thirtsa ; il la frappa parce qu’elle ne lui avait point ouvert ses portes. Il fendit le ventre de toutes les femmes enceintes.
\VS{17}La trente-neuvième année d’Azaria, roi de Juda, Menahem, fils de Gadi, commença à régner sur Israël. Il régna dix ans à Samarie.
\VS{18}Il fit ce qui est mal aux yeux de Yahweh ; il ne se détourna point des péchés de Jéroboam, fils de Nebath, par lesquels il avait fait pécher Israël.
\TextTitle{Invasion d'Israël par le roi d'Assyrie\FTNTT{1 Ch. 5:26}}
\VS{19}Alors Pul, roi d’Assyrie, vint contre le pays ; et Menahem donna mille talents d’argent à Pul, afin qu’il l’aide à affermir son royaume entre ses mains.
\VS{20}Menahem leva cet argent sur tous ceux d’Israël qui avaient de la richesse pour le donner au roi d’Assyrie ; chacun cinquante sicles d’argent. Ainsi le roi d’Assyrie s’en retourna, et ne s’arrêta point dans le pays.
\VS{21}Le reste des actions de Menahem, tout ce qu’il a fait, cela n’est-il pas écrit dans le livre des Chroniques des rois d’Israël ?
\TextTitle{Pekachia, roi d'Israël}
\VS{22}Menahem se coucha avec ses pères, et Pekachia, son fils, régna à sa place.
\VS{23}La cinquantième année d’Azaria, roi de Juda, Pekachia, fils de Menahem, commença à régner sur Israël à Samarie. Il régna deux ans.
\VS{24}Il fit ce qui est mal aux yeux de Yahweh ; il ne se détourna point des péchés de Jéroboam, fils de Nebath, par lesquels il avait fait pécher Israël.
\TextTitle{Pekach s'empare du trône et devient roi d'Israël}
\VS{25}Pékach, fils de Remalia, son officier, conspira contre lui ; il le frappa à Samarie, dans le palais de la maison royale, de même qu’Argob et Arié ; il avait avec lui cinquante hommes d’entre les fils des Galaadites. Il fit ainsi mourir Pekachia, et il régna à sa place.
\VS{26}Le reste des actions de Pekachia, tout ce qu’il a fait, cela est écrit dans le livre des Chroniques des rois d’Israël.
\VS{27}La cinquante-deuxième année d’Azaria, roi de Juda, Pékach, fils de Remalia, commença à régner sur Israël à Samarie. Il régna vingt ans.
\VS{28}Il fit ce qui est mal aux yeux de Yahweh et ne se détourna point des péchés de Jéroboam, fils de Nebath, par lesquels il avait fait pécher Israël.
\VS{29}Du temps de Pékach, roi d’Israël, Tiglath-Piléser, roi d’Assyrie, vint et prit Ijjon, Abel-Beth-Maaca, Janoach, Kédesch, Hatsor, Galaad et la Galilée, et même tout le pays de Nephthali, et il emmena captifs les habitants en Assyrie.
\TextTitle{Osée conspire contre Pékach et règne sur Israël}
\VS{30}Osée, fils d’Ela, forma une conspiration contre Pékach, fils de Remalia, le frappa et le fit mourir. Il régna à sa place la vingtième année de Jotham, fils d’Ozias.
\VS{31}Le reste des actions de Pékach, tout ce qu’il a fait, cela est écrit dans le livre des Chroniques des rois d’Israël.
\TextTitle{Jotham, roi de Juda\FTNTT{2 R. 15:2 ; 2 Ch. 26:23 ; 27:1-9}}
\VS{32}La seconde année de Pékach, fils de Remalia, roi d’Israël, Jotham, fils d’Ozias, roi de Juda, commença à régner.
\VS{33}Il était âgé de vingt-cinq ans lorsqu’il commença à régner. Il régna seize ans à Jérusalem. Sa mère s’appelait Jeruscha, fille de Tsadok.
\VS{34}Il fit ce qui est droit aux yeux de Yahweh ; il agit entièrement comme avait agi Ozias, son père.
\VS{35}Seulement, les hauts lieux ne disparurent point ; et le peuple offrait encore des sacrifices et des parfums sur les hauts lieux. Jotham bâtit la porte supérieure de la maison de Yahweh.
\VS{36}Le reste des actions de Jotham, tout ce qu’il a fait, cela n’est-il pas écrit dans le livre des Chroniques des rois de Juda ?
\VS{37}Dans ce temps-là, Yahweh commença à envoyer contre Juda Retsin, roi de Syrie, et Pékach, fils de Remalia.
\VS{38}Jotham se coucha avec ses pères, et il fut enseveli dans la cité de David, son père. Et Achaz, son fils, régna à sa place.
\Chap{16}
\TextTitle{Achaz, roi de Juda\FTNTT{2 R. 15:38 ; 2 Ch. 28:1-4}}
\VerseOne{}La dix-septième année de Pékach, fils de Remalia, Achaz, fils de Jotham, roi de Juda, commença à régner.
\VS{2}Achaz était âgé de vingt ans lorsqu’il commença à régner. Il régna seize ans à Jérusalem. Il ne fit point ce qui est droit aux yeux de Yahweh, son Dieu, comme avait fait David, son père.
\VS{3}Mais il suivit la voie des rois d’Israël, et il fit même passer son fils par le feu, selon les abominations des nations que Yahweh avait chassées devant les enfants d’Israël.
\VS{4}Il offrait aussi des sacrifices et des parfums sur les hauts lieux, sur les coteaux, et sous tout arbre vert.
\TextTitle{Juda envahi par les rois d'Assyrie et d'Israël\FTNTT{2 Ch. 28:5-19}}
\VS{5}Alors Retsin, roi de Syrie, et Pékach, fils de Remalia, roi d’Israël, montèrent contre Jérusalem pour lui faire la guerre. Ils assiégèrent Achaz ; mais ne purent en venir à bout par les armes.
\VS{6}Dans ce même temps, Retsin, roi de Syrie, fit rentrer Elath au pouvoir des Syriens ; il expulsa les Juifs d’Elath, et les Syriens vinrent à Elath, où ils ont demeuré jusqu’à ce jour.
\TextTitle{Le roi d'Assyrie vient en aide à Achaz et s'empare de Damas\FTNTT{2 Ch. 28:16-25}}
\VS{7}Achaz envoya des messagers à Tiglath-Piléser, roi d’Assyrie, pour lui dire : Je suis ton serviteur et ton fils ; monte et délivre-moi de la main du roi des Syriens, et de la main du roi d’Israël, qui s’élèvent contre moi.
\VS{8}Alors Achaz prit l’argent et l’or qui se trouvaient dans la maison de Yahweh, et dans les trésors de la maison royale, et il les envoya en présent au roi d’Assyrie.
\VS{9}Le roi d’Assyrie l’écouta ; il monta contre Damas, la prit, emmena les habitants en captivité à Kir, et fit mourir Retsin.
\VS{10}Alors le roi Achaz s’en alla à la rencontre de Tiglath-Piléser, roi d’Assyrie, à Damas. Et ayant vu l’autel\FTNT{Achaz, roi de Juda, se rendit chez le roi d’Assyrie et il fut fasciné par l’autel de son dieu au point de le convoiter. Il demanda au sacrificateur Urie de fabriquer un autel identique, dont le modèle n’était pas celui que Yahweh avait décrit à Moïse. Il introduisit un objet de culte d’origine païenne dans le temple de Jérusalem, sous prétexte d’honorer Yahweh. Certains « Pères de l’Eglise », comme les empereurs Constantin I (285-337) et Théodose I (347-395), se sont comportés exactement comme Achaz en adoptant les pratiques païennes. Les historiens s’accordent pour dire que la diffusion de la Parole de Dieu sous l’empire de Constantin I (285-337), empereur de Rome, avait des fins strictement politiques. Cette politique a eu deux conséquences essentielles concernant l’influence de l’Eglise chrétienne et son fonctionnement de plus en plus éloigné de la Parole de Dieu :
- Les peuples païens ont introduit leurs rites idolâtres au sein de l’Eglise. En effet, les dogmes de l’institution devaient plaire à la majorité.
- L’Eglise chrétienne cessant d’être persécutée, son fonctionnement intimiste fondé sur l’implication de chaque croyant et l’exercice du sacerdoce universel des chrétiens, a changé à cause de l’effet de masse. Devenant numériquement très importante, il a fallu imposer une autorité capable de contenir un nombre de fidèles de plus en plus élevé. Mais à cause de cette augmentation numérique et de la présence de « faux convertis » lié au fait que l'adhésion au christianisme (religion chrétienne fondée par les hommes) devenait une obligation, l’étude de la Parole, la fraction du pain et la prière ne pouvaient plus perdurer. C’est ainsi que beaucoup d’églises ont commencé à subir l’influence du monde.} qui était à Damas, le roi Achaz envoya au sacrificateur Urie la forme et le modèle exact de cet autel.
\VS{11}Le sacrificateur Urie construisit un autel entièrement d’après le modèle envoyé de Damas par le roi Achaz, et le sacrificateur Urie le fit avant que le roi Achaz soit de retour de Damas.
\VS{12}Quand le roi Achaz revint de Damas, et vit l’autel, il s’en approcha, et y monta ;
\VS{13}Il fit brûler son holocauste et son sacrifice, versa ses libations, et répandit sur l’autel le sang de ses sacrifices d’offrande de paix.
\VS{14}Il éloigna de la face de la maison l’autel d’airain qui était devant Yahweh, afin qu’il ne soit pas entre le nouvel autel et la maison de Yahweh ; et il le plaça à côté du nouvel autel, vers le nord.
\VS{15}Et le roi Achaz donna cet ordre au sacrificateur Urie : Fais brûler l’holocauste du matin et l’offrande du soir, l’holocauste du roi et son offrande, les holocaustes de tout le peuple du pays et leurs offrandes, verses-y leurs libations, et répands-y tout le sang des holocaustes et tout le sang des sacrifices ; mais pour ce qui concerne l’autel d’airain, je m’en occuperai.
\VS{16}Le sacrificateur Urie exécuta tout ce que le roi Achaz lui avait ordonné.
\VS{17}Le roi Achaz brisa les panneaux des bases, et en ôta les cuves qui étaient dessus. Il descendit la mer de dessus les bœufs d’airain qui étaient sous elle, et il la posa sur un pavé de pierre.
\VS{18}Il changea aussi de la maison de Yahweh, à cause du roi d’Assyrie, le portique du sabbat qu’on y avait bâti et l’entrée extérieure du roi.
\TextTitle{Mort d'Achaz ; Ezéchias, roi de Juda\FTNTT{2 Ch. 28:26-27}}
\VS{19}Le reste des actions d’Achaz, et ce tout qu’il a fait, cela n’est-il pas écrit dans le livre des Chroniques des rois de Juda ?
\VS{20}Achaz se coucha avec ses pères, et il fut enseveli avec ses pères dans la cité de David. Et Ezéchias, son fils, régna à sa place.
\Chap{17}
\TextTitle{Osée, dernier roi d'Israël}
\VerseOne{}La douzième année d’Achaz, roi de Juda, Osée, fils d’Ela, régna à Samarie sur Israël. Il régna neuf ans.
\VS{2}Il fit ce qui est mal aux yeux de Yahweh, non pas toutefois comme les rois d’Israël qui avaient été avant lui.
\TextTitle{Tentative d'Osée de s'affranchir du tribut imposée par le roi d'Assyrie}
\VS{3}Salmanasar\FTNT{Le royaume d’Israël a été détruit en 722 av. J.-C., par l’empereur assyrien Salmanasar V, ou Salmanazar, (règne : 727-722 av. J.-C.), après  avoir assiégé trois ans le roi Osée (règne : 732-722 av. J.-C.) dans sa capitale Samarie. Celui-ci ne payait plus le tribut et essayait d’obtenir l’appui de l’Egypte pour retrouver l’indépendance. Le royaume d’Israël a disparu au début du 8ème siècle av. J.-C., provoquant la dispersion dans le monde de plusieurs juifs issus des dix tribus. L’origine des Samaritains remonte à cette déportation, après que le royaume du Nord soit tombé aux mains de Salmanasar, roi d’Assyrie. Malgré les déportations, les Assyriens n’avaient pas laissé déserte cette région appelée « Samarie » ; plusieurs Israélites y étaient restés et des colons d’autres provinces assyriennes vinrent s’y établir. Les Samaritains sont issus du mélange de ces populations, et leur religion est un mélange entre le culte à Yahweh avec celui des dieux étrangers.}, roi d’Assyrie, monta contre lui ; et Osée lui fut assujetti, et lui paya un tribut.
\VS{4}Mais le roi d’Assyrie découvrit une conspiration chez Osée, qui avait envoyé des messagers vers So, roi d’Egypte, et qui ne payait plus le tribut tous les ans au roi d’Assyrie. C’est pourquoi le roi d’Assyrie le fit enfermer et enchaîner dans une prison.
\TextTitle{Siège de Samarie par le roi d'Assyrie}
\VS{5}Le roi d’Assyrie parcourut tout le pays, et monta contre Samarie qu’il assiégea pendant trois ans.
\TextTitle{Les causes de la cativité d'Israël en Assyrie}
\VS{6}La neuvième année d’Osée, le roi d’Assyrie prit Samarie, et emmena captifs les Israélites en Assyrie. Il les fit habiter à Chalach, et sur le Chabor, fleuve de Gozan, et dans les villes des Mèdes.
\VS{7}Cela arriva parce que les enfants d’Israël péchèrent contre Yahweh, leur Dieu, qui les avait fait monter hors du pays d’Egypte, de dessous la main de Pharaon, roi d’Egypte, et parce qu’ils craignirent d’autres dieux.
\VS{8}Ils suivirent les coutumes des nations que Yahweh avait chassées devant les enfants d’Israël, et celles des rois d’Israël qu’ils avaient établis.
\VS{9}Les enfants d’Israël firent en secret des choses qui n’étaient point droites, contre Yahweh, leur Dieu. Ils se bâtirent des hauts lieux dans toutes leurs villes, depuis la tour des gardes jusqu’aux villes fortes.
\VS{10}Ils se dressèrent des statues et des idoles, sur toutes les hautes collines et sous tout arbre vert.
\VS{11}Et là, ils brûlèrent des parfums sur tous les hauts lieux, comme les nations que Yahweh avait chassées devant eux, et ils firent des choses mauvaises pour irriter Yahweh.
\VS{12}Ils servirent les idoles, au sujet desquelles Yahweh leur avait dit : Vous ne ferez pas cela\FTNT{1 R. 12:28}.
\VS{13}Yahweh fit avertir Israël et Juda par tous ses prophètes, tous les voyants, en disant : Détournez-vous de toutes vos mauvaises voies, revenez, et gardez mes commandements et mes ordonnances, en suivant entièrement la loi que j’ai prescrite à vos pères et que je vous ai envoyée par mes serviteurs les prophètes.
\VS{14}Mais ils n’écoutèrent point, et ils raidirent leur cou, comme leurs pères avaient raidi leur cou, et qui n’avaient pas cru en Yahweh, leur Dieu.
\VS{15}Ils rejetèrent ses lois, et son alliance qu’il avait traitée avec leurs pères, et ses avertissements, qu’il leur avait adressés. Ils allèrent après des choses de néant et ne furent eux-mêmes que néant, après les nations qui les entouraient et que Yahweh leur avait défendu d’imiter.
\VS{16}Ils abandonnèrent tous les commandements de Yahweh, leur Dieu, ils firent deux veaux en métal fondu, ils fabriquèrent des idoles d’Astarté\FTNT{Voir commentaire en Jg. 2:13.}, ils se prosternèrent devant toute l’armée des cieux, et ils servirent Baal.
\VS{17}Ils firent aussi passer leurs fils, et leurs filles par le feu, ils s’adonnèrent à la divination et aux enchantements, et ils se vendirent pour faire ce qui est mal aux yeux de Yahweh afin de l’irriter.
\VS{18}C’est pourquoi, Yahweh fut très irrité contre Israël, et il les rejeta, il n’est resté que la seule tribu de Juda.
\VS{19}Même Juda n’avait pas gardé les commandements de Yahweh, son Dieu, mais ils avaient suivi les ordonnances qu’Israël avait établies.
\VS{20}C’est pourquoi Yahweh rejeta toute la race d’Israël ; il les a humiliés, il les a livrés entre les mains des pillards, il a fini par les chasser loin de sa face.
\VS{21}Car Israël s’était détaché de la maison de David, et avait établi roi Jéroboam, fils de Nebath. Jéroboam avait détourné Israël de Yahweh, afin qu’il ne le suive plus, et lui avait fait commettre un grand péché.
\VS{22}Les enfants d’Israël s’étaient livrés à tous les péchés que Jéroboam avait commis ; ils ne s’en détournèrent pas,
\VS{23}Jusqu’à ce que Yahweh ait chassé Israël de devant sa face, comme il l’avait annoncé par tous ses serviteurs les prophètes. Et Israël fut emmené captif loin de son pays en Assyrie, jusqu’à ce jour.
\TextTitle{Jugement sur les étrangers occupant les villes d'Israël}
\VS{24}Le roi d’Assyrie fit venir des gens de Babylone, de Cutha, d’Avva, de Hamath, et de Sepharvaïm. Il les fit habiter dans les villes de Samarie, à la place des enfants d’Israël. Ils prirent possession de la Samarie, et habitèrent dans ses villes.
\VS{25}Lorsqu’ils commencèrent à y habiter, ils ne craignirent point Yahweh, et Yahweh envoya contre eux des lions, qui les tuaient.
\VS{26}Et on dit au roi d’Assyrie : Les nations que tu as transportées et fait habiter dans les villes de Samarie ne connaissent pas la manière de servir le dieu du pays, c’est pourquoi il a envoyé contre eux des lions, et voilà, ces lions les tuent, parce qu’ils ne connaissent pas la manière de servir le dieu du pays.
\TextTitle{L'idolâtrie dans les villes occupées}
\VS{27}Alors le roi d’Assyrie donna cet ordre, en disant : Faites-y aller quelqu’un des sacrificateurs que vous avez emmenés de là en captivité ; qu’il parte pour s’y établir et qu’il leur enseigne la manière de servir le dieu du pays.
\VS{28}Alors l’un des sacrificateurs qui avaient été emmenés captifs de Samarie vint s’établir à Béthel, et leur enseigna comment ils devaient craindre Yahweh.
\VS{29}Mais les nations firent chacune leurs dieux dans les villes qu’elles habitaient, et les placèrent dans les maisons des hauts lieux bâties par les Samaritains.
\VS{30}Les gens de Babylone firent Succoth-Benoth, les gens de Cuth firent Nergal, et les gens de Hamath firent Aschima.
\VS{31}Ceux d’Avva firent Nibchaz et Thartak ; ceux de Sepharvaïm brûlaient leurs enfants par le feu à Adrammélec et Anammélec, les dieux de Sepharvaïm.
\VS{32}Toutefois, ils redoutaient Yahweh, et ils établirent des sacrificateurs des hauts lieux pris parmi tout le peuple ; ces sacrificateurs offraient pour eux des sacrifices dans les maisons des hauts lieux.
\VS{33}Ils redoutaient Yahweh, et en même temps ils servaient leurs dieux à la manière des nations d’où on les avait transportés.
\VS{34}Et jusqu’à ce jour, ils font encore selon leurs premières coutumes : ils ne craignirent point Yahweh, et ils ne se conforment ni à leurs lois et à leurs ordonnances, ni à la loi et aux commandements prescrits par Yahweh Dieu aux enfants de Jacob, qu’il appela du nom d’Israël.
\VS{35}Yahweh avait traité alliance avec eux, et leur avait donné cet ordre, en disant : Vous ne craindrez point d’autres dieux ; et ne vous prosternerez point devant eux ; vous ne les servirez point, et vous ne leur offrirez point de sacrifices.
\VS{36}Mais vous craindrez Yahweh, qui vous a fait monter hors du pays d’Egypte avec une grande puissance et à bras étendu ; et vous vous prosternerez devant lui, et vous lui offrirez des sacrifices.
\VS{37}Vous observerez et mettrez toujours en pratique les statuts, les ordonnances, la loi et les commandements, qu’il a écrits pour vous, et vous ne craindrez pas d’autres dieux.
\VS{38}Vous n’oublierez pas l’alliance que j’ai traitée avec vous, et vous ne craindrez point d’autres dieux.
\VS{39}Mais vous craindrez Yahweh, votre Dieu, et il vous délivrera de la main de tous vos ennemis.
\VS{40}Ils n’écoutèrent pas, et ils firent selon leurs premières coutumes.
\VS{41}Ainsi ces nations-là redoutaient Yahweh, et servaient leurs images ; leurs enfants et les enfants de leurs enfants font jusqu’à ce jour ce que leurs pères ont fait.
\Chap{18}
\TextTitle{Règne d'Ezéchias sur Juda\FTNTT{2 R. 16:20 ; 2 Ch. 29:1-31:21}}
\VerseOne{}La troisième année d’Osée, fils d’Ela, roi d’Israël, Ezéchias, fils d’Achaz, roi de Juda, commença à régner.
\VS{2}Il était âgé de vingt-cinq ans lorsqu’il commença à régner, il régna vingt-neuf ans à Jérusalem. Sa mère s’appelait Abi, fille de Zacharie.
\VS{3}Il fit ce qui est droit aux yeux de Yahweh, entièrement comme avait fait David, son père.
\TextTitle{Mouvement de réveil sous Ezéchias\FTNTT{2 Ch. 29:3-31:21}}
\VS{4}Il fit disparaître les hauts lieux, mit en pièces les statues, abattit les idoles d’Astarté, et il brisa le serpent d’airain que Moïse avait fait, car les enfants d’Israël avaient jusqu’alors brûlé des parfums devant lui ; ils l’appelaient Nehuschtan.
\VS{5}Il se confia en Yahweh, le Dieu d’Israël ; et parmi tous les rois de Juda qui vinrent après ou qui le précédèrent, il n’y en eut point de semblable à lui.
\VS{6}Il s’attacha à Yahweh, il ne se détourna point de lui, et il observa les commandements que Yahweh avait prescrits à Moïse.
\TextTitle{Révolte d'Ezéchias contre l'Assyrie}
\VS{7}Et Yahweh fut avec Ezéchias, qui réussit dans toutes ses entreprises. Il se révolta contre le roi d’Assyrie, et ne lui fut plus assujetti.
\VS{8}Il frappa les Philistins jusqu’à Gaza, et ravagea leur territoire depuis les tours des gardes jusqu’aux villes fortes.
\TextTitle{Captivité d'Israël par l'Assyrie\FTNTT{2 R. 17:4-6}}
\VS{9}La quatrième année du roi Ezéchias, qui était la septième du règne d’Osée, fils d’Ela, roi d’Israël, Salmanasar(2), roi d’Assyrie, monta contre Samarie, et l’assiégea.
\VS{10}Il la prit au bout de trois ans, la sixième année du règne d’Ezéchias, qui était la neuvième d’Osée, roi d’Israël, Samarie fut prise.
\VS{11}Le roi d’Assyrie emmena Israël en Assyrie, et il les établit à Chalach, sur le Chabor, fleuve de Gozan, et dans les villes des Mèdes,
\VS{12}parce qu’ils n’avaient point obéi à la voix de Yahweh, leur Dieu, et qu’ils avaient transgressé son alliance, parce qu’ils n’avaient ni écouté ni mis en pratique tout ce qu’avait ordonné Moïse, serviteur de Yahweh.
\TextTitle{Invasion de Juda par Sanchérib\FTNTT{2 Ch. 32:1-15, 30 ; Es. 36:1-10}}
\VS{13}La quatorzième année du roi Ezéchias, Sanchérib, roi d’Assyrie, monta contre toutes les villes fortes de Juda, et les prit.
\VS{14}Ezéchias, roi de Juda, envoya dire au roi d’Assyrie à Lakis : J’ai commis une faute ! Eloigne-toi de moi. Je payerai tout ce que tu m’imposeras. Et le roi d’Assyrie imposa à Ezéchias, roi de Juda, trois cents talents d’argent et trente talents d’or.
\VS{15}Ezéchias donna tout l’argent qui se trouvait dans la maison de Yahweh, et dans les trésors de la maison royale.
\VS{16}En ce temps-là, Ezéchias enleva les lames d’or dont il avait couvert les portes et les linteaux du temple de Yahweh, pour les livrer au roi d’Assyrie.
\VS{17}Puis le roi d’Assyrie envoya de Lakis à Jérusalem, vers le roi Ezéchias, Tharthan, Rab-Saris et Rabschaké avec une puissante armée. Ils montèrent et arrivèrent à Jérusalem. Lorsqu’ils furent montés et arrivés, ils s’arrêtèrent à l’aqueduc de l’étang supérieur, qui est sur le chemin du champ du foulon.
\VS{18}Ils appelèrent le roi tout haut ; alors Eliakim, fils de Hilkija, chef de la maison du roi, Schebna, le secrétaire et Joach, fils d’Asaph, l’archiviste, se rendirent auprès d’eux.
\VS{19}Rabschaké leur dit : Dites maintenant à Ezéchias : Ainsi parle le grand roi, le roi d’Assyrie : Quelle est cette confiance sur laquelle tu t’appuies ?
\VS{20}Tu as dit : Il faut pour la guerre le conseil et la force. Mais ce ne sont que des paroles. Mais en qui donc as-tu placé ta confiance, pour te rebeller contre moi ?
\VS{21}Voici maintenant, tu l’as placée dans l’Egypte, dans ce roseau cassé, qui pénètre et perce la main de quiconque s’appuie dessus : tel est Pharaon, roi d’Egypte, pour tous ceux qui se confient en lui.
\VS{22}Peut-être me direz-vous : Nous nous confions en Yahweh, notre Dieu, mais n’est-ce pas celui dont Ezéchias a détruit les hauts lieux, et les autels, en disant à Juda et à Jérusalem : Vous vous prosternerez devant cet autel à Jérusalem ?
\VS{23}Maintenant, donne des otages au roi d’Assyrie, mon maître, et je te donnerai deux mille chevaux, si tu peux donner autant des cavaliers pour les monter.
\VS{24}Comment donc repousserais-tu un seul gouverneur d’entre les serviteurs de mon maître ? Mais tu mets ta confiance dans l’Egypte, à cause des chars et des cavaliers.
\VS{25}D’ailleurs, est-ce sans l’ordre de Yahweh que je suis monté contre ce lieu, pour le détruire ? Yahweh m’a dit : Monte contre ce pays, et détruis-le.
\TextTitle{Ménaces de Rabschaké\FTNTT{2 Ch. 32:16, 18-19 ; Es. 36:11-21}}
\VS{26}Alors Eliakim, fils de Hilkija, Schebna et Joach dirent à Rabschaké : Nous te prions de parler en araméen à tes serviteurs, car nous le comprenons ; et ne nous parle pas en langue judaïque, aux oreilles du peuple qui est sur la muraille.
\VS{27}Rabschaké leur répondit : Est-ce à ton maître et à toi que mon maître m’a envoyé dire ces paroles ? Ne m’a-t-il pas envoyé vers les hommes qui se tiennent sur la muraille pour leur dire qu’ils mangeront leurs propres excréments, et qu’ils boiront leur urine avec vous ?
\VS{28}Rabschaké, s’étant avancé, cria à haute voix en langue judaïque, il parla et dit : Ecoutez la parole du grand roi, le roi d’Assyrie.
\VS{29}Ainsi parle le roi : Qu’Ezéchias ne vous trompe pas, car il ne pourra pas vous délivrer de ma main.
\VS{30}Qu’Ezéchias ne vous amène pas à vous confier en Yahweh, en disant : Yahweh nous délivrera certainement, et cette ville ne sera pas livrée entre les mains du roi d’Assyrie.
\VS{31}N’écoutez pas Ezéchias ; car ainsi parle le roi d’Assyrie : Faites la paix avec moi, et rendez-vous à moi, et chacun de vous mangera de sa vigne, et de son figuier, et chacun boira de l’eau de sa citerne,
\VS{32}jusqu’à ce que je ne vienne, et que je ne vous emmène dans un pays comme le vôtre, dans un pays de blé et de bon vin, un pays de pain et de vignes, un pays d’oliviers qui portent de l’huile, et de miel, et vous vivrez et vous ne mourrez pas. Mais n’écoutez pas Ezéchias ; car il pourrait vous séduire, en disant : Yahweh nous délivrera.
\VS{33}Les dieux des nations ont-ils délivré chacun leur pays de la main du roi d’Assyrie ?
\VS{34}Où sont les dieux de Hamath et d’Arpad ? Où sont les dieux de Sépharvaïm, d’Héna et d’Ivva ? Et même ont-ils délivré Samarie de ma main ?
\VS{35}Parmi tous les dieux de ces pays, quels sont ceux qui ont délivré leur pays de ma main, pour dire que Yahweh délivrera Jérusalem de ma main ?
\VS{36}Le peuple se tut, et on ne lui répondit pas un mot ; car le roi avait donné cet ordre : Vous ne lui répondrez point(1).
\VS{37}Après cela, Eliakim, fils de Hilkija, chef de la maison du roi, et Schebna le secrétaire, et Joach, fils d’Asaph, l’archiviste, vinrent auprès d’Ezéchias, les vêtements déchirés, et ils lui rapportèrent les paroles de Rabschaké.
\Chap{19}
\TextTitle{Intercession d'Esaïe pour Ezéchias\FTNTT{2 Ch. 32:20-22 ; Es. 36:22-37:5}}
\VerseOne{}Dès que le roi Ezéchias eut entendu ces choses, il déchira ses vêtements, se couvrit d’un sac, et entra dans la maison de Yahweh.
\VS{2}Puis il envoya Eliakim, chef de la maison du roi, et Schebna le secrétaire, et les plus anciens des sacrificateurs, couverts de sacs, vers Esaïe, le prophète, fils d’Amots.
\VS{3}Ils lui dirent : Ainsi parle Ezéchias : Ce jour est un jour d’angoisse, de châtiment et d’opprobre ; car les enfants sont près du sein maternel, mais il n’y a point de force pour enfanter.
\VS{4}Peut-être Yahweh, ton Dieu, a-t-il entendu toutes les paroles de Rabschaké, que le roi d’Assyrie, son maître, a envoyé pour blasphémer le Dieu vivant, et peut-être Yahweh, ton Dieu, exercera-t-il ses châtiments à cause des paroles qu’il a entendues. Fais donc une prière pour le reste qui subsiste encore.
\VS{5}Les serviteurs du roi Ezéchias vinrent donc vers Esaïe.
\TextTitle{Yahweh rassure Ezéchias\FTNTT{Es. 37:6-7}}
\VS{6}Et Esaïe leur dit : Voici ce que vous direz à votre maître : Ainsi parle Yahweh : Ne t’effraie point des paroles que tu as entendues, par lesquelles les serviteurs du roi d’Assyrie m’ont blasphémé.
\VS{7}Voici, je vais mettre en lui un esprit, tel que sur une nouvelle qu’il recevra, il retournera dans son pays ; et je le ferai tomber par l’épée dans son pays.
\TextTitle{Défi du roi d'Assyrie au Dieu d'Israël\FTNTT{2 Ch. 32:17 ; Es. 37:8-13}}
\VS{8}Rabschaké s’étant retiré, trouva le roi d’Assyrie qui attaquait Libna, car il avait appris qu’il était parti de Lakis.
\VS{9}Le roi d’Assyrie reçut une nouvelle au sujet de Tirhaka, roi d’Ethiopie ; on lui dit : Voici, il est sorti pour te combattre. C’est pourquoi le roi d’Assyrie retourna dans son pays, mais il envoya des messagers à Ezéchias, en leur disant :
\VS{10}Vous parlerez ainsi à Ezéchias, roi de Juda, et lui direz : Que ton Dieu, en qui tu te confies, ne t’abuse pas en te disant : Jérusalem ne sera point livrée entre les mains du roi d’Assyrie.
\VS{11}Voici, tu as entendu ce que les rois d’Assyrie ont fait à tous les pays, et comment ils les ont détruits entièrement ; et tu échapperais ?
\VS{12}Les dieux des nations que mes ancêtres ont détruites, les ont-ils délivrées, de Gozan, de Charan, de Retseph, et des fils d’Eden, qui sont à Telassar ?
\VS{13}Où sont le roi de Hamath, le roi d’Arpad, et le roi de la ville de Sepharvaïm, d’Héna et d’Ivva ?
\TextTitle{Le reste de Yahweh assurera la pérénité de Juda\FTNTT{2 Ch. 32:20 ; Es. 37:14-20}}
\VS{14}Quand Ezéchias reçut la lettre de la main des messagers, il la lut. Puis il monta à la maison de Yahweh, et la déploya devant Yahweh,
\VS{15}puis Ezéchias lui adressa cette prière et dit : Ô Yahweh, Dieu d’Israël ! Qui est assis entre les chérubins, c’est toi qui es le seul Dieu de tous les royaumes de la terre, c’est toi qui as fait les cieux et la terre.
\VS{16}Ô Yahweh ! Incline ton oreille, et écoute. Ouvre tes yeux, et regarde. Ecoute les paroles de Sanchérib, et de Rabschaké qu’il a envoyé pour blasphémer le Dieu vivant.
\VS{17}Il est vrai, ô Yahweh ! Que les rois d’Assyrie ont détruit ces nations et ravagé leurs pays,
\VS{18}et qu’ils ont jeté dans le feu leurs dieux ; mais ils n’étaient pas des dieux, mais c’étaient des ouvrages de mains d’homme, du bois, et de la pierre, c’est pourquoi ils les ont détruits.
\VS{19}Maintenant donc, ô Yahweh, notre Dieu ! Je te prie, délivre-nous de la main de Sanchérib, afin que tous les royaumes de la terre sachent que c’est toi, ô Yahweh, qui es le seul Dieu.
\VS{20}Alors Esaïe, fils d’Amots, envoya dire à Ezéchias : Ainsi parle Yahweh, le Dieu d’Israël : Je t’ai exaucé dans ce que tu m’as demandé au sujet de Sanchérib, roi d’Assyrie.
\VS{21}Voici la parole que Yahweh a prononcée contre lui : Elle te méprise, elle se moque de toi, la fille, vierge de Sion ; elle hoche la tête après toi, la fille de Jérusalem.
\VS{22}Qui as-tu outragé et blasphémé ? Contre qui as-tu élevé la voix ? Tu as porté tes yeux en haut, vers le Saint d’Israël !
\VS{23}Tu as insulté le Seigneur par le moyen de tes messagers, et tu as dit : J’ai gravi le sommet des montagnes avec la multitude de mes chars, les extrémités du Liban ; je couperai les plus hauts de ses cèdres et les plus beaux de ses cyprès, et j’atteindrai sa dernière cime, la forêt de son verger.
\VS{24}J’ai creusé des sources, après avoir bu les eaux étrangères, et je tarirai avec la plante de mes pieds tous les fleuves de l’Egypte.
\VS{25}N’as-tu pas appris que j’ai préparé cette ville déjà dès longtemps, et que dès les temps anciens je l’ai ainsi formée ? Et maintenant l’aurais-je conservé pour être réduite en désolation, et les villes fortes, en monceaux de ruines ?
\VS{26}Il est vrai que leurs habitants sont impuissants, épouvantés, et confus ; ils sont devenus comme l’herbe des champs et la tendre verdure, comme le gazon des toits, et le blé brûlé avant la formation de sa tige.
\VS{27}Mais je connais ta demeure, ta sortie et ton entrée, et comment tu es furieux contre moi.
\VS{28}Parce que tu es furieux contre moi, et que ton insolence est montée à mes oreilles, je mettrai ma boucle à tes narines, et mon mors entre tes lèvres, et je te ferai retourner par le chemin par lequel tu es venu.
\VS{29}Que ceci soit un signe pour toi, ô Ezéchias : On mangera cette année le produit du grain tombé, et la deuxième année, ce qui croît de soi-même ; mais la troisième année, vous sèmerez, et vous moissonnerez, vous planterez des vignes, et vous en mangerez le fruit.
\VS{30}Ce qui aura été épargné de la maison de Juda, ce qui sera resté poussera encore des racines par-dessous, et produira du fruit par-dessus.
\VS{31}Car il sortira de Jérusalem un reste, et de la montagne de Sion des réchappés. Voilà ce que fera le zèle de Yahweh des armées.
\VS{32}C’est pourquoi ainsi parle Yahweh, sur le roi d’Assyrie : Il n’entrera point dans cette ville, il n’y lancera aucune flèche, il ne se présentera point contre elle avec le bouclier, et il n’élèvera point des retranchements contre elle.
\VS{33}Il s’en retournera par le chemin par lequel il est venu, et il n’entrera point dans cette ville, dit Yahweh.
\VS{34}Car je protègerai cette ville, afin de la délivrer, par amour pour moi, et par amour pour David, mon serviteur.
\TextTitle{Jugement de Yahweh sur les Assyriens et leur roi\FTNTT{2 Ch. 32:21 ; Es. 37:36-38}}
\VS{35}Cette nuit-là, l’Ange de Yahweh\FTNT{Voir commentaire en Ge. 16:7.} sortit, et frappa cent quatre-vingt-cinq mille hommes dans le camp des Assyriens. Et quand on se leva de bon matin, voici, ils étaient tous morts.
\VS{36}Alors Sanchérib, roi d’Assyrie, leva son camp, partit et s’en retourna ; et il resta à Ninive.
\VS{37}Il arriva, comme il était prosterné dans la maison de Nisroc, son dieu, qu’Adrammélec et Scharetser, ses fils, le tuèrent avec l’épée, puis ils se sauvèrent au pays d’Ararat ; et Esar-Haddon, son fils, régna à sa place.
\Chap{20}
\TextTitle{Yahweh accorde sa miséricorde à Ezéchias\FTNTT{2 Ch. 32:24-29 ; Es. 38}}
\VerseOne{}En ce temps-là, Ezéchias fut malade à la mort. Le prophète Esaïe, fils d’Amots, vint auprès lui, et lui dit : Ainsi parle Yahweh : Donne tes ordres à ta maison, car tu vas mourir, et tu ne vivras plus.
\VS{2}Alors Ezéchias tourna son visage contre le mur, et fit sa prière à Yahweh, en disant :
\VS{3}Je te prie, ô Yahweh ! Souviens-toi que j’ai marché devant toi avec fidélité et intégrité de cœur, et que j’ai fait ce qui est agréable à tes yeux ! Et Ezéchias pleura abondamment.
\VS{4}Esaïe n’était pas encore sorti de la cour du milieu, que la parole de Yahweh lui fut adressée, en disant :
\VS{5}Retourne, et dis à Ezéchias, chef de mon peuple : Ainsi parle Yahweh, le Dieu de David, ton père : J’ai exaucé ta prière, j’ai vu tes larmes. Voici je te guérirai ; dans trois jours tu monteras à la maison de Yahweh.
\VS{6}J’ajouterai quinze ans à tes jours, je te délivrerai, toi et cette ville, de la main du roi d’Assyrie ; et je protégerai cette ville, par amour pour moi, et par amour pour David, mon serviteur.
\VS{7}Puis Esaïe dit : Prenez une masse de figues sèches. Et ils la prirent, et l’appliquèrent sur l’ulcère. Et Ezéchias fut guéri.
\VS{8}Ezéchias avait dit à Esaïe : A quel signe connaîtrai-je que Yahweh me guérira, et qu’au troisième jour je monterai à la maison de Yahweh ?
\VS{9}Esaïe répondit : Voici, de la part de Yahweh, le signe auquel tu connaîtras que Yahweh accomplira la parole qu’il a prononcée : L’ombre s’avancera-t-elle de dix degrés, ou reculera-t-elle en arrière de dix degrés ?
\VS{10}Ezéchias dit : C’est peu de chose que l’ombre s’avance de dix degrés ; mais plutôt que l’ombre recule en arrière de dix degrés.
\VS{11}Alors Esaïe, le prophète, invoqua Yahweh, qui fit reculer l’ombre de dix degrés sur les degrés d’Achaz, où elle était descendue.
\TextTitle{Imprudence d'Ezéchias\FTNTT{2 Ch. 32:25-31 ; Es. 39}}
\VS{12}En ce temps-là, Berodac-Baladan, fils de Baladan, roi de Babylone, envoya une lettre avec un présent à Ezéchias, parce qu’il avait appris la maladie d’Ezéchias.
\VS{13}Et Ezéchias, donna audience aux envoyés, et il leur montra tous les lieux où étaient ses objets les plus précieux, l’argent, l’or, les aromates l’huile précieuse, tout son arsenal, et tout ce qui se trouvait dans ses trésors : Il n’y eut rien qu’Ezéchias ne leur montra dans sa maison et dans tous ses domaines.
\VS{14}Esaïe, le prophète, vint ensuite auprès du roi Ezéchias, et lui dit : Qu’ont dit ces gens-là ? Et d’où sont-ils venus vers toi ? Ezéchias répondit : Ils sont venus d’un pays très éloigné, ils sont venus de Babylone.
\VS{15}Esaïe dit : Qu’ont-ils vu dans ta maison ? Et Ezéchias répondit : Ils ont vu tout ce qui est dans ma maison ; il n’y a rien dans mes trésors que je ne leur aie montré.
\VS{16}Alors Esaïe dit à Ezéchias : Ecoute la parole de Yahweh :
\VS{17}Voici, les jours viendront où tout ce qui est dans ta maison, et ce que tes pères ont amassé dans leurs trésors jusqu’à ce jour, sera emporté à Babylone ; il n’en restera rien dit Yahweh\FTNT{La déportation des juifs à Babylone : voir 2 R. 24 et 25.}.
\VS{18}On prendra même de tes fils\FTNT{2 R. 24:12 ; 2 Ch. 33:11 ; Da. 1.} qui seront sortis de toi, que tu auras engendrés, afin qu’ils soient eunuques dans le palais du roi de Babylone.
\VS{19}Ezéchias répondit à Esaïe : La parole de Yahweh, que tu as prononcée, est bonne. Et il ajouta : N’y aura-t-il pas paix et sécurité pendant mes jours ?
\TextTitle{Mort d'Ezéchias\FTNTT{2 Ch. 32:32-33}}
\VS{20}Le reste des actions d’Ezéchias, tous ses exploits, et comment il fit l’étang et l’aqueduc par lequel il fit entrer les eaux dans la ville, cela n’est-il pas écrit dans le livre des Chroniques des rois de Juda ?
\VS{21}Ezéchias se coucha avec ses pères. Et Manassé, son fils, régna à sa place.
\Chap{21}
\TextTitle{Manassé, roi de Juda ramène l'idolâtrie\FTNTT{2 Ch. 33:1-9}}
\VerseOne{}Manassé était âgé de douze ans, lorsqu’il commença à régner. Il régna cinquante-cinq ans à Jérusalem. Sa mère s’appelait Hephtsiba.
\VS{2}Il fit ce qui est mal aux yeux de Yahweh, selon les abominations des nations que Yahweh avait chassées devant les enfants d’Israël.
\VS{3}Car il rebâtit les hauts lieux qu’Ezéchias, son père, avait détruits, et redressa des autels à Baal, il fit une idole d’Astarté\FTNT{Voir commentaire en Jg. 2:13.}, comme avait fait Achab, roi d’Israël, il se prosterna devant toute l’armée des cieux, et les servit.
\VS{4}Il bâtit aussi des autels dans la maison de Yahweh, quoique Yahweh ait dit : C’est dans Jérusalem que j’établirai mon nom.
\VS{5}Il bâtit des autels à toute l’armée des cieux dans les deux parvis de la maison de Yahweh.
\VS{6}Il fit aussi passer son fils par le feu, il pratiquait l’astrologie et la divination, il établit des gens qui évoquaient les esprits des morts, et qui prédisaient l’avenir. Il fit de plus en plus ce qui est mal aux yeux de Yahweh pour l’irriter.
\VS{7}Il plaça aussi l’idole d’Astarté qu’il avait faite, dans la maison de laquelle Yahweh avait dit à David, et à Salomon, son fils : C’est dans cette maison, et c’est dans Jérusalem, que j’ai choisie parmi toutes les tribus d’Israël, que je veux à toujours établir mon nom.
\VS{8}Je ne ferai plus errer le pied d’Israël hors de cette terre que j’ai donnée à leurs pères, pourvu seulement qu’ils aient soin de mettre en pratique tout ce que je leur ai ordonné et toute la loi que Moïse, mon serviteur, leur a prescrite.
\VS{9}Mais ils n’obéirent point ; car Manassé les fit s’égarer, jusqu’à faire le mal plus que les nations que Yahweh avait exterminées devant les enfants d’Israël.
\TextTitle{Jugement de Dieu sur Manassé\FTNTT{2 Ch. 33:10}}
\VS{10}Alors Yahweh parla par ses serviteurs les prophètes, en disant :
\VS{11}Parce que Manassé, roi de Juda, a commis ces abominations, parce qu’il a fait pis que tout ce qu’avaient fait avant lui les Amoréens, et parce qu’il a aussi fait pécher Juda par ses idoles,
\VS{12}à cause de cela, Yahweh, le Dieu d’Israël, dit : Voici, je vais faire venir sur Jérusalem et sur Juda des malheurs qui étourdiront les oreilles de quiconque en entendra parler.
\VS{13}Car j’étendrai sur Jérusalem le cordeau de Samarie, et le niveau de la maison d’Achab, et je nettoierai Jérusalem comme un plat qu’on nettoie, et qu’on renverse sens dessus dessous après l’avoir nettoyé.
\VS{14}J’abandonnerai le reste de mon héritage, et je les livrerai entre les mains de leurs ennemis ; et ils seront le butin et la proie à tous leurs ennemis.
\VS{15}Parce qu’ils ont fait ce qui est mal à mes yeux, et qu’ils m’ont irrité depuis le jour où leurs pères sont sortis d’Egypte, jusqu’à ce jour.
\TextTitle{Meurtres de Manassé ; sa mort\FTNTT{2 Ch. 33:11-20}}
\VS{16}Manassé répandit aussi beaucoup de sang innocent, jusqu’à en remplir Jérusalem d’un bout à l’autre, outre son péché par lequel il fit commettre à Juda en faisant ce qui est mal aux yeux de Yahweh.
\VS{17}Le reste des actions de Manassé, tout ce qu’il a fait ; et les péchés auxquels il se livra, cela n’est-il pas écrit dans le livre des Chroniques des rois de Juda ?
\VS{18}Manassé se coucha avec ses pères, et il fut enseveli dans le jardin de sa maison, dans le jardin d’Uzza. Amon, son fils, régna à sa place.
\TextTitle{Amon, roi de Juda, perpétue l'idolâtrie\FTNTT{2 Ch. 33:20-23}}
\VS{19}Amon était âgé de vingt-deux ans lorsqu’il commença à régner. Il régna deux ans à Jérusalem. Sa mère s’appelait Meschullémeth, fille de Haruts, de Jotba.
\VS{20}Il fit ce qui est mal aux yeux de Yahweh, comme avait fait Manassé, son père.
\VS{21}Car il marcha dans toute la voie où avait marché son père, il servit les idoles que son père avait servies, et il se prosterna devant elles.
\VS{22}Il abandonna Yahweh, le Dieu de ses pères, et il ne marcha point dans la voie de Yahweh.
\TextTitle{Josias, roi de Juda\FTNTT{2 Ch. 33:24-25}}
\VS{23}Les serviteurs d’Amon firent une conspiration contre lui, et le tuèrent dans sa maison.
\VS{24}Mais le peuple du pays frappa tous ceux qui avaient conspiré contre le roi Amon ; et ils établirent Josias, son fils, roi à sa place.
\VS{25}Le reste des actions d’Amon, ce qu’il a fait, cela n’est-il pas écrit dans le livre des Chroniques des rois de Juda ?
\VS{26}On l’ensevelit dans son sépulcre, dans le jardin d’Uzza. Et Josias, son fils, régna à sa place.
\Chap{22}
\TextTitle{Josias restaure le culte et la crainte de Yahweh\FTNTT{2 Ch. 34:1-13}}
\VerseOne{}Josias était âgé de huit ans lorsqu’il commença à régner. Il régna trente et un ans à Jérusalem. Sa mère s’appelait Jedida, fille d’Adaja, de Botskath.
\VS{2}Il fit ce qui est droit aux yeux de Yahweh, et il marcha dans toute la voie de David, son père ; il ne s’en détourna ni à droite ni à gauche.
\VS{3}La dix-huitième année du roi Josias, le roi envoya dans la maison de Yahweh, Schaphan, le secrétaire, fils d’Atsalia, fils de Meschullam.
\VS{4}Il lui dit : Monte vers Hilkija, le souverain sacrificateur, et dis-lui d’amasser l’argent qui a été apporté dans la maison de Yahweh et que ceux qui ont la garde du seuil ont recueilli du peuple.
\VS{5}On remettra cet argent entre les mains de ceux qui sont chargés de faire exécuter l’ouvrage dans la maison de Yahweh. Et ils l’emploieront pour ceux qui travaillent dans la maison de Yahweh, pour réparer les brèches de la maison,
\VS{6}pour les charpentiers, les architectes et les maçons, pour les achats du bois et des pierres de taille pour réparer la maison.
\VS{7}Mais on ne leur demandera pas de comptes pour l’argent remis entre leurs mains, parce qu’ils agissent fidèlement.
\TextTitle{Le livre de la loi retrouvé\FTNTT{2 Ch. 34:14-21}}
\VS{8}Alors Hilkija, le souverain sacrificateur, dit à Schaphan, le secrétaire : J’ai trouvé le livre de la loi dans la maison de Yahweh. Et Hilkija donna ce livre à Schaphan qui le lut.
\VS{9}Schaphan, le secrétaire, alla vers le roi, et lui rapporta la chose, et dit : Tes serviteurs ont amassé l’argent qui se trouvait dans la maison, et l’ont remis entre les mains de ceux qui sont chargés de faire l’ouvrage dans la maison de Yahweh.
\VS{10}Schaphan, le secrétaire, dit aussi au roi : Le sacrificateur Hilkija m’a donné un livre. Et Schaphan le lut devant le roi.
\VS{11}Lorsque le roi eut entendu les paroles du livre de la loi, il déchira ses vêtements.
\VS{12}Il donna cet ordre au sacrificateur Hilkija, à Achikam, fils de Schaphan, à Acbor, fils de Michée, à Schaphan, le secrétaire, et à Asaja, serviteur du roi :
\VS{13}Allez, consultez Yahweh pour moi, pour le peuple, et pour tout Juda, au sujet des paroles de ce livre qui a été trouvé ; car grande est la colère de Yahweh, qui s’est enflammée contre nous, parce que nos pères n’ont point obéi aux paroles de ce livre et n’ont pas mis en pratique tout ce qui nous y est prescrit.
\VS{14}Le sacrificateur Hilkija, Achikam, Acbor, Schaphan et Asaja, allèrent auprès de la prophétesse Hulda, femme de Schallum, fils de Thikva, fils de Harhas, gardien des vêtements. Elle habitait dans un autre quartier de Jérusalem.
\TextTitle{Yahweh rassure Josias par la prophetesse Hulda\FTNTT{2 Ch. 34:22-28}}
\VS{15}Après qu’ils eurent parlé avec elle, elle leur répondit : Ainsi parle Yahweh, le Dieu d’Israël : Dites à l’homme qui vous a envoyés vers moi :
\VS{16}Ainsi parle Yahweh : Voici, je vais faire venir le malheur sur cette ville, et sur ses habitants, selon toutes les paroles du livre que le roi de Juda a lu.
\VS{17}Parce qu’ils m’ont abandonné, et qu’ils ont offert des parfums à d’autres dieux, pour m’irriter par toutes les actions de leurs mains, ma colère s’est enflammée contre cette ville, et elle ne s’éteindra point.
\VS{18}Mais quant au roi de Juda qui vous a envoyés pour consulter Yahweh vous lui direz : Ainsi parle Yahweh, le Dieu d’Israël, au sujet des paroles que tu as entendues :
\VS{19}Parce que ton cœur a été touché, et que tu t’es humilié devant Yahweh en entendant ce que j’ai prononcé contre cette ville et contre ses habitants, qui seront un objet d’épouvante et de malédiction, et parce que tu as déchiré tes vêtements, et que tu as pleuré devant moi, je t’ai exaucé, dit Yahweh.
\VS{20}C’est pourquoi voici, je vais te recueillir auprès de tes pères, et tu seras recueilli dans ton sépulcre en paix, et tes yeux ne verront point tout ce mal que je vais faire venir sur cette ville. Ils rapportèrent toutes ces paroles au roi.
\Chap{23}
\TextTitle{Josias fait la lecture de la loi et fait alliance avec Yahweh\FTNTT{2 Ch. 34:29-32}}
\VerseOne{}Alors le roi Josias fit assembler auprès de lui tous les anciens de Juda et de Jérusalem.
\VS{2}Le roi monta à la maison de Yahweh, avec tous les hommes de Juda, tous les habitants de Jérusalem, les sacrificateurs, les prophètes, et tout le peuple, depuis le plus petit jusqu’au plus grand. Il lut devant eux toutes les paroles du livre de l’alliance, qui avait été trouvé dans la maison de Yahweh.
\VS{3}Le roi se tenait sur l’estrade, et il traita alliance devant Yahweh, s’engageant à suivre Yahweh, à observer ses ordonnances, ses préceptes et ses lois, de tout son cœur, à persévérer dans les paroles de cette alliance, écrites dans ce livre. Et tout le peuple entra dans cette alliance.
\TextTitle{Josias assinit le pays de toutes les abominations\FTNTT{2 Ch. 34:33}}
\VS{4}Alors le roi donna cet ordre à Hilkija, le souverain sacrificateur, aux sacrificateurs du second ordre, et à ceux qui gardaient le seuil, de sortir hors du temple de Yahweh tous les ustensiles qui avaient été faits pour Baal\FTNT{Voir commentaire en Jg. 2:12.}, pour Astarté\FTNT{Voir commentaire en Jg. 2:13}, et pour toute l’armée des cieux ; et il les brûla hors de Jérusalem, dans les champs de Cédron, et en fit porter la poussière à Béthel.
\VS{5}Il chassa les prêtres des idoles, que les rois de Juda avaient établis pour brûler des parfums sur les hauts lieux, dans les villes de Juda, et aux environs de Jérusalem, et ceux qui offraient des parfums à Baal, au soleil, à la lune, au zodiaque et à toute l’armée des cieux.
\VS{6}Il sortit de la maison Yahweh l’idole d’Astarté, qu’il transporta hors de Jérusalem vers le torrent de Cédron ; il la brûla au torrent de Cédron et la réduisit en poudre, et il en jeta la poussière sur le sépulcre des fils du peuple.
\VS{7}Ensuite, il démolit les maisons des prostitués qui étaient dans la maison de Yahweh, où les femmes tissaient des tentes pour Astarté.
\VS{8}Il fit venir des villes de Juda tous les sacrificateurs ; il profana les hauts lieux où les sacrificateurs brûlaient des parfums, depuis Guéba jusqu’à Beer-Schéba ; il renversa les hauts lieux des portes, celui qui était à l’entrée de la porte de Josué, chef de la ville, et celui qui était à gauche de la porte de la ville.
\VS{9}Toutefois, les sacrificateurs des hauts lieux ne montaient pas à l’autel de Yahweh à Jérusalem, mais ils mangeaient des pains sans levain parmi leurs frères.
\VS{10}Le roi profana aussi Topheth, dans la vallée des fils de Hinnom, afin que personne ne fasse plus passer son fils ou sa fille par le feu, en l’honneur de Moloc\FTNT{Lé. 20:2-3}.
\VS{11}Il fit disparaître de l’entrée de la maison de Yahweh les chevaux que les rois de Juda avaient consacré au soleil, près de la chambre de l’eunuque Nethan-Mélec, situé à Parvarim, et il brûla au feu les chars du soleil.
\VS{12}Le roi démolit les autels qui étaient sur le toit de la chambre haute d’Achaz, que les rois de Juda avaient fait, et les autels que Manassé avait faits dans les deux parvis de la maison de Yahweh ; après les avoir brisés et enlevés de là, il en jeta la poussière dans le torrent de Cédron.
\VS{13}Le roi profana aussi les hauts lieux qui étaient en face de Jérusalem, sur la droite de la montagne de perdition, que Salomon, roi d’Israël, avait bâtis à Astarté, l’abomination des Sidoniens, à Kemosch, l’abomination des Moabites, et à Milcom, l’abomination des fils d’Ammon.
\VS{14}Il brisa aussi les statues, et abattit les idoles d’Astarté, et il remplit d’ossements d’hommes les lieux où elles étaient.
\VS{15}Il renversa l’autel qui était à Béthel, et le haut lieu qu’avait fait Jéroboam, fils de Nebath, qui avait fait pécher Israël ; il brûla le haut lieu et le réduisit en poudre, et il brûla l’idole d’Astarté.
\VS{16}Josias s’étant tourné et ayant vu les sépulcres qui étaient là dans la montagne, envoya prendre les ossements des sépulcres, et il les brûla sur l’autel et le profana, selon la parole de Yahweh prononcée à haute voix par l’homme de Dieu.
\VS{17}Le roi dit : Quel est ce monument que je vois ? Et les hommes de la ville lui répondirent : C’est le sépulcre de l’homme de Dieu qui est venu de Juda qui a crié contre l’autel de Béthel ces choses que tu as accomplies.
\VS{18}Et il dit : Laissez-le ; que personne ne remue ses os ! Ils conservèrent ainsi ses os, avec les os du prophète qui était venu de Samarie.
\VS{19}Josias fit encore disparaître toutes les maisons des hauts lieux, qui étaient dans les villes de Samarie, et qu’avaient faites les rois d’Israël pour irriter Yahweh ; et il fit à leur égard entièrement comme il avait fait à Béthel.
\VS{20}Il immola sur les autels tous les sacrificateurs des hauts lieux qui étaient là, et il y brûla des ossements d’hommes. Puis il retourna à Jérusalem.
\TextTitle{Rétablissement de la Pâque\FTNTT{2 Ch. 35:1-19}}
\VS{21}Alors le roi donna cet ordre à tout le peuple, en disant : Célébrez la Pâque en l’honneur de Yahweh, votre Dieu, comme il est écrit dans le livre de cette alliance\FTNT{Jésus-Christ est notre Pâque. Voir Ex. 12   et 1 Co. 5:7.}.
\VS{22}Aucune Pâque pareille à celle-ci n’avait été célébrée depuis le temps où les juges jugeaient Israël et pendant tous les jours des rois d’Israël et des rois de Juda.
\VS{23}Ce fut la dix-huitième année du roi Josias qu’on célébra cette Pâque en l’honneur de Yahweh à Jérusalem.
\VS{24}Josias extermina aussi ceux qui évoquaient les esprits des morts et les devins, les théraphim, les idoles, et toutes les abominations qui se voyaient dans le pays de Juda et à Jérusalem, afin de mettre en pratique les paroles de la loi, écrites dans le livre que Hilkija, le sacrificateur, avait trouvé dans la maison de Yahweh.
\VS{25}Avant Josias, il n’y eut point de roi qui, comme lui, revienne à Yahweh de tout son cœur, de toute son âme, et de toute sa force, selon toute la loi de Moïse ; et après lui, il n’en a point paru de semblable.
\VS{26}Toutefois, Yahweh ne se détourna point de l’ardeur de sa grande colère dont il était enflammé contre Juda, à cause de tout ce que Manassé avait fait pour l’irriter.
\VS{27}Et Yahweh dit : J’ôterai Juda de devant ma face, comme j’ai ôté Israël, et je rejetterai cette ville de Jérusalem que j’avais choisie, et la maison de laquelle j’avais dit : Là sera mon nom.
\VS{28}Le reste des actions de Josias, tout ce qu’il a fait, cela n’est-il pas écrit dans le livre des Chroniques des rois de Juda ?
\TextTitle{Mort de Josias\FTNTT{2 Ch. 35:1-19}}
\VS{29}De son temps, Pharaon Néco, roi d’Egypte, monta contre le roi d’Assyrie, vers le fleuve d’Euphrate. Le roi Josias s’en alla au-devant de lui ; mais dès que pharaon le vit, il le tua à Meguiddo.
\VS{30}Ses serviteurs l’emportèrent mort sur un char ; ils l’amenèrent de Meguiddo à Jérusalem, et l’ensevelirent dans son sépulcre. Et le peuple du pays prit Joachaz, fils de Josias, ils l’oignirent, et l’établirent roi à la place de son père.
\TextTitle{Pharaon Néco détrône Joachaz et établit Jojakim roi de Juda\FTNTT{2 Ch. 36:1-5}}
\VS{31}Joachaz était âgé de vingt-trois ans, lorsqu’il commença à régner. Il régna trois mois à Jérusalem. Sa mère s’appelait Hamuthal, fille de Jérémie, de Libna.
\VS{32}Il fit ce qui est mal aux yeux de Yahweh, entièrement comme avaient fait ses pères.
\VS{33}Et pharaon Néco l’emprisonna à Ribla, dans le pays de Hamath, afin qu’il ne règne plus à Jérusalem ; et il imposa sur le pays un tribut de cent talents d’argent et d’un talent d’or.
\VS{34}Puis pharaon Néco établit roi Eliakim, fils de Josias, à la place de Josias, son père, et il changea son nom en celui de Jojakim. Il prit Joachaz, qui alla en Egypte, où il mourut.
\VS{35}Jojakim donna cet argent et cet or à pharaon ; mais il taxa le pays pour fournir cet argent, selon l’ordre de pharaon ; il détermina la part de chacun et exigea du peuple du pays l’argent et l’or qu’il devait livrer à pharaon Néco.
\VS{36}Jojakim était âgé de vingt-cinq ans lorsqu’il commença à régner. Il régna onze ans à Jérusalem. Sa mère s’appelait Zebudda, fille de Pedaja, de Ruma.
\VS{37}Il fit ce qui est mal aux yeux de Yahweh, entièrement comme avaient fait ses pères.
\Chap{24}
\TextTitle{Première vague de la déportation à Babylone\FTNTT{2 Ch. 36:6-7 ; Da. 1:1}}
\VerseOne{}De son temps, Nebucadnetsar, roi de Babylone, monta contre Jojakim, et Jojakim lui fut asservi pendant trois ans ; mais il se révolta de nouveau contre lui.
\VS{2}Alors Yahweh envoya contre Jojakim des troupes de Chaldéens, des armées de Syriens, des troupes de Moabites, et des troupes des fils d’Ammon ; il les envoya contre Juda, pour le détruire, selon la parole que Yahweh avait prononcée par ses serviteurs les prophètes.
\VS{3}Cela arriva uniquement sur l’ordre de Yahweh, qui voulait ôter Juda de devant sa face, à cause de tous les péchés commis par Manassé,
\VS{4}et à cause aussi du sang innocent qu’il avait répandu, et dont il avait rempli Jérusalem. Aussi Yahweh ne voulut-il point lui pardonner.
\TextTitle{Jojakin, roi de Juda, infidèle à Yahweh\FTNTT{2 Ch. 36:1-5}}
\VS{5}Le reste des actions de Jojakim, et tout ce qu’il a fait, cela n’est-il pas écrit dans le livre des Chroniques des rois de Juda ?
\VS{6}Ainsi Jojakim se coucha avec ses pères. Et Jojakin, son fils, régna à sa place.
\VS{7}Le roi d’Egypte ne sortit plus de son pays, parce que le roi de Babylone avait pris tout ce qui était au roi d’Egypte, depuis le torrent d’Egypte jusqu’au fleuve d’Euphrate.
\VS{8}Jojakin était âgé de dix-huit ans lorsqu’il commença à régner. Il régna trois mois à Jérusalem. Sa mère s’appelait Nehuschtha, fille d’Elnathan, de Jérusalem.
\VS{9}Il fit ce qui est mal aux yeux de Yahweh, entièrement comme avait fait son père.
\TextTitle{Deuxième vague de la déportation à Babylone\FTNTT{2 Ch. 36:10}}
\VS{10}En ce temps-là, les serviteurs de Nebucadnetsar, roi de Babylone, montèrent contre Jérusalem, et la ville fut assiégée.
\VS{11}Nebucadnetsar, roi de Babylone, arriva devant la ville, pendant que ses serviteurs l’assiégeaient.
\VS{12}Alors Jojakin, roi de Juda, se rendit vers le roi de Babylone, avec sa mère, ses serviteurs, ses chefs, et ses eunuques. Et le roi de Babylone le fit prisonnier, la huitième année de son règne.
\VS{13}Il emporta de là, tous les trésors de la maison de Yahweh et les trésors de la maison royale ; et il mit en pièces tous les ustensiles d’or que Salomon, roi d’Israël, avait faits pour le temple de Yahweh, comme Yahweh l’avait ordonné.
\VS{14}Il emmena en captivité tout Jérusalem\FTNT{Première déportation : 2 R. 24:1-4 et 2 Ch. 36:6-7. La première déportation eut lieu en 597 av. J.-C. pendant le règne de Jojakim, roi de Juda. Les premiers exilés furent installés dans la région du fleuve Kebar (Ez. 1:1-3), un canal de 90 km de long reliant l’Euphrate au nord de Babylone au même fleuve au sud d’Ur en Chaldée. Jérémie savait que leur séjour à l’étranger serait long. Il avait prophétisé qu’il durerait soixante-dix ans (Jé. 25 :1 ; Jé. 25:11-12) et leur conseilla de se construire des maisons, de cultiver des jardins et de se multiplier (Jé. 29). Daniel et ses compagnons furent déportés à Babylone lors de la première déportation (Da. 1). Daniel fut déporté environ huit ans avant Ezéchiel}, à savoir, tous les chefs, et tous les vaillants hommes de guerre, au nombre de dix mille captifs, avec les charpentiers et les serruriers, de sorte qu’il ne resta plus que le peuple pauvre du pays.
\VS{15}Ainsi il transporta Jojakin à Babylone, avec la mère du roi, les femmes du roi et ses eunuques. Il emmena captifs à Babylone tous les grands du pays, de Jérusalem à Babylone,
\VS{16}avec tous les guerriers au nombre de sept mille, les charpentiers, les serruriers au nombre de mille, tous hommes vaillants et propres à la guerre. Le roi de Babylone les emmena captifs à Babylone.
\TextTitle{Révolte de Sédécias, roi de Juda, contre le roi de Babylone\FTNTT{2 Ch. 36:10-16}}
\VS{17}Et le roi de Babylone établit roi, à la place de Jojakin, Matthania, son oncle, et il changea son nom en celui de Sédécias.
\VS{18}Sédécias était âgé de vingt et un ans lorsqu’il commença à régner. Il régna onze ans à Jérusalem. Sa mère s’appelait Hamuthal, fille de Jérémie, de Libna.
\VS{19}Il fit ce qui est mal aux yeux de Yahweh, entièrement comme avait fait Jojakim.
\VS{20}Cela arriva à cause de la colère de Yahweh contre Jérusalem et contre Juda, qu’il voulait rejeter de devant sa face. Et Sédécias se révolta contre le roi de Babylone.
\Chap{25}
\TextTitle{Siège de Jérusalem par Babylone\FTNTT{2 Ch. 36:1-4 ; Jé. 39:8-10}}
\VerseOne{}La neuvième année du règne de Sédécias, le dixième jour du dixième mois, Nebucadnetsar\FTNT{Jérusalem fut assiégée pendant deux ans. Lors de ce siège, des femmes juives faisaient cuire leurs enfants pour les consommer (La. 2:20 ; La. 4:10). }, roi de Babylone, vint avec toute son armée contre Jérusalem ; il campa devant elle, et éleva des retranchements tout autour.
\VS{2}La ville fut assiégée jusqu’à la onzième année du roi Sédécias.
\VS{3}Le neuvième jour du mois, la famine\FTNT{La. 4:10.} augmenta dans la ville, de sorte qu’il n’y avait pas de pain pour le peuple du pays.
\VS{4}Alors la brèche fut faite à la ville ; et tous les gens de guerre s’enfuirent de nuit par le chemin de la porte entre les deux murailles près du jardin du roi, pendant que les Chaldéens environnaient la ville. Les fuyards et le roi prirent le chemin de la plaine.
\VS{5}Mais l’armée des Chaldéens poursuivit le roi et l’atteignit dans les plaines de Jéricho, et toute son armée se dispersa loin de lui.
\VS{6}Ils saisirent donc le roi, et le firent monter vers le roi de Babylone à Ribla ; et l’on prononça contre lui un jugement.
\VS{7}Et on égorgea les fils de Sédécias en sa présence ; puis on creva les yeux à Sédécias, et on le lia de doubles chaînes d’airain, et on le mena à Babylone.
\TextTitle{Jérusalem détruite ; troisième vague de la déportation à Babylone\FTNTT{2 Ch. 36:17-21}}
\VS{8}Le septième jour du cinquième mois, c’était la dix-neuvième année du roi Nebucadnetsar, roi de Babylone, Nebuzaradan, chef des gardes, serviteur du roi de Babylone,\FTNT{Troisième déportation : Le temple fut brûlé, la ville de Jérusalem fut totalement rasée et ses habitants furent déportés (De. 28:49-68). Contrairement à ce que l’on pense, il y a eu d’autres déportations. Voir Jé. 52.} entra dans Jérusalem.
\VS{9}Il brûla la maison de Yahweh, la maison royale, et toutes les maisons de Jérusalem ; il brûla par le feu toutes les grandes maisons.
\VS{10}Toute l’armée des Chaldéens, qui était avec le chef des gardes, démolit les murailles qui entouraient Jérusalem.
\VS{11}Et Nebuzaradan, chef des gardes, emmena captifs le reste du peuple, ceux qui étaient restés dans la ville, ceux qui s’étaient rendus au roi de Babylone, et le reste de la multitude.
\VS{12}Cependant le chef des gardes laissa quelques-uns des plus pauvres du pays comme vignerons et comme laboureurs.
\VS{13}Les Chaldéens brisèrent les colonnes d’airain qui étaient dans la maison de Yahweh, les bases, la mer d’airain qui était dans la maison de Yahweh, et ils en emportèrent l’airain à Babylone.
\VS{14}Ils prirent aussi les cendriers, les pelles, les couteaux, les tasses, et tous les ustensiles d’airain avec lesquels on faisait le service.
\VS{15}Le chef des gardes emporta aussi les encensoirs et les coupes, ce qui était d’or et ce qui était d’argent.
\VS{16}Les deux colonnes, la mer, et les bases, que Salomon avait faits pour la maison de Yahweh, tous ces ustensiles d’airain avaient un poids inconnu.
\VS{17}La hauteur d’une colonne était de dix-huit coudées, et il y avait au-dessus un chapiteau d’airain dont la hauteur était de trois coudées ; autour du chapiteau il y avait un treillis et des grenades, le tout d’airain ; il en était de même pour la seconde colonne avec le treillis.
\VS{18}Le chef des gardes emmena aussi Seraja, le souverain sacrificateur, et Sophonie, le second sacrificateur, et les trois gardiens du seuil.
\VS{19}Et dans la ville il prit un eunuque qui avait sous son commandement des hommes de guerre, cinq hommes de ceux qui voyaient la face du roi et qui furent trouvés dans la ville, il prit aussi le secrétaire du chef de l’armée qui était chargé d’enrôler le peuple du pays, et soixante hommes du peuple du pays qui se trouvaient dans la ville.
\VS{20}Nebuzaradan, chef des gardes, les prit, et les conduisit vers le roi de Babylone à Ribla.
\VS{21}Le roi de Babylone les frappa, et les fit mourir à Ribla, dans le pays de Hamath. Ainsi Juda fut transporté captif hors de sa terre.
\TextTitle{Guedalia, gouverneur du reste de Juda}
\VS{22}Nebucadnetsar, roi de Babylone, plaça le reste du peuple, qu’il laissa dans le pays de Juda, sous le commandement de Guedalia, fils d’Achikam, fils de Schaphan.
\VS{23}Lorsque tous les chefs des troupes, et leurs hommes, eurent appris que le roi de Babylone avait établi Guedalia pour gouverneur, ils allèrent trouver Guedalia à Mitspa, à savoir Ismaël, fils de Nethania, Jochanan, fils de Karéach, Seraja, fils de Thanhumeth, de Nethopha, Jaazania, fils du Maacathien, eux et leurs hommes.
\VS{24}Guedalia leur jura, à eux et à leurs hommes, et leur dit : Ne craignez pas d’être serviteurs des Chaldéens ; demeurez dans le pays, et servez le roi de Babylone, et vous vous en trouverez bien.
\TextTitle{Le peuple se réfugie en Egypte}
\VS{25}Mais au septième mois, Ismaël, fils de Nethania, fils d’Elischama, qui était de race royale, vint, accompagné de dix hommes, et ils frappèrent mortellement Guedalia, ainsi que les Juifs et les Chaldéens qui étaient avec lui à Mitspa.
\VS{26}Alors tout le peuple, depuis le plus petit jusqu’au plus grand, avec les chefs des troupes, se levèrent, et s’en allèrent en Egypte, parce qu’ils avaient peur des Chaldéens.
\TextTitle{Libération de Jojakin\FTNTT{Jé. 52:31-34}}
\VS{27}La trente-septième année de la captivité de Jojakin, roi de Juda, le vingt-septième jour du douzième mois, Evil-Merodac, roi de Babylone, dans la première année de son règne, releva la tête de Jojakin, roi de Juda, et le tira de prison.
\VS{28}Il lui parla avec bonté, et il mit son trône au-dessus du trône des rois qui étaient avec lui à Babylone.
\VS{29}Il lui fit changer ses vêtements de prison, et Jojakin mangea du pain tout le temps de sa vie en sa présence.
\VS{30}Et quant à son entretien, un entretien perpétuel, lui fut accordé par le roi pour chaque jour, tous les jours de sa vie.
\PPE{}
\end{multicols}
