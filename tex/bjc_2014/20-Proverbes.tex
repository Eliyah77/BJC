\ShortTitle{Proverbes}\BookTitle{Proverbes}\BFont
\noindent\hrulefill
{\footnotesize
\textit{
\bigskip
{\centering{}
\\Auteurs : Salomon, Agur et Lemuel
\\(Heb. : Mishlei)
\\Signification : Paraboles
\\Thème : La sagesse
\\Date de rédaction : 10\up{ème} siècle av J.-C.\\}
}
%\bigskip
\textit{
\\Le mot « proverbe » désigne un genre littéraire appliqué à une sentence, une énigme, une comparaison, un oracle, une
parabole ou une parole de sagesse. Le livre des proverbes est donc un recueil de sentences dont la majeure partie
est attribuée à Salomon. Véritable collection de maximes morales et spirituelles, la sagesse, la crainte de Dieu et la
tempérance en sont les thèmes principaux.
%\bigskip
\\Ce livre met en évidence l'opposition entre la voie du méchant et celle du juste, entre la femme étrangère et la femme
vertueuse, entre l'orgueil et l'humilité, entre la sagesse et la folie, entre le chemin de la vie et celui de la mort. Comme il était coutume au Moyen-Orient, ces écrits s'adressaient particulièrement aux jeunes gens en vue de leur instruction.\bigskip
}
}
\par\nobreak\noindent\hrulefill
\begin{multicols}{2}
\Chap{1}
\TextTitle{[But du livre : connaître la sagesse}
\VerseOne{}Les Proverbes de Salomon, fils de David et roi d'Israël.
\VS{2}Pour connaître la sagesse et l'instruction, pour discerner les paroles d'intelligence ;
\VS{3}pour recevoir une leçon de bon sens, de justice, de jugement et d'équité.
\VS{4}Pour donner du discernement aux simples, aux jeunes gens de la connaissance et de la réflexion.
\VS{5}Le sage écoutera, et il augmentera son savoir, et l'homme intelligent acquerra de la prudence ;
\VS{6}afin d'entendre les paraboles et les énigmes ; les discours des sages et leurs énigmes.
\TextTitle{Le fondement de la sagesse : la crainte de Dieu}
\VS{7}La crainte de Yahweh\FTNT{Pr. 8:13.} est la principale de la science ; mais les fous méprisent la sagesse et l'instruction.
\VS{8}Mon fils, écoute l'instruction de ton père, et n'abandonne pas l'enseignement\FTNT{Loi vient de Torah (instruction, enseignement, direction etc.).} de ta mère.
\VS{9}Car se sont des grâces enfilées ensemble autour de ta tête, et des colliers autour de ton cou.
\VS{10}Mon fils, si les pécheurs veulent t'attirer, ne t'y accorde pas.
\VS{11}S'ils disent : Viens avec nous, dressons des embûches pour tuer; épions secrètement l'innocent, quoiqu'il ne nous en ai point donné de sujet aucune raison.
\VS{12}Engloutissons-les tout vifs, comme le scheol ; et tout entiers, comme ceux qui descendent dans la fosse ;
\VS{13}nous trouverons toutes sortes de biens précieux, nous remplirons nos maisons de butin ;
\VS{14}tu auras ta part avec nous, il n'y aura qu'une bourse pour nous tous.
\VS{15}Mon fils, ne te mets point en chemin avec eux ; retires ton pied de leur sentier ;
\VS{16}parce que leurs pieds courent au mal, et se hâtent pour répandre le sang\FTNT{Es. 59:7.}.
\VS{17}Car c'est en vain qu'on jette le filet devant les yeux de tout Baal ailé\FTNT{Ec. 10:20.} ;
\VS{18}ainsi ceux-ci dressent des embûches contre le sang de ceux-là et épient secrètement leurs vies.
\VS{19}Tel est le train de tout homme convoiteux de gain déshonnête, qui ôte la vie à ceux qui y sont adonnés.
\TextTitle{La sagesse crie}
\VS{20}La souveraine sagesse crie hautement au-dehors, elle fait retentir sa voix dans les rues.
\VS{21}Elle crie dans les carrefours, là où on fait le plus de bruit, aux entrées des portes, elle prononce ses paroles dans la ville :
\VS{22}Stupides, dit-elle, jusqu'à quand aimerez-vous la stupidité ? Et jusqu'à quand les moqueurs prendront-ils plaisir à la moquerie, et les haïront-ils la connaissance ?
\VS{23}Etant repris par moi, convertissez-vous ; voici, je vous donnerai de mon Esprit en abondance, et je vous ferai connaître mes paroles.
\VS{24}Parce que je crie, et que vous refusez d'entendre ; parce que j'étends ma main, et que personne n'y prend garde ;
\VS{25}et parce que vous rejetez tout mon conseil, et que vous n'avez accepté je vous reprenne ;
\VS{26}moi aussi je rirai quand vous serez dans le malheur, je me moquerai quand la terreur viendra sur vous.
\VS{27}Quand votre effroi surviendra comme une ruine, et que votre calamité viendra comme un tourbillon; vous enveloppera comme un tourbillon ; quand la détresse et l'angoisse viendront  sur vous ;
\VS{28}alors on criera vers moi, mais je ne répondrai point ; on me cherchera de grand matin, mais on ne me trouvera pas\FTNT{De. 31:18 ; Job. 35:12.}.
\VS{29}Parce qu'ils auront haï la connaissance, et qu'ils n'auront point choisi la crainte de Yahweh.
\VS{30}Ils n'ont point aimé mon conseil ; ils ont rejeté toutes mes réprimandes.
\VS{31}Qu'ils mangent donc le fruit de leur voie, et qu'ils se rassasient de leurs conseils.
\VS{32}Car l'égarement des sots les tue, et la prospérité des insensés les perd.
\VS{33}Mais celui qui m'écoute habitera en sécurité et sera tranquille, sans être effrayé d'aucun mal.
\Chap{2}
\TextTitle{La sagesse nous libère du mal}
\VerseOne{}Mon fils, si tu reçois mes paroles, et que tu gardes précieusement en toi mes commandements,
\VS{2}si tu rends ton oreille attentive à la sagesse, et que tu inclines ton cœur à l'intelligence ;
\VS{3}si tu appelles à toi la sagesse, et que tu adresses ta voix à l'intelligence,
\VS{4}si tu la cherches comme de l'argent, et si tu la recherches soigneusement comme des trésors,
\VS{5}alors tu connaîtras la crainte de Yahweh, et tu trouveras la connaissance de Dieu.
\VS{6}Car Yahweh donne la sagesse, et de sa bouche procède la connaissance et l'intelligence.
\VS{7}Il réserve le salut pour ceux qui sont droits, et il est le bouclier de ceux qui marchent dans l'intégrité,
\VS{8}pour garder les sentiers de la justice ; il gardera la voie de ses bien-aimés.
\VS{9}Alors tu comprendras la justice, le jugement, l'équité, et tout bon chemin.
\VS{10}Si la sagesse vient dans ton coeur, et si la connaissance est agréable à ton âme ;
\VS{11}la réflexion veillera sur toi, et l'intelligence te gardera,
\VS{12}pour te délivrer du mauvais chemin, et de l'homme qui tient de mauvais discours ,
\VS{13}de ceux qui abandonnent les voies de la droiture pour marcher dans les chemins ténébreux,
\VS{14}qui sont joyeux de mal faire, et qui se réjouissent dans la perversité des méchants.
\VS{15}Eux dont les sentiers sont tortueux, et qui dans leur conduite vont de travers.
\VS{16}Afin qu'il te délivre de la femme étrangère\FTNT{La femme étrangère est la prostituée ou l'esprit de Jézabel qui séduit les hommes. Voir Pr. 6:24 ; Pr. 7:5.}, et de la femme d'autrui, dont les paroles sont flatteuses ;
\VS{17}qui abandonne l'ami de sa jeunesse et qui oublie l'alliance de son Dieu.
\VS{18}Car sa maison penche vers la mort, et son chemin mène vers les morts.
\VS{19}Pas un de ceux qui vont vers elle n'en retourne, ni ne reprend les sentiers de la vie.
\VS{20}Ainsi tu marcheras dans la voie des gens de bien, et tu garderas les sentiers des justes.
\VS{21}Car ceux qui sont droits habiteront la terre, les hommes intègres y demeureront.
\VS{22}Mais les méchants seront retranchés de la terre, et ceux qui agissent perfidement seront arrachés.
\Chap{3}
\TextTitle{La sagesse bénit et protège}
\VerseOne{}Mon fils, ne mets pas en oubli mon enseignement, et que ton cœur garde mes commandements.
\VS{2}Car ils t'apportent de longs jours et des années de vie et de paix.
\VS{3}Que la bonté et la vérité ne t'abandonnent pas : Lie-les à ton cou, et écris-les sur la table de ton coeur ;
\VS{4}et tu trouveras la grâce et la prudence au yeux de Dieu et des hommes.
\VS{5}Confie-toi de tout ton coeur en Yahweh et ne t'appuie point sur ton intelligence.
\VS{6}Considère-le dans toutes tes voies et il dirigera tes sentiers.
\VS{7}Ne sois point sage à tes yeux ; crains Yahweh, et détourne-toi du mal.
\VS{8}Ce sera la guérison de ton nombril et un rafraîchissement pour tes os.
\VS{9}Honore Yahweh avec tes biens et les prémices de tout ton revenu\FTNT{De. 12:6.} :
\VS{10}Alors tes greniers seront remplis d'abondance, et tes cuves regorgeront de vin nouveau.
\VS{11}Mon fils, ne rebute pas l'instruction de Yahweh, et ne te fâche pas de ce qu'il te reprend.
\VS{12}Car Yahweh châtie\FTNT{Hé. 12:4-11.} celui qu'il aime, comme un père le fils auquel il prend plaisir.
\VS{13}Heureux l'homme qui a trouvé la sagesse, et l'homme qui possède l'intelligence !
\VS{14}Car le trafic qu'on peut faire d'elle est meilleur que le trafic l'argent; et le profit qu'on en tire est meilleur que l'or fin.
\VS{15}Elle est plus précieuse que les perles, et toutes tes choses désirables ne la valent point.
\VS{16}Il y a de longs jours dans sa main droite, des richesses et de la gloire en sa gauche.
\VS{17}Ses voies sont des voies agréables, et tous ses sentiers ne sont que paix.
\VS{18}Elle est l'arbre de vie pour ceux qui l'embrassent ; et tous ceux qui la tiennent sont heureux\FTNT{Ge. 2:9 ; Ap. 22:2.}.
\VS{19}Yahweh a fondé la terre par la sagesse, il a disposé les cieux par l'intelligence.
\VS{20}C'est par sa science que les abîmes se sont ouverts, et que les nuages distillent la rosée.
\VS{21}Mon fils, que ces enseignements ne s'écartent point de devant tes yeux ; garde la sagesse et la réflexion :
\VS{22}Elles seront la vie de ton âme et l'ornement de ton cou.
\VS{23}Alors tu marcheras avec assurance dans ton chemin, et ton pied ne bronchera pas.
\VS{24}Si tu te couches, tu seras sans crainte, et quand tu seras couché ton sommeil sera doux.
\VS{25}Ne crains ni une terreur soudaine, ni la ruine des méchants, quand elle arrivera.
\VS{26}Car Yahweh sera ton assurance, et il gardera ton pied de toute embûche.
\VS{27}Ne retiens pas le bien à ceux à qui il est dû, quand il est au pouvoir de ta main de le faire\FTNT{Ga. 6:10.}.
\VS{28}Ne dis pas à ton prochain : Va, et reviens, demain je te donnerai ! Quand tu as de quoi donner.
\VS{29}Ne médite pas le mal contre ton prochain, lorsqu'il demeure tranquillement près de toi.
\VS{30}Ne conteste pas sans motif avec quelqu'un, à moins qu'il ne t'ait causé quelque tort\FTNT{Ro. 12:18.}.
\VS{31}Ne porte pas envie à l'homme violent, et ne choisis aucune de ses voies.
\VS{32}Car celui qui va de travers est en abomination à Yahweh ; mais son intimité est pour ceux qui sont justes.
\VS{33}La malédiction de Yahweh est dans la maison du méchant ; mais il bénit la demeure des justes.
\VS{34}Certes il se moque des moqueurs, mais il fait grâce à ceux qui s'humilient.
\VS{35}Les sages hériteront la gloire ; mais la honte élève des insensés.
\Chap{4}
\TextTitle{Instructions et conseils d'un père}
\VerseOne{}Ecoutez, mes fils, l'instruction du père, et soyez attentifs pour connaître l'intelligence.
\VS{2}Car je vous donne une bonne doctrine, ne rejetez donc pas mon enseignement.
\VS{3}J'ai été un fils pour mon père. Un fils tendre et unique auprès de ma mère.
\VS{4}Il m'a enseigné, et m'a dit : Que ton coeur retienne mes paroles ; garde mes commandements et tu vivras.
\VS{5}Acquiers la sagesse, acquiers l'intelligence ; n'oublie pas les paroles de ma bouche, et ne t'en détourne pas.
\VS{6}Ne l'abandonne point, et elle te gardera ; aime-la, et elle te protégera.
\VS{7}La principale chose c'est la sagesse ; donc acquiers la sagesse ; et sur toutes tes acuisitions, acquiers la prudence.
\VS{8}Exalte-la, et elle t'élèvera ; elle te glorifiera quand tu l'auras embrassée.
\VS{9}Elle posera sur ta tête une couronne de grâce, et elle t'ornera d'un magnifique diadème.
\VS{10}Ecoute, mon fils, et reçois mes paroles, ainsi les années de ta vie te seront multipliées.
\VS{11}Je t'ai enseigné le chemin de la sagesse, et je t'ai conduit dans les sentiers de la droiture\FTNT{Ps. 23:3.}.
\VS{12}Quand tu y marcheras, ton pas ne sera pas gêné ; et si tu cours, tu ne chancelleras pas\FTNT{Ps. 121:3.}.
\VS{13}Embrasse l'instruction, ne la lâche pas ; garde-la ; car elle est ta vie.
\VS{14}N'entre pas dans le sentier des méchants, et ne marche pas dans la voie des hommes mauvais.
\VS{15}Détourne-t'en, ne passe pas par là, détourne-t'en, et passe outre.
\VS{16}Car ils ne dormiraient pas, s'ils n'avaient fait quelque mal, et le sommeil leur serait ôté, s'ils n'avaient fait tomber quelqu'un.
\VS{17}Parce qu'ils mangent le pain de méchanceté, et qu'ils boivent le vin de la violence.
\VS{18}Mais le sentier des justes est comme la lumière resplendissante, dont l'éclat augmente jusqu'à ce que le jour soit dans sa perfection.
\VS{19}La voie des méchants est comme l'obscurité ; ils n'aperçoivent pas ce qui les fera tomber.
\VS{20}Mon fils, sois attentif à mes paroles, incline ton oreille à mes discours.
\VS{21}Qu'ils ne s'écarte pas de tes yeux ; garde-les dans le fond de ton coeur.
\VS{22}Car ils sont la vie pour ceux qui les trouvent, et la santé de tout le corps de chacun d'eux.
\VS{23}Garde ton coeur de tout ce dont il faut se garder ; car de lui procèdent les sources de la vie\FTNT{Mt 12:35 ; Mt. 15:18-19.}.
\VS{24}Eloigne de toi la perversité de la bouche et la dépravation des lèvres.
\VS{25}Que tes yeux regardent droit et que tes paupières dirigent ton chemin devant toi.
\VS{26}Pèse le chemin de tes pieds, et que toutes tes voies soient bien stables.
\VS{27}Ne te tourne ni à droite ni à gauche ; détourne ton pied du mal.
\Chap{5}
\TextTitle{[Se garder de l'immoralité]}
\VerseOne{}Mon fils, sois attentif à ma sagesse, incline ton oreille à mon intelligence ;
\VS{2}afin que tu gardes mes avis, et que tes lèvres conservent la connaissance.
\VS{3}Car les lèvres de l'étrangère distillent des rayons de miel, et son palais est plus doux que l'huile.
\VS{4}Mais ce qui en provient est amère comme de l'absinthe, et aigu comme une épée à deux tranchants.
\VS{5}Ses pieds descendent à la mort, ses pas atteignent le scheol.
\VS{6}Afin que tu ne balance pas sur le chemin de la vie car, ses chemins en sont écartés; tu ne le connaîtras pas.
\VS{7}Maintenant donc, fils, écoutez-moi, et ne vous détournez pas des paroles de ma bouche.
\VS{8}Eloigne ton chemin de la femme étrangère et n'approche pas de l'entrée de sa maison.
\VS{9}De peur que tu ne donnes ton honneur à d'autres, et tes années à un homme cruel.
\VS{10}De peur que les étrangers ne se rassasient de tes biens, et que le fruit de ton travail ne soit dans la maison d'un étranger.
\VS{11}De peur que tu ne gémisses quand tu seras près de ta fin, quand ta chair et ton corps seront consumés ;
\VS{12}et que tu ne dises : Comment donc ai-je pu haïr la correction, et comment mon coeur a-t-il dédaigné les réprimandes ?
\VS{13}Et comment n'ai-je point obéi à la voix de ceux qui m'instruisaient, et n'ai-je point incliné mon oreille à ceux qui m'enseignaient ?
\VS{14}Peu s'en est fallu que je n'aie été dans toute sorte de mal, au milieu du peuple et de l'assemblée.
\VS{15}Bois des eaux de ta citerne et de ce qui coule du milieu de ton puits ;
\VS{16}que tes sources se répandent dehors, et les ruisseaux d'eau sur les rues ;
\VS{17}qu'elles soient à toi seul, et non aux étrangers avec toi.
\VS{18}Que ta source soit bénie, et réjouis-toi de la femme de ta jeunesse,
\VS{19}comme d'une biche des amours, et d'une chevrette gracieuse ; que ses mamelles te rassasient en tout temps, et sois continuellement épris de son amour.
\VS{20}Et pourquoi, mon fils, irais-tu errant après l'étrangère et embrasserais-tu le sein de l'inconnue ?
\VS{21}Vu que les voies de l'homme sont devant les yeux de Yahweh et qu'il pèse toutes ses voies\FTNT{Jé 16:17 ; Hé. 4:13.}.
\VS{22}Les iniquités du méchant l'attraperont, et il sera retenu par les cordes de son péché.
\VS{23}Il mourra faute d'instruction et il s'égarera par l'excès de sa folie.
\Chap{6}
\TextTitle{Recommandations diverses}
\VerseOne{}Mon fils, si tu t'es porté caution pour ton prochain, si tu as engagé ta main pour un étranger,
\VS{2}tu es enlacé par les paroles de ta bouche, tu es pris par les paroles de ta bouche.
\VS{3}Mon fils, fais maintenant ceci, et dégage-toi, puisque tu es tombé entre les mains de ton intime ami, va, prosterne-toi, et importune tes amis.
\VS{4}Ne donne point de sommeil à tes yeux et ne laisse point sommeiller tes paupières.
\VS{5}Dégage-toi comme la gazelle de la main du chasseur, et comme l'oiseau de la main de l'oiseleur.
\VS{6}Va, paresseux, vers la fourmi, regarde ses voies, et sois sage.
\VS{7}Elle n'a ni chef, ni directeur, ni gouverneur,
\VS{8}et cependant elle prépare en été son pain, et amasse durant la moisson de quoi manger.
\VS{9}Paresseux, jusqu'à quand resteras-tu couché ? Quand te lèveras-tu de ton sommeil ?
\VS{10}Un peu de sommeil, dis-tu, un peu d'assoupissement, un peu croiser les mains afin de rester couché ;
\VS{11}et ta pauvreté viendra comme un voyageur, et ta disette comme un soldat.
\VS{12}Celui qui marche, la fausseté dans sa bouche, est un homme de Bélial\FTNT{1 S. 2:12.}, un homme inique.
\VS{13}Il cligne des yeux, parle du pied, enseigne de ses doigts.
\VS{14}Il y a la perversité dans son cœur, il machine du mal en tout temps, il fait naître des querelles.
\VS{15}C'est pourquoi sa calamité viendra subitement, il sera subitement brisé, il n'y aura point de guérison.
\VS{16}Il y a six choses que Yahweh hait, et il y en a sept qui sont en abomination à son âme ;
\VS{17}savoir, les yeux hautains\FTNT{Ps. 101:5.}, la langue mensongère\FTNT{Ps. 120:2-3.}, les mains qui répandent le sang innocent\FTNT{Es. 1:15.},
\VS{18}le coeur qui médite des projets iniques\FTNT{Ps. 36:5.}, les pieds qui se hâtent de courir au mal\FTNT{Es. 59:7.},
\VS{19}le faux témoin qui profère des mensonges\FTNT{Ps. 27:12.}, et celui qui sème des querelles entre les frères\FTNT{Jud. 1:16-19.}.
\VS{20}Mon fils, garde le commandement de ton père, et n'abandonne pas l'enseignement de ta mère ;
\VS{21}attache-les continuellement à ton coeur, lie-les à ton cou.
\VS{22}Quand tu marcheras, il te conduira ; et quand tu te coucheras, il te gardera ; et quand tu te réveilleras, il s'entretiendra avec toi.
\VS{23}Car le commandement est une lampe ; et l'enseignement une lumière\FTNT{Ps. 119:105.} ; et les réprimandes propres à instruire sont le chemin de la vie.
\VS{24}Ils te préserveront de la mauvaise femme, de la langue doucereuse de l'étrangère.
\VS{25}Ne la convoite pas dans ton coeur pour sa beauté, et ne te laisse pas prendre par ses yeux\FTNT{Mt. 5:28.}.
\VS{26}Car pour l'amour de la femme prostituée on est réduit à un morceau de pain, et la femme adultère chasse après l'âme précieuse de l'homme.
\VS{27}Un homme peut-il prendre du feu dans son sein, sans que ses habits brûlent ?
\VS{28}Un homme marchera-t-il sur des charbons ardents, sans que ses pieds en soient brûlés ?
\VS{29}Il en est de même pour celui qui va vers la femme de son prochain ; quiconque la touchera ne restera pas impuni.
\VS{30}On ne méprise pas un voleur, s'il vole pour satisfaire son âme quand il a faim ;
\VS{31}si on le trouve, il rendra sept fois autant, il donnera tout ce qu'il a dans sa maison.
\VS{32}Mais celui qui commet un adultère avec une femme est dépourvu de sens ; et celui qui le fera, détruira son âme.
\VS{33}Il trouvera des plaies et de l'ignominie, et son opprobre ne sera pas effacé.
\VS{34}Car la jalousie d'un mari est une fureur, il n'épargnera pas l'adultère au jour de la vengeance.
\VS{35}Il n'aura égard à aucune rançon, et il n'acceptera rien, quand même tu multiplierais les présents.
\Chap{7}
\TextTitle{Mise en garde contre la femme prostituée}
\VerseOne{}Mon fils, observe mes paroles, et garde avec toi mes commandements.
\VS{2}Garde mes commandements, et tu vivras, garde mes enseignements comme la prunelle de tes yeux\FTNT{Lé. 18:5.}.
\VS{3}Lie-les sur tes doigts, écris-les sur la table de ton coeur.
\VS{4}Dis à la sagesse : Tu es ma soeur ; et appelle l'intelligence, ton amie.
\VS{5}Afin qu'elles te préservent de la femme étrangère, de l'étrangère qui emploie des paroles doucereuses.
\VS{6}Comme je regardais de la fenêtre de ma maison à travers mon treillis,
\VS{7}je vis parmi les stupides, et je remarquai parmi les jeunes gens un jeune homme dépourvu de sens.
\VS{8}Il passait dans la rue, près de l'angle où se tenait une de ces femmes, et qui suivait le chemin de sa maison,
\VS{9}au crépuscule, au soir du jour, au milieu de la nuit et de l'obscurité.
\VS{10}Et voici, il fut abordé par une femme, vêtue en tenue de prostituée, et pleine de ruse dans le cœur.
\VS{11}Elle était bruyante et rebelle, ses pieds ne restaient point dans sa maison ;
\VS{12}tantôt dehors, tantôt sur les places, elle était aux aguets à chaque coin de rue.
\VS{13}Elle le saisit, et l'embrassa ; et avec un visage effronté, lui dit :
\VS{14}J'ai chez moi des sacrifices d'offrande de paix ; j'ai aujourd'hui accompli mes voeux.
\VS{15}C'est pourquoi je suis sortie à ta rencontre pour chercher ton visage, et je t'ai trouvé.
\VS{16}J'ai orné mon lit de couvertures, d'étoffes de fil d'Egypte.
\VS{17}J'ai parfumé ma couche de myrrhe, d'aloès et de cinnamome.
\VS{18}Viens, enivrons-nous de plaisir jusqu'au matin, réjouissons-nous en amours.
\VS{19}Car mon mari n'est point à la maison, il est parti pour un voyage lointain.
\VS{20}Il a pris un sac d'argent dans sa main, il ne reviendra à la maison qu'à la nouvelle lune.
\VS{21}Elle l'a fait détourner par beaucoup de douces paroles, et l'a attiré par la flatterie de ses lèvres.
\VS{22}Il s'en alla aussitôt après elle, comme un boeuf qui va à la boucherie, comme le fou qu'on lie pour être châtié ;
\VS{23}jusqu'à ce que la flèche lui ait transpercé le foie ; comme l'oiseau qui se hâte vers le filet, sans savoir que c'est au prix de sa vie.
\VS{24}Maintenant donc, fils, écoutez-moi, et soyez attentifs aux paroles de ma bouche.
\VS{25}Que ton coeur ne se détourne pas vers les voies d'une telle femme, ne t' égare pas dans ses sentiers.
\VS{26}Car elle a fait tomber plusieurs blessés à mort, et tous ceux qu'elle a tués sont nombreux.
\VS{27}Sa maison est le chemin du scheol, qui descend vers les demeures de la mort.
\Chap{8}
\TextTitle{[La sagesse préférable aux richesses]}
\VerseOne{}La sagesse ne crie-t-elle pas ? Et l'intelligence ne fait-elle pas entendre sa voix ?
\VS{2}Elle s'est présentée sur le sommet des lieux élevés, sur le chemin, aux carrefours.
\VS{3}Elle crie près des portes, devant la ville, à l'entrée des portes,
\VS{4}ô vous ! Hommes de qualité, je vous appelle ; et ma voix s'adresse aussi aux fils des hommes.
\VS{5}Vous stupides, apprenez le discernement, et vous tous, devenez intelligents de coeur.
\VS{6}Écoutez, car je dirai des choses importantes : Et j'ouvrirai mes lèvres pour enseigner des choses droites.
\VS{7}Parce que ma bouche proclame la vérité, et mes lèvres ont en horreur le mensonge.
\VS{8}Tous les discours de ma bouche sont selon la justice, il n'y a rien en eux de faux, ni de déformé.
\VS{9}Ils sont tous clairs à l'homme intelligent, et droits pour ceux qui ont trouvé la connaissance.
\VS{10}Recevez mon instruction plutôt que de l'argent, la connaissance à l'or le plus précieux.
\VS{11}Car la sagesse vaut mieux que les perles, et tout ce qu'on pourrait souhaiter ne la vaut pas\FTNT{Ps. 19:11 ; Ps. 119:127 ; Job. 28:18.}.
\VS{12}Moi, la Sagesse, j'habite avec le discernement, et je possède la connaissance de la réflexion.
\VS{13}La crainte de Yahweh c'est la haine du mal. Je hais l'orgueil et l'arrogance, la voie du mal, et la bouche perverse.
\VS{14}A moi appartiennent le conseil et le succès ; je suis l'intelligence, à moi appartient la force.
\VS{15}Par moi règnent les rois, et par moi les princes décrètent ce qui est juste.
\VS{16}Par moi gouvernent les seigneurs, les princes, et tous les juges de la terre.
\VS{17}J'aime ceux qui m'aiment ; et ceux qui me cherchent soigneusement me trouveront\FTNT{Mt. 7:7 ; Lu. 11:9 ; Jn 14:23-24.}.
\VS{18}Avec moi sont la richesse et la gloire, les biens durables et la justice.
\VS{19}Mon fruit est meilleur que le fin or, même que l'or raffiné ; et mon revenu est meilleur que l'argent choisi.
\VS{20}Je marche dans le chemin de la justice, au milieu des sentiers de la droiture ;
\VS{21}pour donner des biens en héritage à ceux qui m'aiment, et pour remplir leurs trésors.
\VS{22}Yahweh m'a acquise dès le commencement de ses voies, avant ses œuvres les plus anciennes.
\VS{23}J'ai été déclarée princesse depuis l'éternité, dès le commencement, avant l'origine de la terre.
\VS{24}J'ai été engendrée lorsqu'il n'y avait point encore d'abîmes, ni de sources chargées d'eaux.
\VS{25}Avant que les montagnes soient affermies, avant que les collines existent, j'ai été engendrée.
\VS{26}Lorsqu'il n'avait pas encore fait la terre et les campagnes, et le commencement de la poussière du monde habitable.
\VS{27}Lorsqu'il disposa les cieux, j'étais là ; lorsqu'il traça un cercle à la surface de l'abîme ;
\VS{28}lorsqu'il fixa les nuages en haut ; et que les sources de l'abîme jaillirent avec force ;
\VS{29}lorsqu'il donna une limite à la mer, pour que les eaux ne franchissent pas les bords ; lorsqu'il posa les fondements de la terre,
\VS{30}j'étais à l'œuvre auprès de lui, je faisais ses délices tous les jours, et toujours j'étais en joie en sa présence.
\VS{31}Je me réjouissais dans la partie habitable de sa terre, trouvant mes délices avec les fils de l'homme.
\VS{32}Maintenant donc, mes fils, écoutez-moi : Heureux sont ceux qui observent mes voies.
\VS{33}Ecoutez l'instruction, et soyez sages, et ne la rejetez point.
\VS{34}Ô ! Heureux est l'homme qui m'écoute, qui veille chaque jour à mes portes. Et qui monte la garde aux montants de mes portes !
\VS{35}Car celui qui me trouve a trouvé la vie, et obtient la faveur de Yahweh.
\VS{36}Mais celui qui pèche contre moi nuit à son âme ; tous ceux qui me haïssent aiment la mort.
\Chap{9}
\TextTitle{[La sagesse, source de vie]}
\VerseOne{}La Souveraine Sagesse a bâti sa maison, elle a taillé ses sept colonnes.
\VS{2}Elle a apprêté sa viande, elle a mêlé son vin ; elle a aussi dressé sa table.
\VS{3}Elle a envoyé ses servantes, elle crie du haut des lieux les plus élevés de la ville, disant : 
\VS{4}Que celui qui est stupide, entre ici ; et elle dit à ceux qui sont dépourvus de sens :
\VS{5}Venez, mangez de mon pain, et buvez du vin que j'ai mêlé.
\VS{6}Abandonnez la stupidité, et vous vivrez ; et marchez droit dans la voie de l'intelligence.
\VS{7}Celui qui instruit le moqueur, en reçoit de l'ignominie ; et celui qui reprend le méchant en reçoit une tache.
\VS{8}Ne reprends point le moqueur, de crainte qu'il ne te haïsse ; reprends le sage, et il t'aimera\FTNT{Ps. 141:5.}.
\VS{9}Donne l'instruction au sage, et il deviendra encore plus sage ; enseigne le juste, et il croîtra en science.
\VS{10}Le commencement de la sagesse est la crainte de Yahweh\FTNT{Ps. 19:10.} ; et la connaissance des saints, c'est l'intelligence.
\VS{11}Car tes jours se multiplieront par moi, et les années de vie augmenteront.
\VS{12}Si tu es sage, tu es sage pour toi-même ; si tu es moqueur, tu en porteras seul la peine.
\VS{13}La femme folle est bruyante, stupide et elle ne connaît rien.
\VS{14}Et elle s'assied à la porte de sa maison sur un siège, dans les lieux élevés de la ville ;
\VS{15}pour appeler les passants qui vont droit leur chemin, disant :
\VS{16}Que celui qui est stupide entre ici ! Elle dit à celui qui est dépourvu de sens :
\VS{17}Les eaux dérobées sont douces, et le pain pris en secret est agréable.
\VS{18}Et il ne sait pas que là sont les défunts, et que ceux qu'elle a conviés sont dans le scheol.
\Chap{10}
\TextTitle{[La justice s'oppose à la méchanceté]}
\VerseOne{}Proverbes de Salomon. Le fils sage réjouit son père, mais le fils insensé est l'ennui de sa mère.
\VS{2}Les trésors de méchanceté ne profitent pas, mais la justice délivre de la mort.
\VS{3}Yahweh ne laisse pas l'âme du juste avoir faim, mais il repousse au loin l'avidité des méchants.
\VS{4}Celui qui agit d'une main nonchalante s'appauvrit, mais la main des diligents enrichit.
\VS{5}L'enfant prudent amasse en été, mais celui qui dort durant la moisson est un enfant qui fait honte.
\VS{6}Les bénédictions seront sur la tête du juste, mais la violence couvrira la bouche des méchants.
\VS{7}La mémoire du juste est en bénédiction\FTNT{Ps 112:6.}, mais la réputation des méchants tombe en pourriture.
\VS{8}Celui qui est sage de coeur reçoit les commandements, mais celui qui est insensé des lèvres, tombera.
\VS{9}Celui qui marche dans l'intégrité marche avec assurance, mais celui qui pervertit ses voies, sera connu.
\VS{10}Celui qui cligne de l'oeil cause du chagrin, et celui qui a les lèvres insensées sera renversé.
\VS{11}La bouche du juste est une source de vie, mais la cruauté couvre la bouche des méchants.
\VS{12}La haine excite les querelles, mais la charité couvre toutes les fautes\FTNT{1 Pi 4:8.}.
\VS{13}La sagesse se trouve sur les lèvres de l'homme intelligent, mais la verge est pour le dos de celui qui est dépourvu de sens.
\VS{14}Les sages tiennent la connaissance en réserve, mais la bouche de l'insensé est une ruine prochaine.
\VS{15}Les biens du riche sont la ville de sa force, mais la pauvreté des misérables est leur ruine.
\VS{16}L'oeuvre du juste est pour la vie, mais le revenu du méchant est pour le péché.
\VS{17}Celui qui garde l'instruction est dans le chemin de la vie, mais celui qui néglige la correction s'y égare.
\VS{18}Celui qui dissimule la haine a des lèvres menteuses, et celui qui répand la calomnie est un insensé.
\VS{19}Dans la multitude de paroles le péché ne manque pas, mais celui qui retient ses lèvres est prudent.
\VS{20}La langue du juste est un argent de choix, mais le coeur des méchants est bien peu de chose.
\VS{21}Les lèvres du juste en instruisent plusieurs, mais les insensés mourront faute de sens.
\VS{22}La bénédiction de Yahweh est celle qui enrichit, et il n'y ajoute aucune peine.
\VS{23}C'est comme un jeu à un insensé de pratiquer l'infamie, mais la sagesse appartient à l'homme intelligent.
\VS{24}Ce que redoute le méchant, c'est ce qui lui arrive ; mais Dieu accorde aux justes ce qu'ils désirent.
\VS{25}Comme le tourbillon passe, ainsi le méchant n'est plus ; mais le juste est un fondement perpétuel.
\VS{26}Ce qu'est le vinaigre aux dents et la fumée aux yeux, tel est le paresseux à ceux qui l'envoient.
\VS{27}La crainte de Yahweh augmente les jours, mais les années des méchants sont raccourcies.
\VS{28}L'espérance des justes n'est que joie, mais l'espérance des méchants périra.
\VS{29}La voie de Yahweh est le refuge de l'homme intègre, mais elle est la ruine pour ceux qui pratiquent l'iniquité.
\VS{30}Le juste ne sera jamais ébranlé, mais les méchants ne demeureront pas sur la terre.
\VS{31}La bouche du juste produit la sagesse, mais la langue perverse sera retranchée.
\VS{32}Les lèvres du juste connaissent ce qui est agréable ; mais la bouche des méchants n'est que perversité.
\Chap{11}
\TextTitle{[La justice s'oppose à la méchanceté (suite)]}
\VerseOne{}La fausse balance est une abomination à Yahweh, mais le poids juste lui est agréable\FTNT{Lé 19:35-36 ; De. 25:13-16.}.
\VS{2}Quand l'orgueil vient, la honte vient aussi ; mais la sagesse est avec ceux qui sont modestes.
\VS{3}L'intégrité des hommes droits les conduit, mais la perversité des perfides les détruit.
\VS{4}Les richesses ne servent à rien au jour de la colère, mais la justice délivre de la mort.
\VS{5}La justice de l'homme intègre rend droite sa voie, mais le méchant tombe par sa méchanceté.
\VS{6}La justice des hommes droits les délivre, mais les perfides sont pris par leur méchanceté.
\VS{7}Quand l'homme méchant meurt, son espoir périt ; et l'espérance des hommes iniques périt.
\VS{8}Le juste est délivré de la détresse, et le méchant y entre à sa place.
\VS{9}Par sa bouche l'impie corrompt son prochain, mais les justes en sont délivrés par la connaissance.
\VS{10}La ville se réjouit quand les justes sont heureux, et quand les méchants périssent, c'est un triomphe.
\VS{11}La ville est élevée par la bénédiction des hommes droits, mais elle est renversée par la bouche des méchants.
\VS{12}Celui qui méprise son prochain est dépourvu de sens, mais l'homme prudent se tait.
\VS{13}Celui qui va rapportant, révèle les secrets, mais celui qui a l'esprit qui supporte les paroles, les couvre.
\VS{14}Le peuple tombe par faute de prudence, mais la délivrance est dans la multitude de conseillers.
\VS{15}Celui qui se porte garant pour un étranger en souffrira, et celui qui hait le cautionnement est assuré.
\VS{16}La femme gracieuse obtient de l'honneur, et les hommes robustes obtiennent les richesses.
\VS{17}L'homme doux fait du bien à son âme, mais le cruel trouble sa chair.
\VS{18}Le méchant fait une oeuvre qui le trompe, mais la récompense est assurée à celui qui sème la justice\FTNT{Os. 10:12.}.
\VS{19}Ainsi la justice conduit à la vie, mais celui qui poursuit le mal aboutit à sa mort.
\VS{20}Ceux qui ont le cœur pervers sont en abomination à Yahweh, mais ceux qui sont intègres dans leurs voies lui sont agréables.
\VS{21}De main en main le méchant ne demeurera point impuni, mais la race des justes sera délivrée.
\VS{22}Une belle femme qui se détourne de la raison est comme un anneau d'or au nez d'un pourceau.
\VS{23}Le souhait des justes n'est que le bien, mais l'attente des méchants c'est l'indignation.
\VS{24}Tel, qui donne libéralement, devient plus riche ; et tel qui épargne à l'excès ne fait que s'appauvrir.
\VS{25}Celui qui bénit sera engraisssé ; et celui qui arrose abondamment sera lui-même arrosé.
\VS{26}Sera maudit du peuple, celui qui cache le froment, mais la bénédiction est sur la tête de celui qui le vend.
\VS{27}Qui recherche le bien cherche la faveur, mais le mal arrive à qui le recherche.
\VS{28}Celui qui se confie dans ses richesses tombera, mais les justes verdiront comme le feuillage\FTNT{Ps. 1:3 ; Jé 17: 8.}.
\VS{29}Celui qui ne gouverne pas sa maison avec ordre, aura le vent pour héritage, et le fou sera le serviteur de celui qui a le coeur sage.
\VS{30}Le fruit du juste est un arbre de vie, et celui qui gagne les âmes est sage.
\VS{31}Voici, le juste reçoit sur la terre sa rétribution, combien plus le méchant et le pécheur la recevront-ils ?
\Chap{12}
\TextTitle{[La justice s'oppose à la méchanceté (suite)]}
\VerseOne{}Celui qui aime la correction aime la connaissance, mais celui qui hait la réprimande est un stupide.
\VS{2}L'homme de bien obtient la faveur de Yahweh, mais Yahweh condamne l'homme qui a des mauvaises pensées.
\VS{3}L'homme ne sera point affermi par la méchanceté, mais la racine des justes ne sera point ébranlée.
\VS{4}La femme vertueuse est la couronne de son mari\FTNT{Pr. 31:10.}, mais celle qui fait honte est comme la pourriture dans ses os.
\VS{5}Les pensées des justes ne sont que jugement, mais les conseils des méchants ne sont que fraude.
\VS{6}Les paroles des méchants ne tendent qu'à dresser des embûches pour répandre le sang, mais la bouche des hommes droits les délivrera.
\VS{7}Les méchants sont renversés, et ils ne sont plus, mais la maison des justes se maintiendra.
\VS{8}L'homme est estimé en raison de sa prudence, mais celui qui a le coeur pervers est l'objet du mépris.
\VS{9}Mieux vaut l'homme qui ne fait pas cas de lui-même, bien qu'il ait des serviteurs, que celui qui se glorifie, et qui manque de pain.
\VS{10}Le juste a égard à la vie de sa bête, mais les entrailles des méchants sont cruelles.
\VS{11}Celui qui cultive son champ sera rassasié de pain, mais celui qui court après des futilités est dépourvu de sens.
\VS{12}Ce que le méchant désire, est un filet des hommes mauvais, mais la racine des justes donnera son fruit.
\VS{13}Il y a dans le péché des lèvres un piège pernicieux, mais le juste sortira de la détresse.
\VS{14}L'homme sera rassasié de biens par le fruit de sa bouche, et on rendra à l'homme la rétribution de ses mains.
\VS{15}La voie de l'insensé est droite à son opinion, mais celui qui écoute le conseil est sage.
\VS{16}Quand à l'insensé, sa colère est révélée le jour même, mais l'homme bien avisé couvre son ignominie.
\VS{17}Celui qui prononce des choses véritables rend un témoignage juste, mais le faux témoin fait des rapports trompeurs.
\VS{18}Il y a tel homme dont les paroles blessent comme des pointes d'épée, mais la langue des sages apporte la guérison.
\VS{19}La lèvre véridique est affermie pour toujours, mais la fausse langue n'est que pour un moment\FTNT{Ps. 52: 6-7.}.
\VS{20}Il y a de la tromperie dans le coeur de ceux qui méditent le mal, mais il y a de la joie pour ceux qui conseillent la paix.
\VS{21}Il n'arrivera aucun outrage aux justes, mais les méchants seront remplis de mal.
\VS{22}Les fausses lèvres sont une abomination à Yahweh\FTNT{Ap. 22:15.}, mais ceux qui agissent fidèlement lui sont agréables.
\VS{23}L'homme bien avisé cache sa connaissance, mais le coeur des insensés publie la folie.
\VS{24}La main des diligents dominera, mais la main paresseuse sera tributaire.
\VS{25}Le chagrin qui est au cœur de l'homme, l'accable ; mais la bonne parole le réjouit.
\VS{26}Le juste a plus de reste que son voisin, mais la voie des méchants les égare.
\VS{27}L'homme paresseux ne rôtit point son gibier ; mais les biens précieux de l'homme sont au diligent.
\VS{28}La vie est dans le chemin de la justice, et la voie de son sentier ne tend point à la mort.
\Chap{13}
\TextTitle{[La justice s'oppose à la méchanceté (suite)]}
\VerseOne{}Un fils sage écoute l'instruction de son père, mais le moqueur n'écoute pas la réprimande\FTNT{Ps. 1:1.}.
\VS{2}L'homme mange du bien par le fruit de sa bouche, mais l'âme de ceux qui agissent perfidement mangent l'injustice.
\VS{3}Celui qui garde sa bouche, garde son âme ; mais celui qui ouvre à tout propos ses lèvres, tombera en ruine\FTNT{Ps. 39:2.}.
\VS{4}L'âme du paresseux a des désirs qu'il ne peut satisfaire, mais l'âme des diligents sera engraissée.
\VS{5}Le juste hait la parole mensongère, mais elle rend le méchant odieux et le fait tomber dans la confusion.
\VS{6}La justice garde celui qui est intègre dans sa voie, mais la méchanceté renversera celui qui s'égare.
\VS{7}Tel fait le riche et n'a rien du tout, tel fait le pauvre et a de grandes fortunes.
\VS{8}Les richesses d'un homme servent de rançon pour sa vie, mais le pauvre n'entend pas des réprimandes.
\VS{9}La lumière des justes remplit de joie, mais la lampe des méchants s'éteint.
\VS{10}L'orgueil ne produit que querelle, mais la sagesse est avec ceux qui écoutent les conseils.
\VS{11}Les richesses provenues de la fraude seront diminuées, mais celui qui amasse peu à peu les augmentera.
\VS{12}Un espoir différé fait languir le cœur, mais un désir accompli est comme un arbre de vie.
\VS{13}Celui qui méprise la parole périra à cause d'elle, mais celui qui craint le commandement en sera récompensé.
\VS{14}L'enseignement du sage est une source de vie, pour se détourner des pièges de la mort.
\VS{15}Le bon sens donne de la grâce ; mais la voie de ceux qui agissent perfidement est raboteuse.
\VS{16}Tout homme bien avisé agira avec connaissance, mais l'insensé fera l'étalage de sa folie\FTNT{Da.11:32}.
\VS{17}Le méchant messager tombe dans le mal, mais l'ambassadeur fidèle apporte la guérison.
\VS{18}La pauvreté et l'ignominie arrivent à celui qui rejette l'instruction, mais celui qui garde la réprimande est honoré.
\VS{19}Le souhait accompli est une chose douce à l'âme, mais se détourner du mal est une abomination aux insensés.
\VS{20}Celui qui marche avec les sages deviendra sage, mais le compagnon des insensés sera accablé.
\VS{21}Le mal poursuit les pécheurs, mais le bien sera rendu aux justes.
\VS{22}L'homme de bien laissera de quoi hériter aux fils de ses fils, mais les richesses du pécheur sont réservées aux justes.
\VS{23}Il y a beaucoup à manger dans les terres défrichées des pauvres, mais il y a tel qui est consumé faute de règles.
\VS{24}Celui qui épargne sa verge hait son fils, mais celui qui l'aime se hâte de le châtier.
\VS{25}Le juste mangera jusqu'à être rassasié à son souhait, mais le ventre des méchants aura la disette.
\Chap{14}
\TextTitle{[La justice s'oppose à la méchanceté (suite)]}
\VerseOne{}Toute femme sage bâtit sa maison, mais la folle la ruine de ses mains.
\VS{2}Celui qui marche dans la droiture craint Yahweh, mais celui dont les voies sont perverses le méprise.
\VS{3}La verge d'orgueil est dans la bouche de l'insensé, mais les lèvres des sages les garderont.
\VS{4}Où il n'y a point de boeuf, la grange est vide ; et l'abondance du revenu provient de la force du boeuf.
\VS{5}Le témoin véritable ne ment jamais, mais le faux témoin avance volontiers des mensonges.
\VS{6}Le moqueur cherche la sagesse et ne la trouve pas, mais la connaissance est aisée à trouver pour l'homme intelligent.
\VS{7}Eloigne-toi de l'homme insensé, puisque tu n'as pas trouvé sur ses lèvres la connaissance.
\VS{8}La sagesse d'un homme avisé est de connaître les règles de sa voie, mais la folie des insensés est la tromperie.
\VS{9}Les insensés se moquent du péché, mais parmi les hommes droits se trouve la bienveillance.
\VS{10}Le cœur d'un chacun connaît l'amertume de son âme, et un autre ne saurait partager sa joie.
\VS{11}La maison des méchants sera abolie, mais la tente des hommes droits fleurira.
\VS{12}Il y a telle voie qui semble droite à l'homme, mais dont l'issue sont les voies de la mort.
\VS{13}Même en riant le coeur sera triste, et la joie finit par l'ennui.
\VS{14}Celui qui a un cœur hypocrite, sera rassasié de ses voies ; mais l'homme de bien de ce qui est en lui.
\VS{15}Le simple croit à toute parole ; mais l'homme bien avisé considère ses pas.
\VS{16}Le sage craint et se retire du mal, mais l'insensé se met en colère et est confiant.
\VS{17}Celui qui est prompt à la colère agit follement\FTNT{Ps. 37:8.}, et l'homme plein de ruse est haï.
\VS{18}Les naïfs hériteront la folie ; mais les prudents seront couronnés de connaissance.
\VS{19}Les malins seront humiliés devant les bons, et les méchants, devant les portes du juste.
\VS{20}Le pauvre est haï même de son ami, mais les amis du riche sont en grand nombre.
\VS{21}Celui qui méprise son prochain commet un péché, mais celui qui a pitié des pauvres affligés est heureux.
\VS{22}Ceux qui méditent le mal ne s'égarent-ils pas ? Mais la bonté et la vérité sont pour ceux qui méditent le bien.
\VS{23}En tout travail il y a quelque profit, mais les vains discours ne tournent qu'à la disette.
\VS{24}Les richesses des sages leur sont comme une couronne, mais la stupidité des insensés est toujours stupidité.
\VS{25}Le témoin fidèle délivre les âmes, mais celui qui prononce des mensonges est trompeur.
\VS{26}En la crainte de Yahweh il y a une ferme assurance, et une retraite pour ses fils.
\VS{27}La crainte de Yahweh est une source de vie pour se détourner des pièges de la mort.
\VS{28}La gloire d'un roi, c'est la multitude du peuple, mais quand le peuple manque, c'est la ruine du prince.
\VS{29}Celui qui est lent à la colère a une grande intelligence, mais celui qui est prompt à s'emporter excite la folie.
\VS{30}Un coeur sain est la vie de la chair, mais l'envie est la pourriture des os.
\VS{31}Celui qui fait tort au pauvre déshonore celui qui l'a fait, mais celui qui a pitié de l'indigent honore Yahweh\FTNT{De. 24:11 ; Ps. 107:41.}.
\VS{32}Le méchant est chassé par sa malice, mais le juste trouve un refuge même dans sa mort.
\VS{33}La sagesse repose au coeur de l'homme intelligent, et elle est même reconnue au milieu des insensés.
\VS{34}La justice élève une nation, mais le péché est l'ignominie des peuples.
\VS{35}Le roi prend plaisir au serviteur prudent, mais son indignation sera contre celui qui lui fait honte.
\Chap{15}
\TextTitle{[La justice s'oppose à la méchanceté (suite)]}
\VerseOne{}La réponse douce apaise la fureur ; mais la parole douloureuse excite la colère
\VS{2}La langue des sages se réjouit de la connaissance, mais la bouche des insensés profère la sottise.
\VS{3}Les yeux de Yahweh sont en tous lieux, observant les méchants et les bons.
\VS{4}La langue qui corrige le prochain est comme l'arbre de vie, mais celle où il y a de la perversité est comme une brèche dans l'esprit.
\VS{5}L'insensé méprise l'instruction de son père, mais celui qui prend garde à la réprimande agit avec prudence.
\VS{6}Il y a un grand trésor dans la maison du juste, mais il y a du trouble dans les revenus du méchant.
\VS{7}Les lèvres des sages répandent partout la connaissance, mais le coeur des insensés ne fait pas ainsi.
\VS{8}Le sacrifice des méchants est en abomination à Yahweh, mais la requête des hommes droits lui est agréable.
\VS{9}La voie du méchant est en abomination à Yahweh, mais il aime celui qui poursuit soigneusement la justice.
\VS{10}Le châtiment est fâcheux à celui qui quitte le droit chemin, mais celui qui hait d'être repris, mourra.
\VS{11}Le schéol et le gouffre sont devant Yahweh ; combien plus les coeurs des fils des hommes !
\VS{12}Le moqueur n'aime pas qu'on le reprenne, et il ne va pas vers les sages.
\VS{13}Le cœur joyeux rend le visage beau, mais l'esprit est abattu par l'ennui du cœur.
\VS{14}Le cœur de l'homme prudent cherche la science ; mais la bouche des insensés se repaît de folie.
\VS{15}Tous les jours de l'affligé sont mauvais, mais quand on a le coeur gai, c'est un festin perpétuel.
\VS{16}Un peu de bien vaut mieux avec la crainte de Yahweh, qu'un grand trésor avec lequel il y a du trouble\FTNT{Ps. 37:16.}.
\VS{17}Mieux vaut un repas d'herbes où il y a de l'amitié, qu'un repas de boeuf bien gras où il y a de la haine.
\VS{18}L'homme furieux excite la querelle, mais l'homme lent à la colère apaise la dispute.
\VS{19}La voie du paresseux est comme une haie d'épines, mais le chemin des hommes droits est aplani.
\VS{20}Un fils sage réjouit le père, et un homme insensé méprise sa mère.
\VS{21}La stupidité est la joie de celui qui est dépourvu de sens, mais un homme prudent dresse ses pas au chemin de la droiture.
\VS{22} Les résolutions deviennent inutiles où il n'y a point de conseil ; mais il y a de la fermeté dans la multitude des conseillers.
\VS{23}L'homme a de la joie dans les réponses de sa bouche ; et combien est bonne une parole dite en son temps !
\VS{24}Le chemin de la vie élève l'homme prudent, afin qu'il se détourne du scheol qui est en bas.
\VS{25}Yahweh renverse la maison des orgueilleux, mais il affermit la borne de la veuve.
\VS{26}Les pensées du malin sont en abomination à Yahweh, mais celles de ceux qui sont purs sont des paroles agréables à ses yeux.
\VS{27}Celui qui est entièrement adonné au gain déshonnête trouble sa maison, mais celui qui hait les présents vivra.
\VS{28}Le coeur du juste médite ce qu'il doit répondre, mais la bouche des méchants profère des choses mauvaises.
\VS{29}Yahweh est loin des méchants, mais il exauce la requête des justes.
\VS{30}La clarté des yeux réjouit le coeur ; et la bonne renommée fortifie les os.
\VS{31}L'oreille qui écoute la correction qui donne la vie habite parmi les sages.
\VS{32}Celui qui rejette l'instruction a en dédain son âme, mais celui qui écoute la réprimande s'acquiert du sens.
\VS{33}La crainte de Yahweh enseigne la sagesse, et l'humilité précède la gloire\FTNT{Ps. 19:10.}.
\Chap{16}
\TextTitle{[La justice s'oppose à la méchanceté (suite)]}
\VerseOne{}Les préparations du cœur sont à l'homme, mais le discours réponse de la langue est de par Yahweh.
\VS{2}Chacune des voies de l'homme lui semble pure à ses yeux; mais Yahweh pèse les esprits.
\VS{3}Recommande tes affaires à Yahweh, et tes pensées seront bien ordonnées.
\VS{4}Yahweh a fait toutes choses pour lui-même ; et même le méchant pour le jour de l'affliction.
\VS{5}Yahweh a en abomination tout homme hautain de coeur ; assurément, il ne demeurera pas impuni.
\VS{6}Il y aura propitiation de l'iniquité par la miséricorde et la vérité ; on se détourne du mal par la crainte de Yahweh. 
\VS{7}Quand Yahweh prend plaisir aux voies d'un homme, il apaise\FTNT{Apaiser vient de shalom qui signifie : être dans une alliance de paix, être en paix, apaiser, vivre dans la paix etc.} envers lui même ses ennemis.
\VS{8}Il vaut mieux un peu de bien avec justice, qu'un gros revenu là où on n'a pas de droit.
\VS{9}Le cœur de l'homme médite sur sa voie, mais Yahweh conduit ses pas.
\VS{10}La divination est sur les lèvres du roi : Sa bouche ne doit pas s'égarer du droit.
\VS{11}La balance et le poids justes sont à Yahweh, tous les poids du sachet sont aussi son oeuvre.
\VS{12}Commettre une injustice doit être en abomination aux rois, parce que le trône est affermi par la justice.
\VS{13}Les rois doivent prendre plaisir aux lèvres de justice, et aimer celui qui profère des paroles justes.
\VS{14}Ce sont autant de messagers de mort que la colère du roi, mais l'homme sage l'apaisera.
\VS{15}Le visage serein du roi c'est la vie, et sa faveur est comme la nuée portant la pluie de la dernière saison.
\VS{16}Combien est-il plus précieux que l'or fin, d'acquérir de la sagesse! Et combien est-il plus excellent que l'argent, d'acquérir de la prudence ! 
\VS{17}Le chemin aplani des hommes droits, c'est de se détourner du mal ; celui qui prend garde de sa voie garde son âme.
\VS{18}L'orgueil va devant l'écrasement, et la fierté d'esprit devant la ruine.
\VS{19}Mieux vaut être humilié d'esprit avec les débonnaires, que de partager le butin avec les orgueilleux.
\VS{20}Celui qui prend garde à la parole trouvera le bien, et celui qui se confie en Yahweh est heureux\FTNT{Ps. 2:12.}.
\VS{21}On appellera prudent le sage de cœur, et la douceur des lèvres augmente l'instruction.
\VS{22}La prudence est à ceux qui la possèdent une source de vie ; mais le l'instruction des fous c'est leur folie.
\VS{23}Celui qui est sage de coeur conduit prudemment sa bouche, et ajoute l'instruction sur ses lèvres.
\VS{24}Les paroles agréables sont des rayons de miel, douces à l'âme et santé pour les os.
\VS{25} II y a telle voie qui semble droite à l'homme, mais dont la fin sont les voies de la mort.
\VS{26}Celui qui travaille, travaille pour lui-même, parce que sa bouche se courbe devant lui\FTNT{Ec. 6:7.}.
\VS{27}L'homme méchant creuse le mal, et il y a comme un feu brûlant sur ses lèvres.
\VS{28}L'homme qui use de perversité sème des querelles, et le rapporteur divise les grands amis.
\VS{29}L'homme violent attire son compagnon et le fait marcher dans une voie qui n'est pas bonne.
\VS{30}Il fait signe des yeux pour méditer des choses perverses, et remuant ses lèvres il exécute le mal.
\VS{31}Les cheveux blancs sont une couronne d'honneur ; elle se trouvera dans la voie de la justice.
\VS{32}Celui qui est lent à la colère vaut mieux que l'homme fort, et celui qui est maître de son cœur, vaut mieux que celui qui prend des villes.
\VS{33}On jette le sort dans le pan de la robe, mais tout ce qui doit arriver est de part Yahweh.
\Chap{17}
\TextTitle{[La justice s'oppose à la méchanceté (suite)]}
\VerseOne{}Mieux vaut un morceau de pain sec là où il y a la paix, qu'une maison pleine de viandes, là où il y a des querelles.
\VS{2}Le serviteur prudent sera maître sur l'enfant qui fait honte, et il partagera l'héritage entre les frères.
\VS{3}Le creuset est pour éprouver l'argent, et le fourneau l'or ; mais Yahweh éprouve les coeurs\FTNT{Jé. 17:10 ; Mal. 3:3 ; Ps. 26:2.}.
\VS{4}L'homme mauvais est attentif à la lèvre trompeuse, et le menteur écoute la mauvaise langue.
\VS{5}Celui qui se moque du pauvre déshonore celui qui l'a fait ; et celui qui se réjouit de l'affliction ne demeurera pas impuni.
\VS{6}Les petits-fils sont la couronne des vieillards\FTNT{Ps. 127:3 ; Ps. 128:3.}, et les pères sont la gloire de leurs fils.
\VS{7}La parole distinguée ne convient pas à un fou ; combien moins aux principaux du peuple des paroles de mensonge!
\VS{8}Le présent est comme une pierre précieuse aux yeux de ceux qui y sont adonnés ; de quelque côté qu'ils se tournent, ils réussissent.
\VS{9}Celui qui couvre les fautes cherche l'amitié, mais celui qui rapporte la chose divise les plus grands amis.
\VS{1}La répréhension se fait mieux sentir sur l'homme prudent que cent coups au fou.
\VS{11}Le méchant ne cherche que rébellion, mais le messager cruel sera envoyé contre lui.
\VS{12}Que l'homme rencontre plutôt une ourse qui a perdu ses petits qu'un fou dans sa folie.
\VS{13}Le mal ne partira point de la maison de celui qui rend le mal pour le bien.
\VS{14}Le commencement d'une querelle est comme quand on lâche une l'eau; mais avant qu'on en vienne à la dispute, retire-toi.
\VS{15}Celui qui déclare juste le méchant et celui qui déclare méchant le juste, sont tous deux en abomination à Yahweh\FTNT{Ex. 23:7 ; Es. 5:23.}.
\VS{16}A quoi sert le prix dans la main du fou pour acheter la sagesse, vu qu'il n'a pas de sens?
\VS{17}L'ami intime aime en tout temps, et il naît comme un frère dans la détresse.
\VS{18}Celui là est dépourvu de sens qui touche à la main et se rend caution pour son ami.
\VS{19}Celui qui aime les querelles aime le péché ; celui qui élève sa porte cherche sa ruine.
\VS{20}Celui qui est pervers de coeur ne trouve pas le bien; et l'hypocrite tombe dans le malheur.
\VS{21}Celui qui engendre un sot en aura de l'ennui, et le père du sot ne se réjouira pas.
\VS{22}Le coeur joyeux est un remède, mais l'esprit abattu dessèche les os.
\VS{23}Le méchant rend les présents en secret, pour pervertir les voies du jugement.
\VS{24}La sagesse est en présence de l'homme prudent; mais les yeux du fou sont à l'extrémité de la terre.
\VS{25}Le fils fou est l'ennui de son père, et l'amertume de celle qui l'a enfanté.
\VS{26}Il n'est pas bon de condamner l'innocent à l'amende, ni que les principaux frappent quelqu'un pur avoir agi avec droiture.
\VS{27}L'homme retenu dans ses paroles sait ce qu'est la connaissance, et l'homme qui est d'un esprit calme est un homme intelligent.
\VS{28}Même le fou, quand il se tait, est réputé sage ; et celui qui ferme ses lèvres est réputé intelligent.
\Chap{18}
\TextTitle{[La justice s'oppose à la méchanceté (suite)]}
\VerseOne{}Celui qui se sépare cherche ce qui lui fait plaisir, et se mêle de savoir comment tout doit aller.
\VS{2}Le fou ne prend pas plaisir à l'intelligence, mais à ce que son cœur soit manifesté.
\VS{3}Quand le méchant vient, le mépris vient aussi, et le reproche avec l'ignominie.
\VS{4}Les paroles de la bouche d'un homme sont des eaux profondes ; et la source de la sagesse est un torrent qui bouillonne\FTNT{Jn. 4:14.}.
\VS{5}Il n'est pas bon d'avoir égard à l'apparence de la personne du méchant, pour renverser le juste en jugement.
\VS{6}La bouche du fou entrent en querelles, et sa bouche appelle les combats.
\VS{7}La bouche du fou lui est une ruine, et ses lèvres sont un piège à son âme.
\VS{8}Les paroles du flatteur sont de ceux qui font semblant d'y toucher ; mais elles pénètrent jusqu'au-dedans des entrailles.
\VS{9}Celui qui se relâche dans son ouvrage est frère de celui qui dissipe ce qu'il a.
\VS{10}Le Nom de Yahweh est une tour forte, le juste y court et y trouve une haute retraite.
\VS{11}Les biens du riche sont sa ville forte et comme une haute muraille de retraite, selon son imagination.
\VS{12}Le coeur de l'homme s'élève avant que la ruine arrive, mais l'humilité précède la gloire.
\VS{13}Celui qui répond à quelque propos avant de l'avoir entendu, agit en fou et s'attire le reproche.
\VS{14}L'esprit d'un homme fort soutiendra dans son infirmité ; mais l'esprit abattu, qui le relèvera ?
\VS{15}Le coeur de l'homme intelligent acquiert la connaissance, et l'oreille des sages cherche la connaissance.
\VS{16}Le présent d'un homme lui fait faire place, et le conduit devant les grands.
\VS{17}Celui qui plaide le premier paraît juste; mais sa partie adverse vient, et examine le tout.
\VS{18}Le sort fait cesser les procès et fait les partages entre les puissants.
\VS{19}Un frère offensé se rend plus difficile qu'une ville forte, et les discordes entre frères sont comme les verrous d'un palais.
\VS{20}Le ventre de chacun est rassasié du fruit de sa bouche, il se rassasie du revenu de ses lèvres.
\VS{21}La mort et la vie sont au pouvoir de la langue\FTNT{Mt. 12:37.}, et celui qui aime à parler mangera de ses fruits.
\VS{22}Celui qui trouve une femme vertueuse trouve le bonheur et il obtient une faveur de Yahweh.
\VS{23}Le pauvre ne prononce que des supplications, mais le riche ne répond que des paroles dures.
\VS{24}L'homme qui a des intimes amis se tiennent à leur amitié parce qu'il y a tel ami qui est plus attaché que le frère.
\Chap{19}
\TextTitle{[La justice s'oppose à la méchanceté (suite)]}
\VerseOne{}Le pauvre qui marche dans son intégrité, vaut mieux que celui qui pervertit ses lèvres et qui est fou.
\VS{2}La vie même sans connaissance n'est pas une bonne personne ; et celui qui hâte ses pas dans le péché, s'égare.
\VS{3}La folie de l'homme renverse son chemin ; et cependant, c'est contre Yahweh que son coeur s'irrite.
\VS{4}Les richesses attirent un grand nombre d'amis, mais celui qui est pauvre est abandonné même par son ami.
\VS{5}Le faux témoin ne restera pas impuni, et celui qui profère des mensonges n'échappera pas.
\VS{6}Plusieurs supplient celui qui est en état de faire du bien, et chacun est ami de celui qui donne.
\VS{7}Tous les frères du pauvre le haïssent ; combien plus ses amis se retirent-ils de lui ! Il les supplie, mais il n'y a que des paroles pour lui.
\VS{8}Celui qui acquiert du sens aime son âme, et celui qui prend garde à l'intelligence c'est pour trouve le bonheur.
\VS{9}Le faux témoin ne restera pas impuni, et celui qui profère des mensonges périra.
\VS{10}Il ne sied pas à un fou de vivre dans les délices ; combien moins sied-il à un esclave de dominer sur les personnes de distinction !
\VS{11}La prudence de l'homme retient à la colère ; c'est un honneur pour lui de passer par dessus le tort qu'on lui fait.
\VS{12}La colère du roi est comme le rugissement d'un jeune lion, mais sa faveur est comme la rosée sur l'herbe.
\VS{13}Un fils insensé est un grand malheur pour son père, et les querelles d'une femme sont une gouttière continuelle.
\VS{14}On peut hériter de ses pères une maison et des richesses, mais la femme prudente est un don de Yahweh.
\VS{15}La paresse fait venir le sommeil, et l'âme paresseuse a faim.
\VS{16}Celui qui garde le commandement garde son âme, mais celui qui méprise ses voies mourra.
\VS{17}Celui qui a pitié du pauvre prête à Yahweh, qui lui rendra son bienfait.
\VS{18}Châtie ton fils tandis qu'il y a de l'espérance, mais ne va pas jusqu'à le faire mourir.
\VS{19}Celui qui est de grande colère en porte la peine ; et si tu l'en retires, tu y ajoute davantage.
\VS{20}Ecoute le conseil et reçois l'instruction, afin que tu deviennes sage en ton dernier temps.
\VS{21}Il y a dans le cœur de l'homme plusieurs pensées, mais le conseil de Yahweh est\FTNT{Es. 46:10 ; Ps. 33:11.}.
\VS{22}Ce que l'homme doit désirer, c'est d'exercer la miséricorde ; et le pauvre vaut mieux qu'un menteur.
\VS{23}La crainte de Yahweh conduit à la vie, et celui qui l'a, passe la nuit étant rassasié, sans qu'il soit visité par aucun mal.
\VS{24}Le paresseux cache sa main dans le sein, et il ne daigne même pas la ramener à sa bouche.
\VS{25}Si tu bats le moqueur, le sot en rend garde ; et si tu reprends l'homme intelligent, il discernera ce qu'il faut savoir.
\VS{26}L'enfant qui fait honte et sème la confusion, détruit le père et met en fuite sa mère.
\VS{27}Mon fils, cesse d'écouter ce qui pourrait t'apprendre à te détourner des paroles de la connaissance.
\VS{28}Le témoin indigne\FTNT{Le mot « pervers » vient de l'hébreu « beliya'al » : « sans valeur », « vaurien » (Jg. 19:22 ; 1. S. 2:12). Bélial est aussi un autre nom de Satan (2 Co. 6:15).} se moque de la justice, et la bouche des méchants avale l'iniquité.
\VS{29}Les jugements sont préparés pour les moqueurs, et les grands coups pour le dos des fous.
\Chap{20}
\TextTitle{[La justice s'oppose à la méchanceté (suite)]}
\VerseOne{}Le vin est moqueur et la boisson forte mutine, et quiconque en fait excès, n'est pas sage.
\VS{2}La terreur du roi est comme le rugissement d'un jeune lion, celui qui se met en colère contre lui pèche contre sa propre âme.
\VS{3}C'est une gloire à l'homme de s'abstenir des disputes, mais chaque insensé s'en mêle.
\VS{4}Le paresseux ne laboure pas à cause du mauvais temps, mais il mendiera durant la moisson, et il n'en aura rien.
\VS{5}Le conseil dans le coeur d'un homme sont comme des eaux profondes, et l'homme intelligent sait y puiser.
\VS{6}Beaucoup de gens prêche leur bonté ; mais qui trouvera un homme véritable?
\VS{7}Ô, que les fils du juste qui marchent dans son intégrité seront heureux après lui !
\VS{8}Le roi assis sur le trône de justice dissipe tout mal par son regard.
\VS{9}Qui est-ce qui peut dire : J'ai purifié mon cœur, je suis net de mon péché ?
\VS{10}Le double poids et la double mesure sont tous deux en abomination à Yahweh.
\VS{11}Un jeune enfant même fait connaître par ses actions si son oeuvre sera pure et  si elle sera droite.
\VS{12}L'oreille qui entend et l'oeil qui voit, Yahweh les a faits tous les deux.
\VS{13}N'aime point le sommeil, de peur que tu ne deviennes pauvre ; ouvre tes yeux, et tu auras suffisamment de pain.
\VS{14}Il est mauvais, il est mauvais, dit l'acheteur ; puis il s'en va, et se vante.
\VS{15}Il y a de l'or et beaucoup de perles ; mais les lèvres qui prononcent la connaissance sont un vase précieux.
\VS{16}Quand quelqu'un se porte garant pour l'étranger, prends son vêtement ; exige de lui des gages pour cet étranger.
\VS{17}Le pain volé est doux à l'homme, mais ensuite sa bouche sera remplie de gravier.
\VS{18}Chaque pensée s'affermissent par le conseil ; fais donc la guerre avec prudence.
\VS{19}Celui qui médit révèle les secrets va médisant ; ne te mêle donc pas avec celui qui séduit par ses lèvres.
\VS{20}La lampe de celui qui traite avec mépris son père ou sa mère, s'éteindra au milieu des ténèbres les plus noires\FTNT{Ex. 21:17 ; Lé. 20:9 ; Mt. 15:4.}.
\VS{21}L'héritage pour lequel on s'est trop hâté au début, ne sera pas béni à la fin.
\VS{22}Ne dis point : Je rendrai le mal ; mais attends Yahweh, et il te délivrera.
\VS{23}Le double poids est en abomination à Yahweh, et la balance fausse n'est pas une chose bonne.
\VS{24}Les pas de l'homme sont dirigés par Yahweh, comment donc l'homme peut-il comprendre sa voie ?
\VS{25}C'est un piège à l'homme que de dévorer la chose sainte, et de ne réfléchir qu'après des vœux.
\VS{26}Un roi sage dissipe les méchants et fait tourner la roue sur eux.
\VS{27}L'esprit de l'homme est une lampe de Yahweh, il pénètre jusqu'au fond plus profond.
\VS{28}La bienveillance et la vérité protègent le roi, et il soutient son trône par la bienveillance.
\VS{29}La force est la gloire des jeunes gens, et les cheveux blancs sont l'honneur des vieillards.
\VS{30}Les meurtrissures et les plaies nettoient le mal, de même les coups qui pénètrent jusqu'au fond des entrailles.
\Chap{21}
\TextTitle{[La justice s'oppose à la méchanceté (suite)]}
\VerseOne{}Le coeur du roi est un courant d'eau dans la main de Yahweh ; il l'incline partout où il veut.
\VS{2}Toutes les voies de l'homme sont droites à ses yeux, mais c'est Yahweh qui pèse les coeurs.
\VS{3}Faire ce qui est juste et droit est une chose que Yahweh préfère aux sacrifices.
\VS{4}Des regards hautains et le coeur qui s'enfle sont la lampe des méchants, ce n'est que péché.
\VS{5}Les projets de l'homme diligent ne mènent qu'à l'abondance, mais celui qui agit avec précipitation ne court qu'à l'indigence.
\VS{6}Des trésors acquis par une langue mensongère, c'est une vanité qu'on ne peut retenir, un signe avant-coureur de la mort.
\VS{7}La violence des méchants les emporte, parce qu'ils refusent de faire ce qui est droit.
\VS{8}La voie d'un homme coupable est détournée, mais l'oeuvre de celui qui est innocent est droite.
\VS{9}Il vaut mieux habiter à l'angle d'un toit qu'avec une femme querelleuse dans une grande maison.
\VS{10}L'âme du méchant désire le mal, son prochain ne trouve pas de grâce à ses yeux.
\VS{11}Quand on punit le moqueur, le sot devient sage ; et quand on instruit le sage, il reçoit la connaissance.
\VS{12}Il y a un juste qui considère attentivement la maison du méchant, Yahweh renverse les méchants dans le malheur.
\VS{13}Celui qui bouche son oreille pour ne pas entendre le cri du pauvre, criera aussi lui-même, et on ne lui répondra point.
\VS{14}Un don fait en secret apaise la colère, et un présent fait en cachette calme une fureur violente.
\VS{15}C'est une joie pour le juste de pratiquer la justice, mais c'est la ruine pour les ouvriers d'iniquité.
\VS{16}L'homme qui s'écarte du chemin de la sagesse aura sa demeure dans l'assemblée des morts.
\VS{17}Celui qui aime les réjouissances reste dans l'indigence ; et celui qui aime le vin et l'huile ne s'enrichira pas.
\VS{18}Le méchant sert de rançon pour le juste, et le déloyal pour les hommes intègres.
\VS{19}Il vaut mieux habiter dans une terre déserte qu'avec une femme querelleuse et qui se dépite.
\VS{20}Des précieux trésors et l'huile sont dans la demeure du sage, mais l'homme insensé les engloutit.
\VS{21}Celui qui poursuit la justice et la bonté, trouve la vie, la justice et la gloire.
\VS{22}Le sage entre dans la ville des forts et il abat la force qui lui donnait de l'assurance.
\VS{23}Celui qui veille sur sa bouche et sur sa langue préserve son âme des angoisses.
\VS{24}On appelle moqueur un superbe arrogant, qui agit avec colère et orgueil.
\VS{25}Les désirs du paresseux le tuent, parce que ses mains refusent de travailler.
\VS{26}Tout le jour il désire avidement, mais le juste donne sans parcimonie.
\VS{27}Le sacrifice des méchants est une abomination ; combien plus quand ils l'apportent avec des mauvaises intentions\FTNT{1 S. 15:22.} ?
\VS{28}Le témoin menteur périra, mais l'homme qui écoute parlera avec gain de cause.
\VS{29}L'homme méchant prend un air effronté, mais l'homme droit règle sa conduite.
\VS{30}Il n'y a ni sagesse, ni intelligence, ni conseil, contre Yahweh.
\VS{31}Le cheval est équipé pour le jour de la bataille, mais la délivrance vient de Yahweh.
\Chap{22}
\TextTitle{[La justice s'oppose à la méchanceté (suite)]}
\VerseOne{}La renommée est préférable aux grandes richesses\FTNT{Ec.7:1.}, et la bonne grâce plus que l'argent et l'or.
\VS{2}Le riche et le pauvre se rencontrent ; celui qui les a faits l'un et l'autre, c'est Yahweh\FTNT{Lu. 16.}.
\VS{3}L'homme prudent voit le mal et se cache, mais les stupides passent et en portent la peine.
\VS{4}Les fruits de l'humilité et de la crainte de Yahweh sont les richesses, la gloire et la vie.
\VS{5}Il y a des épines et des pièges dans la voie de l'homme pervers ; celui qui aime son âme s'en retirera loin.
\VS{6}Instruis le jeune enfant selon la voie qu'il doit suivre, et quand il sera vieux, il ne s'en détournera pas.
\VS{7}Le riche domine sur les pauvres\FTNT{Ja. 2:6.}, et celui qui emprunte est l'esclave de celui qui prête.
\VS{8}Celui qui sème l'injustice moissonne le malheur\FTNT{Job. 4:8 ; Ga. 6:7.}, et la verge de sa fureur prendra fin.
\VS{9}Celui qui a l'oeil bienveillant sera béni, parce qu'il aura donné de son pain au pauvre.
\VS{10}Chasse le moqueur, et la querelle prendra fin ; les disputes et l'ignominie cesseront.
\VS{11}Le roi est ami de celui qui aime la pureté de coeur, et qui a de la grâce dans ses paroles.
\VS{12}Les yeux de Yahweh veillent sur la connaissance, mais il confond les paroles du perfide.
\VS{13}Le paresseux dit : Il y a un lion dehors ! Je serais tué dans les rues !
\VS{14}La bouche des courtisanes est une fosse profonde, celui contre qui Yahweh est irrité y tombera.
\VS{15}La folie est liée au coeur du jeune enfant, mais la verge de la correction l'éloignera de lui.
\VS{16}Celui qui fait tort au pauvre pour s'enrichir et pour donner au riche, ne peut manquer de tomber dans l'indigence.
\VS{17}Prête ton oreille et écoute les paroles des sages, et applique ton coeur à ma connaissance.
\VS{18}Car ce sera une chose agréable pour toi si tu les gardes au-dedans de toi, et qu'elles soient toutes présentes sur tes lèvres.
\VS{19}Je te les ai fait connaître à toi aujourd'hui, dis-je, afin que ta confiance soit en Yahweh.
\VS{20}N'ai-je pas déjà pour toi mis par écrit des choses qui conviennent à ceux qui gouvernent, des conseils et des réflexions,
\VS{21}pour te faire connaître la certitude des paroles vraies, afin que tu répondes par des paroles vraies à celui qui t'envoie ?
\VS{22}Ne dépouille pas le pauvre, parce qu'il est pauvre ; et n'opprime pas le malheureux à la porte.
\VS{23}Car Yahweh défendra leur cause et privera de la vie ceux qui les auront volés.
\VS{24}Ne fréquente pas quelqu'un de coléreux, ne va pas avec l'homme violent ;
\VS{25}de peur que tu n'apprennes ses manières, et qu'ils ne deviennent un piège pour ton âme.
\VS{26}Ne sois pas parmi ceux qui prennent des engagements ni de ceux qui cautionnent les dettes.
\VS{27}Si tu n'as pas de quoi payer, pourquoi prendrait-on ton lit de dessous toi ?
\VS{28}Ne déplace pas la borne ancienne, que tes pères ont posée.
\VS{29}As-tu vu un homme habile en son travail ? Il sera au service des rois, il ne se tiendra pas devant des gens obscurs.
\Chap{23}
\TextTitle{[La justice s'oppose à la méchanceté (suite)]}
\VerseOne{}Quand tu t'assieds pour manger avec un gouverneur, considère avec attention celui qui est devant toi.
\VS{2}Autrement tu te mettras le couteau à la gorge, si ton appétit te domine.
\VS{3}Ne convoite pas ses friandises, car c'est un pain trompeur.
\VS{4}Ne travaille pas en vue d'acquérir des richesses ; désiste-toi de la résolution que tu as prise.
\VS{5}Jetteras-tu tes yeux sur ce qui bientôt n'est plus ? Car certainement, il se fera des ailes, il s'envolera comme un aigle dans les cieux.
\VS{6}Ne mange pas le pain de celui dont le regard est envieux, et ne désire pas ses friandises.
\VS{7}Car il est tel qu'il pense dans son âme. Il te dira bien : Mange et bois, mais son coeur n'est pas avec toi.
\VS{8}Tu voudrais vomir le morceau que tu auras mangé, et tu auras perdu tes paroles agréables.
\VS{9}Ne parle pas aux oreilles de l'insensé, car il méprise le bon sens de ton discours.
\VS{10}Ne déplace pas la borne ancienne et n'entre pas dans les champs des orphelins :
\VS{11}Car leur vengeur est puissant, il défendra leur cause contre toi.
\VS{12}Applique ton coeur à l'instruction, et tes oreilles aux paroles de la connaissance.
\VS{13}Ne te retiens pas de corriger le jeune enfant ; quand tu l'auras frappé de la verge, il n'en mourra pas.
\VS{14}En le frappant de la verge, tu préserves son âme du scheol.
\VS{15}Mon fils, si ton coeur est sage, mon coeur s'en réjouira, oui, moi-même.
\VS{16}Certes, mes reins tressailliront de joie quand tes lèvres proféreront ce qui est droit.
\VS{17}Que ton coeur ne porte pas d'envie aux pécheurs, mais adonne-toi tout le jour à la crainte de Yahweh.
\VS{18}Car il y a véritablement un avenir, et ton espérance ne sera pas retranchée.
\VS{19}Toi, mon fils, écoute et sois sage, et dirige ton coeur dans la bonne voie.
\VS{20}Ne fréquente point les ivrognes ni les gourmands\FTNT{Ro. 13:13 ; Ep. 5:18 ; Ga. 5:18-21.}.
\VS{21}Car l'ivrogne et le gourmand s'appauvrissent ; et l'assoupissement fait porter des vêtements déchirés.
\VS{22}Ecoute ton père, c'est celui qui t'a engendré ; et ne méprise pas ta mère, quand elle est devenue vieille.
\VS{23}Acquiers la vérité, et ne la vends point, acquiers la sagesse, l'instruction et l'intelligence.
\VS{24}Le père du juste aura beaucoup de joie, et celui qui donne naissance à un sage se réjouira en lui.
\VS{25}Que ton père et ta mère se réjouissent, que celle qui t'a enfanté soit dans l'allégresse !
\VS{26}Mon fils, donne-moi ton coeur, et que tes yeux prennent plaisir à mes voies.
\VS{27}Car la femme prostituée est une fosse profonde, et la courtisane un puits de détresse.
\VS{28}Aussi se tient-elle en embûche comme un voleur, et elle augmente parmi les hommes le nombre des infidèles.
\VS{29}Pour qui les « ah » ? Pour qui les « malheur à moi ! » Pour qui les disputes ? Pour qui les plaintes ? Pour qui les blessures sans raison ? Pour qui les yeux rouges ?
\VS{30}Pour ceux qui s'attardent auprès du vin, pour ceux qui vont chercher des vins mélangés.
\VS{31}Ne regarde pas le vin parce qu'il est d'un beau rouge, qu'il donne son éclat dans la coupe, et qu'il coule aisément.
\VS{32}Il finit par mordre par derrière comme un serpent, et par piquer comme un basilic.
\VS{33}Ensuite tes yeux regarderont les femmes étrangères, et ton coeur parlera d'une manière perverse.
\VS{34}Tu seras comme un homme qui dort au milieu de la mer, et comme un homme couché sur le sommet d'un mât.
\VS{35}On m'a battu, diras-tu, et je n'en ai pas été malade, on m'a frappé, et je ne l'ai pas senti, quand me réveillerai-je ? Je me remettrai encore à chercher le vin.
\Chap{24}
\TextTitle{[La justice s'oppose à la méchanceté (suite)]}
\VerseOne{}N'envie pas les hommes qui font le mal, et ne désire pas être avec eux.
\VS{2}Car leur coeur médite la destruction, et leurs lèvres parlent d'iniquité.
\VS{3}C'est par la sagesse qu'une maison est bâtie, et par l'intelligence qu'elle s'affermit.
\VS{4}C'est par la connaissance que les chambres seront remplies de tous les biens précieux et agréables.
\VS{5}Un homme sage est accompagné de force, et celui qui a de la connaissance affermit sa vigueur.
\VS{6}Car avec de bonnes directives tu feras la guerre avantageusement, et le salut est dans le grand nombre des bons conseillers.
\VS{7}La sagesse est trop élevée pour l'insensé, il n'ouvrira pas sa bouche à la porte.
\VS{8}Celui qui médite de faire le mal s'appelle un homme plein de malice.
\VS{9}Le projet de la folie est un péché, et le moqueur est en abomination parmi les hommes.
\VS{10}Si tu perds courage au jour de la détresse, ta force n'est que détresse.
\VS{11}Ne te retiens pas de délivrer ceux qu'on traîne à la mort, ceux qu'on va égorger, agis pour qu'on les épargne !
\VS{12}Si tu dis : Ah ! nous n'en savions rien… Celui qui pèse les cœurs, lui, ne le considérera-t-il pas ? Celui qui garde ton âme, lui, le sait, et il rend à chacun selon son œuvre.
\VS{13}Mon fils, mange le miel, car il est bon ; un rayon de miel sera doux à ton palais.
\VS{14}Ainsi sera à ton âme la connaissance de la sagesse, quand tu l'auras trouvée ; il y a un avenir, et ton espérance ne sera pas anéantie.
\VS{15}Méchant, ne mets pas des embûches dans le domaine du juste, et ne dévaste pas le lieu où il se repose.
\VS{16}Car le juste tombera sept fois, et sera relevé\FTNT{Ps. 34:20. ; Job. 5:19.} ; mais les méchants trébuchent pour tomber dans le malheur.
\VS{17}Si ton ennemi tombe, ne t'en réjouis pas, et que ton coeur ne soit pas dans l'allégresse quand il chancelle,
\VS{18}de peur que Yahweh ne le voie et que cela ne lui déplaise, tellement qu'il détourne sa colère de dessus de lui sur toi.
\VS{19}Ne t'irrite pas à cause de ceux qui font le mal, n'envie pas les méchants,
\VS{20}car il n'y a pas d'avenir pour celui qui fait le mal, la lampe des méchants sera éteinte.
\VS{21}Mon fils, crains Yahweh et le roi ; et ne te mêle pas avec les gens agités.
\VS{22}Car leur ruine s'élèvera tout d'un coup ; et qui sait le malheur qui arrivera à l'un et à l'autre ?
\VS{23}Voici encore ce qui vient des sages : Il n'est pas bon d'avoir égard à l'apparence des personnes dans le jugement.
\VS{24}Celui qui dit au méchant : Tu es juste ! Les peuples le maudiront, et les nations seront indignées contre lui.
\VS{25}Mais pour ceux qui le reprennent, ils en retireront de la satisfaction. Et la bénédiction vient sur eux pour leur bonheur.
\VS{26}Celui qui répond avec justesse fait plaisir à celui qui l'écoute.
\VS{27}Prépare ton ouvrage au-dehors, et apprête ton champ, et après, tu bâtiras ta maison.
\VS{28}Ne témoigne pas sans cause contre ton prochain ; car voudrais-tu tromper de tes lèvres\FTNT{Ep. 4:25.} ?
\VS{29}Ne dis pas : Comme il m'a fait, ainsi je lui ferai ; je rendrai à cet homme selon ce qu'il m'a fait.
\VS{30}J'ai passé près du champ de l'homme de paresseux, et près de la vigne d'un homme dépourvu de sens ;
\VS{31}et voici, tout y était monté en chardons, et les ronces avaient couvert la surface, et le mur de pierres était écroulé.
\VS{32}Et j'ai regardé, j'y ai appliqué mon coeur, j'ai vu et j'en ai tiré instruction.
\VS{33}Un peu de sommeil, un peu d'assoupissement, un peu croiser les mains pour dormir ! …
\VS{34}Et la pauvreté te surprendra comme un rôdeur ; et la disette, comme un homme armé.
\Chap{25}
\TextTitle{[Avertissements et conseils]}
\VerseOne{}Ce sont ici aussi des proverbes de Salomon\FTNT{1 R. 4: 32.}, que les gens d'Ezéchias, roi de Juda, ont transcrits.
\VS{2}La gloire de Dieu est de cacher les choses, et la gloire des rois est de sonder les choses.
\VS{3}Les cieux dans leur hauteur, la terre dans sa profondeur, le coeur des rois sont impénétrables.
\VS{4}Ôte de l'argent les scories, et il en sortira un vase pour l'orfèvre.
\VS{5} De même, ôte le méchant de devant le roi, et son trône sera affermi par la justice.
\VS{6}Ne t'élève pas devant le roi et ne te tiens pas à la place des grands.
\VS{7}Car il vaut mieux qu'on te dise: Monte-ici ! Que si l'on t'abaisse devant un seigneur que tes yeux ont vu\FTNT{Lu. 14:8-11.}.
\VS{8}Ne te hâte pas d'entrer en contestation, de peur que tu ne saches que faire à la fin, lorsque ton prochain t'aura confondu.
\VS{9}Plaide ta cause contre ton prochain, mais ne révèle pas le secret d'autrui.
\VS{10}De peur que celui qui l'écoute ne te couvre de honte, et que ton opprobre ne s'efface pas.
\VS{11}Telles que sont des pommes d'or sur des ciselures d'argent, telle est une parole dite quand il faut.
\VS{12}Comme un anneau d'or ou comme un joyau d'or fin, ainsi est l'oreille obéissante pour le sage qui réprimande.
\VS{13}Le messager fidèle est à ceux qui l'envoient, comme la fraîcheur de la neige au temps de la moisson, il restaure l'âme de son maître.
\VS{14}Celui qui se glorifie faussement de ses libéralités, est comme les nuages et le vent sans pluie.
\VS{15}Un prince est fléchi par la patience, et la langue douce brise les os.
\VS{16}As-tu trouvé du miel, manges-en autant qu'il t'en faut, de peur que tu n'en sois rassasié, que tu ne le vomisses.
\VS{17}De même, mets rarement le pied dans la maison de ton prochain, de peur qu'étant rassasié de toi, il ne te haïsse.
\VS{18}Comme une massue, une épée et une flèche aiguë, ainsi est un homme qui porte un faux témoignage contre son prochain.
\VS{19}Comme une dent qui se rompt, un pied qui glisse, telle est la confiance qu'on met en un traître au jour de la détresse.
\VS{20}Celui qui chante des chansons à un coeur affligé est comme celui qui ôte sa robe dans un jour froid, et comme du vinaigre répandu sur du nitre.
\VS{21}Si celui qui te hait a faim, donne-lui à manger du pain ; et s'il a soif, donne-lui à boire de l'eau\FTNT{Mt 5:39-44.}.
\VS{22}Car ce sont des charbons ardents que tu lui mets sur sa tête, et Yahweh te le rendra.
\VS{23}Le vent du nord engendre les averses, et la langue qui médit en cachette un visage irrité.
\VS{24}Il vaut mieux habiter à l'angle d'un toit que de partager la demeure d'une femme querelleuse.
\VS{25}Comme de l'eau fraîche pour une personne fatiguée et lasse, ainsi est une bonne nouvelle venant d'une terre lointaine.
\VS{26}Le juste qui bronche devant le méchant est une fontaine troublée et une source gâtée.
\VS{27}Comme il n'est pas bon de manger beaucoup de miel, aussi n'y a-t-il pas de gloire pour les hommes de rechercher leur gloire avec ardeur.
\VS{28}L'homme qui n'est pas maître de son esprit est comme une ville où il y a une brèche, et qui est sans murailles.
\Chap{26}
\TextTitle{[Avertissements et conseils (suite)]}
\VerseOne{}Comme la neige en été et la pluie pendant la moisson, ainsi la gloire ne convient pas à un insensé.
\VS{2}Comme l'oiseau est prompt à s'échapper et l'hirondelle à s'envoler, ainsi la malédiction sans cause n'atteint pas.
\VS{3}Le fouet est pour le cheval, le mors pour l'âne, et la verge pour le dos des insensés.
\VS{4}Ne réponds pas à l'insensé selon sa folie, de peur que tu ne lui ressembles toi-même.
\VS{5}Réponds à l'insensé selon sa folie, de peur qu'il ne devienne sage à ses propres yeux.
\VS{6}Celui qui envoie des messages par l'intermédiaire d'un insensé, se coupe les pieds et boit la peine du tort qu'il s'est fait.
\VS{7}Faites marcher un homme boiteux, ainsi il en sera d'un proverbe dans la bouche des insensés.
\VS{8}Celui qui donne de la gloire à un insensé, c'est comme s'il jetait une pierre précieuse dans un monceau de pierres.
\VS{9}Comme une épine dans la main d'un homme ivre, ainsi est un proverbe dans la bouche des insensés.
\VS{10}Les puissants donnent de l'ennui à tout le monde, et prennent à leur service les insensés et les passants.
\VS{11}Comme le chien retourne à ce qu'il a vomi, ainsi l'insensé réitère sa folie\FTNT{2 Pi. 2:22.}.
\VS{12}As-tu vu un homme qui croit être sage ? Il y a plus à espérer d'un insensé que de lui.
\VS{13}Le paresseux dit : Il y a un lion rugissant sur le chemin, il y a un lion dans les rues.
\VS{14}Comme une porte tourne sur ses gonds, ainsi fait le paresseux sur son lit.
\VS{15}Le paresseux plonge sa main dans le plat, et il trouve fatigant de la ramener à sa bouche.
\VS{16}Le paresseux se croit plus sage que sept hommes qui répondent avec bon sens.
\VS{17}Celui qui en passant se met en colère pour une dispute qui ne le touche en rien, est comme celui qui prend un chien par les oreilles.
\VS{18}Tel est celui qui fait l'insensé, et qui cependant jette des feux, des flèches, et des choses propres à tuer,
\VS{19}tel est l'homme qui a trompé son ami, et qui après cela dit : N'était-ce pas pour plaisanter ?
\VS{20}Le feu s'éteint faute de bois, ainsi quand il n'y a pas de rapporteurs la querelle s'apaise.
\VS{21}Le charbon est pour faire de la braise, et le bois pour faire du feu, et l'homme querelleur pour exciter des querelles.
\VS{22}Les paroles d'un rapporteur sont comme des friandises, elles descendent jusqu'au fond des entrailles.
\VS{23}Les lèvres brûlantes de zèle et le coeur mauvais sont comme des scories d'argent appliquées sur un vase de terre.
\VS{24}Celui qui a de la haine se déguise par ses discours, mais au-dedans de lui il nourrit la trahison.
\VS{25}Lorsqu'il prend une voix douce, ne le crois pas, car il y a sept abominations dans son coeur.
\VS{26}S'il cache sa haine sous la dissimulation, sa méchanceté sera révélée dans l'assemblée.
\VS{27}Celui qui creuse la fosse y tombe ; et la pierre retourne sur celui qui la roule\FTNT{Ps. 7:16-17 ; Ps. 57:7 ; Ec. 10:8.}.
\VS{28}La fausse langue hait ceux qu'elle a écrasés ; et la bouche flatteuse prépare la ruine.
\Chap{27}
\TextTitle{[Avertissements et conseils (suite)]}
\VerseOne{}Ne te vante point du lendemain, car tu ne sais pas quelle chose le jour enfantera\FTNT{Ja. 4:13-15.}.
\VS{2}Qu'un autre te loue, et non pas ta propre bouche ; que ce soit l'étranger, et non pas tes lèvres.
\VS{3}La pierre est pesante, et le sable est lourd ; mais l'irritation de l'insensé est plus pesante que tous les deux.
\VS{4}Il y a de la cruauté dans la fureur, et du débordement dans la colère ; mais qui pourra subsister devant la jalousie ?
\VS{5}Mieux vaut une réprimande ouverte qu'une amitié cachée.
\VS{6}Les blessures d'un ami sont dignes de confiance, mais les baisers d'un ennemi sont à craindre\FTNT{Il est question ici de Judas}.
\VS{7}L'âme rassasiée foule aux pieds les rayons de miel ; mais l'âme qui a faim trouve doux même ce qui est amer.
\VS{8}Tel qu'est un oiseau errant loin de son nid, tel est l'homme qui s'écarte de son lieu.
\VS{9}L'huile et les parfums réjouissent le coeur, et il en est ainsi de la douceur d'un ami dont le fruit est un conseil qui vient du coeur.
\VS{10}Ne quitte point ton ami ni l'ami de ton père, et n'entre pas dans la maison de ton frère au jour de ta détresse ; car le voisin qui est proche vaut mieux que le frère qui est éloigné.
\VS{11}Mon fils sois sage, et réjouis mon coeur, afin que j'aie de quoi répondre à celui qui m'outrage.
\VS{12}L'homme prudent voit le malheur arriver et se cache ; mais les stupides passent outre et en payent l'amende.
\VS{13}Quand quelqu'un se portera garant pour l'étranger, prends son vêtement, exige de lui des gages, à cause des étrangers.
\VS{14}Celui qui bénit son ami à haute voix, se levant de grand matin, on le lui comptera comme une malédiction.
\VS{15}Une gouttière continuelle en un jour de grosse pluie, et une femme querelleuse, cela se ressemble.
\VS{16}Celui qui veut la retenir est comme s'il voulait arrêter le vent, et retenir dans sa main une huile qui s'écoule.
\VS{17}Comme le fer aiguise le fer, ainsi l'homme aiguise la personnalité de son prochain.
\VS{18}Comme celui qui garde le figuier mangera de son fruit, ainsi celui qui garde son maître sera honoré.
\VS{19}Comme dans l'eau le visage répond au visage, ainsi le coeur d'un homme répond à celui d'un autre homme.
\VS{20}Le scheol et le gouffre ne sont jamais rassasiés ; de même, les yeux des hommes sont insatiables\FTNT{Ec. 1:8 ; 2 Pi. 2:14.}.
\VS{21}Le fourneau est pour éprouver l'argent, et le creuset pour l'or ; mais un homme est jugé d'après sa renommée.
\VS{22}Quand tu pilerais un insensé dans un mortier, parmi du grain qu'on pile avec un pilon, sa stupidité ne se détacherait pas de lui.
\VS{23}Sois diligent à reconnaître l'état de chacune de tes brebis, et applique ton coeur aux troupeaux.
\VS{24}Car l'abondance ne dure pas à toujours, et une couronne dure-t-elle d'âge en âge ?
\VS{25}Le foin s'enlève et la verdure paraît, et les herbes des montagnes sont amassées.
\VS{26}Les agneaux sont pour te vêtir, et les boucs pour payer le champ ;
\VS{27}et l'abondance du lait des chèvres sera pour ta nourriture et celle de ta maison, et pour la subsistance de tes servantes.
\Chap{28}
\TextTitle{[Avertissements et conseils (suite)]}
\VerseOne{}Le méchant prend la fuite sans qu'on le poursuive, mais les justes seront assurés comme un jeune lion.
\VS{2}Il y a plusieurs chefs, à cause de la rébellion d'un pays, mais pour l'amour de l'homme avisé et intelligent, il y aura prolongation du même gouvernement.
\VS{3}L'homme qui est pauvre et qui opprime les pauvres, est comme une pluie violente qui cause la disette du pain.
\VS{4}Ceux qui abandonnent la loi louent le méchant, mais ceux qui gardent la loi leur font la guerre.
\VS{5}Les gens adonnés au mal n'entendent point ce qui est droit ; mais ceux qui cherchent Yahweh comprennent tout.
\VS{6}Le pauvre qui marche dans son intégrité vaut mieux que le pervers qui marche par deux chemins et qui est riche.
\VS{7}Celui qui garde la loi est un fils prudent, mais celui qui entretient les gourmands fait honte à son père.
\VS{8}Celui qui augmente ses biens par l'intérêt et l'usure, les amasse pour celui qui en fera des libéralités aux pauvres.
\VS{9}Celui qui détourne son oreille pour ne pas écouter la loi, sa prière même est une abomination\FTNT{La prière doit être faite selon la Parole de Dieu, en conformité avec sa volonté. Le Seigneur n'exauce que ceux qui obéissent à sa Parole (Mt. 6:9-10 ; Jn. 9:31 ; Jn. 15:7 ; 1 Jn. 5:14-15).}.
\VS{10}Celui qui égare les hommes droits dans le mauvais chemin tombera dans la fosse qu'il aura creusée, mais ceux qui sont intègres hériteront le bonheur.
\VS{11}L'homme riche pense être sage, mais le pauvre qui est intelligent le sondera.
\VS{12}Quand les justes se réjouissent, la gloire est grande, mais quand les méchants sont élevés, chacun se déguise.
\VS{13}Celui qui cache ses transgressions ne prospère point, mais celui qui les confesse et les délaisse, obtient miséricorde\FTNT{Ec. 1:8 ; 2 Pi. 2:14.}.
\VS{14}Heureux est l'homme qui est continuellement dans la crainte, mais celui qui endurcit son coeur tombera dans la calamité.
\VS{15}Le méchant qui domine sur un peuple pauvre est un lion rugissant et comme un ours quêtant sa proie.
\VS{16}Le conducteur qui manque d'intelligence fait beaucoup d'extorsions, mais celui qui hait le gain déshonnête prolonge ses jours.
\VS{17}L'homme chargé du sang d'une personne fuira jusqu'à la fosse sans qu'aucun ne le retienne.
\VS{18}Celui qui marche dans l'intégrité sera sauvé, mais le pervers qui suit deux chemins tombera tout à coup.
\VS{19}Celui qui laboure sa terre sera rassasié de pain, mais celui qui suit les fainéants sera accablé de misère.
\VS{20}L'homme fidèle abondera en bénédictions, mais celui qui se hâte de s'enrichir ne restera pas impuni.
\VS{21}Il n'est pas bon d'avoir égard à l'apparence des personnes, car pour un morceau de pain l'homme commet un crime.
\VS{22}L'homme qui a l'oeil malin se hâte pour avoir des richesses, et il ne sait pas que la disette lui arrivera.
\VS{23}Celui qui reprend les hommes obtient ensuite plus de faveur que celui qui flatte de sa langue.
\VS{24}Celui qui pille son père ou sa mère, et qui dit que ce n'est point un péché, est compagnon de l'homme dissipateur.
\VS{25}Celui qui a l'âme enflée excite les querelles, mais celui qui se confie en Yahweh sera rassasié.
\VS{26}Celui qui se confie dans son propre coeur est un fou, mais celui qui marche sagement sera délivré.
\VS{27}Celui qui donne au pauvre n'aura point de disette, mais celui qui en détourne ses yeux abondera en malédictions.
\VS{28}Quand les méchants s'élèvent, l'homme se cache ; mais quand ils périssent, les justes se multiplient.
\Chap{29}
\TextTitle{[Avertissements et conseils (suite)]}
\VerseOne{}L'homme qui étant repris, raidit son cou, sera subitement brisé et sans qu'il y ait de guérison.
\VS{2}Quand les justes sont nombreux, le peuple se réjouit ; mais quand le méchant domine, le peuple gémit.
\VS{3}L'homme qui aime la sagesse, réjouit son père, mais celui qui se plaît avec les femmes prostituées dissipe ses richesses.
\VS{4}Le roi affermit le pays par la justice, mais l'homme qui est adonné aux présents le ruinera.
\VS{5}L'homme qui flatte son prochain lui tend un piège sous ses pas.
\VS{6}Le péché de l'homme méchant lui tend un piège dangereux, mais le juste triomphe et se réjouit.
\VS{7}Le juste prend connaissance de la cause des pauvres, mais le méchant n'en prend pas connaissance.
\VS{8}Les hommes moqueurs troublent la ville, mais les sages apaisent la colère.
\VS{9}Un homme sage qui conteste avec un insensé, qu'il se fâche ou qu'il rie, la paix n'aura pas lieu.
\VS{10}Les hommes de sang ont en haine l'homme intègre, mais les hommes droits tiennent chère sa vie.
\VS{11}L'insensé pousse au-dehors tout ce qu'il a dans l'esprit, mais le sage le calme et le retient en arrière.
\VS{12}Tous les serviteurs d'un prince qui prêtent l'oreille à la parole de mensonge sont méchants.
\VS{13}Le pauvre et l'oppresseur se rencontrent, c'est Yahweh qui illumine les yeux de l'un et de l'autre.
\VS{14}Le trône du roi qui fait justice selon la vérité aux pauvres, sera établi à perpétuité.
\VS{15}La verge et la réprimande donnent la sagesse, mais l'enfant livré à lui-même fait honte à sa mère.
\VS{16}Quand les méchants se multiplient, les péchés s'accroissent, mais les justes verront leur ruine.
\VS{17}Corrige ton fils, et il te donnera du repos, et il procurera du plaisir à ton âme.
\VS{18}Lorsqu'il n'y a pas de vision\FTNT{Le manque de vision n'est bon pour personne. Dieu donne une vision aux personnes qu'il a appelées. La vision peut être un songe, une directive, une prophétie, etc. Il s'agit des objectifs à atteindre.}, le peuple est sans frein, mais heureux est celui qui garde la loi !
\VS{19}L'esclave ne se corrige pas par des paroles, même s'il comprend, il n'obéit pas.
\VS{20}As-tu vu un homme irréfléchi dans ses paroles ? Il y a plus à espérer d'un insensé que de lui.
\VS{21}Le serviteur qu'on a traité délicatement dès sa jeunesse finit par se croire un fils.
\VS{22}L'homme coléreux excite des querelles, et l'homme furieux commet beaucoup de péchés.
\VS{23}L'orgueil de l'homme l'abaisse, mais celui qui est humble d'esprit obtient la gloire\FTNT{Mt. 23:12 ; Lu. 14:11 ; 1 Pi. 5:5.}.
\VS{24}Celui qui partage avec un voleur hait son âme ; il entend la malédiction, et il ne révèle rien.
\VS{25}La crainte qu'on a des hommes tend un piège, mais celui qui se confie en Yahweh est élevé dans une haute retraite.
\VS{26}Plusieurs recherchent la face de celui qui domine, mais c'est de Yahweh que vient le jugement des hommes.
\VS{27}L'homme inique est en abomination aux justes, et celui dont la voie est droite est en abomination au méchant.
\Chap{30}
\TextTitle{[Proverbe d'Agur]}
\VerseOne{}Les paroles d'Agur, fils de Jaké, à savoir la sentence prononcée par cet homme pour Ithiel, pour Ithiel et Ucal.
\VS{2}Certainement je suis le plus stupide de tous les hommes, et il n'y a pas en moi l'intelligence humaine.
\VS{3}Et je n'ai pas appris la sagesse ; et je n'ai pas la connaissance des saints.
\VS{4}Qui est celui qui est monté aux cieux, ou qui en est descendu\FTNT{Jn. 3:13 ; Ro. 10:6-7.} ? Qui est celui qui a recueilli le vent dans le creux de sa main, qui a serré les eaux dans son manteau, qui a dressé toutes les bornes de la terre ? Quel est son nom, et quel est le nom de son fils, le sais-tu ?
\VS{5}Toute la parole de Dieu est éprouvée ; il est un bouclier pour ceux qui se réfugient en lui\FTNT{Ps. 18:31 ; Ps. 115:9-11.}.
\VS{6}N'ajoute rien à ses paroles, de peur qu'il ne te reprenne et que tu ne sois trouvé menteur.
\VS{7}Je te demande deux choses : Ne me les refuse pas durant ma vie.
\VS{8}Eloigne de moi la vanité et la parole mensongère ; ne me donne ni pauvreté ni richesse, nourris-moi du pain qui m'est nécessaire.
\VS{9}De peur que dans l'abondance je ne te renie, et que je ne dise : Qui est Yahweh ? Ou que dans la pauvreté, je ne dérobe et que je ne porte atteinte au Nom de mon Dieu.
\VS{10}N'accuse pas un serviteur devant son maître, de peur que ce serviteur ne te maudisse, et qu'il ne t'en arrive du mal.
\VS{11}Il est une race de gens qui maudit son père et qui ne bénit pas sa mère.
\VS{12}Il est une race de gens qui croit être pure, et qui toutefois n'est point lavée de son ordure.
\VS{13}Il est une race de gens dont les yeux sont fort hautains, et les paupières élevées.
\VS{14}Il est une race de gens dont les dents sont des épées, et les mâchoires sont des couteaux pour dévorer les malheureux sur la terre et les pauvres d'entre les hommes.
\VS{15}La sangsue a deux filles qui disent : Apporte ! Apporte ! Il y a trois choses qui sont insatiables, il y en a même quatre qui ne disent point : C'est assez !
\VS{16}Le scheol, la matrice stérile, la terre qui n'est pas rassasiée d'eau, et le feu qui ne dit jamais : C'est assez !
\VS{17}L'oeil de celui qui se moque de son père et qui méprise l'enseignement de sa mère, les corbeaux des torrents le crèveront, et les petits de l'aigle le mangeront.
\VS{18}Il y a trois choses qui sont trop merveilleuses pour moi, même quatre, que je ne connais point :
\VS{19}La trace de l'aigle dans le ciel, la trace du serpent sur un rocher, le chemin d'un navire au milieu de la mer, et la trace de l'homme chez la jeune femme.
\VS{20}Telle est la trace de la femme adultère : Elle mange, et s'essuie la bouche, puis elle dit : Je n'ai pas commis d'iniquité.
\VS{21}La terre tremble pour trois choses, même pour quatre, qu'elle ne peut supporter :
\VS{22}Pour l'esclave quand il vient à régner, pour l'insensé quand il est rassasié de pain,
\VS{23}pour la femme odieuse quand elle se marie, et pour la servante quand elle hérite de sa maîtresse.
\VS{24}Il y a quatre choses des plus petites de la terre qui toutefois sont bien sages entre les sages :
\VS{25}Les fourmis, qui sont un peuple sans force, et qui néanmoins préparent durant l'été leur nourriture ;
\VS{26}les damans, qui sont un peuple qui n'est pas puissant, et qui néanmoins font leurs maisons dans les rochers ;
\VS{27}les sauterelles, qui n'ont point de roi, et qui toutefois sortent toutes par divisions ;
\VS{28}les lézards que tu peux saisir avec les mains, et qui sont pourtant dans les palais des rois.
\VS{29}Il y a trois choses qui ont une belle allure, même quatre, qui ont une belle démarche :
\VS{30}Le lion, qui est le plus fort d'entre les animaux, et qui ne recule pas à la rencontre de qui que ce soit ;
\VS{31}le cheval, qui a les flancs bien troussés ; le bouc, et le roi devant qui personne ne résiste.
\VS{32}Si tu t'es conduit follement en t'emportant, et si tu as des mauvaises intentions, mets la main sur ta bouche.
\VS{33}Comme celui qui bat le lait en fait sortir le beurre, et comme celui qui presse le nez en fait sortir le sang, ainsi celui qui provoque la colère excite la querelle.
\Chap{31}
\TextTitle{Proverbe de Lemuel}
\VerseOne{}Les paroles du roi Lemuel et l'instruction que sa mère lui donna.
\VS{2}Quoi, mon fils ? Quoi, fils de mes entrailles ? Eh quoi, mon fils, pour lequel j'ai tant fait de voeux ?
\VS{3}Ne livre pas ta vigueur aux femmes et tes voies à celles qui perdent les rois.
\VS{4}Lemuel, ce n'est point aux rois, ce n'est point aux rois de boire le vin, ni aux princes de boire la cervoise\FTNT{La cervoise est une bière faite avec de l'orge ou d'autres céréales.}
\VS{5}de peur qu'ayant bu, ils n'oublient ce qui a été prescrit, et qu'ils n'altèrent le jugement de tous les pauvres affligés.
\VS{6}Donnez de la cervoise à celui qui va périr, et du vin à celui qui a l'amertume dans le coeur ;
\VS{7}afin qu'il en boive, et qu'il oublie sa pauvreté, et ne se souvienne plus de ses peines.
\VS{8}Ouvre ta bouche en faveur du muet, pour la cause de tous les fils délaissés qui vont périr.
\VS{9}Ouvre ta bouche, fais justice, et plaide pour le malheureux et l'indigent.
\TextTitle{[La femme vertueuse]}
\VS{10}[Aleph.] Qui est-ce qui trouvera une femme vertueuse ? Car son prix surpasse de beaucoup les perles.
\VS{11}[Beth.] Le coeur de son mari a confiance en elle, et il ne manquera point de dépouilles.
\VS{12}[Guimel.] Elle lui fera du bien tous les jours de sa vie, et jamais du mal.
\VS{13}[Daleth.] Elle cherche de la laine et du lin, et elle en fait ce qu'elle veut avec ses mains.
\VS{14}[He.] Elle est semblable aux navires d'un marchand, elle amène son pain de loin.
\VS{15}[Vav.] Elle se lève lorsqu'il est encore nuit, elle donne la nourriture nécessaire à sa maison et elle donne à ses servantes leur tâche.
\VS{16}[Zayin.] Elle pense à un champ, et l'acquiert ; et elle plante la vigne du fruit de ses mains.
\VS{17}[Heth.] Elle ceint ses reins de force, et affermit ses bras.
\VS{18}[Teth.] Elle sent que ce qu'elle gagne est bon ; sa lampe ne s'éteint pas pendant la nuit.
\VS{19}[Yod.] Elle met sa main à la quenouille, et ses doigts tiennent le fuseau.
\VS{20}[Kaf.] Elle tend sa main au malheureux, elle tend ses mains à l'indigent.
\VS{21}[Lamed.] Elle ne craint point la neige pour sa famille, car toute sa famille est vêtue de vêtements doubles.
\VS{22}[Mem.] Elle se fait des couvertures, le fin lin et l'écarlate sont ce dont elle s'habille.
\VS{23}[Nun.] Son mari est considéré aux portes, lorsqu'il siège avec les anciens du pays.
\VS{24}[Samech.] Elle fait des chemises et les vend, et elle livre des ceintures au marchand.
\VS{25}[Ayin.] Elle est revêtue de force et de gloire, elle se rit du jour à venir.
\VS{26}[Pe.] Elle ouvre sa bouche avec sagesse, et la loi de la charité est sur sa langue.
\VS{27}[Tsade.] Elle veille sur ce qui se passe dans sa maison, et elle ne mange pas le pain de la paresse.
\VS{28}[Qof.] Ses fils se lèvent et la disent bienheureuse ; son mari aussi, et il la loue, en disant :
\VS{29}[Resh.] Plusieurs filles se sont conduites vertueusement, mais toi, tu les surpasses toutes.
\VS{30}[Shin.] La grâce est trompeuse et la beauté vaine, mais la femme qui craint Yahweh est celle qui sera louée.
\VS{31}[Tav.] Récompensez-la du fruit de ses mains, et que ses oeuvres la louent aux portes.
\PPE{}
\end{multicols}
