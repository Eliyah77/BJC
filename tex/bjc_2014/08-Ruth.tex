\ShortTitle{Ruth}\BookTitle{Ruth}\BFont
\noindent\hrulefill
{\footnotesize
\textit{
\bigskip
{\centering{}
\\Auteur : Inconnu
\\(Heb. : Ruwth)
\\Signification : Amitié, une amie
\\Thème : Les origines de la famille messianique
\\Date de rédaction : 11\up{ème} siècle av. J.-C.\\}
}
%\bigskip
\textit{
\\Au temps des juges, une famine qui frappa le pays de Juda poussa Elimélec, sa femme Naomi et ses deux fils à s'installer dans le pays de Moab. Ils y rencontrèrent Ruth qui devint ensuite la belle fille de Naomi. Après la mort de son mari, cette Moabite montra son attachement à sa belle-mère et au Dieu de celle-ci, qui devint le sien. Sa détermination, sa fidélité, son obéissance et son humilité bouleversèrent sa destinée.
%\bigskip
\\Son histoire est l'image du rachat des nations par Jésus-Christ, dont elle fut l'une des ancêtres.\bigskip
}
}
\par\nobreak\noindent\hrulefill
\begin{multicols}{2}
\Chap{1}
\TextTitle{De Juda à Moab}
\VerseOne{}Or il arriva du temps où les Juges jugeaient, qu'il y eut une famine au pays, et un homme de Bethléhem de Juda s'en alla, pour demeurer en quelque lieu sur la terre de Moab, lui et sa femme, et ses deux fils.
\VS{2}Et le nom de cet homme était Elimélec, le nom de sa femme Naomi, et les noms de ses deux fils Machlon et Kiljon, Ephratiens, de Bethléhem de Juda ; et ils vinrent sur la terre de Moab, et y demeurèrent là.
\VS{3}Or Elimélec, mari de Naomi, mourut et elle resta avec ses deux fils,
\VS{4}qui prirent pour eux des femmes Moabites, dont l'une s'appelait Orpa, et la seconde Ruth\FTNT{Ruth, la Moabite, dont l'ancêtre était issu d'une relation incestueuse (Ge. 19:36-37), est devenue l'ancêtre du Messie (Mt. 1:5-6).}, et ils demeurèrent là environ dix ans.
\VS{5}Puis Machlon et Kiljon moururent aussi tous les deux ; ainsi cette femme demeura là, privée de ses deux fils et de son mari.
\TextTitle{Naomi renvoie ses belles-filles dans leur famille}
\VS{6}Puis elle se leva avec ses belles-filles afin de quitter la terre de Moab, car elle entendit, au pays de Moab, que Yahweh avait visité son peuple en leur donnant du pain.
\VS{7}Ainsi elle partit du lieu où elle était, avec ses deux belles-filles, et elle se mirent en chemin pour retourner au pays de Juda.
\VS{8}Naomi dit à ses deux belles-filles : Allez, retournez chacune dans la maison de sa mère ! Que Yahweh vous fasse du bien, comme vous en avez fait à ceux qui sont morts, et à moi.
\VS{9}Que Yahweh vous fasse trouver du repos à chacune dans la maison de son mari ! Et elle les baisa mais elles élevèrent leur voix, et pleurèrent ;
\VS{10}et elles lui dirent : Non, nous retournerons avec toi vers ton peuple.
\TextTitle{Fidélité de Ruth à Naomi}
\VS{11}Et Naomi répondit : Retournez, mes filles ! Pourquoi viendriez-vous avec moi ? Ai-je encore des fils en mon ventre, afin que vous les ayez pour maris ?
\VS{12}Retournez, mes filles, allez-vous-en, car je suis trop âgée pour être remariée. Et quand je dirais : J'ai de l'espérance, quand même dès cette nuit je serais avec un mari, et que j'enfanterais des fils,
\VS{13}les attendriez-vous jusqu'à ce qu'ils fussent grands ? Refuseriez-vous donc des maris ? Non, mes filles ! Certes je suis dans une plus grande amertume que vous, parce que la main de Yahweh s'est éloignée de moi.
\VS{14}Et elles levèrent leur voix, et pleurèrent encore. Orpa baisa sa belle-mère, mais Ruth s'attacha à elle.
\VS{15}Et Naomi dit à Ruth : Voici, ta belle-sœur est retournée vers son peuple et vers ses dieux ; retourne après ta belle-sœur.
\VS{16}Mais Ruth répondit : Ne me prie pas de te laisser, pour m’éloigner de toi, car où tu iras, j’irai ; et où tu demeureras, je demeurerai ; ton peuple sera mon peuple, et ton Dieu sera mon Dieu ;
\VS{17}Là où tu mourras je mourrai, et j'y serai ensevelie. Que Yahweh me traite avec la dernière rigueur, si autre chose que la mort me sépare de toi!
\VS{18}Naomi donc voyant qu’elle était résolue d’aller avec elle, cessa de lui en parler.
\TextTitle{Arrivée à Bethléhem}
\VS{19}Et elles marchèrent toutes deux jusqu'à ce qu'elles arrivèrent à Bethléhem, et comme elles furent entrées dans Bethléhem, toute la ville se mit à parler sur son sujet, et les femmes dirent : N’est-ce pas ici Naomi ?
\VS{20}Et elle leur répondit : Ne m'appelez pas Naomi ; appelez-moi Mara, car le Tout-Puissant m'a remplie d'amertume.
\VS{21}Je m’en allai pleine de biens, et Yahweh me ramène à vide. Pourquoi m'appelleriez-vous Naomi, après que Yahweh m’a abattue, et que le Tout-Puissant m’a affligée ?
\VS{22}C'est ainsi que revinrent de la terre de Moab, Naomi et avec elle, Ruth, sa belle-fille, la Moabite, et elles entrèrent dans Bethléhem au commencement de la moisson des orges.
\Chap{2}
\TextTitle{Ruth trouve grâce aux yeux de Boaz}
\VerseOne{}Or le mari de Nahomi avait là un parent, un homme puissant et riche, de la famille d'Elimélec, et qui s'appelait Boaz.
\VS{2}Et Ruth la Moabite dit à Naomi : Je te prie laisse-moi aller glaner des épis dans le champ de celui aux yeux duquel je trouverai grâce. Et elle lui répondit : Va, ma fille.
\VS{3}Elle s'en alla donc, et entra dans un champ, et glana après les moissonneurs. Et il se trouva par hasard que la parcelle de champ appartenait à Boaz, qui était de la famille d'Elimélec.
\VS{4}Or voici, Boaz vint de Bethléhem, et il dit aux moissonneurs : Yahweh soit avec vous ! Et ils lui répondirent : Yahweh te bénisse !
\VS{5}Et Boaz dit à son serviteur qui était établi sur les moissonneurs : A qui est cette jeune fille ?
\VS{6}Et le serviteur qui était établi sur les moissonneurs répondit et dit : C'est une jeune femme Moabite, qui est revenue avec Naomi de la terre de Moab.
\VS{7}Et elle nous a dit : Je vous prie que je glane, et que je ramasse quelques gerbes, après les moissonneurs. Et elle est venue, et elle a été debout depuis le matin jusqu'à cette heure, et ne s'est reposée qu'un moment dans la maison.
\VS{8}Alors Boaz dit à Ruth : Ecoute, ma fille, ne va pas glaner dans un autre champ et même, ne pars pas au loin, mais reste ici auprès de mes jeunes filles.
\VS{9}Regarde le champ où l’on moissonne, et va après elles ; n’ai-je pas défendu à mes garçons de te toucher ? Et si tu as soif, tu iras aux vases, et tu boiras de ce que les garçons auront puisé.
\VS{10}Alors elle tomba sur sa face, et se prosterna contre terre, et elle lui dit : Comment ai-je trouvé grâce à tes yeux, pour que tu prêtes attention à moi, moi qui suis une étrangère ?
\VS{11}Boaz lui répondit et dit : Tout ce que tu as fait pour ta belle-mère depuis que ton mari est mort, m'a été entièrement rapporté, et comment tu as laissé ton père, et ta mère, et le pays de ta naissance, et tu es venue vers un peuple que tu n’avais point connu auparavant.
\VS{12}Que Yahweh récompense ton œuvre, et que ton salaire soit entier de la part de Yahweh, le Dieu d'Israël, sous les ailes duquel tu es venue te réfugier !
\VS{13}Et elle dit : Mon seigneur, que je trouve grâce à tes yeux ! Car tu m'as consolée, et tu as parlé selon le cœur de ta servante. Et pourtant je ne suis pas, moi, comme l'une de tes servantes.
\VS{14}Boaz lui dit encore à l’heure du repas : Approche-toi d’ici, et mange du pain, et trempe ton morceau dans le vinaigre ; et elle s’assit à côté des moissonneurs, et il lui donna du grain rôti, et elle en mangea, et fut rassasiée, et garda le reste.
\VS{15}Puis elle se leva pour glaner et Boaz ordonna à ses garçons : Qu'elle glane même entre les gerbes, et ne lui faites pas honte.
\VS{16}Et même vous lui retirerez quelques poignées ; vous les lui laisserez, et elle les recueillera, et vous ne l’en censurerez point.
\VS{17}Elle glana donc dans le champ jusqu'au soir, et elle battit ce qu'elle avait recueilli, et il y eut environ un épha d'orge.
\VS{18}Et elle l'emporta, entra dans la ville, et sa belle-mère vit ce qu'elle avait glané. Elle sortit aussi ce qu’elle avait gardé de reste, après avoir été rassasiée, et elle le lui donna.
\VS{19}Alors sa belle-mère lui dit : Où as-tu glané aujourd'hui, et où as-tu travaillé ? Béni soit celui qui t'a reconnue ! Et elle raconta à sa belle-mère chez qui elle avait travaillé, et dit : L'homme chez qui j'ai travaillé aujourd'hui s'appelle Boaz.
\VS{20}Et Naomi dit à sa belle-fille : Béni soit-il de Yahweh, puisqu'il a la même bonté pour les vivants, comme il en eut pour ceux qui sont morts ! Et Naomi lui dit : Cet homme est un proche parent, il est un de ceux qui ont sur nous le droit de rachat\FTNT{Le droit de rachat : Le rédempteur est celui qui rachète une personne moyennant le paiement d'une rançon. Sous la Première Alliance, le rachat se faisait soit par un frère, soit par un proche parent, pour la libération de celui qui s'était fait esclave ou qui avait aliéné sa propriété ou son bien (Lé. 25:25 et 48). Sous la Nouvelle Alliance, Jésus-Christ est désormais notre rédempteur. Romains 3:23-24 nous dit : « Car tous ont péché, et sont entièrement privés de la gloire de Dieu. Et ils sont gratuitement justifiés par sa grâce, par la rédemption qui est en Jésus-Christ ». Christ nous a rachetés de la malédiction de la loi en se donnant lui-même pour nous afin de nous délivrer de toute iniquité (Ga. 3:13 ; Ti. 2:14). Dieu s'est fait homme (Hé. 2:14-17) afin de mieux nous libérer de l'esclavage du diable par sa mort à la croix de Golgotha (Es. 60:16).}.
\VS{21}Et Ruth la Moabite dit : Il m'a même dit : Reste avec mes garçons jusqu'à ce qu'ils aient achevé toute la moisson qui m'appartient.
\VS{22}Et Naomi dit à Ruth, sa belle-fille : Ma fille, il est bon que tu sortes avec ses jeunes filles, et qu'on ne te rencontre pas dans un autre champ.
\VS{23}Elle resta donc avec les jeunes filles de Boaz, afin de glaner jusqu'à la fin de la moisson des orges et la moisson des froments. Puis elle demeurait avec sa belle-mère.
\Chap{3}
\TextTitle{Ruth obéit soigneusement aux instructions}
\VerseOne{}Naomi, sa belle-mère, lui dit : Ma fille, je voudrais chercher ton repos, afin que tu sois heureuse.
\VS{2}Maintenant Boaz, avec les servantes duquel tu as été, n'est-il pas de notre parenté ? Voici, il doit vanner cette nuit les orges qui ont été foulées dans l'aire.
\VS{3}Lave-toi et oins-toi, puis mets tes habits, et descends dans l'aire. Ne te fais pas connaître à lui, jusqu'à ce qu'il ait achevé de manger et de boire.
\VS{4}Quand il se couchera, découvre le lieu où il se couche. Ensuite, entre, découvre ses pieds, et couche-toi. Il te dira ce que tu as à faire.
\VS{5}Elle lui répondit : Je ferai tout ce que tu as dit.
\VS{6}Elle descendit à l'aire, et fit tout ce que sa belle-mère lui avait ordonné.
\VS{7}Boaz mangea et but, et son cœur était joyeux. Il vint se coucher à l'extrémité d'un tas de gerbes. Ruth vint secrètement, découvrit ses pieds, et se coucha.
\VS{8}Au milieu de la nuit, cet homme eut peur ; il se retourna et retira ses pieds, car voici, une femme était couchée à ses pieds.
\VS{9}Il dit : Qui es-tu ? Elle répondit : Je suis Ruth, ta servante ; étends le pan de ta robe sur ta servante, car tu as droit de rachat.
\VS{10}Et il dit : Ma fille, que Yahweh te bénisse ! Ce dernier trait de bonté me réjouit plus que le premier, car tu n'es pas allée après des jeunes gens, pauvres ou riches.
\VS{11}Maintenant, ma fille, ne crains pas ; je te ferai tout ce que tu me diras ; car toute la porte de mon peuple sait que tu es une femme vertueuse.
\VS{12}Il est bien vrai que j'ai droit de rachat, mais il existe un autre plus proche que moi, qui a le droit de rachat.
\VS{13}Passe ici la nuit, et demain, si cet homme veut user envers toi du droit de rachat, à la bonne heure, qu'il te rachète ; mais s'il ne lui plaît pas de te racheter, moi je te rachèterai, Yahweh est vivant ! Couche-toi jusqu'au matin.
\VS{14}Elle se coucha à ses pieds jusqu'au matin, et elle se leva avant qu'on puisse se reconnaître l'un l'autre. Boaz dit : Qu'on ne sache pas qu'une femme est entrée dans l'aire.
\VS{15}Et il dit : Donne-moi le manteau qui est sur toi, et tiens-le. Elle le tint, et il mesura six mesures d'orge, qu'il posa sur elle. Puis il entra dans la ville.
\VS{16}Ruth revint auprès de sa belle-mère, et Naomi dit : Est-ce toi ma fille ? Ruth lui raconta tout ce que cet homme avait fait pour elle.
\VS{17}Elle dit : Il m'a donné ces six mesures d'orge, en disant : Tu n'iras pas à vide vers ta belle-mère.
\VS{18}Et Naomi dit : Ma fille, assieds-toi ici jusqu'à ce que tu saches ce que l'affaire deviendra, car cet homme ne se donnera pas de repos, qu'il n'ait achevé cette affaire aujourd'hui.
\Chap{4}
\TextTitle{Boaz exerce son droit de rachat}
\VerseOne{}Boaz monta à la porte, et s'y assit. Or voici, celui qui avait le droit de rachat, et dont Boaz avait parlé, passa. Boaz lui dit : Ah ! Détourne-toi, reste ici, toi un tel. Et il se détourna, et s'assit.
\VS{2}Boaz prit dix hommes d'entre les anciens de la ville, et leur dit : Asseyez-vous ici. Et ils s'assirent.
\VS{3}Puis il dit à celui qui avait le droit de rachat : Naomi qui est revenue de la terre de Moab, a vendu la parcelle du champ qui appartenait à notre frère Elimélec.
\VS{4}J'ai parlé à tes oreilles afin de te le faire savoir et te le dire : Acquiers-la en la présence de ceux qui sont assis ici et en présence des anciens de mon peuple. Si tu veux racheter par droit de rachat, rachète-la ; mais si tu ne veux pas la racheter, déclare-le-moi, afin que je le sache. Car il n'y a pas d'autre que toi qui ait le droit de rachat, et je l'ai après toi. Et il dit : je rachèterai.
\VS{5}Boaz dit : Le jour où tu acquerras le champ de la main de Naomi, tu l'acquerras aussi de Ruth la Moabite, femme du défunt, pour maintenir le nom du défunt dans son héritage.
\VS{6}Et celui qui avait le droit de rachat dit : Je ne puis pas racheter pour mon compte, de peur de détruire mon héritage ; prends pour toi le droit de rachat, car je ne puis pas le racheter.
\VS{7}Autrefois en Israël, pour confirmer une affaire quelconque relative à un rachat ou à un échange, l'homme ôtait son soulier et le donnait à son parent : C'était là, en Israël, un témoignage qu'on cédait son droit.
\VS{8}Celui qui avait le droit de rachat dit à Boaz : Acquiers-le pour toi ! Et il ôta son soulier.
\VS{9}Alors Boaz dit aux anciens et à tout le peuple : Vous êtes aujourd'hui témoins que j'ai acquis de la main de Naomi tout ce qui appartenait à Elimélec, à Kiljon et à Machlon.
\VS{10}Et que je me suis également acquis pour femme Ruth la Moabite, femme de Machlon, pour maintenir le nom du défunt dans son héritage, et afin que le nom du défunt ne soit pas retranché d'entre ses frères et de la porte de sa ville. Vous en êtes témoins aujourd'hui !
\TextTitle{Boaz prend Ruth pour femme}
\VS{11}Tout le peuple qui était à la porte et les anciens dirent : Nous en sommes témoins ! Que Yahweh rende la femme qui entre dans ta maison semblable à Rachel et à Léa, qui ont bâti toutes deux, la maison d'Israël ! Montre ta puissance dans Ephrata et proclame ton nom dans Bethléhem !
\VS{12}Puisse la postérité que Yahweh te donnera de cette jeune femme, rendre ta maison semblable à la maison de Pérets, que Tamar enfanta à Juda !
\VS{13}Boaz prit Ruth, qui devint sa femme, et il alla vers elle. Yahweh lui fit la grâce de concevoir, et elle enfanta un fils.
\VS{14}Les femmes dirent à Naomi : Béni soit Yahweh qui ne t'a pas laissé manquer aujourd'hui d'un homme, ayant droit de rachat, et dont le nom sera proclamé en Israël !
\VS{15}Cet enfant restaurera ton âme, et sera le soutien de ta vieillesse ; car ta belle-fille, qui t'aime, l'a enfanté, et elle vaut mieux que sept fils.
\VS{16}Naomi prit l'enfant et le posa sur son sein, et elle fut sa nourrice.
\TextTitle{Obed, le grand-père de David}
\VS{17}Les voisines lui donnèrent un nom, en disant : Un fils est né à Naomi ! Et elles l'appelèrent du nom de Obed. Ce fut le père d'Isaï, père de David.
\VS{18}Voici la généalogie de Pérets. Pérets engendra Hetsron ;
\VS{19}Hetsron engendra Ram ; Ram engendra Amminadab ;
\VS{20}Amminadab engendra Nachschon ; Nachschon engendra Salmon ;
\VS{21}Salmon engendra Boaz ; Boaz engendra Obed ;
\VS{22}Obed engendra Isaï, et Isaï engendra David.
\PPE{}
\end{multicols}
