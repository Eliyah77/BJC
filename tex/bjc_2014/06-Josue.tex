\ShortTitle{Josué}\BookTitle{Josué}\BFont
\noindent\hrulefill
{\footnotesize
\textit{
\bigskip
{\centering{}
\\(Yehoshoua)
\\Signifie : Yahweh est salut
\\Thème : La conquête de Canaan
\\Auteur : Probablement Josué
\\Date de rédaction : 14ème siècle av. J.-C.\\}
}
%\bigskip
\textit{
\\Josué, fils de Nun, originaire de la tribu d’Ephraïm et né en Égypte, servit Moïse de la sortie d’Egypte jusqu’à sa mort. Il fut le seul de l’ancienne génération, avec Caleb, à avoir survécu à la longue épreuve du désert et fut choisi par Dieu pour succéder à Moïse. Accompagné de la puissante main de Dieu qui lui accorda de grands succès sur les champs de bataille, il conduisit Israël à entrer en possession de la terre promise. Ainsi, ce livre relate l’histoire de la conquête de Canaan et son partage aux douze tribus d’Israël.\bigskip
}
}
\par\nobreak\noindent\hrulefill
\begin{multicols}{2}
\Chap{1}
\VerseOne{}Or, il arriva après la mort de Moïse, serviteur de Yahweh, que Yahweh parla à Josué, fils de Nun, qui avait servi Moïse, en disant :
\VS{2}Moïse, mon serviteur est mort ; maintenant, lève-toi, passe ce Jourdain, toi et tout ce peuple, pour entrer dans le pays que je donne aux enfants d’Israël.
\VS{3}Tout lieu que foulera la plante de votre pied, je vous le donne, comme je l’ai déclaré à Moïse.
\VS{4}Votre frontière sera depuis le désert et le Liban, jusqu’au grand fleuve, le fleuve de l’Euphrate, tout le pays des Héthiens jusqu’à la grande mer, vers le soleil couchant.
\VS{5}Nul ne tiendra devant toi, tous les jours de ta vie. Je serai avec toi comme j’ai été avec Moïse ; je ne te délaisserai point, et je ne t’abandonnerai point.
\VS{6}Fortifie-toi et prends courage, car c’est toi qui mettras ce peuple en possession du pays dont j’ai juré à leurs pères de leur donner.
\VS{7}Fortifie-toi seulement et prends courage de plus en plus, en agissant fidèlement selon toute la loi que Moïse, mon serviteur t’a prescrite ; ne t’en détourne point ni à droite ni à gauche, afin que tu réussisses partout où tu iras.
\VS{8}Que ce livre de la loi ne s’éloigne point de ta bouche, mais médite-le jour et nuit, pour agir fidèlement selon tout ce qui y est écrit (1) ; car c’est alors que tu auras du succès dans tes entreprises, c’est alors que tu réussiras.
\VS{9}Ne t’ai-je pas donné cet ordre, fortifie-toi et prends courage ? Ne t’épouvante point et ne t’effraie point ; car Yahweh ton Dieu est avec toi partout où tu iras.
\VS{10}Après cela, Josué donna cet ordre aux officiers du peuple, en disant :
\VS{11}Traversez le camp, ordonnez au peuple et dites-lui : Préparez-vous des provisions, car dans trois jours vous passerez ce Jourdain pour vous emparer du pays que Yahweh, votre Dieu, vous donne afin que vous le possédiez.
\VS{12}Josué parla aussi aux Rubénites, aux Gadites et à la demi-tribu de Manassé, en disant :
\VS{13}Souvenez-vous de la parole que Moïse, serviteur de Yahweh, vous a prescrite, en disant : Yahweh votre Dieu vous a accordé du repos, et vous a donné ce pays.
\VS{14}Vos femmes, vos petits-enfants, et vos bêtes resteront dans le pays que Moïse vous a donné de l’autre côté du Jourdain ; mais vous tous, hommes vaillants, vous passerez en armes devant vos frères, et vous les aiderez ;
\VS{15}jusqu’à ce que Yahweh ait accordé du repos à vos frères comme à vous, et qu’ils soient aussi en possession du pays que Yahweh, votre Dieu, leur donne. Puis vous reviendrez prendre possession du pays qui est votre propriété, et que vous a donné Moïse, serviteur de Yahweh, de l’autre côté du Jourdain, vers l’orient.
\VS{16}Ils répondirent à Josué, en disant : Nous ferons tout ce que tu nous as ordonné, et nous irons partout où tu nous enverras.
\VS{17}Nous t’obéirons comme nous avons obéi à Moïse ; seulement que Yahweh ton Dieu soit avec toi, comme il a été avec Moïse.
\VS{18}Tout homme qui sera rebelle à ton ordre, et qui n’obéira point à tes paroles, à tout ce que tu lui commanderas, sera mis à mort ; seulement, fortifie-toi, et prends courage !
\Chap{2}
\VerseOne{}Or, Josué fils de Nun, envoya secrètement de Sittim deux hommes, pour espionner le pays, et il leur dit : Allez, examinez le pays, et Jéricho. Ils partirent donc, et arrivèrent dans la maison d’une femme prostituée, nommée Rahab (1), et ils y couchèrent.
\VS{2}Alors on dit au roi de Jéricho : Voici, des hommes sont venus ici cette nuit de la part des enfants d’Israël pour explorer le pays.
\VS{3}Le roi de Jéricho envoya dire à Rahab : Fais sortir les hommes qui sont venus chez toi et qui sont entrés dans ta maison ; car ils sont venus pour explorer tout le pays.
\VS{4}La femme prit les deux hommes et les cacha ; et elle dit : Il est vrai que des hommes sont venus chez moi, mais je ne savais pas d’où ils étaient ;
\VS{5}et comme la porte a dû se fermer de nuit, ces hommes sont sortis ; j’ignore où ils sont allés ; hâtez-vous de les poursuivre et vous les atteindrez.
\VS{6}Elle les avait fait monter sur le toit et les avait cachés sous des tiges de lin qu’elle avait arrangées sur le toit.
\VS{7}Ces gens les poursuivirent par le chemin du Jourdain jusqu’aux passages ; on ferma la porte après que ceux qui les poursuivaient furent sortis.
\VS{8}Avant qu’ils se couchent, elle monta vers eux sur le toit ;
\VS{9}et leur dit : Je sais que Yahweh vous a donné ce pays, la terreur que votre nom inspire nous a saisis, et tous les habitants du pays tremblent devant vous.
\VS{10}Car nous avons appris que Yahweh a mis à sec devant vous les eaux de la mer Rouge à votre sortie du pays d’Egypte ; et ce que vous avez fait aux deux rois des Amoréens qui étaient de l’autre côté du Jourdain, à Sihon et à Og, que vous avez détruits complètement en les dévouant par le moyen de l'interdit.
\VS{11}Nous l’avons appris, et notre cœur est saisi d’épouvante, tous les hommes ont l’esprit abattu à votre aspect. Car Yahweh, votre Dieu, est le Dieu en haut dans les cieux et en bas sur la terre.
\VS{12}Maintenant donc, je vous prie, jurez-moi par Yahweh, que vous aurez pour la maison de mon père la même bonté que j’ai eue pour vous.
\VS{13}Donnez-moi un signe (2) d’une ferme assurance que vous laisserez vivre mon père, ma mère, mes frères, mes sœurs, et tous ceux qui leur appartiennent, et que vous sauverez nos âmes de la mort.
\VS{14}Ces hommes lui répondirent : Nous sommes prêts à mourir pour vos âmes, si vous ne divulguez pas cette affaire ; et quand Yahweh nous aura donné le pays, nous agirons envers toi avec bonté et fidélité.
\VS{15}Elle les fit donc descendre avec une corde par la fenêtre ; car la maison qu’elle habitait était sur la muraille de la ville.
\VS{16}Elle leur dit : Allez à la montagne, de peur que ceux qui vous poursuivent ne vous rencontrent, et cachez-vous là pendant trois jours jusqu’à ce qu’ils soient de retour. Après cela vous suivrez votre chemin.
\VS{17}Ces hommes lui dirent : Voici comment nous serons quittes de ce serment que tu nous as fait faire.
\VS{18}Quand nous entrerons dans le pays, tu lieras ce (2) cordon de fil cramoisi à la fenêtre par laquelle tu nous auras fait descendre, et tu recueilleras chez toi, dans cette maison, ton père et ta mère, tes frères, et toute la famille de ton père.
\VS{19}Quiconque sortira hors de la porte de ta maison, son sang sera sur sa tête, et nous en serons quittes ; mais quiconque sera avec toi, dans la maison, son sang sera sur notre tête si quelqu’un met la main sur lui.
\VS{20}Et si tu divulgues cette affaire, nous serons quittes du serment que tu nous as fait faire.
\VS{21}Elle répondit : Que cela soit selon vos paroles. Alors elle les laissa aller. Ils s’en allèrent et elle lia le cordon de fil cramoisi à la fenêtre.
\VS{22}Ils partirent, et arrivèrent à la montagne, où ils restèrent trois jours, jusqu’à ce que ceux qui les poursuivaient soient de retour. Ceux qui les poursuivaient les cherchèrent par tout le chemin, mais ils ne les trouvèrent pas.
\VS{23}Ainsi ces deux hommes s’en retournèrent, descendirent de la montagne, passèrent le Jourdain. Ils vinrent auprès de Josué, fils de Nun. Ils lui racontèrent toutes les choses qui leur étaient arrivées.
\VS{24}Ils dirent à Josué : Certainement, Yahweh a livré tout le pays entre nos mains, et même tous les habitants tremblent devant nous.
\Chap{3}
\VerseOne{}Josué se leva de bon matin, lui et tous les enfants d’Israël partirent de Sittim, ils vinrent jusqu’au Jourdain, et ils logèrent là cette nuit, avant de le traverser.
\VS{2}Au bout de trois jours les officiers traversèrent le milieu du camp,
\VS{3}et donnèrent cet ordre au peuple en disant : Dès que vous verrez l’arche de l’alliance de Yahweh, votre Dieu, portée par les sacrificateurs, les Lévites, vous partirez du lieu où vous êtes, et vous marcherez après elle.
\VS{4}Mais il y aura entre vous et elle une distance d’environ deux mille coudées : N’en approchez pas. Elle vous fera connaître le chemin par lequel vous devez marcher ; car vous n’avez pas encore passé par ce chemin.
\VS{5}Josué dit au peuple : Sanctifiez-vous, car Yahweh fera demain des prodiges au milieu de vous.
\VS{6}Josué parla aussi aux sacrificateurs, en disant : Portez l’arche de l’alliance, et passez devant le peuple. Ainsi ils portèrent l’arche de l’alliance, et marchèrent devant le peuple.
\VS{7}Yahweh dit à Josué : Aujourd’hui je commencerai à t’élever aux yeux de tout Israël, afin qu’ils sachent que je serai aussi avec toi, comme j’ai été avec Moïse.
\VS{8}Tu donneras cet ordre aux sacrificateurs qui portent l’arche de l’alliance, en leur disant : Dès que vous arriverez au bord des eaux du Jourdain, vous vous arrêterez dans le Jourdain.
\VS{9}Josué dit aux enfants d’Israël : Approchez-vous d’ici, et écoutez les paroles de Yahweh, votre Dieu.
\VS{10}Josué dit : Vous reconnaîtrez à ceci que le Dieu vivant est au milieu de vous et qu’il chassera et déshéritera devant vous les Cananéens, les Héthiens, les Héviens, les Phéréziens, les Guirgasiens, les Amoréens et les Jébusiens.
\VS{11}Voici, l’arche de l’alliance du Seigneur de toute la terre va passer devant vous dans le Jourdain.
\VS{12}Maintenant, prenez douze hommes des tribus d’Israël, un homme de chaque tribu.
\VS{13}Et aussitôt que les sacrificateurs qui portent l’arche de Yahweh, le Seigneur de toute la terre, poseront la plante des pieds dans les eaux du Jourdain, les eaux du Jourdain seront coupées, les eaux qui descendent d’en haut, et elles s’arrêteront en un monceau.
\VS{14}Le peuple sortit de ses tentes pour passer le Jourdain, et les sacrificateurs, qui portaient l’arche de l’alliance, étaient devant le peuple.
\VS{15}Aussitôt que ceux qui portaient l’arche furent arrivés au Jourdain, et que les pieds des sacrificateurs qui portaient l’arche furent mouillés au bord de l’eau, - le Jourdain regorge par-dessus toutes ses rives durant tout le temps de la moisson, -
\VS{16}les eaux qui descendent d’en haut, s’arrêtèrent, et s’élevèrent en un monceau, à une très grande distance, depuis la ville d’Adam, qui est à côté de Tsarthan ; et celles d’en bas, qui descendaient vers la mer de la plaine, qui est la mer salée, furent totalement coupées. Le peuple passa vis-à-vis de Jéricho.
\VS{17}Les sacrificateurs qui portaient l’arche de l’alliance de Yahweh, s’arrêtèrent de pied ferme sur le sec, au milieu du Jourdain, pendant que tout Israël passait à sec, jusqu’à ce que tout le peuple ait achevé de passer le Jourdain.
\Chap{4}
\VerseOne{}Lorsque tout le peuple eut achevé de passer le Jourdain, Yahweh parla à Josué et dit :
\VS{2}Prenez douze hommes parmi le peuple, un homme de chaque tribu.
\VS{3}Donnez-leur cet ordre, en disant : Enlevez d’ici, du milieu du Jourdain, de la place où les sacrificateurs se sont arrêtés de pied ferme, douze pierres, que vous emporterez avec vous, et vous les poserez au lieu où vous passerez cette nuit.
\VS{4}Josué appela les douze hommes qu’il choisit parmi les enfants d’Israël, un homme de chaque tribu.
\VS{5}Il leur dit : Passez devant l’arche de Yahweh, votre Dieu, au milieu du Jourdain, et que chacun de vous charge une pierre sur son épaule, selon le nombre des tribus des enfants d’Israël ;
\VS{6}afin que cela soit un signe au milieu de vous. Et quand vos fils interrogeront à l’avenir leurs pères, en disant : Qu’est-ce que ces pierres ?
\VS{7}Alors vous leur répondrez : Les eaux du Jourdain ont été coupées devant l’arche de l’alliance de Yahweh ; lorsqu’elle passa le Jourdain, les eaux du Jourdain ont été arrêtées ; c’est pourquoi ces pierres-là seront à jamais un souvenir pour les enfants d’Israël.
\VS{8}Les enfants d’Israël firent ce que Josué leur avait ordonné. Ils prirent douze pierres du milieu du Jourdain, comme Yahweh l’avait ordonné à Josué, selon le nombre des tribus des enfants d’Israël. Ils les emportèrent avec eux et les posèrent au lieu où ils devaient passer la nuit.
\VS{9}Josué dressa aussi douze pierres au milieu du Jourdain, à l’endroit où les pieds des sacrificateurs qui portaient l’arche de l’alliance s’étaient arrêtés ; et elles y sont restées jusqu’à ce jour.
\VS{10}Les sacrificateurs qui portaient l’arche se tinrent debout au milieu du Jourdain, jusqu’à ce que tout ce que Yahweh avait ordonné à Josué de dire au peuple soit accompli, selon tout ce que Moïse avait prescrit à Josué. Et le peuple se hâta de passer.
\VS{11}Lorsque tout le peuple eut achevé de passer, alors l’arche de Yahweh et les sacrificateurs passèrent devant le peuple.
\VS{12}Les fils de Ruben, les fils de Gad, et la demi-tribu de Manassé passèrent en armes devant les enfants d’Israël, comme Moïse le leur avait dit.
\VS{13}Environ quarante mille hommes armés et prêts à combattre passèrent devant Yahweh, vers les plaines de Jéricho.
\VS{14}Ce jour-là, Yahweh éleva Josué à la vue de tout Israël, et ils le craignirent, comme ils avaient craint Moïse, tous les jours de sa vie.
\VS{15}Yahweh parla à Josué, et dit :
\VS{16}Ordonne aux sacrificateurs qui portent l’arche du témoignage de sortir du Jourdain.
\VS{17}Et Josué donna cet ordre aux sacrificateurs, en disant : Sortez du Jourdain.
\VS{18}Aussitôt que les sacrificateurs, qui portaient l’arche de l’alliance de Yahweh, furent sortis du milieu du Jourdain, et qu’ils eurent mis la plante de leurs pieds sur le sec, les eaux du Jourdain retournèrent à leur place, et coulèrent comme auparavant sur tous les rivages.
\VS{19}Le peuple sortit du Jourdain le dixième jour du premier mois, et il campa à Guilgal, à l’orient de Jéricho.
\VS{20}Josué dressa à Guilgal les douze pierres qu’ils avaient prises du Jourdain.
\VS{21}Il parla aux enfants d’Israël et leur dit : Quand vos enfants interrogeront à l’avenir leurs pères, et leur diront : Qu’est-ce que ces pierres-ci ?
\VS{22}Vous en instruirez vos enfants, en leur disant : Israël a passé ce Jourdain à sec.
\VS{23}Car Yahweh, votre Dieu, a mis à sec les eaux du Jourdain devant vous jusqu’à ce que vous eussiez passé, comme Yahweh, votre Dieu, l’avait fait à la mer Rouge, qu’il mit à sec devant nous, jusqu’à ce que nous eussions passé,
\VS{24}afin que tous les peuples de la terre sachent que la main de Yahweh est puissante, et afin que vous ayez toujours la crainte de Yahweh, votre Dieu.
\Chap{5}
\VerseOne{}Lorsque tous les rois des Amoréens qui étaient au-delà du Jourdain, vers l’occident, et tous les rois des Cananéens qui étaient près de la mer, apprirent que Yahweh avait mis à sec les eaux du Jourdain devant les enfants d’Israël, jusqu’à ce que nous eussions passé, leur cœur fut saisi d’épouvante, et leur esprit fut troublé à l’aspect des enfants d’Israël.
\VS{2}En ce temps-là, Yahweh dit à Josué : Fais-toi des couteaux de pierre tranchants, et (1) circoncis de nouveau les enfants d’Israël, une seconde fois.
\VS{3}Josué se fit des couteaux de pierre tranchants, et circoncit les enfants d’Israël sur la colline d’Araloth.
\VS{4}Voici la raison pour laquelle Josué les circoncit : Tout le peuple sorti d’Egypte, tous les hommes, tous les hommes de guerre étaient morts en chemin dans le désert, après leur sortie d’Egypte.
\VS{5}Tout ce peuple sorti d’Egypte était circoncis, mais aucun du peuple né dans le désert en chemin n’avait été circoncis, après leur sortie d’Egypte.
\VS{6}Car les enfants d’Israël avaient marché quarante ans dans le désert jusqu’à la destruction de toute la nation des hommes de guerre qui étaient sortis d’Egypte, et qui n’avaient point écouté la voix de Yahweh ; Yahweh leur jura qu’il ne leur laisserait point voir le pays qu’il avait juré à leurs pères de nous donner, pays où coulent le lait et le miel.
\VS{7}Ce sont leurs enfants qu’il établit à leur place ; et Josué les circoncit, parce qu’ils étaient incirconcis ; parce qu’on ne les avait pas circoncis pendant le voyage.
\VS{8}Lorsqu’on eut achevé de circoncire tout le peuple, ils restèrent dans leur camp, jusqu’à ce qu’ils soient guéris.
\VS{9}Yahweh dit à Josué : Aujourd’hui j’ai roulé de dessus vous l’opprobre de l’Egypte. Et ce lieu-là fut appelé Guilgal jusqu’à ce jour.
\VS{10}Ainsi les enfants d’Israël campèrent à Guilgal, et célébrèrent la Pâque le quatorzième jour du mois, sur le soir, dans les plaines de Jéricho.
\VS{11}Dès le lendemain de la Pâque, ils mangèrent du blé du pays, des pains sans levain et du grain rôti, en ce même jour.
\VS{12}La manne cessa dès le lendemain de la Pâque, après qu’ils mangèrent du blé du pays ; les enfants d’Israël n’eurent plus de manne, mais ils mangèrent les récoltes de la terre de Canaan cette année-là.
\VS{13}Comme Josué était près de Jéricho, il leva les yeux et regarda. Voici, un homme qui avait son épée nue à la main, se tenait debout devant lui. Josué alla vers lui et lui dit : Es-tu des nôtres ou de nos ennemis ?
\VS{14}Il répondit : Non, mais je suis le chef de l’armée de Yahweh, je viens maintenant. Josué tomba à terre sur son visage, l’adora, et lui dit : Qu’est-ce que mon Seigneur dit à son serviteur ?
\VS{15}Et le chef de l’armée de Yahweh dit à Josué : Délie tes souliers de tes pieds ; car le lieu sur lequel tu te tiens est saint. Et Josué fit ainsi.
\Chap{6}
\VerseOne{}Jéricho était barricadée et fermée soigneusement, à cause des enfants d’Israël. Personne ne sortait, et personne n’entrait.
\VS{2}Yahweh dit à Josué : Regarde, je livre entre tes mains Jéricho et son roi, ses hommes vaillants.
\VS{3}Faites le tour de la ville, vous tous les hommes de guerre, faites une fois le tour de la ville. Tu feras ainsi durant six jours.
\VS{4}Sept sacrificateurs porteront sept shofars retentissants devant l’arche. Mais au septième jour, vous ferez sept fois le tour de la ville et les sacrificateurs sonneront des shofars.
\VS{5}Quand ils sonneront avec la corne retentissante, aussitôt que vous entendrez le son du shofar retentissant, tout le peuple poussera un grand cri de guerre et la muraille de la ville tombera sur elle. Et le peuple montera, les hommes devant lui.
\VS{6}Josué, fils de Nun, appela les sacrificateurs et leur dit : Portez l’arche de l’alliance et que sept sacrificateurs portent sept shofars devant l’arche de Yahweh.
\VS{7}Il dit aussi au peuple : Passez et faites le tour de la ville, que tous ceux qui seront armés passent devant l’arche de Yahweh.
\VS{8}Lorsque Josué eut parlé au peuple, les sept sacrificateurs qui portaient les sept shofars retentissants devant Yahweh passèrent et sonnèrent des shofars. Et l’arche de l’alliance de Yahweh allait derrière eux.
\VS{9}Les hommes armés marchaient devant les sacrificateurs qui sonnaient des shofars ; mais l’arrière-garde suivait derrière l’arche ; on sonnait des shofars en marchant.
\VS{10}Josué avait donné cet ordre au peuple, en disant : Vous ne crierez point et vous ne ferez point entendre votre voix. Il ne sortira point un seul mot de votre bouche, jusqu’au jour où je vous dirai : Poussez des cris de guerre ! Alors vous pousserez des cris de guerre.
\VS{11}L’arche de Yahweh fit ainsi le tour de la ville, en tournant une fois autour, puis on revint au camp, et on y passa la nuit.
\VS{12}Ensuite Josué se leva de bon matin, et les sacrificateurs portèrent l’arche de Yahweh.
\VS{13}Les sept sacrificateurs qui portaient les sept shofars retentissants devant l’arche de Yahweh se mirent en marche et sonnèrent du shofar. Et les hommes armés allaient devant eux ; puis l’arrière-garde suivait l’arche de Yahweh ; on sonnait des shofars en marchant.
\VS{14}Ils firent une fois le tour de la ville le deuxième jour, et ils retournèrent au camp. Ils firent de même durant six jours.
\VS{15}Mais le septième jour, ils se levèrent dès le matin à l’aube du jour, et ils firent sept fois le tour de la ville de la même manière ; ce fut le seul jour où ils firent sept fois le tour de la ville.
\VS{16}A la septième fois, comme les sacrificateurs sonnaient des shofars, Josué dit au peuple : Poussez des cris de guerre, car Yahweh vous a donné la ville !
\VS{17}La ville sera dévouée par le moyen de l'interdit à Yahweh, elle et ce qui s’y trouve ; seule Rahab, la prostituée (1), vivra, elle et tous ceux qui seront avec elle dans la maison, parce qu’elle a caché soigneusement les messagers que nous avions envoyés.
\VS{18}Gardez-vous seulement de ce qui sera dévoué par le moyen de l'interdit ; car si vous preniez ce que vous aurez dévoué par le moyen de l'interdit, vous mettriez le camp d’Israël en interdit, et vous y jetteriez le trouble.
\VS{19}Mais tout l’argent et tout l’or, tous les objets d’airain et de fer seront consacrés à Yahweh, ils entreront dans le trésor de Yahweh.
\VS{20}Le peuple cria et on sonna des shofars. Lorsque le peuple entendit le son des shofars, il poussa de grands cris de guerre, et la muraille tomba sur elle-même. Alors le peuple monta dans la ville, les hommes devant le peuple. Ils s’emparèrent de la ville,
\VS{21}et ils la dévouèrent entièrement par le moyen de l'interdit, et passèrent au fil de l’épée tout ce qui était dans la ville, hommes et femmes, enfants et vieillards, même jusqu’aux bœufs, aux brebis et aux ânes.
\VS{22}Josué dit aux deux hommes qui avaient espionné le pays : Entrez dans la maison de la femme prostituée, et faites-la sortir de là, avec tous ceux qui lui appartiennent, comme vous le lui avez juré.
\VS{23}Les jeunes hommes qui avaient espionné le pays, entrèrent et firent sortir Rahab, son père, sa mère et ses frères, avec tous ceux qui lui appartenaient ; ils firent aussi sortir toutes les familles qui lui appartenaient, et les mirent hors du camp d’Israël.
\VS{24}Puis ils allumèrent le feu et brûlèrent la ville et tout ce qui s’y trouvait ; seulement ils mirent l’argent et l’or, les objets d’airain et de fer dans le trésor de la maison de Yahweh.
\VS{25}Ainsi Josué sauva la vie à Rahab la prostituée, la maison de son père, et tous ceux qui lui appartenaient ; et elle a habité au milieu d’Israël jusqu’à ce jour, parce qu’elle avait caché les messagers que Josué avait envoyés pour explorer Jéricho.
\VS{26}En ce temps-là Josué jura, en disant : Maudit soit devant Yahweh l’homme qui se lèvera pour rebâtir cette ville de Jéricho ! Il en posera les fondements au prix de son premier-né, et il y placera ses portes au prix de son plus jeune fils (2).
\VS{27}Yahweh fut avec Josué, et sa renommée se répandit dans tout le pays.
\Chap{7}
\VerseOne{}Mais les enfants d’Israël agirent avec infidélité au sujet des choses dévouées par le moyen de l'interdit. Acan, fils de Carmi, fils de Zabdi, fils de Zérach, de la tribu de Juda, prit des choses dévouées par le moyen de l'interdit, et la colère de Yahweh s’enflamma contre les enfants d’Israël.
\VS{2}Josué envoya de Jéricho des hommes vers Aï, qui est près de Beth-Aven, à l’orient de Béthel. Il leur parla, et dit : Montez, et explorez le pays. Ces hommes montèrent et explorèrent Aï.
\VS{3}Ils revinrent auprès de Josué, ils lui dirent : Que tout le peuple n’y monte point mais qu’environ deux mille ou trois mille hommes y montent, et ils battront Aï. Ne fatigue pas tout le peuple en l’envoyant là, car ils sont en petit nombre.
\VS{4}Environ trois mille hommes du peuple y montèrent, mais ils s’enfuirent devant les gens d’Aï.
\VS{5}Les gens d’Aï leur tuèrent environ trente-six hommes ; ils les poursuivirent depuis la porte jusqu’à Schebarim, et les battirent à la descente. Le cœur du peuple se fondit et devint comme de l’eau.
\VS{6}Alors Josué déchira ses vêtements, et se prosterna jusqu’au soir le visage contre terre devant l’arche de Yahweh, lui et les anciens d’Israël, et ils se couvrirent la tête de poussière.
\VS{7}Josué dit : Ha ! Seigneur Yahweh, pourquoi as-tu fait si magnifiquement passer le Jourdain à ce peuple, pour nous livrer entre les mains des Amoréens, et nous faire périr ? Oh ! Si nous avions su rester de l’autre côté du Jourdain !
\VS{8}De grâce, Seigneur, que dirai-je, puisqu’Israël a tourné le dos devant ses ennemis ?
\VS{9}Les Cananéens et tous les habitants du pays l’apprendront ; ils nous entoureront, et ils supprimeront notre nom de dessus la terre. Et que feras-tu pour ton grand Nom ?
\VS{10}Alors Yahweh dit à Josué : Lève-toi ! Pourquoi te jettes-tu ainsi le visage contre terre ?
\VS{11}Israël a péché ; ils ont transgressé mon alliance que je leur avais prescrite, ils ont pris des choses dévouées par le moyen de l'interdit, ils les ont dérobées, ils ont menti, et ils les ont cachées parmi leurs objets (1).
\VS{12}C’est pourquoi les enfants d’Israël ne pourront s’élever et faire face à leurs ennemis ; ils tourneront le dos devant leurs ennemis ; car ils sont dévoués par le moyen de l'interdit. Je ne serai plus avec vous si vous ne détruisez pas l’interdit du milieu de vous.
\VS{13}Lève-toi, sanctifie le peuple, et dis : Sanctifiez-vous pour demain ; car ainsi parle Yahweh, le Dieu d’Israël : Il y a de l’interdit au milieu de toi, Israël ! Tu ne pourras t’élever et faire face à tes ennemis jusqu’à ce que vous ayez ôté l’interdit du milieu de vous.
\VS{14}Vous vous approcherez donc le matin selon vos tribus ; et la tribu que Yahweh aura désignée s’approchera selon les familles, et la famille que Yahweh aura désignée s’approchera selon les maisons, et la maison que Yahweh aura désignée s’approchera selon les hommes.
\VS{15}Alors celui qui aura été désigné avec la chose dévouée par le moyen de l'interdit sera brûlé au feu, lui et tout ce qui lui appartient parce qu’il a transgressé l’alliance de Yahweh, et qu’il a commis une infamie en Israël.
\VS{16}Josué se leva de bon matin, et fit approcher Israël selon ses tribus, et la tribu de Juda fut désignée.
\VS{17}Puis il fit approcher les familles de Juda, et la famille de Zérach fut désignée. Puis il fit approcher les hommes de la famille de ceux qui étaient descendants de Zérach, et Zabdi fut désigné.
\VS{18}Il fit approcher la maison de Zabdi par hommes, Acan fils de Carmi, fils de Zabdi, fils de Zérach, de la tribu de Juda, fut désigné.
\VS{19}Josué dit à Acan : Mon fils, donne gloire à Yahweh, le Dieu d’Israël, et fais-lui des actions de grâces. Déclare-moi ce que tu as fait, ne me le cache point.
\VS{20}Acan répondit à Josué, et dit : Il est vrai que j’ai péché contre Yahweh, le Dieu d’Israël, et voici ce que j’ai fait.
\VS{21}J’ai vu dans le butin un beau manteau de Schinear (2), deux cents sicles d’argent et un lingot d’or du poids de cinquante sicles ; je les ai convoités, je les ai pris ; ces choses sont cachées dans la terre au milieu de ma tente, et l’argent est sous le manteau.
\VS{22}Alors Josué envoya des messagers qui coururent à la tente ; et voici, le manteau était caché dans la tente d’Acan, et l’argent sous le manteau.
\VS{23}Ils les prirent donc du milieu de la tente, les apportèrent à Josué et à tous les enfants d’Israël, et ils les déposèrent devant Yahweh.
\VS{24}Alors Josué et tout Israël avec lui, prirent Acan, fils de Zérach, l’argent, le manteau, le lingot d’or, ses fils et ses filles, ses bœufs, ses ânes et ses brebis, sa tente et tout ce qui lui appartenait, et ils les firent monter dans la vallée d’Acor.
\VS{25}Josué dit : Pourquoi nous as-tu troublés ? Yahweh te troublera aujourd’hui. Et tout Israël le lapida avec des pierres, et les brûlèrent au feu, après les avoir lapidés avec des pierres.
\VS{26}Ils dressèrent sur lui un grand tas de pierres, qui dure jusqu’à ce jour. Et Yahweh apaisa l’ardeur de sa colère. C’est pourquoi ce lieu-là a été appelé jusqu’à aujourd’hui, la vallée d’Acor.
\Chap{8}
\VerseOne{}Yahweh dit à Josué : Ne crains point, et ne t’effraie point ! Prends avec toi tous les gens de guerre, lève-toi, et monte contre Aï. Regarde, je livre entre tes mains le roi d’Aï et son peuple, sa ville et son pays.
\VS{2}Tu traiteras Aï et son roi, comme tu as traité Jéricho et son roi : Seulement vous pillerez pour vous le butin et les bêtes. Place des gens en embuscade derrière la ville.
\VS{3}Josué se leva avec tous les gens de guerre, pour monter contre Aï. Josué choisit trente mille vaillants hommes armés, et les envoya de nuit.
\VS{4}Et il leur donna cet ordre en disant : Prêtez attention, vous vous mettrez en embuscade derrière la ville ; ne vous éloignez pas beaucoup de la ville, mais tenez-vous prêts.
\VS{5}Mais moi et tout le peuple qui est avec moi, nous nous approcherons de la ville. Et quand ils sortiront à notre rencontre, comme ils ont fait la première fois, nous nous enfuirons devant eux.
\VS{6}Ils sortiront après nous, jusqu’à ce que nous les ayons attirés loin de la ville ; car ils diront : Ils fuient devant nous comme la première fois ; parce que nous fuirons devant eux.
\VS{7}Alors vous vous lèverez de l’embuscade, et vous vous emparerez de la ville ; car Yahweh, votre Dieu, la livrera entre vos mains.
\VS{8}Et quand vous aurez pris la ville, vous y mettrez le feu ; vous agirez selon la parole de Yahweh. Prêtez attention à l’ordre que je vous donne.
\VS{9}Josué les envoya, et ils allèrent se mettre en embuscade, et restèrent entre Béthel et Aï, à l’occident d’Aï. Mais Josué resta cette nuit-là au milieu du peuple.
\VS{10}Puis Josué se leva de bon matin, et passa le peuple en revue ; et il monta lui et les anciens d’Israël, devant le peuple vers Aï.
\VS{11}Tous les gens de guerre qui étaient avec lui, montèrent et s’approchèrent ; lorsqu’ils furent en face de la ville, ils campèrent au nord d’Aï ; et la vallée était entre lui et Aï.
\VS{12}Il prit aussi environ cinq mille hommes, et les mit en embuscade entre Béthel et Aï, à l’occident de la ville.
\VS{13}Après que tout le camp eut pris position au nord de la ville, et l’embuscade à l’occident de la ville, cette nuit-là, Josué s’avança au milieu de la vallée.
\VS{14}Lorsque le roi d’Aï vit cela, les hommes de la ville se levèrent en hâte, et de bon matin le roi et tout son peuple sortirent devant la plaine contre Israël pour le combattre. Or il ne savait pas qu’il y avait des gens en embuscade contre lui derrière la ville.
\VS{15}Alors Josué et tout Israël feignirent d’être battus devant eux, et ils s’enfuirent par le chemin du désert.
\VS{16}Alors tout le peuple qui était dans la ville d’Aï, fut assemblé à grand cri pour les poursuivre. Ils poursuivirent Josué, et ils furent ainsi attirés loin de la ville.
\VS{17}Il ne resta pas un seul homme dans Aï ni dans Béthel qui ne sortit contre Israël. Ils laissèrent la ville ouverte, et ils poursuivirent Israël.
\VS{18}Yahweh dit à Josué : Etends vers Aï le javelot qui est dans ta main, car je la livrerai entre tes mains. Et Josué étendit vers la ville le javelot qui était dans sa main.
\VS{19}Aussitôt qu’il eut étendu sa main, les hommes qui étaient en embuscade se levèrent précipitamment du lieu où ils étaient ; ils pénétrèrent dans la ville, la prirent, et se hâtèrent de mettre le feu dans la ville.
\VS{20}Les gens d’Aï, ayant regardé derrière eux, virent la fumée de la ville monter vers le ciel, et ils ne purent se sauver d’aucun côté. Le peuple qui fuyait vers le désert se tourna contre ceux qui le poursuivaient ;
\VS{21}car Josué et tout Israël, voyant que ceux qui étaient en embuscade avaient pris la ville, et que la fumée de la ville montait, se retournèrent, et frappèrent les gens d’Aï.
\VS{22}Les autres aussi sortirent de la ville contre eux, et ils furent enveloppés par les Israélites de toutes parts. Ils furent tellement battus qu’il ne resta aucun survivant ni aucun fuyard ;
\VS{23}ils prirent aussi vivant le roi d’Aï, et le présentèrent à Josué.
\VS{24}Quand les Israélites eurent achevé de tuer tous les habitants d’Aï dans la campagne, dans le désert, où ils les avaient poursuivis, et que tous furent tombés sous le tranchant de l’épée, jusqu’à être entièrement défaits, tous les Israélites revinrent vers Aï, et la frappèrent au tranchant de l’épée.
\VS{25}Tous ceux qui tombèrent ce jour-là, tant des hommes que des femmes, furent au nombre de douze mille, tous gens d’Aï.
\VS{26}Josué ne retira point sa main qu’il tenait étendue avec le javelot, jusqu’à ce que tous les habitants d’Aï aient été entièrement dévoués par le moyen de l'interdit.
\VS{27}Seulement les Israélites pillèrent pour eux les bêtes et le butin de cette ville-là, suivant ce que Yahweh avait prescrit à Josué.
\VS{28}Josué brûla Aï, et en fit à jamais un monceau de ruines, jusqu’à aujourd’hui.
\VS{29}Puis il fit pendre le roi d’Aï à un arbre, et le laissa jusqu’au soir. Au coucher du soleil, Josué ordonna qu’on descende de l’arbre son cadavre ; on le jeta à l’entrée de la porte de la ville, puis on dressa sur lui un grand amas de pierres, qui subsiste encore aujourd’hui.
\VS{30}Alors Josué bâtit un autel à Yahweh, le Dieu d’Israël, sur la montagne d’Ebal,
\VS{31}comme Moïse, serviteur de Yahweh, l’avait ordonné aux enfants d’Israël, ainsi qu’il est écrit dans le livre de la loi de Moïse : Il fit cet autel de pierres brutes sur lesquelles personne ne porta le fer (1) ; et ils offrirent dessus des holocaustes à Yahweh, et sacrifièrent des sacrifices d’offrande de paix.
\VS{32}Et là Josué écrivit sur les pierres une copie de la loi que Moïse avait mise par écrit devant les enfants d’Israël.
\VS{33}Tout Israël, ses anciens, ses officiers, et ses juges étaient des deux côtés de l’arche, en face des sacrificateurs qui sont de la race de Lévi, qui portaient l’arche de l’alliance de Yahweh, les étrangers comme les Hébreux naturels, une moitié du côté du mont Garizim (2), et l’autre moitié du côté du mont Ebal, selon l’ordre qu’avait précédemment donné Moïse, serviteur de Yahweh, de bénir le peuple d’Israël.
\VS{34}Après cela, Josué lut tout haut toutes les paroles de la loi, tant les bénédictions que les malédictions, selon tout ce qui est écrit dans le livre de la loi.
\VS{35}Il n’y eut rien de tout ce que Moïse avait prescrit, que Josué ne lise tout haut devant toute l’assemblée d’Israël, des femmes et des petits-enfants, et des étrangers qui marchaient au milieu d’eux.
\Chap{9}
\VerseOne{}En entendant ces choses, tous les rois qui étaient au-delà du Jourdain, dans la montagne et dans la plaine, et sur toute la côte de la grande mer, jusque près du Liban, les Héthiens, les Amoréens, les Cananéens, les Phéréziens, les Héviens et les Jébusiens,
\VS{2}s’assemblèrent tous d’un commun accord pour faire la guerre à Josué et à Israël.
\VS{3}Mais les habitants de Gabaon (1), lorsqu’ils apprirent de quelle manière Josué avait traité Jéricho et Aï,
\VS{4}usèrent de ruse, et se mirent en chemin. Ils prirent de vieux sacs pour leurs ânes, et de vieilles outres de vin déchirées et recousues,
\VS{5}ils portaient à leurs pieds de vieux souliers raccommodés, et de vieux habits sur eux ; et tout le pain qu’ils avaient pour nourriture était sec et en miettes.
\VS{6}Ils arrivèrent auprès de Josué au camp de Guilgal, et lui dirent, ainsi qu’à tous ceux d’Israël : Nous venons d’un pays éloigné, et maintenant traitez alliance avec nous.
\VS{7}Les hommes d’Israël répondirent à ces Héviens : Peut-être habitez-vous au milieu de nous, et comment traiterions-nous alliance avec vous ?
\VS{8}Mais ils dirent à Josué : Nous sommes tes serviteurs. Alors Josué leur dit : Qui êtes-vous ? Et d’où venez-vous ?
\VS{9}Ils lui répondirent : Tes serviteurs sont venus d’un pays très éloigné, sur la renommée de Yahweh, ton Dieu ; car nous avons entendu sa renommée, et toutes les choses qu’il a faites en Egypte,
\VS{10}et de la manière dont il a traité les deux rois des Amoréens, qui étaient au-delà du Jourdain, Sihon, roi de Hesbon, et Og, roi de Basan, qui demeurait à Aschtaroth.
\VS{11}Et nos anciens et tous les habitants de notre pays nous ont dit : Prenez avec vous des provisions pour le chemin, et allez au-devant d’eux, et dites-leur : Nous sommes vos serviteurs, et maintenant traitez alliance avec nous.
\VS{12}Voici notre pain : Nous l’avons pris dans nos maisons tout chaud pour notre provision, le jour où nous sommes partis pour venir vers vous, mais maintenant voici, il est devenu sec et en miettes.
\VS{13}Et voici aussi les outres de vin neuves que nous avons remplies, elles se sont déchirées ; nos habits et nos souliers sont usés à cause de la longueur de la marche.
\VS{14}Les hommes d’Israël prirent de leur provision, et aucun d’eux ne consulta la bouche de Yahweh (2).
\VS{15}Alors Josué fit la paix avec eux, et traita avec eux une alliance par laquelle il devait leur laisser la vie, et les chefs de l’assemblée le leur jurèrent.
\VS{16}Mais trois jours après l’alliance traitée avec eux, ils apprirent que c’étaient leurs voisins et qu’ils habitaient au milieu d’eux.
\VS{17}Car les enfants d’Israël partirent, et arrivèrent à leurs villes le troisième jour. Leurs villes étaient Gabaon, Kephira, Beéroth, et Kirjath-Jearim.
\VS{18}Les enfants d’Israël ne les frappèrent point, parce que les chefs de l’assemblée leur avaient juré par Yahweh, le Dieu d’Israël. Mais toute l’assemblée murmura contre les chefs.
\VS{19}Alors tous les chefs dirent à toute l’assemblée : Nous leur avons juré par Yahweh, le Dieu d’Israël, c’est pourquoi maintenant nous ne pouvons pas les frapper.
\VS{20}Voici comment nous les traiterons : Nous leur laisserons la vie, afin de ne pas attirer sur nous la colère, à cause du serment que nous leur avons fait.
\VS{21}Ils vivront, leur dirent les chefs. Mais ils furent employés à couper le bois et à puiser l’eau pour toute l’assemblée, comme les chefs le leur avaient dit (3).
\VS{22}Josué les fit appeler, et leur parla, en disant : Pourquoi nous avez-vous trompés, en nous disant : Nous sommes très éloignés de vous, alors que vous habitez au milieu de nous ?
\VS{23}Maintenant vous êtes maudits il y aura toujours des esclaves parmi vous, des coupeurs de bois et des puiseurs d’eau pour la maison de mon Dieu.
\VS{24}Ils répondirent à Josué, et dirent : Après qu’il ait été exactement rapporté à tes serviteurs les ordres que Yahweh, ton Dieu, avait ordonnés à Moïse, son serviteur, pour vous livrer tout le pays et pour en exterminer tous les habitants, et votre présence nous a inspiré une grande crainte pour nos vies : voilà pourquoi nous avons agi de la sorte.
\VS{25}Maintenant nous voici entre tes mains ; traite-nous comme il te semblera bon et juste de nous traiter.
\VS{26}Josué les traita ainsi, il les délivra de la main des enfants d’Israël, et ils ne les tuèrent point.
\VS{27}En ce jour-là, Josué les établit coupeurs de bois et puiseurs d’eau pour l’assemblée, et pour l’autel de Yahweh, jusqu’à aujourd’hui, dans le lieu qu’il choisirait.
\Chap{10}
\VerseOne{}Quand Adoni-Tsédek, roi de Jérusalem, entendit que Josué s’était emparé d’Aï, et qu’il l’avait entièrement détruite en la dévouant par le moyen de l'interdit, qu’il avait traité Aï et son roi, comme il avait traité Jéricho et son roi, et que les habitants de Gabaon avaient fait la paix avec Israël, et étaient au milieu d’eux.
\VS{2}Il eut une grande frayeur, parce que Gabaon était une grande ville, comme une ville royale, et elle était plus grande qu’Aï, et parce que tous ses hommes étaient vaillants.
\VS{3}C’est pourquoi Adoni-Tsédek, roi de Jérusalem, envoya dire à Hoham, roi d’Hébron, et à Piream, roi de Jarmuth, et à Japhia, roi de Lakis, et à Debir, roi d’Eglon :
\VS{4}Montez vers moi, et aidez-moi afin que nous frappions Gabaon, car elle a fait la paix avec Josué et avec les enfants d’Israël.
\VS{5}Ainsi cinq rois des Amoréens, le roi de Jérusalem, le roi d’Hébron, le roi de Jarmuth, le roi de Lakis, et le roi d’Eglon, s’assemblèrent et montèrent avec toutes leurs armées ; et ils campèrent près de Gabaon, et lui firent la guerre.
\VS{6}Alors les gens de Gabaon dirent à Josué au camp de Guilgal : Ne retire point tes mains de tes serviteurs, monte rapidement vers nous, délivre-nous, et donne-nous du secours ; car tous les rois des Amoréens qui habitent aux montagnes se sont rassemblés contre nous.
\VS{7}Josué monta donc de Guilgal, et avec lui tous les gens de guerre, et tous les hommes forts et vaillants.
\VS{8}Yahweh dit à Josué : Ne les crains point, car je les livre entre tes mains, et aucun d’eux ne tiendra devant toi.
\VS{9}Josué arriva subitement sur eux, après avoir marché toute la nuit depuis Guilgal.
\VS{10}Yahweh les mit en déroute devant Israël, qui leur fit éprouver une grande défaite près de Gabaon, et les poursuivit par le chemin de la montagne de Beth-Horon, les battit jusqu’à Azéka, et jusqu’à Makkéda.
\VS{11}Comme ils s’enfuyaient devant Israël, et qu’ils étaient à la descente de Beth-Horon, Yahweh fit tomber du ciel sur eux de grosses pierres jusqu’à Azéka, et ils périrent ; ceux qui moururent des pierres de grêle furent plus nombreux que ceux qui furent tués avec l’épée par les enfants d’Israël.
\VS{12}Alors Josué parla à Yahweh, le jour où Yahweh livra les Amoréens aux enfants d’Israël, et dit en présence d’Israël : Soleil, arrête-toi sur Gabaon, et toi lune, sur la vallée d’Ajalon !
\VS{13}Et le soleil s’arrêta, et la lune aussi s’arrêta, jusqu’à ce que la nation ait tiré vengeance de ses ennemis. Cela n’est-il pas écrit dans le livre du Juste ? Le soleil s’arrêta au milieu du ciel et ne se hâta point de se coucher environ un jour entier.
\VS{14}Il n’y a point eu de jour semblable à celui-là, ni avant ni après, où Yahweh exauça la voix d’un homme ; car Yahweh combattait pour Israël.
\VS{15}Et Josué, et tout Israël avec lui, retourna au camp à Guilgal.
\VS{16}Les cinq rois restants s’enfuirent, et se cachèrent dans une caverne à Makkéda.
\VS{17}On le rapporta à Josué, en disant : On a trouvé les cinq rois cachés dans une caverne à Makkéda.
\VS{18}Et Josué dit : Roulez de grosses pierres à l’entrée de la caverne et mettez près d’elle quelques hommes pour les garder.
\VS{19}Mais vous, ne vous arrêtez pas, poursuivez vos ennemis, attaquez-les par-derrière jusqu’au dernier, ne les laissez pas entrer dans leurs villes, car Yahweh, votre Dieu, les a livrés entre vos mains.
\VS{20}Après que Josué et les enfants d’Israël eurent achevé de les frapper et de leur faire éprouver une très grande défaite, jusqu’à les détruire entièrement, ceux d’entre eux qui s’étaient échappés se retirèrent dans les villes fortifiées,
\VS{21}tout le peuple revint en paix au camp vers Josué à Makkéda, et personne ne remua sa langue contre les enfants d’Israël.
\VS{22}Josué dit alors : Ouvrez l’entrée de la caverne, et amenez-moi ces cinq rois hors de la caverne.
\VS{23}Ils firent ainsi, et ils lui amenèrent hors de la caverne ces cinq rois : Le roi de Jérusalem, le roi d’Hébron, le roi de Jarmuth, le roi de Lakis et le roi d’Eglon.
\VS{24}Lorsqu’ils eurent amené à Josué ces cinq rois, Josué appela tous les hommes d’Israël, et dit aux chefs des gens de guerre qui étaient allés avec lui : Approchez-vous, mettez vos pieds sur les cous de ces rois. Ils s’approchèrent, et mirent leurs pieds sur leurs cous.
\VS{25}Alors Josué leur dit : Ne craignez point, et ne soyez point effrayés, fortifiez-vous, et ayez du courage, car Yahweh traitera ainsi tous vos ennemis contre lesquels vous combattez.
\VS{26}Après cela Josué les frappa et les fit mourir, il les fit pendre à cinq arbres, et ils restèrent pendus à ces arbres jusqu’au soir.
\VS{27}Vers le coucher du soleil, Josué ordonna qu’on les descende de ces arbres, on les jeta dans la caverne où ils s’étaient cachés, et on mit à l’entrée de la caverne de grosses pierres qui y sont demeurées jusqu’à ce jour.
\VS{28}Josué prit aussi Makkéda le même jour, la frappa du tranchant de l’épée, et dévoua par le moyen de l'interdit le roi, la ville et ceux qui s’y trouvaient ; il n’en laissa échapper aucun, et il traita le roi de Makkéda comme il avait traité le roi de Jéricho.
\VS{29}Josué, et tout Israël avec lui, passa de Makkéda à Libna, et fit la guerre à Libna.
\VS{30}Yahweh la livra aussi entre les mains d’Israël, avec son roi, et il la frappa du tranchant de l’épée, elle et tous ceux qui s’y trouvaient ; il n’en laissa échapper aucun, et il traita son roi comme il avait traité le roi de Jéricho.
\VS{31}Ensuite Josué, et tout Israël avec lui, passa de Libna à Lakis, campa devant elle, et lui fit la guerre.
\VS{32}Yahweh livra Lakis entre les mains d’Israël, qui la prit le deuxième jour, et la frappa du tranchant de l’épée, et toutes les personnes qui s’y trouvaient, comme il avait traité Libna.
\VS{33}Alors Horam, roi de Guézer, monta pour secourir Lakis. Josué le frappa, lui et son peuple, de sorte qu’il n’en laissa pas échapper un seul homme.
\VS{34}Après cela Josué, et tout Israël avec lui, passa de Lakis à Eglon ; ils campèrent devant elle, et lui firent la guerre.
\VS{35}Ils la prirent le jour même, la frappèrent du tranchant de l’épée ; et Josué dévoua par le moyen de l'interdit ce jour-là toutes les personnes qui y étaient, comme il avait traité Lakis.
\VS{36}Josué, et tout Israël avec lui, monta d’Eglon à Hébron, et ils lui firent la guerre.
\VS{37}Ils la prirent, et la frappèrent du tranchant de l’épée, avec son roi, toutes ses villes, et toutes les personnes qui y étaient ; il n’en laissa échapper aucune, comme il avait traité Eglon ; et il dévoua par le moyen de l'interdit, toutes les personnes qui y étaient.
\VS{38}Josué, et tout Israël avec lui, retourna vers Debir, et ils lui firent la guerre.
\VS{39}Il la prit, avec son roi et toutes ses villes ; et ils les frappèrent du tranchant de l’épée, et dévouèrent par le moyen de l'interdit toutes les personnes qui y étaient ; il n’en laissa échapper aucune. Il traita Debir et son roi comme il avait traité Hébron, et comme il avait traité Libna et son roi.
\VS{40}Josué frappa tout ce pays, la montagne et le midi, la plaine et les coteaux, et tous leurs rois ; il n’en laissa échapper aucun, et il dévoua par le moyen de l'interdit toutes les personnes qui y respiraient, comme Yahweh, le Dieu d’Israël, l’avait ordonné.
\VS{41}Josué les battit depuis Kadès-Barnéa jusqu’à Gaza, et tout le pays de Gosen jusqu’à Gabaon.
\VS{42}Josué prit tous ces rois en même temps et leur pays, parce que Yahweh, le Dieu d’Israël, combattait pour Israël.
\VS{43}Après quoi Josué, et tout Israël avec lui, retourna au camp à Guilgal.
\Chap{11}
\VerseOne{}Jabin, roi de Hatsor, ayant appris ces choses, envoya des messagers à Jobab, roi de Madon, au roi de Schimron, et au roi d’Acschaph,
\VS{2}aux rois qui habitaient vers le nord, aux montagnes et dans la plaine, vers le midi de Kinnéreth, dans la vallée, et sur les hauteurs de Dor vers l’occident,
\VS{3}aux Cananéens qui étaient à l’orient et à l’occident, aux Amoréens, aux Héthiens, aux Phéréziens, aux Jébusiens dans les montagnes, et aux Héviens au pied de la montagne de l’Hermon, dans le pays de Mitspa.
\VS{4}Ils sortirent avec toutes leurs armées, un grand peuple par leur grand nombre, comme le sable qui est sur le bord de la mer, il y avait aussi des chevaux et des chars en très grand nombre.
\VS{5}Tous ces rois se réunirent, et campèrent ensemble près des eaux de Mérom, pour combattre contre Israël.
\VS{6}Yahweh dit à Josué : Ne les crains point, car demain, à cette même heure, je les livrerai tous, blessés à mort, devant Israël. Tu couperas les jarrets à leurs chevaux, et brûleras au feu leurs chars.
\VS{7}Josué, et tous les gens de guerre avec lui vinrent subitement sur eux près des eaux de Mérom, et ils se précipitèrent au milieu d’eux.
\VS{8}Yahweh les livra entre les mains d’Israël ; ils les battirent, et les poursuivirent jusqu’à Sidon la grande, jusqu’aux eaux de Misrephoth-Maïm, et jusqu’à la vallée de Mitspa vers l’orient, et ils les battirent tellement qu’ils ne laissèrent aucun survivant.
\VS{9}Josué les traita comme Yahweh lui avait dit ; il coupa les jarrets de leurs chevaux, et brûla au feu leurs chars.
\VS{10}A son retour, et dans le même temps, Josué prit Hatsor, et frappa son roi avec l’épée ; car Hatsor avait été auparavant la capitale de tous ces royaumes.
\VS{11}On frappa aussi du tranchant de l’épée et l’on dévoua par le moyen de l'interdit tous ceux qui s’y trouvaient, il ne resta rien de ce qui respirait, et l’on brûla au feu Hatsor.
\VS{12}Josué prit aussi toutes les villes de ces rois, et tous leurs rois, et les frappa du tranchant de l’épée, et il les dévoua par le moyen de l'interdit, comme Moïse, serviteur de Yahweh, l’avait ordonné.
\VS{13}Mais Israël ne brûla aucune des villes situées sur des collines, à l’exception de Hatsor, que Josué brûla.
\VS{14}Les enfants d’Israël pillèrent pour eux tout le butin de ces villes et le bétail ; mais ils frappèrent du tranchant de l’épée tous les hommes, jusqu’à ce qu’ils les aient exterminés, ils n’y laissèrent aucun qui respirait.
\VS{15}Josué exécuta les ordres de Yahweh à Moïse, son serviteur, et de Moïse à Josué ; il ne s’éloigna pas de toutes les paroles que Yahweh avait prescrites à Moïse.
\VS{16}Josué s’empara de tout ce pays, de la montagne, de tout le pays du midi, de tout le pays de Gosen, de la vallée, et de la plaine, la montagne d’Israël et de ses vallées.
\VS{17}Depuis la montagne de Halak, qui s’élève vers Séir, jusqu’à Baal-Gad dans la vallée du Liban, au pied de la montagne d’Hermon. Il prit aussi tous leurs rois, les battit et les fit mourir.
\VS{18}La guerre que soutint Josué contre tous ces rois fut de plusieurs jours.
\VS{19}Il n’y eut aucune ville qui fit la paix avec les enfants d’Israël, excepté les Héviens qui habitaient à Gabaon ; ils les prirent toutes par la guerre.
\VS{20}Car Yahweh endurcissait leur cœur pour qu’ils sortent en bataille contre Israël, afin qu’il les dévoue par le moyen de l'interdit, sans qu’il y ait pour eux de miséricorde, et qu’il les extermine, comme Yahweh l’avait ordonné à Moïse.
\VS{21}Dans le même temps, Josué se mit en marche, et il extermina les Anakim de la montagne d’Hébron, de Debir, d’Anab, de toute la montagne de Juda, et de toute la montagne d’Israël ; Josué, les dévoua par le moyen de l'interdit avec leurs villes.
\VS{22}Il ne resta point d’Anakim dans le pays des enfants d’Israël ; il n’en resta seulement qu’à Gaza, à Gath et à Asdod.
\VS{23}Josué prit donc tout le pays, suivant tout ce que Yahweh avait dit à Moïse. Et Josué le donna en héritage à Israël, selon leurs portions, et leurs tribus. Puis le pays fut en repos et sans guerre.
\Chap{12}
\VerseOne{}Voici les rois du pays que les enfants d’Israël frappèrent, et dont ils possédèrent le pays de l’autre côté du Jourdain, vers l’orient, depuis le torrent de l’Arnon jusqu’à la montagne de l’Hermon, et toute la plaine vers l’orient.
\VS{2}Sihon, roi des Amoréens, qui habitait à Hesbon, et qui dominait depuis Aroër, qui est sur le bord du torrent de l’Arnon, et depuis le milieu du torrent, sur la moitié de Galaad, jusqu’au torrent de Jabbok, qui est la frontière des enfants d’Ammon ;
\VS{3}et depuis la plaine jusqu’à la mer de Kinnéreth vers l’orient, et jusqu’à la mer de la plaine, qui est la mer salée, vers l’orient, au chemin de Beth-Jeschimoth ; et depuis le midi sur le pied du Pisga.
\VS{4}Les contrées d’Og, roi de Basan, qui était seul reste des Rephaïm, et qui habitait à Aschtaroth et à Edréï.
\VS{5}Sa domination s’étendait sur la montagne de l’Hermon, sur Salca, et sur tout Basan, jusqu’à la frontière des Gueschuriens et des Maacathiens, et sur la moitié de Galaad, frontière de Sihon, roi de Hesbon.
\VS{6}Moïse, serviteur de Yahweh, et les enfants d’Israël, les battirent ; et Moïse, serviteur de Yahweh, en donna la possession aux Rubénites, et aux Gadites, et à la demi-tribu de Manassé.
\VS{7}Voici les rois du pays que Josué et les enfants d’Israël frappèrent de ce côté-ci du Jourdain vers l’occident, depuis Baal-Gad, dans la vallée du Liban, jusqu’à la montagne de Halak qui monte vers Séir, et que Josué donna aux tribus d’Israël en possession, selon leurs portions,
\VS{8}dans la montagne, dans les vallées, dans les plaines, sur les collines, dans le désert et dans le midi ; pays des Héthiens, des Amoréens, des Cananéens, des Phéréziens, des Héviens et des Jébusiens.
\VS{9}Le roi de Jéricho, un ; le roi d’Aï, près de Béthel, un ;
\VS{10}le roi de Jérusalem, un ; le roi d’Hébron, un ;
\VS{11}le roi de Jarmuth, un ; le roi de Lakis, un ;
\VS{12}le roi d’Eglon, un ; le roi de Guézer, un ;
\VS{13}le roi de Debir, un ; le roi de Guéder, un ;
\VS{14}le roi de Horma, un ; le roi d’Arad, un ;
\VS{15}le roi de Libna, un ; le roi d’Adullam, un ;
\VS{16}le roi de Makkéda, un ; le roi de Béthel, un ;
\VS{17}le roi de Tappuach, un ; le roi de Hépher, un ;
\VS{18}le roi d’Aphek, un ; le roi de Lascharon, un ;
\VS{19}le roi de Madon, un ; le roi de Hatsor, un ;
\VS{20}le roi de Schimron-Meron, un ; le roi d’Acschaph, un ;
\VS{21}le roi de Taanac, un ; le roi de Meguiddo, un ;
\VS{22}le roi de Kédesch, un ; le roi de Jokneam, au Carmel, un ;
\VS{23}le roi de Dor, sur les hauteurs de Dor, un ; le roi de Gojim, près de Guilgal, un ;
\VS{24}le roi de Thirtsa, un ; en tout trente et un rois.
\Chap{13}
\VerseOne{}Josué était vieux, fort avancé en âge, Yahweh lui dit : Tu es devenu vieux, fort avancé en âge, et le pays qui te reste à soumettre est très grand.
\VS{2}Voici le pays qui reste, toutes les contrées des Philistins, et des Gueschuriens,
\VS{3}depuis le Schichor, qui coule devant l’Egypte, jusqu’à la frontière d’Ekron au nord, contrée qui doit être tenue pour Cananéenne, et qui est occupée par les cinq princes des Philistins, celui de Gaza, celui d’Asdod, celui d’Askalon, celui de Gath, celui d’Ekron, et par les Avviens ;
\VS{4}du côté du midi, tout le pays des Cananéens, et Meara qui est aux Sidoniens, jusqu’à Aphek, jusqu’à la frontière des Amoréens ;
\VS{5}le pays qui appartient aux Guibliens, et tout le Liban, vers l’orient, depuis Baal-Gad, au pied de la montagne d’Hermon, jusqu’à l’entrée de Hamath ;
\VS{6}tous les habitants de la montagne, depuis le Liban jusqu’aux eaux de Misrephoth-Maïm, tous les Sidoniens. Je les chasserai moi-même devant les enfants d’Israël. Donne seulement ce pays en héritage par le sort à Israël, comme je te l’ai prescrit.
\VS{7}Maintenant divise ce pays en héritage aux neuf tribus, et à la demi-tribu de Manassé.
\VS{8}Les Rubénites et les Gadites, avec l’autre moitié de la tribu de Manassé, ont reçu leur héritage, que Moïse leur a donné de l’autre côté du Jourdain, à l’orient, comme le leur a donné Moïse, serviteur de Yahweh ;
\VS{9}depuis Aroër, qui est sur le bord du torrent de l’Arnon, et la ville qui est au milieu de la vallée, et toute la plaine de Médeba, jusqu’à Dibon ;
\VS{10}toutes les villes de Sihon, roi des Amoréens, qui régnait à Hesbon, jusqu’à la frontière des enfants d’Ammon ;
\VS{11}et Galaad, et les territoires des Gueschuriens et des Maacathiens, toute la montagne de l’Hermon, et tout Basan jusqu’à Salca ;
\VS{12}tout le royaume d’Og en Basan, qui régnait à Aschtaroth, et à Edréï, et qui était resté le seul reste des Rephaïm, Moïse battit ces rois, et les chassa.
\VS{13}Mais les enfants d’Israël ne chassèrent point les Gueschuriens et les Maacathiens, mais les Gueschuriens et les Maacathiens ont habité au milieu d’Israël jusqu’à ce jour.
\VS{14}La tribu de Lévi fut la seule à qui Moïse ne donna point d’héritage ; les sacrifices consumés par le feu devant Yahweh, le Dieu d’Israël, tel fut son héritage, comme il le lui avait dit.
\VS{15}Moïse donna un héritage à la tribu des fils de Ruben selon leurs familles.
\VS{16}Et leurs frontières furent depuis Aroër qui est sur le bord du torrent d’Arnon, et de la ville qui est au milieu du torrent, et toute la plaine qui est près de Médeba.
\VS{17}Hesbon et toutes ses villes, qui étaient dans la plaine, Dibon, Bamoth-Baal, Beth-Baal-Meon,
\VS{18}Jahats, Kedémoth et Méphaath,
\VS{19}Kirjathaïm, Sibma, Tséreth-Haschachar sur la montagne de la vallée,
\VS{20}Beth-Peor, les coteaux du Pisga et Beth-Jeschimoth,
\VS{21}toutes les villes de la plaine, et tout le royaume de Sihon, roi des Amoréens qui régnait à Hesbon ; Moïse l’avait battu, lui et les princes de Madian, Evi, Rékem, Tsur, Hur, et Réba, princes qui relevaient de Sihon, et qui habitaient dans le pays.
\VS{22}Parmi ceux que tuèrent les enfants d’Israël, ils avaient aussi fait périr par l’épée Balaam (1), fils de Beor, le devin.
\VS{23}Le Jourdain servait de frontière au territoire des fils de Ruben. Tel fut l’héritage des fils de Ruben selon leurs familles ; les villes et leurs villages.
\VS{24}Moïse donna aussi un héritage à la tribu de Gad, pour les fils de Gad, selon leurs familles.
\VS{25}Et leur territoire fut Jaezer, toutes les villes de Galaad et la moitié du pays des enfants d’Ammon, jusqu’à Aroër, qui est vis-à-vis de Rabba,
\VS{26}depuis Hesbon jusqu’à Ramath-Mitspé, et Bethonim, et depuis Mahanaïm jusqu’à la frontière de Debir,
\VS{27}et, dans la vallée, Beth-Haram, Beth-Nimra, Succoth et Tsaphon, reste du royaume de Sihon, roi de Hesbon, ayant le Jourdain pour frontière jusqu’à l’extrémité de la mer de Kinnéreth, de l’autre côté du Jourdain, vers l’orient.
\VS{28}Tel fut l’héritage des fils de Gad, selon leurs familles ; les villes et leurs villages.
\VS{29}Moïse donna aussi à la demi-tribu de Manassé un héritage, qui est resté à la demi-tribu des fils de Manassé, selon leurs familles.
\VS{30}Leur pays fut depuis Mahanaïm, tout Basan, et tout le royaume d’Og, roi de Basan, et tous les bourgs de Jaïr qui sont en Basan, soixante villes.
\VS{31}La moitié de Galaad, Aschtaroth et Edréï, villes du royaume d’Og en Basan, furent aux fils de Makir, fils de Manassé, à la moitié des enfants de Makir, selon leurs familles.
\VS{32}Ce sont là les pays que Moïse avait donnés en héritage, lorsqu’il était dans les plaines de Moab, de l’autre côté du Jourdain, vis-à-vis de Jéricho, à l’orient.
\VS{33}Mais Moïse ne donna point d’héritage à la tribu de Lévi ; car Yahweh, le Dieu d’Israël, fut leur héritage, comme il le lui avait dit.
\Chap{14}
\VerseOne{}Voici les terres que les enfants d’Israël reçurent en héritage dans le pays de Canaan, ce que partagèrent entre eux le sacrificateur Eléazar, Josué, fils de Nun, et les chefs des familles des tribus des enfants d’Israël.
\VS{2}Le partage eut lieu d’après le sort, comme Yahweh l’avait ordonné par Moïse, pour les neuf tribus, et pour la demi-tribu.
\VS{3}Car Moïse avait donné un héritage aux deux tribus et à la demi-tribu de l’autre côté du Jourdain, mais il n’avait point donné de part aux Lévites parmi eux.
\VS{4}Les fils de Joseph, Manassé et Ephraïm, formaient deux tribus ; et l’on ne donna point de part aux Lévites dans le pays, excepté des villes pour habitation, et les banlieues pour leurs troupeaux, et pour le reste de leurs biens.
\VS{5}Les enfants d’Israël firent comme Yahweh l’avait ordonné à Moïse, et ils partagèrent le pays.
\VS{6}Les fils de Juda s’approchèrent de Josué à Guilgal ; et Caleb, fils de Jephunné, le Kenizien, lui dit : Tu sais la parole que Yahweh a déclarée à Moïse, homme de Dieu, à mon sujet et au tien à Kadès-Barnéa.
\VS{7}J’étais âgé de quarante ans lorsque Moïse, serviteur de Yahweh, m’envoya à Kadès-Barnéa pour espionner le pays, et je lui fis un rapport avec droiture de cœur.
\VS{8}Mes frères qui étaient montés avec moi découragèrent le cœur du peuple, mais moi je persévérai à suivre Yahweh, mon Dieu.
\VS{9}Et ce jour-là Moïse jura, en disant : La terre que ton pied a foulée sera ton héritage à perpétuité, pour toi et pour tes fils, parce que tu as persévéré à suivre Yahweh, mon Dieu.
\VS{10}Maintenant voici, Yahweh m’a fait vivre comme il l’a dit. Il y a déjà quarante-cinq ans que Yahweh déclarait cette parole à Moïse, lorsqu’Israël marchait dans le désert. Et maintenant voici, je suis aujourd’hui âgé de quatre-vingt-cinq ans.
\VS{11}Aujourd’hui, je suis encore vigoureux comme au jour où Moïse m’envoya ; et j’ai toujours la même force que j’avais alors pour le combat, soit pour sortir et pour entrer.
\VS{12}Maintenant, donne-moi donc cette montagne, dont Yahweh a parlé ce jour-là ; car tu as appris en ce jour qu’il s’y trouve des Anakim, et qu’il y a de grandes villes fortifiées. Yahweh sera peut-être avec moi, et je les chasserai, comme Yahweh a dit.
\VS{13}Josué bénit Caleb, fils de Jephunné, et lui donna Hébron pour héritage.
\VS{14}C’est ainsi que Caleb, fils de Jephunné, le Kenizien, a eu jusqu’à ce jour Hébron pour héritage, parce qu’il avait persévéré à suivre Yahweh, le Dieu d’Israël.
\VS{15}Or Hébron s’appelait autrefois Kirjath-Arba ; et Arba avait été le plus grand homme parmi les Anakim. Le pays fut en repos et sans guerre.
\Chap{15}
\VerseOne{}La part échue par le sort à la tribu des fils de Juda selon leurs familles s’étendait vers la frontière d’Edom, jusqu’au désert de Tsin, vers le midi, à l’extrémité méridionale.
\VS{2}Ainsi leur frontière méridionale partait de l’extrémité de la mer salée, de la langue de mer qui fait face au sud.
\VS{3}Elle se prolongeait au midi de la montée d’Akrabbim, passait par Tsin, et montait au midi de Kadès-Barnéa ; elle passait de là par Hetsron, puis montait vers Addar, et tournait à Karkaa ;
\VS{4}puis elle passait par Atsmon, et continuait jusqu’au torrent d’Egypte, et les extrémités de cette frontière aboutissent à la mer. Ce sera là votre frontière du côté du midi.
\VS{5}La frontière orientale était la mer salée jusqu’à l’embouchure du Jourdain. La frontière septentrionale partait de la langue de mer, qui est à l’embouchure du Jourdain.
\VS{6}Cette frontière montait jusqu’à Beth-Hogla, et passait du côté du nord de Beth-Araba, et s’élevait jusqu’à la pierre de Bohan, fils de Ruben.
\VS{7}Elle montait à Debir, depuis la vallée d’Acor, et se dirigeait vers le nord, du côté de Guilgal, qui est vis-à-vis de la montée d’Adummim, au sud du torrent. Elle passait près des eaux d’En-Schémesch, et ses extrémités se prolongeaient à En-Roguel.
\VS{8}Elle montait de là par la vallée de Ben-Hinnom, au côté méridional de Jebus, qui est Jérusalem, puis s’élevait jusqu’au sommet de la montagne, qui est vis-à-vis de la vallée de Hinnom, à l’occident, et à l’extrémité de la vallée des Rephaïm, au nord.
\VS{9}Elle s’étendait du sommet de la montagne jusqu’à la source des eaux de Nephthoach, et continuait vers les villes de la montagne d’Ephron, puis se prolongeait par Baala, qui est Kirjath-Jearim.
\VS{10}De Baala elle tournait à l’occident, jusqu’à la montagne de Séir, puis elle traversait le côté septentrional de la montagne de Jearim, à Kesalon, puis descendait à Beth-Schémesch, et passait par Thimna.
\VS{11}Cette frontière continuait sur le côté septentrional d’Ekron, s’étendait vers Schicron, puis passait par la montagne de Baala, et se prolongeait jusqu’à Jabneel, pour aboutir à la mer.
\VS{12}La frontière occidentale était la grande mer. Telles furent de tous les côtés les frontières des fils de Juda, selon leurs familles.
\VS{13}On donna à Caleb, fils de Jephunné, une part au milieu des fils de Juda, comme Yahweh l’avait ordonné à Josué ; on lui donna, Kirjath-Arba, or Arba était père d’Anak ; et Kirjath-Arba c’est Hébron.
\VS{14}Caleb chassa de là les trois fils d’Anak : Schéschaï, Ahiman, et Talmaï, fils d’Anak.
\VS{15}De là il monta contre les habitants de Debir ; Debir s’appelait autrefois Kirjath-Sépher.
\VS{16}Caleb dit : Je donnerai ma fille Acsa pour femme à celui qui battra Kirjath-Sépher, et la prendra.
\VS{17}Et Othniel, fils de Kenaz, frère de Caleb, s’en empara ; et Caleb lui donna sa fille Acsa pour femme.
\VS{18}Lorsqu’elle fut entrée chez Othniel, elle l’incita à demander à son père un champ. Puis elle descendit de son âne, et Caleb lui dit : Qu’as-tu ?
\VS{19}Elle répondit : Donne-moi un présent, puisque tu m’as donné une terre du sud, donne-moi aussi des sources d’eau. Et il lui donna les sources supérieures et les sources inférieures.
\VS{20}Tel fut l’héritage de la tribu des fils de Juda, selon leurs familles.
\VS{21}Les villes situées dans la contrée du midi, à l’extrémité de la tribu des fils de Juda, près la frontière d’Edom, étaient : Kabtseel, Eder, Jagur,
\VS{22}Kina, Dimona, Adada,
\VS{23}Kédesch, Hatsor, Ithnan,
\VS{24}Ziph, Thélem, Bealoth,
\VS{25}Hatsor-Hadattha, Kerijoth-Hetsron qui est Hatsor,
\VS{26}Amam, Schema, Molada,
\VS{27}Hatsar-Gadda, Heschmon, Beth-Paleth,
\VS{28}Hatsar-Schual, Beer-Schéba, Bizjothja,
\VS{29}Baala, Ijjim, Atsem,
\VS{30}Eltholad, Kesil, Horma,
\VS{31}Tsiklag, Madmanna, Sansanna,
\VS{32}Lebaoth, Schilhim, Aïn et Rimmon. Total des villes : Vingt-neuf villes, et leurs villages.
\VS{33}Dans la plaine : Eschthaol, Tsorea, Aschna,
\VS{34}Zanoach, En-Gannim, Tappuach, Enam,
\VS{35}Jarmuth, Adullam, Soco, Azéka,
\VS{36}Schaaraïm, Adithaïm, Guedéra et Guedérothaïm ; quatorze villes, et leurs villages.
\VS{37}Tsenan, Hadascha, Migdal-Gad,
\VS{38}Dilean, Mitspé, Joktheel,
\VS{39}Lakis, Botskath, Eglon,
\VS{40}Cabbon, Lachmas, Kithlisch,
\VS{41}Guedéroth, Beth-Dagon, Naama, et Makkéda ; seize villes, et leurs villages.
\VS{42}Libna, Ether, Aschan,
\VS{43}Jiphtach, Aschna, Netsib,
\VS{44}Keïla, Aczib et Maréscha ; neuf villes, et leurs villages.
\VS{45}Ekron, et les villes de son ressort, et ses villages.
\VS{46}Depuis Ekron et à l’occident, toutes les villes près d’Asdod, et leurs villages.
\VS{47}Asdod, les villes de son ressort, et ses villages, Gaza, les villes de son ressort, et ses villages, jusqu’au torrent d’Egypte, et à la grande mer, qui sert de limite.
\VS{48}Dans la montagne : Schamir, Jatthir, Soco,
\VS{49}Danna, Kirjath-Sanna, qui est Debir,
\VS{50}Anab, Eschthemo, Anim,
\VS{51}Gosen, Holon, et Guilo ; onze villes et leurs villages.
\VS{52}Arab, Duma, Eschean,
\VS{53}Janum, Beth-Tappuach, Aphéka,
\VS{54}Humta, Kirjath-Arba, qui est Hébron, et Tsior ; neuf villes, et leurs villages.
\VS{55}Maon, Carmel, Ziph, Juta,
\VS{56}Jizreel, Jokdeam, Zanoach,
\VS{57}Kaïn, Guibea, et Thimna ; dix villes, et leurs villages.
\VS{58}Halhul, Beth-Tsur, Guedor,
\VS{59}Maarath, Beth-Anoth, et Elthekon ; six villes, et leurs villages.
\VS{60}Kirjath-Baal, qui est Kirjath-Jearim, et Rabba ; deux villes, et leurs villages.
\VS{61}Au désert : Beth-Araba, Middin, Secaca,
\VS{62}Nibschan, Ir-Hammélach, et En-Guédi : Six villes et leurs villages.
\VS{63}Au reste, les fils de Juda ne purent pas chasser les Jébusiens qui habitaient à Jérusalem, c’est pourquoi les Jébusiens ont habité avec les fils de Juda à Jérusalem jusqu’à ce jour.
\Chap{16}
\VerseOne{}La part échue par le sort aux fils de Joseph s’étendait depuis le Jourdain près de Jéricho, vers les eaux de Jéricho, à l’orient. La frontière suivait le désert qui s’élève de Jéricho à la montagne jusqu’à Béthel.
\VS{2}Cette frontière continuait de Béthel à Luz, puis passait vers la frontière des Arkiens jusqu’à Atharoth.
\VS{3}Elle descendait à l’occident, vers la frontière des Japhléthiens, jusqu’à celle de Beth-Horon la basse et jusqu’à Guézer, de sorte que ses extrémités aboutissaient à la mer.
\VS{4}Ainsi les fils de Joseph, Manassé et Ephraïm, reçurent leur héritage.
\VS{5}La frontière des fils d’Ephraïm, selon leurs familles, la frontière de leur héritage était à l’orient, Atharoth-Addar, jusqu’à Beth-Horon la haute.
\VS{6}Cette frontière continuait du côté de l’occident vers Micmethath au nord, tournait à l’orient, jusqu’à Thaanath-Silo, et passait dans la direction de l’orient, par Janoach.
\VS{7}Elle descendait de Janoach à Atharoth et à Naaratha, touchait à Jéricho, et se prolongeait jusqu’au Jourdain.
\VS{8}Elle allait de Tappuach, vers l’occident, jusqu’au torrent de Kana, tellement que ses extrémités aboutissaient à la mer. Ce fut là l’héritage de la tribu des fils d’Ephraïm, selon leurs familles.
\VS{9}Les fils d’Ephraïm avaient aussi des villes séparées au milieu de l’héritage des fils de Manassé, toutes ces villes, avec leurs villages.
\VS{10}Ils ne chassèrent point les Cananéens qui habitaient à Guézer, c’est pourquoi les Cananéens ont habité parmi Ephraïm jusqu’à ce jour, mais ils furent réduits à la servitude et assujettis à un tribut.
\Chap{17}
\VerseOne{}Une part échut aussi par le sort à la tribu de Manassé qui était le premier-né de Joseph. Quant à Makir, premier-né de Manassé, et père de Galaad, il avait eu Galaad et Basan parce qu’il était un homme de guerre.
\VS{2}Puis on jeta donc le sort pour les autres enfants de Manassé, selon ses familles ; aux fils d’Abiézer, aux fils de Hélek, aux fils d’Asriel, aux fils de Sichem, aux fils de Hépher, et aux fils de Schemida. Ce sont là les enfants mâles de Manassé fils de Joseph, selon leurs familles.
\VS{3}Tselophchad, fils de Hépher, fils de Galaad, fils de Makir, fils de Manassé, n’eut point de fils, mais il eut des filles dont voici les noms : Machla, Noa, Hogla, Milca et Thirtsa.
\VS{4}Elles se présentèrent devant le sacrificateur Eléazar, devant Josué, fils de Nun, et devant les princes, en disant : Yahweh a ordonné à Moïse de nous donner un héritage parmi nos frères. C’est pourquoi on leur donna un héritage parmi les frères de leur père, selon l’ordre de Yahweh.
\VS{5}Il échut dix portions à Manassé, outre le pays de Galaad et de Basan, qui est de l’autre côté du Jourdain.
\VS{6}Car les filles de Manassé eurent un héritage parmi ses fils, et le pays de Galaad fut pour les autres des fils de Manassé.
\VS{7}La frontière de Manassé s’étendait d’Aser à Micmethath, qui est près de Sichem, puis allait à Jamin vers les habitants d’En-Tappuach.
\VS{8}Le pays de Tappuach appartenait à Manassé, mais Tappuach qui était près de la frontière de Manassé, appartenait aux fils d’Ephraïm.
\VS{9}De là cette frontière descendait au torrent de Kana, au midi du torrent. Ces villes étaient à Ephraïm parmi les villes de Manassé. La frontière de Manassé était au côté du nord du torrent, et ses extrémités aboutissaient à la mer.
\VS{10}Le territoire du midi était à Ephraïm, et celui qui était vers le nord était à Manassé, et la mer leur servait de frontière ; et du côté du nord, les frontières se rencontraient à Aser, à Issacar, vers l’orient.
\VS{11}Manassé possédait dans Issacar et dans Aser : Beth-Schean et les villes de son ressort, Jibleam et les villes de son ressort, les habitants de Dor et les villes de son ressort, les habitants d’En-Dor, et les villes de son ressort, les habitants de Thaanac et les villes de son ressort, les habitants de Meguiddo et les villes de son ressort, qui sont trois contrées.
\VS{12}Les fils de Manassé ne purent pas chasser les habitants de ces villes, et les Cananéens voulurent rester dans le même pays.
\VS{13}Mais lorsque les fils d’Israël furent assez forts, ils assujettirent les Cananéens à un tribut, mais ils ne les chassèrent pas entièrement.
\VS{14}Les fils de Joseph parlèrent à Josué, et dirent : Pourquoi nous as-tu donné en héritage un seul lot, et une seule part, vu que nous sommes un peuple nombreux, et que Yahweh nous a bénis jusqu’à présent ?
\VS{15}Josué leur dit : Si vous êtes un peuple nombreux, montez à la forêt, et vous l’abattrez, pour vous y faire de la place dans le pays des Phéréziens et des Rephaïm, si la montagne d’Ephraïm est trop étroite pour vous.
\VS{16}Les fils de Joseph répondirent : Cette montagne ne sera pas suffisante pour nous, et tous les Cananéens qui habitent la vallée ont des chars de fer, et ceux qui sont à Beth-Schean, et dans les villes de son ressort, et ceux qui habitent dans la vallée de Jizreel.
\VS{17}Josué parla à la maison de Joseph, à Ephraïm et à Manassé, et dit : Vous êtes un peuple nombreux, et vous avez de grandes forces, vous n’aurez pas qu’une seule part.
\VS{18}Mais vous aurez la montagne, car c’est une forêt que vous abattrez et dont les extrémités vous appartiendront, et vous chasserez les Cananéens, quoiqu’ils aient des chars de fer, et qu’ils soient puissants.
\Chap{18}
\VerseOne{}Toute l’assemblée des enfants d’Israël s’assembla à (1) Silo, et ils y posèrent la tente d’assignation. Le pays était soumis devant eux.
\VS{2}Mais il restait sept tribus des enfants d’Israël qui n’avaient pas encore reçu leur héritage.
\VS{3}Josué dit aux enfants d’Israël : Jusqu’à quand négligerez-vous de prendre possession du pays que Yahweh, le Dieu de vos pères, vous a donné ?
\VS{4}Prenez trois hommes de chaque tribu, que j’enverrai. Ils se lèveront, traverseront le pays, traceront un plan en vue de l’héritage, puis ils reviendront auprès de moi.
\VS{5}Ils le diviseront en sept parts ; Juda restera dans ses limites au midi, et la maison de Joseph restera dans ses limites au nord.
\VS{6}Faites-vous un plan du pays en sept parts, et apportez-le-moi ici. Puis je jetterai pour vous le sort devant Yahweh, notre Dieu.
\VS{7}Mais il n’y aura point de part pour les Lévites au milieu de vous, parce que le sacerdoce de Yahweh est leur héritage. Quant à Gad et à Ruben, et à la demi-tribu de Manassé, ils ont reçu leur héritage de l’autre côté du Jourdain, vers l’orient, que Moïse, serviteur de Yahweh, leur a donné.
\VS{8}Ces hommes-là se levèrent, et s’en allèrent pour tracer un plan du pays, Josué leur donna cet ordre en disant : Allez et traversez le pays, et tracez-en un plan, puis revenez auprès de moi, et je jetterai ici le sort pour vous devant Yahweh, à Silo.
\VS{9}Ces hommes s’en allèrent, parcoururent le pays, et en tracèrent un plan dans un livre en sept parts selon les villes ; puis ils revinrent auprès de Josué dans le camp à Silo.
\VS{10}Josué jeta le sort pour eux à Silo devant Yahweh, et Josué fit le partage du pays entre les enfants d’Israël, selon leurs parts.
\VS{11}Le sort tomba sur la tribu des fils de Benjamin selon leurs familles, et la part qui leur échut par le sort avait ses frontières entre les fils de Juda et les fils de Joseph.
\VS{12}Du côté septentrional, leur frontière partait du Jourdain. Elle montait au nord de Jéricho, puis s’élevait dans la montagne vers l’occident, et ses extrémités aboutissaient au désert de Beth-Aven.
\VS{13}Puis elle passait de là par Luz, au midi de Luz, qui est Béthel, et elle descendait à Atharoth-Addar par-dessus la montagne qui est au midi de Beth-Horon la basse.
\VS{14}Du côté occidental, cette frontière se prolongeait et tournait au midi depuis la montagne qui est vis-à-vis de Beth-Horon, vers le midi, et ses extrémités aboutissaient à Kirjath-Baal, qui est Kirjath-Jearim, ville des fils de Juda. C’était le côté occidental.
\VS{15}Le côté méridional commençait à l’extrémité de Kirjath-Jearim. Et la frontière se prolongeait vers l’occident, jusqu’à la source des eaux de Nephthoach.
\VS{16}Elle descendait à l’extrémité de la montagne qui est vis-à-vis de la vallée de Ben-Hinnom, dans la vallée des Rephaïm, vers le nord. Elle descendait par la vallée de Hinnom, sur le côté méridional des Jébusiens, puis descendait jusqu’à En-Roguel.
\VS{17}Elle se dirigeait vers le nord, et sortait à En-Schémesch, de là à Gueliloth, qui est vis-à-vis de la montée d’Adummim, et elle descendait à la pierre de Bohan, fils de Ruben.
\VS{18}Elle passait sur le côté septentrional en face d’Araba, et descendait à Araba,
\VS{19}puis elle continuait sur le côté septentrional de Beth-Hogla, de sorte que ses extrémités aboutissaient à la langue septentrionale de la mer salée, vers l’embouchure du Jourdain au midi. C’était la frontière méridionale.
\VS{20}Et du côté oriental, le Jourdain en formait la frontière. Ce fut là l’héritage des fils de Benjamin avec ses frontières tout autour, selon leurs familles.
\VS{21}Les villes de la tribu des fils de Benjamin, selon leurs familles, étaient : Jéricho, Beth-Hogla, Emek-Ketsits,
\VS{22}Beth-Araba, Tsemaraïm, Béthel,
\VS{23}Avvim, Para, Ophra,
\VS{24}Kephar-Ammonaï, Ophni et Guéba ; douze villes, et leurs villages.
\VS{25}Gabaon, Rama, Beéroth,
\VS{26}Mitspé, Kephira, Motsa,
\VS{27}Rékem, Jirpeel, Thareala,
\VS{28}Tséla, Eleph, Jebus, qui est Jérusalem, Guibeath et Kirjath ; quatorze villes, et leurs villages. Tel fut l’héritage des fils de Benjamin selon leurs familles.
\Chap{19}
\VerseOne{}La deuxième part échut par le sort à Siméon, pour la tribu des fils de Siméon, selon leurs familles. Leur héritage était parmi l’héritage des fils de Juda.
\VS{2}Ils eurent dans leur héritage Beer-Schéba, Schéba, Molada,
\VS{3}Hatsar-Schual, Bala, Atsem,
\VS{4}Eltholad, Bethul, Horma,
\VS{5}Tsiklag, Beth-Marcaboth, Hatsar-Susa,
\VS{6}Beth-Lebaoth et Scharuchen ; treize villes et leurs villages.
\VS{7}Aïn, Rimmon, Ether, et Aschan ; quatre villes et leurs villages ;
\VS{8}et tous les villages qui étaient autour de ces villes-là jusqu’à Baalath-Beer, qui est Ramath du midi. Tel fut l’héritage de la tribu des fils de Siméon, selon leurs familles.
\VS{9}L’héritage des fils de Siméon fut pris sur la portion des fils de Juda ; car la portion des fils de Juda était trop grande pour eux ; c’est pourquoi les fils de Siméon reçurent leur héritage parmi le leur.
\VS{10}La troisième part échut par le sort aux fils de Zabulon, selon leurs familles.
\VS{11}La frontière de leur héritage s’étendait jusqu’à Sarid. Elle montait à l’occident vers Mareala, puis touchait à Dabbéscheth, et de là au torrent qui est vis-à-vis de Jokneam.
\VS{12}Cette frontière tournait de Sarid à l’orient, vers le soleil levant, jusqu’à la frontière de Kisloth-Thabor, puis continuait à Dabrath, et montait à Japhia.
\VS{13}De là elle passait à l’orient, par Guittha-Hépher, par Ittha-Katsin, puis continuait à Rimmon, jusqu’à Néa.
\VS{14}Elle tournait du côté du nord vers Hannathon, et ses extrémités aboutissaient à la vallée de Jiphthach-El.
\VS{15}Avec Katthath, Nahalal, Schimron, Jideala, et Bethléhem. Il y avait douze villes, et leurs villages.
\VS{16}Tel fut l’héritage des fils de Zabulon selon leurs familles, ces villes-là, et leurs villages.
\VS{17}La quatrième part échut par le sort à Issacar, aux fils d’Issacar, selon leurs familles.
\VS{18}Leur frontière passait par Jizreel, Kesulloth, Sunem,
\VS{19}Hapharaïm, Schion, Anacharath,
\VS{20}Rabbith, Kischjon, Abets,
\VS{21}Rémeth, En-Gannim, En-Hadda et Beth-Patsets ;
\VS{22}elle atteignait Thabor, Schachatsima et Beth-Schémesch, et les extrémités de leur frontière aboutissaient au Jourdain. Seize villes, et leurs villages.
\VS{23}Tel fut l’héritage de la tribu des fils d’Issacar, selon leurs familles, ces villes-là, et leurs villages.
\VS{24}La cinquième part échut par le sort à la tribu des fils d’Aser, selon leurs familles.
\VS{25}Leur frontière passait par Helkath, Hali, Béthen, Acschaph,
\VS{26}Allammélec, Amead et Mischeal ; elle aboutissait à Carmel, au quartier vers la mer, et à Schichor-Libnath.
\VS{27}Puis elle tournait vers l’orient, à Beth-Dagon, et atteignait Zabulon, et à la vallée de Jiphthach-El, vers le nord de Beth-Emek et de Neïel, puis se prolongeait vers Cabul, à gauche,
\VS{28}et vers Ebron, Rehob, Hammon et Kana, jusqu’à Sidon la grande.
\VS{29}Elle tournait ensuite vers Rama, jusqu’à la ville forte de Tyr, et aboutissait à Hosa, et ses extrémités aboutissaient au quartier qui est vers la mer, par la contrée d’Aczib.
\VS{30}Avec Umma, Aphek et Rehob. Vingt-deux villes, et leurs villages.
\VS{31}Tel fut l’héritage de la tribu des fils d’Aser, selon leurs familles ; ces villes-là et leurs villages.
\VS{32}La sixième part échut par le sort aux fils de Nephthali, selon leurs familles.
\VS{33}Leur frontière s’étendait depuis Héleph, depuis Allon par Tsaanannim, Adami-Nékeb et Jabneel, jusqu’à Lakkum, et ses extrémités aboutissaient au Jourdain.
\VS{34}Puis cette frontière tournait vers l’occident, à Aznoth-Thabor, et de là continuait à Hukkok ; elle touchait à Zabulon du côté du midi, du côté de l’occident elle touchait Aser et à Juda ; le Jourdain était du côté de l’orient.
\VS{35}Les villes fortifiées étaient : Tsiddim, Tser, Hammath, Rakkath, Kinnéreth,
\VS{36}Adama, Rama, Hatsor,
\VS{37}Kédesch, Edréï, En-Hatsor,
\VS{38}Jireon, Migdal-El, Horem, Beth-Anath et Beth-Schémesch ; dix-neuf villes et leurs villages.
\VS{39}Tel fut l’héritage de la tribu des fils de Nephthali, selon leurs familles ; ces villes-là, et leurs villages.
\VS{40}La septième part échut par le sort à la tribu des fils de Dan selon leurs familles.
\VS{41}La limite de leur héritage fut, Tsorea, Eschthaol, Ir-Schémesch,
\VS{42}Schaalabbin, Ajalon, Jithla,
\VS{43}Elon, Thimnatha, Ekron,
\VS{44}Eltheké, Guibbethon, Baalath,
\VS{45}Jehud, Bené-Berak, Gath-Rimmon,
\VS{46}Mé-Jarkon et Rakkon, avec le territoire qui est vis-à-vis de Japho.
\VS{47}Le territoire échu aux fils de Dan était trop petit pour eux. C’est pourquoi les fils de Dan montèrent, et combattirent contre Léschem ; ils s’en emparèrent et la frappèrent du tranchant de l’épée ; ils en prirent possession, s’y établirent, et l’appelèrent Léschem, Dan, du nom de Dan leur père.
\VS{48}Tel fut l’héritage de la tribu des fils de Dan selon leurs familles ; ces villes-là, et leurs villages.
\VS{49}Après qu’on eut achevé de partager le pays selon ses frontières, les enfants d’Israël donnèrent à Josué, fils de Nun, une possession au milieu d’eux.
\VS{50}Selon l’ordre de Yahweh, ils lui donnèrent la ville qu’il demanda, Thimnath-Sérach, dans la montagne d’Ephraïm. Il rebâtit la ville, et y habita.
\VS{51}Ce sont là les héritages que le sacrificateur Eléazar, Josué, fils de Nun, et les chefs de pères des tribus des enfants d’Israël partagèrent par le sort à Silo, devant Yahweh, à l’entrée de la tente d’assignation, et ils achevèrent ainsi le partage du pays.
\Chap{20}
\VerseOne{}Puis Yahweh parla à Josué et dit :
\VS{2}Parle aux enfants d’Israël et dis : Etablissez-vous des villes de refuge comme je vous l’ai ordonné par Moïse,
\VS{3}où pourra s’enfuir le meurtrier qui aura tué quelqu’un involontairement, sans intention, et elles vous serviront de refuge devant celui qui a le droit de venger le sang.
\VS{4}Le meurtrier s’enfuira dans l’une de ces villes, s’arrêtera à l’entrée de la porte de la ville, et il exposera son affaire aux anciens de cette ville-là, ils l’écouteront, et le recevront chez eux dans la ville, et lui donneront une demeure, afin qu’il habite avec eux.
\VS{5}Quand celui qui a le droit de venger le sang le poursuivra, ils ne livreront pas le meurtrier entre ses mains ; puisque c’est involontairement qu’il a tué son prochain, et qu’il ne le haïssait point auparavant.
\VS{6}Mais il demeurera dans cette ville-là, jusqu’à ce qu’il comparaisse devant l’assemblée pour être jugé, jusqu’à la mort du souverain sacrificateur qui sera en fonction ce temps-là. Alors le meurtrier s’en retournera, et reviendra dans sa ville et dans sa maison, dans la ville d’où il s’était enfui.
\VS{7}Ils consacrèrent donc Kédesch, en Galilée, dans la montagne de Nephthali ; Sichem dans la montagne d’Ephraïm ; et Kirjath-Arba, qui est Hébron, dans la montagne de Juda.
\VS{8}Et de l’autre côté du Jourdain, à l’orient de Jéricho, ils choisirent Betser, dans la tribu de Ruben, dans le désert, dans la plaine ; Ramoth en Galaad, dans la tribu de Gad ; et Golan en Basan, dans la tribu de Manassé.
\VS{9}Telles furent les villes désignées pour tous les enfants d’Israël et pour l’étranger en séjour au milieu d’eux, afin que quiconque aurait tué quelqu’un involontairement puisse s’y réfugier, et qu’il ne meure pas de la main de celui qui a le droit de venger le sang, avant d’avoir comparu devant l’assemblée.
\Chap{21}
\VerseOne{}Les chefs des pères de famille des Lévites s’approchèrent d’Eléazar, le sacrificateur, de Josué, fils de Nun, et des chefs des pères de famille des tribus des enfants d’Israël.
\VS{2}Ils leur parlèrent à Silo, dans le pays de Canaan, et dirent : Yahweh a ordonné par Moïse qu’on nous donne des villes pour habiter, et leurs banlieues pour nos bêtes.
\VS{3}Alors les enfants d’Israël donnèrent aux Lévites, sur leur héritage, les villes suivantes et leurs banlieues, d’après l’ordre de Yahweh.
\VS{4}On tira au sort pour les familles des Kehathites ; et les Lévites, fils d’Aaron, le sacrificateur eurent par le sort treize villes de la tribu de Juda, de la tribu de Siméon, et de la tribu de Benjamin.
\VS{5}Les autres fils de Kehath eurent par le sort dix villes des familles de la tribu d’Ephraïm, de la tribu de Dan, et de la demi-tribu de Manassé.
\VS{6}Les fils de Guerschon eurent par le sort treize villes, des familles de la tribu d’Issacar, de la tribu d’Aser, de la tribu de Nephthali, et de la demi-tribu de Manassé en Basan.
\VS{7}Les fils de Merari selon leurs familles, eurent douze villes, de la tribu de Ruben, de la tribu de Gad, et de la tribu de Zabulon.
\VS{8}Les enfants d’Israël donnèrent donc par le sort aux Lévites ces villes-là avec leurs banlieues, comme Yahweh l’avait ordonné par Moïse.
\VS{9}Ils donnèrent de la tribu des fils de Juda et de la tribu des fils de Siméon, ces villes, qui vont être nommées par leurs noms,
\VS{10}et qui furent pour les fils d’Aaron, qui étaient des familles des Kehathites, et des fils de Lévi, car le sort les avait indiqués les premiers.
\VS{11}Ils leur donnèrent Kirjath-Arba, qui est Hébron, dans la montagne de Juda, avec ses banlieues tout autour : Arba était le père d’Anak.
\VS{12}Mais quant au territoire de la ville, et à ses villages, on les donna à Caleb, fils de Jephunné, pour sa possession.
\VS{13}Ils donnèrent donc aux fils d’Aaron, le sacrificateur, les villes de refuge pour les meurtriers, Hébron, avec ses banlieues, et Libna avec ses banlieues.
\VS{14}Jatthir, avec ses banlieues, Eschthemoa, avec ses banlieues,
\VS{15}Holon, avec ses banlieues, Debir, avec ses banlieues,
\VS{16}Aïn, avec ses banlieues, Jutta, avec ses banlieues ; et Beth-Schémesch, avec ses banlieues ; neuf villes de ces deux tribus-là ;
\VS{17}et de la tribu de Benjamin, Gabaon, avec ses banlieues, et Guéba, avec ses banlieues,
\VS{18}Anathoth, avec ses banlieues, et Almon, avec ses banlieues ; quatre villes.
\VS{19}Toutes les villes des sacrificateurs, fils d’Aaron, furent treize villes, avec leurs banlieues.
\VS{20}Quant aux Lévites, appartenant aux familles des autres fils de Kehath, ils eurent par le sort des villes de la tribu d’Ephraïm.
\VS{21}On leur donna donc les villes de refuge pour les meurtriers, Sichem, avec ses banlieues, dans la montagne d’Ephraïm, et Guézer avec ses banlieues.
\VS{22}Kibtsaïm, avec ses banlieues, et Beth-Horon, avec ses banlieues ; quatre villes ;
\VS{23}et de la tribu de Dan, Eltheké, avec ses banlieues ; Guibbethon, avec ses banlieues,
\VS{24}Ajalon, avec ses banlieues, Gath-Rimmon, avec ses banlieues ; quatre villes.
\VS{25}Et de la demi-tribu de Manassé, Thaanac, avec ses banlieues ; et Gath-Rimmon, avec ses banlieues, deux villes.
\VS{26}Total des villes : dix villes avec leurs banlieues, pour les familles des autres fils de Kehath.
\VS{27}On donna aussi aux fils de Guerschon, d’entre les familles des Lévites : De la demi-tribu de Manassé les villes de refuge pour les meurtriers, Golan en Basan, avec ses banlieues, et Beeschthra, avec ses banlieues ; deux villes ;
\VS{28}et de la tribu d’Issacar, Kischjon, avec ses banlieues, Dabrath, avec ses banlieues,
\VS{29}Jarmuth, avec ses banlieues, En-Gannim, avec ses banlieues ; quatre villes ;
\VS{30}et de la tribu d’Aser, Mischeal, avec ses banlieues, Abdon, avec ses banlieues,
\VS{31}Helkath, avec ses banlieues, et Rehob, avec ses banlieues ; quatre villes ;
\VS{32}et de la tribu de Nephthali, les villes de refuge pour les meurtriers, Kédesch en Galilée avec ses banlieues, Hammoth-Dor, avec ses banlieues, et Karthan, avec ses banlieues ; trois villes.
\VS{33}Total des villes des Guerschonites, selon leurs familles : Treize villes, et leurs banlieues.
\VS{34}On donna aussi au reste des Lévites, qui appartenaient aux familles des fils de Merari : De la tribu de Zabulon, Jokneam, avec ses banlieues, Kartha, avec ses banlieues,
\VS{35}Dimna, avec ses banlieues, et Nahalal, avec ses banlieues ; quatre villes ;
\VS{36}et de la tribu de Ruben, Betser, avec ses banlieues, et Jahtsa, avec ses banlieues ;
\VS{37}Kedémoth, avec ses banlieues, et Méphaath, avec ses banlieues ; quatre villes ;
\VS{38}et de la tribu de Gad, les villes de refuge pour les meurtriers, Ramoth en Galaad, avec ses banlieues, et Mahanaïm, avec ses banlieues,
\VS{39}Hesbon, avec ses banlieues, et Jaezer, avec ses banlieues ; en tout quatre villes.
\VS{40}Total des villes qui échurent par le sort aux fils de Merari, selon leurs familles, formant le reste des familles des Lévites : Douze villes.
\VS{41}Total des villes des Lévites qui étaient parmi la possession des enfants d’Israël : Quarante-huit villes, et leurs banlieues.
\VS{42}Chacune de ces villes avait ses banlieues autour d’elle ; il en était ainsi de toutes ces villes-là.
\VS{43}Yahweh donna donc à Israël tout le pays qu’il avait juré de donner à leurs pères ; ils le possédèrent, et y habitèrent (1).
\VS{44}Yahweh leur accorda un parfait repos tout autour, selon tout ce qu’il avait juré à leurs pères ; aucun de leurs ennemis ne put leur résister, car Yahweh les livra entre leurs mains.
\VS{45}Il ne tomba pas un seul mot de toutes les bonnes paroles que Yahweh avait dites à la maison d'Israël : Toutes s’accomplirent.
\Chap{22}
\VerseOne{}Alors Josué appela les Rubénites, les Gadites et la demi-tribu de Manassé.
\VS{2}Il leur dit : Vous avez observé tout ce que Moïse, serviteur de Yahweh, vous a prescrit, et vous avez obéi à ma voix dans tout ce que je vous ai ordonné.
\VS{3}Vous n’avez pas abandonné vos frères, depuis une très longue période jusqu’à ce jour ; et vous avez gardé les ordres, les commandements de Yahweh votre Dieu.
\VS{4}Maintenant que Yahweh, votre Dieu, a donné du repos à vos frères, comme il le leur avait dit, retournez et allez dans vos tentes, dans le pays qui vous appartient, et que Moïse, serviteur de Yahweh, vous a donné de l’autre côté du Jourdain.
\VS{5}Ayez seulement soin d’observer les ordonnances et les lois que Moïse, serviteur de Yahweh, vous a prescrites : Aimez Yahweh votre Dieu, marchez dans toutes ses voies, gardez ses commandements, attachez-vous à lui, et servez-le de tout votre cœur et de toute votre âme.
\VS{6}Puis Josué les bénit et les renvoya ; et ils s’en allèrent vers leurs tentes.
\VS{7}Moïse avait donné à la moitié de la tribu de Manassé son héritage en Basan ; et Josué donna à l’autre moitié son héritage avec leurs frères de l’autre côté du Jourdain vers l’occident. Josué les renvoya dans leurs tentes, et les bénit.
\VS{8}Et il leur parla et dit : Vous retournez à vos tentes avec de grandes richesses, une très nombreuse quantité de bétail, avec une quantité considérable d’argent, d’or, d’airain, de fer, et de vêtements. Partagez avec vos frères le butin de vos ennemis.
\VS{9}Ainsi les fils de Ruben, les fils de Gad, et la demi-tribu de Manassé s’en retournèrent, et partirent de Silo, dans le pays de Canaan, après avoir quitté les enfants d’Israël, pour s’en aller dans le pays de Galaad, sur la terre de leur possession et où ils s’établirent, suivant ce que Yahweh avait ordonné par Moïse.
\VS{10}Quand ils furent arrivés aux frontières du Jourdain, qui appartiennent au pays de Canaan, les fils de Ruben, les fils de Gad, et la demi-tribu de Manassé y bâtirent un autel, près du Jourdain, un autel dont la grandeur frappait les regards.
\VS{11}Les enfants d’Israël apprirent que l’on disait : Voici, les fils de Ruben, les fils de Gad, et la demi-tribu de Manassé ont bâti un autel en face du pays de Canaan, sur les frontières du Jourdain, du côté des enfants d’Israël.
\VS{12}Lorsque les enfants d’Israël entendirent cela, toute l’assemblée des enfants d’Israël se réunit à Silo, pour monter en guerre contre eux.
\VS{13}Les enfants d’Israël envoyèrent vers les fils de Ruben, vers les fils de Gad, et vers la demi-tribu de Manassé, au pays de Galaad, Phinées, fils du sacrificateur Eléazar,
\VS{14}et avec lui dix princes, un prince par maison paternelle pour chacune des tribus d’Israël ; tous étaient chefs de maison paternelle parmi les milliers d’Israël.
\VS{15}Ils se rendirent auprès des fils de Ruben, des fils de Gad et de la demi-tribu de Manassé au pays de Galaad, et leur parlèrent, en disant :
\VS{16}Ainsi parle toute l’assemblée de Yahweh : Quelle est cette infidélité que vous avez commise contre le Dieu d’Israël, et pourquoi vous détournez-vous aujourd’hui de Yahweh, en vous bâtissant un autel, pour vous rebeller aujourd’hui contre Yahweh ?
\VS{17}Regardons-nous comme peu de chose l’iniquité de Peor (1), dont nous ne nous sommes pas encore bien purifiés jusqu’à présent, malgré la plaie qu’il attira sur l’assemblée de Yahweh ?
\VS{18}Et vous vous détournez aujourd’hui de Yahweh ! Si vous vous rebellez aujourd’hui contre Yahweh, demain il s’irritera contre toute l’assemblée d’Israël.
\VS{19}Si vous tenez pour impure la terre qui est votre propriété, passez sur la terre qui est la possession de Yahweh, où est fixé le Tabernacle de Yahweh, ayez votre possession parmi nous, mais ne vous révoltez point contre Yahweh, et ne soyez point rebelles contre nous, en vous bâtissant un autel, outre l’autel de Yahweh notre Dieu.
\VS{20}Acan (2), fils de Zérach, ne commit-il pas une infidélité en prenant des choses dévouées par le moyen de l'interdit, et la colère de Yahweh ne s’enflamma-t-elle pas contre toute l’assemblée d’Israël ? Cependant, cet homme ne fut pas le seul qui périt à cause de son iniquité.
\VS{21}Mais les fils de Ruben, les fils de Gad, et la demi-tribu de Manassé répondirent, et dirent aux chefs des milliers d’Israël :
\VS{22}Dieu (3), Dieu (4) Yahweh, Dieu (3), Dieu (4)Yahweh, le sait, et Israël lui-même le saura ! Si c’est par rébellion et par infidélité envers Yahweh, alors qu’il ne nous vienne point en aide aujourd’hui.
\VS{23}Si nous nous sommes bâti un autel pour nous détourner de Yahweh, si c’est pour y offrir des holocaustes, ou des offrandes, ou si c’est pour y faire des sacrifices d’offrande de paix, que Yahweh lui-même nous en demande compte !
\VS{24}C’est bien plutôt par une sorte d’inquiétude que nous avons fait cela, en pensant que vos fils pourraient un jour parler à nos fils et leur dire : Qu’y a-t-il de commun entre vous et Yahweh, le Dieu d’Israël ?
\VS{25}Puisque Yahweh a mis le Jourdain pour frontière entre nous et vous, fils de Ruben, et fils de Gad ; vous n’avez point de part à Yahweh ! Et ainsi vos fils feraient qu’un jour nos fils cesseraient de craindre Yahweh.
\VS{26}C’est pourquoi nous avons dit : Mettons-nous maintenant à bâtir un autel, non pour des holocaustes ni pour des sacrifices ;
\VS{27}mais afin qu’il serve de témoignage entre nous et vous, et entre nos descendants et les vôtres, que nous voulons servir Yahweh devant sa face par nos holocaustes et nos sacrifices d’expiation et d’offrande de paix, afin que vos fils ne disent pas un jour à nos fils : Vous n’avez point de part à Yahweh !
\VS{28}C’est pourquoi nous avons dit : Lorsqu’ils nous tiendront ce discours, ou à nos descendants, nous leur dirons : Voyez la forme de l’autel de Yahweh qu’ont fait nos pères, non pour des holocaustes, ni pour des sacrifices, mais afin qu’il soit témoin entre nous et vous.
\VS{29}Loin de nous la pensée de nous révolter contre Yahweh et de nous détourner aujourd’hui de Yahweh, en bâtissant un autel pour des holocaustes, pour des offrandes, et pour des sacrifices, outre l’autel de Yahweh notre Dieu, qui est devant son tabernacle !
\VS{30}Lorsque le sacrificateur Phinées, et les princes de l’assemblée, les chefs des milliers d’Israël qui étaient avec lui, eurent entendu les paroles que les fils de Ruben, les fils de Gad, et les fils de Manassé leur dirent, ils furent satisfaits.
\VS{31}Et Phinées, fils du sacrificateur Eléazar, dit aux fils de Ruben, aux fils de Gad, et aux fils de Manassé : Nous reconnaissons aujourd’hui que Yahweh est au milieu de nous, puisque vous n’avez point commis cette infidélité contre Yahweh ; vous avez ainsi délivré les enfants d’Israël de la main de Yahweh.
\VS{32}Phinées, fils du sacrificateur Eléazar, et les princes, quittèrent les fils de Ruben, les fils de Gad, et revinrent du pays de Galaad dans le pays de Canaan, auprès des enfants d’Israël, auxquels ils firent un rapport.
\VS{33}Et la chose plut aux enfants d’Israël ; ils bénirent Dieu, et ne parlèrent plus de monter en armes contre eux pour détruire le pays où habitaient les fils de Ruben, et les fils de Gad.
\VS{34}Les fils de Ruben, et les fils de Gad appelèrent l’autel Ed ; car, dirent-ils, il est témoin entre nous que Yahweh est Dieu.
\Chap{23}
\VerseOne{}Depuis de nombreux jours, Yahweh avait donné du repos à Israël devant tous les ennemis qui l’entouraient. Josué était vieux, fort avancé en âge.
\VS{2}Alors Josué convoqua tout Israël, ses anciens, ses chefs, ses juges, ses officiers, et leur dit : Je suis devenu vieux, fort avancé en âge.
\VS{3}Vous avez vu tout ce que Yahweh, votre Dieu, a fait à toutes ces nations devant vous ; car Yahweh, votre Dieu, est celui qui combat pour vous.
\VS{4}Voyez, je vous ai donné en héritage par le sort, selon vos tribus, ces nations qui sont restées, depuis le Jourdain, et toutes les nations que j’ai exterminées, jusqu’à la grande mer vers le soleil couchant.
\VS{5}Yahweh, votre Dieu, les repoussera devant vous et les chassera ; et vous posséderez leur pays en héritage, comme Yahweh, votre Dieu, vous l’a dit.
\VS{6}Appliquez-vous avec force à observer et à mettre en pratique tout ce qui est écrit dans le livre de la loi de Moïse, sans vous en détourner ni à droite ni à gauche.
\VS{7}Ne vous mêlez point avec ces nations qui sont restées parmi vous ; et ne faites point mention du nom de leurs dieux, et ne faites jurer personne par eux, ne les servez point, et ne vous prosternez point devant eux.
\VS{8}Mais attachez-vous à Yahweh, votre Dieu, comme vous l’avez fait jusqu’à ce jour.
\VS{9}C’est pour cela que Yahweh a chassé devant vous des nations grandes et puissantes ; nul n’a pu vous résister jusqu’à ce jour.
\VS{10}Un seul homme d’entre vous en poursuivait mille ; car Yahweh votre Dieu est celui qui combat pour vous, comme il vous l’a dit.
\VS{11}Veillez donc attentivement sur vos âmes, afin d’aimer Yahweh, votre Dieu.
\VS{12}Si vous vous détournez et que vous vous attachez au reste de ces nations qui sont demeurées parmi vous, si vous faites alliance par des mariages avec elles, et si vous formez ensemble des relations,
\VS{13}soyez certain que Yahweh, votre Dieu, ne continuera pas à chasser ces nations devant vous ; mais elles seront pour vous un piège et un filet, un fouet dans vos côtés et des épines dans vos yeux, jusqu’à ce que vous ayez péri de dessus cette bonne terre que Yahweh, votre Dieu, vous a donnée.
\VS{14}Voici, je m’en vais aujourd’hui par le chemin de toute la terre. Reconnaissez de tout votre cœur et de toute votre âme qu’aucune de toutes les bonnes paroles prononcées sur vous par Yahweh, votre Dieu, n’est restée sans effet ; toutes se sont accomplies pour vous, aucune n’est restée sans effet (1).
\VS{15}Et comme toutes les bonnes paroles que Yahweh, votre Dieu, vous a dites se sont accomplies pour vous, de même Yahweh accomplira pour vous toutes les paroles mauvaises, jusqu’à ce qu’il vous ait exterminés de dessus cette bonne terre que Yahweh, votre Dieu, vous a donnée.
\VS{16}Si vous transgressez l’alliance que Yahweh, votre Dieu, vous a prescrite, et si vous allez servir d’autres dieux et vous prosterner devant eux, la colère de Yahweh s’enflammera contre vous, et vous périrez promptement de dessus cette bonne terre qu’il vous a donnée.
\Chap{24}
\VerseOne{}Josué assembla toutes les tribus d’Israël à Sichem, et il convoqua les anciens d’Israël, ses chefs, ses juges, et ses officiers, qui se présentèrent devant Dieu.
\VS{2}Josué dit à tout le peuple : Ainsi parle Yahweh, le Dieu d’Israël : Vos pères, Térach père d’Abraham, et père de Nachor, ont anciennement habité de l’autre côté du fleuve, où ils servaient d’autres dieux.
\VS{3}Mais j’ai pris votre père Abraham de l’autre côté du fleuve, je lui fis parcourir tout le pays de Canaan, je multipliai sa postérité, et lui donnai Isaac.
\VS{4}Je donnai à Isaac, Jacob et Esaü ; et je donnai à Esaü le mont de Séir, pour le posséder ; mais Jacob et ses fils descendirent en Egypte.
\VS{5}Puis j’envoyai Moïse et Aaron, et je frappai l’Egypte, par les prodiges que j’opérai au milieu d’elle ; puis je vous en fis sortir.
\VS{6}Je fis donc sortir vos pères hors de l’Egypte, et vous arrivâtes à la mer. Les Egyptiens poursuivirent vos pères avec des chars et des cavaliers, jusqu’à la mer Rouge.
\VS{7}Alors ils crièrent à Yahweh. Et il mit des ténèbres entre vous et les Egyptiens, et ramena sur eux la mer, qui les couvrit. Vos yeux ont vu ce que j’ai fait aux Egyptiens. Puis vous restâtes longtemps dans le désert.
\VS{8}Ensuite je vous conduisis dans le pays des Amoréens, qui habitaient de l’autre côté du Jourdain, et ils combattirent contre vous. Mais je les livrai entre vos mains ; vous prîtes possession de leur pays, et je les détruisis devant vous.
\VS{9}Balak (1) aussi, fils de Tsippor, roi de Moab, se leva, et fit la guerre à Israël. Il fit appeler Balaam (2), fils de Beor, pour qu’il vous maudisse.
\VS{10}Mais je ne voulus point écouter Balaam ; il s’agenouilla et vous bénit, et je vous délivrai de la main de Balak.
\VS{11}Et vous passâtes le Jourdain, et arrivâtes près de Jéricho. Les habitants de Jéricho, les Amoréens, les Phéréziens, les Cananéens, les Héthiens, les Guirgasiens, les Héviens et les Jébusiens vous firent la guerre. Je les livrai entre vos mains,
\VS{12}et j’envoyai devant vous des frelons qui les chassèrent loin de votre face, comme les deux rois des Amoréens : Ce ne fut ni par ton épée, ni par ton arc.
\VS{13}Je vous donnai une terre que vous n’aviez point cultivée, des villes que vous n’aviez point bâties, et que vous habitez, et vous mangez les fruits des vignes et des oliviers que vous n’avez point plantés.
\VS{14}Maintenant, craignez Yahweh, et servez-le avec intégrité et avec fidélité. Ôtez les dieux que vos pères ont servis de l’autre côté du fleuve et en Egypte, et servez Yahweh.
\VS{15}Et s’il vous déplaît de servir Yahweh, choisissez aujourd’hui qui vous voulez servir, ou les dieux que servaient vos pères au-delà du fleuve, ou les dieux des Amoréens dans le pays desquels vous habitez. Mais moi et ma maison, nous servirons Yahweh.
\VS{16}Alors le peuple répondit, et dit : Que Dieu nous garde d’abandonner Yahweh pour servir d’autres dieux !
\VS{17}Car Yahweh, notre Dieu, est celui qui nous a fait monter, nous et nos pères, hors du pays d’Egypte, de la maison de servitude, qui a fait devant nos yeux ces grands signes, qui nous a gardés dans tout le chemin par lequel nous avons marché, et entre tous les peuples parmi lesquels nous avons passé.
\VS{18}Yahweh a chassé devant nous tous les peuples, et même les Amoréens qui habitaient ce pays. Nous servirons aussi Yahweh, car il est notre Dieu.
\VS{19}Josué dit au peuple : Vous ne pourrez pas servir Yahweh, car c’est un Dieu Saint, qui est jaloux, il ne pardonnera point votre rébellion et vos péchés.
\VS{20}Lorsque vous abandonnerez Yahweh et que vous servirez les dieux des étrangers, il reviendra vous faire du mal, et il vous consumera après vous avoir fait du bien.
\VS{21}Le peuple dit à Josué : Non ! Car nous servirons Yahweh.
\VS{22}Josué dit au peuple : Vous êtes témoins contre vous-mêmes que c’est vous qui avez choisi Yahweh pour le servir. Et ils répondirent : Nous en sommes témoins.
\VS{23}Maintenant donc ôtez les dieux étrangers qui sont au milieu de vous, et tournez votre cœur vers Yahweh, le Dieu d’Israël.
\VS{24}Et le peuple répondit à Josué : Nous servirons Yahweh notre Dieu et nous obéirons à sa voix.
\VS{25}Ce jour-là, Josué traita alliance avec le peuple, et lui donna des lois et des ordonnances à Sichem.
\VS{26}Josué écrivit ces paroles dans le livre de la loi de Dieu. Il prit aussi une grande pierre (3), qu’il dressa là sous le chêne qui était dans le lieu consacré à Yahweh.
\VS{27}Josué dit à tout le peuple : Voici, cette pierre servira de témoin contre nous, car elle a entendu toutes les paroles que Yahweh nous a déclarées ; elle servira de témoin contre vous, afin que vous ne reniiez pas votre Dieu.
\VS{28}Puis Josué renvoya le peuple, chacun dans son héritage.
\VS{29}Après ces choses, Josué, fils de Nun, serviteur de Yahweh, mourut, âgé de cent dix ans.
\VS{30}Et on l’ensevelit dans le territoire de son héritage, à Thimnath-Sérach, dans la montagne d’Ephraïm, du côté du nord de la montagne de Gaasch.
\VS{31}Israël servit Yahweh tout le temps de Josué, et tout le temps des anciens qui survécurent à Josué, qui avaient connu toutes les œuvres que Yahweh avait faites pour Israël.
\VS{32}Les os de Joseph (4), que les enfants d’Israël avaient rapportés d’Egypte, furent ensevelis à Sichem, dans la portion du champ que Jacob avait achetée des fils de Hamor, père de Sichem, pour cent kesita, et qui appartint à l’héritage des fils de Joseph.
\VS{33}Eléazar, fils d’Aaron, mourut, on l’enterra à Guibeath-Phinées, qui avait été donnée à son fils Phinées, dans la montagne d’Ephraïm.
\PPE{}
\end{multicols}
