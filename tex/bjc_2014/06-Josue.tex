\ShortTitle{Jos.}\BookTitle{Josué}\BFont
\noindent\hrulefill
{\footnotesize
\textit{
\bigskip
{\centering{}
\\Auteur~: Probablement Josué
\\(Heb.~: Yehowshuwa)
\\Signification~: Yahweh est salut
\\Thème~: La conquête de Canaan
\\Date de rédaction~: 14\up{ème} siècle av. J.-C.\\}
}
\textit{
\\Né en Egypte, Josué, fils de Nun, originaire de la tribu d'Ephraïm, servit Moïse de la sortie d'Egypte jusqu'à sa mort. Choisi par Dieu pour succéder au prophète, il fut le seul de l'ancienne génération, avec Caleb, à avoir survécu à la longue épreuve du désert. Ce livre relate les étapes du voyage du peuple d'Israël et sa conquête de la terre promise. Il présente par ailleurs les victoires acquises par la puissance de Yahweh sous la conduite de Josué. C'est l'histoire de la prise de Canaan et de son partage entre les douze tribus d'Israël.\bigskip
}
}
\par\nobreak\noindent\hrulefill
\begin{multicols}{2}
\Chap{1}
\TextTitle{Josué succède à Moïse, à sa mort\FTNTT{De. 34:9.}}
\VerseOne{}Or il arriva après la mort de Moïse, serviteur de Yahweh, que Yahweh parla à Josué, fils de Nun, qui avait servi Moïse, en disant~:
\VS{2}Moïse, mon serviteur est mort~; maintenant donc, lève-toi, passe ce Jourdain, toi et tout ce peuple, pour entrer dans le pays que je donne aux enfants d'Israël.
\VS{3}Tout lieu que foulera la plante de votre pied, je vous l'ai donné, comme je l'ai déclaré à Moïse\FTNT{De. 11:24.}.
\VS{4}Vos frontières seront depuis ce désert et le Liban, jusqu'au grand fleuve, le fleuve de l'Euphrate, tout le pays des Héthiens jusqu'à la grande mer, vers le soleil couchant.
\VS{5}Nul ne tiendra devant toi, tous les jours de ta vie. Je serai avec toi comme j'ai été avec Moïse~; je ne te délaisserai point, et je ne t'abandonnerai point\FTNT{De. 31:6~; Hé. 13:5-6.}.
\VS{6}Fortifie-toi et prends courage, car c'est toi qui mettras ce peuple en possession du pays dont j'ai juré à leurs pères de leur donner.
\VS{7}Seulement fortifie-toi et renforce-toi de plus en plus, afin que tu prennes garde de faire selon toute la loi que Moïse mon serviteur t'a ordonnée~; ne t'en détourne point ni à droite ni à gauche, afin que tu prospères partout où tu iras.
\VS{8}Que ce livre de la loi ne s'éloigne point de ta bouche, mais médite-le jour et nuit, pour agir fidèlement selon tout ce qui y est écrit\FTNT{La clé d'une vie chrétienne épanouie est la Parole de Dieu. Méditer signifie~: 
\\- Murmurer la Parole de Dieu~: Partout où nous sommes, nous pouvons dans nos cœurs murmurer les promesses du Seigneur (Ps. 63:5-8~; Ps. 119:11).
\\- Proclamer à haute voix~: Il est intéressant de noter que le mot hébreu traduit dans Jos. 1:8 par méditer est traduit par «~proclamer~» ou «~dire~» dans Pr. 8:7~; Ps. 35:8~; Ps. 77:13. 
\\- Réfléchir profondément~: Il faut être dans le lieu secret (Mt. 6:5-6). En Israël il est de coutume d'aller étudier la Torah à l'ombre d'un figuier. Voir Jn. 1:43-51.}~; car c'est alors que tu auras du succès dans tes entreprises, c'est alors que tu réussiras.
\VS{9}Ne t'ai-je pas donné cet ordre, fortifie-toi et prends courage~? Ne t'épouvante point et ne t'effraie point~; car Yahweh ton Dieu est avec toi partout où tu iras.
\TextTitle{Josué prend la direction du peuple}
\VS{10}Après cela, Josué donna cet ordre aux officiers du peuple, en disant~:
\VS{11}Passez par le camp, ordonnez au peuple et dites-lui~: Préparez-vous des provisions, car dans trois jours vous passerez ce Jourdain pour aller prendre possession du pays que Yahweh, votre Dieu, vous donne afin que vous le possédiez.
\VS{12}Josué parla aussi aux Rubénites, aux Gadites et à la demi-tribu de Manassé, en disant~:
\VS{13}Souvenez-vous de la parole que Moïse, serviteur de Yahweh, vous a prescrite, en disant~: Yahweh votre Dieu vous a accordé du repos, et vous a donné ce pays.
\VS{14}Vos femmes, vos petits-enfants, et vos bêtes resteront dans le pays que Moïse vous a donné de l'autre côté du Jourdain~; mais vous tous, hommes vaillants, vous passerez en armes devant vos frères, et vous les aiderez\FTNT{Ex. 13:18.}~;
\VS{15}jusqu'à ce que Yahweh ait accordé du repos à vos frères comme à vous, et qu'ils soient aussi en possession du pays que Yahweh, votre Dieu, leur donne. Puis vous reviendrez prendre possession du pays qui est votre propriété, et que vous a donné Moïse, serviteur de Yahweh, de l'autre côté du Jourdain, vers l'orient.
\VS{16}Ils répondirent à Josué, en disant~: Nous ferons tout ce que tu nous as ordonné, et nous irons partout où tu nous enverras.
\VS{17}Nous t'obéirons comme nous avons obéi à Moïse~; seulement que Yahweh ton Dieu soit avec toi, comme il a été avec Moïse.
\VS{18}Tout homme qui sera rebelle à ton ordre, et qui n'obéira point à tes paroles dans tout ce que tu lui commanderas, sera mis à mort~; seulement, fortifie-toi, et sois courageux~!
\Chap{2}
\TextTitle{Josué envoie deux espions à Jéricho~; ils sont reçus par Rahab\FTNTT{Ja. 2:25.}}
\VerseOne{}Or Josué fils de Nun, envoya secrètement de Sittim deux hommes, pour épier secrètement le pays, et il leur dit~: Allez, examinez le pays, et Jéricho. Ils partirent donc et entrèrent dans la maison d'une femme prostituée, nommée Rahab\FTNT{Rahab avait entendu parler du Dieu des Hébreux et avait placé son espérance de salut en lui (Ro. 10:11). Par cet acte de foi, sa destinée a changé. Cette femme qui était vouée à une double condamnation du fait de sa condition de prostituée (De. 23:17) et de son appartenance à une nation païenne qui devait être dévouée à la façon de l'interdit (Jos. 6), a été sauvée avec sa famille (Ac. 2:21~; Ac. 16:31 ). Ainsi, bien des siècles plus tard, on ne la mentionnera plus comme une prostituée, mais comme une ancêtre du Sauveur et une héroïne de la foi (Mt. 1:5~; Hé. 11:21). Rahab est donc l'archétype des païens qui sont rentrés dans l'alliance de Dieu par la foi.}, et ils y couchèrent.
\VS{2}Alors on dit au roi de Jéricho~: Voici, des hommes sont venus ici cette nuit de la part des enfants d'Israël pour explorer le pays.
\VS{3}Et le roi de Jéricho envoya dire à Rahab~: Fais sortir les hommes qui sont venus chez toi et qui sont entrés dans ta maison~; car ils sont venus pour explorer tout le pays.
\VS{4}Or la femme prit les deux hommes et les cacha~; et elle dit~: Il est vrai que des hommes sont venus chez moi, mais je ne savais pas d'où ils étaient~;
\VS{5}et comme on fermait la porte sur le soir, ces hommes sont sortis~; je ne sais pas où ces hommes sont allés~; poursuivez-les bien vite car vous les atteindrez.
\VS{6}Or elle les avait fait monter sur le toit et les avait cachés sous des tiges de lin qu'elle avait arrangées sur le toit.
\VS{7}Et quelques gens les poursuivirent par le chemin du Jourdain jusqu'aux passages~; et on ferma la porte après que ceux qui les poursuivaient furent sortis.
\VS{8}Or avant qu'ils se couchent, elle monta vers eux sur le toit~;
\VS{9}et leur dit~: Je sais que Yahweh vous a donné ce pays, et que la terreur de votre nom nous a saisis, et que tous les habitants du pays perdent courage à cause de vous\FTNT{Ex. 23:27.}.
\VS{10}Car nous avons entendu que Yahweh a mis à sec devant vous les eaux de la Mer Rouge à votre sortie du pays d'Egypte~; et ce que vous avez fait aux deux rois des Amoréens qui étaient de l'autre côté du Jourdain, à Sihon et à Og, que vous avez détruits complètement en les dévouant par le moyen de l'interdit.
\VS{11}Nous l'avons entendu, et notre cœur a fondu, et depuis aucun homme n'a plus eu de courage à cause de vous. Car Yahweh, votre Dieu, est le Dieu des cieux en haut et de la terre\FTNT{De. 4:39.} en bas.
\VS{12}Maintenant donc, je vous prie, jurez-moi par Yahweh, que puisque j'ai usé de bonté envers vous, vous userez aussi de bonté envers la maison de mon père, 
\VS{13}et que vous me donnerez un signe de votre fidélité et qu'assurément vous laisserez vivre mon père, ma mère, mes frères, mes sœurs, et tous ceux qui leur appartiennent, et que vous sauverez nos âmes de la mort.
\VS{14}Et ces hommes lui répondirent~: Nos personnes répondront pour vous jusqu'à la mort, pourvu que vous ne divulguiez pas cette affaire~; et quand Yahweh nous aura donné le pays nous userons envers toi de bonté et de vérité. 
\TextTitle{Les espions s'enfuient aidés par Rahab}
\VS{15}Elle les fit donc descendre avec une corde par la fenêtre~; car sa maison était sur la muraille de la ville, et elle habitait sur la muraille de la ville. 
\VS{16}Et elle leur dit~: Allez à la montagne, de peur que ceux qui vous poursuivent ne vous rencontrent, et cachez-vous là pendant trois jours jusqu'à ce qu'ils soient de retour. Après cela vous suivrez votre chemin.
\VS{17}Et ces hommes lui dirent~: Voici comment nous serons quittes de ce serment que tu nous as fait faire.
\VS{18}Voici, quand nous entrerons dans le pays, tu lieras ce cordon de fil d'écarlate à la fenêtre par laquelle tu nous auras fait descendre, et tu recueilleras chez toi, dans cette maison, ton père et ta mère, tes frères, et toute la famille de ton père.
\VS{19}Et quiconque sortira hors de la porte de ta maison, son sang sera sur sa tête, et nous en serons quittes~; mais quiconque sera avec toi, dans la maison, son sang sera sur notre tête si quelqu'un met la main sur lui.
\VS{20}Et si tu divulgues cette affaire, nous serons quittes du serment que tu nous as fait faire.
\VS{21}Et elle répondit~: Que cela soit tel que vous l'avez dit. Alors elle les laissa aller. Ils s'en allèrent et elle lia le cordon de fil d'écarlate à la fenêtre.
\VS{22}Et ils marchèrent et arrivèrent à la montagne, où ils restèrent trois jours, jusqu'à ce que ceux qui les poursuivaient soient de retour. Ceux qui les poursuivaient les cherchèrent par tout le chemin, mais ils ne les trouvèrent pas.
\VS{23}Ainsi ces deux hommes s'en retournèrent, descendirent de la montagne, passèrent le Jourdain. Ils vinrent auprès de Josué, fils de Nun. Ils lui racontèrent toutes les choses qui leur étaient arrivées.
\VS{24}Et ils dirent à Josué~: Certainement, Yahweh a livré tout le pays entre nos mains, et même, tous les habitants ont perdu courage à notre vue.
\Chap{3}
\TextTitle{Israël traverse le Jourdain à sec}
\VerseOne{}Or Josué se leva de bon matin, lui et tous les enfants d'Israël partirent de Sittim, ils vinrent jusqu'au Jourdain, et ils logèrent là cette nuit, avant de le traverser.
\VS{2}Et au bout de trois jours les officiers traversèrent le milieu du camp,
\VS{3}et donnèrent cet ordre au peuple, en disant~: Dès que vous verrez l'arche de l'alliance de Yahweh, votre Dieu, portée par les prêtres, les Lévites, vous partirez de votre quartier, et vous marcherez après elle.
\VS{4}Et afin que vous n'approchiez point d'elle, il y aura entre vous et elle une distance de la mesure d'environ deux mille coudées. Elle vous fera connaître le chemin par lequel vous devez marcher~; car vous n'avez pas encore passé par ce chemin.
\VS{5}Josué dit au peuple~: Sanctifiez-vous, car Yahweh fera demain des choses merveilleuses au milieu de vous\FTNT{Ex. 19:10-11.}.
\VS{6}Josué parla aussi aux prêtres, en disant~: Portez l'arche de l'alliance, et passez devant le peuple. Ainsi ils portèrent l'arche de l'alliance, et marchèrent devant le peuple.
\VS{7}Or Yahweh dit à Josué~: Aujourd'hui je commencerai à t'élever aux yeux de tout Israël, afin qu'ils sachent que je serai aussi avec toi, comme j'ai été avec Moïse.
\VS{8}Tu donneras cet ordre aux prêtres qui portent l'arche de l'alliance, en leur disant~: Dès que vous arriverez au bord des eaux du Jourdain, vous vous arrêterez dans le Jourdain.
\VS{9}Et Josué dit aux enfants d'Israël~: Approchez-vous d'ici, et écoutez les paroles de Yahweh, votre Dieu.
\VS{10}Puis Josué dit~: Vous reconnaîtrez à ceci que le Dieu vivant est au milieu de vous et qu'il chassera et déshéritera devant vous les Cananéens, les Héthiens, les Héviens, les Phéréziens, les Guirgasiens, les Amoréens et les Jébusiens.
\VS{11}Voici, l'arche de l'alliance du Seigneur de toute la terre va passer devant vous dans le Jourdain.
\VS{12}Maintenant, prenez douze hommes des tribus d'Israël, un homme de chaque tribu.
\VS{13}Et il arrivera qu'aussitôt que les plantes des pieds des prêtres qui portent l'arche de Yahweh, le Seigneur de toute la terre, seront posés dans les eaux du Jourdain, les eaux du Jourdain seront coupées, les eaux, dis-je, qui descendent d'en haut, et elles s'arrêteront en un monceau\FTNT{Ps. 114:3.}.
\VS{14}Et il arriva que le peuple étant parti de ses tentes pour passer le Jourdain, et les prêtres qui portaient l'arche de l'alliance, étaient devant le peuple.
\VS{15}Aussitôt que ceux qui portaient l'arche furent arrivés au Jourdain, et que les pieds des prêtres qui portaient l'arche furent mouillés au bord de l'eau. C'était à l'époque ou le Jourdain regorge par-dessus toutes ses rives durant le temps de la moissonFTNT{\vref{La traversée c'est faite à la période ou le Jourdain était en crue, c'est à dire la fin de la saison des pluie, au début des moisson. Cette époque correspondait à la fin du moi d'avril, dans cette région du monde, ce qui, dans le calendrier hébreu, équivaut au mois de Nissan ou Abib. Voir 1 Ch. 12:15}}.
\VS{16}Les eaux qui descendent d'en haut, s'arrêtèrent, et s'élevèrent en un monceau, à une très grande distance, depuis la ville d'Adam, qui est à côté de Tsarthan~; et celles d'en bas, qui descendaient vers la mer de la plaine, qui est la mer salée, furent totalement coupées. Le peuple passa vis-à-vis de Jéricho.
\VS{17}Mais les prêtres qui portaient l'arche de l'alliance de Yahweh, s'arrêtèrent de pied ferme sur le sec, au milieu du Jourdain, pendant que tout Israël passait à sec, jusqu'à ce que tout le peuple ait achevé de passer le Jourdain.
\Chap{4}
\TextTitle{Josué dresse un monument de pierres en souvenir de la traversée}
\VerseOne{}Or il arriva que quand tout le peuple eut achevé de passer le Jourdain, que Yahweh parla à Josué et dit~:
\VS{2}Prenez douze hommes parmi le peuple, un homme de chaque tribu.
\VS{3}Et donnez-leur cet ordre, en disant~: Prenez ici, du milieu du Jourdain, de la place où les prêtres se sont arrêtés de pied ferme, douze pierres, que vous emporterez avec vous, et vous les poserez au lieu où vous passerez cette nuit.
\VS{4}Josué appela les douze hommes qu'il choisit parmi les enfants d'Israël, un homme de chaque tribu.
\VS{5}Et il leur dit~: Passez devant l'arche de Yahweh, votre Dieu, au milieu du Jourdain, et que chacun de vous charge une pierre sur son épaule, selon le nombre des tribus des enfants d'Israël~;
\VS{6}afin que cela soit un signe au milieu de vous. Et quand vos fils interrogeront à l'avenir leurs pères, en disant~: Que signifient ces pierres-ci~?
\VS{7}Alors vous leur répondrez~: Les eaux du Jourdain ont été coupées devant l'arche de l'alliance de Yahweh~; lorsqu'elle passa le Jourdain, les eaux du Jourdain ont été arrêtées~; c'est pourquoi ces pierres-là seront à jamais un souvenir pour les enfants d'Israël.
\VS{8}Les enfants d'Israël firent donc comme Josué leur avait ordonné. Ils prirent douze pierres du milieu du Jourdain, comme Yahweh l'avait ordonné à Josué, selon le nombre des tribus des enfants d'Israël. Ils les emportèrent avec eux et les posèrent au lieu où ils devaient passer la nuit.
\VS{9}Josué dressa aussi douze pierres au milieu du Jourdain, à l'endroit où les pieds des prêtres qui portaient l'arche de l'alliance s'étaient arrêtés~; et elles y sont restées jusqu'à ce jour.
\VS{10}Les prêtres donc qui portaient l'arche se tinrent debout au milieu du Jourdain, jusqu'à ce que tout ce que Yahweh avait ordonné à Josué de dire au peuple soit accompli, selon tout ce que Moïse avait prescrit à Josué. Et le peuple se hâta de passer.
\VS{11}Et quand tout le peuple eut achevé de passer, alors l'arche de Yahweh et les prêtres passèrent devant le peuple.
\VS{12}Et les fils de Ruben, les fils de Gad, et la demi-tribu de Manassé passèrent en armes devant les enfants d'Israël, comme Moïse le leur avait dit\FTNT{No. 32:20-29.}.
\VS{13}Ils passèrent, dis-je, dans les plaines de Jérico environ quarante mille hommes en équipage de guerre, devant Yahweh, pour combattre. 
\VS{14}Ce jour-là, Yahweh éleva Josué à la vue de tout Israël, et ils le craignirent, comme ils avaient craint Moïse, tous les jours de sa vie.
\VS{15}Yahweh parla à Josué, et dit~:
\VS{16}Ordonne aux prêtres qui portent l'arche du témoignage qu'ils montent hors du Jourdain.
\VS{17}Et Josué donna cet ordre aux prêtres, en disant~: Montez hors du Jourdain.
\VS{18}Or sitôt que les prêtres, qui portaient l'arche de l'alliance de Yahweh furent montés hors du milieu du Jourdain, et qu'ils eurent mis la plante de leurs pieds sur le sec, les eaux du Jourdain retournèrent à leur place, et coulèrent comme auparavant sur tous les rivages.
\VS{19}Le peuple donc monta hors du Jourdain le dixième jour du premier mois, et il campa à Guilgal, à l'orient de Jéricho.
\VS{20}Josué aussi dressa à Guilgal les douze pierres qu'ils avaient prises du Jourdain.
\VS{21}Et il parla aux enfants d'Israël et leur dit~: Quand vos enfants interrogeront à l'avenir leurs pères, et leur diront~: Que signifient ces pierres-ci~?
\VS{22}Vous l'apprendrez à vos enfants, en leur disant~: Israël a passé ce Jourdain à sec.
\VS{23}Car Yahweh, votre Dieu, a fait tarir les eaux du Jourdain devant vous jusqu'à ce que vous eussiez passé, comme Yahweh, votre Dieu, l'avait fait à la Mer Rouge, qu'il mit à sec devant nous, jusqu'à ce que nous eussions passé,
\VS{24}afin que tous les peuples de la terre sachent que la main de Yahweh est puissante, et afin que vous ayez toujours la crainte de Yahweh, votre Dieu.
\Chap{5}
\TextTitle{La crainte s'empare des Amoréens}
\VerseOne{}Or il arriva qu'aussitôt que tous les rois des Amoréens qui étaient au-delà du Jourdain, vers l'occident, et tous les rois des Cananéens qui étaient près de la mer, apprirent que Yahweh avait mis à sec les eaux du Jourdain devant les enfants d'Israël, jusqu'à ce qu'ils eussent passé, leur cœur fondit, et il n'y avait plus de courage en eux à cause des enfants d'Israël.
\TextTitle{Israël circoncis à nouveau~; la fin de la manne}
\VS{2}En ce temps-là, Yahweh dit à Josué~: Fais-toi des couteaux de pierre tranchants, et circoncis de nouveau les enfants d'Israël, une seconde fois.
\VS{3}Et Josué se fit des couteaux de pierre tranchants, et circoncit les enfants d'Israël sur la colline d'Araloth.
\VS{4}Or la raison pour laquelle Josué les circoncit, c'est que tout le peuple sorti d'Egypte, tous les mâles, dis-je, hommes de guerre étaient morts en chemin dans le désert, après leur sortie d'Egypte.
\VS{5}Et tout le peuple sorti d'Egypte était circoncis, mais aucun du peuple né dans le désert en chemin n'avait été circoncis, après leur sortie d'Egypte.
\VS{6}Car les enfants d'Israël avaient marché dans le désert quarante ans jusqu'à ce que soit consummée toute la nation des hommes de guerre qui étaient sortis d'Egypte, et qui n'avaient point écouté la voix de Yahweh~; auxquels Yahweh avait juré qu'il ne leur laisserait point voir le pays qu'il avait juré à leurs pères de nous donner, pays où coulent le lait et le miel\FTNT{No. 14:32-33.}.
\VS{7}Et il a suscité à leur place leurs enfants que Josué circoncit, parce qu'ils étaient incirconcis~; car on ne les avait pas circoncis pendant le voyage.
\VS{8}Et quand on eut achevé de circoncire tout le peuple, ils restèrent dans leur camp, jusqu'à ce qu'ils soient guéris.
\VS{9}Et Yahweh dit à Josué~: Aujourd'hui j'ai roulé de dessus vous l'opprobre de l'Egypte. Et ce lieu-là fut appelé Guilgal jusqu'à ce jour.
\VS{10}Ainsi les enfants d'Israël campèrent à Guilgal, et célébrèrent la Pâque le quatorzième jour du mois, sur le soir, dans les plaines de Jéricho\FTNT{Ex. 12:6.}.
\VS{11}Et dès le lendemain de la Pâque, ils mangèrent du blé du pays, savoir, des pains sans levain et du grain rôti, en ce même jour\FTNT{Ex. 12:39~; Lé. 2:14}.
\VS{12}Et la manne cessa dès le lendemain de la Pâque, après qu'ils eurent manger du blé du pays~; les enfants d'Israël n'eurent plus de manne, mais ils mangèrent les récoltes de la terre de Canaan cette année-là\FTNT{Ex. 16:35.}.
\TextTitle{Rencontre avec le chef de l'armée de Yahweh}
\VS{13}Or il arriva, comme Josué était près de Jéricho, qu'il leva les yeux et regarda. Voici, un homme qui avait son épée nue à la main, se tenait debout devant lui. Josué alla vers lui et lui dit~: Es-tu des nôtres ou de nos ennemis~?
\VS{14}Et il répondit~: Non, mais je suis le Chef de l'armée de Yahweh, je viens maintenant. Josué tomba à terre sur son visage, l'adora, et lui dit~: Qu'est-ce que mon Seigneur dit à son serviteur~?
\VS{15}Et le Chef de l'armée de Yahweh dit à Josué~: Délie tes souliers de tes pieds~; car le lieu sur lequel tu te tiens est saint\FTNT{Ex. 3:5.}. Et Josué fit ainsi.
\Chap{6}
\TextTitle{Jéricho miraculeusement livré à Israël~; Rahab sauvée}
\VerseOne{}Or Jéricho était barricadée et soigneusement fermée, à cause des enfants d'Israël. Personne ne sortait, et personne n'entrait.
\VS{2}Et Yahweh dit à Josué~: Regarde, j'ai livré entre tes mains Jéricho et son roi, ses hommes vaillants.
\VS{3}Vous tous donc, hommes de guerre, vous ferez le tour de la ville, en tournant une fois autour d'elle. Tu feras ainsi durant six jours.
\VS{4}Et Sept prêtres porteront sept shofars retentissants devant l'arche. Mais au septième jour, vous ferez sept fois le tour de la ville et les prêtres sonneront des shofars.
\VS{5}Et quand ils sonneront avec la corne de bélier, aussitôt que vous entendrez le son du shofar retentissant, tout le peuple poussera un grand cri de joie et la muraille de la ville tombera sur elle. Et le peuple montera, les hommes devant lui.
\VS{6}Josué donc, fils de Nun, appela les prêtres et leur dit~: Portez l'arche de l'alliance et que sept prêtres portent sept shofars devant l'arche de Yahweh.
\VS{7}Il dit aussi au peuple~: Passez et faites le tour de la ville, que tous ceux qui seront armés passent devant l'arche de Yahweh.
\VS{8}Et quand Josué eut parlé au peuple, les sept prêtres qui portaient les sept cornes de béliers devant Yahweh passèrent et sonnèrent des cornes. Et l'arche de l'alliance de Yahweh les suivait.
\VS{9}Et les hommes qui étaient armés marchaient devant les prêtres qui sonnaient des shofars~; mais l'arrière-garde suivait derrière l'arche~; on sonnait des shofars en marchant.
\VS{10}Or Josué avait donné cet ordre au peuple, en disant~: Vous ne pousserez point de cris de joie et vous ne ferez point entendre votre voix. Et il ne sortira point un seul mot de votre bouche, jusqu'au jour où je vous dirai~: Poussez des cris de joie~! Alors vous crierez.
\VS{11}L'arche de Yahweh fit ainsi le tour de la ville, en tournant tout autour une fois, puis on revint au camp, et on y passa la nuit.
\VS{12}Ensuite Josué se leva de bon matin, et les prêtres portèrent l'arche de Yahweh.
\VS{13}Et les sept prêtres qui portaient les sept cornes de bélier devant l'arche de Yahweh se mirent en marche et sonnèrent du shofar. Et les hommes armés allaient devant eux~; puis l'arrière-garde suivait l'arche de Yahweh~; on sonnait des shofars en marchant.
\VS{14}Ainsi ils firent une fois le tour de la ville le deuxième jour, et ils retournèrent au camp. Ils firent de même durant six jours.
\VS{15}Mais quand le septième jour fut venu, ils se levèrent dès le matin à l'aube du jour, et ils firent sept fois le tour de la ville de la même manière~; ce fut le seul jour où ils firent sept fois le tour de la ville.
\VS{16}Et à la septième fois, comme les prêtres sonnaient des shofars, Josué dit au peuple~: Poussez des cris de joie, car Yahweh vous a donné la ville~!
\VS{17}La ville sera dévouée par le moyen de l'interdit à Yahweh, elle et toutes les choses qui y sont~; seulement Rahab, la prostituée\FTNT{Rahab sauva sa famille par sa foi en Dieu (Ac. 16:31). Voir Josué 2.}, vivra, elle et tous ceux qui seront avec elle dans la maison, parce qu'elle a caché soigneusement les messagers que nous avions envoyés.
\VS{18}Mais quoi qu'il en soit gardez-vous de l'interdit, de peur que vous ne vous mettiez en interdit, et que vous ne mettriez le camp d'Israël en interdit et que vous le troubliez\FTNT{De. 7:26.}.
\VS{19}Mais tout l'argent et tout l'or, tous les objets d'airain et de fer seront consacrés à Yahweh, ils entreront dans le trésor de Yahweh\FTNT{No. 31:54.}.
\VS{20}Le peuple donc poussa de cris de joie et on sonna des shofars. Et quand le peuple entendit le son des shofars, il poussa de grands cris de joie et la muraille tomba sur elle-même\FTNT{Hé. 11:30.}. Alors le peuple monta dans la ville, les hommes devant le peuple. Et ils prirent la ville. 
\VS{21}Et ils la dévouèrent entièrement par le moyen de l'interdit, et passèrent au fil de l'épée tout ce qui était dans la ville, depuis l'homme jusqu'à la femme, depuis l'enfant jusqu'au vieillard, même jusqu'aux bœufs, aux brebis et aux ânes.
\VS{22}Mais Josué dit aux deux hommes qui avaient espionné le pays~: Entrez dans la maison de cette femme prostituée, et faites-la sortir de là, avec tous ceux qui lui appartiennent, selon que vous lui avez juré.
\VS{23}Les jeunes hommes donc qui avaient espionné le pays, entrèrent et firent sortir Rahab, et son père, et sa mère et ses frères, avec tous ceux qui lui appartenaient~; ils firent aussi sortir toutes les familles qui lui appartenaient, et les mirent hors du camp d'Israël.
\VS{24}Puis ils allumèrent le feu et brûlèrent la ville et tout ce qui s'y trouvait~; seulement ils mirent l'argent et l'or, les objets d'airain et de fer dans le trésor de la maison de Yahweh.
\VS{25}Ainsi Josué sauva la vie à Rahab la prostituée, la maison de son père, et tous ceux qui lui appartenaient~; et elle a habité au milieu d'Israël jusqu'à ce jour, parce qu'elle avait caché les messagers que Josué avait envoyés pour explorer Jéricho.
\VS{26}Et en ce temps-là Josué jura, en disant~: Maudit soit devant Yahweh l'homme qui se mettra à rebâtir cette ville de Jéricho~! Il la fondera sur son premier-né, et il posera ses portes sur son puîné\FTNT{Le mot «~puîné~» désigne l'enfant fille ou garçon né après l'aîné. Cette parole s'est accomplie en 1 R. 16:34.}.
\VS{27}Yahweh fut avec Josué, et sa renommée se répandit dans tout le pays.
\Chap{7}
\TextTitle{Israël battu à Aï suite au péché d'Acan}
\VerseOne{}Mais les enfants d'Israël se rendirent coupables au sujet de l'interdit. Car Acan, fils de Carmi, fils de Zabdi, fils de Zérach, de la tribu de Juda, prit de l'interdit, et la colère de Yahweh s'enflamma contre les enfants d'Israël.
\VS{2}Car Josué envoya de Jéricho des hommes vers Aï, qui est près de Beth-Aven, à l'orient de Béthel. Il leur parla, et dit~: Montez, et reconnaissez le pays. Ces hommes donc montèrent et reconnurent Aï.
\VS{3}Et étant retournés vers Josué, ils lui dirent~: Que tout le peuple n'y monte point mais qu'environ deux mille ou trois mille hommes y montent, et ils battront Aï. Ne fatigue pas tout le peuple en l'envoyant là, car ils sont en petit nombre.
\VS{4}Ainsi, environ trois mille hommes du peuple y montèrent, mais ils s'enfuirent devant les gens d'Aï.
\VS{5}Et les gens d'Aï leur tuèrent environ trente-six hommes~; car ils les poursuivirent depuis la porte jusqu'à Schebarim, et les battirent à la descente. Le cœur du peuple se fondit et devint comme de l'eau.
\VS{6}Alors Josué déchira ses vêtements, et se jeta sur le visage contre terre, devant l'arche de Yahweh jusqu'au soir, lui et les anciens d'Israël, et ils jettèrent de la poussière sur leur tête.
\VS{7}Et Josué dit~: Helas~! Seigneur Yahweh, pourquoi as-tu fait si magnifiquement passer le Jourdain à ce peuple, pour nous livrer entre les mains des Amoréens, et nous faire périr~? Oh~! Que n'avons-nous eu dans l'esprit de demeurer de l'autre côté du Jourdain~!
\VS{8}Hélas~! Seigneur, que dirai-je, puisqu'Israël a tourné le dos devant ses ennemis~?
\VS{9}Les Cananéens et tous les habitants du pays l'entendront~; ils nous envelepperont, et ils retrancheront notre nom de dessus la terre. Et que feras-tu à ton grand Nom~?
\VS{10}Alors Yahweh dit à Josué~: Lève-toi~! Pourquoi te jettes-tu ainsi le visage contre terre~?
\VS{11}Israël a péché~; ils ont transgressé mon alliance que je leur avais prescrite~; même ils ont pris de l'interdit~; même ils en ont dérobé~; même ils ont menti~; et même ils l'ont caché parmi leurs objets\FTNT{Il est impossible de remporter une victoire contre Satan en ayant avec soi des choses qui lui appartiennent (Jn. 14:30). Celui qui pèche est du diable nous dit la Parole de Dieu (1 Jn. 3:4-10). Les grandes victoires sont remportées par ceux qui se sanctifient et invoquent le Nom de Jésus-Christ.}.
\VS{12}C'est pourquoi les enfants d'Israël ne pourront subsister devant leurs ennemis~; ils tourneront le dos devant leurs ennemis~; car ils sont devenus un interdit. Je ne serai plus avec vous si vous ne détruisez pas l'interdit du milieu de vous.
\VS{13}Lève-toi, sanctifie le peuple, et dis~: Sanctifiez-vous pour demain~; car ainsi parle Yahweh, le Dieu d'Israël~: Il y a de l'interdit au milieu de toi, Israël~! Tu ne pourras subsister et faire face à tes ennemis jusqu'à ce que vous ayez ôté l'interdit du milieu de vous.
\VS{14}Vous vous approcherez donc le matin selon vos tribus~; et la tribu que Yahweh aura saisi s'approchera selon les familles, et la famille que Yahweh aura saisie s'approchera selon les maisons, et la maison que Yahweh aura saisie s'approchera selon les hommes.
\VS{15}Alors celui qui aura été saisi avec l'interdit sera brûlé au feu, lui et tout ce qui lui appartient parce qu'il a transgressé l'alliance de Yahweh, et qu'il a commis une infamie en Israël.
\VS{16}Josué donc se leva de bon matin, et fit approcher Israël selon ses tribus, et la tribu de Juda fut saisie.
\VS{17}Puis il fit approcher les familles de Juda, et la famille de Zérach fut saisie. Puis il fit approcher les hommes de la famille de ceux qui étaient descendants de Zérach, et Zabdi fut saisie.
\VS{18}Et quand il fit approcher la maison de Zabdi par hommes, Acan fils de Carmi, fils de Zabdi, fils de Zérach, de la tribu de Juda, fut saisi.
\VS{19}Josué dit à Acan~: Mon fils, je te prie donne gloire à Yahweh, le Dieu d'Israël, et fais-lui confession. Déclare-moi je te prie ce que tu as fait, ne me le cache point.
\VS{20}Et Acan répondit à Josué, et dit~: J'ai péché il est vrai, contre Yahweh, le Dieu d'Israël, et voici ce que j'ai fait.
\VS{21}J'ai vu parmi le butin un beau manteau de Schinear\FTNT{Ge. 10:6-12.}, deux cents sicles d'argent et un lingot d'or du poids de cinquante sicles~; je les ai convoités, je les ai pris et voilà, ces choses sont cachées dans la terre au milieu de ma tente, et l'argent est sous le manteau.
\VS{22}Alors Josué envoya des messagers qui coururent à cette tente~; et voici, le manteau était caché dans la tente d'Acan, et l'argent sous le manteau.
\VS{23}Ils les tirèrent donc du milieu de la tente et les apportèrent à Josué et à tous les enfants d'Israël, et ils les déposèrent devant Yahweh.
\VS{24}Alors Josué et tout Israël avec lui, prirent Acan, fils de Zérach, l'argent, le manteau, le lingot d'or, ses fils et ses filles, ses bœufs, ses ânes et ses brebis, sa tente et tout ce qui lui appartenait, et ils les firent monter dans la vallée d'Acor.
\VS{25}Et Josué dit~: Pourquoi nous as-tu troublés~? Yahweh te troublera aujourd'hui. Et tout Israël le lapida avec des pierres, et les brûlèrent au feu, après les avoir lapidés avec des pierres.
\VS{26}Et ils dressèrent sur lui un grand monceau de pierres, qui dure jusqu'à ce jour. Et Yahweh apaisa l'ardeur de sa colère. C'est pourquoi ce lieu-là a été appelé jusqu'à aujourd'hui, la vallée d'Acor\FTNT{2 S. 18:17.}.
\Chap{8}
\TextTitle{Victoire d'Israël à Aï}
\VerseOne{}Puis Yahweh dit à Josué~: Ne crains point, et ne t'effraie de rien\FTNT{De. 1:21~; De. 7:18.}~! Prends avec toi tout le peuple propre à la guerre et lève-toi, et monte contre Aï. Regarde, j'ai livré entre tes mains le roi d'Aï et son peuple, sa ville et son pays.
\VS{2}Et tu traiteras Aï et son roi, comme tu as fais Jéricho et son roi~: Seulement vous pillerez pour vous le butin et les bêtes. Place des gens en embuscade derrière la ville.
\VS{3}Josué donc se leva avec tout le peuple propre à la guerre, pour monter contre Aï. Josué choisit trente mille vaillants hommes armés, et les envoya de nuit.
\VS{4}Et il leur donna cet ordre en disant~: Voyez, vous qui serez en embuscade derrière la ville~; ne vous éloignez pas beaucoup de la ville, mais tenez-vous prêts.
\VS{5}Et moi et tout le peuple qui est avec moi, nous nous approcherons de la ville. Et quand ils sortiront à notre rencontre, comme ils ont fait la première fois, nous nous enfuirons devant eux.
\VS{6}Ainsi ils sortiront après nous, jusqu'à ce que nous les ayons attirés hors de la ville~; car ils diront~: Ils fuient devant nous comme la première fois~; parce que nous fuirons devant eux.
\VS{7}Alors vous vous lèverez de l'embuscade, et vous vous saisirez de la ville~; car Yahweh, votre Dieu, la livrera entre vos mains.
\VS{8}Et quand vous aurez pris la ville, vous y mettrez le feu~; vous agirez selon la parole de Yahweh. Regardez, je vous l'ai ordonné.
\VS{9}Josué donc les envoya, et ils allèrent se mettre en embuscade, et se tinrent entre Béthel et Aï, à l'occident d'Aï. Mais Josué resta cette nuit-là au milieu du peuple.
\VS{10}Puis Josué se leva de bon matin, et dénombra le peuple~; et il monta lui et les anciens d'Israël, devant le peuple vers Aï.
\VS{11}Et tout le peuple propre à la guerre qui étaient avec lui, monta et s'approcha~; et ils vinrent en face de la ville et campèrent du côté du nord d'Aï~; et la vallée était entre lui et Aï.
\VS{12}Il prit aussi environ cinq mille hommes, et les mit en embuscade entre Béthel et Aï, à l'occident de la ville.
\VS{13}Après que tout le camp eut pris position au nord de la ville, et l'embuscade à l'occident de la ville, cette nuit-là, Josué s'avança au milieu de la vallée.
\VS{14}Or il arriva qu'aussitôt que le roi d'Aï l'eut vu, les hommes de la ville se hâtèrent, et se levèrent de bon matin, et au temps marqué, le Roi et tout son peuple sortirent à la campagne contre Israël pour le combattre. Or il ne savait pas qu'il y eût des gens en embuscade contre lui derrière la ville.
\VS{15}Alors Josué et tout Israël feignirent d'être battus devant eux, et ils s'enfuirent par le chemin du désert.
\VS{16}Alors tout le peuple qui était dans la ville d'Aï, fut assemblé à grand cri pour les poursuivre. Ils poursuivirent Josué, et ils furent ainsi attirés loin de la ville.
\VS{17}Il ne resta pas un seul homme dans Aï ni dans Béthel qui ne sortit contre Israël. Ils laissèrent la ville ouverte, et ils poursuivirent Israël.
\VS{18}Alors Yahweh dit à Josué~: Etends vers Aï le javelot qui est dans ta main, car je la livrerai entre tes mains. Et Josué étendit vers la ville le javelot qui était dans sa main.
\VS{19}Aussitôt qu'il eut étendu sa main, les hommes qui étaient en embuscade se levèrent précipitamment du lieu où ils étaient~; ils pénétrèrent dans la ville, la prirent, et se hâtèrent de mettre le feu dans la ville.
\VS{20}Et les gens d'Aï, se tournant derrière eux, regardèrent~; et voici, la fumée de la ville montait vers le ciel, et ils n'y eut en eux aucune force pour fuir ça ou là. Et le peuple qui fuyait vers le désert se tourna contre ceux qui le poursuivaient.
\VS{21}Et Josué et tout Israël, voyant que ceux qui étaient en embuscade avaient pris la ville, et que la fumée de la ville montait, se retournèrent, et frappèrent les gens d'Aï.
\VS{22}Les autres aussi sortirent de la ville contre eux ; ainsi ils furent encerclés par les Israélites, les uns d'un côté, et les autres de l'autre ; et ils furent tellement battus qu'on n'en laissa aucun qui demeurât en vie, ou qui échappât\FTNT{De. 7:2.}.
\VS{23}ils prirent aussi vivant le roi d'Aï, et le présentèrent à Josué.
\VS{24}Et quand les Israélites eurent achevé de tuer tous les habitants d'Aï dans la campagne, dans le désert, où ils les avaient poursuivis, et que tous furent tombés sous le tranchant de l'épée, jusqu'à être entièrement défaits, tous les Israélites revinrent vers Aï, et la frappèrent au tranchant de l'épée.
\VS{25}Et tous ceux qui tombèrent ce jour-là, tant des hommes que des femmes, furent au nombre de douze mille, tous gens d'Aï.
\VS{26}Et Josué ne retira point sa main qu'il tenait étendue avec l'étandard, jusqu'à ce que tous les habitants d'Aï aient été entièrement dévoués par le moyen de l'interdit.
\VS{27}Seulement les Israélites pillèrent pour eux les bêtes et le butin de cette ville-là, suivant ce que Yahweh avait prescrit à Josué\FTNT{No. 31:22-26.}.
\VS{28}Josué donc brûla Aï, et en fit un monceau perpétuel de ruines, jusqu'à aujourd'hui.
\VS{29}Puis il fit pendre le roi d'Aï à un arbre jusqu'au temps du soir. Et comme le soleil se couchait, Josué ordonna qu'on descende de l'arbre son cadavre~; on le jeta à l'entrée de la porte de la ville, puis on dressa sur lui un grand amas de pierres, qui subsiste encore aujourd'hui.
\TextTitle{Sacrifices offerts à Yahweh et lecture de la loi de Moïse}
\VS{30}Alors Josué bâtit un autel à Yahweh, le Dieu d'Israël, sur la montagne d'Ebal,
\VS{31}comme Moïse, serviteur de Yahweh, l'avait ordonné aux enfants d'Israël, ainsi qu'il est écrit dans le livre de la loi de Moïse~: Il fit cet autel de pierres brutes sur lesquelles personne ne porta le fer\FTNT{L'autel devait être construit avec des pierres taillées par Dieu lui-même dans la nature (Ex. 20:25). L'Eglise du Seigneur est construite avec des pierres vivantes, taillées par Dieu et non par les hommes (Mt. 16:18). Babylone est construite avec des briques, œuvre des hommes (Ge. 11:1-3).}~; et ils offrirent dessus des holocaustes à Yahweh, et sacrifièrent des sacrifices d'offrande de paix\FTNT{Voir commentaire en Lé. 3:1.}.
\VS{32}Il écrivit aussi là, sur les pierres une copie de la loi que Moïse avait mise par écrit devant les enfants d'Israël.
\VS{33}Et tout Israël, ses anciens, ses officiers et ses juges étaient des deux côtés de l'arche, en face des prêtres qui sont de la race de Lévi, qui portaient l'arche de l'alliance de Yahweh, les étrangers comme les Hébreux naturels, une moitié du côté du mont Garizim\FTNT{Voir Jn. 4:19-24.}, et l'autre moitié du côté du mont Ebal, selon l'ordre qu'avait précédemment donné Moïse, serviteur de Yahweh, de bénir le peuple d'Israël.
\VS{34}Et après cela, il lut tout haut toutes les paroles de la loi, tant les bénédictions que les malédictions, selon tout ce qui est écrit dans le livre de la loi.
\VS{35}Il n'y eut rien de tout ce que Moïse avait prescrit, que Josué ne lise tout haut devant toute l'assemblée d'Israël, des femmes et des petits-enfants, et des étrangers qui marchaient au milieu d'eux.
\Chap{9}
\TextTitle{Josué tombe dans la ruse des Gabaonites}
\VerseOne{}Or dès que tous les rois qui étaient au-delà du Jourdain, dans la montagne et dans la plaine, et sur toute la côte de la grande mer, jusque près du Liban~: les Héthiens, les Amoréens, les Cananéens, les Phéréziens, les Héviens et les Jébusiens, eurent appris ces choses,
\VS{2}ils s'assemblèrent tous d'un commun accord pour faire la guerre à Josué et à Israël.
\VS{3}Mais les habitants de Gabaon, ayant entendu ce que Josué avait fait à Jéricho et à Aï,
\VS{4}usèrent de ruse, car ils se mirent en chemin et contrefirent les ambassadeurs et prirent de vieux sacs pour leurs ânes, et de vieilles outres de vin déchirées et recousues,
\VS{5}Et ils avaient à leurs pieds de vieux souliers raccommodés et de vieux habits sur eux~; et tout le pain qu'ils avaient pour nourriture était sec et moisi.
\VS{6}Et ils arrivèrent auprès de Josué au camp de Guilgal, et lui dirent, ainsi qu'à tous les hommes d'Israël~: Nous sommes venus d'un pays éloigné, maintenant donc traitez alliance avec nous.
\VS{7}Et les hommes d'Israël répondirent à ces Héviens~: Peut-être que vous habitez au milieu de nous, et comment traiterions-nous alliance avec vous~?
\VS{8}Mais ils dirent à Josué~: Nous sommes tes serviteurs. Alors Josué leur dit~: Qui êtes-vous~? Et d'où venez-vous~?
\VS{9}Ils lui répondirent~: Tes serviteurs sont venus d'un pays très éloigné, sur la renommée de Yahweh, ton Dieu~; car nous avons entendu sa renommée, et toutes les choses qu'il a faites en Egypte,
\VS{10}et tout ce qu'il a fait aux deux rois des Amoréens, qui étaient au-delà du Jourdain, Sihon, roi de Hesbon, et Og, roi de Basan, qui demeurait à Aschtaroth.
\VS{11}Et nos anciens et tous les habitants de notre pays nous ont dit~: Prenez avec vous des provisions pour le chemin, et allez au-devant d'eux, et dites-leur~: Nous sommes vos serviteurs, et maintenant traitez alliance avec nous.
\VS{12}Voci notre pain~: Nous l'avons pris dans nos maisons tout chaud pour notre provision, le jour où nous sommes partis pour venir vers vous, mais maintenant voici, il est devenu sec et moisi.
\VS{13}Et voici aussi les outres de vin neuves que nous avons remplies, elles se sont déchirées~; nos habits et nos souliers sont usés à cause de la longueur de la marche.
\VS{14}Les hommes d'Israël prirent de leur provision, et aucun d'eux ne consulta la bouche de Yahweh.
\VS{15}Car Josué fit la paix avec eux, et traita avec eux une alliance par laquelle il devait leur laisser la vie, et les chefs de l'assemblée le leur jurèrent.
\TextTitle{Les Gabaonites démasqués}
\VS{16}Mais il arriva, trois jours après l'alliance traitée avec eux, qu'ils apprirent que c'étaient leurs voisins et qu'ils habitaient parmi eux.
\VS{17}Car les enfants d'Israël partirent, et arrivèrent à leurs villes le troisième jour. Leurs villes étaient Gabaon, Kephira, Beéroth, et Kirjath-Jearim.
\VS{18}Et les enfants d'Israël ne les frappèrent point, parce que les chefs de l'assemblée leur avaient juré par Yahweh, le Dieu d'Israël. Mais toute l'assemblée murmura contre les chefs.
\VS{19}Alors tous les chefs dirent à toute l'assemblée~: Nous leur avons juré par Yahweh, le Dieu d'Israël, c'est pourquoi maintenant nous ne pouvons pas les frapper.
\VS{20}Faisons-leur ceci, et qu'on les laisse vivre afin qu'il n'y ait pas de colère contre nous, à cause du serment que nous leur avons fait.
\VS{21}Ils vivront, leur dirent les chefs. Mais ils furent employés à couper le bois et à puiser l'eau pour toute l'assemblée, comme les chefs le leur avaient dit\FTNT{2 S. 21:1-14. La présence des Gabaonites en plein centre de Canaan tendait à isoler les tribus du nord de celles du sud, favorisant ainsi le schisme des deux royaumes (1 R. 12).}.
\VS{22}Car Josué les fit appeler, et leur parla, en disant~: Pourquoi nous avez-vous trompés, en nous disant~: Nous sommes très éloignés de vous, alors que vous habitez au milieu de nous~?
\VS{23}Maintenant vous êtes maudits~; il y aura toujours des esclaves parmi vous, des coupeurs de bois et des puiseurs d'eau pour la maison de mon Dieu.
\VS{24}Et ils répondirent à Josué, et dirent~: Après qu'il ait été exactement rapporté à tes serviteurs les ordres que Yahweh, ton Dieu, avait ordonnés à Moïse, son serviteur, pour vous donner tout le pays et pour en exterminer tous les habitants devant vous~; nous avons extrêmement crains pour nos personnes à cause de vous et nous avons fait ceci. 
\VS{25}Et maintenant nous voici entre tes mains~; fais-nous comme il te semblera bon et juste de nous faire.
\VS{26}Il leur fit donc ainsi et il les délivra de la main des enfants d'Israël, de sorte qu'il ne les tuèrent point.
\VS{27}Et en ce jour-là, Josué les établit coupeurs de bois et puiseurs d'eau pour l'assemblée, et pour l'autel de Yahweh, jusqu'à aujourd'hui, dans le lieu qu'il choisirait.
\Chap{10}
\TextTitle{Josué secoure Gabaon des cinq rois des Amoréens}
\VerseOne{}Or quand Adoni-Tsédek, roi de Jérusalem, entendit que Josué avait pris Aï, et qu'il l'avait entièrement détruite par le moyen de l'interdit, ayant fait à Aï et à son roi, comme il avait fait à Jéricho et à son roi, et que les habitants de Gabaon avaient fait la paix avec Israël, et étaient au milieu d'eux,
\VS{2}Il eut une grande frayeur. Parce que Gabaon était une grande ville, comme une ville royale, et elle était plus grande qu'Aï, et parce que tous ses hommes étaient vaillants.
\VS{3}C'est pourquoi Adoni-Tsédek, roi de Jérusalem, envoya dire à Hoham, roi d'Hébron, et à Piream, roi de Jarmuth, et à Japhia, roi de Lakis, et à Debir, roi d'Eglon~:
\VS{4}Montez vers moi, et aidez-moi afin que nous frappions Gabaon, car elle a fait la paix avec Josué et avec les enfants d'Israël.
\VS{5}Ainsi cinq rois des Amoréens, savoir, le roi de Jérusalem, le roi d'Hébron, le roi de Jarmuth, le roi de Lakis, et le roi d'Eglon, s'assemblèrent et montèrent avec toutes leurs armées~; et ils campèrent près de Gabaon, et lui firent la guerre.
\VS{6}Alors les gens de Gabaon dirent à Josué au camp de Guilgal~: Ne retire point tes mains de tes serviteurs, monte rapidement vers nous, délivre-nous, et donne-nous du secours~; car tous les rois des Amoréens qui habitent aux montagnes se sont rassemblés contre nous.
\VS{7}Josué donc monta de Guilgal, et avec lui tout le peuple qui était propre à la guerre, et tous les hommes forts et vaillants.
\TextTitle{Yahweh accorde à Israël une grande victoire à Makkéda}
\VS{8}Et Yahweh dit à Josué~: Ne les crains point, car je les ai livré entre tes mains, et aucun d'eux ne tiendra devant toi.
\VS{9}Josué arriva subitement sur eux, après avoir marché toute la nuit depuis Guilgal.
\VS{10}Yahweh les mit en déroute devant Israël, qui en fit un grand carnage près de Gabaon, et les poursuivit par le chemin de la montagne de Beth-Horon, les battit jusqu'à Azéka, et jusqu'à Makkéda.
\VS{11}Et comme ils s'enfuyaient devant Israël, et qu'ils étaient à la descente de Beth-Horon, Yahweh fit tomber du ciel sur eux de grosses pierres jusqu'à Azéka, et ils périrent~; ceux qui moururent des pierres de grêle furent plus nombreux que ceux qui furent tués avec l'épée par les enfants d'Israël.
\VS{12}Alors Josué parla à Yahweh, le jour où Yahweh livra les Amoréens aux enfants d'Israël, et dit en présence d'Israël~: Soleil, arrête-toi sur Gabaon, et toi lune, sur la vallée d'Ajalon~!
\VS{13}Et le soleil s'arrêta, et la lune aussi s'arrêta, jusqu'à ce que le peuple ait tiré vengeance de ses ennemis. Cela n'est-il pas écrit dans le livre du Juste~? Le soleil s'arrêta au milieu du ciel et ne se hâta point de se coucher environ un jour entier\FTNT{Ha. 3:11.}.
\VS{14}Et il n'y a point eu de jour semblable à celui-là, ni avant ni après, où Yahweh exauça la voix d'un homme~; car Yahweh combattait pour Israël.
\VS{15}Et Josué, et tout Israël avec lui, retourna au camp à Guilgal.
\VS{16}Au reste, ces cinq rois restants s'enfuirent, et se cachèrent dans une caverne à Makkéda.
\VS{17}Et on le rapporta à Josué, en disant~: On a trouvé les cinq rois cachés dans une caverne à Makkéda.
\VS{18}Et Josué dit~: Roulez de grosses pierres à l'entrée de la caverne et mettez près d'elle quelques hommes pour les garder.
\VS{19}Mais vous, ne vous arrêtez pas, poursuivez vos ennemis, attaquez-les par-derrière jusqu'au dernier, ne les laissez pas entrer dans leurs villes, car Yahweh, votre Dieu, les a livrés entre vos mains.
\VS{20}Et quand Josué et les enfants d'Israël eurent achevé d'en faire une très grande boucherie, jusqu'à les détruire entièrement, ceux d'entre eux qui s'étaient échappés se retirèrent dans les villes fortifiées,
\VS{21}tout le peuple revint en paix au camp vers Josué à Makkéda, et personne ne remua sa langue contre les enfants d'Israël.
\VS{22}Alors Josué dit~: Ouvrez l'entrée de la caverne, et amenez-moi ces cinq rois hors de la caverne.
\VS{23}Et ils firent ainsi, et ils lui amenèrent hors de la caverne ces cinq rois~: Le roi de Jérusalem, le roi d'Hébron, le roi de Jarmuth, le roi de Lakis et le roi d'Eglon.
\VS{24}Et après qu'ils eurent amené à Josué ces cinq rois hors de la caverne, Josué appela tous les hommes d'Israël, et dit aux chefs des gens de guerre qui étaient allés avec lui~: Approchez-vous, mettez vos pieds sur les cous de ces rois. Ils s'approchèrent, et mirent leurs pieds sur leurs cous\FTNT{Ps. 110:1.}.
\VS{25}Alors Josué leur dit~: Ne craignez point, et ne soyez point effrayés, fortifiez-vous, et ayez du courage, car Yahweh traitera ainsi tous vos ennemis contre lesquels vous combattez.
\VS{26}Et après cela, Josué les frappa et les fit mourir, il les fit pendre à cinq arbres, et ils restèrent pendus à ces arbres jusqu'au soir.
\VS{27}Et comme le soleil se couchait, Josué ordonna qu'on les descende de ces arbres, et on les jeta dans la caverne où ils s'étaient cachés, et on mit à l'entrée de la caverne de grosses pierres qui y sont demeurées jusqu'à ce jour\FTNT{De. 21:23.}.
\VS{28}Josué prit aussi Makkéda le même jour, la frappa du tranchant de l'épée, et dévoua à la façon de l'interdit son roi et ses habitants, et ne laissa échapper personne qui était dans cette ville. Et il fit au roi de Makkéda comme il fait au roi de Jéricho.
\TextTitle{Conquête des territoires du sud}
\VS{29}Après cela, Josué, et tout Israël avec lui, passa de Makkéda à Libna, et fit la guerre à Libna.
\VS{30}Et Yahweh la livra aussi entre les mains d'Israël, avec son roi, et il la frappa du tranchant de l'épée, elle et tous ceux qui s'y trouvaient~; il n'en laissa échapper aucune personne qui était dans cette ville~; et il fit à son roi comme il avait fait au roi de Jéricho.
\VS{31}Ensuite Josué, et tout Israël avec lui, passa de Libna à Lakis, campa devant elle, et lui fit la guerre.
\VS{32}Et Yahweh livra Lakis entre les mains d'Israël, qui la prit le deuxième jour, et la frappa du tranchant de l'épée, et toutes les personnes qui s'y trouvaient, comme il avait fait à Libna.
\VS{33}Alors Horam, roi de Guézer, monta pour secourir Lakis. Josué le frappa, lui et son peuple, de sorte qu'il n'en laissa pas échapper un seul homme.
\VS{34}Après cela Josué, et tout Israël avec lui, passa de Lakis à Eglon~; ils campèrent devant elle, et lui firent la guerre.
\VS{35}Ils la prirent le jour même, la frappèrent du tranchant de l'épée~; et Josué dévoua à la façon de l'interdit ce jour-là toutes les personnes qui y étaient, comme il avait fait à Lakis.
\VS{36}Puis Josué, et tout Israël avec lui, monta d'Eglon à Hébron, et ils lui firent la guerre.
\VS{37}Et ils la prirent, et la frappèrent du tranchant de l'épée, avec son roi, toutes ses villes, et toutes les personnes qui y étaient~; il n'en laissa échapper aucune, comme il avait fait à Eglon~; et il dévoua à la façon de l'interdit, toutes les personnes qui y étaient.
\VS{38}Ensuite Josué, et tout Israël avec lui, retourna vers Debir, et ils lui firent la guerre.
\VS{39}Et il la prit, avec son roi et toutes ses villes~; et ils les frappèrent du tranchant de l'épée, et dévouèrent à la façon de l'interdit toutes les personnes qui y étaient~; il n'en laissa échapper aucune. Il fait à Debir et à son roi comme il avait fait à Hébron, et comme il avait fait à Libna et à son roi.
\VS{40}Josué donc frappa tout ce pays, la montagne et le midi, la plaine et les coteaux, et tous leurs rois~; il n'en laissa échapper aucun, et il dévoua par le moyen de l'interdit toutes les personnes qui y respiraient, comme Yahweh, le Dieu d'Israël, l'avait ordonné\FTNT{De. 20:16-17.}.
\VS{41}Ainsi Josué les battit depuis Kadès-Barnéa jusqu'à Gaza, et tout le pays de Gosen jusqu'à Gabaon.
\VS{42}Josué prit tous ces rois en même temps et leur pays, parce que Yahweh, le Dieu d'Israël, combattait pour Israël.
\VS{43}Après quoi Josué, et tout Israël avec lui, retourna au camp à Guilgal.
\Chap{11}
\TextTitle{Conquête des territoires du nord}
\VerseOne{}Et aussitôt que Jabin, roi de Hatsor, eut appris ces choses, il envoya des messagers à Jobab, roi de Madon, au roi de Schimron, et au roi d'Acschaph,
\VS{2}et aux rois qui habitaient vers le nord, aux montagnes et dans la plaine, vers le midi de Kinnéreth, dans la vallée, et sur les hauteurs de Dor vers l'occident,
\VS{3}aux Cananéens qui étaient à l'orient et à l'occident, aux Amoréens, aux Héthiens, aux Phéréziens, aux Jébusiens dans les montagnes, et aux Héviens au pied de la montagne de l'Hermon, dans le pays de Mitspa.
\VS{4}Ils sortirent donc avec toutes leurs armées, un grand peuple par leur grand nombre, comme le sable qui est sur le bord de la mer, il y avait aussi des chevaux et des chars en très grand nombre.
\VS{5}Tous ces rois se réunirent, et campèrent ensemble près des eaux de Mérom, pour combattre contre Israël.
\VS{6}Et Yahweh dit à Josué~: Ne les crains point, car demain, à cette même heure, je les livrerai tous, blessés à mort, devant Israël. Tu couperas les jarrets à leurs chevaux, et brûleras au feu leurs chars\FTNT{2 S. 8:4.}.
\VS{7}Josué donc, et tous les gens de guerre avec lui vinrent subitement sur eux près des eaux de Mérom, et ils se précipitèrent au milieu d'eux.
\VS{8}Et Yahweh les livra entre les mains d'Israël~; ils les battirent, et les poursuivirent jusqu'à Sidon la grande, jusqu'aux eaux de Misrephoth-Maïm, et jusqu'à la vallée de Mitspa vers l'orient, et ils les battirent tellement qu'ils ne laissèrent aucun survivant.
\VS{9}Et Josué leur fit comme Yahweh lui avait dit~; il coupa les jarrets de leurs chevaux, et brûla au feu leurs chars.
\VS{10}A son retour, et dans le même temps, Josué prit Hatsor, et frappa son roi avec l'épée~; car Hatsor avait été auparavant la capitale de tous ces royaumes.
\VS{11}On frappa aussi du tranchant de l'épée et l'on dévoua à la façon de l'interdit tous ceux qui s'y trouvaient, il ne resta rien de ce qui respirait, et l'on brûla au feu Hatsor.
\VS{12}Josué prit aussi toutes les villes de ces rois, et tous leurs rois, et les frappa du tranchant de l'épée, et il les dévoua à la façon de l'interdit, comme Moïse, serviteur de Yahweh, l'avait ordonné.
\VS{13}Mais Israël ne brûla aucune des villes situées sur des collines, excepté de Hatsor seule, que Josué brûla.
\VS{14}Et les enfants d'Israël pillèrent pour eux tout le butin de ces villes et le bétail~; mais ils frappèrent du tranchant de l'épée tous les hommes, jusqu'à ce qu'ils les aient exterminés, ils n'y laissèrent aucun qui respirait.
\VS{15}Comme Yahweh l'avait ordonné à Moïse son serviteur, ainsi Moïse l'avait ordonné à Josué~; et Josué le fit ainsi~; de sorte qu'il n'omit rien de tout ce que Yahweh avait ordonné à Moïse. 
\TextTitle{Josué s'empare de tout le pays}
\VS{16}Josué donc prit tout ce pays-là, la montagne et tout le pays du midi, avec tout le pays de Gosen, la vallée et la plaine, la montagne d'Israël et ses vallées.
\VS{17}Depuis la montagne de Halak, qui s'élève vers Séir, jusqu'à Baal-Gad dans la vallée du Liban, au pied de la montagne d'Hermon. Il prit aussi tous leurs rois, les battit et les fit mourir.
\VS{18} Josué fit la guerre plusieurs jours contre tous ces rois.
\VS{19}Il n'y eut aucune ville qui fit la paix avec les enfants d'Israël, excepté les Héviens qui habitaient à Gabaon~; ils les prirent toutes par la guerre.
\VS{20}Car cela venait de Yahweh, qu'ils endurcissent leur cœur pour qu'ils sortent en bataille contre Israël, afin qu'il les dévoue à la façon de l'interdit, sans qu'il y ait pour eux de miséricorde, et qu'il les extermine, comme Yahweh l'avait ordonné à Moïse\FTNT{Ex. 4:21~; De. 2:30~; 1 R. 12:15.}.
\VS{21}En ce même temps-là aussi, Josué se mit en marche, et il extermina les Anakim des montagnes d'Hébron, de Debir, d'Anab, et de toute la montagne de Juda, et de toute la montagne d'Israël~; Josué, dis-je, les dévoua à la façon de l'interdit avec leurs villes.
\VS{22}Il ne resta aucun Anakim dans le pays des enfants d'Israël~; il n'en resta seulement qu'à Gaza, à Gath et à Asdod\FTNT{2 S. 21:20.}.
\VS{23}Josué donc prit tout le pays, suivant tout ce que Yahweh avait dit à Moïse. Et Josué le donna en héritage à Israël, selon leurs portions, et leurs tribus. Et le pays fut en repos et sans avoir guerre.
\Chap{12}
\TextTitle{Liste des rois vaincus par Moïse et Josué}
\VerseOne{}Voici les rois du pays que les enfants d'Israël frappèrent, et dont ils possédèrent le pays de l'autre côté du Jourdain, vers l'orient, depuis le torrent de l'Arnon jusqu'à la montagne de l'Hermon, et toute la plaine vers l'orient.
\VS{2}Savoir, Sihon, roi des Amoréens, qui habitait à Hesbon, et qui dominait depuis Aroër, qui est sur le bord du torrent de l'Arnon, et depuis le milieu du torrent, sur la moitié de Galaad, jusqu'au torrent de Jabbok, qui est la frontière des enfants d'Ammon\FTNT{De. 3:8-16.}~;
\VS{3}et depuis la plaine jusqu'à la mer de Kinnéreth vers l'orient, et jusqu'à la mer de la plaine, qui est la mer salée, vers l'orient, au chemin de Beth-Jeschimoth~; et depuis le midi sur le pied du Pisga.
\VS{4}Et les contrées d'Og, roi de Basan, qui était seul reste des Rephaïm, et qui habitait à Aschtaroth et à Edréï.
\VS{5}Et sa domination s'étendait sur la montagne de l'Hermon, sur Salca, et sur tout Basan, jusqu'à la frontière des Gueschuriens et des Maacathiens, et sur la moitié de Galaad, frontière de Sihon, roi de Hesbon.
\VS{6}Moïse, serviteur de Yahweh, et les enfants d'Israël, les battirent~; et Moïse, serviteur de Yahweh, en donna la possession aux Rubénites, aux Gadites, et à la demi-tribu de Manassé\FTNT{No. 32:33.}.
\VS{7}Voici les rois du pays que Josué et les enfants d'Israël frappèrent de ce côté-ci du Jourdain vers l'occident, depuis Baal-Gad, dans la vallée du Liban, jusqu'à la montagne de Halak qui monte vers Séir, et que Josué donna aux tribus d'Israël en possession, selon leurs portions,
\VS{8}pays consistant en montagnes et en vallées, en plaines et en collines, en pays de désert et de midi: Les Héthiens, les Amoréens, les Cananéens, les Phéréziens, les Héviens et les Jébusiens.
\VS{9}Le roi de Jéricho, un~; le roi d'Aï, près de Béthel, un~;
\VS{10}le roi de Jérusalem, un~; le roi d'Hébron, un~;
\VS{11}le roi de Jarmuth, un~; le roi de Lakis, un~;
\VS{12}le roi d'Eglon, un~; le roi de Guézer, un~;
\VS{13}le roi de Debir, un~; le roi de Guéder, un~;
\VS{14}le roi de Horma, un~; le roi d'Arad, un~;
\VS{15}le roi de Libna, un~; le roi d'Adullam, un~;
\VS{16}le roi de Makkéda, un~; le roi de Béthel, un~;
\VS{17}le roi de Tappuach, un~; le roi de Hépher, un~;
\VS{18}le roi d'Aphek, un~; le roi de Lascharon, un~;
\VS{19}le roi de Madon, un~; le roi de Hatsor, un~;
\VS{20}le roi de Schimron-Meron, un~; le roi d'Acschaph, un~;
\VS{21}le roi de Taanac, un~; le roi de Meguiddo, un~;
\VS{22}le roi de Kédesch, un~; le roi de Jokneam, au Carmel, un~;
\VS{23}le roi de Dor, sur les hauteurs de Dor, un~; le roi de Gojim, près de Guilgal, un~;
\VS{24}le roi de Thirtsa, un~; en tout trente et un rois.
\Chap{13}
\TextTitle{Les territoires de Ruben, de Gad et de la demi-tribu de Manassé}
\VerseOne{}Or quand Josué fut devenu vieux, fort avancé en âge, Yahweh lui dit~: Tu es devenu vieux, fort avancé en âge, et il te reste encore un très grand pays à posséder.
\VS{2}Voici le pays qui reste, toutes les contrées des Philistins, et des Gueschuriens,
\VS{3}depuis le Schichor, qui coule devant l'Egypte, jusqu'à la frontière d'Ekron au nord, contrée qui doit être tenue pour Cananéenne, et qui est occupée par les cinq princes des Philistins, celui de Gaza, celui d'Asdod, celui d'Askalon, celui de Gath, celui d'Ekron, et par les Avviens~;
\VS{4}du côté du midi, tout le pays des Cananéens, et Meara qui est aux Sidoniens, jusqu'à Aphek, jusqu'à la frontière des Amoréens~;
\VS{5}le pays qui appartient aux Guibliens, et tout le Liban, vers l'orient, depuis Baal-Gad, au pied de la montagne d'Hermon, jusqu'à l'entrée de Hamath~;
\VS{6}tous les habitants de la montagne, depuis le Liban jusqu'aux eaux de Misrephoth-Maïm, tous les Sidoniens. Je les chasserai moi-même devant les fils d'Israël. Donne seulement ce pays en héritage par le sort à Israël, comme je te l'ai prescrit.
\VS{7}Maintenant donc divise ce pays en héritage aux neuf tribus, et à la demi-tribu de Manassé.
\VS{8}Avec l'autre moitié de laquelle les Rubénites et les Gadites ont pris leur héritage, lequel Moïse leur a donné au delà du Jourdain, vers l'orient, selon que Moïse, serviteur de Yahweh, le leur a donné~;
\VS{9}depuis Aroër, qui est sur le bord du torrent de l'Arnon, et la ville qui est au milieu de la vallée, et toute la plaine de Médeba, jusqu'à Dibon~;
\VS{10}et toutes les villes de Sihon, roi des Amoréens, qui régnait à Hesbon, jusqu'à la frontière des enfants d'Ammon~;
\VS{11}et Galaad, et les territoires des Gueschuriens et des Maacathiens, toute la montagne de l'Hermon, et tout Basan jusqu'à Salca~;
\VS{12}tout le royaume d'Og en Basan, qui régnait à Aschtaroth, et à Edréï, et qui était resté le seul reste des Rephaïm~; Moïse battit ces rois, et les chassa.
\VS{13}Or les fils d'Israël ne chassèrent point les Gueschuriens et les Maacathiens, mais les Gueschuriens et les Maacathiens ont habité au milieu d'Israël jusqu'à ce jour.
\VS{14}Seulement il ne donna point d'héritage à la tribu de Lévi~; les sacrifices consumés par le feu devant Yahweh, le Dieu d'Israël, tel fut son héritage, comme il le lui avait dit\FTNT{No. 18:20-24~; De. 10:9~; De. 18:2~; Ez. 44:28.}.
\VS{15}Moïse donc donna un héritage à la tribu des fils de Ruben selon leurs familles.
\VS{16}Et leurs frontières furent depuis Aroër qui est sur le bord du torrent de l'Arnon, et de la ville qui est au milieu du torrent, et toute la plaine qui est près de Médeba.
\VS{17}Hesbon et toutes ses villes, qui étaient dans la plaine, Dibon, Bamoth-Baal, Beth-Baal-Meon,
\VS{18}Jahats, Kedémoth et Méphaath,
\VS{19}Kirjathaïm, Sibma, Tséreth-Haschachar sur la montagne de la vallée,
\VS{20}Beth-Peor, les coteaux du Pisga et Beth-Jeschimoth,
\VS{21}et toutes les villes de la plaine, et tout le royaume de Sihon, roi des Amoréens qui régnait à Hesbon~; Moïse l'avait battu, lui et les princes de Madian, Evi, Rékem, Tsur, Hur, et Réba, princes qui relevaient de Sihon, et qui habitaient dans le pays.
\VS{22}Les enfants d'Israël firent passer aussi par l'épée Balaam\FTNT{Voir No. 22. Balaam était l'exemple type du prophète corrompu, soucieux de tirer profit de son service.}, fils de Beor, le devin, avec les autres qui y furent tués.
\VS{23}Et les frontières des enfants d'Israël fut le Jourdain et sa frontière. Tel fut l'héritage des fils de Ruben, selon leurs familles~; savoir, ces villes-là et leurs villages\FTNT{No. 34:14-15.}.
\VS{24}Moïse donna aussi un héritage à la tribu de Gad, pour les fils de Gad, selon leurs familles.
\VS{25}Et leur pays fut Jaezer, et toutes les villes de Galaad et la moitié du pays des enfants d'Ammon, jusqu'à Aroër, qui est vis-à-vis de Rabba,
\VS{26}et depuis Hesbon jusqu'à Ramath-Mitspé, et Bethonim, et depuis Mahanaïm jusqu'à la frontière de Debir,
\VS{27}et, dans la vallée, Beth-Haram, Beth-Nimra, Succoth et Tsaphon, reste du royaume de Sihon, roi de Hesbon, ayant le Jourdain pour frontière jusqu'à l'extrémité de la mer de Kinnéreth, de l'autre côté du Jourdain, vers l'orient.
\VS{28}Tel fut l'héritage des fils de Gad, selon leurs familles~; savoir, les villes et leurs villages.
\VS{29}Moïse donna aussi à la demi-tribu de Manassé un héritage, qui est resté à la demi-tribu des fils de Manassé, selon leurs familles.
\VS{30}Leur pays fut depuis Mahanaïm, tout Basan, et tout le royaume d'Og, roi de Basan, et tous les villages de Jaïr qui sont en Basan, soixante villes.
\VS{31}Et la moitié de Galaad, Aschtaroth et Edréï, villes du royaume d'Og en Basan, furent aux fils de Makir, fils de Manassé, à la moitié des enfants de Makir, selon leurs familles.
\VS{32}Ce sont là les pays que Moïse avait donnés en héritage, lorsqu'il était dans les plaines de Moab, de l'autre côté du Jourdain, vis-à-vis de Jéricho, à l'orient.
\VS{33}Mais Moïse ne donna point d'héritage à la tribu de Lévi~; car Yahweh, le Dieu d'Israël, fut leur héritage, comme il le lui avait dit.
\Chap{14}
\TextTitle{Caleb reçoit Hébron}
\VerseOne{}Voici les terres que les enfants d'Israël eurent pour héritage dans le pays de Canaan, ce que partagèrent entre eux le prêtre Eléazar, Josué, fils de Nun, et les chefs des familles des tribus des enfants d'Israël.
\VS{2}Selon le sort de leur héritage~; comme Yahweh l'avait ordonné par le moyen de Moïse. A savoir, à neuf tribus et à la demi-tribu\FTNT{No. 26:55.}.
\VS{3}Car Moïse avait donné un héritage aux deux tribus et à la demi-tribu de l'autre côté du Jourdain, mais il n'avait point donné de part aux Lévites parmi eux.
\VS{4}Parce que les fils de Joseph, savoir, Manassé et Ephraïm, formaient deux tribus~; et l'on ne donna point de part aux Lévites dans le pays, excepté des villes pour habitation, et les faubourgs pour leurs troupeaux, et pour le reste de leurs biens.
\VS{5}Les enfants d'Israël firent comme Yahweh l'avait ordonné à Moïse, et ils partagèrent le pays.
\VS{6}Or les fils de Juda s'approchèrent de Josué à Guilgal~; et Caleb, fils de Jephunné, le Kenizien, lui dit~: Tu sais la parole que Yahweh a déclarée à Moïse, homme de Dieu, à mon sujet et au tien à Kadès-Barnéa\FTNT{No. 14:24~; No. 32:12~; De. 1:36.}.
\VS{7}J'étais âgé de quarante ans quand Moïse, serviteur de Yahweh, m'envoya à Kadès-Barnéa pour espionner le pays, et je lui fis un rapport avec droiture de cœur.
\VS{8}Et mes frères qui étaient montés avec moi firent fondre le cœur du peuple, mais moi je persévérai à suivre Yahweh, mon Dieu.
\VS{9}Et ce jour-là Moïse jura, en disant~: La terre que ton pied a foulée sera ton héritage à perpétuité, pour toi et pour tes fils, parce que tu as persévéré à suivre Yahweh, mon Dieu.
\VS{10}Or maintenant voici, Yahweh m'a fait vivre comme il l'a dit. Il y a déjà quarante-cinq ans que Yahweh déclarait cette parole à Moïse, lorsqu'Israël marchait dans le désert. Et maintenant voici, je suis aujourd'hui âgé de quatre-vingt-cinq ans.
\VS{11}Et je suis encore aujourd'hui aussi vigoureux que j'étais le jour où Moïse m'envoya~; et j'ai maintenant la même force que j'avais alors pour le combat, soit pour sortir et pour entrer.
\VS{12}Maintenant, donne-moi donc cette montagne, dont Yahweh a parlé ce jour-là~; car tu as appris en ce jour qu'il s'y trouve des Anakim, et qu'il y a de grandes villes fortifiées. Yahweh sera peut-être avec moi, et je les chasserai, comme Yahweh a dit.
\VS{13}Josué donc bénit Caleb, fils de Jephunné, et lui donna Hébron pour héritage.
\VS{14}C'est ainsi que Caleb, fils de Jephunné, le Kenizien, a eu jusqu'à ce jour Hébron pour héritage, parce qu'il avait persévéré à suivre Yahweh, le Dieu d'Israël.
\VS{15}Or Hébron s'appelait autrefois Kirjath-Arba~; et Arba avait été le plus grand homme parmi les Anakim. Le pays fut en repos et sans guerre.
\Chap{15}
\TextTitle{Le territoire de Juda}
\VerseOne{} Ce sont ici la part échue par le sort à la tribu des enfants de Juda, selon leurs familles~; à la frontière d'Edom, au désert de Tsin, vers le midi, fut la dernière extrémité de leurs pays vers le midi~;
\VS{2}tellement que leur frontière, du côté du midi, fut la dernière extrémité de la mer Salée, depuis le bras qui regarde vers le midi. 
\VS{3}Et elle devait sortir vers le midi de la montée d'Akrabbim, et passer vers Tsin~; et montant du midi de Kadès-Barnéa, passer à Hetsron~; puis montant vers Addar, se tourner vers Karkaa~; 
\VS{4}puis, passant vers Atsmon, sortir au torrent d'Egypte~; tellement que les extrémités de cette frontière devaient se rendre à la mer. Ce sera là, dit Josué, votre frontière, du côté du midi.
\VS{5}Et la frontière vers l'orient était la mer salée jusqu'à l'embouchure du Jourdain. La frontière du côté du nord sera depuis la langue de mer, qui est à l'embouchure du Jourdain.
\VS{6}Et cette frontière montera jusqu'à Beth-Hogla, et passera du côté du nord de Beth-Araba~; et cette frontière montera jusqu'à la pierre de Bohan, fils de Ruben.
\VS{7}Puis cette frontière montera vers Debir, depuis la vallée d'Acor, et même vers le nord, du côté de Guilgal, qui est vis-à-vis de la montée d'Adummim, au sud du torrent. Puis cette frontière passera près des eaux d'En-Schémesch, et ses extrémités se prolongeront à En-Roguel.
\VS{8}Puis cette frontière montera de là par la vallée de Ben-Hinnom, au côté du midi de Jebus, qui est Jérusalem, puis cette frontière montera jusqu'au sommet de la montagne, qui est vis-à-vis de la vallée de Hinnom, à l'occident, et à l'extrémité de la vallée des Rephaïm, au nord.
\VS{9}Et cette frontière s'alignera, depuis le sommet de la montagne jusqu'à la source des eaux de Nephthoach, et continuera vers les villes de la montagne d'Ephron, puis cette frontière s'alignera à Baala, qui est Kirjath-Jearim.
\VS{10}Et cette frontière se tournera depuis Baala, vers l'occident, jusqu'à la montagne de Séir, puis elle traversera le côté nord de la montagne de Jearim, à Kesalon, puis descendait à Beth-Schémesch, et passera par Thimna.
\VS{11}Et cette frontière sortira jusqu'au côté d'Ekron, vers le nord et cette frontière s'alignera vers Schicron, puis ayant passé la montagne de Baala, elle se sortira jusqu'à Jabneel~; tellement que les extrémités de cette frontière se rendront à la mer. 
\VS{12}Or la frontière du côté de l'occident sera ce qui est vers la grande mer et ses limites. Telles furent de tous les côtés les frontières des fils de Juda, selon leurs familles.
\VS{13}Au reste, on donna à Caleb, fils de Jephunné, une part au milieu des fils de Juda, comme Yahweh l'avait ordonné à Josué~;savoir, Kirjath-Arba, or Arba était père d'Anak~; et Kirjath-Arba c'est Hébron.
\VS{14}Et Caleb chassa de là les trois fils d'Anak~: Schéschaï, Ahiman, et Talmaï, fils d'Anak.
\VS{15}Et de là il monta contre les habitants de Debir~; Debir s'appelait autrefois Kirjath-Sépher.
\VS{16}Et Caleb dit~: Je donnerai ma fille Acsa pour femme à celui qui battra Kirjath-Sépher, et la prendra\FTNT{Jg. 1:12-14.}.
\VS{17}Et Othniel, fils de Kenaz, frère de Caleb, la prit~; et Caleb lui donna sa fille Acsa pour femme.
\VS{18}Et il arriva que comme elle s'en allait, elle l'incita à demander à son père un champ~; puis elle descendit impétueusement de dessus son âne, et Caleb lui dit~: Qu'as-tu~? 
\VS{19}Elle répondit~: Donne-moi un présent, puisque tu m'as donné une terre du sud, donne-moi aussi des sources d'eau. Et il lui donna les sources supérieures et les sources inférieures.
\VS{20}Tel fut l'héritage de la tribu des fils de Juda, selon leurs familles.
\VS{21}Les villes situées dans la contrée du midi, à l'extrémité de la tribu des fils de Juda, près de la frontière d'Edom, étaient~: Kabtseel, Eder, Jagur,
\VS{22}Kina, Dimona, Adada,
\VS{23}Kédesch, Hatsor, Ithnan,
\VS{24}Ziph, Thélem, Bealoth,
\VS{25}Hatsor-Hadattha, Kerijoth-Hetsron qui est Hatsor,
\VS{26}Amam, Schema, Molada,
\VS{27}Hatsar-Gadda, Heschmon, Beth-Paleth,
\VS{28}Hatsar-Schual, Beer-Schéba, Bizjothja,
\VS{29}Baala, Ijjim, Atsem,
\VS{30}Eltholad, Kesil, Horma,
\VS{31}Tsiklag, Madmanna, Sansanna,
\VS{32}Lebaoth, Schilhim, Aïn et Rimmon. Total des villes~: Vingt-neuf villes, et leurs villages.
\VS{33}Dans la plaine~: Eschthaol, Tsorea, Aschna,
\VS{34}Zanoach, En-Gannim, Tappuach, Enam,
\VS{35}Jarmuth, Adullam, Soco, Azéka,
\VS{36}Schaaraïm, Adithaïm, Guedéra et Guedérothaïm~; quatorze villes, et leurs villages.
\VS{37}Tsenan, Hadascha, Migdal-Gad,
\VS{38}Dilean, Mitspé, Joktheel,
\VS{39}Lakis, Botskath, Eglon,
\VS{40}Cabbon, Lachmas, Kithlisch,
\VS{41}Guedéroth, Beth-Dagon, Naama, et Makkéda~; seize villes, et leurs villages.
\VS{42}Libna, Ether, Aschan,
\VS{43}Jiphtach, Aschna, Netsib,
\VS{44}Keïla, Aczib et Maréscha~; neuf villes, et leurs villages.
\VS{45}Ekron, et les villes de son ressort, et ses villages.
\VS{46}Depuis Ekron et à l'occident, toutes les villes près d'Asdod, et leurs villages.
\VS{47}Asdod, les villes de son ressort, et ses villages, Gaza, les villes de son ressort, et ses villages, jusqu'au torrent d'Egypte, et à la grande mer, qui sert de limite.
\VS{48}Dans la montagne~: Schamir, Jatthir, Soco,
\VS{49}Danna, Kirjath-Sanna, qui est Debir,
\VS{50}Anab, Eschthemo, Anim,
\VS{51}Gosen, Holon, et Guilo~; onze villes et leurs villages.
\VS{52}Arab, Duma, Eschean,
\VS{53}Janum, Beth-Tappuach, Aphéka,
\VS{54}Humta, Kirjath-Arba, qui est Hébron, et Tsior~; neuf villes, et leurs villages.
\VS{55}Maon, Carmel, Ziph, Juta,
\VS{56}Jizreel, Jokdeam, Zanoach,
\VS{57}Kaïn, Guibea, et Thimna~; dix villes, et leurs villages.
\VS{58}Halhul, Beth-Tsur, Guedor,
\VS{59}Maarath, Beth-Anoth, et Elthekon~; six villes, et leurs villages.
\VS{60}Kirjath-Baal, qui est Kirjath-Jearim, et Rabba~; deux villes, et leurs villages.
\VS{61}Au désert~: Beth-Araba, Middin, Secaca,
\VS{62}Nibschan, Ir-Hammélach, et En-Guédi~: Six villes et leurs villages.
\VS{63}Au reste, les fils de Juda ne purent pas chasser les Jébusiens qui habitaient à Jérusalem, c'est pourquoi les Jébusiens ont habité avec les fils de Juda à Jérusalem jusqu'à ce jour.
\Chap{16}
\TextTitle{Le territoire d'Ephraïm}
\VerseOne{}La part échue par le sort aux fils de Joseph depuis le Jourdain près de Jéricho, aux eaux de Jéricho, vers l'orient qui est le désert~; montant de Jéricho par la montagne jusqu'à Béthel.
\VS{2}Et cette frontière devait sortir de Béthel à Luz, puis passer vers la frontière des Arkiens jusqu'à Atharoth.
\VS{3}Et elle devait descendre tirant vers l'occident, vers la frontière des Japhléthiens, jusqu'à celle de Beth-Horon la basse et jusqu'à Guézer, de sorte que ses extrémités aboutissent à la mer.
\VS{4}Ainsi les fils de Joseph, à savoir, Manassé et Ephraïm, reçurent leur héritage.
\VS{5}Or la frontière des fils d'Ephraïm, selon leurs familles, la frontière de leur héritage était à l'orient, Atharoth-Addar, jusqu'à Beth-Horon la haute.
\VS{6}Et cette frontière devait sortir vers la mer à Micmethath, du côté du nord~; et cette frontière devait se tourner vers l'orient jusqu'à Thaanath-Silo, et passant du côté d'orient, se rendre à Janoach.
\VS{7}Puis descendre de Janoach à Atharoth et à Naaratha, se rencontrer à Jéricho, et sortir au Jourdain.
\VS{8}Et cette frontière devait aller de Tappuach, vers l'occident, jusqu'au torrent de Kana, tellement que ses extrémités devaient se rendre à la mer. Ce fut là l'héritage de la tribu des fils d'Ephraïm, selon leurs familles.
\VS{9}Les fils d'Ephraïm avaient aussi des villes séparées au milieu de l'héritage des fils de Manassé, toutes ces villes, avec leurs villages.
\VS{10}Or ils ne chassèrent point les Cananéens qui habitaient à Guézer, c'est pourquoi les Cananéens ont habité parmi Ephraïm jusqu'à ce jour, mais ils furent réduits à la servitude et assujettis à un tribut\FTNT{Jg. 1:29~; 1 R. 9:16.}.
\Chap{17}
\TextTitle{Le territoire de Manassé}
\VerseOne{}Il y eut aussi une part échut par le sort à la tribu de Manassé qui était le premier-né de Joseph. Quant à Makir, premier-né de Manassé, et père de Galaad, il avait eu Galaad et Basan parce qu'il était un homme de guerre.
\VS{2}Puis on jeta donc le sort pour les autres enfants de Manassé, selon ses familles~; aux fils d'Abiézer, aux fils de Hélek, aux fils d'Asriel, aux fils de Sichem, aux fils de Hépher, et aux fils de Schemida. Ce sont là les enfants mâles de Manassé fils de Joseph, selon leurs familles.
\VS{3}Or Tselophchad, fils de Hépher, fils de Galaad, fils de Makir, fils de Manassé, n'eut point de fils, mais il eut des filles dont voici les noms~: Machla, Noa, Hogla, Milca et Thirtsa.
\VS{4}Elles vinrent se présenter devant le prêtre Eléazar, devant Josué, fils de Nun, et devant les princes, en disant~: Yahweh a ordonné à Moïse de nous donner un héritage parmi nos frères. C'est pourquoi on leur donna un héritage parmi les frères de leur père, selon l'ordre de Yahweh\FTNT{No. 27:7~; No. 36:2.}.
\VS{5}Et dix portions échurent à Manassé, outre le pays de Galaad et de Basan, qui est de l'autre côté du Jourdain.
\VS{6}Car les filles de Manassé eurent un héritage parmi ses fils, et le pays de Galaad fut pour les autres fils de Manassé.
\VS{7}Or la frontière de Manassé fut du côté d'Aser, venant à Micmethath, qui est près de Sichem~; puis cette frontière devait aller à main droite vers les habitants d'En-Tappuach.
\VS{8}Or le pays de Tappuach appartenait à Manassé, mais Tappuach qui était près de la frontière de Manassé, appartenait aux fils d'Ephraïm.
\VS{9}De là, cette frontière devait descendre au torrent de Kana, au midi du torrent. Ces villes étaient à Ephraïm parmi les villes de Manassé. La frontière de Manassé était au côté du nord du torrent, et ses extrémités devaient se rendre à la mer.
\VS{10}Ce qui était vers le midi était à Ephraïm, et celui qui était vers le nord était à Manassé, et la mer leur servait de frontière~; et du côté du nord, les frontières se rencontraient à Aser, à Issacar, vers l'orient.
\VS{11}Car Manassé possédait dans Issacar et dans Aser~: Beth-Schean et les villes de son ressort, Jibleam et les villes de son ressort, les habitants de Dor et les villes de son ressort, les habitants d'En-Dor, et les villes de son ressort, les habitants de Thaanac et les villes de son ressort, les habitants de Meguiddo et les villes de son ressort, qui sont trois contrées.
\VS{12}Au reste, les fils de Manassé ne purent pas chasser les habitants de ces villes, et les Cananéens voulurent rester dans le même pays.
\VS{13}Mais lorsque les fils d'Israël furent assez forts, ils assujettirent les Cananéens à un tribut, mais ils ne les chassèrent pas entièrement.
\VS{14}Or les fils de Joseph parlèrent à Josué, et dirent~: Pourquoi nous as-tu donné en héritage un seul lot, et une seule part, vu que nous sommes un peuple nombreux, et que Yahweh nous a bénis jusqu'à présent~?
\VS{15}Et Josué leur dit~: Si vous êtes un peuple nombreux, montez à la forêt, et vous l'abattrez, pour vous y faire de la place dans le pays des Phéréziens et des Rephaïm, si la montagne d'Ephraïm est trop étroite pour vous.
\VS{16}Et les fils de Joseph répondirent~: Cette montagne ne sera pas suffisante pour nous, et tous les Cananéens qui habitent la vallée ont des chars de fer, et ceux qui sont à Beth-Schean, et dans les villes de son ressort, et ceux qui habitent dans la vallée de Jizreel\FTNT{Jg. 1:19~; Jg. 4:3.}.
\VS{17}Donc Josué parla à la maison de Joseph, à Ephraïm et à Manassé, et dit~: Vous êtes un peuple nombreux, et vous avez de grandes forces, vous n'aurez pas qu'une seule part.
\VS{18}Mais vous aurez la montagne, car c'est une forêt que vous abattrez et dont les extrémités vous appartiendront, et vous chasserez les Cananéens, quoiqu'ils aient des chars de fer, et qu'ils soient puissants.
\Chap{18}
\TextTitle{La tente d'assignation à Silo}
\VerseOne{}Or toute l'assemblée des enfants d'Israël s'assembla à Silo\FTNT{Silo fut pendant la période des Juges le centre religieux d'Israël car c'est dans cette ville que l'on avait déposé l'arche jusqu'à ce que le roi David l'amène à Jérusalem (Jos. 18:1~; 2 S. 6~; 1 Ch. 15:3). Durant le schisme, Silo, située en Samarie, fit office de capitale du royaume de sud. La ville fut finalement détruite par les Philistins aux alentours de 1050 av. J.-C.}, et ils y posèrent la tente d'assignation, après que le pays leur ait été assujetti. 
\VS{2}Mais il restait sept tribus des enfants d'Israël qui n'avaient pas encore reçu leur héritage.
\VS{3}Josué dit aux enfants d'Israël~: Jusqu'à quand négligerez-vous de prendre possession du pays que Yahweh, le Dieu de vos pères, vous a donné~?
\VS{4}Prenez trois hommes de chaque tribu, que j'enverrai. Ils se lèveront, traverseront le pays, traceront un plan en vue de l'héritage, puis ils reviendront auprès de moi.
\VS{5}Ils le diviseront en sept parts~; Juda restera dans ses limites au midi, et la maison de Joseph restera dans ses limites au nord.
\VS{6}Vous donc faites-vous un plan du pays en sept parts, et apportez-le-moi ici. Puis je jetterai pour vous le sort devant Yahweh, notre Dieu.
\VS{7}Et il n'y aura point de part pour les Lévites au milieu de vous, parce que le sacerdoce de Yahweh est leur héritage. Quant à Gad et à Ruben, et à la demi-tribu de Manassé, ils ont reçu leur héritage de l'autre côté du Jourdain, vers l'orient, que Moïse, serviteur de Yahweh, leur a donné.
\VS{8}Ces hommes-là donc se levèrent et s'en allèrent pour tracer un plan du pays, Josué leur donna cet ordre, en disant~: Allez et traversez le pays, et tracez-en un plan, puis revenez auprès de moi, et je jetterai ici le sort pour vous devant Yahweh, à Silo.
\VS{9}Ces hommes-là donc s'en allèrent, parcoururent le pays, et en tracèrent un plan dans un livre en sept parts selon les villes~; puis ils revinrent auprès de Josué dans le camp à Silo.
\VS{10}Et Josué jeta le sort pour eux à Silo devant Yahweh, et Josué fit le partage du pays entre les enfants d'Israël, selon leurs parts.
\TextTitle{Le territoire de Benjamin}
\VS{11}Et le sort tomba sur la tribu des fils de Benjamin selon leurs familles, et la part qui leur échut par le sort avait ses frontières entre les fils de Juda et les fils de Joseph.
\VS{12}Et leur frontière du côté du nord fut depuis le Jourdain~; et cette frontière devait monter à côté de Jéricho vers le nord, puis monter en la montagne tirant vers l'Occident~; de sorte que ses extrémités devaient se rendre au désert de Beth-aven. 
\VS{13}Puis cette frontière devait passer de là vers Luz, à côté de Luz, qui est Béthel tirant vers le midi~; et cette frontière devait descendre à Hatroth-Addar, près de la montagne qui est du côté du midi de Beth-Horon la basse. 
\VS{14}Et cette frontière devait s'aligner et tourner du côté occidental qui regarde vers le midi, depuis la montagne qui est vis-à-vis de Beth-Horon, vers le midi~; tellement que ses extrémités devaient se rendre à Kirjath Baal, qui est Kirjath Jearim, ville des enfants de Juda. C'est là le côté d'occident. 
\VS{15}Mais le côté méridional est l'extrémité de Kirjath Jearim~; et cette frontière devait sortir vers l'Occident, puis elle devait sortir à la fontaine des eaux de Nephtoah. 
\VS{16}Et cette frontière devait descendre à l'extrémité de la montagne qui est vis-à-vis de la vallée de Ben-Hinnom, dans la vallée des Rephaïm, vers le nord, et descendre par la vallée de Hinnom, sur le côté méridional des Jébusiens, puis descendre jusqu'à En-Roguel.
\VS{17}Et elle devait s'aligner vers le nord, et sortir à En-Schémesch, de là à Gueliloth, qui est vis-à-vis de la montée d'Adummim, et descendre à la pierre de Bohan, fils de Ruben,
\VS{18}et passer sur le côté nord en face d'Araba, et descendre à Araba,
\VS{19}puis cette frontière devait passer à côté de Beth-Hogla vers le nord~; de sorte que les extrémités de cette frontière aboutissent à la langue de la mer salée vers le nord, à l'embouchure du Jourdain vers le midi. C'était la frontière du midi.
\VS{20}Et le Jourdain devait borner du côté de l'orient. Ce fut là l'héritage des fils de Benjamin avec ses frontières tout autour, selon leurs familles.
\VS{21}Les villes de la tribu des fils de Benjamin, selon leurs familles, étaient~: Jéricho, Beth-Hogla, Emek-Ketsits,
\VS{22}Beth-Araba, Tsemaraïm, Béthel,
\VS{23}Avvim, Para, Ophra,
\VS{24}Kephar-Ammonaï, Ophni et Guéba~; douze villes et leurs villages.
\VS{25}Gabaon, Rama, Beéroth,
\VS{26}Mitspé, Kephira, Motsa,
\VS{27}Rékem, Jirpeel, Thareala,
\VS{28}Tséla, Eleph, Jebus, qui est Jérusalem, Guibeath et Kirjath~; quatorze villes et leurs villages. Tel fut l'héritage des fils de Benjamin selon leurs familles.
\Chap{19}
\TextTitle{Le territoire de Siméon}
\VerseOne{}La deuxième part échut par le sort à Siméon, pour la tribu des fils de Siméon, selon leurs familles. Leur héritage était parmi l'héritage des fils de Juda\FTNT{Ge. 49:5-7.}.
\VS{2}Ils eurent dans leur héritage Beer-Schéba, Schéba, Molada,
\VS{3}Hatsar-Schual, Bala, Atsem,
\VS{4}Eltholad, Bethul, Horma,
\VS{5}Tsiklag, Beth-Marcaboth, Hatsar-Susa,
\VS{6}Beth-Lebaoth et Scharuchen~; treize villes et leurs villages.
\VS{7}Aïn, Rimmon, Ether, et Aschan~; quatre villes et leurs villages~;
\VS{8}et tous les villages qui étaient autour de ces villes-là jusqu'à Baalath-Beer, qui est Ramath du midi. Tel fut l'héritage de la tribu des fils de Siméon, selon leurs familles.
\VS{9}L'héritage des fils de Siméon fut pris sur la portion des fils de Juda~; car la portion des fils de Juda était trop grande pour eux~; c'est pourquoi les fils de Siméon reçurent leur héritage parmi le leur.
\TextTitle{Le territoire de Zabulon}
\VS{10}La troisième part échut par le sort aux fils de Zabulon, selon leurs familles.
\VS{11}Et leur frontière devait monter vers le quartier devers la mer, même jusqu'à Mareala, puis se rencontrer à Dabbéscheth, et de là au torrent qui est vis-à-vis de Jokneam.
\VS{12}Or cette frontière devait retourner vers Sarid à l'orient, vers le soleil levant, jusqu'à la frontière de Kisloth-Thabor, puis continuer à Dabrath, et monter à Japhia,
\VS{13}de là passer à l'orient, par Guittha-Hépher, par Ittha-Katsin, puis continuer à Rimmon, jusqu'à Néa.
\VS{14}Puis cette frontière devait tourner du côté du nord vers Hannathon, et ses extrémités devaient se rendre à la vallée de Jiphthach-El.
\VS{15}Avec Katthath, Nahalal, Schimron, Jideala, et Bethléhem~; il y avait douze villes et leurs villages.
\VS{16}Tel fut l'héritage des fils de Zabulon selon leurs familles, ces villes-là, et leurs villages.
\TextTitle{Le territoire d'Issacar}
\VS{17}La quatrième part échut par le sort à Issacar, aux fils d'Issacar, selon leurs familles.
\VS{18}Et leur frontière devaient passer par Jizreel, Kesulloth, Sunem,
\VS{19}Hapharaïm, Schion, Anacharath,
\VS{20}Rabbith, Kischjon, Abets,
\VS{21}Rémeth, En-Gannim, En-Hadda et Beth-Patsets~;
\VS{22}elle devait se rencontrer à Thabor, et vers Schachatsima et Beth-Schémesch, et les extrémités de leur frontière devaient se rendre au Jourdain. Seize villes et leurs villages.
\VS{23}Tel fut l'héritage de la tribu des fils d'Issacar, selon leurs familles, ces villes-là et leurs villages.
\TextTitle{Le territoire d'Aser}
\VS{24}La cinquième part échut par le sort à la tribu des fils d'Aser, selon leurs familles.
\VS{25}Et leur frontière fut Helkath, Hali, Béthen, Acschaph,
\VS{26}Allammélec, Amead et Mischeal~; et elle devait se rencontrer à Carmel, au quartier vers la mer, et à Schichor-Libnath.
\VS{27}Puis elle devait retourner vers l'orient, à Beth-Dagon, et se rencontrer à Zabulon, et à la vallée de Jiphthach-El, vers le nord de Beth-Emek et de Neïel, puis sortir vers Cabul, à gauche,
\VS{28}et vers Ebron, Rehob, Hammon et Kana, jusqu'à Sidon la grande.
\VS{29}Puis la frontière devait retourner à Rama, jusqu'à la ville forte de Tyr, et cette frontière devait retourner à Hosa~; de sorte que ses extrémités se rencontrent au quartier qui est vers la mer, par la contrée d'Aczib.
\VS{30}Avec Umma, Aphek et Rehob~; vingt-deux villes et leurs villages.
\VS{31}Tel fut l'héritage de la tribu des fils d'Aser, selon leurs familles~; ces villes-là et leurs villages.
\TextTitle{Le territoire de Nephthali}
\VS{32}La sixième part échut par le sort aux fils de Nephthali, selon leurs familles.
\VS{33}Leur frontière fut depuis Héleph, depuis Allon par Tsaanannim, Adami-Nékeb et Jabneel, jusqu'à Lakkum, et ses extrémités devaient se rendre au Jourdain.
\VS{34}Puis cette frontière devait retourner du côté d'occident, vers Aznoth-Thabor, et sortir de là à Hukkok~; de sorte que du côté du midi elle devait se rencontrer à Zabulon, et du côté d'occident elle devait se rencontrer à Aser et à Juda~; le Jourdain était du côté au soleil levant.
\VS{35}Au reste, les villes fortifiées étaient~: Tsiddim, Tser, Hammath, Rakkath, Kinnéreth,
\VS{36}Adama, Rama, Hatsor,
\VS{37}Kédesch, Edréï, En-Hatsor,
\VS{38}Jireon, Migdal-El, Horem, Beth-Anath et Beth-Schémesch~; dix-neuf villes et leurs villages.
\VS{39}Tel fut l'héritage de la tribu des fils de Nephthali, selon leurs familles~; ces villes-là, et leurs villages.
\TextTitle{Le territoire de Dan}
\VS{40}La septième part échut par le sort à la tribu des fils de Dan selon leurs familles.
\VS{41}La limite de leur héritage fut, Tsorea, Eschthaol, Ir-Schémesch,
\VS{42}Schaalabbin, Ajalon, Jithla,
\VS{43}Elon, Thimnatha, Ekron,
\VS{44}Eltheké, Guibbethon, Baalath,
\VS{45}Jehud, Bené-Berak, Gath-Rimmon,
\VS{46}Mé-Jarkon et Rakkon, avec le territoire qui est vis-à-vis de Japho.
\VS{47}Le territoire échu aux fils de Dan était trop petit pour eux. C'est pourquoi les fils de Dan montèrent, et combattirent contre Léschem~; ils s'en emparèrent et la frappèrent du tranchant de l'épée~; ils en prirent possession, s'y établirent, et l'appelèrent Léschem, Dan, du nom de Dan leur père.
\VS{48}Tel fut l'héritage de la tribu des fils de Dan selon leurs familles~; ces villes-là et leurs villages.
\TextTitle{Josué reçoit Thimnath-Sérach}
\VS{49}Après qu'on eut achevé de partager le pays selon ses frontières, les enfants d'Israël donnèrent à Josué, fils de Nun, une possession au milieu d'eux.
\VS{50}Selon l'ordre de Yahweh, ils lui donnèrent la ville qu'il demanda, Thimnath-Sérach, dans la montagne d'Ephraïm. Il rebâtit la ville, et y habita.
\VS{51}Ce sont là les héritages que le prêtre Eléazar, Josué, fils de Nun, et les chefs de pères des tribus des enfants d'Israël partagèrent par le sort à Silo, devant Yahweh, à l'entrée de la tente d'assignation, et ils achevèrent ainsi le partage du pays.
\Chap{20}
\TextTitle{Les six villes de refuge\FTNTT{No. 35.}}
\VerseOne{}Puis Yahweh parla à Josué et dit :
\VS{2}Parle aux enfants d'Israël et dis-leur : Etablissez-vous les villes de refuge dont je vous ai parlé par le moyen de Moïse,
\VS{3}afin que le meurtrier qui aura tué quelqu'un involontairement sans y penser, s'y enfuie ; et elles seront pour vous un refuge devant celui qui a le droit de venger le sang.
\VS{4}Et le meurtrier s'enfuira dans l'une de ces villes, s'arrêtera à l'entrée de la porte de la ville, et il dira ses raisons aux anciens de cette ville-là, ils l'écouteront, et le recevront chez eux dans la ville, et lui donneront un lieu, afin qu'il demeure avec eux.
\VS{5}Et quand celui qui a le droit de venger le sang le poursuivra, ils ne livreront pas le meurtrier entre ses mains ; puisque c'est sans y penser qu'il a tué son prochain, et qu'il ne le haïssait point auparavant.
\VS{6}Mais il demeurera dans cette ville-là, jusqu'à ce qu'il comparaisse devant l'assemblée en jugement, même jusqu'à la mort du grand-prêtre qui sera en ce temps-là. Alors le meurtrier s'en retournera, et reviendra dans sa ville et dans sa maison, dans la ville d'où il s'était enfui\FTNT{Ex. 21:13~; No. 35:9-34~; De. 19.}.
\VS{7}Ils consacrèrent donc Kédesch, en Galilée, dans la montagne de Nephthali~; Sichem dans la montagne d'Ephraïm~; et Kirjath-Arba, qui est Hébron, dans la montagne de Juda.
\VS{8}Et au delà du Jourdain, à l'orient de Jéricho, ils choisirent Betser, dans la tribu de Ruben, dans le désert, dans la plaine~; Ramoth en Galaad, dans la tribu de Gad ; et Golan en Basan, dans la tribu de Manassé\FTNT{De. 4:43.}.
\VS{9}Ce furent là les villes assignées à tous les enfants d'Israël et à l'étranger demeurant parmi eux ; afin que quiconque aurait tué quelqu'un involontairement, s'enfuit là, et ne meure pas de la main de celui qui a le droit de venger le sang, jusqu'à ce qu'il comparaisse devant l'assemblée.
\Chap{21}
\TextTitle{Les quarante-huit villes des Lévites}
\VerseOne{}Or les chefs des pères de famille des Lévites s'approchèrent d'Eléazar, le prêtre, de Josué, fils de Nun, et des chefs des pères de famille des tribus des enfants d'Israël.
\VS{2}Et leur parlèrent à Silo, dans le pays de Canaan, en disant~: Yahweh a ordonné par le moyen de Moïse qu'on nous donne des villes pour habiter, et leurs faubourgs pour nos bêtes\FTNT{No. 35:2-3.}.
\VS{3}Et ainsi les enfants d'Israël donnèrent aux Lévites, sur leur héritage, les villes suivantes et leurs faubourgs, d'après l'ordre de Yahweh.
\VS{4}Et on tira au sort pour les familles des Kehathites~; et les Lévites, fils d'Aaron, le prêtre eurent, par le sort, treize villes de la tribu de Juda, de la tribu de Siméon, et de la tribu de Benjamin.
\VS{5}Et il échut par sort au reste des enfants de Kehath, dix villes des familles de la tribu d'Ephraïm, de la tribu de Dan, et de la demi-tribu de Manassé.
\VS{6}Et les enfants de Guerschon eurent par le sort treize villes, des familles de la tribu d'Issacar, de la tribu d'Aser, de la tribu de Nephthali, et de la demi-tribu de Manassé en Basan.
\VS{7}Et les enfants de Merari, selon leurs familles, eurent douze villes, de la tribu de Ruben, de la tribu de Gad, et de la tribu de Zabulon.
\VS{8}Les enfants d'Israël donnèrent donc par le sort aux Lévites ces villes-là avec leurs faubourgs, comme Yahweh l'avait ordonné par le moyen de Moïse.
\VS{9}Ils donnèrent donc de la tribu des fils de Juda et de la tribu des fils de Siméon, ces villes, qui vont être nommées par leurs noms,
\VS{10}et elles furent pour ceux des enfants d'Aaron, qui étaient des familles des Kehathites, et des fils de Lévi, car le premier sort fut pour eux. 
\VS{11}On leur donna donc Kirjath-Arba ; or Arba était le père d'Anak, et Kirjath-Arba est Hébron, dans la montagne de Juda, avec ses faubourgs tout autour~: 
\VS{12}Mais quant au territoire de la ville, et à ses villages, on les donna à Caleb, fils de Jephunné, pour sa possession.
\VS{13}On donna donc aux enfants d'Aaron, le prêtre, les villes de refuge pour les meurtriers, Hébron, avec ses faubourgs, et Libna avec ses faubourgs.
\VS{14}Et Jatthir, avec ses faubourgs, Eschthemoa, avec ses faubourgs,
\VS{15}Et Holon, avec ses faubourgs, Debir, avec ses faubourgs,
\VS{16}Et Aïn, avec ses faubourgs, Jutta, avec ses faubourgs~; et Beth-Schémesch, avec ses faubourgs~; neuf villes de ces deux tribus-là~;
\VS{17}et de la tribu de Benjamin, Gabaon, avec ses faubourgs, et Guéba, avec ses faubourgs,
\VS{18}Anathoth, avec ses faubourgs, et Almon, avec ses faubourgs~; quatre villes.
\VS{19}Toutes les villes des prêtres, fils d'Aaron, furent treize villes, avec leurs faubourgs.
\VS{20}Or quant aux familles des enfants de Kehath, Lévites, qui étaient le reste des enfants de Kehath, il y eut dans leur sort des villes de la tribu d'Ephraïm.
\VS{21}On leur donna donc les villes de refuge pour les meurtriers, Sichem, avec ses faubourgs, dans la montagne d'Ephraïm, et Guézer avec ses faubourgs~;
\VS{22}Et Kibtsaïm, avec ses faubourgs, et Beth-Horon, avec ses faubourgs~; quatre villes~;
\VS{23}et de la tribu de Dan, Eltheké, avec ses faubourgs~; Guibbethon, avec ses faubourgs,
\VS{24}Ajalon, avec ses faubourgs, Gath-Rimmon, avec ses faubourgs~; quatre villes.
\VS{25}Et de la demi-tribu de Manassé, Thaanac, avec ses faubourgs~; et Gath-Rimmon, avec ses faubourgs, deux villes.
\VS{26}Total des villes~: Dix villes avec leurs faubourgs, pour les familles des autres fils de Kehath.
\VS{27}On donna aussi aux fils de Guerschon, d'entre les familles des Lévites~: De la demi-tribu de Manassé les villes de refuge pour les meurtriers, Golan en Basan, avec ses faubourgs, et Beeschthra, avec ses faubourgs~; deux villes~;
\VS{28}et de la tribu d'Issacar, Kischjon, avec ses faubourgs, Dabrath, avec ses faubourgs,
\VS{29}Jarmuth, avec ses faubourgs, En-Gannim, avec ses faubourgs~; quatre villes~;
\VS{30}et de la tribu d'Aser, Mischeal, avec ses faubourgs, Abdon, avec ses faubourgs,
\VS{31}Helkath, avec ses faubourgs, et Rehob, avec ses faubourgs~; quatre villes~;
\VS{32}et de la tribu de Nephthali, les villes de refuge pour les meurtriers, Kédesch en Galilée avec ses faubourgs, Hammoth-Dor, avec ses faubourgs, et Karthan, avec ses faubourgs~; trois villes.
\VS{33}Total des villes des Guerschonites, selon leurs familles~: Treize villes, et leurs faubourgs.
\VS{34}On donna aussi au reste des Lévites, qui appartenaient aux familles des enfants de Merari~: De la tribu de Zabulon, Jokneam, avec ses faubourgs, Kartha, avec ses faubourgs,
\VS{35}Dimna, avec ses faubourgs, et Nahalal, avec ses faubourgs~; quatre villes~;
\VS{36}et de la tribu de Ruben, Betser, avec ses faubourgs, et Jahtsa, avec ses faubourgs~;
\VS{37}Kedémoth, avec ses faubourgs, et Méphaath, avec ses faubourgs~; quatre villes~;
\VS{38}et de la tribu de Gad, les villes de refuge pour les meurtriers, Ramoth en Galaad, avec ses faubourgs, et Mahanaïm, avec ses faubourgs,
\VS{39}Hesbon, avec ses faubourgs, et Jaezer, avec ses faubourgs~; en tout quatre villes.
\VS{40}Total des villes qui échurent par le sort aux enfants de Merari, selon leurs familles, formant le reste des familles des Lévites~: Douze villes.
\VS{41}Total des villes des Lévites qui étaient parmi la possession des enfants d'Israël~: Quarante-huit villes, et leurs faubourgs.
\VS{42}Chacune de ces villes avait ses faubourgs autour d'elle~; il en était ainsi de toutes ces villes-là.
\TextTitle{Yahweh accomplit sa promesse}
\VS{43}Yahweh donna donc à Israël tout le pays qu'il avait juré de donner à leurs pères~; ils le possédèrent, et y habitèrent\FTNT{Dieu accomplit toujours ses promesses (Jé. 1:12).}.
\VS{44}Yahweh leur accorda un parfait repos tout autour, selon tout ce qu'il avait juré à leurs pères~; aucun de leurs ennemis ne put leur résister, car Yahweh les livra entre leurs mains.
\VS{45}Il ne tomba pas un seul mot de toutes les bonnes paroles que Yahweh avait dites à la maison d'Israël~: Tout arriva.
\Chap{22}
\TextTitle{Ruben, Gad et la demi-tribu de Manassé retournent sur leurs terres}
\VerseOne{}Alors Josué appela les Rubénites, les Gadites et la demi-tribu de Manassé.
\VS{2}Et il leur dit~: Vous avez gardé tout ce que Moïse, serviteur de Yahweh, vous a prescrit, et vous avez obéi à ma voix dans tout ce que je vous ai ordonné.
\VS{3}Vous n'avez pas abandonné vos frères, depuis une très longue période jusqu'à ce jour~; et vous avez gardé les ordres, les commandements de Yahweh, votre Dieu.
\VS{4}Or maintenant, Yahweh, votre Dieu a donné du repos à vos frères, selon qu'il leur en avait parlé. Maintenant donc, retournez et allez-vous-en dans vos demeures, dans la terre de votre possession, que Moïse, serviteur de Yahweh, vous a donnée de l'autre côté du Jourdain\FTNT{No. 32:33~; De. 3:13~; De. 29:8.}.
\VS{5}Prenez seulement bien garde d'observer les ordonnances et les lois que Moïse, serviteur de Yahweh, vous a prescrites, qui sont~: Que vous aimiez Yahweh, votre Dieu, et que vous marchiez dans toutes ses voies, et que vous gardiez ses commandements, et que vous vous attachiez à lui, et le serviez de tout votre cœur et de toute votre âme\FTNT{De. 10:12.}.
\VS{6}Puis Josué les bénit et les renvoya~; et ils s'en allèrent dans leurs demeures.
\VS{7}Or Moïse avait donné à la moitié de la tribu de Manassé son héritage en Basan~; et Josué donna à l'autre moitié son héritage avec leurs frères de l'autre côté du Jourdain vers l'occident. Josué les renvoya dans leurs demeures, et les bénit.
\VS{8}Et il leur parla, en disant~: Vous retournez à vos demeures avec de grandes richesses, une très nombreuse quantité de bétail, avec une quantité considérable d'argent, d'or, d'airain, de fer, et de vêtements. Partagez avec vos frères le butin de vos ennemis.
\VS{9}Ainsi donc les enfants de Ruben, les enfants de Gad, et la demi-tribu de Manassé s'en retournèrent, et partirent de Silo, dans le pays de Canaan, après avoir quitté les enfants d'Israël, pour s'en aller dans le pays de Galaad, sur la terre de leur possession, de laquelle on les avait fait jouir, suivant ce que Yahweh avait ordonné par le moyen de Moïse.
\TextTitle{L'autel Ed, sujet d'incompréhension}
\VS{10}Or ils vinrent aux frontières du Jourdain, qui appartiennent au pays de Canaan, et les enfants de Ruben, les enfants de Gad, et la demi-tribu de Manassé y bâtirent un autel, joignant le Jourdain, qui était un autel de grande apparence.
\VS{11}Et les enfants d'Israël entendirent dire~: Voici, les enfants de Ruben, les enfants de Gad, et la demi-tribu de Manassé ont bâti un autel en face du pays de Canaan, sur les frontières du Jourdain, du côté des enfants d'Israël.
\VS{12}Les enfants d'Israël entendirent donc cela, et toute l'assemblée des enfants d'Israël s'assembla à Silo, pour monter en bataille contre eux.
\VS{13}Cependant les enfants d'Israël envoyèrent vers les enfants de Ruben, vers les enfants de Gad, et vers la demi-tribu de Manassé, au pays de Galaad, Phinées, fils du prêtre Eléazar,
\VS{14}et avec lui dix princes, à savoir, un prince de chaque maison des pères de toutes les tribus d'Israël~; car il y avait dans tous les milliers d'Israël un chef de chaque maison de leurs pères. 
\VS{15}Ceux-ci vinrent donc vers les enfants de Ruben, des enfants de Gad et de la demi-tribu de Manassé au pays de Galaad, et leur parlèrent, en disant~:
\VS{16}Ainsi parle toute l'assemblée de Yahweh~: Quelle est cette infidélité que vous avez commise contre le Dieu d'Israël, et pourquoi vous détournez-vous aujourd'hui de Yahweh, en vous bâtissant un autel, pour vous rebeller aujourd'hui contre Yahweh~?
\VS{17}Regardons-nous comme peu de chose l'iniquité de Peor\FTNT{Peor~: No. 25:1-9.}, dont nous ne nous sommes pas encore bien nettoyés jusqu'à présent, malgré la plaie qu'il attira sur l'assemblée de Yahweh,
\VS{18}que vous vous détourniez aujourd'hui de Yahweh, et que vous vous rebelliez aujourd'hui contre Yahweh, afin que demain sa colère s'enflamme contre toute l'assemblée d'Israël ?
\VS{19}Toutefois, si la terre de votre possession est souillée, passez sur la terre qui est la possession de Yahweh, où est fixé le tabernacle de Yahweh, et ayez votre possession parmi nous, et ne vous révoltez point contre Yahweh, et ne soyez point rebelles contre nous, en vous bâtissant un autel, outre l'autel de Yahweh, notre Dieu.
\VS{20}Acan\FTNT{Acan~: Jos. 7:1-26.}, fils de Zérach, ne commit-il pas une infidélité en prenant des choses dévouées par le moyen de l'interdit, et la colère de Yahweh ne s'enflamma-t-elle pas contre toute l'assemblée d'Israël~? Cependant, cet homme ne fut pas le seul qui périt à cause de son iniquité.
\VS{21}Mais les enfants de Ruben, les enfants de Gad, et la demi-tribu de Manassé répondirent, et dirent aux chefs des milliers d'Israël~:
\VS{22}Dieu\FTNT{Dieu~: de l'hébreu «~El~»~: puissant, etc.}, Dieu\FTNT{Dieu~: de l'hébreu «~elohim~»~: juge, ange.} Yahweh, Dieu\FTNT{Dieu~: de l'hébreu «~El~»~: puissant, etc.}, Dieu\FTNT{Dieu~: de l'hébreu «~elohim~»~: juge, ange.}Yahweh, le sait, et Israël lui-même le saura~! Si c'est par rébellion et par infidélité envers Yahweh, alors qu'il ne nous vienne point en aide aujourd'hui.
\VS{23}Si nous nous sommes bâti un autel pour nous détourner de Yahweh, si c'est pour y offrir des holocaustes, ou des offrandes, ou si c'est pour y faire des sacrifices d'offrande de paix, que Yahweh lui-même nous en demande compte~!
\VS{24}C'est bien plutôt par une sorte de crainte que nous avons fait cela, en pensant que vos enfants pourraient un jour parler à nos enfants et leur dire~: Qu'y a-t-il de commun entre vous et Yahweh, le Dieu d'Israël~?
\VS{25}Puisque Yahweh a mis le Jourdain pour frontière entre nous et vous, enfants de Ruben, et enfants de Gad~; vous n'avez point de part à Yahweh~! Et ainsi vos enfants feraient qu'un jour nos enfants cesseraient de craindre Yahweh\FTNT{Né. 2:20~; Ac. 8:21.}.
\VS{26}C'est pourquoi nous avons dit~: Mettons-nous maintenant à bâtir un autel, non pour des holocaustes ni pour des sacrifices~;
\VS{27}mais afin qu'il serve de témoignage entre nous et vous, et entre nos descendants et les vôtres, que nous voulons servir Yahweh devant sa face par nos holocaustes et nos sacrifices d'expiation et d'offrande de paix, afin que vos fils ne disent pas un jour à nos fils~: Vous n'avez point de part à Yahweh\FTNT{Ge. 31:48.}~!
\VS{28}C'est pourquoi nous avons dit~: Lorsqu'ils nous tiendront ce discours, ou à nos descendants, nous leur dirons~: Voyez la forme de l'autel de Yahweh qu'ont fait nos pères, non pour des holocaustes, ni pour des sacrifices, mais afin qu'il soit témoin entre nous et vous.
\VS{29}A Dieu ne plaise que nous nous révoltions contre Yahweh et que nous nous détournions aujourd'hui de Yahweh, en bâtissant un autel pour des holocaustes, pour des offrandes, et pour des sacrifices, outre l'autel de Yahweh, notre Dieu, qui est devant son tabernacle~!
\VS{30}Or après que le prêtre Phinées, et les princes de l'assemblée, les chefs des milliers d'Israël qui étaient avec lui, eurent entendu les paroles que les fils de Ruben, les fils de Gad, et les fils de Manassé leur dirent, ils furent satisfaits.
\VS{31}Et Phinées, fils du prêtre Eléazar, dit aux fils de Ruben, aux fils de Gad, et aux fils de Manassé~: Nous reconnaissons aujourd'hui que Yahweh est au milieu de nous, puisque vous n'avez point commis cette infidélité contre Yahweh~; vous avez ainsi délivré les enfants d'Israël de la main de Yahweh.
\VS{32}Ainsi Phinées, fils du prêtre Eléazar, et les princes, quittèrent les fils de Ruben, les fils de Gad, et revinrent du pays de Galaad dans le pays de Canaan, auprès des enfants d'Israël, auxquels ils firent un rapport.
\VS{33}Et la chose plut aux enfants d'Israël~; ils bénirent Dieu, et ne parlèrent plus de monter en armes contre eux pour détruire le pays où habitaient les fils de Ruben, et les fils de Gad.
\VS{34}Les fils de Ruben, et les fils de Gad appelèrent l'autel Ed~; car, dirent-ils, il est témoin entre nous que Yahweh est Dieu.
\Chap{23}
\TextTitle{Avertissements de Josué}
\VerseOne{}Or il arriva, plusieurs jours après, que Yahweh ayant donné du repos à Israël de tous les ennemis qui l'entouraient, Josué était vieux, fort avancé en âge.
\VS{2}Et Josué convoqua tout Israël, ses anciens, ses chefs, ses juges, ses officiers, et leur dit~: Je suis devenu vieux, fort avancé en âge.
\VS{3}Vous avez vu tout ce que Yahweh, votre Dieu, a fait à toutes ces nations à cause de vous~; car Yahweh, votre Dieu, est celui qui combat pour vous.
\VS{4}Voyez, je vous ai donné en héritage par le sort, selon vos tribus, ces nations qui sont restées, depuis le Jourdain, et toutes les nations que j'ai exterminées, jusqu'à la grande mer vers le soleil couchant.
\VS{5}Yahweh, votre Dieu, les repoussera devant vous et les chassera~; et vous posséderez leur pays en héritage, comme Yahweh, votre Dieu, vous l'a dit\FTNT{Ex. 14:14~; Ex. 23:27~; No. 33:53~; De. 6:18-19.}.
\VS{6}Fortifiez-vous donc de plus en plus pour garder et faire tout ce qui est écrit dans le livre de la loi de Moïse, afin que vous ne vous détourniez ni à droite ni à gauche\FTNT{De. 5:32~; De. 28:14.}.
\VS{7}Et que vous ne vous mêliez point avec ces nations qui sont restées parmi vous~; et que vous ne fassiez point mention de leurs dieux ; et que vous ne fassiez jurer personne par eux, et que vous ne les serviez point, et ne vous prosterniez point devant eux\FTNT{Ex. 23:13~; De. 12:3~; Jé. 5:7~; De. 6:14.}.
\VS{8}Mais attachez-vous à Yahweh, votre Dieu, comme vous l'avez fait jusqu'à ce jour\FTNT{De. 11:22.}.
\VS{9}C'est pour cela que Yahweh a chassé devant vous des nations grandes et puissantes~; nul n'a pu vous résister jusqu'à ce jour.
\VS{10}Un seul homme d'entre vous en poursuivait mille~; car Yahweh, votre Dieu est celui qui combat pour vous, comme il vous l'a dit\FTNT{Lé. 26:8~; De. 32:30.}.
\VS{11}Veillez donc attentivement sur vos âmes, afin d'aimer Yahweh, votre Dieu.
\VS{12}Autrement, si vous vous détournez et que vous vous attachez au reste de ces nations qui sont demeurées parmi vous, si vous faites alliance par des mariages avec elles, et si vous formez ensemble des relations,
\VS{13}sachez certainement que Yahweh, votre Dieu, ne continuera pas à chasser ces nations devant vous~; mais elles seront pour vous un piège et un filet, un fouet dans vos côtés et des épines dans vos yeux, jusqu'à ce que vous ayez péri de dessus cette bonne terre que Yahweh, votre Dieu, vous a donnée\FTNT{Ex. 23:33~; De. 7:16~; Jg. 2:3.}.
\VS{14}Voici, je m'en vais aujourd'hui par le chemin de toute la terre. Reconnaissez de tout votre cœur et de toute votre âme qu'aucune de toutes les bonnes paroles prononcées sur vous par Yahweh, votre Dieu, n'est restée sans effet~; toutes se sont accomplies pour vous, aucune n'est restée sans effet\FTNT{Jos. 21:45~; 2 R. 10:10.}.
\VS{15}Et il arrivera que comme toutes les bonnes paroles que Yahweh, votre Dieu, vous a dites vous sont arrivées~; ainsi Yahweh fera venir sur vous toutes les paroles mauvaises, jusqu'à ce qu'il vous ait exterminés de dessus cette bonne terre que Yahweh, votre Dieu, vous a donnée.
\VS{16}Si vous transgressez l'alliance que Yahweh, votre Dieu, vous a prescrite, et si vous allez servir d'autres dieux et vous prosterner devant eux, la colère de Yahweh s'enflammera contre vous, et vous périrez promptement de dessus cette bonne terre qu'il vous a donnée.
\Chap{24}
\TextTitle{Josué rappelle à Israël son histoire}
\VerseOne{}Josué assembla toutes les tribus d'Israël à Sichem, et il convoqua les anciens d'Israël, ses chefs, ses juges, et ses officiers, qui se présentèrent devant Dieu.
\VS{2}Et Josué dit à tout le peuple~: Ainsi parle Yahweh, le Dieu d'Israël~: Vos pères, Térach, père d'Abraham et père de Nachor, ont anciennement habité de l'autre côté du fleuve, où ils servaient d'autres dieux.
\VS{3}Mais j'ai pris votre père Abraham de l'autre côté du fleuve, je lui fis parcourir tout le pays de Canaan, je multipliai sa postérité, et lui donnai Isaac\FTNT{Ge. 12~; Ge. 21:2.}.
\VS{4}Je donnai à Isaac, Jacob et Esaü~; et je donnai à Esaü le mont de Séir, pour le posséder~; mais Jacob et ses fils descendirent en Egypte\FTNT{Ge. 25:24~; Ge. 36:6.}.
\VS{5}Puis j'envoyai Moïse et Aaron, et je frappai l'Egypte, par les prodiges que j'opérai au milieu d'elle~; puis je vous en fis sortir\FTNT{Ex. 3:10.}.
\VS{6}Je fis donc sortir vos pères hors de l'Egypte, et vous arrivâtes à la mer. Les Egyptiens poursuivirent vos pères avec des chars et des cavaliers, jusqu'à la Mer Rouge\FTNT{Ex. 14:9.}.
\VS{7}Alors ils crièrent à Yahweh. Et il mit des ténèbres entre vous et les Egyptiens, et ramena sur eux la mer, qui les couvrit. Vos yeux ont vu ce que j'ai fait aux Egyptiens. Puis vous restâtes longtemps dans le désert.
\VS{8}Ensuite je vous conduisis dans le pays des Amoréens, qui habitaient de l'autre côté du Jourdain, et ils combattirent contre vous. Mais je les livrai entre vos mains~; vous prîtes possession de leur pays, et je les détruisis devant vous.
\VS{9}Balak\FTNT{Balaak~: Voir No. 22:2-14.} aussi, fils de Tsippor, roi de Moab, se leva, et fit la guerre à Israël. Il fit appeler Balaam\FTNT{Balaam~: Voir No. 22.}, fils de Beor, pour qu'il vous maudisse.
\VS{10}Mais je ne voulus point écouter Balaam~; il s'agenouilla et vous bénit, et je vous délivrai de la main de Balak.
\VS{11}Et vous passâtes le Jourdain, et arrivâtes près de Jéricho. Les habitants de Jéricho, les Amoréens, les Phéréziens, les Cananéens, les Héthiens, les Guirgasiens, les Héviens et les Jébusiens vous firent la guerre. Je les livrai entre vos mains,
\VS{12}et j'envoyai devant vous des frelons qui les chassèrent loin de votre face, comme les deux rois des Amoréens~: Ce ne fut ni par ton épée, ni par ton arc\FTNT{Ex. 23:28~; De. 7:20.}.
\VS{13}Je vous donnai une terre que vous n'aviez point cultivée, des villes que vous n'aviez point bâties, et que vous habitez, et vous mangez les fruits des vignes et des oliviers que vous n'avez point plantés\FTNT{De. 6:10~; Ps. 105:44~; Né. 9:25.}.
\TextTitle{Le peuple choisit de servir Yahweh}
\VS{14}Maintenant, craignez Yahweh, et servez-le avec intégrité et avec fidélité. Ôtez les dieux que vos pères ont servis de l'autre côté du fleuve et en Egypte, et servez Yahweh\FTNT{1 S. 12:23-24~; Ez. 20:7-44.}.
\VS{15}Et s'il vous déplaît de servir Yahweh, choisissez aujourd'hui qui vous voulez servir, ou les dieux que servaient vos pères au-delà du fleuve, ou les dieux des Amoréens dans le pays desquels vous habitez. Mais moi et ma maison, nous servirons Yahweh.
\VS{16}Alors le peuple répondit, et dit~: Que Dieu nous garde d'abandonner Yahweh pour servir d'autres dieux~!
\VS{17}Car Yahweh, notre Dieu, est celui qui nous a fait monter, nous et nos pères, hors du pays d'Egypte, de la maison de servitude, qui a fait devant nos yeux ces grands signes, qui nous a gardés dans tout le chemin par lequel nous avons marché, et entre tous les peuples parmi lesquels nous avons passé.
\VS{18}Yahweh a chassé devant nous tous les peuples, et même les Amoréens qui habitaient ce pays. Nous servirons aussi Yahweh, car il est notre Dieu.
\VS{19}Josué dit au peuple~: Vous ne pourrez pas servir Yahweh, car c'est un Dieu Saint, qui est jaloux, il ne pardonnera point votre rébellion et vos péchés.
\VS{20}Lorsque vous abandonnerez Yahweh et que vous servirez les dieux des étrangers, il reviendra vous faire du mal, et il vous consumera après vous avoir fait du bien.
\VS{21}Le peuple dit à Josué~: Non~! Car nous servirons Yahweh.
\VS{22}Et Josué dit au peuple~: Vous êtes témoins contre vous-mêmes que c'est vous qui avez choisi Yahweh pour le servir. Et ils répondirent~: Nous en sommes témoins.
\VS{23}Maintenant donc ôtez les dieux étrangers qui sont au milieu de vous, et tournez votre cœur vers Yahweh, le Dieu d'Israël.
\VS{24}Et le peuple répondit à Josué~: Nous servirons Yahweh, notre Dieu et nous obéirons à sa voix.
\VS{25}Ce jour-là, Josué traita alliance avec le peuple, et lui donna des lois et des ordonnances à Sichem.
\VS{26}Josué écrivit ces paroles dans le livre de la loi de Dieu. Il prit aussi une grande pierre\FTNT{Cette Pierre est une préfiguration de Jésus-Christ, le «~témoin fidèle~» (Ap. 19:11). Voir en commentaire en (Es. 8:13-16).}, qu'il dressa là sous le chêne qui était dans le lieu consacré à Yahweh.
\VS{27}Josué dit à tout le peuple~: Voici, cette pierre servira de témoin contre nous, car elle a entendu toutes les paroles que Yahweh nous a déclarées~; elle servira de témoin contre vous, afin que vous ne reniiez pas votre Dieu.
\VS{28}Puis Josué renvoya le peuple, chacun dans son héritage.
\TextTitle{Mort de Josué et d'Eléazar~; ensevelissement des os de Joseph (Ge. 50:26)}
\VS{29}Or il arriva, après ces choses, que Josué, fils de Nun, serviteur de Yahweh, mourut, âgé de cent dix ans.
\VS{30}Et on l'ensevelit dans le territoire de son héritage, à Thimnath-Sérach, dans la montagne d'Ephraïm, du côté du nord de la montagne de Gaasch.
\VS{31}Et Israël servit Yahweh tout le temps de Josué, et tout le temps des anciens qui survécurent à Josué, qui avaient connu toutes les œuvres que Yahweh avait faites pour Israël.
\VS{32}Les os de Joseph\FTNT{(Ge. 50:25~; Ex. 13:19~; Hé. 11:22).}, que les enfants d'Israël avaient rapportés d'Egypte, furent ensevelis à Sichem, dans la portion du champ que Jacob avait achetée des fils de Hamor, père de Sichem, pour cent kesita, et qui appartint à l'héritage des fils de Joseph.
\VS{33}Et Eléazar, fils d'Aaron, mourut, on l'enterra à Guibeath-Phinées, qui avait été donnée à son fils Phinées, dans la montagne d'Ephraïm.
\PPE{}
\end{multicols}
