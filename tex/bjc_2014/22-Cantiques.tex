\ShortTitle{Ca.}\BookTitle{Cantique des cantiques}\BFont
\noindent\hrulefill
{\footnotesize
\textit{
\bigskip
{\centering{}
\\Auteur : Salomon
\\(Heb. : ShirHashirim)
\\Signification : Cantique des cantiques
\\Thème : L'amour
\\Date de rédaction : 10\up{ème} siècle av J.-C.\\}
}
%\bigskip
\textit{
\\Le Cantique des cantiques est une suite de poèmes célébrant l'amour. Il présente la relation d'un homme et d'une femme, dont l'attrait réciproque les pousse à dévoiler leurs sentiments avec douceur. L'union de Salomon et de la Sulamithe exprime différents aspects de l'amour tels que l'attente, le désir, la souffrance ou la passion, et ce dans toute la pureté et la sainteté qui conviennent à la maison de Dieu.
%\bigskip
\\Au-delà de l'image du couple, on peut découvrir au travers de ces écrits l'expression de l'amour de Dieu à l'égard de l'homme, tout comme celui de l'Epoux divin envers l'Eglise.
\bigskip
}
}
\par\nobreak\noindent\hrulefill
\begin{multicols}{2}
\Chap{1}
\TextTitle{La fiancée et le fiancé expriment leur amour mutuel.}
\VerseOne{}Le Cantique des cantiques qui est de Salomon.
\VS{2}[La Sulamithe :] Qu'il me baise des baisers de sa bouche ! [Les filles de Jérusalem :] Car tes amours sont plus agréables que le vin,
\VS{3}à cause de l'odeur de tes excellents parfums, ton nom est comme un parfum répandu ; c'est pourquoi les filles t'aiment.
\VS{4}[La Sulamithe :] Tire-moi [Les filles de Jérusalem :] et nous courrons après toi ! [La Sulamithe :] Lorsque le roi m'introduit dans ses chambres, [Les filles de Jérusalem :] Nous nous égaierons et nous nous réjouirons en toi ; nous célébrerons tes amours plus que le vin ; les hommes droits t'aiment.
\VS{5}[La Sulamithe :] Ô filles de Jérusalem, je suis noire, je suis belle. Je suis comme les tentes de Kédar, et comme les courtines de Salomon.
\VS{6}Ne prenez pas garde à moi, de ce que je suis noire : Car le soleil m'a regardée. Les fils de ma mère se sont mis en colère contre moi, ils m'ont faite gardienne des vignes et je n'ai pas gardé la vigne qui était à moi.
\VS{7}Déclare-moi, toi que mon âme aime, où tu fais paître ton troupeau, et où tu les fais reposer à midi ; car pourquoi serais-je comme une femme errante vers les troupeaux de tes compagnons ?
\VS{8}[Salomon :] Si tu ne le sais pas, ô la plus belle des femmes, sors sur les traces du troupeau, et fais paître tes chevrettes près des cabanes des bergers.
\VS{9}Ma grande amie, je te compare au plus beau couple de chevaux que j'ai aux chars de Pharaon.
\VS{10}Tes joues ont bonne grâce avec les atours, et ton cou avec les colliers.
\VS{11}[Les filles de Jérusalem :] Nous te ferons des atours avec des boutons d'argent.
\VS{12}[La Sulamithe :] Tandis que le roi est assis à table, mon nard exhale son parfum.
\VS{13}Mon bien-aimé est pour moi comme un sachet de myrrhe, il passe la nuit entre mes seins.
\VS{14}Mon bien-aimé m'est comme une grappe de troëne dans les vignes d'En-Guédi.
\VS{15}[Salomon :] Te voilà belle, ma grande amie, te voilà belle ! Tes yeux sont comme ceux des colombes.
\VS{16}[La Sulamithe :] Te voilà beau, mon bien-aimé, que tu es agréable ! Aussi, notre couche est-elle verdoyante.
\VS{17}Les poutres de nos maisons sont de cèdre, et nos chevrons de cyprès.
\Chap{2}
\TextTitle{L'amour entre la fiancée et le fiancé}
\VerseOne{}[La Sulamithe :] Je suis la rose de Saron, et le lis des vallées.
\VS{2}[Salomon :] Tel qu'est le lis entre les épines\FTNT{L'Eglise du Seigneur est au milieu des loups (Mt. 10:16).}, telle est ma grande amie entre les filles.
\VS{3}[La Sulamithe :] Tel qu'est le pommier entre des arbres de la forêt, tel est mon bien-aimé entre les jeunes hommes. J'ai désiré son ombre et m'y suis assise et son fruit a été doux à mon palais.
\VS{4}Il m'amenait dans la salle du festin\FTNT{La maison du vin ou des festins (Ap. 19:7-9).} ; et sa bannière que je porte c'est l'amour\FTNT{Jésus-Christ est notre bannière (Ex. 17:15).}.
\VS{5} Ranimez-moi avec des gâteaux de raisins, fortifiez-moi avec des pommes, car je suis malade d'amour.
\VS{6}Que sa main gauche soit sous ma tête et que sa droite m'embrasse !
\VS{7}[Salomon :] Filles de Jérusalem, je vous en conjure, par les chevreuils et par les biches des champs, ne réveillez pas celle que j'aime, ne la réveillez pas jusqu'à ce qu'elle le veuille.
\TextTitle{La Sulamithe parle de Salomon}
\VS{8}[La Sulamithe :] C'est ici la voix de mon bien-aimé ! Le voici qui vient, sautant sur les montagnes et bondissant sur les collines.
\VS{9}Mon bien-aimé est semblable au chevreuil ou au faon des biches. Le voici qui se tient derrière notre muraille, il regarde par les fenêtres, il se fait voir par les treillis.
\VS{10}[La Sulamithe rapporte les paroles de Salomon :] Mon bien-aimé parle et me dit : Lève-toi, ma grande amie, ma belle, et viens !
\VS{11}Car voici, l'hiver est passé ; la pluie a cessé, elle s'en est allée.
\VS{12}Les fleurs paraissent sur la terre, le temps des chansons est venu, et la voix de la tourterelle se fait déjà entendre dans notre contrée.
\VS{13}Le figuier produit ses premiers fruits, et les vignes en fleurs exhalent le parfum. Lève-toi, ma grande amie, ma belle, et viens !
\VS{14}Ma colombe qui te tiens dans les fentes du rocher\FTNT{Jésus-Christ notre rocher (Es. 8:13-17). L'Eglise véritable est fondée sur le roc, c'est-à-dire Jésus-Christ lui-même (Mt. 16:18).}, dans les lieux secrets\FTNT{Les lieux secrets (Mt. 6:6).} et escarpés, fais-moi voir ta figure, fais-moi entendre ta voix ; car ta voix est douce, et ta figure est belle.
\VS{15}Prenez-nous les renards, et les petits renards qui ravagent les vignes, car nos vignes produisent des grappes.
\VS{16}[La Sulamithe :] Mon bien-aimé est à moi, et je suis à lui ; il fait paître son troupeau parmi les lis.
\VS{17}Avant que le jour se rafraîchisse et que les ombres s'enfuient, reviens mon bien-aimé ! Sois comme le chevreuil ou le faon des biches, sur les montagnes de Bether\FTNT{Montagnes de séparation : région montagneuse de Palestine, site inconnu.}.
\Chap{3}
\TextTitle{La fiancée recherche son fiancé et le trouve.}
\VerseOne{}[La Sulamithe :] J'ai cherché pendant les nuits, sur ma couche, celui que mon âme aime ; je l'ai cherché, mais je ne l'ai point trouvé.
\VS{2}Je me lèverai maintenant, et je ferai le tour de la ville, par les carrefours et par les places ; et je chercherai celui que mon âme aime. Je l'ai cherché, mais je ne l'ai point trouvé.
\VS{3}Les gardes qui font la ronde dans la ville m'ont rencontrée : N'avez-vous pas vu, leur ai-je dit, celui que mon âme aime ?
\VS{4}A peine les avais-je passés, que j'ai trouvé celui que mon âme aime ; je l'ai saisi et je ne l'ai pas lâché jusqu'à ce que je l'aie amené dans la maison de ma mère et dans la chambre de celle qui m'a conçue.
\VS{5}[Salomon :] Filles de Jérusalem, je vous en conjure par les chevreuils et par les biches des champs, ne réveillez pas celle que j'aime, ne la réveillez pas, jusqu'à ce qu'elle le veuille.
\TextTitle{Salomon entre avec sa fiancée dans Sion}
\VS{6}[La Sulamithe :] Qui est celle qui monte du désert, comme des colonnes de fumée en forme de palmiers, parfumée de myrrhe et d'encens, et de tous les aromates du parfumeur ?
\VS{7}[Un officier des gardes du roi Salomon :] Voici le lit de Salomon, autour duquel il y a soixante vaillants hommes, des plus vaillants d'Israël.
\VS{8}Tous maniant l'épée et très bien exercés à la guerre ; chacun porte son épée sur sa cuisse, à cause des frayeurs de la nuit.
\VS{9}Le roi Salomon s'est fait un lit de bois du Liban.
\VS{10}Il a fait ses piliers d'argent et l'intérieur d'or, son siège d'écarlate ; son intérieur tapissé avec amour, par les filles de Jérusalem.
\VS{11}[Les filles de Jérusalem :] Sortez, filles de Sion, et regardez le roi Salomon avec la couronne dont sa mère l'a couronné le jour de ses noces, le jour de la joie de son cœur.
\Chap{4}
\TextTitle{Le fiancé déclare son amour}
\VerseOne{}[Salomon :] Te voilà belle, ma grande amie, te voilà belle ! Tes yeux sont comme ceux des colombes derrière ton voile. Tes cheveux sont comme un troupeau de chèvres sur les pentes de la montagne de Galaad.
\VS{2}Tes dents sont comme un troupeau de brebis tondues qui remontent du lavoir, qui sont toutes deux à deux, et dont aucune ne manque.
\VS{3}Tes lèvres sont comme un fil teint d'écarlate, et ta bouche est gracieuse ; ta tempe est comme une pièce de pomme de grenade, derrière ton voile.
\VS{4}Ton cou est comme la tour de David, bâtie pour les armes à laquelle sont suspendus mille boucliers et tous les boucliers des hommes vaillants.
\VS{5}Tes deux seins sont comme deux faons, comme les jumeaux d'une gazelle qui paissent au milieu des lis.
\VS{6}Avant que le vent du jour souffle et que les ombres s'enfuient, je m'en irai à la montagne de myrrhe, et à la colline de l'encens.
\VS{7}Tu es toute belle, ma grande amie, et il n'y a point de tache\FTNT{Ep. 5 :27.} en toi.
\VS{8}Viens du Liban avec moi, mon épouse, viens du Liban avec moi ! Regarde du sommet de l'Amana, du sommet du Senir et de l'Hermon, des repaires des lions et des montagnes des léopards.
\VS{9}Tu me ravis le cœur, ma sœur, mon épouse, tu me ravis le cœur, par l'un de tes regards et par l'un des colliers de ton cou.
\VS{10}Combien sont belles tes amours, ma sœur, mon épouse ! Combien sont tes amours meilleures que le vin et l'odeur de tes parfums exhale plus que tous les aromates !
\VS{11}Tes lèvres, mon épouse, distillent des rayons de miel ; le miel et le lait sont sous ta langue, et l'odeur de tes vêtements est comme l'odeur du Liban.
\VS{12}Ma sœur, mon épouse, tu es un jardin clos, une source close, une fontaine scellée\FTNT{La virginité de l'Eglise : « un jardin fermé », « une source fermée » et « une fontaine scellée ». Nous avons ici une belle image de l'Eglise (Ep. 5:25-27) comparée par l'apôtre Paul à une vierge pure destinée à être présentée à son Christ, son Epoux (2 Co. 11:2). Selon la loi de Moïse, un homme qui séduisait une vierge devait l'épouser (Ex. 22:16) et ne pouvait plus jamais la répudier (De. 22:28-29). Tout comme la femme adultère, la vierge fiancée qui se rendait coupable d'infidélité était lapidée (De. 22:22-24), sauf si elle n'avait pas pu appeler à l'aide lors de son viol (De. 22:25-27). Un homme qui accusait faussement sa femme de ne pas avoir été vierge lors du mariage était sévèrement puni, mais si l'accusation se révélait exacte, la femme était lapidée (De. 22:13-21). Si une femme était faussement accusée, ses parents pouvaient produire, pour la disculper, les « signes de sa virginité » (De. 22:15), c'est-à-dire le drap nuptial portant les traces sanglantes de la perforation de l'hymen. Notons que le grand prêtre n'avait le droit d'épouser qu'une vierge (Lé. 21:13-14). De même, Jésus-Christ, le grand prêtre par excellence, désire aussi une Eglise vierge.}.
\VS{13}Tes rejetons sont un jardin clos de grenadiers avec des fruits délicieux, des troënes avec le nard.
\VS{14}Le nard et le safran, le roseau aromatique et le cinnamome, avec tous les arbres d'encens ; la myrrhe et l'aloès, avec tous les principaux aromates.
\VS{15}Ô fontaine des jardins ! Ô source d'eaux vives ! Ruisseaux coulant du Liban !
\VS{16}Lève-toi, aquilon\FTNTT{Vent du nord.} et viens autan\FTNTT{Vent du sud.} du midi ! Soufflez sur mon jardin afin que ses drogues aromatiques distillent ! [La Sulamithe :] Que mon bien-aimé entre dans son jardin et qu'il mange de ses fruits délicieux !
\Chap{5}
\VerseOne{}[Salomon :] J'entre dans mon jardin, ma sœur, mon épouse ; je cueille ma myrrhe avec mes drogues aromatiques, je mange mes rayons de miel et mon miel, je bois mon vin et mon lait. Mes amis, mangez, buvez ; faites bonne chère, mes bien-aimés.
\TextTitle{La Sulamithe raconte son rêve}
\VS{2}[La Sulamithe :] J'étais endormie, mais mon cœur veillait\FTNT{Mt. 25:1-13.}, et voici c'est la voix de mon bien-aimé qui frappe, en disant : [La Sulamithe rapporte les propos de Salomon :] Ouvre-moi, ma sœur, ma grande amie, ma colombe, ma parfaite ! Car ma tête est pleine de rosée et mes cheveux de l'humidité de la nuit.
\VS{3}[La Sulamithe :] J'ai enlevé ma tunique, lui dis-je, comment la revêtirais-je ? J'ai lavé mes pieds, comment les souillerais-je ?
\VS{4}Mon bien-aimé a avancé la main par le trou de la porte, et mes entrailles ont été émues à cause lui.
\VS{5}Je me suis levée pour ouvrir à mon bien-aimé, et la myrrhe a distillé de mes mains, et la myrrhe coulante de mes doigts sur la poignée du verrou.
\VS{6}J'ai ouvert à mon bien-aimé, mais mon bien-aimé s'était retiré, il était parti au loin ; mon âme était hors de moi quand il me parlait. Je l'ai cherché, mais je ne l'ai point trouvé ; je l'ai appelé, mais il ne m'a point répondu.
\VS{7}Les gardes qui font la ronde dans la ville m'ont rencontrée ; ils m'ont battue, ils m'ont blessée ; les gardes des murailles m'ont enlevé mon voile.
\VS{8}Filles de Jérusalem, je vous en conjure, si vous trouvez mon bien-aimé, que lui direz-vous ? Que je suis malade d'amour.
\VS{9}[Les filles de Jérusalem :] Qu'a ton bien-aimé de plus qu'un autre, ô la plus belle d'entre les femmes ? Qu'a ton bien-aimé de plus qu'un autre pour que tu nous conjures ainsi ?
\TextTitle{La Sulamithe décrit Salomon}
\VS{10}[La Sulamithe :] Mon bien-aimé est éblouissant et vermeil, un porte-enseigne choisi entre dix mille.
\VS{11}Sa tête est un or très fin ; ses cheveux sont crépus, noirs comme un corbeau.
\VS{12}Ses yeux sont comme ceux des colombes sur les ruisseaux des eaux courantes, lavés dans du lait, et comme enchâssés dans des chatons d'anneau. 
\VS{13}Ses joues sont comme un parterre de drogues aromatiques, comme des fleurs parfumées ; ses lèvres sont comme des lis, elles distillent la myrrhe coulante. 
\VS{14}Ses mains sont comme des anneaux d'or où il y a des chrysolithes enchâssées ; son ventre est comme de l'ivoire bien poli, couvert de saphirs.
\VS{15}Ses jambes sont comme des piliers de marbre, posés sur des bases d'or fin. Son apparence est comme le Liban, elle est précieuse comme les cèdres.
\VS{16}Son palais n'est que douceur, et tout ce qui est en lui est désirable\FTNT{Mt. 11:29.}. Tel est mon bien-aimé, tel est mon ami, filles de Jérusalem !
\Chap{6}
\TextTitle{Les filles de Jérusalem aident la Sulamithe à chercher Salomon}
\VerseOne{}[Les filles de Jérusalem :] Où est allé ton bien-aimé, ô la plus belle des femmes ? De quel côté est allé ton bien-aimé ? Nous le chercherons avec toi.
\VS{2}[La Sulamithe :] Mon bien-aimé est descendu à son jardin, au parterre de drogues aromatiques, pour faire paître son troupeau dans les jardins, et pour cueillir des lis.
\VS{3}Je suis à mon bien-aimé et mon bien-aimé est à moi ; il fait paître son troupeau parmi les lis.
\TextTitle{Salomon à la Sulamithe}
\VS{4}[Salomon :] Tu es belle, grande amie, comme Thirtsa ; agréable comme Jérusalem, redoutable comme des armées qui marchent sous leurs bannières déployées. 
\VS{5}Détourne de moi tes yeux, car ils me troublent. Tes cheveux sont comme un troupeau de chèvres suspendus aux flancs de Galaad.
\VS{6}Tes dents sont comme un troupeau de brebis qui remontent du lavoir, qui sont toutes deux à deux: il n'en manque aucune.
\VS{7}Ta tempe est comme une pièce de pomme de grenade derrière ton voile.
\VS{8}Qu'il y ait soixante reines, quatre-vingts concubines, et des vierges sans nombre ;
\VS{9}ma colombe, ma parfaite est unique ; elle est l'unique de sa mère, de celle qui l'a enfantée. Les filles l'ont vue et l'ont dite bienheureuse ; les reines et les concubines l'ont louée en disant :
\VS{10}Qui est celle qui paraît comme l'aube du jour, belle comme la lune, brillante comme le soleil, redoutable comme des armées qui marchent avec des bannières déployées ?
\VS{11}[La Sulamithe :] Je suis descendue au jardin des noyers pour voir les fruits de la vallée qui mûrissent, et pour voir si la vigne pousse, et si les grenadiers fleurissent.
\VS{12}Je ne me suis pas aperçue que mon affection m'a rendue semblable aux chars d'Amminadib\FTNT{Amminadib : Mon peuple est bien disposé.}.
\Chap{7}
\TextTitle{La beauté de la Sulamithe}
\VerseOne{}[Les filles de Jérusalem :] Reviens, reviens, ô Sulamithe ! Reviens, reviens, que nous te contemplions. [La Sulamithe :] Qu'avez-vous à contempler la Sulamithe comme une danse de deux armées ?
\VS{2}[Les filles de Jérusalem :] Fille de prince combien sont belles tes démarches avec ta chaussure ! Les contours de ta hanche sont comme des colliers travaillés de la main d'un excellent ouvrier.
\VS{3}Ton nombril est comme une tasse ronde, toute comble de breuvage, ton ventre est comme un tas de blé entouré de lis.
\VS{4}Tes deux seins sont comme deux faons jumeaux d'une gazelle.
\VS{5}Ton cou est comme une tour d'ivoire, tes yeux sont comme les étangs qui sont à Hesbon, près de la porte de Bath-Rabbim et ton visage est comme la tour du Liban qui regarde vers Damas.
\VS{6}Ta tête est sur toi comme le Carmel, et les cheveux fins de ta tête sont comme de l'écarlate ; le roi est attaché aux galeries pour te contempler.
\VS{7}[Salomon :] Que tu es belle, que tu es agréable, amour délicieuse !
\VS{8}Ta taille est semblable à un palmier, et tes seins à des grappes.
\VS{9}Je me dis : Je monterai sur le palmier, j'empoignerai ses branches et tes seins me seront maintenant comme des grappes de vigne et l'odeur de tes narines comme l'odeur des pommes,
\VS{10}et ta bouche comme le bon vin [La Sulamithe :] …qui coule en faveur de mon bien-aimé et qui fait parler les lèvres de ceux qui s'endorment.
\VS{11}Je suis à mon bien-aimé et son désir est vers moi.
\VS{12}Viens, mon bien-aimé, sortons dans les champs, passons la nuit dans les villages !
\VS{13}Levons-nous dès le matin pour aller aux vignes, nous verrons si la vigne pousse, si la fleur s'ouvre, si les grenadiers fleurissent. Là je te donnerai mes amours.
\VS{14}Les mandragores répandent leur parfum, et à nos portes il y a toutes sortes de fruits exquis, des fruits nouveaux, et des fruits gardés que je t'ai conservés ô mon bien-aimé !
\Chap{8}
\VerseOne{}[La Sulamithe :] Plaise à Dieu ! Que tu es comme mon frère, qui a sucé les seins de ma mère ! Je te rencontrerais dehors, je t'embrasserais, et on ne m'en mépriserait point.
\VS{2}Je te conduirais, je t'introduirais dans la maison de ma mère ; tu m'instruiras, et je te ferais boire du vin parfumé d'aromates et du moût de mon grenadier.
\VS{3}Que sa main gauche soit sous ma tête, et que sa droite m'embrasse !
\VS{4}[La Sulamithe (citant Salomon) :] Je vous en conjure, filles de Jérusalem, ne réveillez pas celle que j'aime, ne la réveillez pas, jusqu'à ce qu'elle le veuille.
\VS{5}[Les frères de la Sulamithe :] Qui est celle-ci qui monte du désert, mollement appuyée sur son bien-aimé ? [Salomon :] Je t'ai réveillée sous le pommier ; là où ta mère t'a enfantée, là où celle qui t'a conçue t'a donné le jour.
\VS{6}Place-moi comme un sceau sur ton cœur\FTNT{l'Eglise du Seigneur est scellée du Saint-Esprit (Ep. 1:13 ; Ep. 4:30).}, comme un sceau sur ton bras ; car l'amour est fort comme la mort, et la jalousie est cruelle comme le scheol ; leurs embrasements sont des embrasements de feu et une flamme de Yah\FTNT{Yah est le diminutif de Yahweh.}.
\VS{7}[La Sulamithe (à Salomon) :] Beaucoup d'eaux ne pourraient éteindre cet amour-là et même les fleuves ne pourraient le submerger ; si un homme donnait toutes les richesses de sa maison pour cet amour, certainement on n'en tiendrait aucun compte.
\VS{8}[La Sulamithe (racontant ce que ses frères lui ont dit) :] Nous avons une petite sœur qui n'a pas encore de seins ; que ferons-nous à notre sœur le jour où on parlera d'elle ?
\VS{9}Si elle est comme une muraille, nous bâtirons sur elle un palais d'argent ; et si elle est comme une porte, nous la renforcerons avec une planche de cèdre.
\VS{10}[La Sulamithe :] Je suis comme une muraille, et mes seins sont comme des tours ; j'ai été alors à ses yeux comme celle qui trouve la paix. 
\VS{11}Salomon avait une vigne à Baal-Hamon, qu'il donna à des gardes et chacun d'eux doit en apporter pour son fruit mille pièces d'argent. 
\VS{12}Ma vigne, qui est à moi, est à mon commandement. Ô Salomon, que les mille pièces d'argent soient à toi, et qu'il y en ait deux cents pour les gardes du fruit de la vigne !
\VS{13}[Les frères de la Sulamithe :] Ô toi qui habites dans les jardins ! Les amis sont attentifs à ta voix. [Salomon :] Fais que je l'entende. 
\VS{14}[La Sulamithe :] Mon bien-aimé enfuis-toi aussi vite qu'un chevreuil ou qu'un faon des biches, sur les montagnes des drogues aromatiques. 
\PPE{}
\end{multicols}
