\ShortTitle{Romains}\BookTitle{Romains}\BFont
\noindent\hrulefill
{\footnotesize
\textit{
\bigskip
{\centering{}
\\Thème : L'Evangile de Dieu
\\Auteur : Paul
\\Date de rédaction : Env. 56\\}
}
%\bigskip
\textit{
\\Rome est une ville située dans la région du Latium, au centre de l’Italie, à la confluence de l’Aniene et du Tibre. Centre de l’Empire romain,  elle domina  l’Europe, l’Afrique du Nord et le Moyen-Orient du 1er siècle avant J.-C. au 5ème siècle après J.-C.
%\bigskip
\\La lettre était destinée à l’Eglise de Rome, fondée sans doute par des chrétiens convertis au  travers du ministère de Paul et d’autres apôtres itinérants. Cette Eglise comptait quelques juifs mais surtout des membres d’origine païenne. Cette épître fut rédigée au cours du 3ème voyage missionnaire de Paul,  pendant les trois mois que l’apôtre passa à Corinthe. En attendant de leur rendre visite physiquement, Paul avait le désir de communiquer aux chrétiens de Rome les grandes lignes du principe de la grâce, dont il avait eu la révélation. Il y aborda plusieurs doctrines majeures comme le salut par la foi et la grâce ainsi que des enseignements pratiques sur l’amour, le devoir du chrétien et la sainteté.\bigskip
}
}
\par\nobreak\noindent\hrulefill
\begin{multicols}{2}
\TextTitle{[L'Evangile de Christ, puissance de Dieu pour le salut de tous]}
\Chap{1}
\VerseOne{}Paul, serviteur de Jésus-Christ, appelé à être apôtre, mis à part pour annoncer l'Evangile de Dieu,
\VS{2}qu’il avait auparavant promis par ses prophètes dans les saintes Ecritures,
\VS{3}et qui concerne son Fils, qui est né de la postérité de David, selon la chair,
\VS{4}et qui a été pleinement déclaré Fils de Dieu avec puissance, selon l'Esprit de sainteté, par sa résurrection d'entre les morts, c'est-à-dire, Jésus-Christ notre Seigneur,
\VS{5}par qui nous avons reçu la grâce et l’apostolat, pour amener en son Nom tous les gentils à l’obéissance de la foi,
\VS{6}parmi lesquels aussi vous êtes, vous qui êtes appelés par Jésus-Christ.
\VS{7}A vous tous qui êtes à Rome, bien-aimés de Dieu, appelés à être saints\FTNT{Le terme «~saint~» est tiré du grec «~hagios~» qui signifie «~consacré à Dieu~», «~saint~», «~sacré~», «~pieux~». Ce mot est souvent utilisé au pluriel dans le testament de Jésus (Ac. 9:13 ;  Ac. 26:10). Il n’y a aucun rapport entre la compréhension catholique romaine du terme «~saint~» et l’enseignement biblique. Dans l’enseignement catholique romain, une personne ne devient pas sainte tant qu’elle n’a pas été béatifiée ou canonisée par le pape ou l’éminent évêque. Dans la Bible, tous ceux qui reçoivent Jésus-Christ par la foi sont appelés saints car ils sont mis à part pour Dieu. Dans le culte de l’église catholique romaine, les saints sont vénérés, priés, et parfois adorés. Dans la Bible, les saints sont appelés à adorer et à prier Dieu seul par Jésus-Christ homme, le seul Médiateur entre Dieu et les hommes (1 Ti. 2:5). Dans le Tanakh, les mots «~sanctifié~», «~saint~» et leurs dérivés viennent du mot hébreu «~Quodesch~» dont le sens général est : «~Mis à part pour Dieu~». Dans la Bible, ces mots sont appliqués à des objets et à des personnes. Le terme «~sanctification~» appliqué à des objets sous-entend l'idée qu'ils sont réservés uniquement pour le service de Dieu ; ils sont sanctifiés, mis à part pour Dieu. Dans le Testament de Jésus, il est appliqué aux personnes et comprend plusieurs sens : - Les croyants, par leur position, sont éternellement mis à part pour Dieu par la rédemption (Hé. 10:10-14). Ils sont donc considérés comme «~saints~», «~sanctifiés~» dès leur conversion (Ph. 1:1 ; Hé. 3:1). - Les croyants sont amenés à la sanctification par l'action du Saint-Esprit au moyen des Ecritures (Jn. 15:3 ; Jn. 17:17 ; 2 Co. 3:18 ; Ep. 5:25-26 ; 1Th. 5:23-24). - Les croyants attendent la venue du Seigneur pour la réalisation complète de leur sanctification (1 Co. 15:29-50 ; Ph. 3:20-21 ; Ep. 5:27 ; 1 Jn. 3:2 ; Ap. 22:12).} ; que la grâce et la paix vous soient données de la part de Dieu notre Père et du Seigneur Jésus-Christ.
\VS{8}Je rends premièrement grâces à mon Dieu par Jésus-Christ, au sujet de vous tous, de ce que votre foi est renommée dans le monde entier.
\VS{9}Car Dieu, que je sers en mon esprit dans l'Evangile de son Fils, m'est témoin que je fais sans cesse mention de vous,
\VS{10}demandant continuellement dans mes prières que je puisse enfin trouver, par la volonté de Dieu, quelque moyen favorable pour aller vers vous.
\VS{11}Car je désire extrêmement vous voir, pour vous communiquer quelque don spirituel, afin que vous soyez affermis ;
\VS{12}et aussi, afin qu'étant parmi vous, nous nous consolions ensemble par la foi qui nous est commune.
\VS{13}Or mes frères, je ne veux pas que vous ignoriez que j’ai souvent formé le dessein d'aller vers vous, afin de recueillir quelque fruit parmi vous, comme parmi les autres nations ; mais j'en ai été empêché jusqu'à présent.
\VS{14}Je me dois aux Grecs et aux barbares, aux sages et aux ignorants.
\VS{15}Ainsi, autant qu’il dépend de moi, je suis prêt à vous annoncer aussi l'Evangile à vous qui êtes à Rome.
\VS{16}Car je n'ai point honte de l'Evangile de Christ, vu qu'il est la puissance de Dieu pour le salut de tous ceux qui croient : Du Juif premièrement, puis du Grec,
\VS{17}parce qu’en lui est révélée la justice de Dieu pleinement de foi en foi, selon qu'il est écrit : Le juste vivra par la foi\FTNT{Ha. 2:4.}.
\TextTitle{[Jugement sur ceux qui retienne la vérité captive]}
\VS{18}Car la colère de Dieu se révèle pleinement du ciel contre toute impiété et injustice des hommes qui retiennent injustement la vérité captive.
\VS{19}Car ce qu’on peut connaître de Dieu est manifesté parmi eux ; car Dieu le leur a fait connaître.
\VS{20}En effet, les perfections invisibles de Dieu, à savoir sa puissance éternelle et sa divinité, se voient comme à l’œil nu, depuis la création du monde, quand on les considère dans ses ouvrages, de sorte qu'ils sont inexcusables.
\TextTitle{[Egarement des hommes]}
\VS{21}Parce qu'ayant connu Dieu, ils ne l'ont point glorifié comme Dieu, et ils ne lui ont point rendu grâces, mais ils se sont égarés dans leurs pensées, et leur cœur sans intelligence a été plongé dans les ténèbres.
\VS{22}Se vantant d’être sages, ils sont devenus fous.
\VS{23}Et ils ont changé la gloire du Dieu incorruptible en images\FTNT{Ex. 20:4-5 ; Mt. 22:20 ; Mc. 12:16 ; Lu. 20:24. Ap.13:14-15 ; Ap. 14:9-11 ; Ap. 15:2 ; Ap. 16:2 ; 19:20 ;  Ap. 20:4.} représentant l'homme corruptible, des oiseaux, des quadrupèdes, et des reptiles.
\TextTitle{[Les conséquences de l'endurcissement des hommes]}
\VS{24}C'est pourquoi aussi Dieu les a livrés aux convoitises de leurs cœurs et à l’impureté\FTNT{Dieu les a livrés à l’esprit d’égarement (1 R. 22 ; 2 Th. 2:10-13).}, ainsi ils déshonorent eux-mêmes leurs propres corps ;
\VS{25}eux qui ont changé la vérité de Dieu en mensonge, et qui ont adoré et servi la créature, au lieu du Créateur, qui est béni éternellement. Amen !
\VS{26}C'est pourquoi Dieu les a livrés à des passions infâmes, car leurs femmes ont changé l'usage naturel en celui qui est contre la nature.
\VS{27}Et de même les hommes, abandonnant l'usage naturel de la femme, se sont enflammés dans leurs désirs les uns envers les autres, commettant homme avec homme des choses infâmes, et recevant en eux-mêmes le salaire que méritait leur égarement.
\VS{28}Car comme ils ne se sont pas souciés de connaître Dieu, aussi Dieu les a livrés à leur sens réprouvé, pour commettre des choses indignes.
\VS{29}Etant remplis de toute espèce d’injustice, d'impureté, de méchanceté, d'avarice, de malignité, pleins d'envie, de meurtre, de querelles, de fraude, de mauvaises mœurs,
\VS{30}rapporteurs, médisants, haïssant Dieu, outrageux, orgueilleux, vains, ingénieux au mal, rebelles à leurs parents,
\VS{31}dépourvus d’intelligence, de loyauté, d’affection naturelle, de miséricorde.
\VS{32}Et bien qu'ils connaissent le jugement de Dieu, déclarant dignes de mort ceux qui commettent de telles choses, non seulement ils les font, mais encore ils approuvent ceux qui les font.
\TextTitle{[Ceux qui jugent les autres sans être meilleurs]}
\Chap{2}
\VerseOne{}C'est pourquoi, ô homme, qui que tu sois, toi qui juges les autres, tu es donc inexcusable ; car en jugeant les autres, tu te condamnes toi-même, puisque toi qui juges, tu commets les mêmes choses.
\VS{2}Or nous savons que le jugement de Dieu est selon la vérité pour ceux qui commettent de telles choses.
\VS{3}Et penses-tu, ô homme, qui juges ceux qui commettent de telles choses, et qui les commets, que tu échapperas au jugement de Dieu ?
\VS{4}Ou méprises-tu les richesses de sa douceur, et de sa patience, et de sa bonté ; ne reconnaissant pas que la bonté de Dieu te convie à la repentance ?
\VS{5}Mais par ta dureté, et par ton cœur qui est sans repentance, tu t'amasses la colère pour le jour de la colère, et de la manifestation du juste jugement de Dieu,
\VS{6}qui rendra à chacun selon ses œuvres ;
\VS{7}à savoir la vie éternelle à ceux qui, en persévérant dans les bonnes œuvres, cherchent la gloire, l'honneur et l'immortalité.
\VS{8}Mais il y aura de l'indignation et de la colère contre ceux qui ont un esprit de dispute, et qui se rebellent contre la vérité, et obéissent à l'injustice.
\VS{9}Il y aura tribulation et angoisse sur toute âme d'homme qui fait le mal, pour le Juif premièrement, puis pour le Grec.
\VS{10}Mais gloire, honneur, et paix pour quiconque fait le bien ; pour le Juif premièrement, puis pour le Grec.
\VS{11}Car Dieu n'a point d'égard à l'apparence des personnes.
\VS{12}Tous ceux qui auront péché sans la loi, périront aussi sans la loi ; et tous ceux qui auront péché ayant la loi, seront jugés par la loi.
\VS{13}Car ce ne sont pas, en effet, ceux qui écoutent la loi qui sont justes devant Dieu, mais ce sont ceux qui la mettent en pratique qui seront justifiés.
\VS{14}Or quand les gentils, qui n'ont point la loi, font naturellement ce que prescrit la loi, n'ayant point la loi, ils sont une loi pour eux-mêmes.
\VS{15}Et ils montrent par-là que l’œuvre de la loi est écrite dans leurs cœurs, puisque leur conscience leur rend témoignage, et que leurs pensées les accusent ou les défendent.
\VS{16}Tous, dis-je, donc seront jugés le jour où Dieu jugera les secrets des hommes par Jésus-Christ, selon mon Evangile.
\TextTitle{[Les Juifs, connaissant la loi, sont condamnés par la loi]}
\VS{17}Voici, tu portes le nom de Juif, tu te reposes entièrement sur la loi, et tu te glorifies de Dieu ;
\VS{18}tu connais sa volonté, et tu sais discerner ce qui est contraire, étant instruit par la loi ; 
\VS{19}et tu te crois être le conducteur des aveugles, la lumière de ceux qui sont dans les ténèbres,
\VS{20}le docteur des ignorants, le maître des ignorants, ayant le modèle de la science et de la vérité dans la loi.
\VS{21}Toi donc qui enseignes les autres, tu ne t’enseignes pas toi-même ! Toi qui prêches de ne pas dérober, tu dérobes !
\VS{22}Toi qui dis de ne pas commettre d’adultère, tu commets l’adultère ! Toi qui as en abomination les idoles, tu commets des sacrilèges !
\VS{23}Toi qui te glorifies de la loi, tu déshonores Dieu par la transgression de la loi.
\VS{24}Car le nom de Dieu est blasphémé parmi les gentils à cause de vous comme cela est écrit.
\VS{25}Il est vrai que la circoncision est profitable, si tu gardes la loi ; mais si tu es transgresseur de la loi, ta circoncision devient incirconcision.
\VS{26}Si donc l’incirconcis observe les ordonnances de la loi, son incirconcision ne sera-t-elle pas tenue pour circoncision ?
\VS{27}L’incirconcis de nature, qui accomplit la loi, ne te condamnera-t-il pas, toi qui la transgresses, tout en ayant la lettre de la loi et la circoncision ?
\VS{28}Le Juif, ce n’est pas celui qui en a les apparences\FTNT{Le formalisme (2 Ti. 3:5). L’apparence de la piété correspond aux vêtements des brebis : «~Gardez-vous des faux prophètes, ils viennent à vous en habits de brebis, mais au-dedans ce sont des loups ravisseurs.~» (Mt 7:15). «~Puis je vis une autre bête qui montait de la terre, et qui avait deux cornes semblables à celles de l'Agneau ; mais elle parlait comme le dragon.~» Ap. 13:11. Il a l’apparence d’un agneau, mais sa voix est celle du dragon, c’est-à-dire Satan.} ; et la circoncision, ce n’est pas celle qui est visible dans la chair.
\VS{29}Mais le Juif, c’est celui qui l’est intérieurement ; et la circoncision, c’est celle du cœur, selon l’Esprit et non selon la lettre. La louange de ce Juif ne vient pas des hommes, mais de Dieu.
\TextTitle{[L'avantage du Juif peut devenir une condamnation]}
\Chap{3}
\VerseOne{}Quel est donc l'avantage du Juif, ou quelle est l’utilité de la circoncision ?
\VS{2}Cet avantage est grand de toute manière, et tout d’abord en ce que les oracles de Dieu leur ont été confiés.
\VS{3}Eh quoi ! Si quelques-uns n'ont pas cru, leur incrédulité anéantira-t-elle la fidélité de Dieu ?
\VS{4}Nullement ! Que Dieu au contraire soit reconnu pour vrai, et tout homme pour menteur ; selon ce qui est écrit : Afin que tu sois trouvé juste dans tes paroles, et que tu triomphes lorsqu’on te juge\FTNT{Ps. 51:6.}.
\VS{5}Mais si notre injustice établit la justice de Dieu, que dirons-nous ? Dieu est-il injuste quand il déchaine sa colère ? (Je parle à la manière des hommes.)
\VS{6}Nullement ! Autrement, comment Dieu jugera-t-il le monde ?
\VS{7}Et si par mon mensonge la vérité de Dieu est plus abondante pour sa gloire, pourquoi suis-je encore condamné comme pécheur ?
\VS{8}Et pourquoi ne ferions-nous pas le mal, afin qu'il en arrive du bien, comme quelques-uns, qui nous calomnient, prétendent que nous le disons ? La condamnation de ces gens est juste.
\TextTitle{[Juifs et Grecs coupables devant Dieu]}
\VS{9}Quoi donc ! Sommes-nous plus excellents ? Nullement. Car nous avons déjà prouvé que tous, tant Juifs que Grecs, sont assujettis au péché.
\VS{10}Selon qu'il est écrit : Il n'y a point de juste, pas même un seul\FTNT{Ps. 14:3.}.
\VS{11}Il n'y a personne qui ait de l'intelligence, il n'y a personne qui recherche Dieu.
\VS{12}Ils se sont tous égarés, ils se sont tous corrompus : Il n'y en a aucun qui fasse le bien, pas même un seul.
\VS{13}Leur gosier est un sépulcre ouvert ; ils se servent de leur langue pour tromper ; il y a du venin d'aspic sous leurs lèvres.
\VS{14}Leur bouche est pleine de malédictions et d'amertume.
\VS{15}Leurs pieds sont légers pour répandre le sang.
\VS{16}La destruction et la misère sont sur leurs voies.
\VS{17}Et ils n'ont point connu la voie de la paix.
\VS{18}La crainte de Dieu n'est pas devant leurs yeux\FTNT{Ps. 14.}.
\VS{19}Or nous savons que tout ce que la loi dit, elle le dit à ceux qui sont sous la loi, afin que toute bouche soit fermée, et que tout le monde soit reconnu coupable devant Dieu.
\VS{20}C'est pourquoi personne ne sera justifié devant lui par les œuvres de la loi, puisque c’est par la loi que vient la connaissance du péché.
\TextTitle{[La justification par la foi]}
\VS{21}Mais maintenant, sans la loi, la justice de Dieu est manifestée, à laquelle rendent témoignage la loi et les prophètes.
\VS{22}La justice, dis-je, de Dieu par la foi en Jésus-Christ, envers tous et sur tous ceux qui croient. Car il n'y a point de distinction.
\VS{23}Car tous ont péché\FTNT{Le mot péché vient du terme grec «~hamartano~» : «~manquer la marque, manquer le chemin de la droiture et de l’honneur, s’éloigner de la loi de Dieu~». Le péché est la violation délibérée de la loi divine et l’absence de la droiture.} et sont entièrement privés de la gloire de Dieu.
\VS{24}Et ils sont gratuitement justifiés par sa grâce, par la rédemption\FTNT{La rédemption est la délivrance par le paiement d’un prix. Trois termes grecs sont utilisés pour parler de la rédemption : - Agorazo : acheter un objet au marché (agora signifiant marché). Les pécheurs sont considérés comme des esclaves vendus au marché (Ro. 7:14). - Exagorazo : acheter et amener un objet hors du marché (Ga. 3:13 ; Ga. 4:5). L’esclave acheté et amené hors du marché est définitivement délivré. - Lutroo : détacher, rendre libre (Lu. 24:21 ; Tit. 2:14 ; 1 P. 1:18.) Jésus-Christ nous a délivrés du péché, de la puissance de Satan et de la loi mosaïque. (Col. 1:12-14 ; Col. 2:14-17 ; 1 Jn. 3:5).} qui est en Jésus-Christ.
\VS{25}C’est lui que Dieu a destiné à être, par son sang, la victime propitiatoire\FTNT{Le terme propitiation vient du grec «~hilastérion~» qui signifie «~ce qui est expié, ce qui rend propice ou le don qui assure la propitiation~». C’est aussi le lieu où s’accomplit la propitiation (Hé. 9:5), c’est-à-dire le couvercle de l’arche. Lors du grand jour des expiations (Yom Kippour en hébreu), l’aspersion du sang était faite sur le propitiatoire (Lé. 16:14). Le Seigneur Jésus-Christ est notre victime expiatoire (1 Jn. 2:2 ; 1 Jn. 4:10).} pour ceux qui croiraient, afin de montrer sa justice, parce qu’il avait laissé impunis les péchés commis auparavant, au temps de sa patience.
\VS{26}Il montre, dis-je, sa justice dans le temps présent, de manière à être trouvé juste tout en justifiant celui qui a la foi en Jésus.
\VS{27}Où est donc le sujet de se glorifier ? Il est exclu. Par quelle loi ? Est-ce par la loi des œuvres ? Non, mais par la loi de la foi.
\VS{28}Nous concluons donc que l'homme est justifié par la foi, sans les œuvres de la loi.
\TextTitle{[Circoncis et incirconcis ont part à la justification]}
\VS{29}Dieu est-il seulement le Dieu des Juifs ? Ne l'est-il pas aussi des gentils ? Certes, il l'est aussi des gentils,
\VS{30}puisqu’il y a un seul Dieu qui justifiera par la foi les circoncis, et aussi les incirconcis par la foi.
\VS{31}Anéantissons-nous donc la loi par la foi ? Nullement ! Mais au contraire, nous affermissons la loi.
\TextTitle{[Abraham et David justifiés par la foi]
\\(cp. v. 18-25)}
\Chap{4}
\VerseOne{}Que dirons-nous donc, qu'Abraham, notre père, a obtenu selon la chair ?
\VS{2}Certes, si Abraham a été justifié par les œuvres, il a de quoi se glorifier, mais non pas envers Dieu.
\VS{3}Car que dit l'Ecriture ? Qu’Abraham a cru en Dieu, et que cela lui a été imputé à justice\FTNT{Ge. 15:6.}.
\VS{4}Or à celui qui fait les œuvres, le salaire ne lui est pas imputé comme une grâce, mais comme une chose due.
\VS{5}Mais à celui qui ne fait pas les œuvres, mais qui croit en celui qui justifie le méchant, sa foi lui est imputée à justice.
\VS{6}De même, David exprime le bonheur de l'homme à qui Dieu impute la justice sans les œuvres, en disant :
\VS{7}Heureux sont ceux à qui les iniquités sont pardonnées, et dont les péchés sont couverts.
\VS{8}Heureux l'homme à qui le Seigneur n’impute pas son péché\FTNT{Ps. 32:1-2.}.
\TextTitle{[La justice par la foi, avant la circoncision]}
\VS{9}Cette déclaration de bénédiction, est-elle seulement pour les circoncis, ou également pour les incirconcis ? Car nous disons que la foi a été imputée à Abraham à justice.
\VS{10}Comment donc lui a-t-elle été imputée ? Etait-ce après, ou avant sa circoncision ? Il n’était pas encore circoncis, il était incirconcis.
\VS{11}Et il reçut le signe de la circoncision comme sceau de la justice, qu’il avait obtenue par la foi, quand il était incirconcis, afin d’être le père de tous les incirconcis qui croient, pour que la justice leur soit aussi imputée ;
\VS{12}et le père des circoncis, qui ne sont pas seulement circoncis, mais encore qui marchent sur les traces de la foi de notre père Abraham, quand il était incirconcis.
\TextTitle{[La justification s'accomplit sans la loi]}
\VS{13}En effet, ce n’est pas par loi que la promesse d'être héritier du monde a été faite à Abraham, ou à sa postérité, mais par la justice de la foi.
\VS{14}Car, si les héritiers le sont par la loi, la promesse est annulée, et la foi est vaine
\VS{15}car la loi produit la colère ; car là où il n'y a point de loi, il n'y a point non plus de transgression.
\VS{16}C'est pourquoi les héritiers le sont par la foi, pour que ce soit par la grâce, et afin que la promesse soit assurée à toute la postérité ; non seulement à celle qui est de la loi, mais aussi à celle qui est de la foi d'Abraham, qui est le père de nous tous,
\VS{17}selon qu'il est écrit : Je t'ai établi père de plusieurs nations\FTNT{Ge 17:4-5.}. Il est notre père devant celui auquel il a cru, Dieu qui donne la vie aux morts, et qui appelle les choses qui ne sont point, comme si elles étaient.
\VS{18}Et Abraham ayant espéré contre toute espérance, crut qu'il deviendrait le père de plusieurs nations, selon ce qui lui avait été dit : Ainsi sera ta postérité.
\VS{19}Et sans faiblir dans la foi, il ne considéra point que son corps était déjà usé ; puisqu’il avait environ cent ans, et que Sara n’était plus en âge d'avoir des enfants.
\VS{20}Et il ne douta point de la promesse de Dieu par incrédulité, mais il fut fortifié par la foi, donnant gloire à Dieu,
\VS{21}étant pleinement persuadé que celui qui lui avait fait la promesse était aussi puissant pour l'accomplir.
\VS{22}C'est pourquoi cela lui fut imputé à justice.
\VS{23}Mais ce n’est pas à cause de lui seul qu’il est écrit que cela lui fut imputé à justice ;
\VS{24}c’est encore à cause de nous, à qui cela sera imputé, à nous, dis-je, qui croyons en celui qui a ressuscité des morts, Jésus notre Seigneur,
\VS{25}qui a été livré pour nos offenses, et est ressuscité pour notre justification.
\TextTitle{[Réconciliation avec Dieu par la justification]}
\Chap{5}
\VerseOne{}Etant donc justifiés\FTNT{La justification est l’œuvre de Dieu par laquelle la justice de Jésus est comptée en faveur du pécheur, de sorte que le pécheur est déclaré juste par Dieu (Ro. 4 : 3 ; Ro. 5:1-9 ; Ga. 2:16 ; Ga. 3:11). Cette justice n’est pas obtenue par les efforts de la personne sauvée. La justification est une action instantanée qui a pour résultat la vie éternelle. Elle repose totalement et exclusivement sur le sacrifice de Jésus à la croix (1 Pi. 2:24). Elle ne peut être reçue que par la foi en Jésus-Christ (Ep. 2:8-9). La justification est un acte d’imputation divine et non une reconnaissance personnelle de l’homme. Elle provient de la grâce (Ro. 3:24 ; Tit. 3:7).} par la foi, nous avons la paix avec Dieu, par notre Seigneur Jésus-Christ.
\VS{2}Par lequel aussi nous avons été amenés par la foi à cette grâce, dans laquelle nous tenons ferme ; et nous nous glorifions dans l'espérance de la gloire de Dieu.
\VS{3}Bien plus, nous nous glorifions même dans les afflictions ; sachant que l'affliction produit la persévérance ;
\VS{4}et la persévérance l'épreuve ; et l'épreuve l'espérance.
\VS{5}Or l'espérance ne trompe point, parce que l'amour de Dieu est répandu dans nos cœurs par le Saint-Esprit qui nous a été donné.
\VS{6}Car lorsque nous étions encore sans force, Christ est mort en son temps pour nous qui étions des impies.
\VS{7}A peine mourrait-on pour un juste ; quelqu’un peut-être mourrait pour un homme de bien.
\VS{8}Mais Dieu prouve son amour envers nous, en ce que lorsque nous étions encore pécheurs, Christ est mort pour nous.
\VS{9}Etant donc maintenant justifiés par son sang, à plus forte raison serons-nous sauvés par lui de la colère.
\VS{10}Car si, lorsque nous étions ennemis, nous avons été réconciliés avec Dieu par la mort de son Fils, à plus forte raison, étant réconciliés, serons-nous sauvés par sa vie.
\VS{11}Et non seulement cela, mais encore nous nous glorifions même en Dieu par notre Seigneur Jésus-Christ, par qui nous avons maintenant obtenu la réconciliation.
\TextTitle{[Parallèle entre l'œuvre de Jésus-Christ et celle d'Adam]}
\VS{12}C'est pourquoi comme par un seul homme le péché est entré dans le monde, et par le péché la mort, et qu’ainsi la mort s’est étendue sur tous les hommes, parce que tous ont péché…
\VS{13}Car jusqu'à la loi le péché était dans le monde ; or le péché n'est point imputé quand il n'y a point de loi.
\VS{14}Mais la mort a régné depuis Adam jusqu'à Moïse, même sur ceux qui n'avaient pas péché par une transgression semblable à celle d’Adam, lequel est la figure de celui qui devait venir.
\VS{15}Mais il n'en est pas du don gratuit comme de l'offense ; car si par l'offense d'un seul il en est beaucoup qui sont morts, à plus forte raison la grâce de Dieu, et le don de la grâce, venant d'un seul homme, à savoir de Jésus-Christ, ont-ils été abondamment répandus sur plusieurs.
\VS{16}Et il n'en est pas du don comme de ce qui est arrivé par un seul qui a péché ; car c’est après une seule offense que le jugement est devenu condamnation, mais le don gratuit devient justification après plusieurs offenses.
\VS{17}Si par l'offense d'un seul la mort a régné par lui seul, à plus forte raison ceux qui reçoivent l'abondance de la grâce, et du don de la justice, régneront-ils dans la vie par Jésus-Christ lui seul.
\VS{18}Ainsi donc, comme par une seule offense la condamnation est venue sur tous les hommes, de même par un acte de justice la justification qui donne la vie s’étend à tous les hommes.
\VS{19}Car, comme par la désobéissance d'un seul homme plusieurs ont été rendus pécheurs, de même par l'obéissance d'un seul plusieurs seront rendus justes.
\VS{20}Or la loi est intervenue afin que l'offense abonde, mais là où le péché a abondé, la grâce a surabondé,
\VS{21}afin que, comme le péché a régné par la mort, ainsi la grâce règne par la justice pour donner la vie éternelle, par Jésus-Christ notre Seigneur.
\TextTitle{[Le baptème en la mort et la résurrection de Jésus-Christ]}
\Chap{6}
\VerseOne{}Que dirons-nous donc ? Demeurerions-nous dans le péché, afin que la grâce abonde ?
\VS{2}A Dieu ne plaise ! Car nous qui sommes morts au péché, comment vivrions-nous encore dans le péché ?
\VS{3}Ignoriez-vous que nous tous qui avons été baptisés en Jésus-Christ, c’est en sa mort que nous avons été baptisés ?
\VS{4}Nous avons donc été ensevelis avec lui par le baptême en sa mort ; afin que comme Christ est ressuscité des morts par la gloire du Père, de même nous aussi nous marchions en nouveauté de vie.
\VS{5}Car, si nous sommes devenus une même plante avec lui par la conformité à sa mort, nous le serons aussi par la conformité à sa résurrection.
\VS{6}Sachant que notre vieil homme a été crucifié avec lui, afin que le corps du péché soit détruit, pour que nous ne soyons plus esclaves du péché.
\VS{7}Car celui qui est mort est libre du péché.
\VS{8}Or si nous sommes morts avec Christ, nous croyons que nous vivrons aussi avec lui,
\VS{9}sachant que Christ ressuscité des morts ne meurt plus, et que la mort n'a plus de pouvoir sur lui.
\VS{10}Car il est mort, et c’est pour le péché qu’il est mort une fois pour toutes ; il est revenu à la vie, et c’est pour Dieu qu’il est vivant.
\TextTitle{[Mourir au péché pour vivre selon Dieu]}
\VS{11}Ainsi vous-mêmes, considérez-vous comme morts au péché, et comme vivants pour Dieu en Jésus-Christ notre Seigneur.
\VS{12}Que le péché ne règne donc point dans votre corps mortel, et n’obéissez pas à ses convoitises.
\VS{13}Et ne livrez pas vos membres au péché comme des instruments d'iniquité ; mais donnez-vous vous-mêmes à Dieu comme de morts étant devenus vivants, et offrez vos membres à Dieu pour être des instruments de justice.
\VS{14}Car le péché n'aura pas de domination sur vous, parce que vous n'êtes point sous la loi, mais sous la grâce.
\VS{15}Quoi donc ? Pécherions-nous parce que nous ne sommes point sous la loi, mais sous la grâce ? A Dieu ne plaise !
\VS{16}Ne savez-vous pas qu’en vous livrant à quelqu’un comme esclaves pour lui obéir, vous êtes esclaves de celui à qui vous obéissez, soit du péché qui conduit à la mort, soit de l'obéissance qui conduit à la justice ?
\VS{17}Mais grâces à Dieu de ce qu'ayant été les esclaves du péché, vous avez obéi de cœur à la forme expresse de la doctrine dans laquelle vous avez été élevés.
\VS{18}Ayant donc été affranchis du péché, vous avez été asservis à la justice.
\VS{19}Je parle à la façon des hommes, à cause de l'infirmité de votre chair. Comme donc vous avez appliqué vos membres pour servir à la souillure et à l’iniquité, ainsi appliquez vos membres pour servir à la justice en sainteté.
\VS{20}Car lorsque vous étiez esclaves du péché, vous étiez libres à l'égard de la justice.
\VS{21}Quel fruit portiez-vous alors ? Des fruits dont vous avez honte maintenant. Car la fin de ces choses c’est la mort.
\VS{22}Mais maintenant que vous êtes affranchis du péché, et asservis à Dieu, vous avez pour fruit la sanctification, et pour fin la vie éternelle.
\VS{23}Car le salaire du péché, c'est la mort ; mais le don gratuit de Dieu, c'est la vie éternelle par Jésus-Christ notre Seigneur.
\TextTitle{[Le chrétien lié à Christ comme à un époux]}
\Chap{7}
\VerseOne{}Ignorez-vous, frères, car je parle à des gens qui connaissent la loi, que la loi exerce son pouvoir sur l’homme aussi longtemps qu’il vit ?
\VS{2}Car la femme qui est sous la puissance d'un mari, est liée à son mari par la loi tandis qu'il est en vie ; mais si son mari meurt, elle est délivrée de la loi du mari.
\VS{3}Si donc, du vivant de son mari, elle épouse un autre homme, elle sera appelée adultère ; mais si son mari meurt, elle est délivrée de la loi, de sorte qu'elle ne sera point adultère si elle épouse un autre homme.
\VS{4}Ainsi donc, vous aussi, mes frères, vous avez été, par le corps de Christ, mis à mort en ce qui concerne la loi, pour que vous apparteniez à un autre, à savoir, à celui qui est ressuscité des morts, afin que nous portions des fruits pour Dieu.
\VS{5}Car lorsque nous étions dans la chair, les passions des péchés excitées par la loi, agissaient dans nos membres de manière à produire des fruits pour la mort.
\VS{6}Mais maintenant nous sommes délivrés de la loi, étant morts à cette loi sous laquelle nous étions retenus ; afin que nous servions Dieu dans un esprit nouveau, et non selon la lettre qui a vieilli.
\TextTitle{[La loi a révélé le péché mais la délivrance vient par Jésus-Christ]}
\VS{7}Que dirons-nous donc ? La loi est-elle péché ? Nullement ! Au contraire, je n'ai connu le péché que par la loi ; car je n’aurais pas connu la convoitise, si la loi n’avait pas dit : Tu ne convoiteras point\FTNT{Ex. 20:17.}.
\VS{8}Et le péché, saisissant l’occasion, produisit en moi, par le commandement, toutes sortes de convoitises ; parce que sans la loi le péché est mort.
\VS{9}Pour moi, étant autrefois sans loi, je vivais. Mais quand le commandement vint, le péché reprit vie, et moi je mourus.
\VS{10}Ainsi, le commandement qui conduit à la vie se trouva pour moi conduire à la mort.
\VS{11}Car le péché, saisissant l’occasion, me séduisit par le commandement, et par lui me fit mourir.
\VS{12}La loi donc est sainte, et le commandement est saint, juste, et bon.
\VS{13}Ce qui est bon a-t-il donc été pour moi une cause de mort ? Nullement ! Mais c’est le péché, afin qu'il se manifeste comme péché, en me donnant la mort par ce qui est bon, et que par le commandement, il devienne condamnable au plus haut point.
\VS{14}Car nous savons, en effet, que la loi est spirituelle ; mais moi, je suis charnel, vendu au péché.
\TextTitle{[la connaissance du bien incapable de délivrer l'homme du péché}
\VS{15}Car je n'approuve pas ce que je fais, puisque je ne fais point ce que je veux, mais je fais ce que je hais.
\VS{16}Or si je fais ce que je ne veux pas, je reconnais par cela même que la loi est bonne.
\VS{17}Et maintenant donc ce n'est plus moi qui fais cela, mais c'est le péché qui habite en moi.
\VS{18}Ce qui est bon, je le sais, n’habite pas en moi, c’est-à-dire dans ma chair. J’ai la volonté, mais non le pouvoir de faire le bien.
\VS{19}Car je ne fais pas le bien que je veux, mais je fais le mal que je ne veux point.
\VS{20}Or si je fais ce que je ne veux point, ce n'est plus moi qui le fais, mais c'est le péché qui habite en moi.
\VS{21}Je trouve donc cette loi au-dedans de moi : Quand je veux faire le bien, le mal est attaché à moi.
\VS{22}Car je prends bien plaisir à la loi de Dieu quant à l'homme intérieur,
\VS{23}mais je vois dans mes membres une autre loi, qui combat contre la loi de mon entendement\FTNT{Entendement : Du grec «~nous~», c’est-à-dire l’esprit, l’intelligence, le bon sens, la raison.}, et qui me rend prisonnier à la loi du péché qui est dans mes membres.
\VS{24}Ah, misérable que je suis ! Qui me délivrera du corps de cette mort ?
\TextTitle{[Seul l'Esprit de Christ libère de la loi du péché]}
\VS{25}Je rends grâces à Dieu par Jésus-Christ notre Seigneur !… Ainsi donc, moi-même, je suis par l’entendement esclave de la loi de Dieu, et je suis par la chair esclave de la loi du péché.
\Chap{8}
\VerseOne{}Il n'y a donc maintenant aucune condamnation pour ceux qui sont en Jésus-Christ, qui marchent, non selon la chair, mais selon l'Esprit.
\VS{2}Parce que la loi de l'Esprit de vie qui est en Jésus-Christ m'a affranchi de la loi du péché et de la mort.
\VS{3}Car chose impossible à la loi, parce que la chair la rendait impuissante, Dieu a condamné le péché dans la chair, en envoyant, à cause du péché, son propre Fils dans une chair semblable à celle du péché.
\VS{4}Afin que la justice de la loi soit accomplie en nous, qui ne marchons point selon la chair, mais selon l'Esprit.
\TextTitle{[L'affection de l'Esprit opposée à celle de la chair]
\\(cp. Ga. 5:15-18)}
\VS{5}Car ceux, en effet, qui vivent selon la chair, s’affectionnent aux choses de la chair, tandis que ceux qui vivent selon l'Esprit, s’affectionnent aux choses de l'Esprit.
\VS{6}Or l'affection de la chair c’est la mort, tandis que l'affection de l'Esprit c’est la vie et la paix.
\VS{7}Car l'affection de la chair est inimitié contre Dieu, parce qu’elle ne se soumet pas à la loi de Dieu, et qu’elle ne le peut même pas.
\VS{8}C'est pourquoi ceux qui vivent selon la chair ne sauraient plaire à Dieu.
\VS{9}Pour vous, vous ne vivez pas selon la chair, mais selon l'Esprit, si du moins l'Esprit de Dieu habite en vous. Si quelqu'un n'a pas l'Esprit de Christ\FTNT{Notez que le Saint-Esprit est aussi appelé l’Esprit de Jésus (Ac. 16:7).}, il ne lui appartient pas.
\VS{10}Et si Christ est en vous, le corps est bien mort à cause du péché, mais l'Esprit est vie à cause de la justice.
\VS{11}Et si l'Esprit de celui qui a ressuscité Jésus d’entre les morts habite en vous, celui qui a ressuscité Christ d’entre les morts rendra aussi la vie à vos corps mortels par son Esprit qui habite en vous.
\VS{12}Ainsi donc, mes frères, nous ne sommes point redevables à la chair, pour vivre selon la chair.
\VS{13}Car si vous vivez selon la chair, vous mourrez ; mais si par l'Esprit vous faites mourir les actions du corps, vous vivrez.
\TextTitle{[L'Esprit d'adoption]
\\(Ga. 4:7)}
\VS{14}Car tous ceux qui sont conduits par l'Esprit de Dieu sont enfants de Dieu.
\VS{15}Et vous n'avez point reçu un esprit de servitude pour être encore dans la crainte ; mais vous avez reçu l'Esprit d'adoption, par lequel nous crions Abba, c'est-à-dire Père.
\VS{16}L’Esprit lui-même rend témoignage à notre esprit que nous sommes enfants de Dieu.
\VS{17}Et si nous sommes enfants, nous sommes aussi héritiers : Héritiers, dis-je, de Dieu, et cohéritiers de Christ ; si toutefois nous souffrons avec lui, afin d’être glorifiés avec lui.
\TextTitle{[La gloire à venir]
\\(cp. Ge. 3:18-19)}
\VS{18}Car tout bien compté, j'estime que les souffrances du temps présent ne sauraient être comparables à la gloire à venir qui doit être révélée pour nous.
\VS{19}Aussi, la création attend-elle avec un ardent désir la révélation des fils de Dieu.
\VS{20}Car la création a été soumise à la vanité, non de son gré, mais à cause de celui qui l’y a soumise ;
\VS{21}avec l’espérance qu’elle aussi sera affranchie de la servitude de la corruption, pour avoir part à la liberté de la gloire des enfants de Dieu.
\VS{22}Or, nous savons que jusqu’à ce jour, toutes les créatures soupirent et souffrent les douleurs de l’enfantement.
\VS{23}Et non seulement elles, mais nous aussi, qui avons les prémices de l'Esprit ; nous-mêmes, dis-je, soupirons en nous-mêmes, en attendant l'adoption, c'est-à-dire la rédemption de notre corps\FTNT{1 Co. 15:35-43 ; 1 Co. 15:51-54.}.
\VS{24}Car c’est en espérance que nous sommes sauvés. Or l’espérance qu’on voit n’est plus espérance : Ce qu’on voit, peut-on l’espérer encore ?
\VS{25}Mais si nous espérons ce que nous ne voyons pas, c'est que nous l'attendons avec patience.
\TextTitle{[L'Esprit intercède pour les saints]
\\(Hé. 7:25)}
\VS{26}De même aussi l’Esprit nous aide dans notre faiblesse, car nous ne savons pas ce qu’il nous convient de demander dans nos prières. Mais l’Esprit lui-même intercède par des soupirs inexprimables.
\VS{27}Et celui qui sonde les cœurs connaît quelle est la pensée de l'Esprit, car il intercède en faveur des saints, selon Dieu.
\TextTitle{[Le plan de Dieu s'accomplit par l'Evangile]}
\VS{28}Or nous savons aussi que toutes choses concourent au bien de ceux qui aiment Dieu, c'est-à-dire de ceux qui sont appelés selon son dessein.
\VS{29}Car ceux qu'il a connus d’avance, il les a aussi prédestinés à être semblables à l'image de son Fils, afin qu'il soit le premier-né de beaucoup de frères.
\VS{30}Et ceux qu'il a prédestinés, il les a aussi appelés ; et ceux qu'il a appelés, il les a aussi justifiés ; et ceux qu'il a justifiés, il les a aussi glorifiés.
\VS{31}Que dirons-nous donc à l’égard de ces choses ? Si Dieu est pour nous, qui sera contre nous ?
\VS{32}Lui qui n'a point épargné son propre Fils, mais qui l'a livré pour nous tous, comment ne nous donnera-t-il point aussi toutes choses avec lui ?
\VS{33}Qui accusera les élus de Dieu ? Dieu est celui qui justifie.
\VS{34}Qui les condamnera ? Christ est mort ; et bien plus, il est ressuscité, il est à la droite de Dieu, et il intercède pour nous.
\TextTitle{[L'amour de Christ résiste contre tout]}
\VS{35}Qui nous séparera de l'amour de Christ ? Sera-ce l'oppression, ou l'angoisse, ou la persécution, ou la famine, ou la nudité, ou le péril, ou l'épée ?
\VS{36}Ainsi qu'il est écrit : C’est à cause de toi que nous sommes livrés à la mort tous les jours, et qu’on nous regarde comme des brebis destinées à la boucherie\FTNT{Ps. 44:23.}.
\VS{37}Mais dans toutes ces choses nous sommes plus que vainqueurs par celui qui nous a aimés.
\VS{38}Car j’ai l’assurance que ni la mort, ni la vie, ni les anges, ni les principautés, ni les puissances, ni les choses présentes, ni les choses à venir,
\VS{39}ni la hauteur, ni la profondeur, ni aucune autre créature, ne pourra nous séparer de l'amour de Dieu manifesté en Jésus-Christ notre Seigneur.
\TextTitle{[Le chagrin de Paul pour Israël son peuple]}
\Chap{9}
\VerseOne{}Je dis la vérité en Christ, je ne mens point, ma conscience m’en rend témoignage par le Saint-Esprit :
\VS{2}J’éprouve une grande tristesse et un chagrin continuel dans mon cœur.
\VS{3}Car moi-même je souhaiterais être anathème et séparé de Christ pour mes frères, mes parents selon la chair,
\TextTitle{[Les enfants de la chair et ceux de la promesse]}
\VS{4}qui sont Israélites, à qui appartiennent l'adoption, la gloire, les alliances, l'ordonnance de la loi, le culte,
\VS{5}les promesses, les patriarches, et de qui est issu selon la chair Christ, qui est Dieu au-dessus de toutes choses, béni éternellement, Amen !
\VS{6}Toutefois il ne peut pas se faire que la parole de Dieu soit anéantie. Car tous ceux qui descendent d’Israël ne sont pas Israël.
\VS{7}Et bien qu’ils soient de la postérité d'Abraham, ils ne sont pas tous ses enfants, car il est dit : C'est en Isaac que tu auras une postérité appelée de ton nom ;
\VS{8}c'est-à-dire que ce ne sont pas ceux qui sont enfants de la chair qui sont enfants de Dieu, mais que ce sont les enfants de la promesse qui sont regardés comme la postérité.
\VS{9}Car voici la parole de la promesse : Je viendrai à cette même époque, et Sara aura un fils\FTNT{Ge. 18:10.}.
\VS{10}Et de plus, il en fut ainsi de Rébecca, qui conçut du seul Isaac notre père ;
\VS{11}car les enfants n’étaient pas encore nés et ils n’avaient fait ni bien ni mal, afin que le dessein arrêté selon l'élection de Dieu subsiste, sans dépendre des œuvres, mais par la volonté de celui qui appelle,
\VS{12}il lui fut dit : L’aîné sera assujetti au plus petit\FTNT{Ge. 25:23.}, selon qu’il est écrit :
\VS{13}J'ai aimé Jacob, et j'ai haï Esaü\FTNT{Mal. 1:2-3.}.
\TextTitle{[La volonté souveraine de Dieu]}
\VS{14}Que dirons-nous donc : Y a-t-il de l’injustice en Dieu ? A Dieu ne plaise !
\VS{15}Car il dit à Moïse : J'aurai compassion de celui de qui j’aurai compassion et je ferai miséricorde à celui à qui je ferai miséricorde\FTNT{Ex. 33:19.}.
\VS{16}Ainsi donc, cela ne vient pas de celui qui veut, ni de celui qui court, mais de Dieu qui fait miséricorde.
\VS{17}Car l'Ecriture dit à Pharaon : Je t'ai suscité dans le but de démontrer en toi ma puissance, et afin que mon Nom soit publié par toute la terre\FTNT{Ex. 9:16.}.
\VS{18}Ainsi, il fait miséricorde à qui il veut, et il endurcit qui il veut.
\VS{19}Tu me diras : Pourquoi se plaint-il encore ? Car qui est celui qui peut résister à sa volonté ?
\VS{20}Mais plutôt, ô homme, qui es-tu, toi qui contestes contre Dieu ? Le vase d’argile dira-t-il à celui qui l'a formé : Pourquoi m'as-tu ainsi fait ?
\VS{21}Le potier n'a-t-il pas le pouvoir de faire avec la même masse de terre un vase d’honneur et un vase d’un usage vil ?
\VS{22}Et que dire, si Dieu, en voulant montrer sa colère, et faire connaître sa puissance, a supporté avec une grande patience les vases de colère, préparés pour la perdition ?
\VS{23}Et s’il a voulu faire connaître les richesses de sa gloire envers les vases de miséricorde, qu'il a préparés d’avance pour la gloire ?
\VS{24}Ainsi il nous a appelés, non seulement d'entre les juifs, mais aussi d'entre les gentils,
\TextTitle{[Les prophéties concernant l'aveuglement d'Israël et la grâce sur les gentils]}
\VS{25}selon ce qu'il dit dans Osée : J'appellerai mon peuple celui qui n'était point mon peuple ; et la bien-aimée, celle qui n'était point la bien-aimée ;
\VS{26}et il arrivera qu'au lieu où il leur a été dit : Vous n’êtes pas mon peuple, là ils seront appelés les fils du Dieu vivant\FTNT{Os. 2:1.}.
\VS{27}Aussi Esaïe s'écrie au sujet d'Israël : Quand le nombre des enfants d'Israël serait comme le sable de la mer, un petit reste seulement sera sauvé.
\VS{28}Car le Seigneur exécutera pleinement et promptement sa parole sur la terre ce qu’il a résolu\FTNT{Es. 10:22-23.}.
\VS{29}Et comme Esaïe avait dit auparavant : Si le Seigneur des armées ne nous avait laissé une postérité, nous serions devenus comme Sodome, et nous aurions été semblables à Gomorrhe\FTNT{Es. 1:9.}.
\VS{30}Que dirons-nous donc ? Que les gentils, qui ne cherchaient pas la justice, ont obtenu la justice, la justice qui vient de la foi,
\VS{31}tandis qu’Israël qui cherchait la loi de la justice, n'est pas parvenu à cette loi.
\VS{32}Pourquoi ? Parce qu’Israël l’a cherchée non par la foi, mais comme provenant des œuvres de la loi. Ils se sont heurtés contre la pierre d'achoppement,
\VS{33}selon qu’il est écrit : Voici, je mets en Sion la pierre d'achoppement ; et un rocher de scandale, et quiconque croit en lui ne sera point confus\FTNT{Es. 28:16.}.
\TextTitle{[La foi, seule condition du salut]}
\Chap{10}
\VerseOne{}Mes frères, le souhait de mon cœur, et la prière que je fais à Dieu pour les Israélites, c'est qu'ils soient sauvés.
\VS{2}Car je leur rends témoignage qu'ils ont du zèle pour Dieu, mais sans connaissance.
\VS{3}Parce que ne connaissant point la justice de Dieu, et cherchant à établir leur propre justice, ils ne se sont point soumis à la justice de Dieu.
\VS{4}Car Christ est la fin de la loi\FTNT{Il est question de la loi cérémonielle relative au culte mosaïque. Avant sa mort, Jésus qui était né sous la loi (Ga. 4:4), demandait aux gens de l’appliquer. Ainsi, il demanda au lépreux qu’il avait guéri de présenter une offrande pour sa purification au temple (Mt. 8:1-4) et à ses disciples d'observer l'enseignement des scribes (Mt. 23:1-2). En effet, il fallait que les lois cérémonielles soient respectées jusqu’à sa résurrection. Une fois que Jésus eut dit «~tout est accompli~» (Jn. 19:30), toutes ces lois n’avaient plus aucune raison d’être (Col. 2:14-17 ; Hé. 7:11-22 ; Hé. 10:1-2).} pour la justification de tous ceux qui croient.
\VS{5}En effet, Moïse décrit ainsi la justice qui vient de la loi : L'homme qui fera ces choses vivra par elles\FTNT{Lé. 18:5.}.
\VS{6}Mais voici comment s'exprime la justice qui vient de la foi : Ne dis pas en ton cœur : Qui montera au ciel ? C’est en faire descendre Christ.
\VS{7}Ou : Qui descendra dans l'abîme ? C’est faire remonter Christ d’entre les morts.
\VS{8}Mais que dit-elle ? La parole est près de toi, dans ta bouche, et dans ton cœur. Or voilà la parole de foi que nous prêchons.
\VS{9}C'est pourquoi, si tu confesses de ta bouche le Seigneur Jésus, et si tu crois dans ton cœur que Dieu l'a ressuscité des morts, tu seras sauvé.
\VS{10}Car c’est en croyant du cœur qu’on parvient à la justice, et c’est en confessant de la bouche qu’on parvient au salut, selon ce que dit l’Ecriture :
\VS{11}Quiconque croit en lui ne sera point confus\FTNT{Es. 49:23.}.
\VS{12}Parce qu'il n'y a point de différence, en effet, entre le Juif et le Grec, puisqu’ils ont un même Seigneur, qui est riche pour tous ceux qui l'invoquent.
\VS{13}Car quiconque invoquera le nom du Seigneur sera sauvé\FTNT{Joë. 2:32.}.
\TextTitle{[La proclamation de l'Evangile dans les nations]}
\VS{14}Mais comment invoqueront-ils celui en qui ils n'ont point cru ? Et comment croiront-ils en celui dont ils n'ont point entendu parler ? Et comment en entendront-ils parler s'il n'y a personne qui leur prêche ?
\VS{15}Et comment y aura-t-il des prédicateurs, s’ils ne sont pas envoyés ? Selon qu'il est écrit : Qu’ils sont beaux les pieds de ceux qui annoncent la paix, de ceux qui annoncent de bonnes nouvelles\FTNT{Es. 52:7.} !
\VS{16}Mais tous n'ont pas obéi à l'Evangile ; car Esaïe dit : Seigneur, qui a cru à notre prédication\FTNT{Es. 53:1.} ?
\VS{17}Ainsi la foi vient de ce qu’on entend, et ce qu’on entend vient de la parole de Christ.
\VS{18}Mais je dis : Ne l'ont-ils point entendue ? Au contraire, leur voix est allée par toute la terre, et leur parole jusqu’aux extrémités du monde.
\VS{19}Mais je dis : Israël ne l'a-t-il point su ? Moïse le premier dit : J’exciterai votre jalousie par ce qui n'est point une nation, je provoquerai votre colère par une nation sans intelligence\FTNT{De. 32:21.}.
\VS{20}Et Esaïe pousse la hardiesse jusqu’à dire : J'ai été trouvé par ceux qui ne me cherchaient point, et je me suis clairement manifesté à ceux qui ne me demandaient pas\FTNT{Es. 65:1.}.
\VS{21}Mais au sujet d’Israël, il dit : J'ai tout le jour tendu mes mains vers un peuple rebelle et contredisant\FTNT{Es. 65:2.}.
\TextTitle{[Un reste d'Israël participe à la grâce]}
\Chap{11}
\VerseOne{}Je dis donc : Dieu a-t-il rejeté son peuple ? A Dieu ne plaise ! Car je suis aussi Israélite, de la postérité d'Abraham, de la tribu de Benjamin.
\VS{2}Dieu n'a point rejeté son peuple, qu’il a connu d’avance. Et ne savez-vous pas ce que l'Ecriture dit d'Elie, comment il a fait requête à Dieu contre Israël, disant :
\VS{3}Seigneur, ils ont tué tes prophètes, et ils ont démoli tes autels, et je suis resté moi seul ; et ils cherchent à m'ôter la vie\FTNT{1 R. 19:10.}.
\VS{4}Mais quelle réponse Dieu lui donna-t-il ? Je me suis réservé sept mille hommes, qui n'ont point fléchi le genou devant Baal\FTNT{1 R.19:18.}.
\VS{5}De même aussi dans le temps présent, il y a un reste selon l'élection de la grâce.
\VS{6}Or si c'est par la grâce, ce n'est plus par les œuvres ; autrement la grâce n'est plus la grâce. Mais si c'est par les œuvres, ce n'est plus par une grâce ; autrement l’œuvre n'est plus une œuvre.
\TextTitle{[La nation d'Israël est temporairement mise à l'écart mais non rejetée]}
\VS{7}Quoi donc ? Ce qu'Israël cherche, il ne l'a point obtenu ; mais les élus l’ont obtenu, tandis que les autres ont été endurcis,
\VS{8}selon qu'il est écrit : Dieu leur a donné un esprit d’assoupissement, des yeux pour ne point voir, et des oreilles pour ne point entendre\FTNT{Es. 29:10.}, jusqu’à ce jour. Et David dit :
\VS{9}Que leur table soit pour eux un filet, un piège, une occasion de chute, et cela pour leur récompense.
\VS{10}Que leurs yeux soient obscurcis pour ne point voir\FTNT{Ps. 69:23-24.} ; et tiens continuellement leur dos courbé !
\VS{11}Mais je dis : Est-ce pour tomber qu’ils ont bronché ? Nullement ! Mais par leur chute, le salut est accordé aux Gentils, afin qu’ils soient excités à la jalousie.
\VS{12}Or si leur chute est la richesse du monde, et leur amoindrissement la richesse des Gentils, combien plus en sera-t-il quand ils se convertiront tous ?
\TextTitle{[Avertissement aux gentils]}
\VS{13}Car je vous parle à vous, Gentils, en tant qu’apôtre des Gentils, je glorifie mon ministère,
\VS{14}afin, s’il est possible, d’exciter la jalousie de ceux de ma race et d’en sauver quelques-uns.
\VS{15}Car si leur mise à l’écart a été la réconciliation du monde, quelle sera leur réintégration, sinon le passage de la mort à la vie ?
\VS{16}Or si les prémices sont saintes, la masse l'est aussi ; et si la racine est sainte, les branches le sont aussi.
\VS{17}Mais si quelques-unes des branches ont été retranchées, et si toi qui étais un olivier sauvage, tu as été greffé à leur place et rendu participant de la racine et de la graisse de l'olivier,
\VS{18}ne te glorifie pas contre ces branches ; car si tu te glorifies, ce n'est pas toi qui portes la racine, mais c'est la racine qui te porte.
\VS{19}Mais tu diras : Les branches ont été retranchées, afin que moi je sois greffé.
\VS{20}Cela est vrai, elles ont été retranchées à cause de leur incrédulité, et tu es debout par la foi ; ne t'élève donc point par orgueil, mais crains.
\VS{21}Car si Dieu n'a point épargné les branches naturelles, prends garde qu'il ne t'épargne pas non plus.
\VS{22}Considère donc la bonté et la sévérité de Dieu ; la sévérité envers ceux qui sont tombés ; et la bonté envers toi, si tu persévères dans cette bonté : Car autrement tu seras aussi retranché.
\VS{23}Eux de même, s'ils ne persistent pas dans leur incrédulité, ils seront greffés ; car Dieu est puissant pour les greffer de nouveau.
\VS{24}Car si toi tu as été coupé de l'olivier sauvage selon sa nature, et greffé contrairement à ta nature sur l'olivier franc, à plus forte raison eux seront-ils greffés selon leur nature sur leur propre olivier.
\VS{25}Car mes frères, je ne veux pas que vous ignoriez ce mystère, afin que vous ne vous regardiez point comme sages : Une partie d’Israël est tombée dans l’endurcissement, jusqu’à ce que la totalité des gentils soit entrée.
\TextTitle{[Yahweh prédit le salut futur d'Israël]
\\(Es. 66:8)}
\VS{26}Et ainsi tout Israël sera sauvé, selon qu’il est écrit : Le Libérateur viendra de Sion, et il détournera de Jacob les infidélités ;
\VS{27}et c'est là l'alliance que je ferai avec eux, lorsque j'ôterai leurs péchés\FTNT{Es. 59:20-21.}.
\VS{28}Ils sont certes ennemis par rapport à l'Evangile, à cause de vous ; mais en ce qui concerne l’élection, ils sont aimés à cause de leurs pères.
\VS{29}Car Dieu ne se repent pas de ses dons et de sa vocation.
\VS{30}De même que vous avez autrefois désobéi à Dieu et que par leur désobéissance vous avez maintenant obtenu miséricorde,
\VS{31}de même ils ont maintenant désobéi, afin que par la miséricorde qui vous a été faite, ils obtiennent aussi miséricorde.
\VS{32}Car Dieu les a tous renfermés sous la rébellion afin de faire miséricorde à tous.
\TextTitle{[Les voies incompréhensibles de Dieu]}
\VS{33}Ô profondeur de la richesse, de la sagesse et de la connaissance de Dieu ! Que ses jugements sont insondables et ses voies incompréhensibles !
\VS{34}Car qui a connu la pensée du Seigneur ? Ou qui a été son conseiller ?
\VS{35}Qui lui a donné le premier, pour qu’il ait à recevoir en retour ?
\VS{36}Car c’est de lui, par lui, et pour lui que sont toutes choses. A lui soit la gloire éternellement. Amen !
\TextTitle{[Le culte raisonnable]}
\Chap{12}
\VerseOne{}Je vous exhorte donc, mes frères, par les compassions de Dieu, à offrir vos corps comme un sacrifice vivant, saint, agréable à Dieu, ce qui est votre culte raisonnable.
\VS{2}Et ne vous conformez pas au siècle présent, mais soyez transformés\FTNT{Le verbe «~transformer~» est la traduction du terme grec «~metamorphoo~» qui a donné en français «~transfigurer~». C’est le même terme qui a été utilisé en Mt. 17:2 pour parler de la transfiguration du Seigneur. Si Paul recommandait cela à des personnes déjà converties, c’est parce que Dieu les appelait à aller plus loin. La transformation d’une chenille en papillon est un très bel exemple pour illustrer le changement radical qui doit s’opérer en nous. Pour atteindre ce stade, cet insecte passe par plusieurs étapes. La transformation nous permet de croître spirituellement. En effet, tout enfant de Dieu est appelé à devenir mature, à passer du stade de petit enfant à celui de jeune homme, et de celui de jeune homme à celui de père (1 Jn. 2:12-14).} par le renouvellement de votre entendement, afin que vous discerniez quelle est la volonté de Dieu, ce qui est bon, agréable et parfait.
\TextTitle{[Exhortation à l'humilité et au service selon les dons de l'Esprit]}
\VS{3}Par la grâce qui m’a été donnée, je dis à chacun de vous que nul ne présume d'être plus sage qu'il ne faut, mais d’avoir des sentiments modestes, selon la mesure de foi que Dieu a départie à chacun.
\VS{4}Car comme nous avons plusieurs membres dans un seul corps, et que tous les membres n'ont pas la même fonction,
\VS{5}ainsi, nous qui sommes plusieurs, nous formons un seul corps en Christ, et nous sommes tous membres les uns des autres.
\VS{6}Puisque nous avons des dons différents, selon la grâce qui nous est donnée, que celui qui a le don de prophétie l’exerce en analogie de la foi ;
\VS{7}que celui qui est appelé au ministère, s’attache à son ministère ; que celui qui enseigne s’attache à son enseignement,
\VS{8}et celui qui exhorte, à l’exhortation ; que celui qui donne, le fasse avec simplicité ; que celui qui préside, le fasse avec zèle ; que celui qui exerce la miséricorde, le fasse avec joie.
\TextTitle{[Les relations mutuelles entre chrétien]}
\VS{9}Que la charité soit sincère. Ayez en horreur le mal, attachez-vous fortement au bien.
\VS{10}Par charité fraternelle, soyez pleins d’affection les uns pour les autres ; par honneur, usez de prévenances réciproques.
\VS{11}Ne soyez point paresseux à vous employer pour autrui. Soyez fervents d'esprit. Servez le Seigneur.
\VS{12}Soyez joyeux dans l'espérance. Soyez patients dans la tribulation. Persévérez dans la prière.
\VS{13}Pourvoyez aux besoins des saints. Exercez l'hospitalité.
\VS{14}Bénissez ceux qui vous persécutent ; bénissez-les, et ne les maudissez point.
\VS{15}Réjouissez-vous avec ceux qui se réjouissent. Pleurez avec ceux qui pleurent.
\VS{16}Ayez les mêmes sentiments les uns envers les autres. N’aspirez pas à ce qui est élevé, mais laissez-vous attirer par ce qui est humble. Ne soyez point sages à votre propre jugement.
\TextTitle{[Les relations du chrétiens avec ceux du dehors]}
\VS{17}Ne rendez à personne le mal pour le mal. Recherchez les choses honnêtes devant tous les hommes.
\VS{18}S’il est possible, autant que cela dépend de vous, soyez en paix avec tous les hommes.
\VS{19}Ne vous vengez point vous-mêmes, mes bien-aimés, mais laissez agir la colère de Dieu, car il est écrit : A moi appartient la vengeance, à moi la rétribution, dit le Seigneur\FTNT{De. 32:35.}.
\VS{20}Si donc ton ennemi a faim, donne-lui à manger ; s'il a soif, donne-lui à boire, car en faisant cela, tu amasseras des charbons ardents sur sa tête.
\VS{21}Ne te laisse pas vaincre par le mal, mais surmonte le mal par le bien.
\TextTitle{[Le chrétien et les autorités]}
\Chap{13}
\VerseOne{}Que toute personne soit soumise aux autorités supérieures, car il n'y a point d’autorité qui ne vienne pas de Dieu, et les autorités qui existent ont été instituées de Dieu.
\VS{2}C'est pourquoi celui qui s’oppose à l’autorité résiste à l’ordre de Dieu ; et ceux qui y résistent attireront la condamnation sur eux-mêmes.
\VS{3}Car ce n’est pas pour une bonne action, c’est pour une mauvaise que les magistrats sont à craindre. Veux-tu ne pas craindre l’autorité ? Fais le bien, et tu auras sa louange.
\VS{4}Car le magistrat est un serviteur de Dieu pour ton bien. Mais si tu fais le mal, crains, car ce n’est pas en vain qu’il porte l'épée, étant serviteur de Dieu, ordonné pour faire justice en punissant celui qui fait le mal.
\VS{5}C'est pourquoi il faut être soumis, non seulement à cause de la punition, mais aussi à cause de la conscience.
\VS{6}Car c'est aussi pour cela que vous payez les impôts, parce que les magistrats sont les ministres de Dieu, s'employant à rendre la justice.
\VS{7}Rendez donc à tous ce qui leur est dû : L’impôt à qui vous devez l’impôt, le tribut à qui vous devez le tribut, le péage à qui vous devez le péage, la crainte à qui vous devez la crainte, l’honneur à qui vous devez l'honneur.
\TextTitle{[L'amour de son prochain : accomplissement de la loi]
\\(cp. Lu. 10:29-37)}
\VS{8}Ne devez rien à personne, si ce n’est de vous aimer les uns les autres ; car celui qui aime les autres a accompli la loi.
\VS{9}En effet, les commandements : Tu ne commettras point d’adultère, tu ne tueras point, tu ne déroberas point, tu ne convoiteras point, et ceux qu’il peut encore y avoir, se résument dans cette parole : Tu aimeras ton prochain comme toi-même\FTNT{Ex. 20:12-17 ; Mt. 22:39.}.
\VS{10}La charité ne fait point de mal au prochain ; la charité est donc l'accomplissement de la loi.
\VS{11}Cela importe d’autant plus que vous savez en quelle saison nous sommes ; parce qu'il est déjà l’heure de nous réveiller du sommeil ; car maintenant le salut est plus près de nous que lorsque nous avons cru.
\VS{12}La nuit est avancée\FTNT{Mt. 25:1-13.} et le jour approche. Rejetons donc les œuvres des ténèbres, et soyons revêtus des armes de lumière.
\VS{13}Marchons honnêtement, comme en plein jour, loin des orgies\FTNT{Orgies : Du grec «~komos~». Ce terme désigne la procession nocturne et rituelle, qui avait lieu après un souper, de gens à moitié ivres, à l'esprit folâtre, qui défilaient à travers les rues avec torches et musique en l'honneur de Bacchus ou quelque autre divinité, et chantaient et jouaient devant les maisons de leurs amis, hommes ou femmes. Ce mot est aussi utilisé pour les fêtes et beuveries de nuit qui se terminaient en orgies.} et de l’ivrognerie, de la luxure et de la débauche, des querelles et des jalousies.
\VS{14}Mais revêtez-vous du Seigneur Jésus-Christ, et n'ayez point soin de la chair pour en satisfaire les convoitises.
\TextTitle{[L'attitude du chrétien face aux opinions différentes]
\\(cp. 1 Co. 8:1-10:33}
\Chap{14}
\VerseOne{}Or quant à celui qui est faible dans la foi, recevez-le, et n'ayez point avec lui des discussions sur les opinions.
\VS{2}L'un croit qu'on peut manger de tout, et l'autre, qui est faible, mange des légumes.
\VS{3}Que celui qui mange de tout ne méprise pas celui qui n'en mange point ; et que celui qui n'en mange point, ne juge point celui qui en mange, car Dieu l'a accueilli.
\VS{4}Qui es-tu, toi qui juges le serviteur d'autrui ? S’il se tient ferme ou s'il tombe, c’est à son maître de le juger ; mais il sera affermi, car Dieu est Puissant pour l'affermir.
\VS{5}Tel fait une distinction entre les jours, tel autre les estime tous égaux. Que chacun ait en son esprit une pleine conviction.
\VS{6}Celui qui distingue entre les jours agit ainsi pour le Seigneur. Celui qui mange, c’est pour le Seigneur qu’il mange, car il rend grâces à Dieu ; celui qui ne mange pas, c’est pour le Seigneur qu’il ne mange pas, et il rend grâces à Dieu.
\VS{7}Car nul de nous ne vit pour lui-même, et nul ne meurt pour lui-même.
\VS{8}Car si nous vivons, nous vivons pour le Seigneur ; et si nous mourons, nous mourons pour le Seigneur. Soit donc que nous vivions, soit que nous mourions, nous sommes au Seigneur.
\VS{9}Car c'est pour cela que Christ est mort, qu'il est ressuscité, et qu'il a repris la vie, afin de dominer sur les morts et sur les vivants.
\VS{10}Mais toi, pourquoi juges-tu ton frère ? Ou toi, pourquoi méprises-tu ton frère ? Puisque nous comparaîtrons tous devant le tribunal de Christ.
\VS{11}Car il est écrit : Je suis vivant, dit le Seigneur, tout genou fléchira devant moi, et toute langue donnera gloire à Dieu\FTNT{Es. 45:23 ; Ph. 2:10-11.}.
\VS{12} Ainsi, chacun de nous rendra compte à Dieu pour lui-même.
\TextTitle{[Se garder d'être une occasion de chute]}
\VS{13}Ne nous jugeons donc plus les uns les autres ; mais pensez plutôt à ne rien faire qui soit pour votre frère une pierre d’achoppement ou une occasion de chute.
\VS{14}Je sais, et je suis persuadé par le Seigneur Jésus, que rien n'est souillé en soi, et qu’une chose n’est souillée que par celui qui la croit souillée.
\VS{15}Mais si ton frère est attristé au sujet d’un aliment, tu ne marches plus selon la charité ; ne détruis point, par ton aliment, celui pour qui Christ est mort.
\VS{16}Que votre privilège ne soit pas un sujet de calomnie.
\VS{17}Car le Royaume de Dieu ne consiste ni dans le manger ni dans le boire, mais dans la justice, la paix et la joie par le Saint-Esprit.
\VS{18}Celui qui sert Christ de cette manière est agréable à Dieu et approuvé des hommes.
\VS{19}Recherchons donc ce qui contribue à la paix et à l’édification mutuelle.
\VS{20}Ne détruis pas l’œuvre de Dieu pour un aliment. Il est vrai que toutes choses sont pures, mais il est mal à l’homme, quand il mange, de devenir une pierre d’achoppement.
\VS{21}Il est bien de ne pas manger de viande, de ne pas boire de vin, et de s’abstenir de ce qui peut être pour ton frère une occasion de chute, de scandale ou de faiblesse.
\VS{22}As-tu la foi ? Garde-la devant Dieu. Heureux est celui qui ne se condamne pas lui-même dans ce qu'il approuve.
\VS{23}Mais celui qui a des doutes au sujet de ce qu’il mange est condamné, parce qu’il n’agit pas avec foi. Tout ce que l’on ne fait pas avec foi est un péché.
\Chap{15}
\VerseOne{}Nous devons, nous qui sommes forts, supporter les infirmités des faibles, et ne pas nous complaire en nous-mêmes.
\VS{2}Que chacun de nous plaise au prochain pour ce qui est bien, en vue de l’édification.
\VS{3}Car même Jésus-Christ n'a pas cherché ce qui lui plaisait, mais, selon qu’il est écrit : Les outrages de ceux qui t’insultent sont tombés sur moi\FTNT{Ps. 69:10.}.
\TextTitle{[Les Juifs et gentils rachetés par un même salut]}
\VS{4}Or tout ce qui a été écrit autrefois, a été écrit pour notre instruction, afin que par la patience et la consolation que donnent les Ecritures, nous possédions l’espérance.
\VS{5}Que le Dieu de patience et de consolation vous donne d’avoir les mêmes sentiments les uns envers les autres, selon Jésus-Christ,
\VS{6}afin que tous d'un même cœur et d'une même bouche, vous glorifiiez Dieu, qui est le Père de notre Seigneur Jésus-Christ.
\VS{7}C'est pourquoi, accueillez-vous les uns les autres, comme Christ nous a accueillis, pour la gloire de Dieu.
\VS{8}Je dis donc que Jésus-Christ a été Ministre des circoncis, pour prouver la vérité de Dieu, afin de confirmer les promesses faites aux pères,
\VS{9}afin que les gentils glorifient Dieu pour sa miséricorde, selon ce qui est écrit : C’est pourquoi je te louerai parmi les nations, et je chanterai à la gloire de ton Nom\FTNT{Ps. 18:50.}. Et il est dit encore :
\VS{10}Nations, réjouissez-vous avec son peuple\FTNT{De. 32:43.} !
\VS{11}Et encore : Louez le Seigneur, vous toutes les nations, et célébrez-le, vous tous les peuples\FTNT{Ps. 117:1.}. Esaïe dit aussi :
\VS{12}Il sortira d’Isaï un rejeton, qui se lèvera pour régner sur les nations ; les nations espéreront en lui\FTNT{Es. 11:1 ; Es. 11:10.}.
\VS{13}Que le Dieu de l’espérance vous remplisse de toute joie et de toute paix, dans la foi, afin que vous abondiez en espérance par la puissance du Saint-Esprit.
\TextTitle{[Paul envisage d'aller à Jérusalem, à Rome et en Espagne]}
\VS{14}Pour moi, mes frères, je suis persuadé que vous êtes pleins de bonté, remplis de toute connaissance, et capables de vous exhorter les uns les autres.
\VS{15}Cependant, mes frères, je vous ai écrit en quelque sorte plus librement, comme pour réveiller vos souvenirs, à cause de la grâce que Dieu m’a faite,
\VS{16}d’être ministre de Jésus Christ parmi les gentils ; je m’acquitte du divin service de l'Evangile de Dieu, afin que les gentils lui soient une offrande agréable, étant sanctifiée par le Saint-Esprit.
\VS{17}J'ai donc sujet de me glorifier en Jésus-Christ pour ce qui regarde les choses de Dieu.
\VS{18}Car je n’oserais parler de quoi que ce soit que Christ n’ait opéré par moi, pour amener les gentils à son obéissance, par la parole et par les œuvres,
\VS{19}par la puissance des prodiges et des miracles, par la puissance de l'Esprit de Dieu. Ainsi, depuis Jérusalem et les pays voisins jusqu’en Illyrie, j’ai abondamment répandu l’Evangile de Christ.
\VS{20}M'attachant ainsi avec affection à annoncer l’Evangile là où Christ n’avait point encore été prêché, afin que je ne bâtisse pas sur le fondement qu'un autre a déjà posé. 
\VS{21}Mais selon qu'il est écrit : Ceux à qui il n'a point été annoncé le verront ; et ceux qui n'en avaient point entendu parler l’entendront\FTNT{Es. 52:15.}.
\VS{22}Et c’est ce qui m'a souvent empêché d’aller vous voir.
\VS{23}Mais maintenant, n’ayant plus rien qui me retienne dans ces contrées, et ayant depuis plusieurs années le désir d’aller vers vous,
\VS{24}j’espère vous voir en passant, quand je me rendrai en Espagne, et y être accompagné par vous, après que j’aurai satisfait en partie mon désir de me trouver chez vous.
\VS{25}Maintenant je vais à Jérusalem pour assister les saints.
\VS{26}Car il a semblé bon à ceux de Macédoine et d’Achaïe de s’imposer une contribution pour les pauvres parmi les saints de Jérusalem.
\VS{27}Ils l’ont bien voulu, et ils le leur devaient, car si les gentils ont eu part à leurs avantages spirituels, ils doivent aussi les assister dans les choses temporelles.
\VS{28}Dès que j'aurai achevé cette affaire, et que je leur aurai remis ce fruit, j'irai en Espagne en passant par vos quartiers.
\VS{29}Et je sais qu’en allant vers vous, j’irai avec une pleine bénédiction de l'Evangile de Christ.
\VS{30}Je vous exhorte, mes frères, par notre Seigneur Jésus-Christ, et par la charité de l'Esprit, à combattre avec moi en adressant des prières à Dieu en ma faveur,
\VS{31}afin que je sois délivré des incrédules de Judée, et que mon ministère\FTNT{Ministère : Du grec «~diakonia~», terme qui désigne le service ou le ministère de ceux qui répondent aux besoins des autres. Ce vocable fait aussi allusion à l’office des diacres.} à Jérusalem soit agréable aux saints,
\VS{32}en sorte que, par la volonté de Dieu, j’arrive chez vous avec joie, et que je me repose avec vous.
\VS{33}Que le Dieu de paix soit avec vous tous. Amen !
\TextTitle{[Salutations personnelles de Paul]}
\Chap{16}
\VerseOne{}Je vous recommande notre sœur Phœbé, qui est diaconesse de l'église de Cenchrées,
\VS{2}afin que vous la receviez selon le Seigneur, comme il faut recevoir les saints, et que vous l'assistiez dans tout ce dont elle aura besoin ; car elle a exercé l'hospitalité à l'égard de plusieurs, et même à mon égard.
\VS{3}Saluez Priscille et Aquilas, mes compagnons d’œuvre en Jésus-Christ,
\VS{4}qui ont exposé leur cou pour ma vie ; ce n’est pas moi seul qui leur rends grâces, mais aussi toutes les églises des gentils.
\VS{5}Saluez aussi l'église qui est dans leur maison. Saluez Epaïnète, mon bien-aimé, qui a été pour Christ les prémices d'Achaïe.
\VS{6}Saluez Marie, qui a beaucoup travaillé pour nous.
\VS{7}Saluez Andronicus et Junias, mes parents, qui ont été prisonniers avec moi, et qui sont distingués parmi les apôtres, et qui ont même été en Christ avant moi.
\VS{8}Saluez Amplias, mon bien-aimé dans le Seigneur.
\VS{9}Saluez Urbain, notre compagnon d’œuvre en Christ, et Stachys, mon bien-aimé.
\VS{10}Saluez Apellès, qui est éprouvé en Christ. Saluez ceux de chez Aristobule.
\VS{11}Saluez Hérodion, mon parent. Saluez ceux de chez Narcisse qui sont dans le Seigneur.
\VS{12}Saluez Tryphène et Tryphose, qui travaillent pour le Seigneur. Saluez Perside, la bien-aimée qui a beaucoup travaillé pour le Seigneur.
\VS{13}Saluez Rufus, l’élu du Seigneur, et sa mère, qui est aussi la mienne.
\VS{14}Saluez Asyncrite, Phlégon, Hermas, Patrobas, Hermès, et les frères qui sont avec eux.
\VS{15}Saluez Philologue et Julie, Nérée et sa sœur, et Olympe, et tous les saints qui sont avec eux.
\VS{16}Saluez-vous les uns les autres par un saint baiser. Les églises de Christ vous saluent.
\TextTitle{[Se garder de ceux qui causent des divisions et des scandales]}
\VS{17}Je vous exhorte, mes frères, à prendre garde à ceux qui causent des divisions et des scandales contre la doctrine que vous avez apprise. Eloignez-vous d'eux.
\VS{18}Car de tels hommes ne servent point notre Seigneur Jésus-Christ, mais leur propre ventre, et par des paroles douces et flatteuses, ils séduisent les cœurs des simples.
\VS{19}Pour vous, votre obéissance est connue de tous ; je me réjouis donc à votre sujet, et je désire que vous soyez sages à l’égard du bien, et purs à l’égard du mal.
\VS{20}Le Dieu de paix brisera bientôt Satan sous vos pieds. Que la grâce de notre Seigneur Jésus-Christ soit avec vous. Amen !
\VS{21}Timothée, mon compagnon d’œuvre, vous salue, ainsi que Lucius, et Jason et Sosipater, mes parents.
\VS{22}Je vous salue dans le Seigneur, moi Tertius, qui ai écrit cette lettre.
\VS{23}Gaïus, mon hôte, et celui de toute l'église, vous salue. Eraste, l’économe de la ville, vous salue, et Quartus, notre frère.
\TextTitle{[Bénédiction]}
\VS{24}Que la grâce de notre Seigneur Jésus-Christ soit avec vous tous. Amen !
\VS{25}or à celui qui est puissant pour vous affermir selon mon Evangile, et selon la prédication de Jésus-Christ, conformément à la révélation du mystère qui a été caché dans les temps passés.
\VS{26}mais manifesté maintenant par les écrits des prophètes, d’après l’ordre du Dieu éternel, et porté à la connaissance de toutes les nations, afin qu’elles obéissent à la foi.
\VS{27}A Dieu, seul sage, soit la gloire éternellement, par Jésus-Christ. Amen !
\PPE{}
\end{multicols}
