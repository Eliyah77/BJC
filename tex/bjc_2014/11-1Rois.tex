\ShortTitle{1 Rois}\BookTitle{1 Rois}\BFont
\noindent\hrulefill
{\footnotesize
\textit{
\bigskip
{\centering{}
\\(Melakhim)
\\Signifie : Roi, Règne
\\Thème : Unité du royaume après le schisme
\\Auteur : Inconnu
\\Date de rédaction : 6ème siècle av. J.-C.\\}
}
%\bigskip
\textit{
\\Ce livre relate la vie de Salomon : son accès à la royauté suite à la mort de son père David, son alliance avec Dieu qui lui accorda une sagesse exceptionnelle ainsi que la construction du Temple de Dieu et du palais royal.
%\bigskip
\\Les premières années du règne de Salomon furent exemplaires. Malheureusement, il ne fit pas preuve de la même piété que son père et développa une affection particulière pour les femmes étrangères, ce qui le plongea dans l’idolâtrie. A sa mort, son fils Roboam accéda au pouvoir et provoqua la division du royaume en deux : d’un côté les dix tribus du nord gardant le nom d’Israël, et de l’autre côté les deux tribus du sud, Juda et Benjamin sur qui il régna.
%\bigskip
\\Ce livre raconte également le règne et la conduite parfois abominable des rois d’Israël et Juda jusqu’à Achab et Josaphat. Il présente la puissance de l’appel prophétique d’Elie, le Thischbite que Dieu suscita pour ramener son peuple à lui et montrer sa souveraineté.\bigskip
}
}
\par\nobreak\noindent\hrulefill
\begin{multicols}{2}
\Chap{1}
\TextTitle{[Viellesse de David]}
\VerseOne{}Le roi David était vieux et avancé en âge ; on le couvrait de vêtements parce qu’il ne parvenait point à se réchauffer.
\VS{2}Ses serviteurs lui dirent : Que l'on cherche pour le roi notre seigneur, une jeune fille vierge ; qu’elle se tienne devant le roi, qu’elle le soigne et qu'elle dorme en son sein, afin que le roi notre seigneur, se réchauffe.
\VS{3}On chercha donc dans toutes les contrées d'Israël une jeune et belle femme, et on trouva Abischag la Sunamite, que l’on amena auprès du roi.
\VS{4}Cette jeune femme était fort belle. Elle prit soin du roi et le servit, mais le roi ne la connut point.
\TextTitle{[Adonija veut prendre la royauté, Nathan etBath-Schéba informent David]}
\VS{5}Alors Adonija, fils de Haggith, se laissa emporter par l’orgueil en disant : Je suis le roi ! Il se procura un char, des cavaliers et cinquante hommes qui couraient devant lui.
\VS{6}Son père ne lui avait jamais fait un reproche jusqu’à ce jour-là, en disant : Pourquoi agis-tu ainsi ? Adonija était très beau de figure, il était né après Absalom.
\VS{7}Il s’entendit avec Joab, fils de Tseruja, et avec le sacrificateur Abiathar, qui embrassèrent son parti.
\VS{8}Mais le sacrificateur Tsadok, Benaja fils de Jehojada, Nathan le prophète, Schimeï, Reï et les vaillants hommes de David ne furent point du parti d'Adonija.
\VS{9}Or, Adonija fit tuer des brebis, des bœufs et des veaux gras près de la pierre de Zohéleth, qui est auprès d’En-Roguel ; il invita tous ses frères, fils du roi, et tous les hommes de Juda qui étaient au service du roi.
\VS{10}Mais il ne convia point Nathan le prophète, ni Benaja, ni les vaillants hommes, ni Salomon, son frère.
\VS{11}Alors Nathan parla à Bath-Schéba, mère de Salomon, en disant : N'as-tu pas entendu qu'Adonija, fils de Haggith, a été fait roi ? Et David notre Seigneur n'en sait rien.
\VS{12}Maintenant donc viens, je t’en donne le conseil afin que tu sauves ta vie et la vie de ton fils Salomon.
\VS{13}Va, entre chez le roi David et dis-lui : Ô roi, mon seigneur, n'as-tu pas fait serment à ta servante, en disant : ton fils Salomon régnera après moi et sera assis sur mon trône ? Pourquoi donc Adonija règne-t-il ?
\VS{14}Et voici, lorsque tu seras encore là et que tu parleras avec le roi, je viendrai après toi et je confirmerai tes dires.
\VS{15}Bath-Schéba se rendit dans la chambre du roi. Or, le roi était très vieux et Abischag, la Sunamite, le servait.
\VS{16}Bath-Schéba s'inclina et se prosterna devant le roi. Et le roi lui dit : Qu'as-tu ?
\VS{17}Et elle lui répondit : Mon seigneur, tu as juré par Yahweh, ton Dieu à ta servante, en lui disant : Ton fils Salomon régnera après moi et s’assiéra sur mon trône.
\VS{18}Mais maintenant voici, Adonija est proclamé roi ! Et tu ne le sais pas, ô roi, mon seigneur !
\VS{19}Il a fait tuer des bœufs, des veaux gras et des brebis en grand nombre, il a convié tous les fils du roi, avec Abiathar, le sacrificateur, et Joab, chef de l'armée, mais il n'a point convié ton serviteur Salomon.
\VS{20}Ô roi mon seigneur ! Les yeux de tout Israël sont sur toi, afin que tu lui fasses connaître qui s’assiéra sur le trône du roi mon seigneur après lui.
\VS{21}Aussi, lorsque le roi mon seigneur sera endormi avec ses pères, nous serons traités comme des coupables, moi et mon fils Salomon.
\VS{22}Tandis qu’elle parlait encore avec le roi, Nathan le prophète se présenta.
\VS{23}On l’annonça au roi en disant : Voici Nathan le prophète ! Il se présenta devant le roi et se prosterna devant lui, le visage contre terre.
\VS{24}Et Nathan dit : Ô roi mon seigneur ! Tu as dit : Adonija régnera après moi et sera assis sur mon trône !
\VS{25}Car il est descendu aujourd'hui, il a sacrifié des bœufs, des veaux gras et des brebis en grand nombre. Il a convié tous les fils du roi, les chefs de l'armée et le sacrificateur Abiathar. Et voici, ils mangent et boivent devant lui ; ils disent : Vive le roi Adonija !
\VS{26}Mais il n'a convié ni moi, ton serviteur, ni le sacrificateur Tsadok, ni Benaja, fils de Jehojada, ni Salomon ton serviteur.
\VS{27}Est-ce bien par ordre de mon seigneur le roi que cette chose a lieu et sans que tu aies fait connaître à ton serviteur quel est celui qui doit s'asseoir sur le trône du roi mon seigneur après lui ?
\VS{28}Et le roi David répondit, en disant : Appelez-moi Bath-Schéba ; elle entra et se présenta devant le roi.
\VS{29}Alors le roi jura et dit : Yahweh, qui m'a délivré de toute détresse, est vivant !
\VS{30}Comme je te l'ai juré par Yahweh, le Dieu d'Israël, en disant : Ton fils Salomon régnera après moi et sera assis sur mon trône à ma place ; ainsi ferai-je aujourd'hui.
\VS{31}Alors Bath-Schéba s'inclina le visage contre terre et se prosterna devant le roi en disant : Que le roi David mon seigneur vive éternellement !
\VS{32}Et le roi David dit : Appelez-moi le sacrificateur Tsadok, le prophète Nathan et Benaja, fils de Jehojada ; et ils se présentèrent devant le roi.
\VS{33}Le roi leur dit : Prenez avec vous les serviteurs de votre seigneur, faites monter mon fils Salomon sur ma mule, et faites-le descendre à Guihon.
\VS{34}Que Tsadok le sacrificateur et Nathan le prophète, l'oignent en ce lieu-là pour roi sur Israël, puis vous sonnerez du shofar et vous direz : Vive le roi Salomon !
\VS{35}Vous monterez après lui et il viendra, il s'assiéra sur mon trône et il régnera à ma place ; car j'ai ordonné qu'il soit le chef d'Israël et de Juda.
\VS{36}Et Benaja fils de Jehojada répondit au roi : Amen ! Ainsi parle Yahweh, le Dieu de mon seigneur le roi !
\VS{37}Comme Yahweh a été avec mon seigneur le roi, qu'il soit aussi avec Salomon, et qu'il élève son trône encore plus que le trône du roi David mon seigneur !
\TextTitle{[Tsadok oint roi Salomon\FTNTT{cp. 1 Ch. 29:22}]}
\VS{38}Puis Tsadok le sacrificateur descendit avec Nathan le prophète et Benaja, fils de Jehojada, les Kéréthiens et les Péléthiens ; ils firent monter Salomon sur la mule du roi David et le menèrent à Guihon.
\VS{39}Tsadok le sacrificateur prit du tabernacle une corne d'huile dont il oignit Salomon. On sonna du shofar et tout le peuple dit : Vive le roi Salomon !
\VS{40}Et tout le monde monta après lui et le peuple jouait de la flûte, en se livrant à une grande joie, au point que la terre se fendait par leurs cris.
\VS{41}Ce bruit fut entendu d’Adonija et de tous les conviés qui étaient avec lui comme ils achevaient de manger ; et Joab entendant le son du shofar, dit : Pourquoi ce bruit de la ville en tumulte ?
\VS{42}Et comme il parlait encore, voici Jonathan, fils du sacrificateur Abiathar, arriva et Adonija lui dit : Entre, car tu es un vaillant homme et tu apportes de bonnes nouvelles.
\VS{43}Oui ! répondit Jonathan à Adonija : Le roi David, notre seigneur, a établi Salomon roi.
\VS{44}Et le roi a envoyé avec lui Tsadok le sacrificateur, Nathan le prophète, Benaja, fils de Jehojada, les Kéréthiens, et les Péléthiens, et ils l'ont fait monter sur la mule du roi.
\VS{45}Tsadok le sacrificateur, et Nathan le prophète l'ont oint pour roi à Guihon, d'où ils sont remontés avec joie, et la ville est ainsi émue ; c'est là le bruit que vous avez entendu.
\VS{46}Salomon s'est même assis sur le trône royal.
\VS{47}Et les serviteurs du roi sont venus pour bénir le roi David notre seigneur, en disant : Que ton Dieu rende le nom de Salomon encore plus grand que ton nom, et qu'il élève son trône encore plus que ton trône ! Et le roi s'est prosterné sur son lit.
\VS{48}Le roi a ainsi parlé : Béni soit Yahweh, le Dieu d'Israël, qui a aujourd’hui établi sur mon trône un successeur, et qui m’a permis de le voir !
\VS{49}Alors tous les conviés d’Adonija furent saisis de frayeur, ils se levèrent et s'en allèrent chacun son chemin.
\VS{50}Adonija eut peur de Salomon ; il se leva aussi et s'en alla empoigner les cornes de l'autel.
\VS{51}On vint l’apprendre à Salomon, en disant : Voici Adonija a peur du roi Salomon et il a saisi les cornes de l'autel, en disant : Que le roi Salomon me jure aujourd'hui qu'il ne fera point mourir son serviteur par l'épée.
\VS{52}Et Salomon dit : A l’avenir, s’il se comporte en homme de bien il ne tombera pas un seul de ses cheveux à terre ; mais s'il se trouve du mal en lui, il mourra.
\VS{53}Alors le roi Salomon envoya des personnes qui le firent descendre de l'autel. Il vint et se prosterna devant le roi Salomon, et Salomon lui dit : Va dans ta maison.
\Chap{2}
\TextTitle{[Dernières paroles de David à Salomon]}
\VerseOne{}David approchait du moment de sa mort, et il donna ses ordres à Salomon, son fils, en disant :
\VS{2}Je m'en vais par le chemin de toute la terre, fortifie-toi et comporte-toi en homme.
\VS{3}Observe les commandements de Yahweh, ton Dieu, en marchant dans ses voies, en gardant ses statuts, ses commandements, ses ordonnances et ses préceptes, selon ce qui est écrit dans la loi de Moïse, afin que tu réussisses dans tout ce que tu feras et dans tout ce que tu entreprendras ;
\VS{4}et afin que s’accomplisse cette parole de Yahweh déclarée sur moi : Si tes fils prennent garde à leur voie, pour marcher devant moi dans la vérité, de tout leur cœur et de toute leur âme, tu ne manqueras jamais de successeur sur le trône d'Israël.
\VS{5}Tu sais ce que m'a fait Joab, fils de Tseruja et ce qu'il a fait aux deux chefs des armées d'Israël, Abner, fils de Ner, et à Amasa, fils de Jéther, qu'il a tués, en versant pendant la paix le sang de la guerre ; il a mis de ce sang sur la ceinture qu'il avait sur ses reins et sur les chaussures qu'il avait aux pieds.
\VS{6}Tu agiras selon ta sagesse, en sorte que tu ne laisseras point ses cheveux blancs descendre en paix dans le scheol.
\VS{7}Tu traiteras avec bienveillance les fils de Barzillaï, le Galaadite, et ils seront du nombre de ceux qui mangent à ta table ; car ils se sont approchés de moi quand je fuyais Absalom, ton frère.
\VS{8}Voici, tu as avec toi Schimeï, fils de Guéra, le benjamite de Bachurim, qui proféra contre moi des malédictions violentes le jour où je m'en allais à Mahanaïm. Mais il descendit au-devant de moi vers le Jourdain et je lui jurai par Yahweh, en disant : Je ne te ferai point mourir par l'épée.
\VS{9}Maintenant donc tu ne le laisseras point impuni, car tu es sage, pour savoir comment tu dois le traiter ; et tu feras descendre ses cheveux blancs ensanglantés au scheol.
\TextTitle{[Mort de David, Salomon s'assoit sur le trône\FTNTT{1 Ch. 29:23-30}]}
\VS{10}Ainsi David se coucha avec ses pères, il fut enseveli dans la cité de David.
\VS{11}Et le temps que David régna sur Israël fut quarante ans. Il régna sept ans à Hébron et il régna trente-trois ans à Jérusalem.
\VS{12}Et Salomon s'assit sur le trône de David, son père, et son règne fut très affermi.
\TextTitle{[Mort d'Adonija]}
\VS{13}Alors Adonija, fils de Haggith, vint vers Bath-Schéba, mère de Salomon et elle dit : Amènes-tu la paix ? Et il répondit : Je viens en paix.
\VS{14}Il ajouta : J'ai un mot à te dire. Elle répondit : Parle !
\VS{15}Et il dit : Tu sais bien que le royaume m'appartenait et que tout Israël s'attendait à ce que je règne. Mais la royauté s’est détournée de moi, elle est échue à mon frère parce que Yahweh la lui a donnée.
\VS{16}Maintenant donc je te demande une chose, ne me la refuse point. Elle lui répondit : Parle !
\VS{17}Et il dit : Je te prie, dis au roi Salomon, car il ne te refusera rien, qu'il me donne Abischag, la Sunamite, pour femme.
\VS{18}Bath-Schéba répondit : Et bien, je parlerai pour toi au roi.
\VS{19}Bath-Schéba se rendit auprès du roi Salomon pour lui parler en faveur d’Adonija ; et le roi se leva pour aller au-devant d’elle, il se prosterna devant elle, puis il s'assit sur son trône. On plaça un siège pour la mère du roi, et elle s'assit à sa droite.
\VS{20}Elle dit alors : J'ai une petite demande à te faire : ne me la refuse pas ! Et le roi lui répondit : Demande, ma mère, car je ne te la refuserai point.
\VS{21}Et elle dit : Qu'on donne Abischag, la Sunamite, pour femme à Adonija, ton frère.
\VS{22}Mais le roi Salomon répondit à sa mère et dit : Et pourquoi demandes-tu Abischag, la Sunamite, pour Adonija ? Demande plutôt le royaume pour lui, parce qu'il est mon frère aîné ; demande-le pour lui, pour Abiathar, le sacrificateur, et pour Joab, fils de Tseruja !
\VS{23}Alors le roi Salomon jura par Yahweh, en disant : Que Dieu me traite dans toute sa rigueur, si Adonija n'a dit cette parole contre sa propre vie !
\VS{24}Maintenant Yahweh est vivant, lui qui m'a établi, qui m'a fait asseoir sur le trône de David, mon père, et qui m'a donné une maison, selon sa promesse ! Aujourd’hui Adonija mourra.
\VS{25}Et le roi Salomon envoya Benaja, fils de Jehojada, qui le frappa, et Adonija mourut.
\TextTitle{[Abiathar destitué de la sacrificature]}
\VS{26}Puis le roi dit à Abiathar, le sacrificateur : Va-t'en à Anathoth sur tes terres, car tu mérites la mort ; toutefois je ne te ferai point mourir aujourd'hui, parce que tu as porté l'arche du Seigneur Yahweh devant David, mon père ; et parce que tu as eu part à toutes les afflictions de mon père.
\VS{27}Ainsi Salomon dépouilla Abiathar de ses fonctions, afin qu'il ne fût plus sacrificateur de Yahweh pour accomplir la parole de Yahweh, qu'il avait prononcée à Silo contre la maison d'Eli.
\TextTitle{[Mort de Joab, Benaja à la tête de l'armée]}
\VS{28}Le bruit en parvint à Joab, qui avait suivi le parti d’Adonija, quoiqu'il n’eût pas suivi le parti d’Absalom. Joab s'enfuit au tabernacle de Yahweh et empoigna les cornes de l'autel.
\VS{29}On alla l’apprendre au roi Salomon, en disant : Joab s'en est enfui dans la tente de Yahweh et il est auprès de l'autel. Salomon envoya Benaja, fils de Jehojada, et lui dit : Va et frappe-le.
\VS{30}Benaja entra dans la tente de Yahweh et dit à Joab : Ainsi a parlé le roi : Sors de là ! Mais il répondit : Non ! Je veux mourir ici. Et Benaja rapporta la chose au roi en disant : Joab m'a parlé ainsi et c’est ainsi qu’il m’a répondu.
\VS{31}Et le roi dit à Benaja : Fais comme il t'a dit, frappe-le et enterre-le ; tu ôteras ainsi de dessus moi et de dessus la maison de mon père le sang que Joab a répandu sans cause.
\VS{32}Et Yahweh fera retomber son sang sur sa tête, car il a frappé deux hommes plus justes et meilleurs que lui et les a tués par l'épée, sans que mon père David n’en sût rien : Abner, fils de Ner, chef de l'armée d'Israël, et Amassa, fils de Jéther, chef de l'armée de Juda.
\VS{33}Leur sang retombera sur la tête de Joab et sur la tête de sa postérité à perpétuité ; mais il y aura paix à toujours de par Yahweh, pour David, pour sa postérité, pour sa maison et pour son trône.
\VS{34}Donc Benaja, fils de Jehojada, monta, et il frappa Joab à mort. On l'ensevelit dans sa maison, dans le désert.
\VS{35}Alors le roi établit Benaja, fils de Jehojada, sur l'armée à la place de Joab ; le roi établit aussi Tsadok sacrificateur à la place d'Abiathar.
\TextTitle{[Mort de Schimeï]}
\VS{36}Puis le roi fit appeler Schimeï et lui dit : Bâtis-toi une maison à Jérusalem, et demeures-y, et n'en sors point pour aller de côté ou d'autre.
\VS{37}Car sache que le jour où tu en sortiras et que tu passeras le torrent de Cédron, tu mourras certainement ; ton sang sera sur ta tête.
\VS{38}Schimeï répondit au roi : Cette parole est bonne ! Ton serviteur fera tout ce que le roi mon Seigneur a dit. Ainsi Schimeï demeura à Jérusalem plusieurs jours.
\VS{39}Mais il arriva qu'au bout de trois ans, deux serviteurs de Schimeï s'enfuirent vers Akisch, fils de Maaca, roi de Gath, et on le rapporta à Schimeï en disant : Voilà tes serviteurs sont à Gath.
\VS{40}Alors Schimeï se leva, sella son âne, et s'en alla à Gath vers Akisch pour chercher ses serviteurs. Schimeï s'en alla donc et ramena de Gath ses serviteurs.
\VS{41}On rapporta à Salomon que Schimeï était allé de Jérusalem à Gath, et qu'il était de retour.
\VS{42}Et le roi envoya appeler Schimeï, et lui dit : Ne t'avais-je pas fait jurer par Yahweh, et ne t'avais-je pas fait cette déclaration formelle : Sache-le, sache bien que le jour que tu sortiras pour aller de côté ou d’autre, tu mourras ? Et ne me répondis-tu pas : La parole que j'ai entendue est bonne ?
\VS{43}Pourquoi donc n'as-tu pas observé le serment que tu as fait par Yahweh et le commandement que je t'avais donné ?
\VS{44}Le roi dit aussi à Schimeï : Tu sais en ton cœur tout le mal que tu as fait à David, mon père ; c'est pourquoi Yahweh a fait retomber ta méchanceté sur ta tête.
\VS{45}Mais le roi Salomon sera béni, et le trône de David sera affermi devant Yahweh à jamais.
\VS{46}Et le roi donna commission à Benaja fils de Jehojada, qui sortit, et frappa Schimeï et Schimeï mourut. La royauté fut ainsi affermie entre les mains de Salomon.
\Chap{3}
\TextTitle{[Salomon s'allie à Pharaon]}
\VerseOne{}Or, Salomon s'allia avec Pharaon roi d'Egypte. Il prit pour femme la fille de Pharaon, et l'amena en la cité de David, jusqu'à ce qu'il eût achevé de bâtir sa maison, la maison de Yahweh, et la muraille de Jérusalem tout alentour.
\VS{2}Seulement le peuple sacrifiait dans les hauts lieux, parce que jusqu’alors on n’avait pas bâti de maison au nom de Yahweh.
\Chap{3}
\TextTitle{[Salomon demande la sagesse à Yahweh\FTNTT{2 Ch. 1:2-10}]}
\VS{3}Salomon aimait Yahweh, il marchait selon les ordonnances de David son père. Seulement, c’était sur les hauts lieux qu’il offrait des sacrifices et des parfums.
\VS{4}Le roi se rendit à Gabaon pour y sacrifier, car c'était le plus grand des hauts lieux. Et Salomon offrit mille holocaustes sur cet autel.
\VS{5}Et Yahweh apparut de nuit à Salomon à Gabaon dans un songe, et Dieu lui dit : Demande ce que tu veux que je te donne.
\VS{6}Et Salomon répondit : Tu as usé d'une grande bienveillance envers ton serviteur David, mon père, parce qu’il a marché devant toi fidèlement, dans la justice, et dans la droiture de cœur envers toi. Tu as gardé cette grande bienveillance envers lui en lui donnant un fils qui est assis sur son trône, comme on le voit aujourd'hui.
\VS{7}Or, maintenant, ô Yahweh mon Dieu ! Tu as fait régner ton serviteur à la place de David, mon père, et je ne suis qu'un jeune homme, je ne sais comment me conduire.
\VS{8}Ton serviteur est parmi ce peuple que tu as choisi, un peuple nombreux qui ne peut être compté ni dénombré à cause de sa multitude.
\VS{9}Accorde donc à ton serviteur un cœur intelligent pour juger ton peuple, pour discerner le bien du mal ! Car qui pourrait juger ce peuple si grand ?
\TextTitle{[Yahweh exauce Salomon\FTNTT{2 Ch. 1:11-13}]}
\VS{10}Cette demande de Salomon plut à Yahweh.
\VS{11}Et Dieu lui dit : Puisque c’est là ta demande et que tu n'as point demandé une longue vie, ni les richesses, ni la mort de tes ennemis, mais que tu as demandé de l'intelligence pour rendre justice,
\VS{12}voici, je fais selon ta parole. Voici, je te donne un cœur sage et intelligent, de sorte qu'il n'y aura eu personne de semblable avant toi et qu’il n'y en aura jamais de semblable après toi.
\VS{13}Et même, je te donne ce que tu n'as point demandé, les richesses et la gloire, de sorte qu'il n'y aura point de roi semblable à toi entre les rois, tant que tu vivras.
\VS{14}Et si tu marches dans mes voies pour garder mes ordonnances et mes commandements, comme David, ton père, je prolongerai tes jours.
\VS{15}Salomon s’éveilla. Et voilà le songe. Puis il s'en retourna à Jérusalem et se tint devant l'Arche de l'alliance de Yahweh. Là, il offrit des holocaustes et des offrandes de paix et fit un festin à tous ses serviteurs.
\VS{16}Alors deux femmes prostituées vinrent au roi et se présentèrent devant lui.
\VS{17}Et l'une de ces femmes dit : Hélas, mon Seigneur ! Nous demeurions cette femme-ci et moi dans une même maison et j'ai accouché près d’elle dans cette maison-là.
\VS{18}Trois jours après, cette femme a aussi accouché. Et nous étions ensemble, il n'y avait aucun étranger avec nous dans cette maison, il n’y avait que nous deux.
\VS{19}Or, l'enfant de cette femme est mort la nuit, parce qu'elle s'était couchée sur lui.
\VS{20}Elle s'est levée au milieu de la nuit, et a pris mon fils à mes côtés pendant que ta servante dormait, et l'a couché dans son sein. Et son fils mort, elle l’a couché dans mon sein.
\VS{21}Le matin, je me suis levée pour allaiter mon fils. Et voici, il était mort. Je l’ai regardé attentivement ce matin-là ; et voici, ce n'était point mon fils que j'avais enfanté.
\VS{22}L’autre femme dit : Non, c’est mon fils qui est vivant, et c’est ton fils qui est mort. Mais la première répliqua : Nullement ! Celui qui est mort est ton fils, et c’est mon fils qui vit. Elles parlaient ainsi devant le roi.
\VS{23}Et le roi dit : L’une dit : C’est mon fils qui est vivant, et c’est ton fils qui est mort ; l’autre dit : Nullement ! C’est ton fils qui est mort, et c’est mon fils qui est vivant.
\VS{24}Alors le roi dit : Apportez-moi une épée ! Et on apporta une épée devant le roi.
\VS{25}Puis le roi dit : Partagez en deux l'enfant qui vit, et donnez-en la moitié à l'une et la moitié à l'autre.
\VS{26}Alors la femme dont le fils était vivant sentit ses entrailles s’émouvoir pour son fils, et elle dit au roi : Ah ! Mon seigneur, qu'on donne à celle-ci l'enfant qui vit et qu'on ne le fasse pas mourir ! Mais l'autre dit : Il ne sera ni à moi ni à toi ; qu'on le partage.
\VS{27}Alors le roi répondit et dit : Donnez à la première l'enfant qui vit, et ne le faites pas mourir. C’est elle qui est sa mère.
\VS{28}Tout Israël entendit parler du jugement que le roi avait prononcé. Et l’on craignit le roi, car l’on reconnut que la sagesse divine était en lui pour rendre justice.
\Chap{4}
\TextTitle{[Salomon établit onze chefs et douze intendants]}
\VerseOne{}Le roi Salomon était roi sur tout Israël.
\VS{2}Voici les chefs qu’il avait à son service. Azaria, fils du sacrificateur Tsadok,
\VS{3}Elihoreph et Achija, enfants de Schischa, secrétaires ; Josaphat, fils d'Achilud, archiviste ;
\VS{4}Benaja, fils de Jehojada, commandait l'armée ; Tsadok et Abiathar étaient sacrificateurs ;
\VS{5}Azaria, fils de Nathan était chef des intendants ; Zabud, fils de Nathan, était le ministre d’état, favori du roi ;
\VS{6}Achischar, chef de la maison du roi ; et Adoniram, fils d’Abda, préposé sur les impôts.
\VS{7}Or, Salomon avait douze intendants sur tout Israël, qui veillaient à l’entretien du roi et de sa maison ; et chacun pendant un mois de l'année.
\VS{8}Voici leurs noms : Le fils de Hur, sur la montagne d'Ephraïm.
\VS{9}Le fils de Déker, sur Makats, sur Saalbim, sur Beth-Schémesch, à Elon de Beth-Hanan.
\VS{10}Le fils de Hésed, à Arubboth ; il avait Soco et tout le pays de Hépher.
\VS{11}Le fils d'Abinadab avait toute la contrée de Dor ; il avait Thaphath, fille de Salomon, pour femme.
\VS{12}Baana, fils d'Achilud, avait Thaanac et Meguiddo, et tout le pays de Beth-Schean qui est près de Tsarthan au-dessous de Jizreel, depuis Beth-Schean jusqu'à Abel-Mehola et jusqu'au-delà de Jokmeam.
\VS{13}Le fils de Guéber, à Ramoth en Galaad ; il avait les bourgs de Jaïr, fils de Manassé, en Galaad ; il avait aussi toute la contrée d'Argob en Basan, soixante grandes villes à murailles et garnies de barres d'airain.
\VS{14}Achinadab, fils d’Iddo, à Mahanaïm.
\VS{15}Achimaats, qui avait pour femme Basmath, fille de Salomon, en Nephthali.
\VS{16}Baana, fils de Huschaï, en Aser et sur Bealoth.
\VS{17}Josaphat, fils de Paruach, à Issacar.
\VS{18}Schimeï, fils d'Ela, en Benjamin.
\VS{19}Guéber, fils d'Uri, dans le pays de Galaad, le pays de Sihon, roi des Amoréens, et d’Og, roi de Basan ; et il était seul intendant de ce pays-là.
\TextTitle{[L'étendue de la domination du royaume]}
\VS{20}Juda et Israël étaient en grand nombre, semblable au sable sur le bord de la mer ; ils mangeaient, buvaient et se réjouissaient.
\VS{21}Et Salomon dominait sur tous les royaumes depuis le fleuve jusqu'au pays des Philistins et jusqu'à la frontière d'Egypte ; ils apportaient des présents, et lui furent assujettis pendant toute sa vie.
\VS{22}Or, les vivres de Salomon pour chaque jour étaient de trente cors de fine farine et soixante d'autre farine,
\VS{23}dix bœufs gras, vingt bœufs de pâturages, et cent moutons, outre les cerfs, les daims et les volailles engraissées.
\VS{24}Il dominait sur toutes les contrées de l’autre côté du fleuve, depuis Thiphsach jusqu'à Gaza, sur tous les rois qui étaient de l’autre côté du fleuve. Il était en paix avec tous les pays alentour.
\VS{25}Juda et Israël habitèrent en sécurité chacun sous sa vigne et sous son figuier, depuis Dan jusqu'à Beer-Schéba, durant toute la vie de Salomon.
\VS{26}Salomon avait aussi quarante mille crèches pour les chevaux destinés à ses chars et douze mille hommes de cheval.
\VS{27}Or, les intendants pourvoyaient à l’entretien du roi Salomon et de tous ceux qui s'approchaient de sa table, chacun en son mois ; ils ne les laissaient manquer de rien.
\VS{28}Ils faisaient aussi venir de l'orge et de la paille pour les chevaux et les coursiers dans le lieu où se trouvait le roi, chacun selon les ordres qu'il avait reçus.
\TextTitle{[La sagesse de Salomon façonne sa renommée]}
\VS{29}Dieu donna à Salomon de la sagesse, une très grande intelligence, et des connaissances multipliées comme le sable qui est sur le bord de la mer.
\VS{30}La sagesse de Salomon surpassait la sagesse de tous les fils de l’Orient et toute la sagesse des égyptiens.
\VS{31}Il était plus sage qu’aucun homme, plus qu'Ethan, l’Ezrachite, plus qu'Héman, Calcol et Darda, les fils de Machol ; et sa renommée était répandue parmi toutes les nations d'alentour.
\VS{32}Il a prononcé trois mille paraboles et composa cinq mille cantiques.
\VS{33}Il a aussi parlé des arbres, depuis le cèdre du Liban jusqu'à l'hysope qui sort de la muraille ; il a aussi parlé sur les animaux, sur les oiseaux, sur les reptiles et sur les poissons.
\VS{34}Il venait des gens d'entre tous les peuples pour entendre la sagesse de Salomon, de la part de tous les rois de la terre qui avaient entendu parler de sa sagesse.
\Chap{5}
\TextTitle{[Salomon prépare la construction du temple\FTNTT{2 Ch. 2:1 ; 13:16}]}
\VerseOne{}Hiram, roi de Tyr, envoya ses serviteurs vers Salomon, car il apprit qu'on l'avait oint pour roi à la place de son père, car Hiram avait toujours aimé David.
\VS{2}Et Salomon fit dire à Hiram :
\VS{3}Tu sais que David, mon père, n'a pu bâtir une maison à Yahweh, son Dieu, à cause des guerriers qui l'ont encerclé, jusqu'à ce que Yahweh les ait mis sous la plante de ses pieds.
\VS{4}Maintenant Yahweh, mon Dieu, m'a donné du repos de toutes parts, et je n'ai plus d’adversaires, plus de calamités !
\VS{5}Voici donc j’ai l’intention de bâtir une maison au nom de Yahweh, mon Dieu, comme Yahweh l’a promis à David, mon père, en disant : Ton fils que je mettrai à ta place sur ton trône sera celui qui bâtira une maison à mon nom.
\VS{6}Ordonne maintenant que l’on coupe des cèdres du Liban pour moi. Mes serviteurs seront avec les tiens, et je donnerai pour tes serviteurs le salaire que tu auras fixé ; car tu sais qu'il n'y a personne parmi nous qui sache couper le bois comme les Sidoniens.
\VS{7}Lorsque Hiram eut entendu les paroles de Salomon, il eut une grande joie et il dit : Béni soit aujourd'hui Yahweh, qui a donné à David un fils sage pour chef de ce grand peuple !
\VS{8}Hiram fit répondre à Salomon : J'ai entendu ce que tu m'as envoyé dire et je ferai tout ce qui te plaira au sujet des bois de cèdre et des bois de cyprès.
\VS{9}Mes serviteurs les descendront du Liban à la mer, puis je les expédierai sur la mer par radeaux jusqu'au lieu que tu m'auras indiqué ; là je les ferai délier, et tu les prendras. Ce que je désire en retour, c’est que tu fournisses des vivres à ma maison.
\VS{10}Hiram donna du bois de cèdre et du bois de cyprès à Salomon autant qu'il en voulait.
\VS{11}Et Salomon donna à Hiram vingt mille cors de froment pour la nourriture de sa maison et vingt cors d'huile d’olives concassées ; Salomon en donna autant à Hiram chaque année.
\VS{12}Et Yahweh donna de la sagesse à Salomon, comme il le lui avait promis ; et il y eut paix entre Hiram et Salomon, et ils firent alliance ensemble.
\TextTitle{[Les hommes de corvée\FTNTT{2 Ch. 2:2 ; 17:18}]}
\VS{13}Le roi Salomon leva sur tout Israël des hommes de corvée ; ils étaient au nombre de trente mille hommes.
\VS{14}Il en envoya dix mille au Liban chaque mois, tour à tour, ils étaient un mois au Liban, et deux mois chez eux. Adoniram était préposé sur les hommes de corvée.
\VS{15}Salomon avait aussi soixante-dix mille hommes qui portaient les fardeaux et quatre-vingt mille qui taillaient les pierres dans la montagne,
\VS{16}sans compter les chefs au nombre de trois mille trois cents, préposés par Salomon sur le suivi des travaux, et chargés de surveiller les ouvriers.
\VS{17}Le roi ordonna d’extraire de grandes et précieuses pierres, pour faire le fondement de la maison, qui soient toutes taillées,
\VS{18}de sorte que les maçons de Salomon et ceux d'Hiram, taillèrent les pierres et préparèrent le bois et les pierres pour bâtir la maison.
\Chap{6}
\TextTitle{[Sept ans de construction du temple, les dimensions et caractéristiques\FTNTT{2 Ch. 3:1-14}]}
\VerseOne{}Ce fut la quatre cent quatre-vingtième année après la sortie des enfants d'Israël du pays d'Egypte que Salomon bâtit la maison de Yahweh, la quatrième année du règne de Salomon sur Israël, au mois de Ziv, qui est le second mois.
\VS{2}La maison que le roi Salomon bâtit à Yahweh avait soixante coudées de long, vingt de large, et trente de haut.
\VS{3}Le portique devant le temple de la maison avait vingt coudées de longueur, répondant à la largeur de la maison, et il avait dix coudées de profondeur sur le devant de la maison.
\VS{4}Il fit placer des fenêtres à la maison, fenêtres solidement grillées.
\VS{5}Il bâtit contre la muraille de la maison, à l’entour, des étages qui entouraient les murs de la maison, le temple et le sanctuaire ainsi il fit des chambres latérales tout autour.
\VS{6}L’étage inférieur était large de cinq coudées, celui du milieu de six coudées et le troisième de sept coudées ; car il avait aménagé des retraites à la maison tout autour en dehors, afin que la charpente n'entrât pas dans les murailles de la maison.
\VS{7}Pour bâtir la maison, on se servit de pierres déjà taillées, de sorte qu'en bâtissant la maison on n'entendit ni marteau, ni hache, ni aucun outil de fer.
\VS{8}L'entrée des chambres de l’étage inférieur était au côté droit de la maison, et on montait à l’étage du milieu par un escalier tournant, et de l’étage du milieu au troisième.
\VS{9}Après avoir achevé de bâtir la maison, Salomon couvrit la maison de planches et de poutres de cèdre.
\VS{10}Et il bâtit les étages joignant toute la maison, avec chacun cinq coudées de haut, et il les lia à la maison par des bois de cèdre.
\VS{11}Alors la parole de Yahweh fut adressée à Salomon, en ces termes :
\VS{12}Quant à cette maison que tu bâtis, si tu marches dans mes statuts, si tu pratiques mes ordonnances et que tu gardes tous mes commandements pour y marcher, j’accomplirai en ta faveur la parole que j'ai dite à David, ton père.
\VS{13}Et j'habiterai au milieu des enfants d'Israël, et je n'abandonnerai point mon peuple d'Israël.
\VS{14}Ainsi Salomon bâtit la maison et l'acheva.
\VS{15}Il revêtit de cèdre les murs de la maison, depuis le sol jusqu'au plafond ; il revêtit ainsi de bois l’intérieur, et il couvrit le sol de la maison de planches de cyprès.
\VS{16}Il revêtit aussi l'espace de vingt coudées de planches de cèdre à partir du fond de la maison, depuis le sol jusqu'au haut des murailles, et il bâtit cet espace au dedans pour en faire le sanctuaire, le saint des saints.
\VS{17}Les quarante coudées sur le devant formaient la maison, c’est-à-dire le temple.
\VS{18}Le bois de cèdre à l’intérieur de la maison était sculpté en coloquintes et en fleurs épanouies ; tout l’intérieur était de cèdre, on ne voyait aucune pierre.
\VS{19}Salomon disposa aussi le sanctuaire, au dedans de la maison vers le fond, pour y mettre l'arche de l'alliance de Yahweh.
\VS{20}Le sanctuaire avait par devant vingt coudées de long, vingt coudées de large, et vingt coudées de haut, et on le couvrit d’or pur ; on en couvrit aussi l'autel, fait de planches de bois de cèdre.
\VS{21}Salomon couvrit d’or pur l’intérieur de la maison, et fit passer un voile avec des chaînes d'or au-devant du sanctuaire, qu’il couvrit également d'or.
\VS{22}Ainsi il couvrit d'or la maison tout entière. Il couvrit aussi d'or tout l'autel qui était devant le sanctuaire.
\VS{23}Et il fit dans le sanctuaire deux chérubins de bois d'olivier sauvage, qui avaient chacun dix coudées de haut.
\VS{24}Chacune des ailes de l'un des chérubins avait cinq coudées et les ailes de l’autre chérubin avaient aussi cinq coudées ; depuis le bout d'une aile jusqu'au bout de l'autre aile il y avait donc dix coudées.
\VS{25}Le second chérubin était aussi de dix coudées. Les deux chérubins étaient d'une même mesure et taillés l'un comme l'autre.
\VS{26}La hauteur de chacun des deux chérubins était de dix coudées.
\VS{27}Salomon plaça les chérubins à l’intérieur, au milieu de la maison. Les ailes des chérubins étaient déployées : l'aile de l'un touchait à l’un des murs, l'aile de l'autre chérubin touchait à l'autre mur ; et leurs autres ailes se rencontraient par l’extrémité au milieu de la maison.
\VS{28}Salomon couvrit d'or les chérubins.
\VS{29}Il fit sculpter sur tout le pourtour des murs de la maison, à l’intérieur et à l’extérieur, des sculptures en relief de chérubins, des palmes et des fleurs épanouies.
\VS{30}Il couvrit aussi d'or le sol de la maison, tant à l’intérieur qu’au-dehors.
\VS{31}A l'entrée du sanctuaire, il fit une porte à deux battants de bois d'olivier sauvage, dont les linteaux avec les poteaux équivalaient à un cinquième du mur.
\VS{32}Les deux battants étaient de bois d'olivier sauvage. Il y fit sculpter des chérubins, des palmes et des fleurs épanouies qu’il couvrit d'or, étendant également l'or sur les chérubins et sur les palmes.
\VS{33}Il fit aussi, à l'entrée du temple, des poteaux de bois d'olivier sauvage, du quart de la dimension du mur.
\VS{34}Les deux battants étaient de bois de sapin ; chacun des battants était formé de deux planches brisées.
\VS{35}Il y fit sculpter des chérubins, des palmes et des fleurs épanouies, et les couvrit d'or, proprement posé sur la sculpture.
\VS{36}Il bâtit aussi le parvis de l’intérieur de trois rangées de pierres de taille et d'une rangée de poutres de cèdre.
\VS{37}La quatrième année, au mois de Ziv, les fondements de la maison de Yahweh furent posés.
\VS{38}Et la onzième année, au mois de Bul, qui est le huitième mois, la maison fut achevée dans toutes ses parties et telle qu’elle devait être. Salomon la construisit en l’espace de sept années.
\Chap{7}
\TextTitle{[Le palais royal]}
\VerseOne{}Salomon bâtit aussi sa maison, et l'acheva complètement en treize ans.
\VS{2}Il bâtit d’abord la maison de la forêt du Liban, de cent coudées de long, de cinquante coudées de large, et de trente coudées de haut, sur quatre rangées de colonnes de cèdre ; et sur les colonnes il y avait des poutres de cèdre.
\VS{3}On couvrit de bois de cèdre les chambres qui portaient sur les colonnes qui étaient au nombre de quarante-cinq, quinze par étages.
\VS{4}Et il y avait trois rangées de fenêtrages ; et une fenêtre répondait à l'autre en trois endroits.
\VS{5}Toutes les portes et tous les poteaux étaient formés de poutres carrées, avec les fenêtres ; et à chacun des trois étages, les ouvertures étaient en vis-à-vis les unes des autres.
\VS{6}Il fit aussi le portique de colonnes, long de cinquante coudées, et large de trente coudées ; et un autre portique en avant avec des colonnes et des degrés sur leur front.
\VS{7}Il fit aussi le portique du trône sur lequel il rendait ses jugements, appelé le portique du jugement ; on le couvrit de cèdre depuis un bout du sol jusqu'à l'autre.
\VS{8}La maison où il demeurait fut construite de la même manière, dans une autre cour, derrière le portique. Salomon fit une maison bâtie comme ce portique à la fille de Pharaon, qu'il avait prise pour femme.
\VS{9}Toutes ces constructions étaient de pierres de prix, taillées d’après des mesures, sciées à la scie, en dedans et en dehors, depuis les fondements jusqu'aux corniches, et par dehors jusqu'au grand parvis.
\VS{10}Le fondement était en pierres magnifiques et de grand prix, de grandes pierres, des pierres de dix coudées et des pierres de huit coudées.
\VS{11}Et par-dessus il y avait des pierres de prix, taillées d’après des mesures, et du bois de cèdre.
\VS{12}Et le grand parvis avait aussi tout alentour trois rangées de pierres de taille et une rangée de poutres de cèdre, comme le parvis intérieur de la maison de Yahweh, et le portique de la maison.
\TextTitle{[Fabrication d'ouvrages pour le temple, Hiram l'expert\FTNTT{2 Ch. 2:12-13}]}
\VS{13}Or, le roi Salomon fit venir de Tyr Hiram ;
\VS{14}fils d'une femme veuve de la tribu de Nephthali, et d’un père tyrien, Hiram travaillait le cuivre ; fort expert, intelligent et savant pour faire toutes sortes d'ouvrages d'airain ; il arriva auprès du roi Salomon, et il fit tout son ouvrage.
\TextTitle{[les colonnes du temple\FTNTT{2 Ch. 3:15-17}]}
\VS{15}Il fit les deux colonnes d'airain, la première avait dix-huit coudées de hauteur ; et un cordon de douze coudées mesurait le tour de la seconde.
\VS{16}Il fit aussi deux chapiteaux d'airain fondu pour mettre sur les sommets des colonnes ; le premier chapiteau était de cinq coudées de hauteur, le second était aussi de cinq coudées.
\VS{17}Il fit des treillis en forme de maillages, des festons façonnés en forme de chaînes, pour les chapiteaux qui étaient sur le sommet des colonnes, sept pour le premier des chapiteaux, et sept pour le second.
\VS{18}Il fit deux rangs de grenades autour de l’un des treillis, pour couvrir le chapiteau qui était sur le sommet d'une des colonnes ; et il fit de même pour l'autre chapiteau.
\VS{19}Dans le portique, les chapiteaux qui étaient sur le sommet des colonnes figuraient des fleurs de lis hautes de quatre coudées au porche.
\VS{20}Ces chapiteaux placés sur les deux colonnes étaient entourés de deux cents grenades, en haut, depuis le renflement qui était au-delà du treillis ; il y avait aussi deux cents grenades, disposées par rangs, autour du second chapiteau.
\VS{21}Il dressa donc les colonnes au portique du temple. Il dressa la colonne de droite qu’il nomma Jakin ; puis il dressa la colonne de gauche qu’il nomma Boaz.
\VS{22}Et l’on mit sur le chapiteau des colonnes l'ouvrage figurant des fleurs de lis ; ainsi l'ouvrage des colonnes fut achevé.
\TextTitle{[la mer de fonte\FTNTT{2 Ch. 4:2-5}]}
\VS{23}Il fit aussi la mer de fonte. Elle avait dix coudées d'un bord à l'autre, ronde tout autour, avec cinq coudées de haut ; et un cordon de trente coudées en mesurait le tour.
\VS{24}Au-dessous de son bord, des coloquintes l'environnaient, dix à chaque coudée, lesquelles faisaient tout le tour de la mer. Il y avait deux rangées de coloquintes, jetées en fonte.
\VS{25}Et elle était posée sur douze bœufs, dont trois regardaient le nord et trois regardaient l'occident, trois regardaient le sud et trois regardaient l'orient. La mer était sur eux et toute la partie postérieure de leur corps était tournée en dedans.
\VS{26}Son épaisseur était d'une paume, et son bord était comme le bord d'une coupe en fleur de lis ; elle contenait deux mille baths.
\TextTitle{[Les dix socles d'airain]}
\VS{27}Il fit aussi dix socles d'airain, ayant chacun quatre coudées de long, quatre coudées de large et trois coudées de haut.
\VS{28}Ces socles étaient réalisés de telle manière qu'il y avait des panneaux enchâssés entre leurs bordures.
\VS{29}Sur les panneaux qui étaient entre les bordures, il y avait des lions, des bœufs et des chérubins. Et sur les bordures, au-dessus et en dessous des lions et des bœufs, il y avait des ornements qui pendaient en festons.
\VS{30}Chaque socle avait quatre roues d'airain avec des essieux d'airain. Ses quatre pieds leur servaient d’appuis. Ces appuis étaient fondus au-dessous de la cuve, et au-dessus étaient les festons.
\VS{31}Le couronnement offrait à son intérieur une ouverture avec un prolongement d'une coudée vers le haut ; cette ouverture était arrondie comme pour les ouvrages de ce genre et elle avait une coudée et demie de largeur. Il s’y trouvait aussi des sculptures ; les panneaux étaient carrés, et non arrondis.
\VS{32}Les quatre roues étaient sous les panneaux, et les essieux des roues fixés à la base ; chaque roue était haute d'une coudée et demie.
\VS{33}Les roues étaient faites comme les roues de chars ; leurs essieux, leurs jantes, leurs rais et leurs moyeux étaient tous de fonte.
\VS{34}Il y avait aux quatre angles de chaque socle quatre consoles d’une même pièce que la base.
\VS{35}La partie supérieure de la base se terminait par un cercle d’une demi-coudée de hauteur, et elle avait ses appuis et ses panneaux de la même pièce.
\VS{36}Puis, on sculpta sur la surface de ses appuis et sur ses panneaux, des chérubins, des lions et des palmes, selon les espaces libres, et des ornements tout autour.
\VS{37}Ainsi les dix socles étaient tous d’une même fonte, d’une même mesure et d’une même forme.
\TextTitle{[Les dix cuves d'airain\FTNTT{2 Ch. 4:6}]}
\VS{38}Il fit aussi dix cuves d'airain, dont chacune contenait quarante baths, et chaque cuve était de quatre coudées, chaque cuve était sur l’un des dix socles.
\VS{39}Il mit cinq socles au côté droit de la maison, et cinq au côté gauche de la maison ; quant à la mer, il l’a mis au côté droit de la maison, vers l'orient du côté sud.
\VS{40}Ainsi Hiram fit les cuves, les pelles et les bassins, et il acheva tout l'ouvrage qu'il faisait au roi Salomon pour la maison de Yahweh.
\VS{41}Savoir, deux colonnes avec les deux chapiteaux qui étaient sur le sommet des colonnes ; et deux maillages pour couvrir les deux bourrelets des chapiteaux qui étaient sur le sommet des colonnes ;
\VS{42}les quatre cents grenades pour les deux maillages, deux rangs de grenades pour chaque réseau, pour couvrir les deux renflements des chapiteaux, qui étaient sur les colonnes ;
\VS{43}les dix socles ; et les dix cuves pour mettre sur les socles ;
\VS{44}la mer avec les douze bœufs sous la mer ;
\VS{45}les pots, les pelles et les bassins. Tous ces ustensiles que Hiram fit au roi Salomon pour la maison de Yahweh étaient d'airain poli.
\VS{46}Le roi les fit fondre dans la plaine du Jourdain, dans un sol argileux, entre Succoth et Tsarthan.
\VS{47}Et Salomon ne pesa aucun de ces ustensiles, parce qu'ils étaient en trop grand nombre, de sorte qu'on ne rechercha point le poids de l’airain.
\TextTitle{[Les ustensiles pour la maison de Yahweh]}
\VS{48}Salomon fit aussi tous les ustensiles pour la maison de Yahweh, savoir l'autel d'or, et les tables d'or, sur lesquelles étaient les pains de proposition ;
\VS{49}les chandeliers d’or pur, cinq à droite et cinq à gauche devant le sanctuaire, avec les fleurs, les lampes et les mouchettes d'or ;
\VS{50}les coupes, les couteaux, les bassins, les tasses et les brasiers d’or pur. Les gonds, même des portes de la maison, à l’entrée du saint des saints, à la porte de la maison et à l’entrée du temple, étaient d'or.
\VS{51}Ainsi fut achevé tout l'ouvrage que le roi Salomon fit pour la maison de Yahweh ; puis il y fit apporter l'or, l’argent et les ustensiles que David, son père, avait consacrés ; il les mit dans les trésors de la maison de Yahweh.
\Chap{8}
\TextTitle{[Installation de l'arche dans le temple, la nuée de Yahweh\FTNTT{2 Ch. 5:2-14}]}
\VerseOne{}Alors le roi Salomon convoqua près de lui à Jérusalem les anciens d'Israël, tous les chefs des tribus et les chefs de famille des fils d'Israël, pour transporter l'arche de l'alliance de Yahweh de la cité de David, qui est Sion.
\VS{2}Tous les hommes d'Israël s’assemblèrent auprès du roi Salomon, au mois d'Ethanim, qui est le septième mois, pendant la fête.
\VS{3}Une fois tous les anciens d'Israël arrivés, les sacrificateurs portèrent l'arche.
\VS{4}Ils transportèrent l'arche de Yahweh, la tente d'assignation, et tous les ustensiles qui étaient dans le tabernacle ; les sacrificateurs et les Lévites les emportèrent.
\VS{5}Le roi Salomon et toute l'assemblée d'Israël convoquée auprès de lui se tinrent devant l'arche. Ils sacrifièrent du gros et du menu bétail en si grand nombre, qu'on ne pouvait ni nombrer ni compter.
\VS{6}Et les sacrificateurs portèrent l'arche de l'alliance de Yahweh à sa place, dans le sanctuaire de la maison, dans le saint des saints, sous les ailes des chérubins.
\VS{7}Car les chérubins avaient les ailes étendues sur l’emplacement de l'arche, et ils couvraient l'arche et ses barres par-dessus.
\VS{8}On avait donné aux barres une longueur telle que leurs extrémités se voyaient du lieu saint devant le sanctuaire, mais elles ne se voyaient point du dehors. Elles sont demeurées là jusqu'à ce jour.
\VS{9}Il n'y avait rien dans l'arche que les deux tables de pierre que Moïse y déposa en Horeb, lorsque Yahweh fit alliance avec les enfants d'Israël à leur sortie du pays d'Egypte.
\VS{10}Au moment où les sacrificateurs sortirent du lieu saint, la nuée remplit la maison de Yahweh.
\VS{11}Les sacrificateurs ne purent pas y rester pour faire le service, à cause de la nuée ; car la gloire de Yahweh remplissait la maison de Yahweh.
\TextTitle{[Discours de Salomon\FTNTT{2 Ch. 6:1-11}]}
\VS{12}Alors Salomon dit : Yahweh veut habiter dans l'obscurité !
\VS{13}J'ai achevé de bâtir une maison pour ta demeure ô Yahweh ! Ce sera une demeure, un lieu où tu résideras éternellement.
\VS{14}Le roi tourna son visage, et bénit toute l'assemblée d'Israël ; car toute l'assemblée d'Israël se tenait là debout.
\VS{15}Et il dit : Béni soit Yahweh, le Dieu d'Israël, qui a parlé de sa propre bouche à David, mon père, et qui a accompli par sa puissance ce qu’il avait déclaré en disant :
\VS{16}Depuis le jour où je fis sortir mon peuple d'Israël hors d'Egypte, je n'ai choisi aucune ville d'entre toutes les tribus d'Israël pour y bâtir une maison afin que mon nom y fût, mais j'ai choisi David pour qu’il règne sur mon peuple d'Israël.
\VS{17}David, mon père, avait à cœur de bâtir une maison au nom de Yahweh, le Dieu d'Israël.
\VS{18}Et Yahweh dit à David, mon père : Puisque tu as eu à cœur de bâtir une maison à mon nom, tu as bien fait d’avoir eu cette intention.
\VS{19}Néanmoins, tu ne bâtiras point cette maison, mais ton fils qui sortira de tes entrailles sera celui qui bâtira cette maison à mon Nom.
\VS{20}Yahweh a donc accompli la parole qu'il avait prononcée. Je me suis élevé à la place de David, mon père, et me suis assis sur le trône d'Israël, comme Yahweh l’avait annoncé, et j'ai bâti cette maison au Nom de Yahweh, le Dieu d'Israël.
\VS{21}J'y ai établi ici un lieu pour l'arche, dans lequel est l'alliance de Yahweh, qu'il traita avec nos pères quand il les fit sortir hors du pays d'Egypte.
\TextTitle{[Prière de Salomon\FTNTT{2 Ch. 6:12-42}]}
\VS{22}Ensuite Salomon se tint devant l'autel de Yahweh en la présence de toute l'assemblée d'Israël, et étendant ses mains vers les cieux,
\VS{23}il dit : Ô Yahweh, Dieu d'Israël ! Il n'y a point de Dieu semblable à toi en haut dans les cieux, ni en bas sur la terre ; tu gardes l'alliance et la miséricorde envers tes serviteurs qui marchent devant ta face de tout cœur !
\VS{24}Ainsi tu as tenu parole à ton serviteur David, mon père, car ce que tu as déclaré de ta bouche, tu l'as accompli en ce jour par ta main puissante.
\VS{25}Maintenant donc, ô Yahweh, Dieu d'Israël, prête attention à la promesse faite à ton serviteur David, mon père, en lui disant : Tu ne manqueras jamais devant moi d’un successeur assis sur le trône d'Israël, pourvu seulement que tes fils prennent garde à leur voie et qu’ils marchent devant ma face, comme tu y as marché.
\VS{26}Et maintenant, ô Dieu d'Israël ! Je te prie, que s’accomplisse la promesse que tu as faite à ton serviteur David, mon père.
\VS{27}Mais Dieu habiterait-il véritablement sur la terre ? Voilà, les cieux, même les cieux des cieux ne peuvent te contenir ; combien moins cette maison que j'ai bâtie !
\VS{28}Toutefois, ô Yahweh, mon Dieu, sois attentif à la prière que t’adresse ton serviteur et à sa supplication, pour entendre le cri et la prière que ton serviteur t’adresse aujourd'hui.
\VS{29}Que tes yeux soient ouverts jour et nuit sur cette maison, sur le lieu dont tu as dit : Là sera mon Nom ! Ecoute la prière que ton serviteur fait en ce lieu.
\VS{30}Daigne exaucer la supplication de ton serviteur et de ton peuple d'Israël lorsqu’ils te prieront en ce lieu ; exauce du lieu de ta demeure. Des cieux, exauce, et pardonne !
\VS{31}Si quelqu'un pèche contre son prochain et qu’on lui impose un serment pour le faire jurer, et que le serment aura été fait devant ton autel dans cette maison ;
\VS{32}écoute-le des cieux, et agis. Juge tes serviteurs, condamne le coupable en lui rendant selon sa conduite ; rends justice à l’innocent, et traite-le selon son innocence !
\VS{33}Quand ton peuple d'Israël sera battu par l'ennemi, pour avoir péché contre toi, s’il revient à toi et rend gloire à ton Nom, en t’adressant des prières et des supplications dans cette maison,
\VS{34}exauce-le des cieux, et pardonne le péché de ton peuple d'Israël, et ramène-le dans la terre que tu as donnée à leurs pères.
\VS{35}Quand les cieux seront fermés et qu'il n'y aura point de pluie, à cause de ses péchés contre toi, s'il te fait une prière en ce lieu-ci, qu’il loue ton Nom, et s’il se détourne de ses péchés, parce que tu les auras affligés,
\VS{36}exauce-le des cieux, pardonne le péché de tes serviteurs et de ton peuple d'Israël, à qui tu enseigneras quel est le chemin par lequel ils doivent marcher et envoie-leur la pluie sur la terre que tu as donnée à ton peuple pour héritage !
\VS{37}Quand il y aura dans le pays, famine, peste, jaunisse, nielle, sauterelles d’une espèce ou d’une autre, même quand les ennemis assiégeront ton peuple dans son propre pays, quand il y aura un fléau ou une maladie quelconque ;
\VS{38}si un homme, si tout ton peuple d'Israël fait entendre des prières et des supplications, que chacun reconnaisse la plaie de son cœur et étende les mains vers cette maison,
\VS{39}exauce-le des cieux, du lieu de ta demeure, pardonne, et agis. Rends à chacun selon toutes ses voies, parce que tu auras connu leurs cœurs ; car toi seul connais le cœur de tous les fils des hommes ;
\VS{40}et ils te craindront toute leur vie dans le pays que tu as donné à nos pères !
\VS{41}Et même lorsque l'étranger, qui n’est pas de ton peuple d'Israël, viendra d'un pays éloigné à cause de ton Nom,
\VS{42}car on saura que ton Nom est grand, ta main puissante et ton bras étendu, quand il viendra prier dans cette maison,
\VS{43}exauce-le des cieux, du lieu de ta demeure, et fais à cet étranger selon ce qu’il t’aura demandé, afin que tous les peuples de la terre connaissent ton Nom pour te craindre, comme ton peuple d'Israël ; et pour connaître que ton Nom est invoqué sur cette maison que j'ai bâtie !
\VS{44}Quand ton peuple sortira pour combattre son ennemi, par la voie par laquelle tu l’auras envoyé, s'ils prient Yahweh en regardant vers cette ville que tu as choisie et vers cette maison que j'ai bâtie à ton Nom !
\VS{45}Exauce des cieux leurs prières et leurs supplications, et fais leur justice !
\VS{46}Quand ils pécheront contre toi, car il n'y a point d'homme qui ne pèche, et que tu seras irrité contre eux et que tu les auras livrés à leurs ennemis, qui les emmènera captifs dans un pays ennemi, lointain ou proche ;
\VS{47}si dans le pays où ils auront été menés captifs, ils reviennent à toi et t’adressent des supplications, se repentent et te prient au pays de ceux qui les auront emmenés captifs, en disant : Nous avons péché, nous avons commis l’iniquité, nous avons fait le mal !
\VS{48}S'ils reviennent à toi de tout leur cœur et de toute leur âme, dans le pays de leurs ennemis, qui les auront emmenés captifs, et s'ils t'adressent leurs prières, les regards tournés vers le pays que tu as donné à leurs pères, vers la ville que tu as choisie, vers la maison que j'ai bâtie à ton Nom,
\VS{49}exauce des cieux, du lieu de ta demeure, leurs prières et leurs supplications, et fais-leur justice.
\VS{50}Pardonne à ton peuple ses offenses et ses péchés envers toi, et fais que ceux qui les auront emmenés captifs aient pitié d'eux et leur fassent grâce,
\VS{51}car ils sont ton peuple et ton héritage, et tu les as fait sortir hors d'Egypte, du milieu d'une fournaise de fer !
\VS{52}Que tes yeux donc soient ouverts sur la supplication de ton serviteur et celle de ton peuple d'Israël, pour les exaucer dans tout ce pourquoi ils crieront à toi !
\VS{53}Car tu les as séparés de tous les autres peuples de la terre pour être ton héritage, comme tu l’as déclaré par Moïse, ton serviteur, quand tu fis sortir nos pères hors d'Egypte, ô Seigneur Yahweh !
\TextTitle{[Bénédictions et réjouissances\FTNTT{2 Ch. 7:4-10}]}
\VS{54}Lorsque Salomon eut achevé de faire cette prière et cette supplication à Yahweh, il se leva de devant l'autel de Yahweh où il était agenouillé et les mains étendues vers les cieux.
\VS{55}Il se tint debout, et bénit toute l'assemblée d'Israël à haute voix, en disant :
\VS{56}Béni soit Yahweh, qui a donné du repos à son peuple d'Israël, comme il l’avait annoncé ! De toutes les paroles qu'il avait prononcées par le moyen de Moïse, son serviteur, aucune n’est restée sans effet.
\VS{57}Que Yahweh, notre Dieu, soit avec nous, comme il a été avec nos pères ; qu'il ne nous abandonne point et qu'il ne nous délaisse point,
\VS{58}mais qu'il incline nos cœurs vers lui, afin que nous marchions dans toutes ses voies, et que nous observions ses commandements, ses statuts et ses ordonnances, qu'il a prescrits à nos pères !
\VS{59}Que ces paroles, par lesquelles j'ai fait supplication à Yahweh, soient présentes devant Yahweh, notre Dieu, jour et nuit ; afin qu'il fasse justice à son serviteur et à son peuple d’Israël en tout temps,
\VS{60}afin que tous les peuples de la terre reconnaissent que c'est Yahweh qui est Dieu et qu'il n'y en a point d'autre !
\VS{61}Que votre cœur soit intègre envers Yahweh, notre Dieu, comme aujourd'hui, pour marcher dans ses statuts et pour garder ses commandements.
\VS{62}Le roi et tout Israël avec lui offrirent des sacrifices devant Yahweh.
\VS{63}Salomon offrit un sacrifice d'offrande de paix à Yahweh, savoir vingt-deux mille bœufs et cent vingt mille brebis. Ainsi le roi et tous les enfants d'Israël firent la dédicace de la maison de Yahweh.
\VS{64}En ce jour-là, le roi consacra le milieu du parvis, qui est devant la maison de Yahweh ; car il offrit là les holocaustes, les offrandes et les graisses des sacrifices d’offrandes de paix, parce que l'autel d'airain qui est devant Yahweh, était trop petit pour contenir les holocaustes, les offrandes et les graisses des offrandes de paix.
\VS{65}Et en ce temps-là, Salomon célébra une fête solennelle ; et tout Israël avec lui, venu en grande multitude depuis les environs de Hamath jusqu'au torrent d'Egypte, devant Yahweh, notre Dieu, pendant sept jours, et sept autres jours, soit quatorze jours.
\VS{66}Le huitième jour, il renvoya le peuple. Et ils bénirent le roi, et s'en allèrent dans leurs demeures, en se réjouissant, et le cœur heureux pour tout le bien que Yahweh avait fait à David, son serviteur, et à Israël, son peuple.
\Chap{9}
\TextTitle{[Yahweh apparaît à Salomon\FTNTT{2 Ch. 7:11-22}]}
\VerseOne{}Lorsque Salomon eut achevé de bâtir la maison de Yahweh, la maison royale, et tout ce que Salomon prit plaisir à faire,
\VS{2}Yahweh apparut à Salomon une seconde fois, comme il lui était apparu à Gabaon.
\VS{3}Et Yahweh lui dit : J'exauce ta prière, et la supplication que tu as faite devant moi, j'ai sanctifié cette maison que tu as bâtie pour y mettre mon Nom à jamais, et mes yeux et mon cœur seront toujours là.
\VS{4}Quant à toi, si tu marches devant moi comme David, ton père, a marché, avec intégrité et de cœur et avec droiture, en faisant tout ce que je t'ai commandé, et si tu gardes mes statuts et mes ordonnances,
\VS{5}j’affermirai le trône de ton royaume sur Israël à jamais, comme je l’ai déclaré à David, ton père, en disant : Tu ne manqueras jamais d’un successeur sur le trône d'Israël.
\VS{6}Mais si vous et vos fils, vous vous détournez de moi et que vous ne gardiez pas mes commandements, mes lois que je vous ai prescrites, et si vous allez servir d'autres dieux et vous prosterner devant eux,
\VS{7}je retrancherai Israël de la terre que je lui ai donnée, je rejetterai loin de moi cette maison que j'ai consacrée à mon Nom et Israël sera un sujet de sarcasme et de moquerie parmi tous les peuples.
\VS{8}Et si haut placée qu’ait été cette maison, quiconque passera auprès d'elle sera étonné et sifflera. Et on dira : Pourquoi Yahweh a-t-il ainsi traité ce pays et cette maison ?
\VS{9}Et on répondra : Parce qu'ils ont abandonné Yahweh, leur Dieu, qui avait tiré leurs pères hors du pays d'Egypte, qu'ils se sont attachés à d'autres dieux, se sont prosternés devant eux et les ont servis, voilà pourquoi Yahweh a fait venir sur eux tous ces maux.
\TextTitle{[Les réalisations de Salomon\FTNTT{2 Ch. 8:1-18}]}
\VS{10}Au bout de vingt ans, Salomon avait bâti les deux maisons, la maison de Yahweh et la maison royale.
\VS{11}Hiram, roi de Tyr, avait fourni à Salomon du bois de cèdre, du bois de sapin et de l'or, autant qu'il en avait voulu, le roi Salomon donna à Hiram vingt villes dans le pays de Galilée.
\VS{12}Hiram sortit de Tyr, pour voir les villes que Salomon lui avait données. Mais elles ne lui plurent point,
\VS{13}et il dit : Quelles villes m'as-tu assignées, mon frère ? Et il les appela, pays de Cabul, nom qu’elles ont conservé jusqu'à ce jour.
\VS{14}Hiram avait aussi envoyé au roi cent vingt talents d'or.
\VS{15}Voici ce qui concerne les hommes de corvée que le roi Salomon leva pour bâtir la maison de Yahweh, sa maison, Millo, la muraille de Jérusalem, Hatsor, Meguiddo et Guézer.
\VS{16}Pharaon, roi d'Egypte, était venu s’emparer de Guézer et l'avait incendiée, il avait tué les Cananéens qui habitaient dans la ville. Puis il la donna pour dot à sa fille, femme de Salomon.
\VS{17}Salomon donc bâtit Guézer, et Beth-Horon la basse,
\VS{18}Baalath et Thadmor, dans le désert qui est au pays,
\VS{19}toutes les villes servant de magasins et lui appartenant, les villes pour les chars et les villes pour la cavalerie, et tout ce qu’il plut à Salomon de bâtir à Jérusalem, au Liban, et dans tout le pays dont il était le souverain.
\VS{20}Tout le peuple qui était resté des Amoréens, des Héthiens, des Phéréziens, des Héviens et des Jébusiens ne faisaient point partie des fils d'Israël,
\VS{21}leurs descendants qui étaient demeurés après eux dans le pays et que les fils d'Israël n'avaient pu dévouer par le moyen de l'interdit, Salomon les fit placer à son service comme gens de corvée à toujours.
\VS{22}Mais Salomon n’employa aucun des fils d'Israël comme esclaves ; car ils étaient ses hommes de guerre, ses serviteurs, ses chefs, ses officiers, les chefs de ses chars et ses hommes d'armes.
\VS{23}Les chefs préposés aux travaux par Salomon étaient au nombre de cinq cent cinquante, lesquels géraient l'intendance des ouvriers.
\VS{24}La fille de Pharaon monta de la cité de David dans la maison que Salomon lui avait bâtie. Ce fut alors qu’il bâtit Millo.
\VS{25}Trois fois par an, Salomon offrait des holocaustes et des offrandes de paix sur l'autel qu'il avait bâti à Yahweh, et il brûlait des parfums sur celui qui était devant Yahweh. Et il acheva la maison.
\VS{26}Le roi Salomon construisit des navires à Etsjon-Guéber, près d'Eloth, sur le rivage de la Mer Rouge, au pays d'Edom.
\VS{27}Et Hiram envoya sur ces navires, auprès des serviteurs de Salomon, ses propres serviteurs, des hommes connaissant la mer.
\VS{28}Ils allèrent en Ophir, et ils prirent de là quatre cent vingt talents d'or qu’ils apportèrent au roi Salomon.
\Chap{10}
\TextTitle{[La reine de Séba chez Salomon\FTNTT{2 Ch. 7:1-12}]}
\VerseOne{}Or, la reine de Séba ayant appris la renommée de Salomon, à cause du Nom de Yahweh, vint l’éprouver par des énigmes.
\VS{2}Elle entra dans Jérusalem avec une suite fort nombreuse, et avec des chameaux qui portaient des aromates, une grande quantité d'or, et des pierres précieuses. Elle se rendit auprès de Salomon, et lui parla de tout ce qu'elle avait dans le cœur.
\VS{3}Salomon répondit à toutes ses questions, et il n’y eut aucune parole à laquelle le roi ne put fournir une explication.
\VS{4}La reine de Séba vit toute la sagesse de Salomon et la maison qu'il avait bâtie,
\VS{5}les mets de sa table, la demeure de ses serviteurs, l’ordre de service, leurs vêtements, ses échansons, et les holocaustes qu'il offrait dans la maison de Yahweh.
\VS{6}Elle fut toute ravie en elle-même, elle parla ainsi au roi : Ce que j'ai entendu dire dans mon pays au sujet de ta sagesse était donc vrai !
\VS{7}Je ne croyais pas ce qu’on en disait avant d’être venue et que mes yeux ne l'aient vu. Et voici, on ne m'en avait point rapporté la moitié. Ta sagesse et ta prospérité surpassent tout ce que j'en avais entendu.
\VS{8}Heureux sont tes gens ! Heureux tes serviteurs qui se tiennent continuellement devant toi, et qui entendent ta sagesse !
\VS{9}Béni soit Yahweh, ton Dieu, qui t’a accordé la faveur de t’établir sur le trône d'Israël ! Car Yahweh a aimé Israël à toujours ; et t'a établi roi pour faire droit et justice.
\VS{10}Puis elle donna au roi cent vingt talents d'or, une très grande quantité d’aromates et des pierres précieuses. Il ne vint jamais depuis une aussi grande abondance d’aromates que la reine de Séba en donna au roi Salomon.
\VS{11}Et les navires de Hiram, qui amenèrent de l'or d'Ophir, amenèrent aussi d’Ophir une grande quantité de bois de santal et de pierres précieuses.
\VS{12}Le roi fit des supports de ce bois de santal pour la maison de Yahweh et pour la maison royale ; il en fit aussi des harpes et des luths pour les chantres ; il ne vint plus de ce bois de santal et on n’en a plus vu jusqu'à ce jour-là.
\VS{13}Le roi Salomon donna à la reine de Séba tout ce qu'elle désira et répondit à tout et ce qu'elle lui demanda. Il lui fit en outre des présents dignes d'un roi tel que Salomon. Puis elle s'en retourna et alla dans son pays, elle et ses serviteurs.
\TextTitle{[Les richesses de Salomon\FTNTT{2 Ch. 9:13-28}]}
\VS{14}Le poids de l'or qui revenait à Salomon chaque année, était de six cent soixante-six talents d'or,
\VS{15}outre ce qui lui revenait des négociants, du trafic des marchands, de tous les rois d'Arabie, et des gouverneurs de ce pays-là.
\VS{16}Le roi Salomon fit aussi deux cents grands boucliers d'or battu au marteau, employant six cents sicles d'or pour chaque bouclier,
\VS{17}et trois cents autres boucliers d'or battu au marteau, pour chacun desquels il employa trois mines d'or ; et le roi les mit dans la maison de la forêt du Liban.
\VS{18}Le roi fit aussi un grand trône d'ivoire, qu'il couvrit d’or pur.
\VS{19}Ce trône avait six degrés, et la partie supérieure, le haut du trône était arrondi par derrière. Il y avait des accoudoirs de chaque côté du siège et deux lions se tenaient auprès des accoudoirs.
\VS{20}Il y avait aussi douze lions sur les six degrés du trône, de part et d'autre. Il ne s'est rien fait de tel dans aucun royaume.
\VS{21}Toute la vaisselle du buffet du roi Salomon était d'or, et toutes les coupes de la maison de la forêt du Liban étaient d’or pur. Il n'y en avait point en argent ; on n’en faisait aucun cas du temps de Salomon.
\VS{22}Car le roi avait en mer des navires de Tarsis avec la flotte d'Hiram ; et tous les trois ans la flotte de Tarsis revenait, apportant de l'or, de l'argent, de l'ivoire, des singes et des paons.
\VS{23}Le roi Salomon fut plus grand que tous les rois de la terre, tant en richesses qu'en sagesse.
\VS{24}Tous les habitants de la terre cherchaient à voir la face de Salomon, pour écouter la sagesse que Dieu avait mise en son cœur.
\VS{25}Et chacun d'eux lui apportait son présent, des vases d’or et d'argent, des vêtements, des armes, des aromates, des chevaux et des mulets, tous les ans.
\VS{26}Salomon rassembla ses chars et sa cavalerie ; il y avait mille quatre cents chars et douze mille chevaliers, qu'il plaça dans les villes où il tenait ses chars et à Jérusalem près du roi.
\VS{27}Le roi rendit l'argent aussi commun à Jérusalem que les pierres ; et les cèdres que les sycomores qui croissent dans les plaines, tant il y en avait.
\VS{28}C’est d’Egypte que provenaient les chevaux de Salomon ; une caravane de marchands du roi allait les chercher par troupes, à un prix fixe :
\VS{29}Un char montait et sortait d'Egypte pour six cents sicles d'argent et chaque cheval pour cent cinquante sicles ; ils en amenaient de même avec eux pour tous les rois des Héthiens et pour les rois de Syrie.
\Chap{11}
\TextTitle{[Salomon détourne son cœur de Yahweh]}
\VerseOne{}Le roi Salomon aima plusieurs femmes étrangères, outre la fille de Pharaon ; savoir des Moabites, des Ammonites, des Edomites, des Sidoniennes et des Héthiennes.
\VS{2}Elles étaient d'entre les nations dont Yahweh avait dit aux enfants d'Israël : Vous n'irez point vers elles, et elles ne viendront point vers vous ; car certainement elles feraient détourner vos cœurs pour suivre leurs dieux. Salomon s'attacha à elles et les aima.
\VS{3}Il eut donc pour femmes sept cents princesses et trois cents concubines ; et ses femmes détournèrent son cœur.
\VS{4}Au temps de la vieillesse de Salomon, ses femmes firent détourner son cœur vers d'autres dieux ; et son cœur ne fut point intègre devant Yahweh, son Dieu, comme David, son père.
\VS{5}Salomon alla après Astarté, la divinité des Sidoniens, et après Milcom, l'abomination des Ammonites.
\VS{6}Ainsi Salomon fit ce qui est mal aux yeux de Yahweh, et il ne persévéra point à suivre Yahweh, comme David, son père.
\VS{7}Et Salomon bâtit un haut lieu à Kemosch, l'abomination des Moabites, sur la montagne qui est vis-à-vis de Jérusalem ; et à Moloc, l'abomination des fils d’Ammon.
\VS{8}Il en fit de même pour toutes ses femmes étrangères, qui offraient des parfums et des sacrifices à leurs dieux.
\VS{9}C'est pourquoi Yahweh fut irrité contre Salomon, parce qu'il avait détourné son cœur de Yahweh, le Dieu d'Israël, qui lui était apparu deux fois.
\VS{10}Il lui avait donné cet ordre de ne point aller après d'autres dieux ; mais il ne garda point ce que Yahweh lui avait ordonné.
\VS{11}Et Yahweh dit à Salomon : Puisque tu as agi de la sorte, et que tu n'as pas observé l’alliance et les ordonnances que je t'avais prescrites, je déchirerai le royaume afin qu'il ne soit plus à toi et je le donnerai à ton serviteur.
\VS{12}Toutefois je ne le ferai point en ton temps, pour l’amour de David, ton père. Ce sera d'entre les mains de ton fils que je déchirerai le royaume.
\VS{13}Néanmoins je ne déchirerai pas tout le royaume, j'en donnerai une tribu à ton fils, pour l'amour de David, mon serviteur, et pour l'amour de Jérusalem, que j'ai choisie.
\VS{14}Yahweh donc suscita un ennemi à Salomon, savoir Hadad, l’Edomite, qui était de la race royale d'Edom.
\VS{15}Car il était arrivé qu'au temps que David était en Edom, Joab, chef de l'armée, étant monté pour ensevelir les morts, tua tous les mâles qui étaient en Edom ;
\VS{16}Joab demeura là six mois avec tout Israël, jusqu'à ce qu'il eût exterminé tous les mâles d'Edom.
\VS{17}Ce fut alors qu’Hadad prit la fuite avec des Edomites d'entre les serviteurs de son père, pour se retirer en Egypte. Hadad était alors un jeune garçon.
\VS{18}Une fois partis de Madian, ils allèrent à Paran, prirent avec eux des hommes de Paran, et arrivèrent en Egypte auprès de Pharaon, roi d'Egypte, qui lui donna une maison, pourvut à sa subsistance et lui donna aussi une terre.
\VS{19}Et Hadad trouva grâce aux yeux de Pharaon, de sorte que Pharaon lui donna pour femme la sœur de sa propre femme, la sœur de la reine Thachpenès.
\VS{20}Et la sœur de Thachpenès lui enfanta son fils Guenubath. Thachpenès le sevra dans la maison de Pharaon. Ainsi Guenubath fut dans la maison de Pharaon, parmi les fils de Pharaon.
\VS{21}Lorsque Hadad apprit en Egypte que David s'était endormi avec ses pères, et que Joab, chef de l'armée, était mort, il dit à Pharaon : Laisse-moi partir dans mon pays.
\VS{22}Et Pharaon lui répondit : Que te manque-t-il auprès de moi, pour désirer ainsi t'en aller dans ton pays ? Et il répondit : Je n’ai besoin de rien, mais cependant laisse-moi partir.
\VS{23}Dieu suscita aussi un autre ennemi à Salomon, savoir Rezon, fils d'Eliada, qui s'était enfui de chez son maître Hadadézer, roi de Tsoba,
\VS{24}Il avait rassemblé des gens auprès de lui, et était devenu chef de bandes, lorsque David les fit périr ; et ils s'en allèrent à Damas, s’y établirent et y régnèrent.
\VS{25}Rezon fut ennemi d'Israël au temps de Salomon, en même temps qu’Hadad le mettait à mal, il avait en aversion Israël et il régna sur la Syrie.
\VS{26}Jéroboam aussi, serviteur de Salomon, s'éleva également contre le roi. Il était fils de Nebath, Ephratien, de Tseréda, dont la mère s’appelait Tserua, femme veuve.
\VS{27}Voici à quelle occasion il s'éleva contre le roi. Salomon bâtissait Millo, et fermait la brèche de la cité de David, son père.
\VS{28}Jéroboam était un homme fort et vaillant ; et Salomon, voyant ce jeune homme à l’ouvrage, lui assigna la charge de toute la maison de Joseph.
\VS{29}Dans ce même temps, Jéroboam, étant sorti de Jérusalem, rencontra en chemin le prophète Achija de Silo, revêtu d'un manteau neuf, et ils étaient eux deux tout seuls dans les champs.
\VS{30}Et Achija prit le manteau neuf qu'il avait sur lui et le déchira en douze morceaux,
\VS{31}et il dit à Jéroboam : Prends-en pour toi dix morceaux ! Car ainsi parle Yahweh, le Dieu d'Israël : Voici, je vais arracher le royaume d'entre les mains de Salomon, et je t'en donnerai dix tribus.
\VS{32}Mais il aura une tribu, pour l'amour de David, mon serviteur, et pour l'amour de Jérusalem, qui est la ville que j'ai choisie d'entre toutes les tribus d'Israël.
\VS{33}Parce qu'ils m'ont abandonné, et se sont prosternés devant Astarté, la déesse des Sidoniens, devant Kemosch, dieu de Moab, et devant Milcom, le dieu des fils d’Ammon, et qu'ils n'ont point marché dans mes voies, pour faire ce qui est droit à mes yeux et garder mes statuts, et mes ordonnances, comme l’a fait David, père de Salomon.
\VS{34}Toutefois, je n'ôterai pas de sa main tout le royaume, car pendant toute sa vie je le maintiendrai prince, pour l'amour de David, mon serviteur, que j'ai choisi et qui a observé mes commandements et mes lois.
\VS{35}Mais j'ôterai le royaume d'entre les mains de son fils, je t'en donnerai dix tribus ;
\VS{36}j'en donnerai une tribu à son fils, afin que David, mon serviteur, ait une lampe à toujours devant moi dans Jérusalem, qui est la ville que j'ai choisie pour y mettre mon Nom.
\VS{37}Je te prendrai donc, tu régneras sur tout ce que ton âme désirera, tu seras roi sur Israël.
\VS{38}Et il arrivera que si tu m'obéis en tout ce que je te commanderai, que tu marches dans mes voies, en faisant tout ce qui est droit à mes yeux, en gardant mes statuts et mes commandements, comme l’a fait David, mon serviteur, je serai avec toi, je te bâtirai une maison qui sera stable, comme j'en ai bâti une à David, et je te donnerai Israël.
\VS{39}Ainsi j’humilierai la postérité de David à cause de cela, mais non pas à toujours.
\VS{40}Salomon chercha à faire mourir Jéroboam, mais Jéroboam se leva et s'enfuit en Egypte vers Schischak, roi d'Egypte ; et il demeura en Egypte jusqu'à la mort de Salomon.
\TextTitle{[Mort de Salomon\FTNTT{2 Ch. 9:29-31}]}
\VS{41}Or, le reste des faits de Salomon, tout ce qu'il a fait et sa sagesse, cela n'est-il pas écrit dans le livre des actes de Salomon ?
\VS{42}Salomon régna à Jérusalem sur tout Israël pendant quarante ans.
\VS{43}Ainsi Salomon s'endormit avec ses pères, il fut enseveli dans la cité de David, son père. Et Roboam, son fils, régna en sa place.
\Chap{12}
\TextTitle{[Règne de Roboam\FTNTT{2 Ch. 10:1 ; cp. Ec. 2:18-19}]}
\VerseOne{}Roboam se rendit à Sichem, parce que tout Israël était venu à Sichem pour l'établir roi.
\VS{2}Or, Jéroboam, fils de Nebath, était encore en Egypte, où il s'était enfui de devant le roi Salomon, quand il l'apprit, et c’était en Egypte qu’il habitait.
\VS{3}On l'envoya appeler. Ainsi Jéroboam et toute l'assemblée d'Israël vinrent, ils parlèrent à Roboam, en disant :
\VS{4}Ton père a mis sur nous un pesant joug ; mais toi allège maintenant cette rude servitude de ton père et ce pesant joug qu'il a mis sur nous ; et nous te servirons.
\VS{5}Il leur répondit : Allez, et dans trois jours revenez vers moi. Et le peuple s'en alla.
\VS{6}Le roi Roboam consulta les vieillards qui avaient été auprès de Salomon, son père, pendant sa vie et leur dit : Que me conseillez-vous de répondre à ce peuple ?
\VS{7}Et ils lui répondirent, en disant : Si aujourd'hui tu rends service à ce peuple et que tu leur cèdes, et si tu leur réponds avec des paroles bienveillantes, ils seront tes serviteurs à toujours.
\VS{8}Mais Roboam laissa le conseil que les vieillards lui avaient donné et consulta les jeunes gens qui avaient grandi avec lui et qui se tenaient près de lui.
\VS{9}Il leur dit : Que me conseillez-vous de répondre à ce peuple qui m'a parlé, en disant : Allège le joug que ton père a mis sur nous ?
\VS{10}Alors les jeunes gens qui avaient grandi avec lui, lui dirent : Tu parleras ainsi à ce peuple qui t'est venu dire : Ton père a mis sur nous un pesant joug, mais toi allège-le-nous ! Tu leur parleras ainsi : Mon petit doigt est plus gros que les reins de mon père.
\VS{11}Or, mon père a mis sur vous un pesant joug, mais moi je rendrai votre joug encore plus pesant ; mon père vous a châtiés avec des fouets, mais moi je vous châtierai avec des scorpions.
\VS{12}Or, trois jours après, Jéroboam avec tout le peuple vint vers Roboam, selon que le roi leur avait dit : Retournez vers moi dans trois jours.
\VS{13}Mais le roi répondit durement au peuple, laissant le conseil que les anciens lui avaient donné.
\VS{14}Il leur parla selon le conseil des jeunes gens, en leur disant : Mon père a mis sur vous un pesant joug, mais moi, je rendrai votre joug plus pesant encore ; mon père vous a châtiés avec des fouets, mais moi, je vous châtierai avec des scorpions.
\VS{15}Le roi donc n'écouta point le peuple ; car cela était ainsi conduit par Yahweh, en vue d’accomplir la parole qu'il avait prononcée par le ministère d'Achija de Silo, à Jéroboam, fils de Nebath.
\TextTitle{[Jéroboam se se détache de la maison de David avec Israël\FTNTT{2 Ch. 10:12-19 ; 11:1-4}]}
\VS{16}Et quand tout Israël vit que le roi ne les avait point écoutés, le peuple fit cette réponse au roi, en disant : Quelle part avons-nous avec David ? Nous n'avons point de propriété avec le fils d'Isaï ! A tes tentes, Israël ! Et toi David, pourvois maintenant à ta maison ! Ainsi Israël s'en alla dans ses tentes.
\VS{17}Les fils d'Israël qui habitaient dans les villes de Juda furent les seuls sur qui Roboam régna.
\VS{18}Or, le roi Roboam envoya Adoram, qui était préposé aux impôts, mais tout Israël le lapida, et il mourut. Alors le roi Roboam se hâta de monter sur un char pour s'enfuir à Jérusalem.
\VS{19}C’est ainsi qu’Israël s’est détaché de la maison de David jusqu'à ce jour.
\VS{20}Tout Israël apprit que Jéroboam était de retour, ils l'envoyèrent appeler dans l'assemblée, et l'établirent roi sur tout Israël. La tribu de Juda fut la seule qui suivit la maison de David.
\VS{21}Roboam arriva à Jérusalem, il rassembla toute la maison de Juda et la tribu de Benjamin, savoir cent quatre-vingt mille hommes d’élite choisis et disposés à faire la guerre, pour combattre contre la maison d'Israël, et ramener la domination à Roboam, fils de Salomon.
\VS{22}Mais la parole de Dieu fut ainsi adressée à Schemaeja, homme de Dieu, disant :
\VS{23}Parle à Roboam, fils de Salomon, roi de Juda, et à toute la maison de Juda, et de Benjamin, et au reste du peuple, en disant :
\VS{24}Ainsi parle Yahweh : Vous ne monterez point et vous ne combattrez point contre vos frères, les fils d'Israël ! Que chacun de vous retourne dans sa maison, car ceci a été fait de par moi. Ils obéirent à la parole de Yahweh, et s'en retournèrent, selon la parole de Yahweh.
\TextTitle{[Idolatrie de Jéroboam]}
\VS{25}Or, Jéroboam bâtit Sichem sur la montagne d'Ephraïm, et y demeura, puis il en sortit et bâtit Penuel.
\VS{26}Et Jéroboam dit en son cœur : Maintenant le royaume pourrait bien retourner à la maison de David.
\VS{27}Si ce peuple monte à Jérusalem pour faire des sacrifices dans la maison de Yahweh, le cœur de ce peuple se tournera vers son seigneur, Roboam, roi de Juda, et ils me tueront, et ils retourneront à Roboam, roi de Juda.
\VS{28}Sur quoi le roi ayant pris conseil, fit deux veaux d'or et dit au peuple : Vous êtes longtemps montés à Jérusalem ! Voici ton dieu, ô Israël, qui t'a fait sortir hors du pays d'Egypte.
\VS{29}Il plaça un de ces veaux à Béthel, et il mit l'autre à Dan.
\VS{30}Et cela fut une occasion de péché, car le peuple allait jusqu'à Dan, pour se prosterner devant l'un des veaux.
\VS{31}Il fit aussi des maisons dans les hauts lieux, et établit des sacrificateurs pris parmi tout le peuple, qui n'étaient point des enfants de Lévi.
\VS{32}Jéroboam ordonna aussi une fête solennelle au huitième mois, le quinzième jour du mois, à l'imitation de la fête solennelle qu'on célébrait en Juda, et il offrait des sacrifices sur un autel. Il fit ainsi à Béthel, sacrifiant aux veaux qu'il avait faits, et il établit à Béthel des sacrificateurs des hauts lieux qu'il avait élevés.
\VS{33}Or, le quinzième jour du huitième mois, savoir au mois qu'il avait choisi lui-même, il monta sur l'autel qu'il avait fait à Béthel, et célébra cette fête solennelle pour les enfants d'Israël ; et fit brûler des parfums sur l'autel.
\Chap{13}
\TextTitle{[Un prophète envoyé vers Jéroboam]}
\VerseOne{}Et voici, un homme de Dieu vint de Juda à Béthel avec la parole de Yahweh, pendant que Jéroboam se tenait près de l'autel pour brûler des parfums.
\VS{2}Et il cria contre l'autel selon la parole de Yahweh, et dit : Autel ! Autel ! Ainsi parle Yahweh : Voici, un fils naîtra à la maison de David, qui aura pour nom Josias ; il immolera sur toi les sacrificateurs des hauts lieux qui brûlent des parfums sur toi, et on brûlera sur toi des ossements d’hommes !
\VS{3}Le même jour il donna un signe, en disant : C'est ici le signe dont Yahweh a parlé : Voici, l'autel se fendra, et la cendre qui est dessus sera répandue.
\VS{4}Lorsque le roi entendit la parole que l'homme de Dieu avait criée contre l'autel de Béthel, Jéroboam étendit sa main de l'autel, en disant : Saisissez-le ! Et la main qu'il étendit contre lui devint sèche, et il ne put la ramener à lui.
\VS{5}L'autel aussi se fendit, et la cendre qui était sur l'autel fut répandue, selon le signe que l'homme de Dieu avait donné par la parole de Yahweh.
\VS{6}Alors le roi prit la parole et dit à l'homme de Dieu : Implore Yahweh, ton Dieu, et prie pour moi, afin que ma main revienne à moi. L'homme de Dieu implora Yahweh, et la main du roi put revenir à lui et elle fut comme auparavant.
\VS{7}Alors le roi dit à l'homme de Dieu : Entre avec moi dans la maison, tu prendras quelque nourriture et je te donnerai un présent.
\VS{8}Mais l'homme de Dieu répondit au roi : Quand tu me donnerais la moitié de ta maison, je n'entrerais point chez toi, je ne mangerais point de pain, ni ne boirais d'eau en ce lieu.
\VS{9}Car cela m'a été ordonné par Yahweh, qui m'a dit : Tu ne mangeras point de pain, tu ne boiras point d'eau et tu ne t'en retourneras point par le chemin par lequel tu y seras allé.
\VS{10}Il s'en alla donc par un autre chemin, et ne s'en retourna point par le chemin par lequel il était venu à Béthel.
\TextTitle{[L'homme de Dieu écoute le vieux prophète plutôt que Dieu]}
\VS{11}Or, il y avait un vieux prophète qui demeurait à Béthel. Ses fils vinrent raconter toutes les choses que l'homme de Dieu avait faites ce jour-là à Béthel, et les paroles qu'il avait dites au roi ; et comme les fils de ce prophète les rapportaient à leur père,
\VS{12}il leur demanda : Par quel chemin s'en est-il allé ? Or, ses fils avaient vu le chemin par lequel l'homme de Dieu qui était venu de Juda s'en était allé.
\VS{13}Et il dit à ses fils : Sellez-moi un âne. Ils lui sellèrent, puis il monta dessus.
\VS{14}Et il s'en alla après l'homme de Dieu, et le trouva assis sous un chêne. Et il lui dit : Es-tu l'homme de Dieu qui est venu de Juda ? Et il lui répondit : C'est moi.
\VS{15}Alors il lui dit : Viens avec moi dans la maison, et tu prendras de quoi te nourrir.
\VS{16}Mais il répondit : Je ne puis retourner avec toi, ni entrer chez toi et je ne mangerai point de pain, ni ne boirai d'eau avec toi en ce lieu ;
\VS{17}Car il m'a été dit de la part de Yahweh : Tu ne mangeras point de pain, tu ne boiras point d'eau, et tu ne t'en retourneras point par le chemin par lequel tu seras allé.
\VS{18}Et il lui dit : Et moi aussi je suis prophète comme toi ; et un ange m'a parlé de la part de Yahweh, en disant : Ramène-le avec toi dans ta maison, qu'il mange du pain, et qu'il boive de l'eau ; mais il lui mentait.
\VS{19}Il s'en retourna donc avec lui, il mangea du pain et but de l'eau dans sa maison.
\VS{20}Et il arriva que comme ils étaient assis à table, la parole de Yahweh fut adressée au prophète qui l'avait ramené.
\VS{21}Et il cria à l'homme de Dieu qui était venu de Juda, en disant : Ainsi a parlé Yahweh : Parce que tu as été rebelle au commandement de Yahweh et que tu n'as point gardé l’ordre que Yahweh, ton Dieu, t'avait donné ;
\VS{22}mais tu t'en es retourné, tu as mangé du pain et bu de l'eau dans le lieu dont Yahweh t'avait dit : N'y mange point de pain et n'y bois point d'eau, ton cadavre n'entrera point au sépulcre de tes pères.
\VS{23}Et quand le prophète qu'il avait ramené eut mangé du pain et bu de l’eau, il sella l’âne pour lui.
\VS{24}L’homme de Dieu s'en alla, et un lion le rencontra dans le chemin, et le tua. Son corps était étendu dans le chemin, l'âne resta auprès du corps, et le lion aussi resta à côté du cadavre.
\VS{25}Et voici des passants virent le corps étendu dans le chemin et le lion qui se tenait auprès du corps ; et ils vinrent le dire dans la ville où le vieux prophète demeurait.
\VS{26}Et le prophète qui avait ramené du chemin l'homme de Dieu, l'ayant appris, dit : C'est l'homme de Dieu qui a été rebelle au commandement de Yahweh, c'est pourquoi Yahweh l'a livré au lion, qui l'aura déchiré après l'avoir tué, selon la parole que Yahweh avait dite à ce prophète.
\VS{27}Et il parla à ses fils, en disant : Sellez-moi un âne. Ils le lui sellèrent,
\VS{28}Et il s'en alla et trouva le corps de l'homme de Dieu étendu dans le chemin, l'âne et le lion qui se tenaient auprès du corps. Le lion n'avait pas dévoré le cadavre, ni déchiré l'âne.
\VS{29}Alors le prophète leva le corps de l'homme de Dieu, le plaça sur l'âne et le ramena ; et ce vieux prophète revint dans la ville pour le pleurer et l'enterrer.
\VS{30}Il mit le corps de ce prophète dans le sépulcre, et il pleura sur lui, en disant : Hélas, mon frère !
\VS{31}Après l’avoir enterré, il parla à ses fils, en disant : Quand je serai mort, enterrez-moi au sépulcre où est enterré l'homme de Dieu, et vous déposerez mes os à côté de ses os.
\VS{32}Car elle s’accomplira, la parole qu’il a criée de la part de Yahweh, contre l'autel qui est à Béthel et contre toutes les maisons des hauts lieux qui sont dans les villes de Samarie.
\TextTitle{[Jéroboam continue dans le mal]}
\VS{33}Néanmoins, Jéroboam ne se détourna point de sa mauvaise voie, mais il établit de nouveau des sacrificateurs de hauts lieux pris parmi tout le peuple ; quiconque le voulait, Jéroboam le consacrait sacrificateur des hauts lieux.
\VS{34}Cela fut une occasion de péché pour la maison de Jéroboam, qui fut effacée et exterminée de dessus la terre.
\Chap{14}
\TextTitle{[Maladie et mort du fils de Jéroboam]}
\VerseOne{}En ce temps-là, Abija, fils de Jéroboam, devint malade.
\VS{2}Et Jéroboam dit à sa femme : Lève-toi maintenant et déguise-toi, en sorte qu'on ne reconnaisse point que tu es la femme de Jéroboam, et va à Silo. Voici, là est Achija, le prophète, qui m'a dit que je serais roi sur ce peuple.
\VS{3}Emmène avec toi dix pains, des gâteaux et un vase de miel, et entre chez lui ; il te dira ce qui arrivera à l’enfant.
\VS{4}La femme de Jéroboam fit donc ainsi ; elle se leva et s'en alla à Silo puis elle entra dans la maison d'Achija. Or, Achija ne pouvait plus voir, parce qu’il avait les yeux figés à cause de sa vieillesse.
\VS{5}Et Yahweh dit à Achija : Voici, la femme de Jéroboam, qui vient te consulter concernant l’état de son fils, parce qu'il est malade. Tu lui parleras de telle et de telle manière. Quand elle arrivera, elle se sera déguisée.
\VS{6}Lorsque Achija eut entendu le bruit de ses pas, comme elle franchissait la porte, il dit : Entre, femme de Jéroboam. Pourquoi fais-tu semblant d'être quelqu’un d’autre ? Je suis chargé de t’annoncer des choses dures.
\VS{7}Va, dis à Jéroboam : Ainsi parle Yahweh, le Dieu d'Israël : Parce que je t'ai élevé du milieu du peuple et que je t'ai établi pour chef sur mon peuple d'Israël,
\VS{8}j'ai arraché le royaume de la maison de David et je te l'ai donné ; mais parce que tu n'as point été comme David, mon serviteur, qui a gardé mes commandements et qui a marché après moi de tout son cœur, ne faisant que ce qui est droit à mes yeux.
\VS{9}Tu as fait pire que tous ceux qui ont été devant toi, tu es allé te faire d'autres dieux et des images de fonte, pour m'irriter, et tu m'as rejeté derrière ton dos !
\VS{10}A cause de cela, voici, je vais faire venir le malheur sur la maison de Jéroboam ; je retrancherai ce qui appartient à Jéroboam, ce qu’il détient et ce qu’il néglige en Israël, et je brûlerai la maison de Jéroboam, comme on brûle les ordures, jusqu'à ce qu'il n'en reste plus.
\VS{11}Celui de la maison de Jéroboam qui mourra dans la ville, les chiens le mangeront, et celui qui mourra aux champs, les oiseaux du ciel le mangeront. Car Yahweh a parlé.
\VS{12}Toi donc lève-toi, va dans ta maison. Dès que tes pieds entreront dans la ville, l'enfant mourra.
\VS{13}Tout Israël le pleurera et on l’enterrera ; car lui seul de la famille de Jéroboam entrera au sépulcre, parce que Yahweh, le Dieu d'Israël, a trouvé quelque chose de bon en lui seul dans toute la maison de Jéroboam.
\VS{14}Yahweh s'établira un roi sur Israël qui retranchera la maison de Jéroboam. Ce jour-là, n’est-ce pas déjà ce qui arrive ?
\VS{15}Yahweh frappera Israël, l'agitant comme le roseau est agité dans l'eau ; et il arrachera Israël de ce bon pays qu'il a donné à leurs pères, et les dispersera au-delà du fleuve, parce qu'ils se sont fait des idoles, irritant Yahweh.
\VS{16}Il livrera Israël à cause des péchés que Jéroboam a commis et qu’il a fait commettre à Israël.
\VS{17}Alors la femme de Jéroboam se leva et s'en alla, elle vint à Thirtsa : et comme elle franchit le seuil de la maison, le jeune garçon mourut.
\VS{18}Il fut enseveli et tout Israël le pleura, selon la parole de Yahweh, proférée par son serviteur Achija, le prophète.
\TextTitle{[Nadab, roi d'Israël\FTNTT{cp. 2 Ch. 13:20}]}
\VS{19}Quant au reste des faits de Jéroboam, comment il a fait la guerre et comment il a régné, cela est écrit dans le livre des Chroniques des rois d'Israël.
\VS{20}Jéroboam régna vingt-deux ans, puis il s'endormit avec ses pères. Et Nadab, son fils, régna à sa place.
\TextTitle{[Juda dans l'apostasie\FTNTT{2 Ch. 12:1}]}
\VS{21}Roboam, fils de Salomon, régna en Juda. Il avait quarante et un ans quand il devint roi, et il régna dix-sept ans à Jérusalem, la ville que Yahweh avait choisie d'entre toutes les tribus d'Israël pour y mettre son nom. Sa mère s’appelait Naama, l’Ammonite.
\VS{22}Juda fit ce qui est mal aux yeux de Yahweh ; et par les péchés qu'ils commirent, ils excitèrent sa jalousie plus que leurs pères ne l'avaient jamais fait.
\VS{23}Ils se bâtirent, eux aussi, des hauts lieux avec des statues et des idoles sur toute colline élevée, et sous tout arbre verdoyant.
\VS{24}Il y avait dans le pays des prostitués. Et ils firent selon toutes les abominations des nations que Yahweh avait chassées devant les enfants d'Israël.
\TextTitle{[Schischak, roi d'Egypte monte contre Juda, mort de Roboam\FTNTT{2 Ch. 12:2-16}]}
\VS{25}La cinquième année du roi Roboam, Schischak, roi d'Egypte, monta contre Jérusalem.
\VS{26}Il prit les trésors de la maison de Yahweh et les trésors de la maison royale, et il emporta tout. Il prit aussi tous les boucliers d'or que Salomon avait faits.
\VS{27}Le roi Roboam fit des boucliers d'airain au lieu de ceux-là, et les mit entre les mains des chefs des coureurs, qui gardaient l’entrée de la maison du roi.
\VS{28}Toutes les fois où le roi entrait dans la maison de Yahweh, les coureurs les portaient, et ensuite ils les rapportaient dans la chambre des coureurs.
\VS{29}Le reste des actions de Roboam, et tout ce qu'il a fait, n'est-il pas écrit au livre des Chroniques des rois de Juda ?
\VS{30}Il y eut toujours guerre entre Roboam et Jéroboam.
\VS{31}Roboam s'endormit avec ses pères et fut enseveli avec eux dans la cité de David. Sa mère avait pour nom Naama, l’Ammonite. Et Abijam, son fils, régna à sa place.
\Chap{15}
\TextTitle{[Règne d'Abijam (ou Abija) sur Juda\FTNTT{2 Ch. 13:1-2}]}
\VerseOne{}La dix-huitième année du roi Jéroboam, fils de Nebath, Abijam commença à régner sur Juda.
\VS{2}Il régna trois ans à Jérusalem. Sa mère s’appelait Maaca et était fille d'Abisalom.
\VS{3}Il marcha dans tous les péchés que son père avait commis avant lui ; son cœur ne fut point intègre envers Yahweh, son Dieu, comme l'avait été le cœur de David, son père.
\VS{4}Mais pour l'amour de David, Yahweh, son Dieu, lui donna une lampe dans Jérusalem, lui suscitant son fils après lui et laissant subsister Jérusalem ;
\VS{5}Parce que David avait fait ce qui est droit devant Yahweh, et que pendant toute sa vie il ne s'était point détourné d’aucun de ses commandements, hormis dans l'affaire d'Urie, le Héthien.
\VS{6}Or, il y eut toujours guerre entre Roboam et Jéroboam, pendant toute la vie de Roboam.
\VS{7}Le reste des actions d'Abijam, et même tout ce qu'il fit, n'est-il pas écrit au livre des Chroniques des rois de Juda ? Il y eut aussi guerre entre Abijam et Jéroboam.
\VS{8}Ainsi Abijam s'endormit avec ses pères, et on l'enterra dans la cité de David. Et Asa, son fils, régna à sa place.
\TextTitle{[Asa, roi de Juda\FTNTT{2 Ch. 14:1-5 ; 15:1-19}]}
\VS{9}La vingtième année de Jéroboam, roi d'Israël, Asa commença à régner sur Juda.
\VS{10}Il régna quarante et un ans à Jérusalem. Sa mère avait pour nom Maaca, elle était fille d'Abisalom.
\VS{11}Asa fit ce qui est droit devant Yahweh, comme David, son père.
\VS{12}Il ôta du pays les prostitués, et ôta toutes les idoles que ses pères avaient faites.
\VS{13}Et même il ôta la dignité de reine à sa mère Maaca, parce qu'elle avait fait une idole pour Astarté. Asa mit en pièces l’idole qu'elle avait faite, et la brûla au torrent de Cédron.
\VS{14}Mais les hauts lieux ne furent point ôtés. Néanmoins, le cœur d'Asa fut intègre envers Yahweh pendant toute sa vie.
\TextTitle{[Asa s'allie avec le roi de Syrie contre le  roi d'Israël\FTNTT{1 Ch. 14:6-15 ; 16:1-10}]}
\VS{15}Il remit dans la maison de Yahweh les choses qui avaient été consacrées par son père et par lui-même, de l'argent, de l'or et les ustensiles.
\VS{16}Or, il y eut guerre entre Asa et Baescha, roi d'Israël, pendant toute leur vie.
\VS{17}Baescha, roi d'Israël, monta contre Juda, et bâtit Rama, pour empêcher quiconque de sortir et entrer vers Asa, roi de Juda.
\VS{18}Asa prit tout l'argent et l'or qui était resté dans les trésors de Yahweh et dans les trésors de la maison royale, et les donna à ses serviteurs ; le roi Asa les envoya vers Ben-Hadad, fils de Thabrimmon, fils de Hezjon, roi de Syrie, qui demeurait à Damas, pour lui dire :
\VS{19}Qu’il y ait alliance entre moi et toi, comme entre mon père et le tien. Voici, je t'envoie un présent en argent et en or. Va, romps l'alliance que tu as avec Baescha, roi d'Israël, afin qu'il se retire de moi.
\VS{20}Et Ben-Hadad écouta le roi Asa ; il envoya les chefs de son armée contre les villes d'Israël, et il battit Ijjon, Dan, Abel-Beth-Maaca, tout Kinneroth, et tout le pays de Nephthali.
\VS{21}Lorsque Baescha l’apprit, il cessa de bâtir Rama et demeura à Thirtsa.
\VS{22}Alors le roi Asa fit publier par tout Juda que tous, sans en excepter aucun, eussent à emporter les pierres et le bois de Rama, que Baescha faisait bâtir, et le roi Asa s’en servit pour bâtir Guéba de Benjamin, et Mitspa.
\TextTitle{[Maladie et mort d'Asa, Josaphat roi de Juda\FTNTT{1 Ch. 16:11-17:1}]}
\VS{23}Le reste de toutes les actions d'Asa, tous ses exploits, tout ce qu'il fit, et les villes qu'il a bâties, cela n'est-il pas écrit au livre des Chroniques des rois de Juda ? Au reste, il fut malade de ses pieds au temps de sa vieillesse.
\VS{24}Et Asa s'endormit avec ses pères, avec lesquels il fut enseveli en la cité de David, son père. Et, son fils, Josaphat, régna à sa place.
\TextTitle{[Baescha tue Nadab et devient roi d'Israël]}
\VS{25}Or, Nadab, fils de Jéroboam, régna sur Israël la seconde année d'Asa, roi de Juda, et il régna deux ans sur Israël.
\VS{26}Il fit ce qui est mal aux yeux de Yahweh ; et il marcha dans la voie de son père, se livrant aux péchés que son père avait fait commettre à Israël.
\VS{27}Et Baescha, fils d'Achija, de la maison d'Issacar, fit une conspiration contre lui. Il le tua devant Guibbethon, qui était aux Philistins, lorsque Nadab et tout Israël assiégeaient Guibbethon.
\VS{28}Baescha le fit donc mourir la troisième année d'Asa, roi de Juda, et il régna à sa place.
\VS{29}Une fois proclamé roi, il frappa toute la maison de Jéroboam et ne laissa échapper aucune âme vivante, il détruisit tout ce qui respirait, selon la parole de Yahweh qu'il avait proférée par son serviteur Achija, Silonite,
\VS{30}A cause des péchés que Jéroboam avait commis et fait commettre à Israël, irritant ainsi Yahweh, le Dieu d'Israël.
\VS{31}Le reste des faits de Nadab, et même tout ce qu'il a fait, n'est-il pas écrit au livre des Chroniques des rois d'Israël ?
\VS{32}Or, il y eut guerre entre Asa et Baescha, roi d'Israël, pendant toute leur vie.
\VS{33}La troisième année d'Asa, roi de Juda, Baescha, fils d'Achija, commença à régner sur tout Israël à Thirtsa, il régna vingt-quatre ans.
\VS{34}Et il fit ce qui est mal aux yeux de Yahweh, et marcha dans la voie de Jéroboam, en se livrant aux péchés que Jéroboam avait fait commettre à Israël.
\Chap{16}
\TextTitle{[Yahweh avertit Baescha avant sa mort]}
\VerseOne{}Alors la parole de Yahweh fut adressée à Jéhu, fils de Hanani, contre Baescha, en ces mots :
\VS{2}Je t'ai élevé de la poussière et je t'ai établi chef de mon peuple d'Israël ; malgré cela tu as suivi la voie de Jéroboam et fait pécher mon peuple d'Israël, pour m'irriter par leurs péchés.
\VS{3}Voici, je m'en vais entièrement consumer Baescha et sa maison, et je rendrai ta maison semblable à la maison de Jéroboam, fils de Nebath.
\VS{4}Celui de la maison de Baescha qui mourra dans la ville, les chiens le mangeront, et celui des siens qui mourra aux champs, les oiseaux du ciel le mangeront.
\VS{5}Le reste des faits de Baescha, ce qu'il a fait et ses exploits, n'est-il pas écrit au livre des Chroniques des rois d'Israël ?
\VS{6}Ainsi Baescha s'endormit avec ses pères et fut enseveli à Thirtsa. Ela, son fils, régna à sa place.
\VS{7}La parole de Yahweh fut aussi adressée par le moyen de Jéhu, fils de Hanani, le prophète, contre Baescha et contre sa maison, tant à cause de tout le mal qu'il avait fait devant Yahweh, en l'irritant par l'œuvre de ses mains et en devenant comme la maison de Jéroboam, que parce qu'il l'avait détruite.
\TextTitle{[Ela, puis Zimri, règnent sur Israël]}
\VS{8}La vingt-sixième année d'Asa, roi de Juda, Ela, fils de Baescha, commença à régner sur Israël et il régna deux ans à Thirtsa.
\VS{9}Son serviteur, Zimri, capitaine de la moitié des chars, fit une conspiration contre Ela, lorsqu'il était à Thirtsa, buvant et s'enivrant dans la maison d'Artsa, chef de la maison du roi à Thirtsa.
\VS{10}Alors, Zimri vint, le frappa et le tua, la vingt-septième année d'Asa, roi de Juda, et il régna à sa place.
\VS{11}Dès qu’il fut roi et qu'il fut assis sur son trône, il frappa toute la maison de Baescha, il n'en laissa échapper personne qui lui appartint, ni parent, ni ami.
\VS{12}Ainsi Zimri extermina toute la maison de Baescha, selon la parole que Yahweh avait proférée contre Baescha, par Jéhu, le prophète,
\VS{13}A cause de tous les péchés de Baescha, et des péchés d'Ela, son fils, qu’ils avaient commis et qu’ils avaient fait commettre à Israël, irritant Yahweh, le Dieu d'Israël, par leurs idoles.
\VS{14}Le reste des faits d'Ela, et même tout ce qu'il a fait, n'est-il pas écrit au livre des Chroniques des rois d'Israël ?
\VS{15}La vingt-septième année d'Asa, roi de Juda, Zimri régna sept jours à Thirtsa. Or, le peuple était campé contre Guibbethon qui appartenait aux Philistins.
\VS{16}Et le peuple qui était campé là entendit que l'on disait : Zimri a fait une conspiration, et il a même tué le roi ! En ce même jour, tout Israël établit dans le camp pour roi d’Israël Omri, chef de l'armée d'Israël.
\VS{17}Omri et tout Israël avec lui partirent de Guibbethon, et assiégèrent Thirtsa.
\VS{18}Mais dès que Zimri vit que la ville était prise, il entra au palais de la maison royale et brûla sur lui la maison royale, il mourut ainsi,
\VS{19}A cause des péchés qu’il avait commis, faisant ce qui est mal aux yeux de Yahweh, en suivant la voie de Jéroboam et le péché qu'il avait fait commettre à Israël.
\VS{20}Le reste des actions de Zimri et la conspiration qu'il forma, cela n’est-il pas écrit dans le livre des Chroniques des rois d'Israël ?
\TextTitle{[Omri, roi d'Israël]}
\VS{21}Alors le peuple d'Israël se divisa en deux partis : la moitié du peuple voulait faire roi Thibni, fils de Guinath ; et l'autre moitié suivait Omri.
\VS{22}Mais le peuple qui suivait Omri, fut plus fort que le peuple qui suivait Thibni, fils de Guinath. Thibni mourut et Omri régna.
\VS{23}La trente et unième année d'Asa, roi de Juda, Omri commença à régner sur Israël et il régna douze ans après avoir régné six ans à Thirtsa.
\VS{24}Puis il acheta de Schémer la montagne de Samarie, deux talents d'argent ; il bâtit une ville sur cette montagne et nomma la ville qu'il bâtit, du nom de Schémer, seigneur de la montagne.
\VS{25}Omri fit ce qui est mal aux yeux de Yahweh ; il agit même plus mal que tous ceux qui avaient été avant lui.
\VS{26}Il marcha dans la voie de Jéroboam, fils de Nebath, et se livra aux péchés que Jéroboam avait fait commettre à Israël, irritant Yahweh, le Dieu d'Israël, par leurs idoles.
\VS{27}Le reste des actions d’Omri, tout ce qu'il a fait et ses exploits, cela n’est-il pas écrit au livre des Chroniques des rois d'Israël ?
\VS{28}Ainsi Omri s'endormit avec ses pères et fut enseveli à Samarie. Achab, son fils, régna à sa place.
\TextTitle{[Achab roi d'Israël et Jézabel sa femme]}
\VS{29}Achab, fils d’Omri, régna sur Israël la trente-huitième année d'Asa, roi de Juda. Et Achab, fils d’Omri, régna sur Israël à Samarie vingt-deux ans.
\VS{30}Et Achab, fils d’Omri, fit ce qui est mal aux yeux de Yahweh, plus que tous ceux qui avaient été avant lui.
\VS{31}Et il arriva que, comme si ce lui eût été peu de chose de marcher dans les péchés de Jéroboam, fils de Nebath, il prit pour femme Jézabel, fille d'Ethbaal, roi des Sidoniens, puis il alla servir Baal et se prosterna devant lui.
\VS{32}Il dressa un autel à Baal, dans la maison de Baal, qu'il bâtit à Samarie.
\VS{33}Et Achab fit une idole d’Astarté. De sorte qu'Achab fit plus encore que tous les rois d'Israël qui avaient été avant lui, pour irriter Yahweh, le Dieu d'Israël.
\VS{34}En son temps, Hiel de Béthel bâtit Jéricho ; il en jeta les fondements au prix d’Abiram, son premier-né, et posa ses portes sur Segub, son plus jeune fils, selon la parole que Yahweh avait proférée par le moyen de Josué, fils de Nun.
\Chap{17}
\TextTitle{[Elie annonce trois ans de famine\FTNTT{1 R. 17-2 R. 1}]}
\VerseOne{}Alors Elie, Thischbite, l’un des habitants de Galaad, dit à Achab : Yahweh, le Dieu d'Israël, en la présence duquel je me tiens, est vivant ! Il n'y aura ces années-ci ni rosée ni pluie, sinon à ma parole.
\TextTitle{[Elie à Kerith]}
\VS{2}Puis la parole de Yahweh fut adressée à Elie, en disant :
\VS{3}Va-t'en d'ici et tourne-toi vers l'orient ; cache-toi près du torrent de Kerith, qui est en face du Jourdain.
\VS{4}Tu boiras de l’eau du torrent, et j'ai commandé aux corbeaux de t'y nourrir.
\VS{5}Il partit donc et fit selon la parole de Yahweh, il s'en alla et demeura au torrent de Kerith, vis-à-vis du Jourdain.
\VS{6}Les corbeaux lui apportaient du pain et de la viande le matin, et du pain et de la viande le soir, et il buvait de l’eau du torrent.
\VS{7}Mais il arriva qu'au bout d’un certain temps le torrent tarit, parce qu'il n'y avait point eu de pluie dans le pays.
\TextTitle{[Elie à Sarepta]}
\VS{8}Alors la parole de Yahweh lui fut adressée, en ces mots :
\VS{9}Lève-toi, va à Sarepta, qui appartient à Sidon, et demeure-là. Voici, j'ai commandé là à une femme veuve de t'y nourrir.
\VS{10}Il se leva donc et s'en alla à Sarepta. Et comme il fut arrivé à l’entrée de la ville, voici, une femme veuve était là, qui ramassait du bois. Et il l'appela et lui dit : Apporte-moi, je te prie, un peu d'eau dans un vase et que je boive.
\VS{11}Elle alla en chercher. Il l’appela de nouveau et dit : Apporte-moi, je te prie, un morceau de pain de ta main.
\VS{12}Mais elle répondit : Yahweh, ton Dieu, est vivant ! Je n'ai rien de cuit, je n'ai qu’une poignée de farine dans un pot et un peu d'huile dans une cruche. Et voici, j'amasse deux morceaux de bois, puis je rentrerai, je l'apprêterai pour moi et pour mon fils, nous le mangerons, après quoi nous mourrons.
\VS{13}Et Elie lui dit : Ne crains point, va, fais comme tu dis. Seulement, fais-moi d’abord avec cela un petit gâteau et tu me l’apporteras, tu en feras ensuite pour toi et pour ton fils.
\VS{14}Car ainsi parle Yahweh, le Dieu d'Israël : La farine qui est dans le pot ne finira point et l'huile qui est dans la cruche ne diminuera point, jusqu'à ce que Yahweh donne de la pluie sur la terre.
\VS{15}Elle s'en alla donc, et fit selon la parole d'Elie. Et elle eut à manger, elle et sa famille, ainsi qu’Elie pendant plusieurs jours.
\VS{16}La farine du pot ne finit point, et l'huile de la cruche ne diminua point, selon la parole que Yahweh avait prononcée par le moyen d'Elie.
\TextTitle{[Résurrection du fils de la veuve à Sarepta]}
\VS{17}Après ces choses, il arriva que le fils de la femme, maîtresse de la maison, devint malade ; et la maladie fut si forte, qu'il expira.
\VS{18}Et elle dit à Elie : Qu'y a-t-il entre moi et toi, homme de Dieu ? Es-tu venu chez moi pour rappeler le souvenir de mon iniquité, et pour faire mourir mon fils ?
\VS{19}Et il lui dit : Donne-moi ton fils. Et il le prit du sein de cette femme, le porta dans la chambre haute où il demeurait, et le coucha sur son lit.
\VS{20}Puis il cria à Yahweh, et dit : Yahweh, mon Dieu ! Affligeras-tu cette veuve au point de faire mourir son fils, elle qui a m’a reçu comme un hôte ?
\VS{21}Et il s'étendit sur l'enfant par trois fois, et cria à Yahweh, en disant : Yahweh, mon Dieu ! Je te prie que l'âme de cet enfant revienne au-dedans de lui.
\VS{22}Et Yahweh écouta la voix d'Elie, l'âme de l'enfant revint au-dedans de lui, et il fut rendu à la vie.
\VS{23}Elie prit l'enfant, le descendit de la chambre haute dans la maison et le donna à sa mère, en lui disant : Regarde, ton fils est vivant.
\VS{24}Et la femme dit à Elie : Je reconnais maintenant, que tu es un homme de Dieu et que la parole de Yahweh, qui est dans ta bouche, est vérité.
\Chap{18}
\TextTitle{[Abdias cache les prophètes]}
\VerseOne{}Et il arriva, après bien des jours, que la parole de Yahweh fut adressée à Elie, dans la troisième année, en disant : Va, montre-toi à Achab et je ferai tomber de la pluie sur la terre.
\VS{2}Et Elie s'en alla pour se présenter devant Achab. Il y avait alors une grande famine en Samarie.
\VS{3}Achab avait appelé Abdias, chef de sa maison ; or, Abdias craignait beaucoup Yahweh ;
\VS{4}quand Jézabel exterminait les prophètes de Yahweh, Abdias prit cent prophètes et les cacha, cinquante dans une caverne et cinquante dans une autre, et il les y nourrit de pain et d'eau.
\VS{5}Achab dit alors à Abdias : Va par le pays vers toutes les sources d'eaux et vers tous les torrents ; peut-être que nous trouverons de l'herbe, nous garderons ainsi en vie les chevaux et les mulets, et nous n’aurons pas besoin d’abattre du bétail.
\VS{6}Ils se partagèrent donc entre eux le pays pour le parcourir ; Achab allait seul par un chemin et Abdias allait seul par un autre chemin.
\VS{7}Comme Abdias était en chemin, voici, Elie le rencontra. Abdias reconnut Elie, il tomba sur son visage et lui dit : N'es-tu pas mon seigneur Elie ?
\VS{8}Il lui répondit : C'est moi ; va et dis à ton seigneur : Voici Elie !
\VS{9}Et Abdias dit : Quel péché ai-je commis, pour que tu livres ton serviteur entre les mains d'Achab pour me faire mourir ?
\VS{10}Yahweh, ton Dieu, est vivant ! Il n'y a ni nation, ni royaume, où mon seigneur n'ait envoyé pour te chercher ; et quand on répondait que tu n'y étais pas, il faisait jurer aux rois et au peuple que l'on ne t’avait pas trouvé.
\VS{11}Et maintenant tu dis : Va, dis à ton seigneur, voici Elie !
\VS{12}Puis, lorsque je t’aurai quitté, l'Esprit de Yahweh te transportera je ne sais où et j’irai informer Achab qui ne te trouvera pas et qui me tuera. Or, ton serviteur craint Yahweh dès sa jeunesse.
\VS{13}N'a-t-on point dit à mon seigneur ce que je fis quand Jézabel tuait les prophètes de Yahweh, comment j'en cachai cent, cinquante dans une caverne et cinquante dans une autre et les y ai nourris de pain et d'eau ?
\VS{14}Et maintenant tu dis : Va, dis à ton seigneur : Voici Elie ! Il me tuera !
\VS{15}Mais Elie lui répondit : Yahweh des armées, devant lequel je me tiens, est vivant ! Aujourd'hui, je me montrerai à Achab.
\VS{16}Abdias étant allé à la rencontre Achab, l’informa de la chose ; puis Achab alla au-devant d'Elie.
\VS{17}Et aussitôt qu'Achab eut vu Elie, il lui dit : Est-ce toi qui jettes le trouble en Israël ?
\VS{18}Et Elie lui répondit : Je n'ai point troublé Israël ; c'est toi et la maison de ton père, puisque vous avez abandonné les commandements de Yahweh et que vous êtes allés après les Baals.
\VS{19}Fais maintenant se rassembler tout Israël auprès de moi, sur le mont Carmel, les quatre cent cinquante prophètes de Baal et les quatre cents prophètes d’Astarté qui mangent à la table de Jézabel.
\TextTitle{[Confrontation entre Elie et les prophètes de Baal sur le mont Carmel]}
\VS{20}Ainsi Achab envoya des messagers vers tous les fils d'Israël et, il rassembla les prophètes sur le mont Carmel.
\VS{21}Alors Elie s'approcha de tout le peuple et dit : Jusqu'à quand clocherez-vous des deux côtés ? Si Yahweh est Dieu, suivez-le ; mais si Baal est dieu, suivez-le. Et le peuple ne lui répondit pas un seul mot.
\VS{22}Alors Elie dit au peuple : Je suis demeuré seul prophète de Yahweh ; et voici quatre cent cinquante prophètes de Baal.
\VS{23}Que l’on nous donne deux veaux, qu'ils en choisissent l'un pour eux, qu'ils le coupent en pièces et qu'ils le mettent sur du bois ; mais qu'ils n'y mettent point de feu ; et je préparerai l'autre veau, je le mettrai sur du bois, sans y mettre le feu.
\VS{24}Puis invoquez le nom de vos dieux, et moi j'invoquerai le nom de Yahweh ; que le dieu qui répondra par le feu, soit reconnu pour être Dieu. Et tout le peuple répondit et dit : C'est bien !
\VS{25}Et Elie dit aux prophètes de Baal : Choisissez un veau et préparez-le les premiers, car vous êtes en plus grand nombre et invoquez le nom de vos dieux ; mais n'y mettez point de feu.
\VS{26}Ils prirent donc un veau qu'on leur donna, ils l'apprêtèrent et ils invoquèrent le nom de Baal depuis le matin jusqu'à midi, en disant : Baal exauce-nous ! Mais il n'y avait ni voix ni réponse et ils sautaient devant l'autel qu'ils avaient fait.
\VS{27}A midi, Elie se moqua d'eux et dit : Criez à haute voix, puisqu’il est dieu ; mais il pense à quelque chose, ou il est occupé, ou il est en voyage ; peut-être qu'il dort et il se réveillera.
\VS{28}Ils criaient donc à haute voix ; ils se faisaient des incisions avec des couteaux et des lances, selon leur coutume, en sorte que le sang coulait sur eux.
\VS{29}Lorsque midi fut passé et qu'ils eurent fait les prophètes jusqu'au temps où l’on offre l'oblation, sans qu'il y eût ni voix, ni réponse, ni signe d’attention.
\VS{30}Elie dit alors à tout le peuple : Approchez-vous de moi ! Et tout le peuple s'approcha de lui et il répara l'autel de Yahweh, qui avait été renversé.
\VS{31}Puis Elie prit douze pierres, selon le nombre des tribus des fils de Jacob, auquel la parole de Yahweh avait été adressée, en disant : Israël sera ton nom.
\VS{32}Et il rebâtit de ces pierres l'autel au nom de Yahweh. Puis il fit un fossé de la capacité de deux mesures de semence autour de l'autel.
\VS{33}Il rangea le bois, il coupa le veau en pièces, et il le plaça sur le bois.
\VS{34}Puis il dit : Remplissez quatre cruches d'eau, puis versez-les sur l'holocauste et sur le bois. Puis il dit : Faites-le encore une seconde fois. Et ils le firent une seconde fois. Il dit : Faites-le une troisième fois. Et ils le firent pour la troisième fois ;
\VS{35}de sorte que les eaux allaient à l'entour de l'autel ; et il remplit aussi d’eau le fossé.
\VS{36}Et au moment de la présentation de l’offrande, Elie, le prophète, s'approcha et dit : Ô Yahweh ! Dieu d'Abraham, d'Isaac et d'Israël ! Que l’on sache aujourd'hui que tu es Dieu en Israël et que je suis ton serviteur ; et que j'ai fait toutes ces choses par ta parole !
\VS{37}Réponds-moi, Ô Yahweh ! Réponds-moi, afin que ce peuple connaisse que c’est toi, Yahweh, qui es Dieu et que c'est toi qui ramènes leur cœur.
\VS{38}Alors le feu de Yahweh tomba et consuma l'holocauste, le bois, les pierres et la terre, et il absorba toute l'eau qui était dans le fossé.
\VS{39}Quand tout le peuple vit cela, ils tombèrent sur leur visage et dirent : C'est Yahweh qui est Dieu ! C'est Yahweh qui est Dieu !
\VS{40}Et Elie leur dit : Saisissez les prophètes de Baal et qu'il n'en échappe aucun ! Ils les saisirent. Elie les fit descendre au torrent de Kison, où il les fit égorger là.
\TextTitle{[Fin du temps de sécheresse, la prière d'Elie\FTNTT{Ja. 5:17-18}]}
\VS{41}Puis Elie dit à Achab : Monte, mange et bois ; car il se fait un bruit qui annonce la pluie.
\VS{42}Ainsi Achab monta pour manger et pour boire tandis qu’Elie monta au sommet du Carmel ; et, se penchant contre terre, il mit son visage entre ses genoux ;
\VS{43}Et il dit à son serviteur : Monte maintenant et regarde vers la mer. Le serviteur monta, il regarda et dit : Il n'y a rien. Elie dit par sept fois : Retournes-y.
\VS{44}A la septième fois, il dit : Voici un petit nuage qui s’élève de la mer et qui est comme la paume de la main d'un homme, laquelle monte de la mer. Elie dit : Monte et dis à Achab : Attelle ton char et descends de peur que la pluie ne t’arrête.
\VS{45}Ici et là, les cieux s'obscurcirent de nuages accompagnés de vent et il y eut une forte pluie. Achab monta sur son char et partit pour Jizreel.
\VS{46}Et la main de Yahweh fut sur Elie, qui se ceignit les reins et courut devant Achab, jusqu'à l'entrée de Jizreel.
\Chap{19}
\TextTitle{[Elie fuit Jézabel]}
\VerseOne{}Achab rapporta à Jézabel tout ce qu'Elie avait fait, et comment il avait tué par l'épée tous les prophètes.
\VS{2}Et Jézabel envoya un messager vers Elie, pour lui dire : Que les dieux me traitent dans toute leur rigueur, si demain, à cette heure-ci, je ne fais de ta vie ce que tu as fait de la vie de chacun d'eux !
\VS{3}Elie, voyant cela, se leva et s'en alla pour sauver sa vie. Il arriva à Beer-Schéba, qui appartient à Juda ; et il laissa là son serviteur.
\TextTitle{[L'ange de Yahweh au secours d'Elie]}
\VS{4}Mais lui s'en alla dans le désert où, après une journée de marche, il s'assit sous un genêt et demanda la mort, en disant : C'en est assez, Ô Yahweh ! Prends mon âme, car je ne suis pas meilleur que mes pères.
\VS{5}Puis il se coucha et s'endormit sous un genêt. Voici un ange le toucha et lui dit : Lève-toi, mange.
\VS{6}Et il regarda, et voici à son chevet, un gâteau cuit sur des pierres chauffées et une cruche d'eau. Il mangea et but, puis se recoucha.
\VS{7}Et l'ange de Yahweh vint une seconde fois, le toucha et lui dit : Lève-toi, mange, car le chemin est trop long pour toi.
\TextTitle{[Elie marche jusqu'à Horeb]}
\VS{8}Il se leva donc, mangea et but ; puis avec la force que lui donna cette nourriture, il marcha quarante jours et quarante nuits jusqu'à Horeb, la montagne de Dieu.
\VS{9}Et là, il entra dans une caverne et y passa la nuit. Et voici, la parole de Yahweh lui fut adressée en ces mots : Que fais-tu ici, Elie ?
\VS{10}Et il répondit : J'ai déployé mon zèle pour Yahweh, le Dieu des armées, parce que les enfants d'Israël ont abandonné ton alliance, ils ont renversé tes autels, ils ont tué tes prophètes par l'épée ; je suis resté, moi seul et ils me cherchent pour m'ôter la vie.
\VS{11}Yahweh lui dit : Sors et tiens-toi sur la montagne devant Yahweh. Et voici, Yahweh passa. Et devant Yahweh, il y eut un grand vent impétueux qui déchirait les montagnes et brisait les rochers, mais Yahweh n'était point dans ce vent. Après le vent, ce fut un tremblement de terre ; mais Yahweh n'était point dans ce tremblement de terre.
\VS{12}Après le tremblement de terre, un feu ; mais Yahweh n'était pas dans le feu. Et après le feu vint un murmure doux et léger.
\VS{13}Quand Elie l'entendit, il s’enveloppa le visage de son manteau, il sortit et se tint à l'entrée de la caverne. Et voici, une voix lui fit entendre ces paroles : Que fais-tu ici Elie ?
\VS{14}Et il répondit : J'ai déployé mon zèle pour Yahweh, le Dieu des armées, parce que les enfants d'Israël ont abandonné ton alliance, ils ont renversé tes autels, ils ont tué par l'épée tes prophètes ; je suis resté moi seul, et ils cherchent ma vie pour me l'ôter.
\VS{15}Yahweh lui dit : Va, retourne-t'en par ton chemin vers le désert de Damas ; et quand tu seras arrivé, tu oindras Hazaël pour roi de Syrie.
\VS{16}Tu oindras aussi Jéhu, fils de Nimschi, pour roi d’Israël ; et tu oindras Elisée, fils de Schaphath, d'Abel-Mehola, pour prophète à ta place.
\VS{17}Et il arrivera que quiconque échappera de l'épée de Hazaël, Jéhu le fera mourir ; et quiconque échappera de l'épée de Jéhu, Elisée le fera mourir.
\VS{18}Mais je me suis réservé sept mille hommes de reste en Israël, tous ceux qui n'ont point fléchi les genoux devant Baal, et dont la bouche ne l'a point baisé.
\TextTitle{[Elisée disciple d'Elie]}
\VS{19}Elie partit donc de là, et il trouva Elisée, fils de Schaphath, qui labourait. Il y avait douze paires de bœufs devant soi et il était avec la douzième. Quand Elie passa près de lui, il jeta sur lui son manteau.
\VS{20}Elisée laissa ses bœufs et courut après Elie, en disant : Je t’en prie, laisse-moi embrasser mon père et ma mère, et je te suivrai. Elie lui répondit : Va, et reviens ; car pense à ce que je t'ai fait.
\VS{21}Après s’être éloigné d’Elie, il revint prendre une paire de bœufs qu’il offrit en sacrifice ; et avec l'attelage des bœufs, il en fit bouillir la chair, et la donna au peuple ; ils mangèrent ; puis il se leva et suivit Elie. Dès lors, il fut à son service.
\Chap{20}
\TextTitle{[Achab monte contre Ben-Hadad]}
\VerseOne{}Alors Ben-Hadad, roi de Syrie rassembla toute son armée ; il avait avec lui trente-deux rois, des chevaux et des chars. Puis il monta, assiégea Samarie et il lui fit la guerre.
\VS{2}Il envoya des messagers à Achab, roi d'Israël, dans la ville ;
\VS{3}Et il lui fit dire : Ainsi parle Ben-Hadad : Ton argent et ton or sont à moi, tes femmes aussi et tes beaux enfants sont à moi.
\VS{4}Et le roi d'Israël répondit, et dit : Mon seigneur, je suis à toi, comme tu le dis, avec tout ce que j'ai.
\VS{5}Ensuite les messagers retournèrent, et dirent : Ainsi parle Ben-Hadad : Puisque je t'ai envoyé dire : Donne-moi ton argent et ton or, ta femme et tes enfants ;
\VS{6}A la même heure demain, j'enverrai chez toi mes serviteurs, ils fouilleront ta maison et les maisons de tes serviteurs, et se saisiront de tout ce que tu as de précieux, et ils l'emporteront.
\VS{7}Alors le roi d'Israël appela tous les anciens du pays, et il dit : Sachez et considérez, je vous prie, combien cet homme nous veut du mal ; car il m’a envoyé demander mes femmes, mes enfants, mon argent et mon or, et je ne lui avais rien refusé.
\VS{8}Et tous les anciens et tout le peuple lui dirent : Ne l'écoute point et ne consens pas.
\VS{9}Il répondit donc aux messagers de Ben-Hadad : Dites au roi, mon seigneur : Je ferai tout ce que tu as envoyé demander la première fois à ton serviteur, mais je ne pourrai faire ceci. Les messagers s'en allèrent et lui rapportèrent cette réponse.
\VS{10}Et Ben-Hadad envoya dire à Achab : Que les dieux me traitent dans toute leur rigueur, si la poudre de Samarie suffit pour remplir le creux de la main de tout le peuple qui me suit.
\VS{11}Mais le roi d'Israël répondit, et dit : Dites-lui : Que celui qui revêt une armure ne se glorifie point comme celui qui la dépose.
\VS{12}Lorsque Ben-Hadad entendit cette réponse, il était à boire avec les rois sous les tentes et il dit à ses serviteurs : Rangez-vous en bataille ! Et ils se rangèrent en bataille contre la ville.
\TextTitle{[Victoire Achab]}
\VS{13}Alors voici, un prophète s’approcha d’Achab, roi d'Israël et lui dit : Ainsi parle Yahweh : N'as-tu pas vu cette grande multitude ? Voilà, je m'en vais la livrer aujourd'hui entre tes mains, et tu sauras que je suis Yahweh.
\VS{14}Et Achab dit : Par qui ? Et il lui répondit : Ainsi parle Yahweh : Ce sera par les serviteurs des chefs des provinces. Et Achab dit : Qui engagera le combat ? Et il lui répondit : Toi.
\VS{15}Alors il passa en revue les serviteurs des chefs des provinces, qui furent deux cent trente-deux ; et après eux, il dénombra tout le peuple de tous les enfants d'Israël qui furent sept mille.
\VS{16}Ils firent une sortie en plein midi, lorsque Ben-Hadad buvait et s'enivrait dans les tentes, lui et les trente-deux rois qui étaient ses auxiliaires.
\VS{17}Les serviteurs des chefs des provinces sortirent les premiers et Ben-Hadad envoya quelques-uns qui le lui rapportèrent en disant : Des hommes sont sortis de Samarie.
\VS{18}Et il dit : Qu’ils soient sortis pour la paix, ou qu'ils soient sortis pour faire la guerre, saisissez-les tous vivants.
\VS{19}Les serviteurs des chefs de province sortirent de la ville puis l'armée qui était après eux.
\VS{20}Chacun d'eux frappa son homme, de sorte que les Syriens s'enfuirent et Israël les poursuivit. Ben-Hadad, roi de Syrie, se sauva sur un cheval, avec des cavaliers.
\VS{21}Et le roi d'Israël sortit et frappa les chevaux et les chars, en sorte qu'il fit éprouver une grande défaite aux Syriens.
\TextTitle{[Achab monte de nouveau contre les Syriens]}
\VS{22}Alors le prophète s’approcha du roi d'Israël, et lui dit : Va, fortifie-toi ; considère et vois ce que tu auras à faire ; car l’année révolue, le roi de Syrie montera contre toi.
\VS{23}Or, les serviteurs du roi de Syrie lui dirent : Leur dieu est un dieu de montagnes, c'est pourquoi ils ont été plus forts que nous. Mais combattons contre eux dans la plaine, et certainement, nous serons plus forts qu'eux.
\VS{24}Fais donc ceci : Ote chacun de ces rois de sa place, et remplace-les par des chefs ;
\VS{25}Puis lève une armée pareille à celle que tu as perdue, avec autant de chevaux et de chars, puis nous les combattrons dans la plaine et l’on verra si nous ne sommes pas plus forts qu'eux. Il les écouta, et fit ainsi.
\VS{26}L’année suivante, Ben-Hadad dénombra les Syriens et monta à Aphek pour combattre contre Israël.
\VS{27}On fit aussi le dénombrement des enfants d'Israël ; ils reçurent des vivres, et ils marchèrent à la rencontre des Syriens. Les enfants d'Israël campèrent vis-à-vis d'eux ; semblables à deux petits troupeaux de chèvres, tandis que les Syriens remplissaient le pays.
\VS{28}Alors l'homme de Dieu vint, et dit au roi d'Israël : Ainsi parle Yahweh : Parce que les Syriens ont dit : Yahweh est un dieu des montagnes et non un dieu des vallées, je livrerai entre tes mains toute cette grande multitude, et vous saurez que je suis Yahweh.
\VS{29}Sept jours durant ils campèrent vis-à-vis les uns des autres. Le septième jour, ils entrèrent en bataille, et les enfants d'Israël tuèrent en un seul jour cent mille hommes de pied des Syriens.
\VS{30}Le reste s'enfuit à la ville d'Aphek, où la muraille tomba sur vingt-sept mille hommes demeurés de reste. Ben-Hadad s'était réfugié dans la ville où il allait de chambre en chambre.
\TextTitle{[Faute d'Achab qui épargne Ben-Hadad]}
\VS{31}Ses serviteurs lui dirent : Voici maintenant, nous avons appris que les rois de la maison d'Israël sont des rois miséricordieux ; maintenant donc mettons des sacs sur nos reins et des cordes à nos têtes, sortons vers le roi d'Israël, peut-être qu'il te laissera la vie sauve.
\VS{32}Ils se mirent donc des sacs autour des reins et des cordes autour de leurs têtes. Ils allèrent auprès du roi d'Israël. Ils lui dirent : Ton serviteur Ben-Hadad dit : Laisse-moi la vie ! Achab répondit : Est-il encore vivant ? Il est mon frère.
\VS{33}Ces hommes tirèrent de là un bon augure, ils se hâtèrent de le prendre au mot et ils dirent : Ben-Hadad est-il ton frère ! Et il répondit : Allez, amenez-le. Ben-Hadad vint vers lui, et il le fit monter sur son char.
\VS{34}Et Ben-Hadad lui dit : Je te rendrai les villes que mon père avait prises à ton père ; et tu te feras des rues en Damas comme mon père avait fait en Samarie. Et moi, répondit Achab, je te laisserai aller en faisant alliance. Il traita donc alliance avec lui, et le laissa aller.
\VS{35}Alors un homme d'entre les fils des prophètes dit à son compagnon, sur l’ordre de Yahweh : Frappe-moi, je te prie ! Mais celui-là refusa de le frapper.
\VS{36}Et il lui dit : Parce que tu n'as point obéi à la parole de Yahweh, voilà, quand tu m’auras quitté, un lion te frappera. Quand il se fut séparé de lui, un lion survint et le frappa.
\VS{37}Puis il trouva un autre homme, et lui dit : Frappe-moi, je te prie. Cet homme-là le frappa et il le blessa.
\VS{38}Après cela le prophète s'en alla, et se plaça sur le chemin du roi ; il se déguisa avec un bandeau sur ses yeux.
\VS{39}Lorsque le roi passa, il cria vers lui, et dit : Ton serviteur était allé au milieu de la bataille ; et voici quelqu'un s'étant retiré, m'a amené un homme, en disant : Garde cet homme, s'il vient à s'échapper, ta vie en répondra, ou tu paieras un talent d'argent.
\VS{40}Et pendant que ton serviteur faisait quelques affaires çà et là, cet homme a disparu. Et le roi d'Israël lui répondit : Telle est ta condamnation, tu l’as toi-même prononcée.
\VS{41}Alors le prophète ôta promptement le bandeau de dessus ses yeux et le roi d'Israël reconnut que c'était l’un des prophètes.
\VS{42}Et il dit : Ainsi parle Yahweh : Parce que tu as laissé échapper de tes mains l'homme que j'avais dévoué par la voie de l'interdit, ta vie répondra de sa vie, et ton peuple de son peuple.
\VS{43}Mais le roi d'Israël se retira en sa maison, triste et irrité ; et il arriva en Samarie.
\Chap{21}
\TextTitle{[Naboth meurt pour sa vigne]}
\VerseOne{}Après ces choses, voici ce qui arriva. Naboth de Jizreel, ayant une vigne à Jizreel, près du palais d'Achab, roi de Samarie.
\VS{2}Achab parla à Naboth et lui dit : Cède-moi ta vigne, afin que j'en fasse un jardin potager, car elle est proche de ma maison et je te donnerai à la place une vigne meilleure ; ou, si cela te semble bon, je te paierai l'argent qu'elle vaut.
\VS{3}Mais Naboth répondit à Achab : Que Yahweh me garde de te donner l'héritage de mes pères !
\VS{4}Et Achab vint en sa maison tout triste et irrité, à cause de cette parole que lui avait dite Naboth de Jizreel, en disant : Je ne te donnerai point l'héritage de mes pères ! Il se coucha sur son lit, détourna son visage, et ne mangea rien.
\VS{5}Alors Jézabel, sa femme, vint auprès de lui, et lui dit : D'où vient que ton esprit est si triste ? Et pourquoi ne manges-tu point ?
\VS{6}Et il lui répondit : J’ai parlé à Naboth de Jizreel, et je lui ai dit : Donne-moi ta vigne pour de l'argent, ou si tu le désires, je te donnerai une autre vigne pour celle-là, mais il m'a dit : Je ne te céderai point ma vigne !
\VS{7}Alors Jézabel, sa femme, lui dit : Est-ce bien toi maintenant qui exerces la royauté sur Israël ? Lève-toi, prends un repas et que ton cœur se réjouisse ; je te ferai avoir la vigne de Naboth de Jizreel.
\VS{8}Et elle écrivit au nom d'Achab des lettres qu’elle scella du sceau du roi, et elle envoya aux anciens et magistrats qui habitaient avec Naboth, dans sa ville.
\VS{9}Voici ce qu’elle écrivit dans ces lettres : Publiez un jeûne et placez Naboth à la tête du peuple.
\VS{10}Mettez face à lui deux méchants hommes et qu'ils témoignent contre lui, en disant : Tu as maudit Dieu et le roi ! Puis vous le mènerez dehors et le lapiderez afin qu'il meure.
\VS{11}Les gens donc de la ville de Naboth, les anciens et les magistrats qui habitaient dans sa ville, agirent comme Jézabel le leur avait dit, et d’après ce qui était écrit dans les lettres qu'elle leur avait envoyées.
\VS{12}Ils publièrent un jeûne et ils placèrent Naboth à la tête du peuple.
\VS{13}Les deux méchants hommes vinrent et se mirent face à lui, et ces méchants hommes déclarèrent contre Naboth en la présence du peuple : Naboth a maudit Dieu et le roi ! Puis ils le menèrent hors de la ville, ils le lapidèrent, et il mourut.
\VS{14}Après cela, ils envoyèrent dire à Jézabel : Naboth a été lapidé, et il est mort.
\VS{15}Lorsque Jézabel apprit que Naboth avait été lapidé et qu'il était mort, elle dit à Achab : Lève-toi, mets-toi en possession de la vigne de Naboth de Jizreel, qu’il avait refusé de te donner pour de l'argent ; car Naboth n'est plus en vie, il est mort.
\VS{16}Ainsi dès qu'Achab eut entendu que Naboth était mort, il se leva pour descendre à la vigne de Jizreel et pour s'en mettre en possession.
\TextTitle{[Elie condamne Achab et Jézabel, Achab se repent]}
\VS{17}Alors la parole de Yahweh fut adressée à Elie, le Thischbite, en ces mots :
\VS{18}Lève-toi, descends au-devant d'Achab, roi d'Israël, lorsqu'il sera à Samarie. Le voilà dans la vigne de Naboth, où il est descendu pour en prendre possession.
\VS{19}Et tu lui diras : Ainsi parle Yahweh : N’es-tu pas un meurtrier et un voleur ? Puis tu lui diras : Ainsi parle Yahweh : Comme les chiens ont léché le sang de Naboth, les chiens lécheront aussi ton propre sang.
\VS{20}Et Achab dit à Elie : M'as-tu trouvé mon ennemi ? Mais il lui répondit : Oui, je t'ai trouvé, parce que tu t'es vendu pour faire ce qui est mal aux yeux de Yahweh.
\VS{21}Voici je vais faire venir le malheur sur toi, et je te consumerai, j’exterminerai quiconque appartient à Achab, tant celui qui est esclave, que celui qui est libre en Israël.
\VS{22}Je rendrai ta maison semblable à la maison de Jéroboam, fils de Nebath, et la maison de Baescha, fils d'Achija, parce que tu m'as irrité et fait pécher Israël.
\VS{23}Yahweh parla aussi contre Jézabel en disant : Les chiens mangeront Jézabel près du rempart de Jizreel.
\VS{24}Celui de la maison d’Achab qui mourra dans la ville, les chiens le mangeront, et celui qui mourra aux champs, les oiseaux des cieux le mangeront.
\VS{25}En effet, il n'y en avait point eu de personne comme Achab, qui se soit vendu pour faire ce qui est mal aux yeux de Yahweh, et sa femme Jézabel l’y excitait ;
\VS{26}de sorte qu'il se rendit fort abominable, allant après les idoles, comme l'avaient fait les Amoréens, que Yahweh avait chassés de devant les enfants d'Israël.
\VS{27}Après avoir entendu les paroles d’Elie, Achab déchira ses vêtements, il mit un sac sur son corps, et jeûna. Il se tenait couché avec ce sac, et il marchait lentement.
\VS{28}Et la parole de Yahweh fut adressée à Elie, le Thischbite, en disant :
\VS{29}As-tu vu comment Achab s'est humilié devant moi ? Parce qu'il s'est humilié devant moi, je ne ferai pas venir le malheur pendant sa vie, ce sera aux jours de son fils que je ferai venir le malheur sur sa maison.
\Chap{22}
\TextTitle{[Josaphat aide Achab contre les Syriens]}
\VerseOne{}Et on resta trois ans sans qu'il y eût guerre entre la Syrie et Israël.
\VS{2}Puis il arriva, dans la troisième année, que Josaphat, roi de Juda, descendit vers le roi d'Israël.
\VS{3}Le roi d'Israël dit à ses serviteurs : Ne savez-vous pas que Ramoth de Galaad nous appartient ? Et nous ne nous inquiétons pas de la reprendre des mains du roi de Syrie !
\VS{4}Puis il dit à Josaphat : Viendras-tu avec moi à la guerre contre Ramoth de Galaad ? Et Josaphat répondit au roi d'Israël : Nous irons, moi comme toi, mon peuple comme ton peuple, et mes chevaux comme tes chevaux.
\TextTitle{[Les prophètes de mensonge\FTNTT{2 Ch. 18:4-5, 9-11}]}
\VS{5}Josaphat dit encore au roi d'Israël : Consulte aujourd'hui, je te prie, la parole de Yahweh.
\VS{6}Et le roi d'Israël assembla les prophètes, au nombre de quatre cents environ, auxquels il dit : Irai-je à la guerre contre Ramoth de Galaad, ou dois-je y renoncer ? Et ils répondirent : Monte, car le Seigneur la livrera entre les mains du roi.
\VS{7}Mais Josaphat dit : N'y a-t-il point ici encore quelque prophète de Yahweh, afin que nous le consultions ?
\VS{8}Et le roi d'Israël dit à Josaphat : Il y a encore un homme par qui l’on puisse consulter Yahweh, mais je le hais, car il ne prophétise rien de bon, mais seulement du mal, c'est Michée, fils de Jimla. Josaphat dit : Que le roi ne parle point ainsi !
\VS{9}Alors le roi d'Israël appela un eunuque auquel il dit : Fais venir promptement Michée, fils de Jimla.
\VS{10}Or, le roi d'Israël et Josaphat, roi de Juda, étaient assis chacun sur son trône, revêtus de leurs habits, dans la place, vers l'entrée de la porte de Samarie ; et tous les prophètes prophétisaient en leur présence.
\VS{11}Sédécias, fils de Kenaana, s'était fait des cornes de fer et il dit : Ainsi parle Yahweh : De ces cornes-ci tu heurteras les Syriens, jusqu'à les détruire.
\VS{12}Et tous les prophètes prophétisaient de même, en disant : Monte à Ramoth de Galaad et tu réussiras ; et Yahweh la livrera entre les mains du roi.
\TextTitle{[Michée annonce la défaite et la mort d'Achab\FTNTT{2 Ch. 18:6-8, 12-27, 28-34}]}
\VS{13}Le messager qui était allé appeler Michée, lui parla ainsi : Voici, les prophètes parlent d'un commun accord au sujet du roi ; je te prie que ta parole soit semblable à celle de chacun d’eux ! Annonce du bien !
\VS{14}Mais Michée lui répondit : Yahweh est vivant ! J’annoncerai ce que Yahweh me dira.
\VS{15}Il vint donc vers le roi, et le roi lui dit : Michée, irons-nous à la guerre contre Ramoth de Galaad, ou devons-nous y renoncer ? Et il lui dit : Monte et tu réussiras, et Yahweh la livrera entre les mains du roi.
\VS{16}Et le roi lui dit : Jusqu'à combien de fois te conjurerai-je de ne me dire que la vérité au nom de Yahweh ?
\VS{17}Et il répondit : J'ai vu tout Israël dispersé par les montagnes, comme un troupeau de brebis qui n'a point de berger ; et Yahweh a dit : Ces gens n’ont point de maître, que chacun retourne en paix dans sa maison !
\VS{18}Alors le roi d'Israël dit à Josaphat : Ne t'ai-je pas bien dit que quand il est question de moi il ne prophétise rien de bon, mais seulement du mal ?
\VS{19}Et Michée lui dit : Ecoute néanmoins la parole de Yahweh ! J'ai vu Yahweh assis sur son trône, et toute l'armée des cieux se tenant devant lui, à sa droite et à sa gauche.
\VS{20}Et Yahweh a dit : Quel est celui qui séduira Achab, afin qu'il monte et qu'il périsse en Ramoth de Galaad ? Et ils répondaient, l'un parlait d'une manière et l'autre d'une autre.
\VS{21}Alors un esprit s'avança et se tint devant Yahweh, il déclara : Je le séduirai. Et Yahweh lui dit : Comment ?
\VS{22}Et il répondit : Je sortirai et je serai un esprit de mensonge dans la bouche de tous ses prophètes. Et Yahweh dit : Tu le séduiras et même tu en viendras à bout ; sors et fais ainsi !
\VS{23}Et maintenant, voici, Yahweh a mis un esprit de mensonge dans la bouche de tous tes prophètes que voilà et Yahweh a prononcé du mal contre toi.
\VS{24}Alors Sédécias, fils de Kenaana, s'approcha et frappa Michée sur la joue et dit : Par où l'Esprit de Yahweh est-il sorti de moi pour s'adresser à toi ?
\VS{25}Et Michée répondit : Voici, tu le verras le jour où tu iras de chambre en chambre pour te cacher.
\VS{26}Alors le roi d'Israël dit : Qu'on prenne Michée et qu'on le mène vers Amon, capitaine de la ville et vers Joas, le fils du roi.
\VS{27}Et tu diras : Ainsi a parlé le roi : Mettez cet homme en prison, nourrissez-le de pain et de l’eau d’affliction, jusqu'à ce que je revienne en paix.
\VS{28}Et Michée répondit : Si tu reviens en paix, Yahweh n'a point parlé par moi. Il dit aussi : Vous tous, peuples, entendez !
\VS{29}Le roi d'Israël monta avec Josaphat, roi de Juda, contre Ramoth de Galaad.
\VS{30}Et le roi d'Israël dit à Josaphat : Que je me déguise et que j'aille à la bataille ; mais toi, revêts-toi de tes habits. Le roi d'Israël donc se déguisa et alla au combat.
\VS{31}Or, le roi de Syrie avait donné un ordre aux trente-deux chefs de ses chars, en disant : Vous n’attaquerez ni petits ni grands, mais contre le seul roi d'Israël.
\VS{32}Quand les chefs des chars aperçurent Josaphat, ils dirent : C'est certainement le roi d'Israël. Et ils s’approchèrent de lui pour le combattre, mais Josaphat s'écria.
\VS{33}Et quand les chefs des chars virent que ce n'était pas le roi d'Israël, ils se détournèrent de lui.
\VS{34}Alors un homme tira de son arc au hasard, et frappa le roi d'Israël entre les jointures de la cuirasse. Et le roi dit à son conducteur de char : Tourne et fais-moi sortir du champ de bataille, car je suis blessé.
\VS{35}Or, le combat devint acharné ce jour-là. Le roi d'Israël fut arrêté dans son char en face des Syriens et il mourut sur le soir. Le sang de sa blessure coulait à l’intérieur du char.
\VS{36}Au coucher du soleil, on cria par tout le camp, en disant : Que chacun se retire en sa ville et chacun en son pays !
\VS{37}Ainsi mourut le roi, qui fut ramené à Samarie ; et l’on enterra le roi à Samarie.
\VS{38}Lorsqu’on lava le char à l’étang de Samarie, les chiens léchèrent le sang d’Achab, et les prostituées s’y baignèrent, selon la parole que Yahweh avait prononcée.
\VS{39}Le reste des actions d'Achab, tout ce qu’il a fait, la maison d'ivoire qu'il construisit et toutes les villes qu'il a bâties, toutes ces choses ne sont-elles pas écrites au livre des Chroniques des rois d'Israël ?
\TextTitle{[Achazia règne après Achab]}
\VS{40}Ainsi Achab se coucha avec ses pères. Et Achazia, son fils, régna à sa place.
\TextTitle{[Règne de Josaphat sur Juda\FTNTT{2 Ch. 17:19-20}]}
\VS{41}Josaphat, fils d'Asa, régna sur Juda, la quatrième année d'Achab, roi d'Israël.
\VS{42}Josaphat avait trente-cinq ans lorsqu’il devint roi, et il régna vingt-cinq ans à Jérusalem. Sa mère s’appelait Azuba, fille de Schilchi.
\VS{43}Il suivit entièrement la voie d'Asa, son père, et ne s'en détourna point, faisant tout ce qui est droit aux yeux de Yahweh.
\VS{44}Toutefois les hauts lieux ne disparurent pas ; le peuple offrait encore des sacrifices et offrait encore des parfums sur les hauts lieux.
\VS{45}Josaphat fit aussi la paix avec le roi d'Israël.
\VS{46}Le reste des actions de Josaphat, ses exploits et les guerres qu'il mena ne sont-elles pas écrites au livre des Chroniques des rois de Juda ?
\VS{47}Il extermina du pays le reste des prostitués, qui étaient demeurés là depuis le temps d'Asa, son père.
\VS{48}Il n'y avait point alors de roi en Edom : c’était un intendant qui gouvernait.
\VS{49}Josaphat construisit des navires de Tarsis pour aller chercher de l'or à Ophir ; mais il n'y alla point, parce que les navires se brisèrent à Etsjon-Guéber.
\VS{50}Alors Achazia, fils d'Achab, dit à Josaphat : Que mes serviteurs aillent sur les navires avec les tiens, mais Josaphat ne le voulut point.
\TextTitle{[Joram, roi de Juda\FTNTT{2 Ch. 21:1}]}
\VS{51}Et Josaphat s’endormit avec ses pères et fut enterré avec eux en la cité de David, son père. Et Joram, son fils, régna à sa place.
\TextTitle{[Règne d'Achazia  sur Israël]}
\VS{52}Achazia, fils d'Achab, régna sur Israël à Samarie, la dix-septième année de Josaphat, roi de Juda. Et il régna deux ans sur Israël.
\VS{53}Il fit ce qui est mal aux yeux de Yahweh : il marcha dans la voie de son père, de sa mère et celle de Jéroboam, fils de Nebath, qui avait fait pécher Israël.
\VS{54}Il servit Baal, il se prosterna devant lui et il irrita Yahweh, le Dieu d'Israël, comme l’avait fait son père.
\PPE{}
\end{multicols}
