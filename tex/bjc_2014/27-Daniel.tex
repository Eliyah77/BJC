\ShortTitle{Daniel}\BookTitle{Daniel}\BFont
\noindent\hrulefill
{\footnotesize
\textit{
\bigskip
{\centering{}
\\Auteur : Daniel
\\(Heb. : Daniye'l)
\\Signification : Dieu est mon juge
\\Thème : Ascension et chute des royaumes
\\Date de rédaction : 6\up{ème} siècle av. J.-C.\\}
}
%\bigskip
\textit{
\\Issu d'une famille princière de Juda, Daniel fut déporté de Jérusalem à Babylone pendant sa jeunesse, sous le règne de Nebucadnestar. Lui et trois de ses amis – eux aussi de noble lignée - furent choisis pour être instruits selon la sagesse babylonienne en vue de servir le roi. Fervent dans sa foi en Yahweh, Daniel - imité ensuite par ses compagnons – résolut de ne point se souiller et obtint ainsi la faveur de son Dieu. Son intégrité et sa crainte de Dieu lui valurent de miraculeuses victoires, de nombreuses distinctions et une grande sagesse. Daniel avait reçu du discernement pour expliquer songes et visions et délivra plusieurs prophéties dont certaines se sont déjà accomplies, d'autres se réaliseront à la fin du temps des nations, au moment du retour de Christ.
%\bigskip
\\Dieu témoigna de la justice de Daniel au prophète Ezéchiel dont il fut contemporain.\bigskip
}
}
\par\nobreak\noindent\hrulefill
\begin{multicols}{2}
\Chap{1}
\TextTitle{Juda livré à la captivité babylonienne}
\VerseOne{}La troisième année du règne de Jojakim, roi de Juda, Nebucadnetsar, roi de Babylone, vint contre Jérusalem et l'assiégea.
\VS{2}Et le Seigneur livra entre ses mains Jojakim\FTNT{En 597 av. J.-C., la ville de Jérusalem tomba entre les mains des Babyloniens qui déportèrent le roi Jojakim et nommèrent comme roi à sa place son oncle Sédécias. Une petite partie de la population fut déportée à cette occasion. Cette première déportation ne concernait que l'élite administrative et sacerdotale : Prêtres, scribes, hauts fonctionnaires, membres de la famille royale et artisans métallurgistes. Pour de nombreux historiens, il s'agissait moins d'une déportation que d'une constitution d'un groupe d'otages. Le roi, quelques membres de sa famille, et de diverses familles de notables, furent tenus en résidence surveillée à la cour babylonienne pour s'assurer que le royaume de Juda resterait pacifié.}, roi de Juda et une partie des vases de la maison de Dieu. Nebucadnetsar emporta les vases au pays de Schinear\FTNT{Schinear : « le pays des deux fleuves ». C'est l'ancien nom du territoire qui est devenu Babylonie ou Chaldée. C'est le pays de Nimrod (Ge. 10:6-12). C'est à Schinear qu'on tenta de construire la tour de Babel et de mettre en place le premier gouvernement mondial.}, dans la maison de son dieu, il les mit dans la maison du trésor de son dieu.
\VS{3}Et le roi dit à Aschpenaz, chef de ses eunuques, d'amener quelques-uns des fils d'Israël de race royale\FTNT{Daniel était de la race royale (2 R. 20:16-19 ; Es. 39:6-7).} et des principaux seigneurs,
\VS{4}quelques jeunes garçons en qui il n'y avait aucun défaut corporel, beaux de figure, instruits en toute sagesse, connaissant les sciences, pleins d'intelligence, et capables de se tenir dans le palais du roi ; et à qui l'on enseignerait les lettres et la langue des Chaldéens.
\VS{5}Le roi leur assigna pour provision chaque jour une portion de la viande royale et du vin dont il buvait, afin qu'on les nourrisse ainsi pendant trois ans au bout desquels ils se tiendraient devant le roi.
\VS{6}Il y avait parmi eux, d'entre les fils de Juda, Daniel, Hanania, Mischaël et Azaria.
\VS{7}Mais le chef des eunuques leur donna d'autres noms, il donna à Daniel le nom de Beltschatsar, à Hanania celui de Schadrac, à Mischaël celui de Méschac et à Azaria celui d'Abed-Nego.
\TextTitle{La fermeté de Daniel à Babylone}
\VS{8}Or Daniel résolut dans son cœur de ne pas se souiller par la portion de la viande du roi, ni par le vin dont le roi buvait; c'est pourquoi il supplia le chef des eunuques afin qu'il ne l'oblige pas à se souiller.
\VS{9}Et Dieu fit trouver à Daniel faveur et grâce auprès du chef des eunuques. 
\VS{10}Toutefois le chef des eunuques dit à Daniel : Je crains le roi, mon seigneur, qui a ordonné ce que vous devez manger et boire ; car pourquoi verrait-il vos visages plus défaits que ceux des jeunes gens de votre âge, et rendriez-vous ma tête coupable envers le roi ?
\VS{11}Mais Daniel dit à Meltsar, l'intendant à qui le chef des eunuques avait remis la surveillance de Daniel, Hanania, Mischaël et Azaria :
\VS{12}Eprouve, je te prie, tes serviteurs pendant dix jours, et qu'on nous donne des légumes à manger et de l'eau à boire.
\VS{13}Après cela, tu regarderas nos visages et ceux des jeunes gens qui mangent la portion de la viande royale; puis tu feras à tes serviteurs selon ce que tu auras vu.
\VS{14}Et il les écouta dans cette affaire et les éprouva pendant dix jours.
\VS{15}Au bout des dix jours, leurs visages parurent en meilleur état et avaient plus d'embonpoint que tous les jeunes gens qui mangeaient la portion de la viande royale.
\VS{16} Ainsi Meltsar prenait la portion de leur viande et le vin qu'ils devaient boire, et leur donnait des légumes.
\VS{17}Et Dieu donna à ces quatre jeunes gens de la science et de l'intelligence dans toutes les lettres, et de la sagesse ; et Daniel comprenait toutes les visions et tous les songes.
\VS{18}Et à la fin des jours fixés par le roi pour qu'on les lui amène, le chef des eunuques les présenta à Nebucadnetsar.
\VS{19}Le roi s'entretint avec eux ; mais entre eux tous il ne s'en trouva pas de tels que Daniel, Hanania, Mischaël et Azaria ; et ils entrèrent au service du roi.
\VS{20}Sur toutes les questions savantes qui réclamaient de la sagesse et de l'intelligence, et sur lesquelles le roi les interrogeait, il trouva dix fois plus de science en eux que dans tous les magiciens et les astrologues qui étaient dans tout son royaume.
\VS{21}Et Daniel fut là jusqu'à la première année du roi Cyrus.
\Chap{2}
\TextTitle{Les sages de Babylone tous condamnés à mort}
\VerseOne{}Or la deuxième année du règne de Nebucadnetsar, Nebucadnetsar eut des songes, et son esprit fut agité, et son sommeil fut interrompu.
\VS{2}Alors le roi fit appeler les magiciens, les astrologues, les enchanteurs et les Chaldéens, pour qu'ils lui expliquent ses songes ; ils vinrent donc et se présentèrent devant le roi.
\VS{3}Le roi leur dit : J'ai eu un songe, mon esprit est agité, tâchant de connaître ce songe\FTNT{Ce songe annonce la future mise en place d'un gouvernement mondial. Voir commentaire en Da. 7:3.}.
\VS{4}Et les Chaldéens répondirent au roi en langue araméenne\FTNT{L'araméen : De Daniel 2:5 à 7:28, le livre est écrit en araméen.} : Ô roi, vis éternellement ! Dis le songe à tes serviteurs et nous en donnerons l'interprétation.
\VS{5}Mais le roi répondit et dit aux Chaldéens : La chose m'a échappé ; si vous ne me faites connaître le songe et son interprétation, vous serez mis en pièces et vos maisons seront réduites en un tas d'immondices.
\VS{6}Mais si vous me faites connaître le songe et son interprétation, vous recevrez de moi, des dons, des présents et un grand honneur. Quoi qu'il en soit, faites-moi connaître le songe et son interprétation.
\VS{7}Ils répondirent pour la seconde fois et dirent : Que le roi dise le songe à ses serviteurs et nous en donnerons l'interprétation.
\VS{8}Le roi répondit et dit : Je connais maintenant que vous ne cherchez qu'à gagner du temps, parce que vous voyez que la chose m'a échappée.
\VS{9}Mais si vous ne me faites pas connaître le songe, il y aura une même sentence contre vous tous ; car vous vous êtes préparés à dire devant moi des mensonges et des faussetés en attendant que le temps soit changé. Quoi qu'il en soit, dites-moi le songe et je saurai que vous pouvez m'en donner l'interprétation.
\VS{10}Les Chaldéens répondirent au roi et dirent : Il n'y a aucun homme sur la terre qui puisse exécuter ce que le roi demande. Et aussi il n'y a ni roi, ni seigneur, ni gouverneur qui ait jamais demandé une telle chose à quelque magicien, astrologue ou Chaldéen que ce soit.
\VS{11}Car la chose que le roi demande est extrêmement difficile et il n'y a personne qui puisse la faire connaître au roi, excepté les dieux dont la demeure n'est pas parmi les hommes.
\VS{12}A cause de cela, le roi s'irrita et se mit dans une très grande colère, et ordonna qu'on mette à mort tous les sages de Babylone.
\VS{13}La sentence fut donc publiée; on mettait à mort les sages et l'on cherchait Daniel et ses compagnons pour les tuer.
\TextTitle{Daniel implore la miséricorde de Dieu}
\VS{14}Alors Daniel détourna l'exécution du conseil et l'arrêt donné à Arjoc, chef des gardes du roi, qui était sorti pour tuer les sages de Babylone.
\VS{15}Et il demanda et dit à Arjoc, commandant du roi : Pourquoi la sentence du roi est-elle si sévère ? Arjoc exposa la chose à Daniel.
\VS{16}Et Daniel entra et pria le roi de lui accorder du temps pour donner l'interprétation au roi.
\VS{17}Alors Daniel alla dans sa maison et informa de cette affaire Hanania, Mischaël et Azaria, ses compagnons,
\VS{18}pour implorer la miséricorde du Dieu des cieux sur ce secret, afin qu'on ne mette pas à mort Daniel et ses compagnons avec le reste des sages de Babylone. 
\TextTitle{Le songe de la grande statue révélé à Daniel}
\VS{19}Et le secret fut révélé à Daniel dans une vision pendant la nuit. Et Daniel bénit le Dieu des cieux.
\VS{20}Daniel prit donc la parole et dit : Béni soit le nom de Dieu, d'éternité en éternité, car c'est à lui qu'appartiennent la sagesse et la force\FTNT{Job. 12:13 ; Ap. 5:12 ; Ap. 7:12.}!
\VS{21}C'est lui qui change les temps et les saisons, qui ôte et qui établit les rois, qui donne la sagesse aux sages et la connaissance à ceux qui ont de l'intelligence.
\VS{22}C'est lui qui révèle les choses profondes et cachées, il connaît les choses qui sont dans les ténèbres et la lumière demeure avec lui\FTNT{De. 29:29 ; Es. 48:6 ; Jé. 33:3 ; Lu. 12:2-3.}.
\VS{23}Ô Dieu de nos pères ! Je te glorifie et te loue de ce que tu m'as donné de la sagesse et de la force, et de ce que tu m'as maintenant fait connaître ce que nous t'avons demandé, en nous ayant fait connaître le secret du roi.
\VS{24}C'est pourquoi Daniel alla auprès d'Arjoc, à qui le roi avait ordonné de faire périr les sages de Babylone. Il alla et lui parla ainsi : Ne fais pas périr les sages de Babylone, mais fais-moi entrer devant le roi et je donnerai au roi l'interprétation qu'il souhaite.
\VS{25}Alors Arjoc conduisit promptement Daniel devant le roi et lui parla ainsi : J'ai trouvé parmi les captifs de Juda un homme qui donnera au roi l'interprétation de son songe.
\VS{26}Le roi prit la parole et dit à Daniel, qu'on nommait Beltschatsar : Es-tu capable de me faire connaître le songe que j'ai eu et son interprétation ?
\VS{27}Et Daniel répondit en présence du roi et dit : Ce que le roi demande est un secret que les sages, les astrologues, les magiciens et les devins ne sont pas capables de révéler au roi.
\VS{28}Mais il y a dans les cieux un Dieu qui révèle les secrets et qui a fait connaître au roi Nebucadnetsar ce qui doit arriver dans les derniers jours\FTNT{« Les derniers jours » voir commentaires dans Ge. 49:1-2.}. Voici ton songe et les visions de ta tête que tu as eues sur ta couche.
\VS{29}Sur ta couche, ô roi, il t'est monté des pensées touchant ce qui arrivera après ce temps-ci ; et celui qui révèle les secrets t'a fait connaître ce qui doit arriver.
\VS{30}Si ce secret m'a été révélé, ce n'est point qu'il y ait en moi une sagesse supérieure à celle de tous les vivants, mais c'est afin de donner au roi l'interprétation de son songe et afin que tu connaisses les pensées de ton cœur.
\VS{31}Ô roi, tu regardais et tu voyais une grande statue\FTNT{Voir annexe « La statue de Nebucadnetsar ».} ; cette grande statue, dont la splendeur était extraordinaire, était debout devant toi et son apparence était terrible.
\VS{32}La tête de cette statue était d'un or très fin, sa poitrine et ses bras étaient d'argent ; son ventre et ses cuisses étaient d'airain ;
\VS{33}ses jambes étaient de fer et ses pieds étaient en partie de fer et en partie d'argile.
\VS{34}Tu regardais cela, jusqu'à ce qu'une pierre se détacha sans main, frappa les pieds de fer et d'argile de la statue et les brisa.
\VS{35}Alors le fer, l'argile, l'airain, l'argent et l'or furent brisés ensemble et devinrent comme la paille de l'aire en été que le vent transporte çà et là ; et nulle trace n'en fut retrouvée. Mais la pierre qui avait frappé la statue devint une grande montagne et remplit toute la terre.
\TextTitle{Premier empire universel : Babylone\FTNT{Cp. Da. 7:4.}}
\VS{36}C'est là le songe. Nous en donnerons maintenant l'interprétation devant le roi.
\VS{37}Ô roi, tu es le roi des rois, parce que le Dieu des cieux t'a donné le royaume, la puissance, la force et la gloire.
\VS{38}Il a remis entre tes mains, en quelque lieu qu'ils habitent, les enfants des hommes, les bêtes des champs et les oiseaux du ciel, et il t'a fait dominer sur eux tous ; c'est toi qui es la tête d'or\FTNT{Jé. 27:6-7.}.
\TextTitle{Deuxième et troisième empire : Les Mèdes et les Perses\FTNTT{cp. Da. 7:5 ; 8:20} et la Grèce\FTNTT{cp. Da. 7:6 ; 8:21}}
\VS{39}Mais après toi, il s'élèvera un autre royaume, moindre que le tien ; et ensuite un troisième royaume qui sera d'airain et qui dominera sur toute la terre.
\TextTitle{Quatrième empire : Rome\FTNTT{Cp. Da. 7:7 ; 9:26.}}
\VS{40}Puis il y aura un quatrième royaume, fort comme du fer ; de même que le fer brise et rompt tout, ainsi il brisera et rompra tout, comme le fer qui met tout en pièces.
\VS{41}Et quant à ce que tu as vu, que les pieds et les orteils étaient en partie d'argile de potier et en partie de fer, c'est que ce royaume sera divisé, mais il y aura en lui de la force du fer, parce que tu as vu le fer mêlé avec l'argile de potier.
\VS{42}Et comme les doigts des pieds étaient en partie de fer et en partie d'argile, ce royaume sera en partie fort et en partie fragile.
\VS{43} Quant à ce que tu as vu, le fer mêlé avec l'argile de potier, c'est qu'ils se mêleront par des alliances humaines\FTNT{Le mot « alliances » vient de l'araméen « zera » qui signifie « semence » et « descendant ».} ; mais ils ne seront point unis l'un à l'autre de même que le fer ne s'allie point avec l'argile.
\TextTitle{Le royaume du Messie}
\VS{44}Dans le temps de ces rois, le Dieu des cieux suscitera un Royaume qui ne sera jamais détruit, et ce Royaume ne passera point à un autre peuple ; il brisera et anéantira tous ces royaumes-là, et lui-même sera établi éternellement.
\VS{45}Selon que tu as vu que de la montagne une pierre a été coupée sans main et qu'elle a brisé le fer, l'airain, l'argile, l'argent et l'or. Le grand Dieu a fait connaître au Roi ce qui arrivera ci-après ; or le songe est véritable et son interprétation est certaine.
\TextTitle{Yahweh, le Dieu qui revèle les secrets}
\VS{46}Alors le roi Nebucadnetsar tomba sur sa face et se prosterna devant Daniel et il ordonna qu'on lui donne de quoi faire des sacrifices et des offrandes de bonne odeur.
\VS{47}Le roi parla à Daniel et lui dit : Certainement, votre Dieu est le Dieu des dieux, et le Seigneur des rois, et il révèle les secrets, puisque tu as pu déclarer ce secret.
\VS{48}Alors le roi éleva Daniel en dignité et lui fit de nombreux et riches présents ; il l'établit gouverneur sur toute la province de Babylone et chef suprême de tous les sages de Babylone.
\VS{49}Daniel pria le roi de remettre l'intendance de la province de Babylone à Schadrac, Méschac et Abed-Négo. Et Daniel se tenait à la porte du roi.
\Chap{3}
\TextTitle{La statue d'or de Nebucadnetsar}
\VerseOne{}Le roi Nebucadnetsar fit une statue d'or\FTNT{Nebucadnetsar est un type de l'antéchrist qui s'oppose aux plans de Dieu. Contrairement à la statue composée de plusieurs métaux qu'il avait vue en songe, et où il est représenté par la tête en or (Da. 2:38), il s'est fait construire une statue entièrement en or, se déclarant ainsi symboliquement invincible et immortel. En agissant de la sorte, Nebucadnetsar se fait Dieu et exige d'être adoré (2 Th. 2:3-4). Cette statue annonçait prophétiquement la mise en place d'une religion mondiale, fruit d'un mélange entre la politique et la religion. Ces choses sont déjà bien installées, il ne manque plus que la révélation de l'impie. Nous vivons dans une époque où on oblige les chrétiens à adhérer à des organisations politiques, religieuses, sous contrôle de l'état, et cela dans le but de contrôler les individus et le message qu'ils entendent et diffusent. Ainsi, les dirigeants actuels doivent passer au préalable par des études théologiques, dont l'enseignement contredit de plus en plus la vérité biblique pour se conformer aux préceptes de ce monde. Une fois ordonnés, ils doivent affilier leurs églises à des fédérations qui sont sous contrôle de l'état. En échange des subventions, beaucoup accepteront de diluer l'évangile, privant ainsi les âmes de la vérité.}, dont la hauteur était de soixante coudées, et la largeur de six coudées. Il la dressa dans la vallée de Dura, dans la province de Babylone.
\VS{2}Puis le roi Nebucadnetsar envoya pour rassembler les satrapes, les intendants, les gouverneurs, les conseillers, les trésoriers, les jurisconsultes, les juges, et tous les magistrats des provinces, afin qu'ils se rendent à la dédicace de la statue que le roi Nebucadnetsar avait dressée.
\VS{3}Ainsi furent assemblés les satrapes, les intendants, les gouverneurs, les conseillers, les trésoriers, les jurisconsultes, les juges, et tous les magistrats des provinces, pour la dédicace de la statue que le roi Nebucadnetsar avait dressée. Ils se tenaient debout devant la statue que le roi Nebucadnetsar avait dressée.
\VS{4}Alors un héraut cria à haute voix, en disant : On vous fait savoir, ô peuples, nations, et hommes de toutes langues !
\VS{5}Au moment où vous entendrez le son du cor, du chalumeau, de la guitare, de la sambuque, du psaltérion, de la cornemuse, et de toutes sortes d'instruments de musique, vous vous jetterez à terre et vous adorerez la statue d'or que le roi Nebucadnetsar a dressée.
\VS{6}Quiconque ne se jettera pas à terre et n'adorera pas sera jeté à l'instant même au milieu de la fournaise de feu ardent.
\VS{7}C'est pourquoi, au moment même où tous les peuples entendirent le son du cor, du chalumeau, de la guitare, de la sambuque, du psaltérion, et de toutes sortes d'instruments de musique, tous les peuples, les nations, et les hommes de toutes les langues, se jetèrent à terre et adorèrent la statue d'or que le roi avait dressée.
\TextTitle{Le refus de l'idolâtrie}
\VS{8}Alors à ce même moment, certains chaldéens s'approchèrent et accusèrent les Juifs.
\VS{9}Et ils parlèrent et dirent au roi Nebucadnetsar : Roi, vis éternellement !
\VS{10}Toi, ô roi, tu as donné un ordre d'après lequel tout homme qui entendrait le son du cor, du chalumeau, de la guitare, de la sambuque, du psaltérion, de la cornemuse, et de toutes sortes d'instruments de musique, devrait se jeter à terre et adorer la statue d'or,
\VS{11}et que quiconque ne se jetterait pas à terre et ne l'adorerait pas, serait jeté au milieu d'une fournaise de feu ardent.
\VS{12}Or il y a certains Juifs que tu as établis sur les affaires de la province de Babylone, Schadrac, Méschac, et Abed-Négo ; ces hommes-là, ô roi, ne tiennent aucun compte de toi ; ils ne servent pas tes dieux, et ils n'adorent pas la statue d'or que tu as dressée.
\VS{13}Alors le roi Nebucadnetsar, saisi de colère et de fureur, ordonna qu'on amène Schadrac, Méschac, et Abed-Négo. Et ces hommes furent amenés devant le roi.
\VS{14}Et le roi Nebucadnetsar prit la parole et leur dit : Est-il vrai, Schadrac, Méschac, et Abed-Négo, que vous ne servez pas mes dieux, et que vous ne vous prosternez pas devant la statue d'or que j'ai dressée ?
\VS{15}Maintenant si vous êtes prêts, au moment où vous entendrez le son du cor, du chalumeau, de la guitare, de la sambuque, du psaltérion, de la cornemuse, et de toutes sortes d'instruments de musique, vous vous jetterez à terre, et vous adorerez la statue que j'ai faite ; si vous ne l'adorez pas, vous serez jetés à l'instant au milieu de la fournaise de feu ardent. Et qui est le dieu qui vous délivrera de mes mains ?
\VS{16}Schadrac, Méschac et Abed-Négo répondirent et dirent au roi Nebucadnetsar : Nous n'avons pas besoin de te répondre sur ce sujet.
\VS{17}Voici, notre Dieu que nous servons peut nous délivrer de la fournaise de feu ardent, et il nous délivrera de ta main, ô roi !
\VS{18}Sinon, sache, ô roi, que nous ne servirons pas tes dieux, et que nous n'adorerons pas la statue d'or que tu as dressée.
\TextTitle{L'épreuve de la fournaise de feu ardent}
\VS{19}Alors Nebucadnetsar fut rempli de fureur, et il changea de visage en tournant ses regards contre Schadrac, Méschac, et Abed-Négo. Il prit la parole et ordonna de chauffer la fournaise sept fois plus qu'on avait coutume de la chauffer.
\VS{20}Puis il commanda aux hommes les plus forts et les plus vaillants qui étaient dans son armée de lier Schadrac, Méschac, et Abed-Négo, pour les jeter dans la fournaise de feu ardent.
\VS{21}Et en même temps ces hommes furent liés avec leurs caleçons, leurs chaussures, leurs tiares, et leurs vêtements, et furent jetés au milieu de la fournaise de feu ardent.
\VS{22}Et parce que l'ordre du roi était sévère, et que la fournaise était extraordinairement chauffée, la flamme tua les hommes qui y avaient jetés, Schadrac, Méschac, et Abed-Négo.
\VS{23}Et ces trois hommes, Schadrac, Méschac, et Abed-Négo, tombèrent tous liés au milieu de la fournaise de feu ardent.
\TextTitle{La grandeur de Yahweh, le Dieu qui délivre}
\VS{24}Alors le roi Nebucadnetsar fut effrayé, et se leva précipitamment. Il prit la parole et il dit à ses conseillers : N'avons-nous pas jeté trois hommes liés au milieu du feu ? Ils répondirent et dirent au roi : Certainement, ô roi !
\VS{25}Il reprit et dit : Voici, je vois quatre hommes sans liens qui marchent au milieu du feu, et il n'y a en eux aucun dommage ; et la forme du quatrième est semblable à celle d'un fils de Dieu.
\VS{26}Alors Nebucadnetsar s'approcha vers la porte de la fournaise de feu ardent ; et prenant la parole, il dit : Schadrac, Méschac, et Abed-Négo, serviteurs du Dieu Très-Haut, sortez et venez ! Alors Schadrac, Méschac, et Abed-Négo sortirent du milieu du feu.
\VS{27}Puis les satrapes, les intendants, les gouverneurs, et les conseillers du roi s'assemblèrent pour contempler ces hommes-là, et le feu n'avait eu aucun pouvoir sur leurs corps, et aucun cheveu de leur tête n'était brûlé, et leurs caleçons n'étaient point endommagés, et l'odeur du feu n'avait pas passé sur eux.
\VS{28}Alors Nebucadnetsar prit la parole et dit : Béni soit le Dieu de Schadrac, Méschac, et Abed-Négo, lequel a envoyé son ange et délivré ses serviteurs qui ont eu confiance en lui, et qui ont violé l'ordre du roi et livré leur corps plutôt que de servir et d'adorer aucun autre dieu que leur Dieu\FTNT{Mt. 4:10 ; Ac. 4:19 ; Ac. 5:29.}.
\TextTitle{Schadrac, Méschac et Abed-Nego élevé par le roi}
\VS{29}Voici maintenant l'ordre que je donne : Tout homme, à quelque nation ou langue qu'il appartienne, qui parlera mal du Dieu de Schadrac, Méschac, et Abed-Négo, sera mis en pièces, et sa maison sera réduite en un tas d'immondices, parce qu'il n'y a aucun autre dieu qui puisse délivrer comme lui.
\VS{30}Alors le roi fit réussir Schadrac, Méschac, et Abed-Négo dans la province de Babylone.
\Chap{4}
\TextTitle{La suprématie de Yahweh déclarée aux nations}
\VerseOne{}Le roi Nebucadnetsar, à tous les peuples, aux nations, aux hommes de toutes langues qui habitent sur toute la terre : Que votre paix soit multipliée !
\VS{2}Il m'a semblé bon de vous déclarer les signes et les merveilles que le Dieu Très-Haut a opérés à mon égard.
\VS{3}Ô que ses signes sont grands, et ses merveilles pleines de force ! Son règne est un règne éternel, et sa domination subsiste de génération en génération\FTNT{Ps 102:13 ; La. 5:19 ; Lu. 1:33.}.
\TextTitle{La vision du grand arbre}
\VS{4}Moi, Nebucadnetsar, j'étais tranquille dans ma maison, et heureux dans mon palais,
\VS{5}lorsque je vis un songe qui m'épouvanta ; les pensées que j'ai eues sur ma couche et les visions de ma tête me troublèrent.
\VS{6}J'ordonnai qu'on fasse venir devant moi tous les sages de Babylone, afin qu'ils me donnent l'interprétation du songe.
\VS{7}Alors vinrent les magiciens, les astrologues, les Chaldéens et les devins. Je leur dis le songe, mais ils ne purent m'en donner l'interprétation.
\VS{8}En dernier lieu, se présenta devant moi Daniel, nommé Beltschatsar, selon le nom de mon Dieu, et qui a en lui l'Esprit des dieux saints. Je lui dis le songe :
\VS{9}Beltschatsar, chef des magiciens, comme je sais que l'Esprit des dieux saints est en toi, et que nul secret ne t'est difficile, écoute les visions que j'ai eues en songe, et donne-moi son interprétation.
\VS{10}Voici les visions de ma tête, pendant que j'étais sur ma couche. Je regardais, et voici, il y avait un arbre au milieu de la terre d'une grande hauteur.
\VS{11}Cet arbre était devenu grand et fort, sa cime atteignait les cieux, et on le voyait des extrémités de toute la terre.
\VS{12}Son feuillage était beau, et son fruit abondant, et il portait de la nourriture pour tous ; les bêtes des champs s'abritaient sous son ombre, les oiseaux du ciel habitaient dans ses branches, et tout être vivant tirait de lui sa nourriture.
\VS{13}Dans les visions de ma tête que j'avais sur ma couche, je regardais, et voici, un de ceux qui veillent et qui sont saints descendit des cieux.
\VS{14}Il cria à haute voix et parla ainsi : Coupez l'arbre, et ébranchez-le ! Secouez son feuillage, et dispersez son fruit ; que les bêtes s'enfuient de dessous, et les oiseaux du milieu de ses branches !
\VS{15}Mais laissez en terre le tronc où se trouvent ses racines, et liez-le avec des chaînes de fer et d'airain, qu'il soit parmi l'herbe des champs. Qu'il soit trempé de la rosée des cieux, et qu'il ait, comme les bêtes, l'herbe de la terre pour partage.
\VS{16}Que son cœur d'homme soit changé, et qu'un cœur de bête lui soit donné ; et que sept temps passent sur lui.
\VS{17}Cette sentence est le décret de ceux qui veillent, cette résolution est un ordre des saints, afin que les vivants sachent que le Très-Haut domine sur le royaume des hommes, qu'il le donne à qui il lui plaît, et qu'il y établit le plus vil des hommes\FTNT{Cette vérité est confirmée par l'apôtre Paul dans Ro. 13:1. C'est Dieu qui choisit souverainement qui il établit à la tête d'un pays. Selon les Ecritures, toute autorité vient de Dieu.}.
\VS{18}Voilà le songe que j'ai eu, moi, le roi Nebucadnetsar. Toi donc Beltschatsar, donnes-en l'interprétation, puisque tous les sages de mon royaume ne peuvent me la donner ; mais toi, tu le peux, parce l'Esprit des dieux saints est en toi.
\TextTitle{Interprétation de la vision}
\VS{19}Alors Daniel, nommé Beltschatsar, demeura stupéfait environ une heure, et ses pensées le troublaient. Le roi reprit et dit : Beltschatsar, que le songe et son interprétation ne te troublent pas ! Et Beltschatsar répondit : Mon seigneur, que le songe soit pour ceux qui te haïssent, et son interprétation pour tes ennemis !
\VS{20}L'arbre que tu as vu, qui était devenu grand et fort, dont la cime s'élevait jusqu'aux cieux, et qu'on voyait de tous les points de la terre ;
\VS{21}cet arbre, dont le feuillage était beau, et les fruits abondants, qui portait de la nourriture pour tous, sous lequel s'abritaient les bêtes des champs, et parmi les branches duquel les oiseaux du ciel habitaient,
\VS{22}c'est toi, ô roi, qui es devenu grand et fort, tant ta grandeur s'est accrue et s'est élevée jusqu'aux cieux, et dont la domination s'étend jusqu'aux extrémités de la terre.
\VS{23}Le roi a vu un de ceux qui veillent et qui sont saints descendre des cieux et dire : Coupez l'arbre, et ébranchez-le ! Toutefois, laissez en terre le tronc où se trouvent ses racines, et liez-le avec des chaînes de fer et d'airain, parmi l'herbe des champs, qu'il soit arrosé de la rosée du ciel, et que son partage soit avec les bêtes des champs, jusqu'à ce que sept temps soient passés sur lui.
\VS{24}Voici l'interprétation, ô roi, voici le décret du Très-Haut, qui s'accomplira sur mon seigneur, le roi.
\VS{25}On te chassera du milieu des hommes, tu auras ta demeure avec les bêtes des champs, et l'on te donnera de l'herbe comme aux bœufs, et tu seras arrosé de la rosée du ciel ; et sept temps passeront sur toi, jusqu'à ce que tu reconnaisses que le Très-Haut domine sur le royaume des hommes, et qu'il le donne à qui il lui plaît.
\VS{26}L'ordre de laisser le tronc où se trouvent les racines de cet arbre signifie que ton royaume te sera rendu, dès que tu auras reconnu que les cieux dominent.
\VS{27}C'est pourquoi, ô roi, que mon conseil te soit agréable : Rachète tes péchés par la justice, et tes iniquités en faisant miséricorde aux pauvres, et ta paix pourra se prolonger.
\TextTitle{Le roi déchu à cause de son orgueil}
\VS{28}Toutes ces choses arrivèrent au roi Nebucadnetsar.
\VS{29}Au bout de douze mois, comme il se promenait dans le palais royal de Babylone,
\VS{30}le roi prit la parole et dit : N'est-ce pas ici Babylone la grande, que j'ai bâtie pour être la maison royale, par la puissance de ma force et pour la gloire de ma magnificence ?
\VS{31}La parole était encore dans la bouche du roi, qu'une voix descendit du ciel, disant : Roi Nebucadnetsar, on t'annonce que ton royaume va t'être ôté.
\VS{32}On te chassera du milieu des hommes, tu auras ta demeure avec les bêtes des champs ; on te donnera de l'herbe à manger comme aux bœufs ; et sept temps passeront sur toi, jusqu'à ce que tu reconnaisses que le Très-Haut domine sur le royaume des hommes, et qu'il le donne à qui il lui plaît.
\VS{33}Au même instant, la parole s'accomplit sur Nebucadnetsar. Il fut chassé du milieu des hommes, il mangea de l'herbe comme les bœufs, et son corps fut arrosé de la rosée du ciel jusqu'à ce que ses cheveux croissent comme les plumes des aigles, et ses ongles comme ceux des oiseaux.
\TextTitle{Le roi est rétabli ; il loue le Dieu Très-Haut}
\VS{34}Mais à la fin de ces jours-là, moi Nebucadnetsar, je levai mes yeux vers le ciel, et la raison me revint. J'ai béni le Très-Haut, j'ai loué et glorifié celui qui vit éternellement, celui dont la domination est une domination éternelle, et dont le règne subsiste de génération en génération.
\VS{35}Tous les habitants de la terre ne sont à ses yeux que néant ; il agit comme il lui plaît avec l'armée des cieux et avec les habitants de la terre, et il n'y a personne qui empêche sa main, et qui lui dise : Que fais-tu\FTNT{Es. 45:9 ; Jé. 23:18-22 ; Ps.115:3 ; Job 9:12.} ?
\VS{36}En ce temps, la raison me revint, et je retournai à la gloire de mon royaume, ma magnificence et ma splendeur me furent rendues ; mes conseillers et mes grands me redemandèrent ; je fus rétabli dans mon royaume, et ma gloire fut augmentée.
\VS{37}Maintenant, moi, Nebucadnetsar, je loue, j'exalte, et je glorifie le Roi des cieux, dont toutes les œuvres sont véritables et les voies justes, et qui peut abaisser ceux qui marchent avec orgueil\FTNT{De. 32:4 ; Es. 13:11 ; Ez. 17:24 ; Ps. 145:17.}.
\Chap{5}
\TextTitle{Les vases du temple souillés}
\VerseOne{}Le roi Belschatsar donna un grand festin à ses grands au nombre de mille, et il buvait le vin devant ces mille courtisans.
\VS{2}Et ayant goûté au vin, Belschatsar ordonna qu'on apporte les vases d'or et d'argent que Nebucadnetsar, son père, avait enlevés du temple de Jérusalem\FTNT{2 Ch. 36:10.}, afin que le roi et ses grands, ses femmes et ses concubines, s'en servent pour boire.
\VS{3}Alors furent apportés les vases d'or qui avaient été enlevés du temple, de la maison de Dieu qui était à Jérusalem ; et le roi, et ses grands, ses femmes et ses concubines, s'en servirent pour boire.
\VS{4}Ils burent du vin, et ils louèrent leurs dieux d'or, d'argent, d'airain, de fer, de bois et de pierre.
\TextTitle{L'écriture sur la muraille}
\VS{5}Et à cette même heure-là sortirent de la muraille des doigts d'une main d'homme, qui écrivaient à l'endroit du chandelier, sur l'enduit de la muraille du palais royal ; et le roi voyait cette partie de main qui écrivait.
\VS{6}Alors l'aspect du roi changea, et ses pensées l'effrayèrent, si bien que les jointures de ses reins se desserrèrent, et ses genoux se cognèrent l'un contre l'autre.
\VS{7}Puis le roi cria avec force qu'on fasse venir les astrologues, les Chaldéens et les devins ; et le roi prit la parole et dit aux sages de Babylone : Quiconque lira cette écriture, et m'en donnera l'interprétation, sera revêtu de pourpre, il aura un collier d'or à son cou, et sera le troisième dans le gouvernement du royaume.
\VS{8}Alors tous les sages du roi entrèrent, mais ils ne purent pas lire l'écriture et en donner au roi l'interprétation.
\VS{9}Sur quoi le roi Belschatsar fut très effrayé, il changea de couleur, et ses grands furent consternés.
\TextTitle{Interprétation de l'écriture: Division de l'empire babylonien}
\VS{10}Or la reine entra dans la maison du festin, à cause de ce qui était arrivé au roi et à ses grands. La reine prit la parole et dit : Ô roi, vis éternellement ! Que tes pensées ne te troublent pas, et que ton visage ne change pas de couleur !
\VS{11}Il y a dans ton royaume un homme qui a en lui l'Esprit des dieux saints ; et du temps de ton père, on trouva en lui une lumière, une intelligence, et une sagesse semblable à la sagesse des dieux. Aussi, le roi Nebucadnetsar, ton père, et le roi, ton père\FTNT{Belschatsar était le petit-fils de Nebucadnetsar qui avait régné conjointement avec son père, Nabonide, à partir de 552 av. J.-C.}, ô roi, l'établirent chef des magiciens, des astrologues, des Chaldéens et des devins,
\VS{12}parce qu'on trouva chez lui, chez Daniel, que le roi avait nommé Beltschatsar, un esprit supérieur, de la connaissance et de l'intelligence, pour interpréter les songes, pour expliquer les questions obscures et résoudre les choses difficiles. Que Daniel soit donc appelé et il donnera l'interprétation que tu souhaites.
\VS{13}Alors Daniel fut introduit devant le roi. Le roi prit la parole et dit à Daniel : Es-tu ce Daniel, l'un des captifs de Juda, que le roi, mon père, a amenés de Juda ?
\VS{14}J'ai appris sur ton compte que tu as en toi l'esprit des dieux, et qu'on trouve en toi une lumière, une intelligence et une sagesse extraordinaires.
\VS{15}On vient d'amener devant moi les sages et les astrologues, afin qu'ils lisent cette écriture et m'en donnent l'interprétation, mais ils n'ont pas pu donner l'interprétation de la chose.
\VS{16}J'ai appris que tu peux interpréter et résoudre les choses difficiles ; maintenant donc si tu peux lire cette écriture, et m'en donner l'interprétation, tu seras revêtu de pourpre, tu porteras à ton cou un collier d'or, et tu seras le troisième dans le gouvernement du royaume.
\VS{17}Alors Daniel répondit et dit en présence du roi : Que tes dons restent à toi, et donne tes présents à un autre ; toutefois je lirai l'écriture au roi, et je lui en donnerai l'interprétation.
\VS{18}Ô roi ! Le Dieu Très-Haut avait donné à Nebucadnetsar, ton père, le royaume, la magnificence, la gloire et l'honneur.
\VS{19}Et à cause de la grandeur qu'il lui avait donnée, tous les peuples, les nations, et les hommes de toutes langues tremblaient devant lui et le redoutaient ; car il faisait mourir ceux qu'il voulait, et il laissait la vie à ceux qu'il voulait ; il élevait ceux qu'il voulait, et il abaissait ceux qu'il voulait.
\VS{20}Mais lorsque son cœur s'éleva et que son esprit s'endurcit jusqu'à l'arrogance, il fut renversé de son trône royal et dépouillé de sa gloire ;
\VS{21}il fut chassé du milieu des fils des hommes, son cœur fut rendu semblable à celui des bêtes, et sa demeure fut avec les ânes sauvages ; on lui donna comme aux bœufs de l'herbe à manger, et son corps fut arrosé de la rosée du ciel, jusqu'à ce qu'il reconnaisse que le Dieu Très-Haut domine sur les royaumes des hommes, et qu'il y établit ceux qu'il lui plaît.
\VS{22}Et toi aussi, Belschatsar, son fils, tu n'as pas humilié ton cœur, quoique tu saches toutes ces choses.
\VS{23}Mais tu t'es élevé contre le Seigneur des cieux ; les vases de sa maison ont été apportés devant toi, et vous vous en êtes servis pour boire du vin, toi et tes grands, tes femmes et tes concubines ; et tu as loué les dieux d'argent, d'or, d'airain, de fer, de bois et de pierre, qui ne voient, ni n'entendent, ni ne connaissent, et tu n'as pas glorifié le Dieu dans la main duquel est ton souffle, et toutes tes voies\FTNT{Job. 12:10 ; 33:4.}.
\VS{24}Alors de sa part a été envoyée cette partie de main, et cette écriture a été gravée.
\VS{25}Voici l'écriture qui a été gravée : Compté, compté, pesé et divisé.
\VS{26}Et voici l'interprétation de ces paroles. Compté : Dieu a compté ton règne, et y a mis la fin.
\VS{27}Pesé : Tu as été pesé dans la balance, et tu as été trouvé léger.
\VS{28}Mesuré : Ton royaume a été divisé, et donné aux Mèdes et aux Perses.
\VS{29}Aussitôt, Belschatsar donna des ordres, et l'on revêtit Daniel de pourpre, on lui mit un collier d'or au cou, et on publia qu'il serait le troisième dans le gouvernement du royaume.
\VS{30}Cette même nuit, Belschatsar, roi des Chaldéens, fut tué ;
\VS{31}et Darius, le Mède, reçut le royaume, étant âgé d'environ soixante-deux ans\FTNT{Es. 13:17 ; Es. 21:2 ; Jé. 51:11.}.
\Chap{6}
\TextTitle{Règne de Darius, le Mède}
\VerseOne{}Or Darius trouva bon d'établir sur le royaume cent vingt satrapes, qui devaient être répartis dans tout le royaume.
\VS{2}Il mit à leur tête trois chefs, au nombre desquels était Daniel, afin que ces satrapes leur rendent compte, et que le roi ne souffre aucun préjudice.
\VS{3}Mais Daniel surpassait les autres chefs et satrapes, parce qu'il y avait en lui un esprit supérieur ; et le roi pensait à l'établir sur tout le royaume.
\TextTitle{Daniel refuse l'idolâtrie et persévère dans la prière}
\VS{4}Alors les chefs et les satrapes cherchèrent une occasion d'accuser Daniel en ce qui concerne les affaires du royaume. Mais ils ne purent trouver en lui aucune occasion, ni aucune fausseté, parce qu'il était fidèle, et il ne se trouvait en lui ni faute ni vice.
\VS{5}Et ces hommes dirent : Nous ne trouverons aucune occasion d'accuser ce Daniel, à moins que nous n'en trouvions une dans la loi de son Dieu.
\VS{6}Alors ces chefs et ces satrapes s'assemblèrent vers le roi, et lui parlèrent ainsi : Roi Darius, vis éternellement !
\VS{7}Tous les chefs de ton royaume, les intendants, les satrapes, les conseillers, et les gouverneurs, sont d'avis d'établir une ordonnance royale et un décret ferme, portant que quiconque, dans l'espace de trente jours, adressera des prières à quelque dieu ou à quelque homme, excepté à toi, ô roi, sera jeté dans la fosse aux lions.
\VS{8}Maintenant donc, ô roi, établis cette ordonnance, et écris le décret afin qu'il soit irrévocable, selon la loi des Mèdes et des Perses, qui est immuable.
\VS{9}Là-dessus, le roi Darius écrivit l'ordonnance et le décret.
\VS{10}Lorsque Daniel sut que le décret était écrit, il entra dans sa maison, où les fenêtres de sa chambre étaient ouvertes dans la direction de Jérusalem ; et trois fois par jour, il se mettait à genoux, il priait, et il louait son Dieu, comme il le faisait auparavant\FTNT{1 R. 8:44 ; Ps. 55:17-18.}.
\VS{11}Alors ces hommes s'assemblèrent, et ils trouvèrent Daniel qui priait et invoquait son Dieu.
\VS{12}Puis ils s'approchèrent du roi, et lui dirent au sujet du décret royal : N'as-tu pas écrit ce décret portant que tout homme dans l'espace de trente jours qui adresserait des prières à quelque dieu ou à quelque homme, excepté à toi, ô roi, serait jeté dans la fosse aux lions ? Le roi répondit : La chose est certaine, selon la loi des Mèdes et des Perses, qui est irrévocable.
\VS{13}Ils prirent de nouveau la parole et dirent au roi : Daniel, l'un des captifs de Juda, n'a tenu aucun compte de toi, ô roi, ni du décret que tu as écrit, et il fait sa prière trois fois par jour.
\VS{14}Le roi fut très affligé quand il entendit cela ; il prit à cœur de délivrer Daniel, et jusqu'au coucher du soleil il s'efforça de le sauver.
\VS{15}Mais ces hommes s'assemblèrent auprès du roi, et lui dirent : Sache, ô roi, que la loi des Mèdes et des Perses exige que toute ordonnance ou tout décret établi par le roi soit irrévocable.
\TextTitle{Daniel demeure fidèle à Dieu face à la mort}
\VS{16}Alors le roi commanda qu'on amène Daniel, et qu'on le jette dans la fosse aux lions. Et le roi prenant la parole dit à Daniel : Ton Dieu que tu sers constamment sera celui qui te délivrera.
\VS{17}On apporta une pierre, et on la mit sur l'ouverture de la fosse ; le roi la scella de son anneau, et de l'anneau de ses grands, afin que rien ne soit changé à l'égard de Daniel.
\TextTitle{Yahweh fait justice à Daniel}
\VS{18}Après quoi le roi se rendit ensuite dans son palais ; il passa la nuit à jeun, il ne fit point venir des danseuses\FTNT{Le mot « danseuse » vient de l'araméen « dachavah » qui signifie « divertissement », « instrument de musique », « danseuse », « concubine », « musique ».} auprès de lui, et il ne put se livrer au sommeil.
\VS{19}Puis le roi se leva au point du jour, avec l'aurore, et il alla précipitamment à la fosse aux lions.
\VS{20}En s'approchant de la fosse, il cria d'une voix triste : Daniel ! Le roi prit la parole et dit à Daniel : Daniel, serviteur du Dieu vivant, ton Dieu, que tu sers avec persévérance, a-t-il pu te délivrer des lions ?
\VS{21}Alors Daniel dit au roi : Ô roi, vis éternellement !
\VS{22}Mon Dieu a envoyé son ange, et a tellement fermé la gueule des lions, qu'ils ne m'ont fait aucun mal, parce que j'ai été trouvé innocent devant lui ; et même à ton égard, ô roi, je n'ai commis aucune faute. 
\VS{23}Alors le roi fut extrêmement heureux pour lui et il ordonna qu'on fasse retirer Daniel de la fosse. Ainsi Daniel fut retiré de la fosse, et on ne trouva sur lui aucune blessure, parce qu'il avait cru en son Dieu.
\VS{24}Le roi ordonna que ces hommes qui avaient accusé Daniel, soient amenés et jetés dans la fosse aux lions, eux, leurs enfants et leurs femmes, et avant qu'ils soient parvenus au fond de la fosse, les lions se saisirent d'eux, et leur brisèrent tous les os.
\TextTitle{Les merveilles de Yahweh proclamées aux nations}
\VS{25}Après cela, le roi Darius écrivit à tous les peuples, à toutes les nations, aux hommes de toutes les langues, qui habitent sur toute la terre : Que votre paix soit multipliée !
\VS{26}J'ordonne que dans toute l'étendue de mon royaume on ait de la crainte et de la frayeur pour le Dieu de Daniel, car c'est le Dieu vivant, et il subsiste éternellement ; son Royaume ne sera jamais détruit, et sa domination durera jusqu'à la fin\FTNT{Lu. 1:33.}.
\VS{27}Il sauve et délivre, il fait des prodiges et des merveilles dans les cieux et sur la terre, et il a délivré Daniel de la puissance des lions.
\VS{28}Ainsi Daniel prospéra sous le règne de Darius, et sous le règne de Cyrus, roi de Perse.
\Chap{7}
\TextTitle{Songe des quatre animaux ; Explication des visions de Daniel}
\VerseOne{}La première année de Belschatsar, roi de Babylone, Daniel eut un songe et des visions en sa tête, étant sur sa couche. Ensuite il écrivit le songe, et il relata les principales choses.
\VS{2}Daniel donc parla et dit : Je regardais dans ma vision nocturne, et voici, les quatre vents des cieux se levèrent avec impétuosité sur la grande mer.
\VS{3}Puis quatre grandes bêtes\FTNT{Les quatre bêtes représentent les quatre empires historiques. 
Le lion :
V. 4 : Le premier animal est un lion, il représente l'Empire néo-babylonien (625 – 539 av. J.-C.). Les ailes suggèrent la rapidité de la conquête babylonienne (Jé. 4:13 ; Ha.1:6-8). En 30 ans, l'Arabie, la Judée, la Syrie et la Phénicie furent conquises.
Les ailes arrachées annoncent l'arrêt des grandes conquêtes avec la mort de Nebucadnetsar. 
Le cœur d'homme donné au lion symbolise la conversion de Nebucadnetsar et le changement dans l'attitude des rois babyloniens (2 R. 25:27-30 ; Da.4:30-31).
L'ours :
V. 5 : Le deuxième animal est un ours. Il représente l'empire médo-perse (539–331 av. J.-C.) qui succéda à l'Empire babylonien.
Le fait que l'ours se tienne sur un côté indique que les Mèdes sont soumis aux Perses qui sont les véritables maîtres de l'empire. Les trois côtes dans la gueule de l'ours symbolisent trois grandes conquêtes médo-perses : La Lydie (546 av. J.-C.), la Babylonie (539 av. J.-C.) et l'Egypte (524 av. J.-C.).
Le léopard :
V. 6 : Le troisième animal est un léopard, qui représente l'Empire gréco-macédonien (331 – 146 av. J.-C.). En 331 av. J.-C., le coup de grâce est donné aux Médo-Perses à la bataille d'Arbèles.
Les quatre ailes symbolisent la grande rapidité des conquêtes. Quand Alexandre le Grand mourut à l'âge de 33 ans, il avait le plus grand Empire jamais vu jusqu'à l'époque. Ses conquêtes s'étendaient jusqu'en Inde !
Les quatre têtes symbolisent quatre de ses généraux qui, à la mort d'Alexandre, se partagèrent l'immense empire : Cassandre en Grèce et en Macédoine, Lysimaque en Thrace et en Asie Mineure, Séleucus en Syrie et en orient, Ptolémée en Égypte.
Très rapidement, la Palestine, qui se trouvait au croisement des routes, fut l'objet de rivalités entre les généraux et leurs successeurs. Après quelques années de stabilité, les généraux luttèrent entre eux jusqu'au maintien de deux dynasties : les Séleucides, au nord, et les Lagides, au sud, en Égypte. Cela dura jusqu'à l'apparition de l'Empire romain.
La quatrième bête, est différente des autres
V. 7, 19, 24 : La quatrième bête est extraordinaire, terrible, effrayante, elle ne porte même pas de nom ! Elle représente l'Empire romain qui succéda à l'Empire gréco-macédonien (146 av. J.-C. – 476 ap. J.-C.). En 168 av. J.-C., la Macédoine passa sous le contrôle de Rome, puis, en 146 av. J.-C., c'est au tour de la Grèce de devenir une province romaine.
 Le quatrième empire ne peut être que celui de Rome comme l'enseigne l'histoire de l'antiquité. 
Dès le quatrième siècle, l'Empire romain fut assailli par les tribus barbares venues du nord (Alamans, Wisigoths, Goths, Vandales, Burgondes, Ostrogoths, etc.) et, en 476, le dernier empereur romain d'occident, Romulus Augustule, fut chassé par le roi barbare Odoacre (Goth). L'Empire romain n'est plus.
Les orteils en partie de fer et en partie d'argile représentent les nations européennes issues de la fragmentation de l'Empire romain qui a eu lieu le 4 septembre 476.} montèrent de la mer, différentes les unes des autres.
\TextTitle{Premier empire universel: Babylone\FTNTT{Cp. Da. 2:36-38.}}
\VS{4}La première était semblable à un lion, et avait des ailes d'aigle ; je la regardai jusqu'à ce que les plumes de ses ailes furent arrachées ; elle fut enlevée de terre et dressée sur ses pieds comme un homme, et un cœur d'homme lui fut donné.
\TextTitle{Deuxième empire: Les Mèdes et les Perses\FTNTT{Cp. Da. 2:39 ; 8:20.}}
\VS{5}Et voici, une deuxième bête était semblable à un ours, et se tenait sur un côté ; elle avait trois côtes dans la gueule entre ses dents ; et on lui disait ainsi : Lève-toi, mange beaucoup de chair.
\TextTitle{Troisième empire: La Grèce\FTNTT{Cp. Da. 2:39 ; 8:21-22 ; 11:2-4.}}
\VS{6}Après cela je regardai, et voici une autre bête, semblable à un léopard, qui avait sur son dos quatre ailes d'oiseau, et cette bête avait quatre têtes, et la domination lui fut donnée.
\TextTitle{Quatrième empire: Rome\FTNTT{Cp. Da. 2:40-43 ; 7:23-24 ; 9:26.}}
\VS{7}Après cela, je regardai dans mes visions nocturnes, et voici, il y avait une quatrième bête, terrible, épouvantable et extraordinairement forte ; elle avait de grandes dents de fer, elle mangeait, brisait, et elle foulait à ses pieds ce qui restait ; elle était différente de toutes les bêtes qui avaient été avant elle, et elle avait dix cornes.
\TextTitle{Les dix cornes et la petite corne\FTNTT{Da. 7:24-27.}}
\VS{8}Je considérai ses cornes, et voici, une autre petite corne sortit du milieu d'elles, et trois des premières cornes furent arrachées par elle ; et voici, il y avait en cette corne des yeux comme des yeux d'homme, et une bouche qui proférait de grandes choses.
\TextTitle{Le règne de Yahweh, l'Ancien des jours\FTNTT{Cp. Mt. 24:27-30 ; Ap. 19:11-21.}}
\VS{9}Je regardai jusqu'à ce que les trônes soient placés. Et l'Ancien des jours s'assit. Son vêtement était blanc comme la neige, et les cheveux de sa tête étaient comme de la laine pure ; son trône était des flammes de feu, et ses roues un feu ardent.
\VS{10}Un fleuve de feu sortait et se répandait de devant lui. Mille milliers le servaient, et dix mille millions se tenaient en sa présence. Le jugement se tint, et les livres furent ouverts\FTNT{1 R. 22:19 ; Ps. 68:18 ; Ap. 5:11.}.
\VS{11}Je regardai alors, à cause du bruit des paroles arrogantes que proférait la corne ; et tandis que je regardais, la bête fut tuée, et son corps fut détruit et livré pour être brûlé au feu.
\VS{12}Les autres bêtes furent dépouillées de leur domination, mais une prolongation de vie leur fut accordée jusqu'à un temps déterminé.
\TextTitle{La domination du Fils de l'homme est éternelle\FTNTT{Cp. Ap. 5:1-14.}}
\VS{13}Je regardai encore dans les visions nocturnes, et je vis, comme le Fils de l'homme, qui venait avec les nuées des cieux, et il vint jusqu'à l'Ancien des jours, et se tint devant lui\FTNT{Jud. 1:14 ; Ap. 19:14.}.
\VS{14}Et il lui donna la domination, la gloire et le règne ; et tous les peuples, les nations et les langues le serviront. Sa domination est une domination éternelle qui ne passera point, et son règne ne sera jamais détruit.
\TextTitle{Interprétation de la vision du quatrième animal}
\VS{15}Moi, Daniel, j'eus l'esprit troublé au-dedans de moi, et les visions de ma tête m'effrayèrent.
\VS{16}Je m'approchai de l'un des assistants, et lui demandai ce qu'il y avait de vrai dans toutes ces choses. Il me parla, et me donna l'interprétation de ces choses, en disant :
\VS{17}Ces quatre grandes bêtes sont quatre rois, qui s'élèveront de la terre.
\VS{18}Mais les saints du Très-Haut recevront le Royaume, et ils posséderont le Royaume éternellement, d'éternité en éternité.
\VS{19}Alors, je désirai savoir la vérité sur la quatrième bête, qui était différente de toutes les autres, extraordinairement terrible, qui avait des dents de fer et des ongles d'airain, qui mangeait, brisait, et foulait à ses pieds ce qui restait ;
\VS{20}et sur les dix cornes qu'elle avait à la tête, et sur l'autre corne qui était sortie et devant laquelle trois étaient tombées, sur cette corne qui avait des yeux, une bouche parlant avec arrogance, et une plus grande apparence que celle de ses associées.
\VS{21}Je regardai comment cette corne faisait la guerre aux saints et l'emportait sur eux\FTNT{Ap. 13:2-7.},
\VS{22}jusqu'au moment où l'Ancien des jours vint donner droit aux saints du Très-Haut, et que le temps arriva où les saints furent en possession du Royaume.
\VS{23}Il me parla donc ainsi : La quatrième bête est un quatrième royaume qui sera sur la terre, différent de tous les royaumes, et qui dévorera toute la terre, la foulera, et la brisera.
\TextTitle{Règne de l'homme impie et jugement de Dieu}
\VS{24}Mais les dix cornes sont dix rois qui s'élèveront de ce royaume. Un autre s'élèvera après eux, il sera différent des premiers, et il abattra trois rois.
\VS{25}Il proférera des paroles contre le Très-Haut, il harcelera les saints du Très-Haut, et il aura l'intention de changer les temps et la loi ; et les saints seront livrés entre ses mains pendant un temps, des temps, et la moitié d'un temps.
\VS{26}Mais le jugement se tiendra, et on lui ôtera sa domination, en la détruisant et la faisant périr, jusqu'à en voir la fin.
\VS{27}Afin que le règne, la domination, et la grandeur de tous les royaumes qui sont sous les cieux, soient donnés au peuple des saints du Très-Haut. Son royaume est un royaume éternel, et tous les royaumes lui seront assujettis et lui obéiront.
\VS{28}Jusqu'ici est la fin de cette parole. Quant à moi, Daniel, mes pensées m'effrayèrent beaucoup, et mon visage changea en moi, toutefois je gardai cette parole dans mon cœur.
\Chap{8}
\TextTitle{Vision du bélier et du bouc}
\VerseOne{}La troisième année du règne du roi Belschatsar, moi, Daniel, j'eus cette vision, en plus de celle que j'avais eue auparavant.
\VS{2}Je vis cette vision, et il arriva, comme je regardais, que j'étais à Suse, la capitale, dans la province d'Elam, et dans ma vision, je me trouvais près du fleuve d'Ulaï.
\VS{3}Et je levai mes yeux, je regardai, et voici, un bélier se tenait devant le fleuve, et il avait deux cornes ; et les deux cornes étaient hautes, mais l'une était plus haute que l'autre, et la plus haute s'éleva la dernière.
\VS{4}Je vis ce bélier qui frappait de ses cornes à l'occident, au nord, et au midi ; aucune bête ne pouvait subsister devant lui, et il n'y avait personne qui puisse délivrer de sa puissance ; et il agissait selon sa volonté et devint grand.
\VS{5}Comme je regardais attentivement, voici, un bouc d'entre les chèvres venait de l'occident, et parcourait toute la terre à sa surface, sans la toucher ; ce bouc avait entre les yeux une corne considérable.
\VS{6}Il arriva jusqu'au bélier qui avait deux cornes et que j'avais vu se tenant devant le fleuve, et il courut sur lui dans la fureur de sa force.
\VS{7}Je le vis qui s'approchait du bélier et s'irritait contre lui ; il frappa le bélier et lui brisa les deux cornes, et le bélier n'avait aucune force pour tenir ferme contre lui ; et quand il l'eut jeté par terre, il le foula; et il n'y eut personne pour délivrer le bélier de sa puissance.
\VS{8}Alors le bouc d'entre les chèvres grandit extrêmement ; mais lorsqu'il fut puissant, sa grande corne se brisa. Quatre grandes cornes s'élevèrent pour la remplacer, aux quatre vents des cieux.
\TextTitle{La petite corne renverse la vérité}
\VS{9}De l'une d'elles sortit une petite corne\FTNT{Antiochus IV Epiphane est le fils d'Antiochos III le Grand, né vers 215 av. J.-C. Il gouverna le royaume séleucide de 175 av. J.-C. à 164 av. J.-C., date de sa mort. Ce dernier avait profané le temple de Jérusalem en sacrifiant des porcs sur l'autel (Voir commentaire en Mt. 24:15). Cette petite corne, qui fait tomber par terre une partie de l'armée des étoiles, agit de même que Satan au ciel qui avait fait chuter un tiers des étoiles, soit des anges (Ap. 12:3-4).}, qui s'agrandit beaucoup vers le midi, et vers l'orient, et vers le pays de noblesse.
\VS{10}Elle s'éleva même jusqu'à l'armée des cieux, elle fit tomber à terre une partie de l'armée et des étoiles, et elle les foula\FTNT{Es. 14:12-15 ; Ez. 28:12-19.}.
\VS{11}Et elle s'éleva même jusqu'au chef de l'armée, lui enleva le sacrifice perpétuel, et renversa la demeure de son sanctuaire.
\VS{12}L'armée fut livrée avec le sacrifice perpétuel, à cause du péché ; la corne jeta la vérité par terre, et fit de grands exploits, et prospéra.
\VS{13}Alors j'entendis un saint qui parlait ; et un autre saint disait à celui qui parlait : Jusqu'à quand durera cette vision sur le sacrifice perpétuel et sur le péché qui cause la désolation ? Jusqu'à quand le sanctuaire et l'armée seront-ils foulés ?
\VS{14}Et il me dit : Deux mille trois cents soirs et matins ; puis le sanctuaire sera purifié.
\TextTitle{La vision du bélier et du bouc interprétée}
\VS{15}Et quand à moi, Daniel, j'avais cette vision et que je désirais la comprendre, voici, quelqu'un qui avait l'apparence d'un homme se tenait devant moi.
\VS{16}Et j'entendis la voix d'un homme au milieu du fleuve Ulaï ; il cria et dit : Gabriel, explique-lui la vision.
\VS{17}Puis Gabriel vint alors près du lieu où je me tenais ; et à son approche, je fus effrayé, et je tombai sur ma face. Il me dit : Comprends, fils de l'homme, car la vision est pour le temps de la fin.
\VS{18}Comme il me parlait, je restai frappé d'étourdissement, la face contre terre. Il me toucha, et me fit tenir debout à la place où je me trouvais.
\VS{19}Et il dit : Voici, je vais t'apprendre ce qui arrivera à la fin de la colère, car il y a un temps marqué pour la fin.
\VS{20}Le bélier que tu as vu qui avait deux cornes, ce sont les rois des Mèdes et des Perses ;
\VS{21}et le bouc velu, c'est le roi de Javan\FTNT{Javan ou Grèce.} ; et la grande corne entre ses yeux, c'est le premier roi.
\VS{22}Les quatre cornes qui se sont élevées pour remplacer cette corne brisée, ce sont quatre royaumes qui s'élèveront de cette nation, mais qui n'auront pas autant de force.
\TextTitle{Le roi impie, adversaire de Dieu ; la vision scellée}
\VS{23}A la fin de leur règne, lorsque les pécheurs seront consumés, il se lèvera un roi cruel et artificieux.
\VS{24}Sa puissance s'accroîtra, mais non par sa propre force ; il fera d'incroyables ravages, il réussira dans ses entreprises, et il détruira les puissants et le peuple des saints.
\VS{25}Et par la subtilité de son esprit, il fera prospérer la fraude dans sa main. Il aura de l'arrogance dans le cœur, et fera périr beaucoup d'hommes qui vivaient dans la paix, et il s'élèvera contre le Prince des princes ; mais il sera brisé, sans l'effort d'aucune main.
\VS{26}Et la vision du soir et du matin, dont il s'agit, est véritable. Mais toi, scelle la vision, car elle se rapporte à un temps éloigné.
\VS{27}Moi Daniel, je fus tout défait et malade pendant quelques jours ; puis je me levai, et je m'occupai des affaires du roi. J'étais étonné de la vision, et personne n'en eut connaissance.
\Chap{9}
\TextTitle{Supplications de Daniel à Yahweh}
\VerseOne{}La première année de Darius, fils d'Assuérus, de la race des Mèdes, lequel était établi roi sur le royaume des Chaldéens.
\VS{2}La première année, dis-je, de son règne, moi Daniel, je discernai par les livres, que le nombre des années dont Yahweh avait parlé au prophète Jérémie\FTNT{Jé. 25:11.} pour finir les désolations de Jérusalem, était de soixante-dix ans. 
\VS{3}Et je tournai ma face vers le Seigneur Dieu, pour le chercher par la prière et des supplications, avec le jeûne, et le sac et la cendre.
\VS{4}Je priai Yahweh, mon Dieu, et je lui fis ma confession : Ah ! Seigneur, Dieu grand et redoutable, toi qui gardes ton alliance et qui fais miséricorde à ceux qui t'aiment et qui gardent tes commandements !
\VS{5}Nous avons péché, nous avons commis l'iniquité, nous avons agi méchamment, nous avons été rebelles, et nous nous sommes détournés de tes commandements et de tes ordonnances.
\VS{6}Nous n'avons pas écouté tes serviteurs, les prophètes, qui ont parlé en ton Nom à nos rois, à nos chefs, à nos pères, et à tout le peuple du pays.
\VS{7}Ô Seigneur ! A toi est la justice, et à nous la confusion de face, en ce jour, aux hommes de Juda, aux habitants de Jérusalem, et à tout Israël, à ceux qui sont près et à ceux qui sont loin, dans tous les pays où tu les as dispersés, à cause des infidélités dont ils se sont rendus coupables envers toi\FTNT{Ps. 106:6 ; La. 3:42 ; Né. 9:30.}.
\VS{8}Seigneur, à nous est la confusion de face, à nos rois, à nos chefs, et à nos pères, parce que nous avons péché contre toi.
\VS{9}Auprès du Seigneur, notre Dieu, la miséricorde et le pardon, car nous avons été rebelles envers lui.
\VS{10}Nous n'avons pas écouté la voix de Yahweh, notre Dieu, pour marcher dans ses lois, qu'il avait mises devant nous par le moyen de ses serviteurs, les prophètes.
\VS{11}Et tout Israël a transgressé ta loi, et s'est détourné pour ne pas écouter ta voix. Alors se sont répandues sur nous les malédictions et les imprécations qui sont écrites dans la loi de Moïse, serviteur de Dieu, parce que nous avons péché contre Dieu\FTNT{Lé. 26:14-39 ; Né. 1:6.}.
\VS{12}Il a accompli les paroles qu'il avait prononcées contre nous, et contre nos chefs qui nous ont gouvernés, et il a fait venir sur nous un grand mal, et il n'en est jamais arrivé sous le ciel entier un semblable à celui qui est arrivé à Jérusalem.
\VS{13}Comme cela est écrit dans la loi de Moïse, ce mal est venu sur nous ; et nous n'avons pas imploré Yahweh, notre Dieu, pour nous détourner de nos iniquités, et pour nous rendre attentifs à ta vérité.
\VS{14}Yahweh a veillé sur le mal que nous avons fait et il l'a fait venir sur nous ; car Yahweh, notre Dieu, est juste dans toutes les œuvres qu'il a faites, vu que nous n'avons point obéi à sa voix.
\VS{15}Or maintenant, Seigneur, notre Dieu ! Toi qui as tiré ton peuple du pays d'Egypte par ta main puissante, et qui t'es acquis un Nom comme il l'est aujourd'hui, nous avons péché, nous avons été méchants.
\VS{16}Seigneur, je te prie que selon ta justice, que ta colère et ton indignation se détournent de ta ville de Jérusalem, de la montagne de ta sainteté ; car à cause de nos péchés et des iniquités de nos pères, Jérusalem et ton peuple sont en opprobre à tous ceux qui nous entourent.
\VS{17}Maintenant donc, ô notre Dieu, écoute la prière et les supplications de ton serviteur, et pour l'amour du Seigneur, fais briller ta face sur ton sanctuaire dévasté !
\VS{18}Mon Dieu ! Prête l'oreille, et écoute ; ouvre tes yeux, et regarde nos ruines, et la ville sur laquelle ton Nom est invoqué ; car ce n'est pas à cause de notre justice que nous te présentons nos supplications, c'est à cause de tes grandes compassions.
\VS{19}Seigneur, exauce ! Seigneur pardonne ! Seigneur sois attentif, et opère ! Ne tarde pas, par amour pour toi, ô mon Dieu ! Car ton Nom est invoqué sur ta ville, et sur ton peuple.
\VS{20}Je parlais encore, je priais, je confessais mon péché, et le péché de mon peuple d'Israël, et je présentais ma supplication à Yahweh, mon Dieu, en faveur de la sainte montagne de mon Dieu.
\TextTitle{Les soixante-dix semaines}
\VS{21}Je parlais encore dans ma prière, quand l'homme Gabriel, que j'avais vu précédemment dans une vision, s'approcha de moi d'un vol rapide au moment de l'offrande du soir.
\VS{22}Il m'instruisit, et s'entretint avec moi. Il me dit : Daniel, je suis venu maintenant pour ouvrir ton intelligence.
\VS{23}La parole est sortie dès le commencement de tes supplications, et je suis venu pour te la déclarer, car tu es un bien-aimé. Sois attentif à la parole, et comprends la vision.
\VS{24}Il y a soixante-dix semaines\FTNT{Le verset 24 concerne la chronologie de l'accomplissement de la prophétie de Jérémie (25 : 11). Les soixante-dix semaines auxquelles elle fait allusion représentent une période de 490 ans, conformément au principe biblique prophétique selon lequel un jour prophétique équivaut à une année (No. 14:33-34 ; Ez. 4:4-6). Dans les versets 25 et 27, les soixante-dix semaines sont divisées en trois périodes : 7 semaines (49 ans), 62 semaines (434 ans), et une semaine (7 ans). Les soixante-dix semaines devaient débuter au moment où la parole a annoncé que Jérusalem serait rebâtie (v. 25). En 445 avant notre ère, dans la vingtième année de son règne, le roi Artaxerxès publia un décret permettant à Esdras de retourner à Jérusalem pour achever la reconstruction de la ville (Esd. 7:6-10 ; Esd. 9:9 ; Né 2:5). Il est attesté par l'histoire profane que cette date est le point de départ de la soixante-dixième semaine de Daniel. Les 69 premières semaines vont jusqu'au Messie conducteur. La semaine qui reste (7 ans) concerne la période du règne de l'Antéchrist, celle-ci est divisée en deux période : trois ans et demi de fausse paix (1 Th. 5:3), et trois ans et demi concernent la Grande Tribulation (Ap. 7:9-17 ; Ap. 11:1-3 ; Ap. 12:6 ; Ap. 13:5).} fixées sur ton peuple et sur ta ville sainte, pour abolir la transgression et mettre fin aux péchés, faire la propitiation pour l'iniquité, pour amener la justice éternelle, pour mettre le sceau à la vision et à la prophétie et pour oindre le Saint des saints.
\VS{25}Tu sauras donc et tu comprendras, que depuis le moment où la parole a annoncé que Jérusalem sera rebâtie jusqu'au Messie, le Conducteur, il y a sept semaines et soixante-deux semaines ; et les places et les brèches seront rebâties, mais en des temps d'angoisse.
\VS{26}Et après ces soixante-deux semaines, le Messie sera retranché, mais non pas pour lui. Le peuple du chef qui viendra, détruira la ville et le sanctuaire, et sa fin arrivera comme par une inondation ; il est déterminé que les dévastations dureront jusqu'à la fin de la guerre.
\VS{27}Et il confirmera l'alliance à plusieurs pour une semaine, et à la moitié de cette semaine il fera cesser le sacrifice, et l'offrande ; puis par le moyen des ailes abominables, qui causeront la désolation, même jusqu'à une consomption déterminée, la désolation fondra sur le désolé.
\Chap{10}
\TextTitle{Daniel voit la gloire du Messie}
\VerseOne{}La troisième année de Cyrus, roi de Perse, une parole fut révélée à Daniel, qu'on nommait Beltschatsar. Cette parole est véritable et annonce une grande guerre. Il fut attentif à cette parole, et il eut l'intelligence de la vision.
\VS{2}En ce temps-là, moi Daniel, je fus dans le deuil pendant trois semaines entières.
\VS{3}Je ne mangeai aucun mets délicat, il n'entra ni viande ni vin dans ma bouche, et je ne m'oignis point, jusqu'à ce que ces trois semaines entières soient accomplies.
\VS{4}Le vingt-quatrième jour du premier mois, j'étais au bord du grand fleuve qui est Hiddékel.
\VS{5}Je levai les yeux, et je regardai, et voici, il y avait un homme vêtu de lin, et ayant sur les reins une ceinture d'or fin d'Uphaz.
\VS{6}Son corps était comme de chrysolithe, et son visage brillait comme l'éclair, ses yeux étaient comme des flammes de feu, ses bras et ses pieds ressemblaient à de l'airain poli, et le son de sa voix était comme le bruit d'une multitude de gens\FTNT{Ap. 1:12-15.}.
\VS{7}Moi, Daniel, je vis seul la vision, et les hommes qui étaient avec moi ne la virent point ; mais ils furent saisis d'une grande frayeur, et ils s'enfuirent pour se cacher.
\VS{8}Je restai seul, et je vis cette grande vision ; les forces me manquèrent, aussi mon visage changea jusqu'à être tout défait, et je ne conservai aucune vigueur.
\VS{9}J'entendis le son de ses paroles ; et comme j'entendais le son de ses paroles, je tombai frappé d'étourdissement, la face contre terre.
\TextTitle{Le combat du monde spirituel}
\VS{10}Et voici, une main me toucha et me fit mettre sur mes genoux, et sur les paumes de mes mains.
\VS{11}Puis il me dit : Daniel, homme aimé de Dieu, sois attentif aux paroles que je vais te dire, et tiens-toi debout à la place où tu es ; car je suis maintenant envoyé vers toi. Lorsqu'il m'eut ainsi parlé, je me tins debout en tremblant.
\VS{12}Et il me dit : Ne crains rien, Daniel, car dès le premier jour où tu as appliqué ton cœur à comprendre, et à t'humilier devant ton Dieu, tes paroles ont été exaucées, et c'est à cause de tes paroles que je viens.
\VS{13}Mais le chef du royaume de Perse m'a résisté vingt et un jours ; mais voici, Micaël, l'un des principaux chefs, est venu à mon secours, et je suis demeuré là auprès des rois de Perse.
\VS{14}Je viens maintenant pour te faire connaître ce qui doit arriver à ton peuple dans les derniers jours, car la vision s'étend jusqu'à ces jours-là.
\VS{15}Pendant qu'il m'adressait ces paroles, je mis mon visage contre terre, et je gardai le silence.
\VS{16}Et voici, quelqu'un qui avait l'apparence des fils de l'homme toucha mes lèvres. J'ouvris la bouche, je parlai, et je dis à celui qui se tenait devant moi : Mon seigneur ! La vision m'a rempli d'effroi, et j'ai perdu toute vigueur.
\VS{17}Comment le serviteur de mon seigneur pourrait-il parler avec mon seigneur ? Maintenant les forces me manquent, et je n'ai plus de souffle.
\VS{18}Alors celui qui avait l'apparence d'un homme me toucha encore, et me fortifia.
\VS{19}Puis il me dit : Ne crains rien, homme bien-aimé, que la paix soit avec toi ! Fortifie-toi, fortifie-toi ! Et comme il me parlait, je repris des forces, et je dis : Que mon seigneur parle, car tu m'as fortifié.
\VS{20}Il me dit : Ne sais-tu pas pourquoi je suis venu vers toi ? Maintenant je m'en retournerai pour combattre le chef de Perse ; et quand je partirai, voici, le chef de Javan viendra.
\VS{21}Mais je veux te faire connaître ce qui est écrit dans le livre de vérité. Et il n'y a personne qui me soutienne contre ceux-là, excepté Micaël, votre chef.
\Chap{11}
\TextTitle{Succession des monarques jusqu'à l'homme impie\FTNTT{Da. 11:1 - 12:13.}}
\VerseOne{}Et moi, dans la première année de Darius, le Mède, je me tenais auprès de lui pour l'aider et le fortifier.
\VS{2}Et maintenant, je vais te faire connaître la vérité : Voici, il y aura encore trois rois en Perse. Le quatrième amassera plus de richesses que les autres ; et quand il sera puissant par ses richesses, il soulèvera tout le monde contre le royaume de Javan.
\VS{3}Mais il s'élèvera un vaillant roi\FTNT{Ce vaillant roi est Alexandre le Grand qui règna de 336 à 323 av. J.-C.}, qui dominera avec une grande puissance, et fera ce qu'il voudra.
\VS{4}Et sitôt qu'il sera élevé, son royaume sera brisé et sera divisé\FTNT{A la mort d'Alexandre le Grand, ses quatre principaux généraux se partagèrent l'empire : 
-Lysimaque régna sur l'Asie mineure. 
-Cassandre régna sur la Grèce et la Macédoine.
-Seleucos régna en Syrie, en Babylonie et sur toutes les régions à l'est jusqu'aux Indes. 
-Ptolémée régna sur l'Égypte, la Judée et une partie de la Syrie.} vers les quatre vents des cieux ; il ne passera point à ses descendants, et n'aura pas la même puissance qu'il a exercée, car son royaume sera déchiré, et il passera à d'autres qu'à eux.
\VS{5}Le roi du midi\FTNT{Le roi du midi est Ptolémée 1er Soter (règne : 323-285 av. J.-C.), le chef plus fort que lui est Séleucus 1er Nicator (règne : 305-281 av. J.-C.).} deviendra fort et puissant. Mais un de ses chefs [roi de Javan] sera plus puissant que lui et dominera ; sa domination sera puissante.
\VS{6}Au bout de quelques années, ils s'allieront, et la fille du roi du midi viendra vers le roi du nord pour redresser les affaires. Mais elle ne conservera pas la force de son bras, et il ne résistera pas, ni lui ni son bras ; elle sera livrée avec ceux qui l'auront amenée, avec son père et avec celui qui aura été son soutien dans ce temps-là.
\VS{7}Mais un rejeton de ses racines s'élèvera pour le remplacer\FTNT{Ptolémée III Evergète (règne : 246-222 av. J.-C.)} ; il viendra à l'armée, il entrera dans les forteresses du roi du nord, il en disposera à son gré, et il se rendra puissant.
\VS{8}Et même il emmènera captifs en Egypte leurs dieux, avec leurs images de fonte et avec leurs vases précieux d'argent et d'or. Puis il restera quelques années de plus que le roi du nord.
\VS{9}Et le roi du midi entrera dans son royaume, mais il s'en retournera dans son pays.
\VS{10}Ses fils\FTNT{Ses fils : Ce sont les deux rois de Syrie, Seuleucus III Ceraunus (règne : 225-223 av. J.-C.), et Antiochus III le Grand (223-187 av. J.-C.).} entreront en guerre et rassembleront une multitude nombreuse de troupes ; l'un d'eux s'avancera et se répandra comme un torrent, débordera, puis reviendra ; et il poussera la guerre jusqu'à la forteresse du roi du midi.
\VS{11}Et le roi du midi sera irrité, il sortira et combattra contre lui, savoir contre le roi du nord ; il soulèvera une grande multitude, et les troupes du roi du nord seront livrées entre les mains du roi du midi.
\VS{12}Et après avoir défait cette multitude, le cœur du roi s'élèvera ; il fera tomber des milliers, mais il ne triomphera pas.
\VS{13}Car le roi du nord reviendra et rassemblera une plus grande multitude que la première ; et au bout de quelque temps, de quelques années, il viendra avec une grande armée et de grandes richesses.
\VS{14}Et en ce temps-là, plusieurs s'élèveront contre le roi du midi ; et des hommes violents parmi ton peuple se révolteront pour accomplir la vision, mais ils succomberont.
\VS{15}Le\FTNT{Le roi du nord est Séleucos IV Philopator (règne :187-175 av. J.-C.)} roi du nord viendra, il élèvera des terrasses, et prendra les villes fortes. Les bras du midi et l'élite du roi ne résisteront pas, ils manqueront de force pour résister.
\VS{16}Celui qui marchera contre lui fera ce qu'il voudra, et personne ne lui résistera ; il s'arrêtera dans le pays de noblesse, exterminant ce qui tombera sous sa main.
\VS{17}Puis il tournera sa face pour entrer avec la force de tout son royaume, et fera un accord avec le roi du midi, et il lui donnera sa fille pour femme, pour ruiner le royaume ; mais cela ne tiendra pas, et elle ne sera pas pour lui. 
\VS{18}Puis il tournera ses vues vers les îles, et il en prendra plusieurs ; mais un chef mettra fin à l'opprobre qu'il voulait lui attirer, et le fera retomber sur lui.
\VS{19}Il se dirigera ensuite vers les forteresses de son pays ; et il chancellera, il tombera, et on ne le trouvera plus.
\VS{20}Et un autre sera établi à sa place, qui fera passer un exacteur dans l'ornement du royaume, et en peu de jours il sera brisé, et ce ne sera ni par la colère ni par la guerre.
\TextTitle{Usage de la tromperie pour régner}
\VS{21}Et à sa place il en sera établi un autre qui sera méprisé, auquel on ne donnera pas l'honneur royal ; mais il viendra en paix, et il s'emparera du royaume par des flatteries.
\VS{22}Les troupes qui se répandront comme un torrent seront submergées devant lui, et brisées, de même qu'un chef de l'alliance.
\VS{23}Mais après les accords faits avec lui, il usera de tromperie, et il montera, et il aura le dessus avec peu de gens.
\VS{24} Il entrera tranquillement dans les lieux les plus riches de la province, et il fera ce que n'avaient pas fait ses pères, ni les pères de ses pères ; il distribuera le butin, le pillage et les richesses ; et il formera des desseins contre les places fortes, et cela jusqu'à un certain temps.
\VS{25}Puis il réveillera sa force et son coeur contre le roi du midi avec une grande armée. Et le roi du midi s'avancera en bataille avec une très grande et très forte armée ; mais il ne résistera pas, car on formera des complots contre lui.
\VS{26}Ceux qui mangent les mets de sa table le mettront en pièces ; son armée se répandra comme un torrent, et beaucoup de gens tomberont blessés à mort.
\VS{27}Et les deux rois chercheront en leur cœur à se nuire, et à la même table ils parleront avec fausseté. Mais cela ne réussira pas ; car la fin ne viendra qu'au temps marqué.
\VS{28}Après quoi il retournera dans son pays avec de grandes richesses ; et son cœur sera contre la sainte alliance, il agira contre elle, puis retournera dans son pays.
\VS{29}Ensuite il retournera au temps fixé, et il viendra contre le midi ; mais cette dernière expédition ne sera pas comme la précédente,
\VS{30}car les navires de Kittim viendront contre lui ; affligé, il rebroussera chemin. Puis irrité contre la sainte alliance, il agira contre elle, il retournera et s'entendra avec les apostats de la sainte alliance.
\VS{31}Et les forces seront de son côté, et on profanera le sanctuaire qui est la forteresse, et on fera cesser le sacrifice perpétuel, et on y dressera l'abomination qui causera la désolation.
\VS{32}Et il corrompra par des flatteries ceux qui agissent méchamment à l'égard de l'alliance. Mais ceux du peuple qui connaîtront leur Dieu agiront avec courage.
\VS{33}Et les plus intelligents parmi le peuple donneront instruction à plusieurs. Il en est qui succomberont pour un temps à l'épée et à la flamme, à la captivité et au pillage.
\VS{34}Dans le temps où ils succomberont, ils seront un peu secourus, et plusieurs se joindront à eux par hypocrisie.
\VS{35}Et quelques-uns des hommes intelligents succomberont, afin qu'ils soient épurés, purifiés et blanchis, jusqu'au temps de la fin, car elle n'arrivera qu'au temps marqué.
\TextTitle{Blasphème du roi contre Yahweh, le Dieu des dieux}
\VS{36}Le roi fera ce qu'il voudra, il s'élèvera, il se glorifiera au-dessus de tous les dieux ; il proférera des choses étranges contre le Dieu des dieux, il prospérera jusqu'à ce que la colère soit consommée, car ce qui est décrété sera exécuté.
\VS{37}Il n'aura égard ni aux dieux de ses pères, ni à l'amour des femmes ; il n'aura égard à aucun dieu ; car il s'élèvera au-dessus de tout.
\VS{38}Mais, à la place, il honorera le dieu Mahuzzim ; ce dieu que ses pères n'ont pas connu, il lui rendra des hommages avec de l'or et de l'argent, et des pierres précieuses, et des objets de prix.
\VS{39}C'est avec le dieu étranger qu'il agira contre les lieux les plus fortifiés ; et il comblera d'honneurs ceux qui le reconnaîtront, il les fera dominer sur plusieurs, il leur partagera des terres à prix d'argent.
\VS{40}Au temps de la fin, le roi du midi se heurtera contre lui avec ses cornes. Et le roi du nord fondra sur lui comme une tempête, avec des chars et des cavaliers, et avec de nombreux navires ; il s'avancera dans les terres, se répandra comme un torrent et débordera.
\VS{41}Et il entrera dans le pays de noblesse, et plusieurs pays succomberont ; mais Edom, Moab et les principaux des enfants d'Ammon échapperont de sa main.
\VS{42}Il étendra sa main sur ces pays-là, et le pays d'Egypte n'échappera point.
\VS{43}Il se rendra maître des trésors d'or et d'argent, et de toutes les choses précieuses de l'Egypte ; les Libyens et les Ethiopiens seront à sa suite.
\VS{44}Mais des nouvelles de l'orient et du nord viendront le troubler, et il partira avec une grande fureur, pour détruire et exterminer beaucoup de gens.
\VS{45}Et il dressera les tentes de son palais entre les mers, vers la glorieuse et sainte montagne. Puis il arrivera à la fin, et personne ne lui donnera du secours.
\Chap{12}
\TextTitle{La résurrection pour le jugement éternel}
\VerseOne{}Or en ce temps-là Michaël, ce grand chef qui tient ferme pour les enfants de ton peuple, tiendra ferme ; et ce sera un temps de détresse, tel qu'il n'y en a point eu de semblable depuis que les nations existent jusqu'à ce temps-là. En ce temps-là, ceux de ton peuple qui seront trouvés inscrits dans le livre seront sauvés.
\TextTitle{Les deux résurrections}
\VS{2}Et plusieurs de ceux qui dorment dans la poussière de la terre se réveilleront\FTNT{Il est question ici de la résurrection. Tout d'abord, il y aura la résurrection des morts en Christ, lors du retour de Jésus-Christ (1 Th. 4:12-17). Ensuite, il y aura celle de tous les saints lors du retour de Christ avec l'Eglise (Ap. 19 et 20). Enfin, la dernière résurrection interviendra à l'issue du millénium. Il s'agit de la résurrection des impies (Ap. 20:11-15). Voir également Jn. 5 : 24-29 ; Jn. 11:25.}, les uns pour la vie éternelle, et les autres pour l'opprobre et pour l'infamie éternelle.
\VS{3}Ceux qui auront été intelligents, brilleront comme la splendeur du ciel, et ceux qui auront amené plusieurs à la justice brilleront comme les étoiles, à toujours et à perpétuité\FTNT{Mt. 13:43.}.
\TextTitle{Dernières paroles de Yahweh à Daniel ; le livre scellé jusqu'au temps de la fin}
\VS{4}Mais toi, Daniel, ferme ces paroles, et scelle le livre jusqu'au temps de la fin. Beaucoup courront ça et là, et la connaissance augmentera\FTNT{Ap. 10:4 ; Ap. 5:2.}.
\VS{5}Et moi, Daniel, je regardai, et voici, deux autres hommes se tenaient debout, l'un en deçà du bord du fleuve, et l'autre au-delà du bord du fleuve.
\VS{6}Et l'un d'eux dit à l'homme vêtu de lin, qui se tenait au-dessus des eaux du fleuve : Quand sera la fin de ces merveilles ?
\VS{7}Et j'entendis l'homme vêtu de lin, qui se tenait au-dessus des eaux du fleuve ; il leva sa main droite et sa main gauche vers les cieux, et il jura par celui qui vit éternellement que ce sera dans un temps, des temps, et la moitié d'un temps, et que toutes ces choses s'accompliront quand la force du peuple saint sera entièrement brisée.
\VS{8}J'entendis, mais je ne compris pas ; et je dis : Mon seigneur, quelle sera l'issue de ces choses ?
\VS{9}Il répondit : Va, Daniel, car ces paroles seront tenues closes et scellées jusqu'au temps de la fin.
\VS{10}Plusieurs seront purifiés, blanchis et éprouvés ; mais les méchants agiront avec méchanceté, et aucun des méchants ne comprendra, mais les intelligents comprendront.
\VS{11}Or depuis le temps où cessera le sacrifice perpétuel et où sera dressée l'abomination de la désolation, il y aura mille deux cent quatre-vingt-dix jours\FTNT{Mt. 24:15 ; Mc. 13:14 ; Lu. 21:20.}.
\VS{12}Heureux celui qui attendra et qui parviendra jusqu'à mille trois cent trente-cinq jours.
\VS{13}Mais toi, va ton chemin jusqu'à la fin ; car tu te reposeras, et tu demeureras dans ton héritage à la fin des jours.
\PPE{}
\end{multicols}
