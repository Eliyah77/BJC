\ShortTitle{1 Th.}\BookTitle{1 Thessaloniciens}\BFont
\noindent\hrulefill
{\footnotesize
\textit{
\bigskip
{\centering{}
\\Auteur~: Paul
\\Thème~: Le retour de Christ
\\Date de rédaction~: Env. 51 ap. J.-C.\\}
}
\textit{
\\Autrefois appelée Therme ou Therma, qui signifie «~source chaude~», Thessalonique reçut son nouveau nom de Cassandre, en l'honneur de sa femme Thessalonike, qui était aussi la sœur d'Alexandre le Grand (356 av. J.-C. - 323 av. J.-C.), à qui il succéda. Cette ville est située au nord de la Grèce actuelle, sur la côte de la mer Egée. Du temps de Paul, ce pays était divisé en deux parties. Dans la région du nord, la Macédoine, se trouvaient les villes de Philippes, Thessalonique et Bérée. Quant à la région du sud, l'Achaïe, elle comportait les villes d'Athènes et de Corinthe. Aujourd'hui, la ville s'appelle Salonique.
\\En ce temps-là, Thessalonique comptait environ 200 000 habitants (Grecs, Romains et Juifs) et jouissait d'une importante fréquentation puisqu'elle figurait parmi les trois ports principaux de la Méditerranée et se situait sur l'une des plus grandes routes commerciales de l'époque~: La Voie Egnatienne reliant Rome à Byzance.
\\Sur le plan religieux, les habitants étaient polythéistes et pratiquaient une variété de cultes, dont le culte impérial. Durant trois semaines, Paul enseigna dans une synagogue à Thessalonique et réussit à constituer un groupe de croyants composé de Juifs, de Gentils, de pauvres et de plusieurs femmes de la haute société. Toutefois, une violente persécution l'obligea à quitter promptement la ville, laissant la communauté nouvellement formée vulnérable et fragile.
\\La première épître adressée par Paul aux Thessaloniciens leur parvint quelques mois après le passage de l'équipe apostolique et après la visite de Timothée. Cette lettre avait pour but d'affermir les Thessaloniciens dans les vérités fondamentales qui leur avaient été enseignées, de les exhorter à vivre une vie de sainteté pour être agréables à Dieu, de les éclairer quant au devenir des défunts et de les assurer du retour certain du Seigneur.\bigskip
}
}
\par\nobreak\noindent\hrulefill
\begin{multicols}{2}
\Chap{1}
\TextTitle{Introduction}
\VerseOne{}Paul, et Silvain, et Timothée, à l'église des Thessaloniciens qui est en Dieu le Père, et en Jésus-Christ, notre Seigneur~: Que la grâce et la paix vous soient données de la part de Dieu notre Père, et du Seigneur Jésus-Christ~!
\VS{2}Nous rendons toujours grâces à Dieu pour vous tous, faisant mention de vous dans nos prières,
\VS{3}en nous rappelant sans cesse l'œuvre de votre foi, le travail de votre charité, et l'immuabilité de votre espérance en notre Seigneur Jésus-Christ devant notre Dieu et Père,
\VS{4}sachant, mes frères bien-aimés de Dieu, votre élection.
\TextTitle{Proclamation de l'Evangile avec puissance et avec l'Esprit Saint}
\VS{5}Car notre Evangile ne vous a pas été prêché en paroles seulement, mais aussi en puissance, avec l'Esprit Saint, et avec une pleine persuasion~; car vous n'ignorez pas que nous nous sommes montrés ainsi parmi vous, à cause de vous.
\VS{6}Aussi avez-vous été nos imitateurs et ceux du Seigneur, ayant reçu avec la joie du Saint-Esprit, la parole au milieu de grandes afflictions,
\VS{7}de sorte que vous avez été des modèles à tous les fidèles de la Macédoine\FTNT{La Macédoine était le pays natal d'Alexandre le Grand. Elle fut conquise par les Romains et devint une province romaine, dont la capitale était Thessalonique.} et de l'Achaïe\FTNT{L'Achaïe était une province romaine placée sous l'autorité d'un proconsul résidant dans la capitale qui était Corinthe (2 Co. 1:1).}.
\VS{8}Car la parole du Seigneur a retenti de chez vous, non seulement dans la Macédoine et dans l'Achaïe mais aussi en tous lieux, et votre foi envers Dieu est si répandue, que nous n'avons pas besoin d'en parler.
\VS{9}Car eux-mêmes racontent de nous quel accès nous avons eu auprès de vous, et comment vous vous êtes convertis à Dieu en vous séparant des idoles, pour servir le Dieu vivant et vrai,
\VS{10}et pour attendre des cieux son Fils Jésus, qu'il a ressuscité des morts, et qui nous délivre de la colère à venir\FTNT{La colère à venir. Voir les sept coupes de la colère de Dieu (Ap. 15:5-8~; 16:1-21).}.
\Chap{2}
\TextTitle{Annoncer l'Evangile en recherchant l'approbation de Dieu et non celle des hommes}
\VerseOne{}Car, mes frères, vous savez vous-mêmes que notre entrée au milieu de vous n'a point été vaine. 
\VS{2}Après avoir souffert et reçu des outrages à Philippes\FTNT{Philippes était une ville de Macédoine située en Thrace, près de la côte nord de la mer Egée. Voir Ac. 16:12-40 et l'épître de Paul aux Philippiens.}, comme vous le savez, nous avons pris de l'assurance en notre Dieu, pour vous annoncer l'Evangile de Dieu au milieu de beaucoup de combats.
\VS{3}Car il n'y a eu dans notre prédication ni séduction, ni motif impur, ni fraude.
\VS{4}Mais comme Dieu nous a considérés dignes de nous confier la prédication de l'Evangile, ainsi nous parlons non comme pour plaire aux hommes, mais à Dieu qui éprouve nos cœurs.
\VS{5}Car, en effet, nous n'avons jamais été surpris avec des paroles flatteuses, comme vous le savez~; jamais nous n'avons eu pour prétexte la cupidité, Dieu en est témoin.
\VS{6}Et nous n'avons point cherché la gloire qui vient des hommes, ni de vous, ni des autres~; nous aurions pu nous imposer comme apôtres de Christ,
\VS{7}mais nous avons été doux au milieu de vous, de même qu'une nourrice chérit ses enfants.
\VS{8}Nous aurions voulu dans notre affection envers vous, non seulement vous donner l'Evangile de Dieu, mais encore notre propre vie, tant vous nous étiez devenus chers.
\VS{9}Car, mes frères, vous vous souvenez de notre peine et de notre travail~; vu que nous vous avons prêché l'Evangile de Dieu, en travaillant nuit et jour, pour n'être point à charge à aucun de vous.
\VS{10}Vous êtes témoins et Dieu aussi, combien notre conduite envers vous qui croyez a été sainte, juste, et irréprochable.
\VS{11}Et vous savez que nous avons exhorté chacun de vous, comme un père exhorte ses enfants,
\VS{12}en vous exhortant, vous encourageant et vous conjurant de vous conduire d'une manière digne de Dieu, qui vous appelle à son Royaume et à sa gloire.
\VS{13}C'est pourquoi nous rendons sans cesse grâces à Dieu, de ce que, quand vous avez reçu de nous la parole de la prédication de Dieu, vous l'avez reçue non comme une parole des hommes, mais ainsi qu'elle est véritablement, comme la parole de Dieu, laquelle aussi agit avec efficacité en vous qui croyez.
\VS{14}En effet, mes frères, vous êtes devenus les imitateurs des églises de Dieu qui sont en Jésus-Christ dans la Judée, parce que vous aussi, vous avez souffert de la part de ceux de votre propre nation les mêmes choses qu'elles ont souffertes de la part des Juifs,
\VS{15}qui ont même mis à mort le Seigneur Jésus, et leurs propres prophètes, qui nous ont persécutés, qui ne plaisent point à Dieu, et qui sont ennemis de tous les hommes,
\VS{16}nous empêchant de parler aux Gentils afin qu'ils soient sauvés, comblant ainsi toujours plus la mesure de leurs péchés. Mais la colère de Dieu est venue sur eux jusqu'au plus haut degré.
\VS{17}Pour nous, mes frères, après avoir été quelque temps séparés de vous de corps et non de cœur, nous avons eu d'autant plus d'ardeur et d'empressement de vous revoir.
\VS{18}Nous avons donc voulu, une et même deux fois, aller chez vous, au moins, moi Paul~; mais Satan nous en a empêchés.
\VS{19}Car quelle est notre espérance, ou notre joie, ou notre couronne de gloire~? N'est-ce pas vous qui l'êtes, devant notre Seigneur Jésus-Christ lors de son avènement~?
\VS{20}Certes, vous êtes notre gloire et notre joie.
\Chap{3}
\TextTitle{La persévérance des Thessaloniciens dans l'affliction}
\VerseOne{}C'est pourquoi ne pouvant plus soutenir la privation de vos nouvelles, nous avons trouvé bon de demeurer seuls à Athènes.
\VS{2}Et nous avons envoyé Timothée, notre frère, serviteur de Dieu, et notre compagnon d'œuvre dans l'Evangile de Christ, pour vous affermir et vous exhorter au sujet de votre foi,
\VS{3}afin que nul ne soit troublé dans ces afflictions, puisque vous savez vous-mêmes que nous sommes destinés à cela.
\VS{4}Et lorsque nous étions avec vous, nous vous annoncions d'avance que nous aurions à souffrir des afflictions, comme cela est aussi arrivé, et vous le savez.
\VS{5}C'est pourquoi, dis-je, ne pouvant plus soutenir cette inquiétude, j'ai envoyé Timothée pour connaître l'état de votre foi, de peur que le tentateur ne vous ait tentés en quelque sorte, et que nous n'ayons travaillé en vain.
\VS{6}Mais Timothée étant revenu depuis peu de chez vous, nous a apporté d'agréables nouvelles de votre foi et de votre charité, et nous a dit que vous conservez toujours un bon souvenir de nous, désirant nous voir, comme nous désirons aussi vous voir.
\VS{7}C'est pourquoi, mes frères, nous avons été consolés par votre foi, dans toutes nos afflictions et dans toutes nos détresses.
\VS{8}Car maintenant nous vivons puisque vous demeurez fermes dans le Seigneur.
\VS{9}Et quelles actions de grâces ne pouvons-nous pas rendre à Dieu à votre sujet, pour toute la joie que nous éprouvons devant notre Dieu, à cause de vous.
\VS{10}Nuit et jour, nous le prions avec une extrême ardeur de nous permettre de vous voir, et de compléter\FTNT{Compléter~: du grec «~katartizo~» qui signifie «~redresser~», «~ajuster~», «~compléter~», «~raccommoder~» (ce qui a été abîmé), «~réparer~». Ce verbe est également utilisé dans Mt. 4:21 lorsque Jacques et Jean réparaient leurs filets. Le terme «~katartismos~» traduit par «~perfectionnement~» dans Ep. 4:11 vient de ce verbe. Ainsi, l'un des rôles de ces services est le perfectionnement des saints et non leur destruction.} ce qui manque à votre foi.
\VS{11}Que Dieu lui-même, notre Père, et notre Seigneur Jésus-Christ, aplanisse\FTNT{Le verbe «~aplanir~» vient du grec «~kateuthuno~». On constate que ce verbe est conjugué au singulier, y compris dans le texte original grec, ce qui atteste l'unité entre le Père et le Fils (voir 2 Th. 2:16-17).} notre chemin pour que nous allions vers vous.
\VS{12}Et que le Seigneur vous fasse croître et abonder de plus en plus en charité les uns envers les autres, et envers tous, comme nous abondons aussi en charité envers vous~;
\VS{13}qu'il affermisse vos cœurs pour qu'ils soient irréprochables dans la sainteté, devant Dieu qui est notre Père, lors de l'avènement de notre Seigneur Jésus-Christ, accompagné de tous ses saints.
\Chap{4}
\TextTitle{Appel à la sanctification et à l'amour fraternel}
\VerseOne{}Au reste, mes frères, nous vous prions donc, et nous vous conjurons par le Seigneur Jésus, que comme vous avez appris de nous de quelle manière on doit se conduire, et plaire à Dieu, vous y fassiez tous les jours de nouveaux progrès.
\VS{2}Car vous savez quels préceptes nous vous avons donnés de la part du Seigneur Jésus.
\VS{3}Parce que c'est ici la volonté de Dieu~; savoir votre sanctification\FTNT{La sanctification personnelle (1 Pi. 1:15-18~; Hé. 12:14~; Ap. 22:11). Chaque chrétien doit fournir un effort, en se servant quotidiennement de la Parole de Dieu et de la prière, pour se maintenir dans la sanctification. Cela implique la séparation d'avec le mal et des mauvaises compagnies (2 Co. 6:14-18). Elle se développe au prix de nombreuses souffrances et de multiples sacrifices (Ro. 12:1-3).}, et que vous vous absteniez de la fornication,
\VS{4}c'est que chacun de vous sache posséder son corps dans la sanctification et dans l'honneur,
\VS{5}et sans se laisser aller aux désirs de la convoitise, comme les Gentils qui ne connaissent point Dieu.
\VS{6}Que personne n'use de fraude envers son frère et de cupidité dans les affaires, parce que le Seigneur tire vengeance de toutes ces choses, comme nous vous l'avons dit et attesté.
\VS{7}Car Dieu ne nous a pas appelés à l'impureté, mais à la sanctification.
\VS{8}C'est pourquoi celui qui rejette ceci ne rejette pas un homme, mais Dieu qui a aussi donné son Saint-Esprit.
\VS{9}Quant à la charité fraternelle\FTNT{Le mot grec employé ici est «~philadelphia~». Ce terme désigne l'amour fraternel, l'amour que les chrétiens se portent entre eux.}, vous n'avez pas besoin que je vous en écrive~; car vous-mêmes vous êtes enseignés de Dieu à vous aimer les uns les autres,
\VS{10}et c'est aussi ce que vous faites à l'égard de tous les frères qui sont dans toute la Macédoine. Mais, mes frères, nous vous prions de vous perfectionner tous les jours davantage,
\VS{11}et de tâcher de vivre paisiblement~; de faire vos propres affaires, et de travailler de vos propres mains, ainsi que nous vous l'avons ordonné.
\VS{12}En sorte que vous vous conduisiez honnêtement envers ceux du dehors, et que vous n'ayez besoin de rien.
\TextTitle{L'enlèvement de l'Eglise}
\VS{13}Or, mes frères, je ne veux pas que vous soyez dans l'ignorance au sujet de ceux qui dorment, afin que vous ne soyez point attristés comme les autres qui n'ont point d'espérance. 
\VS{14}Car si nous croyons que Jésus est mort, et qu'il est ressuscité~; de même aussi ceux qui dorment en Jésus, Dieu les ramènera avec lui.
\VS{15}Car nous vous disons ceci par la parole du Seigneur, que nous qui vivrons et resterons pour l'avènement du Seigneur, ne précéderons point ceux qui dorment.
\VS{16}Car le Seigneur lui-même, avec un cri de commandement\FTNT{L'expression «~cri de commandement~» vient du grec «~keleuma~», ce mot signifie un ordre, et en particulier un cri stimulant, comme celui que reçoit un animal pressé par un homme, tels les chevaux par les conducteurs de chariots, les chiens de chasse par les chasseurs, etc.~; ou par lequel un ordre est donné par le capitaine d'un navire, aux soldats par un chef, un appel de trompette. La sagesse de Dieu crie (Pr. 8). Esaïe devait crier à plein gosier (Es. 58:1). Le cri du Seigneur ne sera entendu que par l'Eglise véritable qui est son épouse (Mt. 25:6).}, et une voix d'archange, et avec la trompette de Dieu, descendra du ciel, et les morts en Christ ressusciteront premièrement.
\VS{17}Puis nous qui vivrons et qui resterons, serons enlevés ensemble avec eux dans les nuées, à la rencontre du Seigneur, dans les airs et ainsi nous serons toujours avec le Seigneur. 
\VS{18}C'est pourquoi consolez-vous les uns les autres par ces paroles.
\Chap{5}
\TextTitle{Veiller en attendant le jour du Seigneur~; encouragements divers\FTNTT{Joë. 1:15.}}
\VerseOne{}Pour ce qui est des temps et des moments, mes frères, vous n'avez pas besoin qu'on vous en écrive,
\VS{2}puisque vous savez vous-mêmes très bien que le jour du Seigneur viendra comme un voleur dans la nuit\FTNT{Mt. 25:6~; 2 Pi. 3:10~; Ap. 3:3~; 16:15.}.
\VS{3}Quand ils diront~: Nous sommes en paix et en sûreté. Alors une destruction soudaine les surprendra, comme les douleurs de l'enfantement surprennent la femme enceinte, et ils n'échapperont point.
\VS{4}Mais quant à vous, mes frères, vous n'êtes pas dans les ténèbres pour que ce jour-là vous surprenne comme un voleur.
\VS{5}Vous êtes tous des enfants de la lumière, et des enfants du jour. Nous ne sommes point de la nuit ni des ténèbres.
\VS{6}Ne dormons donc point comme les autres, mais veillons et soyons sobres.
\VS{7}Car ceux qui dorment, dorment la nuit, et ceux qui s'enivrent, s'enivrent la nuit.
\VS{8}Mais nous qui sommes enfants du jour, soyons sobres, ayant revêtu la cuirasse de la foi et de la charité, et ayant pour casque l'espérance du salut\FTNT{Ro. 13:12~; Ep. 6:14~; 6:17.}.
\VS{9}Car Dieu ne nous a pas destinés à la colère\FTNT{La colère à venir. Voir 1 Th. 1:9-10.}, mais à l'acquisition du salut par notre Seigneur Jésus-Christ,
\VS{10}qui est mort pour nous, afin que soit que nous veillons, soit que nous dormions, nous vivions avec lui.
\VS{11}C'est pourquoi exhortez-vous réciproquement, et édifiez-vous tous, les uns les autres, comme aussi vous le faites.
\VS{12}Nous vous prions, mes frères, d'avoir de la considération pour ceux qui travaillent parmi vous, qui dirigent dans le Seigneur, et qui vous exhortent.
\VS{13}Ayez pour eux beaucoup d'affection\FTNT{Littéralement «~agape~»~: amour, charité, affection.} à cause de l'œuvre qu'ils font. Soyez en paix entre vous.
\VS{14}Nous vous en prions aussi, mes frères, avertissez ceux qui vivent dans le désordre\FTNT{Mt. 18:15~; Ga. 6:1.}, consolez ceux qui ont l'esprit abattu, supportez les faibles, et soyez patients envers tous.
\VS{15}Prenez garde que personne ne rende à autrui le mal pour le mal\FTNT{Mt. 5:44~; Ro. 12:21.}~; mais recherchez toujours ce qui est bon, soit entre vous, soit envers tous les hommes.
\VS{16}Soyez toujours joyeux.
\VS{17}Priez sans cesse.
\VS{18}Rendez grâces pour toutes choses, car c'est la volonté de Dieu par Jésus-Christ.
\VS{19}N'éteignez point l'Esprit.
\VS{20}Ne méprisez point les prophéties.
\VS{21}Eprouvez toutes choses~; retenez ce qui est bon.
\VS{22}Abstenez-vous de toute apparence de mal.
\VS{23}Que le Dieu de paix veuille vous sanctifier entièrement, et faire que votre être entier, l'esprit, l'âme et le corps soient conservés sans reproche lors de la venue de notre Seigneur Jésus-Christ\FTNT{L'avènement du Seigneur. Voir Mt. 24:1-3.}.
\VS{24}Celui qui vous appelle est fidèle, c'est pourquoi il fera ces choses en vous.
\TextTitle{Salutations}
\VS{25}Mes frères, priez pour nous.
\VS{26}Saluez tous les frères par un saint baiser.
\VS{27}Je vous en conjure par le Seigneur que cette épître soit lue à tous les saints frères.
\VS{28}Que la grâce de notre Seigneur Jésus-Christ soit avec vous~! Amen~!
\PPE{}
\end{multicols}
