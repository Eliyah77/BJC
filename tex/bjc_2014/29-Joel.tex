\ShortTitle{Joël}\BookTitle{Joël}\BFont
\noindent\hrulefill
{\footnotesize
\textit{
\bigskip
{\centering{}
\\(Yoël)
\\Signifie : Yahweh est Dieu
\\Thème : Le jour de Dieu
\\Auteur : Joël
\\Date de rédaction : 9ème ou 8ème siècle av. J.-C.\\}
}
%\bigskip
\textit{
\\Joël, fils de Pethuel, exerça son ministère dans le royaume de Juda. Son message faisait suite à deux fléaux qui s’étaient abattus sur Juda, à savoir une invasion de sauterelles et la sécheresse. Il s’agissait d’un avertissement de Yahweh qui appelait le peuple à revenir à lui avec la promesse de le restaurer dans tout ce qu’il avait perdu. Joël annonça en outre l’effusion de l’Esprit sur toute chair dans un avenir lointain, prophétie ayant trouvé son accomplissement à la naissance de l’Eglise.\bigskip
}
}
\par\nobreak\noindent\hrulefill
\begin{multicols}{2}
\TextTitle{[Introduction]}
\Chap{1}
\VerseOne{}La parole de Yahweh qui fut adressée à Joël, fils de Pethuel.
\VS{2}Anciens, écoutez ceci ! Et vous tous, habitants du pays, prêtez l'oreille ! Rien de pareil est-il arrivé de votre temps, ou même du temps de vos pères ?
\VS{3}Racontez-le à vos enfants, et que vos enfants le racontent à leurs enfants, et leurs enfants à la génération suivante !
\TextTitle{[Désolation consécutive à l'invasion des sauterelles]}
\VS{4}Ce qu’a laissé le gazam, la sauterelle l’a dévoré ; ce qu’a laissé la sauterelle, le jélek l’a dévoré ; ce qu’a laissé le jélek, le hasil l’a dévoré.
\VS{5}Réveillez-vous, ivrognes, et pleurez ! Et vous tous, buveurs de vin, gémissez, parce que le moût vous est retiré de la bouche !
\VS{6}Car une nation puissante et innombrable est montée contre mon pays. Elle a les dents d’un lion, les mâchoires d’une lionne.
\VS{7}Elle a ravagé ma vigne ; elle a mis en morceaux mon figuier, elle l'a dépouillé, abattu ; les sarments de la vigne ont blanchi.
\VS{8}Lamente-toi, comme une vierge qui se revêt d'un sac pour pleurer le mari de sa jeunesse !
\VS{9}Offrandes et libations sont retranchées de la maison de Yahweh ; et les sacrificateurs, serviteurs de Yahweh, sont dans le deuil.
\VS{10}Les champs sont ravagés, la terre est dans le deuil ; parce que le blé est détruit, le moût est tari, l'huile est desséchée.
\VS{11}Les laboureurs sont confus, les vignerons gémissent, à cause du froment et de l'orge, car la moisson des champs est perdue.
\VS{12}La vigne est desséchée, le figuier languissant ; le grenadier, le palmier, le pommier, tous les arbres des champs ont séché, la joie a cessé parmi les fils de l’homme !
\VS{13}Sacrificateurs, ceignez-vous, et pleurez ! Lamentez-vous, serviteurs de l’autel ! Venez, passez la nuit revêtus de sacs, serviteurs de mon Dieu ! Car offrandes et libations ont disparu de la maison de votre Dieu.
\TextTitle{[Désolation consécutive à la sécheresse et à la famine]}
\VS{14}Consacrez un jeûne, publiez une convocation solennelle ! Assemblez les anciens, et tous les habitants du pays, dans la maison de Yahweh, votre Dieu, et criez à Yahweh !
\VS{15}Ah ! Quel Jour ! Car le jour de Yahweh\FTNT{Jour de Yahweh : Voir commentaire en Za.14:1.} est proche : Il vient comme un ravage du Tout-Puissant.
\VS{16}La nourriture n’est-elle pas retranchée sous nos yeux ? Ainsi que la joie et l'allégresse, de la maison de notre Dieu ?
\VS{17}Les semences ont séché sous les mottes, les greniers sont vides, les granges sont en ruines, car le blé s’est desséché.
\VS{18}Comme les bêtes gémissent ! Les troupeaux de bœufs sont confus, parce qu'ils n'ont point de pâturage ; et même les troupeaux de brebis en souffrent.
\VS{19}C’est vers toi que je crie, ô Yahweh ! Car le feu a consumé les pâturages du désert, et la flamme a consumé tous les arbres des champs.
\VS{20}Les bêtes des champs crient aussi vers toi ; car les torrents d’eau sont à sec, et le feu a consumé les pâturages du désert.
\TextTitle{[L'invasion victorieuse de l'armée assyrienne, venue du nord]}
\Chap{2}
\VerseOne{}Sonnez du shofar en Sion ! Faites-le retentir sur ma montagne sainte ! Que tous les habitants du pays tremblent ! Car le jour de Yahweh vient, car il est proche,
\VS{2}Jour de ténèbres et d'obscurité, jour de nuées et de brouillards, il vient comme l'aurore s'étend sur les montagnes. Voici un peuple nombreux et puissant, tel qu’il n’y en a jamais eu, et qu’il n’y en aura jamais dans la suite des siècles.
\VS{3}Devant lui est un feu dévorant, et derrière lui une flamme brûle ; le pays était, avant sa venue, comme le jardin d’Eden, et depuis, c’est un désert affreux : rien ne lui échappe.
\VS{4}Leur aspect est comme l’aspect des chevaux, et ils courent comme des cavaliers.
\VS{5}A les entendre, sur le sommet des montagnes où ils sautent, on dirait un bruit de chars, un pétillement de flamme de feu, quand elle consume le chaume. C’est comme un peuple puissant qui se prépare au combat.
\VS{6}Devant eux les peuples tremblent, tous les visages deviennent pâles et livides.
\VS{7}Ils courent comme des hommes vaillants, et montent sur les murailles comme des hommes de guerre ; chacun va son chemin, sans s’écarter de sa route.
\VS{8}Ils ne se pressent point les uns les autres, chacun va son chemin ; ils se jettent au travers des épées sans être blessés.
\VS{9}Ils courent çà et là dans la ville, se précipitent sur les murailles, montent sur les maisons, entrent par les fenêtres comme le voleur.
\VS{10}Devant eux la terre tremble, les cieux sont ébranlés, le soleil et la lune s’obscurcissent, et les étoiles retirent leur éclat.
\VS{11}Yahweh fait entendre sa voix devant son armée ; car son camp est immense, et l'exécuteur de sa parole est puissant ; car le jour de Yahweh est grand et terrible. Qui pourra le soutenir ?
\TextTitle{[Seule la repentance peut empêcher l'invasion]}
\VS{12}Maintenant encore, dit Yahweh, revenez à moi de tout votre cœur, avec des jeûnes, avec des pleurs et des lamentations !
\VS{13}Déchirez vos cœurs et non vos vêtements, et revenez à Yahweh, votre Dieu ; car il est compatissant et miséricordieux, lent à la colère et riche en bonté, et il se repent d’avoir affligé.
\VS{14}Qui sait si Yahweh, votre Dieu, ne reviendra pas et ne se repentira pas, et s'il ne laissera point après lui la bénédiction, des offrandes et des libations ?
\VS{15}Sonnez du shofar en Sion ! Sanctifiez un jeûne, publiez une convocation solennelle !
\VS{16}Assemblez le peuple, sanctifiez l’assemblée ! Réunissez les anciens, assemblez les enfants, même les nourrissons à la mamelle ! Que l’époux sorte de sa demeure, et l’épouse de sa chambre nuptiale !
\VS{17}Que les sacrificateurs qui font le service de Yahweh pleurent entre le portique et l'autel, et qu'ils disent : Yahweh ! Protège ton peuple ! N’expose pas ton héritage à l'opprobre, que les nations n’en fassent pas un sujet de railleries ! Pourquoi dirait-on parmi les peuples : Où est leur Dieu ?
\TextTitle{[Si Israël se repent, la délivrance lui est promise]}
\VS{18}Yahweh fut ému de jalousie pour son pays, et il épargna son peuple.
\VS{19}Yahweh répondit à son peuple : Voici, je vous enverrai du blé, du moût, et de l'huile, et vous en serez rassasiés ; et je ne vous exposerai plus à l'opprobre parmi les nations.
\VS{20}J'éloignerai de vous l’armée venue du nord, je la chasserai vers une terre aride et déserte, son avant-garde dans la mer orientale, son arrière-garde dans la mer occidentale ; et sa puanteur montera, et son infection s’élèvera dans les airs, parce qu’il a fait de grandes choses.
\VS{21}Terre, ne crains pas, sois dans l’allégresse et réjouis-toi, car Yahweh fait de grandes choses !
\VS{22}Bêtes des champs, ne craignez pas, car les pâturages du désert reverdiront, car les arbres porteront leurs fruits, le figuier et la vigne donneront leurs richesses.
\VS{23}Et vous, enfants de Sion, soyez dans l’allégresse et réjouissez-vous en Yahweh, votre Dieu, car il vous donnera la pluie selon sa justice, il vous enverra la pluie de la première\FTNT{La pluie de la première saison : En Orient, la première pluie tombe au moment des semailles d’automne. Elle est nécessaire afin que la semence puisse germer. Sous l'influence des pluies fertilisantes, les tendres pousses sortent du sol.} et de l’arrière-saison\FTNT{La pluie de l’arrière-saison : Elle tombe vers la fin de la saison, mûrit le grain et le prépare pour la moisson. C’est la pluie du printemps. Voir Za. 10:1 ; Os. 6:1-3 ; Jé. 5:24.}, comme autrefois.
\VS{24}Les aires se rempliront de blé, et les cuves regorgeront de moût et d'huile.
\VS{25}Je vous rendrai les fruits des années qu’ont dévorés la sauterelle, le jélek, le hasil et le gazam, ma grande armée que j’avais envoyée contre vous.
\VS{26}Vous aurez donc abondamment de quoi manger et être rassasiés, et vous louerez le Nom de Yahweh votre Dieu, qui aura fait pour vous des choses merveilleuses ; et mon peuple ne sera plus jamais dans la confusion.
\VS{27}Et vous saurez que je suis au milieu d'Israël, que je suis Yahweh, votre Dieu, et qu'il n'y en a point d'autre, et mon peuple ne sera plus jamais dans la confusion.
\TextTitle{[La promesse de l'Esprit]}
\VS{28}Après cela, je répandrai mon Esprit sur toute chair\FTNT{Cette promesse s’est réalisée dans Actes 2. Elle se réalise encore aujourd’hui dans la vie de chaque enfant de Dieu. Enfin, elle sera pleinement réalisée lors du retour du Messie en Israël (Za. 12:10-14) puisque cette prophétie annonce la repentance nationale d’Israël (Ro. 11:26-27).} ; vos fils et vos filles prophétiseront ; vos vieillards auront des songes, et vos jeunes gens des visions.
\VS{29}Même en ces jours-là, je répandrai mon Esprit sur les serviteurs et sur les servantes.
\TextTitle{[Signes précédant le jour de Yahweh]
\\(Es. 13:9-10 ; 24:21-23 ; Ez. 32:7-10 ; Mt. 24:29-30)}
\VS{30}Je ferai des prodiges dans les cieux et sur la terre, du sang, du feu, et des colonnes de fumée ;
\VS{31}Le soleil se changera en ténèbres, et la lune en sang, avant que le grand et terrible jour de Yahweh vienne.
\VS{32}Alors quiconque invoquera le Nom de Yahweh\FTNT{Quiconque invoquera le Nom de Yahweh sera sauvé. Ce passage nous confirme que Jésus-Christ est vraiment Yahweh. En effet, Paul, apôtre des païens, attribue le Nom de Yahweh et cette prophétie à Jésus-Christ (Ro. 10:9-13). C’est bien le Nom de Jésus-Christ qu'il faut invoquer pour être sauvé (Ac. 4:12 ; Ac. 9:21 ; 1 Co. 1:2). Les éditeurs de la Traduction du Monde Nouveau (bible des témoins de Jéhovah) se sont permis de «~» restituer «~» le Nom divin YHWH qui apparaît près de 6000 fois dans le Tanak, en 237 endroits dans les écrits de la nouvelle alliance, alors qu’aucun ancien manuscrit de la nouvelle alliance ne le contient. Ils affirment, sur la base d’éléments de preuves indirectes, que les scribes du IIème remplacé le Nom divin dans la nouvelle alliance par «~» Seigneur «~» ou «~» Dieu «~». Pour restituer ce Nom (YHWH), ils se basent sur les citations du Tanak où celui-ci figure et sur des versions hébraïques de la nouvelle alliance dont la plus ancienne date du XIVème pour la plupart des copies de textes plus anciens. On constate cependant qu’ils n’ont pas restitué le Nom divin en 1 Pierre 2:3 qui est pourtant une citation du Psaumes 34:8. Pourquoi ? 
Parce que l’application de ce texte à Jésus-Christ, la pierre rejetée, est évidente. Si ce texte du Tanak mentionnant Yahweh est appliqué à Jésus que penser des autres ? Jésus-Christ est vraiment Yahweh qui s’est incarné pour nous sauver. D'ailleurs, le Nom de Jésus veut dire «~» YHWH est Sauveur «~» (Es. 9:5 ; Es. 7:14 ; Mt. 1 ; Lu. 1 ; 1 Ti. 3:16).} sera sauvé ; car le salut sera sur la montagne de Sion et à Jérusalem, comme l’a dit Yahweh, et parmi les réchappés que Yahweh appellera.
\TextTitle{[Rétablissement d'Israël]
\\(Es. 11:10-12 ; Jé. 23:5-8 ; Ez. 37:21-28 ; Ac. 15:15-17)}
\Chap{3}
\VerseOne{}Car voici, en ces jours-là, et en ce temps-là, quand je ramènerai les captifs de Juda et de Jérusalem,
\TextTitle{[Jugements des nations étrangères]
\\(Za. 12:2-3 ; 14:9)}
\VS{2}Je rassemblerai toutes les nations\FTNT{Dieu rassemblera les nations dans la vallée de Josaphat (de l'hébreu «~» yehôsâphât «~», «~» Dieu juge «~) pour leur jugement. Cette vallée est peut-être celle où le roi Josaphat remporta une grande victoire, avec beaucoup de facilité, sur les Moabites, les Ammonites et les Maonites (2 Ch. 20). Cette vallée s'étend à l'orient de Jérusalem, entre la ville et le Mont des Oliviers, et traverse le torrent de Cédron.}, et je les ferai descendre dans la vallée de Josaphat ; là, j'entrerai en jugement avec elles, au sujet de mon peuple, d'Israël, mon héritage, qu’elles ont dispersé parmi les nations, et à cause de mon pays qu'elles se sont partagé.
\VS{3}Ils ont tiré mon peuple au sort ; ils ont donné l’enfant pour une prostituée, ils ont vendu la jeune fille pour du vin, et ils ont bu.
\VS{4}Que me voulez-vous, Tyr et Sidon, et vous tous, territoires des Philistins ? Voulez-vous tirer vengeance de moi ? Si vous voulez vous venger, je vous rendrai promptement et sans délai votre vengeance sur vos têtes.
\VS{5}Car vous avez pris mon argent et mon or ; et vous avez emporté dans vos temples ce que j’avais de plus précieux et de plus beau.
\VS{6}Vous avez vendu les enfants de Juda et de Jérusalem aux enfants de Javan\FTNT{Il s’agit de la Grèce.}, afin de les éloigner de leur territoire.
\VS{7}Voici, je les ferai revenir du lieu où vous les avez vendus, et je ferai retomber votre vengeance sur vos têtes.
\VS{8}Je livrerai vos fils et vos filles aux mains des enfants de Juda, et ils les vendront aux Sabéens, à une nation lointaine ; car Yahweh a parlé.
\VS{9}Publiez ceci parmi les nations ! Préparez la guerre ! Réveillez les hommes vaillants ! Qu’ils s’approchent, et qu’ils montent, tous les hommes de guerre !
\VS{10}Forgez des épées de vos hoyaux, et des lances de vos serpes ! Et que le faible dise : Je suis fort !
\VS{11}Hâtez-vous et venez, vous toutes les nations d'alentour, et rassemblez-vous ! Là, ô Yahweh, fais descendre tes hommes vaillants !
\VS{12}Que les nations se réveillent, et qu'elles montent vers la vallée de Josaphat ! Car là je siégerai pour juger toutes les nations d'alentour.
\VS{13}Saisissez la faucille, car la moisson est mûre ! Venez, et descendez, car le pressoir est plein, les cuves regorgent ! Car leur méchanceté est grande,
\VS{14}Des multitudes, des multitudes, dans la vallée de décision ; car le jour de Yahweh est proche, dans la vallée du jugement.
\VS{15}Le soleil et la lune s’obscurcissent, et les étoiles retirent leur éclat.
\VS{16}De Sion Yahweh rugit, de Jérusalem il fait entendre sa voix ; les cieux et la terre sont ébranlés. Mais Yahweh est un asile pour son peuple, un rocher protecteur\FTNT{Jésus-Christ est notre rocher (Es. 8:13-17 ; Ps. 78:35 ; 1 Co. 10:4).} pour les enfants d’Israël.
\VS{17}Et vous saurez que je suis Yahweh, votre Dieu, qui habite à Sion, ma sainte montagne. Jérusalem sera sainte, et les étrangers n'y passeront plus.
\TextTitle{[Restauration finale et pleine bénédiction du royaume]}
\VS{18}En ce jour-là, le moût ruissellera des montagnes, le lait coulera des collines, il y aura de l’eau dans tous les torrents de Juda ; et une source\FTNT{Jésus est celui qui fait jaillir en nous une source d’eau qui étanche notre soif à jamais et nous donne la vie éternelle (Jé. 2:13 ; Jé. 17:13 ; Ez. 47:1-12 ; Za. 14:8 ; Jn. 4:14 ; Ap. 22:1).} sortira de la maison de Yahweh, et arrosera la vallée de Sittim.
\VS{19}L'Egypte sera dévastée, Edom sera réduit en désert de désolation, à cause de la violence faite aux enfants de Juda, dont ils ont répandu le sang innocent dans leur pays.
\VS{20}Mais Juda sera éternellement habité, et Jérusalem, d’âge en âge.
\VS{21}Je les purifierai du sang dont je ne les avais pas encore purifiés. Et Yahweh habitera dans Sion.
\PPE{}
\end{multicols}
