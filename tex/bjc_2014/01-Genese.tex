\ShortTitle{Genèse}\BookTitle{Genèse}\BFont
\noindent\hrulefill
{\footnotesize
\textit{
\bigskip
{\centering{}
\\(Bereshit)
\\Signifie : Au commencement
\\Thème : Le Messie d'Israël
\\Auteur : Probablement Moïse
\\Date de rédaction : Env. 1450-1410 av. J.-C.\\}
}
%\bigskip
\textit{
\\Ce livre relate l’histoire des origines de l’univers, la création des cieux, de la terre et de tout ce qui s’y trouve par Yahweh, le Dieu créateur, puis la dégradation de ce même univers à cause du péché de l’homme.
%\bigskip
\\Dieu y agit comme un juge en détruisant la terre par le déluge en réponse à la méchanceté du cœur de l’homme mais Il s’y révéla aussi comme un sauveur en accordant grâce à Noé et à sa famille. Après cet évènement, les hommes se tournèrent une fois de plus vers le mal en tentant Dieu par la construction de la tour de Babel, œuvre à l’origine de la dispersion des nations. 
%\bigskip
\\On nous raconte ensuite l’élection d’un homme originaire d’Ur en Chaldée (actuelle Mésopotamie), Abraham, qui reçut la promesse de devenir une grande nation et en qui toutes les familles de la terre seraient bénies. Le récit se poursuit par l’histoire de ses descendants Isaac, Jacob et les douze fils de ce dernier qui formèrent par la suite la nation d’Israël.\bigskip
}
}
\par\nobreak\noindent\hrulefill
\begin{multicols}{2}
\TextTitle{[Création des cieux et de la terre]}
\Chap{1}
\VerseOne{}Au commencement, Dieu créa les cieux et la terre.
\TextTitle{[La terre devint informe et vide]}
\VS{2}Et la terre devint informe et vide\FTNT{Les termes « informe » et « vide » viennent des mots hébreux « tohuw » et « bohuw »  qui désignent la confusion, le chaos, la vanité.}, les ténèbres étaient à la surface de l'abîme ; et l'Esprit de Dieu se mouvait au-dessus des eaux.
\TextTitle{[Jour « un » : la lumière apparut]}
\VS{3}Dieu dit : Que la lumière apparaisse\FTNT{« Que la lumière apparaisse ! » (Es. 9:1 ; Mt. 4:12-18 ; Jn. 1:1-5). Cette lumière n’est autre que Yahweh lui-même qui va s’incarner en la personne de Jésus-Christ pour chasser les ténèbres (2 S. 22:9-12 ; Es. 60:1 ; Es. 60:19-20 ; Jn. 1:1-2 ; Jn. 8:13-14 ; 2 Co. 4:6).} ; et la lumière apparut.
\VS{4}Et Dieu vit que la lumière était bonne ; et Dieu sépara la lumière d’avec les ténèbres.
\VS{5}Dieu appela la lumière jour, et il appela les ténèbres nuit\FTNT{La lumière et les ténèbres, ainsi que leurs champs lexicaux respectifs, personnifient  souvent Jésus et Satan.  Ainsi, Jésus est la Lumière du monde (Jn. 9:5), l’Etoile brillante du matin (Ap. 22:16), le Soleil levant ou le Soleil de la justice (Ps. 19:6 ;  Mal. 4:2 ; Lc. 1:78). Il est associé au jour (Jn. 9:4) d’où les expressions  « Jour du Seigneur » (1 Th. 5:2) ou « Jour de Yahweh » (Joë. 1:15). A l’inverse, la Bible associe Satan aux ténèbres (Ps. 143:3 ; Es. 8:23 ; Ep. 6:12 ; Col. 1:13) et à la nuit (Jn. 9:4 ; Ro. 13:12).}. Ainsi fut le soir, ainsi fut le matin\FTNT{Contrairement au calendrier grégorien où le jour commence à minuit, selon Dieu et le calendrier hébraïque, le jour commence le soir à 18 heures pour se terminer le lendemain à la même heure. Voir commentaire en Mc. 16:9.} ; ce fut le  jour un\FTNT{L’hébreu utilise le terme « ehad » qui signifie « un », au sens de l’indivisible, pour qualifier le premier jour. Ce jour nous parle de Yahweh tel qu’il s’est présenté à son peuple sur le mont Sinaï en De. 6:4:« Shema Yisrael Yahweh elohénou Yahweh ehad » (« écoute Israël, Yahweh [est] notre Dieu, Yahweh [est] UN »). Un n’est pas divisible sinon on obtient un zéro ce qui équivaut au néant. Dieu est tout sauf le néant, il remplit tout (Ep. 1:23), il est partout (Ps. 139:7-13), les cieux des cieux ne peuvent le contenir (1 R. 8:27).}.
\TextTitle{[Second jour : une étendue entre les eaux]}
\VS{6}Puis Dieu dit : Qu'il y ait une étendue entre les eaux, et qu'elle sépare les eaux d'avec les eaux.
\VS{7}Dieu donc fit l'étendue, et il sépara les eaux qui sont au-dessous de l'étendue d'avec celles qui sont au-dessus de l'étendue, et il fut ainsi.
\VS{8}Et Dieu appela l'étendue cieux. Ainsi fut le soir, ainsi fut le matin ; ce fut le second jour.
\TextTitle{[Troisième jour : les mers et la terre, la végétation]}
\VS{9}Puis Dieu dit : Que les eaux qui sont au-dessous des cieux soient rassemblées en un lieu, et que le sec paraisse ; et il fut ainsi.
\VS{10}Et Dieu appela le sec terre ; et il appela l'amas des eaux mers ; et Dieu vit que cela était bon.
\VS{11}Puis Dieu dit : Que la terre produise de la verdure, de l'herbe portant de la semence, et des arbres fruitiers portant du fruit selon leur espèce, qui aient leur semence en eux-mêmes sur la terre ; et il fut ainsi.
\VS{12}La terre donc produisit de la verdure, de l'herbe portant de la semence selon son espèce ; et des arbres portant du fruit qui avaient leur semence en eux-mêmes selon leur espèce ; et Dieu vit que cela était bon.
\VS{13}Ainsi fut le soir, ainsi fut le matin ; ce fut le troisième jour.
\TextTitle{[Quatrième jour : les luminaires du ciel]}
\VS{14}Puis Dieu dit : Qu'il y ait des luminaires dans l'étendue du ciel pour séparer la nuit d'avec le jour, et qui servent de signes pour les saisons, pour les jours, et pour les années ;
\VS{15}et qu’ils servent de luminaires dans l'étendue du ciel afin d'éclairer la terre ; et il fut ainsi.
\VS{16}Dieu donc fit deux grands luminaires, le plus grand luminaire pour présider au jour, et le plus petit luminaire pour présider à la nuit ; il fit aussi les étoiles.
\VS{17}Dieu les plaça dans l'étendue du ciel pour éclairer la terre,
\VS{18}pour présider au jour et à la nuit, et pour séparer la lumière d’avec les ténèbres ; et Dieu vit que cela était bon.
\VS{19}Ainsi fut le soir, ainsi fut le matin ; ce fut le quatrième jour.
\TextTitle{[Cinquième jour : les animaux vivants dans les eaux et les airs]
\\( Ge. 2:19)}
\VS{20}Puis Dieu dit : Que les eaux produisent en toute abondance des reptiles vivants ; et qu'il y ait des oiseaux qui volent sur la terre vers l'étendue du ciel.
\VS{21}Dieu créa les grands poissons et tous les animaux vivants qui se meuvent et que les eaux produisirent en toute abondance selon leur espèce ; il créa aussi tout oiseau ayant des ailes selon son espèce ; et Dieu vit que cela était bon.
\VS{22}Dieu les bénit en disant : Soyez féconds, multipliez, et remplissez les eaux des mers ; et que les oiseaux multiplient sur la terre.
\VS{23}Ainsi fut le soir, ainsi fut le matin ; ce fut le cinquième jour.
\TextTitle{[Sixième jour : les animaux terrestres]}
\VS{24}Puis Dieu dit : Que la terre produise des animaux selon leur espèce, le bétail, les reptiles, et les bêtes de la terre selon leur espèce ; et il fut ainsi.
\VS{25}Dieu donc fit les animaux de la terre selon leur espèce, et le bétail selon son espèce, et les reptiles de la terre selon leur espèce ; et Dieu vit que cela était bon.
\TextTitle{[Sixième jour : création de l'homme]}
\VS{26}Puis Dieu dit : Faisons l'homme à notre image, selon notre ressemblance\FTNT{L’image de Dieu n’est autre que Jésus-Christ lui-même (Col. 1:15). Adam, qui signifie terrien, a été crée à  l’image du dernier Adam (1 Co. 15:40-49) qui est venu comme Fils, afin de nous montrer le modèle de fils et de filles que Dieu souhaite (Ro. 8:29). Nous avons ici une autre image de l’incarnation de Dieu en la personne de Jésus-Christ. Ainsi, avant que l’homme ne pèche, le projet de la rédemption était déjà là (1 Pi. 1:19-21).}, et qu'il domine sur les poissons de la mer, sur les oiseaux du ciel, sur le bétail, sur toute la terre, et sur tout reptile qui rampe sur la terre.
\VS{27}Dieu créa l'homme à son image, il le créa à l'image de Dieu, il les créa mâle et femelle.
\TextTitle{[Autorité de l'homme sur la création]}
\VS{28}Dieu les bénit et leur dit : Soyez féconds, multipliez, remplissez la terre, et assujettissez-la ; et dominez sur les poissons de la mer, sur les oiseaux du ciel, et sur toute bête qui se meut sur la terre.
\VS{29}Et Dieu dit : Voici, je vous donne toute herbe portant de la semence qui est sur toute la terre, et tout arbre ayant en lui du fruit d'arbre et portant de la semence, ce sera votre nourriture.
\VS{30}Et à tout animal de la terre, à tout oiseau du ciel, et à tout ce qui se meut sur la terre, ayant en soi un souffle de vie, je donne toute herbe verte pour nourriture. Et cela fut ainsi.
\VS{31}Dieu vit tout ce qu'il avait fait, et voici cela était très bon. Ainsi fut le soir, ainsi fut le matin ; ce fut le sixième jour.
\TextTitle{[Septième jour : le sabbat]}
\Chap{2}
\VerseOne{}Les cieux donc et la terre furent achevés, avec toute leur armée.
\VS{2}Dieu acheva au septième jour son œuvre qu'il avait faite, et il se reposa au septième jour de toute son œuvre qu'il avait faite.
\VS{3}Dieu bénit le septième jour, et le sanctifia, parce qu'en ce jour-là il s'était reposé de toute son œuvre qu'il avait créée en la faisant.
\VS{4}Telles sont les origines des cieux et de la terre, lorsqu'ils furent créés.
\VS{5}Lorsque Yahweh Dieu fit la terre et les cieux, aucun arbuste des champs n’était encore sur la terre, et aucune herbe des champs ne germait encore ; car Yahweh Dieu n'avait pas fait pleuvoir sur la terre, et il n'y avait point d'homme pour cultiver la terre.
\TextTitle{[Yahweh forma l'homme et le plaça en Eden]
\\(Job 10:8-9 ; Ps. 119:73)}
\VS{6}Et il monta une vapeur de la terre qui arrosa toute la surface de la terre.
\VS{7}Yahweh Dieu forma l'homme de la poussière de la terre, et il souffla dans ses narines un souffle de vie ; et l'homme devint une âme vivante.
\VS{8}Aussi Yahweh Dieu planta un jardin en Eden, du côté de l’orient, et il y mit l'homme qu'il avait formé.
\TextTitle{[L'innocence]
\\(Ge. 1:28-3:6)}
\VS{9}Yahweh Dieu fit germer de la terre des arbres de toute espèce, agréables à voir et bons à manger, et l'arbre de la vie au milieu du jardin, et l'arbre de la connaissance du bien et du mal.
\VS{10}Un fleuve sortait d'Eden pour arroser le jardin ; et de là il se divisait en quatre bras.
\VS{11}Le nom du premier est Pischon ; c'est le fleuve qui coule en entourant tout le pays de Havila où se trouve l'or.
\VS{12}L'or de ce pays est pur ; c'est là aussi que se trouvent le bdellium et la pierre d'onyx.
\VS{13}Le nom du second fleuve est Guihon ; c'est celui qui coule en entourant tout le pays de Cusch.
\VS{14}Le nom du troisième fleuve est Hiddékel, qui coule vers l'Assyrie ; et le quatrième fleuve est l'Euphrate.
\TextTitle{[L'alliance d'Eden entre Yahweh et l'homme]
\\(Ge. 1:28)}
\VS{15}Yahweh Dieu prit donc l'homme et le mit dans le jardin d'Eden pour le cultiver et pour le garder.
\VS{16}Puis Yahweh Dieu donna cet ordre à l'homme, en disant : Tu mangeras, tu mangeras\FTNT{En hebreu, le mot «akal» signifie "manger", "se nourrir", "goûter", "jouir", "dévorer", "consumer", et il a été utilisé deux fois de suite.} de tout arbre du jardin.
\VS{17}Mais quant à l'arbre de la connaissance du bien et du mal, tu n'en mangeras point, car le jour où tu en mangeras, tu mourras, tu mourras\FTNT{Dans la plupart des versions ce passage est mal traduit par "tu mourras certainement », alors que le terme mort en hébreu « muwth » est utilisé deux fois et les écritures nous parlent de la mort physique et de la seconde mort, qui est le lac de feu (Ap. 2:11 ; Ap. 20:6,14). La mort physique précède la seconde physique.}.
\TextTitle{[Yahweh forma une femme pour l'homme]
\\(Ge. 1:27)}
\VS{18}Yahweh Dieu dit : Il n'est pas bon que l'homme soit seul ; je lui ferai une aide semblable à lui.
\VS{19}Car Yahweh Dieu forma de la terre tous les animaux des champs et tous les oiseaux du ciel, puis il les fit venir vers Adam pour voir comment il les nommerait, et afin que le nom qu'Adam donnerait à tout animal fût, son nom.
\VS{20}Et Adam donna des noms à tout le bétail, et aux oiseaux du ciel, et à tous les animaux des champs ; mais pour Adam, il ne trouva point d'aide semblable à lui.
\VS{21}Et Yahweh Dieu fit tomber un profond sommeil sur Adam, qui s'endormit ; et Dieu prit une de ses côtes, et referma la chair à la place de cette côte.
\VS{22}Yahweh Dieu forma une femme de la côte qu'il avait prise d'Adam, et il l’amena vers Adam\FTNT{1 Co. 11:8.}.
\TextTitle{[Yahweh institue le mariage : Adam et Eve]}
\VS{23}Alors Adam dit : Voici cette fois celle qui est os de mes os et chair de ma chair ; on l’appellera femme, parce qu'elle a été prise de l'homme.
\VS{24}C'est pourquoi l'homme quittera son père et sa mère et s’attachera à sa femme, et ils deviendront une seule chair\FTNT{Ep. 5:30-31 ; Mt. 19:5 ; Mc. 10:7 ; 1 Co. 6:16.}.
\VS{25}Adam et sa femme étaient tous deux nus, et ils n’en avaient pas honte.
\TextTitle{[Tentation et rupture de l'alliance]}
\Chap{3}
\VerseOne{}Or le serpent\FTNT{Satan ou le serpent ancien (Ap. 12:9 ; Ap. 20:2).} était le plus prudent\FTNT{La prudence, la ruse, la subtilité du serpent, sont marquées dans l'Ecriture comme des qualités qui le distinguent des autres animaux (Mt. 10:16).} de tous les animaux des champs que Yahweh Dieu avait faits ; et il dit à la femme : Quoi ! Dieu a dit : Vous ne mangerez pas de tous les arbres du jardin ?
\VS{2}La femme répondit au serpent : Nous mangeons du fruit des arbres du jardin ;
\VS{3}mais quant au fruit de l'arbre qui est au milieu du jardin, Dieu a dit : Vous n'en mangerez point, et vous ne le toucherez point, de peur que vous ne mouriez.
\VS{4}Alors le serpent dit à la femme : Vous ne mourrez nullement ;
\VS{5}mais Dieu sait que le jour où vous en mangerez, vos yeux seront ouverts, et vous serez comme des dieux, connaissant le bien et le mal.
\VS{6}La femme donc voyant que le fruit de l'arbre était bon à manger et agréable à la vue, et que cet arbre était désirable pour donner de la science ; elle prit de son fruit, et en mangea, et elle en donna aussi à son mari qui était auprès d’elle, et il en mangea.
\TextTitle{[La connaissance du bien et du mal]}
\VS{7}Les yeux de tous les deux s’ouvrirent, ils connurent qu'ils étaient nus, et ils cousirent ensemble des feuilles de figuier, et s'en firent des ceintures.
\VS{8}Alors ils entendirent au vent du jour la voix de Yahweh Dieu qui se promenait par le jardin ; et Adam et sa femme se cachèrent loin de la face de Yahweh Dieu, au milieu des arbres du jardin.
\VS{9}Mais Yahweh Dieu appela Adam et lui dit : Où es-tu ?
\VS{10}Il répondit : J'ai entendu ta voix dans le jardin, et j'ai eu peur parce que je suis nu, et je me suis caché.
\VS{11}Et Dieu dit : Qui t'a appris que tu es nu ? Est-ce que tu as mangé du fruit de l'arbre dont je t'avais défendu de manger ?
\VS{12}Adam répondit : La femme que tu m'as donnée pour être avec moi m'a donné du fruit de l'arbre, et j'en ai mangé.
\VS{13}Et Yahweh Dieu dit à la femme : Pourquoi as-tu fait cela ? Et la femme répondit : Le serpent m'a séduite, et j'en ai mangé.
\TextTitle{[La création soumise à la vanité]
\\(Ro. 8:20-22)}
\VS{14}Alors Yahweh Dieu dit au serpent : Parce que tu as fait cela, tu seras maudit entre tout le bétail et entre tous les animaux des champs ; tu marcheras sur ton ventre, et tu mangeras la poussière\FTNT{La poussière dont il est question n’est autre que l’homme pécheur (Ge. 3:19). Satan ne peut rien contre les véritables enfants de Dieu (Mt. 16:18 ; Lu. 10:19).} tous les jours de ta vie.
\VS{15}Je mettrai inimitié entre toi et la femme\FTNT{La femme représente en premier lieu Eve, la mère de tous les hommes. Elle représente aussi Israël, l’épouse de Yahweh selon Ge. 37:5-11 et Ap. 12:1.}, et entre ta postérité\FTNT{La postérité du serpent comprend les démons (Ap. 12:3-4), l’homme impie (2 Th. 2:3-4 ; 1 Jn. 2:18-22), le monde et tous ceux qui n’ont pas reçu Jésus-Christ comme Seigneur et Sauveur. En effet, seuls ceux qui ont reçu Jésus dans leur vie sont appelés enfants de Dieu (Jn. 1:12 ; 1 Jn. 3:8-10 ; 1 Jn. 5:19).} et sa postérité\FTNT{La postérité de la femme c’est tout d’abord Israël, les Juifs étant la première postérité physique d’Eve (Ge. 49). Ensuite, il y a Jésus-Christ homme, la postérité par excellence (Es. 7:14 ; Lu. 2:4-7). Enfin, l’Église, le Corps de Christ (Col. 1:24), est une autre postérité de la femme.} ; celle-ci te brisera la tête, et tu lui blesseras le talon.
\VS{16}Et il dit à la femme : J'augmenterai beaucoup la souffrance de tes grossesses ; tu enfanteras dans la douleur tes enfants ; tes désirs se porteront vers ton mari, et il dominera sur toi.
\VS{17}Puis il dit à Adam : Parce que tu as obéi à la parole de ta femme, et que tu as mangé le fruit de l'arbre au sujet duquel je t'avais donné cet ordre, en disant : Tu n'en mangeras point, la terre sera maudite à cause de toi ; tu en mangeras les fruits dans la peine, tous les jours de ta vie.
\VS{18}Et elle te produira des épines et des chardons ; et tu mangeras l'herbe des champs.
\VS{19}C’est à la sueur de ton visage que tu mangeras du pain, jusqu'à ce que tu retournes dans la terre, d’où tu as été pris ; car tu es poussière, et tu retourneras dans la poussière.
\TextTitle{[Yahweh fit une tunique de peaux]}
\VS{20}Et Adam donna à sa femme le nom d’Eve, parce qu'elle a été la mère de tous les vivants.
\VS{21}Yahweh Dieu fit à Adam et à sa femme des tuniques de peaux, et il les en revêtit.
\VS{22}Yahweh Dieu dit : Voici, l'homme est devenu comme l'un de nous, connaissant le bien et le mal. Mais maintenant il faut prendre garde, qu’il n’avance sa main, et aussi qu’il ne prenne de l'arbre de vie, et qu’il n’en mange, et ne vive éternellement.
\TextTitle{[Yahweh ferme l'accès du jardin d'Eden]}
\VS{23}Et Yahweh Dieu le chassa du jardin d’Eden pour qu’il cultive la terre d’où il avait été pris.
\VS{24}C’est ainsi qu’il chassa l'homme, et il mit à l’orient du jardin d’Eden des chérubins qui tournent ça et là une épée flamboyante pour garder le chemin de l'arbre de vie.
\TextTitle{[La jalousie de Caïn contre son frère Abel]}
\Chap{4}
\VerseOne{}Adam connut Eve sa femme ; elle conçut, et enfanta Caïn ; et elle dit : J'ai acquis un homme de par Yahweh.
\VS{2}Elle enfanta encore Abel, son frère ; et Abel fut berger, et Caïn laboureur.
\VS{3}Or, au bout de quelque temps, Caïn offrit à Yahweh une offrande des fruits de la terre\FTNT{Caïn était du diable, il est l’archétype du religieux qui pense pouvoir être sauvé par les œuvres (Lu. 11:51 ; 1 Jn. 3:12). Son offrande fut rejetée car il avait apporté devant Dieu le fruit de la terre qui avait été maudite (Ge. 3:17).  Cela revenait à offrir à Dieu le péché, la malédiction.} ;
\VS{4}et Abel, de son côté, offrit des premiers-nés de son troupeau, et de leur graisse\FTNT{Abel était juste et pieux, aussi il sut instinctivement apporter une offrande agréable à Dieu (Mt. 23:35 ; Lu. 11:51 ; Hé. 11:4). En l’occurrence, son offrande préfigurait le sacrifice du Seigneur.}. Yahweh eut égard à Abel, et à son offrande.
\VS{5}Mais il n'eut point d'égard à Caïn ni à son offrande ; et Caïn fut fort irrité, et son visage fut abattu.
\TextTitle{[Yahweh avertit Caïn]}
\VS{6}Et Yahweh dit à Caïn : Pourquoi es-tu irrité, et pourquoi ton visage est-il abattu ?
\VS{7}Si tu agis bien, tu relèveras ton visage, et si tu agis mal, le péché est couché à la porte, et ses désirs se portent vers toi ;  mais toi, domine sur lui.
\TextTitle{[intrusion de la violence : Caïn tue son frère Abel]
\\(Ge. 4:23)}
\VS{8}Et Caïn parla avec Abel son frère, et comme ils étaient dans les champs, Caïn se jeta sur Abel, son frère, et le tua.
\VS{9}Yahweh dit à Caïn : Où est Abel ton frère ? Et il lui répondit : Je ne sais, suis-je le gardien de mon frère, moi ?
\VS{10}Et Dieu dit : Qu'as-tu fait ? La voix du sang de ton frère crie de la terre à moi.
\VS{11}Maintenant donc tu seras maudit de la terre, qui a ouvert sa bouche pour recevoir de ta main le sang de ton frère.
\VS{12}Quand tu cultiveras la terre, elle ne te donnera plus son fruit, et tu seras vagabond et fugitif sur la terre.
\VS{13}Caïn dit à Yahweh : Mon châtiment est trop grand pour être supporté.
\VS{14}Voici, tu me chasses aujourd'hui de cette terre ; je serai caché loin de ta face, je serai vagabond et fugitif sur la terre, et quiconque me trouvera me tuera.
\VS{15}Yahweh lui dit : Si quelqu’un tuait Caïn, Caïn serait vengé sept fois. Ainsi Yahweh mit une marque sur Caïn afin que quiconque le trouverait ne le tue point.
\TextTitle{[Caïn bâtit une cité loin de Yahweh]}
\VS{16}Alors, Caïn s’éloigna de la face de Yahweh, et habita dans la terre de Nod, à l'orient d’Eden.
\VS{17}Puis Caïn connut sa femme ; elle conçut et enfanta Hénoc. Il bâtit une ville, et il donna à cette ville le nom de son fils Hénoc.
\VS{18}Hénoc engendra Irad, Irad engendra Mehujaël, Mehujaël engendra Metuschaël, et Métuschaël engendra Lémec.
\VS{19}Lémec prit deux femmes ; le nom de l'une était Ada, et le nom de l'autre Tsilla.
\VS{20}Ada enfanta Jabal : Il fut le père de ceux qui habitent dans les tentes et près des troupeaux.
\VS{21}Le nom de son frère était Jubal : Il fut le père de tous ceux qui jouent de la harpe et du chalumeau.
\VS{22}Tsilla aussi enfanta Tubal-Caïn, qui forgeait toutes sortes d'instruments d'airain et de fer. La soeur de Tubal-Caïn était Naama.
\VS{23}Lémec dit à Ada et à Tsilla ses femmes : Ecoutez ma voix femmes de Lémec, écoutez ma parole ! J’ai tué un homme pour ma blessure et un jeune homme pour ma meurtrissure.
\VS{24}Car si Caïn est vengé sept fois, Lémec le sera soixante-dix-sept fois.
\VS{25}Adam connut encore sa femme ; elle enfanta un fils, et il l’appela du nom de Seth, car, dit-il, Dieu m'a donné un autre fils à la place d'Abel, que Caïn a tué.
\TextTitle{[Naissance de Seth]}
\VS{26}Il naquit aussi un fils à Seth, et il l'appela du nom d’Enosch. C’est alors que l’on commença à proclamer le nom de Yahweh.
\TextTitle{[La postérité d'Adam soumise à la mort]
\\(Ro. 5:12)}
\Chap{5}
\VerseOne{}Voici le livre de la postérité d'Adam, depuis le jour où Dieu créa l'homme, il le fit à la ressemblance de Dieu.
\VS{2}Il les créa mâle et femelle, et les bénit, et il leur donna le nom d'homme, le jour où ils furent créés.
\VS{3}Adam vécut cent trente ans, et engendra un fils à sa ressemblance, selon son image\FTNT{Désormais les hommes naissent à la ressemblance d’Adam, c’est-à-dire pécheurs (Ro. 3:23 ; Ro. 5:14-17).}, et il lui donna le nom de Seth.
\VS{4}Les jours d'Adam, après qu'il eut engendré Seth, furent de huit cents ans, et il engendra des fils et des filles.
\VS{5}Tous les jours qu'Adam vécut furent de neuf cent trente ans ; puis il mourut.
\TextTitle{[De Seth aux fils de Noé]
\\(Ro. 5:12)}
\VS{6}Seth aussi vécut cent cinq ans, et engendra Enosch.
\VS{7}Seth, après qu'il eut engendré Enosch, vécut huit cent sept ans ; et il engendra des fils et des filles.
\VS{8}Tous les jours que Seth vécut furent de neuf cent douze ans ; puis il mourut.
\VS{9}Enosch, ayant vécu quatre-vingt-dix ans, engendra Kénan.
\VS{10}Enosch, après qu'il eut engendré Kénan, vécut huit cent quinze ans, et il engendra des fils et des filles.
\VS{11}Tous les jours qu'Enosch vécut furent de neuf cent cinq ans ; puis il mourut.
\VS{12}Kénan, ayant vécu soixante-dix ans, engendra Mahalaleel.
\VS{13}Kénan, après qu'il eut engendré Mahalaleel, vécut huit cent quarante ans ; et il engendra des fils et des filles.
\VS{14}Tous les jours que Kénan vécut furent de neuf cent dix ans ; puis il mourut.
\VS{15}Mahalaleel vécut soixante-cinq ans ; et il engendra Jéred.
\VS{16}Et Mahalaleel, après qu'il eut engendré Jéred, vécut huit cent trente ans, et il engendra des fils et des filles.
\VS{17}Tous les jours donc que Mahalaleel vécut furent de huit cent quatre-vingt-quinze ans ; puis il mourut.
\VS{18}Jéred, ayant vécu cent soixante-deux ans, engendra Hénoc.
\VS{19}Jéred, après avoir engendré Hénoc, vécut huit cents ans, et il engendra des fils et des filles.
\VS{20}Tous les jours que Jéred vécut furent de neuf cent soixante-deux ans ; puis il mourut.
\VS{21}Hénoc vécut soixante-cinq ans, et engendra Metuschélah.
\VS{22}Hénoc, après qu'il eut engendré Metuschélah, marcha avec Dieu trois cents ans ; et il engendra des fils et des filles.
\VS{23}Tous les jours qu'Hénoc vécut furent de trois cent soixante-cinq ans.
\VS{24}Hénoc marcha avec Dieu ; mais il ne parut plus parce que Dieu le prit.
\VS{25}Metuschélah, ayant vécu cent quatre-vingt-sept ans, engendra Lémec.
\VS{26}Metuschélah, après qu'il eut engendré Lémec, vécut sept cent quatre-vingt-deux ans ; et il engendra des fils et des filles.
\VS{27}Tous les jours que Metuschélah vécut furent de neuf cent soixante-neuf ans ; puis il mourut.
\VS{28}Lémec aussi vécut cent quatre-vingt-deux ans, et il engendra un fils.
\VS{29}Il lui donna le nom de Noé, en disant : Celui-ci nous consolera de nos fatigues, et du travail pénible de nos mains, sur la terre que Yahweh a maudite.
\VS{30}Lémec, après qu'il eut engendré Noé, vécut cinq cent quatre-vingt-quinze ans ; et il engendra des fils et des filles.
\VS{31}Tous les jours que Lémec vécut furent de sept cent soixante-dix-sept ans ; puis il mourut.
\VS{32}Noé, âgé de cinq cents ans, engendra Sem, Cham, et Japhet.
\TextTitle{[Le mal dans le cœur de l'homme]
\\(Ro. 5:12)}
\Chap{6}
\VerseOne{}Lorsque les hommes eurent commencé à se multiplier sur la face de la terre, et qu'ils eurent engendré des filles,
\VS{2}les fils de Dieu\FTNT{Les fils de Dieu sont des anges qui ont quitté leur demeure (Jud. 1:5-7).} virent que les filles des hommes étaient belles, et ils en prirent pour femmes parmi toutes celles qu'ils choisirent.
\TextTitle{[Yahweh ne conteste plus avec les hommes]}
\VS{3}Yahweh dit : Mon Esprit ne contestera point à toujours avec les hommes\FTNT{C’est le Saint-Esprit qui nous convainc de péché, de jugement et de justice (Jn. 16:8). Lorsqu’il constate que le cœur d’une personne est définitivement endurci au point de refuser la repentance, il renonce à la convaincre de péché et il se retire. La génération antédiluvienne avait définitivement rejeté Dieu en choisissant de faire du mal son idole (Ge. 6:5). Elle était allée si loin dans l’abomination au point de s’accoupler avec des anges déchus (Ge. 6:4), ce qui laisse supposer un culte volontaire aux démons. Lorsque le Saint-Esprit est retiré d’une personne, il est remplacé par l’esprit d’égarement qui enferme le pécheur dans l’erreur et l’entraîne ainsi à sa condamnation éternelle (Mt. 12:31 ; 2 Th. 2:11 )}, car les hommes ne sont que chair, et leurs jours seront de cent vingt ans.
\TextTitle{[Le monde avant le déluge]
\\(Luc 17.27)}
\VS{4}Les géants étaient sur la terre en ce temps-là. Il en fut de même après que les fils de Dieu furent venus vers les filles des hommes, et qu’elles leur eurent donné des enfants. Ce sont ces hommes vaillants qui furent des gens de renom dans l’antiquité.
\TextTitle{[Yahweh prépare un jugement]}
\VS{5}Yahweh vit que la méchanceté des hommes était très grande sur la terre, et que toute l'imagination des pensées de leur cœur n'était que mauvaise en tout temps.
\VS{6}Yahweh se repentit d'avoir fait l'homme sur la terre, et il fut affligé en son cœur.
\VS{7}Et Yahweh dit : J'exterminerai de la face de la terre les hommes que j'ai créés, depuis les hommes jusqu'au bétail, jusqu'aux reptiles, et même jusqu'aux oiseaux du ciel ; car je me repens de les avoir faits.
\TextTitle{[La grâce de Yahweh sur Noé : construction de l'arche]}
\VS{8}Mais Noé trouva grâce aux yeux de Yahweh.
\VS{9}Voici la postérité de Noé. Noé était un homme juste et intègre en son temps ; Noé marchait avec Dieu.
\VS{10}Noé engendra trois fils : Sem, Cham, et Japhet.
\VS{11}Et la terre était corrompue devant Dieu, et remplie d'extorsion.
\VS{12}Dieu donc regarda la terre, et voici elle était corrompue ; car toute chair avait corrompu sa voie sur la terre.
\VS{13}Et Dieu dit à Noé : La fin de toute chair est venue devant moi ; car ils ont rempli la terre de violence, et voici, je les détruirai avec la terre.
\VS{14}Fais-toi une arche\FTNT{L’arche  est un type de Christ et du salut en lui et par lui.  On peut voir plusieurs aspects de Jésus-Christ et de la rédemption dans la structure de l’arche :
- L’arche a été une révélation de Jésus-Christ donnée à Noé.  C’est en Jésus-Christ que nous avons le salut et la protection (Col. 1:12-13 ; Col. 3:3). 
-L’arche était faite de bois de gopher, probablement du cèdre.  Ce bois est un bois qui ne pourrit pas en condition normale. Ce bois préfigurait l’incorruptibilité de Jésus-Christ homme (Es. 53:9 ;  Hé. 4:15 ; 1 Pi. 2:22). 
-Au verset 14, on lit que Dieu demande à Noé d’enduire l’arche en dedans et en dehors avec de la  poix, c’est-à-dire du bitume.  Le mot « poix » vient de l’hébreu « kaphar », qui signifie « expiation ».  Ce mot est traduit près de 70 fois dans le Tanakh par expiation.  Il est également traduit par « réconciliation », « pardon », « miséricordieux » et « apaiser ».  L’allusion à l’expiation des péchés faite par Jésus-Christ est claire. Par son sacrifice, nous sommes rendus parfaits à jamais (Hé. 10:14-15).} de bois de gopher ; tu feras cette arche en cellules, et tu l’enduiras de poix en dedans et en dehors.
\VS{15}Et voici comment tu la feras : La longueur de l'arche sera de trois cents coudées ; sa largeur de cinquante coudées, et sa hauteur de trente coudées.
\VS{16}Tu feras une fenêtre à l'arche, et feras son comble d'une coudée de hauteur, et tu mettras la porte de l'arche à son côté, et tu la feras avec un bas, un second, et un troisième étage.
\VS{17}Et voici, je ferai venir un déluge d'eau sur la terre, pour détruire toute chair dans laquelle il y a souffle de vie sous les cieux ; et tout ce qui est sur la terre expirera.
\VS{18}Mais j'établirai mon alliance avec toi ; et tu entreras dans l'arche toi et tes fils, et ta femme, et les femmes de tes fils avec toi.
\VS{19}Et de tout ce qui a vie d'entre toute chair, tu en feras entrer deux de chaque espèce dans l'arche, pour les conserver en vie avec toi, à savoir le mâle et la femelle.
\VS{20}Des oiseaux, selon leur espèce, des bêtes à quatre pattes, selon leur espèce, et de tous les reptiles, selon leur espèce. Ils y entreront tous par paires avec toi, afin que tu les conserves en vie.
\VS{21}Prends aussi avec toi de tous les aliments que l’on mange, et rassemble-les auprès de toi, afin qu'ils servent pour ta nourriture et pour celle des animaux.
\VS{22}Et Noé fit selon tout ce que Dieu lui avait ordonné ; il le fit ainsi.
\TextTitle{[Le jugement par le déluge]}
\Chap{7}
\VerseOne{}Yahweh dit à Noé : Entre dans l’arche, toi et toute ta maison ; car je t'ai vu juste devant moi parmi cette génération. 
\VS{2} Tu prendras de toutes les bêtes pures sept de chaque espèce, le mâle et sa femelle ; mais des bêtes qui ne sont point pures, un couple, le mâle et la femelle.
\VS{3}Tu prendras aussi des oiseaux du ciel sept de chaque espèce, le mâle et sa femelle ; afin d'en conserver la race sur toute la terre.
\VS{4}Car dans sept jours, je ferai pleuvoir sur la terre pendant quarante jours et quarante nuits ; et j'exterminerai de la surface de la terre tous les êtres qui subsistent que j'ai faits.
\VS{5}Noé fit selon tout ce que Yahweh lui avait ordonné.
\VS{6}Noé était âgé de six cents ans quand le déluge des eaux vint sur la terre.
\VS{7}Noé donc entra dans l’arche avec ses fils, sa femme, et les femmes de ses fils, pour échapper aux eaux du déluge.
\VS{8}Des bêtes pures, des bêtes qui ne sont point pures, des oiseaux, et tout ce qui se meut sur la terre.
\VS{9}Elles entrèrent deux à deux vers Noé dans l'arche, le mâle et la femelle, comme Dieu l’avait ordonné à Noé.
\VS{10}Sept jours après, les eaux du déluge furent sur la terre.
\VS{11}En l'an six cent de la vie de Noé, au second mois, le dix-septième jour du mois, en ce jour-là toutes les sources du grand abîme furent rompues, et les écluses des cieux furent ouvertes.
\VS{12}La pluie tomba sur la terre pendant quarante jours et quarante nuits.
\VS{13}Ce même jour entrèrent dans l’arche Noé, Sem, Cham, et Japhet, fils de Noé, avec la femme de Noé, et les trois femmes de ses fils avec eux.
\VS{14}Eux, et tous les animaux selon leur espèce, et tout le bétail selon son espèce, et tous les reptiles qui se meuvent sur la terre selon leur espèce, et tous les oiseaux selon leur espèce ; et tout petit oiseau ayant des ailes, de quelque sorte que ce soit.
\VS{15}Ils entrèrent dans l'arche auprès de Noé, deux à deux, de toute chair ayant souffle de vie.
\VS{16}Il en entra mâle et femelle de toute chair comme Dieu l’avait ordonné à Noé, puis Yahweh ferma l'arche sur lui.
\VS{17}Le déluge fut pendant quarante jours sur la terre ; et les eaux crurent et élevèrent l'arche, et elle fut élevée au-dessus de la terre.
\VS{18}Les eaux  grossirent et s'accrurent beaucoup sur la terre, et l'arche flottait au-dessus des eaux.
\VS{19}Les eaux grossirent de plus en plus sur la terre, et toutes les hautes montagnes qui sont sous le ciel entier en furent couvertes.
\VS{20}Les eaux s’élevèrent de quinze coudées au-dessus des montagnes  qui furent couvertes.
\VS{21}Toute chair qui se mouvait sur la terre périt, tant les oiseaux que le bétail et les animaux, tous les reptiles qui rampaient sur la terre, et tous les hommes.
\VS{22}Tout ce qui avait respiration, souffle de vie dans ses narines, et qui était sur la terre sèche mourut.
\VS{23}Tous les êtres qui étaient sur la face de la terre furent donc exterminés, depuis les hommes jusqu’au bétail, aux reptiles et aux oiseaux du ciel ; ils furent exterminés de la face de la terre ; il ne resta seulement que Noé, et ce qui était avec lui dans l'arche.
\VS{24}Les eaux furent grosses sur la terre pendant cent cinquante jours.
\TextTitle{[Fin du déluge]}
\Chap{8}
\VerseOne{}Dieu se souvint de Noé, de tous les animaux et de tout le bétail qui étaient avec lui dans l'arche ; et Dieu fit passer un vent sur la terre, et les eaux s’apaisèrent.
\VS{2}Les sources de l'abîme et les écluses des cieux furent fermées et la pluie ne tomba plus du ciel.
\VS{3}Au bout de cent cinquante jours, les eaux se retirèrent sans interruption de dessus la terre, et diminuèrent.
\VS{4}Le dix-septième jour du septième mois, l'arche s'arrêta sur les montagnes d'Ararat.
\VS{5}Les eaux allèrent en diminuant de plus en plus jusqu'au dixième mois ; et au premier jour du dixième mois, les sommets des montagnes apparurent.
\VS{6}Au bout de quarante jours, Noé ouvrit la fenêtre qu’il avait faite à l'arche.
\VS{7}Il lâcha le corbeau, qui sortit, allant et revenant, jusqu'à ce que les eaux aient séché sur la terre.
\VS{8}Il lâcha aussi une colombe pour voir si les eaux avaient diminué à la surface de la terre.
\VS{9}Mais la colombe ne trouvant aucun lieu pour poser la plante de son pied, retourna à lui dans l'arche, car les eaux étaient sur toute la terre ; et Noé avançant sa main la reprit et la fit entrer dans l'arche.
\VS{10}Il attendit encore sept autres jours, il lâcha de nouveau la colombe hors de l'arche.
\VS{11}Sur le soir, la colombe revint à lui ; et voici, elle avait dans son bec une feuille d'olivier qu'elle avait arrachée ; et Noé connut que les eaux avaient diminué sur la terre.
\VS{12}Il attendit encore sept autres jours, puis il lâcha la colombe qui ne retourna plus à lui.
\VS{13}L’an six cent un de l'âge de Noé, le premier jour du premier mois, les eaux avaient diminué sur la terre. Noé ôta la couverture de l'arche, regarda, et voici, la surface de la terre avait séché.
\VS{14}Le vingt-septième jour du second mois la terre fut sèche.
\TextTitle{[Noé sort de l'arche: Le règne des hommes]
\\(Ge. 8:15-11 ; 32)}
\VS{15}Puis Dieu parla à Noé, en disant :
\VS{16}Sors de l'arche, toi et ta femme, tes fils, et les femmes de tes fils avec toi.
\VS{17}Fais sortir avec toi tous les animaux qui sont avec toi, de toute chair, tant les oiseaux que le bétail, et tous les reptiles qui rampent sur la terre ; qu'ils se répandent sur la terre, et qu'ils soient féconds et multiplient sur la terre.
\VS{18}Noé donc sortit, et avec lui ses fils, sa femme, et les femmes de ses fils.
\VS{19}Tous les animaux, tous les reptiles, tous les oiseaux, tout ce qui se meut sur la terre, selon leurs espèces, sortirent de l'arche.
\VS{20}Noé bâtit un autel à Yahweh, il prit de toutes les bêtes pures, et de tout oiseau pur, et il offrit des holocaustes sur l'autel.
\VS{21}Yahweh respira une odeur agréable, et dit en son cœur : Je ne maudirai plus la terre à cause des hommes, quoique les dispositions du coeur des hommes soient mauvaises dès leur jeunesse ; et je ne frapperai plus tout ce qui est vivant, comme je l’ai fait.
\VS{22}Tant que la terre subsistera, les semailles et les moissons, le froid et la chaleur, l'été et l'hiver, le jour et la nuit ne cesseront point.
\TextTitle{[Yahweh établit une alliance avec Noé]
\\(Ge. 9:16)}
\Chap{9}
\VerseOne{}Dieu bénit Noé et ses fils, et leur dit : Soyez féconds, multipliez, et remplissez la terre.
\VS{2}Vous serez un sujet de crainte et d’effroi pour tout animal de la terre, pour tout oiseau du ciel, pour tout ce qui se meut sur la terre, et pour tous les poissons de la mer : Ils sont livrés entre vos mains.
\VS{3}Tout ce qui se meut et qui a vie sera votre nourriture ; je vous donne tout cela comme l'herbe verte.
\VS{4}Seulement, vous ne mangerez point de chair avec son âme, c'est-à-dire, son sang.
\VS{5}Sachez-le aussi, je redemanderai votre sang, le sang de vos âmes, je le redemanderai à tout animal ; et je redemanderai l’âme de l’homme de la main de l’homme, de la main de son frère.
\VS{6}Celui qui aura versé le sang de l'homme, par l'homme son sang sera versé ; car Dieu a fait l'homme à son image.
\VS{7}Vous donc, soyez féconds et multipliez, répandez-vous sur la terre et multipliez sur elle.
\VS{8}Dieu parla aussi à Noé et à ses fils qui étaient avec lui, en disant :
\VS{9}Et quant à moi, voici, j'établis mon alliance avec vous, et avec votre postérité après vous ;
\VS{10}avec tous les êtres vivants qui sont avec vous, tant les oiseaux que le bétail, et tous les animaux de la terre qui sont avec vous, tous ceux qui sont sortis de l'arche jusqu'à tous les animaux de la terre.
\VS{11}J'établis donc mon alliance avec vous ; aucune chair ne sera plus exterminée par les eaux du déluge, et il n'y aura plus de déluge pour détruire la terre.
\VS{12}Puis Dieu dit : C'est ici le signe de l'alliance que j’établis entre moi et vous, et tous les êtres vivants qui sont avec vous, pour les générations à toujours :
\VS{13}J’ai placé mon arc dans la nuée, et il servira de signe de l'alliance entre moi et la terre.
\VS{14}Quand j’aurai rassemblé des nuages au-dessus de la terre, l’arc paraîtra dans la nuée ;
\VS{15}et je me souviendrai de mon alliance entre moi et vous, et tous les êtres vivants de toute chair, et les eaux ne deviendront plus un déluge pour détruire toute chair.
\VS{16}L'arc donc sera dans la nuée, et je le regarderai, et je me souviendrai de l'alliance perpétuelle entre Dieu et tous les êtres vivants  de toute chair qui est sur la terre.
\VS{17}Dieu donc dit à Noé : C'est là le signe de l'alliance que j'ai établie entre moi et toute chair qui est sur la terre.
\VS{18}Les fils de Noé qui sortirent de l'arche étaient Sem, Cham, et Japhet. Cham fut père de Canaan.
\VS{19}Ce sont là les trois fils de Noé, et c’est leur postérité qui peupla toute la terre.
\TextTitle{[Le péché de Noé]}
\VS{20}Or, Noé commença à cultiver la terre, et planta de la vigne.
\VS{21}Il but du vin, s'enivra, et se découvrit au milieu de sa tente.
\VS{22}Cham, père de Canaan, vit la nudité de son père\FTNT{Lé. 18:6-19 ; Lé. 20:11-21.}, et il le rapporta dehors à ses deux frères.
\VS{23}Alors Sem et Japhet prirent un manteau qu'ils mirent sur leurs deux épaules, et marchant à reculons, ils couvrirent la nudité de leur père ; et leurs visages étaient tournés en arrière, de sorte qu'ils ne virent point la nudité de leur père.
\VS{24}Et quand Noé se réveilla de son vin, il apprit ce que lui avait fait son fils cadet.
\TextTitle{[Noé prononce une malédiction contre Canaan]}
\VS{25}C'est pourquoi il dit : Maudit soit Canaan\FTNT{Une idée erronée selon laquelle les noirs auraient été maudits par Dieu au travers de la malédiction de Canaan s’est répandue pendant des siècles. On a ainsi légitimé la domination des peuples africains par les puissances occidentales blanches, et par la même occasion l’esclavage. Il faut préciser que les descendants de Cham furent Cush (Ethiopie), Mitsraïm (Egypte), Puth (les Celtes) et Canaan (Palestine, pays que Dieu a donné aux descendants de Sem, selon Ge. 15). Cham est le fils cadet, c’est-à-dire le coupable aux yeux de Noé. Mais c’est à Canaan (Palestine), le fils de Cham, donc petit-fils de Noé, que s’adresse la malédiction. Selon la Bible, les peuples africains sont des descendants de Cham, mais par son fils Cush et non par Canaan. La prétendue malédiction des noirs n’a donc aucun fondement.} ; il sera serviteur des serviteurs de ses frères.
\VS{26}Il dit aussi : Béni soit Yahweh, Dieu de Sem ; et que Canaan soit leur serviteur.
\VS{27}Que Dieu étende en douceur Japhet, et qu’il habite dans les tentes de Sem ; et que Canaan soit leur serviteur.
\VS{28}Noé vécut après le déluge trois cent cinquante ans.
\VS{29}Tout le temps donc que Noé vécut fut de neuf cent cinquante ans ; puis il mourut.
\TextTitle{[La postérité de Noé]
\\(Ge. 9:28-10:32)}
\Chap{10}
\VerseOne{}Voici la postérité des enfants de Noé, Sem, Cham et Japhet ; il leur naquit des fils après le déluge.
\VS{2}Les fils de Japhet furent : Gomer, Magog, Madaï, Javan, Tubal, Mésech, et Tiras.
\VS{3}Les fils de Gomer : Aschkenaz, Riphat, et Togarma.
\VS{4}Les fils de Javan : Elischa, Tarsis, Kittim, et Dodanim.
\VS{5}C’est par eux qu’ont été peuplées les îles des nations selon leurs terres, chacun selon sa langue, selon leurs familles, entre leurs nations.
\VS{6}Les fils de Cham furent : Cusch, Mitsraïm, Puth, et Canaan.
\VS{7}Les fils de Cusch : Saba, Havila, Sabta, Raema, et Sabteca. Les fils de Raema : Séba et Dedan.
\VS{8}Cusch engendra aussi Nimrod\FTNT{Nimrod ou Nemrod, dont le nom signifie « rebelle », fut le premier roi de l’histoire biblique. Fils de Cusch (Ethiopie), lui-même premier-né de Cham, fils de Noé (Ge. 10:8-10), il fut à la tête du premier empire après le déluge. Il se distingua en qualité de puissant chasseur  « devant Yahweh » ou « contre Yahweh ». Le contexte du chapitre 10 laisse entendre que Nimrod était un puissant chasseur qui provoquait Dieu. Fondateur de Ninive, il est surtout connu pour avoir été à l’origine du projet de la tour de Babel.}, c’est lui qui commença à être puissant sur la terre.
\VS{9}Il fut un puissant chasseur devant Yahweh, c'est pourquoi l'on a dit : Comme Nimrod, le puissant chasseur devant Yahweh.
\VS{10}Il régna d’abord sur Babel\FTNT{Le nom Babel signifie confusion par le mélange.}, Erec, Accad, et Calné au pays de Schinear.
\VS{11}De ce pays-là sortit Assur, et il bâtit Ninive et les rues de la ville, Rehoboth-Hir et Calach,
\VS{12}et Résen, entre Ninive et Calach, qui est une grande ville.
\VS{13}Mitsraïm engendra les Ludim, les Anamim, les Lehabim, les Naphtuhim,
\VS{14}les Patrusim, les Casluhim, d’où sont sortis les Philistins, et les Caphtorim.
\VS{15}Canaan engendra Sidon, son premier-né, et Heth ;
\VS{16}et les Jébusiens, les Amoréens, les Guirgasiens,
\VS{17}les Héviens, les Arkiens, les Siniens,
\VS{18}les Arvadiens, les Tsemariens, les Hamathiens. Ensuite, les familles des Cananéens se sont dispersées.
\VS{19}Les limites des Cananéens furent depuis Sidon, quand on vient vers Guérar, jusqu'à Gaza, en allant vers Sodome et Gomorrhe, Adma, et Tseboïm, jusqu'à Léscha.
\VS{20}Ce sont là les fils de Cham selon leurs familles et leurs langues, selon leurs pays, et selon leurs nations.
\VS{21}Il naquit aussi des fils à Sem, père de tous les fils d'Héber, et frère aîné de Japhet.
\VS{22}Les fils de Sem furent : Elam, Assur, Arpacschad, Lud et Aram.
\VS{23}Les fils d'Aram : Uts, Hul, Guéter et Masch.
\VS{24}Arpacschad engendra Schélach ; et Schélach engendra Héber.
\VS{25}Il naquit à Héber deux fils : Le nom de l'un était Péleg, parce que de son temps la terre fut partagée ; et le nom de son frère était Jokthan.
\VS{26}Jokthan engendra Almodad, Schéleph, Hatsarmaveth, Jérach,
\VS{27}Hadoram, Uzal, Dikla,
\VS{28}Obal, Abimaël, Séba,
\VS{29}Ophir, Havila, et Jobab. Tous ceux-là sont les enfants de Jokthan.
\VS{30}Ils habitèrent depuis Méscha, du côté de Sephar jusqu’à la montagne de l’orient.
\VS{31}Ce sont là les fils de Sem, selon leurs familles, selon leurs langues, selon leurs pays, et selon leurs nations.
\VS{32}Telles sont les familles des fils de Noé, selon leurs lignées, selon leurs nations. Et c’est d’eux que sont sorties les nations qui se sont répandues sur la terre après le déluge.
\TextTitle{[Un projet humain : la tour de Babel]}
\Chap{11}
\VerseOne{}Alors toute la terre avait un même langage et les mêmes paroles.
\VS{2}Mais il arriva qu'étant partis d'orient, ils trouvèrent une vallée au pays de Schinear où ils habitèrent.
\VS{3}Et ils se dirent l'un à l'autre : Allons ! Faisons des briques, et cuisons-les très bien au feu. Et la brique leur servit de pierre, et le bitume leur servit d’argile.
\VS{4}Puis ils dirent : Allons ! Bâtissons-nous une ville, et une tour dont le sommet soit jusqu’aux cieux ; et faisons-nous un nom, de peur que nous ne soyons dispersés sur toute la terre.
\VS{5}Alors Yahweh descendit pour voir la ville et la tour que les fils des hommes bâtissaient.
\VS{6}Et Yahweh dit : Voici, ce n'est qu'un seul et même peuple, ils ont un même langage, et ils commencent à travailler ; et maintenant rien ne les empêchera d'exécuter ce qu'ils ont projeté.
\TextTitle{[Yahweh confond le langage humain]}
\VS{7}Allons ! Descendons, et là confondons leur langage afin qu'ils n'entendent point le langage les uns des autres.
\VS{8}Ainsi Yahweh les dispersa de là par toute la terre, et ils cessèrent de bâtir la ville.
\VS{9}C'est pourquoi on l’appela du nom de Babel, car c’est là que Yahweh confondit le langage de toute la terre, et c’est de là que Yahweh les dispersa sur toute la terre.
\TextTitle{[La postérité de Sem à Abram]}
\VS{10}Voici la postérité de Sem : Sem, âgé de cent ans, engendra Arpacschad, deux ans après le déluge.
\VS{11}Sem, après qu'il eut engendré Arpacschad, vécut cinq cents ans, et engendra des fils et des filles.
\VS{12}Arpacschad vécut trente-cinq ans, et engendra Schélach.
\VS{13}Arpacschad, après qu'il eut engendré Schélach, vécut quatre cent trois ans, et engendra des fils et des filles.
\VS{14}Schélach, ayant vécu trente ans, engendra Héber.
\VS{15}Schélach, après qu'il eut engendré Héber, vécut quatre cent trois ans, et engendra des fils et des filles.
\VS{16}Héber, ayant vécu trente-quatre ans, engendra Péleg.
\VS{17}Héber, après qu'il eut engendré Péleg, vécut quatre cent trente ans, et engendra des fils et des filles.
\VS{18}Péleg, ayant vécu trente ans, engendra Rehu.
\VS{19}Péleg, après qu'il eut engendré Rehu, vécut deux cent neuf ans, et engendra des fils et des filles.
\VS{20}Rehu, ayant vécu trente-deux ans, engendra Serug.
\VS{21}Rehu, après qu'il eut engendré Serug, vécut deux cent sept ans, et engendra des fils et des filles.
\VS{22}Serug, ayant vécu trente ans, engendra Nachor.
\VS{23}Serug, après qu'il eut engendré Nachor, vécut deux cents ans, et engendra des fils et des filles.
\VS{24}Nachor, ayant vécu vingt-neuf ans, engendra Térach.
\VS{25}Nachor, après qu'il eut engendré Térach, vécut cent dix-neuf ans, et engendra des fils et des filles.
\VS{26}Térach, ayant vécu soixante-dix ans, engendra Abram, Nachor, et Haran.
\VS{27}Voici la postérité de Térach : Térach engendra Abram, Nachor, et Haran ; et Haran engendra Lot.
\VS{28}Et Haran mourut en présence de son père, au pays de sa naissance, à Ur en Chaldée.
\VS{29}Abram et Nachor prirent chacun une femme. Le nom de la femme d'Abram était Saraï ; et le nom de la femme de Nachor était Milca, fille de Haran, père de Milca et de Jisca.
\VS{30}Saraï était stérile, elle n'avait point d'enfants.
\TextTitle{[Séjour à Charan]}
\VS{31}Térach prit son fils Abram, et Lot fils de son fils, qui était fils de Haran, et Saraï, sa belle-fille, femme d'Abram, son fils, et ils sortirent ensemble d'Ur en Chaldée pour aller au pays de Canaan, et ils vinrent jusqu'à Charan, et ils y habitèrent.
\VS{32}Les jours de Térach furent de deux cent cinq ans ; puis il mourut à Charan.
\TextTitle{[Appel d'Abram : La promesse de Yahweh]
\\(Ge. 12:2 ; 13:14-18 ; 15:1-21 ; 17:4-8 ; 22:15-24 ; 26:1-5 ; 28 : 10-15)}
\Chap{12}
\VerseOne{}Yahweh dit à Abram : Va pour toi, hors de ta terre, de ta patrie, et de la maison de ton père, vers la terre que je te montrerai\FTNT{Ac. 7:3 ; Hé. 11:8.}.
\VS{2}Je te ferai devenir une grande nation, et je te bénirai, je rendrai ton nom grand, et tu seras béni.
\VS{3}Je bénirai ceux qui te béniront, et je maudirai ceux qui te maudiront ; et toutes les familles de la terre seront bénies en toi\FTNT{Ac. 3:25 ; Ga. 3:8.}.
\TextTitle{[Abram sur la terre de Canaan]}
\VS{4}Abram donc partit, comme Yahweh le lui avait dit, et Lot alla avec lui. Abram était âgé de soixante-quinze ans quand il sortit de Charan.
\VS{5}Abram prit aussi Saraï, sa femme, et Lot, fils de son frère, avec tous les biens qu'ils avaient acquis, et les personnes qu'ils avaient eues à Charan ; et ils partirent pour aller dans le pays de Canaan, et ils arrivèrent au pays de Canaan\FTNT{Ac. 7:4.}.
\VS{6}Abram parcourut le pays jusqu'au lieu nommé Sichem, et jusqu'aux chênes de Moré ; et les Cananéens étaient alors dans le pays.
\VS{7}Yahweh apparut à Abram, et lui dit : Je donnerai ce pays à ta postérité. Et Abram bâtit là un autel à Yahweh qui lui était apparu.
\VS{8}Il se transporta de là vers la montagne, à l'orient de Béthel, et il dressa ses tentes, ayant Béthel à l'occident, et Aï à l'orient ; et il bâtit là un autel à Yahweh, et invoqua le nom de Yahweh.
\VS{9}Puis Abram partit de là, marchant et s'avançant vers le midi.
\TextTitle{[Abram descend en Egypte à cause de la famine]}
\VS{10}Mais la famine étant survenue dans le pays, Abram descendit en Egypte pour s'y retirer, car la famine était grande dans le pays.
\VS{11}Comme il était près d'entrer en Egypte, il dit à Saraï, sa femme : Voici, je sais que tu es une fort belle femme ;
\VS{12}c'est pourquoi, quand les Egyptiens te verront, ils diront : C'est la femme de cet homme, et ils me tueront, mais ils te laisseront vivre.
\VS{13}Dis donc, je te prie, que tu es ma sœur, afin que je sois bien traité à cause de toi, et que par ton moyen, ma vie soit préservée.
\VS{14}Il arriva donc qu'aussitôt qu'Abram fut arrivé en Egypte, les Egyptiens virent que cette femme était fort belle.
\VS{15}Les principaux de la cour de Pharaon la virent aussi et la vantèrent à Pharaon, et elle fut enlevée pour être menée dans la maison de Pharaon.
\VS{16}Il traita bien Abram à cause d'elle, de sorte qu'il en eut des brebis, des bœufs, des ânes, des serviteurs, des servantes, des ânesses, et des chameaux.
\VS{17}Mais Yahweh frappa de grandes plaies Pharaon et sa maison, à cause de Saraï, femme d'Abram.
\VS{18}Alors Pharaon appela Abram, et lui dit : Qu'est-ce que tu m'as fait ? Pourquoi ne m'as-tu pas déclaré que c'était ta femme ?
\VS{19}Pourquoi as-tu dit : C'est ma sœur ? Car je l'avais prise pour ma femme ; mais maintenant, voici ta femme, prends-la, et va-t'en.
\VS{20}Et Pharaon ayant donné ordre à ses gens, ils le renvoyèrent, lui, sa femme, et tout ce qui était à lui.
\TextTitle{[Retour d'Abram à Canaan]}
\Chap{13}
\VerseOne{}Abram donc monta d'Egypte vers le midi, lui, sa femme, et tout ce qui lui appartenait, et Lot avec lui.
\VS{2}Et Abram était très riche en bétail, en argent, et en or.
\VS{3}Et il s'en retourna en suivant la route qu'il avait suivie du midi à Béthel, jusqu'au lieu où il avait dressé ses tentes au commencement, entre Béthel et Aï,
\VS{4}au même lieu où était l'autel qu'il y avait bâti au commencement, et Abram invoqua là le nom de Yahweh.
\TextTitle{[Abram se sépare de Lot]
\\(Ge. 13:12; 19:1 , 33)}
\VS{5}Lot aussi, qui marchait avec Abram, avait des brebis, des boeufs, et des tentes.
\VS{6}Et le pays ne pouvait les porter pour demeurer ensemble ; car leurs biens étaient si grand qu'ils ne pouvaient demeurer ensemble.
\VS{7}Il y eut querelle entre les bergers du bétail d'Abram et les bergers du bétail de Lot ; or en ce temps-là, les Cananéens et les Phérésiens habitaient dans le pays.
\VS{8}Et Abram dit à Lot : Je te prie qu'il n'y ait point de dispute entre moi et toi, ni entre mes bergers et les tiens, car nous sommes frères.
\VS{9}Tout le pays n'est-il pas devant toi ? Sépare-toi je te prie d'avec moi. Si tu vas à gauche, j’irai à droite ; et si tu vas à droite, j’irai à gauche.
\VS{10}Lot, levant les yeux, vit que toute la plaine du Jourdain était entièrement arrosée. Avant que Yahweh ait détruit Sodome et Gomorrhe, c’était, jusqu'à Tsoar, comme le jardin de Yahweh, et comme le pays d'Egypte.
\VS{11}Lot choisit pour lui toute la plaine du Jourdain, et alla du côté de l’orient ; ainsi ils se séparèrent l'un de l'autre.
\TextTitle{[Lot s'établit à Sodome]
\\(Ge. 13:10; 19:1 , 33)}
\VS{12}Abram habita dans le pays de Canaan, et Lot habita dans les villes de la plaine, et dressa ses tentes jusqu'à Sodome.
\VS{13}Les habitants de Sodome étaient méchants et de grands pécheurs contre Yahweh.
\TextTitle{[Yahweh confirme l'alliance avec Abram]}
\VS{14}Yahweh dit à Abram, après que Lot se fut séparé de lui : Lève maintenant tes yeux, et regarde du lieu où tu es vers le nord, le midi, l'orient, et l'occident.
\VS{15}Car je te donnerai, à toi et à ta postérité pour toujours, tout le pays que tu vois.
\VS{16}Je rendrai ta postérité comme la poussière de la terre ; en sorte que si quelqu'un peut compter la poussière de la terre, il comptera aussi ta postérité\FTNT{Ro. 4:18 ; Hé. 11:12.}.
\VS{17}Lève-toi donc et promène-toi dans le pays, dans sa longueur et dans sa largeur, car je te le donnerai.
\VS{18}Abram ayant transporté ses tentes, alla habiter dans les plaines de Mamré, qui sont près d’Hébron et là, il bâtit un autel à Yahweh.
\TextTitle{[Abram va au secours de Lot]}
\Chap{14}
\VerseOne{}Dans le temps d'Amraphel, roi de Schinear, d'Arjoc, roi d'Ellasar, de Kedorlaomer, roi d'Elam, et de Tideal, roi de Gojim,
\VS{2}il arriva qu’ils firent la guerre contre Béra, roi de Sodome, et contre Birscha, roi de Gomorrhe, et contre Schineab, roi d'Adma, et contre Schémeéber, roi de Tseboïm, et contre le roi de Béla, qui est Tsoar.
\VS{3}Tous ceux-ci se joignirent dans la vallée de Siddim, qui est la mer salée.
\VS{4}Ils avaient été asservis douze années à Kedorlaomer, et la treizième année, ils s'étaient révoltés.
\VS{5}A la quatorzième année, Kedorlaomer et les rois qui étaient avec lui vinrent et ils battirent les Rephaïm à Aschteroth-Karnaïm, les Zuzim à Ham, et les Emin à la plaine de Schavé-Kirjathaïm,
\VS{6}et les Horiens dans leur montagne de Séir, jusqu'au chêne de Paran, qui est près du désert.
\VS{7}Puis ils s’en retournèrent et vinrent à En-Mischpath, qui est Kadès ; et ils frappèrent tout le pays des Amalécites et des Amoréens qui habitaient dans Hatsatson-Thamar.
\VS{8}Alors le roi de Sodome, le roi de Gomorrhe, le roi d'Adma, le roi de Tseboïm, et le roi de Béla qui est Tsoar, sortirent et rangèrent leurs troupes contre eux dans la vallée de Siddim.
\VS{9}C'est-à-dire contre Kedorlaomer, roi d'Elam, et contre Tideal, roi de Gojim, et contre Amraphel, roi de Schinear, et contre Arjoc, roi d'Ellasar : Quatre rois contre cinq.
\VS{10}La vallée de Siddim était pleine de puits de bitume ; les rois de Sodome et de Gomorrhe s'enfuirent et y tombèrent, et le reste s'enfuit dans la montagne.
\VS{11}Ils prirent donc toutes les richesses de Sodome et de Gomorrhe, et tous leurs vivres ; puis ils se retirèrent.
\VS{12}Ils prirent aussi Lot, fils du frère d'Abram, qui habitait dans Sodome, et tous ses biens ; puis ils s'en allèrent.
\VS{13}Un fuyard vint avertir Abram, l’Hébreu, qui demeurait dans les plaines de Mamré, l’Amoréen, frère d'Eschcol, et frère d’Aner, qui avaient fait alliance avec Abram.
\VS{14}Dès qu’Abram eut appris que son frère avait été emmené prisonnier, il arma trois cent dix-huit de ses plus braves serviteurs, nés dans sa maison, et il poursuivit ces rois jusqu'à Dan.
\VS{15}Il divisa sa troupe, il se jeta sur eux de nuit, lui et ses serviteurs ; il les battit et les poursuivit jusqu'à Choba, qui est à la gauche de Damas.
\VS{16}Il ramena tous les biens qu'ils avaient pris ; il ramena aussi Lot, son frère, ses biens, les femmes et le peuple.
\TextTitle{[Melchisédek, sacrificateur d'El Elyon (Dieu Très-Haut)]}
\VS{17}Le roi de Sodome sortit à la rencontre d’Abram qui revenait vainqueur de Kedorlaomer, et des rois qui étaient avec lui, dans la vallée de la plaine, qui est la vallée royale.
\VS{18}Melchisédek\FTNT{Melchisédek est un type de Christ (Ps. 110:4 ; Hé. 5:5-6 ; Hé. 6:20 ; Hé. 7:1-2). Ce personnage nous montre l’aspect de Christ en tant que roi de Salem, ce qui signifie « paix », et Souverain Sacrificateur possédant un sacerdoce non transmissible (Hé. 7:24).}, roi de Salem, fit apporter du pain et du vin, or il était Sacrificateur du Dieu Très-Haut.
\VS{19}Il bénit Abram en disant : Béni soit Abram par le Dieu Très-Haut, Maître du ciel et de la terre.
\VS{20}Béni soit le Dieu Très-Haut qui a livré tes ennemis entre tes mains. Et Abram lui donna la dîme\FTNT{Voir commentaire sur la dîme en No. 18:21 et Mal. 3:10.} de tout.
\VS{21}Le roi de Sodome dit à Abram : Donne-moi les personnes, et prends pour toi les richesses.
\VS{22}Abram répondit au roi de Sodome : Je lève ma main vers Yahweh, le Dieu Très-Haut, Maître du ciel et de la terre :
\VS{23}Je ne prendrai rien de tout ce qui est à toi, pas même un fil, ni un cordon de soulier, afin que tu ne dises point : J'ai enrichi Abram.
\VS{24}Seulement, ce que les jeunes gens ont mangé, et la part des hommes qui sont venus avec moi, Aner, Eschcol, et Mamré, qui prendront leur part.
\TextTitle{[Yahweh confirme l'alliance avec Abram]}
\Chap{15}
\VerseOne{}Après ces choses, la parole de Yahweh fut adressée à Abram dans une vision, en disant : Abram, ne crains point, je suis ton bouclier\FTNT{Voir commentaire sur la dîme en No. 18:21 et Mal. 3:10.}, et ta récompense sera très grande.
\VS{2}Abram répondit : Seigneur Yahweh, que me donneras-tu ? Je m'en vais sans laisser d'enfants après moi, et l’héritier de ma maison c'est Eliézer de Damas.
\VS{3}Abram dit aussi : Voici, tu ne m'as point donné d'enfants ; et voilà, le serviteur né dans ma maison sera mon héritier.
\VS{4}Alors la parole de Yahweh lui fut adressée ainsi : Ce n’est pas lui qui sera ton héritier, mais c’est celui qui sortira de tes entrailles qui sera ton héritier.
\VS{5}Puis l'ayant fait sortir dehors, il lui dit : Lève maintenant les yeux au ciel et compte les étoiles si tu peux les compter. Et il lui dit : Ainsi sera ta postérité.
\VS{6}Abram crut à Yahweh qui lui imputa cela à justice\FTNT{Ga. 3:6 ; Ja. 2:23 ; Ro. 4:3.}.
\TextTitle{[Yahweh annonce l'esclavage de la postérté d'Abram]}
\VS{7}Et il lui dit : Je suis Yahweh qui t'ai fait sortir d'Ur en Chaldée, afin de te donner ce pays-ci pour le posséder.
\VS{8}Abram répondit : Seigneur Yahweh, à quoi connaîtrai-je que je le posséderai ?
\VS{9}Et Yahweh lui répondit : Prends une génisse de trois ans,  une chèvre de trois ans, un bélier de trois ans, une tourterelle, et un pigeon.
\VS{10}Abram prit tous ces animaux, les coupa par le milieu, et mit chaque morceau l’un vis-à-vis de l’autre, mais il ne partagea point les oiseaux.
\VS{11}Les oiseaux de proie descendirent sur les cadavres, mais Abram les chassa.
\VS{12}Au coucher du soleil, un profond sommeil tomba sur Abram, et voici, une frayeur d'une grande obscurité tomba sur lui.
\VS{13}Et Yahweh dit à Abram : Sache comme une chose certaine que tes descendants habiteront quatre cents ans comme étrangers dans un pays qui ne leur appartiendra point, et qu’ils seront asservis aux habitants du pays qui les opprimera\FTNT{Ex. 12:10 ; Ac. 7:6 ; Ga. 3:17.}.
\VS{14}Mais je jugerai la nation à laquelle ils seront asservis, et après cela ils sortiront avec de grands biens\FTNT{Ex. 3:22.}.
\VS{15}Et toi tu iras vers tes pères en paix, et tu seras enterré après une heureuse vieillesse.
\VS{16}A la quatrième génération, ils reviendront ici ; car l'iniquité des Amoréens n'est pas encore à son comble.
\VS{17}Quand le soleil fut couché, il y eut une obscurité profonde, et voici, ce fut une fournaise fumante, et des flammes passèrent entre les animaux qui avaient été partagés.
\VS{18}En ce jour-là, Yahweh traita alliance avec Abram, en disant : Je donne ce pays à ta postérité, depuis le fleuve d'Egypte jusqu'au grand fleuve, le fleuve d'Euphrate ;
\VS{19}le pays des Kéniens, des Keniziens, des Kadmoniens,
\VS{20}des Héthiens, des Phéréziens, des Rephaïm,
\VS{21}des Amoréens, des Cananéens, des Guirgasiens, et des Jébusiens.
\TextTitle{[Saraï pousse Abram dans les bras de sa servante]}
\Chap{16}
\VerseOne{}Saraï, femme d'Abram, ne lui avait enfanté aucun enfant, mais elle avait une servante égyptienne nommée Agar.
\VS{2}Et Saraï dit à Abram : Voici, Yahweh m'a rendue stérile ; viens je te prie vers ma servante, peut-être aurai-je des enfants par elle. Et Abram écouta la voix de Saraï.
\VS{3}Alors Saraï, femme d'Abram, prit Agar, sa servante égyptienne, et la donna pour femme à Abram, son mari, après qu’Abram eut habité dix ans dans le pays de Canaan.
\VS{4}Il alla donc vers Agar, et elle conçut. Quand Agar se vit enceinte, elle regarda sa maîtresse avec mépris.
\VS{5}Et Saraï dit à Abram : L'outrage qui m'est fait retombe sur toi. J’ai mis ma servante dans ton sein, mais quand elle a vu qu'elle avait conçu, elle m'a regardée avec mépris. Que Yahweh soit juge entre moi et toi !
\VS{6}Alors Abram répondit à Saraï : Voici, ta servante est entre tes mains, traite-la comme il te plaira. Saraï donc la maltraita, et Agar s'enfuit de devant elle.
\VS{7}Mais l'Ange de Yahweh\FTNT{L’Ange de Yahweh n’est autre que Yahweh lui-même, c’est-à-dire une manifestation de Jésus-Christ. Agar a réalisé que c’était bel et bien Dieu qui s’était révélé à elle (v. 13). Gédéon et Manoach ont  identifié de façon précise l'Ange de Yahweh comme étant Yahweh lui-même (Jg. 13:21-22 ; Jg. 6:11-14). Nous pouvons encore faire remarquer qu’en Ge. 22:1-2, c’est Dieu qui demanda à Abraham de lui sacrifier Isaac, et qu’en Ge. 22:11-12 et 15-16, c’est l’Ange de Yahweh qui lui dit:« Ne porte pas ta main sur l'enfant, et ne lui fais rien ; car maintenant je sais que tu crains Dieu, puisque tu ne m’as point refusé ton fils, ton unique. ». Ce dernier jura d’ailleurs par lui-même, lorsqu’il confirma son alliance avec Abram, puisqu’il n’existe personne au-dessus de lui (Ge. 22:15-16 ; Hé. 6:13-14). Ce même ange s’est présenté à Jacob comme étant le Dieu de Béthel (Ge. 31:11-17). Les écritures disent que le nom de Dieu est en lui (Ex. 23:21) et que sa présence équivaut à la présence divine (Ex. 32:34 ; Ex. 33:14; Es. 63:9). Jésus-Christ est appelé l’Ange de l’alliance dans Mal. 3:1.} la trouva auprès d'une fontaine d'eau dans le désert, près de la fontaine qui est sur le chemin de Schur.
\VS{8}Il lui dit : Agar, servante de Saraï, d'où viens-tu ? Et où vas-tu ? Et elle répondit : Je m'enfuis de devant Saraï, ma maîtresse.
\VS{9}L'Ange de Yahweh lui dit : Retourne vers ta maîtresse et humilie-toi sous sa main.
\VS{10}L'Ange de Yahweh lui dit : Je multiplierai beaucoup ta postérité, elle sera si nombreuse qu'on ne pourra la compter.
\VS{11}L'Ange de Yahweh lui dit aussi : Voici, tu as conçu, et tu enfanteras un fils que tu appelleras Ismaël, car Yahweh a entendu ton affliction.
\VS{12}Et ce sera un homme farouche comme un âne sauvage ; sa main sera contre tous, et la main de tous contre lui ; et il habitera en face de tous ses frères.
\VS{13}Alors elle appela Atta-El-roï (tu es le Dieu qui me voit) le nom de Yahweh qui lui avait parlé ; car elle dit : N'ai-je pas même, ici, vu celui qui me voyait ?
\VS{14}C'est pourquoi on a appelé ce puits le puits du vivant qui me voit ; lequel est entre Kadès et Bared.
\TextTitle{[Naissance d'Ismaël]}
\VS{15}Agar donc enfanta un fils à Abram ; et Abram donna le nom d’Ismaël  au fils qu'Agar lui avait enfanté\FTNT{Ga. 4:22.}.
\VS{16}Abram était âgé de quatre-vingt-six ans quand Agar enfanta Ismaël à Abram.
\TextTitle{[El Schaddaï (Dieu Tout-Puissant) confirme sa promesse]}
\Chap{17}
\VerseOne{}Lorsqu’Abram fut âgé de quatre-vingt-dix-neuf ans, Yahweh lui apparut et lui dit : Je suis le Dieu Tout-Puissant\FTNT{Dieu se révèle ici à Abraham comme le Dieu Tout-Puissant. Or Christ s’est présenté à l’apôtre Jean comme le Dieu Tout Puissant (Ap. 1:8).  Plus loin en Ap. 5:6, le Seigneur apparaît au milieu du trône céleste sous la forme d’un Agneau ayant sept cornes qui représentent sa toute-puissance. Jésus est bien le Dieu Tout-Puissant qui s’était révélé à Abraham (Da. 8:20-22).}. Marche devant ma face, et sois intègre.
\VS{2}J’établirai mon alliance entre moi et toi, et je te multiplierai très abondamment.
\VS{3}Alors Abram tomba sur sa face, et Dieu lui parla et lui dit :
\TextTitle{[Abram devient Abraham]}
\VS{4}Quant à moi, voici, mon alliance est avec toi, et tu deviendras père d'une multitude de nations\FTNT{Ro. 4:17.}.
\VS{5}On ne t’appellera plus Abram, mais ton nom sera Abraham ; car je t'ai établi père d'une multitude de nations\FTNT{Né. 9:7.}.
\TextTitle{[L'alliance éternelle]}
\VS{6}Je te rendrai fécond  à l’extrême, et je te ferai devenir des nations ; même des rois sortiront de toi\FTNT{Mt. 1:6.}.
\VS{7}J'établirai donc mon alliance entre moi et toi, et entre ta postérité après toi, selon leurs générations, ce sera une alliance éternelle en vertu de laquelle je serai ton Dieu et celui de ta postérité après toi.
\VS{8}Je te donnerai, et à ta postérité après toi, le pays où tu demeures comme étranger, à savoir tout le pays de Canaan, en possession perpétuelle, et je leur serai Dieu.
\TextTitle{[La circoncision, signe de l'alliance]}
\VS{9}Dieu dit encore à Abraham : Tu garderas donc mon alliance, toi et ta postérité après toi, selon leurs générations.
\VS{10}C’est ici mon alliance entre moi et vous, et entre ta postérité après toi, que vous garderez : Tout mâle parmi vous sera circoncis.
\VS{11}Vous circoncirez la chair de votre prépuce ; et cela sera le signe de l'alliance entre moi et vous\FTNT{Ac. 7:8 ; Ro. 4:11.}.
\VS{12}Tout enfant mâle de huit jours sera circoncis parmi vous dans vos générations, tant celui qui est né dans la maison que l'esclave acquis à prix d’argent de tout étranger qui n'est point de ta race\FTNT{Lu. 2:21 ; Lé. 12:3.}.
\VS{13}On ne manquera donc point de circoncire celui qui est né dans ta maison, et celui qui est acquis à prix d’argent, et mon alliance sera dans votre chair pour être une alliance perpétuelle.
\VS{14}Et le mâle incirconcis qui n’aura pas été circoncis dans sa  chair sera retranché du milieu de son peuple parce qu'il aura violé mon alliance.
\TextTitle{[Saraï devient Sara, promesse d'Isaac]}
\VS{15}Dieu dit aussi à Abraham : Quant à Saraï, ta femme, tu n'appelleras plus son nom Saraï, mais son nom sera Sara.
\VS{16}Je la bénirai, et même je te donnerai un fils d'elle. Je la bénirai et elle deviendra des nations ; des rois, chefs de peuples sortiront d'elle.
\VS{17}Alors Abraham se prosterna la face contre terre, et sourit en disant en son cœur : Naîtrait-il un fils à un homme âgé de cent ans ? Et Sara, âgée de quatre-vingt-dix ans, aurait-elle un enfant ?
\VS{18}Et Abraham dit à Dieu : Je te prie, qu'Ismaël vive devant toi.
\VS{19}Et Dieu dit : Certainement Sara, ta femme, t'enfantera un fils, et tu appelleras son nom Isaac ; et j'établirai mon alliance avec lui pour être une alliance perpétuelle pour sa postérité après lui.
\TextTitle{[Une nation sortira d'Ismaël]}
\VS{20}Je t'ai aussi exaucé touchant Ismaël : Voici, je le bénirai, et je le ferai croître et multiplier très abondamment. Il engendrera douze princes, et je le ferai devenir une grande nation.
\VS{21}Mais j'établirai mon alliance avec Isaac, que Sara t'enfantera l'année qui vient, en cette même saison.
\VS{22}Et Dieu ayant achevé de parler, s’éleva au-dessus d'Abraham.
\VS{23}Et Abraham prit son fils Ismaël, avec tous ceux qui étaient nés dans sa maison, et tous ceux qu'il avait acquis à prix d’argent, tous les mâles qui étaient des gens de sa maison, et il circoncit la chair de leur prépuce en ce même jour-là, comme Dieu le lui avait dit.
\VS{24}Abraham était âgé de quatre-vingt-dix-neuf ans quand il circoncit la chair de son prépuce ;
\VS{25}et Ismaël, son fils, était âgé de treize ans lorsqu'il fut circoncis.
\VS{26}En ce même jour, Abraham fut circoncis, et son fils Ismaël aussi.
\VS{27}Et tous les gens de sa maison, tant ceux qui étaient nés dans sa maison que ceux qui avaient été acquis à prix d’argent des étrangers, furent circoncis avec lui.
\TextTitle{[Abraham, ami de Yahweh]
\\(Jn 3:29 ; 15:13-35)}
\Chap{18}
\VerseOne{}Puis Yahweh lui apparut dans les plaines de Mamré, comme il était assis à la porte de sa tente, pendant la chaleur du jour.
\VS{2}Levant ses yeux, il regarda : Et voici, trois hommes parurent devant lui. Quand il les vit, il courut au-devant d'eux depuis la porte de sa tente, et se prosterna à terre\FTNT{Hé. 13:2.} ;
\VS{3}Et il dit : Mon Seigneur, je te prie, si j'ai trouvé grâce devant tes yeux, ne passe point outre, je te prie, et arrête-toi chez ton serviteur.
\VS{4}Qu'on prenne, je vous prie, un peu d'eau, et lavez vos pieds, et reposez-vous sous un arbre.
\VS{5}J’apporterai un morceau de pain pour fortifier votre cœur, après quoi vous passerez outre ; car c'est pour cela que vous êtes venus vers votre serviteur. Et ils dirent : Fais ce que tu as dit.
\VS{6}Abraham donc s'en alla en hâte dans la tente vers Sara, et lui dit : Hâte-toi, prends trois mesures de fleur de farine, pétris-les, et fais des gâteaux.
\VS{7}Puis Abraham courut au troupeau et prit un veau tendre et bon, et le donna à un serviteur qui se hâta de l'apprêter.
\VS{8}Ensuite, il prit du beurre et du lait, et le veau qu'on avait apprêté, et le mit devant eux ; et il se tint auprès d'eux sous l'arbre, et ils mangèrent.
\VS{9}Et ils lui dirent : Où est Sara ta femme ? Et il répondit : La voilà dans la tente.
\VS{10}Et l'un d'entre eux dit : Je ne manquerai pas de revenir vers toi en ce même temps où nous sommes, et voici, Sara, ta femme, aura un fils. Et Sara écoutait à la porte de la tente qui était derrière lui\FTNT{Ro. 9:9.}.
\VS{11}Or Abraham et Sara étaient vieux, fort avancés en âge ; et Sara n'avait plus ce que les femmes sont accoutumées d'avoir\FTNT{Ro. 4:19 ; Hé. 11:11.}.
\VS{12}Et Sara rit en elle-même et dit : Etant vieille, et mon Seigneur étant fort âgé, aurai-je encore des désirs ?
\VS{13}Et Yahweh dit à Abraham : Pourquoi Sara a-t-elle ri en disant : Serait-il vrai que j'aurais un enfant, étant vieille comme je suis ?
\VS{14}Y a-t-il quelque chose qui soit difficile à Yahweh ? Je reviendrai vers toi à cette époque, en ce même temps où nous sommes et Sara aura un fils\FTNT{Mt. 19:26 ; Lu. 1:37.}.
\VS{15}Et Sara le nia en disant : Je n'ai point ri ; car elle avait peur. Mais il dit : Cela n'est pas, car tu as ri.
\VS{16}Et ces hommes se levèrent de là, et regardèrent vers Sodome ; et Abraham alla  avec eux pour les accompagner.
\VS{17}Et Yahweh dit : Cacherai-je à Abraham ce que je vais faire ?
\VS{18}Abraham deviendra certainement une nation grande et puissante, et toutes les nations de la terre seront bénies en lui\FTNT{Ac. 3:25 ; Ga. 3:8.}.
\VS{19}Car je le connais, et je sais qu'il ordonnera à ses enfants, et à sa maison après lui, de garder la voie de Yahweh, pour faire ce qui est juste et droit ; afin que Yahweh fasse venir sur Abraham tout ce qu'il lui a dit.
\VS{20}Et Yahweh dit : Le cri contre Sodome et Gomorrhe s’est accru, et leur péché s’est fort aggravé.
\VS{21}Je descendrai maintenant, et je verrai s'ils ont fait entièrement selon le cri qui est venu jusqu'à moi ; et si cela n'est pas, je le saurai.
\VS{22}Ces hommes donc partant de là allèrent vers Sodome ; mais Abraham se tint encore devant Yahweh.
\TextTitle{[Intercession d'Abraham]}
\VS{23}Et Abraham s'approcha et dit : Feras-tu périr le juste avec le méchant ?
\VS{24}Peut-être y a-t-il cinquante justes dans la ville, les feras-tu périr aussi ? Ne pardonneras-tu point à la ville à cause des cinquante justes qui sont au milieu d’elle ?
\VS{25}Non, il n'arrivera pas que tu fasses une telle chose, que tu fasses mourir le juste avec le méchant, et que le juste soit traité comme le méchant ! Non, tu ne le feras point. Celui qui juge toute la terre ne fera-t-il point justice\FTNT{Ro. 3:5-6.} ?
\VS{26}Et Yahweh dit : Si je trouve dans Sodome cinquante justes au milieu de la ville, je pardonnerai à toute la ville à cause d'eux.
\VS{27}Et Abraham répondit en disant : Voici, j'ai pris maintenant la hardiesse de parler au Seigneur, moi qui ne suis que poussière et cendres.
\VS{28}Peut-être en manquera-t-il cinq des cinquante justes ; détruiras-tu toute la ville pour ces cinq-là ? Et Yahweh lui répondit : Je ne la détruirai point si j'y trouve quarante-cinq justes.
\VS{29}Abraham continua de lui parler en disant : Peut-être s'y trouvera-t-il quarante ? Et il dit : Je ne la détruirai point pour l'amour des quarante.
\VS{30}Abraham dit : Je prie le Seigneur de ne pas s'irriter si je parle encore. Peut-être s'en trouvera-t-il trente ? Et il dit : Je ne la détruirai point si j'y trouve trente.
\VS{31}Abraham dit : Voici, maintenant j'ai pris la hardiesse de parler au Seigneur : Peut-être s'en trouvera-t-il vingt ? Et il dit : Je ne la détruirai point pour l'amour des vingt.
\VS{32}Abraham dit : Je prie le Seigneur de ne pas s'irriter, je parlerai encore une seule fois : Peut-être s'y trouvera-t-il dix. Et Yahweh dit : Je ne la détruirai point pour l'amour des dix.
\VS{33}Yahweh s'en alla quand il eut achevé de parler avec Abraham. Et Abraham retourna dans sa demeure.
\TextTitle{[Des anges chez Lot]
\\(Ge. 13:10, 12 ; 19:33)}
\Chap{19}
\VerseOne{}Sur le soir, les deux anges arrivèrent à Sodome, et Lot était assis à la porte de Sodome. Quand Lot les vit, il se leva pour aller au-devant d'eux, et se prosterna la face contre terre.
\VS{2}Et il leur dit : Voici, je vous prie, mes seigneurs, entrez maintenant dans la maison de votre serviteur, et passez-y la nuit ; lavez-vous les pieds ; puis vous vous lèverez dès le matin et continuerez votre chemin ; et ils dirent : Non, mais nous passerons la nuit dans la rue.
\VS{3}Mais il les pressa tellement qu'ils se retirèrent chez lui ; et quand ils furent entrés dans sa maison, il leur fit un festin, et fit cuire des pains sans levain, et ils mangèrent.
\VS{4}Ils n’étaient pas encore couchés que les hommes de la ville, les hommes de Sodome, environnèrent la maison, depuis les plus jeunes jusqu'aux vieillards, tout le peuple était ensemble.
\VS{5}Ils appelèrent Lot et ils lui dirent : Où sont les hommes qui sont venus cette nuit chez toi ? Fais-les sortir afin que nous les connaissions.
\VS{6}Mais Lot sortit de sa maison pour leur parler à la porte, et ayant fermé la porte après lui,
\VS{7}il leur dit : Je vous prie, mes frères, ne leur faites point de mal.
\VS{8}Voici, j'ai deux filles qui n'ont point encore connu d'homme ; je vous les amènerai et vous les traiterez comme il vous plaira. Seulement, ne faites pas de mal à ces hommes, car ils sont venus à l'ombre de mon toit.
\VS{9}Ils lui dirent : Retire-toi de là. Ils dirent aussi : Cet homme seul est venu pour habiter ici comme étranger, et il veut nous gouverner ? Maintenant nous te ferons pis qu'à eux. Et faisant violence à Lot,  ils s'approchèrent pour briser la porte\FTNT{2 Pi. 2:7-8.}.
\VS{10}Mais les hommes étendirent leurs mains, firent rentrer Lot vers eux dans la maison, et fermèrent la porte.
\VS{11}Et ils frappèrent d’aveuglement les hommes qui étaient à la porte de la maison, depuis le plus petit jusqu'au plus grand, de sorte qu'ils se lassèrent à chercher la porte.
\VS{12}Alors ces hommes dirent à Lot : Qui as-tu encore ici qui t'appartienne ? Gendres, fils et filles, et tout ce qui t'appartient dans la ville, fais-les sortir de ce lieu.
\VS{13}Car nous allons détruire ce lieu parce que le cri contre ses habitants est grand devant Yahweh. Yahweh nous a envoyés pour le détruire.
\VS{14}Lot sortit donc et parla à ses gendres, qui devaient prendre ses filles, et leur dit : Levez-vous, sortez de ce lieu, car Yahweh va détruire la ville. Mais aux yeux de ses gendres, il parut plaisanter.
\TextTitle{[Jugement sur Sodome]
\\(Jn. 3:29 ; 15:13-35)}
\VS{15}Dès l’aube du jour, les anges pressèrent Lot en disant : Lève-toi, prends ta femme et tes deux filles qui se trouvent ici, de peur que tu ne périsses dans le châtiment de la ville.
\VS{16}Et comme il tardait, ces hommes le prirent par la main, et ils prirent aussi par la main sa femme et ses deux filles, parce que Yahweh voulait l'épargner ; et ils l'emmenèrent et le mirent hors de la ville.
\VS{17}Après les avoir fait sortir, l'un d’eux dit : Sauve ta vie, ne regarde point derrière toi, et ne t'arrête en aucun endroit de la plaine ; sauve-toi sur la montagne, de peur que tu ne périsses.
\VS{18}Lot leur répondit : Non, Seigneur, je te prie.
\VS{19}Voici, ton serviteur a maintenant trouvé grâce devant toi, et tu as montré la grandeur de ta bonté à mon égard en préservant ma vie, mais je ne pourrai pas me sauver vers la montagne avant que le mal ne m'atteigne, et je mourrai.
\VS{20}Voici, je te prie, cette ville-là est proche ; je puis m'y enfuir, et elle est petite. Je te prie, que je m'y sauve ; n'est-elle pas petite ? Et mon âme vivra.
\VS{21}Et il lui dit : Voici, je t'ai exaucé encore en cela, de ne point détruire la ville dont tu as parlé.
\VS{22}Hâte-toi, sauve-toi là, car je ne pourrai rien faire jusqu'à ce que tu y sois entré ; c'est pourquoi cette ville fut appelée Tsoar.
\VS{23}Comme le soleil se levait sur la terre, Lot entra dans Tsoar.
\VS{24}Alors Yahweh fit pleuvoir du ciel, sur Sodome et sur Gomorrhe, du soufre et du feu, de la part de Yahweh\FTNT{De. 29:23 ; Lu. 17:29 ; Jud. 1:7.} ;
\VS{25}et il détruisit ces villes-là, et toute la plaine, et tous les habitants des villes, et les herbes de la terre.
\VS{26}Mais la femme de Lot regarda en arrière, et elle devint une statue de sel\FTNT{Lu. 17:31-33.}.
\VS{27}Abraham se leva de bon matin et vint au lieu où il s'était tenu devant Yahweh ;
\VS{28}et regardant vers Sodome et Gomorrhe, et vers toute la terre de cette plaine-là, il vit monter de la terre une fumée comme la fumée d'une fournaise.
\VS{29}Lorsque Dieu détruisit les villes de la plaine, il se souvint d'Abraham, et laissa Lot s’en aller  du milieu du désastre par lequel il détruisit les villes où Lot avait établi sa demeure.
\TextTitle{[Une abomination commise dans la famille de Lot]
\\(Ge. 13:10,12 ; 19:1 ; Lu.22:31-62)}
\VS{30}Lot quitta Tsoar et habita sur la montagne avec ses deux filles, car il craignait de demeurer dans Tsoar, et il se retira dans une caverne avec ses deux filles.
\VS{31}L'aînée dit à la plus jeune : Notre père est vieux, et il n'y a personne sur la terre pour venir vers nous, selon la coutume de tous les pays.
\VS{32}Viens, donnons du vin à notre père, et couchons avec lui  afin que nous conservions la race de notre père.
\VS{33}Elles donnèrent donc du vin à boire à leur père cette nuit-là ; et l'aînée vint, et coucha avec son père, mais il ne s'aperçut point ni quand elle se coucha ni quand elle se leva.
\VS{34}Le lendemain, l'aînée dit à la plus jeune : Voici, j'ai couché la nuit dernière avec mon père, donnons-lui encore du vin à boire cette nuit, puis va et couche avec lui, et nous conserverons la race de notre père.
\VS{35}Elles firent boire du vin à leur père encore cette nuit-là ; et la plus jeune se leva et coucha avec lui ; mais il ne s'aperçut point ni quand elle se coucha ni quand elle se leva.
\VS{36}Ainsi, les deux filles de Lot conçurent de leur père.
\VS{37}L’aînée enfanta un fils qu’elle appela du nom de Moab ; c'est le père des Moabites jusqu'à ce jour.
\VS{38}La plus jeune aussi enfanta un fils qu’elle appela du nom de Ben-Ammi ; c'est le père des Ammonites jusqu'à ce jour.
\TextTitle{[Faute d'Abraham à Guérar]
\\(Ge. 26:6-32)}
\Chap{20}
\VerseOne{}Abraham s'en alla de là pour le pays du midi ; il demeura entre Kadès et Schur, et il habita comme étranger à Guérar.
\VS{2}Abraham disait de Sara sa femme : C'est ma sœur. Et Abimélec, roi de Guérar, envoya des gens prendre Sara.
\VS{3}Mais Dieu apparut la nuit dans un songe à Abimélec, et lui dit : Voici, tu vas mourir, à cause de la femme que tu as prise, car elle a un mari.
\VS{4}Abimélec, qui ne s'était point approché d'elle, répondit : Seigneur, feras-tu donc mourir une nation juste ?
\VS{5}Ne m'a-t-il pas dit : C'est ma sœur? Et elle-même aussi n'a-t-elle pas dit : C'est mon frère ? J'ai fait ceci dans l'intégrité de mon cœur et dans la pureté de mes mains.
\VS{6}Dieu lui dit en songe : Je sais que tu l'as fait dans l'intégrité de ton cœur, aussi ai-je empêché que tu ne pèches contre moi ; c'est pourquoi je n'ai pas permis que tu la touches.
\VS{7}Maintenant donc rends la femme de cet homme, car il est prophète ; et il priera pour toi et tu vivras. Mais si tu ne la rends pas, sache que tu mourras toi et tout ce qui est à toi.
\VS{8}Abimélec se leva de bon matin, appela tous ses serviteurs, et rapporta à leurs oreilles  toutes ces choses, et ils furent saisis de crainte.
\VS{9}Puis Abimélec appela Abraham et lui dit : Que nous as-tu fait ? Et en quoi t'ai-je offensé que tu aies fait venir sur moi et sur mon royaume un grand péché ? Tu m'as fait des choses qui ne doivent point se faire.
\VS{10}Abimélec dit aussi à Abraham : Qu'as-tu vu qui t'aie obligé de faire cela ?
\VS{11}Abraham répondit : C'est parce que je disais : Assurément, il n'y a point de crainte de Dieu dans ce pays, et ils me tueront à cause de ma femme.
\VS{12}De plus, il est vrai qu’elle est ma soeur, fille de mon père ; mais elle n'est pas fille de ma mère ; et elle m'a été donnée pour femme.
\VS{13}Lorsque Dieu me fit errer loin de la maison de mon père, je dis à Sara : Voici la grâce que tu me feras, dis de moi dans tous les lieux où nous irons : C'est mon frère.
\VS{14}Alors Abimélec prit des brebis, des bœufs, des serviteurs et des servantes, et les donna à Abraham, et lui rendit Sara, sa femme.
\VS{15}Abimélec lui dit : Voici, mon pays est à ta disposition, demeure où il te plaira.
\VS{16}Et il dit à Sara : Voici, je donne à ton frère mille pièces d'argent ; cela te sera un voile sur les yeux  pour tous ceux qui sont avec toi, et envers tous les autres ; et ainsi elle fut reprise.
\VS{17}Abraham pria Dieu, et Dieu guérit Abimélec, sa femme, et ses servantes ; et elles eurent des enfants.
\VS{18}Car Yahweh avait frappé de stérilité en  fermant toute matrice de la maison d'Abimélec, à cause de Sara, femme d'Abraham.
\TextTitle{[Naissance d'Isaac]}
\Chap{21}
\VerseOne{}Et Yahweh visita Sara, comme il avait dit ; et il agit selon ses paroles.
\VS{2}Sara donc conçut, et enfanta un fils à Abraham dans sa vieillesse, au temps précis que Dieu lui avait dit.
\VS{3}Abraham donna le nom d’Isaac au fils qui lui était né, que Sara lui avait enfanté.
\VS{4}Abraham circoncit son fils Isaac âgé de huit jours, comme Dieu le lui avait ordonné.
\VS{5}Abraham était âgé de cent ans quand Isaac, son fils, lui naquit.
\VS{6}Et Sara dit : Dieu m'a donné de quoi rire ; tous ceux qui l'apprendront riront avec moi.
\VS{7}Elle dit aussi : Qui aurait dit à Abraham que Sara allaiterait des enfants ? Car je lui ai enfanté un fils dans sa vieillesse.
\VS{8}L'enfant grandit et fut sevré ; et Abraham fit un grand festin le jour où Isaac fut sevré.
\TextTitle{[Abraham chasse Agar avec Ismaël]
\\(Ga. 4:21-31)}
\VS{9}Sara vit rire le fils qu’Agar, l’Egyptienne, avait enfanté à Abraham ;
\VS{10}et elle dit à Abraham : Chasse cette servante et son fils, car le fils de cette servante n'héritera point avec mon fils, avec Isaac\FTNT{Ga. 4:30.}.
\VS{11}Cette parole déplut fort à Abraham à cause de son fils.
\VS{12}Mais Dieu dit à Abraham : N'aie point de chagrin au sujet de l'enfant ni de ta servante ;  écoute la parole de Sara dans toutes les choses qu’elle te dira, car en Isaac te sera donnée une postérité.
\VS{13}Je ferai aussi devenir le fils de la servante une nation, parce qu'il est ta semence.
\VS{14}Puis Abraham se leva de bon matin et prit du pain et une outre d'eau, et il les donna à Agar en les mettant sur son épaule. Il lui donna aussi l'enfant et la renvoya. Elle se mit en chemin et fut errante au désert de Beer-Schéba.
\VS{15}Quand l'eau de l’outre fut épuisée, elle jeta l'enfant sous un arbrisseau,
\VS{16}et elle alla s’asseoir vis-à-vis, à une portée d’arc, car elle dit : Que je ne voie pas mourir mon enfant. Elle s’assit donc vis-à-vis de lui, éleva la voix et pleura.
\VS{17}Dieu entendit la voix de l'enfant, et l'Ange de Dieu appela des cieux Agar et lui dit : Qu'as-tu Agar ? Ne crains point, car Dieu a entendu la voix de l'enfant du lieu où il est.
\VS{18}Lève-toi, lève l'enfant, et prends-le par la main, car je le ferai devenir une grande nation.
\VS{19}Et Dieu lui ouvrit les yeux et elle vit un puits d'eau ; elle alla remplir d'eau l’outre, et donna à boire à l'enfant.
\VS{20}Dieu fut avec l'enfant, qui devint grand, et demeura dans le désert ; et il fut tireur d'arc.
\VS{21}Il habita dans le désert de Paran ; et sa mère lui prit une femme du pays d'Egypte.
\TextTitle{[Abraham à Beer-Schéba]}
\VS{22}Et il arriva en ce temps-là qu'Abimélec, et Picol, chef de son armée, parla à Abraham en disant : Dieu est avec toi dans toutes les choses que tu fais.
\VS{23}Maintenant donc jure-moi ici par le nom de Dieu que tu ne me mentiras point, ni à mes enfants ni aux enfants de mes enfants, et que selon la faveur que je t'ai faite, tu agiras envers moi et envers le pays où tu séjournes comme étranger.
\VS{24}Abraham répondit : Je te le jurerai.
\VS{25}Mais Abraham fit des reproches à Abimélec au sujet d'un puits d'eau, dont les serviteurs d'Abimélec s'étaient emparés de force.
\VS{26}Abimélec répondit : J’ignore qui a fait cela, et aussi tu ne m'en as point informé, et moi, je ne l’apprends qu’aujourd’hui.
\VS{27}Alors Abraham prit des brebis et des bœufs, et les donna à Abimélec, et ils firent alliance ensemble.
\VS{28}Abraham mit à part sept jeunes brebis de son troupeau.
\VS{29}Et Abimélec dit à Abraham : Que veulent dire ces sept jeunes brebis que tu as mises à part ?
\VS{30}Il répondit : C'est que tu prendras ces sept jeunes brebis de ma main pour me servir de témoignage que j'ai creusé ce puits.
\VS{31}C'est pourquoi on appela ce lieu-là Beer-Schéba, car tous deux y jurèrent.
\VS{32}Ils traitèrent donc alliance à Beer-Schéba, puis Abimélec se leva avec Picol, chef de son armée, et ils retournèrent au pays des Philistins.
\VS{33}Abraham planta des tamaris à Beer-Schéba ; et là il invoqua le nom de Yahweh, le Dieu de l’éternité.
\VS{34}Abraham séjourna beaucoup de jours comme étranger dans le pays des Philistins.
\TextTitle{[Abraham présente Isaac en sacrifice]
\\(Hé. 11:17-19)}
\Chap{22}
\VerseOne{}Or, il arriva après ces choses, que Dieu éprouva Abraham et lui dit : Abraham ! Et il répondit : Me voici.
\VS{2}Et Dieu lui dit : Prends maintenant ton fils, ton unique, celui que tu aimes, Isaac, et va-t'en au pays de Morija, et là offre-le en holocauste sur l'une des montagnes que je te dirai.
\VS{3}Abraham donc s'étant levé de bon matin, sella son âne, et prit deux de ses serviteurs avec lui, et Isaac son fils ; et ayant fendu le bois pour l'holocauste, il se mit en chemin et s'en alla au lieu que Dieu lui avait dit.
\VS{4}Le troisième jour, Abraham levant ses yeux, vit le lieu de loin.
\VS{5}Et Abraham dit à ses serviteurs : Restez ici avec l'âne ; moi et l'enfant nous irons jusque-là pour adorer, après quoi nous reviendrons auprès de vous.
\VS{6}Abraham prit le bois de l'holocauste et le mit sur Isaac, son fils, et prit le feu dans sa main, et un couteau ; et ils s'en allèrent tous deux ensemble.
\VS{7}Alors Isaac parla à Abraham, son père, et dit : Mon père ! Abraham répondit : Me voici mon fils. Et il dit : Voici le feu et le bois, mais où est l’agneau pour l'holocauste\FTNT{Isaac est un autre type de Christ qui s’offre en sacrifice pour l’expiation de nos péchés. La réponse à sa question au v. 7:« Voici le feu et le bois, mais où est l’agneau pour l’holocauste ? », a été apportée bien des siècles plus tard par Jean-Baptiste:« Voici l’agneau de Dieu, qui ôte le péché du monde ». (Jn. 1:29).} ?
\VS{8}Abraham répondit : Mon fils, Dieu se pourvoira lui-même de l’agneau pour l'holocauste. Et ils marchèrent tous deux ensemble.
\VS{9}Et étant arrivés au lieu que Dieu lui avait dit, Abraham bâtit là un autel, et rangea le bois, et ensuite il lia Isaac, son fils, et le mit sur l'autel, par-dessus le bois\FTNT{Ja. 2:21.}.
\VS{10}Puis Abraham étendit sa main et prit le couteau pour égorger son fils.
\VS{11}Mais l'Ange de Yahweh l’appela des cieux et dit : Abraham, Abraham ! Il répondit : Me voici.
\VS{12}L’ange lui dit : Ne porte pas ta main sur l'enfant, et ne lui fais rien ; car maintenant je sais que tu crains Dieu, puisque tu ne m’as point refusé ton fils, ton unique.
\VS{13}Abraham leva les yeux et regarda ; et voici,  il vit derrière lui un bélier qui était retenu à un buisson par ses cornes ; et Abraham alla prendre le bélier et l'offrit en holocauste à la place de son fils.
\VS{14}Abraham donna à ce lieu le nom de Yahweh-Jiré (Yahweh pourvoira) ; c'est pourquoi on dit aujourd'hui : Dans la montagne de Yahweh il y sera pourvu.
\VS{15}L'Ange de Yahweh appela des cieux Abraham pour la seconde fois,
\VS{16}et dit : Je le jure par moi-même\FTNT{Hé. 6:13-15.}, parole de Yahweh ! Parce que tu as fait cela, et que tu n'as point refusé ton fils, ton unique,
\VS{17}certainement je te bénirai, et je multiplierai très abondamment ta postérité, comme les étoiles du ciel et comme le sable qui est sur le bord de la mer ; et ta postérité possédera la porte de ses ennemis.
\VS{18}Toutes les nations de la terre seront bénies en ta postérité, parce que tu as obéi à ma voix.
\VS{19}Ainsi Abraham retourna vers ses serviteurs, et ils se levèrent et s'en allèrent ensemble à Beer-Schéba ; car Abraham demeurait à Beer-Schéba.
\VS{20}Après ces choses, quelqu'un apporta des nouvelles à Abraham, en disant : Voici, Milca a aussi enfanté des fils à Nachor, ton frère.
\VS{21}Uts, son premier-né, et Buz, son frère, Kemuel, père d'Aram,
\VS{22}Késed, Hazo, Pildasch, Jidlaph et Bethuel.
\VS{23}Bethuel a engendré Rebecca. Milca enfanta ces huit fils à Nachor, frère d'Abraham.
\VS{24}Sa concubine, nommée Réuma, enfanta aussi Thébach, Gaham, Tahasch, et Maaca.
\TextTitle{[Mort de Sara]}
\Chap{23}
\VerseOne{}Or, Sara vécut cent vingt-sept ans ; ce sont là les années de la vie de Sara.
\VS{2}Sara mourut à Kirjath-Arba, qui est Hébron, dans le pays de Canaan ; et Abraham vint pour mener deuil sur Sara et pour la pleurer.
\VS{3}Et Abraham se leva de devant son mort, il parla aux fils de Heth, en disant :
\VS{4}Je suis étranger et habitant parmi vous ; donnez-moi une possession de sépulcre parmi vous, afin que j'enterre mon mort et que je l'ôte de devant moi\FTNT{Ac. 7:5.}.
\VS{5}Les fils de Heth répondirent à Abraham et lui dirent :
\VS{6}Mon seigneur, écoute-nous ! Tu es un prince de Dieu parmi nous, enterre ton mort dans le plus distingué de nos sépulcres ; nul de nous ne te refusera son sépulcre afin que tu y enterres ton mort.
\VS{7}Alors Abraham se leva et se prosterna devant le peuple du pays, devant les Héthiens.
\VS{8}Et il leur parla et dit : S'il vous plaît que j'enterre mon mort et que je l'ôte de devant moi ; écoutez-moi, et intercédez pour moi envers Ephron, fils de Tsochar,
\VS{9}afin qu'il me cède sa caverne de Macpéla, qui est à l’extrémité de son champ ; qu'il me la cède contre sa valeur en argent, afin qu’elle me serve de possession sépulcrale au milieu de vous.
\VS{10}Ephron était assis parmi les fils de Heth. Et Ephron, l’Héthien, répondit à Abraham, en présence des fils de Heth qui l'écoutaient, devant tous ceux qui entraient par la porte de sa ville, et dit :
\VS{11}Non, mon Seigneur, écoute-moi ! Je te donne le champ, je te donne aussi la caverne qui y est, je te la donne en présence des enfants de mon peuple ; enterres-y ton mort.
\VS{12}Abraham se prosterna devant le peuple du pays.
\VS{13}Et il parla ainsi à Ephron, en présence de tout le peuple du pays qui écoutait et dit : S'il te plaît, je te prie, écoute-moi ! Je donnerai l'argent du champ ; reçois-le de moi, et j'y enterrerai mon mort.
\VS{14}Et Ephron répondit à Abraham, en disant :
\VS{15}Mon Seigneur, écoute-moi ! La terre vaut quatre cents sicles d'argent, qu’est-ce que cela entre moi et toi ? Enterre donc ton mort.
\VS{16}Abraham ayant entendu Ephron, lui paya l'argent dont il avait parlé, en présence des fils de Heth, à savoir quatre cents sicles d'argent ayant cours chez les marchands\FTNT{Ac. 7:16.}.
\VS{17}Le champ d'Ephron, qui était à Macpéla, vis-à-vis de Mamré, le champ et la caverne qui y est, et tous les arbres qui sont dans le champ et dans toutes ses limites à l’entour,
\VS{18}tout fut acquis comme propriété d’Abraham, en présence des fils de Heth, et de tous ceux qui entraient par la porte de la ville.
\VS{19}Après cela, Abraham enterra Sara, sa femme, dans la caverne du champ de Macpéla, vis-à-vis de Mamré, qui est Hébron, dans le pays de Canaan.
\VS{20}Le champ et la caverne qui y est demeurèrent à Abraham comme possession sépulcrale, acquise des fils de Heth.
\TextTitle{[Abraham recherche une épouse à Isaac]}
\Chap{24}
\VerseOne{}Or, Abraham devint vieux et fort avancé en âge ; et Yahweh avait béni Abraham en toute chose.
\VS{2}Abraham dit à son serviteur, le plus ancien des serviteurs de sa maison, l’intendant de tout ce qui lui appartenait : Mets, je te prie, ta main sous ma cuisse ;
\VS{3}et je te ferai jurer par Yahweh, le Dieu du ciel et le Dieu de la terre, que tu ne prendras point de femme pour mon fils parmi les filles des Cananéens, au milieu desquels j'habite.
\VS{4}Mais tu iras dans mon pays et vers mes parents, et tu y prendras une femme pour mon fils Isaac.
\VS{5}Le serviteur lui répondit : Peut-être que la femme ne voudra-t-elle pas me suivre dans ce pays ; me faudra-t-il nécessairement ramener ton fils dans le pays d'où tu es sorti ?
\VS{6}Abraham lui dit : Garde-toi bien d'y ramener mon fils !
\VS{7}Yahweh, le Dieu du ciel, qui m'a fait sortir de la maison de mon père et de ma patrie, qui m'a parlé et qui m’a juré en disant : Je donnerai ce pays à ta postérité, enverra lui-même son ange devant toi ; et c’est là que tu prendras une femme pour mon fils.
\VS{8}Si la femme ne veut pas te suivre, tu seras quitte de ce serment que je te fais faire. Quoi qu'il en soit, tu n’y ramèneras point mon fils.
\VS{9}Le serviteur mit la main sous la cuisse d'Abraham, son Seigneur, et lui jura d’observer ces choses.
\VS{10}Alors le serviteur prit dix chameaux parmi les chameaux de son maître, et s'en alla, ayant à sa disposition tous les biens. Il partit donc et s'en alla en Mésopotamie, à la ville de Nachor.
\VS{11}Il fit reposer les chameaux sur leurs genoux hors de la ville, près d'un puits d'eau, sur le soir, au temps où sortent celles qui vont puiser de l'eau.
\VS{12}Et il dit : Ô Yahweh, Dieu de mon Seigneur Abraham, fais que j'aie une heureuse rencontre aujourd'hui ; et sois favorable à mon Seigneur Abraham.
\VS{13}Voici, je me tiens près de la source d'eau, et les filles des gens de la ville vont sortir pour puiser de l'eau.
\VS{14}Fais donc que la jeune fille à laquelle je dirai : Penche ta cruche, je te prie, afin que je boive, et qui me répondra : Bois, et je donnerai aussi à boire à tes chameaux, soit celle que tu as destinée à ton serviteur Isaac, et par là je connaîtrai que tu es favorable à mon Seigneur.
\VS{15}Il n’avait pas encore fini de parler que sortit sa cruche sur l’épaule, Rebecca, fille de Bethuel, fils de Milca, femme de Nachor, frère d'Abraham.
\VS{16}Et la jeune fille était très belle de figure ; elle était vierge, et aucun homme ne l'avait connue. Elle descendit donc à la source, et comme elle remontait après avoir rempli sa cruche,
\VS{17}le serviteur courut au-devant d'elle et lui dit : Laisse-moi boire, je te prie, un peu d’eau de ta cruche.
\VS{18}Elle répondit : Mon seigneur, bois. Elle s’empressa d’abaisser sa cruche sur sa main,  et elle lui donna à boire.
\VS{19}Quand elle eut achevé de lui donner à boire, elle dit : Je puiserai aussi pour tes chameaux jusqu'à ce qu'ils aient achevé de boire.
\VS{20}Et elle s’empressa de vider sa cruche dans l’abreuvoir ; elle courut encore au puits pour puiser de l'eau, et elle puisa pour tous ses chameaux.
\VS{21}L’homme la regardait avec étonnement et sans rien dire, pour voir si Yahweh faisait réussir son voyage ou non.
\VS{22}Quand les chameaux eurent fini de boire, l’homme prit un anneau d'or, du poids d'un demi-sicle, et deux bracelets, pour les mettre sur les mains de cette fille, pesant dix sicles d'or.
\VS{23}Et il lui dit : De qui es-tu fille ? Je te prie, fais-le-moi savoir. Y-a-t-il dans la maison de ton père de la place pour nous loger ?
\VS{24}Elle lui répondit : Je suis fille de Bethuel, fils de Milca et de Nachor.
\VS{25}Elle lui dit encore : Il y a chez nous de la paille et du fourrage en abondance, et de la place pour loger.
\VS{26}Alors l’homme s'inclina et adora Yahweh,
\VS{27}et dit : Béni soit Yahweh, le Dieu de mon seigneur Abraham, qui n'a point cessé d'exercer sa bonté et sa fidélité envers mon Seigneur ! Lorsque j'étais en chemin, Yahweh m'a conduit dans la maison des frères de mon seigneur.
\VS{28}La jeune fille courut et rapporta toutes ces choses à la maison de sa mère.
\VS{29}Rebecca avait un frère nommé Laban, qui courut dehors vers l’homme près de la source.
\VS{30}Il avait vu l’anneau et les bracelets aux mains de sa sœur, et il avait entendu les paroles de Rebecca sa sœur, disant : Ainsi m’a parlé l’homme. Il vint donc à cet homme qui se tenait auprès des chameaux, près de la source,
\VS{31}et il lui dit : Entre, béni de Yahweh ! Pourquoi te tiens-tu dehors ? J'ai préparé la maison et une place pour tes chameaux.
\VS{32}L'homme donc entra dans la maison. Laban fit décharger les chameaux, et il donna de la paille et du fourrage  aux chameaux ; et il apporta de l'eau pour laver les pieds de l’homme et les pieds de ceux qui étaient avec lui.
\VS{33}Et il lui présenta à manger. Mais il dit : Je ne mangerai point avant d’avoir dit  ce que j'ai à dire. Parle ! dit Laban.
\VS{34}Alors il dit : Je suis serviteur d'Abraham.
\VS{35}Yahweh a comblé de bénédictions mon seigneur qui est devenu puissant. Il lui a donné des brebis, des bœufs, de l'argent, de l'or, des serviteurs, des servantes, des chameaux, et des ânes.
\VS{36}Sara, la femme de mon Seigneur, a enfanté dans sa vieillesse un fils à mon Seigneur ; et il lui a donné tout ce qu'il possède.
\VS{37}Mon Seigneur m'a fait jurer en disant : Tu ne prendras point de femme pour mon fils parmi les filles des Cananéens dans le pays desquels j’habite ;
\VS{38}mais tu iras dans la maison de mon père et de ma famille prendre une femme pour mon fils.
\VS{39}J’ai dit à mon Seigneur : Peut-être que la femme ne voudra-t-elle pas me suivre.
\VS{40}Et il m’a répondu : Yahweh, devant la face de qui j'ai marché, enverra son ange avec toi, et fera réussir ton voyage ; et tu prendras pour mon fils une femme de ma famille et de la maison de mon père.
\VS{41}Quand tu auras été vers ma famille, tu seras alors dégagé de la punition du serment que je te fais faire ; et si on ne te la donne pas, tu seras dégagé de la punition du serment que je te fais faire.
\VS{42}Je suis arrivé aujourd'hui à la source et j'ai dit : Ô Yahweh ! Dieu de mon seigneur Abraham, si tu daignes faire réussir le voyage que j'ai entrepris,
\VS{43}voici, je me tiendrai près de la source d'eau, et la jeune fille qui sortira pour puiser à qui je dirai : Laisse-moi boire, je te prie, un peu d’eau de ta cruche ; et qui me répondra :
\VS{44}Bois toi-même, et je puiserai aussi pour tes chameaux, que cette jeune fille soit la femme que Yahweh a destinée au fils de mon seigneur.
\VS{45}Avant que j’aie fini de parler en mon cœur, voici, Rebecca est sortie, ayant sa cruche sur son épaule ; elle est descendue à la source et a puisé de l'eau ; et je lui ai dit : Donne-moi, à boire, je te prie.
\VS{46}Elle s’est empressée d’abaisser sa cruche de dessus son épaule et m'a dit : Bois, et même je donnerai à boire à tes chameaux. J'ai donc bu, et elle a aussi donné à boire aux chameaux.
\VS{47}Puis je l'ai interrogée en disant : De qui es-tu fille ? Elle a répondu : Je suis fille de Bethuel, fils de Nachor et de Milca. Alors je lui ai mis un anneau à son nez et les bracelets à ses mains.
\VS{48}Puis je me suis incliné, j’ai adoré Yahweh, et j'ai béni Yahweh, le Dieu de mon seigneur Abraham, qui m'a conduit fidèlement, afin que je prenne la fille du frère de mon seigneur pour son fils.
\VS{49}Maintenant donc, si vous voulez user de bonté et de fidélité envers mon seigneur, déclarez-le-moi ; sinon, déclarez-le-moi aussi ; et je me tournerai à droite ou à gauche.
\VS{50}Laban et Bethuel répondirent et dirent : Cette affaire vient de Yahweh, nous ne pouvons te parler ni en bien ni en mal.
\VS{51}Voici Rebecca est devant toi, prends-la et va, et qu'elle soit la femme du fils de ton seigneur, comme Yahweh l’a dit.
\VS{52}Lorsque le serviteur d'Abraham eut entendu leurs paroles, il se prosterna à terre devant Yahweh.
\VS{53}Et le serviteur sortit des objets d'argent et d'or, et des vêtements, et les donna à Rebecca. Il donna aussi de riches présents à son frère et à sa mère.
\VS{54}Puis ils mangèrent et burent, lui et les gens qui étaient avec lui, et ils passèrent la nuit.  Le matin,  quand ils furent levés, le serviteur dit : Laissez-moi retourner vers mon seigneur.
\VS{55}Le frère et la mère lui dirent : Que la jeune fille reste avec nous quelques jours encore, une dizaine de jours ;  après quoi, elle s'en ira.
\VS{56}Il leur répondit : Ne me retardez pas puisque Yahweh a fait réussir mon voyage ; laissez-moi partir  afin que je m'en aille vers mon seigneur.
\VS{57}Alors ils dirent : Appelons la jeune fille et demandons-lui son avis.
\VS{58}Ils appelèrent donc Rebecca et lui dirent : Veux-tu aller avec cet homme ? Et elle répondit : J'irai.
\VS{59}Ainsi ils laissèrent partir Rebecca, leur sœur, et sa nourrice, avec le serviteur d'Abraham et ses gens.
\VS{60}Ils bénirent Rebecca et lui dirent : Tu es notre sœur, puisses-tu devenir des milliers de myriades, et que ta postérité possède la porte de ses ennemis !
\VS{61}Alors Rebecca se leva avec ses servantes, et elles montèrent sur les chameaux et suivirent l’homme. Et le serviteur prit Rebecca et s'en alla.
\VS{62}Or Isaac revenait du puits de Lachaï-roï, et il habitait dans le pays du midi.
\VS{63}Un soir qu’Isaac était sorti dans les champs pour prier, il leva les yeux et regarda, et voici, des chameaux arrivaient.
\VS{64}Rebecca leva aussi les yeux, vit Isaac, et descendit de son chameau ;
\VS{65}car elle avait dit au serviteur : Qui est cet homme qui marche dans les champs à notre rencontre ? Et le serviteur avait répondu : C'est mon seigneur ; et elle prit son voile et se couvrit.
\VS{66}Le serviteur raconta à Isaac toutes les choses qu'il avait faites.
\VS{67}Alors Isaac conduisit Rebecca dans la tente de Sara, sa mère ; il prit Rebecca pour sa femme\FTNT{Pr. 18:22 ; Pr. 31:10-31.} et l'aima. Ainsi Isaac fut consolé après la mort de sa mère.
\TextTitle{[Ketura, femme d'Abraham]}
\Chap{25}
\VerseOne{}Or, Abraham prit une autre femme nommée Ketura.
\VS{2}Elle lui enfanta Zimram, Jokschan, Medan, Madian, Jischbak, et Schuach.
\VS{3}Jokschan engendra Séba et Dedan. Les fils de Dedan furent Aschurim, Letuschim et Leummim.
\VS{4}Les fils de Madian furent Epha, Epher, Hénoc, Abida, Eldaa. Ce sont là tous les fils de Ketura.
\TextTitle{[Isaac hérite d'Abraham]
\\(Hé. 1:2)}
\VS{5}Abraham donna tout ce qui lui appartenait à Isaac.
\VS{6}Mais il fit des dons aux fils de ses concubines, et tandis qu’il vivait encore, il les envoya loin de son fils Isaac, du côté de l'orient, dans le pays d’Orient.
\TextTitle{[Mort d'Abraham]}
\VS{7}Voici les jours des années de la vie d’Abraham : Il vécut cent soixante-quinze ans.
\VS{8}Abraham expira et mourut après une heureuse vieillesse, fort âgé et rassasié de jours, et il fut recueilli auprès de son peuple.
\VS{9}Isaac et Ismaël, ses fils, l'enterrèrent dans la caverne de Macpéla, dans le champ d'Ephron, fils de Tschoar, le Héthien, qui est vis-à-vis de Mamré.
\VS{10}C’est le champ qu'Abraham avait acheté des fils de Heth. Là furent enterrés Abraham et Sara, sa femme.
\VS{11}Après la mort d'Abraham, Dieu bénit Isaac son fils.  Isaac habitait près du puits de Lachaï-roï.
\TextTitle{[Postérité d'Ismaël]}
\VS{12}Voici la postérité d'Ismaël, fils d'Abraham, qu'Agar l’Egyptienne, servante de Sara, avait enfanté à Abraham.
\VS{13}Voici les noms des fils d'Ismaël, par leurs noms, selon leurs générations. Le premier-né d'Ismaël fut Nebajoth, puis Kédar, Adbeel, Mibsam,
\VS{14}Mischma, Duma, Massa,
\VS{15}Hadad, Théma, Jethur, Naphisch, et Kedma.
\VS{16}Ce sont là les fils d'Ismaël, et ce sont là leurs noms, selon leurs parcs, et selon leurs enclos ; douze princes de leurs peuples.
\VS{17}Et voici les années de la vie d'Ismaël : Cent trente-sept ans. Il expira et mourut, et il fut recueilli auprès de son peuple.
\VS{18}Ses descendants habitèrent depuis Havila jusqu'à Schur, qui est vis-à-vis de l'Egypte, en allant vers l'Assyrie. Et le pays qui était échu à Ismaël était à la vue de tous ses frères.
\TextTitle{[Postérité d'Isaac]}
\VS{19}Voici la postérité d'Isaac, fils d'Abraham.
\VS{20}Abraham engendra Isaac. Isaac était âgé de quarante ans quand il épousa Rebecca, fille de Bethuel, le Syrien, de Paddan-Aram, sœur de Laban, le Syrien.
\VS{21}Isaac pria instamment Yahweh au sujet de sa femme parce qu'elle était stérile ; et Yahweh exauça ses prières ; et Rebecca, sa femme, conçut.
\VS{22}Mais les enfants se heurtaient dans son ventre, et elle dit : S'il en est ainsi, pourquoi suis-je enceinte ? Et elle alla consulter Yahweh.
\VS{23}Et Yahweh lui dit : Deux nations sont dans ton ventre, et deux peuples se sépareront au sortir de tes entrailles ; un de ces peuples sera plus fort que l'autre, et le plus grand sera asservi au plus petit\FTNT{Ro. 9:12.}.
\TextTitle{[Naissance des jumeaux : Esaü et Jacob]}
\VS{24}Les jours où elle devait accoucher s’accomplirent ; et voici, il y avait deux jumeaux dans son ventre.
\VS{25}Celui qui sortit le premier était roux et tout velu, comme un manteau de poil ; et on lui donna le nom d’Esaü.
\VS{26}Ensuite sortit son frère, tenant de sa main le talon d'Esaü ; c'est pourquoi il fut appelé Jacob\FTNT{Jacob:« celui qui prend par le talon » ou « qui supplante ».}. Isaac était âgé de soixante ans quand ils naquirent.
\TextTitle{[Esaü méprise son droit d'ainesse]}
\VS{27}Depuis, les enfants devinrent grands. Esaü devint un habile chasseur, et un homme des champs ; mais Jacob fut un homme intègre, se tenant dans les tentes.
\VS{28}Isaac aimait Esaü ; car le gibier était sa nourriture. Mais Rebecca aimait Jacob.
\VS{29}Comme Jacob faisait cuire du potage, Esaü arriva des champs, et il était fatigué.
\VS{30}Et Esaü dit à Jacob : Donne-moi, je te prie, à manger de ce roux, de ce roux-là\FTNT{Probablement un plat de lentilles.} ; car je suis fatigué. C'est pourquoi on appela son nom, Edom\FTNT{Edom:« rouge, de couleur rousse ».}.
\VS{31}Mais Jacob lui dit : Vends-moi aujourd'hui ton droit d'aînesse.
\VS{32}Et Esaü répondit : Voici, je m'en vais mourir ; et de quoi me servira le droit d'aînesse ?
\VS{33}Et Jacob dit : Jure-moi aujourd'hui ; et il lui jura ; ainsi il vendit son droit d'aînesse à Jacob\FTNT{Hé. 12:16.}.
\VS{34}Et Jacob donna à Esaü du pain et du potage de lentilles ; et il mangea et but ; puis il se leva et s'en alla ; ainsi Esaü méprisa son droit d'aînesse.
\TextTitle{[Yahweh confirme son alliance à Isaac]}
\Chap{26}
\VerseOne{}Or, il y eut une famine dans le pays, outre la première famine qui eut lieu du temps d'Abraham ; et Isaac s'en alla vers Abimélec, roi des Philistins, à Guérar.
\VS{2}Yahweh lui apparut et lui dit : Ne descends pas en Egypte ; demeure dans le pays que je te dirai.
\VS{3}Demeure dans ce pays-ci, et je serai avec toi, et je te bénirai ; car je donnerai toutes ces contrées à toi et à ta postérité, et j’accomplirai le serment que j'ai fait à ton père Abraham.
\VS{4}Je multiplierai ta postérité comme les étoiles du ciel ; et je donnerai ces contrées à ta postérité ; et toutes les nations de la terre seront bénies en ta postérité,
\VS{5}parce qu'Abraham a obéi à ma voix, et qu'il a gardé mon ordonnance, mes commandements, mes statuts et mes lois.
\TextTitle{[Faute d'Isaac à Guérar]
\\(Ge. 20)}
\VS{6}Isaac donc demeura à Guérar.
\VS{7}Et quand les gens du lieu posaient des questions sur sa femme, il disait : C'est ma sœur ; car il craignait de dire : C'est ma femme ; de peur, disait-il, que les habitants du lieu ne me tuent à cause de Rebecca, car elle est belle de figure.
\VS{8}Comme son séjour se prolongeait, il arriva qu'Abimélec, roi des Philistins, regardant par la fenêtre, vit Isaac qui plaisantait avec Rebecca, sa femme\FTNT{Ge. 20.}.
\VS{9}Alors Abimélec appela Isaac et lui dit : Voici, c'est véritablement ta femme. Comment as-tu pu dire : C'est ma soeur ? Et Isaac lui répondit : C'est parce que j'ai dit : Il ne faut pas que je meure à cause d'elle.
\VS{10}Et Abimélec dit : Que nous as-tu fait ? Il s'en est peu fallu que quelqu'un du peuple n'ait couché avec ta femme, et tu nous aurais rendus coupables.
\VS{11}Abimélec donc fit une ordonnance à tout le peuple en disant : Celui qui touchera cet homme, ou à sa femme, sera certainement puni de mort.
\VS{12}Isaac sema dans cette terre-là et il recueillit cette année-là le centuple ; car Yahweh le bénit.
\VS{13}Cet homme devint riche, et il alla s’enrichissant de plus en plus, jusqu'à ce qu'il devint fort riche.
\VS{14}Il avait des troupeaux de menu bétail et des troupeaux de gros bétail, et un grand nombre de serviteurs ; aussi les Philistins lui portèrent-ils envie.
\VS{15}Et tous les puits que les serviteurs de son père avaient creusés, du temps de son père Abraham, les Philistins les bouchèrent et les remplirent de terre.
\VS{16}Abimélec aussi dit à Isaac : Va-t’en de chez nous, car tu es devenu beaucoup plus puissant que nous.
\TextTitle{[Les puits d'Isaac]}
\VS{17}Isaac donc partit de là, et campa dans la vallée de Guérar, où il s’établit.
\VS{18}Isaac creusa de nouveau les puits d'eau qu'on avait creusés du temps d'Abraham, son père, et que les Philistins avaient bouchés après la mort d'Abraham, et il leur donna les mêmes noms que son père leur avait donnés.
\VS{19}Les serviteurs d'Isaac creusèrent dans cette vallée et y trouvèrent un puits d'eau vive.
\VS{20}Mais les bergers de Guérar eurent une querelle avec les bergers d'Isaac, disant : L'eau est à nous. Et il appela le nom du puits Esek ; parce qu'ils avaient contesté avec lui.
\VS{21}Ensuite, ils creusèrent un autre puits, pour lequel ils contestèrent aussi ; et il appela son nom Sitna.
\VS{22}Alors il se transporta de là et creusa un autre puits pour lequel ils ne contestèrent point, et il le nomma Rehoboth, en disant : C'est parce que Yahweh nous a maintenant mis au large, et nous fructifierons dans le pays.
\VS{23}Et de là il remonta à Beer-Schéba.
\VS{24}Yahweh lui apparut cette nuit-là et lui dit : Je suis le Dieu d'Abraham, ton père ; ne crains point, car je suis avec toi, je te bénirai et je multiplierai ta postérité à cause d'Abraham, mon serviteur.
\VS{25}Alors il bâtit là un autel, et invoqua le nom de Yahweh, et il y dressa ses tentes. Et les serviteurs d'Isaac y creusèrent un puits.
\VS{26}Abimélec vint à lui de Guérar avec Ahuzath, son ami, et Picol, chef de son armée.
\VS{27}Mais Isaac leur dit : Pourquoi venez-vous vers moi, puisque vous me haïssez et que vous m'avez renvoyé de chez vous ?
\VS{28}Ils répondirent : Nous avons vu clairement que Yahweh est avec toi ; et nous avons dit : Qu'il y ait maintenant un serment solennel entre nous, c'est-à-dire entre nous et toi ; et traitons alliance avec toi.
\VS{29}Jure que tu ne nous feras aucun mal, de même que nous ne t'avons point maltraité, que nous t'avons fait seulement du bien, et que nous t’avons laissé partir en paix. Toi qui es maintenant béni de Yahweh.
\VS{30}Alors il leur fit un festin, et ils mangèrent et burent.
\VS{31}Ils se levèrent de bon matin, et jurèrent l'un à l'autre. Puis Isaac les renvoya, et ils s'en allèrent en paix.
\VS{32}Ce même jour, les serviteurs d'Isaac vinrent lui parler du puits qu'ils avaient creusé, et lui dirent : Nous avons trouvé de l'eau.
\VS{33}Et il l'appela Schiba. C'est pourquoi le nom de la ville a été Beer-Schéba jusqu'à aujourd'hui.
\VS{34}Esaü, âgé de quarante ans, prit pour femmes Judith, fille de Beéri, le Héthien, et Basmath, fille d'Elon, le Héthien.
\VS{35}Elles furent un sujet d’amertume pour l’esprit d’Isaac et de Rebecca.
\TextTitle{[Jacob prend la bénédiction d'Isaac à la place d'Esaü]}
\Chap{27}
\VerseOne{}Et il arriva que quand Isaac fut devenu vieux, et que ses yeux furent si affaiblis qu'il ne pouvait plus voir, il appela Esaü, son fils aîné, et lui dit : Mon fils ! Et il lui répondit : Me voici.
\VS{2}Isaac lui dit : Voici, maintenant je suis devenu vieux, et je ne connais pas le jour de ma mort.
\VS{3}Maintenant donc, je te prie, prends tes armes, ton carquois et ton arc, va dans les champs, et chasse-moi du gibier.
\VS{4}Apprête-moi un mets comme j’aime, et apporte-le-moi, afin que je mange, et que mon âme te bénisse avant que je meure.
\VS{5}Or Rebecca écoutait pendant qu'Isaac parlait à Esaü, son fils. Esaü donc s'en alla dans les champs pour chasser du gibier et pour le rapporter.
\VS{6}Et Rebecca parla à Jacob, son fils, et lui dit : Voici, j'ai entendu parler ton père à Esaü, ton frère, disant :
\VS{7}Apporte-moi du gibier, et fais-moi un mets, afin que je le mange ; et je te bénirai devant Yahweh avant de mourir.
\VS{8}Maintenant donc, mon fils, obéis à ma parole, et fais ce que je vais te commander.
\VS{9}Va maintenant à la bergerie, et prends-moi là deux bons chevreaux parmi les chèvres, et j'en ferai un mets pour ton père comme il aime.
\VS{10}Et tu le porteras à ton père, afin qu'il le mange et qu'il te bénisse avant sa mort.
\VS{11}Jacob répondit à Rebecca sa mère : Voici, Esaü, mon frère, est un homme velu, et je suis un homme sans poil.
\VS{12}Peut-être que mon père me touchera-t-il, et il me regardera comme un homme qui a voulu le tromper, et j'attirerai sur moi sa malédiction et non pas sa bénédiction.
\VS{13}Sa mère lui dit : Mon fils, que la malédiction que tu crains retombe sur moi ! Obéis seulement à ma parole, et va me prendre ce que je t'ai dit.
\TextTitle{[Déception d'Esaü]
\\(Hé. 12:16-17)}
\VS{14}Jacob alla les prendre et les apporta à sa mère ; et sa mère fit un mets comme son père aimait.
\VS{15}Puis Rebecca prit les plus précieux habits d'Esaü, son fils aîné, qu'elle avait dans la maison, et elle les fit mettre à Jacob, son fils cadet.
\VS{16}Elle couvrit ses mains et son cou, qui étaient sans poil, des peaux des chevreaux.
\VS{17}Puis elle mit entre les mains de son fils Jacob le mets et le pain qu'elle avait apprêtés.
\VS{18}Il vint vers son père, et lui dit : Mon père ! Il répondit : Me voici ; qui es-tu, mon fils ?
\VS{19}Jacob répondit à son père : Je suis Esaü, ton fils aîné ; j'ai fait ce que tu m’as dit. Lève-toi, je te prie, assieds-toi et mange de mon gibier, afin que ton âme me bénisse.
\VS{20}Isaac dit à son fils : Eh quoi ! Tu en as déjà trouvé, mon fils ! Et il dit : Yahweh ton Dieu l'a fait venir devant moi.
\VS{21}Isaac dit à Jacob : Approche-toi, je te prie, mon fils, et que je te touche, afin que je sache si tu es mon fils Esaü ou non.
\VS{22}Jacob donc s'approcha de son père Isaac, qui le toucha et dit : Cette voix est la voix de Jacob, mais ces mains sont les mains d'Esaü.
\VS{23}Et il ne le reconnut pas, car ses mains étaient velues comme les mains de son frère Esaü ; et il le bénit.
\VS{24}Il dit : C’est toi, mon fils Esaü ? Il répondit : Je le suis.
\VS{25}Isaac lui dit : Apporte-moi donc la viande, et que je mange du gibier de mon fils, afin que mon âme te bénisse. Jacob l'apporta, et Isaac mangea ; il lui apporta aussi du vin, et il but.
\VS{26}Puis Isaac, son père, lui dit : Approche-toi, je te prie, et embrasse-moi mon fils.
\VS{27}Jacob s'approcha et l’embrassa. Isaac sentit l'odeur de ses habits, et le bénit en disant : Voici l'odeur de mon fils, comme l'odeur d'un champ que Yahweh a béni.
\VS{28}Que Dieu te donne de la rosée du ciel, et de la graisse de la terre, du blé et du vin en abondance\FTNT{Hé. 11:20.} !
\VS{29}Que des peuples te servent, et que des nations se prosternent devant toi ! Sois le maître de tes frères, et que les fils de ta mère se prosternent devant toi ! Maudit soit quiconque te maudira, et béni soit quiconque te bénira.
\VS{30}Isaac avait fini de bénir Jacob, et Jacob avait à peine quitté son père Isaac, qu’Esaü, son frère, revint de la chasse.
\VS{31}Il apprêta aussi un mets, l’apporta à son père, et lui dit : Que mon père se lève et mange du gibier de son fils, afin que ton âme me bénisse.
\VS{32}Isaac, son père, lui dit : Qui es-tu ? Et il dit : Je suis ton fils, ton fils aîné, Esaü.
\VS{33}Isaac fut saisi d'une grande, d’une violente émotion, et dit : Qui est donc celui qui a chassé du gibier et me l’a apporté ? J'ai mangé de tout avant que tu ne viennes, et je l'ai béni. Aussi sera-t-il béni !
\VS{34}Dès qu'Esaü entendit les paroles de son père, il poussa de forts cris, pleins d’amertume, et il dit à son père : Bénis-moi aussi, bénis-moi, mon père !
\VS{35}Mais il dit : Ton frère est venu avec tromperie, et il a enlevé ta bénédiction.
\VS{36}Esaü dit : N'est-ce pas avec raison qu'on a appelé son nom Jacob ? Car il m'a déjà supplanté deux fois ; il m'a enlevé mon droit d'aînesse, et voici, maintenant il a enlevé ma bénédiction. Puis il dit : Ne m'as-tu point réservé de bénédiction ?
\VS{37}Isaac répondit à Esaü en disant : Voici, je l'ai établi ton maître, et lui ai donné tous ses frères pour serviteurs, et je l'ai pourvu de blé et de vin ; et que ferai-je maintenant pour toi, mon fils ?
\VS{38}Esaü dit à son père : N'as-tu qu'une bénédiction, mon père ? Bénis-moi aussi, bénis-moi, mon père ! Et Esaü éleva la voix et pleura\FTNT{Hé. 12:17.}.
\VS{39}Isaac, son père, répondit, et dit : Voici, ta demeure sera privée de la graisse de la terre, et de la rosée du ciel, d'en haut.
\VS{40}Tu vivras par ton épée, et tu seras asservi à ton frère ; mais il arrivera qu'étant devenu maître, tu briseras son joug de dessus ton cou.
\VS{41}Esaü conçut de la haine contre Jacob, à cause de la bénédiction dont son père l'avait béni ; et Esaü dit en son cœur : Les jours du deuil de mon père approchent, et je tuerai Jacob, mon frère.
\VS{42}On rapporta à Rebecca les paroles d'Esaü, son fils aîné ; et elle fit alors appeler Jacob, son fils cadet, et lui dit : Voici, Esaü, ton frère, se console dans l'espérance qu'il a de te tuer.
\VS{43}Maintenant donc, mon fils, obéis à ma parole ! Lève-toi, et enfuis-toi à Charan, vers Laban, mon frère.
\VS{44}Et reste avec lui quelque temps, jusqu'à ce que la fureur de ton frère soit passée ;
\VS{45}jusqu’à ce que la colère de ton frère se détourne de toi, et qu'il oublie ce que tu lui as fait. Pourquoi serais-je privée de vous deux en un même jour ?
\VS{46}Rebecca dit à Isaac : Je suis dégoûtée de la vie, à cause de filles de Heth. Si Jacob prend une femme, comme celles-ci, parmi les filles de Heth, parmi les filles du pays, à quoi me sert la vie ?
\TextTitle{[A Béthel Yahweh confirme son alliance à Jacob]}
\Chap{28}
\VerseOne{}Isaac donc appela Jacob, et le bénit, et lui donna cet ordre : Tu ne prendras point de femme parmi les filles de Canaan.
\VS{2}Lève-toi, va à Paddan-Aram, à la maison de Bethuel, père de ta mère, et prends-toi une femme de là, parmi les filles de Laban, frère de ta mère.
\VS{3}Que le Dieu Tout-Puissant te bénisse, te rende fécond et te multiplie, afin que tu deviennes une assemblée de peuples.
\VS{4}Qu’il te donne la bénédiction d'Abraham, à toi et à ta postérité avec toi, afin que tu obtiennes en héritage le pays où tu as été étranger, que Dieu a donné à Abraham.
\VS{5}Isaac donc fit partir Jacob, qui s'en alla à Paddan-Aram, vers Laban, fils de Bethuel, le Syrien, frère de Rebecca, mère de Jacob et d'Esaü.
\VS{6}Esaü vit qu'Isaac avait béni Jacob, et qu'il l'avait envoyé à Paddan-Aram afin qu'il prenne une femme de ce pays-là pour lui, et qu'il lui avait donné cet ordre, quand il le bénissait, disant : Ne prends point de femme parmi les filles de Canaan ;
\VS{7}il vit que Jacob avait obéi à son père et à sa mère, et qu’il était parti à Paddan-Aram.
\VS{8}Esaü comprit ainsi que les filles de Canaan déplaisaient à Isaac, son père.
\VS{9}Et Esaü s’en alla vers Ismaël. Il prit pour femme, outre ses autres femmes, Mahalath, fille d'Ismaël, fils d'Abraham, sœur de Nebajoth.
\VS{10}Jacob partit de Beer-Schéba et s'en alla à Charan.
\VS{11}Il arriva dans un lieu où il passa la nuit, parce que le soleil était couché. Il y prit donc une pierre\FTNT{1 Pi. 2:4. Voir  commentaire en Es. 8:13-15.}, et en fit son chevet, et il se coucha dans ce lieu-là.
\VS{12}Il eut un songe ; et voici, une échelle dressée sur la terre, dont le sommet touchait le ciel. Et voici, les anges de Dieu montaient et descendaient par cette échelle\FTNT{Jn. 1:51.}.
\VS{13}Et voici, Yahweh se tenait sur l'échelle, et il lui dit : Je suis Yahweh, le Dieu d'Abraham, ton père, et le Dieu d'Isaac ; je te donnerai à toi et à ta postérité, la terre sur laquelle tu es couché.
\VS{14}Ta postérité sera comme la poussière de la terre, et tu t'étendras à l'occident et à l'orient, au nord et au midi, et toutes les familles de la terre seront bénies en toi et en ta postérité.
\VS{15}Voici, je suis avec toi ; et je te garderai partout où tu iras ; et je te ramènerai dans ce pays ; car je ne t'abandonnerai point que je n'aie exécuté ce que je t'ai dit.
\VS{16}Et quand Jacob fut réveillé de son sommeil, il dit : Certainement, Yahweh est en ce lieu-ci, et moi, je ne le savais pas !
\VS{17}Il eut peur et dit : Que ce lieu-ci est effrayant ! C'est ici la maison de Dieu, et c'est ici la porte des cieux !
\VS{18}Et Jacob se leva de bon matin, prit la pierre dont il avait fait son chevet, il la dressa pour monument, et versa de l'huile sur son sommet.
\VS{19}Il donna à ce lieu le nom de Béthel ; mais auparavant la ville s'appelait Luz.
\VS{20}Jacob fit un vœu en disant : Si Dieu est avec moi, et s'il me garde pendant le voyage que je fais, s'il me donne du pain à manger, et des habits pour me vêtir,
\VS{21}et si je retourne en paix à la maison de mon père, certainement Yahweh sera mon Dieu.
\VS{22}Cette pierre que j'ai dressée pour monument sera la maison de Dieu ; et de tout ce que tu m'auras donné, je t'en donnerai entièrement la dîme\FTNT{Voir commentaire sur la dîme en No. 18:21 et Mal. 3:10.}.
\TextTitle{[Jacob épouse Léa et Rachel chez Laban]}
\Chap{29}
\VerseOne{}Jacob donc se mit en chemin, et s'en alla au pays des fils de l’orient.
\VS{2}Il regarda. Et voici, il y avait un puits dans un champ ; et voici il y avait à côté trois troupeaux de brebis couchées près du puits, car c’était à ce puits qu’on abreuvait les troupeaux.  Et il y avait une grosse pierre sur l'ouverture du puits.
\VS{3}Tous les troupeaux se rassemblaient là ; on roulait la pierre de dessus l'ouverture du puits, et on abreuvait les troupeaux ; et ensuite on remettait la pierre à sa place, sur l'ouverture du puits.
\VS{4}Jacob leur dit : Mes frères, d'où êtes-vous ? Ils répondirent : Nous sommes de Charan.
\VS{5}Il leur dit : Connaissez-vous Laban, fils de Nachor ? Ils répondirent : Nous le connaissons.
\VS{6}Il leur dit : Se porte-t-il bien ? Ils lui répondirent : Il se porte bien ; et voici Rachel, sa fille, qui vient avec le troupeau.
\VS{7}Il dit : Voici, il est encore grand jour, et il n'est pas temps de rassembler les troupeaux ; abreuvez les brebis, puis allez et faites-les paître.
\VS{8}Ils répondirent : Nous ne le pouvons pas, jusqu'à ce que tous les troupeaux soient rassemblés et qu'on ait ôté la pierre de dessus l'ouverture du puits, afin d'abreuver les troupeaux.
\VS{9}Comme il parlait encore avec eux, Rachel arriva avec le troupeau de son père ; car elle était bergère.
\VS{10}Lorsque Jacob vit Rachel, fille de Laban, frère de sa mère, et le troupeau de Laban, frère de sa mère, il s'approcha et roula la pierre de dessus l'ouverture du puits, et abreuva le troupeau de Laban, frère de sa mère.
\VS{11}Et Jacob embrassa Rachel, et il éleva sa voix et pleura.
\VS{12}Jacob apprit à Rachel qu'il était frère de son père, et qu'il était fils de Rebecca ; et elle courut le rapporter à son père.
\VS{13}Dès que Laban eut entendu parler de Jacob, fils de sa soeur, il courut au-devant de lui, il le prit dans ses bras et l’embrassa, et il le fit venir dans sa maison ; et Jacob raconta à Laban tout ce qui lui était arrivé.
\VS{14}Et Laban lui dit : Certainement, tu es mon os et ma chair. Jacob demeura un mois entier chez Laban.
\VS{15}Puis Laban dit à Jacob : Me serviras-tu pour rien parce que tu es mon frère ? Dis-moi quel sera ton salaire ?
\VS{16}Or Laban avait deux filles : L'aînée s'appelait Léa, et la cadette Rachel.
\VS{17}Léa avait les yeux délicats, mais Rachel était belle de taille et belle de figure.
\VS{18}Jacob aimait Rachel, et il dit : Je te servirai sept ans pour Rachel, ta cadette.
\VS{19}Et Laban répondit : Il vaut mieux que je te la donne que de la donner à un autre homme ; demeure avec moi.
\VS{20}Ainsi Jacob servit sept ans pour Rachel ; et elles furent à ses yeux comme quelques jours, parce qu'il l'aimait.
\VS{21}Et Jacob dit à Laban : Donne-moi ma femme, car mon temps est accompli, et j’irai vers elle.
\VS{22}Laban réunit tous les gens du lieu et fit un festin.
\VS{23}Mais quand le soir fut venu, il prit Léa, sa fille, et l'amena vers Jacob qui s’approcha d’elle.
\VS{24}Et Laban donna Zilpa, sa servante, à Léa, sa fille, pour servante.
\VS{25}Le lendemain matin, voilà que c'était Léa. Alors Jacob dit à Laban : Qu'est-ce que tu m'as fait ? N'ai-je pas servi chez toi pour Rachel ? Et pourquoi m'as-tu trompé ?
\VS{26}Laban répondit : On ne fait pas ainsi dans ce lieu de donner la plus jeune avant l'aînée.
\VS{27}Achève la semaine avec celle-ci, et nous te donnerons aussi l'autre, pour le service que tu feras encore chez moi sept autres années.
\VS{28}Jacob donc fit ainsi, et il acheva la semaine avec Léa ; et Laban lui donna aussi pour femme Rachel, sa fille.
\VS{29}Et Laban donna Bilha, sa servante, à Rachel, sa fille, pour servante.
\VS{30}Jacob alla aussi vers Rachel, et il aima Rachel plus que Léa ; et il servit encore chez Laban sept autres années.
\VS{31}Yahweh vit que Léa était haïe, et il ouvrit sa matrice, tandis que Rachel était stérile.
\VS{32}Léa conçut et enfanta un fils à qui elle donna le nom de Ruben, car elle dit : C'est parce que Yahweh a vu mon affliction, et maintenant mon mari m'aimera.
\VS{33}Elle conçut encore et enfanta un fils, et elle dit : Parce que Yahweh a entendu que j'étais haïe, il m'a aussi donné celui-ci. Et elle lui donna le nom de Siméon.
\VS{34}Elle conçut encore et enfanta un fils, et elle dit : Maintenant mon mari s'attachera à moi, car je lui ai enfanté trois fils. C'est pourquoi on lui donna le nom de Lévi.
\VS{35}Elle conçut encore et enfanta un fils, et elle dit : Cette fois je louerai Yahweh. C'est pourquoi elle lui donna le nom de Juda. Et elle cessa d'avoir des enfants.
\TextTitle{[Naissance des enfants de Jacob]}
\Chap{30}
\VerseOne{}Alors Rachel, voyant qu’elle ne donnait point d'enfants à Jacob, fut jalouse de Léa, sa sœur, et elle dit à Jacob : Donne-moi des enfants, autrement je meurs !
\VS{2}La colère de Jacob s’enflamma contre Rachel, et il dit : Suis-je à la place de Dieu pour t’empêcher d'avoir des enfants ?
\VS{3}Elle dit : Voici ma servante Bilha ; va vers elle ; qu’elle enfante sur mes genoux, et que j’aie des fils par elle.
\VS{4}Et elle lui donna pour femme Bilha, sa servante, et Jacob alla vers elle.
\VS{5}Bilha conçut et enfanta un fils à Jacob.
\VS{6}Rachel dit : Dieu a jugé en ma faveur, et il a aussi exaucé ma voix, et m'a donné un fils ; c'est pourquoi elle l’appela du nom de Dan.
\VS{7}Bilha, servante de Rachel, conçut encore et enfanta un second fils à Jacob.
\VS{8}Rachel dit : J'ai fortement lutté contre ma sœur, aussi j'ai eu la victoire ; c'est pourquoi elle l’appela du nom de Nephthali.
\VS{9}Alors Léa, voyant qu'elle avait cessé de faire des enfants, prit Zilpa, sa servante, et la donna pour femme à Jacob.
\VS{10}Zilpa, servante de Léa, enfanta un fils à Jacob.
\VS{11}Léa dit : Le bonheur est arrivé, c'est pourquoi elle l’appela du nom de Gad.
\VS{12}Zilpa, servante de Léa, enfanta un second fils à Jacob.
\VS{13}Léa dit : C'est pour me rendre heureuse, car les filles me diront bienheureuse ; c'est pourquoi elle l’appela du nom d’Aser.
\VS{14}Ruben sortit au temps de la moisson des blés, trouva des mandragores\FTNT{La mandragore, appelée pomme d'amour, était utilisée comme excitant du désir sexuel ainsi que pour favoriser la procréation. On attribuait à cette plante aux propriétés hallucinogènes des vertus magiques.} aux champs, et les apporta à Léa, sa mère ; et Rachel dit à Léa : Donne-moi, je te prie, des mandragores de ton fils.
\VS{15}Elle lui répondit : Est-ce peu que tu aies pris mon mari, pour que tu prennes aussi les mandragores de mon fils ? Et Rachel dit : Qu'il couche donc cette nuit avec toi pour les mandragores de ton fils.
\VS{16}Le soir, comme Jacob revenait des champs, Léa sortit au-devant de lui et lui dit : Tu viendras vers moi, car je t'ai acheté pour les mandragores de mon fils ; et il coucha avec elle cette nuit-là.
\VS{17}Dieu exauça Léa, et elle conçut et enfanta à Jacob un cinquième fils.
\VS{18}Léa dit : Dieu m'a récompensée, parce que j'ai donné ma servante à mon mari ; c'est pourquoi elle l’appela du nom d’Issacar.
\VS{19}Léa conçut encore et enfanta un sixième fils à Jacob.
\VS{20}Léa dit : Dieu m'a donné un beau don ; maintenant mon mari habitera avec moi, car je lui ai enfanté six fils ; c'est pourquoi elle l’appela du nom de Zabulon.
\VS{21}Puis elle enfanta une fille et la nomma Dina.
\VS{22}Dieu se souvint de Rachel, il l’exauça et il ouvrit sa matrice.
\VS{23}Alors elle conçut et enfanta un fils, et elle dit : Dieu a ôté mon opprobre.
\VS{24}Et elle lui donna le nom de Joseph, en disant : Que Yahweh m'ajoute un autre fils !
\TextTitle{[yahweh multiplie les troupeaux de Jacob]}
\VS{25}Lorsque Rachel eut enfanté Joseph, Jacob dit à Laban : Laisse-moi partir, pour que je m’en aille chez moi, dans mon pays.
\VS{26}Donne-moi mes femmes et mes enfants, pour lesquels je t'ai servi, et je m'en irai ; car tu sais de quelle manière je t'ai servi.
\VS{27}Laban lui répondit : Ecoute, je te prie, si j'ai trouvé grâce à tes yeux ; j’ai deviné que Yahweh m'a béni à cause de toi.
\VS{28}Il lui dit aussi : Fixe-moi le salaire que tu veux, et je te le donnerai.
\VS{29}Jacob lui répondit : Tu sais comment je t'ai servi et ce qu'est devenu ton bétail avec moi.
\VS{30}Car le peu que tu avais avant que je vienne s’est beaucoup accru, et Yahweh t'a béni depuis que j’ai mis mes pieds chez toi. Et maintenant, quand ferai-je aussi quelque chose pour ma maison ?
\VS{31}Laban lui dit : Que te donnerai-je ? Et Jacob répondit : Tu ne me donneras rien ; mais je ferai paître encore tes troupeaux, et je les garderai, si tu consens à ce que je vais te dire.
\VS{32}Je parcourrai aujourd'hui tes troupeaux, mets à part parmi toutes les brebis tachetées et marquetées, et tous les agneaux noirs, et les chèvres marquetées et tachetées. Ce sera mon salaire.
\VS{33}Ma justice me rendra témoignage à l’avenir devant toi ; quand tu viendras reconnaître mon salaire, en ta présence ; et tout ce qui ne sera pas marqueté ou tacheté parmi les chèvres, et noirs parmi les agneaux, sera considéré comme un vol s'il est trouvé chez moi.
\VS{34}Laban dit : Voici, qu'il te soit fait comme tu l'as dit.
\VS{35}Ce même jour, il sépara les boucs rayés et marquetés, et toutes les chèvres tachetées et marquetées, toutes celles où il y avait du blanc, et tous les agneaux noirs. Il les remit entre les mains de ses fils.
\VS{36}Puis il mit l'espace de trois journées de chemin entre lui et Jacob ; et Jacob fit paître le reste des troupeaux de Laban.
\VS{37}Mais Jacob prit des branches vertes de peuplier, d’amandier et de platane ; il y pela des bandes blanches,  mettant à nu le blanc qui était sur les branches.
\VS{38}Puis il plaça les branches qu’il avait pelées dans les auges, dans les abreuvoirs, sous les yeux des brebis qui venaient boire, et elles entraient en chaleur quand elles venaient boire.
\VS{39}Les brebis entraient en chaleur près des branches, et elles faisaient des brebis rayées, tachetées et marquetées.
\VS{40}Jacob séparait les agneaux, et il mettait ensemble ce qui était rayé et tout ce qui était noir dans les troupeaux de Laban. Il se fit ainsi des troupeaux à part, qu’il ne réunit point aux troupeaux de Laban.
\VS{41}Toutes les fois que les brebis vigoureuses entraient en chaleur, Jacob mettait les branches dans les auges sous les yeux des brebis, afin qu'elles entrent en chaleur près des  branches.
\VS{42}Mais pour les brebis chétives, il ne les mettait point ; de sorte que les chétives appartenaient à Laban et les vigoureuses à Jacob.
\VS{43}Ainsi cet homme devint de plus en plus riche ; il eut du menu bétail en abondance, des servantes et des serviteurs, des chameaux et des ânes.
\TextTitle{[Yahweh demande à Jacob de retourner à Béthel]}
\Chap{31}
\VerseOne{}Or Jacob entendit les discours des fils de Laban qui disaient : Jacob a pris tout ce qui appartenait à notre père, et c’est de ce qui était à notre père qu’il s’est acquis toute cette richesse.
\VS{2}Jacob regarda le visage de Laban, et voici, il n'était plus à son égard comme auparavant.
\VS{3}Alors Yahweh dit à Jacob : Retourne au pays de tes pères et vers ta parenté, et je serai avec toi.
\VS{4}Jacob fit appeler Rachel et Léa qui étaient aux champs vers son troupeau,
\VS{5}et leur dit : Je vois au visage de votre père qu'il n'est plus envers moi comme il était auparavant ; toutefois le Dieu de mon père a été avec moi.
\VS{6}Vous savez que j'ai servi votre père de tout mon pouvoir.
\VS{7}Mais votre père s'est moqué de moi et a changé dix fois mon salaire ; mais Dieu ne lui a pas permis de me faire du mal.
\VS{8}Quand il disait : Les tachetées seront ton salaire, alors toutes les brebis faisaient des agneaux tachetés ; et quand il disait : Les marquetées seront ton salaire, alors toutes les brebis faisaient des agneaux marquetés.
\VS{9}Ainsi Dieu a ôté à votre père son bétail et me l'a donné.
\VS{10}Au temps où les brebis entraient en chaleur, je levai mes yeux et vis en songe que les boucs qui couvraient les brebis étaient rayés, tachetés, et marquetés.
\VS{11}Et l'Ange de Dieu\FTNT{Gn. 7:7} me dit en songe : Jacob ! Et je répondis : Me voici.
\VS{12}Il dit : Lève maintenant tes yeux et regarde : Tous les boucs qui couvrent les brebis sont rayés, tachetés et marquetés, car j'ai vu tout ce que te fait Laban.
\VS{13}Je suis le Dieu de Béthel, où tu oignis la pierre que tu dressas pour monument, où tu me fis un vœu.  Maintenant lève-toi, sors de ce pays, et retourne au pays de ta naissance.
\TextTitle{[Jacob fuit de chez Laban avec sa famille]}
\VS{14}Alors Rachel et Léa lui répondirent et dirent : Avons-nous encore quelque portion et quelque héritage dans la maison de notre père ?
\VS{15}Ne nous a-t-il pas traitées comme des étrangères ? Car il nous a vendues, et même il a entièrement mangé notre argent.
\VS{16}Car toutes les richesses que Dieu a ôtées à notre père nous appartenaient ainsi qu’à nos enfants. Maintenant donc fais tout ce que Dieu t'a dit.
\VS{17}Ainsi Jacob se leva, et fit monter ses enfants et ses femmes sur des chameaux.
\VS{18}Il emmena tout son bétail et tous les biens qu'il avait acquis, et tout ce qu'il possédait et qu'il avait acquis à Paddan-Aram, pour aller vers Isaac, son père, au pays de Canaan.
\VS{19}Tandis que Laban était allé tondre ses brebis, Rachel déroba les théraphim de son père\FTNT{Les théraphim étaient des idoles utilisées dans un sanctuaire de maison ou dans un lieu de culte. Voir Jg. 18:14 ; 2 R. 23:24.}.
\VS{20}Et Jacob trompa Laban, le Syrien, en ne l’avertissant pas de son dessein, parce qu'il s'enfuyait.
\VS{21}Il s'enfuit avec tout ce qui lui appartenait ; il se leva, traversa le fleuve, et se dirigea vers la montagne de Galaad.
\VS{22}Le troisième jour, on rapporta à Laban que Jacob s’était enfui.
\VS{23}Alors il prit avec lui ses frères, et il le poursuivit sept journées de marche, et l'atteignit à la montagne de Galaad.
\VS{24}Mais Dieu apparut à Laban, le Syrien, en songe la nuit, et lui dit : Garde-toi de parler à Jacob ni en bien ni en mal.
\VS{25}Laban donc atteignit Jacob. Jacob avait dressé ses tentes sur la montagne ; et Laban dressa aussi les siennes avec ses frères sur la montagne de Galaad.
\VS{26}Et Laban dit à Jacob : Qu'as-tu fait ? Tu m’as trompé, tu as emmené mes filles comme des prisonnières de guerre.
\VS{27}Pourquoi as-tu pris la fuite secrètement, m’as-tu trompé et ne m’as-tu pas averti ? Car je t'aurais laissé partir avec joie et avec des chansons, au son des tambours et des violons.
\VS{28}Tu ne m'as pas laissé embrasser mes fils et mes filles ! C’est en insensé que tu as agi.
\VS{29}J'ai en main le pouvoir de vous faire du mal, mais le Dieu de votre père m'a parlé la nuit passée et m'a dit : Garde-toi de ne parler à Jacob ni en bien ni en mal.
\VS{30}Maintenant que tu es parti, parce que tu  languissais après la maison de ton père, pourquoi as-tu dérobé mes dieux ?
\VS{31}Jacob répondit et dit à Laban : Je me suis enfui parce que je craignais ; car je me disais qu'il fallait prendre garde que tu ne me ravisses tes filles.
\VS{32}Mais celui chez qui tu trouveras tes dieux ne vivra point. En présence de nos frères, examine  s'il y a chez moi quelque chose qui t'appartienne, et prends-le ; car Jacob ignorait que Rachel les avait dérobés.
\VS{33}Alors Laban entra dans la tente de Jacob, et dans celle de Léa, et dans la tente des deux servantes, et il ne les trouva point ; et étant sorti de la tente de Léa, il entra dans la tente de Rachel.
\VS{34}Mais Rachel avait prit les théraphim et les avait mis dans le bât d'un chameau, et s’était assise dessus ; et Laban fouilla toute la tente et ne les trouva point.
\VS{35}Elle dit à son père : Que mon seigneur ne se fâche point de ce que je ne puis me lever devant lui, car j'ai ce que les femmes ont coutume d'avoir ; et il fouilla, mais il ne trouva point les théraphim.
\VS{36}Jacob se mit en colère et querella Laban. Il reprit la parole et lui dit : Quel est mon crime ? Quel est mon péché, pour que tu me poursuives avec tant d’ardeur ?
\VS{37}Car tu as fouillé tous mes effets, qu'as-tu trouvé des effets de ta maison ? Mets-les ici devant mes frères et les tiens, et qu'ils soient juges entre nous deux.
\VS{38}Voilà vingt ans que j’ai passés chez toi ; tes brebis et tes chèvres n'ont point avorté, je n'ai point mangé les moutons de tes troupeaux.
\VS{39}Je ne t'ai point rapporté de bêtes déchirées par les bêtes sauvages, j'en ai moi-même subi la perte ; et tu redemandais de ma main ce qui avait été dérobé de jour et ce qui avait été dérobé de nuit.
\VS{40}Le jour la chaleur me consumait, et la nuit le froid ; et le sommeil fuyait de mes yeux.
\VS{41}Voilà vingt ans que j’ai passés dans ta maison, quatorze ans pour tes deux filles, et six ans pour tes troupeaux, et tu m'as changé dix fois mon salaire.
\VS{42}Si je n’avais pas eu pour moi le Dieu de mon père, le Dieu d'Abraham, et celui que craint Isaac, certes tu m’aurais maintenant renvoyé à vide. Mais Dieu a regardé mon affliction et le travail de mes mains, et il t'a repris la nuit passée.
\VS{43}Laban répondit à Jacob et dit : Ces filles sont mes filles, et ces enfants sont mes enfants, et ces troupeaux sont mes troupeaux, et tout ce que tu vois est à moi ; et que ferais-je aujourd'hui à mes filles et aux enfants qu'elles ont enfantés ?
\VS{44}Maintenant donc, viens, faisons ensemble une alliance, et qu’elle serve de témoignage entre moi et toi.
\VS{45}Jacob prit une pierre et il la dressa pour monument.
\VS{46}Jacob dit à ses frères : Ramassez des pierres. Et ils prirent des pierres et ils en firent un monceau, et ils mangèrent là sur ce monceau.
\VS{47}Laban l'appela Jegar-Sahadutha, et Jacob l'appela Galed.
\VS{48}Et Laban dit : Ce monceau sera aujourd'hui témoin entre moi et toi ; c'est pourquoi il fut nommé Galed (poste d’observation).
\VS{49}Il fut aussi appelé Mitspa ; parce que Laban dit : Que Yahweh veille sur moi et sur toi, quand nous nous serons l'un et l'autre perdus de vue.
\VS{50}Si tu maltraites mes filles et si tu prends une autre femme que mes filles, ce n’est pas un homme qui sera témoin entre nous, prends-y garde ; c'est Dieu qui est témoin entre moi et toi.
\VS{51}Laban dit encore à Jacob : Regarde ce monceau, et considère le monument que j'ai dressé entre moi et toi.
\VS{52}Que ce monceau soit témoin et que ce monument soit témoin que je n’irai pas vers toi au-delà de ce monceau, et que tu ne viendras pas vers moi au-delà de ce monceau et de ce monument pour me faire du mal.
\VS{53}Que le Dieu d'Abraham et le Dieu de Nachor, le Dieu de leur père, juge entre nous ; mais Jacob jura par celui que craignait Isaac, son père.
\VS{54}Jacob offrit un sacrifice sur la montagne et invita ses frères pour manger du pain ; ils mangèrent donc du pain et passèrent la nuit sur la montagne.
\VS{55}Laban se leva de bon matin, embrassa ses fils et ses filles, et les bénit. Ensuite il s'en alla. Ainsi Laban retourna chez lui.
\TextTitle{[Jacob devient Israël]}
\Chap{32}
\VerseOne{}Et Jacob continua son chemin, et des anges de Dieu le rencontrèrent.
\VS{2}En les voyant, Jacob dit : C'est ici le camp de Dieu ! Et il donna à ce lieu le nom de Mahanaïm.
\VS{3}Jacob envoya devant lui des messagers vers Esaü, son frère, au pays de Séir, dans le territoire d'Edom.
\VS{4}Il leur donna cet ordre : Vous parlerez de cette manière à mon seigneur Esaü : Ainsi a dit ton serviteur Jacob : J'ai séjourné comme étranger chez Laban, et j’y ai habité jusqu'à présent ;
\VS{5}j’ai des bœufs, des ânes, des brebis, des serviteurs, et des servantes ; et j'envoie l’annoncer à mon seigneur, afin de trouver grâce à  tes yeux.
\VS{6}Et les messagers revinrent auprès de Jacob et lui dirent : Nous sommes allés vers ton frère Esaü, et il marche aussi à ta rencontre avec quatre cents hommes.
\VS{7}Alors Jacob fut très effrayé et rempli d’angoisse ; et il partagea le peuple qui était avec lui, et les brebis, et les boeufs, et les chameaux, en deux camps ; et  il dit :
\VS{8}Si Esaü attaque l'un des camps et le frappe, le camp qui restera pourra s’échapper.
\VS{9}Jacob dit aussi : Ô Dieu de mon père Abraham, Dieu de mon père Isaac, ô Yahweh qui m'as dit : Retourne dans ton pays, et vers ta parenté, et je te ferai du bien.
\VS{10}Je suis trop petit pour toutes les faveurs et pour toute la fidélité dont tu as usé envers ton serviteur ; car j'ai passé ce Jourdain avec mon bâton, et maintenant je forme deux camps.
\VS{11}Je te prie, délivre-moi de la main de mon frère Esaü ; car je crains qu'il ne vienne, et qu'il ne me frappe, et qu'il ne tue la mère avec les enfants.
\VS{12}Et toi, tu as dit : Certes, je te ferai du bien, et je rendrai ta postérité comme le sable de la mer, si abondant qu’on ne saurait le compter.
\VS{13}C’est dans ce lieu-là que Jacob passa la nuit.  Il prit de ce qu’il avait sous la main pour faire un présent à Esaü, son frère :
\VS{14}à savoir deux cents chèvres, vingt boucs, deux cents brebis et vingt béliers.
\VS{15}Trente femelles de chameaux qui allaitaient, et leurs petits ; quarante jeunes vaches, dix jeunes taureaux, vingt ânesses et dix ânes.
\VS{16}Il les mit entre les mains de ses serviteurs, chaque troupeau à part, et leur dit : Passez devant moi, et faites qu'il y ait un intervalle entre chaque troupeau.
\VS{17}Il donna cet ordre au premier, disant : Quand Esaü, mon frère, te rencontrera et te demandera, disant : A qui es-tu ? Et où vas-tu ? Et à qui sont ces choses qui sont devant toi ?
\VS{18}Alors tu diras : Je suis à ton serviteur Jacob ; c'est un présent qu'il envoie à mon seigneur Esaü ; et voici, il vient lui-même derrière nous.
\VS{19}Il donna le même ordre au deuxième, au troisième, et à tous ceux qui suivaient les troupeaux, disant : C’est ainsi que vous parlerez à mon seigneur Esaü, quand vous le rencontrerez.
\VS{20}Vous lui direz : Voici, ton serviteur Jacob vient aussi derrière nous. Car il se disait : J'apaiserai sa colère par ce présent qui va devant moi, et après cela, je verrai sa face ; peut-être qu'il me regardera favorablement.
\VS{21}Le présent passa devant lui ; mais il resta cette nuit-là dans le camp.
\VS{22}Il se leva cette nuit, et prit ses deux femmes, ses deux servantes, et ses onze enfants, et passa le gué de Jabbok.
\VS{23}Il les prit donc, et leur fit passer le torrent ; il fit aussi passer tout ce qu'il avait.
\VS{24}Jacob demeura seul. Alors un homme lutta avec lui jusqu'au lever de l’aurore.
\VS{25}Et quand cet homme vit qu'il ne pouvait pas le vaincre, il frappa à l'emboîture de la hanche de Jacob ; ainsi l'emboîture de l'os de la hanche de Jacob se démit pendant qu’il luttait avec lui.
\VS{26}Et cet homme lui dit : Laisse-moi, car l'aube du jour est levée. Mais il dit : Je ne te laisserai point que tu ne m'aies béni.
\VS{27}Cet homme lui dit : Quel est ton nom ? Il répondit : Jacob.
\VS{28}Alors il dit : Ton nom ne sera plus Jacob, mais tu seras appelé Israël ; car tu as été le vainqueur en luttant avec Dieu et avec les hommes, et tu as été le plus fort.
\VS{29}Jacob l’interrogea en disant : Je te prie, déclare-moi ton nom. Et il répondit : Pourquoi demandes-tu mon nom ? Et il le bénit là\FTNT{Jg. 13:18.}.
\VS{30}Jacob appela ce lieu du nom de Peniel ; car, dit-il, j’ai vu Dieu face à face, et mon âme a été délivrée.
\VS{31}Le soleil se levait lorsqu’il passa Peniel. Jacob boitait de la hanche.
\VS{32}C'est pourquoi, jusqu'à ce jour, les enfants d'Israël ne mangent point le tendon qui est à l’emboîture de la hanche ; parce que Dieu frappa Jacob à l'emboîture de la hanche, au tendon.
\TextTitle{[Jacob demande pardon à son frère Esaü]}
\Chap{33}
\VerseOne{}Et Jacob leva ses yeux et regarda ; et voici, Esaü arrivait avec quatre cents hommes. Et Jacob répartit les enfants entre Léa, Rachel, et les deux servantes.
\VS{2}Il plaça en tête les servantes avec leurs enfants ; Léa et ses enfants ensuite ; et Rachel et Joseph au dernier rang.
\VS{3}Quant à lui,  il passa devant eux et se prosterna à terre sept fois, jusqu'à ce qu'il soit près de son frère.
\VS{4}Esaü courut à sa rencontre ; il le prit dans ses bras, se jeta sur son cou, et l’embrassa. Et ils pleurèrent.
\VS{5}Esaü leva ses yeux, vit les femmes et les enfants, et dit : Qui sont ceux-là ? Sont-ils à toi ? Jacob lui répondit : Ce sont les enfants que Dieu, par sa grâce, a donnés à ton serviteur.
\VS{6}Les servantes s'approchèrent, elles et leurs enfants, et se prosternèrent.
\VS{7}Puis Léa aussi s'approcha avec ses enfants, et ils se prosternèrent, et ensuite Joseph et Rachel s'approchèrent et se prosternèrent aussi.
\VS{8}Esaü dit : Que veux-tu faire avec tout ce camp que j'ai rencontré ? Et Jacob répondit : C'est pour trouver grâce aux yeux de mon seigneur.
\VS{9}Esaü dit : Je suis dans l’abondance, mon frère ; garde ce qui est à toi.
\VS{10}Et Jacob répondit : Non, je te prie, si j'ai maintenant trouvé grâce à tes yeux, reçois ce présent de ma main ; parce que j'ai vu ta face comme si j'avais vu la face de Dieu, et parce que tu m’as accueilli favorablement.
\VS{11}Accepte, je te prie, mon présent qui t'a été offert ; car Dieu m’a comblé de grâce, et je ne manque de rien. Il le pressa tant qu'il le prit.
\VS{12}Esaü dit : Partons et marchons, et je marcherai devant toi.
\VS{13}Mais Jacob lui dit : Mon seigneur sait que ces enfants sont jeunes et que j’ai des brebis et de vaches qui allaitent ; si l’on forçait leur marche un seul jour, tout le troupeau mourra.
\VS{14}Je te prie que mon seigneur passe devant son serviteur, et je m’avancerai tout doucement, au pas de ce bétail qui est devant moi, et au pas de ces enfants, jusqu'à ce que j'arrive chez mon seigneur à Séir.
\VS{15}Esaü dit : Je te prie, je vais au moins laisser avec toi une partie de ce peuple qui est avec moi ; et il répondit : Pourquoi cela ? Je te prie que je trouve grâce aux yeux de mon seigneur.
\VS{16}Ainsi Esaü  retourna ce jour-là par son chemin à Séir.
\TextTitle{[Jacob dresse un autel à El-Elohé-Israël (Dieu Fort, Dieu d'Israël)]}
\VS{17}Jacob partit pour Succoth. Il bâtit une maison pour lui, et il fit des cabanes pour son bétail. C’est pourquoi il appela ce lieu du nom de Succoth.
\VS{18}A son retour de  Paddan-Aram, Jacob arriva sain et sauf à la ville de Sichem, dans le pays de Canaan, et il campa devant la ville.
\VS{19}Il acheta une portion du champ où il avait dressé sa tente de la main des fils de Hamor, père de Sichem, pour cent pièces d'argent.
\VS{20}Et là, il dressa un autel qu'il appela El-Elohé-Israël (le Dieu Fort, le Dieu d'Israël).
\TextTitle{[Le péché dans la famille de Jacob]}
\Chap{34}
\VerseOne{}Or Dina, la fille que Léa avait enfantée à Jacob, sortit pour voir les filles du pays.
\VS{2}Elle fut aperçue de Sichem, fils de Hamor, le Hévien, prince du pays. Il l'enleva et coucha avec elle, et la déshonora.
\VS{3}Son cœur fut attaché à Dina, fille de Jacob ; il aima la jeune fille et sut parler au cœur de la jeune fille.
\VS{4}Et Sichem parla à Hamor, son père, en disant : Prends-moi cette fille pour femme.
\VS{5}Jacob apprit qu'il avait déshonoré Dina, sa fille. Or ses fils étaient avec son bétail aux champs ; Jacob garda le silence jusqu’à leur retour.
\VS{6}Hamor, père de Sichem, sortit vers Jacob pour lui  parler.
\VS{7}Et les fils de Jacob revinrent des champs dès qu’ils apprirent ce qui était arrivé ; ces hommes furent dans  une grande douleur, et furent fort irrités de l'infamie que Sichem avait commise contre Israël, en couchant avec la fille de Jacob, ce qui ne devait point se faire.
\VS{8}Hamor leur parla en disant : L’âme de Sichem, mon fils, s’est attachée à votre fille ; donnez-la-lui je vous prie pour femme.
\VS{9}Alliez-vous avec nous, vous nous donnerez vos filles, et vous prendrez pour vous les nôtres.
\VS{10}Vous habiterez avec nous, et le pays sera à votre disposition ; restez pour y trafiquer et y acquérir des possessions.
\VS{11}Sichem dit aussi au père et aux frères de la fille : Que je trouve grâce à vos yeux, et je donnerai tout ce que vous me direz.
\VS{12}Exigez de moi une forte dot, et beaucoup de présents que vous voudrez, et je les donnerai comme vous me direz ; et donnez-moi la jeune fille pour femme.
\VS{13}Alors les fils de Jacob répondirent avec ruse à Sichem et à Hamor, son père ; ils parlèrent ainsi parce que Sichem avait déshonoré Dina, leur sœur.
\VS{14}Ils leur dirent : C’est une chose que ne pouvons pas faire, que de donner notre sœur à un homme incirconcis, car ce serait un opprobre pour nous.
\VS{15}Mais nous ne consentirons à ce que vous demandez que si vous deveniez semblables à nous en circoncisant tous les mâles qui sont parmi vous.
\VS{16}Alors nous vous donnerons nos filles, et nous prendrons vos filles pour nous, et nous habiterons avec vous, et nous ne serons qu'un seul peuple.
\VS{17}Mais si vous ne voulez pas nous écouter et vous circoncire, nous prendrons notre fille et nous nous en irons.
\VS{18}Leurs discours plurent à Hamor et à Sichem, fils d'Hamor.
\VS{19}Le jeune homme ne tarda point à faire ce qu'on lui avait proposé, car la fille de Jacob lui plaisait beaucoup ; et il était le plus considéré de tous ceux de la maison de son père.
\VS{20}Hamor et Sichem, son fils, se rendirent à la porte de leur ville et parlèrent aux gens de leur ville en leur disant :
\VS{21}Ces hommes sont paisibles à notre égard ; qu'ils habitent dans le pays et qu'ils y trafiquent ; car voici, le pays est assez vaste pour eux. Nous prendrons pour femmes leurs filles, et nous leur donnerons nos filles.
\VS{22}Mais ces hommes ne consentiront à habiter avec nous, pour former un seul peuple, que si tout mâle qui est parmi nous est circoncis, comme ils sont eux-mêmes circoncis.
\VS{23}Leur bétail, et leurs biens, et toutes leurs bêtes, ne seront-ils pas à nous ? Accordons-leur seulement cela, et qu'ils demeurent avec nous.
\VS{24}Tous ceux qui sortaient par la porte de leur ville obéirent à Hamor et à Sichem, son fils ; et tout mâle d'entre tous ceux qui sortaient par la porte de leur ville fut circoncis.
\VS{25}Le troisième jour, pendant qu’ils étaient souffrants, deux des fils de Jacob, Siméon et Lévi, frères de Dina, prirent leurs épées, entrèrent hardiment dans la ville et tuèrent tous les mâles.
\VS{26}Ils passèrent aussi au tranchant de l'épée Hamor et Sichem, son fils ; ils enlevèrent Dina de la maison de Sichem, et sortirent.
\VS{27}Les fils de Jacob se jetèrent sur les morts et pillèrent la ville, parce qu'on avait déshonoré leur sœur.
\VS{28}Ils prirent leurs troupeaux, leurs bœufs, leurs ânes, et ce qui était dans la ville et dans les champs ;
\VS{29}et toutes leurs richesses, leurs petits enfants, et ils emmenèrent prisonnières leurs femmes ; et ils les pillèrent avec tout ce qui était dans les maisons.
\VS{30}Alors Jacob dit à Siméon et Lévi : Vous m'avez troublé en me rendant odieux aux habitants du pays, aux Cananéens et aux Phérésiens, et je n'ai qu’un petit nombre d’hommes ; ils s'assembleront contre moi, et me frapperont, et me détruiront, moi et ma maison.
\VS{31}Ils répondirent : Doit-on traiter notre sœur comme une prostituée ?
\TextTitle{[Jacob revient à Béthel pour adorer Yahweh]}
\Chap{35}
\VerseOne{}Or Dieu dit à Jacob : Lève-toi, monte à Béthel, et demeures-y ; là, tu y dresseras un autel au Dieu qui t'apparut lorsque tu fuyais Esaü, ton frère.
\VS{2}Jacob dit à sa famille, et à tous ceux qui étaient avec lui : Otez les dieux des étrangers qui sont au milieu de vous, purifiez-vous, et changez de vêtements\FTNT{Jos. 24:23.}.
\VS{3}Levons-nous et montons à Béthel ; là je dresserai un autel au Dieu qui m'a exaucé dans le jour de ma détresse, et qui a été avec moi dans le chemin où j'ai marché.
\VS{4}Alors ils donnèrent à Jacob tous les dieux des étrangers qui étaient entre leurs mains, et les anneaux qui étaient à leurs oreilles, et il les cacha sous un térébinthe qui est près de Sichem.
\VS{5}Puis ils partirent. Et Dieu frappa de terreur les villes qui les entouraient, et l’on ne poursuivit point les fils de Jacob.
\VS{6}Ainsi Jacob, et tout le peuple qui était avec lui, arrivèrent à Luz, qui est Béthel, dans le pays de Canaan.
\VS{7}Il bâtit là un autel, et il appela ce lieu El-Béthel (le Dieu Puissant de Béthel) ; car c’est là que Dieu s’était révélé à lui lorsqu’il fuyait son frère.
\VS{8}Débora,  nourrice de Rebecca, mourut ; et elle fut ensevelie au-dessous de Béthel sous un chêne, auquel on donna le nom d’Allon-Bacuth (chêne des pleurs).
\VS{9}Dieu apparut encore à Jacob, après son retour de Paddan-Aram, et il le bénit\FTNT{Os. 12:5.}.
\VS{10}Dieu lui dit : Ton nom est Jacob, mais tu ne seras plus appelé Jacob, car ton nom sera Israël. Et il lui donna le nom d’Israël.
\VS{11}Dieu lui dit aussi : Je suis le Dieu Fort, Tout-Puissant. Sois fécond et multiplie : Une nation et une multitude de nations naîtront de toi, et des rois sortiront de tes reins.
\VS{12}Je te donnerai le pays que j'ai donné à Abraham et à Isaac, et je le donnerai à ta postérité après toi.
\VS{13}Dieu s’éleva au-dessus de lui dans le lieu où il lui avait parlé.
\VS{14}Et Jacob dressa un monument dans le lieu où Dieu lui avait parlé, à savoir un monument de pierre, et il fit dessus une aspersion et y versa de l'huile.
\VS{15}Jacob donna le nom de Béthel au lieu où Dieu lui avait parlé.
\VS{16}Puis ils partirent de Béthel, et il y avait encore une certaine distance jusqu’à Ephrata (lieu de la fécondité) lorsque Rachel accoucha. Elle eut un accouchement difficile ;
\VS{17}et comme elle avait beaucoup de peine à accoucher, la sage-femme lui dit : Ne crains point, car tu as encore un fils.
\VS{18}Et comme elle rendait l'âme, car elle était mourante, elle lui donna le nom de Ben-Oni\FTNT{Ben-Oni:« fils de ma douleur ».}, mais son père l’appela Benjamin\FTNT{Benjamin:« fils de ma main droite », « fils de félicité ».}.
\VS{19}C'est ainsi que mourut Rachel, et elle fut ensevelie sur le chemin d'Ephrata, qui est Bethléhem.
\VS{20}Jacob dressa un monument sur son sépulcre. C'est le monument du sépulcre de Rachel qui subsiste encore aujourd'hui.
\VS{21}Puis Israël partit et dressa ses tentes au-delà de Migdal-Eder.
\VS{22}Pendant qu’Israël habitait dans ce pays, Ruben alla coucher avec Bilha, concubine de son père.  Et Israël l'apprit. Or Jacob avait douze fils.
\VS{23}Les fils de Léa étaient Ruben, premier-né de Jacob, Siméon, Lévi, Juda, Issacar, et Zabulon.
\VS{24}Les fils de Rachel : Joseph et Benjamin.
\VS{25}Les fils de Bilha, servante de Rachel : Dan et Nephthali.
\VS{26}Les fils de Zilpa, servante de Léa : Gad et Aser. Ce sont là les enfants de Jacob qui lui naquirent à Paddan-Aram.
\TextTitle{[Jacob revient vers son père Isaac avant sa mort]}
\VS{27}Jacob arriva auprès d’Isaac, son père, à la plaine de Mamré, à Kirjath-Arba, qui est Hébron, où Abraham et Isaac avaient séjourné comme étrangers.
\VS{28}Les jours d’Isaac furent de cent quatre-vingts ans.
\VS{29}Isaac expira et mourut, et fut recueilli auprès de son peuple, âgé et rassasié de jours ; et Esaü et Jacob ses fils l'ensevelirent.
\TextTitle{[Postérité d'Esaü (Edom)]}
\Chap{36}
\VerseOne{}Et Voici la postérité d'Esaü, qui est Edom.
\VS{2}Esaü prit ses femmes parmi les filles de Canaan, à savoir Ada, fille d'Elon, le Héthien, Oholibama, fille d’Ana, petite- fille de Tsibeon, le Hévien.
\VS{3}Il prit aussi Basmath, fille d'Ismaël, sœur de Nebajoth.
\VS{4}Ada enfanta à Esaü Eliphaz ; et Basmath enfanta Réuel.
\VS{5}Et Oholibama enfanta Jéusch, Jaelam et Koré. Ce sont là les enfants d'Esaü qui lui naquirent dans le  pays de Canaan.
\VS{6}Esaü prit ses femmes, ses fils et ses filles, et toutes les personnes de sa maison, tous ses troupeaux, ses bêtes, et tout le bien qu'il avait acquis dans le pays de Canaan, et il s'en alla dans un autre pays, loin de Jacob, son frère.
\VS{7}Car leurs richesses étaient si grandes qu'ils n'auraient pas pu demeurer ensemble ; et le pays où ils séjournaient comme étrangers ne pouvait plus les contenir à cause de leurs troupeaux.
\VS{8}Ainsi Esaü habita dans la montagne de Séir ; Esaü est Edom.
\VS{9}Voici la postérité d'Esaü, père d'Edom, dans la montagne de Séir.
\VS{10}Voici les noms des fils d'Esaü : Eliphaz fils d’Ada, femme d'Esaü ; Réuel, fils de Basmath, femme d'Esaü.
\VS{11}Les fils d'Eliphaz furent : Théman, Omar, Tsepho, Gaetham et Kenaz.
\VS{12}Et Timna était la concubine d'Eliphaz, fils d'Esaü, et elle enfanta à Eliphaz Amalek. Ce sont là les fils d’Ada, femme d'Esaü.
\VS{13}Voici les fils de Réuel : Nahath, Zérach, Schamma et Mizza. Ce sont là les fils de Basmath, femme d'Esaü.
\VS{14}Voici les fils d'Oholibama, fille d’Ada, petite fille de Tsibeon, femme d'Esaü ; elle enfanta à Esaü Jéusch, Jaelam et Koré.
\VS{15}Voici les chefs des fils d'Esaü. Voici les fils d'Eliphaz, premier-né d'Esaü, le chef Théman, le chef Omar, le chef Tsepho, le chef Kenaz,
\VS{16}le chef Koré, le chef Gaetham, le chef Amalek. Ce sont là les chefs d'Eliphaz dans le  pays d'Edom. Ce sont les fils d’Ada.
\VS{17}Voici les fils de Réuel, fils d'Esaü : le chef Nahath, le chef Zérach, le chef Schamma, et le chef Mizza. Ce sont là les chefs sortis de Réuel, dans le pays d'Edom.  Ce sont là les fils de Basmath, femme d'Esaü.
\VS{18}Voici les fils d'Oholibama, femme d'Esaü : Le chef Jéusch, le chef Jaelam, le chef Koré. Ce sont là les chefs sortis d'Oholibama, fille d’Ana, femme d'Esaü.
\VS{19}Ce sont là les fils d'Esaü, qui est Edom, et ce sont là leurs chefs.
\VS{20}Voici les fils de Séir, le Horien, qui avaient habité dans le pays : Lothan, Schobal, Tsibeon, Ana,
\VS{21}Dischon, Etser, et Dischan. Ce sont là les chefs des Horiens, fils de Séir, dans le pays d'Edom.
\VS{22}Les fils de Lothan furent Hori et Héman.  Et Thimna était sœur de Lothan.
\VS{23}Voici les fils de Schobal : Alvan, Manahath, Ebal, Schepho et Onam.
\VS{24}Voici les fils de Tsibeon : Ajja et Ana. C’est cet Ana qui trouva les sources chaudes dans le désert, quand il faisait paître les ânes de Tsibeon, son père.
\VS{25}Voici les fils d’Ana : Dischon, et Oholibama, fille d’Ana.
\VS{26}Voici les fils de Dischon : Hemdan, Eschban, Jithran et Keran.
\VS{27}Voici les fils d'Etser : Bilhan, Zaavan et Akan.
\VS{28}Voci les fils de Dischan : Huts et Aran.
\VS{29}Voici les chefs des Horiens : Le chef Lothan, le chef Schobal, le chef Tsibeon, le chef Ana.
\VS{30}Le chef Dischon, le chef Etser, le chef Dischan. Ce sont là les chefs des Horiens, les chefs qu’ils établirent dans le pays de Séir.
\VS{31}Voici les rois qui ont régné dans le pays d'Edom, avant qu’un roi règne sur les enfants d'Israël.
\VS{32}Béla, fils de Béor, régna sur Edom, et le nom de sa ville était Dinhaba.
\VS{33}Béla mourut, et Jobab, fils de Zérach de Botsra, régna à sa place.
\VS{34}Jobab mourut, et Huscham, du pays des Thémanites, régna à sa place.
\VS{35}Huscham mourut, et Hadad, fils de Bédad, régna à sa place. C’est lui qui frappa Madian dans le territoire de Moab ; et le nom de sa ville était Avith.
\VS{36}Hadad mourut, et Samla, de Masréka, régna à sa place.
\VS{37}Samla mourut, et Saül de Réhoboth sur le fleuve, régna à sa place.
\VS{38}Saül mourut, et Baal-Hanan, fils d’Acbor, régna à sa place.
\VS{39}Baal-Hanan, fils de Hacbor mourut, et Hadar régna à sa place. Le nom de sa ville était Pau ; et le nom de sa femme Mehéthabeel, fille de Mathred, petite fille de Mézahab.
\VS{40}Voici les noms des chefs d'Esaü selon leurs familles, selon leurs territoires, et d’après leurs noms : Le chef Thimna, le chef Alva, le chef Jétheth,
\VS{41}le chef Oholibama, le chef Ela, le chef Pinon,
\VS{42}Le chef Kenaz, le chef Théman, le chef Mibtsar,
\VS{43}le chef Magdiel, et le chef Iram. Ce sont là les chefs d'Edom, selon leurs habitations dans le pays qu’ils possédaient. C'est Esaü le père d'Edom.
\TextTitle{[Jacob aime Joseph plus que ses autres fils]}
\Chap{37}
\VerseOne{}Or Jacob demeura dans le pays de Canaan, pays où avait séjourné son père comme étranger.
\VS{2}Voici la postérité de Jacob. Joseph, âgé de dix-sept ans, faisait paître le troupeau avec ses frères ; et il était jeune garçon auprès des fils de Bilha et des fils de Zilpa, femmes de son père. Et Joseph rapportait à leur père leurs mauvais propos.
\VS{3}Or Israël aimait Joseph plus que tous ses autres fils, parce qu'il l'avait eu dans sa vieillesse, et il lui fit une tunique de plusieurs couleurs.
\VS{4}Ses frères voyant que leur père l'aimait plus qu'eux tous, le haïssaient et ne pouvaient lui parler paisiblement.
\VS{5}Joseph eut un songe et il raconta à ses frères ; et ils le haïrent encore davantage.
\VS{6}Il leur dit donc : Ecoutez, je vous prie, le songe que j'ai eu.
\VS{7}Voici, nous étions à lier des gerbes au milieu d'un champ ; et voici, ma gerbe se leva et se tint droite ; et voici, vos gerbes l’entourèrent et se prosternèrent devant elle.
\TextTitle{[Joseph haï par ses frères]}
\VS{8}Alors ses frères lui dirent : Régnerais-tu sur nous ? Et dominerais-tu sur nous ? Et ils le haïrent encore plus pour ses songes et pour ses paroles.
\VS{9}Il eut encore un autre songe, et il le raconta à ses frères, en disant : Voici, j'ai eu encore un songe ; et voici, le soleil, la lune et onze étoiles se prosternaient devant moi\FTNT{Ap. 12:1.}.
\VS{10}Il le raconta à son père et à ses frères. Son père le réprimanda et lui dit : Que veut dire ce songe que tu as eu ? Faut-il que nous venions moi, ta mère, et tes frères, nous prosterner à terre devant toi ?
\VS{11}Ses frères eurent de l'envie contre lui, mais son père garda ses discours\FTNT{Ac. 7:9.}.
\VS{12}Les frères de Joseph s'en allèrent paître les troupeaux de leur père à Sichem.
\VS{13}Israël dit à Joseph : Tes frères ne font-ils pas paître le troupeau à Sichem ? Viens, que je t'envoie vers eux ; et il lui répondit : Me voici.
\VS{14}Israël lui dit : Va maintenant, vois si tes frères se portent bien, et si le troupeau est en bon état, et rapporte-le-moi. Ainsi il l'envoya de la vallée d’Hébron, et il alla jusqu'à Sichem.
\VS{15}Un homme le rencontra, comme il errait dans les champs ; et cet homme le questionna et lui dit : Que cherches-tu ?
\VS{16}Joseph répondit : Je cherche mes frères ; je te prie, dis-moi où ils font paître leur troupeau.
\VS{17}Et l'homme dit : Ils sont partis d'ici, et je les ai entendus dire : Allons à Dothan. Joseph alla après ses frères et les trouva à Dothan.
\VS{18}Ils le virent de loin ; et avant qu'il soit près d’eux, ils complotèrent contre lui pour le tuer.
\VS{19}Ils se dirent l'un à l'autre : Voici ce maître songeur qui arrive.
\TextTitle{[Jacob dans la citerne]}
\VS{20}Venez maintenant, tuons-le, et jetons-le dans l’une de ces citernes ; et nous dirons qu'une bête féroce l'a dévoré, et nous verrons ce que deviendront ses songes.
\VS{21}Mais Ruben entendit cela et le délivra de leurs mains en disant : Ne lui ôtons point la vie.
\VS{22}Ruben leur dit encore : Ne répandez point le sang ; jetez-le dans cette citerne qui est au désert, mais ne mettez point la main sur lui. C'était pour le délivrer de leurs mains et le renvoyer à son père.
\VS{23}Lorsque Joseph fut arrivé auprès de ses frères, ils le dépouillèrent de sa tunique, de cette tunique de plusieurs couleurs qui était sur lui.
\VS{24}Ils le prirent et le jetèrent dans la citerne.  Cette citerne était vide, il n'y avait point d'eau.
\VS{25}Ensuite, ils s'assirent pour manger du pain ; et levant les yeux, ils virent une caravane d'Ismaélites qui passait et qui venait de Galaad ; et leurs chameaux étaient chargés d’aromates, du baume et de la myrrhe, qu’ils transportaient en Egypte.
\VS{26}Et Juda dit à ses frères : Que gagnerons-nous à tuer notre frère et à cacher son sang ?
\VS{27}Venez, vendons-le à ces Ismaélites, et ne mettons point notre main sur lui, car il est notre frère, notre chair ; et ses frères lui obéirent.
\TextTitle{[Jacob vendu à des marchands et emmené en Egypte]}
\VS{28}Et comme les marchands Madianites passaient, ils tirèrent et firent remonter Joseph de la citerne, et le vendirent pour vingt pièces d'argent aux Ismaélites, qui emmenèrent Joseph en Egypte\FTNT{Ps. 105:17.}.
\VS{29}Puis Ruben revint à la citerne, et voici, Joseph n'était plus dans la citerne. Alors il déchira ses vêtements.
\VS{30}Il retourna vers ses frères et leur dit : L'enfant n’y est plus ! Et moi ! Moi ! Où irai-je ?
\VS{31}Ils prirent la tunique de Joseph et tuèrent un bouc d'entre les chèvres, ils plongèrent la tunique dans le sang.
\VS{32}Puis ils envoyèrent et firent porter à leur père la tunique de plusieurs couleurs, en lui disant : Voici ce que avons trouvé ! Reconnais maintenant si c'est la tunique de ton fils ou non.
\VS{33}Jacob la reconnut, et dit : C'est la tunique de mon fils ! Une bête féroce l'a dévoré ! Certainement Joseph a été déchiré !
\VS{34}Et Jacob déchira ses vêtements, il mit un sac sur ses reins, et il porta le deuil de son fils durant plusieurs jours.
\VS{35}Tous ses fils et toutes ses filles vinrent pour le consoler, mais il rejeta toute consolation. Il disait : C’est en pleurant que je descendrai vers mon fils dans le scheol ! C'est ainsi que son père le pleurait.
\VS{36}Les Madianites le vendirent en Egypte à Potiphar, eunuque de Pharaon, chef des gardes.
\TextTitle{[Péché de Juda]}
\Chap{38}
\VerseOne{}Il arriva qu’en ce temps-là, Juda s’éloigna de ses frères et se retira vers un homme d’Adullam, nommé Hira.
\VS{2}Là, Juda vit la fille d'un Cananéen, nommé Schua, il la prit pour femme et alla vers elle.
\VS{3}Elle conçut et enfanta un fils qu’elle appela Er.
\VS{4}Elle conçut encore et enfanta un fils qu’elle appela Onan.
\VS{5}Elle enfanta de nouveau un fils qu’elle appela Schéla. Juda était à Czib quand elle l’enfanta.
\VS{6}Juda prit une femme pour Er, son premier-né, une femme nommée Tamar.
\VS{7}Mais Er le premier-né de Juda était méchant devant Yahweh, et Yahweh le fit mourir\FTNT{No. 26:19.}.
\VS{8}Alors Juda dit à Onan : Va vers la femme de ton frère, et prends-la pour femme, comme tu es son beau-frère, et suscite des enfants à ton frère\FTNT{Lé. 25:25  ; Lé. 25:48 ; Voir commentaire en Ru. 2:20.}.
\VS{9}Mais Onan, sachant que les enfants ne seraient pas à lui, se souillait à terre lorsqu’il allait vers la femme de son frère, afin de ne pas donner de postérité à son frère.
\VS{10}Ce qu'il faisait déplut à Yahweh, c'est pourquoi il le fit aussi mourir.
\VS{11}Et Juda dit à Tamar, sa belle-fille : Demeure veuve dans la maison de ton père, jusqu'à ce que Schéla, mon fils, soit grand ; car il dit : Il faut prendre garde qu'il ne meure comme ses frères. Ainsi Tamar s'en alla et demeura dans la maison de son père.
\VS{12}Et après plusieurs jours, la fille de Schua, femme de Juda, mourut ; lorsque Juda fut consolé, il monta vers ceux qui tondaient ses brebis à Thimna, avec Hira, l’Adullamite, son ami intime.
\VS{13}On en informa Tamar et on lui dit : Voici, ton beau-père monte à Thimna pour tondre ses brebis.
\VS{14}Alors elle ôta ses habits de veuve, se couvrit d'un voile, et s'enveloppa, et elle s’assit à l’entrée d’Enaïm, sur le chemin de Thimna ; car elle voyait que Schéla était devenu grand et qu’elle ne lui était point donnée pour femme.
\VS{15}Et quand Juda la vit, il s'imagina que c'était une prostituée, car elle avait couvert son visage.
\VS{16}Il l’aborda sur le chemin et lui dit : Permets, je te prie, que je vienne vers toi ; car il ne savait pas que c’était sa belle-fille. Et elle répondit : Que me donneras-tu pour venir vers moi ?
\VS{17}Il répondit : Je t'enverrai un chevreau d'entre les chèvres du troupeau. Elle répondit : Me donneras-tu un gage jusqu'à ce que tu l'envoies ?
\VS{18}Il répondit : Quel gage te donnerai-je ? Et elle répondit : Ton cachet, ton cordon, et ton bâton que tu as à la main. Et il les lui donna. Il alla vers elle, et elle devint enceinte de lui.
\VS{19}Puis elle se leva et s'en alla ; elle ôta son voile et remit ses habits de veuve.
\VS{20}Juda envoya un chevreau d'entre ses chèvres par son ami intime l’Adullamite, pour qu'il retire le gage de la main de la femme, mais il ne la trouva point.
\VS{21}Il interrogea les hommes du lieu où elle avait été, en disant : Où est cette prostituée qui était à Enaïm, sur le chemin ? Ils répondirent : Il n'y a point eu ici de prostituée.
\VS{22}Il retourna auprès de Juda et lui dit : Je ne l'ai point trouvée ; et même les gens du lieu m'ont dit : Il n'y a point eu ici de prostituée.
\VS{23}Juda dit : Qu'elle garde le gage, il ne faut pas nous faire mépriser. Voici, j'ai envoyé ce chevreau, mais tu ne l'as point trouvée.
\VS{24}Environ trois mois après, on fit un rapport à Juda, en disant : Tamar, ta belle-fille, a commis un adultère, et voici elle est même enceinte. Et Juda dit : Faites-la sortir, et qu'elle soit brûlée.
\VS{25}Comme on la faisait sortir, elle envoya dire à son beau-père : Je suis enceinte de l'homme à qui ces choses appartiennent. Elle dit aussi : Reconnais, je te prie, à qui est ce cachet, ce cordon, et ce bâton.
\VS{26}Alors Juda les reconnut et il dit : Elle est plus juste que moi, parce que je ne l'ai point donnée à Schéla, mon fils ; et il ne la connut plus.
\VS{27}Quand elle fut au moment d'accoucher, voici, des jumeaux étaient dans son ventre.
\VS{28}Et pendant qu’elle accouchait, il y en eut un qui présenta la main ; la sage-femme la prit et y attacha un fil cramoisi, en disant : Celui-ci sort le premier.
\VS{29}Mais il retira la main, et son frère sortit. Alors la sage-femme dit : Quelle brèche tu as faite ! Et elle lui donna le nom de Pérets.
\VS{30}Ensuite sortit son frère, qui avait à la main le fil cramoisi ; et on lui donna le nom de Zérach.
\TextTitle{[Joseph fidèle à Yahweh devant la tentation]}
\Chap{39}
\VerseOne{}Or, quand on fit descendre Joseph en Egypte, Potiphar, eunuque de Pharaon, chef des gardes, Egyptien, l'acheta de la main des Ismaélites qui l'y avaient amené.
\VS{2}Yahweh était avec Joseph ; et il prospéra, et demeura dans la maison de son maître,  l’Egyptien.
\VS{3}Son maître vit que Yahweh était avec lui, et que Yahweh faisait prospérer entre ses mains tout ce qu'il faisait.
\VS{4}C'est pourquoi Joseph trouva grâce aux yeux de son maître, qui l’employa à son service. Et son maître l'établit sur sa maison, et lui remit entre les mains tout ce qui lui appartenait.
\VS{5}Dès que Potiphar l’eut établi sur sa maison et sur tout ce qu’il possédait, Yahweh bénit la maison de l’Egyptien, à cause de Joseph ; et la bénédiction de Yahweh fut sur tout ce qui lui appartenait, soit à la maison, soit aux champs.
\VS{6}Il abandonna aux mains de Joseph tout ce qui lui appartenait, et il n’avait avec lui d’autre soin que celui de prendre sa nourriture. Or Joseph était beau de  taille et beau de figure.
\VS{7}Après ces choses, il arriva que la femme de son maître porta les yeux sur Joseph, et elle lui dit : Couche avec moi\FTNT{Pr. 7:9-13.} !
\VS{8}Mais il le refusa, et dit à la femme de son maître : Voici, mon maître ne prend avec moi connaissance de rien dans la maison, et il a remis entre mes mains tout ce qui lui appartient.
\VS{9}Il n'y a personne dans cette maison qui soit plus grand que moi, et il ne m'a rien interdit excepté toi, parce que tu es sa femme ; et comment ferais-je un si grand mal et pécherais-je contre Dieu ?
\VS{10}Quoiqu’elle parlât tous les jours à Joseph, il refusa de coucher auprès d’elle, d’être avec elle.
\VS{11}Un jour qu'il était entré dans la maison pour faire son ouvrage, et qu'il n'y avait là aucun des gens dans la maison,
\VS{12}elle le saisit par son vêtement et lui dit : Couche avec moi ! Mais il laissa son vêtement entre ses mains, s'enfuit, et sortit dehors\FTNT{1 Co. 6:18.}.
\TextTitle{[Fausse accusation contre Joseph]}
\VS{13}Et lorsqu'elle vit qu'il lui avait laissé son vêtement entre les mains, et qu'il s'était enfui dehors,
\VS{14}elle appela les gens de sa maison, et leur parla en disant : Voyez, on nous a amené un Hébreu pour se moquer de nous.  Cet homme est venu vers moi pour coucher avec moi ; mais j'ai crié à haute voix.
\VS{15}Et dès qu’il a entendu que j’élevais la voix et que je  criais, il a laissé son vêtement à côté de moi, et s’est enfui dehors.
\VS{16}Et elle garda le vêtement de Joseph jusqu'à ce que son maître rentre à la maison.
\VS{17}Alors elle lui parla en ces mêmes termes et dit : Le serviteur Hébreu que tu nous as amené est venu vers moi pour se moquer de moi.
\VS{18}Mais comme j'ai élevé ma voix et que j'ai crié, il a laissé son vêtement à côté de moi et s'est enfui.
\VS{19}Et dès que le maître de Joseph eut entendu les paroles de sa femme qui lui disait : Ton serviteur m'a fait ce que je t'ai dit, sa colère s'enflamma.
\VS{20}Et le maître de Joseph le prit et le mit dans une étroite prison ; dans l'endroit où les prisonniers du roi étaient enfermés, et il fut là en prison.
\VS{21}Mais Yahweh fut avec Joseph ; il étendit sa bonté sur lui et lui fit trouver grâce auprès du chef de la prison.
\VS{22}Et le chef de la prison mit entre les mains de Joseph tous les prisonniers qui étaient dans la prison, et tout ce qu'il y avait à faire, il le faisait.
\VS{23}Le chef de la prison ne prenait aucune connaissance de ce que Joseph avait en main, parce que Yahweh était avec lui. Et Yahweh faisait prospérer tout ce qu'il faisait.
\TextTitle{[Joseph demeure dans en prison]}
\Chap{40}
\VerseOne{}Après ces choses, il arriva que l'échanson et le panetier du roi d'Egypte offensèrent leur maître, le roi d'Egypte.
\VS{2}Pharaon fut fort irrité contre ces deux eunuques, contre le chef des échansons, et contre le chef des panetiers.
\VS{3}Et il les fit mettre dans la maison du chef des gardes, dans la prison étroite, dans le même lieu où Joseph était enfermé.
\VS{4}Le chef des gardes les mit entre les mains de Joseph qui les servait ; et ils furent quelques jours en prison.
\VS{5}Pendant une même nuit, l’échanson et le panetier du roi d’Egypte, qui étaient enfermés dans la prison, eurent tous les deux un songe, chacun le sien, pouvant recevoir une explication distincte.
\VS{6}Joseph, étant venu le matin vers eux, les regarda ; et voici, ils étaient fort tristes.
\VS{7}Et il interrogea ces deux eunuques de Pharaon, qui étaient avec lui dans la prison de son maître, et leur dit : Pourquoi avez-vous mauvais visage aujourd'hui ?
\VS{8}Ils lui répondirent : Nous avons eu des songes, et il n'y a personne qui les interprète. Et Joseph leur dit : Les interprétations n’appartiennent-elles pas à Dieu ? Je vous prie, racontez-moi vos songes\FTNT{1 Co. 12:8-10 ; Job. 33:15.}.
\VS{9}Le chef des échansons raconta son songe à Joseph et lui dit : Dans mon songe, voici, il y avait un cep devant moi.
\VS{10}Ce cep avait trois sarments. Quand il eut poussé, sa fleur se développa et ses grappes donnèrent des raisins mûrs.
\VS{11}La coupe de Pharaon était dans ma main. Je pris les raisins, je les pressai dans la coupe de Pharaon, et je mis la coupe dans la main de Pharaon.
\VS{12}Joseph lui dit : Voici son interprétation : Les trois sarments sont trois jours.
\VS{13}Dans trois jours Pharaon élèvera ta tête et te rétablira dans ta charge, et tu mettras la coupe dans sa main, comme tu le faisais auparavant, lorsque tu étais son échanson.
\VS{14}Mais souviens-toi de moi quand tu  seras heureux, et use de bonté envers moi je te prie ; fais mention de moi à Pharaon, afin  qu’il me fasse sortir de cette maison.
\VS{15}Car certainement j'ai été enlevé du pays des Hébreux ; ici non plus je n’ai rien fait  pour  être mis en prison.
\VS{16}Le chef des panetiers, voyant que Joseph avait interprété favorablement ce songe, lui dit : Voici, il y avait aussi dans mon songe trois corbeilles de pain blanc sur ma tête.
\VS{17}Dans la corbeille la plus élevée, il y avait pour Pharaon des mets de toute espèce, cuits au four ; et les oiseaux les mangeaient dans la corbeille au-dessus de ma tête.
\VS{18}Joseph répondit et dit : Voici son interprétation : Les trois corbeilles sont trois jours.
\VS{19}Dans trois jours Pharaon enlèvera ta tête de dessus toi et te fera pendre à un bois, et les oiseaux mangeront ta chair sur toi.
\VS{20}Le troisième jour, jour de la naissance de Pharaon, il fit un festin à tous ses serviteurs ; et il éleva la tête du chef des échansons et la tête du chef des panetiers, au milieu de ses serviteurs.
\VS{21}Il rétablit le chef des échansons dans sa charge d’échanson, pour qu’il mette la coupe dans la main de Pharaon.
\VS{22}Mais il fit pendre le chef des panetiers, selon l’explication que Joseph leur avait donnée.
\VS{23}Cependant, le chef des échansons ne pensa plus à Joseph. Il l’oublia.
\TextTitle{[Songe de Pharaon]}
\Chap{41}
\VerseOne{}Mais il arriva qu’au bout de deux ans entiers, Pharaon eut un songe. Et il lui semblait qu'il était près du fleuve.
\VS{2}Et voici, sept jeunes vaches belles à voir, grasses de chair, montèrent hors du fleuve et se mirent à paître dans les  prairies.
\VS{3}Et voici sept autres jeunes vaches, laides à voir, et maigres de chair, montèrent hors du fleuve derrière les autres et se tinrent auprès des autres jeunes vaches sur le bord du fleuve.
\VS{4}Les jeunes vaches laides à voir, et maigres, mangèrent les sept jeunes vaches belles à voir, et grasses. Alors Pharaon s'éveilla.
\VS{5}Il se rendormit et il eut un second songe. Voici, sept épis gras et beaux montèrent sur une même tige.
\VS{6}Et sept épis maigres et brûlés par le vent d’orient poussèrent après eux.
\VS{7}Les épis maigres engloutirent les sept épis gras et pleins. Et Pharaon s'éveilla ; et voilà le songe.
\VS{8}Le matin, Pharaon eut l’esprit troublé, et il envoya appeler tous les magiciens et tous les sages d'Egypte, et leur raconta ses songes.  Mais personne ne put les interpréter à Pharaon.
\VS{9}Alors le chef des échansons parla à Pharaon en disant : Je rappellerai aujourd'hui le souvenir de mes fautes.
\VS{10}Lorsque Pharaon fut irrité contre ses serviteurs, et nous fit mettre, le chef des panetiers et moi, en prison, dans la maison du chef des gardes,
\VS{11}nous eûmes l’un et l’autre un songe dans une même nuit ; et chacun de nous reçut une interprétation en rapport avec le songe qu’il avait eu.
\VS{12}Il y avait là avec nous un garçon Hébreu, esclave du chef des gardes. Nous lui racontâmes nos songes, et il nous les expliqua.
\VS{13}Les choses sont arrivées comme il nous les avait interprétées ; car le roi me rétablit dans ma charge et fit pendre le chef des panetiers.
\TextTitle{[Joseph sort de prison et est établi sur l'Egypte par Pharaon]}
\VS{14}Alors Pharaon envoya appeler Joseph.  On le fit sortir en hâte de la prison ; on le rasa, et on lui fit changer de vêtements ; puis il se rendit vers Pharaon.
\VS{15}Pharaon dit à Joseph : J'ai eu un songe, et personne ne peut  l'expliquer ; or j'ai appris que tu sais expliquer les songes.
\VS{16}Joseph répondit à Pharaon en disant : Ce n’est pas moi !  C’est Dieu qui donnera une réponse concernant la paix de Pharaon.
\VS{17}Pharaon dit alors à Joseph : Dans mon songe, voici, je me tenais sur le bord du fleuve.
\VS{18}Et voici, sept vaches grasses de chair et belles d’apparence montèrent hors du fleuve et se mirent à paître dans la prairie.
\VS{19}Sept autres vaches montèrent derrière elles, maigres, fort laides d’apparence, et décharnées ; je n’en ai point vu d’aussi laides dans tout le pays d’Egypte.
\VS{20}Les vaches décharnées et laides mangèrent les sept premières vaches qui étaient grasses ;
\VS{21}elles les engloutirent dans leur ventre, sans qu’on s’aperçoive qu’elles y étaient entrées ; et leur apparence était laide comme auparavant. Et je m’éveillai.
\VS{22}Je vis encore en songe sept épis pleins et beaux, qui montèrent sur une même tige.
\VS{23}Et sept épis vides, maigres, brûlés par le vent d’orient, poussèrent après eux.
\VS{24}Les épis maigres engloutirent les sept beaux épis. Je l’ai dit aux magiciens, mais personne ne m’a donné l’explication. 25 Joseph dit à Pharaon : Ce qu’a rêvé Pharaon est une seule chose ; Dieu a fait connaître à Pharaon ce qu’il va faire.
\VS{26}Les sept vaches belles sont sept années ; et les sept épis beaux sont sept années ; c’est un seul songe.
\VS{27}Les sept vaches décharnées et laides, qui montaient derrière les premières, sont sept années ; et les sept épis vides, brûlés par le vent d’orient, seront sept années de famine.
\VS{28}Ainsi, comme je viens de le dire à Pharaon, Dieu a fait connaître à Pharaon ce qu’il va faire.
\VS{29}Voici, il y aura sept années de grande abondance dans tout le pays d’Egypte.
\VS{30}Sept années de famine viendront après elles ; et l’on oubliera toute cette abondance au pays d’Egypte, et la famine consumera le pays.
\VS{31}Cette famine qui suivra sera si forte qu’on ne s’apercevra plus de l’abondance dans le pays.
\VS{32}Si Pharaon a vu le songe se répéter une seconde fois, c’est que la chose est arrêtée de la part de Dieu, et que Dieu se hâtera de l’exécuter.
\VS{33}Maintenant que Pharaon choisisse un homme intelligent et sage, et qu'il l'établisse sur le pays d'Egypte.
\VS{34}Que Pharaon établisse et institue des commissaires sur le pays, et qu'ils prennent la cinquième partie du revenu du pays d'Egypte durant les sept années d'abondance.
\VS{35}Qu’ils rassemblent tous les produits de ces bonnes années  qui viennent ; qu’ils fassent sous l’autorité de Pharaon des amas de blé, des approvisionnements dans les villes, et qu’ils en aient la garde.
\VS{36}Ces provisions seront en réserve pour le pays durant les sept années de famine qui seront dans le  pays d'Egypte, afin que le pays ne soit pas consumé par la famine.
\VS{37}Ces paroles plurent à Pharaon et à tous ses serviteurs\FTNT{Ac. 7:10.}.
\VS{38}Et Pharaon dit à ses serviteurs : Trouverions-nous un homme semblable à celui-ci, qui a l'Esprit de Dieu ?
\VS{39}Et Pharaon dit à Joseph : Puisque Dieu t'a fait connaître toutes ces choses, il n'y a personne qui soit aussi intelligent et aussi sage que toi.
\VS{40}C’est toi qui seras sur ma maison, et tout mon peuple obéira à tes ordres ; je serai seulement plus grand que toi par le trône.
\VS{41}Pharaon dit encore à Joseph : Regarde, je t’établis sur tout le pays d'Egypte.
\VS{42}Alors Pharaon ôta son anneau de sa main et le mit à la main de Joseph ; il le fit revêtir d'habits de fin lin et lui mit un collier d'or au cou.
\VS{43}Il le fit monter sur le char qui suivait le sien, et on criait devant lui : A genoux ! Et il l'établit sur tout le pays d'Egypte.
\VS{44}Et Pharaon dit à Joseph : Je suis Pharaon ! Et sans toi nul ne lèvera la main ni le pied dans tout le pays d'Egypte.
\TextTitle{[Joseph épouse une égyptienne]}
\VS{45}Pharaon appela Joseph du nom de Tsaphnath-Paenéach ; et il lui donna pour femme Asnath, fille de Poti-Phéra, prêtre d'On. Et Joseph alla visiter le pays d'Egypte.
\VS{46}Joseph était âgé de trente ans lorsqu’il se présenta devant Pharaon, roi d'Egypte ; et il quitta Pharaon et parcourut tout le pays d'Egypte.
\VS{47}Et la terre rapporta très abondamment pendant les sept années de fertilité.
\VS{48}Joseph rassembla tous les produits de ces sept années dans le pays d’Egypte ; il fit des approvisionnements dans les villes, mettant dans l’intérieur de chaque ville les productions des champs d’alentour.
\VS{49}Ainsi Joseph amassa une grande quantité de blé, comme le sable de la mer ; tellement qu'on cessa de le compter, parce qu’il n’y avait plus de nombre.
\VS{50}Avant les années de famine, il naquit à Joseph deux fils, que lui enfanta Asnath, fille de Poti-Phéra, prêtre  d'On.
\VS{51}Joseph donna au premier-né le nom de Manassé, parce que, dit-il, Dieu m'a fait oublier toute ma peine et toute la maison de mon père.
\VS{52}Et il donna au second le nom d’Ephraïm, parce que, dit-il, Dieu m'a fait fructifier dans le  pays de mon affliction.
\VS{53}Alors finirent les sept années de l'abondance qui avaient été dans le pays d'Egypte.
\VS{54}Et les sept années de la famine commencèrent à venir comme Joseph l'avait prédit. Et la famine fut dans tous les pays ; mais il y avait du pain dans tout le pays d'Egypte.
\VS{55}Ensuite tout le pays d'Egypte fut affamé, et le peuple cria à Pharaon pour avoir du pain. Et Pharaon répondit à tous les Egyptiens : Allez vers Joseph, et faites ce qu'il vous dira.
\VS{56}La famine régnait dans tout le pays. Joseph ouvrit tous les lieux d’approvisionnements et vendit du blé aux Egyptiens. La famine augmentait dans le pays d’Egypte.
\VS{57}On venait de tous les pays jusqu’en Egypte, pour acheter du blé  auprès de Joseph ; car la famine était fort grande sur toute la terre.
\TextTitle{[Les frères de Joseph viennent acheter des vivres en Egypte]}
\Chap{42}
\VerseOne{}Et Jacob, voyant qu'il y avait du blé à vendre en Egypte, dit à ses fils : Pourquoi vous regardez-vous les uns les autres ?
\VS{2}Il leur dit aussi : Voici, j'ai appris qu'il y a du blé à vendre en Egypte, descendez-y pour nous en acheter là, afin que nous vivions, et que nous ne mourions point.
\VS{3}Alors les frères de Joseph descendirent pour acheter du blé en Egypte.
\VS{4}Mais Jacob n'envoya point Benjamin, frère de Joseph, avec ses frères ; car il disait : Il faut prendre garde qu’un malheur ne lui arrive.
\VS{5}Ainsi les fils d'Israël allèrent en Egypte pour acheter du blé avec ceux qui y allaient, car la famine était dans le pays de Canaan.
\TextTitle{[Joseph met ses frères à l'épreuve]}
\VS{6}Joseph commandait dans le pays, et c’était lui qui vendait le blé à tous les peuples de la terre. Les frères de Joseph vinrent et se prosternèrent devant lui la face contre terre.
\VS{7}Joseph vit ses frères et les reconnut ; mais il feignit d’être un étranger pour eux, et il leur parla rudement, en leur disant : D'où venez-vous ? Et ils répondirent : Du pays de Canaan, pour acheter des vivres.
\VS{8}Joseph reconnut ses frères, mais eux ne le connurent point.
\VS{9}Alors Joseph se souvint des songes qu'il avait eus à leur sujet et leur dit : Vous êtes des espions, vous êtes venus pour observer les lieux faibles du pays.
\VS{10}Et ils lui répondirent : Non, mon seigneur, mais tes serviteurs sont venus pour acheter des vivres.
\VS{11}Nous sommes tous enfants d'un même homme, nous sommes des gens de bien ; tes serviteurs ne sont pas des espions.
\VS{12}Et il leur dit : Nullement ; vous êtes venus pour observer les lieux faibles du pays.
\VS{13}Et ils répondirent : Nous, tes serviteurs, étions douze frères, fils d'un même homme, dans le pays de Canaan. Et voici, le plus jeune est aujourd'hui avec notre père, et l'un n'est plus.
\VS{14}Joseph leur dit : C'est ce que je vous disais, vous êtes des espions.
\VS{15}Voici comment vous serez éprouvés : Par la vie de Pharaon ! Vous ne sortirez pas d'ici que votre jeune frère ne soit venu ici.
\VS{16}Envoyez l’un de vous et qu’il amène votre frère ; et  vous, restez prisonniers. Vos paroles seront éprouvées et je saurai si vous avez dit la vérité. Autrement, par la vie de Pharaon ! Vous êtes des espions.
\VS{17}Et il les mit tous ensemble en prison pendant trois jours.
\VS{18}Le troisième jour, Joseph leur dit : Faites ceci, et vous vivrez. Je crains Dieu !
\VS{19}Si vous êtes sincères, que l'un de vos frères reste enfermé dans votre prison ; et vous, partez et emportez du blé pour nourrir vos familles.
\VS{20}Puis amenez-moi votre jeune frère afin que vos paroles soient éprouvées, et vous ne mourrez point ; et ils firent ainsi.
\TextTitle{[Joseph exige que Benjamen viennent en Egypte]}
\VS{21}Et ils se dirent alors l'un à l'autre : Nous sommes certainement coupables à l'égard de notre frère ; car nous avons vu l'angoisse de son âme quand il nous demandait grâce, et nous ne l'avons point écouté ; c'est pour cela que cette détresse nous est arrivée.
\VS{22}Ruben leur répondit en disant : Ne vous disais-je pas : Ne commettez point ce péché contre l'enfant ? Et vous ne m’avez point écouté ; et voici que son sang vous est redemandé.
\VS{23}Ils ne savaient pas que Joseph les comprenait, parce qu'il se servait d’un interprète pour leur parler.
\VS{24}Il s’éloigna d’eux pour pleurer. Et il revint, leur parla ; puis il prit parmi eux Siméon, et le fit enchaîner sous leurs yeux.
\VS{25}Et Joseph ordonna qu'on remplisse leurs sacs de blé, et qu'on remette l'argent de chacun d’eux dans son sac, et qu'on leur donne de la provision pour la route ; et cela fut fait ainsi.
\VS{26}Ils chargèrent donc leur blé sur leurs ânes, et s'en allèrent.
\VS{27}L’un d'eux ouvrit son sac pour donner du fourrage à son âne dans l'hôtellerie ; et il vit son argent qui était à l’entrée de son sac.
\VS{28}Il dit à ses frères : Mon argent m'a été rendu ; et le voici dans mon sac. Alors leur cœur fut en défaillance ; et ils furent saisis de peur, et se dirent l'un à l'autre : Qu'est-ce que Dieu nous a fait ?
\VS{29}Et étant arrivés dans le pays de Canaan, vers Jacob leur père, ils lui racontèrent toutes les choses qui leur étaient arrivées, en disant :
\VS{30}L'homme qui est le seigneur du pays, nous a parlé rudement et nous a pris pour des espions du pays.
\VS{31}Mais nous lui avons répondu : Nous sommes sincères, nous ne sommes point des espions.
\VS{32}Nous étions douze frères, fils de notre père ; l'un n'est plus, et le plus jeune est aujourd'hui avec notre père dans le pays de Canaan.
\VS{33}Et cet homme, qui est le seigneur  du pays, nous a dit : A ceci je connaîtrai que vous êtes sincères : Laissez-moi l'un de vos frères, et prenez de quoi nourrir vos familles et partez.
\VS{34}Puis amenez-moi votre jeune frère, et je saurai que vous n'êtes point des espions, que vous êtes sincères ; je vous rendrai votre frère, et vous pourrez librement trafiquer dans le  pays.
\VS{35}Lorsqu’ils vidèrent leurs sacs, voici, le paquet d’argent de chacun était dans son sac. Ils virent, eux et leur père, leurs paquets d’argent, et ils eurent peur.
\VS{36}Jacob leur père leur dit : Vous me privez de mes enfants ! Joseph n'est plus, et Siméon n'est plus, et vous prendriez Benjamin ! C’est sur moi que tout cela retombe.
\VS{37}Ruben parla à son père et lui dit : Fais mourir deux de mes fils si je ne te ramène pas Benjamin ! Remets-le entre mes mains et je te le ramènerai.
\VS{38}Jacob répondit : Mon fils ne descendra point avec vous, car son frère est mort, et il reste seul ; s’il lui arrivait un malheur dans le voyage que vous allez faire, vous feriez descendre mes cheveux blancs avec douleur dans le scheol.
\TextTitle{[Juda se porte garant de Benjamin]
\\(Ge. 37:26-28)}
\Chap{43}
\VerseOne{}Or la famine devint fort grande dans le pays.
\VS{2}Et quand  ils eurent achevé de manger le blé qu'ils avaient apporté d'Egypte, leur père leur dit : Retournez, achetez-nous un peu de vivres.
\VS{3}Juda lui répondit et lui dit : Cet homme nous a expressément déclaré, disant : Vous ne verrez point ma face, à moins que votre frère ne soit avec vous.
\VS{4}Si donc tu envoies notre frère avec nous, nous descendrons en Egypte et nous t'achèterons des vivres.
\VS{5}Mais si tu ne l'envoies pas, nous n'y descendrons point ; car cet homme nous a dit : Vous ne verrez point ma face, à moins que votre frère ne soit avec vous.
\VS{6}Et Israël dit : Pourquoi avez-vous mal agi à mon égard, en disant à cet homme que vous aviez encore un frère ?
\VS{7}Ils répondirent : Cet homme nous a interrogés sur nous et sur notre famille, en disant : Votre père vit-il encore ? N'avez-vous point de frère ? Et nous lui avons déclaré selon ce qu'il nous avait demandé ; pouvions-nous savoir qu'il dirait : Faites descendre votre frère ?
\VS{8}Juda dit à Israël, son père : Laisse venir l'enfant avec moi, afin que nous nous levions et que nous partions ; et nous vivrons et nous ne mourons point, nous, toi et nos enfants.
\VS{9}Je réponds de lui, tu le redemanderas de ma main. Si je ne te le ramène pas auprès de toi et si je ne le remets pas devant ta face, je serai coupable toute ma vie envers toi.
\VS{10}Car si nous n’avions pas tardé, certainement nous serions déjà de retour deux fois.
\VS{11}Alors Israël leur père leur dit : Si cela est ainsi, faites ceci, prenez dans vos bagages les meilleures productions du pays, pour en  porter un présent à cet homme, un peu de baume, et un peu de miel, des épices, de la myrrhe, des dattes, et des amandes.
\VS{12}Prenez avec vous de l'argent au double dans vos mains, et rapportez l’argent qu’on avait mis à l’entrée de vos sacs ; peut-être était-ce une erreur.
\VS{13}Prenez votre frère, et levez-vous, retournez vers cet homme.
\VS{14}Que le Dieu Tout-Puissant vous fasse trouver grâce devant cet homme, afin qu'il relâche votre autre frère et Benjamin ; et s'il faut que je sois privé de ces deux fils, que j'en sois privé.
\VS{15}Alors ils prirent le présent, et ayant pris de l'argent au double dans leurs mains, et Benjamin, ils se levèrent et descendirent en Egypte ; puis ils se présentèrent devant Joseph.
\VS{16}Dès que Joseph vit Benjamin avec eux, il dit à l’intendant de sa maison : Fais entrer ces gens dans la maison, tue et apprête quelques bêtes, car ils mangeront à midi avec moi.
\VS{17}Cet homme fit ce que Joseph lui avait dit ; et il conduit ces gens dans la maison de Joseph.
\VS{18}Ils eurent peur lorsqu’ils furent conduits dans la maison de Joseph, et ils dirent : Nous sommes emmenés à cause de l'argent remis l’autre fois dans nos sacs ; c’est pour se jeter sur nous, se précipiter sur nous ; c’est pour nous prendre comme esclaves et s’emparer de nos ânes.
\VS{19}Ils s’approchèrent de l’intendant de la maison de Joseph, et lui adressèrent la parole, à l’entrée de la maison.
\VS{20}Ils dirent : Pardon ! Mon seigneur, nous sommes déjà descendus une fois pour acheter des vivres.
\VS{21}Puis, quand nous arrivâmes, au lieu où nous devions passer la nuit, nous avons ouvert nos sacs ; et voici, l’argent de chacun était à l’entrée de son sac, notre argent selon son poids ;  nous le rapportons avec nous.
\VS{22}Nous avons aussi apporté d'autre argent dans nos mains pour acheter des vivres ; et nous ne savons point qui a remis notre argent dans nos sacs.
\VS{23}L’intendant leur dit : Tout va bien pour vous, ne craignez point. C’est votre Dieu, le Dieu de votre père vous a donné un trésor dans vos sacs ; votre argent est parvenu jusqu'à moi ; et il leur amena Siméon.
\VS{24}Cet homme les fit entrer dans la maison de Joseph, et leur donna de l'eau, et ils lavèrent leurs pieds ; il donna aussi à manger à leurs ânes.
\VS{25}Ils préparèrent leur présent en attendant que Joseph revienne à midi ; car ils avaient appris qu'ils mangeraient du pain chez lui.
\VS{26}Quand  Joseph fut arrivé à la maison, ils lui offrirent le présent qu'ils avaient dans leurs mains, et se prosternèrent à terre devant lui dans la maison.
\VS{27}Il leur demanda comment ils se portaient et leur dit : Votre vieux père, dont vous m'avez parlé, se porte-t-il bien ? Vit-il encore ?
\VS{28}Ils répondirent : Ton serviteur, notre père, se porte bien, il vit encore. Et ils s’inclinèrent et se prosternèrent.
\VS{29}Joseph leva les yeux, il vit Benjamin, son frère, fils de sa mère, et il dit : Est-ce là votre jeune frère dont vous m'avez parlé ? Et il ajouta : Mon fils, Dieu te fasse grâce !
\VS{30}Et Joseph se retira promptement, car ses entrailles étaient émues à la vue de son frère, et il cherchait un lieu pour pleurer ; il entra dans sa chambre et il y pleura.
\VS{31}Après s’être lavé le visage, il sortit de là, et faisant des efforts pour se contenir, il dit : Servez le pain.
\VS{32}On servit Joseph à part, et ses frères à part, et les Egyptiens qui mangeaient avec lui furent aussi servis à part, car les Egyptiens ne pouvaient manger du pain avec les Hébreux,  parce que c’est à leurs yeux une abomination.
\VS{33}Les frères de Joseph s’assirent en sa présence, le premier-né selon son droit d’aînesse, et le plus jeune selon son âge ; et ils se regardaient les uns les autres avec étonnement.
\VS{34}Joseph leur fit porter des mets qui étaient devant lui, et Benjamin en eut cinq fois plus que les autres. Ils burent et s’enivrèrent  avec lui.
\TextTitle{[Juda prend la place de Benjamin]
\\(Ge. 43:9)}\Chap{44}
\VerseOne{}Et Joseph donna un ordre à son intendant, en disant : Remplis de vivres les sacs de ces gens, autant qu'ils en pourront porter, et remets l'argent de chacun à l’entrée de son sac.
\VS{2}Tu mettras aussi ma coupe, la coupe d'argent, à l’entrée du sac du plus petit avec l'argent de son blé ; et il fit comme Joseph lui avait dit.
\VS{3}Le matin, dès qu'il fit jour, on renvoya ces hommes avec leurs ânes.
\VS{4}Ils étaient sortis de la ville, ils n’en étaient guère éloignés, lorsque Joseph dit à son intendant : Va, poursuis ces hommes, et quand tu les auras atteints, tu leur diras : Pourquoi avez-vous rendu le mal pour le bien ?
\VS{5}N'est-ce pas la coupe dont se sert mon seigneur pour boire et pour deviner ? Vous avez mal fait d’agir ainsi.
\VS{6}L’intendant les atteignit, et leur dit ces paroles.
\VS{7}Ils lui répondirent : Pourquoi mon seigneur parle-t-il ainsi ? Loin de tes serviteurs la pensée de faire pareille chose !
\VS{8}Voici, nous t'avons rapporté du pays de Canaan l'argent que nous avions trouvé à l’entrée de nos sacs, et comment aurions-nous dérobé de l'argent ou de l'or de la maison de ton maître ?
\VS{9}Que celui de tes serviteurs sur qui se trouvera la coupe meure ; et nous serons aussi esclaves de mon seigneur !
\VS{10}Il leur dit : Qu'il soit fait maintenant selon vos paroles ! Qu’il en soit ainsi ! Que celui sur qui se trouvera la coupe soit mon esclave, et vous, vous serez innocents.
\VS{11}Et ils se hâtèrent de déposer chacun son sac à terre ; et chacun ouvrit son sac.
\VS{12}L’intendant les fouilla, en commençant par le plus âgé, et finissant par le plus jeune ; et la coupe fut trouvée dans le sac de Benjamin.
\VS{13}Alors ils déchirèrent leurs vêtements, et chacun rechargea son âne, et ils retournèrent à la ville.
\VS{14}Juda et ses frères arrivèrent à la maison de Joseph, qui était encore là, et ils se jetèrent à terre devant lui.
\VS{15}Joseph leur dit : Quelle action avez-vous faite ? Ne savez-vous pas qu'un homme tel que moi ne manque pas de deviner ?
\VS{16}Juda lui répondit : Que dirons-nous à mon Seigneur ? Comment parlerons-nous ? Et comment nous justifierons-nous ? Dieu a trouvé l'iniquité de tes serviteurs ; voici, nous sommes esclaves de mon seigneur, nous, et celui entre les mains de qui la coupe a été trouvée.
\VS{17}Mais il dit : Loin de moi la pensée d’agir ainsi ! L’homme dans la main duquel la coupe a été trouvée sera mon esclave ; mais vous, remontez en paix vers votre père.
\VS{18}Alors Juda s'approcha de lui en disant : Pardon mon seigneur ! Je te prie, que ton serviteur dise un mot, je te prie aux oreilles de mon seigneur, et que ta colère ne s'enflamme point contre ton serviteur, car tu es comme Pharaon.
\VS{19}Mon seigneur interrogea ses serviteurs en disant : Avez-vous un père ou un frère ?
\VS{20}Nous avons répondu à mon seigneur : Nous avons notre père qui est âgé, et un enfant de sa vieillesse, et qui est le plus jeune d'entre nous ; son frère est mort, et celui-ci est resté le seul enfant  de sa mère ; et son père l'aime.
\VS{21}Tu as dis à tes serviteurs : Faites-le descendre vers moi, et que je le voie de mes yeux.
\VS{22}Nous avons répondu  à mon seigneur : Cet enfant ne peut quitter son père, car s'il le quitte, son père mourra.
\VS{23}Alors tu dis à tes serviteurs : Si votre petit frère ne descend avec vous, vous ne verrez plus ma face.
\VS{24}Lorsque nous sommes remontés auprès de ton serviteur, mon père, nous lui avons rapporté les paroles de mon seigneur.
\VS{25}Notre père nous a dit : Retournez, et achetez-nous un peu de vivres.
\VS{26}Nous lui avons répondu : Nous ne pouvons pas descendre ; mais si notre petit frère est avec nous, nous descendrons, car nous ne pouvons pas voir la face de cet homme, à moins que notre jeune frère ne soit avec nous.
\VS{27}Ton serviteur, mon père, nous répondit : Vous savez que ma femme m'a enfanté deux fils.
\VS{28}L’un étant sorti de chez moi, je pense qu’il a été sans doute déchiré, car je ne l’ai pas revu jusqu’à présent.
\VS{29}Si vous me prenez encore celui-ci, et qu’il lui arrive un malheur, vous ferez descendre mes cheveux blancs avec douleur dans le scheol.
\VS{30}Maintenant, si je retourne auprès de ton serviteur, mon père, sans avoir avec nous l’enfant à l’âme duquel son âme est attachée,
\VS{31}il mourra, en voyant que l’enfant n’y est pas ; et tes serviteurs feront descendre avec douleur dans le scheol les cheveux blancs de ton serviteur, notre père.
\VS{32}De plus, ton serviteur a répondu pour l'enfant, en le prenant à mon père, en disant : Si je ne te le ramène pas, je serai pour toujours coupable envers mon père.
\VS{33}Permets donc, je te prie, à ton serviteur de rester à la place de l’enfant, comme esclave de mon seigneur ; et que l’enfant remonte avec ses frères.
\VS{34}Car comment pourrai-je remonter vers mon père, si l'enfant n'est pas avec moi ? Que je ne voie point l'affliction qu'en aurait mon père !
\TextTitle{[Joseph  revèle qui il est à ses frères]}
\Chap{45}
\VerseOne{}Alors Joseph, ne pouvant plus se contenir devant tous ceux qui étaient là présents, cria : Faites sortir tout le monde ! Et il ne resta personne quand il se fit connaître à ses frères.
\VS{2}Et en pleurant, il éleva sa voix, et les Egyptiens l'entendirent, et la maison de Pharaon l'entendit aussi.
\VS{3}Et Joseph dit à ses frères : Je suis Joseph ! Mon père vit-il encore ? Mais ses frères ne pouvaient lui répondre, car ils étaient tout troublés en sa présence.
\VS{4}Joseph dit encore à ses frères : Je vous prie, approchez-vous de moi ; et ils s'approchèrent, et il leur dit : Je suis Joseph, votre frère, que vous avez vendu pour être mené en Egypte\FTNT{Ac. 7:13.}.
\VS{5}Mais maintenant ne soyez pas en peine, et n'ayez point de regret de ce que vous m'avez vendu pour être mené ici, car Dieu m'a envoyé devant vous pour la conservation de votre vie.
\VS{6}Car voici, il y a déjà deux ans que la famine est sur la terre, et il y aura encore cinq ans pendant lesquels il n'y aura ni labour ni moisson.
\VS{7}Mais Dieu m'a envoyé devant vous, pour vous faire subsister sur la terre, et vous faire vivre par une grande délivrance.
\VS{8}Maintenant donc ce n'est pas vous qui m'avez envoyé ici, mais c'est Dieu ; il m'a établi père de Pharaon, et seigneur sur toute sa maison, et gouverneur de tout le pays d'Egypte.
\VS{9}Hâtez-vous d'aller vers mon père, et dites-lui : Ainsi a dit ton fils, Joseph : Dieu m'a établi seigneur sur toute l'Egypte, descends vers moi, ne t'arrête point.
\VS{10}Et tu habiteras dans la contrée de Gosen, et tu seras près de moi, toi, tes fils, et les fils de tes fils, tes brebis, et tes bœufs, et tout ce qui est à toi.
\VS{11}Là, je te nourrirai, car il y aura encore cinq années de famine ; et ainsi tu ne périras point, toi et ta maison, et tout ce qui est à toi.
\VS{12}Et voici, vous voyez de vos yeux, et Benjamin mon frère voit aussi de ses yeux, que c'est moi qui vous parle de ma propre bouche.
\VS{13}Rapportez donc à mon père quelle est ma gloire en Egypte, et tout ce que vous avez vu ; hâtez-vous, et faites descendre ici mon père.
\VS{14}Alors il se jeta sur le cou de Benjamin, son frère, et pleura. Benjamin pleura aussi sur son cou.
\VS{15}Puis il embrassa tous ses frères et pleura sur eux ; après cela ses frères parlèrent avec lui.
\TextTitle{[Joseph pardonne à ses frères et fait venir Jacob]
\\(Ge. 43:9)}
\VS{16}Et le bruit se répandit dans la maison de Pharaon que les frères de Joseph étaient venus, ce qui plut fort à Pharaon et à ses serviteurs.
\VS{17}Alors Pharaon dit à Joseph : Dis à tes frères : Faites ceci : Chargez vos bêtes, et allez, retournez dans le pays de Canaan ;
\VS{18}et prenez votre père et vos familles, et revenez vers moi, et je vous donnerai le meilleur du pays d'Egypte ; et vous mangerez la graisse de la terre.
\VS{19}Tu as ordre de leur dire: Faites ceci : Prenez dans le pays d’Egypte des chars pour vos enfants et pour vos femmes; amenez votre père, et venez.
\VS{20}Ne regrettez point ce que vous laisserez, car ce qu’il y a de meilleur dans tout le pays d’Egypte sera pour vous.
\VS{21}Et les fils d'Israël firent ainsi. Et Joseph leur donna des chars selon l'ordre de Pharaon ; il leur donna aussi de la provision pour la route.
\VS{22}Il leur donna à chacun des vêtements de rechange ; et il donna à Benjamin trois cents pièces d'argent et cinq vêtements de rechange.
\VS{23}Il envoya aussi à son père dix ânes chargés des plus excellentes choses qu'il y avait en Egypte, et dix ânesses portant du blé, du pain, et des vivres à son père pour la route.
\VS{24}Il renvoya donc ses frères, et ils partirent ; et il leur dit : Ne vous querellez point en chemin.
\VS{25}Ainsi ils remontèrent d'Egypte, et vinrent dans le  pays de Canaan auprès de Jacob, leur père.
\VS{26}Et ils lui rapportèrent et lui dirent : Joseph vit encore, et même c’est lui qui gouverne tout le pays d'Egypte ; mais le cœur de Jacob resta froid, parce qu’il ne les croyait pas.
\VS{27}Et ils lui dirent toutes les paroles que Joseph leur avait dites ; puis il vit les chars que Joseph avait envoyés pour le porter ; et l'esprit de Jacob, leur père, se ranima.
\VS{28}Alors Israël dit : C'est assez ! Joseph, mon fils, vit encore ! J'irai, et je le verrai avant que je meure.
\TextTitle{[Jacob se rend en Egypte]}
\Chap{46}
\VerseOne{}Israël donc partit avec tout ce qui lui appartenait, et vint à Beer-Schéba, et il offrit des sacrifices au Dieu de son père Isaac.
\VS{2}Et Dieu parla à Israël dans une vision pendant la nuit et lui dit : Jacob, Jacob ! Et il répondit : Me voici.
\VS{3}Et Dieu lui dit : Je suis le Dieu, le Dieu de ton père. Ne crains point de descendre en Egypte, car là je te ferai devenir une grande nation.
\VS{4}Je descendrai avec toi en Egypte, et je t'en ferai aussi très certainement remonter ; et Joseph te fermera les yeux avec sa main.
\VS{5}Ainsi Jacob partit de Beer-Schéba, et les fils d'Israël mirent Jacob, leur père, et leurs petits enfants, et leurs femmes, sur les chars que Pharaon avait envoyés pour le porter.
\VS{6}Ils emmenèrent aussi leur bétail et leur bien qu'ils avaient acquis dans le pays de Canaan ; et Jacob et toute sa famille avec lui vinrent en Egypte.
\VS{7}Il amena avec lui en Egypte ses fils, et les fils de ses fils, ses filles, et les filles de ses fils, et toute sa famille.
\TextTitle{[Les fils de Jacob descendus Egypte]}
\VS{8}Voici les noms des fils d'Israël qui vinrent en Egypte : Jacob et ses fils. Le premier-né de Jacob fut Ruben.
\VS{9}Et les fils de Ruben : Hénoc, Pallu, Hetsron, et Carmi.
\VS{10}Et les fils de Siméon : Jemuel, Jamin, Ohad, Jakin, Tsochar, et Saül, fils d'une Cananéenne.
\VS{11}Et les fils de Lévi : Guerschon, Kehath, et Merari.
\VS{12}Et les fils de Juda : Er, Onan, Schéla, Pérets et Zérach ; mais Er et Onan moururent au pays de Canaan. Les fils de Pérets furent Hetsron et Hamul.
\VS{13}Et les fils d'Issacar : Thola, Puva, Job et Schimron.
\VS{14}Et les fils de Zabulon : Séred, Elon et Jahleel.
\VS{15}Ce sont là les fils de Léa, qu'elle enfanta à Jacob à Paddan-Aram, avec Dina, sa fille. Ses fils et ses filles formaient en tout trente-trois personnes.
\VS{16}Et les fils de Gad : Tsiphjon, Haggi, Schuni, Etsbon, Eri, Arodi et Areéli.
\VS{17}Et les fils d'Aser : Jimna, Jischva, Jischvi, Beria et Sérach, leur sœur. Les fils de Beria : Héber et Malkiel.
\VS{18}Ce sont là les fils de Zilpa que Laban donna à Léa, sa fille ; et elle les enfanta à Jacob. En tout seize personnes.
\VS{19}Les fils de Rachel, femme de Jacob, furent Joseph et Benjamin.
\VS{20}Et il naquit à Joseph dans le  pays d'Egypte, Manassé et Ephraïm, qu'Asnath, fille de Poti-Phéra, prêtre d'On, lui enfanta.
\VS{21}Et les fils de Benjamin étaient Béla, Béker, Aschbel, Guéra, Naaman, Ehi, Rosch, Muppim, Huppim et Ard.
\VS{22}Ce sont là les fils de Rachel, qu'elle enfanta à Jacob. En tout quatorze personnes.
\VS{23}Et les fils de Dan : Huschim.
\VS{24}Et les enfants de Nephthali : Jahtseel, Guni, Jetser, et Schillem.
\VS{25}Ce sont là les fils de Bilha, que Laban donna à Rachel, sa fille, et elle les enfanta à Jacob. En tout sept personnes.
\VS{26}Toutes les personnes appartenant à Jacob qui vinrent en Egypte, et qui étaient issues de lui, sans les femmes des fils de Jacob, furent en tout soixante-dix.
\VS{27}Et les fils de Joseph qui lui étaient nés en Egypte furent deux personnes. Toutes les personnes de la maison de Jacob qui vinrent en Egypte furent soixante-dix.
\VS{28}Jacob envoya Juda devant lui vers Joseph, pour l’informer qu’il se rendait en Gosen. Ils vinrent donc dans la contrée de Gosen.
\VS{29}Et Joseph fit atteler son char, et y monta pour aller à la rencontre d'Israël, son père, en Gosen. Dès qu’il le vit, il se jeta à son cou, et pleura longtemps sur son cou.
\VS{30}Et Israël dit à Joseph : Que je meure à présent, puisque j'ai vu ton visage, et que tu vis encore.
\VS{31}Puis Joseph dit à ses frères et à la famille de son père : Je monterai pour informer Pharaon, et je lui dirai : Mes frères et la famille de mon père, qui étaient au pays de Canaan, sont arrivés auprès de moi.
\VS{32}Et ces hommes sont bergers, ils se sont toujours occupés du bétail, et ils ont amené leurs brebis et leurs bœufs, et tout ce qui était à eux.
\VS{33}Et quand Pharaon vous fera appeler et vous dira : Quel est votre métier ?
\VS{34}Vous direz : Tes serviteurs se sont toujours occupés de bétail dès leur jeunesse jusqu'à maintenant, nous, et nos pères. De cette manière, vous habiterez dans le pays de Gosen, car les Egyptiens ont en abomination les bergers.
\TextTitle{[La famille de Jacob honoré en Egypte]}
\Chap{47}
\VerseOne{}Joseph alla avertir Pharaon, et lui dit : Mon père  et mes frères sont arrivés du pays de Canaan avec leurs troupeaux et leurs bœufs, et tout ce qui est à eux ; et voici, ils sont dans le pays de Gosen.
\VS{2}Et il prit une partie de ses frères, à savoir cinq, et il les présenta à Pharaon.
\VS{3}Et Pharaon dit aux frères de Joseph : Quel est votre métier ? Ils répondirent à Pharaon : Tes serviteurs sont bergers, comme l'ont été nos pères.
\VS{4}Ils dirent aussi à Pharaon : Nous sommes venus séjourner comme étrangers dans ce pays, parce qu'il n'y a plus de pâturages pour les troupeaux de tes serviteurs, et il y a une grande famine au pays de Canaan ; maintenant nous te prions que tes serviteurs demeurent dans le pays de Gosen.
\VS{5}Et Pharaon parla à Joseph et lui dit : Ton père et tes frères sont arrivés auprès de toi.
\VS{6}Le pays d'Egypte est à ta disposition ; fais habiter ton père et tes frères dans le meilleur endroit du pays ; qu'ils demeurent dans la terre de Gosen ; et si tu connais parmi eux des hommes habiles tu les établiras chefs de tous mes troupeaux.
\VS{7}Alors Joseph amena Jacob, son père, et le présenta à Pharaon ; et Jacob bénit Pharaon.
\VS{8}Et Pharaon dit à Jacob : Quel est le nombre de jours de tes années ?
\VS{9}Jacob répondit à Pharaon : Les jours des années de mes pèlerinages sont de cent trente ans ; les jours des années de ma vie ont été courts et mauvais et n'ont point atteint les jours des années de la vie de mes pères, du temps de leurs pèlerinages.
\VS{10}Jacob donc bénit Pharaon, et sortit de devant lui.
\VS{11}Et Joseph assigna une demeure à son père et à ses frères, et leur donna une possession au pays d'Egypte, au meilleur endroit du pays, dans le pays d'Egypte, comme Pharaon l'avait ordonné.
\VS{12}Et Joseph fournit du pain à son père et à ses frères, et à toute la maison de son père, selon le nombre de leurs familles.
\VS{13}Or il n'y avait point de pain sur toute la terre, car la famine était très grande ; et le pays d'Egypte et le pays de Canaan étaient épuisés par la famine.
\VS{14}Et Joseph amassa tout l'argent qui se trouva dans le pays d'Egypte, et dans le pays de Canaan, contre le blé qu'on achetait ; et il apporta l'argent à la maison de Pharaon.
\VS{15}Quand l'argent du pays d'Egypte et du pays de Canaan fut épuisé, tous les Egyptiens vinrent à Joseph en disant : Donne-nous du pain ; et pourquoi mourrions-nous en ta présence, parce que l'argent manque ?
\VS{16}Joseph répondit : Donnez votre bétail, et je vous en donnerai pour votre bétail, puisque l'argent manque.
\VS{17}Alors ils amenèrent à Joseph leur bétail, et Joseph leur donna du pain pour des chevaux, pour des troupeaux de brebis, pour des troupeaux de boeufs, et pour des ânes ; ainsi il leur fournit du pain en échange de leurs troupeaux cette année-là.
\VS{18}Lorsque cette année fut écoulée, ils revinrent à Joseph l'année suivante et lui dirent : Nous ne cacherons point à mon seigneur que l'argent est épuisé et les troupeaux de bétail ont été amenés à mon seigneur, il ne nous reste plus rien devant mon seigneur que nos corps et nos terres.
\VS{19}Pourquoi mourrions-nous sous tes yeux ? Achète-nous avec nos terres, pour du pain ; et nous serons esclaves de Pharaon, et nos terres seront à lui ; donne-nous aussi de quoi semer, afin que nous vivions et ne mourions point, et que nos terres ne soient point désolées.
\VS{20}Ainsi, Joseph acheta toutes les terres de l’Egypte pour Pharaon ; car les Egyptiens vendirent chacun son champ, parce que la famine les pressait. Et le pays devint la propriété de Pharaon.
\VS{21}Et il fit passer le peuple dans les villes, d’un bout à l’autre des frontières de l’Egypte.
\VS{22}Seulement, il n’acheta point les terres des prêtres, parce qu’il y avait une loi de Pharaon en faveur des prêtres, qui vivaient du revenu que leur assurait Pharaon, c’est pourquoi ils ne vendirent point leurs terres.
\VS{23}Et Joseph dit au peuple : Voici, je vous ai achetés aujourd'hui, vous et vos terres pour Pharaon, voilà de la semence pour ensemencer la terre.
\VS{24}Et quand le temps de la récolte viendra, vous donnerez la cinquième partie à Pharaon, et les quatre autres seront à vous, pour ensemencer les champs, et pour votre nourriture, et pour celle de ceux qui sont dans vos maisons, et pour la nourriture de vos petits enfants.
\VS{25}Et ils dirent : Tu nous sauves la vie ! Que nous trouvions grâce aux yeux de mon seigneur, et nous serons esclaves de Pharaon.
\VS{26}Et Joseph fit de cela une loi qui a subsisté jusqu’à ce jour, et d’après laquelle un cinquième du revenu des terres de l’Egypte appartient à Pharaon ; il n’y a que les terres des prêtres qui ne soient point à Pharaon.
\TextTitle{[Jacob demande à être enterré à Canaan]}
\VS{27}Israël habita dans le pays d’Egypte, dans le pays de Gosen. Ils eurent des possessions, ils furent féconds et multiplièrent beaucoup.
\VS{28}Jacob vécut dix-sept ans dans le pays d’Egypte ; et les jours des années de la vie de Jacob furent de cent quarante-sept ans.
\VS{29}Et quand le jour de la mort d'Israël approcha, il appela Joseph, son fils, et lui dit : Je te prie, si j'ai trouvé grâce à tes yeux, mets présentement ta main sous ma cuisse, et jure-moi que tu useras envers moi de bonté et de fidélité : Je te prie, ne m'enterre point en Egypte !
\VS{30}Quand  je serai couché avec mes pères, tu me transporteras hors de l'Egypte, et m'enterreras dans leur sépulcre. Et il répondit : Je le ferai selon ta parole.
\VS{31}Et Jacob lui dit : Jure-le-moi ; et il le lui jura. Et Israël se prosterna sur le chevet du lit.
\TextTitle{[Bénédiction de Jacob sur les fils de Joseph]}
\Chap{48}
\VerseOne{}Or il arriva après ces choses que l'on vint dire à Joseph : Voici, ton père est malade. Et il prit avec lui ses deux fils, Manassé et Ephraïm.
\VS{2}On avertit Jacob et on lui dit : Voici Joseph, ton fils, qui vient vers toi. Alors Israël rassembla ses forces et s’assit sur son lit.
\VS{3}Puis Jacob dit à Joseph : Le Dieu Tout-Puissant m’est apparu  à Luz, au pays de Canaan, et m’a béni.
\VS{4}Et il m’a dit : Voici, je te ferai croître et multiplier, et je te ferai devenir une assemblée de peuples, et je donnerai ce pays en possession perpétuelle à ta postérité après toi.
\VS{5}Et maintenant tes deux fils, qui te sont nés au pays d'Egypte, avant mon arrivée vers toi, seront à moi : Ephraïm et Manassé seront à moi comme Ruben et Siméon.
\VS{6}Mais les enfants que tu auras engendrés après eux, seront à toi, et ils seront appelés selon le nom de leurs frères dans leur héritage.
\VS{7}A mon retour de Paddan, Rachel mourut en route auprès de moi, dans le pays de Canaan, à quelque distance d’Ephrata ; et c’est là que je l’ai enterrée, sur le chemin d’Ephrata, qui est Bethléhem.
\VS{8}Puis Israël vit les fils de Joseph, et il dit : Qui sont ceux-ci ?
\VS{9}Et Joseph répondit à son père : Ce sont mes fils que Dieu m'a donnés ici ; et il dit : Amène-les-moi, je te prie, afin que je les bénisse.
\VS{10}Or les yeux d'Israël étaient appesantis par la vieillesse, et il ne pouvait plus voir ; et il les fit approcher de lui, les embrassa et les prit dans ses bras.
\VS{11}Et Israël dit à Joseph : Je ne pensais pas revoir ton visage ; et voici, Dieu m'a fait voir et toi et ta postérité.
\VS{12}Et Joseph les retira des genoux de son père, et se prosterna le visage contre terre.
\VS{13}Puis Joseph les prit tous deux, Ephraïm de sa main droite à la gauche d’Israël, et Manassé de sa main gauche à la droite d’Israël, et il les fit approcher de lui.
\VS{14}Israël étendit sa main droite et la posa sur la tête d’Ephraïm qui était le plus jeune, et il posa sa main gauche sur la tête de Manassé ; ce fut avec intention qu’il posa ses mains ainsi, car Manassé était le premier-né.
\VS{15}Il bénit Joseph et dit : Que le Dieu en présence duquel ont marché mes pères, Abraham et Isaac, que le Dieu qui m’a conduit depuis que j’existe jusqu’à ce jour\FTNT{Hé. 11:21.},
\VS{16}que l’Ange qui m’a délivré de tout mal, bénisse ces enfants ! Qu’ils soient appelés de mon nom et du nom de mes pères, Abraham et Isaac, et qu’ils multiplient en abondance comme les poissons au milieu du pays.
\VS{17}Joseph vit avec déplaisir que son père posait sa main droite sur la tête d’Ephraïm ; il saisit la main de son père, pour la détourner de dessus la tête d’Ephraïm, et la diriger sur celle de Manassé.
\VS{18}Et Joseph dit à son père : Ce n'est pas ainsi mon père ! Car celui-ci est l'aîné ; mets ta main droite sur sa tête.
\VS{19}Mais son père le refusa en disant : Je le sais, mon fils, je le sais. Celui-ci deviendra aussi un peuple, et même il sera grand ; mais toutefois son frère, qui est plus jeune, sera plus grand que lui, et sa postérité sera une multitude de nations.
\VS{20}Il les bénit ce jour-là et dit : C’est par toi qu’Israël bénira en disant : Que Dieu te traite comme Ephraïm et comme Manassé ! Et il mit Ephraïm avant Manassé.
\VS{21}Puis Israël dit à Joseph : Voici, je  vais mourir, mais Dieu sera avec vous, et vous fera retourner au pays de vos pères.
\VS{22}Et je te donne une portion  de plus qu'à tes frères, celle que j'ai prise avec mon épée et mon arc sur les Amoréens.
\TextTitle{[Prophétie de Jacob qui bénit ses fils]}
\Chap{49}
\VerseOne{}Puis Jacob appela ses fils et leur dit : Assemblez-vous, et je vous annoncerai ce qui vous arrivera dans les derniers jours\FTNT{L’expression « dans les derniers jours » vient de l’hébreu « achariyth » qui veut dire « dernier ». Son équivalent grec est « eschatos »:« dernier », « extrémité » etc. Jacob est le premier homme à avoir utilisé  cette expression. Cette promesse de Jacob devait arriver à Israël dans les derniers jours, selon leurs tribus. Ainsi, les promesses du droit d'aînesse de Ge. 49 étaient pour l'âge messianique, lequel est associé aux derniers jours, et a commencé à la Fête de la Pentecôte (Ac. 2:14-21). 
Ces jours impliquent :
-	L’effusion de l’Esprit, le réveil de l’Eglise de Christ (Mt. 25:1-13 ; Ac. 2)
-	Le réveil des faux prophètes ou l’apostasie (2 Pi. 3:3 ; 1 Jn. 2)
-	La dégradation de la moralité (2 Ti. 3)
-	L’enrichissement des hommes de ce monde (Ja. 5:3 ; Ap. 3:14-22)
-	Le fait que Dieu nous parle par le Fils (Hé. 1:2)
-	La future résurrection des saints lors du retour du Messie (Jn. 6:39-54 ; 1 Th. 4:12-17).Le temps des nations (fin des temps) s’achèvera lors du retour visible de Jésus-Christ pour établir son règne sur toute la terre. Le temps des nations a commencé lorsque, à la suite de l’infidélité d’Israël, la gloire de Dieu a quitté le temple et la ville de Jérusalem (Ez. 11), la puissance fut confiée aux nations en la personne de Nebucadnetsar qui s’empara de Jérusalem (2 R. 24 et 25 ; 2 Ch. 36:6-21 ; Da. 1 ; Jé 39). Ces temps dureront jusqu’à la destruction finale du dernier empire des nations représenté par la Bête romaine ressuscitée (Ap. 13:3). Cette destruction n’aura lieu que lorsque Jésus-Christ, la pierre détachée sans le secours d’aucune main, deviendra une grande montagne qui remplira toute la terre (Da. 2:34 ; Mi. 4). Jérusalem ne sera délivrée du joug des nations qu’à ce moment-là. Les temps des nations ne seront accomplis que lorsque le trône de Dieu sera de nouveau établi à Jérusalem.}.
\VS{2}Rassemblez-vous, et écoutez, fils de Jacob ; écoutez Israël\FTNT{Ecoutez Israël:Le shema Israël est le texte principal de la liturgie juive. Composé de trois extraits de la Torah, on le récite matin et soir accompagné de bénédictions. Voir De. 6:4-9.}, votre père.
\VS{3}Ruben, tu es mon premier-né, ma force et le commencement de ma vigueur, qui excelle en dignité et qui excelle aussi en force ;
\VS{4}impétueux comme les eaux ; tu n'auras pas la prééminence, car tu es monté sur la couche de ton père, et tu as souillé mon lit en y montant.
\VS{5}Siméon et Lévi, sont frères, leurs glaives sont des instruments de violence dans leurs demeures.
\VS{6}Que mon âme n'entre point dans leur conseil secret, que ma gloire ne soit point jointe à leur compagnie, car ils ont tué les gens dans leur colère, et ont enlevé les bœufs pour leur plaisir.
\VS{7}Maudite soit leur colère, car elle a été violente ; et leur fureur, car elle a été cruelle ; je les diviserai dans Jacob, et les disperserai dans Israël.
\VS{8}Juda, quant à toi, tes frères te loueront ; ta main sera sur la nuque de tes ennemis ; les fils de ton père se prosterneront devant toi.
\VS{9}Juda est un jeune lion. Mon fils, tu reviens du carnage, mon fils ! Il ploie les genoux, il se couche comme un lion, comme une lionne : Qui le fera lever ?
\VS{10}Le sceptre ne s’éloignera point de Juda, ni le bâton de législateur d'entre ses pieds, jusqu'à ce que le Schilo vienne, et que  les peuples lui obéissent.
\VS{11}Il attache à la vigne son ânon, et au cep excellent le petit de son ânesse ; il lavera son vêtement dans le vin, et son vêtement dans le sang des raisins.
\VS{12}Il a les yeux rouges de vin, et les dents blanches de lait.
\VS{13}Zabulon habitera sur la côte des mers, il sera un port des navires ; et ses côtés s'étendront vers Sidon.
\VS{14}Issacar est un âne robuste, couché entre les barres des étables.
\VS{15}Il voit que le lieu où il repose est agréable et que la contrée est magnifique ; et il courbe son épaule sous le fardeau, il s’assujettit à un tribut.
\VS{16}Dan jugera son peuple, comme l’une des tribus d'Israël.
\VS{17}Dan sera un serpent sur le chemin, une vipère sur le sentier, mordant les talons du cheval, pour que le cavalier tombe à la renverse.
\VS{18}Ô Yahweh ! J’espère en ton salut\FTNT{Le mot secours se dit « yeshuw`ah »  en hébreu et veut littéralement dire  salut, délivrance. C’est de cette même racine hébraïque que le nom du Seigneur Jésus, « Yehowshuwa », est tiré. Avant de mourir, Jacob a  donc  placé son espérance en Jésus-Christ qui est la résurrection et la vie (Jn. 11:25). Voir commentaire en Es. 26:1.} !
\VS{19}Quant à Gad, des troupes viendront l’attaquer, mais il ravagera leur arrière-garde.
\VS{20}Le pain excellent viendra d'Aser, et il fournira les mets délicats des rois.
\VS{21}Nephtali est une biche en liberté ; il profère des belles paroles.
\VS{22}Joseph est un fils fertile, un rameau fertile près d'une fontaine ; ses branches se sont étendues sur la muraille.
\VS{23}Des archers l’ont provoqué, ils ont lancé des traits ; les archers l’ont poursuivi de leur haine.
\VS{24}Mais son arc est demeuré ferme, et ses mains ont été fortifiées par les mains du Puissant de Jacob : Il est ainsi devenu le pasteur, le rocher d’Israël.
\VS{25}C’est l’œuvre du Dieu de ton père qui t’aidera ; c’est l’œuvre du Tout-Puissant qui te bénira des bénédictions des cieux en haut, des bénédictions des eaux en bas, des bénédictions des mamelles et du sein maternel.
\VS{26}Les bénédictions de ton père ont surpassé les bénédictions de ceux qui m'ont engendré, jusqu'à la cime des antiques collines ; elles seront sur la tête de Joseph, et sur le sommet de la tête du Nazaréen d'entre ses frères.
\VS{27}Benjamin est un loup qui déchirera ; le matin il dévorera la proie, et sur le soir il partagera le butin.
\VS{28}Ce sont là tous ceux qui forment les douze tribus d'Israël.  Et c’est là ce que leur père leur dit en les bénissant. Il bénit chacun d'eux selon la bénédiction qui lui était propre.
\VS{29}Il leur donna aussi cet ordre : Je vais être recueilli auprès de mon peuple, enterrez-moi avec mes pères dans la caverne qui est au champ d'Ephron, le Héthien,
\VS{30}dans la caverne du champ de Macpéla, vis-à-vis de Mamré, dans le pays de Canaan. C’est le champ qu’Abraham a acheté d’Ephron, le Héthien, comme propriété sépulcrale.
\VS{31}C'est là qu'on a enterré Abraham avec Sara, sa femme ; c'est là qu'on a enterré Isaac et Rebecca, sa femme ; et c'est là que j'ai enterré Léa.
\VS{32}Le champ a été acquis des fils de Heth avec la caverne qui s’y trouve.
\VS{33}Lorsque Jacob eut achevé de donner ses ordres à ses fils, il retira ses pieds dans le lit, il expira, et fut recueilli auprès de son peuple.
\TextTitle{[Mort de Jacob]}
\Chap{50}
\VerseOne{}Alors Joseph se jeta sur le visage de son père, pleura sur lui et l’embrassa.
\VS{2}Et Joseph ordonna à ceux de ses serviteurs qui étaient médecins d'embaumer son père ; et les médecins embaumèrent Israël.
\VS{3}Et on employa quarante jours à l'embaumer, car c'était la coutume d'embaumer les corps pendant quarante jours ; et les Egyptiens le pleurèrent soixante-dix jours.
\VS{4}Quand les jours du deuil furent passés, Joseph s’adressa aux gens de la maison de Pharaon, et leur dit : Si j’ai trouvé grâce à vos yeux, rapportez, je vous prie, à Pharaon ce que je vous dis.
\VS{5}Mon père m’a fait jurer en disant : Voici, je vais mourir ! Tu m’enterreras dans le sépulcre que je me suis acheté au pays de Canaan. Je voudrais donc y monter, pour enterrer mon père ; et je reviendrai.
\VS{6}Et Pharaon répondit : Monte, et enterre ton père comme il t'a fait jurer.
\VS{7}Alors Joseph monta pour enterrer son père, et les serviteurs de Pharaon, les anciens de la maison de Pharaon, et tous les anciens du pays d'Egypte montèrent avec lui.
\VS{8}Et toute la maison de Joseph, et ses frères, et la maison de son père y montèrent aussi, laissant seulement leurs familles, et leurs troupeaux, et leurs bœufs dans le pays de Gosen.
\VS{9}Il y avait encore avec Joseph des chars et des cavaliers, en sorte que le cortège était très nombreux.
\VS{10}Arrivés à l’aire d’Athad, qui est au-delà du Jourdain, ils firent entendre de grandes et profondes lamentations ; et Joseph fit en l’honneur de son père un deuil de sept jours.
\VS{11}Et les Cananéens, habitants du pays, voyant ce deuil dans l'aire d'Athad, dirent : Ce deuil est grand pour les Egyptiens ; c'est pourquoi cette aire, qui est au-delà du Jourdain, fut nommée Abel-Mitsraïm\FTNT{Abel-Mitsraïm:« Pré du deuil de l’Egypte ».}.
\VS{12}Les fils de Jacob firent à l'égard de son corps ce qu'il leur avait ordonné.
\VS{13}Ils le transportèrent au pays de Canaan, et l’enterrèrent dans la caverne du champ de Macpéla, qu’Abraham avait achetée d’Ephron, le Héthien, comme propriété sépulcrale, et qui est vis-à-vis de Mamré.
\VS{14}Et après que Joseph eut enseveli son père, il retourna en Egypte avec ses frères et tous ceux qui étaient montés avec lui pour ensevelir son père.
\VS{15}Et les frères de Joseph, voyant que leur père était mort, se dirent entre eux : Peut-être que Joseph nous aura en haine, et ne manquera pas de nous rendre tout le mal que nous lui avons fait.
\VS{16}Et ils firent dire à Joseph : Ton père a donné cet ordre avant de mourir en disant :
\VS{17}Vous parlerez ainsi à Joseph : Je te prie, pardonne maintenant l'iniquité de tes frères, et leur péché, car ils t'ont fait du mal. Maintenant, je te supplie, pardonne le crime des serviteurs du Dieu de ton père. Et Joseph pleura quand on lui parla.
\VS{18}Ses frères vinrent eux-mêmes se prosterner devant lui, et ils dirent : Nous sommes tes serviteurs.
\VS{19}Et Joseph leur dit : Ne craignez point, car suis-je à la place de Dieu ?
\VS{20}Vous aviez médité de me faire du mal : Dieu l’a changé en bien, pour accomplir ce qui arrive aujourd’hui, pour sauver la vie à un peuple nombreux.
\VS{21}Soyez donc sans crainte ; je vous entretiendrai, vous et vos familles ; et il les consola en parlant à leur cœur.
\VS{22}Joseph demeura donc en Egypte, lui et la maison de son père, et vécut cent dix ans.
\VS{23}Et Joseph vit les fils d'Ephraïm jusqu'à la troisième génération. Makir aussi, fils de Manassé, eut des fils qui furent élevés sur les genoux de Joseph.
\VS{24}Et Joseph dit à ses frères : Je vais mourir ! Mais Dieu ne manquera pas de vous visiter, et il vous fera remonter de ce pays au pays qu’il a juré  de donner à Abraham, Isaac et à Jacob.
\VS{25}Et Joseph fit jurer les enfants d'Israël et leur dit : Dieu ne manquera pas de vous visiter, et alors vous transporterez mes os d'ici\FTNT{Hé. 11:22 ; Ex. 13:19.}.
\VS{26}Puis Joseph mourut, âgé de cent dix ans. On l'embauma, et on le mit dans un cercueil en Egypte.
\PPE{}
\end{multicols}
