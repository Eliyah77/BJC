\ShortTitle{Néhémie}\BookTitle{Néhémie}\BFont
\noindent\hrulefill
{\footnotesize
\textit{
\bigskip
{\centering{}
\\Auteur : Néhémie
\\(Heb. : Nechemyah)
\\Signification : Yahweh a consolé
\\Thème : Reconstruction des murailles de Jérusalem
\\Date de rédaction : 5\up{ème} siècle av. J.-C.\\}
}
%\bigskip
\textit{
\\En apprenant l'état de ruine dans lequel se trouvait Jérusalem, Néhémie, échanson du roi perse Artaxerxés Ier, fut profondément affecté. Après plusieurs jours dans la désolation et l'humiliation, le Seigneur toucha le cœur du roi qui lui donna l'autorisation et le matériel nécessaire pour rebâtir la muraille de Jérusalem. Malgré les nombreuses oppositions dont il fit l'objet au cours de son entreprise, Néhémie acheva l'œuvre qui lui avait été confiée. Dans le même temps, il mit en place de profondes réformes dans le cadre du retour à la loi de Yahweh.
%\bigskip
\\Complément du livre d'Esdras avec lequel il ne formait initialement qu'un ouvrage, le livre de Néhémie présente un homme de prière, un serviteur œuvrant pour, avec, et au Nom de Yahweh.\bigskip
}
}
\par\nobreak\noindent\hrulefill
\begin{multicols}{2}
\Chap{1}
\TextTitle{La détresse du peuple resté à Jérusalem est racontée à Néhémie}
\VerseOne{}Paroles de Néhémie, fils de Hacalia. Il arriva au mois de Kisleu, la vingtième année, comme j'étais à Suse, la capitale,
\VS{2}Hanani, l'un de mes frères et quelques hommes arrivèrent de Juda. Je les questionnai au sujet des Juifs réchappés qui étaient restés de la captivité et au sujet de Jérusalem.
\VS{3}Et ils me dirent : Ceux qui sont restés de la captivité sont là dans la province, dans une grande misère et dans l'opprobre ; et la muraille de Jérusalem demeure renversée et ses portes ont été consumées par le feu.
\TextTitle{Néhémie prie Yahweh et implore sa grâce}
\VS{4}Or il arriva que, dès que j'entendis ces paroles, je m'assis, je pleurai et je fus dans le deuil plusieurs jours. Je jeûnai et je priai devant le Dieu des cieux,
\VS{5} et je dis : Je te prie, ô Yahweh ! Dieu des cieux, Dieu grand et redoutable, qui garde l'alliance et la miséricorde de ceux qui t'aiment et qui observent tes commandements !
\VS{6}Je te prie que ton oreille soit attentive et que tes yeux soient ouverts pour entendre la prière que ton serviteur te présente en ce temps-ci, jour et nuit, pour tes serviteurs les enfants d'Israël, en confessant les péchés des enfants d'Israël, que nous avons commis contre toi ; même moi et la maison de mon père, nous avons péché.
\VS{7}Certainement nous sommes coupables devant toi, nous n'avons pas gardé les commandements, les lois et les ordonnances que tu prescrivis à Moïse, ton serviteur.
\VS{8}Mais, je te prie, souviens-toi de la parole que tu chargeas Moïse, ton serviteur, de dire : Vous pécherez et je vous disperserai parmi les peuples\FTNT{De. 28:63-67.} ;
\VS{9}mais si vous revenez à moi, et si vous gardez mes commandements et les observez ; et s'il y en a d'entre vous qui ont été chassés jusqu'à l'extrémité du ciel, je vous rassemblerai de là, et je vous ramènerai au lieu que j'aurai choisi pour y faire habiter mon Nom\FTNT{De. 30:1-10.}.
\VS{10}Ils sont tes serviteurs et ton peuple, que tu as rachetés par ta grande puissance et par ta main forte.
\VS{11}Je te prie donc, Seigneur, que ton oreille soit maintenant attentive à la prière de ton serviteur, et à la suplication de tes serviteurs qui prennent plaisir à craindre ton Nom ! Je te prie, donne aujourd'hui du succès à ton serviteur, et fais-lui trouver grâce devant cet homme ! J'étais alors échanson du roi.
\Chap{2}
\TextTitle{Yahweh exauce Néhémie et lui donne la faveur du roi}
\VerseOne{}Et il arriva, au mois de Nisan, la vingtième année du roi Artaxerxès, comme le vin était devant lui, je pris le vin et le présentai au roi. Or je n'avais jamais eu mauvais visage devant lui\FTNT{Pr. 15:13.}.
\VS{2}Et le roi me dit : Pourquoi as-tu mauvais visage, puisque tu n'es point malade ? Cela ne peut être qu'une tristesse de cœur. Je fus alors saisi d'une grande crainte,
\VS{3}et je répondis au roi : Que le roi vive éternellement ! Comment n'aurais-je pas mauvais visage, puisque la ville où sont les sépulcres de mes pères demeure désolée et que ses portes ont été consumées par le feu ?
\VS{4}Et le roi dit : Que me demandes-tu ? Alors je priai le Dieu des cieux,
\VS{5}et je dis au roi : Si le roi le trouve bon, et si ton serviteur lui est agréable, envoie-moi en Juda, vers la ville des sépulcres de mes pères, pour la rebâtir\FTNT{La reconstruction de la ville de Jérusalem sous Néhémie date, selon certains, de l'an 445 av. J.-C., suite au décret d'Artaxerxès. Cette date marquerait le point de départ des soixante-dix semaines d'années annoncées par Daniel (Da. 9:24-27).}.
\VS{6}Le roi me dit, et sa femme aussi qui était assise auprès de lui : Combien ton voyage durera-t-il, et quand seras-tu de retour ? Je lui précisai le temps, et le roi trouva bon de m'envoyer.
\VS{7}Puis je dis au roi : Si le roi le trouve bon, qu'on me donne des lettres pour les gouverneurs de l'autre côté du fleuve, afin qu'ils me laissent passer, jusqu'à ce que j'arrive en Juda ;
\VS{8}et des lettres pour Asaph, le garde de la forêt du roi, afin qu'il me donne du bois pour la charpente des portes de la forteresse près de la maison, pour les murailles de la ville, et pour la maison dans laquelle j'entrerai. Et le roi me l'accorda, car la main de mon Dieu était bonne sur moi.
\TextTitle{Arrivée à Jérusalem, constat des murailles en ruines}
\VS{9}J'allai donc vers les gouverneurs qui sont de l'autre côté du fleuve et je leur donnai les lettres du roi. Le roi avait aussi envoyé avec moi des chefs de l'armée et des cavaliers.
\VS{10}Quand Sanballat, le Horonite, et Tobija, le serviteur Ammonite, l'ayant appris, ils eurent un très grand déplaisir de ce qu'il venait un homme pour procurer du bien aux enfants d'Israël.
\VS{11}Ainsi j'arrivai à Jérusalem et j'y passai trois jours.
\VS{12}Puis je me levai de nuit, avec quelques hommes ; mais je ne dis à personne ce que Dieu avait mis dans mon cœur de faire pour Jérusalem. Il n'y avait point d'autre bête avec moi que celle sur laquelle j'étais monté.
\VS{13}Je sortis donc de nuit par la porte de la vallée et me dirigeai vers la source du dragon, vers la porte du fumier ; et je considérai les murailles de Jérusalem qui étaient en ruines\FTNT{Jé. 39:8.}, et ses portes consumées par le feu.
\VS{14}Je passai près de la porte de la source et vers l'étang du roi ; et il n'y avait point de place par où je puisse passer avec ma monture.
\VS{15}Je montai de nuit par le torrent et je considérai la muraille. Puis en revenant, je rentrai par la porte de la vallée ; et ainsi je fus de retour.
\VS{16}Or les magistrats ne savaient pas où j'étais allé, ni ce que je faisais ; car je n'avais rien dit jusqu'à ce moment, ni aux Juifs, ni aux sacrificateurs, ni aux chefs, ni aux magistrats, ni au reste de ceux qui s'occupaient des affaires.
\TextTitle{Néhémie partage sa vision de rebâtir la muraille}
\VS{17}Alors je leur dis : Vous voyez la misère dans laquelle nous sommes ! Comment Jérusalem demeure désolée et ses portes brûlées par le feu ! Venez et rebâtissons les murailles de Jérusalem et nous ne serons plus dans l'opprobre.
\VS{18}Et je leur déclarai comment la main de mon Dieu avait été bonne sur moi, et quelles paroles le roi m'avait dites. Alors ils dirent : Levons-nous et bâtissons ! Ils fortifièrent leurs mains pour bien faire.
\TextTitle{Premières oppositions}
\VS{19}Mais Sanballat, le Horonite, Tobija, le serviteur Ammonite, et Guéschem, l'Arabe, l'ayant appris, se moquèrent de nous et nous méprisèrent. Ils dirent : Qu'est-ce que vous faites ? Ne vous rebellez-vous pas contre le roi ?
\VS{20}Et je leur répondis cette parole : Le Dieu des cieux lui-même nous donnera le succès ! Nous donc, qui sommes ses serviteurs, nous nous lèverons et nous bâtirons ; mais vous, vous n'avez aucune part, ni droit, ni souvenir, à Jérusalem.
\Chap{3}
\TextTitle{Les participants à la reconstruction de la muraille}
\VerseOne{}Eliaschib, le souverain sacrificateur, se leva donc avec ses frères, les sacrificateurs et ils rebâtirent la porte des brebis\FTNT{La première porte qui fut reconstruite fut la porte des brebis. Cette porte est très proche du temple, c'est par elle que l'on faisait entrer les brebis destinées aux sacrifices dans la cour du temple. Cette porte est la préfiguration de Jésus-Christ qui s'est lui-même présenté comme étant la « porte des brebis » (Jn. 10:7).}. Ils la sanctifièrent, ils y posèrent ses battants. Ils la consacrèrent jusqu'à la tour de Méa jusqu'à la tour de Hananeel.
\VS{2}Et les gens de Jéricho rebâtirent à son côté ; et à côté d'eux Zaccur, fils d'Imri, rebâtit aussi.
\VS{3}Les fils de Senaa rebâtirent la porte des poissons. Ils en firent la charpente et y mirent ses portes, ses serrures et ses barres.
\VS{4}Et à leur côté travailla aux réparations Merémoth, fils d'Urie, fils d'Hakkots ; et à leur côté travailla Meschullam, fils de Bérékia, fils de Meschézabeel, et à leur côté travailla Tsadok, fils de Baana.
\VS{5}A leur côté travaillèrent les Tekoïtes ; mais les chefs d'entre eux ne vinrent point au service de leur Seigneur.
\VS{6}Et Jojada, fils de Paséach, et Meschullam, fils de Besodia, réparèrent la vieille porte. Ils en firent la charpente, y mirent ses battants, ses serrures et ses barres.
\VS{7}A leur côté travaillèrent Melatia, le Gabaonite, Jadon, le Méronothite, et les hommes de Gabaon et de Mitspa, vers le siège du gouverneur de ce côté du fleuve.
\VS{8}A côté d'eux travailla Uzziel, fils de Harhaja, d'entre les orfèvres, et à côté de lui travailla Hanania, d'entre les parfumeurs. Et ainsi ils relevèrent Jérusalem jusqu'à la muraille large.
\VS{9}Et à leur côté travailla Rephaja, fils de Hur, chef d'un demi-quartier de Jérusalem.
\VS{10}Puis à leur côté travailla Jedaja, fils de Harumaph, devant sa maison ; et à son côté travailla Hattusch, fils de Haschabnia.
\VS{11}Et Malkija, fils de Harim, et Haschub, fils de Pachath-Moab, en réparèrent une seconde section, et la tour des fours.
\VS{12}Et à leur côté travailla, avec ses filles, Schallum, fils de d'Hallochesch, chef de la moitié du quartier de Jérusalem.
\VS{13}Hanun et les habitants de Zanoach réparèrent la porte de la vallée. Ils la rebâtirent et mirent ses battants, ses serrures, et ses barres, et ils bâtirent mille coudées de muraille, jusqu'à la porte du fumier.
\VS{14}Et Malkija, fils de Récab, chef du quartier de Beth-Hakkérem, répara la porte du fumier. Il la rebâtit et mit ses battants, ses serrures et ses barres.
\VS{15}Schallum, fils de Col-Hozé, chef du quartier de Mitspa, répara la porte de la source. Il la rebâtit et la couvrit, et mit ses portes, ses serrures, et ses barres. Il répara aussi la muraille de l'étang de Siloé, vers le jardin du roi, et jusqu'aux marches qui descendent de la cité de David.
\VS{16}Après lui travailla Néhémie, fils d'Azbuk, chef de la moitié du quartier de Beth-Tsur, jusqu'à l'endroit des sépulcres de David, et jusqu'à l'étang qui avait été refait, et jusqu'à la maison des hommes vaillants.
\VS{17}Après lui travaillèrent les Lévites, Rehum, fils de Bani ; et à son côté travailla Haschabia, chef de la moitié du quartier de Keïla, pour ceux de son quartier.
\VS{18}Après lui travaillèrent leurs frères, Bavvaï, fils de Hénadad, chef de la moitié du quartier de Keïla.
\VS{19}A son côté, Ezer, fils de Josué, chef de Mitspa, en répara autant, à l'endroit où l'on monte à l'arsenal, à l'angle.
\VS{20}Après lui Baruc, fils de Zabbaï, répara avec ardeur une seconde section, depuis l'angle jusqu'à la porte de la maison d'Eliaschib, le souverain sacrificateur.
\VS{21}Après lui Merémoth, fils d'Urie, fils d'Hakkots, répara une seconde section, depuis l'entrée de la maison d'Eliaschib, jusqu'à l'extrémité de la maison d'Eliaschib.
\VS{22}Et après lui travaillèrent les sacrificateurs, habitants des environs.
\VS{23}Après eux, Benjamin et Haschub travaillèrent devant leur maison. Après eux, Azaria, fils de Maaséja, fils d'Anania, travailla auprès de sa maison.
\VS{24}Après lui, Binnuï, fils de Hénadad, répara une seconde section, depuis la maison d'Azaria jusqu'à l'angle et jusqu'au coin.
\VS{25}Palal, fils d'Uzaï, travailla vis-à-vis de l'angle, et de la tour qui sort de la tour supérieure du roi, qui est auprès de la cour de la prison. Après lui travailla Pedaja, fils de Pareosch.
\VS{26}Les Néthiniens, qui demeuraient sur la colline, réparèrent vers l'orient, jusqu'à l'endroit de la porte des eaux, et vers la tour qui sort.
\VS{27}Après eux, les Tekoïtes réparèrent une seconde section, depuis l'endroit de la grande tour qui sort en dehors, jusqu'à la muraille de la colline.
\VS{28}Au-dessus de la porte des chevaux, les sacrificateurs travaillèrent, chacun devant de sa maison.
\VS{29}Après eux, Tsadok, fils d'Immer, travailla devant sa maison. Après lui répara Schemaeja, fils de Schecania, gardien de la porte orientale.
\VS{30}Après lui, Hanania, fils de Schélémia et Hanun le sixième fils de Tsalaph, en réparèrent une seconde section. Après eux, Meschullam, fils de Bérékia, travailla vis-à-vis de sa chambre.
\VS{31}Après lui, Malkija, fils de l'orfèvre, répara jusqu'à la maison des Néthiniens et des marchands, vis-à-vis de la porte de Miphkad, et jusqu'à la chambre haute du coin.
\VS{32}Et les orfèvres et les marchands travaillèrent entre la chambre haute du coin et la porte des brebis.
\Chap{4}
\TextTitle{La prière, solution pour faire face aux attaques et moqueries}
\VerseOne{}Or il arriva que Sanballat apprit que nous rebâtissions la muraille, il devint furieux et très fâché. Il se moqua des Juifs.
\VS{2}Et il dit en présence de ses frères, et des gens de guerre de Samarie : Que font ces faibles Juifs ? Les laissera-t-on faire ? Sacrifieront-ils ? Et achèveront-ils tout en un jour ? Pourront-ils faire revenir à la vie les pierres des monceaux de poussière, puisqu'elles sont brûlées ?
\VS{3}Et Tobija, l'Ammonite, qui était auprès de lui, dit : Qu'ils bâtissent encore ! Si un renard monte, il rompra leur muraille de pierre !
\VS{4}Ô notre Dieu, écoute comment nous sommes méprisés ! Fais retourner leurs insultes sur leur tête, et donne-les en pillage dans un pays de captivité.
\VS{5}Ne couvre point leur iniquité, et que leur péché ne soit point effacé de devant ta face ; car ils ont irrité les bâtisseurs.
\VS{6}Nous rebâtîmes donc la muraille, et tout le mur fut achevé jusqu'à sa moitié ; et le peuple avait le cœur au travail.
\VS{7}Mais quand Sanballat et Tobija, les Arabes, les Ammonites et les Asdodiens eurent appris que la muraille de Jérusalem avait été refaite, et qu'on avait commencé à fermer les brèches, ils s'enflammèrent de colère.
\VS{8}Et ils se liguèrent tous ensemble pour venir faire la guerre contre Jérusalem, et pour les faire échouer.
\VS{9}Alors nous priâmes notre Dieu, et ayant peur d'eux, nous établîmes une garde jour et nuit pour nous défendre contre leurs attaques.
\TextTitle{Persévérance du peuple prêt à se battre à tout moment}
\VS{10}Et Juda disait : La force des ouvriers est affaiblie, et il y a beaucoup de débris, en sorte que nous ne pourrons pas bâtir la muraille.
\VS{11}Et nos ennemis disaient : Qu'ils n'en sachent rien et qu'ils ne voient rien, jusqu'à ce que nous entrions au milieu d'eux ; nous les tuerons et ferons ainsi cesser l'ouvrage.
\VS{12}Mais il arriva que les Juifs, qui habitaient près d'eux, vinrent dix fois nous avertir, de tous les lieux d'où ils se rendaient vers nous.
\VS{13}C'est pourquoi je plaçai le peuple depuis le bas, derrière la muraille, et sur des lieux élevés, secs et lumineux, selon leurs familles, avec leurs épées, leurs lances et leurs arcs.
\VS{14}Puis je regardai et m'étant levé, je dis aux chefs, aux magistrats et au reste du peuple : N'ayez point peur d'eux ! Souvenez-vous du Seigneur, qui est grand et terrible, et combattez pour vos frères, pour vos fils et pour vos filles, pour vos femmes et pour vos maisons !
\VS{15}Et quand nos ennemis entendirent que nous étions avertis, Dieu fit échouer leur projet, et nous retournâmes tous aux murailles, chacun à son travail.
\VS{16}Depuis ce jour-là, la moitié de mes serviteurs travaillait, et l'autre moitié avait des lances, des boucliers, des arcs et des cuirasses. Les gouverneurs suivaient chaque maison de Juda.
\VS{17}Ceux qui bâtissaient la muraille, et ceux qui portaient ou chargeaient les fardeaux, travaillaient chacun d'une main, et de l'autre ils tenaient une arme.
\VS{18}Car chacun de ceux qui bâtissaient avait son épée ceinte autour des reins. Et celui qui sonnait du shofar se tenait près de moi.
\VS{19}Et je dis aux chefs, aux magistrats et au reste du peuple : L'ouvrage est grand et étendu, et nous sommes séparés sur la muraille, éloignés les uns des autres.
\VS{20}En quelque lieu donc d'où vous entendrez le son du shofar, courez-y vers nous ; notre Dieu combattra pour nous\FTNT{Ex. 14:14 ; De. 1:30 ; 2 Ch. 20:29.}.
\VS{21}C'est donc ainsi que nous accomplissions le travail ; la moitié tenait des lances, depuis le lever du jour jusqu'à l'apparition des étoiles.
\VS{22}En ce temps-là, je dis aussi au peuple : Que chacun passe la nuit dans Jérusalem avec son serviteur, afin de faire la garde la nuit et de travailler le jour.
\VS{23}Et nous ne quitterons point nos vêtements, ni moi, ni mes frères, ni mes serviteurs, ni les hommes de garde qui me suivaient; chacun n'avait que ses armes et de l'eau.
\Chap{5}
\TextTitle{Cupidité des chefs dévoilée ; rétablissement de la justice}
\VerseOne{}Or il y eut un grand cri du peuple et de leurs femmes, contre les Juifs, leurs frères.
\VS{2}Et il y en avait qui disaient : Nous, nos fils et nos filles, nous sommes nombreux, et nous demandons du blé afin que nous mangions et que nous vivions.
\VS{3}Et d'autres disaient : Nous engageons nos champs, nos vignes et nos maisons, pour avoir du blé pendant la famine.
\VS{4}D'autres disaient : Nous avons emprunté de l'argent sur nos champs et sur nos vignes pour le tribut du roi.
\VS{5}Toutefois notre chair est comme la chair de nos frères, et nos fils sont comme leurs fils ; et voici, nous assujettissons nos fils et nos filles pour être esclaves ; et quelques-unes de nos filles sont déjà assujetties, et ne sont plus en notre pouvoir ; et nos champs et nos vignes sont à d'autres.
\VS{6}Et je fus très en colère quand j'entendis leur cri et ces paroles-là.
\VS{7}Je résolus dans mon cœur; puis je censurai les principaux et les magistrats, et je leur dis : Vous exigez rigoureusement ce que chacun de vous a imposé à son frère ; et je fis convoquer contre eux la grande assemblée. 
\VS{8}Et je leur dis : Nous avons racheté selon notre pouvoir nos frères Juifs, qui avaient été vendus aux nations, et vous vendriez vous-mêmes vos frères, ou nous seraient-ils vendus ? Alors ils se turent, et ne surent que dire. 
\VS{9}Et je dis : Ce que vous faites n'est pas bien. Ne voulez-vous pas marcher dans la crainte de notre Dieu, plutôt que d'être en opprobre aux nations qui sont nos ennemies ?
\VS{10}Moi aussi, mes frères et mes serviteurs, nous leur avons prêté de l'argent et du blé. Abandonnons je vous prie, cette dette !
\VS{11}Rendez-leur, je vous prie, aujourd'hui leurs champs, leurs vignes, leurs oliviers et leurs maisons ; et outre cela, le centième de l'argent, du blé, du vin, et de l'huile que vous exigez d'eux.
\VS{12}Et ils répondirent : Nous les rendrons, et nous ne leur demanderons rien ; nous ferons ce que tu dis ; alors j'appelai les sacrificateurs, et je les fis jurer qu'ils le feraient ainsi.
\VS{13}Et je secouai mon bras et je dis : Que Dieu secoue ainsi de sa maison et de son travail tout homme qui n'aura pas tenu parole, et qu'il soit ainsi secoué et vidé ! Et toute l'assemblée répondit : Amen ! Et ils louèrent Yahweh. Et le peuple fit selon cette parole.
\TextTitle{Néhémie, modèle de dévouement}
\VS{14}Et même, depuis le jour où le roi m'avait commandé d'être leur gouverneur au pays de Juda, qui est depuis la vingtième année jusqu'à la trente-deuxième année du roi Artaxerxes, pendant douze ans, moi et mes frères, nous n'avons point pris ce qui était assigné au gouverneur pour son plat.
\VS{15}Quoique les premiers gouverneurs qui avaient été avant moi, eurent chargé le peuple, et eurent pris d'eux du pain et du vin, outre quarante sicles d'argent, et qu'aussi leurs serviteurs eurent dominé sur le peuple ; mais je n'ai pas agi ainsi, à cause de la crainte de mon Dieu.
\VS{16}Et même j'ai réparé une partie de cette muraille, et nous n'avons pas acheté de champ, et tous mes serviteurs ont été assemblés là pour travailler.
\VS{17}Et outre cela, j'avais aussi à ma table les Juifs et les magistrats, au nombre de cent cinquante hommes, et ceux qui venaient vers nous des nations d'alentour.
\VS{18}On m'apprêtait chaque jour un bœuf, six moutons choisis et aussi des volailles ; et tous les dix jours on me présentait toutes sortes de vins en abondance. Malgré cela, je n'ai pas réclamé le pain du gouverneur ; parce que les travaux étaient à la charge de ce peuple.
\VS{19}Ô mon Dieu ! Souviens-toi de moi en bien, à cause de tout ce que j'ai fait pour ce peuple.
\Chap{6}
\TextTitle{Complot et mensonge contre Néhémie ; fermeté et confiance en Dieu}
\VerseOne{}Or il arriva que quand Sanballat, Tobija, et Guéschem l'Arabe, et le reste de nos ennemis apprirent que j'avais rebâti la muraille, et qu'il n'y restait aucune brèche (bien que jusqu'à ce temps-là, je n'avais pas encore mis les battants aux portes.)
\VS{2}Alors Sanballat et Guéschem envoyèrent vers moi, pour dire : Viens, et ayons ensemble une rencontre dans les villages qui sont dans la vallée d'Ono. Or ils avaient comploté de me faire du mal.
\VS{3}Mais j'envoyai des messagers vers eux pour leur dire : J'ai un grand ouvrage à faire, et je ne puis descendre. Le travail serait interrompu pendant que je le quitterais pour aller vers vous.
\VS{4}Ils m'adressèrent la même chose quatre fois ; et je leur répondis la même réponse.
\VS{5}Alors Sanballat m'envoya son serviteur pour me tenir le même discours une cinquième fois ; et il avait dans sa main une lettre ouverte.
\VS{6}Il y était écrit : On entend dire parmi les nations, et Gaschmu le dit, que vous pensez, toi et les Juifs, à vous révolter, et que c'est pour cela que tu rebâtis la muraille. Et tu vas, dit-on, devenir leur roi ;
\VS{7}Même que tu as ordonné des prophètes pour te louer dans Jérusalem, et pour dire : Il est roi de Juda. Et maintenant, on fera entendre au roi ces mêmes choses. Viens donc afin que nous consultions ensemble.
\VS{8}Et je renvoyai vers lui pour lui dire : Ce que tu dis là n'est point, mais c'est toi qui l'inventes dans ton propre coeur !
\VS{9}Car tous ces gens voulaient nous effrayer, en disant : Leurs mains relâcheront le travail, de sorte qu'il ne se fera point. Maintenant donc, ô Dieu, fortifie-moi !
\VS{10}Je me rendis à la maison de Schemaeja, fils de Delaja, fils de Mehétabeel. Il s'était enfermé et il me dit : Assemblons-nous dans la maison de Dieu, au milieu du temple et fermons les portes du temple ; car ils doivent venir pour te tuer, et ils viendront pendant la nuit pour te tuer.
\VS{11}Mais je répondis : Un homme tel que moi s'enfuirait-il ? Et quel homme tel que moi pourrait entrer dans le temple pour sauver sa vie ? Je n'y entrerai point.
\VS{12}Et voilà, je reconnus bien que Dieu ne l'avait point envoyé, mais qu'il avait prononcé cette prophétie contre moi parce que Sanballat et Tobija lui avaient donné de l'argent.
\VS{13}Car il était leur pensionnaire pour m'épouvanter, et pour m'obliger à agir de la sorte, et à commettre cette faute, afin qu'ils aient quelque mauvaise chose à me reprocher.
\VS{14}Ô mon Dieu ! Souviens-toi de Tobija et de Sanballat, et de leurs actions et aussi de Noadia, la prophétesse, et du reste des prophètes qui cherchaient à m'effrayer !
\TextTitle{Achèvement de la muraille}
\VS{15}Néanmoins, la muraille fut achevée le vingt-cinquième jour du mois d'Elul, en cinquante-deux jours.
\VS{16}Quand donc tous nos ennemis l'apprirent et qu'ils la virent, toutes les nations qui étaient autour de nous furent dans la crainte ; elles éprouvèrent une grande humiliation, et ils reconnurent que cet ouvrage s'était accompli par le secours de notre Dieu.
\VS{17}Mais aussi en ce temps-là, les chefs de Juda adressaient fréquemment des lettres à Tobija, et celles de Tobija venaient à eux.
\VS{18}Car il y en avait plusieurs en Juda qui s'étaient liés à lui par serment, parce qu'il était gendre de Schecania, fils d'Arach, et que son fils Jochanan avait pris la fille de Meschullam, fils de Bérékia.
\VS{19}Ils racontaient même du bien de lui en ma présence, et lui rapportaient mes paroles. Et Tobija envoyait des lettres pour m'effrayer.
\Chap{7}
\TextTitle{Instructions spécifiques à Hanani et Hanania}
\VerseOne{}Or après que la muraille fut rebâtie, et que j'aie mis les portes, et qu'on ait fait la revue des chantres et des Lévites ; 
\VS{2}je donnai cet ordre à Hanani, mon frère, et à Hanania, chef de la forteresse de Jérusalem ; car il était tel qu'un homme fidèle doit être, et il craignait Dieu plus que plusieurs autres ;
\VS{3}et je leur dis : Que les portes de Jérusalem ne s'ouvrent point avant la chaleur du soleil ; et pendant que les gardes seront encore là, que l'on ferme les portes, et qu'on y mette les barres ; que l'on place comme gardes les habitants de Jérusalem, chacun à son poste, et chacun devant de sa maison.
\VS{4}Or la ville était spacieuse et grande, mais il y avait peu de gens, et ses maisons n'étaient point bâties\FTNT{De. 4:27.}.
\TextTitle{Liste des familles revenues de captivité avec Zorobabel}
\VS{5}Et mon Dieu me mit à coeur d'assembler les chefs, les magistrats et le peuple, pour en faire le dénombrement selon leurs généalogies. Je trouvai le registre du dénombrement selon les généalogies de ceux qui étaient montés la première fois. Et j'y trouvai ainsi écrit :
\VS{6}Ce sont ici ceux de la province qui remontèrent de la captivité, d'entre ceux que Nebucadnetsar, roi de Babylone, avait transportés en exil, et qui retournèrent à Jérusalem et en Juda, chacun dans sa ville.
\VS{7}Ils vinrent avec Zorobabel\FTNT{Esd. 5:2.}, Josué, Néhémie, Azaria, Raamia, Nachamani, Mardochée, Bilschan, Mispéreth, Bigvaï, Nehum, et Baana. Nombre des hommes du peuple d'Israël :
\VS{8}Les fils de Pareosch, deux mille cent soixante-douze.
\VS{9}Les fils de Schephathia, trois cent soixante-douze.
\VS{10}Les fils d'Arach, six cent cinquante-deux.
\VS{11}Les fils de Pachath-Moab, des fils de Josué et de Joab, deux mille huit cent dix-huit.
\VS{12}Les fils d'Elam, mille deux cent cinquante-quatre.
\VS{13}Les fils de Zatthu, huit cent quarante-cinq.
\VS{14}Les fils de Zaccaï, sept cent soixante.
\VS{15}Les fils de Binnuï, six cent quarante-huit.
\VS{16}Les fils de Bébaï, six cent vingt-huit.
\VS{17}Les fils d'Azgad, deux mille trois cent vingt-deux.
\VS{18}Les fils d'Adonikam, six cent soixante-sept.
\VS{19}Les fils de Bigvaï, deux mille soixante-sept.
\VS{20}Les fils d'Adin, six cent cinquante-cinq.
\VS{21}Les fils d'Ather, issu d'Ezéchias, quatre-vingt-dix-huit.
\VS{22}Les fils de Haschum, trois cent vingt-huit.
\VS{23}Les fils de Betsaï, trois cent vingt-quatre.
\VS{24}Les fils de Hariph, cent douze.
\VS{25}Les fils de Gabaon, quatre-vingt-quinze.
\VS{26}Les gens de Bethléhem et de Netopha, cent quatre-vingt-huit.
\VS{27}Les gens d'Anathoth, cent vingt-huit.
\VS{28}Les gens de Beth-Azmaveth, quarante-deux.
\VS{29}Les gens de Kirjath-Jearim, de Kephira et de Beéroth, sept cent quarante-trois.
\VS{30}Les gens de Rama et de Guéba, six cent vingt et un.
\VS{31}Les gens de Micmas, cent vingt-deux.
\VS{32}Les gens de Béthel et d'Aï, cent vingt-trois.
\VS{33}Les gens de l'autre Nebo, cinquante-deux.
\VS{34}Les fils d'un autre Elam, mille deux cent cinquante-quatre.
\VS{35}Les fils de Harim, trois cent vingt.
\VS{36}Les fils de Jéricho, trois cent quarante-cinq.
\VS{37}Les fils de Lod, de Hadid et d'Ono, sept cent vingt et un.
\VS{38}Les fils de Senaa, trois mille neuf cent trente.
\TextTitle{Liste des sacrificateurs revenus de captivité}
\VS{39}Sacrificateurs : Les fils de Jedaeja, de la maison de Josué, neuf cent soixante-treize.
\VS{40}Les fils d'Immer, mille cinquante-deux.
\VS{41}Les fils de Paschhur, mille deux cent quarante-sept.
\VS{42}Les fils de Harim, mille dix-sept.
\TextTitle{Liste des Lévites revenus de captivité}
\VS{43}Lévites : Les fils de Josué et de Kadmiel, d'entre les fils de Hodva, soixante quatorze.
\VS{44}Chantres : Les fils d'Asaph, cent quarante-huit.
\VS{45}Portiers : Les fils de Schallum, les fils d'Ather, les fils de Thalmon, les fils d'Akkub, les fils de Hathitha, les fils de Schobaï, cent trente-huit.
\TextTitle{Liste des Néthiniens revenus de captivité}
\VS{46}Néthiniens : Les fils de Tsicha, les fils de Hasupha, les fils de Thabbaoth,
\VS{47}les fils de Kéros, les fils de Sia, les fils de Padon,
\VS{48}les fils de Lebana, les fils de Hagaba, les fils de Salmaï,
\VS{49}les fils de Hanan, les fils de Guiddel, les fils de Gachar,
\VS{50}les fils de Reaja, les fils de Retsin, les fils de Nekoda,
\VS{51}les fils de Gazzam, les fils d'Uzza, les fils de Paséach,
\VS{52}les fils de Bésaï, les fils de Mehunim, les fils de Nephischsim,
\VS{53}les fils de Bakbuk, les fils de Hakupha, les fils de Harhur,
\VS{54}les fils de Batslith, les fils de Mehida, les fils de Harscha,
\VS{55}les fils de Barkos, les fils de Sisera, les fils de Thamach,
\VS{56}les fils de Netsiach, les fils de Hathipha.
\TextTitle{Liste des fils des serviteurs de Salomon revenus de captivité}
\VS{57}Fils des serviteurs de Salomon : Les fils de Sothaï, les fils de Sophéreth, les fils de Perida,
\VS{58}les fils de Jaala, les fils de Darkon, les fils de Guiddel,
\VS{59}les fils de Schephathia, les fils de Hatthil, les fils de Pokéreth-Hatsebaïm, les fils d'Amon.
\VS{60}Tous les Néthiniens, et les fils des serviteurs de Salomon, étaient trois cent quatre-vingt-douze.
\VS{61}Voici ceux qui montèrent de Thel-Mélach, de Thel-Harscha, de Kerub-Addon et d'Immer, lesquels ne purent montrer la maison de leurs pères, ni leur race, pour prouver qu'ils étaient d'Israël.
\VS{62}Les fils de Delaja, les fils de Tobija, les fils de Nekoda, six cent quarante-deux.
\TextTitle{Liste des sacrificateurs exclus de la sacrificature}
\VS{63}Et les sacrificateurs : Les fils de Hobaja, les fils d'Hakkots, les fils de Barzillaï, qui prit pour femme une des filles de Barzillaï, le Galaadite, et qui fut appelé de leur nom.
\VS{64}Ils cherchèrent leur registre généalogique, mais ils n'y furent point trouvés ; c'est pourquoi ils furent exclus de la sacrificature.
\VS{65}Et le gouverneur leur dit de ne pas manger des choses très saintes, jusqu'à ce que le sacrificateur eût consulté l'urim et le thummim\FTNT{Ex. 28:30.}.
\TextTitle{Somme des Israélites revenus de captivité}
\VS{66}Toute l'assemblée réunie était de quarante-deux mille trois cent soixante ;
\VS{67}sans leurs serviteurs et leurs servantes, qui étaient sept mille trois cent trente-sept ; et ils avaient deux cent quarante-cinq chantres ou chanteuses.
\TextTitle{Dons des fils d'Israël pour le trésor}
\VS{68}Ils avaient sept cent trente-six chevaux, deux cent quarante-cinq mulets ;
\VS{69}quatre cent trente-cinq chameaux et six mille sept cent vingt ânes.
\VS{70}Or quelques-uns des chefs des pères firent des dons pour l'ouvrage. Le gouverneur donna au trésor mille drachmes d'or, cinquante coupes, cinq cent trente tuniques de sacrificateurs.
\VS{71}Quelques autres d'entre les chefs des pères donnèrent pour le trésor de l'ouvrage vingt mille drachmes d'or et deux mille deux cent mines d'argent.
\VS{72}Le reste du peuple donna vingt mille drachmes d'or, deux mille mines d'argent et soixante-sept tuniques de sacrificateurs.
\VS{73}Et ainsi les sacrificateurs, les Lévites, les portiers, les chantres, quelques-uns du peuple, les Néthiniens, et tous ceux d'Israël habitèrent dans leurs villes. Ainsi, quand le septième mois approcha, les enfants d'Israël étaient dans leurs villes.
\Chap{8}
\TextTitle{Lecture du livre de la loi, conviction de péché du peuple}
\VerseOne{}Or tout le peuple s'assembla, comme un seul homme, sur la place qui est devant la porte des eaux. Et ils dirent à Esdras, le scribe, d'apporter le livre de la loi de Moïse, que Yahweh avait ordonnée à Israël.
\VS{2}Et ainsi le premier jour du septième mois Esdras, le sacrificateur apporta la loi devant l'assemblée, composée d'hommes et de femmes, et de tous ceux qui étaient capables d'entendre, afin qu'on l'écoutât.
\VS{3}Et il lut dans le livre, sur la place qui était devant la porte des eaux, depuis le matin jusqu'au milieu du jour, en présence des hommes et des femmes, et de ceux qui étaient capables d'entendre. Et les oreilles de tout le peuple étaient attentives à la lecture du livre de la loi.
\VS{4}Ainsi Esdras, le scribe, était debout sur une tour bâtie de bois, qu'on avait dressée pour cela. Il avait auprès de lui, à sa droite, Matthithia, Schéma, Anaja, Urie, Hilkija et Maaséja ; et à sa gauche étaient Pedaja, Mischaël, Malkija, Haschum, Haschbaddana, Zacharie, et Meschullam.
\VS{5}Esdras ouvrit le livre devant les yeux de tout le peuple ; car il était au-dessus de tout le peuple ; et sitôt qu'il l'eut ouvert, tout le peuple se tint debout.
\VS{6}Puis Esdras bénit Yahweh, le grand Dieu ; et tout le peuple répondit en élevant leurs mains: Amen ! Amen ! Et ils s'inclinèrent et se prosternèrent devant Yahweh, le visage contre terre.
\VS{7}Aussi Josué, Bani, Schérébia, Jamin, Akkub, Schabbethaï, Hodija, Maaséja, Kelitha, Azaria, Jozabad, Hanan, Pelaja, et les Lévites, faisaient comprendre la loi au peuple, et le peuple se tenait à sa place.
\VS{8}Et ils lisaient dans le livre de la loi de Dieu, ils l'expliquaient et en donnaient l'intelligence, la faisant comprendre par l'Ecriture elle-même.
\VS{9}Or Néhémie, qui est le gouverneur, Esdras, le sacrificateur et le scribe, et les Lévites qui instruisaient le peuple dirent à tout le peuple : Ce jour est consacré à Yahweh, notre Dieu ; ne soyez pas dans les lamentations, et ne pleurez point ! Car tout le peuple pleurait en entendant les paroles de la loi.
\VS{10}Puis on leur dit : Allez, mangez des viandes grasses, et buvez du vin doux ; et envoyez-en des portions à ceux qui n'ont rien de prêt ; car ce jour est consacré à notre Seigneur. Ne soyez donc point tristes, car la joie de Yahweh est votre force.
\VS{11}Et les Lévites faisaient faire silence parmi tout le peuple, en disant : Taisez-vous, car ce jour est saint, et ne vous affligez point.
\VS{12}Ainsi tout le peuple s'en alla pour manger et pour boire, pour envoyer des portions, et pour faire une grande réjouissance, parce qu'ils avaient bien compris les paroles qu'on leur avait fait connaître.
\TextTitle{Célébration de la fête des tabernacles}
\VS{13}Et le second jour, les chefs des pères de tout le peuple, les sacrificateurs et les Lévites, s'assemblèrent auprès d'Esdras, le scribe, pour sagement comprendre les paroles de la loi.
\VS{14}Et ils trouvèrent écrit dans la loi que Yahweh avait ordonnée par Moïse, que les enfants d'Israël devaient habiter sous des tentes\FTNT{Voir les sept fêtes de Yahweh en Lé. 23.} pendant la fête solennelle au septième mois.
\VS{15}Ce qu'ils firent savoir et qu'ils publièrent dans toutes leurs villes et à Jérusalem, en disant : Allez sur la montagne, et apportez des rameaux d'oliviers, et des rameaux d'autres arbres huileux, des rameaux de myrte, des rameaux de palmier, et des rameaux d'arbres touffus, afin de faire des tentes, selon ce qui est écrit.
\VS{16}Alors le peuple alla et apporta des rameaux. Ils se firent des tentes, chacun sur son toit, dans les cours de leurs maisons, et dans les parvis de la maison de Dieu, sur la place de la porte des eaux, et sur la place de la porte d'Ephraïm.
\VS{17}Ainsi toute l'assemblée de ceux qui étaient revenus de la captivité fit des tentes, et ils habitèrent sous ces tentes. Or les enfants d'Israël n'en avaient point fait de telles depuis les jours de Josué, fils de Nun, jusqu'à ce jour ; et il y eut une très grande joie.
\VS{18}On lut dans le livre de la loi de Dieu chaque jour, depuis le premier jour jusqu'au dernier. On célébra la fête pendant sept jours, et il y eut une assemblée solennelle au huitième jour, comme cela est ordonné.
\Chap{9}
\TextTitle{Confession, jeune et prière du peuple}
\VerseOne{}Et le vingt-quatrième jour du même mois, les enfants d'Israël s'assemblèrent, jeûnant, revêtus de sacs, et ayant de la terre sur eux.
\VS{2}Et la race d'Israël se sépara de tous les étrangers, et ils se présentèrent confessant leurs péchés et les iniquités de leurs pères.
\VS{3}Ils se levèrent donc à leur place, et on lut dans le livre de la loi de Yahweh, leur Dieu, pendant un quart de la journée, et pendant un autre quart, ils faisaient confession de leurs péchés, et se prosternaient devant Yahweh, leur Dieu.
\TextTitle{Prière des Lévites, alliance avec Yahweh}
\VS{4}Josué, Bani, Kadmiel, Schebania, Bunni, Schérébia, Bani et Kenani se levèrent sur le lieu qu'on avait élevé pour les Lévites, et crièrent à haute voix à Yahweh, leur Dieu.
\VS{5}Et les Lévites Josué, Kadmiel, Bani, Haschabnia, Schérébia, Hodija, Schebania et Pethachja, dirent : Levez-vous, bénissez Yahweh, votre Dieu, d'éternité en éternité ! Que l'on bénisse ton Nom glorieux, qui est au-dessus de toute bénédiction et de toute louange !
\VS{6}Toi seul, Yahweh, tu as fait les cieux, les cieux des cieux, et toute leur armée ; la terre, et tout ce qui y est ; les mers, et toutes les choses qui y vivent. Tu donnes la vie à toutes ces choses, et l'armée des cieux se prosterne devant toi.
\VS{7}Tu es Yahweh, notre Dieu, qui as choisi Abram, et qui l'as fait sortir d'Ur en Chaldée, et qui lui as donné le nom d'Abraham\FTNT{Ge. 11:31 ; Ge. 17:5.}.
\VS{8}Tu trouvas son coeur fidèle devant toi, et tu traitas avec lui cette alliance que tu donneras à sa postérité le pays des Cananéens, des Héthiens, des Amoréens, des Phéréziens, des Jébusiens, et des Guirgasiens. Et tu as accompli ce que tu as promis, parce que tu es juste.
\VS{9}Car tu vis l'affliction de nos pères en Egypte et tu entendis leurs cris près de la Mer Rouge\FTNT{Ex. 2:23-25.}.
\VS{10}Tu fis des miracles et des prodiges sur Pharaon et sur tous ses serviteurs, et sur tout le peuple de son pays ; parce que tu connus qu'ils s'étaient orgueilleusement élevés contre eux, et tu t'es acquis un renom, tel qu'il paraît aujourd'hui.
\VS{11}Tu fendis aussi la mer devant eux, et ils passèrent à sec au milieu de la mer ; et tu jetas dans l'abîme ceux qui les poursuivaient, comme une pierre dans les eaux violentes.
\VS{12}Tu les fis marcher de jour par la colonne de nuée, et de nuit par la colonne de feu, pour les éclairer dans le chemin par où ils devaient aller\FTNT{Ex. 13:21.}.
\VS{13}Tu descendis sur la montagne de Sinaï, tu parlas avec eux du haut des cieux, tu leur donnas des ordonnances justes et des lois de vérité, des statuts et des commandements bons.
\VS{14}Tu leur fis connaître ton saint sabbat\FTNT{Ge 2:1-3 ; Ex. 20:8-11.} ; et tu leur donnas les commandements, les statuts, et la loi par Moïse, ton serviteur.
\VS{15}Tu leur donnas aussi, du haut des cieux, du pain quand ils avaient faim, et tu fis sortir de l'eau du rocher quand ils avaient soif\FTNT{Ex. 16:13-36 ; No. 20 : 8.}. Et tu leur dis d'entrer et de posséder le pays que tu avais juré de leur donner.
\VS{16}Mais nos pères s'élevèrent orgueilleusement et raidirent leur cou. Ils n'écoutèrent point tes commandements.
\VS{17}Ils refusèrent d'écouter et ne se souvinrent point des merveilles que tu avais faites en leur faveur. Mais ils raidirent leur cou, et par leur rébellion, ils s'attribuèrent un chef pour retourner à leur servitude. Mais toi, tu es un Dieu qui pardonne, miséricordieux, compatissant, lent à la colère et abondant en bonté, et tu ne les abandonnas pas.
\VS{18}Et quand ils se firent un veau en métal fondu et qu'ils dirent : Voici ton Dieu qui t'a fait sortir hors d'Egypte, et qu'ils te firent de grands outrages\FTNT{Ex. 32:1-14.} ;
\VS{19}dans ton immense miséricorde, tu ne les abandonnas pourtant pas dans le désert ; et la colonne de nuée ne se retira point pour les conduire le jour par le chemin, ni la colonne de feu la nuit, pour les éclairer dans le chemin par lequel ils devaient aller.
\VS{20}Tu leur donnas ton bon Esprit pour les rendre sages ; tu ne retiras point ta manne de leur bouche, et tu leur donnas de l'eau pour leur soif.
\VS{21}Tu les nourris ainsi quarante ans au désert, en sorte que rien ne leur manqua. Leurs vêtements ne s'usèrent point, et leurs pieds ne s'enflèrent point.
\VS{22}Tu leur donnas les royaumes et les peuples, dont tu partageas entre eux les contrées ; et ils possédèrent le pays de Sihon, le pays du roi de Hesbon, et le pays d'Og, roi de Basan.
\VS{23}Et tu multiplias leurs fils comme les étoiles des cieux, et les fis entrer au pays dont tu avais dit à leurs pères qu'ils y entreraient pour le posséder.
\VS{24}Ainsi leurs fils y entrèrent et possédèrent le pays ; tu humilias devant eux les habitants du pays, les Cananéens, et les livras entre leurs mains, eux et leurs rois, et les peuples du pays, afin qu'ils en fissent selon leur volonté.
\VS{25}Ils prirent les villes fortifiées et la terre grasse, ils possédèrent les maisons remplies de toutes sortes de biens, les puits qu'on avait creusés, les vignes, les oliviers, et les arbres fruitiers en abondance ; ils mangèrent, ils se rassasièrent ; ils s'engraissèrent et ils vécurent dans les délices de ta grande bonté.
\VS{26}Mais ils se rebellèrent et se révoltèrent contre toi. Ils jetèrent ta loi derrière leur dos, ils tuèrent tes prophètes qui les avertissaient pour les ramener à toi, et ils te firent de grands outrages.
\VS{27}C'est pourquoi tu les donnas aux mains de leurs ennemis, qui les opprimèrent. Mais au temps de leur détresse, ils crièrent à toi, et tu les entendis des cieux ; et selon ta grande miséricorde, tu leur donnas des libérateurs qui les délivrèrent de la main de leurs ennemis.
\VS{28}Mais dès qu'ils eurent du repos, ils recommencèrent à faire le mal devant toi. Alors tu les abandonnas entre les mains de leurs ennemis, qui dominèrent sur eux. Puis ils revinrent et crièrent vers toi, et tu les entendis des cieux. Ainsi tu les délivras selon tes miséricordes, plusieurs fois, et en divers temps.
\VS{29}Et tu les exhortas à revenir à ta loi, mais ils s'élevèrent orgueilleusement et n'écoutèrent pas tes commandements ; ils péchèrent contre tes ordonnances, qui font vivre l'homme qui les observe. Ils tirèrent l'épaule en arrière, raidirent leur cou et n'écoutèrent pas.
\VS{30}Tu les supportas patiemment plusieurs années, et tu les avertissais par ton Esprit, par la main de tes prophètes ; mais ils ne prêtèrent point l'oreille. C'est pourquoi tu les livras entre les mains des peuples des pays étrangers.
\VS{31}Néanmoins, dans ta grande miséricorde, tu ne les anéantis pas et tu ne les abandonnas pas ; car tu es un Dieu compatissant et miséricordieux.
\VS{32}Et maintenant donc, ô notre Dieu ! Grand, puissant et terrifiant, qui garde ton alliance et la miséricorde ; ne regarde pas comme peu de chose cette affliction qui nous est arrivée, à nous, à nos rois, à nos chefs, à nos sacrificateurs, à nos prophètes, à nos pères et à tout ton peuple, depuis le temps des rois d'Assyrie jusqu'à aujourd'hui.
\VS{33}Tu as été juste dans toutes les choses qui nous sont arrivées ; car tu as agi avec fidélité, mais nous, nous avons agi méchamment.
\VS{34}Nos rois, nos chefs, nos sacrificateurs et nos pères n'ont point pratiqué ta loi et n'ont point été attentifs à tes commandements ni à tes témoignages par lesquels tu les as avertis.
\VS{35}Ils ne t'ont point servi durant leur règne ni durant les grands biens que tu leur as faits, même dans le pays vaste et riche que tu leur avais donné pour être à leur disposition, et ils ne se sont point détournés de leurs mauvaises oeuvres.
\VS{36}Voici, nous sommes aujourd'hui esclaves ! Sur la terre que tu as donnée à nos pères pour en manger le fruit et les biens ; voici, nous y sommes esclaves !
\VS{37}Elle rapporte ses produits en abondance pour les rois que tu as établis sur nous à cause de nos péchés, et qui dominent sur nos corps et sur nos bêtes, à leur volonté, de sorte que nous sommes dans une grande angoisse !
\VS{38}C'est pourquoi, à cause de toutes ces choses, nous contractâmes une alliance et nous l'écrivîmes ; et les chefs d'entre nous, nos Lévites et nos sacrificateurs y apposèrent leur sceau.
\Chap{10}
\TextTitle{Liste des contractants et termes de l'alliance}
\VerseOne{}Voici ceux qui apposèrent leur sceau. Néhémie, qui est le gouverneur, fils de Hacalia, et Sédécias.
\VS{2}Seraja, Azaria, Jérémie,
\VS{3}Paschhur, Amaria, Malkija,
\VS{4}Hattusch, Schebania, Malluc,
\VS{5}Harim, Merémoth, Abdias,
\VS{6}Daniel, Guinnethon, Baruc,
\VS{7}Meschullam, Abija, Mijamin,
\VS{8}Maazia, Bilgaï et Schemaeja. Ce sont les sacrificateurs.
\VS{9}Des Lévites : Josué, fils d'Azania, Binnuï d'entre les fils de Hénadad, et Kadmiel.
\VS{10}Et leurs frères, Schebania, Hodija, Kelitha, Pelaja, Hanan,
\VS{11}Michée, Rehob, Haschabia.
\VS{12}Zaccur, Schérébia, Schebania,
\VS{13}Hodija, Bani et Beninu.
\VS{14}Des chefs du peuple : Pareosch, Pachath-Moab, Elam, Zatthu, Bani,
\VS{15}Bunni, Azgad, Bébaï,
\VS{16}Adonija, Bigvaï, Adin,
\VS{17}Ather, Ezéchias, Azzur,
\VS{18}Hodija, Haschum, Betsaï,
\VS{19}Hariph, Anathoth, Nébaï,
\VS{20}Magpiasch, Meschullam, Hézir,
\VS{21}Meschézabeel, Tsadok, Jaddua,
\VS{22}Pelathia, Hanan, Anaja,
\VS{23}Hosée, Hanania, Haschub,
\VS{24}Hallochesch, Pilcha, Schobek,
\VS{25}Rehum, Haschabna, Maaséja,
\VS{26}Achija, Hanan, Anan,
\VS{27}Malluc, Harim et Baana.
\VS{28}Quant au reste du peuple, les sacrificateurs, les Lévites, les portiers, les chantres, les Néthiniens et tous ceux qui s'étaient séparés des peuples de ces pays pour suivre la loi de Dieu, leurs femmes, leurs fils et leurs filles, tous ceux qui étaient capables de connaissance et d'intelligence,
\VS{29}se joignirent à leurs frères les plus considérables d'entre eux. Ils s'engagèrent par serment et jurèrent de marcher dans la loi de Dieu, qui avait été donnée par Moïse, serviteur de Dieu ; de garder et faire tous les commandements de Yahweh, notre Seigneur, ses jugements et ses ordonnances ;
\VS{30}de ne pas donner nos filles aux peuples du pays, et de ne pas prendre leurs filles pour nos fils ;
\VS{31}de ne rien prendre le jour du sabbat, ou tel autre jour consacré, des peuples du pays qui apporteraient des marchandises et toutes sortes de denrées, le jour du sabbat, pour les vendre, d'abandonner la septième année et de faire remise de toute dette.
\VS{32}Nous fîmes aussi des ordonnances, nous chargeant de donner chaque année le tiers d'un sicle, pour le service de la maison de notre Dieu,
\VS{33}pour les pains de proposition, pour l'offrande perpétuelle et pour l'holocauste perpétuel ; pour ceux des sabbats, des nouvelles lunes et des fêtes ; pour les choses consacrées, pour les sacrifices d'expiation afin de faire propitiation pour Israël ; et pour toute l'oeuvre de la maison de notre Dieu.
\VS{34}Nous tirâmes au sort, pour l'offrande du bois, tant les sacrificateurs et les Lévites, que le peuple, afin de l'amener dans la maison de notre Dieu, selon les maisons de nos pères, et dans les temps fixés, d'année en année, pour le brûler sur l'autel de Yahweh, notre Dieu, ainsi qu'il est écrit dans la loi.
\VS{35}Nous décidâmes aussi d'apporter dans la maison de Yahweh, d'année en année, les premiers fruits de notre terre, et les prémices de tous les fruits de tous les arbres ;
\VS{36}d'amener les premiers-nés de nos fils, et de nos bêtes, comme il est écrit dans la loi ; et d'amener dans la maison de notre Dieu, aux sacrificateurs qui font le service dans la maison de notre Dieu, les premiers-nés de nos boeufs et de notre menu bétail;
\VS{37}d'apporter les prémices de notre pâte, nos offrandes, les fruits de tous les arbres, le vin, et l'huile aux sacrificateurs, dans les chambres de la maison de notre Dieu, et la dîme de notre terre aux Lévites, et que les Lévites prendraient les dîmes dans toutes les villes agricoles.
\VS{38}Le sacrificateur, fils d'Aaron, sera avec les Lévites, lorsque les Lévites paieront la dîme\FTNT{Il est question de la dîme des Lévites (No. 18:24 ; De. 14:28-29).}; et les Lévites apporteront la dîme de la dîme\FTNT{Il s'agit ici de la dîme de la dîme que les Lévites donnaient aux sacificateurs. Elle était apportée aux magasins du temple. Voir commentaires en No. 18:21 et Mal. 3:10.} à la maison de notre Dieu, dans les chambres de la maison où sont les magasins\FTNT{Le mot hébreu « owtsar » (trésor) signifie aussi magasin (Né. 12:44 ; Né. 13:12. Né. 13:13).}.
\VS{39}Car les enfants d'Israël et les fils de Lévi apporteront dans ces chambres les offrandes du blé, du vin et de l'huile ; là sont les ustensiles du sanctuaire, et les sacrificateurs qui font le service, les portiers, et les chantres. Et nous n'abandonnâmes point la maison de notre Dieu.
\Chap{11}
\TextTitle{Les habitants de Jérusalem}
\VerseOne{}Les chefs du peuple demeurèrent à Jérusalem. Mais tout le reste du peuple tira au sort, afin qu'un sur dix vînt habiter à Jérusalem, la ville sainte, et que les neuf autres parties demeurassent dans les autres villes.
\VS{2}Et le peuple bénit tous ceux qui se présentèrent volontairement pour habiter à Jérusalem.
\VS{3}Voici les chefs de la province qui habitèrent à Jérusalem ; les autres s'étant établis dans les villes de Juda, chacun dans sa propriété, selon sa ville, Israélites, sacrificateurs, Lévites, Néthiniens, et les fils des serviteurs de Salomon.
\VS{4}A Jérusalem habitèrent donc des fils de Juda et des fils de Benjamin. Des fils de Juda : Athaja, fils d'Ozias, fils de Zacharie, fils d'Amaria, fils de Schephathia, fils de Mahalaleel, d'entre les fils de Pérets,
\VS{5}et Maaséja, fils de Baruc, fils de Col-Hozé, fils de Hazaja, fils d'Adaja, fils de Jojarib, fils de Zacharie, fils de Schiloni.
\VS{6}Total des fils de Pérets, qui s'établirent à Jérusalem : Quatre cent soixante-huit vaillants hommes.
\VS{7}Voici les fils de Benjamin : Sallu, fils de Meschullam, fils de Joëd, fils de Pedaja, fils de Kolaja, fils de Maaséja, fils d'Ithiel, fils d'Esaïe,
\VS{8}et après lui, Gabbaï et Sallaï : Neuf cent vingt-huit.
\VS{9}Joël, fils de Zicri, était leur chef ; et Juda, fils de Senua, était le second chef de la ville.
\VS{10}Des sacrificateurs : Jedaeja, fils de Jojarib, Jakin,
\VS{11}Seraja, fils de Hilkija, fils de Meschullam, fils de Tsadok, fils de Merajoth, fils d'Achithub, prince de la maison de Dieu,
\VS{12}et leurs frères, faisant le service de la maison : Huit cent vingt-deux. Adaja, fils de Jerocham, fils de Pelalia, fils d'Amtsi, fils de Zacharie, fils de Paschhur, fils de Malkija,
\VS{13}et ses frères, chefs des pères : Deux cent quarante-deux ; et Amaschsaï, fils d'Azareel, fils d'Achzaï, fils de Meschillémoth, fils d'Immer,
\VS{14}et leurs frères, forts et vaillants: Cent vingt-huit. Zabdiel, fils de Guedolim, était leur chef.
\VS{15}Des Lévites : Schemaeja, fils de Haschub, fils d'Azrikam, fils de Haschabia, fils de Bunni,
\VS{16}Schabbethaï et Jozabad chargés des travaux extérieurs pour la maison de Dieu, étant d'entre les chefs des Lévites ;
\VS{17}Matthania, fils de Michée, fils de Zabdi, fils d'Asaph, était le chef qui commençait le premier à chanter les louanges dans la prière, et Bakbukia, le second parmi ses frères, puis Abda, fils de Schammua, fils de Galal, fils de Jeduthun.
\VS{18}Total des Lévites dans la ville sainte : Deux cent quatre-vingt-quatre.
\VS{19}Et les portiers : Akkub, Thalmon, et leurs frères qui gardaient les portes : Cent soixante-douze.
\TextTitle{Les habitants des autres villes}
\VS{20}Le reste d'Israël, des sacrificateurs et des Lévites, fut dans toutes les villes de Juda, chacun dans son héritage.
\VS{21}Mais les Néthiniens habitèrent sur la colline ; et Tsicha et Guischpa étaient leurs chefs.
\VS{22}Celui qui avait la charge des Lévites à Jérusalem était Uzzi, fils de Bani, fils de Haschabia, fils de Matthania, fils de Michée, d'entre les fils d'Asaph, chantres, pour l'ouvrage de la maison de Dieu ;
\VS{23}car il y avait un commandement du roi à leur égard, et il y avait chaque jour un salaire assuré pour les chantres.
\VS{24}Pethachja, fils de Meschézabeel, d'entre les fils de Zérach, fils de Juda, était commissaire du roi pour toutes les affaires du peuple.
\VS{25}Dans les villages et leurs territoires, quelques-uns des fils de Juda habitèrent à Kirjath-Arba, et dans les lieux de son ressort ; à Dibon, et dans les lieux de son ressort ; à Jekabtseel, et dans les villages de son ressort,
\VS{26}à Jéschua, à Molada, à Beth-Paleth,
\VS{27}à Hatsar-Schual, à Beer-Schéba, et dans les lieux de son ressort,
\VS{28}à Tsiklag, à Mecona, et dans les lieux de son ressort,
\VS{29}à En-Rimmon, à Tsorea, à Jarmuth,
\VS{30}à Zanoach, à Adullam, et dans leurs villages, à Lakis et dans ses territoires, à Azéka et dans les lieux de son ressort. Ils habitèrent depuis Beer-Schéba jusqu'à la vallée de Hinnom.
\VS{31}Et les fils de Benjamin habitèrent depuis Guéba à Micmasch, à Ajja, à Béthel, et dans les lieux de son ressort,
\VS{32}à Anathoth, à Nob, à Hanania,
\VS{33}à Hatsor, à Rama, à Guitthaïm,
\VS{34}à Hadid, à Tseboïm, à Neballath,
\VS{35}à Lod, et à Ono, la vallée des ouvriers.
\VS{36}D'entre les Lévites, des classes de Juda se rattachèrent à Benjamin.
\Chap{12}
\TextTitle{Les sacrificateurs et les Lévites montés avec Zorobabel}
\VerseOne{}Voici les sacrificateurs et les Lévites qui montèrent avec Zorobabel, fils de Schealthiel, et avec Josué : Seraja, Jérémie, Esdras,
\VS{2}Amaria, Malluc, Hattusch,
\VS{3}Schecania, Rehum, Merémoth,
\VS{4}Iddo, Guinnethoï, Abija,
\VS{5}Mijamin, Maadia, Bilga,
\VS{6}Schemaeja, Jojarib, Jedaeja,
\VS{7}Sallu, Amok, Hilkija, Jedaeja. Ce furent là les chefs des sacrificateurs, et de leurs frères, du temps de Josué.
\VS{8}Lévites : Josué, Binnuï, Kadmiel, Schérébia, Juda, Matthania, qui dirigeait les louanges, lui et ses frères.
\VS{9}Bakbukia et Unni, leurs frères, étaient avec eux pour la surveillance.
\TextTitle{Les fils des sacrificateurs}
\VS{10}Josué engendra Jojakim, Jojakim engendra Eliaschib, Eliaschib engendra Jojada,
\VS{11}Jojada engendra Jonathan, et Jonathan engendra Jaddua.
\VS{12}Au temps de Jojakim, étaient sacrificateurs, chefs des pères : Pour Seraja, Meraja ; pour Jérémie, Hanania ;
\VS{13}pour Esdras, Meschullam ; pour Amaria, Jochanan ;
\VS{14}pour Meluki, Jonathan ; pour Schebania, Joseph ;
\VS{15}pour Harim, Adna ; pour Merajoth, Helkaï ;
\VS{16}pour Iddo, Zacharie ; pour Guinnethon, Meschullam ;
\VS{17}pour Abija, Zicri ; pour Minjamin et Moadia, Pilthaï ;
\VS{18}pour Bilga, Schammua ; pour Schemaeja, Jonathan ;
\VS{19}pour Jojarib, Matthnaï ; pour Jedaeja, Uzzi ;
\VS{20}pour Sallaï, Kallaï ; pour Amok, Eber ;
\VS{21}pour Hilkija, Haschabia ; pour Jedaeja, Nethaneel.
\TextTitle{Les chefs des fils de Lévi}
\VS{22}Au temps d'Eliaschib, de Jojada, de Jochanan et de Jaddua, les Lévites, chefs de famille, et les sacrificateurs, furent inscrits sous le règne de Darius, le Perse.
\VS{23}Les fils de Lévi, chefs des pères, furent enregistrés dans le livre des Chroniques jusqu'au temps de Jochanan, fils d'Eliaschib.
\VS{24}Les chefs des Lévites, Haschabia, Schérébia, et Josué, fils de Kadmiel, et leurs frères, étaient vis-à-vis d'eux, pour louer et célébrer, selon l'ordre de David, homme de Dieu.
\VS{25}Matthania, Bakbukia, Abdias, Meschullam, Thalmon, et Akkub, les portiers, faisaient la garde au seuil des portes.
\VS{26}Ce fut du temps de Jojakim, fils de Josué, fils de Jotsadak, et du temps de Néhémie, le gouverneur, et d'Esdras, sacrificateur et scribe.
\TextTitle{La dédicace de la muraille de Jérusalem}
\VS{27}Lors de la dédicace de la muraille de Jérusalem, on envoya chercher les Lévites de tous les lieux où ils étaient, pour les faire venir à Jérusalem, afin de célébrer la dédicace avec joie, par des louanges, et par des chants sur des cymbales, des luths et des harpes.
\VS{28}Les fils des chantres se rassemblèrent des plaines aux alentours de Jérusalem, des villages des Nethophatiens,
\VS{29}de Beth-Guilgal, et des territoires de Guéba et d'Azmaveth ; car les chantres s'étaient bâtis des villages aux alentours de Jérusalem.
\VS{30}Les sacrificateurs et les Lévites se purifièrent, et ils purifièrent le peuple, les portes et la muraille.
\VS{31}Puis je fis monter sur la muraille les chefs de Juda, et j'établis deux grands chœurs. Le premier se mit en marche du côté droit sur la muraille, vers la porte du fumier.
\VS{32}Et après eux marchait Hosée, avec la moitié des chefs de Juda,
\VS{33}Azaria, Esdras, Meschullam,
\VS{34}Juda, Benjamin, Schemaeja et Jérémie,
\VS{35}des fils des sacrificateurs avec les trompettes, Zacharie, fils de Jonathan, fils de Schemaeja, fils de Matthania, fils de Michée, fils de Zaccur, fils d'Asaph,
\VS{36}et ses frères, Schemaeja, Azareel, Milalaï, Guilalaï, Maaï, Nethaneel, Juda, et Hanani, avec les instruments des cantiques de David, homme de Dieu. Esdras, le scribe, marchait devant eux.
\VS{37}A la porte de la source, qui était vis-à-vis d'eux, ils montèrent aux marches de la cité de David, par la montée de la muraille, depuis la maison de David, jusqu'à la porte des eaux, vers l'orient.
\VS{38}Le second choeur de ceux qui chantaient les louanges allait à l'opposé. J'allais après lui, avec l'autre moitié du peuple, allant sur la muraille. Passant par-dessus la tour des fours, jusqu'à la muraille large ;
\VS{39}puis vers la porte d'Ephraïm, vers la vieille porte, vers la porte des poissons, la tour de Hananeel, et la tour de Méa, jusqu'à la porte des brebis. Et l'on s'arrêta à la porte de la prison.
\VS{40}Les deux choeurs s'arrêtèrent dans la maison de Dieu ; moi aussi, avec les magistrats qui étaient avec moi,
\VS{41}et les sacrificateurs Eliakim, Maaséja, Minjamin, Michée, Eljoénaï, Zacharie, Hanania, avec les trompettes,
\VS{42}et Maaséja, Schemaeja, Eléazar, Uzzi, Jochanan, Malkija, Elam et Ezer. Puis les chantres, desquels Jizrachja avait la charge, se firent entendre.
\VS{43}On offrit ce jour-là de nombreux sacrifices, et on se réjouit, parce que Dieu leur avait donné un grand sujet de joie. Les femmes et les enfants se réjouirent aussi ; et la joie de Jérusalem fut entendue au loin.
\TextTitle{Les sacrificateurs et les Lévites à leur poste}
\VS{44}En ce jour-là, on établit des hommes sur les chambres des trésors, des offrandes, des prémices et des dîmes ; pour rassembler du territoire des villes les portions ordonnées par la loi aux sacrificateurs et aux Lévites. Car Juda se réjouissait de ce que les sacrificateurs et de ce que les Lévites étaient à leur poste,
\VS{45}et parce qu'ils avaient gardé la charge qui leur avait été donnée de la part de leur Dieu, et la charge de la purification. Les chantres et les portiers remplissaient aussi leurs fonctions, selon le commandement de David, et de Salomon, son fils.
\VS{46}Car autrefois, du temps de David et d'Asaph, on avait établi des chefs de chantres et des cantiques de louange et de reconnaissance à Dieu.
\VS{47}Tout Israël, du temps de Zorobabel et de Néhémie, donna les portions des chantres et des portiers, jour par jour, et les consacraient aux Lévites, et les Lévites les consacraient aux fils d'Aaron.
\Chap{13}
\TextTitle{Lecture du livre de Moïse, séparation d'avec les étrangers}
\VerseOne{}En ce temps-là, on lut aux oreilles du peuple dans le livre de Moïse, et l'on y trouva écrit que les Ammonites et les Moabites ne devaient jamais entrer dans l'assemblée de Dieu,
\VS{2}parce qu'ils n'étaient pas venus au-devant des enfants d'Israël avec du pain et de l'eau ; et qu'ils avaient loué  Balaam\FTNT{Balaam : voir No. 22,23 et 24.} contre eux pour les maudire ; mais notre Dieu avait changé la malédiction en bénédiction.
\VS{3} C'est pourquoi il arriva que dès qu'on eut entendu la loi, on sépara d'Israël tout mélange.
\TextTitle{Purification des chambres du temple}
\VS{4}Or, avant que ceci arrive, Eliaschib, sacrificateur établi sur les chambres de la maison de notre Dieu, s'était allié à Tobija ;
\VS{5}et lui avait disposé une grande chambre, où on mettait auparavant les offrandes, l'encens, les ustensiles, les dîmes du blé, du vin et de l'huile, qui étaient ordonnées pour les Lévites, pour les chantres et pour les portiers, avec les contributions pour les sacrificateurs.
\VS{6}Or je n'étais point à Jérusalem pendant tout cela, car j'étais retourné vers le roi la trente-deuxième année d'Artaxerxès, roi de Babylone. Et à la fin de l'année, j'obtins du roi la permission,
\VS{7}de revenir à Jérusalem, et je m'aperçus du mal qu'Eliaschib avait fait, en disposant une chambre pour Tobija dans le parvis de la maison de Dieu.
\VS{8}Ce qui me déplut fort, et je jetai tous les meubles de Tobija hors de la chambre ;
\VS{9}Et on nettoya les chambres, selon que je l'avais ordonné et j'y fis rapporté les ustensiles de la maison de Dieu, les offrandes et l'encens.
\TextTitle{Sur les portions des Lévites}
\VS{10}J'appris aussi que les portions des Lévites ne leur avaient point été données ; et que les Lévites et les chantres qui faisaient le service s'étaient enfuis chacun sur sa terre.
\VS{11}Je fis des réprimandes aux magistrats, leur disant : Pourquoi a-t-on abandonné la maison de Dieu ? Je rassemblai les Lévites et les chantres, et les rétablis à leur place.
\VS{12}Alors tous ceux de Juda apportèrent dans le trésor les dîmes du blé, du vin et de l'huile.
\VS{13}Je confiai la surveillance du trésor à Schélémia, le sacrificateur, et Tsadok, le scribe, et Pedaja, l'un des Lévites ; et pour les aider, Hanan, fils de Zaccur, fils de Matthania, parce qu'ils étaient considérés comme très fidèles. Ils furent chargés de faire les distributions à leurs frères.
\VS{14}Souviens-toi de moi, ô mon Dieu, à cause de cela et n'efface point ce que j'ai fait avec fidélité pour la maison de mon Dieu, et pour ce qu'il est ordonné d'y faire !
\TextTitle{Avertissement pour le respect du sabbat}
\VS{15}En ces jours-là, je vis quelques-uns de Juda fouler aux pressoirs le jour du sabbat, et d'autres apporter des gerbes, et charger sur des ânes du vin, des raisins, des figues, et toutes autres sortes de fardeaux, et les apporter à Jérusalem le jour du sabbat ; et je les avertis le jour où ils vendaient leurs denrées.
\VS{16}Les Tyriens, qui demeuraient aussi à Jérusalem, apportaient du poisson, et plusieurs autres marchandises, et les vendaient aux fils de Juda dans Jérusalem le jour du sabbat.
\VS{17}Je fis des réprimandes aux chefs de Juda, et leur dis : Quel mal ne faites-vous pas, en violant le jour du sabbat ?
\VS{18}Vos pères n'ont-ils pas fait la même chose, et n'est-ce pas pour cela que notre Dieu a fait venir tout ce mal sur nous et sur cette ville ? Et vous amenez de nouveau son ardente colère contre Israël, en violant le sabbat !
\VS{19}C'est pourquoi, dès que le soleil s'était retiré des portes de Jérusalem, avant le sabbat, par mon commandement, on ferma les portes ; j'ordonnai aussi qu'on ne les ouvre point jusqu'après le sabbat. Et je plaçai quelques-uns de mes serviteurs aux portes, afin d'empêcher l'entrée des fardeaux le jour du sabbat.
\VS{20}Alors les marchands et les vendeurs de toutes sortes de denrées passèrent une ou deux fois la nuit hors de Jérusalem.
\VS{21}Je les avertis et je leur dis : Pourquoi passez-vous la nuit devant la muraille ? Si vous le faites encore, je mettrai la main sur vous. Ainsi, depuis ce temps-là, ils ne vinrent plus le jour du sabbat.
\VS{22}J'ordonnai aussi aux Lévites de se purifier, et de venir garder les portes pour sanctifier le jour du sabbat. Souviens-toi de moi, ô mon Dieu, à cause de cela, et ai compassion de moi selon la grandeur de ta miséricorde !
\TextTitle{Condamnation des unions mixtes ; rétablissement des fonctions des sacrificateurs et des Lévites}
\VS{23}En ces jours-là, je vis des Juifs qui avaient pris des femmes Asdodiennes, Ammonites et Moabites.
\VS{24}La moitié de leurs fils parlaient en partie asdodien et ne savaient point parler l'hébreu ; mais ils parlaient la langue de divers peuples.
\VS{25}Je leur fis des réprimandes et les maudis ; j'en frappai même quelques-uns, leur arrachai les cheveux et les fis jurer par Dieu, qu'ils ne donneraient point leurs filles à leurs fils, et qu'ils ne prendraient point leurs filles pour leurs fils, ou pour eux.
\VS{26}Salomon, le roi d'Israël, n'avait-il point péché par ce moyen ? Il n'y avait point de roi semblable à lui parmi un grand nombre de nations, il était aimé de son Dieu, et Dieu l'avait établi pour roi sur tout Israël ; toutefois, les femmes étrangères l'amenèrent à pécher.
\VS{27}Faut-il donc apprendre que vous fassiez tout ce grand mal, de commettre ce péché contre notre Dieu, en prenant des femmes étrangères ?
\VS{28}Or un des fils de Jojada, fils d'Eliaschib, grand sacrificateur, était gendre de Sanballat, le Horonite. Je le chassai loin de moi.
\VS{29}Souviens-toi d'eux, ô mon Dieu, car ils ont souillé la sacrificature et l'alliance contractée par les sacrificateurs et les Lévites.
\VS{30}Ainsi je les nettoyai de tous les étrangers et je rétablis les fonctions des sacrificateurs et des Lévites, chacun selon ce qu'il avait à faire,
\VS{31}et ce qui concernait l'offrande du bois aux temps fixés, de même que les prémices. Souviens-toi de moi en bien, ô mon Dieu !
\PPE{}
\end{multicols}
