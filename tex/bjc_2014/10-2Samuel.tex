\ShortTitle{2 Samuel}\BookTitle{2 Samuel}\BFont
\noindent\hrulefill
{\footnotesize
\textit{
\bigskip
{\centering{}
\\Auteur : Inconnu
\\(Heb. : Shemuw'el)
\\Signifie : Entendu, Exaucé de Dieu
\\Thème : Le règne de David
\\Date de rédaction : 10\up{ème} siècle av. J.-C.\\}
}
%\bigskip
\textit{
\\Suite du premier livre de Samuel, ce livre commence par le récit de la mort de Saül et l'accession progressive à la royauté de David. La faveur de Dieu dans sa vie lui donna du succès et lui permit d'étendre son royaume jusqu'au nord de Damas. Au détriment de sa piété et de son alliance avec Dieu, David commit de lourdes erreurs. Il s'en repentit sincèrement, mais il dut en assumer les conséquences…\bigskip
}
}
\par\nobreak\noindent\hrulefill
\begin{multicols}{2}
\Chap{1}
\TextTitle{Attitude de David à la mort de Saül}
\VerseOne{}Et il arriva qu'après la mort de Saül, David, qui était revenu vainqueur des Amalécites, resta deux jours à Tsiklag.
\VS{2}Le troisième jour, un homme arriva du camp de Saül, avec ses vêtements déchirés et de la terre sur sa tête. Il se présenta à David, se jeta par terre et se prosterna.
\VS{3}David lui dit : D'où viens-tu? Il lui répondit : Je me suis échappé du camp d'Israël.
\VS{4}David lui dit : Qu'est-il arrivé ? Je te prie, raconte-le-moi! Il répondit : Le peuple s'est enfui de la bataille, et il y a eu beaucoup du peuple qui sont tombés morts ; Saül aussi et Jonathan, son fils, sont morts.
\VS{5}David dit à ce jeune garçon qui lui disait ces nouvelles : Comment sais-tu que Saül et Jonathan, son fils, sont morts ?
\VS{6}Le jeune garçon qui lui disait ces nouvelles lui répondit : Je me trouvais par hasard sur la montagne de Guilboa ; et voici, Saül s'appuyait sur sa lance, et voici, les chars et quelques chefs des cavaliers le poursuivaient.
\VS{7}S'étant retourné, il m'aperçut et m'appela. Je lui répondis : Me voici !
\VS{8}Il me dit : Qui es-tu ? Je lui répondis : Je suis amalécite.
\VS{9}Et il dit : Approche-toi de moi et tue-moi ; car je suis dans une grande angoisse, et ma vie est encore toute en moi.
\VS{10}Je m'approchai de lui, et je lui donnai la mort\FTNT{Cet homme amalécite a peut-être menti afin de gagner la faveur de David (1 S. 31:3-5).}, sachant bien qu'il ne survivrait après s'être jeté sur sa lance. J'ai pris la couronne qu'il avait sur sa tête, et le bracelet qui était à son bras, et je les apporte ici à mon seigneur.
\VS{11}Alors David saisit ses vêtements et les déchira, et tous les hommes qui étaient avec lui firent de même.
\VS{12}Ils furent dans le deuil, ils pleurèrent et ils jeûnèrent jusqu'au soir, à cause de Saül, de Jonathan, son fils, à cause du peuple de Yahweh, et de la maison d'Israël, parce qu'ils étaient tombés par l'épée.
\VS{13}David dit au jeune garçon qui lui avait apporté ces nouvelles : D'où es-tu ? Et il répondit : Je suis fils d'un homme étranger, d'un Amalécite.
\VS{14}David lui dit : Comment n'as-tu pas craint d'avancer ta main pour tuer l'oint de Yahweh ?
\VS{15}Et David appela l'un de ses serviteurs et lui dit : Approche-toi, jette-toi sur lui ! Ce dernier le frappa et il mourut.
\VS{16}Et David lui dit : Que ton sang retombe sur ta tête, car ta bouche a témoigné contre toi, en disant : J'ai fait mourir l'oint de Yahweh !
\TextTitle{Chant funèbre de David}
\VS{17}Alors David composa sur Saül et sur Jonathan, son fils, un chant funèbre,
\VS{18}qu'il ordonna d'enseigner aux fils de Juda. C'est le cantique de l'arc : Il est écrit dans le livre du Juste.
\VS{19}L'élite d'Israël a succombé sur tes collines ! Comment des héros sont-ils tombés ?
\VS{20}Ne l'annoncez pas dans Gath, et n'en publiez point la nouvelle dans les rues d'Askalon, de peur que les filles des Philistins ne se réjouissent, de peur que les filles des incirconcis n'en tressaillent de joie.
\VS{21}Montagnes de Guilboa ! Qu'il n'y ait sur vous ni rosée, ni pluie, ni champs qui donnent des prémices pour les offrandes ! Car là ont été jetés les boucliers des héros, le bouclier de Saül ; l'huile a cessé de les oindre.
\VS{22}L'arc de Jonathan ne revenait jamais sans être teint du sang des blessés et de la graisse des hommes forts ; et l'épée de Saül ne retournait jamais sans effet.
\VS{23}Saül et Jonathan, aimables et agréables pendant leur vie, n'ont point été séparés dans leur mort ; ils étaient plus légers que les aigles, ils étaient plus forts que des lions.
\VS{24}Filles d'Israël ! Pleurez sur Saül, qui vous revêtait magnifiquement de cramoisi, qui mettait des ornements d'or à vos habits.
\VS{25}Comment les héros sont-ils tombés au milieu du combat ? Comment Jonathan a-t-il été tué sur tes collines ?
\VS{26}Jonathan, mon frère, je suis dans la douleur à cause de toi ! Tu faisais tout mon plaisir ; l'amour que j'avais pour toi était plus grand que celui qu'on a pour les femmes.
\VS{27}Comment sont tombés les héros ? Comment se sont perdus les instruments de guerre ?
\Chap{2}
\TextTitle{David oint roi de Juda}
\VerseOne{}Et il arriva après cela que David consulta Yahweh, en disant : Monterai-je dans l'une des villes de Juda ? Yahweh lui répondit : Monte. David dit : Où monterai-je ? Yahweh répondit : A Hébron.
\VS{2}David y monta, avec ses deux femmes, Achinoam de Jizreel, et Abigaïl de Carmel, qui avait été femme de Nabal.
\VS{3}David fit monter aussi les hommes qui étaient avec lui, chacun avec sa famille ; et ils habitèrent dans les villes d'Hébron.
\VS{4}Les hommes de Juda vinrent, et là ils oignirent David pour roi sur la maison de Juda. On fit un rapport à David en disant : Les gens de Jabès en Galaad ont enseveli Saül.
\VS{5}Alors David envoya des messagers vers les gens de Jabès en Galaad, pour leur dire : Soyez bénis de Yahweh, puisque vous avez montré de la bienveillance envers Saül, votre seigneur, et que vous l'avez enseveli.
\VS{6}Maintenant donc que Yahweh use envers vous de bonté et de fidélité. Moi aussi je vous ferai du bien, parce que vous avez agi de la sorte.
\VS{7}Que maintenant vos mains se fortifient, et soyez des vaillants hommes ; car Saül, votre maître, est mort, et la maison de Juda m'a oint pour être roi sur elle.
\TextTitle{Isch-Boscheth établi roi d'Israël}
\VS{8}Cependant Abner, fils de Ner, chef de l'armée de Saül, prit Isch-Boscheth, fils de Saül, et le fit passer à Mahanaïm.
\VS{9}Il l'établit roi sur Galaad, sur les Gueschuriens, sur Jizreel, sur Ephraïm, sur Benjamin, et sur tout Israël.
\VS{10}Isch-Boscheth, fils de Saül, était âgé de quarante ans quand il commença à régner sur Israël, et il régna deux ans. Il n'y eut que la maison de Juda qui suivit David.
\VS{11}Le nombre de jours pendant lesquels David régna à Hébron sur la maison de Juda fut de sept ans et six mois.
\TextTitle{Guerre entre Juda et Israël}
\VS{12}Abner, fils de Ner, et les gens d'Isch-Boscheth, fils de Saül, sortirent de Mahanaïm pour marcher vers Gabaon.
\VS{13}Joab, fils de Tseruja, et les gens de David, sortirent aussi. Ils se rencontrèrent ensemble près de l'étang de Gabaon, et les uns se tinrent d'un côté de l'étang, et les autres du côté opposé de l'étang.
\VS{14}Alors Abner dit à Joab : Que ces jeunes gens se lèvent maintenant, et qu'ils se battent devant nous ! Et Joab répondit : Qu'ils se lèvent !
\VS{15}Ils se levèrent donc, et s'avancèrent en nombre égal, douze pour Benjamin et pour Isch-Boscheth, fils de Saül, et douze des gens de David.
\VS{16}Alors, chacun saisissant son adversaire par la tête, lui enfonça son épée dans le flanc, et ils tombèrent tous ensemble. Et l'on donna à ce lieu, qui est près de Gabaon, le nom de Helkath-Hatsurim.
\VS{17}Il y eut ce jour-là un combat très rude, dans lequel Abner et les hommes d'Israël furent battus par les gens de David.
\VS{18}Les trois fils de Tseruja, Joab, Abischaï et Asaël étaient là. Asaël avait les pieds légers comme une gazelle des champs.
\VS{19}Asaël poursuivit Abner, sans se détourner de lui ni à droite ni à gauche.
\VS{20}Abner regarda derrière lui, et dit : Est-ce toi, Asaël ? Et il répondit : C'est moi.
\VS{21}Abner lui dit : Détourne-toi à droite ou à gauche ; saisis-toi de l'un de ces jeunes gens, et prends sa dépouille. Mais Asaël ne voulut point se détourner de lui.
\VS{22}Et Abner dit encore à Asaël : Détourne-toi de moi ; pourquoi te frapperais-je et t'abattrais-je à terre ? Comment ensuite lèverais-je le visage devant ton frère Joab ?
\VS{23}Mais Asaël refusa de se détourner. Abner le frappa au ventre avec l'extrémité inférieure de sa lance, qui sortit par-derrière. Il tomba là, roide mort sur place. Tous ceux qui arrivaient au lieu où Asaël était tombé mort, s'y arrêtaient.
\VS{24}Joab et Abischaï poursuivirent Abner, et le soleil se couchait quand ils arrivèrent au coteau d'Amma, qui est en face de Guiach, sur le chemin du désert de Gabaon.
\VS{25}Les fils de Benjamin s'assemblèrent auprès d'Abner et formèrent un corps de troupe, et ils s'arrêtèrent sur le sommet d'une colline.
\VS{26}Alors Abner appela Joab, et dit : L'épée dévorera-t-elle sans cesse ? Ne sais-tu pas qu'il y aura de l'amertume à la fin ? Jusqu'à quand tarderas-tu à dire au peuple qu'il cesse de poursuivre ses frères ?
\VS{27}Joab répondit : Dieu est vivant ! Si tu n'avais parlé ainsi, le peuple n'aurait pas cessé avant le matin de poursuivre ses frères.
\VS{28}Joab sonna du shofar, et tout le peuple s'arrêta ; ils ne poursuivirent plus Israël, et ils ne continuèrent plus à se battre.
\VS{29}Ainsi, Abner et ses gens marchèrent toute la nuit dans la plaine ; ils passèrent le Jourdain, traversèrent tout le Bithron, et arrivèrent à Mahanaïm.
\VS{30}Joab aussi revint de la poursuite d'Abner, et rassembla tout le peuple ; il manquait dix-neuf hommes des gens de David, et Asaël.
\VS{31}Mais les gens de David avaient frappé à mort trois cent soixante hommes de Benjamin, et des gens d'Abner.
\VS{32}Ils emportèrent Asaël, et l'ensevelirent dans le sépulcre de son père à Bethléhem. Joab et ses gens marchèrent toute la nuit et arrivèrent à Hébron au point du jour.
\Chap{3}
\TextTitle{L'autorité de David s'accroit\FTNTT{1 Ch. 3:1-4}}
\VerseOne{}Or il y eut une longue guerre entre la maison de Saül et la maison de David. David devenait de plus en plus fort, et la maison de Saül allait en s'affaiblissant.
\VS{2}Il naquit à David des fils à Hébron. Son premier-né fut Amnon, d'Achinoam de Jizreel ;
\VS{3}le second, Kileab, d'Abigaïl de Carmel, femme de Nabal ; le troisième, Absalom, fils de Maaca, fille de Talmaï, roi de Gueschur ;
\VS{4}le quatrième, Adonija, fils de Haggith ; le cinquième, Schephathia, fils d'Abithal ;
\VS{5}et le sixième, Jithream, d'Egla, femme de David. Ce sont là ceux qui naquirent à David à Hébron.
\TextTitle{Abner fait alliance avec David}
\VS{6}Et il arriva que pendant la guerre entre la maison de Saül et la maison de David, Abner tint ferme pour la maison de Saül.
\VS{7}Or Saül avait eu une concubine, nommée Ritspa, fille d'Ajja. Et Isch-Boscheth dit à Abner : Pourquoi es-tu venu vers la concubine de mon père ?
\VS{8}Abner fut très irrité à cause du discours d'Isch-Boscheth, et il lui dit : Suis-je une tête de chien, au service de Juda ? Je fais aujourd'hui preuve de bienveillance envers la maison de Saül, ton père, envers ses frères et ses amis, je ne t'ai pas livré entre les mains de David, et c'est aujourd'hui que tu me reproches une faute avec cette femme ?
\VS{9}Que Dieu punisse sévèrement Abner, si je n'agis pas avec David selon ce que Yahweh a juré à David,
\VS{10}en disant qu'il ferait passer la royauté de la maison de Saül à la sienne, et qu'il établirait le trône de David sur Israël et sur Juda depuis Dan jusqu'à Beer-Schéba.
\VS{11}Isch-Boscheth n'osa pas répondre un seul mot à Abner, parce qu'il le craignait.
\VS{12}Abner envoya des messagers à David pour lui dire de sa part : A qui est le pays ? Fais alliance avec moi, et voici, ma main sera avec toi, pour tourner vers toi tout Israël.
\VS{13}David répondit : Je le veux bien ! Je ferai alliance avec toi ; je te demande seulement une chose, c'est que tu ne voies point ma face, à moins que tu n'amènes d'abord Mical, fille de Saül, quand tu viendras me voir.
\VS{14}Et David envoya des messagers à Isch-Boscheth, fils de Saül, pour lui dire : Rends-moi ma femme Mical, que j'ai épousée pour cent prépuces des Philistins.
\VS{15}Isch-Boscheth envoya et l'ôta à son mari Palthiel, fils de Laïsch.
\VS{16}Et son mari la suivit, marchant et pleurant continuellement après elle jusqu'à Bachurim. Alors Abner lui dit : Va, retourne-t'en ! Et il s'en retourna.
\VS{17}Abner parla aux anciens d'Israël, et leur dit : Vous désiriez autrefois avoir David pour roi ;
\VS{18}établissez-le maintenant, car Yahweh a parlé de David et a dit : C'est par David, mon serviteur, que je délivrerai mon peuple d'Israël de la main des Philistins et de la main de tous ses ennemis.
\VS{19}Abner parla aussi aux oreilles de ceux de Benjamin, puis il alla faire entendre expressément à David, qui était à Hébron, ce qui semblait bon aux yeux d'Israël et aux yeux de toute la maison de Benjamin.
\VS{20}Abner vint donc vers David à Hébron, accompagné de vingt hommes ; et David fit un festin à Abner et aux hommes qui étaient avec lui.
\VS{21}Abner dit à David : Je me lèverai, et je partirai pour rassembler tout Israël auprès du roi, mon seigneur ; ils feront alliance avec toi, et tu régneras selon le désir de ton âme. David renvoya Abner, qui s'en alla en paix.
\VS{22}Voici, les gens de David et Joab revinrent d'une excursion, et amenèrent avec eux un grand butin. Abner n'était plus avec David à Hébron, car David l'avait renvoyé, et il s'en était allé en paix.
\VS{23}Lorsque Joab et toute l'armée qui était avec lui revinrent, on fit ce rapport à Joab en ces mots : Abner, fils de Ner, est venu auprès du roi, qui l'a renvoyé, et il s'en est allé en paix.
\VS{24}Joab vint vers le roi, et dit : Qu'as-tu fait ? Voici, Abner est venu vers toi ; pourquoi l'as-tu ainsi renvoyé, en sorte qu'il s'en est allé ?
\VS{25}Tu connais Abner, fils de Ner ! C'est pour te tromper qu'il est venu, pour épier tes démarches, tes allées et venues, et pour savoir tout ce que tu fais.
\VS{26}Puis Joab, après avoir quitté David, envoya sur les traces d'Abner des messagers, qui le ramenèrent de la fosse de Sira, sans que David n'en sache rien.
\TextTitle{Mort d'Abner}
\VS{27}Lorsque Abner revint à Hébron, Joab le tira à l'écart au milieu de la porte, comme pour lui parler en secret ; et là il le frappa à la cinquième côte ; et ainsi Abner mourût à cause du sang d'Asaël, frère de Joab.
\VS{28}David apprit ce qui était arrivé et dit : Je suis à jamais innocent, mon royaume et moi, devant Yahweh, du sang d'Abner, fils de Ner.
\VS{29}Que ce sang retombe sur la tête de Joab, et sur toute la maison de son père ! Que soit retranchée la maison de Joab, qu'il y ait toujours un homme qui soit atteint d'un flux ou de la lèpre, ou qui s'appuie sur un bâton, ou qui tombe par l'épée, ou qui manque de pain !
\VS{30}Ainsi Joab et Abischaï, son frère, tuèrent Abner, parce qu'il avait tué Asaël, leur frère, à Gabaon, dans la bataille.
\VS{31}David dit à Joab et à tout le peuple qui était avec lui : Déchirez vos vêtements, ceignez-vous de sacs, et menez le deuil en marchant devant Abner ! Et le roi David marcha derrière le cercueil.
\VS{32}On ensevelit Abner à Hébron. Le roi éleva la voix et pleura sur la tombe d'Abner, et tout le peuple pleura.
\VS{33}Le roi fit une complainte sur Abner, et dit : Abner devait-il mourir comme meurt un insensé ?
\VS{34}Tes mains n'étaient pas liées, et tes pieds n'étaient pas mis dans des chaînes ! Tu es tombé comme on tombe devant les méchants. Et tout le peuple recommença à pleurer sur Abner.
\VS{35}Puis tout le peuple vint pour faire prendre quelque nourriture à David, pendant qu'il était encore jour ; mais David jura, en disant : Que Dieu me punisse sévèrement, si je goûte du pain ou quelque chose d'autre avant le coucher du soleil !
\VS{36}Tout le peuple l'entendit, et l'approuva, et tout le peuple trouva bon tout ce qu'avait fait le roi.
\VS{37}En ce jour, tout le peuple et tout Israël surent que ce n'était pas par ordre du roi qu'Abner, fils de Ner, avait été tué.
\VS{38}Le roi dit à ses serviteurs : Ne savez-vous pas qu'un chef, un grand homme, est tombé aujourd'hui en Israël ?
\VS{39}Je suis encore faible aujourd'hui, bien que j'aie été oint roi ; et ces gens, les fils de Tseruja, sont trop puissants pour moi. Que Yahweh rende à celui qui fait le mal selon sa méchanceté !
\Chap{4}
\TextTitle{Mort d'Ish-Boscheth}
\VerseOne{}Quand le fils de Saül apprit qu'Abner était mort à Hébron, ses mains restèrent sans force, et tout Israël fut dans l'épouvante.
\VS{2}Le fils de Saül avait deux chefs de bandes, dont l'un s'appelait Baana et l'autre Récab ; ils étaient fils de Rimmon de Beéroth, d'entre les fils de Benjamin. Car Beéroth était regardée comme appartenant à Benjamin,
\VS{3}et les Beérothiens s'étaient enfuis à Guitthaïm, où ils y ont habité jusqu'à ce jour.
\VS{4}Jonathan, fils de Saül, avait un fils perclus des pieds ; il était âgé de cinq ans lorsque la nouvelle de la mort de Saül et de Jonathan arriva de Jizreel ; sa nourrice le prit et s'enfuit, et comme elle se hâtait de fuir, il tomba et devint boiteux ; son nom était Mephiboscheth.
\VS{5}Les fils de Rimmon de Beéroth, Récab et Baana, se rendirent pendant la chaleur du jour à la maison d'Isch-Boscheth, qui était couché pour son repos du midi.
\VS{6}Ils pénétrèrent jusqu'au milieu de la maison, comme pour y prendre du froment, et ils le frappèrent à la cinquième côte ; puis Récab et Baana, son frère, se sauvèrent.
\VS{7}Ils entrèrent donc dans la maison lorsqu'Isch-Boscheth était couché sur son lit dans la chambre à coucher où il dormait, ils le frappèrent et le tuèrent, puis ils lui coupèrent la tête. Ils prirent sa tête, et ils marchèrent toute la nuit au travers de la plaine.
\VS{8}Ils apportèrent la tête d'Isch-Boscheth à David dans Hébron, et ils dirent au roi : Voici la tête d'Isch-Boscheth, fils de Saül, ton ennemi, qui en voulait à ta vie ; Yahweh venge aujourd'hui le roi mon seigneur de Saül et de sa race.
\VS{9}Mais David répondit à Récab et à Baana, son frère, fils de Rimmon de Beéroth, et leur dit : Yahweh, qui a délivré mon âme de toute angoisse est vivant !
\VS{10}J'ai saisi celui qui est venu m'annoncer et me dire : Voilà, Saül est mort, et qui pensait m'apprendre de bonnes nouvelles, je l'ai fait saisir et tuer à Tsiklag, pour lui donner le salaire de ses bonnes nouvelles ;
\VS{11}combien plus, quand des méchants ont tué un homme juste dans sa maison et sur sa couche, ne redemanderai-je pas maintenant son sang de vos mains et ne vous exterminerai-je pas de la terre ?
\VS{12}David ordonna à ses gens de les tuer ; ils leur coupèrent les mains et les pieds, et les pendirent près de l'étang d'Hébron. Ils prirent la tête d'Isch-Boscheth, et l'ensevelirent dans le sépulcre d'Abner à Hébron.
\Chap{5}
\TextTitle{David oint roi sur tout Israël\FTNTT{1 Ch. 11:1-3}}
\VerseOne{}Alors toutes les tribus d'Israël vinrent auprès de David, à Hébron, et dirent : Voici, nous sommes tes os et ta chair.
\VS{2}Autrefois déjà, quand Saül était roi sur nous, c'est toi qui conduisais et qui ramenais Israël. Yahweh t'a dit : Tu paîtras mon peuple d'Israël, et tu seras le chef d'Israël.
\VS{3}Tous les anciens d'Israël vinrent donc vers le roi à Hébron, et le roi David fit alliance avec eux à Hébron, devant Yahweh. Ils oignirent David pour roi sur Israël.
\VS{4}David était âgé de trente ans lorsqu'il commença à régner ; il régna quarante ans.
\VS{5}Il régna sur Juda à Hébron sept ans et six mois, puis il régna trente-trois ans à Jérusalem sur tout Israël et Juda.
\TextTitle{Jérusalem, capitale de tout Israël\FTNTT{1 Ch. 11:4-9}}
\VS{6}Le roi marcha avec ses gens sur Jérusalem contre les Jébusiens qui habitaient ce pays. Ils dirent à David : Tu n'entreras point ici, car les aveugles mêmes et les boiteux te repousseront ! Ce qui voulait dire : David n'entrera point ici.
\VS{7}Mais David s'empara de la forteresse de Sion : C'est la cité de David.
\VS{8}David avait dit en ce jour-là : Quiconque battra les Jébusiens et atteindra le canal, ces aveugles et ces boiteux, qui sont haïs de l'âme de David, sera récompensé… C'est pourquoi l'on dit : Aucun aveugle ni boiteux n'entrera dans cette maison.
\VS{9}Et David habita dans la forteresse, et l'appela la cité de David. Il bâtit tout autour, depuis Millo jusqu'au-dedans.
\VS{10}David devenait de plus en plus grand, et Yahweh, le Dieu des armées, était avec lui.
\TextTitle{Yahweh affermit le règne de David}
\VS{11}Hiram, roi de Tyr, envoya des messagers à David, du bois de cèdre, des charpentiers et des tailleurs de pierres à bâtir, et ils bâtirent la maison de David.
\VS{12}David reconnut que Yahweh l'affermissait comme roi sur Israël, et qu'il élevait son royaume à cause de son peuple d'Israël.
\TextTitle{Fils de David nés à Jérusalem\FTNTT{2 S. 3:2-5 ; 1 Ch. 3:1-4}}
\VS{13}David prit encore des concubines et des femmes de Jérusalem, après qu'il fut venu d'Hébron, et il lui naquit encore des fils et des filles.
\VS{14}Voici les noms de ceux qui lui naquirent à Jérusalem : Schammua, Schobad, Nathan, Salomon,
\VS{15}Jibhar, Elischua, Népheg, Japhia,
\VS{16}Elischama, Eliada et Eliphéleth.
\TextTitle{Yahweh livre les Philistins à David\FTNTT{2 S. 23:13-17 ; 1 Ch. 14:8-17 ; 11:15-19 ; 12:8-15}}
\VS{17}Or quand les Philistins apprirent qu'on avait oint David pour roi sur Israël, ils montèrent tous pour chercher David. Et David l'ayant appris, il descendit vers la forteresse.
\VS{18}Les Philistins arrivèrent et se répandirent dans la vallée des Rephaïm.
\VS{19}Alors David consulta Yahweh, en disant : Monterai-je contre les Philistins ? Les livreras-tu entre mes mains ? Et Yahweh parla à David : Monte, car certainement je livrerai les Philistins entre tes mains.
\VS{20}Alors David vint à Baal-Peratsim, où il les battit. Puis il dit : Yahweh a dispersé mes ennemis devant moi, comme des eaux qui s'écoulent. C'est pourquoi il nomma ce lieu-là Baal-Peratsim.
\VS{21}Ils laissèrent là leurs faux dieux que David et ses gens emportèrent.
\VS{22}Les Philistins montèrent encore une autre fois, et se répandirent dans la vallée des Rephaïm.
\VS{23}David consulta Yahweh. Et Yahweh dit : Tu ne monteras pas ; contourne-les par-derrière, et tu les atteindras vis-à-vis des mûriers.
\VS{24}Quand tu entendras un bruit comme des gens qui marchent au sommet des mûriers, alors hâte-toi, car c'est Yahweh qui sort devant toi pour battre l'armée des Philistins.
\VS{25}David fit ce que Yahweh lui avait ordonné, et il battit les Philistins depuis Guéba jusqu'à Guézer.
\Chap{6}
\TextTitle{Désobéissance dans le transport de l'arche\FTNTT{1 Ch. 13:1-14}}
\VerseOne{}David rassembla encore toute l'élite d'Israël, au nombre de trente mille hommes.
\VS{2}Puis David se leva, ainsi que tout le peuple qui était avec lui, et se mit en marche de Baalé-Juda, pour faire monter de là l'arche de Dieu, qui est appelée du Nom, du Nom de Yahweh des armées, qui siège entre les chérubins.
\VS{3}Ils mirent l'arche\FTNT{L'arche ne devait être portée que par les Lévites. Les meilleures intentions pour le service de Yahweh ne suffisent pas pour que le Seigneur nous agrée. Nous devons nous conformer à la Parole de Dieu (1 R. 18:36-39).} de Dieu sur un char tout neuf, et l'emmenèrent de la maison d'Abinadab qui était sur la colline ; Uzza et Achjo, fils d'Abinadab, conduisaient le char neuf.
\VS{4}Ils l'emportèrent donc de la maison d'Abinadab sur la colline ; et Achjo allait devant l'arche.
\VS{5}David et toute la maison d'Israël jouaient devant Yahweh de toutes sortes d'instruments faits de bois de cyprès, de harpes, de luths, de tambourins, de sistres et de cymbales.
\VS{6}Quand ils furent arrivés à l'aire de Nacon, Uzza étendit la main vers l'arche de Dieu et la saisit, parce que les bœufs la faisaient pencher.
\VS{7}La colère de Yahweh s'enflamma contre Uzza, et Dieu le frappa là à cause de sa faute. Il mourut là, près de l'arche de Dieu.
\VS{8}David fut irrité de ce que Yahweh avait fait une brèche en la personne d'Uzza. C'est pourquoi on a appelé ce lieu jusqu'à ce jour Pérets-Uzza.
\VS{9}David eut peur de Yahweh en ce jour-là, et il dit : Comment l'arche de Yahweh entrerait-elle chez moi ?
\VS{10}David ne voulut pas déposer l'arche de Yahweh chez lui dans la cité de David, mais il la fit conduire dans la maison d'Obed-Edom de Gath.
\VS{11}L'arche de Yahweh resta trois mois dans la maison d'Obed-Edom de Gath, et Yahweh bénit Obed-Edom et toute sa maison.
\TextTitle{Accueil de l'arche à Jérusalem\FTNTT{1 Ch. 15:26-16:1}}
\VS{12}Puis on vint dire au roi David : Yahweh a béni la maison d'Obed-Edom et tout ce qui lui appartient, pour l'amour de l'arche de Dieu. Alors David s'y rendit, et il fit monter l'arche de Dieu depuis la maison d'Obed-Edom jusqu'à la cité de David, au milieu des réjouissances.
\VS{13}Et il arriva que quand ceux qui portaient l'arche de Dieu eurent fait six pas, on sacrifia des taureaux et des béliers gras.
\VS{14}David dansait de toute sa force devant Yahweh, et il était ceint d'un éphod de lin.
\VS{15}Ainsi, David et toute la maison d'Israël firent monter l'arche de Yahweh avec des cris de joie et au son du shofar.
\VS{16}Comme l'arche de Yahweh entrait dans la cité de David, Mical, fille de Saül, regardait par la fenêtre, et voyant le roi David sauter et danser devant Yahweh, elle le méprisa en son cœur.
\VS{17}Ils amenèrent l'arche de Yahweh, et la posèrent au milieu de la tente que David avait dressée pour elle ; et David offrit des holocaustes et des sacrifices d'offrande de paix devant Yahweh.
\VS{18}Quand David eut achevé d'offrir des holocaustes et des sacrifices d'offrande de paix, il bénit le peuple au Nom de Yahweh des armées.
\VS{19}Et il partagea à tout le peuple, à toute la multitude d'Israël, tant aux hommes qu'aux femmes, à chacun un pain, une portion de viande, un gâteau de raisins, et une ration de vin. Puis tout le peuple s'en alla, chacun dans sa maison.
\VS{20}David s'en retourna pour bénir aussi sa maison, et Mical, fille de Saül, sortit à sa rencontre. Elle dit : Quel honneur s'est fait aujourd'hui le roi d'Israël, en se découvrant aux yeux des servantes et de ses serviteurs, comme se découvrirait un homme de néant sans en avoir honte !
\VS{21}David répondit à Mical : C'est devant Yahweh, qui m'a choisi plutôt que ton père et toute sa maison pour m'établir chef sur le peuple de Yahweh, sur Israël, c'est devant Yahweh que je me suis réjoui.
\VS{22}Je me rendrai encore plus insignifiant que je n'ai été cette fois, et je m'estimerai encore moins à mes propres yeux ; malgré cela, je serai en honneur auprès des servantes dont tu parles.
\VS{23}Or Mical, fille de Saül, n'eut point d'enfants jusqu'au jour de sa mort.
\Chap{7}
\TextTitle{David veut construire une maison à Yahweh\FTNTT{1 Ch. 17:1-2}}
\VerseOne{}Et il arriva, lorsque le roi fut établi dans sa maison, et que Yahweh lui eut donné du repos de tous ses ennemis qui l'entouraient,
\VS{2}qu'il dit à Nathan le prophète : Regarde maintenant ! J'habite dans une maison de cèdres, et l'arche de Dieu habite sous des tapis\FTNT{Dans la plupart des versions, on a traduit ce mot par « tente », alors que le terme hébreu est « yeriy'ah », ce qui signifie « rideau », « drap », « tapis ». Voir Ex. 26.}.
\VS{3}Alors Nathan répondit au roi : Va, fais tout ce qui est dans ton cœur, car Yahweh est avec toi.
\TextTitle{Yahweh traite alliance avec David et sa postérité\FTNTT{1 Ch. 17:3-15}}
\VS{4}Mais il arriva cette nuit-là que la parole de Yahweh fut adressée à Nathan, en disant :
\VS{5}Va, et dis à David, mon serviteur : Ainsi parle Yahweh : Me bâtirais-tu une maison afin que j'y habite ?
\VS{6}Puisque je n'ai point habité dans une maison depuis le jour où j'ai fait monter les enfants d'Israël hors d'Egypte jusqu'à ce jour ; mais j'ai marché ça et là sous une tente et dans un tabernacle.
\VS{7}Partout où j'ai marché avec tous les enfants d'Israël, ai-je dit un seul mot à quelqu'une des tribus d'Israël à qui j'avais ordonné de paître mon peuple d'Israël, ai-je dit : Pourquoi ne me bâtissez-vous pas une maison de cèdres ?
\VS{8}Maintenant tu diras à David, mon serviteur : Ainsi parle Yahweh des armées : Je t'ai pris d'une cabane, d'auprès des brebis, afin que tu sois le conducteur de mon peuple, Israël ;
\VS{9}j'ai été avec toi partout où tu as marché, j'ai exterminé tous tes ennemis devant toi, et j'ai rendu ton nom grand, comme le nom des grands qui sont sur la terre ;
\VS{10}j'ai établi une demeure à mon peuple, à Israël, et je l'ai planté pour qu'il y habite et ne soit plus agité, pour que les méchants ne l'affligent plus comme auparavant,
\VS{11}et comme du temps où j'avais établi des juges sur mon peuple d'Israël. Je t'ai accordé du repos face à tous tes ennemis. Et Yahweh t'annonce qu'il te bâtira une maison.
\VS{12}Quand tu seras endormi avec tes pères, je susciterai après toi, ton fils, qui sera sorti de tes entrailles, et j'affermirai son règne.
\VS{13}Ce sera lui qui bâtira une maison à mon Nom, et j'affermirai pour toujours le trône de son règne\FTNT{Le royaume millénaire était promis à David et à sa postérité. Il fut proclamé par Jean-Baptiste (Mt. 3:1-12), le Messie (Mt. 4:17) et les apôtres (Mt. 10) comme étant proche. Présentement, le royaume de Dieu se manifeste par la vie sanctifiée des saints en Christ (Lu. 17:20 ; Jn. 3:1-8 ; Ro. 14:17). Il n'apparaîtra pas de manière visible avant la «moisson», c'est-à-dire le jugement des nations (Mt. 13:39-50). En effet, ce n'est qu'après cette moisson que le royaume sera installé ici-bas, lorsque le Messie rétablira la monarchie et la dynastie de David en sa propre personne. Il rassemblera alors les enfants d'Israël dispersés dans le monde entier et établira sa domination sur toute la terre pendant mille ans. Ce royaume sera remis au Père par le Messie après avoir vaincu le dernier ennemi, c'est-à-dire la mort (1 Co. 15:24-26). De ce fait, personne ne mourra pendant le millénium. Toutes les nations monteront tous les ans à Jérusalem pour adorer Yahweh et célébrer la fête des tabernacles qui sera restaurée (Za. 14). Le gouvernement théocratique en Israël sera alors restauré (Es. 1:26).}.
\VS{14}Je serai pour lui un père, et il sera pour moi un fils. S'il fait le mal, je le châtierai avec une verge d'hommes et avec des plaies des fils des hommes ;
\VS{15}mais ma grâce ne se retirera point de lui, comme je l'ai retirée de Saül, que j'ai ôté de devant toi.
\VS{16}Ainsi ta maison et ton règne seront assurés à jamais devant tes yeux, et ton trône sera pour toujours affermi.
\VS{17}Nathan rapporta à David toutes ces paroles et toute cette vision.
\TextTitle{Louange et reconnaissance de David envers Yahweh\FTNTT{1 Ch. 17:16-27}}
\VS{18}Alors le roi David alla se présenter devant Yahweh, et dit : Qui suis-je, Seigneur Yahweh, et quelle est ma maison, que tu m'aies fait arriver au point où je suis ?
\VS{19}C'est encore peu de choses à tes yeux, ô Seigneur Yahweh ! Car tu as même parlé sur la maison de ton serviteur pour les temps éloignés. Est-ce là la manière d'agir des hommes, ô Seigneur Yahweh ?
\VS{20}Et que pourrait dire de plus David ? Car, Seigneur Yahweh, tu connais ton serviteur !
\VS{21}Tu as fait toutes ces grandes choses pour l'amour de ta parole, et selon ton cœur, pour les révéler à ton serviteur.
\VS{22}C'est pourquoi tu t'es montré grand, ô Yahweh Dieu ! Car nul n'est semblable à toi, et il n'y a point d'autre Dieu que toi, d'après tout ce que nous avons entendu de nos oreilles.
\VS{23}Et qui est comme ton peuple, comme Israël, la seule nation de la terre que Dieu est venu racheter pour en faire son peuple, y mettre son Nom et pour accomplir dans ton pays, devant ton peuple que tu t'es racheté d'Egypte, des choses grandes et terribles contre les nations et contre leurs dieux ?
\VS{24}Tu as affermi ton peuple d'Israël pour qu'il soit ton peuple pour toujours ; et toi, Yahweh, tu es devenu son Dieu.
\VS{25}Maintenant donc, ô Yahweh Dieu, confirme pour toujours la parole que tu as prononcée sur ton serviteur et sur sa maison, et agis selon ta parole.
\VS{26}Que ton Nom soit à jamais glorifié, et que l'on dise : Yahweh des armées est le Dieu d'Israël ! Et que la maison de David, ton serviteur, demeure stable devant toi !
\VS{27}Car toi, Yahweh des armées, Dieu d'Israël, tu as révélé ces choses à l'oreille de ton serviteur, en disant : Je te bâtirai une maison ! C'est pourquoi ton serviteur a pris courage pour t'adresser cette prière.
\VS{28}Maintenant, Seigneur Yahweh, tu es Dieu, tes paroles sont vérité, et tu as promis cette grâce à ton serviteur.
\VS{29}Veuille donc bénir la maison de ton serviteur, afin qu'elle soit éternellement devant toi ! Car c'est toi, Seigneur Yahweh, qui a parlé, et par ta bénédiction la maison de ton serviteur sera comblée de bénédictions éternellement.
\Chap{8}
\TextTitle{Yahweh donne à David la victoire sur ses ennemis\FTNTT{1 Ch. 18:1-17}}
\VerseOne{}Après cela il arriva que David battit les Philistins et les humilia, et il prit Métheg-Amma de la main des Philistins.
\VS{2}Il battit aussi les Moabites, et les mesura au cordeau, en les faisant coucher par terre ; il en mesura deux cordeaux pour les faire mourir, et un plein cordeau pour leur laisser la vie. Et les Moabites furent assujettis à David, et lui payèrent un tribut.
\VS{3}David battit aussi Hadadézer, fils de Rehob, roi de Tsoba, lorsqu'il alla rétablir sa domination sur le fleuve de l'Euphrate.
\VS{4}David lui prit mille sept cents cavaliers, et vingt mille hommes de pied ; il coupa les jarrets aux chevaux de tous les chars, et ne conserva que cent attelages.
\VS{5}Les Syriens de Damas vinrent au secours d'Hadadézer, roi de Tsoba, et David battit vingt-deux mille Syriens.
\VS{6}David mit des garnisons dans la Syrie de Damas. Et les Syriens furent assujettis à David, et lui payèrent un tribut. Yahweh protégeait David partout où il allait.
\VS{7}Et David prit les boucliers d'or qui étaient aux serviteurs d'Hadadézer, et les apporta à Jérusalem.
\VS{8}Le roi David emporta aussi une grande quantité d'airain de Béthach, et de Bérothaï, villes d'Hadadézer.
\VS{9}Thoï, roi de Hamath, apprit que David avait battu toute l'armée d'Hadadézer,
\VS{10}et il envoya Joram, son fils, vers le roi David, pour le saluer et pour le féliciter d'avoir fait la guerre contre Hadadézer et de l'avoir battu. Car Hadadézer était continuellement en guerre avec Thoï. Joram apporta des vases d'argent, des vases d'or, et des vases d'airain.
\VS{11}Le roi David les consacra à Yahweh, avec l'argent et l'or qu'il avait déjà consacrés du butin de toutes les nations qu'il s'était assujetties,
\VS{12}de la Syrie, de Moab, des fils d'Ammon, des Philistins, d'Amalek, et du butin d'Hadadézer, fils de Rehob, roi de Tsoba.
\VS{13}Au retour de la défaite des Syriens, David se fit encore un nom, en battant dans la vallée du sel dix-huit mille Edomites.
\VS{14}Il mit des garnisons dans Edom, il mit des garnisons dans tout Edom. Et tout Edom fut assujetti à David. Yahweh protégeait David partout où il allait.
\VS{15}Ainsi David régna sur tout Israël, et il faisait droit et justice à tout son peuple.
\VS{16}Joab, fils de Tseruja, commandait l'armée ; Josaphat, fils d'Achilud, était archiviste ;
\VS{17}Tsadok, fils d'Achithub, et Achimélec, fils d'Abiathar, étaient sacrificateurs ; Seraja était secrétaire ;
\VS{18}Benaja, fils de Jehojada, était chef des Kéréthiens et des Péléthiens ; et les fils de David étaient ministres d'Etat.
\Chap{9}
\TextTitle{Mephiboscheth à la table de David}
\VerseOne{}Alors David dit : Ne reste-t-il donc personne de la maison de Saül, afin que je lui fasse du bien pour l'amour de Jonathan ?
\VS{2}Il y avait dans la maison de Saül un serviteur nommé Tsiba, que l'on fit venir auprès de David. Le roi lui dit : Es-tu Tsiba ? Et il répondit : Je suis ton serviteur !
\VS{3}Le roi dit : N'y a-t-il plus personne de la maison de Saül, pour que j'use envers lui de la bonté de Dieu ? Tsiba répondit au roi : Il y a encore un des fils de Jonathan, qui est perclus des pieds.
\VS{4}Le roi lui dit : Où est-il ? Et Tsiba répondit au roi : Il est dans la maison de Makir, fils d'Ammiel, à Lodebar.
\VS{5}Alors le roi David l'envoya chercher dans la maison de Makir, fils d'Ammiel, à Lodebar.
\VS{6}Quand Mephiboscheth, fils de Jonathan, fils de Saül, vint auprès de David, il tomba sur sa face et se prosterna. David dit : Mephiboscheth ! Et il répondit : Voici ton serviteur.
\VS{7}David lui dit : Ne crains point, car certainement je te ferai du bien pour l'amour de Jonathan, ton père. Je te restituerai toutes les terres de Saül, ton père, et tu mangeras toujours du pain à ma table.
\VS{8}Il se prosterna, et dit : Qui suis-je, moi ton serviteur, pour que tu regardes un chien mort tel que moi ?
\VS{9}Le roi appela Tsiba, serviteur de Saül, et lui dit : Je donne au fils de ton maître tout ce qui appartenait à Saül et à toute sa maison.
\VS{10}Tu cultiveras pour lui ces terres, toi, tes fils, et tes serviteurs, et tu en recueilleras les fruits, afin que le fils de ton maître ait du pain à manger ; et Mephiboscheth, fils de ton maître, mangera toujours du pain à ma table. Or Tsiba avait quinze fils et vingt serviteurs.
\VS{11}Tsiba dit au roi : Ton serviteur fera tout ce que le roi, mon seigneur, ordonne à son serviteur. Et Mephiboscheth mangea à la table de David comme l'un des fils du roi.
\VS{12}Mephiboscheth avait un jeune fils, nommé Mica, et tous ceux qui demeuraient dans la maison de Tsiba étaient serviteurs de Mephiboscheth.
\VS{13}Mephiboscheth habitait à Jérusalem parce qu'il mangeait toujours à la table du roi. Il était boiteux des deux pieds.
\Chap{10}
\TextTitle{Double bataille contre les Ammonites et les Syriens}
\VerseOne{}Or il arriva après cela que le roi des fils d'Ammon mourût, et Hanun, son fils, régna à sa place.
\VS{2}Et David dit : J'userai de bonté envers Hanun, fils de Nachasch, comme son père en a usé envers moi. Ainsi David lui envoya ses serviteurs pour le consoler au sujet de son père. Lorsque les serviteurs de David arrivèrent dans le pays des fils d'Ammon,
\VS{3}les chefs des fils d'Ammon dirent à Hanun, leur maître : Penses-tu que ce soit pour honorer ton père que David t'envoie des consolateurs ? N'est-ce pas pour reconnaître exactement la ville et pour l'épier, afin de la détruire, que David envoie ses serviteurs auprès de toi ?
\VS{4}Alors Hanun saisit les serviteurs de David, et fit raser la moitié de leur barbe, et couper la moitié de leurs habits jusqu'aux hanches. Puis il les renvoya.
\VS{5}David en fut informé et envoya des gens à leur rencontre, car ces hommes étaient accablés de honte ; et le roi leur fit dire : Restez à Jéricho jusqu'à ce que votre barbe ait repoussé, et revenez ensuite.
\VS{6}Les fils d'Ammon, voyant qu'ils s'étaient rendus odieux à David, firent enrôler à leur solde vingt mille hommes de pied chez les Syriens de Beth-Rehob, et chez les Syriens de Tsoba, mille hommes chez le roi de Maaca, et douze mille hommes chez les gens de Tob.
\VS{7}David l'ayant appris, envoya Joab et toute l'armée, les hommes les plus vaillants.
\VS{8}Les fils d'Ammon sortirent et se rangèrent en bataille à l'entrée de la porte ; les Syriens de Tsoba de Rehob, et les hommes de Tob et de Maaca étaient à part dans la campagne.
\VS{9}Joab, voyant que leur armée était tournée contre lui devant et derrière, choisit alors des hommes d'élite parmi tous ceux d'Israël, et les rangea contre les Syriens ;
\VS{10}et il donna le commandement du reste du peuple à Abischaï, son frère, pour le ranger en bataille contre les fils d'Ammon.
\VS{11}Il dit : Si les Syriens sont plus forts que moi, tu viendras à mon secours ; et si les fils d'Ammon sont plus forts que toi, j'irai te secourir.
\VS{12}Sois vaillant, et portons-nous vaillamment pour notre peuple et pour les villes de notre Dieu, et que Yahweh fasse ce qu'il lui semblera bon !
\VS{13}Alors Joab et le peuple qui était avec lui s'approchèrent pour livrer bataille aux Syriens, et ils s'enfuirent devant lui.
\VS{14}Quand les fils d'Ammon virent que les Syriens avaient pris la fuite, ils s'enfuirent aussi devant Abischaï et rentrèrent dans la ville. Joab s'éloigna des fils d'Ammon et revint à Jérusalem.
\VS{15}Les Syriens, voyant qu'ils avaient été battus par Israël, se rassemblèrent.
\VS{16}Hadarézer envoya chercher les Syriens qui étaient de l'autre côté du fleuve ; et ils arrivèrent à Hélam, et Schobac, chef de l'armée d'Hadarézer, les conduisait.
\VS{17}Cela fut rapporté à David, qui assembla tout Israël, passa le Jourdain, et vint à Hélam. Les Syriens se rangèrent en bataille contre David, et combattirent contre lui.
\VS{18}Mais les Syriens s'enfuirent devant Israël. Et David défit sept cents chars des Syriens et quarante mille cavaliers ; il frappa aussi Schobac, le chef de leur armée, qui mourut sur place.
\VS{19}Tous les rois soumis à Hadarézer, se voyant battus par Israël, firent la paix avec Israël et lui furent assujettis. Et les Syriens craignirent désormais de secourir les fils d'Ammon.
\Chap{11}
\TextTitle{Péché de David avec Bath-Schéba}
\VerseOne{}Et il arriva, l'année suivante, au temps où les rois partaient en guerre, que David envoya Joab, avec ses serviteurs et tout Israël, pour détruire les fils d'Ammon et assiéger Rabba. Mais David resta à Jérusalem\FTNT{Au lieu d'aller en guerre et de diriger les troupes, David resta à Jérusalem. Cette négligence l'a conduit à la convoitise, à l'adultère et au meurtre d'Urie. La distraction peut conduire à la mort. Il y a un temps pour toutes choses (Ec. 3).}.
\VS{2}Et il arriva, sur le soir, que David se leva de sa couche ; et comme il se promenait sur le toit de la maison royale, il aperçut de là une femme qui se baignait, et cette femme était très belle de figure.
\VS{3}David envoya demander qui était cette femme, et on lui dit : N'est-ce pas Bath-Schéba, fille d'Eliam, femme d'Urie, le Héthien ?
\VS{4}Et David envoya des messagers pour la chercher. Elle vint vers lui, et il coucha avec elle. Après s'être purifiée de sa souillure, elle retourna dans sa maison.
\VS{5}Cette femme devint enceinte, et elle fit dire à David : Je suis enceinte.
\VS{6}Alors David envoya dire à Joab : Envoie-moi Urie, le Héthien. Et Joab envoya Urie à David.
\VS{7}Urie se rendit auprès de David, qui l'interrogea sur l'état de Joab, sur l'état du peuple, et sur l'état de la guerre.
\VS{8}Puis David dit à Urie : Descends dans ta maison, et lave tes pieds. Urie sortit de la maison du roi, et on fit porter après lui un présent royal.
\VS{9}Mais Urie se coucha à la porte de la maison du roi, avec tous les serviteurs de son maître, et il ne descendit point dans sa maison.
\VS{10}On le rapporta à David, et on lui dit : Urie n'est pas descendu dans sa maison. David dit à Urie : N'arrives-tu pas de voyage ? Pourquoi n'es-tu pas descendu dans ta maison ?
\VS{11}Urie répondit à David : L'arche et Israël et Juda habitent sous des tentes, mon seigneur Joab et les serviteurs de mon seigneur campent aux champs, et moi j'entrerais dans ma maison pour manger et boire et pour coucher avec ma femme ! Tu es vivant, et ton âme est vivante, je ne ferai point une telle chose.
\VS{12}David dit à Urie : Reste ici encore aujourd'hui, et demain je te renverrai. Urie resta donc ce jour-là et le lendemain à Jérusalem.
\VS{13}David l'invita à manger et à boire en sa présence, et il l'enivra ; néanmoins le soir, Urie sortit pour dormir sur sa couche, avec tous les serviteurs de son maître, et il ne descendit point dans sa maison.
\VS{14}Le lendemain matin, David écrivit une lettre à Joab, et l'envoya par la main d'Urie.
\VS{15}Il écrivit en ces termes : Placez Urie à l'endroit où sera le plus fort de la bataille et éloignez-vous de lui, afin qu'il soit frappé et qu'il meure.
\VS{16}Joab, en observant la ville, plaça Urie à l'endroit qu'il savait défendu par de vaillants soldats.
\VS{17}Les hommes de la ville sortirent et combattirent contre Joab ; et quelques-uns du peuple qui étaient des serviteurs de David moururent, et Urie, le Héthien, mourut aussi.
\VS{18}Alors Joab envoya un messager à David pour lui faire savoir tout ce qui était arrivé dans ce combat.
\VS{19}Il donna cet ordre au messager : Quand tu auras achevé de raconter au roi tout ce qui est arrivé au combat, peut-être se mettra-t-il en fureur et te dira : Pourquoi vous êtes-vous approchés de la ville pour combattre ? Ne savez-vous pas bien qu'on tire de dessus la muraille ?
\VS{21}Qui a tué Abimélec, fils de Jerubbéscheth ? N'est-ce pas une femme qui lança sur lui de dessus la muraille une pièce de meule de moulin, et n'en est-il pas mort à Thébets ? Pourquoi vous êtes-vous approchés de la muraille ? Alors tu lui diras : Ton serviteur Urie, le Héthien, est mort aussi.
\VS{22}Le messager partit. A son arrivée, il fit savoir à David tout ce pourquoi Joab l'avait envoyé.
\VS{23}Le messager dit à David : Ces gens ont été plus forts que nous; ils avaient fait une sortie contre nous dans les champs, mais nous les avons repoussés jusqu'à l'entrée de la porte ;
\VS{24}les archers ont tiré sur tes serviteurs du haut de la muraille, et plusieurs des serviteurs du roi ont été tués, ton serviteur Urie, le Héthien, est mort aussi.
\VS{25}David dit au messager : Tu diras ainsi à Joab : Ne sois point peiné de cette affaire, car l'épée dévore tantôt l'un, tantôt l'autre ; attaque vigoureusement la ville, et détruis-la. Et toi, encourage-le !
\VS{26}La femme d'Urie apprit qu'Urie, son mari, était mort, et elle pleura son mari.
\VS{27}Quand le deuil fut passé, David l'envoya chercher et la recueillit dans sa maison. Elle devint sa femme, et lui enfanta un fils. Ce que David avait fait était mal aux yeux de Yahweh.
\Chap{12}
\TextTitle{Le prophète Nathan envoyé pour reprendre David}
\VerseOne{}Yahweh envoya Nathan vers David. Nathan vint à lui, et lui dit : Il y avait deux hommes dans une ville, l'un riche et l'autre pauvre.
\VS{2}Le riche avait des brebis et des bœufs en très grand nombre.
\VS{3}Le pauvre n'avait rien du tout sauf une petite brebis, qu'il avait achetée ; il la nourrissait et elle grandissait chez lui avec ses enfants ; elle mangeait de son pain, buvait dans sa coupe, dormait sur son sein et elle était comme sa fille.
\VS{4}Un voyageur arriva chez l'homme riche. Ce riche a épargné ses brebis et ses bœufs, pour préparer un repas au voyageur qui était venu chez lui ; il a pris la brebis du pauvre homme, et l'a apprêtée pour l'homme qui était venu chez lui.
\VS{5}Alors la colère de David s'enflamma violemment contre cet homme, et il dit à Nathan : Yahweh est vivant ! L'homme qui a fait cela mérite la mort.
\VS{6}Parce qu'il a fait cela et qu'il n'a pas épargné cette brebis, pour une brebis il en rendra quatre.
\VS{7}Alors Nathan dit à David : Tu es cet homme-là ! Ainsi parle Yahweh, le Dieu d'Israël : Je t'ai oint pour roi sur Israël, et je t'ai délivré de la main de Saül ;
\VS{8}je t'ai même donné la maison de ton maître, et les femmes de ton maître dans ton sein, et je t'ai donné la maison d'Israël, et de Juda. Et si cela avait été peu, j'y aurais encore ajouté.
\VS{9}Pourquoi donc as-tu méprisé la parole de Yahweh, en faisant ce qui est mal à ses yeux ? Tu as frappé de l'épée Urie, le Héthien ; tu as pris sa femme pour en faire ta femme, et tu l'as tué par l'épée des fils d'Ammon.
\VS{10}Maintenant, l'épée ne s'éloignera jamais de ta maison, parce que tu m'as méprisé, et que tu as pris la femme d'Urie, le Héthien, pour en faire ta femme.
\VS{11}Ainsi parle Yahweh : Voici, je vais faire sortir de ta propre maison le malheur contre toi, et je vais prendre sous tes yeux tes propres femmes pour les donner à un homme de ta maison, qui couchera avec elles à la vue de ce soleil.
\VS{12}Car tu as agi en secret ; mais moi, je le ferai en présence de tout Israël et à la face du soleil.
\TextTitle{Repentance de David}
\VS{13}David dit à Nathan : J'ai péché contre Yahweh ! Et Nathan dit à David : Yahweh passe par-dessus ton péché, tu ne mourras point.
\VS{14}Toutefois, parce qu'en commettant cela, tu as donné l'occasion aux ennemis de Yahweh de le blasphémer, à cause de cela le fils qui t'est né mourra certainement.
\VS{15}Et Nathan retourna dans sa maison. Yahweh frappa l'enfant que la femme d'Urie avait enfanté à David, et il devint gravement malade.
\VS{16}David pria Dieu pour l'enfant, et David jeûna ; et quand il rentra, il passa la nuit couché par terre.
\VS{17}Les anciens de sa maison se levèrent et vinrent vers lui pour le faire lever de terre ; mais il ne voulut point, et il ne mangea rien avec eux.
\VS{18}Et il arriva que l'enfant mourut le septième jour. Les serviteurs de David craignaient de lui annoncer que l'enfant était mort. Car ils disaient : Voici, quand l'enfant vivait encore, nous lui avons parlé, et il n'a pas écouté notre voix ; comment donc lui dirions-nous : L'enfant est mort ? Il s'affligera bien davantage.
\VS{19}David vit que ses serviteurs parlaient à voix basse, et il comprit que l'enfant était mort. David dit à ses serviteurs : L'enfant est-il mort ? Ils répondirent : Il est mort.
\VS{20}Alors David se leva de terre. Il se lava, s'oignit, et changea de vêtements ; il alla dans la maison de Yahweh, et se prosterna. De retour chez lui, il demanda à manger ; on mit de la viande devant lui et il mangea.
\VS{21}Ses serviteurs lui dirent : Qu'est-ce que tu fais ? Tu jeûnais et pleurais pour l'amour de l'enfant lorsqu'il vivait encore ; et maintenant que l'enfant est mort, tu te lèves et tu manges !
\VS{22}Mais il répondit : Quand l'enfant vivait encore, je jeûnais et pleurais, car je disais : Qui sait si Yahweh n'aura pas pitié de moi et si l'enfant ne vivra pas ?
\VS{23}Maintenant qu'il est mort, pourquoi jeûnerais-je ? Puis-je le faire revenir ? J'irai vers lui, mais il ne reviendra pas vers moi.
\TextTitle{Naissance de Salomon}
\VS{24}David consola sa femme Bath-Schéba, et il alla auprès d'elle et coucha avec elle. Elle lui enfanta un fils qu'il nomma Salomon et qui fut aimé de Yahweh.
\VS{25}Il le remit entre les mains de Nathan, le prophète, qui lui donna le nom de Jedidja, à cause de Yahweh.
\TextTitle{Le pays et le roi de Rabba livrés à Joab et David (1 Ch. 20:1-3)}
\VS{26}Joab combattait contre Rabba, qui appartenait aux fils d'Ammon, il s'empara de la ville royale,
\VS{27}et envoya des messagers à David pour lui dire : J'ai attaqué Rabba, et j'ai pris la ville des eaux ;
\VS{28}rassemble maintenant le reste du peuple, campe contre la ville, et prends-la, de peur que je ne m'en empare et que la gloire m'en soit attribuée.
\VS{29}David rassembla tout le peuple et marcha contre Rabba ; il l'attaqua et la prit.
\VS{30}Il enleva la couronne de dessus la tête de son roi ; elle pesait un talent d'or et était garnie de pierres précieuses. On la mit sur la tête de David, qui emporta de la ville un très grand butin.
\VS{31}Il fit sortir aussi le peuple qui s'y trouvait, et il les plaça sous des scies, des herses de fer, des haches de fer et le fit passer par un fourneau où l'on cuit les briques ; il traita ainsi toutes les villes des fils d'Ammon. Puis David retourna avec tout le peuple à Jérusalem.
\Chap{13}
\TextTitle{David subit les conséquences de son péché}
\VerseOne{}Or il arriva après cela qu'Absalom, fils de David, ayant une sœur qui était belle et qui se nommait Tamar ; et Amnon, fils de David, l'aima.
\TextTitle{Inceste au sein de la famille royale}
\VS{2}Et Amnon fut si tourmenté qu'il tomba malade à cause de Tamar sa sœur, car elle était vierge ; et il paraissait trop difficile à Amnon d'obtenir la moindre chose d'elle.
\VS{3}Amnon avait un ami, nommé Jonadab, fils de Schimea, frère de David, et Jonadab était un homme très rusé.
\VS{4}Il lui dit : Fils de roi, pourquoi maigris-tu ainsi de jour en jour ? Ne veux-tu pas me le dire ? Amnon lui dit : J'aime Tamar, la sœur de mon frère, Absalom.
\VS{5}Jonadab lui dit : Couche-toi dans ton lit et fais le malade. Quand ton père viendra te voir, tu lui diras : Permets à Tamar, ma sœur, de venir pour me donner à manger ; qu'elle prépare un mets sous mes yeux, afin que je le voie et que je le prenne de sa main.
\VS{6}Amnon se coucha et fit le malade. Le roi vint le voir, et Amnon dit au roi : Je te prie, que ma sœur Tamar vienne faire deux beignets sous mes yeux, et que je les mange de sa main.
\VS{7}David envoya dire à Tamar dans la maison : Va dans la maison de ton frère Amnon, et prépare-lui quelque chose d'appétissant.
\VS{8}Tamar alla dans la maison de son frère Amnon, qui était couché. Elle prit de la pâte, la pétrit, et en fit devant lui des beignets et les fit cuire.
\VS{9}Puis elle prit la poêle, et elle les versa devant lui. Mais Amnon refusa d'en manger. Il dit : Faites sortir tous ceux qui sont auprès de moi. Et tout le monde se retira.
\VS{10}Alors Amnon dit à Tamar : Apporte-moi le mets dans la chambre, et que je le mange de ta main. Tamar prit les beignets qu'elle avait faits, et les apporta à Amnon, son frère dans la chambre.
\VS{11}Comme elle les lui présentait pour qu'il en mange, il se saisit d'elle et lui dit : Viens, couche avec moi, ma sœur !
\VS{12}Elle lui répondit : Non, mon frère, ne me déshonore pas, car cela ne se fait point en Israël ; ne commets pas cette infamie.
\VS{13}Et moi, où irais-je avec mon opprobre ? Et toi, tu serais comme l'un des infâmes en Israël. Maintenant, je te prie, parle au roi, et il ne s'opposera pas à ce que je sois ta femme.
\VS{14}Mais il ne voulut pas écouter sa parole ; il fut plus fort qu'elle, lui fit violence et coucha avec elle\FTNT{Le viol et l'inceste que commit Amnon, fils de David, sur Tamar, sa demi-sœur, furent les conséquences du péché de David avec Bath-Schéba.}.
\VS{15}Après cela, Amnon eut pour elle une très grande haine, en sorte que la haine qu'il lui portait était plus grande que l'amour qu'il avait eu pour elle. Ainsi, Amnon lui dit : Lève-toi, va-t'en !
\VS{16}Elle lui répondit : Tu n'as aucune raison de me faire ce mal, que de me chasser, ce mal est plus grand que l'autre que tu m'as fait.
\VS{17}Mais il ne voulut point l'écouter, et appelant le garçon qui le servait, il dit : Qu'on chasse cette femme loin de moi, qu'on la mette dehors. Et ferme la porte après elle !
\VS{18}Elle était habillée d'une tunique de couleurs ; car les filles du roi, qui étaient encore vierges, s'habillaient ainsi. Le serviteur d'Amnon la mit dehors, et ferma la porte après elle.
\VS{19}Alors Tamar répandit de la cendre sur sa tête, et déchira sa tunique de couleurs ; elle mit la main sur sa tête, et s'en alla en poussant des cris.
\VS{20}Et son frère Absalom lui dit : Ton frère, Amnon, a-t-il été avec toi ? Maintenant, ma sœur, tais-toi, c'est ton frère ; ne prends pas cette affaire à cœur. Et Tamar, désolée, demeura dans la maison d'Absalom, son frère.
\VS{21}Quand le roi David eut appris toutes ces choses, il fut très irrité.
\VS{22}Absalom ne parla ni en bien ni en mal avec Amnon ; mais il le prit en haine, parce qu'il avait déshonoré Tamar, sa sœur.
\TextTitle{Vengeance d'Absalom sur Amnon}
\VS{23}Et il arriva au bout de deux années entières, qu'Absalom ayant les tondeurs à Baal-Hatsor, près d'Ephraïm, invita tous les fils du roi.
\VS{24}Absalom alla vers le roi, et dit : Voici, ton serviteur a les tondeurs ; je te prie que le roi et ses serviteurs viennent avec ton serviteur.
\VS{25}Et le roi dit à Absalom : Non, mon fils, nous n'irons pas tous, de peur que nous ne te soyons à charge. Absalom le pressa ; mais le roi ne voulut point aller, et il le bénit.
\VS{26}Absalom dit : Permets au moins à Amnon, mon frère, de venir avec nous. Le roi lui répondit : Pourquoi irait-il ?
\VS{27}Absalom le pressa tellement qu'il laissa aller Amnon et tous les fils du roi avec lui.
\VS{28}Or Absalom avait donné cet ordre à ses serviteurs, en disant : Prenez bien garde, je vous prie, quand le cœur d'Amnon sera égayé par le vin et que je vous dirai : Frappez Amnon ! Tuez-le ; ne craignez point, n'est-ce pas moi qui vous l'ordonne ? Fortifiez-vous et portez-vous en vaillants hommes !
\VS{29}Les serviteurs d'Absalom traitèrent Amnon comme Absalom l'avait ordonné. Et tous les fils du roi se levèrent, montèrent chacun sur son mulet, et s'enfuirent.
\VS{30}Et il arriva, comme ils étaient en chemin, que le bruit parvint à David qu'Absalom avait tué tous les fils du roi, et qu'il n'en était pas resté un seul d'entre eux.
\VS{31}Le roi se leva, déchira ses vêtements, et se coucha par terre ; et tous ses serviteurs étaient là, avec leurs vêtements déchirés.
\VS{32}Jonadab, fils de Schimea, frère de David, prit la parole, et dit : Que mon seigneur ne dise point que tous les jeunes hommes, fils du roi, ont été tués, car seul Amnon est mort ; car c'était là le dessein d'Absalom, depuis le jour où Amnon a violé Tamar, sa sœur ; car il a été exécuté selon son commandement.
\VS{33}Maintenant donc, que le roi mon seigneur ne prenne point la chose à cœur, en disant que tous les fils du roi sont morts, car Amnon seul est mort.
\VS{34}Absalom prit la fuite. Or le jeune homme placé en sentinelle leva les yeux et regarda. Et voici, un grand peuple venait par le chemin qui était derrière lui, du côté de la montagne.
\VS{35}Jonadab dit au roi : Voici les fils du roi qui arrivent ! Ainsi se confirme ce que disait ton serviteur.
\VS{36}Comme il achevait de parler, voici, les fils du roi arrivèrent. Ils élevèrent la voix et pleurèrent ; le roi aussi et tous ses serviteurs versèrent d'abondantes larmes.
\TextTitle{Absalom s'enfuit loin de son père}
\VS{37}Absalom s'était enfui, et il alla chez Talmaï, fils d'Ammihur, roi de Gueschur\FTNT{Absalom s'était réfugié chez Talmaï, roi de Gueschur (Transjordanie, au nord de la Syrie), qui était le père de Maaca, sa mère (2 S. 3:3). Il est donc allé chez son grand-père maternel.}. Et David pleurait tous les jours son fils.
\VS{38}Absalom resta trois ans à Gueschur, où il était allé, après avoir pris la fuite.
\VS{39}Le roi David cessa de poursuivre Absalom, car il était consolé de la mort d'Amnon.
\Chap{14}
\TextTitle{Joab convainc le roi de faire revenir Absalom}
\VerseOne{}Alors Joab, fils de Tseruja, s'aperçut que le cœur du roi était pour Absalom.
\VS{2}Il envoya chercher à Tekoa une femme habile, et il lui dit : Fais semblant de te lamenter, et revêts des habits de deuil ; ne t'oins pas d'huile, mais sois comme une femme qui depuis longtemps pleure un mort.
\VS{3}Ensuite va vers le roi, et tu lui parleras de cette manière. Joab lui mit dans la bouche ce qu'elle devait dire.
\VS{4}La femme de Tekoa alla parler au roi. Elle tomba la face contre terre, se prosterna et dit : Ô roi, sauve-moi !
\VS{5}Le roi lui dit : Qu'as-tu ? Elle répondit : Certainement, je suis une femme veuve, et mon mari est mort !
\VS{6}Or ta servante avait deux fils ; ils se sont tous deux querellés dans les champs, et il n'y avait personne pour les séparer ; l'un a frappé l'autre et l'a tué.
\VS{7}Et voici, toute la famille s'est élevée contre ta servante, en disant : Donne-nous le meurtrier de son frère ! Nous voulons le faire mourir, pour la vie de son frère qu'il a tué ; et que nous exterminions même l'héritier ! Ils veulent ainsi éteindre le charbon vif qui me restait, pour ne laisser à mon mari ni nom ni survivant sur la face de la terre.
\VS{8}Le roi dit à la femme : Va-t-en dans ta maison, et je donnerai des ordres en ta faveur.
\VS{9}Alors la femme de Tekoa dit au roi : Mon seigneur et mon roi ! Que l'iniquité soit sur moi et sur la maison de mon père, et que le roi et son trône en soient innocents.
\VS{10}Et le roi répondit : Si quelqu'un parle contre toi, amène-le-moi, et jamais il ne lui arrivera de te toucher.
\VS{11}Et elle dit : Je te prie, que le roi se souvienne de Yahweh, son Dieu, afin que le vengeur de sang n'augmente pas la ruine et qu'on ne fasse pas périr mon fils. Et il répondit : Yahweh est vivant ! Il ne tombera pas à terre un seul des cheveux de ton fils.
\VS{12}La femme dit : Je te prie que ta servante dise un mot au roi, mon seigneur. Et il répondit : Parle !
\VS{13}La femme dit : Mais pourquoi as-tu pensé une chose comme celle-ci contre le peuple de Dieu ? Puisqu'en tenant ce discours, le roi se déclare coupable en ce qu'il n'a pas fait revenir celui qu'il a banni ?
\VS{14}Car nous mourrons certainement, et nous sommes comme l'eau versée sur la terre qu'on ne peut recueillir. Dieu n'ôte pas la vie, mais il médite les moyens de ne pas repousser loin de lui celui qui est banni de sa présence.
\VS{15}Maintenant, si je suis venue pour tenir ce discours au roi, mon seigneur, c'est parce que le peuple m'a effrayée. Et ta servante a dit : Je veux parler maintenant au roi ; peut-être que le roi fera ce que sa servante lui dira.
\VS{16}Oui, car le roi écoutera sa servante pour la délivrer de la main de celui qui veut nous exterminer, moi et mon fils, de l'héritage de Dieu.
\VS{17}Ta servante a dit : Que la parole du roi, mon seigneur, nous apporte du repos. Car le roi mon seigneur est comme un ange de Dieu, pour entendre le bien et le mal. Que Yahweh, ton Dieu, soit avec toi !
\VS{18}Le roi répondit, et dit à la femme : Je te prie, ne me cache rien de ce que je te vais te demander. Et la femme dit : Que le roi mon seigneur parle !
\VS{19}Et le roi dit : La main de Joab n'est-elle pas avec toi dans tout ceci ? Et la femme répondit et dit : Ton âme vit, ô mon seigneur, qu'on ne saurait se détourner ni à droite ni à gauche de tout ce que dit le roi mon seigneur. C'est en effet ton serviteur Joab qui m'a donné des ordres et qui a mis dans la bouche de ta servante toutes ces paroles.
\VS{20}C'est ton serviteur Joab qui a fait que j'ai ainsi tourné ce discours. Mais mon seigneur est sage comme un ange de Dieu, pour savoir tout ce qui se passe sur la terre.
\TextTitle{Retour d'Absalom à Jérusalem}
\VS{21}Alors le roi dit à Joab : Voici, maintenant c'est toi qui as conduit cette affaire ; va donc, et fais revenir le jeune homme Absalom.
\VS{22}Et Joab tomba la face contre terre et se prosterna, et il bénit le roi. Puis il dit : Aujourd'hui, ton serviteur sait qu'il a trouvé grâce à tes yeux, ô roi mon seigneur, puisque le roi agit selon ce que son serviteur lui a dit.
\VS{23}Joab se leva et partit pour Gueschur, et il ramena Absalom à Jérusalem.
\VS{24}Mais le roi dit : Qu'il se retire dans sa maison, et qu'il ne voie point ma face. Et Absalom se retira dans sa maison, et ne vit point la face du roi.
\VS{25}Il n'y avait point d'homme dans tout Israël aussi renommé qu'Absalom pour sa beauté ; depuis la plante des pieds jusqu'au sommet de la tête, il n'y avait point en lui de défaut.
\VS{26}Et quand il faisait couper ses cheveux, or il arrivait tous les ans qu'il les faisait couper, parce que sa chevelure lui pesait trop, le poids de sa chevelure était de deux cents sicles, poids du roi.
\VS{27}Il naquit à Absalom trois fils, et une fille nommée Tamar, qui était une femme belle de figure.
\VS{28}Et Absalom demeura deux ans entiers à Jérusalem, sans voir la face du roi.
\VS{29}Absalom fit demander Joab, pour l'envoyer vers le roi ; mais Joab ne voulut pas venir vers lui ; il le fit demander encore pour la seconde fois ; mais Joab ne voulut point venir.
\VS{30}Absalom dit alors à ses serviteurs : Voyez le champ de Joab qui est à côté du mien ; il y a de l'orge ; allez et mettez-y le feu. Et les serviteurs d'Absalom mirent le feu au champ.
\VS{31}Alors Joab se leva et vint vers Absalom dans sa maison. Il lui dit : Pourquoi tes serviteurs ont-ils mis le feu à mon champ?
\VS{32}Et Absalom répondit à Joab : Voici, je t'ai fait dire : Viens ici, et je t'enverrai vers le roi, afin que tu lui dises : Pourquoi suis-je revenu de Gueschur ? Il vaudrait mieux pour moi que j'y fusse encore. Je désire maintenant voir la face du roi ; et s'il y a de l'iniquité en moi, qu'il me fasse mourir.
\VS{33}Joab alla vers le roi, et lui rapporta cela. Et le roi appela Absalom, qui vint vers lui et se prosterna le visage contre terre devant le roi. Le roi embrassa Absalom.
\Chap{15}
\TextTitle{Mauvaises intentions d'Absalom}
\VerseOne{}Or il arriva qu'après cela, Absalom se procura des chars et des chevaux, et il avait cinquante hommes qui couraient devant lui\FTNT{La révolte d'Absalom était une autre conséquence du péché de David avec Bath-Schéba.}.
\VS{2}Absalom se levait de bon matin et se tenait au bord du chemin de la porte. Et chaque fois qu'un homme ayant une contestation se rendait auprès du roi pour obtenir justice, Absalom l'appelait, et lui disait : De quelle ville es-tu ? Et il répondait : Ton serviteur est de l'une des tribus d'Israël.
\VS{3}Absalom lui disait : Vois, ta cause est bonne et droite ; mais personne de chez le roi ne t'écoutera.
\VS{4}Absalom disait encore : Qui m'établira juge dans le pays ? Tout homme qui aurait une contestation et un procès viendrait vers moi, et je lui ferais justice.
\VS{5}Et il arrivait aussi que quand quelqu'un s'approchait de lui pour se prosterner, il lui tendait sa main, le saisissait, et l'embrassait.
\VS{6}Absalom faisait ainsi à tous ceux d'Israël qui venaient vers le roi pour demander justice. Et Absalom gagnait les cœurs des hommes d'Israël.
\TextTitle{Conspiration d'Absalom}
\VS{7}Et il arriva qu'au bout de quarante ans, Absalom dit au roi : Permets que j'aille à Hébron, pour accomplir le vœu que j'ai fait à Yahweh.
\VS{8}Car quand ton serviteur demeurait à Gueschur en Syrie, il fit un vœu, en disant : Si Yahweh me ramène à Jérusalem, j'en témoignerai ma reconnaissance à Yahweh.
\VS{9}Et le roi lui répondit : Va en paix. Et Absalom se leva et s'en alla à Hébron.
\VS{10}Absalom envoya des espions dans toutes les tribus d'Israël, pour dire : Aussitôt que vous entendrez le son du shofar, vous direz : Absalom est établi roi à Hébron !
\VS{11}Deux cents hommes de Jérusalem, qui avaient été invités, s'en allèrent avec Absalom ; ils y allèrent en toute simplicité de coeur, ne sachant rien de cette affaire.
\VS{12}Pendant qu'Absalom offrait les sacrifices, il envoya chercher à la ville de Guilo, Achitophel, le Guilonite, conseiller de David. Il se forma une puissante conspiration, parce que le peuple était de plus en plus nombreux auprès d'Absalom.
\TextTitle{David fuit son fils Absalom}
\VS{13}Un messager se rendit auprès de David, et lui dit : Le cœur des hommes d'Israël s'est tourné vers Absalom.
\VS{14}Et David dit à tous ses serviteurs qui étaient avec lui à Jérusalem : Levez-vous, fuyons, car nous ne pourrons échapper à Absalom. Hâtez-vous de partir ; sinon, il ne tarderait pas à nous atteindre, et il nous précipiterait dans le malheur et frapperait la ville du tranchant de l'épée.
\VS{15}Les serviteurs du roi lui répondirent : Tes serviteurs feront tout ce que le roi, notre seigneur, voudra.
\VS{16}Le roi sortit, et toute sa maison le suivait, mais le roi laissa dix femmes, ses concubines, pour garder la maison.
\VS{17}Le roi sortit, et tout le peuple le suivait, et ils s'arrêtèrent à Beth-Merkhak.
\VS{18}Tous ses serviteurs marchaient à côté de lui ; tous les Kéréthiens, tous les Péléthiens, et tous les Gathiens, qui étaient six cents hommes venus de Gath, pour être à sa suite, marchaient devant le roi.
\VS{19}Mais le roi dit à Ittaï de Gath : Pourquoi viendrais-tu aussi avec nous ? Retourne et reste avec le roi, car tu es étranger, et même tu vas retourner bientôt en ton lieu.
\VS{20}Tu es arrivé hier, et te ferais-je aujourd'hui errer çà et là avec nous ? Quant à moi, je m'en vais où je pourrai ! Retourne et emmène tes frères avec toi. Que la bonté et la vérité t'accompagnent !
\VS{21}Mais Ittaï répondit au roi, et dit : Yahweh est vivant, et le roi mon seigneur est vivant ! Quel que soit le lieu où le roi mon seigneur sera, soit pour mourir, soit pour vivre, ton serviteur y sera aussi.
\VS{22}David donc dit à Ittaï : Viens, et marche ! Alors Ittaï de Gath marcha avec tous ses gens et tous les enfants qui étaient avec lui.
\VS{23}Et tout le pays pleurait à grands cris et tout le peuple passait plus avant. Puis le roi passa le torrent de Cédron, et tout le peuple passa en face du chemin qui mène au désert.
\TextTitle{L'arche de l'alliance à Jérusalem}
\VS{24}Tsadok était aussi là, et avec lui tous les Lévites portant l'arche de l'alliance de Dieu ; et ils posèrent là l'arche de Dieu, et Abiathar montait, pendant que tout le peuple achevait de sortir de la ville.
\VS{25}Le roi dit à Tsadok : Rapporte l'arche de Dieu dans la ville. Si je trouve grâce aux yeux de Yahweh, il me ramènera, et il me fera voir l'arche et sa demeure.
\VS{26}Mais s'il dit : Je ne prends point de plaisir en toi ! Me voici, qu'il fasse de moi ce qui lui semblera bon.
\VS{27}Le roi dit encore au sacrificateur Tsadok : N'es-tu pas le voyant ? Retourne en paix dans la ville, avec Achimaats, ton fils, et Jonathan, fils d'Abiathar, vos deux fils.
\VS{28}Voyez, j'attendrai dans les plaines du désert, jusqu'à ce qu'on vienne m'apporter des nouvelles de votre part.
\VS{29}Ainsi, Tsadok et Abiathar rapportèrent l'arche de Dieu à Jérusalem, et ils y restèrent.
\VS{30}David monta par la montée des oliviers. Il montait en pleurant, la tête couverte, et marchait pieds nus ; tout le peuple qui était avec lui se couvrit aussi la tête, et il montait en pleurant.
\VS{31}Alors on vint dire à David : Achitophel est parmi ceux qui ont conspiré avec Absalom. Et David dit : Je te prie, ô Yahweh, abolis les conseils d'Achitophel !
\TextTitle{Huschaï, espion pour David dans la cour d'Absalom}
\VS{32}Et il arriva que quand David fut arrivé au sommet de la montagne, où il se prosterna devant Dieu, Huschaï, l'Arkien, vint au-devant de lui, la tunique déchirée et de la terre sur sa tête.
\VS{33}David lui dit : Tu me seras à charge si tu viens avec moi.
\VS{34}Et au contraire, tu anéantiras en ma faveur les conseils d'Achitophel, si tu retournes à la ville, et que tu dis à Absalom : Ô roi, je serai ton serviteur, comme je fus autrefois le serviteur de ton père ; mais maintenant je serai ton serviteur.
\VS{35}Les sacrificateurs Tsadok et Abiathar ne seront-ils pas là avec toi ? Tout ce que tu entendras de la maison du roi, tu le rapporteras aux sacrificateurs Tsadok et Abiathar.
\VS{36}Voici, ils ont là avec eux leurs deux fils, Achimaats, fils de Tsadok, et Jonathan, fils d'Abiathar ; c'est par eux que vous me ferez savoir tout ce que vous aurez entendu.
\VS{37}Huschaï, l'ami de David, retourna donc dans la ville, et Absalom entra à Jérusalem.
\Chap{16}
\TextTitle{Tsiba retrouve David en fuite}
\VerseOne{}Quand David eut un peu dépassé le sommet, voici, Tsiba, serviteur de Mephiboscheth, vint au-devant de lui avec deux ânes bâtés, sur lesquels il y avait deux cents pains, cent paquets de raisins secs, cent de fruits d'été, et une outre de vin.
\VS{2}Le roi dit à Tsiba : Que veux-tu faire de cela ? Et Tsiba répondit : Les ânes serviront de montures pour la maison du roi, le pain et les autres fruits d'été sont pour nourrir les jeunes gens, et le vin pour désaltérer ceux qui se seront fatigués dans le désert.
\VS{3}Le roi lui dit : Mais où est le fils de ton maître ? Et Tsiba répondit au roi : Voici, il est resté à Jérusalem, car il a dit : Aujourd'hui, la maison d'Israël me rendra le royaume de mon père.
\VS{4}Alors le roi dit à Tsiba : Voici, tout ce qui est à Mephiboscheth est à toi. Et Tsiba dit : Je me prosterne ! Que je trouve grâce à tes yeux, ô roi, mon seigneur !
\TextTitle{Schimeï maudit le roi David}
\VS{5}Le roi David était arrivé jusqu'à Bachurim. Et voici, il sortit de là un homme de la famille et de la maison de Saül, nommé Schimeï, fils de Guéra. Il s'avança en prononçant des malédictions,
\VS{6}il jeta des pierres contre David, contre tous ses serviteurs, et contre tout le peuple ; tous les hommes vaillants étaient à la droite et à la gauche du roi.
\VS{7}Schimeï parlait ainsi en le maudissant : Sors, sors, homme de sang, méchant homme !
\VS{8}Yahweh fait retomber sur toi tout le sang de la maison de Saül, à la place duquel tu régnais, et Yahweh a mis le royaume entre les mains de ton fils, Absalom ; et voilà, tu souffres le mal que tu as fait, parce que tu es un homme de sang !
\VS{9}Alors Abischaï, fils de Tseruja, dit au roi : Pourquoi ce chien mort maudit-il le roi, mon seigneur ? Permets que je m'avance et que je lui ôte la tête.
\VS{10}Mais le roi répondit : Qu'ai-je à faire avec vous, fils de Tseruja ? S'il maudit, c'est que Yahweh lui a dit : Maudis David ! Qui donc lui dira : Pourquoi agis-tu ainsi ?
\VS{11}Et David dit à Abischaï et à tous ses serviteurs : Voici, mon propre fils, qui est sorti de mes entrailles, en veut à ma vie ; à plus forte raison ce Benjamite ! Laissez-le, et qu'il maudisse, car Yahweh lui a parlé.
\VS{12}Peut-être Yahweh regardera mon affliction, et que Yahweh me rendra le bien au lieu des malédictions d'aujourd'hui.
\VS{13}David donc, et ses gens, continuèrent leur chemin. Et Schimeï marchait sur le flanc de la montagne vis-à-vis de lui, continuant à maudire, jetant des pierres contre lui et de la poussière en l'air.
\VS{14}Le roi David et tout le peuple qui était avec lui arrivèrent fatigués et là ils se rafraîchirent.
\TextTitle{Abominations d'Absalom à Jérusalem}
\VS{15}Absalom, et tout le peuple, les hommes d'Israël, étaient entrés dans Jérusalem ; et Achitophel était avec lui.
\VS{16}Quand Huschaï, l'Arkien, ami de David, fut arrivé auprès d'Absalom, il lui dit : Vive le roi ! Vive le roi !
\VS{17}Et Absalom dit à Huschaï : Est-ce donc là l'affection que tu as pour ton ami ? Pourquoi n'es-tu pas allé avec ton ami ?
\VS{18}Huschaï répondit à Absalom : Non, mais je serai à celui qui a été choisi par Yahweh, par ce peuple et par tous les hommes d'Israël, et je demeurerai avec lui.
\VS{19}D'ailleurs, qui servirai-je ? Ne sera-ce pas son fils ? Je serai ton serviteur, comme j'ai été le serviteur de ton père.
\VS{20}Absalom dit à Achitophel : Donnez un conseil sur ce que nous ferons.
\VS{21}Achitophel dit à Absalom : Va vers les concubines que ton père a laissées pour garder la maison ; ainsi tout Israël saura que tu t'es rendu odieux envers ton père, et les mains de tous ceux qui sont avec toi se fortifieront.
\VS{22}On dressa une tente pour Absalom sur le toit et Absalom alla vers les concubines de son père, aux yeux de tout Israël\FTNT{2 S. 12:11-12.}.
\VS{23}Les conseils que donnait Achitophel en ce temps-là étaient autant estimés que si l'on eût demandé la parole de Dieu. C'est ainsi qu'on considérait tous les conseils qu'Achitophel donnait, tant à David qu'à Absalom.
\Chap{17}
\TextTitle{Schimeï maudit le roi David}
\VerseOne{}Après cela, Achitophel dit à Absalom : Je choisirai maintenant douze mille hommes, et je me lèverai, et je poursuivrai David cette nuit.
\VS{2}Je l'atteindrai pendant qu'il est fatigué, et que ses mains sont affaiblies ; je l'épouvanterai tellement que tout le peuple qui est avec lui s'enfuira, et je frapperai seulement le roi ;
\VS{3}et je ramènerai à toi tout le peuple ; car l'homme que tu cherches vaut autant que si tous retournaient à toi ; ainsi tout le peuple sera en paix.
\VS{4}Cette parole plut à Absalom, et à tous les anciens d'Israël.
\VS{5}Cependant Absalom dit : Qu'on appelle maintenant aussi Huschaï, l'Arkien, et que nous entendions aussi son avis.
\VS{6}Huschaï vint vers Absalom et Absalom lui dit : Achitophel a donné un tel avis ; devons-nous faire ce qu'il a dit ou non ? Parle, toi aussi.
\VS{7}Alors Huschaï dit à Absalom : Cette fois, le conseil qu'Achitophel a donné n'est pas bon.
\VS{8}Huschaï dit encore : Tu connais ton père et ses gens, ce sont des hommes forts, et ils ont l'amertume dans l'âme comme une ourse des champs privée de ses petits. Ton père est un homme de guerre, il ne passera pas la nuit avec le peuple.
\VS{9}Voici, il est maintenant caché dans quelque fosse, ou dans quelque autre lieu ; et si, dès le commencement, il en est qui tombent sous leurs coups, on ne tardera pas à l'apprendre et l'on dira : Il y a une défaite parmi le peuple qui suit Absalom !
\VS{10}Alors le plus vaillant, celui-là même qui avait le cœur comme un lion, se découragera ; car tout Israël sait que ton père est un homme de cœur, et que ceux qui sont avec lui sont vaillants.
\VS{11}Je conseille donc que tout Israël se rassemble auprès de toi, depuis Dan jusqu'à Beer-Schéba, multitude pareille au sable qui est sur le bord de la mer, et qu'en personne tu marches au combat.
\VS{12}Alors nous viendrons à lui en quelque lieu que nous le trouvions, et nous nous jetterons sur lui, comme la rosée tombe sur la terre ; et il ne lui restera aucun de tous les hommes qui sont avec lui.
\VS{13}S'il se retire dans une ville, tout Israël portera des cordes vers cette ville-là, et nous la traînerons jusque au torrent, jusqu'à ce qu'on n'en trouve plus une pierre.
\VS{14}Alors Absalom et tous les hommes d'Israël dirent : Le conseil de Huschaï, l'Arkien, est meilleur que le conseil d'Achitophel. Car Yahweh avait résolu de dissiper le conseil d'Achitophel, qui était bon, afin de faire venir le mal sur Absalom.
\TextTitle{Huschaï avertit David du danger}
\VS{15}Alors Huschaï dit aux sacrificateurs Tsadok et Abiathar : Achitophel a donné tel et tel conseil à Absalom, et aux anciens d'Israël ; mais moi, j'ai conseillé telle et telle chose.
\VS{16}Maintenant donc, envoyez tout de suite informer David, en disant : Ne passe point la nuit dans les plaines du désert, mais va plus loin, de peur que le roi et tout le peuple qui est avec lui ne soient exposés au péril.
\VS{17}Jonathan et Achimaats se tenaient à En-Roguel (la fontaine du foulon). Une servante vint leur dire d'aller informer le roi David ; car ils n'osaient pas se montrer et entrer dans la ville.
\VS{18}Mais un garçon les aperçut, et le rapporta à Absalom. Et ils partirent tous deux en hâte et ils arrivèrent à Bachurim, à la maison d'un homme qui avait un puits dans sa cour, dans lequel ils descendirent.
\VS{19}La femme de cet homme prit une couverture, qu'elle étendit sur l'ouverture du puits, et y répandit dessus du grain pilé en sorte qu'on ne s'aperçut de rien.
\VS{20}Les serviteurs d'Absalom entrèrent dans la maison auprès de cette femme, et lui dirent : Où sont Achimaats et Jonathan ? La femme leur répondit : Ils ont passé le ruisseau. Ils cherchèrent, et ne les trouvant pas, ils retournèrent à Jérusalem.
\VS{21}Après leur départ, Achimaats et Jonathan remontèrent du puits et allèrent informer le roi David. Ils lui dirent : Levez-vous, et hâtez-vous de passer l'eau, car Achitophel a conseillé telle chose contre vous.
\VS{22}Alors David et tout le peuple qui était avec lui se levèrent et ils passèrent le Jourdain ; à la lumière du matin, il n'en manqua pas un qui n'eût passé le Jourdain.
\VS{23}Or Achitophel voyant qu'on n'avait point fait ce qu'il avait conseillé, fit seller son âne, se leva, et s'en alla en sa maison, dans sa ville. Après avoir donné des ordres à sa maison, il s'étrangla et mourut. On l'enterra dans le sépulcre de son père.
\TextTitle{Absalom et Israël en marche contre David}
\VS{24}David arriva à Mahanaïm. Et Absalom passa le Jourdain, lui et tous les hommes d'Israël avec lui.
\VS{25}Absalom établit Amasa sur l'armée, à la place de Joab. Or Amasa était fils d'un homme nommé Jithra, l'Israélite, qui était allé vers Abigaïl, fille de Nachasch, et soeur de Tseruja, mère de Joab.
\VS{26}Israël et Absalom campèrent dans le pays de Galaad.
\TextTitle{Mahanaïm bienveillant envers David}
\VS{27}Or il arriva qu'aussitôt que David fut arrivé à Mahanaïm, Schobi, fils de Nachasch de Rabba, des fils d'Ammon, Makir, fils d'Ammiel de Lodebar, et Barzillaï, le Galaadite de Roguelim,
\VS{28}apportèrent des lits, des bassins, des vases de terre, du froment, de l'orge, de la farine, du grain rôti, des fèves, des lentilles, des pois rôtis,
\VS{29}du miel, de la crème, des brebis, et des fromages de vache. Ils apportèrent ces choses à David et au peuple qui était avec lui, afin qu'ils mangent, car ils disaient : Ce peuple a dû souffrir de la faim, de la fatigue et de la soif dans le désert.
\Chap{18}
\TextTitle{Bataille dans la forêt d'Ephraïm ; instructions de David sur Absalom}
\VerseOne{}David fit le dénombrement du peuple qui était avec lui, et il établit sur eux des chefs de milliers et des chefs de centaines.
\VS{2}David envoya le peuple, un tiers sous le commandement de Joab, un tiers sous le commandement d'Abischaï, fils de Tseruja, frère de Joab, et un tiers sous le commandement d'Ittaï, de Gath. Et le roi dit au peuple : Moi aussi, je veux sortir avec vous.
\VS{3}Mais le peuple lui dit : Tu ne sortiras point ! Car si nous prenons la fuite, ce n'est pas sur nous que l'attention se portera ; et même quand la moitié d'entre nous y serait tuée, on n'y ferait pas attention ; mais toi, tu es comme dix mille de nous, et maintenant il vaut mieux que de la ville tu puisses venir à notre secours.
\VS{4}Le roi leur répondit : Je ferai ce qui est bon à vos yeux. Le roi s'arrêta donc à la place de la porte, pendant que tout le peuple sortait par centaines et par milliers.
\VS{5}Le roi donna cet ordre à Joab, à Abischaï, et à Ittaï, et dit : Epargnez-moi le jeune homme Absalom ! Et tout le peuple entendit ce que le roi commandait à tous les chefs au sujet d'Absalom.
\VS{6}Ainsi le peuple sortit dans les champs à la rencontre d'Israël, et la bataille eut lieu dans la forêt d'Ephraïm.
\VS{7}Là, le peuple d'Israël fut battu par les serviteurs de David, et il y eut en ce jour-là dans ce même lieu, une grande défaite de vingt mille hommes.
\VS{8}La bataille s'étendit sur toute la contrée, et la forêt dévora ce jour-là beaucoup plus de peuple que l'épée.
\TextTitle{Joab tue Absalom}
\VS{9}Absalom se retrouva devant les serviteurs de David. Il était monté sur un mulet. Le mulet entra sous les branches entrelacées d'un grand chêne, et la tête d'Absalom fut prise dans le chêne ; il demeura suspendu entre le ciel et la terre, et le mulet qui était sous lui passa outre.
\VS{10}Un homme ayant vu cela, le rapporta à Joab, et lui dit : Voici, j'ai vu Absalom suspendu à un chêne.
\VS{11}Et Joab répondit à l'homme qui lui rapportait cela : Tu l'as vu ! Pourquoi ne l'as-tu pas tué là, le jetant par terre ? Je t'aurais donné dix sicles d'argent et une ceinture.
\VS{12}Mais cet homme dit à Joab : Quand je pèserais dans ma main mille pièces d'argent, je ne mettrais pas ma main sur le fils du roi ; car nous avons entendu ce que le roi vous a ordonné, à toi, à Abischaï et à Ittaï, en disant : Prenez garde chacun au jeune homme Absalom !
\VS{13}Autrement j'aurais commis une lâcheté au péril de ma vie, car rien ne serait caché au roi, et toi-même tu te lèverais contre moi.
\VS{14}Joab répondit : Je ne m'attarderai pas auprès de toi ! Et il prit en sa main trois javelots, et les enfonça dans le cœur d'Absalom qui était encore vivant au milieu du chêne.
\VS{15}Puis dix jeunes hommes, qui portaient les armes de Joab, entourèrent Absalom, le frappèrent et le firent mourir\FTNT{La mort d'Absalom fut une conséquence du péché de David avec Bath-Schéba. Le péché a donc des conséquences graves et cause beaucoup de souffrances.}.
\VS{16}Alors Joab fit sonner la trompette ; et le peuple cessa de poursuivre Israël, parce que Joab le retint.
\VS{17}Ils prirent Absalom, le jetèrent dans la forêt dans une grande fosse, et mirent sur lui un très grand monceau de pierres. Tout Israël s'enfuit, chacun dans sa tente.
\VS{18}Or Absalom s'était fait ériger, de son vivant, un monument dans la vallée du roi ; car il disait : Je n'ai point de fils pour conserver la mémoire de mon nom. Et il donna son propre nom au monument, qu'on appelle encore aujourd'hui la place d'Absalom.
\TextTitle{David apprend la mort d'Absalom}
\VS{19}Et Achimaats, fils de Tsadok, dit : Laisse-moi courir, et porter au roi la bonne nouvelle que Yahweh lui a rendu justice en jugeant ses ennemis.
\VS{20}Joab lui répondit : Tu ne seras pas aujourd'hui porteur de bonnes nouvelles ; tu le seras un autre jour ; car aujourd'hui tu ne porterais pas de bonnes nouvelles, puisque le fils du roi est mort.
\VS{21}Et Joab dit à Cuschi : Va, et annonce au roi ce que tu as vu. Cuschi se prosterna devant Joab, puis il se mit à courir.
\VS{22}Achimaats, fils de Tsadok, dit encore à Joab : Quoi qu'il arrive, laisse-moi courir après Cuschi. Joab lui dit : Pourquoi veux-tu courir, mon fils, puisque tu n'as pas de bonnes nouvelles à apporter ?
\VS{23}Quoiqu'il arrive, je veux courir, reprit Achimaats. Et Joab lui dit : Cours ! Achimaats courut par le chemin de la plaine, et il devança Cuschi.
\VS{24}David était assis entre les deux portes. La sentinelle alla sur le toit de la porte vers la muraille ; elle leva les yeux et elle regarda. Et voici un homme qui courait tout seul.
\VS{25}Alors la sentinelle cria, et avertit le roi. Le roi dit : S'il est seul, il apporte des bonnes nouvelles. Et cet homme marchait incessamment et approchait.
\VS{26}Puis la sentinelle vit un autre homme qui courait ; et elle cria au portier : Voici un homme qui court tout seul. Le roi dit : Il apporte aussi des bonnes nouvelles.
\VS{27}La sentinelle dit : La manière de courir du premier me paraît celle d'Achimaats, fils de Tsadok. Et le roi dit : C'est un homme de bien, il vient quand il y a des bonnes nouvelles.
\VS{28}Achimaats cria, et il dit au roi : Tout va bien ! Et il se prosterna devant le roi, le visage contre terre, et dit : Béni soit Yahweh, ton Dieu, qui a livré les hommes qui levaient leurs mains contre le roi, mon seigneur !
\VS{29}Le roi dit : Le jeune homme Absalom se porte-t-il bien ? Achimaats lui répondit : J'ai vu s'élever un grand tumulte au moment où Joab envoya le serviteur, du roi et moi ton serviteur ; mais je ne sais pas exactement ce que c'était.
\VS{30}Et le roi lui dit : Mets-toi là de côté. Et Achimaats se tint de côté.
\VS{31}Aussitôt arriva Cuschi. Et il dit : Que le roi, mon seigneur apprenne, ces bonnes nouvelles ! Aujourd'hui, Yahweh t'a rendu justice en jugeant tous ceux qui s'élevaient contre toi.
\VS{32}Le roi dit à Cuschi : Le jeune homme Absalom se porte-t-il bien ? Et Cuschi lui répondit : Que les ennemis du roi, mon seigneur, et tous ceux qui s'élèvent contre toi pour te faire du mal soient comme ce jeune homme !
\VS{33}Alors le roi, saisi d'émotion, monta à la chambre haute de la porte, et alla pleurer. Il disait ainsi en marchant : Mon fils Absalom ! Mon fils, mon fils Absalom ! Plaise à Dieu que je sois moi-même mort à ta place ! Absalom, mon fils, mon fils !
\Chap{19}
\TextTitle{Souffrance de David ; indignation de Joab}
\VerseOne{}Et on fit ce rapport à Joab : Voici, le roi pleure et se lamente à cause d'Absalom.
\VS{2}Ainsi, la victoire fut en ce jour-là changée en deuil pour tout le peuple, car en ce jour-là le peuple entendait dire : Le roi est affligé à cause de son fils.
\VS{3}Ce même jour, le peuple rentra dans la ville à la dérobée, comme l'auraient fait des gens honteux d'avoir pris la fuite dans la bataille.
\VS{4}Le roi s'était couvert le visage, et il criait à haute voix : Mon fils Absalom ! Absalom, mon fils, mon fils !
\VS{5}Joab entra dans la chambre où était le roi, et lui dit : Tu couvres aujourd'hui de confusion les faces de tous tes serviteurs, qui ont en ce jour sauvé ta vie, celle de tes fils et de tes filles, celle de tes femmes et de tes concubines.
\VS{6}Tu aimes ceux qui te haïssent, et tu hais ceux qui t'aiment, car tu montres aujourd'hui que tes chefs et tes serviteurs ne te sont rien ; et je sais maintenant que si Absalom vivait, et que nous tous fussions morts aujourd'hui, cela serait agréable à tes yeux.
\VS{7}Maintenant donc lève-toi, sors, et parle selon le coeur de tes serviteurs ! Car je jure par Yahweh que si tu ne sors pas, il ne restera pas un seul homme avec toi cette nuit ; et ce mal sera pire que tous ceux qui te sont arrivés depuis ta jeunesse jusqu'à présent.
\TextTitle{Retour du roi David à Jérusalem}
\VS{8}Alors le roi se leva et s'assit à la porte. On fit dire à tout le peuple : Voici, le roi est assis à la porte. Et tout le peuple vint devant le roi. Cependant, Israël s'était enfui, chacun dans sa tente.
\VS{9}Et dans toutes les tribus d'Israël, tout le peuple était en contestation, disant : Le roi nous a délivrés de la main de nos ennemis, c'est lui qui nous a sauvés de la main des Philistins, et maintenant il a dû fuir du pays devant Absalom.
\VS{10}Or Absalom, que nous avions oint pour roi sur nous, est mort dans la bataille. Maintenant donc, pourquoi ne parlez-vous pas de faire revenir le roi ?
\VS{11}Le roi David envoya dire aux sacrificateurs Tsadok et Abiathar : Parlez aux anciens de Juda, et dites-leur : Pourquoi seriez-vous les derniers à ramener le roi en sa maison ? Car les discours que tout Israël avait tenus étaient parvenus jusqu'au roi dans sa maison.
\VS{12}Vous êtes mes frères, vous êtes mes os et ma chair ; pourquoi seriez-vous les derniers à ramener le roi ?
\VS{13}Dites même à Amasa : N'es-tu pas mon os et ma chair ? Que Dieu me traite dans toute sa rigueur si tu ne deviens pas devant moi pour toujours chef de l'armée à la place de Joab !
\VS{14}Ainsi David fléchit le cœur de tous les hommes de Juda, comme s'ils n'eussent été qu'un seul homme ; et ils envoyèrent dire au roi : Reviens, toi, et tous tes serviteurs.
\VS{15}Le roi revint et arriva jusqu'au Jourdain ; et Juda se rendit jusqu'à Guilgal, pour aller à la rencontre du roi afin de lui faire repasser le Jourdain.
\VS{16}Et Schimeï, fils de Guéra, Benjamite, qui était de Bachurim, se hâta de descendre avec les hommes de Juda à la rencontre du roi David.
\VS{17}Il avait avec lui mille hommes de Benjamin, et Tsiba, serviteur de la maison de Saül, ses quinze enfants, et ses vingt serviteurs étaient aussi avec lui. Ils passèrent le Jourdain en présence du roi.
\VS{18}Le bateau, mis à la disposition du roi, faisait la traversée pour transporter sa maison ; et au moment où le roi allait passer le Jourdain, Schimeï, fils de Guéra, se prosterna devant lui.
\VS{19}Et il dit au roi : Que mon seigneur ne m'impute pas mon iniquité, et ne se souvienne pas de ce que ton serviteur a fait de mal le jour où le roi mon seigneur sortait de Jérusalem, et que le roi ne le prenne point à cœur !
\VS{20}Car ton serviteur sait qu'il a péché. Et voici, je viens aujourd'hui le premier de toute la maison de Joseph à la rencontre du roi, mon seigneur.
\VS{21}Mais Abischaï, fils de Tseruja, répondit et dit : A cause de cela, ne fera-ton par mourir Schimeï, puisqu'il a maudit l'oint de Yahweh ?
\VS{22}Et David dit : Qu'ai-je à faire avec vous, fils de Tseruja ? Et pourquoi vous montrez-vous aujourd'hui mes adversaires ? Ferait-on mourir aujourd'hui quelqu'un en Israël ? Ne sais-je donc pas que je règne aujourd'hui sur Israël ?
\VS{23}Et le roi dit à Schimeï : Tu ne mourras point ! Et le roi le lui jura.
\VS{24}Après cela, Mephiboscheth, fils de Saül, descendit aussi à la rencontre du roi. Il n'avait point lavé ses pieds, ni fait sa barbe, ni lavé ses vêtements, depuis que le roi s'en était allé, jusqu'au jour où il revenait en paix.
\VS{25}Il se trouva donc au-devant du roi comme il entrait dans Jérusalem, et le roi lui dit : Pourquoi n'es-tu pas venu avec moi, Mephiboscheth ?
\VS{26}Et il lui répondit : Ô roi, mon seigneur, mon serviteur m'a trompé, car ton serviteur qui est boiteux avait dit : Je ferai seller mon âne, je monterai dessus, et j'irai avec le roi.
\VS{27}Et il a calomnié ton serviteur auprès du roi, mon seigneur. Mais le roi mon seigneur est comme un ange de Dieu. Fais donc ce qui semblera bon à tes yeux.
\VS{28}Car bien que tous ceux de la maison de mon père n'ont été que des gens dignes de mort devant le roi mon seigneur ; cependant tu as mis ton serviteur parmi ceux qui mangent à ta table. Quel droit puis-je encore avoir, pour me plaindre encore au roi ?
\VS{29}Et le roi lui dit : Pourquoi toutes ces paroles ? Je l'ai dit : Toi et Tsiba, vous partagerez les terres.
\VS{30}Et Mephiboscheth dit au roi : Qu'il prenne même tout, puisque le roi mon seigneur rentre en paix dans sa maison.
\VS{31}Barzillaï, le Galaadite, descendit de Roguelim, et passa le Jourdain avec le roi, pour l'accompagner jusqu'au-delà du Jourdain.
\VS{32}Barzillaï était très vieux, âgé de quatre-vingts ans. Il avait nourri le roi pendant qu'il avait séjourné à Mahanaïm, car c'était un homme fort riche.
\VS{33}Le roi dit à Barzillaï : Viens avec moi, je te nourrirai chez moi à Jérusalem.
\VS{34}Mais Barzillaï répondit au roi : Combien d'années vivrai-je encore pour que je monte avec le roi à Jérusalem ?
\VS{35}Je suis aujourd'hui âgé de quatre-vingts ans. Puis-je encore discerner ce qui est bon de ce qui est mauvais ? Ton serviteur peut-il savourer ce qu'il mange et ce qu'il boit ? Puis-je encore entendre la voix des chanteurs et des chanteuses ? Et pourquoi ton serviteur serait-il encore à charge à mon seigneur, le roi ?
\VS{36}Ton serviteur ira un peu au-delà du Jourdain avec le roi. Pourquoi le roi voudrait-il me donner une telle récompense ?
\VS{37}Je te prie que ton serviteur s'en retourne, et que je meure dans ma ville, près du sépulcre de mon père et de ma mère ! Mais voici ton serviteur Kimham, passera avec le roi mon seigneur ; fais-lui ce qui semblera bon à tes yeux.
\VS{38}Le roi dit : Que Kimham passe avec moi, et je lui ferai ce qui sera bon à tes yeux ; et tout ce que tu voudras de moi, je te l'accorderai.
\VS{39}Tout le peuple passa donc le Jourdain avec le roi. Puis le roi embrassa Barzillaï et le bénit. Et Barzillaï retourna dans sa demeure.
\TextTitle{Juda et Israël se disputent le roi}
\VS{40}De là, le roi passa à Guilgal, et Kimham passa avec lui. Ainsi, tout le peuple de Juda, et même la moitié du peuple d'Israël ramenèrent le roi.
\VS{41}Mais voici, tous les hommes d'Israël vinrent vers le roi, et lui dirent : Pourquoi nos frères, les hommes de Juda, t'ont-ils enlevé, et ont-ils fait passer le Jourdain au roi et à sa maison, et à tous les gens de David ?
\VS{42}Alors tous les hommes de Juda répondirent aux hommes d'Israël : Parce que le roi nous est plus proche ; pourquoi vous fâchez-vous de cela ? Avons-nous vécu aux dépens du roi ? Nous a-t-il fait des présents ?
\VS{43}Les hommes d'Israël répondirent aux hommes de Juda, et dirent : Le roi nous appartient dix fois autant, et David même plus qu'à vous. Pourquoi nous avez-vous méprisés ? N'avons-nous pas parlé les premiers de ramener notre roi ? Mais les hommes de Juda parlèrent avec plus de violence que les hommes d'Israël.
\Chap{20}
\TextTitle{Juda reste fidèle au roi David}
\VerseOne{}Et il se trouvait là un méchant\FTNT{Littéralement « beliya`al » : « méchant, pervers », « ruine, destruction ». Voir commentaire en 1 S. 2:12.} homme, nommé Schéba, fils de Bicri, Benjamite. Il sonna du shofar et dit : Nous n'avons point de part avec David ni d'héritage avec le fils d'Isaï ! Israël, chacun à ses tentes !
\VS{2}Ainsi tous les hommes d'Israël se séparèrent de David, et suivirent Schéba, fils de Bicri. Mais les hommes de Juda s'attachèrent à leur roi, et l'accompagnèrent depuis le Jourdain jusqu'à Jérusalem.
\VS{3}David rentra dans sa maison à Jérusalem. Il prit ses dix femmes concubines qu'il avait laissées pour garder sa maison, et les mit en un lieu où elles étaient gardées ; il pourvut à leur entretien, mais il n'alla point vers elles. Ainsi, elles furent enfermées jusqu'au jour de leur mort, vivant dans le veuvage.
\TextTitle{Bataille contre Schéba ; Joab tue Amasa}
\VS{4}Puis le roi dit à Amasa : Rassemble-moi dans trois jours les hommes de Juda ; et toi, sois ici présent.
\VS{5}Amasa donc s'en alla pour rassembler Juda ; mais il tarda au-delà du temps que le roi lui avait fixé.
\VS{6}Alors David dit à Abischaï : Maintenant Schéba, fils de Bicri, nous fera plus de mal qu'Absalom. Prends toi-même les serviteurs de ton maître et poursuis-le, de peur qu'il ne trouve des villes fortes, et que nous ne le perdions de vue.
\VS{7}Et Abischaï partit, suivi des gens de Joab, des Kéréthiens et des Péléthiens, et de tous les hommes forts ; ils sortirent de Jérusalem, pour poursuivre Schéba, fils de Bicri.
\VS{8}Et comme ils furent près de la grande pierre qui est à Gabaon, Amasa vint au-devant d'eux. Joab était ceint d'une épée par-dessus les habits dont il était revêtu ; elle était attachée à ses reins dans le fourreau, et comme il s'avançait, elle tomba.
\VS{9}Joab dit à Amasa : Te portes-tu bien, mon frère ? Puis Joab prit de sa main droite la barbe d'Amasa pour l'embrasser.
\VS{10}Amasa ne prit point garde à l'épée qui était dans la main de Joab ; et Joab l'en frappa au ventre et répandit ses entrailles à terre, sans le frapper une seconde fois. Et il mourut. Après cela, Joab et Abischaï, son frère, poursuivirent Schéba, fils de Bicri.
\VS{11}Un des serviteurs de Joab resta près d'Amasa, et il disait : Qui aime Joab et qui est pour David ? Qu'il suive Joab !
\VS{12}Amasa était vautré dans son sang au milieu de la route ; et cet homme-là, ayant vu que tout le peuple s'arrêtait, poussa Amasa hors de la route dans un champ, et jeta un vêtement sur lui, lorsqu'il vit que tous ceux qui arrivaient près de lui s'arrêtaient.
\VS{13}Quand il fut ôté de la route, tous les hommes qui suivaient Joab passaient au-delà, afin de poursuivre Schéba, fils de Bicri.
\TextTitle{La révolte de Schéba}
\VS{14}Joab passa par toutes les tribus d'Israël jusqu'à Abel-Beth-Maaca, avec tous les Bériens, qui s'étaient assemblés et qui l'avaient suivi.
\VS{15}Les gens donc de Joab vinrent assiéger Schéba dans Abel-Beth-Maaca, et ils élevèrent contre la ville une terrasse qui atteignait le rempart. Tout le peuple qui était avec Joab rompait la muraille pour la faire tomber.
\VS{16}Lorsqu'une femme sage de la ville se mit à crier : Ecoutez, écoutez ! Dites, je vous prie, à Joab : Approche jusqu'ici, je veux te parler !
\VS{17}Il s'approcha d'elle, et la femme dit : Es-tu Joab ? Il répondit : Je le suis. Elle lui dit : Ecoute les paroles de ta servante. Il répondit : J'écoute.
\VS{18}Et elle dit : Autrefois on avait coutume de dire : Que l'on consulte Abel ! Et tout se terminait ainsi.
\VS{19}Je suis une des cités paisibles et fidèles en Israël ; tu cherches à détruire une ville qui est une mère en Israël ! Pourquoi détruirais-tu l'héritage de Yahweh ?
\VS{20}Joab lui répondit : A Dieu ne plaise, à Dieu ne plaise que je détruise et que je ruine !
\VS{21}La chose n'est pas ainsi. Mais un homme de la montagne d'Ephraïm, nommé Schéba, fils de Bicri, a levé sa main contre le roi David ; livrez-le, lui seul, et je m'éloignerai de la ville. La femme dit à Joab : Voici, sa tête te sera jetée par-dessus la muraille.
\VS{22}Et la femme alla vers tout le peuple, et leur parla sagement ; et ils coupèrent la tête de Schéba, fils de Bicri, et la jetèrent à Joab. Alors il sonna du shofar ; et on se dispersa loin de la ville, et chacun s'en alla dans sa tente. Puis Joab retourna vers le roi à Jérusalem.
\VS{23}Joab était le chef de toute l'armée d'Israël ; Benaja, fils de Jehojada, était à la tête des Kéréthiens et des Péléthiens ;
\VS{24}et Adoram était préposé aux impôts ; Josaphat, fils d'Achilud, était archiviste.
\VS{25}Scheja était le secrétaire ; Tsadok et Abiathar étaient les sacrificateurs ;
\VS{26}et Ira de Jaïr était ministre d'Etat de David.
\Chap{21}
\TextTitle{Vengeance des Gabaonites sur la maison de Saül}
\VerseOne{}Or il y eut du temps de David, une famine qui dura trois ans de suite. David chercha la face de Yahweh, et Yahweh lui répondit : C'est à cause de Saül et de sa maison sanguinaire, parce qu'il a fait mourir les Gabaonites.
\VS{2}Alors le roi appela les Gabaonites pour leur parler. Or les Gabaonites n'étaient point des enfants d'Israël, mais un reste des Amoréens ; les enfants d'Israël leur avaient juré de les laisser vivre\FTNT{Jos. 9.}, mais Saül dans son zèle pour les enfants d'Israël et de Juda, avait cherché à les faire mourir.
\VS{3}Et David dit aux Gabaonites : Que ferais-je pour vous, et par quel moyen vous appaserai-je, afin que vous bénissiez l'héritage de Yahweh ?
\VS{4}Les Gabaonites lui répondirent : Il ne s'agit pas pour nous d'argent ou d'or avec Saül et avec sa maison, et ce n'est pas à nous de faire mourir un homme en Israël. Le roi leur dit : Que voulez-vous donc que je fasse pour vous ?
\VS{5}Ils répondirent au roi : Puisque cet homme nous a consumés, et qu'il avait résolu de nous exterminer pour nous faire disparaître de tout le territoire d'Israël,
\VS{6}qu'on nous livre sept hommes d'entre ses fils, et nous les pendrons devant Yahweh à Guibea de Saül, l'élu de Yahweh. Et le roi dit : Je vous les livrerai.
\VS{7}Le roi épargna Mephiboscheth, fils de Jonathan, fils de Saül, à cause du serment que David et Jonathan, fils de Saül, avaient fait entre eux, devant Yahweh.
\VS{8}Mais le roi prit les deux fils que Ritspa, fille d'Ajja, avait enfantés à Saül, Armoni et Mephiboscheth, et les cinq fils que Mérab, fille de Saül, avait enfantés à Adriel de Mehola, fils de Barzillaï,
\VS{9}et il les livra entre les mains des Gabaonites, qui les pendirent sur la montagne, devant Yahweh. Tous les sept furent tués ensemble ; on les fit mourir dans les premiers jours de la moisson, au commencement de la moisson des orges.
\VS{10}Alors Ritspa, fille d'Ajja, prit un sac et l'étendit sous elle au-dessus d'un rocher, depuis le commencement de la moisson jusqu'à ce que l'eau du ciel tombât sur eux ; et elle ne permit pas aux oiseaux du ciel de s'approcher d'eux pendant le jour, ni aux bêtes des champs pendant la nuit.
\VS{11}On informa David de ce qu'avait fait Ritspa, fille d'Ajja, concubine de Saül.
\VS{12}Et David alla prendre les os de Saül et les os de Jonathan, son fils, chez les habitants de Jabès en Galaad, qui les avaient enlevés de la place de Beth-Schan, où les Philistins les avaient pendus lorsqu'ils tuèrent Saül à Guilboa.
\VS{13}Il emporta de là les os de Saül et les os de Jonathan, son fils ; on recueillit aussi les os de ceux qui avaient été pendus.
\VS{14}On les enterra avec les os de Saül et de Jonathan, son fils, au pays de Benjamin, à Tséla, dans le sépulcre de Kis, père de Saül. Et l'on fit tout ce que le roi avait ordonné. Après cela, Dieu fut apaisé envers le pays.
\TextTitle{Nouvelles batailles contre les Philistins}
\VS{15}Il y eut encore une guerre entre les Philistins et Israël. David descendit avec ses serviteurs avec lui, et ils combattirent tellement contre les Philistins que David était fatigué.
\VS{16}Et Jischbi-Benob, qui était un des enfants de Rapha, eut l'intention de tuer David ; il avait une lance dont le fer pesait trois cents sicles d'airain, et il était ceint d'une armure neuve.
\VS{17}Mais Abischaï, fils de Tseruja, vint au secours de David, frappa le Philistin, et le tua. Alors les gens de David jurèrent, en disant : Tu ne sortiras plus avec nous à la bataille, de peur que tu n'éteignes la lampe d'Israël.
\VS{18}Après cela, il y eut encore une autre guerre à Gob avec les Philistins. Sibbecaï, le Huschatite, tua Saph, qui était un des enfants de Rapha.
\VS{19}Il y eut encore une autre guerre à Gob avec les Philistins. Et Elchanan, fils de Jaaré-Oreguim, de Bethléhem, tua Goliath de Gath, qui avait une lance dont le bois était comme une ensouple de tisserand.
\VS{20}Il y eut encore une guerre à Gath. Il s'y trouva un homme de haute taille, qui avait six doigts à chaque main, et six orteils à chaque pied, en tout vingt-quatre, lequel était aussi issu de Rapha.
\VS{21}Il jeta un défi à Israël ; et Jonathan, fils de Schimea, frère de David, le tua.
\VS{22}Ces quatre-là étaient nés à Gath, de la race de Rapha. Ils moururent par les mains de David, ou par les mains de ses serviteurs.
\Chap{22}
\TextTitle{Louange à Yahweh, le Dieu qui délivre}
\VerseOne{}Après cela, David adressa à Yahweh les paroles de ce cantique, le jour où Yahweh l'eut délivré de la main de tous ses ennemis, et de la main de Saül.
\VS{2}Il dit : Yahweh est mon rocher, ma forteresse, mon libérateur.
\VS{3}Dieu est mon rocher, où je trouve un abri, mon bouclier et la force qui me sauve, ma haute retraite et mon refuge. Ô mon Sauveur ! Tu me délivres de la violence.
\VS{4}Je m'écrie : Loué soit Yahweh! Et je suis délivré de mes ennemis\FTNT{Ps. 18:4.}.
\VS{5}Car les flots de la mort m'avaient environné, les torrents des méchants m'avaient épouvanté ;
\VS{6}les liens du scheol m'avaient entouré, les filets de la mort m'avaient surpris.
\VS{7}Dans ma détresse, j'ai invoqué Yahweh, j'ai crié à mon Dieu ; de son palais, il a entendu ma voix, et mon cri est parvenu à ses oreilles.
\VS{8}Alors la terre fut ébranlée et trembla, les fondements des cieux s'agitèrent, et ils furent ébranlés, parce qu'il était irrité.
\VS{9}Une fumée montait de ses narines, et de sa bouche sortait un feu dévorant : Il en jaillissait des charbons embrasés.
\VS{10}Il abaissa les cieux, et descendit : Il y avait une épaisse nuée sous ses pieds.
\VS{11}Il était monté sur un chérubin, et il volait, il paraissait sur les ailes du vent.
\VS{12}Il mit autour de lui les ténèbres pour tabernacle, des amas d'eaux, des nuées épaisses.
\VS{13}Des charbons de feu étaient embrasés de la splendeur qui le précédait.
\VS{14}Yahweh tonna des cieux, et le Très-Haut fit retentir sa voix ;
\VS{15}il lança ses flèches, et dispersa mes ennemis ; la foudre, et les mit en déroute.
\VS{16}Alors le fond de la mer apparut, et les fondements de la terre habitable furent mis à découvert, par la menace de Yahweh, par le souffle du vent de sa colère.
\VS{17}Il étendit sa main d'en haut, il me saisit, il me retira des grandes eaux ;
\VS{18}il me délivra de mon ennemi puissant, de ceux qui me haïssaient, car ils étaient plus forts que moi.
\VS{19}Ils m'avaient surpris au jour de ma détresse, mais Yahweh fut mon appui.
\VS{20}Il m'a mis au large, il m'a sauvé, parce qu'il a pris son plaisir en moi.
\VS{21}Yahweh m'a traité selon ma droiture, il m'a rendu selon la pureté de mes mains ;
\VS{22}parce que j'ai gardé les voies de Yahweh, et que je ne me suis point détourné de mon Dieu.
\VS{23}Toutes ses ordonnances ont été devant moi, et je ne me suis point écarté de ses lois.
\VS{24}J'ai été intègre envers lui, et je me suis gardé de mon iniquité.
\VS{25}Yahweh donc m'a rendu selon ma droiture, selon ma pureté devant ses yeux.
\VS{26}Avec celui qui est bon tu es bon, avec l'homme intègre tu es intègre,
\VS{27}avec celui qui est pur tu te montres pur, mais avec le pervers tu agis selon sa perversité.
\VS{28}Tu sauves le peuple qui s'humilie, et de ton regard, tu abaisses les orgueilleux.
\VS{29}Tu es ma lampe, ô Yahweh ! Et Yahweh éclaire mes ténèbres.
\VS{30}Avec toi je me précipite sur une troupe en armes, avec mon Dieu je franchis une muraille.
\VS{31}La voie de Dieu est parfaite, la parole de Yahweh est éprouvée ; il est le bouclier de tous ceux qui se confient en lui.
\VS{32}Car qui est Dieu, si ce n'est Yahweh ? Et qui est un rocher, si ce n'est notre Dieu ?
\VS{33}C'est Dieu qui est ma puissante forteresse, et qui me conduit dans la voie droite.
\VS{34}Il a rendu mes pieds semblables à ceux des biches, et il me fait tenir debout sur mes lieux élevés.
\VS{35}Il exerce mes mains au combat, et mes bras tendent l'arc d'airain.
\VS{36}Tu me donnes le bouclier de ton salut, et ta bonté me fait devenir plus grand.
\VS{37}Tu élargis le chemin sous mes pas, et mes pieds ne chancellent point.
\VS{38}Je poursuis mes ennemis, et je les détruis ; je ne reviens qu'après les avoir exterminés.
\VS{39}Je les anéantis, je les transperce, et ils ne se relèvent plus ; ils tombent sous mes pieds.
\VS{40}Tu me ceins de force pour le combat, tu fais plier sous moi mes adversaires.
\VS{41}Tu fais tourner le dos à mes ennemis devant moi, et j'extermine ceux qui me haïssent.
\VS{42}Ils regardent autour d'eux, et il n'y a point de sauveur ! Ils crient à Yahweh, mais il ne leur répond pas !
\VS{43}Je les broie comme la poussière de la terre, je les écrase, je les foule, comme la boue des rues.
\VS{44}Tu me délivres des dissensions de mon peuple ; tu me gardes pour être chef des nations ; un peuple que je ne connaissais pas m'est asservi.
\VS{45}Les fils de l'étranger me flattent, dès qu'ils ont entendu parler de moi, ils se sont rendus obéissants.
\VS{46}Les fils de l'étranger défaillent, et sortent tremblants de leurs forteresses.
\VS{47}Yahweh est vivant, et béni soit mon rocher ! Que Dieu, le rocher de mon salut, soit exalté,
\VS{48}le Dieu qui me donne vengeance, qui m'assujettit les peuples,
\VS{49}et qui me fait échapper à mes ennemis ! Tu m'élèves au-dessus de mes adversaires, tu me délivres de l'homme violent.
\VS{50}C'est pourquoi, ô Yahweh, je te louerai parmi les nations, et je chanterai des psaumes à ton Nom.
\VS{51}C'est lui qui est la tour de délivrance de son roi, et qui fait miséricorde à son oint, à David, et à sa postérité, à jamais.
\Chap{23}
\TextTitle{Paroles prophétiques de David}
\VerseOne{}Voici les dernières paroles de David. Parole de David, fils d'Isaï, parole de l'homme qui a été élevé, de l'oint du Dieu de Jacob, du chantre agréable d'Israël :
\VS{2}L'Esprit de Yahweh parle par moi, et sa parole est sur ma langue.
\VS{3}Le Dieu d'Israël a parlé, le Rocher\FTNT{Voir commentaire Es. 8 :13-17.} d'Israël m'a dit : Celui qui règne parmi les hommes avec justice, celui qui règne dans la crainte de Dieu,
\VS{4}est comme la lumière du matin quand le soleil se lève, un matin sans nuage ; son éclat fait germer de la terre la verdure après la pluie.
\VS{5}N'en est-il pas ainsi de ma maison devant Dieu, puisqu'il a traité avec moi une alliance éternelle, bien ordonnée, et gardée ? Tout mon salut et tout mon plaisir, ne les fera-t-il pas germer ?
\VS{6}Mais les méchants sont tous comme des épines que l'on jette au loin, parce qu'on ne les prend pas avec la main ;
\VS{7}celui qui les touche, s'arme du fer ou du bois d'une lance, et on les brûle au feu sur place.
\TextTitle{Les vaillants hommes de David\FTNTT{1 Ch. 11:10-47}}
\VS{8}Voici les noms des vaillants hommes qui étaient au service de David. Joscheb-Basschébeth, le Tachkemonite, était l'un des principaux chefs. C'était Hadino le Hetsnite, qui eut le dessus sur huit cents hommes qu'il tua en une seule fois.
\VS{9}Après lui, Eléazar, fils de Dodo, fils d'Achochi. Il était l'un des trois vaillants hommes qui étaient avec David lorsqu'ils défièrent les Philistins rassemblés pour combattre, tandis que les hommes d'Israël se retiraient.
\VS{10}Il se leva, et frappa les Philistins jusqu'à ce que sa main fut lasse et qu'elle restât attachée à l'épée. Ce jour-là, Yahweh opéra une grande délivrance. Le peuple revint après Eléazar, seulement pour prendre les dépouilles.
\VS{11}Après lui, Schamma, fils d'Agué d'Harar. Les Philistins s'étaient rassemblés en troupe. Il y avait là une parcelle de champ pleine de lentilles ; et le peuple fuyait devant les Philistins.
\VS{12}Schamma se mit au milieu de cette parcelle, la défendit, et frappa les Philistins. Et Yahweh opéra une grande délivrance.
\VS{13}Trois des trente chefs descendirent au temps de la moisson et vinrent vers David, dans la caverne d'Adullam, lorsqu'une troupe de Philistins était campée dans la vallée des Rephaïm.
\VS{14}David était alors dans la forteresse, et la garnison des Philistins était en ce temps-là à Bethléhem.
\VS{15}Et David eut un désir, et dit : Qui est-ce qui me fera boire de l'eau de la citerne qui est à la porte de Bethléhem ?
\VS{16}Alors ces trois vaillants hommes passèrent au travers du camp des Philistins, et puisèrent de l'eau de la citerne qui est à la porte de Bethléhem. Ils l'apportèrent, et ils la présentèrent à David ; mais il ne voulut pas la boire, et il la répandit devant Yahweh.
\VS{17}Car il dit : Loin de moi, ô Yahweh, de faire une telle chose ! N'est-ce pas le sang de ces hommes qui sont allés au péril de leur vie ? Il ne voulut pas la boire. Voilà ce que firent ces trois vaillants hommes.
\VS{18}Il y avait aussi Abischaï, frère de Joab, fils de Tseruja, qui était le chef des trois. Il brandit sa lance sur trois cents hommes, les blessa à mort ; et il eut du renom parmi les trois.
\VS{19}Il était le plus considéré des trois, et il fut leur chef ; cependant il n'égala point les trois premiers.
\VS{20}Benaja, fils de Jehojada, fils d'un vaillant homme de Kabtseel, rempli de force, avait fait de grands exploits. Il frappa deux des plus puissants hommes de Moab. Il descendit au milieu d'une fosse, où il frappa un lion, un jour de neige.
\VS{21}Il frappa aussi un Egyptien d'un aspect formidable et ayant une lance à la main ; Benaja descendit contre lui avec un bâton, arracha la lance de la main de l'Egyptien, et s'en servit pour le tuer.
\VS{22}Benaja, fils de Jehojada, fit ces choses-là ; et fut illustre parmi les trois vaillants hommes.
\VS{23}Il était le plus considéré des trente ; mais il n'égala pas les trois premiers. C'est pourquoi David l'établit dans son conseil secret.
\VS{24}Asaël, frère de Joab, était des trente. Elchanan, fils de Dodo, de Bethléhem.
\VS{25}Schamma, de Harod. Elika, de Harod.
\VS{26}Hélets, de Péleth. Ira, fils d'Ikkesch, de Tekoa.
\VS{27}Abiézer, d'Anathoth. Mebunnaï, de Huscha.
\VS{28}Tsalmon, d'Achoach. Maharaï, de Nethopha.
\VS{29}Héleb, fils de Baana, de Nethopha. Ittaï, fils de Ribaï, de Guibea des fils de Benjamin.
\VS{30}Benaja, de Pirathon. Hiddaï, de Nachalé-Gaasch.
\VS{31}Abi-Albon, d'Araba. Azmaveth, de Barchum.
\VS{32}Eliachba, de Schaalbon. Bené-Jaschen. Jonathan.
\VS{33}Schamma, d'Harar. Achiam, fils de Scharar, d'Arar.
\VS{34}Eliphéleth, fils d'Achasbaï, fils d'un Maacathien. Eliam, fils d'Achitophel, de Guilo.
\VS{35}Hetsraï, de Carmel. Paaraï, d'Arab.
\VS{36}Jigueal, fils de Nathan, de Tsoba. Bani, de Gad.
\VS{37}Tsélek, l'Ammonite. Naharaï, de Beéroth, qui portait les armes de guerre de Joab, fils de Tseruja.
\VS{38}Ira, de Jéther. Gareb, de Jéther.
\VS{39}Urie, le Héthien. En tout, trente-sept.
\Chap{24}
\TextTitle{Péché de David ; plaie mortelle sur Israël\FTNTT{1 Ch. 21:1-17}}
\VerseOne{}La colère de Yahweh s'enflamma encore contre Israël, parce que David fut incité contre eux, en disant : Va, fais le dénombrement d'Israël et de Juda\FTNT{1 Ch. 21.}.
\VS{2}Le roi dit donc à Joab, chef de l'armée qui se trouvait près de lui : Parcours toutes les tribus d'Israël, depuis Dan jusqu'à Beer-Schéba ; et dénombre le peuple, afin que je sache le nombre du peuple.
\VS{3}Joab dit au roi : Que Yahweh, ton Dieu, veuille augmenter ton peuple cent fois plus, et que les yeux du roi mon seigneur le voient ! Mais pourquoi le roi mon seigneur prend-il plaisir à cela ?
\VS{4}Néanmoins, la parole du roi l'emporta sur Joab, et sur les chefs de l'armée ; et Joab et les chefs de l'armée sortirent de la présence du roi pour dénombrer le peuple d'Israël.
\VS{5}Ils passèrent le Jourdain, et ils campèrent à Aroër, à droite de la ville qui est au milieu de la vallée du torrent de Gad, et vers Jaezer.
\VS{6}Ils allèrent en Galaad et dans le territoire de ceux qui habitent vers le bas du pays de Thachthim-Hodschi. Ils allèrent à Dan-Jaan, et aux environs de Sidon.
\VS{7}Ils vinrent jusqu'à la forteresse de Tyr, et dans toutes les villes des Héviens et des Cananéens. Ils sortirent vers le midi de Juda à Beer-Schéba.
\VS{8}Ainsi ils parcoururent tout le pays, et arrivèrent à Jérusalem au bout de neuf mois et vingt jours.
\VS{9}Et Joab donna au roi le rôle du dénombrement du peuple : Il y avait en Israël huit cent mille hommes de guerre tirant l'épée, et en Juda cinq cent mille hommes.
\VS{10}Alors David sentit battre son cœur, après qu'il eut fait ainsi dénombrer le peuple. Et David dit à Yahweh : J'ai commis un grand péché en faisant cela ! Mais, je te prie, ô Yahweh, de pardonner l'iniquité de ton serviteur, car j'ai agi en insensé !
\VS{11}Après cela, David se leva dès le matin, et la parole de Yahweh fut adressée à Gad le prophète, qui était le voyant de David :
\VS{12}Va dire à David : Ainsi parle Yahweh : J'apporte trois choses contre toi ; choisis l'une d'elles afin que je te la fasse.
\VS{13}Gad alla vers David, et lui rapporta cela en disant : Que veux-tu qu'il t'arrive : Sept ans de famine sur ton pays, ou que durant trois mois tu fuies devant tes ennemis qui te poursuivront, ou que durant trois jours la peste soit dans ton pays ? Choisis maintenant, et regarde ce que tu veux que je réponde à celui qui m'a envoyé.
\VS{14}David répondit à Gad : Je suis dans une très grande détresse ! Tombons entre les mains de Yahweh, car ses compassions sont en grand nombre ; mais que je ne tombe pas entre les mains des hommes !
\VS{15}Yahweh envoya donc la peste en Israël, depuis le matin jusqu'au temps fixé ; et depuis Dan jusqu'à Beer-Schéba, il mourut soixante-dix mille hommes parmi le peuple.
\VS{16}Mais quand l'ange étendait sa main sur Jérusalem pour la ravager, Yahweh se repentit de ce mal et dit à l'ange qui ravageait le peuple : C'est assez ! Retire maintenant ta main. Or l'Ange de Yahweh était près de l'aire d'Aravna, le Jébusien.
\VS{17}Car David voyant l'ange qui frappait le peuple, parla à Yahweh, et dit : Voici, c'est moi qui ai péché ! C'est moi qui ai commis l'iniquité ; mais ces brebis, qu'ont-elles fait ? Je te prie que ta main soit contre moi et contre la maison de mon père !
\TextTitle{Sacrifice de David ; Yahweh met fin à la plaie\FTNTT{1 Ch. 21:18-30}}
\VS{18}Ce jour-là, Gad vint vers David, et lui dit : Monte, et dresse un autel à Yahweh dans l'aire d'Aravna, le Jébusien.
\VS{19}Et David monta, selon la parole de Gad, comme Yahweh l'avait ordonné.
\VS{20}Aravna regarda, et vit le roi et ses serviteurs qui venaient vers lui ; et Aravna sortit, et se prosterna devant le roi, le visage contre terre.
\VS{21}Aravna dit : Pourquoi le roi mon seigneur vient-il vers son serviteur ? Et David répondit : Pour acheter ton aire, et y bâtir un autel à Yahweh, afin que cette plaie se retire de dessus le peuple.
\VS{22}Aravna dit à David : Que le roi mon seigneur prenne et offre ce qu'il lui plaira ; vois les bœufs seront pour l'holocauste, et les chars avec l'attelage de bœufs serviront de bois.
\VS{23}Aravna donna tout cela au roi. Et Aravna dit au roi : Que Yahweh, ton Dieu, te soit favorable !
\VS{24}Mais le roi répondit à Aravna : Non ! Je veux l'acheter de toi pour un certain prix, et je n'offrirai point à Yahweh, mon Dieu, des holocaustes qui ne me coûtent rien. Ainsi, David acheta l'aire et les bœufs pour cinquante sicles d'argent.
\VS{25}David bâtit là un autel à Yahweh, et offrit des holocaustes et des sacrifices d'offrande de paix. Alors Yahweh fut apaisé envers le pays, et la plaie se retira d'Israël.
\PPE{}
\end{multicols}
