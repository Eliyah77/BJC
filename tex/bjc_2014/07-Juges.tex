\ShortTitle{Juges}\BookTitle{Juges}\BFont
\noindent\hrulefill
{\footnotesize
\textit{
\bigskip
{\centering{}
\\Auteur : Inconnu
\\(Heb. : Shoftim)
\\Signification : Être juge, prononcer, punir
\\Thème : Défaites et délivrances
\\Date de rédaction : Environ 1100 av. J.-C.\\}
}
%\bigskip
\textit{
\\A la mort de Josué et des anciens, il s'éleva en Israël une nouvelle génération qui n'avait pas connu l'expérience du désert. Elle fit ce qui est mal aux yeux de Dieu, l'abandonna et tomba dans l'idolâtrie. Ainsi, la colère de Yahweh s'abattit sur Israël et il livra le peuple entre les mains de ses ennemis. Dans ces temps de troubles, Dieu suscita des juges - douze hommes et une femme - pour délivrer Israël de ses oppresseurs. Aussi longtemps que le juge était en vie, Israël était en paix. Mais dès qu'il venait à mourir, le peuple se corrompait de nouveau et ses oppressions recommençaient.\bigskip
}
}
\par\nobreak\noindent\hrulefill
\begin{multicols}{2}
\Chap{1}
\TextTitle{Poursuite de la conquête de Canaan}
\VerseOne{}Or il arriva qu'après la mort de Josué, les fils d'Israël consultèrent Yahweh, en disant : Qui de nous montera le premier contre les Cananéens pour leur faire la guerre ?
\VS{2}Et Yahweh répondit : Juda montera ; voici, j'ai livré le pays entre ses mains.
\VS{3}Juda dit à Siméon son frère : Monte avec moi dans mon lot et nous ferons la guerre aux Cananéens ; et j'irai aussi avec toi dans ton lot. Ainsi Siméon alla avec lui.
\TextTitle{Victoires de Juda ; Caleb prend possession d'Hébron}
\VS{4}Juda monta, et Yahweh livra les Cananéens et les Phéréziens entre leurs mains ; ils battirent dix mille hommes à Bézek.
\VS{5}Et ils trouvèrent Adoni-Bézek à Bézek ; ils l'attaquèrent et frappèrent les Cananéens et les Phéréziens.
\VS{6}Adoni-Bézek s'enfuit mais ils le poursuivirent ; et l'ayant pris, ils lui coupèrent les pouces des mains et des pieds.
\VS{7}Alors Adoni-Bézek dit : Soixante-dix rois, dont les pouces des mains et des pieds avaient été coupés, ramassaient du pain sous ma table ; Dieu me rend ce que j'ai fait. On l'amena à Jérusalem et il y mourut\FTNT{Es. 33:1}.
\VS{8}Les fils de Juda firent la guerre contre Jérusalem et la prirent, ils frappèrent ses habitants du tranchant de l'épée et mirent le feu à la ville.
\VS{9}Puis les fils de Juda descendirent pour faire la guerre aux Cananéens, qui habitaient la montagne, la contrée du midi et la plaine.
\VS{10}Juda marcha contre les Cananéens qui habitaient à Hébron ; or le nom d'Hébron était auparavant Kirjath-Arba ; et il battit Schéschaï, Ahiman et Talmaï\FTNT{Jos. 15:14.}.
\VS{11}De là, il marcha contre les habitants de Debir ; Debir s'appelait auparavant Kirjath-Sépher\FTNT{Jos. 15:15.}.
\VS{12}Caleb dit : Je donnerai ma fille Acsa pour femme à celui qui frappera Kirjath-Sépher et qui la prendra\FTNT{Jos. 15:16.}.
\VS{13}Othniel, fils de Kenaz, frère cadet de Caleb, la prit ; et Caleb lui donna sa fille Acsa pour femme.
\VS{14}Et il arriva que comme elle s'en allait, elle l'incita à demander à son père un champ. Puis elle descendit impétueusement de dessus son âne ; et Caleb lui dit : Qu'as-tu ?\FTNT{Jos. 15:18.}
\VS{15}Elle lui répondit : Donne-moi un présent, puisque tu m'as donné une terre du midi ; donne-moi aussi des sources d'eau. Et Caleb lui donna les sources supérieures et les sources inférieures.
\VS{16}Les fils du Kénien, beau-père de Moïse, montèrent de la ville des palmiers avec les fils de Juda, dans le désert de Juda, qui est au midi d'Arad, et ils allèrent et demeurèrent avec le peuple\FTNT{Jg. 4:11}.
\VS{17}Puis Juda se mit en marche avec Siméon son frère et ils frappèrent les Cananéens qui habitaient à Tsephath ; et ils détruisirent la ville par le moyen de l'interdit, c'est pourquoi on appela la ville du nom de Horma.
\VS{18}Juda prit aussi Gaza avec ses territoires ; Askalon avec ses territoires ; et Ekron avec ses territoires.
\TextTitle{Des victoires en demi-teintes}
\VS{19}Yahweh fut avec Juda et il se rendit maître de la montagne, mais il ne pût chasser les habitants de la vallée, parce qu'ils avaient des chars de fer.
\VS{20}On donna Hébron à Caleb, comme Moïse l'avait dit ; et il en chassa les trois fils d'Anak\FTNT{No. 14:24}.
\VS{21}Quant aux fils de Benjamin, ils ne chassèrent pas les Jébusiens qui habitaient à Jérusalem ; c'est pourquoi les Jébusiens ont habité avec les fils de Benjamin à Jérusalem jusqu'à ce jour.
\VS{22}Ceux de la maison de Joseph montèrent aussi contre Béthel, et Yahweh fut avec eux.
\VS{23}Ceux de la maison de Joseph firent explorer Béthel, dont le nom était auparavant Luz.
\VS{24}Les espions virent un homme qui sortait de la ville, et ils dirent : Nous te prions de nous montrer un endroit par où l'on puisse entrer dans la ville, et nous te ferons grâce.
\VS{25}Il leur montra donc un endroit où l'on pouvait entrer dans la ville. Et ils frappèrent la ville du tranchant de l'épée ; mais ils laissèrent aller cet homme et toute sa famille.
\VS{26}Puis cet homme se rendit dans le pays des Héthiens ; il bâtit une ville et lui donna le nom de Luz, nom qu'elle a porté jusqu'à ce jour.
\VS{27}Manassé aussi ne chassa pas les habitants de Beth-Schean et des villes de son ressort, de Thaanac et des villes de son ressort, de Dor et des villes de son ressort, les habitants de Jibleam et des villes de son ressort, les habitants de Meguiddo et des villes de son ressort ; et les Cananéens persistèrent à habiter dans ce pays-là.
\VS{28}Il est vrai qu'il arriva que quand Israël fut devenu plus fort, il assujettit les Cananéens à un tribut mais il ne les chassa pas entièrement.
\VS{29}Ephraïm aussi ne chassa pas les Cananéens qui habitaient à Guézer, mais les Cananéens habitèrent avec lui à Guézer.
\VS{30}Zabulon ne chassa pas les habitants de Kitron, ni les habitants de Nahalol ; et les Cananéens habitèrent avec lui et lui furent tributaires.
\VS{31}Aser ne chassa pas les habitants d'Acco, ni les habitants de Sidon, ni ceux d'Achlal, ni d'Aczib, ni d'Helba, ni d'Aphik, ni de Rehob ;
\VS{32}Mais ceux d'Aser habitèrent parmi les Cananéens, habitants du pays ; car ils ne les chassèrent pas.
\VS{33}Nephthali ne chassa pas les habitants de Beth-Schémesch, ni les habitants de Beth-Anath, mais il habita parmi les Cananéens habitants du pays ; et les habitants de Beth-Schémesch, et de Beth-Anath lui furent tributaires.
\VS{34}Les Amoréens repoussèrent les fils de Dan dans la montagne et ne les laissèrent pas descendre dans la vallée.
\VS{35}Les Amoréens voulurent encore habiter à Har-Hérès, à Ajalon et à Schaalbim ; mais la main de la maison de Joseph étant devenue plus forte, ils furent assujettis à un tribut.
\VS{36}Le territoire des Amoréens s'étendait depuis la montée d'Akrabbim, depuis Séla et en dessus.
\Chap{2}
\TextTitle{Le peuple repris à cause de sa désobéissance}
\VerseOne{}Or l'Ange de Yahweh monta de Guilgal à Bokim, et dit : Je vous ai fait monter hors d'Egypte, et je vous ai fait entrer dans le pays que j'avais juré à vos pères, et j'ai dit : Je n'enfreindrai jamais mon alliance que j'ai traitée avec vous\FTNT{Ge. 17:7.} ;
\VS{2}et vous aussi, vous ne traiterez pas alliance avec les habitants de ce pays, vous démolirez leurs autels. Mais vous n'avez pas obéi à ma voix. Pourquoi avez-vous fait cela\FTNT{Ex. 23:32 ; De. 7:2 ; De. 12:3.} ?
\VS{3}J'ai dit alors : Je ne les chasserai pas devant vous, mais ils seront à vos côtés, et leurs dieux vous seront un piège\FTNT{Ex. 23:33 ; Jos. 23:13.}.
\VS{4}Et il arriva que, comme l'Ange de Yahweh disait ces paroles à tous les fils d'Israël, le peuple éleva la voix et pleura.
\VS{5}C'est pourquoi ils appelèrent ce lieu Bokim et ils y offrirent des sacrifices à Yahweh.
\VS{6}Josué renvoya le peuple, et les enfants d'Israël allèrent chacun dans son héritage pour prendre possession du pays\FTNT{Jos. 24:28–32.}.
\VS{7}Le peuple servit Yahweh tout le temps de Josué, et tout le temps des anciens qui survécurent à Josué et qui avaient vu toutes les grandes œuvres que Yahweh avait faites en faveur d'Israël\FTNT{Jos. 24:31.}.
\VS{8}Puis Josué, fils de Nun, serviteur de Yahweh, mourut, âgé de cent dix ans\FTNT{Jos. 24:29.}.
\VS{9}On l'ensevelit dans le territoire qu'il avait eu en partage à Thimnath-Hérès, dans la montagne d'Ephraïm, au nord de la montagne de Gaasch\FTNT{Jos. 24:30.}.
\TextTitle{La nouvelle génération abandonne Yahweh}
\VS{10}Toute cette génération fut recueillie auprès de ses pères, puis il s'éleva après elle une autre génération, qui ne connaissait pas Yahweh ni les œuvres qu'il avait faites en faveur d'Israël.
\VS{11}Les enfants d'Israël firent alors ce qui est mal aux yeux de Yahweh et ils servirent les Baals\FTNT{L'appellatif « Baal » en hébreu, ou « Bel » en phénicien, signifie « seigneur », « maître », ou « possesseur », et désigne parfois « l'époux ». Ce terme qui sert avant tout de titre, signale aussi bien des divinités, des êtres humains ou encore des villes portant le nom du dieu tutélaire de la cité. Comme il existait une multitude de « baalim » (Jg. 2:13), ce nom était la plupart du temps associé à un autre nom ou à un qualificatif : Baal-Hanan, seigneur de compassion (Ge. 36:38-39 ; 1 Ch. 1:49-50 ; 1 Ch. 27:28), Baal-Tsephon, seigneur du nord (Ex. 14:2 ; Ex. 14:9 ; No. 33:7), Baal-Peor, seigneur de la brèche (No. 25:3 ; No. 25:5 ; De. 4:3 ; Ps. 106:28 ; Os. 9:10), Baal-Gad, seigneur des richesses (Jos. 11:17 ; Jos. 12:7 ; Jos. 13:5), Kitjath-Baal, ville de Baal (Jos. 15:60 ; Jos. 18:14), Beth-Baal-Meon, maison de Baal (Jos. 13:17 ; Jé. 48:23), Baal-Berith, seigneur de l'alliance (Jg. 8:33 ; Jg. 9:4), Baal-Zebub, seigneur des mouches (2 R. 1:2-3 ; 2 R. 1:6 ; 2 R. 1:16). Toutefois, l'histoire antique du Proche-Orient fut marquée par la figure emblématique d'un Baal, équivalent de Seth chez les Egyptiens à l'époque des Ramessides. Connu pour être le dieu de la vie, le seigneur de la terre et du ciel, « le chevaucheur des nuées », c'est à lui qu'on attribuait la fertilisation du sol par l'envoi de la pluie. Selon la mythologie Cananéenne, il était condamné à livrer une guerre perpétuelle à Mot, dieu de la guerre et de la stérilité. Si Baal était vainqueur, la terre bénéficiait d'un cycle de sept ans de fertilité ; s'il perdait, Mot installait un cycle de sept ans de sécheresse et de famine. C'est précisément ce Baal et ses prophètes qu'Elie défia au nom de Yahweh (1 R. 17:1 ; 1 R. 18:21-46). Toujours accompagné d'une déesse (« baalat »), le plus souvent Astarté, son culte était licencieux et se déroulait dans les hauts lieux, près des bosquets ou des bocages (« asherah » en hébreu, 2 R. 17 :10) Baal avait ses propres prêtres (So. 1:4), ses prophètes (2 R. 18), et exigeait, selon la circonstance, diverses offrandes, des sacrifices d'animaux ou d'humains (Jé. 7:9 ; Jé. 19:5).}.
\VS{12}Et ils abandonnèrent Yahweh, le Dieu de leurs pères, qui les avait fait sortir du pays d'Egypte, ils allèrent après d'autres dieux, d'entre les dieux des peuples qui les entouraient ; et ils se prosternèrent devant eux, irritant ainsi Yahweh.
\VS{13}Ils abandonnèrent donc Yahweh, et servirent Baal et les Astartés\FTNT{Le nom « Astarté » vient de hébreu « Ashtarowth », ce qui signifie « étoiles » ou encore « accroissement ». Employé au pluriel, ce terme renvoie généralement aux divinités féminines (Jg. 2:13 ; Jg. 10:6 ; 1 S. 7:3-4 ; 1 S. 12:10 ; 1 S. 13:10 ; 2Ch. 24:18). Manifestation sémitique de la déesse Isthar des Babyloniens, ou encore d'Inanna chez les Sumériens, elle était connue dans l'Egypte des Ramessides comme la fille de Rê ou de Ptah, puis la compagne de Seth, équivalent de Baal, auquel elle était immanquablement associée. Initialement connue pour son caractère belliqueux, elle était représentée à cheval, faisant bénéficier le souverain de sa protection. Déesse de la terre et de la nature, on lui attribuait la fertilité du sol et l'opulence des moissons. Comme pour Baal, son culte licencieux était célébré sur les hauteurs, et accompagné de sacrifices sanglants, y compris humains (2 R. 23:7). Au fil du temps, elle prit une telle ampleur au point que Salomon, entraîné par ses concubines, devint son adorateur (1 R. 11:5). Devenue l'incontournable « reine des cieux », les Hébreux continuèrent à la vénérer avec assiduité, et ce, en dépit de la déportation babylonienne qui était la conséquence directe de leur idolâtrie (Jé. 7:18 ; Jé. 44:15-26).}.
\VS{14}La colère de Yahweh s'enflamma contre Israël. Il les livra entre les mains de pillards\FTNT{Lorsqu'un enfant de Dieu ouvre la porte au péché, il s'expose aux pillards, c'est-à-dire à Satan et ses démons (Jn. 10:10).} qui les pillèrent, et il les vendit entre les mains de leurs ennemis d'alentour, de sorte qu'ils ne purent plus résister face à leurs ennemis\FTNT{Es. 50:1 ; Ps. 44:12-13.}.
\VS{15}Partout où ils allaient, la main de Yahweh était contre eux pour leur faire du mal, comme Yahweh en avait parlé et comme Yahweh le leur avait juré, ils furent dans une grande détresse\FTNT{Lé. 26:25 ; De. 28:25.}.
\TextTitle{Yahweh suscite des libérateurs : Les juges}
\VS{16}Yahweh leur suscita des juges\FTNT{Les Juges étaient principalement des libérateurs de l'oppression des ennemis d'Israël.} et ils les délivrèrent de la main de ceux qui les pillaient.
\VS{17}Mais ils ne voulurent pas écouter leurs juges, ils se prostituèrent auprès d'autres dieux, et se prosternèrent devant eux. Ils se détournèrent promptement du chemin qu'avaient suivi leurs pères et ils n'obéirent pas comme eux aux commandements de Yahweh.
\VS{18}Quand Yahweh leur suscitait des juges, Yahweh était avec le juge, et il les délivrait de la main de leurs ennemis pendant tout le temps de la vie du juge ; car Yahweh se repentait à cause de leurs gémissements contre ceux qui les opprimaient et les tourmentaient.
\VS{19}Puis il arrivait que quand le juge mourrait, ils se corrompaient de nouveau plus que leurs pères en allant après d'autres dieux pour les servir et se prosterner devant eux, et ils persévéraient dans la même conduite et dans la même voie obstinée\FTNT{Jg. 3:12.}.
\TextTitle{IYahweh éprouve Israël et ne chasse pas ses ennemis}
\VS{20}C'est pourquoi la colère de Yahweh s'enflamma contre Israël, et il dit : Puisque cette nation a transgressé mon alliance que j'avais prescrite à leurs pères, et puisqu'ils n'ont pas obéi à ma voix,
\VS{21}aussi je ne chasserai plus devant eux aucune des nations que Josué laissa quand il mourut\FTNT{Jos. 23:13.},
\VS{22}afin d'éprouver par elles Israël, et voir s'ils garderont la voie de Yahweh pour y marcher, comme leurs pères l'ont gardée, ou non.
\VS{23}Yahweh laissa en repos ces nations qu'il n'avait pas livrées entre les mains de Josué et il ne se hâta pas de les chasser\FTNT{Jg. 3:1-3.}.
\Chap{3}
\VerseOne{}Voici les nations que Yahweh laissa pour éprouver par elles Israël, tous ceux qui n'avaient pas connu toutes les guerres de Canaan\FTNT{Jg. 2:21-23.} ; 
\VS{2}afin qu'au moins les générations des enfants d'Israël connaissent et apprennent la guerre, ceux qui ne l'avaient pas connue auparavant.
\VS{3}Ces nations étaient : Les cinq princes des Philistins, tous les Cananéens, les Sidoniens et les Héviens qui habitaient la montagne du Liban, depuis la montagne de Baal-Hermon, jusqu'à l'entrée de Hamath\FTNT{No. 13:22.}.
\VS{4}Ces nations, dis-je, servirent à éprouver Israël, pour voir s'ils obéiraient aux commandements que Yahweh avait donnés à leurs pères par le moyen de Moïse.
\TextTitle{Israël se mélange aux nations païennes}
\VS{5}Ainsi les enfants d'Israël habitèrent parmi les Cananéens, les Héthiens, les Amoréens, les Phéréziens, les Héviens et les Jébusiens.
\VS{6}Ils prirent leurs filles pour femmes, ils donnèrent leurs filles à leurs fils et servirent leurs dieux.
\VS{7}Les enfants d'Israël firent donc ce qui est mal aux yeux de Yahweh, ils oublièrent Yahweh et servirent les Baals et les Astartés\FTNT{Jg. 2:11.}.
\TextTitle{Othniel, premier juge suscité par Yahweh}
\VS{8}C'est pourquoi la colère de Yahweh s'enflamma contre Israël, et il les vendit entre la main de Cuschan-Rischeathaïm, roi de Mésopotamie. Et les enfants d'Israël furent asservis à Cuschan-Rischeathaïm durant huit ans.
\VS{9}Puis les enfants d'Israël crièrent à Yahweh, et Yahweh leur suscita un libérateur qui les délivra, Othniel, fils de Kenaz, frère cadet de Caleb.
\VS{10}L'Esprit de Yahweh fut sur lui. Il devint juge en Israël, et il sortit pour la guerre. Yahweh livra entre ses mains Cuschan-Rischeathaïm, roi de Mésopotamie ; et sa main fut puissante contre Cuschan-Rischeathaïm.
\VS{11}Le pays fut en repos pendant quarante ans. Puis Othniel, fils de Kenaz, mourut.
\TextTitle{Ehud, juge en Israël}
\VS{12}Les enfants d'Israël firent encore ce qui est mal aux yeux de Yahweh ; et Yahweh fortifia Eglon, roi de Moab, contre Israël, parce qu'ils avaient fait ce qui est mal aux yeux de Yahweh.
\VS{13}Eglon réunit auprès de lui les fils d'Ammon et les Amalécites et il se mit en marche. Il battit Israël et ils s'emparèrent de la ville des palmiers\FTNT{Palmiers: un autre nom de Jéricho.}.
\VS{14}Et les enfants d'Israël furent asservis à Eglon, roi de Moab, durant dix-huit ans.
\VS{15}Puis les enfants d'Israël crièrent à Yahweh, et Yahweh leur suscita un libérateur, Ehud, fils de Guéra, Benjamite, qui ne se servait pas de sa main droite. Les enfants d'Israël envoyèrent par lui un présent à Eglon, roi de Moab.
\VS{16}Ehud se fit une épée à deux tranchants, de la longueur d'une coudée\FTNT{Une coudée correspond environ à 45 cm.} et il la ceignit sous ses vêtements, sur sa cuisse droite.
\VS{17}Il offrit le présent à Eglon, roi de Moab ; et Eglon était un homme fort gras.
\VS{18}Or il arriva que lorsqu'il eut achevé d'offrir le présent, il renvoya le peuple qui avait apporté le présent.
\VS{19}Mais Ehud revint depuis les idoles de pierre, qui étaient près de Guilgal et il dit : Ô roi ! J'ai quelque chose de secret à te dire. Et il lui répondit : Tais-toi ! Et tous ceux qui étaient auprès de lui sortirent de là.
\VS{20}Ehud s'approcha de lui, comme il était assis seul dans sa chambre d'été, et il dit : J'ai un mot à te dire de la part de Dieu, alors le roi se leva du trône.
\VS{21}Et Ehud avança sa main gauche, tira l'épée de son côté droit et la lui enfonça dans le ventre.
\VS{22}Et la poignée entra après la lame, et la graisse serra tellement la lame, qu'il ne pouvait retirer l'épée du ventre, et il en sortit de l'excrément.
\VS{23}Après cela, Ehud sortit par le portique, ferma après lui les portes de la chambre et tira le verrou.
\VS{24}Quand il fut sorti, les serviteurs d'Eglon vinrent et regardèrent ; et voici, les portes de la chambre étaient fermées au verrou. Ils dirent : Sans doute il se couvre les pieds dans sa chambre d'été.
\VS{25}Et ils attendirent tant qu'ils en furent déconcertés ; et voyant qu'il n'ouvrait pas les portes de la chambre, ils prirent la clef et ouvrirent ; et voici, leur maître était mort, étendu à terre.
\VS{26}Mais Ehud s'échappa pendant qu'ils hésitaient ; et il dépassa les carrières de pierre et se sauva à Seïra.
\VS{27}Dès qu'il fut arrivé, il sonna du shofar dans la montagne d'Ephraïm. Les fils d'Israël descendirent avec lui de la montagne et il marchait à leur tête.
\VS{28}Il leur dit : Suivez-moi, car Yahweh a livré entre vos mains les Moabites, vos ennemis. Ainsi ils descendirent après lui, s'emparèrent des passages du Jourdain vis-à-vis de Moab et ne laissèrent passer personne.
\VS{29}Ils battirent dans ce temps-là environ dix mille hommes de Moab, tous robustes, tous vaillants et il n'en échappa aucun.
\VS{30}En ce jour, Moab fut humilié sous la main d'Israël. Et le pays fut en repos pendant quatre-vingts ans.
\TextTitle{Schamgar, juge en Israël}
\VS{31}Après lui, il y eut Schamgar, fils d'Anath. Il battit six cents Philistins avec un aiguillon à bœufs et délivra Israël.
\Chap{4}
\TextTitle{Débora et Barak, juges en Israël}
\VerseOne{}Mais les enfants d'Israël firent encore ce qui est mal aux yeux de Yahweh après qu'Ehud fut mort.
\VS{2}C'est pourquoi Yahweh les vendit entre les mains de Jabin, roi de Canaan, qui régnait à Hatsor. Le chef de son armée était Sisera, qui habitait à Haroscheth-Goïm\FTNT{Jg. 3:8-16 ; Jos. 11:11-13 ; 1 S. 12:9.}.
\VS{3}Les enfants d'Israël crièrent à Yahweh car Jabin avait neuf cents chars de fer, et il avait violemment opprimé les enfants d'Israël durant vingt ans\FTNT{Jg. 1:19.}.
\VS{4}Dans ce temps-là, Débora, prophétesse, femme de Lappidoth, était juge en Israël.
\VS{5}Débora se tenait sous un palmier, entre Rama et Béthel, dans la montagne d'Ephraïm ; et les enfants d'Israël montaient vers elle pour être jugés.
\VS{6}Elle envoya appeler Barak, fils d'Abinoam, de Kédesch-Nephthali et elle lui dit : Yahweh, le Dieu d'Israël, n'a-t-il pas donné cet ordre ? En disant : Va, et dirige-toi sur la montagne de Thabor et prends avec toi dix mille hommes des fils de Nephthali, et des fils de Zabulon\FTNT{Hé. 11:32.} ;
\VS{7}j'attirerai vers toi, au torrent de Kison, Sisera, chef de l'armée de Jabin, avec ses chars et ses troupes et je le livrerai entre tes mains\FTNT{Ps. 83:9-10.}.
\VS{8}Barak lui dit : Si tu viens avec moi, j'irai ; mais si tu ne viens pas avec moi, je n'irai pas.
\VS{9}Elle répondit : J'irai, j'irai avec toi, mais tu n'auras pas d'honneur sur le chemin où tu marches ; car Yahweh livrera Sisera entre les mains d'une femme. Débora se leva et elle alla avec Barak à Kédesch.
\VS{10}Barak convoqua Zabulon et Nephthali à Kédesch ; dix mille hommes marchèrent à sa suite ; et Débora monta avec lui.
\VS{11}Héber, le Kénien, s'était séparé des fils de Hobab, beau-père de Moïse et il avait dressé ses tentes jusqu'au chêne de Tsaannaïm, près de Kédesch\FTNT{No. 10:29.}.
\TextTitle{Yahweh accorde la victoire à Israël}
\VS{12}On rapporta à Sisera que Barak, fils d'Abinoam, s'était dirigé sur la montagne de Thabor.
\VS{13}Et Sisera rassembla tous ses chars, neuf cents chars de fer, et tout le peuple qui était avec lui, depuis Haroscheth-Goïm, jusqu'au torrent de Kison.
\VS{14}Alors Débora dit à Barak : Lève-toi, car voici le jour où Yahweh livre Sisera entre tes mains. Yahweh ne marche-t-il pas devant toi ? Barak descendit de la montagne de Thabor, ayant dix mille hommes à sa suite.
\VS{15}Yahweh mit en déroute devant Barak, Sisera, tous ses chars et toute l'armée, par le tranchant de l'épée. Sisera descendit du char et s'enfuit à pied\FTNT{Ps. 83:9-10.}.
\VS{16}Barak poursuivit les chars et l'armée jusqu'à Haroscheth-Goïm ; et toute l'armée de Sisera fut passée au fil de l'épée ; il n'en resta pas un seul.
\VS{17}Sisera se sauva à pied dans la tente de Jaël, femme de Héber, le Kénien ; car il y avait paix entre Jabin, roi de Hatsor et la maison de Héber, le Kénien.
\VS{18}Jaël étant sortie au-devant de Sisera, lui dit : Entre, mon seigneur, entre chez moi, ne crains pas. Il entra donc chez elle dans la tente et elle le cacha sous une couverture.
\VS{19}Puis il lui dit : Je te prie, donne-moi un peu d'eau à boire, car j'ai soif. Et elle ouvrit une outre de lait, lui donna à boire et le couvrit\FTNT{Jg. 5:25.}.
\VS{20}Il lui dit encore : Tiens-toi à l'entrée de la tente et si l'on vient t'interroger, en disant : Y a-t-il ici quelqu'un ? Alors tu répondras : Non.
\VS{21}Jaël, femme de Héber, saisit un pieu de la tente, prit en sa main un marteau, s'approcha de lui doucement, et lui enfonça dans la tempe le pieu, qui pénétra en terre, pendant qu'il dormait profondément, car il était accablé de fatigue. Et ainsi il mourut.
\VS{22}Et voici, Barak poursuivait Sisera, Jaël sortit au-devant de lui et lui dit : Viens, et je te montrerai l'homme que tu cherches. Barak entra chez elle, et voici, Sisera était étendu mort, et le pieu était dans sa tempe.
\VS{23}En ce jour-là, Dieu humilia Jabin, roi de Canaan, devant les enfants d'Israël.
\VS{24}Et la puissance des enfants d'Israël s'avançait et se renforçait de plus en plus contre Jabin, Roi de Canaan, jusqu'à ce qu'ils l'eurent exterminé.
\Chap{5}
\TextTitle{Cantique à la gloire de Yahweh, le Dieu qui délivre}
\VerseOne{}En ce jour-là, Débora chanta ce cantique avec Barak, fils d'Abinoam, en disant :
\VS{2}Bénissez Yahweh de ce qu'il a fait de telles vengeances en Israël, et de ce que le peuple s'est offert volontairement.
\VS{3}Vous, rois, écoutez ! Vous, princes, prêtez l'oreille ! Moi, je chanterai à Yahweh, je chanterai un hymne à Yahweh, le Dieu d'Israël.
\VS{4}Ô Yahweh ! Quand tu sortis de Séir, quand tu t'avanças des champs d'Edom, la terre trembla, les cieux se fondirent, les nuées fondirent en eaux ;
\VS{5}les montagnes s'ébranlèrent devant Yahweh, ce Sinaï devant Yahweh, le Dieu d'Israël\FTNT{Ps. 68:8-9}.
\VS{6}Aux jours de Schamgar, fils d'Anath, aux jours de Jaël, les grandes routes étaient délaissées, et ceux qui voyageaient prenaient des chemins détournés.
\VS{7}Les villes sans murailles n'étaient plus habitées en Israël, elles n'étaient point habitées, jusqu'à ce que je me sois levée, moi Débora, jusqu'à ce que je me sois levée pour être mère en Israël.
\VS{8}Israël choisissait-il des dieux nouveaux ? Aussitôt la guerre était aux portes. On ne voyait ni bouclier ni lance chez quarante milliers en Israël.
\VS{9}J'ai mon cœur vers les chefs d'Israël, qui se sont portés volontairement d'entre le peuple. Bénissez Yahweh !
\VS{10}Vous qui montez sur les ânesses blanches, vous qui avez pour sièges des tapis et vous qui marchez sur le chemin, méditez !
\VS{11}Le bruit des archers ayant cessé dans les abreuvoirs, qu'on s'y entretienne des justices de Yahweh et des justices de ses villes sans murailles en Israël ; alors le peuple de Dieu descendra aux portes.
\VS{12}Réveille-toi, réveille-toi, Débora ! Réveille-toi, réveille-toi, dit le cantique, lève-toi Barak et emmène en captivité ceux que tu as faits captifs, toi fils d'Abinoam.
\VS{13}Yahweh a fait dominer un reste du peuple sur les puissants ; Yahweh m'a fait dominer sur les héros.
\VS{14}Leur racine est depuis Ephraïm jusqu'à Amalek. A ta suite marcha Benjamin parmi ta troupe. De Makir descendirent les chefs, et de Zabulon ceux qui manient la plume du scribe.
\VS{15}Et les chefs d'Issacar ont été avec Débora, et Issacar ainsi que Barak ; il a été envoyé avec sa suite dans la vallée ; il y a eu aux ruisseaux de Ruben, de grandes considérations dans leur cœur.
\VS{16}Pourquoi es-tu resté entre les barres des étables, à écouter le bêlement des troupeaux ? Aux ruisseaux de Ruben, grandes furent les résolutions du cœur !
\VS{17}Galaad est resté au-delà du Jourdain ; et pourquoi Dan est-il resté sur ses navires ? Aser s'est tenu sur le rivage de la mer, et s'est reposé dans ses ports.
\VS{18}Mais pour Zabulon, c'est un peuple qui a exposé son âme à la mort ; et Nephthali de même, sur les hauteurs des champs.
\VS{19}Les rois vinrent, ils combattirent. Alors combattirent les rois de Canaan, à Thaanac, près des eaux de Meguiddo ; mais ils ne remportèrent nul butin, nul argent.
\VS{20}On a combattu des cieux, les étoiles, dis-je, ont combattu du lieu de leur cours contre Sisera.
\VS{21}Le torrent de Kison les a emportés, le torrent des anciens temps, le torrent de Kison. Mon âme tu as foulé aux pieds les héros.
\VS{22}Alors les talons des chevaux battirent le sol à cause de la course rapide, de la course rapide de ses puissants chevaux.
\VS{23}Maudissez Méroz, dit l'Ange de Yahweh ; maudissez, maudissez ses habitants, car ils ne sont pas venus au secours de Yahweh, au secours de Yahweh, avec les héros.
\VS{24}Bénie soit par-dessus toutes les femmes Jaël, femme de Héber, le Kénien ! Qu'elle soit bénie entre les femmes qui habitent sous les tentes !
\VS{25}Il demanda de l'eau, elle lui a donné du lait ; elle lui a présenté de la crème dans la coupe des chefs.
\VS{26}Elle a saisi de sa main gauche le pieu et de sa main droite le marteau des ouvriers ; elle a frappé Sisera et lui a fendu la tête ; elle a fracassé et transpercé ses tempes.
\VS{27}Il s'est affaissé aux pieds de Jaël, il est tombé, il s'est couché aux pieds de Jaël ; il s'est affaissé, il est tombé ; là où il s'est affaissé, il est tombé là tout défiguré.
\VS{28}La mère de Sisera regardait par la fenêtre et s'écriait en regardant par les treillis : Pourquoi son char tarde-t-il à venir ? Pourquoi ses chars vont-ils si lentement ?
\VS{29} Et les plus sages de ses dames lui ont répondu ; et elle aussi se répondait à elle-même :
\VS{30}N'ont-ils pas trouvé ? Ils partagent le butin ; une fille, deux filles à chacun par tête. Le butin des vêtements de couleurs est à Sisera, le butin de couleurs de broderie ; couleur de broderie à deux endroits, autour du cou de ceux du butin.
\VS{31}Qu'ainsi périssent tous tes ennemis ô Yahweh ! Et que ceux qui t'aiment soient comme le soleil quand il sort dans sa force ! Et le pays fut en repos pendant quarante ans.
\Chap{6}
\TextTitle{Israël assujetti par Madian}
\VerseOne{}Or les enfants d'Israël firent ce qui est mal aux yeux de Yahweh ; et Yahweh les livra entre les mains de Madian pendant sept ans.
\VS{2}La main de Madian fut puissante contre Israël. Pour échapper aux Madianites, les enfants d'Israël se retiraient dans les ravins des montagnes, dans des cavernes et sur les rochers fortifiés.
\VS{3}Car il arrivait que quand Israël avait semé, Madian montait avec Amalek et les fils de l'orient, et ils montaient contre lui.
\VS{4}Ils faisaient un camp contre lui, ravageaient les fruits du pays jusqu'à Gaza et ne laissaient en Israël ni vivres, ni brebis, ni bœufs, ni ânes.
\VS{5}Car ils montaient avec leurs troupeaux et leurs tentes, ils arrivaient comme une multitude de sauterelles, ils étaient innombrables, eux et leurs chameaux et ils venaient dans le pays pour le ravager.
\VS{6}Israël fut très appauvri par Madian, et les enfants d'Israël crièrent à Yahweh.
\VS{7}Lorsque les enfants d'Israël crièrent à Yahweh au sujet de Madian,
\VS{8}Yahweh envoya un prophète aux enfants d'Israël, qui leur dit : Ainsi parle Yahweh, le Dieu d'Israël : Je vous ai fait monter hors d'Égypte et je vous ai retirés de la maison de servitude.
\VS{9}Je vous ai délivrés de la main des Egyptiens et de la main de tous ceux qui vous opprimaient ; je les ai chassés devant vous et je vous ai donné leur pays.
\VS{10}Je vous ai dit : Je suis Yahweh, votre Dieu ; vous ne craindrez pas les dieux des Amoréens, dans le pays desquels vous habitez. Mais vous n'avez pas obéi à ma voix.
\TextTitle{Gédéon rencontre l'Ange de Yahweh}
\VS{11}Puis l'Ange de Yahweh vint et s'assit sous le térébinthe d'Ophra, qui appartenait à Joas, de la famille d'Abiézer. Gédéon, son fils, battait du froment au pressoir pour le mettre à l'abri de Madian.
\VS{12}Alors l'Ange de Yahweh lui apparut et lui dit : Très fort et vaillant héros, Yahweh est avec toi !
\VS{13}Gédéon lui répondit : Hélas mon Seigneur ! Est-il possible que Yahweh soit avec nous ? Pourquoi donc toutes ces choses nous sont-elles arrivées ? Et où sont tous ces prodiges que nos pères nous ont racontés, en disant : Yahweh ne nous a-t-il pas fait monter hors d'Egypte ? Car maintenant Yahweh nous a abandonnés et nous a livrés entre les mains des Madianites.
\VS{14}Yahweh le regarda et lui dit : Va avec cette force que tu as et tu délivreras Israël de la main des Madianites ; ne t'ai-je pas envoyé\FTNT{Hé.11:32} ?
\VS{15}Et il lui répondit : Hélas, mon Seigneur ! Avec quoi délivrerai-je Israël ? Voici, mon millier de bétail est le plus pauvre en Manassé et je suis le plus petit de la maison de mon père\FTNT{1 S. 9:21 ; 1 S. 16:11.}.
\VS{16}Yahweh lui dit : Parce que je serai avec toi, tu frapperas les Madianites comme s'ils n'étaient qu'un seul homme.
\VS{17}Et il lui répondit : Je te prie, si j'ai trouvé grâce à tes yeux, donne-moi un signe pour montrer que c'est toi qui me parles.
\VS{18}Je te prie, ne t'éloigne pas d'ici jusqu'à ce que je revienne auprès de toi, que j'apporte mon offrande et que je la dépose devant toi. Yahweh dit : Je resterai jusqu'à ce que tu reviennes.
\VS{19}Alors Gédéon rentra et apprêta un chevreau de lait, et fit avec un épha de farine des pains sans levain. Il mit la chair dans un panier, le jus dans un pot et il les lui apporta sous le térébinthe, et les présenta.
\VS{20}L'Ange de Dieu lui dit : Prends la chair et les pains sans levain et pose-les sur ce rocher\FTNT{Voir commentaire en Es. 8:13-17} et répands le jus. Et il fit ainsi.
\VS{21}Alors l'Ange de Yahweh avança l'extrémité du bâton qu'il avait à la main, et toucha la chair et les pains sans levain. Le feu monta du rocher, et consuma la chair et les pains sans levain. Puis l'Ange de Yahweh disparut à ses yeux.
\VS{22}Gédéon, voyant que c'était l'Ange de Yahweh, dit : Ah, malheur à moi, Seigneur Yahweh ! Car j'ai vu l'Ange de Yahweh face à face.
\VS{23}Et Yahweh lui dit : Sois en paix, ne crains pas, tu ne mourras pas.
\VS{24}Gédéon bâtit là un autel à Yahweh, et lui donna pour nom Yahweh-Shalom. Cet autel, qui appartenait à la famille d'Abiézer, existe encore aujourd'hui à Ophra.
\TextTitle{Gédéon détruit les idoles ; Yahweh lui confirme sa mission}
\VS{25}Or il arriva dans cette nuit-là que Yahweh lui dit : Prends un jeune taureau d'entre les bœufs qui sont à ton père et un deuxième taureau de sept ans ; et démolis l'autel de Baal qui est à ton père, et abats l'idole d'Astarté qui est dessus.
\VS{26}Tu bâtiras ensuite et tu disposeras, sur le haut de ce rocher, un autel à Yahweh, ton Dieu. Tu prendras ce deuxième taureau, et tu l'offriras en holocauste avec le bois de l'emblème d'Astarté que tu auras démoli.
\VS{27}Gédéon ayant pris dix hommes parmi ses serviteurs, fit comme Yahweh lui avait dit ; et parce qu'il craignait la maison de son père et les gens de la ville, il l'exécuta de nuit et non de jour.
\VS{28}Lorsque les gens de la ville se levèrent de bon matin, voici, l'autel de Baal était démoli, et l'idole d'Astarté qui est dessus était abattue, et le deuxième taureau était offert en holocauste sur l'autel qui avait été bâti.
\VS{29}Ils se dirent les uns aux autres : Qui a fait cela ? Et ils s'informèrent et firent des recherches. On leur dit : C'est Gédéon, fils de Joas, qui a fait cela.
\VS{30}Puis les gens de la ville dirent à Joas : Fais sortir ton fils et qu'il meure ; car il a démoli l'autel de Baal et abattu l'idole d'Astarté qui était dessus.
\VS{31}Joas répondit à tous ceux qui s'adressèrent à lui : Est-ce à vous de prendre parti pour Baal, est-ce à vous de venir à son secours ? Quiconque prendra parti pour Baal sera mis à mort avant le matin. Si Baal est un dieu, qu'il défende lui-même sa cause puisqu'on a démoli son autel.
\VS{32}Et en ce jour on donna à Gédéon le nom de Jerubbaal, en disant : Que Baal défende sa cause, puisque Gédéon a démoli son autel.
\VS{33}Tout Madian, Amalek, et les fils de l'orient se rassemblèrent ; ils passèrent le Jourdain et campèrent dans la vallée de Jizréel.
\VS{34}Gédéon fut revêtu de l'Esprit de Yahweh ; il sonna du shofar et Abiézer fut convoqué pour marcher à sa suite\FTNT{Jg. 11:29 ; Jg. 13:25.}.
\VS{35}Il envoya des messagers dans tout Manassé qui fut aussi convoqué pour marcher à sa suite. Puis il envoya des messagers dans Aser, dans Zabulon et dans Nephthali, qui montèrent à leur rencontre.
\VS{36}Gédéon dit à Dieu : Si tu veux délivrer Israël par ma main, comme tu l'as dit,
\VS{37}voici, je vais mettre une toison de laine dans l'aire de battage ; si la toison seule se couvre de rosée et que tout le terrain reste sec, je connaîtrai que tu délivreras Israël par ma main, comme tu l'as dit.
\VS{38}Et il arriva ainsi. Le jour suivant, il se leva de bon matin, pressa la toison et en fit sortir la rosée qui donna de l'eau plein une coupe.
\VS{39}Gédéon dit encore à Dieu : Que ta colère ne s'enflamme pas contre moi, et je ne parlerai plus que cette fois : Je te prie, je voudrais seulement faire encore une épreuve avec la toison : Que la toison seule reste sèche et que tout le terrain se couvre de rosée.
\VS{40}Et Dieu fit ainsi cette nuit-là. La toison seule resta sèche, et tout le terrain se couvrit de rosée.
\Chap{7}
\TextTitle{Yahweh sélectionne un petit nombre pour le combat}
\VerseOne{}Jerubbaal qui est Gédéon, et tout le peuple qui était avec lui, se levèrent de bon matin et campèrent près de la source de Harod. Le camp de Madian était au nord, vers la colline de Moré, dans la vallée.
\VS{2}Yahweh dit à Gédéon : Le peuple qui est avec toi est trop nombreux pour que je livre Madian entre ses mains, de peur qu'Israël ne se glorifie contre moi, en disant : C'est ma main qui m'a délivré.
\VS{3}Maintenant donc fais publier ceci aux oreilles du peuple, et qu'on dise : Que celui qui est craintif et qui a peur s'en retourne et s'éloigne de la montagne de Galaad. Vingt-deux mille hommes parmi le peuple s'en retournèrent et il en resta dix mille\FTNT{De. 20:8.}.
\VS{4}Yahweh dit à Gédéon : Le peuple est encore trop nombreux. Fais-les descendre vers l'eau et là je les épurerai\FTNT{C'est Dieu qui qualifie ses ouvriers, il les éprouve et les épure pour les rendre inébranlables. Voir le test de l'épreuve des Hébreux dans le désert de Sinaï (De. 8).} ; et celui dont je te dirai : Que celui-ci aille avec toi, ira avec toi ; et celui dont je te dirai : Que celui-ci n'aille pas avec toi, n'ira pas avec toi.
\VS{5}Il fit donc descendre le peuple vers l'eau ; et Yahweh dit à Gédéon : Tous ceux qui laperont l'eau avec la langue comme lape le chien, tu les sépareras de tous ceux qui se mettront à genoux pour boire.
\VS{6}Ceux qui lapèrent l'eau en la portant à la bouche avec leur main furent au nombre de trois cents hommes et tout le reste du peuple se mit à genoux pour boire.
\VS{7}Alors Yahweh dit à Gédéon : C'est par les trois cents hommes qui ont lapé, que je vous délivrerai et que je livrerai Madian entre tes mains. Que tout le reste du peuple s'en aille donc chacun chez soi.
\VS{8}Ainsi le peuple prit entre ses mains des provisions et ses shofars. Gédéon renvoya tous les hommes d'Israël chacun dans sa tente et il retint les trois cents hommes. Or le camp de Madian était au-dessous de lui, dans la vallée.
\TextTitle{Victoire de Gédéon sur Madian}
\VS{9}Et il arriva cette nuit-là que Yahweh lui dit : Lève-toi, descends au camp car je l'ai livré entre tes mains.
\VS{10}Et si tu crains de descendre, descends vers le camp, toi et Pura, ton serviteur.
\VS{11}Tu écouteras ce qu'ils diront et après cela, tes mains seront fortifiées ; descends donc au camp. Il descendit avec Pura, son serviteur, jusqu'aux avant-postes du camp.
\VS{12}Or Madian, Amalek et tous les fils de l'orient étaient répandus dans la vallée comme des sauterelles, tant il y en avait, et leurs chameaux étaient sans nombre, comme le sable qui est sur le bord de la mer, tant il y en avait\FTNT{Jg. 6:3-33.}.
\VS{13}Gédéon arriva ; et voici, un homme racontait à son compagnon un songe. Il lui disait : Voici, j'ai eu un songe ; il me semblait qu'un gâteau de pain d'orge roulait dans le camp de Madian ; et il est venu heurter jusqu'à la tente et elle est tombée ; il l'a retournée sens dessus dessous et elle a été renversée.
\VS{14}Alors son compagnon répondit et dit : Ce n'est pas autre chose que l'épée de Gédéon, fils de Joas, homme d'Israël ; Dieu a livré Madian et tout le camp entre ses mains.
\VS{15}Lorsque Gédéon eut entendu le récit du songe et son interprétation, il se prosterna, revint au camp d'Israël et dit : Levez-vous car Yahweh a livré le camp de Madian entre vos mains.
\VS{16}Puis il divisa les trois cents hommes en trois corps et il leur donna à chacun des shofars à la main et des cruches vides, avec des flambeaux dans les cruches.
\VS{17}Il leur dit : Regardez-moi et faites comme je ferai. Dès que je serai arrivé à l'extrémité du camp, vous ferez comme je ferai.
\VS{18}Quand je sonnerai du shofar, moi et tous ceux qui sont avec moi, alors vous sonnerez aussi du shofar tout autour du camp et vous direz : L'épée de Yahweh et de Gédéon !
\VS{19}Gédéon et les cent hommes qui étaient avec lui arrivèrent à l'extrémité du camp, au commencement de la veille de la nuit, comme on venait de placer les gardes. Ils sonnèrent du shofar et brisèrent les cruches qu'ils avaient à la main.
\VS{20}Ainsi les trois corps sonnèrent du shofar, et brisèrent les cruches ; ils saisirent de la main gauche les flambeaux et de la main droite les shofars pour sonner et ils s'écrièrent : L'épée de Yahweh et de Gédéon !
\VS{21}Ils restèrent chacun à sa place autour du camp, et tout le camp se mit à courir ça et là, à pousser des cris et à prendre la fuite.
\VS{22}Car comme les trois cents hommes sonnèrent encore du shofar, Yahweh leur fit tourner l'épée les uns contre les autres. Le camp s'enfuit jusqu'à Beth-Schitta, vers Tseréra, jusqu'au bord d'Abel-Mehola, près de Tabbath\FTNT{1 S. 14:20 ; Ez. 38:21.}.
\VS{23}Les hommes d'Israël, à savoir ceux de Nephthali, d'Aser et de tout Manassé, se rassemblèrent et ils poursuivirent Madian.
\VS{24}Alors Gédéon envoya des messagers dans toute la montagne d'Ephraïm pour leur dire : Descendez pour aller à la rencontre de Madian, et coupez-leur les premiers le passage des eaux jusqu'à Beth-Bara et celui du Jourdain. Tous les hommes d'Ephraïm se rassemblèrent, et ils s'emparèrent du passage des eaux jusqu'à Beth-Bara et de celui du Jourdain.
\VS{25}Ils saisirent deux des chefs de Madian, Oreb et Zeeb ; ils tuèrent Oreb au rocher d'Oreb, et ils tuèrent Zeeb au pressoir de Zeeb. Ils poursuivirent Madian, et ils apportèrent les têtes d'Oreb et de Zeeb à Gédéon de l'autre côté du Jourdain\FTNT{Es. 10:26 ; Ps. 83:11-12.}.
\Chap{8}
\TextTitle{Poursuite de Zébach et Tsalmunna ; exécution des rois de Madian}
\VerseOne{}Alors les hommes d'Ephraïm dirent à Gédéon : Que signifie cette manière d'agir envers nous ? Pourquoi ne pas nous avoir appelés quand tu es allé à la guerre contre Madian ? Et ils s'emportèrent fortement contre lui\FTNT{Jg. 12:1.}.
\VS{2}Et il leur répondit : Qu'ai-je fait maintenant au prix de ce que vous avez fait ? Les grappillages d'Ephraïm ne sont-ils pas meilleurs que la vendange d'Abiézer ?
\VS{3}Dieu a livré entre vos mains les chefs de Madian, Oreb et Zeeb. Qu'ai-je pu faire au prix de ce que vous avez fait ? Et leur esprit fut apaisé envers lui lorsqu'il eut ainsi parlé.
\VS{4}Gédéon arriva au Jourdain et il le passa, lui et les trois cents hommes qui étaient avec lui, fatigués, mais poursuivant toujours l'ennemi.
\VS{5}C'est pourquoi il dit aux gens de Succoth : Donnez, je vous prie, quelques pains aux hommes qui m'accompagnent car ils sont fatigués et ainsi, je poursuivrai Zébach et Tsalmunna, rois de Madian.
\VS{6}Mais les chefs de Succoth répondirent : La main de Zébach et celle de Tsalmunna sont-elles déjà en ton pouvoir, pour que nous donnions du pain à ton armée ?
\VS{7}Et Gédéon dit : Eh bien ! Quand Yahweh aura livré Zébach et Tsalmunna entre mes mains, je foulerai au pied votre chair avec des épines du désert et avec des chardons.
\VS{8}Puis de là, il monta à Penuel et il fit la même demande aux gens de Penuel. Les gens de Penuel lui répondirent comme avaient répondu ceux de Succoth.
\VS{9}Et il dit aussi aux gens de Penuel : Quand je reviendrai en paix, je démolirai cette tour.
\VS{10}Zébach et Tsalmunna étaient à Karkor et leurs armées avec eux, environ quinze mille hommes, tous ceux qui étaient restés de l'armée entière des fils de l'orient ; cent vingt mille hommes tirant l'épée avaient été tués.
\VS{11}Gédéon monta par le chemin de ceux qui habitent sous les tentes, à l'orient de Nobach et de Jogbeha, et il battit l'armée qui se croyait en sûreté.
\VS{12}Et comme Zébach et Tsalmunna s'enfuyaient, il les poursuivit et prit les deux rois de Madian, Zébach et Tsalmunna, et mit en déroute toute l'armée\FTNT{Ps. 83:11-12.}.
\TextTitle{Vengeance sur Succoth et Penuel ; exécution de Zébach et Tsalmunna}
\VS{13}Puis Gédéon, fils de Joas, revint de la bataille par la montée de Hérès.
\VS{14}Il saisit un garçon d'entre les hommes de Succoth, il l'interrogea et ce garçon lui donna par écrit le nom des chefs et des anciens de Succoth, au nombre de soixante-dix-sept hommes.
\VS{15}Et il vint auprès des gens de Succoth et leur dit : Voici Zébach et Tsalmunna, au sujet desquels vous m'avez insulté en disant : La main de Zébach et celle de Tsalmunna sont-elles déjà en ton pouvoir pour que nous donnions du pain à tes hommes fatigués ?
\VS{16}Il prit donc les anciens de la ville et châtia les hommes de Succoth avec des épines du désert et des chardons.
\VS{17}Il démolit la tour de Penuel et tua les gens de la ville.
\VS{18}Puis il dit à Zébach et à Tsalmunna : Comment étaient les hommes que vous avez tués à Thabor ? Ils répondirent : Ils étaient entièrement comme toi, chacun d'eux avait l'air d'un fils de roi.
\VS{19}Il leur dit : C'étaient mes frères, fils de ma mère. Yahweh est vivant, si vous les aviez laissés vivre, je ne vous tuerais pas.
\VS{20}Puis il dit à Jéther, son premier-né : Lève-toi, tue-les ! Mais le jeune garçon ne tira pas son épée car il avait peur, parce qu'il était encore un enfant.
\VS{21}Et Zébach et Tsalmunna dirent : Lève-toi toi-même et jette-toi sur nous ! Car tel est l'homme, telle est sa force. Et Gédéon se leva et tua Zébach et Tsalmunna. Il prit ensuite les croissants qui étaient aux cous de leurs chameaux.
\TextTitle{Gédéon recommande au peuple le règne de Yahweh}
\VS{22}Les hommes d'Israël dirent tous d'un commun accord à Gédéon : Domine sur nous, tant toi que ton fils, et le fils de ton fils, car tu nous as délivrés de la main de Madian.
\VS{23}Gédéon leur répondit : Je ne dominerai pas sur vous et mon fils ne dominera pas sur vous ; c'est Yahweh qui dominera sur vous\FTNT{De. 17:15.}.
\TextTitle{Gédéon introduit une occasion de chute en Israël}
\VS{24}Mais Gédéon leur dit : J'ai une demande à vous faire : Donnez-moi chacun les anneaux que vous avez eus pour butin. Les ennemis avaient des anneaux d'or car ils étaient Ismaélites.
\VS{25}Ils répondirent : Nous les donnerons volontiers. Et ils étendirent un manteau sur lequel chacun jeta les anneaux de son butin.
\VS{26}Le poids des anneaux d'or que Gédéon demanda fut de mille sept cents sicles d'or, sans les croissants, les pendants d'oreilles, et les vêtements d'écarlate que portaient les rois de Madian, et sans les colliers qui étaient aux cous de leurs chameaux.
\VS{27}Puis Gédéon en fit un éphod\FTNT{Sous Moïse, il y avait deux sortes d'éphods, le premier était de simple lin pour les sacrificateurs, et le deuxième de broderie pour le souverain sacrificateur. Comme celui des simples sacrificateurs n'avait rien de particulier, Moïse ne s'est pas arrêté à le décrire. Mais il décrit longuement celui du souverain sacrificateur. (Ex. 28:6-9). Il était composé d'or, d'hyacinthe, de pourpre, de cramoisi, de coton retors ; c'était un tissu de différentes couleurs. Il y avait à l'endroit de l'éphod qui venait sur les deux épaules du souverain sacrificateur, deux grosses pierres précieuses, qui étaient chargées du nom des douze tribus d'Israël, six noms sur chaque pierre. A l'endroit où l'éphod se croisait sur la poitrine du grand prêtre, il y avait un ornement carré, nommé le rational, en hébreu « choschen », dans lequel étaient enchâssées douze pierres précieuses, où l'on avait gravé les noms des douze tribus d'Israël ; un sur chacune des pierres.}, et le mit dans sa ville, à Ophra, où il devint un objet de prostitution pour tout Israël ; il fut un piège pour Gédéon et pour sa maison.
\TextTitle{Fin de la vie de Gédéon ; rechute d'Israël après sa mort}
\VS{28}Ainsi Madian fut humilié devant les enfants d'Israël et il ne leva plus la tête. Le pays fut en repos pendant quarante ans, durant les jours de Gédéon.
\VS{29}Jerubbaal, fils de Joas s'en retourna dans sa ville et demeura dans sa maison.
\VS{30}Gédéon eut soixante-dix fils, issus de ses reins, car il eut plusieurs femmes.
\VS{31}Sa concubine, qui était à Sichem, lui enfanta aussi un fils et il lui donna le nom d'Abimélec.
\VS{32}Puis Gédéon, fils de Joas, mourut après une heureuse vieillesse ; et il fut enseveli dans le sépulcre de Joas, son père, à Ophra, qui appartenait à la famille d'Abiézer.
\TextTitle{Retour à l'idolâtrie}
\VS{33}Et il arriva après que Gédéon fut mort, que les enfants d'Israël se détournèrent et se prostituèrent aux Baals, et ils établirent Baal-Berith pour leur dieu\FTNT{Jg. 2:11-17 ; 10:6.}.
\VS{34}Ainsi les enfants d'Israël ne se souvinrent pas de Yahweh, leur Dieu, qui les avait délivrés de la main de tous leurs ennemis qui les entouraient.
\VS{35}Et ils n'usèrent d'aucune loyauté envers la maison de Jerubbaal, de Gédéon, après tout le bien qu'il avait fait à Israël.
\Chap{9}
\TextTitle{Conspiration d'Abimélec pour régner sur Israël}
\VerseOne{}Et Abimélec, fils de Jerubbaal, s'en alla à Sichem vers les frères de sa mère et leur parla, ainsi qu'à toute la maison du père de sa mère :
\VS{2}Je vous prie, faites entendre ces paroles à tous les seigneurs de Sichem : Lequel vous semble le meilleur, que soixante-dix hommes, tous fils de Jerubbaal dominent sur vous, ou qu'un seul homme domine sur vous ? Et souvenez-vous que je suis votre os et votre chair\FTNT{Ge. 29:14.}.
\VS{3}Les frères de sa mère dirent de sa part toutes ces paroles aux oreilles de tous les seigneurs de Sichem et leur cœur se tourna après Abimélec, car ils disaient : C'est notre frère.
\VS{4}Ils lui donnèrent soixante-dix sicles d'argent de la maison de Baal-Berith. Abimélec s'en servit pour acheter des hommes misérables et turbulents, qui allèrent après lui.
\VS{5}Et il vint dans la maison de son père à Ophra et tua sur une seule pierre ses frères, fils de Jerubbaal, qui étaient soixante-dix hommes. Il ne resta que Jotham, le plus jeune fils de Jerubbaal, parce qu'il s'était caché.
\VS{6}Et tous les seigneurs de Sichem s'assemblèrent avec toute la maison de Millo ; ils vinrent et firent d'Abimélec leur roi près du chêne à Sichem.
\VS{7}On le rapporta à Jotham qui alla se tenir au sommet de la montagne de Garizim et les appelant, il dit en élevant la voix : Écoutez-moi, seigneurs de Sichem, et que Dieu vous entende !
\VS{8}Les arbres allèrent pour oindre un roi et ils dirent à l'olivier : Règne sur nous.
\VS{9}Mais l'olivier leur répondit : Renoncerai-je à mon huile par laquelle Dieu et les hommes sont honorés, pour aller m'agiter sur les arbres\FTNT{Ps. 104:15.} ?
\VS{10}Puis les arbres dirent au figuier : Viens, toi, règne sur nous.
\VS{11}Mais le figuier leur répondit : Renoncerai-je à ma douceur et à mon bon fruit, pour aller m'agiter sur les arbres ?
\VS{12}Puis les arbres dirent à la vigne : Viens, toi, et règne sur nous.
\VS{13}Mais la vigne répondit : Renoncerai-je à mon vin qui réjouit Dieu et les hommes, pour aller m'agiter sur les arbres ?
\VS{14}Alors tous les arbres dirent à l'épine : Viens, toi, et règne sur nous.
\VS{15}Et l'épine répondit aux arbres : Si c'est en vérité que vous m'oignez pour roi, venez, et réfugiez-vous sous mon ombrage ; sinon, que le feu sorte de l'épine et qu'il dévore les cèdres du Liban.
\VS{16}Maintenant donc, est-ce en vérité et avec intégrité que vous avez agi en établissant Abimélec pour roi ? Avez-vous bien fait envers Jerubbaal et sa maison ? L'avez-vous fait selon les bienfaits qu'il a rendus de sa main ?
\VS{17}Car mon père a combattu pour vous, il a exposé sa vie devant vous et vous a délivrés de la main de Madian ;
\VS{18}mais vous vous êtes levés aujourd'hui contre la maison de mon père et avez tué sur une pierre ses fils, soixante-dix hommes, et avez établi pour roi Abimélec, fils de sa servante, sur les habitants de Sichem, parce qu'il est votre frère.
\VS{19}Si, dis-je, vous avez agi aujourd'hui en vérité et avec intégrité envers Jerubbaal et sa maison, réjouissez-vous d'Abimélec et qu'il se réjouisse aussi de vous !
\VS{20}Sinon, que le feu sorte d'Abimélec et qu'il dévore les seigneurs de Sichem, et la maison de Millo ; et que le feu sorte des seigneurs de Sichem, et de la maison de Millo, et qu'il dévore Abimélec !
\VS{21}Puis Jotham s'enfuit rapidement ; il s'en alla à Beer, où il demeura loin d'Abimélec, son frère.
\TextTitle{Sichem se retourne contre Abimélec}
\VS{22}Abimélec gouverna sur Israël durant trois ans.
\VS{23}Alors Dieu envoya un mauvais esprit entre Abimélec et les seigneurs de Sichem, et les seigneurs de Sichem furent infidèles à Abimélec.
\VS{24}Afin que la violence faite aux soixante-dix fils de Jerubbaal vienne et que leur sang se tourne contre Abimélec, leur frère, qui les avait tués, et sur les seigneurs de Sichem, qui l'avaient aidé par leur main à tuer ses frères.
\VS{25}Les seigneurs de Sichem mirent des embûches sur le sommet des montagnes, des gens pillaient tous ceux qui passaient près d'eux sur le chemin. Cela fut rapporté à Abimélec.
\VS{26}Alors Gaal, fils d'Ebed, vint avec ses frères, et ils passèrent à Sichem. Les seigneurs de Sichem eurent confiance en lui.
\VS{27}Puis étant sortis aux champs, ils vendangèrent leurs vignes, foulèrent les raisins et se livrèrent à des réjouissances ; ils entrèrent dans la maison de leur dieu, ils mangèrent et burent, et ils maudirent Abimélec.
\VS{28}Alors Gaal, fils d'Ebed, dit : Qui est Abimélec, et qui est Sichem pour que nous servions Abimélec ? N'est-il pas le fils de Jerubbaal et Zebul, n'est-il pas son commissaire ? Servez plutôt les hommes de Hamor, père de Sichem ; mais pour quelle raison servirions-nous Abimélec ?
\VS{29}Plaise à Dieu ! Qu'on mette ce peuple sous mon pouvoir et je chasserais Abimélec. Et il disait d'Abimélec : Multiplie ton armée et sors !
\VS{30}Zebul, gouverneur de la ville, entendit les paroles de Gaal, fils d'Ebed, et sa colère s'enflamma.
\VS{31}Puis il envoya astucieusement des messagers vers Abimélec pour lui dire : Voici, Gaal, fils d'Ebed, et ses frères sont entrés dans Sichem, et voici, ils assiègent la ville contre toi.
\VS{32}Maintenant donc, lève-toi de nuit, toi et le peuple qui est avec toi, et mets-toi en embuscade dans les champs.
\VS{33}Et le matin, au lever du soleil, tu te lèveras et tu te jetteras sur la ville. Gaal et le peuple qui est avec lui sortiront contre toi, ta main lui fera selon les forces que tu trouveras.
\VS{34}Abimélec et tout le peuple qui était avec lui se levèrent de nuit, et ils se mirent en embuscade contre Sichem, divisés en quatre bandes.
\VS{35}Alors Gaal, fils d'Ebed, sortit, et il se tint à l'entrée de la porte de la ville. Abimélec et tout le peuple qui était avec lui se levèrent de l'embuscade.
\VS{36}Gaal voyant le peuple, dit à Zebul : Voici un peuple qui descend du sommet des montagnes. Zebul lui dit : Tu vois l'ombre des montagnes comme des hommes.
\VS{37}Gaal, parla encore et dit : C'est bien un peuple qui descend des hauteurs du pays et une bande vient du chemin du chêne des devins.
\VS{38}Et Zebul lui dit : Où est donc ta bouche, toi qui disais : Qui est Abimélec, pour que nous le servions ? N'est-ce pas ici ce peuple que tu méprisais ? Sors maintenant, je te prie, et combats !
\VS{39}Alors, Gaal sortit conduisant les seigneurs de Sichem et combattit contre Abimélec.
\VS{40}Abimélec le poursuivit et il s'enfuit de devant lui, et plusieurs tombèrent morts jusqu'à l'entrée de la porte.
\VS{41}Abimélec s'arrêta à Aruma. Zebul repoussa Gaal et ses frères afin qu'ils ne restent plus à Sichem.
\VS{42}Et il arriva, dès le lendemain, que le peuple sortit aux champs. Cela fut rapporté à Abimélec,
\VS{43}qui prit son peuple, et le divisa en trois bandes, et les mit en embuscade dans les champs. Ayant vu que le peuple sortait de la ville, il se leva contre eux, et les battit.
\VS{44}Abimélec et la bande qui était avec lui se répandirent, et se tinrent à l'entrée de la porte de la ville ; mais les deux autres bandes se jetèrent sur tous ceux qui étaient aux champs, et les battirent.
\VS{45}Ainsi Abimélec combattit contre la ville toute la journée ; il prit la ville et tua le peuple qui y était. Il la rasa et y sema du sel.
\VS{46}Ayant appris cela, tous les seigneurs de la tour de Sichem entrèrent dans la forteresse de la maison du dieu Berith\FTNT{Jg. 9:4 ; 8:33.}.
\VS{47}On rapporta à Abimélec que tous les seigneurs de la tour de Sichem s'étaient assemblés dans la forteresse.
\VS{48}Alors Abimélec monta sur la montagne de Tsalmon, lui et tout le peuple qui était avec lui. Il prit en main une hache, coupa une branche d'arbre, et l'ayant mise sur son épaule, la porta et dit au peuple qui était avec lui : Avez-vous vu ce que j'ai fait ? Hâtez-vous de faire comme moi.
\VS{49}Chacun donc de tout le peuple coupa une branche et ils marchèrent derrière Abimélec ; ils mirent ces branches tout autour de la forteresse et y mirent le feu. Il brûlèrent la forteresse, et toutes les personnes de la tour de Sichem moururent ; au nombre d'environ mille, tant hommes que femmes.
\TextTitle{Abimélec meurt}
\VS{50}Puis Abimélec marcha contre Thébets, y mit son camp et la prit.
\VS{51}Il y avait au milieu de la ville une forte tour, où s'enfuirent tous les hommes et toutes les femmes, et tous les seigneurs de la ville, et ayant fermé les portes après eux, ils montèrent sur le toit de la tour.
\VS{52}Alors Abimélec alla jusqu'à la tour, l'attaqua et s'approcha jusqu'à la porte pour la brûler par le feu.
\VS{53}Mais une femme jeta une pièce de meule de moulin sur la tête d'Abimélec et lui brisa le crâne\FTNT{2 S. 11:21.}.
\VS{54}Rapidement, il appela le garçon qui portait ses armes et lui dit : Tire ton épée, et tue-moi, de peur qu'on ne dise de moi : C'est une femme qui l'a tué. Le garçon le transperça et il mourut\FTNT{1 S. 31:4.}.
\VS{55}Quand les hommes d'Israël virent qu'Abimélec était mort, ils s'en allèrent chacun en son lieu.
\VS{56}Ainsi Dieu rendit à Abimélec le mal qu'il avait fait contre son père, en tuant ses soixante-dix frères,
\VS{57}et toute la méchanceté des hommes de Sichem ; Dieu, dis-je, la fit retourner sur leurs têtes ; et ainsi la malédiction de Jotham, fils de Jérubbaal, vint sur eux.
\Chap{10}
\TextTitle{Thola, juge en Israël}
\VerseOne{}Après Abimélec, Thola fils de Pua, fils de Dodo, homme d'Issacar, se leva pour délivrer Israël ; il habitait à Schamir, dans la montagne d'Ephraïm.
\VS{2}Il fut juge en Israël pendant vingt-trois ans ; puis il mourut, et fut enterré à Schamir.
\TextTitle{Jaïr, juge en Israël}
\VS{3}Après lui, se leva Jaïr, le Galaadite, qui fut juge en Israël pendant vingt-deux ans.
\VS{4}Il avait trente fils qui montaient sur trente ânons et qui avaient trente villes qu'on appelle jusqu'à ce jour, bourgs de Jaïr, lesquelles sont situées au pays de Galaad.
\VS{5}Et Jaïr mourut, et fut enterré à Kamon.
\TextTitle{Idolâtrie d'Israël et oppression par ses ennemis}
\VS{6}Puis les enfants d'Israël firent encore ce qui est mal aux yeux de Yahweh ; et servirent les Baals et les Astartés, les dieux de Syrie, les dieux de Sidon, les dieux de Moab, les dieux des fils d'Ammon et les dieux des Philistins, et ils abandonnèrent Yahweh, et ne le servirent plus\FTNT{Jg. 2:11 ; 3:7 ; 8:33.}.
\VS{7}Alors la colère de Yahweh s'enflamma contre Israël, et il les vendit entre les mains des Philistins et des fils d'Ammon.
\VS{8}Ils opprimèrent et écrasèrent les enfants d'Israël cette année-là, et pendant dix-huit ans tous les enfants d'Israël qui étaient au-delà du Jourdain, au pays des Amoréens en Galaad.
\VS{9}Même les fils d'Ammon passèrent le Jourdain pour combattre contre Juda, contre Benjamin et contre la maison d'Ephraïm. Israël fut dans une grande détresse.
\VS{10}Alors les enfants d'Israël crièrent à Yahweh, en disant : Nous avons péché contre toi, et certes, nous avons abandonné notre Dieu et nous avons servi les Baals.
\VS{11}Mais Yahweh répondit aux enfants d'Israël : N'avez-vous pas été opprimés par les Egyptiens, les Amoréens, les fils d'Ammon et les Philistins ?
\VS{12}Et lorsque les Sidoniens, Amalek et Maon, vous opprimèrent, et que vous criâtes à moi, ne vous ai-je pas délivrés de leurs mains ?
\VS{13}Mais vous, vous m'avez abandonné, et vous avez servi d'autres dieux. C'est pourquoi je ne vous délivrerai plus.
\VS{14}Allez et criez vers les dieux que vous avez choisis ; qu'ils vous délivrent au temps de votre détresse !
\VS{15}Mais les enfants d'Israël répondirent à Yahweh : Nous avons péché ; traite-nous comme tu le trouveras bon. Nous te prions seulement que tu nous délivres aujourd'hui !
\VS{16}Alors ils ôtèrent du milieu d'eux les dieux étrangers, et servirent Yahweh, qui fut affligé des souffrances d'Israël.
\VS{17}Les fils d'Ammon se rassemblèrent et campèrent en Galaad, et les enfants d'Israël se rassemblèrent et campèrent à Mitspa.
\VS{18}Le peuple, les chefs de Galaad se dirent l'un à l'autre : Qui sera l'homme qui commencera à combattre contre les fils d'Ammon ? Il sera chef de tous les habitants de Galaad.
\Chap{11}
\TextTitle{Jephté, juge en Israël}
\VerseOne{}Or Jephthé, le Galaadite, était un fort et vaillant homme. Il était le fils d'une femme prostituée ; et c'est Galaad qui l'avait engendré.
\VS{2}La femme de Galaad lui enfanta des fils ; et quand les fils de cette femme furent grands, ils chassèrent Jephthé, en lui disant : Tu n'auras pas d'héritage dans la maison de notre père, car tu es fils d'une autre femme.
\VS{3}Jephthé s'enfuit donc de devant ses frères, et habita au pays de Tob. Des misérables se rassemblèrent auprès de Jephthé, et ils sortirent dehors avec lui\FTNT{Jg. 9:4 ; 1 S. 22:2.}.
\VS{4}Et il arriva, quelque temps après, que les fils d'Ammon firent la guerre à Israël.
\VS{5}Et comme les fils d'Ammon faisaient la guerre à Israël, les anciens de Galaad s'en allèrent pour emmener Jephthé du pays de Tob.
\VS{6}Ils dirent à Jephthé : Viens, et sois notre chef, afin que nous combattions contre les fils d'Ammon.
\VS{7}Jephthé répondit aux anciens de Galaad : N'est-ce pas vous qui m'avez haï et chassé de la maison de mon père ? Pourquoi êtes-vous venus à moi maintenant que vous êtes dans la détresse ?
\VS{8}Alors les anciens de Galaad dirent à Jephthé : La raison pour laquelle nous retournons à toi maintenant, c'est afin que tu viennes avec nous, que tu combattes contre les fils d'Ammon, et que tu sois notre chef, celui de tous les habitants de Galaad.
\VS{9}Jephthé répondit aux anciens de Galaad : Si vous me ramenez pour combattre contre les fils d'Ammon et que Yahweh les livre devant moi, je serai votre chef.
\VS{10}Les anciens de Galaad dirent à Jephthé : Que Yahweh nous entende et qu'il juge, si nous ne faisons pas ce que tu dis.
\VS{11}Jephthé donc s'en alla avec les anciens de Galaad. Le peuple le mit à sa tête et l'établit pour chef, et Jephthé déclara devant Yahweh, à Mitspa, toutes les paroles qu'il avait dites.
\VS{12}Puis Jephthé envoya des messagers au roi des fils d'Ammon pour lui dire : Qu'y a-t-il entre toi et moi, que tu viennes contre moi pour faire la guerre à mon pays ?
\VS{13}Le roi des fils d'Ammon répondit aux messagers de Jephthé : C'est parce qu'Israël a pris mon pays quand il est monté d'Egypte, depuis l'Arnon jusqu'à Jabbok, et même jusqu'au Jourdain. Maintenant rends-le de bon gré.
\VS{14}Mais Jephthé envoya encore des messagers au roi des fils d'Ammon,
\VS{15}qui lui dirent : Ainsi parle Jephthé : Israël n'a rien pris du pays de Moab, ni du pays des fils d'Ammon.
\VS{16}Mais lorsqu'Israël est monté d'Egypte, il est venu par le désert jusqu'à la Mer Rouge et il a atteint Kadès.
\VS{17}Alors Israël envoya des messagers au roi d'Edom, pour lui dire : Que je passe, je te prie, par ton pays. Le roi d'Edom ne voulut pas l'entendre. Il en envoya aussi au roi de Moab, qui ne voulut pas non plus l'entendre. Et Israël demeura à Kadès.
\VS{18}Puis il marcha par le désert, tourna le pays d'Edom et le pays de Moab, et vint à l'orient du pays de Moab ; il campa au-delà de l'Arnon, et n'entra pas sur les frontières de Moab, car l'Arnon était la frontière de Moab.
\VS{19}Mais Israël envoya des messagers à Sihon, roi des Amoréens, roi de Hesbon, auquel Israël dit : Laisse-nous passer par ton pays jusqu'au lieu où nous allons.
\VS{20}Mais Sihon n'eut pas assez confiance en Israël pour le laisser passer sur son territoire ; il rassembla tout son peuple, ils campèrent vers Jahats, et combattirent contre Israël.
\VS{21}Et Yahweh, le Dieu d'Israël, livra Sihon et tout son peuple entre les mains d'Israël, qui les battit. Israël prit possession de tout le pays des Amoréens qui habitaient cette terre.
\VS{22}Ils conquirent donc tout le pays des Amoréens, depuis l'Arnon jusqu'à Jabbok, et depuis le désert jusqu'au Jourdain.
\VS{23}Et maintenant que Yahweh, le Dieu d'Israël, a dépossédé les Amoréens de devant son peuple d'Israël, aurais-tu la possession de leur pays ?
\VS{24}Ce que ton dieu Kemosch te donne à posséder, ne le posséderais-tu pas ? Et tout ce que Yahweh, notre Dieu, a mis en notre possession devant nous, nous ne le posséderions pas !
\VS{25}Or maintenant vaux-tu mieux en quelque sorte que ce Balak, fils de Tsippor, roi de Moab ? A-t-il contesté et combattu contre Israël ?
\VS{26}Voilà trois cents ans qu'Israël demeure à Hesbon, et dans les villes de son ressort, à Aroër, et dans les villes de son ressort, et dans toutes les villes qui sont le long de l'Arnon : Pourquoi ne les avez-vous pas saisies pendant ce temps-là ?
\VS{27}Je ne t'ai pas offensé mais tu fais mal de me faire la guerre. Que Yahweh, qui est le juge, juge aujourd'hui entre les enfants d'Israël et les fils d'Ammon !
\VS{28}Le roi des fils d'Ammon n'écouta pas les paroles que Jephthé lui fit dire.
\VS{29}L'Esprit de Yahweh fut sur Jephthé. Il passa au travers de Galaad et de Manassé ; il passa jusqu'à Mitspé de Galaad, et de Mitspé de Galaad, il passa jusqu'aux fils d'Ammon.
\TextTitle{Jephté fait un vœu ; Ammon livré entre ses mains}
\VS{30}Jephthé fit un vœu à Yahweh, et dit : Si tu livres les fils d'Ammon entre mes mains,
\VS{31}alors tout ce qui sortira des portes de ma maison au-devant de moi, quand je retournerai en paix chez les fils d'Ammon, sera consacré à Yahweh et je l'offrirai en holocauste.
\VS{32}Jephthé passa jusqu'où étaient les fils d'Ammon, et Yahweh les livra entre ses mains.
\VS{33}Il les battit par une grande défaite, depuis Aroër jusqu'à Minnith, espace qui renfermait vingt villes et jusqu'à Abel-Keramim. Et les fils d'Ammon furent humiliés devant les fils d'Israël.
\VS{34}Puis comme Jephthé retourna à Mitspa dans sa maison, voici, sa fille, qui était seule et unique, sans qu'il eût d'autres fils ou filles, sortit au-devant de lui avec des tambourins et des danses.
\VS{35}Et il arriva qu'aussitôt qu'il l'eut aperçue, il déchira ses vêtements, et dit : Ha ! Ma fille ! Tu m'as entièrement abaissé, tu es du nombre de ceux qui me troublent ! J'ai ouvert ma bouche à Yahweh, et je ne puis le révoquer.
\VS{36}Elle répondit : Mon père, si tu as ouvert ta bouche à Yahweh, fais-moi selon ce qui est sorti de ta bouche, puisque Yahweh t'a fait vengeance de tes ennemis, des fils d'Ammon.
\VS{37}Toutefois, elle dit à son père : Que ceci me soit fait : Laisse-moi pendant deux mois ! Je m'en irai, je descendrai par les montagnes, et je pleurerai ma virginité avec mes compagnes.
\VS{38}Il répondit : Va ! Et il la laissa aller pour deux mois. Elle s'en alla donc avec ses compagnes, et pleura sa virginité dans les montagnes.
\VS{39}Et au bout de deux mois, elle retourna vers son père ; et il lui fit selon le vœu qu'il avait fait\FTNT{Yahweh interdit les sacrifices humains (Lé. 18:21 ; Lé. 20:2-5 ; Lé. 21 ; De. 12:31 ; De. 18:10).}. Elle n'avait pas connu d'homme. Dès lors, ce fut une coutume en Israël,
\VS{40}tous les ans, les filles d'Israël allaient pour célébrer la fille de Jephthé, le Galaadite, quatre jours par an.
\Chap{12}
\TextTitle{Querelle entre Jephthé et Ephraïm}
\VerseOne{}Or les hommes d'Ephraïm se rassemblèrent, passèrent par le nord et dirent à Jephthé : Pourquoi es-tu passé pour combattre contre les fils d'Ammon, sans nous avoir appelés pour aller avec toi ? Nous brûlerons ta maison, et toi aussi\FTNT{Jg. 8:1.}.
\VS{2}Et Jephthé leur dit : J'ai eu un grand différend avec les fils d'Ammon, moi et mon peuple, et quand je vous ai appelés, vous ne m'avez point délivré de leurs mains.
\VS{3}Voyant que vous ne me délivriez pas, j'ai exposé ma vie, et je suis passé jusqu'où étaient les fils d'Ammon. Yahweh les a livrés entre mes mains. Pourquoi donc aujourd'hui montez-vous vers moi pour me faire la guerre ?
\VS{4}Puis Jephthé assembla tous les hommes de Galaad, et combattit contre Ephraïm. Les hommes de Galaad battirent Ephraïm, parce qu'ils disaient : Vous êtes des fugitifs d'Ephraïm ! Galaad est au milieu d'Ephraïm, au milieu de Manassé !
\VS{5}Les Galaadites se saisirent des gués du Jourdain du côté d'Ephraïm. Et quand l'un des fuyards d'Ephraïm disait : Que je passe ! Les hommes de Galaad lui disaient : Es-tu Ephraïmite ? Il répondait : Non.
\VS{6}Alors ils lui disaient : Dis un peu « Schibboleth ». Et il disait « Sibboleth », car il ne pouvait pas le prononcer. Sur quoi, se saisissant de lui, ils le tuaient aux gués du Jourdain. En ce temps-là quarante-deux mille hommes d'Ephraïm périrent.
\VS{7}Jephthé fut juge en Israël pendant six ans ; puis Jephthé le Galaadite mourut, et fut enterré dans l'une des villes de Galaad.
\TextTitle{Ibstan, juge en Israël}
\VS{8}Après lui, Ibtsan de Bethléhem fut juge en Israël.
\VS{9}Il eut trente fils, il envoya trente filles au-dehors, et il fit venir du dehors trente filles pour ses fils. Il fut juge en Israël pendant sept ans.
\VS{10}Puis Ibtsan mourut, et fut enterré à Bethléhem.
\TextTitle{Elon, juge en Israël}
\VS{11}Après lui, Elon de Zabulon fut juge en Israël pendant dix ans.
\VS{12}Puis Elon de Zabulon mourut, et fut enterré à Ajalon, dans le pays de Zabulon.
\TextTitle{Abdon, juge en Israël}
\VS{13}Après lui, Abdon fils d'Hillel, le Pirathonite, fut juge en Israël.
\VS{14}Il eut quarante fils et trente petits-fils, qui montaient sur soixante-dix ânons. Il fut juge en Israël pendant huit ans\FTNT{Jg. 10:4.}.
\VS{15}Puis Abdon, fils d'Hillel, le Pirathonite, mourut et fut enterré à Pirathon, dans le pays d'Ephraïm, sur la montagne des Amalécites.
\Chap{13}
\TextTitle{Yahweh livre Israël aux Philistins}
\VerseOne{}Et les enfants d'Israël recommencèrent à faire ce qui est mal aux yeux de Yahweh ; et Yahweh les livra entre les mains des Philistins, pendant quarante ans.
\TextTitle{Annonce de la naissance de Samson}
\VS{2}Or il y avait un homme de Tsorea, de la famille des Danites, dont le nom était Manoach. Sa femme était stérile, et n'enfantait pas.
\VS{3}L'Ange de Yahweh apparut à la femme, et lui dit : Voici, tu es stérile et tu n'as jamais eu d'enfants ; mais tu concevras et tu enfanteras un fils.
\VS{4}Prends donc bien garde dès maintenant de ne boire ni vin ni liqueur forte, et de ne manger aucune chose impure.
\VS{5}Car voici, tu vas être enceinte et tu enfanteras un fils. Le rasoir ne s'élèvera pas sur sa tête, parce que l'enfant sera Naziréen\FTNT{Naziréen vient du mot « nazir » qui signifie « consacré » ou « séparé ». Voir No. 6.} pour Dieu dès le ventre de sa mère ; et ce sera lui qui commencera à délivrer Israël de la main des Philistins.
\VS{6}Et La femme vint, et parla à son mari en disant : Un homme de Dieu est venu vers moi et il avait l'aspect d'un ange de Dieu, un aspect fort redoutable. Je ne lui ai pas demandé d'où il était et il ne m'a pas déclaré son nom.
\VS{7}Mais il m'a dit : Tu vas être enceinte et tu enfanteras un fils ; maintenant donc, ne bois ni vin ni liqueur forte, et ne mange aucune chose impure, car cet enfant sera naziréen pour Dieu dès le ventre de sa mère jusqu'au jour de sa mort.
\TextTitle{Prière de Manoach exaucée ; rencontre avec l'Ange de Yahweh}
\VS{8}Et Manoach pria instamment Yahweh, et dit : Hélas, Seigneur ! Que l'homme de Dieu que tu as envoyé vienne encore vers nous, et qu'il nous enseigne ce que nous devons faire à l'enfant quand il naîtra !
\VS{9}Et Dieu exauça la prière de Manoach, et l'Ange de Dieu vint encore vers la femme lorsqu'elle était assise dans un champ ; mais Manoach, son mari, n'était pas avec elle.
\VS{10}Et la femme courut vite le rapporter à son mari, en lui disant : Voici, l'homme qui était venu vers moi l'autre jour m'est apparu.
\VS{11}Manoach se leva, suivit sa femme et venant vers l'homme, il lui dit : Es-tu cet homme qui a parlé à cette femme ? Il répondit : C'est moi.
\VS{12}Manoach dit : Tout ce que tu as dit arrivera, quelle conduite faudra-t-il tenir envers l'enfant, et que lui faudra-t-il faire ?
\VS{13}L'Ange de Yahweh répondit à Manoach : La femme se gardera de tout ce que je lui ai dit.
\VS{14}Elle ne mangera rien qui sorte de la vigne, elle ne boira ni vin ni liqueur forte, et ne mangera aucune chose impure ; elle prendra garde à tout ce que je lui ai ordonné.
\VS{15}Alors Manoach dit à l'Ange de Yahweh : Permets que nous te retenions, et que nous apprêtions un chevreau en ta présence.
\VS{16}Et l'Ange de Yahweh répondit à Manoach : Quand tu me retiendrais, je ne mangerai pas de ton mets ; mais si tu fais un holocauste, tu l'offriras à Yahweh. Manoach ne savait pas que ce fût un Ange de Yahweh.
\VS{17}Et Manoach dit à l'Ange de Yahweh : Quel est ton nom, afin que nous te rendions les honneurs lorsque ta parole viendra ?
\VS{18}Et l'Ange de Yahweh lui répondit : Pourquoi demandes-tu mon nom ? Il est merveilleux.
\VS{19}Alors Manoach prit un chevreau, et une offrande, et les offrit à Yahweh sur le rocher. Il se produisit une chose merveilleuse à la vue de Manoach et de sa femme.
\VS{20}Comme la flamme montait de dessus l'autel vers les cieux, l'Ange de Yahweh monta aussi avec la flamme de l'autel. A cette vue, Manoach et sa femme tombèrent la face contre terre.
\VS{21}L'Ange de Yahweh n'apparut plus à Manoach ni à sa femme. Alors Manoach sut que c'était l'Ange de Yahweh.
\VS{22}Et Manoach dit à sa femme : Certainement nous mourrons, car nous avons vu Dieu.
\VS{23}Mais sa femme lui répondit : Si Yahweh avait voulu nous faire mourir, il n'aurait pas pris de nos mains l'holocauste ni l'offrande, il ne nous aurait pas fait voir toutes ces choses ni fait entendre les choses que nous avons entendues.
\TextTitle{Naissance de Samson}
\VS{24}Puis cette femme enfanta un fils, et elle l'appela du nom de Samson. L'enfant devint grand, et Yahweh le bénit.
\VS{25}Et l'Esprit de Yahweh commença à l'agiter à Machané-Dan\FTNT{qui signifie « camp de Dan »}, entre Tsorea et Eschthaol.
\Chap{14}
\TextTitle{Samson veut épouser une femme philistine, Yahweh au contrôle de la situation}
\VerseOne{}Samson descendit à Thimna, et il y vit une femme d'entre les filles des Philistins.
\VS{2}Etant remonté dans sa maison, il le déclara à son père et à sa mère, en disant : J'ai vu une femme à Thimna d'entre les filles des Philistins ; prenez-la maintenant, afin qu'elle soit ma femme.
\VS{3}Son père et sa mère lui dirent : N'y a-t-il pas de femme parmi les filles de tes frères et parmi tout notre peuple, pour que tu ailles prendre une femme d'entre les Philistins, ces incirconcis ? Et Samson dit à son père : Prenez-la pour moi car elle est droite à mes yeux.
\VS{4}Mais son père et sa mère ne savaient pas que cela venait de Yahweh : Car Samson cherchait une occasion de dispute de la part des Philistins. Or en ce temps-là, les Philistins dominaient sur Israël.
\TextTitle{Festin des noces ; énigme de Samson}
\VS{5}Samson descendit avec son père et sa mère à Thimna. Ils allèrent jusqu'aux vignes de Thimna, et voici, un jeune lion rugissant vint à sa rencontre.
\VS{6}Et l'Esprit de Yahweh saisit Samson ; sans avoir rien en sa main, il déchira le lion comme on déchire un chevreau. Il ne déclara pas à son père ni à sa mère ce qu'il avait fait\FTNT{1 S. 17:34-35.}.
\VS{7}Il descendit et parla à la femme, et elle fut trouvée droite à ses yeux.
\VS{8}Puis quelque temps après, il retourna à Thimna pour la prendre et se détourna pour voir la carcasse du lion. Et voici, il y avait dans la carcasse du lion un essaim d'abeilles et du miel.
\VS{9}Il en prit entre ses mains, et s'en alla en mangeant ; et lorsqu'il fut arrivé vers son père et sa mère, il leur en donna et ils en mangèrent. Mais il ne leur déclara pas qu'il avait pris ce miel dans la carcasse du lion.
\VS{10}Son père descendit chez la femme. Samson fit là un festin ; car c'est ainsi que les jeunes gens faisaient.
\VS{11}Dès qu'on le vit, on prit trente compagnons qui furent avec lui.
\VS{12}Samson leur dit : Je vous propose une énigme. Si vous me l'expliquez au cours des sept jours du festin et si vous la trouvez, je vous donnerai trente chemises et trente vêtements de rechange.
\VS{13}Mais si vous ne pouvez pas me l'expliquer, vous me donnerez trente chemises et trente vêtements de rechange. Ils lui répondirent : Propose ton énigme, et nous l'écouterons.
\VS{14}Et il leur dit : De celui qui mange est sorti ce qui se mange, et du fort est sorti le doux. Pendant trois jours, ils ne purent pas expliquer l'énigme.
\VS{15}Et au septième jour, ils dirent à la femme de Samson : Persuade ton mari de nous expliquer l'énigme ; de peur que nous ne te brûlions au feu, toi et la maison de ton père. C'est pour nous déposséder que vous nous avez appelés ici, n'est-ce pas ?
\VS{16}La femme de Samson pleurait auprès de lui, et disait : Certainement tu me hais,et tu ne m'aimes pas ; tu as proposé une énigme aux enfants de mon peuple, et tu ne me l'as pas expliquée ! Et il lui répondait : Je ne l'ai expliquée ni à mon père ni à ma mère ; est-ce à toi que je l'expliquerais ?
\VS{17}Elle pleura ainsi auprès de lui durant les sept jours du festin ; mais au septième jour il la lui expliqua, parce qu'elle le tourmentait. Puis elle l'expliqua aux enfants de son peuple.
\VS{18}Les gens de la ville lui dirent au septième jour, avant le coucher du soleil : Qu'y a-t-il de plus doux que le miel et qu'y a-t-il de plus fort que le lion ? Et il leur dit : Si vous n'aviez pas labouré avec ma génisse vous n'auriez pas trouvé mon énigme.
\VS{19}L'Esprit de Yahweh le saisit, et il descendit à Askalon. Il tua trente hommes, il prit leurs dépouilles, et donna les vêtements de rechange à ceux qui avaient expliqué l'énigme. Sa colère s'enflamma et il monta à la maison de son père.
\VS{20}Et la femme de Samson fut donnée à son compagnon, avec lequel il était lié.
\Chap{15}
\TextTitle{Querelle entre Samson et les Philistins}
\VerseOne{}Et il arriva quelque jours après, au jour de la moisson des blés, que Samson alla visiter sa femme, et lui porta un chevreau. Il dit : J'entrerai vers ma femme dans sa chambre. Mais le père de sa femme ne lui permit pas d'y entrer.
\VS{2}Car il lui dit : J'ai cru que tu avais de la haine pour elle, c'est pourquoi je l'ai donnée à ton compagnon. Sa jeune sœur n'est-elle pas plus belle qu'elle ? Prends-la donc à sa place.
\VS{3}Samson leur dit : Cette fois je serai innocent à l'égard des Philistins si je leur fais du mal.
\VS{4}Samson s'en alla donc. Il prit trois cents renards, il prit aussi des torches ; puis il tourna les renards queue contre queue, et mit une torche entre les deux queues, au milieu.
\VS{5}Puis il mit le feu aux torches, et lâcha les renards dans les blés des Philistins, et brûla le tas de gerbes, le blé sur pied jusqu'aux plantations d'oliviers.
\VS{6}Les Philistins dirent : Qui a fait cela ? On répondit : Samson, le gendre du Thimnien, parce qu'il lui a pris sa femme et l'a donnée à son compagnon. Les Philistins montèrent, et ils la brûlèrent au feu, elle avec son père.
\VS{7}Alors Samson leur dit : Est-ce donc ainsi que vous faites ? Je ne cesserai qu'après m'être vengé de vous.
\VS{8}Et il les battit dos et ventre, et en fit un grand carnage ; puis il descendit, et demeura dans une caverne du rocher d'Etam.
\TextTitle{Samson livré aux Philistins par Juda ; mille hommes tués avec une mâchoire d'âne}
\VS{9}Alors les Philistins montèrent, campèrent en Juda, et s'étendirent jusqu'à Léchi.
\VS{10}Les hommes de Juda dirent : Pourquoi êtes-vous montés contre nous ? Ils répondirent : Nous sommes montés pour lier Samson, afin que nous lui fassions comme il nous a fait.
\VS{11}Alors trois mille hommes de Juda descendirent à la caverne du rocher d'Etam, et dirent à Samson : Ne sais-tu pas que les Philistins dominent sur nous ? Que nous as-tu donc fait ? Il leur répondit : Je leur ai fait comme ils m'ont fait.
\VS{12}Ils lui dirent : Nous sommes descendus pour te lier, afin de te livrer entre les mains des Philistins. Samson leur dit : Jurez-moi que vous ne me tuerez pas.
\VS{13}Ils lui répondirent, en disant : Non ; mais nous te lierons, afin de te livrer entre leurs mains, mais nous ne te tuerons pas. Ils le lièrent avec deux cordes neuves, et le firent monter hors du rocher.
\VS{14}Lorsqu'il entra à Léchi, les Philistins poussèrent des cris de joie à sa rencontre. Alors l'Esprit de Yahweh le saisit. Les cordes qui étaient sur ses bras devinrent comme du lin brûlé par le feu, et les liens tombèrent de ses mains.
\VS{15}Il trouva une mâchoire d'âne fraîche, il étendit sa main, la prit, et il en tua mille hommes.
\VS{16}Puis Samson dit : Avec une mâchoire d'âne, un monceau, deux monceaux ; avec une mâchoire d'âne, j'ai tué mille hommes.
\VS{17}Quand il cessa de parler, il jeta de sa main la mâchoire. On appela ce lieu Ramath-Léchi.
\VS{18}Il eut extrêmement soif, et invoqua Yahweh en disant : Tu as accordé par la main de ton serviteur cette grande délivrance ; et maintenant mourrais-je de soif, et tomberais-je entre les mains des incirconcis\FTNT{1 S. 17:26.} ?
\VS{19}Alors Dieu fendit un creux qui était dans la mâchoire, et il en sortit de l'eau. Samson but, l'Esprit lui revint, et il reprit vie. C'est pourquoi on a appelé cette source du nom d'En-Hakkoré ; elle existe encore aujourd'hui à Léchi.
\VS{20}Samson fut juge en Israël, au temps des Philistins, pendant vingt ans\FTNT{Jg. 16:31.}.
\Chap{16}
\TextTitle{Samson séduit par Delila}
\VerseOne{}Or Samson s'en alla à Gaza ; il y vit une femme prostituée et il entra chez elle.
\VS{2}On dit aux gens de Gaza : Samson est venu ici. Ils l'entourèrent, et se tinrent en embuscade toute la nuit à la porte de la ville. Ils restèrent tranquilles toute la nuit, en disant : Au point du jour, nous le tuerons.
\VS{3}Samson demeura couché jusqu'à minuit. Au milieu de la nuit, il se leva ; et il saisit les battants des portes de la ville et les deux poteaux, les retira avec la barre, les mit sur ses épaules et les porta sur le sommet de la montagne qui est en face d'Hébron.
\VS{4}Après cela, il aima une femme dans la vallée de Sorek. Elle se nommait Delila.
\VS{5}Les princes des Philistins montèrent vers elle, et lui dirent : Séduis-le, jusqu'à ce que tu saches de lui en quoi consiste sa grande force, et comment pourrions-nous le vaincre ; et nous le lierons pour l'abattre, et nous te donnerons chacun mille cent sicles d'argent.
\VS{6}Delila dit à Samson : Dis-moi, je te prie, en quoi consiste ta grande force, et avec quoi il faudrait te lier pour t'abattre.
\VS{7}Samson lui répondit : Si on me liait avec sept cordes fraîches, qui ne soient pas encore sèches, je deviendrais faible et je serais comme un autre homme.
\VS{8}Les princes des Philistins emmenèrent à Delila sept cordes fraîches, qui n'étaient pas encore sèches. Et elle le lia.
\VS{9}Or il y avait chez elle, dans une chambre, des gens qui se tenaient en embuscade. Elle lui dit : Les Philistins sont sur toi, Samson ! Alors il rompit les cordes comme se romprait un cordon d'étoupe dès qu'il sent le feu. Et l'on ne connut pas d'où lui venait sa force.
\VS{10}Puis Delila dit à Samson : Voici, tu t'es moqué de moi, car tu m'as dit des mensonges. Je te prie, déclare-moi maintenant avec quoi il faut te lier.
\VS{11}Il lui répondit : Si on me liait avec des cordes neuves, dont on ne se serait jamais servi pour un quelconque ouvrage, je deviendrais faible et je serais comme un autre homme.
\VS{12}Delila prit des cordes neuves avec lesquelles elle le lia. Puis elle lui dit : Les Philistins sont sur toi, Samson ! Or il y avait des gens en embuscade dans une chambre. Et il rompit les cordes comme un fil.
\VS{13}Puis Delila dit à Samson : Tu t'es moqué de moi, jusqu'ici tu m'as dit des mensonges. Déclare-moi avec quoi il faut te lier. Il lui dit : Tu n'as qu'à tresser les sept tresses de ma tête avec la chaîne du tissu.
\VS{14}Et elle les fixa par la cheville. Puis elle dit : Les Philistins sont sur toi, Samson ! Alors il se réveilla de son sommeil et il retira la chaîne du tissu.
\TextTitle{Samson révèle le secret de sa force}
\VS{15}Alors elle lui dit : Comment peux-tu dire : Je t'aime ! Puisque ton cœur n'est pas avec moi ? Tu t'es moqué de moi par trois fois et tu ne m'as pas déclaré en quoi consiste ta grande force.
\VS{16}Comme elle le tourmentait et l'importunait tous les jours par ses paroles, son âme en fut affligée jusqu'à la mort,
\VS{17}alors il lui ouvrit tout son cœur, et lui dit : Le rasoir n'est jamais passé sur ma tête car je suis Naziréen de Dieu dès le ventre de ma mère. Si j'étais rasé, ma force partirait, je me trouverais faible et je serais comme tous les autres hommes.
\VS{18}Delila, voyant qu'il lui avait ouvert tout son cœur, envoya appeler les princes des Philistins, et leur fit dire : Montez cette fois, car il m'a ouvert tout son cœur. Les princes des Philistins montèrent vers elle et emmenèrent l'argent dans leurs mains.
\VS{19}Elle l'endormit sur ses genoux. Et ayant appelé un homme, elle rasa les sept tresses de la tête de Samson et commença à le dompter. Sa force partit.
\VS{20}Alors elle dit : Les Philistins sont sur toi, Samson ! Et il se réveilla de son sommeil, et dit : Je m'en sortirai comme les autres fois et je me dégagerai. Mais il ne savait pas que Yahweh s'était retiré de lui\FTNT{L'immoralité sexuelle de Samson et sa désobéissance à Yahweh, dues à son manque de caractère, ont ruiné à jamais son service et compromis l'avenir du peuple d'Israël qu'il devait diriger (Jg. 16). Cet homme avait reçu un appel puissant dès le sein de sa mère, mais il ne vivait pas dans la crainte de Dieu. Le manque de discernement de Samson lui coûta ainsi toutes les grâces que le Seigneur lui avait accordées : La sainteté symbolisée par ses sept tresses, la force ou l'onction, la vision et la liberté (Jg. 16:21).}.
\TextTitle{Samson entre les mains des Philistins ; dernière vengeance sur ses ennemis}
\VS{21}Les Philistins donc le saisirent, et lui crevèrent les yeux ; ils le descendirent à Gaza, et le lièrent de deux chaînes d'airain. Il tournait la meule dans la prison\FTNT{2 S. 3:34.}.
\VS{22}Les cheveux de sa tête commencèrent à repousser, depuis qu'il avait été rasé.
\VS{23}Or les princes des Philistins s'assemblèrent pour offrir un grand sacrifice à Dagon, leur dieu, et pour se réjouir. Ils disaient : Notre dieu a livré en nos mains Samson, notre ennemi.
\VS{24}Et quand le peuple le vit, il loua son dieu, en disant : Notre dieu a livré entre nos mains notre ennemi, celui qui ravageait notre pays et qui multipliait nos morts.
\VS{25}Comme ils avaient le cœur joyeux, ils dirent : Qu'on appelle Samson, afin qu'il nous fasse rire ! Ils appelèrent Samson et le tirèrent de la prison ; et il joua devant eux. Ils le firent tenir entre les colonnes.
\VS{26}Alors Samson dit au garçon qui le tenait par la main : Laisse-moi afin que je puisse toucher les colonnes sur lesquelles repose la maison pour que je m'y appuie.
\VS{27}Or la maison était remplie d'hommes et de femmes ; tous les princes des Philistins y étaient, et il y avait même sur le toit près de trois mille personnes, hommes et femmes, qui regardaient Samson jouer.
\VS{28}Alors Samson invoqua Yahweh, et dit : Seigneur Yahweh ! Je te prie, souviens-toi de moi ; ô Dieu ! Fortifie-moi seulement cette fois et que par un coup, je me venge des Philistins pour mes deux yeux\FTNT{Hé. 11:32.} !
\VS{29}Samson embrassa les deux colonnes du milieu sur lesquelles reposait la maison et il s'appuya contre elles ; l'une à sa droite et l'autre à sa gauche.
\VS{30}Et il dit : Que mon âme meure avec les Philistins ! Il se pencha donc de toute sa force, et la maison tomba sur les princes et sur tout le peuple qui y était. Et il fit mourir beaucoup plus de gens à sa mort qu'il n'en avait fait mourir pendant sa vie.
\VS{31}Ensuite ses frères et toute la maison de son père descendirent et le transportèrent. Lorsqu'ils furent montés, ils l'enterrèrent entre Tsorea et Eschthaol dans le sépulcre de Manoach, son père. Il avait été juge en Israël pendant vingt ans\FTNT{Jg. 13:2.}.
\Chap{17}
\TextTitle{L'idole de Mica}
\VerseOne{}Il y avait un homme de la montagne d'Ephraïm, nommé Mica.
\VS{2}Il dit à sa mère : Les mille cent sicles d'argent qu'on t'a pris et pour lesquels tu as fait des imprécations même à mes oreilles, voici, j'ai cet argent, c'est moi qui l'avais pris. Alors sa mère dit : Béni soit mon fils par Yahweh !
\VS{3}Et il rendit à sa mère les mille cent sicles d'argent ; sa mère dit : Je consacre de ma main cet argent à Yahweh, afin d'en faire pour mon fils une image taillée, et une image en métal fondu ; et c'est ainsi que je te le rendrai.
\VS{4}Et il rendit l'argent à sa mère. Elle prit deux cents sicles d'argent et les donna au fondeur, qui en fit une image taillée, et une image en métal fondu. On les plaça dans la maison de Mica.
\VS{5}Ainsi cet homme, savoir Mica, avait une maison de dieux ; il fit un éphod et des téraphim, et il consacra par sa main l'un de ses fils, qui lui servit de sacrificateur.
\VS{6}En ce temps-là, il n'y avait pas de roi en Israël. Chacun faisait ce qui lui semblait être droit à ses yeux.
\VS{7}Or il y avait un jeune homme de Bethléhem de Juda, ville de la famille de Juda ; il était Lévite et il séjournait là.
\VS{8}Cet homme partit de la ville de Bethléhem de Juda, pour trouver une demeure qui lui convienne. En chemin, il entra dans la montagne d'Ephraïm jusqu'à la maison de Mica.
\VS{9}Mica lui dit : D'où viens-tu ? Il lui répondit : Je suis Lévite, de Bethléhem de Juda, et je voyage pour trouver une demeure qui me convienne.
\VS{10}Mica lui dit : Demeure avec moi ; tu me serviras de père et de sacrificateur, et je te donnerai dix sicles d'argent par an, les vêtements d'ordre dont tu auras besoin et ton entretien. Et le Lévite vint\FTNT{Jg. 18:19.}.
\VS{11}Ainsi le Lévite convint de demeurer avec cet homme, qui regarda le jeune homme comme l'un de ses fils.
\VS{12}Mica consacra\FTNT{« Consacrer » signifie litteralement « remplir la main ».} le Lévite, qui lui servit de sacrificateur, et demeura dans sa maison.
\VS{13}Mica dit : Maintenant je sais que Yahweh me fera du bien, parce que j'ai un Lévite pour sacrificateur.
\Chap{18}
\TextTitle{Dan recherche un territoire}
\VerseOne{}En ce temps-là, il n'y avait pas de roi en Israël ; et en ce même temps la tribu des Danites cherchait un héritage afin de pouvoir s'établir, car jusqu'à ce jour, il ne lui était pas échu d'héritage au milieu des tribus d'Israël\FTNT{Jg. 17:6.}.
\VS{2}C'est pourquoi les fils de Dan envoyèrent de leur famille cinq hommes vaillants, de Tsorea et d'Eschthaol, pour explorer le pays et l'examiner. Ils leur dirent : Allez examiner le pays. Ils entrèrent dans la montagne d'Ephraïm jusqu'à la maison de Mica, et ils y passèrent la nuit.
\VS{3}Comme ils étaient près de la maison de Mica, ils reconnurent la voix du jeune homme Lévite et lui dirent : Qui t'a amené ici ? Qu'y fais-tu ? Qu'as tu ici ?
\VS{4}Il leur répondit : Mica fait pour moi telle et telle chose, il me donne un salaire et je lui sers de sacrificateur.
\VS{5}Ils lui dirent : Nous te prions de consulter Dieu afin que nous sachions si le voyage que nous entreprenons prospérera.
\VS{6}Et le sacrificateur leur répondit : Allez en paix ; Yahweh a sous ses yeux le voyage que vous mènerez.
\VS{7}Ces cinq hommes s'en allèrent et entrèrent à Laïs. Ils virent le peuple qui y habitait en sécurité selon les coutumes des Sidoniens, tranquille et en confiance ; il n'y avait personne au pays qui les humiliait en quelque chose en dominant sur eux ; ils étaient éloignés des Sidoniens et ils n'avaient aucune affaire avec d'autres hommes.
\VS{8}Puis ils vinrent auprès de leurs frères à Tsorea et à Eschthaol, et leurs frères leur dirent : Quelle nouvelle rapportez-vous ?
\VS{9}Et ils répondirent : Allons ! Montons contre eux ; car nous avons vu le pays et nous l'avons trouvé très bon. Quoi ! Vous restez sans rien faire ? Ne soyez pas paresseux à marcher pour aller posséder ce pays.
\VS{10}Quand vous y entrerez, vous irez vers un peuple en sécurité. Le pays est vaste, Dieu l'a livré entre vos mains ; c'est un lieu où il ne manque rien de tout ce qui est sur la terre.
\VS{11}Il partit de Tsorea et d'Eschthaol, six cents hommes de la famille de Dan, munis de leurs armes de guerre.
\VS{12}Ils montèrent, et campèrent à Kirjath-Jearim en Juda ; c'est pourquoi on a appelé ce lieu qui est derrière Kirjath-Jearim jusqu'à ce jour, Machané-Dan.
\VS{13}Puis ils passèrent par la montagne d'Ephraïm et ils arrivèrent à la maison de Mica.
\TextTitle{Les Danites volent l'idole de Mica et son sacrificateur ; ils prennent possession de Laïs}
\VS{14}Alors les cinq hommes qui étaient allés explorer le pays de Laïs prirent la parole et dirent à leurs frères : Savez-vous qu'il y a dans ces maisons-là un éphod, des théraphim, une image taillée et une image en métal fondu ? Voyez maintenant ce que vous avez à faire.
\VS{15}Alors ils se détournèrent de ce lieu et entrèrent dans la maison où était le jeune homme Lévite, dans la maison de Mica, et lui demandèrent comment il se portait.
\VS{16}Et les six cents hommes d'entre les fils de Dan, qui étaient munis de leurs armes de guerre, se tenaient à l'entrée de la porte.
\VS{17}Mais les cinq hommes, qui étaient allés explorer le pays, montèrent et entrèrent dans la maison ; ils prirent l'image taillée, l'éphod, les théraphim, et l'image en métal fondu, pendant que le sacrificateur était à l'entrée de la porte avec les six cents hommes munis de leurs armes de guerre.
\VS{18}Etant entrés dans la maison de Mica, ils prirent l'image taillée, l'éphod, les théraphim et l'image en métal fondu. Le sacrificateur leur dit : Que faites-vous ?
\VS{19}Ils lui répondirent : Tais-toi, mets ta main sur ta bouche et viens avec nous ; sois pour nous un père et un sacrificateur. Vaut-il mieux que tu serves de sacrificateur à la maison d'un homme seul, ou que tu serves de sacrificateur à une tribu et à une famille en Israël\FTNT{Jg. 17:10.} ?
\VS{20}Le sacrificateur eut de la joie dans son cœur ; il prit l'éphod, les théraphim et l'image taillée, et vint au milieu du peuple.
\VS{21}Après quoi, ils se retournèrent et marchèrent en mettant devant eux les petits enfants, le bétail et les bagages.
\VS{22}Comme ils étaient loin de la maison de Mica, les gens qui habitaient les maisons voisines de celle de Mica furent assemblés à grand cri ; et poursuivirent les fils de Dan.
\VS{23}Et ils crièrent aux fils de Dan, qui se tournèrent de face et dirent à Mica : Qu'as-tu, que tu te sois ainsi écrié pour rassembler ces gens ?
\VS{24}Il répondit : Vous avez enlevé mes dieux que j'avais faits, vous avez pris le sacrificateur et vous vous en êtes allés : Que me reste-t-il ? Comment pouvez-vous me dire : Qu'as-tu\FTNT{Ge. 31:30.} ?
\VS{25}Les fils de Dan lui dirent : Ne fais pas entendre ta voix après nous, de peur que des hommes exaspérés ne se jettent sur vous et que vous n'y laissiez la vie, toi, et tous ceux de ta famille.
\VS{26}Les fils de Dan firent leur chemin. Mica, voyant qu'ils étaient plus forts que lui, s'en retourna et revint dans sa maison.
\VS{27}Ainsi, ils prirent les choses que Mica avait faites et le sacrificateur qu'il avait, et ils entrèrent à Laïs, vers un peuple tranquille et en sécurité ; ils les firent passer au fil de l'épée et ils brûlèrent la ville.
\VS{28}Et il n'y eut personne qui la délivrât car elle était éloignée de Sidon, et ses habitants n'avaient pas d'affaires avec les autres hommes : Elle était située dans la vallée qui appartenait au pays de Beth-Rehob. Les fils de Dan rebâtirent la ville et y demeurèrent.
\VS{29}Ils appelèrent la ville Dan, selon le nom de Dan, leur père qui était né à Israël ; mais la ville s'appelait auparavant Laïs\FTNT{Jos. 19:47.}.
\VS{30}Et les fils de Dan dressèrent l'image taillée ; et Jonathan, fils de Guerschom, fils de Manassé, lui et ses fils, furent sacrificateurs pour la tribu des Danites, jusqu'au jour de la captivité du pays.
\VS{31}Ils y dressèrent donc l'image taillée que Mica avait faite pendant tout le temps que la maison de Dieu fut à Silo.
\Chap{19}
\TextTitle{Dégradation morale du peuple}
\VerseOne{}Il arriva aussi en ce temps-là, où il n'y avait pas de roi en Israël, qu'un Lévite qui habitait aux côtés de la montagne d'Ephraïm, prit pour concubine une femme de Bethléhem de Juda\FTNT{Jg. 17:6 ; 21:25.}.
\VS{2}Mais sa concubine se prostitua chez lui et elle s'en alla pour aller dans la maison de son père à Bethléhem de Juda, où elle resta pendant quatre mois.
\VS{3}Puis son mari se leva et alla après elle, pour parler à son cœur, et la ramener. Il avait avec lui son serviteur et deux ânes. Elle le fit entrer dans la maison de son père ; et quand le père de la jeune femme le vit, il s'approcha avec joie.
\VS{4}Son beau-père, le père de la jeune femme, le retint avec grande instance, de sorte qu'il demeura trois jours avec lui. Ils mangèrent et burent, et logèrent là.
\VS{5}Le quatrième jour, ils se levèrent de bon matin, et le Lévite se levait pour s'en aller. Mais le père de la jeune femme dit à son gendre : Fortifie ton cœur avec un morceau de pain, et vous partirez ensuite.
\VS{6}Ils s'assirent, et ils mangèrent et burent eux deux ensemble. Puis le père de la jeune femme dit au mari : Je te prie consens à passer encore ici cette nuit, et que ton cœur se réjouisse.
\VS{7}Le mari se levait pour s'en aller ; mais son beau-père le pressa tellement, qu'il s'en retourna, et y passa encore la nuit.
\VS{8}Le cinquième jour, il se leva de bon matin pour s'en aller. Alors le père de la jeune femme dit : Fortifie ton cœur ; et attendez le déclin du jour. Et ils mangèrent eux deux.
\VS{9}Puis le mari se levait pour s'en aller, avec sa concubine et son serviteur ; mais son beau-père, le père de la jeune femme, lui dit : Voici, maintenant le jour baisse, il se fait tard, je vous prie passez ici la nuit ; voici le jour est sur son déclin, passe ici la nuit, et que ton cœur se réjouisse ; demain matin vous vous mettrez en route, et tu t'en iras à ta tente.
\VS{10}Mais le mari ne voulut pas y passer la nuit il se leva, et s'en alla. Il vint jusque vis-à-vis de Jébus, qui est Jérusalem, avec les deux ânes bâtés et sa concubine.
\VS{11}Comme ils étaient près de Jébus, le jour avait beaucoup baissé. Le serviteur dit à son maître : Allons, détournons-nous vers cette ville des Jébusiens afin que nous y passions la nuit.
\VS{12}Son maître lui répondit : Nous ne nous détournerons pas vers une ville d'étrangers, où il n'y a pas d'enfants d'Israël mais nous passerons par Guibea.
\VS{13}Il dit aussi à son serviteur : Allons, approchons-nous de l'un de ces lieux, Guibea ou Rama et passons-y la nuit.
\VS{14}Ils continuèrent à marcher et le soleil se coucha quand ils furent près de Guibea, qui appartient à Benjamin.
\VS{15}Alors ils se détournèrent vers Guibea, et y entrèrent pour passer la nuit. Le Lévite entra et il s'assit sur la place de la ville. Il n'y eut aucun homme qui les reçut dans sa maison afin qu'ils y passent la nuit.
\VS{16}Et voici, sur le soir, un vieil homme venait de travailler dans les champs ; cet homme était de la montagne d'Ephraïm, il séjournait à Guibea et les gens du lieu étaient Benjamites.
\VS{17}Et levant ses yeux, il vit le voyageur sur la place de la ville. Le vieil homme lui dit : Où vas-tu et d'où viens-tu ?
\VS{18}Il lui répondit : Nous passons de Bethléhem de Juda vers les côtés de la montagne d'Ephraïm, d'où je suis. J'étais allé jusqu'à Bethléhem de Juda, mais maintenant je m'en vais à la maison de Yahweh. Mais il n'y a aucun homme qui me reçoive dans sa maison.
\VS{19}Nous avons pourtant de la paille et du fourrage pour nos ânes ; du pain et du vin pour moi, pour ta servante et pour le garçon qui est avec tes serviteurs. Nous n'avons besoin d'aucune chose.
\VS{20}Le vieil homme dit : Pourvu que la paix soit ! Quoi qu'il en soit, je me charge de tous tes besoins, je te prie seulement de ne pas passer la nuit sur la place.
\VS{21}Alors il les fit entrer dans sa maison, et il donna du fourrage aux ânes. Les voyageurs se lavèrent les pieds ; puis ils mangèrent et burent\FTNT{Ge. 43:24.}.
\VS{22}Comme ils se réjouissaient, voici, les hommes de la ville, fils d'hommes pervers, environnèrent la maison, frappèrent à la porte et dirent au vieil homme, maître de la maison : Fais sortir l'homme qui est entré dans ta maison, afin que nous le connaissions\FTNT{Ge. 19:4 ; Jg. 20:13 ; Os. 9:9 ; 10:9.}.
\VS{23}Mais cet homme, à savoir le maître de la maison, sortit vers eux, et leur dit : Non, mes frères, ne lui faites pas de mal, je vous prie ; puisque cet homme est entré dans ma maison ne faites pas une telle infamie.
\VS{24}Voici, j'ai une fille vierge, et cet homme a une concubine ; je vous les amènerai dehors ; vous les déshonorerez, et vous ferez d'elles comme il semblera bon à vos yeux. Mais ne faites pas cette action infâme à l'égard de cet homme.
\TextTitle{Viol et meurtre de la concubine du Lévite}
\VS{25}Mais ces gens ne voulurent pas l'écouter. C'est pourquoi l'homme saisit sa concubine et la leur amena dehors. Ils la connurent et abusèrent d'elle toute la nuit jusqu'au matin ; puis ils la renvoyèrent au lever de l'aurore.
\VS{26}Vers le matin, cette femme alla tomber à la porte de la maison de l'homme où était son mari et elle y demeura jusqu'au jour.
\VS{27}Et le matin, son mari se leva et ayant ouvert la porte de la maison, il sortit pour poursuivre son chemin. Mais voici, la femme concubine était tombée à la porte de la maison et avait les mains sur le seuil.
\VS{28}Il lui dit : Lève-toi, et allons-nous-en. Mais elle ne répondit pas. Alors il l'emmena sur un âne, se mit en chemin et s'en alla dans sa demeure.
\VS{29}En entrant en sa maison, il prit un couteau et saisissant sa concubine, il la coupa avec ses os en douze morceaux, qu'il envoya dans tout le territoire d'Israël.
\VS{30}Et il arriva que tous ceux qui virent cela dirent : Une telle chose n'a été faite ni vue depuis le jour où les enfants d'Israël sont montés hors du pays d'Egypte, jusqu'à ce jour ; prenez la chose à cœur, consultez-vous et parlez !
\Chap{20}
\TextTitle{Réunion des enfants d'Israël à Mitspa}
\VerseOne{}Alors tous les enfants d'Israël sortirent et toute l'assemblée se réunit comme un seul homme, depuis Dan jusqu'à Beer-Schéba et jusqu'au pays de Galaad, devant Yahweh, à Mitspa.
\VS{2}Les chefs de tout le peuple, toutes les tribus d'Israël, se présentèrent à l'assemblée du peuple de Dieu, au nombre de quatre cent mille hommes de pied, tirant l'épée.
\VS{3}Les fils de Benjamin entendirent que les enfants d'Israël étaient montés à Mitspa. Les enfants d'Israël dirent : Parlez, comment ce mal est arrivé ?
\VS{4}Alors le Lévite, mari de la femme tuée, répondit et dit : J'étais venu à Guibea de Benjamin, avec ma concubine, pour y passer la nuit.
\VS{5}Les seigneurs de Guibea se sont élevés contre moi et ont encerclé de nuit la maison où j'étais. Ils avaient l'intention de me tuer et ils ont tellement violé ma concubine qu'elle en est morte.
\VS{6}C'est pourquoi j'ai saisi ma concubine, je l'ai coupée en morceaux et je les ai envoyés dans tout le territoire de l'héritage d'Israël ; car ils ont commis un crime et une infamie en Israël.
\VS{7}Vous voici tous, enfants d'Israël ; consultez-vous sur la question et prenez ici une décision !
\VS{8}Tout le peuple se leva comme un seul homme, et ils dirent : Aucun homme n'ira dans sa tente, et aucun homme ne se retirera dans sa maison.
\VS{9}Et maintenant voici ce que nous ferons à Guibea : Nous marcherons contre elle d'après le sort.
\VS{10}Nous prendrons dans toutes les tribus d'Israël dix hommes sur cent, cent sur mille, et mille sur dix mille ; nous prendrons des provisions pour le peuple, afin qu'en entrant à Guibea de Benjamin, on leur fasse selon toute l'infamie qu'elle a commise en Israël.
\VS{11}Ainsi tous les hommes d'Israël s'assemblèrent contre la ville, unis comme un seul homme.
\TextTitle{Benjamin refuse de livrer Guibea ; guerre entre Benjamin et le reste d'Israël}
\VS{12}Alors les tribus d'Israël envoyèrent des hommes vers la maison de Benjamin, pour dire : Quelle méchanceté a été faite parmi vous ?
\VS{13}Maintenant donc livrez-nous les fils des hommes pervers qui sont à Guibea, afin que nous les fassions mourir et que nous ôtions le mal du milieu d'Israël. Mais les fils de Benjamin ne voulurent pas écouter la voix de leurs frères, les enfants d'Israël.
\VS{14}Et les fils de Benjamin s'assemblèrent à Guibea pour sortir en guerre contre les fils d'Israël.
\VS{15}En ce jour-là, on fit le dénombrement des fils de Benjamin qui étaient dans ces villes, et il se trouva vingt-six mille hommes, tirant l'épée, sans compter les habitants de Guibea formant sept cents hommes d'élite.
\VS{16}De tout ce peuple, il y avait sept cents hommes d'élite qui ne se servaient pas de la main droite ; tous tirant la pierre avec la fronde, à un cheveu près, ils n'y manquaient pas.
\VS{17}On fit aussi le dénombrement des hommes d'Israël, excepté ceux de Benjamin, et l'on en trouva quatre cent mille hommes tirant l'épée, tous gens de guerre.
\VS{18}Et les fils d'Israël se levèrent, montèrent vers Dieu à Béthel pour le consulter, en disant : Qui d'entre nous montera le premier pour faire la guerre aux fils de Benjamin ? Yahweh répondit : Juda montera le premier.
\VS{19}Puis les fils d'Israël se levèrent de bon matin et campèrent près de Guibea.
\VS{20}Et les hommes d'Israël sortirent pour combattre ceux de Benjamin, et se rangèrent en bataille près de Guibea.
\VS{21}Les fils de Benjamin sortirent de Guibea et ils tuèrent ce jour-là vingt-deux mille hommes d'Israël.
\VS{22}Toutefois le peuple, les hommes d'Israël, se fortifièrent et se rangèrent de nouveau en bataille au lieu où ils s'étaient rangés le premier jour.
\VS{23}Et les fils d'Israël montèrent, et ils pleurèrent devant Yahweh jusqu'au soir ; ils consultèrent Yahweh en disant : M'approcherai-je encore pour combattre contre les fils de Benjamin, mon frère ? Yahweh dit : Montez contre lui.
\VS{24}Le second jour, les fils d'Israël s'approchèrent des fils de Benjamin.
\VS{25}Ce même jour, les Benjamites sortirent de Guibea à leur rencontre, et ils tuèrent encore dix-huit mille hommes des fils d'Israël, tous tirant l'épée.
\VS{26}Alors tous les fils d'Israël et tout le peuple montèrent et vinrent vers Dieu à Béthel ; ils pleurèrent, et restèrent là devant Yahweh. Ce jour-là ils jeûnèrent jusqu'au soir, et ils offrirent des holocaustes, et des offrandes de paix\FTNT{Voir commentaire en Lé. 3:1.} devant Yahweh.
\VS{27}Ensuite les enfants d'Israël consultèrent Yahweh, c'était là que se trouvait l'arche de l'alliance de Dieu ;
\VS{28}et Phinées, fils d'Eléazar, fils d'Aaron, se tenait devant Yahweh en ce temps-là en disant : Sortirai-je encore en guerre contre les fils de Benjamin, mon frère, ou dois-je m'en abstenir ? Yahweh répondit : Montez, car demain je les livrerai entre vos mains.
\VS{29}Alors Israël mit une embuscade autour de Guibea.
\VS{30}Le troisième jour, les fils d'Israël montèrent contre les fils de Benjamin, et ils se rangèrent en bataille contre Guibea, comme les autres fois.
\VS{31}Alors les fils de Benjamin sortirent à la rencontre du peuple, et ils furent attirés hors de la ville. Ils commencèrent à frapper à mort quelques-uns du peuple comme les autres fois, environ trente hommes d'Israël, sur les routes dont l'une monte à Béthel et l'autre à Guibea, par les champs.
\VS{32}Les fils de Benjamin disaient : Ils tombent battus devant nous, comme la première fois ! Mais les fils d'Israël disaient : Fuyons et attirons-les hors de la ville dans les chemins.
\VS{33}Tous les hommes d'Israël se levant de leur lieu, se rangèrent à Baal-Thamar ; et l'embuscade sortit du lieu où ils étaient, de Maaré-Guibea.
\VS{34}Dix mille hommes choisis sur tout Israël vinrent contre Guibea. La bataille fut rude et les Benjamites ne surent pas que le mal les atteindrait.
\TextTitle{Défaite écrasante de Benjamin}
\VS{35}Yahweh battit Benjamin devant Israël et les fils d'Israël tuèrent ce jour-là vingt-cinq mille cent hommes de Benjamin, tous tirant l'épée.
\VS{36}Les fils de Benjamin regardaient comme battus les hommes d'Israël, qui cédaient du terrain à Benjamin et se reposaient sur l'embuscade qu'ils avaient mise près de Guibea.
\VS{37}Ceux qui étaient en embuscade se jetèrent promptement sur Guibea, ils se portèrent en avant et frappèrent toute la ville au tranchant de l'épée.
\VS{38}Et le signal convenu entre les hommes d'Israël et l'embuscade était qu'ils fassent monter beaucoup de fumée de la ville.
\VS{39}Les hommes d'Israël avaient donc tourné le dos dans la bataille, et les Benjamites avaient commencé de frapper et de blesser à mort environ trente hommes de ceux d'Israël ; et ils disaient : Certainement ils tombent devant nous comme à la première bataille !
\VS{40}Mais quand l'épaisse colonne de fumée commençait à monter de la ville, les Benjamites se tournèrent ; et voici, derrière eux toute la ville disparaissait montant en feu vers le ciel.
\VS{41}Les hommes d'Israël tournèrent le visage ; et ceux de Benjamin furent épouvantés en voyant le mal qui allait les atteindre.
\VS{42}Ils tournèrent le dos devant les hommes d'Israël par le chemin du désert. Mais les assaillants s'attachaient à leurs pas, et ils détruisirent ceux qui étaient sortis des villes.
\VS{43}Ils environnèrent Benjamin, le poursuivirent, l'écrasèrent dès qu'il voulut se reposer jusqu'en face de Guibea, du côté du soleil levant.
\VS{44}Il tomba dix-huit mille hommes de Benjamin, tous des vaillants hommes.
\VS{45}Et parmi ceux de Benjamin qui tournèrent le dos pour s'enfuir vers le désert au rocher de Rimmon, les hommes d'Israël en firent périr cinq mille hommes sur les routes ; et les poursuivant de près jusqu'à Guideom, ils frappèrent deux mille hommes.
\VS{46}En ce jour-là, le nombre de Benjamites qui tombèrent fut de vingt-cinq mille hommes tirant l'épée et tous étaient des vaillants hommes.
\VS{47}Et il y eut six cents hommes de ceux qui avaient tourné le dos, qui s'échappèrent vers le désert au rocher de Rimmon, et qui demeurèrent au rocher de Rimmon pendant quatre mois.
\VS{48}Les hommes d'Israël retournèrent vers les fils de Benjamin et ils les frappèrent du tranchant de l'épée, depuis les hommes des villes jusqu'aux bêtes, et tout ce qui s'y trouva. Ils brûlèrent toutes les villes qu'ils trouvaient.
\Chap{21}
\TextTitle{La tribu de Benjamin menacée d'extinction ; regret d'Israël}
\VerseOne{}Les hommes d'Israël avaient juré à Mitspa, en disant : Aucun homme ne donnera sa fille pour femme à un Benjamite.
\VS{2}Puis le peuple vint vers Dieu à Béthel, jusqu'au soir. Ils élevèrent leurs voix et pleurèrent grandement,
\VS{3}Et ils dirent : Ô Yahweh, Dieu d'Israël, pourquoi est-il arrivé en Israël qu'une tribu d'Israël ait été aujourd'hui punie ?
\VS{4}Le lendemain, le peuple se leva de bon matin ; ils bâtirent là un autel et ils offrirent des holocaustes et des sacrifices d'offrande de paix.
\VS{5}Alors les enfants d'Israël dirent : Quel est celui d'entre toutes les tribus d'Israël qui n'est pas monté à l'assemblée vers Yahweh ? Car on avait fait un grand serment contre tout homme qui ne monterait pas vers Yahweh à Mitspa, en disant : Il sera puni de mort.
\VS{6}Les enfants d'Israël se repentaient de ce qui était arrivé à Benjamin, leur frère, et ils disaient : Aujourd'hui une tribu a été retranchée d'Israël.
\VS{7}Comment ferons-nous pour donner des femmes à ceux qui ont survécu, puisque nous avons juré par Yahweh que nous ne leur donnerions pas nos filles pour femmes ?
\VS{8}Ils dirent donc : Y a-t-il quelqu'un d'entre les tribus d'Israël qui ne soit pas monté vers Yahweh à Mitspa ? Et voici, aucun homme de Jabès en Galaad n'était venu au camp, à l'assemblée.
\VS{9}Quand on fit le dénombrement du peuple, il n'y avait aucun des hommes habitant à Jabès en Galaad.
\VS{10}C'est pourquoi l'assemblée envoya contre eux douze mille hommes des fils vaillants, en leur donnant cet ordre : Allez, et frappez du tranchant de l'épée les habitants de Jabès en Galaad, tant les femmes que les enfants.
\VS{11}Voici les choses que vous ferez : Vous détruirez par le moyen de l'interdit tout mâle et toute femme qui a connu la couche d'un homme.
\VS{12}Ils trouvèrent parmi les habitants de Jabès en Galaad quatre cents filles vierges, qui n'avaient pas connu d'homme en couchant avec lui, et ils les amenèrent au camp de Silo, qui est sur la terre de Canaan.
\VS{13}Alors toute l'assemblée envoya parler aux fils de Benjamin qui étaient au rocher de Rimmon, pour leur proclamer la paix.
\VS{14}En ce temps-là, les Benjamites revinrent et on leur donna pour femmes celles qui avaient été conservées en vie d'entre les femmes de Jabès en Galaad. Mais ils n'en trouvèrent pas assez pour eux.
\VS{15}Le peuple se repentit de ce qui avait été fait à Benjamin, car Yahweh avait fait une brèche dans les tribus d'Israël.
\VS{16}Les anciens de l'assemblée dirent : Comment ferons-nous pour donner des femmes à ceux qui restent, car les femmes de Benjamin ont été détruites ?
\VS{17}Et ils dirent : Que ceux qui sont réchappés de Benjamin possèdent leur héritage, afin qu'une tribu d'Israël ne soit pas effacée.
\VS{18}Cependant, nous ne pouvons pas leur donner des femmes d'entre nos filles, car les enfants d'Israël ont juré, en disant : Maudit soit celui qui donnera une femme à un Benjamite !
\VS{19}Et ils dirent : Voici, il y a chaque année une fête de Yahweh à Silo, qui est au nord de Béthel, à l'orient qui monte à Béthel, à Sichem, et au midi de Lebona.
\VS{20}Puis ils ordonnèrent aux fils de Benjamin : Allez, et placez-vous en embuscade dans les vignes.
\VS{21}Vous verrez, et voici, lorsque les filles de Silo sortiront pour danser, alors vous sortirez des vignes, vous enlèverez chacun une des filles de Silo pour en faire votre femme, et vous vous en irez dans le pays de Benjamin.
\VS{22}Si leurs pères ou leurs frères viennent se plaindre auprès de nous, nous leur dirons : Accordez-nous cette faveur, puisque nous n'avons pas pris de femmes pour chaque homme dans cette guerre. Ce n'est pas vous qui les leur avez données ; sinon vous en seriez coupables en ce temps.
\VS{23}Les fils de Benjamin firent ainsi ; ils prirent des femmes selon leur nombre, parmi les danseuses qu'ils saisirent, puis ils s'en allèrent et retournèrent dans leur héritage ; ils rebâtirent les villes et y habitèrent.
\VS{24}Ainsi en ce temps-là, chacun des enfants d'Israël s'en alla de là dans sa tribu et dans sa famille, et ils se retirèrent de là chacun dans son héritage.
\VS{25}En ce temps-là, il n'y avait pas de roi en Israël. L'homme faisait ce qui lui semblait être droit à ses yeux.
\PPE{}
\end{multicols}
