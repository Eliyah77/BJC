\ShortTitle{1 Jean}\BookTitle{1 Jean}\BFont
\noindent\hrulefill
{\footnotesize
\textit{
\bigskip
{\centering{}
\\Auteur : Jean
\\(Gr. : Ioannes)
\\Signification : Yahweh a fait grâce
\\Thème : La communion fraternelle, la connaissance et l'amour
\\Date de rédaction : Env. 85 ap. J.-C.\\}
}
%\bigskip
\textit{
\\Cette épître, écrite par Jean à Ephèse, était destinée aux églises de la province d’Asie qu’il connaissait bien. Il souhaite rendre leur joie parfaite en fortifiant leur foi en Christ et en leur donnant l’assurance de la vie éternelle ;  tout en les mettant en garde contre les faux docteurs.\bigskip
}
}
\par\nobreak\noindent\hrulefill
\begin{multicols}{2}
\Chap{1}
\TextTitle{La Parole incarnée}
\VerseOne{}Ce qui était dès le commencement, ce que nous avons entendu, ce que nous avons vu de nos propres yeux, ce que nous avons contemplé, et que nos propres mains ont touché concernant la Parole de vie,
\VS{2}car la vie a été manifestée, et nous l'avons vue et nous lui rendons témoignage, et nous vous annonçons la vie éternelle, qui était avec le Père, et qui nous a été manifestée.
\TextTitle{Communion avec le Père et le Fils}
\VS{3}Ce que nous avons vu dis-je, et ce que nous avons entendu, nous vous l'annonçons, afin que vous soyez en communion avec nous, et que notre communion soit avec le Père et avec son Fils Jésus-Christ.
\VS{4}Et nous vous écrivons ces choses afin que votre joie soit parfaite.
\TextTitle{De la communion avec Dieu, qui est Lumière, et de la confession des péchés}
\TextTitle{Conditions de la communion avec Dieu
\\a. Position de l'enfant de Dieu dans la Lumière}
\VS{5}Or c'est ici la déclaration que nous avons entendue de lui et que nous vous annonçons, à savoir que Dieu est Lumière et qu'il n'y a point en lui de ténèbres.
\VS{6}Si nous disons que nous sommes en communion avec lui, et que nous marchions dans les ténèbres, nous mentons, et nous n'agissons pas selon la vérité.
\VS{7}Mais si nous marchons dans la Lumière, comme Dieu est dans la Lumière, nous sommes en communion les uns avec les autres, et le sang de son Fils Jésus-Christ nous purifie de tout péché.
\TextTitle{b. Reconnaissance de la présence du péché en nous}
\VS{8}Si nous disons que nous n'avons point de péché, nous nous séduisons nous-mêmes, et la vérité n'est point en nous.
\TextTitle{c. la confession des péchés, le pardon et la purification}
\VS{9}Si nous confessons nos péchés, il est fidèle et juste pour nous les pardonner, et pour nous purifier de toute iniquité.
\VS{10}Si nous disons que nous n'avons point de péché, nous le faisons menteur, et sa Parole n'est point en nous.
\TextTitle{Celui qui connaît Jésus-Christ garde ses commandements}
\Chap{2}
\TextTitle{d. Christ, notre avocat pour nos péchés}
\VerseOne{}Mes petits-enfants, je vous écris ces choses afin que vous ne péchiez point. Et si quelqu'un a péché, nous avons un avocat\FTNT{Jésus, notre Avocat. Le mot grec «~parakletos~», traduit ici par «~avocat~», se trouve également en Jean 14 et 16, où il est traduit par «~Consolateur~» et s’applique au Saint-Esprit. Le Seigneur exerce la fonction d'avocat actuellement pour nous dans le ciel. Voir Ro. 8:33 ; Hé. 7:25.} auprès du Père, Jésus-Christ, le Juste.
\VS{2}Car c'est lui qui est la victime de propitiation pour nos péchés, et non seulement pour les nôtres, mais aussi pour ceux de tout le monde.
\TextTitle{e. reconnaissance de la sainteté de Dieu}
\VS{3}Et nous savons que nous l'avons connu, si nous gardons ses commandements.
\VS{4}Celui qui dit : Je l'ai connu, et qui ne garde point ses commandements, est un menteur, et il n'y a point de vérité en lui.
\VS{5}Mais celui qui garde sa Parole, l'amour de Dieu est véritablement parfait en lui : et c'est par cela que nous savons que nous sommes en lui.
\VS{6}Celui qui dit qu'il demeure en lui doit aussi vivre comme Jésus-Christ lui-même a vécu.
\VS{7}Mes frères, je ne vous écris point un commandement nouveau, mais un commandement ancien, que vous avez eu dès le commencement ; et ce commandement ancien c'est la Parole que vous avez entendue dès le commencement.
\VS{8}Cependant, le commandement que je vous écris est un commandement nouveau, c’est une chose véritable en lui et en vous, parce que les ténèbres sont passées, et que la véritable Lumière paraît déjà.
\VS{9}Celui qui dit qu'il est dans la Lumière, et qui hait son frère, est dans les ténèbres jusqu'à présent.
\VS{10}Celui qui aime son frère demeure dans la Lumière, et il n'y a rien en lui qui puisse le faire tomber.
\VS{11}Mais celui qui hait son frère est dans les ténèbres, et il marche dans les ténèbres, et il ne sait pas où il va car les ténèbres ont aveuglé ses yeux.
\TextTitle{Exhortations à la famille spirituelle}
\VS{12}Mes petits enfants, je vous écris parce que vos péchés vous sont pardonnés à cause de son Nom.
\VS{13}Pères, je vous écris parce que vous avez connu celui qui est dès le commencement. Jeunes gens, je vous écris parce que vous avez vaincu l’esprit du malin.
\VS{14}Jeunes enfants, je vous écris parce que vous avez connu le Père. Pères, je vous ai écrit parce que vous avez connu celui qui est dès le commencement. Jeunes gens, je vous ai écrit parce que vous êtes forts et que la Parole de Dieu demeure en vous, et que vous avez vaincu l’esprit du malin.
\TextTitle{Les enfants de Dieu ne doivent pas aimer le monde}
\VS{15}N'aimez point le monde ni les choses qui sont dans le monde ; si quelqu'un aime le monde, l'amour du Père n'est point en lui.
\VS{16}Car tout ce qui est dans le monde, c'est-à-dire la convoitise de la chair, la convoitise des yeux et l'orgueil de la vie, ne vient point du Père, mais vient du monde.
\VS{17}Et le monde passe avec sa convoitise ; mais celui qui fait la volonté de Dieu demeure éternellement.
\TextTitle{Les enfants de Dieu mis en garde contre les apostats}
\VS{18}Petits enfants, c'est ici la dernière\FTNT{Dernière, du grec «~eschatos~», signifie «~dernier dans une succession dans le temps~». Voir Ge. 49:1-2.} heure ; et comme vous avez entendu que l'Antéchrist viendra, il y a maintenant plusieurs antéchrists ; et par là nous connaissons que c'est la dernière heure.
\VS{19}Ils sont sortis du milieu de nous, mais ils n'étaient pas des nôtres ; car s'ils avaient été des nôtres, ils seraient demeurés avec nous, mais c'est afin qu'il soit manifeste que tous ne sont point des nôtres.
\VS{20}Mais vous avez été oints par le Saint-Esprit, et vous connaissez toutes choses.
\VS{21}Je ne vous ai pas écrit comme si vous ne connaissiez point la vérité, mais parce que vous la connaissez, et qu'aucun mensonge ne vient de la vérité.
\VS{22}Qui est le menteur, sinon celui qui nie que Jésus est le Christ ? Celui-là est l'Antéchrist qui nie le Père et le Fils.
\VS{23}Quiconque nie le Fils, n'a point non plus le Père ; quiconque confesse le Fils, a aussi le Père.
\VS{24}Que ce que vous avez entendu dès le commencement demeure en vous, car si ce que vous avez entendu dès le commencement demeure en vous, vous demeurerez aussi dans le Fils et dans le Père.
\VS{25}Et c'est ici la promesse qu'il nous a faite, à savoir la vie éternelle.
\VS{26}Je vous ai écrit ces choses au sujet de ceux qui vous séduisent.
\VS{27}Mais l'onction que vous avez reçue de lui demeure en vous, et vous n'avez pas besoin qu'on vous enseigne ; mais comme la même onction vous enseigne toutes choses, qu'elle est véritable et n'est pas un mensonge, demeurez en lui selon les enseignements qu’elle vous a donnés.
\TextTitle{Exhortation à demeurer en Christ}
\VS{28}Maintenant donc, mes petits enfants, demeurez en lui ; afin que quand il apparaîtra, nous ayons de l'assurance, et que nous ne soyons point confus devant lui lors de son avènement\FTNT{Avènement, du grec «~parousia~», veut dire «~l’arrivée~» ou «~la présence~». Lors de cette seconde venue, le Messie prendra son Epouse pour les noces, ensuite il posera ses pieds sur le Mont des Oliviers, détruira les armées de l'Antichrist, puis commencera son règne de mille ans. Voir Za. 14.}.
\VS{29}Si vous savez qu'il est juste, sachez que quiconque fait ce qui est juste est né de lui.
\Chap{3}
\VerseOne{}Voyez quelle charité le Père nous a témoignée, pour que nous soyons appelés enfants de Dieu ! Mais le monde ne nous connaît point parce qu'il ne l'a point connu.
\VS{2}Mes bien-aimés, nous sommes maintenant enfants de Dieu, et ce que nous serons n'est pas encore manifesté ; or nous savons que lorsque le Fils de Dieu apparaîtra, nous serons semblables à lui, car nous le verrons tel qu'il est.
\VS{3}Et quiconque a cette espérance en lui se purifie, comme lui aussi est pur.
\TextTitle{Caractéristiques des enfants de Dieu et des enfants du diable}
\VS{4}Quiconque pèche, transgresse la loi, car le péché est la transgression de la loi.
\VS{5}Or vous savez qu'il est apparu pour ôter nos péchés ; et il n'y a point de péché en lui.
\VS{6}Quiconque demeure en lui ne pèche point ; quiconque pèche, ne l'a pas vu, et ne l'a pas connu.
\VS{7}Mes petits-enfants, que personne ne vous séduise. Celui qui fait ce qui est juste est une personne juste, comme Jésus-Christ est juste.
\VS{8}Celui qui vit dans le péché est du diable, car le diable pèche dès le commencement. Or le Fils de Dieu est apparu afin de détruire les œuvres du diable.
\VS{9}Quiconque est né de Dieu ne vit pas dans le péché, car la semence de Dieu demeure en lui ; et il ne peut pécher, parce qu'il est né de Dieu.
\VS{10}Et c'est par là que nous connaissons les enfants de Dieu et les enfants du diable. Quiconque ne fait pas ce qui est juste et qui n'aime pas son frère n'est point de Dieu.
\VS{11}Car ce qui vous a été annoncé et ce que vous avez entendu dès le commencement c’est que nous nous aimions les uns les autres.
\VS{12}Et que nous ne soyons pas comme Caïn\FTNT{La doctrine de la semence du serpent est présentée par certains comme une explication du sens caché de la chute de l’homme dans le jardin en Eden et du péché originel. Selon cette doctrine, l’acte sexuel serait le fruit de l’arbre de la connaissance du bien et du mal. Cependant, cette doctrine n’est pas biblique, Eve n’a jamais eu de relations sexuelles avec le serpent. Dans Jean 8:44, lorsque Jésus dit aux pharisiens «~vous avez pour père le diable…~» suppose-t-il que le diable engendre des enfants physiquement ? Bien sûr que non !}, qui était de l'esprit malin et qui tua son frère. Et pourquoi le tua-t-il ? C’est parce que ses œuvres étaient mauvaises, et que celles de son frère étaient justes.
\VS{13}Mes frères, ne vous étonnez point si le monde vous hait.
\VS{14}Nous savons que nous sommes passés de la mort à la vie parce que nous aimons nos frères. Celui qui n'aime pas son frère demeure dans la mort.
\VS{15}Quiconque hait son frère est un meurtrier, et vous savez qu'aucun meurtrier ne possède la vie éternelle.
\VS{16}Nous avons connu la charité en ce qu'il a donné sa vie pour nous ; nous aussi, nous devons donner nos vies pour nos frères\FTNT{Jn. 15:13.}.
\VS{17}Si quelqu’un possède les biens du monde, et que voyant son frère dans la nécessité, il lui ferme ses entrailles, comment la charité de Dieu demeure-t-elle en lui ?
\VS{18}Mes petits-enfants, n'aimons pas en paroles et avec la langue, mais par des œuvres et en vérité.
\VS{19}Car c'est par là que nous connaissons que nous sommes de la vérité ; et nous rassurerons ainsi nos cœurs devant lui.
\VS{20}Si notre cœur nous condamne, certes Dieu est plus grand que notre cœur, et il connaît toutes choses.
\VS{21}Mes bien-aimés, si notre cœur ne nous condamne point, nous avons de l’assurance devant Dieu.
\VS{22}Et quoi que nous demandions, nous le recevons de lui, parce que nous gardons ses commandements, et que nous faisons les choses qui lui sont agréables.
\VS{23}Et c'est ici son commandement, que nous croyions au Nom de son Fils Jésus-Christ, et que nous nous aimions les uns les autres, selon le commandement qu’il nous a donné.
\VS{24}Celui qui garde ses commandements demeure en Jésus-Christ, et Jésus-Christ demeure en lui ; et par là nous connaissons qu'il demeure en nous, par l'Esprit qu'il nous a donné.
\Chap{4}
\TextTitle{L’amour de Dieu, dans le cœur de ses enfants}
\VerseOne{}Mes bien-aimés, ne croyez pas à tout esprit, mais éprouvez les esprits pour savoir s'ils sont de Dieu, car plusieurs faux prophètes sont venus dans le monde.
\TextTitle{[Caractéristiques des faux prophètes
\\a. leur confession sur Jésus-Christ]}
\VS{2}Reconnaissez à cette marque l'Esprit de Dieu : Tout esprit qui confesse que Jésus-Christ est venu en chair est de Dieu.
\VS{3}Et tout esprit qui ne confesse point que Jésus-Christ est venu en chair n'est point de Dieu ; c’est l'esprit de l'Antéchrist, dont vous avez appris la venue, et qui maintenant est déjà dans le monde.
\VS{4}Mes petits-enfants, vous êtes de Dieu, et vous les avez vaincus, parce que celui qui est en vous est plus grand que celui qui est dans le monde.
\TextTitle{b. leur appartenance au monde}
\VS{5}Eux, ils sont du monde, c'est pourquoi ils parlent comme étant du monde, et le monde les écoute.
\VS{6}Nous sommes de Dieu ; celui qui connaît Dieu nous écoute ; mais celui qui n'est pas de Dieu ne nous écoute point ; c’est par là que nous connaissons l'esprit de vérité et l'esprit de l’erreur.
\TextTitle{La charité de Dieu}
\VS{7}Mes bien-aimés, aimons-nous les uns les autres, car la charité est de Dieu ; et quiconque aime son prochain est né de Dieu et connaît Dieu.
\VS{8}Celui qui n'aime point son prochain n'a pas connu Dieu, car Dieu est Charité\FTNT{« Agapé » en grec.}.
\VS{9}La charité de Dieu a été manifestée envers nous en ce que Dieu a envoyé son Fils unique dans le monde, afin que nous vivions par lui.
\VS{10}Et cette charité consiste, non point en ce que nous avons aimé Dieu, mais en ce qu'il nous a aimés, et qu'il a envoyé son Fils pour être la propitiation\FTNT{Du grec «~hilasmos~» qui signifie «~apaisement~». Les écritures nous parlent aussi du «~propitiatoire~», c'est-à-dire « le siège de la miséricorde » ou « Lieu de l'expiation ». Le propitiatoire était une plaque en or du sommet de l'Arche de l'Alliance. Le souverain sacrificateur l'aspergeait sept fois, le jour de l'expiation afin de reconcilier symboliquement Yahweh et son peuple. Voir Ex. 25:17-22} pour nos péchés.
\VS{11}Mes bien-aimés, si Dieu nous a ainsi aimés, nous devons aussi nous aimer les uns les autres.
\VS{12}Personne n'a jamais vu Dieu ; si nous nous aimons les uns les autres, Dieu demeure en nous et sa charité est parfaite en nous.
\VS{13}A ceci nous connaissons que nous demeurons en lui, et lui en nous, c'est qu'il nous a donné de son Esprit.
\VS{14}Et nous l'avons vu, et nous témoignons que le Père a envoyé le Fils pour être le sauveur du monde.
\VS{15}Quiconque confessera que Jésus est le Fils de Dieu, Dieu demeure en lui, et lui en Dieu.
\VS{16}Et nous, nous avons connu et cru en la charité que Dieu a pour nous. Dieu est charité ; et celui qui demeure dans la charité, demeure en Dieu, et Dieu en lui.
\VS{17}Tel il est, tels aussi nous sommes dans ce monde : C’est en cela que la charité est parfaite en nous, afin que nous ayons de l’assurance au jour du jugement.
\VS{18}Il n'y a point de crainte dans la charité, mais la parfaite charité bannit la crainte, car la crainte suppose un châtiment ; or celui qui craint n'est pas accompli dans la charité.
\VS{19}Nous l'aimons, parce qu'il nous a aimés le premier.
\VS{20}Si quelqu'un dit : J'aime Dieu, et qu’il haïsse son frère, c’est un menteur ; car comment celui qui n'aime point son frère, qu'il voit, peut-il aimer Dieu, qu’il ne voit pas ?
\VS{21}Et nous avons ce commandement de sa part, que celui qui aime Dieu, aime aussi son frère.
\Chap{5}
\TextTitle{La foi, principe qui triomphe des conflits avec le monde}
\VerseOne{}Quiconque croit que Jésus est le Christ, est né de Dieu, et quiconque aime celui qui l'a engendré, aime aussi celui qui est né de lui.
\VS{2}Nous connaissons à ceci que nous aimons les enfants de Dieu, lorsque nous aimons Dieu et que nous gardons ses commandements.
\VS{3}Car c'est en ceci que consiste notre amour pour Dieu : Que nous gardions ses commandements. Et ses commandements ne sont point pénibles.
\VS{4}Parce que tout ce qui est né de Dieu est victorieux du monde ; et ce qui nous fait remporter la victoire sur le monde, c'est notre foi.
\VS{5}Qui est celui qui a remporté la victoire sur le monde, sinon celui qui croit que Jésus est le Fils de Dieu ?
\VS{6}C'est ce Jésus, le Christ, qui est venu avec l’eau et le sang, et pas seulement avec l'eau, mais avec l'eau et le sang ; et c'est l'Esprit qui rend témoignage, or l'Esprit est la vérité.
\VS{7}Car il y en a trois dans le ciel qui rendent témoignage, le Père, la Parole, et le Saint-Esprit ; et ces trois-là ne sont qu'un\FTNT{Dieu est UN. Voir De. 6:4.}.
\VS{8}Il y en a aussi trois qui rendent témoignage sur la terre, à savoir l'Esprit, l'eau, et le sang, et ces trois-là se rapportent à un.
\TextTitle{Une assurance bénie}
\VS{9}Si nous recevons le témoignage des hommes, le témoignage de Dieu est plus grand, car le témoignage de Dieu consiste en ce qu’il a rendu témoignage à son Fils.
\VS{10}Celui qui croit au Fils de Dieu a le témoignage de Dieu en lui-même ; mais celui qui ne croit pas Dieu, le fait menteur, car il ne croit pas au témoignage que Dieu a rendu de son Fils.
\VS{11}Et c'est ici le témoignage, à savoir que Dieu nous a donné la vie éternelle, et cette vie est dans son Fils.
\VS{12}Celui qui a le Fils a la vie, celui qui n'a pas le Fils de Dieu n'a pas la vie.
\VS{13}Je vous ai écrit ces choses, à vous qui croyez au Nom du Fils de Dieu, afin que vous sachiez que vous avez la vie éternelle, et afin que vous croyiez au Nom du Fils de Dieu.
\VS{14}Et c'est ici l’assurance que nous avons en Dieu, que si nous demandons quelque chose selon sa volonté, il nous exauce.
\VS{15}Et si nous savons qu'il nous exauce, quelque chose que nous demandions, nous savons que nous possédons la chose que nous lui avons demandée.
\VS{16}Si quelqu'un voit son frère commettre un péché qui ne mène point à la mort\FTNT{Le péché qui mène à la mort c’est le blasphème contre le Saint-Esprit. Voir commentaire en Mt. 12:32.}, qu’il prie pour lui, et Dieu donnera la vie à ce frère. Il la donnera à ceux qui commettent un péché qui ne mène point à la mort. Il y a un péché qui mène à la mort ; je ne te dis point de prier pour ce péché-là.
\VS{17}Toute iniquité est un péché, mais il y a quelque péché qui ne mène pas à la mort.
\VS{18}Nous savons que quiconque est né de Dieu ne pèche point ; mais celui qui est engendré de Dieu se garde lui-même, et le malin ne le touche point.
\VS{19}Nous savons que nous sommes nés de Dieu, mais le monde entier est plongé dans le mal.
\TextTitle{Conclusion}
\VS{20}Or nous savons que le Fils de Dieu est venu, et il nous a donné l'intelligence pour connaître le Véritable ; et nous sommes dans le Véritable, en son Fils Jésus-Christ. Il est le vrai\FTNT{Dans Jean 17:3, le Père est présenté comme le Vrai Dieu ; le terme grec traduit par «~vrai~» dans Jean est aussi appliqué à Jésus dans ce passage. Jésus est donc le Vrai Dieu.}Dieu, et la vie éternelle.
\VS{21}Mes petits enfants, gardez-vous des idoles. Amen.
\PPE{}
\end{multicols}
