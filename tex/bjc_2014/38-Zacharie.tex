\ShortTitle{Zacharie}\BookTitle{Zacharie}\BFont
\noindent\hrulefill
{\footnotesize
\textit{
\bigskip
{\centering{}
\\Signifie : Yahweh se souvient
\\Thème : Les deux avènements du Messie
\\Auteur : Zacharie
\\Date de rédaction : 6ème siècle av. J.-C.\\}
}
%\bigskip
\textit{
\\Zacharie, contemporain d’Aggée, exerça son ministère en Juda au retour des exilés de Babylone - où il était né. Il annonça la venue du Messie et raconta de manière très précise différents épisodes de sa vie, également retrouvés dans le récit des évangiles. Il dévoila également quelques-uns des attributs du Sauveur, premièrement rejeté mais finalement accepté par le peuple juif pendant le millenium. On y découvre ainsi le Christ en tant que souverain sacrificateur, germe, serviteur, ange de l’Eternel, roi de paix, fils de David…\bigskip
}
}
\par\nobreak\noindent\hrulefill
\begin{multicols}{2}
\TextTitle{[Yahweh avertit son peuple]}
\Chap{1}
\VerseOne{}Le huitième mois de la deuxième année de Darius, la parole de Yahweh fut adressée à Zacharie, le prophète, fils de Bérékia, fils d’Iddo, en ces mots :
\VS{2}Yahweh a été extrêmement irrité contre vos pères.
\VS{3}C'est pourquoi tu leur diras : Ainsi parle Yahweh des armées : Revenez à moi, dit Yahweh des armées, et je reviendrai à vous, dit Yahweh des armées\FTNT{Joë. 2:12 ; Es. 31:6 ; Jé. 3:12.}.
\VS{4}Ne soyez point comme vos pères, auxquels s’adressaient les premiers prophètes, en disant : Ainsi a dit Yahweh des armées : Détournez-vous maintenant de vos mauvaises voies et de vos mauvaises actions ! Mais ils n’écoutèrent pas, ils ne furent pas attentifs à ce que je leur disais, dit Yahweh\FTNT{2 Ch. 29:6 ; Esd. 9:7 ; Né. 9:16 ; La. 5:7.}.
\VS{5}Vos pères où sont-ils ? Et ces prophètes-là pouvaient-ils vivre éternellement ?
\VS{6}Cependant mes paroles et mes ordonnances que j'avais données aux prophètes, mes serviteurs, n'ont-elles pas atteint vos pères ? De sorte qu’étant revenus, ils ont dit : Yahweh des armées nous a traités comme il avait résolu de le faire, selon nos voies et nos actions.
\TextTitle{[Le cavalier sur le cheval roux]}
\VS{7}Le vingt-quatrième jour du onzième mois, qui est le mois de Schebat, la deuxième année de Darius, la parole de Yahweh fut adressée à Zacharie, le prophète, fils de Bérékia, fils d’Iddo, en ces mots :
\VS{8}Je voyais de nuit une vision, et voici, un homme était monté sur un cheval roux, et il se tenait parmi des myrtes qui étaient dans un lieu creux ; il y avait derrière lui des chevaux roux, fauves et blancs\FTNT{Ap. 6:2-4.}.
\VS{9}Je dis : Mon Seigneur ! Que signifient ces choses ? Et l’Ange qui me parlait me dit : Je te montrerai ce que signifient ces choses.
\VS{10}L’homme qui se tenait parmi les myrtes répondit et dit : Ce sont ceux que Yahweh a envoyés pour parcourir la terre.
\VS{11}Et ils répondirent à l'Ange de Yahweh\FTNT{Voir commentaire en Ge. 16:7.} qui se tenait parmi les myrtes, et dirent : Nous avons parcouru la terre ; et voici, toute la terre est en repos et tranquille.
\TextTitle{[La compassion de Yahweh pour Jérusalem]}
\VS{12}Alors l'Ange de Yahweh répondit et dit : Yahweh des armées, jusqu'à quand n'auras-tu pas compassion de Jérusalem et des villes de Juda, contre lesquelles tu es irrité depuis soixante-dix ans\FTNT{Jérémie prophétisa que la captivité babylonienne durerait soixante-dix ans (Jé. 25:11-12 ; Jé. 29:10). Les soixante-dix ans commencèrent à la déportation de la famille royale à Babylone en 605 av. J.-C. (2 R. 24 ; Da. 1) et se terminèrent avec la première vague de retours conduite par Zorobabel (Esd. 1). Les Israélites furent emmenés en captivité en plusieurs vagues. Le livre d’Esdras raconte les deux premières. En 538 av. J.-C., Zorobabel mena la première vague et fut nommé gouverneur (Ag. 1:1). Le sacrificateur Josué  (Esd. 3:2) et les prophètes Aggée et Zacharie (Es. 5:1-2) le secondaient. Leur plus grand défi fut de rebâtir le temple. Puisque la seule tribu à retourner en masse fut celle de Juda, dès lors, le reste du peuple fut appelé «~les Juifs~» (Esd. 4:23).} ?
\VS{13}Yahweh répondit à l'Ange qui me parlait, par de bonnes paroles, par des paroles de consolation.
\VS{14}Puis l'Ange qui me parlait me dit : Crie, en disant : Ainsi parle Yahweh des armées : Je suis ému d'une grande jalousie pour Jérusalem et pour Sion,
\VS{15}et je suis extrêmement irrité contre les nations qui sont à leur aise ; car je n’étais que peu irrité, mais elles ont contribué au mal.
\VS{16}C'est pourquoi ainsi parle Yahweh : Je reviens à Jérusalem avec compassion, et ma maison y sera rebâtie, dit Yahweh des armées ; et le cordeau sera étendu sur Jérusalem.
\VS{17}Crie encore, et dis : Ainsi parle Yahweh des armées : Mes villes regorgeront encore de biens, et Yahweh consolera encore Sion, il choisira encore Jérusalem.
\TextTitle{[Les quatre cornes et les quatre forgerons]}
\VS{18}Puis je levai les yeux et je regardai ; et voici, quatre cornes\FTNT{Da. 7:7-11 ; Da. 8:22 ; Ap. 13:1-11.}.
\VS{19}Alors je dis à l'Ange qui me parlait : Que veulent dire ces choses ? Et il me répondit : Ce sont les cornes qui ont dispersé Juda, Israël et Jérusalem.
\VS{20}Puis Yahweh me fit voir quatre forgerons.
\VS{21}Je dis : Que viennent-ils faire ? Et il répondit et dit : Ce sont les cornes qui ont dispersé Juda, au point que personne ne lève la tête ; et ces forgerons sont venus pour les effrayer, et pour abattre les cornes des nations qui ont levé la corne contre le pays de Juda, pour le disperser.
\TextTitle{[L'homme tenant dans sa main le cordeau pour mésurer]}
\Chap{2}
\VerseOne{}Je levai encore mes yeux et je regardai, et voici, il y avait un homme tenant dans la main un cordeau pour mesurer,
\VS{2}auquel je dis : Où vas-tu ? Et il me répondit : Je vais mesurer Jérusalem, pour voir quelle est sa largeur et quelle est sa longueur.
\VS{3}Et voici, l'Ange qui me parlait s’avança, et un autre ange sortit à sa rencontre.
\TextTitle{[Yahweh, la gloire de Jérusalem]}
\VS{4}Il lui dit : Cours, et parle à ce jeune homme, et dis : Jérusalem sera habitée comme les villes sans murailles, à cause de la multitude d'hommes et de bêtes qui seront au milieu d'elle\FTNT{Né. 1:3 ; Né. 2:13.}.
\VS{5}Mais je serai pour elle, dit Yahweh, une muraille de feu tout autour, et je serai sa gloire au milieu d'elle\FTNT{Es. 60:19.}.
\VS{6}Ha ! Fuyez, fuyez hors du pays du nord ! dit Yahweh. Car je vous ai dispersés aux quatre vents des cieux, dit Yahweh.
\VS{7}Ha ! Sauve-toi, Sion, toi qui habites chez la fille de Babylone\FTNT{Jé. 50:8 ; Es. 48:20 ; Es. 52:11 ; Jé. 51:6.} !
\VS{8}Car ainsi parle Yahweh des armées, lequel après la gloire, il m'a envoyé vers les nations qui ont fait de vous leur proie ; car celui qui vous touche, touche à la prunelle de son œil\FTNT{De. 32:10 ; Ps. 17:8.}.
\VS{9}Car voici, je vais lever ma main contre elles, et elles seront la proie de ceux qui leur étaient asservis. Et vous saurez que Yahweh des armées m'a envoyé.
\VS{10}Pousse des cris d’allégresse et réjouis-toi, fille de Sion ! Car voici, je viens\FTNT{Jésus-Christ est Yahweh qui vient ! C’est la seconde venue de Jésus-Christ qui est évoquée ici (Es. 40:10-11 ; Za. 12:10-14 ; Za. 14:1-10 ; Ac. 1:1-11 ; Ap. 1:7-8 ; Ap. 22:12-17).}, et j'habiterai au milieu de toi, dit Yahweh.
\VS{11}Beaucoup de nations se joindront à Yahweh en ce jour-là, et deviendront mon peuple ; et j'habiterai au milieu de toi ; et tu sauras que Yahweh des armées m'a envoyé vers toi.
\VS{12}Yahweh possédera Juda comme sa part dans la terre sainte, et il choisira encore Jérusalem.
\VS{13}Que toute chair fasse silence devant la face de Yahweh ! Car il s'est réveillé de sa demeure sainte.
\TextTitle{[Yahweh enlève l'iniquité du pays]}
\Chap{3}
\VerseOne{}Puis Yahweh me fit voir Josué, le souverain sacrificateur, se tenant debout devant l'Ange de Yahweh\FTNT{Voir commentaire en Ge. 16:7.}, et Satan qui se tenait debout à sa droite, pour l’accuser.
\VS{2}Yahweh dit à Satan : Que Yahweh te réprime, ô Satan ! Que Yahweh, dis-je, qui a choisi Jérusalem, te réprime ! N’est-ce pas là un tison qui a été retiré du feu\FTNT{Jud. 1:9 ; Am. 4:11.} ?
\VS{3}Or Josué était vêtu de vêtements sales, et il se tenait debout devant l'Ange.
\VS{4}L’Ange prit la parole et dit à ceux qui étaient debout devant lui : Otez-lui ces vêtements sales ! Et il dit à Josué : Regarde, je t’enlève ton iniquité, et je te revêts d’habits de fête.
\VS{5}Je dis : Qu'on mette sur sa tête un turban pur ! Et ils mirent un turban pur sur sa tête, puis ils lui mirent des vêtements\FTNT{Ap. 19:8.}. L’Ange de Yahweh était présent.
\VS{6}Alors l'Ange de Yahweh fit à Josué cette déclaration, en disant :
\VS{7}Ainsi parle Yahweh des armées : Si tu marches dans mes voies, et si tu observes mes commandements, tu jugeras ma maison, tu garderas mes parvis, et je te donnerai libre accès parmi ceux qui se tiennent devant moi.
\VS{8}Ecoute maintenant, Josué, souverain sacrificateur, toi, et tes compagnons qui sont assis devant toi ! Car ce sont des hommes qui serviront de signes. Certainement voici je ferai venir mon serviteur, le Germe\FTNT{Le Germe est un autre nom de Jésus-Christ, notre Seigneur (Es. 4:2).}.
\VS{9}Car voici, quant à la pierre\FTNT{Jésus-Christ est le Rocher des âges (Es. 8:13-17; Ap. 5:1-7).} que j'ai mise devant Josué, sur cette pierre, qui n'est qu'une\FTNT{Cette pierre est UNE (~E’had~) c’est-à-dire indivisible (De. 6:4).}, il y a sept yeux. Voici, je graverai moi-même ce qui doit y être gravé, dit Yahweh des armées ; et j'ôterai en un jour l'iniquité de ce pays.
\VS{10}En ce jour-là, dit Yahweh des armées, chacun de vous appellera son prochain sous la vigne et sous le figuier.
\TextTitle{[Le peuple de Yahweh peut tout par son Esprit]}
\Chap{4}
\VerseOne{}Puis l'Ange qui me parlait revint, et il me réveilla comme un homme que l’on réveille de son sommeil.
\VS{2}Il me dit : Que vois-tu ? Et je répondis : Je regarde, et voici, il y a un chandelier tout en or, surmonté d’un vase et portant ses sept lampes, avec sept conduits pour les sept lampes qui sont au sommet du chandelier\FTNT{Ap. 1:12-13.} ;
\VS{3}et il y a deux oliviers près de lui, l'un à la droite du vase, et l'autre à sa gauche.
\VS{4}Alors je pris la parole et je dis à l'Ange qui me parlait : Mon Seigneur, que signifient ces choses ?
\VS{5}L'Ange qui me parlait répondit et me dit : Ne sais-tu pas ce que signifient ces choses ? Je dis : Non, mon Seigneur !
\VS{6}Alors il reprit et me dit : C'est ici la parole que Yahweh adresse à Zorobabel : Ce n'est point par la puissance ni par la force, mais par mon Esprit, dit Yahweh des armées.
\VS{7}Qui es-tu, grande montagne, devant Zorobabel ? Tu seras aplanie. Il fera sortir la pierre principale ; il y aura des sons éclatants : Grâce, grâce pour elle !
\TextTitle{[Yahweh encourage son peuple à achever l'oeuvre commencée]}
\VS{8}Aussi la parole de Yahweh me fut adressée en ces mots :
\VS{9}Les mains de Zorobabel ont fondé cette maison, et ses mains l'achèveront ; et tu sauras que Yahweh des armées m'a envoyé vers vous.
\VS{10}Car qui est-ce qui a méprisé le jour des faibles commencements ? Ils se réjouiront en voyant le niveau dans la main de Zorobabel.  Ces sept\FTNT{Les sept yeux de Yahweh sont aussi les sept yeux de l’Agneau (Ap. 5 : 6). Ces yeux représentent l’omniscience et l’omniprésence de Jésus-Christ (Za. 3:9 ; Za 4:10. ; Za. 14:7 ; Jn. 16:30 ; Ac. 1:24 ; Ap. 21:17).} sont les yeux de Yahweh qui parcourent toute la terre.
\VS{11}Je pris la parole et je lui dis : Que signifient ces deux oliviers\FTNT{L’identité de ces deux individus est inconnue.  Selon Ap. 11:3, ces deux hommes recevront des pouvoirs incroyables pour les trois années et demi de la grande tribulation qui précéderont le retour du Christ. Si quiconque tente de leur faire du mal ou d’interférer dans leur ministère et leur témoignage, «~… du feu sort de leur bouche et dévore leurs ennemis~», (Ap. 11:5). Ils auront aussi le pouvoir de provoquer la sécheresse et la famine sur la terre, tout comme l’avait fait Elie (1 R. 17:1-7 ; 2 R. 1:9-15 ; Lu. 4:25). Ils auront également le pouvoir de frapper la Terre par des plaies diverses, semblables à celles provoquées par Moïse (Chapitres 7, 8, 9, 10, 11 d’Exode ; Ap. 11:6).}, à la droite et à la gauche du chandelier ?
\VS{12}Je pris la parole pour la seconde fois et je lui dis : Que signifient ces deux branches d'olivier qui sont près des deux conduits d'or, d’où l'or découle ?
\VS{13}Il me répondit et dit : Ne sais-tu pas ce que signifient ces choses ? Et je dis : Non, mon Seigneur.
\VS{14}Et il dit : Ce sont les  deux fils oints, qui se tiennent devant le Seigneur de toute la terre.
\TextTitle{[La malédiction se répands sur Israël]}
\Chap{5}
\VerseOne{}Puis je me retournai, et levai mes yeux pour regarder ; et voici, un rouleau qui volait.
\VS{2}Alors il me dit : Que vois-tu ? Je répondis : Je vois un rouleau qui vole, dont la longueur est de vingt coudées, et la largeur de dix coudées.
\VS{3}Et il me dit : C'est l’exécration du serment qui sort sur la face de tout le pays ; car selon elle, quiconque d'entre ce peuple-ci vole, sera puni comme elle ; et selon elle, quiconque d'entre ce peuple parjure, sera puni comme elle.
\VS{4}Je déploierai cette exécration dit Yahweh des armées, et elle entrera dans la maison du voleur, et dans la maison de celui qui jure faussement en mon Nom, et elle logera au milieu de leur maison, et la consumera avec son bois et ses pierres.
\TextTitle{[L'épha au pays de Schinear]}
\VS{5}L’Ange qui me parlait sortit, et me dit : Lève maintenant tes yeux, et regarde ce qui sort là.
\VS{6}Et je dis : Qu'est-ce ? Et il répondit : C'est l’épha\FTNT{L'épha était une unité de mesure utilisée dans le commerce des céréales, souvent à des fins frauduleuses (De. 25:14 ; Mi. 6:10 ; Am. 8:5).} qui sort dehors. Puis il dit : C'est ici leur aspect dans tout le pays.
\VS{7}Et voici, on portait une masse de plomb, et une femme était assise au milieu de l'épha\FTNT{Zacharie voit une femme assise au milieu de l'épha. L'ange déclare : «~C'est la méchanceté ou l’iniquité~». Elle représente la grande prostituée décrite en Ap. 17, avec sa coupe d'or pleine de ses abominations et des impuretés de sa fornication (v. 4). Cette femme est la figure du «~mystère de l'iniquité qui opère déjà~» (2 Th. 2:7).}.
\VS{8}Il dit : C'est là l’iniquité\FTNT{L’iniquité ou la méchanceté.} ; puis il la repoussa dans l'épha, et il jeta la masse de plomb sur l’ouverture.
\VS{9}Je levai les yeux et je regardai, et voici deux femmes sortirent\FTNT{Les deux femmes ayant «~des ailes de cigogne~» apparaissent portées par le vent. Sous Moïse, cet oiseau devait être considéré comme impur (Lé. 11 : 19). Dans les Ecritures, le vent est constamment en relation avec le jugement (Job. 27:20-22 ; Job. 30:22 ; Es. 7:2 ; Es. 26:6 ; Es. 41:16). Elles soulèvent l'épha et l'emportent dans son lieu d'origine, le pays de Schinear, c’est-à-dire Babylone, pour lui bâtir une maison, au siège même de l'idolâtrie et de la révolte contre Dieu. (Ge. 11:2-9 ; 2 R. 17:24).}. Le vent soufflait dans leurs ailes : Elles avaient des ailes comme les ailes de la cigogne. Et elles enlevèrent l'épha entre la terre et le ciel.
\VS{10}Je dis à l'Ange qui me  parlait : Où emportent-elles l'épha ?
\VS{11}Il me répondit : C'est pour lui bâtir une maison dans le pays de Schinear\FTNT{Schinear ou Babylone (Ge. 10:6-12).} ; et quand elle sera prête, il sera déposé là, sur sa base.
\TextTitle{[Les quatre vents des cieux]}
\Chap{6}
\VerseOne{}Je levai encore les yeux et je regardai, et voici quatre chars\FTNT{Dans les Ecritures, les chars et les chevaux représentent toujours la puissance de Dieu exerçant un jugement sur la terre (Jé. 46:9-10 ; Joë. 2:3-11). Ce jugement concerne le monde entier (Ap. 6:1-8).} sortaient d'entre deux montagnes ; et ces montagnes étaient des montagnes d'airain.
\VS{2}Au premier char, il y avait des chevaux roux ; au deuxième char, des chevaux noirs,
\VS{3}au troisième char, des chevaux blancs, et au quatrième char, des chevaux tachetés, rouges.
\VS{4}Je pris la parole et je dis à l'Ange qui me parlait : Mon Seigneur, que veulent dire ces choses ?
\VS{5}L’Ange répondit et me dit : Ce sont les quatre vents des cieux, qui sortent du lieu où ils se tenaient devant le Seigneur de toute la terre.
\VS{6}Quant au char où sont les chevaux noirs, ils se dirigent vers le pays du nord, et les blancs sortent après eux ; les tachetés se dirigent vers le pays du midi.
\VS{7}Ensuite les rouges sortirent et demandèrent à aller parcourir la terre. L’Ange leur dit : Allez, et parcourez la terre ! Et ils parcoururent la terre.
\VS{8}Puis il m'appela, et me parla, en disant : Voici, ceux qui se dirigent vers le pays du nord ont apaisé mon Esprit dans le pays du nord.
\TextTitle{[Prophétie sur le règne du germe de Yahweh]}
\VS{9}La parole de Yahweh me fut adressée en ces mots :
\VS{10}Tu recevras les dons de ceux qui sont de retour de la captivité : Heldaï, Tobija et Jedaeja. Et tu iras toi-même ce même jour-là, et tu iras dans la maison de Josias, fils de Sophonie, où ils se sont rendus en arrivant de Babylone.
\VS{11}Tu prendras de l'argent et de l'or, et tu en feras des couronnes que tu mettras sur la tête de Josué, fils de Jotsadak, le souverain sacrificateur.
\VS{12}Tu lui diras : Ainsi parle Yahweh des armées : Voici un homme, dont le nom est Germe\FTNT{Es. 4:2.}, germera dans son lieu, et bâtira le temple de Yahweh\FTNT{C’est Yahweh, c’est-à-dire Jésus-Christ lui-même, qui bâtit son temple (Ps. 127:1-2 ; Mt. 16:18).}.
\VS{13}Oui, lui-même bâtira le temple de Yahweh ; et lui-même sera rempli de majesté. Il s’assiéra et dominera sur son trône, il sera Sacrificateur\FTNT{Jésus-Christ est Souverain Sacrificateur (Hé. 6:20 ; Hé. 7:1-28).}, étant sur son trône ; et il y aura un conseil de paix entre les deux.
\VS{14}Les couronnes seront pour Hélem, Tobija et Jedaeja, et pour Hen, fils de Sophonie, un souvenir dans le temple de Yahweh.
\VS{15}Ceux qui sont éloignés viendront, et travailleront au temple de Yahweh ; et vous saurez que Yahweh des armées m'a envoyé vers vous. Cela arrivera, si vous écoutez attentivement la voix de Yahweh, votre Dieu.
\TextTitle{[Yahweh dénonce le jeûne formaliste]}
\Chap{7}
\VerseOne{}La quatrième année du roi Darius, la parole de Yahweh fut adressée à Zacharie, le quatrième jour du neuvième mois, qui est le mois de Kisleu.
\VS{2}On avait envoyé à Béthel Scharetser et Réguem-Mélec avec ses gens, pour supplier Yahweh,
\VS{3}et pour parler aux sacrificateurs de la maison de Yahweh des armées, et aux prophètes, en disant : Dois-je pleurer au cinquième mois, et faire abstinence, comme j'ai déjà fait pendant plusieurs années ?
\VS{4}La parole de Yahweh des armées me fut adressée en ces mots :
\VS{5}Parle à tout le peuple du pays et aux sacrificateurs, et dis-leur : Quand vous avez jeûné et pleuré au cinquième mois et au septième, et cela depuis soixante-dix ans, avez-vous célébré ce jeûne par amour pour moi ?
\VS{6}Et quand vous buvez et mangez, n'est-ce pas vous qui mangez et vous qui buvez\FTNT{Es. 58:3-4.} ?
\VS{7}Ne connaissez-vous pas les paroles qu’a proclamées Yahweh par les premiers prophètes, lorsque Jérusalem était habitée et paisible avec ses villes à l’entour, et que le midi et la plaine étaient habités ?
\TextTitle{[Yahweh n'exauce pas les pécheurs]}
\VS{8}Puis la parole de Yahweh fut adressée à Zacharie en ces mots :
\VS{9}Ainsi parlait Yahweh des armées, en disant : Rendez véritablement la justice, et exercez la miséricorde et la compassion chacun envers son frère.
\VS{10}N’opprimez pas la veuve et l'orphelin, l'étranger et le pauvre, et ne méditez aucun mal dans vos cœurs chacun contre son frère\FTNT{Ex. 22:21 ; Es. 1:23 ; Jé. 5:28 ; Pr. 22:22-23.}.
\VS{11}Mais ils refusèrent d’être attentifs, ils eurent l'épaule rebelle, et ils endurcirent leurs oreilles pour ne pas entendre.
\VS{12}Ils rendirent leur cœur dur comme le diamant, pour ne pas écouter la loi et les paroles que Yahweh des armées adressait par son Esprit, par les premiers prophètes. C’est pourquoi  Yahweh des armées s’enflamma d’une grande colère.
\VS{13}Quand il appelait, ils n'ont pas écouté. Aussi n’ai-je pas écouté, quand ils ont appelé, dit Yahweh des armées\FTNT{Pr. 1:28 ; Es. 1:15 ; Jé. 11:11.}.
\VS{14}Je les ai dispersés comme par un tourbillon parmi toutes les nations qu'ils ne connaissaient pas ; le pays a été dévasté derrière eux, il n’y a plus eu ni allants ni venants ; et d’un pays de délices ils ont fait un désert.
\TextTitle{[Futur royaume d'Israël rétabli dans la justice]}
\Chap{8}
\VerseOne{}La parole de Yahweh des armées me fut encore adressée en ces mots :
\VS{2}Ainsi parle Yahweh des armées : Je suis jaloux pour Sion d'une grande jalousie, et je suis jaloux pour elle d’une grande fureur.
\VS{3}Ainsi parle Yahweh : Je retourne à Sion, et j'habiterai au milieu de Jérusalem ; et Jérusalem sera appelée ville fidèle ; et la montagne de Yahweh des armées sera appelée montagne sainte\FTNT{Es. 1:26.}.
\VS{4}Ainsi parle Yahweh des armées : Il y aura encore des vieillards et des femmes âgées, assis dans les rues de Jérusalem, et chacun aura son bâton à la main, à cause du grand nombre de leurs jours.
\VS{5}Les rues de la ville seront remplies de fils et de filles, jouant dans les rues.
\VS{6}Ainsi parle Yahweh des armées : S’il semble difficile aux yeux du reste de ce peuple que cela arrive, en ces jours-là, sera-t-il de même difficile à mes yeux ? dit Yahweh des armées.
\VS{7}Ainsi parle Yahweh des armées : Voici, je délivre mon peuple du pays de l'orient et du pays du soleil couchant.
\VS{8}Je les ramènerai, et ils habiteront au milieu de Jérusalem ; ils seront mon peuple, et je serai leur Dieu avec vérité et droiture.
\TextTitle{[Juger selon la vérité]}
\VS{9}Ainsi parle Yahweh des armées : Que vos mains soient fortifiées, vous qui entendez aujourd’hui ces paroles de la bouche des prophètes qui parurent au jour où la maison de Yahweh fut fondée, et où le temple allait être bâti\FTNT{Ag. 2:4.}.
\VS{10}Car avant ces jours-là, il n'y avait pas de salaire pour l'homme ni de salaire pour la bête ; et il n'y avait pas de paix pour ceux qui entraient et sortaient, à cause de la détresse ; et je lâchais tous les hommes les uns contre les autres.
\VS{11}Mais maintenant je ne serai pas pour le reste de ce peuple comme les premiers jours,  dit Yahweh des armées.
\VS{12}Car les semailles prospéreront, la semence de paix sera là ; la vigne rendra son fruit, et la terre donnera ses produits ; les cieux donneront leur rosée, et je ferai hériter toutes ces choses au reste de ce peuple.
\VS{13}De même que vous avez été en malédiction parmi les nations, ô maison de Juda et maison d'Israël, de même je vous délivrerai, et vous serez en bénédiction. Ne craignez pas, mais que vos mains soient fortifiées\FTNT{Ge. 1:11 ; Ap. 22:2.}.
\VS{14}Car ainsi parle Yahweh des armées : Comme j'ai eu la pensée de vous affliger, lorsque vos pères ont provoqué ma colère, dit Yahweh des armées, et que je ne m'en suis point repenti,
\VS{15}ainsi je reviens en arrière et j’ai résolu en ces jours de faire du bien à Jérusalem, et à la maison de Juda. Ne craignez pas !
\VS{16}Voici les choses que vous devez faire : Que chacun dise la vérité à son prochain ; jugez selon la vérité et prononcez un jugement en vue de la paix dans vos portes\FTNT{Ep. 4:25 ; Ex. 20:16 ; Mt. 19:18 ; Lu. 18:20.} ;
\VS{17}que personne ne projette du mal dans son cœur contre son prochain ; et n'aimez point le faux serment, car ce sont là des choses que je hais, dit Yahweh\FTNT{Ps. 5:5 ; Ps. 11:5 ; Pr. 6:16-19.}.
\VS{18}Puis la parole de Yahweh des armées me fut adressée en ces mots :
\VS{19}Ainsi parle Yahweh des armées : Le jeûne du quatrième mois, le jeûne du cinquième, le jeûne du septième et le jeûne du dixième seront changés pour la maison de Juda en joie et en allégresse, et en fêtes solennelles de réjouissance. Aimez donc la vérité et la paix\FTNT{Ep. 4:15.}.
\TextTitle{[Les nations reconnaissent que Yahweh est le seul Dieu]}
\VS{20}Ainsi parle Yahweh des armées : Il viendra encore des peuples et des habitants de plusieurs villes.
\VS{21}Les habitants d’une ville  iront à l'autre, en disant : Allons, allons implorer Yahweh et chercher Yahweh des armées ! Nous irons aussi !
\VS{22}Et beaucoup de peuples et de puissantes nations viendront rechercher Yahweh des armées à Jérusalem\FTNT{Jérusalem est appelée à devenir le centre d’adoration de la terre et la capitale du monde à cause de la présence de Dieu (Es. 66:23 ; Za. 14:16-21).}, et implorer Yahweh.
\VS{23}Ainsi parle Yahweh des armées : En ce jour-là, dix hommes de toutes les langues des nations saisiront le pan de la robe d'un homme Juif, et diront : Nous irons avec vous, car nous avons entendu que Dieu est avec vous.
\TextTitle{[Le jugement de Yahweh sur les nations]}
\Chap{9}
\VerseOne{}Oracle, parole de Yahweh sur le pays de Hadrac. Elle s’arrête sur Damas, car Yahweh a l’œil sur les hommes et sur toutes les tribus d'Israël.
\VS{2}Il s’arrête aussi sur Hamath, à la frontière de Damas, sur Tyr, et Sidon, quoique chacune d'elles soit fort sage.
\VS{3}Car Tyr s'est bâti une forteresse ; elle a amassé l'argent comme la poussière, et l’or fin comme la boue des rues\FTNT{Ez. 28:3-17.}.
\VS{4}Voici, le Seigneur l'appauvrira, et en la frappant, il jettera sa puissance dans la mer, et elle sera consumée par le feu\FTNT{Ez. 26:3-4.}.
\VS{5}Askalon le verra, et elle sera dans la crainte ; Gaza aussi le verra, et un violent tremblement la saisira ; Ekron aussi, car son espoir sera confondu. Et il n'y aura plus de roi à Gaza, et Askelon ne sera plus habitée\FTNT{So. 2:4.}.
\VS{6}Et le bâtard habitera à Asdod ; et j’abattrai l'orgueil des Philistins.
\VS{7}J'ôterai le sang de la bouche de chacun d'eux, et leurs abominations d'entre leurs dents ; et lui aussi restera pour notre Dieu, il sera comme un chef en Juda, et Ekron sera comme le Jébusien.
\VS{8}Je camperai autour de ma maison, pour la défendre contre une armée, contre les allants et les venants, et l’oppresseur ne passera plus près d’eux ; car maintenant mes yeux sont fixés sur elle.
\TextTitle{[Prophétie sur la première venue du Messie]}
\VS{9}Sois transportée d’allégresse, fille de Sion ! Pousse des cris de joie, fille de Jérusalem ! Voici, ton Roi vient à toi ; il est juste et vainqueur, il est monté sur un âne, sur un âne, le petit d'une ânesse\FTNT{Cette prophétie s’est accomplie 500 ans après. Jésus est effectivement entré à Jérusalem monté sur un âne (Mt. 21:1-11 ; Lu. 19:28-40 ; Jn. 12:12-19).}.
\TextTitle{[La vision du Messie pour Israël]}
\VS{10}Je détruirai les chars d'Ephraïm, et les chevaux de Jérusalem ; et les arcs de guerre seront aussi retranchés.  Et le Roi parlera de paix aux nations ; et sa domination s'étendra d’une mer à l’autre, depuis le fleuve jusqu'aux extrémités de la terre\FTNT{Es. 57:19 ; Ps. 2:8 ; Ps. 72:8.}.
\VS{11}Quant à toi, à cause de ton alliance scellée par le sang, je retirerai tes captifs de la fosse où il n'y a pas d'eau.
\VS{12}Retournez à la forteresse, captifs pleins d’espérance ! Aujourd'hui même je le déclare, je te rendrai le double.
\VS{13}Car je bande Juda comme un arc, je m’arme d’Ephraïm comme d’un arc, et j’exciterai tes enfants, ô Sion, contre tes enfants, ô Javan ! Je te rendrai pareille à l’épée d’un vaillant homme.
\VS{14}Alors Yahweh au-dessus d’eux apparaîtra, et ses dards partiront comme l'éclair, et le Seigneur, Yahweh, sonnera du shofar, il s’avancera dans le tourbillon du midi.
\VS{15}Yahweh des armées sera leur protecteur ; ils dévoreront, après avoir subjugué ceux qui tirent les pierres de fronde ; ils boiront, ils seront bruyants comme des hommes ivres, ils se rempliront de vin comme un bassin, et comme les coins de l'autel.
\VS{16}Yahweh, leur Dieu, les sauvera en ce jour-là, comme le troupeau de son peuple ; car ils sont les pierres d’une couronne  qui brilleront dans son pays.
\VS{17}Car combien est grande sa bonté ! Quelle beauté ! Le froment fera croître les jeunes hommes, et le vin doux rendra ses vierges éloquentes.
\TextTitle{[Yahweh rassemblera son peuple]}
\Chap{10}
\VerseOne{}Demandez à Yahweh la pluie\FTNT{Les pluies en Israël : En Israël, la saison de pluie commence généralement vers la fin du mois d’octobre avec de légères pluies qui ramollissent la terre (Ps. 65:10), et se poursuit ensuite par de fortes précipitations intermittentes durant deux ou trois jours, tout au long des mois de novembre et de décembre. Ces fortes précipitations étaient appelées dans les écritures «~la pluie de la première saison~» (en hébreu «~yoreb~» ou «~moreh~). Les fermiers dépendaient de la pluie de la première saison pour que la terre dure comme le roc soit rendue apte au labour et à l’ensemencement. Quand ces fortes précipitations s’achèvent, des pluies plus fines continuent encore de façon intermittente. Toutefois, à l’approche de la moisson,  la forte pluie revenait gonfler le grain et le fruit en préparation. Celle-ci était connue comme étant «~la pluie de l’arrière-saison~» (Jé. 5:24; Joë. 2:23-24 ; Os. 6:3).}, la pluie au temps de la dernière saison ! Yahweh produira des éclairs, et il vous donnera une abondante pluie, il donnera à chacun de l'herbe dans son champ.
\VS{2}Car les théraphim ont des paroles vaines, et les devins prophétisent le mensonge, ils profèrent des songes vains et consolent par la vanité. C'est pourquoi ils sont errants comme des brebis, ils sont malheureux, parce qu'il n'y a point de pasteur\FTNT{Mt. 9:36 ; Ez. 34:2 ; Jé. 23:21-30.}.
\VS{3}Ma colère s'est enflammée contre ces pasteurs, et je châtierai ces boucs ; car Yahweh des armées visite son troupeau, la maison de Juda ; et il les a rangés en bataille comme son cheval d'honneur.
\VS{4}De lui sortira l’Angle\FTNT{De lui (Juda) sortira l’Angle ou la pierre angulaire (Jésus-Christ), (1 Pi. 2:7 ; Es. 8:13-17).}, de lui sortira le clou, de lui sortira l'arc de bataille, et de lui sortiront tous les chefs ensemble.
\VS{5}Ils seront comme des vaillants hommes foulant la boue des rues dans la bataille, et ils combattront, parce que Yahweh sera avec eux ; et les cavaliers seront confus.
\VS{6}Car je fortifierai la maison de Juda, et je sauverai la maison de Joseph ; je les ramènerai, et je les ferai habiter en repos, parce que j'aurai compassion d'eux, et ils seront comme si je ne les avais point rejetés ; car je suis Yahweh, leur Dieu, et je les exaucerai.
\VS{7}Et ceux d'Ephraïm seront comme un héros, et leur cœur se réjouira comme par le vin ; leurs fils le verront, et se réjouiront ; leur cœur se réjouira en Yahweh.
\VS{8}Je les sifflerai et les rassemblerai, car je les rachète ; et ils seront multipliés comme ils l'ont été auparavant.
\VS{9}Et après que je les aurai dispersés parmi les peuples, ils se souviendront de moi dans les pays éloignés, et ils vivront avec leurs enfants, et ils reviendront.
\VS{10}Ainsi je les ramènerai du pays d'Egypte, je les rassemblerai de l'Assyrie, je les ferai venir au pays de Galaad, et au Liban, et il n'y aura point assez d'espace pour eux.
\VS{11}Il passera la mer de détresse, et il frappera les flots de la mer ; et toutes les profondeurs du fleuve seront desséchées ; l'orgueil de l'Assyrie sera abattu, et le sceptre d'Egypte sera ôté.
\VS{12}Je les fortifierai en Yahweh, et ils marcheront en son Nom, dit Yahweh.
\TextTitle{[Les houlettes du vrai berger]}
\Chap{11}
\VerseOne{}Liban, ouvre tes portes, et que le feu dévore tes cèdres !
\VS{2}Cyprès, gémis, car le cèdre est tombé, parce que les choses magnifiques ont été ravagées ! Chênes de Basan, gémissez, car la forêt inaccessible est coupée !
\VS{3}Les pasteurs poussent des cris de lamentations, parce que leur magnificence est ravagée ; on entend le rugissement des lionceaux, parce que l'orgueil du Jourdain est abattu.
\VS{4}Ainsi parle Yahweh, mon Dieu : Pais les brebis exposées au carnage !
\VS{5}Leurs possesseurs les égorgent, sans qu'on les tienne pour coupables, et celui qui les vend dit : Béni soit Yahweh, car je m’enrichis !  Et leurs pasteurs ne les épargnent pas.
\VS{6}Car je n’ai plus de pitié pour les habitants du pays, dit Yahweh ; et voici, je livre les hommes aux mains les uns des autres et aux mains de leur roi ; ils ravageront le pays, et je ne le délivrerai pas de leur main.
\VS{7}Alors je me mis donc à paître les brebis exposées au carnage, qui sont véritablement les plus misérables du troupeau. Puis je pris deux verges : J'appelai l'une Grâce, et l'autre Cordon ; et je me mis à paître les brebis.
\VS{8}Et je supprimai les trois pasteurs en un mois ; car mon âme était impatiente à leur sujet, et leur âme aussi avait pour moi du dégoût.
\VS{9}Et je dis : Je ne vous paîtrai plus ; que celle qui va mourir meure, et que celle qui va périr périsse, et que celles qui restent se dévorent la chair les unes les autres.
\VS{10}Puis je pris ma verge, appelée Grâce, et je la brisai, pour rompre mon alliance que j'avais traitée avec tous ces peuples.
\VS{11}Elle fut rompue en ce jour-là ; et les plus malheureuses brebis\FTNT{Les plus malheureuses des brebis sont le reste d’Israël.}, qui prirent garde à moi, reconnurent ainsi que c'était la parole de Yahweh.
\VS{12}Je leur dis : S'il vous semble bon, donnez-moi mon salaire ; sinon, ne me le donnez pas.  Alors ils pesèrent\FTNT{Mt. 26:15 ; Mt. 27:9-10.}  pour mon salaire trente pièces d'argent\FTNT{Selon la loi de Moïse, pour racheter un mâle de 20 à 60 ans, ayant fait un vœu, il fallait payer cinquante sicles d'argent (Lé. 27:3). Pour dédommager un préjudice causé par un bœuf ayant frappé un esclave, on devait donner trente sicles d'argent au maître de l'esclave et lapider le bœuf (Ex. 21:32). Or le prix du Seigneur a été estimé à trente sicles d’argent, comme pour les esclaves.}.
\VS{13}Yahweh me dit : Jette-le au potier, ce prix honorable auquel ils m’ont estimé ! Alors je pris les trente pièces d'argent, et les jetai dans la maison de Yahweh, pour le potier.
\VS{14}Puis je brisai ma seconde verge, appelée Cordon, pour rompre la fraternité entre Juda et Israël.
\TextTitle{[Caractéristiques du faux berger]}
\VS{15}Yahweh me dit : Prends-toi encore l'équipage d'un berger insensé\FTNT{}.
\VS{16}Car voici, je susciterai dans le pays un pasteur, qui ne visitera pas les brebis qui périssent ; il ne cherchera pas celles qui s’égarent, il ne guérira pas celles qui sont blessées, et il ne soutiendra pas celles qui sont saines, mais il dévorera la chair des plus grasses, et il déchirera jusqu’aux cornes de leurs pieds.
\VS{17}Malheur au pasteur inutile qui abandonne les brebis ! Que l’épée fonde sur son bras et sur son œil droit ! Que son bras se dessèche, et que son œil droit s’éteigne entièrement !
\TextTitle{[Jérusalem, une coupe d'étourdissement pour les nations]}
\Chap{12}
\VerseOne{}Oracle, parole de Yahweh, sur Israël. Ainsi parle Yahweh, qui a étendu les cieux et fondé la terre, et qui a formé l'esprit de l'homme au-dedans de lui :
\VS{2}Voici, je ferai de Jérusalem une coupe d'étourdissement pour tous les peuples d'alentour ; et aussi pour Juda dans le siège de Jérusalem\FTNT{Ap. 16:12-16.}.
\VS{3}En ce jour-là, je ferai de Jérusalem une pierre pesante pour tous les peuples ; tous ceux qui en porteront le poids seront entièrement écrasés, car toutes les nations de la terre s'assembleront contre elle.
\VS{4}En ce temps-là, dit Yahweh, je frapperai d'étourdissement tous les chevaux, et de folie ceux qui les monteront ; mais j’aurai les yeux ouverts sur la maison de Juda, et je frapperai d'aveuglement tous les chevaux des peuples.
\VS{5}Les chefs de Juda diront en leur cœur : Les habitants de Jérusalem sont notre force, par Yahweh des armées, leur Dieu.
\VS{6}En ce jour-là, je ferai des chefs de Juda comme un foyer de feu parmi du bois, et comme une torche enflammée parmi des gerbes ; ils dévoreront à droite et à gauche tous les peuples d'alentour ; et Jérusalem sera encore habitée à sa place, à Jérusalem.
\VS{7}Yahweh sauvera premièrement les tentes de Juda, afin que la gloire de la maison de David, la gloire des habitants de Jérusalem, ne s'élève point au-dessus de Juda.
\VS{8}En ce jour-là, Yahweh sera le protecteur des habitants de Jérusalem ; et le plus faible parmi eux sera en ce jour-là comme David ; la maison de David sera comme Dieu, comme l'Ange de Yahweh devant leur face.
\VS{9}En ce jour-là, je chercherai à détruire toutes les nations qui viendront contre Jérusalem.
\TextTitle{[Repentance et délivrance d'Israël]}
\VS{10}Et je répandrai sur la maison de David, et sur les habitants de Jérusalem, l'Esprit de grâce et de supplications\FTNT{Joë. 2:28-30.}, et ils regarderont vers moi\FTNT{Au retour du Messie, il y aura une repentance et une conversion nationale d’Israël (Ro. 11:26).}, celui qu’ils ont percé, et ils pleureront sur lui\FTNT{Celui qu’ils ont percé : Il est question ici du Seigneur Jésus, le Messie (Ap. 1:7).}, comme on pleure sur un fils unique, et ils pleureront amèrement sur lui, comme quand on pleure sur un premier-né.
\VS{11}En ce jour-là, il y aura un grand deuil à Jérusalem, comme le deuil d'Hadadrimmon dans la vallée de Meguiddon.
\VS{12}Le pays sera dans le deuil, chaque famille à part : La famille de la maison de David à part, et les femmes de cette maison-là à part ; la famille de la maison de Nathan à part, et les femmes de cette maison-là à part.
\VS{13}La famille de la maison de Lévi à part, et les femmes de cette maison-là à part ; la famille de Schimeï à part, et ses femmes à part.
\VS{14}Toutes les autres  familles, chaque famille à part, et leurs femmes à part.
\TextTitle{[Dieu frappe les faux prophètes]}
\Chap{13}
\VerseOne{}En ce jour-là, il y aura une source ouverte en faveur de la maison de David et des habitants de Jérusalem, pour le péché et pour la souillure.
\VS{2}En ce jour-là, dit Yahweh des armées, je retrancherai du pays les noms des faux dieux, et on n'en fera plus mention. J'ôterai aussi du pays les faux prophètes et l'esprit d'impureté.
\VS{3}Et il arrivera que si quelqu'un prophétise encore, son père et sa mère qui l’ont engendré, lui diront : Tu ne vivras plus ; car tu as prononcé des mensonges au Nom de Yahweh ; et son père et sa mère qui l’ont engendré, le transperceront quand il prophétisera.
\VS{4}En ce jour-là, les prophètes seront confus de leurs visions, quand ils prophétiseront ; et ils ne revêtiront plus un manteau de poil pour mentir.
\VS{5}Chacun d’eux dira : Je ne suis pas prophète, mais je suis laboureur, car on m'a appris à gouverner du bétail dès ma jeunesse.
\VS{6}Et si on lui demande : Que veulent donc dire ces blessures que tu as aux mains ? Et il répondra : C’est dans la maison de mes amis qu’on me les a faites.
\TextTitle{[Prophétie sur le vrai berger, le Messie]}
\VS{7}Epée, réveille-toi contre mon Berger\FTNT{Mon Berger : Il est question de Jésus-Christ, le Bon Berger (Ps. 23 ; Jn. 10:1-17).}, et sur l'homme qui est mon compagnon ! dit Yahweh des armées frappe le Berger, et les brebis seront dispersées\FTNT{Frappe le Berger : Cette prophétie fait référence à la crucifixion du Seigneur Jésus-Christ (Ge. 3:15 ; Mt. 26:31 ; Mc. 14:27 ; Mc. 14:50 ; Mc. 15:19).} ; et je tournerai ma main vers les faibles.
\TextTitle{[Le reste de Yahweh épuré à travers l'épreuve]}
\VS{8}Dans tout le pays, dit Yahweh, les deux tiers seront retranchées et périront, et l’autre tiers restera.
\VS{9}Je mettrai ce tiers dans le feu, et je le purifierai comme on purifie l'argent, je les éprouverai comme on éprouve l'or. Il invoquera mon Nom, et je l'exaucerai ; je dirai : C'est ici mon peuple ! Et il dira : Yahweh est mon Dieu\FTNT{1 Pi. 1:6-7 ; Ps. 50:15 ; Ps. 91:15 ; Ps. 144:15.} !
\TextTitle{[Imminence du jour de Yahweh]}
\Chap{14}
\VerseOne{}Voici, le jour de Yahweh\FTNT{L’expression «~le jour du Seigneur~» ou «~le jour de Yahweh~» est utilisée dix-neuf fois dans le Tanakh (Es. 2:12 ;  Es. 13:6 ; Es. 13:9 ; Ez. 13:5 ; Ez. 30:3 ; Joë. 1:15 ; Joë. 2:1 ; Joë. 2:11 ; Joë. 2:31 ;  Joë. 3:14 ; Am. 5:18-20 ; Ab. 1:15 ; So. 1:7 ; So. 1:14 ; Za. 14:1 ; Mal. 4:5) et quatre fois dans les textes de la nouvelle alliance (Ac. 2:20 ; 2 Th. 2:2 ; 2 Pi. 3:10 ; Ap. 6:17 ; Ap. 16:14). Cette expression désigne habituellement des événements qui se déroulent à la fin des temps (Es. 7:18-25). Elle désigne un espace de temps au cours duquel Dieu va intervenir personnellement dans l’histoire des hommes. Appelé  «~jour de colère~», «~jour de  visitation~», et «~grand jour du Dieu Tout-Puissant~» ; il se réfère ainsi à un accomplissement encore futur, quand la colère de Dieu viendra s’abattre sur l’Israël qui n’aura pas cru (Es. 22 ; Jé. 30:1-17 ; Joë. 1 et 2 ; Am. 5 ; So. 1) et sur tous les incrédules du monde (Ez. 38 et 39 ; Za. 14). Ce jour sera aussi un temps de salut puisque Dieu va délivrer «~le reste~» d’Israël, accomplissant ainsi sa promesse selon laquelle «~tout Israël sera sauvé~» (Ro. 11:26) : il pardonnera leurs péchés et restaurera le peuple qu’Il s’est choisi sur la terre promise à Abraham (Es. 10:27 ; Jé. 30:19-31 ; Mi. 4 ; Za. 13).} arrive, et tes dépouilles seront partagées au milieu de toi, Jérusalem.
\VS{2}Je rassemblerai toutes les nations à Jérusalem pour qu’elles lui fassent la guerre\FTNT{Joë. 3 ; Ap. 16:12-16.} ; la ville sera prise, les maisons pillées, et les femmes violées ; la moitié de la ville ira en captivité, mais le reste du peuple ne sera pas retranché de la ville.
\VS{3}Yahweh sortira, et il combattra contre ces nations, comme il a combattu au jour de la bataille.
\TextTitle{[Retour visible et en gloire du Seigneur]}
\VS{4}Ses pieds se poseront en ce jour sur la montagne des Oliviers\FTNT{Ce sont les pieds de Jésus-Christ (Ac. 1:10-11).}, qui est vis-à-vis de Jérusalem, du côté de l’orient ; et la montagne des Oliviers se fendra par le milieu, à l'orient et à l'occident, de sorte qu'il y aura une très grande vallée ; une moitié de la montagne reculera vers le nord, et l'autre moitié vers le midi.
\VS{5}Vous fuirez alors dans la vallée de mes montagnes ; car la vallée des montagnes s’étendra jusqu'à Atzel ; et vous fuirez comme vous avez fui devant le tremblement de terre, aux jours d’Ozias, roi de Juda. Alors Yahweh, mon Dieu, viendra, et tous les saints seront avec lui\FTNT{Ce passage confirme clairement que Jésus-Christ est Yahweh (1 Th. 3:13 ; Jud. 14-15 ; Es. 34:5 ; Es. 40:10-11 ; Es. 62:11-15).}.
\VS{6}Et il arrivera qu'en ce jour-là, la lumière précieuse ne sera pas mêlée de ténèbres.
\VS{7}Ce sera un jour unique, connu de Yahweh, et qui ne sera ni jour ni nuit ; mais au temps du soir il y aura de la lumière.
\VS{8}Et il arrivera qu'en ce jour-là, des eaux vives\FTNT{Ez. 47:1-12 ;  Ap. 22:1-2.} sortiront de Jérusalem, la moitié d'elles coulera vers la mer orientale, et l'autre moitié, vers la mer occidentale ; il en sera ainsi été et hiver.
\TextTitle{[Le royaume messianique]}
\VS{9}Yahweh sera Roi sur toute la terre ; en ce jour-là, Yahweh sera Un, et son nom sera Un\FTNT{Littéralement «~E’had~». Le jour du Seigneur est un comme le jour un de Ge. 1:5. Yahweh est Un et non trois (De. 6:4). Son Nom est Un (Ac. 4:12).}.
\VS{10}Toute la terre deviendra comme la plaine, depuis Guéba jusqu'à Rimmon, au midi de Jérusalem ; et Jérusalem sera exaltée et restera à sa place, depuis la porte de Benjamin, jusqu'à l'endroit de la première porte, jusqu'à la porte des angles, et depuis la tour de Hananeel, jusqu'aux pressoirs du roi.
\VS{11}On habitera dans son sein, et il n'y aura plus d'interdit, mais Jérusalem sera habitée en sûreté.
\VS{12}Voici la plaie dont Yahweh frappera tous les peuples qui auront fait la guerre contre Jérusalem ; il fera que la chair de chacun tombera en pourriture tandis qu’ils seront sur leurs pieds, leurs yeux tomberont en pourriture dans leurs orbites, et leur langue tombera en pourriture dans leur bouche.
\VS{13}Et il arrivera en ce jour-là que Yahweh produira un grand trouble parmi eux ; car chacun saisira la main de son prochain, et la main de l'un s'élèvera contre la main de l'autre.
\VS{14}Juda combattra aussi dans Jérusalem, et les richesses de toutes les nations d'alentour y seront amassées : L'or, l'argent, et des vêtements en très grand nombre.
\VS{15}Et la même plaie sera sur les chevaux, les mulets, les chameaux, les ânes et sur toutes les bêtes qui seront dans ces camps, cette plaie sera semblable à l’autre.
\TextTitle{[Adoration de Yahweh des armées dans le royaume]}
\VS{16}Et il arrivera que tous ceux qui resteront de toutes les nations venues contre Jérusalem, monteront en foule chaque année pour adorer le Roi, Yahweh des armées, et pour célébrer la fête des tabernacles.
\VS{17}S’il y a des familles de la terre qui ne montent pas à Jérusalem, pour adorer le Roi, Yahweh des armées, la pluie ne tombera pas sur elles.
\VS{18}Si la famille d'Egypte ne monte pas, si elle ne vient pas, la pluie ne tombera pas sur elle ; elle sera frappée de la plaie dont Yahweh frappera les nations qui ne monteront pas pour célébrer la fête des tabernacles.
\VS{19}Ce sera la peine du péché de l’Egypte, et du péché de toutes les nations qui ne monteront pas pour célébrer la fête des tabernacles.
\VS{20}En ce jour-là, il sera écrit sur les clochettes des chevaux : Sainteté à Yahweh ! Et les chaudières dans la maison de Yahweh seront comme les coupes devant l'autel.
\VS{21}Toute chaudière qui sera à Jérusalem et dans Juda, sera consacrée à Yahweh des armées ; et tous ceux qui offriront des sacrifices viendront, et s’en serviront pour cuire  les viandes ; et il n'y aura plus de marchands dans la maison de Yahweh des armées, en ce jour-là.
\PPE{}
\end{multicols}
