\ShortTitle{So.}\BookTitle{Sophonie}\BFont
\noindent\hrulefill
{\footnotesize
\textit{
\bigskip
{\centering{}
\\Auteur : Sophonie
\\(Heb. : Tsephanyah)
\\Signification : Yahweh a caché, protégé
\\Thème : Le jour de Yahweh
\\Date de rédaction : 7\up{ème} siècle av. J.-C.\\}
}
\textit{
\\De lignée royale, Sophonie exerça son service dans le royaume de Juda au temps du roi Josias et fut contemporain de Jérémie, Habakuk, Ezéchiel et Abdias. A une époque où l'iniquité s'était accrue au point où les quelques personnes fidèles à Dieu étaient persécutées, Sophonie fut suscité par Yahweh pour annoncer le jugement de Juda, d'Israël et de quelques nations païennes.\bigskip
}
}
\par\nobreak\noindent\hrulefill
\begin{multicols}{2}
\Chap{1}
\TextTitle{Yahweh annonce son jugement sur Juda, conséquence de son idolâtrie}
\VerseOne{}C'est ici la parole de Yahweh qui fut adressée à Sophonie, fils de Cuschi, fils de Guedalia, fils d'Amaria, fils d'Ezéchias, du temps de Josias, fils d'Amon, roi de Juda.
\VS{2}Je ferai entièrement périr toutes choses de dessus cette terre, dit Yahweh.
\VS{3}Je ferai périr l'homme et le bétail ; je consumerai les oiseaux des cieux et les poissons de la mer ; et la ruine arrivera aux méchants, et je retrancherai les hommes de dessus cette terre, dit Yahweh.
\VS{4}J'étendrai ma main sur Juda, et sur tous les habitants de Jérusalem ; je retrancherai de ce lieu-ci le reste de Baal\FTNT{Voir commentaire en \vref{Jg. 2:13}.}, les noms des prêtres des faux dieux, les prêtres,
\VS{5}ceux qui se prosternent sur les toits devant l'armée des cieux, ceux qui se prosternent devant Yahweh, qui jurent par lui, et qui jurent aussi par Malcom\FTNT{\vref{2 R. 17:33} ; \vref{2 R. 23:11-12} ; \vref{Jé. 19:13}.},
\VS{6}ceux qui se détournent de Yahweh, ceux qui n'ont point cherché Yahweh, qui ne l'ont point consulté.
\VS{7}Silence, à cause de la présence du Seigneur Yahweh, car le jour de Yahweh est proche\FTNT{Voir commentaire en \vref{Za. 14:1}.} ; Yahweh a préparé le sacrifice, il a invité ses conviés.
\VS{8}Et il arrivera au jour du sacrifice de Yahweh que je punirai les chefs, et les enfants du roi, et tous ceux qui portent des vêtements étrangers.
\VS{9}Et je punirai, en ce jour-là, tous ceux qui sautent par-dessus le seuil, et ceux qui remplissent de violence et de fraude la maison de leurs maîtres.
\VS{10}Et en ce jour-là dit Yahweh, il y aura de grands cris vers la porte des poissons, et des hurlements vers la seconde partie de la ville, et une grande désolation sur les collines.
\VS{11}Vous qui habitez dans Macthesch\FTNT{Macthesch était un bas-quartier de Jérusalem où se tenait les marchés.}, hurlez ! Car tous ceux qui trafiquaient ont été détruits, et tous ceux qui apportaient de l'argent ont été retranchés.
\VS{12}Et il arrivera en ce temps-là que je fouillerai Jérusalem avec des lampes, que je punirai les hommes qui sont figés sur leurs lies, et qui disent dans leurs cœurs : Yahweh ne nous fera ni bien ni mal.
\VS{13}Leurs biens seront au pillage et leurs maisons en désolation ; et ils auront bâti des maisons, mais ils ne les habiteront pas ; ils auront planté des vignes, mais ils n'en boiront pas le vin.
\VS{14}Le grand jour de Yahweh est proche, il est proche, et il se hâte beaucoup ; le jour de Yahweh n'est que bruit ; celui qui est dans l'amertume, crie de toute sa force. Là sont les hommes vaillants\FTNT{\vref{Jé. 30:7} ; \vref{Joë. 2:11} ; \vref{Am. 5:18}.}.
\VS{15}Ce jour est un jour de fureur, un jour de détresse et d'angoisse, un jour de bruit éclatant et effrayant, un jour de ténèbres et d'obscurité, un jour de nuées et de brouillards ;
\VS{16}un jour de shofar et de cris de guerre contre les villes fortifiées, et contre les hautes tours.
\VS{17}Je mettrai les hommes dans la détresse, et ils marcheront comme des aveugles, parce qu'ils ont péché contre Yahweh ; et leur sang sera répandu comme de la poussière, et leur chair comme des ordures.
\VS{18}Ni leur argent ni leur or ne pourront les délivrer au jour de la fureur de Yahweh ; et tout ce pays sera dévoré par le feu de sa jalousie, car il se hâtera de consumer tous les habitants de ce pays\FTNT{\vref{Ez. 7:19} ; \vref{Pr. 11:4}.}.
\Chap{2}
\TextTitle{Yahweh invite Israël à la repentance}
\VerseOne{}Examinez-vous, examinez-vous avec soin ô nation non désirée\FTNT{\vref{1 Th. 5:21} ; \vref{2 Co. 13:5} ; \vref{Ep. 5:10}.} !
\VS{2}Avant que le décret enfante, et que le jour passe comme la balle ; avant que l'ardeur de la colère de Yahweh vienne sur vous, avant que le jour de la colère de Yahweh vienne sur vous !
\VS{3}Vous, tous les pauvres du pays, qui faites ce qu'il ordonne, cherchez Yahweh, cherchez la justice, cherchez l'humilité ; peut-être serez-vous protégés au jour de la colère de Yahweh\FTNT{\vref{Am. 5:15}.}.
\VS{4}Mais Gaza sera abandonnée, et Askalon sera en désolation ; on chassera les habitants d'Asdod en plein midi, et Ekron sera arrachée\FTNT{\vref{Am. 8:9} ; \vref{Za. 9:5}.}.
\VS{5}Malheur aux habitants de la contrée maritime, à la nation des Kéréthiens ! La parole de Yahweh est contre vous ; Canaan, qui est le pays des Philistins, je te détruirai, si bien que, personne n'y habitera.
\VS{6}Et la contrée maritime sera des pâturages, des demeures pour les bergers, et des parcs pour les troupeaux.
\VS{7}Et cette contrée sera pour le reste de la maison de Juda ; ils paîtront dans ces lieux-là, et le soir ils feront leur gîte dans les maisons d'Askalon ; car Yahweh, leur Dieu, les visitera, et il ramènera leurs captifs.
\VS{8}J'ai entendu les insultes de Moab, et les outrages des fils d'Ammon, quand ils ont diffamé mon peuple, et l'ont bravé sur leur frontière\FTNT{\vref{Ez. 25:3-6}.}.
\VS{9}C'est pourquoi, je suis vivant, dit Yahweh des armées, le Dieu d'Israël, Moab sera comme Sodome, et les fils d'Ammon comme Gomorrhe, un lieu couvert d'orties, et une carrière de sel et de désolation à jamais ; les restes de mon peuple les pilleront, et les restes de ma nation les posséderont.
\VS{10}Ceci leur arrivera en échange de leur orgueil, parce qu'ils ont usé d'insultes et d'arrogance, contre le peuple de Yahweh des armées\FTNT{\vref{Es. 16:6} ; \vref{Jé. 48:29}.}.
\VS{11}Yahweh sera terrible contre eux, car il anéantira tous les dieux du pays ; et on se prosternera devant lui, chacun de son lieu, même dans toutes les îles des nations\FTNT{\vref{Mal. 1:11} ; \vref{Jn. 4:21}.}.
\VS{12}Vous aussi, habitants de Cusch, vous serez blessés à mort par mon épée.
\VS{13}Il étendra aussi sa main sur le nord, et il détruira l'Assyrie, et il fera de Ninive une désolation, dans un lieu aride comme un désert.
\VS{14}Et les troupeaux feront leur gîte au milieu d'elle, et toutes les bêtes des nations, même le pélican et le hérisson, habiteront parmi les chapiteaux de ses colonnes ; la voix des oiseaux retentira à la fenêtre, la désolation sera au seuil, parce qu'il en aura abattu les cèdres\FTNT{\vref{Es. 14:23} ; \vref{Es. 34:11}.}.
\VS{15}C'est là cette ville remplie de joie, qui se tenait assurée, et qui disait en son cœur : C'est moi, et il n'y en a point d'autre que moi ! Comment a-t-elle été réduite en désert, pour être le repère des bêtes ? Quiconque passera près d'elle sifflera et secouera sa main.
\Chap{3}
\TextTitle{Israël persiste dans l'immoralité}
\VerseOne{}Malheur à la ville immonde et souillée et qui ne fait qu'opprimer !
\VS{2}Elle n'a point écouté la voix, elle n'a point reçu d'instruction, elle ne s'est point confiée en Yahweh, elle ne s'est point approchée de son Dieu.
\VS{3}Ses chefs au milieu d'elle sont des lions rugissants, et ses juges sont des loups du soir, qui ne gardent pas les os pour les ronger le matin\FTNT{\vref{Ez. 22:27} ; \vref{Pr. 28:15}.}.
\VS{4}Ses prophètes sont des téméraires, et des hommes infidèles ; ses prêtres ont souillé les choses saintes, ils ont fait violence à la loi\FTNT{\vref{Jé. 23:11-32}.}.
\VS{5}Yahweh est juste au milieu d'elle, il ne commet point d'iniquité\FTNT{\vref{De. 32:4}.}. Chaque matin il met en lumière son jugement, il n'y manque pas ; mais celui qui est inique ne sait ce que c'est que d'avoir honte.
\VS{6}J'ai exterminé les nations, et leurs forteresses ont été désolées ; j'ai rendu désertes leurs places, si bien que personne n'y passe ; leurs villes ont été détruites, sans qu'il y soit resté un seul homme, et sans qu'il y ait aucun habitant.
\VS{7}Et je disais : Au moins tu me craindras, tu recevras instruction, et sa demeure ne sera pas retranchée, quelque soit la punition que je lui envoie. Mais ils se sont levés de bon matin, ils ont corrompu toutes leurs actions.
\VS{8}C'est pourquoi attendez-moi, dit Yahweh, au jour où je me lèverai pour le butin ; car j'ai résolu de rassembler les nations et de réunir les royaumes, pour répandre sur eux mon indignation, et toute l'ardeur de ma colère ; car tout le pays sera dévoré par le feu de ma jalousie.
\TextTitle{Un reste trouve refuge en Yahweh}
\VS{9}Alors je transformerai les langues\FTNT{Il est question ici de la conversion des peuples issus des nations (\vref{Ap. 7:9-17}).} des nations en des langues pures, afin qu'elles invoquent toutes le Nom de Yahweh, pour qu'elles le servent d'un commun accord.
\VS{10}Mes adorateurs qui sont au-delà des fleuves de Cusch, à savoir la fille de mes dispersés, m'apporteront mes offrandes\FTNT{\vref{Es. 19:21} ; \vref{Es. 27:13} ; \vref{Ps. 68:31-32} ; \vref{Ps. 72:10-11}.}.
\VS{11}En ce jour-là, tu ne seras plus confuse à cause de toutes tes actions, par lesquelles tu as péché contre moi ; parce qu'alors j'aurai ôté du milieu de toi ceux qui se réjouissent de ton orgueil, et désormais tu ne t'enorgueilliras plus de la montagne de ma sainteté.
\VS{12}Et je laisserai au milieu de toi un peuple humble et faible, et il mettra sa confiance dans le Nom de Yahweh.
\VS{13}Les restes d'Israël ne commettront point d'iniquité, et ne proféreront point de mensonge, et il n'y aura point dans leur bouche de langue trompeuse ; aussi ils paîtront et se reposeront, et il n'y aura personne qui les épouvante.
\TextTitle{Israël délivré et restauré}
\VS{14}Réjouis-toi avec chant de triomphe, fille de Sion ! Pousse des cris de réjouissance, ô Israël ! Réjouis-toi et triomphe de tout ton cœur, fille de Jérusalem !
\VS{15}Yahweh a aboli ta condamnation, il a éloigné ton ennemi. Le Roi d'Israël, Yahweh, est au milieu de toi ; tu ne verras plus de mal\FTNT{\vref{Ps. 46:5-6} ; \vref{Col. 2:14}.}.
\VS{16}En ce temps-là, on dira à Jérusalem : Ne crains point Sion, que tes mains ne défaillent point !
\VS{17}Yahweh, ton Dieu, est au milieu de toi comme le Puissant qui sauve ; il se réjouira à cause de toi d'une grande joie ; il se taira à cause de son amour, et se réjouira à cause de toi avec chant de triomphe.
\VS{18}Je rassemblerai ceux qui sont tristes à cause de l'assemblée solennelle, ils sont sortis de toi ; sur eux pèse l'opprobre.
\VS{19}Voici, je détruirai en ce temps-là tous ceux qui t'auront affligé ; je sauverai la boiteuse, je recueillerai celle qui avait été chassée, et je les ferai louer et devenir célèbres, dans tous les pays où ils auront été couverts de honte.
\VS{20}En ce temps-là, je vous ramènerai, et en ce temps-là je vous rassemblerai ; car je vous rendrai célèbres et un sujet de louange parmi tous les peuples de la terre, quand je ramènerai vos captifs sous vos yeux, dit Yahweh.
\PPE{}
\end{multicols}
