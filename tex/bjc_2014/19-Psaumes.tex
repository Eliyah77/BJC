\ShortTitle{Psaumes}\BookTitle{Psaumes}\BFont
\noindent\hrulefill
{\footnotesize
\textit{
\bigskip
{\centering{}
\\(Tehilim)
\\Signifie : Louanges 
\\Thème : La louange
\\Auteur : Plusieurs écrivains, notamment David
\\Date de rédaction : A compter du 10ème siècle et au-delà\\}
}
%\bigskip
\textit{
\\Le terme «~psaume~» désigne un poème chanté avec l’accompagnement d’un instrument. En effet, c’est ainsi que furent initialement contés les récits de la création divine, la captivité ou encore la gloire de Jérusalem. Expressions de la repentance, l’angoisse ou la vulnérabilité de l’homme, les psaumes étaient parfois des prières adressées à Dieu dans la détresse. D’autres encore parlaient de la fin de temps et annonçaient prophétiquement les souffrances de Christ.
%\bigskip
\\Utilisé comme recueil de chants du second temple,  le livre des psaumes exalte la grandeur de Dieu, sa souveraineté, sa miséricorde et son omniscience. Il est le fruit d’une grande variété d’expériences spirituelles du fait de la diversité de ses auteurs. 
%\bigskip
\\De plus, il contient une richesse de styles considérable, ce qui en fait le chef-d’œuvre de la poésie hébraïque.\bigskip
}
}
\par\nobreak\noindent\hrulefill
\begin{multicols}{2}
\TextTitle{[Destinées du juste et du pécheur]}
\Chap{1}
\VerseOne{}Heureux l'homme qui ne marche pas selon le conseil des méchants, et qui ne s'arrête pas sur la voie des pécheurs, qui ne s'assied pas dans l’assemblée des moqueurs\FTNT{Jé. 15:17 ; 1 Co. 15 : 33 ; Ep. 5:11.},
\VS{2}mais qui prend plaisir dans la loi de Yahweh, et qui médite sa loi jour et nuit\FTNT{De. 6:6 ; De. 17:19 ; Jos. 1:8.}.
\VS{3}Il est comme un arbre planté près des ruisseaux d'eaux, qui rend son fruit en sa saison, et dont le feuillage ne se flétrit point\FTNT{Jé. 17:7-8 ; Ez. 47:12 ; Jn. 15:8 ; Ap. 22:2.}. Et ainsi tout ce qu'il fera réussira.
\VS{4}Il n'en est pas ainsi des méchants : Ils sont comme la balle que le vent chasse au loin\FTNT{Job. 21:17-18 ; Os. 13:3.}.
\VS{5}C'est pourquoi les méchants ne résistent pas dans le jugement ni les pécheurs dans l'assemblée des justes.
\VS{6}Car Yahweh connaît la voie des justes, mais la voie des méchants périra.
\TextTitle{[Complot des nations contre le messie]}
\Chap{2}
\VerseOne{}Pourquoi cette agitation parmi les nations, et pourquoi les peuples projettent-ils des choses vaines ?
\VS{2}Pourquoi les rois de la terre se lèvent-ils en personne, et les princes se liguent-ils avec eux contre Yahweh, et contre son Messie\FTNT{Cette prophétie concerne le complot des Juifs, de Pilate et d’Hérode contre Jésus-Christ, notre Seigneur. Il est également question du gouvernement mondial dirigé par Satan. Mt. 12:14 ; Mt. 26:3-4 ; Mt. 26:59-66. ; Mt. 27:1-2 ; Mc. 3:6 ; Mc. 11:18 ; Ac. 4:23-29.}.
\VS{3}Rompons leurs liens, et jetons loin de nous leurs cordes.
\VS{4}Celui qui habite dans les cieux se rit d'eux, le Seigneur se moque d’eux.
\VS{5}Il leur parle dans sa colère, et il les remplira de terreur par la grandeur de son courroux\FTNT{Pr. 1:26.}.
\VS{6}C’est moi qui ai consacré mon roi sur Sion, la montagne de ma sainteté\FTNT{Mi. 4:7.}.
\VS{7}Je vous réciterai cette ordonnance ; Yahweh m'a dit : Tu es mon Fils, je t'ai engendré aujourd’hui\FTNT{Ac. 13:33 ; Hé. 1:5 ; Hé. 5:5.}.
\VS{8}Demande-moi, et je te donnerai les nations pour héritage, et les extrémités de la terre pour possession.
\VS{9}Tu les briseras avec un sceptre de fer, et tu les mettras en pièces comme un vase de potier\FTNT{Da. 2:44 ; Ap. 2:27.}.
\VS{10}Maintenant donc, rois, ayez de l'intelligence ! Juges de la terre, recevez instruction !
\VS{11}Servez Yahweh avec crainte, et réjouissez-vous avec tremblement\FTNT{Ps. 19:10.}.
\VS{12}Embrassez le Fils, de peur qu'il ne s'irrite et que vous ne périssiez dans cette conduite, quand sa colère s'embrasera promptement. Heureux sont tous ceux qui se confient en lui !
\TextTitle{[Yahweh le vrai secours]}
\Chap{3}
\VerseOne{}Psaume de David au sujet de sa fuite devant Absalom son fils.
\VS{2}Ô Yahweh, combien sont multipliés ceux qui pressent ! Beaucoup de gens se lèvent contre moi.
\VS{3}Plusieurs disent à mon âme : Plus de salut pour lui auprès de Dieu. Pause.
\VS{4}Mais toi, ô Yahweh ! Tu es un bouclier autour de moi, tu es ma gloire, et tu relèves ma tête.
\VS{5}De ma voix je crie à Yahweh, et il me répond de sa sainte montagne. Pause.
\VS{6}Je me couche, je m’endors, je me réveille, car Yahweh me soutient\FTNT{Lé. 26:6.}.
\VS{7}Je ne crains pas les myriades de peuples qui m’assiègent de toutes parts.
\VS{8}Lève-toi, Yahweh mon Dieu ! Délivre-moi ! Car tu frappes à la joue tous mes ennemis, tu brises les dents des méchants.
\VS{9}La délivrance vient de Yahweh\FTNT{Pr. 21:31 ; Es. 43:11 ; Jé. 3:23 ; Ap. 7:10.} ! Que ta bénédiction soit sur ton peuple. Pause.
\TextTitle{[Se confier en Yahweh]}
\Chap{4}
\VerseOne{}Psaume de David, donné au chef des chantres pour le chanter avec instruments à cordes.
\VS{2}Ô Dieu de ma justice, puisque je crie, réponds-moi ! Quand j’étais à l’étroit, tu m’as mis au large ! Aie pitié de moi, et exauce ma prière\FTNT{Ps. 28:1-2.} !
\VS{3}Fils des hommes, jusqu'à quand ma gloire sera-t-elle diffamée ? Jusqu’à quand aimerez-vous la vanité et chercherez-vous le mensonge ? Pause.
\VS{4}Or sachez que Yahweh s'est choisi un bien-aimé. Yahweh m'exauce quand je crie à lui\FTNT{1 Jn. 5:14.}.
\VS{5}Tremblez, et ne péchez point ; parlez en vos cœurs sur votre couche, et taisez-vous. Pause.
\VS{6}Offrez des sacrifices de justice\FTNT{Ps. 51:19.}, et confiez-vous en Yahweh.
\VS{7}Plusieurs disent : Qui nous fera voir le bonheur ? Lève sur nous la lumière de ta face, ô Yahweh !
\VS{8}Tu mets plus de joie dans mon cœur qu'ils n'en ont quand abondent leur froment et leur vin.
\VS{9}Je me couche et je m’endors en paix, car toi seul, ô Yahweh ! Tu fais habiter dans l’assurance\FTNT{Pr. 3:24.}.
\TextTitle{[La protection de Yahweh]}
\Chap{5}
\VerseOne{}Psaume de David, donné au chef des chantres, pour le chanter avec les flûtes.
\VS{2}Yahweh ! Prête l'oreille à mes paroles, Ecoute ma méditation !
\VS{3}Mon Roi et mon Dieu ! Sois attentif à la voix de mon cri ; car c'est à toi que j'adresse ma requête.
\VS{4}Yahweh, le matin tu entends ma voix, dès le matin je me tourne vers toi, et je veille.
\VS{5}Car tu n'es point un Dieu qui prenne plaisir au mal ; le méchant n’a point sa demeure auprès de toi.
\VS{6}Les orgueilleux ne subsistent pas devant tes yeux ; tu hais tous ceux qui commettent l’iniquité\FTNT{Ps. 1 :5 ; Ha. 1:13.}.
\VS{7}Tu fais périr les menteurs ; Yahweh a en abomination l'homme sanguinaire et le trompeur.
\VS{8}Mais moi, comblé de tes bienfaits, j'entrerai dans ta maison, je me prosternerai dans le palais de ta sainteté avec les sentiments d’une crainte respectueuse.
\VS{9}Yahweh ! Conduis-moi dans ta justice, à cause de mes ennemis, aplanis ta voie sous mes pas\FTNT{Ps. 25 : 4-5 ; Ps. 27:11.}.
\VS{10}Car il n'y a rien de droit dans leur bouche, leur cœur est rempli de malice, leur gosier est un sépulcre ouvert, ils flattent de leur langue\FTNT{Ps. 10:7 ; Ps. 12:3 ; Ro. 3:13.}.
\VS{11}Ô Dieu ! Fais-leur leur procès, qu'ils échouent dans leurs entreprises ! Chasse-les au loin, à cause du grand nombre de leurs transgressions ! Car ils se sont rebellés contre toi.
\VS{12}Mais que tous ceux qui se confient en toi se réjouiront, qu’ils soient dans la joie perpétuellement, et que tu sois leur protecteur ; et que ceux qui aiment ton Nom s’égayent en toi !
\VS{13}Car tu bénis le juste, ô Yahweh ! Et tu l'entoures de ta bienveillance comme d’un bouclier.
\TextTitle{[La miséricorde de Dieu]}
\Chap{6}
\VerseOne{}Psaume de David, donné au chef des chantres, pour le chanter sur la harpe à huit cordes.
\VS{2}Yahweh ! Ne me punis pas dans ta colère, et ne me châtie pas dans ta fureur\FTNT{Jé. 10 : 24.}.
\VS{3}Yahweh, aie pitié de moi, car je suis sans aucune force. Guéris-moi, ô Yahweh ! Car mes os sont épouvantés.
\VS{4}Même mon âme est fort troublée ; et toi, ô Yahweh ! Jusqu’à quand ?
\VS{5}Reviens, Yahweh ! Délivre mon âme. Sauve-moi, à cause de ta miséricorde.
\VS{6}Car celui qui meurt n’a plus ton souvenir ; qui te célébrera dans le scheol\FTNT{Ps. 88 : 11 ; Ps. 115 : 17 ; Es. 38 : 18.} ?
\VS{7}Je m’épuise à force de gémir, chaque nuit ma couche est baignée de mes larmes\FTNT{Job 7 : 3-4.}. Mon lit est arrosé de mes pleurs.
\VS{8}J’ai le visage usé par le chagrin\FTNT{Ps. 31 : 10.}, il vieillit à cause de tous ceux qui m’oppriment.
\VS{9}Retirez-vous loin de moi, vous tous ouvriers d'iniquité\FTNT{Mt. 7 : 23 ; Mt. 25 : 41 ; Lu. 13 : 27.}, car Yahweh a entendu la voix de mes pleurs.
\VS{10}Yahweh a entendu ma supplication, Yahweh a reçu ma prière.
\VS{11}Tous mes ennemis sont confondus, saisis d’épouvante, ils reculent soudain honteux.
\TextTitle{[Yahweh, le Dieu qui sonde les coeurs et les reins]}
\Chap{7}
\VerseOne{}Complainte de David, chantée à Yahweh au sujet de Cusch, le Benjamite.
\VS{2}Yahweh, mon Dieu ! Je cherche en toi mon refuge. Sauve-moi de tous mes persécuteurs, et délivre-moi,
\VS{3}afin qu'ils ne me déchirent pas comme un lion qui dévore sans qu'il n’y ait personne qui me secoure.
\VS{4}Yahweh, mon Dieu ! Si j'ai commis une telle action, s'il y a de l'iniquité dans mes mains,
\VS{5}si j’ai rendu le mal à celui qui était paisible envers moi, si j’ai dépouillé celui qui m’opprimait sans cause,
\VS{6}que l'ennemi me poursuive et m'atteigne, qu'il foule à terre ma vie, et qu'il couche ma gloire dans la poussière ! Pause.
\VS{7}Lève-toi, ô Yahweh ! Dans ta colère, lève-toi contre la fureur de mes adversaires. Réveille-toi pour me secourir. Ordonne un jugement !
\VS{8}Que l'assemblée des peuples t'environne ! Monte au-dessus d’elle vers les lieux élevés !
\VS{9}Yahweh juge les peuples : Rends-moi justice, ô Yahweh\FTNT{Ps. 9 : 5.} ! Selon ma droiture, et selon mon intégrité !
\VS{10}Que la malice des méchants prenne fin, et affermis le juste, toi qui sondes les cœurs et les reins\FTNT{Jé. 17:10 ; Jé. 11 : 20.}, ô Dieu juste !
\VS{11}Mon bouclier est en Dieu, qui délivre ceux qui sont droits de cœur.
\VS{12}Dieu est un juste juge, Dieu s'irrite en tout temps.
\VS{13}Si le méchant ne se convertit pas, Dieu aiguise son épée\FTNT{De. 32 : 41.}, il bande son arc et vise.
\VS{14}Il dirige sur lui des traits meurtriers, il rend ses flèches brûlantes.
\VS{15}Voici le méchant prépare le mal, il conçoit l’iniquité et il enfante le mensonge\FTNT{Ja. 1:15.}.
\VS{16}Il fait une fosse, il la creuse, et il tombe dans la fosse qu'il a faite\FTNT{Ps. 9 : 16.}.
\VS{17}Son travail retourne sur sa tête, et sa violence redescend sur son front.
\VS{18}Je célébrerai Yahweh à cause de sa justice, je psalmodierai le Nom de Yahweh, du Très-Haut.
\TextTitle{[La gloire de Dieu et la vanité de l'homme]}
\Chap{8}
\VerseOne{}Psaume de David, donné au chef des chantres, pour le chanter sur la guitthith.
\VS{2}Yahweh notre Seigneur ! Que ton Nom est magnifique sur toute la terre ! Ta majesté s’élève au-dessus des cieux\FTNT{Es. 6 : 3.}.
\VS{3}Par la bouche des petits enfants et de ceux qui tètent\FTNT{Mt. 21 : 16.}, tu as fondé ta puissance, à cause de tes adversaires, afin de faire cesser l'ennemi et le vindicatif.
\VS{4}Quand je regarde tes cieux, l'ouvrage de tes doigts, la lune et les étoiles que tu as fixées,
\VS{5}qu'est-ce que l'homme, pour que tu te souviennes de lui ? Et le fils de l'homme, pour que tu le visites\FTNT{Dans ce passage, il est question de l’incarnation de Yahweh afin de nous sauver (1 Ti. 3:16 ; 1 Co. 15:45-49 ; Hé. 2:14). Jésus-Christ s’est lui-même nommé « fils de l’homme » (Lu. 9:22-26), littéralement « fils d’Adam », d’ailleurs cette expression apparaît dans les évangiles plus de quatre-vingt fois.} ?
\VS{6}Tu l'as fait de peu inférieur aux anges, et tu l'as couronné de gloire et d'honneur.
\VS{7}Tu lui as donné la domination sur les œuvres de tes mains, tu as tout mis sous ses pieds\FTNT{1 Co. 15:27.},
\VS{8}les brebis comme les bœufs, les animaux des champs,
\VS{9}les oiseaux du ciel et les poissons de la mer, tout ce qui parcourt les sentiers des mers.
\VS{10}Yahweh, notre Seigneur ! Que ton Nom est magnifique sur toute la terre !
\TextTitle{[Louange à Yahweh, l'auteur de nos victoires]}
\Chap{9}
\VerseOne{}Psaume de David, donné au chef des chantres, pour le chanter sur Muth-Labben «~la mort du fils~».
\VS{2}Je célébrerai de tout mon cœur Yahweh, je raconterai toutes tes merveilles.
\VS{3}Je me réjouirai et je m'égaierai en toi, je chanterai ton Nom, ô Très-Haut !
\VS{4}Mes ennemis reculent, ils trébuchent, ils périssent devant ta face.
\VS{5}Car tu soutiens mon droit et ma cause, tu sièges sur ton trône en juste juge.
\VS{6}Tu châties les nations, tu détruis le méchant, tu effaces leur nom pour toujours, et à perpétuité.
\VS{7}Plus d’ennemis ! Les désolations ont-elles pris fin ? As-tu aussi rasé les villes pour toujours ? Leur mémoire est perdue avec elles.
\VS{8}Mais Yahweh sera assis éternellement, il a établi son trône pour juger.
\VS{9}Il juge le monde avec justice, il juge les peuples avec droiture\FTNT{Ps. 96:13 ; Ps. 98:9.}.
\VS{10}Yahweh est un refuge pour l’opprimé, un refuge au temps de la détresse\FTNT{Ps. 37:39 ; Ps. 46:2 ; Ps. 91:2.}.
\VS{11}Ceux qui connaissent ton Nom se confient en toi\FTNT{Pr. 3:5.}. Car tu n'abandonnes point ceux qui te cherchent, ô Yahweh !
\VS{12}Chantez à Yahweh qui habite en Sion, annoncez ses exploits parmi les peuples !
\VS{13}Lorsqu'il recherche le sang versé, il se souvient des malheureux ; il n’oublie pas le cris des affligés.
\VS{14}Aie pitié de moi, Yahweh ! Vois la misère où me réduisent mes ennemis, enlève-moi des portes de la mort,
\VS{15}afin que je raconte toutes tes louanges, dans les portes de la fille de Sion. Je me réjouirai de la délivrance\FTNT{Voir commentaire en Es. 26:1.} que tu m'auras donnée.
\VS{16}Les nations tombent dans la fosse qu'elles ont faite\FTNT{Ps. 10:2 ; Ps. 35:7.}, leur pied se prend au filet qu'elles ont caché.
\VS{17}Yahweh se fait connaître, il fait justice, le méchant est enlacé dans l'ouvrage de ses mains. Jeu d’instruments. Pause.
\VS{18}Les méchants retournent dans le scheol, toutes les nations qui oublient Dieu.
\VS{19}Car le pauvre n’est point oublié à jamais, l’espérance des affligés ne périt pas à toujours.
\VS{20}Lève-toi, ô Yahweh ! Que l'homme mortel ne triomphe point ! Que les nations soient jugées devant ta face !
\VS{21}Frappe-les de terreur, ô Yahweh ! Que les peuples sachent qu'ils ne sont que des hommes mortels\FTNT{Es. 51:12.}. Pause.
\TextTitle{[Le jugement de Dieu sur les méchants]}
\Chap{10}
\VerseOne{}Pourquoi, ô Yahweh ! Te tiens-tu éloigné ? Pourquoi te caches-tu au temps où nous sommes dans la détresse ?\FTNT{Ps. 13:2 ; Ps. 44:24.}
\VS{2}Le méchant par son orgueil poursuit ardemment les affligés, mais ils seront pris par les machinations qu’ils ont préméditées\FTNT{Ps. 7:15-16 ; Ps. 9:16 ; Ps. 35:8.}.
\VS{3}Car le méchant se glorifie du désir de son âme, il bénit l'avare, et il méprise Yahweh.
\VS{4}Le méchant dit avec arrogance : Il ne fera pas d’enquête ! Il n’y a point de Dieu\FTNT{Ps. 14:1 ; Ps. 53:2.} ! Voilà toutes ses pensées.
\VS{5}Ses voies réussissent en tout temps, tes jugements sont éloignés de lui, il souffle contre tous ses adversaires.
\VS{6}Il dit en son cœur : Je ne chancelle pas, je suis pour toujours à l’abri du malheur !
\VS{7}Sa bouche est pleine de malédictions, de tromperies, et de fraudes ; il n'y a sous sa langue qu'oppression et outrage\FTNT{Ps. 59:7-8 ; Ps. 64:3-4 ; Job. 20:12.}.
\VS{8}Il se tient aux embûches dans des villages, il tue l'innocent dans des lieux cachés, ses yeux épient le malheureux.
\VS{9}Il se tient aux aguets dans un lieu caché, comme un lion dans sa tanière, il se tient aux aguets pour attraper l'affligé ; il attrape l'affligé, l'attirant dans son filet.
\VS{10}Il se courbe, il se baisse, et les malheureux tombent dans ses griffes.
\VS{11}Il dit en son cœur : Dieu oublie ! Il cache sa face, il ne le verra jamais\FTNT{Ps 94:7.}.
\VS{12}Lève-toi, ô Yahweh ! Lève ta main ! N’oublie pas les malheureux !
\VS{13}Pourquoi le méchant méprise-t-il Dieu ? Il dit en son cœur que tu ne punis pas.
\VS{14}Tu regardes cependant, car tu vois la peine et la souffrance, pour prendre en main leur cause ; c’est toi qui viens en aide à l’orphelin.
\VS{15}Brise le bras du méchant, punis ses iniquités, et qu’il disparaisse à tes yeux !
\VS{16}Yahweh est Roi à toujours et à perpétuité\FTNT{Ps. 29:10 ; Ps. 145:13 ; Ps. 146:10 ; La. 5:19.} ; les nations sont exterminées de sa terre.
\VS{17}Tu entends les vœux de ceux qui souffrent, ô Yahweh ! Tu affermis leur cœur, tu prêtes l’oreille
\VS{18}pour rendre justice à l'orphelin et à l’opprimé, afin que l'homme mortel tiré de la terre cesse d’inspirer l’effroi.
\TextTitle{[Yahweh, le refuge des saints]}
\Chap{11}
\VerseOne{}Psaume de David, donné au chef des chantres. C’est en Yahweh que je cherche un refuge. Comment un homme peut-il dire à mon âme : Fuis dans vos montagnes comme un oiseau ?
\VS{2}En effet, les méchants bandent l'arc\FTNT{Ps. 37:14.}, ils ajustent leur flèche sur la corde, pour tirer dans l’ombre ceux dont le cœur est droit.
\VS{3}Quand les fondements sont renversés, que fera le juste ?
\VS{4}Yahweh est dans son saint temple, Yahweh a son trône dans les cieux ; ses yeux contemplent, ses paupières sondent les fils des hommes.
\VS{5}Yahweh sonde le juste et le méchant ; et son âme hait celui qui aime la violence.
\VS{6}Il fait pleuvoir sur les méchants des charbons, du feu et du soufre\FTNT{Ez. 38:22.} ; un vent brûlant, c’est le calice qu’ils ont en partage.
\VS{7}Car Yahweh est juste, il aime la justice ; les hommes droits contemplent sa face.
\TextTitle{[Le langage des orgueilleux]}
\Chap{12}
\VerseOne{}Psaume de David, donné au chef des chantres pour le chanter sur la harpe à huit cordes.
\VS{2}Sauve, ô Yahweh ! Car les hommes pieux s’en vont, les fidèles disparaissent parmi les fils de l’homme.
\VS{3}Chacun dit des faussetés à son compagnon avec des lèvres flatteuses, et ils parlent avec un cœur double.
\VS{4}Que Yahweh retranche toutes les lèvres flatteuses, la langue qui parle fièrement\FTNT{Ps. 17:10.},
\VS{5}parce qu'ils disent : Nous sommes puissants par nos langues, nous avons nos lèvres avec nous, qui serait notre maître ?
\VS{6}A cause du mauvais traitement que l'on fait aux malheureux, à cause du gémissement des pauvres, je me lèverai maintenant, dit Yahweh, je mettrai en sûreté celui à qui l'on tend des pièges.
\VS{7}Les paroles de Yahweh sont des paroles pures, c'est un argent éprouvé sur terre au creuset\FTNT{Ps. 19:10 ; Ps. 119:140 ; Pr. 30:5.}, et sept fois épuré.
\VS{8}Toi, Yahweh ! Garde-les, préserve cette race à jamais.
\VS{9}Les méchants se promènent de toutes parts, tandis que des gens abjects sont élevés parmi les fils des hommes.
\TextTitle{[Savoir attendre le secours de Dieu]}
\Chap{13}
\VerseOne{}Psaume de David, donné au chef des chantres.
\VS{2}Yahweh, jusqu’à quand m'oublieras-tu ? Sera-ce pour toujours ? Jusqu’à quand me cacheras-tu ta face\FTNT{Ps. 10:1 ; Ps. 27:9.} ?
\VS{3}Jusqu’à quand consulterai-je mon âme, affligerai-je mon cœur tous les jours ? Jusqu’à quand mon ennemi s'élèvera-t-il contre moi ?
\VS{4}Yahweh, mon Dieu ! Regarde, exauce-moi, illumine mes yeux, de peur que je ne dorme du sommeil de la mort.
\VS{5}De peur que mon ennemi ne dise : J'ai eu le dessus ; que mes adversaires ne se réjouissent si je venais à tomber\FTNT{Ps. 25:2.}.
\VS{6}Mais moi, je me confie en ta bonté, mon cœur se réjouira de la délivrance que tu m'auras donnée ; je chanterai à Yahweh parce qu’il m’a fait du bien.
\TextTitle{[L'insensé ne cherche pas Dieu]}
\Chap{14}
\VerseOne{}Psaume de David, donné au chef des chantres. L'insensé dit en son cœur : Il n'y a point de Dieu\FTNT{Ceux qui ne croient pas en l’existence de Dieu sont appelés insensés. En effet, la création révèle l’existence du Créateur (Ro. 1:19-20).} ! Ils se sont corrompus, ils ont commis des actions abominables ; il n'y a personne qui fasse le bien.
\VS{2}Yahweh regarde des cieux les fils de l’homme, pour voir s'il y a quelqu'un qui soit intelligent, qui cherche Dieu\FTNT{Ps. 33:13 ; Job. 28:24.}.
\VS{3}Ils se sont tous égarés, ils se sont tous ensemble rendus odieux, il n'y a personne qui fasse le bien, pas même un seul\FTNT{Tous les hommes naissent pécheurs (Ro. 3:10-23).}.
\VS{4}Tous ces ouvriers d'iniquité n'ont-ils point de connaissance ? Ils dévorent mon peuple, ils le prennent pour nourriture ; ils n'invoquent point Yahweh,
\VS{5}là, ils seront saisis d'une grande frayeur, car Dieu est avec la race des justes.
\VS{6}Jetez l’opprobre sur l’espérance du malheureux…Yahweh est son refuge.
\VS{7}Oh ! qui fera partir de Sion la délivrance d'Israël\FTNT{C’est le Messie qui délivrera Israël (Ro. 11:25-27).} ? Quand Yahweh ramenera son peuple captif, Jacob se réjouira, Israël se réjouira.
\TextTitle{[L'homme que Yahweh agrée]}
\Chap{15}
\VerseOne{}Psaume de David. Yahweh, qui séjournera dans ta tente ? Qui demeurera sur ta montagne sainte ?\FTNT{Ps. 24:3-4.}
\VS{2}Celui qui marche dans l'intégrité, qui fait ce qui est juste, et qui profère la vérité telle qu'elle est dans son cœur,
\VS{3}qui ne calomnie point avec sa langue, qui ne fait point de mal à son ami, qui ne diffame point son prochain.
\VS{4}Il regarde avec dédain celui qui est méprisable, mais il honore ceux qui craignent Yahweh, il ne se rétracte point s’il fait un serment à son préjudice.
\VS{5}Il n’exige point d’intérêt de son argent, et il n’accepte point de présent contre l'innocent\FTNT{Lé. 25:36 ; De. 16:19 ; De. 27:25.}. Celui qui fait ces choses ne sera jamais ébranlé.
\TextTitle{[Yahweh, la source de notre vie]}
\Chap{16}
\VerseOne{}Hymne de David. Garde-moi, ô Dieu ! Car je cherche en toi mon refuge.
\VS{2}Je dis à Yahweh : Tu es mon Seigneur, tu es mon bonheur.
\VS{3}Les saints qui sont dans le pays, les hommes pieux sont l’objet de toute mon affection.
\VS{4}On multiplie les peines, on court après les dieux étrangers : Je ne répands pas leurs libations de sang, et je ne mets pas leurs noms sur mes lèvres.
\VS{5}Yahweh est la part de mon héritage et ma coupe ; tu maintiens mon lot.
\VS{6}Un héritage délicieux m’est échu, une belle possession m’est accordée.
\VS{7}Je bénirai Yahweh qui me donne conseille, je le bénirai même durant les nuits dans lesquelles mes reins m’enseignent.
\VS{8}J’ai constamment Yahweh sous mes yeux ; quand il est à ma droite, je ne chancelle pas\FTNT{Ac. 2:25 ; Ps. 109:31 ; Ps. 110:5.}.
\VS{9}C'est pourquoi mon cœur se réjouit, mon esprit se réjouit, et mon corps repose en sécurité.
\VS{10}Car tu n'abandonneras point mon âme au scheol, tu ne permettras point que ton bien-aimé voie la corruption\FTNT{Le roi David prophétise ici la résurrection du Messie.}.
\VS{11}Tu me feras connaître le chemin de la vie ; il y a d’abondantes joies devant ta face, des délices éternels à ta droite.
\TextTitle{[Avoir confiance en Dieu]}
\Chap{17}
\VerseOne{}Prière de David. Yahweh ! Ecoute la droiture, sois attentif à mon cri, prête l'oreille à ma prière faite avec des lèvres sans tromperie !
\VS{2}Que ma justice paraisse devant ta face, que tes yeux contemplent mon intégrité !
\VS{3}Tu as sondé mon cœur\FTNT{Ps. 139:1 ; Jé. 12:3.}, tu l'as visité de nuit, tu m'as examiné, tu n'as rien trouvé : Ma pensée ne va point au-delà de ma parole.
\VS{4}Quant aux actions des hommes, selon la parole de tes lèvres, je me tiens en garde contre la voie du violent.
\VS{5}Mes pas sont fermes dans tes sentiers, mes pieds ne chancellent point.
\VS{6}Je t'invoque car tu m'exauces, ô Dieu ! Incline ton oreille vers moi, écoute mes paroles !
\VS{7}Signale ta bonté, toi qui sauve ceux qui cherchent un refuge, et qui par ta droite les délivres de leurs adversaires !
\VS{8}Garde-moi comme la prunelle de l'œil, cache-moi à l'ombre de tes ailes\FTNT{Mt. 23:37.},
\VS{9}contre les méchants qui me traitent violemment, mes ardents ennemis qui m'entourent.
\VS{10}Ils sont enfermés dans leur propre graisse, leur bouche parle avec orgueil.
\VS{11}Maintenant, ils nous environnent à chaque pas que nous faisons ; ils jettent leur regard pour nous étendre par terre.
\VS{12}Ils ressemblent au lion qui ne demande qu'à déchirer, et au lionceau qui se tient dans les lieux cachés.
\VS{13}Lève-toi, ô Yahweh, devance-les, renverse-les ! Délivre mon âme du méchant par ton épée.
\VS{14}Yahweh, délivre-moi par ta main de ces gens, des gens de ce monde ! Leur part est dans cette vie, et tu remplis leur ventre de tes biens ; leurs enfants sont rassasiés, et ils laissent leurs restes à leurs petits-enfants.
\VS{15}Mais moi, dans mon innocence, je verrai ta face\FTNT{Job. 19:26-27 ; Ps. 16:10-11.}, et je me rassasierai de ton image, dès mon réveil.
\TextTitle{[Louange à Dieu, le bouclier des saints]}
\Chap{18}
\VerseOne{}Psaume de David, serviteur de Yahweh qui adressa à Yahweh les paroles de ce cantique le jour où Yahweh l'eut délivré de la main de Saül. Au chef des chantres.
\VS{2}Il dit donc : Je t’aime, ô Yahweh, ma force !
\VS{3}Yahweh est mon rocher\FTNT{Yahweh est le rocher sur lequel s’appuyait David. Paul enseigne que ce rocher était Jésus-Christ (1 Co. 10:1-4). Voir commentaire Es. 8:13-17.}, ma forteresse, et mon libérateur ! Mon Dieu, mon rocher où je trouve un refuge ! Mon bouclier, la force qui me sauve, ma haute retraite !
\VS{4}Je crie : Loué soit Yahweh ! Et je suis délivré de mes ennemis.
\VS{5}Les liens de la mort m'avaient environné, et des torrents de destruction m'avaient épouvanté.
\VS{6}Les liens du scheol m'avaient entouré, les filets de la mort m'avaient surpris\FTNT{Ps. 116:3.}.
\VS{7}Dans ma détresse, j'ai invoqué Yahweh, j'ai crié à mon Dieu ; il a entendu ma voix de son palais, mon cri est parvenu devant lui à ses oreilles.
\VS{8}La terre fut ébranlée et trembla, les fondements des montagnes croulèrent\FTNT{Ps. 104:32 ; Es. 5:25 ; Es. 64:1-3 ; Jé. 4:24.}, et ils furent ébranlés, parce qu'il était irrité.
\VS{9}Une fumée montait de ses narines, et de sa bouche sortait un feu dévorant, des charbons embrasés.
\VS{10}Il abaissa les cieux et descendit : Il y avait une épaisse nuée sous ses pieds.
\VS{11}Il était monté sur un chérubin, et il volait, il était porté sur les ailes du vent\FTNT{Ps. 104:3.}.
\VS{12}Il faisait des ténèbres sa demeure secrète, autour de lui était sa tente, il était enveloppé des eaux obscures et de sombres nuages.
\VS{13}De la splendeur qui le précédait s’échappaient les nuées, lançant de la grêle et des charbons de feu.
\VS{14}Yahweh tonna dans les cieux, le Très-Haut fit retentir sa voix avec de la grêle et des charbons de feu.
\VS{15}Il tira ses flèches, et écarta mes ennemis, il lança des éclairs et les mit en déroute\FTNT{Ps. 77:18.}.
\VS{16}Le fond des eaux parut, les fondements du monde furent découverts, par ta menace, ô Yahweh ! Par le souffle du vent de tes narines.
\VS{17}Il étendit la main d'en haut, il m'enleva, et me retira des grandes eaux\FTNT{2 S. 22:17.} ;
\VS{18}il me délivra de mon puissant ennemi, et de ceux qui me haïssaient, car ils étaient plus forts que moi.
\VS{19}Ils m'avaient surpris au jour de ma détresse, mais Yahweh, fut mon appui.
\VS{20}Il m'a mis au large, il m'a délivré, parce qu'il m’aime.
\VS{21}Yahweh m'a rendu selon ma justice, il m'a traité selon la pureté de mes mains\FTNT{Ps. 18:25 ; Ps. 7:9.},
\VS{22}car j’ai observé les voies de Yahweh, et je n’ai point été coupable envers mon Dieu.
\VS{23}Car j'ai eu devant moi toutes ses ordonnances, et je ne me suis point écarté de ses lois.
\VS{24}J'ai été intègre envers lui, et je me suis tenu en garde contre mon iniquité.
\VS{25}Aussi Yahweh m'a rendu selon ma justice, selon la pureté de mes mains devant ses yeux.
\VS{26}Avec celui qui est bon, tu te montres bon, avec l'homme droit tu agis selon la droiture.
\VS{27}Avec celui qui est pur, tu te montres pur, et avec le pervers tu agis selon sa perversité.
\VS{28}Car tu sauves le peuple affligé, et tu abaisses les yeux hautains\FTNT{Es. 2:11 ; Es. 5:15.}.
\VS{29}Tu fais briller ma lumière ; Yahweh, mon Dieu, éclaire mes ténèbres.
\VS{30}Avec toi, je me précipite sur un corps d’armée, avec mon Dieu je franchis la muraille.
\VS{31}Les voies de Dieu sont sans défaut ; la parole de Yahweh est éprouvée\FTNT{Ps. 19:8-9 ; De. 32:4 ; Da. 4:37.} ; il est un bouclier pour tous ceux qui se confient en lui.
\VS{32}Car qui est Dieu, si ce n’est Yahweh ? Et qui est un rocher, si ce n’est notre Dieu ?\FTNT{2 S. 22:32 ; 1 S. 2:2.}
\VS{33}C'est le Dieu qui me ceint de force, et qui me conduit dans la voie droite.
\VS{34}Il rend mes pieds semblables à ceux des biches\FTNT{2 S. 2:18.}, et il me place sur mes lieux élevés.
\VS{35}Il exerce mes mains au combat, tellement qu’un arc d’airain a été rompu avec mes bras.\FTNT{Job. 20:24.}.
\VS{36}Tu me donnes le bouclier de ton salut, ta droite me soutient, et je deviens puissant par ta bonté.
\VS{37}Tu élargis le chemin sous mes pas, et mes pieds ne chancellent point.
\VS{38}Je poursuis mes ennemis, je les atteints, et je ne reviens pas avant de les avoir anéantis.
\VS{39}Je les brise, et ils ne peuvent se relever ; ils tombent sous mes pieds.
\VS{40}Car tu m'as ceint de force pour le combat, tu fais plier sous moi ceux qui s'élevaient contre moi.
\VS{41}Tu fais tourner le dos à mes ennemis devant moi, et j’extermine ceux qui me haïssaient.
\VS{42}Ils crient, mais il n'y a point de libérateur ; ils crient à Yahweh, mais il ne leur répond point.
\VS{43}Je les brise comme la poussière qui est dispersée par le vent, et je les foule comme la boue des rues.
\VS{44}Tu me délivres des séditions du peuple, tu m'établis chef des nations. Un peuple que je ne connais point m'est asservi.
\VS{45}Ils m’obéissent au premier ordre, les fils de l’étranger me flattent.
\VS{46}Les étrangers s’enfuient, et ils tremblent de peur dans leurs forteresses.
\VS{47}Yahweh est vivant, et béni soit mon rocher ! Que le Dieu de mon salut soit exalté !
\VS{48}Le Dieu qui est mon vengeur, et qui m’assujettit les peuples.
\VS{49}C'est lui qui me délivre de mes ennemis ! Tu m'élèves au-dessus de mes adversaires, tu me sauves de l’homme violent.
\VS{50}C'est pourquoi je te louerai parmi les nations, ô Yahweh ! Et je chanterai des louanges à la gloire de ton Nom.
\VS{51}Il accorde de grandes délivrances à son roi, et il fait miséricorde à son oint, à David, et à sa postérité pour toujours.
\TextTitle{[La création exalte la grandeur de Dieu]}
\Chap{19}
\VerseOne{}Psaume de David, donné au chef des chantres.
\VS{2}Les cieux racontent la gloire de Dieu, et l'étendue met en évidence l'oeuvre de ses mains.
\VS{3}Un jour en instruit un autre jour, et une nuit fait connaître sa science à l'autre nuit.
\VS{4}Ce n’est pas un langage, ce ne sont pas des paroles dont le cri ne soit point entendu :
\VS{5}Leur retentissement couvre toute la terre, et leur voix est allée jusqu'aux extrémités du monde\FTNT{Ro. 10:18.}. Il a dressé une tente pour le soleil.
\VS{6}Et le soleil est semblable à un époux sortant de sa chambre ; il s’élance sur le sentier avec la joie d’un homme vaillant.
\VS{7}Il se lève à l’extrémité des cieux, et achève sa course à l’autre extrémité\FTNT{Ec. 1:5.}. Rien ne se dérobe à sa chaleur.
\VS{8}La loi de Yahweh est parfaite, elle restaure l'âme ; le témoignage de Yahweh est fidèle, il donne la sagesse au simple\FTNT{Ps. 18:31 ; 2 S. 22:31 ; Ps. 119:130.}.
\VS{9}Les ordonnances de Yahweh sont droites, elles réjouissent le cœur ; les commandements de Yahweh sont purs, ils éclairent les yeux.
\VS{10}La crainte de Yahweh est pure, elle subsiste à toujours ; les jugements de Yahweh sont vrais, et ils sont tous justes.
\VS{11}Ils sont plus précieux que l'or, que beaucoup d’or fin ; et plus doux que le miel, que celui qui coule des rayons de miel\FTNT{Ps. 119:103.}.
\VS{12}Ton serviteur aussi en reçoit l’éclairage ; pour qui les observe la récompense est grande.
\VS{13}Qui connaît ses fautes commises par erreur ? Purifie-moi de mes fautes cachées.
\VS{14}Eloigne aussi ton serviteur des actions commises par fierté, en sorte qu'elles ne dominent point sur moi, qu’elles cessent et que je sois nettoyé de mes grands péchés.
\VS{15}Que les propos de ma bouche et la méditation de mon cœur te soient agréables, ô Yahweh ! Mon rocher et mon rédempteur\FTNT{Voir commentaire Es. 60:16.}.
\TextTitle{[Yahweh exauce les justes]}
\Chap{20}
\VerseOne{}Psaume de David, donné au chef des chantres.
\VS{2}Que Yahweh te réponde au jour de la détresse, que le Nom du Dieu de Jacob te protège !
\VS{3}Qu'il envoie ton secours du saint lieu, et qu'il te soutienne de Sion !
\VS{4}Qu'il se souvienne de toutes tes offrandes, qu'il réduise en cendres ton holocauste ! Pause.
\VS{5}Qu'il te donne ce que ton cœur désire, et qu'il fasse réussir tes desseins !
\VS{6}Nous nous réjouirons de ton salut, nous lèverons la bannière au nom de notre Dieu ; Yahweh exaucera tous tes vœux.
\VS{7}Je sais déjà que Yahweh sauve son oint ; il l’exaucera des cieux, de sa sainte demeure, par le secours puissant de sa droite.
\VS{8}Les uns se vantent de leurs chars, et les autres de leurs chevaux ; mais nous, nous glorifierons du Nom de Yahweh notre Dieu.
\VS{9}Eux ils plient, et ils tombent ; nous, nous tenons ferme, et restons debout.
\VS{10}Yahweh, sauve le roi ! Qu’il nous réponde quand nous crions à lui !
\TextTitle{[La protection de Dieu sur le roi]}
\Chap{21}
\VerseOne{}Psaume de David, donné au chef des chantres.
\VS{2}Yahweh, le roi se réjouit de ta puissance, ton secours le remplit d’allégresse !
\VS{3}Tu lui as donné ce que désirait son cœur, et tu n’as point refusé ce que demandaient ses lèvres. Pause.
\VS{4}Car tu l'as prévenu par les bénédictions de ta bonté, et tu as mis sur sa tête une couronne d’or pur.
\VS{5}Il t'avait demandé la vie, et tu la lui as donnée, une vie longue pour toujours et à perpétuité.
\VS{6}Sa gloire est grande à cause de ton salut, tu l'as couvert de majesté et d'honneur.
\VS{7}Tu le rends à jamais un objet de bénédictions, tu le combles de joie devant ta face\FTNT{Ps. 16:11.}.
\VS{8}Le roi se confie en Yahweh, et par la bonté du Très-Haut, il ne chancelle pas\FTNT{Ps. 16:8.}.
\VS{9}Ta main trouvera tous tes ennemis, ta droite trouvera tous ceux qui te haïssent.
\VS{10}Tu les rendras tels qu’une fournaise ardente le jour où l’on verra ta face ; Yahweh les engloutira dans sa colère, et le feu les consumera.
\VS{11}Tu feras périr leur fruit de la terre, et leur race du milieu des fils des hommes.
\VS{12}Car ils ont projeté du mal contre toi, et ils ont conçu de mauvais desseins dont ils ne pourront venir à bout.
\VS{13}Parce que tu leur feras tourner le dos, et avec ton arc tu tireras sur eux.
\VS{14}Elève-toi Yahweh, par ta force ! Nous chanterons et célébrerons ta puissance.
\TextTitle{[Souffrances du messie]}
\Chap{22}
\VerseOne{}Psaume de David, donné au chef des chantres, pour le chanter sur biche de l’aurore.
\VS{2}Mon Dieu ! Mon Dieu ! Pourquoi m'as-tu abandonné\FTNT{Le Ps. 22 est une description détaillée de la mort par crucifixion du Seigneur Jésus-Christ (Mt. 27:45-46).} et t’éloignes-tu sans me secourir, sans écouter mes plaintes ?
\VS{3}Mon Dieu ! Je crie le jour, mais tu ne réponds point ; la nuit, et je n’ai point de repos.
\VS{4}Pourtant tu es le Saint, tu habites au milieu des louanges d'Israël.
\VS{5}Nos pères se sont confiés en toi ; ils se sont confiés, et tu les as délivrés.
\VS{6}Ils ont crié vers toi, et ils ont été délivrés ; ils se sont appuyés sur toi, et ils n'ont point été confus\FTNT{Ps. 25:3 ; Ps. 31:2 ; Es. 49:23.}.
\VS{7}Et moi, je suis un ver, et non un homme, l'opprobre des hommes et le méprisé du peuple\FTNT{Es. 53:2-3.}.
\VS{8}Tous ceux qui me voient, se moquent de moi, ils ouvrent les lèvres, secouent la tête\FTNT{Ps. 109:25 ; Mt. 27:39.} :
\VS{9}Recommande-toi à Yahweh ! Qu'il te délivre, et qu'il te sauve, puisqu'il prend plaisir en toi\FTNT{Mt. 27:43.} !
\VS{10}Cependant c'est toi qui m'as tiré hors du ventre de ma mère, qui m'as mis en sûreté lorsque j'étais sur les mamelles de ma mère.
\VS{11}J'ai été sous ta garde dès le sein maternel, tu as été mon Dieu dès le ventre de ma mère\FTNT{Es. 49:1.}.
\VS{12}Ne t'éloigne point de moi, car la détresse est près de moi, et il n'y a personne qui me secoure\FTNT{Ps. 69:21.} !
\VS{13}Plusieurs taureaux sont autour de moi, de puissants taureaux de Basan m'entourent.
\VS{14}Ils ouvrent leur gueule contre moi, comme un lion qui déchire et rugit.
\VS{15}Je suis comme de l'eau qui s’écoule, et tous mes os se séparent ; mon cœur est comme de la cire, il se fond dans mes entrailles.
\VS{16}Ma force se dessèche comme l’argile, et ma langue s’attache à mon palais ; tu me réduis à la poussière de la mort.
\VS{17}Car des chiens m'environnent, une assemblée de méchants m'entoure, ils ont percé mes mains et mes pieds.
\VS{18}Je pourrais compter tous mes os un par un. Eux, ils m’examinent, ils me regardent.
\VS{19}Ils se partagent mes vêtements, et tirent au sort ma tunique\FTNT{Mt. 27:35 ; Mc. 15:24 ; Lu. 23:33.}.
\VS{20}Et toi, Yahweh, ne t'éloigne point ! Ma force, hâte-toi de me secourir !
\VS{21}Délivre ma vie de l'épée, ma vie contre le pouvoir des chiens !
\VS{22}Sauve-moi de la gueule du lion, délivre-moi des cornes du buffle !
\VS{23}Je déclarerai ton Nom à mes frères, je te louerai au milieu de l'assemblée\FTNT{Hé. 2:12.}.
\VS{24}Vous qui craignez Yahweh, louez-le ! Toute la race de Jacob, glorifiez-le ! Toute la race d'Israël, redoutez-le !
\VS{25}Car il n'a ni mépris ni dédain pour les peines du misérable, et il ne lui cache point sa face, mais il l’écoute quand il crie à lui.
\VS{26}Tu seras l’objet de mes louanges dans la grande assemblée ; j’accomplirai mes vœux en présence de ceux qui te craignent\FTNT{Ps. 56:13.}.
\VS{27}Les malheureux mangeront et seront rassasiés ; ceux qui cherchent Yahweh le loueront, votre cœur vivra à perpétuité !
\VS{28}Toutes les extrémités de la terre s'en souviendront, ils se convertiront à Yahweh, et toutes les familles des nations se prosterneront devant toi\FTNT{Ps. 72:8-11 ; Ps. 86:9.}.
\VS{29}Car le règne appartient à Yahweh : Il domine sur les nations.
\VS{30}Tous les gens de la terre mangeront et se prosterneront devant lui ; tous ceux qui descendent dans la poussière s'inclineront, même celui qui ne peut conserver sa vie.
\VS{31}La postérité le servira, on parlera du Seigneur de génération en génération\FTNT{Es. 59:21 ; Es. 65:23 ; Ps. 110:3.}.
\VS{32}Ils viendront et ils publieront sa justice au peuple qui naîtra, parce qu'il aura fait ces choses.
\TextTitle{[Le bon berger]}
\Chap{23}
\VerseOne{}Psaume de David. Yahweh est mon berger\FTNT{Yahweh, le bon berger, est notre Seigneur Jésus-Christ. Jn. 10:11 ; Es. 40:11 ; Jé. 23:4.}, je ne manquerai de rien.
\VS{2}Il me fait reposer dans de verts pâturages, il me dirige près des eaux paisibles.
\VS{3}Il restaure mon âme, et me conduit dans les sentiers de la justice, à cause de son nom.
\VS{4}Quand je marche dans la vallée de l'ombre de la mort, je ne crains aucun mal\FTNT{Ps. 118:6.}, car tu es avec moi : Ton bâton et ta houlette me consolent.
\VS{5}Tu dresses devant moi une table, en face de mes adversaires. Tu oins d'huile ma tête et ma coupe déborde.
\VS{6}Le bonheur et la grâce m'accompagneront tous les jours de ma vie, et j’habiterai dans la maison de Yahweh jusqu’à la fin de mes jours.
\TextTitle{[Accueil de Yahweh, le roi de gloire]}
\Chap{24}
\VerseOne{}Psaume de David. La terre appartient à Yahweh, avec tout ce qui est en elle\FTNT{Ps. 50:12 ; Ex. 19:5 ; De. 10:14 ; Job. 41:2 ; 1 Co. 10:26.}, le monde et ceux qui y habitent.
\VS{2}Car il l'a fondée sur les mers, et affermie sur les fleuves.
\VS{3}Qui pourra monter à la montagne de Yahweh ? Qui s’élèvera jusqu’à son lieu saint\FTNT{Ps. 15:1-2 ; Ps. 118:19.} ?
\VS{4}Celui qui a les mains pures et le cœur pur, qui ne livre point son âme au mensonge, et qui ne jure pas pour tromper.
\VS{5}Il obtiendra la bénédiction de Yahweh, et la justice du Dieu de son salut.
\VS{6}Voilà le partage de la génération qui l'invoque, de ceux qui cherchent ta face, de Jacob ! Pause.
\VS{7}Portes, élevez vos linteaux, élevez-vous portes éternelles ! Que le Roi de gloire fasse son entrée !
\VS{8}Qui est ce Roi de gloire ? C'est Yahweh fort et puissant, Yahweh puissant dans les combats.
\VS{9}Portes, élevez vos linteaux, élevez-les aussi, vous portes éternelles, que le Roi de gloire fasse son entrée !
\VS{10}Qui est ce Roi de gloire ? Yahweh des armées : Voilà le Roi de gloire : Pause.
\TextTitle{[Dieu conduit son peuple dans la vérité]}
\Chap{25}
\VerseOne{}Psaume de David. [Aleph.] Yahweh, j'élève mon âme à toi.
\VS{2}[Beth.] Mon Dieu ! Je me confie en toi : Que je ne sois point honteux\FTNT{Ps. 22:5 ; Ps. 31:2.} ! Que mes ennemis ne triomphent point de moi.
\VS{3}[Guimel.] Tous ceux qui espèrent en toi ne seront point confus\FTNT{Ro. 10:11.} ; ceux qui agissent avec tromperie sans cause seront honteux.
\VS{4}[Daleth.] Yahweh ! Fais-moi connaître tes voies, enseigne-moi tes sentiers\FTNT{Ps. 27:11 ; Ps. 86:11 ; Ps. 143:10.}.
\VS{5}[He. Vau.] Fais-moi marcher selon la vérité, et instruis-moi, car tu es le Dieu de ma délivrance, je m'attends à toi tous les jours.
\VS{6}[Zain.] Yahweh ! Souviens-toi de ta miséricorde et de ta bonté, car elles sont éternelles\FTNT{Ps. 103:17 ; Ps. 106:1 ; Ps. 107:1 ; Ps. 117:2 ; Ps. 136:1-2 ; Jé. 33:11.}.
\VS{7}[Heth.] Ne te souviens point des péchés de ma jeunesse ni de mes transgressions ; souviens-toi de moi selon ta miséricorde, à cause de ta bonté, ô Yahweh !
\VS{8}[Teth.] Yahweh est bon et droit : C'est pourquoi il enseigne aux pécheurs la voie.
\VS{9}[Jod.] Il conduit les humbles dans la justice, et il leur enseigne sa voie.
\VS{10}[Caph.] Tous les sentiers de Yahweh sont miséricorde et fidélité, pour ceux qui gardent son alliance et son témoignage.
\VS{11}[Lamed.] Pour l'amour de ton Nom, ô Yahweh ! Tu me pardonneras mon iniquité, car elle est grande\FTNT{2 S. 24:10.}.
\VS{12}[Mem.] Qui est l'homme qui craint Yahweh ? Yahweh lui enseignera la voie qu'il doit choisir.
\VS{13}[Nun.] Son âme demeurera dans le bonheur, et sa postérité possédera la terre en héritage.
\VS{14}[Samech.] Le secret de Yahweh est pour ceux qui le craignent, et son alliance leur donne le savoir.
\VS{15}[Hajin.] Mes yeux sont continuellement sur Yahweh, car c'est lui qui sortira mes pieds du filet.
\VS{16}[Pe.] Tourne ta face vers moi, et aie pitié de moi, car je suis seul et affligé.
\VS{17}[Tsade.] Les angoisses de mon cœur augmentent ; sors-moi de ma détresse.
\VS{18}[Res.] Vois ma misère et ma peine, et pardonne tous mes péchés.
\VS{19}[Res.] Vois combien mes ennemis sont nombreux et me haïssent d'une haine pleine de violence\FTNT{Jn. 15:25.}.
\VS{20}[Scin.] Garde mon âme et délivre-moi ! Que je ne sois point confus, car je me suis réfugié en toi.
\VS{21}[Thau.] Que l'innocence et la droiture me protègent, car je m’attends à toi.
\VS{22}[Pe.] Ô Dieu ! Rachète Israël de toutes ses détresses !
\TextTitle{[Demeurer dans l'intégrité]}
\Chap{26}
\VerseOne{}Psaume de David. Yahweh, rends-moi justice\FTNT{Ps. 43:1 ; Ps. 54:3.} ! Car je marche dans l’intégrité, je me confie en Yahweh, je ne chancelle pas.
\VS{2}Sonde-moi et éprouve-moi\FTNT{Ps. 11:4-5 ; Ps. 17:3 ; Ps. 139:23.}, Yahweh ! Fais passer au creuset mes reins et mon cœur ;
\VS{3}car ta grâce est devant mes yeux, et je marche dans ta vérité.
\VS{4}Je ne m’assieds pas avec les hommes faux\FTNT{Ps. 1:1 ; 1 Co. 5:9-11 ; 1 Co. 15:33.}, et je ne vais point avec les gens dissimulés.
\VS{5}Je hais la compagnie de ceux qui font le mal\FTNT{Ps. 101:2-7 ; Ps. 119:113.}, et je ne m’assieds pas avec les méchants.
\VS{6}Je lave mes mains dans l'innocence et je fais le tour de ton autel\FTNT{Ps. 73:13.}, ô Yahweh !
\VS{7}Pour faire entendre le cri de reconnaissance, et pour raconter toutes tes merveilles.
\VS{8}Yahweh, j'aime la demeure de ta maison, le lieu dans lequel est le tabernacle de ta gloire.
\VS{9}N'enlève pas mon âme avec les pécheurs, ma vie avec les hommes de sang,
\VS{10}dont les mains sont criminelles, et la droite pleine de présents.
\VS{11}Moi, je marche dans l’intégrité ; délivre-moi et aie pitié de moi !
\VS{12}Mon pied se tient dans la droiture ; je bénirai Yahweh dans les assemblées.
\TextTitle{[La foi qui triomphe des épreuves]}
\Chap{27}
\VerseOne{}Psaume de David. Yahweh est ma lumière\FTNT{Es. 60:19-20 ; Mi. 7:8 ; Jn. 8:12 ; Ap. 21:23 ; Ps. 118:6.} et mon salut : De qui aurai-je peur ? Yahweh est le soutien de ma vie : De qui aurai-je peur ?
\VS{2}Lorsque les méchants s’avancent contre moi pour dévorer ma chair, ce sont mes adversaires et mes ennemis qui chancellent et tombent.
\VS{3}Si toute une armée campait contre moi, mon cœur ne craindrait point ; si une guerre s’élevait contre moi, je serai plein de confiance.
\VS{4}Je demande une chose à Yahweh, que je désire ardemment : C'est d’habiter dans la maison de Yahweh tous les jours de ma vie, pour contempler la beauté de Yahweh et pour admirer son temple.
\VS{5}Car il me cachera dans son tabernacle au jour du malheur, il me tiendra caché sous l’abri de sa tente, il m'élèvera sur un rocher.
\VS{6}Même maintenant ma tête s'élève par-dessus mes ennemis qui m’entourent ; et j’offrirai des sacrifices dans sa tente, au son de la trompette, je chanterai et célèbrerai Yahweh.
\VS{7}Yahweh ! Ecoute ma voix, je t'invoque : Aie pitié de moi et exauce-moi !
\VS{8}Mon cœur dit de ta part : Cherche ma face ! Je chercherai ta face, ô Yahweh !
\VS{9}Ne me cache point ta face, ne rejette point avec colère ton serviteur ! Tu es mon secours, ne me laisse pas, ne m'abandonne pas, Dieu de mon salut !
\VS{10}Car mon père et ma mère m'abandonnent, mais Yahweh me recueillera\FTNT{Es. 49:15.}.
\VS{11}Yahweh, enseigne-moi ta voie, et conduis-moi dans le sentier de la droiture, à cause de mes ennemis\FTNT{Ps. 25:4-5 ; Ps. 5:9 ; Ps. 25:4.}.
\VS{12}Ne me livre pas au désir de mes adversaires, car s’élèvent contre moi de faux témoins et des gens qui ne respirent que la violence.
\VS{13}Oh ! si je n’étais pas sûr de voir la bonté de Yahweh sur la terre des vivants…
\VS{14}Espère en Yahweh ! Fortifie-toi et que ton cœur s’affermisse\FTNT{Ps. 31:25 ; Es. 33:2.} ! Espère en Yahweh !
\TextTitle{[Louanges à Yahweh, le rocher de son peuple]}
\Chap{28}
\VerseOne{}Psaume de David. Je crie à toi, ô Yahweh ! Mon rocher, ne te rends point sourd envers moi, de peur que si tu ne me réponds pas, je ne sois semblable à ceux qui descendent dans la fosse\FTNT{Ps. 4:2 ; Ps. 143:7. 
Voir commentaire Es. 8:13-17.}.
\VS{2}Ecoute la voix de mes supplications, lorsque je crie à toi, quand j'élève mes mains vers ton saint sanctuaire.
\VS{3}Ne m’emporte pas avec les méchants ni avec les ouvriers d'iniquité, qui parlent de paix avec leur prochain pendant que la malice est dans leur cœur\FTNT{Ps. 26:9 ; Jé. 9:8.}.
\VS{4}Traite-les selon leurs œuvres et selon la malice de leurs actions, traite-les selon l'ouvrage de leurs mains, rends-leur ce qu'ils ont mérité\FTNT{2 Ti. 4:14.}.
\VS{5}Parce qu'ils ne prennent point garde aux œuvres de Yahweh, à l'œuvre de ses mains. Qu’il les renverse et ne les édifie point !
\VS{6}Béni soit Yahweh ! Car il exauce la voix de mes supplications.
\VS{7}Yahweh est ma force et mon bouclier ; mon cœur se confie en lui, et je suis secouru ; mon cœur se réjouit, c'est pourquoi je le loue par mes chants.
\VS{8}Yahweh est la force de son peuple, il est le refuge des délivrances de son oint.
\VS{9}Sauve ton peuple et bénis ton héritage ! Nourris-les et élève-les éternellement.
\TextTitle{[La toute-puissance de Dieu]}
\Chap{29}
\VerseOne{}Psaume de David. Fils de Dieu, rendez à Yahweh, rendez à Yahweh la gloire et la force\FTNT{Ps. 96:7-8.} !
\VS{2}Rendez à Yahweh la gloire due à son Nom ! Prosternez-vous devant Yahweh avec des ornements sacrés !
\VS{3}La voix de Yahweh est sur les eaux, le Dieu de gloire fait tonner ; Yahweh est sur les grandes eaux.
\VS{4}La voix de Yahweh est forte, la voix de Yahweh est majestueuse.
\VS{5}La voix de Yahweh brise les cèdres, Yahweh brise les cèdres du Liban,
\VS{6}il les fait sauter comme un veau, le Liban et le Sirion comme de jeunes buffles.
\VS{7}La voix de Yahweh fait jaillir des flammes de feu.
\VS{8}La voix de Yahweh fait trembler le désert, Yahweh fait trembler le désert de Kadès.
\VS{9}La voix de Yahweh fait naître les biches, et dépouille les forêts. Dans son palais tout s’écrie : Gloire !
\VS{10}Yahweh était assis lors du déluge ; Yahweh est assis comme roi éternellement\FTNT{Ps. 146:10.}.
\VS{11}Yahweh donne de la force à son peuple ; Yahweh bénit son peuple en paix.
\TextTitle{[De la délivrance découle la louange]}
\Chap{30}
\VerseOne{}Psaume, cantique pour la dédicace de la maison de David.
\VS{2}Yahweh, je t'exalte parce que tu m'as relevé, tu n'as pas voulu que mes ennemis se réjouissent à mon sujet.
\VS{3}Yahweh mon Dieu ! J'ai crié à toi et tu m'as guéri.
\VS{4}Yahweh ! Tu as fait remonter mon âme du scheol, tu m'as rendu la vie afin que je ne descende point dans la fosse.
\VS{5}Chantez à Yahweh, vous ses bien-aimés, et célébrez la mémoire de sa sainteté\FTNT{Ps. 97:12.} !
\VS{6}Car sa colère dure un instant, mais sa grâce toute la vie. Le soir arrivent les pleurs, et le matin les cris de louange.
\VS{7}Dans ma sécurité, je disais : Je ne serai jamais ébranlé\FTNT{Ps. 10:6.} !
\VS{8}Yahweh ! Par ta faveur tu avais affermi ma montagne… Tu cachas ta face, et je fus terrifié\FTNT{Ps. 13:2 ; Ps. 88:15 ; Ps. 102:3 ; Ps. 143:7.}.
\VS{9}Yahweh, j'ai crié à toi, j'ai présenté ma supplication à Yahweh :
\VS{10}Que gagnes-tu à verser mon sang si je descends dans la fosse ? La poussière te célébrera-t-elle ? Racontera-t-elle ta fidélité\FTNT{Es. 38:18.} ?
\VS{11}Yahweh, écoute, et aie pitié de moi ! Yahweh, secours-moi !
\VS{12}Tu as changé mon deuil en allégresse, tu as détaché mon sac et tu m'as ceint de joie\FTNT{Ps. 30:12.},
\VS{13}afin que ma langue te loue\FTNT{Ps. 57:10.} et ne se taise point. Yahweh, mon Dieu ! Je te célébrerai toujours.
\TextTitle{[Appel à la protection divine]}
\Chap{31}
\VerseOne{}Psaume de David, au chef des chantres.
\VS{2}Yahweh ! Tu es mon refuge : Que je ne sois jamais confus ! Délivre-moi par ta justice\FTNT{Ps. 25:2-20 ; Ps. 71:1-2.} !
\VS{3}Incline ton oreille vers moi, hâte-toi de me délivrer ! Sois pour moi un rocher protecteur, une forteresse, afin que je puisse m’y sauver !
\VS{4}Car tu es mon rocher, ma forteresse ; tu me dirigeras et tu me donneras du repos, à cause de ton Nom.
\VS{5}Tire-moi du filet qu’ils m’ont tendu en secret, car tu es ma vigueur.
\VS{6}Je remets mon esprit entre tes mains\FTNT{Lu. 23:46.} ; tu me rachèteras, Yahweh, Dieu de vérité !
\VS{7}Je hais ceux qui s’adonnent aux vanités trompeuses, et je me confie en Yahweh.
\VS{8}Je serai par ta bonté dans l’allégresse et la dans joie ; car tu vois mon affliction, tu sais les angoisses de mon âme,
\VS{9}Tu ne m’a pas livré entre les mains de l'ennemi, mais tu feras tenir mes pieds au large.
\VS{10}Yahweh, aie pitié de moi, car je suis dans la détresse ; mes yeux, mon âme et mon corps dépérissent de chagrin\FTNT{Ps. 6:8 ; Ps. 88:10.}.
\VS{11}Ma vie se consume dans la douleur, et mes années dans les soupirs ; ma force chancelle à cause de mon iniquité, et mes os sont consumés.
\VS{12}J'ai été un objet d’opprobre à cause de tous mes adversaires, de grand opprobre pour mes voisins, et de terreur pour ceux qui me connaissent ; ceux qui me voient dehors s'enfuient loin de moi\FTNT{Job. 19:13-14 ; Ps. 38:12.}.
\VS{13}Je suis oublié des cœurs comme un mort, je suis comme un vase détruit.
\VS{14}J’entends les calomnies de plusieurs, la crainte m’environne, quand ils se concertent unis contre moi : Ils projettent de m'ôter la vie\FTNT{Jé. 20:10.}.
\VS{15}Toutefois, je me confie en toi, ô Yahweh ! Je dis : Tu es mon Dieu !
\VS{16}Ma destinée est entre tes mains ; délivre-moi de la main de mes ennemis et de ceux qui me poursuivent !
\VS{17}Fais luire ta face sur ton serviteur\FTNT{Ps. 4:7 ; Ps. 67:2.}, délivre-moi par ta bonté !
\VS{18}Yahweh, que je ne sois point confus puisque je t'ai invoqué. Que les méchants soient confus, qu'ils soient couchés dans le scheol !
\VS{19}Que les lèvres menteuses soient muettes, elles profèrent des paroles dures contre le juste, avec orgueil et avec mépris.
\VS{20}Que ta bonté est grande\FTNT{Ps. 36:6.} ! Toi qui la réserves pour ceux qui te craignent, tu leur fais un refuge à la vue des fils de l’homme !
\VS{21}Tu les caches sous l’abri de ta face, loin du complot des hommes, tu les caches sous ton abri contre les langues querelleuses.
\VS{22}Béni soit Yahweh ! Car il a rendu merveilleuse sa bonté envers moi, comme si j’avais été dans une ville retranchée.
\VS{23}Je disais dans ma précipitation : Je suis retranché loin de ton regard ! Mais tu as entendu la voix de mes supplications quand j'ai crié vers toi.
\VS{24}Aimez Yahweh vous tous ses bien-aimés ; Yahweh garde les fidèles, et il punit sévèrement les orgueilleux.
\VS{25}Fortifiez-vous et que votre esprit s’affermisse, espérez en Yahweh\FTNT{Ps. 27:14.} !
\TextTitle{[La force du pardon]}
\Chap{32}
\VerseOne{}Cantique de David. Heureux celui à qui la transgression est pardonnée, et dont le péché est couvert !
\VS{2}Heureux l'homme à qui Yahweh n'impute point son iniquité\FTNT{Ro. 4:6-8.} et dans l'esprit duquel il n'y a point de fraude !
\VS{3}Quand je me suis tu, mes os se sont consumés, je n'ai fait que gémir tout le jour.
\VS{4}Parce que jour et nuit ta main s'appesantissait sur moi\FTNT{Ps. 38:3.}, ma vigueur s'est changée en une sécheresse d'été. Pause.
\VS{5}Je t'ai fait connaître mon péché, et je n'ai point caché mon iniquité, j'ai dit : J’avouerai mes transgressions à Yahweh\FTNT{Pr. 28:13 ; 1 Jn 1:9.} ! Et tu as porté la peine de mon péché. Pause.
\VS{6}Que tout fidèle te prie au temps convenable\FTNT{So. 2:3 ; Ps. 69:14 ; Es. 55:6.}. Si de grandes eaux débordent, elles ne l'atteindront point.
\VS{7}Tu es mon asile, tu me gardes de la détresse, tu m'environnes de chants de triomphe à cause de ta délivrance. Pause.
\VS{8}Je te rendrai intelligent, je t'enseignerai la voie dans laquelle tu dois marcher ; je te guiderai, mon oeil sera sur toi.
\VS{9}Ne soyez pas comme le cheval ni comme le mulet qui sont sans intelligence ; il faut brider leur bouche avec un mors et un frein, de peur qu'ils ne s’approchent de toi\FTNT{Job. 18:3 ; Ja. 3:3.}.
\VS{10}Beaucoup de douleurs atteindront le méchant\FTNT{Pr. 19:29.}, mais la bonté environne l'homme qui se confie en Yahweh.
\VS{11}Vous justes, réjouissez-vous en Yahweh, soyez dans l’allégresse ! Criez de joie, vous tous qui êtes droits de cœur\FTNT{Ps. 33:1 ; Ps. 64:11.} !
\TextTitle{[Louanges à Yahweh, le Dieu fidèle]}
\Chap{33}
\VerseOne{}Vous justes, poussez un cri de joie à cause de Yahweh\FTNT{Ps. 32:11 ; Ps. 97:12 ; Ps. 147:1.} ! Sa louange sied aux hommes droits.
\VS{2}Célébrez Yahweh avec la harpe, chantez-le sur le luth à dix cordes.
\VS{3}Chantez-lui un cantique nouveau\FTNT{Ps. 40:4 ; Ps. 96:1 ; Ps. 98:1 ; Ps. 144:9 ; Ap. 5:9 ; Ap. 14:3.} ! Jouez de vos instruments avec un cri de réjouissance !
\VS{4}Car la parole de Yahweh est droite, et toutes ses œuvres s’accomplissent avec fidélité ;
\VS{5}il aime la justice et la droiture\FTNT{Ps. 45:8 ; Hé. 1:9.} ; la terre est remplie de la bonté de Yahweh.
\VS{6}Les cieux ont été faits par la parole de Yahweh, et toute leur armée par le souffle de sa bouche\FTNT{Ge. 2:1.}.
\VS{7}Il amoncelle en un tas les eaux de la mer, il met les abîmes dans des réservoirs.
\VS{8}Que toute la terre craigne Yahweh ! Que tous les habitants du monde le redoutent !
\VS{9}Car il dit et la chose arrive ; il ordonne, et la chose se présente.
\VS{10}Yahweh rompt le conseil des nations, il anéantit les desseins des peuples ;
\VS{11}mais le conseil de Yahweh subsiste à toujours, les desseins de son cœur subsistent d'âge en âge\FTNT{Pr. 19:21.}.
\VS{12}Heureuse la nation dont Yahweh est le Dieu\FTNT{Ps. 144:15.} ! Et le peuple qu'il s'est choisi pour héritage !
\VS{13}Yahweh regarde des cieux, il voit tous les fils des hommes\FTNT{Job. 28:24.}.
\VS{14}Du lieu de sa demeure, il observe tous les habitants de la terre.
\VS{15}C'est lui qui forme également leur cœur et qui prend garde à toutes leurs actions.
\VS{16}Le roi n'est point sauvé par une grande armée, l'homme puissant n'échappe point par sa grande force.
\VS{17}Le cheval est impuissant pour sauver, et ne délivre point par la grandeur de sa force\FTNT{Ps. 147:10.}.
\VS{18}Voici, l'œil de Yahweh est sur ceux qui le craignent\FTNT{Ps. 34:16 ; 1 Pi. 3:12.}, sur ceux qui s'attendent à sa bonté,
\VS{19}afin qu'il les délivre de la mort, et les fasse vivre durant la famine.
\VS{20}Notre âme espère en Yahweh, il est notre aide et notre bouclier.
\VS{21}Notre cœur se réjouit en lui, car nous avons confiance en son saint Nom.
\VS{22}Que ta bonté soit sur nous, ô Yahweh ! Nous nous attendons à toi.
\TextTitle{[Yahweh sauve les siens]}
\Chap{34}
\VerseOne{}Psaume de David, lorsqu’il contrefît l’insensé en présence d'Abimélec, qui s’en alla, chassé par lui.
\VS{2}[Aleph.] Je bénirai Yahweh en tout temps, sa louange sera continuellement dans ma bouche.
\VS{3}[Beth.] Mon âme se glorifie en Yahweh ! Que les pauvres écoutent et se réjouissent.
\VS{4}[Guimel.] Glorifiez Yahweh avec moi ! Elevons son Nom tous ensemble !
\VS{5}[Daleth.] J'ai cherché Yahweh et il m'a répondu ; il m'a délivré de toutes mes frayeurs.
\VS{6}[He. Vau.] Quand on le regarde, on est illuminé, et la face n’est point confuse.
\VS{7}[Zain.] Cet affligé a crié et Yahweh l'a exaucé, et l'a délivré de toutes ses détresses.
\VS{8}[Heth.] L'ange de Yahweh campe tout autour de ceux qui le craignent, et les équipe.
\VS{9}[Teth.] Goûtez et voyez combien Yahweh est bon ! Heureux l'homme qui se confie en lui !
\VS{10}[Jod.] Craignez Yahweh vous ses saints ! Car rien ne manque à ceux qui le craignent.
\VS{11}[Caph.] Les lionceaux éprouvent la disette et la faim, mais ceux qui cherchent Yahweh ne manquent d'aucun bien.
\VS{12}[Lamed.] Venez, mes fils, écoutez-moi ! Je vous enseignerai la crainte de Yahweh.
\VS{13}[Mem.] Qui est l'homme qui prend plaisir à la vie, qui aime la prolonger pour jouir du bonheur ?
\VS{14}[Nun.] Garde ta langue du mal et tes lèvres des paroles trompeuses\FTNT{1 Pi. 3:10.} ;
\VS{15}[Samech.] détourne-toi du mal et fais-le bien ; cherche la paix et poursuis-la\FTNT{Hé. 12:14.}.
\VS{16}[Hajin.] Les yeux de Yahweh sont sur les justes et ses oreilles sont attentives à leur cri.
\VS{17}[Pe.] La face de Yahweh est contre ceux qui font le mal, pour retrancher de la terre leur mémoire\FTNT{Jé. 44:11 ; Lé. 17:10.}.
\VS{18}[Tsade.] Quand les justes crient, Yahweh les exauce et il les délivre de toutes leurs détresses.
\VS{19}[Koph.] Yahweh est près de ceux qui ont le cœur déchiré par la douleur, et il délivre ceux qui ont l'esprit abattu.
\VS{20}[Res.] Le juste a des maux en grand nombre, mais Yahweh le délivre de tous\FTNT{2 Ti. 3:11.}.
\VS{21}[Scin.] Il garde tous ses os, aucun d’eux n'est brisé.
\VS{22}[Thau.] Le mauvais tue le méchant, et ceux qui haïssent le juste sont détruits.
\VS{23}[Pe.] Yahweh rachète l'âme de ses serviteurs, et aucun de ceux qui se confient en lui ne sera détruit.
\TextTitle{Prière pour que Dieu fasse justice]}
\Chap{35}
\VerseOne{}Psaume de David. Yahweh, défends-moi contre mes adversaires, combats ceux qui me combattent !
\VS{2}Prends le petit et le grand bouclier, et lève-toi pour me secourir !
\VS{3}Brandis la lance et le javelot contre mes persécuteurs ! Dis à mon âme : Je suis ta délivrance !
\VS{4}Que ceux qui en veulent à ma vie soient honteux et confus\FTNT{Ps. 40:15 ; Ps. 70:3 ; Jé. 17:18.} ! Que ceux qui méditent ma perte reculent et rougissent !
\VS{5}Qu'ils soient comme la balle emportée par le vent\FTNT{Es. 29:5 ; Os. 13:3.}, et que l'ange de Yahweh les chasse !
\VS{6}Que leur chemin soit ténébreux et glissant, et que l'ange de Yahweh les poursuive.
\VS{7}Car sans cause ils m'ont tendu leur filet sur une fosse, sans cause ils l’ont creusée pour m’ôter la vie\FTNT{Ps. 57:6 ; Ps. 140:5 ; Ps. 141:9 ; Jé. 18:20.}.
\VS{8}Que la ruine les atteigne sans qu’ils le sachent, qu’ils soient capturés dans le filet qu’ils ont caché. Qu’ils y tombent et soient ravagés !
\VS{9}Mon âme aura de la joie en Yahweh, de l’allégresse en sa délivrance.
\VS{10}Tous mes os diront : Yahweh ! Qui est semblable à toi ? Qui délivre l'affligé de la main de celui qui est plus fort que lui ? L'affligé et le pauvre de celui qui le pille ?
\VS{11}De faux témoins s'élèvent contre moi : On m’interroge sur ce que j’ignore.
\VS{12}Ils me rendent le mal pour le bien, tâchant de m'ôter la vie\FTNT{Ps. 38:21 ; Ps. 109:5.}.
\VS{13}Mais moi, quand ils étaient malades, je me couvrais d'un sac, j'affligeais mon âme par le jeûne, je priais dans mon sein,
\VS{14}comme pour un ami, pour un frère, j’étais abattu, en pleurs, comme pour le deuil d’une mère.
\VS{15}Mais quand je chancelle, ils se réjouissent et s'assemblent, ils s’assemblent contre moi sans que je le sache pour me frapper, ils me déchirent pour que je sois silencieux ;
\VS{16}avec les hypocrites, les railleurs qui suivent les bonnes tables, ils grincent des dents contre moi.
\VS{17}Seigneur ! Jusqu'à quand le verras-tu ? Détourne mon âme de leurs tempêtes, mon unique des lionceaux.
\VS{18}Je te célébrerai dans la grande assemblée, je te louerai parmi un peuple nombreux\FTNT{Ps. 111:1.}.
\VS{19}Que ceux qui sont mes ennemis par leur mensonge ne se réjouissent point de moi, que ceux qui me haïssent sans cause ne m'insultent point par leurs regards\FTNT{Jn. 15:25.}.
\VS{20}Car ils ne parlent point de paix, mais ils préméditent des choses pleines de fraudes contre les tranquilles de la terre.
\VS{21}Ils ont ouvert leur bouche autant qu'ils ont pu contre moi, et ont dit : Ah ! Ah ! Nos yeux l’ont vu !
\VS{22}Yahweh ! Tu le vois : Ne te tais point\FTNT{Ps. 83:2.} ! Seigneur, ne t'éloigne point de moi !
\VS{23}Réveille-toi, réveille-toi pour me rendre justice\FTNT{Ps. 44:24.} ! Mon Dieu et mon Seigneur, défends ma cause !
\VS{24}Juge-moi selon ta justice, Yahweh mon Dieu ! et qu'ils ne se réjouissent point de moi !
\VS{25}Qu'ils ne disent point en leur cœur : Ah ! Notre âme ! Et qu'ils ne disent point : Nous l'avons englouti !
\VS{26}Que ceux qui se réjouissent de mon mal soient honteux et rougissent tous ensemble ! Que ceux qui s'élèvent contre moi soient couverts de honte et de confusion !
\VS{27}Mais que ceux qui prennent plaisir à ma justice se réjouissent avec des chants de triomphe, qu'ils disent sans cesse : Grand est Yahweh qui désire la paix de son serviteur !
\VS{28}Alors ma langue criera ta justice et ta louange tous les jours.
\TextTitle{[Etat du juste et du méchant]}
\Chap{36}
\VerseOne{}Psaume de David, serviteur de Yahweh, donné au chef des chantres.
\VS{2}La transgression du méchant me dit, au dedans de mon cœur, qu'il n'y a point de crainte de Dieu devant ses yeux.
\VS{3}Car il se flatte à ses propres yeux pour consumer, pour assouvir sa haine.
\VS{4}Les paroles de sa bouche ne sont que méchanceté et tromperie, il cesse d’être sage et de faire le bien.
\VS{5}Il projette le malheur sur sa couche, il se tient sur un chemin qui n'est pas bon, il ne rejette pas le mal.
\VS{6}Yahweh ! Ta bonté atteint jusqu'aux cieux, ta fidélité jusqu'aux nues\FTNT{Ps. 57:11 ; Ps. 108:5.}.
\VS{7}Ta justice est comme les montagnes de Dieu, tes jugements sont un grand abîme. Yahweh ! Tu sauves les hommes et les bêtes.
\VS{8}Ô Dieu ! Combien est précieuse ta bonté ! Aussi les fils des hommes se retirent à l'ombre de tes ailes\FTNT{Ps. 17:8 ; Ps. 57:2.}.
\VS{9}Ils seront abondamment rassasiés de la graisse de ta maison, et tu les abreuveras au fleuve de tes délices.
\VS{10}Car la source de la vie est auprès de toi, et par ta lumière nous voyons la lumière.
\VS{11}Etends ta bonté sur ceux qui te connaissent, et ta justice sur ceux qui ont le cœur droit !
\VS{12}Que le pied de l'orgueilleux ne s'avance point sur moi, et que la main des méchants ne m'ébranle point !
\VS{13}Là sont tombés les ouvriers d'iniquité ; ils sont renversés et ne peuvent se relever.
\TextTitle{[Mettre sa confiance en Yahweh]}
\Chap{37}
\VerseOne{}Psaume de David. [Aleph.] Ne t’irrite pas contre les méchants, ne jalouse pas ceux qui s'adonnent à la perversité\FTNT{Pr. 23:17 ; Pr. 24:19.}.
\VS{2}Car ils seront soudainement retranchés comme le foin, et ils se faneront comme l'herbe verte.
\VS{3}[Beth.] Confie-toi en Yahweh, et fais ce qui est bon ; aie le pays pour demeure et la fidélité pour pâture.
\VS{4}Fais de Yahweh tes délices et il t'accordera ce que ton cœur désire.
\VS{5}[Guimel.] Recommande tes voies à Yahweh, confie-toi en lui et il agira\FTNT{Ps. 22:9 ; Ps. 55:23 ; Pr. 16:3.}.
\VS{6}Il manifestera ta justice comme la lumière et ton droit comme le soleil à son midi\FTNT{Pr. 4:18.}.
\VS{7}[Daleth.] Garde le silence devant Yahweh et tremble devant lui ; ne t’irrite point contre celui qui réussit dans ses voies, contre celui qui vient à bout de ses mauvais desseins.
\VS{8}[He.] Laisse la colère et abandonne la rage\FTNT{Ep. 4:26.} ; ne t’irrite pas pour faire le mal.
\VS{9}Car les méchants seront retranchés, mais ceux qui se confient en Yahweh hériteront la terre.
\VS{10}[Vau.] Encore un peu de temps et le méchant ne sera plus ; tu regardes le lieu où il était et il n'y est plus\FTNT{Job. 7:10 ; Job. 20:9.}.
\VS{11}Les pauvres prennent possession du pays et jouissent abondamment de la paix.
\VS{12}[Zain.] Le méchant complote contre le juste et grince ses dents contre lui.
\VS{13}Le Seigneur se rit de lui, car il voit que son jour approche.
\VS{14}[Heth.] Les méchants tirent leur épée et bandent leur arc pour faire tomber le malheureux et le pauvre, pour massacrer ceux qui marchent dans la droiture\FTNT{Ps. 11:2.}.
\VS{15}Mais leur épée entre dans leur propre cœur, et leurs arcs se brisent.
\VS{16}[Teth.] Mieux vaut au juste le peu qu'il a, que l'abondance de beaucoup de méchants\FTNT{Pr. 15:16-17 ; Ec. 4:6.} ;
\VS{17}car les bras des méchants seront brisés, mais Yahweh soutient les justes.
\VS{18}[Jod.] Yahweh connaît les jours de ceux qui sont intègres, et leur héritage demeure à jamais.
\VS{19}Ils ne sont pas honteux au jour du malheur, mais ils sont rassasiés au jour de la famine.
\VS{20}[Caph.] Mais les méchants périssent, et les ennemis de Yahweh, comme les beaux pâturages, s'évanouissent, ils s’évanouissent en fumée.
\VS{21}[Lamed.] Le méchant emprunte et ne rend point ; mais le juste a compassion et donne.
\VS{22}Car les bénis de Yahweh hériteront la terre, mais ceux qu'il a maudits seront retranchés.
\VS{23}[Mem.] Yahweh affermit les pas de l’homme, et il prend plaisir à ses voies.
\VS{24}S'il tombe, il ne sera pas entièrement abattu, car Yahweh le soutient de sa main.
\VS{25}[Nun.] J'ai été jeune et j'ai vieilli ; et je n'ai point vu le juste abandonné ni sa postérité mendiant son pain.
\VS{26}Il est compatissant tout le temps, et il prête ; et sa postérité est bénie.
\VS{27}[Samech.] Retire-toi du mal et fais le bien ; et tu auras une demeure éternelle.
\VS{28}Car Yahweh aime ce qui est juste, et il n'abandonne point ses fidèles ; c'est pourquoi ils sont sous sa garde pour toujours, mais la postérité des méchants est retranchée.
\VS{29}[Hajin.] Les justes hériteront la terre et y habiteront à perpétuité.
\VS{30}[Pe.] La bouche du juste prononce la sagesse et sa langue déclare la justice.
\VS{31}La loi de son Dieu est dans son cœur\FTNT{Ps. 40:8.}, aucun de ses pas ne chancellera.
\VS{32}[Tsade.] Le méchant épie le juste et cherche à le faire mourir.
\VS{33}Yahweh ne l'abandonne point entre ses mains et ne le laisse point condamner quand on le juge.
\VS{34}[Koph.] Espère en Yahweh et garde sa voie, et il t’élèvera pour que tu hérites la terre ; tu verras les méchants retranchés.
\VS{35}[Res.] J'ai vu le méchant dans toute sa puissance, il s’étendait comme un arbre verdoyant.
\VS{36}Il a passé, et voici, il n'est plus ; je le cherche et il ne se trouve plus.
\VS{37}[Scin.] Observe l'homme intègre et considère l'homme droit, car il y a une issue pour l’homme de paix.
\VS{38}Mais les rebelles seront tous détruits et ce qui sera resté des méchants sera retranché.
\VS{39}[Thau.] Mais la délivrance des justes viendra de Yahweh, il sera leur force au temps de la détresse.
\VS{40}Yahweh les secourt et les délivre ; il les délivre des méchants et les sauve, parce qu'ils se confient en lui.
\TextTitle{[Tristesse selon Dieu qui emmène à la repentance]}
\Chap{38}
\VerseOne{}Psaume de David. Pour souvenir.
\VS{2}Yahweh ! Ne me juge pas dans ta colère et ne me châtie pas dans ta fureur.
\VS{3}Car tes flèches m’ont atteint, et ta main s'est appesantie sur moi.
\VS{4}Il n'y a rien de sain dans ma chair, à cause de ta colère, ni de paix dans mes os, à cause de mon péché.
\VS{5}Car mes iniquités s’élèvent au-dessus de ma tête, elles se sont appesanties comme un pesant fardeau, au-delà de mes forces\FTNT{Ps. 40:13.}.
\VS{6}Mes plaies ont une mauvaise odeur et sont purulentes à cause de ma folie.
\VS{7}Je suis courbé et abattu outre mesure ; je marche en pleurs tout le jour.
\VS{8}Car un mal brûlant remplit mes reins, et dans ma chair il n'y a rien de sain.
\VS{9}Je suis affaibli et brisé, je rougis le cœur troublé.
\VS{10}Seigneur, tout mon désir est devant toi, et mon soupir ne t'est point caché.
\VS{11}Mon cœur est agité çà et là, ma force m'abandonne, et la lumière de mes yeux n’est plus avec moi.
\VS{12}Ceux qui m'aiment, et même mes amis intimes, se tiennent loin de ma plaie, et mes proches se tiennent loin de moi\FTNT{Job. 19:13-14.}.
\VS{13}Ceux qui en veulent à ma vie me tendent des pièges ; ceux qui cherchent ma perte parlent de calamités et méditent des tromperies tous les jours.
\VS{14}Mais moi je suis comme un sourd, comme un muet qui n'ouvre point sa bouche.
\VS{15}Je suis, dis-je, comme un homme qui n'entend pas et qui n'a point de réplique dans sa bouche.
\VS{16}Car je m’attends à toi, ô Yahweh ! Tu me répondras, Seigneur mon Dieu !
\VS{17}Je dis : Il faut prendre garde qu'ils ne triomphent de moi, quand mon pied chancelle, ils s'élèvent contre moi\FTNT{Ps. 94:18.} !
\VS{18}Car je suis près de tomber et ma douleur est continuellement devant moi.
\VS{19}Car je reconnais mon iniquité et je suis dans la crainte à cause de mon péché.
\VS{20}Cependant mes ennemis qui sont vivants se renforcent, et ceux qui me haïssent à tort se multiplient.
\VS{21}Ceux qui me rendent le mal pour le bien sont mes adversaires, parce que je recherche le bien\FTNT{Ps. 109:5 ; Jé. 18:20.}.
\VS{22}Ne m'abandonne pas Yahweh ! Mon Dieu, ne t'éloigne pas de moi !
\VS{23}Hâte-toi de venir à mon secours, Seigneur, tu es ma délivrance !
\TextTitle{[La fragilité de l'homme]}
\Chap{39}
\VerseOne{}Psaume de David, donné au chef des chantres, à Jeduthun.
\VS{2}J'ai dit : Je prends garde à mes voies, de peur de pécher par ma langue ; je mettrai un frein à ma bouche tant que le méchant sera devant moi.
\VS{3}Je suis resté muet, dans le silence ; je me suis tu, quoique malheureux ; et ma douleur n’était pas moins vive.
\VS{4}Mon cœur brûlait au-dedans de moi, un feu intérieur me consumait, et la parole est venue sur ma langue.
\VS{5}Yahweh ! Dis-moi quel est le terme de ma vie et quelle est la mesure de mes jours\FTNT{Ps. 119:84.} ; que je sache combien je suis fragile.
\VS{6}Voici, tu as réduit mes jours à la largeur de ma main, et ma vie est comme un rien devant toi. Oui, tout homme debout n’est qu’un souffle\FTNT{Ja. 4:14.}. Pause.
\VS{7}Oui, l'homme se promène comme une ombre, il s'agite inutilement ; il amasse des biens et il ne sait pas qui les recueillera.
\VS{8}Maintenant que puis-je espérer, Seigneur ? Mon espérance est en toi.
\VS{9}Délivre-moi de toutes mes transgressions ! Ne permets pas l’opprobre des insensés.
\VS{10}Je me suis tu et je n'ai point ouvert ma bouche, parce que c'est toi qui agis.
\VS{11}Détourne de moi tes coups ! Je suis consumé par les attaques de ta main.
\VS{12}Aussitôt que tu châties quelqu'un, en le punissant à cause de son iniquité, tu détruis comme la teigne ce qu’il a de plus cher. Oui, tout homme est une vapeur. Pause.
\VS{13}Yahweh, écoute ma prière et prête l'oreille à mon cri ! Ne sois point sourd à mes larmes ! Car je suis un voyageur et un étranger chez toi, comme tous mes pères\FTNT{1 Pi. 2:11 ; Hé. 11:13 ; Ps. 119:19 ; Lé. 25:23.}.
\VS{14}Détourne ton regard de moi, afin que je reprenne mes forces, avant que je m'en aille et que je ne sois plus.
\TextTitle{[Un cantique nouveau à Yahweh]}
\Chap{40}
\VerseOne{}Psaume de David, donné au chef des chantres.
\VS{2}J'ai attendu patiemment Yahweh, et il s'est tourné vers moi et a entendu mon cri.
\VS{3}Il m'a retiré de la fosse de destruction, du fond de la boue ; il a mis mes pieds sur un roc et a assuré mes pas.
\VS{4}Il a mis dans ma bouche un cantique nouveau, qui est la louange de notre Dieu ; plusieurs verront cela et ils craindront et se confieront en Yahweh.
\VS{5}Heureux l'homme qui place sa confiance en Yahweh et qui ne se tourne pas vers les orgueilleux et les menteurs !
\VS{6}Yahweh, mon Dieu ! Tu as multiplié tes merveilles et tes desseins envers nous ; nul n’est comparable à toi ; je voudrais les annoncer et les déclarer, mais leur nombre est trop grand pour que je les raconte.
\VS{7}Tu ne désires ni sacrifice ni offrande. Tu m'as percé les oreilles ; tu ne demandes ni holocauste ni victime expiatoire pour le péché\FTNT{Hé. 10:5.}.
\VS{8}Alors je dis : Voici, je viens avec le rouleau du livre écrit pour moi.
\VS{9}Mon Dieu, je prends plaisir à faire ta volonté, et ta loi est au fond de mes entrailles\FTNT{Ps. 37:31 ; Es. 51:7.}.
\VS{10}J’annonce ta justice dans la grande assemblée ; voilà, je ne ferme pas mes lèvres, Yahweh, tu le sais !
\VS{11}Je ne cache pas ta justice, qui est dans mon cœur ; je déclare ta fidélité et ta délivrance ; je ne cache pas ta bonté ni ta vérité dans la grande assemblée.
\VS{12}Toi, Yahweh ! Ne m'épargne point tes compassions, que ta bonté et ta vérité me gardent continuellement.
\VS{13}Car des maux sans nombre m'environnent ; mes iniquités m'atteignent, et je ne supporte pas leur vue ; elles surpassent en nombre les cheveux de ma tête, et mon cœur m'abandonne.
\VS{14}Yahweh, veuille me délivrer ! Yahweh, hâte-toi de venir à mon secours !
\VS{15}Que tous ensemble ils soient honteux et confus, ceux qui cherchent mon âme pour la perdre ; et que ceux qui prennent plaisir à mon malheur retournent en arrière et rougissent.
\VS{16}Que ceux qui disent de moi : Ah ! Ah ! Soient consumés, en récompense de la honte qu'ils m'ont faite.
\VS{17}Que tous ceux qui te cherchent soient dans l’allégresse et se réjouissent en toi\FTNT{Ps. 70:4.} ! Que ceux qui aiment ta délivrance disent continuellement : Grand est Yahweh !
\VS{18}Moi, je suis affligé et misérable, mais le Seigneur prend soin de moi. Tu es mon secours et mon libérateur : Mon Dieu ne tarde point\FTNT{Ps. 70:6.} !
\TextTitle{[Intervention de Yahweh dans le malheur]}
\Chap{41}
\VerseOne{}Psaume de David, donné au chef des chantres.
\VS{2}Heureux celui qui s’intéresse au pauvre ! Yahweh le délivrera au jour du malheur ;
\VS{3}Yahweh le garde et lui conserve la vie. Il est heureux sur la terre, et tu ne le livres pas au bon plaisir de ses ennemis.
\VS{4}Yahweh le soutient sur son lit de douleur ; tu le soulages dans toutes ses maladies.
\VS{5}Je dis : Yahweh ! Aie pitié de moi, guéris mon âme, car j’ai péché contre toi.
\VS{6}Mes ennemis disent du mal de moi : Quand mourra-t-il ? Et quand périra son nom ?
\VS{7}Si quelqu'un vient me voir, il dit des mensonges, il recueille de mauvais desseins\FTNT{Ps. 5:10 ; Ps. 10:7 ; Ps. 12:3.}, il s’en va et il parle au-dehors.
\VS{8}Tous ceux qui m’ont en haine murmurent sourdement ensemble contre moi, et machinent du mal contre moi.
\VS{9}Quelque action criminelle\FTNT{Le mot "criminelle" donne en hébreu "Belial"} pèse sur lui ; le voilà couché, disent–ils, il ne se relèvera plus !
\VS{10}Même celui qui était en paix avec moi, qui avait ma confiance et qui mangeait mon pain, a levé le talon contre moi\FTNT{Il est question ici de la trahison du Messie par Judas (Jn. 13:18-19).}.
\VS{11}Mais toi, ô Yahweh ! Aie pitié de moi et relève-moi ! Et je leur rendrai ce qui leur est dû.
\VS{12}Je connaîtrai que tu prends plaisir en moi, si mon ennemi ne triomphe pas de moi.
\VS{13}Pour moi, tu m'as soutenu à cause de mon intégrité, et tu m'as établi pour toujours en ta présence.
\VS{14}Béni soit Yahweh, le Dieu d'Israël, d’éternité en éternité. Amen ! Amen !
\TextTitle{[Avoir soif de Dieu]}
\Chap{42}
\VerseOne{}Cantique des fils de Koré, donné au chef des chantres.
\VS{2}Comme une biche soupire après des courants d’eau, ainsi mon âme soupire ardemment après toi, ô Dieu !
\VS{3}Mon âme a soif de Dieu, du Dieu vivant\FTNT{Ps. 63:2 ; Ps. 84:3.} : Ô quand entrerai-je et me présenterai-je devant la face de Dieu ?
\VS{4}Mes larmes sont ma nourriture jour et nuit, quand on me dit chaque jour : Où est ton Dieu\FTNT{Ps. 80:6 ; Ps. 115:2.} ?
\VS{5}Je rappelais ces choses dans mon souvenir, en répandant mon âme au-dedans de moi, savoir que je marchais dans la foule, et que je m’en allais tout doucement en leur compagnie, avec une voix de triomphe et de louange, jusqu’à la maison de Dieu, et qu’une grande multitude de gens sautait alors de joie.
\VS{6}Mon âme, pourquoi t'abats-tu et murmures-tu au-dedans de moi ? Attends-toi à Dieu, car je le célébrerai encore ; sa face est la délivrance même.
\VS{7}Mon Dieu ! Mon âme est abattue au-dedans de moi, aussi je me souviens de toi depuis la terre du Jourdain, depuis l’Hermon, et depuis la montagne de Mitsear.
\VS{8}Un flot appelle un autre flot au bruit de tes ondées ; toutes tes vagues et tes flots passent sur moi.
\VS{9}Toutefois, Yahweh enverra sa bonté compatissante de jour, et de nuit son cantique sera avec moi, et ma prière sera au Dieu qui est ma vie.
\VS{10}Je dis à Dieu, mon rocher : Pourquoi m'oublies-tu ? Pourquoi marcherais-je dans la tristesse à cause de l'oppression de l'ennemi ?
\VS{11}Comme avec une épée dans mes os, mes ennemis m’outragent, tandis qu’ils me disent chaque jour : Où est ton Dieu ?
\VS{12}Mon âme, pourquoi t'abats-tu et pourquoi murmures-tu au-dedans de moi ? Attends-toi à Dieu, car je le célébrerai encore, il est ma délivrance et mon Dieu.
\TextTitle{[Espérer en Dieu]}
\Chap{43}
\VerseOne{}Fais-moi justice, ô Dieu ! Et défends ma cause contre une nation infidèle\FTNT{Ps. 26:1 ; Ps. 35:1.} ! Délivre-moi de l'homme trompeur et pervers.
\VS{2}Toi, mon Dieu protecteur, pourquoi me repousses-tu ? Pourquoi marcherais-je dans la tristesse à cause de l'oppression de l'ennemi ?
\VS{3}Envoie ta lumière et ta vérité, afin qu'elles me conduisent et m'introduisent dans ta sainte montagne, et dans tes demeures.
\VS{4}Alors je viendrai à l'autel de Dieu, au Dieu de ma joie et mon allégresse, et je te célébrerai sur la harpe, ô Dieu ! Mon Dieu !
\VS{5}Mon âme, pourquoi t'abats-tu et pourquoi murmures-tu au-dedans de moi ? Attends-toi à Dieu, car je le célébrerai encore ; il est ma délivrance et mon Dieu\FTNT{Ps. 42:6.}.
\TextTitle{[Yaweh se lève pour les affligés]}
\Chap{44}
\VerseOne{}Cantique des fils de Koré, donné au chef des chantres.
\VS{2}Ô Dieu ! Nous avons entendu de nos oreilles et nos pères nous ont raconté les exploits que tu as faits de leur temps, aux jours d'autrefois\FTNT{Jg. 6:13 ; Ps. 77:12.}.
\VS{3}Tu as de ta main chassé les nations, et tu as affermi nos pères, tu as affligé les peuples, et tu as fait prospérer nos pères.
\VS{4}Car ce n'est point par leur épée qu'ils ont conquis le pays, et ce n'est pas leur bras qui les a délivrés, mais c’est ta droite, c’est ton bras, c’est la lumière de ta face, parce que tu les aimais.
\VS{5}Ô Dieu ! Tu es mon roi : Ordonne la délivrance de Jacob !
\VS{6}Avec toi nous battrons nos adversaires, par ton Nom nous foulerons ceux qui s'élèvent contre nous.
\VS{7}Car je ne me confie point en mon arc, et ce n’est pas mon épée qui me délivrera.
\VS{8}Mais tu nous délivreras de nos adversaires, et tu rendras confus ceux qui nous haïssent.
\VS{9}Nous nous glorifierons en Dieu chaque jour et nous célébrerons à jamais ton Nom. Pause.
\VS{10}Mais tu nous rejettes, tu nous confonds, et tu ne sors plus avec nos armées.
\VS{11}Tu nous fais reculer devant l'adversaire, et ceux qui nous haïssent enlèvent nos dépouilles.
\VS{12}Tu nous livres comme des brebis destinées à être dévorées, et tu nous as dispersés parmi les nations.
\VS{13}Tu as vendu ton peuple pour rien, et tu ne l’estimes pas d’une grande valeur\FTNT{Es. 52:3 ; Jé. 15:13.}.
\VS{14}Tu nous as mis en opprobre chez nos voisins, en dérision, et en sujet de moquerie auprès de ceux qui habitent autour de nous\FTNT{Jé. 24:9 ; Ps. 79:4.}.
\VS{15}Tu fais de nous un objet de sarcasmes parmi les nations, et de hochement de tête parmi les peuples.
\VS{16}Ma confusion est tout le jour devant moi, et la honte couvre ma face,
\VS{17}à cause des discours de celui qui nous fait des reproches et qui nous injurie, et à cause de l'ennemi et du vindicatif.
\VS{18}Tout cela nous est arrivé, et cependant nous ne t'avons point oublié, et nous n'avons point violé ton alliance.
\VS{19}Notre cœur ne s’est point détourné, nos pas ne se sont point éloignés de tes sentiers,
\VS{20}pour que tu nous écrases dans le lieu du serpent, et que tu nous couvres de l'ombre de la mort\FTNT{Ps. 23:4.}.
\VS{21}Si nous avions oublié le Nom de notre Dieu et étendu nos mains vers un dieu étranger,
\VS{22}Dieu ne le saurait-il pas, lui qui connaît les secrets du cœur ?
\VS{23}Mais nous sommes tous les jours mis à mort pour l'amour de toi, nous sommes regardés comme des brebis destinées à la boucherie\FTNT{Es. 53:7.}.
\VS{24}Lève-toi, pourquoi dors-tu Seigneur ? Réveille-toi ! Ne nous rejette point à jamais !
\VS{25}Pourquoi caches-tu ta face, pourquoi oublies-tu notre affliction et notre oppression ?
\VS{26}Car notre âme est abattue dans la poussière et notre ventre est attaché à la terre.
\VS{27}Lève-toi pour nous secourir ! Délivre-nous à cause de ta bonté.
\TextTitle{[La beauté du Roi]}
\Chap{45}
\VerseOne{}Cantique des fils de Koré, qui est un chant nuptial donné au chef des chantres pour le chanter sur les lis.
\VS{2}Des paroles agréables bouillonnent dans mon cœur, et j'ai dit : Mon œuvre est pour le roi ! Ma langue sera comme la plume d'un habile écrivain !\FTNT{Ca. 5:13 ; Ca. 5:16.}
\VS{3}Tu es le plus beau des fils de l’homme, la grâce est répandue sur tes lèvres : C’est pourquoi Dieu t'a béni éternellement.
\VS{4}Ô Très–puissant, ceins ton épée sur ta cuisse, ta majesté et ta magnificence,
\VS{5}et prospère dans ta magnificence. Sois porté sur la parole de vérité, de douceur, et de justice, et ta droite versera des choses terribles !
\VS{6}Tes flèches sont aiguës, les peuples tomberont sous toi, elles perceront le cœur des ennemis du roi.
\VS{7}Ô Dieu, ton trône est à toujours et à perpétuité ! Le sceptre de ton règne est un sceptre d'équité.
\VS{8}Tu aimes la justice et tu hais la méchanceté : C'est pourquoi, ô Dieu, ton Dieu t'a oint d'une huile de joie par privilège sur tes compagnons\FTNT{Hé. 1:8-9.}.
\VS{9}Tous tes vêtements sont parfumés de myrrhe, d’aloès et de casse. Dans les palais d'ivoire les instruments à cordes te réjouissent.
\VS{10}Des filles de rois sont parmi tes bien-aimées ; la reine est à ta droite, parée d'or d'Ophir.
\VS{11}Ecoute, jeune fille, vois et prête l'oreille ; oublie ton peuple et la maison de ton père.
\VS{12}Le roi porte ses désirs sur ta beauté ; puisqu'il est ton Seigneur, prosterne-toi devant lui.
\VS{13}La fille de Tyr et les plus riches des peuples te supplieront avec des présents.
\VS{14}La fille du roi est intérieurement pleine de gloire. Elle porte un vêtement tissé d'or.
\VS{15}Elle sera présentée au roi en vêtements de broderie, et les filles qui viennent après elle, et qui sont ses compagnes, seront amenées vers toi.
\VS{16}Elles te seront présentées avec réjouissance et allégresse, et elles entreront au palais du roi.
\VS{17}Tes fils seront au lieu de tes pères, tu les établiras pour princes sur toute la terre.
\VS{18}Je rendrai ton Nom mémorable dans tous les âges, et à cause de cela les peuples te célébreront pour toujours et à perpétuité\FTNT{Ps. 67:3-5.}.
\TextTitle{[L'assurance du peuple de Dieu]}
\Chap{46}
\VerseOne{}Cantique des fils de Koré, donné au chef des chantres pour le chanter sur Alamoth. Cantique.
\VS{2}Dieu est notre retraite, notre force, et notre secours qui ne manque jamais dans les détresses\FTNT{Ps. 9:10.}.
\VS{3}C'est pourquoi nous ne craindrons point quand la terre est bouleversée et que les montagnes chancellent au cœur des mers\FTNT{Es. 54:10.},
\VS{4}quand ses eaux mugissent et écument, se soulèvent jusqu’à faire trembler les montagnes. Pause.
\VS{5}Il est un fleuve dont les courants réjouissent la cité de Dieu, le lieu saint des demeures du Très-Haut\FTNT{Ez. 47:1-2 ; Jn. 7:38 ; Za. 14:8-9 ; Ap. 22:1-2.}.
\VS{6}Dieu est au milieu d'elle : Elle n’est point ébranlée. Dieu la secourt dès le point du jour\FTNT{So. 3:16-17.}.
\VS{7}Les nations murmurent, les royaumes s’ébranlent ; il a fait entendre sa voix et la terre se fond.
\VS{8}Yahweh des armées est avec nous, le Dieu de Jacob est pour nous une haute retraite. Pause.
\VS{9}Venez, contemplez les œuvres de Yahweh et voyez quels ravages il a faits sur la terre.
\VS{10}Il a fait cesser les guerres jusqu’au bout de la terre, il a brisé l’arc et rompu la lance, il a consumé par le feu les chars de guerre\FTNT{Es. 2:4.}.
\VS{11}Arrêtez, et sachez que je suis Dieu : Je suis élevé parmi les nations, je suis élevé sur toute la terre.
\VS{12}Yahweh des armées est avec nous, le Dieu de Jacob est pour nous une haute retraite. Pause.
\TextTitle{[Yahweh, le Dieu élevé]}
\Chap{47}
\VerseOne{}Psaume des fils de Koré, donné au chef des chantres.
\VS{2}Peuples battez tous des mains ! Poussez vers Dieu des cris de joie avec une voix de triomphe !
\VS{3}Car Yahweh, le Très-Haut, est terrible. Il est un grand Roi sur toute la terre.
\VS{4}Il nous assujettit des peuples et des nations sous nos pieds.
\VS{5}Il nous choisit notre héritage, la gloire de Jacob qu’il aime. Pause.
\VS{6}Dieu est monté avec un cri de réjouissance, Yahweh monte au son du shofar.
\VS{7}Chantez à Dieu, chantez ! Chantez à notre roi, chantez !
\VS{8}Car Dieu est le Roi de toute la terre : Chantez un cantique !
\VS{9}Dieu règne sur les nations, Dieu est assis sur son saint trône.
\VS{10}Les princes des peuples se rassemblent vers le peuple du Dieu d'Abraham, car les boucliers de la terre sont à Dieu : Il est puissamment élevé.
\TextTitle{[Sion, la cité du grand roi]}
\Chap{48}
\VerseOne{}Cantique. Psaume des fils de Koré.
\VS{2}Yahweh est grand, il est l’objet de toutes les louanges dans la ville de notre Dieu, sur sa montagne sainte.
\VS{3}Belle est la colline, joie de toute la terre, la montagne de Sion, le côté nord, c'est la ville du grand Roi.
\VS{4}Dieu est connu dans ses palais pour une haute retraite.
\VS{5}Ils l’ont vue, et aussitôt ils ont été émerveillés ; ils ont été troublés et se sont enfuis à la hâte.
\VS{6}Ils ont regardé tout stupéfaits, ils ont eu peur et ont pris la fuite.
\VS{7}Là un tremblement les a saisis, une douleur comme celle de l’enfantement\FTNT{Es. 13:8.}.
\VS{8}Ils ont été chassés comme par le vent d'orient qui brise les navires de Tarsis.
\VS{9}Comme nous l'avions entendu, ainsi l'avons-nous vu dans la ville de Yahweh des armées, dans la ville de notre Dieu : Dieu l’établira à toujours. Pause.
\VS{10}Ô Dieu ! Nous pensons à ta bonté au milieu de ton temple.
\VS{11}Ô Dieu ! Comme ton Nom, ta louange retentit jusqu'aux extrémités de la terre ; ta droite est pleine de justice.
\VS{12}La montagne de Sion se réjouit, et les filles de Juda sont dans la joie, à cause de tes jugements.
\VS{13}Entourez Sion, faites-en le tour, comptez ses tours.
\VS{14}Observez son rempart, examinez ses palais pour le raconter à la génération future.
\VS{15}Car ce Dieu-là est notre Dieu éternellement et à jamais ; il nous accompagnera jusqu’à la mort.
\TextTitle{[Vanité des richesses terrestres]}
\Chap{49}
\VerseOne{}Psaume des fils de Koré, au chef des chantres.
\VS{2}Vous tous peuples, entendez ceci, vous habitants du monde, prêtez l'oreille,
\VS{3}petits et grands, riches et pauvres !
\VS{4}Ma bouche prononcera des discours pleins de sagesse, et les pensées de mon cœur sont pleines de sens.
\VS{5}Je prête l'oreille aux sentences qui me sont inspirées, je poserai mes questions au son de la harpe.
\VS{6}Pourquoi craindrais-je au jour du malheur, quand l'iniquité de mes adversaires m'entoure ?
\VS{7}Ils mettent leur confiance dans leurs biens et se glorifient de l'abondance de leurs richesses.
\VS{8}Ils ne peuvent se racheter l’un l’autre ni donner à Dieu le prix du rachat\FTNT{Mt. 16:26 ; Mc. 8:36-37 ; Lu. 12:15-21.}.
\VS{9}Car le rachat de leur âme est trop considérable, et il ne se fera jamais ;
\VS{10}ils ne vivront pas toujours et n’éviteront pas la vue de la fosse.
\VS{11}Car on voit que les sages meurent, l’insensé et le stupide périssent également, et ils laissent à d’autres leurs biens\FTNT{Ec. 2:21 ; Ec. 6:2.}.
\VS{12}Leur intention est que leurs maisons durent éternellement, et que leurs habitations demeurent d'âge en âge, ils ont donné leurs noms à leurs terres.
\VS{13}mais l'homme qui est en honneur n’a point de durée, il est semblable aux bêtes que l’on égorge.
\VS{14}Tel est leur chemin, leur folie, et ceux qui les suivent se plaisent à leurs discours. Pause.
\VS{15}Ils seront mis dans le scheol comme des brebis, la mort en fait sa pâture, et au matin les hommes droits les foulent aux pieds, leur beau rocher s’use, le scheol est leur résidence\FTNT{Job. 24:19.}.
\VS{16}Mais Dieu rachètera mon âme du pouvoir du scheol, quand il m’enlèvera de sa captivité\FTNT{Ps. 68:19 ; Ep. 4:8-9.}. Pause.
\VS{17}Ne crains point quand tu verras quelqu'un s’enrichir et quand les trésors de sa maison se multiplient.
\VS{18}Car lorsqu'il mourra, il n'emportera rien, ses trésors ne descendront point après lui\FTNT{Job. 27:19 ; 1 Ti. 6:7.}.
\VS{19}Il aura beau s’estimer heureux pendant sa vie, on aura beau te louer des jouissances que tu te donnes,
\VS{20}tu iras néanmoins au séjour de tes pères, qui jamais ne reverront la lumière.
\VS{21}L'homme qui est en honneur, qui n'a pas d'intelligence, est semblable aux bêtes que l’on égorge.
\TextTitle{[Les actions de grâces offertes à Dieu]}
\Chap{50}
\VerseOne{}Psaume d'Asaph. Le Dieu puissant, Dieu, Yahweh a parlé et il a appelé toute la terre, depuis le soleil levant jusqu’au soleil couchant.
\VS{2}De Sion Dieu a fait luire sa splendeur qui est d'une beauté parfaite,
\VS{3}notre Dieu viendra, il ne se taira point : Il y aura devant lui un feu dévorant, et tout autour de lui une grosse tempête.
\VS{4}Il appellera les cieux d'en haut, et la terre pour juger son peuple :
\VS{5}Rassemblez-moi mes bien-aimés qui ont traité alliance avec moi par le sacrifice\FTNT{Mt. 24:29-31.}.
\VS{6}Les cieux aussi annonceront sa justice parce que Dieu est le juge. Pause.
\VS{7}Ecoute, ô mon peuple ! Et je parlerai. Entends, Israël ! Et je t’avertirai. Moi je suis Dieu, ton Dieu.
\VS{8}Je ne te réprimande pas pour tes sacrifices, tes holocaustes sont continuellement devant moi.
\VS{9}Je ne prendrai point de taureaux de ta maison ni de boucs de tes bergeries\FTNT{Ps. 40:7.}.
\VS{10}Car tous les animaux des forêt sont à moi, toutes les bêtes qui paissent sur mille montagnes.
\VS{11}Je connais tous les oiseaux des montagnes, et tout ce qui se meut dans les champs m’appartient.
\VS{12}Si j'avais faim, je ne t'en dirais rien, car le monde est à moi et tout ce qu’il renferme.
\VS{13}Mangerais-je la chair des gros taureaux ? Et boirais-je le sang des boucs ?
\VS{14}Offre à Dieu la reconnaissance et accomplis tes vœux envers le Très-Haut.
\VS{15}Invoque-moi au jour de ta détresse, je te délivrerai, et tu me glorifieras\FTNT{Ps. 37:5.}.
\VS{16}Dieu dit au méchant : Quoi donc ? Tu énumères mes lois ! Et tu as mon alliance dans ta bouche !
\VS{17}Toi qui hais la correction, et qui jettes mes paroles derrière toi !
\VS{18}Si tu vois un voleur, tu te plais avec lui, et ta part est avec les adultères.
\VS{19}Tu livres ta bouche au mal, et ta langue est un tissu de tromperies.
\VS{20}Tu t'assieds et parles contre ton frère, tu couvres d'opprobre le fils de ta mère.
\VS{21}Tu as fait ces choses-là, et je me suis tu. Tu as estimé que je te ressemble, mais je vais te reprendre et tout mettre sous tes yeux.
\VS{22}Comprenez cela maintenant, vous qui oubliez Dieu, de peur que je ne déchire sans que personne ne vous délivre.
\VS{23}Celui qui offre la louange me glorifie, et à celui qui veille sur sa voie, je lui montrerai le salut de Dieu.
\TextTitle{[Le coeur repentant]}
\Chap{51}
\VerseOne{}Psaume de David, au chef des chantres.
\VS{2}Lorsque Nathan le prophète vint à lui, après que David fut allé vers Bath-Schéba\FTNT{2 Sa. 11 et 12.}.
\VS{3}Ô Dieu ! Aie pitié de moi dans ta bonté, selon ta grande miséricorde, efface mes transgressions ;
\VS{4}lave-moi parfaitement de mon iniquité et purifie-moi de mon péché.
\VS{5}Car je reconnais mes transgressions, et mon péché est continuellement devant moi\FTNT{Es. 59:12.}.
\VS{6}J'ai péché contre toi, contre toi seul, et j'ai fait ce qui déplaît à tes yeux : En sorte que tu seras juste dans ta sentence, sans reproche dans ton jugement.
\VS{7}Voici, je suis né dans l'iniquité, et ma mère m'a conçu dans le péché.
\VS{8}Mais tu prends plaisir à la vérité au fond du cœur, et tu me fais connaître la sagesse au-dedans de moi.
\VS{9}Purifie-moi de mon péché avec de l'hysope, et je serai pur ; lave-moi, et je serai plus blanc que la neige.
\VS{10}Fais-moi entendre la joie et l'allégresse, et les os que tu as brisés se réjouiront.
\VS{11}Détourne ta face de mes péchés, et efface toutes mes iniquités.
\VS{12}Ô Dieu ! Crée en moi un cœur pur et renouvelle en moi un esprit ferme\FTNT{Mt. 5:8.}.
\VS{13}Ne me rejette pas loin de ta face et ne m'ôte pas ton Esprit Saint.
\VS{14}Rends-moi la joie de ton salut et qu’un esprit bien disposé me soutienne.
\VS{15}J'enseignerai tes voies aux transgresseurs et les pécheurs reviendront à toi.
\VS{16}Ô Dieu, Dieu de mon salut ! Délivre-moi de tant de sang, et ma langue chantera hautement ta justice.
\VS{17}Seigneur, ouvre mes lèvres, et ma bouche annoncera ta louange.
\VS{18}Car tu ne prends point plaisir aux sacrifices, autrement j'en donnerais ; l'holocauste ne t'est point agréable.
\VS{19}Les sacrifices à Dieu, c’est un esprit brisé. Ô Dieu ! Tu ne méprises point un cœur brisé et contrit.
\VS{20}Répands par ta grâce, tes bienfaits sur Sion, édifie les murs de Jérusalem.
\VS{21}Alors tu prendras plaisir aux sacrifices de justice, à l'holocauste, et aux sacrifices qui se consument entièrement par le feu ; alors on offrira des taureaux sur ton autel.
\TextTitle{[Sort de l'homme qui se confie en ses richesses]}
\Chap{52}
\VerseOne{}Cantique de David, donné au chef des chantres.
\VS{2}A l’occasion du rapport que Doëg, l’Edomite, vint faire à Saül, en lui disant : David s’est rendu dans la maison d’Achimélec.
\VS{3}Pourquoi te vantes-tu du mal, vaillant homme ? La bonté de Dieu dure à toujours.
\VS{4}Ta langue trame des méchancetés, elle est comme un rasoir affilé qui trompe.
\VS{5}Tu aimes plus le mal que le bien, le mensonge plutôt que de dire la vérité. Pause.
\VS{6}Tu aimes tous les discours pernicieux, le langage trompeur.
\VS{7}Aussi Dieu te détruira pour toujours, il t'enlèvera et t'arrachera de ta tente ; il te déracinera de la terre des vivants. Pause.
\VS{8}Les justes le verront et auront de la crainte, et ils se riront d'un tel homme, disant :
\VS{9}Voilà cet homme qui ne tenait point Dieu pour sa protection, mais qui se confiait en ses grandes richesses et qui mettait sa force dans ses mauvais désirs\FTNT{Es. 47:10 ; Lu. 12:15-21.}.
\VS{10}Mais moi, je serai dans la maison de Dieu comme un olivier verdoyant. Je me confie dans la bonté de Dieu pour toujours et à jamais.
\VS{11}Je te célébrerai à jamais, car tu agis ; et je mettrai mon espérance en ton Nom, parce qu'il est bon envers tes fidèles.
\TextTitle{[Egarement des impies]}
\Chap{53}
\VerseOne{}Cantique de David, donné au chef des chantres, pour le chanter sur la flûte.
\VS{2}L'insensé dit en son cœur : Il n'y a point de Dieu ! Ils se sont corrompus, ils ont commis des injustices abominables ; il n'y a personne qui fasse le bien\FTNT{Ps. 10:4 ; Ro. 1:20-21 ; Ro. 3:12.}.
\VS{3}Dieu a regardé des cieux les fils des hommes, pour voir s'il y a quelqu'un qui soit intelligent, qui cherche Dieu.
\VS{4}Ils se sont tous détournés et se sont tous rendus odieux. Il n'y a personne qui fasse le bien, pas même un seul.
\VS{5}Les ouvriers d'iniquité n'ont-ils point de connaissance ? Ils mangent mon peuple comme s'ils mangeaient du pain. Ils n'invoquent point Dieu.
\VS{6}Ils seront épouvantés sans qu’il y ait sujet d’épouvante, car Dieu a dispersé les os de celui qui campe contre toi. Tu les confondras, car Dieu les a rejetés.
\VS{7}Oh ! Qui fera partir de Sion les délivrances d'Israël ? Quand Dieu aura ramené son peuple captif, Jacob s'égayera, Israël se réjouira.
\TextTitle{[La délivrance vient de Yahweh]}
\Chap{54}
\VerseOne{}Cantique de David, donné au chef des chantres, pour le chanter avec instruments à cordes.
\VS{2}Lorsque les Ziphiens vinrent dire à Saül : David n’est-il pas caché parmi nous\FTNT{1 S. 23:19 ; 1 S. 26:1.} ?
\VS{3}Ô Dieu ! Délivre-moi par ton Nom et fais-moi justice par ta puissance.
\VS{4}Ô Dieu ! Ecoute ma prière, prête l'oreille aux paroles de ma bouche !
\VS{5}Car des étrangers se sont élevés contre moi, et des gens terribles qui ne mettent pas Dieu devant eux en veulent à ma vie. Pause.
\VS{6}Voilà, Dieu m'accorde son secours, le Seigneur est de ceux qui soutiennent mon âme.
\VS{7}Il fera retourner le mal sur ceux qui m'épient ; détruis-les selon ta vérité.
\VS{8}Je t’offrirai de bon cœur des sacrifices ; Yahweh ! Je célébrerai ton Nom parce qu'il est bon.
\VS{9}Car il m'a délivré de toute détresse ; et mes yeux se réjouissent à la vue de mes ennemis.
\TextTitle{[Se garder des méchants]}
\Chap{55}
\VerseOne{}Cantique de David, donné au chef des chantres, pour le chanter avec instruments à cordes.
\VS{2}Ô Dieu ! Prête l'oreille à ma prière et ne te cache pas de mes supplications !
\VS{3}Ecoute-moi et réponds-moi ! J’erre çà et là dans ma méditation et je suis agité
\VS{4}à cause du bruit que fait l'ennemi, à cause de l'oppression du méchant ; car ils font tomber sur moi les outrages, et ils me haïssent jusqu’à la fureur.
\VS{5}Mon cœur tremble au-dedans de moi et les terreurs de la mort tombent sur moi.
\VS{6}La crainte et l’épouvante m’atteignent et le frisson m’habille.
\VS{7}Je dis : Qui me donnera des ailes de colombe ? Je m'envolerai et je trouverai ma demeure.
\VS{8}Voilà, je m'enfuirais bien loin et je me tiendrais au désert. Pause.
\VS{9}Je m’échapperais en toute hâte, plus rapide que le vent impétueux, que la tempête.
\VS{10}Seigneur, réduis à néant, divise leur langue, car j'ai vu la violence et les querelles dans la ville.
\VS{11}Elles font jour et nuit le tour sur les murailles ; l’iniquité et la malice sont dans son sein.
\VS{12}Les calamités sont au milieu d'elle, et la tromperie et la fraude ne partent point de ses places.
\VS{13}Car ce n'est pas mon ennemi qui m'a diffamé, je le supporterais ; ce n'est point celui qui m'a en haine qui s'élève contre moi, je me cacherais de lui.
\VS{14}Mais c'est toi, ô homme ! Que j’estimais mon égal, mon confident et mon ami\FTNT{Ps. 41:10.} !
\VS{15}Nous prenions plaisir à communiquer nos secrets ensemble, nous allions avec la multitude dans la maison de Dieu.
\VS{16}Que la mort les séduise ! Qu'ils descendent vivants dans le scheol ! Car le mal est dans leur demeure, parmi eux dans leur assemblée.
\VS{17}Mais moi je crie à Dieu, et Yahweh me délivrera.
\VS{18}Le soir, le matin, et à midi je me plains et je gémis, et il entendra ma voix.
\VS{19}Il me délivrera mon âme de la guerre et me rendra la paix ; car ils sont nombreux contre moi.
\VS{20}Dieu entendra et témoignera en ma faveur. Lui qui de toute éternité est assis sur son trône. Pause. Car il n'y a point de changement en eux, et ils ne craignent point Dieu.
\VS{21}Chacun d'eux porte la main sur ceux qui vivaient en paix avec lui, et viole son alliance.
\VS{22}Les paroles de sa bouche sont plus douces que la crème, mais la guerre est dans son cœur ; ses paroles sont plus douces que l'huile, néanmoins elles sont tout autant d'épées nues.
\VS{23}Remets ton sort à Yahweh et il te soulagera, il ne permettra jamais que le juste tombe.
\VS{24}Mais toi, ô Dieu ! Tu les précipiteras au puits de la perdition ; les hommes sanguinaires et trompeurs ne parviendront point à la moitié de leurs jours. C’est en toi que je me confie.
\TextTitle{[Se glorifier en la Parole de Yahweh]}
\Chap{56}
\VerseOne{}Hymne de David, donné au chef des chantres, pour le chanter sur «~Colombe des térébinthes lointains~». Lorsque les Philistins le saisirent à Gath\FTNT{1 S. 21:10-14.}.
\VS{2}Dieu ! Aie pitié de moi, car des hommes m’écrasent et m'oppriment, me faisant tout le jour la guerre, ils m’oppressent.
\VS{3}Mes adversaires me piétinent tout le jour ; car, ô Très-Haut, plusieurs me font la guerre comme des hautains.
\VS{4}Le jour où j'aurai peur, je me confierai en toi.
\VS{5}Je me glorifierai en Dieu, en sa parole ; je me confie en Dieu, je ne craindrai rien. Que peuvent me faire les hommes\FTNT{Ps. 118:6 ; Hé. 13:6.} ?
\VS{6}Tout le jour ils tordent mes propos, et toutes leurs pensées tendent à me nuire.
\VS{7}Ils s'assemblent, ils se tiennent cachés, ils observent mes pas, s’attendant à m’ôter la vie.
\VS{8}C’est par l’iniquité qu’ils espèrent échapper. Dans ta colère, ô Dieu, précipite les peuples !
\VS{9}Tu comptes mes allées et venues ; recueille mes larmes dans tes outres : Ne sont-elles pas écrites dans ton livre ?
\VS{10}Le jour où je crierai à toi, mes ennemis reculeront ; je sais que Dieu est pour moi.
\VS{11}Je me glorifierai en Dieu, en sa parole, je me glorifierai en Yahweh, en sa parole.
\VS{12}Je me confie en Dieu, je ne craindrai rien : Que me fera l'homme ?
\VS{13}Ô Dieu ! Les vœux que je t’ai fait s’accompliront, je te louerai.
\VS{14}Car tu as délivré mon âme de la mort, tu as garanti mes pieds de la chute, afin que je marche devant Dieu, à la lumière des vivants.
\TextTitle{[Avoir confiance en Dieu dans les difficultés]}
\Chap{57}
\VerseOne{}Hymne de David, donné au chef des chantres, pour le chanter sur Al-Thasheth\FTNT{Al-Thasheth signifie «~Ne détruis pas~»}. Lorsqu’il se réfugia dans la caverne, poursuivi par Saül\FTNT{1 S. 22:1.}.
\VS{2}Aie pitié de moi, ô Dieu, aie pitié de moi ! Car mon âme cherche un refuge ; je cherche un refuge à l'ombre de tes ailes, jusqu'à ce que les calamités soient passées\FTNT{Ps. 17:8.}.
\VS{3}Je crie au Dieu Très-Haut, au Dieu qui accomplit son œuvre pour moi.
\VS{4}Il m’enverra des cieux la délivrance, il rendra honteux celui qui veut me dévorer. Pause. Dieu enverra sa bonté et sa vérité.
\VS{5}Mon âme est parmi des lions ; je suis couché au milieu de gens qui vomissent la flamme, parmi des hommes dont les dents sont des lances et des flèches, et dont la langue est une épée aiguë\FTNT{Ps. 59:8 ; Ps. 64:4 ; Ja. 3:5-12.}.
\VS{6}Ô Dieu, élève-toi sur les cieux ! Que ta gloire soit sur toute la terre !
\VS{7}Ils avaient tendu un filet sous mes pas : Mon âme se courbait. Ils avaient creusé une fosse devant moi, mais ils y sont tombés. Pause.
\VS{8}Mon cœur est affermi, ô Dieu ! Mon cœur est affermi, je chanterai et je ferai retentir mes instruments.
\VS{9}Réveille-toi ma gloire ! Réveillez-vous mon luth et ma harpe ! Je me réveillerai à l'aube du jour.
\VS{10}Seigneur, je te célébrerai parmi les peuples, je te chanterai parmi les nations.
\VS{11}Car ta bonté est grande jusqu'aux cieux, et ta vérité jusqu'aux nues\FTNT{Ps. 118:4-5.}.
\VS{12}Ô Dieu ! Elève-toi sur les cieux ! Que ta gloire soit sur toute la terre !
\TextTitle{[Yahweh rend justice sur la terre]}
\Chap{58}
\VerseOne{}Hymne de David, donné au chef des chantres, pour le chanter sur Al-Thasheth «~Ne détruis pas~».
\VS{2}En vérité, vous, gens de l'assemblée, prononcez-vous ce qui est juste ? Vous, fils des hommes, jugez-vous avec droiture ?
\VS{3}Au contraire, vous tramez des injustices dans votre cœur. Sur la terre, c’est la violence de vos mains que vous placez sur la balance.
\VS{4}Les méchants se sont égarés dès le sein maternel, ils ont erré dès le ventre de leur mère, en parlant faussement.
\VS{5}Ils ont un venin semblable au venin du serpent, ils sont comme l'aspic sourd, qui ferme son oreille,
\VS{6}qui n'entend pas la voix des enchanteurs, du magicien le plus sage.
\VS{7}Ô Dieu, brise-leur les dents dans leur bouche ! Yahweh, brise les mâchoires des lionceaux !
\VS{8}Qu'ils s'écoulent comme de l'eau, et qu'ils se fondent ! Que chacun d'eux bande son arc, mais que ses flèches soient comme si elles étaient rompues !
\VS{9}Qu'ils s'en aillent comme un limaçon qui se fond ! Qu’ils ne voient point le soleil comme l'avorton d'une femme !
\VS{10}Avant que vos chaudières aient senti le feu des épines, l'ardeur de la colère, semblable à un tourbillon, enlèvera chacun d'eux comme de la chair crue..
\VS{11}Le juste se réjouira quand il aura vu la vengeance, il lavera ses pieds avec le sang du méchant.
\VS{12}Et chacun dira : Quoi qu'il en soit, il y a une récompense pour le juste ; quoi qu'il en soit, il y a un Dieu qui juge sur la terre.
\TextTitle{[Intervention divine]}
\Chap{59}
\VerseOne{}Hymne de David, donné au chef des chantres, pour le chanter sur Al-Thasheth «~Ne détruis pas~». Lorsque Saül envoya des gens qui épièrent sa maison afin de le tuer\FTNT{1 S. 19:11.}.
\VS{2}Mon Dieu ! Délivre-moi de mes ennemis, protège-moi de ceux qui s'élèvent contre moi !
\VS{3}Délivre-moi des ouvriers d'iniquité et garde-moi des hommes sanguinaires !
\VS{4}Car voici, ils m'ont dressé des embûches, et des gens robustes se sont assemblés contre moi, bien qu'il n'y ait point en moi de transgression ni de péché, ô Yahweh !
\VS{5}Ils courent çà et là, et se mettent en ordre, bien qu'il n'y ait point d'iniquité en moi. Réveille-toi pour venir au-devant de moi ! Et regarde !
\VS{6}Toi donc, ô Yahweh ! Dieu des armées, Dieu d'Israël, réveille-toi pour visiter toutes les nations ! Ne fais point de grâce à aucun de ceux qui me trahissent ! Pause.
\VS{7}Ils reviennent chaque soir, ils hurlent comme des chiens, ils font le tour de la ville.
\VS{8}Voici, de leur bouche ils font jaillir le mal, il y a des épées sur leurs lèvres\FTNT{Ja. 3:5-12.} ; car, disent-ils, qui nous entend ?
\VS{9}Mais toi, Yahweh ! Tu te riras d'eux, tu te moqueras de toutes les nations\FTNT{Ps. 2:4.}.
\VS{10}Quelle que soit leur force, je m'attends à toi, car Dieu est ma haute retraite.
\VS{11}Dieu qui me favorise me préviendra, Dieu me fera voir mes adversaires\FTNT{Ps. 118:7.}.
\VS{12}Ne les tue pas, de peur que mon peuple ne l'oublie ; fais-les errer par ta puissance et abats-les ; Seigneur, notre bouclier !
\VS{13}Leur bouche pèche à chaque parole de leurs lèvres ; qu'ils soient pris par leur orgueil ! Ils ne tiennent que des discours de malédiction et de mensonge.
\VS{14}Consume-les avec fureur, consume-les de sorte qu'ils ne soient plus ! Qu'on sache que Dieu domine sur Jacob et jusqu'aux extrémités de la terre ! Pause.
\VS{15}Qu’ils reviennent le soir, et qu’ils hurlent comme un chien, et qu’ils fassent le tour de la ville.
\VS{16}Qu'ils errent çà et là cherchant leur nourriture, et qu'ils passent la nuit sans être rassasiés.
\VS{17}Mais moi je chanterai ta force, je louerai dès le matin à haute voix ta bonté\FTNT{Ps. 88:14.}. Car tu es pour moi une haute retraite, et mon asile au jour de ma détresse.
\VS{18}Ma force ! Je te chanterai ; car Dieu est ma haute retraite, le Dieu qui me favorise.
\TextTitle{[Yahweh, le meilleur secours]}
\Chap{60}
\VerseOne{}Hymne de David, pour enseigner, donné au chef des chantres, pour le chanter sur le lis lyrique,
\VS{2}lorsqu’il fit la guerre contre les Syriens de Mésopotamie, et contre les syriens de Tsoba, et que Joab revint et défit douze mille Edomites dans la vallée du sel\FTNT{2 S. 8:3-13 ; 1 Ch. 18:3-12.}.
\VS{3}Ô Dieu ! Tu nous as rejetés, tu nous as dispersés, tu t'es irrité : Reviens vers nous !
\VS{4}Tu as ébranlé la terre et l'as mise en pièces ; répare ses brèches, car elle chancelle !
\VS{5}Tu as fait voir à ton peuple des choses dures, tu nous as abreuvés d’un vin d'étourdissement\FTNT{Es. 51:17-21 ; Ap. 14:10.}.
\VS{6}Mais tu as donné une bannière à ceux qui te craignent, afin de l'élever bien haut pour l'amour de ta vérité. Pause.
\VS{7}Afin que ceux que tu aimes soient délivrés ; sauve-moi par ta droite et exauce-moi\FTNT{Ps. 108:7.}.
\VS{8}Dieu a parlé dans son lieu saint : Je me réjouirai, je partagerai Sichem, je mesurerai la vallée de Succoth ;
\VS{9}Galaad est à moi, Manassé aussi est à moi, et Ephraïm est la protection de ma tête et Juda mon sceptre.
\VS{10}Moab est le bassin où je me lave ; je jette mon soulier sur Edom ; pays des Philistins, pousse des cris de guerre à mon sujet\FTNT{2 S. 2:8 ; 1 Ch. 18:2.}.
\VS{11}Qui me conduira dans la ville forte ? Qui me conduira jusqu’en Edom ?
\VS{12}Ne sera-ce pas toi, ô Dieu, qui nous avais rejetés, et qui ne sortais plus, ô Dieu, avec nos armées ?
\VS{13}Donne-nous du secours pour sortir de la détresse ! Car la délivrance qu'on attend de l'homme est vanité\FTNT{Ps. 118:8 ; Jé. 17:5.}.
\VS{14}Avec le secours de Dieu, nous ferons des exploits, et il foulera nos ennemis.
\TextTitle{[Dieu, le Refuge]}
\Chap{61}
\VerseOne{}Psaume de David, donné au chef des chantres, pour le chanter sur instruments à cordes.
\VS{2}Ô Dieu, je crie à toi, sois attentif à ma prière !
\VS{3}Je crie à toi du bout de la terre, le cœur abattu ; conduis-moi sur le rocher qui est trop haut pour moi !
\VS{4}Car tu es mon refuge, une tour forte au-devant de l'ennemi.
\VS{5}Je séjournerai éternellement dans ta tente, je me retirerai à l'ombre de tes ailes. Pause.
\VS{6}Car, ô Dieu ! Tu exauces mes vœux, et tu me donnes l'héritage de ceux qui craignent ton Nom.
\VS{7}Tu ajoutes des jours aux jours du roi ; que ses années se prolongent à jamais !
\VS{8}Qu’il demeure toujours dans la présence de Dieu ! Que la bonté et la vérité le gardent !
\VS{9}Ainsi je chanterai ton Nom à perpétuité, en rendant mes vœux chaque jour.
\TextTitle{[S'attendre à Dieu]}
\Chap{62}
\VerseOne{}Psaume de David, donné au chef des chantres, d’après Jeduthun.
\VS{2}Quoiqu'il en soit, mon âme se repose en Dieu ; c'est de lui que vient ma délivrance.
\VS{3}Quoiqu'il en soit, il est mon rocher, ma délivrance, et ma haute retraite ; je ne serai pas entièrement ébranlé.
\VS{4}Jusqu’à quand accablerez-vous de maux un homme ? Vous serez tous mis à mort, et vous serez comme le mur qui penche, comme une cloison qui a été ébranlée.
\VS{5}Ils ne font que consulter pour le faire déchoir de son élévation ; ils prennent plaisir au mensonge ; ils bénissent de leur bouche, mais au-dedans ils maudissent. Pause.
\VS{6}Mais toi mon âme, demeure tranquille, regarde à Dieu, car mon espérance est en lui.
\VS{7}Quoiqu'il en soit, il est mon rocher, ma délivrance, et ma haute retraite ; je ne serai point ébranlé.
\VS{8}En Dieu est ma délivrance et ma gloire ; en Dieu est le rocher de ma force et ma retraite.
\VS{9}Peuples, confiez-vous en lui en tout temps, déchargez votre cœur sur lui ! Dieu est notre retraite. Pause.
\VS{10}Oui, vanité, les fils de l’homme ! Mensonge, les fils de l’homme ! Dans une balance, ils monteraient tous ensemble, plus légers qu’un souffle.
\VS{11}Ne vous confiez pas dans la violence ni dans la rapine ; ne devenez point vains ; quand les richesses abonderont, n'y mettez point votre cœur.
\VS{12}Dieu a parlé une fois, j'ai entendu cela deux fois : C’est que la force est à Dieu.
\VS{13}Et c'est à toi, Seigneur, qu'appartient la bonté ; certainement tu rendras à chacun selon son œuvre\FTNT{Job. 34:11 ; Pr. 24:12 ; Jé. 32:19 ; Mt. 16:27 ; Ro. 2:6.}.
\TextTitle{[Soif de la présence de Dieu]}
\Chap{63}
\VerseOne{}Psaume de David, lorsqu'il était dans le désert de Juda\FTNT{1 S. 22:5 ; 1 S. 23:14-15.}.
\VS{2}Ô Dieu ! Tu es mon Dieu, je te cherche au point du jour ; mon âme a soif de toi, mon corps soupire après toi sur cette terre aride, desséchée, et sans eau\FTNT{Ps. 42:2 ; Ps. 84:3 ; Ps. 143:6.}.
\VS{3}Ainsi je te contemple dans ton lieu saint pour voir ta force et ta gloire.
\VS{4}Car ta bonté vaut mieux que la vie, mes lèvres te louent.
\VS{5}Et ainsi je te bénirai donc toute ma vie\FTNT{Ps. 104:33.}, j'élèverai mes mains en ton Nom.
\VS{6}Mon âme est rassasiée comme de mets gras et succulents, et ma bouche te loue avec un chant de réjouissance.
\VS{7}Quand je me souviens de toi dans mon lit, je médite sur toi durant les veilles de la nuit\FTNT{Ps. 16:7 ; Ps. 119:55.}.
\VS{8}Car tu m'as secouru, je me réjouirai à l'ombre de tes ailes.
\VS{9}Mon âme s'est attachée à toi pour te suivre, ta droite me soutient.
\VS{10}Mais ceux-ci qui demandent que mon âme tombe en ruine entreront au plus bas de la terre.
\VS{11}On les détruira à coups d'épée, ils seront la proie des chacals.
\VS{12}Mais le roi se réjouira en Dieu ; quiconque jure par lui s'en glorifiera, car la bouche de ceux qui mentent sera fermée\FTNT{Ps. 107:42 ; Job. 5:16.}.
\TextTitle{[Chercher en Yahweh son seul refuge]}
\Chap{64}
\VerseOne{}Psaume de David, donné au chef des chantres.
\VS{2}Ô Dieu ! Ecoute ma voix quand je m'écrie. Protège ma vie contre l'ennemi que je crains !
\VS{3}Cache-moi des complots des méchants, de l'assemblée tumultueuse des ouvriers d'iniquité !
\VS{4}Ils aiguisent leur langue comme une épée\FTNT{Ps. 59:8 ; Jé. 9:3 ; Ps. 11:2.}, ils tirent comme des flèches leurs paroles amères,
\VS{5}afin de tirer sur l’innocent dans sa cachette ; ils tirent soudainement sur lui et n'ont aucune crainte.
\VS{6}Ils se fortifient dans leur méchanceté, tiennent des discours pour tendre des pièges, ils disent : Qui les verra\FTNT{Job. 24:15.} ?
\VS{7}Ils cherchent curieusement des méchancetés ; ils ont sondé tout ce qui se peut sonder, même ce qui peut être au–dedans de l’homme, et au cœur le plus profond.
\VS{8}Mais Dieu lance contre eux ses traits, soudain les voilà frappés.
\VS{9}Leur langue a causé leur chute ; tous ceux qui les voient secouent leur tête.
\VS{10}Et tous les hommes craindront et raconteront l'œuvre de Dieu, et considéreront ce qu'il aura fait.
\VS{11}Le juste se réjouira en Yahweh, et se retirera vers lui, et tous ceux qui sont droits de cœur s'en glorifieront\FTNT{Ps. 58:10 ; Ps. 63:12 ; Ps. 97:12.}.
\TextTitle{[Yahweh règne sur toute la nature]}
\Chap{65}
\VerseOne{}Psaume de David. Cantique. Donné au chef des chantres.
\VS{2}Ô Dieu ! Dans le calme, on te louera dans Sion, et l’on accomplira nos vœux\FTNT{Ps. 50:14 ; Ps. 66:13.}.
\VS{3}Tu entends nos prières, toute chair viendra jusqu'à toi.
\VS{4}Les iniquités prévalent sur moi, mais tu feras l'expiation de nos transgressions.
\VS{5}Heureux celui que tu choisis et que tu admets dans ta présence pour qu'il habite dans tes parvis ! Nous serons rassasiés des biens de ta maison, des biens du saint lieu de ton temple.
\VS{6}Dans ta justice, tu nous réponds par des choses terribles, ô Dieu de notre salut, espoir de toutes les extrémités lointaines de la terre et de la mer.
\VS{7}Il affermit les montagnes par sa force, il est ceint de puissance.
\VS{8}Il apaise le mugissement de la mer, le mugissement de leurs flots, et le tumulte des peuples.
\VS{9}Ceux qui habitent aux extrémités de la terre ont peur de tes prodiges ; tu réjouis l'orient et l'occident.
\VS{10}Tu visites la terre, tu lui donnes l’abondance, tu la combles de richesses ; le ruisseau de Dieu est plein d'eau ; tu prépares le blé, quand tu l'établis ainsi.
\VS{11}Tu arroses ses sillons, et tu aplanis ses mottes ; tu l'amollis par la pluie, et tu bénis son germe\FTNT{Ps. 104:13-14 ; Es. 55:10.}.
\VS{12}Tu couronnes l'année de tes biens, et tes voies versent l’abondance.
\VS{13}Les plaines du désert sont abreuvées et les collines sont ceintes de joie.
\VS{14}Les pâturages se couvrent de brebis, et les vallées se revêtent de froments ; les cris de joie et les chants retentissent.
\TextTitle{[Louange au Dieu de grâces]}
\Chap{66}
\VerseOne{}Cantique. Psaume, donné au chef des chantres. Vous tous habitants de toute la terre, poussez des cris de triomphe à Dieu.
\VS{2}Chantez la gloire de son Nom, faites éclater sa gloire par vos louanges.
\VS{3}Dites à Dieu : Que tes œuvres sont redoutables ! Tes ennemis te mentiront à cause de la grandeur de ta force.
\VS{4}Toute la terre se prosterne devant toi et te chante ; elle chante ton Nom. Pause.
\VS{5}Venez et voyez les œuvres de Dieu : Il est redoutable quand il agit sur les fils des hommes.
\VS{6}Il a fait de la mer une terre sèche ; on a passé le fleuve à pied sec ; là, nous nous sommes réjouis en lui\FTNT{Ex. 14:21 ; Jos. 3:14-17.}.
\VS{7}Il domine par sa puissance éternellement ; ses yeux prennent garde sur les nations\FTNT{Ps. 14:2 ; Ps. 33:13 ; Job. 28:24.} ; les rebelles ne pourront point s’élever. Pause.
\VS{8}Peuples, bénissez notre Dieu, et faites retentir le son de sa louange.
\VS{9}C'est lui qui a remis notre âme en vie, et qui n'a point permis que nos pieds chancellent.
\VS{10}Car, ô Dieu, tu nous as éprouvés ! Tu nous a fait passer au creuset comme l'argent.
\VS{11}Tu nous as amenés dans le filet, tu as mis sur nos reins un pesant fardeau.
\VS{12}Tu as fait monter des hommes sur notre tête, et nous avons passé par le feu et par l'eau. Mais tu nous as fait entrer dans un lieu d’abondance.
\VS{13}J'entrerai dans ta maison avec des holocaustes, j’accomplirai mes vœux envers toi\FTNT{Ps. 22:26 ; Ps. 76:12 ; Ps. 116:14.}.
\VS{14}Pour eux, mes lèvres se sont ouvertes et ma bouche les a prononcés dans ma détresse.
\VS{15}Je t'offrirai en holocauste des brebis grasses, avec la graisse des béliers, je te sacrifierai des taureaux et des boucs. Pause.
\VS{16}Vous tous qui craignez Dieu, venez, écoutez, et je raconterai ce qu'il a fait à mon âme.
\VS{17}Je l'ai invoqué de ma bouche, et la louange a été sur ma langue.
\VS{18}Si j’avais conçu l’iniquité dans mon cœur, le Seigneur ne m’aurait pas écouté\FTNT{Jn. 9:31.}.
\VS{19}Mais certainement Dieu m'a écouté, il a été attentif à la voix de ma prière.
\VS{20}Béni soit Dieu qui n'a point rejeté ma prière, et qui n'a point éloigné de moi sa bonté.
\TextTitle{[Les nations louent Dieu]}
\Chap{67}
\VerseOne{}Psaume. Cantique donné au chef des chantres, pour le chanter avec instruments à cordes.
\VS{2}Que Dieu ait pitié de nous et qu’il nous bénisse, qu'il fasse luire sa face sur nous\FTNT{Ps. 4:7 ; Ps. 31:17 ; Ps. 119:135 ; No. 6:25.}. Pause.
\VS{3}Afin que ta voie soit connue sur la terre et ta délivrance parmi toutes les nations.
\VS{4}Les peuples te célébreront, ô Dieu ! Tous les peuples te célébreront\FTNT{Ps. 22:27 ; Ps. 68:33.} !
\VS{5}Les peuples se réjouissent et chantent de joie, car tu juges les peuples avec droiture et tu conduis les nations sur la terre\FTNT{Ps. 96:10.}. Pause.
\VS{6}Les peuples te célébreront, ô Dieu ! Tous les peuples te célébreront !
\VS{7}La terre produira son fruit ; Dieu, notre Dieu, nous bénira.
\VS{8}Dieu nous bénira, et toutes les extrémités de la terre le craindront.
\TextTitle{[Yahweh, le Dieu glorieux]}
\Chap{68}
\VerseOne{}Psaume.Cantique de David, donné au chef des chantres.
\VS{2}Que Dieu se lève, et ses ennemis seront dispersés, et ceux qui le haïssent s'enfuiront devant lui\FTNT{No. 10:35.}.
\VS{3}Tu les chasseras comme la fumée est chassée par le vent ; comme la cire se fond devant le feu, ainsi les méchants périront devant Dieu\FTNT{Ps. 37:20 ; Ps. 97:5.}.
\VS{4}Mais les justes se réjouiront et s'égayeront devant Dieu, et tressailliront de joie\FTNT{Ps. 67:4-5.}.
\VS{5}Chantez à Dieu, célébrez son Nom ! Exaltez celui qui est monté sur les cieux ! Son Nom est Yahweh ! Réjouissez-vous dans sa présence.
\VS{6}Il est le père des orphelins et le juge des veuves ; Dieu est dans sa demeure sainte\FTNT{Ps. 146:9}.
\VS{7}Dieu donne une famille à ceux qui étaient abandonnés, il délivre ceux qui étaient enchaînés, mais les rebelles habitent sur une terre déserte.
\VS{8}Ô Dieu ! Quand tu sortis devant ton peuple, quand tu marchais dans le désert ! Pause.
\VS{9}La terre trembla et les cieux répandirent leurs eaux à cause de la présence de Dieu, le mont Sinaï trembla à cause de la présence de Dieu, du Dieu d'Israël\FTNT{Ex. 19:18 ; Jg. 5:5.}.
\VS{10}Ô Dieu ! Tu as fait tomber une pluie abondante sur ton héritage, et quand il était épuisé, tu l'as rétabli.
\VS{11}Ton troupeau établit sa demeure dans le pays, que par ta bonté tu avais préparé pour les malheureux, ô Dieu !
\VS{12}Le Seigneur donne une parole, et les messagères de bonnes nouvelles sont une grande armée.
\VS{13}Les rois des armées se sont enfuis, ils se sont enfuis, et celle qui se tenait à la maison a partagé le butin\FTNT{1 S. 30:16.}.
\VS{14}Tandis que vous vous couchez dans les étables, les ailes de la colombe sont couvertes d'argent, et son plumage est d’un jaune d’or.
\VS{15}Quand le Tout-Puissant dispersa les rois dans le pays, il devint blanc comme la neige du Tsalmon.
\VS{16}La montagne de Dieu est un mont de Basan ; une montagne élevée, un mont de Basan.
\VS{17}Pourquoi l’insultez-vous, montagnes dont le sommet est élevé ? Dieu a désiré cette montagne pour y habiter, et Yahweh y demeurera à jamais.
\VS{18}Les chars de Dieu se comptent par vingt-mille, par milliers et par milliers ; le Seigneur est au milieu d'eux ; le Sinaï est dans le sanctuaire.
\VS{19}Tu es monté dans les hauteurs, tu as emmené des captifs, tu as pris des dons pour les distribuer parmi les hommes, et même parmi les rebelles, afin qu'ils habitent dans le lieu de Yahweh Dieu\FTNT{Ep. 4:8. Cette prophétie concerne la résurrection du Seigneur Jésus-Christ.}.
\VS{20}Béni soit le Seigneur, qui tous les jours nous comble de ses biens ; Dieu est notre délivrance. Pause.
\VS{21}Dieu est pour nous le Dieu de délivrance, et les issues de la mort sont à Yahweh le Seigneur.
\VS{22}Certainement, Dieu écrasera la tête de ses ennemis\FTNT{Ge. 3:15.}, le sommet de la tête de ceux qui vivent dans le péché.
\VS{23}Le Seigneur dit : Je les ramènerai de Basan\FTNT{No. 21:33.}, je les ramènerai du fond de la mer.
\VS{24}Afin que tu plonges ton pied dans le sang\FTNT{Ps. 58:11.}, et que la langue de tes chiens ait sa part de tes ennemis.
\VS{25}Ils voient ta marche, ô Dieu ! Ils ont vu ta marche dans le lieu saint, la marche de mon Dieu, mon Roi.
\VS{26}Les chantres allaient devant, ensuite les joueurs d'instruments, et au milieu les jeunes filles jouant du tambour\FTNT{Ex. 15:20 ; 1 S. 18:6.}.
\VS{27}Bénissez Dieu dans les assemblées, bénissez le Seigneur, vous qui êtes descendants d'Israël.
\VS{28}Là sont Benjamin, le plus jeune qui domine sur eux, les chefs de Juda et leur corps d’armée, les chefs de Zabulon, et les chefs de Nephthali.
\VS{29}Ton Dieu ordonne que tu sois puissant. Affermis, ô Dieu, ce que tu as fait pour nous.
\VS{30}Dans ton temple, à Jérusalem, les rois t'amèneront des présents\FTNT{Ps. 72:10 ; 1 R. 10:10 ; 2 Ch. 32:23.}.
\VS{31}Epouvante les bêtes sauvages des roseaux, la troupe des taureaux, et les veaux des peuples, et ceux qui se prosternent avec des pièces d'argent. Disperse les peuples qui prennent plaisir à la guerre.
\VS{32}De grands seigneurs viendront d'Egypte ; l’Ethiopie se hâtera d'étendre ses mains vers Dieu.
\VS{33}Royaumes de la terre, chantez à Dieu, célébrez le Seigneur ! Pause.
\VS{34}Chantez celui qui est monté dans les cieux des cieux, les cieux éternels ; voilà, il fait retentir de sa voix un son puissant.
\VS{35}Attribuez la force à Dieu ; sa majesté est sur Israël, et sa force est dans les nuées.
\VS{36}Dieu ! Tu es redouté à cause de ton lieu saint. Le Dieu d'Israël est celui qui donne la force et la puissance à son peuple. Béni soit Dieu !
\TextTitle{[Dieu attentif à la prière de ceux qui s'humilient]}
\Chap{69}
\VerseOne{}Psaume de David, donné au chef des chantres, pour le chanter sur les lis.
\VS{2}Délivre-moi, ô Dieu, car les eaux menacent ma vie\FTNT{Ps. 124:4 ; Ps. 144:7.}.
\VS{3}Je suis enfoncé dans un bourbier profond, sans appui ; je suis entré au plus profond des eaux, et les courants d’eau me submergent.
\VS{4}Je suis las de crier, mon gosier se dessèche, mes yeux se consument pendant que je m’attends à Dieu.
\VS{5}Ceux qui me haïssent sans cause\FTNT{Jn. 15:25.} dépassent en nombre les cheveux de ma tête ; ceux qui tâchent de me ruiner et qui sont mes ennemis à tort se sont renforcés ; je dois rendre ce que je n'avais point ravi.
\VS{6}Ô Dieu ! Tu connais ma folie et mes fautes ne te sont point cachées.
\VS{7}Ô Seigneur Yahweh des armées ! Que ceux qui se confient en toi ne soient point honteux à cause de moi ; et que ceux qui te cherchent ne soient point humiliés à cause de moi, ô Dieu d'Israël !
\VS{8}Car pour l'amour de toi j'ai souffert l'opprobre, la honte a couvert mon visage.
\VS{9}Je suis devenu un étranger pour mes frères, et un homme de dehors pour les fils de ma mère\FTNT{Jn. 7:3-5 ; Ge. 31:15.}.
\VS{10}Car le zèle de ta maison me dévore\FTNT{Jn. 2:17 ; Ro. 15:3.}, et les outrages de ceux qui t’insultaient sont tombés sur moi.
\VS{11}Je pleure et je jeûne : C’est ce qui m’attire l’opprobre.
\VS{12}Je prends un sac pour vêtement, et je suis l’objet de leurs discours moqueurs.
\VS{13}Ceux qui sont assis à la porte parlent de moi, et les buveurs de boissons fortes me mettent en chanson\FTNT{Job. 30:9 ; La. 3:14.}.
\VS{14}Mais je t’adresse ma prière, ô Yahweh\FTNT{Ps. 102:2.} ! Que ce soit le temps favorable, ô Dieu ! Par ta grande bonté. Réponds-moi en m’assurant ta délivrance.
\VS{15}Délivre-moi de la boue, que je ne m’y enfonce point\FTNT{Ps. 40:3.}, et que je sois délivré de ceux qui me haïssent, et des eaux profondes.
\VS{16}Que les courants d’eau ne me submergent plus, que l’abîme ne m'engloutisse point, et que le puits ne ferme point sa bouche sur moi.
\VS{17}Yahweh ! Exauce-moi, car ta bonté est agréable ; dans tes grandes compassions, tourne ta face vers moi ;
\VS{18}et ne cache point ta face à ton serviteur, car je suis en détresse. Hâte-toi, exauce-moi !
\VS{19}Approche-toi de mon âme, rachète-la ; délivre-moi à cause de mes ennemis.
\VS{20}Tu connais toi-même mon opprobre, et ma honte, et mon ignominie ; tous mes ennemis sont devant toi.
\VS{21}L'opprobre m'a brisé le cœur, et je suis languissant ; j'ai attendu que quelqu'un ait compassion de moi, mais il n'y en a point eu. J'ai attendu des consolateurs, mais je n'en ai point trouvé.
\VS{22}Ils m'ont au contraire donné du fiel\FTNT{Mt. 27:34 ; Mt. 27:48.} pour mon repas ; et dans ma soif, ils m'ont abreuvé de vinaigre.
\VS{23}Que leur table soit pour eux un piège et un appât au sein de leur perfection.
\VS{24}Que leurs yeux soient tellement obscurcis, qu'ils ne puissent point voir ; et fais continuellement chanceler leurs reins.
\VS{25}Répands ton indignation sur eux, et que l'ardeur de ta colère les saisisse.
\VS{26}Que leur campement soit désolé, et qu'il n'y ait personne qui habite dans leurs tentes.
\VS{27}Car ils persécutent celui que tu avais frappé, et racontent les souffrances de ceux que tu blesses.
\VS{28}Mets des iniquités à leurs iniquités ; et qu'ils n'entrent point dans ta justice.
\VS{29}Qu'ils soient effacés du livre de vie, et qu'ils ne soient point inscrits avec les justes.
\VS{30}Mais pour moi, qui suis affligé et dans la douleur, ta délivrance, ô Dieu, m’élèvera en une haute retraite.
\VS{31}Je louerai le Nom de Dieu par des cantiques et je le glorifierai par des louanges.
\VS{32}Cela est agréable à Yahweh plus qu'un taureau avec des cornes et des sabots fendus.
\VS{33}Les malheureux le voient et ils se réjouissent ; que votre cœur vive, vous qui cherchez Dieu.
\VS{34}Car Yahweh exauce les misérables et ne méprise point ses prisonniers.
\VS{35}Que les cieux et la terre le louent ; que la mer et tout ce qui s’y meut le louent aussi\FTNT{Ps. 96:11.}.
\VS{36}Car Dieu délivrera Sion et bâtira les villes de Juda ; on y habitera et on la possèdera.
\VS{37}Et la postérité de ses serviteurs en fera son héritage, et ceux qui aiment son Nom y auront leur demeure.
\TextTitle{[Le pauvre et l'indigent]}
\Chap{70}
\VerseOne{}Psaume de David, pour souvenir, donné au chef des chantres.
\VS{2}Dieu ! Hâte-toi de me délivrer, ô Dieu ! Hâte-toi de venir à mon secours\FTNT{Ps. 40:14 ; Ps. 71:12.}.
\VS{3}Que ceux qui cherchent mon âme soient honteux et rougissent\FTNT{Ps. 71:13 ; Ps. 35:4.} ; et que ceux qui prennent plaisir à mon mal soient repoussés en arrière et soient confus.
\VS{4}Que ceux qui disent : Aha ! Aha ! Retournent en arrière par l’effet de leur honte.
\VS{5}Que tous ceux qui te cherchent exultent et se réjouissent en toi ; et que ceux qui aiment ta délivrance disent toujours : Glorifié soit Dieu !
\VS{6}Moi, je suis affligé et misérable, ô Dieu ! Hâte-toi de venir vers moi ; tu es mon secours et mon libérateur, ô Yahweh ! Ne tarde point.
\TextTitle{[Demeurer en Dieu jusqu'au bout]}
\Chap{71}
\VerseOne{}Yahweh ! Je cherche en toi mon refuge : Que je ne sois jamais confus !
\VS{2}Délivre-moi par ta justice et sauve-moi. Incline ton oreille vers moi, mets-moi en sûreté.
\VS{3}Sois pour moi le rocher de mon refuge, afin que je puisse toujours m’y retirer ; tu as donné l’ordre de me mettre en sûreté, car tu es mon rocher et ma forteresse.
\VS{4}Mon Dieu ! Délivre-moi de la main du méchant, de la main du pervers et de l'oppresseur.
\VS{5}Car tu es mon espérance, Seigneur Yahweh ! Ma confiance dès ma jeunesse.
\VS{6}Je m’appuie sur toi dès le ventre de ma mère ; c'est toi qui m'as tiré hors des entrailles de ma mère\FTNT{Ps. 22:10-11.} ; tu es le sujet continuel de mes louanges.
\VS{7}Je suis pour plusieurs comme un miracle, mais tu es mon puissant refuge.
\VS{8}Que ma bouche soit remplie de ta louange et de ta gloire chaque jour.
\VS{9}Ne me rejette point au temps de ma vieillesse ; ne m'abandonne point maintenant que ma force est consumée.
\VS{10}Car mes ennemis ont parlé de moi, et ceux qui épient mon âme ont pris conseil ensemble,
\VS{11}disant : Dieu l'a abandonné. Poursuivez-le et saisissez-le, car il n'y a personne qui le délivre.
\VS{12}Dieu, ne t'éloigne point de moi ! Mon Dieu hâte-toi de venir à mon secours !
\VS{13}Que ceux qui sont les ennemis de mon âme soient honteux et défaits ; et que ceux qui cherchent mon malheur soient enveloppés d'opprobre et de honte.
\VS{14}Mais moi, j’espèrerai toujours et je te louerai tous les jours davantage.
\VS{15}Ma bouche racontera chaque jour ta justice et ta délivrance, bien que je n'en sache point le nombre.
\VS{16}Je marcherai par la force du Seigneur Yahweh ; je raconterai ta seule justice.
\VS{17}Ô Dieu ! Tu m'as enseigné dès ma jeunesse et j'ai annoncé jusqu’à présent tes merveilles.
\VS{18}Ô Dieu ! Ne m'abandonne pas, même dans la blanche vieillesse. Afin que j’annonce ta force à cette génération présente, ta puissance à la génération à venir.
\VS{19}Car ta justice, ô Dieu, est haut élevée, car tu as fait de grandes choses. Ô Dieu, qui est semblable à toi ?
\VS{20}Tu m'a fait éprouver bien des détresses et des malheurs, mais tu me redonneras la vie et tu me feras remonter hors des abîmes de la terre.
\VS{21}Relève ma grandeur et console-moi encore.
\VS{22}Je te louerai au son du luth, je te chanterai ta fidélité, mon Dieu, je te célèbrerai avec la harpe, saint d’Israël !
\VS{23}Mes lèvres et mon âme, que tu auras rachetée, pousseront des cris de joie quand je te chanterai.
\VS{24}Ma langue aussi publiera chaque jour ta justice, car ceux qui cherchent mon malheur seront honteux et rougiront.
\TextTitle{[Le royaume messianique]}
\Chap{72}
\VerseOne{}De Salomon. Ô Dieu, donne tes jugements au roi et ta justice au fils du roi.
\VS{2}Qu'il juge avec justice ton peuple, et tes malheureux avec équité.
\VS{3}Que les montagnes portent la paix pour le peuple, et que les collines la portent en justice.
\VS{4}Qu'il fasse droit aux malheureux du peuple, qu'il délivre les fils du misérable, et qu'il écrase l'oppresseur !
\VS{5}Ils te craindront tant que le soleil et la lune dureront d’âge en âge.
\VS{6}Il descendra comme la pluie sur l’herbe fauchée, comme les ondées qui arrosent la terre.
\VS{7}En son temps, le juste fleurira, et il y aura abondance de paix jusqu'à ce qu'il n'y ait plus de lune.
\VS{8}Il dominera depuis une mer jusqu'à l'autre, et depuis le fleuve jusqu'aux extrémités de la terre.
\VS{9}Les habitants des déserts se courberont devant lui, et ses ennemis lécheront la poussière.
\VS{10}Les rois de Tarsis et des îles lui rapporteront des dons ; les rois de Saba et de Séba lui apporteront des présents.
\VS{11}Tous les rois aussi se prosterneront devant lui, toutes les nations le serviront.
\VS{12}Car il délivrera le pauvre qui crie vers lui, l'affligé et celui qui n’a personne qui l’aide\FTNT{Ps. 34:18 ; Job. 29:12.}.
\VS{13}Il aura compassion du pauvre et du misérable, et il sauvera les âmes des misérables.
\VS{14}Il garantira leur âme de la fraude et de la violence, et leur sang sera précieux devant ses yeux.
\VS{15}Il vivra donc, et on lui donnera de l'or de Séba, et on fera des prières pour lui continuellement ; on le bénira chaque jour.
\VS{16}Les blés abonderont dans le pays, au sommet des montagnes, et leurs épis s’agiteront comme les arbres du Liban ; les hommes fleuriront dans les villes comme l'herbe de la terre.
\VS{17}Sa renommée durera à toujours ; sa renommée ira de père en fils tant que le soleil durera ; et on se bénira en lui ; toutes les nations le diront heureux.
\VS{18}Béni soit Yahweh Dieu, le Dieu d'Israël, qui seul fait des choses merveilleuses !
\VS{19}Béni soit éternellement son Nom glorieux, et que toute la terre soit remplie de sa gloire. Amen ! Oui, amen !
\VS{20}Fin des prières de David, fils d'Isaï.
\TextTitle{[L'orgueil des méchants]}
\Chap{73}
\VerseOne{}Psaume d'Asaph. Quoi qu’il en soit, Dieu est bon pour Israël, pour ceux qui ont le cœur pur\FTNT{Mt. 5:8.}.
\VS{2}Toutefois, mes pieds allaient fléchir, mes pas étaient sur le point de glisser.
\VS{3}Car j'ai porté envie aux insensés en voyant la prospérité des méchants.
\VS{4}Rien ne les tourmente jusqu’à leur mort, et leur corps est gras.
\VS{5}Ils n’ont point de part aux peines des humains, et ils ne sont point frappés avec les autres hommes.
\VS{6}C'est pourquoi l'orgueil les environne comme un collier, et un vêtement de violence les couvre.
\VS{7}Les yeux leur sortent dehors à force de graisse ; ils surpassent les desseins de leur cœur.
\VS{8}Ils sont pernicieux, et parlent méchamment d'opprimer ; ils parlent d’une manière hautaine.
\VS{9}Ils élèvent leur bouche jusqu’aux cieux et leur langue parcourt la terre.
\VS{10}C'est pourquoi son peuple se tourne de leur côté, il avale l'eau abondamment.
\VS{11}Ils disent : Comment Dieu saurait-il ? Comment le Très-Haut connaîtrait-il\FTNT{Ps. 94:7 ; Job. 22:12-13 ; Es. 29:15 ; Ez. 8:12.} ?
\VS{12}Voilà, ceux-ci sont méchants, ils prospèrent toujours dans ce monde et acquièrent de plus en plus de richesses.
\VS{13}Quoi qu’il en soit, c'est donc en vain que j'ai purifié mon cœur et que j'ai lavé mes mains dans l'innocence\FTNT{Job. 35:3 ; Mal. 3:14.}.
\VS{14}Je suis frappé tous les jours, et tous les matins mon châtiment est là.
\VS{15}Si je disais : Je veux parler comme eux, voici je trahirais la génération de tes fils.
\VS{16}Toutefois, j'ai tâché de connaître cela, mais cela m'a paru fort difficile,
\VS{17}jusqu’à ce que je sois entré dans le sanctuaire de Dieu et que j'aie considéré la fin de telles gens.
\VS{18}Quoi qu'il en soit, tu les as mis sur des voies glissantes, tu les fais tomber dans des précipices.
\VS{19}Comment ont-ils été ainsi détruits en un instant ? Ont-ils défailli ? Ont-ils été consumés d'épouvante ?
\VS{20}Ils sont comme un songe lorsqu'on s'est réveillé. Seigneur, tu méprises leur image à ton réveil.
\VS{21}Quand mon cœur s'aigrissait et que je me sentais percé dans les entrailles,
\VS{22}j'étais alors stupide, et je n'avais aucune connaissance ; j'étais comme une bête dans ta présence.
\VS{23}Je serai donc toujours avec toi ; tu m'as pris par la main droite.
\VS{24}Tu me conduiras par ton conseil, et tu me recevras dans la gloire.
\VS{25}Quel autre ai-je au ciel ? Or sur la terre je ne prends plaisir qu’en toi seul.
\VS{26}Ma chair et mon cœur étaient consumés, mais Dieu est le rocher de mon cœur, et mon partage pour toujours.
\VS{27}Car voilà, ceux qui s'éloignent de toi périront ; tu retrancheras tous ceux qui se détournent de toi.
\VS{28}Mais pour moi, m’approcher de Dieu c’est mon bien ; j'ai mis toute mon espérance dans le Seigneur Yahweh, afin de raconter toutes tes œuvres.
\TextTitle{[Appel au secours du peuple de Dieu]}
\Chap{74}
\VerseOne{}Cantique d'Asaph. Ô Dieu, pourquoi nous as-tu rejetés pour toujours ? Et pourquoi ta colère fume-t-elle contre le troupeau de ton pâturage\FTNT{Ps. 79:5.} ?
\VS{2}Souviens-toi de ton assemblée que tu as acquise autrefois. Tu t'es approprié cette montagne de Sion, sur laquelle tu habitais, afin qu'elle soit la portion de ton héritage.
\VS{3}Elève tes pas vers les lieux constamment dévastés ; l'ennemi a tout renversé dans le lieu saint.
\VS{4}Tes adversaires ont rugi au milieu de ton assemblée ; ils ont mis leurs signes pour signes.
\VS{5}On les a vus pareils à celui qui lève la cognée dans une épaisse forêt.
\VS{6}Et maintenant, avec des haches et des marteaux, ils brisent les sculptures.
\VS{7}Ils ont mis le feu à ton lieu saint. Ils ont abattu à terre et profané la demeure dédiée à ton Nom\FTNT{2 R. 25:9.}.
\VS{8}Ils ont dit en leur cœur : Saccageons-les tous ensemble ! Ils ont brûlé dans le pays tous les lieux saints de Dieu.
\VS{9}Nous ne voyons plus nos signes ; il n'y a plus de prophètes ; et personne parmi nous qui sache jusqu’à quand\FTNT{La. 2:9-10.}.
\VS{10}Ô Dieu ! Jusqu’à quand l'adversaire te couvrira-t-il d'opprobres et l’ennemi méprisera-t-il ton Nom à jamais ?
\VS{11}Pourquoi retires-tu ta main, même ta droite ? Consume-les en la tirant du milieu de ton sein !
\VS{12}Or Dieu est mon Roi dès les temps anciens, faisant des délivrances au milieu de la terre.
\VS{13}Tu as fendu la mer par ta force ; tu as brisé les têtes des serpents sur les eaux.
\VS{14}Tu as brisé les têtes du léviathan, tu l'as donné pour nourriture au peuple du désert.
\VS{15}Tu as ouvert la fontaine et le torrent, tu as desséché les grosses rivières.
\VS{16}A toi est le jour, à toi aussi est la nuit ; tu as établi la lumière et le soleil.
\VS{17}Tu as posé toutes les limites de la terre ; tu as formé l'été et l'hiver.
\VS{18}Souviens-toi de ceci : Que l'ennemi a blasphémé Yahweh et qu'un peuple insensé a outragé ton Nom.
\VS{19}Ne livre pas aux vivants l’âme de la tourterelle, n'oublie pas à toujours la vie de tes affligés.
\VS{20}Regarde à ton alliance, car les lieux ténébreux de la terre sont remplis d’habitations de violence.
\VS{21}Ne permets pas que celui qui est foulé s'en retourne tout confus. Que l'affligé et le pauvre louent ton Nom !
\VS{22}Ô Dieu ! Lève-toi, défends ta cause, souviens-toi de l'opprobre qui t'est fait tous les jours par l'insensé !
\VS{23}N'oublie pas le cri de tes adversaires, le bruit de ceux qui s'élèvent contre toi monte continuellement !
\TextTitle{[L'élevation vient de Yahweh]}
\Chap{75}
\VerseOne{}Psaume d'Asaph. Cantique donné au chef des chantres, pour le chanter sur Al-Thasheth\FTNT{Voir Ps. 57:1.}.
\VS{2}Nous te célébrons, ô Dieu ! Nous te célébrons et ton Nom est près de nous ; nous racontons tes merveilles.
\VS{3}Au temps que j’aurai fixé, je jugerai avec droiture.
\VS{4}La terre se dissout avec tous ceux qui y habitent, mais j'affermis ses piliers. Pause.
\VS{5}Je dis aux insensés : N'agissez point follement ; et aux méchants : N’élevez pas la tête.
\VS{6}N’élevez pas si haut votre tête, et ne parlez point avec fierté.
\VS{7}Car l'élévation ne vient point d'orient, ni d'occident ni du désert.
\VS{8}Car c'est Dieu qui gouverne ; il abaisse l'un, et élève l'autre\FTNT{1 S. 2:7.}.
\VS{9}Il y a une coupe dans la main de Yahweh\FTNT{Es. 51:17-22 ; Jé. 25:27-28 ; Ap. 14:10 ; Ap. 16:19.}, et le vin rougit dedans ; il est plein de mélange, et Dieu en verse ; certainement, tous les méchants de la terre en suceront et en boiront jusqu’à la lie.
\VS{10}Mais moi, je raconterai ces choses à jamais, je chanterai au Dieu de Jacob.
\VS{11}J'humilierai tous les méchants, mais les justes seront élevés.
\TextTitle{[La Puissance du Dieu redoutable]}
\Chap{76}
\VerseOne{}Psaume d'Asaph. Cantique donné au chef des chantres, pour le chanter avec instruments à cordes.
\VS{2}Dieu est connu en Judée, sa renommée est grande en Israël ;
\VS{3}sa tente est à Salem et sa demeure à Sion.
\VS{4}Là il a brisé les arcs étincelants, le bouclier, l'épée et les armes de guerre. Pause.
\VS{5}Tu es resplendissant, plus magnifique que les montagnes des ravisseurs.
\VS{6}Les plus courageux sont étourdis, ils sont dans un profond assoupissement, et aucun de ces hommes vaillants n'a trouvé ses mains.
\VS{7}Ô Dieu de Jacob, les cavaliers et les chevaux se sont endormis quand tu les as menacés.
\VS{8}Tu es redoutable, toi. Qui peut se tenir devant toi quand ta colère éclate ?
\VS{9}Tu fais entendre des cieux le jugement ; la terre en a eu peur et s'est tenue dans le silence.
\VS{10}Quand tu te lèves, ô Dieu, pour faire jugement, pour délivrer tous les malheureux de la terre ! Pause.
\VS{11}L’homme te célèbre, même dans sa fureur, quand tu te ceints de toute ta colère.
\VS{12}Faites vos vœux à Yahweh votre Dieu, et accomplissez-les ! Que tous ceux qui l’environnent apportent des dons au Dieu terrible !
\VS{13}Il coupe le souffle des princes ; il est redoutable aux rois de la terre.
\TextTitle{[Ne pas oublier les hauts faits de Yahweh]}
\Chap{77}
\VerseOne{}Psaume d'Asaph, donné au chef des chantres, d'après Jeduthun.
\VS{2}Ma voix s’élève à Dieu, et je crie ; ma voix s'adresse à Dieu, et il m'écoutera.
\VS{3}Je cherche le Seigneur au jour de ma détresse ; sans cesse mes mains s’étendent durant la nuit ; mon âme refuse d'être consolée.
\VS{4}Je me souviens de Dieu, et je gémis ; je médite, et mon esprit est affaibli. Pause.
\VS{5}Tu empêches mes yeux de dormir ; je suis troublé, et ne peux parler.
\VS{6}Je pense aux jours d'autrefois et aux années des siècles passées\FTNT{Ps. 143:5.}.
\VS{7}Je me souviens de mes chants pendant la nuit, je médite en mon cœur, et mon esprit cherche diligemment.
\VS{8}Le Seigneur m'a-t-il rejeté pour toujours ? Ne me sera-t-il plus favorable ?
\VS{9}Sa bonté est-elle disparue pour toujours ? Sa parole a-t-elle pris fin pour l’éternité ?
\VS{10}Dieu a-t-il oublié d'avoir compassion ? A-t-il dans sa colère retiré sa miséricorde ? Pause.
\VS{11}Je dis : Ce qui me fait devenir malade, je me souviendrai des années de la droite du Très–haut.
\VS{12}Je me souviens des exploits de Yahweh ; je me suis, dis-je, souvenu de tes merveilles d'autrefois.
\VS{13}Je méditerai toutes tes œuvres, et je parlerai de tes œuvres.
\VS{14}Ô Dieu ! Tes voies sont saintes. Quel dieu est grand comme Dieu ?
\VS{15}Tu es le Dieu qui fait des merveilles ! Tu as fait connaître ta force parmi les peuples.
\VS{16}Tu as délivré par ton bras ton peuple, les fils de Jacob et de Joseph. Pause.
\VS{17}Les eaux t'ont vu, ô Dieu ! Les eaux t'ont vu et ont tremblé, même les abîmes en ont été émus.
\VS{18}Les nuées ont versé un déluge d'eau, les nuées ont fait retentir leur son ; tes flèches ont volé de toutes parts.
\VS{19}La voix de ton tonnerre était dans le tourbillon, les éclairs ont éclairé le monde, la terre en a été émue et en a tremblé.
\VS{20}Tu te frayas un chemin par la mer, un sentier par les grosses eaux ; et tes traces ne furent plus reconnues.
\VS{21}Tu as mené ton peuple comme un corps d’armée sous la conduite de Moïse et d'Aaron\FTNT{Mi. 6:4.}.
\TextTitle{[Dieu à l'oeuvre pour Israël]}
\Chap{78}
\VerseOne{}Cantique d'Asaph. Mon peuple, écoute ma loi, prêtez vos oreilles aux paroles de ma bouche.
\VS{2}J’ouvrirai ma bouche en une parabole ; je proférerai les énigmes cachées des temps anciens\FTNT{Mt. 13:35.}.
\VS{3}Ce que nous avons entendu et connu, et que nos pères nous ont raconté\FTNT{Ps. 44:2.},
\VS{4}nous ne le cacherons point à leurs fils. Ils raconteront à la génération à venir les louanges de Yahweh, sa puissance et ses merveilles qu'il a faites.
\VS{5}Car il a établi le témoignage en Jacob, et il a mis la loi en Israël ; il a donné cet ordre à nos pères de la faire connaître à leurs fils\FTNT{De. 4:9.},
\VS{6}pour qu’elle soit connue de la génération future, des fils qui naîtraient, et pour que lorsqu’ils seront grands, ils la relatent à leurs fils,
\VS{7}afin qu'ils mettent leur confiance en Dieu, et qu'ils n'oublient point les œuvres de Dieu, et qu'ils gardent ses commandements.
\VS{8}Afin qu'ils ne soient point comme leurs pères, une génération revêche et rebelle, une génération insoumise de cœur, dont l’esprit est infidèle à Dieu\FTNT{Ex. 32:9 ; Ac. 7:51.}.
\VS{9}Les fils d'Ephraïm, armés et tirant de l’arc, tournèrent le dos le jour de la bataille.
\VS{10}Ils ne gardèrent point l'alliance de Dieu et refusèrent de marcher selon sa loi.
\VS{11}Ils oublièrent ses œuvres et ses merveilles qu'il leur avait fait voir.
\VS{12}Il avait fait des miracles en présence de leurs pères, dans le pays d'Egypte, dans le champ de Tsoan.
\VS{13}Il fendit la mer et les fit passer au travers ; et il fit arrêter les eaux comme un monceau de pierres.
\VS{14}Il les conduisit de jour par la nuée, et toute la nuit par une lumière de feu\FTNT{Ex. 13:21.}.
\VS{15}Il fendit les rochers au désert, et leur donna à boire d’abondantes eaux, comme s'il eût puisé des abîmes.
\VS{16}Il fit sortir des ruisseaux de la roche\FTNT{Ex. 17:6 ; No. 20:11 ; 1 Co. 10:5.} et fit couler des eaux comme des rivières.
\VS{17}Toutefois, ils continuèrent à pécher contre lui, irritant le Très-Haut dans le désert.
\VS{18}Ils tentèrent Dieu dans leurs cœurs, en demandant de la viande selon leur désir.
\VS{19}Ils parlèrent contre Dieu, disant : Dieu pourrait-il dresser une table dans ce désert\FTNT{No. 11:4.} ?
\VS{20}Voilà, dirent-ils, il a frappé le rocher, et les eaux ont coulé et des torrents ont débordé ; mais pourrait-il aussi nous donner du pain ? Fournirait-il de la viande à son peuple ?
\VS{21}C'est pourquoi, Yahweh les ayant entendus, se mit dans une grande colère, et le feu s'embrasa contre Jacob, et sa colère s'excita contre Israël.
\VS{22}Parce qu'ils n'avaient point cru en Dieu et ne s'étaient point confiés en sa délivrance.
\VS{23}Il ordonna aux nuées d'en haut et il ouvrit les portes des cieux ;
\VS{24}il fit pleuvoir la manne sur eux pour leur nourriture et il leur donna le blé du ciel\FTNT{Jn. 6:31 ; Ex. 16:14.}.
\VS{25}Ils mangèrent tous le pain des grands. Il leur envoya de la viande pour s'en rassasier.
\VS{26}Il excita dans les cieux le vent d'orient et il amena par sa puissance le vent du sud.
\VS{27}Il fit pleuvoir sur eux de la viande comme de la poussière, et comme le sable des mers des oiseaux ailées.
\VS{28}Il les fit tomber au milieu du camp, autour de leurs demeures.
\VS{29}Ils en mangèrent et en furent pleinement rassasiés, car il leur donna selon leur désir.
\VS{30}Mais ils ne furent pas encore dégoûtés de leur désir, et leur viande était encore dans leur bouche
\VS{31}quand la colère de Dieu s'excita contre eux, et qu'il mit à mort les plus gras d'entre eux, et abattit les gens d'élite d'Israël\FTNT{1 Co. 10:5.}.
\VS{32}Malgré cela, ils péchèrent encore et ne crurent point à ses prodiges\FTNT{No. 14:2.}.
\VS{33}C'est pourquoi il consuma leurs jours par la vanité et leurs années par une fin soudaine.
\VS{34}Quand il les mettait à mort, alors ils le recherchaient ; ils se repentaient et ils cherchaient Dieu dès le matin.
\VS{35}Ils se souvenaient que Dieu était leur rocher, et Dieu, le Très-Haut, était leur libérateur.
\VS{36}Mais ils le trompaient de leur bouche et ils lui mentaient de leur langue\FTNT{Es. 29:13 ; Mt. 15:8 ; Jé. 12:2.} ;
\VS{37}car leur cœur n'était point droit envers lui, et ils ne furent point fidèles à son alliance.
\VS{38}Toutefois, dans sa compassion, il pardonne l’iniquité, et ne détruit point ; mais il apaise souvent sa colère et ne se livre pas à toute sa fureur.
\VS{39}Il se souvint qu'ils n’étaient que chair, qu'un vent qui passe et qui ne revient point.
\VS{40}Combien de fois l’ont–ils irrité au désert, et combien de fois l’ont–ils attristé dans ce lieu inhabitable ?
\VS{41}Ils ne cessèrent de tenter Dieu et de provoquer le Saint d'Israël.
\VS{42}Ils ne se souvinrent point de sa puissance, du jour où il les délivra de la main de l’ennemi,
\VS{43}des miracles qu’il accomplit en Egypte, et de ses merveilles dans les champs de Tsoan.
\VS{44}Il changea en sang leurs fleuves et leurs ruisseaux , et ils ne purent en boire les eaux\FTNT{Ex. 7:20.}.
\VS{45}Il envoya contre eux des mouches qui les dévorèrent et des grenouilles qui les détruisirent\FTNT{Ex. 8:6-24.}.
\VS{46}Il livra leurs récoltes aux sauterelles, le produit de leur travail aux sauterelles\FTNT{Ex. 10:13.}.
\VS{47}Il détruisit leurs vignes par la grêle, et leurs sycomores par les orages\FTNT{Ex. 9:23.}.
\VS{48}Il livra leur bétail à la grêle, et leurs troupeaux aux foudres étincelantes.
\VS{49}Il envoya sur eux l'ardeur de sa colère, la fureur, la rage et la détresse, un corps d’armée de messagers de malheur.
\VS{50}Il donna libre cours à sa colère, et ne retira point leur âme de la mort ; il livra leur vie à la peste\FTNT{Ex. 9:6.}.
\VS{51}Il frappa tout premier-né en Egypte, les prémices de la vigueur dans les tentes de Cham\FTNT{Ex. 12:29.}.
\VS{52}Il fit partir son peuple comme des brebis, il les mena comme un corps d’armée dans le désert.
\VS{53}Il les conduisit sûrement, et sans qu’ils eussent aucune frayeur, là où la mer couvrit leurs ennemis.
\VS{54}Il les amena vers sa frontière sainte, vers cette montagne que sa droite a acquise\FTNT{Ex. 15:17.}.
\VS{55}Il chassa devant eux les nations, leur distribua le pays en héritage, et fit habiter les tribus d'Israël dans les tentes de ces nations.
\VS{56}Mais ils tentèrent et irritèrent le Dieu Très-Haut, et ne gardèrent point ses préceptes.
\VS{57}Et ils se retirèrent en arrière et furent infidèles comme leurs pères ; ils tournèrent comme un arc trompeur.
\VS{58}Ils le provoquèrent à la colère par leurs hauts lieux, et l’émurent à la jalousie par leurs images taillées\FTNT{De. 32:16-21.}.
\VS{59}Dieu l'entendit et se mit dans une grande colère, et il méprisa fortement Israël.
\VS{60}Il abandonna la demeure de Silo, la tente où il habitait parmi les hommes.
\VS{61}Il livra en captivité sa force et son ornement entre les mains de l'ennemi.
\VS{62}Il livra son peuple à l'épée et se mit dans une grande colère contre son héritage.
\VS{63}Le feu consuma leurs gens d'élite, et leurs vierges ne furent point louées.
\VS{64}Leurs sacrificateurs tombèrent par l'épée, et leurs veuves ne les pleurèrent point.
\VS{65}Puis le Seigneur se réveilla comme un homme qui se serait endormi, et comme un puissant homme qui s'écrie ayant encore le vin dans la tête.
\VS{66}Il frappa ses adversaires par derrière et les mit en opprobre perpétuel.
\VS{67}Mais il dédaigna la tente de Joseph, et ne choisit point la tribu d'Ephraïm.
\VS{68}Mais il choisit la tribu de Juda, la montagne de Sion, celle qu’il aime.
\VS{69}Il bâtit son lieu saint dans les lieux élevés, et l'établit comme la terre qu'il a fondée pour toujours.
\VS{70}Il choisit David, son serviteur, et le prit de la bergerie\FTNT{1 S. 16:11 ; 2 S. 7:8.} ;
\VS{71}il le prit derrière les brebis qui allaitent et l'amena pour paître Jacob, son peuple, et Israël, son héritage.
\VS{72}Aussi il les dirigea selon l'intégrité de son cœur, et les conduisit avec des mains intelligentes.
\TextTitle{[Appel au jugement de Dieu]}
\Chap{79}
\VerseOne{}Psaume d'Asaph. Ô Dieu ! Les nations sont entrées dans ton héritage ; on a profané ton saint temple, on a mis Jérusalem en monceaux de pierres.
\VS{2}On a livré les cadavres de tes serviteurs pour viande aux oiseaux du ciel, et la chair de tes fidèles aux bêtes de la terre.
\VS{3}On a répandu leur sang comme de l'eau autour de Jérusalem, et il n'y a eu personne pour les enterrer.
\VS{4}Nous sommes un sujet d’opprobre à nos voisins, de moquerie et de risée à ceux qui habitent autour de nous\FTNT{Ps. 44:14 ; Ps. 80:7.}.
\VS{5}Jusqu’à quand, ô Yahweh, t’irriteras-tu sans cesse et ta jalousie s'embrasera-t-elle comme un feu\FTNT{Ps. 89:47.} ?
\VS{6}Répands ta fureur sur les nations qui ne te connaissent point et sur les royaumes qui n'invoquent point ton Nom\FTNT{Jé. 10:25.}.
\VS{7}Car on a dévoré Jacob et on a ravagé ses demeures.
\VS{8}Ne rappelle point devant nous les iniquités passées. Que tes compassions viennent en hâte au-devant de nous, car nous sommes dans une extrême détresse.
\VS{9}Ô Dieu de notre délivrance ! Aide-nous pour pour l’amour de la gloire de ton Nom, et délivre-nous ! Pardonne-nous nos péchés pour l’amour de ton Nom !
\VS{10}Pourquoi les nations diraient-elles : Où est leur Dieu ? Que la vengeance du sang de tes serviteurs, qui a été répandu, soit manifestée parmi les nations en notre présence.
\VS{11}Que le gémissement des captifs parviennent jusqu’à toi. Par ton bras puissant sauve tes fils, ceux qui vont périr !
\VS{12}Et rends à nos voisins, dans leur sein, sept fois au double l'opprobre qu’ils t’ont fait, ô Yahweh !
\VS{13}Mais nous, ton peuple, et le troupeau de ton pâturage, nous te louerons pour toujours, et de génération en génération nous publierons tes louanges.
\TextTitle{[Retrouver la faveur de Dieu]}
\Chap{80}
\VerseOne{}Psaume d'Asaph, donné au chef des chantres, pour le chanter sur les lis lyriques.
\VS{2}Toi qui pais Israël, prête l'oreille ! Toi qui mènes Joseph comme un troupeau, toi qui es assis entre les chérubins\FTNT{Ps. 99:1 ; Es. 37:16 ; 2 S. 6:2.}, fais briller ta splendeur !
\VS{3}Réveille ta puissance au-devant d'Ephraïm, de Benjamin et de Manassé ; et viens pour notre délivrance !
\VS{4}Dieu, ramène-nous et fais briller ta face ! Et nous serons délivrés !
\VS{5}Ô Yahweh, Dieu des armées, jusqu’à quand seras-tu irrité contre la prière de ton peuple ?
\VS{6}Tu les nourris de pain de larmes et tu les abreuves de larmes à pleine mesure.
\VS{7}Tu fais de nous un sujet de dispute entre nos voisins, et nos ennemis se moquent de nous.
\VS{8}Ô Dieu des armées, ramène-nous et fais briller ta face ! Et nous serons délivrés.
\VS{9}Tu avais retiré une vigne hors d'Egypte, tu as chassé les nations, et tu l’as plantée\FTNT{Es. 5:1-7 ; Os. 10:1 ; Mt. 20:1 ; Mt. 21:28-33.}.
\VS{10}Tu as préparé une place devant elle, tu lui as fait prendre racine, et elle a rempli la terre.
\VS{11}Les montagnes étaient couvertes de son ombre, et ses rameaux étaient comme de hauts cèdres de Dieu.
\VS{12}Elle étendait ses branches jusqu'à la mer, et ses rejetons jusqu'au fleuve.
\VS{13}Pourquoi as-tu rompu ses clôtures, de sorte que tous les passants sur la route cueillent ses raisins ?
\VS{14}Les sangliers de la forêt l'ont détruite, et toutes les bêtes des champs en font leur pâture.
\VS{15}Ô Dieu des armées, reviens ! Regarde des cieux, vois, et visite cette vigne ;
\VS{16}et le plant que ta droite avait planté, et le fils que tu t’es choisi.
\VS{17}Elle est brûlée par le feu, elle est coupée ; ils périssent devant ta face menaçante.
\VS{18}Que ta main soit sur l'homme de ta droite, sur le fils de l'homme que tu t'es choisi.
\VS{19}Et nous ne nous éloignerons plus de toi. Rends-nous la vie, et nous invoquerons ton Nom.
\VS{20}Ô Yahweh ! Dieu des armées, ramène-nous, fais briller ta face, et nous serons délivrés !
\TextTitle{[Se débarasser des dieux étrangers]}
\Chap{81}
\VerseOne{}Psaume d'Asaph, donné au chef des chantres, pour le chanter sur la Guitthith.
\VS{2}Chantez avec allégresse à notre Dieu, notre force ! Poussez des cris de joie en l'honneur du Dieu de Jacob.
\VS{3}Sonnez du shofar, prenez le tambour, la harpe mélodieuse et le luth.
\VS{4}Sonnez du shofar à la nouvelle lune, à la pleine lune, au jour de notre fête\FTNT{No. 10:10.}.
\VS{5}Car c'est une loi pour Israël, une ordonnance du Dieu de Jacob.
\VS{6}Il établit un statut à Joseph, lorsqu'il marcha contre le pays d'Egypte, où j'entendis un langage que je ne connaissais pas.
\VS{7}J'ai retiré son épaule du fardeau, et ses mains ont lâché les corbeilles.
\VS{8}Tu as crié dans la détresse, et je t’ai sauvé ; je t'ai répondu dans le lieu caché du tonnerre ; je t'ai éprouvé auprès des eaux de Mériba. Pause.
\VS{9}Ecoute mon peuple, je te relèverai. Israël, si tu m'écoutais !
\VS{10}Qu’il n’y ait point de dieu étranger au milieu de toi, et ne te prosterne point devant les dieux des étrangers.
\VS{11}Je suis Yahweh, ton Dieu, qui t'ai fait monter hors du pays d'Egypte. Ouvre ta bouche et je la remplirai.
\VS{12}Mais mon peuple n'a point écouté ma voix, et Israël ne m'a point obéi.
\VS{13}C'est pourquoi je les ai abandonnés aux penchants de leur cœur, et ils ont suivi leurs propres conseils\FTNT{Es. 63:17 ; Es. 65:2 ; 2 Pi. 3:3.}.
\VS{14}Ô si mon peuple m’écoutait ! Si Israël marchait dans mes voies !
\VS{15}J'abattrais en un instant leurs ennemis et je tournerais ma main contre leurs adversaires.
\VS{16}Ceux qui haïssent Yahweh le flatteraient, et le bonheur de mon peuple durerait toujours.
\VS{17}Dieu le nourrirait du meilleur froment ; et je le rassasierais du miel du rocher.
\TextTitle{[Dieu dénonce l'injustice des hommes]}
\Chap{82}
\VerseOne{}Psaume d'Asaph. Dieu se tient dans l'assemblée de Dieu, il juge au milieu des juges.
\VS{2}Jusqu’à quand jugerez-vous injustement et aurez-vous égard à l'apparence de la personne des méchants\FTNT{Ps. 58:2.} ? Pause.
\VS{3}Faites droit à celui qu'on opprime et à l'orphelin ; faites justice à l'affligé et au pauvre ;
\VS{4}délivrez celui qu'on maltraite et le misérable, retirez-le de la main des méchants.
\VS{5}Ils ne connaissent ni n'entendent rien ; ils marchent dans les ténèbres, tous les fondements de la terre sont ébranlés.
\VS{6}J'ai dit : Vous êtes des dieux\FTNT{Jn. 10:34.}, et vous êtes tous fils du Très-Haut.
\VS{7}Toutefois, vous mourrez comme des hommes, et vous les princes vous tomberez comme les autres.
\VS{8}Ô Dieu ! Lève-toi, juge la terre ; car tu auras en héritage toutes les nations\FTNT{Hé. 1:2 ; Ps. 2:8.}.
\TextTitle{[Dessein et confusion des ennemis d'Israël]}
\Chap{83}
\VerseOne{}Cantique et psaume d'Asaph.
\VS{2}Ô Dieu ! Ne garde point le silence, ne te tais point, et ne te tiens point en repos, ô Dieu\FTNT{Ps. 35:22.} !
\VS{3}Car voici, tes ennemis s’agitent, et ceux qui te haïssent ont levé la tête.
\VS{4}Ils ont consulté finement en secret contre ton peuple, et ils ont tenu conseil contre ceux qui se sont retirés vers toi pour se cacher\FTNT{Ps. 2:2.}.
\VS{5}Ils disent : Venez et détruisons-les, en sorte qu'ils ne soient plus une nation, et qu'on ne fasse plus mention du nom d'Israël\FTNT{Ce verset fait prophétiquement écho aux déclarations de l’ancien président iranien, Mahmoud Ahmadinejad, qui lors d’une conférence en 2005, déclara qu’Israël devait être rayé de la carte.
Mi. 4:11 ; Ap. 11:1-2.}.
\VS{6}Car ils consultent ensemble d'un même esprit ; ils font alliance contre toi.
\VS{7}Les tentes d’Edom et des Ismaélites, des Moabites et des Hagaréniens ;
\VS{8}de Guebal, d’Ammon, d’Amalek, les Philistins avec les habitants de Tyr.
\VS{9}L’Assyrie aussi se joint à eux ; ils ont servi de bras aux fils de Lot. Pause.
\VS{10}Fais-leur comme tu fis à Madian\FTNT{Jg. 7:15.}, comme à Sisera\FTNT{Jg. 4:15.}, et comme à Jabin, auprès du torrent de Kison !
\VS{11}Ils furent détruits à En-Dor et servirent de fumier à la terre.
\VS{12}Que leurs chefs soient traités comme Oreb et comme Zeeb ; et que tous leurs princes soient comme Zébach et Tsalmunna\FTNT{Jg. 7:25.} ;
\VS{13}parce qu'ils ont dit : Prenons possession des habitations agréables de Dieu.
\VS{14}Mon Dieu ! Rends-les semblables au tourbillon et au chaume chassé par le vent,
\VS{15}comme le feu brûle une forêt, et comme la flamme embrase les montagnes.
\VS{16}Poursuis-les ainsi par ta tempête et épouvante-les par ton tourbillon !
\VS{17}Couvre leurs visages d'ignominie afin qu'on cherche ton Nom, ô Yahweh !
\VS{18}Qu'ils soient honteux et épouvantés à jamais, qu'ils rougissent, et qu'ils périssent ;
\VS{19}afin qu'on sache que toi seul, dont le nom est Yahweh, tu es le Très-Haut sur toute la terre.
\TextTitle{[Délices de ceux qui ont Yahweh pour appui]}
\Chap{84}
\VerseOne{}Psaume des fils de Koré, donné au chef des chantres, pour le chanter sur la Guitthith.
\VS{2}Yahweh des armées, que tes demeures sont aimables !
\VS{3}Mon âme soupire et languit après les parvis de Yahweh ; mon cœur et ma chair poussent des cris de joie vers le Dieu vivant.
\VS{4}Le passereau même trouve sa maison, et l'hirondelle son nid où elle a mis ses petits… Tes autels, ô Yahweh des armées ! Mon Roi et mon Dieu !
\VS{5}Heureux ceux qui habitent ta maison et qui te louent sans cesse ! Pause.
\VS{6}Heureux l'homme dont la force est en toi, ils trouvent dans leur cœur des chemins tout tracés !
\VS{7}Passant par la vallée de Baca, ils la réduisent en fontaines ; la pluie la couvre de bénédictions.
\VS{8}Ils marchent avec force pour se présenter devant Dieu à Sion.
\VS{9}Yahweh Dieu des armées, écoute ma prière, Dieu de Jacob, prête l'oreille. Pause.
\VS{10}Ô Dieu, notre bouclier, vois et regarde la face de ton oint !
\VS{11}Car mieux vaut un jour dans tes parvis, que mille ailleurs. J'aimerais mieux me tenir à la porte dans la maison de mon Dieu, que de demeurer dans les tentes des méchants.
\VS{12}Car Yahweh Dieu est un soleil et un bouclier\FTNT{Ps. 89:19 ; Ps. 144:2 ; Ge. 15:1.} ; Yahweh donne la grâce et la gloire, et il ne refuse aucun bien à ceux qui marchent dans l'intégrité.
\VS{13}Yahweh des armées, heureux l'homme qui se confie en toi\FTNT{Ps. 2:12.} !
\TextTitle{[Les rescapés de l'exil]}
\Chap{85}
\VerseOne{}Psaume des fils de Koré, donné au chef des chantres.
\VS{2}Yahweh, tu as été favorable à ta terre, tu as ramené et mis en repos les prisonniers de Jacob.
\VS{3}Tu as pardonné l'iniquité de ton peuple, tu as couvert tous leurs péchés. Pause.
\VS{4}Tu as retiré toute ta colère, tu es revenu de l'ardeur de ton indignation.
\VS{5}Ô Dieu de notre délivrance, rétablis-nous et fais cesser la colère que tu as contre nous.
\VS{6}Seras-tu irrité à jamais contre nous ? Feras-tu durer ta colère de génération en génération ?
\VS{7}Ne reviendras–tu pas nous rendre la vie\FTNT{Ps. 71:20.}, afin que ton peuple se réjouisse en toi ?
\VS{8}Yahweh, fais-nous voir ta miséricorde et accorde-nous ta délivrance !
\VS{9}J'écouterai ce que dira Dieu, Yahweh ; car il parlera de paix à son peuple et à ses bien-aimés, pourvu que jamais ils ne retournent à leur folie.
\VS{10}Certainement sa délivrance est proche de ceux qui le craignent, la gloire habite dans notre pays.
\VS{11}La bonté et la vérité se rencontrent ; la justice et la paix s’embrassent\FTNT{Hé. 7:2.}.
\VS{12}La vérité germe de la terre et la justice regarde des cieux.
\VS{13}Yahweh aussi donne le bien et notre terre rendra son fruit.
\VS{14}La justice marchera devant lui, et il la mettra partout où il passera.
\TextTitle{[Coeur disposé à la crainte de Dieu]}
\Chap{86}
\VerseOne{}Prière de David. Yahweh, écoute, réponds-moi, car je suis affligé et misérable.
\VS{2}Garde mon âme, car je suis un de tes bien-aimés ; ô toi mon Dieu, délivre ton serviteur qui se confie en toi !
\VS{3}Seigneur, aie pitié de moi, car je crie à toi tout le jour.
\VS{4}Réjouis l'âme de ton serviteur, car j'élève mon âme à toi, Seigneur.
\VS{5}Yahweh ! Tu es bon et clément, et d'une grande bonté envers tous ceux qui t'invoquent\FTNT{Joë. 2:13.}.
\VS{6}Yahweh, prête l'oreille à ma prière, et sois attentif à la voix de mes supplications.
\VS{7}Je t'invoque au jour de ma détresse, car tu m'exauces\FTNT{Ps. 50:15.}.
\VS{8}Seigneur, nul n’est comme toi parmi les dieux, et rien ne ressemble à tes œuvres\FTNT{Ps. 95:3 ; De. 3:24.}.
\VS{9}Seigneur, toutes les nations que tu as faites viendront et se prosterneront devant toi, et glorifieront ton Nom,
\VS{10}car tu es grand, et tu fais des choses merveilleuses ; tu es Dieu, toi seul.
\VS{11}Yahweh ! Enseigne-moi tes voies et je marcherai dans ta vérité\FTNT{Ps. 25:4 ; Ps. 27:11.} ; lie mon cœur à la crainte de ton Nom.
\VS{12}Seigneur, mon Dieu, je te célébrerai de tout mon cœur, et je glorifierai ton Nom à toujours.
\VS{13}Car ta bonté est grande envers moi, et tu as retiré mon âme du profond scheol.
\VS{14}Ô Dieu ! Des gens orgueilleux se sont élevés contre moi, et un corps d’armée de méchants en veut à ma vie ; ils ne portent pas leurs pensées sur toi.
\VS{15}Mais toi, Seigneur, tu es le Dieu compatissant, miséricordieux, lent à la colère, riche en bonté et en vérité.
\VS{16}Tourne-toi vers moi, et aie pitié de moi ! Donne ta force à ton serviteur, délivre le fils de ta servante !
\VS{17}Accorde–moi un signe de ta faveur, et que ceux qui me haïssent le voient et soient honteux, parce que tu m’aideras, ô Yahweh ! Tu me consoleras !
\TextTitle{[Sion, la cité de Dieu]}
\Chap{87}
\VerseOne{}Psaume. Cantique des fils de Koré. Elle est fondée sur les montagnes saintes.
\VS{2}Yahweh aime les portes de Sion, plus que toutes les demeures de Jacob.
\VS{3}Ce qui se dit de toi, cité de Dieu, sont des choses glorieuses. Pause.
\VS{4}Je ferai mention de Rahab et de Babylone parmi ceux qui me connaissent ; voici le pays des philistins, et Tyr avec l’Ethiopie. C’est dans Sion qu’ils sont nés.
\VS{5}Et de Sion il est dit : Un homme y est né ; le Très-Haut lui-même l'établira.
\VS{6}Yahweh compte en inscrivant les peuples : C’est là qu’ils sont nés. Pause.
\VS{7}Et les chantres, de même que les joueurs de flûte, toutes mes sources sont en toi.
\TextTitle{[Lamentation dans l'affliction]}
\Chap{88}
\VerseOne{}Cantique. Psaume des fils de Koré, donné au chef des chantres. Pour chanter sur la flûte. Cantique d’Héman, l’Ezrahite.
\VS{2}Yahweh ! Dieu de ma délivrance, je crie jour et nuit devant toi\FTNT{Lu. 18:7.}.
\VS{3}Que ma prière parvienne en ta présence ; étends ton oreille à mon cri.
\VS{4}Car mon âme est rassasiée de maux, et ma vie atteint le scheol\FTNT{Lu. 16:23.}.
\VS{5}On m'a mis au rang de ceux qui descendent dans la fosse\FTNT{Ps. 28:1 ; Ps. 31:13.} ; je suis devenu comme un homme qui n'a plus de vigueur.
\VS{6}Je suis étendu parmi les morts, semblable à ceux qui sont tués et couchés dans la tombe, à ceux dont tu n’as plus le souvenir, et qui sont séparés par ta main.
\VS{7}Tu m'as jeté dans une fosse profonde, dans les ténèbres, dans les abîmes.
\VS{8}Ta fureur se pose sur moi, et tu m'as accablé de tous tes flots. Pause.
\VS{9}Tu as éloigné de moi ceux de qui j'étais connu, tu m'as mis en abomination devant eux ; je suis enfermé et je ne peux sortir.
\VS{10}Mes yeux se consument dans la souffrance ; Yahweh ! Je crie à toi tout le jour ! J'étends mes mains vers toi\FTNT{Job. 17:7 ; Ex. 9:29 ; 1 R. 8:22.} !
\VS{11}Est-ce pour les morts que tu fais des miracles ? Les morts se relèveront-ils pour te célébrer\FTNT{1 Co. 15:12-13 ; 1 Th. 4:16.} ? Pause.
\VS{12}Parle-t-on de ta bonté dans le sépulcre, de ta fidélité dans le tombeau\FTNT{Ep. 4:9-10 ; 1 Pi. 3:18-20.} ?
\VS{13}Connaîtra-t-on tes merveilles dans les ténèbres et ta justice dans la terre de l'oubli ?
\VS{14}Mais moi, ô Yahweh ! J’implore ton secours, ma prière s’élève dès le matin.
\VS{15}Yahweh ! Pourquoi rejettes-tu mon âme, pourquoi me caches-tu ta face\FTNT{Mt. 27:46 ; Mc. 15:34.} ?
\VS{16}Je suis malheureux et moribond dès ma jeunesse ; j'ai été exposé à tes terreurs, et je ne sais pas où j'en suis.
\VS{17}Les ardeurs de ta colère sont passées sur moi et tes terreurs m’anéantissent\FTNT{Es. 53:5.}.
\VS{18}Elles m’environnent tout le jour comme des eaux, elles m’enveloppent toutes à la fois.
\VS{19}Tu as éloigné de moi mon ami et mon compagnon, mes connaissances ont disparu\FTNT{Mt. 26:56.}.
\TextTitle{[«~Heureux le peuple qui connait le son de la trompette~»]}
\Chap{89}
\VerseOne{}Cantique d'Ethan, l’Ezrachite.
\VS{2}Je chanterai toujours les bontés de Yahweh ; je ferai connaître de ma bouche ta fidélité de génération en génération.
\VS{3}Car je dis : Ta bonté a des fondements éternels, tu établis ta fidélité dans les cieux quand tu dis :
\VS{4}J'ai traité alliance avec mon élu, j'ai fait serment à David mon serviteur :
\VS{5}J’affermirais ta postérité pour toujours, et j'établirai ton trône de génération en génération\FTNT{2 S. 7:8-16.}. Pause.
\VS{6}Les cieux célèbrent tes merveilles, ô Yahweh ! Ta fidélité aussi est célébrée dans l'assemblée des saints.
\VS{7}Car qui dans le ciel peut se comparer à Yahweh ? Qui est semblable à Yahweh parmi les fils de Dieu ?
\VS{8}Dieu se rend extrêmement terrible dans le conseil secret des saints, il est plus redouté que tous ceux qui sont à l’entour de lui.
\VS{9}Ô Yahweh Dieu des armées ! Qui est semblable à toi, puissant Yahweh ? Aussi ta fidélité t’environne.
\VS{10}Tu domines l’élévation des flots de la mer ; quand ses vagues s'élèvent, tu les calmes\FTNT{Job. 26:12 ; Job. 38:8-12.}.
\VS{11}Tu écrasas Rahab\FTNT{Ce terme hébreu fait référence au nom emblèmatique de l'Egypte, il signifie "largeur", "arrogance".} comme un homme blessé à mort ; tu dispersas tes ennemis par le bras de ta force.
\VS{12}A toi sont les cieux, à toi aussi est la terre ; tu as fondé le monde, et tout ce qui est en lui.
\VS{13}Tu as créé le nord et le sud ; le Thabor et l’Hermon se réjouissent en ton Nom.
\VS{14}Ton bras est puissant, ta main est forte, ta droite est haut élevée.
\VS{15}La justice et l'équité sont la base de ton trône ; la bonté et la vérité marchent devant ta face.
\VS{16}Heureux le peuple qui connaît le son de la trompette\FTNT{1 Co. 15:52 ; Ap. 10:7.} ! Il marche, ô Yahweh ! A la clarté de ta face.
\VS{17}Il se réjouit chaque jour en ton Nom, et il se glorifie de ta justice.
\VS{18}Parce que tu es la gloire de leur force ; et notre pouvoir est distingué par ta faveur.
\VS{19}Car notre bouclier est Yahweh, et notre Roi est le Saint d'Israël.
\VS{20}Tu as autrefois parlé en vision touchant ton bien-aimé, et tu as dit : J'ai ordonné mon secours en faveur d'un homme vaillant ; j'ai élevé l'élu du milieu du peuple.
\VS{21}J'ai trouvé David mon serviteur, je l'ai oint de ma sainte huile\FTNT{1 S. 16:13 ; Ac. 13:22.} ;
\VS{22}ma main sera ferme avec lui, et mon bras le renforcera.
\VS{23}L'ennemi ne le surprendra point, et l'inique ne l'affligera point ;
\VS{24}mais j’écraserai devant lui ses adversaires, et je détruirai ceux qui le haïssent.
\VS{25}Ma fidélité et ma bonté seront avec lui, et sa gloire sera élevée en mon Nom.
\VS{26}Je mettrai sa main sur la mer, et sa droite sur les fleuves.
\VS{27}Il m'invoquera : Tu es mon Père, mon Dieu, et le rocher\FTNT{Voir commentaire en Es. 8:14.} de ma délivrance.
\VS{28}Aussi je ferai de lui le premier-né\FTNT{Col. 1:15.}, le plus élevé des rois de la terre.
\VS{29}Je lui garderai ma bonté à toujours, et mon alliance lui sera assurée.
\VS{30}Je rendrai éternelle sa postérité, et son trône comme les jours des cieux.
\VS{31}Mais si ses fils abandonnent ma loi, et ne marchent point selon mes ordonnances ;
\VS{32}s'ils violent mes statuts, et qu'ils ne gardent point mes commandements ;
\VS{33}je punirai de la verge leur transgression, et de plaie leur iniquité.
\VS{34}Mais je ne retirerai point de lui ma bonté, et je ne lui trahirai point ma fidélité.
\VS{35}Je ne violerai point mon alliance, et je ne changerai point ce qui est sorti de mes lèvres.
\VS{36}J'ai une fois juré par ma sainteté : Mentirais-je à David\FTNT{Hé. 6:13.} ?
\VS{37}Sa postérité sera à toujours, et son trône sera devant moi comme le soleil.
\VS{38}Il aura une durée éternelle comme la lune ; le témoin qui est dans le ciel est fidèle. Pause.
\VS{39}Néanmoins, tu l'as rejeté et dédaigné\FTNT{Es. 53:3.} ; tu t'es mis en grande colère contre ton oint.
\VS{40}Tu as rejeté l'alliance faite avec ton serviteur ; tu as souillé sa couronne en la jetant par terre.
\VS{41}Tu as rompu toutes ses murailles ; tu as mis en ruines ses forteresses.
\VS{42}Tous ceux qui passaient par le chemin l'ont pillé ; il a été mis en opprobre à ses voisins.
\VS{43}Tu as élevé la droite de ses adversaires, tu as réjoui tous ses ennemis.
\VS{44}Tu as fait reculer le tranchant de son épée, et tu ne l'as point élevé dans le combat.
\VS{45}Tu as fait cesser sa splendeur, et tu as jeté par terre son trône.
\VS{46}Tu as abrégé les jours de sa jeunesse et l'as couvert de honte. Pause.
\VS{47}Jusqu’à quand, ô Yahweh ? Te cacheras-tu à jamais ? Ta fureur s'embrasera-t-elle comme un feu ?
\VS{48}Souviens-toi quelle est la durée de ma vie ; pourquoi aurais-tu créé en vain tous les fils des hommes ?
\VS{49}Qui est l'homme qui vivra et ne verra point la mort, et qui garantira son âme de la main du scheol\FTNT{1 Co. 15:54-57.} ? Pause.
\VS{50}Seigneur, où sont tes bontés premières que tu juras à David dans ta fidélité ?
\VS{51}Seigneur ! Souviens-toi de l'opprobre de tes serviteurs, et comment je porte dans mon sein l'opprobre qui nous a été fait par tous les peuples nombreux.
\VS{52}Souviens-toi des outrages de tes ennemis, ô Yahweh ! Des outrages contre les pas de ton oint.
\VS{53}Béni soit à toujours Yahweh ; amen ! Oui, amen !
\TextTitle{[Dieu est éternel, l'homme mortel]}
\Chap{90}
\VerseOne{}Prière de Moïse, homme de Dieu\FTNT{De. 33:1.}. Seigneur ! Tu as été pour nous un refuge de génération en génération.
\VS{2}Avant que les montagnes soient nées et que tu aies formé la terre et le monde, d’éternité en éternité, tu es Dieu\FTNT{Ge. 17:1 ; Es. 40:28.}.
\VS{3}Tu fais revenir l'homme à la poussière, et tu dis : Fils des hommes, retournez\FTNT{Ec. 12:7 ; Ge. 3:19.} !
\VS{4}Car mille ans sont à tes yeux comme le jour d'hier qui est passé, et comme une veille de la nuit\FTNT{Ps. 39:5 ; 2 Pi. 3:8.}.
\VS{5}Tu les emportes semblables à un songe qui, le matin, passe comme l’herbe :
\VS{6}Elle fleurit au matin et reverdit ; le soir on la coupe et elle se fane\FTNT{1 Pi. 1:24.}.
\VS{7}Car nous sommes consumés par ta colère et nous sommes troublés par ta fureur.
\VS{8}Tu as mis devant toi nos iniquités, et à la lumière de ta face nos fautes cachées.
\VS{9}Car tous nos jours s'en vont par ta grande colère, et nos années se consument dans un soupir.
\VS{10}Les jours de nos années reviennent à soixante-dix ans, et pour les plus forts, à quatre-vingts ans ; l’orgueil qu’ils en tirent n’est que peine et misère ; car il passe vite, et nous nous envolons.
\VS{11}Qui connaît, selon ta crainte, la force de ton indignation et de ta grande colère ?
\VS{12}Enseigne-nous à compter nos jours, afin que nous puissions avoir un cœur rempli de sagesse.
\VS{13}Yahweh ! Reviens ! Jusqu’à quand ? Sois apaisé envers tes serviteurs.
\VS{14}Rassasie-nous chaque matin de ta bonté, afin que nous nous réjouissions et que nous soyons joyeux tout le long de nos jours.
\VS{15}Réjouis-nous autant de jours que tu nous as affligés, autant d’années que nous avons vu le malheur.
\VS{16}Que ton œuvre se voie sur tes serviteurs, et ta gloire sur leurs fils.
\VS{17}Que la grâce de Yahweh notre Dieu, soit sur nous, et affermis l'œuvre de nos mains ; oui, affermis l'œuvre de nos mains !
\TextTitle{[La sécurité et le fidélité de Dieu]}
\Chap{91}
\VerseOne{}Celui qui demeure sous la couverture\FTNT{Jésus est notre couverture spirituelle.} du Très-Haut, repose à l'ombre du Tout-Puissant.
\VS{2}Je dis à Yahweh : Tu es ma retraite et ma forteresse, tu es mon Dieu en qui je me confie.
\VS{3}Certes, il te délivre du filet de l’oiseleur, de la peste et de ses ravages.
\VS{4}Il te couvrira de ses plumes et tu trouveras un refuge sous ses ailes ; sa fidélité est un bouclier et une cuirasse.
\VS{5}Tu ne craindras ni les terreurs de la nuit, ni la flèche qui vole le jour\FTNT{Pr. 3:23-24.},
\VS{6}ni la peste qui marche dans les ténèbres, ni la destruction qui frappe en plein midi.
\VS{7}Que mille tombent à ton côté et dix mille à ta droite, tu ne seras pas atteint.
\VS{8}De tes yeux tu regarderas et tu verras la rétribution des méchants.
\VS{9}Car tu es mon refuge, ô Yahweh ! Tu fais du Très-Haut ta demeure.
\VS{10}Aucun malheur ne s’approchera de toi, aucun fléau n'approchera de ta tente\FTNT{Ps. 121:6-8 ; Ex. 8:18-19.}.
\VS{11}Car il ordonnera à ses anges de te garder dans toutes tes voies.
\VS{12}Ils te porteront sur les mains, de peur que ton pied ne heurte contre une pierre\FTNT{Mt. 4:5-6 ; Lu. 4:9-11.}.
\VS{13}Tu marcheras sur le lion et sur l'aspic, tu piétineras le lionceau et le dragon.
\VS{14}Puisqu'il m'aime, je le délivrerai ; je le mettrai sur les hauteurs, parce qu'il connaît mon Nom.
\VS{15}Il m'invoquera et je l'exaucerai ; je serai avec lui dans la détresse, je le délivrerai et le glorifierai.
\VS{16}Je le rassasierai de jours et je lui ferai voir ma délivrance.
\TextTitle{[Proclamer la louange de Dieu]}
\Chap{92}
\VerseOne{}Psaume. Cantique pour le jour du sabbat.
\VS{2}C'est une belle chose que de célébrer Yahweh, et de chanter ton Nom, ô Très-Haut\FTNT{Ps. 147:1.} !
\VS{3}Afin d'annoncer chaque matin ta bonté et ta fidélité toutes les nuits\FTNT{Ps. 59:17 ; Ps. 88:14 ; Ps. 89:2.}.
\VS{4}Sur l'instrument à dix cordes, sur le luth, et par un cantique prémédité sur la harpe.
\VS{5}Car, ô Yahweh ! Tu me réjouis par tes œuvres, je me réjouis des œuvres de tes mains.
\VS{6}Ô Yahweh ! Que tes œuvres sont magnifiques ! Tes pensées sont merveilleusement profondes\FTNT{Job. 5:9 ; Es. 55:8-9.}.
\VS{7}L'homme stupide n'y connaît rien et le fou n’y prend point garde\FTNT{Es. 5:12 ; Ro. 1:21.}.
\VS{8}Les méchants croissent comme l'herbe, et tous les ouvriers d'iniquité fleurissent pour être exterminés éternellement\FTNT{Ps. 37:2 ; Jé. 12:1-2 ; Ps. 73:1-20 ; Mal. 3:15.}.
\VS{9}Mais toi, ô Yahweh ! Tu es élevé à toujours.
\VS{10}Car voici tes ennemis, ô Yahweh ! Car voici, tes ennemis périssent, tous les ouvriers d'iniquité sont dispersés.
\VS{11}Mais tu élèveras ma corne comme celle d'un buffle, je serai oint d’une huile fraîche\FTNT{Ps. 23:5 ; Hé. 1:9.}.
\VS{12}Mes yeux se plaisent à regarder ceux qui m'épient, et mes oreilles à entendre les méchants qui s'élèvent contre moi.
\VS{13}Le juste fleurit comme le palmier, il croît comme le cèdre au Liban.
\VS{14}Etant plantés dans la maison de Yahweh, ils fleurissent dans les parvis de notre Dieu.
\VS{15}Ils portent encore des fruits dans la blanche vieillesse ; ils sont gras et verdoyants\FTNT{Ps. 1:3 ; Os. 14:6.},
\VS{16}afin d'annoncer que Yahweh est droit ; c’est mon rocher, et il n'y a point d'injustice en lui.
\TextTitle{[La majesté divine]}
\Chap{93}
\VerseOne{}Yahweh règne, il est revêtu de majesté ; Yahweh est revêtu de force, il s'en est ceint ; aussi le monde est ferme, tellement qu’il ne sera point ébranlé.
\VS{2}Ton trône est établi dès lors, tu es de toute éternité\FTNT{Ps. 9:8 ; Hé. 1:8.}.
\VS{3}Les fleuves élevés, ô Yahweh ! Les fleuves augmentent leur bruit, les fleuves élèvent leurs flots\FTNT{Ps. 46:4 ; Ps. 65:7-8.} ;
\VS{4}Yahweh, qui est dans les lieux élevés, est plus puissant que le bruit des grandes eaux, et que les fortes vagues de la mer\FTNT{Ac. 7:49 ; Es. 57:15.}.
\VS{5}Tes préceptes sont entièrement fidèles. Yahweh ! La sainteté orne ta maison pour une longue durée.
\TextTitle{[A Dieu seul la vengeance]}
\Chap{94}
\VerseOne{}Ô Yahweh ! Dieu des vengeances, Dieu des vengeances, fais briller ta splendeur !
\VS{2}Toi, juge de la terre, élève-toi ! Rends aux orgueilleux selon leurs œuvres.
\VS{3}Jusqu’à quand les méchants, ô Yahweh ! Jusqu’à quand les méchants se réjouiront-ils ?
\VS{4}Jusqu’à quand tous les ouvriers d'iniquité discourront-ils et diront-ils des paroles rudes et se vanteront-ils ?
\VS{5}Yahweh, ils écrasent ton peuple et affligent ton héritage.
\VS{6}Ils tuent la veuve et l'étranger, et ils mettent à mort les orphelins.
\VS{7}Ils disent : Yahweh ne le voit point, le Dieu de Jacob n’entend rien.
\VS{8}Vous les plus abrutis d'entre les peuples, prenez garde à ceci ; et vous insensés, quand serez-vous intelligents ?
\VS{9}Celui qui a planté l'oreille, n'entendrait-t-il point ? Celui qui a formé l'œil, ne verrait-t-il point\FTNT{Pr. 20:12 ; Ex. 4:11.} ?
\VS{10}Celui qui châtie les nations, celui qui enseigne la science aux hommes, ne réprimanderait-il point\FTNT{Ap. 19:15.} ?
\VS{11}Yahweh connaît les pensées des hommes qui ne sont que vanité.
\VS{12}Heureux l'homme que tu châties, ô Yahweh\FTNT{Hé. 12:6.} ! Que tu instruis par ta loi,
\VS{13}afin qu’il soit dans la paix aux jours du malheur, jusqu’à ce que la fosse soit creusée pour le méchant !
\VS{14}Car Yahweh ne délaisse point son peuple et n'abandonne point son héritage\FTNT{Ro. 11:2 ; Es. 49:15.}.
\VS{15}C'est pourquoi le jugement s'unira à la justice, et tous ceux qui sont droits de cœur le suivront.
\VS{16}Qui se lèvera pour moi contre les méchants\FTNT{Ro. 8:31 ; Job. 19:25.} ? Qui m'assistera contre les ouvriers d'iniquité ?
\VS{17}Si Yahweh n’était pas mon secours, mon âme serait bien vite dans la demeure du silence.
\VS{18}Quand je dis : Mon pied chancelle, ta bonté me soutient, ô Yahweh !
\VS{19}Quand j’ai beaucoup des pensées au-dedans de moi, tes consolations font les délices de mon âme.
\VS{20}Serais–tu l’allié du trône de méchanceté, qui forge des injustices contre les règles de la justice ?
\VS{21}Ils se rassemblent contre l'âme du juste et condamnent le sang innocent\FTNT{Mt. 27:1-4 ; Mt. 27:24.}.
\VS{22}Or Yahweh est pour moi une haute retraite ; mon Dieu est le rocher de mon refuge.
\VS{23}Il fera retourner sur eux leur iniquité et les détruira par leur propre méchanceté. Yahweh notre Dieu les détruira\FTNT{Ap. 20:14-15 ; Mt. 13:30.}.
\TextTitle{[Adoration à Yahweh]}
\Chap{95}
\VerseOne{}Venez, chantons à Yahweh, poussons des cris de réjouissance au rocher de notre salut.
\VS{2}Allons au-devant de lui en lui présentant nos louanges ; et poussons devant lui des cris de réjouissance en chantant des psaumes.
\VS{3}Car Yahweh est un grand Dieu, et il est un grand Roi au-dessus de tous les dieux.
\VS{4}Les lieux les plus profonds de la terre sont dans sa main, et les sommets des montagnes sont à lui.
\VS{5}C'est à lui qu'appartient la mer, car lui-même l'a faite, et ses mains ont formé la terre.
\VS{6}Venez, prosternons-nous, inclinons-nous, et mettons-nous à genoux devant Yahweh qui nous a faits\FTNT{Ps. 96:9 ; Ph. 2:10-11.}.
\VS{7}Car il est notre Dieu, et nous sommes le peuple de son pâturage, et les brebis que sa main conduit\FTNT{Ps. 100:3 ; Ps. 23:1 ; Jn. 10:11.}. Si vous entendez aujourd'hui sa voix,
\VS{8}n'endurcissez point votre cœur\FTNT{Hé. 4:7 ; Hé. 3:8.}, comme à Meriba, comme à la journée de Massa, au désert ;
\VS{9}là où vos pères m'ont tenté et éprouvé bien qu’ils virent mes œuvres\FTNT{Ex. 17:7.}.
\VS{10}J'ai eu cette génération en dégoût durant quarante ans, et j'ai dit : C'est un peuple dont le cœur s'égare ; et ils n'ont point connu mes voies ;
\VS{11}c'est pourquoi j'ai juré dans ma colère, ils n’entreront pas dans mon repos\FTNT{Hé. 3:15-19 ; No. 14:22-23 ; Hé. 4:3.}.
\TextTitle{[La grandeur et la gloire de Dieu]}
\Chap{96}
\VerseOne{}Chantez à Yahweh un cantique nouveau\FTNT{Ps. 98:1 ; Es. 42:10 ; Ap. 5:9 ; Ap. 14:3.} ! Vous tous habitants de la terre chantez à Yahweh !
\VS{2}Chantez à Yahweh, bénissez son Nom ! Prêchez de jour en jour sa délivrance !
\VS{3}Racontez sa gloire parmi les nations, ses merveilles parmi tous les peuples\FTNT{Ps. 67:5.}.
\VS{4}Car Yahweh est grand et digne d'être loué ; il est redoutable au-dessus de tous les dieux\FTNT{Ph. 2:9 ; Ap. 5:9.} ;
\VS{5}car tous les dieux des peuples ne sont que des idoles, mais Yahweh a fait les cieux.
\VS{6}La splendeur et la magnificence marchent devant lui, la force et la beauté sont dans son lieu saint.
\VS{7}Familles des peuples, rendez à Yahweh, rendez à Yahweh la gloire et la puissance !
\VS{8}Rendez à Yahweh la gloire due à son Nom ! Apportez des offrandes et entrez dans ses parvis !
\VS{9}Prosternez–vous devant Yahweh avec des ornements sacrés ; tremblez devant lui, vous toute la terre !
\VS{10}Dites parmi les nations : Yahweh règne ; même le monde est affermi, il ne sera point ébranlé ; il jugera les peuples avec équité.
\VS{11}Que les cieux se réjouissent et que la terre soit dans l’allégresse ! Que la mer tonne avec tout ce qui la remplit !
\VS{12}Que les champs s'égayent avec tout ce qui est en eux. Alors tous les arbres de la forêt chanteront de joie
\VS{13}devant Yahweh, car il vient ! Car il vient pour juger la terre ; il jugera avec justice le monde habitable et les peuples selon sa fidélité.
\TextTitle{[Aimer Dieu, c'est haïr le mal]}
\Chap{97}
\VerseOne{}Yahweh règne, que la terre soit dans l’allégresse et que les îles nombreuses s'en réjouissent\FTNT{Ps. 93:1 ; Ps. 86:10 ; Ps. 99:1 ; Es. 42:10.} !
\VS{2}La nuée et l'obscurité sont autour de lui ; la justice et le jugement sont la base de son trône.
\VS{3}Le feu marche devant lui et embrase tout autour ses adversaires.
\VS{4}Ses éclairs éclairent le monde, et la terre le voit et tremble tout étonnée\FTNT{Ap. 4:5 ; Job. 38:35.}.
\VS{5}Les montagnes se fondent comme de la cire\FTNT{Mi. 1:4.}, à cause de la présence de Yahweh, à cause de la présence du Seigneur de toute la terre.
\VS{6}Les cieux annoncent sa justice et tous les peuples voient sa gloire.
\VS{7}Que tous ceux qui servent les images et qui se glorifient des idoles soient confus\FTNT{De. 4:25-26 ; 1 S. 5:1-5.} ; vous dieux, prosternez-vous tous devant lui.
\VS{8}Sion l'a entendu et s'en est réjouie ; et les filles de Juda se sont égayées pour l'amour de tes jugements, ô Yahweh !
\VS{9}Yahweh, tu es le Très-Haut sur toute la terre ; tu es élevé au-dessus de tous les dieux.
\VS{10}Vous qui aimez Yahweh, haïssez le mal\FTNT{Ro. 12:9 ; Am. 5:14-15.} ! Il garde les âmes de ses bien-aimés et les délivre de la main des méchants\FTNT{Jn. 10:28-29 ; Ps. 34:8.}.
\VS{11}La lumière est faite pour le juste\FTNT{Mt. 5:15-16.} et la joie pour ceux qui sont droits de cœur.
\VS{12}Justes, réjouissez-vous en Yahweh et célébrez la mémoire de sa sainteté.
\TextTitle{[Louange à Yahweh]}
\Chap{98}
\VerseOne{}Psaume. Chantez à Yahweh un cantique nouveau, car il a fait des choses merveilleuses ; sa droite et le bras de sa sainteté l'ont délivré\FTNT{Es. 63:3-5 ; Es 53:1 ; Es. 52:10.}.
\VS{2}Yahweh a fait connaître son salut\FTNT{Il est question de la révélation de Jésus. Voir commentaire Es. 26:1.}, il a révélé sa justice devant les yeux des nations.
\VS{3}Il s'est souvenu de sa bonté et de sa fidélité envers la maison d’Israël ; toutes les extrémités de la terre ont vu la délivrance de notre Dieu\FTNT{Lu. 1:72 ; Ac. 13:47 ; Es. 49:6.}.
\VS{4}Vous tous habitants de la terre, poussez des cris de réjouissance à Yahweh ! Faites retentir vos cris et chantez de joie !
\VS{5}Chantez à Yahweh avec la harpe, avec la harpe et avec une voix mélodieuse !
\VS{6}Poussez des cris de réjouissance avec le shofar au son du cor devant le Roi, Yahweh !
\VS{7}Que la mer tonne avec tout ce qu'elle contient, que la terre et ceux qui y habitent fassent éclater leurs cris !
\VS{8}Que les fleuves frappent des mains et que les montagnes chantent de joie
\VS{9}devant Yahweh ! Car il vient pour juger la terre\FTNT{Yahweh qui vient pour juger la terre est Jésus-Christ ( 
Za. 14:1-7 ; 2 Ti. 4:1 ; Ap. 19:15).} ; il jugera le monde habitable avec justice et les peuples avec équité.
\TextTitle{[La grandeur et la sainteté de Dieu]}
\Chap{99}
\VerseOne{}Yahweh règne, que les peuples tremblent ; il est assis entre les chérubins, que la terre soit ébranlée\FTNT{Ex. 25:22 ; Es. 37:16.}.
\VS{2}Yahweh est grand en Sion, et il est élevé au-dessus de tous les peuples.
\VS{3}Ils célébreront ton Nom, grand et terrible, car il est saint.
\VS{4}Qu’on célèbre la force du roi qui aime la justice ! Tu as ordonné l'équité, tu as prononcé des jugements justes en Jacob.
\VS{5}Exaltez Yahweh notre Dieu et prosternez-vous devant son marchepied ! Il est saint !
\VS{6}Moïse et Aaron étaient parmi ses sacrificateurs\FTNT{Ex. 31:10 ; Lé. 2:2.} ; et Samuel parmi ceux qui invoquaient son Nom ; ils invoquaient Yahweh et il leur répondait\FTNT{1 S. 12:18-19.}.
\VS{7}Il leur parlait de la colonne de nuée ; ils ont gardé ses préceptes et l'ordonnance qu'il leur avait donnée.
\VS{8}Ô Yahweh, mon Dieu ! Tu les as exaucés, tu as été pour eux un Dieu qui pardonne\FTNT{Hé. 10:16-17.}, mais tu les as punis de leurs fautes.
\VS{9}Exaltez Yahweh notre Dieu ! Prosternez-vous sur la montagne de sa sainteté ! Car Yahweh, notre Dieu est saint !
\TextTitle{[Célébrer et bénir le nom de Yahweh]}
\Chap{100}
\VerseOne{}Psaume de louange. Vous tous habitants de la terre, poussez des cris de réjouissance à Yahweh !
\VS{2}Servez Yahweh avec allégresse, venez devant lui avec un chant de joie !
\VS{3}Sachez que Yahweh est Dieu. C'est lui qui nous a faits, ce n'est pas nous qui nous sommes faits ; nous sommes son peuple et le troupeau de son pâturage\FTNT{Ps. 95:6 ; Ps. 119:73 ; Ps. 79:13 ; Ps. 80:1.}.
\VS{4}Entrez dans ses portes avec des louanges ; et dans ses parvis, avec des cantiques. Célébrez-le, bénissez son Nom !
\VS{5}Car Yahweh est bon ; sa bonté demeure à toujours et sa fidélité de génération en génération.
\TextTitle{[Appel à l'intégrité]}
\Chap{101}
\VerseOne{}Psaume de David. Je chanterai la miséricorde et la justice ; Yahweh ! Je te chanterai.
\VS{2}Je me rendrai attentif à une conduite pure jusqu’à ce que tu viennes à moi ; je marcherai dans l'intégrité de mon cœur au milieu de ma maison.
\VS{3}Je ne mettrai point devant mes yeux des choses méchantes\FTNT{Ps. 26:5 ; Ps. 119:115.} ; j'ai en haine les actions des débauchés ; elles ne s’attacheront pas à moi.
\VS{4}Le cœur mauvais s’éloignera de moi ; je ne connaîtrai pas le méchant.
\VS{5}Je retrancherai celui qui calomnie en secret son prochain ; je ne supporterai pas celui qui a les yeux élevés et le cœur enflé\FTNT{Pr. 6:16-17.}.
\VS{6}Je prendrai garde aux gens de bien du pays afin qu'ils demeurent avec moi ; celui qui marche dans la voie de l’intégrité me servira.
\VS{7}Celui qui usera de tromperie ne demeurera point dans ma maison ; celui qui profèrera des mensonges ne sera point affermi devant mes yeux.
\VS{8}Je retrancherai chaque matin tous les méchants du pays, afin d'exterminer de la cité de Yahweh tous les ouvriers d'iniquité.
\TextTitle{[Yahweh, le Dieu immuable]}
\Chap{102}
\VerseOne{}Prière de l'affligé étant dans l'angoisse et répandant sa plainte devant Yahweh.
\VS{2}Yahweh ! Ecoute ma prière, et que mon cri parvienne jusqu’à toi\FTNT{Ps. 69:14.}.
\VS{3}Ne cache pas ta face arrière de moi ; au jour où je suis en détresse, prête l'oreille à ma prière ; au jour où je t'invoque, hâte-toi de me répondre.
\VS{4}Car mes jours se sont évanouis comme la fumée et mes os brûlent comme dans un foyer.
\VS{5}Mon cœur est frappé et se dessèche comme l'herbe, car j'ai oublié de manger mon pain\FTNT{Mt. 4:4 ; Lu. 4:4.}.
\VS{6}Le gémissement de ma voix est tel que mes os s’attachent à ma chair\FTNT{Job. 19:20.}.
\VS{7}Je suis devenu semblable au pélican du désert ; et je suis comme la chouette des lieux sauvages.
\VS{8}Je veille et je suis semblable au passereau solitaire sur le toit.
\VS{9}Mes ennemis m’outragent tous les jours, et ceux qui sont furieux contre moi jurent contre moi.
\VS{10}Car j'ai mangé la cendre comme le pain et j'ai mêlé des larmes à ma boisson,
\VS{11}à cause de ta colère et de ta fureur ; car après m'avoir soulevé, tu m'as jeté par terre.
\VS{12}Mes jours sont comme l'ombre qui décline et je deviens sec comme l'herbe.
\VS{13}Mais toi, ô Yahweh ! Tu demeures éternellement, et ta mémoire est de génération en génération.
\VS{14}Tu te lèveras, tu auras compassion de Sion ; car il est temps d'en avoir pitié, parce que le temps assigné est échu.
\VS{15}Car tes serviteurs aiment ses pierres et chérissent sa poussière.
\VS{16}Alors les nations redouteront le Nom de Yahweh, et tous les rois de la terre ta gloire.
\VS{17}Quand Yahweh aura édifié Sion, quand il aura été vu dans sa gloire,
\VS{18}quand il aura eu égard à la prière du désolé et qu'il n'aura point méprisé leur supplication.
\VS{19}Cela sera enregistré pour la génération à venir, le peuple qui sera créé louera Yahweh !
\VS{20}Car il regarde du lieu élevé de sa sainteté. Du haut des cieux, Yahweh regarde la terre,
\VS{21}pour entendre le gémissement des prisonniers, pour délier ceux qui étaient voués à la mort\FTNT{Es. 42:6-7 ; Es. 61:1 ; Lu. 4:18-19.},
\VS{22}afin qu'on annonce le Nom de Yahweh dans Sion et sa louange dans Jérusalem,
\VS{23}quand les peuples se seront joints ensemble, et les royaumes aussi, pour servir Yahweh.
\VS{24}Il a brisé ma force en chemin, il a abrégé mes jours.
\VS{25}Je dis : Mon Dieu, ne m'enlève point au milieu de mes jours dont les années durent éternellement.
\VS{26}Tu as jadis fondé la terre, et les cieux sont l'ouvrage de tes mains.
\VS{27}Ils périront, mais tu subsisteras, ils s’useront tous comme un vêtement ; tu les changeras comme un habit, et ils seront changés.
\VS{28}Mais toi, tu es toujours le même et tes années ne seront jamais achevées.
\VS{29}Les fils de tes serviteurs habiteront près de toi et leur postérité sera établie devant toi.
\TextTitle{[Yahweh, le Dieu miséricordieux et compatissant]}
\Chap{103}
\VerseOne{}Psaume de David. Mon âme, bénis Yahweh, et que tout ce qui est en moi bénisse son saint Nom.
\VS{2}Mon âme, bénis Yahweh, et n'oublie pas un de ses bienfaits\FTNT{De. 6:12.}.
\VS{3}C'est lui qui pardonne toutes tes iniquités, qui guérit toutes tes infirmités\FTNT{Ps. 130:3-4 ; Es. 33:24 ; Mt. 9:6 ; Lu. 7:47 ; Jé. 17:14 ; Es. 53:5.} ;
\VS{4}qui garantit ta vie de la fosse\FTNT{Ps. 106:10 ; Es. 59:20.}, qui te couronne de bonté et de compassions ;
\VS{5}qui rassasie ta bouche de biens ; ta jeunesse est renouvelée comme celle de l'aigle\FTNT{Es. 40:31.}.
\VS{6}Yahweh fait justice et droit à tous les opprimés\FTNT{Ps. 146:7.}.
\VS{7}Il a fait connaître ses voies à Moïse, et ses exploits aux fils d'Israël\FTNT{Ex. 33:12-17.}.
\VS{8}Yahweh est compatissant, miséricordieux, lent à la colère, et riche en bonté.
\VS{9}Il ne conteste pas éternellement, et il ne garde point à toujours sa colère\FTNT{Es. 57:16 ; Jé. 3:5 ; Mi. 7:18.}.
\VS{10}Il ne nous traite pas selon nos péchés, et ne nous rend point selon nos iniquités\FTNT{Esd. 9:13.}.
\VS{11}Car autant les cieux sont élevés au-dessus de la terre, autant sa bonté est grande sur ceux qui le craignent.
\VS{12}Il éloigne de nous nos transgressions, autant que l'orient est éloigné de l'occident\FTNT{Es. 38:17.}.
\VS{13}Comme un père a compassion de ses fils, Yahweh a compassion de ceux qui le craignent\FTNT{Mal. 3:17 ; Lu. 11:11-13.}.
\VS{14}Car il sait bien de quoi nous sommes faits, se souvenant que nous ne sommes que poussière.
\VS{15}L’homme ! Ses jours sont comme l’herbe, il fleurit comme la fleur d'un champ.
\VS{16}Car le vent étant passé par-dessus, elle n'est plus, et son lieu ne la reconnaît plus.
\VS{17}Mais la miséricorde de Yahweh est de tout temps, et elle sera pour toujours en faveur de ceux qui le craignent ; et sa justice en faveur des fils de leurs fils ;
\VS{18}pour ceux qui gardent son alliance, et qui se souviennent de ses commandements pour les faire\FTNT{De. 7:9.}.
\VS{19}Yahweh a établi son trône dans les cieux, et son règne domine sur tout.
\VS{20}Bénissez Yahweh, vous ses anges puissants en force, qui faites ses affaires, en obéissant à la voix de sa parole.
\VS{21}Bénissez Yahweh, vous toutes ses armées, qui êtes ses serviteurs faisant sa volonté.
\VS{22}Bénissez Yahweh, vous toutes ses œuvres, par tous les lieux de sa domination. Mon âme, bénis Yahweh !
\TextTitle{[Yahweh, le Dieu de toute la création]}
\Chap{104}
\VerseOne{}Mon âme, bénis Yahweh. Ô Yahweh mon Dieu, tu es merveilleusement grand, tu es revêtu de majesté et de splendeur.
\VS{2}Il s'enveloppe de lumière comme d'un vêtement, il étend les cieux comme un voile\FTNT{Job. 9:8 ; 1 Ti. 6:16 ; Es. 40:22.}.
\VS{3}Avec les eaux, il va à la rencontre de sa demeure ; il fait des grosses nuées son char, il se promène sur les ailes du vent\FTNT{Ps. 18:10 ; Es. 19:1 ; Ap. 14:14.}.
\VS{4}Il fait des vents ses messagers, et des flammes de feu ses serviteurs\FTNT{Hé. 1:7 ; Ps. 148:8 ; Jn. 3:8.}.
\VS{5}Il a fondé la terre sur ses bases, elle ne sera jamais ébranlée\FTNT{Ps. 24:1-2 ; Ps. 78:69 ; Ps. 93:1 ; Job. 26:7 ; Job. 38:4-6.}.
\VS{6}Tu l'avais couverte de l'abîme comme d'un vêtement, les eaux se tenaient sur les montagnes\FTNT{Ge. 1:2.}.
\VS{7}Elles s'enfuirent à ta menace et se mirent promptement en fuite au son de ton tonnerre.
\VS{8}Les montagnes s'élevèrent et les vallées s'abaissèrent au même lieu que tu leur avais fixé.
\VS{9}Tu as posé une limite que les eaux ne doivent point franchir, afin qu’elles ne reviennent plus couvrir la terre\FTNT{Jé. 5:2 ; Job. 26:10 ; Ge. 1:9 ; Pr. 8:29.}.
\VS{10}C'est lui qui conduit les sources par les vallées, elles se promènent entre les monts.
\VS{11}Elles abreuvent toutes les bêtes des champs, les ânes sauvages y étanchent leur soif.
\VS{12}Les oiseaux des cieux se tiennent auprès d'elles, et font résonner leur voix parmi les rameaux.
\VS{13}Il abreuve les montagnes de ses chambres hautes ; la terre est rassasiée du fruit de tes œuvres.
\VS{14}Il fait germer l’herbe pour le bétail, et les plantes pour le besoin de l'homme, faisant sortir le pain de la terre,
\VS{15}et le vin qui réjouit le cœur de l'homme\FTNT{Jg. 9:11 ; Pr. 31:6-7.}, qui fait resplendir son visage avec l'huile, et qui soutient le cœur de l'homme avec le pain.
\VS{16}Les hauts arbres de Yahweh en sont rassasiés, ainsi que les cèdres du Liban qu'il a plantés,
\VS{17}afin que les oiseaux y fassent leurs nids. Quant à la cigogne, les sapins sont sa demeure.
\VS{18}Les hautes montagnes sont pour les chamois, et les rochers sont la retraite des lapins.
\VS{19}Il a fait la lune pour les saisons, et le soleil sait quand il doit se coucher\FTNT{Ge. 1:16.}.
\VS{20}Tu amènes les ténèbres, et il fait nuit ; alors toutes les bêtes de la forêt sont en mouvement.
\VS{21}Les lionceaux rugissent après la proie pour demander à Dieu leur nourriture.
\VS{22}Le soleil se lève-t-il ? Ils se retirent et se couchent dans leurs tanières.
\VS{23}Alors l'homme sort pour se rendre à son ouvrage, et à son travail jusqu’au soir.
\VS{24}Ô Yahweh, que tes œuvres sont en grand nombre ! Tu les as toutes faites avec sagesse ; la terre est pleine de tes richesses.
\VS{25}Cette mer grande et spacieuse, là où des animaux sans nombre se meuvent, des petites bêtes avec des grandes !
\VS{26}Là se promènent les navires, et ce léviathan que tu as formé pour jouer dans les flots.
\VS{27}Ils s'attendent tous à toi afin que tu leur donnes la nourriture en leur temps.
\VS{28}Quand tu la leur donnes, ils la recueillent, et quand tu ouvres ta main, ils sont rassasiés de biens.
\VS{29}Caches-tu ta face ? Ils sont troublés ; retires-tu leur souffle ? Ils défaillent et retournent dans leur poussière.
\VS{30}Tu envoies ton souffle, ils sont créés ; et tu renouvelles la face de la terre.
\VS{31}Que la gloire de Yahweh subsiste à toujours, que Yahweh se réjouisse dans ses œuvres !
\VS{32}Il jette son regard sur la terre et elle tremble ; il touche les montagnes et elles fument.
\VS{33}Je chanterai à Yahweh durant ma vie ; je chanterai à mon Dieu tant que j'existerai.
\VS{34}Ma méditation lui sera agréable, et je me réjouirai en Yahweh.
\VS{35}Que les pécheurs soient consumés de dessus la terre et qu'il n'y ait plus de méchants ! Mon âme, bénis Yahweh ! Louez Yahweh !
\TextTitle{[La fidélité de Dieu envers son peuple]}
\Chap{105}
\VerseOne{}Célébrez Yahweh, invoquez son Nom, faites connaître parmi les peuples ses exploits.
\VS{2}Chantez-le, chantez-le, parlez de toutes ses merveilles !
\VS{3}Glorifiez-vous de son saint Nom, et que le cœur de ceux qui cherchent Yahweh se réjouisse.
\VS{4}Recherchez Yahweh et sa puissance ; cherchez continuellement sa face.
\VS{5}Souvenez-vous de ses merveilles qu'il a faites, de ses miracles, et des jugements de sa bouche.
\VS{6}La postérité d’Abraham sont ses serviteurs ; les enfants de Jacob sont ses élus !
\VS{7}Il est Yahweh notre Dieu, ses jugements sont sur toute la terre.
\VS{8}Il s'est souvenu pour toujours de son alliance, de la parole qu'il a ordonnée pour mille générations,
\VS{9}du traité qu'il a fait avec Abraham et du serment qu'il a fait à Isaac\FTNT{Ge. 17:2 ; Ge. 22:16 ; Ge. 26:3 ; Ge. 28:13 ; Ge. 33:11 ; Lu. 1:73.}.
\VS{10}Il l’a érigé pour être une ordonnance à Jacob, et à Israël pour être une alliance éternelle,
\VS{11}en disant : Je te donnerai le pays de Canaan, comme héritage qui vous est échu\FTNT{Ge. 13:15 ; Ge. 15:18.}.
\VS{12}Ils étaient alors un petit nombre de gens, très peu nombreux, et étrangers dans le pays.
\VS{13}Car ils allaient de nation en nation, et d'un royaume vers un autre peuple.
\VS{14}Il ne permit à personne de les opprimer, il châtia des rois à cause d'eux\FTNT{Ge. 35:5.},
\VS{15}disant : Ne touchez point à mes oints et ne faites point de mal à mes prophètes\FTNT{1 Ch. 16:22.} !
\VS{16}Il appela aussi la famine sur la terre, rompit le bâton du pain\FTNT{Lé. 26:26 ; Es. 3:1 ; Ez. 4:16.}.
\VS{17}Il envoya un homme devant eux ; Joseph fut vendu pour esclave\FTNT{Ge. 37:28-36.}.
\VS{18}On serra ses pieds dans des ceps, sa personne fut mise aux fers.
\VS{19}Jusqu’au temps où arriva ce qu’il avait annoncé, et où la parole de Yahweh l’éprouva.
\VS{20}Le roi le relâcha et le laissa aller ; le dominateur des peuples le délivra.
\VS{21}Il l'établit pour maître sur sa maison, et pour gouverneur sur tout son domaine\FTNT{Ge. 41:40.} ;
\VS{22}pour soumettre les princes à ses désirs, et pour instruire ses anciens.
\VS{23}Puis Israël entra en Egypte, et Jacob séjourna dans le pays de Cham\FTNT{Ge. 46:6 ; Ps. 78:51.}.
\VS{24}Yahweh rendit son peuple très fécond et le rendit plus puissant que ceux qui l'opprimaient.
\VS{25}Il changea leur cœur, au point qu'ils haïrent son peuple jusqu’à conspirer contre ses serviteurs\FTNT{Ex. 1:7-12.}.
\VS{26}Il envoya Moïse son serviteur, et Aaron, qu'il avait élu\FTNT{Ex. 4:14.}.
\VS{27}Ils accomplirent au milieu d’eux des prodiges et des miracles qu’ils avaient eu la charge de faire dans le pays de Cham.
\VS{28}Il envoya les ténèbres et fit venir l’obscurité ; et ils ne furent point rebelles à sa parole.
\VS{29}Il changea leurs eaux en sang et fit mourir leurs poissons.
\VS{30}Leur terre produisit en abondance des grenouilles jusque dans les chambres de leurs rois.
\VS{31}Il dit, et des mouches vinrent, des poux sur tout leur pays.
\VS{32}Il leur donna pour pluie de la grêle, et un feu flamboyant sur la terre.
\VS{33}Il frappa leurs vignes et leurs figuiers, et il brisa les arbres du pays.
\VS{34}Il ordonna et les sauterelles vinrent, des jeunes sauterelles sans nombre
\VS{35}qui dévorèrent toute l'herbe du pays, et qui dévorèrent le fruit de leur terroir.
\VS{36}Il frappa tous les premiers-nés du pays, les prémices de toute leur vigueur\FTNT{Ex. 7 ; Ex. 8 ; Ex. 9 ; Ex. 10 ; Ex. 11 ; Ex. 12.}.
\VS{37}Puis il les fit sortir avec de l'or et de l'argent, et nul ne chancela parmi ses tribus.
\VS{38}Les Egyptiens se réjouirent à leur départ, car la peur qu'ils avaient d'eux les avait saisis.
\VS{39}Il étendit la nuée pour couverture, et le feu pour éclairer la nuit.
\VS{40}Le peuple demanda et il fit venir des cailles ; et il les rassasia du pain des cieux\FTNT{Ex. 16:12-13.}.
\VS{41}Il ouvrit le rocher et les eaux en coulèrent ; elles se répandirent comme un fleuve dans les lieux arides\FTNT{Ex. 17:6.}.
\VS{42}Car il se souvint de sa parole sainte qu’il avait donnée à Abraham son serviteur\FTNT{Ge. 15:13-16.}.
\VS{43}Il fit sortir son peuple dans l’allégresse, ses élus au milieu des cris retentissants\FTNT{Ex. 15:1.}.
\VS{44}Il leur donna les terres des nations et ils possédèrent le fruit du travail des peuples,
\VS{45}afin qu'ils gardent ses statuts et qu'ils observent ses lois. Louez Yahweh !
\TextTitle{[L'infidélité d'Israël]}
\Chap{106}
\VerseOne{}Louez Yahweh ! Célébrez Yahweh car il est bon, car sa bonté demeure à toujours !
\VS{2}Qui pourrait réciter les exploits de Yahweh ? Qui pourrait faire retentir toute sa louange ?
\VS{3}Heureux ceux qui observent la justice, qui font en tout temps ce qui est juste !
\VS{4}Yahweh, souviens-toi de moi selon la bienveillance que tu portes à ton peuple, aie soin de moi selon ta délivrance !
\VS{5}Afin que je voie le bien de tes élus, que je me réjouisse dans la joie de ta nation, que je me glorifie avec ton héritage.
\VS{6}Nous avons péché avec nos pères, nous avons agi dans l’iniquité, nous avons fait le mal\FTNT{Esd. 9:7 ; Né. 1:6 ; Da. 9:16.}.
\VS{7}Nos pères n'ont point été attentifs à tes merveilles en Egypte ; ils ne se sont point souvenus de la multitude de tes faveurs ; mais ils furent rebelles près de la mer, vers la Mer Rouge\FTNT{Ex. 14:11.}.
\VS{8}Toutefois, il les délivra pour l'amour de son Nom, afin de faire connaître sa puissance.
\VS{9}Il menaça la Mer Rouge et elle se sécha ; et il les conduisit à travers les profondeurs de la mer comme un désert ;
\VS{10}il les délivra de la main de ceux qui les haïssaient et les racheta de la main de l'ennemi.
\VS{11}Les eaux couvrirent leurs oppresseurs, il n'en resta pas un seul\FTNT{Ex. 14:27.}.
\VS{12}Alors ils crurent à ses paroles et ils chantèrent sa louange.
\VS{13}Mais ils oublièrent vite ses œuvres et ne s'attendirent point à son conseil.
\VS{14}Ils furent épris de convoitise au désert et ils tentèrent Dieu dans le désert.
\VS{15}Alors il leur donna ce qu'ils avaient demandé, toutefois il leur envoya le dépérissement dans leur corps.
\VS{16}Ils jalousèrent dans le camp Moïse et Aaron, le saint de Yahweh.
\VS{17}La terre s'ouvrit et engloutit Dathan ; et recouvrit de terre Abiram\FTNT{No. 16.}.
\VS{18}Le feu s'alluma au milieu de leur assemblée, la flamme brûla les méchants.
\VS{19}Ils firent un veau en Horeb, et se prosternèrent devant une image de métal fondu\FTNT{Ex. 32.}.
\VS{20}Ils changèrent leur gloire contre la figure d’un bœuf qui mange l'herbe.
\VS{21}Ils oublièrent Dieu, leur libérateur, qui avait fait de grandes choses en Egypte,
\VS{22}des choses merveilleuses dans le pays de Cham, et des prodiges sur la Mer Rouge.
\VS{23}C'est pourquoi il dit qu'il les détruirait ; mais Moïse, son élu, se tint à la brèche devant lui pour détourner sa fureur, afin qu'il ne les détruisît point\FTNT{Ex. 32:11.}.
\VS{24}Ils méprisèrent le pays désirable et ne crurent point à sa parole.
\VS{25}Ils murmurèrent dans leurs tentes et n'obéirent point à la voix de Yahweh.
\VS{26}C'est pourquoi il leur jura la main levée de les faire tomber dans le désert,
\VS{27}d’accabler leur postérité parmi les nations, et de les disperser au milieu des pays\FTNT{No. 14:22.}.
\VS{28}Ils se joignirent aux adorateurs de Baal-Peor et mangèrent des victimes sacrifiées aux morts.
\VS{29}Ils irritèrent Dieu par leurs actions, au point qu'une plaie fit une brèche parmi eux.
\VS{30}Mais Phinées se présenta et fit justice ; et la plaie fut arrêtée.
\VS{31}Cela lui fut imputé à justice de génération en génération, pour toujours\FTNT{No. 25:3-8.}.
\VS{32}Ils excitèrent aussi sa colère près des eaux de Meriba, et Moïse fut puni à cause d'eux.
\VS{33}Car ils aigrirent son esprit et il parla avec légèreté de ses lèvres\FTNT{No. 20:12.}.
\VS{34}Ils ne détruisirent point les peuples que Yahweh leur avait dit de détruire,
\VS{35}mais ils se mêlèrent parmi ces nations et apprirent leurs manières de faire.
\VS{36}Ils servirent leurs faux dieux qui furent un piège pour eux.
\VS{37}Car ils sacrifièrent leurs fils et leurs filles aux démons\FTNT{Lé. 18:21 ; De. 12:31 ; 2 R. 16:3 ; Ez. 20:26.}.
\VS{38}Ils répandirent le sang innocent, le sang de leurs fils et de leurs filles, ils sacrifièrent aux faux dieux de Canaan ; et le pays fut souillé de sang\FTNT{No. 35:33.}.
\VS{39}Ils se souillèrent par leurs œuvres et se prostituèrent par leurs actions.
\VS{40}C'est pourquoi la colère de Yahweh s'embrasa contre son peuple et il eut en abomination son héritage.
\VS{41}Il les livra entre les mains des nations, et ceux qui les haïssaient dominèrent sur eux.
\VS{42}Leurs ennemis les opprimèrent et ils furent humiliés sous leur main.
\VS{43}Il les délivra souvent, mais ils se montrèrent rebelles dans leurs desseins et furent humiliés par leur iniquité.
\VS{44}Toutefois, il vit leur détresse lorsqu’il entendit leurs supplications.
\VS{45}Il se souvint en leur faveur de son alliance et se repentit selon la grandeur de ses compassions.
\VS{46}Il fit que ceux qui les avaient emmenés captifs eurent pitié d'eux.
\VS{47}Yahweh notre Dieu, délivre-nous et rassemble-nous du milieu des nations ! Afin que nous célébrions ton saint Nom et que nous mettions notre gloire à te louer !
\VS{48}Béni soit Yahweh, le Dieu d'Israël, d’éternité en éternité ! Et que tout le peuple dise amen ! Louez Yahweh.
\TextTitle{[La grâce de Dieu pour les rachetés]}
\Chap{107}
\VerseOne{}Célébrez Yahweh car il est bon, parce que sa bonté demeure à toujours.
\VS{2}Qu’ainsi disent les rachetés de Yahweh, ceux qu’il a rachetés de la main de l'oppresseur,
\VS{3}et qu'il a rassemblés de tous les pays, de l'orient et de l’occident, du nord et de la mer.
\VS{4}Ils erraient dans le désert, ils marchaient dans la solitude, sans trouver une ville où ils puissent habiter.
\VS{5}Ils étaient affamés et assoiffés, leur âme était languissante.
\VS{6}Alors ils crièrent vers Yahweh dans leur détresse et il les délivra de leurs angoisses ;
\VS{7}il les conduisit sur le droit chemin pour aller dans une ville habitée.
\VS{8}Qu'ils célèbrent Yahweh pour sa bonté et ses merveilles envers les fils des hommes !
\VS{9}Parce qu'il a désaltéré l'âme altérée et rassasié de ses biens l'âme affamée\FTNT{Ps. 146:7 ; Lu. 1:53.}.
\VS{10}Ceux qui avaient pour demeure les ténèbres et l'ombre de la mort, vivaient captifs dans l’affliction et dans les chaînes,
\VS{11}parce qu'ils furent rebelles aux paroles de Dieu, et parce qu'ils avaient rejeté le conseil du Très-Haut\FTNT{De. 31:20 ; La. 3:42.}.
\VS{12}Il humilia leur cœur par la souffrance, ils furent abattus ; et personne ne les secourut.
\VS{13}Alors ils crièrent vers Yahweh dans leur détresse, et il les délivra de leurs angoisses.
\VS{14}Il les fit sortir hors des ténèbres et de l'ombre de la mort ; et il rompit leurs liens\FTNT{Ps. 68:19 ; Ep. 4:8 ; Col. 1:12-13.}.
\VS{15}Qu'ils célèbrent Yahweh pour sa bonté et ses merveilles envers les fils des hommes !
\VS{16}Parce qu'il a brisé les portes d'airain et cassé les barreaux de fer.
\VS{17}Les insensés sont affligés à cause de leurs transgressions et à cause de leurs iniquités.
\VS{18}Leur âme avait en horreur toute nourriture, et ils touchaient aux portes de la mort.
\VS{19}Alors ils crièrent vers Yahweh dans leur détresse, et il les délivra de leurs angoisses\FTNT{Ps. 50:15 ; Os. 5:15.}.
\VS{20}Il envoya sa parole et les guérit ; et il les délivra de leurs tombeaux.
\VS{21}Qu'ils célèbrent Yahweh pour sa bonté et ses merveilles envers les fils des hommes !
\VS{22}Qu'ils offrent des sacrifices de remerciements, et qu'ils racontent ses œuvres avec des cris de joie.
\VS{23}Ceux qui descendaient sur la mer dans des navires, faisant commerce sur les grandes eaux,
\VS{24}ceux-là virent les œuvres de Yahweh et ses merveilles dans les lieux profonds,
\VS{25}car il dit, et il fit paraître la tempête qui souleva les vagues de la mer.
\VS{26}Ils montaient vers les cieux, ils descendaient dans l’abîme ; leur âme se fondait d'angoisse.
\VS{27}Saisis de vertiges, ils chancelaient comme un homme ivre ; et toute leur sagesse était anéantie\FTNT{Es. 51:17-21 ; Jé. 13:13.}.
\VS{28}Alors ils crièrent vers Yahweh dans leur détresse, et il les tira hors de leurs angoisses.
\VS{29}Il arrêta la tempête, la changeant en calme, et les ondes se turent.
\VS{30}Puis ils se réjouirent de ce qu'elles s’étaient apaisées, et il les conduisit au port qu'ils désiraient.
\VS{31}Qu'ils célèbrent Yahweh pour sa bonté et ses merveilles envers les fils des hommes !
\VS{32}Et qu'ils l'exaltent dans l’assemblée du peuple et le louent dans l'assemblée des anciens.
\VS{33}Il réduit les fleuves en désert, et les sources d'eaux en sécheresse ;
\VS{34}la terre fertile en terre salée, à cause de la méchanceté de ses habitants\FTNT{Jé. 12:4 ; Jé. 17:6.}.
\VS{35}Il transforme le désert en étangs d’eaux, et la terre sèche en des sources d'eaux\FTNT{Es. 41:18.} ;
\VS{36}il y établit ceux qui sont affamés, ils bâtissent des villes pour l’habiter.
\VS{37}Ils ensemencent des champs et plantent des vignes qui rendent du fruit tous les ans.
\VS{38}Il les bénit et ils se multiplient extrêmement ; et il ne laisse point diminuer leur bétail.
\VS{39}Puis ils sont amoindris et humiliés par l'oppression, le malheur et la souffrance.
\VS{40}Il répand le mépris sur les princes et les fait errer dans des lieux déserts sans chemin.
\VS{41}Mais il relève le pauvre et le délivre de la misère, il établit les familles comme des troupeaux\FTNT{Ps. 113:7 ; 1 S. 2:8.}.
\VS{42}Les hommes droits le voient et se réjouissent, mais toute iniquité a la bouche fermée.
\VS{43}Quiconque est sage prendra garde à ces choses, afin qu'on considère les bontés de Yahweh.
\TextTitle{[Louange à Yahweh qui affermit son peuple]}
\Chap{108}
\VerseOne{}Cantique. Psaume de David. Mon cœur est affermi, ô Dieu ! Je chante et je joue de mes instruments, c’est ma gloire !
\VS{2}Réveillez-vous, mon luth et ma harpe ! Je me réveillerai à l'aube du jour.
\VS{3}Yahweh, je te célébrerai parmi les peuples et je te chanterai parmi les nations.
\VS{4}Car ta bonté est grande par-dessus les cieux, et ta vérité atteint jusqu’aux nues.
\VS{5}Ô Dieu ! Elève-toi sur les cieux, et que ta gloire soit sur toute la terre !
\VS{6}Afin que ceux que tu aimes soient délivrés ; sauve-moi par ta droite et exauce-moi !
\VS{7}Dieu a dit dans sa sainteté : Je me réjouirai, je partagerai Sichem et mesurerai la vallée de Succoth.
\VS{8}Galaad sera à moi, Manassé sera à moi, et Ephraïm sera le sommet de ma forteresse, Juda mon législateur.
\VS{9}Moab sera le bassin où je me laverai, je jetterai mon soulier sur Edom, je triompherai des Philistins.
\VS{10}Qui me conduira dans la ville forte ? Qui me conduira jusqu’en Edom ?
\VS{11}N'est-ce pas toi, ô Dieu, qui nous avais rejetés, et qui ne sortais plus, ô Dieu, avec nos armées ?
\VS{12}Donne–nous du secours pour sortir de la détresse ! Car la délivrance qu'on attend de l'homme est vaine.
\VS{13}Avec Dieu, nous ferons des exploits ; il foulera nos ennemis\FTNT{Ps. 60:5-14.}.
\TextTitle{[La méchanceté de l'homme]}
\Chap{109}
\VerseOne{}Psaume de David, donné au chef des chantres\FTNT{Les Psaumes d’imprécations (Ps. 35, 52, 55, 58, 59, 79, 109, 137) sont des demandes faites à Dieu pour qu’il punisse les méchants. Le Seigneur Jésus-Christ nous demande aujourd’hui de bénir nos ennemis (Lu. 6:27-37).}. Dieu de ma louange, ne te tais point !
\VS{2}Car la bouche du méchant et la bouche remplie de fraude se sont ouvertes contre moi ; ils parlent contre moi avec une langue mensongère !
\VS{3}Ils m’entourent de paroles pleines de haine et ils me font la guerre sans cause !
\VS{4}Tandis que je les aime, ils sont mes ennemis ; mais moi, je n'ai fait que prier en leur faveur !
\VS{5}Ils me rendent le mal pour le bien, et la haine pour l'amour que je leur porte.
\VS{6}Etablis le méchant sur lui, et que Satan se tienne à sa droite !
\VS{7}Quand il sera jugé, fais qu'il soit déclaré méchant, et que sa prière soit regardée comme un crime !
\VS{8}Que sa vie soit courte\FTNT{Il est question ici du suicide de Judas (Mt. 27:3-5).} et qu'un autre prenne sa charge\FTNT{Ce passage parle de Judas (Ac. 1:20).} !
\VS{9}Que ses enfants soient orphelins et sa femme veuve !
\VS{10}Que ses enfants soient entièrement vagabonds, et qu'ils mendient et quêtent en sortant de leurs maisons détruites\FTNT{Job. 20:10.} !
\VS{11}Que le créancier usant d'exaction attrape tout ce qui est à lui et que les étrangers butinent tout son travail !
\VS{12}Qu'il n'y ait personne qui étende sa compassion sur lui, et qu'il n'y ait personne qui ait pitié de ses orphelins !
\VS{13}Que sa postérité soit exposée à être retranchée ; que leur nom soit effacé dans la génération qui le suivra !
\VS{14}Que l'iniquité de ses pères revienne en mémoire à Yahweh, et que le péché de sa mère ne soit point effacé !
\VS{15}Qu'ils soient continuellement devant Yahweh, et qu'il retranche leur mémoire de la terre\FTNT{Ps. 34:17.},
\VS{16}parce qu'il ne s'est point souvenu d'user de miséricorde, mais il a persécuté l'homme affligé et misérable, dont le cœur est brisé, et cela pour le faire mourir !
\VS{17}Puisqu'il aime la malédiction, que la malédiction tombe sur lui ! Puisqu’il ne prends pas plaisir à la bénédiction, que la bénédiction aussi s'éloigne de lui !
\VS{18}Et qu'il soit revêtu de la malédiction comme de sa robe ; qu'elle entre dans son corps comme de l'eau, et dans ses os comme de l'huile !
\VS{19}Qu'elle lui soit comme un vêtement dont il se couvre, et comme une ceinture dont il se ceigne continuellement !
\VS{20}Telle soit, de la part de Yahweh, la récompense de mes adversaires, et de ceux qui parlent mal de moi !
\VS{21}Mais toi, Yahweh, Seigneur, agis avec moi pour l'amour de ton Nom ! Et parce que ta miséricorde est grande, délivre-moi !
\VS{22}Car je suis affligé et misérable, et mon cœur est blessé au-dedans de moi.
\VS{23}Je m'en vais comme l'ombre quand elle décline, et je suis chassé comme une sauterelle.
\VS{24}Mes genoux sont affaiblis par le jeûne, et mon corps est épuisé de maigreur au lieu d’être gras.
\VS{25}Je suis pour eux un objet d’opprobre ; quand ils me voient, ils secouent la tête.
\VS{26}Yahweh, mon Dieu ! Aide-moi, délivre-moi selon ta miséricorde.
\VS{27}Afin qu'on sache que c’est ta main, que c’est toi, ô Yahweh, qui l’as fait.
\VS{28}Ils maudiront, mais tu béniras ; ils s'élèveront, mais ils seront confus ; et ton serviteur se réjouira.
\VS{29}Que mes adversaires soient revêtus de confusion et couverts de leur honte comme d'un manteau.
\VS{30}Je célébrerai hautement de ma bouche Yahweh, et je le louerai au milieu de plusieurs nations.
\VS{31}De ce qu'il se tient à la droite du misérable pour le délivrer de ceux qui condamnent son âme.
\TextTitle{[Le Roi-Sacrificateur]}
\Chap{110}
\VerseOne{}Psaume de David. Yahweh a dit à mon Seigneur : Assieds-toi à ma droite, jusqu’à ce que je fasse de tes ennemis ton marchepied\FTNT{Ce psaume affirme la divinité de Jésus-Christ (Mt. 22:41-46 ; Mc. 12:35-37 ; Lu. 20:41-44 ; Ac. 2:34-35 ; Hé. 1:13 ; Hé. 10:12-13).}.
\VS{2}Yahweh étendra de Sion le sceptre de ta puissance, en disant : Domine au milieu de tes ennemis\FTNT{Es. 2:2-3 ; Da. 7:13.} !
\VS{3}Ton peuple est plein d’ardeur quand tu rassembles ton armée ; avec des ornements sacrés, du sein de l’aurore, ta jeunesse vient à toi comme une rosée.
\VS{4}Yahweh l'a juré, et il ne s'en repentira point : Tu es sacrificateur éternellement, à la manière de Melchisédek\FTNT{Hé. 5:6 ; Hé. 6:20 ; Hé. 7:17 ; Ge. 14:18.}.
\VS{5}Le Seigneur est à ta droite, il brisera les rois au jour de sa colère.
\VS{6}Il exercera le jugement sur les nations, il remplira tout de cadavre ; il brisera le chef qui domine sur un grand pays\FTNT{Ap. 14 ; Ap. 16.}.
\VS{7}Il boit au torrent pendant la marche : C'est pourquoi il lève haut la tête.
\TextTitle{[Les oeuvres magnifiques de Dieu]}
\Chap{111}
\VerseOne{}Louez Yahweh. [Aleph.] Je célébrerai Yahweh de tout mon cœur, [Beth.] dans la compagnie des hommes droits et dans l'assemblée.
\VS{2}[Guimel.] Les œuvres de Yahweh sont grandes, [Daleth.] elles sont recherchées par tous ceux qui y prennent plaisir.
\VS{3}[He.] Son œuvre n'est que majesté et magnificence, [Vau.] et sa justice demeure à perpétuité.
\VS{4}[Zaïn.] Il a rendu ses merveilles mémorables. [Heth.] Yahweh est miséricordieux et compatissant.
\VS{5}[Teth.] Il a donné de la nourriture à ceux qui le craignent ; [Jod.] il s'est souvenu pour toujours de son alliance.
\VS{6}[Caph.] Il a manifesté à son peuple la puissance de ses œuvres, [Lamed.] en leur donnant l'héritage des nations.
\VS{7}[Mem.] Les œuvres de ses mains ne sont que vérité et équité. [Nun.] Tous ses commandements sont véritables,
\VS{8}[Samech.] appuyés à perpétuité, éternellement, [Hajin.] faits avec fidélité et droiture.
\VS{9}[Pe.] Il a envoyé la rédemption à son peuple\FTNT{Ex. 6:6 ; Jn. 3:16.} ; [Tsade.] il lui a donné une alliance éternelle ; [Koph.] son nom est saint et redoutable.
\VS{10}[Resch.] Le commencement de la sagesse c'est la crainte de Yahweh : [Scin.] Tous ceux qui s'adonnent à faire ce qu'elle prescrit sont sages\FTNT{Pr. 1:7 ; Pr. 9:10 ; Pr. 8:13 ; De. 4:6.}. [Thau.] Sa louange demeure à perpétuité.
\TextTitle{[La crainte de Yahweh donne de l'assurance]}
\Chap{112}
\VerseOne{}Louez Yahweh. [Aleph.] Heureux l'homme qui craint Yahweh [Beth.] et qui prend un grand plaisir à ses commandements !
\VS{2}[Guimel.] Sa postérité sera puissante sur la terre, [Daleth.] la génération des hommes droits sera bénie\FTNT{Pr. 20:7.}.
\VS{3}[He.] Il y aura des biens et des richesses dans sa maison ; [Vau.] et sa justice demeure à perpétuité.
\VS{4}[Zaïn.] La lumière s'est levée dans les ténèbres sur ceux qui sont justes\FTNT{Ps. 37:6 ; Pr. 4:18.} ; [Heth.] il est compatissant, miséricordieux et juste.
\VS{5}[Teth.] Heureux l'homme de bien qui exerce la miséricorde et prête, [Jod.] qui règle ses actions avec justice.
\VS{6}[Caph.] Il ne chancelle jamais. [Lamed.] La mémoire du juste dure toujours\FTNT{Pr. 10:7.}.
\VS{7}[Mem.] Il ne craint pas les mauvaises nouvelles ; [Nun.] son cœur est ferme, confiant en Yahweh.
\VS{8}[Samech.] Son cœur est bien affermi, il ne craint pas, [Hajin.] jusqu’à ce qu'il mette son plaisir à regarder ses adversaires.
\VS{9}[Pe.] Il fait des largesses, il donne aux pauvres ; [Tsade.] sa justice demeure à perpétuité ; [Koph.] sa corne s’élève en gloire.
\VS{10}[Resch.] Le méchant le voit et s’irrite. [Scin.] Il grince des dents et se consume ; [Thau.] les désirs des méchants périssent.
\TextTitle{[Yahweh, le Dieu élevé au-dessus de toutes les nations]}
\Chap{113}
\VerseOne{}Louez Yahweh\FTNT{Les psaumes dits «~Alléluia~» sont les Ps. 104 à 106, 111 à 113, 115 à 117, 135 à 136, 146 à 150. Parmi eux, les psaumes 135 et 146 à 150, étaient chantés durant le service quotidien d’adoration dans la synagogue. Les psaumes 115 à 118, appelés «~le grand Hallel~», étaient chantés lors des fêtes de Pâque. Alléluia veut dire «~Louez Yahweh~» (Ap. 19:1).} ! Louez, vous serviteurs de Yahweh, louez le Nom de Yahweh !
\VS{2}Que le Nom de Yahweh soit béni dès maintenant et à toujours !
\VS{3}Le Nom de Yahweh est digne de louanges depuis le soleil levant jusqu’au soleil couchant.
\VS{4}Yahweh est élevé par-dessus toutes les nations, sa gloire est au-dessus des cieux.
\VS{5}Qui est semblable à Yahweh notre Dieu, qui habite dans les lieux très hauts ?
\VS{6}Il s'abaisse pour regarder sur le ciel et sur la terre,
\VS{7}de la poussière il retire le pauvre, du fumier il relève l’indigent\FTNT{Ps. 107:41 ; 1 S. 2:8.},
\VS{8}pour le faire asseoir avec les nobles, avec les nobles de son peuple\FTNT{Job. 36:7.}.
\VS{9}Il donne une maison à la femme stérile, il en fait une mère joyeuse au milieu de ses fils\FTNT{1 S. 2:5 ; Ps. 68:6 ; Ge. 17:17-21.}. Louez Yahweh !
\TextTitle{[La terre tremble devant le Dieu Tout-Puissant]}
\Chap{114}
\VerseOne{}Quand Israël sortit d'Egypte, quand la maison de Jacob s’éloigna d’un peuple barbare,
\VS{2}Juda devint son lieu saint, Israël son domaine\FTNT{Jé. 2:2-3.}.
\VS{3}La mer le vit et s'enfuit, le Jourdain retourna en arrière\FTNT{Ps. 77:17 ; Jos. 3:13-16.}.
\VS{4}Les montagnes sautèrent comme des béliers, les collines comme des agneaux\FTNT{Ps. 68:9 ; Jg. 5:5 ; Ha. 3:10.}.
\VS{5}Ô Mer ! Qu’avais-tu pour t'enfuir ? Jourdain, pour retourner en arrière ?
\VS{6}Et vous montagnes, pour sauter comme des béliers ? Et vous collines, comme des agneaux ?
\VS{7}Ô Terre ! Tremble devant la présence du Seigneur, devant la présence du Dieu de Jacob,
\VS{8}qui a changé le rocher en un étang d'eaux, la pierre très dure en une source d'eaux.
\TextTitle{[A Dieu seul la gloire]}
\Chap{115}
\VerseOne{}Non point à nous, ô Yahweh ! Non point à nous, mais à ton Nom donne gloire, à cause de ta bonté, à cause de ta fidélité !
\VS{2}Pourquoi les nations diraient-elles : Où est maintenant leur Dieu ?
\VS{3}Certes notre Dieu est au ciel, il fait tout ce qu'il veut\FTNT{Ps. 135:6 ; Job. 23:13.}.
\VS{4}Leurs idoles sont des dieux d’or et d'argent, elles sont l’ouvrage de mains d'homme.
\VS{5}Elles ont une bouche, et ne parlent point ; elles ont des yeux, et ne voient point ;
\VS{6}elles ont des oreilles, et n'entendent point ; elles ont un nez, et ne sentent point ;
\VS{7}elles ont des mains, et elles ne touchent point ; elles ont des pieds, et elles ne marchent point ; et elles ne rendent aucun son de leur gosier\FTNT{Ex. 32:2-8 ; Es. 44:9 ; Ez. 8:8-12 ; 1 R. 18:25-26.}.
\VS{8}Ils leur ressemblent, ceux qui les fabriquent, tous ceux qui se confient en eux.
\VS{9}Israël confie-toi en Yahweh ; il est leur secours et leur bouclier de ceux qui se confient en lui.
\VS{10}Maison d'Aaron, confie-toi en Yahweh ; il est leur secours et leur bouclier.
\VS{11}Vous qui craignez Yahweh, confiez-vous en Yahweh ; il est leur secours et leur bouclier.
\VS{12}Yahweh s'est souvenu de nous, il bénira, il bénira la maison d'Israël, il bénira la maison d'Aaron.
\VS{13}Il bénira ceux qui craignent Yahweh, tant les petits que les grands.
\VS{14}Yahweh vous multipliera ses bénédictions, à vous et à vos fils.
\VS{15}Vous êtes bénis de Yahweh, qui a fait les cieux et la terre.
\VS{16}Les cieux, sont les cieux de Yahweh, mais il a donné la terre aux fils des hommes.
\VS{17}Ce ne sont pas les morts qui célèbrent Yahweh, ce n’est aucun de ceux qui descendent dans le lieu du silence\FTNT{s. 6:6 ; Ps. 88:11 ; Es. 38:18-19.}.
\VS{18}Mais nous, nous bénirons Yahweh dès maintenant et pour toujours. Louez Yahweh !
\TextTitle{[La reconaissance des rachetés]}
\Chap{116}
\VerseOne{}J'aime Yahweh, car il a entend ma voix et mes supplications.
\VS{2}Car il a incliné son oreille vers moi, c'est pourquoi je l'invoquerai durant mes jours.
\VS{3}Les liens de la mort m'avaient environné, et les angoisses du scheol m'avaient trouvé\FTNT{Ps. 18:5 ; 2 S. 22:5.} ; j'avais trouvé la détresse et la douleur.
\VS{4}Mais j'invoquai le Nom de Yahweh en disant : Je te prie, délivre mon âme, ô Yahweh !
\VS{5}Yahweh est compatissant et juste, et notre Dieu fait miséricorde.
\VS{6}Yahweh garde les simples ; j'étais devenu misérable et il m'a sauvé.
\VS{7}Mon âme, retourne dans ton repos, car Yahweh t'a fait du bien.
\VS{8}Parce que tu as retiré mon âme de la mort, mes yeux des larmes et mes pieds de la chute,
\VS{9}je marcherai dans la présence de Yahweh, sur la terre des vivants.
\VS{10}J'ai cru, c'est pourquoi j'ai parlé\FTNT{2 Co. 4:13.} ; j'ai été fort affligé.
\VS{11}Je disais dans ma précipitation : Tout homme est menteur\FTNT{Ro. 3:4.}.
\VS{12}Que rendrai-je à Yahweh ? Tous ses bienfaits envers moi ?
\VS{13}J’élèverai la coupe des délivrances et j'invoquerai le Nom de Yahweh.
\VS{14}J’accomplirai maintenant mes vœux à Yahweh, devant tout son peuple.
\VS{15}Elle a du prix aux yeux de Yahweh, la mort de ceux qu’il aime.
\VS{16}Ecoute-moi, ô Yahweh ! Car je suis ton serviteur, je suis ton serviteur, fils de ta servante. Tu as délié mes liens.
\VS{17}Je t’offrirai le sacrifice de remerciement et j'invoquerai le Nom de Yahweh.
\VS{18}J’accomplirai maintenant mes vœux à Yahweh, devant tout son peuple,
\VS{19}dans les parvis de la maison de Yahweh, au milieu de toi, Jérusalem ! Louez Yahweh !
\TextTitle{[Toutes les nations louent Dieu]}
\Chap{117}
\VerseOne{}Toutes les nations, louez Yahweh ! Tous les peuples, célébrez-le !
\VS{2}Car sa miséricorde est grande envers nous, et sa fidélité dure à toujours. Louez Yahweh !
\TextTitle{[Louange à Yahweh, la pierre principale de l'angle]}
\Chap{118}
\VerseOne{}Célébrez Yahweh, car il est bon, parce que sa bonté dure à toujours !
\VS{2}Qu'Israël dise maintenant : Car sa bonté dure à toujours !
\VS{3}Que la maison d'Aaron dise maintenant : Car sa bonté dure à toujours !
\VS{4}Que ceux qui craignent Yahweh disent maintenant : Car sa bonté dure à toujours !
\VS{5}Me trouvant dans la détresse, j'ai invoqué Yahweh\FTNT{Ps. 120:1.} ; et Yahweh m'a répondu et m'a mis au large.
\VS{6}Yahweh est pour moi, je ne craindrai point. Que me ferait l'homme ?
\VS{7}Yahweh est pour moi parmi ceux qui me secourent, c’est pourquoi je verrai en ceux qui me haïssent ce que je désire.
\VS{8}Mieux vaut se confier en Yahweh que se confier en l'homme\FTNT{Ps. 62:9 ; Es. 2:22 ; Jé. 17:5.}.
\VS{9}Mieux vaut se confier en Yahweh que se reposer sur les grands d'entre les peuples.
\VS{10}Toutes les nations m'avaient environné, mais au Nom de Yahweh je les taille en pièces.
\VS{11}Elles m'avaient environné, elles m'avaient, dis-je, environné ; mais au Nom de Yahweh je les taille en pièces.
\VS{12}Elles m'avaient environné comme des abeilles, elles s’éteignent comme un feu d'épines\FTNT{De. 1:44.}, car au Nom de Yahweh je les taille en pièces.
\VS{13}Tu me poussais violemment pour me faire tomber, mais Yahweh m'a secouru.
\VS{14}Yahweh est ma force et le sujet de mes louanges, et il a été ma délivrance\FTNT{Ex. 15:2 ; Es. 12:2.}.
\VS{15}Une voix de chant de triomphe et de délivrance retentit dans les tentes des justes : La droite de Yahweh exerce la puissance !
\VS{16}La droite de Yahweh est élevée, la droite de Yahweh exerce sa puissance.
\VS{17}Je ne mourrai pas, je vivrai et je raconterai les œuvres de Yahweh.
\VS{18}Yahweh m'a châtié sévèrement, mais il ne m'a point livré à la mort.
\VS{19}Ouvrez-moi les portes de la justice ; j'y entrerai et je célébrerai Yahweh.
\VS{20}C'est ici la porte de Yahweh, les justes y entreront.
\VS{21}Je te célébrerai parce que tu m'as exaucé et tu as été mon libérateur.
\VS{22}La Pierre que les architectes avaient rejetée, est devenue la principale de l’angle\FTNT{Le Messie est présenté comme la pierre ou le rocher. 
Es. 8:13-17 ; 1 Pi. 2:7.}.
\VS{23}Ceci a été fait par Yahweh, c’est un prodige à nos yeux.
\VS{24}C'est ici la journée que Yahweh a faite, qu’elle soit pour nous un sujet d’allégresse et de joie.
\VS{25}Yahweh, je te prie, délivre maintenant. Yahweh, je te prie, donne maintenant la prospérité !
\VS{26}Béni soit celui qui vient au Nom de Yahweh ! Nous vous bénissons de la maison de Yahweh.
\VS{27}Yahweh est Dieu, et il nous a éclairés. Liez avec des cordes la bête du sacrifice, et amenez-la jusqu’aux cornes de l'autel.
\VS{28}Tu es mon Dieu, c'est pourquoi je te célébrerai. Tu es mon Dieu, je t'exalterai.
\VS{29}Célébrez Yahweh car il est bon, parce que sa miséricorde demeure à toujours !
\TextTitle{[Marcher et examiner ses voies d'après la Parole de Yahweh]}
\Chap{119}
\VerseOne{}[Aleph.] Heureux ceux qui sont intègres dans leur voie, qui marchent selon la loi de Yahweh.
\VS{2}Heureux sont ceux qui gardent ses préceptes et qui le cherchent de tout leur cœur\FTNT{Jos. 1:8.} ;
\VS{3}qui ne font point d'iniquité, qui marchent dans ses voies\FTNT{1 Jn. 3:9 ; 1 Jn. 5 :18.}.
\VS{4}Tu as donné tes commandements afin qu'on les garde soigneusement.
\VS{5}Oh ! Que mes voies soient bien établies pour garder tes statuts !
\VS{6}Et je ne rougirai point de honte quand je regarderai à tous tes commandements.
\VS{7}Je te célébrerai avec droiture de cœur quand j'aurai appris les ordonnances de ta justice.
\VS{8}Je veux garder tes statuts, ne me délaisse point entièrement.
\VS{9}[Beth.] Par quel moyen le jeune homme rendra-t-il pure sa voie ? Ce sera en y prenant garde selon ta parole.
\VS{10}Je te recherche de tout mon cœur, ne me laisse pas m’égarer loin de tes commandements.
\VS{11}Je serre ta parole dans mon cœur afin de ne pas pécher contre toi.
\VS{12}Yahweh ! Tu es béni ; enseigne-moi tes statuts.
\VS{13}De mes lèvres je raconte toutes les ordonnances de ta bouche.
\VS{14}Je me réjouis dans le chemin de tes préceptes comme si je possédais toutes les richesses du monde.
\VS{15}Je médite tes commandements et j’observe tes voies.
\VS{16}Je prends plaisir à tes statuts et je n'oublie pas tes paroles.
\VS{17}[Guimel.] Fais du bien à ton serviteur afin que je vive, et je garderai ta parole\FTNT{Ps. 116:7.}.
\VS{18}Ouvre mes yeux afin que je regarde aux merveilles de ta loi\FTNT{Ep. 1:18.} !
\VS{19}Je suis voyageur sur la terre, ne me cache pas tes commandements.
\VS{20}Mon âme est brisée par le désir qui toujours me porte vers tes ordonnances.
\VS{21}Tu réprimandes les orgueilleux, ces maudits, qui se détournent de tes commandements.
\VS{22}Décharge-moi de l'opprobre et du mépris, car j'ai gardé tes préceptes\FTNT{Ps. 3:9.}.
\VS{23}Même les princes s’assoient et parlent contre moi pendant que ton serviteur médite tes statuts.
\VS{24}Tes préceptes font mes délices, ce sont mes conseillers.
\VS{25}[Daleth.] Mon âme est attachée à la poussière, fais-moi revivre selon ta parole\FTNT{Ps. 44:26 ; Ps. 143:11.}.
\VS{26}Je te raconte mes voies et tu me réponds ; enseigne-moi tes statuts.
\VS{27}Fais-moi entendre la voie de tes commandements, et je parlerai de tes merveilles\FTNT{Ps. 145:6.}.
\VS{28}Mon âme pleure de chagrin, relève-moi selon tes paroles.
\VS{29}Eloigne de moi la voie du mensonge et accorde–moi la grâce d’observer ta loi\FTNT{Le mot "loi" vient de l'hebreu "towrah" qui donne "Thora" en français}.
\VS{30}Je choisis la voie de la vérité et je place tes ordonnances sous mes yeux.
\VS{31}Je m’attache à tes préceptes, ô Yahweh ! Ne me fais point rougir de honte.
\VS{32}Je cours dans la voie de tes commandements car tu élargis mon cœur.
\VS{33}[He.] Yahweh, enseigne-moi la voie de tes statuts, et je la garderai jusqu’au bout.
\VS{34}Donne-moi de l'intelligence ; je garderai ta loi et je l'observerai de tout mon cœur\FTNT{Pr. 2:6 ; Ja. 1:5.}.
\VS{35}Fais-moi marcher sur le sentier de tes commandements car j'y prends plaisir.
\VS{36}Incline mon cœur à tes préceptes et non point au profit\FTNT{Hé. 13:5 ; Mc 7:21-22 ; Ez. 33:31.}.
\VS{37}Détourne mes yeux de la vue des choses vaines ; fais-moi vivre dans ta voie.
\VS{38}Accomplis ta parole envers ton serviteur, parole qui est pour ceux qui te craignent.
\VS{39}Eloigne de moi l’opprobre que je redoute, car tes ordonnances sont bonnes.
\VS{40}Voici, je désire pratiquer tes commandements, fais-moi vivre dans ta justice.
\VS{41}[Vau.] Que ta miséricorde vienne sur moi, ô Yahweh ! Et ta délivrance aussi, selon ta promesse !
\VS{42}Et je pourrai répondre à celui qui m’outrage, car je me confie en ta parole.
\VS{43}N'arrache pas de ma bouche la parole de vérité, car j’espère en tes jugements.
\VS{44}Je garderai continuellement ta loi, à toujours et à perpétuité.
\VS{45}Je marcherai au large parce que je recherche tes commandements.
\VS{46}Je parlerai de tes préceptes devant les rois et je ne rougirai pas de honte\FTNT{Ps. 138:1-4 ; Mt. 10:18-19 ; Ac. 26.}.
\VS{47}Je fais mes délices de tes commandements que j’aime ;
\VS{48}j'étends mes mains vers tes commandements que j'aime ; et je médite tes statuts.
\VS{49}[Zain.] Souviens-toi de la parole donnée à ton serviteur, sur laquelle tu m’as fait espérer.
\VS{50}C'est ici ma consolation dans mon affliction, car ta parole me rend la vie.
\VS{51}Les orgueilleux se sont fort moqués de moi, mais je ne me suis pas détourné de ta loi.
\VS{52}Yahweh, je me souviens de tes jugements anciens et je me suis consolé en eux.
\VS{53}L'horreur me saisit à cause des méchants qui abandonnent ta loi.
\VS{54}Tes statuts sont le sujet de mes cantiques dans la maison où je suis étranger.
\VS{55}Yahweh, je me souviens de ton Nom pendant la nuit et je garde ta loi.
\VS{56}Cela m'arrive parce que je garde tes commandements.
\VS{57}[Heth.] Ô Yahweh ! J’en conclus que ma part est de garder tes paroles.
\VS{58}Je te supplie de tout mon cœur : Aie pitié de moi selon ta parole.
\VS{59}Je fais le compte de mes voies et je rebrousse chemin vers tes préceptes\FTNT{La. 3:40 ; Os. 6:3.}.
\VS{60}Je me hâte, je ne diffère point de garder tes commandements.
\VS{61}Une compagnie de méchants me pille, mais je n'oublie pas ta loi.
\VS{62}Je me lève au milieu de la nuit pour te célébrer à cause des ordonnances de ta justice.
\VS{63}Je suis l’ami de tous ceux qui te craignent et qui gardent tes commandements.
\VS{64}Yahweh, la terre est pleine de ta bonté ; enseigne-moi tes statuts.
\VS{65}[Teth.] Yahweh, tu fais du bien à ton serviteur selon ta parole.
\VS{66}Enseigne-moi le bon sens et la connaissance car je crois à tes commandements.
\VS{67}Avant d’avoir été humilié, je m’égarais, mais maintenant j'observe ta parole.
\VS{68}Tu es bon et bienfaisant, enseigne-moi tes statuts.
\VS{69}Les orgueilleux imaginent des faussetés contre moi, mais je garde de tout mon cœur tes commandements.
\VS{70}Leur cœur est insensible comme la graisse, mais moi, je prends plaisir dans ta loi\FTNT{De. 32:15 ; Jé. 5:28.}.
\VS{71}Il est bon que je sois humilié afin que j'apprenne tes statuts.
\VS{72}La loi que tu as prononcée de ta bouche m'est plus précieuse que mille pièces d'or ou d'argent\FTNT{Ps. 19:10-11 ; Job. 22:2.}.
\VS{73}[Jod.] Tes mains m'ont façonné, elles m’ont formé\FTNT{Jé. 1:5 ; Job. 10:9 ; Ps. 137:15-16.} ; donne-moi l’intelligence afin que j'apprenne tes commandements.
\VS{74}Ceux qui te craignent me verront et se réjouiront, parce que j’espère en tes promesses.
\VS{75}Je reconnais, ô Yahweh, que tes jugements sont justes, et que tu m’as humilié par ta fidélité\FTNT{Hé. 12:10.}.
\VS{76}Que ta bonté soit ma consolation, comme tu l’as promis à ton serviteur.
\VS{77}Que tes compassions viennent sur moi et je vivrai ; car ta loi fait mes délices.
\VS{78}Que les orgueilleux rougissent de honte, de ce qu’ils m’oppriment sans cause ; mais moi, je médite sur tes ordonnances.
\VS{79}Que ceux qui te craignent et ceux qui connaissent tes préceptes reviennent vers moi.
\VS{80}Que mon cœur soit intègre dans tes statuts afin que je ne sois pas couvert de honte.
\VS{81}[Caph.] Mon âme se consume en attendant ta délivrance ; j’espère en ta promesse.
\VS{82}Mes yeux s’épuisent en attendant ta promesse, lorsque je dis : Quand me consoleras-tu ?
\VS{83}Car je suis comme une outre dans la fumée, je n'oublie pas tes statuts.
\VS{84}Quel est le nombre de jours de ton serviteur ? Quand jugeras-tu ceux qui me poursuivent\FTNT{Ap. 6:10.} ?
\VS{85}Les orgueilleux me creusent des fosses, ils n’agissent pas selon ta loi.
\VS{86}Tous tes commandements ne sont que fidélité ; on me persécute sans cause, aide-moi\FTNT{Mt. 5:10.} !
\VS{87}On m'a presque réduit à rien et mis par terre ; mais je n'ai point abandonné tes commandements.
\VS{88}Fais-moi revivre selon ta miséricorde et je garderai les préceptes de ta bouche.
\VS{89}[Lamed.] Ô Yahweh ! Ta parole subsiste à toujours dans les cieux.
\VS{90}Ta fidélité dure d'âge en âge ; tu as établi la terre, et elle demeure ferme\FTNT{Pr. 1:4.}.
\VS{91}Ces choses subsistent aujourd'hui selon tes ordonnances, car toutes choses te servent.
\VS{92}Si ta loi n’avait pas fait mes délices, j’aurais déjà péri dans mon affliction.
\VS{93}Je n'oublierai jamais tes commandements car c’est par eux que tu m'as fait revivre.
\VS{94}Je suis à toi, sauve-moi ; car je recherche tes commandements.
\VS{95}Les méchants m'attendent pour me faire périr, mais je suis attentif à tes préceptes.
\VS{96}Je vois des bornes à tout ce qui est parfait, mais tes commandements n’ont point de limites.
\VS{97}[Mem.] Combien j'aime ta loi\FTNT{Ps. 1:2.} ! Elle est tout le jour l’objet de ma méditation.
\VS{98}Par tes commandements, tu m'as rendu plus sage que mes ennemis, parce que tes commandements sont toujours avec moi.
\VS{99}J'ai surpassé en prudence tous ceux qui m'avaient enseigné parce que tes préceptes sont l’objet de ma méditation.
\VS{100}Je suis devenu plus intelligent que les vieillards parce que j'observe tes commandements.
\VS{101}Je garde mes pieds de toute mauvaise voie afin d’observer ta parole.
\VS{102}Je ne me suis point détourné de tes ordonnances parce que tu me les enseignes.
\VS{103}Que ta parole est douce à mon palais ! Plus douce que le miel à ma bouche.
\VS{104}Je suis devenu intelligent par tes commandements, c'est pourquoi je hais toute voie de mensonge.
\VS{105}[Nun.] Ta parole est une lampe à mes pieds et une lumière sur mon sentier\FTNT{Pr. 6:23 ; 2 Pi. 1:19.}.
\VS{106}J'ai juré et je le tiendrai, d'observer les lois de ta justice\FTNT{Né. 10:29.}.
\VS{107}Yahweh, je suis extrêmement affligé, fais-moi revivre selon ta parole.
\VS{108}Yahweh, je te prie, agrée les sentiments que ma bouche exprime, et enseigne-moi tes ordonnances\FTNT{Os. 14:2 ; Hé. 13:15.}.
\VS{109}Ma vie est continuellement en danger, toutefois je n'oublie pas ta loi.
\VS{110}Les méchants m'ont tendu des pièges, toutefois je ne me suis point égaré de tes commandements.
\VS{111}J'ai pris pour héritage perpétuel tes préceptes car ils sont la joie de mon cœur.
\VS{112}J'ai incliné mon cœur à accomplir toujours tes statuts jusqu’au bout.
\VS{113}[Samech.] Je hais les hommes indécis\FTNT{1 R. 18:21 ; Ja. 1:6 ; Ja. 4:8.}, mais j'aime ta loi.
\VS{114}Tu es mon refuge et mon bouclier, je m’attends à ta parole.
\VS{115}Méchants, retirez-vous de moi\FTNT{Mt. 7:23 ; Ps. 6:9.} ! Et je garderai les commandements de mon Dieu.
\VS{116}Soutiens-moi suivant ta parole, et je vivrai ; et ne me fais point rougir de honte en me refusant ce que j'espérais.
\VS{117}Soutiens-moi, et je serai en sûreté ; et j'aurai continuellement les yeux sur tes statuts.
\VS{118}Tu as foulé aux pieds tous ceux qui se détournent de tes statuts, car le mensonge est le moyen dont ils se servent pour tromper.
\VS{119}Tu réduis à néant tous les méchants de la terre, comme de l’écume ; c'est pourquoi j'aime tes préceptes.
\VS{120}Ma chair frissonne de l’effroi que tu m’inspires et je crains tes jugements\FTNT{Ha. 3:16.}.
\VS{121}[Hajin.] J'ai exercé le jugement et la justice, ne m'abandonne pas à ceux qui me font tort.
\VS{122}Prends sous ta garantie le bien de ton serviteur et ne permets pas que je sois opprimé par les orgueilleux.
\VS{123}Mes yeux s’épuisent en attendant ta délivrance et la parole de ta justice.
\VS{124}Agis envers ton serviteur suivant ta miséricorde et enseigne-moi tes statuts.
\VS{125}Je suis ton serviteur, donne-moi l’intelligence, et je connaîtrai tes préceptes\FTNT{Pr. 1:4 ; Pr. 6:23.}.
\VS{126}Il est temps que Yahweh opère ; ils ont aboli ta loi.
\VS{127}C'est pourquoi j'aime tes commandements, plus que l'or et l’or fin.
\VS{128}C'est pourquoi je trouve justes tous tes commandements, je hais toute voie de mensonge.
\VS{129}[Pe.] Tes préceptes sont merveilleux, c'est pourquoi mon âme les garde.
\VS{130}La révélation de tes paroles éclaire, elle donne de l'intelligence aux simples.
\VS{131}J'ouvre ma bouche et je soupire, car je désire tes commandements.
\VS{132}Regarde-moi, et aie pitié de moi, selon tes jugements à l’égard de ceux qui aiment ton Nom.
\VS{133}Affermis mes pas sur ta parole, et que l'iniquité n'ait point d'emprise sur moi.
\VS{134}Délivre-moi de l'oppression des hommes afin que je garde tes commandements.
\VS{135}Fais luire ta face sur ton serviteur et enseigne-moi tes statuts.
\VS{136}Mes yeux répandent des torrents d'eau parce qu'on n'observe point ta loi.
\VS{137}[Tsade.] Tu es juste, ô Yahweh, et droit dans tes jugements.
\VS{138}Tu ordonnes tes préceptes avec justice et grande fidélité.
\VS{139}Mon zèle me consume parce que mes adversaires oublient tes paroles.
\VS{140}Ta parole est entièrement éprouvée, c'est pourquoi ton serviteur l'aime.
\VS{141}Je suis petit et méprisé, toutefois je n'oublie point tes commandements.
\VS{142}Ta justice est une justice éternelle, et ta loi est la vérité.
\VS{143}La détresse et l'angoisse m’atteignent, mais tes commandements font mes délices.
\VS{144}Tes préceptes ne sont que justice éternelle ; donne-moi l’intelligence afin que je vive.
\VS{145}[Koph.] Je crie de tout mon cœur, réponds-moi, ô Yahweh ! Je garde tes statuts.
\VS{146}Je crie vers toi, sauve-moi afin que j'observe tes préceptes.
\VS{147}Je devance l’aurore et je crie ; je m’attends à ta parole.
\VS{148}Mes yeux ont devancé les veilles de la nuit pour méditer ta parole.
\VS{149}Ecoute ma voix selon ta miséricorde, ô Yahweh ! Fais-moi revivre selon ton ordonnance.
\VS{150}Ceux qui poursuivent le crime s’approchent de moi, et ils s’éloignent de ta loi.
\VS{151}Yahweh, tu es près de moi ; et tous tes commandements ne sont que vérité.
\VS{152}Depuis longtemps, je sais par tes préceptes, que tu les as établis pour toujours.
\VS{153}[Resch.] Regarde mon affliction et sauve-moi, car je n'oublie pas ta loi.
\VS{154}Soutiens ma cause et rachète-moi ; fais-moi revivre selon ta parole.
\VS{155}La délivrance est loin des méchants parce qu'ils ne recherchent pas tes statuts.
\VS{156}Tes compassions sont en grand nombre, ô Yahweh ! Fais-moi revivre selon tes ordonnances.
\VS{157}Ceux qui me persécutent et qui me pressent sont en grand nombre, toutefois je ne me détourne pas de tes préceptes.
\VS{158}Je vois avec dégoût les traîtres et je suis rempli de tristesse car ils n'observent pas ta parole.
\VS{159}Regarde combien j'aime tes commandements, Yahweh ! Fais-moi revivre selon ta miséricorde !
\VS{160}Le fondement de ta parole est la vérité, et toutes les lois de ta justice sont éternelles.
\VS{161}[Scin.] Les princes du peuple me persécutent sans cause, mais mon cœur tremble à cause de ta parole.
\VS{162}Je me réjouis de ta parole comme ferait celui qui aurait trouvé un grand butin.
\VS{163}J’ai en haine et en abomination le mensonge ; j’aime ta loi.
\VS{164}Sept fois le jour je te loue à cause des ordonnances de ta justice.
\VS{165}Il y a une grande paix pour ceux qui aiment ta loi, et rien ne peut les renverser\FTNT{Ph. 4:7 ; Es. 32:17.}.
\VS{166}Yahweh, j'espère en ta délivrance et je pratique tes commandements.
\VS{167}Mon âme observe tes préceptes, je les aime beaucoup.
\VS{168}J'observe tes commandements et tes préceptes, car toutes mes voies sont devant toi.
\VS{169}[Thau.] Yahweh, que mon cri parvienne jusqu’à toi, donne-moi l’intelligence selon ta parole.
\VS{170}Que ma supplication vienne devant toi, délivre-moi selon ta parole.
\VS{171}Mes lèvres publieront ta louange quand tu m'auras enseigné tes statuts.
\VS{172}Ma langue ne s'entretiendra que de ta parole, parce que tous tes commandements ne sont que justice.
\VS{173}Que ta main me soit en aide, parce que j'ai choisi tes commandements.
\VS{174}Yahweh, je souhaite ta délivrance, et ta loi fait mes délices.
\VS{175}Que mon âme vive afin qu'elle te loue, et que tes ordonnances me soient en aide !
\VS{176}Je suis errant comme une brebis perdue\FTNT{Es. 53:6 ; 1 Pi. 2:25 ; Lu. 15:4.} ; cherche ton serviteur, car je n'oublie pas tes commandements.
\TextTitle{[Dans la détresse, crier à Yahweh]}
\Chap{120}
\VerseOne{}Cantique des degrés\FTNT{Les psaumes 120 à 134 sont appelés «~psaumes des degrés~» ou de «~l’ascension~». Ces psaumes furent chantés par les Israélites montant à Jérusalem au retour de la captivité de Babylone.}. J'ai invoqué Yahweh dans ma grande détresse, et il m'a exaucé.
\VS{2}Yahweh, délivre mon âme des lèvres mensongères et de la langue trompeuse.
\VS{3}Que te donne, que te rapporte la langue trompeuse ?
\VS{4}Ce sont des flèches aiguës tirées par un homme puissant et des charbons ardents du genêt\FTNT{Jé. 9:3 ; Ja. 3:5-6.}.
\VS{5}Malheureux que je suis de séjourner à Méschec et de demeurer aux tentes de Kédar !
\VS{6}Assez longtemps mon âme a demeuré auprès de ceux qui haïssent la paix !
\VS{7}Je ne cherche que la paix, mais lorsque j'en parle, ils sont pour la guerre.
\TextTitle{[Yahweh, le Dieu qui ne sommeille ni ne dort]}
\Chap{121}
\VerseOne{}Cantique des degrés. J'élève mes yeux vers les montagnes, d'où me viendra le secours.
\VS{2}Mon secours vient de Yahweh qui a fait les cieux et la terre\FTNT{Ps. 124:8.}.
\VS{3}Il ne permettra point que ton pied chancelle, celui qui te garde ne sommeillera point\FTNT{Es. 27:3 ; Pr. 3:23.}.
\VS{4}Voici, il ne sommeille ni ne dort celui qui garde Israël.
\VS{5}Yahweh est celui qui te garde, Yahweh est ton ombre à ta main droite\FTNT{Es. 25:4.}.
\VS{6}Pendant le jour, le soleil ne te frappera point, ni la lune pendant la nuit\FTNT{Es. 49:10. Ap. 7:16.}.
\VS{7}Yahweh te gardera de tout mal, il gardera ton âme.
\VS{8}Yahweh gardera ton départ et ton arrivée, dès maintenant et à jamais\FTNT{De. 28:6.}.
\TextTitle{[Dieu garde Jérusalem]}
\Chap{122}
\VerseOne{}Cantique des degrés de David. Je me réjouis à cause de ceux qui me disent : Allons à la maison de Yahweh\FTNT{Ps. 84:1-5.} !
\VS{2}Nos pieds s’arrêtent dans tes portes, ô Jérusalem !
\VS{3}Jérusalem, qui est bâtie comme une ville dont les édifices sont joints ensemble,
\VS{4}à laquelle montent les tribus, les tribus de Yahweh, selon le témoignage d’Israël, pour célébrer le Nom de Yahweh.
\VS{5}Car c’est là qu’ont été posés les trônes pour juger\FTNT{Mt. 19:28.}. Les trônes de la maison de David.
\VS{6}Demandez la paix de Jérusalem ; que ceux qui t'aiment jouissent du repos.
\VS{7}Que la paix soit dans tes murs et la tranquillité dans tes palais.
\VS{8}Pour l'amour de mes frères et de mes amis, je prie maintenant pour ta paix.
\VS{9}A cause de la maison de Yahweh notre Dieu, je fais une requête pour ton bonheur.
\TextTitle{[Lever les yeux vers Yahweh]}
\Chap{123}
\VerseOne{}Cantique des degrés. J'élève mes yeux vers toi qui habites dans les cieux.
\VS{2}Voici, comme les yeux des serviteurs regardent la main de leurs maîtres, comme les yeux de la servante regardent la main de sa maîtresse, ainsi nos yeux regardent à Yahweh notre Dieu, jusqu’à ce qu'il ait pitié de nous\FTNT{Ps. 25:15.}.
\VS{3}Aie pitié de nous, ô Yahweh ! Aie pitié de nous ! Car nous sommes assez rassasiés de mépris !
\VS{4}Notre âme est assez rassasiée des moqueries des orgueilleux, du mépris des hautains.
\TextTitle{[Le secours est dans le nom de Yahweh]}
\Chap{124}
\VerseOne{}Cantique des degrés, de David. Sans Yahweh, qui nous protégea, qu’Israël le dise !
\VS{2}Sans Yahweh, qui nous protégea, quand les hommes s’élevèrent contre nous ?
\VS{3}Ils nous auraient engloutis tous vivants quand leur colère s’enflamma contre nous.
\VS{4}Alors les eaux nous auraient submergés, les torrents auraient passé sur notre âme.
\VS{5}Alors les flots impétueux auraient passé sur notre âme.
\VS{6}Béni soit Yahweh qui ne nous a point livrés en proie à leurs dents !
\VS{7}Notre âme s’est échappée comme l'oiseau du filet des oiseleurs ; le filet a été rompu, et nous nous sommes échappés\FTNT{Pr. 6:5.}.
\VS{8}Notre secours est dans le Nom Yahweh\FTNT{Ac. 4:11-12.} qui a fait les cieux et la terre.
\TextTitle{[Qui se confie en Yahweh ne chancelle pas]}
\Chap{125}
\VerseOne{}Cantique des degrés. Ceux qui se confient en Yahweh sont comme la montagne de Sion : Elle ne chancelle point et est affermie pour toujours.
\VS{2}Quant à Jérusalem, il y a des montagnes autour d'elle, ainsi Yahweh entoure son peuple, dès maintenant et à jamais.
\VS{3}Car la verge de la méchanceté ne restera pas sur le lot des justes, de peur que les justes n’étendent leurs mains vers l'iniquité\FTNT{Es. 14:5.}.
\VS{4}Yahweh, répands tes bienfaits sur les bons et sur ceux dont le cœur est droit.
\VS{5}Mais ceux qui s’engagent dans des voies détournées, que Yahweh les fasse marcher avec les ouvriers d'iniquité\FTNT{Mt. 7:23.}. La paix sera sur Israël.
\TextTitle{[Souvenir des bénédictions passées]}
\Chap{126}
\VerseOne{}Cantique des degrés. Quand Yahweh ramena les captifs de Sion, nous étions comme ceux qui font un rêve.
\VS{2}Alors notre bouche était remplie de joie, et notre langue de chants de triomphe, alors on disait parmi les nations : Yahweh a fait de grandes choses pour eux !
\VS{3}Yahweh a fait de grandes choses pour nous ; nous sommes dans la joie.
\VS{4}Ô Yahweh ! Ramène nos captifs, comme des ruisseaux dans le midi\FTNT{Os. 6:11 ; Joë. 3:11.} !
\VS{5}Ceux qui sèment avec larmes moissonneront avec chants d’allégresse\FTNT{Ga. 6:9.}.
\VS{6}Celui qui marche en pleurant quand il porte la semence pour la mettre en terre, revient avec des chants d’allégresse quand il porte ses gerbes.
\TextTitle{[Le meilleur batisseur]}
\Chap{127}
\VerseOne{}Cantique des degrés, de Salomon. Si Yahweh ne bâtit la maison, ceux qui la bâtissent travaillent en vain ; si Yahweh ne garde la ville, celui qui la garde fait le guet en vain.
\VS{2}C'est en vain que vous vous levez de grand matin, que vous vous couchez tard, et que vous mangez le pain de douleurs ; certes c'est Dieu qui donne du repos à celui qu'il aime\FTNT{Mc. 2:27 ; Ez. 20:20.}.
\VS{3}Voici, les fils sont un héritage donné par Yahweh et le fruit du ventre est une récompense de Dieu\FTNT{Ps. 113:9 ; Ps. 128:3-6.}.
\VS{4}Telles sont les flèches dans la main d'un homme puissant, tels sont les fils de la jeunesse.
\VS{5}Heureux l'homme qui en a rempli son carquois ! Ils ne seront pas honteux quand ils parleront avec leurs ennemis à la porte.
\TextTitle{[Dieu agrée le foyer qui le craint]}
\Chap{128}
\VerseOne{}Cantique des degrés. Heureux tout homme qui craint Yahweh et marche dans ses voies !
\VS{2}Tu jouis du travail de tes mains ; tu es heureux et tu prospères\FTNT{Es. 3:10.}.
\VS{3}Ta femme est dans ta maison comme une vigne qui porte du fruit ; tes fils sont autour de ta table comme des plants d'oliviers.
\VS{4}Voici, certainement ainsi sera béni l'homme qui craint Yahweh.
\VS{5}Yahweh te bénira de Sion et tu verras le bien de Jérusalem tous les jours de ta vie.
\VS{6}Tu verras les fils de tes fils. La paix sera sur Israël.
\TextTitle{[Position de celui qui est persécuté]}
\Chap{129}
\VerseOne{}Cantique des degrés. Qu'Israël dise maintenant : Ils m'ont souvent tourmenté dès ma jeunesse.
\VS{2}Ils m'ont assez opprimé dès ma jeunesse, mais ils ne m’ont pas vaincu.
\VS{3}Des laboureurs ont labouré mon dos, ils y ont tracé de longs sillons.
\VS{4}Yahweh est juste, il a coupé les cordes des méchants.
\VS{5}Qu’ils soient honteux et qu’ils reculent, tous ceux qui haïssent Sion !
\VS{6}Qu’ils soient comme l'herbe des toits qui sèche avant qu’on l’arrache !
\VS{7}Le moissonneur n’en remplit point sa main, ni celui qui lie les gerbes n'en remplit point ses bras ;
\VS{8}et les passants ne disent pas : Que la bénédiction de Yahweh soit sur vous ! Nous vous bénissons au nom de Yahweh !
\TextTitle{[La rédemption en abondance auprès de Yahweh]}
\Chap{130}
\VerseOne{}Cantique des degrés. Ô Yahweh ! Je t'invoque du fond de l’abîme.
\VS{2}Seigneur, écoute ma voix ! Que tes oreilles soient attentives à la voix de mes supplications !
\VS{3}Yahweh ! si tu prends garde aux iniquités, Seigneur, qui subsistera ?
\VS{4}Mais le pardon se trouve auprès de toi, afin qu’on te craigne\FTNT{Col. 1:12-14 ; Mt. 26:28 ; Ro. 3:24.}.
\VS{5}J’espère en Yahweh, mon âme espère, et j’attends sa parole.
\VS{6}Mon âme attend le Seigneur plus que les sentinelles n'attendent le matin, plus que les sentinelles n'attendent le matin.
\VS{7}Israël, attends-toi à Yahweh, car Yahweh est miséricordieux et la rédemption est auprès de lui en abondance.
\VS{8}Lui-même rachètera Israël de toutes ses iniquités.
\TextTitle{[Mettre son espoir en Yahweh seul]}
\Chap{131}
\VerseOne{}Cantique des degrés, de David. Ô Yahweh ! Je n’ai ni un cœur qui s’élève ni un regard hautain\FTNT{Pr. 16:5 ; Pr. 6:17.} ; je ne m’occupe pas de choses trop grandes et trop extraordinaires pour moi.
\VS{2}J’ai l’âme calme et tranquille comme un enfant sevré de sa mère ; j’ai l’âme comme un enfant sevré.
\VS{3}Israël attends-toi à Yahweh dès maintenant et à jamais !
\TextTitle{[Yahweh, le Dieu de David]}
\Chap{132}
\VerseOne{}Cantique des degrés. Ô Yahweh ! Souviens-toi de David et de toute son affliction !
\VS{2}Il a juré à Yahweh et fait ce vœu au puissant de Jacob :
\VS{3}Je n’entrerai pas dans la tente où j’habite, je ne monterai pas sur le lit où je couche,
\VS{4}je ne donnerai pas du sommeil à mes yeux, je ne laisserai pas sommeiller mes paupières,
\VS{5}jusqu’à ce que j'aie trouvé un lieu pour Yahweh, une demeure pour le puissant de Jacob\FTNT{1 Ch. 15:1.}.
\VS{6}Voici, nous avons entendu parler d'elle à Ephrata, nous l'avons trouvée dans les champs de Jaar.
\VS{7}Entrons dans sa demeure, prosternons-nous devant son marchepied.
\VS{8}Lève-toi, ô Yahweh, pour venir à ton lieu de repos, toi et l'arche de ta force\FTNT{2 Ch. 6:41 ; No. 10:35-36.}.
\VS{9}Que tes sacrificateurs soient revêtus de justice et que tes bien-aimés chantent de joie\FTNT{Es. 11:5 ; Ap. 19:8.} !
\VS{10}Pour l'amour de David, ton serviteur, ne permets pas que ton oint retourne en arrière !
\VS{11}Yahweh a juré la vérité à David, et il ne se rétractera pas, disant : Je mettrai le fruit de tes entrailles\FTNT{2 S. 7:12 ; 1 R. 8:25 ; 2 Ch. 6:16 ; Lu. 1:69 ; Ac. 2:30.} sur ton trône.
\VS{12}Si tes fils gardent mon alliance et mon témoignage que je leur enseignerai, leurs fils aussi seront assis à perpétuité sur ton trône.
\VS{13}Car Yahweh a choisi Sion, il l'a préférée pour être son trône :
\VS{14}Elle est mon lieu de repos à perpétuité, j'y habiterai parce que je l'ai désirée.
\VS{15}Je bénirai abondamment sa nourriture, je rassasierai de pain ses pauvres.
\VS{16}Je revêtirai de salut ses sacrificateurs, et ses bien-aimés chanteront avec des cris de joie.
\VS{17}Je ferai qu'en elle germera une corne à David ; je préparerai une lampe à mon oint,
\VS{18}je revêtirai de honte ses ennemis, et sur lui fleurira son diadème.
\TextTitle{[La communion fraternelle]}
\Chap{133}
\VerseOne{}Cantique des degrés. De David. Voici, oh ! Que c'est une chose bonne et que c'est une chose agréable que des frères demeurent unis ensemble\FTNT{Hé. 13:1 ; Ac. 2:46.} !
\VS{2}C'est comme cette huile précieuse, répandue sur la tête, qui coule sur la barbe d'Aaron\FTNT{Ex. 30:22-30.}, sur le bord de ses vêtements ;
\VS{3}comme la rosée de l’Hermon, celle qui descend sur les montagnes de Sion. Car c'est là que Yahweh a ordonné la bénédiction et la vie, pour l’éternité.
\TextTitle{[Bénissez Yahweh, vous tous ses serviteurs]}
\Chap{134}
\VerseOne{}Cantique des degrés. Voici, bénissez Yahweh ! Vous tous les serviteurs de Yahweh ! Qui vous tenez toutes les nuits dans la maison de Yahweh !
\VS{2}Elevez vos mains vers le lieu saint ! Et bénissez Yahweh !
\VS{3}Que Yahweh, qui a fait les cieux et la terre, te bénisse de Sion !
\TextTitle{[La souveraineté de Dieu]}
\Chap{135}
\VerseOne{}Louez le Nom de Yahweh ! Vous serviteurs de Yahweh ! Louez-le !
\VS{2}Vous qui vous tenez dans la maison de Yahweh, dans les parvis de la maison de notre Dieu,
\VS{3}louez Yahweh, car Yahweh est bon ! Chantez son Nom, car il est agréable !
\VS{4}Car Yahweh s'est choisi Jacob et Israël pour sa possession\FTNT{Ex. 19:5 ; De. 7:6 ; Tit. 2:14 ; 1 Pi. 2:9.}.
\VS{5}Certainement, je sais que Yahweh est grand et que notre Seigneur est au-dessus de tous les dieux.
\VS{6}Yahweh fait tout ce qu'il lui plaît, dans les cieux et sur la terre, dans la mer et dans tous les abîmes.
\VS{7}C'est lui qui fait monter les vapeurs des extrémités de la terre ; il fait les éclairs et la pluie ; il tire le vent hors de ses trésors.
\VS{8}C'est lui qui a frappé les premiers-nés d'Egypte, tant des hommes que des bêtes ;
\VS{9}qui a envoyé des prodiges et des miracles au milieu de toi, ô Egypte ! Contre Pharaon et contre tous ses serviteurs ;
\VS{10}qui a frappé plusieurs nations et tué les puissants rois ;
\VS{11}Sihon, roi des Amoréens, et Og, roi de Basan, et ceux de tous les royaumes de Canaan\FTNT{No. 21:33-35 ; De. 3:11.} ;
\VS{12}qui a donné leur pays en héritage, en héritage à Israël son peuple.
\VS{13}Yahweh, ton Nom est pour toujours ! Yahweh, ta mémoire de génération en génération !
\VS{14}Car Yahweh jugera son peuple et se repentira à l'égard de ses serviteurs.
\VS{15}Les dieux des nations ne sont que de l'or et de l'argent, un ouvrage de mains d'homme.
\VS{16}Ils ont une bouche, et ne parlent point ; ils ont des yeux, et ne voient point ;
\VS{17}ils ont des oreilles, et n'entendent point ; il n'y a point de souffle dans leur bouche.
\VS{18}Ils leur ressemblent ceux qui les font, et tous ceux qui s'y confient.
\VS{19}Maison d'Israël, bénissez Yahweh ! Maison d'Aaron, bénissez Yahweh !
\VS{20}Maison des Lévites, bénissez Yahweh ! Vous qui craignez Yahweh, bénissez Yahweh !
\VS{21}Béni soit de Sion Yahweh qui habite dans Jérusalem ! Louez Yahweh !
\TextTitle{[Miséricorde constante de Yahweh]}
\Chap{136}
\VerseOne{}Célébrez Yahweh, car il est bon, car sa bonté demeure à toujours !
\VS{2}Célébrez le Dieu des dieux, car sa bonté demeure à toujours !
\VS{3}Célébrez le Seigneur des seigneurs, car sa bonté demeure à toujours !
\VS{4}Célébrez celui qui seul fait de grandes merveilles, car sa bonté demeure à toujours !
\VS{5}Celui qui a fait avec intelligence les cieux, car sa bonté demeure à toujours !
\VS{6}Celui qui a étendu la terre sur les eaux, car sa bonté demeure à toujours !
\VS{7}Celui qui a fait les grands luminaires, car sa bonté demeure à toujours !
\VS{8}Le soleil pour dominer sur le jour, car sa bonté demeure à toujours !
\VS{9}La lune et les étoiles pour dominer la nuit, car sa bonté demeure à toujours !
\VS{10}Celui qui a frappé l'Egypte dans leurs premiers-nés, car sa bonté demeure à toujours !
\VS{11}Qui a fait sortir Israël du milieu d'eux, car sa bonté demeure à toujours.
\VS{12}Et cela avec main forte et bras étendu, car sa bonté demeure à toujours !
\VS{13}Il a fendu la Mer Rouge en deux, car sa bonté demeure à toujours !
\VS{14}Il a fait passer Israël par le milieu d'elle, car sa bonté demeure à toujours !
\VS{15}Il a renversé Pharaon et son armée dans la Mer Rouge, car sa bonté demeure à toujours !
\VS{16}Il a conduit son peuple dans le désert, car sa bonté demeure à toujours !
\VS{17}Il a frappé les grands rois, car sa bonté demeure à toujours !
\VS{18}Qui a tué des grands rois, car sa bonté demeure à toujours !
\VS{19}Sihon, roi des Amoréens, car sa bonté demeure à toujours !
\VS{20}Og, roi de Basan, car sa bonté demeure à toujours !
\VS{21}Il a donné leur pays en héritage, car sa bonté demeure à toujours\FTNT{Jos. 12:7.} !
\VS{22}En héritage à Israël son serviteur, car sa bonté demeure à toujours !
\VS{23}Et qui, lorsque nous étions humiliés, s'est souvenu de nous, car sa bonté demeure à toujours !
\VS{24}Il nous a délivrés de la main de nos adversaires, car sa bonté demeure à toujours !
\VS{25}Il donne la nourriture à toute chair, car sa bonté demeure à toujours\FTNT{Ps. 104:21 ; Mt. 6:26 ; Ps. 147:9.} !
\VS{26}Célébrez le Dieu des cieux, car sa bonté demeure à toujours !
\TextTitle{[Le coeur des captifs]}
\Chap{137}
\VerseOne{}Sur les bords des fleuves de Babylone, nous étions assis et nous pleurions en nous souvenant de Sion.
\VS{2}Nous avions suspendu nos harpes au milieu des saules.
\VS{3}Là, ceux qui nous avaient emmenés en captivité, nous ont demandé des paroles de chants, et nos oppresseurs de la joie, en nous disant : Chantez-nous quelques cantiques de Sion ! Nous avons répondu :
\VS{4}Comment chanterions-nous les cantiques de Yahweh sur une terre étrangère ?
\VS{5}Si je t'oublie, Jérusalem, que ma droite s'oublie elle-même.
\VS{6}Que ma langue soit attachée à mon palais\FTNT{Ez. 3:26.}, si je ne me souviens pas de toi, si je ne fais pas de Jérusalem le sujet de ma réjouissance.
\VS{7}Ô Yahweh, souviens-toi des fils d'Edom, qui dans la journée de Jérusalem disaient : Rasez, rasez jusqu’à ses fondements\FTNT{Jé. 25:15-21 ; Jé. 49:7-8 ; La. 4:21 ; Ez. 25:12 ; Am. 1:11.} !
\VS{8}Fille de Babylone, qui va être détruite, heureux celui qui te rend la pareille de ce que tu nous as fait\FTNT{Jé. 50:15-29 ; Ap. 18:6.} !
\VS{9}Heureux celui qui saisit tes petits enfants et qui les écrase contre le rocher\FTNT{Es. 13:16.} !
\TextTitle{[La renommée de Yahweh dans les nations]}
\Chap{138}
\VerseOne{}Psaume de David. Je te célèbre de tout mon cœur, je te chante des louanges dans la présence de Dieu.
\VS{2}Je me prosterne dans ton saint temple, et je célèbre ton Nom à cause de ta bonté et de ta fidélité ; car ta renommée s’est accrue par l’accomplissement de ta promesse.
\VS{3}Le jour où je t’ai invoqué, tu m'as exaucé, tu m'as rassuré, tu m'as fortifié d’une nouvelle force en mon âme.
\VS{4}Yahweh ! Tous les rois de la terre te célèbrent, quand ils entendent les paroles de ta bouche.
\VS{5}Ils chantent les voies de Yahweh, car la gloire de Yahweh est grande.
\VS{6}Car Yahweh est haut élevé, il voit les humbles et il reconnaît de loin les orgueilleux.
\VS{7}Quand je marche au milieu de l'adversité, tu me rends la vie, tu avances ta main contre la colère de mes ennemis, et ta droite me délivre.
\VS{8}Yahweh achèvera ce qui me concerne. Yahweh, ta bonté demeure toujours ; tu n'abandonnes pas l’œuvre de tes mains\FTNT{Ph. 1:6.}.
\TextTitle{[Dieu voit tout et connait tout]}
\Chap{139}
\VerseOne{}Psaume de David, donné au chef des chantres. Yahweh, tu me sondes et tu me connais\FTNT{Ps. 17:3 ; Jé. 12:3.}.
\VS{2}Tu sais quand je m'assieds et quand je me lève ; tu discernes de loin ma pensée.
\VS{3}Tu sais quand je marche et quand je me couche ; tu connais parfaitement toutes mes voies.
\VS{4}Avant que la parole soit sur ma langue, voici, ô Yahweh, tu la connais déjà !
\VS{5}Tu m’entoures par derrière et par devant, et tu mets ta main sur moi.
\VS{6}Ta science est trop merveilleuse pour moi, elle est si haut élevée que je ne saurais l’atteindre\FTNT{Ps. 92:6 ; Ro. 11:33 ; Job. 42:3.}.
\VS{7}Où irai-je loin de ton Esprit, et où fuirai-je loin de ta face\FTNT{Jé. 23:24 ; Am. 9:2-4 ; Jon. 1:3.} ?
\VS{8}Si je monte aux cieux, tu y es ; si je me couche dans le scheol, t'y voilà.
\VS{9}Si je prends les ailes de l’aurore et que je demeure à l’extrémité de la mer,
\VS{10}là aussi ta main me conduira et ta droite me saisira.
\VS{11}Si je dis : Au moins les ténèbres me couvriront, la nuit même sera une lumière tout autour de moi.
\VS{12}Même les ténèbres ne me cacheront point de toi, et la nuit resplendira comme le jour, et les ténèbres comme la lumière.
\VS{13}Tu as créé mes reins, tu me couvres du sein de ma mère.
\VS{14}Je te célèbre de ce que je suis une créature redoutée et merveilleuse ; tes œuvres sont merveilleuses, et mon âme le reconnaît très bien.
\VS{15}Mon corps n’était pas caché devant toi lorsque j'ai été fait dans un lieu secret et brodé dans les profondeurs de la terre\FTNT{Ps. 119:73 ; Ec. 11:5.}.
\VS{16}Tes yeux me voyaient quand je n’étais qu’un embryon, et sur ton livre étaient inscrits tous les jours qui m’étaient destinés\FTNT{Ap. 3:5 ; Ap. 20:15 ; Ph. 4:3.}.
\VS{17}Dieu ! Que tes pensées sont précieuses ! Que le nombre en est grand !
\VS{18}Si je les compte, elles sont plus nombreuses que les grains de sable. Je m’éveille et je suis encore avec toi.
\VS{19}Ô Dieu ! Ne tueras-tu pas le méchant ? C’est pourquoi, hommes sanguinaires, retirez-vous loin de moi !
\VS{20}Car ils ont parlé de toi en pensant à quelque méchanceté ; ils ont élevé tes ennemis en mentant.
\VS{21}Yahweh, n'aurais-je point en haine ceux qui te haïssent ; et ne serais-je point irrité contre ceux qui s'élèvent contre toi ?
\VS{22}Je les hais d'une parfaite haine ; ils sont pour moi des ennemis.
\VS{23}Ô Dieu ! Sonde-moi et considère mon cœur ! Eprouve-moi et considère mes discours !
\VS{24}Et regarde si je suis sur une mauvaise voie ; conduis-moi sur la voie de l’éternité.
\TextTitle{[Appel au soutien de Dieu]}
\Chap{140}
\VerseOne{}Psaume de David, donné au chef des chantres. Yahweh, délivre-moi de l'homme méchant, garde-moi de l'homme violent.
\VS{2}Ils méditent des méchancetés dans leur cœur, tous les jours ils complotent des guerres.
\VS{3}Ils aiguisent leur langue comme un serpent, il y a du venin de vipère sous leurs lèvres. Pause.
\VS{4}Yahweh, garde-moi de la main du méchant, préserve-moi de l'homme violent, de ceux qui méditent de me faire tomber.
\VS{5}Les orgueilleux me tendent un piège et des filets, et ils étendent des rets le long du chemin, ils me dressent des embûches. Pause.
\VS{6}Je dis à Yahweh : Tu es mon Dieu, Yahweh ! Prête l'oreille à la voix de mes supplications !
\VS{7}Ô Yahweh ! Seigneur ! La force de mon salut ! Tu couvres ma tête au jour de la bataille.
\VS{8}Yahweh n'accorde point au méchant ses désirs ; qu’il n’apporte pas ses méchants desseins, ils s'élèveraient. Pause.
\VS{9}Quant à la tête de ceux qui m’environnent, que la méchanceté de leurs lèvres les recouvre.
\VS{10}Que des charbons ardents soient jetés sur eux ! Qu’ils tombent sur eux ! Qu'il les fasse tomber dans le feu, et dans des fosses profondes, sans qu'ils se relèvent\FTNT{Pr. 25:21-22 ; Ro. 12:20.} !
\VS{11}Que l'homme à la langue méchante ne soit point affermi sur la terre ; quant à l'homme violent et mauvais, qu'on le chasse jusqu’à ce qu'il soit exterminé.
\VS{12}Je sais que Yahweh fera justice au malheureux et droit aux indigents.
\VS{13}Quoi qu'il en soit, les justes célébreront ton Nom, les hommes droits habiteront devant ta face.
\TextTitle{[Se garder des hommes violents]}
\Chap{141}
\VerseOne{}Psaume de David. Yahweh, je t'invoque, hâte-toi de venir vers moi ; prête l'oreille à ma voix lorsque je crie à toi.
\VS{2}Que ma prière te soit agréable comme l’encens, et l'élévation de mes mains comme l’offrande du soir\FTNT{Ex. 30:1 ; Ap. 5:8 ; Ap. 8:3.}.
\VS{3}Yahweh, mets une garde à ma bouche, garde l'entrée de mes lèvres.
\VS{4}N'incline point mon cœur à des choses mauvaises, au point que je commette quelques méchantes actions par malice, avec les hommes qui font le mal ; et que je ne mange point de leurs délices.
\VS{5}Que le juste me frappe, ce me sera une faveur ; et qu'il me réprimande, ce sera pour moi un baume excellent\FTNT{Pr. 27:6 ; Ec. 7:5.} ; il ne blessera point ma tête ; car ma prière sera pour eux leur calamité.
\VS{6}Que leurs juges soient précipités le long des rochers, et l’on écoutera mes paroles, car elles sont agréables.
\VS{7}Nos os sont dispersés dans la bouche du scheol comme quand on laboure la terre et on fend le bois.
\VS{8}C'est pourquoi, ô Yahweh, Seigneur, mes yeux sont sur toi, je me suis retiré vers toi, n'abandonne point mon âme !
\VS{9}Garde-moi du piège qu'ils m'ont tendu et des filets de ceux qui font le mal.
\VS{10}Que tous les méchants tombent dans leurs filets, jusqu’à ce que je sois passé.
\TextTitle{[Expérimenter la délivrance de Dieu]}
\Chap{142}
\VerseOne{}Cantique de David. Prière qu'il fit lorsqu'il était dans la caverne\FTNT{1 S. 24:4.}.
\VS{2}Je crie de ma voix à Yahweh, je supplie de ma voix Yahweh.
\VS{3}Je répands devant lui ma complainte, je déclare mon angoisse devant lui\FTNT{1 S. 1:15 ; La. 2:19.}.
\VS{4}Quand mon esprit est abattu en moi, toi, tu connais mon sentier. Ils me tendent un piège sur le chemin par lequel je marche.
\VS{5}Je contemple à ma droite et je regarde, et il n’y a personne qui me reconnaît ; tout refuge s’évanouit devant moi, il n'y a personne qui prend soin de mon âme.
\VS{6}Yahweh, je crie vers toi ; je dis : Tu es mon refuge, ma part sur la terre des vivants.
\VS{7}Sois attentif à mon cri car je suis devenu très affaibli. Délivre-moi de ceux qui me poursuivent car ils sont plus puissants que moi.
\VS{8}Retire mon âme de sa prison afin que je célèbre ton Nom ! Les justes viendront m’entourer quand tu m'auras fait du bien.
\TextTitle{[Dieu nous enseigne à faire sa volonté]}
\Chap{143}
\VerseOne{}Psaume de David. Yahweh, écoute ma requête, prête l'oreille à mes supplications ! Exauce-moi dans ta fidélité, réponds-moi à cause de ta justice !
\VS{2}N'entre point en jugement avec ton serviteur, car aucun homme vivant n’est juste devant toi.
\VS{3}Car l'ennemi poursuit mon âme, il foule ma vie par terre ; il me fait habiter dans les ténèbres comme ceux qui sont morts depuis longtemps.
\VS{4}Et mon esprit est abattu au-dedans de moi, mon cœur est épouvanté en mon sein.
\VS{5}Je me souviens des jours anciens, je médite sur toutes tes œuvres, je médite sur l’ouvrage de tes mains\FTNT{Ps. 77:11-13.}.
\VS{6}J'étends mes mains vers toi ; mon âme s'adresse à toi comme une terre desséchée\FTNT{Ps. 28:1 ; Ps. 42:1.}. Pause.
\VS{7}Ô Yahweh, hâte-toi, réponds-moi ! Mon esprit se consume ! Ne me cache point ta face au point que je devienne semblable à ceux qui descendent dans la fosse !
\VS{8}Fais-moi entendre dès le matin ta miséricorde, car je me confie en toi ; fais-moi connaître le chemin par lequel je dois marcher, car j'ai élevé mon cœur vers toi\FTNT{Ps. 25:1.}.
\VS{9}Yahweh, délivre-moi de mes ennemis, car je me suis réfugié auprès de toi !
\VS{10}Enseigne-moi à faire ta volonté, car tu es mon Dieu ! Que ton bon Esprit me conduise sur la voie de la droiture\FTNT{Jn. 16:13.} !
\VS{11}Yahweh, rends-moi la vie pour l'amour de ton Nom ! Retire mon âme de la détresse à cause de ta justice !
\VS{12}Et selon la bonté que tu as pour moi, retranche mes ennemis ! Détruis tous ceux qui tiennent mon âme oppressée, parce que je suis ton serviteur !
\TextTitle{[Se confier en Yahweh, le rocher]}
\Chap{144}
\VerseOne{}Psaume de David. Béni soit Yahweh, mon rocher\FTNT{Voir commentaire Es. 8:13-17.} qui exerce mes mains au combat et mes doigts à la bataille,
\VS{2}qui déploie sa bonté envers moi, qui est ma forteresse, ma haute retraite, mon libérateur\FTNT{Es. 59:20-21 ; Ro. 11:26.}, mon bouclier\FTNT{Ep. 6:16.}, mon refuge\FTNT{Ps. 91 ; Mt. 11:28-30.}, qui m’assujettit mon peuple.
\VS{3}Ô Yahweh ! Qu’est-ce que l'homme pour que tu aies soin de lui\FTNT{Job. 7:17 ; Ps. 8:5 ; Hé. 2:6-7.} ? Le fils de l'homme mortel pour que tu prennes garde à lui ?
\VS{4}L'homme est semblable à la vanité, ses jours sont comme une ombre qui passe\FTNT{Ec. 6:12 ; Job. 14:1-2 ; Ps. 102:12.}.
\VS{5}Yahweh abaisse tes cieux et descends ! Touche les montagnes et qu'elles soient fumantes\FTNT{Ps. 18:7-8 ; Es. 63:19.}.
\VS{6}Lance les éclairs et disperse mes ennemis ! Lance tes flèches et mets-les en déroute !
\VS{7}Etends tes mains d'en haut ; sauve-moi et délivre-moi des grandes eaux, de la main des fils de l'étranger,
\VS{8}dont la bouche profère le mensonge, et dont la droite est une droite trompeuse !
\VS{9}Ô Dieu ! Je chanterai un cantique nouveau ! Je te célèbrerai sur le luth à dix cordes !
\VS{10}Toi qui donnes la délivrance aux rois et qui délivres de l'épée meurtrière David, ton serviteur.
\VS{11}Retire-moi et délivre-moi de la main des fils de l'étranger, dont la bouche profère le mensonge et dont la droite est une droite trompeuse ;
\VS{12}afin que nos fils soient comme des plantes qui croissent dans leur jeunesse et nos filles comme des pierres angulaires taillées pour l'ornement d'un palais.
\VS{13}Que nos greniers soient pleins, fournissant toute espèce de provision ; que nos troupeaux multiplient par milliers, même par dix milliers dans nos rues.
\VS{14}Que nos bœufs soient chargés de graisse. Qu'il n'y ait ni brèche, ni sortie dans nos murailles, ni cri dans nos places.
\VS{15}Heureux le peuple pour qui il en est ainsi ! Heureux le peuple dont Yahweh est le Dieu !
\TextTitle{[La compassion de Dieu]}
\Chap{145}
\VerseOne{}Psaume de louange, composé par David. [Aleph.] Mon Dieu, mon roi, je t'exalterai et je bénirai ton Nom à toujours, et à perpétuité !
\VS{2}[Beth.] Je te bénirai chaque jour, et je louerai ton Nom à toujours, et à perpétuité !
\VS{3}[Guimel.] Yahweh est grand et très digne de louanges, il n'est pas possible de sonder sa grandeur.
\VS{4}[Daleth.] Que chaque génération célèbre tes œuvres et publie tes hauts faits !
\VS{5}[He.] Je dirai la splendeur glorieuse de ta majesté et de tes faits merveilleux.
\VS{6}[Vau.] On parlera de ta puissance redoutable, et je raconterai ta grandeur.
\VS{7}[Zaïn.] Ils proclameront le souvenir de ton immense bonté, et ils raconteront avec chants de triomphe ta justice.
\VS{8}[Heth.] Yahweh est miséricordieux et compatissant, lent à la colère et grand en bonté.
\VS{9}[Teth.] Yahweh est bon envers tous et ses compassions sont au-dessus de toutes ses œuvres.
\VS{10}[Jod.] Yahweh, toutes tes œuvres te célébreront, et tes fidèles te béniront.
\VS{11}[Caph.] Ils diront la gloire de ton règne, et ils proclameront ta puissance
\VS{12}[Lamed.] pour faire connaître aux fils de l’homme ta puissance et la splendeur glorieuse de ton règne.
\VS{13}[Mem.] Ton règne est un règne de tous les siècles et ta domination subsiste dans tous les âges.
\VS{14}[Samech.] Yahweh soutient tous ceux qui tombent et redresse tous ceux qui sont courbés\FTNT{Ps. 146:8.}.
\VS{15}[Hajin.] Les yeux de tous les animaux s'attendent à toi et tu leur donnes leur nourriture en leur temps.
\VS{16}[Pe.] Tu ouvres ta main et tu rassasies à souhait toute créature vivante.
\VS{17}[Tsade.] Yahweh est juste dans toutes ses voies et plein de bonté dans toutes ses œuvres\FTNT{Da. 4:37.}.
\VS{18}[Koph.] Yahweh est près de tous ceux qui l'invoquent, de tous ceux qui l'invoquent avec vérité\FTNT{Ps. 34:18.}.
\VS{19}[Resch.] Il accomplit le désir de ceux qui le craignent, il entend leur cri et les délivre.
\VS{20}[Scin.] Yahweh garde tous ceux qui l'aiment, mais il exterminera tous les méchants.
\VS{21}[Thau.] Ma bouche racontera la louange de Yahweh, et toute chair bénira le Nom de sa sainteté à toujours, et à perpétuité\FTNT{Ps. 103:1.}.
\TextTitle{[Dieu garde sa fidélité à toujours]}
\Chap{146}
\VerseOne{}Louez Yahweh ! Mon âme, loue Yahweh !
\VS{2}Je louerai Yahweh durant ma vie, je chanterai mon Dieu tant que je vivrai !
\VS{3}Ne vous confiez pas aux grands, ni en aucun fils de l’homme qui ne peuvent délivrer.
\VS{4}Son esprit s’en va et l'homme retourne dans sa terre, et ce même jour ses desseins périssent.
\VS{5}Heureux celui qui a pour secours le Dieu de Jacob, qui met son espoir en Yahweh, son Dieu !
\VS{6}Il a fait les cieux et la terre, la mer et tout ce qui s’y trouve. Il garde la vérité à toujours !
\VS{7}Il fait droit aux opprimés, il donne du pain aux affamés ; Yahweh délie ceux qui sont liés\FTNT{Jn. 11:43-44.}.
\VS{8}Yahweh ouvre les yeux des aveugles\FTNT{Les miracles de Jésus-Christ confirment sa divinité (Es. 35:4-6 ; Lu. 7:19-23).} ; Yahweh redresse ceux qui sont courbés\FTNT{Lu. 13:11-13.} ; Yahweh aime les justes.
\VS{9}Yahweh protège les étrangers, il soutient l'orphelin et la veuve, mais il renverse la voie des méchants.
\VS{10}Yahweh règne éternellement. Ô Sion ! Ton Dieu subsiste d'âge en âge. Louez Yahweh !
\TextTitle{[Craindre Yahweh et espérer en sa bonté]}
\Chap{147}
\VerseOne{}Louez Yahweh ! Car il est beau de chanter à notre Dieu ! Car il est doux et bienséant de le louer !
\VS{2}Yahweh est celui qui bâtit Jérusalem ; il rassemblera ceux d'Israël qui sont dispersés çà et là.
\VS{3}Il guérit ceux qui ont le cœur brisé et il bande leurs plaies\FTNT{Ex. 15:26 ; Job. 5:18 ; De. 32:39.}.
\VS{4}Il compte le nombre des étoiles, il les appelle toutes par leur nom.
\VS{5}Notre Seigneur est grand, puissant par sa force, son intelligence n’a point de limites.
\VS{6}Yahweh soutient les malheureux, mais il abaisse les méchants jusqu’à terre.
\VS{7}Chantez à Yahweh avec reconnaissance ! Célébrez notre Dieu avec la harpe !
\VS{8}Il couvre les cieux de nuées, il prépare la pluie pour la terre ; il fait germer l’herbe sur les montagnes.
\VS{9}Il donne la nourriture au bétail et aux petits du corbeau qui crient.
\VS{10}Il ne prend point plaisir dans la force du cheval ; il ne fait point cas des jambes de l'homme.
\VS{11}Yahweh aime ceux qui le craignent, ceux qui s'attendent à sa bonté.
\VS{12}Jérusalem, loue Yahweh ! Sion, loue ton Dieu !
\VS{13}Car il a affermi les barres de tes portes, il a béni tes fils au milieu de toi.
\VS{14}Il rend la paix à son territoire et te rassasie du meilleur froment.
\VS{15}C'est lui qui envoie ses ordres sur la terre, sa parole court avec rapidité\FTNT{Es. 55:10-11.}.
\VS{16}C'est lui qui donne la neige comme des flocons de laine et qui répand la gelée blanche comme de la cendre.
\VS{17}C'est lui qui lance sa glace comme par morceaux, qui peut résister devant son froid ?
\VS{18}Il envoie sa parole, et il les fond ; il fait souffler son vent, et les eaux coulent\FTNT{Ps. 135:7.}.
\VS{19}Il déclare ses paroles à Jacob, ses statuts et ses ordonnances à Israël\FTNT{Ps. 78:5.}.
\VS{20}Il n'a pas agi de même pour toutes les nations, c'est pourquoi elles ne connaissent point ses ordonnances. Louez Yahweh !
\TextTitle{[La création loue son Dieu]}
\Chap{148}
\VerseOne{}Louez Yahweh ! Louez des cieux Yahweh ! Louez-le dans les lieux élevés !
\VS{2}Louez-le, vous tous anges ! Louez-le, vous toutes ses armées !
\VS{3}Louez-le, vous, soleil et lune ! Louez-le, vous toutes, étoiles lumineuses !
\VS{4}Louez-le, vous, cieux des cieux ! Et vous, eaux qui êtes au-dessus des cieux !
\VS{5}Qu’ils louent le Nom de Yahweh ! Car il a commandé et ils ont été créés\FTNT{Ge. 1:3-6 ; Jé. 31:35.}.
\VS{6}Il les a établis à perpétuité et à toujours ; il a donné des lois, et il ne les violera pas\FTNT{Ps. 104:5 ; Ps. 119:91 ; Job. 14:5.}.
\VS{7}De la terre, louez Yahweh ! Louez-le, monstres marins et tous les abîmes !
\VS{8}Feu et grêle, neige et brouillard, vent impétueux qui exécutez ses ordres,
\VS{9}montagnes et toutes les collines, arbres fruitiers et tous les cèdres,
\VS{10}bêtes sauvages et tout le bétail, reptiles et oiseaux ailés,
\VS{11}rois de la terre et tous les peuples, princes et tous les juges de la terre,
\VS{12}ceux qui sont à la fleur de leur âge, et les vierges aussi, les vieillards, et les jeunes gens !
\VS{13}Qu'ils louent le Nom de Yahweh ! Car son Nom seul est haut élevé ! Sa majesté est au-dessus de la terre et des cieux.
\VS{14}Il a relevé la force de son peuple, sujet de louange pour tous ses fidèles, pour les fils d'Israël, du peuple qui est près de lui. Louez Yahweh !
\TextTitle{[L'adoration sied à Yahweh]}
\Chap{149}
\VerseOne{}Louez Yahweh ! Chantez à Yahweh un cantique nouveau et louez-le dans l'assemblée de ses fidèles !
\VS{2}Qu'Israël se réjouisse en celui qui l'a fait ! Et que les fils de Sion soient dans l’allégresse à cause de leur Roi\FTNT{Ps. 100:3 ; Za. 9:9 ; Mt. 21:5.} !
\VS{3}Qu'ils louent son Nom avec des danses ! Qu'ils le chantent avec le tambourin et la harpe !
\VS{4}Car Yahweh prend plaisir à son peuple, il glorifie les pauvres en les délivrant.
\VS{5}Que les fidèles se réjouissent dans la gloire, qu’ils poussent des cris de joie sur leur couche.
\VS{6}Les louanges de Dieu sont dans leur bouche et les épées affilées à deux tranchants dans leur main,
\VS{7}pour se venger des nations, pour châtier les peuples,
\VS{8}pour lier leurs rois avec des chaînes, et les plus honorables parmi eux avec des ceps de fer,
\VS{9}pour exercer sur eux le jugement qui est écrit ! Cet honneur est pour tous ses fidèles. Louez Yahweh !
\TextTitle{[La véritable louange]}
\Chap{150}
\VerseOne{}Louez Yahweh ! Louez Dieu à cause de sa sainteté ! Louez-le dans l’étendue de toute sa puissance !
\VS{2}Louez-le pour ses hauts faits ! Louez-le selon la grandeur de sa magnificence !
\VS{3}Louez-le au son du shofar ! Louez-le avec le luth et la harpe !
\VS{4}Louez-le avec le tambour et avec des danses ! Louez-le avec des instruments à cordes et le chalumeau !
\VS{5}Louez-le avec les cymbales sonores ! Louez-le avec les cymbales de cri de joie !
\VS{6}Que tout ce qui respire loue Yahweh ! Louez Yahweh !
\PPE{}
\end{multicols}
