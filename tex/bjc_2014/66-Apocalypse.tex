\ShortTitle{Ap.}\BookTitle{Apocalypse}\BFont
\noindent\hrulefill
{\footnotesize
\textit{
\bigskip
{\centering{}
\\Auteur~: Jean
\\Thème~: L'aboutissement de toutes choses
\\(Gr.~: Apokalupsis)
\\Signification~: Mettre à nu, révélation d'une vérité, action de révéler
\\Date de rédaction~: Env. 95 ap. J.-C.\\}
}
\textit{
\\Le terme apocalypse, du grec «~apokalupsis~», évoque «~l'action de révéler ce qui était caché ou inconnu~». Ce mot a pour racine «~apokalupto~» qui signifie aussi «~découvrir, dévoiler ce qui est voilé ou recouvert~».
\\C'est à Patmos, île grecque de la mer Egée - où il s'exila en raison de la persécution de l'empereur Domitien (51 - 96 ap. J.C.) - que Jean reçut une révélation de Jésus-Christ ainsi qu'un message s'adressant aux «~sept églises~» qui constituaient certainement les villes de l'Asie Mineure où se trouvaient les principales concentrations de chrétiens. Si Ephèse figure dans les écrits de la Nouvelle Alliance et que Thyatire et Laodicée y sont brièvement mentionnées, les quatre autres églises - qu'on ne retrouve nulle part ailleurs dans les Ecritures - étaient sans doute le fruit du travail missionnaire de Paul. Les sept lettres s'adressent à l'ange de chacune de ces assemblées locales, autrement dit aux messagers de celles-ci (probablement un ancien ou un responsable).
\\Ce livre, qui arrive en conclusion des Ecritures, annonce les événements qui doivent précéder la fin de l'histoire de l'humanité.\bigskip
}
} 
\par\nobreak\noindent\hrulefill
\begin{multicols}{2}
\Chap{1}
\TextTitle{Introduction}
\VerseOne{}La révélation\FTNT{«~Apokalupsis~» en grec. Voir l'introduction du livre.} de Jésus-Christ, que Dieu lui a donnée pour montrer à ses serviteurs les choses qui doivent arriver bientôt, et qui les a fait connaître en les envoyant par son ange à Jean, son serviteur,
\VS{2}qui a annoncé la parole de Dieu, et le témoignage de Jésus-Christ, et toutes les choses qu'il a vues.
\VS{3}Béni est celui qui lit et ceux qui écoutent les paroles de cette prophétie, et qui gardent les choses qui y sont écrites~! Car le temps est proche.
\TextTitle{Jésus-Christ}
\VS{4}Jean aux sept églises qui sont en Asie~: Que la grâce et la paix vous soient données de la part de celui QUI EST, QUI ETAIT, et QUI VIENT\FTNT{Les prophètes ont prophétisé la venue de Yahweh en personne~: Es. 35:4~; Es. 40:10-11~; Es. 60:1-2~; Za. 14:1-21~; Jn. 14:1-3. Jésus-Christ est bien Yahweh qui vient.}, et de la part des sept Esprits qui sont devant son trône,
\VS{5}et de la part de Jésus-Christ, qui est le témoin fidèle, le premier-né d'entre les morts\FTNT{Voir commentaire Col. 1:15.}, et le Prince des rois de la terre.
\VS{6}A lui, dis-je, qui nous a aimés, et qui nous a lavés de nos péchés dans son sang, et qui a fait de nous des rois et des prêtres pour Dieu, son Père, à lui soient la gloire et la force aux siècles des siècles. Amen~!
\TextTitle{La venue du Christ}
\VS{7}Voici, il vient avec les nuées, et tout œil le verra, et même ceux qui l'ont percé~; et toutes les tribus de la terre se lamenteront devant lui. Oui, amen~!
\VS{8}Je suis l'Alpha et l'Oméga, le commencement et la fin, dit le Seigneur, QUI EST, QUI ETAIT, et QUI VIENT, le Tout-Puissant.
\TextTitle{La vision doit être écrite}
\VS{9}Moi Jean, qui suis aussi votre frère et qui participe à la tribulation, au règne, et à la patience de Jésus-Christ, j'étais sur l'île appelée Patmos à cause de la parole de Dieu, et du témoignage de Jésus-Christ.
\VS{10}Je fus ravi en esprit au jour du Seigneur, et j'entendis derrière moi une voix forte, comme le son d'une trompette,
\VS{11}qui disait~: Je suis l'Alpha et l'Oméga, le premier et le dernier. Ecris dans un livre ce que tu vois, et envoie-le aux sept églises qui sont en Asie, à savoir à Ephèse, à Smyrne, à Pergame, à Thyatire, à Sardes, à Philadelphie, et à Laodicée.
\VS{12}Alors je me retournai pour voir celui dont la voix m'avait parlé, et après m'être retourné, je vis sept chandeliers d'or,
\VS{13}et au milieu des sept chandeliers d'or, quelqu'un qui ressemblait à un fils d'homme, vêtu d'une longue robe, et ayant une ceinture d'or sur la poitrine.
\VS{14}Sa tête et ses cheveux étaient blancs comme de la laine blanche, et comme de la neige, et ses yeux étaient comme une flamme de feu.
\VS{15}Ses pieds étaient semblables à de l'airain ardent, comme s'ils avaient été embrasés dans une fournaise~; et sa voix était comme le bruit des grandes eaux.
\VS{16}Et il avait dans sa main droite sept étoiles, et de sa bouche sortait une épée aiguë à deux tranchants, et son visage était semblable au soleil lorsqu'il brille dans sa force.
\VS{17}Quand je le vis, je tombai à ses pieds comme mort, et il mit sa main droite sur moi, en me disant~: Ne crains pas~!
\VS{18}Je suis le premier et le dernier, et je vis~; j'étais mort, et voici, je suis vivant aux siècles des siècles. Amen~! Et je tiens les clefs de Hadès\FTNT{Voir commentaire Mt. 16:18.} et de la mort.
\VS{19}Ecris les choses que tu as vues, celles qui sont présentement, et celles qui doivent arriver ensuite.
\VS{20}Le mystère des sept étoiles que tu as vues dans ma main droite, et les sept chandeliers d'or. Les sept étoiles sont les anges des sept églises~; et les sept chandeliers que tu as vus sont les sept églises.
\Chap{2}
\TextTitle{Ephèse~: L'église qui a perdu le premier amour}
\VerseOne{}Ecris à l'ange\FTNT{Ange, du grec «~aggelos~»~: Envoyé, messager, un ange. Un messager de Dieu. Ce terme sert à désigner aussi bien les créatures spirituelles que les êtres humains.} de l'église d'Ephèse~: Voici ce que dit celui qui tient les sept étoiles dans sa main droite, et qui marche au milieu des sept chandeliers d'or~:
\VS{2}Je connais tes œuvres, et ton travail, et ta patience, et je sais que tu ne peux pas supporter les méchants, et que tu as éprouvé ceux qui se disent être apôtres et qui ne le sont pas, et que tu les as trouvés menteurs~;
\VS{3}et que tu as souffert, et que tu as eu de la patience, et que tu as travaillé pour mon Nom, et que tu ne t'es pas lassé.
\VS{4}Mais j'ai quelque chose contre toi, c'est que tu as abandonné ta première charité\FTNT{Il est question ici de l'amour «~agape~»~: L'amour fraternel, l'amour divin.}.
\VS{5}C'est pourquoi souviens-toi donc d'où tu es tombé, repens-toi, et fais tes premières œuvres. Autrement, je viendrai à toi à toute vitesse, et j'ôterai ton chandelier de sa place si tu ne te repens pas.
\VS{6}Mais pourtant tu as ceci de bon, c'est que tu hais les œuvres des Nicolaïtes\FTNT{Nicolaïtes~: Tiré du nom Nicolas, qui signifie littéralement «~victorieux du peuple~». Il s'agit d'une secte dont les membres furent peut-être des disciples d'un certain Nicolas, l'un des diacres de l'église d'Antioche qui aurait dévié (Ac. 6:5). Ces derniers suivaient la doctrine de Balaam, enseignant aux chrétiens qu'à cause du principe de liberté, ils pouvaient manger des viandes sacrifiées aux idoles et commettre des actes immoraux comme les Gentils.}, œuvres que je hais moi aussi.
\VS{7}Que celui qui a des oreilles entende ce que l'Esprit dit aux églises~! A celui qui vaincra, je lui donnerai à manger de l'arbre de vie, qui est au milieu du paradis de Dieu.
\TextTitle{Smyrne~: L'église sous la persécution}
\VS{8}Ecris aussi à l'ange de l'église de Smyrne~: Voici ce que dit celui qui est le premier et le dernier, qui a été mort, et qui est revenu à la vie~:
\VS{9}Je connais tes œuvres, ton affliction et ta pauvreté, quoique tu sois riche, et le blasphème de ceux qui se disent être Juifs et qui ne le sont pas, mais qui sont la synagogue de Satan.
\VS{10}Ne crains rien des choses que tu as à souffrir. Voici, il arrivera que le diable mettra quelques-uns d'entre vous en prison, afin que vous soyez éprouvés~; et vous aurez une affliction de dix jours. Sois fidèle jusqu'à la mort, et je te donnerai la couronne de vie.
\VS{11}Que celui qui a des oreilles, entende ce que l'Esprit dit aux églises~! Celui qui vaincra n'aura pas à souffrir la seconde mort.
\TextTitle{Pergame~: L'église établie dans le monde}
\VS{12}Ecris aussi à l'ange de l'église de Pergame~: Voici ce que dit celui qui a l'épée aiguë à deux tranchants\FTNT{Hé. 4:12.}~: 
\VS{13}Je connais tes œuvres, et le lieu où tu habites, à savoir là où est le trône de Satan. Et que cependant tu retiens mon Nom, et tu n'as pas renié ma foi, même aux jours d'Antipas, mon fidèle martyr\FTNT{Du grec «~martus~» qui signifie «~témoin~».}, qui a été mis à mort chez vous, là où Satan habite.
\VS{14}Mais j'ai quelque chose contre toi, c'est que tu as là des gens attachés à la doctrine de Balaam, qui enseignait à Balak à mettre un scandale devant les enfants d'Israël, afin qu'ils mangent des viandes sacrifiées aux idoles, et qu'ils se livrent à la fornication\FTNT{No. 25:1-2~; No. 31:16.}.
\VS{15}De même, toi aussi tu as des gens attachés à la doctrine des Nicolaïtes~; ce que je hais~!
\VS{16}Repens-toi donc, autrement je viendrai à toi à toute vitesse, et je les combattrai avec l'épée de ma bouche.
\VS{17}Que celui qui a des oreilles, entende ce que l'Esprit dit aux églises~! A celui qui vaincra, je lui donnerai à manger de la manne qui est cachée, et je lui donnerai un caillou blanc, et sur ce caillou sera écrit un nouveau nom, que nul ne connaît, sinon celui qui le reçoit.
\TextTitle{Thyatire~: L'église en temps d'idolâtrie}
\VS{18}Ecris aussi à l'ange de l'église de Thyatire~: Voici ce que dit le Fils de Dieu, qui a ses yeux comme une flamme de feu, et dont les pieds sont semblables à de l'airain ardent.
\VS{19}Je connais tes œuvres, ta charité, ton service, ta foi, ta patience, et que tes dernières œuvres surpassent les premières.
\VS{20}Mais j'ai quelque peu de chose contre toi, c'est que tu laisses cette femme Jézabel\FTNT{1 R. 16:31~; 1 R. 21:25~; 2 R. 9:7~; 2 R. 9:22.}, qui se dit prophétesse, enseigner et séduire mes serviteurs pour les porter à la fornication, et leur faire manger des choses sacrifiées aux idoles.
\VS{21}Et je lui ai donné du temps, afin qu'elle se repente de sa prostitution, mais elle ne s'est pas repentie.
\VS{22}Voici, je vais la jeter sur un lit, et mettre dans une grande affliction ceux qui commettent l'adultère avec elle, s'ils ne se repentent pas de leurs œuvres.
\VS{23}Et je ferai mourir de mort ses enfants~; et toutes les églises connaîtront que je suis celui qui sonde les reins et les cœurs, et je rendrai à chacun de vous selon ses œuvres.
\VS{24}Mais je vous dis à vous et aux autres qui sont à Thyatire, à tous ceux qui n'ont pas cette doctrine, et qui n'ont pas connu les profondeurs de Satan, comme ils disent, je vous dis~: Je ne mettrai pas sur vous d'autre charge.
\VS{25}Mais retenez ce que vous avez, jusqu'à ce que je vienne.
\VS{26}Car à celui qui aura vaincu, et qui aura gardé mes œuvres jusqu'à la fin, je lui donnerai autorité sur les nations.
\VS{27}Et il les gouvernera avec un sceptre de fer, et elles seront brisées comme les vases d'un potier, ainsi que j'en ai moi-même reçu le pouvoir de mon Père.
\VS{28}Et je lui donnerai l'étoile du matin.
\VS{29}Que celui qui a des oreilles entende ce que l'Esprit dit aux églises~!
\Chap{3}
\TextTitle{Sardes~: L'église morte}
\VerseOne{}Ecris aussi à l'ange de l'église de Sardes~: Voici ce que dit celui qui a les sept Esprits de Dieu, et les sept étoiles~: Je connais tes œuvres. Tu as la réputation d'être vivant, mais tu es mort.
\VS{2}Sois vigilant, et affermis le reste qui va mourir~; car je n'ai pas trouvé tes œuvres parfaites devant Dieu.
\VS{3}Souviens-toi donc des choses que tu as reçues et entendues, garde-les, et repens-toi. Si tu ne veilles pas, je viendrai contre toi comme un voleur, et tu ne sauras pas à quelle heure je viendrai contre toi\FTNT{Mt. 24:43~; Lu. 12:39~; 1 Th. 5:2~; 2 Pi. 3:10.}.
\VS{4}Toutefois, tu as quelque peu de personnes à Sardes qui n'ont pas souillé leurs vêtements, et qui marcheront avec moi en vêtements blancs, car ils en sont dignes.
\VS{5}Celui qui vaincra sera vêtu de vêtements blancs, et je n'effacerai pas son nom du Livre de vie, mais je confesserai son nom devant mon Père, et devant ses anges.
\VS{6}Que celui qui a des oreilles, entende ce que l'Esprit dit aux églises~!
\TextTitle{Philadelphie~: L'église réveillée et fidèle}
\VS{7}Ecris aussi à l'ange de l'église de Philadelphie~: Voici ce que dit le Saint et le Véritable, qui a la clef de David, qui ouvre et nul ne ferme, qui ferme et nul n'ouvre.
\VS{8}Je connais tes œuvres. Voici, j'ai ouvert une porte devant toi, et personne ne peut la fermer~; parce que tu as peu de puissance, que tu as gardé ma parole, et que tu n'as pas renié mon Nom.
\VS{9}Voici, je ferai venir ceux de la synagogue de Satan qui se disent Juifs, et ne le sont pas, mais qui mentent~; voici, dis-je, je les ferai venir et se prosterner à tes pieds, et ils connaîtront que je t'aime.
\VS{10}Parce que tu as gardé la parole de la persévérance en moi, je te garderai aussi de l'heure de la tentation qui doit arriver dans le monde entier, pour éprouver les habitants de la terre.
\VS{11}Voici, je viens à toute vitesse. Tiens ferme ce que tu as, afin que personne ne t'enlève ta couronne.
\VS{12}Celui qui vaincra, je ferai de lui une colonne dans le temple de mon Dieu, et il n'en sortira plus~; et j'écrirai sur lui le Nom de mon Dieu, et le nom de la cité de mon Dieu, qui est la nouvelle Jérusalem qui descend du ciel d'auprès de mon Dieu, et mon nouveau Nom.
\VS{13}Que celui qui a des oreilles entende ce que l'Esprit dit aux églises~!
\TextTitle{Laodicée~: L'église apostate}
\VS{14}Ecris aussi à l'ange de l'église de Laodicée~: Voici ce que dit l'Amen, le témoin fidèle et véritable, le commencement de la création de Dieu~:
\VS{15}Je connais tes œuvres. Je sais que tu n'es ni froid ni bouillant~; puisses-tu être ou froid ou bouillant~!
\VS{16}Parce que tu es tiède, et que tu n'es ni froid ni bouillant, je te vomirai de ma bouche.
\VS{17}Car tu dis~: Je suis riche, je suis dans l'abondance, et je n'ai besoin de rien~; mais tu ne sais pas que tu es malheureux, misérable, pauvre, aveugle et nu.
\VS{18}Je te conseille d'acheter de moi de l'or éprouvé par le feu, afin que tu deviennes riche; et des vêtements blancs, afin que tu sois vêtu et que la honte de ta nudité ne paraisse pas~; et d'oindre tes yeux de collyre, afin que tu voies.
\VS{19}Moi, je reprends et châtie tous ceux que j'aime. Aie donc du zèle et repens-toi.
\TextTitle{Le Messie se retrouve hors des églises apostates}
\VS{20}Voici, je me tiens à la porte, et je frappe. Si quelqu'un entend ma voix et m'ouvre la porte, j'entrerai chez lui, et je souperai avec lui, et lui avec moi.
\VS{21}Celui qui vaincra, je le ferai asseoir avec moi sur mon trône, ainsi que j'ai vaincu et me suis assis avec mon Père sur son trône.
\VS{22}Que celui qui a des oreilles entende ce que l'Esprit dit aux églises~!
\Chap{4}
\TextTitle{Vision avant l'ouverture des sceaux}
\VerseOne{}Après ces choses, je regardai, et voici une porte était ouverte dans le ciel. Et la première voix que j'avais entendue, comme le son d'une trompette, et qui parlait avec moi, me dit~: Monte ici, et je te montrerai les choses qui doivent arriver à l'avenir.
\VS{2}Aussitôt, je fus ravi en esprit. Et voici, un trône était dressé dans le ciel, et sur ce trône, quelqu'un était assis.
\VS{3}Et celui qui y était assis était semblable à une pierre de jaspe et de sardoine~; et le trône était environné d'un arc-en-ciel semblable à de l'émeraude.
\TextTitle{Les trônes des vingt-quatre vieillards}
\VS{4}Et il y avait autour du trône vingt-quatre trônes et je vis sur ces trônes vingt-quatre anciens assis, vêtus de vêtements blancs, et ayant sur leurs têtes des couronnes d'or.
\VS{5}Et du trône sortaient des éclairs, des tonnerres, et des voix~; et il y avait devant le trône sept lampes de feu ardentes, qui sont les sept Esprits de Dieu.
\TextTitle{Le Messie est digne de recevoir la louange et la gloire}
\VS{6}Et devant le trône, il y avait une mer de verre semblable à du cristal~; et au milieu du trône et autour du trône quatre animaux, pleins d'yeux devant et derrière.
\VS{7}Et le premier animal était semblable à un lion~; le second animal était semblable à un veau~; le troisième animal avait la face comme un homme~; et le quatrième animal était semblable à un aigle qui vole.
\VS{8}Et les quatre animaux avaient chacun six ailes, et tout autour et au-dedans ils étaient pleins d'yeux~; et ils ne cessent pas de dire jour et nuit~: Saint~! Saint~! Saint est le Seigneur Dieu Tout-puissant, QUI ETAIT, QUI EST, et QUI VIENT.
\VS{9}Et quand ces animaux rendaient gloire et honneur et des actions de grâces à celui qui était assis sur le trône, à celui qui est vivant aux siècles des siècles,
\VS{10}les vingt-quatre anciens se prosternaient devant celui qui était assis sur le trône, et adoraient celui qui est vivant aux siècles des siècles, et ils jetaient leurs couronnes devant le trône, en disant~:
\VS{11}Seigneur, tu es digne de recevoir gloire, honneur et puissance~; car tu as créé toutes choses, et c'est par ta volonté qu'elles existent et qu'elles ont été créées.
\Chap{5}
\TextTitle{Le Messie est le seul digne d'ouvrir le livre}
\VerseOne{}Puis je vis dans la main droite de celui qui était assis sur le trône, un livre écrit en dedans et en dehors, scellé de sept sceaux.
\VS{2}Et je vis aussi un ange remarquable par sa force, qui proclamait d'une voix forte~: Qui est digne d'ouvrir le livre, et d'en rompre les sceaux~?
\VS{3}Et il n'y avait personne, ni dans le ciel, ni sur la terre, ni sous la terre qui pouvait ouvrir le livre, ni le regarder.
\VS{4}Et je pleurais beaucoup parce que personne n'était trouvé digne d'ouvrir le livre, ni de le lire, ni de le regarder.
\VS{5}Et l'un des anciens me dit~: Ne pleure pas, voici le Lion qui vient de la tribu de Juda, de la racine de David, a vaincu pour ouvrir le livre et pour en rompre les sept sceaux.
\VS{6}Et je regardai, et voici il y avait au milieu du trône et des quatre animaux, et au milieu des anciens, un Agneau qui se tenait là comme immolé, ayant sept cornes, et sept yeux, qui sont les sept Esprits de Dieu envoyés par toute la terre.
\VS{7}Et il vint et prit le livre de la main droite de celui qui était assis sur le trône.
\TextTitle{L'Agneau est adoré\FTNTT{Ph. 2:9-11.}}
\VS{8}Et quand il eut pris le livre, les quatre animaux et les vingt-quatre anciens se prosternèrent devant l'Agneau, ayant chacun des harpes et des coupes d'or pleines de parfums, qui sont les prières des saints.
\VS{9}Et ils chantaient un cantique nouveau, en disant~: Tu es digne de prendre le livre, et d'en ouvrir les sceaux~; car tu as été mis à mort, et tu nous as rachetés pour Dieu par ton sang, de toute tribu, de toute langue, de tout peuple, et de toute nation~;
\VS{10}et tu as fait de nous des rois et des prêtres pour notre Dieu~; et nous régnerons sur la terre.
\VS{11}Puis je regardai, et j'entendis la voix de plusieurs anges autour du trône, et des anciens; et leur nombre était de plusieurs millions.
\VS{12}Et ils disaient à haute voix~: L'Agneau qui a été mis à mort est digne de recevoir puissance, richesses, sagesse, force, honneur, gloire et louange.
\VS{13}J'entendis aussi toutes les créatures qui sont dans le ciel, sur la terre, et sous la terre, et dans la mer, et toutes les choses qui y sont, disant~: A celui qui est assis sur le trône et à l'Agneau, soient louange, honneur, gloire, et force, aux siècles des siècles~!
\VS{14}Et les quatre animaux disaient~: Amen~! Et les vingt-quatre vieillards se prosternèrent et adorèrent celui qui est vivant aux siècles des siècles.
\Chap{6}
\TextTitle{Premier sceau~: Le cavalier qui part pour vaincre}
\VerseOne{}Et quand l'Agneau eut ouvert l'un des sceaux, je regardai, et j'entendis l'un des quatre animaux qui disait comme avec une voix de tonnerre~: Viens, et vois.
\VS{2}Je regardai, et je vis un cheval blanc~; celui qui était monté dessus avait un arc, et il lui fut donné une couronne~; et il est sortit en vainqueur pour vaincre\FTNT{Contrairement aux apparences, ce cavalier couronné d'un diadème et qui monte un cheval blanc n'est pas Jésus-Christ, mais l'Antichrist qui singe le retour glorieux du Seigneur~: Da. 7:21~; Mt. 24:4-5~; 2 Th. 2:9-12~; Ap. 13:7. Le vrai Christ revenant triomphalement avec son Eglise est décrit en Ap. 19:11-16.}.
\TextTitle{Deuxième sceau~: La guerre}
\VS{3}Et quand il eut ouvert le second sceau, j'entendis le second animal qui disait~: Viens, et vois.
\VS{4}Et il sortit un autre cheval qui était roux~; il fut donné à celui qui était monté dessus de pouvoir ôter la paix de la terre, afin que les hommes se tuent les uns les autres~; et il lui fut donné une grande épée.
\TextTitle{Troisième sceau~: La famine}
\VS{5}Et quand il eut ouvert le troisième sceau, j'entendis le troisième animal qui disait~: Viens, et vois. Je regardai, et je vis un cheval noir, et celui qui était monté dessus avait une balance dans sa main.
\VS{6}Et j'entendis au milieu des quatre animaux une voix qui disait~: Une mesure de blé pour un denier, et les trois mesures d'orge pour un denier~; mais ne fais pas de mal au vin et à l'huile.
\TextTitle{Quatrième sceau~: La mort}
\VS{7}Et quand il eut ouvert le quatrième sceau, j'entendis la voix du quatrième animal qui disait~: Viens, et vois.
\VS{8}Je regardai, et je vis un cheval verdâtre~; et celui qui était monté dessus se nommait la Mort, et le Hadès l'accompagnait. Il leur fut donné le pouvoir sur le quart de la terre pour tuer par l'épée, par la famine, par la mortalité, et par les bêtes sauvages de la terre.
\TextTitle{Cinquième sceau~: Les martyrs}
\VS{9}Et quand il eut ouvert le cinquième sceau, je vis sous l'autel les âmes de ceux qui avaient été tués pour la parole de Dieu, et pour le témoignage qu'ils avaient gardé.
\VS{10}Et elles criaient à haute voix, disant~: Jusqu'à quand, Seigneur qui es saint et véritable, ne jugeras-tu pas et ne vengeras-tu pas notre sang de ceux qui habitent sur la terre~?
\VS{11}Et il leur fut donné à chacun des robes blanches, et il leur fut dit de se tenir en repos encore un peu de temps, jusqu'à ce que le nombre de leurs compagnons de service, et de leurs frères qui doivent être mis à mort comme eux, soit complet.
\TextTitle{Sixième sceau~: L'anarchie}
\VS{12}Et je regardai quand il eut ouvert le sixième sceau, et voici, il se fit un grand tremblement de terre, et le soleil devint noir comme un sac de crin, et la lune entière devint comme du sang.
\VS{13}Et les étoiles du ciel tombèrent sur la terre\FTNT{Mt. 24:29~; Mc. 13:25.}, comme lorsque le figuier est agité par un grand vent et laisse tomber ses figues encore vertes.
\VS{14}Et le ciel se retira comme un livre qu'on roule~; et toutes les montagnes et les îles furent remuées de leurs places.
\VS{15}Et les rois de la terre, et les princes, et les riches, et les capitaines, et les puissants, et tout esclave, et tout homme libre se cachèrent dans les cavernes et entre les rochers des montagnes.
\VS{16}Et ils disaient aux montagnes et aux rochers~: Tombez sur nous\FTNT{Lu. 23:30.}, et cachez-nous devant la face de celui qui est assis sur le trône, et devant la colère de l'Agneau~;
\VS{17}car le grand jour de sa colère est venu, et qui peut subsister~?
\Chap{7}
\TextTitle{Les 144 000 marqués du sceau de Dieu}
\VerseOne{}Après cela, je vis quatre anges qui se tenaient aux quatre coins de la terre, et qui retenaient les quatre vents de la terre, afin qu'ils ne soufflent pas sur la terre, ni sur la mer, ni sur aucun arbre.
\VS{2}Puis je vis un autre ange qui montait du côté de l'orient, tenant le sceau du Dieu vivant, et il cria d'une voix forte aux quatre anges à qui il avait été donné de faire du mal à la terre et à la mer,
\VS{3}et leur dit~: Ne faites pas de mal à la terre, ni à la mer, ni aux arbres, jusqu'à ce que nous ayons marqué du sceau les serviteurs de notre Dieu sur leurs fronts.
\VS{4}Et j'entendis que le nombre de ceux qui avaient été marqués du sceau était de cent quarante-quatre mille, de toutes les tribus des enfants d'Israël.
\VS{5}De la tribu de Juda, douze mille marqués du sceau~; de la tribu de Ruben, douze mille marqués du sceau~; de la tribu de Gad, douze mille marqués du sceau~;
\VS{6}De la tribu d'Aser, douze mille marqués du sceau~; de la tribu de Nephthali, douze mille marqués du sceau~; de la tribu de Manassé, douze mille marqués du sceau~;
\VS{7}De la tribu de Siméon, douze mille marqués du sceau~; de la tribu de Lévi, douze mille marqués du sceau~; de la tribu d'Issacar, douze mille marqués du sceau~;
\VS{8}De la tribu de Zabulon, douze mille marqués du sceau~; de la tribu de Joseph, douze mille marqués du sceau~; de la tribu de Benjamin, douze mille marqués du sceau.
\TextTitle{Multitude de sauvés pendant la grande tribulation}
\VS{9}Après cela, je regardai, et voici une grande multitude de gens, que personne ne pouvait compter, de toute nation, de toute tribu, de tout peuple et de toute langue, se tenaient devant le trône, et devant l'Agneau, vêtus de longues robes blanches, et ils avaient des palmes dans leurs mains.
\VS{10}Et ils criaient d'une voix forte, en disant~: Le salut est à notre Dieu, qui est assis sur le trône, et à l'Agneau.
\VS{11}Et tous les anges se tenaient autour du trône, et des anciens, et des quatre animaux, et ils se prosternèrent devant le trône sur leurs faces et adorèrent Dieu,
\VS{12}en disant~: Amen~! La louange, la gloire, la sagesse, les actions de grâces, l'honneur, la puissance et la force soient à notre Dieu, aux siècles des siècles. Amen~!
\VS{13}Et l'un des anciens prit la parole et me dit~: Ceux qui sont revêtus de longues robes blanches, qui sont-ils et d'où sont-ils venus~?
\VS{14}Et je lui dis~: Seigneur, tu le sais. Et il me dit~: Ce sont ceux qui sont venus de la grande tribulation\FTNT{Les saints ont toujours été persécutés. Cela a débuté dès la Genèse avec Caïn qui tua son frère Abel (Ge. 4:5-10). La grande tribulation correspond néanmoins à une période de persécutions particulièrement cruelles qui seront orchestrées par l'homme impie (à la tête de plusieurs nations) principalement contre les Juifs (Jé. 30:7~; Da. 9:24~; Lu. 21:20-24) et sans doute contre les personnes converties à Christ issues des nations (Ap. 7:9-17~; Ap. 12:17). Le Seigneur Jésus a prédit la grande tribulation à ses disciples (Mt. 24:15-29~; Mc. 13:14-19) en précisant qu'en ce temps là on verrait «~l'abomination de la désolation~» établie en lieu saint prophétisée par Daniel (Da. 11:31). La grande tribulation durera trois ans et demi, c'est ce que Daniel appelle «~un temps, des temps et la moitié d'un temps~» (Da. 7:25~; Ap. 11:3.) L'ère de paix factice instaurée par l'impie cédera alors soudainement la place à un temps d'angoisse sans précédent (1 Th. 5:3).}, et qui ont lavé et blanchi leurs longues robes dans le sang de l'Agneau.
\VS{15}C'est pourquoi ils sont devant le trône de Dieu, et ils le servent jour et nuit dans son temple~; et celui qui est assis sur le trône habitera avec eux.
\VS{16}Ils n'auront plus faim ni soif, et le soleil ne les frappera plus, ni aucune chaleur.
\VS{17}Car l'Agneau qui est au milieu du trône les paîtra, et les conduira aux sources des eaux de la vie, et Dieu essuiera toutes les larmes de leurs yeux.
\Chap{8}
\TextTitle{Septième sceau~: Annonce des sept trompettes\FTNTT{Ap. 4:1.}}
\VerseOne{}Et quand il eut ouvert le septième sceau, il y eut un silence dans le ciel d'environ une demi-heure.
\VS{2}Et je vis les sept anges qui se tiennent devant Dieu, et sept trompettes leur furent données.
\VS{3}Et un autre ange vint et se tint devant l'autel, ayant un encensoir d'or, et plusieurs parfums lui furent donnés pour les offrir, avec les prières de tous les saints, sur l'autel d'or qui est devant le trône.
\VS{4}Et la fumée des parfums monta avec les prières des saints de la main de l'ange devant Dieu.
\VS{5}Puis l'ange prit l'encensoir, et l'ayant rempli du feu de l'autel, il le jeta sur la terre~; et il y eut des tonnerres, des voix, des éclairs, et un tremblement de terre.
\VS{6}Alors les sept anges qui avaient les sept trompettes se préparèrent à en sonner.
\TextTitle{Première trompette~: Grêle et feu mêlés de sang}
\VS{7}Et le premier ange sonna de la trompette. Et il y eut de la grêle et du feu mêlés de sang, qui furent jetés sur la terre~; et le tiers des arbres fut brûlé, et toute herbe verte aussi fut brûlée.
\TextTitle{Deuxième trompette~: La montagne embrasée}
\VS{8}Et le second ange sonna de la trompette, et je vis comme une grande montagne embrasée de feu, qui fut jetée dans la mer~; et le tiers de la mer devint du sang,
\VS{9}et le tiers des créatures vivantes qui étaient dans la mer mourut, et le tiers des navires périt.
\TextTitle{Troisième trompette~: Absinthe, l'étoile tombée du ciel}
\VS{10}Et le troisième ange sonna de la trompette, et il tomba du ciel une grande étoile ardente comme un flambeau, et elle tomba sur le tiers des fleuves et sur les sources des eaux.
\VS{11}Le nom de l'étoile est Absinthe~; et le tiers des eaux fut changé en absinthe, et beaucoup d'hommes moururent par les eaux, parce qu'elles étaient devenues amères.
\TextTitle{Quatrième trompette~: Des signes dans le ciel}
\VS{12}Puis le quatrième ange sonna de la trompette, et le tiers du soleil fut frappé, ainsi que le tiers de la lune, et le tiers des étoiles, afin que le tiers en soit obscurci~; le jour fut privé d'un tiers de sa clarté, et la nuit de même.
\VS{13}Je regardai, et j'entendis un ange qui volait au milieu du ciel et qui disait à haute voix~: Malheur~! Malheur~! Malheur aux habitants de la terre à cause des autres sons de trompettes que les trois autres anges vont faire retentir.
\Chap{9}
\TextTitle{Cinquième trompette~: Ouverture du puits de l'abîme}
\VerseOne{}Le cinquième ange sonna de la trompette, et je vis une étoile qui tomba du ciel sur la terre, et la clef du puits de l'abîme fut donnée à cet ange.
\VS{2}Et il ouvrit le puits de l'abîme, et une fumée monta du puits comme la fumée d'une grande fournaise~; et le soleil et l'air furent obscurcis par la fumée du puits.
\VS{3}Des sauterelles sortirent de la fumée du puits et se répandirent sur la terre, et il leur fut donné un pouvoir comme le pouvoir qu'ont les scorpions de la terre.
\VS{4}Et il leur fut dit de ne pas faire de mal à l'herbe de la terre, ni à aucune verdure, ni à aucun arbre, mais seulement aux hommes qui n'avaient pas la marque de Dieu sur leurs fronts.
\VS{5}Et il leur fut donné, non de les tuer, mais de les tourmenter pendant cinq mois~; et le tourment qu'elles causaient était comme le tourment que cause le scorpion quand il pique un homme.
\VS{6}Et en ces jours-là, les hommes chercheront la mort, mais ils ne la trouveront pas~; et ils désireront mourir, mais la mort fuira loin d'eux.
\VS{7}Ces sauterelles ressemblaient à des chevaux préparés pour la guerre, et sur leurs têtes il y avait comme des couronnes semblables à de l'or, et leurs faces étaient comme des faces d'hommes.
\VS{8}Elles avaient les cheveux comme des cheveux de femmes~; et leurs dents étaient comme des dents de lions.
\VS{9}Elles avaient des cuirasses comme des cuirasses de fer~; et le bruit de leurs ailes était comme le bruit des chars à plusieurs chevaux qui courent à la guerre.
\VS{10}Elles avaient des queues armées d'aiguillons, comme les scorpions, et c'est dans leurs queues qu'était le pouvoir de faire du mal aux hommes pendant cinq mois.
\VS{11}Elles avaient sur elles comme roi l'ange de l'abîme, dont le nom en hébreu est Abaddon, mais en grec son nom est Apollyon\FTNT{Abaddon ou Apollyon~: Le nom de ce démon signifie «~Le destructeur~».}.
\VS{12}Le premier malheur est passé, et voici venir encore deux malheurs après celui-ci.
\TextTitle{Sixième trompette~: Les quatre anges de l'Euphrate déliés\FTNTT{Ap. 16:12.}}
\VS{13}Alors le sixième ange sonna de sa trompette, et j'entendis une voix sortant des quatre cornes de l'autel d'or qui est devant Dieu,
\VS{14}et disant au sixième ange qui avait la trompette~: Délie les quatre anges qui sont liés sur le grand fleuve, l'Euphrate.
\VS{15}On délia donc les quatre anges qui étaient prêts pour l'heure, le jour, le mois et l'année, afin de tuer le tiers des hommes.
\VS{16}Le nombre des cavaliers de l'armée était de deux cents millions, car j'en entendis le nombre.
\VS{17}Et je vis aussi dans la vision les chevaux et ceux qui étaient montés dessus, ayant des cuirasses de feu, d'hyacinthe et de soufre~; et les têtes des chevaux étaient comme des têtes de lions~; et de leurs bouches sortaient du feu, de la fumée et du soufre.
\VS{18}Le tiers des hommes fut tué par ces trois fléaux, par le feu, et par la fumée et par le soufre qui sortaient de leur bouche.
\VS{19}Car le pouvoir des chevaux était dans leurs bouches et dans leurs queues~; et leurs queues étaient semblables à des serpents ayant des têtes, et c'est avec elles qu'ils faisaient du mal.
\VS{20}Mais les autres hommes qui ne furent pas tués par ces fléaux, ne se repentirent pas des œuvres de leurs mains, ils ne cessèrent pas d'adorer les démons, les idoles d'or, d'argent, de cuivre, de pierre, et de bois, qui ne peuvent ni voir, ni entendre, ni marcher.
\VS{21}Et ils ne se repentirent pas aussi de leurs meurtres, ni de leurs enchantements, ni de leur impudicité, ni de leurs vols.
\Chap{10}
\TextTitle{Un ange puissant descend du ciel}
\VerseOne{}Je vis un autre ange puissant qui descendait du ciel, environné d'une nuée, au-dessus de sa tête était l'arc-en-ciel, son visage était comme le soleil, et ses pieds comme des colonnes de feu.
\VS{2}Et il avait dans sa main un petit livre ouvert, et il posa son pied droit sur la mer, et le pied gauche sur la terre~;
\VS{3}et il cria d'une voix forte, comme lorsqu'un lion rugit. Et quand il eut crié, les sept tonnerres firent entendre leurs voix.
\VS{4}Et après que les sept tonnerres eurent fait entendre leurs voix, j'allais écrire, mais j'entendis une voix du ciel qui me disait~: Scelle les choses que les sept tonnerres ont fait entendre, et ne les écris pas.
\VS{5}Et l'ange que j'avais vu se tenant sur la mer et sur la terre, leva sa main vers le ciel,
\VS{6}et jura par celui qui est vivant aux siècles des siècles, qui a créé le ciel avec les choses qui y sont, et la terre avec les choses qui y sont, et la mer avec les choses qui y sont, qu'il n'y aurait plus de temps~;
\VS{7}mais qu'aux jours de la voix du septième ange, quand il commencera à sonner de la trompette, le mystère de Dieu sera accompli, comme il l'a déclaré à ses serviteurs les prophètes.
\TextTitle{Nouvelle mission de Jean}
\VS{8}Et la voix que j'avais entendue du ciel me parla encore et me dit~: Va, et prends le petit livre ouvert qui est dans la main de l'ange qui se tient sur la mer et sur la terre.
\VS{9}Et j'allai vers l'ange, en lui disant~: Donne-moi le petit livre~; et il me dit~: Prends-le et mange-le~; il remplira tes entrailles d'amertume, mais il sera doux dans ta bouche comme du miel\FTNT{Ez. 3:1-3.}.
\VS{10}Je pris donc le petit livre de la main de l'ange, et je le mangeai~; il fut doux dans ma bouche comme du miel, mais quand je l'eus mangé, mes entrailles furent remplies d'amertume.
\VS{11}Alors il me dit~: Il faut que tu prophétises de nouveau sur beaucoup de peuples, et sur plusieurs nations, sur plusieurs langues et plusieurs rois.
\Chap{11}
\TextTitle{Le temps des nations}
\VerseOne{}On me donna un roseau semblable à une verge, et l'ange se présenta et me dit~: Lève-toi et mesure le temple de Dieu et l'autel, et ceux qui y adorent.
\VS{2}Mais laisse de côté le parvis extérieur du temple, et ne le mesure pas~; car il est donné aux Gentils, et ils fouleront aux pieds la ville sainte pendant quarante-deux mois\FTNT{C'est le temps que durera la grande tribulation, soit trois ans et demi. Daniel parle d'une semaine, un jour comptant pour une année (Da. 9:27). La grande tribulation débutera à la moitié de cette semaine, ce qui correspond bien à quarante-deux mois (Ap. 13:5) et à mille deux cent soixante jours (Ap. 11:3~; Ap. 12:6).}.
\TextTitle{Les deux témoins ressuscitent}
\VS{3}Mais je donnerai à mes deux témoins de prophétiser pendant mille deux cent soixante jours, revêtus de sacs.
\VS{4}Ce sont les deux oliviers\FTNT{Za. 4:14.} et les deux chandeliers qui se tiennent devant le Dieu de la terre. 
\VS{5}Et si quelqu'un veut leur faire du mal, du feu sort de leurs bouches et dévore leurs ennemis~; car si quelqu'un veut leur faire du mal, il faut qu'il soit tué de cette manière.
\VS{6}Ils ont le pouvoir de fermer le ciel, afin qu'il ne pleuve pas pendant les jours de leur prophétie~; ils ont aussi le pouvoir de changer les eaux en sang, et de frapper la terre de toutes sortes de plaies, toutes les fois qu'ils le voudront.
\VS{7}Et quand ils auront achevé de rendre leur témoignage, la bête qui monte de l'abîme\FTNT{L'homme impie, l'Antichrist, ou encore le fils de la perdition dont il est question dans 2 Th. 2:3~; 2 Th. 2:8-9.} leur fera la guerre, les vaincra, et les tuera.
\VS{8}Et leurs cadavres seront étendus sur les places de la grande ville, qui est appelée spirituellement Sodome et Egypte, où aussi notre Seigneur a été crucifié.
\VS{9}Et ceux des tribus, des peuples, des langues, et des nations verront leurs cadavres pendant trois jours et demi, et ils ne permettront pas que leurs cadavres soient mis dans des sépulcres.
\VS{10}Et les habitants de la terre se réjouiront, ils seront dans l'allégresse, ils s'enverront des présents les uns aux autres, parce que ces deux prophètes ont tourmenté les habitants de la terre.
\VS{11}Mais après ces trois jours et demi, l'Esprit de vie venant de Dieu entra en eux, et ils se tinrent sur leurs pieds, et une grande crainte saisit ceux qui les virent.
\VS{12}Après cela, ils entendirent une forte voix du ciel, leur disant~: Montez ici~! Et ils montèrent au ciel sur une nuée, et leurs ennemis les virent.
\VS{13}Et à cette même heure-là, il eut un grand tremblement de terre, et la dixième partie de la ville tomba, et sept mille hommes furent tués par ce tremblement de terre~; et les autres furent épouvantés et donnèrent gloire au Dieu du ciel.
\VS{14}Le second malheur est passé. Voici, le troisième malheur vient bientôt.
\TextTitle{Septième trompette~: Le règne du Messie annoncé, cantique des vingt-quatre vieillards\FTNTT{Ap. 8:2.}}
\VS{15}Le septième ange sonna de la trompette, et il se fit entendre au ciel de grandes voix qui disaient~: Les royaumes du monde sont soumis à notre Seigneur et à son Christ, et il régnera aux siècles des siècles.
\VS{16}Alors les vingt-quatre anciens qui étaient assis devant Dieu sur leurs trônes, se prosternèrent sur leurs faces et adorèrent Dieu,
\VS{17}en disant~: Nous te rendons grâces, Seigneur Dieu Tout-Puissant, QUI ES, QUI ETAIS, et QUI VIENS, de ce que tu as fait éclater ta grande puissance, et de ce que tu as agi en Roi.
\VS{18}Les nations se sont irritées, mais ta colère est venue, et le temps est venu de juger les morts, et de donner la récompense à tes serviteurs les prophètes et aux saints, et à ceux qui craignent ton Nom, petits et grands, et de détruire ceux qui corrompent la terre.
\VS{19}Et le temple de Dieu fut ouvert dans le ciel, et l'arche de son alliance apparut dans son temple. Et il y eut des éclairs, des voix, des tonnerres, un tremblement de terre, et une grosse grêle.
\Chap{12}
\TextTitle{Vision de la femme et du dragon}
\VerseOne{}Et un grand signe parut dans le ciel~: Une femme revêtue du soleil, la lune sous ses pieds, et sur sa tête une couronne de douze étoiles\FTNT{En Ge. 37:9-10, Joseph raconte à ses parents et à ses frères un songe particulier où il voyait le soleil, la lune et onze étoiles se prosterner devant lui. Jacob comprit que les onze étoiles représentaient ses enfants, la lune sa femme Rachel, qui était la mère de Joseph, et que le soleil c'était lui-même. Il est donc question ici d'Israël, qui a toujours été identifié à une femme (Ez. 16) de qui est issu le Messie selon la chair (Ro. 9:5).}.
\VS{2}Elle était enceinte, et elle criait, étant en travail d'enfant, souffrant les grandes douleurs de l'enfantement.
\VS{3}Il parut aussi un autre signe dans le ciel, et voici un grand dragon rouge feu ayant sept têtes et dix cornes, et sur ses têtes sept diadèmes.
\VS{4}Sa queue entraînait le tiers des étoiles du ciel et les jeta sur la terre\FTNT{Da. 8:10.}. Puis le dragon s'arrêta devant la femme qui devait accoucher, afin de dévorer son enfant\FTNT{Cet enfant est évidemment Jésus-Christ (Mt. 2:16.)}, dès qu'elle l'aurait mis au monde.
\TextTitle{La naissance du Messie}
\VS{5}Et elle accoucha d'un fils, qui doit gouverner toutes les nations avec un sceptre de fer\FTNT{Ps. 2:8-9.}. Et son enfant fut enlevé vers Dieu et vers son trône\FTNT{Lu. 24:51~; Ac. 1:9-11.}.
\VS{6}Et la femme s'enfuit dans un désert, où elle avait un lieu préparé par Dieu, afin d'y être nourrie pendant mille deux cent soixante jours.
\TextTitle{Guerre entre l'archange Michel et le dragon}
\VS{7}Et il y eut une guerre dans le ciel. Michel et ses anges combattirent contre le dragon. Et le dragon et ses anges combattirent contre Michel,
\VS{8}mais ils ne furent pas les plus forts, et leur place ne fut plus trouvée dans le ciel.
\VS{9}Et il fut précipité le grand dragon, le serpent ancien, appelé le diable et Satan, celui qui séduit toute la terre, il fut précipité sur la terre, et ses anges furent précipités avec lui\FTNT{Es. 14:12-15~; Ez. 28~; Lu. 10:18.}.
\VS{10}Et j'entendis une voix forte dans le ciel qui disait~: Maintenant le salut est arrivé, ainsi que la force, le règne de notre Dieu, et la puissance de son Christ~; car l'accusateur de nos frères, qui les accusait devant notre Dieu jour et nuit, a été précipité.
\VS{11}Et ils l'ont vaincu à cause du sang de l'Agneau, et à cause de la parole de leur témoignage, et ils n'ont pas aimé leurs vies, mais les ont exposées à la mort.
\VS{12}C'est pourquoi réjouissez-vous cieux, et vous qui y habitez. Mais malheur à vous habitants de la terre et de la mer~! Car le diable est descendu vers vous animé d'une grande fureur, sachant qu'il a peu de temps.
\TextTitle{Le dragon persécute la femme, sa postérité et les témoins du Messie}
\VS{13}Quand le dragon vit qu'il avait été précipité sur la terre, il persécuta la femme qui avait enfanté le fils.
\VS{14}Mais deux ailes d'un grand aigle furent données à la femme, afin qu'elle s'envole de devant le serpent au désert, où elle est nourrie un temps, des temps, et la moitié d'un temps.
\VS{15}Et de sa gueule, le serpent lança de l'eau comme un fleuve derrière la femme, afin de l'entraîner par le fleuve.
\VS{16}Mais la terre secourut la femme, elle ouvrit sa bouche, et elle engloutit le fleuve que le dragon avait lancé de sa gueule.
\VS{17}Alors le dragon fut irrité contre la femme, et s'en alla faire la guerre contre les autres qui sont de la semence de la femme, qui gardent les commandements de Dieu, et qui ont le témoignage de Jésus-Christ.
\VS{18}Et je me tins sur le sable qui borde la mer.
\Chap{13}
\TextTitle{La bête qui monte de la mer, l'antichrist}
\VerseOne{}Et je vis monter de la mer une bête\FTNT{Cette bête représente deux entités. Tout d'abord l'homme impie, l'Antichrist, et ensuite un système politique. Les dix cornes sur sa tête symbolisent les dix nations les plus puissantes de la terre avec lesquelles il imposera sa dictature mondiale (Da. 7:16-25). L'alliage des quatre métaux dans la statue de Nébucadnetsar en Da. 2 et la vision des quatre animaux en Da. 7, annoncent l'instauration d'un quatrième empire ou encore le système politique à la tête duquel sera la bête.} qui avait sept têtes et dix cornes, et sur ses cornes dix diadèmes, et sur ses têtes des noms de blasphème\FTNT{Voir annexe «~La bête d'apocalypse~».}.
\VS{2}Et la bête que je vis était semblable à un léopard, ses pieds étaient comme ceux d'un ours~; sa gueule était comme la gueule d'un lion\FTNT{Da. 7:7.}. Et le dragon lui donna sa puissance, son trône, et une grande autorité.
\VS{3}Et je vis l'une de ses têtes comme blessée à mort, mais sa blessure mortelle fut guérie. Remplie d'admiration, la terre entière suivit la bête.
\VS{4}Et ils adorèrent le dragon, parce qu'il avait donné l'autorité à la bête, et ils adorèrent aussi la bête, en disant~: Qui est semblable à la bête, et qui peut combattre contre elle~?
\VS{5}Et il lui fut donné une bouche qui proférait des discours pleins d'orgueil, et des blasphèmes~; et il lui fut aussi donné le pouvoir d'agir pendant quarante-deux mois.
\VS{6}Elle ouvrit sa bouche pour blasphémer contre Dieu, pour blasphémer son Nom et son tabernacle, et ceux qui habitent dans le ciel.
\VS{7}Et il lui fut donné de faire la guerre aux saints et de les vaincre. Il lui fut aussi donné autorité sur toute tribu, toute langue et toute nation.
\VS{8}Et tous les habitants de la terre l'adoreront, ceux dont les noms n'ont pas été écrits dans le livre de vie de l'Agneau immolé dès la fondation du monde.
\VS{9}Si quelqu'un a des oreilles qu'il entende.
\VS{10}Si quelqu'un est destiné à la captivité, il ira en captivité~; si quelqu'un tue avec l'épée, il faut qu'il soit lui-même tué avec l'épée. C'est ici la persévérance et la foi des saints.
\TextTitle{La bête qui monte de la terre, le faux prophète}
\VS{11}Puis je vis une autre bête qui montait de la terre\FTNT{Cette bête est identifiée au faux-prophète car son rôle consiste à amener les habitants de la terre à adorer la première bête, tout comme les vrais prophètes invitent les gens à l'adoration du Dieu véritable (Mt. 7:15).}, et qui avait deux cornes semblables à celles de l'Agneau~; mais elle parlait comme le dragon.
\VS{12}Et elle exerçait toute l'autorité de la première bête en sa présence, et elle obligeait la terre et ses habitants à adorer la première bête, dont la blessure mortelle avait été guérie\FTNT{Cette bête a existé par le passé sous la forme de l'empire romain qui s'est écroulé le 4 septembre 476. Ce régime a marqué l'histoire par son caractère universel et brutal. Le fait que cette bête blessée à mort reprenne vie, annonce l'instauration d'un empire universel qui aura les caractéristiques combinées de l'empire babylonien, médo-perse, gréco-macédonien et romain, ceux-ci correspondant aux quatre animaux de la vision de Da. 7:1-8~: Le lion, l'ours, le léopard et le quatrième animal.}.
\VS{13}Elle opérait de grands prodiges, même jusqu'à faire descendre le feu du ciel sur la terre devant les hommes.
\VS{14}Et elle séduisait les habitants de la terre, à cause des prodiges qu'il lui était donné d'opérer en présence de la bête, disant aux habitants de la terre de faire une image\FTNT{Dieu interdit la vénération des images (Ex. 20:4-5) La particularité de l'image de la bête est qu'elle possède un esprit (démon).} de la bête qui avait reçu le coup mortel de l'épée, et qui était bien vivante.
\VS{15}Et il lui fut donné de mettre un esprit à l'image de la bête, afin que même l'image de la bête parle, et qu'elle fasse que tous ceux qui n'adoreraient pas l'image de la bête soient mis à mort.
\VS{16}Elle fit que tous, petits et grands, riches et pauvres, libres et esclaves, reçoivent une marque sur leur main droite, ou sur leur front\FTNT{Il s'agit d'une marque qui est avant tout spirituelle. Car de la même façon que nous sommes scellés et marqués par l'Esprit de Dieu qui produit en nous la sainteté (Ga. 5:22~; Ro. 6:20-22~; Ep. 1:13~; Ep. 4:30) Satan marque les siens par le péché (1 Ti. 4:1-2~; 2 Ti. 3:1-5).}~;
\VS{17}et que personne ne puisse acheter ni vendre, sans avoir la marque ou le nom de la bête, ou le nombre de son nom.
\VS{18}Ici est la sagesse~: Que celui qui a de l'intelligence compte le nombre de la bête, car c'est un nombre d'homme, et son nombre est six cent soixante-six.
\Chap{14}
\TextTitle{L'Agneau et les 144 000}
\VerseOne{}Puis je regardai, et voici, l'Agneau se tenait sur la montagne de Sion, et il y avait avec lui cent quarante-quatre mille personnes qui avaient son Nom et le Nom de son Père écrit sur leurs fronts.
\VS{2}Et j'entendis une voix du ciel comme le bruit des grandes eaux, et comme le bruit d'un grand tonnerre~; et j'entendis une voix de joueurs de harpe jouant de leurs harpes.
\VS{3}Et ils chantaient comme un cantique nouveau devant le trône, et devant les quatre animaux, et devant les anciens. Et personne ne pouvait apprendre le cantique, si ce n'est les cent quarante-quatre mille qui avaient été rachetés de la terre.
\VS{4}Ce sont ceux qui ne se sont pas souillés avec les femmes, car ils sont vierges~; ce sont ceux qui suivent l'Agneau partout où il va. Ils ont été rachetés d'entre les hommes pour être des prémices pour Dieu et pour l'Agneau.
\VS{5}Et dans leur bouche il ne s'est pas trouvé de fraude, car ils sont sans tache devant le trône de Dieu\FTNT{Ps. 32:2.}.
\TextTitle{l'Evangile éternel et la chute de Babylone}
\VS{6}Puis je vis un autre ange qui volait au milieu du ciel, il avait l'Evangile éternel pour évangéliser les habitants de la terre, de toute nation, de toute tribu, de toute langue et de tout peuple.
\VS{7}Il disait d'une voix forte~: Craignez Dieu, et donnez-lui gloire, car l'heure de son jugement est venue~; et adorez celui qui a fait le ciel et la terre, la mer et les sources des eaux.
\VS{8}Et un autre ange le suivit, disant~: Elle est tombée, elle est tombée Babylone, la grande ville, parce qu'elle a abreuvé toutes les nations du vin de la fureur de son impudicité~!
\TextTitle{Le jugement des adorateurs de la bête}
\VS{9}Et un troisième ange les suivit, disant d'une voix forte~: Si quelqu'un adore la bête et son image, et reçoit la marque sur son front ou sur sa main,
\VS{10}il boira, lui aussi, du vin de la colère de Dieu, du vin pur versé dans la coupe de sa colère, et il sera tourmenté dans le feu et le soufre devant les saints anges et devant l'Agneau.
\VS{11}Et la fumée de leur tourment montera aux siècles des siècles, et ils n'auront de repos ni jour ni nuit, ceux qui adorent la bête et son image, et quiconque reçoit la marque de son nom.
\VS{12}Ici est la persévérance des saints~; ici sont ceux qui gardent les commandements de Dieu, et la foi de Jésus.
\TextTitle{Bénédiction de ceux qui meurent en Christ}
\VS{13}Alors j'entendis une voix du ciel qui me disait~: Ecris~: Bénis sont dès à présent les morts qui meurent dans le Seigneur~! Oui, c'est vrai~! Dit l'Esprit, afin qu'ils se reposent de leurs travaux, car leurs œuvres les suivent.
\TextTitle{Prophétie sur Harmaguédon}
\VS{14}Et je regardai, et voici, il y avait une nuée blanche, et sur la nuée était assis quelqu'un qui ressemblait à un homme\FTNT{Ez. 1:26~; Da. 7:13~; Mt. 24:30~; Mt. 26:64~; Ap. 1:13.}, ayant sur sa tête une couronne d'or, et dans sa main une faucille tranchante.
\VS{15}Et un autre ange sortit du temple, criant à haute voix à celui qui était assis sur la nuée~: Jette ta faucille, et moissonne~; car c'est ton heure de moissonner, parce que la moisson de la terre est mûre\FTNT{Jé. 51:33~; Mt. 13:30-39.}.
\VS{16}Alors celui qui était assis sur la nuée jeta sa faucille sur la terre, et la terre fut moissonnée.
\VS{17}Et un autre ange sortit du temple qui est dans le ciel, ayant lui aussi une faucille tranchante.
\VS{18}Et un autre ange, qui avait autorité sur le feu, sortit de l'autel, et s'adressant d'une voix forte à celui qui avait la faucille tranchante, dit~: Jette ta faucille tranchante, et vendange les grappes de la vigne de la terre, car ses raisins sont mûrs.
\VS{19}Et l'ange jeta sa faucille tranchante sur la terre et vendangea la vigne de la terre, et il jeta la vendange dans la grande cuve de la colère de Dieu.
\VS{20}Et la cuve fut foulée hors de la ville~; et du sang sortit de la cuve, jusqu'aux mors des chevaux, sur une étendue de mille six cents stades\FTNT{Es. 63:1-6.}.
\Chap{15}
\TextTitle{Une scène glorieuse au ciel}
\VerseOne{}Puis je vis dans le ciel un autre signe, grand et admirable~: Sept anges qui tenaient les sept derniers fléaux, car c'est par eux que s'accomplit la colère de Dieu.
\VS{2}Et je vis aussi comme une mer de verre mêlée de feu, et ceux qui avaient vaincu la bête et son image, et sa marque, et le nombre de son nom, étaient debout sur la mer qui était comme de verre, et ayant les harpes de Dieu.
\VS{3}Ils chantaient le cantique de Moïse, serviteur de Dieu, et le cantique de l'Agneau, en disant~: Tes œuvres sont grandes et merveilleuses, ô Seigneur Dieu Tout-Puissant~! Tes voies sont justes et véritables, ô Roi des nations~!
\VS{4}Seigneur, qui ne te craindrait, et qui ne glorifierait ton Nom~? Car toi seul tu es Saint, c'est pourquoi toutes les nations viendront et se prosterneront devant toi~; car tes jugements sont pleinement manifestés.
\VS{5}Et après ces choses, je regardai, et voici le temple du tabernacle du témoignage fut ouvert dans le ciel.
\VS{6}Et les sept anges qui avaient les sept fléaux sortirent du temple, revêtus d'un lin pur et blanc, et ayant des ceintures d'or autour de leurs poitrines.
\VS{7}Et l'un des quatre animaux donna aux sept anges sept coupes d'or, pleines de la colère du Dieu qui vit aux siècles des siècles.
\VS{8}Et le temple fut rempli de la fumée à cause de la gloire de Dieu et de sa puissance~; et personne ne pouvait entrer dans le temple jusqu'à ce que les sept fléaux des sept anges soient accomplis.
\Chap{16}
\TextTitle{Première coupe~: Les ulcères}
\VerseOne{}Et j'entendis du temple une voix éclatante qui disait aux sept anges~: Allez, et versez sur la terre les coupes de la colère de Dieu.
\VS{2}Et le premier ange s'en alla, et versa sa coupe sur la terre. Et un ulcère malin et dangereux frappa les hommes qui avaient la marque de la bête, et ceux qui adoraient son image.
\TextTitle{Deuxième coupe~: La mer changée en sang}
\VS{3}Et le second ange versa sa coupe sur la mer, et elle devint comme le sang d'un corps mort, et tout être qui vivait dans la mer mourut.
\TextTitle{Troisième coupe~: Les sources changées en sang}
\VS{4}Et le troisième ange versa sa coupe sur les fleuves et sur les sources des eaux, et elles devinrent du sang.
\VS{5}Et j'entendis l'ange des eaux qui disait~: Seigneur, QUI ES, QUI ETAIS, et QUI VIENS, tu es juste, parce que tu as exercé ce jugement.
\VS{6}Parce qu'ils ont répandu le sang des saints et des prophètes, tu leur as aussi donné du sang à boire, car ils le méritent.
\VS{7}Et j'entendis un autre ange de l'autel, qui disait~: Certainement, Seigneur Dieu Tout-Puissant, tes jugements sont véritables et justes.
\TextTitle{Quatrième coupe~: Une chaleur extrême}
\VS{8}Ensuite, le quatrième ange versa sa coupe sur le soleil, et le pouvoir lui fut donné de brûler les hommes par le feu,
\VS{9}de sorte que les hommes furent brûlés par de grandes chaleurs, et ils blasphémèrent le Nom de Dieu qui a puissance sur ces fléaux~; et ils ne se repentirent pas pour lui donner gloire.
\TextTitle{Cinquième coupe~: Les ténèbres sur le trône de la bête}
\VS{10}Après cela, le cinquième ange versa sa coupe sur le trône de la bête. Et son royaume fut couvert de ténèbres, et les hommes se mordaient la langue à cause de la douleur qu'ils ressentaient.
\VS{11}Et ils blasphémèrent le Dieu du ciel à cause de leurs douleurs et de leurs ulcères~; et ils ne se repentirent pas de leurs œuvres.
\TextTitle{Sixième coupe~: L'Euphrate asséché}
\VS{12}Puis le sixième ange versa sa coupe sur le grand fleuve, l'Euphrate. Et son eau tarit, afin de préparer la voie des rois venant du côté où le soleil se lève.
\VS{13}Et je vis sortir de la gueule du dragon, et de la gueule de la bête, et de la bouche du faux prophète, trois esprits impurs semblables à des grenouilles.
\VS{14}Car ce sont des esprits de démons, qui font des prodiges, et qui vont vers les rois de la terre et du monde entier, afin de les assembler pour le combat de ce grand jour du Dieu Tout-Puissant.
\VS{15}Voici, je viens comme un voleur. Béni est celui qui veille et qui garde ses vêtements, afin de ne pas marcher nu, et qu'on ne voie pas sa honte~!
\VS{16}Et ils les assemblèrent dans le lieu qui est appelé en hébreu Harmaguédon\FTNT{Le terme «~Harmaguédon~», mentionné uniquement dans ce passage, vient du mot hébreu «~Har-Magidown~», ce qui signifie «~Montagne de Megiddo~». Bien qu'il n'existe pas de montagne portant spécifiquement ce nom, l'emplacement probable de cet endroit est la plaine de Meggido se trouvant à proximité de Jérusalem. Par le passé, elle fut le théâtre de la victoire de Barak sur les Cananéens (Jg. 4:15) et de celle de Gédéon sur les Madianites (Jg. 7). C'est aussi à cet endroit que Saül et ses fils (1 Sa. 31~:8) ainsi que le roi Josias (2 R. 23:29-30~; 2 Ch. 35:22) trouvèrent la mort. Pour toutes ces raisons, elle devint au fil du temps le symbole de l'affrontement entre Dieu et la puissance des ténèbres. Selon les prophéties bibliques, la plaine de Meggido et la vallée de Jizréel constitueront le site de l'ultime guerre mondiale, celle opposant l'Antichrist et ses alliés (dirigeants des nations) contre Israël. Le Seigneur interviendra alors ouvertement dans les affaires humaines pour déverser la coupe de sa colère (Ap. 16:1) et anéantir l'homme impie et toute son armée (Ez. 38-39~; Joë. 3~; Mi. 4:11~; So. 1~; Za. 14~; Mt. 24:29-30~; Ap. 20:1-3~; Ap. 20:7-10).}.
\TextTitle{Septième coupe~: Une grosse grêle tombe du ciel}
\VS{17}Puis le septième ange versa sa coupe dans l'air~; et il sortit du temple du ciel une voix forte qui venait du trône, disant~: C'en est fait.
\VS{18}Et il y eut des éclairs, et des voix, et des tonnerres, et il se fit un grand tremblement de terre, dis-je, tel qu'il n'y en avait jamais eu depuis que les hommes sont sur la terre.
\VS{19}La grande ville fut divisée en trois parties, et les villes des nations tombèrent, et Dieu se souvint de Babylone la grande, pour lui donner la coupe du vin de son ardente colère.
\VS{20}Toutes les îles s'enfuirent et les montagnes ne furent plus retrouvées.
\VS{21}Une grosse grêle, dont les grêlons pesaient un talent\FTNT{Un talent d'argent pesait 45 kg, un talent d'or pesait 90 kg.}, tomba du ciel sur les hommes~; et les hommes blasphémèrent Dieu, à cause du fléau de la grêle, car le fléau qu'elle causa fut très grand.
\Chap{17}
\TextTitle{La prostituée}
\VerseOne{}Puis l'un des sept anges qui tenaient les sept coupes vint, et il m'adressa la parole, en disant~: Viens, je te montrerai le jugement de la grande prostituée, qui est assise sur les grandes eaux.
\VS{2}Avec elle, les rois de la terre ont commis la fornication, et les habitants de la terre ont été enivrés du vin de sa prostitution.
\VS{3}Il me transporta en esprit dans un désert~; et je vis une femme assise sur une bête écarlate, pleine de noms de blasphème, ayant sept têtes et dix cornes.
\VS{4}Et la femme était vêtue de pourpre et d'écarlate, et parée d'or, de pierres précieuses, et de perles~; et elle tenait à la main une coupe d'or, pleine des abominations de l'impureté de sa prostitution.
\VS{5}Et il y avait sur son front un nom écrit, un mystère~: Babylone la grande, la mère des impudicités et des abominations de la terre\FTNT{Symboliquement, Babylone la grande incarne l'Eglise apostate. Elle est soutenue par la bête qu'elle chevauche, c'est-à-dire l'homme impie. Ces deux entités forment un système impie où la politique et la religion se mélangent (Da. 2:43).}.
\VS{6}Et je vis cette femme ivre du sang des saints, et du sang des martyrs de Jésus. Et quand je la vis, je fus saisi d'un grand étonnement.
\TextTitle{Alliance entre la prostituée et la bête}
\VS{7}Et l'ange me dit~: Pourquoi t'étonnes-tu~? Je te dirai le mystère de la femme et de la bête qui la porte, qui a les sept têtes et les dix cornes.
\VS{8}La bête que tu as vue, était, et elle n'est plus. Elle doit monter de l'abîme, et aller à la perdition. Et les habitants de la terre, ceux dont les noms ne sont pas écrits dans le Livre de vie dès la fondation du monde, s'étonneront en voyant la bête parce qu'elle était, et qu'elle n'est plus, et qui toutefois est.
\VS{9}C'est ici qu'il faut un esprit intelligent et qui ait de la sagesse. Les sept têtes sont sept montagnes sur lesquelles la femme est assise.
\VS{10}Ce sont aussi sept rois, les cinq sont tombés~; l'un est, et l'autre n'est pas encore venu~; et quand il sera venu, il faut qu'il demeure pour un peu de temps.
\VS{11}Et la bête qui était, et qui n'est plus, est elle-même un huitième roi, et elle est du nombre des sept, mais elle tend à sa ruine.
\VS{12}Et les dix cornes que tu as vues sont dix rois, qui n'ont pas encore commencé à régner, mais ils recevront autorité comme rois en même temps avec la bête, pour une heure.
\VS{13}Ils ont un même dessein, et ils donneront leur puissance et leur autorité à la bête.
\TextTitle{Victoire de l'Agneau sur la prostituée}
\VS{14}Ils combattront contre l'Agneau et l'Agneau les vaincra, parce qu'il est le Seigneur des seigneurs, et le Roi des rois~; et les appelés, les élus et les fidèles qui sont avec lui, les vaincront aussi.
\VS{15}Puis il me dit~: Les eaux que tu as vues, et sur lesquelles la prostituée est assise, sont des peuples, des nations et des langues.
\VS{16}Les dix cornes que tu as vues et la bête haïront la prostituée, la rendront désolée et nue, la dépouilleront, et mangeront sa chair, et la brûleront au feu.
\VS{17}Car Dieu a mis dans leurs cœurs de faire ce qu'il lui plaît, et de former un même dessein, et de donner leur royaume à la bête, jusqu'à ce que les paroles de Dieu soient accomplies.
\VS{18}Et la femme que tu as vue, c'est la grande ville, qui règne sur les rois de la terre.
\Chap{18}
\TextTitle{Babylone détruite}
\VerseOne{}Après ces choses, je vis descendre du ciel un autre ange, qui avait une grande autorité, et la terre fut illuminée de sa gloire.
\VS{2}Il cria avec force à haute voix, et il dit~: Elle est tombée, elle est tombée Babylone la grande, et elle est devenue la demeure de démons, et la retraite de tout esprit impur, et le repaire de tout oiseau impur et exécrable.
\VS{3}Car toutes les nations ont bu du vin de sa prostitution effrénée, et les rois de la terre ont commis la fornication avec elle, et les marchands de la terre se sont enrichis par l'excès de son luxe.
\VS{4}Puis j'entendis une autre voix du ciel, qui disait~: Sortez de Babylone mon peuple, afin que vous ne participiez pas à ses péchés, et que vous n'ayez pas de part à ses fléaux.
\VS{5}Car ses péchés sont montés jusqu'au ciel, et Dieu s'est souvenu de ses iniquités.
\VS{6}Rendez-lui selon ce qu'elle vous a fait, et payez-lui au double selon ses œuvres~; et dans la même coupe où elle vous a versé à boire versez-lui au double.
\VS{7}Autant elle s'est glorifiée et plongée dans le luxe, autant donnez-lui de tourment et de deuil~; car elle dit en son cœur~: Je siège en reine, je ne suis pas veuve, et je ne verrai pas de deuil.
\VS{8}C'est pourquoi ses plaies, qui sont la mort, le deuil, et la famine, viendront en un même jour, et elle sera entièrement brûlée au feu~; car le Seigneur Dieu qui la jugera est puissant.
\TextTitle{Conséquence de la chute de Babylone~: Gémissements des habitants de la terre}
\VS{9}Et les rois de la terre, qui ont commis la fornication avec elle, et qui ont vécu dans le luxe, la pleureront, et mèneront deuil sur elle en se frappant la poitrine, quand ils verront la fumée de son embrasement~;
\VS{10}et ils se tiendront éloignés dans la crainte de son tourment, et diront~: Malheur~! Malheur~! Babylone la grande, cette ville si puissante~! Comment ta condamnation est-elle venue en une seule heure~?
\VS{11}Les marchands de la terre aussi pleureront, et seront dans le deuil à cause d'elle, parce que personne n'achète plus de leurs marchandises,
\VS{12}qui sont des marchandises d'or, d'argent, de pierres précieuses, de perles, de fin lin, de pourpre, de soie, d'écarlate, de toute sorte de bois odoriférant, de toute espèce de bois de senteur, d'ivoire, et de toute espèce de vaisseaux de bois très précieux, d'airain, de fer, et de marbre,
\VS{13}du cinnamome, des parfums, des essences, de l'encens, du vin, de l'huile, de la fine fleur de farine, du blé, des bœufs, des brebis, des chevaux, des chars, des esclaves, et des âmes d'hommes.
\VS{14}Car les fruits du désir de ton âme se sont éloignés de toi, et toutes les choses délicates et excellentes sont perdues pour toi, et dorénavant tu ne les trouveras plus.
\VS{15}Les marchands, dis-je, de ces choses, qui se sont enrichis par elle, se tiendront éloignés, dans la crainte de son tourment~; ils pleureront et seront dans le deuil,
\VS{16}et diront~: Malheur~! Malheur~! La grande ville qui était vêtue de fin lin, de pourpre, d'écarlate, qui était parée d'or, ornée de pierres précieuses, et de perles~! Comment en une seule heure tant de richesses ont été détruites~?
\VS{17}Et tous les pilotes aussi, tous ceux qui naviguent vers ce lieu, tous les marins, et tous ceux qui exploitent la mer, se tiendront éloignés,
\VS{18}et, en voyant la fumée de son embrasement, ils s'écrieront, en disant~: Quelle ville était semblable à cette grande ville~?
\VS{19}Ils jetteront de la poussière sur leurs têtes, pleurant et menant deuil, ils crieront, en disant~: Malheur~! Malheur~! La grande ville, où se sont enrichis par son opulence tous ceux qui ont des navires sur la mer~! Comment a-t-elle été réduite en désert en une seule heure~?
\TextTitle{Réjouissance des anges suite à la chute de Babylone}
\VS{20}Ô ciel~! Réjouis-toi à cause d'elle~; et vous aussi, les saints, les apôtres et les prophètes, réjouissez-vous~! Car Dieu l'a punie à cause de vous.
\VS{21}Alors un ange d'une grande force prit une pierre semblable à une grande meule, et la jeta dans la mer, en disant~: Ainsi sera précipitée avec impétuosité Babylone, cette grande ville, et elle ne sera plus retrouvée\FTNT{Jé. 51:63-64.}.
\VS{22}Et l'on entendra plus chez toi les sons des joueurs de harpe, des musiciens, des joueurs de flûte, et de ceux qui sonnent de la trompette~; et on ne trouvera plus chez toi aucun artisan d'un métier quelconque, on n'entendra plus chez toi le bruit de la meule,
\VS{23}et la lumière de la lampe ne brillera plus chez toi, et la voix de l'époux et de l'épouse ne sera plus entendue chez toi~; car tes marchands étaient des princes de la terre, et parce que par tes enchantements toutes les nations ont été séduites,
\VS{24}et l'on a trouvé chez elle le sang des prophètes et des saints, et de tous ceux qui ont été mis à mort sur la terre.
\Chap{19}
\TextTitle{Allégresse dans les cieux suite au jugement de la grande prostituée\FTNTT{Ap. 17:16-17~; 18:8.}}
\VerseOne{}Après cela, j'entendis dans le ciel une voix forte d'une foule nombreuse, disant~: Alléluia~! Le salut, la gloire, l'honneur et la puissance appartiennent au Seigneur, notre Dieu,
\VS{2}car ses jugements sont véritables et justes~; car il a jugé la grande prostituée qui a corrompu la terre par son impudicité, et parce qu'il a vengé le sang de ses serviteurs versé de la main de la prostituée.
\VS{3}Et ils dirent encore~: Alléluia~! Et sa fumée monte aux siècles des siècles.
\VS{4}Et les vingt-quatre anciens et les quatre animaux se prosternèrent sur leurs faces, et adorèrent Dieu, qui était assis sur le trône, en disant~: Amen~! Alléluia~!
\VS{5}Et il sortit du trône une voix qui disait~: Louez notre Dieu, vous tous ses serviteurs, et vous qui le craignez, tant les petits que les grands\FTNT{Ps. 134.}.
\VS{6}J'entendis ensuite comme la voix d'une grande assemblée, et comme le bruit de grandes eaux, et comme l'éclat de grands tonnerres, disant~: Alléluia~! Car le Seigneur notre Dieu Tout-Puissant a pris possession de son Royaume.
\TextTitle{Festin des noces de l'Agneau}
\VS{7}Réjouissons-nous et tressaillons de joie, et donnons-lui gloire, car les noces de l'Agneau sont venues, et son Epouse s'est préparée.
\VS{8}Et il lui a été donné de se revêtir d'un fin lin pur et éclatant. Car le fin lin désigne la justice des saints.
\VS{9}Alors il me dit~: Ecris~: Bénis sont ceux qui sont appelés au festin des noces de l'Agneau\FTNT{Mt. 22:1-13~; Lu. 14:15-24.}~! Il me dit aussi~: Ces paroles de Dieu sont véritables.
\VS{10}Alors je tombai à ses pieds pour l'adorer, mais il me dit~: Garde-toi de le faire~! Je suis ton compagnon de service, et celui de tes frères qui ont le témoignage de Jésus. Adore Dieu~! Car le témoignage de Jésus est l'Esprit de la prophétie.
\TextTitle{Seconde venue du Messie dans la gloire\FTNTT{Mt. 24:16-30.}}
\VS{11}Puis je vis le ciel ouvert, et voici parut un cheval blanc. Et celui qui était monté dessus s'appelle FIDELE et VERITABLE, et il juge et combat avec justice.
\VS{12}Et ses yeux étaient comme une flamme de feu~; il y avait sur sa tête plusieurs diadèmes, et il avait un nom écrit que personne ne connaît, si ce n'est lui-même.
\VS{13}Il était revêtu d'un vêtement teint de sang, et son Nom s'appelle LA PAROLE DE DIEU.
\VS{14}Les armées qui sont dans le ciel le suivaient sur des chevaux blancs, revêtues de fin lin blanc et pur.
\VS{15}De sa bouche sortait une épée tranchante\FTNT{Es. 11:4~; 2 Th. 2:8~; Hé. 4:12.}, pour frapper les nations~; il les gouvernera avec un sceptre de fer\FTNT{Ps. 2:8-9.}, et il foulera la cuve du vin de l'indignation et de la colère du Dieu Tout-Puissant.
\VS{16}Et sur son vêtement et sur sa cuisse étaient écrits ces mots~: LE ROI DES ROIS ET LE SEIGNEUR DES SEIGNEURS.
\TextTitle{Bataille d'Harmaguédon\FTNTT{Ap. 16:16.}}
\VS{17}Puis je vis un ange qui se tenait dans le soleil. Il cria d'une voix forte, et dit à tous les oiseaux qui volaient au milieu du ciel~: Venez et rassemblez-vous pour le grand festin de Dieu,
\VS{18}afin de manger la chair des rois, la chair des chefs militaires, la chair des puissants, la chair des chevaux et de ceux qui les montent, et la chair de toute sorte de personnes libres, esclaves, petits et grands.
\VS{19}Alors je vis la bête et les rois de la terre, et leurs armées rassemblées pour faire la guerre\FTNT{Guerre d' Harmaguédon~: Voir commentaire Ap. 16:16.} contre celui qui était monté sur le cheval et contre son armée.
\TextTitle{Condamnation de la bête et du faux prophète}
\VS{20}Et la bête fut prise, et avec elle le faux prophète qui avait fait devant elle les prodiges par lesquels il avait séduit ceux qui avaient pris la marque de la bête, et adoré son image. Et ils furent tous deux jetés vivants dans l'étang ardent de feu et de soufre.
\TextTitle{Condamnation des rois et des armées}
\VS{21}Et le reste fut tué par l'épée qui sortait de la bouche de celui qui était monté sur le cheval, et tous les oiseaux furent rassasiés de leur chair.
\Chap{20}
\TextTitle{Satan lié pour mille ans et règne du Messie}
\VerseOne{}Après cela, je vis descendre du ciel un ange, qui avait la clef de l'abîme et une grande chaîne dans sa main.
\VS{2}Il saisit le dragon, le serpent ancien, qui est le diable et Satan, et le lia pour mille ans.
\VS{3}Il le jeta dans l'abîme, et il l'enferma et mit le sceau sur lui, afin qu'il ne séduise plus les nations, jusqu'à ce que les mille ans soient accomplis. Après quoi, il faut qu'il soit délié pour un peu de temps.
\TextTitle{Dernière phase de la première résurrection}
\VS{4}Je vis des trônes, sur lesquels des gens s'assirent, à qui l'autorité de juger fut donnée\FTNT{1 Co. 6:2.}. Et je vis les âmes de ceux qui avaient été décapités pour le témoignage de Jésus, et pour la parole de Dieu, et de ceux qui n'avaient pas adoré la bête ni son image, et qui n'avaient pas pris sa marque sur leurs fronts, ou sur leurs mains. Et ils vécurent\FTNT{Jn. 14:19.} et régnèrent avec Christ mille ans.
\VS{5}Les autres morts ne revinrent pas à la vie jusqu'à ce que les mille ans soient accomplis. C'est la première résurrection.
\VS{6}Bénis et saints sont ceux qui ont part à la première résurrection~! La seconde mort n'a pas de puissance sur eux, mais ils seront prêtres de Dieu, et de Christ, et ils régneront avec lui mille ans.
\TextTitle{Satan délié~; sa chute finale}
\VS{7}Et quand les mille ans seront accomplis, Satan sera délié de sa prison.
\VS{8}Et il sortira pour séduire les nations qui sont aux quatre coins de la terre, Gog et Magog, afin de les rassembler pour la guerre, et leur nombre est comme le sable de la mer.
\VS{9}Ils montèrent et se répandirent à la surface de la terre, et ils environnèrent le camp des saints, et la ville bien-aimée. Mais Dieu fit descendre un feu du ciel qui les dévora.
\TextTitle{Satan jeté dans l'étang de feu}
\VS{10}Et le diable qui les séduisait fut jeté dans l'étang de feu et de soufre, où sont la bête et le faux prophète. Et ils seront tourmentés jour et nuit, aux siècles des siècles.
\TextTitle{Résurrection des impies et jugement dernier~; l'Hadès (ou enfer) et la mort jetés dans l'étang de feu}
\VS{11}Puis je vis un grand trône blanc, et celui qui était assis dessus. La terre et le ciel s'enfuirent devant sa face, et il ne fut plus trouvé de place pour eux.
\VS{12}Et je vis les morts, les grands et les petits, qui se tenaient devant le trône de Dieu. Des livres furent ouverts. Et un autre livre fut ouvert, celui qui est le Livre de vie. Et les morts furent jugés selon les choses qui étaient écrites dans les livres, c'est-à-dire selon leurs œuvres.
\VS{13}Et la mer rendit les morts qui étaient en elle, et la mort et l'enfer\FTNT{le mot «~enfer~» vient de l'hébreu «~Hadès~». Voir commentaire dans Mt. 16:18.} rendirent les morts qui étaient en eux~; et ils furent jugés chacun selon ses œuvres.
\VS{14}Et la mort et l'enfer furent jetés dans l'étang de feu\FTNT{L'étang de feu est aussi appelé «~seconde mort~», c'est la destination finale de tous les impies, des démons et de Satan. On l'appelle «~la seconde mort~» parce qu'elle a été précédée de la mort physique. Cette mort n'est pas un anéantissement, mais une condition de souffrances éternelles. C'est la séparation définitive d'avec Dieu. A l'issue du jugement dernier, le séjour des morts (le dieu Hadès ou l'enfer) sera jeté dans le lac de feu (voir commentaire en Mt. 16:18). La Bible utilise également le mot «~géhenne~» pour décrire l'endroit où les impies passeront l'éternité. Ce terme vient de l'hébreu «~ge-hinnom~», autrement dit vallée de Ben Hinnom (littéralement «~le lieu du feu~») qui se trouve en Israël, en contrebas du mont Sion sur lequel est bâtie la ville de Jérusalem (Mt. 5:22~; Mt. 5:29-30~; Mt. 10:28~; Mt. 18:9~; Mt. 23:15~; Mt. 23:33~; Mc. 9:47~; Lu. 12:5~; Ja. 3:6). Autrefois, on y brûlait des enfants en l'honneur de Moloc, une divinité ammonite (2 R. 23:10~; Jé. 32:35), puis des immondices. Ce lieu est devenu avec le temps le symbole du péché et de l'affliction et c'est ainsi qu'il finit par désigner le lieu du châtiment éternel.}. C'est la seconde mort.
\VS{15}Et quiconque ne fut pas trouvé écrit dans le Livre de vie fut jeté dans l'étang de feu.
\Chap{21}
\TextTitle{Nouveaux cieux et une nouvelle terre~; la nouvelle Jérusalem}
\VerseOne{}Puis je vis un nouveau ciel et une nouvelle terre~; car le premier ciel et la première terre avaient disparu, et la mer n'était plus.
\VS{2}Et moi, Jean, je vis la ville sainte, la nouvelle Jérusalem, qui descendait du ciel, d'auprès de Dieu, parée comme une épouse qui s'est ornée pour son mari.
\VS{3}Et j'entendis du trône une voix forte qui disait~: Voici le tabernacle de Dieu avec les hommes~! Il habitera avec eux, et ils seront son peuple, et Dieu lui-même sera leur Dieu, et il sera avec eux.
\VS{4}Et Dieu essuiera toute larme de leurs yeux, et la mort ne sera plus~; et il n'y aura plus ni deuil, ni cri, ni douleur, car les premières choses sont passées.
\VS{5}Et celui qui était assis sur le trône dit~: Voici, je fais toutes choses nouvelles. Puis il me dit~: Ecris, car ces paroles sont véritables et certaines.
\VS{6}Il me dit aussi~: Tout est accompli. Je suis l'Alpha et l'Oméga, le commencement et la fin. A celui qui a soif, je lui donnerai de la source d'eau vive gratuitement\FTNT{Es. 55:1-2~; Mt. 10:8~; Ap. 22:17. Voir commentaire Mt. 10:8.}.
\VS{7}Celui qui vaincra héritera toutes choses~; je serai son Dieu, et il sera mon fils.
\VS{8}Mais pour les timides, les incrédules, les abominables, les meurtriers, les fornicateurs, les sorciers, les idolâtres et tous les menteurs, leur part sera dans l'étang ardent de feu et de soufre, qui est la seconde mort.
\TextTitle{L'Epouse de l'Agneau et la nouvelle Jérusalem}
\VS{9}Puis l'un des sept anges qui tenaient les sept coupes pleines des sept derniers fléaux s'approcha de moi et me parla, en disant~: Viens, et je te montrerai l'Epouse, la femme de l'Agneau.
\VS{10}Et il me transporta en esprit sur une grande et haute montagne, et il me montra la grande ville, la sainte Jérusalem, qui descendait du ciel d'auprès de Dieu,
\VS{11}ayant la gloire de Dieu. Son éclat était semblable à une pierre très précieuse, comme à une pierre de jaspe transparente comme du cristal.
\VS{12}Et elle avait une grande et haute muraille, avec douze portes, et aux portes douze anges, et des noms écrits sur elles, qui sont les noms des douze tribus des fils d'Israël\FTNT{Ez. 48:31-34.}.
\VS{13}A l'orient, trois portes, au nord, trois portes, du côté du sud, trois portes et du côté de l'occident, trois portes.
\VS{14}Et la muraille de la ville avait douze fondements, et les noms des douze apôtres de l'Agneau étaient écrits dessus\FTNT{Lu. 22:29-30~; Ep. 2:20.}.
\VS{15}Et celui qui parlait avec moi avait un roseau d'or pour mesurer la ville, ses portes et sa muraille.
\VS{16}Et la ville était bâtie en carré, et sa longueur était aussi grande que sa largeur. Il mesura donc la ville avec le roseau d'or, jusqu'à douze mille stades~; la longueur, la largeur et la hauteur étaient égales.
\VS{17}Puis il mesura la muraille qui fut de cent quarante-quatre coudées, de la mesure du personnage, c'est-à-dire de l'ange.
\VS{18}Et le bâtiment de la muraille était de jaspe, mais la ville était d'or pur, semblable à du verre fort transparent.
\VS{19}Et les fondements de la muraille de la ville étaient ornés de toutes sortes de pierres précieuses\FTNT{Es. 54:11-12.}~: Le premier fondement était de jaspe, le second de saphir, le troisième de calcédoine, le quatrième d'émeraude,
\VS{20}le cinquième de sardonyx, le sixième de sardoine, le septième de chrysolithe, le huitième de béryl, le neuvième de topaze, le dixième de chrysoprase, le onzième d'hyacinthe, le douzième d'améthyste.
\VS{21}Et les douze portes étaient douze perles~; chacune des portes était d'une seule perle. Et la place de la ville était d'or pur, comme du verre transparent.
\VS{22}Et je ne vis pas de temple dans la ville, parce que le Seigneur Dieu Tout-Puissant et l'Agneau en sont le Temple.
\VS{23}Et la ville n'a pas besoin du soleil ni de la lune pour l'éclairer, car la gloire de Dieu l'éclaire, et l'Agneau est son flambeau\FTNT{Es. 60:19.}.
\VS{24}Et les nations qui auront été sauvées marcheront à la faveur de sa lumière, et les rois de la terre y apporteront ce qu'ils ont de plus magnifique et de plus précieux.
\VS{25}Et ses portes ne se fermeront pas le jour, car il n'y aura pas de nuit\FTNT{Es. 60:11.}.
\VS{26}Et on y apportera la gloire et l'honneur des nations.
\VS{27}Il n'entrera chez elle rien de souillé, ni personne qui s'abandonne à l'abomination et au mensonge~; mais seulement ceux qui sont écrits dans le Livre de vie de l'Agneau.
\Chap{22}
\TextTitle{Règne éternel des saints avec l'Agneau}
\VerseOne{}Puis il me montra un fleuve d'eau de la vie\FTNT{Ce fleuve représente le Saint-Esprit~: Ez. 47:1-12~; Ps. 46:5~; Da. 7:9-10~; Jn. 7:38-39.}, transparent comme du cristal, qui sortait du trône de Dieu et de l'Agneau.
\VS{2}Et au milieu de la place de la ville, et des deux côtés du fleuve, était l'arbre de vie, portant douze fruits, et rendant son fruit chaque mois et les feuilles de l'arbre servaient à la guérison des nations\FTNT{Ge. 2:9~; Ge. 3:22~; Ez. 47:12.}.
\VS{3}Et il n'y aura plus d'anathème. Le trône de Dieu et de l'Agneau sera dans la ville, et ses serviteurs le serviront,
\VS{4}et ils verront sa face, et son Nom sera sur leurs fronts.
\VS{5}Et il n'y aura plus de nuit~; et ils n'auront besoin ni de lumière, ni de lampe, ni du soleil, parce que le Seigneur Dieu les éclairera, et ils régneront aux siècles des siècles.
\TextTitle{Certitude des prophéties de ce livre}
\VS{6}Puis il me dit~: Ces paroles sont certaines et véritables~; et le Seigneur, le Dieu des esprits des prophètes, a envoyé son ange pour manifester à ses serviteurs les choses qui doivent arriver bientôt.
\VS{7}Voici, je viens à toute vitesse\FTNT{Dans la plupart des traductions, ce passage a été traduit par «~Je viens bientôt~». Or le texte grec utilise le mot «~tachu~» qui signifie «~rapidement, à toute vitesse (sans tarder)~». Beaucoup doutent de cette promesse du Seigneur en faisant la même réflexion évoquée par Pierre~: «~Où est la promesse de son avènement~? Car depuis que les pères sont morts, toutes choses demeurent comme elles ont été dès le commencement de la création.~» (2 Pi. 3:4). Or le Seigneur ne tarde pas dans l'accomplissement de sa promesse, car il a fixé de sa propre autorité une date pour son retour, que lui seul connaît (Za. 14:7~; Mt. 24:36~; Mc. 13:32~; Ac. 1:6-7). Il sera donc fidèle à son calendrier et ne tardera pas (2 Pi. 3:9.~; Hé. 10:37).}. Béni est celui qui garde les paroles de la prophétie de ce livre~!
\VS{8}C'est moi, Jean, qui ai entendu et vu ces choses. Et après les avoir entendues et vues, je tombai à terre aux pieds de l'ange qui me les montrait pour l'adorer.
\VS{9}Mais il me dit~: Garde-toi de le faire~! Car je suis ton compagnon de service\FTNT{Hé. 1:14.} et celui de tes frères les prophètes, et de ceux qui gardent les paroles de ce livre. Adore Dieu~!
\VS{10}Il me dit aussi~: Ne scelle pas les paroles de la prophétie de ce livre. Car le temps est proche.
\VS{11}Que celui qui est injuste soit encore injuste, et que celui qui est souillé se souille encore~; et que celui qui est juste pratique encore la justice~; et que celui qui est saint se sanctifie encore~!
\VS{12}Voici, je viens à toute vitesse, et ma rétribution est avec moi\FTNT{Jésus affirme de nouveau ici sa divinité et confirme les prophéties d'Es. 35:4~; Es. 40:10~; Es. 62:11, où il est dit que Yahweh lui-même viendra avec ses rétributions.} pour rendre à chacun selon son œuvre.
\VS{13}Je suis l'Alpha et l'Oméga, le premier et le dernier, le commencement et la fin.
\VS{14}Bénis sont ceux qui lavent leurs robes afin d'avoir droit à l'arbre de vie, et d'entrer par les portes dans la ville.
\VS{15}Mais seront laissés dehors les chiens, les empoisonneurs, les fornicateurs, les meurtriers, les idolâtres et quiconque aime et pratique le mensonge.
\VS{16}Moi, Jésus, j'ai envoyé mon ange\FTNT{Cette déclaration de Jésus fait écho au verset 6 où il est dit que le Seigneur, le Dieu des esprits des prophètes, a envoyé son ange. Jésus confirme donc qu'il est Seigneur et Dieu.} pour vous confirmer ces choses dans les églises. Je suis le rejeton et la postérité de David, l'étoile brillante du matin.
\VS{17}Et l'Esprit et l'Epouse disent~: Viens~! Et que celui qui entend dise~: Viens~! Et que celui qui a soif vienne~; que celui qui veut, prenne gratuitement de l'eau de la vie.
\TextTitle{Nul ne doit y ajouter ou y retrancher}
\VS{18}Je le déclare à quiconque entend les paroles de la prophétie de ce livre~: Si quelqu'un y ajoute quelque chose, Dieu le frappera des fléaux décrits dans ce livre,
\VS{19}et si quelqu'un retranche quelque chose des paroles du livre de cette prophétie, Dieu retranchera sa part de l'arbre de vie et de la ville sainte, décrits dans ce livre.
\VS{20}Celui qui rend témoignage de ces choses, dit~: Certainement, je viens à toute vitesse. Amen~! Oui, Seigneur Jésus, viens~!
\VS{21}Que la grâce de notre Seigneur Jésus-Christ soit avec vous tous~! Amen~!
\PPE{}
\end{multicols}
