\ShortTitle{Hébreux}\BookTitle{Hébreux}\BFont
\noindent\hrulefill
{\footnotesize
\textit{
\bigskip
{\centering{}
\\Auteur : Inconnu
\\Thème : Le sacerdoce de Christ
\\Date de rédaction : Env. 68 ap. J.-C.\\}
}
%\bigskip
\textit{
\\Cette épître fut rédigée avant la destruction de Jérusalem, car le temple y subsistait encore. Elle s'adressait à des juifs convertis connaissant bien l'auteur. Parmi eux, certains étaient tentés de retourner au judaïsme à cause des persécutions. L'auteur désire affermir ces chrétiens en leur montrant que l'objectif de la loi avait été réalisé par Christ qui est supérieur aux anges, aux prophètes et à Moïse. Il leur montre combien son œuvre rédemptrice est parfaite et les invite à suivre le Seigneur avec une foi indéfectible en persévérant dans l'amour fraternel.\bigskip
}
}
\par\nobreak\noindent\hrulefill
\begin{multicols}{2}
\Chap{1}
\TextTitle{Dieu parle par le Fils}
\VerseOne{}Dieu ayant anciennement parlé à nos pères par les prophètes, à plusieurs reprises et de plusieurs manières,
\VS{2}nous a parlé dans ces derniers jours\FTNT{Les derniers jours ont commencé avec la naissance de l'Eglise. Voir Joë. 2:28 ; Ac. 2:14-17.} par son Fils, qu'il a établi héritier de toutes choses, et par lequel il a aussi créé l'univers ;
\VS{3}et qui étant la splendeur de sa gloire, et l'empreinte de sa substance, et soutenant toutes choses par sa parole puissante, ayant fait par lui-même la purification de nos péchés, s'est assis à la droite de la Majesté divine dans les lieux très hauts.
\TextTitle{Le Fils supérieur aux anges}
\VS{4}Etant devenu d'autant supérieur aux anges, il a hérité d'un nom plus excellent que le leur.
\VS{5}Car auquel des anges a-t-il jamais dit : Tu es mon Fils, je t'ai engendré aujourd'hui\FTNT{Ps. 2:7.} ? Et encore : Je serai pour lui un Père, et il sera pour moi un Fils\FTNT{2 S. 7:14.} ?
\VS{6}Et quand il introduit de nouveau dans le monde son Fils premier-né\FTNT{Voir commentaire en Col. 1:15.}, il est dit : Et que tous les anges de Dieu l'adorent\FTNT{Ps. 97:7.} !
\VS{7}Car quant aux anges, il est dit : Il fait de ses anges des vents, et de ses serviteurs des flammes de feu\FTNT{Ps. 104:4.}.
\VS{8}Mais à l'égard du Fils, il dit : Ô Dieu, ton trône demeure aux siècles des siècles ; et le sceptre de ton Royaume est un sceptre d'équité ;
\VS{9}tu as aimé la justice, et tu as haï l'iniquité ; c'est pourquoi, ô Dieu, ton Dieu t'a oint d'une huile de joie par-dessus tous tes semblables\FTNT{Ps. 45:7-8.} !
\VS{10}Et dans un autre endroit : Toi, Seigneur, tu as fondé la terre dès le commencement, et les cieux sont les ouvrages de tes mains ;
\VS{11}ils périront, mais tu es permanent ; et ils vieilliront tous comme un vêtement,
\VS{12}et tu les rouleras comme un manteau et ils seront changés ; mais toi, tu restes le même, et tes années ne finiront point\FTNT{Es. 50:9 ; Es. 51:6 ; Ps. 102:27-28.}.
\VS{13}Et auquel des anges a-t-il jamais dit : Assieds-toi à ma droite, jusqu'à ce que j'aie mis tes ennemis pour le marchepied de tes pieds\FTNT{Ps. 110:1.} ?
\VS{14}Ne sont-ils pas tous des esprits administrateurs, envoyés pour servir en faveur de ceux qui doivent recevoir l'héritage du salut ?
\Chap{2}
\TextTitle{Ne pas négliger le salut}
\VerseOne{}C'est pourquoi il nous faut prendre garde de plus près aux choses que nous avons entendues, de peur que nous les laissions s'échapper.
\VS{2}Car, si la parole prononcée par les anges a été ferme, et si toute transgression et toute désobéissance a reçu une juste rétribution,
\VS{3}comment échapperons-nous, si nous négligeons un si grand salut, qui, ayant été premièrement annoncé par le Seigneur, nous a été confirmé par ceux qui l'avaient entendu ?
\VS{4}Dieu confirmant aussi leur témoignage par des prodiges, et des miracles, et par plusieurs autres différents effets de sa puissance, et par les dons du Saint-Esprit, selon sa volonté.
\TextTitle{Toutes choses doivent être soumises à Christ}
\VS{5}Car, ce n'est pas aux anges qu'il a soumis le monde à venir dont nous parlons.
\VS{6}Et quelqu'un a rendu ce témoignage en quelque autre endroit, disant : Qu'est-ce que l'homme, pour que tu te souviennes de lui, ou le fils de l'homme, pour que tu le visites ?
\VS{7}Tu l'as fait un peu moindre que les anges, tu l'as couronné de gloire et d'honneur, et l'as établi sur les œuvres de tes mains.
\VS{8}Tu as assujetti toutes choses sous ses pieds\FTNT{Ps. 8:5-7.}. En effet, en lui assujettissant toutes choses, il n'a rien laissé qui ne lui soit assujetti. Mais, nous ne voyons pourtant pas encore que toutes choses lui soient assujetties.
\TextTitle{Jésus abaissé un peu de temps pour sauver l'homme}
\VS{9}Mais celui qui a été fait un peu moindre que les anges, Jésus, nous le voyons couronné de gloire et d'honneur par la passion de sa mort, afin que par la grâce de Dieu, il souffrît la mort pour tous.
\VS{10}Car il était convenable, que celui pour qui sont toutes choses et par qui sont toutes choses, puisqu'il a amené plusieurs enfants à la gloire, consacre le Prince de leur salut par les afflictions.
\VS{11}Car, et celui qui sanctifie et ceux qui sont sanctifiés descendent tous d'un même père. C'est pourquoi il n'a pas honte de les appeler ses frères,
\VS{12}disant : J'annoncerai ton Nom à mes frères, et je te louerai au milieu de l'assemblée\FTNT{Ps. 22:23.}.
\VS{13}Et encore : Je me confierai en lui. Et encore : Me voici, moi et les enfants que Dieu m'a donnés\FTNT{Es. 8:17-18.}.
\VS{14}Ainsi donc, puisque les enfants participent à la chair et au sang, lui aussi de même a participé aux mêmes choses, afin que, par la mort, il rende impuissant celui qui avait le pouvoir de la mort, c'est-à-dire le diable,
\VS{15}et qu'il délivrât tous ceux qui, par crainte de la mort, étaient assujettis toute leur vie à la servitude.
\VS{16}Car, certes, il n'a nullement secouru les anges, mais il a secouru la postérité d'Abraham.
\VS{17}C'est pourquoi il a fallu qu'il soit semblable en toutes choses à ses frères, afin qu'il soit un Souverain Sacrificateur miséricordieux et fidèle dans les choses qui doivent être faites envers Dieu, pour faire la propitiation pour les péchés du peuple ;
\VS{18}car, parce qu'il a souffert lui-même, étant tenté, il est puissant pour secourir ceux qui sont tentés.
\Chap{3}
\TextTitle{Christ supérieur à Moïse}
\VerseOne{}C'est pourquoi, mes frères saints, qui avez part à la vocation céleste, considérez attentivement Jésus-Christ, l'Apôtre et le Souverain Sacrificateur de notre profession,
\VS{2}qui a été fidèle à celui qui l'a établi, comme le fut Moïse dans toute sa maison.
\VS{3}Car Jésus-Christ a été jugé digne d'une gloire d'autant supérieure à celle de Moïse, que celui qui a construit une maison, a plus d'honneur que la maison même.
\VS{4}Car chaque maison est construite par quelqu'un, mais celui qui a construit toutes choses, c'est Dieu.
\VS{5}Et quant à Moïse, il a été fidèle dans toute sa maison, comme serviteur, pour témoigner des choses qui devait être dites ;
\VS{6}mais Christ l'est comme Fils sur sa maison ; et nous sommes sa maison\FTNT{L'Eglise véritable est la maison de Dieu. Voir Es. 66:1 ; 1 Co. 3:16 ; 1 Cor. 6:19 ; Ep. 2:21-22. Les bâtiments ne sont pas la maison de Dieu. Le premier bâtiment d'église avait été édifié par des fidèles sous le règne d'Alexandre Sévère en 222-235. L'Eglise véritable est composée de pierres vivantes qui ont pour fondement le Roc (Jésus), parce qu'elle est bâtie par Jésus-Christ lui-même et qu'elle est sa propriété ; les démons ne peuvent pas la détruire. L'Eglise véritable ne peut donc être confondue avec un bâtiment ou une maison physique.}, pourvu que nous retenions fermement jusqu'à la fin l'assurance et la gloire de l'espérance.
\TextTitle{Résultat de l'incrédulité de la génération qui sortit d'Egypte}
\VS{7}C'est pourquoi, comme dit le Saint-Esprit : Aujourd'hui, si vous entendez sa voix,
\VS{8}n'endurcissez point vos cœurs, comme il arriva dans le lieu de la rébellion, au jour de la tentation dans le désert,
\VS{9}où vos pères me tentèrent et m'éprouvèrent, et ils virent mes œuvres pendant quarante ans\FTNT{Ps. 95:8-11.}.
\VS{10}C'est pourquoi je fus irrité contre cette génération, et je dis : Leur cœur s'égare toujours. Et ils n'ont pas connu mes voies.
\VS{11}Aussi, je jurai dans ma colère : Ils n'entreront pas dans mon repos !
\VS{12}Mes frères, prenez garde que quelqu'un de vous n'ait un cœur mauvais et incrédule, au point de se révolter contre le Dieu vivant.
\VS{13}Mais exhortez-vous les uns les autres chaque jour, aussi longtemps qu'on peut dire : Aujoud'hui ! De peur que quelqu'un d'entre vous ne s'endurcisse par la séduction du péché.
\VS{14}Car nous sommes devenus participants de Christ, pourvu que nous gardions ferme jusqu'à la fin notre première assurance,
\VS{15}pendant qu'il est dit : Aujourd'hui, si vous entendez sa voix, n'endurcissez pas vos cœurs, comme il arriva dans le lieu de la rébellion.
\VS{16}Car, quelques-uns l'ayant entendue, le provoquèrent à la colère ; mais ce ne furent pas tous ceux qui étaient sortis d'Egypte par Moïse. 
\VS{17}Et contre qui Dieu fut-il irrité pendant quarante ans ? Ne fut-ce pas contre ceux qui péchèrent, et dont les cadavres tombèrent dans le désert ?
\VS{18}Et à qui jura-t-il qu'ils n'entreraient point dans son repos, sinon à ceux qui furent rebelles ?
\VS{19}Aussi, nous voyons qu'ils ne purent y entrer à cause de leur incrédulité.
\Chap{4}
\TextTitle{Le repos}
\VerseOne{}Craignons donc, que quelqu'un d'entre vous, venant à négliger la promesse d'entrer dans son repos, ne s'en trouve privé.
\VS{2}Car il nous a été évangélisé, aussi bien qu'à eux ; mais la parole qu'ils entendirent ne leur servit de rien, parce qu'elle n'était pas mêlée avec la foi dans ceux qui l'entendirent.
\VS{3}Pour nous qui avons cru, nous entrons dans le repos, suivant ce qui a été dit : C'est pourquoi je jurai dans ma colère, ils n'entreront pas dans mon repos\FTNT{Hé. 3:11.} ! Il dit cela, quoique ses œuvres aient été achevées depuis la fondation du monde.
\VS{4}Car il a parlé quelque part ainsi du septième jour : Et Dieu se reposa de toutes ses œuvres le septième jour\FTNT{Ge. 2:2.}.
\VS{5}Et encore dans ce passage : Ils n'entreront pas dans mon repos !
\VS{6}Puisqu'il reste donc à quelques-uns d'y entrer, et que ceux à qui d'abord il a été évangélisé n'y sont pas entrés à cause de leur désobéissance,
\VS{7}Dieu détermine de nouveau un certain jour, qu'il appelle aujourd'hui, en disant par David si longtemps après, selon ce qui a été dit : Aujourd'hui, si vous entendez sa voix, n'endurcissez point vos cœurs\FTNT{Ps. 95:8-11.}.
\VS{8}Car, si Josué les avait introduits dans le repos, jamais après cela il n'aurait parlé d'un autre jour.
\TextTitle{Entrer dans le repos de Dieu}
\VS{9}Il reste donc encore un repos réservé au peuple de Dieu.
\VS{10}Car celui qui est entré dans son repos, se repose aussi de ses œuvres, comme Dieu s'est reposé des siennes.
\VS{11}Efforçons-nous donc d'entrer dans ce repos-là, de peur que quelqu'un ne tombe en imitant une semblable désobéissance.
\VS{12}Car la Parole de Dieu est vivante et efficace, et plus pénétrante qu'une épée quelconque à deux tranchants, et atteignant jusqu'à la division de l'âme et de l'esprit, et des jointures et des moelles ; et elle juge les pensées et les intentions du cœur.
\VS{13}Et il n'y a aucune créature qui soit cachée devant lui, mais toutes choses sont nues et entièrement découvertes aux yeux de celui devant lequel nous devons rendre compte.
\VS{14}Ainsi, puisque nous avons un grand Souverain Sacrificateur, Jésus, le Fils de Dieu, qui a traversé les cieux, tenons ferme notre profession.
\VS{15}Car nous n'avons pas un Souverain Sacrificateur qui ne puisse avoir compassion de nos infirmités ; mais, nous avons celui qui a été tenté comme nous en toutes choses, mais sans pécher.
\VS{16}Approchons donc avec assurance du trône de la grâce, afin d'obtenir miséricorde et de trouver grâce, pour être secourus dans le temps convenable.
\Chap{5}
\TextTitle{Le service du souverain sacrificateur}
\VerseOne{}Or tout souverain sacrificateur pris d'entre les hommes est établi pour les hommes dans les choses qui concernent Dieu, afin qu'il offre des dons et des sacrifices pour les péchés.
\VS{2}Etant capable d'avoir de l'indulgence pour les ignorants et les égarés, puisqu'il est aussi lui-même enveloppé d'infirmité.
\VS{3}Et à cause de cette infirmité, il doit offrir pour les péchés, non seulement pour le peuple, mais aussi pour lui-même.
\VS{4}Et nul ne s'attribue cet honneur, si ce n'est celui qui est appelé de Dieu, comme Aaron.
\TextTitle{Christ dans l'ordre de Melchisédek}
\VS{5}De même, aussi Christ ne s'est point glorifié lui-même d'être fait Souverain Sacrificateur, mais celui qui lui a dit : C'est toi qui es mon Fils, je t'ai engendré aujourd'hui\FTNT{Ps. 2:7.} !
\VS{6}Comme il dit encore ailleurs : Tu es Sacrificateur éternellement, selon l'ordre de Melchisédek\FTNT{Ps. 110:4.}.
\VS{7}C'est lui qui, pendant les jours de sa chair, a offert avec de grands cris et avec larmes des prières et des supplications à celui qui pouvait le sauver de la mort, et il a été exaucé à cause de sa piété.
\VS{8}Quoiqu'il soit le Fils de Dieu, il a pourtant appris l'obéissance par les choses qu'il a souffertes.
\VS{9}Après avoir été consacré, il est devenu l'auteur du salut éternel pour tous ceux qui lui obéissent,
\VS{10}étant appelé de Dieu à être Souverain Sacrificateur selon l'ordre de Melchisédek ;
\TextTitle{Du lait à la nourriture solide\FTNTT{jusqu'à Hé. 6:12}}
\VS{11}de qui nous avons beaucoup de choses à dire, mais elles sont difficiles à expliquer, parce que vous êtes devenus lents à comprendre.
\VS{12}En effet, tandis que vous devriez être maîtres depuis longtemps, vous avez encore besoin qu'on vous enseigne quels sont les premiers rudiments des oracles de Dieu, et vous êtes devenus tels, que vous avez encore besoin de lait et non d'une nourriture solide.
\VS{13}Or, quiconque use de lait, ne sait point ce que c'est que la parole de la justice, parce qu'il est un enfant\FTNT{Le mot enfant dans ce passage vient du grec « nepios » qui signifie « ignorant ».}.
\VS{14}Mais la viande solide est pour ceux qui sont déjà hommes faits, {c'est-à-dire}, pour ceux qui, pour y être habitués, ont les sens exercés à discerner le bien et le mal.
\Chap{6}
\TextTitle{Tendre à la perfection}
\VerseOne{}C'est pourquoi, laissant la parole qui n'enseigne que les premiers principes de Christ, tendons à la perfection, ne posant pas de nouveau le fondement de la repentance des œuvres mortes, et de la foi en Dieu,
\VS{2}de la doctrine des baptêmes, et de l'imposition des mains, et de la résurrection des morts, et du jugement éternel.
\VS{3}Et c'est ce que nous ferons, si Dieu le permet.
\VS{4}Or il est impossible que ceux qui ont été une fois illuminés, et qui ont goûté le don céleste, et qui ont été fait participants au Saint-Esprit,
\VS{5}qui ont goûté la bonne parole de Dieu, et les puissances du siècle à venir,
\VS{6}s'ils retombent, soient changés de nouveau par la repentance, vu que, quant à eux, ils crucifient de nouveau le Fils de Dieu, et l'exposent à l'opprobre.
\VS{7}Car la terre qui est abreuvée par la pluie qui tombe souvent sur elle, et qui produit des herbes propres à ceux par qui elle est labourée, reçoit la bénédiction de Dieu ;
\VS{8}mais, celle qui produit des épines et des chardons, est rejetée et proche de malédiction, et sa fin est d'être brûlée.
\VS{9}Mais nous sommes persuadés, quoique nous parlions ainsi, en ce qui vous concerne, mes bien-aimés, des choses meilleures et qui tiennent au salut.
\VS{10}Car Dieu n'est pas injuste, pour oublier votre œuvre, et le travail de la charité que vous avez témoigné pour son Nom, en ce que vous avez secouru les saints, et que vous les secourez encore.
\VS{11}Or nous souhaitons que chacun de vous montre jusqu'à la fin le même empressement pour la pleine certitude de l'espérance,
\VS{12}afin que vous ne vous relâchiez point, mais que vous imitiez ceux qui, par la foi et par la patience, héritent ce qui leur a été promis.
\TextTitle{Christ entré au-delà du voile}
\VS{13}Car, lorsque Dieu fit la promesse à Abraham, ne pouvant jurer par un plus grand, il jura par lui-même,
\VS{14}en disant : Certainement, je te bénirai abondement et je te multiplierai merveilleusement\FTNT{Ge. 22:16-17.}.
\VS{15}Et ainsi, Abraham ayant attendu patiemment, obtint ce qui lui avait été promis.
\VS{16}Or les hommes jurent par celui qui est plus grand qu'eux, et le serment qu'ils font pour confirmer leur parole met fin à tous leurs différends.
\VS{17}C'est pourquoi Dieu, voulant faire mieux connaître aux héritiers de la promesse la fermeté immuable de sa résolution, il y a fait intervenir le serment,
\VS{18}afin que, par deux choses immuables, dans lesquelles il est impossible que Dieu mente, nous ayons une ferme consolation, nous qui avons notre refuge à obtenir l'espérance qui nous est proposée.
\VS{19}Laquelle nous tenons comme une ancre sûre et ferme de l'âme, et qui pénètre jusqu'au-delà du voile,
\VS{20}où Jésus est entré comme notre précurseur, ayant été fait Souverain Sacrificateur éternellement, selon l'ordre de Melchisédek\FTNT{Voir Ge. 14.}.
\Chap{7}
\TextTitle{Melchisédek, type de Christ\FTNTT{Ge. 14}}
\VerseOne{}En effet, ce Melchisédek était Roi de Salem et Sacrificateur du Dieu Très-Haut\FTNT{Ge. 14:18.}. Il alla au-devant d'Abraham lorsqu'il revenait de la défaite des rois, et il le bénit,
\VS{2}et auquel Abraham donna pour sa part la dîme de tout\FTNT{Ge. 14:20. Pour en savoir plus sur la dîme, voir les commentaires en De. 14:22, No. 18:21 et Mal. 3:10.}. Son nom signifie premièrement Roi de justice, et puis il a été Roi de Salem, c'est-à-dire, Roi de paix.
\VS{3}Il est sans père, sans mère, sans généalogie, n'ayant ni commencement de jours ni fin de vie, mais il est rendu semblable au Fils de Dieu. Il demeure Sacrificateur continuellement.
\TextTitle{La sacrificature de Melchisédek supérieur à celle d'Aaron}
\VS{4}Considérez donc combien est grand celui à qui même Abraham, le patriarche, donna la dîme du butin.
\VS{5}Car, quant à ceux d'entre les fils de Lévi qui reçoivent la sacrificature, ils ont bien une ordonnance de dîmer le peuple selon la loi, c'est-à-dire, de dîmer leurs frères, bien qu'ils soient sortis des reins d'Abraham.
\VS{6}Mais celui qui n'était pas de la même famille qu'eux reçut d'Abraham la dîme, et bénit celui qui avait les promesses.
\VS{7}Or sans contredit, celui qui est le moindre est béni par celui qui est le plus grand.
\VS{8}Et ici, ce sont les hommes mortels qui prennent les dîmes ; mais là, c'est celui de qui il est rendu témoignage qu'il est vivant.
\VS{9}Et pour ainsi dire, Lévi même qui prend des dîmes, les a payées en Abraham ;
\VS{10}car il était encore dans les reins de son père, quand Melchisédek alla au-devant de lui.
\TextTitle{La sacrificature selon l'ordre d'Aaron n'a rien amené à la perfection}
\VS{11}Si donc la perfection s'était trouvée dans la sacrificature lévitique, (car c'est sous elle que le peuple a reçu la loi) quel besoin était-il après cela qu'un autre sacrificateur se lève selon l'ordre de Melchisédek, et qui ne soit point nommé selon l'ordre d'Aaron ?
\VS{12}Or la sacrificature étant changée, il est nécessaire qu'il y ait aussi un changement de loi.
\VS{13}Car, celui à l'égard duquel ces choses sont dites, appartient à une autre tribu, de laquelle nul n'a assisté à l'autel ;
\VS{14}car il est évident que notre Seigneur est descendu de la tribu de Juda\FTNT{Mt. 1:2.}, à l'égard de laquelle Moïse n'a rien dit de la sacrificature.
\VS{15}Et cela est encore plus incontestable, en ce qu'un autre Sacrificateur, à la ressemblance de Melchisédek, est suscité ;
\VS{16}qui n'a point été fait Sacrificateur selon la loi du commandement charnel, mais selon la puissance de la vie impérissable.
\VS{17}Car Dieu lui rend ce témoignage : Tu es Sacrificateur éternellement, selon l'ordre de Melchisédek.
\VS{18}Or il se fait une abolition du commandement qui a précédé, à cause de sa faiblesse, et parce qu'il ne pouvait point profiter.
\VS{19}Car la loi n'a rien amené à la perfection, mais ce qui a amené à la perfection, c'est ce qui a été introduit par-dessus, à savoir une meilleure espérance, par laquelle nous approchons de Dieu.
\VS{20}D'autant plus, même que cela n'a pas été sans serment,
\VS{21}car les Lévites sont devenus sacrificateurs sans serment, mais celui-ci l'est devenu avec serment par celui qui lui a dit : Le Seigneur l'a juré, et il ne s'en repentira pas\FTNT{Voir Ps. 110:4} : Tu es Sacrificateur éternellement, selon l'ordre de Melchisédek.
\VS{22}C'est donc d'une alliance d'autant plus excellente que Jésus a été fait le garant.
\TextTitle{Les sacrificateurs sont mortels, seul Christ est éternel}
\VS{23}Et quant aux sacrificateurs, il y en a eu plusieurs qui se sont succédés parce que la mort les empêchait d'être perpétuels.
\VS{24}Mais lui, parce qu'il demeure éternellement, possède une sacrificature qui n'est pas transmissible.
\VS{25}C'est pourquoi aussi il peut sauver parfaitement ceux qui s'approchent de Dieu par lui, étant toujours vivant pour intercéder\FTNT{Le Seigneur Jésus-Christ est le modèle parfait en ce qui concerne la prière d'intercession. Il se tient devant le Père pour nous. En tant qu'homme (1 Ti. 2:5) et Souverain Sacrificateur, il se tient entre le Père et l'homme pécheur, comme le faisaient les sacrificateurs sous la loi mosaïque. Voir Lu. 22:31-32 ; Ro. 8:34 ; 1 Jn. 2:1-2.} pour eux.
\VS{26}Or, il nous était convenable d'avoir un tel Souverain Sacrificateur, saint, innocent, sans tache, séparé des pécheurs, et élevé au-dessus des cieux,
\VS{27}qui n'avait pas besoin, comme les souverains sacrificateurs, d'offrir tous les jours des sacrifices, premièrement pour ses péchés, et ensuite pour ceux du peuple, vu qu'il a fait cela une fois, s'étant offert lui-même.
\VS{28}Car, la loi établit souverains sacrificateurs des hommes faibles ; mais la parole du serment qui a été fait après la loi, établit le Fils, qui est parfait pour toujours.
\Chap{8}
\TextTitle{L'ancienne sacrificature : L'ombre des choses célestes}
\VerseOne{}La chose principale de notre discours, c'est que nous avons un tel Souverain Sacrificateur, qui est assis à la droite du trône de la majesté de Dieu dans les cieux,
\VS{2}serviteur du sanctuaire, et du véritable tabernacle, que le Seigneur a dressé et non pas les hommes.
\VS{3}Car tout souverain sacrificateur est établi pour offrir des offrandes et des sacrifices ; c'est pourquoi il est nécessaire que celui-ci ait aussi quelque chose à offrir.
\VS{4}Vu même que s'il était sur la terre, il ne serait pas sacrificateur, pendant qu'il y aurait encore des sacrificateurs qui offrent les offrandes selon la loi ;
\VS{5}lesquels font le service dans le lieu qui n'est que l'image et l'ombre des choses célestes, selon que Dieu le dit à Moïse, quand il devait achever le tabernacle : Or prends garde, lui dit-il, de faire toutes choses selon le modèle qui t'a été montré sur la montagne\FTNT{Ex. 25:40.}.
\TextTitle{Christ, le Médiateur d'une alliance plus excellente}
\VS{6}Mais maintenant, notre Souverain Sacrificateur a obtenu un service d'autant supérieur qu'il est le Médiateur d'une alliance plus excellente, qui a été établie sur de meilleures promesses.
\TextTitle{Les prophètes ont annoncé la Nouvelle Alliance}
\VS{7}En effet, si la Première Alliance avait été irréprochable, il n'y aurait pas eu lieu d'en chercher une seconde.
\VS{8}Car en censurant les Juifs, Dieu leur dit : Voici, les jours viendront, dit le Seigneur, où je traiterai avec la maison d'Israël et avec la maison de Juda une Alliance Nouvelle,
\VS{9}non selon l'alliance que je traitai avec leurs pères, le jour où je les saisis par la main pour les tirer du pays d'Egypte ; car ils n'ont pas persévéré dans mon alliance, c'est pourquoi je les ai méprisés, dit le Seigneur.
\VS{10}Mais voici l'alliance que je traiterai, après ces jours-là, avec la maison d'Israël, dit le Seigneur : Je mettrai mes lois dans leur esprit, et je les écrirai dans leur cœur, je serai leur Dieu, et ils seront mon peuple.
\VS{11}Personne n'enseignera plus son prochain, ni personne son frère, en disant : Connais le Seigneur ! Parce que tous me connaîtront, depuis le plus petit jusqu'au plus grand d'entre eux ;
\VS{12}car je serai miséricordieux par rapport à leurs injustices, et je ne me souviendrai plus de leurs péchés, ni de leurs iniquités\FTNT{Jé. 31:31-34.}.
\VS{13}En disant une Nouvelle Alliance, il a déclaré vieille la première ; or, ce qui devient vieux et ancien, est près d'être aboli.
\Chap{9}
\TextTitle{Les ordonnances et le sanctuaire de la Première Alliance : Des symboles}
\VerseOne{}En vérité, la Première Alliance avait aussi des ordonnances touchant le service divin, et un sanctuaire terrestre.\FTNT{Ex. 25:1-9.}.
\VS{2}Car il fut construit un premier tabernacle, appelé le lieu saint, dans lequel étaient le chandelier, et la table, et les pains de proposition\FTNT{Ex. 25:30.}.
\VS{3}Et après le second voile\FTNT{Ex. 26:31-35.} était le tabernacle, qui était appelé le Saint des saints,
\VS{4}ayant un encensoir d'or\FTNT{Encensoir ou autel d'or pour les parfums : Lé. 16:12.}, et l'arche de l'alliance\FTNT{Ex. 25:10.}, entièrement couverte d'or tout autour, dans laquelle était le vase d'or\FTNT{Ex. 16:33.} où était la manne, et la verge d'Aaron\FTNT{No. 17:1-10.} qui avait fleuri, et les tables de l'alliance\FTNT{Les tables de l'alliance ou tables du témoignage : Ex. 34:29 ; De. 10:2-5.}.
\VS{5}Et au-dessus de l'arche étaient les chérubins de la gloire, couvrant de leur ombre le propitiatoire\FTNT{Propitiatoire ou couvercle de l'arche de l'alliance : Lé. 9:7 ; Lé. 16:15-17.}. Ce n'est pas le moment de parler en détail là-dessus.
\VS{6}Or ces choses étant ainsi disposées, les sacrificateurs qui font le service entrent en tout temps dans le premier tabernacle\FTNT{No. 28:3.} ;
\VS{7}mais seul le souverain sacrificateur entre dans le second une fois par an, non sans y porter du sang, qu'il offre pour lui-même et pour les péchés du peuple\FTNT{Lé. 16:34.}.
\VS{8}Le Saint-Esprit faisant connaitre par là que le chemin du Saint des saints n'était pas encore manifesté, tandis que le premier tabernacle était encore debout,
\VS{9}lequel était une figure destinée pour le temps présent, durant lequel étaient offerts des offrandes et des sacrifices qui ne pouvaient point sanctifier la conscience de celui qui faisait le service,
\VS{10}ordonnés seulement en aliments, et en breuvages, en diverses ablutions, et en des cérémonies charnelles, jusqu'au temps de la réforme.
\TextTitle{La réalité du sacrifice s'accomplit en Christ}
\VS{11}Mais Christ est venu comme Souverain Sacrificateur des biens à venir ; il a traversé un tabernacle plus excellent et plus parfait, qui n'est pas un tabernacle construit de main d'homme, c'est-à-dire, qui n'est pas de cette création ;
\VS{12}et il est entré une fois pour toutes dans le Saint des saints, non avec le sang des veaux ou des boucs, mais avec son propre sang, après avoir obtenu une rédemption éternelle.
\VS{13}Car si le sang des taureaux et des boucs, et la cendre de la génisse\FTNT{No. 19:1-12.}, répandue sur ceux qui sont souillés, sanctifient et procurent la pureté de la chair,
\VS{14}combien plus le sang de Christ, qui, par l'Esprit éternel, s'est offert lui-même à Dieu sans nulle tache, purifiera-t-il votre conscience des œuvres mortes, pour servir le Dieu vivant ?
\VS{15}C'est pourquoi il est le Médiateur de la Nouvelle Alliance, afin que, la mort étant intervenue pour la rançon des transgressions commises sous la Première Alliance, ceux qui ont été appelés reçoivent l'héritage éternel qui leur a été promis.
\TextTitle{Les clauses du testament du Messie}
\VS{16}Car là où il y a un testament, il est nécessaire que la mort du testateur intervienne,
\VS{17}parce que c'est par la mort du testateur qu'un testament est rendu ferme, puisqu'il n'a aucune force tant que le testateur est en vie.
\VS{18}C'est pourquoi la Première Alliance elle-même n'a point été confirmée sans le sang.
\VS{19}Car Moïse, après avoir prononcé devant tout le peuple tous les commandements de la loi, prit le sang des veaux et des boucs, avec de l'eau, et de la laine écarlate, et de l'hysope ; et il en fit l'aspersion sur le livre et sur tout le peuple, en disant :
\VS{20}Ceci est le sang de l'Alliance que Dieu vous a ordonné d'observer\FTNT{Ex. 24:3-8.}.
\VS{21}Puis il fit aussi aspersion avec du sang sur le tabernacle et sur tous les ustensiles du service\FTNT{Ex. 29:12 ; Ex. 29:36.}.
\VS{22}Et presque toutes choses, selon la loi, sont purifiées par le sang, et sans effusion de sang il n'y a pas de rémission des péchés.
\TextTitle{Un sacrifice plus excellent\FTNTT{Lé. 16:33}}
\VS{23}Il a donc fallu que les choses qui représentaient celles qui sont aux cieux, soient purifiées par de telles choses, mais que les célestes le soient par des sacrifices plus excellents que ceux-là.
\VS{24}Car Christ n'est pas entré dans un sanctuaire fait de main d'homme, et qui n'était que la figure du véritable, mais il est entré dans le ciel même, afin de comparaître maintenant pour nous devant la face de Dieu.
\VS{25}Et ce n'est pas pour s'offrir lui-même plusieurs fois qu'il y est entré, ainsi que le souverain sacrificateur entre dans le Saint des saints, chaque année, avec un autre sang ;
\VS{26}autrement, il aurait fallu qu'il ait souffert plusieurs fois depuis la création du monde ; mais maintenant, à la fin des siècles, il a paru une seule fois pour l'abolition du péché par son sacrifice.
\VS{27}Et comme il est réservé aux hommes de mourir une seule fois\FTNT{Ce passage réfute la doctrine de la réincarnation.}, et après cela suit le jugement,
\VS{28}de même aussi Christ, qui s'est offert une seule fois pour ôter les péchés de plusieurs, apparaîtra sans péché une seconde fois à ceux qui l'attendent pour le salut.
\Chap{10}
\TextTitle{Le sacrifice unique de Christ est supérieur à tous les sacrifices}
\VerseOne{}Car la loi qui possède l'ombre des biens à venir, et non l'image exacte des choses, ne peut jamais, par les mêmes sacrifices que l'on offre continuellement chaque année, sanctifier ceux qui s'y attachent.
\VS{2}Autrement, n'auraient-ils pas cessé d'être offerts ? Parce que les adorateurs une fois expurgés, n'auraient plus eu conscience des péchés.
\VS{3}Or le souvenir des péchés est réitéré dans ces sacrifices chaque année ;
\VS{4}car il est impossible que le sang des taureaux et des boucs ôte les péchés.
\VS{5}C'est pourquoi Jésus-Christ, en entrant dans le monde, a dit : Tu n'as pas voulu de sacrifice, ni d'offrande, mais tu m'as formé un corps ;
\VS{6}tu n'as pas pris plaisir aux holocaustes, ni aux sacrifices pour le péché\FTNT{Ps. 40:7-9.}.
\VS{7}Alors j'ai dit : Me voici, je viens, il est écrit de moi au commencement du livre : Que je fasse, ô Dieu, ta volonté !
\VS{8}Après avoir dit d'abord : Tu n'as pas voulu de sacrifice, ni d'offrande, ni d'holocauste, ni d'offrande pour le péché et tu n'y as point pris plaisir, lesquelles choses sont pourtant offertes selon la loi, alors il dit : Me voici, je viens afin de faire, ô Dieu, ta volonté !
\VS{9}Il abolit ainsi le premier afin d'établir le second.
\VS{10}Or c'est par cette volonté que nous sommes sanctifiés, à savoir par l'offrande du corps de Jésus-Christ qui a été faite une fois pour toutes.
\VS{11}De plus, tout sacrificateur fait chaque jour le service et offre souvent les mêmes sacrifices, qui ne peuvent jamais ôter les péchés,
\VS{12}mais lui, après avoir offert un seul sacrifice pour les péchés, s'est assis pour toujours à la droite de Dieu,
\VS{13}attendant désormais que ses ennemis soient mis pour le marchepied de ses pieds.
\VS{14}Car, par une seule offrande, il a rendu parfaits pour toujours ceux qui sont sanctifiés.
\VS{15}Et c'est aussi ce que le Saint-Esprit nous témoigne ; car, après avoir dit premièrement :
\VS{16}Voici l'alliance que je ferai avec eux, après ces jours-là, dit le Seigneur\FTNT{Voir Jé. 31:31-34.} : C'est que je mettrai mes lois dans leur cœur, et je les écrirai dans leur esprit ;
\VS{17}et je ne me souviendrai plus de leurs péchés, ni de leurs iniquités.
\VS{18}Or, là où les péchés sont pardonnés, il n'y a plus d'offrande pour le péché.
\TextTitle{Exhortation à s'approcher de Dieu avec foi}
\VS{19}Ainsi donc, mes frères, nous avons la liberté d'entrer dans le Saint des saints au moyen du sang de Jésus,
\VS{20}qui est le chemin\FTNT{Jésus est le chemin qui conduit au Saint des saints, à la vie (Voir Jn. 14:6), et ce chemin n'était pas encore manifesté avant sa naissance. Hé. 9:8.} nouveau et vivant qu'il a inauguré pour nous à travers le voile, c'est-à-dire sa propre chair,
\VS{21}et ayant un Souverain Sacrificateur établi sur la maison de Dieu,
\VS{22}approchons-nous de lui avec un cœur sincère, et une foi inébranlable, ayant les cœurs purifiés d'une mauvaise conscience, et le corps lavé d'une eau pure.
\VS{23}Retenons fermement la profession de notre espérance, car celui qui nous a fait la promesse est fidèle.
\VS{24}Veillons les uns sur les autres pour nous exciter à la charité et aux bonnes œuvres.
\VS{25}N'abandonnons pas notre assemblée\FTNT{Assemblée : Du grec « episunagoge » qui veut dire « être assemblé en un lieu, assemblée religieuse des chrétiens ». Il est question de ne pas abandonner la communion fraternelle et non une église locale. En effet, il est du devoir du chrétien de se séparer des faux frères de peur d'être entraîné dans leur égarement (Mt. 18:15-17 ; 1 Co. 5:11 ; 1 Co. 15:33).}, comme c'est la coutume de quelques-uns ; mais exhortons-nous les uns les autres, et cela d'autant plus que vous voyez approcher le jour.
\TextTitle{Ne pas mépriser le sacrifice de Christ}
\VS{26}Car, si nous péchons volontairement après avoir reçu la connaissance de la vérité, il ne reste plus de sacrifice pour les péchés,
\VS{27}mais une attente terrible du jugement et l'ardeur d'un feu qui doit dévorer les adversaires.
\VS{28}Si quelqu'un avait méprisé la loi de Moïse, il mourait sans miséricorde, sur la déposition de deux ou de trois témoins\FTNT{De. 17:6.} ;
\VS{29}de combien pires tourments pensez-vous donc que sera jugé digne celui qui aura foulé aux pieds le Fils de Dieu, et qui aura tenu pour une chose profane le sang de l'Alliance, par lequel il avait été sanctifié, et qui aura outragé l'Esprit de grâce ?
\VS{30}Car nous connaissons celui qui a dit : C'est à moi que la vengeance appartient, et je le rendrai ! Dit le Seigneur. Et encore : Le Seigneur jugera son peuple.\FTNT{De. 32:35-36.}
\VS{31}C'est une chose terrible que de tomber entre les mains du Dieu vivant.
\VS{32}Or rappelez-vous des premiers jours, où, après avoir été éclairés, vous avez soutenu un grand combat de souffrances,
\VS{33}ayant été, d'une part, exposés à la vue de tout le monde par des opprobres et des afflictions, et de l'autre, ayant participé aux maux de ceux qui ont souffert de semblables indignités.
\VS{34}Car vous avez aussi été participants de l'affliction de mes liens, et vous avez reçu avec joie l'enlèvement de vos biens, sachant en vous-mêmes que vous avez dans les cieux des biens meilleurs et permanents.
\VS{35}N'abandonnez donc pas cette fermeté que vous avez fait paraître, et qui sera bien récompensée.
\VS{36}Parce que vous avez besoin de patience, afin qu'après avoir fait la volonté de Dieu, vous receviez l'effet de sa promesse.
\TextTitle{La marche par la foi : Exemples d'hommes et de femmes de foi}
\VS{37}Car, encore un peu de temps, et celui qui doit venir, viendra, et il ne tardera point.
\VS{38}Or le juste vivra de la foi ; mais si quelqu'un se retire, mon âme ne prend point de plaisir en lui\FTNT{Ha. 2:4.}.
\VS{39}Mais pour nous, nous ne sommes pas de ceux qui se retirent ; ce serait notre perdition ; mais nous persévérons dans la foi, pour le salut de l'âme.
\Chap{11}
\VerseOne{}Or la foi, rend présentes les choses qu'on espère, et elle est une démonstration de celles qu'on ne voit point.
\VS{2}Car c'est par elle que les anciens ont obtenu un bon témoignage.
\VS{3}Par la foi, nous comprenons que l'univers a été fait par la parole de Dieu, de sorte que les choses qui se voient, n'ont pas été faites des choses visibles.
\VS{4}[Abel] Par la foi, Abel\FTNT{Ge. 4:3-5.} offrit à Dieu un sacrifice plus excellent que Caïn ; et par elle il obtînt le témoignage d'être juste, parce que Dieu rendait témoignage de ses offrandes ; et c'est par elle qu'il parle encore, quoique mort.
\VS{5}[Hénoc]Par la foi, Hénoc\FTNT{Ge. 5:22-24.} fut enlevé pour ne pas voir la mort, et il ne parut plus parce que Dieu l'avait enlevé ; car, avant qu'il soit enlevé, il avait obtenu le témoignage d'avoir été agréable à Dieu.
\VS{6}Or il est impossible de lui être agréable sans la foi ; car il faut que celui qui vient à Dieu, croie que Dieu est, et qu'il est le rémunérateur de ceux qui le cherchent.
\VS{7}[Noé]Par la foi, Noé\FTNT{Ge. 6:14-22.}, ayant été divinement averti des choses qui ne se voyaient point encore, craignit, et bâtit l'arche pour la conservation de sa famille ; et par cette arche, il condamna le monde, et devint héritier de la justice qui est selon la foi.
\VS{8}[Abraham et Sara]Par la foi, Abraham\FTNT{Ge 12:1-4.}, étant appelé, obéit, pour aller sur la terre, qu'il devait recevoir en héritage, et il partit sans savoir où il allait.
\VS{9}Par la foi, il demeura comme étranger sur la terre, qui lui avait été promise, comme si elle ne lui avait point appartenu, demeurant sous des tentes avec Isaac et Jacob, qui étaient héritiers avec lui de la même promesse.
\VS{10}Car il attendait la cité qui a des fondements, celle dont Dieu est l'architecte et le constructeur.
\VS{11}Par la foi, aussi Sara\FTNT{Ge. 21:1-2.} reçut la force de concevoir un enfant, et elle enfanta hors d'âge, parce qu'elle fut persuadée que celui qui le lui avait promis, était fidèle.
\VS{12}C'est pourquoi d'un seul homme, et qui était déjà affaibli, il est né une multitude aussi nombreuse que les étoiles du ciel, et que le sable du bord de la mer, qui ne peut se compter\FTNT{Ge. 22:17.}.
\VS{13}Tous ceux-ci sont morts dans la foi, sans avoir reçu les choses dont ils avaient eu les promesses, mais ils les ont vues de loin, crues, et saluées, et ils ont fait profession qu'ils étaient étrangers et voyageurs sur la terre\FTNT{1 Pi. 2:11.}.
\VS{14}Car ceux qui tiennent ces discours montrent clairement qu'ils cherchent encore leur patrie.
\VS{15}Et certes, s'ils avaient eu en vue celle d'où ils étaient sortis, ils auraient eu le temps d'y retourner.
\VS{16}Mais maintenant, ils en désirent une meilleure, c'est-à-dire une céleste. C'est pourquoi Dieu n'a pas honte d'être appelé leur Dieu, parce qu'il leur a préparé une cité\FTNT{Jn. 14:2 ; Ap. 21:2.}.
\VS{17}Par la foi, Abraham étant éprouvé, offrit Isaac ; celui qui avait reçu les promesses offrit même son fils unique\FTNT{Ge. 22:1.},
\VS{18}à l'égard duquel il lui avait été dit : Les descendants d'Isaac seront ta véritable postérité\FTNT{Ge. 21:12.}.
\VS{19}Ayant estimé que Dieu pouvait même le ressusciter d'entre les morts ; c'est pourquoi aussi il le recouvra par une espèce de résurrection.
\VS{20}[Isaac]Par la foi, Isaac bénit Jacob et Esaü, en vue des choses à venir\FTNT{Ge. 27:26-40.}.
\VS{21}[Jacob]Par la foi, Jacob mourant bénit chacun des fils de Joseph\FTNT{Ge. 48:1-22.}, et adora Dieu, appuyé sur l'extrémité de son bâton\FTNT{Ge. 47:31.}.
\VS{22}[Joseph]Par la foi, Joseph mourant fit mention de la sortie des enfants d'Israël, et il donna des ordres au sujet de ses os\FTNT{Ge. 50:24-25.}.
\VS{23}[Les parents de Moïse]Par la foi, Moïse\FTNT{Ex. 2:1-3.}, à sa naissance, fut caché pendant trois mois par son père et sa mère, parce qu'ils virent que l'enfant était beau, et ils ne craignirent pas l'ordre du roi.
\VS{24}[Moïse]Par la foi, Moïse, devenu grand, refusa d'être nommé fils de la fille de Pharaon,
\VS{25}choisissant plutôt d'être affligé avec le peuple de Dieu, que de jouir pour un peu de temps des délices du péché.
\VS{26}Et ayant estimé que l'opprobre de Christ était un plus grand trésor que les richesses de l'Egypte, parce qu'il avait égard à la rémunération.
\VS{27}Par la foi, il quitta l'Egypte, sans craindre la fureur du roi ; car il demeura ferme, comme voyant celui qui est invisible.
\VS{28}Par la foi, il fit la Pâque et l'aspersion du sang, afin que le destructeur qui tuait les premiers-nés, ne touche pas aux premiers-nés des Israélites\FTNT{Ex. 12:1-51.}.
\VS{29}Par la foi, ils traversèrent la Mer Rouge, comme un lieu sec, ce que les Egyptiens essayèrent de tenter, ils furent engloutis dans les eaux\FTNT{Ex. 14:13-31.}.
\VS{30}[Josué]Par la foi, les murs de Jéricho tombèrent, après qu'on en eut fait le tour pendant sept jours\FTNT{Jos. 6:1-20.}.
\VS{31}[Rahab]Par la foi, Rahab, la prostituée, ne périt pas avec les incrédules, parce qu'elle avait reçu les espions et les avait renvoyés en paix\FTNT{Jos. 2:1-21 ; Jos. 6:23.}.
\VS{32}[Autres hommes et femmes de foi]Et que dirai-je encore ? Car le temps me manquerait si je voulais parler de Gédéon\FTNT{Jg. 6:11.}, et de Barak\FTNT{Jg. 4:6.}, et de Samson\FTNT{Jg. 13:24.}, et de Jephté\FTNT{Jg. 11:1.}, et de David\FTNT{1 S. 16 ; 17.}, et de Samuel\FTNT{1 S. et 2 S.}, et des prophètes,
\VS{33}qui par la foi combattirent des royaumes, exercèrent la justice, obtinrent des promesses, fermèrent la gueule des lions,
\VS{34}éteignirent la force du feu, échappèrent au tranchant des épées, des malades devinrent vigoureux, se montrèrent fort dans la bataille, et mirent en fuite des armées étrangères.
\VS{35}Des femmes recouvrèrent leurs morts par le moyen de la résurrection ; et d'autres furent livrés aux tourments et n'acceptèrent point d'être délivrés, afin d'obtenir une meilleure résurrection.
\VS{36}Et d'autres subirent les moqueries et le fouet, les chaînes et la prison ;
\VS{37}ils furent lapidés, sciés, subirent de rudes épreuves, ils furent mis à mort par le tranchant de l'épée, ils errèrent çà et là, vêtus de peaux de brebis et de chèvres, réduits à la misère, affligés, tourmentés,
\VS{38}eux dont le monde n'était pas digne, errant dans les déserts et dans les montagnes, et dans les cavernes et dans les trous de la terre.
\VS{39} Et quoiqu'ils aient tous été recommandables par leur foi, ils n'ont pourtant point reçu l'effet de la promesse,
\VS{40}Dieu ayant pourvu quelque chose de meilleur pour nous, en sorte qu'ils ne parviennent pas à la perfection sans nous.
\Chap{12}
\TextTitle{Fixer les regards sur Jésus}
\VerseOne{}Nous donc aussi, puisque nous sommes environnés d'une si grande nuée de témoins\FTNT{Témoin : du grec « martus », terme qui dans un sens légal et historique signifie « celui qui est spectateur d'une chose ». Dans un sens éthique, il est question de « ceux qui ont prouvé la force et l'authenticité de leur foi en Christ en supportant une mort violente ». « Martus » a donné le mot « martyr » en français.}, rejetons tout fardeau, et le péché qui nous enveloppe si aisément, et poursuivons constamment la course qui nous est proposée,
\VS{2}portant les yeux sur Jésus, le chef et le consommateur de la foi qui en échange de la joie qui lui était réservée, il a souffert la croix, ayant méprisé la honte, et s'est assis à la droite du trône de Dieu.
\VS{3}C'est pourquoi, considérez soigneusement celui qui a supporté contre sa personne une telle opposition de la part des pécheurs, afin que vous ne succombiez point, en perdant courage.
\TextTitle{La correction du Père}
\VS{4}Vous n'avez pas encore résisté jusqu'à répandre votre sang en combattant contre le péché.
\VS{5}Et cependant vous avez oublié l'exhortation qui vous est adressée comme à ses fils, disant : Mon fils, ne méprise pas le châtiment du Seigneur, et ne perds point courage lorsqu'il te reprend ;
\VS{6}car le Seigneur châtie celui qu'il aime, et il frappe de la verge tous ceux qu'il reconnaît pour ses fils\FTNT{Pr. 3:11-12.}.
\VS{7}Si vous endurez le châtiment, Dieu se présente à vous comme à ses fils ; car qui est le fils que le père ne châtie point ?
\VS{8}Mais si vous êtes sans châtiment auquel tous participent, vous êtes donc des enfants illégitimes, et non pas des fils.
\VS{9}Et puisque nos pères selon la chair nous ont châtiés, et que malgré cela nous les avons respectés, ne serons-nous pas beaucoup plus soumis au Père des esprits, pour avoir la vie ?
\VS{10}Car par rapport à ceux-là, ils nous châtiaient pour un peu de temps, suivant leur volonté, mais celui-ci nous châtie pour notre profit, afin que nous soyons participants de sa sainteté.
\VS{11}Or tout châtiment ne semble pas sur l'heure être un sujet de joie, mais de tristesse ; mais ensuite il produit un fruit paisible de justice à ceux qui sont exercés par ce moyen.
\VS{12}Fortifiez donc vos mains languissantes et vos genoux affaiblis ;
\VS{13}et suivez avec vos pieds des chemins droits, afin que ce qui est boiteux ne dévie pas, mais plutôt se consolide.
\VS{14}Recherchez la paix avec tous, et la sanctification, sans laquelle nul ne verra le Seigneur.
\TextTitle{Que nul ne se prive de la grâce de Dieu}
\VS{15}Veillez à ce que personne ne se prive de la grâce de Dieu ; à ce qu'aucune racine d'amertume, poussant des rejetons, ne vous trouble, et que plusieurs n'en soient souillés par elles ;
\VS{16}que nul de vous ne soit fornicateur, ou profane comme Esaü, qui pour un aliment vendit son droit d'aînesse\FTNT{Ge. 25:33}.
\VS{17}Car vous savez que plus tard, désirant hériter la bénédiction, il fut rejeté, car il ne trouva point de lieu à la repentance, quoiqu'il l'ait demandée avec larmes.
\TextTitle{L'Eglise véritable s'est approchée de Sion}
\VS{18}Vous ne vous êtes pas approchés d'une montagne qu'on pouvait toucher avec la main\FTNT{Ex. 19:12.}, ni du feu brûlant, ni de la nuée épaisse, ni des ténèbres, ni de la tempête,
\VS{19}ni du retentissement de la trompette, ni du son des paroles, au sujet duquel ceux qui l'entendirent prièrent que la parole ne leur soit plus adressée\FTNT{Ex. 20:18-26.},
\VS{20}car ils ne pouvaient pas supporter ce qui était ordonné, que si même une bête touche la montagne, elle sera lapidée ou percée d'un dard\FTNT{Ex. 19:13.}.
\VS{21}Et ce spectacle était si terrible que Moïse dit : Je suis épouvanté et tout tremblant !
\VS{22}Mais vous vous êtes approchés de la montagne de Sion, de la Cité du Dieu vivant, la Jérusalem céleste, d'une multitude innombrable d'anges,
\VS{23}et de l'assemblée et de l'Eglise des premiers-nés qui sont inscrits dans les cieux, du Dieu qui est le juge de tous, et des esprits des justes qui ont été rendus parfaits,
\VS{24}de Jésus, qui est le médiateur de la Nouvelle Alliance, et du sang de l'aspersion, qui prononce des meilleurs choses que celui d'Abel.
\TextTitle{Exhortation à la crainte de Dieu}
\VS{25}Prenez garde de ne pas mépriser celui qui vous parle ; car si ceux qui méprisèrent celui qui leur parlait sur la terre, n'ont pas échappé, nous serons punis beaucoup plus, si nous nous détournons de celui qui parle des cieux,
\VS{26}lui, dont la voix ébranla alors la terre, mais à l'égard du temps présent, il a fait cette promesse, disant : J'ébranlerai encore une fois non seulement la terre, mais aussi le ciel\FTNT{Ag. 2:6.}.
\VS{27}Or ces mots : Une fois encore, marquent le changement des choses ébranlées, comme étant faites pour un temps, afin que celles qui sont inébranlables demeurent.
\VS{28}C'est pourquoi, saisissant le Royaume qui ne peut point être ébranlé, retenons la grâce par laquelle nous servions Dieu, en sorte que nous lui soyons agréables avec respect et avec crainte,
\VS{29}car notre Dieu est aussi un feu dévorant\FTNT{De. 4:24.}.
\Chap{13}
\TextTitle{Exhortations ; invariabilité de Christ}
\VerseOne{}Que la charité fraternelle demeure dans vos cœurs.
\VS{2}N'oubliez pas l'hospitalité ; car, par elle quelques-uns ont logé des anges sans le savoir.
\VS{3}Souvenez-vous des prisonniers, comme si vous étiez emprisonnés avec eux ; et de ceux qui sont maltraités, comme étant aussi vous-mêmes du même corps.
\VS{4}Le mariage est honorable entre tous, et le lit sans souillure ; mais Dieu jugera les fornicateurs et les adultères.
\VS{5}Que votre conduite soit sans avarice, étant contents de ce que vous avez présentement ; car lui-même a dit : Je ne te délaisserai point, et je ne t'abandonnerai point\FTNT{De. 31:6.}.
\VS{6}De sorte que nous pouvons dire avec assurance : Le Seigneur est mon aide, et je ne craindrai point ce que l'homme pourrait me faire\FTNT{Ps. 118:6.}.
\VS{7}Souvenez-vous de vos conducteurs qui vous ont annoncé la parole de Dieu ; considérez quelle a été la fin de leur vie, et imitez leur foi.
\VS{8}Jésus-Christ est le même hier, aujourd'hui, et il l'est aussi éternellement.
\VS{9}Ne soyez point emportés çà et là par des doctrines diverses et étrangères ; car il est bon que le cœur soit affermi par la grâce, et non point par les aliments, lesquelles n'ont en rien profité à ceux qui s'y sont attachés.
\TextTitle{Porter ses regards sur la cité céleste}
\VS{10}Nous avons un autel dont ceux qui servent dans le tabernacle n'ont pas le droit de manger.
\VS{11}Car les corps des animaux, dont le sang est porté dans le sanctuaire par le souverain sacrificateur pour le péché, sont brûlés hors du camp.
\VS{12}C'est pourquoi aussi Jésus, afin de sanctifier le peuple par son propre sang, a souffert hors de la porte\FTNT{Ex. 29:14. Jésus a souffert hors de Jérusalem (Jn. 19:17-18).}.
\VS{13}Sortons donc vers lui, hors du camp\FTNT{Le mot « camp » dans ce passage vient du grec « parambole », terme faisant référence au judaïsme antique dans lequel s'étaient embourbés les chrétiens d'origine hébraïque. Aujourd'hui, il représente plutôt le christianisme paganisé, essentiellement basé sur la loi de Moïse et constituant une prison qui empêche certains enfants de Dieu de vivre pleinement leur liberté en Christ.}, en portant son opprobre.
\VS{14}Car nous n'avons point ici-bas de cité permanente, mais nous recherchons celle qui est à venir.
\TextTitle{Le sacrifice de louange et du serviteur de Dieu}
\VS{15}Offrons donc par lui sans cesse à Dieu un sacrifice de louange, c'est-à-dire, le fruit des lèvres, en confessant son Nom.
\VS{16}Or, n'oubliez pas la bienfaisance et de faire part de vos biens, car Dieu prend plaisir à de tels sacrifices.
\TextTitle{L'obéissance aux conducteurs}
\VS{17}Obéissez\FTNT{Le terme « obéissez », en grec « peitho », veut dire « se laisser persuader par des mots ». Il signifie aussi « donner avec persuasion l'envie à quelqu'un de faire quelque chose en le rassurant ». Par conséquent, les conducteurs doivent comprendre que la soumission et l'obéissance des chrétiens n'a rien à voir avec la dictature et l'autoritarisme. Ils doivent les rassurer et les convaincre - car tout ce qui n'est pas fait avec foi est péché (Ro. 14:23) - et ne pas tyranniser leurs frères en les obligeant à leur obéir (Mt. 20:25 ; 1 Pi. 5:2-3).} à vos conducteurs, et soyez-leur soumis, car ils veillent pour vos âmes, comme devant en rendre compte ; afin que ce qu'ils en font, ils le fassent avec joie, et non en gémissant, car cela ne vous serait pas profitable.
\VS{18} Priez pour nous, car nous nous assurons que nous avons une bonne conscience, désirant nous conduire honnêtement parmi tous.
\VS{19}C'est avec instance que je vous demande de le faire, afin que je vous sois rendu plus tôt.
\TextTitle{Bénédictions et salutations}
\VS{20}Que le Dieu de paix, qui a ramené d'entre les morts le grand Pasteur des brebis, par le sang de l'Alliance éternelle, notre Seigneur Jésus-Christ,
\VS{21}vous rende capables de toute bonne œuvre pour faire sa volonté ; qu'il fasse en vous ce qui lui est agréable par Jésus-Christ ; auquel soit la gloire aux siècles des siècles ! Amen !
\VS{22}Aussi, mes frères, je vous prie de supporter la parole d'exhortation, car je vous ai écrit en peu de mots.
\VS{23}Sachez que notre frère Timothée a été relâché ; s'il vient bientôt, je vous verrai avec lui.
\VS{24}Saluez tous vos conducteurs, et tous les saints. Ceux d'Italie vous saluent.
\VS{25}Que la grâce soit avec vous tous ! Amen !
\PPE{}
\end{multicols}
