\ShortTitle{Hébreux}\BookTitle{Hébreux}\BFont
\noindent\hrulefill
{\footnotesize
\textit{
\bigskip
{\centering{}
\\Thème : Le sacerdoce de Christ
\\Auteur : Probablement Paul
\\Date de rédaction : Env. 68\\}
}
%\bigskip
\textit{
\\La lettre fut rédigée avant la destruction de Jérusalem car le temple y subsistait encore. Elle s’adressait à des juifs convertis connaissant bien l’auteur. Parmi eux, certains étaient tentés de retourner au judaïsme du fait des persécutions. Mais n’ayant pas résisté jusqu’au sang, il semblerait que ces chrétiens se situaient à Rome avant la persécution de Néron (37-68). L’auteur désirait affermir ces chrétiens en leur montrant que l’objectif de la loi a été réalisé par Christ qui est supérieur aux anges, aux prophètes et à Moïse. Il leur montra combien son œuvre rédemptrice était parfaite et les appela à suivre le Seigneur avec une foi indéfectible en persévérant dans l’amour fraternel.\bigskip
}
}
\par\nobreak\noindent\hrulefill
\begin{multicols}{2}
\TextTitle{[Dieu parle par le Fils]}
\Chap{1}
\VerseOne{}Dieu ayant autrefois parlé à nos pères par les prophètes, à plusieurs reprises et de plusieurs manières,
\VS{2}nous a parlé dans ces derniers jours{\FTNT{Les derniers jours ont commencé avec la naissance de l’église. Voir Joë. 2:28 ; Ac. 2:14-17.}} par son Fils, qu'il a établi héritier de toutes choses, et par lequel il a aussi créé les siècles,
\VS{3}et qui étant la splendeur de sa gloire, et l'empreinte de sa substance, et soutenant toutes choses par sa parole puissante, après avoir fait par lui-même la purification de nos péchés, s'est assis à la droite de la Majesté divine dans les lieux très hauts.
\TextTitle{[Le Fils supérieur aux anges]}
\VS{4}Il est devenu d'autant supérieur aux anges, qu'il a hérité d’un Nom plus excellent que le leur.
\VS{5}Car auquel des anges Dieu a-t-il jamais dit : Tu es mon Fils, je t'ai engendré aujourd'hui\FTNT{Ps. 2:7.} ? Et encore : Je serai pour lui un Père, et il sera pour moi un Fils{\FTNT{2 S. 7:14.}} ?
\VS{6}Et quand il introduit de nouveau dans le monde son Fils premier-né\FTNT{Voir commentaire en Co. 1:15.}, il dit : Que tous les anges de Dieu l'adorent\FTNT{Ps. 97:7.}.
\VS{7}A l’égard des anges, Dieu dit : Il fait de ses anges des vents, et ses ministres des flammes de feu\FTNT{Ps. 104:4.}.
\VS{8}Mais à l’égard du Fils, il dit : Ô Dieu ! Ton trône demeure aux siècles des siècles, et le sceptre de ton Royaume est un sceptre d'équité ;
\VS{9}Tu as aimé la justice, et tu as haï l'iniquité ; c'est pourquoi, ô Dieu ! Ton Dieu t'a oint d'une huile de joie au-dessus de tous tes semblables\FTNT{Ps. 45:7-8.}.
\VS{10}Et dans un autre endroit : Toi, Seigneur, tu as fondé la terre dès le commencement, et les cieux sont les ouvrages de tes mains ;
\VS{11}Ils périront, mais tu subsistes ; ils vieilliront tous comme un vêtement ;
\VS{12}Et tu les rouleras comme un habit, et ils seront changés ; mais toi, tu restes le même, et tes années ne finiront point\FTNT{Ps. 102:27-28 ; Es. 50:9 ; Es. 51:6.}.
\VS{13}Et auquel des anges a-t-il jamais dit : Assieds-toi à ma droite, jusqu'à ce que je fasse de tes ennemis ton marchepied\FTNT{Ps. 110:1.} ?
\VS{14}Ne sont-ils pas tous des esprits administrateurs, envoyés pour servir en faveur de ceux qui doivent hériter du salut ?
\TextTitle{[Avertissement à ne pas négliger le salut]}
\Chap{2}
\VerseOne{}C'est pourquoi il faut prendre garde de plus près aux choses que nous avons entendues de peur que nous soyons emportés loin d’elles.
\VS{2}Car si la parole annoncée par les anges a eu son effet, et si toute transgression et toute désobéissance ont reçu une juste rétribution,
\VS{3}comment échapperons-nous, si nous négligeons un si grand salut, qui, d’abord annoncé par le Seigneur, nous a été confirmé par ceux qui l'avaient entendu ?
\VS{4}Dieu appuyant leur témoignage par des prodiges et des miracles, et par plusieurs autres différents effets de sa puissance, et par les dons du Saint-Esprit, selon sa volonté.
\TextTitle{[Toutes choses doivent être soumises à Christ]}
\VS{5}Car ce n'est pas aux anges qu'il a soumis le monde à venir dont nous parlons.
\VS{6}Or quelqu'un a rendu quelque part ce témoignage : Qu'est-ce que l'homme pour que tu te souviennes de lui ? Ou le fils de l'homme pour que tu le visites ?
\VS{7}Tu l'as abaissé pour un peu de temps au-dessous des anges, tu l'as couronné de gloire et d'honneur, et tu l'as établi sur les ouvrages de tes mains.
\VS{8}Tu as mis toutes choses sous ses pieds\FTNT{Ps. 8:5-7.}. En effet, en lui soumettant toutes choses, Dieu n'a rien laissé qui ne lui soit pas soumis. Cependant, nous ne voyons pas encore que toutes choses lui soient soumises.
\TextTitle{[Jésus abaissé un peu de temps pour sauver l'homme]}
\VS{9}Mais celui qui a été abaissé pour un peu de temps au-dessous des anges, Jésus, nous le voyons couronné de gloire et d'honneur, à cause de la mort qu’il a soufferte ; ainsi par la grâce de Dieu, il a souffert la mort pour tous.
\VS{10}Car il était convenable que celui pour qui et par qui sont toutes choses, et qui voulait conduire à la gloire beaucoup de fils, ait élevé à la perfection par les souffrances le Prince de leur salut.
\VS{11}Car celui qui sanctifie et ceux qui sont sanctifiés descendent tous d'un même père ; c'est pourquoi il n’a pas honte de les appeler ses frères,
\VS{12}lorsqu’il dit : J'annoncerai ton Nom à mes frères, et je te louerai au milieu de l'assemblée\FTNT{Ps. 22:23.}.
\VS{13}Et encore : Je me confierai en lui. Et encore : Me voici, moi et les enfants que Dieu m'a donnés\FTNT{Es. 8:17-18.}.
\VS{14}Ainsi donc, puisque les enfants participent au sang et à la chair, il y a également participé lui-même, afin que par la mort il rende impuissant celui qui avait la puissance de la mort, c'est-à-dire le diable.
\VS{15}Ainsi, il délivre tous ceux qui par crainte de la mort, étaient retenus dans la servitude toute leur vie.
\VS{16}Car assurément ce n’est pas à des anges qu’il vient en aide, mais c’est à la postérité d'Abraham.
\VS{17}De sorte qu’il a dû être rendu semblable en toutes choses à ses frères, afin qu'il soit un Souverain Sacrificateur miséricordieux et fidèle dans le service de Dieu, pour faire l’expiation des péchés du peuple ;
\VS{18}car du fait qu'il a souffert lui-même et qu’il a été tenté, il peut secourir ceux qui sont tentés.
\TextTitle{[Christ, le Fils, supérieur à Moïse, le serviteur]}
\Chap{3}
\VerseOne{}C'est pourquoi, mes frères saints, qui avez part à la vocation céleste, considérez attentivement Jésus-Christ l'Apôtre et le Souverain Sacrificateur de notre profession,
\VS{2}qui est fidèle à celui qui l'a établi, comme le fut Moïse dans toute sa maison.
\VS{3}Car Jésus-Christ a été jugé digne d'une gloire d'autant supérieure à celle de Moïse, que celui qui a construit une maison, a plus d’honneur que la maison même.
\VS{4}Car chaque maison est construite par quelqu'un, mais celui qui a construit toutes choses, c'est Dieu.
\VS{5}Et pour ce qui est de Moïse, il a été fidèle dans toute la maison de Dieu, comme serviteur, pour rendre témoignage de ce qui devait être annoncé.
\VS{6}Mais Christ l’est comme Fils sur sa maison ; et nous sommes sa maison\FTNT{L’Eglise véritable est la maison de Dieu. Voir Es. 66:1 ; 1 Co. 3:16 ; 1 Cor. 6:19 ; Ep. 2:21-22. Les bâtiments ne sont pas la maison de Dieu. Le premier bâtiment d’église avait été édifié par des fidèles sous le règne d’Alexandre Sévère en 222-235. L’Eglise véritable est composée de pierres vivantes qui ont pour fondement le Roc (Jésus), parce qu’elle est bâtie par Jésus-Christ lui-même et qu’elle est sa propriété ; les démons ne peuvent pas la détruire. L’Eglise véritable ne peut alors pas être confondue avec un bâtiment ou une maison physique.}, pourvu que nous retenions fermement jusqu’à la fin l'assurance et l'espérance dont nous nous glorifions.
\TextTitle{[la génération qui sortit d'Egypte a manqué le repos à cause de l'incrédulité]}
\VS{7}C'est pourquoi, comme dit le Saint-Esprit: Aujourd'hui, si vous entendez sa voix,
\VS{8}n'endurcissez point vos cœurs, comme lors de la révolte, au jour de la tentation dans le désert,
\VS{9}où vos pères me tentèrent pour m’éprouver, et ils virent mes œuvres pendant quarante ans\FTNT{Ps. 95:8-11.}.
\VS{10}C'est pourquoi je fus irrité contre cette génération, et je dis : Leur cœur s'égare toujours et ils n'ont pas connu mes voies.
\VS{11}Aussi, je jurai dans ma colère : Ils n’entreront pas dans mon repos.
\VS{12}Mes frères, prenez garde que quelqu'un de vous n’ait un cœur mauvais et incrédule, au point de se révolter contre le Dieu vivant.
\VS{13}Mais exhortez-vous les uns les autres chaque jour, pendant que ce jour nous éclaire, de peur que quelqu'un de vous ne s'endurcisse par la séduction du péché.
\VS{14}Car nous sommes devenus participants de Christ, pourvu que nous retenions fermement jusqu'à la fin l’assurance que nous avions au commencement,
\VS{15}pendant qu'il est dit : Aujourd'hui si vous entendez sa voix n'endurcissez pas vos cœurs comme lors de la révolte.
\VS{16}Car quelques-uns de ceux qui l’entendirent s’irritèrent ; mais non pas tous ceux qui sortirent d’Egypte sous la conduite de Moïse.
\VS{17}Et contre qui Dieu fut-il irrité pendant quarante ans, sinon contre ceux qui péchèrent, et dont les cadavres tombèrent dans le désert ?
\VS{18}Et à qui jura-t-il qu'ils n'entreraient point dans son repos, sinon à ceux qui furent rebelles ?
\VS{19}Aussi, voyons-nous qu'ils ne purent y entrer à cause de leur incrédulité.
\TextTitle{[Le repos]}
\Chap{4}
\VerseOne{}Craignons donc que quelqu'un d'entre vous, venant à négliger la promesse d'entrer dans son repos, ne s'en trouve privé.
\VS{2}Car il nous a été évangélisé, aussi bien qu’à eux ; mais cette parole de la prédication ne leur servit de rien, parce qu'elle n'était pas mêlée avec la foi dans ceux qui l’entendirent.
\VS{3}Pour nous qui avons cru, nous entrerons dans le repos, suivant ce qui a été dit : C'est pourquoi je jurai dans ma colère, ils n’entreront pas dans mon repos\FTNT{Hé. 3:11.} ! Il dit cela quoique ses œuvres aient été achevées depuis la création du monde.
\VS{4}Car il a parlé quelque part ainsi du septième jour : Et Dieu se reposa de toutes ses œuvres le septième jour\FTNT{Ge. 2:2.}.
\VS{5}Et encore dans ce passage : Ils n’entreront pas dans mon repos.
\VS{6}Or puisqu’il est encore réservé à quelques-uns d’y entrer, et que ceux à qui d’abord il a été évangélisé n’y sont pas entrés à cause de leur incrédulité,
\VS{7}Dieu détermine de nouveau un certain jour, par ce mot : Aujourd'hui, en disant par David si longtemps après, selon ce qui a été dit : Aujourd'hui, si vous entendez sa voix, n'endurcissez point vos cœurs\FTNT{Ps. 95:8-11.}.
\VS{8}Car si Josué les avait introduits dans le repos, Dieu ne parlerait pas d'un autre jour.
\TextTitle{[Entrer dans le repos de Dieu pour se reposer de ses œuvres]}
\VS{9}Il reste donc encore un repos de sabbat réservé au peuple de Dieu.
\VS{10}Car celui qui est entré dans le repos de Dieu se repose de ses œuvres, comme Dieu s’est reposé après avoir achevé les siennes.
\VS{11}Efforçons-nous donc d'entrer dans ce repos, de peur que quelqu'un de nous ne tombe dans une semblable rébellion.
\VS{12}Car la Parole de Dieu est vivante et efficace, plus pénétrante qu’une épée quelconque à deux tranchants, et atteignant jusqu’à la division de l'âme et de l'esprit, des jointures et des moelles, et elle juge les pensées et les intentions du cœur.
\VS{13}Et il n'y a aucune créature qui soit cachée devant lui, mais toutes choses sont nues et entièrement découvertes aux yeux de celui à qui nous devons rendre compte.
\VS{14}Ainsi, puisque nous avons un Grand Souverain Sacrificateur, Jésus, le Fils de Dieu, qui est entré dans les cieux, demeurons ferme dans notre profession.
\VS{15}Car nous n'avons pas un Souverain Sacrificateur qui ne puisse pas compatir à nos faiblesses, puisqu’il a été tenté comme nous en toutes choses, sans commettre le péché.
\VS{16}Approchons donc avec assurance du trône de la grâce, afin d’obtenir miséricorde et de trouver grâce, pour être secourus dans nos besoins.
\TextTitle{[Le service du souverain sacrificateur]}
\Chap{5}
\VerseOne{}En effet, tout souverain sacrificateur pris du milieu des hommes est établi pour les hommes dans les choses qui concernent le service de Dieu, afin de présenter des offrandes et des sacrifices pour les péchés.
\VS{2}Il peut avoir de l’indulgence pour les ignorants et les égarés, parce que la faiblesse est aussi son partage.
\VS{3}Et c’est à cause de cette faiblesse qu’il doit offrir des sacrifices pour ses propres péchés, comme pour ceux du peuple.
\VS{4}Or, personne ne s'attribue cet honneur si ce n’est celui qui est appelé de Dieu comme le fut Aaron.
\TextTitle{[Christ dans l'ordre de Melchisédek]}
\VS{5}De même aussi Christ ne s'est point attribué la gloire d’être Souverain Sacrificateur, mais il l’a reçue de celui qui lui a dit : C’est toi qui es mon Fils, je t'ai engendré aujourd'hui\FTNT{Ps. 2:7.} !
\VS{6}Comme il est dit encore ailleurs : Tu es Sacrificateur éternellement selon l'ordre de Melchisédek\FTNT{Ps. 110:4.}.
\VS{7}C’est lui qui, pendant les jours de sa chair, a offert avec de grands cris et avec larmes des prières et des supplications à celui qui pouvait le sauver de la mort, et il a été exaucé à cause de sa piété.
\VS{8}Il a appris, bien qu’il soit le Fils de Dieu, l'obéissance par les choses qu'il a souffertes.
\VS{9}Après avoir été élevé à la perfection, il est devenu l'auteur du salut éternel pour tous ceux qui lui obéissent,
\VS{10}Dieu l’ayant déclaré Souverain Sacrificateur selon l'ordre de Melchisédek.
\TextTitle{[Exhortation à croître du lait à la nourriture solide\FTNTT{jusqu'à Hé. 6:12}]}
\VS{11}Nous avons beaucoup de choses à dire là-dessus, et des choses difficiles à expliquer parce que vous êtes devenus lents à comprendre.
\VS{12}Vous, en effet, qui depuis longtemps devriez être des maîtres, vous avez encore besoin qu'on vous enseigne les premiers éléments de la parole de Dieu, et vous êtes dans un tel état, que vous avez encore besoin de lait et non d’une nourriture solide.
\VS{13}Or celui qui ne se nourrit que de lait, ne saurait comprendre la parole de justice, car il est encore un enfant\FTNT{Le mot enfant dans ce passage vient du grec «~nepios~» qui signifie «~ignorant~».}.
\VS{14}Mais la nourriture solide est pour les hommes faits, c'est-à-dire pour ceux dont le jugement est exercé par l’usage à discerner ce qui est bien et ce qui mal.
\TextTitle{[Tendre à la perfection]}
\Chap{6}
\VerseOne{}C'est pourquoi, laissant les premiers principes de la parole de Christ, tendons à la perfection, sans poser de nouveau le fondement de la repentance des œuvres mortes,
\VS{2}de la foi en Dieu, de la doctrine des baptêmes, et de l'imposition des mains, de la résurrection des morts, et du jugement éternel.
\VS{3}Et c'est ce que nous ferons, si Dieu le permet.
\VS{4}Car il est impossible que ceux qui ont été une fois illuminés, et qui ont goûté le don céleste, et qui ont eu part au Saint-Esprit,
\VS{5}qui ont goûté la bonne parole de Dieu, et les puissances du siècle à venir,
\VS{6}et qui sont tombés, soient de nouveau renouvelés par la repentance, puisqu’ils crucifient de nouveau le Fils de Dieu, et l'exposent à l'opprobre.
\VS{7}Lorsqu’une terre est abreuvée par la pluie qui tombe souvent sur elle, et qu’elle produit des herbes utiles à ceux pour qui elle est labourée, elle participe à la bénédiction de Dieu.
\VS{8}Mais si elle produit des épines et des chardons, elle est abandonnée, et près d’être maudite, et on finit par y mettre le feu.
\VS{9}Quoique nous parlions ainsi, mes bien-aimés, nous attendons de vous des choses meilleures et convenables au salut.
\VS{10}Car Dieu n'est pas injuste pour oublier vos bonnes œuvres, votre travail et la charité que vous avez montré pour son Nom, ayant rendu et rendant encore des services aux saints.
\VS{11}Nous souhaitons que chacun de vous montre le même zèle pour conserver jusqu'à la fin une pleine espérance,
\VS{12}en sorte que vous ne vous relâchiez point, mais que vous imitiez ceux qui par la foi et par la patience héritent ce qui leur a été promis.
\TextTitle{[Le souverain sacrificateur entré au-delà du voile comme précurseur]}
\VS{13}Lorsque Dieu fit la promesse à Abraham, ne pouvant jurer par un plus grand, il jura par lui-même,
\VS{14}en disant : Certainement, je te bénirai et je multiplierai ta postérité\FTNT{Ge. 22:16-17.}.
\VS{15}Et ainsi, Abraham ayant attendu patiemment, obtint ce qui lui avait été promis.
\VS{16}Or les hommes jurent par celui qui est plus grand qu'eux, et le serment qu'ils font pour confirmer leur parole met fin à tous leurs différends.
\VS{17}C'est pourquoi Dieu voulant faire mieux connaître aux héritiers de la promesse la fermeté immuable de sa résolution, intervint par un serment,
\VS{18}afin que par deux choses immuables, dans lesquelles il est impossible que Dieu mente, nous trouvions une ferme consolation, nous dont le seul refuge a été de saisir l'espérance qui nous était proposée.
\VS{19}Cette espérance, nous la possédons comme une ancre sûre et ferme de l'âme ; elle pénètre jusqu'au-delà du voile,
\VS{20}là où Jésus est entré comme notre précurseur, ayant été fait Souverain Sacrificateur éternellement, selon l'ordre de Melchisédek\FTNT{Voir Ge. 14.}.
\TextTitle{[Melchisédek, type de Christ\FTNTT{Ge. 14}]}
\Chap{7}
\VerseOne{}En effet, ce Melchisédek était Roi de Salem et Sacrificateur du Dieu Très-Haut\FTNT{Ge. 14:18.}. Il alla au-devant d'Abraham lorsqu'il revenait de la défaite des rois, il le bénit,
\VS{2}et Abraham lui donna pour sa part la dîme de tout\FTNT{Ge. 14:20. Pour en savoir plus sur la dîme, voir les commentaires en De. 14:22, No. 18:21 et Mal. 3:10.}. Il est d’abord Roi de justice, d’après la signification de son nom, ensuite Roi de Salem, c’est-à-dire Roi de paix.
\VS{3}Il est sans père, sans mère, sans généalogie, n'ayant ni commencement de jours ni fin de vie, mais il est rendu semblable au Fils de Dieu ; il demeure Sacrificateur à perpétuité.
\TextTitle{[La sacrificature de Melchisédek supérieure à celle d'Aaron]}
\VS{4}Considérez donc combien est grand celui à qui même Abraham le patriarche donna la dîme du butin.
\VS{5}Ceux des fils de Lévi qui exercent la sacrificature, ont d’après la loi, l’ordre de lever la dîme sur le peuple, c'est-à-dire sur leurs frères, qui cependant sont issus des reins d'Abraham.
\VS{6}Mais celui qui n’était pas de la même famille qu’eux reçut d’Abraham la dîme, et bénit celui à qui les promesses avaient été faites.
\VS{7}Or, sans contredit, celui qui bénit est plus grand que celui qui est béni.
\VS{8}Et ici, ce sont les hommes mortels qui prennent les dîmes ; mais là, c’est celui dont il est attesté qu’il est vivant.
\VS{9}De plus, Lévi qui perçoit les dîmes, les a payées, pour ainsi dire, par Abraham ;
\VS{10}car il était encore dans les reins de son père quand Melchisédek alla au-devant d’Abraham.
\TextTitle{[La sacrificature selon l'ordre d'Aaron n'a rien amené à la perfection]}
\VS{11}Si donc la perfection avait été possible par la sacrificature lévitique, (car c'est sur cette sacrificature que repose la loi donnée au peuple), était-il encore nécessaire qu’il paraisse un autre sacrificateur selon l'ordre de Melchisédek, et non selon l'ordre d'Aaron ?
\VS{12}Car, la sacrificature étant changée, il y a aussi nécessairement un changement de loi.
\VS{13}En effet, celui de qui ces choses sont dites appartient à une autre tribu, dont aucun membre n'a fait le service à l'autel.
\VS{14}Car il est évident que notre Seigneur est sorti de la tribu de Juda\FTNT{Mt. 1:2.}, tribu dont Moïse n'a rien dit au sujet de la sacrificature.
\VS{15}Cela devient plus manifeste encore quand un autre sacrificateur, à la ressemblance de Melchisédek, est suscité,
\VS{16}lequel n'a pas été établi Sacrificateur selon la loi d’une ordonnance charnelle, mais selon la puissance d’une vie impérissable.
\VS{17}Car Dieu lui rend ce témoignage : Tu es Sacrificateur éternellement, selon l'ordre de Melchisédek.
\VS{18}Ainsi, l’ordonnance précédente a été abolie à cause de sa faiblesse et de son inutilité,
\VS{19}car la loi n’a rien amené à la perfection ; mais ce qui a amené à la perfection, c’est ce qui a été introduit par-dessus, à savoir une meilleure espérance par laquelle nous nous approchons de Dieu.
\VS{20}Et cela n’a pas eu lieu sans serment ;
\VS{21}car les Lévites sont devenus sacrificateurs sans serment, mais Jésus l’est devenu avec serment par celui qui lui a dit : Le Seigneur l'a juré, et il ne s'en repentira pas\FTNT{Voir Ps. 110:4} : Tu es Sacrificateur éternellement selon l'ordre de Melchisédek.
\VS{22}Jésus est par cela même le garant d’une alliance plus excellente.
\TextTitle{[Les sacrificateurs selon l'ordre d'Aaron sont mortels, Christ demeure éternellement]}
\VS{23}A l’égard des sacrificateurs, il y en a eu plusieurs qui se sont succédés parce que la mort les empêchait d’être permanents.
\VS{24}Mais lui, parce qu'il demeure éternellement, possède un sacerdoce qui n’est pas transmissible.
\VS{25}C'est aussi pour cela qu’il peut parfaitement sauver pour toujours ceux qui s'approchent de Dieu par lui, étant toujours vivant pour intercéder\FTNT{Le Seigneur Jésus-Christ est le modèle parfait en ce qui concerne la prière d’intercession. Il se tient devant le Père pour nous. En tant qu’homme (1 Ti. 2:5) et Souverain Sacrificateur, il se tient entre le Père et l’homme pécheur, comme le faisaient les sacrificateurs sous la loi mosaïque. Voir Lu. 22:31-32 ; Ro. 8:34 ; 1 Jn. 2:1-2.} en leur faveur.
\VS{26}Il nous convenait, en effet, d'avoir un tel souverain sacrificateur, saint, innocent, sans tache, séparé des pécheurs, et élevé au-dessus des cieux,
\VS{27}qui n’a pas besoin, comme les autres souverains sacrificateurs, d'offrir tous les jours des sacrifices, premièrement pour ses péchés, et ensuite pour ceux du peuple, car il a fait cela une seule fois en s’offrant lui-même.
\VS{28}Car la loi établit souverains sacrificateurs des hommes faibles ; mais la parole du serment qui a été fait après la loi, établit le Fils, qui est parfait pour toujours.
\TextTitle{[Le culte de l'ancienne sacrificature est l'ombre des choses célestes, mais la réalité est en Christ]}
\Chap{8}
\VerseOne{}Le point capital de notre discours, c'est que nous avons un tel Souverain Sacrificateur qui est assis à la droite du trône de la majesté de Dieu dans les cieux,
\VS{2}comme ministre du sanctuaire, et du véritable tabernacle, dressé par le Seigneur et non par les hommes.
\VS{3}Car tout souverain sacrificateur est établi pour offrir des offrandes et des sacrifices ; c'est pourquoi il est nécessaire que celui-ci ait aussi quelque chose à offrir.
\VS{4}S’il était sur la terre, il ne serait même pas sacrificateur, puisqu’il y a encore des sacrificateurs qui offrent les offrandes selon la loi,
\VS{5}lesquels célèbrent un culte, image et ombre des choses célestes, comme Moïse en fut divinement averti lorsqu’il allait construire le tabernacle : Prends garde, lui dit-il, de faire toutes choses selon le modèle qui t'a été montré sur la montagne\FTNT{Ex. 25:40.}.
\TextTitle{[Christ, le Médiateur d'une alliance plus excellente]}
\VS{6}Mais maintenant, notre Souverain Sacrificateur a obtenu un ministère d'autant supérieur qu'il est le Médiateur d'une alliance plus excellente, qui a été établie sur de meilleures promesses.
\TextTitle{[Les prophètes ont annoncés la nouvelle alliance]}
\VS{7}En effet, si la première alliance avait été irréprochable, il n’aurait jamais été question de la remplacer par une seconde.
\VS{8}Car en censurant les Juifs Dieu leur dit : Voici, les jours viendront, dit le Seigneur, où je ferai avec la maison d'Israël et avec la maison de Juda une alliance nouvelle,
\VS{9}non comme l'alliance que je traitai avec leurs pères, le jour où je les saisis par la main pour les faire sortir du pays d'Egypte, car ils n'ont pas persévéré dans mon alliance ; c'est pourquoi je les ai rejetés, dit le Seigneur.
\VS{10}Mais voici l'alliance que je ferai avec la maison d'Israël en ces jours-là, dit le Seigneur : Je mettrai mes lois dans leur esprit, et je les écrirai dans leur cœur, je serai leur Dieu, et ils seront mon peuple.
\VS{11}Personne n'enseignera plus son prochain, ni personne son frère, en disant : Connais le Seigneur ! Car tous me connaîtront, depuis le plus petit jusqu'au plus grand d'entre eux ;
\VS{12}car je pardonnerai leurs injustices, et je ne me souviendrai plus de leurs péchés ni de leurs iniquités\FTNT{(2)	Jé. 31:31-34.}.
\VS{13}En disant une nouvelle alliance, il a déclaré ancienne la première ; or, ce qui est ancien, ce qui a vieilli, est près de disparaître.
\TextTitle{[Les ordonnances et le sanctuaire de la première alliance étaient des symboles]}
\Chap{9}
\VerseOne{}La première alliance avait donc des ordonnances relatives au culte divin et au sanctuaire terrestre\FTNT{Ex. 25:1-9.}.
\VS{2}En effet, un tabernacle fut construit. Dans la partie antérieure, appelée le lieu saint, étaient le chandelier, la table, et les pains de proposition\FTNT{Ex. 25:30.}.
\VS{3}Et au-delà du second voile\FTNT{Ex. 26:31-35.} était la partie du tabernacle appelée le Saint des saints,
\VS{4}renfermant l’encensoir d'or\FTNT{Encensoir ou autel d'or pour les parfums : Lé. 16:12.} pour les parfums et l'arche de l'alliance\FTNT{Ex. 25:10.} entièrement recouverte d'or. Il y avait dans l’arche un vase d'or\FTNT{Ex. 16:33.} contenant la manne, la verge d'Aaron\FTNT{No. 17:1-10.} qui avait fleuri, et les tables de l'alliance\FTNT{Les tables de l'alliance ou tables du témoignage : Ex. 34:29 ; De. 10:2-5.}.
\VS{5}Et au-dessus de l'arche étaient les chérubins de la gloire, couvrant de leur ombre sur le propitiatoire\FTNT{Propitiatoire ou couvercle de l’arche de l'alliance : Lé. 9:7 ; Lé. 16:15-17.}. Ce n’est pas le moment de parler en détail là-dessus.
\VS{6}Or ces choses étant ainsi disposées, les sacrificateurs qui font le service entrent en tout temps dans la première partie du tabernacle\FTNT{No. 28:3.}.
\VS{7}Mais seul le souverain sacrificateur entre dans la seconde une fois par an, non sans y porter du sang, qu’il offre pour lui-même et pour les péchés du peuple\FTNT{Lé. 16:34.}.
\VS{8}Le Saint-Esprit montrait par là que le chemin du Saint des saints n'était pas encore ouvert tant que le premier tabernacle subsistait,
\VS{9}lequel était une figure pour le temps présent, où l’on présente des offrandes et des sacrifices qui ne peuvent purifier la conscience de celui qui rend ce culte.
\VS{10}Ils étaient avec les aliments, les boissons et les diverses ablutions, des ordonnances charnelles imposées seulement jusqu’à une époque de réformation.
\TextTitle{[La réalité du sacrifice s'accomplit en Christ]}
\VS{11}Mais Christ est venu comme Souverain Sacrificateur des biens à venir ; il a traversé un tabernacle plus excellent et plus parfait, qui n'est pas un tabernacle construit de main d’homme, c'est-à-dire qui n’est pas de cette création.
\VS{12}Et il est entré une fois pour toutes dans le Saint des saints, non avec le sang des veaux ou des boucs, mais avec son propre sang, après avoir obtenu une rédemption éternelle.
\VS{13}Car si le sang des taureaux et des boucs, et la cendre de la génisse\FTNT{No. 19:1-12.}, répandue sur ceux qui sont souillés, sanctifient, et procurent la pureté de la chair,
\VS{14}combien plus le sang de Christ, qui par l'Esprit éternel s'est offert lui-même à Dieu sans tache, purifiera-t-il votre conscience des œuvres mortes, afin que vous serviez le Dieu vivant !
\VS{15}C'est pourquoi il est le médiateur d’une nouvelle alliance, afin que la mort étant intervenue pour la rançon des transgressions commises sous la première alliance, ceux qui ont été appelés reçoivent l’héritage éternel qui leur a été promis.
\TextTitle{[Les clauses du testament du Messie]}
\VS{16}Car là où il y a un testament, il est nécessaire que la mort du testateur soit constatée.
\VS{17}Parce que c'est par la mort du testateur qu'un testament est valable, puisqu’il n'a aucune force tant que le testateur vit.
\VS{18}Voilà pourquoi c’est avec du sang que même la première alliance fut inaugurée.
\VS{19}Car Moïse, après avoir prononcé devant tout le peuple tous les commandements de la loi, prit le sang des veaux et des boucs, avec de l'eau, de la laine écarlate, et de l'hysope, et il fit l’aspersion sur le livre et sur tout le peuple, en disant :
\VS{20}Ceci est le sang de l’alliance que Dieu vous a ordonné d'observer\FTNT{Ex. 24:3-8.}.
\VS{21}Puis il fit de même l’aspersion avec le sang sur le tabernacle et sur tous les ustensiles du culte\FTNT{Ex. 29:12 ; Ex. 29:36.}.
\VS{22}Et presque tout, selon la loi, est purifié avec le sang, et sans effusion de sang il n’y a pas de rémission des péchés.
\TextTitle{[Le sanctuaire céleste a été purifié par un sacrifice plus excellent\FTNTT{Lé. 16:33}]}
\VS{23}Il était donc nécessaire, puisque les images des choses qui sont dans les cieux devaient être purifiées de cette manière, que les choses célestes elles-mêmes le soient par des sacrifices plus excellents que ceux-là.
\VS{24}Car Christ n'est pas entré dans un sanctuaire fait de main d’homme, et qui n’était que la figure du véritable, mais il est entré dans le ciel même, afin de comparaître maintenant pour nous devant la face de Dieu.
\VS{25}Et ce n’est pas pour s’offrir lui-même plusieurs fois qu’il y est entré, comme le souverain sacrificateur entre chaque année dans le Saint des saints, mais pour offrir un autre sang que le sien ;
\VS{26}autrement il aurait fallu qu'il ait souffert plusieurs fois depuis la création du monde ; mais maintenant, à la fin des siècles, il a paru une seule fois pour abolir le péché par son sacrifice.
\VS{27}Et comme il est réservé aux hommes de mourir une seule fois\FTNT{Ce passage réfute la doctrine de la réincarnation.}, après quoi vient le jugement,
\VS{28}de même aussi Christ, qui s’est offert une seule fois pour ôter les péchés de plusieurs, apparaîtra sans péché une seconde fois à ceux qui l'attendent pour leur salut.
\TextTitle{[Le sacrifice unique de Christ est supérieur à tous les sacrifices de Moïse]}
\Chap{10}
\VerseOne{}En effet, la loi qui possède l'ombre des biens à venir, et non l’image exacte des choses, ne peut jamais par les mêmes sacrifices que l'on offre continuellement chaque année, amener les assistants à la perfection.
\VS{2}Autrement n’aurait-on pas cessé de les offrir, puisque ceux qui rendent ce culte, une fois purifiés, n’auraient plus eu conscience de leurs péchés ?
\VS{3}Mais le souvenir des péchés est renouvelé chaque année par ces sacrifices,
\VS{4}car il est impossible que le sang des taureaux et des boucs ôte les péchés.
\VS{5}C'est pourquoi Jésus-Christ en entrant dans le monde a dit : Tu n'as point voulu de sacrifice ni d'offrande, mais tu m'as formé un corps.
\VS{6}Tu n'as pas pris plaisir aux holocaustes ni aux sacrifices pour le péché\FTNT{Ps. 40:7-9.}.
\VS{7}Alors j'ai dit : Voici, je viens, ô Dieu ! Pour faire ta volonté, comme cela est écrit de moi dans le rouleau du livre.
\VS{8}Après avoir dit d’abord : Tu n'as point voulu de sacrifice, ni d'offrande, ni d'holocauste, ni de sacrifice pour le péché (choses qui sont offertes selon la loi), et tu n'y as point pris plaisir, il dit ensuite : Voici, je viens pour faire, ô Dieu, ta volonté !
\VS{9}Il abolit ainsi le premier sacrifice pour établir le second.
\VS{10}C’est en vertu de cette volonté que nous sommes sanctifiés par l’offrande du corps de Jésus-Christ, une fois pour toutes.
\VS{11}De plus, tout sacrificateur fait chaque jour le service et offre souvent les mêmes sacrifices qui ne peuvent jamais ôter les péchés ;
\VS{12}mais lui, après avoir offert un seul sacrifice pour les péchés, s’est assis pour toujours à la droite de Dieu ;
\VS{13}attendant désormais que ses ennemis deviennent son marchepied.
\VS{14}Car par une seule offrande il a rendu parfaits pour toujours ceux qui sont sanctifiés.
\VS{15}Et c'est aussi ce que le Saint-Esprit nous témoigne, car après avoir dit premièrement :
\VS{16}Voici l'alliance que je ferai avec eux après ces jours-là, dit le Seigneur\FTNT{Voir Jé. 31:31-34.} : Je mettrai mes lois dans leur cœur, je les écrirai dans leur esprit ; il ajoute :
\VS{17}Et je ne me souviendrai plus de leurs péchés, ni de leurs iniquités.
\VS{18}Or là où les péchés sont pardonnés, il n'y a plus d’offrande pour le péché.
\TextTitle{[Exhortation à s'approcher de Dieu avec foi]}
\VS{19}Ainsi donc, mes frères, nous avons la liberté d'entrer dans le Saint des saints au moyen du sang de Jésus,
\VS{20}qui est le chemin nouveau et vivant qu'il nous a frayé au travers du voile, c’est-à-dire de sa chair.
\VS{21}Et ayant un Souverain Sacrificateur établi sur la maison de Dieu,
\VS{22}approchons-nous de lui avec un cœur sincère et une foi inébranlable, ayant les cœurs purifiés d’une mauvaise conscience, et le corps lavé d’une eau pure.
\VS{23}Retenons sans fléchir la profession de notre espérance, car celui qui nous a fait la promesse est fidèle.
\VS{24}Veillons les uns sur les autres pour nous exciter à la charité et aux bonnes œuvres.
\VS{25}N’abandonnons pas notre assemblée\FTNT{Le Seigneur Jésus se définit comme l’unique chemin qui conduit au Saint des saints (Jn. 14:6).}, comme c’est la coutume de quelques-uns, mais exhortons-nous réciproquement, et cela d'autant plus que vous voyez s’approcher le jour.
\TextTitle{[Avertissement à ne pas mépriser le sacrifice de Christ]}
\VS{26}Car si nous péchons volontairement après avoir reçu la connaissance de la vérité, il ne reste plus de sacrifice pour les péchés,
\VS{27}mais une attente terrible du jugement et l'ardeur d'un feu qui dévorera les adversaires.
\VS{28}Si quelqu'un avait méprisé la loi de Moïse, il mourait sans miséricorde, sur la déposition de deux ou de trois témoins\FTNT{Assemblée : Du grec «~episunagoge~» qui veut dire «~être assemblés en un lieu, assemblée religieuse des chrétiens~». Il est question de ne pas abandonner la communion fraternelle et non une église locale. En effet, il est du devoir du chrétien de se séparer des faux frères de peur d’être entraîné dans leur égarement (Mt. 18:15-17 ; 1 Co. 5:11 ; 1 Co. 15:33).}.
\VS{29}De quel plus grand supplice pensez-vous que sera jugé digne celui qui foulera aux pieds le Fils de Dieu et qui aura tenu pour profane le sang de l'alliance, par lequel il a été sanctifié, et qui aura outragé l'Esprit de la grâce ?
\VS{30}Car nous connaissons celui qui a dit : A moi la vengeance, à moi la rétribution !\FTNT{De. 17:6.} Et encore : Le Seigneur jugera son peuple.
\VS{31}C'est une chose terrible que de tomber entre les mains du Dieu vivant.
\VS{32}Souvenez-vous de ces premiers jours, où, après avoir été illuminés, vous avez soutenu un grand combat au milieu des souffrances,
\VS{33}d'une part, exposés comme en spectacle aux opprobres et aux afflictions, et de l'autre, vous associant à ceux dont la position était la même.
\VS{34}En effet, vous avez eu de la compassion pour mes liens et vous avez accueilli avec joie l'enlèvement de vos biens ; sachant en vous-mêmes que vous avez dans les cieux des biens plus excellents et permanents.
\VS{35}N’abandonnez donc pas votre assurance, à laquelle est attachée une grande rémunération.
\VS{36}Car vous avez besoin de persévérance, afin qu'après avoir fait la volonté de Dieu, vous receviez ce qui vous est promis.
\TextTitle{[La marche par la foi: quelques hommes de foi]}
\VS{37}Encore un peu, un peu de temps, celui qui doit venir viendra, et il ne tardera pas.
\VS{38}Et le juste vivra par la foi ; mais s’il se retire, mon âme ne prend pas plaisir en lui\FTNT{Ha. 2:4.}.
\VS{39}Nous, nous ne sommes pas de ceux qui se retirent de leur Maître pour se perdre, mais de ceux qui persévèrent dans la foi pour le salut de leur âme.
\Chap{11}
\VerseOne{}Or la foi rend présente les choses qu'on espère, et elle est une démonstration de celles qu'on ne voit point.
\VS{2}Pour l’avoir possédée, les anciens ont obtenu un bon témoignage.
\VS{3}C’est par la foi que nous comprenons que le monde a été formé par la parole de Dieu, en sorte que ce qu’on voit n’a pas été fait de choses visibles.
\TextTitle{[Abel]}
\VS{4}C’est par la foi qu’Abel\FTNT{Ge. 4:3-5.} offrit à Dieu un sacrifice plus excellent que celui de Caïn ; c’est par elle qu’il a reçu le témoignage d’être juste, Dieu rendant témoignage à ses offrandes ; et c’est par elle qu’il parle encore quoique mort.
\TextTitle{[Hénoc]}
\VS{5}C’est par la foi qu’Enoch\FTNT{Ge. 5:22-24.} fut enlevé pour qu’il ne vît pas la mort, et il ne parut plus parce que Dieu l’avait enlevé ; car avant son enlèvement, il avait reçu le témoignage qu’il était agréable à Dieu.
\VS{6}Or sans la foi, il est impossible de lui être agréable ; car il faut que celui qui s’approche de Dieu croie que Dieu existe et qu'il est le rémunérateur de ceux qui le cherchent.
\TextTitle{[Noé]}
\VS{7}C’est par la foi que Noé\FTNT{Ge. 6:14-22.}, divinement averti des choses qu’on ne voyait pas encore, fut rempli de crainte, et construisit une arche pour le salut de sa famille ; par cette arche, il condamna le monde et devint héritier de la justice qui s’obtient par la foi.
\TextTitle{[Abraham et Sara]}
\VS{8}C’est par la foi qu’Abraham\FTNT{Ge 12:1-4.}, lors de sa vocation, obéit et partit pour une terre qu'il devait recevoir en héritage, et il partit sans savoir où il allait.
\VS{9}C’est par la foi qu’il demeura comme étranger dans la terre qui lui avait été promise, habitant sous des tentes, aussi bien qu’Isaac et Jacob qui étaient héritiers avec lui de la même promesse.
\VS{10}Car il attendait la cité qui a des fondements, celle dont Dieu est l'architecte et le constructeur.
\VS{11}C’est par la foi que Sara\FTNT{Ge. 21:1-2.} elle-même, malgré son âge avancé, fut rendue capable d’avoir une postérité, parce qu’elle crut à la fidélité de celui qui avait fait la promesse.
\VS{12}C'est pourquoi d'un seul, déjà usé de corps, naquit une postérité nombreuse comme les étoiles du ciel, comme le sable qui est sur le bord de mer et qu’on ne peut pas compter\FTNT{Ge. 22:17.}.
\VS{13}Tous ceux-là sont morts dans la foi, sans avoir obtenu les choses promises, mais ils les ont vues de loin, crues, et saluées, reconnaissant qu'ils étaient étrangers et voyageurs sur la terre\FTNT{1 Pi. 2:11.}.
\VS{14}Ceux qui parlent ainsi montrent clairement qu'ils cherchent encore une patrie.
\VS{15}S’ils avaient eu en vue celle d’où ils étaient sortis, ils auraient eu le temps d’y retourner.
\VS{16}Mais ils en désirent une meilleure, c'est-à-dire une céleste. C’est pourquoi Dieu n’a pas honte d'être appelé leur Dieu, parce qu'il leur a préparé une cité\FTNT{Jn. 14:2 ; Ap. 21:2.}.
\VS{17}C’est par la foi qu’Abraham offrit Isaac lorsqu’il fut mis à l’épreuve, et qu’il offrit son fils unique\FTNT{Ge. 22:1.}, lui qui avait reçu les promesses,
\VS{18}et à qui il avait été dit : C’est en Isaac que ta postérité sera appelée de ton nom\FTNT{Ge. 21:12.}.
\VS{19}Il estimait que Dieu est puissant même pour ressusciter Isaac d'entre les morts ; c'est pourquoi aussi il retrouva son fils, ce qui est une préfiguration.
\TextTitle{[Isaac]}
\VS{20}C’est par la foi qu’Isaac bénit Jacob et Esaü, en vue des choses à venir\FTNT{Ge. 27:26-40.}.
\TextTitle{[Jacob]}
\VS{21}C’est par la foi que Jacob mourant bénit chacun des fils de Joseph\FTNT{Ge. 48:1-22.}, et adora Dieu, appuyé sur l’extrémité de son bâton\FTNT{Ge. 47:31.}.
\TextTitle{[Joseph]}
\VS{22}C’est par la foi que Joseph mourant fit mention de la sortie des enfants d'Israël, et qu’il donna des ordres au sujet de ses os\FTNT{Ge. 50:24-25.}.
\TextTitle{[Les parents de Moïse]}
\VS{23}C’est par la foi que Moïse\FTNT{Ex. 2:1-3.}, à sa naissance, fut caché pendant trois mois par ses parents, parce qu’ils virent que l’enfant était beau, et ils ne craignirent pas l’ordre du roi.
\TextTitle{[Moïse]}
\VS{24}C’est par la foi que Moïse, devenu grand, refusa d'être appelé fils de la fille de Pharaon ;
\VS{25}choisissant d’être affligé avec le peuple de Dieu, plutôt que de jouir pour un peu de temps des délices du péché.
\VS{26}Il regarda l'opprobre de Christ comme un trésor plus grand que les richesses de l'Egypte, car il avait les yeux fixés sur la rémunération.
\VS{27}C’est par la foi qu’il quitta l'Egypte, sans être effrayé de la colère du roi ; car il demeura ferme, comme voyant celui qui est invisible.
\VS{28}C’est par la foi qu’il célébra la Pâque et fit l'aspersion du sang, afin que l’exterminateur qui tuait les premiers-nés, ne touche pas aux premiers-nés des Israélites\FTNT{Ex. 12:1-51.}.
\VS{29}C’est par la foi qu’ils traversèrent la Mer Rouge, comme un lieu sec, tandis que les Egyptiens qui tentèrent de passer furent engloutis\FTNT{Ex. 14:13-31.}.
\TextTitle{[Josué et Jéricho]}
\VS{30}C’est par la foi que les murs de Jéricho tombèrent, après qu'on en eut fait le tour pendant sept jours\FTNT{Jos. 6:1-20.}.
\TextTitle{[Rahab]}
\VS{31}C’est par la foi que Rahab, la prostituée, ne périt pas avec les incrédules, parce qu’elle avait reçu les espions et les avait renvoyés en paix\FTNT{Jos. 2:1-21 ; Jos. 6:23.}.
\TextTitle{[De Nombreux hommes et femmes de la foi]}
\VS{32}Et que dirai-je encore ? Car le temps me manquerait si je voulais parler de Gédéon\FTNT{Jg. 6:11.}, de Barak\FTNT{Jg. 4:6.}, de Samson\FTNT{Jg. 13:24.}, de Jephté\FTNT{Jg. 11:1.}, de David\FTNT{1 S. 16 et 17.}, de Samuel\FTNT{1 S. et 2 S.}, et des prophètes,
\VS{33}qui par la foi vainquirent des royaumes, exercèrent la justice, obtinrent des promesses, fermèrent la gueule des lions,
\VS{34}éteignirent la puissance du feu, échappèrent au tranchant de l’épée, guérirent de leurs maladies, furent vaillants à la guerre, mirent en fuite des armées étrangères.
\VS{35}Des femmes recouvrèrent leurs morts par le moyen de la résurrection ; d'autres furent livrés aux tourments et n’acceptèrent point d'être délivrés afin d'obtenir une meilleure résurrection.
\VS{36}Et d'autres subirent les moqueries et le fouet, les chaînes et la prison ;
\VS{37}ils furent lapidés, sciés, subirent de rudes épreuves ; ils moururent tués par l'épée ; ils allèrent çà et là, vêtus de peaux de brebis et de chèvres, réduits à la misère, affligés, maltraités,
\VS{38}eux dont le monde n'était pas digne ; errant dans les déserts, et dans les montagnes, dans les cavernes, et dans les antres de la terre.
\VS{39}Et tous ceux-là, ayant obtenu un bon témoignage par leur foi, n'ont pourtant pas obtenu ce qui leur était promis,
\VS{40}Dieu ayant en vue quelque chose de meilleur pour nous, afin qu'ils ne parviennent pas à la perfection sans nous.
\TextTitle{[Fixer les regards sur Jésus]}
\Chap{12}
\VerseOne{}Nous donc aussi, puisque nous sommes environnés d'une si grande nuée de témoins\FTNT{Témoin : du grec «~martus~», terme qui dans un sens légal et historique signifie «~celui qui est spectateur d’une chose~». Dans un sens éthique, il est question de «~ceux qui ont prouvé la force et l’authenticité de leur foi en Christ en supportant une mort violente.~» «~Martus~» a donné le mot «~martyr~» en français.}, rejetons tout fardeau, et le péché qui nous enveloppe si aisément, et poursuivons constamment la course qui nous est proposée,
\VS{2}ayant les yeux sur Jésus, le chef et le consommateur de la foi. En échange de la joie qui lui était réservée, il a souffert la croix, ayant méprisé la honte, et s'est assis à la droite du trône de Dieu.
\VS{3}C'est pourquoi, considérez soigneusement celui qui a supporté contre sa personne une telle opposition de la part des pécheurs, afin que vous ne succombiez point en perdant courage.
\TextTitle{[La correction du Père]}
\VS{4}Vous n'avez pas encore résisté jusqu'à répandre votre sang en combattant contre le péché ;
\VS{5}et vous avez oublié l'exhortation qui vous est adressée comme à des fils : Mon fils ne méprise pas le châtiment du Seigneur, et ne perds point courage lorsqu’il te reprend ;
\VS{6}car le Seigneur châtie celui qu'il aime, et il frappe de la verge tous ceux qu’il reconnaît pour ses fils\FTNT{Pr. 3:11-12.}.
\VS{7}Supportez le châtiment, c’est comme des fils que Dieu vous traite, car quel est le fils qu’un père ne châtie pas ?
\VS{8}Mais si vous êtes exempts du châtiment auquel tous ont part, vous êtes donc des bâtards, et non des enfants légitimes.
\VS{9}Et puisque nos pères selon la chair nous ont châtiés et que malgré cela nous les avons respectés, ne devons-nous pas à plus forte raison nous soumettre au Père des esprits pour avoir la vie ?
\VS{10}Nos pères nous châtiaient pour un peu de jours, comme ils le trouvaient bon ; mais Dieu nous châtie pour notre bien, afin de nous rendre participants de sa sainteté.
\VS{11}Or tout châtiment semble d’abord un sujet de tristesse et non de joie ; mais il produit plus tard pour ceux qui ont été ainsi exercés un fruit paisible de justice.
\VS{12}Fortifiez donc vos mains languissantes et vos genoux affaiblis,
\VS{13}et suivez avec vos pieds des chemins droits, afin que ce qui est boiteux ne dévie pas, mais plutôt se raffermisse.
\VS{14}Recherchez la paix avec tous, et la sanctification, sans laquelle nul ne verra le Seigneur.
\TextTitle{[Que nul ne se prive de la grâce de Dieu]}
\VS{15}Veillez à ce que personne ne se prive de la grâce de Dieu ; à ce qu’aucune racine d'amertume, poussant des rejetons, ne vous trouble, et que plusieurs n’en soient infectés ;
\VS{16}à ce qu’il n’y ait ni débauché, ni profane comme Esaü, qui pour un mets vendit son droit d’aînesse\FTNT{Ge. 25:33}.
\VS{17}Car vous savez que plus tard, désirant obtenir la bénédiction, il fut rejeté, car il ne trouva point de lieu à la repentance, quoiqu'il l’ait demandée avec larmes.
\TextTitle{[L'Eglise véritable s'est approchée de la montagne de Sion (la Jérusalem céleste) et non du mont Sinaï (la loi)]}
\VS{18}Vous ne vous êtes pas approchés d’une montagne qu’on pouvait toucher avec la main\FTNT{Ex. 19 : 12.}, ni du feu brûlant, ni de la nuée épaisse, ni de l'obscurité, ni de la tempête,
\VS{19}ni du retentissement de la trompette, ni du bruit des paroles, tel que ceux qui l'entendirent prièrent que la parole ne leur soit plus adressée\FTNT{Ex. 20:18-26.} ;
\VS{20}car ils ne pouvaient pas supporter ce qui était ordonné : Même si une bête touche la montagne, elle sera lapidée ou percée d'un dard\FTNT{Ex. 19:13.}.
\VS{21}Et ce spectacle était si terrible que Moïse dit : Je suis épouvanté et tout tremblant !
\VS{22}Mais vous vous êtes approchés de la montagne de Sion, de la Cité du Dieu vivant, la Jérusalem céleste, des myriades d'anges,
\VS{23}et de l'assemblée et de l'Eglise des premiers-nés inscrits dans les cieux, du Dieu qui est le juge de tous, et des esprits des justes parvenus à la perfection,
\VS{24}de Jésus, qui est le médiateur de la nouvelle alliance, et du sang de l'aspersion, qui parle mieux que celui d'Abel.
\TextTitle{[Exhortation à la crainte de Dieu]}
\VS{25}Prenez garde de ne pas mépriser celui qui vous parle ; car si ceux qui méprisèrent celui qui leur parlait sur la terre, n’ont pas échappé combien moins échapperons-nous, si nous nous détournons de celui qui nous parle des cieux,
\VS{26}lui, dont la voix ébranla alors la terre, et qui maintenant a fait cette promesse, disant : Une fois encore j'ébranlerai non seulement la terre, mais aussi le ciel\FTNT{Ag. 2:6.}.
\VS{27}Or ce mot : Une fois encore, marque le changement des choses ébranlées, comme étant faites pour un temps, afin que celles qui sont inébranlables demeurent.
\VS{28}C'est pourquoi, saisissant le Royaume inébranlable, montrons notre reconnaissance en rendant à Dieu un culte qui lui soit agréable, avec piété et avec crainte,
\VS{29}car notre Dieu est aussi un feu dévorant\FTNT{De. 4:24.}.
\TextTitle{[Conseils de Paul qui rappelle que le Messie reste le même]}
\Chap{13}
\VerseOne{}Que la charité fraternelle demeure dans vos cœurs.
\VS{2}N'oubliez pas l'hospitalité, car c’est par elle que quelques-uns ont logé des anges sans le savoir.
\VS{3}Souvenez-vous des prisonniers, comme si vous étiez emprisonnés avec eux ; et de ceux qui sont maltraités, comme étant aussi vous-mêmes du même corps.
\VS{4}Que le mariage soit honoré de tous, et le lit conjugal sans souillure, car Dieu jugera les débauchés et les adultères.
\VS{5}Ne vous livrez pas à l’amour de l’argent, contentez-vous de ce que vous avez, car Dieu lui-même a dit : Je ne te délaisserai point, et je ne t'abandonnerai point\FTNT{De. 31:6.},
\VS{6}de sorte que nous pouvons dire avec assurance : Le Seigneur est mon aide, je ne craindrai point ce que l'homme pourrait me faire\FTNT{Ps. 118:6.}.
\VS{7}Souvenez-vous de vos conducteurs qui vous ont annoncé la parole de Dieu ; considérez quelle a été la fin de leur vie, et imitez leur foi.
\VS{8}Jésus-Christ est le même hier, aujourd'hui, et éternellement.
\VS{9}Ne vous laissez pas entrainer çà et là par des doctrines diverses et étrangères ; car il est bon que le cœur soit affermi par la grâce, et non par des aliments, qui n'ont servi à rien à ceux qui s'y sont attachés.
\TextTitle{[Porter ses regards sur la cité céleste]}
\VS{10}Nous avons un autel dont ceux qui font le service au tabernacle n'ont pas le droit de manger.
\VS{11}Car les corps des animaux, dont le sang est porté dans le sanctuaire par le souverain sacrificateur pour le péché, sont brûlés hors du camp.
\VS{12}C'est pour cela que Jésus aussi, afin de sanctifier le peuple par son propre sang, a souffert hors de la porte\FTNT{Ex. 29:14. Jésus a souffert hors de Jérusalem (Jn. 19:17-18).}.
\VS{13}Sortons donc pour aller à lui hors du camp\FTNT{Le mot «~camp~» dans ce passage vient du grec «~parambole~», terme faisant référence au judaïsme antique dans lequel s’étaient embourbés les chrétiens d'origine hébraïque. Aujourd’hui, il représente plutôt le christianisme paganisé, essentiellement basé sur la loi de Moïse et constituant une prison qui empêche certains enfants de Dieu de vivre pleinement leur liberté en Christ.}, en portant son opprobre.
\VS{14}Car nous n'avons point ici-bas de cité permanente, mais nous recherchons celle qui est à venir.
\TextTitle{[Sacrifice de louange et le sacrifice du serviteur de Dieu]}
\VS{15}Offrons donc par lui sans cesse à Dieu un sacrifice de louange, c'est-à-dire, le fruit des lèvres, en confessant son Nom.
\VS{16}Et n'oubliez pas la bienveillance et de faire part de vos biens, car Dieu prend plaisir à de tels sacrifices.
\TextTitle{[L'obéissance aux conducteurs]}
\VS{17}Obéissez\FTNT{Le terme «~obéissez~», en grec «~peitho~», veut dire «~se laisser persuader par des mots~». Il signifie aussi «~donner avec persuasion l’envie à quelqu’un de faire quelque chose en le rassurant~». Par conséquent, les conducteurs doivent comprendre que la soumission et l’obéissance des chrétiens n’a rien à voir avec la dictature et l’autoritarisme. Ils doivent les rassurer et les convaincre - car tout ce qui n’est pas fait avec foi est péché (Ro. 14:23) - et ne pas tyranniser leurs frères en les obligeant à leur obéir (Mt. 20:25 ; 1 Pi. 5:2-3).} à vos conducteurs, et soyez-leur soumis, car ils veillent pour vos âmes, dont ils devront rendre compte ; qu’il en soit ainsi afin qu’ils le fassent avec joie, et non en gémissant ce qui ne vous serait d’aucun avantage.
\VS{18}Priez pour nous ; car nous croyons avoir une bonne conscience, voulant en toutes choses nous conduire honnêtement.
\VS{19}C’est avec instance que je vous demande de le faire, afin que je vous sois rendu plutôt.
\VS{20}Que le Dieu de paix, qui a ramené d'entre les morts le grand Pasteur des brebis, par le sang de l'alliance éternelle, notre Seigneur Jésus-Christ,
\VS{21}Vous rende capables de toute bonne œuvre pour faire sa volonté ; qu’il fasse en vous ce qui lui est agréable par Jésus-Christ ; auquel soit la gloire aux siècles des siècles, amen !
\VS{22}Je vous prie, mes frères, de supporter ces paroles d'exhortation, car je vous ai écrit en peu de mots.
\VS{23}Sachez que notre frère Timothée a été relâché ; s'il vient bientôt, je vous verrai avec lui.
\VS{24}Saluez tous vos conducteurs, et tous les saints ; ceux d'Italie vous saluent.
\VS{25}Que la grâce soit avec vous tous, amen !
\PPE{}
\end{multicols}
