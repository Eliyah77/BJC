\ShortTitle{Marc}\BookTitle{Marc}\BFont
\noindent\hrulefill
{\footnotesize
\textit{
\bigskip
{\centering{}
\\Auteur : Marc
\\(Gr. : Markos)
\\Signification : Une défense, grand marteau
\\Thème : Jésus, le Serviteur
\\Date de rédaction : Env. 68 ap. J.-C.\\}
}
%\bigskip
\textit{
\\Originaire de Jérusalem, Marc, également appelé Jean, fut l'auteur de l'évangile du même nom. Cousin de Barnabas et collaborateur de Paul, ce dernier l'éconduit lors d'un voyage, car Marc l'avait abandonné lors d'une précédente mission ; ce fut d'ailleurs la cause de la séparation de Barnabas et Paul. Par la suite, il renoua le contact avec Paul et devint un de ses fidèles compagnons de ministère. Lié à l'apôtre Pierre tel un fils, ce fut probablement sous son autorité qu'il écrivit. En effet, l'évangile de Marc expose le témoignage de Pierre sur Christ.
%\bigskip
\\Adressé aux Gentils, cet évangile contient peu de références à la Première Alliance. On y découvre Jésus, l'inlassable serviteur de Dieu et des hommes. Marc y présente la richesse de ses bonnes œuvres, son incomparable dévouement, révélant les sentiments intimes du Maître, tandis que le récit des miracles met en exergue toute la puissance du Christ.\bigskip
}
}
\par\nobreak\noindent\hrulefill
\begin{multicols}{2}
\Chap{1}
\TextTitle{Ministère de Jean-Baptiste\FTNTT{Mt. 3:1-12 ; Lu. 3:1-20 ; Jn. 1:6-8,15-37}}
\VerseOne{}Le commencement de l'Evangile de JESUS-CHRIST, Fils de Dieu ;
\VS{2}selon qu'il est écrit dans les prophètes : Voici, j'envoie mon messager devant ta face, lequel préparera ta voie devant toi.
\VS{3}La voix de celui qui crie dans le désert est : Préparez le chemin du Seigneur, aplanissez ses sentiers\FTNT{Es. 40:3 ; Mal. 3:1.}.
\VS{4}Jean baptisait dans le désert et prêchait le baptême de repentance, pour obtenir la rémission des péchés.
\VS{5}Et tout le pays de Judée, et les habitants de Jérusalem allaient vers lui, et confessant leurs péchés, ils se faisaient tous baptiser par lui dans le fleuve du Jourdain.
\VS{6}Or Jean était vêtu de poils de chameau, il avait une ceinture de cuir autour de ses reins, et mangeait des sauterelles et du miel sauvage.
\VS{7}Et il prêchait, en disant : Il vient après moi, celui qui est plus puissant que moi, et je ne suis pas digne de délier en me baissant la courroie de ses souliers.
\VS{8}Pour moi, je vous ai baptisés d'eau ; mais lui, il vous baptisera du Saint-Esprit.
\TextTitle{Baptême de Jésus-Christ\FTNTT{Mt. 3:13-17 ; Lu. 3:21-22 ; Jn. 1:31-34}}
\VS{9}Or il arriva en ce temps-là que Jésus vint de Nazareth, ville de Galilée, et il fut baptisé par Jean dans le Jourdain.
\VS{10}Au moment où il sortait de l'eau, Jean vit les cieux s'ouvrir, et le Saint-Esprit descendre sur lui comme une colombe.
\VS{11}Et une voix fit entendre des cieux ces paroles : Tu es mon Fils bien-aimé, en qui j'ai mis toute mon affection.
\TextTitle{La tentation\FTNTT{Mt. 4:1-11 ; Lu. 4:1-13}}
\VS{12}Aussitôt l'Esprit le poussa à se rendre dans un désert,
\VS{13}où il passa quarante jours, tenté par Satan. Il était avec les bêtes sauvages et les anges le servaient.
\TextTitle{Jésus en Galilée\FTNTT{Mt. 4:12-17 ; Lu. 4:14-15}}
\VS{14}Or après que Jean eut été mis en prison, Jésus alla dans la Galilée, prêchant l'Evangile du Royaume de Dieu.
\VS{15}Il disait : Le temps est accompli et le Royaume de Dieu est proche. Repentez-vous, et croyez à l'Evangile.
\TextTitle{Appel de Simon (Pierre), André, Jacques et Jean\FTNTT{Lu. 5:1-11 ; Jn. 1:35-51}}
\VS{16}Comme il marchait près de la mer de Galilée, il vit Simon et André son frère qui jetaient leurs filets dans la mer, car ils étaient pêcheurs.
\VS{17}Et Jésus leur dit : Suivez-moi et je vous ferai pêcheurs d'hommes.
\VS{18}Et ayant aussitôt quitté leurs filets, ils le suivirent.
\VS{19}Etant allé un peu plus loin, il vit Jacques, fils de Zébédée et Jean son frère, qui raccommodaient leurs filets dans la barque.
\VS{20}Et aussitôt, il les appela ; et eux, laissant leur père Zébédée dans la barque avec les ouvriers, le suivirent.
\TextTitle{Jésus chasse un démon dans la synagogue\FTNTT{Lu. 4:31-37}}
\VS{21}Puis ils entrèrent dans Capernaüm. Et le jour du sabbat, Jésus entra d'abord dans la synagogue et il enseigna.
\VS{22}Ils étaient étonnés de sa doctrine ; car il les enseignait comme ayant autorité et non pas comme les scribes.
\VS{23}Or il se trouva dans leur synagogue un homme qui avait un esprit impur, et qui s'écria,
\VS{24}en disant : Ah ! Qu'y a-t-il entre toi et nous, Jésus de Nazareth ? Es-tu venu pour nous détruire ? Je sais qui tu es : Tu es le Saint de Dieu.
\VS{25}Mais Jésus le menaça, disant : Tais-toi, et sors de cet homme.
\VS{26}Alors l'esprit impur sortit de cet homme, en l'agitant avec violence et en poussant un grand cri.
\VS{27}Et tous en furent étonnés, de sorte qu'ils se demandaient les uns aux autres, et disaient : Qu'est-ce que ceci ? Quelle est cette nouvelle doctrine ? Il commande avec autorité même aux esprits impurs, et ils lui obéissent.
\VS{28}Et sa renommée se répandit aussitôt dans tout le pays des environs de la Galilée.
\TextTitle{Jésus guérit la belle-mère de Pierre\FTNTT{Mt. 8:14-15 ; Lu. 4:38-39}}
\VS{29}Et aussitôt après, étant sortis de la synagogue, ils se rendirent avec Jacques et Jean à la maison de Simon et d'André.
\VS{30}Or la belle-mère de Simon était couchée, ayant la fièvre ; et au même moment, on parla d'elle à Jésus.
\VS{31}Et s'étant approché, il la fit lever en la prenant par la main ; et à l'instant la fièvre la quitta ; et elle les servit.
\TextTitle{Jésus guérit les malades et chasse des démons ; prédications en Galilée\FTNTT{Mt. 8:16-17 ; Lu. 4:40-44}}
\VS{32}Or le soir étant venu, comme le soleil se couchait, on lui amena tous les malades et les démoniaques.
\VS{33}Et toute la ville était assemblée devant la porte.
\VS{34}Et il guérit beaucoup de malades qui avaient différentes maladies, chassa beaucoup de démons hors des possédés, et il ne permit pas aux démons de dire qu'ils le connaissaient.
\VS{35}Puis au matin, pendant qu'il faisait encore très sombre, il se leva, et sortit pour aller dans un lieu désert, où il pria.
\VS{36}Et Simon et ceux qui étaient avec lui se mirent à sa recherche,
\VS{37}et quand ils l'eurent trouvé, ils lui dirent : Tous te cherchent.
\VS{38}Et il leur dit : Allons aux villages voisins, afin que j'y prêche aussi ; car c'est pour cela que je suis venu.
\VS{39}Il prêchait donc dans leurs synagogues, par toute la Galilée, et chassait les démons hors des possédés.
\TextTitle{Jésus guérit un lépreux\FTNTT{Mt. 8:2-4 ; Lu. 5:12-14}}
\VS{40}Et un lépreux vint à lui, le priant et se mettant à genoux devant lui, et lui dit : Si tu veux, tu peux me rendre pur.
\VS{41}Jésus, ému de compassion, étendit sa main et le toucha, en lui disant : Je le veux, sois pur.
\VS{42}Aussitôt la lèpre quitta cet homme et il fut purifié.
\VS{43}Puis l'ayant menacé, il le renvoya incessamment,
\VS{44}et lui dit : Garde-toi de ne rien dire à personne ; mais va, et montre-toi au sacrificateur, et présente pour ta purification les choses que Moïse a commandées, pour leur servir de témoignage\FTNT{Loi sur la purification de la lèpre : Lé. 14:1-32. Avant sa mort et sa résurrection, Jésus-Christ observait la loi de Moïse (Mt. 23:1-3).}.
\VS{45}Mais cet homme, s'en étant allé, commença à publier ouvertement plusieurs choses et à divulguer ce qui s'était passé, de sorte que Jésus ne pouvait plus entrer publiquement dans la ville ; mais il se tenait dehors, dans des lieux déserts, et l'on venait à lui de toutes parts.
\Chap{2}
\TextTitle{Jésus guérit un paralytique\FTNTT{Mt. 9:2-8 ; Lu. 5:18-26}}
\VerseOne{}Quelques jours après, il revint à Capernaüm. Et on apprit qu'il était à la maison,
\VS{2}et aussitôt il s'assembla un si grand nombre de personnes, tellement que l'espace même devant la porte ne pouvait plus les contenir. Il leur annonçait la parole.
\VS{3}Et quelques-uns vinrent à lui, amenant un paralytique qui était porté par quatre personnes.
\VS{4}Mais parce qu'ils ne pouvaient s'approcher de lui à cause de la foule, ils découvrirent le toit du lieu où il était, et l'ayant percé, ils descendirent le petit lit dans lequel le paralytique était couché.
\VS{5}Et Jésus, voyant leur foi, dit au paralytique : Mon enfant, tes péchés te sont pardonnés.
\VS{6}Et quelques scribes qui étaient assis là, raisonnaient dans leurs coeurs :
\VS{7}Pourquoi celui-ci prononce-t-il ainsi des blasphèmes ? Qui peut pardonner les péchés, si ce n'est Dieu seul ?
\VS{8}Et Jésus, ayant aussitôt connu par son esprit qu'ils raisonnaient ainsi en eux-mêmes, leur dit : Pourquoi avez-vous de telles pensées dans vos cœurs ?
\VS{9}Lequel est le plus aisé de dire au paralytique : Tes péchés te sont pardonnés, ou de dire : Lève-toi, prends ton petit lit et marche ?
\VS{10}Mais afin que vous sachiez que le Fils de l'homme a le pouvoir sur la terre de pardonner les péchés, il dit au paralytique :
\VS{11}Je te dis : Lève-toi, prends ton lit, et va dans ta maison.
\VS{12}Et il se leva aussitôt, et ayant pris son lit, il sortit en présence de tous ; de sorte qu'ils en furent tous étonnés, et ils glorifièrent Dieu, en disant : Nous n'avons jamais rien vu de pareil.
\TextTitle{Appel de Lévi\FTNTT{Mt. 9:9 ; Lu. 5:27-28}}
\VS{13}Et Jésus sortit de nouveau du côté de la mer, et toute la foule venait à lui, et il les enseignait.
\VS{14}En passant, il vit Lévi, fils d'Alphée, assis au bureau des péages, et il lui dit : Suis-moi. Et Lévi s'étant levé, le suivit.
\TextTitle{Appel des pécheurs à la repentance\FTNTT{Mt. 9:10-15 ; Lu. 5:29-35}}
\VS{15}Or il arriva que comme Jésus était à table dans la maison de Lévi, plusieurs publicains et des gens de mauvaise vie se mirent aussi à table avec lui et avec ses disciples ; car il y avait là beaucoup de gens qui l'avaient suivi.
\VS{16}Mais les scribes et les pharisiens voyant qu'il mangeait avec les publicains et les gens de mauvaise vie, disaient à ses disciples : Pourquoi mange-t-il et boit-il avec les publicains et les gens de mauvaise vie ?
\VS{17}Jésus ayant entendu cela, leur dit : Ce ne sont pas ceux qui se portent bien qui ont besoin de médecin, mais les malades. Je ne suis pas venu appeler à la repentance les justes, mais les pécheurs.
\TextTitle{Les pharisiens et les disciples de Jean interrogent Jésus sur le jeûne}
\VS{18}Or les disciples de Jean et ceux des pharisiens jeûnaient ; ils vinrent à Jésus et lui dirent : Pourquoi les disciples de Jean et ceux des pharisiens jeûnent-ils, tandis que tes disciples ne jeûnent point ?
\VS{19}Et Jésus leur répondit : Les amis de l'Epoux peuvent-ils jeûner pendant que l'Epoux est avec eux ? Aussi longtemps qu'ils ont avec eux l'Epoux, ils ne peuvent jeûner.
\VS{20}Mais les jours viendront où l'Epoux leur sera ôté, alors ils jeûneront en ce jour-là.
\TextTitle{Parabole du drap neuf et des outres neuves\FTNTT{Mt. 9:16-17 ; Lu. 5:36-39}}
\VS{21}Personne ne coud une pièce de drap neuf à un vieil habit ; autrement, la pièce du drap neuf emporterait une partie du vieux et la déchirure serait pire.
\VS{22}Et personne ne met du vin nouveau dans de vieilles outres ; autrement, le vin nouveau ferait rompre les outres et le vin se répandrait, et les outres seraient perdues ; mais le vin nouveau doit être mis dans des outres neuves.
\TextTitle{Jésus, le Maître du sabbat\FTNTT{Mt. 12:1-8 ; Lu. 6:1-5}}
\VS{23}Et il arriva, un jour de sabbat, qu'il traversa des champs de blé. Ses disciples en marchant se mirent à arracher des épis.
\VS{24}Et les pharisiens lui dirent : Regarde, pourquoi font-ils ce qui n'est pas permis les jours de sabbat ?
\VS{25}Mais il leur dit : N'avez-vous jamais lu ce que fit David quand il fut dans la nécessité, et qu'il eut faim, lui et ceux qui étaient avec lui ?
\VS{26}Comment il entra dans la maison de Dieu, au temps du souverain sacrificateur Abiathar, et mangea les pains de proposition\FTNT{1 S. 21:1-7.}, qu'il n'est permis qu'aux sacrificateurs de manger ; et il en donna même à ceux qui étaient avec lui!
\VS{27}Puis il leur dit : Le sabbat a été fait pour l'homme, et non l'homme pour le sabbat ;
\VS{28}de sorte que le Fils de l'homme est Maître même du sabbat.
\Chap{3}
\TextTitle{Jésus-Christ guérit un homme à la main sèche le jour du sabbat\FTNTT{Mt. 12:9-13 ; Lu. 6:6-11}}
\VerseOne{}Puis il entra de nouveau dans la synagogue, et il y avait là un homme qui avait une main sèche.
\VS{2}Et ils l'observaient, pour voir s'il le guérirait le jour du sabbat, afin de l'accuser.
\VS{3}Et Jésus dit à l'homme qui avait la main sèche : Lève-toi, et tiens-toi là au milieu.
\VS{4}Puis il leur dit : Est-il permis de faire du bien les jours de sabbat, ou de faire du mal, de sauver une personne, ou de la tuer ? Mais ils gardèrent le silence.
\VS{5}Alors, les regardant tous avec indignation, et étant affligé de l'endurcissement de leur cœur, il dit à cet homme : Etends ta main. Il l'étendit, et sa main fut rendue saine comme l'autre.
\TextTitle{Nombreuses guérisons de Jésus\FTNTT{Mt. 12:15-16 ; Lu. 6:17-19}}
\VS{6}Alors les pharisiens sortirent, et aussitôt, ils se consultèrent contre lui avec les Hérodiens, sur comment ils feraient pour le détruire.
\VS{7}Mais Jésus se retira vers la mer avec ses disciples. Une grande multitude le suivit de la Galilée,
\VS{8}de Judée, de Jérusalem, de l'Idumée, d'au-delà du Jourdain, et des environs de Tyr et de Sidon, une grande multitude, ayant entendu les grandes choses qu'il faisait, vint vers lui en grand nombre.
\VS{9}Et il dit à ses disciples de tenir toujours à sa disposition une petite barque afin de ne pas être pressé par la foule.
\VS{10}Car, comme il guérissait beaucoup de gens, tous ceux qui avaient des maladies se jetaient sur lui pour le toucher.
\VS{11}Et les esprits impurs, quand ils le voyaient, se prosternaient devant lui, et s'écriaient en disant : Tu es le Fils de Dieu.
\VS{12}Mais il leur défendait avec de grandes menaces de le faire connaître.
\TextTitle{L'appel des douze apôtres\FTNTT{Mt. 10:1-4 ; Lu. 6:13-16}}
\VS{13}Puis il monta sur une montagne, appela ceux qu'il voulut, et ils vinrent auprès de lui.
\VS{14}Et il en établit douze pour être avec lui,
\VS{15}et pour les envoyer prêcher, avec la puissance de guérir les maladies, et de chasser les démons hors des possédés.
\VS{16}Et ce sont ici les noms de ces douze : Simon qu'il nomma Pierre ;
\VS{17}Jacques fils de Zébédée, et Jean, frère de Jacques, auxquels il donna le nom de Boanergès, ce qui veut dire fils de tonnerre.
\VS{18}André, Philippe, Barthélemy, Matthieu, Thomas, Jacques, fils d'Alphée, Thaddée, Simon le Cananite,
\VS{19}et Judas Iscariot, celui qui le trahit.
\VS{20}Puis ils se rendirent à la maison, et une grande multitude s'assembla de nouveau, en sorte qu'ils ne pouvaient même pas prendre leur repas.
\VS{21}Quand ses parents apprirent cela, ils sortirent pour se saisir de lui. Car ils disaient : Il est hors de sens.
\TextTitle{Le blasphème contre le Saint-Esprit\FTNTT{Mt. 12:24-32 ; Lu. 11:15-23}}
\VS{22}Et les scribes, qui étaient descendus de Jérusalem, disaient : Il est possédé par Béelzébul ; c'est par le prince des démons qu'il chasse les démons.
\VS{23}Mais Jésus les appela, et leur dit sous forme de paraboles : Comment Satan peut-il chasser Satan ?
\VS{24}Car si un royaume est divisé contre lui-même, ce royaume ne peut pas subsister ;
\VS{25}et si une maison est divisée contre elle-même, cette maison ne peut pas subsister.
\VS{26}Si donc Satan s'élève contre lui-même, il est divisé, il ne peut subsister, mais il tend vers sa fin.
\VS{27}Personne ne peut entrer dans la maison d'un homme fort et piller ses biens, sans avoir auparavant lié cet homme fort ; alors il pillera sa maison.
\VS{28}En vérité je vous dis, que toutes sortes de péchés seront pardonnés aux enfants des hommes, et aussi toutes sortes de blasphèmes par lesquels ils auront blasphémé ;
\VS{29}mais quiconque blasphémera contre le Saint-Esprit n'obtiendra jamais de pardon : Il est coupable et subira une condamnation éternelle \FTNT{Voir commentaire Mt. 12:32.}.
\VS{30}Or c'était parce qu'ils disaient : Il est possédé d'un esprit impur.
\TextTitle{La famille spirituelle\FTNTT{Mt. 12:46-50 ; Lu. 8:19-21}}
\VS{31}Sur cela, arrivèrent là ses frères et sa mère qui, se tenant dehors, ils l'envoyèrent appeler. La multitude était assise autour de lui,
\VS{32}et on lui dit : Voici, ta mère et tes frères sont là dehors et te demandent.
\VS{33}Mais il leur répondit en disant : Qui est ma mère et qui sont mes frères ?
\VS{34}Et, jetant les regards sur ceux qui étaient assis tout autour de lui, il dit : Voici ma mère et mes frères.
\VS{35}Car quiconque fera la volonté de Dieu, celui-là est mon frère, ma sœur, et ma mère.
\Chap{4}
\TextTitle{Parabole du semeur et des quatre terrains\FTNTT{Mt. 13:1-9 ; Lu. 8:4-10}}
\VerseOne{}Puis il se mit de nouveau à enseigner près de la mer, et une grande foule s'étant assemblée auprès de lui, il monta dans une barque et s'assit dans la barque, sur la mer. Toute la foule était à terre sur le rivage de la mer.
\VS{2}Et il leur enseignait beaucoup de choses en paraboles, et il leur dit dans son enseignement :
\VS{3}Ecoutez. Un semeur sortit pour semer.
\VS{4}Et il arriva qu'en semant, une partie de la semence tomba le long du chemin, et les oiseaux du ciel vinrent, et la mangèrent toute.
\VS{5}Une autre partie tomba dans les endroits pierreux, où elle n'avait pas beaucoup de terre ; elle leva aussitôt, parce qu'elle n'entrait pas profondément dans la terre ;
\VS{6}mais, quand le soleil parut, elle fut brûlée, et parce qu'elle n'avait pas de racine, elle sécha.
\VS{7}Une autre partie tomba parmi les épines ; et les épines montèrent, et l'étouffèrent, et elle ne donna pas de fruit.
\VS{8}Une autre partie tomba dans la bonne terre, et donna du fruit qui montait et croissait en sorte qu'un grain en rapporta trente, un autre soixante et un autre cent.
\VS{9}Et il leur dit : Que celui qui a des oreilles pour entendre, qu'il entende !
\VS{10}Lorsqu'il fut à l'écart, ceux qui étaient autour de lui avec les douze, l'interrogèrent touchant cette parabole.
\VS{11}Et il leur dit : Il vous est donné de connaître le mystère du Royaume de Dieu ; mais pour ceux qui sont dehors, tout se passe en paraboles,
\VS{12}afin qu'en voyant, ils voient et n'aperçoivent pas, et qu'en entendant, ils entendent et ne comprennent pas, de peur qu'ils ne se convertissent, et que leurs péchés ne leur soient pardonnés.
\TextTitle{Explication de la parabole\FTNTT{Mt. 13:18-23 ; Lu. 8:11-15}}
\VS{13}Puis il leur dit : Ne comprenez-vous pas cette parabole ? Et comment donc comprendrez-vous toutes les paraboles ?
\VS{14}Le semeur c'est celui qui sème la parole.
\VS{15}Et voici, ceux qui reçoivent la semence le long du chemin, ce sont ceux en qui la parole est semée. Mais après qu'ils l'ont entendue, aussitôt Satan vient et enlève la parole qui a été semée dans leurs cœurs.
\VS{16}De même, ceux qui reçoivent la semence dans les endroits pierreux, ce sont ceux qui entendent la parole, ils la reçoivent aussitôt avec joie ;
\VS{17}mais ils n'ont pas de racine en eux-mêmes, ils ne durent qu'un temps, et dès que survient une tribulation ou une persécution à cause de la parole, ils sont aussitôt scandalisés.
\VS{18}D'autres reçoivent la semence parmi les épines ; ce sont ceux qui entendent la parole,
\VS{19}mais en qui les soucis de ce monde, la séduction des richesses et les convoitises des autres choses étant entrées dans leurs esprits, étouffent la parole, et elle devient infructueuse.
\VS{20}Mais ceux qui ont reçu la semence dans une bonne terre, ce sont ceux qui entendent la parole, la reçoivent, et portent du fruit : L'un trente, et l'autre soixante, et l'autre cent.
\TextTitle{Parabole de la lampe\FTNTT{Mt. 5:15-16 ; Lu. 8:16-18 ; 11:33-36}}
\VS{21}Il leur disait aussi : Apporte-t-on la lampe pour la mettre sous un boisseau ou sous un lit ? N'est-ce pas pour la mettre sur un chandelier ?
\VS{22}Car il n'y a rien de secret qui ne soit manifesté, rien de caché qui ne vienne en évidence.
\VS{23}Si quelqu'un a des oreilles pour entendre, qu'il entende.
\VS{24}Il leur dit encore : Prenez garde à ce que vous entendez. De la mesure dont vous mesurerez, il vous sera mesuré ; mais à vous qui entendez, il sera ajouté.
\VS{25}Car on donnera à celui qui a ; mais à celui qui n'a pas, on ôtera même ce qu'il a.
\TextTitle{Parabole de la semence et de la croissance spirituelle}
\VS{26}Il disait aussi : Il en est du Royaume de Dieu comme quand un homme jette la semence en terre ;
\VS{27}qu'il dorme ou qu'il veille, nuit et jour, la semence germe et croit, sans qu'il sache comment.
\VS{28}Car la terre produit d'elle-même, premièrement l'herbe, ensuite l'épi, et puis le grain formé dans l'épi ;
\VS{29}et quand le fruit est mûr, on y met aussitôt la faucille, parce que la moisson est prête.
\TextTitle{Parabole du grain de moutarde\FTNTT{Mt. 13:31-32 ; Lu. 13:18-19}}
\VS{30}Il disait encore : A quoi comparerons-nous le Royaume de Dieu, ou par quelle parabole le représenterons-nous ?
\VS{31}Il en est comme du grain de moutarde, qui, lorsqu'on le sème dans la terre, est la plus petite de toutes les semences qui sont jetées dans la terre.
\VS{32}Mais après qu'il a été semé, il monte et devient plus grand que toutes les autres plantes, et pousse de grandes branches, en sorte que les oiseaux du ciel peuvent faire leurs nids sous son ombre.
\VS{33}Ainsi c'est par beaucoup de paraboles de cette sorte qu'il leur annonçait la parole de Dieu, selon qu'ils pouvaient l'entendre.
\VS{34}Et il ne leur parlait point sans paraboles ; mais en particulier, il expliquait tout à ses disciples.
\TextTitle{Jésus apaise la tempête\FTNTT{Mt. 8:23-27 ; Lu. 8:22-25}}
\VS{35}Or en ce même jour, comme le soir fut venu, il leur dit : Passons sur l'autre bord.
\VS{36}Après avoir renvoyé la foule, ils l'emmenèrent avec eux dans la barque ; et il y avait aussi d'autres petites barques avec lui.
\VS{37}Et il se leva un grand tourbillon de vent, et les flots se jetaient dans la barque, de sorte qu'elle se remplissait déjà.
\VS{38}Et lui, il dormait à la poupe sur un oreiller. Et ils le réveillèrent, et lui dirent : Maître, ne t'inquiètes-tu pas de ce que nous périssons ?
\VS{39}S'étant réveillé, il menaça le vent, et dit à la mer : Silence ! Tais-toi ! Et le vent cessa, et il eut un grand calme.
\VS{40}Puis il leur dit : Pourquoi avez-vous si peur ? Comment n'avez-vous point de foi ?
\VS{41}Et ils furent saisis d'une grande crainte, et ils se dirent les uns les autres : Quel est donc celui-ci, à qui obéissent le vent et la mer ?
\Chap{5}
\TextTitle{Jésus-Christ délivre un possédé à Gadara\FTNTT{Mt. 8:28-34 ; Lu. 8:26-40}}
\VerseOne{}Et ils arrivèrent sur l'autre bord de la mer, dans le pays des Gadaréniens.
\VS{2}Aussitôt que Jésus fut descendu de la barque, un homme possédé d'un esprit impur, sortit d'abord des sépulcres, et vint le rencontrer.
\VS{3}Cet homme avait sa demeure dans les sépulcres, et personne ne pouvait plus le lier, pas même avec des chaînes.
\VS{4}Car souvent, il avait eu les fers aux pieds et avait été lié de chaînes, mais il avait rompu les chaînes et brisé les fers, et personne ne pouvait le dompter.
\VS{5}Il était continuellement, nuit et jour sur les montagnes et dans les sépulcres, criant et se meurtrissant avec des pierres.
\VS{6}Mais ayant vu Jésus de loin, il courut et se prosterna devant lui.
\VS{7}Et criant d'une voix forte, il dit : Qu'y a-t-il entre toi et moi, Jésus, Fils du Dieu Très-Haut ? Je te conjure au Nom de Dieu de ne pas me tourmenter.
\VS{8}Car Jésus lui disait : Sors de cet homme, esprit impur.
\VS{9}Alors il lui demanda : Quel est ton nom ? Légion\FTNT{Une légion romaine contenait entre trois et six mille soldats. C'est d'autant de démons que cet homme était possédé.} est mon nom, lui répondit-il, car nous sommes plusieurs.
\VS{10}Et il le priait instamment de ne pas les envoyer hors de cette contrée.
\VS{11}Or il y avait là, vers les montagnes, un grand troupeau de pourceaux qui paissaient.
\VS{12}Et tous ces démons le priaient en disant : Envoie-nous dans les pourceaux, afin que nous entrions en eux ;
\VS{13}et aussitôt Jésus le leur permit. Alors ces esprits impurs étant sortis, entrèrent dans les pourceaux, qui étaient environ deux mille, et le troupeau se précipita des pentes escarpées dans la mer ; et ils se noyèrent dans la mer.
\VS{14}Ceux qui paissaient les pourceaux s'enfuirent, et répandirent la nouvelle dans la ville et dans les campagnes. Ceux de la ville sortirent pour voir ce qui était arrivé.
\VS{15}Ils vinrent à Jésus et ils virent le démoniaque, celui qui avait eu la légion, assis et vêtu, et dans son bon sens ; et ils furent saisis de crainte.
\VS{16}Et ceux qui avaient vu le miracle leur racontèrent ce qui était arrivé au démoniaque et aux pourceaux.
\VS{17}Alors ils se mirent à supplier Jésus de quitter leur territoire.
\VS{18}Comme il montait dans la barque, celui qui avait été démoniaque le pria de lui permettre de rester avec lui.
\VS{19}Mais Jésus ne le lui permit pas, mais il lui dit : Va dans ta maison, vers les tiens, et raconte-leur les grandes choses que le Seigneur t'a faites, et comment il a eu pitié de toi.
\VS{20}Il s'en alla donc, et se mit à publier dans la Décapole les grandes choses que Jésus lui avait faites. Et tous furent dans l'étonnement.
\TextTitle{La résurrection de la fille de Jaïrus et la guérison de la femme atteinte d'une perte de sang\FTNTT{Mt. 9:18-26 ; Lu. 8:41-56}}
\VS{21}Et quand Jésus regagna de nouveau l'autre rive, dans la barque, une grande foule s'assembla près de lui ; il était près de la mer.
\VS{22}Alors vint un des chefs de la synagogue, nommé Jaïrus, qui l'ayant aperçu, se jeta à ses pieds,
\VS{23}et le pria instamment, en disant : Ma petite fille est à l'extrémité. Je te prie de venir et de lui imposer les mains, afin qu'elle soit guérie et qu'elle vive.
\VS{24}Jésus s'en alla donc avec lui. Et de grandes foules de gens le suivaient et le pressaient.
\VS{25}Or il y avait une femme qui avait une perte de sang depuis douze ans,
\VS{26}et qui avait beaucoup souffert entre les mains de plusieurs médecins. Elle avait dépensé tout ce qu'elle possédait, sans avoir éprouvé aucun soulagement, mais était allée plutôt en empirant.
\VS{27}Ayant entendu parler de Jésus, elle vint dans la foule par derrière et toucha son vêtement.
\VS{28}Car elle disait : Si je puis seulement toucher ses vêtements, je serai guérie.
\VS{29}Au même instant, la perte de sang s'arrêta ; et elle sentit en son corps qu'elle était guérie de son fléau.
\VS{30}Et aussitôt Jésus connut en lui-même qu'une force était sortie de lui, et, se retournant vers la foule, il dit : Qui a touché mes vêtements ?
\VS{31}Et ses disciples lui dirent : Tu vois que la foule te presse, et tu dis : Qui m'a touché ?
\VS{32}Mais il regardait tout autour pour voir celle qui avait fait cela.
\VS{33}Alors la femme saisie de crainte et toute tremblante, sachant ce qui s'était passé en elle, vint et se jeta à ses pieds, et lui déclara toute la vérité.
\VS{34}Et il lui dit : Ma fille ! Ta foi t'a sauvée. Va en paix, et sois guérie de ton fléau.
\VS{35}Comme il parlait encore, il vint des gens de chez le chef de la synagogue qui lui dirent : Ta fille est morte, pourquoi donnes-tu encore de la peine au Maître ?
\VS{36}Mais aussitôt que Jésus eut entendu cela, il dit au chef de la synagogue : Ne crains pas, crois seulement.
\VS{37}Et il ne permit à personne de le suivre, si ce n'est à Pierre, à Jacques et à Jean, frère de Jacques.
\VS{38}Ils arrivèrent à la maison du chef de la synagogue, où Jésus vit le tumulte, c'est-à-dire ceux qui pleuraient et qui poussaient de grands cris.
\VS{39}Il entra, et leur dit : Pourquoi faites-vous tout ce bruit, et pourquoi pleurez-vous ? La petite fille n'est pas morte, mais elle dort.
\VS{40}Et ils se moquèrent de lui. Mais Jésus les ayant tous fait sortir, prit le père et la mère de la petite fille, et ceux qui étaient avec lui, et entra là où la petite fille était couchée.
\VS{41}Il la saisit par la main, et lui dit : Talitha koumi, qui étant expliqué, veut dire : Jeune fille, je te dis lève-toi.
\VS{42}Aussitôt la petite fille se leva, et se mit à marcher ; car elle était âgée de douze ans. Et ils furent dans un grand étonnement.
\VS{43}Jésus leur recommanda fort expressément que personne ne le sache ; et il dit qu'on lui donne à manger.
\Chap{6}
\TextTitle{Jésus à Nazareth}
\VerseOne{}Puis il partit de là, et se rendit dans sa patrie. Ses disciples le suivirent.
\VS{2}Quand le jour du sabbat fut venu, il se mit à enseigner dans la synagogue. Et beaucoup de ceux qui l'entendaient étaient dans l'étonnement, et ils disaient : D'où lui viennent ces choses ? Et quelle est cette sagesse qui lui a été donnée, et comment de tels prodiges se font-ils par ses mains ?
\VS{3}N'est-ce pas le charpentier, le fils de Marie, frère de Jacques, de Joseph, de Jude, et de Simon ? Et ses sœurs ne sont-elles pas ici parmi nous ? Et ils étaient scandalisés à cause de lui.
\VS{4}Mais Jésus leur dit : Un prophète n'est sans honneur que dans sa patrie, parmi ses parents et dans sa famille.
\VS{5}Et il ne put faire là aucun miracle, si ce n'est qu'il guérit quelques malades en leur imposant les mains.
\VS{6}Et il s'étonnait de leur incrédulité. Il parcourait les villages d'alentour en enseignant.
\TextTitle{Mission des apôtres\FTNTT{Mt. 10:1-42 ; Lu. 9:1-6}}
\VS{7}Alors il appela les douze, et commença à les envoyer deux à deux, en leur donnant pouvoir sur les esprits impurs.
\VS{8}Il leur commanda de ne rien prendre pour le chemin, si ce n'est un bâton, et de ne porter ni sac, ni pain, ni monnaie dans leur ceinture ;
\VS{9}de chausser des sandales, et de ne pas porter deux tuniques.
\VS{10}Il leur disait aussi : Dans quelque maison que vous entrerez, demeurez-y jusqu'à ce que vous partiez de là.
\VS{11}Et tous ceux qui ne vous recevront pas, et ne vous écouteront pas, en partant de là, secouez la poussière de vos pieds en témoignage contre eux. Je vous le dis en vérité que ceux de Sodome et de Gomorrhe seront traités moins rigoureusement au jour du jugement que cette ville-là.
\VS{12}Ils partirent donc, et ils prêchèrent la repentance.
\VS{13}Ils chassèrent beaucoup de démons hors des possédés, et ils oignirent d'huile beaucoup de malades et les guérirent.
\TextTitle{Jean-Baptiste décapité\FTNTT{Mt. 14:1-14 ; Lu. 9:7-9}}
\VS{14}Or le roi Hérode entendit parler de Jésus dont le nom était devenu fort célèbre, et il dit : Ce Jean, qui baptisait, est ressuscité des morts ; c'est pourquoi la puissance de faire des miracles agit puissamment en lui.
\VS{15}Les autres disaient : C'est Elie. Et les autres disaient : C'est un prophète, ou comme l'un des prophètes.
\VS{16}Mais Hérode en apprenant cela, disait : C'est Jean que j'ai fait décapiter, il est ressuscité des morts.
\VS{17}Car Hérode avait fait arrêter Jean, et l'avait fait lier en prison, à cause d'Hérodias, femme de Philippe son frère, parce qu'il l'avait prise en mariage.
\VS{18}Car Jean disait à Hérode : Il ne t'est pas permis d'avoir la femme de ton frère.
\VS{19}C'est pourquoi Hérodias était irritée contre Jean, et voulait le faire mourir,
\VS{20}mais elle ne le pouvait pas, car Hérode craignait Jean, sachant que c'était un homme juste et saint ; il avait du respect pour lui, et lorsqu'il l'avait entendu, il faisait beaucoup de choses que Jean avait dit de faire, car il l'écoutait volontiers.
\VS{21} Mais un jour propice arriva, lorsque Hérode, à l'occasion du jour de sa naissance, donna un festin aux grands seigneurs de sa cour, aux chefs militaires et aux principaux de la Galilée.
\VS{22}La fille d'Hérodias entra dans la salle ; elle dansa et plut à Hérode, et à ceux qui étaient à table avec lui. Le roi dit à la jeune fille : Demande-moi ce que tu voudras, et je te le donnerai.
\VS{23}Et il ajouta avec serment : Tout ce que tu me demanderas, je te le donnerai, serait-ce la moitié de mon royaume.
\VS{24}Elle, étant sortie, dit à sa mère : Que demanderai-je ? Et sa mère lui dit : La tête de Jean-Baptiste.
\VS{25}Puis étant revenue en toute hâte vers le roi, elle lui fit sa demande en disant : Je veux que tu me donnes à l'instant sur un plat, la tête de Jean-Baptiste.
\VS{26}Le roi fut attristé, mais à cause de son serment et des convives, il ne voulut pas refuser.
\VS{27}Il envoya sur-le-champ l'un de ses gardes avec ordre d'apporter la tête de Jean.
\VS{28}Le garde alla décapiter Jean dans la prison, et apporta sa tête dans un plat, et la donna à la jeune fille. Et la jeune fille la donna à sa mère.
\VS{29}Ce que les disciples de Jean ayant appris, ils vinrent et emportèrent son corps, et le mirent dans un sépulcre.
\TextTitle{Les apôtres rendent compte de leur mission à Jésus\FTNTT{Lu. 9:10}}
\VS{30}Les apôtres se rassemblèrent auprès de Jésus, et lui racontèrent tout ce qu'ils avaient fait et enseigné.
\VS{31}Et il leur dit : Venez à l'écart dans un lieu désert, et reposez-vous un peu ; car il y avait beaucoup de gens qui allaient et qui venaient, de sorte qu'ils n'avaient même pas l'occasion de manger.
\TextTitle{Multiplication des pains\FTNTT{Mt. 14:12-21 ; Lu. 9:15-17 ; Jn. 6:1-14}}
\VS{32}Ils s'en allèrent donc dans une barque, à l'écart, dans un lieu désert.
\VS{33}Mais le peuple vit qu'ils s'en allaient, et plusieurs l'ayant reconnu, y accoururent à pied de toutes les villes, et y arrivèrent avant eux, et s'assemblèrent auprès de lui.
\VS{34}Quand il sortit, Jésus vit une grande foule, et fut ému de compassion pour elle, parce qu'ils étaient comme des brebis qui n'ont pas de pasteur ; et il se mit à leur enseigner plusieurs choses.
\VS{35}Comme il était déjà tard, ses disciples s'approchèrent de lui, en disant : Ce lieu est désert, et il est déjà tard,
\VS{36}renvoie-les, afin qu'ils s'en aillent dans les campagnes et dans les villages des environs pour s'acheter des pains ; car ils n'ont rien à manger.
\VS{37}Jésus leur répondit et dit : Donnez-leur vous-mêmes à manger. Et ils lui dirent : Irions-nous acheter des pains pour deux cents deniers, afin de leur donner à manger ?
\VS{38}Et il leur dit : Combien avez-vous de pains ? Allez et regardez. Et quand ils le surent, ils dirent : Cinq, et deux poissons.
\VS{39}Alors il leur commanda de les faire tous asseoir par groupes sur l'herbe verte.
\VS{40}Et ils s'assirent par rangées de cent et de cinquante personnes.
\VS{41}Et quand il prit les cinq pains et les deux poissons, il leva les yeux vers le ciel, il bénit Dieu et rompit les pains, puis il les donna à ses disciples, afin qu'ils les mettent devant eux. Il partagea aussi les deux poissons entre tous.
\VS{42}Et tous mangèrent et furent rassasiés.
\VS{43}Et l'on emporta douze paniers pleins de morceaux de pains et de ce qui restait des poissons.
\VS{44}Or ceux qui avaient mangé les pains étaient environ cinq mille hommes.
\TextTitle{Jésus marche sur la mer\FTNTT{Mt. 14:22-33 ; Jn. 6:15-21}}
\VS{45}Et aussitôt après, il obligea ses disciples à monter sur la barque, et à le devancer sur l'autre bord, vers Bethsaïda, pendant que lui-même renverrait la foule.
\VS{46}Quand il l'eut renvoyée, il s'en alla sur la montagne pour prier.
\VS{47}Le soir étant venu, la barque était au milieu de la mer, et Jésus était seul à terre.
\VS{48}Il vit qu'ils avaient beaucoup de peine à ramer parce que le vent leur était contraire. Vers la quatrième veille de la nuit, il alla vers eux marchant sur la mer, et il voulait les devancer.
\VS{49}Mais quand ils le virent marcher sur la mer, ils crurent que c'était un fantôme, et ils poussèrent des cris ;
\VS{50}car ils le voyaient tous, et ils furent troublés. Mais il leur parla aussitôt, et leur dit : Rassurez-vous, c'est moi. N'ayez pas peur.
\VS{51}Et il monta vers eux dans la barque, et le vent cessa. Et ils furent en eux-mêmes excessivement étonnés et remplis d'admiration.
\VS{52}Car ils n'avaient pas compris le miracle des pains, parce que leur cœur était endurci.
\TextTitle{Jésus guérit les malades à Génésareth\FTNTT{Mt. 14:34-36}}
\VS{53}Après avoir traversé la mer, ils arrivèrent dans la contrée de Génésareth, où ils abordèrent.
\VS{54}Et après qu'ils furent sortis de la barque, les gens, ayant aussitôt reconnu Jésus,
\VS{55}parcoururent tous les environs, et se mirent à lui apporter de tous côtés les malades sur de petits lits, partout où ils apprenaient qu'il était.
\VS{56}Et partout où il entrait, dans les villages, dans les villes, ou dans les campagnes, ils mettaient les malades dans les places publiques, et ils le priaient de leur permettre seulement de toucher le bord de son vêtement. Et tous ceux qui le touchaient étaient guéris.
\Chap{7}
\TextTitle{Jésus condamne les traditions\FTNTT{Mt. 15:1-9}}
\VerseOne{}Alors les pharisiens, et quelques scribes qui étaient venus de Jérusalem, s'assemblèrent auprès de lui.
\VS{2}Et ils virent quelques-uns de ses disciples mangeant du pain avec des mains impures, c'est-à-dire non lavées, et ils les blâmèrent.
\VS{3}Car les pharisiens et tous les Juifs ne mangent pas sans s'être lavé leurs mains jusqu'au coude, conformément à la tradition des anciens.
\VS{4}Et quand ils reviennent de la place publique, ils ne mangent qu'après s'être lavés\FTNT{Le verbe laver vient du grec « baptizo » : « Plonger », « immerger », « submerger », « purifier en plongeant ou en submergeant », « laver », « rendre pur avec de l’eau », « se baigner » (Mt. 3 :6-16 ; Mt. 28 :19 ; Ac. 1 :5 ; Ac. 2 :38 ; 1 Co. 12 :13, etc.). Craignant de se retrouver en état d’impureté, beaucoup  de Juifs du premier siècle, en particulier parmi les esséniens et les pharisiens, se soumettaient quotidiennement à de nombreux rituels de purification avec de l’eau. A titre d’exemple, les jarres de Cana étaient utilisées à cet effet (Jn. 2 :6).}. Il y a plusieurs autres observances dont ils se sont chargés, comme le lavage des coupes, de cruches, des vases d'airain, et des lits.
\VS{5}Sur cela, les pharisiens et les scribes l'interrogèrent, en disant : Pourquoi tes disciples ne se conduisent-ils pas selon la tradition des anciens, mais prennent-ils leur repas sans se laver les mains ?
\VS{6}Jésus leur répondit et leur dit : Certainement Esaïe a bien prophétisé sur vous, hypocrites, comme il est écrit : Ce peuple m'honore des lèvres, mais leur cœur est éloigné de moi\FTNT{Es. 29:13.}.
\VS{7}C'est en vain qu'ils m'honorent, en enseignant des doctrines qui sont des commandements d'hommes.
\VS{8}Vous abandonnez le commandement de Dieu, et vous retenez la tradition des hommes, à savoir le lavage des cruches et des coupes, et vous faites beaucoup d'autres choses semblables.
\VS{9}Il leur dit aussi : Vous rejetez bien le commandement de Dieu, afin de garder votre tradition.
\VS{10}Car Moïse a dit : Honore ton père et ta mère ; et : Celui qui maudira son père ou sa mère, finira à la mort.
\VS{11}Mais vous, vous dites : Si quelqu'un dit à son père ou à sa mère : Tout ce dont je pourrais t'assister est corban, c'est-à-dire une offrande à Dieu, il ne sera point coupable.
\VS{12}Et vous ne lui permettez plus de rien faire pour son père ou pour sa mère,
\VS{13}anéantissant ainsi la parole de Dieu par votre tradition que vous avez établie. Et vous faites encore beaucoup d'autres choses semblables.
\TextTitle{Le coeur humain\FTNTT{Mt. 15:10-20.}}
\VS{14}Puis, ayant appelé la foule, il leur dit : Ecoutez-moi vous tous, et entendez.
\VS{15}Il n'est hors de l'homme rien qui, entrant en lui, puisse le souiller ; mais ce qui sort de l'homme, c'est ce qui le souille.
\VS{16}Si quelqu'un a des oreilles pour entendre, qu'il entende.
\VS{17}Puis quand il fut entré dans la maison, loin de la foule, ses disciples l'interrogèrent sur cette parabole.
\VS{18}Et il leur dit : Vous aussi, êtes-vous sans intelligence ? Ne comprenez-vous pas que rien de ce qui du dehors entre dans l'homme ne peut le souiller ?
\VS{19}Parce qu'il n'entre pas dans son cœur, mais dans son ventre, puis s'en va ensuite dans les lieux secrets, qui purifient le corps de tous les aliments.
\VS{20}Mais il leur disait : Ce qui sort de l'homme, c'est ce qui souille l'homme.
\VS{21}Car c'est du dedans, c'est-à-dire du cœur des hommes, que sortent les mauvaises pensées, les adultères, les fornications, les meurtres,
\VS{22}les vols, les cupidités, les méchancetés, la fraude, l'impudicité, le regard envieux, les discours outrageux, l'orgueil, la folie.
\VS{23}Tous ces maux sortent du dedans, et souillent l'homme.
\TextTitle{Jésus et la femme syro-phénicienne\FTNTT{Mt. 15:21-28}}
\VS{24}Puis, étant parti de là, il s'en alla dans le territoire de Tyr et de Sidon. Il entra dans une maison, désirant que personne ne le sache ; mais il ne put rester caché.
\VS{25}Car une femme, dont la petite fille était possédée d'un esprit impur, ayant entendu parler de lui, vint et se jeta à ses pieds.
\VS{26}Or cette femme était grecque, syro-phénicienne d'origine. Elle le pria de chasser le démon hors de sa fille. Jésus lui dit :
\VS{27}Laisse premièrement les enfants se rassasier ; car il n'est pas raisonnable de prendre le pain des enfants, et de le jeter aux petits chiens.
\VS{28}Et elle lui répondit et dit : Cela est vrai, Seigneur ! Cependant les petits chiens mangent sous la table les miettes que les enfants laissent tomber.
\VS{29}Alors il lui dit : A cause de cette parole, va, le démon est sorti de ta fille.
\VS{30}Et quand elle rentra dans sa maison, elle trouva sa fille couchée sur le lit, le démon étant sorti.
\TextTitle{Jésus guérit un sourd-muet\FTNTT{Mt. 15:29-31}}
\VS{31}Puis Jésus quitta le territoire de Tyr et de Sidon, et revint vers la mer de Galilée en traversant le pays de la Décapole.
\VS{32}On lui amena un sourd qui avait la parole empêchée, et on le pria de lui imposer les mains.
\VS{33}Et Jésus le prit à part, hors de la foule, et lui mit les doigts dans les oreilles, et lui toucha la langue avec sa propre salive.
\VS{34}Puis, levant les yeux vers le ciel, il soupira, et lui dit : Ephphatha, c'est-à-dire : Ouvre-toi.
\VS{35}Aussitôt ses oreilles s'ouvrirent, et le lien de sa langue se délia, et il parla aisément.
\VS{36}Et Jésus leur recommanda de ne le dire à personne ; mais plus il le leur recommandait, plus ils le publiaient.
\VS{37}Et ils en étaient extrêmement étonnés, et disaient : Il fait tout à merveille ; même il fait entendre les sourds, et parler les muets.
\Chap{8}
\TextTitle{Seconde multiplication des pains\FTNTT{Mt. 15:32-39}}
\VerseOne{}En ces jours-là, une grande foule s'était réunie et n'avait rien à manger. Jésus appela ses disciples, et leur dit :
\VS{2}Je suis ému de compassion pour cette foule, car il y a déjà trois jours qu'ils restent auprès de moi, et ils n'ont rien à manger.
\VS{3}Si je les renvoie chez eux à jeun, ils tomberont en défaillance en chemin, car quelques-uns d'eux sont venus de loin.
\VS{4}Et ses disciples lui répondirent : Comment pourrait-on les rassasier de pains, ici, dans un désert ?
\VS{5}Jésus leur demanda : Combien avez-vous de pains ? Sept lui répondirent-ils.
\VS{6}Alors il ordonna à la foule de s'asseoir par terre, et il prit les sept pains, et après avoir béni Dieu, il les rompit, et les donna à ses disciples pour les mettre devant la foule ; et ils les mirent devant elle.
\VS{7}Ils avaient aussi quelques petits poissons ; et après avoir béni Dieu, il ordonna qu'on les mette aussi devant.
\VS{8}Et ils mangèrent, et furent rassasiés ; et l'on remporta sept corbeilles pleines des morceaux qui restaient.
\VS{9}Or ceux qui avaient mangé étaient environ quatre mille. Ensuite Jésus les renvoya.
\TextTitle{Mise en garde de l'enseignement des pharisiens, un levain corrompu\FTNTT{Mt. 16:1-12}}
\VS{10}Aussitôt après, il monta dans la barque avec ses disciples, et se rendit dans la contrée de Dalmanutha.
\VS{11}Les pharisiens survinrent, se mirent à discuter avec lui, et pour l'éprouver, lui demandèrent un signe venant du ciel.
\VS{12}Alors, Jésus soupirant profondément en son esprit, dit : Pourquoi cette génération demande-t-elle un signe ? Je vous le dis en vérité, il ne sera point donné de signe à cette génération.
\VS{13}Puis il les quitta, et remonta dans la barque, pour passer à l'autre rivage.
\VS{14}Les disciples avaient oublié de prendre des pains ; et ils n'en avaient qu'un seul avec eux dans la barque.
\VS{15}Jésus leur fit cette recommandation : Gardez-vous avec soin du levain des pharisiens et du levain d'Hérode.
\VS{16}Ils raisonnaient entre eux, disant : C'est parce que nous n'avons pas de pains.
\VS{17}Et Jésus, le sachant, leur dit : Pourquoi discourez-vous sur ce que vous n'avez pas de pains ? N'entendez-vous pas encore, et ne comprenez-vous pas ?
\VS{18}Avez-vous encore votre cœur endurci ? Ayant des yeux, ne voyez-vous point ? Ayant des oreilles, n'entendez-vous point ? Et n'avez-vous point de mémoire ?
\VS{19}Quand j'ai rompu les cinq pains pour les cinq mille hommes, combien de paniers pleins de morceaux avez-vous emportés ? Douze, lui répondirent-ils.
\VS{20}Et quand j'ai rompu les sept pains pour quatre mille hommes, combien de corbeilles pleines de morceaux avez-vous emportées ? Sept, répondirent-ils.
\VS{21}Et il leur dit : Comment n'avez-vous pas d'intelligence ?
\TextTitle{Jésus guérit un aveugle}
\VS{22}Puis ils se rendirent à Bethsaïda, et on lui présenta un aveugle qu'on le pria de toucher.
\VS{23}Alors il prit la main de l'aveugle, et le conduisit hors du village ; puis il lui mit de la salive sur les yeux, lui imposa les mains, et lui demanda s'il voyait quelque chose.
\VS{24}Et cet homme ayant regardé, dit : Je vois des hommes qui marchent, et qui me paraissent comme des arbres.
\VS{25}Jésus lui mit encore les mains sur les yeux, et lui dit de regarder ; et il fut rétabli, et les voyait tous clairement.
\VS{26}Puis il le renvoya dans sa maison, en lui disant : N'entre pas dans le village, et ne le dis à personne du village.
\TextTitle{Pierre reconnaît Jésus comme le Messie\FTNTT{Mt. 16:13-16 ; Lu. 9:18-21 ; Jn. 6:67-71}}
\VS{27}Et Jésus s'en alla, avec ses disciples, dans les villages de Césarée de Philippe, et sur le chemin il interrogea ses disciples, leur disant : Qui dit-on que je suis ?
\VS{28}Ils répondirent : Les uns disent que tu es Jean-Baptiste ; les autres, Elie ; et les autres, l'un des prophètes.
\VS{29}Alors il leur dit : Et vous, qui dites-vous que je suis ? Pierre lui répondit : Tu es le Christ.
\VS{30}Et il leur défendit très sévèrement de ne dire cela de lui à personne.
\VS{31}Et il commença à leur enseigner qu'il fallait que le Fils de l'homme souffre beaucoup, et qu'il soit rejeté par les anciens, par les principaux sacrificateurs et par les scribes, et qu'il soit mis à mort, et qu'il ressuscite trois jours après.
\VS{32}Il leur tenait ces discours ouvertement. Et Pierre l'ayant pris à part, se mit à le reprendre.
\VS{33}Mais lui, se retournant et regardant ses disciples, réprimanda Pierre en lui disant : Va arrière de moi, Satan ! Car tu ne comprends pas les choses qui sont de Dieu, mais celles qui sont des hommes.
\TextTitle{La consécration du disciple\FTNTT{Mt. 16:24-28 ; Lu. 9:23-26}}
\VS{34}Puis, ayant appelé la foule et ses disciples, il leur dit : Si quelqu'un veut venir après moi, qu'il renonce à lui-même, qu'il se charge de sa croix et qu'il me suive.
\VS{35}Car quiconque voudra sauver son âme, la perdra ; mais quiconque perdra son âme pour l'amour de moi et de l'Evangile, celui-là la sauvera.
\VS{36}Car que sert-il à un homme de gagner tout le monde, s'il perd son âme ?
\VS{37}Que donnerait l'homme en échange de son âme ?
\VS{38}Car quiconque aura honte de moi et de mes paroles au milieu de cette génération adultère et pécheresse, le Fils de l'homme aura aussi honte de lui, quand il viendra environné de la gloire de son Père avec les saints anges.
\Chap{9}
\TextTitle{La transfiguration\FTNTT{Mt. 17:1-9 ; Lu. 9:27-36}}
\VerseOne{}Il leur disait aussi : En vérité je vous le dis, quelques-uns de ceux qui sont ici présents ne mourront point qu'ils n'aient vu le Royaume de Dieu venir avec puissance\FTNT{Voir commentaire Mt. 16:28.}.
\VS{2}Et six jours après, Jésus prit avec lui Pierre, Jacques et Jean, et les conduisit seuls à l'écart sur une haute montagne. Il fut transfiguré devant eux,
\VS{3}et ses vêtements devinrent resplendissants, et blancs comme de la neige, tels qu'il n'est pas de foulon sur la terre qui puisse blanchir ainsi.
\VS{4}Et en même temps leur apparurent Moïse et Elie, qui s'entretenaient avec Jésus.
\VS{5}Alors Pierre prenant la parole, dit à Jésus : Rabbi, il est bon que nous soyons ici ; faisons donc trois tentes, une pour toi, une pour Moïse, et une pour Elie.
\VS{6}Or il ne savait pas ce qu'il disait, car ils étaient épouvantés.
\VS{7}Une nuée vint les couvrir de son ombre, et de la nuée sortit une voix : Celui-ci est mon Fils bien-aimé, écoutez-le.
\VS{8}Et aussitôt les disciples regardèrent tout autour, et ils ne virent que Jésus seul avec eux.
\VS{9}Comme ils descendaient de la montagne, il leur recommanda expressément de ne raconter à personne ce qu'ils avaient vu, jusqu'à ce que le Fils de l'homme soit ressuscité des morts.
\VS{10}Et ils retinrent cette parole en eux-mêmes, se demandant entre eux ce que c'était que ressusciter des morts.
\VS{11}Puis ils l'interrogèrent, disant : Pourquoi les scribes disent-ils qu'il faut qu'Elie vienne premièrement ?
\VS{12}Il leur répondit et leur dit : Il est vrai, Elie viendra premièrement, et rétablira toutes choses, et comme il est écrit du Fils de l'homme, il faut qu'il souffre beaucoup, et qu'il soit chargé de mépris.
\VS{13}Mais je vous dis même qu'Elie est venu, et qu'ils lui ont fait tout ce qu'ils ont voulu, selon qu'il est écrit de lui ;
\TextTitle{L'incapacité des disciples et la toute-puissance de Jésus-Christ\FTNTT{Mt. 17:14-21 ; Lu. 9:37-43}}
\VS{14}et lorsqu'il fut arrivé près des disciples, il vit une grande foule autour d'eux, et des scribes qui discutaient avec eux.
\VS{15}Dès que la foule vit Jésus, elle fut saisie d'étonnement, et accourut pour le saluer.
\VS{16}Alors il interrogea les scribes disant : De quoi discutez-vous avec eux ?
\VS{17}Et un homme de la foule prenant la parole, dit : Maître, je t'ai amené mon fils qui est possédé d'un esprit muet.
\VS{18}En quelque lieu qu'il le saisisse, il le jette par terre ; l'enfant écume, grince des dents, et devient tout raide. J'ai prié tes disciples de chasser ce démon, mais ils n'ont pas pu.
\VS{19}Alors Jésus leur répondit et dit : Ô génération incrédule ! Jusqu'à quand serai-je avec vous ? Jusqu'à quand vous supporterai-je ? Amenez-le-moi. Ils le lui amenèrent  ;
\VS{20}et aussitôt que l'enfant vit Jésus, l'esprit l'agita sur-le-champ avec violence ; il tomba par terre, et se roulait en écumant.
\VS{21}Jésus demanda au père de l'enfant : Combien y a-t-il de temps que cela lui arrive ? Et il dit : Dès son enfance.
\VS{22}Et souvent l'esprit l'a jeté dans le feu et dans l'eau pour le faire périr. Mais si tu peux quelque chose, secours-nous, aie compassion de nous.
\VS{23}Alors Jésus lui dit : Si tu peux croire, tout est possible à celui qui croit.
\VS{24}Et aussitôt le père de l'enfant s'écriant avec larmes, dit : Je crois, Seigneur ! Secours-moi dans mon incrédulité.
\VS{25}Et quand Jésus vit la foule accourir ensemble, il reprit sévèrement l'esprit impur, et lui dit : Esprit muet et sourd, je te l'ordonne, sors de cet enfant, et n'y rentre plus !
\VS{26}Et le démon sortit, en poussant des cris, et en l'agitant avec une grande violence. L'enfant devint comme mort, de sorte que plusieurs disaient qu'il était mort.
\VS{27}Mais Jésus, l'ayant pris par la main, le fit lever. Et il se tint debout.
\VS{28}Puis Jésus fut entré dans la maison, ses disciples lui demandèrent en particulier : Pourquoi n'avons-nous pas pu le chasser ?
\VS{29}Il leur répondit : Cette sorte de démons ne peut sortir que par la prière et par le jeûne.
\TextTitle{Jésus annonce sa mort et sa résurrection\FTNTT{Mt. 17:22-23 ; Lu. 9:44-45}}
\VS{30}Puis étant partis de là, ils traversèrent la Galilée. Jésus ne voulait pas qu'on le sache.
\VS{31}Or il enseignait ses disciples, et il leur dit : Le Fils de l'homme va être livré entre les mains des hommes, et ils le feront mourir, mais après qu'il aura été mis à mort, il ressuscitera le troisième jour.
\VS{32}Mais ils ne comprenaient point ce discours, et ils craignaient de l'interroger.
\TextTitle{L'humilité, secret de la vraie grandeur\FTNTT{Mt. 18:1-6 ; Lu. 9:46-48}}
\VS{33}Après ces choses il vint à Capernaüm, et quand il fut arrivé à la maison, il leur demanda : De quoi discutiez-vous ensemble en chemin ?
\VS{34}Mais ils gardèrent le silence, car ils avaient discuté entre eux en chemin sur celui qui serait le plus grand.
\VS{35}Alors il s'assit, appela les douze, et leur dit : Si quelqu'un veut être le premier parmi vous, il sera le dernier de tous, et le serviteur de tous.
\VS{36}Et ayant pris un petit enfant, il le mit au milieu d'eux, et après l'avoir pris entre ses bras, il leur dit :
\VS{37}Quiconque reçoit en mon Nom un de ces petits enfants, me reçoit ; et quiconque me reçoit, ce n'est pas moi qu'il reçoit, mais celui qui m'a envoyé.
\TextTitle{Jésus condamne l'esprit sectaire\FTNTT{Lu. 9:49-50}}
\VS{38}Alors Jean prit la parole, et dit : Maître, nous avons vu quelqu'un qui chasse les démons en ton Nom et qui pourtant ne nous suit pas, et nous l'en avons empêché, parce qu'il ne nous suit pas.
\VS{39}Mais Jésus leur dit : Ne l'en empêchez pas ; parce qu'il n'est personne qui, faisant un miracle en mon Nom, puisse aussitôt après parler mal de moi.
\VS{40}Qui n'est pas contre nous est pour nous.
\VS{41}Et quiconque vous donnera à boire un verre d'eau en mon Nom, parce que vous êtes à Christ, je vous le dis en vérité, il ne perdra point sa récompense.
\TextTitle{Avertissement de Jésus concernant les occasions de chute}
\VS{42}Mais quiconque scandalisera un de ces petits qui croient en moi, il vaudrait mieux pour lui qu'on lui mette une pierre de moulin autour de son cou, et qu'on le jette dans la mer.
\VS{43}Or si ta main est pour toi une occasion de chute, coupe-la ; mieux vaut pour toi entrer manchot dans la vie, que d'avoir les deux mains, et d'aller dans la géhenne, dans le feu qui ne s'éteint point ;
\VS{44}là où leur ver ne meurt point, et où le feu ne s'éteint point.
\VS{45}Et si ton pied est pour toi une occasion de chute, coupe-le ; mieux vaut pour toi entrer boiteux dans la vie, que d'avoir les deux pieds, et d'être jeté dans la géhenne, dans le feu qui ne s'éteint point ;
\VS{46}là où leur ver ne meurt point, et où le feu ne s'éteint point.
\VS{47}Et si ton œil est pour toi une occasion de chute, arrache-le ; mieux vaut pour toi entrer dans le Royaume de Dieu n'ayant qu'un œil, que d'avoir les deux yeux, et d'être jeté dans le feu de la géhenne,
\VS{48}là où leur ver ne meurt point, et où le feu ne s'éteint point.
\VS{49}Car chacun sera salé de feu ; et toute offrande sera salée de sel.
\VS{50}Le sel est une bonne chose ; mais si le sel devient sans saveur, avec quoi lui rendra-t-on sa saveur ?
\VS{51}Ayez du sel en vous-mêmes, et soyez en paix les uns avec les autres.
\Chap{10}
\TextTitle{Enseignement de Jésus sur le mariage et le divorce\FTNTT{Mt. 5:31-32 ; 19:1-9 ; Lu. 16:18 ; 1 Co. 7:10-16 ; Ro. 7:1-3.}}
\VerseOne{}Puis, étant parti de là, il se rendit dans le territoire de la Judée, au-delà du Jourdain. La foule s'assembla de nouveau auprès de lui, et selon sa coutume, il se mit à l'enseigner.
\VS{2}Alors les pharisiens vinrent à lui, et, pour l'éprouver, ils lui demandèrent s'il est permis à un homme de répudier sa femme.
\VS{3}Il répondit et leur dit : Qu'est-ce que Moïse vous a commandé ?
\VS{4}Moïse, dirent-ils, a permis d'écrire une lettre de divorce, et de répudier ainsi sa femme\FTNT{De. 24:1.}.
\VS{5}Et Jésus répondant leur dit : C'est à cause de la dureté de votre cœur que Moïse vous a donné ce commandement.
\VS{6}Mais au commencement de la création, Dieu fit un homme et une femme.
\VS{7}C'est pourquoi l'homme quittera son père et sa mère, et s'attachera à sa femme,
\VS{8}et les deux deviendront une seule chair. Ainsi, ils ne sont plus deux, mais ils sont une seule chair.
\VS{9}Que l'homme donc ne sépare pas ce que Dieu a mis ensemble sous un joug\FTNT{Voir commentaire Mt. 19:6.}.
\VS{10}Lorsqu'ils furent dans la maison, ses disciples l'interrogèrent encore là-dessus.
\VS{11}Il leur dit : Celui qui répudie sa femme et qui en épouse une autre, commet un adultère contre elle.
\VS{12}Pareillement si la femme répudie son mari, et se marie à un autre, elle commet un adultère.
\TextTitle{Le Royaume des cieux, pour ceux qui ressemblent aux petits enfants\FTNTT{Mt. 19:13-15 ; Lu. 18:15-17}}
\VS{13}On lui amena de petits enfants afin qu'il les touche. Mais les disciples reprirent ceux qui les amenaient.
\VS{14}Jésus, voyant cela, fut indigné, et leur dit : Laissez venir à moi les petits enfants et ne les en empêchez point, car le Royaume de Dieu appartient à ceux qui leur ressemblent.
\VS{15}Je vous le dis en vérité, quiconque ne recevra pas comme un petit enfant le Royaume de Dieu, il n'y entrera point.
\VS{16}Après les avoir donc pris entre ses bras, il les bénit, en leur imposant les mains.
\TextTitle{Le jeune homme riche\FTNTT{Mt. 19:16-26 ; Lu. 18:18-27.}}
\VS{17}Et comme il sortait pour se mettre en chemin, un homme accourut, et se mit à genoux devant lui, et lui fit cette demande : Maître qui es bon, que ferai-je pour hériter la vie éternelle ?
\VS{18}Et Jésus lui répondit : Pourquoi m'appelles-tu bon ? Il n'y a nul être qui soit bon que Dieu\FTNT{La même histoire est racontée en Lu. 18:18 qui précise que c'était un chef qui avait interrogé Jésus. La réponse du Seigneur est ironique. Jésus aurait aussi pu lui poser la question comme suit : « Puisque tu penses que je ne suis qu'un simple homme, pourquoi m'appelles-tu bon ? ».}.
\VS{19}Tu connais les commandements : Ne commets point d'adultère ; ne tue point ; ne dérobe point ; ne dis point de faux témoignage ; ne fais aucun tort à personne ; honore ton père et ta mère.
\VS{20}Il lui répondit et lui dit : Maître, j'ai gardé toutes ces choses dès ma jeunesse.
\VS{21}Jésus, l'ayant regardé, l'aima, et lui dit : Il te manque une chose : Va et vends tout ce que tu as, et donne-le aux pauvres, et tu auras un trésor dans le ciel. Puis, viens et suis-moi en te chargeant de ta croix.
\VS{22}Mais, fâché de cette parole, il s'en alla tout triste parce qu'il avait de grands biens.
\TextTitle{Rien n'est impossible à Dieu}
\VS{23}Alors Jésus, ayant regardé autour de lui, dit à ses disciples : Qu'il est difficile à ceux qui ont des richesses d'entrer dans le Royaume de Dieu.
\VS{24}Et ses disciples furent étonnés de ces paroles ; mais Jésus reprenant la parole, leur dit : Mes enfants, qu'il est difficile à ceux qui se confient dans les richesses d'entrer dans le Royaume de Dieu !
\VS{25}Il est plus facile à un chameau de passer par le trou d'une aiguille\FTNT{Voir commentaire Mt. 19:24.}, qu'à un riche d'entrer dans le Royaume de Dieu.
\VS{26}Les disciples furent encore plus étonnés, et ils se dirent les uns les autres : Et qui peut être sauvé ?
\VS{27}Jésus les ayant regardés, leur dit : cela est impossible quant aux hommes, mais non pas quant à Dieu ; car toutes choses sont possibles à Dieu.
\TextTitle{La fidélité à Jésus-Christ sera récompensée}
\VS{28}Alors Pierre se mit à lui dire : Voici, nous avons tout quitté et nous t'avons suivi.
\VS{29}Et Jésus répondit, disant : Je vous le dis en vérité, il n'est personne qui, ayant quitté pour l'amour de moi et de l'Evangile, sa maison, ou ses frères, ou ses sœurs, ou son père, ou sa mère, ou sa femme, ou ses enfants, ou ses terres,
\VS{30}ne reçoive au centuple, présentement dans ce temps-ci, des maisons, des frères, des sœurs, des mères, des enfants, et des terres, avec des persécutions ; et dans le siècle à venir, la vie éternelle.
\VS{31}Mais plusieurs des premiers seront les derniers ; et plusieurs des derniers seront les premiers.
\TextTitle{Jésus annonce sa mort et sa résurrection\FTNTT{Mt. 20:17-19 ; Lu. 18:31-34}}
\VS{32}Or ils étaient en chemin, montant à Jérusalem, et Jésus allait devant eux. Les disciples étaient troublés, et le suivaient avec crainte. Et Jésus prit de nouveau à l'écart les douze, et commença à leur déclarer ce qui devait lui arriver,
\VS{33}disant : Voici, nous montons à Jérusalem, et le Fils de l'homme sera livré aux principaux sacrificateurs et aux scribes. Ils le condamneront à mort, et le livreront aux Gentils
\VS{34}qui se moqueront de lui, le battront de verges, cracheront sur lui, et le feront mourir ; et il ressuscitera trois jours après.
\TextTitle{Jésus répond à la question de Jacques et Jean}
\VS{35}Alors Jacques et Jean, fils de Zébédée, s'approchèrent de Jésus et lui dirent : Maître, nous voudrions que tu fasses pour nous ce que nous te demanderons.
\VS{36}Il leur dit : Que voulez-vous que je fasse pour vous ?
\VS{37}Et ils lui dirent : Accorde-nous, lui dirent-ils, d'être assis l'un à ta droite et l'autre à ta gauche, quand tu seras dans ta gloire.
\VS{38}Jésus leur dit : Vous ne savez pas ce que vous demandez. Pouvez-vous boire la coupe que je dois boire, et être baptisés du baptême dont je dois être baptisé ?
\VS{39}Ils lui répondirent : Nous le pouvons. Et Jésus leur répondit : Il est vrai que vous boirez la coupe que je dois boire, et que vous serez baptisés du baptême dont je dois être baptisé ;
\VS{40}mais pour ce qui est d'être assis à ma droite et à ma gauche, ce n'est pas à moi de l'accorder ; mais cela ne sera donné qu'à ceux à qui cela est préparé.
\VS{41}Les dix autres, ayant entendu cela, commencèrent à s'indigner contre Jacques et Jean.
\VS{42}Jésus les appela et leur dit : Vous savez que ceux qu'on regarde comme les chefs des nations les dominent, et que les grands les asservissent.
\VS{43}Il n'en sera pas de même parmi vous. Mais quiconque veut être le plus grand parmi vous, qu'il soit votre serviteur,
\VS{44}et quiconque veut être le premier parmi vous, qu'il soit l'esclave de tous.
\VS{45}Car le Fils de l'homme est venu, non pour être servi, mais pour servir et donner sa vie en rançon pour plusieurs.
\TextTitle{Jésus guérit l'aveugle Bartimée\FTNTT{Mt. 20:29-34 ; Lu. 18:35-43}}
\VS{46}Puis ils arrivèrent à Jéricho. Et lorsque Jésus en sortit avec ses disciples et une grande foule, un aveugle, appelé Bartimée, c'est-à-dire le fils de Timée, était assis au bord du chemin et mendiait.
\VS{47}Il entendit que c'était Jésus de Nazareth, et il se mit à crier et à dire : Jésus, Fils de David, aie pitié de moi !
\VS{48}Et plusieurs le reprenaient pour le faire taire ; mais il criait beaucoup plus fort : Fils de David, aie pitié de moi !
\VS{49}Et Jésus s'arrêta, et dit : Appelez-le. Ils appelèrent donc l'aveugle en lui disant : Prends courage, lève-toi, il t'appelle.
\VS{50}Et il jeta son manteau, il se leva et vint vers Jésus.
\VS{51}Jésus, prenant la parole, lui dit : Que veux-tu que je te fasse ? Et l'aveugle lui dit : Maître, que je recouvre la vue.
\VS{52}Et Jésus lui dit : Va, ta foi t'a sauvé.
\VS{53}Et aussitôt il recouvra la vue, et suivit Jésus dans le chemin.
\Chap{11}    
\TextTitle{Entrée de Jésus à Jérusalem\FTNTT{Za. 9:9 ; Mt. 21:1-11 ; Lu. 19:28-40 ; Jn. 12:12-19.}}
\VerseOne{}Lorsqu'ils approchaient de Jérusalem, et qu'ils furent près de Bethphagé et de Béthanie, vers le Mont des Oliviers, Jésus envoya deux de ses disciples,
\VS{2}en leur disant : Allez au village qui est devant vous. Dès que vous y serez entrés, vous trouverez un ânon attaché, sur lequel aucun homme ne s'est encore assis. Détachez-le, et amenez-le.
\VS{3}Si quelqu'un vous dit : Pourquoi faites-vous cela ? Dites que le Seigneur en a besoin ; et à l'instant, il le laissera venir ici.
\VS{4}Ils partirent donc, et trouvèrent l'ânon qui était attaché dehors, près d'une porte, au contour du chemin, et ils le détachèrent.
\VS{5}Et quelques-uns de ceux qui étaient là leur dirent : Pourquoi détachez-vous cet ânon ?
\VS{6}Ils leur répondirent comme Jésus l'avait ordonné ; et on les laissa faire.
\VS{7}Ils amenèrent donc l'ânon à Jésus, sur lequel ils jetèrent leurs vêtements, et Jésus s'assit dessus.
\VS{8}Plusieurs étendirent leurs vêtements sur le chemin, et d'autres des branches qu'ils coupèrent dans les champs.
\VS{9}Ceux qui allaient devant, et ceux qui suivaient, criaient en disant : Hosanna ! Béni soit celui qui vient au Nom du Seigneur !
\VS{10}Béni soit le règne de David notre père, le règne qui vient au Nom du Seigneur ! Hosanna dans les lieux très hauts !
\VS{11}Jésus entra ainsi à Jérusalem, dans le temple. Et après avoir regardé de tous côtés, comme il était déjà tard, il sortit pour aller à Béthanie avec les douze.
\TextTitle{Le figuier sans fruit\FTNTT{Mt. 21:18-22}}
\VS{12}Et le lendemain, après qu'ils furent sortis de Béthanie, Jésus eut faim.
\VS{13}Apercevant de loin un figuier qui avait des feuilles, il alla voir s'il y trouverait quelque chose ; mais s'en étant approché, il ne trouva que des feuilles, car ce n'était pas la saison des figues.
\VS{14}Jésus prenant la parole dit au figuier : Que jamais personne ne mange de ton fruit ! Et les disciples l'entendirent.
\TextTitle{Jésus chasse les marchands du temple\FTNTT{Mt. 21:12-13 ; Lu. 19:45-46 ; Jn. 2:13-17}}
\VS{15}Ils arrivèrent donc à Jérusalem, et Jésus entra dans le temple. Il se mit à chasser dehors ceux qui vendaient, et ceux qui achetaient dans le temple, et il renversa les tables des changeurs, et les sièges de ceux qui vendaient des pigeons.
\VS{16}Il ne laissait personne porter aucun objet à travers le temple.
\VS{17}Et il les enseignait, en leur disant : N'est-il pas écrit : Ma maison sera appelée une maison de prière pour toutes les nations ? Mais vous, vous en avez fait une caverne de voleurs\FTNT{Jé. 7:11.}.
\VS{18}Les scribes et les principaux sacrificateurs l'ayant entendu, cherchèrent les moyens de le détruire ; car ils le craignaient, parce que toute la foule était dans l'étonnement à l'egard de sa doctrine.
\VS{19}Le soir étant venu, il sortit de la ville.
\TextTitle{Prier avec foi\FTNTT{1 Jn. 5:14-15}}
\VS{20}Et le matin, en passant, les disciples virent le figuier séché jusqu'aux racines.
\VS{21}Pierre s'étant souvenu de ce qui s'était passé, dit à Jésus : Maître, voici, le figuier que tu as maudit a séché.
\VS{22}Jésus répondant, leur dit : Ayez foi en Dieu.
\VS{23}Car je vous le dis en vérité, si quelqu'un dit à cette montagne : Quitte ta place, et jette-toi dans la mer, et s'il ne doute pas dans son cœur, mais croit que ce qu'il dit se fera, tout ce qu'il aura dit lui sera fait.
\VS{24}C'est pourquoi je vous dis : Tout ce que vous demanderez en priant, croyez que vous l'avez reçu, et vous le verrez s'accomplir.
\TextTitle{Le pardon}
\VS{25}Mais quand vous vous présenterez pour faire votre prière, si vous avez quelque chose contre quelqu'un, pardonnez-lui, afin que votre Père qui est dans les cieux vous pardonne aussi vos fautes.
\VS{26}Mais si vous ne pardonnez pas, votre Père qui est dans les cieux ne vous pardonnera point aussi vos fautes.
\TextTitle{L'autorité de Jésus-Christ mise en doute\FTNTT{Mt. 21:23-27 ; Lu. 20:1-8}}
\VS{27}Ils se rendirent de nouveau à Jérusalem, et pendant que Jésus marchait dans le temple, les principaux sacrificateurs, les scribes et les anciens vinrent à lui,
\VS{28}et lui dirent : Par quelle autorité fais-tu ces choses, et qui t'a donné cette autorité pour faire les choses que tu fais ?
\VS{29}Jésus leur répondit et leur dit : Je vous demanderai aussi une chose, et répondez-moi ; puis je vous dirai par quelle autorité je fais ces choses.
\VS{30}Le baptême de Jean venait-il du ciel ou des hommes ? Répondez-moi.
\VS{31}Et ils raisonnaient entre eux, disant : Si nous disons : Du ciel, il nous dira : Pourquoi donc n'avez-vous pas cru en lui ?
\VS{32}Et si nous disons : Des hommes, nous avons à craindre le peuple, car tous croyaient que Jean était un vrai prophète.
\VS{33}Alors pour réponse, ils dirent à Jésus : Nous ne savons pas. Et Jésus répondant leur dit : Moi non plus je ne vous dirai pas par quelle autorité je fais ces choses.
\Chap{12}
\TextTitle{Parabole des vignerons\FTNTT{E s. 5:1-7 ; Mt. 21:33-46 ; Lu. 20:9-18.}}
\VerseOne{}Jésus se mit à leur parler en paraboles : Quelqu'un, dit-il, planta une vigne, et l'environna d'une haie, et il y creusa une fosse pour un pressoir, et bâtit une tour ; puis il la loua à des vignerons, et partit pour un pays lointain.
\VS{2}Or au temps de la récolte, il envoya un serviteur vers les vignerons, pour recevoir d'eux le fruit de la vigne.
\VS{3}S'étant saisi de lui, ils le battirent, et le renvoyèrent à vide.
\VS{4}Il envoya encore un autre serviteur vers eux. Ils lui jetèrent des pierres, le frappèrent à la tête, et le renvoyèrent après l'avoir outragé.
\VS{5}Il en envoya de nouveau un troisième, qu'ils tuèrent ; et plusieurs autres, et ils battirent les uns, et tuèrent les autres.
\VS{6}Mais il avait encore un fils, son bien-aimé, il le leur envoya le dernier, disant : Ils auront du respect pour mon fils.
\VS{7}Mais ces vignerons dirent entre eux : Voici l'héritier, venez, tuons-le, et l'héritage sera à nous.
\VS{8}Ils se saisirent de lui, le tuèrent, et le jetèrent hors de la vigne.
\VS{9}Que fera donc le maître de la vigne ? Il viendra, et fera périr ces vignerons, et donnera la vigne à d'autres.
\VS{10}N'avez-vous pas lu cette parole de l'Ecriture ? La pierre qu'ont rejetée ceux qui bâtissaient est devenue la principale de l'angle\FTNT{Jésus-Christ, la pierre angulaire : Es. 8:13-15 ; Ps. 118:22-23.} ?
\VS{11}Cela a été fait par le Seigneur, et c'est une chose merveilleuse à nos yeux.
\VS{12}Alors ils cherchaient à se saisir de lui, mais ils craignirent la foule. Ils avaient compris que c'était contre eux qu'il avait dit cette parabole. C'est pourquoi, ils le laissèrent, et s'en allèrent.
\TextTitle{Le tribut dû à César\FTNTT{Mt. 22:15-22 ; Lu. 20:19-26}}
\VS{13}Puis ils envoyèrent quelques-uns des pharisiens et des hérodiens auprès de lui afin de le surprendre par ses discours.
\VS{14}Et ils vinrent lui dire : Maître, nous savons que tu es véritable, et que tu ne considères personne ; car tu n'as point d'égard à l'apparence des hommes, mais tu enseignes la voie de Dieu selon la vérité ; est-il permis de payer le tribut à César ou non ? Le payerons-nous, ou ne le payerons-nous point ?
\VS{15}Mais Jésus, connaissant leur hypocrisie, leur dit : Pourquoi me tentez-vous ? Apportez-moi un denier afin que je le voie.
\VS{16}Et ils lui en apportèrent un. Alors il leur dit : De qui porte-t-il l'image et l'inscription ? De César, lui répondirent-ils.
\VS{17}Et Jésus répondant leur dit : Rendez à César ce qui est à César, et à Dieu ce qui est à Dieu. Et ils furent remplis d'admiration pour lui.
\TextTitle{Jésus répond aux sadducéens sur la résurrection\FTNTT{Mt. 22:23-33 ; Lu. 20:27-38}}
\VS{18}Alors les sadducéens, qui disent qu'il n'y a point de résurrection, vinrent à lui, et l'interrogèrent, disant :
\VS{19}Maître, voici ce que Moïse nous a prescrit : Si le frère de quelqu'un meurt, et laisse sa femme sans avoir d'enfants, son frère épousera sa veuve et suscitera une postérité à son frère.
\VS{20}Or il y avait sept frères dont le premier prit une femme et mourut sans laisser d'enfants.
\VS{21}Or le deuxième prit la veuve pour femme, et mourut sans laisser de postérité. Il en fut de même du troisième,
\VS{22}et les sept l'épousèrent sans laisser de postérité. Après eux tous, la femme mourut aussi.
\VS{23}A la résurrection donc, quand ils seront ressuscités, duquel d'entre eux sera-t-elle la femme ? Car les sept l'ont eue pour femme.
\VS{24}Et Jésus leur répondit : La raison pour laquelle vous tombez dans l'erreur, c'est que vous ne connaissez ni les Ecritures ni la puissance de Dieu.
\VS{25}Car à la résurrection des morts, ils ne prendront point de femmes, et on ne leur donnera point de femmes en mariage, mais ils seront comme les anges qui sont dans les cieux.
\VS{26}Et quant aux morts, pour vous montrer qu'ils ressuscitent, n'avez-vous point lu dans le livre de Moïse, comment Dieu lui parla dans le buisson, en disant : Je suis le Dieu d'Abraham, et le Dieu d'Isaac, et le Dieu de Jacob ?
\VS{27}Or il n'est pas le Dieu des morts, mais le Dieu des vivants. Vous êtes donc dans une grande erreur.
\TextTitle{Aimer son prochain, le plus grand des commandements\FTNTT{Mt. 22:34-40 ; Lu. 10:25-28}}
\VS{28}Un des scribes, qui les avait entendus discuter, voyant qu'il leur avait bien répondu, s'approcha de lui, et lui demanda : Quel est le premier de tous les commandements ?
\VS{29}Jésus lui répondit : Le premier de tous les commandements est : Ecoute Israël\FTNT{Ecoute Israël : Jésus se réfère ici à De. 6:4 : « Ecoute, Israël ! Yahweh, notre Dieu Yahweh est Un ». Le Shema Israël est le noyau central de la prière que le Juif adulte doit lire matin et soir. C'est la confession de foi juive. Jacob est le premier à l'avoir enseignée à ses enfants dans Ge. 49:1-2.}, le Seigneur notre Dieu, le Seigneur est Un\FTNT{Jésus-Christ, notre Seigneur et notre modèle, a confirmé le Shema Israël qui déclare haut et fort que Dieu est Un et non trois en un. Le scribe, homme versé dans les Ecritures, était satisfait de la réponse de Jésus car il croyait aussi en un seul Dieu. Or le monothéisme est le fondement de la foi juive et des premiers chrétiens.}.
\VS{30}Et tu aimeras le Seigneur ton Dieu de tout ton cœur, de toute ton âme, de toute ta pensée, et de toute ta force. C'est là le premier commandement.
\VS{31}Et le second, qui est semblable au premier est celui-ci : Tu aimeras ton prochain comme toi-même. Il n'y a pas d'autre commandement plus grand que ceux-ci.
\VS{32}Et le scribe lui dit : Maître, tu as bien dit selon la vérité, qu'il y a un seul Dieu, et qu'il n'y en a point d'autre que lui ;
\VS{33}et que l'aimer de tout son cœur, de toute son intelligence, de toute son âme, et de toute sa force ; et d'aimer son prochain comme soi-même, c'est plus que tous les holocaustes et les sacrifices.
\VS{34}Et Jésus voyant que ce scribe avait répondu prudemment, lui dit : Tu n'es pas loin du Royaume de Dieu. Et personne n'osait plus l'interroger.
\TextTitle{Jésus dénonce les scribes\FTNTT{Mt. 22:41-46 ; Lu. 20:39-44}}
\VS{35}Comme Jésus enseignait dans le temple, il prit la parole et dit : Comment les scribes disent-ils que le Christ est le Fils de David ?
\VS{36}Car David lui-même a dit par le Saint-Esprit : Le Seigneur a dit à mon Seigneur : Assieds-toi à ma droite, jusqu'a ce que j'aie mis tes ennemis pour le marchepied de tes pieds.
\FTNT{Ps. 110:1.}.
\VS{37}David lui-même l'appelle son Seigneur, comment est-il son fils ? Et une grande foule l'écoutait avec plaisir.
\VS{38}Il leur disait dans son enseignement : Gardez-vous des scribes qui prennent plaisir à se promener en robes longues, et qui aiment les salutations dans les places publiques,
\VS{39}qui recherchent les premiers sièges dans les synagogues, et les premières places dans les festins ;
\VS{40}qui dévorent entièrement les maisons des veuves, même sous le prétexte de faire de longues prières. Ils seront jugés plus sévèrement.
\TextTitle{L'offrande de la pauvre veuve\FTNTT{Lu. 21:1-4}}
\VS{41}Jésus, s'étant assis vis-à-vis du tronc, regardait comment la foule y mettait de l'argent. Plusieurs riches y mettaient beaucoup.
\VS{42}Et une pauvre veuve vint, elle y mit deux petites pièces, faisant le quart d'un sou.
\VS{43}Et Jésus, ayant appelé ses disciples, leur dit : Je vous le dis en vérité, cette pauvre veuve a plus mis dans le tronc que tous ceux qui y ont mis.
\VS{44}Car tous ont mis de leur superflu ; mais celle-ci a mis de son nécessaire, tout ce qu'elle possédait, tout ce qu'elle avait pour vivre.
\TextTitle{Prophétie sur la destruction du temple\FTNTT{Mt. 24:3 ; Lu. 21:7.}}
\Chap{13}
\VerseOne{}Comme il sortit du temple, un de ses disciples lui dit : Maître, regarde quelles pierres et quels bâtiments !
\VS{2}Et Jésus répondant lui dit : Vois-tu ces grands bâtiments ? Il ne restera pas pierre sur pierre qui ne soit démolie.
\TextTitle{Les temps de la fin}
\VS{3}Comme il s'assit sur le Mont des Oliviers, en face du temple, Pierre, Jacques, Jean et André, l'interrogèrent en particulier,
\VS{4}disant : Dis-nous quand ces choses arriveront, et quel signe il y aura quand toutes ces choses devront s'accomplir ?
\VS{5}Et Jésus leur répondant, se mit à leur dire : Prenez garde que personne ne vous séduise.
\VS{6}Car plusieurs viendront en mon Nom, disant : C'est moi qui suis le Christ. Et ils séduiront beaucoup de gens.
\VS{7}Or quand vous entendrez parler de guerres et des bruits de guerres, ne soyez point troublés ; parce qu'il faut que ces choses arrivent ; mais ce ne sera pas encore la fin.
\VS{8}Car une nation s'élèvera contre une autre nation, et un royaume contre un autre royaume ; et il y aura des tremblements de terre en divers lieux, et il y aura des famines et des troubles. Ces choses ne seront que les premières douleurs.
\VS{9}Mais prenez garde à vous-mêmes. Car ils vous livreront aux tribunaux, et aux synagogues, vous serez battus de verges et vous serez présentés devant les gouverneurs et devant les rois, à cause de moi, pour leur servir de témoignage.
\VS{10}Mais il faut premièrement que l'Evangile soit prêché à toutes les nations.
\VS{11}Et quand ils vous emmèneront pour vous livrer, ne soyez point en peine par avance de ce que vous aurez à dire, et ne le méditez point ; mais dites ce qui vous sera donné en ce moment-là ; car ce ne sera pas vous qui parlerez, mais le Saint-Esprit.
\VS{12}Or le frère livrera son frère à la mort, et le père son enfant ; et les enfants se soulèveront contre leurs pères et leurs mères, et les feront mourir.
\VS{13}Vous serez haïs de tous à cause de mon Nom ; mais celui qui persévérera jusqu'à la fin, celui-là sera sauvé.
\TextTitle{L'abomination de la désolation\FTNTT{Ps. 2.5 ; Mt. 24:15-28 ; Lu. 21:20-24 ; Ap. 7:14}}
\VS{14}Or lorsque vous verrez l'abomination qui cause la désolation\FTNT{Voir commentaire Mt. 24:15.} qui a été prédite par Daniel, le prophète, être établie là où elle ne doit pas être, que celui qui lit ce prophète fasse attention ! Alors que ceux qui seront en Judée fuient dans les montagnes.
\VS{15}Que celui qui sera sur le toit ne descende pas dans la maison, et n'entre pas pour emporter quoi que ce soit de sa maison,
\VS{16}et que celui qui sera dans les champs ne retourne pas en arrière pour emporter son manteau.
\VS{17}Mais malheur à celles qui seront enceintes, et à celles qui allaiteront en ces jours-là.
\VS{18}Priez Dieu que votre fuite n'arrive pas en hiver.
\VS{19}Car la détresse, en ces jours, sera telle qu'il n'y en a point eu de semblable depuis le commencement du monde que Dieu a créé jusqu'à présent, et qu'il n'y en aura jamais qui l'égale.
\VS{20}Et si le Seigneur n'avait abrégé ces jours, il n'y aurait personne de sauvé ; mais il a abrégé ces jours, à cause des élus qu'il a élus.
\VS{21}Et alors si quelqu'un vous dit : Voici, le Christ est ici ; ou voici, il est là, ne le croyez point.
\VS{22}Car il s'élèvera des faux christs et des faux prophètes, qui feront des prodiges et des miracles pour séduire même les élus, s'il était possible.
\VS{23}Mais soyez sur vos gardes ; voici, je vous ai tout annoncé d'avance.
\TextTitle{Retour du Messie sur la terre\FTNTT{Mt. 24:29-31 ; Lu. 21:25-28}}
\VS{24}Or dans ces jours, après cette détresse, le soleil s'obscurcira, et la lune ne donnera plus sa clarté ;
\VS{25}les étoiles du ciel tomberont, et les puissances qui sont dans les cieux seront ébranlées.
\VS{26}Et ils verront alors le Fils de l'homme venant sur les nuées, avec une grande puissance et une grande gloire.
\VS{27}Alors il enverra ses anges, et il rassemblera ses élus des quatre vents, de l'extrémité de la terre jusqu'à l'extrémité du ciel.
\TextTitle{Parabole du figuier\FTNTT{Mt. 24:32-35 ; Lu. 21:29-33}}
\VS{28}Mais apprenez la leçon tirée de la parabole du figuier. Dès que ses jeunes branches deviennent tendres, et que ses feuilles poussent, vous savez que l'été est proche.
\VS{29}Ainsi, quand vous verrez ces choses arriver, sachez que le Fils de l'homme est proche, à la porte.
\VS{30}Je vous le dis en vérité, cette génération ne passera point, que toutes ces choses ne soient arrivées.
\VS{31}Le ciel et la terre passeront, mais mes paroles ne passeront point.
\TextTitle{Exhortation de Jésus sur la vigilance\FTNTT{Mt. 24:36-51 ; Lu. 21:34-38}}
\VS{32}Or pour ce qui est du jour ou de l'heure, personne ne le sait, ni les anges dans le ciel, ni le Fils\FTNT{Comment expliquer l'ignorance du Fils quant à l'heure de son retour ? En prenant la condition d'un homme, Jésus s'est dépouillé de ses prérogatives divines et a connu des limites propres au genre humain (Ph. 2:7) : la fatigue (Jn. 4:6 ; Mc. 4:38), la faim (Mc. 11:12), l'angoisse et la peur (Mc. 14:33), la mortalité physique… Ce dépouillement incluait le renoncement à l'omniscience, d'où le fait que Jésus-Christ homme ne connaissait pas le jour et l'heure de son retour.}, mais mon Père seul.
\VS{33}Faites attention à tout, veillez et priez ; car vous ne savez quand ce temps viendra.
\VS{34}Il en sera comme d'un homme qui, partant pour un voyage, laisse sa maison, remet l'autorité à ses serviteurs, marquant à chacun sa tâche, et ordonne au portier de veiller.
\VS{35}Veillez donc, car vous ne savez quand le Maître de la maison viendra, ou le soir, ou à minuit, ou à l'heure où le coq chante, ou le matin ;
\VS{36}craignez qu'il ne vous trouve endormis à son arrivée soudaine.
\VS{37}Or ce que je vous dis, je le dis à tous : Veillez.
\Chap{14}
\TextTitle{Le complot\FTNTT{Mt. 26:1-5 ; Lu. 22:1-2}}
\VerseOne{}Or la fête de Pâque et des pains sans levain devait avoir lieu deux jours après. Les principaux sacrificateurs et les scribes cherchaient les moyens de se saisir de Jésus par ruse, et de le faire mourir.
\VS{2}Mais ils disaient : Que ce ne soit pas pendant la fête, afin qu'il n'y ait pas de tumulte parmi le peuple.
\TextTitle{Jésus oint par Marie de Béthanie\FTNTT{Mt. 26:6-13 ; Jn. 12:1-8}}
\VS{3}Et comme il était à Béthanie, dans la maison de Simon le lépreux, et pendant qu'il était à table, une femme vint à lui avec un vase d'albâtre, rempli d'un parfum de nard pur et de grand prix ; et ayant rompu le vase, elle répandit le parfum sur la tête de Jésus.
\VS{4}Quelques-uns en furent indignés en eux-mêmes, et ils disaient : A quoi sert la perte de ce parfum ?
\VS{5}Car on aurait pu le vendre plus de trois cents deniers, et les donner aux pauvres. Ainsi ils murmuraient contre elle.
\VS{6}Mais Jésus dit : Laissez-la. Pourquoi lui faites-vous de la peine ? Elle a fait une bonne action à mon égard.
\VS{7}Parce que vous aurez toujours des pauvres avec vous, et vous pouvez leur faire du bien quand vous voulez ; mais vous ne m'aurez pas toujours.
\VS{8}Elle a fait ce qu'elle a pu ; elle a d'avance embaumé mon corps pour la sépulture.
\VS{9}Je vous le dis en vérité, partout où cet Evangile sera prêché, dans le monde entier, on racontera aussi en mémoire de cette femme ce qu'elle a fait.
\TextTitle{La trahison de Judas\FTNTT{Mt. 26:14-16 ; Lu. 22:3-6}}
\VS{10}Alors Judas Iscariot, l'un des douze, alla vers les principaux sacrificateurs pour le livrer.
\VS{11}Après l'avoir entendu, ils furent dans la joie, et promirent de lui donner de l'argent. Et Judas cherchait une occasion favorable pour le livrer.
\TextTitle{La dernière Pâque\FTNTT{Mt. 26:17-25 ; Lu. 22:7-20 ; Jn. 13:1-12}}
\VS{12}Or le premier jour des pains sans levain, où l'on sacrifiait l'agneau de Pâque, ses disciples lui dirent : Où veux-tu que nous allions te préparer l'agneau de Pâque afin que tu manges ?
\VS{13}Et il envoya deux de ses disciples, et leur dit : Allez dans la ville, vous rencontrerez un homme portant une cruche d'eau, suivez-le.
\VS{14}Où qu'il entre, dites au maître de la maison : Le Maître dit : Où est le lieu où je mangerai l'agneau de Pâque avec mes disciples ?
\VS{15}Et il vous montrera une grande chambre haute, meublée et toute prête : C'est là que vous nous préparerez l'agneau de Pâque.
\VS{16}Ses disciples partirent, arrivèrent dans la ville, et ils trouvèrent les choses comme il l'avait dit ; et ils apprêtèrent l'agneau de Pâque.
\VS{17}Et le soir étant venu, Jésus arriva avec les douze.
\VS{18}Pendant qu'ils étaient à table, et qu'ils mangeaient, Jésus leur dit : Je vous le dis en vérité, l'un de vous, qui mange avec moi, me trahira.
\VS{19}Ils commencèrent à s'attrister, et ils lui dirent l'un après l'autre : Est-ce moi ? Et l'autre : Est-ce moi ?
\VS{20}Mais il répondit, et leur dit: C'est l'un des douze qui trempe avec moi dans le plat.
\VS{21}Certes le Fils de l'homme s'en va, selon qu'il est écrit de lui. Mais malheur à l'homme par qui le Fils de l'homme est trahi ! Mieux vaudrait pour cet homme qu'il ne soit pas né.
\TextTitle{Le repas de la Pâque\FTNTT{Mt. 26:26-29 ; Lu. 22:17-20 ; Jn. 13:12-30 ; 1 Co. 11:23-26}}
\VS{22}Pendant qu'ils mangeaient, Jésus prit du pain, et après avoir béni Dieu, il le rompit et le leur donna, et leur dit : Prenez, mangez, ceci est mon corps.
\VS{23}Puis il prit ensuite une coupe, et après avoir rendu grâces, il la leur donna, et ils en burent tous.
\VS{24}Et il leur dit : Ceci est mon sang\FTNT{Nouvelle Alliance : Voir Jn. 19:30.}, le sang de la Nouvelle Alliance, qui est répandu pour plusieurs.
\VS{25}Je vous le dis en vérité, je ne boirai plus du fruit de la vigne jusqu'au jour où j'en boirai du nouveau dans le Royaume de Dieu.
\TextTitle{Jésus avertit Pierre de son triple reniement\FTNTT{Mt. 26:30-35 ; Lu. 22:31-34 ; Jn. 13:36-38}}
\VS{26}Et après avoir chanté le cantique\FTNT{Cantique : Voir Mt. 26:30.}, ils se rendirent à la Montagne des Oliviers.
\VS{27}Jésus leur dit : Vous serez tous cette nuit scandalisés en moi ; car il est écrit : Je frapperai le Berger, et les brebis seront dispersées\FTNT{Za. 13:7.}.
\VS{28}Mais après que je serai ressuscité, je vous précéderai en Galilée.
\VS{29}Et Pierre lui dit : Quand même tous seraient scandalisés, je ne le serai pourtant pas moi.
\VS{30}Et Jésus lui dit : Je te le dis en vérité, qu'aujourd'hui, cette nuit même, avant que le coq chante deux fois, tu me renieras trois fois.
\VS{31}Mais Pierre disait encore plus fortement : Quand même il me faudrait mourir avec toi, je ne te renierai pas. Et tous lui dirent la même chose.
\TextTitle{Jésus dans le jardin de Gethsémané\FTNTT{Mt. 26:36-46 ; Lu. 22:39-46 ; Jn. 18:1}}
\VS{32}Ils allèrent dans un lieu appelé Gethsémané, et Jésus dit à ses disciples : Asseyez-vous ici jusqu'à ce que j'aie prié.
\VS{33}Il prit avec lui Pierre, Jacques et Jean, et il commença à être effrayé et fort agité.
\VS{34}Il leur dit : Mon âme est saisie de tristesse jusqu'à la mort, restez ici, et veillez.
\TextTitle{Première prière de Jésus\FTNTT{Mt. 26:39 ; Lu. 22:41-42}}
\VS{35}Puis s'en allant un peu plus en avant, il se jeta contre terre, et pria que s'il était possible, cette heure s'éloigne de lui.
\VS{36}Il disait : Abba, Père, toutes choses te sont possibles, éloigne de moi cette coupe ! Toutefois, non pas ce que je veux, mais ce que tu veux.
\VS{37}Puis il revint vers les disciples qu'il trouva endormis, et il dit à Pierre : Simon, tu dors ! Tu n'as pas pu veiller une heure !
\VS{38}Veillez et priez afin que vous ne tombiez pas en tentation, car quant à l'esprit, il est prompt, mais la chair est faible.
\TextTitle{Deuxième prière\FTNTT{Mt. 26:42 ; Lu. 22:44}}
\VS{39}Il s'éloigna de nouveau, et fit la même prière, disant les mêmes paroles.
\VS{40}Puis il revint, et les trouva encore endormis, car leurs yeux étaient appesantis. Ils ne surent que lui répondre.
\TextTitle{Troisième prière\FTNTT{Mt. 26:44}}
\VS{41}Il revint encore, pour la troisième fois, et leur dit : Dormez maintenant, et reposez-vous ! C'est assez ! L'heure est venue ; voici, le Fils de l'homme s'en va être livré entre les mains des méchants.
\VS{42}Levez-vous, allons ; voici, celui qui me trahit s'approche.
\TextTitle{Jésus trahi, abandonné et arrêté\FTNTT{Mt. 26:47-56 ; Lu. 22:47-53 ; Jn. 18:2-11}}
\VS{43}Et aussitôt, comme il parlait encore, Judas, l'un des douze, vint, et avec lui une grande foule ayant des épées et des bâtons, envoyée par les principaux sacrificateurs, par les scribes et par les anciens.
\VS{44}Or celui qui le trahissait leur avait donné ce signe : Celui que j'embrasserai, c'est lui ; saisissez-le, et emmenez-le sûrement.
\VS{45}Dès qu'il fut arrivé, il s'approcha aussitôt de lui, et lui dit : Maître, Maître ! Et il le baisa.
\VS{46}Alors ils mirent la main sur Jésus, et le saisirent.
\VS{47}Un de ceux qui étaient là présents, tirant son épée, frappa le serviteur du souverain sacrificateur et lui emporta l'oreille.
\VS{48}Alors Jésus prit la parole, et leur dit : Vous êtes venus comme après un brigand, avec des épées et des bâtons, pour m'arrêter.
\VS{49}J'étais tous les jours parmi vous, enseignant dans le temple, et vous ne m'avez point saisi ; mais tout ceci est arrivé afin que les Ecritures soient accomplies.
\VS{50}Alors tous ses disciples l'abandonnèrent et s'enfuirent.
\VS{51}Et un certain jeune homme le suivait, enveloppé d'un linceul sur le corps nu ; et quelques jeunes gens le saisirent.
\VS{52}Mais il abandonna son linceul, et se sauva tout nu.
\TextTitle{Jésus devant Caïphe et le sanhédrin\FTNTT{Mt. 26:57-68 ; Jn. 18:12-14,19-24}}
\VS{53}Et ils emmenèrent Jésus chez le souverain sacrificateur, où s'assemblèrent tous les principaux sacrificateurs, les anciens et les scribes.
\VS{54}Pierre le suivait de loin jusque dans la cour du souverain sacrificateur ; et il était assis avec les serviteurs, et se chauffait près du feu.
\VS{55}Les principaux sacrificateurs et tout le sanhédrin cherchaient quelque témoignage contre Jésus pour le faire mourir, mais ils n'en trouvaient point.
\VS{56}Car plusieurs rendaient de faux témoignages contre lui, mais les témoignages n'étaient point suffisants.
\VS{57}Alors quelques-uns s'élevèrent, et portèrent de faux témoignages contre lui, disant :
\VS{58}Nous l'avons entendu dire : Je détruirai ce temple qui est fait de main, et en trois jours, j'en rebâtirai un autre qui ne sera pas fait de main d'homme.
\VS{59}Même sur ce point-là, leurs témoignages n'étaient point suffisants.
\VS{60}Alors le souverain sacrificateur se levant au milieu, interrogea Jésus, disant : Ne réponds-tu rien ? Qu'est-ce que ces gens déposent contre toi ?
\VS{61}Mais il garda le silence, et ne répondit rien. Le souverain sacrificateur l'interrogea de nouveau, et lui dit : Es-tu le Christ, le Fils du Dieu béni ?
\VS{62}Et Jésus lui répondit : Je le suis. Et vous verrez le Fils de l'homme assis à la droite de la puissance de Dieu, et venant sur les nuées du ciel.
\VS{63}Alors le souverain sacrificateur déchira ses vêtements et dit : Qu'avons-nous encore besoin de témoins ?
\VS{64}Vous avez entendu le blasphème. Que vous en semble ? Alors tous le condamnèrent comme étant digne de mort.
\VS{65}Et quelques-uns se mirent à cracher sur lui, à lui voiler le visage, et à lui donner des soufflets, en lui disant : Prophétise ! Et les serviteurs lui donnaient des coups avec leurs verges.
\TextTitle{Triple reniement de Pierre\FTNTT{Mt. 26:69-75 ; Lu. 22:54-62 ; Jn. 18:15-18,25-27}}
\VS{66}Or pendant que Pierre était en bas dans la cour, une des servantes du souverain sacrificateur vint.
\VS{67}Apercevant Pierre qui se chauffait, elle le regarda en face, et lui dit : Toi aussi, tu étais avec Jésus de Nazareth.
\VS{68}Mais il le nia, disant : Je ne le connais pas, et je ne sais pas ce que tu dis ; puis il sortit dehors pour aller dans le vestibule. Et le coq chanta.
\VS{69}Et la servante l'ayant vu de nouveau, elle se mit à dire à ceux qui étaient là présents : Celui-ci est de ces gens-là. Et il le nia une seconde fois.
\VS{70}Peu après, ceux qui étaient là présents dirent à Pierre : Certainement tu es de ces gens-là, car tu es Galiléen, et ton langage s'y rapporte.
\VS{71}Alors il se mit à se maudire, et à jurer, disant : je ne connais point cet homme-là dont vous parlez.
\VS{72}Et le coq chanta pour la seconde fois. Et Pierre se souvint de cette parole que Jésus lui avait dite : Avant que le coq chante deux fois, tu me renieras trois fois. Et étant sorti promptement, il pleura.
\Chap{15}
\TextTitle{Jésus livré à Pilate\FTNTT{Mt. 27:1-2,11-15 ; Lu. 23:1-7,13-16 ; Jn. 18:28-38 ; 19:1-15}}
\VerseOne{}Et dès le matin, les principaux sacrificateurs tinrent conseil avec les anciens et les scribes, et tout le sanhédrin. Après avoir lié Jésus, ils l'emmenèrent, et le livrèrent à Pilate.
\VS{2}Et Pilate l'interrogea, disant : Es-tu le roi des Juifs ? Et Jésus répondant, lui dit : Tu le dis.
\VS{3}Les principaux sacrificateurs l'accusaient de plusieurs choses, mais il ne répondit rien.
\VS{4}Pilate l'interrogea de nouveau : Ne réponds-tu rien ? Vois de combien de choses ils t'accusent.
\VS{5}Mais Jésus ne donna plus aucune réponse, ce qui étonna Pilate.
\TextTitle{Jésus ou Barabbas ?\FTNTT{Mt. 27:15-26 ; Lu. 23:17-25 ; Jn. 18:39-40}}
\VS{6}Or à chaque fête, il relâchait un prisonnier, celui que demandait la foule.
\VS{7}Et il y en avait un, nommé Barabbas, qui était prisonnier avec ses complices pour une sédition, dans laquelle ils avaient commis un meurtre.
\VS{8}La foule se mit à demander à Pilate, avec de grands cris, ce qu'il avait coutume de leur accorder.
\VS{9}Mais Pilate leur répondit en disant : Voulez-vous que je vous relâche le Roi des Juifs ?
\VS{10}Car il savait bien que les principaux sacrificateurs l'avaient livré par envie.
\VS{11}Mais les principaux sacrificateurs excitèrent la foule, afin que Pilate leur relâche plutôt Barabbas.
\VS{12}Et Pilate reprenant la parole, leur dit encore : Que voulez-vous donc que je fasse de celui que vous appelez Roi des Juifs ?
\VS{13}Ils s'écrièrent de nouveau : Crucifie-le !
\VS{14}Alors Pilate leur dit : Mais quel mal a-t-il fait ? Et ils s'écrièrent encore plus fort : Crucifie-le !
\VS{15}Pilate, voulant satisfaire la foule, leur relâcha Barabbas ; et après avoir fait battre de verges Jésus, il le livra pour être crucifié.
\TextTitle{Jésus couronné d'épines\FTNTT{Mt. 27:27-31 ; Jn. 19:1-3}}
\VS{16}Alors les soldats emmenèrent Jésus dans l'intérieur de la cour, c'est-à-dire dans le prétoire, et ils assemblèrent toute la cohorte.
\VS{17}Ils le revêtirent d'une robe de pourpre, et posèrent sur sa tête une couronne d'épines qu'ils avaient tressée.
\VS{18}Puis ils commencèrent à le saluer, en lui disant : Nous te saluons, Roi des Juifs !
\VS{19}Et ils lui frappaient la tête avec un roseau, et crachaient contre lui, et fléchissant les genoux, ils se prosternaient devant lui.
\VS{20}Et après s'être ainsi moqués de lui, ils le dépouillèrent de la robe de pourpre, lui remirent ses habits, et l'emmenèrent dehors pour le crucifier.
\VS{21}Et un certain homme de Cyrène, nommé Simon, père d'Alexandre et de Rufus, passant par là en revenant des champs, fut forcé à porter la croix de Jésus.
\VS{22}Et ils le conduisirent au lieu appelé Golgotha\FTNT{Golgotha : Le Golgotha (crâne) était une colline située à l'extérieur de Jérusalem, sur laquelle les Romains crucifiaient les condamnés.}, c'est-à-dire, le lieu du Crâne.
\VS{23}Ils lui donnèrent à boire du vin mêlé de myrrhe, mais il ne le prit pas.
\TextTitle{La crucifixion de Jésus\FTNTT{Mt. 27:33-56 ; Lu. 23:33-49 ; Jn. 19:16-37}}
\VS{24}Ils le crucifièrent, et se partagèrent ses vêtements en tirant au sort pour savoir ce que chacun aurait.
\VS{25}Or c'était la troisième heure, quand ils le crucifièrent.
\VS{26}Et l'écriteau indiquant la cause de sa condamnation portait ces mots : LE ROI DES JUIFS.
\VS{27}Ils crucifièrent aussi avec lui deux brigands, l'un à sa main droite, et l'autre à sa gauche.
\VS{28}Et ainsi fut accomplie l'Ecriture, qui dit : Et il a été mis au rang des malfaiteurs\FTNT{Es. 53:12.}.
\VS{29}Et les passants l'injuriaient, et secouaient la tête, en disant : Hé ! Toi qui détruis le temple et qui le rebâtis en trois jours,
\VS{30}sauve-toi toi-même, et descends de la croix !
\VS{31}Les principaux sacrificateurs aussi avec les scribes se moquaient entre eux, et disaient : Il a sauvé les autres, et il ne peut se sauver lui-même.
\VS{32}Que le Christ, le Roi d'Israël, descende maintenant de la croix, afin que nous le voyions et que nous croyions ! Ceux qui étaient crucifiés avec lui l'insultaient aussi.
\VS{33}La sixième heure étant venue, il y eut des ténèbres sur toute la terre jusqu'à la neuvième heure.
\VS{34}Et à la neuvième heure, Jésus s'écria d'une voix forte : Eloï, Eloï, lama sabachthani ? C'est-à-dire : Mon Dieu ! Mon Dieu ! Pourquoi m'as-tu abandonné ?
\VS{35}Quelques-uns de ceux qui étaient là présents, l'ayant entendu, dirent : Voici, il appelle Elie.
\VS{36}Et l'un d'eux courut remplir une éponge de vinaigre\FTNT{Le vinaigre : Voir Mt. 27:34.}, et l'ayant fixée au bout d'un roseau, il lui donna à boire, en disant : Laissez, voyons si Elie viendra le descendre de la croix.
\VS{37}Mais Jésus, ayant poussé un grand cri, rendit l'esprit.
\VS{38}Et le voile du temple se déchira en deux, depuis le haut jusqu'en bas\FTNT{Hé 10:19-20.}.
\TextTitle{Fin de la Première Alliance\FTNTT{Hé. 9:16-18}}
\VS{39}Et le centenier, qui était en face de lui, voyant qu'il avait rendu l'esprit en criant de la sorte, dit : Certainement cet homme était Fils de Dieu.
\VS{40}Il y avait là aussi des femmes qui regardaient de loin. Parmi elles, étaient Marie de Madgala, Marie mère de Jacques le mineur et de Joses, et Salomé,
\VS{41}qui le suivaient et le servaient lorsqu'il était en Galilée, et plusieurs autres qui étaient montées avec lui à Jérusalem.
\TextTitle{Jésus enseveli}
\VS{42}Et le soir étant venu, comme c'était la préparation, c'est-à-dire le sabbat,
\VS{43}arriva Joseph d'Arimathée, conseiller honorable, qui attendait aussi le Royaume de Dieu. Il osa se rendre vers Pilate pour demander le corps de Jésus.
\VS{44}Et Pilate s'étonna qu'il soit déjà mort ; il fit venir le centenier, et lui demanda s'il était mort depuis longtemps.
\VS{45}S'en étant assuré par le centenier, il donna le corps à Joseph.
\VS{46}Et Joseph ayant acheté un linceul, descendit Jésus de la croix, et l'enveloppa du linceul, et le déposa dans un sépulcre taillé dans le roc. Puis il roula une pierre sur l'entrée du sépulcre.
\VS{47}Et Marie de Magdala, et Marie, mère de Joses, regardaient où on le mettait.
\Chap{16}
\TextTitle{Jésus, ressuscité, apparaît à plusieurs disciples\FTNTT{Mt. 28:1-15 ; Lu. 24:1-49 ; Jn. 20:1-23}}
\VerseOne{}Or lorsque le sabbat fut passé, Marie de Magdala, Marie mère de Jacques, et Salomé, achetèrent des aromates pour venir l'embaumer.
\VS{2}Et le premier jour de la semaine, de grand matin, elles se rendirent au sépulcre, comme le soleil venait de se lever.
\VS{3}Elles disaient entre elles : Qui nous roulera la pierre de l'entrée du sépulcre ?
\VS{4}Et levant les yeux, elles virent que la pierre, qui était très grande, avait été roulée.
\VS{5}Puis elles entrèrent dans le sépulcre, elles virent un jeune homme assis à droite, vêtu d'une robe blanche, et elles furent épouvantées.
\VS{6}Mais il leur dit : Ne vous épouvantez pas. Vous cherchez Jésus de Nazareth qui a été crucifié. Il est ressuscité, il n'est point ici ; voici le lieu où on l'avait mis.
\VS{7}Mais allez, et dites à ses disciples, et à Pierre, qu'il vous précède en Galilée. C'est là que vous le verrez, comme il vous l'a dit.
\VS{8}Elles partirent aussitôt et s'enfuirent du sépulcre. La peur et le trouble les avaient saisies ; et elles ne dirent rien à personne, à cause de la peur.
\VS{9}Jésus étant ressuscité, le matin du premier jour de la semaine\FTNT{Pour beaucoup, Jésus-Christ serait mort le vendredi soir et ressuscité le dimanche matin. Cette théorie ne tient cependant pas lorsqu’elle est confrontée au récit des évangiles. Le Seigneur a déclaré qu’il resterait « trois jours et trois nuits dans le cœur de la terre » (Mt. 12:40), et qu’il ressusciterait « après trois jours » (Mc. 8 :31). Or de toute évidence, si Jésus est mort le vendredi pour ressusciter le dimanche matin, cela ne fait pas trois jours et trois nuits. Les Ecritures ne précisent pas quel jour le Seigneur est mort, mais elles nous donnent quelques indices. Tout d’abord, il convient de signaler que chez les Hébreux (selon Ge. 1), le jour commence au coucher du soleil, aux environs de dix-huit heures, et s’achève le lendemain au coucher du soleil. Chez les Romains, le jour commence à minuit et se termine le lendemain à minuit. C’est de cette manière que l’évangile de Jean compte les heures. Dans les autres évangiles, les journées commencent avec le lever du soleil. Nous savons que Jésus a été crucifié à « la troisième heure » (Mc. 15:25), ce qui correspond à 9 heures du matin. Ensuite, il est précisé qu’il y a eu des ténèbres sur la terre de la sixième à la neuvième heure, donc de midi à 15 heures (Mt. 27:45-46 ; Mc. 15:33-34 ; Lu. 23:44). Jésus est donc mort avant dix-huit heures. Ainsi, il est évident qu’il n’a pas pu passer toute la journée du vendredi au tombeau. Nous savons aussi que Jésus a été crucifié « la veille du sabbat » (Mc. 15 :42), ce qui pourrait donner raison à la théorie selon laquelle il serait mort le vendredi. Or les Hébreux ont des sabbats hebdomadaires (le samedi) et des grands sabbats annuels, qui correspondent aux fêtes de Yahweh (Lé. 23). Ainsi, le sabbat en question était sans doute la Pâque (Mt. 26 ; Mc. 14 ; Lu. 22). Selon toute vraisemblance, cette semaine-là il y a eu deux sabbats : le sabbat de Pâque et le sabbat hebdomadaire. Le premier sabbat, celui de la Pâque, est relaté en Mc. 16 :1, où nous apprenons que des femmes sont allées acheter des aromates pour embaumer le corps du Seigneur « lorsque le sabbat fut passé ». Le deuxième sabbat était hebdomadaire ; il est relaté en Lu. 23 :54-56 où il est dit que « la veille du sabbat », les femmes préparèrent des aromates et des parfums, et qu’ensuite elles se reposèrent. Or il est impossible qu’elles aient acheté les aromates après le sabbat et qu’elles les aient préparées avant, à moins qu’il n’y ait eu deux sabbats cette semaine-là.  Ainsi, pour que le Seigneur puisse effectivement passer trois jours et trois nuits dans le cœur de la terre (Mt. 12 :40), il a dû être arrêté la nuit de mardi à mercredi à Getsémané (en sachant que pour les Hébreux, le mardi à 18 heures correspond au début du mercredi). Il comparut devant le sanhédrin le mercredi matin, et à 9 heures il fut crucifié. Il resta à la croix six heures (de 9 heures à 15 heures) et il fut mis au tombeau le soir, vers 18 heures (ce qui correspond au début du jeudi pour les Hébreux ; Mt. 27 :57-60). Nous avons donc :
-Du mercredi soir au jeudi soir : un jour et une nuit.
-Du jeudi soir au vendredi soir : deux jours et deux nuits.
-Du vendredi soir au samedi soir : trois jours et trois nuits. 
Le Seigneur est donc ressuscité juste à la fin du sabbat hebdomadaire, soit le samedi soir à partir de 18 heures (ce qui correspond au début du dimanche pour les Hébreux et au samedi soir pour les Romains). Quand les femmes arrivèrent au sépulcre à l’aube du dimanche matin, le Seigneur n’y était plus (Mt. 28 :1 ; Mc. 16 :2 ; Lu. 24 :1-2).}, apparut d'abord à Marie de Madgala, de laquelle il avait chassé sept démons.
\VS{10}Elle alla l'annoncer à ceux qui avaient été avec lui, et qui étaient dans le deuil et pleuraient.
\VS{11}Mais quand ils entendirent qu'il était vivant, et qu'elle l'avait vu, ils ne la crurent point.
\VS{12}Après cela, il se montra sous une autre forme à deux d'entre eux, qui étaient en chemin pour aller à la campagne.
\VS{13}Ils revinrent l'annoncer aux autres, mais ils ne les crurent pas non plus.
\VS{14}Enfin, il se montra aux onze, qui étaient assis ensemble, et il leur reprocha leur incrédulité et leur dureté de cœur, parce ce qu'ils n'avaient pas cru ceux qui l'avaient vu ressuscité.
\TextTitle{Nouvelle mission aux onze apôtres\FTNTT{Mt. 28:16-20 ; Lu. 24:46-48 ; Jn. 17:18 ; 20:21 ; Ac. 1:8}}
\VS{15}Puis il leur dit : Allez par tout le monde, et prêchez l'Evangile à toute créature.
\VS{16}Celui qui croira et qui sera baptisé, sera sauvé ; mais celui qui ne croira pas sera condamné.
\VS{17}Et voici les miracles qui accompagneront ceux qui auront cru : Ils chasseront les démons en mon Nom ; ils parleront de nouvelles langues ;
\VS{18}ils saisiront les serpents avec la main, et s'ils boivent quelque breuvage mortel, il ne leur fera point de mal ; ils imposeront les mains aux malades, et ils seront guéris.
\TextTitle{Jésus enlevé au ciel\FTNTT{Lu. 24:50-53 ; Ac. 1:9-11}}
\VS{19}Le Seigneur, après leur avoir parlé de la sorte, fut enlevé au ciel, et il s'assit à la droite de Dieu.
\VS{20}Et ils s'en allèrent prêcher partout. Le Seigneur coopérait avec eux, et confirmait la parole par les miracles qui l'accompagnaient.
\PPE{}
\end{multicols}
