\ShortTitle{Tite}\BookTitle{Tite}\BFont
\begin{multicols}{2}
\TextTitle{[Adresse]}
\Chap{1}
\VerseOne{}Paul, serviteur de Dieu, et apôtre de Jésus-Christ, selon la foi des élus de Dieu, et la connaissance de la vérité, qui est selon la piété ;
\VS{2}lesquelles reposent sur l'espérance de la vie éternelle, que Dieu, qui ne peut mentir, avait promise avant les temps éternels ;
\VS{3}mais qui a manifesté sa parole en son propre temps, par la prédication qui m’a été confiée, d’après l’ordre de Dieu notre Sauveur,
\VS{4}à Tite mon vrai fils, selon la foi qui nous est commune ; que la grâce, la miséricorde, et la paix te soient données de la part de Dieu notre Père, et de la part du Seigneur Jésus-Christ, notre Sauveur !
\TextTitle{[Qualités nécessaires aux pasteurs]}
\VS{5}La raison pour laquelle je t'ai laissé en Crète, c'est afin que tu achèves de mettre en ordre les choses qui restent à régler, et que tu établisses des anciens de ville en ville, suivant mes instructions.
\VS{6}S’il s’y trouve quelque homme qui soit irrépréhensible, mari d'une seule femme, ayant des enfants fidèles, qui ne soient ni accusés de dissolution, ni désobéissants.
\VS{7}Car il faut que l'évêque soit irrépréhensible, comme étant économe dans la maison de Dieu ; qu’il ne soit ni arrogant, ni coléreux, ni adonné au vin, ni violent, ni porté à un gain déshonnête.
\VS{8}Mais hospitalier, aimant les gens de bien, sage, juste, saint, continent,
\VS{9}attaché à la parole de la vérité comme elle lui a été enseignée, afin qu'il soit capable tant d'exhorter par la saine doctrine que de convaincre les contredisants.
\TextTitle{[Vices des Crétois]}
\VS{10}Car il y a principalement parmi les circoncis plusieurs rebelles, vains discoureurs, et séducteurs d'esprits,
\VS{11}auxquels il faut fermer la bouche. Ils renversent des maisons tout entières, enseignant pour un gain honteux des choses qu'on ne doit point enseigner.
\VS{12}Quelqu'un d'entre eux, leur propre prophète, a dit : Les Crétois sont toujours menteurs, de mauvaises bêtes, des ventres paresseux.
\VS{13}Ce témoignage est véritable ; c'est pourquoi reprends-les vivement, afin qu'ils soient sains dans la foi,
\VS{14}et qu’ils ne s’attachent point aux fables judaïques, et aux commandements d’hommes qui se détournent de la vérité.
\VS{15}Tout est pur, en effet pour ceux qui sont purs ; mais rien n'est pur pour ceux qui sont souillés et infidèles ; leur entendement et leur conscience sont souillés.
\VS{16}Ils font profession de connaître Dieu, mais ils le renient par leurs œuvres ; car ils sont abominables, et rebelles, et réprouvés pour toute bonne œuvre.
\TextTitle{[Devoirs des vieillards, des jeunes gens, des pasteurs, des serviteurs]}
\Chap{2}
\VerseOne{}Mais toi, annonce les choses qui conviennent à la saine doctrine.
\VS{2}Que les vieillards soient sobres, honnêtes, prudents, sains dans la foi, dans la charité, et dans la patience.
\VS{3}De même, que les femmes âgées règlent leur extérieur d'une manière convenable à la sainteté ; qu'elles ne soient ni médisantes, ni sujettes à beaucoup de vin, mais qu'elles enseignent de bonnes choses,
\VS{4}afin d’apprendre aux jeunes femmes à être modestes, à aimer leurs maris et leurs enfants ;
\VS{5}à être modérées, pures, occupées aux soins domestiques, bonnes, soumises à leurs maris ; afin que la parole de Dieu ne soit point blasphémée.
\VS{6}Exhorte aussi les jeunes hommes à être modérés,
\VS{7}te montrant toi-même en toutes choses un modèle de bonnes œuvres, par une doctrine exempte de toute altération, pure et digne,
\VS{8}une doctrine saine, dans laquelle il n’y a rien à reprendre, afin que les adversaires soient confus, n'ayant aucun mal à dire de vous.
\VS{9}Exhorte les serviteurs à être soumis à leurs maîtres, à leur plaire en toutes choses, à n’être point contredisants,
\VS{10}ne dérobant rien de ce qui appartient à leurs maîtres, mais faisant toujours paraître une grande fidélité, afin de rendre honorable en toutes choses la doctrine de Dieu, notre Sauveur.
\TextTitle{[Enseignements de la grâce de Dieu]}
\VS{11}Car la grâce de Dieu salutaire à tous les hommes a été manifestée.
\VS{12}Et elle nous enseigne à renoncer à l'impiété et aux passions mondaines, et à vivre dans le présent siècle, selon la sagesse, la justice et la piété.
\VS{13}En attendant la bienheureuse espérance, et l'apparition de la gloire du grand Dieu et notre Sauveur Jésus-Christ,
\VS{14}qui s'est donné lui-même pour nous, afin de nous racheter de toute iniquité, et de nous purifier, pour lui être un peuple qui lui appartienne en propre, et qui soit zélé pour les bonnes œuvres.
\VS{15}Enseigne ces choses, exhorte et reprends avec toute autorité. Que personne ne te méprise.
\TextTitle{[Soumission aux puissances et charité envers le prochain - De la miséricorde de Dieu en ChristGénéalogie de Jésus-Christ]}
\Chap{3}
\VerseOne{}Rappelle-leur d'être soumis aux magistrats et aux autorités, d'obéir aux gouverneurs, d'être prêts à faire toute sorte de bonnes actions.
\VS{2}De ne médire de personne ; de n'être point querelleurs, mais doux, et montrant toute douceur envers tous les hommes.
\VS{3}Car nous aussi, nous étions autrefois insensés, désobéissants, égarés, asservis à toute espèce de convoitises et de voluptés, vivant dans la méchanceté et dans l’envie, dignes d’être haïs, et nous haïssant les uns les autres.
\VS{4}Mais quand la bonté de Dieu notre Sauveur, et son amour envers les hommes ont été manifestés, il nous a sauvés,
\VS{5}non à cause des œuvres de justice que nous aurions faites, mais selon la miséricorde ; par le bain de la régénération, et le renouvellement du Saint-Esprit,
\VS{6}qu’il a répandu abondamment sur nous par Jésus-Christ notre Sauveur.
\VS{7}Afin qu'ayant été justifiés par sa grâce, nous soyons les héritiers de la vie éternelle selon notre espérance.
\TextTitle{[Nécessité des bonnes œuvres]}
\VS{8}Cette parole est certaine, et je veux que tu affirmes ces choses, afin que ceux qui ont cru en Dieu aient soin principalement de s'appliquer à pratiquer les bonnes œuvres ; voilà les choses qui sont bonnes et utiles aux hommes.
\VS{9}Mais évite les discussions folles, les généalogies, les querelles et les disputes de la loi ; car elles sont inutiles et vaines.
\VS{10}Rejette l'homme hérétique, après le premier et le second avertissement,
\VS{11}sachant qu'un tel homme est perverti, et qu'il pèche en se condamnant lui-même.
\TextTitle{[Recommandations diverses]}
\VS{12}Quand je t’enverrai Artémas, ou Tychique, hâte-toi de venir vers moi à Nicopolis ; car j'ai résolu d'y passer l'hiver.
\VS{13}Accompagne soigneusement Zénas, docteur de la loi, et Apollos, afin que rien ne leur manque.
\VS{14}Que les nôtres aussi apprennent à être les premiers à s'appliquer aux bonnes œuvres, pour les usages nécessaires, afin qu'ils ne soient point sans fruits.
\VS{15}Tous ceux qui sont avec moi te saluent. Salue ceux qui nous aiment dans la foi. Grâce soit avec vous tous, Amen !
\PPE{}
\end{multicols}
