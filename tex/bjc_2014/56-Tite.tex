\ShortTitle{Tite}\BookTitle{Tite}\BFont
\noindent\hrulefill
{\footnotesize
\textit{
\bigskip
{\centering{}
\\Auteur : Paul
\\(Gr. : Titos)
\\Signifie : Nourrice, honorable
\\Thème : L'ordre dans les églises
\\Date de rédaction : Env. 65 ap. J.-C.\\}
}
%\bigskip
\textit{
\\Cette épître pastorale fut écrite après la libération de Paul de sa première captivité romaine, peut-être dans la ville de Philippes. Tite, disciple d'origine païenne et collaborateur de Paul, se trouvait alors en Crète où Paul l'avait laissé afin qu'il organise les églises. Dans cette lettre, l'apôtre traite des conditions requises pour assumer la charge d'ancien en mettant l'accent sur la saine doctrine. Mentionnant également les obligations morales des jeunes, des personnes âgées, ainsi que des serviteurs, il exhorte Tite à veiller et à s'éloigner des apostats.\bigskip
}
}
\par\nobreak\noindent\hrulefill
\begin{multicols}{2}
\Chap{1}
\TextTitle{Introduction}
\VerseOne{}Paul, serviteur de Dieu, et apôtre de Jésus-Christ, selon la foi des élus de Dieu et la connaissance de la vérité qui est selon la piété,
\VS{2}dans l'espérance de la vie éternelle, que Dieu, qui ne peut mentir, avait promise avant les temps éternels,
\VS{3}mais qu'il a manifestée en son propre temps par sa parole, dans la prédication qui m'a été confiée, par le commandement de Dieu notre Sauveur,
\VS{4}à Tite mon vrai fils, selon la foi qui nous est commune: Que la grâce, la miséricorde, et la paix te soient données de la part de Dieu notre Père, et de la part du Seigneur Jésus-Christ, notre Sauveur !
\TextTitle{Les caractéristiques d'un ancien}
\VS{5}La raison pour laquelle je t'ai laissé en Crète, c'est afin que tu achèves de mettre en bon ordre les choses qui restent à régler, et que tu établisses des anciens de ville en ville, suivant ce que je t'ai ordonné,
\VS{6}s'il s'y trouve un homme qui soit irrépréhensible, mari d'une seule femme, ayant des enfants fidèles, qui ne soient ni accusés de dissolution, ni rebelles.
\VS{7}Car il faut que l'évêque soit irrépréhensible, comme étant économe dans la maison de Dieu ; qu'il ne soit ni arrogant, ni coléreux, ni adonné au vin, ni violent, non convoiteux d'un gain déshonnête ;
\VS{8}mais hospitalier, aimant les gens de bien, sage, juste, saint, tempérant,
\VS{9}retenant ferme la parole de la vérité comme elle lui a été enseignée, afin qu'il soit capable tant d'exhorter par la saine doctrine, que de convaincre les contredisants\FTNT{Du grec « antilego » (« anti » : « contre » et « lego » : « parler ») qui signifie : « parler contre », « contredire », « contradiction », « opposition » (Lu. 2:34 ; Jn. 19:12 ; Ac. 13:45 ; 28:19 ; 28:22 ; Ro. 10:21).}.
\VS{10}Car il y en a plusieurs qui ne veulent pas se soumettre, vains discoureurs, et séducteurs d'esprits, principalement ceux qui sont de la circoncision,
\VS{11}auxquels il faut fermer la bouche, et qui renversent les maisons tout entières enseignant pour un gain déshonnête des choses qu'on ne doit point enseigner.
\VS{12}Quelqu'un d'entre eux, qui était leur propre prophète, a dit : Les Crétois sont toujours menteurs, de mauvaises bêtes, des ventres paresseux.
\VS{13}Ce témoignage est véritable. C'est pourquoi reprends-les vivement, afin qu'ils soient sains dans la foi,
\VS{14}et qu'ils ne s'attachent point aux fables judaïques et aux commandements d'hommes qui se détournent de la vérité.
\VS{15}Toutes choses sont bien pures pour ceux qui sont purs, mais rien n'est pur pour les impurs et les infidèles; mais leur entendement et leur conscience sont souillés.
\VS{16}Ils font profession de connaître Dieu, mais ils le renient par leurs œuvres, car ils sont abominables, et rebelles, et réprouvés pour toute bonne œuvre.
\Chap{2}
\TextTitle{Recommandations de Paul à Tite}
\VerseOne{}Mais toi, annonce les choses qui conviennent à la saine doctrine.
\VS{2}Que les vieillards soient sobres, honnêtes, prudents, sains dans la foi, dans la charité, et dans la patience.
\VS{3}De même, que les femmes âgées règlent leur extérieur d'une manière convenable à la sainteté ; qu'elles ne soient ni médisantes, ni sujettes à beaucoup de vin, mais qu'elles enseignent de bonnes choses,
\VS{4}afin qu'elles instruisent les jeunes femmes à être modestes, à aimer leurs maris, à aimer leurs enfants,
\VS{5}à être modérées, pures, occupées aux soins domestiques, bonnes, soumises à leurs maris, afin que la Parole de Dieu ne soit point blasphémée.
\VS{6}Exhorte aussi les jeunes hommes à être modérés,
\VS{7}te montrant toi-même un modèle de bonnes œuvres en toutes choses, en une doctrine exempte de toute altération, en pureté, en intégrité,
\VS{8}en paroles saines, que l'on ne puisse point condamner, afin que celui qui vous est contraire, soit rendu confus, n'ayant aucun mal à dire de vous.
\VS{9}Que les serviteurs soient soumis à leurs maîtres, leur complaisant en toutes choses, n'étant point contredisants,
\VS{10}ne dérobant rien de ce qui appartient à leurs maîtres, mais faisant toujours paraître une grande fidélité, afin de rendre honorable en toutes choses la doctrine de Dieu, notre Sauveur.
\VS{11}Car la grâce de Dieu, salutaire à tous les hommes, a été manifestée.
\VS{12}Et elle nous enseigne à renoncer à l'impiété et aux passions mondaines, et à vivre dans le présent siècle, selon la sagesse, la justice et la piété,
\VS{13}en attendant la bienheureuse espérance, et l'apparition de la gloire du grand Dieu et notre Sauveur Jésus-Christ,
\VS{14}qui s'est donné lui-même pour nous, afin de nous racheter de toute iniquité, et de nous purifier, pour lui être un peuple qui lui appartienne en propre, et qui soit zélé pour les bonnes œuvres.
\VS{15}Enseigne ces choses, exhorte, et reprends avec une pleine autorité. Et que personne ne te méprise.
\Chap{3}
\TextTitle{Conseils pratiques de Paul}
\VerseOne{}Rappelle-leur d'être soumis aux magistrats et aux autorités, d'obéir aux gouverneurs, d'être prêts à faire toute sorte de bonnes actions,
\VS{2}de ne médire de personne, de n'être point querelleurs, mais doux, et montrant une parfaite douceur envers tous les hommes.
\VS{3}Car nous aussi, nous étions autrefois insensés, désobéissants, égarés, asservis à toute espèce de convoitises et de voluptés, vivant dans la méchanceté et dans l'envie, dignes d'être haïs, et nous haïssant les uns les autres.
\VS{4}Mais, quand la bonté de Dieu notre Sauveur et son amour envers les hommes ont été manifestés, il nous a sauvés,
\VS{5}non par des œuvres de justice que nous aurions faites, mais selon la miséricorde, par le bain de la régénération et le renouvellement du Saint-Esprit,
\VS{6}qu'il a répandu abondamment sur nous par Jésus-Christ notre Sauveur,
\VS{7}afin qu'ayant été justifiés par sa grâce, nous soyons les héritiers de la vie éternelle selon notre espérance.
\VS{8}Cette parole est certaine, et je veux que tu affirmes ces choses, afin que ceux qui ont cru en Dieu aient soin principalement de s'appliquer à pratiquer les bonnes œuvres. Voilà les choses qui sont bonnes et utiles aux hommes.
\VS{9}Mais évite les discussions folles, les généalogies, et les querelles et les disputes de la loi ; car elles sont inutiles et vaines.
\VS{10}Rejette l'homme hérétique, après le premier et le second avertissement,
\VS{11}sachant qu'un tel homme est perverti, et qu'il pèche en se condamnant lui-même.
\TextTitle{Salutations}
\VS{12}Quand je t'enverrai Artémas ou Tychique, hâte-toi de venir vers moi à Nicopolis ; car j'ai résolu d'y passer l'hiver.
\VS{13}Accompagne soigneusement Zénas, docteur de la loi, et Apollos, afin que rien ne leur manque.
\VS{14}Que les nôtres aussi apprennent à être les premiers à s'appliquer aux bonnes œuvres, pour les usages nécessaires, afin qu'ils ne soient point sans fruits.
\VS{15}Tous ceux qui sont avec moi te saluent. Salue ceux qui nous aiment dans la foi. Grâce soit avec vous tous ! Amen !
\PPE{}
\end{multicols}
