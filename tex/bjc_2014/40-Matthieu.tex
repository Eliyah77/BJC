\ShortTitle{Matthieu}\BookTitle{Matthieu}\BFont
\begin{multicols}{2}
\TextTitle{[Présentation du Roi : La généalogie de Jésus-Christ]}
\Chap{1}
\VerseOne{}Livre de la généalogie de Jésus-Christ, fils de David, fils d’Abraham.
\VS{2}Abraham engendra Isaac ; Isaac engendra Jacob ; Jacob engendra Juda et ses frères ;
\VS{3}Juda engendra Pérets et Zara, de Thamar ; Pérets engendra Esrom ; Esrom engendra Aram ;
\VS{4}Aram engendra Aminadab ; Aminadab engendra Naasson ; et Naasson engendra Salmon ;
\VS{5}Salmon engendra Boaz, de Rahab{\FTNT{Rahab était une prostituée Cananéenne qui est devenue l’ancêtre du Messie (Jos. 6).}} ; Boaz engendra Obed, de Ruth{\FTNT{Ruth était une Moabite, son peuple était issu de la relation incestueuse de Lot et sa fille aînée (Ge. 19:36-37). Elle est devenue l’ancêtre du Messie (Ru. 4:17).}} ; Obed engendra Isaï ;
\VS{6}Isaï engendra le roi David ; le roi David engendra Salomon, de la femme d'Urie ;
\VS{7}Salomon engendra Roboam ; Roboam engendra Abia ; Abia engendra Asa ;
\VS{8}Asa engendra Josaphat ; Josaphat engendra Joram ; Joram engendra Ozias ;
\VS{9}Ozias engendra Joatham ; Joatham engendra Achaz ; Achaz engendra Ezéchias ;
\VS{10}Ezéchias engendra Manassé ; Manassé engendra Amon ; Amon engendra Josias ;
\VS{11}Josias engendra Jéconias et ses frères, au temps de la déportation à Babylone.
\VS{12}Après la déportation à Babylone, Jéchonias engendra Salathiel ; Salathiel engendra Zorobabel ;
\VS{13}Zorobabel engendra Abiud ; Abiud engendra Eliakim ; Eliakim engendra Azor ;
\VS{14}Azor engendra Sadok ; Sadok engendra Achim ; Achim engendra Eliud ;
\VS{15}Eliud engendra Eléazar ; Eléazar engendra Matthan ; Matthan engendra Jacob ;
\VS{16}Jacob engendra Joseph, l’époux de Marie, de laquelle est né Jésus, qui est appelé Christ{\FTNT{Christ : du grec «~christos~», ce qui signifie «~oint~», est l’équivalent grec du mot hébreu «~mashiyach~», traduit par «~messie~» en français (Da. 9:25-26). C’est le titre officiel du Seigneur Jésus.}}.
\VS{17}Ainsi il y a en tout quatorze générations depuis Abraham jusqu'à David, quatorze générations depuis David jusqu'à la déportation à Babylone, et quatorze générations depuis la déportation à Babylone jusqu'au Christ.
\TextTitle{[Naissance miraculeuse de Jésus-Christ]
\\(Lu. 1:26-38 ; 2:1-7 ; Jn. 1:1-2,14)}
\VS{18}Voici de quelle manière arriva la naissance de Jésus-Christ. Marie, sa mère, ayant été fiancée à Joseph, se trouva enceinte par l’opération du Saint-Esprit, avant qu'ils aient habité ensemble.
\VS{19}Joseph, son époux, qui était un homme juste et qui ne voulait pas la diffamer, se proposa de la répudier secrètement.
\VS{20}Mais comme il y pensait, voici, l'Ange du Seigneur{\FTNT{L’ange du Seigneur est l’ange de Yahweh, c’est-à-dire Jésus-Christ lui-même (Ge. 16:7). Tous les commentateurs bibliques s’accordent pour dire que l’ange de Yahweh est Jésus-Christ avant son incarnation. Toutefois, il n’a pas cessé de se manifester sous cette forme après son incarnation. En effet, dans les passages suivants : Mt. 3:3 ; Mt. 4:10 ; Mt. 21:9 ; Mc.12:29 ; Lu. 1:68 ; Lu. 2:9, Ro. 10:13, on cite respectivement les prophéties de : Es. 40:3 ; De. 6:16 ; De. 6:13 ; Ps. 118:26 ; De. 6:5 ; 1 R. 1:48 ; Ps. 106:48 ; Joë. 2:32. Tous ces passages des évangiles et des épîtres, qui sont des citations des écrits prophétiques, nous montrent que dans la pensée de leurs auteurs il s'agissait du même Ange. De plus en Mt. 1:20, l’ange de Yahweh, donc Jésus-Christ lui-même, est apparu à Joseph alors qu’il se trouvait en même temps dans le ventre de Marie ! Ce même ange s’était adressé à Manoach et sa femme pour leur annoncer la naissance de Samson (Jg. 13). Nous avons ici une preuve merveilleuse de la toute-puissance et de l’omniprésence de notre Dieu qui peut se manifester sous diverses formes, à plusieurs endroits et en même temps, sans l’aide de personne.}} lui apparut en songe et lui dit : Joseph, fils de David, ne crains point de prendre avec toi Marie, ta femme, car l’enfant qu’elle a conçu est du Saint-Esprit.
\VS{21}Elle enfantera un fils, et tu lui donneras le nom de Jésus. C’est lui qui sauvera son peuple de ses péchés.
\VS{22}Tout cela arriva afin que s’accomplisse ce que le Seigneur avait annoncé par le prophète :
\VS{23}Voici, la vierge deviendra enceinte, elle enfantera un fils ; et on lui donnera le nom d’Emmanuel{\FTNT{Es. 7:14.}}, ce qui signifie, Dieu est avec nous.
\VS{24}Joseph s’étant donc réveillé de son sommeil, fit ce que l'ange du Seigneur lui avait ordonné, et il prit sa femme.
\VS{25}Mais il ne la connut point jusqu'à ce qu'elle ait enfanté son fils premier-né, auquel il donna le nom de Jésus.
\TextTitle{[Arrivée des mages]}
\Chap{2}
\VerseOne{}Jésus étant né à Bethléem, ville de Juda, au temps du roi Hérode, voici des mages d'Orient arrivèrent à Jérusalem.
\VS{2}En disant : Où est le Roi des Juifs qui vient de naître ? Car nous avons vu son étoile en Orient, et nous sommes venus l'adorer.
\VS{3}Le roi Hérode ayant entendu, fut troublé et tout Jérusalem avec lui.
\VS{4}Et ayant assemblé tous les principaux sacrificateurs, et les scribes du peuple, il s'informa auprès d'eux où le Christ devait naître.
\VS{5}Et ils lui dirent : à Bethléem, ville de Judée ; car voici ce qui a été écrit par le prophète :
\VS{6}Et toi, Bethléem, terre de Juda, tu n'es certes pas la plus petite entre les principales villes de Juda, car de toi sortira le chef qui paîtra mon peuple d'Israël{\FTNT{Mi. 5:1.}}.
\VS{7}Alors Hérode ayant appelé en secret les mages, s'informa soigneusement auprès d’eux depuis combien de temps brillait l'étoile.
\VS{8}Puis les envoya à Bethléem, en leur disant : Allez, et prenez des informations exactes sur le petit enfant ; et quand vous l'aurez trouvé, faites-le-moi savoir, afin que j’aille aussi moi-même l’adorer.
\VS{9}Après avoir entendu le roi, ils partirent. Et voici, l'étoile{\FTNT{Il est évident que cette étoile qui se déplace pour guider les mages n’est pas une étoile ordinaire. Il s’agit de Jésus-Christ lui-même (Ap. 22:16).}} qu'ils avaient vue en Orient allait devant eux, jusqu’au moment où, arrivée au-dessus du lieu où était le petit enfant, elle s’arrêta.
\VS{10}Quand ils virent l'étoile, ils furent saisis d’une très grande joie.
\VS{11}Ils entrèrent dans la maison, virent le petit enfant avec Marie sa mère, se prosternèrent et l’adorèrent. Ils ouvrirent ensuite leurs trésors et lui offrirent des présents : De l'or, de l'encens et de la myrrhe.
\VS{12}Puis, divinement avertis en songe de ne pas retourner vers Hérode, ils regagnèrent leur pays par un autre chemin.
\TextTitle{[Fuite en Egypte]}
\VS{13}Lorsqu’ils furent partis, voici, l’Ange du Seigneur apparut dans un songe à Joseph, et lui dit : Lève-toi et prends le petit enfant et sa mère, fuis en Egypte, et demeure là, jusqu'à ce que je te le dise ; car Hérode cherchera le petit enfant pour le faire mourir.
\VS{14}Joseph donc étant réveillé, prit de nuit le petit enfant et sa mère, et se retira en Egypte.
\VS{15}Il y resta là jusqu’à la mort d'Hérode ; afin que s’accomplisse ce que le Seigneur avait annoncé par le prophète : J'ai appelé mon Fils hors d'Egypte{\FTNT{Os. 11:1}}.
\TextTitle{[Hérode envoie tuer des enfants innocents]}
\VS{16}Alors Hérode voyant que les mages s'étaient moqués de lui, se mit dans une grande colère, et il envoya tuer tous les enfants qui étaient à Bethléem, et dans tout son territoire ; depuis l'âge de deux ans, et au-dessous, selon la date dont il s'était exactement enquis auprès des mages.
\VS{17}Alors s’accomplit ce qui avait été annoncé par Jérémie le prophète :
\VS{18}On a entendu à Rama des cris, des lamentations, des plaintes, et des grands gémissements : Rachel pleure ses enfants, et n’a pas voulu être consolée, parce qu’ils ne sont plus{\FTNT{Jé. 31:15}}.
\TextTitle{[Joseph revient en Israël et vint habiter à Nazareth]
\\(cp. Lu. 2:39-52)}
\VS{19}Mais après qu'Hérode fut mort, voici, l'Ange du Seigneur apparut dans un songe à Joseph, en Egypte,
\VS{20}et lui dit : Lève-toi, et prends le petit enfant et sa mère, et va dans le pays d'Israël ; car ceux qui cherchaient à ôter la vie au petit enfant sont morts.
\VS{21}Joseph donc s'étant réveillé, prit le petit enfant et sa mère, et alla dans le pays d'Israël.
\VS{22}Mais quand il eut appris qu'Archélaüs régnait en Judée, à la place d'Hérode son père, il craignit d'y aller ; et étant divinement averti dans un songe, il se retira dans le territoire de la Galilée,
\VS{23}et vint habiter dans la ville appelée Nazareth ; afin que s’accomplisse ce qui avait été dit par les prophètes : Il sera appelé Nazaréen{\FTNT{Es. 11:1}}.
\TextTitle{[Ministère de Jean-Baptiste]
\\(Mc. 1:1-8 ; Lu. 3:1-20 ; Jn. 1:6-8,15-37)}
\Chap{3}
\VerseOne{}Or, en ce temps-là arriva Jean-Baptiste, prêchant dans le désert de la Judée.
\VS{2}Il disait : Repentez-vous, car le Royaume des cieux est proche.
\VS{3}Car c'est celui dont Esaïe le prophète a parlé, en disant : C’est ici la voix de celui qui crie dans le désert : Préparez le chemin du Seigneur, aplanissez ses sentiers.
\VS{4}Jean avait un vêtement de poils de chameau, et une ceinture de cuir autour de ses reins. Et il se nourrissait de sauterelles et de miel sauvage.
\VS{5}Alors les habitants de Jérusalem, et de toute la Judée, et de tout le pays des environs du Jourdain, vinrent à lui ;
\VS{6}et confessant leurs péchés, ils se faisaient baptiser par lui dans le Jourdain.
\VS{7}Mais, voyant venir à son baptême beaucoup de pharisiens et de sadducéens, il leur dit : Race de vipères, qui vous a appris à fuir la colère à venir ?
\VS{8}Produisez donc des fruits convenables à la repentance,
\VS{9}et ne prétendez pas dire en vous-mêmes : Nous avons Abraham pour père ! Car je vous dis que Dieu peut faire naître de ces pierres mêmes des enfants à Abraham.
\VS{10}Et déjà la cognée est mise à la racine des arbres ; c'est pourquoi tout arbre qui ne produit pas de bons fruits sera coupé et jeté au feu.
\VS{11}Pour moi, je vous baptise d'eau en signe de repentance ; mais celui qui vient après moi est plus puissant que moi, et je ne suis pas digne de porter ses souliers ; celui-là vous baptisera du Saint-Esprit et de feu{\FTNT{Le baptême du Saint-Esprit ne doit pas être confondu avec la plénitude du Saint-Esprit. Le baptême est un acte définitif qui nous greffe au corps du Christ lors de la conversion (1 Co. 12:13). La plénitude consiste quant à elle en un constant renouvellement que nous devons impérativement rechercher (Ep. 5:18). Certains courants chrétiens charismatiques enseignent que le parler en langues est le signe distinctif du baptême du Saint-Esprit. Cette doctrine est basée sur au moins trois passages : Ac. 2:4 ; Ac. 10:44-46 et Ac. 19:1-7. Si cela était vraiment le cas, plusieurs chrétiens seraient encore dans leurs péchés et n’appartiendraient pas au Seigneur Jésus-Christ. En effet, Ro. 8:9 déclare ceci : «~Si quelqu'un n'a pas l'Esprit de Christ, il ne lui appartient pas~». Or il est manifeste que bon nombre de chrétiens nés d’en haut ne parlent pas en langues, ce qui est d’ailleurs attesté par l’apôtre Paul (1 Co. 12:30). Il n’y a aucun verset dans les Ecritures qui nous ordonne de chercher le baptême du Saint-Esprit pour la bonne et simple raison que nous le recevons à la conversion.}}.
\VS{12}Il a son van à la main, et il nettoiera entièrement son aire, et il assemblera son froment dans le grenier ; mais il brûlera la paille dans un feu qui ne s'éteint point.
\TextTitle{[Jean baptise Jésus-Christ]
\\(Mc. 1:9-11 ; Lu. 3:21-22 ; cp. Jn. 1:31-34)}
\VS{13}Alors Jésus vint de Galilée au Jourdain vers Jean pour être baptisé par lui.
\VS{14}Mais Jean l'en empêchait avec force en lui disant : J'ai besoin d'être baptisé par toi, et tu viens vers moi ?
\VS{15}Et Jésus répondit en disant : Laisse-moi faire pour le moment, car il nous est ainsi convenable d'accomplir tout ce qui est juste. Et alors il le laissa faire.
\VS{16}Dès que Jésus eut été baptisé, il sortit aussitôt hors de l'eau. Et voici, les cieux lui furent ouverts, et Jean vit l'Esprit de Dieu descendant comme une colombe et venant sur lui.
\VS{17}Et voici une voix du ciel déclara : Celui-ci est mon Fils bien-aimé, en qui j'ai mis toute mon affection.
\TextTitle{[Tentation de Jésus-Christ]
\\(Mc. 1:12-13 ; Lu. 4:1-13 ; cp. Ge. 3:6 ; 1 Jn. 2:16)}
\Chap{4}
\VerseOne{}Alors Jésus fut emmené par l'Esprit dans le désert, pour être tenté par le diable.
\VS{2}Après avoir jeûné quarante jours et quarante nuits, finalement il eut faim.
\VS{3}Et le tentateur s’étant approché, lui dit : Si tu es le Fils de Dieu, ordonne que ces pierres deviennent des pains.
\VS{4}Mais Jésus répondit et dit : Il est écrit : L'homme ne vivra point de pain seulement, mais de toute parole qui sort de la bouche de Dieu{\FTNT{De. 8:3.}}.
\VS{5}Alors le diable le transporta dans la sainte ville et le mit sur le haut du temple ;
\VS{6}et il lui dit : Si tu es le Fils de Dieu, jette-toi en bas ; car il est écrit : Il ordonnera à ses anges de te porter sur leurs mains, de peur que ton pied ne heurte contre une pierre{\FTNT{Ps. 91:12-13.}}.
\VS{7}Jésus lui dit : Il est aussi écrit : Tu ne tenteras point le Seigneur ton Dieu{\FTNT{De. 6:16.}}.
\VS{8}Le diable le transporta encore sur une forte haute montagne, et lui montra tous les royaumes du monde et leur gloire ;
\VS{9}et il lui dit : Je te donnerai toutes ces choses, si tu te prosternes et m'adores.
\VS{10}Mais Jésus lui dit : Retire-toi Satan ! Car il est écrit : Tu adoreras le Seigneur ton Dieu, et tu le serviras lui seul{\FTNT{De. 6:13 ; De. 10:20.}}.
\VS{11}Alors le diable le laissa. Et voici, des anges s'approchèrent et le servirent.
\TextTitle{[Jésus commence son ministère en Galilée]
\\(Mc. 1:14-15 ; Lu. 4:14-15)}
\VS{12}Jésus, ayant appris que Jean avait été mis en prison, se retira dans la Galilée.
\VS{13}Et ayant quitté Nazareth, il alla demeurer à Capernaüm, ville maritime, sur les confins de Zabulon et de Nephthali.
\VS{14}Afin que s’accomplisse ce qui avait été annoncé par Esaïe, le prophète, en disant :
\VS{15}Le pays de Zabulon et le pays de Nephthali, de la contrée voisine de la mer, au-delà du Jourdain, et la Galilée des gentils ;
\VS{16}Ce peuple, assis dans les ténèbres, a vu une grande lumière ; et à ceux qui étaient assis dans la région et l'ombre de la mort, la lumière elle-même s'est levée{\FTNT{Es. 9:1.}}.
\VS{17}Dès lors, Jésus commença à prêcher et à dire : Repentez-vous, car le Royaume des cieux est proche.
\TextTitle{[Appel de Pierre, André, Jacques et Jean]
\\(Mc. 1:16-20 ; cp. Lu. 5:1-11 ; Jn. 1:35-51)}
\VS{18}Comme Jésus marchait le long de la mer de Galilée, il vit deux frères, Simon, appelé Pierre, et André, son frère, qui jetaient leurs filets dans la mer ; car ils étaient pêcheurs.
\VS{19}Et il leur dit : Suivez-moi, et je vous ferai pêcheurs d'hommes.
\VS{20}Et ayant aussitôt quitté leurs filets, ils le suivirent.
\VS{21}Et de là étant allé plus avant, il vit deux autres frères, Jacques, fils de Zébédée, et Jean, son frère, dans une barque, avec Zébédée leur père, qui réparaient leurs filets, et il les appela.
\VS{22}Et ayant aussitôt quitté leur barque et leur père, ils le suivirent.
\TextTitle{[Ministère de Jésus-Christ en Galilée]}
\VS{23}Jésus allait par toute la Galilée, enseignant dans leurs synagogues, prêchant l'Evangile du Royaume, et guérissant toutes sortes de maladies, et toutes sortes d’infirmités parmi le peuple.
\VS{24}Et sa renommée se répandit par toute la Syrie ; et on lui présentait tous ceux qui se portaient mal, tourmentés de diverses maladies, des démoniaques, des lunatiques, des paralytiques ; et il les guérissait.
\VS{25}Une grande foule le suivit, de Galilée, de la Décapole, de Jérusalem, de Judée, et au-delà du Jourdain.
\TextTitle{[L'enseignement de Jésus sur la montagne]
\\(Luc. 6:20-49)}
\Chap{5}
\VerseOne{}Voyant la foule, Jésus monta sur la montagne ; puis s'étant assis, ses disciples s'approchèrent de lui.
\VS{2}Puis, ayant ouvert la bouche, il les enseigna de la sorte :
\VS{3}Heureux les pauvres en esprit, car le Royaume des cieux est à eux.
\VS{4}Heureux ceux qui pleurent, car ils seront consolés.
\VS{5}Heureux les humbles, car ils hériteront la terre.
\VS{6}Heureux ceux qui ont faim et soif de la justice, car ils seront rassasiés.
\VS{7}Heureux les miséricordieux, car ils obtiendront miséricorde.
\VS{8}Heureux sont ceux qui sont purs de cœur, car ils verront Dieu.
\VS{9}Heureux ceux qui procurent la paix, car ils seront appelés enfants de Dieu.
\VS{10}Heureux ceux qui sont persécutés pour la justice, car le Royaume des cieux est à eux.
\VS{11}Heureux serez-vous lorsqu’on vous outragera, qu’on vous persécutera et qu’on dira faussement de vous toute sorte de mal à cause de moi.
\VS{12}Réjouissez-vous et soyez dans l’allégresse, parce que votre récompense sera grande dans les cieux ; car c’est ainsi qu’on a persécuté les prophètes qui ont été avant vous.
\TextTitle{[Deux paraboles : le sel de la terre et la lumière du monde]
\\(Mc. 4:21-23 ; Lu. 8:16-18 ; 11:33-36)}
\VS{13}Vous êtes le sel de la terre ; mais si le sel perd sa saveur, avec quoi le salera-t-on ? Il ne sert plus qu’à être jeté dehors, et foulé aux pieds par les hommes.
\VS{14}Vous êtes la lumière du monde. Une ville située sur une montagne ne peut être cachée,
\VS{15}et on n'allume point la lampe pour la mettre sous un boisseau, mais sur un chandelier, et elle éclaire tous ceux qui sont dans la maison.
\VS{16}Ainsi, que votre lumière luise devant les hommes, afin qu'ils voient vos bonnes œuvres, et qu'ils glorifient votre Père qui est dans les cieux.
\TextTitle{[Relation de Christ avec la loi]}
\VS{17}Ne croyez pas que je sois venu abolir la loi ou les prophètes ; je ne suis pas venu les abolir, mais les accomplir.
\VS{18}Car, je vous le dis en vérité, tant que le ciel et la terre ne passeront point, il ne disparaîtra pas de la loi un seul iota ou un seul trait de lettre jusqu’à ce que tout soit arrivé.
\VS{19}Celui donc qui aura violé l'un de ces petits commandements, et qui aura enseigné les hommes à faire de même, sera appelé le plus petit au Royaume des cieux ; mais celui qui les observera, et qui enseignera à les observer, celui-là sera appelé grand au Royaume des cieux.
\VS{20}Car je vous dis que si votre justice ne surpasse celle des scribes et des pharisiens, vous n'entrerez point dans le Royaume des cieux.
\TextTitle{[Injure, offrande, pardon]}
\VS{21}Vous avez entendu qu'il a été dit aux anciens : Tu ne tueras point, et celui qui tuera, sera puni par les juges.
\VS{22}Mais moi je vous dis que quiconque se met en colère sans cause contre son frère sera puni par les juges ; et celui qui dira à son frère : Raca ! sera puni par le conseil ; et celui qui lui dira : Insensé ! sera puni par le feu de la géhenne{\FTNT{La géhenne ou le lac de feu : Voir commentaire  Ap. 20:14.}}.
\VS{23}Si donc tu apportes ton offrande à l'autel, et que là tu te souviennes que ton frère a quelque chose contre toi,
\VS{24}laisse là ton offrande devant l'autel, et va te réconcilier d’abord avec ton frère, puis viens et offre ton offrande.
\VS{25}Accorde-toi rapidement avec ta partie adverse, tandis que tu es en chemin avec elle ; de peur que ta partie adverse ne te livre au juge, et que le juge ne te livre à l’officier de justice, et que tu ne sois mis en prison.
\VS{26}En vérité, je te dis, tu ne sortiras point de là, jusqu'à ce que tu n’aies payé le dernier quart de sou.
\TextTitle{[Convoitise, adultère et divorce]
\\(cp. Mt. 19:3-11 ; Mc. 10:2-12 ; 1 Co. 7:1-16)}
\VS{27}Vous avez entendu qu'il a été dit aux anciens : Tu ne commettras point d’adultère.
\VS{28}Mais moi, je vous dis que quiconque regarde une femme pour la convoiter a déjà commis dans son cœur un adultère avec elle.
\VS{29}Si ton œil droit est pour toi une occasion de chute, arrache-le et jette-le loin de toi ; car il est avantageux pour toi qu'un seul de tes membres périsse, et que ton corps entier ne soit pas jeté dans la géhenne.
\VS{30}Si ta main droite est pour toi une occasion de chute, coupe-la et jette-la loin de toi ; car il est avantageux pour toi qu’un seul de tes membres périsse, et que ton corps entier ne soit pas jeté dans la géhenne.
\VS{31}Il a été dit encore : Si quelqu'un répudie sa femme, qu'il lui donne une lettre de divorce.
\VS{32}Mais moi, je vous dis que celui qui répudie sa femme, si ce n'est pour cause d'adultère, l’expose à devenir adultère ; et que celui qui épouse une femme répudiée commet un adultère.
\TextTitle{[Interdiction du parjure et de la vengeance]}
\VS{33}Vous avez aussi appris qu'il a été dit aux anciens : Tu ne te parjureras point, mais tu rendras au Seigneur ce que tu auras promis par serment.
\VS{34}Mais moi, je vous dis de ne jurer aucunement, ni par le ciel, parce que c'est le trône de Dieu ;
\VS{35}ni par la terre, parce que c'est son marchepied, ni par Jérusalem, parce que c'est la ville du grand Roi.
\VS{36}Ne jure pas non plus par ta tête, car tu ne peux pas rendre blanc ou noir un seul cheveu.
\VS{37}Mais que votre parole soit : Oui, oui ; non, non ; car ce qui est de plus, vient du malin.
\VS{38}Vous avez appris qu'il a été dit : Œil pour œil et dent pour dent.
\VS{39}Mais moi, je vous dis : Ne résistez point au méchant. Si quelqu'un te frappe sur ta joue droite, présente-lui aussi l'autre.
\VS{40}Si quelqu'un veut plaider contre toi, et prendre ta tunique, laisse-lui encore ton manteau.
\VS{41}Si quelqu'un te force à faire un mille, fais-en deux avec lui.
\VS{42}Donne à celui qui te demande, et ne te détourne point de celui qui veut emprunter de toi.
\TextTitle{[Aimer ses ennemis et bénir ceux qui nous maudissent]
\\(Lu. 6:27-36)}
\VS{43}Vous avez appris qu'il a été dit : Tu aimeras ton prochain, et tu haïras ton ennemi.
\VS{44}Mais moi, je vous dis : Aimez vos ennemis, bénissez ceux qui vous maudissent, faites du bien à ceux qui vous haïssent, et priez pour ceux qui vous maltraitent et vous persécutent,
\VS{45}afin que vous soyez les fils de votre Père qui est dans les cieux ; car il fait lever son soleil sur les méchants, et sur les gens de bien, et il envoie sa pluie sur les justes, et sur les injustes.
\VS{46}Car si vous aimez seulement ceux qui vous aiment, quelle récompense en aurez-vous ? Les publicains aussi n'en font-ils pas tout autant ?
\VS{47}Et si vous faites accueil seulement à vos frères, que faites-vous de plus que les autres ? Les publicains aussi ne le font-ils pas de même ?
\VS{48}Soyez donc parfaits, comme votre Père qui est dans les cieux est parfait.
\TextTitle{[Jésus condamne l'hypocrisie]}
\Chap{6}
\VerseOne{}Gardez-vous de pratiquer votre justice devant les hommes, pour en être vus ; autrement, vous ne recevrez point la récompense de votre Père qui est dans les cieux.
\VS{2}Donc, lorsque tu fais ton aumône, ne fais point sonner la trompette devant toi, comme font les hypocrites dans les synagogues, et dans les rues, afin d’être glorifiés par les hommes. Je vous le dis en vérité, ils reçoivent leur récompense.
\VS{3}Mais quand tu fais ton aumône, que ta main gauche ne sache pas ce que fait ta droite.
\VS{4}Afin que ton aumône se fasse en secret, et ton Père, qui voit ce qui se fait dans le secret, te récompensera publiquement.
\VS{5}Et quand tu pries, ne sois point comme les hypocrites ; car ils aiment à prier en se tenant debout dans les synagogues et aux coins des rues, pour être vus des hommes. Je vous le dis en vérité, ils reçoivent leur récompense.
\VS{6}Mais toi, quand tu pries, entre dans ta chambre, et ayant fermé ta porte, prie ton Père, qui est là dans ce lieu secret ; et ton Père qui te voit dans ce lieu secret, te récompensera publiquement.
\VS{7}Quand vous priez, ne multipliez pas de vaines paroles, comme font les païens ; car ils s'imaginent qu’à force de paroles ils seront exaucés.
\VS{8}Ne leur ressemblez donc point ; car votre Père sait de quoi vous avez besoin, avant que vous le lui demandiez.
\TextTitle{[Instructions de Jésus sur la prière]}
\VS{9}Voici donc comment vous devez prier : Notre Père qui es aux cieux, que ton Nom soit sanctifié.
\VS{10}Que ton règne vienne. Que ta volonté soit faite sur la terre comme au ciel.
\VS{11}Donne-nous aujourd'hui notre pain quotidien.
\VS{12}Pardonne-nous nos offenses, comme nous aussi nous pardonnons à ceux qui nous ont offensés.
\VS{13}Ne nous induis pas en tentation ; mais délivre-nous du mal. Car c’est à toi qu’appartiennent, dans tous les siècles, le règne, et la puissance et la gloire. Amen !
\VS{14}Car si vous pardonnez aux hommes leurs offenses, votre Père céleste vous pardonnera aussi les vôtres.
\VS{15}Mais si vous ne pardonnez point aux hommes leurs offenses, votre Père ne vous pardonnera point non plus vos offenses.
\VS{16}Et quand vous jeûnerez, ne prenez pas un air triste, comme font les hypocrites ; car ils se rendent le visage tout défait, afin de montrer aux hommes qu'ils jeûnent. Je vous le dis en vérité, ils reçoivent leur récompense.
\VS{17}Mais toi, quand tu jeûnes, parfume ta tête et lave ton visage,
\VS{18}afin de ne pas montrer aux hommes que tu jeûnes, mais à ton Père qui est là dans ton lieu secret ; et ton Père qui te voit dans ton lieu secret, te récompensera publiquement.
\TextTitle{[Jésus exhorte ses disciples à chercher d'abord le Royaume de Dieu]}
\VS{19}Ne vous amassez point des trésors sur la terre, où les vers et la rouille détruisent, et où les voleurs percent et dérobent.
\VS{20}Mais amassez-vous des trésors dans le ciel, où les vers et la rouille ne détruisent point, et où les voleurs ne percent ni ne dérobent.
\VS{21}Car là où est ton trésor, là aussi sera ton cœur.
\VS{22}L’œil est la lampe du corps. Si donc ton œil est en bon état, tout ton corps sera éclairé.
\VS{23}Mais si ton œil est mal disposé, tout ton corps sera ténébreux. Si donc la lumière qui est en toi est ténèbres, combien seront grandes ces ténèbres ?
\VS{24}Nul ne peut servir deux maîtres. Car, ou il haïra l'un, et aimera l'autre ; ou il s'attachera à l'un, et méprisera l'autre. Vous ne pouvez pas servir Dieu et Mammon{\FTNT{Mammon : Mot d'origine araméenne signifiant «~riche~». Certains le rapprochent de l’hébreu «~matmon~» : trésor, argent. D'autres le rapprochent du phénicien «~mommon~» : bénéfice. Dans les évangiles, il signifie «~possession~» (matérielle), mais il est parfois personnifié.}}.
\TextTitle{[Jésus exhorte à ne pas s'inquiéter]}
\VS{25}C'est pourquoi je vous dis : Ne vous inquiétez pas pour votre vie, de ce que vous mangerez, et de ce que vous boirez ; ni pour votre corps, de quoi vous serez vêtus. La vie n'est-elle pas plus que la nourriture, et le corps plus que le vêtement ?
\VS{26}Considérez les oiseaux du ciel ; car ils ne sèment, ni ne moissonnent, ni n'assemblent dans des greniers, et cependant votre Père céleste les nourrit. Ne valez-vous pas mieux qu'eux ?
\VS{27}Et qui de vous, par ses inquiétudes, peut ajouter une coudée à la durée de sa vie ?
\VS{28}Et pourquoi vous inquiéter au sujet du vêtement ? Apprenez comment croissent les lis des champs : Ils ne travaillent ni ne filent ;
\VS{29}cependant je vous dis que Salomon même, dans toute sa gloire, n'a pas été vêtu comme l'un d'eux.
\VS{30}Si donc Dieu revêt ainsi l'herbe des champs, qui est aujourd'hui sur pied, et qui demain sera jetée au four, ne vous vêtira-t-il pas à plus forte raison, ô gens de petite foi ?
\VS{31}Ne vous inquiétez donc point en disant : Que mangerons-nous ? Ou, que boirons-nous ? Ou de quoi serons-nous vêtus ?
\VS{32}Car toutes ces choses, ce sont les païens qui les recherchent. Car votre Père céleste sait que vous en avez besoin.
\VS{33}Mais cherchez premièrement le Royaume de Dieu et sa justice, et toutes ces choses vous seront données par-dessus.
\VS{34}Ne vous inquiétez donc pas pour le lendemain ; car le lendemain prendra soin de lui-même. A chaque jour suffit sa peine.
\TextTitle{[Fin du discours de Jésus-Christ
\\Blâme des critiques injustes]
\\(Luc. 6:37-42)}
\Chap{7}
\VerseOne{}Ne jugez point, afin que vous ne soyez point jugés.
\VS{2}Car on vous jugera du jugement dont vous jugez, et l’on vous mesurera avec la mesure dont vous mesurez.
\VS{3}Et pourquoi vois-tu la paille qui est dans l’œil de ton frère, et n’aperçois-tu pas la poutre qui est dans ton œil ?
\VS{4}Ou comment peux-tu dire à ton frère : Permets que j'ôte de ton œil cette paille, et n’aperçois-tu pas la poutre dans ton œil ?
\VS{5}Hypocrite, ôte premièrement de ton œil la poutre, et après cela tu verras comment tu ôteras la paille de l’œil de ton frère.
\VS{6}Ne donnez point les choses saintes aux chiens, et ne jetez point vos perles devant les pourceaux, de peur qu'ils ne les foulent aux pieds, ne se retournent et ne vous déchirent.
\TextTitle{[Encouragement à la prière]}
\VS{7}Demandez, et il vous sera donné. Cherchez, et vous trouverez. Frappez, et l’on vous ouvrira.
\VS{8}Car quiconque demande, reçoit ; et celui qui cherche trouve ; et l’on ouvre à celui qui frappe.
\VS{9}Lequel de vous donnera une pierre à son fils, s'il lui demande du pain ?
\VS{10}Ou, s’il lui demande un poisson, lui donnera-t-il un serpent ?
\VS{11}Si donc vous, méchants comme vous l’êtes, savez donner à vos enfants de bonnes choses, à combien plus forte raison votre Père qui est dans les cieux, donnera-t-il des bonnes choses à ceux qui les lui demandent ?
\TextTitle{[La règle d'or de loi et des prophètes]
\\(Lu. 6:31; cp. Ep.4:32)}
\VS{12}Tout ce que vous voulez que les hommes fassent pour vous, faites-le de même pour eux, car c’est la loi et les prophètes.
\TextTitle{[Les deux chemins]
\\(cp. Ps.1)}
\VS{13}Entrez par la porte étroite. Car large est la porte, spacieux est le chemin qui mène à la perdition, et il y en a beaucoup qui entrent par là.
\VS{14}Mais étroite est la porte, resserré le chemin qui mène à la vie, et il y en a peu qui les trouvent.
\TextTitle{[Les faux prophètes]
\\(Lu. 6:43-45)}
\VS{15}Gardez-vous des faux prophètes, ils viennent à vous en habits de brebis, mais au-dedans ce sont des loups ravisseurs.
\VS{16}Vous les reconnaîtrez à leurs fruits. Cueille-t-on des raisins sur des épines, ou des figues sur des chardons ?
\VS{17}Ainsi tout bon arbre porte de bons fruits ; mais le mauvais arbre porte de mauvais fruits.
\VS{18}Un bon arbre ne peut porter de mauvais fruits ni le mauvais arbre porter de bons fruits.
\VS{19}Tout arbre qui ne porte pas de bons fruits est coupé et jeté au feu.
\VS{20}Vous les reconnaîtrez donc à leurs fruits.
\TextTitle{[Fausse profession religieuse]
\\(Lu. 6:46)}
\VS{21}Ceux qui me disent : Seigneur ! Seigneur ! N’entreront pas tous dans le Royaume des cieux ; mais celui qui fait la volonté de mon Père qui est dans les cieux.
\VS{22}Plusieurs me diront en ce jour-là : Seigneur ! Seigneur ! N’avons-nous pas prophétisé en ton Nom ? N'avons-nous pas chassé les démons en ton Nom ? N'avons-nous pas fait beaucoup de miracles en ton Nom ?
\VS{23}Alors je leur dirai ouvertement : Je ne vous ai jamais connus. Retirez-vous de moi, vous qui commettez l'iniquité.
\TextTitle{[Parabole des deux constructeurs et des deux fondements]
\\(lu. 6:47-49)}
\VS{24}Quiconque entend ces paroles que je dis, et les met en pratique, je le comparerai à un homme prudent qui a bâti sa maison sur le roc.
\VS{25}La pluie est tombée, les torrents sont venus, les vents ont soufflé contre cette maison : Elle n'est point tombée, parce qu'elle était fondée sur le roc{\FTNT{Jésus-Christ le rocher : voir Es. 8:13-17.}}.
\VS{26}Mais quiconque entend ces paroles que je dis, et ne les met point en pratique, sera semblable à un homme insensé qui a bâti sa maison sur le sable.
\VS{27}La pluie est tombée, les torrents sont venus, les vents ont soufflé contre cette maison : Elle est tombée et sa ruine a été grande.
\TextTitle{[Effet du sermon]}
\VS{28}Après que Jésus eut achevé ce discours, la foule fut frappée de sa doctrine ;
\VS{29}car il enseignait comme ayant autorité, et non pas comme leurs scribes.
\TextTitle{[Jésus guérit un lépreux]
\\(Mc. 1:40-45}
\Chap{8}
\VerseOne{}Et quand il fut descendu de la montagne, de grandes foules le suivirent.
\VS{2}Et voici, un lépreux vint et se prosterna devant lui, en lui disant : Seigneur{\FTNT{Seigneur : Du grec «~kurios~». C’est la première fois que ce terme est appliqué à Jésus. Notez que c’est un lépreux qui a la révélation que Jésus-Christ est YHWH. Voir aussi le commentaire en Mt. 1:20.}}, si tu veux, tu peux me rendre pur.
\VS{3}Et Jésus étendit la main, le toucha, en disant : Je le veux, sois pur. A l’instant même il fut purifié de sa lèpre.
\VS{4}Puis Jésus lui dit : Prends garde de ne le dire à personne ; mais va te montrer au sacrificateur, et offre l’offrande que Moïse a prescrite, afin que cela leur serve de témoignage.
\TextTitle{[Guérison du serviteur d'un centenier]
\\(Lu. 7:1-10)}
\VS{5}Et quand Jésus fut entré dans Capernaüm, un centenier vint à lui, le priant
\VS{6}et disant : Seigneur, mon serviteur qui est paralytique est couché à la maison et il souffre extrêmement.
\VS{7}Jésus lui dit : J'irai, et je le guérirai.
\VS{8}Mais le centenier lui répondit : Seigneur, je ne suis pas digne que tu entres sous mon toit ; mais dis seulement une parole, et mon serviteur sera guéri.
\VS{9}Car moi-même qui suis un homme soumis à l’autorité d’un autre, j'ai des soldats sous mes ordres, et je dis à l'un : Va ! et il va ; et à un autre : Viens ! et il vient ; et à mon serviteur : Fais cela ! et il le fait.
\VS{10}Après l’avoir entendu, Jésus fut étonné et dit à ceux qui le suivaient : Je vous le dis en vérité, même en Israël je n'ai pas trouvé une aussi grande foi.
\VS{11}Or, je vous dis que plusieurs viendront de l’orient et de l’occident, et seront à table dans le Royaume des cieux, avec Abraham, Isaac et Jacob.
\VS{12}Et les enfants du Royaume seront jetés dans les ténèbres du dehors, où il y aura des pleurs et des grincements de dents.
\VS{13}Alors Jésus dit au centenier : Va, et qu'il te soit fait selon ta foi. Et à l'heure même, son serviteur fut guéri.
\TextTitle{[Guérison de la belle-mère de Pierre]
\\(Mc. 1:29-34 ; Lu. 4:38-41)}
\VS{14}Puis Jésus alla à la maison de Pierre, dont il vit la belle-mère couchée et ayant la fièvre.
\VS{15}Il toucha sa main, et la fièvre la quitta ; puis elle se leva, et les servit.
\VS{16}Et le soir étant venu, on lui amena plusieurs démoniaques. Et il chassa par sa parole les esprits malins, et guérit tous ceux qui étaient malades,
\VS{17}afin que s’accomplisse ce qui avait été annoncé par Esaïe le prophète, en disant : Il a pris nos faiblesses et a porté nos maladies{\FTNT{Es. 53:4.}}.
\TextTitle{[Mise à l'épreuve de la consécration des disciples]
\\(Lu. 9:57-62)}
\VS{18}Or Jésus voyant autour de lui de grandes foules, donna l’ordre de passer à l'autre rive.
\VS{19}Et un scribe s'approchant, lui dit : Maître, je te suivrai partout où tu iras.
\VS{20}Jésus lui dit : Les renards ont des tanières, et les oiseaux du ciel ont des nids ; mais le Fils de l'homme n'a pas de place pour reposer sa tête.
\VS{21}Puis un autre de ses disciples lui dit : Seigneur, permets-moi d'aller d’abord ensevelir mon père.
\VS{22}Et Jésus lui dit : Suis-moi, et laisse les morts ensevelir leurs morts.
\TextTitle{[Jésus apaise la tempête]
\\(Mc. 4:35-41 ; Lu 8:22-25)}
\VS{23}Il monta dans -la barque, et ses disciples le suivirent.
\VS{24}Et voici, il s'éleva sur la mer une si grande tempête que la barque était couverte de flots ; et Jésus dormait.
\VS{25}Et ses disciples vinrent le réveiller, en lui disant : Seigneur, sauve-nous, nous périssons !
\VS{26}Et il leur dit : Pourquoi avez-vous peur, gens de peu de foi ? Alors s'étant levé, il menaça les vents et la mer, et il se fit un grand calme.
\VS{27}Et les gens qui étaient là furent étonnés, et dirent : Qui est celui-ci à qui obéissent même les vents et la mer ?
\TextTitle{[Jésus chasse les démons à gadara]
\\(Mc. 5:1-20 ; Lu. 8:26-40)}
\VS{28}Et quand il fut passé de l'autre côté, dans le pays des Gadaréniens, deux démoniaques sortant des sépulcres, vinrent le rencontrer. Ils étaient si dangereux que personne ne pouvait passer par ce chemin-là.
\VS{29}Et voici, ils s'écrièrent : Qu'y a-t-il entre nous et toi, Jésus Fils de Dieu ? Es-tu venu ici nous tourmenter avant le temps ?
\VS{30}Et il y avait loin d'eux un grand troupeau de pourceaux qui paissaient.
\VS{31}Et les démons le priaient en disant : Si tu nous chasses dehors, permets-nous d’entrer dans ce troupeau de pourceaux.
\VS{32}Et il leur dit : Allez ! Et ils sortirent et entrèrent dans le troupeau de pourceaux. Et voici, tout le troupeau de pourceaux se précipita des pentes escarpées dans la mer, et ils périrent dans les eaux.
\VS{33}Ceux qui les gardaient s'enfuirent et allèrent dans la ville, ils racontèrent toutes ces choses, et ce qui était arrivé aux démoniaques.
\VS{34}Et voici, toute la ville alla à la rencontre de Jésus, et l'ayant vu, ils le prièrent de se retirer de leur pays.
\TextTitle{[Guérison d'un paralytique]
\\(Mc. 2:3-12 ; Lu. 5:18-26)}
\Chap{9}
\VerseOne{}Alors, étant monté dans une barque, il traversa la mer et vint dans sa ville.
\VS{2}Et voici, on lui présenta un paralytique couché sur un lit. Et Jésus voyant leur foi, dit au paralytique : Prends courage, mon enfant ! Tes péchés te sont pardonnés.
\VS{3}Et voici, quelques-uns des scribes disaient au dedans d’eux : Cet homme blasphème.
\VS{4}Mais Jésus, connaissant leurs pensées, leur dit : Pourquoi avez-vous de mauvaises pensées dans vos cœurs ?
\VS{5}Car lequel est le plus aisé de dire : Tes péchés te sont pardonnés ; ou de dire : Lève-toi, et marche ?
\VS{6}Or afin que vous sachiez que le Fils de l'homme a le pouvoir sur la terre de pardonner les péchés : Lève-toi, dit-il au paralytique, prends ton lit, et va dans ta maison.
\VS{7}Et il se leva, et s'en alla dans sa maison.
\VS{8}Quand la foule vit cela, elle fut saisie d’étonnement, et elle glorifiait Dieu qui a donné aux hommes un tel pouvoir.
\TextTitle{[Appel de Matthieu]
\\(Mc. 2:14 ; Lu. 5:27-28)}
\VS{9}De là, étant allé plus loin, Jésus vit un homme nommé Matthieu, assis au bureau du péage, et il lui dit : Suis-moi ; et il se leva, et le suivit.
\TextTitle{[Jésus appelle des pécheurs, non des justes]
\\(Mc. 2:15-20 ; Lu. 5:29-35)}
\VS{10}Comme Jésus était à table dans la maison de Matthieu, beaucoup de publicains et des gens de mauvaise vie, qui étaient venus là, se mirent à table avec Jésus et avec ses disciples.
\VS{11}Les pharisiens virent cela et ils dirent à ses disciples : Pourquoi votre Maître mange-t-il avec les publicains et les gens de mauvaise vie ?
\VS{12}Jésus l'ayant entendu, leur dit : Ce ne sont pas ceux qui sont en bonne santé qui ont besoin de médecin, mais les malades.
\VS{13}Mais allez et apprenez ce que veulent dire ces paroles : Je prends plaisir à la miséricorde, et non aux sacrifices{\FTNT{Os. 6:6.}}. Car je ne suis pas venu appeler à la repentance les justes, mais les pécheurs.
\VS{14}Alors les disciples de Jean vinrent auprès de lui et lui dirent : Pourquoi nous et les pharisiens jeûnons-nous souvent, tandis que tes disciples ne jeûnent point ?
\VS{15}Et Jésus leur répondit : Les amis de l’époux peuvent-ils s'affliger pendant que l’époux est avec eux ? Mais les jours viendront où l’époux leur sera enlevé, alors ils jeûneront.
\TextTitle{[Parabole du drap neuf et des outres neuves]
\\(Mc. 2:21-22 ; Lu. 5:36-39)}
\VS{16}Aussi personne ne met une pièce de drap neuf à un vieil habit, car la pièce emporterait une partie de l’habit, et la déchirure serait pire.
\VS{17}On ne met pas non plus du vin nouveau dans de vieilles outres ; autrement les outres se rompent, et le vin se répand, et les outres sont perdues ; mais on met le vin nouveau dans des outres neuves, et l'un et l'autre se conservent.
\TextTitle{[Résurrection de la fille de Jaïrus et guérison de la femme à la perte de sang]
\\(Mc. 5:21-43 ; Lu. 8:41-56)}
\VS{18}Tandis qu’il leur disait ces choses, voici, arriva un chef qui se prosterna devant lui, en lui disant : Ma fille est morte il y a un instant, mais viens, et impose-lui ta main, et elle vivra.
\VS{19}Et Jésus s'étant levé le suivit avec ses disciples.
\VS{20}Et voici, une femme atteinte d'une perte de sang depuis douze ans s’approcha par-derrière, et toucha le bord de son vêtement.
\VS{21}Car elle disait en elle-même : Si je puis seulement toucher son vêtement, je serai guérie.
\VS{22}Et Jésus se retourna, et dit en la voyant : Prends courage, ma fille ! Ta foi t'a sauvée. Et cette femme fut guérie à l’heure même.
\VS{23}Lorsque Jésus fut arrivé à la maison du chef, et qu'il vit les joueurs de flûte et une foule bruyante,
\VS{24}il leur dit : Retirez-vous, car la jeune fille n'est pas morte, mais elle dort ; et ils se moquaient de lui.
\VS{25}Quand la foule eut été renvoyée, il entra, prit la main de la jeune fille, et elle se leva.
\VS{26}Et le bruit s'en répandit dans toute la contrée.
\TextTitle{[Guérison de deux aveugles et d'un démoniaque]}
\VS{27}Etant parti de là, Jésus fut suivi par deux aveugles, qui criaient : Fils de David, aie pitié de nous !
\VS{28}Et quand il fut arrivé dans la maison, les aveugles s’approchèrent de lui, et Jésus leur dit : Croyez-vous que je puisse faire ce que vous me demandez ? Ils lui répondirent : Oui, Seigneur !
\VS{29}Alors il toucha leurs yeux en disant : Qu'il vous soit fait selon votre foi.
\VS{30}Et leurs yeux s’ouvrirent. Alors Jésus leur dit sévèrement : Prenez garde que personne ne le sache.
\VS{31}Mais, dès qu’ils furent sortis, ils répandirent sa renommée dans tout le pays.
\VS{32}Comme ils s’en allaient, voici, on présenta à Jésus un homme muet et démoniaque.
\VS{33}Et le démon ayant été chassé, le muet parla ; et les foules étonnées disaient : Jamais pareille chose ne s’est vue en Israël.
\VS{34}Mais les pharisiens disaient : Il chasse les démons par le prince des démons.
\VS{35}Jésus allait dans toutes les villes et les villages, enseignant dans leurs synagogues, et prêchant l'Evangile du Royaume, et guérissant toutes sortes de maladies et toutes sortes d'infirmités parmi le peuple.
\TextTitle{[Jésus ému de compassion pour la foule]
\\(Mc. 6:34)}
\VS{36}Et voyant les foules, il fut ému de compassion, parce qu'elles étaient dispersées et errantes comme des brebis qui n'ont point de pasteur.
\VS{37}Et il dit à ses disciples : La moisson est grande, mais il y a peu d'ouvriers.
\VS{38}Priez donc le Maître de la moisson d’envoyer des ouvriers dans sa moisson.
\TextTitle{[Appel et Mission des douze apôtres]
\\(Mc. 6:7-13 ; Lu. 9:1-6)}
\Chap{10}
\VerseOne{}Alors Jésus ayant appelé ses douze disciples, leur donna le pouvoir de chasser les esprits impurs et de guérir toutes sortes de maladies et toutes sortes d'infirmités.
\VS{2}Et voici les noms des douze apôtres : Le premier est Simon, nommé Pierre, et André son frère ; Jacques, fils de Zébédée, et Jean, son frère ;
\VS{3}Philippe et Barthélemy ; Thomas, et Matthieu le péager ; Jacques, fils d'Alphée, et Lebbée, surnommé Thaddée.
\VS{4}Simon le Cananite, et Judas Iscariot, celui qui le livra.
\VS{5}Tels sont les douze que Jésus envoya, et leur donna ses ordres en disant : N'allez point vers les gentils, et n'entrez point dans aucune ville des Samaritains ;
\VS{6}Mais allez plutôt vers les brebis perdues de la maison d'Israël.
\VS{7}Et quand vous serez partis, prêchez, en disant : Le Royaume des cieux est proche.
\VS{8}Guérissez les malades, purifiez les lépreux, ressuscitez les morts, chassez les démons. Vous avez reçu gratuitement, donnez gratuitement{\FTNT{Vous avez reçu gratuitement : Aucun chrétien, quel que soit son appel ou son don ne peut prétendre qu’il a payé pour avoir les talents qu’il a reçus du Seigneur. Dans 1 Co. 4:7 Paul nous pose une question : «~Qu’as-tu que tu n’aies reçu et si tu l’as reçu pourquoi te glorifies-tu ?~» Dieu interroge également Job : «~De qui suis-je le débiteur ?~» (Job 41:2). Vendre quelque chose qu’on a reçu gratuitement n’est rien d’autre que du vol. Donnez gratuitement : C’est la suite logique des choses, on reçoit gratuitement et on donne gratuitement. Si nous aimons Dieu, nous devons garder sa Parole et marcher comme lui a marché (Jn. 14:15 ; 1 Jn. 2:6). Il a donné ses enseignements et nourri les gens gratuitement. Dans Ap. 21:6 et 22:17, le Seigneur invite toutes les personnes qui ont soif à venir s’abreuver gratuitement. Alors pourquoi vendre la parole qu’on a reçue gratuitement ? Le Seigneur a envoyé les douze en mission et leur a demandé d’apporter l’évangile du Royaume, de guérir les malades et de délivrer les possédés gratuitement (Ac. 8:18-24 ; Ac. 20:33-35 ; Ap. 21:6 ; Ap. 22:17).}}.
\VS{9}Ne prenez ni or, ni argent, ni monnaie dans vos ceintures ;
\VS{10}ni de sac pour le voyage, ni deux tuniques, ni souliers, ni bâton ; car l'ouvrier mérite sa nourriture.
\VS{11}Et dans quelque ville ou village que vous entriez, informez-vous qui y est digne de vous loger ; et demeurez chez lui jusqu'à ce que vous partiez de là.
\VS{12}Et quand vous entrerez dans quelque maison, saluez-la.
\VS{13}Et si cette maison en est digne, que votre paix vienne sur elle ; mais si elle n'en est pas digne, que votre paix retourne à vous.
\VS{14}Mais lorsque quelqu'un ne vous recevra point et n'écoutera point vos paroles, secouez, en partant de cette maison ou de cette ville, la poussière de vos pieds.
\VS{15}Je vous dis en vérité que ceux du pays de Sodome et de Gomorrhe seront traités moins rigoureusement au jour du jugement que cette ville-là.
\TextTitle{[L'Evangile du royaume doit être proclamé avant le retour de Christ]}
\VS{16}Voici, je vous envoie comme des brebis au milieu des loups ; soyez donc prudents comme des serpents, et simples comme des colombes.
\VS{17}Et mettez-vous en garde contre les hommes ; car ils vous livreront aux tribunaux, et vous battront de verges dans leurs synagogues.
\VS{18}Et vous serez menés devant des gouverneurs, et même devant des rois, à cause de moi, pour rendre témoignage de moi devant eux et aux nations.
\VS{19}Mais, quand ils vous livreront, ne vous inquiétez pas de ce que vous aurez à dire, ni comment vous parlerez. Ce que vous aurez à dire vous sera donné à l’heure même.
\VS{20}Car ce n'est pas vous qui parlez, mais c'est l'Esprit de votre Père qui parlera en vous.
\VS{21}Le frère livrera son frère à la mort, et le père son enfant ; et les enfants s'élèveront contre leurs pères et leurs mères, et les feront mourir.
\VS{22}Et vous serez haïs de tous à cause de mon Nom ; mais celui qui persévérera jusqu’à la fin sera sauvé.
\VS{23}Quand ils vous persécuteront dans une ville, fuyez dans une autre. Je vous le dis en vérité, vous n'aurez pas achevé de parcourir toutes les villes d'Israël, que le Fils de l'homme sera venu.
\TextTitle{[Prix de la consécration du disciple et sa récompense]}
\VS{24}Le disciple n'est pas plus que le maître, ni le serviteur plus que son Seigneur.
\VS{25}Il suffit au disciple d'être traité comme son maître, et au serviteur comme son Seigneur. S’ils ont appelé le père de famille Béelzébul, à combien plus forte raison appelleront-ils ainsi ses domestiques ?
\VS{26}Ne les craignez donc point. Car il n'y a rien de caché qui ne doive être découvert, ni rien de secret qui ne doive être connu.
\VS{27}Ce que je vous dis dans les ténèbres, dites-le dans la lumière ; et ce que je vous dis à l'oreille, prêchez-le sur les toits.
\VS{28}Et ne craignez point ceux qui tuent le corps, et qui ne peuvent tuer l'âme ; mais craignez plutôt celui qui peut faire périr et l'âme et le corps en les jetant dans la géhenne.
\VS{29}Ne vend-on pas deux passereaux pour un sou ? Cependant, il n’en tombe pas un à terre sans la volonté de votre Père.
\VS{30}Et même les cheveux de votre tête sont tous comptés.
\VS{31}Ne craignez donc point : Vous valez plus que beaucoup de passereaux.
\VS{32}Quiconque donc me confessera devant les hommes, je le confesserai aussi devant mon Père qui est aux cieux.
\VS{33}Mais quiconque me reniera devant les hommes, je le renierai aussi devant mon Père qui est dans les cieux.
\VS{34}Ne croyez pas que je sois venu apporter la paix sur la terre. Je ne suis pas venu apporter la paix, mais l'épée.
\VS{35}Car je suis venu mettre en division le fils contre son père, et la fille contre sa mère, et la belle-fille contre sa belle-mère.
\VS{36}Et les propres domestiques d'un homme seront ses ennemis.
\VS{37}Celui qui aime son père ou sa mère plus que moi, n'est pas digne de moi ; et celui qui aime son fils ou sa fille plus que moi, n'est pas digne de moi.
\VS{38}Et quiconque ne prend pas sa croix, et ne vient pas après moi, n'est pas digne de moi.
\VS{39}Celui qui aura conservé sa vie la perdra ; mais celui qui aura perdu sa vie pour l'amour de moi la retrouvera.
\VS{40}Celui qui vous reçoit me reçoit, et celui qui me reçoit, reçoit celui qui m'a envoyé.
\VS{41}Celui qui reçoit un prophète en qualité de prophète, recevra la récompense d'un prophète ; et celui qui reçoit un juste en qualité de juste recevra la récompense d'un juste.
\VS{42}Et quiconque aura donné à boire seulement un verre d'eau froide à l’un de ces petits parce qu’il est mon disciple, je vous le dis en vérité qu'il ne perdra point sa récompense.
\TextTitle{[Réponse de Jésus-Christ aux disciples de Jean-Baptiste]}
\Chap{11}
\VerseOne{}Lorsque Jésus eut achevé de donner ses ordres à ses douze disciples, il partit de là pour aller enseigner et prêcher dans leurs villes.
\VS{2}Jean, ayant entendu parler dans sa prison des œuvres du Christ, envoya deux de ses disciples pour lui dire :
\VS{3}Es-tu celui qui devait venir, ou devons-nous en attendre un autre ?
\VS{4}Et Jésus leur répondit : Allez, et rapportez à Jean les choses que vous entendez et que vous voyez.
\VS{5}Les aveugles recouvrent la vue, les boiteux marchent, les lépreux sont purifiés, les sourds entendent, les morts sont ressuscités, et l'Evangile est annoncé aux pauvres{\FTNT{Jésus-Christ est le Dieu véritable dont la venue était annoncée par Esaïe (Es. 35:4-6).}}.
\VS{6}Mais, heureux celui pour qui je ne serai pas une occasion de chute !
\VS{7}Et comme ils s'en allaient, Jésus se mit à dire à la foule au sujet de Jean : Mais qu'êtes-vous allés voir dans le désert ? Un roseau agité par le vent ?
\VS{8}Mais qu'êtes-vous allés voir ? Un homme vêtu de précieux vêtements ? Voici, ceux qui portent des habits précieux sont dans les maisons des rois.
\VS{9}Mais qu'êtes-vous allés voir ? Un prophète ? Oui, vous dis-je, et plus qu'un prophète.
\VS{10}Car c’est celui dont il est écrit : Voici, j'envoie mon messager{\FTNT{Mal. 3:1.}} devant ta face, pour préparer ton chemin devant toi.
\VS{11}En vérité, je vous le dis, parmi ceux qui sont nés de femmes, il n'en a point paru de plus grand que Jean-Baptiste. Cependant, le plus petit dans le Royaume des cieux, est plus grand que lui.
\VS{12}Or depuis le temps de Jean-Baptiste jusqu’à maintenant, le Royaume des cieux est forcé, et ce sont les violents qui s’en emparent.
\VS{13}Car tous les prophètes et la loi ont prophétisé jusqu'à Jean.
\VS{14}Et si vous voulez recevoir mes paroles, c'est lui qui est l'Elie{\FTNT{Mal. 4:5-6.}} qui devait venir.
\VS{15}Que celui qui a des oreilles pour entendre, entende.
\VS{16}Mais à qui comparerai-je cette génération ? Elle est semblable aux petits-enfants qui sont assis sur les places publiques, et qui crient à leurs compagnons
\VS{17}et leur disent : Nous vous avons joué de la flûte, et vous n'avez point dansé ; nous vous avons chanté des complaintes, et vous ne vous êtes point lamentés.
\VS{18}Car Jean est venu ne mangeant ni ne buvant et ils disent : Il a un démon.
\VS{19}Le Fils de l'homme est venu mangeant et buvant et ils disent : C’est un mangeur et un buveur, un ami des publicains et des gens de mauvaise vie. Mais la sagesse a été justifiée par ses enfants.
\TextTitle{[Jésus dénonce les indifférents]}
\VS{20}Alors il se mit à faire des reproches aux villes où il avait fait beaucoup de miracles, parce qu’elles ne s'étaient point repenties.
\VS{21}Malheur à toi, Chorazin ! Malheur à toi, Bethsaïda ! Car si les miracles qui ont été faits au milieu de vous, avaient été faits dans Tyr et dans Sidon, il y a longtemps qu'elles se seraient repenties, en prenant le sac et la cendre.
\VS{22}C'est pourquoi je vous dis que Tyr et Sidon seront traitées moins rigoureusement que vous, au jour du jugement.
\VS{23}Et toi Capernaüm, qui as été élevée jusqu’au ciel, tu seras abaissée jusqu’en enfer ; car si les miracles qui ont été faits au milieu de toi, avaient été faits dans Sodome, elle subsisterait encore.
\VS{24}C'est pourquoi je vous dis que ceux de Sodome seront traités moins rigoureusement que toi, au jour du jugement.
\TextTitle{[La relation personnelle du disciple avec son Seigneur]}
\VS{25}En ce temps-là Jésus prenant la parole dit : Je te loue, ô mon Père ! Seigneur du ciel et de la terre, de ce que tu as caché ces choses aux sages et aux intelligents, et que tu les as révélées aux petits enfants.
\VS{26}Oui, Père, je te loue parce que telle a été ta bonne volonté.
\VS{27}Toutes choses m'ont été données par mon Père ! Et personne ne connaît le Fils si ce n’est le Père ; et personne ne connaît le Père si ce n’est le Fils, et celui à qui le Fils veut le révéler.
\VS{28}Venez à moi vous tous qui êtes fatigués et chargés, et je vous donnerai du repos.
\VS{29}Prenez mon joug sur vous et recevez mes instructions, car je suis doux et humble de cœur ; et vous trouverez le repos pour vos âmes.
\VS{30}Car mon joug est doux et mon fardeau est léger.
\TextTitle{[Jésus est maître du sabbat]
\\(Mc. 2:23-28 ; Lu. 6:1-5)}
\Chap{12}
\VerseOne{}En ce temps-là, Jésus traversa des champs de blé un jour de sabbat. Et ses disciples qui avaient faim se mirent à arracher des épis et à les manger.
\VS{2}Les pharisiens voyant cela, lui dirent : Voici, tes disciples font ce qu'il n'est pas permis de faire le jour du sabbat.
\VS{3}Mais il leur dit : N'avez-vous pas lu ce que fit David quand il eut faim, lui et ceux qui étaient avec lui ?
\VS{4}Comment il entra dans la maison de Dieu, et mangea les pains de proposition, qu’il ne lui était pas permis de manger ni à lui, ni à ceux qui étaient avec lui, mais aux sacrificateurs seulement ?
\VS{5}Ou n'avez-vous pas lu dans la loi, qu'aux jours du sabbat les sacrificateurs violent le sabbat dans le temple, sans se rendre coupables ?
\VS{6}Or, je vous le dis, qu'il y a ici quelqu'un de plus grand que le temple.
\VS{7}Si vous saviez ce que signifient ces paroles : Je prends plaisir à la miséricorde, et non aux sacrifices, vous n'auriez pas condamné des innocents{\FTNT{1 S. 15:22 ; Os. 6:6.}}.
\VS{8}Car le Fils de l'homme est Maître du sabbat.
\TextTitle{[Jésus guérit l'homme à la main sèche le jour du sabbat]
\\(Mc. 3:1-5 ; Lu. 6:6-11)}
\VS{9}Puis étant parti de là, il entra dans leur synagogue.
\VS{10}Et voici, il s’y trouvait un homme qui avait la main sèche. Et pour avoir sujet de l'accuser, ils l'interrogèrent en disant : Est-il permis de guérir les jours du sabbat ?
\VS{11}Et il répondit : Lequel d’entre vous s'il n’a qu’une brebis, et qu'elle vienne à tomber dans une fosse le jour du sabbat, ne la saisira-t-il pas pour l’en retirer ?
\VS{12}Combien un homme ne vaut-il pas plus qu'une brebis ! Il est donc permis de faire du bien les jours du sabbat.
\VS{13}Alors il dit à cet homme : Etends ta main. Il l'étendit et elle devint saine comme l'autre.
\TextTitle{[Jésus accomplit de nombreuses guérisons]}
\VS{14}Les pharisiens sortirent et ils se consultèrent sur les moyens de le faire périr.
\VS{15}Mais Jésus, l’ayant su, partit de là, et de grandes foules le suivirent. Il les guérit tous
\VS{16}et il leur recommanda sévèrement de ne pas le faire connaître,
\VS{17}afin que s’accomplisse ce qui avait été annoncé par Esaïe le prophète :
\VS{18}Voici mon serviteur que j'ai élu, mon bien-aimé, qui est l'objet de mon amour, je mettrai mon Esprit en lui, et il annoncera le jugement aux nations.
\VS{19}Il ne contestera point, il ne criera point, et personne n'entendra sa voix dans les rues.
\VS{20}Il ne brisera point le roseau cassé, et n'éteindra point le lumignon qui fume, jusqu'à ce qu'il ait fait triompher la justice.
\VS{21}Et les nations espéreront en son nom{\FTNT{Es. 42:1-4.}}.
\VS{22}Alors on lui amena un homme tourmenté d'un démon, aveugle et muet, et il le guérit ; de sorte que celui qui avait été aveugle et muet, parlait et voyait.
\VS{23}Et toutes les foules en furent étonnées, et elles disaient : Celui-ci n'est-il pas le Fils de David ?
\TextTitle{[Les pharisiens attribuent à Satan les oeuvres du Saint-Esprit]
\\(Mc. 3:22-30 ; Lu. 11:15-23)}
\VS{24}Mais les pharisiens ayant entendu cela, disaient : Celui-ci ne chasse les démons que par Béelzébul, prince des démons.
\VS{25}Mais Jésus connaissant leurs pensées, leur dit : Tout royaume divisé contre lui-même sera réduit en désert ; et toute ville, ou maison, divisée contre elle-même ne subsistera point.
\VS{26}Si Satan chasse Satan, il est divisé contre lui-même ; comment donc son royaume subsistera-t-il ?
\VS{27}Et si je chasse les démons par Béelzébul, par qui vos fils les chassent-ils ? C’est pourquoi ils seront eux-mêmes vos juges.
\VS{28}Mais si je chasse les démons par l'Esprit de Dieu, certes le Royaume de Dieu est donc venu jusqu'à vous.
\VS{29}Ou, comment quelqu'un peut-il entrer dans la maison d'un homme fort et piller ses biens, sans avoir auparavant lié cet homme fort ? Alors il pillera sa maison.
\VS{30}Celui qui n'est pas avec moi, est contre moi, et celui qui n'assemble pas avec moi disperse.
\TextTitle{[Blasphème contre le Saint-Esprit]
\\(Mc. 3:28-30)}
\VS{31}C'est pourquoi je vous dis, que tout péché et tout blasphème sera pardonné aux hommes ; mais le blasphème contre l'Esprit ne leur sera point pardonné.
\VS{32}Quiconque parlera contre le Fils de l'homme, il lui sera pardonné ; mais quiconque parlera contre le Saint-Esprit, il ne lui sera pardonné ni dans ce siècle, ni dans le siècle à venir{\FTNT{Le blasphème contre le Saint-Esprit : Le blasphème est un outrage, une calomnie à l’encontre de Dieu. En attribuant l’œuvre de Dieu à Satan, les pharisiens ont commis l’impardonnable. Beaucoup de personnes craignent d’avoir commis ce péché par inadvertance, en ayant par exemple un doute sur l’origine d’un miracle. La Parole nous recommande de ne pas ajouter foi à tout esprit, mais d’éprouver les esprits pour savoir s’ils sont de Dieu (1 Jn. 4:1). On ne pèche donc pas lorsqu’on exerce son discernement. De plus, si l’on a commis une erreur de jugement par ignorance, le Seigneur ne nous en tiendra pas rigueur (Ac. 17:30). Le blasphème contre le Saint-Esprit est commis par des personnes qui, bien qu’ayant la connaissance et la capacité de différencier le bien du mal, font preuve de mauvaise foi. Ainsi, les pharisiens avaient constaté les bons fruits portés par Jésus, mais ils ont hypocritement qualifié de mal le bien qu’il faisait (Es. 5:20). Ceux qui blasphèment contre le Saint-Esprit sont loin d’être ignorants. Comme nous l’atteste Hé. 6:4-6, parmi ces gens, certains «~ont goûté le don céleste~», «~ont eu part au Saint-Esprit, et ont goûté la bonne parole de Dieu, et les puissances du siècle à venir~». En choisissant sciemment de pécher, alors qu’ils ont expérimenté la grâce de Dieu, ils retournent à ce qu’ils ont vomi et outragent ainsi le Seigneur (2 Pi. 2:18-22). Leur cœur endurci à l’extrême rejette volontairement la vérité pour s’attacher au mensonge. Constatant leur refus définitif de se repentir, le Saint-Esprit finit par se retirer pour laisser la place à l’esprit d’égarement qui les maintiendra dans l’erreur (2 Th. 2:9-12). Enfin, il est à noter qu’en Ap. 14:9-11, ceux qui ont reçu la marque de la bête sont condamnés d’office. Il ne faut nullement conclure que le Seigneur leur a refusé son pardon, mais plutôt que les personnes ayant reçu cette marque ont aussi blasphémé contre le Saint-Esprit. Voir commentaire en Ap. 13:16.}}.
\TextTitle{[Toute parole appelle un jugement]}
\VS{33}Ou dites que l'arbre est bon, et son fruit est bon ; ou dites que l'arbre est mauvais, et son fruit est mauvais ; car on connaît l'arbre par le fruit.
\VS{34}Race de vipères, comment pourriez-vous dire de bonnes choses, méchants comme vous l’êtes ? Car c’est de l'abondance du cœur que la bouche parle.
\VS{35}L'homme bon tire de bonnes choses du bon trésor de son cœur ; et l'homme méchant tire de mauvaises choses du mauvais trésor de son cœur.
\VS{36}Je vous le dis : Les hommes rendront compte au jour du jugement, de toute parole vaine qu'ils auront proférée.
\VS{37}Car tu seras justifié par tes paroles, et tu seras condamné par tes paroles.
\TextTitle{[Le miracle du prophète Jonas]
\\(Lu. 11:29-32 ; cp. Jon. 2:1)}
\VS{38}Alors quelques-uns des scribes et des pharisiens lui dirent : Maître, nous voudrions bien te voir faire quelque miracle.
\VS{39}Mais il leur répondit et dit : Une génération méchante et adultère demande un miracle, mais il ne lui sera point donné d'autre miracle que celui de Jonas le prophète.
\VS{40}Car, de même que Jonas fut trois jours et trois nuits dans le ventre d'un grand poisson, de même le Fils de l'homme sera trois jours et trois nuits dans le sein de la terre.
\VS{41}Les Ninivites se lèveront au jour du jugement contre cette génération et la condamneront, parce qu'ils se repentirent à la prédication de Jonas ; et voici, il y a ici plus que Jonas.
\TextTitle{[La reine de Séba condamnera cette génération]
\\(cp. 2 ch. 9:1-12)}
\VS{42}La reine du Midi se lèvera au jour du jugement contre cette nation et la condamnera, parce qu'elle vint des extrémités de la terre pour entendre la sagesse de Salomon ; et voici, il y a ici plus que Salomon.
\TextTitle{[Le retour de l'esprit impur]
\\(Lu. 11:24-26)}
\VS{43}Lorsque l'esprit impur est sorti d'un homme, il va par des lieux arides, cherchant du repos, mais il n'en trouve point.
\VS{44}Alors il dit : Je retournerai dans ma maison, d'où je suis sorti ; et quand il arrive, il la trouve vide, balayée et ornée.
\VS{45}Puis il s'en va et il prend avec lui sept autres esprits plus méchants que lui ; ils entrent dans la maison, s’y établissent, et la dernière condition de cet homme est pire que la première. Il en sera de même pour cette génération perverse.
\TextTitle{[Les nouvelles relations de la famille spirituelle]
\\(Mc. 3:31-35 ; Lu. 8:19-21)}
\VS{46}Et comme il parlait encore à la foule, voici, sa mère et ses frères qui étaient dehors cherchèrent à lui parler.
\VS{47}Et quelqu'un lui dit : Voici, ta mère et tes frères sont dehors et ils cherchent à te parler.
\VS{48}Mais il répondit à celui qui lui avait dit cela : Qui est ma mère, et qui sont mes frères ?
\VS{49}Et étendant sa main sur ses disciples, il dit : Voici ma mère et mes frères.
\VS{50}Car, quiconque fait la volonté de mon Père qui est dans les cieux, celui-là est mon frère, et ma sœur, et ma mère.
\TextTitle{[MYSTERES DU ROYAUME
\\1. Parabole du semeur et des quatre terrains]}
\Chap{13}
\VerseOne{}Ce même jour, Jésus sortit de la maison et s'assit au bord de la mer.
\VS{2}Une grande foule s'assembla auprès de lui, c’est pourquoi il monta dans une barque, et il s'assit. Aussi, toute la foule se tenait sur le rivage.
\VS{3}Et il leur parla en paraboles sur beaucoup de choses, et il dit : Un semeur sortit pour semer.
\VS{4}Et comme il semait, une partie de la semence tomba le long du chemin, et les oiseaux vinrent, et la mangèrent toute.
\VS{5}Et une autre partie tomba dans les endroits pierreux, où elle n'avait pas beaucoup de terre : Elle leva aussitôt, parce qu'elle n'entrait pas profondément dans la terre ;
\VS{6}mais, quand le soleil parut, elle fut brûlée et sécha parce qu'elle n'avait point de racines.
\VS{7}Une autre partie tomba parmi les épines ; et les épines montèrent et l'étouffèrent.
\VS{8}Une autre partie tomba dans la bonne terre : Et elle donna du fruit, un grain en donna cent, un autre, soixante, et un autre, trente.
\VS{9}Que celui qui a des oreilles pour entendre, qu'il entende.
\TextTitle{[Explication aux disciples]}
\VS{10}Alors les disciples s'approchèrent et lui dirent : Pourquoi leur parles-tu en paraboles ?
\VS{11}Il leur répondit et dit : Parce qu'il vous a été donné de connaître les mystères du Royaume des cieux, et que cela ne leur a pas été donné de les connaître.
\VS{12}Car on donnera à celui qui a, et il sera dans l’abondance, mais à celui qui n'a pas, on ôtera même ce qu’il a.
\VS{13}C'est pourquoi je leur parle en paraboles, parce qu'en voyant ils ne voient point, et qu'en entendant ils n'entendent point et ne comprennent point.
\VS{14}Et ainsi s'accomplit pour eux la prophétie d'Esaïe qui dit : Vous entendrez de vos oreilles, et vous ne comprendrez point ; et vous regarderez des yeux, et vous ne verrez point.
\VS{15}Car le cœur de ce peuple est engraissé, et ils ont endurci leurs oreilles, et ils ont fermé leurs yeux, de peur qu'ils ne voient de leurs yeux, qu'ils n'entendent de leurs oreilles, qu'ils ne comprennent de leur cœur, qu’ils ne se convertissent, et que je ne les guérisse{\FTNT{Es. 6:9-10.}}.
\VS{16}Mais heureux sont vos yeux, car ils voient ; et vos oreilles, parce qu’elles entendent.
\VS{17}Je vous le dis en vérité, beaucoup de prophètes et de justes ont désiré voir les choses que vous voyez, et ils ne les ont point vues, entendre les choses que vous entendez, et ils ne les ont point entendues.
\VS{18}Vous donc, écoutez la signification de la parabole du semeur.
\VS{19}Lorsqu’un homme écoute la parole du Royaume, et ne la comprend pas, le malin vient et ravit ce qui est semé dans son cœur ; cet homme est celui qui a reçu la semence le long du chemin.
\VS{20}Et celui qui a reçu la semence dans les endroits pierreux, c'est celui qui entend la parole et la reçoit aussitôt avec joie ;
\VS{21}mais il n'a point de racine en lui-même, il croit pour un temps, et dès que survient une tribulation ou une persécution à cause de la parole, il y trouve une occasion de chute.
\VS{22}Et celui qui a reçu la semence parmi les épines, c'est celui qui entend la parole de Dieu, mais en qui les soucis du siècle et la séduction des richesses étouffent la parole, et la rendent infructueuse.
\VS{23}Mais celui qui a reçu la semence dans la bonne terre, c'est celui qui entend la parole et la comprend. Il porte du fruit, et un grain donne cent, un autre soixante, et un autre trente{\FTNT{La parabole du semeur met en évidence trois niveaux de croissance spirituelle chez celui qui reçoit et garde la Parole de Dieu. Les chiffres trente, soixante et cent ont chacun une valeur numérique correspondant à une lettre spécifique de l’alphabet hébraïque. 30 a pour valeur numérique la lettre hébraïque ל (lamed). Cette lettre en forme de bâton de berger au bout recourbé ou encore de fouet, est la douzième de l’alphabet hébraïque. Le chiffre douze est associé au fondement : les douze tribus d’Israël (Ge. 49:1-28), les douze apôtres de Jésus (Mt. 10:2-4), les douze pierres précieuses ornant les fondements de la Nouvelle Jérusalem et ses douze portes qui sont des perles (Ap. 21:18-21). Or nous sommes bâtis sur le fondement, c’est-à-dire l’enseignement, des apôtres et des prophètes (Ep. 2:20). La stabilité d’une maison dépend de ses fondations. Ainsi, comme nous l’enseigne le Seigneur, celui qui écoute et met en pratique sa Parole est semblable à un homme prudent qui a bâti sa maison sur le roc (Mt. 7:24-25), ce roc étant Jésus lui-même (voir commentaires (1) et (2) en Mt. 16). A l’inverse, celui qui la néglige est comparé à un insensé qui bâtit sur le sable, exposant ainsi sa maison, et donc sa propre vie, à une ruine certaine (Mt. 7:26-27). Le fondement symbolise également les racines. De la même manière qu’il est impossible qu’un arbre subsiste sans racines, le chrétien qui ne s’est pas enraciné dans le Seigneur sera incapable de persévérer dans la foi et de porter du fruit (Lu. 8:13 ; Jn. 15:5). C’est donc parce que nous sommes portés par la sainte racine qu’est Jésus-Christ que nous sommes saints (Ro. 11:16-18 ; Ro. 6:22). Le fouet symbolise l’instruction qui ne peut être dissociée de la correction que le Seigneur nous inflige pour notre bien (Pr. 22:6 ; Jn. 2:13-16 ; Hé. 12:10 ; 2 Ti. 3:16). 60 a pour valeur numérique la lettre ס (samèkh). Le chiffre 60 nous parle du tronc, du pilier ou encore de la colonne. L’Eglise de Christ est la colonne et l’appui de la vérité (1 Ti. 3:15). Les apôtres Jacques, Pierre et Jean étaient regardés comme des colonnes (Ga. 2:9). Chaque chrétien est appelé à devenir à son tour une colonne du temple de Dieu (Ap. 3:12). Ainsi, la croissance du disciple est comparée à celle d'un arbre. Elle commence d’abord au niveau des racines et se poursuit avec le tronc de l’arbre qui porte les branches et leurs fruits. 100 correspond à la lettre ך (kaph), laquelle est composée des lettres ו (vav) et ק (qof) qui ont respectivement pour valeurs les chiffres 6 et 20. Le tétragramme YHWH a pour valeur numérique 26 (la somme de 6 et 20). Ainsi, lorsque le disciple multiplie les fruits de l’esprit, ces derniers l'amènent à la connaissance du nom de Dieu, à savoir Jésus. La connaissance de Dieu est par conséquent la finalité de la marche chrétienne. Pour y parvenir, chaque disciple doit nécessairement passer par les niveaux précédents. Jésus est la tête du corps de l’Eglise (Col. 1:18 ; 1 Co. 11:3), le toit de la maison de Dieu et l’arbre de vie qui porte du fruit chaque mois (Ap. 22:2). Ainsi, lorsque l’Eglise aura la révélation de Jésus-Christ, elle portera toutes sortes de fruits. Le Seigneur viendra les cueillir lorsqu’ils seront mûrs, ce sera alors l’enlèvement de l’Eglise.}}.
\TextTitle{[2. Parabole du blé et de l'ivraie]}
\VS{24}Il leur proposa une autre parabole et il dit{\FTNT{La Parabole du blé et de l’ivraie. En méditant cette parabole, nous remarquons que lorsque le blé eut poussé et donné du fruit, l’ivraie parut aussi. Il est vrai que lorsqu’il y a un réveil spirituel divin dans une assemblée ou dans un pays, l’ennemi suscite aussi un faux réveil avec des faux ouvriers et des fausses manifestations spirituelles. Voilà pourquoi l’ivraie côtoiera le blé jusqu’à la fin du monde. Le mot «~ivraie~» se dit «~ebriacus~» en latin, ce qui donne «~ébriété~» en français. Nous comprenons donc que l’un des rôles de l’ivraie est d’enivrer le blé (les enfants de Dieu). Dans les Ecritures, l’ivresse est synonyme de la débauche spirituelle ou physique. En grec l’ivraie se dit «~zizanion~» qui donne en français «~zizanie~». Voir Mt. 12:25. La division est l’œuvre de l’ivraie dans les églises qui cherche à créer des sectes et des partis pris.}} : Le Royaume des cieux est semblable à un homme qui a semé de la bonne semence dans son champ.
\VS{25}Mais, pendant que les hommes dormaient, son ennemi vint, sema de l'ivraie parmi le blé, puis s'en alla.
\VS{26}Lorsque l’herbe eut poussé et donné du fruit, l’ivraie parut aussi.
\VS{27}Et les serviteurs du maître de la maison vinrent à lui et lui dirent : Seigneur, n'as-tu pas semé de la bonne semence dans ton champ ? D’où vient donc qu'il y a de l'ivraie ?
\VS{28}Mais il leur répondit : C’est un ennemi qui a fait cela. Et les serviteurs lui dirent : Veux-tu donc que nous allions l’arracher ?
\VS{29}Et il leur dit : Non, de peur qu’en arrachant l'ivraie, vous ne déraciniez le blé en même temps.
\VS{30}Laissez-les croître tous deux ensemble, jusqu'à la moisson ; et au temps de la moisson, je dirai aux moissonneurs : Arrachez premièrement l'ivraie, et liez là en gerbes pour la brûler, mais amassez le blé dans mon grenier.
\TextTitle{[3. Parabole du grain de sénevé
\\Mc. 4:30-32 ; Lu. 13:18-19]}
\VS{31}Il leur proposa une autre parabole et il dit : Le Royaume des cieux est semblable au grain de sénevé qu’un homme a pris et semé dans son champ.
\VS{32}C’est la plus petite de toutes les semences ; mais, quand il a poussé, il est plus grand que les autres plantes et devient un arbre, de sorte que les oiseaux du ciel viennent habiter et font leurs nids dans ses branches.
\TextTitle{[4. Parabole du levain
\\(Lu. 13:20-21)]}
\VS{33}Il leur dit une autre parabole : Le Royaume des cieux est semblable à du levain qu'une femme a pris et mis dans trois mesures de farine, jusqu'à ce que toute la pâte soit levée.
\VS{34}Jésus dit à la foule toutes ces choses en paraboles, et il ne lui parlait point sans paraboles,
\VS{35}afin que s’accomplisse ce qui avait été annoncé par le prophète : J'ouvrirai ma bouche en paraboles, je déclarerai les choses qui ont été cachées dès la fondation du monde{\FTNT{Ps. 78:2.}}.
\TextTitle{[Explication de la parabole du blé et de l'ivraie]}
\VS{36}Alors Jésus renvoya la foule et entra dans la maison, et ses disciples s’approchèrent de lui et lui dirent : Explique-nous la parabole de l'ivraie du champ.
\VS{37}Et il leur répondit et dit : Celui qui sème la bonne semence, c'est le Fils de l'homme ;
\VS{38}le champ, c'est le monde ; la bonne semence ce sont les fils du Royaume, et l'ivraie ce sont les fils du malin ;
\VS{39}et l'ennemi qui l'a semée, c'est le diable ; la moisson, c'est la fin du monde, et les moissonneurs sont les anges.
\VS{40}Or, comme on arrache l'ivraie, et qu’on la brûle au feu, il en sera de même à la fin de ce monde.
\VS{41}Le Fils de l'homme enverra ses anges qui arracheront de son Royaume tous les scandales et ceux qui commettent l'iniquité,
\VS{42}et les jetteront dans la fournaise ardente, où il y aura des pleurs et des grincements de dents.
\VS{43}Alors les justes resplendiront comme le soleil dans le Royaume de leur Père. Que celui qui a des oreilles pour entendre, qu'il entende.
\TextTitle{[5. Parabole du trésor caché]}
\VS{44}Le Royaume des cieux est encore semblable à un trésor caché dans un champ. L’homme qui l’a trouvé, le cache ; puis dans sa joie, il va vendre tout ce qu'il a, et achète ce champ.
\TextTitle{[6. Parabole de la perle]}
\VS{45}Le Royaume des cieux est encore semblable à un marchand qui cherche de bonnes perles.
\VS{46}Il a trouvé une perle de grand prix et il est allé vendre tout ce qu'il avait, et l'a achetée.
\TextTitle{[7. Parabole du filet]}
\VS{47}Le Royaume des cieux est encore semblable à un filet jeté dans la mer et ramassant toutes sortes de choses.
\VS{48}Quand il est rempli, les pêcheurs le tirent en haut sur le rivage, puis s'étant assis, ils mettent ce qu'il y a de bon à part dans leurs vases, et jettent dehors ce qui est mauvais.
\VS{49}Il en sera de même à la fin du monde, les anges viendront séparer les méchants d'avec les justes,
\VS{50}et les jetteront dans la fournaise ardente, où il y aura des pleurs, et des grincements de dents.
\VS{51}Jésus leur dit : Avez-vous compris toutes ces choses ? Ils lui répondirent : Oui, Seigneur.
\TextTitle{[8. Le maître de la maison]}
\VS{52}Et il leur dit : C’est pourquoi, tout scribe instruit de ce qui regarde le Royaume des cieux, est semblable à un père de famille qui tire de son trésor des choses nouvelles et des choses anciennes.
\TextTitle{[Jésus se rend à Nazareth]
\\(Mc. 6:1-6)}
\VS{53}Et quand Jésus eut achevé ces paraboles, il partit de là.
\VS{54}Et s’étant rendu dans sa patrie, il enseignait dans la synagogue, de telle sorte que ceux qui l’entendirent étaient étonnés et disaient : D’où lui viennent cette sagesse et ces miracles ?
\VS{55}Celui-ci n'est-il pas le fils du charpentier ? Sa mère ne s'appelle-t-elle pas Marie ? Et ses frères ne s'appellent-ils pas Jacques, Joseph, Simon et Jude ?
\VS{56}Et ses sœurs ne sont-elles pas toutes parmi nous ? D'où lui viennent donc toutes ces choses ?
\VS{57}Et il était pour eux une occasion de chute. Mais Jésus leur dit : Un prophète n'est méprisé que dans sa patrie et dans sa maison.
\VS{58}Et il ne fit là que peu de miracles, à cause de leur incrédulité.
\TextTitle{[Mort de Jean-Baptiste]
\\(Mc. 6:14-29; cp. Lu. 9:7-9)}
\Chap{14}
\VerseOne{}En ce temps-là, Hérode le tétrarque ayant entendu parler de Jésus, dit à ses serviteurs : C’est Jean-Baptiste !
\VS{2}Il est ressuscité des morts, c'est pourquoi la puissance de faire des miracles agit puissamment en lui.
\VS{3}Car Hérode avait fait arrêter Jean, et l'avait fait lier et mettre en prison, à cause d'Hérodias, femme de Philippe son frère.
\VS{4}Parce que Jean lui disait : Il ne t'est pas permis de l'avoir pour femme.
\VS{5}Et il voulait le faire mourir, mais il craignait la foule, parce qu’elle regardait Jean comme un prophète.
\VS{6}Or, le jour où l’on célébra la naissance d'Hérode, la fille d'Hérodias dansa au milieu de l’assemblée et plut à Hérode.
\VS{7}C'est pourquoi il lui promit avec serment de lui donner tout ce qu'elle demanderait.
\VS{8}A l’instigation de sa mère, elle dit : Donne-moi ici, sur un plat, la tête de Jean-Baptiste.
\VS{9}Le roi fut attristé ; mais à cause de ses serments et de ceux qui étaient à table avec lui, il commanda qu'on la lui donne.
\VS{10}Et il envoya décapiter Jean dans la prison.
\VS{11}Et sa tête fut apportée sur un plat et donnée à la fille qui la présenta à sa mère.
\VS{12}Puis ses disciples vinrent, et emportèrent son corps, et l'ensevelirent. Et ils allèrent l’annoncer à Jésus.
\VS{13}Et Jésus, ayant appris ce qu’Hérode avait fait, partit de là dans une barque, pour se retirer à l’écart dans un lieu désert ; et la foule l’ayant appris, sortit des villes voisines, et le suivit à pied.
\VS{14}Et Jésus étant sorti, vit une grande foule, et il fut ému de compassion pour elle, et guérit les malades.
\TextTitle{[Multiplication des pains pour les cinq mille hommes]
\\(Mc. 6:32-44 ; Lu. 9:12-17 ; Jn. 6:1-14)}
\VS{15}Et comme il se faisait tard, ses disciples vinrent à lui et lui dirent : Ce lieu est désert et l'heure est déjà avancée. Renvoie la foule, afin qu'elle aille dans les villages, pour s’acheter des vivres.
\VS{16}Mais Jésus leur dit : Ils n'ont pas besoin de s'en aller ; donnez-leur vous-mêmes à manger.
\VS{17}Et ils lui dirent : Nous n'avons ici que cinq pains et deux poissons.
\VS{18}Et il leur dit : Apportez-les-moi ici.
\VS{19}Et après avoir ordonné à la foule de s'asseoir sur l'herbe, il prit les cinq pains et les deux poissons, et levant les yeux au ciel, il rendit grâces à Dieu. Puis ayant rompu les pains, il les donna aux disciples qui les distribuèrent à la foule.
\VS{20}Tous en mangèrent et furent rassasiés, et l’on emporta douze paniers pleins des morceaux qui restaient.
\VS{21}Ceux qui avaient mangé étaient environ cinq mille hommes, sans compter les femmes et les petits enfants.
\TextTitle{[Jésus marche sur les eaux]
\\(Mc. 6:45-56 ; Jn. 6:15-21)}
\VS{22}Aussitôt après, Jésus obligea ses disciples à monter dans la barque, et à passer avant lui de l'autre côté, pendant qu'il renverrait la foule.
\VS{23}Et quand il l’eut renvoyée, il monta sur une montagne pour être à part, afin de prier ; et le soir étant venu, il était là seul.
\VS{24}La barque, déjà au milieu de la mer, était battue par les flots ; car le vent était contraire.
\VS{25}Et vers la quatrième veille de la nuit, Jésus alla vers eux, marchant sur la mer.
\VS{26}Et ses disciples le voyant marcher sur la mer, ils furent troublés, et ils dirent : C’est un fantôme ! Et, dans leur frayeur, ils poussèrent des cris.
\VS{27}Jésus leur dit aussitôt : Rassurez-vous, c'est moi, n'ayez pas de peur !
\VS{28}Et Pierre lui répondit : Seigneur, si c'est toi, ordonne que j'aille vers toi sur les eaux.
\VS{29}Et il lui dit : Viens ! Pierre sortit de la barque, marcha sur les eaux pour aller vers Jésus.
\VS{30}Mais voyant que le vent était fort, il eut peur ; et comme il commençait à enfoncer, il s'écria : Seigneur ! Sauve-moi !
\VS{31}Et aussitôt Jésus étendit sa main et le prit en lui disant : Homme de peu foi, pourquoi as-tu douté ?
\VS{32}Et quand ils furent montés dans la barque, le vent s'apaisa.
\VS{33}Alors ceux qui étaient dans la barque, vinrent adorer Jésus et dirent : Certes, tu es le Fils de Dieu.
\TextTitle{[Jésus guérit des malades à Génésareth]
\\(Mc. 6:53-56)}
\VS{34}Après avoir traversé la mer, ils vinrent dans le pays de Génézareth.
\VS{35}Les gens de ce lieu ayant reconnu Jésus, envoyèrent des messagers dans tous les environs et on lui amena tous les malades.
\VS{36}Et ils le prièrent de leur permettre de toucher seulement le bord de son vêtement. Et tous ceux qui le touchèrent furent guéris.
\TextTitle{[Jésus-Christ condamne les traditions]
\\(Mc. 7:1-13)}
\Chap{15}
\VerseOne{}Alors des scribes et des pharisiens vinrent de Jérusalem auprès de Jésus et lui dirent :
\VS{2}Pourquoi tes disciples transgressent-ils la tradition des anciens ? Car ils ne se lavent point les mains quand ils prennent leur repas.
\VS{3}Il leur répondit : Et vous, pourquoi transgressez-vous le commandement de Dieu par votre tradition ?
\VS{4}Car Dieu a dit : Honore ton père et ta mère. Et il a dit aussi : Celui qui maudira son père ou sa mère sera puni de mort.
\VS{5}Mais vous, vous dites : Celui qui dira à son père ou à sa mère : Tout ce dont j’aurais pu t’assister est une offrande à Dieu, n’est pas coupable, quoiqu’il n’honore pas son père ou sa mère.
\VS{6}Vous annulez ainsi le commandement de Dieu par votre tradition.
\VS{7}Hypocrites, Esaïe a bien prophétisé de vous, en disant :
\VS{8}Ce peuple s’approche de moi de sa bouche et m’honore des lèvres ; mais son cœur est éloigné de moi.
\VS{9}C’est en vain qu’ils m'honorent, en enseignant des doctrines qui ne sont que des commandements d'hommes{\FTNT{Es. 29:13.}}.
\TextTitle{[Verdict sur le coeur humain]
\\(Mc. 7:14-23)}
\VS{10}Puis ayant appelé à lui la foule, il lui dit : Ecoutez, et comprenez ceci :
\VS{11}Ce n'est pas ce qui entre dans la bouche qui souille l'homme ; mais ce qui sort de la bouche c'est ce qui souille l'homme.
\VS{12}Alors ses disciples s'approchant, lui dirent : Sais-tu que les pharisiens ont été scandalisés des paroles qu’ils ont entendues ?
\VS{13}Il leur répondit : Toute plante que mon Père céleste n'a pas plantée sera déracinée.
\VS{14}Laissez-les, ce sont des aveugles qui conduisent des aveugles ; si un aveugle conduit un autre aveugle, ils tomberont tous deux dans une fosse.
\VS{15}Alors Pierre prenant la parole, lui dit : Explique-nous cette parabole.
\VS{16}Et Jésus dit : Vous aussi, êtes-vous encore sans intelligence ?
\VS{17}Ne comprenez-vous pas encore que tout ce qui entre dans la bouche va dans le ventre, puis est jeté dans les lieux secrets ?
\VS{18}Mais ce qui sort de la bouche vient du cœur, et c’est ce qui souille l'homme.
\VS{19}Car c’est du cœur que viennent les mauvaises pensées, les meurtres, les adultères, les fornications, les vols, les faux témoignages, les médisances.
\VS{20}Ce sont ces choses-là qui souillent l'homme ; mais manger sans s’être lavé les mains, cela ne souille point l'homme.
\TextTitle{[Jésus et la femme cananéenne]
\\(Mc. 7:24-30)}
\VS{21}Jésus, étant parti de là, se retira dans le territoire de Tyr et de Sidon.
\VS{22}Et voici, une femme Cananéenne, qui venait de ces contrées, lui cria : Seigneur ! Fils de David, aie pitié de moi ! Ma fille est cruellement tourmentée par le démon.
\VS{23}Mais il ne lui répondit pas un mot. Et ses disciples s'approchèrent, et lui dirent : Renvoie-la, car elle crie derrière nous.
\VS{24}Et il répondit : Je n’ai été envoyé qu'aux brebis perdues de la maison d'Israël.
\VS{25}Mais elle vint, et l'adora, disant : Seigneur, secours-moi !
\VS{26}Et il lui répondit en disant : Il n’est pas juste de prendre le pain des enfants et de le jeter aux petits chiens.
\VS{27}Mais elle dit : Cela est vrai, Seigneur ! Cependant les petits chiens mangent des miettes qui tombent de la table de leurs maîtres.
\VS{28}Alors Jésus répondit et dit : Ô femme ! Ta foi est grande. Qu'il te soit fait comme tu le souhaites. Et, à l’heure même, sa fille fut guérie.
\TextTitle{[Nouvelles guérisons]}
\VS{29}Et Jésus quitta ces lieux, et vint près de la mer de Galilée. Etant monté sur une montagne, il s’y assit.
\VS{30}Et une grande foule vint à lui, ayant avec elle des boiteux, des aveugles, des muets, des estropiés, et beaucoup d’autres malades. On les mit aux pieds de Jésus, et il les guérit.
\VS{31}C’est ainsi que la foule était dans l’admiration de voir que les muets parlaient, que les estropiés étaient guéris, que les boiteux marchaient, que les aveugles voyaient ; et elle glorifiait le Dieu d'Israël.
\TextTitle{[Seconde multiplication des pains pour les quatre mille hommes]
\\(Mc. 8:1-9)}
\VS{32}Alors Jésus, ayant appelé ses disciples, dit : Je suis ému de compassion pour cette foule de gens ; car voilà trois jours qu'ils sont près de moi, et ils n'ont rien à manger. Je ne veux pas les renvoyer à jeun, de peur que les forces ne leur manquent en chemin.
\VS{33}Et ses disciples lui dirent : Comment nous procurer dans ce lieu désert assez de pains pour rassasier une si grande foule ?
\VS{34}Et Jésus leur dit : Combien avez-vous de pains ? Ils lui dirent : Sept, et quelques petits poissons.
\VS{35}Alors il fit asseoir la foule par terre,
\VS{36}prit les sept pains et les poissons, et après avoir rendu grâces à Dieu, il les rompit et les donna à ses disciples, qui les distribuèrent à la foule.
\VS{37}Tous mangèrent et furent rassasiés, et l’on emporta sept corbeilles pleines des morceaux qui restaient.
\VS{38}Ceux qui avaient mangé étaient quatre mille hommes, sans compter les femmes et les petits enfants.
\VS{39}Et Jésus renvoya la foule, monta sur une barque, et se rendit dans le territoire de Magdala.
\TextTitle{[Reproches aux pharisiens aveugles]
\\(Mc. 8:10-14)}
\Chap{16}
\VerseOne{}Les pharisiens et les sadducéens s’approchèrent de Jésus, et pour l'éprouver, lui demandèrent de leur faire voir un signe venant du ciel.
\VS{2}Jésus leur répondit : Quand le soir est venu, vous dites : Il fera beau temps, car le ciel est rouge.
\VS{3}Et le matin vous dites : Il y aura de l'orage aujourd'hui, car le ciel est d’un rouge sombre. Hypocrites, vous savez bien discerner l’aspect du ciel, et vous ne pouvez discerner les signes des temps !
\VS{4}Une génération méchante et adultère demande un miracle ; mais il ne lui sera point donné d'autre miracle que celui de Jonas le prophète. Puis il les quitta, et s'en alla.
\VS{5}Et ses disciples, en passant sur l’autre bord, avaient oublié de prendre des pains.
\TextTitle{[Le levain des pharisiens et des sadducéens]
\\(Mc. 8:15-21 ; cp. Lu. 12:1-15)}
\VS{6}Et Jésus leur dit : Gardez-vous avec soin du levain des pharisiens et des sadducéens.
\VS{7}Ils résonnaient en eux-mêmes, et disaient : C’est parce que nous n'avons pas pris de pains.
\VS{8}Et Jésus connaissant leurs pensées leur dit : Gens de peu de foi, pourquoi raisonnez-vous en vous-mêmes sur le fait que vous n'avez pas pris de pains ?
\VS{9}Etes-vous encore sans intelligence, et ne vous rappelez-vous plus les cinq pains des cinq mille hommes et combien de paniers vous avez emportés,
\VS{10}ni des sept pains des quatre mille hommes et combien de corbeilles vous avez emportées ?
\VS{11}Comment ne comprenez-vous pas que je ne vous parlais pas de pain, lorsque je vous ai dit de vous garder du levain des pharisiens et des sadducéens ?
\VS{12}Alors ils comprirent que ce n'était pas du levain du pain qu'il leur avait dit de se garder, mais de la doctrine des pharisiens et des sadducéens.
\TextTitle{[Pierre reconnaït Jésus comme le Messie]
\\(Mc. 8:27-30 ; Lu. 9:18-21 ; cp. Jn. 6:66-71)}
\VS{13}Jésus, étant arrivé dans le territoire de Césarée de Philippe, demanda à ses disciples : Qui suis-je aux dires des hommes, moi le Fils de l'homme ?
\VS{14}Et ils lui répondirent : Les uns disent que tu es Jean-Baptiste ; les autres, Elie ; et les autres, Jérémie, ou l'un des prophètes.
\VS{15}Et vous, leur dit-il, qui dites-vous que je suis ?
\VS{16}Simon Pierre répondit : Tu es le Christ, le Fils du Dieu vivant.
\TextTitle{[Jésus bâtit son Eglise]}
\VS{17}Et Jésus lui répondit et dit : Tu es heureux, Simon, fils de Jonas, car ce ne sont pas la chair et le sang qui t’ont révélé cela, mais mon Père qui est dans les cieux.
\VS{18}Et moi je te dis, que tu es Pierre, et que sur ce Roc{\FTNT{Le Roc : Ce passage a été mal traduit dans beaucoup de Bibles comme suit : «~Et moi, je te dis que tu es Pierre, et que sur cette pierre je bâtirai mon Eglise…~». Or pour une bonne compréhension des propos de Jésus, il est important d’insister sur la distinction que le grec fait entre «~Petros~» (pierre, caillou), l’apôtre Pierre, et «~Petra~» (roc, rocher), qui n’est autre que Jésus-Christ, le rocher des siècles (Es. 17:10 ; Es. 26:4 ; 1 Co. 10:4). De là en découle un enseignement fondamental : l’Eglise n’est bâtie ni par un homme ni sur homme, en l’occurrence Pierre et ses supposés successeurs (papes), mais par Jésus-Christ lui-même qui en est la Pierre Angulaire et le fondement inébranlable (Ac. 4:11 ; Ep. 2:20).}} je bâtirai mon Eglise ; et les portes de l’enfer{\FTNT{(2) Enfer : du grec «~Hades~». Hadès chez les Grecs ou Pluton chez les Romains, était considéré comme le dieu des profondeurs souterraines et le maître des enfers. Ce terme est parfois traduit par «~séjour des morts~», équivalent hébreu de «~Scheol~». Les Grecs utilisaient l’euphémisme Pylartes, signifiant «~aux portes solidement closes~», pour parler du très craint Hadès. En effet, Juifs, Grecs et Romains avaient conscience que les portes closes de l’enfer ne laissaient personne sortir du royaume de la mort. Tous les impies, et même les croyants d’avant Jésus-Christ, étaient retenus par les portes de l’enfer. Toutefois, les croyants allaient dans une partie de l’enfer que les juifs appelaient «~sein d’Abraham~» ou «~paradis~» (Lu. 16:22-25 ; Lu. 23:43) où ils ne subissaient pas les tourments infligés aux impies. Lorsque le Seigneur est mort, il est descendu «~dans les régions inférieures de la terre~» pour dépouiller Hadès des clés du séjour des morts (Col. 2:15 ; Ap. 1:17-18) et libérer les captifs pieux. Ainsi, certains ont rejoint le paradis tandis que d’autres ont ressuscité au moment où le Seigneur expira (Ep. 4:8-9 ; Mt. 27:52). Jésus affirme que les portes de l’enfer ne prévaudront jamais contre son Eglise puisque c’est lui qui l’a bâtie. Malgré tout, Hadès, bien que vaincu par le Seigneur, essaie d’attirer l’Eglise que le Seigneur a établie dans les lieux célestes (Ep. 2:4-9 ; Col. 3:1) vers le royaume des ténèbres, au travers des fausses doctrines et du péché. Au jour du jugement dernier, Hadès et la mort, qui sont deux démons, seront jetés dans l’étang de feu et de soufre (Ap. 20:11-15).}} ne prévaudront point contre elle.
\VS{19}Je te donnerai les clefs du Royaume des cieux ; et tout ce que tu lieras sur la terre, sera lié dans les cieux ; et tout ce que tu délieras sur la terre, sera délié dans les cieux{\FTNT{Une mauvaise compréhension de ce verset a contribué à propager l’idée erronée selon laquelle Pierre serait le médiateur entre Dieu et les hommes, puisque c’est lui qui détiendrait les clés du Royaume des cieux. Toutefois, Es. 22:22 affirme que seul Jésus-Christ détient ces clés qui symbolisent l’autorité et la domination. Or dans le cadre de l’héritage que le Seigneur nous a laissé, cette autorité est désormais exercée en son Nom par tous les membres du corps de Christ (Mt. 18:18).}}.
\VS{20}Alors il recommanda expressément à ses disciples de ne dire à personne qu’il était le Christ.
\TextTitle{[Jésus annonce sa mort et sa résurrection]
\\(Mc. 8:31-33 ; Lu. 9:22)}
\VS{21}Dès lors Jésus commença à déclarer à ses disciples qu'il fallait qu'il aille à Jérusalem, qu'il souffre beaucoup de la part des anciens, des principaux sacrificateurs et des scribes, qu'il soit mis à mort, et qu'il ressuscite le troisième jour.
\VS{22}Mais Pierre l'ayant pris à part, se mit à le reprendre en lui disant : Seigneur, aie pitié de toi, cela ne t'arrivera point !
\VS{23}Mais Jésus, se retournant, dit à Pierre : Arrière de moi, Satan ! Tu m'es en scandale, car tu ne comprends pas les choses qui sont de Dieu, mais seulement celles qui sont des hommes.
\TextTitle{[Prix de la consécration du disciple]
\\(Mc. 8:34-38 ; Lu. 9:23-26)}
\VS{24}Alors Jésus dit à ses disciples : Si quelqu'un veut venir après moi, qu'il renonce à lui-même, et qu'il se charge de sa croix, et qu’il me suive.
\VS{25}Car quiconque voudra sauver son âme, la perdra ; mais quiconque perdra son âme à cause de son amour pour moi, la trouvera.
\VS{26}Et que servirait-il à un homme de gagner tout le monde, s'il perdait son âme ? Ou, que donnerait un homme en échange de son âme ?
\VS{27}Car le Fils de l'homme doit venir dans la gloire de son Père avec ses anges ; et alors il rendra à chacun selon ses œuvres.
\VS{28}Je vous le dis en vérité, quelques-uns de ceux qui sont ici présents, ne mourront point, qu’ils n’aient vu le Fils de l'homme venir dans son règne{\FTNT{Ce passage doit être lu de concert avec Mt. 24:32-34. Jésus utilise un langage prophétique pour expliquer deux réalités. La première réalité est spirituelle et concerne ses contemporains qui allaient vivre l’effusion de l’Esprit pour rétablir le Royaume de Dieu dans le cœur des gens. En effet, le Seigneur ne les a pas laissés orphelins, mais il est revenu sous la forme de l’Esprit (Jn. 14:17-18 ; Ac. 2 ; Ac. 16:7). Aussi, les apôtres ont pu proclamer ce Royaume partout où ils allaient (Ac. 20:25). La deuxième réalité est matérielle et concerne le fleurissement du figuier, c’est-à-dire Israël. L’histoire atteste le fleurissement de ce figuier tant sur le plan géographique que sur le plan numérique. Depuis le 14 mai 1948, date de la naissance officielle de l’état hébreu, Israël ne cesse de s’étendre. Cette nation est l’horloge des temps car le Messie gouvernera le monde entier depuis Jérusalem (Mi. 4 ; Za. 14).}}.
\TextTitle{[Transfiguration de Jésus-Christ]
\\(Mc. 9:1-8 ; Lu. 9:27-36)}
\Chap{17}
\VerseOne{}Six jours après, Jésus prit Pierre, Jacques et Jean son frère, et les conduisit à l'écart sur une haute montagne.
\VS{2}Et il fut transfiguré devant eux et son visage resplendit comme le soleil ; et ses vêtements devinrent blancs comme la lumière.
\VS{3}Et voici, Moïse et Elie leur apparurent, s'entretenant avec lui.
\VS{4}Alors Pierre prenant la parole, dit à Jésus : Seigneur, il est bon que nous soyons ici. Faisons-y, si tu le veux, trois tentes, une pour toi, une pour Moïse, et une pour Elie.
\VS{5}Et comme il parlait encore, voici une nuée resplendissante les couvrit de son ombre. Et voici, une voix fit entendre de la nuée ces paroles : Celui-ci est mon Fils bien-aimé, en qui j'ai pris mon bon plaisir : Ecoutez-le !
\VS{6}Lorsque les disciples entendirent cette voix, ils tombèrent le visage contre terre et furent saisis d’une très grande frayeur.
\VS{7}Mais Jésus, s'approchant, les toucha, et leur dit : Levez-vous, et n'ayez pas peur.
\VS{8}Ils levèrent les yeux, et ne virent personne, excepté Jésus tout seul.
\VS{9}Et comme ils descendaient de la montagne, Jésus leur donna cet ordre, en disant : Ne parlez à personne de cette vision, jusqu'à ce que le Fils de l'homme soit ressuscité des morts.
\VS{10}Et ses disciples l'interrogèrent, en disant : Pourquoi donc les scribes disent-ils qu'il faut qu'Elie vienne premièrement ?
\VS{11}Et Jésus répondit : Il est vrai qu'Elie viendra premièrement et rétablira toutes choses.
\VS{12}Mais je vous dis qu'Elie est déjà venu, et ils ne l'ont pas reconnu et ils l’ont traité comme ils ont voulu. De même, le Fils de l'homme doit souffrir aussi de leur part.
\VS{13}Alors les disciples comprirent que c'était de Jean-Baptiste qu'il leur parlait.
\TextTitle{[Jésus reprend ses disciples et leur reproche leur manque de foi]
\\(Mc. 9:30-32 ; Lu. 9:44-45)}
\VS{14}Et quand ils furent arrivés près de la foule, un homme s'approcha, et se mit à genoux devant lui,
\VS{15}et lui dit : Seigneur ! Aie pitié de mon fils qui est lunatique et misérablement affligé ; car il tombe souvent dans le feu, et souvent dans l'eau.
\VS{16}Et je l'ai présenté à tes disciples, mais ils n’ont pas pu le guérir.
\VS{17}Et Jésus répondit et dit : Ô race incrédule et perverse, jusqu’à quand serai-je avec vous ? Jusqu’à quand vous supporterai-je ? Amenez-le-moi ici.
\VS{18}Et Jésus parla sévèrement au démon, qui sortit de lui, et à l'heure même l'enfant fut guéri.
\VS{19}Alors les disciples s’approchèrent de Jésus et lui dirent en particulier : Pourquoi n’avons-nous pas pu chasser ce démon ?
\VS{20}Et Jésus leur répondit : C'est à cause de votre incrédulité. Je vous le dis en vérité, si vous aviez de la foi, comme un grain de sénevé, vous diriez à cette montagne : Transporte-toi d'ici là, et elle se transporterait ; et rien ne vous serait impossible.
\VS{21}Mais cette sorte de démon ne sort que par la prière et par le jeûne.
\TextTitle{[Jésus annonce à nouveau sa mort et sa résurrection]
\\(Mc. 9:30-32 ; Lu. 9:44-45)}
\VS{22}Et comme ils se trouvaient en Galilée, Jésus leur dit : Le Fils de l'homme doit être livré entre les mains des hommes ;
\VS{23}ils le feront mourir, mais le troisième jour il ressuscitera. Et les disciples furent profondément attristés.
\TextTitle{[Miracle de Jésus : Les deux drachmes dans la bouche du poisson]
\\(cp. Mc. 12:13-17)}
\VS{24}Et lorsqu'ils arrivèrent à Capernaüm, ceux qui percevaient les deux drachmes s'adressèrent à Pierre et lui dirent : Votre Maître ne paye-t-il pas les deux drachmes ?
\VS{25}Oui dit-il. Et quand il fut entré dans la maison, Jésus le prévint en lui disant : Qu'est-ce qu'il t'en semble, Simon ? Les rois de la terre, de qui perçoivent-ils des tributs, ou des impôts ? Est-ce de leurs enfants, ou des étrangers ?
\VS{26}Pierre dit : Des étrangers. Jésus lui répondit : Les enfants en sont donc exempts.
\VS{27}Mais, pour ne pas les scandaliser, va à la mer et jette l'hameçon, et prends le premier poisson qui viendra ; ouvre-lui la bouche, tu trouveras un statère. Prends-le, et donne-le-leur pour moi et pour toi.
\TextTitle{[L'humilité, secret de la vraie grandeur]
\\(Mc. 9:33-37; Lu. 9:46-48)}
\Chap{18}
\VerseOne{}En cette même heure-là, les disciples s’approchèrent de Jésus, et dirent : Qui est le plus grand dans le Royaume des cieux ?
\VS{2}Et Jésus ayant appelé un petit enfant, le mit au milieu d'eux,
\VS{3}et leur dit : Je vous le dis en vérité, que si vous ne vous convertissez pas et si vous ne devenez pas comme les petits enfants, vous n'entrerez pas dans le Royaume des cieux.
\VS{4}C'est pourquoi quiconque deviendra humble, comme ce petit enfant, sera le plus grand dans le Royaume des cieux.
\VS{5}Et quiconque reçoit en mon Nom un petit enfant comme celui-ci, me reçoit moi-même.
\VS{6}Mais, si quelqu’un scandalisait un de ces petits qui croient en moi, il vaudrait mieux pour lui qu'on mette à son cou une meule d'âne, et qu'on le jette au fond de la mer.
\TextTitle{[Les scandales et les occasions de chute]
\VS{7}Malheur au monde à cause des scandales ! Car il est nécessaire qu'il arrive des scandales ; mais malheur à l'homme par qui le scandale arrive !
\VS{8}Si ta main ou ton pied est pour toi une occasion de chute, coupe-les et jette-les loin de toi ; car il vaut mieux que tu entres boiteux ou manchot dans la vie, que d'avoir deux pieds ou deux mains, et d'être jeté dans le feu éternel.
\VS{9}Et si ton œil est pour toi une occasion de chute, arrache-le et jette-le loin de toi ; car il vaut mieux que tu entres dans la vie n'ayant qu'un œil, que d'avoir deux yeux, et d'être jeté dans le feu de la géhenne.
\VS{10}Gardez-vous de mépriser un seul de ces petits ; car je vous dis que dans les cieux leurs anges voient continuellement la face de mon Père qui est aux cieux.
\VS{11}Car le Fils de l'homme est venu pour sauver ce qui était perdu.
\TextTitle{[Parabole de la brebis égarée]
\\(Lu. 15:3-7)}
\VS{12}Que vous en semble ? Si un homme a cent brebis, et que l’une d’elles s’égare, ne laisse-t-il pas les quatre-vingt-dix-neuf autres, pour aller dans les montagnes chercher celle qui s'est égarée ?
\VS{13}Et, s'il arrive qu'il la trouve, je vous le dis en vérité, elle lui cause plus de joie que les quatre-vingt-dix-neuf qui ne se sont pas égarées.
\VS{14}De même, ce n’est pas la volonté de votre Père qui est dans les cieux qu'un seul de ces petits périsse.
\TextTitle{[Discipline dans les assemblées]}
\VS{15}Si ton frère a péché contre toi, va, et reprends-le entre toi et lui seul. S’il t'écoute, tu as gagné ton frère.
\VS{16}Mais s'il ne t'écoute pas, prends encore avec toi une ou deux personnes, afin que toute l’affaire se règle sur la déclaration de deux ou de trois témoins{\FTNT{De. 19:15.}}.
\VS{17}S’il refuse de les écouter, dis-le à l'Eglise ; et s'il refuse aussi d’écouter l'Eglise, qu'il soit pour toi comme un païen et comme un publicain.
\VS{18}Je vous le dis en vérité, tout ce que vous lierez sur la terre, sera lié dans le ciel ; et tout ce que vous délierez sur la terre sera délié dans le ciel{\FTNT{Voir commentaire en Mt. 16:19.}}.
\VS{19}Je vous dis aussi que si deux d'entre vous s'accordent sur la terre pour demander une chose quelconque, elle leur sera accordée par mon Père qui est dans les cieux.
\VS{20}Car là où deux ou trois sont assemblés en mon Nom, je suis là au milieu d'eux.
\VS{21}Alors Pierre s'approchant, lui dit : Seigneur, combien de fois pardonnerai-je à mon frère lorsqu’il péchera contre moi ? Sera-ce jusqu'à sept fois ?
\VS{22}Jésus lui répondit : Je ne te dis pas jusqu'à sept fois, mais jusqu'à soixante-dix fois sept fois.
\VS{23}C'est pourquoi le Royaume des cieux est semblable à un Roi qui voulut faire rendre compte à ses serviteurs.
\VS{24}Et quand il se mit à compter, on lui en présenta un qui lui devait dix mille talents.
\VS{25}Et parce qu'il n'avait pas de quoi payer, son maître ordonna qu'il fût vendu, lui, sa femme, ses enfants et tout ce qu'il avait, et que la dette fût acquittée.
\VS{26}Mais ce serviteur se jetant à ses pieds, le suppliait en disant : Seigneur, aie patience envers moi et je te paierai tout.
\VS{27}Alors le maître de ce serviteur, ému de compassion, le laissa aller et lui remit la dette.
\VS{28}Après qu’il fut sorti, ce serviteur rencontra un de ses compagnons de service, qui lui devait cent deniers. Il le saisit et l'étranglait, en lui disant : Paye-moi ce que tu me dois !
\VS{29}Mais son compagnon de service se jetant à ses pieds, le suppliait en disant : Aie patience envers moi, et je te paierai le tout.
\VS{30}Mais l’autre ne voulut pas, et il alla le jeter en prison, jusqu'à ce qu'il ait payé la dette.
\VS{31}Or ses autres compagnons de service voyant ce qui était arrivé, en furent extrêmement attristés, et ils allèrent raconter à leur maître tout ce qui s'était passé.
\VS{32}Alors son maître le fit venir, et lui dit : Méchant serviteur, je t’avais remis en entier ta dette, parce que tu m'en avais supplié.
\VS{33}Ne devais-tu pas aussi avoir pitié de ton compagnon de service, comme j’ai eu pitié de toi ?
\VS{34}Et son maître étant irrité le livra aux sergents, jusqu'à ce qu'il lui ait payé tout ce qu’il devait.
\VS{35}C'est ainsi que mon Père céleste vous traitera, si chacun de vous ne pardonne pas à son frère de tout son cœur.
\TextTitle{[Enseignement de Jésus sur le mariage et le divorce]
\\(Mt. 5:31-32 ; Mc. 10:2-12 ; Lu. 16:18 ; cp. Ro. 7:1-3 ; 1 Co. 7:10-16)}
\Chap{19}
\VerseOne{}Lorsque Jésus eut achevé ces discours, il quitta la Galilée, et alla dans le territoire de la Judée, au-delà du Jourdain.
\VS{2}Une grande foule le suivit, et là il guérit les malades.
\VS{3}Alors les pharisiens vinrent à lui pour l'éprouver, et ils lui dirent : Est-il permis à un homme de répudier sa femme pour quelque cause que ce soit ?
\VS{4}Et il répondit et leur dit : N'avez-vous pas lu que le Créateur, au commencement, fit l’homme et la femme ?
\VS{5}Et qu'il dit : C’est pourquoi l'homme quittera son père et sa mère, et s’attachera à sa femme, et les deux deviendront une seule chair ?
\VS{6}Ainsi ils ne sont plus deux, mais une seule chair (1). Que l’homme donc ne sépare pas ce que Dieu a mis ensemble sous un joug (2).
\VS{7}Ils lui dirent : Pourquoi donc Moïse a-t-il prescrit de donner la lettre de divorce, et de répudier sa femme (3) ?
\VS{8}Il leur dit : C'est à cause de la dureté de votre cœur que Moïse vous a permis de répudier vos femmes ; mais au commencement il n'en était pas ainsi.
\VS{9}Mais je vous dis que celui qui répudie sa femme, si ce n'est pour cause d'adultère (4), et qui en épouse une autre, commet un adultère ; et celui qui épouse une femme répudiée commet un adultère.
\VS{10}Ses disciples lui dirent : Si telle est la condition de l'homme à l'égard de sa femme, il n’est pas avantageux de se marier.
\VS{11}Mais il leur répondit : Tous ne comprennent pas cette parole, mais seulement ceux à qui cela est donné.
\VS{12}Car il y a des eunuques qui le sont dès le ventre de leur mère ; et il y en a qui ont été faits eunuques par les hommes ; et il y a des eunuques qui se sont faits eux-mêmes eunuques pour le Royaume des cieux. Que celui qui peut comprendre ceci, le comprenne.
\TextTitle{[Jésus bénit les petits enfants]
\\(Mc. 10:13-16 ; Lu. 18:15-17)}
\VS{13}Alors on lui présenta des petits enfants, afin qu'il leur impose les mains et prie pour eux. Mais les disciples les repoussèrent.
\VS{14}Et Jésus leur dit : Laissez venir à moi les petits enfants, et ne les empêchez pas ; car le Royaume des cieux est pour ceux qui leur ressemblent.
\VS{15}Il leur imposa les mains, et il partit de là.
\TextTitle{[Le jeune homme riche]
\\(Mc. 10:17-31 ; Lu. 18:18-27 ; cp. Lu. 10:25-37)}
\VS{16}Et voici, un homme s'approcha, et lui dit : Mon bon Maître, que dois-je faire pour avoir la vie éternelle ?
\VS{17}Il lui répondit : Pourquoi m'appelles-tu bon ? Dieu est le seul être qui soit bon. Si tu veux entrer dans la vie, garde les commandements.
\VS{18}Il lui dit : Lesquels ? Et Jésus lui répondit : Tu ne tueras point. Tu ne commettras point d’adultère. Tu ne déroberas point. Tu ne diras point de faux témoignage.
\VS{19}Honore ton père et ta mère ; et tu aimeras ton prochain comme toi-même (5).
\VS{20}Le jeune homme lui dit : J’ai gardé toutes ces choses dès ma jeunesse. Que me manque-t-il encore ?
\VS{21}Jésus lui dit : Si tu veux être parfait, va, vends ce que tu possèdes, et donne-le aux pauvres, et tu auras un trésor dans le ciel ; puis viens, et suis-moi.
\VS{22}Mais quand ce jeune homme eut entendu cette parole, il s'en alla tout triste, parce qu'il avait de grands biens.
\VS{23}Alors Jésus dit à ses disciples : Je vous le dis en vérité, un riche entrera difficilement dans le Royaume des cieux.
\VS{24}Je vous le dis encore : Il est plus facile à un chameau de passer par le trou d'une aiguille (6), qu’à un riche d’entrer dans le Royaume de Dieu.
\VS{25}Ses disciples ayant entendu ces choses furent très étonnés, et dirent : Qui peut donc être sauvé ?
\VS{26}Et Jésus les regarda, et leur dit : Aux hommes, cela est impossible, mais à Dieu tout est possible.
\TextTitle{[Récompenses maintenant et dans le royaume à venir]
\\(Mc. 10:28-31 ; Lu. 18:28-30)}
\VS{27}Alors Pierre prenant la parole, lui dit : Voici, nous avons tout quitté et nous t'avons suivi ; qu’en serait-il pour nous ?
\VS{28}Et Jésus leur dit : Je vous le dis en vérité, quand le Fils de l’homme, au renouvellement de toutes choses, sera assis sur le trône de sa gloire, vous qui m’avez suivi, vous serez assis sur douze trônes, et vous jugerez les douze tribus d'Israël.
\VS{29}Et quiconque aura quitté, à cause de mon Nom, ses maisons, ses frères, ou ses sœurs, ou son père, ou sa mère, ou sa femme, ou ses enfants, ou ses champs, recevra le centuple, et héritera la vie éternelle.
\VS{30}Plusieurs des premiers seront les derniers, et plusieurs des derniers seront les premiers.
\TextTitle{[Parabole des ouvriers]}
\Chap{20}
\VerseOne{}Car le Royaume des cieux est semblable à un maître de maison, qui sortit dès le matin, afin de louer des ouvriers pour sa vigne.
\VS{2}Il convint avec eux d’un denier par jour, et il les envoya à sa vigne.
\VS{3}Il sortit vers la troisième heure, il en vit d'autres qui étaient sur la place publique, sans rien faire.
\VS{4}Il leur dit : Allez aussi à ma vigne, et je vous donnerai ce qui sera raisonnable.
\VS{5}Et ils y allèrent. Puis il sortit de nouveau vers la sixième heure et vers la neuvième, il fit de même.
\VS{6}Et étant sorti vers la onzième heure, il en trouva d'autres qui étaient sur la place publique sans rien faire, et il leur dit : Pourquoi vous tenez-vous ici toute la journée sans rien faire ?
\VS{7}Ils lui répondirent : Parce que personne ne nous a loués. Et il leur dit : Allez-vous aussi à ma vigne, et vous recevrez ce qui sera raisonnable.
\VS{8}Et le soir étant venu, le maître de la vigne dit à son intendant : Appelle les ouvriers et paye-leur le salaire, en commençant depuis les derniers jusqu’aux premiers.
\VS{9}Alors ceux qui avaient été loués vers la onzième heure vinrent et reçurent chacun un denier.
\VS{10}Or quand les premiers furent venus, ils croyaient recevoir davantage, mais ils reçurent aussi chacun un denier.
\VS{11}Et l'ayant reçu, ils murmuraient contre le maître de la maison,
\VS{12}en disant : Ces derniers n'ont travaillé qu'une heure, et tu les traites à l’égal de nous qui avons supporté la fatigue du jour et la chaleur.
\VS{13}Et il répondit à l'un d'eux et lui dit : Mon ami, je ne te fais pas de tort, n'as-tu pas convenu avec moi d’un denier ?
\VS{14}Prends ce qui te revient, et va-t’en. Je veux donner à ce dernier autant qu'à toi,
\VS{15}ne m'est-il pas permis de faire ce que je veux de mes biens ? Ou vois-tu d’un mauvais œil que je sois bon ?
\VS{16}Ainsi les derniers seront les premiers, et les premiers seront les derniers, car il y a beaucoup d'appelés, mais peu d'élus.
\TextTitle{[Jésus annonce à nouveau sa mort et sa résurrection]
\\(Mc. 10:32-34 ; Lu. 18:31-34 ; cp. Mt. 12:38-42 ; 16:21-28 ; 17:22-23)}
\VS{17}Pendant que Jésus montait à Jérusalem, il prit à part ses douze disciples, et il leur dit en chemin :
\VS{18}Voici, nous montons à Jérusalem, et le Fils de l'homme sera livré aux principaux sacrificateurs et aux scribes, et ils le condamneront à la mort.
\VS{19}Ils le livreront aux païens pour qu’ils se moquent de lui, le battent de verges, et le crucifient ; et le troisième jour il ressuscitera.
\TextTitle{[Demande des fils de Zébédée]
\\(Mc. 10:35-45)}
\VS{20}Alors la mère des fils de Zébédée s’approcha de lui avec ses fils, et se prosterna, pour lui demander quelque chose.
\VS{21}Et il lui dit : Que veux-tu ? Elle lui dit : Ordonne que mes deux fils, que voici, soient assis l'un à ta droite, et l'autre à ta gauche dans ton Royaume.
\VS{22}Et Jésus répondit et dit : Vous ne savez pas ce que vous demandez. Pouvez-vous boire la coupe que je dois boire, et être baptisés du baptême dont je dois être baptisé ? Ils lui répondirent : Nous le pouvons.
\VS{23}Et il leur dit : Il est vrai que vous boirez ma coupe et que vous serez baptisés du baptême dont je serai baptisé ; mais pour ce qui est d'être assis à ma droite ou à ma gauche, cela ne dépend pas de moi, et ne sera donné qu’à ceux à qui mon Père l’a réservé.
\VS{24}Les dix autres disciples ayant entendu cela, furent indignés contre les deux frères.
\VS{25}Mais Jésus les appela et leur dit : Vous savez que les princes des nations les dominent, et que les grands les asservissent.
\VS{26}Mais il n'en sera pas de même au milieu de vous. Au contraire, quiconque veut être grand parmi vous, qu'il soit votre serviteur.
\VS{27}Et quiconque veut être le premier parmi vous, qu'il soit votre serviteur.
\VS{28}C’est ainsi que le Fils de l'homme est venu, non pour être servi, mais pour servir et donner sa vie en rançon pour plusieurs.
\TextTitle{[Jésus guérit deux aveugles]
\\(cp. Mc. 10:46-53 ; Lu. 18:35-43)}
\VS{29}Lorsqu’ils sortirent de Jéricho, une grande foule suivit Jésus.
\VS{30}Et voici, deux aveugles qui étaient assis au bord du chemin, entendirent que Jésus passait, et crièrent en disant : Seigneur, Fils de David ! Aie pitié de nous !
\VS{31}Et la foule les reprenait pour les faire taire ; mais ils criaient encore plus fort : Seigneur, Fils de David ! Aie pitié de nous !
\VS{32}Jésus s'arrêta, les appela, et leur dit : Que voulez-vous que je vous fasse ?
\VS{33}Ils lui dirent : Seigneur, que nos yeux s’ouvrent.
\VS{34}Et Jésus étant ému de compassion, toucha leurs yeux, et aussitôt ils recouvrèrent la vue, et ils le suivirent.
\TextTitle{[Jésus-Christ se présente publiquement comme Roi]
\\(Mc. 11:1-11 ; Lu. 19:28-40 ; Jn. 12:12-19 ; cp. Za.9:9)}
\Chap{21}
\VerseOne{}Lorsqu’ils approchèrent de Jérusalem, et qu'ils furent arrivés à Bethphagé vers le Mont des oliviers, Jésus envoya alors deux disciples,
\VS{2}en leur disant : Allez au village qui est devant vous. Vous trouverez une ânesse attachée, et son ânon avec elle. Détachez-les et amenez-les-moi.
\VS{3}Et si quelqu'un vous dit quelque chose, vous direz que le Seigneur en a besoin ; et aussitôt il les laissera aller.
\VS{4}Or, tout cela arriva afin que s’accomplisse ce qui avait été annoncé par le prophète, en disant :
\VS{5}Dites à la fille de Sion : Voici, ton Roi vient à toi, plein de douceur, et monté sur un âne, sur un ânon, le petit d'une ânesse (1).
\VS{6}Les disciples donc s'en allèrent, et firent ce que Jésus leur avait ordonné.
\VS{7}Et ils amenèrent l'ânesse et l'ânon, et mirent leurs vêtements sur eux, et le firent asseoir dessus.
\VS{8}Alors de grandes foules étendirent leurs vêtements sur le chemin, et les autres coupaient des rameaux des arbres, et les étendaient sur le chemin.
\VS{9}Ceux qui précédaient et ceux qui suivaient Jésus criaient : Hosanna au Fils de David ! Béni soit celui qui vient au Nom du Seigneur ! Hosanna dans les lieux très hauts !
\VS{10}Lorsqu’il entra dans Jérusalem, toute la ville fut émue, et l’on disait : Qui est celui-ci ?
\VS{11}Et la foule répondait : C’est Jésus, le prophète de Nazareth en Galilée.
\TextTitle{[Jésus chasse les marchands du temple]
\\(Mc. 11:15-18 ; Lu. 19:45-46 ; cp. Jn. 2:13-16)}
\VS{12}Jésus entra dans le temple de Dieu. Il chassa dehors tous ceux qui vendaient et qui achetaient dans le temple ; il renversa les tables des changeurs, et les sièges de ceux qui vendaient des pigeons ;
\VS{13}et il leur dit : Il est écrit : Ma maison sera appelée une maison de prière, mais vous en avez fait une caverne de voleurs (2).
\VS{14}Alors des aveugles et des boiteux s’approchèrent de lui dans le temple, et il les guérit.
\VS{15}Mais les principaux sacrificateurs et les scribes furent indignés à la vue des choses merveilleuses qu'il avait faites, et des enfants qui criaient dans le temple : Hosanna au Fils de David !
\VS{16}Et ils lui dirent : Entends-tu ce qu’ils disent ? Oui, leur répondit Jésus. N'avez-vous jamais lu ces paroles : Tu as tiré des louanges de la bouche des enfants, et de ceux qui sont à la mamelle (3) ?
\VS{17}Et, les ayant laissés, il sortit de la ville, pour aller à Béthanie, où il passa la nuit.
\TextTitle{[Le figuier stérile]
\\(Mc. 11:12-14,20-26)}
\VS{18}Le matin, comme il retournait à la ville, il eut faim.
\VS{19}Et voyant un figuier qui était sur le chemin, il s'en approcha, mais il n'y trouva que des feuilles ; et il lui dit : Qu'aucun fruit ne naisse plus jamais de toi ! Et aussitôt le figuier sécha.
\VS{20}Les disciples qui virent cela furent étonnés, et dirent : Comment ce figuier est-il devenu sec en un instant ?
\VS{21}Jésus leur répondit : Je vous le dis en vérité, si vous aviez la foi, et que vous ne doutiez point, non seulement vous ferez ce qui a été fait à ce figuier, mais quand vous diriez à cette montagne : Ôte-toi de là et jette-toi dans la mer, cela se ferait.
\VS{22}Tout ce que vous demanderez avec foi par la prière à Dieu, vous le recevrez.
\TextTitle{[Mise en doute de l'autorité de Jésus-Christ]
\\(Mc. 11:27-33 ; Lu. 20:1-8)}
\VS{23}Puis, s’étant rendu dans le temple, les principaux sacrificateurs et les anciens du peuple vinrent auprès de lui, pendant qu’il enseignait, et lui dirent : Par quelle autorité fais-tu ces choses ; et qui t'a donné cette autorité ?
\VS{24}Jésus répondit : Je vous interrogerai aussi sur une chose, et si vous me répondez, je vous dirai par quelle autorité je fais ces choses.
\VS{25}Le baptême de Jean d'où venait-il ? Du ciel, ou des hommes ? Mais ils raisonnèrent ainsi entre eux : Si nous disons : Du ciel, il nous dira : Pourquoi n’avez-vous pas cru en lui ?
\VS{26}Et si nous répondons : Des hommes, nous craignons la foule, car tous tiennent Jean pour un prophète.
\VS{27}Alors ils répondirent à Jésus : Nous ne savons pas. Et il leur dit : Moi non plus, je ne vous dirai pas par quelle autorité je fais ces choses.
\TextTitle{[Parabole des deux fils]}
\VS{28}Que vous en semble ? Un homme avait deux fils ; et s’adressant au premier, il lui dit : Mon fils, va travailler aujourd'hui dans ma vigne.
\VS{29}Il répondit : Je ne veux pas y aller. Ensuite il se repentit et y alla.
\VS{30}S’adressant à l’autre, il lui dit la même chose. Et ce fils répondit : Je veux bien, seigneur. Et il n’alla pas.
\VS{31}Lequel des deux a fait la volonté du père ? Ils lui répondirent : Le premier. Et Jésus leur dit : Je vous le dis en vérité, les publicains et les prostituées vous devanceront dans le Royaume de Dieu.
\VS{32}Car Jean est venu à vous dans la voie de la justice, et vous ne l'avez pas cru ; mais les publicains et les prostituées ont cru en lui. Et vous, qui avez vu cela, vous ne vous êtes pas ensuite repentis pour croire en lui.
\TextTitle{[Parabole des vignerons]
\\(Mc. 12:1-12 ; Lu. 20:9-18 ; cp. Es. 5:1-7)}
\VS{33}Ecoutez une autre parabole : Il y avait un père de famille qui planta une vigne, et l’entoura d'une haie, et y creusa un pressoir, et bâtit une tour ; puis il l’afferma à des vignerons, et quitta le pays.
\VS{34}Lorsque le temps de la récolte fut arrivé, il envoya ses serviteurs vers les vignerons, pour recevoir les fruits.
\VS{35}Mais les vignerons s’étant saisis de ses serviteurs, fouettèrent l'un, tuèrent l'autre, et lapidèrent le troisième.
\VS{36}Il envoya encore d'autres serviteurs en plus grand nombre que les premiers, et ils leur firent de même.
\VS{37}Enfin, il envoya vers eux son propre fils, en disant : Ils auront du respect pour mon fils.
\VS{38}Mais quand les vignerons virent le fils, ils dirent entre eux : Voici l'héritier ; venez, tuons-le, et emparons-nous de son héritage.
\VS{39}Et ils se saisirent de lui, le jetèrent hors de la vigne, et le tuèrent.
\VS{40}Maintenant donc, lorsque le maître de la vigne viendra, que fera-t-il à ces vignerons ?
\VS{41}Ils lui dirent : Il fera périr malheureusement ces méchants, et il affermera la vigne à d'autres vignerons, qui lui en rendront les fruits en leur saison.
\VS{42}Et Jésus leur dit : N'avez-vous jamais lu dans les Ecritures : La pierre qu’ont rejetée ceux qui bâtissaient, est devenue la principale de l’angle. C’est du Seigneur que cela est venu, et c’est un prodige à nos yeux (4) ?
\VS{43}C'est pourquoi je vous dis que le Royaume de Dieu vous sera enlevé, et il sera donné à une nation qui en rendra les fruits.
\VS{44}Celui qui tombera sur cette pierre s’y brisera, et celui sur qui elle tombera sera écrasé.
\VS{45}Après avoir entendu ses paraboles, les principaux sacrificateurs et les pharisiens comprirent qu'il parlait d'eux.
\VS{46}Et ils cherchaient à se saisir de lui, mais ils craignaient la foule, parce qu’elle le tenait pour un prophète.
\TextTitle{[Parabole des noces]
\\(Lu. 14:16-24)}
\Chap{22}
\VerseOne{}Jésus, prenant la parole, leur parla de nouveau en paraboles, et il dit :
\VS{2}Le Royaume des cieux est semblable à un roi qui fit des noces pour son fils.
\VS{3}Il envoya ses serviteurs pour appeler ceux qui avaient été conviés aux noces ; mais ils ne voulurent pas venir.
\VS{4}Il envoya encore d'autres serviteurs, disant : Dites aux conviés : Voici, j'ai préparé mon festin ; mes bœufs et mes bêtes grasses sont tués, et tout est prêt ; venez aux noces.
\VS{5}Mais, sans tenir compte de l’invitation, ils s’en allèrent l'un à son champ, et l'autre à son trafic.
\VS{6}Et les autres se saisirent de ses serviteurs, les outragèrent, et les tuèrent.
\VS{7}Quand le roi l'entendit, il se mit en colère ; il envoya ses troupes, fit périr ces meurtriers, et brûla leur ville.
\VS{8}Puis il dit à ses serviteurs : Les noces sont prêtes, mais les conviés n’en étaient pas dignes.
\VS{9}Allez donc dans les carrefours des chemins, et autant de gens que vous trouverez, appelez-les aux noces.
\VS{10}Ces serviteurs allèrent dans les chemins, et rassemblèrent tous ceux qu'ils trouvèrent, méchants et bons, et la salle des noces fut remplie de conviés qui étaient à table.
\VS{11}Et le roi étant entré pour voir ceux qui étaient à table, il aperçut là un homme qui n’avait pas revêtu un habit de noces (1).
\VS{12}Et il lui dit : Mon ami, comment es-tu entré ici sans avoir un habit de noces ? Cet homme eut la bouche fermée.
\VS{13}Alors le roi dit aux serviteurs : Liez-lui les pieds et les mains, emportez-le et jetez-le dans les ténèbres de dehors, où il y aura des pleurs et des grincements de dents.
\VS{14}Car il y a beaucoup d'appelés, mais peu d'élus.
\TextTitle{[Le tribut dû à César]
\\(Mc. 12:13-17 ; Lu. 20:19-26)}
\VS{15}Alors les pharisiens allèrent se consulter ensemble sur les moyens de le surprendre par ses propres paroles.
\VS{16}Ils envoyèrent auprès de lui leurs disciples, avec des hérodiens, qui dirent : Maître, nous savons que tu es véritable, que tu enseignes la voie de Dieu selon la vérité, sans t’inquiéter de personne ; car tu ne regardes point à l'apparence des hommes.
\VS{17}Dis-nous donc ce qu'il t’en semble : Est-il permis de payer le tribut à César, ou non ?
\VS{18}Et Jésus connaissant leur malice, dit : Hypocrites, pourquoi me tentez-vous ?
\VS{19}Montrez-moi la monnaie avec laquelle on paie le tribut ; et ils lui présentèrent un denier.
\VS{20}Il leur demanda : De qui porte-t-il l’image et l’inscription ?
\VS{21}De César, lui répondirent-ils. Alors il leur dit : Rendez donc à César ce qui est à César, et à Dieu, ce qui est à Dieu.
\VS{22}Après avoir entendu cela, ils furent étonnés, ils le quittèrent, et s’en allèrent.
\TextTitle{[Enseignement de Jésus sur la résurrection]}
\VS{23}Le même jour, les sadducéens, qui disent qu'il n'y a pas de résurrection, vinrent auprès de lui, et lui posèrent cette question :
\VS{24}Maître, Moïse a dit : Si quelqu'un meurt sans enfants, son frère épousera sa femme et suscitera une postérité à son frère.
\VS{25}Or, il y avait parmi nous sept frères. Le premier se maria et mourut ; et, n'ayant pas eu d'enfants, il laissa sa femme à son frère.
\VS{26}Il en fut de même du deuxième, puis du troisième, jusqu’au septième.
\VS{27}Après eux tous, la femme mourut aussi.
\VS{28}A la résurrection, duquel des sept sera-t-elle la femme ? Car tous l'ont eue.
\VS{29}Jésus leur répondit : Vous êtes dans l’erreur, parce que vous ne connaissez ni les Ecritures, ni la puissance de Dieu.
\VS{30}Car, à la résurrection, les hommes ne prendront point de femmes, ni les femmes de maris, mais ils seront comme les anges de Dieu dans le ciel.
\VS{31}Et quant à la résurrection des morts, n'avez-vous point lu ce que Dieu vous a dit :
\VS{32}Je suis le Dieu d'Abraham, le Dieu d'Isaac, et le Dieu de Jacob (2). Or Dieu n'est pas le Dieu des morts, mais des vivants.
\VS{33}La foule, qui écoutait, fut frappée de sa doctrine.
\TextTitle{[Le plus grand commandement de la loi]
\\(Mc. 12:28-34 ; cp. Lu. 10:25-28)}
\VS{34}Quand les pharisiens apprirent qu'il avait réduit au silence les sadducéens, ils se rassemblèrent dans un même lieu,
\VS{35}et l'un d'eux, qui était docteur de la loi, lui posa cette question, pour l'éprouver :
\VS{36}Maître, quel est le plus grand commandement de la loi ?
\VS{37}Jésus lui dit : Tu aimeras le Seigneur ton Dieu de tout ton cœur, de toute ton âme, et de toute ta pensée.
\VS{38}C’est le premier et le plus grand commandement.
\VS{39}Et voici le deuxième qui lui est semblable : Tu aimeras ton prochain comme toi-même.
\VS{40}De ces deux commandements dépendent toute la loi et les prophètes.
\TextTitle{[Jésus interroge les pharisiens au sujet du Messie]
\\(Mc. 12:35-37 ; Lu. 20:39-44)}
\VS{41}Comme les pharisiens étaient assemblés, Jésus les interrogea,
\VS{42}en disant : Que pensez-vous du Christ ? De qui est-il Fils ? Ils lui répondirent : De David.
\VS{43}Et il leur dit : Comment donc David, animé par l'Esprit, l'appelle-t-il son Seigneur ? Lorsqu’il dit :
\VS{44}Le Seigneur a dit à mon Seigneur, assieds-toi à ma droite, jusqu'à ce que je fasse de tes ennemis ton marchepied (3).
\VS{45}Si donc David l'appelle son Seigneur, comment est-il son Fils ?
\VS{46}Et personne ne pouvait lui répondre un seul mot. Et depuis ce jour, personne n'osa plus lui poser des questions.
\TextTitle{[Caractéristiques des scribes et des pharisiens]
\\(Mc. 12:38-40 ; Lu. 20:45-47)}
\Chap{23}
\VerseOne{}Alors Jésus parlant à la foule, et à ses disciples,
\VS{2}dit : Les scribes et les pharisiens sont assis dans la chaire de Moïse.
\VS{3}Faites donc et observez tout ce qu’ils vous disent ; mais n’agissez pas selon leurs œuvres. Car ils disent, et ne font pas.
\VS{4}Ils lient ensemble des fardeaux pesants et insupportables, et les mettent sur les épaules des hommes ; mais ils ne veulent point les remuer de leur doigt.
\VS{5}Et ils font toutes leurs œuvres pour être vus des hommes. Ainsi, ils portent de larges phylactères, et de longues franges à leurs vêtements.
\VS{6}Ils aiment les premières places dans les festins, et les premiers sièges dans les synagogues.
\VS{7}Ils aiment les salutations dans les places publiques, et à être appelés par les hommes : Rabbi, Rabbi.
\VS{8}Mais vous, ne vous faites pas appeler Rabbi ; car Christ seul est votre Maître ; et vous êtes tous frères.
\VS{9}Et n'appelez personne sur la terre votre père ; car un seul est votre Père, celui qui est dans les cieux.
\VS{10}Ne vous faites pas appeler directeurs ; car Christ seul est votre Directeur.
\VS{11}Que le plus grand parmi vous, soit votre serviteur.
\VS{12}Car quiconque s'élèvera sera abaissé ; et quiconque s'abaissera, sera élevé.
\TextTitle{[Malheurs sur les scribes et les pharisiens]
\\(Mc. 12:38-40 ; Lu. 11:39-54;20:45-47}
\VS{13}Malheur à vous, scribes et pharisiens hypocrites ! Parce que vous fermez le Royaume des cieux aux hommes ; vous n'y entrez pas vous-mêmes, et vous n’y laissez pas entrer ceux qui veulent entrer.
\VS{14}Malheur à vous, scribes et pharisiens hypocrites ! Parce que vous dévorez les maisons des veuves, et que vous faites pour l’apparence de longues prières, c'est pourquoi vous serez jugés plus sévèrement.
\VS{15}Malheur à vous, scribes et pharisiens hypocrites ! Parce que vous courez la mer et la terre pour faire un prosélyte, et quand il l'est devenu, vous le rendez fils de la géhenne, deux fois plus que vous.
\VS{16}Malheur à vous conducteurs aveugles, qui dites : Si quelqu’un jure par le temple, ce n'est rien ; mais si quelqu’un jure par l'or du temple, il est engagé.
\VS{17}Insensés et aveugles ! Car lequel est le plus grand, l'or, ou le temple qui sanctifie l'or ?
\VS{18}Si quelqu’un, dites-vous encore, jure par l'autel, ce n'est rien ; mais si quelqu’un jure par l’offrande qui est sur l'autel, il est engagé.
\VS{19}Insensés et aveugles ! Car lequel est le plus grand, l’offrande, ou l'autel qui sanctifie l’offrande ?
\VS{20}Celui donc qui jure par l'autel, jure par l'autel et par toutes les choses qui sont dessus.
\VS{21}Celui qui jure par le temple, jure par le temple et par celui qui y habite ;
\VS{22}et celui qui jure par le ciel, jure par le trône de Dieu et par celui qui y est assis.
\VS{23}Malheur à vous, scribes et pharisiens hypocrites ! Parce que vous payez la dîme (1) de la menthe, de l'aneth et du cumin ; et vous laissez les choses les plus importantes de la loi, c'est-à-dire la justice, la miséricorde et la fidélité. Il fallait pratiquer ces choses-là, sans négliger les autres choses.
\VS{24}Conducteurs aveugles ! Vous coulez le moucheron et vous engloutissez le chameau (2).
\VS{25}Malheur à vous, scribes et pharisiens hypocrites ! Parce que vous nettoyez le dehors de la coupe et du plat ; alors qu’au-dedans ils sont pleins de rapine et d'intempérance.
\VS{26}Pharisien aveugle, nettoie premièrement l’intérieur de la coupe et du plat, afin que l’extérieur aussi devienne net.
\VS{27}Malheur à vous, scribes et pharisiens hypocrites ! Parce que vous êtes semblables aux sépulcres blanchis, qui paraissent beaux au-dehors, et qui au-dedans sont pleins d'ossements de morts, et de toutes espèces d’impuretés.
\VS{28}Ainsi, au-dehors vous paraissez justes aux hommes, mais au-dedans vous êtes pleins d'hypocrisie et d'iniquité.
\VS{29}Malheur à vous, scribes et pharisiens hypocrites ! Parce que vous bâtissez les tombeaux des prophètes, et vous ornez les sépulcres des justes ;
\VS{30}et vous dites : Si nous avions vécu du temps de nos pères, nous n'aurions pas participé avec eux au meurtre des prophètes.
\VS{31}Ainsi vous êtes témoins contre vous-mêmes, que vous êtes les enfants de ceux qui ont fait mourir les prophètes.
\VS{32}Comblez donc la mesure de vos pères.
\VS{33}Serpents, race de vipères ! Comment éviterez-vous le supplice de la géhenne ?
\VS{34}Car voici, je vous envoie des prophètes, des sages et des scribes. Vous tuerez et crucifierez les uns, vous battrez de verges les autres dans vos synagogues, et vous les persécuterez de ville en ville,
\VS{35}afin que retombe sur vous tout le sang innocent qui a été répandu sur la terre, depuis le sang d'Abel le juste, jusqu’au sang de Zacharie, fils de Barachie, que vous avez tué entre le temple et l'autel.
\VS{36}Je vous le dis en vérité, que toutes ces choses viendront sur cette génération.
\TextTitle{[Jésus se lamente sur Jérusalem]
\\(Lu. 13:34-35 ; 19:41-44 ; cp. Jé. 22:5)}
\VS{37}Jérusalem, Jérusalem, qui tues les prophètes, et qui lapides ceux qui te sont envoyés, combien de fois ai-je voulu rassembler tes enfants, comme la poule rassemble ses poussins sous ses ailes, et vous ne l'avez point voulu !
\VS{38}Voici, votre maison va devenir déserte.
\VS{39}Car je vous dis, que désormais vous ne me verrez plus, jusqu'à ce que vous disiez : Béni soit celui qui vient au Nom du Seigneur (3) !
\TextTitle{[Prophétie sur la destruction du temple de Jérusalem]
\\(Mc. 13:1-2 ; Lu. 21:5-6)}
\Chap{24}
\VerseOne{}Comme Jésus sortait et s'en allait du temple, ses disciples s'approchèrent de lui pour lui faire remarquer les constructions.
\VS{2}Mais Jésus leur dit : Voyez-vous tout cela ? Je vous le dis en vérité, il ne restera pas ici pierre sur pierre qui ne soit démolie.
\TextTitle{["Quand cela arrivera-t-il?" "Quel sera le signe?"]
\\(Mc. 13:3-4 ; Lu. 21:7)}
\VS{3}Il s’assit sur la montagne des oliviers. Et les disciples vinrent en particulier lui poser cette question : Dis-nous quand cela arrivera-t-il, et quel sera le signe de ton avènement, et de la fin du monde ?
\TextTitle{[La 70ème semaine d'années de Daniel]
\\(Da. 9:27) ; les temsp de la fin (Mc. 13:5-13 ; cp. Lu. 21:8-11)}
\VS{4}Jésus leur répondit : Prenez garde que personne ne vous séduise.
\VS{5}Car plusieurs viendront sous mon Nom, disant : Je suis le Christ. Et ils séduiront beaucoup de gens.
\VS{6}Vous entendrez parler de guerres et de bruits de guerres ; gardez-vous d’être troublés ; car il faut que toutes ces choses arrivent ; mais ce ne sera pas encore la fin.
\VS{7}Une nation s'élèvera contre une autre nation, et un royaume contre un autre royaume ; et il y aura des famines, des pestes, et des tremblements de terre en divers lieux.
\VS{8}Mais toutes ces choses ne seront que le commencement des douleurs.
\VS{9}Alors ils vous livreront aux tourments, et vous tueront ; et vous serez haïs de toutes les nations, à cause de mon Nom.
\VS{10}Alors aussi plusieurs seront scandalisés, se trahiront et se haïront les uns les autres.
\VS{11}Plusieurs faux prophètes s'élèveront, et ils séduiront beaucoup de gens.
\VS{12}Et parce que l'iniquité sera multipliée, la charité de plusieurs se refroidira.
\VS{13}Mais celui qui persévérera jusqu'à la fin, sera sauvé.
\VS{14}Cet Evangile du Royaume sera prêché dans toute la terre habitable, pour servir de témoignage à toutes les nations, et alors viendra la fin.
\TextTitle{[Au milieu de la 70ème semaine d'années de Daniel : l'abomination de la désolation]
\\(Mc. 13:14-18 ; Lu. 21:20-23)}
\VS{15}C’est pourquoi, lorsque vous verrez l'abomination de la désolation, qui a été prédite par Daniel le prophète (1), être établie dans le lieu saint, que celui qui lit ce prophète fasse attention !
\VS{16}Alors, que ceux qui seront en Judée fuient dans les montagnes ;
\VS{17}que celui qui sera sur le toit ne descende pas pour emporter quoi que ce soit de sa maison ;
\VS{18}que celui qui sera dans les champs ne retourne pas en arrière pour prendre ses habits.
\VS{19}Malheur aux femmes enceintes, et à celles qui allaiteront en ces jours-là.
\VS{20}Priez pour que votre fuite n’arrive pas en hiver, ni un jour de sabbat (2).
\TextTitle{[La seconde moitié de la 70ème semaine d'années de Daniel : la grande Tribulation]
\\(mc. 13:19-23 ; cp. Ps. 2:5 ; lu. 21:23-24)}
\VS{21}Car alors, la détresse sera si grande qu’il n’y en a point eu de semblable depuis le commencement du monde jusqu’à présent, et qu’il n’y en aura jamais.
\VS{22}Et si ces jours n’étaient abrégés, personne ne serait sauvé ; mais à cause des élus, ces jours seront abrégés.
\VS{23}Alors si quelqu'un vous dit : Voici, le Christ est ici ; ou, il est là ; ne le croyez point.
\VS{24}Car il s'élèvera de faux christs et de faux prophètes, ils feront de grands prodiges et des miracles, pour séduire même les élus, s'il était possible.
\VS{25}Voici, je vous l'ai prédit.
\VS{26}Si on vous dit : Voici, il est dans le désert, ne sortez point ; voici, il est dans les chambres, ne le croyez point.
\VS{27}Car, comme l'éclair part de l'Orient, et se montre jusqu'en Occident, il en sera de même de l'avènement du Fils de l'homme.
\VS{28}Car là où est le cadavre, là s'assembleront les vautours.
\TextTitle{[Retour du Roi de la terre à la fin de la tribulation]
\\(Mc. 13:24-27 ; Lu. 21:25-28)}
\VS{29}Aussitôt après ces jours de détresse, le soleil s’obscurcira, la lune ne donnera plus sa lumière, et les étoiles tomberont du ciel, et les puissances des cieux seront ébranlées.
\VS{30}Alors le signe du Fils de l'homme paraîtra dans le ciel, toutes les tribus de la terre se lamenteront en se frappant la poitrine, et verront le Fils de l'homme venant sur les nuées du ciel, avec une grande puissance, et une grande gloire.
\VS{31}Il enverra ses anges avec un grand son de trompette, et ils rassembleront ses élus, des quatre vents, d’une extrémité des cieux à l’autre.
\TextTitle{[Parabole du figuier]
\\(Mc. 13:28-31 ; Lu. 21:29-33)}
\VS{32}Instruisez-vous par la parabole tirée du figuier. Dès que ses branches deviennent tendres, et que les feuilles poussent, vous savez que l'été est proche.
\VS{33}De même, quand vous verrez toutes ces choses, sachez que le Fils de l'homme est proche, à la porte.
\VS{34}Je vous le dis en vérité, cette génération ne passera point, jusqu’à ce que tout cela n’arrive.
\VS{35}Le ciel et la terre passeront, mais mes paroles ne passeront point.
\TextTitle{[Exhortation à la vigilance]
\\(Mc. 13:32-37 ; Lu. 21:34-38)}
\VS{36}Pour ce qui est du jour et de l’heure, personne ne le sait, ni les anges des cieux, mais mon Père seul.
\VS{37}Ce qui arriva du temps de Noé, arrivera de même à l'avènement du Fils de l'homme.
\VS{38}Car, dans les jours qui précédèrent le déluge, les hommes mangeaient et buvaient, se mariaient, et donnaient en mariage leurs enfants, jusqu'au jour où Noé entra dans l'arche ;
\VS{39}et ils ne se doutèrent de rien, jusqu’à ce que le déluge, vienne et les emporte tous ; il en sera de même à l'avènement du Fils de l'homme.
\VS{40}Alors, de deux hommes qui seront dans un champ ; l'un sera pris, et l'autre laissé ;
\VS{41}de deux femmes qui moudront à la meule, l'une sera prise, et l'autre laissée.
\VS{42}Veillez donc, car vous ne savez point à quelle heure votre Seigneur doit venir.
\VS{43}Sachez-le bien, si un père de famille savait à quelle veille de la nuit le voleur doit venir, il veillerait et ne laisserait pas percer sa maison.
\VS{44}C'est pourquoi, vous aussi tenez-vous prêts ; car le Fils de l'homme viendra à l'heure où vous n'y penserez pas.
\VS{45}Quel est donc le serviteur fidèle et prudent, que son maître a établi sur tous ses serviteurs, pour leur donner la nourriture au temps convenable ?
\VS{46}Heureux est ce serviteur que son maître en arrivant trouvera agir de cette manière.
\VS{47}Je vous le dis en vérité, il l'établira sur tous ses biens.
\VS{48}Mais si c'est un méchant serviteur, qui dit en lui-même : Mon maître tarde à venir ;
\VS{49}et s’il se met à battre ses compagnons de service, s’il mange et boit avec les ivrognes,
\VS{50}le maître de ce serviteur viendra le jour où il ne s’y attend pas, et à l'heure qu'il ne connaît pas.
\VS{51}Et il le mettra en pièces, et lui donnera sa part avec les hypocrites ; là où il y aura des pleurs et des grincements de dents.
\TextTitle{[Parabole des dix vierges]}
\Chap{25}
\VerseOne{}Alors le Royaume des cieux sera semblable à dix vierges qui, ayant pris leurs lampes, allèrent à la rencontre de l'époux.
\VS{2}Cinq d’entre elles étaient sages, et cinq folles.
\VS{3}Les folles, en prenant leurs lampes, ne prirent pas d'huile avec elles ;
\VS{4}mais les sages prirent de l'huile dans leurs vases avec leurs lampes.
\VS{5}Et comme l'époux tardait à venir, elles s’assoupirent et s'endormirent toutes.
\VS{6}A minuit un cri s’éleva : Voici, l'époux vient, allez à sa rencontre !
\VS{7}Alors toutes ces vierges se réveillèrent (1), et préparèrent leurs lampes.
\VS{8}Et les folles dirent aux sages : Donnez-nous de votre huile, car nos lampes s'éteignent.
\VS{9}Mais les sages répondirent : Nous ne pouvons pas vous en donner, de peur que nous n'en ayons pas assez pour nous et pour vous ; mais allez plutôt chez ceux qui en vendent, et achetez-en pour vous.
\VS{10}Pendant qu'elles allaient en acheter, l'époux arriva. Celles qui étaient prêtes entrèrent avec lui dans la salle des noces, puis la porte fut fermée.
\VS{11}Plus tard, les autres vierges vinrent, et dirent : Seigneur ! Seigneur ! Ouvre-nous !
\VS{12}Mais il leur répondit : Je vous le dis en vérité, je ne vous connais point.
\VS{13}Veillez donc, puisque vous ne savez ni le jour ni l'heure où le Fils de l'homme viendra.
\TextTitle{[Parabole des talents]}
\VS{14}Car il en sera comme d'un homme qui, partant pour un voyage, appela ses serviteurs, et leur remit ses biens.
\VS{15}Il donna à l'un cinq talents, à l'autre deux, et au troisième un ; à chacun selon sa capacité ; et aussitôt après il partit.
\VS{16}Celui qui avait reçu les cinq talents, s'en alla, et les fit valoir, et gagna cinq autres talents.
\VS{17}De même, celui qui avait reçu les deux talents, en gagna aussi deux autres.
\VS{18}Mais celui qui n'en avait reçu qu'un, alla et creusa dans la terre, et y cacha l'argent de son maître.
\VS{19}Longtemps après, le maître de ces serviteurs revint, et leur fit rendre compte.
\VS{20}Alors celui qui avait reçu les cinq talents, vint, et présenta cinq autres talents, en disant : Seigneur, tu m'as confié cinq talents, voici, j'en ai gagné cinq autres par-dessus.
\VS{21}Et son Seigneur lui dit : C’est bien, bon et fidèle serviteur ; tu as été fidèle en peu de choses, je t'établirai sur beaucoup ; viens participer à la joie de ton Seigneur.
\VS{22}Ensuite, celui qui avait reçu les deux talents, vint, et dit : Seigneur, tu m'as confié deux talents ; voici, j'en ai gagné deux autres par-dessus.
\VS{23}Et son Seigneur lui dit : C’est bien, bon et fidèle serviteur, tu as été fidèle en peu de choses, je t'établirai sur beaucoup ; viens prendre part à la joie de ton Seigneur.
\VS{24}Mais celui qui n'avait reçu qu'un talent, vint, et dit : Seigneur, je savais que tu es un homme dur, qui moissonnes où tu n'as point semé, et qui amasses où tu n'as point vanné,
\VS{25}c'est pourquoi craignant de perdre ton talent, je suis allé le cacher dans la terre. Voici, tu as ici ce qui t'appartient.
\VS{26}Et son Seigneur lui répondit : Méchant et paresseux serviteur, tu savais que je moissonnais où je n'ai point semé, et que j'amassais où je n'ai point vanné,
\VS{27}il te fallait donc remettre mon argent aux banquiers, et à mon retour je l'aurais retiré avec l'intérêt.
\VS{28}Ôtez-lui donc le talent, et donnez-le à celui qui a les dix talents.
\VS{29}Car on donnera à celui qui a, et il sera dans l’abondance, mais à celui qui n’a pas on ôtera même ce qu’il a.
\VS{30}Jetez donc le serviteur inutile dans les ténèbres de dehors ; où il y aura des pleurs et des grincements de dents.
\TextTitle{[Jugement des individus membres des nations]
\\(1 Co 6:2)}
\VS{31}Lorsque le Fils de l'homme viendra dans sa gloire, avec tous les saints anges, il s'assiéra sur le trône de sa gloire.
\VS{32}Et toutes les nations seront assemblées devant lui ; et il séparera les uns d'avec les autres, comme le berger sépare les brebis d'avec les boucs.
\VS{33}Et il mettra les brebis à sa droite, et les boucs à sa gauche.
\VS{34}Alors le Roi dira à ceux qui seront à sa droite : Venez, vous qui êtes bénis de mon Père, prenez possession du Royaume qui vous a été préparé dès la fondation du monde.
\VS{35}Car j'ai eu faim, et vous m'avez donné à manger ; j'ai eu soif, et vous m'avez donné à boire ; j'étais étranger, et vous m'avez recueilli ;
\VS{36}j'étais nu, et vous m'avez vêtu ; j'étais malade, et vous m'avez visité ; j'étais en prison, et vous êtes venus vers moi.
\VS{37}Alors les justes lui répondront : Seigneur, quand t'avons-nous vu avoir faim, et t'avons-nous donné à manger ; ou avoir soif, et t'avons-nous donné à boire ?
\VS{38}Quand t'avons-nous vu étranger, et t'avons-nous recueilli ; ou nu, et t'avons-nous vêtu ?
\VS{39}Quand t'avons-nous vu malade, ou en prison, et sommes-nous allés vers toi ?
\VS{40}Et le Roi leur répondra : Je vous le dis en vérité, toutes les fois que vous avez fait ces choses à l'un de ces plus petits de mes frères, c’est à moi que vous les avez faites.
\VS{41}Alors il dira aussi à ceux qui seront à sa gauche : Maudits, retirez-vous de moi, et allez dans le feu éternel, qui a été préparé pour le diable et pour ses anges.
\VS{42}Car j'ai eu faim, et vous ne m'avez point donné à manger ; j'ai eu soif et vous ne m'avez point donné à boire ;
\VS{43}j'étais étranger, et vous ne m'avez point recueilli ; j'ai été nu, et vous ne m'avez point vêtu ; j'ai été malade et en prison, et vous ne m'avez point visité.
\VS{44}Ils répondront aussi : Seigneur, quand t’avons-nous vu avoir faim, ou avoir soif, ou être étranger, ou nu, ou malade, ou en prison, et ne t'avons-nous point secouru ?
\VS{45}Alors il leur répondra : Je vous le dis en vérité, toutes les fois que vous n'avez pas fait ces choses à l'un de ces plus petits, c’est à moi que vous ne les avez pas faites.
\VS{46}Et ceux-ci iront au châtiment éternel, mais les justes à la vie éternelle.
\TextTitle{[Les autorités juives complotent la mort de Jésus-Christ]
\\(Mc. 14:1-2 ; Lu. 22:1-2)}
\Chap{26}
\VerseOne{}Lorsque Jésus eut achevé tous ces discours, il dit à ses disciples :
\VS{2}Vous savez que la fête de Pâque a lieu dans deux jours ; et le Fils de l'homme sera livré pour être crucifié.
\VS{3}Alors les principaux sacrificateurs, les scribes, et les anciens du peuple, se réunirent dans la cour du souverain sacrificateur, appelé Caïphe ;
\VS{4}et ils tinrent conseil ensemble sur les moyens d’arrêter Jésus par ruse, afin de le faire mourir.
\VS{5}Mais ils dirent : Que ce ne soit pas pendant la fête, afin qu’il n’y ait pas de tumulte parmi le peuple.
\TextTitle{[Marie de Béthanie oint Jésus pour Sa sépulture]
\\(Mc. 14:3-9 ; Jn. 12:1-8)}
\VS{6}Comme Jésus était à Béthanie, dans la maison de Simon le lépreux,
\VS{7}une femme s’approcha de lui tenant un vase d'albâtre, plein d'un parfum de grand prix, et pendant qu’il était à table, elle répandit le parfum sur sa tête.
\VS{8}Ses disciples voyant cela, furent indignés, et dirent : A quoi sert cette perte ?
\VS{9}On aurait pu vendre ce parfum très cher, et en donner le prix aux pauvres.
\VS{10}Mais Jésus, s’en étant aperçu, leur dit : Pourquoi faites-vous de la peine à cette femme ? Car elle a fait une bonne action à mon égard ;
\VS{11}car vous aurez toujours des pauvres avec vous ; mais vous ne m'aurez pas toujours.
\VS{12}En répandant ce parfum sur mon corps, elle l'a fait pour ma sépulture.
\VS{13}Je vous le dis en vérité, partout où cet Evangile sera prêché, dans le monde entier, on racontera aussi en mémoire de cette femme ce qu’elle a fait.
\TextTitle{[Judas trahit Jésus]
\\(Mc. 14:10-11 ; Lu. 22:3-6)}
\VS{14}Alors l'un des douze, appelé Judas Iscariot, alla vers les principaux sacrificateurs,
\VS{15}et dit : Que voulez-vous me donner, et je vous le livrerai ? Et ils lui payèrent trente pièces d'argent (1).
\VS{16}Depuis ce moment, il cherchait une occasion favorable pour livrer Jésus.
\TextTitle{[Préparation de la Pâque]
\\(Mc. 14:12-16 ; Lu. 22:7-13)}
\VS{17}Le premier jour des pains sans levain, les disciples s’approchèrent de Jésus pour lui dire : Où veux-tu que nous te préparions le repas de la Pâque ?
\VS{18}Il répondit : Allez à la ville chez un tel, et dites-lui : Le Maître dit : Mon temps est proche ; je ferai la Pâque chez toi avec mes disciples.
\VS{19}Les disciples firent ce que Jésus leur avait ordonné, et préparèrent la Pâque.
\TextTitle{[La dernière Pâque]
\\(Mc. 14:17-21 ; Lu. 22:14-20 ; cp. Jn. 13:1-12)}
\VS{20}Le soir étant venu, il se mit à table avec les douze.
\VS{21}Pendant qu’ils mangeaient, il dit : Je vous le dis en vérité, l'un de vous me trahira.
\VS{22}Ils furent profondément attristés, et chacun d'eux commença à lui dire : Seigneur, est-ce moi ?
\VS{23}Mais il leur répondit : Celui qui a mis avec moi la main dans le plat pour tremper, c'est celui qui me trahira.
\VS{24}Le Fils de l'homme s'en va, selon qu'il est écrit de lui ; mais malheur à cet homme par qui le Fils de l'homme est trahi ! Mieux vaudrait pour cet homme qu’il ne soit pas né.
\VS{25}Judas qui le trahissait, prit la parole et dit : Maître, est-ce moi ? Jésus lui dit : Tu l'as dit.
\TextTitle{[Le repas de la pâque]
\\(Mc. 14:22-25 ; Lu. 22:17-20 ; Jn. 13:12-30 ; cp. 1 Co. 11:23-26)}
\VS{26}Pendant qu’ils mangeaient, Jésus prit le pain, et après avoir rendu grâces à Dieu, il le rompit et le donna à ses disciples, et leur dit : Prenez, mangez, ceci est mon corps.
\VS{27}Il prit ensuite la coupe, et après avoir rendu grâces à Dieu, il la leur donna, en leur disant : Buvez-en tous ;
\VS{28}car ceci est mon sang, le sang de l’alliance, qui est répandu pour beaucoup, pour la rémission des péchés.
\VS{29}Je vous le dis, je ne boirai plus désormais de ce fruit de la vigne, jusqu'au jour où j’en boirai du nouveau avec vous, dans le Royaume de mon Père.
\TextTitle{[Jésus avertit Pierre de son triple reniement]
\\(Mc. 14:26-31 ; Lu. 22:31-34 ; Jn. 13:36-38)}
\VS{30}Après avoir chanté les cantiques (2), ils se rendirent à la montagne des oliviers.
\VS{31}Alors Jésus leur dit : Je serai pour vous tous, cette nuit, une occasion de chute ; car il est écrit : Je frapperai le Berger, et les brebis du troupeau seront dispersées (3).
\VS{32}Mais, après que je serai ressuscité, je vous précéderai en Galilée.
\VS{33}Pierre, prenant la parole, lui dit : Même si tu étais pour tous une occasion de chute, tu ne le seras jamais pour moi.
\VS{34}Jésus lui dit : Je te le dis en vérité, cette nuit même, avant que le coq chante, tu me renieras trois fois.
\VS{35}Pierre lui répondit : Même s’il me fallait mourir avec toi, je ne te renierai pas. Et tous les disciples dirent la même chose.
\TextTitle{[Jésus dans le jardin de Gethsémané]
\\(Mc. 14:32-42 ; Lu. 22:39-46 ; Jn. 18:1)}
\VS{36}Alors Jésus alla avec eux dans un lieu appelé Gethsémané, et il dit à ses disciples : Asseyez-vous ici, pendant que je m’éloignerai pour prier.
\VS{37}Il prit avec lui Pierre et les deux fils de Zébédée, et il commença à éprouver de la tristesse et des angoisses.
\VS{38}Il leur dit : Mon âme est tristesse jusqu’à la mort ; restez ici, et veillez avec moi.
\TextTitle{[première prière de Jésus]
\\(Mc. 14:35-38 ; Lu. 22:41-42)}
\VS{39}Puis, ayant fait quelques pas en avant, il se prosterna le visage contre terre, et pria ainsi : Mon Père, s'il est possible, que cette coupe s’éloigne de moi ! Toutefois, non pas ce que je veux, mais ce que tu veux.
\TextTitle{[Jésus trouve les disciples endormis]
\\(Mc. 14:37-40 ; Lu. 22:45-46)}
\VS{40}Puis il vint vers ses disciples, qu’il trouva endormis, et il dit à Pierre : Vous n’avez pas pu veiller une heure avec moi ?
\VS{41}Veillez et priez, afin que vous ne tombiez pas en tentation ; l'esprit est bien disposé, mais la chair est faible.
\TextTitle{[Deuxième prière]
\\(Mc. 14:39 ; Lu. 22:44)}
\VS{42}Il s’éloigna encore pour la seconde fois, et il pria, disant : Mon Père, s'il n'est pas possible que cette coupe s’éloigne sans que je la boive, que ta volonté soit faite.
\VS{43}Il revint ensuite, et les trouva encore endormis ; car leurs yeux étaient appesantis.
\TextTitle{[Troisième prière]
\\(Mc. 14:41)}
\VS{44}Et les ayant laissés, il s'en alla encore, et pria pour la troisième fois, disant les mêmes paroles.
\VS{45}Puis il alla vers ses disciples, et leur dit : Dormez maintenant, et reposez-vous ; voici, l'heure est proche, et le Fils de l'homme va être livré entre les mains des méchants.
\VS{46}Levez-vous, allons. Voici, celui qui me trahit s'approche.
\TextTitle{[Jésus trahi et arrêté]
\\(Mc. 14:43-50 ; Lu. 22:47-53 ; Jn. 18:2-11)}
\VS{47}Comme il parlait encore, voici, Judas, l'un des douze, vint, et avec lui une grande foule, avec des épées et des bâtons, envoyée par les principaux sacrificateurs et par les anciens du peuple.
\VS{48}Celui qui le trahissait leur avait donné ce signe : Celui à qui je donnerai un baiser, c'est lui, saisissez-le.
\VS{49}Aussitôt, s'approchant de Jésus, il lui dit : Salut. Rabbi ! Et il lui donna un baiser.
\VS{50}Jésus lui dit : Mon ami, ce que tu es venu faire, fais-le. Alors s'étant approchés, ils mirent la main sur Jésus, et le saisirent.
\VS{51}Et voici, l'un de ceux qui étaient avec Jésus, étendit la main et tira son épée ; il frappa le serviteur du souverain sacrificateur, et lui emporta l'oreille.
\VS{52}Alors Jésus lui dit : Remets ton épée à sa place ; car tous ceux qui prendront l'épée, périront par l'épée.
\VS{53}Crois-tu que je ne puisse pas maintenant prier mon Père, qui me donnerait à l’instant plus de douze légions d'anges ?
\VS{54}Comment donc s’accompliraient les Ecritures qui disent qu'il faut que cela arrive ainsi ?
\VS{55}En ce même instant Jésus dit à la foule : Vous êtes venus avec des épées et des bâtons, comme après un brigand, pour me prendre ; j'étais tous les jours assis parmi vous, enseignant dans le temple, et vous ne m'avez pas saisi.
\VS{56}Mais tout ceci est arrivé afin que les Ecritures des prophètes soient accomplies. Alors tous les disciples l'abandonnèrent, et s'enfuirent.
\TextTitle{[Jésus comparaît devant Caïphe et le sanhédrin]
\\(Mc. 14:53-65 ; cp. Jn. 18:12-14, 19-24)}
\VS{57}Ceux qui avaient saisi Jésus l'amenèrent chez Caïphe, le souverain sacrificateur, où les scribes et les anciens étaient assemblés.
\VS{58}Pierre le suivit de loin, jusqu’à la cour du souverain sacrificateur, y entra, et s’assit avec les officiers pour voir comment cela finirait.
\VS{59}Les principaux sacrificateurs, les anciens et tout le sanhédrin cherchaient des faux témoignages contre Jésus, pour le faire mourir.
\VS{60}Mais ils n'en trouvèrent point, bien que plusieurs faux témoins se soient présentés. Mais à la fin, deux faux témoins s'approchèrent,
\VS{61}et dirent : Celui-ci a dit : Je puis détruire le temple de Dieu, et le rebâtir en trois jours.
\VS{62}Alors le souverain sacrificateur se leva, et lui dit : Ne réponds-tu rien ? Qu’est-ce que ces hommes déposent contre toi ?
\VS{63}Jésus garda le silence. Et le souverain sacrificateur prenant la parole, lui dit : Je t’adjure par le Dieu vivant, de nous dire si tu es le Christ, le Fils de Dieu.
\VS{64}Jésus lui dit : Tu l'as dit. De plus, je vous dis que désormais vous verrez le Fils de l'homme assis à la droite de la puissance de Dieu, et venant sur les nuées du ciel.
\VS{65}Alors le souverain sacrificateur déchira ses vêtements, en disant : Il a blasphémé ! Qu’avons-nous encore besoin de témoins ? Voici, vous avez entendu maintenant son blasphème. Que vous en semble ?
\VS{66}Ils répondirent : Il mérite la mort.
\VS{67}Alors ils lui crachèrent au visage, et lui donnèrent des coups de poing et des soufflets, et les autres le frappaient avec leurs bâtons ;
\VS{68}en disant : Christ, prophétise, dis-nous qui t’a frappé.
\TextTitle{[Triple reniement de Pierre]
\\(Mc. 14:66-72 , Lu. 22:55-62 ; Jn. 18:15-18, 25-27)}
\VS{69}Cependant, Pierre était assis dehors dans la cour. Une servante s'approcha de lui, et lui dit : Toi aussi, tu étais aussi avec Jésus le Galiléen.
\VS{70}Mais il le nia devant tous, en disant : Je ne sais pas ce que tu veux dire.
\VS{71}Et comme il se dirigeait vers la porte, une autre servante le vit, et elle dit à ceux qui étaient là : Celui-ci aussi était avec Jésus de Nazareth.
\VS{72}Et il le nia encore avec serment, disant : Je ne connais pas cet homme.
\VS{73}Peu après, ceux qui se trouvaient là s'approchèrent, et dirent à Pierre : Certainement tu es aussi de ces gens-là, car ton langage te fait reconnaître.
\VS{74}Alors il commença à faire des imprécations et à jurer, en disant : Je ne connais pas cet homme. Et aussitôt le coq chanta.
\VS{75}Et Pierre se souvint de la parole de Jésus, qui lui avait dit : Avant que le coq chante, tu me renieras trois fois. Et étant sorti dehors, il pleura amèrement.
\TextTitle{[Jésus devant Pilate]}
\Chap{27}
\VerseOne{}Dès que le matin fut venu, tous les principaux sacrificateurs et les anciens du peuple tinrent conseil contre Jésus pour le faire mourir.
\VS{2}Après l’avoir lié, ils l'amenèrent et le livrèrent à Ponce Pilate, qui était le gouverneur.
\TextTitle{[Judas se suicide]
\\(cp. Ac. 1:16-19)}
\VS{3}Alors Judas qui l'avait trahi, voyant qu'il était condamné, se repentit, et rapporta les trente pièces d'argent aux principaux sacrificateurs et aux anciens,
\VS{4}en leur disant : J’ai péché en trahissant le sang innocent ; mais ils lui dirent : Que nous importe ? Cela te regarde.
\VS{5}Et après avoir jeté les pièces d'argent dans le temple, il se retira, et alla se pendre.
\VS{6}Mais les principaux sacrificateurs prirent les pièces d'argent, et dirent : Il n'est pas permis de les mettre dans le trésor ; car c’est le prix du sang.
\VS{7}Et, après en avoir délibéré, ils achetèrent avec cet argent le champ d'un potier, pour la sépulture des étrangers.
\VS{8}C'est pourquoi ce champ-là a été appelé jusqu'à aujourd'hui, le champ du sang.
\VS{9}Alors s’accomplit ce qui avait été annoncé par Jérémie le prophète : Ils ont pris les trente pièces d'argent, le prix de celui qui a été estimé, qu’on a estimé de la part des enfants d'Israël ;
\VS{10}et ils les ont données pour acheter le champ d'un potier, selon ce que le Seigneur l’avait ordonné (1).
\TextTitle{[Jésus comparaît devant Pilate]
\\(Mc. 15:2-5 ; Lu. 23:2-3 ; Jn. 18:28-38)}
\VS{11}Jésus comparut devant le gouverneur. Le gouverneur l'interrogea : Es-tu le Roi des Juifs ? Jésus lui répondit : Tu le dis.
\VS{12}Mais il ne répondit rien aux accusations des principaux sacrificateurs et des anciens.
\VS{13}Alors Pilate lui dit : N'entends-tu pas de combien de choses ils t’accusent ?
\VS{14}Mais il ne lui donna de réponse sur aucune parole, ce qui étonna beaucoup le gouverneur.
\TextTitle{[Jésus ou Barabbas ?]
\\(Mc. 15:6-15 ; Lu. 23:17-25 ; cp. Jn. 18:39-40)}
\VS{15}Or le gouverneur avait coutume de relâcher un prisonnier à chaque fête, celui que demandait la foule.
\VS{16}Et il y avait alors un prisonnier fameux, nommé Barabbas.
\VS{17}Comme ils étaient assemblés, Pilate leur dit : Lequel voulez-vous que je vous relâche ? Barabbas, ou Jésus qu'on appelle Christ ?
\VS{18}Car il savait bien qu'ils l'avaient livré par envie.
\VS{19}Pendant qu’il siégeait au tribunal, sa femme envoya lui dire : Ne te mêle point de l'affaire de ce juste, car j'ai beaucoup souffert aujourd'hui en songe à cause de lui.
\VS{20}Les principaux sacrificateurs et les anciens persuadèrent la multitude du peuple de demander Barabbas, et de faire périr Jésus.
\VS{21}Et le gouverneur prenant la parole leur dit : Lequel des deux voulez-vous que je vous relâche ? Ils dirent : Barabbas.
\VS{22}Pilate leur dit : Que ferai-je donc de Jésus qu'on appelle Christ ? Ils lui dirent tous : Qu’il soit crucifié !
\VS{23}Et le gouverneur leur dit : Mais quel mal a-t-il fait ? Et ils crièrent encore plus fort, en disant : Qu’il soit crucifié !
\VS{24}Alors Pilate voyant qu'il ne gagnait rien, mais que le tumulte s'augmentait, prit de l'eau et lava ses mains devant le peuple, en disant : Je suis innocent du sang de ce juste. Cela vous regarde.
\VS{25}Et tout le peuple répondit : Que son sang retombe sur nous et sur nos enfants !
\VS{26}Alors il leur relâcha Barabbas ; et après avoir fait battre de verges Jésus, il le livra pour être crucifié.
\TextTitle{[Le Roi couronné d'épines et conduit à Golgotha]
\\(Mc. 15:16-23 ; Lu. 23:26-32 ; Jn. 19:16-17)}
\VS{27}Les soldats du gouverneur amenèrent Jésus dans le prétoire, et assemblèrent devant lui toute la cohorte.
\VS{28}Et après l'avoir dépouillé, ils le revêtirent d’un manteau d'écarlate.
\VS{29}Puis, ayant fait une couronne d'épines entrelacées, ils la mirent sur sa tête, et ils lui mirent un roseau dans sa main droite ; puis s'agenouillant devant lui, ils se moquaient de lui, en disant : Nous te saluons, Roi des Juifs !
\VS{30}Et ils crachaient contre lui, prenaient le roseau, et frappaient sur sa tête.
\VS{31}Après s'être ainsi moqués de lui, ils lui ôtèrent le manteau, et lui remirent ses vêtements, et l'amenèrent pour le crucifier.
\VS{32}Comme ils sortaient, ils rencontrèrent un homme de Cyrène, appelé Simon, et ils le forcèrent à porter la croix de Jésus.
\TextTitle{[Jésus crucifié]
\\(Mc. 15:24-32 ; Lu. 23:33-43 ; Jn. 19:17-24)}
\VS{33}Arrivés au lieu appelé Golgotha, c'est-à-dire le lieu du crâne,
\VS{34}ils lui donnèrent à boire du vinaigre mêlé avec du fiel (2) ; mais quand il l’eut goûté, il ne voulut pas boire.
\VS{35}Et après l'avoir crucifié, ils partagèrent ses vêtements, en tirant au sort, afin que s’accomplisse ce qui avait été annoncé par le prophète : Ils se sont partagé mes vêtements, et ont jeté ma tunique au sort (3).
\VS{36}Puis ils s’assirent, et le gardèrent.
\VS{37}Ils mirent aussi au-dessus de sa tête un écriteau, où la cause de sa condamnation était marquée en ces mots : Celui-ci est Jésus, le Roi des Juifs.
\VS{38}Avec lui furent crucifiés deux brigands, l'un à sa droite, et l'autre à sa gauche.
\VS{39}Ceux qui passaient par là, l’injuriaient et secouaient la tête
\VS{40}en disant : Toi qui détruis le temple, et qui le rebâtis en trois jours, sauve-toi toi-même ! Si tu es le Fils de Dieu, descends de la croix !
\VS{41}De même, les principaux sacrificateurs avec les scribes et les anciens, se moquaient aussi de lui, et disaient :
\VS{42}Il a sauvé les autres, et il ne peut pas se sauver lui-même ! S’il est le Roi d'Israël, qu'il descende maintenant de la croix, et nous croirons en lui.
\VS{43}Il s’est confié en Dieu ; que Dieu le délivre maintenant, s’il aime, car il a dit : Je suis le Fils de Dieu.
\VS{44}Les brigands, crucifiés avec lui, l’insultaient de la même manière.
\TextTitle{[Jésus accomplit la loi par Sa mort]
\\(Mc. 15:33-41 ; Lu. 23:44-49 ; Jn. 19:30-37 ; Hé. 9:3-8 ; 10:19-20)}
\VS{45}Depuis la sixième heure jusqu’à la neuvième, il y eut des ténèbres sur toute la terre.
\VS{46}Et vers la neuvième heure, Jésus s'écria d’une voix forte : Eli, Eli, lama sabachthani ? C’est-à-dire : Mon Dieu ! Mon Dieu ! Pourquoi m'as-tu abandonné ?
\VS{47}Quelques-uns de ceux qui étaient là présents, ayant entendu cela, disaient : Il appelle Elie.
\VS{48}Et aussitôt l’un d'entre eux courut prendre une éponge, qu’il remplit de vinaigre, et l’ayant fixée au bout d'un roseau, lui donna à boire.
\VS{49}Mais les autres disaient : Laisse, voyons si Elie viendra le sauver.
\VS{50}Alors Jésus, poussa de nouveau un grand cri, et rendit l'esprit.
\TextTitle{[Le voile est déchiré : fin de la loi Mosaïque ou de la première alliance]}
\VS{51}Et voici, le voile du temple se déchira en deux, depuis le haut jusqu'en bas (4) ; et la terre trembla, et les pierres se fendirent.
\VS{52}Et les sépulcres s'ouvrirent, et plusieurs corps des saints qui étaient morts ressuscitèrent.
\VS{53}Et étant sortis des sépulcres après la résurrection de Jésus, ils entrèrent dans la ville sainte, et se montrèrent à plusieurs.
\VS{54}Le centenier, et ceux qui étaient avec lui pour garder Jésus, ayant vu le tremblement de terre, et tout ce qui venait d'arriver, furent saisis d’une grande frayeur, et dirent : Certainement cet homme était le Fils de Dieu.
\VS{55}Il y avait là aussi plusieurs femmes qui regardaient de loin, et qui avaient suivi Jésus depuis la Galilée, pour le servir.
\VS{56}Parmi elles étaient Marie de Magdala, Marie mère de Jacques et de Joseph, et la mère des fils de Zébédée.
\TextTitle{[Jésus enseveli]
\\(Mc. 15:42-47 ; Lu. 23:50-56 ; Jn. 19:38-42)}
\VS{57}Le soir étant venu, un homme riche d'Arimathée, appelé Joseph, qui était aussi disciple de Jésus,
\VS{58}se rendit vers Pilate, et demanda le corps de Jésus. Et Pilate ordonna qu’on le lui donne.
\VS{59}Joseph prit le corps, et l'enveloppa d'un linceul blanc ;
\VS{60}et le mit dans un sépulcre neuf, qu'il s’était fait tailler dans le roc. Puis il roula une grande pierre à l'entrée du sépulcre, et il s'en alla.
\VS{61}Marie de Magdala et l’autre Marie étaient là, assises vis-à-vis du sépulcre.
\TextTitle{[Le sépulcre scellé et gardé]}
\VS{62}Le lendemain, qui était le jour de la préparation du sabbat, les principaux sacrificateurs et les pharisiens allèrent ensemble auprès de Pilate,
\VS{63}et lui dirent : Seigneur ! Nous nous souvenons que ce séducteur disait, quand il était encore en vie : Après trois jours je ressusciterai.
\VS{64}Ordonne donc que le sépulcre soit gardé sûrement jusqu’au troisième jour ; de peur que ses disciples ne viennent de nuit, et ne dérobent son corps, et qu'ils ne disent au peuple : Il est ressuscité des morts. Cette dernière imposture serait pire que la première.
\VS{65}Pilate leur dit : Vous avez une garde ; allez, et faites-le garder comme vous l’entendez.
\VS{66}Ils s'en allèrent donc, et s’assurèrent du sépulcre, au moyen d’une garde, après avoir scellé la pierre.
\TextTitle{[Résurrection et apparition de Jésus-Christ]
\\(Mc. 16:1-14 ; Lu. 24:1-49 ; Jn. 20:1-23)}
\Chap{28}
\VerseOne{}Après le sabbat, à l’aube du premier jour de la semaine, Marie de Magdala, et l'autre Marie, allèrent voir le sépulcre.
\VS{2}Et voici, il eut un grand tremblement de terre ; car un ange du Seigneur descendit du ciel, vint rouler la pierre à côté de l'entrée du sépulcre, et s'assit dessus.
\VS{3}Son visage était comme un éclair, et son vêtement blanc comme de la neige.
\VS{4}Les gardes furent tellement saisis de frayeur, qu'ils devinrent comme morts.
\VS{5}Mais l'ange prit la parole, et dit aux femmes : Pour vous, ne craignez pas ; car je sais que vous cherchez Jésus, qui a été crucifié.
\VS{6}Il n'est point ici car il est ressuscité comme il l'avait dit. Venez et voyez le lieu où le Seigneur était couché,
\VS{7}et allez rapidement dire à ses disciples qu'il est ressuscité des morts. Et voici, il vous précède en Galilée ; c’est là que vous le verrez. Voici, je vous l’ai dit.
\VS{8}Alors elles sortirent rapidement du sépulcre avec crainte et une grande joie ; et coururent l'annoncer à ses disciples.
\VS{9}Mais comme elles allaient pour l'annoncer à ses disciples, voici, Jésus se présenta devant elles, et leur dit : Je vous salue. Et elles s'approchèrent, embrassèrent ses pieds, et l'adorèrent.
\VS{10}Alors Jésus leur dit : Ne craignez point. Allez, et dites à mes frères d'aller en Galilée, c’est là qu'ils me verront.
\TextTitle{[Les soldats soudoyés par les sacrificateurs]}
\VS{11}Pendant qu’elles étaient en chemin, quelques hommes de la garde entrèrent dans la ville, et ils rapportèrent aux principaux sacrificateurs tout ce qui était arrivé.
\VS{12}Sur quoi les sacrificateurs s'assemblèrent avec les anciens, et après avoir tenu conseil, donnèrent une forte somme d'argent aux soldats,
\VS{13}en disant : Dites : Ses disciples sont venus de nuit le dérober, pendant que nous dormions.
\VS{14}Et si le gouverneur l’apprend, nous l’apaiserons et nous vous tirerons de peine.
\VS{15}Les soldats prirent l'argent et suivirent les instructions qui leur furent données. Et ce bruit s’est répandu parmi les juifs, jusqu’à aujourd’hui.
\TextTitle{[Mission des apôtres]
\\(cp. Mc. 16:15-18 ; Lu. 24:46-48 ; Jn. 17:18 ; 20:21 ; Ac. 1:8 ; 1 Co. 15:6) }
\VS{16}Mais les onze disciples allèrent en Galilée, sur la montagne, où Jésus leur avait ordonné de se rendre.
\VS{17}Quand ils le virent, ils l'adorèrent, mais quelques-uns doutèrent.
\VS{18}Jésus, s’étant approché, leur parla ainsi : Tout pouvoir m’a été donné dans le ciel et sur la terre.
\VS{19}Allez donc, et faites de toutes les nations mes disciples, les baptisant au Nom du Père, du Fils et du Saint-Esprit ;
\VS{20}et enseignez-leur à observer tout ce que je vous ai ordonné. Et voici, je suis avec vous tous les jours, jusqu’à la fin du monde. Amen.
\PPE{}
\end{multicols}
