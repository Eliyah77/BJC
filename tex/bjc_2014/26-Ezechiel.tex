\ShortTitle{Ez.}\BookTitle{Ezéchiel}\BFont
\noindent\hrulefill
{\footnotesize
\textit{
\bigskip
{\centering{}
\\Auteur~: Ezéchiel
\\(Heb.~: Yechezqe'l)
\\Signification~: Dieu fortifie
\\Thème~: Jugements et gloire
\\Date de rédaction~: 6\up{ème} siècle av. J.-C.\\}
}
\textit{
\\Déporté à Babylone alors qu'il remplissait la fonction de prêtre, Ezéchiel eut la particularité d'exercer le service prophétique hors de la terre d'Israël. Sa mission était en partie d'affermir la foi des déportés par la promesse du jugement de leurs ennemis et du rétablissement de la nation. Il leur rappela aussi que les péchés de leurs pères étaient la raison de leur captivité et que l'occasion leur était donnée de réformer leurs voies.
\\Ezéchiel reçut aussi des prophéties concernant ceux qui étaient restés à Jérusalem, dont la condition n'était guère meilleure que celle des exilés et pour lesquels le pire était à venir. Ses prophéties s'exprimaient en songes et en visions, c'est donc ainsi qu'il vit la gloire de Yahweh quittant le temple de Jérusalem. Il reçut plusieurs prophéties sur les derniers temps, notamment la promesse d'un cœur nouveau en vue de la conversion, le retour de la gloire de Dieu lors du règne millénaire et le rétablissement total d'Israël.\bigskip
}
}
\par\nobreak\noindent\hrulefill
\begin{multicols}{2}
\Chap{1}
\TextTitle{Vision de la gloire de Yahweh}
\VerseOne{}Or il arriva en la trentième année, au cinquième jour du quatrième mois, comme j'étais parmi ceux qui avaient été transportés sur le fleuve de Kebar, que les cieux furent ouverts, et je vis des visions de Dieu.
\VS{2}Au cinquième jour du mois de cette année, qui fut la cinquième après que le roi Jojakin\FTNT{2 R. 24:12-16.}ait été mené en captivité,
\VS{3}la parole de Yahweh fut adressée expressément à Ezéchiel, le prêtre, fils de Buzi, dans le pays des Chaldéens\FTNT{Ezéchiel était en exil à Babylone.}, sur le fleuve de Kebar, et la main de Yahweh fut là sur lui.
\VS{4}Je regardai donc, et voici, un vent impétueux vint du nord, une grosse nuée, et un feu qui prenait de tous côtés. Il y avait autour de la nuée une splendeur, et au milieu de la nuée paraissait comme de l'airain poli, lorsqu'il sort du milieu du feu.
\VS{5}Et du milieu aussi paraissait une ressemblance de quatre animaux\FTNT{Les quatre animaux représentent quatre aspects de Jésus. La face de l'homme correspond à l'humanité du Seigneur mise en exergue dans l'évangile de Luc. La face de lion symbolise la royauté du Messie, mise en évidence dans l'évangile de Matthieu. La face de bœuf fait écho à l'évangile de Marc où le Seigneur y est présenté comme serviteur. La face d'aigle symbolise la divinité du Messie mise en évidence dans l'évangile de Jean. Le Seigneur y est présenté comme le Fils de Dieu et le Dieu véritable.} et voici leur forme: Ils avaient une ressemblance humaine.
\VS{6}Et chacun d'eux avait quatre faces, et chacun avait quatre ailes.
\VS{7}Et leurs pieds étaient des pieds droits, et la plante de leurs pieds était comme la plante d'un pied de veau, ils étincelaient comme la couleur d'un airain poli.
\VS{8}Et il y avait des mains d'homme sous leurs ailes à leurs quatre côtés~; et tous les quatre avaient leurs faces et leurs ailes.
\VS{9}Leurs ailes étaient jointes l'une à l'autre~; ils ne se tournaient point quand ils marchaient, mais chacun marchait droit devant soi.
\VS{10}Leurs faces ressemblaient à la face d'un homme, à la face d'un lion à la main droite, à la face d'un bœuf à la gauche des quatre, et à la face d'un aigle à tous les quatre.
\VS{11}Leurs faces et leurs ailes étaient divisées par le haut~; chacun avait des ailes qui se joignaient l'une à l'autre, et deux couvraient leurs corps.
\VS{12}Chacun marchait droit devant soi~; ils allaient partout où l'Esprit les poussait à aller, et ils ne se tournaient point lorsqu'ils marchaient.
\VS{13}Et quant à l'aspect des animaux, leur regard était comme des charbons de feu ardent, comme des torches~; le feu courait parmi les animaux~; et le feu avait une splendeur, et de ce feu sortait un éclair.
\VS{14}Et les animaux couraient et revenaient selon que l'éclair paraissait.
\VS{15}Je regardais les animaux, et voici, une roue apparut sur la terre auprès des animaux, devant leurs quatre faces.
\VS{16}Et l'aspect et la forme des roues étaient comme la couleur d'un chrysolithe, et toutes les quatre avaient un même aspect~; leur aspect et leur structure étaient comme si chaque roue avait été au milieu d'une autre roue.
\VS{17}En marchant, elles allaient de leurs quatre côtés, et elles ne se tournaient point quand elles marchaient.
\VS{18}Elles avaient des jantes, elles étaient si hautes qu'elles faisaient peur, et leurs jantes étaient pleines d'yeux tout autour des quatre roues.
\VS{19}Quand ils marchaient, elles marchaient auprès d'eux~; et quand ils s'élevaient au-dessus de la terre, elles aussi s'élevaient.
\VS{20}Ils allaient partout où l'Esprit les poussait à aller~; l'Esprit tendait-il là, ils y allaient, et les roues s'élevaient avec eux, car l'esprit des animaux était dans les roues.
\VS{21}Quand ils marchaient, elles marchaient~; et quand ils s'arrêtaient, elles s'arrêtaient~; et quand ils s'élevaient au-dessus de la terre, les roues aussi s'élevaient avec eux, car l'esprit des animaux était dans les roues.
\VS{22}L'aspect de ce qui était au-dessus des têtes des animaux était une étendue semblable à un cristal étincelant et terrible à voir, laquelle s'étendait au-dessus de leurs têtes.
\VS{23}Sous l'étendue, leurs ailes se tenaient droites l'une contre l'autre~; ils avaient chacun deux ailes dont ils se couvraient, chacun, dis-je, en avait deux qui couvraient leurs corps.
\VS{24}Puis j'entendis le bruit que faisaient leurs ailes quand ils marchaient, semblable au bruit des grandes eaux, et au bruit du Tout-Puissant, un bruit éclatant comme le bruit d'une armée~; quand ils s'arrêtaient, ils baissaient leurs ailes.
\VS{25}Et lorsqu'ils s'arrêtaient et laissaient tomber leurs ailes, il se faisait un bruit au-dessus de l'étendue qui était sur leurs têtes.
\VS{26}Et au-dessus de cette étendue, qui était sur leurs têtes, il y avait quelque chose de semblable à une pierre de saphir, en forme de trône~; et sur cette forme de trône apparaissait comme une figure d'homme\FTNT{Il est question ici de la manifestation du Messie.} placée dessus en hauteur.
\VS{27}Je vis encore comme de l'airain poli semblable à un feu, au dedans duquel était cet homme, et qui l'environnait. Depuis la forme de ses reins jusqu'en haut, et depuis la forme de ses reins jusqu'en bas, je vis comme du feu, et il y avait une lumière éclatante autour de lui.
\VS{28}Et la splendeur qui se voyait autour de lui était comme l'arc qui se fait dans la nuée en un jour de pluie. C'est là la vision de la représentation de la gloire de Yahweh. A sa vue, je tombai sur ma face, et j'entendis une voix qui parlait.
\Chap{2}
\TextTitle{Appel d'Ezéchiel}
\VerseOne{}Il me dit~: Fils de l'homme, tiens-toi sur tes pieds, et je parlerai avec toi.
\VS{2}Alors l'Esprit entra en moi, après qu'il m'eut parlé~; et il me releva sur mes pieds, et j'entendis celui qui me parlait.
\VS{3}Il me dit~: Fils de l'homme, je t'envoie vers les enfants d'Israël, vers des nations rebelles qui se sont rebellées contre moi. Eux et leurs pères ont péché contre moi jusqu'à ce jour même\FTNT{Jé. 3:25.}.
\VS{4}Ce sont des enfants à la face dure et au cœur obstiné, vers lesquels je t'envoie~; c'est pourquoi tu leur diras que le Seigneur Yahweh a ainsi parlé.
\VS{5}Et soit qu'ils écoutent, soit qu'ils n'en fassent rien, car ils sont une maison rebelle, ils sauront pourtant qu'il y aura eu un prophète parmi eux\FTNT{Es. 6:9-10.}.
\VS{6}Mais toi, fils de l'homme, ne les crains point, et ne crains point leurs paroles~; quoique des gens rebelles et dont les langues sont perçantes comme des épines soient avec toi, et que tu habites parmi des scorpions~; ne crains point leurs paroles, et ne t'effraie point à cause d'eux, quoiqu'ils soient une maison rebelle\FTNT{Jé. 1:8~; 1 Pi. 3:14.}.
\VS{7}Tu leur prononceras mes paroles, qu'ils écoutent ou qu'ils n'en fassent rien, car ils ne sont que rébellion.
\VS{8}Mais toi, fils de l'homme, écoute ce que je te dis, et ne sois point rebelle comme cette maison rebelle~; ouvre ta bouche et mange ce que je vais te donner\FTNT{Jé. 15:16~; Ap. 10:9.}.
\VS{9}Alors je regardai, et voici, une main fut envoyée vers moi, et voici, elle avait un livre en rouleau.
\VS{10}Et elle l'ouvrit devant moi~; et voici, il était écrit en dedans et en dehors~; des lamentations, des soupirs, et des gémissements y étaient écrits.
\Chap{3}
\TextTitle{Yahweh établit Ezéchiel comme sentinelle}
\VerseOne{}Puis il me dit~: Fils de l'homme, mange ce que tu trouveras, mange ce rouleau, et va, parle à la maison d'Israël~!
\VS{2}J'ouvris donc ma bouche, et il me fit manger ce rouleau.
\VS{3}Il me dit~: Fils de l'homme, nourris ton ventre et remplis tes entrailles de ce rouleau que je te donne~! Je le mangeai, et il fut doux dans ma bouche comme du miel\FTNT{Ps. 119:103.}.
\VS{4}Puis il me dit~: Fils de l'homme, lève-toi et va vers la maison d'Israël, et prononce-leur mes paroles~!
\VS{5}Car tu n'es point envoyé vers un peuple au langage inconnu, ou à la langue barbare~; c'est vers la maison d'Israël~;
\VS{6}ni vers plusieurs peuples ayant un langage inconnu ou une langue barbare, dont tu ne puisses comprendre les paroles. Si je t'envoyais vers eux, ils t'écouteraient.
\VS{7}Mais la maison d'Israël ne voudra pas t'écouter, parce qu'ils ne veulent point m'écouter~; car toute la maison d'Israël a le front dur et le cœur obstiné.
\VS{8}Voici, j'endurcirai ta face contre leurs faces, et j'endurcirai ton front contre leurs fronts\FTNT{Jé. 1:18~; Mi. 3:8.}.
\VS{9}Et j'ai rendu ton front semblable à un diamant, plus dur que le roc. Ne les crains donc point, et ne t'effraie point à cause d'eux, quoiqu'ils soient une maison rebelle.
\VS{10}Puis il me dit~: Fils de l'homme, reçois dans ton cœur et écoute de tes oreilles toutes les paroles que je te dirai.
\VS{11}Lève-toi donc, va vers ceux qui ont été emmenés captifs, vers les enfants de ton peuple, parle-leur et dis-leur que le Seigneur Yahweh a ainsi parlé~; soit qu'ils écoutent ou qu'ils n'en fassent rien.
\VS{12}Puis l'Esprit m'enleva, et j'entendis derrière moi le bruit d'un grand tremblement, disant~: Bénie soit la gloire de Yahweh du lieu de sa demeure~!
\VS{13}Et j'entendis le bruit des ailes des animaux, qui s'entre-touchaient les unes les autres, et le bruit des roues auprès d'eux, et le bruit d'un grand tremblement.
\VS{14}L'Esprit donc m'enleva, et me prit, et j'allai, l'esprit rempli d'amertume et de colère, mais la main de Yahweh me fortifia.
\VS{15}Je vins donc vers ceux qui avaient été transportés à Thel-Abib, vers ceux qui demeuraient auprès du fleuve de Kebar~; et je me tins là où ils se tenaient, même je me tins là parmi eux sept jours, tout étonné.
\VS{16}Et au bout de sept jours, la parole de Yahweh me fut adressée, en disant~:
\VS{17}Fils de l'homme, je t'établis pour être sentinelle sur la maison d'Israël~; tu écouteras donc la parole de ma bouche, et tu les avertiras de ma part\FTNT{Es. 52:8~; Es. 62:6~; Jé. 6:17.}.
\VS{18}Quand je dirai au méchant~: Tu mourras, tu mourras~! Si tu ne l'avertis pas, et si tu ne parles pas pour l'avertir de se détourner de ses mauvaises voies, afin de lui sauver la vie~; ce méchant-là mourra dans son iniquité, mais je redemanderai son sang de ta main.
\VS{19}Et si tu avertis le méchant, et qu'il ne se détourne pas de sa méchanceté ni de ses mauvaises voies, il mourra dans son iniquité, mais toi, tu sauveras ton âme\FTNT{Ez. 18:23-24~; Ez. 33:6.}.
\VS{20}Pareillement, si le juste se détourne de sa justice et commet l'iniquité, lorsque j'aurais mis quelque obstacle devant lui, il mourra~; parce que tu ne l'auras point averti, il mourra dans son péché, et il ne sera point fait mention de ses justices qu'il aura faites~; mais je te redemanderai son sang de ta main.
\VS{21}Et si tu avertis le juste de ne point pécher, et qu'il ne pèche point, il vivra~; il vivra parce qu'il aura été averti~; et toi pareillement, tu sauveras ton âme.
\VS{22}Et la main de Yahweh fut sur moi, et il me dit~: Lève-toi, et sors vers la vallée, et là je te parlerai.
\VS{23}Je me levai donc, et sortis dans la vallée~; voici, la gloire de Yahweh se tenait là, telle que je l'avais vue près du fleuve de Kebar, et je tombai sur ma face.
\VS{24}Alors l'Esprit entra en moi et me releva sur mes pieds~; il me parla et me dit~: Entre, et enferme-toi dans ta maison.
\VS{25}Fils de l'homme, voici, on mettra des cordes sur toi, on te liera, et tu ne sortiras point pour aller parmi eux.
\VS{26} Et j'attacherai ta langue à ton palais, tu seras muet, et tu ne les reprendras point~; parce qu'ils sont une maison rebelle\FTNT{Jn. 1:20-22.}.
\VS{27}Mais quand je te parlerai, j'ouvrirai ta bouche, et tu leur diras~: Ainsi parle le Seigneur Yahweh~: Que celui qui écoute, écoute~; et que celui qui n'écoute pas, n'écoute pas~; car ils sont une maison rebelle.
\Chap{4}
\TextTitle{Signes annonciateurs du jugement de Jérusalem}
\VerseOne{}Toi, fils de l'homme, prends une brique et place-la devant toi, et traces-y la ville de Jérusalem.
\VS{2}Puis tu mettras le siège contre elle, tu bâtiras contre elle des retranchements, tu élèveras contre elle des terrasses, tu mettras des camps contre elle, et tu mettras autour d'elle des béliers pour la battre\FTNT{2 R. 25:1.}.
\VS{3}Tu prendras aussi une plaque de fer, et tu la mettras comme un mur de fer entre toi et la ville~; tu dresseras ta face contre elle, elle sera assiégée, et tu l'assiégeras~; ce sera un signe pour la maison d'Israël.
\VS{4}Après, tu dormiras sur ton côté gauche, mets-y l'iniquité de la maison d'Israël, et tu porteras leur iniquité autant de jours que tu seras couché sur ce côté.
\VS{5}Et je t'ai assigné un nombre de jours égal à celui des années de leur iniquité~: Trois cent quatre-vingt-dix jours~; ainsi tu porteras l'iniquité de la maison d'Israël.
\VS{6}Et quand tu auras accompli ces jours-là, tu dormiras une seconde fois sur ton côté droit, et tu porteras l'iniquité de la maison de Juda pendant quarante jours~; un jour pour chaque année, car je t'ai assigné un jour pour chaque année.
\VS{7}Tu tourneras ta face et ton bras nu vers Jérusalem assiégée, et tu prophétiseras contre elle.
\VS{8}Et voici, j'ai mis sur toi des cordes, afin que tu ne puisses pas te tourner d'un côté sur l'autre, jusqu'à ce que tu aies accompli les jours de ton siège.
\TextTitle{Le pain impur}
\VS{9}Tu prendras aussi du froment, de l'orge, des fèves, des lentilles, du millet, et de l'épeautre~; tu les mettras dans un vase, et tu en feras du pain autant de jours que tu seras couché sur ton côté~; tu en mangeras pendant trois cent quatre-vingt-dix jours.
\VS{10}La viande que tu mangeras sera du poids de vingt sicles par jour~; et tu la mangeras de temps à autre.
\VS{11}Et tu boiras de l'eau par mesure~; à savoir la sixième partie d'un hin~; tu la boiras de temps à autre.
\VS{12}Et tu mangeras aussi des gâteaux d'orge, que tu feras cuire avec des excréments humains en leur présence.
\VS{13}Puis Yahweh dit~: Les enfants d'Israël mangeront ainsi leur pain souillé, parmi les nations vers lesquelles je les chasserai\FTNT{Os. 9:3~; Da. 1:8.}.
\VS{14}Et je dis~: Ah~! Seigneur Yahweh, voici, mon âme n'a point été souillée, et je n'ai mangé d'aucune bête morte d'elle-même, ou déchirée par les bêtes sauvages, depuis ma jeunesse jusqu'à présent, et aucune chair impure n'est entrée dans ma bouche\FTNT{Lé. 17:15~; De. 14:3~; Ac. 10:14.}.
\VS{15}Il me répondit~: Voici, je te donne des excréments de bœuf au lieu d'excréments humains, et tu y feras cuire ton pain.
\VS{16}Puis il me dit~: Fils de l'homme, voici, je m'en vais rompre le bâton du pain dans Jérusalem~; et ils mangeront leur pain au poids et avec chagrin~; et ils boiront de l'eau par mesure et avec horreur\FTNT{Lé. 26:26~; Es. 3:1~; Ps. 105:16~; La. 5:4.}.
\VS{17}Parce que le pain et l'eau leur manqueront, ils seront épouvantés, se regardant les uns les autres, et ils se décomposeront à cause de leur iniquité.
\Chap{5}
\TextTitle{Les cheveux coupés et divisés en trois}
\VerseOne{}Et toi, fils de l'homme, prends un couteau tranchant, prends un rasoir de barbier, et fais-le passer sur ta tête et sur ta barbe. Puis, tu prendras une balance à peser, et tu partageras ce que tu auras rasé\FTNT{Lé. 21:5~; Ez. 44:20.}.
\VS{2}Brûles-en un tiers dans le feu, au milieu de la ville, lorsque les jours du siège seront accomplis~; prends-en un tiers, et frappe-le avec l'épée tout autour de la ville~; disperses-en un tiers au vent, car je tirerai l'épée derrière eux\FTNT{Lé. 26:25~; La. 1:20.}.
\VS{3}Tu en prendras une petite quantité, que tu serreras aux pans de ton manteau.
\VS{4}De ceux-là, tu en prendras encore, les jetteras au milieu du feu, et les brûleras au feu. De là sortira un feu contre toute la maison d'Israël.
\VS{5}Ainsi parle le Seigneur Yahweh~: C'est là cette Jérusalem que j'avais placée au milieu des nations et des pays qui sont autour d'elle.
\VS{6}Elle a changé mes ordonnances et s'est rendue plus coupable que les nations et les pays d'alentour~; car ils ont rejeté mes ordonnances, et n'ont point marché dans mes ordonnances.
\VS{7}C'est pourquoi ainsi parle le Seigneur Yahweh~: Parce que vous avez multiplié vos méchancetés plus que les nations qui vous entourent, et que vous n'avez point suivi mes ordonnances et observé mes lois, et que vous n'avez pas même agi selon les ordonnances des nations qui vous entourent.
\VS{8}A cause de cela, ainsi parle le Seigneur Yahweh~: Voici, j'en veux à toi et j'exécuterai au milieu de toi mes jugements, sous les yeux des nations.
\VS{9}Je te ferai, à cause de toutes tes abominations, des choses que je n'ai jamais faites, et ce que je ne ferai jamais plus\FTNT{Da. 9:12~; Mt. 24:21.}.
\VS{10}Des pères mangeront leurs fils au milieu de toi, et des fils mangeront leurs pères~; j'exécuterai mes jugements sur toi, et je disperserai à tous les vents tout ce qui restera de toi\FTNT{Lé. 26:33~; De. 28:64~; Jé. 9:16~; Za. 2:6.}.
\VS{11}Je suis vivant, dit le Seigneur Yahweh, parce que tu as souillé mon lieu saint par toutes tes infamies, et par toutes tes abominations, moi-même je te raserai, et mon œil ne t'épargnera point, et je n'aurai point de compassion\FTNT{Jé. 7:9-11.}.
\VS{12}Un tiers d'entre vous mourra de la peste, et sera consumé par la famine au milieu de toi~; un tiers tombera par l'épée autour de toi~; et je disperserai à tous les vents l'autre tiers, je tirerai l'épée derrière eux.
\VS{13}Car ma colère sera portée à son comble, je ferai reposer ma fureur sur eux, et je me donnerai satisfaction~; ils sauront que moi, Yahweh, j'ai parlé dans ma jalousie, quand j'aurai consumé ma fureur sur eux.
\VS{14}Je ferai de toi un désert, un sujet d'opprobre parmi les nations qui sont autour de toi, aux yeux de tous les passants\FTNT{Lé. 26:31-32~; Né. 2:17.}.
\VS{15}Tu seras en opprobre, en ignominie, un exemple et un sujet d'étonnement pour les nations qui t'entourent, quand j'aurai exécuté mes jugements sur toi, avec colère, avec fureur, et par des châtiments pleins de fureur~; moi, Yahweh, j'ai parlé\FTNT{De. 28:37~; 1 R. 9:7~; Es. 26:9~; Jé. 24:9~; Ps. 79:4.}.
\VS{16}Quand je lancerai sur eux les flèches douloureuses de la famine, qui seront mortelles, quand je les enverrai pour vous détruire, j'ajouterai la famine sur vous, et romprai pour vous le bâton du pain\FTNT{De. 32:24.}.
\VS{17}Je vous enverrai la famine, et des bêtes féroces, qui te priveront d'enfants~; la peste et le sang passeront au milieu de toi, et je ferai venir l'épée sur toi. Moi,Yahweh, j'ai parlé.
\Chap{6}
\TextTitle{Grâce de Yahweh pour quelques réchappés d'Israël}
\VerseOne{}La parole de Yahweh vint encore à moi en disant~:
\VS{2}Fils de l'homme, tourne ta face contre les montagnes d'Israël, et prophétise contre elles~!
\VS{3}Et dis~: Montagnes d'Israël, écoutez la parole du Seigneur Yahweh. Ainsi parle le Seigneur Yahweh aux montagnes et aux collines, aux cours des rivières, et aux vallées~: Me voici, je vais faire venir l'épée sur vous, et je détruirai vos hauts lieux\FTNT{Lé. 26:30.}.
\VS{4}Vos autels seront dévastés, vos autels d'encens seront brisés, et je ferai tomber vos morts devant vos idoles.
\VS{5}Car je mettrai les cadavres des enfants d'Israël devant leurs idoles, et je disperserai vos os autour de vos autels\FTNT{2 R. 23:14-20.}.
\VS{6}Les villes seront désertes, là où sont vos demeures, et les hauts lieux seront dévastés, vos autels seront délaissés et abandonnés, et vos idoles seront brisées et ne seront plus~; vos autels d'encens seront abattus, et vos ouvrages seront nettoyés.
\VS{7}Les tués tomberont parmi vous~; et vous saurez que je suis Yahweh.
\VS{8}Mais je laisserai quelques restes d'entre vous, afin que vous ayez quelques réchappés de l'épée parmi les nations, quand vous serez dispersés parmi les pays.
\VS{9}Vos réchappés se souviendront de moi\FTNT{Jé. 51:50.} parmi les nations où ils seront captifs, parce que j'aurai brisé leur cœur adonné à la fornication, qui s'est détourné de moi, et à cause de leurs yeux qui se sont livrés à la prostitution après leurs idoles~; ils se prendront eux-mêmes en dégoût, à cause du mal qu'ils ont commis, à cause de leurs abominations.
\VS{10}Ils sauront que je suis Yahweh, que ce n'est point en vain que je les ai menacés.
\TextTitle{Sentence envers les idolâtres}
\VS{11}Ainsi parle le Seigneur Yahweh~: Frappe de ta main et bats de ton pied, et dis~: Hélas~! A cause de toutes les abominations, des maux de la maison d'Israël~; car ils tomberont par l'épée, par la famine, et par la peste.
\VS{12}Celui qui sera loin mourra de la peste, et celui qui sera près tombera par l'épée~; et celui qui restera et sera assiégé, mourra par la famine, ainsi je consumerai ma fureur sur eux\FTNT{Am. 4:10.}.
\VS{13}Vous saurez que je suis Yahweh quand les blessés à mort seront au milieu de leurs idoles, autour de leurs autels, sur toute colline élevée, sur tous les sommets des montagnes, sous tout arbre vert, et sous tout chêne touffu, là où ils offraient des parfums de bonne odeur à toutes leurs idoles\FTNT{Os. 4:13.}.
\VS{14}J'étendrai donc ma main sur eux, et je rendrai leur pays désolé et désert dans toutes leurs demeures, plus que le désert qui est vers Dibla. Et ils sauront que je suis Yahweh.
\Chap{7}
\TextTitle{Attaque babylonienne imminente}
\VerseOne{}Puis la parole de Yahweh vint à moi en disant~:
\VS{2}Et toi, fils de l'homme, écoute~: Ainsi parle le Seigneur Yahweh à la terre d'Israël~: La fin, la fin vient sur les quatre coins de la terre~!
\VS{3}Maintenant la fin vient sur toi, j'enverrai sur toi ma colère, je te jugerai selon ta voie, et je mettrai sur toi toutes tes abominations\FTNT{Ro. 2:6.}.
\VS{4}Et mon œil ne t'épargnera point, et je n'aurai point de compassion~; mais je te chargerai de tes voies, et tes abominations seront au milieu de toi~; et vous saurez que je suis Yahweh.
\VS{5}Ainsi parle le Seigneur Yahweh~: Voici un mal, un seul mal qui vient~!
\VS{6}La fin vient, la fin vient, elle se réveille contre toi~; voici, le mal vient~!
\VS{7}Le matin vient sur toi, qui demeures au pays~; le temps vient, le jour est près de toi, il ne sera que frayeur, et non pas une invitation des montagnes\FTNT{So. 1:14-15.} à s'entre-réjouir.
\VS{8}Maintenant, je répandrai bientôt ma fureur sur toi, et je consumerai ma colère sur toi~; je te jugerai selon ta voie, je mettrai sur toi toutes tes abominations.
\VS{9}Mon œil ne t'épargnera point, et je n'aurai point de compassion, je te punirai selon ta voie, et tes abominations seront au milieu de toi~; et vous saurez que je suis Yahweh qui frappe.
\VS{10}Voici le jour, voici il vient, le matin paraît, la verge fleurit, l'orgueil bourgeonne.
\VS{11}La violence s'élève pour servir de verge à la méchanceté~; il ne restera rien d'eux, ni de leur multitude, ni de leur tumulte, et on ne se lamentera point sur eux.
\VS{12}Le temps vient, le jour est tout proche~: Que celui donc qui achète ne se réjouisse point, et que celui qui vend ne se lamente point~; car il y a une ardente colère sur toute leur multitude.
\VS{13}Car le vendeur ne retournera point à ce qu'il aura vendu, quand ils seraient encore en vie~; parce que la vision concernant toute la multitude de son pays ne sera point révoquée, et chacun portera la peine de son iniquité, tant qu'il vivra~; ils ne reprendront jamais courage.
\VS{14}On sonne de la trompette, tout est prêt, mais il n'y a personne pour aller au combat, parce que l'ardeur de ma colère est sur toute leur multitude.
\VS{15}L'épée est au-dehors, la peste et la famine au-dedans~! Celui qui est aux champs mourra par l'épée~; et celui qui est dans la ville, la famine et la peste le dévoreront.
\VS{16}Les réchappés s'enfuiront et seront sur les montagnes comme les pigeons des vallées, tous gémissant, chacun sur son iniquité.
\VS{17}Toutes les mains deviendront lâches, et tous les genoux se fondront en eau\FTNT{Es. 13:7~; Jé. 6:24.}.
\VS{18}Ils se ceindront de sacs, et le tremblement les couvrira, la confusion sera sur tous leurs visages, et leurs têtes deviendront chauves\FTNT{Es. 3:24~; Jé. 48:37~; Am. 8:10.}.
\VS{19}Ils jetteront leur argent par les rues, et leur or s'en ira au loin~; leur argent et leur or ne pourront pas les délivrer au jour de la grande colère de Yahweh\FTNT{Pr. 11:4~; So. 1:18.}~; ils ne rassasieront point leurs âmes, et ne rempliront point leurs entrailles, parce que leur iniquité aura été leur ruine.
\TextTitle{Violation du temple}
\VS{20}II avait mis entre eux la noblesse de son magnifique ornement~; mais ils y ont placé des images de leurs abominations, et de leurs infamies~; c'est pourquoi j'en ai fait pour eux une chose impure.
\VS{21}Je l'ai livrée au pillage dans la main des étrangers, et en proie aux méchants de la terre qui la profaneront\FTNT{Jé. 20:5.}.
\VS{22}Je détournerai aussi ma face d'eux, et on violera mon lieu secret, et des furieux entreront et le profaneront.
\VS{23}Fais une chaîne~! Car le pays est plein de crimes, de meurtres, et la ville est pleine de violence.
\VS{24}C'est pourquoi je ferai venir les plus méchants des nations, qui possèderont leurs maisons, et je ferai cesser l'orgueil des puissants, et leurs saints lieux seront profanés.
\VS{25}La destruction vient, et ils chercheront la paix, mais il n'y en aura point.
\VS{26}Il viendra malheur sur malheur, et il y aura rumeur sur rumeur~; ils demanderont la vision aux prophètes\FTNT{La. 2:9.}~; la loi périra chez le prêtre, et le conseil chez les anciens.
\VS{27}Le roi se lamentera, les princes se vêtiront de désolation, et les mains du peuple du pays tomberont de frayeur. Je les traiterai selon leur voie, je les jugerai comme ils le méritent, et ils sauront que je suis Yahweh.
\Chap{8}
\TextTitle{Visions divines}
\VerseOne{}Puis il arriva dans la sixième année, au cinquième jour du sixième mois, comme j'étais assis dans ma maison, et que les anciens de Juda étaient assis devant moi, que la main du Seigneur Yahweh tomba là sur moi.
\VS{2}Je regardai, et voici c'était une figure ayant l'aspect d'un feu qui frappe les regards~; depuis ses reins jusqu'en bas, c'était du feu, et depuis ses reins jusqu'en haut, c'était d'un aspect brillant comme de l'airain poli.
\VS{3}Il étendit une forme de main et me prit par les cheveux de ma tête. L'Esprit m'enleva entre la terre et le ciel, et me transporta à Jérusalem, dans des visions de Dieu, à l'entrée de la porte intérieure, du côté nord, où était posée l'idole de jalousie\FTNT{L'idole de la jalousie~: Dans le temple de Jérusalem à l'époque d'Ezéchiel, l'idolâtrie s'y développait sans retenue (2 R. 21-23). Il y avait dans ce temple les idoles d'Astarté et les autels de Baal. Le temple était souillé.} qui provoque la jalousie.
\VS{4}Voici, la gloire du Dieu d'Israël était là, telle que je l'avais vue en vision dans la vallée.
\TextTitle{Abominations dans le temple}
\VS{5}Il me dit~: Fils de l'homme, lève maintenant tes yeux vers le chemin qui tend vers le nord~! J'élevai mes yeux vers le chemin qui tend vers le nord, et voici, du côté nord, à la porte de l'autel, était cette idole de jalousie, à l'entrée.
\VS{6}Il me dit~: Fils de l'homme, ne vois-tu pas ce qu'ils font, les grandes abominations que la maison d'Israël commet ici, pour que je me retire de mon lieu saint~? Mais tourne-toi encore, tu verras de grandes abominations.
\VS{7}Il me conduisit donc à l'entrée du parvis. Je regardai, et voici, il y avait un trou dans le mur.
\VS{8}Il me dit~: Fils de l'homme, perce maintenant le mur~; et quand je perçai le mur, il y avait une porte.
\VS{9}Puis il me dit~: Entre et regarde les méchantes abominations qu'ils commettent ici.
\VS{10}J'entrai donc et je regardai~; et voici, toutes sortes de figures de reptiles et de bêtes abominables, et toutes les idoles de la maison d'Israël étaient peintes sur le mur tout autour\FTNT{Ex. 20:4~; De. 4:16-18~; Ro. 1:23.}.
\VS{11}Soixante-dix hommes des anciens de la maison d'Israël, au milieu desquels était Jaazania, fils de Schaphan, se tenaient debout devant ces idoles, chacun l'encensoir à la main, d'où s'élevait une épaisse nuée d'encens.
\VS{12}Alors il me dit~: Fils de l'homme, n'as-tu pas vu ce que les anciens de la maison d'Israël font dans les ténèbres, chacun dans sa chambre pleine de figures~? Car ils disent~: Yahweh ne nous voit point, Yahweh a abandonné le pays\FTNT{Es. 29:15.}.
\VS{13}Puis il me dit~: Tourne-toi encore, et tu verras les grandes abominations qu'ils commettent.
\VS{14}Il me conduisit donc à l'entrée de la porte de la maison de Yahweh, qui est vers le nord. Et voici, il y avait là des femmes assises qui pleuraient Thammuz\FTNT{Thammuz ou Adonis.}.
\VS{15}Il me dit~: Fils de l'homme, n'as-tu pas vu~? Tourne-toi encore, et tu verras des abominations plus grandes que celles-ci.
\VS{16}Il me fit donc entrer dans le parvis intérieur de la maison de Yahweh. Et voici, à l'entrée du temple de Yahweh, entre le portique et l'autel, environ vingt-cinq hommes avaient le dos tourné contre le temple de Yahweh, leurs visages tournés vers l'orient~; et ils se prosternaient vers l'orient, devant le soleil\FTNT{De. 4:19.}.
\VS{17}Alors il me dit~: Fils de l'homme, n'as-tu pas vu~? Est-ce une chose légère à la maison de Juda de commettre ces abominations qu'ils commettent ici~? Car ils ont rempli le pays de violence, et ils se sont ainsi tournés pour m'irriter~; mais voici, ils mettent une écharde à leur nez.
\VS{18}Et moi, j'agirai dans ma fureur~; mon œil ne les épargnera point, et je n'aurai point de compassion~; quand ils crieront à haute voix à mes oreilles, je ne les exaucerai point\FTNT{Es. 1:15~; Jé. 11:11~; Pr. 1:28~; Mi. 3:4~; Za. 7:13.}.
\Chap{9}
\TextTitle{Marque de Yahweh sur les justes~; extermination des impies}
\VerseOne{}Puis il cria d'une voix forte à mes oreilles~: Faites approcher ceux qui châtient la ville, chacun avec son instrument de destruction à la main~!
\VS{2}Et voici, six hommes venaient par le chemin de la haute porte qui regarde vers le nord, et chacun avait dans sa main son instrument de destruction. Il y avait au milieu d'eux un homme vêtu de lin, avec un encrier d'écrivain à ses reins. Ils entrèrent et se tinrent près de l'autel d'airain.
\VS{3}Alors la gloire du Dieu d'Israël s'éleva du chérubin sur lequel elle était, et vint sur le seuil de la maison. Il cria à l'homme qui était vêtu de lin et qui avait l'écritoire sur ses reins.
\VS{4}Yahweh lui dit~: Passe par le milieu de la ville, par le milieu de Jérusalem, et marque la lettre Tav sur les fronts des hommes qui gémissent et qui soupirent à cause de toutes les abominations qui s'y commettent\FTNT{Ex. 12:7-23~; Ap. 7:3~; Ap. 9:4~; Ap. 13:16-17~; Ap. 20:4.}.
\VS{5}Et s'adressant aux autres en ma présence, il dit~: Passez dans la ville après lui, et frappez~; que votre œil soit sans pitié et n'ayez point de compassion~!
\VS{6}Tuez-les tous, les vieillards, les jeunes gens, les vierges, les enfants et les femmes\FTNT{2 Ch. 36:17.}~; mais n'approchez pas de ceux qui ont la lettre Tav\FTNT{La lettre Tav symbolise le sceau de Dieu. Selon la Bible, la marque des chrétiens est représentée par~: Le Saint-Esprit, le nom de Jésus-Christ (Ep. 1:13-14~; Ep. 4:30~; Ap. 14:1), le nom de la Nouvelle Jérusalem (Ap. 3:12) et le nom du Père. Les chrétiens fidèles à Dieu sont marqués par l'Esprit de Dieu qui est notre sceau.}, et commencez par mon lieu saint\FTNT{Le jugement commencera par la maison de Dieu (1 Pi. 4:17-18).}. Ils commencèrent donc par les vieillards qui étaient devant la maison.
\VS{7}Il leur dit~: Profanez la maison, et remplissez de morts les parvis~! Sortez~! Et ils sortirent, et ils frappèrent dans la ville.
\VS{8}Or il arriva que comme ils frappaient, je restai là, et m'étant prosterné le visage contre terre, je criai et dis~: Ah~! Seigneur Yahweh~! Vas-tu donc détruire tous les restes d'Israël en répandant ta fureur sur Jérusalem~?
\VS{9}Il me dit~: L'iniquité de la maison d'Israël et de Juda est excessivement grande, le pays est rempli de meurtres et la ville remplie de crimes~; car ils ont dit~: Yahweh a abandonné le pays, Yahweh ne nous voit point.
\VS{10}Quant à moi, mon œil aussi ne les épargnera point, et je n'en aurai point compassion~; je mettrai leur voie sur leur tête.
\VS{11}Et voici, l'homme vêtu de lin, qui avait un encrier d'écrivain à ses reins, rapporta ce qui avait été fait, et il dit~: J'ai fait comme tu m'as ordonné.
\Chap{10}
\TextTitle{La gloire de Yahweh quitte le temple}
\VerseOne{}Je regardai, et voici, sur l'étendue qui était au-dessus de la tête des chérubins, parut comme une pierre de saphir~; on voyait au-dessus d'eux quelque chose de semblable à un trône.
\VS{2}On parla à l'homme vêtu de lin, et on lui dit~: Va entre les roues, sous les chérubins, et remplis tes mains de charbons ardents que tu prendras entre les chérubins, et répands-les sur la ville\FTNT{Es. 6:6~; Ap. 8:5.}~! Il y entra devant mes yeux.
\VS{3}Les chérubins étaient à la droite de la maison quand l'homme entra, et une nuée remplit le parvis intérieur\FTNT{1 R. 8:10-11.}.
\VS{4}Puis la gloire de Yahweh s'éleva de dessus les chérubins pour venir sur le seuil de la maison, la maison fut remplie d'une nuée, et le parvis fut rempli de la splendeur de la gloire de Yahweh.
\VS{5}On entendit le bruit des ailes des chérubins jusqu'au parvis extérieur, pareil à la voix du Dieu Tout-Puissant lorsqu'il parle.
\VS{6}Ainsi Yahweh donna cet ordre à l'homme qui était vêtu de lin~: Prends du feu d'entre les roues des chérubins~; il entra et se tint auprès des roues.
\VS{7}L'un des chérubins étendit sa main entre les chérubins, vers le feu qui était entre les chérubins~; il en prit, et le mit entre les mains de l'homme vêtu de lin. Et cet homme le prit et sortit.
\VS{8}On voyait aux chérubins une forme de main d'homme sous leurs ailes.
\VS{9}Puis je regardai, et voici, il y avait quatre roues près des chérubins, une roue près de chaque chérubin~; et ces roues avaient l'aspect d'une pierre de chrysolithe.
\VS{10}A leur aspect, toutes les quatre avaient la même forme~; chaque roue paraissait être au milieu d'une autre roue.
\VS{11}Quand elles marchaient, elles allaient de leurs quatre côtés, et elles ne se tournaient point dans leur marche~; mais elles allaient dans la direction de la tête, sans se tourner dans leur marche.
\VS{12}Tout le corps des chérubins, leur dos, leurs mains, leurs ailes, étaient remplis d'yeux, aussi bien que les roues tout autour, les quatre roues\FTNT{Ap. 4:6-8.}.
\VS{13}J'entendis qu'on appela les roues tourbillon.
\VS{14}Chaque animal avait quatre faces~: La première face était la face d'un chérubin~; la seconde face était la face d'un homme~; la troisième était la face d'un lion~; et la quatrième la face d'un aigle\FTNT{Ez. 1~; Ap. 4:7.}.
\VS{15}Puis les chérubins s'élevèrent. Ce sont là les animaux que j'avais vus près du fleuve de Kebar.
\VS{16}Lorsque les chérubins marchaient, les roues aussi marchaient à côté d'eux~; et quand les chérubins élevaient leurs ailes pour s'élever de terre, les roues ne se détournaient point d'eux.
\VS{17}Lorsqu'ils s'arrêtaient, elles s'arrêtaient~; et lorsqu'ils s'élevaient, elles s'élevaient~; car l'esprit des animaux était dans les roues.
\VS{18}Puis la gloire de Yahweh se retira de dessus le seuil de la maison, et se tint au-dessus des chérubins.
\VS{19}Les chérubins élevant leurs ailes, s'élevèrent de terre sous mes yeux quand ils partirent~; les roues s'élevèrent aussi. Et chacun d'eux s'arrêta à l'entrée de la porte orientale de la maison de Yahweh~; la gloire du Dieu d'Israël était sur eux en haut.
\VS{20}C'étaient les animaux que j'avais vus sous le Dieu d'Israël près du fleuve de Kebar, et je reconnus que c'étaient des chérubins.
\VS{21}Chacun avait quatre faces, et chacun quatre ailes, une forme de main d'homme était sous leurs ailes.
\VS{22}Quant à l'aspect de leurs faces, c'étaient les faces que j'avais vues près du fleuve de Kebar, c'était le même aspect, c'étaient eux-mêmes. Et chacun marchait droit devant soi.
\Chap{11}
\TextTitle{Sentences sur les princes infidèles}
\VerseOne{}Puis l'Esprit m'enleva et me transporta à la porte orientale de la maison de Yahweh, à celle qui regarde vers l'orient. Et il y avait vingt-cinq hommes à l'entrée de la porte, et je vis au milieu d'eux Jaazania, fils d'Azzur, et Pelathia, fils de Benaja, les princes du peuple.
\VS{2}Il me dit~: Fils de l'homme, ce sont les hommes qui ont des pensées d'iniquité, et qui donnent un mauvais conseil dans cette ville\FTNT{Mi. 2:1.}.
\VS{3}Ils disent~: Ce n'est pas le moment~! Bâtissons des maisons~! La ville est la chaudière et nous sommes la viande.
\VS{4}C'est pourquoi prophétise contre eux, prophétise, fils de l'homme~!
\VS{5}L'Esprit de Yahweh tomba sur moi. Et il me dit~: Ainsi parle Yahweh~: Vous parlez de la sorte, maison d'Israël, et je connais toutes les pensées de votre esprit.
\VS{6}Vous avez multiplié les meurtres dans cette ville, et vous avez rempli ses rues de gens que vous avez tués.
\VS{7}C'est pourquoi, ainsi parle le Seigneur Yahweh~: Les gens que vous avez tués, et que vous avez mis au milieu d'elle, sont la viande, et elle est la chaudière, mais je vous tirerai hors du milieu d'elle\FTNT{Mi. 3:3.}.
\VS{8}Vous avez eu peur de l'épée, mais je ferai venir l'épée sur vous, dit le Seigneur Yahweh\FTNT{Jé. 42:16.}.
\VS{9}Je vous tirerai hors de la ville, je vous livrerai entre les mains des étrangers, et j'exécuterai mes jugements contre vous.
\VS{10}Vous tomberez par l'épée~; je vous jugerai dans le pays d'Israël, et vous saurez que je suis Yahweh.
\VS{11}Elle ne sera point une chaudière pour vous, et vous ne serez point au dedans d'elle comme la viande~; je vous jugerai dans le pays d'Israël.
\VS{12}Et vous saurez que je suis Yahweh~; car vous n'avez point suivi mes ordonnances, et vous n'avez pas observé mes lois, mais vous avez agi selon les ordonnances des nations qui sont autour de vous.
\VS{13}Or il arriva comme je prophétisais, que Pelathia, fils de Benaja, mourut. Alors je me prosternai sur mon visage, je criai à haute voix, et dis~: Ah~! Seigneur Yahweh~! Vas-tu consumer entièrement le reste d'Israël~?
\TextTitle{Restauration d'Israël et de ses exilés}
\VS{14}La parole de Yahweh me fut adressée, en disant~:
\VS{15}Fils de l'homme, tes frères, tes frères, les hommes de ta parenté, et la maison d'Israël tout entière, à qui les habitants de Jérusalem ont dit~: Eloignez-vous de Yahweh, la terre nous a été donnée en héritage.
\VS{16}C'est pourquoi dis-leur~: Ainsi parle le Seigneur Yahweh~: Quoique je les aie éloignés des nations, et que je les aie dispersés dans divers pays, je serai pour eux quelque temps un lieu saint\FTNT{Alors que le lieu saint ou la maison terrestre était souillée, Yahweh s'est présenté lui-même comme le Lieu Sacré pour son peuple.} dans les pays où ils sont venus.
\VS{17}C'est pourquoi dis-leur~: Ainsi parle le Seigneur Yahweh~: Je vous rassemblerai du milieu des peuples, et je vous recueillerai des pays auxquels vous avez été dispersés, et je vous donnerai la terre d'Israël\FTNT{Es. 11:11-16~; Jé. 24:6~; Ez. 28:25~; Ez. 34:13~; Ez. 36:24.}.
\VS{18}C'est là qu'ils iront, et ils ôteront hors d'elle toutes ses infamies et toutes ses abominations.
\VS{19}Je leur donnerai un même cœur, et je mettrai en eux un esprit nouveau~; j'ôterai de leur corps le cœur de pierre, et je leur donnerai un cœur de chair\FTNT{Il s'agit d'une allusion à la Nouvelle Alliance (Jé. 31:31-34~; Hé. 8).},
\VS{20}afin qu'ils suivent mes ordonnances, et qu'ils gardent et observent mes lois~; ils seront mon peuple, et je serai leur Dieu.
\VS{21}Quant à ceux dont le cœur se plaît à leurs idoles et à leurs abominations, quant à ceux-là, je ferai tomber sur leur tête les peines que mérite leur conduite, dit le Seigneur Yahweh.
\TextTitle{La gloire de Dieu en mouvement vers le Mont des Oliviers\FTNTT{Ez. 43:1-4.}}
\VS{22}Puis les chérubins élevèrent leurs ailes, accompagnés des roues~; et la gloire du Dieu d'Israël était sur eux, en haut.
\VS{23}La gloire de Yahweh s'éleva du milieu de la ville\FTNT{Le départ de la gloire de Dieu du temple de Jérusalem marque la fin de la théocratie (règne de Dieu) en Israël. Cet événement, comparable au retrait de l'Esprit de Dieu en Ge. 6:3, fut consécutif à la décadence morale d'Israël (voir Ez. 8) qui fut désormais livré aux nations. Certains estiment que la théocratie a cessé au moment où les Israélites ont demandé un roi (voir 1 S. 8). Or bien que cette demande déplut à Yahweh, il continua néanmoins à diriger Israël au travers des souverains tels que David, qu'il établissait à la tête de son peuple. Les Hébreux avaient déjà reçu un sérieux avertissement avec la destruction du temple lors de la première déportation babylonienne (2 R. 24). Cet événement, bien que traumatisant pour beaucoup, n'avait cependant pas provoqué une réelle repentance, c'est pourquoi les Israélites retombèrent rapidement dans leurs travers. Ainsi, comme en témoigne Mal. 2~:17 qui rapporte les propos de certains juifs~: «~Où est le Dieu de la justice~?~», en dépit de la reconstruction du temple sous Néhémie et Esdras, la gloire de Dieu ne s'y manifestait plus depuis longtemps. Ezéchiel ne fait donc qu'assister à la conséquence de plusieurs siècles d'infidélité des Juifs à l'égard de leur Dieu.}, et s'arrêta sur la montagne qui est à l'orient de la ville.
\VS{24}Puis l'Esprit m'enleva et me transporta en Chaldée, vers ceux qui avaient été emmenés captifs, le tout en vision par l'Esprit de Dieu~; et la vision que j'avais vue disparut au-dessus de moi.
\VS{25}Alors je dis à ceux qui avaient été emmenés captifs toutes les paroles que Yahweh m'avait révélées.
\Chap{12}
\TextTitle{Fuite d'Ezéchiel, un signe pour Israël}
\VerseOne{}La parole de Yahweh vint encore à moi en disant~:
\VS{2}Fils de l'homme, tu habites au milieu d'une maison rebelle, au milieu de gens qui ont des yeux pour voir, et ne voient point~; et qui ont des oreilles pour entendre, et n'entendent point~; parce qu'ils sont une maison de rebelles\FTNT{Es. 6:9~; Es. 49:19-20~; Jé. 5:21~; Ac. 28:26.}.
\VS{3}Toi donc, fils de l'homme, fais-toi des bagages d'un homme qui s'exile et pars en exil de jour, sous leurs yeux~; pars en exil, dis-je, de ton lieu pour aller dans un autre lieu, sous leurs yeux. Peut-être qu'ils y prendront garde, quoi qu'ils soient une maison rebelle.
\VS{4}Tu mettras donc dehors pendant le jour tes bagages comme les bagages d'un homme qui s'exile sous leurs yeux, et le soir, tu sortiras sous leurs yeux, comme quand on sort pour s'exiler.
\VS{5}Perce-toi le mur sous leurs yeux et sors par là tes bagages.
\VS{6}Tu les porteras sur tes épaules, sous leurs yeux, et tu sortiras tes bagages pendant l'obscurité. Tu couvriras aussi ton visage, afin que tu ne voies point la terre~; car je t'ai mis pour être un signe pour la maison d'Israël.
\VS{7}Je fis donc ce qui m'avait été ordonné~: Je portai dehors pendant le jour mes bagages comme des bagages d'exil~; le soir je perçai le mur avec la main, et je les sortis pendant l'obscurité, et je les portai sur l'épaule, sous leurs yeux.
\VS{8}Au matin, la parole de Yahweh me fut adressée en ces mots~:
\VS{9}Fils de l'homme, la maison d'Israël, maison rebelle, ne t'a-t-elle pas dit~: Qu'est-ce que tu fais~?
\VS{10}Dis-leur~: Ainsi parle le Seigneur Yahweh~: Ce fardeau dont je suis chargé s'adresse au prince qui est à Jérusalem, et à toute la maison d'Israël qui s'y trouve.
\VS{11}Dis~: Je suis pour vous un signe~; comme j'ai fait, ainsi il leur sera fait. Ils iront en exil, en captivité.
\VS{12}Et le prince qui est parmi eux, mettra son bagage sur l'épaule et sortira~; on percera le mur pour le tirer dehors~; il couvrira son visage, afin qu'il ne voie point de ses yeux la terre\FTNT{2 R. 25:4.}.
\VS{13}J'étendrai mon rets sur lui, et il sera pris dans mes filets~; je le ferai entrer dans Babylone, au pays des Chaldéens, mais il ne la verra point, et il y mourra.
\VS{14}Je disperserai à tout vent tout ce qui est autour de lui, son secours, et tous ses corps d'armées~; et je tirerai l'épée sur eux.
\VS{15}Ils sauront que je suis Yahweh, quand je les aurai répandus parmi les nations, et que je les aurai dispersés dans divers pays.
\VS{16}Je laisserai un reste d'entre eux, quelques hommes, préservés de l'épée, de la famine, et de la peste, afin qu'ils racontent toutes leurs abominations parmi les nations où ils iront~; et ils sauront que je suis Yahweh.
\TextTitle{La captivité du peuple imminente\FTNTT{Cp. 2 R. 25:1-10.}}
\VS{17}Puis la parole de Yahweh vint à moi en disant~:
\VS{18}Fils de l'homme, mange ton pain dans l'agitation, et bois ton eau en tremblant et avec inquiétude.
\VS{19}Puis tu diras au peuple du pays~: Ainsi parle le Seigneur Yahweh, sur les habitants de Jérusalem, à la terre d'Israël~: Ils mangeront leur pain avec chagrin, et ils boiront leur eau avec frayeur, parce que son pays sera désolé, étant privé de son abondance, à cause de la violence de tous ceux qui y habitent.
\VS{20}Les villes peuplées seront désertes, et le pays ne sera que désolation~; et vous saurez que je suis Yahweh.
\VS{21}La parole de Yahweh vint encore à moi en disant~:
\VS{22}Fils de l'homme, que signifient ces discours moqueurs que vous tenez sur la terre d'Israël, en disant~: Les jours seront prolongés, et toute vision périra\FTNT{Es. 5:19~; Am. 6:3~; 2 Pi. 3:3.}~?
\VS{23}C'est pourquoi dis-leur~: Ainsi parle le Seigneur Yahweh~: Je ferai cesser ce proverbe, et on ne s'en servira plus comme proverbe en Israël~; et dis-leur~: Les jours approchent, et toutes les visions s'accompliront.
\VS{24}Car il n'y aura plus désormais aucune vision de vanité ni aucune divination de flatteur, au milieu de la maison d'Israël.
\VS{25}Car moi, Yahweh, je parlerai, et la parole que j'aurai prononcée sera mise en exécution, elle ne sera plus différée~; mais ô maison rebelle~! Je prononcerai en vos jours la parole, et je l'exécuterai, dit le Seigneur Yahweh.
\VS{26}La parole de Yahweh me fut encore adressée en ces mots~:
\VS{27}Fils de l'homme, voici, ceux de la maison d'Israël disent~: La vision que celui-ci voit n'arrivera pas avant longtemps, et il prophétise pour des temps qui sont encore éloignés.
\VS{28}C'est pourquoi dis-leur~: Ainsi parle le Seigneur Yahweh~: Aucune de mes paroles ne sera plus différée, mais la parole que j'aurai prononcée sera exécutée incessament, dit le Seigneur Yahweh.
\Chap{13}
\TextTitle{Jugement sur ceux qui égarent le peuple de Dieu}
\VerseOne{}La parole de Yahweh vint encore à moi en disant~:
\VS{2}Fils de l'homme, prophétise contre les prophètes d'Israël qui prophétisent, et dis à ces prophètes qui prophétisent selon leur propre cœur~: Ecoutez la parole de Yahweh~!
\VS{3}Ainsi parle le Seigneur Yahweh~: Malheur aux prophètes insensés qui suivent leur propre esprit, et qui n'ont point eu de vision.
\VS{4}Israël, tes prophètes ont été comme des renards dans les déserts.
\VS{5}Vous n'êtes point montés devant les brèches, et vous n'avez point réparé les murs pour la maison d'Israël, afin de vous tenir debout pour le combat au jour de Yahweh.
\VS{6}Ils ont eu des visions vaines et des divinations de mensonge, ils disent~: Yahweh a dit~! Et toutefois Yahweh ne les a point envoyés~; et ils font espérer que leur parole s'accomplira\FTNT{Les faux prophètes (Jé. 23)~; Jé. 14:14~; Jé. 28:15.}.
\VS{7}N'avez-vous pas vu des visions de vanité, et prononcé des divinations de mensonge~? Cependant vous dites~: Yahweh a dit~! Et je n'ai point parlé.
\VS{8}C'est pourquoi ainsi parle le Seigneur Yahweh~: Parce que vous avez prononcé des choses vaines, et que vous avez eu des visions de mensonge, à cause de cela j'en veux à vous, dit le Seigneur Yahweh.
\VS{9}Et ma main sera sur les prophètes qui ont des visions de vanité et des divinations de mensonge~; ils ne seront plus admis dans le conseil de mon peuple, ils ne seront plus écrits dans les registres de la maison d'Israël, ils n'entreront plus dans la terre d'Israël~; et vous saurez que je suis le Seigneur Yahweh.
\VS{10}Oui, parce qu'ils ont abusé mon peuple, en disant~: Paix~! Et il n'y avait point de paix\FTNT{Jé. 6:14~; Jé. 8:11.}. L'un bâtissait le mur, et les autres l'enduisaient de mortier mal lié.
\VS{11}Dis à ceux qui enduisent le mur de mortier mal lié qu'il tombera~; il y aura une pluie débordante, et vous, pierres de grêle, vous tomberez sur lui, et un vent de tempête le fendra.
\VS{12}Et voici, le mur est tombé~! Ne vous sera-t-il donc pas dit~: Où est l'enduit dont vous l'avez couvert~?
\VS{13}C'est pourquoi, ainsi parle le Seigneur Yahweh~: Je ferai dans ma fureur éclater un vent impétueux, et dans ma colère, il surviendra une pluie débordante et des pierres de grêle dans ma fureur, pour détruire entièrement.
\VS{14}Je démolirai le mur que vous avez enduit de mortier mal lié, je le jetterai par terre, tellement que son fondement sera découvert, et il tombera~; vous serez consumés au milieu de lui, et vous saurez que je suis Yahweh.
\VS{15}Ainsi, j'accomplirai ma colère contre le mur, et contre ceux qui l'enduisent de mortier mal lié~; et je vous dirai~: Le mur n'est plus ni ceux qui l'ont enduit~;
\VS{16}à savoir les prophètes d'Israël, qui prophétisent sur Jérusalem, et qui voient pour elle des visions de paix~; et néanmoins il n'y a point de paix, dit le Seigneur Yahweh.
\VS{17}Aussi, toi, fils de l'homme, tourne ta face contre les filles de ton peuple qui prophétisent selon leur propre cœur, prophétise contre elles~!
\VS{18} Et dis~: Ainsi parle le Seigneur Yahweh~: Malheur à celles qui cousent des coussins\FTNT{Le mot «~coussin~» vient du terme hébreu «~keceth~», et signifie~: «~bande~», «~filet~», «~faux phylactères~». Il s'agissait d'un tissu utilisé par les fausses prophétesses en Israël dans le but de se faire passer pour de vrais servantes de Yahweh, et ainsi tromper le peuple.} pour s'accouder le long du bras jusqu'aux mains, et qui font des voiles pour mettre sur la tête des personnes de toute taille, pour séduire les âmes. Séduiriez-vous les âmes de mon peuple~; et conserveriez-vous vos âmes~?
\VS{19}Et me profaneriez-vous envers mon peuple pour des poignées d'orge et pour des morceaux de pain, en faisant mourir les âmes qui ne devaient point mourir, et en faisant vivre les âmes qui ne devaient point vivre, en mentant à mon peuple qui écoute le mensonge~?
\VS{20}C'est pourquoi ainsi parle le Seigneur Yahweh~: Voici, j'en veux à vos coussins, par lesquels vous séduisez les âmes pour les faire voler vers vous~; et je déchirerai ces coussins de vos bras, et je ferai échapper les âmes que vous avez attirées afin qu'elles volent vers vous\FTNT{1 Co. 6:10~; 2 Pi. 2:14~; Ap. 18:11-13.}.
\VS{21}Je déchirerai aussi vos voiles, et je délivrerai mon peuple d'entre vos mains, et ils ne seront plus entre vos mains pour en faire votre proie~; et vous saurez que je suis Yahweh.
\VS{22}Parce que vous avez affligé sans raison le cœur du juste, quand moi-même je ne l'ai point attristé, et que vous avez renforcé les mains du méchant, afin qu'il ne se détourne point de son mauvais chemin, et que je lui sauve la vie.
\VS{23}C'est pourquoi, vous n'aurez plus aucune vision de vanité ni aucune divination, mais je délivrerai mon peuple d'entre vos mains~; et vous saurez que je suis Yahweh.
\Chap{14}
\TextTitle{Jugement sur les anciens idolâtres}
\VerseOne{}Or quelques-uns des anciens d'Israël vinrent auprès de moi et s'assirent devant moi.
\VS{2}Et la parole de Yahweh vint à moi en disant~:
\VS{3}Fils de l'homme, ces gens élèvent leurs idoles dans leurs cœurs, et ils attachent les regards sur ce qui les fait tomber dans l'iniquité. Serais-je consulté par eux sérieusement~?
\VS{4}C'est pourquoi parle-leur et dis-leur~: Ainsi parle le Seigneur Yahweh. Quiconque de la maison d'Israël aura élevé ses idoles dans son cœur, et aura mis devant sa face ce qui l'a fait tomber dans son iniquité, s'il vient vers le prophète, je suis Yahweh, je lui répondrai puisqu'il vient avec la multitude de ses idoles,
\VS{5}afin que je saisisse la maison d'Israël par leur propre cœur~; car eux tous se sont éloignés de moi par leurs idoles.
\VS{6}C'est pourquoi dis à la maison d'Israël~: Ainsi parle le Seigneur Yahweh~: Revenez, et détournez-vous de vos idoles, détournez les regards de toutes vos abominations\FTNT{Es. 55:6-7.}.
\VS{7}Car quiconque de la maison d'Israël, ou des étrangers qui séjournent en Israël, qui s'est séparé de moi, qui élève ses idoles dans son cœur, et attache ses regards sur ce qui l'a fait tomber dans l'iniquité, s'il vient vers le prophète pour me consulter par lui, je suis Yahweh, on lui répondra tout ce qu'on a à lui répondre.
\VS{8}Je me tournerai contre cet homme\FTNT{Lé. 17:10~; Lé. 20:3-6~; Jé. 44:11.}, et je ferai de lui un signe, et un sujet de sarcasme\FTNT{No. 26:10~; De. 28:37.}. Je le retrancherai du milieu de mon peuple~; et vous saurez que je suis Yahweh.
\VS{9}S'il arrive que le prophète soit séduit, et qu'il profère quelque parole, moi, Yahweh, je séduirai ce prophète-là\FTNT{1 R. 22:23~; Job 12:16~; 2 Th. 2:11.}~; et j'étendrai ma main sur lui, et je l'exterminerai du milieu de mon peuple d'Israël.
\VS{10}Et ils porteront la peine de leur iniquité~; la peine de l'iniquité du prophète sera comme la peine de celui qui l'aura interrogé~;
\VS{11}afin que la maison d'Israël ne s'éloigne plus de moi, et qu'ils ne se souillent plus par tous leurs crimes\FTNT{Jé. 31:18-19~; Hé. 12:11~; Ja. 1:1-3.}. Alors ils seront mon peuple, et je serai leur Dieu, dit le Seigneur Yahweh.
\TextTitle{Châtiments d'Israël~; Yahweh épargne un reste}
\VS{12}Puis la parole de Yahweh vint à moi en disant~:
\VS{13}Fils de l'homme, lorsqu'un pays aura péché contre moi, en commettant une infidélité, et que j'aurai étendu ma main contre lui, et que je lui aurai rompu le bâton du pain, envoyé la famine et retranché du milieu de lui tant les hommes que les bêtes,
\VS{14}et que ces trois hommes, Noé\FTNT{Ge. 6:8.}, Daniel\FTNT{Da. 1:8-12.} et Job\FTNT{Job 1:8.} s'y trouveraient, ils sauveraient leurs âmes par leur justice, dit le Seigneur Yahweh.
\VS{15}Si je fais passer les bêtes féroces par ce pays-là et qu'elles le privent d'enfants, tellement qu'il soit devenu un désert où personne ne passe à cause des bêtes,
\VS{16}et que ces trois hommes-là s'y trouvent, je suis vivant, dit le Seigneur Yahweh, ils ne sauveraient ni fils ni filles, eux seulement seraient sauvés, et le pays serait un désert.
\VS{17}Si je faisais venir l'épée sur ce pays-là et si je disais~: Que l'épée passe par le pays, et qu'elle en retranche les hommes et les bêtes~!
\VS{18}Si ces trois hommes-là se trouveraient au milieu du pays, je suis vivant, dit le Seigneur, ils ne sauveraient ni fils ni filles~; mais eux seulement seraient sauvés.
\VS{19}Ou si j'envoyais la peste dans ce pays, et que je répandais ma colère contre lui jusqu'à faire ruisseler le sang, au point de retrancher du milieu de lui les hommes et les bêtes,
\VS{20}et que Noé, Daniel et Job, s'y trouveraient, je suis vivant, dit le Seigneur Yahweh, ils ne sauveraient ni fils ni filles, mais ils sauveraient leurs âmes par leur justice.
\VS{21}Car ainsi parle le Seigneur Yahweh~: J'envoie mes quatre plaies mortelles, l'épée, la famine, les bêtes féroces, et la peste, contre Jérusalem, pour en retrancher les hommes et les bêtes\FTNT{Jé. 15:2-3.}~;
\VS{22}et toutefois, il y aura un reste qui échappera, qui en sortira, des fils et des filles. Voici, ils viennent vers vous, et vous verrez leur conduite et leurs actions, et vous serez consolés du malheur que je fais venir contre Jérusalem, et de tout ce que j'aurais fait venir sur elle.
\VS{23}Vous serez consolés, lorsque vous verrez leur conduite et leurs actions~; et vous reconnaîtrez que ce n'est pas sans raison que je fais tout ce que je lui fais, dit le Seigneur Yahweh\FTNT{Jé. 22:8-9.}.
\Chap{15}
\TextTitle{Infidélité d'Israël\FTNTT{Cp. Es. 5:1-24.}}
\VerseOne{}La parole de Yahweh vint encore à moi en disant~:
\VS{2}Fils de l'homme, que vaut le bois de la vigne de plus que les autres bois~? Et les sarments de plus que les branches des arbres d'une forêt~?
\VS{3}Et prendra-t-on de ce bois pour en faire quelque ouvrage~? Ou prendra-t-on un clou pour y pendre quelque chose~?
\VS{4}Voici, on le met au feu pour être consumé~; le feu consume aussitôt ses deux bouts, et le milieu est en feu~; serait-il bon pour quelque ouvrage~?
\VS{5}Voici, quand il est entier, on n'en fait aucun ouvrage~; à plus forte raison quand le feu l'aura consumé et qu'il sera brûlé, sera-t-il bon pour quelque ouvrage~?
\VS{6}C'est pourquoi ainsi parle le Seigneur Yahweh~: Comme le bois de la vigne est parmi les arbres d'une forêt, que j'ai assigné au feu pour être consumé, ainsi je livrerai les habitants de Jérusalem.
\VS{7}Je me tournerai contre eux, seront-ils sortis du feu~? Encore le feu les consumera~; et vous saurez que je suis Yahweh, quand je tournerai ma face contre eux.
\VS{8}Je ferai de ce pays une désolation, parce qu'ils ont commis une infidélité, dit le Seigneur Yahweh.
\Chap{16}
\TextTitle{Bonté de Yahweh~; prostitutions d'Israël}
\VerseOne{}De nouveau la parole de Yahweh vint à moi en disant~:
\VS{2}Fils de l'homme, fais connaître à Jérusalem ses abominations.
\VS{3}Et dis~: Ainsi parle le Seigneur Yahweh à Jérusalem~: Tu as tiré ton origine et ta naissance du pays de Canaan~; ton père était Amoréen, et ta mère Héthienne.
\VS{4}Quant à ta naissance, le jour où tu naquis, ton cordon ombilical n'a pas été coupé, tu n'as pas été lavée dans l'eau pour être nettoyée~; tu n'as pas été salée de sel ni emmaillotée.
\VS{5}Il n'y a pas eu d'œil qui ait eu pitié de toi pour te faire une seule de ces choses, en ayant compassion pour toi~; mais tu as été jetée sur la face des champs le jour de ta naissance, parce qu'on avait horreur de toi.
\VS{6}Et passant près de toi, je te vis gisante par terre, dans ton sang, et je te dis~: Vis dans ton sang~! Et je te redis encore~: Vis dans ton sang~!
\VS{7}Je t'ai fait croître par millions comme l'herbe des champs. Et tu pris de l'accroissement et tu devins grande, tu parvins à une parfaite beauté, tes seins se formèrent, ta chevelure poussa, tu devins nubile~; mais tu étais abandonnée et sans habits.
\VS{8}Je passai près de toi, je te regardai, et voici, le temps était là, le temps des amours. J'étendis sur toi le pan de ma robe, et je couvris ta nudité. Je te jurai, j'entrai en alliance avec toi, dit le Seigneur Yahweh, et tu devins mienne.
\VS{9}Je te lavai dans l'eau en t'y plongeant, j'ôtai le sang de dessus toi, et je t'oignis d'huile.
\VS{10}Je te revêtis de vêtements brodés, je te chaussai de fourrure, je te ceignis de fin lin, et je te couvris de soie.
\VS{11}Je te parai d'ornements, je mis des bracelets sur tes mains, et un collier à ton cou.
\VS{12}Je mis un anneau à ton nez, des pendants à tes oreilles, et une couronne de gloire sur ta tête.
\VS{13}Tu fus donc parée d'or et d'argent, et ton vêtement était de fin lin, de soie, et de broderie. Tu mangeas la fleur de farine, le miel, et l'huile~; tu devins extrêmement belle, et tu prospéras jusqu'à régner.
\VS{14}Ta renommée se répandit parmi les nations à cause de ta beauté, car elle était parfaite, à cause de ma gloire que j'avais mise sur toi, dit le Seigneur Yahweh.
\VS{15}Mais tu t'es confiée dans ta beauté, et tu t'es prostituée à cause de ta renommée, tu t'es abandonnée à tous les passants\FTNT{Es. 1:21~; Jé. 2:20~; Jé. 3:2-6~; Os. 1:2.}.
\VS{16}Tu as pris tes vêtements pour t'en faire des hauts lieux de diverses couleurs, tels qu'il n'y en a point eu et n'y en aura jamais, et tu t'y es prostituée.
\VS{17}Tu as pris ta magnifique parure d'or et d'argent, que je t'avais donnée, et tu t'en es fait des images d'hommes, tu as commis la fornication avec elles.
\VS{18}Tu as pris tes vêtements brodés, tu les en as couvertes, et tu as mis mon huile et mon encens devant elles.
\VS{19}Mon pain que je t'avais donné, la fleur de farine, l'huile, et le miel que je t'avais donné à manger, tu as mis cela devant elles en sacrifice de bonne odeur~; il a été fait ainsi, dit le Seigneur Yahweh.
\VS{20}Tu as aussi pris tes fils et tes filles que tu m'avais enfantés, et tu les as sacrifiés pour être mangés\FTNT{Lé. 18:21~; Lé. 20:2~; Es. 57:5~; Jé. 19:5, Jé. 32:35.}. N'était-ce pas assez de tes prostitutions~?
\VS{21}Tu as égorgé mes enfants, et tu les as livrés pour les faire passer par le feu, en l'honneur de ces idoles\FTNT{2 R. 17:17.}.
\VS{22}Et parmi toutes tes abominations et tes adultères, tu ne t'es point souvenue du temps de ta jeunesse, quand tu étais sans habits et toute nue, gisante par terre dans ton sang.
\VS{23}Après toutes tes méchancetés, malheur, malheur à toi~! dit le Seigneur Yahweh.
\VS{24}Tu t'es bâti un lieu éminent, et tu t'es fait des hauts lieux dans toutes les places.
\VS{25}A l'entrée de chaque chemin tu as bâti un haut lieu, et tu as rendu ta beauté abominable, tu t'es prostituée à tous les passants, tu as multiplié tes adultères.
\VS{26}Tu t'es abandonnée aux enfants d'Egypte, tes voisins au corps avantageux et tu as multiplié tes adultères pour m'irriter.
\VS{27}Et voici, j'ai étendu ma main sur toi, j'ai diminué la portion que je t'avais prescrite, et je t'ai abandonnée à la volonté de celles qui te haïssaient, des filles des Philistins, lesquelles ont honte de tes voies qui ne sont que méchanceté.
\VS{28}Tu t'es aussi abandonnée aux fils des Assyriens\FTNT{2 R. 16:7-10~; Jé. 2:18-36.}, parce que tu n'étais pas encore rassasiée~; et après avoir commis l'adultère avec eux, tu n'as point encore été rassasiée.
\VS{29}Tu as multiplié tes adultères dans le pays de Canaan jusqu'en Chaldée, et avec cela tu n'as pas encore été rassasiée.
\VS{30}Quelle faiblesse de cœur tu as eue, dit le Seigneur Yahweh, d'avoir fait toutes ces choses-là, qui sont les actions d'une femme qui se prostitue avec arrogance~;
\VS{31}de t'être bâti un lieu éminent à chaque entrée de chemin, et d'avoir fait ton haut lieu dans toutes les places. Et tu n'as pas été comme la prostituée, car tu n'as point tenu compte du salaire.
\VS{32}Femme adultère, tu prends des étrangers au lieu de ton mari.
\VS{33}On donne un salaire à toutes les prostituées, mais toi tu as donné à tous tes amants les présents\FTNT{Es. 57:8-9~; Os. 8:9-10.}, que ton mari t'avait fait, tu les as gagnés par des présents, afin que de toutes parts ils viennent vers toi, pour se plonger avec toi dans le crime.
\VS{34}Tu as été le contraire des autres prostituées, parce qu'on ne te recherchait pas~; et en donnant un salaire au lieu d'en recevoir un, tu as été le contraire des autres.
\TextTitle{Conséquences de l'infidélité de Jérusalem}
\VS{35}C'est pourquoi, ô adultère, écoute la parole de Yahweh~!
\VS{36}Ainsi parle le Seigneur Yahweh~: Parce que ton venin s'est répandu, et que dans tes excès tu t'es abandonnée à ceux que tu aimais, à tes abominables idoles, et que tu as mis à mort tes enfants que tu leur as donnés~;
\VS{37}à cause de cela, voici, je vais rassembler tous tes amants, avec lesquels tu te plaisais, et tous ceux que tu as aimés, avec tous ceux que tu as haïs~; je les assemblerai de toutes parts contre toi, je découvrirai ta honte à leurs yeux et ils verront ton infamie.
\VS{38}Et je te jugerai comme on juge les femmes adultères, et celles qui répandent le sang\FTNT{Lé. 20:10~; De. 22:22-30.}~; je te livrerai pour être mise à mort selon ma fureur et ma jalousie.
\VS{39}Je te livrerai, dis-je, entre leurs mains~; et ils détruiront tes maisons de prostitution, ils détruiront tes hauts lieux~; ils te dépouilleront de tes vêtements, emporteront ta magnifique parure, et ils te laisseront sans habits et entièrement nue.
\VS{40}Et on fera monter contre toi une foule de gens qui te lapideront de pierres, et qui te perceront avec leurs épées.
\VS{41}Puis ils brûleront tes maisons, et feront justice de toi aux yeux d'un grand nombre de femmes. Je ferai cesser tes prostitutions et tu ne donneras plus de salaires.
\VS{42}J'abandonnerai alors ma colère contre toi, et ma jalousie se retirera de toi~; je serai en repos, et je ne m'irriterai plus.
\VS{43}Parce que tu ne t'es point souvenue du temps de ta jeunesse, et que tu m'as provoqué par toutes ces choses-là~; à cause de cela, voici, j'ai fait tomber la peine de tes crimes sur ta tête, dit le Seigneur Yahweh~; et tu ne feras plus de mauvais projets avec toutes tes abominations.
\VS{44}Voici, tous ceux qui usent de proverbes feront un proverbe de toi, en disant~: Telle mère, telle fille~!
\VS{45}Tu es la fille de ta mère, qui a dédaigné son mari et ses fils~; et tu es la sœur de chacune de tes sœurs, qui ont dédaigné leurs maris et leurs fils. Votre mère était Héthienne, et votre père était Amoréen.
\VS{46}Ta grande sœur qui demeure à ta gauche, c'est Samarie avec ses filles~; et ta petite sœur qui demeure à ta droite, c'est Sodome avec ses filles.
\VS{47}Et tu n'as pas seulement marché dans leurs voies et fait selon leurs abominations, c'était fort peu~; mais tu t'es corrompue plus qu'elles dans toutes tes voies.
\VS{48}Je suis vivant~! dit le Seigneur Yahweh, Sodome, ta sœur et ses filles, n'ont point fait comme tu as fait, toi et tes filles.
\VS{49}Voici quel a été le crime de Sodome, ta sœur~: Elle avait de l'orgueil, elle vivait dans l'abondance de pain et dans une insouciante tranquillité, elle et ses filles, et elle ne fortifiait pas la main du pauvre et de l'indigent.
\VS{50}Elles se sont élevées, elles ont commis des abominations devant moi, et je me suis détourné quand j'ai vu cela.
\VS{51}Quant à Samarie, elle n'a pas commis la moitié de tes péchés~; car tu as multiplié tes abominations plus qu'elle, et tu as justifié tes sœurs par toutes les abominations que tu as commises.
\VS{52}Porte ta honte, toi qui as jugé chacune de tes sœurs, à cause de tes péchés, par lesquels tu as été rendue plus abominable qu'elles~; elles sont plus justes que toi~; c'est pourquoi sois honteuse, et porte ta confusion, vu que tu as justifié tes sœurs.
\VS{53}Quand je ramènerai leurs captifs, les captifs, dis-je, de Sodome et des villes de son ressort, et les captifs de Samarie et des villes de son ressort, je ramenèrai aussi les captifs de la captivité parmi elles~!
\VS{54}Afin que tu portes ta honte, et que tu sois confuse à cause de tout ce que tu as fait, et que tu les consoles.
\VS{55}Quand ta sœur Sodome et les villes de son ressort retourneront à leur état précédent, Samarie et les villes de son ressort retourneront à leur état précédent~; toi aussi, et les villes de ton ressort retournerez à votre état précédent.
\VS{56}Or ta bouche n'a point fait mention de ta sœur, Sodome, au jour de tes fiertés,
\VS{57}avant que ta méchanceté soit découverte~; lorsque tu as reçu les outrages des filles de Syrie, et de tous ses alentours, des filles des Philistins, qui te pillèrent de tous côtés~!
\VS{58}Tu portes sur toi tes méchancetés et tes abominations, dit Yahweh.
\VS{59}Car ainsi parle le Seigneur Yahweh~: Je te ferai comme tu as fait, quand tu as méprisé le serment en rompant l'alliance.
\TextTitle{Fidélité de Yahweh à son alliance}
\VS{60}Mais je me souviendrai de l'alliance que j'ai traitée avec toi dans les jours de ta jeunesse, et j'établirai avec toi une alliance éternelle\FTNT{Lé. 26:42-45~; Ps. 106:45.}.
\VS{61}Et tu te souviendras de tes voies, et en seras confuse, lorsque tu recevras tes sœurs, tant tes plus grandes que tes plus petites~; et je te les donnerai pour filles, mais pas selon ton alliance.
\VS{62}Car j'établirai mon alliance avec toi, et tu sauras que je suis Yahweh,
\VS{63}afin que tu te souviennes de ta vie passée, que tu en sois honteuse, et que tu n'ouvres plus la bouche, à cause de ta confusion, après que j'aurai été apaisé envers toi, pour tout ce que tu as fait, dit le Seigneur Yahweh.
\Chap{17}
\TextTitle{Enigme de Yahweh}
\VerseOne{}La parole de Yahweh vint à moi en disant~:
\VS{2}Fils de l'homme, propose une énigme, une parabole à la maison d'Israël.
\VS{3}Tu diras~: Ainsi parle le Seigneur Yahweh~: Un grand aigle à grandes ailes, aux ailes déployées, couvert de plumes de toutes les couleurs, vint au Liban, et enleva la cime d'un cèdre.
\VS{4}Il arracha la tête de ses rameaux, l'emmena dans un pays de commerce, et la mit dans une ville marchande.
\VS{5}Il prit de la semence du pays, et la mit dans un champ propre à semer, et l'apportant près des grosses eaux, la planta comme un saule.
\VS{6}Cette semence poussa et devint un cep de vigne étendu, mais de peu d'élévation~; ses rameaux étaient tournés vers l'aigle, et ses racines étaient sous lui~; il devint une vigne, donna des jets, et produisit des branches.
\VS{7}Mais il y avait un autre grand aigle, aux longues ailes, et au plumage épais. Et voici, cette vigne serra vers lui ses racines, et étendit ses rameaux vers lui, afin qu'il l'arrose des eaux qui coulent des terrasses.
\VS{8}Elle était donc plantée dans une bonne terre, près des grosses eaux, en sorte qu'il y sortait des sarments et portait du fruit\FTNT{Mt. 13:8-23~; Mc. 4:8-20~; Lu. 8:8-15.}. Elle était devenue une vigne magnifique.
\VS{9}Dis~: Ainsi parle le Seigneur Yahweh: Prospèrera-t-elle~? N'arrachera-t-il pas ses racines, et ne coupera-t-il pas ses fruits pour qu'ils deviennent secs~? Tous les sarments qu'il a jetés sécheront, et il ne faudra pas un grand effort et beaucoup de monde pour l'enlever de dessus ses racines.
\VS{10}Mais voici, quoique plantée, prospèrera-t-elle~? Quand le vent d'orient l'aura touchée, ne séchera-t-elle pas entièrement~? Elle séchera sur le terrain où elle était plantée.
\TextTitle{Jugement de Dieu sur Sédécias\FTNTT{2 R. 24:17-20~; 25:1-10.}}
\VS{11}Puis la parole de Yahweh me fut adressée en ces mots~:
\VS{12}Parle maintenant à la maison rebelle~: Ne savez-vous pas ce que veulent dire ces choses~? Dis~: Voici, le roi de Babylone est venu à Jérusalem. Il a pris le roi, et les princes, et les a emmenés avec lui à Babylone.
\VS{13}Il en a pris un de la race royale, il a traité alliance avec lui, il lui a fait prêter serment, et il a retenu les puissants du pays,
\VS{14}afin que le royaume soit tenu dans l'abaissement, et qu'il ne s'élève point, mais qu'en gardant son alliance, il subsiste.
\VS{15}Mais celui-ci s'est rebellé contre lui, envoyant ses messagers en Egypte, pour qu'on lui donne des chevaux et un grand peuple. Celui qui fait de telles choses prospérera-t-il, échappera-t-il~? Ayant enfreint l'alliance, échappera-t-il~?
\VS{16}Je suis vivant, dit le Seigneur Yahweh, c'est dans le pays du roi qui l'a établi pour roi, envers qui il a violé son serment et dont il a rompu l'alliance, c'est près de lui, au milieu de Babylone, qu'il mourra\FTNT{Référence à Nebucadnetsar. Sédécias eut les yeux crevés avant d'être emmené captif (2 R. 25:7~; Jé. 34:3~; Jé. 52:11).}.
\VS{17}Pharaon n'ira pas avec une grande armée et un peuple nombreux pour le secourir dans cette guerre, lorsque l'ennemi élèvera des terrasses et fera des retranchements pour exterminer beaucoup d'âmes.
\VS{18}Car il a méprisé le serment en violant l'alliance~; car voici, après avoir donné sa main, il a fait néanmoins toutes ces choses-là~; il n'échappera point~!
\VS{19}C'est pourquoi, ainsi parle le Seigneur Yahweh~: Je suis vivant, si je ne fais tomber sur sa tête mon serment qu'il a méprisé, et mon alliance qu'il a enfreinte.
\VS{20}Et j'étendrai mon rets sur lui, et il sera pris dans mes filets, je le ferai entrer dans Babylone, et là j'entrerai en jugement contre lui pour le crime qu'il a commis contre moi.
\VS{21}Et tous ses fugitifs avec toutes ses troupes tomberont par l'épée, et ceux qui resteront seront dispersés à tout vent~; et vous saurez que moi, Yahweh, j'ai parlé.
\VS{22}Ainsi parle le Seigneur Yahweh~: Je prendrai aussi un rameau de la cime de ce haut cèdre, et je le planterai~; je couperai, dis-je, du bout de ses jeunes branches, un tendre rameau, et je le planterai sur une montagne haute et éminente.
\VS{23}Je le planterai sur la haute montagne d'Israël, et là il produira des branches et produira du fruit, et il deviendra un excellent cèdre. Et des oiseaux de tout plumage demeureront sous lui, et habiteront sous l'ombre de ses branches.
\VS{24}Et tous les bois des champs connaîtront que moi, Yahweh, j'aurai abaissé le grand arbre, et élevé le petit arbre, fait sécher le bois vert, et fait reverdir le bois sec. Moi, Yahweh, j'ai parlé, et je le ferai.
\Chap{18}
\TextTitle{Chacun responsable de son péché}
\VerseOne{}La parole de Yahweh vint encore à moi en disant~:
\VS{2}Que voulez-vous dire, vous qui usez ordinairement de ce proverbe concernant le pays d'Israël, en disant~: Les pères ont mangé des raisins verts et les dents des enfants ont été agacées\FTNT{Jé. 31:29~; La. 5:7.}~?
\VS{3}Je suis vivant, dit le Seigneur Yahweh, et vous n'userez plus de ce proverbe en Israël.
\VS{4}Voici, toutes les âmes sont à moi~; l'âme du fils est à moi comme l'âme du père~; l'âme qui pèche sera celle qui mourra.
\VS{5}Mais l'homme qui est juste et qui pratique la droiture et la justice,
\VS{6}qui ne mange pas sur les montagnes, qui ne lève pas ses yeux vers les idoles de la maison d'Israël, qui ne souille pas la femme de son prochain et ne s'approche pas de la femme dans son état d'impureté\FTNT{Lé. 18:18~; Lé. 20:18.},
\VS{7}qui n'opprime personne, qui rend le gage à son débiteur\FTNT{Ex. 22:26~; De. 24:12-13.}, qui ne ravit pas le bien d'autrui, qui donne son pain à celui qui a faim et qui couvre d'un vêtement celui qui est nu\FTNT{De. 15:11~; Es. 58:7.},
\VS{8}qui ne prête pas à intérêt, qui ne tire pas d'usure, qui détourne sa main de l'iniquité et qui juge selon la vérité entre les parties qui plaident ensemble\FTNT{Ex. 22:25~; Lé. 25:35-37~; De. 23:19.},
\VS{9}qui suit mes lois et garde mes ordonnances pour agir avec fidélité, celui-là est juste~; il vivra, il vivra, dit le Seigneur Yahweh.
\VS{10}Et s'il a engendré un fils qui soit un meurtrier, répandant le sang, et commettant des choses semblables~;
\VS{11}et qui ne fasse aucune de ces choses que j'ai ordonnées, s'il mange sur les montagnes, s'il déshonore la femme de son prochain,
\VS{12}s'il opprime le malheureux et le pauvre, s'il ravit le bien d'autrui, s'il ne rend pas le gage, s'il lève ses yeux vers les idoles et commet des abominations,
\VS{13}s'il prête à intérêt, et tire une usure~; ce fils-là, vivrait-il~? Il ne vivra pas, quand il aura commis toutes ces abominations~; on le fera mourir, et son sang retombera sur lui.
\VS{14}Mais s'il engendre un fils qui voie tous les péchés que commet son père, qui les voie et n'agisse pas de la même manière~;
\VS{15}s'il ne mange pas sur les montagnes et qu'il ne lève point ses yeux vers les idoles de la maison d'Israël, s'il ne déshonore pas la femme de son prochain,
\VS{16}s'il n'opprime personne, s'il ne prend point de gages, s'il ne ravit point le bien d'autrui, s'il donne de son pain à celui qui a faim et couvre celui qui est nu,
\VS{17}s'il retire sa main du pauvre, s'il n'exige ni usure ni intérêt, s'il garde mes ordonnances, et s'il suit mes lois~; il ne mourra point pour l'iniquité de son père, mais il vivra, il vivra.
\VS{18}Mais son père, parce qu'il a usé de fraude, et qu'il a ravi ce qui était à son frère, et fait parmi son peuple ce qui n'est pas bon, voici, il mourra pour son iniquité.
\VS{19}Mais direz-vous~: Pourquoi le fils ne porte-t-il pas l'iniquité de son père\FTNT{Ex. 20:5~; De. 5:9.}~? Parce que le fils a fait ce qui était juste et droit, et qu'il a gardé toutes mes lois et les a observées~; il vivra, il vivra.
\VS{20}L'âme qui pèche est celle qui mourra. Le fils ne portera point l'iniquité du père, et le père ne portera point l'iniquité du fils. La justice du juste sera sur le juste, et la méchanceté du méchant sera sur le méchant.
\VS{21}Si le méchant se détourne de tous ses péchés qu'il aura commis, et qu'il garde toutes mes lois, et fasse ce qui est juste et droit~; il vivra, il vivra, il ne mourra point.
\VS{22}Il ne lui sera point fait mention de tous ses crimes qu'il aura commis, mais il vivra pour sa justice, à laquelle il se sera adonné.
\VS{23}Prendrais-je plaisir à la mort du méchant, dit le Seigneur Yahweh, et non plutôt à ce qu'il se détourne de ses mauvaises voies et qu'il vive~?
\VS{24}Mais si le juste se détourne de sa justice, et commet l'iniquité, selon toutes les abominations que le méchant a l'habitude de commettre, vivra-t-il~? Il ne sera point fait mention de toutes ses justices qu'il aura faites, à cause de son crime qu'il aura commis, et à cause de son péché qu'il aura fait~; il mourra à cause de ces choses-là.
\VS{25}Et vous, vous dites~: La voie du Seigneur n'est pas bien réglée. Ecoutez maintenant maison d'Israël~! Ma voie n'est-elle pas bien réglée~? Ne sont-ce pas plutôt vos voies qui ne sont pas bien réglées~?
\VS{26}Si le juste se détourne de sa justice, et commet l'iniquité, il mourra à cause de ces choses-là~; il mourra à cause de son iniquité qu'il aura commise.
\VS{27}Si le méchant se détourne de sa méchanceté qu'il aura commise, et pratique ce qui est juste et droit, il fera vivre son âme.
\VS{28}Ayant donc considéré sa conduite, et s'étant détourné de tous ses crimes qu'il aura commis~; il vivra, il vivra, il ne mourra point.
\VS{29}La maison d'Israël dit~: La voie du Seigneur Yahweh n'est pas bien réglée. Ô maison d'Israël~! Mes voies ne sont-elles pas bien réglées~? Ne sont-ce pas plutôt vos voies qui ne sont pas bien réglées~?
\VS{30}C'est pourquoi je jugerai chacun de vous selon ses voies, ô maison d'Israël~! dit le Seigneur. Revenez, et détournez-vous de tous vos péchés, et l'iniquité ne vous ruinera pas.
\VS{31}Rejetez loin de vous tous les crimes par lesquels vous avez péché~; et faites-vous un nouveau cœur et un esprit nouveau. Pourquoi mourriez-vous, ô maison d'Israël~?
\VS{32}Car je ne prends point de plaisir à la mort de celui qui meurt, dit le Seigneur Yahweh. Convertissez-vous donc, et vivez\FTNT{Ac. 3:19-20.}.
\Chap{19}
\TextTitle{Complaintes sur les dirigeants d'Israël}
\VerseOne{}Et toi, prononce à haute voix une complainte concernant les princes d'Israël.
\VS{2}Et dis~: Ta mère, qu'était-ce~? C'était une lionne couchée parmi les lions, et qui a élevé ses petits parmi les jeunes lions.
\VS{3}Elle fit croître un de ses petits, qui devint un jeune lion, et qui apprit à déchirer la proie et à dévorer les hommes.
\VS{4}Les nations en entendirent parler, il fut attrapé dans leur fosse~; et elles l'emmenèrent avec des boucles au pays d'Egypte\FTNT{2 R. 23:33-34.}.
\VS{5}Puis ayant vu qu'elle attendait en vain, qu'elle était trompée dans son espérance, elle prit un autre de ses petits, et en fit un jeune lion.
\VS{6}Il marcha parmi les lions et devint un jeune lion, il apprit à déchirer la proie et a dévorer les hommes.
\VS{7}Il désola leurs palais, il ravagea leurs villes, de sorte que le pays, et tout ce qui y est, fut épouvanté par le cri de son rugissement.
\VS{8}Les nations s'armèrent contre lui de toutes les provinces, elles étendirent leurs rets contre lui, et il fut attrapé dans leur fosse\FTNT{2 R. 24:2.}.
\VS{9}Puis ils l'enfermèrent et l'enchaînèrent, pour l'amener au roi de Babylone, et le mettre dans une forteresse, afin que sa voix ne soit plus entendue sur les montagnes d'Israël.
\VS{10}Ta mère était comme une vigne dans ton sang, plantée auprès des eaux, et elle est devenue chargée de fruits et de rameaux, à cause des grandes eaux.
\VS{11}Elle avait de puissantes branches pour en faire des sceptres de souverains~; son tronc s'était élevé jusqu'à ses branches touffues, et on la voyait dans sa hauteur avec la multitude de ses rameaux.
\VS{12}Mais elle a été arrachée avec fureur, et jetée par terre~; le vent d'orient a séché son fruit~; ses puissantes branches se sont rompues et ont séché~; le feu les a consumées.
\VS{13}Maintenant elle est plantée dans le désert, dans une terre sèche et aride.
\VS{14}Le feu est sorti de ses branches, et a consumé son fruit~; et il n'y a plus en elle de puissantes branches pour un sceptre de souverain. C'est là une complainte, et cela servira de complainte.
\Chap{20}
\TextTitle{Compassions de Yahweh face aux infidélités d'Israël}
\VerseOne{}Or il arriva la septième année, au dixième jour du cinquième mois, que quelques-uns des anciens d'Israël vinrent pour consulter Yahweh, et s'assirent devant moi.
\VS{2}La parole de Yahweh vint à moi en disant~:
\VS{3}Fils de l'homme, parle aux anciens d'Israël, et dis-leur~: Ainsi parle le Seigneur Yahweh~: Est-ce pour me consulter que vous venez~? Je suis vivant, dit le Seigneur Yahweh, je ne me laisserai pas consulter par vous.
\VS{4}Ne les jugeras-tu pas, ne les jugeras-tu pas, fils de l'homme~? Donne-leur à connaître les abominations de leurs pères.
\VS{5}Et dis-leur~: Ainsi parle le Seigneur Yahweh~: Le jour où j'ai choisi Israël, j'ai levé ma main vers la postérité de la maison de Jacob, et je me suis fait connaître à eux dans le pays d'Egypte, et j'ai levé ma main vers eux, en disant~: Je suis Yahweh, votre Dieu.
\VS{6}En ce jour, j'ai levé ma main vers eux, pour les faire sortir du pays d'Egypte, pour les amener dans un pays que j'avais cherché pour eux, pays où coulent le lait et le miel, et qui est la noblesse de tous les pays\FTNT{Ex. 3:8~; Ex. 6:7.}.
\VS{7}Alors je leur dis~: Que chacun de vous rejette les abominations qui attirent ses regards, et ne vous souillez point par les idoles d'Egypte~! Je suis Yahweh, votre Dieu\FTNT{Jos. 24:14-23.}.
\VS{8}Mais ils se rebellèrent contre moi, et ils ne voulurent point m'écouter. Aucun ne rejeta les abominations qui attiraient ses regards, et ils n'abandonnèrent point les idoles de l'Egypte. Et je dis que je répandrais ma fureur sur eux, que je consumerais ma colère sur eux au milieu du pays d'Egypte.
\VS{9}Mais je les ai tirés hors du pays d'Egypte, je l'ai fait pour l'amour de mon Nom, afin qu'il ne soit point profané aux yeux des nations parmi lesquelles ils se trouvaient, et aux yeux desquelles je m'étais fait connaître à eux, pour les faire sortir du pays d'Egypte.
\VS{10}Je les fis donc sortir du pays d'Egypte, et je les conduisis dans le désert.
\VS{11}Je leur donnai mes lois et leur fis connaître mes ordonnances, que l'homme doit mettre en pratique, afin de vivre par elles\FTNT{Lé. 18:5~; Ro. 10:5~; Ga. 3:12.}.
\VS{12}Je leur donnai aussi mes sabbats, pour être un signe entre moi et eux, afin qu'ils sachent que je suis Yahweh qui les sanctifie\FTNT{Ex. 20:8~; Ex. 31:13.}.
\VS{13}Mais ceux de la maison d'Israël se rebellèrent contre moi dans le désert. Ils ne suivirent point mes lois, et ils rejetèrent mes ordonnances que l'homme doit mettre en pratique, afin de vivre par elles, et ils profanèrent à l'excès mes sabbats. C'est pourquoi je dis que je répandrais sur eux ma fureur dans le désert pour les consumer\FTNT{Ex. 16:28.}.
\VS{14}Je l'ai fait pour l'amour de mon Nom, afin qu'il ne soit point profané devant les nations, en présence desquelles je les avais fait sortir d'Egypte\FTNT{Ex. 32:12~; No. 14:13-14~; De. 9:28~; Jos. 7:9.}.
\VS{15}Je levai ma main vers eux dans le désert pour ne pas les amener dans le pays que je leur avais donné, pays où coulent le lait et le miel, et qui est la noblesse de tous les pays,
\VS{16}parce qu'ils ont rejeté mes ordonnances, qu'ils n'ont point suivi mes lois, et qu'ils ont profané mes sabbats, car leur cœur ne s'est pas éloigné de leurs idoles.
\VS{17}Toutefois mon oeil les épargna pour ne pas les détruire, et je ne les consumai point entièrement dans le désert.
\VS{18}Mais je dis à leurs fils dans le désert~: Ne marchez point dans les statuts de vos pères, et ne gardez point leurs ordonnances, et ne vous souillez point par leurs idoles.
\VS{19}Je suis Yahweh, votre Dieu. Marchez dans mes statuts, et gardez mes ordonnances et accomplissez-les.
\VS{20}Sanctifiez mes sabbats, et ils seront un signe entre moi et vous, afin que vous reconnaissiez que je suis Yahweh, votre Dieu.
\VS{21}Mais les fils se rebellèrent aussi contre moi, et ils ne marchèrent point dans mes statuts, et ne gardèrent point mes ordonnances pour les faire~; ce que l'homme doit accomplir, pour vivre par elles. Ils profanèrent mes sabbats~; c'est pourquoi je dis que je répandrais ma fureur sur eux, et que je consumerais ma colère sur eux dans le désert.
\VS{22}Toutefois, je retirai ma main, et je le fis pour l'amour de mon Nom, afin qu'il ne soit point profané devant les nations, en présence desquelles je les avais sortis d'Egypte.
\VS{23}Néanmoins, je levai ma main vers eux dans le désert, pour les répandre parmi les nations, et les disperser dans les pays\FTNT{Lé. 26:13-33.},
\VS{24}parce qu'ils n'ont point accompli mes ordonnances, qu'ils ont rejeté mes statuts et profané mes sabbats, et que leurs yeux se sont attachés aux idoles de leurs pères.
\VS{25}A cause de cela, je leur donnai des statuts qui n'étaient pas bons, et des ordonnances par lesquelles ils ne vivraient point.
\VS{26}Je les souillai par leurs dons, quand ils firent passer par le feu tous les premiers-nés, afin de les punir, et que l'on sache que je suis Yahweh.
\VS{27}C'est pourquoi, toi fils de l'homme, parle à la maison d'Israël, et dis-leur~: Ainsi parle le Seigneur Yahweh~: Vos pères m'ont encore outragé, car ils ont commis un crime contre moi.
\VS{28}Je les ai conduits dans le pays que j'avais juré de leur donner, et ils ont regardé toute haute colline, et tout arbre touffu, ils y ont fait leurs sacrifices, ils y ont posé leur oblation pour m'irriter, ils y ont mis leurs parfums, et ils y ont répandu leurs libations.
\VS{29}Je leur ai dit~: Que veulent dire ces hauts lieux où vous allez~? Et le nom de hauts lieux leur a été donné jusqu'à ce jour.
\VS{30}C'est pourquoi dis à la maison d'Israël~: Ainsi parle le Seigneur Yahweh~: Ne vous souillez-vous pas selon les voies de vos pères, et ne vous prostituez-vous point à leurs idoles abominables,
\VS{31}en offrant vos dons, en faisant passer vos fils par le feu, en vous souillant par toutes vos idoles jusqu'à ce jour~? Est-ce ainsi que vous me consultez, ô maison d'Israël~? Je suis vivant, dit le Seigneur Yahweh, vous ne me consultez pas.
\VS{32}Ce que vous pensez n'arrivera nullement, quand vous dites~: Nous serons comme les nations, et comme les familles des pays, en servant le bois et la pierre.
\TextTitle{Restauration future d'Israël}
\VS{33}Je suis vivant~! Dit le Seigneur Yahweh. Je règnerai sur vous avec une main forte, et un bras étendu, et avec effusion de colère.
\VS{34}Je vous sortirai du milieu des peuples, et vous rassemblerai hors des pays dans lesquels vous êtes dispersés, avec une main forte, et un bras étendu et avec effusion de colère.
\VS{35}Je vous ferai venir dans le désert des peuples, et je contesterai là contre vous, face à face,
\VS{36}comme j'ai contesté contre vos pères dans le désert du pays d'Egypte, ainsi je contesterai contre vous, dit le Seigneur Yahweh.
\VS{37}Je vous ferai passer sous la verge, et vous ramènerai au lieu de l'alliance\FTNT{Es. 65:12.}.
\VS{38}Je séparerai de vous les rebelles, et ceux qui se révoltent contre moi~; je les ferai sortir du pays dans lequel ils séjournent, mais ils n'entreront point dans la terre d'Israël~; et vous saurez que je suis Yahweh.
\VS{39}Vous donc, ô maison d'Israël, ainsi parle le Seigneur Yahweh~: Allez, servez chacun vos idoles, puisque vous ne voulez pas m'écouter~! Ainsi vous ne profanerez plus mon saint Nom par vos dons et par vos idoles.
\VS{40}Mais ce sera sur ma sainte montagne, sur la haute montagne d'Israël, dit le Seigneur Yahweh, que toute la maison d'Israël me servira, dans le pays\FTNT{Jn. 4:21-24.}. Là, je prendrai plaisir en eux, et là je demanderai vos offrandes et les prémices de vos dons, et tout ce que vous me consacrerez.
\VS{41}Je prendrai plaisir en vous par vos parfums d'une agréable odeur, quand je vous aurai fait sortir du milieu des peuples, et que je vous aurai rassemblés des pays dans lesquels vous êtes dispersés~; je serai sanctifié par vous, aux yeux des nations.
\VS{42}Vous saurez que je suis Yahweh, quand je vous aurai fait revenir dans le pays d'Israël, dans le pays où j'ai levé ma main pour le donner à vos pères.
\VS{43}Et là, vous vous souviendrez de vos voies, et de toutes vos actions, par lesquelles vous vous êtes souillés~; et vous vous prendrez vous-mêmes en dégoût à cause de tous vos maux que vous aurez faits.
\VS{44}Vous saurez que je suis Yahweh, par tout ce que j'aurai fait pour vous, à cause de mon Nom, et non pas selon vos méchantes voies et vos actions corrompues, ô maison d'Israël~! dit le Seigneur Yahweh.
\Chap{21}
\TextTitle{L'épée de Yahweh}
\VerseOne{}La parole de Yahweh vint encore à moi en disant~:
\VS{2}Fils de l'homme, tourne ta face vers Jérusalem, parle en direction du sud, et prophétise contre la forêt du champ du sud~!
\VS{3}Dis à la forêt du sud~: Ecoute la parole de Yahweh. Ainsi parle le Seigneur Yahweh~: Voici, je m'en vais allumer au-dedans de toi un feu qui consumera tout bois vert et tout bois sec au-dedans de toi~; la flamme de l'embrasement ne s'éteindra point, et tout le dessus en sera brûlé, depuis le sud jusqu'au nord\FTNT{Jé. 21:14~; Jé. 22:7~; Jé. 46:23~; Lu. 23:31.}.
\VS{4}Toute chair verra que moi, Yahweh, j'ai allumé le feu~; et il ne s'éteindra point.
\VS{5}Je dis~: Ah~! Seigneur Yahweh, ils disent de moi~: N'est-il pas vrai que celui-ci ne fait que mettre en avant des paraboles~?
\TextTitle{Parabole de l'épée de Yahweh}
\VS{6}La parole de Yahweh me fut adressée en ces mots~:
\VS{7}Fils de l'homme, tourne ta face vers Jérusalem, et parle en direction du lieu saint, et prophétise contre la terre d'Israël.
\VS{8}Dis à la terre d'Israël~: Ainsi parle Yahweh~: Voici, j'en veux à toi, je tirerai mon épée de son fourreau, et je retrancherai du milieu de toi le juste et le méchant.
\VS{9}Parce que je retrancherai du milieu de toi le juste et le méchant, à cause de cela mon épée sortira de son fourreau contre toute chair, depuis le sud jusqu'au nord.
\VS{10}Toute chair saura que moi, Yahweh, j'ai tiré mon épée de son fourreau, et elle n'y retournera plus.
\VS{11}Aussi, toi, fils de l'homme, gémis en te rompant les reins de douleur, et soupire avec amertume dans leur présence.
\VS{12}Quand ils te diront~: Pourquoi gémis-tu~? Alors tu répondras~: C'est à cause d'une nouvelle, car elle vient, et tout cœur se fondra, et toutes les mains seront baissées, tout esprit sera affaibli, et tous les genoux se fondront en eau~; voici, elle vient, elle arrive, dit le Seigneur Yahweh\FTNT{Jé. 6:24~; Jé. 49:23.}.
\VS{13}Puis la parole de Yahweh me fut adressée en ces mots~:
\VS{14}Fils de l'homme, prophétise, et dis~: Ainsi parle Yahweh~: Dis~: L'épée, l'épée a été aiguisée, elle est polie~!
\VS{15}Elle a été aiguisée pour faire un grand carnage, elle a été polie afin qu'elle brille: Nous réjouirons-nous~? C'est la verge de mon fils, elle dédaigne tout bois.
\VS{16}Yahweh l'a donnée à polir, afin qu'on la tienne à la main~; l'épée a été aiguisée, et elle a été polie pour la mettre dans la main du destructeur.
\VS{17}Crie et hurle, fils de l'homme~! Car elle est contre mon peuple, elle est contre tous les princes d'Israël~; ils sont livrés à l'épée à cause de mon peuple. C'est pourquoi frappe sur ta cuisse~!
\VS{18}Quand ce serait une épreuve, et que serait-ce~? Si même cette épée qui dédaigne tout bois, était une verge, il n'en serait rien, dit le Seigneur Yahweh. 
\VS{19}Toi donc, fils de l'homme, prophétise, et frappe d'une main contre l'autre, que les coups de l'épée soient doublés, soient triplés, c'est l'épée du carnage, l'épée du grand carnage, l'épée qui doit les poursuivre.
\VS{20}J'ai mis à toutes leurs portes l'épée étincelante, afin que le cœur se fonde, et que les ruines soient multipliées. Ah~! Elle est faite pour briller et réservée pour tuer.
\VS{21}Joins-toi épée, frappe à la droite~! Avance-toi, frappe à la gauche, à tous côtés que tu rencontres~!
\VS{22}Je frapperai aussi d'une main contre l'autre, je donnerai du repos à ma colère. Moi, Yahweh, j'ai parlé.
\VS{23}La parole de Yahweh me fut adressée en ces mots~:
\VS{24}Toi, fils de l'homme, pose deux chemins où l'épée du roi de Babylone pourrait venir~; que les deux chemins sortent d'un même pays, et forme-les, forme-les de ta main à l'endroit où commence le chemin de la ville.
\VS{25}Tu poseras le chemin par lequel l'épée doit venir contre Rabbath des fils d'Ammon, et le chemin qui va en Judée, et à Jérusalem, ville forte.
\VS{26}Car le roi de Babylone se tient au carrefour, à l'entrée des deux chemins, pour consulter les devins~; il aiguise les flèches, il interroge les théraphim, il examine le foie.
\VS{27}Dans sa main droite est la divination contre Jérusalem, pour y dresser des béliers, pour publier le carnage, pour pousser des cris de guerre, pour ranger les béliers contre les portes, pour élever des terrasses et construire des remparts.
\VS{28}Mais ce sera pour eux, à leurs yeux, une divination vaine~; il y a de grands serments entre eux. Mais lui, il se souvient de leur iniquité, en sorte qu'ils seront pris.
\VS{29}C'est pourquoi ainsi parle le Seigneur Yahweh~: Parce que vous avez fait revenir le souvenir de votre iniquité, lorsque vos crimes se sont découverts, au point de voir vos péchés dans toutes vos actions~; parce que vous avez fait qu'on se souvienne de vous, vous serez saisis par sa main.
\TextTitle{Quand l'iniquité arrive à son terme\FTNTT{Ap. 19:11-20:6.}}
\VS{30}Et toi, profane, méchant, prince d'Israël, dont le jour arrive au temps où l'iniquité est à son terme~!
\VS{31}Ainsi parle le Seigneur Yahweh~: Qu'on ôte cette tiare, et qu'on enlève cette couronne. Ce ne sera plus celle-ci~; j'élèverai ce qui est bas, et j'abaisserai ce qui est haut\FTNT{Job 5:11~; 1 Co. 1:27.}.
\VS{32}J'en ferai une ruine, une ruine, une ruine, et elle ne sera plus. Mais cela n'aura lieu qu'à la venue de celui à qui appartient le jugement et à qui je le donnerai.
\VS{33}Toi, fils de l'homme, prophétise, et dis~: Ainsi parle le Seigneur Yahweh, au sujet des fils d'Ammon, et de leur opprobre~: Dis donc, épée, épée dégainée, polie pour le massacre, pour dévorer avec son éclat~!
\VS{34}Pendant qu'on voit pour toi des visions de vanité, et qu'on devine pour toi le mensonge, afin qu'on te mette sur le cou des méchants qui sont mis à mort, dont le jour est venu, lorsque leur iniquité aura une fin.
\VS{35}La remettrait-on dans son fourreau~? Je te jugerai sur le lieu où tu as été créé, au pays de ta naissance.
\VS{36}Je répandrai ma colère sur toi, j'allumerai sur toi le feu de ma fureur, et je te livrerai entre les mains d'hommes brutaux, qui ne travaillent qu'à détruire\FTNT{Jé. 25:11~; Jé. 52:30.}.
\VS{37}Tu seras destiné au feu pour être dévoré~; ton sang sera au milieu de la terre~: On ne se souviendra plus de toi, car c'est moi, Yahweh, qui parle.
\Chap{22}
\TextTitle{Les péchés d'Israël}
\VerseOne{}La parole de Yahweh vint encore à moi en disant~:
\VS{2}Et toi, fils de l'homme, ne jugeras-tu pas, ne jugeras-tu pas la ville sanguinaire, et ne lui donneras-tu pas à connaître toutes ses abominations\FTNT{Ez. 24:6-9~;  Na. 3:1-4~; Ha. 1:13.}~?
\VS{3}Tu diras donc, ainsi parle le Seigneur Yahweh~: Ville qui répands le sang au milieu de toi, afin que ton temps vienne, et qui te fais des idoles à ton préjudice, pour en être souillée.
\VS{4}Tu t'es rendue coupable par ton sang que tu as répandu, et tu t'es souillée par tes idoles que tu as faites~; tu as fait approcher tes jours, et tu es venue au terme de tes années~; c'est pourquoi je t'ai exposée en opprobre aux nations, et en dérision dans tous les pays\FTNT{2 R. 21:16~; Jé. 26:21-23.}.
\VS{5}Celles qui sont près de toi, et celles qui en sont loin, se moqueront de toi, infâme de réputation, et remplie de troubles.
\VS{6}Voici, les princes d'Israël ont contribué au dedans de toi, chacun selon sa force, à répandre le sang.
\VS{7}Au dedans de toi, on méprise père et mère, on use de tromperie à l'égard de l'étranger, on opprime l'orphelin et la veuve.
\VS{8}Tu méprises ma sainteté, et profanes mes sabbats.
\VS{9}Des gens médisants sont au milieu de toi pour répandre le sang, ceux qui sont chez toi mangent sur les montagnes~; on commet des actions énormes au milieu de toi\FTNT{Es. 57:7~; Jé. 2:20.}.
\VS{10}L'enfant découvre la nudité du père au milieu de toi, et on humilie au milieu de toi la femme dans le temps de son impureté\FTNT{Lé. 18:6-9~; Ge. 9:22-23.}.
\VS{11}L'un commet l'abomination avec la femme de son prochain~; et l'autre se souille par l'inceste avec sa belle-fille~; chacun humilie sa sœur, fille de son père\FTNT{Ge. 19:32-36~; Lé. 18:15-20~; Jé. 5:8.}.
\VS{12}Chez toi, on reçoit des présents pour répandre le sang~; tu exiges un intérêt et une usure, tu dépouilles ton prochain par l'extorsion, et tu m'oublies, dit le Seigneur Yahweh\FTNT{Ex. 23:8~; De. 27:25.}.
\VS{13}Voici, je frappe de mes mains l'une contre l'autre à cause de ton gain déshonnête que tu fais, et à cause de ton sang qui se répand au milieu de toi.
\VS{14}Ton cœur pourra-t-il tenir ferme, tes mains seront-elles fortes dans les jours où j'agirai contre toi~? Moi, Yahweh, j'ai parlé, et je le ferai.
\VS{15}Je te disperserai parmi les nations, je t'éparpillerai en divers pays, et je consumerai ta souillure, jusqu'à ce qu'il n'y en ait plus en toi.
\VS{16}Tu seras souillée par toi-même aux yeux des nations, et tu sauras que je suis Yahweh.
\TextTitle{La fureur de Yahweh}
\VS{17}Puis la parole de Yahweh me fut adressée en ces mots~:
\VS{18}Fils de l'homme, la maison d'Israël m'est devenue comme de l'écume~; eux tous sont de l'airain, de l'étain, du fer et du plomb dans un creuset~; ils sont devenus comme une écume d'argent.
\VS{19}C'est pourquoi ainsi parle le Seigneur Yahweh~: Parce que vous êtes tous devenus comme de l'écume, voici, je vais à cause de cela vous rassembler au milieu de Jérusalem,
\VS{20}comme on assemble de l'argent, de l'airain, du fer, du plomb, et de l'étain dans un creuset, afin d'y souffler le feu pour les fondre~; je vous rassemblerai ainsi dans ma colère et dans ma fureur, et je vous fondrai.
\VS{21}Je vous assemblerai, je soufflerai contre vous le feu de ma fureur, et vous serez fondus au milieu de Jérusalem.
\VS{22}Comme l'argent se fond dans le creuset, ainsi vous serez fondus au milieu d'elle, et vous saurez que moi, Yahweh, j'ai répandu ma fureur sur vous.
\TextTitle{Se tenir à la brèche devant Yahweh}
\VS{23}La parole de Yahweh vint encore à moi en disant~:
\VS{24}Fils de l'homme, dis-lui~: Tu es une terre qui n'est pas purifiée ni arrosée de pluie au jour de la colère.
\VS{25}Il y a un complot de ses prophètes au milieu d'elle~; ils seront comme des lions rugissants, qui ravissent la proie~: Ils dévorent les âmes, ils emportent les richesses et la gloire, ils multiplient les veuves au milieu d'elle\FTNT{Mt. 23:13~; 1 Pi. 5:8.}.
\VS{26}Ses prêtres ont fait violence à ma loi et ont profané mes choses saintes~; ils ne font pas de différence entre la chose sainte et profane~; ils ne donnent pas à connaître la diffrence qu'il y a entre la chose impure et la pure, et ils cachent leurs yeux de mes sabbats, et je suis profané au milieu d'eux.
\VS{27}Ses princes sont au milieu d'elle comme des loups qui ravissent la proie, pour répandre le sang et pour détruire les âmes, pour s'adonner au gain déshonnête\FTNT{Mi. 3:11~; Mt. 10:16~; 2 Pi. 2:16.}.
\VS{28}Ses prophètes ont pour eux des enduits de plâtre, des visions fausses, et des oracles menteurs, en disant~: Ainsi parle le Seigneur Yahweh~; et cependant Yahweh n'a point parlé.
\VS{29}Le peuple du pays use de tromperies, ravit le bien d'autrui, opprime l'affligé et le pauvre, et foule l'étranger contre tout droit.
\VS{30}Et j'ai cherché quelqu'un\FTNT{Dieu n'a pas besoin d'une foule de gens avant d'agir. Une seule personne suffit.} d'entre eux qui relèverait la clôture, et qui se tiendrait à la brèche devant moi pour le pays, afin que je ne le détruise pas~; mais je n'en ai pas trouvé. 
\VS{31}C'est pourquoi je répandrai sur eux ma colère, et je les consumerai par le feu de ma fureur~; je mettrai leur voie sur leur tête, dit le Seigneur Yahweh.
\Chap{23}
\TextTitle{Prostitutions d'Israël et de Juda}
\VerseOne{}La parole de Yahweh vint encore à moi en disant~:
\VS{2}Fils de l'homme, il y a eu deux femmes, filles d'une même mère,
\VS{3}qui se sont prostituées en Egypte, elles se sont prostituées dans leur jeunesse. Là, leur sein fut déshonoré et leur virginité touchée.
\VS{4}Et c'était ici leurs noms, celui de la plus grande était Ohola, et celui de sa sœur Oholiba\FTNT{Ohola signifie «~sa propre tente~», et Oholiba «~la femme de la tente~» (2 R. 17:23-24).}~; elles étaient à moi, et elles ont enfanté des fils et des filles~; leurs noms donc étaient Ohola, qui était Samarie, et Oholiba, qui est Jérusalem.
\VS{5}Or Ohola a commis l'adultère étant ma femme, et s'est rendue amoureuse de ses amoureux, c'est-à-dire des Assyriens ses voisins,
\VS{6}vêtus de pourpre, gouverneurs et magistrats, tous jeunes et aimables, tous cavaliers, montés sur des chevaux.
\VS{7}Elle a commis ses adultères avec toute l'élite des fils des Assyriens, et avec tous ceux pour qui elle s'était enflammée, et s'est souillée avec toutes leurs idoles.
\VS{8}Elle n'a pas abandonné ses fornications d'Egypte, car ils avaient couché avec elle dans sa jeunesse, ils avaient déshonoré sa virginité et s'étaient livrés à l'impureté avec elle\FTNT{Ac. 7:42.}.
\VS{9}C'est pourquoi je l'ai livrée entre les mains de ses amoureux, entre les mains des fils des Assyriens, dont elle s'était rendue amoureuse.
\VS{10}Ils l'ont couverte d'opprobre, ils ont enlevé ses fils et ses filles, et l'ont tuée elle-même avec l'épée~; elle a été en renom parmi les femmes, après avoir exercé des jugements sur elle.
\VS{11}Quand sa sœur Oholiba a vu cela, elle fut plus déréglée qu'elle dans ses passions~; ses prostitutions dépassèrent celles de sa sœur.
\VS{12}Elle s'enflamma pour les fils des Assyriens, des gouverneurs et des magistrats, ses voisins, vêtus magnifiquement, et des cavaliers montés sur des chevaux, tous jeunes et bien faits.
\VS{13}J'ai vu qu'elle s'était souillée, et que l'une et l'autre avaient suivi la même voie.
\VS{14}Et encore a-t-elle augmenté ses prostitutions~; car ayant vu les peintures d'hommes sur la muraille, les images de Chaldéens peints en rouge,
\VS{15}ceints de ceintures sur leurs reins, ayant des turbans teints de couleurs variées flottant sur leurs têtes, eux tous ayant l'apparence de grands seigneurs, et la ressemblance des enfants de Babylone en Chaldée, terre de leur naissance.
\VS{16}Elle en est tombée amoureuse dès qu'elle les vit de ses yeux, et envoya des messagers vers eux au pays des Chaldéens. 
\VS{17}Les fils de Babylone vinrent vers elle au lit de ses prostitutions, et la souillèrent par leurs adultères~; elle s'est aussi souillée avec eux, et après cela son cœur s'est détaché d'eux.
\VS{18}Elle a manifesté ses fornications et fait connaître son opprobre~; et mon cœur s'est détaché d'elle, comme mon cœur s'était détaché de sa sœur.
\VS{19}Car elle a multiplié ses adultères, jusqu'à rappeler le souvenir des jours de sa jeunesse, lorsqu'elle s'était abandonnée au pays d'Egypte.
\VS{20}Elle s'est enflammée pour des impudiques, dont la chair était comme celle des ânes, et dont la force égale celle des chevaux.
\VS{21}Tu as donc repris les méchancetés de ta jeunesse, lorsque tu as été déshonorée, depuis que tu étais en Egypte, à cause du sein de ta jeunesse.
\VS{22}C'est pourquoi, Oholiba, ainsi parle le Seigneur Yahweh~: Voici, je m'en vais réveiller contre toi tous tes amants, ceux dont ton cœur s'est détaché, et je les amènerai contre toi de toutes parts.
\VS{23}Les fils de Babylone, et tous les Chaldéens, Pekod, Shoa, Koa, et tous les Assyriens avec eux, tous jeunes gens d'élite, gouverneurs et magistrats, grands seigneurs et renommés, tous montant à cheval.
\VS{24}Ils viendront contre toi avec des armes, des chars, et des roues, avec une multitude de peuples, avec le grand bouclier et le petit bouclier, avec les casques, et je leur mettrai le jugement en main, ils te jugeront selon leur jugement.
\VS{25}Je mettrai ma jalousie contre toi, et ils agiront contre toi avec fureur. Ils te retrancheront le nez et les oreilles, et ce qui restera de toi tombera par l'épée. Ils enlèveront tes fils et tes filles, et ce qui restera de toi sera dévoré par le feu.
\VS{26}Ils te dépouilleront de tes vêtements, et t'enlèveront les ornements dont tu te pares.
\VS{27}Je mettrai un terme à tes méchancetés et tes prostitutions du pays d'Egypte~; tu ne lèveras plus tes yeux vers eux, et tu ne te souviendras plus de l'Egypte.
\VS{28}Car ainsi parle le Seigneur Yahweh~: Voici, je te livre entre les mains de ceux que tu hais, entre les mains de ceux dont ton cœur s'est détaché.
\VS{29}Ils te traiteront avec haine~; ils enlèveront tout ton travail, et te laisseront sans habits et découverte~; et la turpitude de tes adultères, de ton énormité, et de tes fornications, sera découverte.
\VS{30}On te fera ces choses-là parce que tu t'es prostituée aux nations, avec lesquelles tu t'es souillée par leurs idoles.
\VS{31}Tu as marché dans la voie de ta sœur, c'est pourquoi je mets sa coupe dans ta main.
\VS{32}Ainsi parle le Seigneur Yahweh~: Tu boiras la coupe profonde et large de ta sœur~; elle sera une coupe d'une grande mesure~; tu seras un sujet de risée et de moquerie\FTNT{Ps. 75:9~; Es. 51:17~; Jé. 25:15.}.
\VS{33}Tu seras remplie d'ivresse et de douleur, par la coupe de désolation et de dégât, qui est la coupe de ta sœur Samarie.
\VS{34}Tu la boiras et la videras, tu briseras ce pot de terre et tu déchireras ton sein. Car j'ai parlé, dit le Seigneur Yahweh.
\VS{35}C'est pourquoi ainsi parle le Seigneur Yahweh~: Parce que tu m'as oublié, et que tu m'as jeté derrière ton dos, aussi porteras-tu la peine de ta méchanceté, et de tes prostitutions.
\TextTitle{Jugement sur Israël et Juda}
\VS{36}Puis Yahweh me dit~: Fils de l'homme, ne jugeras-tu pas Ohola et Oholiba~? Déclare-leur donc leurs abominations.
\VS{37}Déclare-leur comment elles ont commis l'adultère et comment il y a du sang dans leurs mains~; comment, dis-je, elles ont commis l'adultère avec leurs idoles, et ont même fait passer par le feu leurs fils pour les consumer, ces enfants qu'elles m'avaient enfantés.
\VS{38}Voici encore ce qu'elles m'ont fait~: Elles ont souillé mon lieu saint ce même jour, et ont profané mes sabbats.
\VS{39}Car après avoir égorgé leurs fils à leurs idoles, elles sont entrées ce même jour-là dans mon lieu saint pour le profaner~; et voilà, comment elles ont fait au milieu de ma maison\FTNT{2 R. 21:4.}.
\VS{40}Et qui plus est, elles ont fait chercher des hommes venant de loin, elles leur ont envoyé des messagers, et voici, ils sont venus. Pour eux tu t'es lavée, tu as fardé ton visage, et t'es parée d'ornements.
\VS{41}Tu t'es assise sur un lit magnifique, devant lequel a été apprêtée une table, sur laquelle tu as mis mon encens et mon huile.
\VS{42}On entendait le bruit d'une multitude tranquille~; et parmi cette foule d'hommes, on a fait venir du désert des Sabéens, qui ont mis des bracelets aux mains des deux sœurs, et de superbes couronnes sur leurs têtes.
\VS{43}J'ai dit au sujet de celle qui avait vieilli dans l'adultère~: Maintenant ses impudicités prendront fin, et elle aussi.
\VS{44}Toutefois on est venu vers elle comme on vient vers une femme prostituée~; ils sont ainsi venus vers Ohola et vers Oholiba, femmes pleines de méchanceté.
\VS{45}Les hommes justes donc les jugeront comme on juge les femmes adultères, et comme on juge celles qui répandent le sang~; car elles sont adultères, et le sang est dans leurs mains.
\VS{46}C'est pourquoi ainsi parle le Seigneur Yahweh~: Qu'on fasse monter l'assemblée contre elles, et qu'elles soient abandonnées au tumulte et au pillage.
\VS{47}Que l'assemblée les lapide de pierres et les taille en pièces avec leurs épées~; qu'ils tuent leurs fils et leurs filles, et qu'ils brûlent au feu leurs maisons.
\VS{48}Et ainsi je ferai cesser la méchanceté dans le pays, et toutes les femmes seront enseignées à ne point faire selon votre méchanceté.
\VS{49}On mettra votre méchanceté sur vous, et vous porterez les péchés de vos idoles~; et vous saurez que je suis le Seigneur Yahweh.
\Chap{24}
\TextTitle{Malheur à la ville sanguinaire}
\VerseOne{}La neuvième année, au dixième jour du dixième mois, la parole de Yahweh me fut adressée en ces mots~:
\VS{2}Fils de l'homme, mets par écrit la date de ce jour, de ce jour-ci~! Car en ce même jour le roi de Babylone s'approche contre Jérusalem\FTNT{2 R. 25:1.}.
\VS{3}Propose une parabole à la famille de rebelles, et dis-leur~: Ainsi parle le Seigneur Yahweh~: Mets, mets la chaudière, et verse de l'eau dedans.
\VS{4}Mets-y les morceaux, tous les bons morceaux, la cuisse, et l'épaule, et remplis-la des meilleurs os.
\VS{5}Prends la meilleure bête du troupeau, et fais brûler des os sous la chaudière, fais-la bouillir à gros bouillons, et que les os cuisent au-dedans.
\VS{6}Car ainsi parle le Seigneur Yahweh~: Malheur à la ville sanguinaire, à la chaudière pleine de rouille, et de laquelle la rouille n'est point sortie~! Vide-la morceau par morceau, et que le sort ne soit point jeté sur elle.
\VS{7}Parce que son sang est au milieu d'elle, qu'elle l'a mis sur le rocher brillant, et qu'elle ne l'a point répandu sur la terre pour le couvrir de poussière,
\VS{8}j'ai mis son sang sur un rocher brillant, afin qu'il ne soit point couvert, pour faire monter la fureur, et pour me venger.
\VS{9}C'est pourquoi ainsi parle le Seigneur Yahweh~: Malheur à la ville sanguinaire~! J'en ferai aussi un grand tas de bois à brûler~!
\VS{10}Amasse beaucoup de bois, allume le feu, fais cuire la chair entièrement, et fais-la consumer, et que les os soient brûlés.
\VS{11}Puis mets sur les charbons ardents la chaudière toute vide, afin qu'elle s'échauffe et que son airain se brûle, et que sa souillure soit fondue à l'intérieur, et que sa rouille soit consumée.
\VS{12}Les efforts sont inutiles, sa rouille dont elle est pleine n'est point sortie d'elle~; sa rouille ne s'en ira que par le feu.
\VS{13}L'impureté est dans ta souillure~; car je t'avais purifiée, et tu n'as point été pure~; tu ne seras pas encore nettoyée de ta souillure, jusqu'à ce que j'aie assouvi sur toi ma fureur.
\VS{14}Moi, Yahweh, j'ai parlé, cela arrivera, et je le ferai~; et je ne me retirerai point en arrière, je n'épargnerai point, et je ne serai point apaisé. On t'a jugée selon tes voies et selon tes actions, dit le Seigneur Yahweh.
\TextTitle{La vie d'Ezéchiel, un signe pour Israël}
\VS{15}La parole de Yahweh me fut adressée en ces mots~:
\VS{16}Fils de l'homme, voici, je vais t'ôter par une plaie ce que tes yeux voient avec le plus de plaisir. Ne mène point de deuil, ne pleure point, ne fais point couler tes larmes\FTNT{Jé. 16:6-7.}.
\VS{17}Garde-toi de gémir, et ne fais pas le deuil des morts~; attache ton turban sur ta tête, mets tes souliers à tes pieds, ne te couvre pas la barbe, et ne mange pas le pain des autres\FTNT{Lé. 10:6.}.
\VS{18}J'avais parlé au peuple le matin, et ma femme mourut le soir~; le lendemain matin je fis comme il m'avait été ordonné.
\VS{19}Le peuple me dit~: Ne nous déclareras-tu point ce que signifient ces choses-là que tu fais~?
\VS{20}Je leur répondis~: La parole de Yahweh m'a été adressée en ces mots~:
\VS{21}Parle à la maison d'Israël~: Ainsi parle le Seigneur Yahweh~: Voici, je m'en vais profaner mon lieu saint, la magnificence de votre force, ce qui est le plus agréable à vos yeux, ce que vous voudriez épargner sur toutes choses~; et vos fils et vos filles, que vous aurez laissés, tomberont par l'épée.
\VS{22}Vous ferez alors comme j'ai fait~; vous ne couvrirez point vos barbes, et vous ne mangerez point le pain des autres.
\VS{23}Vos turbans seront sur vos têtes, et vos souliers à vos pieds~; vous ne mènerez point de deuil ni ne pleurerez~; mais vous pourrirez à cause de vos iniquités, et vous gémirez les uns avec les autres. Ezéchiel sera pour vous un signe~; 
\VS{24}vous ferez selon toutes les choses qu'il a faites~; et quand cela sera arrivé, vous saurez que je suis le Seigneur Yahweh.
\VS{25}Quant à toi, fils de l'homme, au jour que je leur ôterai leur force, la joie de leur ornement, l'objet le plus agréable à leurs yeux, et l'objet de leurs cœurs, leurs fils et leurs filles,
\VS{26}ce jour-là un fuyard ne viendra-t-il pas vers toi pour te le raconter~?
\VS{27}En ce jour-là ta bouche sera ouverte envers celui qui sera échappé, et tu parleras et ne seras plus muet~; ainsi tu seras pour eux un signe, et ils sauront que je suis Yahweh.
\Chap{25}
\TextTitle{Jugement de Dieu sur Ammon}
\VerseOne{}Puis la parole de Yahweh vint à moi en disant~:
\VS{2}Fils de l'homme, tourne ta face vers les fils d'Ammon, et prophétise contre eux\FTNT{Jé. 49:1.}.
\VS{3}Dis aux fils d'Ammon~: Ecoutez la parole du Seigneur Yahweh~: Parce que vous avez dit~: Ah, ah, contre mon lieu saint, parce qu'il était profané~; et contre la terre d'Israël, parce qu'elle était désolée~; et contre la maison de Juda, parce qu'ils allaient en captivité\FTNT{Am. 1:13~; So. 2:8.}~;
\VS{4}à cause de cela, voici, je m'en vais te donner en héritage aux fils d'orient, et ils bâtiront des palais dans tes villes, et ils demeureront chez toi~; ils mangeront tes fruits et boiront ton lait.
\VS{5}Je livrerai Rabba pour être le repaire des chameaux, et le pays des fils d'Ammon pour être le gîte des brebis, et vous saurez que je suis Yahweh.
\VS{6}Car ainsi parle le Seigneur Yahweh~: Parce que tu as frappé des mains, que tu as battu des pieds, et que tu t'es réjoui de bon cœur avec tout le mépris que tu as eu pour la terre d'Israël,
\VS{7}à cause de cela, voici, j'étends ma main sur toi, et je te livre pour être pillée par les nations, et je te retranche du milieu des peuples, je te fais périr d'entre les pays~; je te détruis~; et tu sauras que je suis Yahweh.
\TextTitle{Jugement sur Moab}
\VS{8}Ainsi parle le Seigneur Yahweh~: Parce que Moab et Séir ont dit~: Voici, la maison de Juda est comme toutes les autres nations~;
\VS{9}à cause de cela voici, j'ouvre le territoire de Moab du côté des villes, de ses villes frontières, la beauté du pays de Beth-Jeschimoth, de Baal-Meon et de Kirjathaïm\FTNT{No. 32:38~; Jos. 12:3.},
\VS{10}je l'ouvre aux fils d'orient, qui sont au-delà du pays des fils d'Ammon, je leur donne en possession, afin qu'on ne se souvienne plus des fils d'Ammon parmi les nations.
\VS{11}J'exercerai aussi des jugements contre Moab, et ils sauront que je suis Yahweh.
\TextTitle{Jugement sur Edom}
\VS{12}Ainsi parle le Seigneur Yahweh~: A cause de ce qu'Edom a fait quand il s'est vengé de la maison de Juda, et parce qu'il s'en est rendu coupable en se vengeant d'eux\FTNT{Ps. 137:7.},
\VS{13}à cause de cela, le Seigneur Yahweh dit~: J'étends ma main sur Edom, j'en retranche les hommes et les bêtes, et j'en fais un désert~; depuis Théman à Dedan ils tomberont par l'épée\FTNT{Jé. 49:7-9~; Am. 1:12~; Ab. 1:9.}.
\VS{14}J'exercerai ma vengeance sur Edom à cause de mon peuple d'Israël, et on traitera Edom selon ma colère, et selon ma fureur, et ils reconnaîtront ma vengeance, dit le Seigneur Yahweh.
\TextTitle{Jugement sur les Philistins}
\VS{15}Ainsi parle le Seigneur Yahweh~: Puisque les Philistins ont agi par vengeance, et qu'ils se sont vengés avec mépris et du fond de leur âme, voulant tout détruire dans leur haine éternelle~;
\VS{16}à cause de cela le Seigneur Yahweh dit~: Voici, je m'en vais étendre ma main sur les Philistins, j'exterminerai les Kéréthiens, et je ferai périr le reste sur le rivage de la mer.
\VS{17}J'exercerai sur eux de grandes vengeances par des châtiments de fureur~; et ils sauront que je suis Yahweh, quand j'aurai exécuté sur eux ma vengeance\FTNT{Es. 14:29~; Jé. 25:20~; So. 2:7.}.
\Chap{26}
\TextTitle{Jugement sur Tyr}
\VerseOne{}Il arriva dans la onzième année, le premier jour du mois, que la parole de Yahweh vint à moi en disant~:
\VS{2}Fils de l'homme, parce que Tyr a dit au sujet de Jérusalem~: Ah~! Ah~! Celle qui était la porte des peuples a été rompue, elle s'est réfugiée chez moi, je serai remplie parce qu'elle a été rendue déserte\FTNT{Am. 1:9~; Za. 9:2-3.}~!
\VS{3}A cause de cela, ainsi parle le Seigneur Yahweh~: Voici, j'en veux à toi, Tyr, et je ferai monter contre toi plusieurs nations, comme la mer fait monter ses flots\FTNT{Jé. 51:42.}.
\VS{4}Elles détruiront les murailles de Tyr, et démoliront ses tours~; j'en raclerai sa poussière, et la rendrai semblable à un rocher nu\FTNT{Es. 23:15.}.
\VS{5}Elle servira à étendre les filets au milieu de la mer~; car j'ai parlé, dit le Seigneur Yahweh, et elle sera en pillage aux nations.
\VS{6}Ses filles sur sa terre seront tuées par l'épée, et elles sauront que je suis Yahweh.
\VS{7}Car ainsi parle le Seigneur Yahweh~: Voici, je m'en vais faire venir du nord contre Tyr, Nebucadnetsar, roi de Babylone, le roi des rois, avec des chevaux, des chars, des cavaliers, et un grand peuple assemblé de toutes parts.
\VS{8}Il tuera par l'épée tes filles sur ta terre, il fera des remparts contre toi, il dressera des terrasses contre toi, et il lèvera les boucliers contre toi.
\VS{9}Il donnera des coups de béliers contre tes murs, et renversera tes tours avec ses épées.
\VS{10}La multitude de ses chevaux te couvrira de poussière, tes murs trembleront au bruit des cavaliers, des roues, et des chars, quand il entrera par tes portes, comme on entre dans une ville qu'on a divisée.
\VS{11}Il foulera toutes tes rues avec les sabots de ses chevaux, il tuera ton peuple avec l'épée, et les trophées de ta force tomberont par terre\FTNT{Es. 5:8~; Jé. 47:3.}.
\VS{12}Puis ils retireront tes biens, et pilleront ta marchandise~; ils renverseront tes murs, et renverseront tes maisons de plaisance~; et ils mettront tes pierres, ton bois et ta poussière au milieu des eaux.
\VS{13}Je ferai cesser le bruit de tes chansons, et le son de tes harpes ne sera plus entendu.
\VS{14}Je te rendrai semblable à un rocher nu~; tu seras un lieu pour étendre les filets, et tu ne seras plus rebâtie, parce que moi, Yahweh, j'ai parlé, dit le Seigneur Yahweh.
\VS{15}Ainsi parle le Seigneur Yahweh, à Tyr~: Les îles ne tremblent-elles pas du bruit de ta ruine, quand ceux qui sont blessés à mort gémissent, quand le carnage se fait au milieu de toi~?
\VS{16}Tous les princes de la mer descendent de leurs trônes, ôtent leurs manteaux, dépouillent leurs vêtements brodés, et s'enveloppent de frayeur~; ils s'assoient sur la terre, ils sont effrayés à chaque instant, et sont désolés à cause de toi.
\VS{17}Ils prononceront à haute voix une complainte sur toi, et te diront~: Comment as-tu péri, toi qui étais fréquentée par ceux qui vont sur la mer, ville renommée, qui étais forte dans la mer, toi et tes habitants, qui inspiraient la terreur à tous ceux qui habitent chez elle\FTNT{Es. 23:15-16~; Ap. 18:9.}~?
\VS{18}Maintenant les îles seront effrayées au jour de ta ruine, et les îles qui sont dans la mer seront terrifiées à cause de ta fuite.
\VS{19}Car ainsi parle le Seigneur Yahweh~: Quand je ferai de toi une ville désolée, comme sont les villes qui ne sont point habitées, quand je ferai tomber sur toi l'abîme, et que les grosses eaux te couvriront~;
\VS{20}alors je te ferai descendre avec ceux qui descendent dans la fosse, vers le peuple d'autrefois, et je te placerai aux lieux les plus bas de la terre, aux endroits désolés depuis longtemps, avec ceux qui descendent dans la fosse, afin que tu ne sois plus habitée, mais je donnerai la gloire pour la terre des vivants.
\VS{21}Je ferai qu'on sera épouvanté à cause de toi, de ce que tu n'es plus~; et quand on te cherchera, on ne te trouvera plus jamais, dit le Seigneur Yahweh.
\Chap{27}
\TextTitle{Lamentation sur Tyr\FTNTT{Ap. 18:1-24.}}
\VerseOne{}La parole de Yahweh vint encore à moi en disant~:
\VS{2}Toi donc, fils de l'homme, prononce à haute voix une complainte sur Tyr.
\VS{3}Tu diras à Tyr~: Toi qui demeures au bord de la mer, qui trafiques avec les peuples dans plusieurs îles~; ainsi parle le Seigneur Yahweh~: Tyr, tu disais~: Je suis parfaite en beauté~!
\VS{4}Ton territoire est au cœur de la mer, ceux qui t'ont bâtie t'ont rendue parfaite en beauté.
\VS{5}Ils t'ont bâti de tous les côtés des navires de sapins de Senir~; ils ont pris les cèdres du Liban pour te faire des mâts.
\VS{6}Ils ont fait tes rames de chênes de Basan, et la troupe des Assyriens a fait tes bancs d'ivoire, apporté des îles de Kittim.
\VS{7}Le fin lin d'Egypte, avec des broderies, te servait de voiles et de pavillon~; des étoffes teintes en pourpre et écarlate des îles d'Elischa formaient tes couvertures.
\VS{8}Les habitants de Sidon et d'Arvad étaient tes rameurs, ô Tyr~! Les plus sages du milieu de toi étaient tes pilotes.
\VS{9}Les anciens de Guebal et ses hommes experts furent parmi toi, réparant tes brèches~; tous les navires de la mer, et leurs mariniers étaient chez toi, pour faire l'échange de tes marchandises.
\VS{10}Ceux de Perse, de Lud, et de Puth servaient dans ton armée. C'étaient des hommes de guerre, ils suspendaient chez toi le bouclier et le casque~; ils te rendaient magnifique.
\VS{11}Les fils d'Arvad avec ton armée étaient autour de tes murs, et des hommes braves étaient dans tes tours~; ils suspendaient leurs boucliers à tous tes murs, ils achevaient de te rendre parfaite en beauté.
\VS{12}Ceux de Tarsis ont trafiqué avec toi de toutes sortes de richesses, d'argent, de fer, d'étain et de plomb.
\VS{13}Javan, Tubal, et Méschec trafiquaient avec toi~; ils donnaient des personnes et des ustensiles d'airain en échange de tes marchandises.
\VS{14}Ceux de la maison de Togarma pourvoyaient tes marchés de chevaux, de cavaliers, et de mulets.
\VS{15}Les fils de Dedan trafiquaient avec toi~; tu avais dans ta main le commerce de plusieurs îles~; et on te rendait en échange des dents d'ivoire et de l'ébène.
\VS{16}La Syrie trafiquait avec toi, en quantité d'ouvrages faits pour toi~; elle pourvoyait tes marchés d'escarboucles, d'écarlate, de broderie, de byssus, de corail, et d'agate.
\VS{17}Juda et le pays d'Israël trafiquaient avec toi, faisant valoir ton commerce en blé de Minnith, en pâtisseries, en miel, en huile, et en baume.
\VS{18}Damas trafiquait avec toi en quantité d'ouvrages faits pour toi, en toutes sortes de richesses, en vin de Helbon, et en laine blanche.
\VS{19}Vedan, et Javan depuis Uzal, pourvoyaient tes marchés~; le fer luisant, la casse et le roseau aromatique furent dans ton commerce.
\VS{20}Dedan trafiquait avec toi en couvertures pour s'asseoir à cheval.
\VS{21}Les arabes, et tous les princes de Kédar, étaient des marchands dans ta main, trafiquant avec toi en agneaux, en moutons, et en boucs.
\VS{22}Les marchands de Séba et de Raema trafiquaient avec toi de tous les meilleurs aromates, de toute sorte de pierres précieuses et d'or.
\VS{23}Charan, Canné, et Eden, les marchands de Séba, d'Assyrie, de Kilmad, trafiquaient avec toi.
\VS{24}Ils trafiquaient avec toi toutes sortes de belles choses, des manteaux en pourpre, en broderie, en riches étoffes contenues dans des coffres attachés avec des cordes, faits en bois de cèdre, et amenés sur tes marchés.
\VS{25}Les navires de Tarsis naviguaient pour ton commerce~; tu étais au comble de la force et de la richesse, au cœur des mers.
\VS{26}Tes rameurs t'ont amenée dans de grosses eaux, le vent d'orient t'a brisée au cœur de la mer.
\VS{27}Tes richesses, tes marchés et tes marchandises, tes mariniers et tes pilotes, ceux qui réparent tes brèches, et ceux qui s'occupent de ton commerce, tous tes hommes de guerre qui sont chez toi, et toute ta multitude au milieu de toi, tomberont dans le cœur de la mer au jour de ta ruine\FTNT{Ap. 18:9.}.
\VS{28}Les faubourgs trembleront au bruit du cri de tes pilotes.
\VS{29}Tous ceux qui manient la rame descendront de leurs navires, les mariniers, et tous les pilotes de la mer~; ils se tiendront sur la terre~;
\VS{30}ils feront entendre leur voix, et crieront amèrement~; ils jetteront de la poussière sur leurs têtes, et se vautreront dans la cendre~;
\VS{31}ils arracheront leurs cheveux, et rendront leur tête chauve à cause de toi, ils se ceindront de sacs, et te pleureront avec l'amertume dans leur âme, en menant un deuil amer.
\VS{32}Ils prononceront à haute voix sur toi une complainte dans leur lamentation, et feront leur complainte sur toi, en disant~: Qui fut jamais comme Tyr, comme cette ville détruite au cœur de la mer~?
\VS{33}Tu rassasiais plusieurs peuples par la traite des marchandises qu'on apportait de tes marchés au-delà des mers~; et tu enrichissais les rois de la terre par la multitude de tes richesses et de ton commerce.
\VS{34}Quand tu as été brisée par la mer au fond des eaux, ton commerce et toute ta multitude sont tombés avec toi.
\VS{35}Tous les habitants des îles sont désolés à cause de toi~; et leurs rois sont saisis d'épouvante, et leur visage pâlit.
\VS{36}Les marchands parmi les peuples t'insultent, tu es réduite au néant, tu ne seras plus à jamais~!
\Chap{28}
\TextTitle{Yahweh réprime l'arrogance du roi de Tyr}
\VerseOne{}La parole de Yahweh vint encore à moi en disant~:
\VS{2}Fils de l'homme, dis au prince de Tyr~: Ainsi parle le Seigneur Yahweh~: Parce que ton cœur s'est élevé et que tu as dit~: Je suis Dieu, je suis assis sur le siège de Dieu, au cœur de la mer, quoique tu sois un homme, et non Dieu, et parce que tu as élevé ton cœur comme si tu étais un dieu.
\VS{3}Voici, tu es plus sage que Daniel, rien de caché ne t'a été rendu obscur.
\VS{4}Tu t'es acquis de la puissance par ta sagesse et par ton intelligence~; et tu as amassé de l'or et de l'argent dans tes trésors\FTNT{Za. 9:2-3.}.
\VS{5}Tu as multiplié ta puissance par la grandeur de ta sagesse dans ton commerce, puis ton cœur s'est élevé à cause de ta puissance.
\VS{6}C'est pourquoi ainsi parle le Seigneur Yahweh~: Parce que tu as élevé ton cœur, comme si tu étais un dieu,
\VS{7}à cause de cela voici, je m'en vais faire venir contre toi des étrangers, les plus terribles parmi les nations, qui tireront leurs épées sur la beauté de ta sagesse, et souilleront ta splendeur.
\VS{8}Ils te feront descendre dans la fosse, et tu mourras comme ceux qui tombent percés de coups, au milieu de la mer.
\VS{9}En face de ton meurtrier, diras-tu~: Je suis Dieu~? Tu seras homme et non Dieu sous la main de celui qui te tuera.
\VS{10}Tu mourras de la mort des incirconcis par la main des étrangers~; car j'ai parlé, dit le Seigneur Yahweh.
\TextTitle{Chute du roi de Tyr représentant Satan\FTNTT{Es. 14:12-17.}}
\VS{11}La parole de Yahweh me fut encore adressée en ces mots~:
\VS{12}Fils de l'homme, prononce à haute voix une complainte sur le roi de Tyr, et dis-lui~: Ainsi parle le Seigneur Yahweh~: Toi à qui rien ne manquait, plein de sagesse, et parfait en beauté~;
\VS{13}tu étais en Eden, le jardin de Dieu~; ta couverture était de pierres précieuses de toutes sortes, de sardoine, de topaze, de diamant, de chrysolithe, d'onyx, de jaspe, de saphir, d'escarboucle, d'émeraude, et d'or~; tes tambourins et tes flûtes étaient à ton service~; préparés pour le jour où tu fus créé.
\VS{14}Tu étais un chérubin, oint pour servir de protection~; je t'avais établi, et tu étais sur la sainte montagne de Dieu~; tu marchais entre les pierres éclatantes.
\VS{15}Tu étais parfait dans tes voies dès le jour où tu fus créé, jusqu'à celui où l'injustice fut trouvée en toi.
\VS{16}Selon la grandeur de ton trafic\FTNT{Satan est le premier commerçant. Il avait transformé ses sanctuaires célestes en un lieu de trafic, en un marché.}, tu as été rempli de violence, et tu as péché~; c'est pourquoi je te jette comme une chose souillée hors de la montagne de Dieu\FTNT{Ap. 12:1-12.}, et je te détruis d'entre les pierres éclatantes, ô chérubin protecteur~!
\VS{17}Ton cœur s'est élevé à cause de ta beauté, tu as corrompu ta sagesse à cause de ton éclat~; je te jette par terre, je te donne en spectacle aux rois, afin qu'ils te regardent.
\VS{18}Tu as profané tes sanctuaires par la multitude de tes iniquités, par l'injustice de ton commerce~; et je fais sortir du milieu de toi un feu qui te consume, je te réduis en cendres sur la terre, dans la présence de tous ceux qui te regardent.
\VS{19}Tous ceux qui te connaissent parmi les peuples sont désolés à cause de toi~; tu es réduit à néant, tu ne seras plus à jamais.
\TextTitle{Jugement sur Sidon}
\VS{20}Puis la parole de Yahweh vint à moi en disant~:
\VS{21}Fils de l'homme, tourne ta face vers Sidon, et prophétise contre elle.
\VS{22}Tu diras~: Ainsi parle le Seigneur Yahweh~: Voici j'en veux à toi, Sidon~! Je serai glorifié au milieu de toi~; et on saura que je suis Yahweh, quand j'aurai exercé des jugements contre elle et que je serai sanctifié.
\VS{23}J'enverrai la peste dans son sein, je ferai couler le sang dans ses rues. Les morts tomberont au milieu d'elle par l'épée qui viendra de toutes parts sur elle~; et ils sauront que je suis Yahweh.
\VS{24}Elle ne sera plus pour la maison d'Israël une épine qui blesse, une ronce déchirante, parmi tous ceux qui l'entourent et qui la méprisent. Et ils sauront que je suis le Seigneur Yahweh.
\TextTitle{Rétablissement d'Israël}
\VS{25}Ainsi parle le Seigneur Yahweh~: Quand j'aurai rassemblé la maison d'Israël d'entre les peuples parmi lesquels ils auront été dispersés, je manifesterai en elle ma sainteté, aux yeux des nations, et ils habiteront sur leur terre que j'ai donnée à mon serviteur Jacob.
\VS{26}Ils y habiteront en sûreté, ils bâtiront des maisons, ils planteront des vignes~; ils y habiteront, dis-je, en sûreté, lorsque j'aurai exercé des jugements contre ceux qui les auront pillés de toutes parts~; et ils sauront que je suis Yahweh, leur Dieu.
\Chap{29}
\TextTitle{Jugement sur l'Egypte}
\VerseOne{}La dixième année, au douzième jour du dixième mois, la parole de Yahweh vint à moi en disant~:
\VS{2}Fils de l'homme, tourne ta face contre Pharaon, roi d'Egypte, prophétise contre lui, et contre toute l'Egypte\FTNT{Jé. 43:8-11.}.
\VS{3}Parle, et dis~: Ainsi parle le Seigneur Yahweh~: Voici, j'en veux à toi, Pharaon, roi d'Egypte, grand serpent couché au milieu de tes fleuves, qui dis~: Mes fleuves sont à moi, et je me les suis faits\FTNT{Es. 27:1~; Ps. 74:13-14.}~!
\VS{4}C'est pourquoi je mettrai des crocs dans ta mâchoire, j'attacherai à tes écailles les poissons de tes fleuves~; je te tirerai hors de tes fleuves, avec tous les poissons de tes fleuves, qui seront attachés à tes écailles.
\VS{5}Et t'ayant tiré dans le désert, je te laisserai là, toi, et tous les poissons de tes fleuves~; tu tomberas sur la face des champs, tu ne seras point recueilli ni ramassé~; je te livrerai aux bêtes de la terre, et aux oiseaux des cieux, pour en être dévoré.
\VS{6}Et tous les habitants d'Egypte sauront que je suis Yahweh~; parce qu'ils ont été un soutien de roseau pour la maison d'Israël\FTNT{2 R. 18:21~; Es. 36:6.}.
\VS{7}Quand ils t'ont pris par la main, tu t'es rompu, et tu leur as percé toute l'épaule~; et quand ils se sont appuyés sur toi, tu t'es cassé, et tu les as fait tomber à la renverse.
\VS{8}C'est pourquoi ainsi parle le Seigneur Yahweh~: Voici, je ferai venir l'épée sur toi, et j'exterminerai du milieu de toi les hommes et les bêtes.
\VS{9}Le pays d'Egypte sera dans la désolation et dans le désert, et ils sauront que je suis Yahweh, parce que le roi d'Egypte a dit~: Les fleuves sont à moi, et je les ai faits~!
\VS{10}C'est pourquoi voici, j'en veux à toi, et à tes fleuves, et je réduirai le pays d'Egypte en désert de sécheresse et de désolation, depuis Migdol jusqu'à Syène, et aux frontières de l'Ethiopie.
\VS{11}Nul pied d'homme ne passera par là, et il n'y passera non plus aucun pied d'animal, il sera quarante ans sans être habité.
\VS{12}Car je réduirai le pays d'Egypte en désolation entre les pays désolés, et ses villes entre les villes réduites en désert~; elles seront en désolation durant quarante ans, je disperserai les Egyptiens parmi les nations, et je les répandrai parmi les pays.
\VS{13}Toutefois, ainsi parle le Seigneur Yahweh~: Au bout de quarante ans, je ramasserai les Egyptiens d'entre les peuples parmi lesquels ils auront été dispersés~;
\VS{14}je ramènerai les captifs d'Egypte, et les ferai retourner au pays de Pathros, au pays de leur origine, mais ils seront là un royaume rabaissé.
\VS{15}Il sera le plus bas des royaumes, et il ne s'élèvera plus au-dessus des nations, je le diminuerai, afin qu'il ne domine point sur les nations.
\VS{16}Ce royaume ne sera plus pour la main d'Israël un sujet de confiance~; il lui rappellera son iniquité, quand elle se tournait vers eux~; et ils sauront que je suis le Seigneur Yahweh.
\VS{17}Il arriva la vingt-septième année, au premier jour du premier mois, que la parole de Yahweh me fut adressée en ces mots~:
\VS{18}Fils de l'homme, Nebucadnetsar, roi de Babylone, a fait servir son armée dans un service pénible contre Tyr~; toute tête en est devenue chauve, et toute épaule en a été foulée, mais il n'a point eu de salaire, ni lui ni son armée, à cause de Tyr, pour le service qu'il a fait contre elle.
\VS{19}C'est pourquoi, ainsi parle le Seigneur Yahweh~: Voici, je m'en vais donner à Nebucadnetsar, roi de Babylone, le pays d'Egypte~; il enlèvera la multitude, il emportera le butin et fera le pillage~; ce sera là le salaire de son armée.
\VS{20}Pour prix du service qu'il a fait contre Tyr, je lui ai donné le pays d'Egypte, parce qu'ils ont travaillé pour moi, dit le Seigneur Yahweh.
\VS{21}En ce jour-là, je ferai germer la corne de la maison d'Israël, et j'ouvrirai ta bouche au milieu d'eux, et ils sauront que je suis Yahweh.
\Chap{30}
\TextTitle{Disgrâce de l'Egypte}
\VerseOne{}La parole de Yahweh vint encore à moi en disant~:
\VS{2}Fils de l'homme, prophétise, et dis~: Ainsi parle le Seigneur Yahweh~: Hurlez, et dites~: Malheureux jour~!
\VS{3}Car le jour est proche, oui le jour de Yahweh\FTNT{Voir commentaire en Zacharie 14:1.} est proche, c'est un jour ténébreux~; ce sera le temps des nations.
\VS{4}L'épée viendra sur l'Egypte, et il y aura de l'effroi en Ethiopie, quand ceux qui seront blessés à mort tomberont dans l'Egypte, quand on enlèvera la multitude de son peuple, et que ses fondements seront ruinés.
\VS{5}L'Ethiopie, Puth, Lud, toute l'Arabie, Cub, et les fils du pays allié tomberont par l'épée avec eux\FTNT{Jé. 46:9~; Na. 3:9-10.}.
\VS{6}Ainsi parle Yahweh~: Ceux qui soutiendront l'Egypte, tomberont~; et l'orgueil de sa force sera renversé~; ils tomberont par l'épée de Migdol à Syène, dit le Seigneur Yahweh.
\VS{7}Ils seront désolés au milieu des pays désolés, et ses villes seront au milieu des villes désertes.
\VS{8}Ils sauront que je suis Yahweh, quand j'aurai mis le feu en Egypte~; et tous ceux qui lui donneront du secours, seront brisés.
\VS{9}En ce jour-là, des messagers sortiront de ma part sur des navires pour effrayer l'Ethiopie dans sa sécurité, et il y aura entre eux un tourment au jour de l'Egypte~; car voici, il vient.
\VS{10}Ainsi parle le Seigneur Yahweh~: Je ferai périr la multitude d'Egypte par la puissance de Nebucadnetsar, roi de Babylone.
\VS{11}Lui et son peuple avec lui, les plus terribles d'entre les nations, seront amenés pour ruiner le pays~; ils tireront leurs épées contre les Egyptiens, et rempliront la terre de morts.
\VS{12}Je mettrai à sec les fleuves et je livrerai le pays entre les mains des méchants~; je désolerai le pays, et tout ce qui y est, par la puissance des étrangers~; moi, Yahweh, j'ai parlé.
\VS{13}Ainsi parle le Seigneur Yahweh~: Je détruirai aussi les idoles, j'anéantirai les faux dieux de Noph, et il n'y aura point de prince qui soit du pays d'Egypte~; je mettrai la frayeur dans le pays d'Egypte\FTNT{Es. 19:1-13~; Jé. 43:12~; Jé. 46:13.}.
\VS{14}Je désolerai Pathros, je mettrai le feu à Tsoan, et j'exercerai mes jugements sur No\FTNT{Jé. 44:1.}.
\VS{15}Je répandrai ma fureur sur Sin, qui est la place forte de l'Egypte, et j'exterminerai la multitude qui est à No.
\VS{16}Quand je mettrai le feu en Egypte, Sin sera grièvement tourmentée, No sera rompue par diverses brèches, et il n'y aura à Noph que détresses en plein jour.
\VS{17}Les jeunes hommes d'On et de Pi-Béseth tomberont par l'épée, et ces villes iront en captivité.
\VS{18}Le jour s'obscurcira à Tachpanès, lorsque j'y romprai le joug de l'Egypte, et que l'orgueil de sa force cessera~; un nuage la couvrira, et les villes de son ressort iront en captivité.
\VS{19}J'exercerai des jugements en Egypte~; et ils sauront que je suis Yahweh.
\TextTitle{Chute et dispersion de l'Egypte}
\VS{20}Or il arriva que dans la onzième année, au septième jour du premier mois, la parole de Yahweh me fut adressée en ces mots~:
\VS{21}Fils de l'homme, j'ai rompu le bras de Pharaon, roi d'Egypte~; et voici on ne l'a point bandé pour le guérir, on ne lui a point mis de linges pour le bander, et pour le fortifier, afin qu'il puisse manier l'épée.
\VS{22}C'est pourquoi ainsi parle le Seigneur Yahweh~: Voici, j'en veux à Pharaon, roi d'Egypte, et je romprai ses bras, tant celui qui est fort que celui qui est rompu, et je ferai tomber l'épée de sa main.
\VS{23}Je disperserai les Egyptiens parmi les nations, et les répandrai parmi les pays.
\VS{24}Je fortifierai les bras du roi de Babylone, je lui mettrai mon épée dans la main~; mais je romprai les bras de Pharaon, et il gémira devant lui comme gémissent les mourants.
\VS{25}Je fortifierai donc les bras du roi de Babylone, mais les bras de Pharaon tomberont~; et on saura que je suis Yahweh, quand je mettrai mon épée dans la main du roi de Babylone, et qu'il l'a tournera contre le pays d'Egypte.
\VS{26}Je disperserai les Egyptiens parmi les nations, les répandrai parmi les pays~; et ils sauront que je suis Yahweh.
\Chap{31}
\TextTitle{Avertissement contre l'arrogance de Pharaon}
\VerseOne{}Il arriva aussi dans la onzième année, au premier jour du troisième mois, que la parole de Yahweh vint à moi en disant~:
\VS{2}Fils de l'homme, parle à Pharaon, roi d'Egypte, et à la multitude de son peuple~: A qui ressembles-tu dans ta grandeur~?
\VS{3}Voici, le roi d'Assyrie a été comme un cèdre du Liban, ayant de belles branches, et des rameaux qui faisaient une grande ombre, et qui étaient d'une grande hauteur~; sa cime était fort touffue.
\VS{4}Les eaux l'ont fait croître, l'abîme l'a fait pousser en hauteur, ses fleuves ont coulé autour de ses plantes, et il a envoyé ses eaux abondantes vers tous les arbres des champs.
\VS{5}C'est pourquoi il s'est élevé au-dessus de tous les autres arbres des champs, ses branches se sont multipliées, et ses rameaux croissaient par les grandes eaux qui faisaient pousser ses branches.
\VS{6}Tous les oiseaux des cieux ont fait leurs nids dans ses branches, toutes les bêtes des champs ont fait leurs petits sous ses rameaux, et toutes les grandes nations ont habité sous son ombre.
\VS{7}Il était beau par sa grandeur, et par l'étendue de ses branches, parce que sa racine était sur de grandes eaux.
\VS{8}Les cèdres du jardin de Dieu ne le surpassaient point~; les cyprès n'égalaient point ses branches, et les platanes n'égalaient point ses rameaux~; aucun arbre du jardin de Dieu ne lui était comparable en beauté.
\VS{9}Je l'avais embelli par la multitude de ses rameaux, au point que tous les arbres d'Eden, qui étaient dans le jardin de Dieu, lui portaient envie.
\VS{10}C'est pourquoi le Seigneur Yahweh dit~: Parce qu'il s'est élevé, parce qu'il lançait sa cime au milieu d'épais rameaux et que son cœur était fier de sa hauteur,
\VS{11}je l'ai livré entre les mains du plus fort des nations, qui l'a traité comme il fallait, et je l'ai chassé à cause de sa méchanceté.
\VS{12}Les étrangers les plus terrifiants parmi les nations l'ont coupé et l'ont laissé là, ses branches sont tombées sur les montagnes et sur toutes les vallées~; ses rameaux se sont rompus dans tous les ravins de la terre, et tous les peuples de la terre se sont retirés de dessous son ombre, et l'ont laissé là.
\VS{13}Tous les oiseaux des cieux se sont tenus sur ses ruines, et toutes les bêtes des champs se sont retirées vers ses rameaux,
\VS{14}afin que tous les arbres près des eaux n'élèvent plus leur hauteur, et qu'ils ne lancent plus leur cime au milieu d'épais rameaux, afin que tous les chênes arrosés d'eau ne gardent plus leur hauteur~; car tous sont livrés à la mort, aux profondeurs de la terre, parmi les fils des hommes, avec ceux qui descendent dans la fosse.
\VS{15}Ainsi parle le Seigneur Yahweh~: Le jour qu'il descendit dans le scheol, j'ai répandu le deuil sur lui, j'ai couvert l'abîme devant lui, j'ai empêché ses fleuves de couler, et les grosses eaux ont été retenues~; j'ai fait que le Liban soit en deuil à cause de lui, et tous les arbres des champs ont été desséchés.
\VS{16}J'ai ébranlé les nations par le bruit de sa ruine, quand je l'ai fait descendre dans le scheol, avec ceux qui descendent dans la fosse\FTNT{Es. 14:9.}~; et tous les arbres d'Eden, les plus beaux et les plus agréables du Liban, tous arrosés par les eaux, ont été consolés dans les profondeurs de la terre.
\VS{17}Eux aussi sont descendus avec lui dans le scheol, vers ceux qui ont péri par l'épée~; ils étaient son bras et ils habitaient sous son ombre parmi les nations.
\VS{18}A qui ressembles-tu ainsi en gloire et en grandeur parmi les arbres d'Eden~? Tu seras précipité avec les arbres d'Eden dans les profondeurs de la terre, tu seras gisant au milieu des incirconcis, avec ceux qui ont péri par l'épée. Voilà Pharaon et toute sa multitude~! Dit le Seigneur Yahweh.
\Chap{32}
\TextTitle{Lamentation sur le pays d'Egypte}
\VerseOne{}Il arriva aussi dans la douzième année, le premier jour du douzième mois, la parole de Yahweh vint à moi en disant~:
\VS{2}Fils de l'homme, prononce à haute voix une complainte sur Pharaon, roi d'Egypte, et dis-lui~: Tu as été parmi les nations semblable à un lionceau, et comme un serpent dans les mers~; tu t'élançais dans tes fleuves, et tu troublais les eaux avec tes pieds, et remplissais de bourbe leurs fleuves.
\VS{3}Ainsi parle le Seigneur Yahweh~: J'étendrai mon rets sur toi dans une assemblée nombreuse de peuples qui te tireront dans mes filets.
\VS{4}Je te laisserai à l'abandon sur la terre~; je te jetterai sur le dessus des champs, et je ferai demeurer sur toi tous les oiseaux des cieux, et rassasierai de toi les bêtes de toute la terre.
\VS{5}Car je mettrai ta chair sur les montagnes, et je remplirai les vallées de tes débris.
\VS{6}J'arroserai de ton sang jusqu'aux montagnes, la terre où tu nages, et les lits des eaux seront remplis de toi.
\VS{7}Quand je t'éteindrai, je couvrirai les cieux et j'obscurcirai leurs étoiles, je couvrirai le soleil de nuages, et la lune ne donnera plus sa lumière\FTNT{Es. 13:10~; Joë. 2:31~; Mt. 24:29.}.
\VS{8}J'obscurcirai à cause de toi tous les luminaires des cieux, et je répandrai les ténèbres sur ton pays, dit le Seigneur Yahweh.
\VS{9}J'affligerai le cœur de beaucoup de peuples, quand j'annoncerai ta ruine parmi les nations, à des pays que tu ne connaissais pas.
\VS{10}Je frapperai de stupeur beaucoup de peuples à cause de toi, et leurs rois seront saisis d'épouvante à cause de toi, quand je ferai luire mon épée à leurs yeux~; ils seront effrayés à chaque instant, chacun pour sa vie, au jour de ta ruine.
\VS{11}Car ainsi parle le Seigneur Yahweh~: L'épée du roi de Babylone viendra sur toi.
\VS{12}J'abattrai ta multitude par les épées des hommes forts, qui tous sont les plus terribles d'entre les nations~; ils détruiront l'orgueil de l'Egypte, et toute la multitude de son peuple sera ruinée.
\VS{13}Je ferai périr tout son bétail près des grandes eaux, et aucun pied d'homme ne les troublera plus, ni aucun pied d'animal ne les agitera plus.
\VS{14}Alors je rendrai profondes leurs eaux, et je ferai couler leurs fleuves comme de l'huile, dit le Seigneur Yahweh.
\VS{15}Quand j'aurai réduit le pays d'Egypte en désolation, et que le pays sera dénué des choses dont il était rempli~; quand je frapperai tous ceux qui y habitent, ils sauront alors que je suis Yahweh.
\VS{16}C'est ici la complainte qu'on fera sur elle, les filles des nations feront cette complainte sur elle~; elles feront cette complainte sur l'Egypte et sur toute la multitude de son peuple, dit le Seigneur Yahweh.
\VS{17}Il arriva aussi dans la douzième année, le quinzième jour du mois, que la parole de Yahweh me fut adressée en ces mots~:
\VS{18}Fils de l'homme, dresse une lamentation sur la multitude d'Egypte, et fais-la descendre, elle et les filles des nations magnifiques, aux plus bas lieux de la terre, avec ceux qui descendent dans la fosse\FTNT{Jé. 1:10~; Jé. 18:7.}.
\VS{19}Qui surpasses-tu en beauté~? Descends, et couche-toi avec les incirconcis~!
\VS{20}Ils tomberont au milieu de ceux qui seront tués par l'épée. L'épée a déjà été donnée~: Entraînez l'Egypte et toute sa multitude~!
\VS{21}Les plus forts d'entre les puissants lui parleront du milieu du scheol, avec ceux qui lui donnaient du secours, et diront~: Ils sont descendus, ils sont couchés, les incirconcis, tués par l'épée.
\VS{22}Là est l'Assyrien, et toute son assemblée~; ses sépulcres sont autour de lui, eux tous, mis à mort, sont tombés par l'épée.
\VS{23}Car ses sépulcres sont posés au fond de la fosse et son assemblée autour de sa sépulture~; eux tous, qui répandaient leur terreur sur la terre des vivants, sont tombés morts par l'épée.
\VS{24}Là est Elam et toute sa multitude autour de son sépulcre~; eux tous sont tombés morts par l'épée, ils sont descendus incirconcis dans les plus bas lieux de la terre~; et après avoir répandu leur terreur sur la terre des vivants, ils ont porté leur ignominie avec ceux qui descendent dans la fosse.
\VS{25}On a mis sa couche parmi ceux qui ont été tués, avec toute sa multitude~; ses sépulcres sont autour de lui~; eux tous incirconcis, tués par l'épée, quoiqu'ils aient répandu leur terreur sur la terre des vivants, toutefois ils ont porté leur ignominie avec ceux qui descendent dans la fosse~; ils ont été placés parmi les morts.
\VS{26}Là est Méschec, Tubal, et toute leur multitude~; leurs sépulcres sont autour d'eux~; eux tous incirconcis, tués par l'épée, quoiqu'ils aient répandu leur terreur sur la terre des vivants.
\VS{27}Ils ne se sont point couchés avec les hommes vaillants qui sont tombés d'entre les incirconcis, lesquels sont descendus dans le scheol avec leurs armes de guerre, dont on a mis les épées sous leurs têtes, et dont les iniquités ont reposé sur leurs os~; parce que la terreur des hommes forts est dans la terre des vivants.
\VS{28}Toi aussi tu seras brisé au milieu des incirconcis, et tu seras couché avec ceux qui sont tués par l'épée.
\VS{29}Là est Edom, ses rois, et tous ses princes, qui ont été placés malgré leur force avec ceux qui sont tués par l'épée~; ils seront couchés avec les incirconcis, et avec ceux qui sont descendus dans la fosse.
\VS{30}Là sont tous les princes du nord, et tous les Sidoniens, qui sont descendus avec ceux qui sont tués, malgré la terreur qu'inspirait leur force~; ils sont couchés incirconcis avec ceux qui sont tués par l'épée~; ils ont porté leur ignominie avec ceux qui sont descendus dans la fosse.
\VS{31}Pharaon les verra, et il se consolera au sujet de toute la multitude de son peuple~; Pharaon, dit le Seigneur Yahweh, verra les blessés par l'épée et toute son armée.
\VS{32}Car je mettrai ma terreur dans la terre des vivants~; c'est pourquoi Pharaon, avec toute la multitude de son peuple, se couchera au milieu des incirconcis, avec ceux qui sont tués par l'épée, dit le Seigneur Yahweh.
\Chap{33}
\TextTitle{Ezéchiel établi comme sentinelle pour avertir le pécheur}
\VerseOne{}La parole de Yahweh vint encore à moi en disant~:
\VS{2}Fils de l'homme, parle aux fils de ton peuple, et dis-leur~: Quand je ferai venir l'épée sur un pays, et que le peuple du pays aura choisi quelqu'un d'entre eux, et l'aura établi pour leur servir de sentinelle,
\VS{3}et que voyant venir l'épée sur le pays, il sonne du shofar et avertit le peuple,
\VS{4}si le peuple ayant bien entendu le son du shofar, ne se tient pas sur ses gardes, et qu'ensuite l'épée vienne le prendre, son sang sera sur sa tête.
\VS{5}Car il a entendu le son du shofar, et ne s'est point tenu sur ses gardes~; son sang sera sur lui~; mais s'il se tient sur ses gardes, il sauvera sa vie.
\VS{6}Si la sentinelle voit venir l'épée, et qu'elle ne sonne point du shofar, en sorte que le peuple ne se tienne point sur ses gardes, et qu'ensuite l'épée survienne et ôte la vie à l'un d'entre eux, celui-ci sera emmené en captivité à cause de son iniquité, mais je redemanderai son sang de la main de la sentinelle.
\VS{7}Toi donc, fils de l'homme, je t'ai établi pour sentinelle sur la maison d'Israël~; tu écouteras donc la parole qui sort de ma bouche, et tu les avertiras de ma part.
\VS{8}Quand je dirai au méchant~: Méchant, tu mourras~! Tu mourras~! Et que tu ne parleras point au méchant pour l'avertir de se détourner de sa voie, ce méchant mourra dans son iniquité~; mais je redemanderai son sang de ta main.
\VS{9}Mais si tu as averti le méchant de se détourner de sa voie, et qu'il ne se détourne pas de sa voie, il mourra dans son iniquité~; mais toi tu auras délivré ton âme.
\VS{10}Toi donc, fils de l'homme, dis à la maison d'Israël~: Vous avez parlé ainsi, en disant~: Puisque nos crimes et nos péchés sont sur nous, et que nous périssons à cause d'eux, comment pourrions-nous vivre\FTNT{Lé. 26:39.}~?
\VS{11}Dis-leur~: Je suis vivant, dit le Seigneur Yahweh, je ne prends point plaisir dans la mort du méchant, mais que le méchant se détourne de sa voie et qu'il vive. Détournez-vous, détournez-vous de votre méchante voie~! Pourquoi mourriez-vous, maison d'Israël~?
\VS{12}Toi donc, fils de l'homme, dis aux enfants de ton peuple~: La justice du juste ne le délivrera point au jour de son péché, le méchant ne tombera point par sa méchanceté au jour où il s'en détournera~; et le juste ne pourra pas vivre par sa justice au jour de son péché.
\VS{13}Quand j'aurai dit au juste qu'il vivra certainement, et que lui, se confiant sur sa justice, aura commis l'iniquité, on ne se souviendra plus d'aucune de ses justices, mais il mourra dans son iniquité qu'il aura commise.
\VS{14}Aussi, quand j'aurai dit au méchant~: Tu mourras~! S'il se détourne de son péché, et qu'il fasse ce qui est juste et droit~;
\VS{15}si le méchant rend le gage et qu'il restitue ce qu'il aura ravi, et qu'il marche dans les statuts de la vie, sans commettre d'iniquité, certainement il vivra, il ne mourra point.
\VS{16}On ne se souviendra plus des péchés qu'il aura commis~; s'il fait ce qui est juste et droit~; certainement il vivra.
\VS{17}Or les enfants de ton peuple ont dit~: La voie du Seigneur n'est pas bien réglée~; mais c'est plutôt leur voie qui n'est pas bien réglée.
\VS{18}Quand le juste se détournera de sa justice, et qu'il commettra l'iniquité, il mourra à cause de cela.
\VS{19}Quand le méchant se détournera de sa méchanceté, et qu'il fera ce qui est juste et droit, il vivra à cause de cela.
\VS{20}Vous avez dit~: La voie du Seigneur n'est pas bien réglée~! Je vous jugerai, maison d'Israël, chacun selon sa voie.
\TextTitle{Exécution du jugement de Yahweh}
\VS{21}Or il arriva dans la douzième année de notre captivité, au cinquième jour du dixième mois, qu'un homme qui s'était échappé de Jérusalem vint vers moi, en disant~: La ville est prise~!
\VS{22}La main de Yahweh fut sur moi le soir, avant l'arrivée du fugitif, et Yahweh ouvrit ma bouche lorsqu'il vint auprès de moi le matin. Ma bouche était ouverte et je n'étais plus muet.
\TextTitle{Ne pas se contenter d'écouter la Parole de Dieu}
\VS{23}La parole de Yahweh vint à moi en disant~:
\VS{24}Fils de l'homme, ceux qui habitent dans ces ruines, sur la terre d'Israël, discourent en disant~: Abraham était seul, et il a possédé le pays\FTNT{Ge. 15:7.}~; mais nous sommes un grand nombre de gens, et le pays nous a été donné en héritage.
\VS{25}C'est pourquoi tu leur diras~: Ainsi parle le Seigneur Yahweh~: Vous mangez la chair avec le sang, et vous levez vos yeux vers vos idoles, vous répandez le sang~; et vous posséderiez le pays\FTNT{Ge. 9:4~; Lé. 3:17~; Lé. 17:10.}~?
\VS{26}Vous vous appuyez sur votre épée, vous commettez des abominations, vous souillez chacun de vous la femme de son prochain~; et vous posséderiez le pays~?
\VS{27}Tu leur diras~: Ainsi parle le Seigneur Yahweh~: Je suis vivant, ceux qui sont dans ces ruines tomberont par l'épée, et je livrerai aux bêtes celui qui est dans les champs, afin qu'elles le mangent~; et ceux qui sont dans les forteresses et dans les cavernes mourront par la peste.
\VS{28}Ainsi je réduirai le pays en désolation et en désert, l'orgueil de sa force sera aboli, et les montagnes d'Israël seront désolées, en sorte qu'il n'y passera plus personne.
\VS{29}Ils sauront que je suis Yahweh, quand j'aurai réduit leur pays en désolation et en désert, à cause de toutes leurs abominations qu'ils ont commises.
\VS{30}Quant à toi, fils de l'homme, les enfants de ton peuple parlent de toi près des murs et aux entrées des maisons, et parlent l'un à l'autre, chacun avec son prochain, en disant~: Venez maintenant, et écoutez la parole qui vient de Yahweh.
\VS{31}Ils viennent vers toi en foule, et mon peuple s'assied devant toi, ils écoutent tes paroles, mais ils ne les mettent point en pratique~; ils les répètent comme si c'était une chanson profane, mais leur cœur marche toujours après leur gain déshonnête.
\VS{32}Voici, tu es pour eux comme un homme qui leur chante une chanson profane avec une belle voix, qui résonne bien~; car ils écoutent bien tes paroles, mais ils ne les mettent point en pratique.
\VS{33}Mais quand ces choses arriveront, et voici, elles arrivent, ils sauront qu'il y avait un prophète au milieu d'eux.
\Chap{34}
\TextTitle{Jugement de Dieu sur les faux bergers}
\VerseOne{}La parole de Yahweh vint encore à moi en disant~:
\VS{2}Fils de l'homme, prophétise contre les pasteurs d'Israël~! Prophétise, et dis à ces pasteurs~: Ainsi parle le Seigneur Yahweh~: Malheur aux pasteurs d'Israël, qui ne paissent qu'eux-mêmes~! Les pasteurs ne paissent-ils pas le troupeau~?
\VS{3}Vous en mangez la graisse, et vous vous habillez de laine~; vous tuez ce qui est gras, vous ne paissez point le troupeau~!
\VS{4}Vous n'avez point fortifié les brebis languissantes, vous n'avez point donné de remède à celle qui était malade, vous n'avez point bandé la plaie de celle qui avait la jambe rompue, vous n'avez point ramené celle qui était chassée, et vous n'avez point cherché celle qui était perdue\FTNT{Lu. 15:4-6~; 1 Pi. 5:1-3.}~; mais vous les avez maîtrisées avec dureté et rigueur.
\VS{5}Elles se sont dispersées, parce qu'elles n'avaient pas de pasteurs, et elles se sont exposées à toutes les bêtes des champs, pour en être dévorées, étant dispersées.
\VS{6}Mes brebis sont errantes sur toutes les montagnes et sur toutes les collines élevées mes brebis sont dispersées sur toute la surface de la terre~; et il n'y a personne qui les recherche, et il n'y a personne pour s'en soucier\FTNT{Za. 13:7~; Mt. 26:31~; Mc. 14:27.}.
\VS{7}C'est pourquoi pasteurs, écoutez la parole de Yahweh~:
\VS{8}Je suis vivant, dit le Seigneur Yahweh, parce que mes brebis sont pillées, et que mes brebis sont la nourriture de toutes les bêtes des champs, parce qu'elles n'ont point de pasteur~; car mes pasteurs n'ont point recherché mes brebis, mais les pasteurs se sont nourris simplement eux-mêmes, et n'ont point fait paître mes brebis.
\VS{9}C'est pourquoi, pasteurs, écoutez la parole de Yahweh~!
\VS{10}Ainsi parle le Seigneur Yahweh~: Voici, j'en veux à ces pasteurs-là, et je redemanderai mes brebis de leur main~; ils cesseront de paître les brebis, et les pasteurs ne se repaîtront plus eux-mêmes, mais je délivrerai mes brebis de leur bouche, et elles ne seront plus dévorées par eux.
\TextTitle{Yahweh, le bon berger qui restaure son troupeau\FTNT{Jn. 10:1-18.}}
\VS{11}Car ainsi parle le Seigneur Yahweh~: Me voici, je redemanderai mes brebis, et je les rechercherai.
\VS{12}Comme le pasteur prend soin de son troupeau quand il est au milieu de ses brebis dispersées, ainsi je rechercherai mes brebis, et les retirerai de tous les lieux où elles auront été dispersées au jour des nuages et de l'obscurité.
\VS{13}Je les retirerai d'entre les peuples et les rassemblerai des territoires, les ramènerai dans leur terre, et les nourrirai sur les montagnes d'Israël, auprès des cours d'eau et dans toutes les demeures du pays.
\VS{14}Je les paîtrai dans de bons pâturages, et leur demeure sera sur les hautes montagnes d'Israël~; et là elles coucheront dans une agréable demeure, et paîtront dans de gras pâturages, sur les montagnes d'Israël.
\VS{15}Moi-même je paîtrai mes brebis et les ferai reposer, dit le Seigneur Yahweh\FTNT{Ps. 23.}.
\VS{16}Je chercherai celle qui était perdue, et je ramènerai celle qui était chassée, je banderai la plaie de celle qui a la jambe rompue, et je fortifierai celle qui est malade~; mais je détruirai la grasse et la forte~; je les paîtrai avec justice.
\VS{17}Quant à vous, mes brebis, ainsi parle le Seigneur Yahweh~: Voici, je m'en vais mettre à part les brebis, les béliers, et les boucs.
\VS{18}Et vous, est-ce peu de chose de vous faire paître dans de bons pâturages, pour que vous fouliez de vos pieds le reste de votre pâture~? Et de boire des eaux claires, pour que vous troubliez le reste avec vos pieds~?
\VS{19}Mais mes brebis sont nourries du pâturage que vous foulez de vos pieds, et boivent ce que vos pieds ont troublé.
\VS{20}C'est pourquoi le Seigneur Yahweh leur dit~: Me voici, je mettrai moi-même à part la brebis grasse et la brebis maigre.
\VS{21}Parce que vous poussez du côté et de l'épaule, et que vous heurtez de vos cornes toutes celles qui sont languissantes, jusqu'à ce que vous les ayez chassées dehors,
\VS{22}je sauverai mes brebis, au point qu'elles ne seront plus au pillage. Voici, je jugerai entre brebis et brebis.
\VS{23}Je susciterai sur elles un pasteur qui les paîtra, mon serviteur David~; il les paîtra, et lui-même sera leur pasteur.
\VS{24}Moi, Yahweh, je serai leur Dieu, et mon serviteur David sera prince au milieu d'elles~; moi,Yahweh, j'ai parlé.
\VS{25}Je traiterai avec elles une alliance de paix~; et je détruirai dans le pays les mauvaises bêtes~; les brebis habiteront dans le désert en sécurité, et dormiront dans les forêts.
\VS{26}Je les comblerai de bénédictions, elles, et tous les environs de mes collines~; je ferai tomber la pluie en sa saison~; ce seront des pluies de bénédiction.
\VS{27}Les arbres des champs produiront leur fruit, et la terre rapportera son revenu~; elles seront dans leur terre en sécurité, et sauront que je suis Yahweh, quand j'aurai rompu les bois de leur joug, et que je les aurai délivrées de la main de ceux qui se les asservissent.
\VS{28}Elles ne seront plus au pillage parmi les nations, et les bêtes de la terre ne les dévoreront plus~; mais elles habiteront en sécurité, et il n'y aura personne pour les effrayer.
\VS{29}Je leur susciterai une plantation de renom~; elles ne mourront plus de faim sur la terre, et ne porteront plus l'opprobre des nations.
\VS{30}Ils sauront que moi, Yahweh, leur Dieu, suis avec eux, et qu'eux, la maison d'Israël, sont mon peuple, dit le Seigneur Yahweh.
\VS{31}Or vous êtes mes brebis, vous hommes, les brebis de mon pâturage, et je suis votre Dieu, dit le Seigneur Yahweh.
\Chap{35}
\TextTitle{Jugement sur Edom}
\VerseOne{}La parole de Yahweh vint encore à moi en disant~:
\VS{2}Fils de l'homme, tourne ta face contre la montagne de Séir, et prophétise contre elle\FTNT{Am. 1:11.}.
\VS{3}Dis-lui~: Ainsi parle le Seigneur Yahweh~: Voici, j'en veux à toi, montagne de Séir, et j'étendrai ma main contre toi, et te réduirai en désolation et en désert.
\VS{4}Je réduirai tes villes en désert, tu ne seras que désolation, et tu sauras que je suis Yahweh.
\VS{5}Parce que tu as eu une inimitié immortelle, et que tu as fait couler le sang des enfants d'Israël à coups d'épée, au temps de leur détresse, au temps où l'iniquité était à son terme\FTNT{Ps. 137:7.}.
\VS{6}C'est pourquoi, je suis vivant, dit le Seigneur Yahweh, je te mettrai à sang, et le sang te poursuivra~; parce que tu n'as point haï le sang, le sang aussi te poursuivra.
\VS{7}Je réduirai la montagne de Séir en désolation et en désert, et j'en éloignerai tous ceux qui la fréquentaient.
\VS{8}Je remplirai de morts ses montagnes~; tes hommes tués par l'épée tomberont sur tes collines, dans tes vallées, et dans tous tes courants d'eau.
\VS{9}Je te réduirai en désolations éternelles, et tes villes ne seront plus habitées~; vous saurez que je suis Yahweh.
\VS{10}Parce que tu as dit~: Les deux nations, et les deux pays seront à moi, nous les posséderons, quand même Yahweh était là~;
\VS{11}à cause de cela, je suis vivant, dit le Seigneur Yahweh, j'agirai avec la colère et la jalousie que tu as montrées dans ta haine contre eux~; et je me ferai connaître au milieu d'eux, quand je te jugerai.
\VS{12}Tu sauras que moi, Yahweh, j'ai entendu toutes les paroles insultantes que tu as prononcées contre les montagnes d'Israël, en disant~: Elles sont dévastées, elles nous sont livrées comme une proie.
\VS{13}Vous m'avez bravé par vos discours, et vous avez multiplié vos paroles contre moi~; je l'ai entendu.
\VS{14}Ainsi parle le Seigneur Yahweh~: Quand toute la terre se réjouira, je te réduirai en désolation.
\VS{15}Comme tu t'es réjouie sur l'héritage de la maison d'Israël et de sa désolation, j'en ferai de même envers toi~; tu ne seras que désolation, ô montagne de Séir~! Ainsi qu'Edom tout entier~; et ils sauront que je suis Yahweh\FTNT{Ab. 1:11-16.}.
\Chap{36}
\TextTitle{Yahweh rétablit Israël}
\VerseOne{}Toi, fils de l'homme, prophétise sur les montagnes d'Israël, et dis~: Montagnes d'Israël, écoutez la parole de Yahweh~!
\VS{2}Ainsi parle le Seigneur Yahweh~: Parce que l'ennemi a dit contre vous~: Ah~! Ah~! Tous ces hauts lieux éternels sont devenus notre possession~!
\VS{3}Prophétise, et dis~: Ainsi parle le Seigneur Yahweh~: Oui, parce qu'on vous a réduites en désolation, et que de toutes parts, on vous a englouties pour que vous soyez la propriété des autres nations, et qu'on vous a exposées à la langue et aux insultes des nations,
\VS{4}à cause de cela, montagnes d'Israël, écoutez la parole du Seigneur Yahweh~: Ainsi parle le Seigneur Yahweh, aux montagnes, aux collines, aux courants d'eau, aux vallées, aux lieux détruits et désolés, et aux villes abandonnées qui sont pillées et sont un sujet de moquerie aux autres nations d'alentour~;
\VS{5}à cause de cela, ainsi parle le Seigneur, Yahweh~: Je parle dans le feu de ma jalousie contre les autres nations, et contre tous ceux d'Edom qui se sont attribués ma terre en possession, avec toute la joie de leur cœur et le mépris de leur âme, afin d'en piller le butin\FTNT{Lé. 25:23~; Es. 14:2~; Jé. 2:7.}.
\VS{6}C'est pourquoi prophétise sur la terre d'Israël, et dis aux montagnes et aux collines, aux courants d'eau et aux vallées~: Ainsi parle le Seigneur Yahweh~: Voici, je parle avec jalousie, et avec fureur, parce que vous portez l'ignominie des nations.
\VS{7}C'est pourquoi ainsi parle le Seigneur Yahweh~: J'ai levé ma main, si les nations qui sont tout autour de vous ne portent leur ignominie.
\VS{8}Mais vous, montagnes d'Israël, vous pousserez vos branches, et vous porterez votre fruit pour mon peuple d'Israël~; car ils sont prêts à venir.
\VS{9}Car me voici, je viens à vous, et je retournerai vers vous, et vous serez labourées et semées.
\VS{10}Je mettrai sur vous des hommes en grand nombre, la maison d'Israël tout entière, et les villes seront habitées, les lieux déserts seront rebâtis.
\VS{11}Je multiplierai sur vous les hommes et les animaux, ils multiplieront et seront féconds~; je veux que vous soyez habitées comme auparavant, et je vous ferai plus de bien que vous n'en avez eu au commencement~; et vous saurez que je suis Yahweh.
\VS{12}Je ferai marcher sur vous des hommes, mon peuple d'Israël, qui vous posséderont, vous serez leur héritage, et vous ne les consumerez plus.
\VS{13}Ainsi parle le Seigneur Yahweh~: Parce qu'on dit de vous~: Tu es un pays qui dévore les hommes, et tu as consumé tes habitants~;
\VS{14}à cause de cela, tu ne dévoreras plus les hommes et ne consumeras plus tes habitants, dit le Seigneur Yahweh.
\VS{15}Je ne te ferai plus entendre l'ignominie des nations, tu ne porteras plus l'opprobre des peuples~; et tu ne feras plus périr tes habitants, dit le Seigneur Yahweh.
\VS{16}Puis la parole de Yahweh me fut adressée en ces mots~:
\VS{17}Fils de l'homme, ceux de la maison d'Israël habitant sur leur terre l'ont souillée par leur voie et par leurs actions~; leur voie est devenue devant moi comme la souillure d'une femme pendant son impureté\FTNT{Lé. 12:2~; Lé. 15:19.}~;
\VS{18}j'ai répandu ma fureur sur eux à cause du sang qu'ils ont répandu sur le pays, et parce qu'ils l'ont souillé par leurs idoles.
\VS{19}Je les ai dispersés parmi les nations, et ils ont été disséminés en divers pays~; je les ai jugés selon leur voie, et selon leurs actions.
\VS{20}Ils sont arrivés chez les nations où ils allaient, ils ont profané mon saint Nom en sorte qu'on disait d'eux~: Ceux-ci sont le peuple de Yahweh, c'est de son pays qu'ils sont sortis\FTNT{Ro. 2:24.}.
\VS{21}Mais j'ai épargné mon saint Nom, que la maison d'Israël avait profané parmi les nations où elle est allée.
\VS{22}C'est pourquoi dis à la maison d'Israël~: Ainsi parle le Seigneur Yahweh~: Je ne le fais point à cause de vous, ô maison d'Israël~! Mais à cause de mon saint Nom, que vous avez profané parmi les nations où vous êtes allés\FTNT{De. 7:7~; De. 9:5~; Es. 43:25~; Ps. 25:11.}.
\VS{23}Je sanctifierai mon grand Nom, qui a été profané parmi les nations, et que vous avez profané au milieu d'elles~; et les nations sauront que je suis Yahweh, dit le Seigneur Yahweh, quand je serai sanctifié par vous, sous leurs yeux.
\VS{24}Je vous retirerai d'entre les nations, je vous rassemblerai de tous les pays, et je vous ramènerai dans votre terre.
\VS{25}Je répandrai sur vous une eau pure\FTNT{Il est question ici de la Nouvelle Alliance (Jé. 31:31-34~; Hé. 8:7-13).}, et vous serez nettoyés~; je vous nettoierai de toutes vos souillures et de toutes vos idoles.
\TextTitle{Prophétie sur la naissance d'en haut}
\VS{26}Je vous donnerai un nouveau cœur, je mettrai au-dedans de vous un Esprit nouveau~; j'ôterai de votre chair le cœur de pierre, et je vous donnerai un cœur de chair\FTNT{Jé. 32:39~; Ez. 11:19~; 2 Co. 3:3.}.
\VS{27}Je mettrai mon Esprit au dedans de vous, je ferai en sorte que vous suiviez mes ordonnances, et que vous observiez et pratiquiez mes lois.
\VS{28}Vous habiterez le pays que j'ai donné à vos pères, vous serez mon peuple, et je serai votre Dieu.
\VS{29}Je vous délivrerai de toutes vos souillures, j'appellerai le blé, je le multiplierai, et je ne vous enverrai plus la famine.
\VS{30}Je multiplierai le fruit des arbres et le revenu des champs, afin que vous ne portiez plus l'opprobre de la famine parmi les nations.
\VS{31}Vous vous souviendrez de votre mauvaise voie et de vos actions, qui n'étaient pas bonnes, et vous prendrez vous-mêmes en dégoût vos iniquités et vos abominations.
\VS{32}Je ne le fais point par amour pour vous, dit le Seigneur Yahweh, sachez-le~! Soyez honteux et confus à cause de votre voie, ô maison d'Israël~!
\VS{33}Ainsi parle le Seigneur Yahweh~: Le jour où je vous aurai purifiés de toutes vos iniquités, je vous ferai habiter dans des villes, et les lieux déserts seront rebâtis.
\VS{34}La terre désolée sera cultivée, tandis qu'elle n'était que désolation aux yeux de tous les passants.
\VS{35}On dira~: Cette terre-ci, qui était désolée, est devenue comme le jardin d'Eden~; et ces villes qui étaient désertes, désolées et détruites, sont fortifiées et habitées\FTNT{Es. 33:20~; Jé. 22:8-9.}.
\VS{36}Les nations qui resteront autour de vous sauront que moi, Yahweh, j'ai rebâti les lieux détruits et planté le pays désolé. Moi, Yahweh, j'ai parlé, et je le ferai.
\VS{37}Ainsi parle le Seigneur Yahweh~: Je me laisserai rechercher par la maison d'Israël. Voici ce que je ferai pour eux~: Je multiplierai les hommes comme un troupeau de brebis.
\VS{38}Les villes qui sont désertes seront remplies de troupeaux d'hommes, pareils aux troupeaux consacrés, aux troupeaux qu'on amène à Jérusalem pendant ses fêtes solennelles. Et ils sauront que je suis Yahweh.
\Chap{37}
\TextTitle{Vision des ossements desséchés, image de la restauration d'Israël}
\VerseOne{}La main de Yahweh fut sur moi, et Yahweh me transporta par son Esprit et me déposa au milieu d'une vallée remplie d'ossements\FTNT{Les ossements desséchés représentent les Israélites dispersés dans les nations.}.
\VS{2}Il me fit passer auprès d'eux, tout autour~; et voici, ils étaient fort nombreux à la surface de cette vallée et complètement secs.
\VS{3}Puis il me dit~: Fils de l'homme, ces os pourront-ils revivre~? Et je répondis~: Seigneur Yahweh, tu le sais.
\VS{4}Alors il me dit~: Prophétise sur ces os, et dis-leur~: Ossements desséchés, écoutez la parole de Yahweh~!
\VS{5}Ainsi parle le Seigneur Yahweh à ces os~: Voici, je ferai entrer un esprit en vous, et vous vivrez\FTNT{Ps. 71:20~; Ro. 8:11.}~;
\VS{6}je mettrai des nerfs sur vous, je ferai croître de la chair sur vous, et j'étendrai la peau sur vous~; puis je mettrai un esprit en vous, et vous vivrez. Et vous saurez que je suis Yahweh.
\VS{7}Alors je prophétisai selon l'ordre que j'avais reçu. Et comme je prophétisais, il se fit un bruit, et voici, il se fit un mouvement, et ces os s'approchèrent les uns des autres.
\VS{8}Puis je regardai, et voici, il vint des nerfs sur eux, et il y crût de la chair, la peau fut étendue par dessus~; mais il n'y avait pas en eux d'esprit.
\VS{9}Alors il me dit~: Prophétise à l'Esprit~! Prophétise, fils de l'homme~! Et dis à l'Esprit~: Ainsi parle le Seigneur Yahweh~: Esprit, viens des quatre vents, et souffle sur ces morts, et qu'ils revivent~!
\VS{10}Je prophétisai donc selon l'ordre qu'il m'avait donné. Et l'Esprit entra en eux, ils reprirent vie, et se tinrent sur leurs pieds~; c'était une armée extrêmement grande.
\VS{11}Alors il me dit~: Fils de l'homme, ces os sont toute la maison d'Israël~; voici, ils disent~: Nos os sont desséchés, et notre attente est perdue, c'en est fait de nous~!
\VS{12}C'est pourquoi prophétise, et dis-leur~: Ainsi parle le Seigneur Yahweh~: Mon peuple, voici, je m'en vais ouvrir vos sépulcres, je vous tirerai hors de vos sépulcres, et vous ferai entrer dans la terre d'Israël\FTNT{Les sépulcres représentent les nations dans lesquelles les Israélites se sont établis. Dieu annonce le retour de son peuple sur la terre d'Israël (Es. 26:19~; Os. 13:14).}.
\VS{13}Et vous, mon peuple, vous saurez que je suis Yahweh quand j'aurai ouvert vos sépulcres, et que je vous aurai tirés hors de vos sépulcres.
\VS{14}Je mettrai mon Esprit en vous, et vous vivrez, je vous rétablirai sur votre terre~; et vous saurez que moi, Yahweh, j'ai parlé et que je l'ai fait, dit Yahweh.
\TextTitle{Prophétie sur l'unité d'Israël}
\VS{15}Puis la parole de Yahweh me fut adressée en ces mots~:
\VS{16}Et toi, fils de l'homme, prends un bois et écris dessus~: Pour Juda, et pour les enfants d'Israël, ses compagnons. Prends encore un autre bois, et écris dessus~: Le bois d'Ephraïm et de toute la maison d'Israël, ses compagnons, pour Joseph.
\VS{17}Puis tu les joindras l'un à l'autre pour ne former qu'un même bois, ils seront unis dans ta main.
\VS{18}Quand les enfants de ton peuple demanderont, en disant~: Ne nous déclareras-tu pas ce que tu veux dire par ces choses~?
\VS{19}Dis-leur~: Ainsi parle le Seigneur Yahweh~: Voici, je m'en vais prendre le bois de Joseph qui est dans la main d'Ephraïm, et des tribus d'Israël, ses compagnons~; je les joindrai au bois de Juda, et j'en formerai un seul bois, ils ne seront qu'un seul bois dans ma main.
\VS{20}Ainsi les bois sur lesquels tu écriras seront dans ta main, sous leurs yeux.
\VS{21}Dis-leur~: Ainsi parle le Seigneur Yahweh~: Voici, je m'en vais prendre les fils d'Israël d'entre les nations parmi lesquelles ils sont allés, je les rassemblerai de toutes parts, et je les ferai entrer dans leur terre.
\VS{22}Je ferai d'eux une seule nation dans le pays, sur les montagnes d'Israël~; un seul roi sera leur roi à tous, ils ne seront plus deux nations, et ils ne seront plus divisés en deux royaumes\FTNT{Es. 11:12-13~; Os. 2:2~; Jn. 10:16.}.
\VS{23}Ils ne se souilleront plus par leurs idoles, ni par leurs infamies, ni par tous leurs crimes, et je les retirerai de toutes leurs demeures dans lesquelles ils ont péché, et je les purifierai~; ils seront mon peuple, et je serai leur Dieu\FTNT{Es. 1:18~; Jé. 33:8~; Jé. 24:7~; Jé. 32:38~; Za. 8:8~; 2 Co. 6:16.}.
\VS{24}David, mon serviteur, sera leur roi, et ils auront tous un seul pasteur~; ils suivront mes ordonnances, ils garderont mes lois et les mettront en pratique.
\VS{25}Ils habiteront dans le pays que j'ai donné à Jacob, mon serviteur, dans lequel vos pères ont habité~; ils y habiteront, dis-je, eux, et leurs fils, et les fils de leurs fils, pour toujours~; et David mon serviteur sera leur prince pour toujours.
\VS{26}Je traiterai avec eux une alliance de paix, et il y aura une alliance éternelle avec eux~; je les établirai, et les multiplierai, je mettrai mon lieu saint au milieu d'eux pour toujours.
\VS{27}Ma demeure sera parmi eux~; je serai leur Dieu, et ils seront mon peuple.
\VS{28}Les nations sauront que je suis Yahweh, qui sanctifie Israël, quand mon lieu saint sera au milieu d'eux pour toujours.
\Chap{38}
\TextTitle{Jugement sur Gog}
\VerseOne{}La parole de Yahweh vint encore à moi en disant~:
\VS{2}Fils de l'homme, tourne ta face vers Gog au pays de Magog\FTNT{Gog est un prince et Magog le pays. Ce chapitre doit être mis en parallèle avec Za. 12:1-4~; Za. 14:1-9~; Mt. 24:14-30~; Ap. 14:14-20~; Ap. 20:8.}, vers le prince de Rosch, de Méschec et de Tubal, et prophétise contre lui~!
\VS{3}Tu diras~: Ainsi parle le Seigneur Yahweh~: Voici, j'en veux à toi, Gog, prince des chefs de Méschec et de Tubal~!
\VS{4}Je te ferai retourner en arrière, et je mettrai des boucles dans tes mâchoires, et te ferai sortir avec toute ton armée, avec les chevaux et les cavaliers, tous parfaitement bien équipés, une grande multitude portant le grand et le petit bouclier, et tous maniant l'épée~;
\VS{5}ceux de Perse, d'Ethiopie, et de Puth avec eux, qui tous ont des boucliers et des casques~;
\VS{6}Gomer et toutes ses troupes, la maison de Togarma à l'extrême nord, avec toutes ses troupes, et plusieurs peuples avec toi.
\VS{7}Apprête-toi, tiens-toi prêt, toi, et toute la multitude assemblée autour de toi~! Sois leur chef~!
\VS{8}Après plusieurs jours, tu seras à leur tête, et dans la suite des années, tu marcheras contre le pays dont les habitants, délivrés de l'épée, auront été rassemblés d'entre plusieurs peuples sur les montagnes d'Israël longtemps désertes~; en ce pays qui aura été retiré d'entre les peuples où tous habiteront en sûreté.
\VS{9}Tu monteras, tu viendras comme une dévastation, tu seras comme une nuée pour couvrir la terre, toi, toutes tes troupes, et plusieurs peuples avec toi\FTNT{Da. 11:40.}.
\VS{10}Ainsi parle le Seigneur Yahweh~: Il arrivera dans ces jours-là, que des pensées s'élèveront dans ton cœur, et que tu formeras un dessein pernicieux.
\VS{11}Car tu diras~: Je monterai contre le pays dont les villes sont sans murailles, j'envahirai ceux qui sont en repos, qui habitent en sécurité, qui demeurent tous dans des villes sans murs, lesquelles n'ont ni barres ni portes\FTNT{Jé. 49:31.}~;
\VS{12}pour enlever un grand butin et faire un grand pillage, pour remettre ta main sur les déserts qui de nouveau étaient habités, et sur le peuple rassemblé d'entre les nations, ayant des troupeaux et des biens, et occupant les lieux élevés du pays.
\VS{13}Séba et Dedan, les marchands de Tarsis, et tous ses lionceaux, te diront~: Ne vas-tu pas pour faire du butin, et n'as-tu pas assemblé ta multitude pour faire un grand pillage, pour emporter de l'argent et de l'or, pour prendre le bétail et les biens, pour enlever un grand butin~?
\VS{14}Toi donc, fils de l'homme, prophétise, et dis à Gog~: Ainsi parle le Seigneur Yahweh~: En ce jour-là, quand mon peuple d'Israël habitera en sécurité, ne le sauras-tu pas~?
\VS{15}Ne viendras-tu pas de ton lieu, de l'extrême nord, toi, et plusieurs peuples avec toi, tous montés sur des chevaux, une grande multitude, et une grosse armée~?
\VS{16}Ne monteras-tu pas contre mon peuple d'Israël, comme une nuée pour couvrir la terre~? Dans la suite de ces jours, je te ferai venir sur ma terre, afin que les nations me connaissent, quand je serai sanctifié par toi sous leurs yeux, ô Gog~!
\VS{17}Ainsi parle le Seigneur Yahweh~: N'est-ce pas de toi que j'ai parlé autrefois par mes serviteurs, les prophètes d'Israël, qui ont prophétisé dans ces jours-là pendant plusieurs années, qu'on te ferait venir contre eux~?
\VS{18}Mais il arrivera dans ce jour-là, au jour de la venue de Gog sur la terre d'Israël, dit le Seigneur Yahweh, que ma colère éclatera.
\VS{19}Je le déclare, dans ma jalousie, dans l'ardeur de ma fureur, en ce jour-là, il y aura une grande agitation sur la terre d'Israël.
\VS{20}Les poissons de la mer, les oiseaux des cieux et les bêtes des champs, et tous les reptiles qui rampent sur la terre, et tous les hommes qui sont sur la surface de la terre seront épouvantés par ma présence~; les montagnes seront renversées, les parois des rochers tomberont, et tous les murs chuteront par terre.
\VS{21}J'appellerai contre lui l'épée sur toutes mes montagnes, dit le Seigneur Yahweh~; l'épée de chacun d'eux sera contre son frère.
\VS{22}J'entrerai en jugement avec lui par la peste, et par le sang~; je ferai pleuvoir sur lui, sur ses troupes, et sur les grands peuples qui seront avec lui, des torrents d'eau, des pierres de grêle, du feu et du soufre\FTNT{Ps. 11:6~; Ap. 8:7~; Ap. 11:19~; Ap. 16:21.}.
\VS{23}Je me glorifierai, je me sanctifierai, je serai connu aux yeux de plusieurs nations~; et elles sauront que je suis Yahweh.
\Chap{39}
\TextTitle{Jugement sur Gog, suite}
\VerseOne{}Toi donc, fils de l'homme, prophétise contre Gog, et dis~: Ainsi parle le Seigneur Yahweh~: Voici, j'en veux à toi, Gog, prince des chefs de Méschec et de Tubal~!
\VS{2}Je te ferai retourner en arrière, je te conduirai, je te ferai monter de l'extrême nord, et je t'amènerai sur les montagnes d'Israël.
\VS{3}Car je frapperai ton arc dans ta main gauche, et je ferai tomber tes flèches de ta main droite.
\VS{4}Tu tomberas sur les montagnes d'Israël, toi et toutes tes troupes, et les peuples qui seront avec toi~; je te livrerai aux oiseaux de proie, à tout ce qui a des ailes, et aux bêtes des champs, pour en être dévoré.
\VS{5}Tu tomberas sur la face des champs, parce que j'ai parlé, dit le Seigneur Yahweh.
\VS{6}Je mettrai le feu dans Magog, et parmi ceux qui demeurent en sécurité dans les îles~; et ils sauront que je suis Yahweh.
\VS{7}Je ferai connaître mon saint Nom au milieu de mon peuple d'Israël~; et je ne profanerai plus mon saint Nom~; les nations sauront que je suis Yahweh, le Saint d'Israël.
\VS{8}Voici, cela arrive et sera fait, dit le Seigneur Yahweh~; c'est ici le jour dont j'ai parlé.
\VS{9}Les habitants des villes d'Israël sortiront, allumeront le feu, brûleront les armes, les petits et les grands boucliers, les arcs, les flèches, les bâtons qu'on lance de la main, et les javelots~; ils en feront du feu pendant sept ans.
\VS{10}On n'apportera point du bois des champs, et on n'en coupera point dans les forêts, parce qu'ils feront du feu de ces armes, lorsqu'ils dépouilleront ceux qui les ont dépouillés, et qu'ils pilleront ceux qui les ont pillés, dit le Seigneur Yahweh.
\VS{11}Il arrivera ce jour-là que je donnerai à Gog dans ces quartiers-là un lieu pour sépulcre en Israël, à savoir la vallée des passants, qui est au-devant de la mer~; elle réduira les passants au silence. On enterrera là Gog, et toute la multitude de son peuple, et on l'appellera la vallée d'Hamon-Gog\FTNT{«~La vallée d'Hamon-Gog~» signifie «~La vallée de la multitude de Gog~».}.
\VS{12}Ceux de la maison d'Israël les enterreront, et cela durera sept mois, afin de purifier le pays.
\VS{13}Tout le peuple du pays les enterrera, et il en aura du renom, le jour où je serai glorifié, dit le Seigneur Yahweh.
\VS{14}Ils mettront à part des gens qui ne feront autre chose que parcourir le pays, et qui enterreront, avec l'aide des passants, les corps restés à la surface de la terre, pour la purifier, et ils seront à la recherche pendant sept mois.
\VS{15}Ils parcourront le pays~; et celui qui verra l'os d'un homme, dressera auprès de lui un signal~; jusqu'à ce que les fossoyeurs l'aient enterré dans la vallée d'Hamon-Gog.
\VS{16}Il y aura aussi une ville nommée Hamona\FTNT{Hamona signifie «~multitude~».}, et on nettoiera le pays.
\VS{17}Toi donc, fils de l'homme, ainsi parle le Seigneur Yahweh~: Dis aux oiseaux de toutes espèces, et à toutes les bêtes des champs~: Assemblez-vous et venez~; amassez-vous de toutes parts vers mon sacrifice que je fais pour vous, qui est un grand sacrifice sur les montagnes d'Israël~! Vous mangerez de la chair, et vous boirez du sang.
\VS{18}Vous mangerez la chair des hommes puissants, et vous boirez le sang des princes de la terre, le sang des moutons, des agneaux, des boucs et des veaux engraissés sur le Basan\FTNT{Es. 34:6~; Jé. 46:10~; So. 1:7~; Job 39:33~; Mt. 24:28.}.
\VS{19}Vous mangerez de la graisse jusqu'à en être rassasiés, et vous boirez du sang jusqu'à en être ivres, de la graisse et du sang de mon sacrifice, que j'aurai sacrifié pour vous.
\VS{20}Vous serez rassasiés à ma table, de chevaux et de bêtes d'attelage, d'hommes forts et de tous hommes de guerre, dit le Seigneur Yahweh.
\VS{21}Je mettrai ma gloire parmi les nations, et toutes les nations verront mon jugement que j'exercerai, et comment je mettrai ma main sur eux.
\VS{22}La maison d'Israël connaîtra dès ce jour-là, et dans la suite, que je suis Yahweh, leur Dieu.
\VS{23}Les nations sauront que la maison d'Israël a été emmenée en captivité à cause de son iniquité, parce qu'ils avaient péché contre moi, et que je leur ai caché ma face~; aussi je les ai livrés entre les mains de leurs ennemis pour qu'ils périssent par l'épée\FTNT{De. 31:17-18~; Ps. 13:2.}.
\VS{24}Je leur avais fait selon leurs souillures, et selon leurs crimes, et je leur avais caché ma face.
\TextTitle{Rétablissement et conversion d'Israël}
\VS{25}C'est pourquoi ainsi parle le Seigneur Yahweh~: Maintenant je ramènerai la captivité de Jacob et j'aurai pitié de toute la maison d'Israël, et je serai jaloux de mon saint Nom,
\VS{26}après avoir porté leur ignominie, et tout leur crime lorsqu'ils avaient péché contre moi, quand ils demeuraient en sûreté dans leur terre, sans qu'il y eût personne pour les effrayer.
\VS{27}Parce que je les ramènerai d'entre les peuples, que je les rassemblerai des pays de leurs ennemis, et que je serai sanctifié par eux sous les yeux de plusieurs nations.
\VS{28}Ils sauront que je suis Yahweh, leur Dieu, lorsqu'après les avoir enlevés parmi les nations, je les rassemblerai sur leurs terres, et que je n'en laisserai chez elles aucun d'eux.
\VS{29}Je ne leur cacherai plus ma face, car je répandrai mon Esprit sur la maison d'Israël, dit le Seigneur Yahweh\FTNT{Joë. 2:28~; Ac. 2:17.}.
\Chap{40}
\TextTitle{Mesures du futur temple}
\VerseOne{}Dans la vingt-cinquième année de notre captivité, au commencement de l'année, au dixième jour du mois, la quatorzième année après que la ville fut prise, en ce même jour, la main de Yahweh fut sur moi, et il m'amena là.
\VS{2}Il m'amena par des visions de Dieu, au pays d'Israël, et me posa sur une montagne fort élevée, sur laquelle du côté sud il y avait comme une ville construite.
\VS{3}Après qu'il m'y fît entrer, voici un homme, dont l'aspect était comme de l'airain, qui avait dans sa main un cordeau de lin, et une canne à mesurer, et qui se tenait debout à la porte.
\VS{4}Cet homme me parla ainsi~: Fils de l'homme, regarde de tes yeux, écoute de tes oreilles~! Et applique ton cœur à toutes les choses que je m'en vais te faire voir, car tu as été amené ici afin que je te les fasse voir, et que tu fasses savoir à la maison d'Israël toutes les choses que tu vas voir.
\VS{5}Voici, un mur extérieur entourait la maison. Cet homme avait dans la main une canne à mesurer longue de six coudées, chaque coudée étant d'une coudée normale et une largeur de main en plus. Il mesura la largeur de ce mur bâti, laquelle était d'une canne, et sa hauteur d'une autre canne.
\VS{6}Puis il vint vers la porte orientale, et monta par ses étages. Il mesura l'un des poteaux de la porte d'une canne en largeur, et l'autre poteau d'une autre canne en largeur.
\VS{7}Puis il mesura chaque chambre d'une canne en longueur, et d'une canne en largeur. L'espace entre les deux chambres était de cinq coudées. Il mesura d'une canne chacun des poteaux de la porte près du vestibule qui menait à la porte la plus intérieure.
\VS{8}Puis il mesura d'une canne le vestibule qui menait à la porte la plus intérieure.
\VS{9}Il mesura de huit coudées le vestibule de la porte et ses poteaux, le vestibule de la porte était en dedans.
\VS{10}Les chambres de la porte orientale étaient au nombre de trois d'un côté et de trois de l'autre, toutes les trois avaient la même mesure, et les poteaux d'un côté et de l'autre étaient d’une même mesure.
\VS{11}Puis il mesura de dix coudées la largeur de l'ouverture de la première porte, et de treize coudées la longueur de la même porte.
\VS{12}Ensuite, il mesura d'un côté un espace limité au-devant des chambres d'une coudée, et une autre coudée d'espace limité de l'autre côté~; chaque chambre avait six coudées d'un côté, et six coudées de l'autre.
\VS{13}Après cela, il mesura le portail depuis le toit d'une chambre jusqu'au toit de l'autre, de la largeur de vingt-cinq coudées entre les deux ouvertures opposées.
\VS{14}Il compta soixante coudées pour les poteaux, près desquels était une cour, autour de la porte.
\VS{15}L'espace entre la porte d'entrée et le vestibule de la porte intérieure était de cinquante coudées.
\VS{16}Il y avait des fenêtres closes aux chambres et à leurs poteaux, à l'intérieur de la porte tout autour. Il y avait aussi des fenêtres dans les vestibules tout autour vers l'intérieur, des palmes étaient sculptées sur les poteaux.
\VS{17}Il me mena dans le parvis extérieur, où se trouvaient des chambres et un pavé tout autour. Il y avait trente chambres sur ce pavé.
\VS{18}Le pavé était au côté des portes et répondait à la longueur des portes~; c'était le pavé inférieur.
\VS{19}Il mesura la largeur du parvis depuis la porte qui menait vers le bas et en dehors jusqu'au parvis intérieur. Il y avait cent coudées à l'orient et au nord.
\VS{20}Après cela, il mesura la longueur et la largeur de la porte nord du parvis extérieur.
\VS{21}Quant aux chambres, au nombre de trois d'un côté et trois de l'autre, ses poteaux et ses vestibules avaient la même mesure que la première porte, cinquante coudées en longueur, et vingt-cinq coudées en largeur.
\VS{22}Ses fenêtres, son vestibule, et ses palmes avaient la même mesure que la porte orientale~; on y montait par sept étages, devant lesquels était son vestibule.
\VS{23}La porte du parvis intérieur était vis-à-vis de la première porte du nord, et vis-à-vis de la porte orientale. Il mesura depuis une porte jusqu'à l'autre cent coudées.
\VS{24}Après cela, il me conduisit du côté sud, où se trouvait la porte méridionale. Il en mesura les poteaux et les vestibules qui avaient la même mesure.
\VS{25}Cette porte et ses vestibules avaient des fenêtres tout autour, comme les autres fenêtres, cinquante coudées de long, et vingt-cinq coudées de large.
\VS{26}On y montait par sept étages, devant lesquels était son vestibule~; il y avait de chaque côté des palmes sur ses poteaux.
\VS{27}Pareillement, le parvis intérieur avait sa porte du côté sud~; il mesura, d'une porte à l'autre au sud, cent coudées.
\VS{28}Après cela il me fit entrer dans le parvis intérieur par la porte sud, et il mesura la porte sud, selon les mesures précédentes.
\VS{29}Ses chambres, ses poteaux et ses vestibules avaient la même mesure. Cette porte et ses vestibules avaient des fenêtres tout autour, cinquante coudées de long, et vingt-cinq coudées de large.
\VS{30}Il y avait tout autour des vestibules de vingt-cinq coudées de long, et cinq coudées de large.
\VS{31}Les vestibules de la porte aboutissaient au parvis extérieur~; il y avait des palmes sur ses poteaux, et huit étages pour y monter.
\VS{32}Il me conduisit dans le parvis intérieur, par l'entrée orientale. Il mesura la porte, qui avait la même mesure.
\VS{33}Ses chambres, ses poteaux et ses vestibules avaient la même mesure. Cette porte et ses vestibules avaient des fenêtres tout autour, cinquante coudées de long, et vingt-cinq de large.
\VS{34}Ses vestibules aboutissaient au parvis extérieur~; il y avait de chaque côté des palmes sur ses poteaux, et huit étages pour y monter.
\VS{35}Il me conduisit vers la porte nord, il la mesura et trouva la même mesure.
\VS{36}Ainsi qu'à ses chambres, à ses poteaux et à ses vestibules~; elle avait des fenêtres tout autour, cinquante coudées de long, et vingt-cinq coudées de large.
\VS{37}Ses vestibules aboutissaient au parvis extérieur~; il y avait de chaque côté des palmes sur ses poteaux, et huit étages pour y monter.
\VS{38}Il y avait une chambre qui s'ouvrait vers les poteaux des portes, et où l'on devait laver les holocaustes.
\VS{39}Il y avait aussi dans le vestibule de la porte de chaque côté deux tables, pour y égorger les bêtes qu'on sacrifierait pour l'holocauste, et le sacrifice pour l'expiation et le sacrifice pour la culpabilité.
\VS{40}Vers l'un des côtés de la porte, au dehors, vers le lieu où l'on montait, à l'entrée de la porte nord, il y avait deux tables~; et de l'autre côté, vers le vestibule de la porte, deux autres tables.
\VS{41}Il se trouvait ainsi, aux côtés de la porte, quatre tables d'une part et quatre tables de l'autre, en tout huit tables, sur lesquelles on devait abattre les victimes.
\VS{42}Les quatre tables qui étaient pour l'offrande entièrement consumée, étaient en pierres de taille, de la longueur d'une coudée et demie, et de la largeur d'une coudée et demie, et de la hauteur d'une coudée~; et même on devait poser sur elles les instruments avec lesquels on tuait les victimes pour les offrandes entièrement consumées, et les autres sacrifices.
\VS{43}Il y avait aussi à l'intérieur de la maison tout autour, des chevilles pour accrocher, larges d'une paume, bien adaptées, d'où l'on apportait la chair des sacrifices sur les tables.
\TextTitle{Répartition des pièces du futur temple}
\VS{44}En dehors de la porte intérieure, il y avait des chambres pour les chantres dans le parvis intérieur, l'une était à côté de la porte nord et avait la face au sud, l'autre était à côté de la porte orientale et avait la face au nord.
\VS{45}Il me dit~: Ces chambres, dont la face est au sud, sont pour les prêtres qui ont la charge de la maison.
\VS{46}Mais ces chambres, dont la face est au nord, sont pour les prêtres qui ont la charge de l'autel, qui sont les fils de Tsadok, qui, parmi les fils de Lévi, s'approchent de Yahweh pour faire son service.
\VS{47}Puis il mesura un parvis de la longueur et de la largeur de cent coudées, en carré~; et l'autel était devant la maison.
\VS{48}Ensuite, il me fit entrer dans le vestibule de la maison~; et il mesura les poteaux du vestibule de cinq coudées d'un côté, et de cinq coudées de l'autre, puis la largeur de la porte de trois coudées d'un côté, et de trois coudées de l'autre.
\VS{49}Le vestibule avait une longueur de vingt coudées, et une largeur de onze coudées~; on y montait par des étages. Il y avait des colonnes près des poteaux, l'une d'un côté, et l'autre de l'autre.
\Chap{41}
\TextTitle{Description du temple}
\VerseOne{}Puis il me fit entrer dans le temple, et il mesura des poteaux de six coudées de largeur d'un côté, et de six coudées de largeur de l'autre côté, largeur de la tente.
\VS{2}Ensuite, il mesura la largeur de l'ouverture de la porte qui était de dix coudées, et les côtés de l'ouverture de cinq coudées, d'une part, et de cinq coudées de l'autre part. Puis il mesura la longueur du temple, quarante coudées, et la largeur, vingt coudées.
\VS{3}Il entra à l'intérieur, et il mesura un poteau d'une ouverture de porte, deux coudées, la hauteur de cette ouverture, six coudées, et la largeur de cette ouverture, sept coudées.
\VS{4}Puis il mesura une longueur de vingt coudées, et une largeur de vingt coudées en face du temple~; et il me dit~: C'est ici le Saint des saints.
\VS{5}Il mesura l'épaisseur du mur de la maison, qui fut de six coudées, et la largeur des chambres qui étaient tout autour de la maison, de quatre coudées.
\VS{6}Les chambres latérales étaient les unes à côté des autres, au nombre de trente, et il y avait trois poutres~; elles entraient dans un mur construit pour ces chambres tout autour de la maison, elles y étaient appuyées sans entrer dans le mur même de la maison.
\VS{7}Les chambres occupaient plus d'espace, à mesure qu'elles s'élevaient, et l'on allait en tournant, car on montait autour de la maison par un escalier tournant. Il y avait plus d'espace dans le haut de la maison, et l'on montait de l'étage inférieur à l'étage supérieur par celui du milieu.
\VS{8}Je considérai la hauteur autour de la maison. Les chambres latérales, à partir de leur fondement, avaient une canne pleine, six grandes coudées.
\VS{9}La largeur du mur extérieur des chambres latérales était de cinq coudées~; l'espace libre entre les chambres latérales de la maison,
\VS{10}et les chambres autour de la maison avait une largeur de vingt coudées.
\VS{11}L'ouverture des chambres latérales donnait sur l'espace libre, une ouverture au nord, et une autre ouverture au sud~; la largeur de l'espace libre était de cinq coudées tout autour.
\VS{12}Le bâtiment qui était devant la place vide, du côté de l'occident, avait une largeur de soixante-dix coudées, un mur de cinq coudées de largeur tout autour, et une longueur de quatre-vingt-dix coudées.
\VS{13}Il mesura la maison, qui avait cent coudées de longueur~; la place vide, le bâtiment et les murs avaient une longueur de cent coudées.
\VS{14}La largeur de la face de la maison et de la place vide, du côté oriental, était de cent coudées.
\VS{15}Et il mesura la longueur du bâtiment devant la place vide, sur le derrière, et ses galeries de côté et d'autre, et elle était de cent coudées~; puis il y avait le temple intérieur, et les allées du parvis.
\VS{16}Les seuils, les fenêtres closes, les galeries du pourtour aux trois étages, en face des seuils, étaient recouverts de bois tout autour. Depuis le sol jusqu'aux fenêtres fermées,
\VS{17}jusqu'au-dessus des ouvertures, et jusqu'à la maison au-dedans comme au dehors, tout le mur du pourtour, à l'intérieur et à l'extérieur, tout était d'après la mesure,
\VS{18}et fait de chérubins et de palmes. Il y avait une palme entre deux chérubins, et chaque chérubin avait deux faces.
\VS{19}Une face d'homme était tournée vers la palme d'un côté, et une face de jeune lion était tournée vers la palme de l'autre côté~; il en était ainsi tout autour de la maison.
\VS{20}Depuis le sol jusqu'au-dessus des ouvertures il y avait des chérubins et des palmes et aussi sur le mur du temple.
\VS{21}Les poteaux du temple étaient carrés~; et la face du lieu saint avait la même apparence.
\VS{22}L'autel était de bois, de la hauteur de trois coudées, et de deux coudées de longueur~; ses angles, ses pieds et ses côtés étaient de bois. Puis il me dit~: C'est ici la table qui est devant Yahweh.
\VS{23}Le temple et le lieu saint avaient deux portes.
\VS{24}Il y avait deux portes, deux battants, qui tous deux tournaient sur les portes, deux battants pour une porte et deux pour l'autre.
\VS{25}Il y avait aussi des chérubins et des palmes façonnés sur les portes du temple, comme sur les murs. Un entablement en bois était sur le front du vestibule en dehors.
\VS{26}Il y avait des fenêtres fermées, et des palmes de part et d'autre, ainsi qu'aux côtés du vestibule, aux chambres latérales de la maison, et aux entablements.
\Chap{42}
\TextTitle{Mesures supplémentaires du temple}
\VerseOne{}Après cela, il me fit sortir vers le parvis extérieur, du côté nord~; et il me conduisit vers les chambres qui étaient vis-à-vis de la place vide et vis-à-vis du bâtiment, au nord.
\VS{2}Sur la face où se trouvait une ouverture au nord, il y avait une longueur de cent coudées, et la largeur était de cinquante coudées.
\VS{3}C'était vis-à-vis des vingt coudées du parvis intérieur, et vis-à-vis du pavé extérieur, là où se trouvaient les galeries des trois étages.
\VS{4}Devant les chambres, il y avait une promenade large de dix coudées, et une voie d'une coudée~; leurs ouvertures donnaient au nord.
\VS{5}Les chambres supérieures étaient plus étroites que les inférieures et que celles du milieu du bâtiment, parce que les galeries leur ôtaient de la place.
\VS{6}Car elles étaient à trois étages, et n'avaient point de colonnes, comme les colonnes des parvis~; c'est pourquoi à partir du sol, les chambres du haut étaient plus étroites que celles du bas et du milieu.
\VS{7}Le mur extérieur parallèle aux chambres, du côté du parvis extérieur devant les chambres, avait cinquante coudées de long.
\VS{8}Car la longueur des chambres du côté du parvis extérieur était de cinquante coudées. Mais sur la face du temple, il y avait cent coudées.
\VS{9}Au bas de ces chambres était l'entrée orientale, quand on y venait du parvis extérieur.
\VS{10}Il y avait encore des chambres sur la largeur du mur du parvis du côté oriental, vis-à-vis de la place vide et vis-à-vis du bâtiment.
\VS{11}Devant elles, il y avait un chemin, comme devant les chambres qui étaient du côté nord. La longueur et la largeur étaient les mêmes~; leurs issues, leur disposition et leurs ouvertures étaient semblables.
\VS{12}Il en était de même pour les ouvertures des chambres du côté sud. Il y avait une ouverture à la tête du chemin, du chemin qui se trouvait droit devant le mur du côté oriental par où l'on y entrait.
\VS{13}Après cela, il me dit~: Les chambres du parvis nord et les chambres du parvis sud, qui sont devant la place vide, ce sont les chambres du lieu saint, où les prêtres qui s'approchent de Yahweh, mangeront les choses très saintes. Ils déposeront là les choses très saintes, à savoir les gâteaux, les offrandes pour l'expiation et les offrandes pour la culpabilité~; car ce lieu est saint.
\VS{14}Quand les prêtres seront entrés, ils ne sortiront point du lieu saint pour venir au parvis extérieur, mais ils déposeront là leurs vêtements avec lesquels ils font le service~; car ces vêtements sont saints~; ils en mettront d'autres pour s'approcher du peuple.
\VS{15}Lorsqu'il eut achevé de mesurer la maison intérieure, il me fit sortir par la porte qui était du côté oriental, puis il mesura l'enceinte tout autour.
\VS{16}Il mesura le côté oriental avec la canne qui servait de mesure, et il y avait tout autour cinq cents cannes.
\VS{17}Ensuite, il mesura le côté nord, avec la canne qui servait de mesure, et il y avait tout autour cinq cents cannes.
\VS{18}Puis il mesura le côté sud avec la canne qui servait de mesure, et il y avait cinq cents cannes.
\VS{19}Il se tourna du côté occidental, et mesura cinq cents cannes avec la canne qui servait de mesure.
\VS{20}Il mesura des quatre côtés le mur formant l'enceinte de la maison~; la longueur était de cinq cents cannes, et la largeur de cinq cents cannes, ce mur marquait la séparation entre le lieu saint et le profane.
\Chap{43}
\TextTitle{La gloire de Yahweh remplit la maison\FTNTT{Cp. Ez. 11:22-24.}}
\VerseOne{}Puis il me ramena à la porte, à la porte qui était du côté oriental.
\VS{2}Et voici, la gloire du Dieu d'Israël s'avançait de l'orient. Sa voix était pareille au bruit des grandes eaux, et la terre resplendissait de sa gloire\FTNT{Ap. 1:15.}.
\VS{3}La vision que j'eus alors était semblable à celle que j'avais vue lorsque j'étais venu pour détruire la ville, ces visions étaient comme la vision que j'avais vue sur le fleuve de Kebar~; et je me prosternai le visage contre terre.
\VS{4}Puis la gloire de Yahweh entra dans la maison par la porte qui était du côté oriental.
\VS{5}L'Esprit m'enleva et me fit entrer dans le parvis intérieur, et voici la gloire de Yahweh remplissait la maison.
\TextTitle{Le trône de Yahweh}
\VS{6}Je l'entendis s'adressant à moi depuis la maison, et l'homme qui me conduisait était debout près de moi.
\VS{7}Yahweh me dit~: Fils de l'homme, c'est ici le lieu de mon trône, et le lieu des plantes de mes pieds, dans lequel je ferai ma demeure éternellement parmi les enfants d'Israël~; et la maison d'Israël ne souillera plus mon saint Nom, ni eux, ni leurs rois, par leurs fornications~; mais ils souilleront leurs hauts lieux par les cadavres de leurs rois.
\VS{8}Car ils ont mis leur seuil près de mon seuil, et leur poteau près de mon poteau, il y avait un mur entre moi et eux~; ils ont souillé mon saint Nom par leurs abominations qu'ils ont faites~; c'est pourquoi je les ai consumés dans ma colère.
\VS{9}Maintenant, ils rejetteront loin de moi leurs adultères et les cadavres de leurs rois, et je ferai ma demeure éternellement parmi eux.
\VS{10}Toi donc, fils de l'homme, montre ce temple à la maison d'Israël~; et qu'ils soient confus à cause de leurs iniquités~; et qu'ils aient honte de leur iniquité.
\VS{11}S'ils rougissent de tout ce qu'ils ont fait, fais-leur connaître la forme de ce temple, sa disposition, avec ses sorties et ses entrées, toutes ses figures et toutes ses ordonnances, toutes ses formes, toutes ses lois, et écris-les sous leurs yeux, afin qu'ils gardent toutes ses formes, et toutes les ordonnances, et qu'ils les pratiquent.
\VS{12}Telle est la loi de la maison. Sur le sommet de la montagne, tout le territoire sera un lieu très saint tout autour. Voilà donc la loi de la maison.
\TextTitle{L'autel pour les holocaustes et les sacrifices}
\VS{13}Voici les mesures de l'autel, d'après les coudées dont chacune était d'une largeur de main plus longue que la coudée ordinaire. Le fond avait une coudée de hauteur et une coudée de largeur, et le rebord qui terminait son contour avait un empan de largeur~; c'était le dos de l'autel.
\VS{14}Depuis le fond sur le sol jusqu'à l'encadrement inférieur, il y avait deux coudées, et une coudée de largeur, et depuis le petit jusqu'au grand encadrement, il y avait quatre coudées et une coudée de largeur.
\VS{15}L'autel avait quatre coudées~; et quatre cornes s'élevaient de l'autel.
\VS{16}L'autel avait douze coudées de longueur, douze coudées de largeur, et formait un carré par ses quatre côtés.
\VS{17}L'encadrement avait quatorze coudées de longueur sur quatorze coudées de largeur à ses quatre côtés, le rebord qui terminait son contour avait une demi-coudée, le fond avait une coudée tout autour, et les étages étaient tournés vers l'orient.
\VS{18}Il me dit~: Fils de l'homme, ainsi parle le Seigneur Yahweh~: Ce sont ici les lois au sujet de l'autel pour le jour où on le fera, afin qu'on y offre l'holocauste, et qu'on y répande le sang.
\VS{19}Tu donneras aux prêtres, aux Lévites, qui sont de la race de Tsadok, et qui s'approchent de moi, dit le Seigneur Yahweh, afin qu'ils y fassent mon service, un jeune veau en sacrifice pour le péché.
\VS{20}Et tu prendras de son sang, et en mettras sur les quatre cornes de l'autel, et sur les quatre angles de l'encadrement et sur le rebord qui l'entoure, ainsi tu purifieras l'autel, et tu feras la propitiation pour lui\FTNT{Ex. 29:36-39.}.
\VS{21}Tu prendras le jeune taureau expiatoire, et on le brûlera dans un lieu réservé de la maison, en dehors du lieu saint.
\VS{22}Le second jour, tu offriras en expiation un bouc, sans défaut, et on purifiera l'autel comme on l'aura purifié avec le jeune taureau.
\VS{23}Quand tu auras achevé de purifier l'autel, tu offriras un jeune taureau sans défaut, et un bélier du troupeau sans défaut.
\VS{24}Tu les offriras devant Yahweh, et les prêtres jetteront du sel par dessus, et les offriront en holocauste à Yahweh\FTNT{Lé. 2:13.}.
\VS{25}Durant sept jours, tu sacrifieras chaque jour un bouc comme victime expiatoire, et les prêtres sacrifieront un jeune taureau et un bélier du troupeau sans défaut.
\VS{26}Pendant sept jours, les prêtres feront la propitiation pour l'autel, on le purifiera et chacun d'eux sera consacré\FTNT{Le terme «~consacré~» veut dire littéralement «~remplir sa main~». Voir aussi Jg. 17:5,12.}
\VS{27}Lorsque ces jours seront accomplis, dès le huitième jour, et à l'avenir, les prêtres offriront sur cet autel vos holocaustes et vos sacrifices d'offrande de paix\FTNT{Voir commentaire en Lé. 3:1.}. Et je serai apaisé envers vous, dit le Seigneur Yahweh.
\Chap{44}
\TextTitle{La porte fermée du sanctuaire}
\VerseOne{}Puis il me ramena vers la porte extérieure du lieu saint, du côté oriental, mais elle était fermée.
\VS{2}Yahweh me dit~: Cette porte-ci sera fermée, et ne sera point ouverte, personne n'y passera, parce que Yahweh, le Dieu d'Israël, est entré par cette porte~; elle sera donc fermée\FTNT{Ap. 3:8.}.
\VS{3}Elle sera pour le prince~; le prince sera le seul qui s'y assiéra pour manger le pain devant Yahweh~; il entrera par le chemin du vestibule de la porte, et sortira par le même chemin.
\TextTitle{La gloire dans la maison de Yahweh}
\VS{4}Il me fit revenir par le chemin de la porte nord, jusque sur le devant de la maison. Je regardai, et voici, la gloire de Yahweh avait rempli la maison de Yahweh, et je me prosternai sur ma face.
\VS{5}Alors Yahweh me dit~: Fils de l'homme, applique ton cœur, et regarde de tes yeux, écoute de tes oreilles tout ce dont je vais te parler, concernant toutes les ordonnances et toutes les lois qui concernent la maison de Yahweh. Applique ton cœur en ce qui concerne l'entrée de la maison et toutes les sorties du lieu saint.
\VS{6}Tu diras aux rebelles, à la maison d'Israël~: Ainsi parle le Seigneur Yahweh~: Maison d'Israël~! Assez de toutes vos abominations~!
\VS{7}Vous avez fait entrer les fils de l'étranger, incirconcis de cœur et incirconcis de chair, pour être dans mon lieu saint, pour profaner ma maison. Vous avez offert mon pain, la graisse et le sang, à toutes vos abominations, vous avez enfreint mon alliance\FTNT{Lé. 3:11-16~; Lé. 22:25~; No. 28:2.}.
\VS{8}Et vous n'avez point ordonné que mes choses saintes soient observées, mais vous avez établi comme il vous a plu dans mon sanctuaire, des gens pour y être les gardes des choses que j'avais commandé de garder.
\TextTitle{Recommandations aux prêtres du futur temple}
\VS{9}Ainsi parle le Seigneur Yahweh~: Pas un de tous ceux qui seront fils d'étranger, incirconcis de cœur et incirconcis de chair, n'entrera dans mon lieu saint, pas même un d'entre tous les fils d'étrangers qui seront parmi les enfants d'Israël.
\VS{10}Mais les Lévites qui se sont éloignés de moi, lorsque Israël s'est égaré, et qui se sont égarés de moi pour suivre leurs idoles, porteront la peine de leur iniquité.
\VS{11}Toutefois, ils seront employés dans mon lieu saint aux charges qui sont vers les portes de la maison, et ils feront le service de la maison~; ils égorgeront pour le peuple les bêtes pour l'holocauste, et pour les autres sacrifices, et se tiendront prêts devant lui pour le servir.
\VS{12}Parce qu'ils l'ont servi se présentant devant leurs idoles, et qu'ils ont fait tomber dans l'iniquité la maison d'Israël, à cause de cela j'ai levé ma main en jurant contre eux, dit le Seigneur Yahweh, qu'ils porteront la peine de leur iniquité.
\VS{13}Ils n'approcheront plus de moi pour exercer la prêtrise, ni pour approcher mes sanctuaires, mes lieux très saints~; mais ils porteront leur confusion et leurs abominations qu'ils ont commises.
\VS{14}C'est pourquoi, je les établirai pour avoir la garde de la maison pour tout son service, et pour tout ce qui s'y fait.
\VS{15}Mais quant aux prêtres et aux Lévites, fils de Tsadok, qui ont soigneusement administré ce qu'il fallait faire dans mon lieu saint, lorsque les enfants d'Israël se sont éloignés de moi, ceux-là s'approcheront de moi pour faire mon service, et se tiendront devant moi pour m'offrir la graisse et le sang, dit le Seigneur Yahweh.
\VS{16}Ceux-là entreront dans mon lieu saint, et s'approcheront de ma table, pour faire mon service, et ils administreront soigneusement ce que j'ai ordonné de faire.
\VS{17}Et il arriva que lorsqu'ils franchiront les portes des parvis intérieurs, ils se vêtiront de robes de lin~; et il n'y aura point de laine sur eux pendant qu'ils feront le service aux portes des parvis intérieurs et dans la maison.
\VS{18}Ils auront des ornements de lin sur leur tête, et des caleçons de lin sur leurs reins, et ne se ceindront point de manière à provoquer la sueur\FTNT{Voir Ge. 3:17.}.
\VS{19}Quand ils sortiront pour aller dans le parvis extérieur, dans le parvis extérieur, vers le peuple, ils se dévêtiront de leurs habits, avec lesquels ils font le service, et les poseront dans les chambres saintes, et se revêtiront d'autres habits, afin qu'ils ne sanctifient point le peuple avec leurs habits.
\VS{20}Ils ne se raseront point la tête, ni ne laisseront point croître leurs cheveux, mais simplement ils tondront leur tête\FTNT{Lé. 19:27.}.
\VS{21}Pas un des prêtres ne boira du vin quand ils entreront au parvis intérieur.
\VS{22}Ils ne prendront point pour femme une veuve, ni une répudiée~; mais ils prendront des vierges, de la race de la maison d'Israël, ou une veuve qui soit veuve d'un prêtre\FTNT{Lé. 21:13-14.}.
\VS{23}Ils enseigneront à mon peuple la différence qu'il y a entre le saint et le profane, et leur feront entendre la différence qu'il y a entre ce qui est souillé et ce qui est pur.
\VS{24}Quand il surviendra quelque procès, ils assisteront au jugement, et jugeront suivant les lois que j'ai données~; et ils garderont mes lois et mes statuts dans toutes mes fêtes, et ils sanctifieront mes sabbats.
\VS{25}Un prêtre n'ira pas vers un mort, de peur d'en être souillé, il pourra se rendre impur que pour un père, pour une mère, pour un fils, pour une fille, pour un frère, et pour une sœur qui n'aura point eu de mari\FTNT{Lé. 21:1-2.}.
\VS{26}Et après que chacun d'eux se sera purifié, on lui comptera sept jours.
\VS{27}Le jour où il entrera dans le lieu saint, dans le parvis intérieur pour faire le service dans le lieu saint, il offrira son sacrifice pour son péché, dit le Seigneur Yahweh.
\VS{28}Et cela leur sera pour héritage. Ce sera moi leur héritage, car vous ne leur donnerez aucune possession en Israël, ce sera moi leur possession\FTNT{No. 18:20~; De. 18:1-2.}.
\VS{29}Ils mangeront donc les gâteaux et ce qui s'offrira pour l'expiation, et ce qui s'offrira pour la culpabilité~; et tout interdit en Israël leur appartiendra.
\VS{30}Les prémices de tous les fruits et toutes les offrandes que vous présenterez par élévation, appartiendront aux prêtres~; vous donnerez aussi les prémices de votre pâte aux prêtres, afin que la bénédiction repose sur votre maison.
\VS{31}Les prêtres ne mangeront aucune créature volante, aucun animal mort ou déchiré\FTNT{Ex. 22:31~; Lé. 22:8.}.
\Chap{45}
\TextTitle{Zone réservée à Yahweh et aux prêtres}
\VerseOne{}Quand vous partagerez au sort le pays en héritage, vous prélèverez comme une offrande en élévation pour Yahweh, une portion du pays longue de vingt-cinq mille cannes, et large de dix mille~; ce sera une chose sainte dans tous ses territoires et aux environs.
\VS{2}De cette portion, vous prendrez pour le lieu saint cinq cents cannes sur cinq cents en carré, et cinquante coudées tout autour pour ses faubourgs.
\VS{3}Sur cette étendue de vingt-cinq mille en longueur cannes, et de dix mille en largeur, tu mesureras un emplacement pour le lieu saint et pour le Saint des saints~; et le Saint des saints sera dans cet espace.
\VS{4}C'est la portion sainte du pays, elle appartiendra aux prêtres qui font le service du lieu saint, qui s'approchent de Yahweh pour le servir~; c'est là que seront leur maison, et ce sera un lieu saint pour le lieu saint.
\VS{5}Vingt-cinq mille cannes en longueur, et dix mille en largeur, formeront la propriété des Lévites, serviteurs de la maison, avec vingt chambres.
\VS{6}Vous donnerez pour la possession de la ville la largeur de cinq mille et la longueur de vingt-cinq mille, suivant la proportion de la portion sanctifiée, qui sera levée pour toute la maison d'Israël.
\TextTitle{Zone réservée au prince}
\VS{7}Pour le prince vous réserverez un espace aux deux côtés de la portion sainte et de la propriété de la ville, le long de la portion sainte et le long de la propriété de la ville, du côté de l'occident vers l'occident, et du côté de l'orient vers l'orient, sur une longueur parallèle à l'une des parts, depuis la limite de l'occident jusqu'à la limite de l'orient.
\VS{8}Ce sera sa terre, sa propriété en Israël~; et mes princes que j'établirai ne fouleront plus mon peuple, mais ils distribueront le pays à la maison d'Israël, selon leurs tribus.
\TextTitle{Le prince, exemple au milieu du peuple~; prescriptions sur les offrandes}
\VS{9}Ainsi parle le Seigneur Yahweh~: Assez, princes d'Israël~! Otez la violence et le pillage, et jugez avec justice~; ôtez vos extorsions de dessus mon peuple~! dit le Seigneur Yahweh.
\VS{10}Ayez la balance juste, l'épha juste, et le bath juste\FTNT{Lé. 19:35-36.}.
\VS{11}L'épha et le bath seront de même mesure~; on prendra un bath pour la dixième partie d'un homer, et l'épha sera la dixième partie d'un homer, la mesure de l'un et de l'autre se rapportera à l'homer.
\VS{12}Le sicle sera de vingt guéras~; vingt sicles, vingt-cinq sicles et quinze sicles feront la mine\FTNT{Ex. 30:13~; Lé. 27:25.}.
\VS{13}C'est ici l'offrande élevée que vous offrirez~: La sixième partie d'un épha d'un homer de blé~; et vous donnerez la sixième partie d'un épha d'un homer d'orge.
\VS{14}Le bath est la mesure pour l'huile, l'offrande ordonnée pour l'huile sera la dixième partie d'un bath sur un cor, qui est égal à un homer de dix baths~; car dix baths feront un homer.
\VS{15}Pareillement, l'offrande ordonnée des bêtes du menu bétail sera de deux cents l'une, même des meilleurs pâturages d'Israël~; toute cette offrande sera employée en gâteaux et en holocaustes, et en offrandes de paix, afin de faire propitiation pour vous, dit le Seigneur Yahweh.
\VS{16}Tout le peuple du pays sera tenu à cette offrande élevée, pour celui qui sera prince en Israël.
\VS{17}Mais le prince sera tenu de fournir les holocaustes, les offrandes et les libations qu'il faudra offrir aux fêtes solennelles, aux nouvelles lunes et aux sabbats, et dans toutes les solennités de la maison d'Israël. Il tiendra prêtes les bêtes qu'on sacrifiera pour l'expiation, et les gâteaux, et les bêtes qu'on sacrifiera pour l'holocauste, et les bêtes qu'on sacrifiera pour les offrandes de paix, afin de faire propitiation pour la maison d'Israël.
\VS{18}Ainsi parle le Seigneur Yahweh~: Au premier mois, au premier jour du mois, tu prendras un jeune taureau sans défaut, et tu feras l'expiation du lieu saint.
\VS{19}Le prêtre prendra du sang de ce sacrifice offert pour le péché, et en mettra sur les poteaux de la maison, et sur les quatre angles de l'encadrement de l'autel, et sur les poteaux de la porte du parvis intérieur.
\VS{20}Tu en feras ainsi au septième jour du même mois, à cause des hommes qui pèchent involontairement et à cause des hommes simples~; et vous ferez ainsi propitiation pour la maison.
\VS{21}Au premier mois, au quatorzième jour du mois, vous aurez la Pâque, fête solennelle qui durera sept jours, pendant lesquels on mangera des pains sans levain\FTNT{Lé. 25:5~; No. 9:3~; Ex. 12.}.
\VS{22}En ce jour-là, le prince offrira un taureau pour le sacrifice d'expiation, tant pour lui que pour tout le peuple du pays.
\VS{23}Pendant les sept jours de cette fête solennelle, il offrira chaque jour sept taureaux et sept béliers sans défaut, pour l'holocauste qu'on offrira à Yahweh, et un bouc en sacrifice d'expiation, chaque jour.
\VS{24}Il offrira un épha pour chaque taureau, et un épha pour chaque bélier, avec un hin d'huile par épha.
\VS{25}Au septième mois, le quinzième jour du mois, à la fête solennelle, il offrira durant sept jours les mêmes choses, le même sacrifice expiatoire, le même holocauste et la même offrande avec l'huile.
\Chap{46}
\TextTitle{Le service le jour du sabbat et les jours de fêtes}
\VerseOne{}Ainsi parle le Seigneur Yahweh~: La porte du parvis intérieur, du côté oriental, sera fermée les six jours ouvrables, mais elle sera ouverte le jour du sabbat, elle sera aussi ouverte le jour de la nouvelle lune.
\VS{2}Et le prince y entrera par le chemin du vestibule de la porte du parvis extérieur, et se tiendra près de l'un des poteaux de l'autre porte~; les prêtres prépareront son holocauste et ses sacrifices d'offrande de paix~; il se prosternera sur le seuil de cette porte, et ensuite il sortira~; et la porte ne sera point fermée jusqu'au soir.
\VS{3}Tellement que le peuple du pays se prosternera devant Yahweh à l'entrée de cette porte, les jours de sabbat et des nouvelles lunes.
\VS{4}L'holocauste que le prince offrira à Yahweh le jour du sabbat sera de six agneaux sans défaut et d'un bélier sans défaut.
\VS{5}L'offrande pour le bélier sera d'un épha, et l'offrande pour chacun des agneaux sera selon ce qu'il pourra donner~; mais il y aura un hin d'huile pour chaque épha.
\VS{6}Au jour de la nouvelle lune, son holocauste sera d'un jeune taureau, sans défaut, de six agneaux et d'un bélier, aussi sans défaut.
\VS{7}Son offrande pour le taureau sera d'un épha, pour l'offrande du bélier, un autre épha, et pour chacun des agneaux selon ce qu'il pourra donner~; mais il y aura un hin d'huile pour chaque épha.
\VS{8}Lorsque le prince entrera, il entrera par le chemin du vestibule de la porte, et il sortira par le même chemin.
\VS{9}Quand le peuple du pays entrera pour se présenter devant Yahweh, aux fêtes solennelles, celui qui y entrera par le chemin de la porte nord pour y adorer Yahweh, sortira par le chemin de la porte sud, et celui qui y entrera par le chemin de la porte sud, sortira par le chemin de la porte nord~; personne ne retournera par le chemin de la porte par laquelle il sera entré, mais il sortira par celle qui lui est opposée.
\VS{10}Alors le prince entrera parmi eux, quand ils entreront~; et quand ils sortiront, ils sortiront ensemble.
\VS{11}Or,dans ces fêtes solennelles et dans ces solennités, l'offrande d'un taureau sera d'un épha, et l'offrande d'un bélier d'un autre épha, l'offrande de chacun des agneaux sera selon ce que le prince pourra donner, et il y aura un hin d'huile pour chaque épha.
\VS{12}Et si le prince offre un sacrifice volontaire, quelque holocauste, soit quelques sacrifices d'offrande de paix en offrande volontaire à Yahweh, on lui ouvrira la porte qui est du côté oriental, et il offrira son holocauste et ses sacrifices d'offrande de paix comme il les offre le jour du sabbat~; puis il sortira, et après qu'il sera sorti, on fermera cette porte.
\VS{13}Tu sacrifieras chaque jour en holocauste à Yahweh un agneau d'un an sans défaut, tu le sacrifieras tous les matins.
\VS{14}Tu lui offriras tous les matins l'offrande, faite de la sixième partie d'un épha, et de la troisième d'un hin d'huile pour pétrir la farine~; c'est l'offrande à Yahweh qu'il faut offrir par ordonnances perpétuelles.
\VS{15}Ainsi on offrira tous les matins en holocauste perpétuel cet agneau et l'offrande avec cette huile.
\VS{16}Ainsi a dit le Seigneur Yahweh~: Quand le Prince aura fait un don de quelque pièce de son héritage à quelqu'un de ses fils, ce don appartiendra à ses fils~; parce qu'ils ont droit de possession en l'héritage.
\VS{17}Mais s'il fait un don pris de son héritage à l'un de ses serviteurs, le don lui appartiendra bien, mais seulement jusqu'à l'année de la liberté, puis il retournera au prince~; ses fils seuls posséderont ce qu'il leur donnera de son héritage\FTNT{Lé. 25:10.}.
\VS{18}Et le prince ne prendra pas de l'héritage du peuple en les opprimant, les chassant de leur possession, c'est de sa propre possession qu'il fera hériter ses fils, afin qu'aucun de mon peuple ne soit dispersé loin de sa possession.
\VS{19}Puis il me mena par l'entrée qui était vers le côté de la porte, aux chambres saintes qui appartenaient aux prêtres, vers le nord. Et voici, il y avait un certain lieu dans le fond du côté occidental.
\VS{20}Et il me dit~: C'est là le lieu où les prêtres feront bouillir le reste de la bête sacrifiée pour la culpabilité, et le reste de la bête sacrifiée pour l'expiation, et où ils feront cuire les offrandes, afin qu'ils ne les emportent point au parvis extérieur de manière à sanctifier le peuple\FTNT{No. 18:9.}.
\VS{21}Puis il me fit sortir vers le parvis extérieur, et me fit traverser vers les quatre angles du parvis, et voici, il y avait une cour à chacun des angles du parvis.
\VS{22}Aux quatre angles de ce parvis, il y avait des cours voûtées, longues de quarante coudées, et larges de trente~; et toutes les quatre avaient la même mesure dans les angles.
\VS{23}Un mur les entourait toutes les quatre, et des foyers étaient faits au bas du campement tout autour.
\VS{24}Et il me dit~: Ce sont ici les cuisines, où ceux qui font le service de la maison cuiront les sacrifices du peuple.
\Chap{47}
\TextTitle{Les eaux pures du sanctuaire\FTNTT{Cp. Za. 14:8-9~; Ap. 22:1-2.}}
\VerseOne{}Puis il me ramena vers l'entrée de la maison, et voici, des eaux sortaient sous le seuil de la maison, vers l'orient, car la face de la maison était vers l'orient~; et ces eaux-là descendaient du côté droit de la maison, du côté sud de l'autel\FTNT{Ps. 46:5~; Joë. 3:18~; Za. 13:1~; Za. 14:8~; Ap. 22:1.}.
\VS{2}Puis il me fit sortir par le chemin de la porte nord, et me fit faire le tour par dehors, jusqu'à la porte extérieure, du côté de l'orient, et voici, les eaux coulaient du côté droit.
\VS{3}Quand cet homme s'avança vers l'orient, il avait dans sa main un cordeau~; et il mesura mille coudées, puis il me fit traverser ces eaux-là, et j'avais de l'eau jusqu'aux chevilles.
\VS{4}Puis il mesura mille autres coudées, et me fit traverser les eaux, j'avais de l'eau jusqu'aux genoux~; puis il mesura mille autres coudées, et me fit traverser, et j'avais de l'eau jusqu'aux reins.
\VS{5}Il mesura mille autres coudées~; mais ces eaux-là étaient déjà un torrent que je ne pouvais traverser~; car ces eaux-là étaient si profondes qu'il fallait y traverser à la nage, c'était un torrent que l'on ne pouvait traverser.
\VS{6}Alors il me dit~: Fils de l'homme, as-tu vu~? Puis il me fit aller et revenir vers le bord du torrent.
\VS{7}Quand je revins, il y avait un grand nombre d'arbres sur les deux bords du torrent.
\VS{8}Il me dit~: Ces eaux couleront vers la Galilée orientale, et elles descendront à la campagne, puis elles entreront dans la mer, et quand elles se seront jetées dans la mer, les eaux deviendront saines.
\VS{9}Il arrivera que tout être vivant qui se meut vivra partout où les deux torrents couleront, et il y aura une grande quantité de poissons~; car là où ces eaux entreront, les eaux deviendront saines, et tout vivra là où ce torrent parviendra.
\VS{10}Il arrivera que des pêcheurs se tiendront le long de cette mer, depuis En-Guédi jusqu'à En-Eglaïm~; on étendra les filets, il y aura des poissons de diverses espèces, comme les poissons de la grande mer, et ils seront très nombreux.
\VS{11}Ses marais et ses fosses ne seront pas assainis, ils seront abandonnés au sel.
\VS{12}Auprès de ce torrent et sur ses deux bords, il croîtra des arbres fruitiers de toutes sortes. Leur feuillage ne se flétrira point, et l'on trouvera toujours du fruit. Tous les mois, ils produiront des fruits mûrs, parce que les eaux de ce torrent sortent du lieu saint, et à cause de cela leur fruit sera bon à manger, et leur feuillage servira de remède\FTNT{Ap. 22:2.}.
\TextTitle{Délimitations du pays\FTNTT{Cp. Ge. 15:18-21.}}
\VS{13}Ainsi parle le Seigneur Yahweh~: Voici les frontières du pays que vous aurez en héritage, selon les douze tribus d'Israël. Joseph aura deux portions.
\VS{14}Vous aurez la possession l'un comme l'autre de ce pays. J'ai levé ma main de le donner à vos pères~; et ce pays-là vous sera donc échu en héritage\FTNT{Ge. 12:7~; Ge. 17:8.}.
\VS{15}Voici la frontière du pays, du côté nord, depuis la grande mer, le chemin de Hethlon jusqu'à Tsedad,
\VS{16}Hamath, Bérotha, et Sibraïm, entre la frontière de Damas et la frontière de Hamath, Hatzer-Hatthicon, vers la frontière de Havran.
\VS{17}La frontière sera depuis la mer, Hatsar-Enon~; la frontière de Damas, Tsaphon au nord et la frontière de Hamath~: Ce sera le côté nord.
\VS{18}Le côté oriental sera le Jourdain entre Havran, Damas et Galaad, et le pays d'Israël~; vous mesurerez depuis la frontière nord jusqu'à la mer orientale~: Ce sera le côté oriental.
\VS{19}Le côté méridional, au midi, ira depuis Thamar jusqu'aux eaux de Meriba à Kadès, jusqu'au torrent vers la grande mer~: Ce sera le côté méridional.
\VS{20}Le côté occidental sera la grande mer, depuis la frontière jusque vis-à-vis de Hamath~: Ce sera le côté occidental.
\VS{21}Vous partagerez ce pays entre vous, selon les tribus d'Israël.
\VS{22}Vous le diviserez en héritage par le sort pour vous et pour les étrangers qui séjourneront au milieu de vous, qui engendreront des enfants au milieu de vous~; vous les regarderez comme natifs des enfants d'Israël~; ils partageront au sort l'héritage avec vous parmi les tribus d'Israël.
\VS{23}Vous donnerez à l'étranger son héritage dans la tribu où il séjournera, dit le Seigneur Yahweh.
\Chap{48}
\TextTitle{Héritage de sept tribus\FTNTT{Cp. Jos. 13:1-19:51.}}
\VerseOne{}Voici les noms des tribus. Depuis l'extrémité qui regarde vers le nord, le long de la contrée du chemin de Hethlon, du quartier par lequel on entre à Hamath, jusqu'à Hatsar-Enon, qui est la frontière de Damas, du côté qui regarde vers le nord, le long de la contrée de Hamath, tellement que cette extrémité ait le canton de l'orient et celui de l'occident~: il y aura une portion pour Dan.
\VS{2}Et tout joignant les frontières de Dan, depuis le canton de l'orient jusqu'au canton qui regarde vers l'occident, il y aura une portion pour Aser.
\VS{3}Et tout joignant les frontières d'Aser, depuis le canton qui regarde vers l'orient jusqu'au canton qui regarde vers l'occident, il y aura une portion pour Nephthali.
\VS{4}Et tout joignant les frontières de Nephthali, depuis le canton qui regarde vers l'orient jusqu'au canton qui regarde vers l'occident, il y aura une portion pour Manassé. 
\VS{5}Et tout joignant les frontières de Manassé, depuis le canton qui regarde vers l'occident jusqu'au canton qui regarde vers l'orient, il y aura une portion pour Ephraïm. 
\VS{6}Et tout joignant les frontières d'Ephraïm, encore depuis le canton de l'orient, jusqu'au canton qui regarde vers l'occident: il y aura une portion pour Ruben. 
\VS{7}Et joignant les frontières de Ruben, depuis le canton de l'orient jusqu'au canton qui regarde vers l'occident, il y aura une portion pour Juda.
\VS{8}Et tout le long des frontières de Juda, depuis le canton de l'orient, jusqu'au canton qui regarde vers l'occident, il y aura une portion que vous prélèverez sur toute la masse du pays, comme une offrande élevée et elle aura vingt-cinq mille cannes de largueur et de longueur, autant que l'une des autres portions, depuis le canton qui regarde vers l'orient jusqu'au canton qui regarde vers l'occident~; de sorte que le sanctuaire sera au milieu.
\VS{9}La portion que vous lèverez pour Yahweh et qui sera offerte en offrande élevée, aura vingt-cinq mille cannes de longueur, et dix mille de largeur.
\TextTitle{Territoire réservé aux prêtres et aux Lévites}
\VS{10}Et cette portion sainte sera pour les prêtres, vingt-cinq mille cannes de longueur au nord, et dix mille de largeur à l'occident, dix mille de largeur à l'orient, et vingt-cinq mille de longueur au sud, et le sanctuaire de Yahweh sera au milieu.
\VS{11}Elle sera pour les prêtres, et quiconque aura été sanctifié d'entre les fils de Tsadok, qui ont fait ce que j'ai ordonné, et qui ne se sont point égarés quand les enfants d'Israël se sont égarés, comme se sont égarés les autres Lévites,
\VS{12}ceux-là auront une portion ainsi prelevée sur l'autre, qui aura été auparavant prelevée sur toute la masse du pays, comme étant une chose très-sainte, et elle sera vers les frontières de la portion des Lévites.
\VS{13}Les Lévites auront, parallèlement à la frontière des prêtres, vingt-cinq mille cannes de longueur et dix mille de largeur, vingt-cinq mille pour toute la longueur et dix mille pour toute la largeur.
\VS{14}Ils n'en pourront ni vendre ni échanger, et les prémices du pays ne seront point transgressées car elles sont mises à part pour Yahweh.
\VS{15}Les cinq mille cannes qui resteront en largeur sur les vingt-cinq mille cannes seront destinées à la ville, pour les habitations et le faubourg, et la ville sera au milieu.
\VS{16}En voici les mesures~: Du côté nord, quatre mille cinq cents cannes, du côté sud, quatre mille cinq cents, du côté oriental, quatre mille cinq cents, et du côté occidental, quatre mille cinq cents.
\VS{17}Puis il y aura des faubourgs pour la ville, vers le nord. La ville aura un faubourg au nord de deux cent cinquante cannes, de deux cent cinquante au sud, de deux cent cinquante à l'orient, et de deux cent cinquante à l'occident.
\VS{18}Quant à ce qui sera de reste sur la longueur et qui sera tout joignant à la portion sanctifiée, et qui aura dix mille cannes à l'orient, et dix mille autres cannes à l'occident, parallèlement à la portion sanctifiée, le revenu qu'on en tirera sera pour nourriture de ceux qui feront le service qu'il faut dans la ville.
\VS{19}Le sol sera travaillé par ceux de toutes les tribus d'Israël qui travailleront pour la ville.
\VS{20}Toute la portion prélevée sera de vingt-cinq mille cannes en longueur sur vingt-cinq mille en largeur~; vous en séparerez un carré pour la propriété de la ville.
\VS{21}Puis le reste sera pour le prince aux deux côtés de la portion sainte et de la possession de la ville, des vingt-cinq mille cannes de la portion prélevée jusqu'à la frontière de l'orient, et à l'occident, des vingt-cinq mille coudées jusqu'à la frontière de l'occident, le long des parts, sera pour le prince, et la portion sainte et le sanctuaire de la Maison seront au milieu de tout le pays. 
\VS{22}Ce qui sera donc pour le prince sera l'espace compris depuis la possession des Lévites, et depuis la possession de la ville~; ce qui sera entre ces possessions là et la frontière de Juda et la frontière de Benjamin, sera pour le prince.
\TextTitle{Héritage de cinq tribus}
\VS{23}Or ce qui sera de reste sera pour les autres tribus~; depuis le canton de ce qui regarde vers l'orient, jusqu'au canton de ce qui regarde vers l'occident, il y aura une portion pour Benjamin.
\VS{24}Puis tout joignant les frontières de Benjamin, depuis le canton de ce qui regarde vers l'orient, jusqu'au canton de ce qui regarde vers l'occident, il y aura une autre portion pour Siméon.
\VS{25}Puis tout joignant les frontières de Siméon, depuis le canton de ce qui regarde vers l'orient, jusqu'au canton de ce qui regarde vers l'occident, il y aura une autre portion pour Issacar.
\VS{26}Puis tout joignant les frontières d'Issacar, depuis le canton de ce qui regarde vers l'orient, jusqu'au canton de ce qui regarde vers l'occident, il y aura une autre portion pour Zabulon.
\VS{27}Puis tout joignant les frontières de Zabulon, depuis le canton de ce qui regarde vers l'orient, jusqu'au canton de ce qui regarde vers l'occident, il y aura une autre portion pour Gad.
\VS{28}Or ce qui appartient au côté du midi qui regarde proprement le vent d'autan, et sur la frontière de Gad, et cette frontière sera depuis Thamar jusqu'aux eaux de contestation, à Kadès, le long du torrent jusqu'à la grande mer.
\VS{29}C'est là le pays que vous partagerez par le sort en héritage aux tribus d'Israël, et ce sont là leurs portions, dit le Seigneur Yahweh.
\VS{30}Et ce sont ici les sorties de la ville~: Du côté du nord, il y aura quatre mille cinq cents mesures.
\VS{31}Et les portes de la ville seront selon les noms des tribus d'Israël~: Trois portes vers le nord, une porte de Ruben, une porte de Juda, une porte de Lévi.
\VS{32}Et du côté de l'orient, quatre mille cinq cents mesures, et trois portes~: Une porte de Joseph, une porte de Benjamin, une porte de Dan.
\VS{33}Et du côté du sud, quatre mille cinq cents mesures, et trois portes~: Une porte de Simeon, une porte d'Issacar, une porte de Zabulon.
\VS{34}Et du côté de l'occident, quatre mille cinq cents mesures, avec leurs trois portes~: Une porte de Gad, une porte d'Aser, une porte de Nephthali.
\VS{35}Ainsi, le circuit de la ville sera de dix-huit mille mesures~; et le nom de la ville depuis ce jour-là sera~: Yahweh est ici.
\PPE{}
\end{multicols}
