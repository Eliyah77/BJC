\ShortTitle{2 Ti.}\BookTitle{2 Timothée}\BFont
\noindent\hrulefill
{\footnotesize
\textit{
\bigskip
{\centering{}
\\Auteur~: Paul
\\(Gr.~: Timotheos)
\\Signification~: Qui adore ou honore Dieu
\\Thème~: Le maintien de la vérité
\\Date de rédaction~: Env. 67 ap. J.-C.\\}
}
\textit{
\\Cette lettre s'adresse à Timothée dont le père était grec et la mère juive. Le jeune homme se convertit à Christ avec sa mère et sa grand-mère dès le premier voyage missionnaire de Paul au cours duquel il passa à Lystre.
\\Paul écrivit cette épître pastorale alors qu'il était emprisonné à Rome, après avoir été arrêté dans une province orientale à Ephèse ou Troas.
Ses conditions de détention étant plus rudes que la première fois, Paul est dubitatif quant à sa remise en liberté. Il demande donc à Timothée, son fils dans la foi et fidèle compagnon d'œuvre, de le rejoindre à Rome afin de recevoir, semble-t-il, ses dernières volontés. Après avoir exposé à Timothée les qualités et les devoirs d'un bon serviteur de l'Evangile, il l'encourage à lutter contre les faux docteurs et l'apostasie en prêchant la parole en toutes circonstances.\bigskip
}
}
\par\nobreak\noindent\hrulefill
\begin{multicols}{2}
\Chap{1}
\TextTitle{Introduction}
\VerseOne{}Paul, apôtre de Jésus-Christ, par la volonté de Dieu, selon la promesse de la vie qui est en Jésus-Christ,
\VS{2}à Timothée, mon fils bien-aimé~: Que la grâce, la miséricorde et la paix te soient données de la part de Dieu le Père et de la part de Jésus-Christ, notre Seigneur~!
\TextTitle{Paul encourage Timothée à souffrir les afflictions de l'Evangile}
\VS{3}Je rends grâces à Dieu, que mes ancêtres ont servi et que je sers avec une conscience pure, faisant sans cesse mention de toi dans mes prières nuit et jour.
\VS{4}Me souvenant de tes larmes, je désire fort de te voir, afin que je sois rempli de joie,
\VS{5}et me souvenant de la foi sincère qui est en toi et qui a premièrement habité en Loïs, ta grand-mère et en Eunice, ta mère, et qui, j'en suis persuadé, habite aussi en toi.
\VS{6}C'est pourquoi je t'exhorte de ranimer le don de Dieu qui est en toi par l'imposition de mes mains.
\VS{7}Car Dieu ne nous a pas donné un esprit de timidité, mais de force, de charité\FTNT{Il est question ici de l'amour «~agape~», c'est-à-dire l'amour divin.} et de sagesse.
\VS{8}N'aie donc point honte du témoignage à rendre à notre Seigneur ni de moi, qui suis son prisonnier~; mais souffre avec moi les afflictions de l'Evangile, selon la puissance de Dieu,
\VS{9}qui nous a sauvés et qui nous a appelés par une sainte vocation, non selon nos œuvres, mais selon son propre dessein et selon la grâce qui nous a été donnée en Jésus-Christ avant les temps éternels,
\VS{10}et qui maintenant a été manifestée par l'apparition de notre Sauveur Jésus-Christ, qui a détruit la mort et qui a mis en lumière la vie et l'immortalité par l'Evangile,
\VS{11}pour lequel j'ai été établi prédicateur, apôtre et docteur des Gentils.
\VS{12}C'est pourquoi aussi je souffre ces choses, mais je n'en ai point de honte, car je connais celui en qui j'ai cru et je suis persuadé qu'il est puissant pour garder mon dépôt\FTNT{Dépôt~: Il est question ici de la connaissance correcte et de la pure doctrine de l'Evangile qui doit être fermement et fidèlement gardée, et qui doit être consciencieusement délivrée aux autres.} jusqu'à ce jour-là.
\VS{13}Retiens, dans la foi et dans la charité qui est en Jésus-Christ, le modèle des saines paroles que tu as apprises de moi.
\VS{14}Garde le bon dépôt par le Saint-Esprit qui habite en nous.
\VS{15}Tu sais que tous ceux qui sont en Asie se sont éloignés de moi~; entre lesquels sont Phygelle et Hermogène.
\VS{16}Que le Seigneur accorde sa miséricorde à la maison d'Onésiphore, car souvent il m'a consolé et il n'a point eu honte de mes chaînes.
\VS{17}Au contraire, quand il a été à Rome, il m'a cherché avec beaucoup d'empressement et il m'a trouvé.
\VS{18}Que le Seigneur lui fasse trouver miséricorde envers le Seigneur en ce jour-là~! Et tu sais mieux que personne combien il m'a rendu de services à Ephèse.
\Chap{2}
\TextTitle{La conduite d'un disciple de Christ~: Transmettre la parole~; combattre le bon combat}
\VerseOne{}Toi donc, mon fils, sois fortifié dans la grâce qui est en Jésus-Christ.
\VS{2}Et les choses que tu as entendues de moi devant plusieurs témoins, confie-les à des personnes fidèles, qui soient capables de les enseigner aussi à d'autres.
\VS{3}Toi donc, souffre avec moi comme un bon soldat de Jésus-Christ.
\VS{4}Il n'est pas de soldat qui s'embarrasse des affaires de cette vie s'il veut plaire à celui qui l'a enrôlé pour la guerre.
\VS{5}De même, l'athlète qui combat n'est point couronné s'il n'a pas combattu selon les règles.
\VS{6}Il faut aussi que le laboureur travaille premièrement, et ensuite il recueille les fruits.
\VS{7}Considère ce que je dis, car le Seigneur te donne de l'intelligence en toutes choses.
\VS{8}Souviens-toi que Jésus-Christ, qui est de la semence de David, est ressuscité des morts, selon mon Evangile,
\VS{9}pour lequel je souffre beaucoup de maux, jusqu'à être mis dans les chaînes comme un malfaiteur~; cependant, la parole de Dieu n'est point liée.
\VS{10}C'est pourquoi je souffre tout pour l'amour des élus, afin qu'eux aussi obtiennent le salut qui est en Jésus-Christ, avec la gloire éternelle.
\VS{11}Cette parole est certaine, que si nous mourrons avec lui, nous vivrons aussi avec lui.
\VS{12}Si nous souffrons avec lui, nous régnerons aussi avec lui. Si nous le renions, il nous reniera aussi\FTNT{Lu. 9:26.}.
\VS{13}Si nous sommes infidèles, il demeure fidèle, car il ne peut pas se renier lui-même.
\VS{14}Remets ces choses en mémoire, protestant devant Dieu qu'on ait pas de disputes de mots, qui est une chose dont il ne revient aucun profit, mais elle est la ruine des auditeurs.
\VS{15}Efforce-toi de te rendre approuvé\FTNT{Le participe passé du verbe «~approuvé~» vient du grec «~dokimos~». Du temps de Paul, les systèmes bancaires actuels n'existaient pas, toute la monnaie était en métal. Pour obtenir les pièces de monnaie, le métal était fondu et versé dans des moules et après le démoulage, il était nécessaire d'enlever les bavures. Or de nombreuses personnes les grattaient pour récupérer le surplus de métal et même davantage, ce qui faussait le poids de la monnaie. Face à ce problème, de nombreuses lois furent promulguées à Athènes pour éradiquer la pratique du rognage des pièces en circulation. Il existait toutefois quelques changeurs intègres qui ne mettaient en circulation que des pièces au bon poids. On appelait ces personnes des «~dokimos~», ce qui signifie «~éprouvés~» ou «~approuvés~».} devant Dieu, comme un ouvrier sans reproche, enseignant purement la parole de la vérité.
\VS{16}Mais évite les discours vains et profanes~; car ceux qui les tiennent avanceront toujours plus dans l'impiété,
\VS{17}et leur parole rongera comme une gangrène. Et parmi ceux-là sont Hyménée et Philète,
\VS{18}qui se sont écartés\FTNT{Ecarter, dévier, s'écarter de, manquer le but. A l'époque des apôtres, il y avait plusieurs faux frères qui semaient la zizanie au milieu des enfants de Dieu. Parmi eux étaient Alexandre, le forgeron (1 Ti. 1:18-20), Hyménée (1 Ti. 1:18-20), Philète (2 Ti. 2:16-18), les judaïsants (Ac. 15:1-29), Diotrèphe (3 Jn. 1:9-11). Les faux frères sont des séducteurs.} de la vérité, en disant que la résurrection est déjà arrivée, et qui renversent la foi de quelques-uns.
\VS{19}Toutefois, le fondement de Dieu demeure ferme, ayant ce sceau~: Le Seigneur connaît ceux qui lui appartiennent\FTNT{Le Seigneur connaît ses brebis. Voir No. 16:5~; Jn. 10:14.}~; et~: Quiconque invoque le Nom du Seigneur, qu'il s'éloigne de l'iniquité.
\VS{20}Or dans une grande maison, il n'y a pas seulement des vases d'or et d'argent, mais il y en a aussi de bois et de terre. Les uns sont des vases d'honneur et les autres sont d'un usage vil.
\VS{21}Si quelqu'un donc se purifie de ces choses, il sera un vase d'honneur, sanctifié et utile au Seigneur, et préparé pour toute bonne œuvre.
\VS{22}Fuis aussi les désirs de la jeunesse, et recherche la justice, la foi, la charité et la paix avec ceux qui invoquent le Seigneur d'un cœur pur.
\VS{23}Et rejette les questions\FTNT{Questions folles. Il est question ici de disputes, débats, discussions ou questions oiseuses.} folles et qui sont sans instruction, sachant qu'elles ne font que produire des querelles.
\VS{24}Or il ne faut pas que le serviteur du Seigneur soit querelleur, il doit au contraire avoir de la douceur envers tout le monde, propre à enseigner, supportant patiemment les mauvais,
\VS{25}enseignant avec douceur ceux qui ont un sentiment contraire, dans l'espérance qu'un jour Dieu leur donnera la repentance pour reconnaître la vérité,
\VS{26}et afin qu'ils se réveillent pour sortir des pièges du diable par lesquels ils ont été pris pour faire sa volonté.
\Chap{3}
\TextTitle{Le caractère de l'homme des derniers jours}
\VerseOne{}Or sache ceci, que dans les derniers jours\FTNT{Les derniers jours. Voir Ge. 49:1.} il surviendra des temps difficiles.
\VS{2}Car les hommes seront idolâtres d'eux-mêmes, amis de l'argent, fanfarons, orgueilleux, blasphémateurs, rebelles à leurs parents, ingrats, irréligieux,
\VS{3}sans affection naturelle, sans fidélité, calomniateurs, intempérants, cruels, haïssant les gens de bien,
\VS{4}traîtres, emportés, enflés d'orgueil, amis des voluptés plutôt qu'amis de Dieu\FTNT{Le mot grec «~philotheos~» (amour de Dieu) du préfixe «~philos~» qui signifie «~amis, être lié d'amitié avec quelqu'un~» (Mt. 11:19~; Lu. 7:6~; Jn. 15:13-15) et de «~theos~» qui signifie «~Dieu~».}
\VS{5}ayant l'apparence\FTNT{L'apparence de la piété. Le mot «~apparence~» vient du grec «~morphosis~» et du latin «~forma~» qui donnent «~forme~» en français. Il est question du formalisme, de l'attachement excessif aux règles, aux rites, aux coutumes et aux traditions. Dans l'église de Laodicée, l'accent est plutôt mis sur les règles à observer et les apparences que sur la vie spirituelle et intérieure. Les manifestations extérieures du formalisme sont~: Les lieux «~sacrés~» pour adorer (temples, cathédrales, pèlerinages, etc.)~; l'observation des jours sacrés (dimanche et sabbat)~; les rituels censés permettre au croyant d'expérimenter Dieu et de rentrer dans une vie bénie (circoncision, ordination, bénédiction nuptiale, paiement de la dîme, présentation des enfants à Dieu par le pasteur, etc.)~; une manière spéciale de s'habiller (toge, soutane, collet clérical, kippa, voile, costume/cravate, un régime alimentaire spécial, etc.). Voir Mt. 6:1-8.} de la piété, mais en ayant renié la force. Eloigne-toi donc de telles gens.
\VS{6}Or de ce nombre sont ceux qui se glissent dans les maisons, et qui tiennent captives les femmes chargées de péchés et agitées de diverses convoitises,
\VS{7}qui apprennent toujours, mais qui ne peuvent jamais parvenir à la pleine connaissance de la vérité.
\VS{8}Et comme Jannès et Jambrès ont résisté à Moïse, ceux-ci de même résistent à la vérité, étant des gens qui ont l'esprit corrompu et qui sont réprouvés quant à la foi.
\VS{9}Mais ils ne feront pas de plus grands progrès, car leur folie sera manifestée à tous, comme le fut celle de ceux-là.
\VS{10}Mais pour toi, tu as pleinement compris ma doctrine, ma conduite, mon intention, ma foi, ma douceur, ma charité, ma persévérance.
\VS{11}Et tu sais les persécutions et les afflictions qui me sont arrivées à Antioche, à Icone et à Lystre. Quelles persécutions n'ai-je pas supportées~? Et comment le Seigneur m'a délivré de toutes.
\VS{12}Or tous ceux aussi qui veulent vivre pieusement en Jésus-Christ seront persécutés.
\VS{13}Mais les hommes méchants et imposteurs iront en empirant, séduisant les autres et étant séduits.
\VS{14}Mais toi, demeure ferme dans les choses que tu as apprises et qui t'ont été confiées, sachant de qui tu les as apprises,
\VS{15}vu même que dès ton enfance tu as la connaissance des saintes lettres, qui peuvent te rendre sage pour le salut par la foi en Jésus-Christ.
\VS{16}Toute l'Ecriture est inspirée de Dieu et utile pour enseigner, pour convaincre, pour corriger et pour instruire selon la justice,
\VS{17}afin que l'homme de Dieu soit accompli et parfaitement instruit pour toute bonne œuvre.
\Chap{4}
\TextTitle{Paul encourage solennellement Timothée à prêcher la parole}
\VerseOne{}Je te somme devant Dieu et devant le Seigneur Jésus-Christ, qui doit juger les vivants et les morts, lors de son apparition et de son règne,
\VS{2}prêche la parole, insiste en toute occasion, favorable ou non. Reprends, censure, exhorte avec toute douceur d'esprit et avec doctrine.
\VS{3}Car il viendra un temps où les hommes ne supporteront pas la saine doctrine, mais aimant qu'on leur chatouille les oreilles par des discours agréables, ils chercheront des docteurs qui répondent à leurs désirs\FTNT{Beaucoup refusent la saine doctrine et acceptent un évangile basé sur les biens matériels.}.
\VS{4}Et ils détourneront leurs oreilles de la vérité et se tourneront vers les fables.
\VS{5}Mais toi, veille en toutes choses, souffre les afflictions, fais l'œuvre d'un évangéliste, rends ton service pleinement approuvé.
\VS{6}Car pour moi, je m'en vais maintenant servir de libation et le temps de mon départ est proche.
\VS{7}J'ai combattu le bon combat, j'ai achevé la course, j'ai gardé la foi.
\VS{8}Au reste, la couronne de justice m'est réservée, et le Seigneur, le juste Juge, me la donnera en ce jour-là, et non seulement à moi, mais aussi à tous ceux qui auront aimé son apparition.
\VS{9}Hâte-toi de venir bientôt vers moi.
\VS{10}Car Démas m'a abandonné, ayant aimé le présent siècle, et il s'en est allé à Thessalonique~; Crescens est allé en Galatie~; et Tite en Dalmatie.
\VS{11}Luc est seul avec moi~; prends Marc et amène-le avec toi, car il m'est fort utile pour le service.
\VS{12}J'ai aussi envoyé Tychique à Ephèse.
\VS{13}Quand tu viendras, apporte avec toi le manteau que j'ai laissé à Troas, chez Carpus, et les livres aussi~; mais principalement mes parchemins.
\VS{14}Alexandre, le forgeron m'a fait beaucoup de mal. Le Seigneur lui rendra selon ses œuvres.
\VS{15}Garde-toi donc de lui, car il s'est fortement opposé à nos paroles.
\VS{16}Personne ne m'a assisté dans ma première défense, mais tous m'ont abandonné~; toutefois que cela ne leur soit point imputé~!
\VS{17}Mais le Seigneur m'a assisté et fortifié, afin que ma prédication soit pleinement approuvée et que tous les Gentils l'entendent~; et j'ai été délivré de la gueule du lion.
\VS{18}Le Seigneur aussi me délivrera de toute mauvaise œuvre et me sauvera dans son Royaume céleste. A lui soit la gloire aux siècles des siècles~! Amen~!
\TextTitle{Salutations}
\VS{19}Salue Priscille et Aquilas, et la famille d'Onésiphore.
\VS{20}Eraste est resté à Corinthe, et j'ai laissé Trophime malade à Milet.
\VS{21}Hâte-toi de venir avant l'hiver. Eubulus et Pudens, et Linus, et Claudia, et tous les frères te saluent.
\VS{22}Que le Seigneur Jésus-Christ soit avec ton esprit~! Que la grâce soit avec vous~! Amen~!
\PPE{}
\end{multicols}
