\ShortTitle{2 Timothée}\BookTitle{2 Timothée}\BFont
\begin{multicols}{2}
\TextTitle{[Affection de Paul pour Timothée. Il l'exhorte à la persévérance]}
\Chap{1}
\VerseOne{}Paul, apôtre de Jésus-Christ, par la volonté de Dieu, selon la promesse de la vie qui est en Jésus-Christ.
\VS{2}A Timothée, mon fils bien-aimé, que la grâce, la miséricorde et la paix te soient données de la part de Dieu le Père, et de la part de Jésus-Christ notre Seigneur.
\VS{3}Je rends grâces à Dieu, que mes ancêtres ont servi et que je sers avec une conscience pure, faisant sans cesse mention de toi dans mes prières nuit et jour,
\VS{4}me souvenant de tes larmes, je désire fort te voir afin que je sois rempli de joie.
\VS{5}Et me souvenant de la foi sincère qui est en toi, et qui a premièrement habité en Loïs, ta grand-mère, et en Eunice, ta mère, et qui, je suis persuadé qu'elle habite aussi en toi.
\VS{6}C'est pourquoi je t'exhorte de ranimer le don de Dieu qui est en toi par l'imposition de mes mains.
\VS{7}Car Dieu ne nous a pas donné un esprit de timidité, mais de force, de charité (1) et de sagesse.
\VS{8}N’aie donc point honte du témoignage à rendre à notre Seigneur ni de moi, qui suis son prisonnier ; mais souffre avec moi les afflictions de l'Evangile, selon la puissance de Dieu,
\VS{9}qui nous a sauvés et qui nous a appelés par une sainte vocation, non selon nos œuvres, mais selon son propre dessein, et selon la grâce qui nous a été donnée en Jésus-Christ avant les temps éternels,
\VS{10}et qui maintenant a été manifestée par l'apparition de notre Sauveur Jésus-Christ, qui a détruit la mort et qui a mis en lumière la vie et l'immortalité par l'Evangile,
\VS{11}pour lequel j'ai été établi prédicateur, apôtre et docteur des Gentils.
\VS{12}C'est pourquoi aussi je souffre ces choses, mais je n'en ai point de honte ; car je connais celui en qui j'ai cru, et je suis persuadé qu'il est Puissant pour garder mon dépôt (2) jusqu'à ce jour-là.
\VS{13}Retiens dans la foi et dans la charité qui est en Jésus-Christ le modèle des saines paroles que tu as apprises de moi.
\VS{14}Garde le bon dépôt par le Saint-Esprit qui habite en nous.
\VS{15}Tu sais que tous ceux qui sont en Asie se sont détournés de moi ; entre lesquels sont Phygelle et Hermogène.
\VS{16}Que le Seigneur accorde sa miséricorde à la maison d'Onésiphore, car souvent il m'a consolé, et il n'a point eu honte de mes chaînes.
\VS{17}Au contraire, quand il a été à Rome, il m'a cherché avec beaucoup d’empressement, et il m'a trouvé.
\VS{18}Que le Seigneur lui fasse trouver miséricorde envers le Seigneur en ce jour-là ; et tu sais mieux que personne combien il m'a rendu de services à Ephèse.
\TextTitle{[Timothée encouragé à poursuivre les travaux de son ministère avec constance et fidélité]}
\Chap{2}
\VerseOne{}Toi donc, mon fils, sois fortifié dans la grâce qui est en Jésus-Christ.
\VS{2}Et les choses que tu as entendues de moi devant plusieurs témoins, confie-les à des personnes fidèles qui soient capables de les enseigner aussi à d'autres.
\VS{3}Toi donc, souffre avec moi comme un bon soldat de Jésus-Christ.
\VS{4}Il n’est pas de soldat qui s'embarrasse des affaires de cette vie s’il veut plaire à celui qui l'a enrôlé pour la guerre.
\VS{5}De même, l’athlète qui combat n'est point couronné s'il n'a pas combattu selon les règles.
\VS{6}Il faut aussi que le laboureur travaille premièrement, et ensuite il recueille les fruits.
\VS{7}Considère ce que je dis, car le Seigneur te donne de l’intelligence en toutes choses.
\VS{8}Souviens-toi que Jésus-Christ, qui est de la semence de David, est ressuscité des morts, selon mon Evangile,
\VS{9}pour lequel je souffre beaucoup de maux, jusqu'à être mis dans les chaînes comme un malfaiteur, cependant, la parole de Dieu n'est point liée.
\VS{10}C'est pourquoi je souffre tout pour l'amour des élus, afin qu'eux aussi obtiennent le salut qui est en Jésus-Christ, avec la gloire éternelle.
\VS{11}Cette parole est certaine : Si nous mourons avec lui, nous vivrons aussi avec lui.
\VS{12}Si nous souffrons avec lui, nous régnerons aussi avec lui. Si nous le renions, il nous reniera aussi (1).
\VS{13}Si nous sommes infidèles, il demeure fidèle, car il ne peut pas se renier lui-même.
\TextTitle{[Eviter les disputes de mots, les discours profanes et les questions inutiles]}
\VS{14}Rappelle ces choses, en conjurant devant Dieu qu'on évite les disputes de mots, qui ne servent qu'à la ruine de ceux qui écoutent.
\VS{15}Efforce-toi de te présenter devant Dieu comme un homme éprouvé (2), un ouvrier sans reproche, enseignant purement la parole de la vérité.
\VS{16}Mais évite les discours vains et profanes ; car ceux qui les tiennent avanceront toujours plus dans l'impiété,
\VS{17}et leur parole rongera comme une gangrène. Et parmi ceux-là sont Hyménée et Philète,
\VS{18}qui se sont écartés (3) de la vérité, en disant que la résurrection est déjà arrivée, et qui renversent la foi de quelques-uns.
\VS{19}Toutefois, le fondement de Dieu demeure ferme, ayant ce sceau : Le Seigneur connaît ceux qui lui appartiennent (4) ; et : Quiconque invoque le nom du Seigneur, qu'il s’éloigne de l'iniquité.
\VS{20}Or dans une grande maison, il n'y a pas seulement des vases d'or et d'argent, mais il y en a aussi de bois et de terre. Les uns sont des vases d’honneur et les autres sont d’un usage vil.
\VS{21}Si quelqu'un donc se purifie de ces choses, il sera un vase d’honneur, sanctifié et utile au Seigneur, et préparé à toute bonne œuvre.
\VS{22}Fuis aussi les désirs de la jeunesse, et recherche la justice, la foi, la charité, et la paix avec ceux qui invoquent le Seigneur d'un cœur pur.
\VS{23}Et rejette les questions (5) folles, et qui sont sans instruction, sachant qu'elles ne font que produire des querelles.
\VS{24}Or, il ne faut pas que le serviteur du Seigneur soit querelleur, il doit au contraire avoir de la douceur envers tout le monde, propre à enseigner, supportant patiemment les mauvais,
\VS{25}il doit corriger avec douceur ceux qui ont un sentiment contraire, afin que Dieu leur donne la repentance pour reconnaître la vérité,
\VS{26}et afin qu'ils se réveillent pour sortir des pièges du diable, par lesquels ils ont été pris pour faire sa volonté.
\TextTitle{[Prédiction d'une grande corruption de mœurs]}
\Chap{3}
\VerseOne{}Or sache que dans les derniers jours (1) il surviendra des temps difficiles.
\VS{2}Car les hommes seront idolâtres d’eux-mêmes, amis de l’argent, fanfarons, orgueilleux, blasphémateurs, rebelles à leurs parents, ingrats, irréligieux,
\VS{3}sans affection naturelle, sans fidélité, calomniateurs, intempérants, cruels, haïssant les gens de bien,
\VS{4}traîtres, emportés, enflés d'orgueil, aimant le plaisir plus que de Dieu,
\VS{5}ayant l'apparence (2) de la piété, mais en ayant renié ce qui en fait la force. Eloigne-toi donc de tous ces hommes là.
\VS{6}Il en est parmi eux qui se glissent dans les maisons et qui capturent les femmes chargées de péchés et agitées de diverses convoitises,
\VS{7}qui apprennent toujours, mais qui ne peuvent jamais parvenir à la pleine connaissance de la vérité.
\VS{8}Et comme Jannès et Jambrès ont résisté à Moïse, de même ces hommes résistent à la vérité, étant des gens qui ont l'esprit corrompu, et qui sont réprouvés quant à la foi.
\VS{9}Mais ils n'avanceront pas toujours, car leur folie sera manifestée à tous, comme le fut celle de ceux-là.
\TextTitle{[Exhortation à la fermeté dans la foi. Excellence de l'Ecriture]}
\VS{10}Mais toi, tu as pleinement compris ma doctrine, ma conduite, mon intention, ma foi, ma douceur, ma charité, ma persévérance.
\VS{11}Et tu sais les persécutions et les afflictions qui me sont arrivées à Antioche, à Iconie, et à Lystre. Quelles persécutions n’ai-je pas supportées ? Et le Seigneur m'a délivré de toutes.
\VS{12}Or tous ceux aussi qui veulent vivre pieusement en Jésus-Christ seront persécutés.
\VS{13}Mais les hommes méchants et imposteurs iront en empirant, séduisant les autres, et étant séduits eux-mêmes.
\VS{14}Mais toi, demeure ferme dans les choses que tu as apprises et reconnues certaines, sachant de qui tu les as apprises,
\VS{15}vu que depuis ton enfance tu as la connaissance des saintes Lettres, qui peuvent te rendre sage pour le salut par la foi en Jésus-Christ.
\VS{16}Toute l'Ecriture est inspirée de Dieu, et utile pour enseigner, pour convaincre, pour corriger, et pour instruire selon la justice,
\VS{17}afin que l'homme de Dieu soit accompli et parfaitement instruit pour toute bonne œuvre.
\TextTitle{[Devoirs des pasteurs - Paul annonce sa mort prochaine ]}
\Chap{4}
\VerseOne{}Je te somme devant Dieu, et devant le Seigneur Jésus-Christ, qui doit juger les vivants et les morts, lors de son apparition et de son règne.
\VS{2}Prêche la parole, insiste en toute occasion, favorable ou non. Reprends, censure, exhorte avec toute douceur d'esprit, et avec doctrine.
\VS{3}Car il viendra un temps où les hommes ne supporteront pas la saine doctrine, mais aimant qu'on leur chatouille les oreilles par des discours agréables, ils chercheront des docteurs qui répondent à leurs désirs (1).
\VS{4}Et ils détourneront leurs oreilles de la vérité, et se tourneront vers les fables.
\VS{5}Mais toi, veille en toutes choses, souffre les afflictions, fais l'œuvre d'un évangéliste, rends ton ministère pleinement approuvé.
\VS{6}Car pour moi, je m'en vais maintenant servir de libation, et le temps de mon départ est proche.
\VS{7}J'ai combattu le bon combat, j'ai achevé la course, j'ai gardé la foi.
\VS{8}Au reste, la couronne de justice m'est réservée, et le Seigneur, juste Juge, me la rendra en ce jour-là, et non seulement à moi, mais aussi à tous ceux qui auront aimé son apparition.
\TextTitle{[Recommandations diverses - Confiance de l'Apôtre dans le Seigneur]}
\VS{9}Hâte-toi de venir bientôt vers moi.
\VS{10}Car Démas m'a abandonné, ayant aimé le présent siècle, et il s'en est allé à Thessalonique ; Crescens est allé en Galatie ; et Tite en Dalmatie.
\VS{11}Luc est seul avec moi ; prends Marc, et amène-le avec toi, car il m'est fort utile pour le ministère.
\VS{12}J'ai aussi envoyé Tychique à Ephèse.
\VS{13}Quand tu viendras, apporte avec toi le manteau que j'ai laissé à Troas, chez Carpus, et les livres aussi ; mais principalement mes parchemins.
\VS{14}Alexandre le forgeron m'a fait beaucoup de mal. Le Seigneur lui rendra selon ses œuvres.
\VS{15}Garde-toi donc de lui, car il s'est fortement opposé à nos paroles.
\VS{16}Personne ne m'a assisté dans ma première défense, mais tous m'ont abandonné ; toutefois que cela ne leur soit point imputé !
\VS{17}Mais le Seigneur m'a assisté et fortifié, afin que ma prédication soit pleinement approuvée, et que tous les Gentils l’entendent ; et j'ai été délivré de la gueule du lion.
\VS{18}Le Seigneur aussi me délivrera de toute mauvaise œuvre, et me sauvera dans son Royaume céleste. A lui soit la gloire aux siècles des siècles. Amen !
\VS{19}Salue Priscille et Aquilas, et la famille d'Onésiphore.
\VS{20}Eraste est resté à Corinthe, et j'ai laissé Trophime malade à Milet.
\VS{21}Hâte-toi de venir avant l'hiver. Eubulus et Pudens, et Linus, et Claudia, et tous les frères te saluent.
\VS{22}Que le Seigneur Jésus-Christ soit avec ton esprit. Que la grâce soit avec vous. Amen !
\PPE{}
\end{multicols}
