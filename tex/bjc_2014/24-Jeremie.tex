\ShortTitle{Jé.}\BookTitle{Jérémie}\BFont
\noindent\hrulefill
{\footnotesize
\textit{
\bigskip
{\centering{}
\\Auteur~: Jérémie
\\(Heb.~: Yirmeyah)
\\Signification~: Celui que Yahweh a désigné
\\Thème~: Avertissements et jugements
\\Date de rédaction~: 7\up{ème} siècle av J.-C.\\}
}
\textit{
\\Issu d'une famille de prêtres, Jérémie fut appelé dès son plus jeune âge au service de Yahweh et exerça un service prophétique avant et pendant les premières années de déportation. Outre son message à Israël et aux nations, le livre de Jérémie révèle sa personnalité. On découvre alors que l'opposition de ses pairs fut l'une de ses expériences les plus douloureuses. En effet, ce récit raconte ses combats contre les faux prophètes et met en évidence les signes accompagnant les prophètes authentiques, à savoir la souffrance, la solitude, l'incompréhension et le rejet.
\\Son message annonçait le jugement imminent de Dieu et invitait le peuple à la repentance pour éviter le châtiment de Yahweh. Après la chute de Jérusalem, alors que Nebucadnetsar lui avait laissé le choix, Jérémie décida de rester avec les plus pauvres plutôt que de partir pour Babylone. Cependant, des Israélites décidèrent de s'expatrier en Egypte et l'entraînèrent avec eux de force. En terre étrangère, Jérémie continua de porter le fardeau de son peuple, l'exhortant à réformer ses voies. 
\\Parmi les prophéties de Jérémie, figure le retour du peuple d'Israël sur la terre promise avant la seconde venue de Christ.\bigskip
}
}
\par\nobreak\noindent\hrulefill
\begin{multicols}{2}
\Chap{1}
\TextTitle{Yahweh appelle Jérémie à son service}
\VerseOne{}Les Paroles de Jérémie, fils de Hilkija, d'entre les prêtres qui étaient à Anathoth, dans le pays de Benjamin~;
\VS{2}auquel fut adressée la parole de Yahweh aux jours de Josias, fils d'Amon, roi de Juda, la treizième année de son règne,
\VS{3}laquelle lui fut aussi adressée aux jours de Jojakim, fils de Josias, roi de Juda, jusqu'à la fin de la onzième année de Sédécias, fils de Josias, roi de Juda~; à savoir jusqu'au temps où Jérusalem fut transportée, ce qui arriva au cinquième mois.
\VS{4}La parole de Yahweh me fut adressée, en disant~:
\VS{5}Avant que je t'aie formé dans le ventre de ta mère, je te connaissais, et avant que tu sois sorti de son sein, je t'avais consacré, je t'avais établi prophète pour les nations\FTNT{Es. 49:5~; Ga. 1:15.}.
\VS{6}Je répondis~: Ah~! Seigneur Yahweh~! Voici, je ne sais pas parler, car je suis un enfant\FTNT{Ex. 4:10-11.}.
\VS{7}Et Yahweh me dit~: Ne dis pas~: Je suis un enfant. Car tu iras partout où je t'enverrai, et tu diras tout ce que je t'ordonnerai.
\VS{8}Ne crains pas de te montrer devant eux, car je suis avec toi pour te délivrer, dit Yahweh.
\VS{9}Puis Yahweh avança sa main, et toucha ma bouche~; et Yahweh me dit~: Voici, je mets mes paroles dans ta bouche.
\VS{10}Regarde, je t'établis aujourd'hui sur les nations et sur les royaumes, pour que tu arraches et que tu démolisses, pour que tu ruines et que tu détruises, pour que tu bâtisses et que tu plantes\FTNT{Jérémie devait d'abord arracher, démolir, ruiner et détruire avant de bâtir et de planter. Il y avait dans le temple de Jérusalem les autels de Baal et le pieu d'Asherah (2 R. 21). De même, avant de planter la Parole de Dieu qui est une semence plantée dans les cœurs (Mc. 4:3-17), il est nécessaire au préalable d'arracher et de renverser les fausses doctrines et le péché en les dénonçant.}.
\TextTitle{Yahweh confirme la mission de Jérémie et l'établit sur Juda}
\VS{11}Puis la parole de Yahweh me fut adressée, en disant~: Que vois-tu, Jérémie~? Et je répondis~: Je vois une branche d'amandier.
\VS{12}Et Yahweh me dit~: Tu as bien vu~; car je me hâte d'exécuter ma parole.
\VS{13}La parole de Yahweh me fut adressée pour la seconde fois, en disant~: Que vois-tu~? Et je répondis~: Je vois un pot bouillant dont le devant est tourné vers le nord.
\VS{14}Et Yahweh me dit~: Le mal se découvrira du côté du nord sur tous les habitants de ce pays-ci.
\VS{15}Car voici, je vais appeler toutes les familles des royaumes du nord, dit Yahweh~; elles viendront, et mettront chacune leur trône à l'entrée des portes de Jérusalem, contre toutes ses murailles à l'entour, et contre toutes les villes de Juda.
\VS{16}Et je prononcerai mes jugements contre eux, à cause de toute leur méchanceté, par laquelle ils m'ont délaissé, et ont fait des parfums à d'autres dieux, et se sont prosternés devant l'ouvrage de leurs mains. 
\VS{17}Toi donc, ceins tes reins, lève-toi, et dis-leur tout ce que je t'ordonnerai. Ne crains pas de te montrer devant eux, de peur que je ne te mette en pièces en leur présence.
\VS{18}Car voici, je t'établis aujourd'hui sur tout le pays comme une ville forte, une colonne de fer, et un mur d'airain, contre les rois de Juda, contre les chefs du pays, contre ses prêtres, et contre le peuple du pays.
\VS{19}Et ils combattront contre toi, mais ils ne seront pas plus forts que toi~; car je suis avec toi, dit Yahweh, pour te délivrer.
\Chap{2}
\TextTitle{Yahweh dénonce l'attitude d'Israël et l'avertit}
\VerseOne{}La parole de Yahweh me fut adressée, en disant~:
\VS{2}Va et crie aux oreilles de Jérusalem, et dis~: Ainsi parle Yahweh~: Je me souviens de la fidélité de ta jeunesse, de l'amour de tes fiançailles, quand tu venais après moi au désert, dans une terre qu'on n'ensemence pas. 
\VS{3}Israël était une chose sainte à Yahweh, il était les prémices de son revenu\FTNT{Lé. 23:20~; Pr. 3:9~; Né. 10:35.}~; tous ceux qui le dévoraient étaient coupables, il leur en arrivait du mal, dit Yahweh.
\VS{4}Ecoutez la parole de Yahweh, maison de Jacob, et vous toutes les familles de la maison d'Israël~!
\VS{5}Ainsi parle Yahweh~: Quelle iniquité vos pères ont-ils trouvée en moi, pour qu'ils se soient éloignés de moi, et qu'ils aient marché après la vanité et soient devenus vains~?
\VS{6}Et ils n'ont pas dit~: Où est Yahweh qui nous a fait remonter du pays d'Egypte, qui nous a conduits par un désert, par un pays de landes et montagneux, par un pays aride et d'ombre de mort, par un pays où aucun homme n'était passé, et où personne n'avait habité~? 
\VS{7}Car je vous ai fait entrer dans un pays de verger, pour que vous en mangiez les fruits et les biens~; mais sitôt que vous y êtes entrés, vous avez souillé mon pays, et vous avez rendu abominable mon héritage.
\VS{8}Les prêtres n'ont pas dit~: Où est Yahweh~? Les dépositaires de la loi ne m'ont pas connu, les pasteurs se sont révoltés contre moi, les prophètes ont prophétisé par Baal\FTNT{Baal. Voir Jg. 2:13.}, et sont allés après ce qui n'est d'aucun profit.
\VS{9}A cause de cela, je veux encore contester avec vous, dit Yahweh, je veux contester avec les fils de vos fils.
\VS{10}Passez par les îles de Kittim et voyez~! Envoyez quelqu'un à Kédar~; observez bien, et voyez s'il n'y a rien de semblable~!
\VS{11}Y a-t-il une nation qui change ses dieux, quoiqu'ils ne soient pas des dieux~? Mais mon peuple a changé sa gloire contre ce qui n'est d'aucun profit\FTNT{Ro. 1:23.}~!
\VS{12}Cieux, soyez étonnés de cela~; frémissez d'horreur et soyez extrêmement asséchés~! dit Yahweh.
\VS{13}Car mon peuple a commis doublement le mal~: Ils m'ont abandonné, moi qui suis la source d'eaux vives\FTNT{Yahweh est la Source d'eaux vives. Jésus-Christ se présente aussi comme la Source d'eau vive (Jn. 4:13-14~; Ap. 21:6).}, pour se creuser des citernes, des citernes crevassées qui ne peuvent pas retenir l'eau.
\VS{14}Israël est-il un esclave, ou un esclave né dans la maison~? Pourquoi donc est-il mis au pillage~?
\VS{15}Les lionceaux rugissent, poussent leurs cris contre lui, et ils mettent son pays en désolation~; ses villes sont brûlées, de sorte que personne n'y habite.
\VS{16}Même les fils de Noph et de Tachpanès te casseront le sommet de la tête.
\VS{17}Cela ne t'arrive-t-il pas parce que tu as abandonné Yahweh, ton Dieu, à l'époque où il te conduisait par le chemin~?
\VS{18}Et maintenant, qu'as-tu à faire d'aller en Egypte, pour boire l'eau du Schichor\FTNT{Schichor, qui signifie~: Sombre, noir, boueux, était l'un des affluent du Nil.}~? Qu'as-tu à faire d'aller en Assyrie, pour boire l'eau du fleuve~?
\VS{19}Ta méchanceté te châtiera, et tes débauches te jugeront~; tu sauras et tu verras que c'est une chose mauvaise et amère d'abandonner Yahweh, ton Dieu, et de n'avoir de moi aucune crainte, dit le Seigneur, Yahweh des armées.
\VS{20}Parce que depuis longtemps j'ai brisé ton joug, et rompu tes liens, tu as dit~: Je ne serai plus dans la servitude~; c'est pourquoi tu as erré en te prostituant sur toute haute colline, et sous tout arbre vert.
\VS{21}Or je t'avais moi-même plantée comme une vigne exquise, dont tout le plant était franc~; comment donc t'es-tu changée pour moi en sarments d'une vigne étrangère~?
\VS{22}Quand tu te laverais avec du nitre, et que tu prendrais beaucoup de savon, ton iniquité resterait encore marquée devant moi, dit le Seigneur, Yahweh.
\VS{23}Comment dis-tu~: Je ne me suis pas souillée, je ne suis pas allée après les Baals~? Regarde tes pas dans la vallée, reconnais ce que tu as fait, dromadaire à la course légère, qui ne tient pas de route certaine~!
\VS{24}Anesse sauvage, accoutumée au désert, humant le vent à son plaisir, qui l'arrêtera dans son ardeur~? Tous ceux qui la cherchent n'ont pas à se fatiguer~; ils la trouvent pendant son mois.
\VS{25}Empêche ton pied de se déchausser, et ton gosier d'avoir soif~! Mais tu dis~: Non, c'est sans espoir~! Car j'aime les étrangers, et j'irai après eux.
\VS{26}Comme un voleur est confus quand il est surpris, ainsi seront confus ceux de la maison d'Israël, eux, leurs rois, leurs chefs, leurs prêtres et leurs prophètes.
\VS{27}Ils disent au bois~: Tu es mon père~! Et à la pierre~: Tu m'as engendré~! Car ils me tournent le dos, et non la face. Et ils disent dans le temps de leur malheur~: Lève-toi, et sauve-nous~!
\VS{28}Où donc sont tes dieux que tu t'es faits~? Qu'ils se lèvent, s'ils peuvent te sauver au temps de ton malheur~! Car tu as autant de dieux que de villes, ô Juda~!
\VS{29}Pourquoi contesteriez-vous avec moi~? Vous vous êtes tous rebellés contre moi, dit Yahweh.
\VS{30}En vain ai-je frappé vos fils~; ils n'ont pas reçu d'instruction~; votre épée a dévoré vos prophètes comme un lion qui ravage tout.
\VS{31}Hommes de cette génération, considérez la parole de Yahweh~! Ai-je été un désert pour Israël, ou un pays de ténèbres~? Pourquoi mon peuple dit-il~: Nous sommes les maîtres, nous ne viendrons plus à toi~?
\VS{32}La vierge oublie-t-elle ses ornements~? L'épouse sa ceinture~? Mais mon peuple m'a oublié depuis des jours sans nombre.
\VS{33}Pourquoi uses-tu de tant d'artifices dans ton attitude pour chercher l'amour~? De la sorte, tu as même enseigné tes manières de faire aux femmes de mauvaise vie~?
\VS{34}Même sur les pans de ta robe se trouve le sang des pauvres innocents, que tu n'as pas trouvés en effraction.
\VS{35}Et tu dis~: Je suis innocente~! Quoi qu'il en soit, sa colère s'est détournée de moi. Voici, je m'en vais contester contre toi, sur ce que tu as dit~: Je n'ai point péché.
\VS{36}Pourquoi tant te précipiter pour changer ton chemin~? Tu auras autant de confusion de l'Egypte, que tu en as eu de l'Assyrie.
\VS{37}Tu sortiras même d'ici, ayant tes mains sur la tête~; car Yahweh rejette les fondements de ta confiance, et tu n'auras aucune prospérité par eux.
\Chap{3}
\TextTitle{Israël comparé à une prostituée}
\VerseOne{}On dit~: Si un homme répudie sa femme, qu'elle le quitte et se joigne à un autre mari, le premier mari retournera-t-il encore vers elle\FTNT{Lé. 21:7~; De. 24:2.}~? Le pays même n'en serait-il pas entièrement souillé~? Or toi, tu t'es prostituée à plusieurs amoureux, toutefois retourne-toi vers moi, dit Yahweh.
\VS{2}Lève tes yeux vers les lieux élevés, et regarde~! Où ne t'es-tu pas prostituée~! Tu te tenais sur les chemins, comme un Arabe dans le désert, et tu as souillé le pays par tes prostitutions et par ta méchanceté.
\VS{3}C'est pourquoi les pluies ont été retenues, et il n'y a pas eu de pluie de l'arrière-saison~; mais tu as eu le front d'une femme prostituée, tu n'as pas voulu avoir honte.
\VS{4}Maintenant, ne crieras-tu pas vers moi~: Mon père~! Tu as été l'ami de ma jeunesse~!
\VS{5}Gardera-t-il à toujours sa colère~? La conservera-t-il à jamais\FTNT{Es. 57:16~; Ps. 103:9.}~? Voici, tu as ainsi parlé, et tu as fait ces maux-là autant que tu as pu.
\TextTitle{Yahweh appelle Israël à la repentance}
\VS{6}Yahweh me dit au temps du roi Josias~: As-tu vu ce qu'a fait Israël, l'infidèle~? Elle est allée sur toute haute colline et sous tout arbre vert, et elle s'y est prostituée.
\VS{7}Je disais~: Après avoir fait toutes ces choses, elle reviendra à moi. Mais elle n'est pas revenue. Et sa sœur Juda, la perfide, l'a vu.
\VS{8}Quoique j'aie répudié Israël, l'infidèle, à cause de tous ses adultères, et que je lui aie donné sa lettre de divorce, j'ai vu que la perfide Juda, sa sœur, n'a pas eu de crainte, mais elle s'en est allée et s'est aussi prostituée.
\VS{9}Et il est arrivé que par la légèreté de sa fornication, elle a souillé le pays, elle a commis un adultère avec la pierre et le bois.
\VS{10}Malgré tout cela, sa sœur Juda, la perfide, n'est pas revenue à moi de tout son cœur~; c'est avec fausseté qu'elle l'a fait, dit Yahweh.
\VS{11}Et Yahweh me dit~: Israël l'infidèle se montre plus juste que Juda la perfide.
\VS{12}Va, crie ces paroles vers le nord, et dis~: Reviens, Israël, l'infidèle, dit Yahweh. Je ne ferai point tomber ma colère sur vous~; car je suis miséricordieux, dit Yahweh, je ne garde pas ma colère à toujours.
\VS{13}Mais reconnais ton iniquité, tu t'es rebellée contre Yahweh, ton Dieu, tu t'es prostituée aux étrangers sous tout arbre vert, et tu n'as pas écouté ma voix, dit Yahweh.
\VS{14}Enfants rebelles, convertissez-vous, dit Yahweh, car j'ai droit de mari sur vous. Je vous prendrai, un d'une ville, deux d'une famille, et je vous ferai entrer dans Sion.
\VS{15}Je vous donnerai des pasteurs selon mon cœur, qui vous paîtront avec intelligence et avec sagesse\FTNT{Jé. 23:5.}.
\VS{16}Lorsque vous aurez multiplié et fructifié dans le pays, en ces jours-là, dit Yahweh, on ne parlera plus de l'arche de l'alliance de Yahweh, elle ne viendra plus à la pensée~; on ne s'en souviendra plus, on ne s'apercevra plus de son absence, et l'on n'en fera pas une autre.
\VS{17}En ce temps-là, on appellera Jérusalem le trône de Yahweh~; toutes les nations s'assembleront à Jérusalem, au nom de Yahweh, et elles ne marcheront plus suivant les penchants de leur mauvais cœur.
\VS{18}En ces jours-là, la maison de Juda marchera avec la maison d'Israël~; elles viendront ensemble du pays du nord au pays que j'ai donné en héritage à vos pères.
\VS{19}Je disais~: Comment te mettrai-je parmi mes fils et te donnerai-je un pays désirable, le plus bel héritage des armées des nations~? Je disais~: Tu m'appelleras~: Mon père~! Et tu ne te détourneras pas de moi.
\VS{20}Mais comme une femme est infidèle à son compagnon, ainsi vous m'avez été infidèles, maison d'Israël, dit Yahweh.
\VS{21}Une voix se fait entendre sur les lieux élevés~; ce sont les pleurs, les supplications des enfants d'Israël~; car ils ont perverti leur voie, ils ont oublié Yahweh, leur Dieu.
\VS{22}Enfants rebelles, convertissez-vous, je guérirai vos infidélités. Nous voici, nous venons à toi, car tu es Yahweh, notre Dieu.
\VS{23}Certainement, on s'attend en vain aux collines et à la multitude des montagnes~; mais c'est en Yahweh, notre Dieu, qu'est la délivrance d'Israël.
\VS{24}Car la honte a dévoré dès notre jeunesse le travail de nos pères, leurs brebis et leurs bœufs, leurs fils et leurs filles.
\VS{25}Nous serons gisants dans notre honte, et notre ignominie nous couvrira~; parce que nous avons péché contre Yahweh, notre Dieu, nous et nos pères, dès notre jeunesse jusqu'à ce jour, et nous n'avons pas obéi à la voix de Yahweh, notre Dieu.
\Chap{4}
\TextTitle{Prophétie sur l'invasion du pays}
\VerseOne{}Israël, si tu reviens, dit Yahweh, si tu reviens à moi, si tu ôtes tes abominations de devant moi, tu ne seras plus errant ça et là.
\VS{2}Alors tu jureras avec vérité, avec droiture et avec justice~: Yahweh est vivant~! Et les nations seront bénies en lui, et se glorifieront en lui.
\VS{3}Car ainsi parle Yahweh aux hommes de Juda et de Jérusalem~: Labourez pour vous une terre arable, et ne semez pas parmi les épines\FTNT{Mt. 13:7~; Mt. 13:22~; Mc. 4:7~; Mc. 4:18~; Lu. 8:14.}.
\VS{4}Hommes de Juda, et vous habitants de Jérusalem, circoncisez-vous pour Yahweh, circoncisez vos cœurs\FTNT{Ro. 2:29.}, de peur que ma fureur ne sorte comme un feu, et qu'elle ne brûle sans qu'on puisse l'éteindre, à cause de la méchanceté de vos actions.
\VS{5}Annoncez en Juda, publiez dans Jérusalem, et dites~: Sonnez du shofar dans le pays~! Criez à pleine voix et dites~: Assemblez-vous, et nous entrerons dans les villes fortes~!
\VS{6}Elevez une bannière vers Sion, fuyez, ne vous arrêtez pas~! Car je fais venir du nord le malheur et une grande calamité.
\VS{7}Le lion\FTNT{Lion est ici une allusion à Nebucadnetsar, roi de Babylone. Voir 2 R. 24-25~; Da. 7:4.} est sorti de la caverne, le destructeur des nations est en marche, il est sorti de son lieu, pour réduire ton pays en désert~; tes villes seront ruinées, il n'y aura personne pour y habiter.
\VS{8}C'est pourquoi ceignez-vous de sacs, lamentez-vous et gémissez~; car l'ardeur de la colère de Yahweh ne se détourne pas de nous.
\VS{9}Et il arrivera ce jour-là, dit Yahweh, que le cœur du roi et le cœur des chefs seront épouvantés, que les prêtres seront étonnés, et que les prophètes seront stupéfaits.
\VS{10}C'est pourquoi je dis~: Ah~! Seigneur Yahweh~! Oui certainement tu as abusé ce peuple et Jérusalem, en disant~: Vous aurez la paix~! Et cependant l'épée est venue jusqu'à l'âme.
\VS{11}En ce temps-là on dira à ce peuple et à Jérusalem~: Un vent brûlant souffle des lieux élevés du désert sur le chemin de la fille de mon peuple, non pas pour vanner ni pour nettoyer.
\VS{12}C'est un vent impétueux qui vient de là jusqu'à moi, et je leur ferai maintenant leur procès. 
\VS{13}Voici, il monte comme des nuées~; ses chars sont comme un tourbillon, ses chevaux sont plus légers que les aigles. Malheur à nous, car nous sommes détruits~!
\VS{14}Jérusalem, lave ton cœur du mal afin que tu sois délivrée~! Jusqu'à quand séjourneront au-dedans de toi les pensées de ton injustice~?
\VS{15}Car une voix apporte des nouvelles de Dan, elle publie depuis la montagne d'Ephraïm le tourment.
\VS{16}Rappelez-le aux nations, faites-le entendre à Jérusalem~: Des observateurs viennent d'un pays éloigné~; ils poussent des cris contre les villes de Juda.
\VS{17}Ils se sont mis tout autour d'elle comme ceux qui gardent un champ, parce qu'elle s'est rebellée contre moi, dit Yahweh.
\VS{18}Ta conduite et tes actions t'ont produit ces choses, telle a été ta méchanceté, parce que cela a été une chose amère, certainement elle t'atteindra jusqu'au cœur.
\VS{19}Mes entrailles~! Mes entrailles~! Je suis dans la douleur au-dedans de mon cœur, mon cœur bat, je ne puis me taire~; car, ô mon âme, tu entends le son du shofar, la clameur de la guerre.
\VS{20}On annonce brèche sur brèche, car tout le pays est dévasté~; mes tentes sont détruites tout à coup, mes pavillons en un moment.
\VS{21}Jusqu'à quand verrai-je la bannière et entendrai-je le son du shofar~?
\VS{22}Car mon peuple est insensé~; ils ne m'ont pas reconnu, ce sont des enfants insensés, qui n'ont pas d'intelligence~; ils sont habiles pour faire le mal, et ils ne savent pas faire le bien.
\VS{23}Je regarde la terre, et voici, elle est informe et vide\FTNT{Voir Ge. 1:2.}~; les cieux et leur lumière ne sont plus.
\VS{24}Je regarde les montagnes, et voici, elles sont ébranlées~; et toutes les collines sont renversées.
\VS{25}Je regarde, et voici, il n'y a pas un seul homme et tous les oiseaux des cieux se sont enfuis.
\VS{26}Je regarde, et voici, le Carmel est un désert~; et toutes ses villes sont détruites, devant Yahweh, devant l'ardeur de sa colère.
\VS{27}Car ainsi parle Yahweh~: Tout le pays sera dévasté, mais je ne ferai pas une entière destruction.
\VS{28}C'est pourquoi le pays mènera deuil et les cieux en haut seront obscurcis, parce que je l'ai dit, je l'ai résolu, et je ne m'en repentirai pas et je ne le révoquerai pas.
\VS{29}Toute la ville s'enfuit à cause du bruit des cavaliers et des archers~; ils entrent dans les bois fourrés, et montent sur les rochers~; toute la ville est abandonnée, et aucun homme n'y habite.
\VS{30}Et quand tu auras été détruite que feras-tu~? Quoique tu te revêtes de pourpre, que tu te pares d'ornements d'or, et que tu bordes tes yeux de fard, tu t'embellis en vain~; tes amants t'ont méprisée, c'est ta vie qu'ils cherchent.
\VS{31}Car j'entends un cri comme celui d'une femme qui est en travail, et une angoisse comme celle d'une femme qui est en travail de son premier-né~; c'est le cri de la fille de Sion~; elle soupire, elle étend ses mains, en disant~: Malheur maintenant à moi, car mon âme a défailli à cause des meurtriers~! 
\Chap{5}
\TextTitle{Raisons du jugement de Yahweh}
\VerseOne{}Parcourez les rues de Jérusalem et regardez maintenant, sachez et cherchez dans les places, si vous y trouvez un homme de bien, s'il y a quelqu'un qui fasse ce qui est droit, qui cherche la vérité, et je pardonne à Jérusalem\FTNT{Es. 59:15~; Mi. 7:2~; Pr. 20:6.}.
\VS{2}Même s'ils disent~: Yahweh est vivant~! En cela, ils jurent faussement.
\VS{3}Yahweh, tes yeux ne regardent-ils pas à la fidélité~? Tu les frappes, et ils ne sentent pas de douleur~; tu les consumes, et ils refusent de recevoir l'instruction~; ils endurcissent leurs faces plus qu'un rocher, ils refusent de se convertir.
\VS{4}Je disais~: Certainement ce ne sont que les plus petits~; ils se montrent insensés parce qu'ils ne connaissent pas la voie de Yahweh, le jugement de leur Dieu.
\VS{5}J'irai donc vers les plus grands, et je leur parlerai~; car ceux-là connaissent la voie de Yahweh, le jugement de leur Dieu~; mais ceux-là même ont brisés le joug et ont rompu les liens.
\VS{6}C'est pourquoi le lion de la forêt les tue, le loup du soir les détruit, et le léopard est aux aguets contre leurs villes~; quiconque en sortira sera déchiré~; car leurs transgressions sont nombreuses, et leurs infidélités se sont renforcées.
\VS{7}Comment te pardonnerais-je en cela~? Tes fils m'ont abandonné, et ils jurent par ceux qui ne sont pas dieux. Je les ai rassasiés, mais ils commettent l'adultère et ils se pressent en foule dans la maison de la prostituée.
\VS{8}Ils sont comme des chevaux bien nourris, quand ils se lèvent le matin, chacun hennit après la femme de son prochain.
\VS{9}Ne punirais-je pas ces choses-là, dit Yahweh~? Et mon âme ne se vengerait-elle pas d'une telle nation~?
\VS{10}Montez sur ses murailles et détruisez-les, mais ne les achevez pas entièrement~! Otez ses sarments car ils ne sont pas à Yahweh\FTNT{Jn. 15:5.}~!
\VS{11}Car la maison d'Israël et la maison de Juda m'ont été infidèles, dit Yahweh.
\VS{12}Ils trompent Yahweh, et disent~: Cela n'arrivera pas, et le malheur ne viendra pas sur nous, nous ne verrons ni l'épée ni la famine.
\VS{13}Et les prophètes sont légers comme le vent, et la parole n'est pas en eux. Qu'il leur soit fait ainsi~!
\VS{14}C'est pourquoi ainsi parle Yahweh, le Dieu des armées~: Parce que vous avez prononcé cette parole-là, voici, je vais mettre mes paroles dans ta bouche pour y être comme un feu, et ce peuple sera comme le bois, et ce feu les consumera.
\VS{15}Maison d'Israël, voici, je fais venir contre vous une nation d'un pays éloigné\FTNT{Il s'agit de Babylone. Voir 2 R. 24-25.}, dit Yahweh, une nation puissante, une nation ancienne, une nation dont tu ne connais pas la langue, et dont tu ne comprendras pas ce qu'elle dira.
\VS{16}Son carquois est comme un sépulcre ouvert, et ils sont tous des hommes vaillants.
\VS{17}Et elle dévorera ta moisson et ton pain, que tes fils et tes filles devaient manger~; elle dévorera tes brebis et tes bœufs~; elle dévorera les fruits de ta vigne et de ton figuier, et réduira à la pauvreté par l'épée tes villes fortes dans lesquelles tu te confies.
\VS{18}Toutefois en ces jours-là, dit Yahweh, je ne vous achèverai pas entièrement.
\VS{19}Et il arrivera que vous direz~: Pourquoi Yahweh, notre Dieu, nous a-t-il fait toutes ces choses~? Tu leur diras ainsi~: Comme vous m'avez abandonné et que vous avez servi les dieux étrangers dans votre pays, ainsi vous servirez des étrangers dans un pays qui n'est pas le vôtre.
\VS{20}Annoncez ceci dans la maison de Jacob, et publiez-le dans Juda, en disant~:
\VS{21}Ecoutez maintenant ceci, peuple insensé, et qui n'avez pas d'intelligence~; qui avez des yeux et ne voyez pas~; et qui avez des oreilles et n'entendez pas\FTNT{Ez. 12:2~; Jn. 12:40.}.
\VS{22}Ne me craindrez-vous pas, dit Yahweh, ne tremblerez-vous pas devant ma face~? C'est moi qui ai mis le sable pour limite à la mer, par une ordonnance perpétuelle et qui ne passera pas~; ses vagues s'agitent, mais elles sont impuissantes~; elles grondent, mais elles ne la passent pas\FTNT{Pr. 8:29~; Job 38:8.}.
\VS{23}Mais ce peuple-ci a un cœur indocile et rebelle~; ils reculent en arrière et s'en vont.
\VS{24}Et ils ne disent pas dans leur cœur~: Craignons maintenant Yahweh, notre Dieu, qui nous donne la pluie en son temps, de la première et de l'arrière-saison, et qui nous réserve les semaines ordonnées pour la moisson.
\VS{25}Vos iniquités ont détourné ces choses, vos péchés retiennent loin de vous le bien.
\VS{26}Car il se trouve parmi mon peuple des méchants~; ils épient comme l'oiseleur qui dresse des pièges, ils tendent des filets et prennent des hommes\FTNT{Ps. 91:3~; 124:7.}.
\VS{27}Comme la cage est remplie d'oiseaux, ainsi leurs maisons sont remplies de fraude~; c'est par ce moyen qu'ils deviennent grands et riches.
\VS{28}Ils s'engraissent, ils sont brillants~; ils surpassent les actions des méchants, ils ne jugent pas la cause, la cause de l'orphelin, et ils prospèrent~; ils ne font pas droit aux pauvres.
\VS{29}Ne punirais-je pas ces choses-là, dit Yahweh~? Et mon âme ne se vengerait-elle pas d'une telle nation~?
\VS{30}Il est arrivé dans le pays une chose étonnante et horrible~:
\VS{31}C'est que les prophètes prophétisent le mensonge, et les prêtres dominent par leur moyen, et mon peuple prend plaisir à cela. Que ferez-vous donc quand elle prendra fin~?
\Chap{6}
\TextTitle{Jérusalem dans la confusion}
\VerseOne{}Fils de Benjamin, fuyez par troupes du milieu de Jérusalem, sonnez du shofar à Tekoa, et élevez un signal de feu à Beth-Hakkérem~! Car on voit venir du nord un malheur et une grande ruine.
\VS{2}La belle et la délicate, la fille de Sion, je la détruis~!
\VS{3}Les pasteurs avec leurs troupeaux viennent contre elle~; ils plantent leurs tentes autour d'elle, chacun paîtra en son quartier.
\VS{4}Préparez le combat contre elle~! Levez-vous, et montons en plein midi~!… Malheur à nous, car le jour décline, les ombres du soir s'étendent.
\VS{5}Levez-vous~! Montons de nuit, et ruinons ses palais~!
\VS{6}Car ainsi parle Yahweh des armées~: Coupez des arbres, élevez des terrasses contre Jérusalem~! C'est la ville qui doit être visitée~; tout est oppression au milieu d'elle.
\VS{7}Comme le puits fait jaillir ses eaux, ainsi elle fait jaillir sa méchanceté~; devant moi, on n'entend continuellement en elle, que violence et ruine, avec des maladies et des plaies.
\VS{8}Jérusalem, reçois l'instruction, de peur que mon âme ne se retire de toi, et que je ne fasse de toi un désert, et une terre inhabitée~!
\VS{9}Ainsi parle Yahweh des armées~: On grappillera entièrement comme une vigne les restes d'Israël. Remets ta main dans les paniers, comme un vendangeur.
\VS{10}A qui parlerai-je, et qui prendrai-je à témoin, pour qu'ils écoutent~? Voici, leur oreille est incirconcise, et ils ne peuvent entendre~; voici, la parole de Yahweh leur est en opprobre, ils n'y prennent point de plaisir.
\VS{11}C'est pourquoi je suis plein de la fureur de Yahweh, et je suis las de la contenir. Répands-la sur les enfants dans la rue, et sur les assemblées des jeunes gens. Car tant le mari que la femme seront pris, le vieillard et celui qui est chargé de jours.
\VS{12}Et leurs maisons passeront à d'autres, les champs et les femmes aussi, quand j'étendrai ma main sur les habitants du pays, dit Yahweh.
\VS{13}Car depuis le plus petit d'entre eux jusqu'au plus grand, chacun s'adonne au gain déshonnête, tant le prophète que le prêtre, tous agissent faussement.
\VS{14}Et ils pansent à la légère la plaie de la fille de mon peuple, disant~: Paix~! Paix~! Et il n'y a pas de paix\FTNT{1 Th. 5:3.}.
\VS{15}Sont-ils confus d'avoir commis des abominations~? Ils n'en ont même aucune honte, et ils ne savent pas ce que c'est que de rougir~; c'est pourquoi ils tomberont parmi ceux qui tombent, ils seront renversés au temps où je les visiterai, dit Yahweh.
\VS{16}Ainsi parle Yahweh~: Tenez-vous sur les chemins, regardez et enquérez-vous des sentiers des siècles passés, quel est le bon chemin~; et marchez-y, et vous trouverez le repos de vos âmes~! Et ils répondent~: Nous n'y marcherons pas.
\VS{17}J'ai aussi établi sur vous des sentinelles\FTNT{Es. 21:6~; Ez. 33:1-19.} qui disent~: Soyez attentifs au son du shofar~! Mais ils répondent~: Nous n'y serons pas attentifs.
\VS{18}Vous donc, nations, écoutez, et toi assemblée, connais ce qui est entre eux.
\VS{19}Ecoute, terre~! Voici, je fais venir un mal sur ce peuple, à savoir le fruit de leurs pensées~; car ils n'ont pas été attentifs à mes paroles, et qu'ils ont rejeté ma loi.
\VS{20}Pourquoi m'offrir de l'encens venu de Séba, et le bon roseau aromatique du pays éloigné~? Vos holocaustes ne me plaisent pas, et vos sacrifices ne me sont pas agréables.
\VS{21}C'est pourquoi ainsi parle Yahweh~: Voici, je mettrai devant ce peuple des pierres d'achoppement, auxquels les pères et les fils, le voisin et son compagnon, se heurteront ensemble, et ils périront.
\VS{22}Ainsi parle Yahweh~: Voici, un peuple vient du pays du nord, et une grande nation se réveille des extrémités de la terre.
\VS{23}Ils prennent l'arc et le javelot~; ils sont cruels et n'ont pas de pitié~; leur voix gronde comme la mer~; ils sont montés sur des chevaux, ils sont rangés comme un seul homme en bataille contre toi, fille de Sion~!
\VS{24}Nous en entendons le bruit, nos mains en deviennent lâches, l'angoisse nous saisit, et une douleur comme celle d'une femme qui enfante.
\VS{25}Ne sortez pas dans les champs, n'allez pas par les chemins~; car l'épée de l'ennemi et la terreur sont partout.
\VS{26}Fille de mon peuple, ceins-toi d'un sac et roule-toi dans la cendre, prends le deuil comme pour un fils unique, fais une lamentation très amère~! Car le dévastateur vient subitement sur nous.
\VS{27}Je t'avais établi en observateur au milieu de mon peuple, comme une forteresse, pour que tu connaisses et que tu éprouves leur voie.
\VS{28}Ils sont tous rebelles et plus que rebelles, des calomniateurs, ils sont comme de l'airain et du fer~; ils sont tous corrompus.
\VS{29}Le soufflet est brûlant, le plomb est consumé par le feu~; c'est en vain que l'on fond et refond, car les mauvais ne sont pas séparés.
\VS{30}On les appelle de l'argent réprouvé, car Yahweh les a réprouvés.
\Chap{7}
\TextTitle{Hypocrisie de Juda}
\VerseOne{}La parole fut adressée à Jérémie de la part de Yahweh, en disant~:
\VS{2}Tiens-toi debout à la porte de la maison de Yahweh, et là, crie cette parole, et dis~: Ecoutez la parole de Yahweh, vous tous, hommes de Juda, qui entrez par ces portes, pour vous prosterner devant Yahweh~!
\VS{3}Ainsi parle Yahweh des armées, le Dieu d'Israël~: Amendez vos voies et vos actions, et je vous ferai habiter en ce lieu-ci.
\VS{4}Ne vous confiez pas en des paroles trompeuses, en disant~: C'est ici le temple de Yahweh, le temple de Yahweh, le temple de Yahweh~!
\VS{5}Mais amendez sérieusement vos voies et vos actions, et appliquez-vous à faire droit à ceux qui plaident l'un contre l'autre,
\VS{6}et ne faites pas de tort à l'étranger, ni à l'orphelin, ni à la veuve, et ne répandez pas en ce lieu-ci le sang innocent, et ne marchez pas après les dieux étrangers, pour votre malheur.
\VS{7}Et je vous ferai habiter depuis un siècle jusqu'à l'autre siècle en ce lieu-ci, dans le pays que j'ai donné à vos pères.
\VS{8}Voici, vous vous confiez en des paroles trompeuses, sans aucun profit.
\VS{9}Ne dérobez-vous pas~? Ne tuez-vous pas~? Ne commettez-vous pas adultère~? Ne jurez-vous pas faussement~? Ne faites-vous pas des encensements à Baal~? N'allez-vous pas après les dieux étrangers, que vous ne connaissez point~?
\VS{10}Toutefois vous venez et vous vous présentez devant moi, dans cette maison sur laquelle mon Nom est invoqué, et vous dites~: Nous sommes délivrés~!… Pour faire toutes ces abominations~!
\VS{11}N'est-elle plus à vos yeux qu'une caverne de voleurs\FTNT{Mt. 21:13~; Mc. 11:17~; Lu. 19:46.}, cette maison sur laquelle mon Nom est invoqué~? Et voici, moi-même je le vois, dit Yahweh.
\VS{12}Mais allez maintenant à mon lieu qui était à Silo, où j'avais fait demeurer mon Nom au commencement. Et regardez ce que je lui ai fait, à cause de la méchanceté de mon peuple d'Israël.
\VS{13}Maintenant donc, puisque vous avez fait toutes ces actions, dit Yahweh, puisque je vous ai parlé, me levant dès le matin et parlant, et que vous n'avez pas écouté, puisque je vous ai appelés et que vous n'avez pas répondu~;
\VS{14}je ferai à cette maison sur laquelle mon Nom est invoqué, et sur laquelle vous vous confiez, et à ce lieu que je vous ai donné à vous et à vos pères, comme j'ai fait à Silo~;
\VS{15}et je vous chasserai de devant ma face, comme j'ai chassé tous vos frères, avec toute la postérité d'Ephraïm.
\VS{16}Toi donc ne prie pas pour ce peuple, n'élève pour eux ni cri ni prière, et n'intercède pas auprès de moi\FTNT{Ez. 3:26-27.}~; car je ne t'écouterai pas.
\VS{17}Ne vois-tu pas ce qu'ils font dans les villes de Juda et dans les rues de Jérusalem~?
\VS{18}Les fils ramassent le bois, et les pères allument le feu, et les femmes pétrissent la pâte pour faire des gâteaux à la reine des cieux\FTNT{La reine des cieux est une déesse qui change de nom en fonction des pays. Asherah, Astarté, Isis, Junon, Cybèle, Diane ou encore la vierge Marie, proclamée mère de Dieu en 431 au concile d'Ephèse. Voir De. 16:2-3.}, et pour faire des libations aux dieux étrangers, afin de m'irriter.
\VS{19}Est-ce moi qu'ils irritent~? dit Yahweh~; n'est-ce pas eux-mêmes, à la confusion de leurs propres faces~?
\VS{20}C'est pourquoi ainsi parle le Seigneur Yahweh~: Voici, ma colère et ma fureur se répandent sur ce lieu-ci, sur les hommes et sur les bêtes, sur les arbres des champs et sur le fruit de la terre~; ma colère brûlera et ne s'éteindra pas.
\VS{21}Ainsi parle Yahweh des armées, le Dieu d'Israël~: Ajoutez vos holocaustes à vos sacrifices, et mangez-en la chair~!
\VS{22}Car je n'ai pas parlé avec vos pères et je ne leur ai pas donné d'ordre au sujet des holocaustes et des sacrifices, le jour où je les ai fait sortir du pays d'Egypte.
\VS{23}Mais voici la parole que je leur ai commandée, disant~: Ecoutez ma voix, et je serai votre Dieu, et vous serez mon peuple~; marchez dans toutes les voies que je vous ordonne, afin que vous soyez heureux\FTNT{Ex. 15:26.}.
\VS{24}Mais ils n'ont pas écouté, et n'ont pas prêté l'oreille~; mais ils ont suivi d'autres conseils, les penchants de leur mauvais cœur~; ils se sont éloignés et ne sont pas revenus à moi.
\VS{25}Depuis le jour où vos pères sont sortis du pays d'Egypte, jusqu'à ce jour, je vous ai envoyé tous mes serviteurs les prophètes, je les ai envoyés chaque jour, dès le matin.
\VS{26}Mais ils ne m'ont pas écouté, et ils n'ont pas prêté l'oreille~; mais ils ont raidi leur cou, ils ont fait le mal plus que leurs pères.
\VS{27}Tu leur diras toutes ces paroles, mais ils ne t'écouteront pas~; et tu crieras après eux, mais ils ne te répondront pas.
\VS{28}C'est pourquoi tu leur diras~: C'est ici la nation qui n'écoute pas la voix de Yahweh, son Dieu, et qui ne reçoit pas d'instruction~; la vérité a disparu, elle s'est retirée de leur bouche.
\VS{29}Coupe ta chevelure, ô Jérusalem~! Et jette-la au loin, et prononce à haute voix ta complainte sur les lieux élevés~! Car Yahweh rejette et abandonne la génération qui a provoqué sa fureur.
\VS{30}Car les enfants de Juda ont fait ce qui est mal à mes yeux, dit Yahweh~; ils ont mis leurs abominations dans cette maison sur laquelle mon Nom est invoqué, afin de la souiller.
\VS{31}Et ils ont bâti les hauts lieux de Topheth, qui est dans la vallée de Ben-Hinnom\FTNT{Voir commentaire en Ap. 16:16.}, pour brûler au feu leurs fils et leurs filles\FTNT{Lé. 18:21. Voir commentaire en Lé. 20:2.}, ce que je n'avais pas ordonné, et à quoi je n'ai jamais pensé.
\VS{32}C'est pourquoi voici, les jours viennent, dit Yahweh, qu'elle ne sera plus appelée Topheth, ni la vallée de Ben-Hinnom, mais la vallée de la tuerie~; et on enterrera les morts à Topheth, parce qu'il n'y aura plus d'autre lieu.
\VS{33}Et les cadavres de ce peuple seront la pâture des oiseaux des cieux et des bêtes de la terre, sans qu'il n'y ait personne qui les effraye.
\VS{34}Je ferai aussi cesser dans les villes de Juda et dans les rues de Jérusalem les cris de joie et les cris d'allégresse, la voix de l'époux et la voix de l'épouse~; car le pays sera un désert.
\Chap{8}
\TextTitle{Juda dans l'égarement}
\VerseOne{}En ce temps-là, dit Yahweh, on sortira les os des rois de Juda, et les os de ses chefs, les os des prêtres, et les os des prophètes, et les os des habitants de Jérusalem, hors de leurs sépulcres.
\VS{2}Et on les étendra devant le soleil, et devant la lune, et devant toute l'armée des cieux, qui sont des choses qu'ils ont aimées, qu'ils ont servies et après lesquelles ils ont marché~; des choses qu'ils ont recherchées, et devant lesquelles ils se sont prosternés~; ils ne seront pas recueillis ni ensevelis, ils seront comme du fumier sur la face du sol.
\VS{3}Et la mort sera plus désirable que la vie pour tous ceux qui resteront de cette race mauvaise, ceux, dis-je, qui seront restés dans tous les lieux où je les aurai chassés, dit Yahweh des armées.
\VS{4}Dis-leur donc~: Ainsi parle Yahweh~: Si on tombe, ne se relève-t-on pas~? Et si on se détourne, ne revient-on pas~?
\VS{5}Pourquoi donc ce peuple de Jérusalem s'abandonne-t-il à de perpétuels égarements~? Ils tiennent ferme à la tromperie, et ils refusent de se convertir.
\VS{6}Je suis attentif et j'écoute, mais nul ne parle selon la justice~; il n'y a personne qui se repente de sa méchanceté, disant~: Qu'ai-je fait~? Ils retournent tous vers les objets qui les entraînent, comme le cheval qui se jette avec impétuosité parmi la bataille.
\VS{7}Même la cigogne connaît dans les cieux ses saisons~; la tourterelle et l'hirondelle, et la grue observent le temps où elles doivent venir~; mais mon peuple ne connaît pas les ordonnances de Yahweh.
\VS{8}Comment dites-vous~: Nous sommes les sages, et la loi de Yahweh est avec nous~? Voilà, certes on a agi faussement, et la plume des scribes est une plume de fausseté.
\VS{9}Les sages sont confus, ils sont épouvantés et pris~; car ils ont rejeté la parole de Yahweh, et quelle sagesse ont-ils~?
\VS{10}C'est pourquoi je donnerai leurs femmes à d'autres, et leurs champs à des gens qui les posséderont en héritage. Car depuis le plus petit jusqu'au plus grand, chacun s'adonne au gain déshonnête, tant le prophète que le prêtre, tous agissent faussement.
\VS{11}Ils pansent à la légère la plaie de la fille de mon peuple, en disant~: Paix~! Paix~! Et il n'y a pas de paix.
\VS{12}Seront-ils confus d'avoir commis des abominations~? Ils n'en ont même aucune honte, et ils ne savent pas ce que c'est que de rougir~; c'est pourquoi ils tomberont parmi ceux qui tombent, ils seront renversés au temps où je les visiterai, dit Yahweh.
\VS{13}Je les ramasserai, j'en finirai avec eux, dit Yahweh~; il n'y aura plus de raisins à la vigne, et il n'y aura plus de figues au figuier, les feuilles se flétriront~; et ce que je leur avais donné sera transporté avec eux.
\VS{14}Pourquoi restons-nous assis~? Assemblez-vous et entrons dans les villes fortes, et nous serons là en repos~! Car Yahweh, notre Dieu, nous réduit au silence, et il nous fait boire des eaux empoisonnées, parce que nous avons péché contre Yahweh.
\VS{15}On attendait la paix, et il n'y a rien de bon~; on attend le temps de guérison, et voici la terreur~!
\VS{16}Le hennissement de ses chevaux se fait entendre de Dan, et tout le pays tremble au bruit des hennissements de ses puissants chevaux~; ils viennent et dévorent le pays et ce qu'il contient, la ville et ceux qui l'habitent.
\VS{17}Qui plus est, voici, j'envoie contre vous des serpents, des vipères, contre lesquels il n'y a pas d'enchantement, et ils vous mordront, dit Yahweh.
\VS{18}Je voudrais prendre des forces pour soutenir la douleur, mais mon cœur est languissant au dedans de moi.
\VS{19}Voici la voix de la fille de mon peuple, qui crie d'un pays éloigné~: Yahweh n'est-il plus à Sion~? Son Roi n'est-il plus au milieu d'elle~? Pourquoi m'ont-ils irrité par leurs images taillées, par les vanités\FTNT{Idoles que Dieu appelle vanité, vapeur ou souffle} étrangères~?
\VS{20}La moisson est passée, l'été est fini, et nous ne sommes pas sauvés~!
\VS{21}Je suis amèrement affligé à cause de la calamité de la fille de mon peuple, j'en suis en deuil, j'en suis tout désolé.
\VS{22}N'y a-t-il pas de baume en Galaad~? N'y a-t-il pas là de médecin~? Pourquoi donc la guérison de la fille de mon peuple ne s'opère-t-elle pas~?
\Chap{9}
\TextTitle{Jérémie pleure sur son peuple}
\VerseOne{}Plaise à Dieu que ma tête soit comme un réservoir d'eau, et que mes yeux soient une vive fontaine de larmes, et je pleurerais jour et nuit les blessés à mort de la fille de mon peuple~!
\VS{2}Plaise à Dieu que j'aie au désert une cabane de voyageurs, j'abandonnerais mon peuple, je m'en irais loin de lui~! Car ils sont tous des adultères, et une assemblée de perfides.
\VS{3}Ils tendent leur langue, qui est comme leur arc pour lancer le mensonge\FTNT{Ps. 64:3-4.}~; et ils se renforcent dans la terre contre la fidélité~; car ils vont de méchanceté en méchanceté, et ne me reconnaissent pas, dit Yahweh.
\VS{4}Gardez-vous chacun de son intime ami, et ne vous confiez en aucun frère\FTNT{Mi. 7:5.}~; car tout frère fait métier de supplanter, et tout intime ami marche dans la calomnie.
\VS{5}Et chacun se moque de son intime ami, et on ne parle pas selon la vérité~; ils ont instruit leur langue à dire le mensonge, ils se tourmentent extrêmement pour faire le mal.
\VS{6}Ta demeure est au milieu de la tromperie~; ils refusent, à cause de la tromperie, de me connaître, dit Yahweh.
\VS{7}C'est pourquoi, ainsi parle Yahweh des armées~: Voici, je vais les fondre, je les éprouverai\FTNT{Mal. 3:3.}. Car comment agir autrement à l'égard de la fille de mon peuple~?
\VS{8}Leur langue est une flèche meurtrière, elle profère des tromperies~; chacun de sa bouche parle de la paix avec son ami, mais au-dedans il lui dresse des embûches\FTNT{Ps. 12:3~; Ps. 28:3.}.
\VS{9}Ne les punirais-je pas pour ces choses-là, dit Yahweh~? Mon âme ne se vengerait-elle pas d'une telle nation~?
\VS{10}J'élève ma voix avec larmes, et je prononce à haute voix une lamentation à cause des montagnes, et une complainte à cause des cabanes du désert, parce qu'elles sont brûlées, de sorte que personne n'y passe et qu'on n'y entend plus la voix des troupeaux~; les oiseaux des cieux et le bétail ont fui, ils s'en sont allés.
\VS{11}Et je ferai de Jérusalem des monceaux de ruines, elle sera un repaire de serpents, et je ferai des villes de Juda un désert sans habitants.
\VS{12}Qui est l'homme sage qui comprenne ceci~? Qui est celui à qui la bouche de Yahweh a parlé~? Qu'il le déclare et qu'il dise pourquoi le pays est-il détruit, brûlé comme un désert, sans que personne y passe~?
\VS{13}Yahweh donc dit~: Parce qu'ils ont abandonné ma loi que j'avais mise devant eux~; parce qu'ils n'ont pas écouté ma voix, et qu'ils n'ont pas marché selon elle~;
\VS{14}mais parce qu'ils ont marché suivant les penchants de leur cœur, et après les Baals, comme leurs pères le leur ont enseigné.
\VS{15}C'est pourquoi, ainsi parle Yahweh des armées, le Dieu d'Israël~: Voici, je vais faire manger de l'absinthe à ce peuple-ci, et je leur ferai boire des eaux empoisonnées.
\VS{16}Je les disperserai parmi les nations que n'ont connues ni eux ni leurs pères, et j'enverrai après eux l'épée, jusqu'à ce que je les aie exterminés.
\VS{17}Ainsi parle Yahweh des armées~: Considérez, et appelez des pleureuses, afin qu'elles viennent, et mandez les femmes sages, et qu'elles viennent~!
\VS{18}Qu'elles se hâtent, et qu'elles prononcent à haute voix une lamentation sur nous~; et que nos larmes tombent de nos yeux, et que l'eau coule de nos paupières.
\VS{19}Car une voix de lamentation se fait entendre de Sion, disant~: Eh quoi~! Nous sommes dévastés~! Nous sommes couverts de honte parce que nous avons abandonné le pays et que nos demeures nous ont jetés dehors~!
\VS{20}C'est pourquoi, vous femmes, écoutez la parole de Yahweh, et que votre oreille reçoive la parole de sa bouche~! Enseignez vos filles à se lamenter, et chacune à sa compagne à faire des complaintes~!
\VS{21}Car la mort est montée par nos fenêtres, elle est entrée dans nos palais, pour exterminer les enfants dans les rues, et les jeunes hommes sur les places.
\VS{22}Dis~: Ainsi parle Yahweh~: Même les cadavres des hommes tomberont comme du fumier sur le dessus des champs, et comme une gerbe après le moissonneur, sans que personne ne les ramasse~!
\VS{23}Ainsi parle Yahweh~: Que le sage ne se glorifie pas de sa sagesse, que le fort ne se glorifie pas de sa force, et que le riche ne se glorifie pas de sa richesse.
\VS{24}Mais que celui qui se glorifie, se glorifie d'avoir de l'intelligence et de me connaître, car je suis Yahweh, qui fais miséricorde, droit et justice sur la terre~; car je prends plaisir en ces choses-là, dit Yahweh\FTNT{Ps. 62:10~; 1 Co. 1:31~; 2 Co. 10:17~; 1 Ti. 6:17.}.
\VS{25}Voici, les jours viennent, dit Yahweh, où je punirai tout circoncis ayant le prépuce,
\VS{26}l'Egypte, Juda, Edom, les fils d'Ammon, Moab, et tous ceux qui se coupent les coins de leur barbe, et qui habitent dans le désert~; car toutes les nations sont incirconcises, et toute la maison d'Israël a le cœur incirconcis.
\Chap{10}
\TextTitle{Dénonciation de l'idolâtrie en Israël}
\VerseOne{}Ecoutez la parole que Yahweh vous adresse, maison d'Israël~!
\VS{2}Ainsi parle Yahweh~: N'apprenez pas les façons de faire des nations\FTNT{Lé. 18:3~; De. 12:30.}, et ne craignez pas les signes des cieux, parce que les nations les craignent.
\VS{3}Car les lois des peuples ne sont que vanité. On coupe le bois dans la forêt~; la main de l'ouvrier le travaille avec la hache\FTNT{Es. 40:20~; Es. 44~:12-18.}~;
\VS{4}on l'embellit avec de l'argent et de l'or, on le fait tenir avec des clous et à coups de marteau, afin qu'il ne vacille pas.
\VS{5}Ils sont façonnés tout droits comme des colonnes massives, et ils ne parlent pas~; on les porte par nécessité, parce qu'ils ne peuvent pas marcher. Ne les craignez pas, car ils ne sauraient faire aucun mal, et aussi ils sont incapables de faire du bien.
\VS{6}Nul n'est semblable à toi, ô Yahweh~! Tu es grand, et ton Nom est grand par ta puissance.
\VS{7}Qui ne te craindrait, Roi des nations~? Car cela t'est dû~; car, parmi tous les sages des nations et dans tous leurs royaumes, nul n'est semblable à toi\FTNT{Ap. 15:4.}.
\VS{8}Et ils sont tous ensemble stupides et insensés~; le bois ne leur enseigne que des vanités\FTNT{Ha. 2:18.}.
\VS{9}L'argent qui est étendu en plaques est apporté de Tarsis, et l'or d'Uphaz, pour être mis en œuvre par l'ouvrier et par les mains du fondeur~; et la pourpre et l'écarlate sont leur vêtement~; toutes ces choses sont l'ouvrage de gens habiles.
\VS{10}Mais Yahweh est le Dieu de vérité, c'est le Dieu vivant et le Roi éternel~; la terre tremble devant sa colère, et les nations ne supportent pas sa fureur.
\VS{11}Vous leur parlerez ainsi~: Les dieux qui n'ont pas fait les cieux et la terre périront de la terre et de dessous les cieux.
\VS{12}Mais Yahweh est celui qui a fait la terre par sa puissance, qui a fondé le monde habitable par sa sagesse, et qui a étendu les cieux par son intelligence.
\VS{13}Sitôt qu'il fait retentir sa voix, il y a un tumulte d'eaux dans les cieux~; il fait monter les vapeurs des extrémités de la terre, il fait les éclairs et la pluie, et il fait sortir le vent de ses réservoirs.
\VS{14}Tout homme devient stupide par sa connaissance, tout fondeur est honteux par les images taillées~; car les idoles en métal fondu ne sont que mensonge, il n'y a pas de souffle en elles~;
\VS{15}elles ne sont que vanité, une œuvre de tromperie~; elles périront au temps de leur châtiment.
\VS{16}La portion de Jacob n'est pas comme ces choses-là~; car c'est lui qui a tout formé, et Israël est la tribu de son héritage. Son Nom est Yahweh des armées.
\VS{17}Toi qui es assise dans la détresse, rassemble du pays tes paquets~!
\VS{18}Car ainsi parle Yahweh~: Voici, cette fois je vais lancer au loin, comme avec une fronde, les habitants du pays~; je vais les mettre à l'étroit, afin qu'on les atteigne.
\VS{19}Malheur à moi, diront-ils, à cause de ma blessure~! Ma plaie est douloureuse~! Mais moi je dis~: Quoi qu'il en soit, c'est une maladie qu'il faudra que je supporte.
\VS{20}Ma tente est dévastée, tous mes cordages sont rompus~; mes fils m'ont quittée, et ils ne sont plus~; il n'y a plus personne qui dresse ma tente, qui relève mes pavillons.
\VS{21}Car les pasteurs ont été stupides, ils n'ont pas cherché Yahweh~; c'est pour cela qu'ils n'ont pas réussi, et que tous leurs troupeaux s'éparpillent.
\VS{22}Voici, une rumeur se fait entendre~; avec une grande secousse qui vient du pays du nord, pour faire des villes de Juda un désert, un repère de serpents.
\VS{23}Yahweh~! Je sais que la voie de l'homme ne dépend pas de lui\FTNT{Pr. 16:1.}, et qu'il n'est pas au pouvoir de l'homme qui marche de diriger ses pas.
\VS{24}Ô Yahweh~! Châtie-moi, mais avec équité, et non dans ta colère, de peur que tu ne me réduises à rien\FTNT{Es. 27:8~; Ps. 38:2.}.
\VS{25}Répands ta fureur sur les nations qui ne te connaissent pas, et sur les familles qui n'invoquent pas ton Nom~! Car ils dévorent Jacob, ils le dévorent et le consument, et ils mettent en désolation son agréable demeure.
\Chap{11}
\TextTitle{Yahweh dénonce la prostitution de Juda}
\VerseOne{}La parole fut adressée à Jérémie de la part de Yahweh, en disant~:
\VS{2}Ecoutez les paroles de cette alliance, et parlez aux hommes de Juda et aux habitants de Jérusalem~!
\VS{3}Dis-leur~: Ainsi parle Yahweh, le Dieu d'Israël~: Maudit soit l'homme qui n'écoute pas les paroles de cette alliance\FTNT{De. 27:26~; Ga. 3:10.},
\VS{4}que j'ai ordonnée à vos pères, le jour où je les ai fait sortir du pays d'Egypte, de la fournaise de fer, en disant~: Ecoutez ma voix et faites toutes les choses que je vous ordonnerai~; alors vous serez mon peuple, et je serai votre Dieu\FTNT{Lé. 26:12~; De. 4:20.},
\VS{5}afin que j'accomplisse le serment que j'ai juré à vos pères, de leur donner un pays où coulent le lait et le miel, comme vous le voyez aujourd'hui. Et je répondis et dis~: Amen~! Ô Yahweh~!
\VS{6}Puis Yahweh me dit~: Crie toutes ces paroles dans les villes de Juda et dans les rues de Jérusalem, en disant~: Ecoutez les paroles de cette alliance et observez-les~!
\VS{7}Car j'ai averti vos pères, depuis le jour où je les ai fait monter du pays d'Egypte jusqu'à ce jour, je les ai avertis dès le matin, en disant~: Ecoutez ma voix~!
\VS{8}Mais ils n'ont pas écouté, ils n'ont pas prêté l'oreille, ils ont marché chacun suivant les penchants de leur mauvais cœur~; c'est pourquoi j'ai fait venir sur eux toutes les paroles de cette alliance, que je leur avais donné l'ordre d'observer, et qu'ils n'ont pas observée.
\VS{9}Yahweh me dit~: Il y a une conspiration entre les hommes de Juda et entre les habitants de Jérusalem.
\VS{10}Ils sont retournés aux iniquités de leurs premiers pères, qui ont refusé d'écouter mes paroles, et ils sont allés après d'autres dieux pour les servir. La maison d'Israël et la maison de Juda ont rompu mon alliance, que j'avais faite avec leurs pères.
\VS{11}C'est pourquoi ainsi parle Yahweh~: Voici, je fais venir sur eux un mal dont ils ne pourront sortir. Ils crieront vers moi, et je ne les écouterai pas\FTNT{Es. 1:15~; Ez. 8:18~; Mi. 3:4~; Pr. 1:28.}.
\VS{12}Et les villes de Juda et les habitants de Jérusalem s'en iront et crieront vers les dieux auxquels ils brûlent de l'encens, mais ces dieux-là ne les sauveront pas au temps de leur malheur.
\VS{13}Car, ô Juda~! Tu as eu autant de dieux que de villes~; et toi Jérusalem, tu as dressé autant d'autels aux choses honteuses que tu as de rues, des autels, dis-je, pour brûler de l'encens à Baal\FTNT{Ez. 16:24-31~; Ac. 17:23.}…
\VS{14}Toi donc, n'intercède pas pour ce peuple, et n'élève pour eux ni cri ni prière~; car je ne les écouterai pas au temps où ils crieront vers moi dans leur malheur.
\VS{15}Qu'est-ce que mon bien-aimé a à faire dans ma maison, que tant de gens se servent d'elle pour y faire leurs complots~? La chair sainte sera transportée loin de toi, et encore quand tu fais le mal, c'est alors que tu triomphes~!
\VS{16}Yahweh avait appelé ton nom olivier verdoyant et beau par la forme de ton fruit~; mais au bruit d'un grand fracas, il y met le feu, et ses rameaux sont brisés.
\VS{17}Yahweh des armées, qui t'a plantée, prononce le mal contre toi, à cause de la méchanceté de la maison d'Israël et de la maison de Juda, qui ont agi pour m'irriter, en brûlant de l'encens à Baal.
\TextTitle{Jugement des ennemis de Jérémie}
\VS{18}Et Yahweh me l'a fait savoir, et je l'ai su~; alors tu m'as fait voir leurs actions.
\VS{19}Mais moi, comme un agneau, ou comme un bœuf qu'on mène pour être égorgé, je ne savais pas qu'ils projetaient de mauvais desseins contre moi, en disant~: Détruisons l'arbre avec son fruit~! Exterminons-le de la terre des vivants, et qu'on ne se souvienne plus de son nom~!
\VS{20}Mais toi, Yahweh des armées, qui juges justement, et qui éprouves les reins et le cœur~! Fais que je voie ta vengeance s'exercer contre eux, car je t'ai découvert ma cause\FTNT{1 S. 16:7~; Ps. 26:2~; 1 Ch. 28:9~; Ap. 2:23.}.
\VS{21}C'est pourquoi ainsi parle Yahweh contre les gens d'Anathoth, qui cherchent ta vie et qui disent~: Ne prophétise plus au Nom de Yahweh, et tu ne mourras pas par nos mains\FTNT{Es. 30:10~; Mi. 2:6.}~!
\VS{22}C'est pourquoi donc ainsi parle Yahweh des armées~: Voici, je vais les punir~; les jeunes hommes mourront par l'épée, leurs fils et leurs filles mourront par la famine.
\VS{23}Et il ne restera rien d'eux~; car je ferai venir le mal sur les gens d'Anathoth, l'année de leur châtiment.
\Chap{12}
\TextTitle{Prière de Jérémie et réponse de Yahweh}
\VerseOne{}Yahweh, quand je contesterai avec toi, tu seras trouvé juste~; mais toutefois j'entrerai en contestation avec toi~: Pourquoi la voie des méchants est-elle prospère~? Pourquoi tous les perfides vivent-ils en paix\FTNT{Job 21:7-9~; Ro. 3:4.}~?
\VS{2}Tu les as plantés, et ils ont pris racine, ils s'avancent, et ils portent du fruit. Tu es près de leurs bouches, mais tu es loin de leurs cœurs\FTNT{Es. 29:13~; Job 21:7-8.}.
\VS{3}Mais, ô Yahweh, tu me connais, tu me vois, tu éprouves mon cœur qui est avec toi. Traîne-les comme des brebis qu'on mène pour être égorgées, et mets-les à part pour le jour de la tuerie~!
\VS{4}Jusqu'à quand le pays mènera-t-il deuil, et l'herbe de tous les champs séchera-t-elle à cause de la méchanceté des habitants qui sont en la terre~? Les bêtes et les oiseaux sont consumés par la disette, parce que ces méchants disent~: On ne verra pas notre dernière fin. 
\VS{5}Si tu cours avec des piétons et qu'ils te fatiguent, comment lutteras-tu avec les chevaux~? Et si tu te crois en sûreté dans une terre de paix, que feras-tu devant l'orgueil du Jourdain~?
\VS{6} Certainement, mêmes tes frères et la maison de ton père, ceux-là mêmes agissent perfidement contre toi, eux-mêmes ont crié après toi à plein gosier~; ne les crois point, quoiqu'ils te parlent amicalement\FTNT{Pr. 26:25.}.
\VS{7}J'ai abandonné ma maison, j'ai quitté mon héritage, ce que mon âme aimait le plus, je l'ai livré aux mains de ses ennemis.
\VS{8}Mon héritage a été pour moi comme un lion dans la forêt, il a poussé contre moi ses rugissements~; c'est pourquoi je l'ai pris en haine.
\VS{9}Mon héritage a-t-il donc été pour moi comme un oiseau de proie tacheté~? Les oiseaux de proie ne seront-ils pas autour de lui~? Venez, assemblez-vous, vous tous les animaux des champs, venez pour le dévorer\FTNT{Es. 56:9.}~!
\VS{10}Plusieurs pasteurs ravagent ma vigne, ils foulent mon champ~; ils réduisent le champ de mes délices en un désert, en une désolation.
\VS{11}Ils le réduisent en un désert~; il est en deuil, il est désolé devant moi. Tout le pays est ravagé, car nul n'y prend garde.
\VS{12}Les destructeurs viennent sur tous les lieux élevés du désert, car l'épée de Yahweh dévore le pays d'un bout à l'autre~; il n'y a de paix pour aucune chair.
\VS{13}Ils ont semé du froment, et ils moissonnent des épines, ils se sont fatigués sans profit. Soyez honteux de vos récoltes, à cause de l'ardeur de la colère de Yahweh\FTNT{Lé. 26:16.}.
\VS{14}Ainsi parle Yahweh contre tous mes mauvais voisins, qui mettent la main sur l'héritage que j'ai donné à mon peuple d'Israël~: Voici, je les arracherai de leur pays, et j'arracherai la maison de Juda du milieu d'eux.
\VS{15}Mais il arrivera qu'après les avoir arrachés, j'aurai encore compassion d'eux, et je les ramènerai chacun dans son héritage, chacun dans son pays\FTNT{De. 30:3.}.
\VS{16}Et il arrivera que s'ils apprennent bien les voies de mon peuple, pour jurer par mon Nom, en disant~: Yahweh est vivant~! Comme ils ont enseigné à mon peuple à jurer par Baal, ils seront édifiés au milieu de mon peuple.
\VS{17}Mais s'ils n'écoutent pas, j'arracherai entièrement une telle nation, et je la ferai périr, dit Yahweh\FTNT{Es. 60:12.}.
\Chap{13}
\TextTitle{La ceinture pourrie, illustration du jugement}
\VerseOne{}Ainsi m'a parlé Yahweh~: Va, et achète-toi une ceinture de lin et mets-la sur tes reins~; et ne la mets pas dans l'eau.
\VS{2}J'achetai donc une ceinture, selon la parole de Yahweh, et je la mis sur mes reins.
\VS{3}Et la parole de Yahweh me fut adressée pour la seconde fois, en disant~:
\VS{4}Prends la ceinture que tu as achetée et qui est sur tes reins~; lève-toi, va-t'en vers l'Euphrate, et là, cache-la dans la fente d'un rocher.
\VS{5}J'allai donc et je la cachai près de l'Euphrate, comme Yahweh me l'avait ordonné.
\VS{6}Et il arriva que plusieurs jours après, Yahweh me dit~: Lève-toi, va vers l'Euphrate et reprends la ceinture que je t'avais ordonné d'y cacher.
\VS{7}Et j'allai vers l'Euphrate, je creusai, et je pris la ceinture dans le lieu où je l'avais cachée~; mais voici, la ceinture était pourrie, elle n'était plus bonne à rien.
\VS{8}Alors la parole de Yahweh me fut adressée, en disant~:
\VS{9}Ainsi parle Yahweh~: Je ferai ainsi pourrir l'orgueil de Juda et le grand orgueil de Jérusalem.
\VS{10}L'orgueil de ce peuple très méchant, qui refuse d'écouter mes paroles, qui marche selon les penchants de son cœur, et qui va après d'autres dieux, pour les servir et pour se prosterner devant eux, qu'il devienne comme cette ceinture qui n'est plus bonne à rien~!
\VS{11}Car comme une ceinture est attachée aux reins d'un homme, ainsi je m'étais attaché toute la maison d'Israël et toute la maison de Juda, dit Yahweh, afin qu'elles soient mon peuple, mon Nom, ma louange, et ma gloire. Mais ils ne m'ont pas écouté.
\VS{12}Tu leur diras donc cette parole-ci~: Ainsi parle Yahweh, le Dieu d'Israël~: Toute outre sera remplie de vin. Et ils te diront~: Ne savons-nous pas que toute outre sera remplie de vin~?
\VS{13}Mais tu leur diras~: Ainsi parle Yahweh~: Voici, je vais remplir d'ivresse tous les habitants de ce pays, les rois qui sont assis sur le trône de David, les prêtres, les prophètes, et tous les habitants de Jérusalem.
\VS{14}Et je les briserai les uns contre les autres, les pères et les fils ensemble, dit Yahweh\FTNT{Es. 51:17-20~; Ps. 60:5.}~; je n'aurai pas de compassion, je n'épargnerai pas, et je n'aurai pas de miséricorde~; rien ne m'empêchera de les détruire.
\VS{15}Ecoutez et prêtez l'oreille~! Ne vous élevez pas~! Car Yahweh parle.
\VS{16}Donnez gloire à Yahweh, votre Dieu, avant qu'il fasse venir les ténèbres, avant que vos pieds se heurtent contre les montagnes du crépuscule~; vous attendrez la lumière, et il la changera en ombre de la mort, il la réduira en obscurité profonde\FTNT{Es. 59:9~; Jn. 12:35.}.
\VS{17}Si vous n'écoutez pas ceci, mon âme pleurera en secret, à cause de votre orgueil~; mes yeux verseront des larmes en abondance, ils se fondront en larmes, parce que le troupeau de Yahweh sera emmené captif\FTNT{La. 1:2-16.}.
\VS{18}Dis au roi et à la reine~: Humiliez-vous et asseyez-vous sur la cendre~! Car elle est tombée de vos têtes, la couronne de votre gloire.
\VS{19}Les villes du midi sont fermées, il n'y a personne qui les ouvre~; tout Juda est transporté en captivité, il est transporté entièrement.
\VS{20}Levez vos yeux et voyez ceux qui viennent du nord. Où est le troupeau qui t'avait été donné, le troupeau qui faisait ta gloire~?
\VS{21}Que diras-tu quand il te punira~? Car tu les as enseignés à dominer en maîtres sur toi. Les douleurs ne te saisiront-elles pas, comme elles saisissent une femme qui enfante~?
\VS{22}Que si tu dis en ton cœur~: Pourquoi cela m'arrive-t-il~? C'est à cause de la multitude de tes iniquités que les pans de ta robe sont relevés, et que tes talons sont violemment mis à nu\FTNT{Es. 47:2-3.}.
\VS{23}L'Ethiopien peut-il changer sa peau et le léopard ses taches~? Pourriez-vous aussi faire quelque bien, vous qui êtes accoutumés à faire le mal~?
\VS{24}C'est pourquoi je les disperserai, comme du chaume, qui est emporté çà et là par le vent du désert.
\VS{25}Voilà ton sort, la portion que je te mesure, dit Yahweh, parce que tu m'as oublié, et que tu as mis ta confiance dans le mensonge.
\VS{26}A cause de cela, je relèverai les pans de ta robe sur ton visage, et ta honte se verra.
\VS{27}Tes adultères et tes hennissements, l'énormité de tes prostitutions sur les collines et dans les champs, tes abominations, je les ai vues. Malheur à toi, Jérusalem~! Ne seras-tu pas purifiée~? Jusqu'à quand cela durera-t-il~?
\Chap{14}
\TextTitle{Le pays frappé par la sécheresse}
\VerseOne{}La parole de Yahweh, qui fut adressée à Jérémie, à l'occasion de la sécheresse.
\VS{2}Juda est dans le deuil, et ses portes sont dans un état pitoyable. Ils sont tous en deuil, gisant par terre~; et les cris de Jérusalem montent au ciel.
\VS{3}Et les personnes distinguées envoient les petits chercher de l'eau, et les petits vont aux citernes, ne trouvent pas d'eau, et reviennent leurs vases vides~; ils sont honteux et confus, ils couvrent leur tête.
\VS{4}Parce que la terre est crevassée, parce qu'il n'y a pas eu de pluie dans le pays, les laboureurs sont honteux, ils se couvrent la tête.
\VS{5}Même la biche met bas son faon dans le champ et l'abandonne, parce qu'il n'y a pas d'herbe.
\VS{6}Et les ânes sauvages se tiennent sur les lieux élevés, humant l'air comme des serpents~; leurs yeux se consument, parce qu'il n'y a pas d'herbe.
\VS{7}Si nos iniquités témoignent contre nous, agis à cause de ton Nom, ô Yahweh\FTNT{Es. 59:12.}~! Car nos infidélités sont nombreuses, c'est contre toi que nous avons péché.
\VS{8}Toi qui es l'espérance d'Israël, son Sauveur au temps de la détresse, pourquoi serais-tu dans le pays comme un étranger, comme un voyageur qui se détourne pour passer la nuit~?
\VS{9}Pourquoi serais-tu comme un homme stupéfait, et comme un héros qui ne peut sauver~? Or tu es au milieu de nous, ô Yahweh, et ton Nom est invoqué sur nous~: Ne nous abandonne pas~!
\VS{10}Voici ce que Yahweh dit de ce peuple~: Parce qu'ils aiment à errer ainsi çà et là, et qu'ils ne savent retenir leurs pieds, Yahweh ne prend pas plaisir en eux, il se souvient maintenant de leurs iniquités, et il punit leurs péchés\FTNT{Os. 8:13.}.
\VS{11}Puis Yahweh me dit~: N'intercède pas en faveur de ce peuple.
\VS{12}Quand ils jeûnent, je n'écouterai pas leurs cris~; et quand ils offrent des holocaustes et des offrandes, je n'y prendrai pas plaisir~; mais je les consumerai par l'épée, par la famine et par la peste.
\VS{13}Et je répondis~: Ah~! Ah~! Seigneur Yahweh~! Voici, les prophètes leur disent~: Vous ne verrez pas l'épée, et vous n'aurez pas de famine~; mais je vous donnerai dans ce lieu-ci une paix assurée.
\VS{14}Et Yahweh me dit~: C'est le mensonge que ces prophètes prophétisent en mon Nom~; je ne les ai pas envoyés, je ne leur ai pas donné d'ordre, je ne leur ai pas parlé~; ils vous prophétisent des visions de mensonge, des divinations de néant et des tromperies de leur cœur\FTNT{De. 18:20-22~; Ez. 13:2-3.}.
\VS{15}C'est pourquoi ainsi parle Yahweh sur les prophètes qui prophétisent en mon Nom, sans que je les ai envoyés, et qui disent~: Il n'y aura ni épée ni la famine dans ce pays~: Ces prophètes-là seront consumés par l'épée et par la famine.
\VS{16}Et le peuple à qui ils prophétisent sera jeté dans les rues de Jérusalem à cause de la famine et de l'épée~; et il n'y aura personne pour les enterrer, ni eux, ni leurs femmes, ni leurs fils, ni leurs filles~; je répandrai sur eux leur méchanceté.
\VS{17}Tu leur diras donc cette parole-ci~: Que mes yeux se fondent en larmes nuit et jour, et qu'ils ne cessent pas\FTNT{La. 1:16.}~; car la vierge, fille de mon peuple, a été frappée d'un grand coup, d'une plaie très douloureuse.
\VS{18}Si je sors dans les champs, voici les gens tués par l'épée~; si j'entre dans la ville, voici les gens consumés par la faim~; même le prophète et le prêtre parcourent le pays, sans savoir où ils vont.
\VS{19}As-tu entièrement rejeté Juda, et ton âme a-t-elle Sion en horreur~? Pourquoi nous frappes-tu sans qu'il y ait pour nous de guérison~? On attendait la paix, mais il n'y a rien de bon, un temps de guérison, et voici la terreur~!
\VS{20}Yahweh, nous reconnaissons notre méchanceté, l'iniquité de nos pères~; car nous avons péché contre toi\FTNT{Ps. 106:6~; Da. 9:8.}.
\VS{21}Ne nous rejette pas, à cause de ton Nom, et ne déshonore pas le trône de ta gloire~! Souviens-toi de ton alliance avec nous, et ne la romps pas~!
\VS{22}Parmi les vanités\FTNT{Ce terme veut aussi dire «~idole~».} des nations, y en a-t-il qui fassent pleuvoir, et les cieux donnent-ils la menue pluie\FTNT{Es. 30:23~; Ac. 14:17.}~? N'est-ce pas toi, ô Yahweh, notre Dieu~? C'est pourquoi nous nous attendons à toi, car c'est toi qui as fait toutes ces choses.
\Chap{15}
\TextTitle{Yahweh fermement décidé à juger son peuple}
\VerseOne{}Et Yahweh me dit~: Quand Moïse et Samuel se tiendraient devant moi, je n'aurais pourtant point d'affection pour ce peuple~; chasse-les de devant ma face et qu'ils sortent.
\VS{2}Que s'ils te disent~: Où irons-nous~? Tu leur répondras~: Ainsi parle Yahweh~: Ceux qui sont destinés à la mort iront à la mort~; et ceux qui sont destinés à l'épée iront à l'épée~; et ceux qui sont destinés à la famine, iront à la famine~; et ceux qui sont destinés à la captivité iront en captivité\FTNT{Za. 11:9.}~!
\VS{3}J'établirai aussi sur eux quatre espèces de punitions, dit Yahweh, l'épée pour tuer, les chiens pour traîner, les oiseaux des cieux et les bêtes de la terre pour dévorer et pour détruire. 
\VS{4}Et je les livrerai pour être agités par tous les royaumes de la terre, à cause de Manassé, fils d'Ezéchias, roi de Juda, pour les choses qu'il a faites dans Jérusalem. 
\VS{5}Car qui aura compassion de toi, Jérusalem, ou qui te plaindra~? Ou qui se détournera pour s'informer de ta paix~? 
\VS{6}Tu m'as abandonné, dit Yahweh, et tu t'en es allée en arrière~; c'est pourquoi j'étends ma main sur toi, et je te détruis, je suis las d'avoir compassion.
\VS{7}Je les vanne avec un van aux portes du pays\FTNT{Mt. 3:12.}~; je prive d'enfants, je fais périr mon peuple, et ils ne se sont pas détournés de leurs voies.
\VS{8}Je multiplie ses veuves plus que le sable de la mer~; je fais venir sur eux, sur la mère du jeune homme, le dévastateur en plein midi~; je fais tomber subitement sur elle l'angoisse et les frayeurs.
\VS{9}Celle qui en avait enfanté sept languit, elle rend l'âme~; son soleil se couche pendant qu'il est encore jour\FTNT{Am. 8:9.}~; elle est confuse, couverte de honte. Ceux qui restent, je les livre à l'épée devant leurs ennemis, dit Yahweh.
\VS{10}Malheur à moi, ô ma mère, de ce que tu m'as enfanté\FTNT{Job 3:1-2.} pour être un homme de contestation et un homme de dispute pour tout le pays~! Je n'emprunte ni ne prête, et néanmoins tous me maudissent et me méprisent.
\VS{11}En vérité tout ira bien pour ton reste~; en vérité je ferai que l'ennemi te traite bien au temps du malheur et au temps de la détresse.
\VS{12}Le fer brisera-t-il le fer du nord et l'airain~?
\VS{13}Je livre au pillage, sans en faire le prix, tes richesses et tes trésors, et cela à cause de tous tes péchés, sur tout ton territoire.
\VS{14}Je te fais passer avec tes ennemis dans un pays que tu ne connais pas, car le feu de ma colère s'est allumé, il brûle sur vous\FTNT{De. 32:22.}.
\TextTitle{La mise à part de Jérémie}
\VS{15}Yahweh~! Tu sais tout, souviens-toi de moi, visite-moi, venge-moi de ceux qui me persécutent\FTNT{Ps. 106:4.}~! Ne m'enlève pas, tandis que tu te montres lent à la colère~! Sache que je supporte l'opprobre à cause de toi.
\VS{16}J'ai trouvé tes paroles, je les ai aussitôt dévorées\FTNT{Ez. 3:3~; Ap. 10:9.}~; tes paroles ont fait la joie et l'allégresse de mon cœur~; car ton Nom est invoqué sur moi, ô Yahweh, Dieu des armées~!
\VS{17}Je ne me suis pas assis dans l'assemblée des moqueurs, et je ne m'y suis pas réjoui~; mais je me suis assis tout seul à cause de ta main, car tu me remplissais d'indignation.
\VS{18}Pourquoi ma douleur est-elle continuelle~? Pourquoi ma plaie est-elle incurable et refuse-t-elle d'être guérie~? Serais-tu pour moi comme une source trompeuse, comme des eaux qui ne durent pas~?
\VS{19}C'est pourquoi ainsi parle Yahweh~: Si tu reviens, je te ramènerai et tu te tiendras devant moi~; et si tu sépares la chose précieuse de la méprisable, tu seras comme ma bouche. Qu'ils reviennent vers toi, mais toi, ne retourne pas vers eux.
\VS{20}Je ferai que tu sois pour ce peuple une muraille d'airain bien forte~; ils combattront contre toi, mais il n'auront pas le dessus sur toi~; car je suis avec toi pour te sauver et te délivrer, dit Yahweh.
\VS{21}Et je te délivrerai de la main des malins, et te rachèterai de la main des méchants.
\Chap{16}
\TextTitle{Célibat de Jérémie, illustration du jugement sur Juda}
\VerseOne{}Puis la parole de Yahweh me fut adressée, en disant~:
\VS{2}Tu ne prendras pas de femme et tu n'auras pas de fils ni de filles dans ce lieu-ci.
\VS{3}Car ainsi parle Yahweh sur les fils et les filles qui naîtront en ce lieu-ci, sur leurs mères qui les auront enfantés, et sur leurs pères qui les auront engendrés dans ce pays~:
\VS{4}Ils mourront de maladie mortelle~; ils ne seront ni pleurés ni enterrés~; ils seront comme du fumier sur la face du sol~; ils seront consumés par l'épée et par la famine~; et leurs cadavres seront la pâture des oiseaux des cieux et des bêtes de la terre.
\VS{5}Car ainsi parle Yahweh~: N'entre pas dans une maison de deuil, ne vas pas te lamenter ni te plaindre avec eux~; car j'ai retiré de ce peuple dit Yahweh, ma paix, ma miséricorde et mes compassions.
\VS{6}Et les grands et petits mourront dans ce pays~; ils ne seront pas enterrés~; on ne les pleurera pas, on ne se fera pas d'incision, et on ne se rasera pas pour eux\FTNT{Lé. 19:28~; De. 14:1~; Ez. 7:11}.
\VS{7}On ne rompra pas le pain dans le deuil pour consoler quelqu'un au sujet d'un mort, et on ne leur donnera pas à boire de la coupe de consolation pour leur père ou pour leur mère.
\VS{8}Aussi n'entre pas non plus dans une maison de festin pour t'asseoir avec eux, pour manger et pour boire.
\VS{9}Car ainsi parle Yahweh des armées, le Dieu d'Israël~: Voici, je vais faire cesser dans ce lieu-ci, devant vos yeux et en vos jours, les cris de joie et les cris d'allégresse, la voix de l'époux et la voix de l'épouse.
\VS{10}Et il arrivera que quand tu annonceras à ce peuple toutes ces paroles-là, ils te diront~: Pourquoi Yahweh parle-t-il de tout ce grand mal contre nous~? Quelle est notre iniquité~? Quel est le péché que nous avons commis contre Yahweh, notre Dieu~?
\VS{11}Et tu leur diras~: Parce que vos pères m'ont abandonné, dit Yahweh, et sont allés après d'autres dieux, les ont servis, se sont prosternés devant eux, m'ont abandonné et n'ont pas gardé ma loi~; 
\VS{12}et que vous avez fait le mal plus encore que vos pères. Car voici, chacun de vous marche selon les penchants de son mauvais cœur pour ne pas m'écouter.
\VS{13}A cause de cela, je vous transporterai de ce pays à un pays que vous n'avez pas connu, ni vous ni vos pères~; et là, vous servirez jour et nuit les autres dieux, car je ne vous aurai pas fait grâce\FTNT{De. 28:64-65.}.
\VS{14}Néanmoins voici, les jours viennent, dit Yahweh, qu'on ne dira plus~: Yahweh est vivant, lui qui a fait monter les enfants d'Israël du pays d'Egypte~!
\VS{15}Mais on dira~: Yahweh est vivant, lui qui a fait monter les enfants d'Israël du pays du nord et de tous les pays où il les avait chassés~; après que je les ramenerai dans leur pays, que j'avais donné à leurs pères.
\VS{16}Voici, j'envoie plusieurs pêcheurs, dit Yahweh, et ils les pêcheront~; et ensuite, j'enverrai plusieurs chasseurs, et ils les chasseront de toutes les montagnes et de toutes les collines, et des fentes des rochers.
\VS{17}Car mes yeux sont sur toutes leurs voies, elles ne sont pas cachées devant ma face, et leur iniquité n'est pas couverte devant mes yeux\FTNT{Pr. 5:21~; Job 34:21.}.
\VS{18}Mais premièrement je leur rendrai le double de leur iniquité et de leur péché, parce qu'ils ont souillé mon pays par les cadavres de leurs idoles, et parce qu'ils ont rempli mon héritage de leurs abominations.
\VS{19}Yahweh, qui est ma force, ma forteresse, et mon refuge au jour de la détresse~! Les nations viendront à toi des extrémités de la terre, et diront~: Certes, nos pères ont hérité le mensonge et la vanité, et les choses auxquelles il n'y a pas de profit.
\VS{20}L'homme se fera-t-il bien des dieux, qui toutefois ne sont pas des dieux~?
\VS{21}C'est pourquoi voici, je leur fais connaître, cette fois, je leur fais connaître ma main et ma force~; et ils sauront que mon Nom est Yahweh.
\Chap{17}
\TextTitle{Le caractère sinueux du cœur}
\VerseOne{}Le péché de Juda est écrit avec un burin de fer et avec une pointe de diamant~; il est gravé sur la table de leur cœur et sur les cornes de leurs autels~;
\VS{2}de sorte que leurs fils se souviennent de leurs autels, et de leurs Asherah, auprès des arbres verts sur les hautes collines.
\VS{3}Ma montagne dans les champs, je livre tes richesses et tous tes trésors au pillage~; tes hauts lieux sont pleins de péché sur tout ton territoire.
\VS{4}Et toi, et ceux qui sont avec toi, vous laisserez vacant l'héritage que je t'avais donné~; et je t'asservirai à tes ennemis dans un pays que tu ne connais pas~; car vous avez allumé le feu de ma colère, et il brûlera à toujours.
\VS{5}Ainsi parle Yahweh~: Maudit soit l'homme qui se confie dans l'homme, et qui fait de la chair sa force, et dont le cœur se retire de Yahweh~!
\VS{6}Car il sera comme la bruyère dans le désert, et il ne voit pas venir le bien~; mais il demeure dans des lieux brûlés du désert, dans une terre salée et inhabitable.
\VS{7}Béni soit l'homme qui se confie en Yahweh, et dont Yahweh est l'espérance~!
\VS{8}Il est comme un arbre planté près des eaux\FTNT{Ps. 23.}, et qui étend ses racines le long d'une eau courante~; quand la chaleur vient, il ne s'en aperçoit pas, et sa feuille reste verte~; il n'est pas en peine dans l'année de la sécheresse, et ne cesse de porter du fruit.
\VS{9}Le cœur est rusé et désespérément malin par-dessus tout~: Qui peut le connaître\FTNT{Ps. 64:7.}~?
\VS{10}Je suis Yahweh, qui sonde le cœur, et qui éprouve les reins~; même pour rendre à chacun selon sa voie, et selon le fruit de ses actions.
\VS{11}Celui qui acquiert des richesses, sans observer la justice, est une perdrix qui couve ce qu'elle n'a pas pondu~; il les laissera au milieu de ses jours, et à la fin il sera trouvé insensé\FTNT{Ec. 4:8}.
\VS{12}Le lieu de notre sanctuaire est un trône de gloire, un lieu haut élevé dès le commencement.
\VS{13}Yahweh, qui es l'espérance d'Israël~! Tous ceux qui t'abandonnent seront honteux~: Ceux qui se détournent de moi seront écrits sur la terre, car ils abandonnent la source des eaux vives, dit Yahweh\FTNT{Es. 1:28~; Ps. 73:28.}.
\VS{14}Yahweh, guéris-moi et je serai guéri~; sauve-moi et je serai sauvé~; car tu es ma louange.
\VS{15}Voici, ceux-ci me disent~: Où est la parole de Yahweh~? Qu'elle vienne présentement\FTNT{Es. 5:19~; Ez. 12:23~; 2 Pi. 3:3-4.}~!
\VS{16}Mais je ne me suis pas avancé plus qu'un pasteur après toi, je n'ai pas non plus désiré le jour du malheur, tu le sais~; et ce qui est sorti de mes lèvres est présent devant toi.
\VS{17}Ne sois pas pour moi un sujet d'effroi, toi, mon refuge au jour du malheur~!
\VS{18}Que ceux qui me persécutent soient honteux, mais que je ne sois pas honteux~; qu'ils soient brisés, mais que je ne sois pas brisé~! Fais venir sur eux le jour du malheur, frappe-les d'une double plaie~!
\TextTitle{Message à propos du sabbat}
\VS{19}Ainsi m'a parlé Yahweh~: Va, et tiens-toi debout à la porte des fils du peuple, par laquelle les rois de Juda entrent et par laquelle ils sortent, et à toutes les portes de Jérusalem.
\VS{20}Tu leur diras~: Ecoutez la parole de Yahweh, rois de Juda, et vous tous hommes de Juda, et vous tous habitants de Jérusalem qui entrez par ces portes~!
\VS{21}Ainsi parle Yahweh~: Prenez garde à vos âmes~; ne portez aucun fardeau le jour du sabbat, et ne les faites pas passer par les portes de Jérusalem\FTNT{Né. 13:19.}.
\VS{22}Ne faites sortir de vos maisons aucun fardeau le jour du sabbat, et ne faites aucune œuvre~; mais sanctifiez le jour du sabbat, comme je l'ai ordonné à vos pères\FTNT{Ex. 20:8~; Ex. 23:12.}.
\VS{23}Mais ils n'ont pas écouté, ils n'ont pas prêté l'oreille~; ils ont raidi leur cou, pour ne pas écouter et ne pas recevoir d'instruction.
\VS{24}Il arrivera donc, si vous m'écoutez attentivement, dit Yahweh, pour ne faire passer aucun fardeau par les portes de cette ville le jour du sabbat, et si vous sanctifiez le jour du sabbat, en ne faisant aucune œuvre ce jour-là,
\VS{25}que les rois et les chefs, ceux qui sont assis sur le trône de David, montés sur des chars et sur des chevaux, eux et les chefs d'entre eux, les hommes de Juda et les habitants de Jérusalem, entreront par les portes de cette ville, et cette ville sera habitée à toujours.
\VS{26}On viendra aussi des villes de Juda et des environs de Jérusalem, et du pays de Benjamin, et du bas pays, des montagnes et du midi, pour apporter des holocaustes, des sacrifices, des offrandes et de l'encens~; pour apporter aussi des sacrifices de louanges dans la maison de Yahweh.
\VS{27}Mais si vous ne m'écoutez pas pour sanctifier le jour du sabbat, pour ne porter aucun fardeau, et n'en faire entrer aucun par les portes de Jérusalem le jour du sabbat, je mettrai le feu à ses portes, et il consumera les palais de Jérusalem et ne s'éteindra pas\FTNT{2 R. 25:9.}.
\Chap{18}
\TextTitle{La maison du potier~; appel à la repentance et avertissement}
\VerseOne{}Cette parole fut adressée à Jérémie de la part de Yahweh, disant~:
\VS{2}Lève-toi et descends dans la maison d'un potier~; et là, je te ferai entendre mes paroles.
\VS{3}Je descendis donc dans la maison d'un potier, et voici, il faisait son ouvrage, assis sur sa selle.
\VS{4}Et le vase qu'il faisait avec l'argile qu'il tenait dans sa main, fut gâté~; et il en fit encore un autre vase, comme il lui sembla bon de le faire.
\VS{5}Alors la parole de Yahweh me fut adressée, en disant~:
\VS{6}Maison d'Israël, ne puis-je pas faire de vous comme a fait ce potier~? Dit Yahweh. Voici, comme l'argile est dans la main d'un potier, ainsi vous êtes dans ma main, maison d'Israël~!
\VS{7}En un instant je parle contre une nation et contre un royaume, pour arracher, pour démolir, et pour détruire~;
\VS{8}mais si cette nation, contre laquelle j'ai parlé, revient de sa méchanceté, je me repentirai aussi du mal que j'avais pensé lui faire\FTNT{Jon. 3:6-10.}.
\VS{9}Et si en un instant je parle d'une nation et d'un royaume, pour l'édifier et pour le planter~;
\VS{10} et que cette nation fasse ce qui est mal à mes yeux, en sorte qu'elle n'écoute pas ma voix, je me repentirai aussi du bien que j'avais dit que je lui ferais.
\VS{11}Or donc, parle maintenant aux hommes de Juda et aux habitants de Jérusalem, en disant~: Ainsi parle Yahweh~: Voici, je projette du mal contre vous, et je forme un dessein contre vous. Détournez-vous donc chacun de votre mauvaise voie, et amendez votre voie et vos actions~!
\VS{12}Et ils répondent~: Il n'y a plus d'espérance~; c'est pourquoi nous suivrons nos pensées, chacun de nous fera selon les penchants de son mauvais cœur.
\VS{13}C'est pourquoi ainsi parle Yahweh~: Demandez maintenant aux nations~! Qui a entendu de telles choses~? La vierge d'Israël a fait une chose très horrible\FTNT{1 Co. 5:1.}.
\VS{14}La neige du Liban abandonne-t-elle le rocher du champ~? Les eaux étrangères, fraîches et ruisselantes, tarissent-elles~?
\VS{15}Mais mon peuple m'a oublié, et il brûle de l'encens à ce qui n'est que vanité, et qui les a fait chanceler dans leurs voies, pour les faire retirer des anciens sentiers, afin de marcher dans les sentiers d'un chemin non frayé~;
\VS{16}pour faire venir sur leur pays une désolation et un opprobre perpétuel~; quiconque passe par là, en est étonné et secoue la tête.
\VS{17}Je les disperserai devant l'ennemi, comme par le vent d'orient~; je leur tournerai le dos, et non pas la face, au jour de leur calamité\FTNT{Es. 27:8.}.
\VS{18}Et ils ont dit~: Venez, et faisons des complots contre Jérémie~! Car la loi ne périra pas chez le prêtre, ni le conseil chez le sage, ni la parole chez le prophète. Venez, et tuons-le avec la langue, et ne soyons pas attentifs à ses discours~!
\VS{19}Yahweh~! Fais attention à moi, et écoute la voix de ceux qui contestent avec moi~!
\VS{20}Le mal sera-t-il rendu pour le bien\FTNT{Ps. 35:12~; Ps. 109:5.}~? Car ils ont creusé une fosse pour mon âme. Souviens-toi que je me suis tenu devant toi, afin de parler pour leur bien, et afin de détourner d'eux ta grande colère.
\VS{21}C'est pourquoi livre leurs fils à la famine, et fais couler leur sang à coups d'épée~; que leurs femmes soient privées d'enfants, et deviennent veuves, et que leurs maris soient enlevés par la mort~; et leurs jeunes gens frappés par l'épée dans la bataille\FTNT{Ps. 109:9-13.}~!
\VS{22}Qu'on entende le cri de leurs maisons, quand tu feras venir subitement des troupes contre eux~! Car ils ont creusé une fosse pour me prendre, et ils ont caché des pièges pour mes pieds.
\VS{23}Or tu sais, ô Yahweh~! Que tout leur conseil est contre moi pour me mettre à mort~; ne sois pas apaisé à l'égard de leur iniquité, et n'efface pas leur péché de devant ta face, mais qu'on les fasse tomber en ta présence~; agis contre eux au temps de ta colère.
\Chap{19}
\TextTitle{Le vase brisé~: Image de Juda}
\VerseOne{}Ainsi a parlé Yahweh~: Va, et achète un vase de terre d'un potier, et prends avec toi des anciens du peuple et des anciens des prêtres.
\VS{2}Et sors à la vallée de Ben-Hinnom, qui est près de l'entrée de la porte de la orientale, et crie là les paroles que je te dirai.
\VS{3}Dis donc~: Rois de Juda, et vous, habitants de Jérusalem, écoutez la parole de Yahweh~! Ainsi parle Yahweh des armées, le Dieu d'Israël~: Voici, je vais faire venir sur ce lieu-ci un mal, tel que quiconque l'entendra, les oreilles lui tinteront\FTNT{1 S. 3:11~; 2 R. 21:12.}~; 
\VS{4}parce qu'ils m'ont abandonné, et qu'ils ont profané ce lieu, et y ont brûlé de l'encens à d'autres dieux, que ni eux, ni leurs pères, ni les rois de Juda n'ont connus, et parce qu'ils ont rempli ce lieu du sang des innocents~;
\VS{5}et qu'ils ont bâti des hauts lieux à Baal, afin de brûler au feu leurs fils pour en faire des holocaustes à Baal~: Ce que je n'avais pas ordonné, dont je n'avais pas parlé, et qui ne m'était pas monté au cœur.
\VS{6}A cause de cela, voici les jours viennent, dit Yahweh, que ce lieu-ci ne sera plus appelé Topheth, ni la vallée de Ben-Hinnom, mais la vallée de la tuerie.
\VS{7}Et j'anéantirai dans ce lieu-ci le conseil de Juda et de Jérusalem~; et je les ferai tomber par l'épée devant leurs ennemis et par la main de ceux qui cherchent leur vie~; et je donnerai leurs cadavres en pâture aux oiseaux des cieux et aux bêtes de la terre.
\VS{8}Je ferai de cette ville un objet de désolation et de moquerie~; quiconque passera près d'elle sera étonné et sifflera à cause de toutes ses plaies.
\VS{9}Et je leur ferai manger la chair de leurs fils et la chair de leurs filles~; et chacun mangera la chair de son compagnon durant le siège, et dans la détresse où les réduiront leurs ennemis et ceux qui cherchent leur vie\FTNT{Lé. 26:29~; De. 28:53~; La. 2:20.}.
\VS{10}Puis tu briseras le vase, sous les yeux des hommes qui seront allés avec toi.
\VS{11}Et tu leur diras~: Ainsi parle Yahweh des armées~: Je briserai ce peuple et cette ville, de même qu'on brise un vase de potier, qui ne peut être réparé. Et ils seront enterrés à Topheth parce qu'il n'y aura plus d'autre lieu pour les enterrer.
\VS{12}Je ferai ainsi à ce lieu-ci, dit Yahweh, et à ses habitants, et je rendrai cette ville semblable à Topheth~;
\VS{13}et les maisons de Jérusalem, et les maisons des rois de Juda, seront impures comme le lieu de Topheth, à cause de toutes les maisons sur les toits desquelles ils brûlaient de l'encens à toute l'armée des cieux, et faisaient des libations à d'autres dieux.
\VS{14}Puis Jérémie revint de Topheth, là où Yahweh l'avait envoyé pour prophétiser. Et il se tint debout dans le parvis de la maison de Yahweh, et il dit à tout le peuple~:
\VS{15}Ainsi parle Yahweh des armées, le Dieu d'Israël~: Voici, je vais faire venir sur cette ville et sur toutes ses villes tout le mal que j'ai prononcés contre elle, parce qu'ils ont raidi leur cou pour ne pas écouter mes paroles.
\Chap{20}
\TextTitle{Paschhur outrage Jérémie}
\VerseOne{}Alors Paschhur, fils d'Immer, qui était prêtre et inspecteur en chef dans la maison de Yahweh, entendit Jérémie qui prophétisait ces choses.
\VS{2}Et Paschhur frappa le prophète Jérémie, et le mit dans la prison qui était à la haute porte de Benjamin, dans la maison de Yahweh.
\VS{3}Et il arriva que dès le lendemain, Paschhur tira Jérémie hors de la prison. Et Jérémie lui dit~: Yahweh ne t'appelle pas du nom de Paschhur, mais Magor-Missabib\FTNT{«~Magor-Missabib~» veut dire «~terreur de chaque côté~».}.
\VS{4}Car ainsi parle Yahweh~: Voici, je vais te livrer à la terreur, toi et tous tes amis qui tomberont par l'épée de leurs ennemis, et tes yeux le verront. Je livrerai tous ceux de Juda entre les mains du roi de Babylone, qui les transportera à Babylone et les frappera de l'épée.
\VS{5}Et je livrerai toutes les richesses de cette ville, tout son travail, et tout ce qu'elle a de précieux, je livrerai, dis-je, tous les trésors des rois de Juda entre les mains de leurs ennemis, qui les pilleront, les enlèveront et les conduiront à Babylone.
\VS{6}Et toi, Paschhur, et tous ceux qui demeurent dans ta maison, vous irez en captivité~; tu iras à Babylone, tu y mourras et y seras enterré, toi et tous tes amis auxquels tu as prophétisé le mensonge.
\TextTitle{Jérémie gémit auprès de Yahweh}
\VS{7}Ô Yahweh~! Tu m'as persuadé, et je me suis laissé persuader~; tu m'as saisi et tu m'as vaincu. Je suis un objet de moquerie chaque jour, chacun se moque de moi.
\VS{8}Car depuis que je parle, je n'ai fait que jeter des cris, que crier violence et dévastation~! Et la parole de Yahweh est pour moi un sujet d'opprobre et de moquerie chaque jour\FTNT{Es. 57:4.}.
\VS{9}C'est pourquoi j'ai dit~: Je ne ferai plus mention de lui, je ne parlerai plus en son Nom~; mais il y a dans mon cœur comme un feu ardent, renfermé dans mes os~; je suis fatigué de le porter, et je n'en peux plus.
\VS{10}Car j'ai entendu les insultes de plusieurs, la frayeur m'a saisi de tous côtés~; rapportez, disent-ils, et nous le rapporterons~! Tous ceux qui étaient en paix avec moi observent si je bronche, et disent~: Peut-être se laissera-t-il séduire, et nous le vaincrons, nous tirerons vengeance de lui~!
\VS{11}Mais Yahweh est avec moi comme un héros puissant~; c'est pourquoi ceux qui me persécutent seront renversés, ils ne me vaincront pas~; ils seront honteux, car ils n'ont pas réussi~: Ce sera une honte éternelle qui ne s'oubliera jamais.
\VS{12}C'est pourquoi, Yahweh des armées qui sonde les justes, qui voit les reins et les cœurs, fais que je voie la vengeance que tu en feras, car je t'ai montré ma cause.
\VS{13}Chantez à Yahweh, louez Yahweh~! Car il délivre l'âme des pauvres de la main des méchants.
\VS{14}Maudit soit le jour où je suis né~! Que le jour où ma mère m'a enfanté ne soit pas béni~!
\VS{15}Maudit soit l'homme qui porta cette nouvelle à mon père, en lui disant~: Un fils mâle t'est né, et qui le combla de joie~!
\VS{16}Que cet homme-là, soit comme les villes que Yahweh a renversées sans s'en repentir~! Qu'il entende la clameur le matin, et le cri de guerre au temps du midi\FTNT{Ge. 19:24-25~; So. 2:4.}~!
\VS{17}Que ne m'a-t-on fait mourir dans le sein de ma mère~! Pourquoi ma mère ne m'a-t-elle pas servi de sépulcre~? Et pourquoi n'est-elle pas restée éternellement enceinte~?
\VS{18}Pourquoi suis-je sorti de son sein pour ne voir que peine et douleur, et pour consumer mes jours dans la honte~?
\Chap{21}
\TextTitle{Prophétie sur les rois de Juda~: Sédécias}
\VerseOne{}La parole qui fut adressée à Jérémie de la part de Yahweh, lorsque le roi Sédécias envoya vers lui Paschhur, fils de Malkija, et Sophonie, fils de Maaséja, le prêtre, pour lui dire~:
\VS{2}Consulte maintenant Yahweh pour nous~; car Nebucadnetsar, roi de Babylone, combat contre nous~; peut-être que Yahweh fera-t-il en notre faveur un de ses miracles, afin qu'il se retire de nous.
\VS{3}Et Jérémie leur dit~: Vous direz ainsi à Sédécias~:
\VS{4}Ainsi parle Yahweh, le Dieu d'Israël~: Voici, je vais détourner les armes de guerre qui sont dans vos mains, et avec lesquelles vous combattez en dehors des murailles contre le roi de Babylone et contre les Chaldéens qui vous assiègent, et je les rassemblerai au milieu de cette ville.
\VS{5}Et je combattrai contre vous, avec une main étendue, et avec un bras puissant, avec colère, avec fureur, et avec une grande indignation.
\VS{6}Et je frapperai les habitants de cette ville, les hommes, et les bêtes~; et ils mourront d'une grande peste.
\VS{7}Et après cela, dit Yahweh, je livrerai Sédécias, roi de Juda, et ses serviteurs, et le peuple, et ceux qui dans cette ville survivront à la peste, à l'épée et à la famine, entre les mains de Nebucadnetsar\FTNT{2 R. 24-25.}, roi de Babylone, et entre les mains de leurs ennemis, et entre les mains de ceux qui cherchent leur vie~; et il les frappera au tranchant de l'épée, il ne les épargnera pas, il n'en aura pas de compassion, il n'en aura pas de pitié.
\VS{8}Tu diras aussi à ce peuple~: Ainsi parle Yahweh~: Voici, je mets devant vous le chemin de la vie et le chemin de la mort\FTNT{De. 30:19.}.
\VS{9}Quiconque restera dans cette ville mourra par l'épée, ou par la famine, ou par la peste, mais celui qui en sortira, et se rendra aux Chaldéens qui vous assiègent vivra et aura sa vie pour butin.
\VS{10}Car je dresse ma face en mal et non en bien contre cette ville, dit Yahweh~; elle sera livrée entre les mains du roi de Babylone, et il la brûlera par le feu.
\VS{11}Et quant à la maison du roi de Juda~: Ecoutez la parole de Yahweh~!
\VS{12}Maison de David~! Ainsi parle Yahweh~: Rendez la justice dès le matin, et délivrez celui qui aura été pillé d'entre les mains de l'oppresseur, de peur que ma fureur ne sorte comme un feu, et qu'elle ne brûle sans qu'on puisse l'éteindre, à cause de la méchanceté de vos actions.
\VS{13}Voici, j'en veux à toi qui habites dans la vallée, et qui es le rocher de la plaine, dit Yahweh~; à vous qui dites~: Qui descendra contre nous, et qui entrera dans nos demeures~?
\VS{14}Et je vous punirai selon le fruit de vos actions, dit Yahweh~; et je mettrai le feu dans sa forêt qui consumera tout ce qui est autour d'elle\FTNT{Ez. 21:2-3.}.
\Chap{22}
\TextTitle{Sédécias averti de la destruction de Jérusalem}
\VerseOne{}Ainsi parle Yahweh~: Descends dans la maison du roi de Juda, et là prononce cette parole.
\VS{2}Tu diras donc~: Ecoute la parole de Yahweh, ô roi de Juda qui es assis sur le trône de David, toi et tes serviteurs, et ton peuple, qui entrez par ces portes~!
\VS{3}Ainsi parle Yahweh~: Faites droit et justice~; et délivrez celui qui aura été pillé d'entre les mains de l'oppresseur~; ne maltraitez pas l'orphelin, ni l'étranger, ni la veuve~; et n'usez d'aucune violence, et ne répandez pas le sang innocent dans ce lieu-ci.
\VS{4}Car si vous mettez exactement en effet cette parole, alors les rois qui sont assis à la place de David sur son trône, montés sur des chars et sur des chevaux, entreront par les portes de cette maison, eux et leurs serviteurs, et leur peuple.
\VS{5}Mais si vous n'écoutez pas ces paroles, je le jure par moi-même\FTNT{Es. 45:23~; Hé. 6:13.}, dit Yahweh, que cette maison deviendra une ruine.
\VS{6}Car ainsi parle Yahweh sur la maison du roi de Juda~: Tu es pour moi un Galaad, et le sommet du Liban~; mais certainement, je ferai de toi un désert, une ville sans habitants.
\VS{7}Je prépare contre toi des destructeurs, chacun avec ses armes, qui couperont tes cèdres de choix, et les jetteront au feu.
\VS{8}Et plusieurs nations passeront près de cette ville, et chacun dira à son compagnon~: Pourquoi Yahweh a-t-il fait ainsi à cette grande ville\FTNT{De. 29:24-28~; 1 R. 9:8.}~?
\VS{9}Et on dira~: C'est parce qu'ils ont abandonné l'alliance de Yahweh, leur Dieu, et qu'ils se sont prosternés devant d'autres dieux et les ont servis.
\TextTitle{Prophétie sur les rois de Juda~: Joachaz (Schallum)}
\VS{10}Ne pleurez pas celui qui est mort, et ne vous lamentez pas sur lui~; mais pleurez amèrement celui qui s'en va, car il ne reviendra plus, il ne reverra plus le pays de sa naissance.
\VS{11}Car ainsi parle Yahweh sur Schallum, fils de Josias, roi de Juda, qui régnait à la place de Josias, son père, et qui est sorti de ce lieu~: Il n'y reviendra plus~;
\VS{12}mais il mourra dans le lieu où on l'a transporté, et ne verra plus ce pays.
\TextTitle{Prophétie sur les rois de Juda~: Jojakim}
\VS{13}Malheur à celui qui bâtit sa maison par l'injustice, et ses chambres hautes sans droiture~; qui fait travailler son prochain pour rien, sans lui donner le salaire de son travail\FTNT{Lé. 19:13~; De. 24:14-15~; Ha. 2:9.}.
\VS{14}Qui dit~: Je me bâtirai une grande maison et des chambres spacieuses, et qui s'y fait percer des fenêtres~; elle est lambrissée de cèdre, et peinte de vermillon.
\VS{15}Régneras-tu, parce que tu t'enfermes dans du cèdre~? Ton père n'a-t-il pas mangé et bu~? Quand il a fait jugement et justice, alors il a prospéré.
\VS{16}Il jugeait la cause du pauvre et de l'indigent, alors il a prospéré. N'est-ce pas là me connaître~? Dit Yahweh.
\VS{17}Mais tes yeux et ton cœur ne sont adonnés qu'à ton gain déshonnête, qu'à répandre le sang innocent, qu'à faire du tort et qu'à opprimer.
\VS{18}C'est pourquoi ainsi parle Yahweh sur Jojakim, fils de Josias, roi de Juda~: On ne le pleurera pas en disant~: Hélas, mon frère~! et hélas, ma sœur~! On ne le pleurera pas en disant~: Hélas, seigneur~! Et hélas, sa majesté~!
\VS{19}Il sera enterré de l'ensevelissement d'un âne, étant traîné et jeté hors des portes de Jérusalem.
\TextTitle{Prophétie sur les rois de Juda~: Jojakin}
\VS{20}Monte sur le Liban, et crie~! Donne de la voix sur le Basan~! Crie du haut d'Abarim~! Car tous ceux qui t'aimaient sont brisés.
\VS{21}Je t'ai parlé durant ta grande prospérité, mais tu disais~: Je n'écouterai pas~; telle est ta voie depuis ta jeunesse, tu n'as pas écouté ma voix.
\VS{22}Tous tes pasteurs, un vent les fera paître, et tes amoureux iront en captivité~; certainement tu seras alors honteuse et confuse, à cause de toute ta malice.
\VS{23}Toi qui habites sur le Liban, et qui fais ton nid dans les cèdres, que tu seras à plaindre quand les douleurs t'atteindront, les douleurs comme celles d'une femme qui enfante~!
\VS{24}Je suis vivant, dit Yahweh, que quand Jéconia, fils de Jojakim, roi de Juda, serait une bague à ma main droite, je t'arracherais de là.
\VS{25}Je te livrerai entre les mains de ceux qui cherchent ta vie, et entre les mains de ceux devant qui tu es craintif, et entre les mains de Nebucadnetsar, roi de Babylone, et entre les mains des Chaldéens\FTNT{2 R. 24:14~; Ez. 17:12~; 2 Ch. 36:10.}.
\VS{26}Et je te jetterai, toi et ta mère qui t'a enfanté, dans un autre pays où vous n'êtes pas nés, et vous y mourrez.
\VS{27}Et quant au pays qu'ils désirent pour y retourner, ils n'y retourneront pas.
\VS{28}Cet homme, Jéconia, est-il un vase méprisé et brisé~? Est-il un objet qui ne fait plus plaisir~? Pourquoi sont-ils jetés là, lui et sa postérité, lancés, dis-je, dans un pays qu'ils ne connaissent pas\FTNT{Os. 8:8.}~?
\VS{29}Ô terre, terre, terre~! Ecoute la parole de Yahweh~!
\VS{30}Ainsi parle Yahweh~: Ecrivez que cet homme-là est privé d'enfants, que c'est un homme qui ne prospérera pas pendant ses jours~; et que même il n'y aura aucun homme de sa postérité qui prospère, étant assis sur le trône de David et dominant encore en Juda\FTNT{2 R. 24:8-16.}.
\Chap{23}
\TextTitle{Israël sera rassemblé par le Messie}
\VerseOne{}Malheur aux pasteurs qui détruisent et dispersent le troupeau de mon pâturage~! Dit Yahweh.
\VS{2}C'est pourquoi ainsi parle Yahweh, le Dieu d'Israël, sur les pasteurs qui paissent mon peuple~: Vous avez dispersé mes brebis et vous les avez chassées, et ne vous en êtes pas occupés~; voici, je vous punirai à cause de la méchanceté de vos actions, dit Yahweh.
\VS{3}Mais je rassemblerai le reste de mes brebis de tous les pays où je les ai chassées~; et je les ramènerai à leur pâturage, et elles seront fécondes et multiplieront.
\VS{4}Je susciterai aussi sur elles des pasteurs qui les paîtront, et elles n'auront plus de peur, et ne s'épouvanteront plus, et il n'en manquera aucune, dit Yahweh.
\VS{5}Voici, les jours viennent, dit Yahweh, où je susciterai à David un Germe juste, qui régnera en Roi~; il prospérera, et exercera le droit et la justice dans le pays\FTNT{Es. 4:2~; Za. 6:12-13~; Ps. 96:13~; Lu. 1:32-33.}.
\VS{6}En son temps, Juda sera sauvé, Israël demeurera en sécurité~; et c'est ici le nom dont on l'appellera~: Yahweh notre justice.
\VS{7}C'est pourquoi, voici, les jours viennent, dit Yahweh, qu'on ne dira plus~: Yahweh est vivant, lui qui a fait monter les enfants d'Israël du pays d'Egypte~!
\VS{8}Mais~: Yahweh est vivant, lui qui a fait monter et qui a ramené la postérité de la maison d'Israël, du pays du nord et de tous les pays où je les avais chassés, et ils habiteront dans leur pays.
\TextTitle{Jugement sur les faux prophètes}
\VS{9}A cause des prophètes mon cœur est brisé au-dedans de moi, tous mes os se relâchent~; je suis comme un homme ivre, et comme un homme que le vin a surmonté, à cause de Yahweh, et à cause des paroles de sa sainteté.
\VS{10}Car le pays est rempli d'hommes qui commettent l'adultère~; et le pays est en deuil à cause de la malédiction~: Les pâturages du désert sont desséchés, leur course ne va qu'au mal, et leur force à ce qui n'est pas droit.
\VS{11}Car le prophète et le prêtre sont corrompus~; j'ai même trouvé dans ma maison leur méchanceté, dit Yahweh.
\VS{12}C'est pourquoi leur chemin sera comme des lieux glissants dans l'obscurité, ils y seront poussés et ils tomberont\FTNT{Ps. 35:6~; Pr. 4:19.}~; car je ferai venir du mal sur eux, dans l'année de leur châtiment, dit Yahweh.
\VS{13}Or j'ai vu de la folie dans les prophètes de Samarie, car ils prophétisaient par Baal, et faisaient égarer mon peuple Israël.
\VS{14}Mais j'ai vu des choses horribles dans les prophètes de Jérusalem car ils commettent des adultères, et ils marchent dans le mensonge~; ils fortifient les mains de ceux qui font le mal, afin qu'aucun ne se détourne de sa méchanceté~; ils me sont tous comme Sodome, et les habitants de la ville comme Gomorrhe\FTNT{Es. 1:9.}.
\VS{15}C'est pourquoi, ainsi parle Yahweh des armées sur les prophètes~: Voici, je vais leur faire manger de l'absinthe, et leur ferai boire des eaux empoisonnées~; car c'est par les prophètes que la profanation est venue dans tout le pays.
\VS{16}Ainsi parle Yahweh des armées~: N'écoutez pas les paroles des prophètes qui vous prophétisent~! Ils vous font devenir vains, ils disent les visions de leur cœur, et ils ne les tiennent pas de la bouche de Yahweh.
\VS{17}Ils ne cessent de dire à ceux qui me méprisent~: Yahweh a dit~: Vous aurez la paix~; et ils disent à tous ceux qui marchent suivant les penchants de leur cœur~: Il ne vous arrivera aucun mal\FTNT{Ez. 13:10.}.
\VS{18}Car qui s'est trouvé au conseil secret de Yahweh~? Et qui a aperçu et entendu sa parole\FTNT{Es. 40:13~; Job 15:8~; 1 Co. 2:16.}~? Qui a été attentif à sa parole, et l'a entendue~?
\VS{19}Voici la tempête de Yahweh, sa fureur va se montrer, et le tourbillon prêt à fondre tombera sur la tête des méchants.
\VS{20}La colère de Yahweh ne se détournera pas jusqu'à ce qu'il ait accompli, exécuté les desseins de son cœur. Vous aurez une claire intelligence de ceci dans les derniers jours\FTNT{Ge. 49:1-2.}.
\VS{21}Je n'ai pas envoyé ces prophètes-là, et ils ont couru~; je ne leur ai pas parlé, et ils ont prophétisé.
\VS{22}S'ils s'étaient trouvés dans mon conseil secret, ils auraient aussi fait entendre mes paroles à mon peuple, et ils les auraient ramenés de leur mauvaise voie, de la méchanceté de leurs actions.
\VS{23}Suis-je un Dieu de près, dit Yahweh, et ne suis-je pas aussi un Dieu de loin~?
\VS{24}Quelqu'un se cachera-t-il dans un lieu secret sans que je ne le voie~? Dit Yahweh. Ne remplis-je pas, moi, les cieux et la terre~? Dit Yahweh\FTNT{Ps. 139:7-8~; Am. 9:2-3.}.
\VS{25}J'ai entendu ce que les prophètes disent, prophétisant le mensonge en mon Nom, et disant~: J'ai eu un songe~! J'ai eu un songe~!
\VS{26}Jusqu'à quand ceci sera-t-il au cœur des prophètes qui prophétisent le mensonge, et qui prophétisent la tromperie de leur cœur~?
\VS{27}Qui pensent à comment ils feront oublier mon Nom à mon peuple, par les songes que chacun d'eux raconte à son compagnon, comme leurs pères ont oublié mon Nom pour Baal\FTNT{Jg. 2:13.}.
\VS{28}Que le prophète qui a eu un songe raconte ce songe~; et que celui qui a ma parole proclame ma parole en vérité. Quelle convenance y a-t-il entre la paille et le froment~? Dit Yahweh.
\VS{29}Ma parole n'est-elle pas comme un feu, dit Yahweh, et comme un marteau qui brise le roc~?
\VS{30}C'est pourquoi voici, dit Yahweh, j'en veux aux prophètes qui se dérobent mes paroles l'un à l'autre.
\VS{31}Voici, dit Yahweh, j'en veux aux prophètes qui accommodent leurs langues, et qui disent~: Il dit.
\VS{32}Voici, dit Yahweh, j'en veux à ceux qui prophétisent des songes faux, et qui les racontent, et font égarer mon peuple par leurs mensonges, et par leur témérité, quoique je ne les ai pas envoyés, et que je ne leur aie pas donné d'ordre~; c'est pourquoi ils ne sont d'aucune utilité à ce peuple, dit Yahweh\FTNT{So. 3:4.}.
\VS{33}Si donc ce peuple t'interroge, ou qu'il interroge le prophète, ou le prêtre, en disant~: Quel est le fardeau de Yahweh~? Tu leur diras~: Quel est ce fardeau~? Je vous abandonnerai, dit Yahweh.
\VS{34}Et quant au prophète, et au prêtre, et au peuple qui dira~: Fardeau de Yahweh~; je punirai cet homme-là et sa maison.
\VS{35}Vous direz ainsi chacun à son compagnon, et chacun à son frère~: Qu'a répondu Yahweh~? Qu'a dit Yahweh~?
\VS{36}Et vous ne mentionnerez plus~: Fardeau de Yahweh~; car la parole de chacun sera pour lui un fardeau~; parce que vous tordez les paroles du Dieu vivant\FTNT{2 Pi. 3:15-16.}, les paroles de Yahweh des armées, notre Dieu.
\VS{37}Tu diras au prophète~: Que t'a répondu Yahweh et que t'a dit Yahweh~?
\VS{38}Et si vous dites~: Fardeau de Yahweh~; à cause de cela, parle Yahweh, parce que vous dites cette parole~: Fardeau de Yahweh~; et que j'ai envoyé vers vous pour dire~: Ne dites plus~: Fardeau de Yahweh~!
\VS{39}A cause de cela, me voici, et je vous oublierai entièrement, et je vous rejetterai loin de ma présence, vous et la ville que j'ai donnée à vous et à vos pères.
\VS{40}Et je mettrai sur vous un opprobre éternel et une honte éternelle, qui ne s'oublieront pas.
\Chap{24}
\TextTitle{Bonnes figues: Futur retour en Juda des captifs de Babylone~; mauvaises figues: Jugement sur Jérusalem}
\VerseOne{}Yahweh me fit voir une vision, et voici deux paniers de figues posés devant le temple de Yahweh, après que Nebucadnetsar, roi de Babylone, eut transporté de Jérusalem, Jéconia, fils de Jojakim, roi de Juda, les chefs de Juda, avec les charpentiers et les serruriers, et les eut conduits à Babylone.
\VS{2}L'un des paniers avait de très bonnes figues, comme les figues de la première récolte~; et l'autre panier avait de très mauvaises figues, qu'on ne pouvait manger à cause de leur mauvaise qualité.
\VS{3}Et Yahweh me dit~: Que vois-tu, Jérémie~? Et je répondis~: Des figues. Les bonnes figues sont très bonnes, et les mauvaises sont très mauvaises et ne peuvent être mangées à cause de leur mauvaise qualité.
\VS{4}Alors la parole de Yahweh me fut adressée, en disant~:
\VS{5}Ainsi parle Yahweh, le Dieu d'Israël~: Comme tu distingues ces bonnes figues, ainsi je me souviendrai, pour leur faire du bien, des captifs de Juda, que j'ai envoyés hors de ce lieu dans le pays des Chaldéens.
\VS{6}Et je les regarderai d'un œil favorable, et je les ramènerai dans ce pays, je les y rétablirai et je ne les détruirai plus, je les planterai et je ne les arracherai pas.
\VS{7}Et je leur donnerai un cœur pour me connaître, pour connaître, dis-je, que je suis Yahweh~; et ils seront mon peuple, et je serai leur Dieu~: Car ils reviendront à moi de tout leur cœur\FTNT{De. 30:6~; Ez. 11:19.}.
\VS{8}Et comme de mauvaises figues qu'on ne peut manger, tant elles sont mauvaises~; ainsi certainement, dit Yahweh, je ferai devenir Sédécias, roi de Juda, ses chefs, et le reste de Jérusalem, ceux qui sont restés dans ce pays, et ceux qui habitent dans le pays d'Egypte.
\VS{9}Et je les livrerai pour être agités, pour leur malheur, par tous les royaumes de la terre, et pour être en opprobre, en proverbe, en raillerie, et en malédiction, par tous les lieux où je les aurai chassés\FTNT{De. 28:37.}.
\VS{10}Et j'enverrai sur eux l'épée, la famine et la peste, jusqu'à ce qu'ils soient consumés du pays que j'avais donné à eux et à leurs pères.
\Chap{25}
\TextTitle{Prophétie sur les soixante-dix ans de captivité babylonienne\FTNTT{Da. 9:2.}}
\VerseOne{}La parole qui fut adressée à Jérémie touchant tout le peuple de Juda, la quatrième année de Jojakim, fils de Josias, roi de Juda, qui est la première année de Nebucadnetsar, roi de Babylone,
\VS{2}parole que Jérémie, le prophète, prononça à tout le peuple de Juda, et à tous les habitants de Jérusalem, en disant~:
\VS{3}Depuis la treizième année de Josias, fils d'Amon, roi de Juda, jusqu'à ce jour, qui est la vingt-troisième année, la parole de Yahweh m'a été adressée~; et je vous ai parlé, me levant dès le matin et parlant, et vous n'avez pas écouté.
\VS{4}Et Yahweh vous a envoyé tous ses serviteurs, les prophètes, se levant dès le matin et les envoyant~; et vous ne les avez pas écoutés, vous n'avez pas prêté l'oreille pour écouter.
\VS{5}Lorsqu'ils disaient~: Détournez-vous maintenant chacun de sa mauvaise voie et de la méchanceté de vos actions, et vous habiterez d'un siècle à l'autre dans le pays que Yahweh a donné à vous et à vos pères\FTNT{Jon. 3:8~; 2 R. 17:13.}~;
\VS{6}et n'allez pas après d'autres dieux pour les servir et pour vous prosterner devant eux, ne m'irritez pas par les œuvres de vos mains, et je ne vous ferai aucun mal.
\VS{7}Mais vous m'avez désobéi, dit Yahweh, pour m'irriter par les œuvres de vos mains, pour votre malheur.
\VS{8}C'est pourquoi ainsi parle Yahweh des armées~: Parce que vous n'avez pas écouté mes paroles,
\VS{9}voici j'enverrai et j'assemblerai toutes les familles du nord, dit Yahweh, et j'enverrai, dis-je, vers Nebucadnetsar, roi de Babylone, mon serviteur~; et je les ferai venir contre ce pays et contre ses habitants, et contre toutes ces nations d'alentour~; je les détruirai à la façon de l'interdit, je les mettrai en désolation, en opprobre et en ruines éternelles\FTNT{De. 28:37~; Es. 10:6.}.
\VS{10}Et je ferai cesser parmi eux les cris de joie et les cris d'allégresse, la voix de l'époux et la voix de l'épouse, le bruit des meules et la lumière des lampes\FTNT{Es. 24:7~; Ez. 26:13.}.
\VS{11}Et tout ce pays sera une ruine jusqu'à s'en étonner, et ces nations seront asservies au roi de Babylone pendant soixante-dix ans\FTNT{Voir Jé. 2~:10. Les soixante-dix ans se rapportent également au temps de la domination mondiale babylonienne. Le peuple avait une dette envers Yahweh de 70 ans de sabbats (Lé. 26:34-43~; 2 Ch. 36:21).}.
\TextTitle{Jugement sur Babylone et les nations impies}
\VS{12}Et il arrivera que quand ces soixante-dix ans seront accomplis, je punirai le roi de Babylone et cette nation, dit Yahweh, à cause de leurs iniquités~; je punirai le pays des Chaldéens, que je mettrai en désolations éternelles\FTNT{Da. 9:2.}.
\VS{13}Et je ferai venir sur ce pays-là toutes mes paroles que j'ai prononcées contre lui, toutes les choses qui sont écrites dans ce livre, ce que Jérémie a prophétisé contre toutes ces nations.
\VS{14}Car de grands rois aussi et de grandes nations se serviront d'eux, et je leur rendrai selon leurs actions et selon l'œuvre de leurs mains.
\VS{15}Car ainsi m'a parlé Yahweh, le Dieu d'Israël~: Prends de ma main cette coupe du vin, à savoir de cette fureur-ci, et fais-la boire à toutes les nations vers lesquelles je t'enverrai\FTNT{Ab. 16.}.
\VS{16}Ils en boiront, et ils chancelleront et seront comme fous, à cause de l'épée que j'enverrai parmi eux.
\VS{17}Je pris donc la coupe de la main de Yahweh, et je la fis boire à toutes les nations vers lesquelles Yahweh m'envoyait~:
\VS{18}A Jérusalem et aux villes de Juda, à ses rois et à ses chefs, pour les mettre en désolation, en étonnement, en opprobre et en malédiction, comme il paraît aujourd'hui~;
\VS{19}à Pharaon, roi d'Egypte, à ses serviteurs, à ses chefs, et à tout son peuple~;
\VS{20}et à tout le mélange des peuples d'Arabie, à tous les rois du pays d'Uts, à tous les rois du pays des Philistins, à Askalon, à Gaza, à Ekron, et au reste d'Asdod~;
\VS{21}à Edom, à Moab, et aux fils d'Ammon~;
\VS{22}à tous les rois de Tyr, à tous les rois de Sidon, et aux rois des îles qui sont au-delà de la mer~;
\VS{23}à Dedan, à Théma, à Buz, et à tous ceux qui se coupent les coins de la barbe~;
\VS{24}à tous les rois d'Arabie, et à tous les rois du mélange qui habitent au désert~;
\VS{25}à tous les rois de Zimri, à tous les rois d'Elam, et à tous les rois de Médie~;
\VS{26}à tous les rois du nord, tant proches qu'éloignés l'un de l'autre, et à tous les royaumes du monde qui sont sur la face de la terre. Et le roi de Schéschac boira après eux.
\VS{27}Et tu leur diras~: Ainsi parle Yahweh des armées, le Dieu d'Israël~: Buvez et soyez enivrés, même vomissez, et tombez sans vous relever, à cause de l'épée que j'enverrai parmi vous~!
\VS{28}Or il arrivera qu'ils refuseront de prendre la coupe de ta main pour boire~; mais tu leur diras~: Ainsi parle Yahweh des armées~: Vous en boirez~! Vous en boirez~!
\VS{29}Car voici, je commence à envoyer du mal sur la ville sur laquelle mon nom est invoqué~; et vous, en seriez-vous exempts en quelque sorte~? Vous n'en serez pas exempts~; car je m'en vais appeler l'épée sur tous les habitants de la terre, dit Yahweh des armées\FTNT{1 Pi. 4:17-18.}.
\VS{30}Tu prophétiseras donc contre eux toutes ces paroles-là, et tu leur diras~: Yahweh rugira d'en haut~; il fera entendre sa voix de la demeure de sa sainteté~; il rugira, il rugira contre son agréable demeure~; il poussera un cri contre tous les habitants de la terre\FTNT{Joë. 3:16~; Am. 1:2.}, comme ceux qui foulent au pressoir.
\VS{31}Le son éclatant est parvenu jusqu'à l'extrémité de la terre~; car Yahweh plaide avec les nations, et il conteste contre toute chair. On livrera les méchants à l'épée, dit Yahweh.
\VS{32}Ainsi parle Yahweh des armées~: Voici, le mal va sortir d'une nation à l'autre, et une grande tempête se lèvera des extrémités de la terre.
\VS{33}Et en ce jour-là, ceux qui auront été mis à mort par Yahweh seront étendus d'un bout de la terre à l'autre bout~; ils ne seront ni pleurés, ni recueillis, ni enterrés, mais ils seront comme du fumier sur la face du sol.
\VS{34}Vous, pasteurs, hurlez et criez~! Et vous, les nobles du troupeau, roulez-vous dans la cendre~; car les jours pour vous massacrer sont accomplis. Je vous disperserai et vous tomberez comme un vase précieux.
\VS{35}Et les pasteurs n'auront aucun moyen de s'enfuir, ni les nobles d'échapper. 
\VS{36}Il y aura la voix du cri des bergers, les hurlements des nobles du troupeau~; parce que Yahweh s'en va ravager leur pâturage.
\VS{37}Et les demeures paisibles seront abattues, à cause de l'ardeur de la colère de Yahweh.
\VS{38}Il a abandonné son tabernacle comme un lionceau~; car leur pays est réduit en désert, à cause de l'ardeur du destructeur et à cause, dis-je, de l'ardeur de sa colère.
\Chap{26}
\TextTitle{Avertissement dans le parvis du temple}
\VerseOne{}Au commencement du règne de Jojakim, fils de Josias, roi de Juda, cette parole fut adressée à Jérémie part Yahweh, en disant~:
\VS{2}Ainsi parle Yahweh~: Tiens-toi debout dans le parvis de la maison de Yahweh, et prononce à toutes les villes de Juda qui viennent pour se prosterner dans la maison de Yahweh toutes les paroles que je t'ordonne de leur dire~; n'en retranche pas un mot.
\VS{3}Peut-être qu'ils écouteront et qu'ils se détourneront chacun de sa mauvaise voie~; et je me repentirai du mal que j'avais pensé leur faire à cause de la méchanceté de leurs actions.
\VS{4}Tu leur diras donc~: Ainsi parle Yahweh~: Si vous ne m'écoutez pas pour marcher selon ma loi que je vous ai proposée,
\VS{5}pour obéir aux paroles des prophètes, mes serviteurs, que je vous envoie, me levant dès le matin, et les envoyant, lesquels vous n'avez pas écoutés,
\VS{6}je mettrai cette maison dans le même état que Silo, et je livrerai cette ville à la malédiction, à toutes les nations de la terre.
\VS{7}Or les prêtres, les prophètes, et tout le peuple, entendirent Jérémie prononcer ces paroles dans la maison de Yahweh.
\TextTitle{Jérémie menacé de mort par les prêtres et les prophètes}
\VS{8}Et il arrivera qu'aussitôt que Jérémie eut achevé de prononcer tout ce que Yahweh lui avait ordonné de dire à tout le peuple, les prêtres, les prophètes, et tout le peuple, le saisirent en disant~: Tu mourras, tu mourras\FTNT{Ge. 2:17.}~!
\VS{9}Pourquoi as-tu prophétisé au nom de Yahweh, en disant~: Cette maison sera comme Silo, et cette ville sera déserte tellement que personne n'y habitera~? Et tout le peuple s'assembla autour de Jérémie dans la maison de Yahweh.
\VS{10}Et les chefs de Juda ayant entendu toutes ces choses, montèrent de la maison du roi à la maison de Yahweh, et s'assirent à l'entrée de la porte neuve de la maison de Yahweh.
\VS{11}Et les prêtres et les prophètes parlèrent aux chefs et à tout le peuple, en disant~: Cet homme mérite d'être condamné à la mort~; car il a prophétisé contre cette ville, comme vous l'avez entendu de vos oreilles.
\VS{12}Et Jérémie parla à tous les chefs et à tout le peuple, en disant~: Yahweh m'a envoyé pour prophétiser contre cette maison et contre cette ville toutes les paroles que vous avez entendues.
\VS{13}Maintenant donc, amendez votre conduite et vos actions, écoutez la voix de Yahweh, votre Dieu, et Yahweh se repentira du mal qu'il a prononcé contre vous.
\VS{14}Pour moi, me voici entre vos mains~; faites-moi ce qui vous semblera bon et juste.
\VS{15}Mais sachez comme une chose certaine, que si vous me faites mourir, vous mettrez du sang innocent sur vous, sur cette ville et sur ses habitants~; car en vérité Yahweh m'a envoyé vers vous pour prononcer à vos oreilles toutes ces paroles.
\VS{16}Alors les chefs et tout le peuple dirent aux prêtres et aux prophètes~: Cet homme ne mérite pas d'être condamné à la mort, car il nous a parlé au nom de Yahweh, notre Dieu.
\VS{17}Et quelques-uns des anciens du pays se levèrent et parlèrent à toute l'assemblée du peuple, en disant~:
\VS{18}Michée, de Moréscheth, prophétisait aux jours d'Ezéchias, roi de Juda, et il parlait à tout le peuple de Juda, en disant~: Ainsi parle Yahweh des armées~: Sion sera labourée comme un champ, Jérusalem sera réduite en un monceau de pierres, et la montagne du temple en des hauts lieux d'une forêt\FTNT{Mi. 1:1~; Mi. 3:12.}.
\VS{19}Ezéchias, roi de Juda, et tous ceux de Juda l'ont-ils fait mourir~? Ezéchias ne craignit-il pas Yahweh~? N'implora-t-il pas Yahweh~? Et Yahweh se repentit du mal qu'il avait prononcé contre eux. Et nous, nous ferions donc un grand mal contre nos âmes\FTNT{2 Ch. 32:26.}~!
\VS{20}Mais aussi, dirent les autres, y eut aussi un homme qui prophétisait au nom de Yahweh, savoir, Urie, fils de Schemaeja, de Kirjath-Jearim. Il prophétisa contre cette même ville et contre ce même pays, de la même manière que Jérémie.
\VS{21}Et le roi Jojakim, tous ses vaillants hommes, et tous ses chefs entendirent ses paroles, et le roi chercha à le faire mourir. Urie, qui en fut informé, eut peur, prit la fuite, et se retira en Egypte.
\VS{22}Et le roi Jojakim envoya des hommes en Egypte, savoir, Elnathan, fils d'Acbor, et quelques hommes avec lui, qui allèrent en Egypte.
\VS{23}Et ils firent sortir d'Egypte Urie et l'amenèrent au roi Jojakim qui le frappa avec l'épée et jeta son cadavre sur les sépulcres des fils du peuple.
\VS{24}Toutefois la main d'Achikam, fils de Schaphan, fut avec Jérémie, et empêcha qu'il ne soit livré au peuple pour être mis à mort.
\Chap{27}
\TextTitle{Prophétie~: Les nations seront asservies à Nebucadnetsar}
\VerseOne{}Au commencement du règne de Jojakim\FTNT{Il est probable que ce soit une erreur de copiste, car bien que l'hébreu dit «~Jojakim~», le contexte se rapporte à Sédécias. Voir Jé. 27:3~; Jé. 27:12~; Jé. 27:20~; Jé. 28:1.}, fils de Josias, roi de Juda, cette parole fut adressée à Jérémie de la part de Yahweh, en disant~:
\VS{2}Ainsi m'a parlé Yahweh~: Fais-toi des liens et des jougs, et mets-les sur ton cou\FTNT{Ez. 7:23.}.
\VS{3}Et envoie-les au roi d'Edom, et au roi de Moab, et au roi des fils d'Ammon, et au roi de Tyr et au roi de Sidon, par les mains des messagers qui sont venus à Jérusalem vers Sédécias, roi de Juda~;
\VS{4}et tu leur donneras mes ordres pour leurs maîtres, en disant~: Ainsi parle Yahweh des armées, le Dieu d'Israël~: Vous direz ainsi à vos maîtres~:
\VS{5}J'ai fait la terre, les hommes et les bêtes qui sont sur la terre, par ma grande force et par mon bras étendu, et je la donne à qui cela me plaît\FTNT{De. 32:8.}.
\VS{6}Et maintenant j'ai livré tous ces pays entre les mains de Nebucadnetsar, roi de Babylone, mon serviteur~; et même je lui ai donné les bêtes des champs pour qu'elles lui soient asservies\FTNT{Da. 2:38.}.
\VS{7}Et toutes les nations lui seront asservies, à lui, à son fils, et au fils de son fils, jusqu'à ce que le temps de son pays vienne aussi, et que plusieurs nations et de grands rois l'asservissent.
\VS{8}Et il arrivera que la nation ou le royaume qui ne se soumettra pas à Nebucadnetsar, roi de Babylone, et qui ne soumettra pas son cou au joug du roi de Babylone, je punirai cette nation par l'épée, par la famine et par la peste, dit Yahweh, jusqu'à ce que je les aie consumés par sa main.
\VS{9}Vous donc, n'écoutez pas vos prophètes, ni vos devins, ni vos songeurs, ni vos augures, ni vos magiciens, qui vous parlent, en disant~: Vous ne serez pas asservis au roi de Babylone.
\VS{10}Car ils vous prophétisent le mensonge pour vous faire aller loin de votre pays, afin que je vous chasse et que vous périssiez.
\VS{11}Mais la nation qui livrera son cou au joug du roi de Babylone, et qui le servira, je la laisserai dans son pays, dit Yahweh, pour qu'elle le cultive et qu'elle y demeure.
\VS{12}Puis j'ai parlé à Sédécias, roi de Juda, selon toutes ces paroles-là, en disant~: Soumettez votre cou au joug du roi de Babylone, et rendez-vous sujets, à lui et son peuple, et vous vivrez.
\VS{13}Pourquoi mourriez-vous, toi et ton peuple, par l'épée, par la famine et par la peste, selon que Yahweh a parlé contre la nation qui ne sera pas soumise au roi de Babylone~?
\VS{14}N'écoutez donc pas les paroles des prophètes qui vous parlent en disant~: Vous ne serez pas asservis au roi de Babylone~! Car ils vous prophétisent le mensonge.
\VS{15}Même je ne les ai pas envoyés, dit Yahweh, et ils vous prophétisent faussement en mon nom, afin que je vous rejette et que vous périssiez, vous et les prophètes qui vous prophétisent.
\VS{16}J'ai parlé aussi aux prêtres et à tout ce peuple, en disant~: Ainsi parle Yahweh~: N'écoutez pas les paroles de vos prophètes qui vous prophétisent, en disant~: Voici, les ustensiles de la maison de Yahweh seront bientôt rapportés de Babylone~! Car ils vous prophétisent le mensonge.
\VS{17}Ne les écoutez donc pas, rendez-vous sujets au roi de Babylone, et vous vivrez. Pourquoi cette ville serait-elle réduite en un désert~?
\VS{18}Et s'ils sont prophètes et si la parole de Yahweh est en eux, qu'ils intercèdent maintenant auprès de Yahweh des armées, afin que les ustensiles qui restent dans la maison de Yahweh, dans la maison du roi de Juda, et dans Jérusalem, n'aillent pas à Babylone.
\VS{19}Car ainsi parle Yahweh des armées au sujet des colonnes, de la mer, des bases, et des autres ustensiles qui sont restés dans cette ville,
\VS{20}que Nebucadnetsar, roi de Babylone, n'a pas emportés, quand il a transporté de Jérusalem à Babylone, Jéconia, fils de Jojakim, roi de Juda, et tous les nobles de Juda et de Jérusalem,
\VS{21}Yahweh, dis-je, des armées le Dieu d'Israël, parle ainsi au sujet des ustensiles qui restent dans la maison de Yahweh, dans la maison du roi de Juda et dans Jérusalem~:
\VS{22}Ils seront emportés à Babylone, et ils y demeureront jusqu'au jour où je les visiterai, dit Yahweh, et où je les ferai remonter et revenir dans ce lieu\FTNT{2 R. 24:14-15~; Esd. 1:7-11~; 2 Ch. 25:13-16~; 2 Ch. 36:18.}.
\Chap{28}
\TextTitle{Hanania meurt suite à sa prophétie mensongère}
\VerseOne{}Il arriva aussi, en cette même année, au commencement du règne de Sédécias, roi de Juda, à savoir, au cinquième mois de la quatrième année, que Hanania, fils d'Azzur, prophète de Gabaon, me parla dans la maison de Yahweh, aux yeux des prêtres et de tout le peuple, en disant~:
\VS{2}Ainsi parle Yahweh des armées, le Dieu d'Israël~: Je romps le joug du roi de Babylone~!
\VS{3}Dans deux années accomplies, et je ferai rapporter dans ce lieu tous les ustensiles de la maison de Yahweh, que Nebucadnetsar, roi de Babylone, a pris de ce lieu, et qu'il a transportés à Babylone.
\VS{4}Et je ferai revenir dans ce lieu, dit Yahweh, Jéconia, fils de Jojakim, roi de Juda, et tous les captifs de Juda qui sont allés à Babylone~; car je romprai le joug du roi de Babylone.
\VS{5}Alors Jérémie, le prophète, répondit à Hanania, le prophète, aux yeux des prêtres, et aux yeux de tout le peuple qui se tenait dans la maison de Yahweh.
\VS{6}Et Jérémie, le prophète, dit~: Ainsi soit-il~! Que Yahweh fasse ainsi~! Que Yahweh accomplisse les paroles que tu as prophétisées, et qu'il fasse revenir de Babylone dans ce lieu-ci les ustensiles de la maison de Yahweh, et tous les captifs de Babylone~!
\VS{7}Toutefois, écoute maintenant cette parole que je prononce, à tes oreilles et aux oreilles de tout le peuple~:
\VS{8}Les prophètes qui ont été avant moi et avant toi, dès les temps anciens, ont prophétisé contre plusieurs pays et de grands royaumes, la guerre, le malheur et la peste~;
\VS{9}Le prophète qui aura prophétisé la paix, quand la parole de ce prophète sera accomplie, ce prophète-là sera reconnu pour avoir été véritablement envoyé par Yahweh.
\VS{10}Alors Hanania, le prophète, prit le joug de dessus le cou de Jérémie, le prophète, et le rompit.
\VS{11}Puis Hanania parla aux yeux de tout le peuple, en disant~: Ainsi parle Yahweh~: C'est ainsi que dans deux années, je romprai le joug de Nebucadnetsar, roi de Babylone, de dessus le cou de toutes les nations. Et Jérémie, le prophète, alla au loin par la route.
\VS{12}Mais la parole de Yahweh fut adressée à Jérémie, après que Hanania, le prophète, eut rompu le joug de dessus le cou de Jérémie, le prophète, en disant~:
\VS{13}Va, et parle à Hanania, en disant~: Ainsi parle Yahweh~: Tu as rompu les jougs de bois, et tu auras à la place un joug de fer.
\VS{14}Car ainsi parle Yahweh des armées, le Dieu d'Israël~: Je mets un joug de fer sur le cou de toutes ces nations, afin qu'elles servent Nebucadnetsar, roi de Babylone, et elles le serviront~; je lui donne aussi les bêtes des champs\FTNT{De. 28:48.}.
\VS{15}Puis Jérémie, le prophète, dit à Hanania, le prophète~: Ecoute maintenant, ô Hanania~! Yahweh ne t'a pas envoyé, et tu as fait que ce peuple se confie au mensonge\FTNT{Ez. 13:3-9.}.
\VS{16}C'est pourquoi ainsi parle Yahweh~: Voici, je te chasse de la face de la terre~; et tu mourras cette année~; car tu as parlé de révolte contre Yahweh.
\VS{17}Et Hanania, le prophète, mourut cette année-là, dans le septième mois.
\Chap{29}
\TextTitle{Message à l'attention des Juifs captifs à Babylone}
\VerseOne{}Or ce sont ici les paroles de la lettre que Jérémie, le prophète, envoya de Jérusalem au reste des anciens en captivité, aux prêtres et aux prophètes, et à tout le peuple, que Nebucadnetsar avait transportés de Jérusalem à Babylone,
\VS{2}après que le roi Jéconia fut sorti de Jérusalem, avec la reine, et les eunuques, et les chefs de Juda et de Jérusalem, et les charpentiers et les serruriers\FTNT{2 R. 24:12.}.
\VS{3}C'est par la main d'Eleasa, fils de Schaphan, et Guemaria, fils de Hilkija, que Sédécias, roi de Juda, l'envoya à Babylone vers Nebucadnetsar, roi de Babylone. La lettre disait~:
\VS{4}Ainsi parle Yahweh des armées, le Dieu d'Israël, à tous les captifs que j'ai fait transporter de Jérusalem à Babylone.
\VS{5}Bâtissez des maisons, et habitez-les~; plantez des jardins, et mangez-en les fruits.
\VS{6}Prenez des femmes, et engendrez des fils et des filles~; prenez aussi des femmes pour vos fils, et donnez vos filles à des hommes, afin qu'elles enfantent des fils et des filles~; multipliez-vous là, et ne diminuez pas.
\VS{7}Et cherchez la paix de la ville où je vous ai transporté, et priez Yahweh pour elle~; parce que dans sa paix vous aurez la paix.
\VS{8}Car ainsi parle Yahweh des armées, le Dieu d'Israël~: Que vos prophètes qui sont au milieu de vous, et vos devins, ne vous séduisent pas, et n'écoutez pas vos songes que vous songez\FTNT{Tous les songes ne viennent pas toujours du Seigneur. Les visions et les songes doivent être en accord avec la Parole de Dieu.}.
\VS{9}Parce qu'ils vous prophétisent faussement en mon Nom. Je ne les ai pas envoyés, dit Yahweh.
\VS{10}Car ainsi parle Yahweh~: Lorsque les soixante-dix ans seront accomplis pour Babylone, je vous visiterai, et j'accomplirai ma bonne parole à votre égard, pour vous faire revenir dans ce lieu.
\VS{11}Car je sais que les pensées que j'ai pour vous, dit Yahweh, sont des pensées de paix et non pas d'adversité, pour vous donner une fin telle que vous espérez\FTNT{Jos. 1:8.}.
\VS{12}Alors vous m'invoquerez, et vous partirez~; vous me prierez, et je vous exaucerai\FTNT{Os. 5:15.}.
\VS{13}Vous me chercherez, et vous me trouverez, après que vous m'aurez recherché de tout votre cœur\FTNT{Mt. 7:7.}.
\VS{14}Car je me laisserai trouver par vous, dit Yahweh, je ramènerai vos captifs~; et je vous rassemblerai d'entre toutes les nations et de tous les lieux où je vous ai chassés, dit Yahweh, et je vous ramènerai dans le lieu d'où je vous ai transportés.
\VS{15}Cependant si vous dites~: Yahweh nous a suscité des prophètes à Babylone~!
\VS{16} A cause de cela, ainsi parle Yahweh sur le roi qui est assis sur le trône de David, sur tout le peuple qui habite dans cette ville, sur vos frères qui ne sont pas allés avec vous en captivité~;
\VS{17}ainsi parle Yahweh des armées~: Voici, je vais envoyer sur eux l'épée, la famine, et la peste, et je les ferai devenir comme des figues affreuses qui ne peuvent être mangées à cause de leur mauvaise qualité.
\VS{18}Et je les poursuivrai par l'épée, par la famine et par la peste, je les abandonnerai pour être agités par tous les royaumes de la terre, et pour être une malédiction, un étonnement, une moquerie et un opprobre parmi toutes les nations où je les chasserai\FTNT{De. 28:25-37.},
\VS{19}parce qu'ils n'ont pas écouté mes paroles, dit Yahweh, eux à qui j'ai envoyé mes serviteurs, les prophètes, en me levant dès le matin~; et ils n'ont pas écouté, dit Yahweh.
\VS{20}Vous tous donc, écoutez la parole de Yahweh, vous les captifs que j'ai envoyés de Jérusalem à Babylone~!
\VS{21}Ainsi parle Yahweh des armées, le Dieu d'Israël sur Achab, fils de Kolaja, et sur Sédécias, fils de Maaséja, qui vous prophétisent faussement en mon Nom~: Voici, je vais les livrer entre les mains de Nebucadnetsar, roi de Babylone~; et il les frappera sous vos yeux.
\VS{22}Et on se servira d'eux comme une formule de malédiction, parmi tous les captifs de Juda qui sont à Babylone, en disant~: Que Yahweh te mette dans un tel état, comme Sédécias et comme Achab, que le roi de Babylone a fait rôtir au feu~!
\VS{23}Parce qu'ils ont commis des impuretés en Israël, parce qu'ils ont commis l'adultère avec les femmes de leurs prochains, et qu'ils ont dit en mon Nom des paroles fausses, alors que je ne leur avais pas commandées. Je le sais, et j'en suis témoin, dit Yahweh.
\VS{24}Parle aussi à Schemaeja, Néchélamite, en disant~:
\VS{25}Ainsi parle Yahweh des armées, le Dieu d'Israël~: Tu as envoyé en ton nom une lettre à tout le peuple de Jérusalem, à Sophonie, fils de Maaséja, le prêtre, et à tous les prêtres, en disant~:
\VS{26}Yahweh t'a établi prêtre à la place de Jehojada, le prêtre, afin qu'il y ait dans la maison de Yahweh des inspecteurs pour surveiller tout homme qui est fou et se donne pour prophète, et afin que tu le mettes en prison et dans les fers.
\VS{27}Et maintenant, pourquoi n'as-tu pas réprimé Jérémie d'Anathoth, qui prophétise parmi vous,
\VS{28}car à cause de cela il nous a envoyé dire à Babylone~: La captivité sera longue~; bâtissez des maisons, et habitez-les~; plantez des jardins, et mangez-en les fruits~!
\VS{29}Or Sophonie, le prêtre, lut cette lettre aux oreilles de Jérémie, le prophète.
\VS{30}C'est pourquoi la parole de Yahweh fut adressée à Jérémie, en disant~:
\VS{31}Envoie dire à tous les captifs~: Ainsi parle Yahweh sur Schemaeja, Néchélamite~: Parce que Schemaeja vous prophétise, quoique que je ne l'aie envoyé, et qu'il vous a fait vous confier dans le mensonge,
\VS{32}à cause de cela, dit Yahweh~: Je vais punir Schemaeja, Néchélamite, et sa postérité~; il n'y aura personne de sa race qui habite au milieu de ce peuple, et il ne verra pas le bien que je ferai à mon peuple, dit Yahweh~; car il a parlé de révolte contre Yahweh.
\Chap{30}
\TextTitle{Le jour de Yahweh}
\VerseOne{}La parole qui fut adressée à Jérémie de la part de Yahweh, en disant~:
\VS{2}Ainsi parle Yahweh, le Dieu d'Israël~: Ecris pour toi dans un livre toutes les paroles que je t'ai dites.
\VS{3}Car voici, les jours viennent, dit Yahweh, où je ramènerai les captifs de mon peuple d'Israël et de Juda, dit Yahweh~; je les ramènerai dans le pays que j'ai donné à leurs pères, et ils le posséderont.
\VS{4}Ce sont ici les paroles que Yahweh a prononcées sur Israël et Juda.
\VS{5}Ainsi parle Yahweh~: Nous entendons des cris d'effroi et de terreur, il n'y a pas de paix.
\VS{6}Informez-vous, je vous prie et voyez si un mâle enfante~! Pourquoi vois-je les hommes les mains sur leurs reins, comme une femme qui enfante~? Pourquoi tous les visages sont-ils devenus pâles~?
\VS{7}Malheur~! Que ce jour est grand~; il n'y en a pas eu de semblable. Il sera un temps de détresse pour Jacob~; mais il en sera pourtant délivré\FTNT{Joë. 2:11~; So. 1:15~; Da. 12:1~; Mt. 24:21.}.
\VS{8}Et il arrivera en ce jour-là, dit Yahweh des armées, que je briserai son joug de dessus ton cou, je romprai tes liens, et les étrangers ne t'asserviront plus.
\VS{9}Mais ils serviront Yahweh, leur Dieu, et David, leur roi, que je leur susciterai\FTNT{Ez. 34:23-24.}.
\VS{10}Toi donc, mon serviteur Jacob, ne crains pas, dit Yahweh, et ne t'épouvante pas, ô Israël~! Car, voici, je te délivrerai du pays éloigné, et ta postérité du pays de leur captivité~; et Jacob reviendra, il sera en repos et sera en paix, et il n'y aura personne qui lui fasse peur\FTNT{Es. 41:13.}.
\VS{11}Car je suis avec toi, dit Yahweh, pour te délivrer~; et même je consumerai entièrement toutes les nations parmi lesquelles je t'ai dispersé, mais quand à toi, je ne te consumerai pas entièrement~; je te châtierai avec équité, je ne te tiens pas entièrement pour innocent\FTNT{Es. 27:7-8.}.
\VS{12}Ainsi parle Yahweh~: Ta blessure est incurable, ta plaie est très douloureuse\FTNT{Mi. 1:9~; 2 Ch. 36:16.}.
\VS{13}Il n'y a personne qui défende ta cause, pour panser ta plaie~; il n'y a pour toi aucun remède, aucun moyen de guérison.
\VS{14}Tous tes amoureux t'oublient, ils ne te recherchent pas~; car je t'ai frappée d'une plaie d'ennemi, d'un châtiment d'homme cruel, à cause de la multitude de tes iniquités, tes péchés se sont renforcés\FTNT{La. 1:2.}.
\VS{15}Pourquoi cries-tu à cause de ta plaie~? Ta douleur est hors d'espérance~; je t'ai fait ces choses à cause de la grandeur de tes iniquités, du grand nombre de tes péchés.
\VS{16}Néanmoins tous ceux qui te dévorent seront dévorés, et tous ceux qui te mettent dans la détresse, iront en captivité~; ceux qui te dépouillent seront dépouillés, et je livrerai au pillage tous ceux qui te pillent\FTNT{Es. 41:11~; Ab. 15.}.
\VS{17}Même, je guérirai tes plaies, et je guérirai tes blessures, dit Yahweh. Car ils t'appellent la repoussée, cette Sion que personne ne recherche.
\TextTitle{Israël délivré par Yahweh}
\VS{18}Ainsi parle Yahweh~: Voici, je ramène les captifs des tentes de Jacob, j'ai compassion de ses demeures~; la ville sera rebâtie sur le monceau de ses ruines et le palais sera rétabli comme il était.
\VS{19}Et il en sortira des remerciements et des cris de joie~; et je les multiplierai, et ils ne diminueront pas~; je les honorerai, et ils ne seront pas amoindris.
\VS{20}Et ses enfants seront comme autrefois, son assemblée sera affermie devant moi, et je punirai tous ceux qui l'oppriment.
\VS{21}Et son chef sera tiré de son sein, son dominateur sortira du milieu de lui~; je le ferai approcher, et il viendra vers moi~; car qui disposerait son cœur pour venir vers moi~? dit Yahweh.
\VS{22}Et vous serez mon peuple, et je serai votre Dieu.
\VS{23}Voici, la tempête de Yahweh, la fureur éclate, un tourbillon qui s'entasse~; il tombera sur la tête des méchants. 
\VS{24}L'ardeur de la colère de Yahweh ne se détournera pas, jusqu'à ce qu'il ait exécuté, accompli les desseins de son cœur~; vous le comprendrez dans les derniers jours.
\Chap{31}
\TextTitle{Communion retrouvée~: La paix et la joie}
\VerseOne{}En ce temps-là, dit Yahweh, je serai le Dieu de toutes les familles d'Israël, et ils seront mon peuple.
\VS{2}Ainsi parle Yahweh~: Le peuple survivant à l'épée a trouvé grâce dans le désert~; Israël marche vers son lieu de repos.
\VS{3}De loin Yahweh m'est apparu, et m'a dit~: Je t'aime d'un amour éternel, c'est pourquoi j'ai prolongé ma bonté envers toi.
\VS{4}Je te rétablirai encore, et tu seras rétablie, ô vierge d'Israël~! Tu te pareras encore de tes tambours, et tu sortiras au milieu des danses joyeuses.
\VS{5}Tu planteras encore des vignes sur les montagnes de Samarie~; les vignerons planteront et recueilleront les fruits pour leur usage\FTNT{Es. 65:21.}.
\VS{6}Car il y a un jour où les gardes crieront sur la montagne d'Ephraïm~: Levez-vous, et montons à Sion, vers Yahweh, notre Dieu~!
\VS{7}Car ainsi parle Yahweh~: Réjouissez-vous avec chant de triomphe, et avec allégresse à cause de Jacob, et égayez-vous à cause du chef des nations~! Faites-le entendre, chantez des louanges, et dites~: Yahweh, délivre ton peuple, le reste d'Israël~!
\VS{8}Voici, je vais les faire venir du pays du nord, et je les rassemblerai des extrémités de la terre~; l'aveugle et le boiteux, la femme enceinte et celle qui enfante seront ensemble parmi eux~; une grande assemblée qui reviendra ici.
\VS{9}Ils y seront allés en pleurant, mais je les ferai retourner avec des supplications, et je les conduirai aux torrents d'eaux, et par un droit chemin, où ils ne broncheront pas~; car j'ai été un père pour Israël, et Ephraïm est mon premier-né\FTNT{Ex. 4:22.}.
\VS{10}Nations, écoutez la parole de Yahweh, et annoncez-la aux îles éloignées~! Dites~: Celui qui a dispersé Israël le rassemblera, et il le gardera comme un berger garde son troupeau.
\VS{11}Car Yahweh rachète Jacob, et le retire de la main d'un ennemi plus fort que lui.
\VS{12}Ils viendront donc, et se réjouiront avec des chants de triomphe sur les hauteurs de Sion~; ils afflueront vers les biens de Yahweh, le blé, le vin, l'huile, et le fruit du gros et du menu bétail~; et leur âme sera comme un jardin arrosé, et ils ne seront plus dans la souffrance\FTNT{Es. 61:11.}.
\VS{13}Alors la vierge se réjouira à la danse, les jeunes hommes et les anciens ensemble~; je changerai leur deuil en joie, et je les consolerai~; et je les réjouirai en les délivrant de leur douleur.
\VS{14}Je rassasierai aussi de graisse l'âme des prêtres, et mon peuple sera rassasié de mes biens, dit Yahweh.
\VS{15}Ainsi parle Yahweh~: On entend des cris à Rama, des lamentations, des larmes amères~; Rachel pleure ses fils~; elle refuse d'être consolée sur ses fils, car ils ne sont plus\FTNT{Mt. 2:17-18.}.
\VS{16}Ainsi parle Yahweh~: Retiens ta voix de pleurer, et tes yeux de verser des larmes, car ton œuvre aura son salaire, dit Yahweh~; et ils reviendront des terres de l'ennemi.
\VS{17}Et il y a de l'espérance pour tes derniers jours, dit Yahweh~; et tes fils reviendront dans leur territoire.
\VS{18}J'ai très bien entendu Ephraïm se plaignant, et disant~: Tu m'as châtié, et j'ai été châtié comme un veau qui n'est pas dompté. Fais-moi revenir et je reviendrai, car tu es Yahweh, mon Dieu\FTNT{Ps. 119:67-71.}.
\VS{19}Certes, après m'être détourné, je me repens~; et après avoir reconnu mes fautes, je frappe sur ma cuisse~; je suis honteux et confus, car je porte l'opprobre de ma jeunesse\FTNT{Ez. 21:17.}.
\VS{20}Ephraïm est-il donc pour moi un cher fils, un fils qui fait mes délices~? Car plus je parle de lui, plus encore son souvenir est en moi~; aussi mes entrailles sont émues en sa faveur~: J'aurai certainement pitié de lui, dit Yahweh\FTNT{Es. 5:7.}.
\VS{21}Dresse-toi des signes sur les chemins, place des poteaux, prends garde à la route, au chemin par lequel tu es venue… Reviens, vierge d'Israël, reviens dans tes villes~!
\VS{22}Jusqu'à quand seras-tu errante, fille rebelle~? Car Yahweh crée une chose nouvelle sur la terre~: La femme entourera l'homme.
\VS{23}Ainsi parle Yahweh des armées, le Dieu d'Israël~: On dira encore cette parole-ci dans le pays de Juda et dans ses villes, quand j'aurai ramené leurs captifs~: Que Yahweh te bénisse, ô agréable demeure de la justice, montagne de sainteté~!
\VS{24}Juda et toutes ses villes ensemble, les laboureurs et ceux qui conduisent les troupeaux, y habiteront.
\VS{25}Car j'abreuverai l'âme épuisée par le travail, et je remplirai toute âme languissante.
\VS{26}C'est pourquoi je me suis réveillé, et j'ai regardé~; mon sommeil m'avait été agréable.
\TextTitle{Promesse d'une nouvelle alliance}
\VS{27}Voici, les jours viennent, dit Yahweh, où j'ensemencerai la maison d'Israël et la maison de Juda d'une semence d'hommes et d'une semence de bêtes.
\VS{28}Et il arrivera que comme j'ai veillé sur eux pour arracher et démolir, pour détruire, pour perdre et pour faire du mal~; ainsi je veillerai sur eux pour bâtir et pour planter, dit Yahweh.
\VS{29}En ces jours-là, on ne dira plus~: Les pères ont mangé des raisins verts, et les dents des fils en ont été agacées\FTNT{Ez. 18:2-3.}.
\VS{30}Mais chacun mourra pour son iniquité~; tout homme qui mangera des raisins verts, ses dents en seront agacées.
\VS{31}Voici, les jours viennent, dit Yahweh, où je traiterai une nouvelle alliance\FTNT{Il s'agit de l'alliance du sang que Jésus, notre Messie, est venu inaugurer en prenant sur lui tous nos péchés et en mourant sur la croix à notre place (Mt. 26:27-29~; Hé. 8:7-13).} avec la maison d'Israël et avec la maison de Juda,
\VS{32}non comme l'alliance que je traitai avec leurs pères, le jour où je les pris par la main, pour les faire sortir du pays d'Egypte, mon alliance qu'ils ont violée~; alors que j'avais été pour eux un mari, dit Yahweh.
\VS{33}Car c'est ici l'alliance que je traiterai avec la maison d'Israël, après ces jours-là, dit Yahweh, je mettrai ma loi au-dedans d'eux, je l'écrirai dans leur cœur~; et je serai leur Dieu, et ils seront mon peuple.
\VS{34}Aucun homme parmi eux n'enseignera plus son prochain, ni personne son frère, en disant~: Connaissez Yahweh~! Car tous me connaîtront, depuis le plus petit jusqu'au plus grand, dit Yahweh~; parce que je pardonnerai leur iniquité, et que je ne me souviendrai plus de leur péché\FTNT{Es. 54:13~; Ha. 2:14~; Jn. 6:45.}.
\VS{35}Ainsi parle Yahweh, qui a donné le soleil pour être la lumière du jour, qui a réglé la lune et les étoiles pour être la lumière de la nuit, qui remue la mer, et qui fait gronder ses flots, lui dont le nom est Yahweh des armées\FTNT{Ge. 1:16~; Es. 51:15.}~:
\VS{36}Si ces lois viennent à cesser devant moi, dit Yahweh, la race d'Israël aussi cessera d'être à jamais une nation devant moi.
\VS{37}Ainsi parle Yahweh~: Si les cieux en haut peuvent être mesurés, si les fondements de la terre en bas peuvent être sondés, alors je rejetterai toute la race d'Israël, à cause de toutes les choses qu'ils ont faites, dit Yahweh.
\VS{38}Voici, les jours viennent, dit Yahweh, où cette ville sera rebâtie à Yahweh, depuis la tour de Hananeel, jusqu'à la porte de l'angle\FTNT{Za. 14:10~; Né. 3:1~; 2 Ch. 26:9.}.
\VS{39}Le cordeau à mesurer sera encore tiré vis-à-vis d'elle, sur la colline de Gareb, et tournera vers Goath.
\VS{40}Et toute la vallée des cadavres et des cendres, et tous les champs jusqu'au torrent de Cédron, jusqu'à l'angle de la porte des chevaux à l'orient, seront consacrés à Yahweh, et ne seront plus jamais arrachés ni détruits.
\Chap{32}
\TextTitle{Le champ de Hanameel~: La pérennité d'Israël}
\VerseOne{}La parole qui fut adressée à Jérémie de la part de Yahweh, la dixième année de Sédécias, roi de Juda. C'était la dix-huitième année de Nebucadnetsar.
\VS{2}Or l'armée du roi de Babylone assiégeait alors Jérusalem~; et Jérémie le prophète était enfermé dans la cour de la prison, qui était dans la maison du roi de Juda~;
\VS{3}car Sédécias, roi de Juda, l'avait fait enfermer, et lui avait dit~: Pourquoi prophétises-tu, en disant~: Ainsi parle Yahweh~: Voici, je vais livrer cette ville entre les mains du roi de Babylone, et il la prendra~;
\VS{4}et Sédécias, roi de Juda, n'échappera pas aux mains des Chaldéens. Mais il sera livré entre les mains du roi de Babylone, et lui parlera bouche à bouche, et ses yeux verront les yeux de ce roi.
\VS{5}Il emmènera Sédécias à Babylone, qui y demeurera jusqu'à ce que je le visite, dit Yahweh~; si vous combattez contre les Chaldéens, vous ne prospérerez pas.
\VS{6}Jérémie donc dit~: La parole de Yahweh m'a été adressée, en disant~:
\VS{7}Voici Hanameel, fils de Schallum, ton oncle, qui vient vers toi pour te dire~: Achète mon champ qui est à Anathoth, car tu as le droit de rachat pour l'acquérir\FTNT{Lé. 25:48~; Ru. 3:12.}.
\VS{8}Hanameel donc, fils de mon oncle, vint à moi, selon la parole de Yahweh, dans la cour de la prison, et me dit~: Achète, je te prie, mon champ, qui est à Anathoth, dans le pays de Benjamin, car tu as le droit d'héritage et de rachat, achète-le~! Et je connus alors que c'était la parole de Yahweh.
\VS{9}Ainsi j'achetai de Hanameel, fils de mon oncle, le champ qui est à Anathoth, et je lui pesai l'argent, qui fut dix-sept sicles d'argent.
\VS{10}Puis j'écrivis le contrat, que je cachetai, je pris des témoins après avoir pesé l'argent sur la balance.
\VS{11}Et je pris le contrat d'acquisition, celui qui était cacheté, selon les ordonnances et les statuts, et celui qui était ouvert.
\VS{12}Et je remis le contrat d'acquisition à Baruc, fils de Nérija, fils de Machséja, sous les yeux de Hanameel, fils de mon oncle, des témoins qui avaient signé le contrat d'acquisition, et sous les yeux de tous les juifs qui étaient assis dans la cour de la prison.
\VS{13}Puis je donnai sous leurs yeux cet ordre à Baruc, en disant~:
\VS{14}Ainsi parle Yahweh des armées, le Dieu d'Israël~: Prends ces contrats-ci, à savoir, ce contrat d'acquisition, celui qui est scellé, et celui qui est ouvert, et mets-les dans un vase de terre, afin qu'ils se conservent longtemps.
\VS{15}Car ainsi parle Yahweh des armées, le Dieu d'Israël~: On achètera encore des maisons, des champs et des vignes, dans ce pays.
\TextTitle{Promesse du retour des Juifs en Israël}
\VS{16}Et après que j'eus donné à Baruc, fils de Nérija, le contrat d'acquisition, je fis cette prière à Yahweh, en disant~:
\VS{17}Ah~! Ah~! Seigneur Yahweh, voici, tu as fait les cieux et la terre par ta grande puissance et par ton bras étendu~: Aucune chose n'est étonnante de ta part.
\VS{18}Tu fais miséricorde jusqu'à la millième génération, et tu punis l'iniquité des pères dans le sein de leurs fils après eux\FTNT{Ex. 34:7~; Es. 65:7~; Ps. 79:12.}. Tu es le Dieu, le Grand, le Puissant, dont le nom est Yahweh des armées.
\VS{19}Tu es grand en conseil et puissant en actions~; tes yeux sont ouverts sur toutes les voies des fils des hommes, pour rendre à chacun selon ses voies, et selon le fruit de ses œuvres.
\VS{20}Tu as fait dans le pays d'Egypte des miracles et des prodiges qui sont connus jusqu'à ce jour, et en Israël et parmi les hommes, tu t'es fait un nom tel qu'il est aujourd'hui.
\VS{21}Car tu as fait sortir du pays d'Egypte ton peuple d'Israël, avec des miracles et des prodiges, et avec une main forte, et avec un bras étendu, et en répandant partout une grande terreur~;
\VS{22}Et tu leur as donné ce pays, que tu avais juré à leurs pères de leur donner, pays où coulent le lait et le miel.
\VS{23}Et ils y sont entrés, ils l'ont possédé~; mais ils n'ont pas obéi à ta voix, et n'ont pas marché dans ta loi, et n'ont pas fait tout ce que tu leur avais ordonné de faire. C'est pourquoi tu as fait arriver sur eux tout ce mal-ci~!
\VS{24}Voilà, les terrasses sont élevées, on est venu contre la ville pour la prendre, et à cause de l'épée, de la famine, et de la peste, la ville est livrée entre la main des Chaldéens qui combattent contre elle~; et ce que tu as dit est arrivé, et voici, tu le vois.
\VS{25}Et cependant, Seigneur Yahweh~! Tu m'as dit~: Achète-toi ce champ à prix d'argent, et prends-en des témoins, quoique la ville soit livrée entre les mains des Chaldéens.
\VS{26}Mais la parole de Yahweh fut adressée à Jérémie, en disant~:
\VS{27}Voici, je suis Yahweh, le Dieu de toute chair. Y a-t-il quelque chose d'étonnant de ma part~?
\VS{28}C'est pourquoi ainsi parle Yahweh~: Voici, je vais livrer cette ville entre les mains des Chaldéens, et entre les mains de Nebucadnetsar, roi de Babylone, qui la prendra.
\VS{29}Et les Chaldéens qui combattent contre cette ville, y entreront, et mettront le feu à cette ville, et la brûleront, avec les maisons sur les toits desquelles on a brûlé de l'encens à Baal, et où l'on a fait des libations à d'autres dieux pour m'irriter.
\VS{30}Car les fils d'Israël et les fils de Juda n'ont fait, dès leur jeunesse, que ce qui est mal à mes yeux~; les fils d'Israël n'ont fait que m'irriter par les œuvres de leurs mains, dit Yahweh.
\VS{31}Car cette ville a été portée à provoquer ma colère et ma fureur, depuis le jour qu'ils l'ont bâtie, jusqu'à ce jour, afin que je l'ôte de devant ma face~;
\VS{32}à cause de tout le mal que les enfants d'Israël et les fils de Juda ont fait pour m'irriter, eux, leurs rois, leurs chefs, leurs prêtres et leurs prophètes, les hommes de Juda et les habitants de Jérusalem.
\VS{33}Ils m'ont tourné le dos, et non la face~; je les ai enseignés, je les ai enseignés dès le matin, mais ils n'ont pas écouté pour recevoir l'instruction.
\VS{34}Mais ils ont mis leurs abominations dans la maison sur laquelle mon Nom est invoqué, pour la souiller.
\VS{35}Et ils ont bâti les hauts lieux de Baal, qui sont dans la vallée de Ben-Hinnom, pour faire passer par le feu leurs fils et leurs filles à Moloc\FTNT{Voir commentaire en Lé. 20:2.}~; ce que je ne leur avais pas ordonné, et il ne m'était pas monté à la pensée qu'ils feraient cette abomination pour faire pécher Juda.
\VS{36}Et maintenant, à cause de cela Yahweh, le Dieu d'Israël, ainsi parle sur cette ville dont vous dites qu'elle est livrée entre les mains du roi de Babylone, à cause que l'épée, la famine, et la peste sont en elle~:
\VS{37}Voici, je vais les rassembler de tous les pays où je les ai chassés, dans ma colère, dans ma fureur et dans mon grand courroux~; et je les ramènerai dans ce lieu-ci, et je les y ferai habiter en sécurité.
\VS{38}Et Ils seront mon peuple, et je serai leur Dieu.
\VS{39}Et je leur donnerai un même cœur et une même voie, afin qu'ils me craignent à toujours, pour leur bien et celui de leurs fils après eux.
\VS{40}Et je traiterai avec eux une alliance éternelle, à savoir, que je ne me détournerai plus d'eux pour leur faire du bien~; et je mettrai ma crainte dans leur cœur, afin qu'ils ne se détournent pas de moi\FTNT{Es. 54:10.}.
\VS{41}Et je me réjouirai à leur faire du bien, et je les planterai dans ce pays-ci solidement, de tout mon cœur et de toute mon âme.
\VS{42}Car ainsi parle Yahweh~: Comme j'ai fait venir tout ce grand mal sur ce peuple, ainsi je ferai venir sur eux tout le bien que je prononce en leur faveur.
\VS{43}Et on achètera des champs dans ce pays, dont vous dites que ce n'est que désolation, sans hommes ni bêtes, et qui est livré entre les mains des Chaldéens.
\VS{44}On achètera, dis-je, des champs à prix d'argent, et on en écrira les contrats, et on les cachettera. On en prendra des témoins dans le pays de Benjamin, aux environs de Jérusalem, dans les villes de Juda, tant dans les villes des montagnes, que dans les villes de la plaine et dans les villes du midi. Car je ramènerai leurs captifs, dit Yahweh.
\Chap{33}
\TextTitle{Jésus, le Germe appelé à régner\FTNTT{Voir 2 S. 7:8-16.}}
\VerseOne{}Et la parole de Yahweh fut adressée une seconde fois à Jérémie, quand il était encore enfermé dans la cour de la prison, en disant~:
\VS{2}Ainsi parle Yahweh, qui fait ces choses, Yahweh qui les forme et les établit, lui dont le nom est Yahweh~:
\VS{3}Crie vers moi\FTNT{Yahweh, qui demandait qu'on l'invoque, n'est autre que Jésus-Christ, notre Seigneur (Joë. 2:32~; 1 Co. 1:2~; Ro. 10:13).}, je te répondrai et je t'annoncerai des choses grandes, des choses cachées, que tu ne connais pas.
\VS{4}Car ainsi parle Yahweh, le Dieu d'Israël, touchant les maisons de cette ville-ci et les maisons des rois de Juda~; elles seront abattues par les terrasses et par l'épée.
\VS{5}Ils sont venus pour combattre contre les Chaldéens, mais ça a été pour remplir leurs maisons des cadavres des hommes que j'ai frappé dans ma colère et dans ma fureur, et parce que j'ai caché ma face de cette ville à cause de toute leur méchanceté.
\VS{6}Voici, je vais lui donner la santé et la guérison, je les guérirai, et je leur ferai découvrir une abondance de paix et de fidélité\FTNT{Ap. 22:1-2.}.
\VS{7}Et je ramènerai les captifs de Juda, et les captifs d'Israël, et je les rétablirai comme autrefois.
\VS{8}Et je les purifierai de toutes leurs iniquités, par lesquelles ils ont péché contre moi~; et je pardonnerai toutes leurs iniquités par lesquelles ils ont péché contre moi, et par lesquelles ils se sont révoltés contre moi\FTNT{Ez. 37:23.}.
\VS{9}Et cette ville sera pour moi un sujet de joie, de louange et de gloire, parmi toutes les nations de la terre qui entendront parler de tout le bien que je leur ferai. Et elles seront dans la crainte et trembleront, à cause de tout le bien et de toute la prospérité que je vais lui donner.
\VS{10}Ainsi parle Yahweh~: Dans ce lieu-ci dont vous dites~: Il est désert, il n'y a plus d'hommes, plus de bêtes dans les villes de Juda, et dans les rues de Jérusalem, qui sont désolées, privées d'hommes, d'habitants, de bêtes.
\VS{11}On y entendra encore les cris de joie et les cris d'allégresse, la voix de l'époux et la voix de l'épouse, et la voix de ceux qui disent~: Louez Yahweh des armées car Yahweh est bon, parce que sa miséricorde demeure à toujours, lorsqu'ils offriront des offrandes de reconnaissance dans la maison de Yahweh~; car je ramènerai les captifs de ce pays, et je les rétablirai comme autrefois, dit Yahweh.
\VS{12}Ainsi parle Yahweh des armées~: Dans ce lieu désert, où il n'y a ni hommes ni bêtes, et dans toutes ses villes, il y aura encore des demeures de bergers qui y feront reposer leurs troupeaux.
\VS{13}Dans les villes des montagnes, et dans les villes de la plaine, dans les villes du midi, dans le pays de Benjamin et aux environs de Jérusalem, et dans les villes de Juda~; les brebis passeront encore sous les mains de celui qui les compte, dit Yahweh.
\VS{14}Voici, les jours viennent, dit Yahweh, où j'accomplirai la bonne parole que j'ai prononcée sur la maison d'Israël et la maison de Juda.
\VS{15}En ces jours et en ce temps-là, je ferai germer à David le Germe de justice, qui exercera le jugement et la justice dans le pays.
\VS{16}En ces jours-là, Juda sera sauvé, Jérusalem habitera en sécurité~; et voici comment on l'appellera~: Yahweh notre justice.
\VS{17}Car ainsi parle Yahweh~: David ne manquera jamais d'un successeur assis sur le trône de la maison d'Israël.
\VS{18}Et d'entre les prêtres lévites, il ne manquera jamais d'y avoir devant moi d'homme offrant des holocaustes, brûlant de l'encens avec les offrandes, et faisant des sacrifices tous les jours.
\VS{19}La parole de Yahweh fut encore adressée à Jérémie, en disant~:
\VS{20}Ainsi parle Yahweh~: Si vous pouvez rompre mon alliance avec le jour et mon alliance avec la nuit, de sorte que le jour et la nuit ne soient plus en leur temps,
\VS{21}alors aussi mon alliance avec David, mon serviteur, sera rompue~; de sorte qu'il n'aura plus de fils régnant sur son trône~; ni de Lévites, ni de prêtres, faisant mon service.
\VS{22}Car comme on ne peut compter l'armée des cieux, ni mesurer le sable de la mer, ainsi je multiplierai la postérité de David mon serviteur, et les Lévites qui font mon service\FTNT{Ge. 2:1~; Ge. 15:5.}.
\VS{23}La parole de Yahweh fut encore adressée à Jérémie, en disant~:
\VS{24}N'as-tu pas vu ce que ce peuple a prononcé, disant~: Yahweh a rejeté les deux familles qu'il avait élues~? Et ils méprisent tellement mon peuple, que devant eux il ne sera plus une nation. 
\VS{25}Ainsi parle Yahweh~: Si je n'ai pas fait mon alliance avec le jour et la nuit, et si je n'ai pas établi les ordonnances des cieux et de la terre~;
\VS{26}aussi rejetterai-je la postérité de Jacob, et celle de David mon serviteur, pour ne plus prendre de sa postérité des gens qui dominent sur les descendants d'Abraham, d'Isaac et de Jacob~; car je ramènerai leurs captifs et j'aurai compassion d'eux.
\Chap{34}
\TextTitle{Désobéissance du peuple~: Jérusalem dévastée}
\VerseOne{}La parole qui fut adressée à Jérémie de la part de Yahweh, lorsque Nebucadnetsar, roi de Babylone, et toute son armée, et tous les royaumes de la terre, et tous les peuples qui étaient sous la puissance de sa main, combattaient contre Jérusalem, et contre toutes ses villes, en disant\FTNT{2 R. 25:1-2.}~:
\VS{2}Ainsi parle Yahweh, le Dieu d'Israël~: Va et parle à Sédécias, roi de Juda, et dis-lui~: Ainsi parle Yahweh~: Voici, je vais livrer cette ville entre les mains du roi de Babylone, et il la brûlera par le feu.
\VS{3}Et tu n'échapperas pas de sa main, car certainement tu seras pris et tu seras livré entre ses mains, et tes yeux verront les yeux du roi de Babylone, et il te parlera bouche à bouche, et tu iras à Babylone.
\VS{4}Toutefois écoute la parole de Yahweh, ô Sédécias, roi de Juda~! Ainsi parle Yahweh sur toi~: Tu ne mourras pas par l'épée~;
\VS{5}mais tu mourras en paix et on brûlera pour toi des parfums aromatiques, comme on en a brûlé pour tes pères, les rois précédents qui ont été avant toi~; et on te pleurera, en disant~: Hélas, Seigneur~! Car j'ai prononcé cette parole, dit Yahweh\FTNT{2 Ch. 16:14.}.
\VS{6}Jérémie, le prophète, dit toutes ces paroles à Sédécias, roi de Juda, à Jérusalem.
\VS{7}Et l'armée du roi de Babylone combattait contre Jérusalem et contre toutes les villes de Juda qui restaient, à savoir, contre Lakis et contre Azéka, car c'étaient les seules villes fortifiées qui restaient parmi les villes de Juda\FTNT{2 R. 18:13.}.
\TextTitle{Jérusalem deviendra une désolation à cause de la désobéissance}
\VS{8}La parole fut adressée à Jérémie de la part de Yahweh, après que le roi Sédécias eut traité une alliance avec tout le peuple de Jérusalem, pour proclamer la liberté,
\VS{9}afin que chacun renvoie libre son esclave et chacun sa servante, l'hébreu ou la femme de l'hébreu, et qu'aucun juif ne soit l'esclave de son frère.
\VS{10}Tous les chefs et tout le peuple, qui étaient entrés dans cette alliance, entendirent que chacun devait renvoyer libre son serviteur et chacun sa servante, sans plus les asservir~; ils obéirent et les renvoyèrent.
\VS{11}Mais ensuite, ils changèrent d'avis~; ils firent revenir leurs esclaves et leurs servantes, qu'ils avaient renvoyés libres, et les assujettirent pour être leurs esclaves et leurs servantes.
\VS{12}Et la parole de Yahweh fut adressée à Jérémie en disant~:
\VS{13}Ainsi parle Yahweh, le Dieu d'Israël~: J'ai traité une alliance avec vos pères, le jour où je les ai fait sortir du pays d'Egypte, de la maison de servitude, en disant~:
\VS{14}A la fin de la septième année, chacun renverra libre son frère hébreu qui aura été vendu~; il te servira six années, puis tu le renverras libre de chez toi. Mais vos pères ne m'ont pas écouté, ils n'ont pas prêté l'oreille\FTNT{Ex. 21:2~; Lé. 25:10-15~; De. 15:12.}.
\VS{15}Et vous vous étiez convertis aujourd'hui, et vous aviez fait ce qui était juste devant moi, en publiant la liberté chacun à son prochain, et vous aviez traité alliance en ma présence, dans la maison sur laquelle mon Nom est invoqué.
\VS{16}Mais vous êtes revenus en arrière, et vous avez souillé mon Nom~; vous avez fait revenir chacun ses esclaves et ses servantes, que vous aviez renvoyés libres, rendus à eux-mêmes, et vous les avez assujettis, afin qu'ils soient pour vous des serviteurs et des servantes.
\VS{17}C'est pourquoi ainsi parle Yahweh~: Vous ne m'avez pas obéi, en publiant la liberté chacun à son frère, et chacun à son prochain. Voici, je publie contre vous, dit Yahweh, la liberté contre vous à l'épée, à la peste, et à la famine~; et je vous livrerai pour être transportés par tous les royaumes de la terre.
\VS{18}Je livrerai les hommes qui ont transgressé mon alliance, et qui n'ont pas observé les paroles de l'alliance qu'ils avaient traitée devant moi, lorsqu'ils sont passés entre les morceaux du veau qu'ils ont coupé en deux~;
\VS{19}les chefs de Juda, et les chefs de Jérusalem, les eunuques, les prêtres, et tout le peuple du pays, qui sont passés au travers des morceaux du veau.
\VS{20}Je les livrerai, dis-je, entre les mains de leurs ennemis, entre les mains de ceux qui cherchent leur vie~; et leurs cadavres seront la pâture des oiseaux des cieux et des bêtes de la terre.
\VS{21}Je livrerai aussi Sédécias, roi de Juda, et les chefs de sa cour, entre les mains de leurs ennemis, entre les mains de ceux qui cherchent leur vie, entre les mains de l'armée du roi de Babylone, qui s'est retiré de devant vous.
\VS{22}Voici, je vais leur donner mes ordres, dit Yahweh, et je les ramènerai contre cette ville-ci~; ils combattront contre elle, la prendront et la brûleront au feu~; et je ferai des villes de Juda un désert sans habitants.
\Chap{35}
\TextTitle{L'obéissance des Récabites}
\VerseOne{}C'est ici la parole qui fut adressée à Jérémie de la part de Yahweh, au temps de Jojakim, fils de Josias, roi de Juda, en disant~:
\VS{2}Va à la maison des Récabites, et parle-leur~; fais les venir à la maison de Yahweh, dans l'une des chambres, et présente-leur du vin à boire\FTNT{2 S. 4:2~; 1 Ch. 2:55.}.
\VS{3}Je pris donc Jaazania, fils de Jérémie, fils de Habazinia, et ses frères, et tous ses fils, et toute la maison des Récabites,
\VS{4}et je les fis venir dans la maison de Yahweh, dans la chambre des fils de Hanan, fils de Jigdalia, homme de Dieu, qui était près de la chambre des chefs, au-dessus de la chambre de Maaséja, fils de Schallum, garde du seuil.
\VS{5}Et je mis devant les fils de la maison des Récabites des coupes pleines de vin, et des calices, et je leur dis~: Buvez du vin~!
\VS{6}Et ils répondirent~: Nous ne buvons pas de vin~; car Jonadab, fils de Récab, notre père, nous a donné cet ordre en disant~: Vous ne boirez jamais de vin, ni vous ni vos fils\FTNT{Lé. 10:9~; No. 6:2-4.}~;
\VS{7}vous ne bâtirez aucune maison, vous ne sèmerez aucune semence, vous ne planterez aucune vigne, et vous n'en aurez pas~; mais vous habiterez sous des tentes toute votre vie, afin que vous viviez longtemps sur la terre où vous êtes étrangers.
\VS{8}Nous avons donc obéi à la voix de Jonadab, fils de Récab, notre père dans toutes les choses qu'il nous a ordonnées, de sorte que nous n'avons pas bu de vin tous les jours de notre vie, ni nous, ni nos femmes, ni nos fils, ni nos filles.
\VS{9}Nous n'avons bâti aucune maison pour notre demeure, et nous n'avons eu ni vigne, ni champ, ni semence.
\VS{10}Mais nous avons habité sous des tentes, nous avons obéi, et nous avons fait selon toutes les choses que Jonadab, notre père, nous a ordonnées.
\VS{11}Mais il est arrivé que quand Nebucadnetsar, roi de Babylone, est monté au pays, nous avons dit~: Venez et entrons dans Jérusalem, pour fuir de devant l'armée des Chaldéens et de devant l'armée de Syrie. C'est ainsi que nous habitons à Jérusalem.
\VS{12}Alors la parole de Yahweh fut adressée à Jérémie, en disant~:
\VS{13}Ainsi parle Yahweh des armées, le Dieu d'Israël~: Va, et dis aux hommes de Juda, et aux habitants de Jérusalem~: Ne recevrez-vous pas d'instruction pour obéir à mes paroles~? Dit Yahweh.
\VS{14}Toutes les paroles de Jonadab, fils de Récab, qui a ordonné à ses fils de ne pas boire de vin, ont été observées, et ils n'en ont pas bu jusqu'à ce jour~; mais ils ont obéi au commandement de leur père~; mais moi, je vous ai parlé, je vous ai parlé dès le matin, et vous ne m'avez pas obéi.
\VS{15}Car je vous ai envoyé tous les prophètes, mes serviteurs, je les ai envoyés dès le matin, pour vous dire~: Revenez maintenant chacun de votre mauvaise voie, et amendez vos actions, et n'allez pas après d'autres dieux pour les servir, afin que vous demeureriez dans le pays que j'ai donné à vous et à vos pères. Mais vous n'avez pas prêté l'oreille, et vous ne m'avez pas écouté.
\VS{16}Parce que les fils de Jonadab, fils de Récab, ont observé le commandement que leur avait donné leur père, et que ce peuple ne m'écoute pas~;
\VS{17}à cause de cela, Yahweh le Dieu des armées, le Dieu d'Israël, parle ainsi~: Voici, je vais faire venir sur Juda et sur tous les habitants de Jérusalem, tout le mal que j'ai prononcé contre eux~; parce que je leur ai parlé, et ils n'ont pas écouté~; et que je les ai appelés, et ils n'ont pas répondu.
\VS{18}Et Jérémie dit à la maison des Récabites: Ainsi parle Yahweh des armées, le Dieu d'Israël~: Parce que vous avez obéi au commandement de Jonadab, votre père~; que vous avez gardé tous ses commandements, et que vous avez fait selon tout ce qu'il vous a ordonné\FTNT{Les Récabites furent bénis parce qu'ils obéirent aux commandements de leur père (Ep. 6:1-3)}.
\VS{19}C'est pourquoi, ainsi parle Yahweh des armées, le Dieu d'Israël~: Jonadab, fils de Récab, ne manquera jamais de descendants qui se tiennent debout devant moi.
\Chap{36}
\TextTitle{Le roi Jojakim brûle le manuscrit de Jérémie}
\VerseOne{}Or il arriva, dans la quatrième année de Jojakim, fils de Josias, roi de Juda, que cette parole fut adressée à Jérémie de la part de Yahweh, en disant~:
\VS{2}Prends-toi un rouleau de livre, et tu y écriras toutes les paroles que je t'ai dites contre Israël et contre Juda, et contre toutes les nations, depuis le jour où je t'ai parlé, c'est-à-dire, depuis les jours de Josias, jusqu'à ce jour.
\VS{3}Peut-être que la maison de Juda entendra tout le mal que je pense leur faire, afin que chaque homme se détourne de sa mauvaise voie, et que je leur pardonne leur iniquité, et leur péché.
\VS{4}Jérémie donc appela Baruc, fils de Nérija, et Baruc écrivit, sous la dictée de Jérémie, dans le rouleau de livre, toutes les paroles que Yahweh lui avait dites.
\VS{5}Puis Jérémie donna cet ordre à Baruc, en disant~: Je suis retenu et je ne peux pas entrer dans la maison de Yahweh.
\VS{6}Tu y entreras donc et tu liras dans le rouleau que tu as écrit sous ma dictée, toutes les paroles de Yahweh, aux oreilles du peuple dans la maison de Yahweh le jour du jeûne. Tu les liras, dis-je, aussi aux oreilles de tous ceux de Juda qui seront venus de leurs villes.
\VS{7}Peut-être que Yahweh écoutera leur supplication et qu'ils reviendront chacun de leur mauvaise voie~; car grande est la colère, la fureur que Yahweh a déclarée contre ce peuple.
\VS{8}Baruc donc, fils de Nérija, fit selon tout ce que lui avait ordonné Jérémie le prophète, lisant dans le rouleau les paroles de Yahweh, dans la maison de Yahweh.
\VS{9}Or il arriva dans la cinquième année de Jojakim, fils de Josias, roi de Juda, le neuvième mois, qu'on publia le jeûne devant Yahweh à tout le peuple de Jérusalem et à tout le peuple venu des villes de Juda à Jérusalem.
\VS{10}Et Baruc lut dans le livre les paroles de Jérémie, aux oreilles de tout le peuple, dans la maison de Yahweh, dans la chambre de Guemaria, fils de Schaphan, le secrétaire, dans le parvis supérieur, à l'entrée de la porte neuve de la maison de Yahweh.
\VS{11}Et quand Michée fils de Guemaria, fils de Schaphan, eut entendu toutes les paroles de Yahweh contenues dans le livre~;
\VS{12}il descendit dans la maison du roi, vers la chambre du secrétaire, et voici tous les chefs y étaient assis, à savoir, Elischama le secrétaire, Delaja, fils de Schemaeja, Elnathan, fils de Acbor, et Guemaria, fils de Schaphan, et Sédécias, fils de Hanania, et tous les chefs.
\VS{13}Et Michée leur rapporta toutes les paroles qu'il avait entendues, quand Baruc lisait dans le livre, aux oreilles du peuple.
\VS{14}C'est pourquoi tous les chefs envoyèrent vers Baruc, Jehudi, fils de Nethania, fils de Schélémia, fils de Cuschi, pour lui dire~: Prends en ta main le rouleau que tu as lu aux oreilles du peuple, et viens ici~! Baruc donc, fils de Nérija, prit le rouleau en sa main, et vint vers eux.
\VS{15}Et ils lui dirent~: Assieds-toi maintenant, et lis-le à nos oreilles~; et Baruc le lut à leurs oreilles.
\VS{16}Et il arriva que sitôt qu'ils eurent entendu toutes les paroles, ils furent effrayés entre eux, et dirent à Baruc~: Nous ne manquerons pas de rapporter au roi toutes ces paroles.
\VS{17}Et ils interrogèrent Baruc, en disant~: Dis-nous comment tu as écrit toutes ces paroles sous sa dictée.
\VS{18}Et Baruc leur dit~: Il me dictait de sa bouche toutes ces paroles, et je les écrivais avec de l'encre dans le livre.
\VS{19}Alors les chefs dirent à Baruc~: Va, et cache-toi, ainsi que Jérémie, et que personne ne sache où vous serez.
\VS{20}Puis ils s'en allèrent vers le roi dans la cour, mais ils prirent soin de laisser le rouleau dans la chambre d'Elischama le secrétaire~; et ils racontèrent toutes ces paroles aux oreilles du roi.
\VS{21}Et le roi envoya Jehudi pour prendre le rouleau~; et quand Jehudi l'eut prit de la chambre d'Elischama le secrétaire, et il le lut aux oreilles du roi et de tous les chefs qui étaient autour de lui.
\VS{22}Or le roi était assis dans la maison d'hiver, au neuvième mois, et un brasier était allumé devant lui.
\VS{23}Et il arriva qu'aussitôt que Jehudi en eut lu trois ou quatre feuilles, le roi déchira le rouleau avec le canif du secrétaire, et le jeta au feu du brasier, jusqu'à ce que tout le rouleau fut consumé au feu du brasier.
\VS{24}Et ni le roi ni tous ses serviteurs qui entendirent toutes ces paroles, n'en furent pas effrayés, et ne déchirèrent pas leurs vêtements.
\VS{25}Toutefois Elnathan, et Delaja et Guemaria intercédèrent envers le roi, afin qu'il ne brûle pas le rouleau~; mais il ne les écouta pas.
\VS{26}Même le roi ordonna à Jerachmeel, fils de Hammélec, et à Seraja, fils d'Azriel, et à Schélémia, fils de Abdeel, de saisir Baruc, le secrétaire, et Jérémie le prophète~; mais Yahweh les cacha.
\TextTitle{Remplacement du manuscrit brûlé~; jugement sur Jojakim}
\VS{27}Et la parole de Yahweh fut adressée à Jérémie, après que le roi eut brûlé le rouleau contenant les paroles que Baruc avait écrites sous la dictée de Jérémie, en disant~:
\VS{28}Prends encore un autre rouleau, et tu y écriras toutes les premières paroles qui étaient dans le premier rouleau que Jojakim, roi de Juda, a brûlé.
\VS{29}Et tu diras à Jojakim, roi de Juda~: Ainsi parle Yahweh~: Tu as brûlé ce rouleau, et tu as dit~: Pourquoi y as-tu écrit ces paroles~: Le roi de Babylone viendra certainement, il détruira ce pays, et il exterminera les hommes et les bêtes~?
\VS{30}C'est pourquoi ainsi parle Yahweh sur Jojakim, roi de Juda~: Aucun des siens ne sera assis sur le trône de David, et son cadavre sera jeté de jour à la chaleur et de nuit à la gelée.
\VS{31}Je le punirai, lui, sa postérité, et ses serviteurs, à cause de leur iniquité~; et je ferai venir sur eux, et sur les habitants de Jérusalem, et sur les hommes de Juda, tout le mal que je leur ai prononcé, et qu'ils n'ont pas écouter.
\VS{32}Jérémie donc prit un autre rouleau, et le donna à Baruc, fils de Nérija secrétaire, lequel y écrivit, sous la dictée de Jérémie, toutes les paroles du rouleau que Jojakim, roi de Juda, avait brûlé au feu. Beaucoup de paroles semblables y furent encore ajoutées.
\Chap{37}
\TextTitle{Sédécias sollicite l'intercession de Jérémie}
\VerseOne{}Or le roi Sédécias, fils de Josias, régna à la place de Jéconia, fils de Jojakim, et il fut établi roi dans le pays de Juda par Nebucadnetsar, roi de Babylone.
\VS{2}Mais, ni lui, ni ses serviteurs, ni le peuple du pays, n'obéirent aux paroles que Yahweh prononça par Jérémie le prophète.
\VS{3}Toutefois le roi Sédécias envoya Jucal, fils de Schélémia, et Sophonie, fils de Maaséja prêtre, vers Jérémie le prophète, pour lui dire~: Intercède pour nous auprès de Yahweh, notre Dieu.
\VS{4}Car Jérémie allait et venait parmi le peuple, parce qu'on ne l'avait pas encore mis en prison.
\VS{5}Alors l'armée de Pharaon sortit d'Egypte, et quand les Chaldéens qui assiégeaient Jérusalem en entendirent cette nouvelle, ils se retirèrent de devant Jérusalem.
\VS{6}Et la parole de Yahweh fut adressée à Jérémie le prophète, en disant~:
\VS{7}Ainsi parle Yahweh, le Dieu d'Israël~: Vous direz ainsi au roi de Juda, qui vous a envoyés me consulter~: Voici, l'armée de Pharaon, qui était sortie à votre secours, retourne dans son pays, en Egypte~;
\VS{8}et les Chaldéens reviendront, et combattront contre cette ville, et la prendront, et la brûleront au feu.
\VS{9}Ainsi parle Yahweh~: Ne vous abusez pas vous-mêmes, en disant~: Les Chaldéens s'en iront loin de nous~; car ils ne s'en iront pas.
\VS{10}Même quand vous auriez battu toute l'armée des Chaldéens qui combattent contre vous, et qu'il n'y aurait de reste entre eux que des hommes percés de blessures, ils se relèveront pourtant chacun dans sa tente, et brûleront cette ville au feu.
\TextTitle{Jérémie calomnié et emprisonné}
\VS{11}Or il arriva, que quand l'armée des Chaldéens se fut retirée de Jérusalem, à cause de l'armée de Pharaon,
\VS{12}Jérémie sortit de Jérusalem, pour s'en aller dans le pays de Benjamin, se glissant hors de là au milieu du peuple.
\VS{13}Mais quand il fut à la porte de Benjamin, il y avait là un commandant de la garde, nommé Jireija, fils de Schélémia, fils de Hanania, qui saisit Jérémie le prophète, en lui disant~: Tu vas te rendre aux Chaldéens~!
\VS{14}Et Jérémie répondit~: C'est un mensonge~! Je ne vais pas me rendre aux Chaldéens. Mais il ne l'écouta pas, et Jireija prit Jérémie, et l'amena vers les chefs.
\VS{15}Et les chefs se mirent en colère contre Jérémie, le frappèrent et le mirent en prison dans la maison de Jonathan le secrétaire, car ils en avaient fait une prison.
\VS{16}Et ainsi Jérémie entra dans la fosse de la maison et dans les cachots~; et Jérémie y demeura plusieurs jours.
\VS{17}Mais le roi Sédécias l'envoya chercher, et l'en tira, puis il l'interrogea en secret dans sa maison, et lui dit~: Y a-t-il une parole de la part de Yahweh~? Et Jérémie répondit~: Il y en a une~: Tu seras livré entre les mains du roi de Babylone.
\VS{18}Puis Jérémie dit au roi Sédécias~: Quel péché ai-je commis contre toi, contre tes serviteurs et contre ce peuple, pour que vous m'ayez mis en prison~?
\VS{19}Mais où sont vos prophètes qui vous prophétisaient, en disant~: Le roi de Babylone ne reviendra pas contre vous, ni contre ce pays~?
\VS{20}Or écoute maintenant, je te prie, ô roi, mon seigneur~! Et que maintenant ma supplication soit reçue devant ta face, et ne me renvoie pas dans la maison de Jonathan le secrétaire, de peur que je n'y meure~!
\VS{21}C'est pourquoi le roi Sédécias ordonna qu'on garde Jérémie dans la cour de la prison, et qu'on lui donne chaque jour un pain de la rue des boulangers, jusqu'à ce que tout le pain de la ville soit épuisé. Ainsi Jérémie demeura dans la cour de la prison.
\Chap{38}
\TextTitle{Jérémie jeté dans la fosse, puis délivré par Ebed-Mélec l'éthiopien}
\VerseOne{}Mais Schephathia, fils de Matthan, et Guedalia, fils de Paschhur, et Jucal, fils de Schélémia, et Paschhur, fils de Malkija, entendirent les paroles que Jérémie prononçait à tout le peuple, en disant~:
\VS{2}Ainsi parle Yahweh~: Celui qui restera dans cette ville mourra par l'épée, par la famine, ou par la peste~; mais celui qui sortira vers les Chaldéens vivra, et sa vie sera son butin, et il vivra.
\VS{3}Ainsi parle Yahweh~: Cette ville sera livrée certainement aux mains de l'armée du roi de Babylone, qui la prendra.
\VS{4}Et les chefs dirent au roi~: Qu'on fasse mourir cet homme~! Car il décourage les mains des hommes de guerre qui restent dans cette ville, et les mains de tout le peuple, en leur disant de telles paroles~; parce que cet homme ne cherche pas le bien de ce peuple, mais le mal.
\VS{5}Et le roi Sédécias dit~: Voici, il est entre vos mains~; car le roi ne peut rien contre vous.
\VS{6}Ils prirent donc Jérémie, et le jetèrent dans la fosse de Malkija, fils de Hammélec, laquelle était dans la cour de la prison, et ils descendirent Jérémie avec des cordes dans cette fosse où Il n'y avait pas d'eau mais de la boue~; et ainsi Jérémie enfonça dans la boue.
\VS{7}Mais Ebed-Mélec l'éthiopien, eunuque, qui était dans la maison du roi, apprit qu'ils avaient mis Jérémie dans cette fosse~; et le roi était assis à la porte de Benjamin.
\VS{8}Et Ebed-Mélec sortit de la maison du roi, et parla au roi, en disant~:
\VS{9}Ô roi, mon seigneur~! Ces hommes-là ont mal fait dans tout ce qu'ils ont fait contre Jérémie le prophète, en le jetant dans la fosse, car il serait déjà mort de faim dans le lieu où il était parce qu'il n'y a plus de pain dans la ville.
\VS{10}C'est pourquoi le roi donna cet ordre à Ebed-Mélec, l'éthiopien, en disant~: Prends ici trente hommes avec toi, et fais remonter hors de la fosse Jérémie le prophète, avant qu'il meure.
\VS{11}Ebed-Mélec prit donc des hommes avec lui, et entra dans la maison du roi, dans un lieu au-dessous du trésor, d'où il prit de vieux lambeaux et de vieux chiffons, et les descendit avec des cordes à Jérémie dans la fosse.
\VS{12}Et Ebed-Mélec, l'éthiopien dit à Jérémie~: Mets ces vieux lambeaux et ces chiffons sous les aisselles de tes bras, au-dessous des cordes. Et Jérémie fit ainsi.
\VS{13}Ainsi ils tirèrent Jérémie dehors avec les cordes, et le firent remonter hors de la fosse~; et Jérémie demeura dans la cour de la prison.
\TextTitle{Jérémie appelle Sédécias à la repentance}
\VS{14}Et le roi Sédécias envoya chercher Jérémie le prophète et le fit amener vers lui à la troisième entrée, qui était dans la maison de Yahweh. Et le roi dit à Jérémie~: Je vais te demander une chose, ne me cache rien.
\VS{15}Et Jérémie répondit à Sédécias~: Quand je te l'aurais déclarée, n'est-il pas vrai que tu me feras mourir~? Et quand je t'aurai donné conseil, tu ne m'écouteras pas.
\VS{16}Alors le roi Sédécias jura secrètement à Jérémie, en disant~: Yahweh, qui a fait notre âme, est vivant~; je ne te ferai pas mourir, et je ne te livrerai pas entre les mains de ces hommes qui cherchent ta vie.
\VS{17}Alors Jérémie dit à Sédécias~: Ainsi parle Yahweh, le Dieu des armées, le Dieu d'Israël~: Si tu sors volontairement pour aller vers les chefs du roi de Babylone, tu auras la vie, et cette ville ne sera pas brûlée par le feu~; et tu vivras toi et ta maison.
\VS{18}Mais si tu ne sors pas vers les chefs du roi de Babylone, cette ville sera livrée entre les mains des Chaldéens, qui la brûleront par le feu~; et tu n'échapperas pas à leurs mains.
\VS{19}Et le roi Sédécias dit à Jérémie~: Je crains à cause des Juifs qui se sont rendus aux Chaldéens, je crains qu'on ne me livre entre leurs mains et qu'ils ne se moquent de moi.
\VS{20}Et Jérémie lui répondit~: On ne te livrera pas à eux. Je te prie, écoute la voix de Yahweh dans ce que je te dis~; tu t'en trouveras bien et tu auras la vie.
\VS{21}Que si tu refuses de sortir, voici ce que Yahweh m'a fait voir~:
\VS{22}C'est que toutes les femmes qui restent dans la maison du roi de Juda seront menées aux chefs du roi de Babylone, et elles diront~: Tu as été séduit, vaincu, par les hommes qui te prédisaient la paix~; et quand tes pieds se sont enfoncés dans la boue, ils se sont retirés en arrière.
\VS{23}Toutes tes femmes et tes fils seront menés dehors aux Chaldéens~; et tu n'échapperas pas à leurs mains, mais tu seras pris, pour être livré entre les mains du roi de Babylone, et à cause de toi, cette ville sera brûlée par le feu.
\VS{24}Alors Sédécias dit à Jérémie~: Que personne ne sache rien de ces paroles, et tu ne mourras pas.
\VS{25}Et si les chefs entendent que je t'ai parlé, et qu'ils viennent vers toi, et te disent~: Déclare-nous maintenant ce que tu as dit au roi, et ce que le roi t'a dit, ne nous en cache rien, et nous ne te ferons pas mourir~;
\VS{26}tu leur diras~: J'ai présenté ma supplication devant le roi afin qu'il ne me renvoie pas dans la maison de Jonathan, pour y mourir.
\VS{27}Tous les chefs donc, vinrent vers Jérémie et l'interrogèrent~; mais il leur répondit exactement comme le roi lui avait ordonné~; et ils gardèrent le silence, car l'affaire n'avait pas été divulguée.
\VS{28}Ainsi Jérémie demeura dans la cour de la prison, jusqu'au jour où Jérusalem fut prise, et il y était lorsque Jérusalem fut prise.
\Chap{39}
\TextTitle{Prise de Jérusalem~; Sédécias déporté à Babylone\FTNTT{2 R. 25:1-7~; Jé. 52:4-17~; 2 Ch. 36:17-21.}}
\VerseOne{}La neuvième année de Sédécias, roi de Juda, au dixième mois, Nebucadnetsar, roi de Babylone, vint avec toute son armée contre Jérusalem, et ils l'assiégèrent.
\VS{2}Et la onzième année de Sédécias, le neuvième jour du quatrième mois, une brèche fut faite à la ville.
\VS{3}Et tous les chefs du roi de Babylone y entrèrent, et s'assirent à la porte du milieu, à savoir, Nergal-Scharetser, Samgar-Nebu, Sarsekim, chef des eunuques, Nergal-Scharetser, chef des devins et tous les autres chefs du roi de Babylone.
\VS{4}Or il arriva qu'aussitôt que Sédécias, roi de Juda, et tous les hommes de guerre les eurent vus, ils s'enfuirent et sortirent de nuit hors de la ville, par le chemin du jardin du roi, par la porte entre les deux murailles, et ils s'en allèrent par le chemin de la plaine.
\VS{5}Mais l'armée des Chaldéens les poursuivit et atteignit Sédécias dans les plaines de Jéricho. Ils le prirent, et le firent monter vers Nebucadnetsar, roi de Babylone, à Ribla, dans le pays de Hamath, où il prononça contre lui une sentence.
\VS{6}Et le roi de Babylone fit égorger à Ribla, les fils de Sédécias sous ses yeux~; le roi de Babylone fit aussi égorger tous les nobles de Juda.
\VS{7}Puis il fit crever les yeux à Sédécias, et le fit lier de doubles chaînes d'airain, pour le conduire à Babylone.
\VS{8}Les Chaldéens brûlèrent par le feu la maison royale et les maisons du peuple, et démolirent\FTNT{Ici débute le «~temps des nations~» (587-586 av. J.-C.), Jérusalem est foulée aux pieds par les nations. Voir aussi 2 R. 25:8-24~; 2 Ch. 36:17-21.} les murailles de Jérusalem.
\VS{9}Et Nebuzaradan, chef des gardes, transporta à Babylone le reste du peuple qui était resté dans la ville, et ceux qui s'étaient rendus à lui, le reste, dis-je, du peuple qui avait été épargné.
\VS{10}Mais Nebuzaradan, chefs des gardes, laissa dans le pays de Juda les plus pauvres du peuple qui n'avaient rien~; et en ce jour-là, il leur donna des vignes et des champs.
\TextTitle{Jérémie libéré de prison}
\VS{11}Or Nebucadnetsar, roi de Babylone, avait donné cet ordre au sujet de Jérémie, à Nebuzaradan, chef des gardes, en disant~:
\VS{12}Prends cet homme et veille sur lui~; ne lui fais aucun mal, mais fais pour lui tout ce qu'il te dira.
\VS{13}Donc Nebuzaradan, chefs des gardes, envoya, et Nebuschazban, Rabsaris, chef des eunuques, et Nergal-Scharetser, Rabmag, chef des devins, et tous les chefs du roi de Babylone~;
\VS{14}ils envoyèrent, dis-je, chercher Jérémie dans la cour de la prison, et le remirent à Guedalia, fils d'Achikam, fils de Schaphan, pour qu'il le conduise dans sa maison. Ainsi il demeura au milieu du peuple.
\TextTitle{Yahweh épargne Ebed-Mélec}
\VS{15}Or la parole de Yahweh fut adressée à Jérémie pendant qu'il était enfermé dans la cour de la prison, en disant~:
\VS{16}Va, et parle à Ebed-Mélec l'éthiopien, et dis-lui~: Ainsi parle Yahweh des armées, le Dieu d'Israël~: Voici, je vais faire venir mes paroles sur cette ville pour son malheur et non pas pour son bien, et elles s'accompliront en ce jour-là devant toi.
\VS{17}Mais je te délivrerai en ce jour-là, dit Yahweh, et tu ne seras pas livré entre les mains des hommes que tu crains.
\VS{18}Car certainement je te ferai échapper, et tu ne tomberas pas sous l'épée~; mais ta vie sera ton butin, parce que tu as eu confiance en moi, dit Yahweh.
\Chap{40}
\TextTitle{Assassinat de Guedalia et meurtres en série d'Ismaël}
\VerseOne{}La parole qui fut adressée à Jérémie de la part de Yahweh, quand Nebuzaradan, chef des gardes, l'eut renvoyé de Rama, après l'avoir pris lorsqu'il était lié de chaînes parmi tous les captifs de Jérusalem et de Juda qu'on transportait à Babylone.
\VS{2}Quand donc le chef des gardes prit Jérémie, et il lui dit~: Yahweh, ton Dieu, a prononcé ce mal contre ce lieu-ci~;
\VS{3}et Yahweh l'a fait venir et a fait comme il avait dit, parce que vous avez péché contre Yahweh, et que vous n'avez pas écouté sa voix, à cause de cela ceci vous est arrivé.
\VS{4}Maintenant donc voici, je t'affranchis aujourd'hui des chaînes que tu as aux mains~; s'il est bon à tes yeux de venir avec moi à Babylone, viens, et j'aurai les yeux sur toi~; mais s'il est mauvais de venir avec moi à Babylone, ne viens pas~; regarde, tout le pays est à ta disposition, va où il te semblera bon et convenable d'aller.
\VS{5}Or Guedalia ne retournera plus ici~; retourne, dit-il, vers Guedalia, fils d'Achikam, fils de Schaphan, que le roi de Babylone a établi sur les villes de Juda, et demeure avec lui parmi le peuple~; ou bien, va partout où il conviendra à tes yeux d'aller. Et le chef des gardes lui donna des vivres et quelques présents, et le renvoya.
\VS{6}Jérémie donc alla vers Guedalia, fils d'Achikam, à Mitspa, et demeura avec lui parmi le peuple qui était resté dans le pays.
\VS{7}Et tous les chefs des armées qui étaient dans les champs, eux et leurs hommes, entendirent que le roi de Babylone avait établi Guedalia, fils d'Achikam, sur le pays, et qu'il lui avait commis les hommes, et les femmes, et les enfants, et ceux-là d'entre les plus pauvres du pays, à savoir, de ceux qui n'avaient pas été transportés à Babylone.
\VS{8}Alors ils allèrent vers Guedalia à Mitspa~; à savoir, Ismaël fils de Nethania, et Jochanan et Jonathan fils de Karéach, et Seraja fils de Thanhumeth, et les fils d'Ephaï de Nethopha, et Jezania fils du Maacatite, eux et leurs hommes.
\VS{9}Et Guedalia, fils d'Achikam, fils de Schaphan, leur jura, à eux et à leurs hommes, en disant~: Ne craignez pas de servir les Chaldéens~; demeurez dans le pays, et servez le roi de Babylone, et vous vous en trouverez bien.
\VS{10}Et pour moi, voici, je resterai à Mitspa, pour me tenir prêt à recevoir les ordres des Chaldéens qui viendront vers nous~; mais vous, recueillez le vin, les fruits d'été et l'huile, et mettez-les dans vos vases, et demeurez dans vos villes que vous avez prises pour votre demeure.
\VS{11}Pareillement aussi tous les Juifs qui étaient au pays de Moab, parmi les Ammonites, au pays d'Edom, et dans toutes ces contrées, quand ils entendirent que le roi de Babylone avait laissé quelque reste à Juda, et qu'il avait établi sur eux Guedalia, fils d'Achikam, fils de Schaphan~;
\VS{12}tous ces juifs-là retournèrent de tous les lieux où ils avaient été chassés, et vinrent dans le pays de Juda vers Guedalia à Mitspa, et recueillirent du vin et des fruits d'été en grande abondance.
\VS{13}Mais Jochanan, fils de Karéach, et tous les chefs des armées qui étaient dans les champs, vinrent vers Guedalia à Mitspa,
\VS{14}et lui dirent~: Ne sais-tu pas certainement que Baalis, roi des Ammonites, a envoyé Ismaël, le fils de Nethania, pour t'ôter la vie~? Mais Guedalia, fils d'Achikam, ne les crut pas.
\VS{15}Et Jochanan, fils de Karéach parla en secret à Guedalia à Mitspa, en disant~: Laisse-moi aller et je frapperai Ismaël, fils de Nethania, et personne ne le saura. Pourquoi t'ôterait-il la vie, de sorte que tous les Juifs qui se sont rassemblés vers toi soient dissipés, et que les restes de Juda périssent~?
\VS{16}Mais Guedalia, fils d'Achikam, dit à Jochanan, fils de Karéach~: Ne fais pas cela car tu parles faussement d'Ismaël.
\Chap{41}
\TextTitle{Assassinat de Guedalia}
\VerseOne{}Or il arriva, au septième mois, qu'Ismaël, fils de Nethania, fils d'Elischama, de la race royale, et l'un des grands du roi et dix hommes avec lui, vinrent vers Guedalia, fils d'Achikam, à Mitspa~; et ils mangèrent là du pain ensemble à Mitspa\FTNT{2 R. 25:25.}.
\VS{2}Mais Ismaël, fils de Nethania, se leva, et les dix hommes qui étaient avec lui, et ils frappèrent avec l'épée Guedalia, fils d'Achikam, fils de Schaphan, et on le fit mourir, lui que le roi de Babylone avait établi sur le pays.
\VS{3}Ismaël frappa aussi tous les juifs qui étaient avec Guedalia à Mitspa, et les Chaldéens, gens de guerre, qui se trouvaient là.
\VS{4}Et il arriva que le second jour après qu'on eut fait mourir Guedalia, avant que personne le sût,
\VS{5}quelques hommes de Sichem, de Silo et de Samarie, au nombre de quatre-vingts hommes, ayant la barbe rasée et les vêtements déchirés, et s'étant fait des incisions, vinrent avec des dons et de l'encens dans leurs mains pour les apporter dans la maison de Yahweh.
\VS{6}Alors Ismaël, fils de Nethania, sortit de Mitspa au-devant d'eux~; il marchait en pleurant, et quand il les rencontra, il leur dit~: Venez vers Guedalia, fils d'Achikam.
\VS{7}Mais sitôt qu'ils arrivèrent au milieu de la ville, Ismaël, fils de Nethania, accompagné des hommes qui étaient avec lui, les égorgea et les jeta dans une fosse.
\VS{8}Mais il se trouva parmi eux dix hommes, qui dirent à Ismaël~: Ne nous fais pas mourir, car nous avons dans les champs des provisions cachées de froment, d'orge, d'huile et de miel~; il les laissa, et ne les fit pas mourir avec leurs frères.
\VS{9} Et la fosse dans laquelle Ismaël jeta les cadavres des hommes qu'il tua, à l'occasion de Guedalia, est celle que le roi Asa avait faite, lorsqu'il craignait Baescha, roi d'Israël~; et Ismaël, fils de Nethania, la remplit de gens tués\FTNT{1 R. 15:22.}.
\VS{10}Et Ismaël emmena captif tout le reste du peuple qui était à Mitspa, les filles du roi et tous ceux du peuple qui demeuraient à Mitspa, que Nebuzaradan, chef des gardes, avait commis à Guedalia, fils d'Achikam~; Ismaël, fils de Nethania, les emmena captifs, et s'en alla pour passer vers les Ammonites.
\TextTitle{Jochanan délivre le peuple~; fuite d'Ismaël}
\VS{11}Mais Jochanan, fils de Karéach, et tous les chefs des armées qui étaient avec lui, entendirent tout le mal qu'Ismaël, fils de Nethania, avait fait~;
\VS{12}et ils prirent tous les hommes, et s'en allèrent pour combattre contre Ismaël, fils de Nethania. Ils le trouvèrent près des grandes eaux qui sont à Gabaon.
\VS{13}Et il arriva qu'aussitôt que tout le peuple qui était avec Ismaël vit Jochanan, fils de Karéach, et tous les chefs des armées qui étaient avec lui, ils s'en réjouirent~;
\VS{14}et tout le peuple qu'Ismaël avait emmené captif de Mitspa tourna visage, et revenant sur leur pas, il s'en alla vers Jochanan, fils de Karéach.
\VS{15}Mais Ismaël, fils de Nethania, échappa avec huit hommes devant Jochanan, et s'en alla vers les Ammonites.
\VS{16}Et Jochanan, fils de Karéach, et tous les chefs des armées qui étaient avec lui, prirent tout le reste du peuple qu'ils avaient retiré des mains d'Ismaël, fils de Nethania, qu'il avait emmené captif de Mitspa, après avoir tué Guedalia, fils d'Achikam, à savoir, les vaillants hommes de guerre, et les femmes, et les enfants et les eunuques~; et les ramenèrent de Gabaon.
\VS{17}Et ils s'en allèrent et demeurèrent à l'hôtellerie de Kimham, près de Bethléhem, pour se retirer ensuite en Egypte,
\VS{18}à cause des Chaldéens~; car ils avaient peur d'eux, parce qu'Ismaël, fils de Nethania, avait tué Guedalia, fils d'Achikam, qui avait été établi sur le pays par le roi de Babylone.
\Chap{42}
\TextTitle{Yahweh défend au reste du peuple de se réfugier en Egypte}
\VerseOne{}Alors tous les chefs des armées, et Jochanan, fils de Karéach, et Jezania, fils d'Hosée, et tout le peuple, depuis le plus petit jusqu'au plus grand, s'approchèrent,
\VS{2}et dirent à Jérémie le prophète~: Que notre supplication soit favorable devant toi~! Intercède auprès de Yahweh, ton Dieu, pour nous, à savoir, pour tout ce reste-ci~; car de beaucoup de monde que nous étions, nous sommes restés peu, comme tes yeux nous voient~;
\VS{3}et que Yahweh, ton Dieu, nous déclare le chemin par lequel nous aurons à marcher, et ce que nous avons à faire~!
\VS{4}Et Jérémie le prophète, leur répondit~: J'ai entendu votre demande~; voici, je vais prier Yahweh, votre Dieu, selon vos paroles~; et il arrivera que je vous déclarerai tout ce que Yahweh vous répondra, et je ne vous en cacherai pas un mot.
\VS{5}Et ils dirent à Jérémie~: Yahweh soit entre nous un témoin véritable et fidèle, si nous ne faisons pas selon toutes les paroles que Yahweh, ton Dieu, t'enverra vers nous~!
\VS{6}Soit bien, soit mal, nous obéirons à la voix de Yahweh, notre Dieu, vers qui nous t'envoyons, afin qu'il nous arrive du bien, quand nous aurons obéi à la voix de Yahweh, notre Dieu.
\VS{7}Et il arriva, au bout de dix jours, que la parole de Yahweh fut adressée à Jérémie.
\VS{8}Et il appela Jochanan, fils de Karéach, tous les chefs des armées qui étaient avec lui, et tout le peuple, depuis le plus petit jusqu'au plus grand~;
\VS{9}et leur dit~: Ainsi parle Yahweh, le Dieu d'Israël, vers qui vous m'avez envoyé, pour présenter votre supplication devant lui~:
\VS{10}Si vous continuez à demeurer dans ce pays, je vous rétablirai et je ne vous détruirai pas~; je vous y planterai et je ne vous arracherai pas, car je me repens du mal que je vous ai fait.
\VS{11}Ne craignez pas le roi de Babylone, dont vous avez peur, ne craignez pas, dit Yahweh, car je suis avec vous pour vous sauver et pour vous délivrer de sa main.
\VS{12}Même je vous ferai obtenir miséricorde, tellement qu'il aura pitié de vous, et vous fera retourner dans votre pays.
\VS{13}Que si vous dites~: Nous ne demeurerons pas dans ce pays, et nous n'écouterons pas la voix de Yahweh, notre Dieu,
\VS{14}en disant~: Non~; mais nous irons au pays d'Egypte, afin que nous ne voyons pas de guerre, et que nous n'entendions pas le son du shofar, et que nous ne manquions pas de pain, et nous demeurerons là.
\VS{15}A cause de cela écoutez maintenant la parole de Yahweh, vous les restes de Juda~! Ainsi parle Yahweh des armées, le Dieu d'Israël~: Si vous tournez le visage pour aller en Egypte, et que vous y entriez pour y demeurer~;
\VS{16}il arrivera que l'épée dont vous avez peur vous attrapera là au pays d'Egypte~; et la famine que vous craignez si fort vous suivra en Egypte, et vous y mourrez\FTNT{Ez. 30:9-11.}.
\VS{17}Et il arrivera que tous les hommes qui tourneront le visage pour aller en Egypte afin d'y demeurer, mourront par l'épée, par la famine et par la peste~; et il n'y aura ni survivant ni réchappé devant le mal que je vais faire venir sur eux.
\VS{18}Car ainsi parle Yahweh des armées, le Dieu d'Israël~: Comme ma colère et ma fureur se sont répandues sur les habitants de Jérusalem, ainsi ma fureur sera versée sur vous, quand vous serez entrés en Egypte~; et vous serez un sujet d'exécration, d'épouvante, de malédiction et d'opprobre, et vous ne verrez plus ce lieu-ci.
\VS{19}Vous, les restes de Juda, Yahweh dit contre vous~: N'allez pas en Egypte~! Sachez certainement que je vous ai avertis aujourd'hui.
\VS{20}Car vous vous êtes séduits vous-mêmes dans vos âmes, quand vous m'avez envoyé vers Yahweh, votre Dieu, en me disant~: Intercède pour nous auprès de Yahweh, notre Dieu, et déclare-nous tout ce que Yahweh, notre Dieu, te dira, et nous le ferons.
\VS{21}Et je vous l'ai déclaré aujourd'hui~; mais vous n'écoutez pas la voix de Yahweh, votre Dieu, ni rien de tout ce pour quoi il m'a envoyé vers vous.
\VS{22}Maintenant donc sachez certainement que vous mourrez par l'épée, par la famine et par la peste, dans le lieu où vous avez désiré d'aller pour y demeurer.
\Chap{43}
\TextTitle{Désobéissance des Hébreux~; jugement sur l'Egypte}
\VerseOne{}Or il arriva qu'aussitôt que Jérémie eut achevé de prononcer à tout le peuple toutes les paroles de Yahweh, leur Dieu, pour lesquelles Yahweh, leur Dieu, l'avait envoyé vers eux, à savoir, toutes ces choses-là~;
\VS{2}Azaria, fils d'Hosée, et Jochanan, fils de Karéach, et tous ces hommes orgueilleux, dirent à Jérémie~: Tu dis un mensonge~; Yahweh, notre Dieu, ne t'a pas envoyé nous dire~: N'allez pas en Egypte pour y demeurer.
\VS{3}Mais Baruc, fils de Nérija, t'incite contre nous, afin de nous livrer entre les mains des Chaldéens, pour nous faire mourir, et pour nous faire transporter à Babylone.
\VS{4}Ainsi Jochanan, fils de Karéach, et tous les chefs des armées, et tout le peuple, n'obéirent pas à la voix de Yahweh, pour demeurer dans le pays de Juda.
\VS{5}Car Jochanan, fils de Karéach, et tous les chefs des armées, prirent tous les restes de Juda qui étaient revenus de toutes les nations, parmi lesquelles ils avaient été chassés, pour demeurer dans le pays de Juda~;
\VS{6}les hommes, et les femmes, et les enfants, et les filles du roi, et toutes les personnes que Nebuzaradan, chef des gardes, avait laissées avec Guedalia, fils d'Achikam, fils de Schaphan~; ils prirent aussi Jérémie le prophète et Baruc, fils de Nérija.
\VS{7}Et ils entrèrent dans le pays d'Egypte, car ils n'obéirent pas à la voix de Yahweh, et ils vinrent jusqu'à Tachpanès.
\VS{8}Alors la parole de Yahweh fut adressée à Jérémie, à Tachpanès, en disant~:
\VS{9}Prends dans ta main de grandes pierres, et cache-les dans l'argile, dans le four à briques qui est à l'entrée de la maison de Pharaon à Tachpanès, sous les yeux des Juifs~;
\VS{10}et dis-leur~: Ainsi parle Yahweh des armées, le Dieu d'Israël~: Voici, j'enverrai chercher Nebucadnetsar, roi de Babylone, mon serviteur. Et je mettrai son trône sur ces pierres que j'ai cachées, et il étendra son dais sur elles~;
\VS{11}et il viendra et frappera le pays d'Egypte. Ceux qui sont destinés à la mort, iront à la mort~; et ceux qui sont destinés à la captivité, iront en captivité~; et ceux qui sont destinés à l'épée, seront livrés à l'épée\FTNT{Ez. 29:9.}~!
\VS{12}Et j'allumerai le feu dans les maisons des dieux d'Egypte, Nebucadnetsar les brûlera, et il emmènera captifs ceux d'Egypte, et il se parera des richesses du pays d'Egypte, comme le pasteur s'enveloppe de son vêtement, et il en sortira en paix\FTNT{Es. 19:1~; Ez. 30:13.}.
\VS{13}Il brisera aussi les statues de Beth-Schémesch, qui est au pays d'Egypte, et il brûlera par le feu les maisons des dieux d'Egypte.
\Chap{44}
\TextTitle{Yahweh avertit les Juifs d'Egypte\FTNTT{Jé. 43:8-13}}
\VerseOne{}La parole qui fut adressée à Jérémie sur tous les Juifs qui demeuraient au pays d'Egypte, qui habitaient à Migdol, à Tachpanès, à Noph, et au pays de Pathros, en disant~:
\VS{2}Ainsi parle Yahweh des armées, le Dieu d'Israël~: Vous avez vu tous les malheurs que j'ai fait venir sur Jérusalem et sur toutes les villes de Juda~: Voici, elles ne sont plus aujourd'hui que des ruines, et personne n'y habite,
\VS{3}à cause des méchancetés qu'ils ont faites pour m'irriter, en allant brûler de l'encens pour servir d'autres dieux, qu'ils n'ont pas connu, ni eux, ni vous, ni vos pères.
\VS{4}Et je vous ai envoyé tous mes serviteurs, les prophètes, me levant dès le matin, et les envoyant, pour vous dire~: Ne commettez pas maintenant cette chose abominable que je hais.
\VS{5}Mais ils n'ont pas écouté, ils n'ont pas prêté l'oreille pour se détourner de leur méchanceté, afin de ne pas faire brûler de l'encens à d'autres dieux.
\VS{6}C'est pourquoi ma fureur et ma colère se sont répandues sur eux, et ont embrasé les villes de Juda et les rues de Jérusalem, qui ne sont réduites en désert et en désolation, comme il paraît aujourd'hui.
\VS{7}Maintenant donc, ainsi parle Yahweh, le Dieu des armées, le Dieu d'Israël~: Pourquoi faites-vous ce grand mal contre vos âmes, pour vous faire exterminer du milieu de Juda, hommes et femmes, petits enfants et ceux qui tètent, afin qu'on ne vous laisse aucun reste~?
\VS{8}En m'irritant par les œuvres de vos mains, en brûlant de l'encens à d'autres dieux au pays d'Egypte, où vous êtes venus pour y demeurer, afin de vous faire exterminer et d'être un objet de malédiction et d'opprobre parmi toutes les nations de la terre~?
\VS{9}Avez-vous oublié les crimes de vos pères, les crimes des rois de Juda, les crimes de leurs femmes, vos propres crimes et les crimes de vos femmes, commis dans le pays de Juda et dans les rues de Jérusalem~?
\VS{10}Jusqu'à ce jour, ils ne se sont pas humiliés, ils n'ont pas eu de crainte, ils n'ont pas marché dans ma loi ni dans mes ordonnances, que j'ai mises devant vous et devant vos pères.
\VS{11}C'est pourquoi ainsi parle Yahweh des armées, le Dieu d'Israël~: Voici, je tourne ma face contre vous pour vous nuire et vous retrancher tout Juda\FTNT{Am. 9:4.}.
\VS{12}Et je prendrai les restes de ceux de Juda qui ont tourné le visage pour aller au pays d'Egypte afin d'y demeurer~; ils seront tous consumés, ils tomberont dans le pays d'Egypte~; ils seront consumés par l'épée, par la famine, depuis le plus petit jusqu'au plus grand~; ils mourront par l'épée et par la famine~; et ils seront en exécration, en étonnement, en malédiction et en opprobre.
\VS{13}Et je punirai ceux qui demeurent au pays d'Egypte, comme j'ai puni Jérusalem, par l'épée, par la famine, et par la peste.
\VS{14}Et il n'y aura personne du reste de Juda qui, venu dans le pays d'Egypte pour y séjourner, n'échappera ou ne restera, pour retourner dans le pays de Juda, où ils ont le désir de retourner s'établir. Car ils n'y retourneront pas, sinon ceux qui se seront échappés.
\VS{15}Mais tous les hommes qui savaient que leurs femmes brûlaient de l'encens à d'autres dieux, toutes les femmes qui se tenaient là en grande compagnie, et tout le peuple qui demeurait au pays d'Egypte, à Pathros, répondirent à Jérémie, en disant~:
\VS{16}Quant à la parole que tu nous as dite au nom de Yahweh, nous ne t'écouterons pas.
\VS{17}Mais nous ferons assurément selon toute parole qui est sortie de notre bouche, brûler de l'encens à la reine des cieux\FTNT{Voir commentaire en Jé. 7:18.}, et lui faire des libations, comme nous l'avons fait, nous et nos pères, nos rois et nos chefs, dans les villes de Juda et dans les rues de Jérusalem. Alors nous étions rassasiés de pain, nous étions heureux, et nous ne voyions pas le malheur\FTNT{Ez. 16:24~; Ez. 20:32.}.
\VS{18}Mais depuis le temps que nous avons cessé de brûler de l'encens à la reine des cieux et de lui faire des libations, nous avons manqué de tout, et nous avons été consumés par l'épée et par la famine…
\VS{19}Et quand nous brûlions de l'encens à la reine des cieux et que nous lui faisions des libations, est-ce à l'insu de nos maris que nous lui avons fait des gâteaux sur lesquels elle est représentée et que nous lui avons fait des libations~?
\VS{20}Alors Jérémie parla à tout le peuple, aux hommes, aux femmes, et à tous ceux qui lui avaient donné cette réponse, et leur dit~:
\VS{21}Yahweh ne s'est-il pas souvenu, ne lui est-il pas monté à cœur l'encens que vous avez brûlé dans les villes de Juda et dans les rues de Jérusalem, vous et vos pères, vos rois et vos chefs, et le peuple du pays~?
\VS{22}Yahweh n'a pas pu le supporter davantage, à cause de la méchanceté de vos actions, à cause des abominations que vous avez faites~; et votre pays est devenu une ruine, un désert et un objet de malédiction, sans que personne y habite, comme on le voit aujourd'hui.
\VS{23}C'est parce que vous avez brûlé de l'encens et que vous avez péché contre Yahweh, parce que vous n'avez pas écouté la voix de Yahweh, et que vous n'avez pas marché dans sa loi, ni dans ses ordonnances, ni dans ses témoignages~; c'est pour cela que ces malheurs vous sont arrivés, comme on le voit aujourd'hui.
\VS{24}Jérémie dit à tout le peuple et à toutes les femmes~: Vous tous de Juda, qui êtes au pays d'Egypte, écoutez la parole de Yahweh~!
\VS{25}Ainsi parle Yahweh des armées, le Dieu d'Israël, en disant~: Vous et vos femmes, vous avez parlé de vos bouches et accompli de vos mains, en disant~: Certainement, nous accomplirons nos vœux que nous avons faits, brûler de l'encens à la reine des cieux, et lui faire des libations. Vous avez entièrement accompli vos vœux, vous les avez effectués très exactement.
\VS{26}C'est pourquoi, écoutez la parole de Yahweh, vous tous de Juda, qui demeurez au pays d'Egypte~! Voici, je le jure par mon grand Nom, dit Yahweh, mon Nom ne sera plus invoqué par la bouche d'aucun homme de Juda, et dans tout le pays d'Egypte aucun ne dira~: Le Seigneur Yahweh est vivant~!
\VS{27}Voici, je veille sur eux pour leur mal et non pour leur bien~; et tous les hommes de Juda qui sont dans le pays d'Egypte seront consumés par l'épée et par la famine, jusqu'à ce qu'ils soient exterminés\FTNT{Da. 9:14.}.
\VS{28}Et ceux qui auront échappés à l'épée, retourneront du pays d'Egypte au pays de Juda en fort petit nombre. Mais tout le reste de Juda, tous ceux qui sont venus dans le pays d'Egypte pour y demeurer, sauront quelle est la parole qui s'accomplira, la mienne ou la leur.
\VS{29}Et ceci sera pour vous le signe, dit Yahweh, que je vous punirai dans ce lieu, afin que vous sachiez que mes paroles s'accompliront infailliblement pour votre malheur.
\VS{30}Ainsi parle Yahweh~: Voici, je livrerai Pharaon Hophra, roi d'Egypte, entre les mains de ses ennemis, entre les mains de ceux qui cherchent sa vie, comme j'ai livré Sédécias, roi de Juda, entre les mains de Nebucadnetsar, roi de Babylone, son ennemi, et qui cherchait sa vie.
\Chap{45}
\TextTitle{Yahweh explique son dessein à Baruc}
\VerseOne{}La parole que Jérémie, le prophète, adressa à Baruc, fils de Nérija, quand il écrivit dans un livre ces paroles, sous la dictée de Jérémie, la quatrième année de Jojakim, fils de Josias, roi de Juda. Il dit~:
\VS{2}Ainsi parle Yahweh, le Dieu d'Israël, sur toi, Baruc~:
\VS{3}Tu dis~: Malheur à moi~! Car Yahweh ajoute la tristesse à ma douleur~; je me suis lassé dans mon gémissement, et je ne trouve pas de repos.
\VS{4}Tu lui diras~: Ainsi parle Yahweh~: Voici, je vais détruire ce que j'ai bâti, et arracher ce que j'ai planté, à savoir tout ce pays.
\VS{5}Et toi, chercherais-tu de grandes choses~? Ne les cherche pas~! Car voici, je vais faire venir du mal sur toute chair, dit Yahweh~; je te donnerai ta vie pour butin, dans tous les lieux où tu iras.
\Chap{46}
\TextTitle{Prophétie contre l'Egypte}
\VerseOne{}La parole de Yahweh qui fut adressée à Jérémie, le prophète, sur les nations.
\VS{2}A l'égard de l'Egypte, contre l'armée de Pharaon Neco, roi d'Egypte, qui était près du fleuve de l'Euphrate, à Carkemisch, et qui fut battue par Nebucadnetsar, roi de Babylone, la quatrième année de Jojakim, fils de Josias, roi de Juda\FTNT{2 R. 24:7.}.
\VS{3}Préparez le bouclier et l'écu et approchez-vous pour la bataille~!
\VS{4}Attelez les chevaux, montez, cavaliers~! Présentez-vous avec vos casques, polissez vos lances, revêtez l'armure~!
\VS{5}D'où vient que je vois ceci~? Ils sont effrayés, ils reviennent en arrière~; leurs hommes vaillants sont battus~; ils s'enfuient avec précipitation sans regarder derrière eux… La frayeur les environne, dit Yahweh.
\VS{6}Que l'homme léger à la course ne s'enfuie pas, et que le fort ne se sauve pas\FTNT{Am. 2:14-16.}~! Ils sont renversés et tombés vers le nord, auprès du rivage du fleuve de l'Euphrate.
\VS{7}Qui est celui qui s'élève comme le Nil et dont les eaux sont agitées comme les fleuves~?
\VS{8}C'est l'Egypte. Elle s'élève comme le Nil et ses eaux agitées comme les fleuves~; et elle dit~: Je m'élèverai et je couvrirai la terre~; je détruirai la ville et ceux qui y habitent.
\VS{9}Montez, chevaux~! Agissez en insensés, chars~! Que les hommes vaillants sortent, ceux d'Ethiopie et de Puth qui manient le bouclier, et ceux de Lud qui manient et tendent l'arc\FTNT{Ez. 30:5-9~; Na. 3:9-10.}~!
\VS{10}Car c'est le jour du Seigneur, Yahweh des armées~; c'est un jour de vengeance, où il se venge de ses ennemis. L'épée dévore, elle se rassasie, elle s'enivre de leur sang. Car il y a des sacrifices pour le Seigneur, Yahweh des armées, dans le pays du nord, sur le fleuve de l'Euphrate\FTNT{Es. 34:5-6~; Ez. 39:17~; So. 1:7.}.
\VS{11}Monte en Galaad, prends du baume, vierge, fille de l'Egypte~! En vain tu multiplies les remèdes, il n'y a pas de guérison pour toi\FTNT{Ez. 30:21-25~; Na. 3:19.}.
\VS{12}Les nations apprennent ta honte, et tes cris remplissent la terre, car les hommes forts chancellent l'un sur l'autre, et ils tombent tous deux ensemble.
\VS{13}La parole que Yahweh prononça à Jérémie, le prophète, sur la venue de Nebucadnetsar, roi de Babylone, pour frapper le pays d'Egypte~:
\VS{14}Déclarez-le en Egypte, et publiez-le à Migdol, à Noph, et à Tachpanès et Dites~: Présente-toi, tiens-toi prêt car l'épée dévore ce qui est autour de toi~!
\VS{15}Pourquoi tes vaillants hommes sont-ils emportés~? Ils ne tiennent pas ferme, parce que Yahweh les pousse.
\VS{16}Il en a terrassé un grand nombre, et même chacun tombe sur son compagnon, et ils disent~: Levons-nous, retournons vers notre peuple, au pays de notre naissance, loin de l'épée de l'oppresseur~!
\VS{17}Là, ils s'écrient~: Pharaon, roi d'Egypte, n'est qu'un bruit~; il a laissé passer le temps fixé.
\VS{18}Je suis vivant~! dit le Roi, dont le Nom est Yahweh des armées~; comme le Thabor entre les montagnes, comme le Carmel qui s'avance dans la mer, ainsi viendra-t-il.
\VS{19}Ô fille, habitante de l'Egypte, fais tes bagages pour la captivité~! Car Noph sera un désert, elle sera brûlée, elle n'aura plus d'habitants.
\VS{20}L'Egypte est une très belle génisse… La destruction vient, elle vient du nord.
\VS{21}Même les mercenaires sont au milieu d'elle comme des veaux engraissés. Et eux aussi tournent le dos, ils fuient tous sans résister. Car le jour de leur malheur, le temps de leur châtiment est venu sur eux.
\VS{22}Elle sifflera comme un serpent~; car ils marcheront avec une puissante armée, ils viendront contre elle avec des haches, comme des bûcherons.
\VS{23}Ils couperont sa forêt, dit Yahweh, quoiqu'elle soit impénétrable~; parce que leur armée est en plus grand nombre que les sauterelles, on ne saurait la compter.
\VS{24}La fille de l'Egypte est confuse, elle est livrée entre les mains du peuple du nord.
\VS{25}Yahweh des armées, le Dieu d'Israël, dit~: Voici, je vais punir Amon de No, Pharaon, l'Egypte, ses dieux et ses rois, Pharaon et ceux qui se confient en lui.
\VS{26}Et je les livrerai entre les mains de ceux qui cherchent leur vie, entre les mains, dis-je, de Nebucadnetsar, roi de Babylone, et entre les mains de ses serviteurs~; mais après cela, l'Egypte sera habitée comme aux temps passés, dit Yahweh.
\VS{27}Et toi, Jacob, mon serviteur, ne crains pas~; ne t'épouvante pas Israël~! Car voici, je te sauverai de la terre lointaine, je sauverai ta postérité du pays de leur captivité~; Jacob reviendra, il sera en repos et en paix, et il n'y aura personne qui lui fasse peur.
\VS{28}Toi donc,Jacob, mon serviteur, ne crains pas~! dit Yahweh~; car je suis avec toi. Et même je consumerai entièrement toutes les nations parmi lesquelles je t'ai chassé, mais je ne te consumerai pas entièrement~; et je te châtierai avec justice, je ne te tiendrai pas tout à fait pour innocent.
\Chap{47}
\TextTitle{Prophétie contre la Philistie et la Phénicie}
\VerseOne{}La parole de Yahweh fut adressée à Jérémie, le prophète, contre les Philistins, avant que Pharaon frappe Gaza.
\VS{2}Ainsi parle Yahweh~: Voici des eaux montent du nord, elles sont comme un torrent qui déborde~; elles inondent le pays et ce qu'il contient, les villes et leurs habitants. Les hommes poussent des cris, et tous les habitants du pays se lamentent,
\VS{3}à cause du bruit des battements de sabots de ses puissants chevaux, du bruit de ses chars et au son de ses roues~; les pères ne se tournent pas vers leurs fils, tant les mains sont affaiblies,
\VS{4}parce que le jour vient où seront détruits tous les Philistins, exterminés tout le reste de ceux qui servaient de secours à Tyr et à Sidon~; car Yahweh va détruire les Philistins, les restes de l'île de Caphtor.
\VS{5}Gaza est devenue chauve, Askalon est perdue, le reste de leur plaine aussi. Jusqu'à quand te feras-tu des incisions~?
\VS{6}Ah~! Epée de Yahweh, quand te reposeras-tu~? Rentre dans ton fourreau, repose-toi, et sois tranquille~!
\VS{7}Mais comment te reposerais-tu~? Car Yahweh lui donne ses ordres, il l'a assignée contre Askalon et contre le rivage de la mer.
\Chap{48}
\TextTitle{Prophétie sur Moab}
\VerseOne{}Sur Moab. Ainsi parle Yahweh des armées, le Dieu d'Israël~: Malheur à Nebo, car elle est dévastée~! Kirjathaïm est honteuse, elle est prise~; Misgab est honteuse et brisée.
\VS{2}Moab ne se glorifiera plus à Hesbon, car on a machiné du mal contre elle en disant~: Allons, exterminons-la, qu'elle ne soit plus une nation~! Toi aussi, Madmen, tu seras détruite~; l'épée te poursuivra.
\VS{3}Il y a des clameurs de détresse qui viennent de Choronaïm~; c'est un ravage, une grande ruine.
\VS{4}Moab est brisé~! On entend les cris des plus jeunes.
\VS{5}Pleurs sur pleurs s'élèveront à la montée de Luchith, car on entendra à la descente de Choronaïm\FTNT{Es. 15:5.}. Ceux qui crieront à cause des plaies que les ennemis leur auront faites.
\VS{6}Fuyez, dira-t-on, sauvez vos vies, et soyez comme de la bruyère dans le désert~!
\VS{7}Car, parce que tu t'es confié dans tes ouvrages, et dans tes trésors, tu seras pris, et Kemosch sortira pour être transporté avec ses prêtres et ses chefs\FTNT{Es. 46:1-7.}.
\VS{8}Et le dévastateur entrera dans toutes les villes, et aucune ville n'échappera~; la vallée périra et la plaine sera détruite, comme Yahweh l'a dit.
\VS{9}Donnez des ailes à Moab, et qu'il parte en volant~! Ses villes seront réduites en désert, elles n'auront plus d'habitants.
\VS{10}Maudit soit celui qui fait l'œuvre de Yahweh avec paresse, maudit soit celui qui garde son épée pour répandre le sang~!
\VS{11}Moab était tranquille depuis sa jeunesse, il reposait sur sa lie, il n'était pas vidé de vase en vase, et il n'allait pas en captivité. C'est pourquoi sa saveur lui est restée, et son odeur ne s'est pas changée.
\VS{12}Mais voici, les jours viennent, dit Yahweh, où je lui enverrai des gens qui l'enlèveront, qui videront ses vases, et qui briseront ses outres.
\VS{13}Moab aura honte à cause de Kemosch, comme la maison d'Israël a eu honte à cause de Béthel, qui était sa confiance.
\VS{14}Comment dites-vous~: Nous sommes de vaillants hommes, des soldats prêts à combattre~?
\VS{15}Moab est dévasté, et chacune de ses villes monte en fumée, l'élite de sa jeunesse est descendue pour être égorgée, dit le Roi, dont le nom est Yahweh des armées.
\VS{16}La calamité de Moab est proche, son malheur avance à grands pas.
\VS{17}Vous tous qui êtes autour de lui, soyez-en émus à compassion, et vous tous qui connaissez son nom, dites~: Comment a été rompue cette forte verge et ce sceptre d'honneur~? 
\VS{18}Toi qui te tiens chez la fille de Dibon, descends de ta gloire, et assieds-toi dans un lieu desséché~! Car le dévastateur de Moab monte contre toi, il détruit tes forteresses.
\VS{19}Habitante d'Aroër, tiens-toi sur le chemin, et regarde~! Interroge celui qui s'enfuit, celui qui s'échappe, et dis~: Qu'est-il arrivé~?
\VS{20}Moab est rendu honteux car il est brisé. Poussez des gémissements et des cris\FTNT{Es. 15:5~; Es. 16:7.}~! Rapportez dans Arnon que Moab est dévasté~!
\VS{21}Et que le jugement est venu sur le pays de la plaine, sur Holon, sur Jahats, sur Méphaath,
\VS{22}et sur Dibon, sur Nebo, sur Beth-Diblathaïm,
\VS{23}et sur Kirjathaïm, sur Beth-Gamul, sur Beth-Meon,
\VS{24}et sur Kerijoth, sur Botsra, sur toutes les villes du pays de Moab, éloignées et proches.
\VS{25}La force de Moab est abattue, et son bras est brisé, dit Yahweh.
\VS{26}Enivrez-le car il s'est élevé contre Yahweh~! Moab se vautrera dans le vin qu'il aura rendu et deviendra aussi un sujet de moquerie~!
\VS{27}Car ô Moab~! Israël n'a-t-il pas été pour toi un objet de moquerie~? Avait-il été trouvé parmi les voleurs, pour que tu ne dises des paroles qu'en secouant la tête~?
\VS{28}Habitants de Moab, quittez les villes et demeurez dans les rochers~! Soyez comme les colombes qui font leur nid aux côtés de l'entrée des cavernes~!
\VS{29}Nous avons appris l'extrême orgueil de Moab, son arrogance, sa fierté, et son cœur hautain\FTNT{Es. 16:6~; So. 2:9-10.}.
\VS{30}Je connais son orgueil, dit Yahweh~; mais il n'en sera pas ainsi~; je connais ceux sur lesquels il s'appuie~; ils ne font rien de droit. 
\VS{31}Je hurle donc à cause de Moab, même je crie à cause de Moab tout entier~; on gémit sur les gens de Kir-Hérès.
\VS{32}Ô vignoble de Sibma, je pleure sur toi du pleur de Jaezer~; tes rameaux allaient au-delà de la mer, ils atteignaient la mer de Jaezer~; le dévastateur s'est jeté sur tes fruits d'été et sur ta vendange.
\VS{33}L'allégresse et la joie se sont retirées loin des campagnes et du pays de Moab~; j'ai fait cesser le vin des cuves~; on ne foule plus gaîment au pressoir~; il y a des cris de guerre, et non des cris de joie\FTNT{Es. 16:10.}.
\VS{34}Les cris de Hesbon parviennent jusqu'à Elealé, et ils font entendre leurs cris jusqu'à Jahats, même depuis Tsoar jusqu'à Choronaïm, jusqu'à Eglath-Schelischija~; car les eaux de Nimrim seront aussi réduites en désolation.
\VS{35}Je ferais qu'il n'y aura plus en Moab, dit Yahweh, aucun qui offre sur les hauts lieux et celui qui brûle de l'encens à ses dieux. 
\VS{36}C'est pourquoi mon cœur mènera un bruit sur Moab, comme des flûtes~; mon cœur mènera un bruit comme des flûtes sur les hommes de Kir-Hérès, parce que tous les biens qu'ils ont acquis ont péri.
\VS{37}Car toutes les têtes sont chauves, toutes les barbes sont coupées~; il y a des incisions sur toutes les mains, et sur les reins des sacs.
\VS{38}Il y aura des lamentations sur tous les toits de Moab et dans ses places, parce que j'aurai brisé Moab comme un vase auquel on ne prend nul plaisir, dit Yahweh.
\VS{39}Hurlez, en disant~: Comment a-t-il été mis en pièces, brisé~! Comment Moab a-t-il tourné honteusement le dos~! Car Moab sera un objet de moquerie et de frayeur pour tous ceux qui sont autour de lui.
\VS{40}Car ainsi parle Yahweh~: Voici, il vole comme un aigle, et il étend ses ailes sur Moab.
\VS{41}Kerijoth est prise, les forteresses sont saisies, et le cœur des hommes forts de Moab est en ce jour comme le cœur d'une femme qui est en travail.
\VS{42}Et Moab sera exterminé, il ne sera plus un peuple, parce qu'il s'est élevé contre Yahweh.
\VS{43}Habitant de Moab, la frayeur, la fosse, et le filet sont sur toi~! dit Yahweh.
\VS{44}Celui qui s'enfuit à cause de la frayeur tombe dans la fosse, et celui qui remonte de la fosse sera pris au filet~; car je fais venir sur lui, sur Moab, l'année de son châtiment, dit Yahweh\FTNT{Es. 24:18.}.
\VS{45}Ils se sont arrêtés à l'ombre de Hesbon, voulant éviter la force~; mais le feu sort de Hesbon, une flamme du milieu de Sihon~; elle dévore les flancs de Moab et le sommet de la tête des fils du tumulte\FTNT{No. 21:28.}.
\VS{46}Malheur à toi, Moab~! Le peuple de Kemosch est perdu~! Car tes fils sont enlevés et emmenés captifs, et tes filles ont été emmenées captives.
\VS{47}Toutefois je ramènerai et mettrai en repos les captifs de Moab, aux derniers jours\FTNT{Ge. 49:1-2.}, dit Yahweh. Là est le jugement de Moab.
\Chap{49}
\TextTitle{Prophétie sur Ammon}
\VerseOne{}Sur les fils d'Ammon. Ainsi parle Yahweh~: Israël n'a-t-il pas de fils~? N'a-t-il pas d'héritier~? Pourquoi donc Malcom hérite-t-il du pays de Gad, et pourquoi son peuple demeure-t-il dans ses villes~?
\VS{2}C'est pourquoi voici, les jours viennent, dit Yahweh, où je ferai entendre le cri de guerre contre Rabbath des fils d'Ammon~; et elle sera réduite en un monceau de ruines, et les villes de son ressort seront brûlées par le feu. Israël possédera ceux qui l'auront possédé, dit Yahweh.
\VS{3}Hurle, ô Hesbon, car Aï est dévastée~! Poussez des cris, filles de Rabba, ceignez-vous de sacs, lamentez-vous, courez ça et là le long des murailles~! Car Malcom s'en va en captivité avec ses prêtres et ses chefs.
\VS{4}Pourquoi te glorifies-tu de tes vallées~? Ta vallée se fond, fille rebelle, qui te confiais dans tes trésors et disait~: Qui viendra contre moi~?
\VS{5}Voici, je fais venir sur toi la terreur, dit le Seigneur, Yahweh des armées, de tous les alentours~; vous serez chassés chacun çà et là, et il n'y aura personne qui rassemblera les fuyards.
\VS{6}Mais après cela, je ramènerai les captifs des fils d'Ammon, dit Yahweh.
\TextTitle{Prophétie sur Edom}
\VS{7}Sur Edom. Ainsi parle Yahweh des armées~: N'y a-t-il plus de sagesse dans Théman~? Le conseil a-t-il manqué aux hommes intelligents~? Leur sagesse s'est-elle évanouie\FTNT{Ab. 1:8.}~?
\VS{8}Fuyez, tournez le dos, vous habitants de Dedan, qui avez fait des creux pour y habiter~! Car je fais venir sa calamité sur Esaü, le temps où je le visiterai.
\VS{9}Si des vendangeurs entrent chez toi, ne laissent-ils rien à grappiller~? Si des voleurs viennent de nuit, ils ne pillent que ce qu'ils peuvent.
\VS{10}Mais je dépouillerai Esaü, je découvrirai ses lieux secrets, il ne pourra se cacher~; sa postérité, ses frères, et ses voisins, périront, et il ne sera plus.
\VS{11}Laisse tes orphelins, je les ferai vivre, et que tes veuves se confient en moi~!
\VS{12}Car ainsi parle Yahweh~: Voici, ceux dont le jugement n'était pas de boire la coupe, la boiront certainement~; et toi, tu resterais impuni~! Tu ne resteras pas impuni, mais tu la boiras certainement.
\VS{13}Car je le jure par moi-même, dit Yahweh, que Botsra sera réduite en désolation, en opprobre, en désert et en malédiction, et que toutes ses villes deviendront des ruines éternelles.
\VS{14}J'ai entendu de Yahweh une nouvelle, et un messager a été envoyé parmi les nations~: Assemblez-vous et venez contre elle~! Levez-vous pour la guerre~!
\VS{15}Car voici, je te rendrai petit entre les nations, méprisé entre les hommes.
\VS{16}Mais ta présomption et l'orgueil de ton cœur t'ont séduit, toi qui habites dans le creux des rochers, et qui occupes le sommet des collines. Quand tu aurais élevé ton nid comme l'aigle, je t'en ferai descendre, dit Yahweh.
\VS{17}Et Edom sera réduite en désolation~; quiconque passera près de lui sera étonné, et sifflera à cause de toutes ses plaies.
\VS{18}Il n'y demeurera personne, a dit Yahweh, aucun fils d'homme n'y séjournera\FTNT{Ge. 19:25~; Am. 4:11.}…, comme dans la destruction de Sodome et de Gomorrhe, et de leurs villes voisines.
\VS{19}Voici, il monte comme un lion à cause du débordement du Jourdain, vers la demeure du pays rude~; et après l'avoir fait reposer, je le ferai courir hors d'Edom. Et qui est l'homme choisi que j'assignerai sur elle~? Car qui est semblable à moi~? Et qui me déterminera le temps~? Et qui sera le pasteur qui tiendra ferme contre moi\FTNT{Job 41:1.} 
\VS{20}C'est pourquoi écoutez la résolution que Yahweh a prise contre Edom, et les desseins qu'il a formés contre les habitants de Théman~! Assurément, les plus petits du troupeau les traîneront par terre, assurément on réduira en désolation leur demeure sur eux.
\VS{21}La terre tremble au bruit de leur chute~; le bruit de leur cri se fait entendre jusqu'à la Mer Rouge…
\VS{22}Voici, il montera comme un aigle, il volera, et il étendra ses ailes sur Botsra, et le cœur des hommes forts d'Edom sera en ce jour-là comme le cœur d'une femme qui est en travail.
\TextTitle{Prophétie sur Damas}
\VS{23}Sur Damas. Hamath et Arpad sont honteuses parce qu'elles ont entendu de très mauvaises nouvelles, elles tremblent~; il y a une tourmente dans la mer qui ne peut se calmer.
\VS{24}Damas est défaillante, elle se tourne pour fuir, et la panique la saisit~; l'angoisse et les douleurs la saisissent comme une femme qui enfante.
\VS{25}Comment n'est-elle pas abandonnée, la ville de louange, la cité de ma joie~?
\VS{26}Car ses jeunes gens tomberont dans les places, et tous ses hommes de guerre périront en ce jour-là, dit Yahweh des armées.
\VS{27}Et je mettrai le feu à la muraille de Damas, qui dévorera les palais de Ben-Hadad.
\TextTitle{Prophétie sur Kédar (les Arabes) et Hatsor}
\VS{28}Sur Kédar et les royaumes de Hatsor, que Nebucadnetsar, roi de Babylone, frappera. Ainsi parle Yahweh~: Levez-vous, montez vers Kédar et détruisez les fils d'orient~!
\VS{29}On prendra leurs tentes et leurs troupeaux, et on prendra leurs tapis, tous leurs bagages et leurs chameaux, et l'on jettera de toutes parts contre eux des cris de terreur.
\VS{30}Fuyez, écartez-vous tant que vous pourrez, vous habitants de Hatsor, qui avez fait des creux pour y habiter~! Dit Yahweh~; car Nebucadnetsar, roi de Babylone, a formé un dessein contre vous, il a pris une résolution contre vous.
\VS{31}Levez-vous, montez vers la nation tranquille qui habite en sécurité, dit Yahweh~; elle n'a ni portes ni barres, elle habite seule\FTNT{Ez. 38:11.}.
\VS{32}Et leurs chameaux seront au pillage, et la multitude de leur bétail sera une proie~; et je les disperserai à tout vent, vers ceux qui se coupent le coin de la barbe, et je ferai venir de tous les côtés leur calamité, dit Yahweh.
\VS{33}Et Hatsor sera le repaire des serpents et un désert pour toujours~; personne n'y habitera, et aucun fils d'homme n'y séjournera.
\TextTitle{Prophétie sur Elam}
\VS{34}La parole de Yahweh qui fut adressée à Jérémie le prophète, contre Elam, au commencement du règne de Sédécias, roi de Juda, en disant~:
\VS{35}Ainsi parle Yahweh des armées~: Voici, je vais briser l'arc d'Elam, qui est leur principale force\FTNT{Ez. 32:24-27.}.
\VS{36}Et je ferai venir contre Elam les quatre vents des quatre extrémités des cieux, et je les disperserai par tous ces vents~; et il n'y aura pas une nation où ne viennent ceux qui seront chassés d'Elam éternellement.
\VS{37}Et je ferai trembler les habitants d'Elam devant leurs ennemis, et devant ceux qui cherchent leur vie~; je ferai venir du mal sur eux, l'ardeur de ma colère, dit Yahweh. J'enverrai l'épée après eux, jusqu'à ce que je les aie consumés.
\VS{38}Je mettrai mon trône dans Elam, et j'en détruirai les rois et les chefs, dit Yahweh.
\VS{39}Mais il arrivera qu'aux derniers jours\FTNT{Ge. 49:1-2.}, je ramènerai les captifs d'Elam, dit Yahweh.
\Chap{50}
\TextTitle{Prophétie sur Babylone}
\VerseOne{}La parole que Yahweh prononça contre Babylone, et contre le pays des Chaldéens, par le moyen de Jérémie le prophète~:
\VS{2}Annoncez-le parmi les nations, entendez-le, levez une bannière~! Entendez-le, ne le cachez pas~! Dites~: Babylone est prise~! Bel est confus, Merodac est brisé~! Ses idoles sont confuses et brisées\FTNT{Es. 46:1.}, et leurs dieux de fiente sont brisés~!
\VS{3}Car une nation monte contre elle du nord~; elle mettra son pays en désolation, et il n'y aura plus personne qui y habite. Les hommes et les bêtes fuient, ils s'en vont.
\VS{4}En ces jours-là, et en ce temps-là, dit Yahweh, les enfants d'Israël viendront, eux et les enfants de Juda ensemble~; ils marcheront allant et pleurant, et cherchant Yahweh, leur Dieu.
\VS{5}Ils s'informeront du chemin de Sion vers lequel leurs faces seront tournées, disant~: Venez et joignons-nous à Yahweh par une alliance éternelle qui ne sera jamais oubliée.
\VS{6}Mon peuple était comme un troupeau de brebis perdues~; leurs pasteurs les égaraient, et les rendaient errantes par les montagnes~; elles allaient de montagne en colline, oubliant leur bercail\FTNT{Ez. 34:5-6~; Za. 10:2~; Mt. 9:36.}.
\VS{7}Tous ceux qui les trouvaient les dévoraient, et leurs ennemis disaient~: Nous ne sommes coupables d'aucun mal, parce qu'ils ont péché contre Yahweh, contre la demeure de la justice, contre Yahweh, l'espérance de leurs pères.
\VS{8}Fuyez hors de Babylone, et sortez du pays des Chaldéens, et soyez comme des boucs qui vont devant le troupeau\FTNT{Es. 48:20~; 2 Co. 6:17~; Ap. 18:4.}~!
\VS{9}Car voici, je vais susciter et faire monter contre Babylone une multitude de grandes nations du pays du nord, qui se rangeront en bataille contre elle, de sorte qu'elle sera prise. Leurs flèches seront comme celles d'un homme puissant, qui ne fait que détruire, et qui ne retourne point à vide.\FTNT{Es. 13:18.}.
\VS{10}Et la Chaldée sera abandonnée au pillage~; tous ceux qui la pilleront seront rassasiés, dit Yahweh~;
\VS{11}parce que vous vous êtes réjouis, parce que vous vous êtes égayés, en ravageant mon héritage, parce que vous avez bondi comme une génisse qui foule l'herbe, et que vous avez henni comme de puissants chevaux.
\VS{12}Votre mère est devenue fort honteuse, celle qui vous a enfantés a rougit~; voici, elle sera la dernière entre les nations, elle sera un désert, un pays sec, une lande.
\VS{13}Elle ne sera plus habitée à cause de la colère de Yahweh, elle ne sera plus qu'une désolation. Quiconque passera près de Babylone sera étonné et sifflera à cause de toutes ses plaies.
\VS{14}Rangez-vous en bataille contre Babylone, mettez-vous tout autour vous tous qui tendez l'arc~! Tirez contre elle et n'épargnez pas les flèches~! Car elle a péché contre Yahweh.
\VS{15}Poussez des cris de guerre contre elle tout alentour~! Elle tend les mains~; ses fondements tombent~; ses murs sont renversés. Car c'est ici la vengeance de Yahweh. Vengez-vous sur elle~! Faites-lui comme elle a fait\FTNT{Ab. 1:15~; Ps. 137:8~; Lu. 6:38.}~!
\VS{16}Retranchez de Babylone le semeur, et celui qui manie la faucille au temps de la moisson~! Que chacun se tourne vers son peuple, et que chacun s'enfuie vers son pays à cause de l'épée de l'oppresseur.
\VS{17}Israël est comme une brebis égarée que les lions ont chassée. Le roi d'Assyrie l'a dévorée le premier, mais ce dernier-ci, Nebucadnetsar, roi de Babylone, lui a brisé les os.
\VS{18}C'est pourquoi ainsi parle Yahweh des armées, le Dieu d'Israël~: Voici, je vais punir le roi de Babylone et son pays, comme j'ai puni le roi d'Assyrie\FTNT{Es. 37:36~; 2 R. 19:35.}.
\VS{19}Et je ramènerai Israël dans sa demeure~; il paîtra au Carmel et au Basan, et son âme se rassasiera sur la montagne d'Ephraïm et de Galaad.
\VS{20}En ces jours-là, et en ce temps-là, dit Yahweh, on cherchera l'iniquité d'Israël, mais il n'y en aura pas~; et les péchés de Juda, mais ils ne seront pas trouvés~; car je pardonnerai au reste que j'aurai fait demeurer.
\VS{21}Monte contre le pays doublement rebelle, contre lui et contre les habitants destinés à la visitation~; dévaste et détruis entièrement derrière eux, dit Yahweh, et fais selon tout ce que je t'ai ordonné.
\VS{22}Un cri de guerre est dans le pays avec une grande ruine.
\VS{23}Comment est-il mis en pièces et rompu, le marteau de toute la terre~! Comment Babylone est-elle réduite en sujet d'étonnement parmi les nations~! 
\VS{24}Je t'ai tendu un piège, et tu as été prise, ô Babylone, et tu n'en savais rien~; tu as été trouvée et même attrapée, parce que tu as lutté contre Yahweh.
\VS{25}Yahweh a ouvert son arsenal et en a sorti les armes de sa colère~; car c'est ici l'œuvre du Seigneur, de Yahweh des armées, dans le pays des Chaldéens.
\VS{26}Venez contre elle de toutes parts, ouvrez ses granges~; entassez-la comme des gerbes, détruisez-la à la façon de l'interdit, et qu'il n'en reste rien~!
\VS{27}Egorgez tous ses taureaux et qu'ils descendent à l'abattage~! Malheur à eux~! Car le jour est venu, le temps de leur visitation.
\VS{28}On entend la voix de ceux qui s'enfuient, de ceux qui sont échappés du pays de Babylone pour annoncer dans Sion la vengeance de Yahweh, notre Dieu, la vengeance de son temple~!
\VS{29}Appelez les archers contre Babylone~; vous tous qui tendez l'arc, campez-vous contre elle tout alentour~; que personne n'échappe~; rendez-lui selon ses œuvres~; faites-lui selon tout ce qu'elle a fait~; car elle s'est fièrement élevée contre Yahweh, contre le Saint d'Israël\FTNT{Es. 13:11~; Joë. 3:4-9~; La. 1:22.}.
\VS{30}C'est pourquoi, ses jeunes hommes tomberont dans les places, et tous ses hommes de guerre périront en ce jour-là, dit Yahweh.
\VS{31}Voici, j'en veux à toi, orgueilleuse~! Dit le Seigneur, Yahweh des armées~; car ton jour est venu, le temps où je te visiterai.
\VS{32}L'orgueilleuse chancellera et tombera, et il n'y aura personne pour la relever~; je mettrai le feu à ses villes, et il dévorera tous ses environs.
\VS{33}Ainsi parle Yahweh des armées~: Les enfants d'Israël et les enfants de Juda sont ensemble opprimés~; tous ceux qui les ont emmenés captifs les retiennent, et refusent de les laisser aller.
\VS{34}Leur Rédempteur est fort, son nom est Yahweh des armées~; il plaidera avec chaleur leur cause, pour donner du repos au pays, et mettre dans le trouble les habitants de Babylone.
\VS{35}L'épée est sur les Chaldéens, dit Yahweh, et sur les habitants de Babylone, sur ses chefs et sur ses sages~!
\VS{36}L'épée est tirée contre ses devins~; ils deviendront insensés~! L'épée est sur ses hommes forts et ils en seront épouvantés~!
\VS{37}L'épée est sur ses chevaux, et sur ses chars et sur les gens de toute espèce qui sont au milieu d'elle, et ils deviendront comme des femmes~! L'épée est sur ses trésors, et ils seront pillés~!
\VS{38}La sécheresse sera sur ses eaux, et elles seront mises à sec~; parce que c'est un pays d'images taillées, et ils agiront en insensés à l'égard de leurs dieux qui les épouvantent\FTNT{Es. 2:8.}.
\VS{39}C'est pourquoi les bêtes sauvages des déserts y habiteront avec les chacals, et les autruches y habiteront aussi~; et elle ne sera plus jamais habitée, et on n'y demeurera plus jamais.
\VS{40}Comme dans la destruction que Dieu fit de Sodome et de Gomorrhe, et de ses villes voisines, dit Yahweh, elle ne sera plus habitée par des hommes et aucun fils d'homme n'y séjournera.
\VS{41}Voici, un peuple vient du nord, une grande nation et plusieurs rois se réveillent des extrémités de la terre.
\VS{42}Ils saisissent l'arc et le javelot~; ils sont cruels, et ils n'ont pas de compassion~; leur voix mugit comme la mer, et ils sont montés sur des chevaux, chacun d'eux est rangé en bataille comme un seul homme, contre toi, fille de Babylone~!
\VS{43}Le roi de Babylone en a entendu la nouvelle, et ses mains sont devenues lâches~; l'angoisse l'a saisit, une douleur comme celle qui enfante…
\VS{44}Voici, il montera comme un lion à cause du débordement du Jourdain, vers la demeure du pays rude. Et après que je les aurai fait reposer, soudain je les ferai courir hors de la Chaldée. Et qui est l'homme choisi que j'assignerai sur elle~? Car qui est semblable à moi~? Et qui me déterminera le temps~? Et qui sera le pasteur qui tiendra ferme contre moi~?
\VS{45}C'est pourquoi écoutez la résolution que Yahweh a prise contre Babylone, et les desseins qu'il a formés contre le pays des Chaldéens~! Assurément, les plus petits du troupeau les traîneront par terre, assurément on réduira en désolation leur demeure sur eux.
\VS{46}Au bruit de la prise de Babylone la terre est ébranlée, et il y a un cri, entendu parmi les nations.
\Chap{51}
\TextTitle{Le jugement de Babylone par Yahweh}
\VerseOne{}Ainsi parle Yahweh~: Voici, je vais faire lever un vent de destruction contre Babylone, et contre ceux qui habitent au cœur du Royaume de ceux qui s'élèvent contre moi.
\VS{2}J'enverrai contre Babylone des vanneurs qui la vanneront, et qui videront son pays~; car de tous côtés ils seront contre elle, au jour du malheur.
\VS{3}Qu'on bande l'arc contre celui qui bande son arc, et contre celui qui s'élève dans son armure~; et n'épargnez pas ses jeunes hommes, exterminez à la façon de l'interdit toute son armée.
\VS{4}Les blessés à mort tomberont dans le pays des Chaldéens, et ceux qui sont transpercés tomberont dans ses rues.
\VS{5}Car Israël et Juda ne sont pas abandonnés de leur Dieu, de Yahweh des armées~; quoique leur pays ai été trouvé par le Saint d'Israël plein de crimes.
\VS{6}Fuyez hors de Babylone et que chacun sauve sa vie. Ne soyez point exterminés dans son iniquité~; car c'est le temps de la vengeance de Yahweh~; il lui rend ce qu'elle a mérité.
\VS{7}Babylone était comme une coupe d'or dans la main de Yahweh, enivrant toute la terre~; les nations ont bu de son vin~: C'est pourquoi les nations sont devenues insensées.
\VS{8}Babylone est tombée\FTNT{Ap. 18.} en un instant, elle est brisée~! Hurlez sur elle, prenez du baume pour sa douleur~: Peut-être qu'elle guérira.
\VS{9}Nous avons pansé Babylone, mais elle n'est pas guérie. Laissez-la et allons-nous-en chacun dans son pays~; car son jugement atteint les cieux et s'élève jusqu'aux nues.
\VS{10}Yahweh a mis en évidence notre justice~; venez et racontons dans Sion l'œuvre de Yahweh, notre Dieu.
\VS{11}Aiguisez les flèches et empoignez à pleines mains les boucliers~! Yahweh a réveillé l'esprit des rois de Médie, car sa pensée est contre Babylone pour la détruire, parce que c'est ici la vengeance de Yahweh, la vengeance de son temple.
\VS{12}Elevez une bannière sur les murs de Babylone, renforcez la garnison, posez les gardes, préparez les embuscades~; car Yahweh a formé un dessein, même il a fait ce qu'il a dit contre les habitants de Babylone.
\VS{13}Tu étais sur plusieurs eaux, abondantes en trésors~; ta fin est venue et ton gain déshonnête est à son comble.
\VS{14}Yahweh des armées a juré par lui-même, en disant~: Certainement je te remplirai d'hommes comme de sauterelles, et ils pousseront un cri contre toi.
\VS{15}C'est lui qui a fait la terre par sa puissance, et qui a fondé le monde habitable par sa sagesse, et qui a étendu les cieux par son intelligence\FTNT{Ge. 1:1~; Es. 40:22~; Ps. 104:2~; Job 9:8.}.
\VS{16}Sitôt qu'il fait entendre sa voix, il y a un tumulte d'eaux dans les cieux, il fait monter les vapeurs des extrémités de la terre, il fait les éclairs et la pluie, et il fait sortir le vent de ses réservoirs.
\VS{17}Tout homme devient stupide par sa connaissance, tout fondeur est honteux par les images taillées~; car ses idoles en métal fondu ne sont que mensonge, il n'y a pas de souffle en elles.
\VS{18}Elles ne sont que vanité et une œuvre de tromperie~; elles périront au temps de leur visitation.
\VS{19}La portion de Jacob n'est pas comme ces choses-là~; car c'est celui qui a tout formé, et Israël est la tribu de son héritage~; son nom est Yahweh des armées.
\VS{20}Tu as été pour moi un marteau et des instruments de guerre~; par toi j'ai mis en pièces les nations, et par toi j'ai détruit les royaumes.
\VS{21}Et par toi j'ai mis en pièces le cheval et son cavalier~; et par toi j'ai mis en pièces le char et celui qui était monté dessus.
\VS{22}Et par toi j'ai mis en pièces l'homme et la femme~; et par toi j'ai mis en pièces le vieillard et le jeune garçon~; et par toi j'ai brisé le jeune homme et la vierge.
\VS{23}Et par toi j'ai mis en pièces le pasteur et son troupeau~; et par toi j'ai mis en pièces le laboureur et ses bœufs accouplés~; et par toi j'ai mis en pièces les gouverneurs et les magistrats.
\VS{24}Mais je rendrai à Babylone, et à tous les habitants de la Chaldée, tout le mal qu'ils ont fait à Sion sous vos yeux, dit Yahweh\FTNT{La. 1:21.}.
\VS{25}Voici, j'en veux à toi, montagne de destruction, dit Yahweh, qui détruis toute la terre~; et j'étendrai ma main sur toi, et je te roulerai du haut des rochers, et je ferai de toi une montagne embrasée.
\VS{26}Et on ne pourra prendre de toi aucune pierre pour la placer à l'angle de l'édifice, ni aucune pierre pour servir de fondement, car tu seras réduite en désolations perpétuelles, dit Yahweh\FTNT{Es. 13:19-20.}…
\VS{27}Elevez une bannière dans le pays, sonnez du shofar parmi les nations~; préparez les nations contre elle~; appelez contre elle les royaumes d'Ararat, de Minni et d'Aschkenaz~; établissez contre elle des chefs, faites monter ses chevaux comme des sauterelles hérissées~!
\VS{28}Préparez contre elle les nations, les rois de Médie, ses gouverneurs et tous ses magistrats, et tout le pays sous leur domination\FTNT{Es. 13:17.}.
\VS{29}Et la terre tremble et est en travail~; car les desseins de Yahweh s'accomplissent contre Babylone, pour réduire le pays en désolation, de sorte qu'il n'y ait pas d'habitant\FTNT{Es. 13:14~; Joë. 3:16.}.
\VS{30}Les hommes forts de Babylone ont cessé de combattre, ils sont restés dans les forteresses, leur force est épuisée, ils sont devenus comme des femmes~; on a mis le feu à leurs demeures, et leurs barres sont brisées.
\VS{31}Le courrier viendra à la rencontre du courrier, et le messager viendra à la rencontre du messager, pour annoncer au roi de Babylone que sa ville est prise par tous les côtés,
\VS{32}et que ses gués sont saisis, et que ses marais sont brûlés par le feu, et que les hommes de guerre sont épouvantés.
\VS{33}Car ainsi parle Yahweh des armées, le Dieu d'Israël~: La fille de Babylone est comme une aire~; il est temps qu'elle soit foulée~; encore un peu, et le temps de sa moisson viendra.
\VS{34}Nebucadnetsar, roi de Babylone, dira~: Jérusalem m'a dévorée et m'a brisée~; il m'a mise dans le même état qu'un vase qui ne sert à rien~; il m'a engloutie comme un dragon~; il a rempli son ventre de mes délices, il m'a chassée au loin.
\VS{35}Que la violence envers moi et ma chair déchirée retombe sur Babylone~! dit l'habitante de Sion. Que mon sang retombe sur les habitants de la Chaldée~! dit Jérusalem.
\VS{36}C'est pourquoi ainsi parle Yahweh~: Voici, je vais plaider ta cause, et je ferai la vengeance pour toi~; je dessécherai sa mer, et je ferai tarir sa source.
\VS{37}Et Babylone sera réduite en monceau, en repaire de serpents, en étonnement, et en opprobre, sans que personne n'y habite.
\VS{38}Ils rugiront ensemble comme des lions et pousseront des cris comme des lionceaux.
\VS{39}Je les ferai échauffer dans leurs festins, et les enivrerai, afin qu'ils se réjouissent, et qu'ils dorment d'un sommeil éternel, et qu'ils ne se réveillent plus, dit Yahweh.
\VS{40}Je les ferai descendre comme des agneaux à la boucherie, et comme on y mène les moutons avec les boucs.
\VS{41}Comment a été prise Schéschac, et comment a été saisie celle qui était la louange de toute la terre~? Comment Babylone a-t-elle été réduite en désolation parmi les nations~!
\VS{42}La mer est montée sur Babylone, elle a été couverte de la multitude de ses flots\FTNT{Es. 8:8~; Ez. 26:3-19~; Lu. 21:25.}.
\VS{43}Ses villes sont devenues une désolation, une terre sèche et de landes, un pays où personne ne demeure, et où il ne passe pas un fils d'homme.
\VS{44}Je punirai aussi Bel à Babylone, je ferai sortir de sa bouche ce qu'il a englouti, et les nations n'aborderont plus vers lui. Le mur même de Babylone est tombé~!
\VS{45}Mon peuple, sortez du milieu d'elle, et que chacun sauve sa vie de l'ardeur de la colère de Yahweh~;
\VS{46}de peur que votre cœur ne s'amollisse, et que vous n'ayez peur des nouvelles qu'on entendra dans tout le pays. Car des nouvelles viendront une année, et après cela d'autres nouvelles une autre année, et il y aura violence dans le pays, et dominateur sur dominateur.
\VS{47}C'est pourquoi voici, les jours viennent où je punirai les images taillées de Babylone, et tout son pays sera honteux~; tous ses blessés à mort tomberont au milieu d'elle.
\VS{48}Les cieux et la terre, et tout ce qui y est, pousseront des cris de joie contre Babylone, parce qu'il viendra du nord des dévastateurs contre elle, dit Yahweh.
\VS{49}Et comme Babylone a fait tomber les blessés à mort d'Israël, ainsi les tués de tout le pays tomberont à Babylone.
\VS{50}Vous qui avez échappé à l'épée, allez, ne vous arrêtez pas~; souvenez-vous de Yahweh dans ces pays éloignés où vous êtes, et que Jérusalem vous revienne au cœur~!
\VS{51}Mais vous direz~: Nous sommes honteux des reproches que nous avons entendus~; la confusion a couvert nos faces, en ce que les étrangers sont venus contre les sanctuaires de la maison de Yahweh.
\VS{52}C'est pourquoi, voici, les jours viennent, dit Yahweh, où je visiterai ses images taillées, et les blessés à mort gémiront dans tout son pays.
\VS{53}Quand Babylone serait montée jusqu'aux cieux, et qu'elle aurait fortifié le plus haut de sa forteresse, toutefois les dévastateurs y entreront par moi, dit Yahweh\FTNT{Am. 9:2~; Ab. 1:4.}…
\VS{54}Un grand cri s'entend de Babylone, et une grande ruine dans le pays des Chaldéens.
\VS{55}Parce que Yahweh s'en va détruire Babylone, et il abolira du milieu d'elle la voix magnifique~; et leurs flots mugiront comme de grandes eaux, l'éclat de leur bruit retentira.
\VS{56}Car le dévastateur est venu contre elle, contre Babylone~; ses hommes forts sont pris, leurs arcs sont brisés. Car Dieu des rétributions, Yahweh, ne manque jamais à rendre la pareille.
\VS{57}J'enivrerai ses princes et ses sages, ses gouverneurs et ses magistrats, et ses hommes forts~; ils dormiront d'un sommeil perpétuel et ils ne se réveilleront plus, dit le Roi dont le nom est Yahweh des armées.
\VS{58}Ainsi parle Yahweh des armées~: Il n'y aura aucune muraille de Babylone, quelque que soit leur largeur, qui ne soit entièrement rasée~; et ses portes, qui sont si hautes, seront brûlées par le feu~; ainsi les peuples auront travaillé en vain, et les nations pour le feu, et elles s'y seront lassées.
\VS{59}C'est ici l'ordre que Jérémie le prophète donna à Seraja, fils de Nérija, fils de Machséja, quand il alla avec Sédécias, roi de Juda, à Babylone, la quatrième année de son règne~; or Seraja était premier chambellan.
\VS{60}Car Jérémie écrivit dans un livre tout le mal qui devait venir sur Babylone~; à savoir, toutes ces paroles qui sont écrites contre Babylone.
\VS{61}Jérémie donc dit à Seraja~: Sitôt que tu seras venu à Babylone, et que tu l'auras vu, tu liras toutes ces paroles-là~;
\VS{62}et tu diras~: Yahweh~! Tu as parlé contre ce lieu-ci pour l'exterminer, en sorte qu'il n'y ait aucun habitant, depuis l'homme jusqu'à la bête, mais qu'il soit réduit en désolations perpétuelles.
\VS{63}Et sitôt que tu auras achevé de lire ce livre, tu le lieras à une pierre, et le jetteras dans l'Euphrate~;
\VS{64}et tu diras~: Babylone sera ainsi plongée, elle ne se relèvera pas du mal que je m'en vais faire venir sur elle, et ils en seront accablés. Jusqu'ici sont les paroles de Jérémie.
\Chap{52}
\TextTitle{Chute de Jérusalem et destruction du temple~; Juda déportée à Babylone\FTNTT{2 R. 25:1-26~; Jé. 39:1-10.}}
\VerseOne{}Sédécias avait vingt et un ans quand il devint roi, et il régna onze ans à Jérusalem. Sa mère se nommait Hamuthal, et elle était fille de Jérémie, de Libna\FTNT{2 R. 24 et 25.}.
\VS{2}Il fit ce qui est mal aux yeux de Yahweh, comme avait fait Jojakim.
\VS{3}Car il arriva, à cause de la colère de Yahweh contre Jérusalem et Juda, jusqu'à ce qu'il les rejette de devant sa face, que Sédécias se rebella contre le roi de Babylone.
\VS{4}Il arriva donc, la neuvième année de son règne, le dixième jour du dixième mois, que Nebucadnetsar, roi de Babylone, vint contre Jérusalem, lui et toute son armée~; et ils campèrent devant elle et construisirent des retranchements tout alentour.
\VS{5}Et la ville fut assiégée jusqu'à la onzième année du roi Sédécias.
\VS{6}Et le neuvième jour du quatrième mois, la famine était forte dans la ville, et il n'y avait pas de pain pour le peuple du pays\FTNT{La. 2:11-12.}.
\VS{7}Alors la brèche fut faite à la ville, et tous les gens de guerre s'enfuirent et sortirent de nuit hors de la ville, par le chemin de la porte entre les deux murailles, près du jardin du roi. Or les Chaldéens étaient tout autour de la ville~; et ils s'en allèrent par le chemin de la plaine.
\VS{8}Mais l'armée des Chaldéens poursuivit le roi, et quand ils atteignirent Sédécias dans les plaines de Jéricho, toute son armée se dispersa d'avec lui.
\VS{9}Ils prirent donc le roi et le firent monter vers le roi de Babylone à Ribla, dans le pays de Hamath, où il prononça contre lui une sentence.
\VS{10}Et le roi de Babylone fit égorger les fils de Sédécias sous ses yeux~; il fit aussi égorger tous les chefs de Juda à Ribla.
\VS{11}Puis il fit crever les yeux à Sédécias, et le fit lier de doubles chaînes d'airain~; le roi de Babylone l'emmena à Babylone, et le mit en prison jusqu'au jour de sa mort.
\VS{12}Et au dixième jour du cinquième mois, la dix-neuvième année du règne de Nebucadnetsar, roi de Babylone, Nebuzaradan, chef des gardes, qui se tenait devant le roi de Babylone, entra dans Jérusalem~;
\VS{13}et brûla la maison de Yahweh, la maison du roi, et toutes les maisons de Jérusalem~; et mit le feu dans toutes les maisons des grands.
\VS{14}Et toute l'armée des Chaldéens, qui était avec le chef des gardes, renversa toutes les murailles qui étaient autour de Jérusalem.
\VS{15}Et Nebuzaradan, chef des gardes, transporta à Babylone des plus pauvres du peuple, le reste du peuple qui était resté dans la ville, et ceux qui s'étaient rendus au roi de Babylone, avec le reste de la multitude.
\VS{16}Toutefois, Nebuzaradan, chef des gardes, laissa quelques-uns des plus pauvres du pays pour être vignerons et laboureurs.
\VS{17}Et les Chaldéens mirent en pièces les colonnes d'airain qui étaient dans la maison de Yahweh, avec les soubassements~; et la mer d'airain qui était dans la maison de Yahweh, et en emportèrent tout l'airain à Babylone.
\VS{18}Ils prirent aussi les chaudrons, les pelles, les serpes, les bassins, les tasses, et tous les ustensiles d'airain avec lesquels on faisait le service.
\VS{19}Le chef des gardes prit aussi les coupes, les encensoirs, les bassins, les chaudrons, les chandeliers, les tasses et les gobelets, ce qui était d'or et ce qui était d'argent.
\VS{20}Quant aux deux colonnes, à la mer et aux douze bœufs d'airain qui servaient de soubassements, que le roi Salomon avait faits pour la maison de Yahweh, on ne pesa pas l'airain de tous ces ustensiles–là.
\VS{21}Or quant aux colonnes, chaque colonne avait dix-huit coudées de haut, et un cordon de douze coudées l'entourait~; la première était épaisse de quatre doigts et creuse~;
\VS{22}et il y avait par-dessus un chapiteau d'airain~; et la hauteur d'un des chapiteaux était de cinq coudées, il y avait aussi un réseau et des grenades tout autour du chapiteau, le tout d'airain. Et la seconde colonne était de même ainsi que les grenades\FTNT{1 R. 7:15-20.}~;
\VS{23}il y avait aussi quatre-vingt-seize grenades de chaque côté, et les grenades qui étaient sur le réseau à l'entour étaient au nombre de cent.
\VS{24}Davantage le chef des gardes prit Seraja, qui était le premier prêtre, et Sophonie, qui était le second prêtre, et les trois gardiens du seuil.
\VS{25}Il emmena aussi de la ville un eunuque qui avait sous son commandement des hommes de guerre, et sept hommes de ceux qui voyaient la face du roi, et qui furent trouvés dans la ville, et le secrétaire du chef de l'armée qui enrôlait le peuple du pays~; et soixante hommes d'entre le peuple du pays, qui se furent trouvés dans la ville.
\VS{26}Nebuzaradan donc, chef des gardes, les prit et les emmena vers le roi de Babylone à Ribla.
\VS{27}Et le roi de Babylone les frappa et les fit mourir à Ribla, dans le pays de Hamath. Ainsi Juda fut transporté hors de son pays.
\VS{28}Et c'est ici le peuple que Nebucadnetsar emmena en captivité~; la septième année, trois mille vingt-trois juifs.
\VS{29}La dix-huitième année de Nebucadnetsar, il emmena de Jérusalem huit cent trente-deux personnes.
\VS{30}La vingt-troisième année de Nebucadnetsar, Nebuzaradan, chef des gardes, transporta en exil sept cent quarante-cinq personnes des Juifs~; toutes les personnes donc furent quatre mille six cents.
\VS{31}Or il arriva la trente-septième année de la captivité de Jojakin, roi de Juda, le vingt-cinquième jour du douzième mois, qu'Evil-Merodac, roi de Babylone, dans la première année de son règne, releva la tête de Jojakin, roi de Juda, et le fit sortir de prison.
\VS{32}Et lui parla avec bonté, et mit son trône au-dessus du trône des autres rois qui étaient avec lui à Babylone.
\VS{33}Et après qu'il eut changé ses vêtements de prison, il mangea du pain constamment en sa présence, tous les jours de sa vie.
\VS{34}Et quant à son entretien régulier, un entretien continuel lui fut donné de la part du roi de Babylone, jour par jour, jusqu'au jour de sa mort, tous les jours de sa vie.
\PPE{}
\end{multicols}
