\ShortTitle{Jean}\BookTitle{Jean}\BFont
\noindent\hrulefill
{\footnotesize
\textit{
\bigskip
{\centering{}
\\Auteur : Jean
\\(Gr. : Ioannes / Origine héb. :  « Yohanan » )
\\Signification : Yahweh a fait grâce
\\Thème : Jésus Dieu
\\Date de rédaction : Env. 85-90 ap. J.-C.\\}
}
%\bigskip
\textit{
\\Auteur d'un des quatre évangiles, des trois épîtres éponymes et de l'Apocalypse, Jean, fils de Zébédée, fut l'un des douze apôtres. Témoin oculaire du ministère terrestre de Jésus-Christ, il atteste par l'essence de ses écrits le caractère divin de ce dernier.
%\bigskip
\\Fidèle au livre d'Exode où Yahweh se révéla comme étant « Je suis », Jean reprend les propos de Jésus et le présente comme la parole incarnée, le pain de vie, la lumière du monde, la porte des brebis, le bon berger, la résurrection, la vie… Proche du maître, Jean fut à même de relater les événements marquants de sa vie comme la gloire de la transfiguration, l'angoisse de la passion exprimée à Gethsémané, ou encore les déclarations solennelles précédées de l'expression « En vérité, en vérité »… Il met également en évidence la controverse suscitée par le Christ et l'opposition dont il fit l'objet de la part de certains pharisiens qui souhaitaient sa mort.
%\bigskip
\\L'évangile de Jean exprime la nécessité de la nouvelle naissance et dévoile les attributs du Fils de Dieu, le Messie tant
attendu.\bigskip
}
}
\par\nobreak\noindent\hrulefill
\begin{multicols}{2}
\Chap{1}
\TextTitle{La divinité de Jésus-Christ\FTNTT{Jn. 10:30 ; Hé. 1:5-13.}}
\VerseOne{}Au commencement était la Parole, et la Parole était avec Dieu, et la Parole était Dieu.
\VS{2}Elle était au commencement avec Dieu.
\TextTitle{L'oeuvre de Jésus avant son incarnation}
\VS{3}Toutes choses ont été faites par elle, et rien de ce qui a été fait n'a été fait sans elle.
\VS{4}En elle était la vie, et la vie était la lumière des hommes\FTNT{Jésus-Christ notre lumière : Es. 60:19-20.}.
\VS{5}Et la lumière luit dans les ténèbres, mais les ténèbres ne l'ont point reçue.
\TextTitle{Ministère de Jean-Baptiste}
\VS{6}Il y eut un homme appelé Jean, qui fut envoyé de Dieu.
\VS{7}Il vint pour rendre témoignage, pour rendre, dis-je, témoignage à la lumière, afin que tous croient par lui.
\VS{8}Il n'était pas la lumière, mais il était envoyé pour rendre témoignage à la lumière.
\TextTitle{Jésus-Christ, la véritable lumière\FTNTT{Jn. 3:17-21 ; 8:12 ; 9:5 ; 12:46.}}
\VS{9}Cette lumière était la véritable lumière, qui, en venant dans le monde éclaire tout homme.
\VS{10}Elle était dans le monde, et le monde a été fait par elle, mais le monde ne l'a point connue.
\VS{11}Il\FTNT{Il est question ici du logos, d'où l'emploi du prénom personnel « il ».} est venu chez les siens, et les siens ne l'ont point reçu.
\VS{12}Mais à tous ceux qui l'ont reçue, à ceux qui croient en son Nom, il leur a donné le pouvoir de devenir enfants de Dieu.
\VS{13}Lesquels sont nés, non du sang, ni de la volonté de la chair, ni de la volonté de l'homme, mais ils sont nés de Dieu.
\TextTitle{La Parole faite chair\FTNTT{Mt. 1:18-23 ; Lu. 1:30-35 ; 2:11 ; Jn. 14:9 ; 1 Tim. 3:16.}}
\VS{14}Et la Parole a été faite chair, elle a habité parmi nous, pleine de grâce et de vérité ; et nous avons contemplé sa gloire, une gloire comme la gloire du Fils unique du Père.
\TextTitle{Témoignage de Jean-Baptiste\FTNTT{Mt. 3:1-12 ; Mc. 1:1-11 ; Lu. 3:1-22.}}
\VS{15}Jean a donc rendu témoignage de lui, et s'est écrié, disant : C'est celui dont j'ai dit : Celui qui vient après moi m'a précédé, car il était avant moi.
\VS{16}Et nous avons tous reçu de sa plénitude, et grâce pour grâce ;
\VS{17}car la loi\FTNT{La loi a été promulguée par Moïse.} a été donnée par Moïse, la grâce et la vérité sont venues par Jésus-Christ.
\VS{18}Personne n'a jamais vu Dieu ; le Fils unique qui est dans le sein du Père, est celui qui nous l'a révélé.
\VS{19}Et c'est ici le témoignage de Jean, lorsque les Juifs envoyèrent de Jérusalem des sacrificateurs et des Lévites pour l'interroger, et lui dire : Toi qui es-tu ?
\VS{20}Il confessa, et ne le nia point, il déclara, en disant : Ce n'est pas moi qui suis le Christ.
\VS{21}Et ils lui demandèrent : Quoi donc ? Es-tu Elie ? Et il dit : Je ne le suis point\FTNT{En Mt. 11:14 Jésus confirme pourtant que Jean-Baptiste est bien l'Elie qui devait venir. Comment expliquer qu'il nia l'être lorsqu'il fut interrogé par les pharisiens ? La seule explication plausible c'est qu'il l'ignorait. Toutefois, il avait conscience qu'il était « la voix » prophétisée par Esaïe. Remarquez que lorsqu'il fut emprisonné, il avait envoyé quelques-uns de ses disciples pour demander à Jésus s'il était bien le Messie (Mt. 11:13 ; Lu. 7:19-20) alors qu'il fut le premier à rendre témoignage du Seigneur. Ces éléments ne sont pas contradictoires, ils ne font que révéler les failles liées à la nature humaine de Jean.}. Es-tu le Prophète ? Et il répondit : Non.
\VS{22}Ils lui dirent donc : Qui es-tu ? Afin que nous donnions une réponse à ceux qui nous ont envoyés. Que dis-tu de toi-même ?
\VS{23}Il dit : Je suis la voix de celui qui crie dans le désert : Aplanissez le chemin du Seigneur, comme a dit Esaïe le prophète\FTNT{Es. 40:3.}.
\VS{24}Or ceux qui avaient été envoyés vers lui étaient des pharisiens.
\VS{25}Ils l'interrogèrent encore, et lui dirent : Pourquoi donc baptises-tu si tu n'es point le Christ, ni Elie, ni le Prophète ?
\VS{26}Jean leur répondit et leur dit : Pour moi, je baptise d'eau, mais il y a quelqu'un au milieu de vous que vous ne connaissez point.
\VS{27}C'est celui qui vient après moi, il m'a précédé, et je ne suis pas digne de délier la courroie de ses souliers.
\VS{28}Ces choses se passèrent à Béthanie, au-delà du Jourdain, où Jean baptisait.
\VS{29}Le lendemain, Jean vit Jésus venir à lui, et il dit : Voici l'Agneau de Dieu, qui ôte le péché du monde.
\VS{30}C'est celui dont j'ai dit : Après moi vient un homme qui m'a précédé, car il était avant moi.
\VS{31}Et pour moi, je ne le connaissais pas ; mais c'est afin qu'il soit manifesté à Israël que je suis venu baptiser d'eau.
\VS{32}Jean rendit aussi témoignage, en disant : J'ai vu l'Esprit descendre du ciel comme une colombe et s'arrêter sur lui.
\VS{33}Et pour moi, je ne le connaissais point, mais celui qui m'a envoyé baptiser d'eau, m'avait dit : Celui sur qui tu verras l'Esprit descendre et s'arrêter, c'est celui qui baptise du Saint-Esprit.
\VS{34}Et je l'ai vu, et j'ai rendu témoignage que c'est lui qui est le Fils de Dieu.
\TextTitle{Premiers disciples de Jésus-Christ\FTNTT{Mt. 4:18-22 ; Mc. 1:16-20 ; Lu. 5:1-11.}}
\VS{35}Le lendemain, Jean était encore là, avec deux de ses disciples ;
\VS{36}et, regardant Jésus qui marchait, il dit : Voici l'Agneau de Dieu.
\VS{37}Et les deux disciples l'entendirent tenant ce discours, et ils suivirent Jésus.
\VS{38}Et Jésus se retournant, et voyant qu'ils le suivaient, il leur dit : Que cherchez-vous ? Ils lui répondirent : Rabbi, c'est-à-dire Maître, où demeures-tu ?
\VS{39}Il leur dit : Venez, et voyez. Ils y allèrent, et ils virent où il demeurait ; et ils demeurèrent avec lui ce jour-là ; car il était environ dix heures.
\VS{40}Or André, frère de Simon Pierre, était l'un des deux qui avaient entendu les paroles de Jean et qui l'avaient suivi.
\VS{41}Ce fut lui qui rencontra le premier Simon son frère, et il lui dit : Nous avons trouvé le Messie, c'est-à-dire le Christ.
\VS{42}Et il le conduisit vers Jésus, et Jésus l'ayant regardé, dit : Tu es Simon, fils de Jonas, tu seras appelé Céphas, c'est-à-dire, Pierre.
\VS{43}Le lendemain, Jésus voulut aller en Galilée, et il trouva Philippe, et il lui dit : Suis-moi.
\VS{44}Philippe était de Bethsaïda, la ville d'André et de Pierre.
\VS{45}Philippe rencontra Nathanaël, et lui dit : Nous avons trouvé celui de qui Moïse a écrit dans la loi et dont les prophètes ont parlé, Jésus qui est de Nazareth, fils de Joseph.
\VS{46}Et Nathanaël lui dit : Peut-il venir quelque chose de bon de Nazareth ? Philippe lui dit : Viens, et vois.
\VS{47}Jésus aperçut Nathanaël venir vers lui, et il dit de lui : Voici vraiment un Israëlite dans lequel il n'y a point de fraude.
\VS{48}Nathanaël lui dit : D'où me connais-tu ? Jésus répondit, et lui dit : Avant que Philippe t'appelle, quand tu étais sous le figuier, je t'ai vu.
\VS{49}Nathanaël répondit et lui dit : Maître, tu es le Fils de Dieu, tu es le Roi d'Israël.
\VS{50}Jésus lui répondit et dit : Parce que je t'ai dit que je t'ai vu sous le figuier, tu crois ; tu verras des choses plus grandes encore.
\VS{51}Il lui dit aussi : En vérité, en vérité je vous dis : Désormais vous verrez le ciel ouvert, et les anges de Dieu monter et descendre sur le Fils de l'homme.
\Chap{2}
\TextTitle{Miracle à Cana}
\VerseOne{}Or trois jours après, on faisait des noces à Cana en Galilée, et la mère de Jésus était là.
\VS{2}Et Jésus fut aussi convié aux noces avec ses disciples.
\VS{3}Et le vin ayant manqué, la mère de Jésus lui dit : Ils n'ont plus de vin.
\VS{4}Mais Jésus lui répondit : Qu'y a-t-il entre moi et toi, femme ? Mon heure n'est point encore venue.
\VS{5}Sa mère dit aux serviteurs : Faites tout ce qu'il vous dira.
\VS{6}Or il y avait là six vases de pierre, destinés aux purifications des Juifs, dont chacun contenant deux ou trois mesures.
\VS{7}Et Jésus leur dit : Remplissez d'eau ces vases. Et ils les remplirent jusqu'au haut.
\VS{8}Puis il leur dit : Puisez-en maintenant, et apportez-en au maître d'hôtel. Et ils lui en apportèrent.
\VS{9}Quand le maître d'hôtel eut goûté l'eau qui avait été changée en vin, (or il ne savait pas d'où cela venait, mais les serviteurs qui avaient puisé l'eau le savaient bien), il s'adressa à l'époux,
\VS{10}et lui dit : Tout homme sert d'abord le bon vin, et ensuite le moins bon, après qu'on s'est enivré ; mais toi, tu as gardé le bon vin jusqu'à maintenant.
\VS{11}Jésus fit ce premier miracle à Cana en Galilée. Et il manifesta sa gloire, et ses disciples crurent en lui.
\VS{12}Après cela, il descendit à Capernaüm avec sa mère et ses frères, et ses disciples mais ils y demeurèrent peu de jours.
\TextTitle{La première Pâque\FTNTT{Jn. 6:4 ; 11:55.}}
\VS{13}Car la Pâque des Juifs était proche, c'est pourquoi Jésus monta à Jérusalem.
\VS{14}Et il trouva dans le temple des vendeurs de bœufs, de brebis, et de pigeons, et les changeurs qui y étaient assis.
\VS{15}Et ayant fait un fouet avec des petites cordes, il les chassa tous du temple, avec les brebis et les bœufs ; et il dispersa la monnaie des changeurs, et renversa les tables.
\VS{16}Et il dit aux vendeurs de pigeons : Otez ces choses d'ici, et ne faites pas de la maison de mon Père une maison de marché.
\VS{17}Alors ses disciples se souvinrent qu'il était écrit : Le zèle de ta maison me dévore\FTNT{Ps. 69:10.}.
\VS{18}Mais les Juifs, prenant la parole, lui dirent : Quel signe nous montres-tu, pour agir de la sorte ?
\VS{19}Jésus répondit et leur dit : Détruisez ce temple, et en trois jours je le relèverai.
\VS{20}Et les Juifs dirent : Il a fallu quarante-six ans pour bâtir ce temple, et toi, tu le relèveras en trois jours !
\VS{21}Mais il parlait du temple de son corps.
\VS{22}C'est pourquoi, lorsqu'il fut ressuscité des morts, ses disciples se souvinrent qu'il leur avait dit cela, et ils crurent à l'Ecriture et à la parole que Jésus avait dite.
\VS{23}Et comme il était à Jérusalem le jour de la fête de Pâque, plusieurs crurent en son Nom, voyant les miracles qu'il faisait.
\VS{24}Mais Jésus ne se fiait point à eux, parce qu'il les connaissait tous,
\VS{25}et parce qu'il n'avait pas besoin qu'on lui rende témoignage d'aucun homme ; car il savait lui-même ce qui était dans l'homme.
\Chap{3}
\TextTitle{La naissance d'en haut}
\VerseOne{}Mais il y eut un homme d'entre les pharisiens, nommé Nicodème, qui était un des chefs des Juifs,
\VS{2}lequel vint de nuit à Jésus, et lui dit : Maître, nous savons que tu es un docteur venu de Dieu, car personne ne peut faire les miracles que tu fais, si Dieu n'est avec lui.
\VS{3}Jésus répondit, et lui dit : En vérité, en vérité je te dis : Si quelqu'un ne naît d'en haut\FTNT{Naître d'en haut : Dans la plupart des Bibles modernes, on trouve l'expression naître « de nouveau », or cette traduction n'est pas correcte puisque le texte grec utilise l'expression naître « d'en haut ». L'adverbe « d'en haut » vient du mot grec « ahothen » qui signifie : depuis le haut, depuis un endroit plus élevé, ce qui vient des cieux ou de Dieu, depuis le début, l'origine. Ce mot se retrouve dans Mt. 27:51 ; Mc. 15:38 ; Lu. 1:3 ; Jn. 3:31 ; Jn. 19:11 ; Jn. 19:23 ; Ja. 1:17 ; Ja. 3:15 ; Ja. 3:17. « Anothen » vient de « ano » : choses d'en haut. En Ga. 4:26 « ano » peut se référer au lieu ou au temps. Le lieu : La Jérusalem qui est au-dessus, dans les cieux. Le temps : La Jérusalem céleste qui a précédé la terrestre. Le mot « ano » a été traduit par « en haut » dans Jn. 8:23 ; Jn. 11:41 ; Ac. 2:19 ; Ga. 4:26 ; Col. 3:1-2 ; et par « céleste » dans Ph. 3:14. Jésus nous enseigne donc que la nouvelle naissance est en réalité la naissance d'en haut, une naissance qui a eu lieu dans la Nouvelle Jérusalem.}, il ne peut voir le Royaume de Dieu.
\VS{4}Nicodème lui dit : Comment un homme peut-il naître quand il est vieux ? Peut-il rentrer dans le sein de sa mère et naître une seconde fois ?
\VS{5}Jésus répondit : En vérité, en vérité, je te dis : Si quelqu'un ne naît d'eau et d'Esprit, il ne peut entrer dans le Royaume de Dieu.
\VS{6}Ce qui est né de la chair est chair, et ce qui est né de l'Esprit est esprit.
\VS{7}Ne t'étonne pas de ce que je t'ai dit : Il faut que vous naissiez d'en haut.
\VS{8}Le vent souffle où il veut, et tu en entends le bruit ; mais tu ne sais pas d'où il vient ni où il va : Il en est ainsi de tout homme qui est né de l'Esprit.
\VS{9}Nicodème répondit, et lui dit : Comment peuvent se faire ces choses ?
\VS{10}Jésus répondit, et lui dit : Tu es docteur d'Israël, et tu ne connais point ces choses !
\VS{11}En vérité, en vérité je te dis, nous disons ce que nous savons, et nous rendons témoignage de ce que nous avons vu ; et vous ne recevez pas notre témoignage.
\VS{12}Si je vous ai parlé des choses terrestres, et que vous ne les croyez point, comment croirez-vous si je vous parle des choses célestes ?
\VS{13}Car personne n'est monté au ciel, si ce n'est celui qui est descendu du ciel, à savoir le Fils de l'homme qui est dans le ciel.
\VS{14}Or comme Moïse éleva le serpent\FTNT{Le serpent d'airain : No. 21:9}dans le désert, ainsi il faut de même que le Fils de l'homme soit élevé,
\VS{15}afin que quiconque croit en lui ne périsse point, mais qu'il ait la vie éternelle.
\VS{16}Car Dieu a tant aimé le monde qu'il a donné son Fils unique, afin que quiconque croit en lui ne périsse point, mais qu'il ait la vie éternelle.
\VS{17}Car Dieu n'a point envoyé son Fils dans le monde pour condamner le monde, mais afin que le monde soit sauvé par lui.
\VS{18}Celui qui croit en lui ne sera point condamné ; mais celui qui ne croit point est déjà condamné ; parce qu'il n'a point cru au Nom du Fils unique de Dieu.
\VS{19}Or c'est ici le sujet de la condamnation: Que la lumière est venue dans le monde et que les hommes ont mieux aimé les ténèbres que la lumière, parce que leurs œuvres étaient mauvaises.
\VS{20}Car quiconque fait le mal, hait la lumière, et ne vient point à la lumière, de peur que ses œuvres ne soient condamnées.
\VS{21}Mais celui qui agit selon la vérité, vient à la lumière, afin que ses œuvres soient manifestées, parce qu'elles sont faites selon Dieu.
\TextTitle{Nouveau témoignage de Jean-Baptiste}
\VS{22}Après ces choses, Jésus vint avec ses disciples au pays de Judée ; et là il demeurait avec eux et il baptisait.
\VS{23}Or Jean baptisait aussi à Enon, près de Salim, parce qu'il y avait là beaucoup d'eau ; et on y venait pour être baptisé.
\VS{24}Car Jean n'avait pas encore été mis en prison.
\VS{25}Or il y eut une dispute entre les disciples de Jean et les Juifs touchant la purification.
\VS{26}Et ils vinrent près de Jean, et lui dirent : Maître, celui qui était avec toi au-delà du Jourdain, et à qui tu as rendu témoignage, voilà, il baptise, et tous vont à lui.
\VS{27}Jean répondit, et dit : L'homme ne peut recevoir aucune chose si elle ne lui est donnée du ciel.
\VS{28}Vous-mêmes m'êtes témoins que j'ai dit : Ce n'est pas moi qui suis le Christ, mais j'ai été envoyé devant lui.
\VS{29}Celui qui possède l'Epouse est l'Epoux ; mais l'ami de l'Epoux qui assiste, et qui l'entend, est tout réjoui par la voix de l'Epoux ; c'est pourquoi cette joie que j'ai, est accomplie.
\VS{30}Il faut qu'il croisse, et que je diminue.
\TextTitle{Conclusion de Jean}
\VS{31}Celui qui vient d'en haut est au-dessus de tous ; celui qui est venu de la terre est de la terre, et il parle comme venant de la terre. Celui qui est venu du ciel est au-dessus de tous,
\VS{32}et ce qu'il a vu et entendu, il le témoigne ; mais personne ne reçoit son témoignage.
\VS{33}Celui qui a reçu son témoignage a certifié que Dieu est véritable ;
\VS{34}car celui que Dieu a envoyé annonce les paroles de Dieu, car Dieu ne lui donne pas l'Esprit par mesure.
\VS{35}Le Père aime le Fils, et il a remis toutes choses entre ses mains.
\VS{36}Celui qui croit au Fils a la vie éternelle ; mais celui qui désobéit au Fils ne verra point la vie, mais la colère de Dieu demeure sur lui.
\Chap{4}
\TextTitle{Jésus se rend en Galilée}
\VerseOne{}Le Seigneur sut que les pharisiens avaient appris qu'il faisait et baptisait plus de disciples que Jean.
\VS{2}Toutefois Jésus ne baptisait point lui-même, mais c'étaient ses disciples.
\VS{3}Il quitta la Judée, et retourna encore en Galilée.
\TextTitle{Jésus et la femme Samaritaine}
\VS{4}Or il fallait qu'il traverse la Samarie,
\VS{5}il arriva dans une ville de Samarie, nommée Sychar, près du champ que Jacob avait donné à Joseph son fils\FTNT{Ge. 48:22.}.
\VS{6}Or il y avait là la fontaine de Jacob ; et Jésus, fatigué du voyage, se tenait là assis sur la fontaine. C'était environ la sixième heure\FTNT{Sixième heure ou midi.}.
\VS{7}Et une femme Samaritaine vint puiser de l'eau, Jésus lui dit : Donne-moi à boire.
\VS{8}Car ses disciples étaient allés à la ville pour acheter des vivres.
\VS{9}Mais cette femme Samaritaine lui dit : Comment toi qui es Juif, me demandes-tu à boire, à moi qui suis une femme Samaritaine ? Car les Juifs n'ont pas de relations avec les Samaritains.
\VS{10}Jésus lui répondit, et lui dit : Si tu connaissais le don de Dieu et qui est celui qui te dit : Donne-moi à boire ! Tu lui aurais toi-même demandé à boire, et il t'aurait donné de l'eau vive.
\VS{11}La femme lui dit : Seigneur, tu n'as rien pour puiser, et le puits est profond ; d'où aurais-tu donc cette eau vive ?
\VS{12}Es-tu plus grand que Jacob notre père, qui nous a donné ce puits, et qui en a bu lui-même, ainsi que ses enfants et son bétail ?
\VS{13}Jésus répondit, et lui dit : Quiconque boit de cette eau-ci aura encore soif ;
\VS{14}mais celui qui boira de l'eau que je lui donnerai n'aura jamais soif, mais l'eau que je lui donnerai deviendra en lui une fontaine d'eau qui jaillira jusque dans la vie éternelle.
\VS{15}La femme lui dit : Seigneur, donne-moi de cette eau, afin que je n'aie plus soif, et que je ne vienne plus ici puiser de l'eau.
\VS{16}Jésus lui dit : Va, et appelle ton mari, et viens ici.
\VS{17}La femme répondit et lui dit : Je n'ai point de mari. Jésus lui dit : Tu as bien dit : Je n'ai point de mari.
\VS{18}Car tu as eu cinq maris, et celui que tu as maintenant n'est point ton mari. En cela tu as dit la vérité.
\VS{19}La femme lui dit : Seigneur, je vois que tu es un prophète.
\VS{20}Nos pères ont adoré sur cette montagne\FTNT{La Samaritaine faisait allusion au Mont Garizim, également appelé Montagne de Sichem. Les Samaritains y avaient construit leur propre temple, mais celui-ci fut détruit vers 129 av. J.-C. par Jean Hyrcan, fils de Simon Maccabée.  
}, et vous, vous dites que le lieu où il faut adorer est à Jérusalem.
\VS{21}Jésus lui dit : Femme, crois-moi, l'heure vient où ce ne sera ni sur cette montagne ni à Jérusalem que vous adorerez le Père.
\VS{22}Vous adorez ce que vous ne connaissez pas ; nous, nous adorons ce que nous connaissons ; car le salut vient des Juifs.
\VS{23}Mais l'heure vient, et elle est déjà venue, où les vrais adorateurs adoreront le Père en esprit et en vérité ; car ce sont là les adorateurs que le Père demande.
\VS{24}Dieu est Esprit, et il faut que ceux qui l'adorent, l'adorent en esprit et en vérité.
\VS{25}La femme lui répondit : Je sais que le Messie, c'est-à-dire le Christ, doit venir ; quand donc il sera venu, il nous annoncera toutes choses.
\VS{26}Jésus lui dit : C'est moi-même qui parle avec toi.
\VS{27}Sur cela ses disciples vinrent, et ils s'étonnèrent de ce qu'il parlait avec une femme; toutefois aucun ne dit : Que demandes-tu ? Ou : Pourquoi parles-tu avec elle ?
\VS{28}La femme, ayant laissé sa cruche, s'en alla dans la ville, et elle dit aux habitants :
\VS{29}Venez voir un homme qui m'a dit tout ce que j'ai fait ; ne serait-ce point le Christ ?
\VS{30}Ils sortirent donc de la ville, et vinrent vers lui.
\VS{31}Cependant, les disciples le priaient, disant : Maître, mange.
\VS{32}Mais il leur dit : J'ai à manger une nourriture que vous ne connaissez point.
\VS{33}Sur quoi les disciples se demandaient entre eux : Quelqu'un lui aurait-il apporté à manger ?
\VS{34}Jésus leur dit : Ma nourriture est de faire la volonté de celui qui m'a envoyé, et d'accomplir son œuvre.
\VS{35}Ne dites-vous pas qu'il y a encore quatre mois jusqu'à la moisson ? Voici, je vous dis, levez vos yeux, et regardez les champs qui déjà blanchissent pour la moisson.
\VS{36}Or celui qui moissonne reçoit le salaire et amasse le fruit pour la vie éternelle, afin que celui qui sème et celui qui moissonne se réjouissent ensemble.
\VS{37}Or ce que l'on dit d'ordinaire, que l'un sème, l'autre moissone, est vrai en ceci,
\VS{38}que je vous ai envoyés moissonner ce en quoi vous n'avez pas travaillé ; d'autres ont travaillé, et vous êtes entrés dans leur travail.
\TextTitle{Jésus et les Samaritains}
\VS{39}Or plusieurs des Samaritains de cette ville-là crurent en lui pour la parole de la femme qui avait rendu ce témoignage : Il m'a dit tout ce que j'ai fait.
\VS{40}Quand donc les Samaritains vinrent le trouver, ils le prièrent de demeurer avec eux. Et il demeura là deux jours.
\VS{41}Et beaucoup plus de gens crurent à cause de sa parole ;
\VS{42}et ils disaient à la femme : Ce n'est plus à cause de ta parole que nous croyons ; car nous l'avons entendu nous-mêmes, et nous savons qu'il est véritablement le Christ, le Sauveur du monde.
\VS{43}Or deux jours après, il partit de là, et s'en alla en Galilée.
\VS{44}Car Jésus avait rendu témoignage qu'un prophète n'est pas honoré dans son pays.
\VS{45}Lorsqu'il arriva en Galilée, les Galiléens le reçurent, ayant vu toutes les choses qu'il avait faites à Jérusalem le jour de la fête, car eux aussi étaient allés à la fête.
\TextTitle{Jésus guérit le fils d'un officier}
\VS{46}Jésus donc vint encore à Cana de Galilée, où il avait changé l'eau en vin. Or il y avait à Capernaüm un officier du roi, dont le fils était malade ;
\VS{47}qui, ayant entendu que Jésus était venu de Judée en Galilée, s'en alla vers lui, et le pria de descendre pour guérir son fils qui était près de mourir.
\VS{48}Mais Jésus lui dit : Si vous ne voyez pas des prodiges et des miracles, vous ne croyez point.
\VS{49}L'officier du roi lui dit : Seigneur, descends avant que mon fils meure.
\VS{50}Jésus lui dit : Va, ton fils vit. Cet homme crut à la parole que Jésus lui avait dite, et il s'en alla.
\VS{51}Et comme il descendait déjà, ses serviteurs vinrent au-devant de lui et lui apportèrent des nouvelles, disant : Ton fils vit.
\VS{52}Et il leur demanda à quelle heure il s'était trouvé mieux ; et ils lui dirent : Hier, à la septième heure, la fièvre l'a quitté.
\VS{53}Le père reconnut que c'était à cette même heure-là que Jésus lui avait dit : Ton fils vit. Et il crut, avec toute sa maison.
\VS{54}Jésus fit encore ce second miracle quand il fut venu de Judée en Galilée.
\Chap{5}
\TextTitle{Guérison d'un paralytique à la piscine de Béthesda}
\VerseOne{}Après ces choses, il y eut une fête des Juifs, et Jésus monta à Jérusalem.
\VS{2}Or à Jérusalem, près de la porte des brebis, il y avait un lavoir appelé en hébreu Béthesda, ayant cinq portiques.
\VS{3}Dans lesquels étaient couchés un grand nombre de malades, des aveugles, des boiteux, des paralytiques, attendant le mouvement de l'eau.
\VS{4}Car un ange descendait en certains temps dans le lavoir, et agitait l'eau ; et alors le premier qui y descendait après que l'eau avait été agitée, était guéri, quelle que fût sa maladie.
\VS{5}Or il y avait là un homme malade depuis trente-huit ans.
\VS{6}Et Jésus, le voyant couché par terre, et sachant qu'il était déjà malade depuis longtemps, lui dit : Veux-tu être guéri ?
\VS{7}Le malade lui répondit : Seigneur, je n'ai personne pour me jeter dans le lavoir quand l'eau est agitée, et pendant que j'y vais, un autre y descend avant moi.
\VS{8}Jésus lui dit : Lève-toi, prends ton lit, et marche.
\VS{9}Et aussitôt l'homme fut guéri ; il prit son lit, et marcha. Or c'était un jour de sabbat.
\VS{10}Les Juifs dirent donc à celui qui avait été guéri : C'est un jour de sabbat ; il ne t'est pas permis de prendre ton lit.
\VS{11}Il leur répondit : Celui qui m'a guéri m'a dit : Prends ton lit et marche.
\VS{12}Alors ils lui demandèrent : Qui est celui qui t'a dit : Prends ton lit et marche ?
\VS{13}Mais celui qui avait été guéri ne savait pas qui c'était ; car Jésus s'était éclipsé du milieu de la foule qui était en ce lieu-là.
\VS{14}Depuis, Jésus le trouva dans le temple, et lui dit : Voici, tu as été guéri ; ne pèche plus désormais, de peur qu'il ne t'arrive quelque chose de pire.
\VS{15}Cet homme s'en alla, et rapporta aux Juifs que c'était Jésus qui l'avait guéri.
\VS{16}C'est pourquoi les Juifs poursuivaient Jésus et cherchaient à le faire mourir, parce qu'il avait fait ces choses le jour du sabbat.
\TextTitle{Egalité de Jésus avec le Père}
\VS{17}Mais Jésus leur répondit : Mon Père travaille jusqu'à présent ; moi aussi, je travaille.
\VS{18}Et à cause de cela, les Juifs cherchaient encore plus à le faire mourir, parce que non seulement il avait violé le sabbat, mais aussi parce qu'il disait que Dieu était son propre Père, se faisant égal à Dieu.
\VS{19}Mais Jésus répondit, et leur dit : En vérité, en vérité je vous dis, le Fils ne peut rien faire de lui-même, il ne fait que ce qu'il voit faire au Père ; et tout ce que le Père fait, le Fils le fait pareillement.
\VS{20}Car le Père aime le Fils, et lui montre toutes les choses qu'il fait ; et il lui montrera de plus grandes œuvres que celles-ci, afin que vous soyez dans l'admiration.
\VS{21}Car, comme le Père ressuscite les morts et donne la vie, de même aussi le Fils donne la vie à ceux qu'il veut.
\VS{22}Car le Père ne juge personne, mais il a donné tout jugement au Fils,
\VS{23}afin que tous honorent le Fils comme ils honorent le Père ; celui qui n'honore point le Fils, n'honore point le Père qui l'a envoyé.
\VS{24}En vérité, en vérité je vous dis : Que celui qui entend ma parole, et croit à celui qui m'a envoyé, a la vie éternelle et ne vient pas en jugement, mais il est passé de la mort à la vie.
\TextTitle{Les deux résurrections}
\VS{25}En vérité, en vérité, je vous dis : Que l'heure vient, et elle est maintenant, où les morts entendront la voix du Fils de Dieu, et ceux qui l'auront entendue, vivront.
\VS{26}Car comme le Père a la vie en lui-même, ainsi il a donné au Fils d'avoir la vie en lui-même.
\VS{27}Et il lui a donné le pouvoir de juger parce qu'il est le Fils de l'homme.
\VS{28}Ne soyez point étonnés de cela ; car l'heure vient où tous ceux qui sont dans les sépulcres entendront sa voix, et en sortiront.
\VS{29} Ceux qui auront fait le bien ressusciteront pour la vie, mais ceux qui auront fait le mal ressusciteront pour la condamnation.
\TextTitle{Témoignages en accord avec celui de Jésus}
\VS{30}Je ne puis rien faire de moi-même : Je juge conformément à ce que j'entends, et mon jugement est juste, car je ne cherche point ma volonté, mais la volonté du Père qui m'a envoyé.
\VS{31}Si je rends témoignage de moi-même, mon témoignage n'est pas digne de foi.
\VS{32}C'est un autre qui rend témoignage de moi, et je sais que le témoignage qu'il rend de moi est digne de foi.
\TextTitle{Le témoignage de Jean-Baptiste}
\VS{33}Vous avez envoyé vers Jean, et il a rendu témoignage à la vérité.
\VS{34}Or je ne cherche point le témoignage des hommes ; mais je dis ces choses afin que vous soyez sauvés.
\VS{35}Il était une lampe ardente et brillante ; et vous avez voulu vous réjouir pour un peu de temps à sa lumière.
\TextTitle{Le témoignage des oeuvres de Jésus}
\VS{36}Mais moi, j'ai un témoignage plus grand que celui de Jean ; car les œuvres que mon Père m'a donné d'accomplir, ces œuvres mêmes que je fais, témoignent de moi que c'est mon Père qui m'a envoyé.
\TextTitle{Le témoignage du Père\FTNTT{Mt. 3:17.}}
\VS{37}Et le Père qui m'a envoyé, a lui-même rendu témoignage de moi. Vous n'avez jamais entendu sa voix, vous n'avez jamais vu sa face.
\VS{38}Et sa parole ne demeure point en vous, puisque vous ne croyez pas à celui qu'il a envoyé.
\TextTitle{Le témoignage de l'Ecriture\FTNTT{Lu. 24:27,44.}}
\VS{39}Vous sondez les Ecritures, car vous pensez avoir en elles la vie éternelle, et ce sont elles qui rendent témoignage de moi.
\VS{40}Mais vous ne voulez pas venir à moi pour avoir la vie.
\VS{41}Je ne tire pas ma gloire des hommes.
\VS{42}Mais je sais que vous n'avez point l'amour de Dieu en vous.
\VS{43}Je suis venu au Nom de mon Père, et vous ne me recevez pas, si un autre vient en son propre nom, vous le recevrez.
\VS{44}Comment pouvez-vous croire, puisque vous recevez la gloire les uns des autres, et ne cherchez point la gloire qui vient de Dieu seul ?
\VS{45}Ne croyez point que je vous accuserai devant mon Père ; Moïse sur qui vous vous fondez, est celui qui vous accusera.
\VS{46}Mais si vous croyiez Moïse, vous me croiriez aussi, parce qu'il a écrit au sujet de moi.
\VS{47}Mais si vous ne croyez pas à ses écrits, comment croirez-vous à mes paroles ?
\Chap{6}
\TextTitle{Multiplication des pains\FTNTT{Mt. 14:15-21 ; Mc. 6:32-44 ; Lu. 9:12-17.}}
\VerseOne{}Après ces choses, Jésus s'en alla au-delà de la mer de Galilée, qui est la mer de Tibériade.
\VS{2}Et une grande foule le suivait, parce qu'elle voyait les miracles qu'il opérait sur les malades.
\VS{3}Mais Jésus monta sur une montagne, et il s'assit là avec ses disciples.
\VS{4}Or la Pâque, la fête des Juifs, était proche.
\VS{5}Et Jésus ayant levé ses yeux, et voyant qu'une grande foule venait à lui, dit à Philippe : Où achèterons-nous des pains, afin que ces gens aient à manger ?
\VS{6}Or il disait cela pour l'éprouver, car il savait bien ce qu'il allait faire.
\VS{7}Philippe lui répondit : Les pains qu'on aurait pour deux cents deniers ne suffiraient pas pour que chacun en reçoive un peu.
\VS{8}Et l'un de ses disciples, André, frère de Simon Pierre, lui dit :
\VS{9}Il y a ici un petit garçon qui a cinq pains d'orge et deux poissons ; mais qu'est-ce que cela pour tant de gens ?
\VS{10}Alors Jésus dit : Faites asseoir les gens. Or il y avait beaucoup d'herbe dans ce lieu. Les gens donc, s'assirent au nombre d'environ cinq mille hommes.
\VS{11}Et Jésus prit les pains, et après avoir rendu grâces, il les distribua aux disciples, et les disciples à ceux qui étaient assis, et de même des poissons, autant qu'ils en voulaient.
\VS{12}Et après qu'ils furent rassasiés, il dit à ses disciples : Ramassez les morceaux qui restent, afin que rien ne soit perdu.
\VS{13}Ils les ramassèrent donc, et ils remplirent douze paniers avec les morceaux qui restèrent des cinq pains d'orge, après que tous eurent mangé.
\VS{14}Or ces gens, ayant vu le miracle que Jésus avait fait, disaient : Celui-ci est véritablement le Prophète qui devait venir dans le monde.
\TextTitle{Jésus marche sur les eaux\FTNTT{Mt. 14:22-33 ; Mc. 6:45-52.}}
\VS{15}Mais Jésus, sachant qu'ils allaient venir l'enlever pour le faire roi, se retira encore, lui seul, sur la montagne.
\VS{16}Et quand le soir fut venu, ses disciples descendirent à la mer.
\VS{17}Et étant montés dans la barque, ils traversaient la mer pour se rendre à Capernaüm. Il faisait déjà nuit, et Jésus n'était pas venu à eux.
\VS{18}Il soufflait un grand vent, et la mer était agitée.
\VS{19}Après avoir ramé environ vingt-cinq ou trente stades, ils virent Jésus marchant sur la mer, et s'approchant de la barque. Et ils eurent peur.
\VS{20}Mais il leur dit : C'est moi, ne craignez point.
\VS{21}Ils le reçurent donc avec plaisir dans la barque, et aussitôt la barque prit terre au lieu où ils allaient.
\TextTitle{Jésus le pain de vie, envoyé du ciel}
\VS{22}Le lendemain, la foule qui était restée de l'autre côté de la mer, vit qu'il ne se trouvait là qu'une seule barque, et que Jésus n'était pas monté avec ses disciples dans la barque, mais qu'ils étaient partis seuls.
\VS{23}Cependant, d'autres barques étaient arrivées de Tibériade près du lieu où ils avaient mangé le pain, après que le Seigneur eut rendu grâces.
\VS{24}Quand la foule vit que ni Jésus ni ses disciples n'étaient là, les gens montèrent eux-mêmes dans ces barques, et allèrent à Capernaüm chercher Jésus.
\VS{25}Et l'ayant trouvé au-delà de la mer, ils lui dirent : Maître, quand es-tu arrivé ici ?
\VS{26}Jésus leur répondit, et leur dit : En vérité, en vérité, je vous dis : Vous me cherchez, non parce que vous avez vu des miracles, mais parce que vous avez mangé des pains et que vous avez été rassasiés.
\VS{27}Travaillez, non pour la nourriture qui périt, mais pour celle qui est permanente jusqu'à la vie éternelle, laquelle le Fils de l'homme vous donnera ; car c'est lui que le Père, que Dieu, a marqué de son sceau.
\VS{28}Ils lui dirent donc : Que devons-nous faire pour accomplir les œuvres de Dieu ?
\VS{29}Jésus répondit, et leur dit : C'est ici l'œuvre de Dieu, que vous croyiez en celui qu'il a envoyé.
\VS{30}Alors ils lui dirent : Quel miracle fais-tu donc, afin que nous le voyions, et que nous croyions en toi ? Quelle œuvre fais-tu ?
\VS{31}Nos pères ont mangé la manne dans le désert, selon ce qui est écrit : Il leur a donné à manger le pain du ciel.
\VS{32}Mais Jésus leur dit : En vérité, en vérité je vous dis : Moïse ne vous a pas donné le pain du ciel ; mais mon Père vous donne le vrai pain du ciel.
\VS{33}Car le pain de Dieu, c'est celui qui est descendu du ciel et qui donne la vie au monde.
\VS{34}Ils lui dirent donc : Seigneur, donne-nous toujours ce pain-là.
\VS{35}Et Jésus leur dit : Je suis le pain de vie. Celui qui vient à moi n'aura jamais faim ; et celui qui croit en moi n'aura jamais soif.
\VS{36}Mais, je vous ai dit que vous m'avez vu, et cependant vous ne croyez point.
\VS{37}Tous ceux que mon Père me donne viendront à moi, et je ne mettrai point dehors celui qui viendra à moi.
\VS{38}Car je suis descendu du ciel, non point pour faire ma volonté, mais la volonté de celui qui m'a envoyé.
\VS{39}Et c'est ici la volonté du Père qui m'a envoyé, que je ne perde rien de tout ce qu'il m'a donné, mais que je le ressuscite au dernier jour.
\VS{40}Et c'est ici la volonté de celui qui m'a envoyé, que quiconque pose son regard sur le Fils, et croit en lui, ait la vie éternelle ; c'est pourquoi je le ressusciterai au dernier jour.
\VS{41}Or les Juifs murmuraient contre lui de ce qu'il avait dit : Je suis le pain qui est descendu du ciel.
\VS{42}Car ils disaient : N'est-ce pas là Jésus, le fils de Joseph, celui dont nous connaissons le père et la mère ? Comment donc dit-il : Je suis descendu du ciel ?
\VS{43}Jésus leur répondit, et leur dit : Ne murmurez pas entre vous.
\VS{44}Nul ne peut venir à moi, si le Père qui m'a envoyé ne l'attire ; et je le ressusciterai au dernier jour.
\VS{45}Il est écrit dans les prophètes : Ils seront tous enseignés de Dieu. Donc, quiconque a entendu le Père et a été instruit de ses intentions, vient à moi.
\VS{46}C'est que nul n'a vu le Père, sinon celui qui vient de Dieu, celui-là a vu le Père.
\VS{47}En vérité, en vérité je vous dis : Celui qui croit en moi a la vie éternelle.
\VS{48}Je suis le pain de vie.
\VS{49}Vos pères ont mangé la manne dans le désert, et ils sont morts.
\VS{50}C'est ici le pain qui est descendu du ciel, afin que celui qui en mange, ne meure point.
\VS{51}Je suis le pain vivant qui est descendu du ciel. Si quelqu'un mange de ce pain, il vivra éternellement ; et le pain que je donnerai, c'est ma chair, que je donnerai pour la vie du monde.
\VS{52}Les Juifs donc discutaient entre eux, et disaient : Comment peut-il nous donner sa chair à manger ?
\VS{53}Et Jésus leur dit : En vérité, en vérité, je vous le dis : Si vous ne mangez pas la chair du Fils de l'homme, et ne buvez pas son sang, vous n'aurez point la vie en vous-mêmes.
\VS{54}Celui qui mange ma chair, et qui boit mon sang, a la vie éternelle ; et je le ressusciterai au dernier jour.
\VS{55}Car ma chair est une véritable nourriture, et mon sang est un véritable breuvage.
\VS{56}Celui qui mange ma chair, et qui boit mon sang, demeure en moi, et moi en lui.
\VS{57}Comme le Père qui est vivant m'a envoyé, et que je suis vivant par le Père, ainsi celui qui me mangera vivra aussi par moi.
\VS{58}C'est ici le pain qui est descendu du ciel. Il n'en est pas comme de vos pères qui ont mangé la manne, et qui sont morts ; celui qui mangera ce pain, vivra éternellement.
\VS{59}Il dit ces choses dans la synagogue, enseignant à Capernaüm.
\TextTitle{Mise à l'épreuve de la consécration des disciples\FTNTT{Mt. 8:19-22 ; 10:36 ; Lu. 9:23-26.}}
\VS{60}Plusieurs de ses disciples l'ayant entendu, dirent : Cette parole est dure, qui peut l'écouter ?
\VS{61}Mais Jésus sachant en lui-même que ses disciples murmuraient à ce sujet, leur dit : Cela vous scandalise-t-il ?
\VS{62}Que sera-ce donc si vous voyez le Fils de l'homme monter où il était auparavant ?
\VS{63}C'est l'Esprit qui vivifie ; la chair ne sert à rien. Les paroles que je vous ai dites sont Esprit et vie.
\VS{64}Mais il en est certains parmi vous qui ne croient point. En effet, Jésus savait dès le commencement qui étaient ceux qui ne croiraient point, et qui était celui qui le trahirait.
\VS{65}Il leur dit donc : C'est pour cela que je vous ai dit, que nul ne peut venir à moi, si cela ne lui a pas été donné par mon Père.
\TextTitle{Pierre reconnaît Jésus comme le Messie\FTNTT{Mt. 16:13-16 ; Mc. 8:27-30 ; Lu. 9:18-21.}}
\VS{66}Dès lors, plusieurs de ses disciples l'abandonnèrent, et ils ne marchèrent plus avec lui.
\VS{67}Et Jésus dit aux douze : Et vous, ne voulez-vous pas aussi vous en aller ?
\VS{68}Mais Simon Pierre lui répondit : Seigneur ! Auprès de qui irions-nous ? Tu as les paroles de la vie éternelle.
\VS{69}Et nous avons cru et nous avons connu que tu es le Christ, le Fils du Dieu vivant.
\VS{70}Jésus leur répondit : Ne vous ai-je pas choisis, vous les douze ? Et toutefois l'un de vous est un démon.
\VS{71}Or il parlait de Judas Iscariot, fils de Simon ; car c'était lui qui devait le trahir, quoiqu'il fût l'un des douze.
\Chap{7}
\TextTitle{Incrédulité des frères de Jésus}
\VerseOne{}Après ces choses, Jésus parcourait la Galilée, car il ne voulait pas parcourir la Judée, parce que les Juifs cherchaient à le faire mourir.
\VS{2}Or la fête des Juifs, appelée la fête des tabernacles, était proche.
\VS{3}Et ses frères lui dirent : Pars d'ici, et va en Judée, afin que tes disciples aussi contemplent les œuvres que tu fais.
\VS{4}Personne n'agit en secret, lorsqu'il cherche à être connu ; si tu fais ces choses, montre-toi toi-même au monde.
\VS{5}Car ses frères non plus ne croyaient pas en lui.
\VS{6}Et Jésus leur dit : Mon temps n'est pas encore venu, mais votre temps est toujours prêt.
\VS{7}Le monde ne peut pas vous haïr mais il me hait parce que je rends témoignage contre lui que ses œuvres sont mauvaises.
\VS{8}Montez, vous, à cette fête ; pour moi, je n'y monte pas encore, parce que mon temps n'est pas encore accompli.
\VS{9}Après leur avoir dit ces choses, il resta en Galilée.
\TextTitle{Jésus à la fête des tabernacles}
\VS{10}Lorsque ses frères furent montés, alors il y monta aussi lui-même, non publiquement, mais comme en secret.
\VS{11}Les Juifs le cherchaient pendant la fête, et ils disaient : Où est-il ?
\VS{12}Et il y avait un grand murmure à son sujet parmi la foule. Les uns disaient : C'est un homme de bien ; et les autres disaient : Non, il séduit le peuple.
\VS{13}Toutefois personne ne parlait franchement de lui, à cause de la crainte qu'on avait des Juifs.
\VS{14}Vers le milieu de la fête, Jésus monta au temple. Et il enseignait.
\VS{15}Les Juifs s'étonnaient, disant : Comment connaît-il les Ecritures, lui qui n'a point étudié ?
\VS{16}Jésus leur répondit et dit : Ma doctrine n'est pas de moi, mais de celui qui m'a envoyé.
\VS{17}Si quelqu'un veut faire sa volonté, il connaîtra si ma doctrine est de Dieu, ou si je parle de moi-même.
\VS{18}Celui qui parle de son propre chef cherche sa propre gloire ; mais celui qui cherche la gloire de celui qui l'a envoyé, est véritable, et il n'y a point d'injustice en lui.
\VS{19}Moïse ne vous a-t-il pas donné la loi ? Cependant, nul de vous n'observe la loi. Pourquoi cherchez-vous à me faire mourir ?
\VS{20}La foule répondit : Tu as un démon ; qui est-ce qui cherche à te faire mourir ?
\VS{21}Jésus répondit, et leur dit : J'ai fait une œuvre, et vous en êtes tous étonnés.
\VS{22}Moïse vous a donné la circoncision, non qu'elle vienne de Moïse, mais des pères, vous circoncisez bien un homme le jour du sabbat.
\VS{23}Si un homme reçoit la circoncision le jour du sabbat, afin que la loi de Moïse ne soit pas violée, pourquoi êtes-vous irrités contre moi de ce que j'ai guéri un homme tout entier le jour du sabbat ?
\VS{24}Ne jugez pas selon les apparences, mais jugez selon la justice.
\VS{25}Alors quelques-uns de ceux de Jérusalem disaient : N'est-ce pas celui qu'ils cherchent à faire mourir ?
\VS{26}Et cependant voici, il parle librement, et ils ne lui disent rien ! Est-ce que vraiment les chefs auraient reconnu qu'il est véritablement le Christ ?
\VS{27}Or nous savons bien d'où est celui-ci, mais quand le Christ viendra, personne ne saura d'où il est.
\VS{28}Jésus donc criait dans le temple enseignant, et disant : et vous me connaissez, et vous savez d'où je suis ; et je ne suis point venu de moi-même, mais celui qui m'a envoyé est véritable, et vous ne le connaissez point. 
\VS{29}Mais moi, je le connais ; car je suis issu de lui, et c'est lui qui m'a envoyé.
\VS{30}Ils cherchaient donc à se saisir de lui, mais personne ne mit la main sur lui, parce que son heure n'était pas encore venue.
\VS{31}Et plusieurs d'entre les foules crurent en lui, et ils disaient : Quand le Christ sera venu, fera-t-il plus de miracles que celui-ci n'a fait ?
\VS{32}Les Pharisiens entendirent la foule murmurant ces choses de lui ; et les Pharisiens, avec les principaux sacrificateurs envoyèrent des huissiers pour le prendre.
\VS{33}Et Jésus leur dit : Je suis encore pour un peu de temps avec vous, puis je m'en vais vers celui qui m'a envoyé.
\VS{34}Vous me chercherez, mais vous ne me trouverez pas, et vous ne pouvez pas venir là où je serai.
\VS{35}Les Juifs dirent donc entre eux : Où ira-t-il, pour que nous ne le trouvions pas ? Ira-t-il parmi ceux qui sont dispersés chez les Grecs, et enseignera-t-il les Grecs ?
\VS{36}Quel est ce discours qu'il a tenu : Vous me chercherez, mais vous ne me trouverez pas, vous ne pouvez pas venir là où je serai ?
\TextTitle{La puissance du Saint-Esprit\FTNTT{Jn. 4:14 ; Ac. 2:2-4.}}
\VS{37}Le dernier jour, le grand jour de la fête, Jésus, se tenant debout, s'écria : Si quelqu'un a soif, qu'il vienne à moi, et qu'il boive.
\VS{38}Celui qui croit en moi, des fleuves d'eau vive couleront de son sein, comme dit l'Ecriture.
\VS{39}Or il dit cela de l'Esprit que devaient recevoir ceux qui croiraient en lui ; car le Saint-Esprit n'était pas encore donné, parce que Jésus n'était pas encore glorifié.
\TextTitle{Opinions diverses au sujet de Jésus}
\VS{40}Plusieurs de la foule ayant entendu ce discours, disaient : Celui-ci est véritablement le Prophète.
\VS{41}Les autres disaient : Celui-ci est le Christ. Et les autres disaient : Mais le Christ viendra-t-il de la Galilée ?
\VS{42}L'Ecriture ne dit-elle pas que le Christ doit venir de la postérité de David, et du village de Bethléhem, où était David ?
\VS{43}Il y eut donc de la division parmi la foule à cause de lui.
\VS{44}Et quelques-uns d'entre eux voulaient le saisir, mais personne ne mit la main sur lui.
\VS{45}Ainsi les huissiers retournèrent vers les principaux sacrificateurs et les pharisiens, qui leur dirent : Pourquoi ne l'avez-vous pas amené ?
\VS{46}Les huissiers répondirent : Jamais homme n'a parlé comme cet homme.
\VS{47}Mais les pharisiens leur répondirent : Est-ce que vous aussi, vous avez été séduits ?
\VS{48}Y a-t-il quelqu'un des chefs ou des pharisiens qui ait cru en lui ?
\VS{49}Mais cette foule, qui ne connaît pas la loi, ce sont des maudits.
\VS{50}Nicodème, celui qui était venu vers Jésus de nuit, et qui était l'un d'entre eux, leur dit :
\VS{51}Notre loi condamne-t-elle un homme avant qu'on l'entende et qu'on ne sache ce qu'il a fait ?
\VS{52}Ils lui répondirent : Es-tu aussi Galiléen ? Examine, et tu verras qu'aucun prophète n'est sorti de la Galilée.
\VS{53}Et chacun s'en alla dans sa maison.
\Chap{8}
\TextTitle{La femme adultère}
\VerseOne{}Mais Jésus se rendit à la Montagne des Oliviers.
\VS{2}Et, dès le matin, il alla de nouveau dans le temple, et tout le peuple vint à lui. Et s'étant assis, il les enseignait.
\VS{3}Alors les scribes et les pharisiens lui amenèrent une femme surprise en adultère ;
\VS{4}et, l'ayant placée au milieu du peuple, ils lui dirent : Maître, cette femme a été surprise en flagrant délit d'adultère.
\VS{5}Or Moïse nous a ordonné dans la loi de lapider celles qui sont dans son cas ; toi donc qu'en dis-tu ?
\VS{6}Or ils disaient cela pour l'éprouver, afin de pouvoir l'accuser. Mais Jésus s'étant penché en bas, écrivait avec son doigt sur la terre.
\VS{7}Et comme ils continuaient à l'interroger, s'étant relevé, il leur dit : Que celui de vous qui est sans péché, jette le premier la pierre contre elle.
\VS{8}Et s'étant encore baissé, il écrivait sur la terre.
\VS{9}Quand ils entendirent cela, accusés par leur conscience, ils se retirèrent un à un, depuis les plus âgés jusqu'aux derniers ; et Jésus resta seul avec la femme qui était là au milieu.
\VS{10}Alors Jésus s'étant relevé, et ne voyant plus que la femme, il lui dit : Femme, où sont ceux qui t'accusaient ? Personne ne t'a-t-il condamnée ?
\VS{11}Elle dit : Personne Seigneur. Et Jésus lui dit : Je ne te condamne pas non plus ; va, et ne pèche plus.
\TextTitle{Témoignage du Messie\FTNTT{Jn. 1:9.}}
\VS{12}Et Jésus leur parla encore, en disant : Je suis la lumière du monde ; celui qui me suit ne marchera pas dans les ténèbres, mais il aura la lumière de la vie.
\VS{13}Alors les pharisiens lui dirent : Tu rends témoignage de toi-même, ton témoignage n'est pas digne de foi.
\VS{14}Jésus répondit, et leur dit : Quoique je rende témoignage de moi-même, mon témoignage est digne de foi ; car je sais d'où je suis venu et où je vais ; mais vous ne savez pas d'où je viens ni où je vais.
\VS{15}Vous jugez selon la chair, mais moi, je ne juge personne.
\VS{16}Et si je juge, mon jugement est digne de foi, car je ne suis pas seul, mais avec moi est le Père qui m'a envoyé.
\VS{17}Il est même écrit dans votre loi que le témoignage de deux hommes est digne de foi\FTNT{De. 19:15.}.
\VS{18}Je rends témoignage de moi-même, et le Père qui m'a envoyé rend aussi témoignage de moi.
\VS{19}Alors ils lui dirent : Où est ton Père ? Jésus répondit : Vous ne connaissez ni moi, ni mon Père. Si vous me connaissiez, vous connaîtriez aussi mon Père.
\VS{20}Jésus dit ces paroles au lieu où était le trésor, enseignant dans le temple ; mais personne ne le saisit, parce que son heure n'était pas encore venue.
\VS{21}Et Jésus leur dit encore : Je m'en vais, et vous me chercherez, et vous mourrez dans vos péchés ; vous ne pouvez pas venir là où je vais.
\VS{22}Les Juifs disaient donc : Se tuera-t-il lui-même, puisqu'il dit : Vous ne pouvez pas venir là où je vais ?
\VS{23}Alors il leur dit : Vous êtes d'en bas, mais moi, je suis d'en haut. Vous êtes de ce monde, mais moi, je ne suis pas de ce monde.
\VS{24}C'est pourquoi je vous ai dit que vous mourrez dans vos péchés ; car si vous ne croyez pas que je suis l'envoyé de Dieu, vous mourrez dans vos péchés.
\VS{25}Alors ils lui dirent : Toi, qui es-tu ? Et Jésus leur dit : Ce que je vous dis dès le commencement.
\VS{26}J'ai beaucoup de choses à dire de vous et à condamner en vous, mais celui qui m'a envoyé est véritable, et les choses que j'ai entendues de lui, je les dis au monde.
\VS{27}Ils ne comprirent point qu'il leur parlait du Père.
\VS{28}Jésus leur dit donc : Quand vous aurez élevé le Fils de l'homme, vous connaîtrez alors que je suis l'envoyé de Dieu, et que je ne fais rien de moi-même, mais que je dis ces choses selon ce que mon Père m'a enseigné.
\VS{29}Car celui qui m'a envoyé est avec moi ; le Père ne m'a pas laissé seul, parce que je fais toujours les choses qui lui plaisent.
\VS{30}Comme il disait ces choses, plusieurs crurent en lui.
\VS{31}Et Jésus disait aux Juifs qui avaient cru en lui : Si vous demeurez dans ma parole, vous serez vraiment mes disciples.
\VS{32}Vous connaîtrez la vérité, et la vérité vous rendra libres.
\VS{33}Ils lui répondirent : Nous sommes la postérité d'Abraham, et nous ne fûmes jamais esclaves de personne ; comment donc dis-tu : Vous serez rendus libres ?
\VS{34}Jésus leur répondit : En vérité, en vérité, je vous dis : Quiconque se livre au péché, est esclave du péché.
\VS{35}Or l'esclave ne demeure pas toujours dans la maison ; le fils y demeure toujours.
\VS{36}Si donc le Fils vous affranchit, vous serez véritablement libres.
\VS{37}Je sais que vous êtes la postérité d'Abraham, pourtant vous cherchez à me faire mourir, parce que ma parole n'est pas reçue dans vos cœurs.
\VS{38}Je vous dis ce que j'ai vu chez mon Père ; et vous aussi vous faites les choses que vous avez vues chez votre père.
\VS{39}Ils répondirent et lui dirent : Notre père, c'est Abraham. Jésus leur dit : Si vous étiez enfants d'Abraham, vous feriez les œuvres d'Abraham.
\VS{40}Mais maintenant vous cherchez à me faire mourir, moi, un homme qui vous ai dit la vérité que j'ai entendue de Dieu. Cela, Abraham ne l'a point fait.
\VS{41}Vous faites les œuvres de votre père. Et ils lui dirent : Nous ne sommes pas issus de la fornication ; nous avons un seul père, Dieu.
\VS{42}Mais Jésus leur dit : Si Dieu était votre Père, certes vous m'aimeriez, puisque je suis issu de Dieu, et que je viens de lui ; car je ne suis point venu de moi-même, mais c'est lui qui m'a envoyé.
\VS{43}Pourquoi ne comprenez-vous pas mon langage ? C'est parce que vous ne pouvez pas écouter ma parole.
\VS{44}Le père dont vous êtes issus c'est le diable, et vous voulez accomplir les désirs de votre père. Il a été meurtrier dès le commencement, et il n'a pas persévéré dans la vérité, car la vérité n'est pas en lui. Toutes les fois qu'il profère le mensonge, il parle de son propre fonds ; car il est menteur et le père du mensonge.
\VS{45}Mais pour moi, parce que je dis la vérité, vous ne me croyez pas.
\VS{46}Qui de vous me convaincra de péché ? Et si je dis la vérité, pourquoi ne me croyez-vous pas ?
\VS{47}Celui qui est de Dieu écoute les paroles de Dieu ; vous n'écoutez pas, parce que vous n'êtes pas de Dieu.
\VS{48}Alors les Juifs répondirent : N'avons-nous pas raison de dire que tu es un Samaritain, et que tu as un démon ?
\VS{49}Jésus répondit : Je n'ai point un démon, mais j'honore mon Père, et vous m'outragez.
\VS{50}Or je ne cherche point ma gloire ; il y en a un qui la cherche, et qui juge.
\VS{51}En vérité, en vérité je vous dis : Si quelqu'un garde ma parole, il ne verra jamais la mort.
\VS{52}Les Juifs lui dirent donc : Maintenant nous savons que tu as un démon. Abraham est mort, et les prophètes aussi, et tu dis : Si quelqu'un garde ma parole, il ne verra jamais la mort.
\VS{53}Es-tu plus grand que notre père Abraham qui est mort ? Les prophètes aussi sont morts. Qui prétends-tu être ?
\VS{54}Jésus répondit : Si je me glorifie moi-même, ma gloire n'est rien ; mon Père est celui qui me glorifie, celui que vous dites être votre Dieu.
\VS{55}Toutefois vous ne l'avez point connu, mais moi je le connais ; et si je disais que je ne le connais point, je serais un menteur, semblable à vous ; mais je le connais, et je garde sa parole.
\VS{56}Abraham votre père a tressailli de joie de ce qu'il verrait mon jour ; et il l'a vu, et il s'est réjoui.
\VS{57}Sur cela les Juifs lui dirent : Tu n'as pas encore cinquante ans, et tu as vu Abraham !
\VS{58}Et Jésus leur dit : En vérité, en vérité je vous le dis : Avant qu'Abraham fût, Je suis\FTNT{Je suis : L'évangile de Jean rapporte plusieurs déclarations incroyables que Jésus a faites à son sujet : Je suis le pain de vie (6:35), Je suis la lumière du monde (8:12), Je suis le bon berger (10:11), Je suis la porte (10:7), Je suis la résurrection (11:25), Je suis le chemin, la vérité et la vie (14:6), Je suis le vrai cep (15:1). Toutefois, dans ce verset, en déclarant être « Je suis », il s'identifie clairement au Nom que YHWH avait révélé à Moïse dans Ex. 3:14. C'est précisément pour cette raison que les Juifs ont voulu le lapider.}.
\VS{59}Alors ils prirent des pierres pour les jeter contre lui, mais Jésus se cacha et sortit du temple, passant au milieu d'eux ; et ainsi il s'en alla.
\Chap{9}
\TextTitle{Jésus guérit un aveugle-né}
\VerseOne{}Comme Jésus passait, il vit un homme aveugle de naissance.
\VS{2}Et ses disciples l'interrogèrent, disant : Maître, qui a péché ? Celui-ci, ou son père, ou sa mère, pour qu'il soit né aveugle ?
\VS{3}Jésus répondit : Ni celui-ci n'a péché, ni son père, ni sa mère ; mais c'est afin que les œuvres de Dieu soient manifestées en lui.
\VS{4}Il faut que je fasse, tandis qu'il est jour, les œuvres de celui qui m'a envoyé. La nuit vient, où personne ne peut travailler.
\VS{5}Pendant que je suis dans le monde, je suis la Lumière du monde.
\VS{6}Ayant dit ces paroles, il cracha à terre et fit de la boue avec sa salive, et mit de cette boue sur les yeux de l'aveugle.
\VS{7}Et il lui dit : Va, et lave-toi au réservoir de Siloé (nom qui veut dire envoyé). Il y alla donc, se lava, et s'en retourna voyant clair.
\VS{8}Or ses voisins et ceux qui auparavant l'avaient connu comme mendiant disaient : N'est-ce pas celui qui était assis et qui mendiait ?
\VS{9}Les uns disaient : C'est lui. Et les autres disaient : Il lui ressemble. Mais lui il disait : C'est moi-même.
\VS{10}Ils lui dirent donc : Comment tes yeux ont-ils été ouverts ?
\VS{11}Il répondit et dit : Cet homme, qu'on appelle Jésus, a fait de la boue et il l'a mise sur mes yeux, et m'a dit : Va au réservoir de Siloé et lave-toi. J'y suis allé, je me suis lavé, et j'ai recouvert la vue.
\VS{12}Alors ils lui dirent : Où est cet homme ? Il répondit : Je ne sais pas.
\VS{13}Ils amenèrent aux pharisiens celui qui auparavant avait été aveugle.
\VS{14}Or c'était en un jour de sabbat que Jésus avait fait de la boue et lui avait ouvert les yeux.
\VS{15}C'est pourquoi les pharisiens l'interrogèrent encore, comment il avait recouvré la vue ; et il leur dit : Il a mis de la boue sur mes yeux, et je me suis lavé, et je vois.
\VS{16}Sur quoi quelques-uns d'entre les pharisiens dirent : Cet homme n'est pas un envoyé de Dieu, car il n'observe pas le sabbat. Mais d'autres disaient : Comment un homme pécheur peut-il faire de tels prodiges ? Et il y avait de la division entre eux.
\VS{17}Ils dirent encore à l'aveugle : Toi, que dis-tu de lui, sur ce qu'il t'a ouvert les yeux ? Il répondit : C'est un Prophète.
\VS{18}Mais les Juifs ne crurent point que cet homme avait été aveugle et qu'il avait recouvré la vue, jusqu'à ce qu'ils eurent appelé le père, et la mère de celui qui avait recouvré la vue.
\VS{19}Et ils les interrogèrent, disant : Est-ce là votre fils, que vous dites être né aveugle ? Comment donc voit-il maintenant ?
\VS{20}Son père et sa mère leur répondirent : Nous savons que c'est ici notre fils et qu'il est né aveugle.
\VS{21}Mais comment il voit maintenant, ou qui lui a ouvert les yeux, nous ne le savons pas ; il a de l'âge, interrogez-le, il parlera de ce qui le regarde.
\VS{22}Son père et sa mère dirent ces choses parce qu'ils craignaient les Juifs ; car les Juifs avaient déjà convenu que si quelqu'un reconnaissait Jésus pour le Christ, il serait exclu de la synagogue.
\VS{23}Pour cette raison son père et sa mère dirent : Il a de l'âge, interrogez-le lui-même.
\VS{24}Ils appelèrent donc pour la seconde fois l'homme qui avait été aveugle et ils lui dirent : Donne gloire à Dieu ; nous savons que cet homme est un pécheur.
\VS{25}Il répondit : Je ne sais pas si c'est un pécheur ; je sais une chose, c'est que j'étais aveugle et que maintenant je vois.
\VS{26}Ils lui dirent donc encore : Que t'a-t-il fait ? Comment a-t-il ouvert tes yeux ?
\VS{27}Il leur répondit : Je vous l'ai déjà dit, et vous ne l'avez point écouté, pourquoi voulez-vous l'entendre encore ? Voulez-vous aussi être ses disciples ?
\VS{28}Alors ils l'injurièrent et lui dirent : Toi sois son disciple ; nous, nous sommes disciples de Moïse.
\VS{29}Nous savons que Dieu a parlé à Moïse ; mais celui-ci, nous ne savons pas d'où il est.
\VS{30}L'homme répondit et leur dit : Certes, c'est une chose étrange que vous ne sachiez point d'où il est ; et toutefois il a ouvert mes yeux.
\VS{31}Or nous savons que Dieu n'exauce point les pécheurs, mais si quelqu'un est pieux envers Dieu, et fait sa volonté, il l'exauce.
\VS{32}On n'a jamais entendu dire que quelqu'un ait ouvert les yeux d'un aveugle-né.
\VS{33}Si cet homme n'était pas un envoyé de Dieu, il ne pourrait rien faire de semblable.
\VS{34}Ils répondirent : Tu es entièrement né dans le péché, et tu nous enseignes ! Et ils le chassèrent dehors.
\TextTitle{Jésus affirme sa divinité}
\VS{35}Jésus apprit qu'ils l'avaient chassé dehors ; et, l'ayant rencontré, il lui dit : Crois-tu au Fils de Dieu ?
\VS{36}Cet homme lui répondit : Qui est-il Seigneur, afin que je croie en lui ?
\VS{37}Jésus lui dit : Tu l'as vu, et c'est celui qui te parle.
\VS{38}Alors il dit : Je crois, Seigneur ; et il l'adora\FTNT{Au travers de la lecture de la Bible, on constate que les anges refusent l'adoration (Ap. 19:9-10) de même que les apôtres (Ac. 10:25-26 ; Ac. 14:5-18). Seul Dieu accepte l'adoration puisqu'il en est le seul digne. Jésus n'a jamais refusé l'adoration des hommes, car il est Dieu.}.
\VS{39}Et Jésus dit : Je suis venu dans ce monde pour exercer le jugement, afin que ceux qui ne voient point voient ; et que ceux qui voient deviennent aveugles.
\VS{40}Et quelques-uns des pharisiens qui étaient avec lui, ayant entendu ces paroles, dirent : Et nous, sommes-nous aussi aveugles ?
\VS{41}Jésus leur répondit : Si vous étiez aveugles, vous n'auriez point de péché ; mais maintenant vous dites : Nous voyons. C'est à cause de cela que votre péché demeure.
\Chap{10}
\TextTitle{Jésus, le bon berger\FTNTT{Ps. 23 ; Hé. 13:20 ; 1 Pi. 5:4.}}
\VerseOne{}En vérité, en vérité, je vous dis : Celui qui n'entre point par la porte dans la bergerie des brebis, mais y monte par ailleurs, est un voleur et un brigand.
\VS{2}Mais celui qui entre par la porte est le berger des brebis.
\VS{3}Le portier lui ouvre, et les brebis entendent sa voix, et il appelle ses propres brebis par leur nom, et il les conduit dehors.
\VS{4}Et quand il a fait sortir toutes ses brebis dehors, il marche devant elles, et les brebis le suivent, parce qu'elles connaissent sa voix.
\VS{5}Mais elles ne suivront point un étranger, au contraire, elles fuiront loin de lui ; parce qu'elles ne connaissent point la voix des étrangers.
\VS{6}Jésus leur dit cette parabole, mais ils ne comprirent pas ce qu'il leur disait.
\VS{7}Jésus donc leur dit encore : En vérité, en vérité, je vous dis : Je suis la porte par où entrent les brebis\FTNT{Les animaux destinés au sacrifice passaient sans doute par la porte des brebis construite du temps de Néhémie (Né. 3 :1)..}.
\VS{8}Tout ceux qui sont venus avant moi sont des brigands et des voleurs ; mais les brebis ne les ont point écoutés.
\VS{9}Je suis la porte. Si quelqu'un entre par moi, il sera sauvé ; il entrera et il sortira, et il trouvera des pâturages.
\VS{10}Le voleur ne vient que pour dérober, tuer et détruire ; moi, je suis venu afin que mes brebis aient la vie, et qu'elles l'aient même en abondance.
\VS{11}Je suis le bon berger : Le bon berger donne sa vie pour ses brebis.
\VS{12}Mais le mercenaire, qui n'est pas le berger, à qui n'appartiennent pas les brebis, voit venir le loup, abandonne les brebis, et s'enfuit ; et le loup ravit et disperse les brebis.
\VS{13}Ainsi le mercenaire s'enfuit, parce qu'il est mercenaire, et qu'il ne se soucie pas des brebis. Je suis le bon berger.
\VS{14}Je connais mes brebis, et mes brebis me connaissent.
\VS{15}Comme le Père me connaît, je connais aussi le Père, et je donne ma vie pour mes brebis.
\VS{16}J'ai encore d'autres brebis qui ne sont pas de cette bergerie ; celles-là, il faut aussi que je les amène ; elles entendront ma voix, et il y aura un seul troupeau, et un seul berger.
\VS{17}Le Père m'aime, parce que je donne ma vie, afin de la reprendre.
\VS{18}Personne ne me l'ôte, mais je la donne de moi-même. J'ai le pouvoir de la donner, et j'ai le pouvoir de la reprendre ; j'ai reçu cet ordre de mon Père.
\VS{19}Il y eut de nouveau de la division parmi les Juifs à cause de ces discours.
\VS{20}Car plusieurs disaient : Il a un démon, il est fou ! Pourquoi l'écoutez-vous ?
\VS{21}Et les autres disaient : Ce ne sont pas les paroles d'un démoniaque ; un démon peut-il ouvrir les yeux des aveugles ?
\TextTitle{Jésus affirme de nouveau sa divinité\FTNTT{Jn. 5:26-27 ; 14:9 ; 20:28-29.}}
\VS{22}Or on célébrait la fête de la dédicace\FTNT{Le mot « dédicace » se dit « chanukkah » en hébreu, ce qui signifie « consécration ». Ce terme est employé pour l’inauguration et la consécration de l’autel destiné aux sacrifices rituels (No. 7 :10-88), du temple (1 R. 8 :63 ; 2 Ch. 7 :5) ou encore pour les murailles de Jérusalem (Né. 12 :27). La fête de Hanoukka, célébrée chaque année le 25 du mois de chisleu (mi-décembre), fut instituée en 164 av. J.-C.  par Judas Maccabée pour commémorer la purification du temple profané par Antiochus Epiphane en 168 et 167 av. J.-C.} à Jérusalem, et c'était l'hiver.
\VS{23}Et Jésus se promenait dans le temple, au portique de Salomon.
\VS{24}Et les Juifs l'entourèrent, et lui dirent : Jusqu'à quand tiendras-tu notre âme en suspens ? Si tu es le Christ, dis-le-nous franchement.
\VS{25}Jésus leur répondit : Je vous l'ai dit, et vous ne le croyez point. Les œuvres que je fais au Nom de mon Père rendent témoignage de moi.
\VS{26}Mais vous ne croyez point, parce que vous n'êtes point de mes brebis, comme je vous l'ai dit.
\VS{27}Mes brebis entendent ma voix ; je les connais, et elles me suivent.
\VS{28}Et moi, je leur donne la vie éternelle, et elles ne périront jamais, et personne ne les ravira de ma main.
\VS{29}Mon Père, qui me les a données, est plus grand que tous ; et personne ne peut les ravir des mains de mon Père.
\VS{30}Moi et le Père nous sommes un.
\VS{31}Alors les Juifs prirent de nouveau des pierres pour le lapider.
\VS{32}Et Jésus leur dit : Je vous ai fait voir plusieurs bonnes œuvres de la part de mon Père : Pour laquelle me lapidez-vous ?
\VS{33}Les Juifs répondirent, en lui disant : Nous ne te lapidons pas pour aucune bonne oeuvre, mais pour un blasphème et parce que n'étant qu'un homme tu te fais Dieu.
\VS{34}Jésus leur répondit : N'est-il pas écrit dans votre loi : J'ai dit : Vous êtes des dieux\FTNT{Ps. 82:6 : Le sens du mot « dieu » peut désigner des personnes ayant un certain pouvoir. D'ailleurs, le mot hébreu utilisé dans Ps. 82:6 est « Elohim », or ce mot signifie aussi « juge ». De plus, dans le contexte du psaume, « vous êtes des dieux » ne s'applique pas à tous, mais seulement à une certaine catégorie de personnes qui exerçaient un pouvoir en Israël : rois, scribes, souverains sacrificateurs, etc. Rappelons-nous aussi que Dieu a fait de Moïse un dieu pour Aaron (Ex. 7:1-2), mais cela n'a pas fait de lui le Dieu Créateur pour autant. En Jn. 17:3, Jésus atteste qu'il n'y a qu'un seul vrai Dieu. Satan veut nous faire croire que nous sommes des dieux et nous amener ainsi à pécher par l'orgueil (Ge. 3:5). Toutefois, comme le souligne si bien l'apôtre Paul, même s'il existe des créatures qu'on appelle dieux ou déesses, il ne reste pas moins vrai qu'il n'y a qu'un seul Dieu (1 Co. 8:5-7).} ?
\VS{35}Si elle a donc appelé dieux ceux à qui la parole de Dieu est adressée, et cependant l'Ecriture ne peut être anéantie.
\VS{36}Celui que le Père a sanctifié et envoyé dans le monde, vous lui dites : Tu blasphèmes ! Et cela parce que j'ai dit : Je suis le Fils de Dieu ?
\VS{37}Si je ne fais pas les œuvres de mon Père, ne me croyez pas.
\VS{38}Mais si je les fais, même si vous ne me croyez pas, croyez à ces œuvres, afin que vous sachiez que le Père est en moi et que je suis dans le Père.
\VS{39}À cause de cela, ils cherchaient encore à le saisir, mais il s'échappa de leurs mains.
\TextTitle{Jésus se retire de Jérusalem}
\VS{40}Il s'en alla de nouveau au-delà du Jourdain, à l'endroit où Jean avait baptisé au commencement, et il demeura là.
\VS{41}Beaucoup de gens vinrent à lui, et ils disaient : Jean n'a fait aucun miracle ; mais tout ce que Jean a dit de cet homme était vrai.
\VS{42}Et dans ce lieu-là, plusieurs crurent en lui.
\Chap{11}
\TextTitle{Résurrection de Lazare de Béthanie}
\VerseOne{}Il y avait un certain homme malade, appelé Lazare, qui était de Béthanie, village de Marie et de Marthe, sa sœur.
\VS{2}C'était cette Marie qui oignit de parfum le Seigneur, et qui essuya ses pieds avec ses cheveux ; et c'était son frère Lazare qui était malade.
\VS{3}Ses sœurs envoyèrent donc dire à Jésus : Seigneur, voici, celui que tu aimes est malade.
\VS{4}Après avoir entendu cela, Jésus dit : Cette maladie n'est point à la mort, mais elle est pour la gloire de Dieu, afin que le Fils de Dieu soit glorifié par elle.
\VS{5}Or Jésus aimait Marthe, sa sœur, et Lazare.
\VS{6}Et après qu'il eut appris que Lazare était malade, il resta deux jours encore dans le lieu où il était,
\VS{7}et il dit à ses disciples : Retournons en Judée.
\VS{8}Les disciples lui dirent : Maître, les Juifs tout récemment cherchaient à te lapider, et tu retournes en Judée !
\VS{9}Jésus répondit : N'y a-t-il pas douze heures au jour ? Si quelqu'un marche pendant le jour, il ne bronche point ; car il voit la lumière de ce monde.
\VS{10}Mais si quelqu'un marche pendant la nuit, il bronche ; car il n'y a point de lumière avec lui.
\VS{11}Après ces paroles, il leur dit : Notre ami Lazare dort, mais je vais le réveiller.
\VS{12}Ses disciples lui dirent : Seigneur, s'il dort, il sera guéri.
\VS{13}Or Jésus avait parlé de sa mort, mais ils pensaient qu'il parlait du repos du sommeil.
\VS{14}Alors Jésus leur dit ouvertement : Lazare est mort.
\VS{15}Et je me réjouis, à cause de vous, de ce que je n'étais pas là, afin que vous croyiez. Mais allons vers lui.
\VS{16}Alors Thomas, appelé Didyme, dit aux autres disciples : Allons-y aussi, afin que nous mourions avec lui.
\VS{17}Jésus, étant donc arrivé, trouva que Lazare était déjà depuis quatre jours dans le sépulcre.
\VS{18}Or Béthanie était près de Jérusalem, à quinze stades environ.
\VS{19}Et beaucoup de Juifs étaient venus vers Marthe et Marie pour les consoler au sujet de leur frère.
\VS{20}Lorsque Marthe apprit que Jésus arrivait, elle alla au-devant de lui ; mais Marie se tenait assise à la maison.
\VS{21}Et Marthe dit à Jésus : Seigneur, si tu avais été ici mon frère ne serait pas mort.
\VS{22}Mais maintenant je sais que tout ce que tu demanderas à Dieu, Dieu te le donnera.
\VS{23}Jésus lui dit : Ton frère ressuscitera.
\VS{24}Marthe lui dit : Je sais qu'il ressuscitera à la résurrection, au dernier jour.
\VS{25}Jésus lui dit : Je suis la résurrection et la vie : Celui qui croit en moi vivra même s'il meurt.
\VS{26}Et quiconque vit et croit en moi ne mourra jamais ; crois-tu cela ?
\VS{27}Elle lui dit : Oui, Seigneur, je crois que tu es le Christ, le Fils de Dieu, qui devait venir dans le monde.
\VS{28}Ayant ainsi parlé, elle alla appeler secrètement Marie sa sœur, en lui disant : Le Maître est ici, et il t'appelle.
\VS{29}Et aussitôt qu'elle eut entendu, elle se leva promptement, et alla vers lui.
\VS{30}Or Jésus n'était pas encore entré dans le village, mais il était au lieu où Marthe l'avait rencontré.
\VS{31}Alors les Juifs qui étaient avec Marie à la maison, et qui la consolaient, ayant vu qu'elle s'était levée si promptement, et qu'elle était sortie, la suivirent en disant : Elle va au sépulcre pour y pleurer.
\VS{32}Lorsque Marie fut arrivée où était Jésus, et qu'elle le vit, elle se jeta à ses pieds, en lui disant : Seigneur, si tu avais été ici, mon frère ne serait pas mort.
\VS{33}Jésus, la voyant pleurer, elle et les Juifs qui étaient venus avec elle, frémit en son esprit et fut tout ému.
\VS{34}Et il dit : Où l'avez-vous mis ? Ils lui répondirent : Seigneur, viens et vois.
\VS{35}Et Jésus pleura.
\VS{36}Sur quoi les Juifs dirent : Voyez comme il l'aimait.
\VS{37}Et quelques-uns d'entre eux disaient : Lui qui a ouvert les yeux de l'aveugle, ne pouvait-il pas faire aussi que cet homme ne meure point ?
\VS{38}Alors Jésus frémissant de nouveau en lui-même, se rendit au sépulcre. C'était une grotte, et il y avait une pierre placée devant.
\VS{39}Jésus dit : Ôtez la pierre. Mais Marthe, la sœur du mort, lui dit : Seigneur, il sent déjà, car il est là depuis quatre jours.
\VS{40}Jésus lui dit : Ne t'ai-je pas dit que si tu crois tu verras la gloire de Dieu ?
\VS{41}Ils ôtèrent donc la pierre de dessus le lieu où le mort était couché. Et Jésus levant ses yeux au ciel, dit : Père, je te rends grâces de ce que tu m'as exaucé.
\VS{42}Or je savais bien que tu m'exauces toujours; mais je l'ai dit à cause de la foule qui est autour de moi, afin qu'elle croit que tu m'as envoyé.
\VS{43}Et ayant dit ces choses, il cria à haute voix : Lazare sors dehors !
\VS{44}Alors le mort sortit, ayant les mains et les pieds liés de bandes ; et son visage était enveloppé d'un linge. Jésus leur dit : Déliez-le, et laissez-le aller.
\TextTitle{Conspiration des pharisiens\FTNTT{Jn. 12:10-11.}}
\VS{45}C'est pourquoi plusieurs des Juifs qui étaient venus vers Marie, et qui avaient vu ce que Jésus avait fait, crurent en lui.
\VS{46}Mais quelques-uns d'entre eux allèrent vers les pharisiens et leur dirent les choses que Jésus avait faites.
\VS{47}Alors les principaux sacrificateurs et les pharisiens assemblèrent le sanhédrin, et ils dirent : Que ferons-nous ? Car cet homme fait beaucoup de miracles.
\VS{48}Si nous le laissons faire, tout le monde croira en lui, et les Romains viendront et ils détruiront et ce lieu et notre nation.
\VS{49}Alors l'un d'eux appelé Caïphe, qui était le souverain sacrificateur cette année-là, leur dit : Vous n'y comprenez rien.
\VS{50}Et vous ne réfléchissez pas qu'il est de notre intérêt qu'un homme meure pour le peuple, et que toute la nation ne périsse point.
\VS{51}Or il ne dit pas cela de lui-même ; mais étant souverain sacrificateur de cette année-là, il prophétisa que Jésus devait mourir pour la nation.
\VS{52}Et non pas seulement pour la nation, mais aussi pour rassembler en un seul corps les enfants de Dieu dispersés.
\VS{53}Depuis ce jour-là, ils se concertèrent ensemble pour le faire mourir.
\VS{54}C'est pourquoi Jésus ne se montrait plus ouvertement parmi les Juifs ; mais il se retira dans la contrée voisine du désert, dans une ville appelée Ephraïm ; et il demeura là avec ses disciples.
\VS{55}Or la Pâque des Juifs était proche. Et beaucoup de gens du pays montèrent à Jérusalem avant Pâque, afin de se purifier.
\VS{56}Et ils cherchaient Jésus, et se disaient les uns les autres dans le temple : Que vous en semble ? Ne viendra-t-il pas à la fête ?
\VS{57}Or les principaux sacrificateurs et les pharisiens avaient donné l'ordre que si quelqu'un savait où il était, il le déclare, afin qu'on se saisisse de lui.
\Chap{12}
\TextTitle{Jésus oint par Marie de Béthanie\FTNTT{Mt. 26:6-13 ; Mc. 14:3-9.}}
\VerseOne{}Six jours avant la Pâque, Jésus arriva à Béthanie, où était Lazare qui avait été mort, et qu'il avait ressuscité des morts.
\VS{2}Là, on lui fit un souper ; Marthe servait, et Lazare était un de ceux qui étaient à table avec lui.
\VS{3}Alors Marie, ayant pris une livre de nard pur de grand prix, en oignit les pieds de Jésus, et les essuya avec ses cheveux ; et la maison fut remplie de l'odeur du parfum.
\VS{4}Alors Judas Iscariot, fils de Simon, l'un de ses disciples, celui qui devait le trahir, dit :
\VS{5}Pourquoi ce parfum n'a-t-il pas été vendu trois cents deniers, pour donner cet argent aux pauvres ?
\VS{6}Il dit cela, non parce qu'il se mettait en peine des pauvres, mais parce qu'il était voleur, et que, tenant la bourse, il prenait ce qu'on y mettait.
\VS{7}Mais Jésus lui dit : Laisse-la faire ; elle l'a gardé pour le jour de ma sépulture.
\VS{8}Car vous aurez toujours des pauvres avec vous, mais vous ne m'aurez pas toujours.
\VS{9}Et une grande multitude des Juifs ayant su qu'il était là, y vinrent, non seulement à cause de Jésus, mais aussi pour voir Lazare, qu'il avait ressuscité des morts.
\VS{10}Sur quoi les principaux sacrificateurs résolurent de faire mourir aussi Lazare.
\VS{11}Car plusieurs des Juifs se retiraient d'avec eux à cause de lui, et croyaient en Jésus.
\TextTitle{Entrée de Jésus à Jérusalem, des Grecs cherchent à le voir\FTNTT{Za. 9:9 ; Mt. 21:1-11 ; Mc. 11:1-11 ; Lu. 19:28-40 ; Ap. 19:11-16.}}
\VS{12}Le lendemain, une grande foule qui était venue à la fête ayant entendu dire que Jésus se rendait à Jérusalem,
\VS{13}prit des branches de palmes, et sortit au-devant de lui en criant : Hosanna ! Béni soit le roi d'Israël qui vient au Nom du Seigneur !
\VS{14}Jésus trouva un ânon, s'assit dessus, selon ce qui est écrit :
\VS{15}Ne crains point, fille de Sion ; voici, ton Roi vient, assis sur le petit d'une ânesse\FTNT{Za. 9:9.}.
\VS{16}Ses disciples ne comprirent pas d'abord ces choses ; mais quand Jésus eut été glorifié, ils se souvinrent alors qu'elles étaient écrites de lui, et qu'elles avaient été accomplies à son égard.
\VS{17}Tous ceux qui avaient été avec Jésus, quand il appela Lazare du sépulcre et le ressuscita des morts, lui rendaient témoignage ;
\VS{18}et la foule alla au-devant de lui, parce qu'elle avait appris qu'il avait fait ce miracle.
\VS{19}Sur quoi les pharisiens dirent entre eux : Vous ne voyez pas que vous ne gagnez rien ? Voici, le monde va après lui.
\VS{20}Or il y avait quelques Grecs d'entre ceux qui étaient montés pour adorer Dieu pendant la fête,
\VS{21}lesquels vinrent à Philippe, qui était de Bethsaïda de Galilée, et le prièrent, disant : Seigneur ! Nous désirons voir Jésus.
\VS{22}Philippe vint, et le dit à André, et André et Philippe le dirent à Jésus.
\TextTitle{Jésus annonce sa crucifixion}
\VS{23}Et Jésus leur répondit, disant : L'heure est venue où le Fils de l'homme doit être glorifié.
\VS{24}En vérité, en vérité je vous dis : Si le grain de blé qui est tombé en terre ne meurt, il reste seul ; mais s'il meurt, il porte beaucoup de fruit.
\VS{25}Celui qui aime sa vie la perdra, et celui qui hait sa vie dans ce monde la conservera pour la vie éternelle.
\VS{26}Si quelqu'un me sert, qu'il me suive ; et là où je serai, là aussi sera celui qui me sert ; et si quelqu'un me sert, mon Père l'honorera.
\VS{27}Maintenant mon âme est troublée. Et que dirai-je ? Ô Père, délivre-moi de cette heure ? Mais c'est pour cela que je suis venu jusqu'à cette heure.
\VS{28}Père, glorifie ton Nom ! Alors une voix vint du ciel, disant : Je l'ai glorifié, et je le glorifierai encore.
\VS{29}Et la foule qui était là, et qui avait entendu cette voix, disait que c'était un coup de tonnerre ; les autres disaient : Un ange lui a parlé.
\VS{30}Jésus prit la parole et dit : Cette voix n'est pas venue pour moi, mais pour vous.
\VS{31}Maintenant est venu le jugement de ce monde ; maintenant le prince de ce monde sera jeté dehors.
\VS{32}Et moi, quand je serai élevé de la terre, j'attirerai tous les hommes à moi.
\VS{33}Or il disait cela indiquant de quelle mort il devait mourir.
\VS{34}La foule lui répondit : Nous avons appris par la loi que le Christ demeure éternellement ; et comment donc dis-tu qu'il faut que le Fils de l'homme soit élevé ? Qui est ce Fils de l'homme ?
\VS{35}Alors Jésus leur dit : La lumière est encore avec vous pour un peu de temps : Marchez, pendant que vous avez la lumière, de peur que les ténèbres ne vous surprennent ; car celui qui marche dans les ténèbres ne sait où il va.
\VS{36}Pendant que vous avez la lumière, croyez en la lumière, afin que vous soyez enfants de lumière. Jésus dit ces choses, puis il s'en alla, et se cacha de devant eux.
\VS{37}Malgré tant de miracles qu'il avait faits en leur présence, ils ne croyaient point en lui,
\VS{38}afin que s'accomplisse cette parole qui a été dite par Esaïe le prophète : Seigneur, qui a cru à notre parole ? Et à qui a été révélé le bras du Seigneur\FTNT{Es. 53:1.} ?
\VS{39}C'est pourquoi ils ne pouvaient croire, parce qu'Esaïe a dit encore :
\VS{40}Il a aveuglé leurs yeux ; et il a endurci leur cœur, de peur qu'ils ne voient de leurs yeux, qu'ils ne comprennent du cœur, qu'ils ne se convertissent, et que je ne les guérisse\FTNT{Es. 6:9-10.}.
\VS{41}Esaïe dit ces choses quand il vit sa gloire, et qu'il parla de lui.
\VS{42}Cependant, même parmi les chefs, plusieurs crurent en lui ; mais ils ne le confessaient pas à cause des pharisiens, de peur d'être exclus de la synagogue.
\VS{43}Car ils aimèrent la gloire des hommes, plus que la gloire de Dieu.
\VS{44}Or Jésus s'écria et dit : Celui qui croit en moi, ne croit pas seulement en moi, mais en celui qui m'a envoyé ;
\VS{45}et celui qui me voit, voit celui qui m'a envoyé.
\VS{46}Je suis venu dans le monde pour en être la Lumière, afin que quiconque croit en moi ne demeure point dans les ténèbres.
\VS{47}Et si quelqu'un entend mes paroles et ne les croit point, je ne le juge point ; car je ne suis point venu pour juger le monde, mais pour sauver le monde.
\VS{48}Celui qui me rejette et qui ne reçoit pas mes paroles, a son juge ; la parole que j'ai annoncée sera celle qui le jugera au dernier jour.
\VS{49}Car je n'ai point parlé de moi-même, mais le Père qui m'a envoyé, m'a prescrit ce que je dois dire et annoncer.
\VS{50}Et je sais que son commandement est la vie éternelle. Les choses donc que je dis, je les dis comme mon Père me les a dites.
\Chap{13}
\VerseOne{}Or avant la fête de Pâque, Jésus, sachant que son heure était venue pour passer de ce monde au Père, et ayant aimé les siens qui étaient dans le monde, il les aima jusqu'à la fin.
\TextTitle{Jésus lave les pieds de ses disciples\FTNTT{Mt. 26:20-24 ; Mc. 14:17 ; Lu. 22:14,21-23.}}
\VS{2}Et après le souper, alors que le diable avait déjà mis dans le cœur de Judas Iscariot, fils de Simon, de le trahir,
\VS{3}Jésus sachant que le Père avait remis toutes choses entre ses mains, qu'il était venu de Dieu, et qu'il s'en allait à Dieu,
\VS{4}se leva de table, ôta ses vêtements, et prit un linge, dont il se ceignit.
\VS{5}Puis il mit de l'eau dans un bassin, et se mit à laver les pieds de ses disciples, et à les essuyer avec le linge dont il se ceignit.
\VS{6}Alors il vint à Simon Pierre, mais Pierre lui dit : Toi, Seigneur, tu me laves les pieds ?
\VS{7}Jésus répondit, et lui dit : Tu ne comprends pas maintenant ce que je fais, mais tu le sauras dans la suite.
\VS{8}Pierre lui dit : Tu ne me laveras jamais les pieds. Jésus lui répondit : Si je ne te lave pas, tu n'auras point de part avec moi.
\VS{9}Simon Pierre lui dit : Seigneur, non seulement mes pieds, mais aussi les mains et la tête.
\VS{10}Jésus lui dit : Celui qui est baigné n'a besoin que de se laver les pieds pour être entièrement pur ; or vous êtes purs, mais non pas tous.
\VS{11}Car il savait qui était celui qui le trahirait ; c'est pourquoi il dit : Vous n'êtes pas tous purs.
\VS{12}Après donc qu'il leur eut lavé les pieds, il reprit ses vêtements, et s'étant remis à table, il leur dit : Comprenez-vous ce que je vous ai fait ?
\VS{13}Vous m'appelez Maître et Seigneur ; et vous dites bien, car je le suis.
\VS{14}Si donc moi, qui suis le Seigneur et le Maître, j'ai lavé vos pieds, vous devez aussi vous laver les pieds les uns des autres.
\VS{15}Car je vous ai donné un exemple, afin que vous fassiez comme je vous ai fait.
\VS{16}En vérité, en vérité, je vous dis : Le serviteur n'est pas plus grand que son maître, ni l'apôtre plus grand que celui qui l'a envoyé.
\VS{17}Si vous savez ces choses, vous êtes heureux, pourvu que vous les pratiquiez.
\VS{18}Je ne parle pas de vous tous ; je connais ceux que j'ai choisis. Mais il faut que l'Ecriture s'accomplisse : Celui qui mange le pain avec moi, a levé son talon contre moi\FTNT{Ps. 41:10.}.
\VS{19}Je vous dis ceci dès maintenant, et avant que la chose arrive, afin que lorsqu'elle arrivera, vous croyiez que c'est moi que le Père a envoyé.
\VS{20}En vérité, en vérité, je vous le dis : Celui qui reçoit celui que j'aurai envoyé, me reçoit ; et celui qui me reçoit, reçoit celui qui m'a envoyé.
\TextTitle{Jésus annonce la trahison de Judas et le reniement de Pierre\FTNTT{Mt. 26:21-25 ; Mc. 14:18-21 ; Lu. 22:21-23.}}
\VS{21}Quand Jésus eut dit ces choses, il fut ému dans son esprit, et il déclara, et dit : En vérité, en vérité, je vous dis, que l'un de vous me trahira.
\VS{22}Alors les disciples se regardaient les uns les autres, ne sachant de qui il parlait.
\VS{23}Or un des disciples, celui que Jésus aimait, était à table couché sur le sein de Jésus.
\VS{24}Et Simon Pierre lui fit signe de demander qui était celui dont Jésus parlait.
\VS{25}Et ce disciple, s'étant penché sur la poitrine de Jésus, lui dit : Seigneur, qui est-ce ?
\VS{26}Jésus répondit : C'est celui à qui je donnerai le morceau trempé. Et, ayant trempé le morceau, il le donna à Judas Iscariot, fils de Simon.
\VS{27}Après que Judas eut pris le morceau, Satan entra en lui. Jésus donc lui dit : Ce que tu fais, fais-le promptement.
\VS{28}Mais aucun de ceux qui étaient à table ne comprit pourquoi il lui avait dit cela ;
\VS{29}car quelques-uns pensaient que, comme Judas avait la bourse, Jésus lui disait : Achète ce qui nous est nécessaire pour la fête, ou qu'il lui commandait de donner quelque chose aux pauvres.
\VS{30}Judas, ayant pris le morceau, sortit aussitôt. Or il faisait nuit.
\VS{31}Lorsqu'il fut sorti, Jésus dit : Maintenant le Fils de l'homme est glorifié ; et Dieu est glorifié en lui.
\VS{32}Si Dieu est glorifié en lui, Dieu aussi le glorifiera en lui-même, et il le glorifiera bientôt.
\VS{33}Mes petits enfants, je suis encore pour un peu de temps avec vous. Vous me chercherez ; mais, comme j'ai dit aux Juifs : Vous ne pouvez pas venir là où je vais, je vous le dis aussi maintenant.
\VS{34}Je vous donne un nouveau commandement : Aimez-vous les uns les autres ; comme je vous ai aimés, vous aussi, aimez-vous les uns les autres.
\VS{35}A ceci tous connaîtront que vous êtes mes disciples, si vous avez de l'amour les uns pour les autres.
\VS{36}Simon Pierre lui dit : Seigneur ! Où vas-tu ? Jésus lui répondit : Là où je vais, tu ne peux pas me suivre maintenant, mais tu me suivras plus tard.
\VS{37}Pierre lui dit : Seigneur ! Pourquoi ne puis-je pas te suivre maintenant ? J'exposerai ma vie pour toi.
\VS{38}Jésus lui répondit : Tu exposeras ta vie pour moi ? En vérité, en vérité je te dis, que le coq ne chantera pas, que tu ne m'aies renié trois fois.
\Chap{14}
\TextTitle{Une place dans la demeure du Seigneur}
\VerseOne{}Que votre cœur ne se trouble point. Vous croyez en Dieu, croyez aussi en moi.
\VS{2}Il y a plusieurs demeures dans la maison de mon Père. Si cela n'était pas, je vous l'aurais dit. Je vais vous préparer une place.
\VS{3}Et quand je m'en serai allé, et que je vous aurai préparé une place, je reviendrai, et je vous prendrai avec moi, afin que là où je suis vous y soyez aussi.
\VS{4}Et vous savez où je vais, et vous en savez le chemin.
\VS{5}Thomas lui dit : Seigneur, nous ne savons point où tu vas, comment donc pouvons-nous en savoir le chemin ?
\VS{6}Jésus lui dit : Je suis le chemin, la vérité, et la vie. Nul ne vient au Père que par moi.
\TextTitle{Le Père et le Fils sont un}
\VS{7}Si vous me connaissiez, vous connaîtriez aussi mon Père. Mais dès maintenant vous le connaissez, et vous l'avez vu.
\VS{8}Philippe lui dit : Seigneur, montre-nous le Père, et cela nous suffit.
\VS{9}Jésus lui répondit : Je suis depuis si longtemps avec vous, et tu ne m'as pas connu ? Philippe ! Celui qui m'a vu a vu mon Père ; et comment dis-tu : Montre-nous le Père ?
\VS{10}Ne crois-tu pas que je suis en mon Père, et que le Père est en moi ? Les paroles que je vous dis, je ne les dis pas de moi-même ; mais le Père qui demeure en moi est celui qui fait les œuvres.
\VS{11}Croyez-moi, je suis en mon Père, et le Père est en moi ; sinon croyez-moi à cause de ces œuvres.
\VS{12}En vérité, en vérité, je vous dis : Celui qui croit en moi fera les œuvres que je fais, et il en fera même de plus grandes que celles-ci, parce que je m'en vais vers mon Père.
\TextTitle{Le privilège de la prière}
\VS{13}Et tout ce que vous demanderez en mon Nom, je le ferai, afin que le Père soit glorifié dans le Fils.
\VS{14}Si vous demandez en mon Nom quelque chose, je le ferai.
\TextTitle{Promesse de l'habitation de l'Esprit}
\VS{15}Si vous m'aimez, gardez mes commandements.
\VS{16}Et moi, je prierai le Père, et il vous donnera un autre consolateur, pour demeurer avec vous éternellement,
\VS{17}l'Esprit de vérité que le monde ne peut recevoir, parce qu'il ne le voit point et qu'il ne le connaît point ; mais vous le connaissez, car il demeure avec vous, et il sera en vous.
\VS{18}Je ne vous laisserai pas orphelins, je viendrai vers vous.
\VS{19}Encore un peu de temps, et le monde ne me verra plus ; mais vous me verrez, et parce que je vis, vous aussi vous vivrez.
\VS{20}En ce jour-là, vous connaîtrez que je suis en mon Père, et vous en moi, et moi en vous.
\VS{21}Celui qui a mes commandements et qui les garde, c'est celui qui m'aime ; et celui qui m'aime sera aimé de mon Père et je l'aimerai, et je me ferai connaître à lui.
\VS{22}Jude, non pas Iscariot, lui dit : Seigneur, d'où vient que tu te feras connaître à nous, et non pas au monde ?
\VS{23}Jésus répondit, et lui dit : Si quelqu'un m'aime, il gardera ma parole, et mon Père l'aimera, et nous viendrons à lui, et nous ferons notre demeure chez lui.
\VS{24}Celui qui ne m'aime point ne garde point mes paroles. Et la parole que vous entendez n'est point ma parole, mais c'est celle du Père qui m'a envoyé.
\VS{25}Je vous ai dit ces choses pendant que je demeure avec vous.
\VS{26}Mais le Consolateur, qui est le Saint-Esprit, que le Père enverra en mon Nom, vous enseignera toutes choses, et il vous rappellera tout ce que je vous ai dit.
\TextTitle{Le Messie nous donne sa paix}
\VS{27}Je vous laisse la paix, je vous donne ma paix. Je ne vous la donne pas comme le monde la donne. Que votre cœur ne se trouble point, et ne s'alarme point.
\VS{28}Vous avez entendu que je vous ai dit : Je m'en vais, et je reviens à vous. Si vous m'aimiez, vous seriez certes joyeux de ce que j'ai dit : Je m'en vais au Père, car le Père est plus grand que moi.
\VS{29}Et maintenant je vous l'ai dit avant que cela soit arrivé, afin que quand il sera arrivé, vous croyiez.
\VS{30}Je ne parlerai plus guère avec vous ; car le prince de ce monde vient. Mais il n'a rien en moi.
\VS{31}Mais afin que le monde sache que j'aime le Père, et que je fais ce que le Père m'a commandé, levez-vous, partons d'ici.
\Chap{15}
\TextTitle{Le cep et les sarments}
\VerseOne{}Je suis le vrai Cep\FTNT{Jésus est l'arbre de vie qui produit de bons fruits en nous, à condition que nous nous tenions loin de l'arbre de la connaissance du bien et du mal. Jésus, le vrai Cep, est la source de vie. La viabilité du sarment dépend de son attachement au Cep. Jésus a été pendu au bois (Ac 5:30), s'est chargé de nos malédictions (Ga. 3:13) et a été retranché à notre place.}, et mon Père est le Vigneron.
\VS{2}Il retranche tout sarment qui est en moi et qui ne porte pas de fruit ; et tout sarment qui porte du fruit, il l'émonde, afin qu'il porte encore plus de fruit.
\VS{3}Vous êtes déjà purs, à cause de la parole que je vous ai enseignée.
\VS{4}Demeurez en moi, et je demeurerai en vous. Comme le sarment ne peut de lui-même porter du fruit, s'il ne demeure pas attaché au cep, ainsi vous ne le pouvez pas non plus, si vous ne demeurez pas en moi.
\VS{5}Je suis le cep, et vous en êtes les sarments. Celui qui demeure en moi et en qui je demeure porte beaucoup de fruit, car hors de moi, vous ne pouvez rien produire.
\VS{6}Si quelqu'un ne demeure point en moi, il est jeté dehors, comme le sarment, et il se sèche ; puis on l'amasse, on le met au feu, et il brûle.
\VS{7}Si vous demeurez en moi, et que mes paroles demeurent en vous, demandez tout ce que vous voudrez, et cela vous sera fait.
\VS{8}Si vous portez beaucoup de fruits, mon Père sera glorifié, et vous serez alors mes disciples.
\VS{9}Comme le Père m'a aimé, ainsi je vous ai aimés. Demeurez dans mon amour.
\VS{10}Si vous gardez mes commandements, vous demeurerez dans mon amour, comme j'ai gardé les commandements de mon Père, et je demeure dans son amour.
\VS{11}Je vous ai dit ces choses, afin que ma joie demeure en vous, et que votre joie soit parfaite.
\VS{12}C'est ici mon commandement : Que vous vous aimiez-vous les uns les autres, comme je vous ai aimés.
\VS{13}Personne n'a de plus grand amour que celui qui donne sa vie pour ses amis.
\VS{14}Vous serez mes amis, si vous faites tout ce que je vous commande.
\TextTitle{Une nouvelle intimité}
\VS{15}Je ne vous appelle plus serviteurs, car le serviteur ne sait pas ce que fait son maître; mais je vous ai appelés mes amis, parce que je vous ai fait connaître tout ce que j'ai appris de mon Père.
\VS{16}Ce n'est pas vous qui m'avez choisi ; mais moi, je vous ai choisis, et je vous ai établis afin que vous alliez partout, et que vous produisiez du fruit, et que votre fruit demeure ; afin que tout ce que vous demanderez au Père en mon Nom, il vous le donne.
\VS{17}Ce que je vous commande, c'est de vous aimer les uns les autres.
\TextTitle{Attitude du monde face au croyant}
\VS{18}Si le monde vous hait, sachez qu'il m'a haï avant vous.
\VS{19}Si vous étiez du monde, le monde aimerait ce qui est à lui ; mais parce que vous n'êtes pas du monde, et que je vous ai choisis du milieu du monde, à cause de cela le monde vous hait.
\VS{20}Souvenez-vous de la parole que je vous ai dite : Le serviteur n'est pas plus grand que son maître. S'ils m'ont persécuté, ils vous persécuteront aussi ; s'ils ont gardé ma parole, ils garderont aussi la vôtre.
\VS{21}Mais ils vous feront toutes ces choses à cause de mon Nom, parce qu'ils ne connaissent point celui qui m'a envoyé.
\VS{22}Si je n'étais pas venu, et que je ne leur avais point parlé, ils n'auraient point de péché, mais maintenant ils n'ont point d'excuse de leur péché.
\VS{23}Celui qui me hait, hait aussi mon Père.
\VS{24}Si je n'avais pas fait parmi eux les œuvres qu'aucun autre n'a faites, ils n'auraient point de péché ; mais maintenant ils les ont vues, et ils ont haï et moi et mon Père.
\VS{25}Mais cela est arrivé afin que s'accomplisse la parole qui est écrite dans leur loi : Ils m'ont haï sans cause\FTNT{Ps. 35:19 ; Ps. 69:5.}.
\VS{26}Mais quand le Consolateur sera venu, que je vous enverrai de la part de mon Père, l'Esprit de vérité, qui procède de mon Père, il rendra témoignage de moi ;
\VS{27}et vous aussi, vous rendrez témoignage, car vous êtes dès le commencement avec moi.
\Chap{16}
\TextTitle{Avertissement de Jésus\FTNTT{Mt. 24:9-10 ; Lu. 21:16-19.}}
\VerseOne{}Je vous ai dit ces choses, afin que vous ne soyez pas scandalisés.
\VS{2}Ils vous chasseront des synagogues ; même l'heure vient où quiconque vous fera mourir croira rendre un culte à Dieu.
\VS{3}Et ils vous feront ces choses, parce qu'ils n'ont connu ni le Père ni moi.
\VS{4}Mais je vous ai dit ces choses, afin que, lorsque l'heure sera venue, vous vous souveniez que je vous les ai dites. Et je ne vous en ai pas parlé dès le commencement, parce que j'étais avec vous.
\VS{5}Mais maintenant je m'en vais vers celui qui m'a envoyé, et aucun de vous ne me demande : Où vas-tu ?
\VS{6}Mais, parce que je vous ai dit ces choses, la tristesse a rempli votre cœur.
\TextTitle{L'Esprit convainc}
\VS{7}Toutefois je vous dis la vérité : Il vous est avantageux que je m'en aille, car si je ne m'en vais, le Consolateur ne viendra pas à vous ; mais si je m'en vais, je vous l'enverrai.
\VS{8}Et quand il sera venu, il convaincra le monde de péché, de justice, et de jugement :
\VS{9}de péché, parce qu'ils ne croient point en moi ;
\VS{10}de justice, parce que je m'en vais à mon Père, et que vous ne me verrez plus ;
\VS{11}de jugement, parce que le prince de ce monde est déjà jugé.
\TextTitle{L'Esprit révèlera la verité}
\VS{12} J'ai à vous dire encore plusieurs choses, mais elles sont encore au-dessus de votre portée.
\VS{13}Mais quand celui-là sera venu, l'Esprit de vérité, il vous conduira dans toute la vérité ; car il ne parlera pas de lui-même, mais il dira tout ce qu'il aura entendu, et il vous annoncera les choses à venir.
\VS{14}Il me glorifiera, car il prendra ce qui est à moi, et vous l'annoncera.
\VS{15}Tout ce que mon Père a, est à moi ; c'est pourquoi j'ai dit qu'il prendra ce qui est à moi, et qu'il vous l'annoncera.
\TextTitle{La mort, la résurrection et l'avènement de Jésus}
\VS{16}Encore un peu de temps, et vous ne me verrez plus ; et après un peu de temps, vous me verrez, car je m'en vais à mon Père.
\VS{17}Et quelques-uns de ses disciples dirent entre eux : Qu'est-ce qu'il nous dit : Encore un peu de temps, et vous ne me verrez plus ; et après un peu de temps, vous me verrez, car je m'en vais à mon Père ?
\VS{18}Ils disaient donc : Que signifient ces mots : Encore un peu de temps ? Nous ne comprenons pas ce qu'il dit.
\VS{19}Jésus, sachant qu'ils voulaient l'interroger, leur dit : Vous vous demandez entre vous sur ce que j'ai dit : Encore un peu de temps, et vous ne me verrez plus, et après un peu de temps, vous me verrez.
\VS{20}En vérité, en vérité, je vous dis : Vous pleurerez et vous vous lamenterez, et le monde se réjouira ; vous serez, dis-je, attristés ; mais votre tristesse sera changée en joie.
\VS{21}Quand une femme accouche, elle a des douleurs, parce que son terme est venue ; mais lorsque l'enfant est né, elle ne se souvient plus de la souffrance, à cause de la joie qu'elle a de ce qu'un homme est né dans le monde.
\VS{22}Vous donc aussi, vous êtes maintenant dans la tristesse ; mais je vous reverrai encore, et votre cœur se réjouira, et personne ne vous ôtera votre joie.
\VS{23}Et en ce jour-là, vous ne m'interrogerez plus sur rien. En vérité, en vérité, je vous dis : Tout ce que vous demanderez au Père en mon Nom, il vous le donnera.
\VS{24}Jusqu'à présent vous n'avez rien demandé en mon Nom. Demandez, et vous recevrez, afin que votre joie soit parfaite.
\VS{25}Je vous ai dit ces choses en paraboles. Mais l'heure vient où je ne vous parlerai plus en paraboles, mais je vous parlerai ouvertement de mon Père.
\VS{26}En ce jour-là, vous demanderez des grâces en mon Nom, et je ne vous dis pas que je prierai le Père pour vous ;
\VS{27}car le Père lui-même vous aime, parce que vous m'avez aimé, et que vous avez cru que je suis issu de Dieu.
\VS{28}Je suis issu du Père, et je suis venu dans le monde ; maintenant je quitte le monde, et je m'en vais au Père.
\VS{29}Ses disciples lui dirent : Voici, maintenant tu parles ouvertement, et tu n'uses plus de paraboles.
\VS{30}Maintenant nous savons que tu sais toutes choses, et que tu n'as pas besoin que quelqu'un t'interroge ; à cause de cela nous croyons que tu es issu de Dieu.
\VS{31}Jésus leur répondit : Croyez-vous maintenant ?
\VS{32}Voici, l'heure vient, et elle est déjà venue, où vous serez dispersés chacun de son côté, et vous me laisserez seul ; mais je ne suis pas seul, car le Père est avec moi.
\VS{33}Je vous ai dit ces choses, afin que vous ayez la paix en moi. Vous aurez des tribulations dans le monde ; mais prenez courage, j'ai vaincu le monde.
\Chap{17}
\TextTitle{L'intercession du Messie, le Souverain Sacrificateur}
\VerseOne{}Jésus dit ces choses ; puis levant ses yeux au ciel, il dit : Père, l'heure est venue ! Glorifie ton Fils, afin que ton Fils te glorifie,
\VS{2}selon que tu lui as donné pouvoir sur tous les hommes, afin qu'il donne la vie éternelle à tous ceux que tu lui as donnés.
\VS{3}Or la vie éternelle, c'est qu'ils te connaissent, toi, le seul vrai Dieu, et celui que tu as envoyé, Jésus-Christ.
\VS{4}Je t'ai glorifié sur la terre, j'ai achevé l'œuvre que tu m'avais donnée à faire.
\VS{5}Et maintenant glorifie-moi, toi Père, auprès de toi, de la gloire que j'avais auprès de toi avant que le monde soit.
\VS{6}J'ai fait connaître ton Nom aux hommes que tu m'as donnés du milieu du monde. Ils étaient à toi, et tu me les as donnés ; et ils ont gardé ta parole.
\VS{7}Maintenant ils ont connu que tout ce que tu m'as donné vient de toi.
\VS{8}Car je leur ai donné les paroles que tu m'as données ; et ils les ont reçues, et ils ont vraiment connu que je suis issu de toi, et ils ont cru que tu m'as envoyé.
\VS{9}Je prie pour eux. Je ne prie pas pour le monde, mais pour ceux que tu m'as donnés, parce qu'ils sont à toi ;
\VS{10}et tout ce qui est à moi est à toi, et ce qui est à toi est à moi ; et je suis glorifié en eux.
\VS{11}Et maintenant je ne suis plus dans le monde, mais ceux-ci sont dans le monde. Et moi je vais à toi. Père saint, garde-les en ton Nom ceux que tu m'as donnés, afin qu'ils soient un comme nous sommes un.
\VS{12}Quand j'étais avec eux dans le monde, je les gardais en ton Nom. J'ai gardé ceux que tu m'as donnés, et aucun d'eux ne s'est perdu, sinon le fils de perdition, afin que l'Ecriture soit accomplie.
\VS{13}Et maintenant je vais à toi, et je dis ces choses étant encore dans le monde, afin qu'ils aient ma joie parfaite en eux-mêmes.
\VS{14}Je leur ai donné ta parole ; et le monde les a haïs, parce qu'ils ne sont pas du monde, comme moi je ne suis pas du monde.
\VS{15}Je ne te prie pas de les ôter du monde, mais de les préserver du mal.
\VS{16}Ils ne sont pas du monde, comme aussi je ne suis pas du monde.
\VS{17}Sanctifie-les par ta vérité ; ta parole est la vérité.
\VS{18}Comme tu m'as envoyé dans le monde, ainsi je les ai envoyés dans le monde.
\VS{19}Et je me sanctifie moi-même pour eux, afin qu'eux aussi soient sanctifiés par la vérité.
\VS{20}Or je ne prie pas seulement pour eux, mais aussi pour ceux qui croiront en moi par leur parole,
\VS{21}afin que tous soient un, ainsi que toi, Père, tu es en moi, et moi en toi ; afin qu'eux aussi soient un en nous, et que le monde croie que c'est toi qui m'as envoyé.
\VS{22}Je leur ai donné la gloire que tu m'as donnée, afin qu'ils soient un comme nous sommes un.
\VS{23}Je suis en eux, et toi en moi, afin qu'ils soient parfaitement un, et que le monde connaisse que c'est toi qui m'as envoyé, et que tu les aimes, comme tu m'as aimé.
\VS{24}Père, mon désir est que ceux que tu m'as donnés soient avec moi là où je suis, afin qu'ils contemplent la gloire que tu m'as donnée, parce que tu m'as aimé avant la fondation du monde.
\VS{25}Père juste, le monde ne t'a point connu ; mais moi je t'ai connu, et ceux-ci ont connu que c'est toi qui m'as envoyé.
\VS{26}Et je leur ai fait connaître ton Nom, et je le leur ferai connaître, afin que l'amour dont tu m'as aimé soit en eux, et que je sois en eux.
\Chap{18}
\TextTitle{Gethsémané\FTNTT{Mt. 26:36-46 ; Mc. 14:32-42 ; Lu. 22:39-46.}}
\VerseOne{}Après que Jésus eut dit ces choses, il s'en alla avec ses disciples au-delà du torrent de Cédron, où il y avait un jardin, dans lequel il entra avec ses disciples.
\TextTitle{La trahison et l'arrestation de Jésus\FTNTT{Mt. 26:47-56 ; Mc. 14:43-50 ; Lu. 22:47-54.}}
\VS{2}Or Judas, qui le trahissait, connaissait aussi ce lieu-là, car Jésus s'y était souvent assemblé avec ses disciples.
\VS{3}Judas donc, ayant pris la cohorte, et des huissiers qu'envoyèrent les principaux sacrificateurs et les pharisiens, s'en vint là avec des lanternes, des flambeaux, et des armes.
\VS{4}Et Jésus, sachant tout ce qui devait lui arriver, s'avança et leur dit : Qui cherchez-vous ?
\VS{5}Ils lui répondirent : Jésus de Nazareth. Jésus leur dit : Moi, Je suis\FTNT{« Moi, Je suis » (En grec « ego eimi »), ce qui fait écho au nom sous lequel Dieu s'était révélé à Moïse en Ex. 3:14.}. Et Judas, qui le trahissait, était aussi avec eux.
\VS{6}Or après que Jésus leur eut dit : Moi Je suis, ils reculèrent et tombèrent par terre.
\VS{7}Il leur demanda une seconde fois : Qui cherchez-vous ? Et ils répondirent : Jésus de Nazareth.
\VS{8}Jésus répondit : Je vous ai dit que moi, Je suis ; si donc vous me cherchez, laissez aller ceux-ci.
\VS{9}Il dit cela, afin que s'accomplisse la parole qu'il avait dite : Je n'ai perdu aucun de ceux que tu m'as donnés.
\TextTitle{Pierre frappe Malchus}
\VS{10}Alors Simon Pierre, qui avait une épée, la tira, et frappa un serviteur du souverain sacrificateur, et lui coupa l'oreille droite. Ce serviteur s'appelait Malchus.
\VS{11}Mais Jésus dit à Pierre : Remets ton épée dans le fourreau. Ne boirai-je pas la coupe que le Père m'a donnée ?
\TextTitle{Jésus devant le souverain sacrificateur\FTNTT{Mt. 26:57-68 ; Mc. 14:53-65 ; Lu. 22:54.}}
\VS{12}La cohorte, le tribun, et les huissiers des Juifs, se saisirent alors de Jésus, et le lièrent.
\VS{13}Et ils l'emmenèrent premièrement chez Anne ; car il était le beau-père de Caïphe, qui était le souverain sacrificateur de cette année-là.
\VS{14}Or Caïphe était celui qui avait donné ce conseil aux Juifs : Qu'il était avantageux qu'un seul homme meure pour le peuple.
\TextTitle{Le triple reniement de Pierre\FTNTT{Mt. 26:69-75 ; Mc. 14:66-72 ; Lu. 22:54-62.}}
\VS{15}Or Simon Pierre, avec un autre disciple, suivait Jésus. Et ce disciple était connu du souverain sacrificateur, et il entra avec Jésus dans la cour du souverain sacrificateur ;
\VS{16}mais Pierre était dehors à la porte. Et l'autre disciple, qui était connu du souverain sacrificateur, sortit dehors et parla à la portière, et fit entrer Pierre.
\VS{17}Et la servante, qui était la portière, dit à Pierre : N'es-tu pas aussi des disciples de cet homme ? Il dit : Je n'en suis point.
\VS{18}Or les serviteurs et les huissiers, qui étaient là, avaient allumé un feu, parce qu'il faisait froid, et ils se chauffaient. Mais Pierre aussi était avec eux, et se chauffait.
\VS{19}Et le souverain sacrificateur interrogea Jésus sur ses disciples et sur sa doctrine.
\VS{20}Jésus lui répondit : J'ai ouvertement parlé au monde ; j'ai toujours enseigné dans la synagogue et dans le temple, où les Juifs s'assemblent toujours, et je n'ai rien dit en secret.
\VS{21}Pourquoi m'interroges-tu ? Interroge ceux qui ont entendu ce que je leur ai dit ; voici, ils savent ce que j'ai dit.
\VS{22}Quand il eut dit ces choses, un des huissiers, qui se tenait là, donna un coup de sa verge à Jésus, en lui disant : Est-ce ainsi que tu réponds au souverain sacrificateur ?
\VS{23}Jésus lui répondit : Si j'ai mal parlé, explique-moi ce que j'ai dit de mal ; et si j'ai bien parlé, pourquoi me frappes-tu ?
\VS{24}Or Anne l'envoya lié à Caïphe, le souverain sacrificateur.
\VS{25}Et Simon Pierre était là, et se chauffait. Alors on lui dit : N'es-tu pas aussi de ses disciples ? Il le nia, et dit : Je n'en suis point.
\VS{26}Et un des serviteurs du souverain sacrificateur, parent de celui à qui Pierre avait coupé l'oreille, dit : Ne t'ai-je pas vu dans le jardin avec lui ?
\VS{27}Mais Pierre le nia de nouveau. Et aussitôt le coq chanta.
\TextTitle{Jésus devant Pilate\FTNTT{Mt. 27:2,11-14 ; Mc. 15:1-5 ; Lu. 23:1-7,13-15.}}
\VS{28}Puis ils menèrent Jésus de chez Caïphe au prétoire\FTNT{Le prétoire était à l'origine le nom du quartier général de la légion romaine. Il s'agissait plus particulièrement de la tente du général en chef d'une armée.} ; et c'était le matin. Mais ils n'entrèrent point eux-mêmes dans le prétoire, afin de ne pas se souiller, et de pouvoir manger l'agneau de Pâque.
\VS{29}C'est pourquoi Pilate\FTNT{Ponce Pilate était le préfet procurateur de la province romaine de Judée au Ier siècle (de 26 à 36).} sortit vers eux, et leur dit : Quelle accusation portez-vous contre cet homme ?
\VS{30}Ils lui répondirent, et lui dirent : Si ce n'était pas un malfaiteur, nous ne te l'aurions pas livré.
\VS{31}Alors Pilate leur dit : Prenez-le vous-mêmes, et jugez-le selon votre loi. Mais les Juifs lui dirent : Il ne nous est pas permis de mettre quelqu'un à mort.
\VS{32}Et cela arriva ainsi afin que s'accomplisse la parole que Jésus avait dite, lorsqu'il indiquait de quelle mort il devait mourir.
\VS{33}Pilate entra de nouveau dans le prétoire, et ayant appelé Jésus, il lui dit : Es-tu le roi des Juifs ?
\VS{34}Jésus lui répondit : Est-ce de toi-même que tu dis cela, ou d'autres te l'ont dit de moi ?
\VS{35}Pilate répondit : Suis-je Juif ? Ta nation et les principaux sacrificateurs t'ont livré à moi ; qu'as-tu fait ?
\VS{36}Jésus répondit : Mon Royaume n'est pas de ce monde. Si mon Royaume était de ce monde, mes serviteurs auraient combattu pour moi afin que je ne sois pas livré aux Juifs ; mais maintenant mon Royaume n'est point d'ici-bas.
\VS{37}Alors Pilate lui dit : Es-tu donc roi ? Jésus répondit : Tu le dis, que je suis Roi ; je suis né pour cela et c'est pour cela que je suis venu dans le monde, pour rendre témoignage à la vérité. Quiconque est de la vérité entend ma voix.
\VS{38}Pilate lui dit : Qu'est-ce que la vérité ? Et quand il eut dit cela, il sortit de nouveau vers les Juifs, et il leur dit : Je ne trouve aucun crime en lui.
\TextTitle{Barabbas libéré et Jésus condamné\FTNTT{Mt. 27:15-21 ; Mc. 15:6-15 ; Lu. 23:18-25.}}
\VS{39}Or comme c'est parmi vous une coutume que je vous relâche un prisonnier à la fête de Pâque ; voulez-vous donc que je vous relâche le roi des Juifs ?
\VS{40}Et tous s'écrièrent encore, disant : Non pas celui-ci, mais Barrabas. Or Barrabas était un brigand.
\Chap{19}
\TextTitle{Jésus couronné d'épines\FTNTT{Mt. 27:-30 ; Mc. 15:16-18.}}
\VerseOne{}Pilate fit donc prendre Jésus, et le fit battre de verges.
\VS{2}Les soldats tressèrent une couronne d'épines qu'ils posèrent sur sa tête, et le vêtirent d'un vêtement de pourpre ;
\VS{3}puis, ils lui disaient : Roi des Juifs ! Nous te saluons. Et ils lui donnaient des coups avec leurs verges.
\TextTitle{Pilate tente de relâcher Jésus, en vain\FTNTT{Mt. 27:22-26 ; Mc. 15:12-15 ; Lu. 23:20-25.}}
\VS{4}Pilate sortit de nouveau dehors, et leur dit : Voici, je vous l'amène dehors, afin que vous sachiez que je ne trouve aucun crime en lui.
\VS{5}Jésus donc sortit, portant la couronne d'épines et le vêtement de pourpre. Et Pilate leur dit : Voici l'homme.
\VS{6}Mais quand les principaux sacrificateurs et leurs huissiers le virent, ils s'écrièrent, en disant : Crucifie ! Crucifie ! Pilate leur dit : Prenez-le vous-mêmes, et crucifiez-le ; car je ne trouve point de crime en lui.
\VS{7}Les Juifs lui répondirent : Nous avons une loi ; et selon notre loi il doit mourir, car il s'est fait Fils de Dieu.
\VS{8}Or quand Pilate entendit cette parole, il craignit encore davantage.
\VS{9}Et il rentra dans le prétoire, et dit à Jésus : D'où es-tu ? Mais Jésus ne lui donna point de réponse.
\VS{10}Et Pilate lui dit : Est-ce à moi que tu ne parles pas ? Ne sais-tu pas que j'ai le pouvoir de te crucifier, et que j'ai le pouvoir de te délivrer ?
\VS{11}Jésus lui répondit : Tu n'aurais aucun pouvoir sur moi, s'il ne t'avait été donné d'en haut. C'est pourquoi celui qui m'a livré à toi commet un plus grand péché.
\VS{12}Dès ce moment, Pilate cherchait à le délivrer. Mais les Juifs criaient en disant : Si tu le délivres, tu n'es pas ami de César. Car quiconque se fait roi s'oppose à César.
\VS{13}Pilate, ayant entendu ces paroles, amena Jésus dehors ; et il siégea au tribunal, au lieu appelé le Pavé, et en hébreu Gabbatha.
\VS{14}Or c'était la préparation de la Pâque, et il était environ la sixième heure. Et Pilate dit aux Juifs : Voilà votre roi.
\VS{15}Mais ils criaient : Ôte, ôte, crucifie-le ! Pilate leur dit : Crucifierai-je votre roi ? Les principaux sacrificateurs répondirent : Nous n'avons pas d'autre roi que César.
\TextTitle{Jésus crucifié\FTNTT{Mt. 27:31-50 ; Mc. 15:20-37 ; Lu. 23:26-46.}}
\VS{16}Alors il le leur livra pour être crucifié. Ils prirent donc Jésus et l'emmenèrent.
\VS{17}Et Jésus, portant sa croix, arriva au lieu appelé le Crâne, qui se dit en hébreu Golgotha,
\VS{18}où ils le crucifièrent, et deux autres avec lui, un de chaque côté, et Jésus au milieu.
\VS{19}Pilate fit un écriteau, qu'il mit sur la croix, où étaient écrits ces mots : Jésus de Nazareth, le roi des Juifs.
\VS{20}Beaucoup des Juifs lurent cet écriteau, parce que le lieu où Jésus était crucifié, était près de la ville ; et cet écriteau était en hébreu, en grec et en latin.
\VS{21}C'est pourquoi les principaux sacrificateurs des Juifs dirent à Pilate : N'écris pas : Le roi des Juifs, mais que celui-ci a dit : Je suis le roi des Juifs.
\VS{22}Pilate répondit : Ce que j'ai écrit, je l'ai écrit.
\VS{23}Les soldats, après avoir crucifié Jésus, prirent ses vêtements, et ils en firent quatre parts, une part pour chaque soldat. Ils prirent aussi sa tunique, qui était sans couture, d'un seul tissu depuis le haut jusqu'en bas.
\VS{24}Ils se dirent entre eux : Ne la déchirons pas mais tirons au sort, pour savoir à qui elle sera. Et cela arriva ainsi, afin que s'accomplisse cette parole de l'Ecriture : Ils ont partagé entre eux mes vêtements, et ils ont tiré au sort ma tunique\FTNT{Ps. 22:19.}. Ainsi firent les soldats.
\VS{25}Or près de la croix de Jésus se tenaient sa mère, et la sœur de sa mère, à savoir Marie femme de Cléopas, et Marie de Magdala.
\VS{26}Et Jésus, voyant sa mère, et auprès d'elle le disciple qu'il aimait, il dit à sa mère : Femme, voilà ton Fils.
\VS{27}Puis il dit au disciple : Voilà ta mère. Et dès ce moment, ce disciple la prit chez lui.
\VS{28}Après cela, Jésus, sachant que toutes choses étaient déjà accomplies, il dit, afin que l'Ecriture soit accomplie : J'ai soif.
\VS{29}Et il y avait là un vase plein de vinaigre. Les soldats en remplirent une éponge et la mirent au bout d'une branche d'hysope, et la lui présentèrent à la bouche.
\VS{30}Et quand Jésus eut pris le vinaigre, il dit : Tout est accompli\FTNT{La fin de la période de la Première Alliance n'a pas eu lieu à la naissance du Seigneur. En effet, Ga. 4:4 nous dit que Jésus est né sous la loi de Moïse et le récit des quatre évangiles atteste que depuis sa naissance jusqu'à sa mort, Jésus a scrupuleusement respecté et accompli toute la loi. En effet, il a lui-même dit : « Ne croyez pas que je sois venu abolir la loi ou les prophètes ; je ne suis pas venu les abolir, mais les accomplir. » (Mt. 5:17). Ainsi, durant son ministère terrestre, le Seigneur demandait à ce qu'on applique la loi (Mt. 8:4 ; Mt. 23:23 ; Lu. 17:11-14) tout en préparant ses disciples à la nouvelle alliance. L'évangile de Matthieu nous relate un événement capital qui a eu lieu juste après la mort du Seigneur : « Alors Jésus, poussa de nouveau un grand cri, et rendit l'esprit. Et voici, le voile du temple se déchira en deux, depuis le haut jusqu'en bas ; et la terre trembla, et les pierres se fendirent. » (Mt. 27:50-51). Il convient de rappeler que le temple était divisé en trois parties : le parvis, le lieu saint et le saint des saints. Le parvis était accessible à tout le monde, y compris aux non-Juifs. Le lieu saint n'était accessible qu'aux Lévites. La troisième partie, le saint des saints, n'était accessible qu'au souverain sacrificateur. Le lieu saint était séparé du saint des saints par un voile qui symbolisait le mur d'inimitié (Es. 59:2 ; Ro. 3:23) qui sépare l'homme pécheur de la présence de Dieu, représentée dans le temple par l'arche de l'alliance. Ce voile n'avait rien d'un tissu léger et vaporeux, mais il ressemblait davantage à un épais tapis, opaque et surtout très résistant, et donc très difficile à déchirer. Le souverain sacrificateur rentrait seulement une fois par an dans le saint des saints pour y offrir le sacrifice d'expiation pour le peuple ainsi que pour lui-même (Lé. 16 ; Hé. 9:7). Toutefois, la nécessité de répéter ce sacrifice chaque année prouvait que les exigences de la justice divine n'étaient pas pleinement satisfaites (Hé. 10:3-4). L'auteur de l'épître aux Hébreux nous apprend que le voile symbolisait également le corps physique du Messie (Hé. 10:19-20). Ainsi, lorsque le Seigneur a succombé à ses meurtrissures, le fameux voile s'est déchiré du haut jusqu'au bas. Or tant que le voile subsistait, l'accès à la présence de Dieu était fermé (Hé. 9:8). La déchirure atteste donc qu'en Christ, nous pouvons désormais nous approcher avec assurance du trône de Dieu, sans autre médiateur que le Seigneur lui-même (1 Ti. 2:5). « Or là où les péchés sont pardonnés, il n'y a plus d'offrande pour le péché. Ainsi donc, mes frères, nous avons la liberté d'entrer dans le saint des saints au moyen du sang de Jésus, qui est le chemin nouveau et vivant qu'il nous a frayé au travers du voile, c'est-à-dire de sa chair. Et ayant un Souverain Sacrificateur établi sur la maison de Dieu, approchons-nous de lui avec un cœur sincère et une foi inébranlable, ayant les cœurs purifiés d'une mauvaise conscience, et le corps lavé d'une eau pure. Retenons sans fléchir la profession de notre espérance, car celui qui nous a fait la promesse est fidèle. » Hé. 10:18-23. Jésus-Christ est notre Pâque (1 Co. 5:5-8), il est le sacrifice parfait qui a expié nos péchés une fois pour toutes (Hé. 10:10). Par conséquent, il est celui à qui nous devons nous adresser pour recevoir pardon, miséricorde et compassion. « Tout est accompli ». En s'écriant de la sorte, Jésus-Christ a proclamé la fin de l'Ancienne Alliance. En effet, la loi a été promulguée par Moïse, mais la grâce et la vérité sont venues par Jésus-Christ (Jn. 1:17). Toutefois, la Nouvelle Alliance n'a réellement débuté qu'à la Pentecôte avec l'effusion du Saint-Esprit. Voir commentaire en Actes 2.}. Et ayant baissé la tête, il rendit l'esprit.
\TextTitle{Fin de la Première Alliance\FTNTT{Mt. 27:50-51 ; Mc. 15:37-38 ; Lu. 23:45-46.}}
\VS{31}Alors les Juifs, afin que les corps ne demeurent pas sur la croix au jour du sabbat, parce que c'était la préparation - or c'était un grand jour de sabbat - prièrent Pilate qu'on leur rompe les jambes, et qu'on les enlève.
\VS{32}Les soldats vinrent donc, et ils rompirent les jambes au premier, et de même à l'autre qui était crucifié avec lui.
\VS{33}Puis étant venus à Jésus, et voyant qu'il était déjà mort, ils ne lui rompirent point les jambes ;
\VS{34}mais un des soldats lui perça le côté avec une lance, et aussitôt il sortit du sang et de l'eau.
\VS{35}Et celui qui l'a vu en a rendu témoignage, et son témoignage est digne de foi ; et il sait qu'il dit vrai, afin que vous le croyiez.
\VS{36}Ces choses-là sont arrivées, afin que l'Ecriture soit accomplie : Aucun de ses os ne sera brisé\FTNT{Ex. 12:46 ; No 9:12 ; Ps. 34:21.}.
\VS{37}Et encore une autre Ecriture, qui dit : Ils verront celui qu'ils ont percé\FTNT{Za. 12:10.}.
\TextTitle{Jésus enseveli\FTNTT{Mt. 27:57-66 ; Mc. 15:42-47 ; Lu. 23:50-56.}}
\VS{38}Or après ces choses, Joseph d'Arimathée, qui était disciple de Jésus, mais en secret parce qu'il craignait les Juifs, demanda à Pilate la permission d'enlever le corps de Jésus. Et Pilate le lui ayant permis, il vint et prit le corps de Jésus.
\VS{39}Nicodème, qui auparavant était allé de nuit vers Jésus, vint aussi, apportant un mélange de myrrhe et d'aloès d'environ cent livres.
\VS{40}Et ils prirent le corps de Jésus, et l'enveloppèrent de linges, avec des aromates, comme les Juifs ont coutume d'ensevelir.
\VS{41}Or il y avait un jardin dans le lieu où Jésus fut crucifié, et dans le jardin un sépulcre neuf, où personne n'avait encore été mis.
\VS{42}Ce fut là qu'ils déposèrent Jésus, à cause de la préparation des Juifs, parce que le sépulcre était près.
\Chap{20}
\TextTitle{Enchaînement des évènements le jour de la résurrection\FTNTT{Mt. 28:1-15 ; Mc. 16:1-14 ; Lu. 24:1-32.}}
\VerseOne{}Le premier jour de la semaine, Marie de Magdala se rendit dès le matin au sépulcre, comme il faisait encore obscur ; et elle vit que la pierre était ôtée du sépulcre.
\VS{2}Elle courut vers Simon Pierre et vers l'autre disciple que Jésus aimait, et elle leur dit : On a enlevé le Seigneur hors du sépulcre, et nous ne savons pas où on l'a mis.
\VS{3}Alors Pierre partit avec l'autre disciple, et ils s'en allèrent au sépulcre.
\VS{4}Ils couraient tous deux ensemble. Mais l'autre disciple courait plus vite que Pierre, et il arriva le premier au sépulcre ;
\VS{5}et s'étant baissé, il vit les linges à terre, mais il n'y entra point.
\VS{6}Alors Simon Pierre, qui le suivait, arriva, et entra dans le sépulcre ; et vit les linges à terre,
\VS{7}et le linge qu'on avait mis sur la tête de Jésus, non pas avec les linges, mais plié dans un lieu à part.
\VS{8}Alors l'autre disciple, qui était arrivé le premier au sépulcre, y entra aussi, et il vit, et crut.
\VS{9}Car ils ne comprenaient pas encore que, selon l'Ecriture, Jésus devait ressusciter des morts.
\VS{10}Et les disciples s'en retournèrent chez eux.
\TextTitle{Jésus apparaît à Marie de Magdala et aux disciples\FTNTT{Mc. 16:14 ; Lu. 24:13-49.}}
\VS{11}Mais Marie se tenait près du sépulcre dehors, et pleurait. Et comme elle pleurait, elle se baissa dans le sépulcre,
\VS{12}et elle vit deux anges vêtus de blanc, assis à la place où avait été couché le corps de Jésus, l'un à la tête et l'autre aux pieds.
\VS{13}Et ils lui dirent : Femme, pourquoi pleures-tu ? Elle leur dit : Parce qu'on a enlevé mon Seigneur, et je ne sais point où on l'a mis.
\VS{14}En disant cela, elle se retourna, et elle vit Jésus qui était là ; mais elle ne savait pas que c'était Jésus.
\VS{15}Jésus lui dit : Femme, pourquoi pleures-tu ? Qui cherches-tu ? Elle, pensant que c'était le jardinier, lui dit : Seigneur, si c'est toi qui l'as emporté, dis-moi où tu l'as mis, et je le prendrai.
\VS{16}Jésus lui dit : Marie ! Et elle se retourna et lui dit : Rabbouni ! C'est-à-dire, mon Maître !
\VS{17}Jésus lui dit : Ne me touche pas ; car je ne suis point encore monté vers mon Père. Mais va vers mes frères, et dis-leur que je monte vers mon Père et votre Père, vers mon Dieu et votre Dieu.
\VS{18}Marie de Magdala alla annoncer aux disciples qu'elle avait vu le Seigneur, et qu'il lui avait dit ces choses.
\VS{19}Le soir de ce jour, qui était le premier de la semaine, les portes du lieu où les disciples étaient assemblés, à cause de la crainte qu'ils avaient des Juifs, étaient fermées. Jésus vint, se présenta au milieu d'eux, et il leur dit : Que la paix soit avec vous !
\VS{20}Et quand il leur eut dit cela, il leur montra ses mains et son côté. Alors les disciples furent dans la joie en voyant le Seigneur.
\VS{21}Jésus leur dit de nouveau : Que la paix soit avec vous ! Comme mon Père m'a envoyé, ainsi je vous envoie.
\VS{22}Après ces paroles, il souffla sur eux, et leur dit : Recevez le Saint-Esprit.
\VS{23}Ceux à qui vous pardonnerez les péchés, ils leur seront pardonnés ; et ceux à qui vous les retiendrez, ils leur seront retenus.
\VS{24}Or Thomas, appelé Didyme, l'un des douze, n'était pas avec eux quand Jésus vint.
\VS{25}Les autres disciples lui dirent : Nous avons vu le Seigneur. Mais il leur dit : Si je ne vois pas les marques des clous dans ses mains, et si je ne mets pas mon doigt où étaient les clous, et si je ne mets pas ma main dans son côté, je ne le croirai point.
\VS{26}Huit jours après, les disciples étaient de nouveau dans la maison, et Thomas était avec eux. Jésus vint, les portes étant fermées, se présenta au milieu d'eux, et il leur dit : Que la paix soit avec vous !
\VS{27}Puis il dit à Thomas : Mets ton doigt ici, et regarde mes mains, avance aussi ta main, et mets-la dans mon côté ; et ne sois point incrédule, mais fidèle.
\VS{28}Et Thomas répondit et lui dit : Mon Seigneur, et mon Dieu !
\VS{29}Jésus lui dit : Parce que tu m'as vu, Thomas, tu as cru. Heureux sont ceux qui n'ont pas vu et qui ont cru.
\TextTitle{But de l'Evangile selon Jean}
\VS{30}Jésus fit encore, en présence de ses disciples, beaucoup d'autres miracles qui ne sont pas écrits dans ce livre.
\VS{31}Mais ces choses sont écrites afin que vous croyiez que Jésus est le Christ, le Fils de Dieu, et qu'en croyant vous ayez la vie par son Nom.
\Chap{21}
\TextTitle{Apparition de Jésus à quelques apôtres et disciples}
\VerseOne{}Après cela, Jésus se fit voir encore à ses disciples, près de la mer de Tibériade. Et il s'y fit voir de cette manière.
\VS{2}Simon Pierre, Thomas, appelé Didyme, Nathanaël, de Cana en Galilée, les fils de Zébédée, et deux autres de ses disciples étaient ensemble.
\VS{3}Simon Pierre leur dit : Je vais pêcher. Ils lui dirent : Nous allons aussi avec toi. Ils partirent donc et montèrent aussitôt dans une barque, mais ils ne prirent rien cette nuit-là.
\VS{4}Le matin étant venu, Jésus se trouva sur le rivage ; mais les disciples ne savaient pas que c'était Jésus.
\VS{5}Jésus leur dit : Mes enfants, avez-vous quelque petit poisson à manger ? Ils lui répondirent : Non.
\TextTitle{Le service sous les directives du Messie}
\VS{6}Et il leur dit : Jetez le filet du côté droit de la barque, et vous en trouverez. Ils le jetèrent donc, et ils ne pouvaient plus le retirer à cause de la grande quantité de poissons.
\VS{7}C'est pourquoi le disciple que Jésus aimait, dit à Pierre : C'est le Seigneur. Et quand Simon Pierre eut entendu que c'était le Seigneur, il ceignit sa tunique, parce qu'il était nu, et se jeta dans la mer.
\VS{8}Et les autres disciples vinrent dans la barque, car ils n'étaient pas loin de terre, mais seulement à environ deux cents coudées, traînant le filet de poissons.
\VS{9}Lorsqu'ils furent descendus à terre, ils virent de la braise, et du poisson dessus, et du pain.
\VS{10}Jésus leur dit : Apportez des poissons que vous venez maintenant de prendre.
\VS{11}Simon Pierre monta, et tira le filet à terre plein de cent cinquante-trois grands poissons ; et quoiqu'il y en eût tant, le filet ne se rompit point.
\TextTitle{Les ressources du Messie pour ses serviteurs\FTNTT{Lu. 22:35 ; Ph. 4:19.}}
\VS{12}Jésus leur dit : Venez et dînez. Et aucun de ses disciples n'osait lui demander : Qui es-tu ? Sachant que c'était le Seigneur.
\VS{13}Jésus donc vint et prit du pain, et leur en donna ; il fit de même du poisson aussi.
\VS{14}C'était déjà la troisième fois que Jésus se fit voir à ses disciples après être ressuscité des morts.
\TextTitle{La charité, seule motivation pour le service\FTNTT{1 Co. 13 ; 2 Co. 5:14 ; Ap. 2:4-5.}}
\VS{15}Et après qu'ils eurent dîné, Jésus dit à Simon Pierre : Simon, fils de Jonas, m'aimes-tu plus que ne m'aiment ceux-ci ? Il lui répondit : Oui, Seigneur, tu sais que je t'aime. Il lui dit : Pais mes agneaux.
\VS{16}Il lui dit encore : Simon, fils de Jonas, m'aimes-tu ? Il lui répondit : Oui, Seigneur, tu sais que je t'aime. Il lui dit : Pais mes brebis.
\VS{17}Il lui dit pour la troisième fois : Simon, fils de Jonas, m'aimes-tu ? Pierre fut attristé de ce qu'il lui avait dit pour la troisième fois : M'aimes-tu ? Et il lui répondit : Seigneur, tu sais toutes choses, tu sais que je t'aime. Jésus lui dit : Pais mes brebis.
\TextTitle{Le Maître révèle à Pierre sa destinée}
\VS{18}En vérité, en vérité je te dis : Quand tu étais plus jeune, tu te ceignais toi-même, et tu allais où tu voulais ; mais quand tu seras vieux, tu étendras tes mains, et un autre te ceindra, et te mènera où tu ne voudras pas.
\VS{19}Or il dit cela pour indiquer par quelle mort Pierre glorifierait Dieu. Et ayant ainsi parlé, il lui dit : Suis-moi.
\TextTitle{Tous ses serviteurs ne mourront pas\FTNTT{1 Co. 15:51-52 ; 1 Th. 4:14-18.}}
\VS{20}Et Pierre, se retournant, vit venir après eux le disciple que Jésus aimait, celui qui pendant le souper, s'était penché sur la poitrine de Jésus et avait dit : Seigneur, qui est celui qui te trahit ?
\VS{21}Quand donc Pierre le vit, il dit à Jésus : Seigneur, et celui-ci, que lui arrivera-t-il ?
\VS{22}Jésus lui dit : Si je veux qu'il demeure jusqu'à ce que je vienne, que t'importe ? Toi, suis-moi.
\VS{23}Là-dessus, le bruit courut parmi les frères que ce disciple ne mourrait point. Cependant Jésus ne lui avait pas dit : Il ne mourra point ; mais : Si je veux qu'il demeure jusqu'à ce que je vienne, que t'importe ?
\VS{24}C'est ce disciple qui rend témoignage de ces choses, et qui les a écrites. Et nous savons que son témoignage est digne de foi.
\VS{25}Jésus a fait beaucoup d'autres choses ; si on les écrivait en détail, je ne pense pas que le monde même pourrait contenir les livres qu'on écrirait. Amen !
\PPE{}
\end{multicols}
