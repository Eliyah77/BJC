\ShortTitle{Lé.}\BookTitle{Lévitique}\BFont
\noindent\hrulefill
{\footnotesize
\textit{
\bigskip
{\centering{}
\\Auteur~: Probablement Moïse
\\(Heb.~: Vayiqra)
\\Signification~: Et Il (Yahweh) appela
\\Thème~: La sainteté
\\Date de rédaction~: Env. 1450-1410 av. J.-C.\\}
}
\textit{
\\Après avoir construit et dressé le tabernacle selon le modèle que Yahweh avait donné à Moïse, les fils d'Israël reçurent le détail des prescriptions relatives aux offrandes, aux sacrifices et aux fêtes en l'honneur de Yahweh. 
\\Ce livre, dont le nom tire son origine de Lévi, explique la manière dont Aaron et ses fils devaient exercer la prêtrise et amener le peuple à s'approcher de Dieu dans le respect de ses ordonnances.
\\Les lois que Moïse avait recueillies présentent la voie du pardon, laquelle est impossible sans effusion de sang. Bien que les mêmes sacrifices furent réitérés tous les ans, ces préceptes mettaient en évidence l'impuissance de l'homme à atteindre la justice de Dieu par ses propres moyens.\bigskip
}
}
\par\nobreak\noindent\hrulefill
\begin{multicols}{2}
\Chap{1}
\TextTitle{L'holocauste\FTNTT{voir Lé. 6:1-6.}}
\VerseOne{}Et Yahweh appela Moïse, et lui parla de la tente d'assignation, en disant~:
\VS{2}Parle aux enfants d'Israël, et dis-leur~: Quand quelqu'un d'entre vous offrira à Yahweh une offrande d'une bête à quatre pattes, il fera son offrande de gros ou de menu bétail.
\VS{3}Si son offrande pour un holocauste est de gros bétail, il offrira un mâle sans défaut\FTNT{L'holocauste était le sacrifice pour l'expiation par excellence. Contrairement aux autres sacrifices, l'holocauste était entièrement consumé sur l'autel. Il symbolisait d'une part le sacrifice parfait de Christ et d'autre part notre vie volontairement offerte à Dieu (Ro. 12:1). Les animaux aptes à être offerts en holocauste devaient être des mâles sans défaut~:\\- Le veau (Lé. 1:5), image de Christ, l'humble serviteur, soumis et obéissant (Mt. 20:28~; Ph. 2:5-8).\\- L'agneau ou le chevreau, image de Christ qui livre sa vie à la croix sans résistance ni contestation, et qui prend sur lui nos péchés (Es. 53:7~; Mt. 26:63~; Ac. 8:32).\\- Les tourterelles ou les jeunes pigeons, image de la simplicité de Christ (Mt. 10:16).\\Toutes les étapes de la réalisation de ce sacrifice enseignent le disciple sur la mort à soi-même et le dépouillement des œuvres de la chair (Ga. 5:19-21).\\Le sang de l'animal égorgé devait être répandu sur l'autel (Lé. 1:5), image de la croix. L'âme (contenue dans le sang selon Lé. 17:14), liée à la chair et à ses désirs, doit être crucifiée (Ga. 2:20~; Ga. 5:24). L'objet de la mise à mort était certainement un couteau tranchant comme une épée, image de la Parole de Dieu (Hé. 4:12). La mise en pratique de la Parole nous amène nécessairement à nous séparer du monde et à renoncer à soi-même.}~; il l'offrira de son bon gré à l'entrée de la tente d'assignation~; devant Yahweh\FTNT{Ex. 29:10-11.}.
\VS{4}Et il posera sa main sur la tête de l'holocauste, et il sera agréé pour lui, afin de faire la propitiation pour lui.
\VS{5}Puis, on égorgera le jeune taureau devant Yahweh~; et les fils d'Aaron, les prêtres, en offriront le sang et ils répandront le sang sur l'autel tout autour, qui est à l'entrée de la tente d'assignation.
\VS{6}Ensuite on égorgera l'holocauste et on le coupera en morceaux.
\VS{7}Les fils du prêtre Aaron mettront le feu sur l'autel, et disposeront le bois sur le feu.
\VS{8}Et les fils d'Aaron, les prêtres, poseront les morceaux, la tête et la graisse sur le bois qui sera au feu sur l'autel.
\VS{9}Mais il lavera avec de l'eau les entrailles et les jambes~; et le prêtre brûlera toutes ces choses sur l'autel. C'est un holocauste, un sacrifice consumé par le feu, d'une bonne odeur à Yahweh.
\VS{10}Si son offrande est un holocauste de menu bétail, d'entre les agneaux ou d'entre les chèvres, il offrira un mâle sans défaut.
\VS{11}Et on l'égorgera à côté de l'autel, vers le nord, devant Yahweh~; et les prêtres, fils d'Aaron, en répandront le sang sur l'autel tout autour.
\VS{12}Puis on le coupera en morceaux, avec sa tête et sa graisse~; et le prêtre les posera sur le bois qui sera au feu sur l'autel.
\VS{13}Mais il lavera avec de l'eau les entrailles et les jambes. Puis le prêtre offrira toutes ces choses, et les brûlera sur l'autel. C'est un holocauste, un sacrifice consumé par le feu, d'une agréable odeur à Yahweh\FTNT{Ez. 40:38.}.
\VS{14}Si son offrande à Yahweh est un holocauste d'oiseaux, il offrira son offrande de tourterelles, ou de jeunes pigeons.
\VS{15}Le prêtre l'apportera sur l'autel, lui ouvrira la tête avec l'ongle, la brûlera sur l'autel, et il en exprimera le sang contre un côté de l'autel.
\VS{16}Il ôtera son jabot avec ses plumes, et le jettera près de l'autel, vers l'orient, dans le lieu où seront les cendres.
\VS{17}Il le déchirera avec ses ailes, sans le séparer~; et le prêtre le brûlera sur l'autel, sur le bois qui sera au feu. C'est un holocauste, un sacrifice consumé par le feu, d'une agréable odeur à Yahweh.
\Chap{2}
\TextTitle{L'offrande de gâteau\FTNTT{Lé. 6:7-16.}}
\VerseOne{}Lorsque quelqu'un offrira l'offrande de gâteau\FTNT{L'offrande de farine ou de gâteau correspond aux perfections de la vie du Seigneur Jésus-Christ en tant qu'homme. Ce sacrifice ne comporte ni victime ni sang, mais seulement de la farine, de l'huile, de l'encens et du sel. Jésus, le grain de blé (Jn. 12:24), a été complètement broyé, pétri et oint d'huile, éprouvé par toutes sortes de douleurs. Sa vie sainte était pour le Père un parfum de bonne odeur. Son amour pour les âmes, sa dépendance totale au Père, sa persévérance, sa douceur, sa sagesse et sa bonté, n'ont pas varié malgré toutes les souffrances par lesquelles il est passé. Voilà quelques-uns des fruits admirables qui correspondent à l'offrande de gâteau saupoudrée d'encens. Le levain, image du péché (1 Co. 5:6-8), n'y entrait pas, ni le miel, symbole des affections humaines (Pr. 5:3). Quant au sel, il préserve de la corruption des aliments, il est comparé à la saveur des disciples de Christ (Mt. 5:13).} à Yahweh, son offrande sera de fine farine~; il versera de l'huile dessus, et mettra de l'encens.
\VS{2}Il l'apportera aux fils d'Aaron, les prêtres, et le prêtre prendra une pleine poignée de cette fine farine, et d'huile, avec tout l'encens, et il brûlera son souvenir\FTNT{En hébreu «~azkarah~», offrande de souvenir, la portion de nourriture offerte et qui est consumée.} sur l'autel. C'est une offrande d'une bonne odeur à Yahweh.
\VS{3}Ce qui restera du gâteau sera pour Aaron et ses fils~; c'est une chose très sainte parmi les offrandes consumées par le feu à Yahweh.
\VS{4}Et quand tu offriras une offrande de gâteaux cuits au four, ce sera de fine farine, des gâteaux sans levain, pétris avec de l'huile, et des galettes sans levain, ointes d'huile.
\VS{5}Si ton offrande est un gâteau cuit sur la plaque, elle sera de fine farine pétrie à l'huile, sans levain.
\VS{6}Tu la rompras en morceaux, et tu verseras de l'huile sur elle~; c'est une offrande de gâteau.
\VS{7}Si ton offrande est un gâteau cuit sur le gril, elle sera faite de fine farine avec de l'huile.
\VS{8}Puis tu apporteras à Yahweh l'offrande de gâteaux qui sera faite de ces choses, et on la présentera au prêtre, qui l'apportera sur l'autel.
\VS{9}Le prêtre lèvera de l'offrande de gâteaux, son souvenir, et le brûlera sur l'autel. C'est une offrande consumée par le feu de bonne odeur à Yahweh.
\VS{10}Ce qui restera de l'offrande de gâteau sera pour Aaron et ses fils~; c'est une chose très sainte parmi les offrandes consumées par le feu devant Yahweh.
\VS{11}Aucune offrande de gâteau que vous offrirez à Yahweh ne sera faite avec du levain~; car vous ne brûlerez point de levain ni de miel, parmi l'offrande consumée par le feu devant Yahweh.
\VS{12}Vous pourrez bien les offrir à Yahweh dans l'offrande des prémices, mais ils ne seront point mis sur l'autel comme offrande d'une bonne odeur.
\VS{13}Tu mettras du sel\FTNT{Voir No. 18:19~; 2 Ch. 13:5. Le sel est un agent purificateur (2 R. 2:19-22). Il préserve de la corruption et conserve les aliments. Les chrétiens sont le sel de la terre (Mt. 5:13). Nos paroles doivent être assaisonnées de sel (Col. 4:6).} sur toutes tes offrandes de gâteaux, et tu ne laisseras point ton offrande de gâteau manquer de sel, signe de l'alliance de ton Dieu~; mais sur toutes tes offrandes, tu offriras du sel.
\VS{14}Si tu offres à Yahweh une offrande de gâteau des premiers fruits, tu offriras, pour l'offrande de gâteau des premiers fruits, des épis qui commencent à mûrir, rôtis au feu, les grains de quelques épis bien grenés, broyés entre les mains.
\VS{15}Puis tu mettras de l'huile sur le gâteau, et tu mettras aussi de l'encens dessus. C'est une offrande de gâteaux.
\VS{16}Et le prêtre brûlera son souvenir, pris de ses grains broyés, et de son huile avec tout l'encens. C'est une offrande consumée par le feu à Yahweh.
\Chap{3}
\TextTitle{Le sacrifice d'offrande de paix\FTNTT{Lé. 7:11-21.}}
\VerseOne{}Si son offrande est un sacrifice d'offrande de paix\FTNT{La plupart des traducteurs ont traduit par «~sacrifice d'actions de grâces~», or l'étymologie hébraïque du mot grâce est «~shelem~», ce qui signifie d'abord «~paix~». Ce terme peut aussi vouloir dire «~remerciement~» ou «~reconnaissance~». La racine de «~shelem~» est «~shalam~»~: «~Etre dans une alliance de paix~», «~être en paix~». Il est donc question ici d'une offrande de paix qui préfigure l'ensemble de l'œuvre de la croix accomplie par le Messie, et grâce à laquelle nous sommes réconciliés avec le Père (Col. 1:20~; Ep. 2:14-17). Cette offrande préfigure aussi la Pâque, incarnée par le Messie (1 Co. 5:7) ainsi que le repas du Seigneur. En effet, sur cette offrande Dieu prenait pour lui la graisse et la queue entière (Lé. 3:3~; Lé. 3:9-17), le prêtre prenait la poitrine et l'épaule droite (Lé. 7:31-34), et celui qui offrait l'animal pouvait consommer le reste avec d'autres personnes pures (Lé. 7:20). Ainsi, comme pour le repas du Seigneur, tous ceux qui étaient saints pouvaient participer au repas (1 Co. 11:27-34).}, et qu'il offre du gros bétail, soit mâle, soit femelle, il l'offrira sans défaut devant Yahweh.
\VS{2}Il posera sa main sur la tête de son offrande, et l'égorgera à l'entrée de la tente d'assignation, et les fils d'Aaron, les prêtres, répandront le sang sur l'autel tout autour.
\VS{3}Puis on offrira de cette offrande de paix, un sacrifice consumé par le feu à Yahweh, à savoir la graisse qui couvre les entrailles et toute la graisse qui est sur les entrailles~;
\VS{4}les deux rognons avec la graisse qui est dessus et qui est sur les flancs~; et on ôtera le grand lobe qui est sur le foie pour le mettre avec les rognons.
\VS{5}Les fils d'Aaron brûleront tout cela sur l'autel, sur l'holocauste, qui sera sur le bois mis au feu. C'est une offrande consumée par le feu d'agréable odeur à Yahweh\FTNT{Ex. 29:13-25.}.
\VS{6}Si son offrande pour le sacrifice d'offrande de paix à Yahweh est de menu bétail, soit mâle, soit femelle, il l'offrira sans défaut.
\VS{7}S'il offre un agneau pour son offrande, il l'offrira devant Yahweh.
\VS{8}Il posera sa main sur la tête de son offrande, et l'égorgera devant la tente d'assignation, et les fils d'Aaron répandront son sang sur l'autel tout autour.
\VS{9}De ce sacrifice d'offrande de paix, il offrira en offrande consumée par le feu à Yahweh, sa graisse, et sa queue entière, séparée jusqu'à l'échine, avec la graisse qui couvre les entrailles et toute la graisse qui est sur les entrailles,
\VS{10}les deux rognons avec la graisse qui est dessus, sur les flancs, et il ôtera le grand lobe qui est sur le foie, jusqu'aux rognons.
\VS{11}Le prêtre brûlera tout cela sur l'autel. C'est un aliment d'offrande consumée par le feu à Yahweh\FTNT{No. 28:2.}.
\VS{12}Si son offrande est une chèvre, il l'offrira devant Yahweh.
\VS{13}Il posera sa main sur sa tête, et l'égorgera devant la tente d'assignation~; et les fils d'Aaron répandront son sang sur l'autel tout autour.
\VS{14}Puis il offrira son offrande en sacrifice consumé par le feu à Yahweh, la graisse qui couvre les entrailles et toute la graisse qui est sur les entrailles,
\VS{15}les deux rognons, et la graisse qui est dessus, sur les flancs, et il ôtera le grand lobe qui est sur le foie, jusqu'aux rognons.
\VS{16}Puis le prêtre brûlera toutes ces choses sur l'autel. C'est un aliment d'offrande consumée par le feu de bonne odeur. Toute graisse appartient à Yahweh.
\VS{17}C'est une loi perpétuelle pour vos descendants, dans toutes vos demeures~: Vous ne mangerez ni graisse ni sang\FTNT{Ge. 9:4~; 1 S. 14:33.}.
\Chap{4}
\TextTitle{Le sacrifice pour l'expiation\FTNTT{Lé. 6:17-23.}}
\VerseOne{}Yahweh parla encore à Moïse, en disant~:
\VS{2}Parle aux enfants d'Israël, et dis-leur~: Quand une personne aura péché involontairement\FTNT{Avant la promulgation de la loi, certains hommes péchaient par ignorance (Ro. 5:13). Néanmoins, ces péchés étaient tout de même punis et nécessitaient un sacrifice (Lé. 4:13-14. No. 15:22-36~; Job. 1). Sous la grâce, l'excuse du péché par ignorance ne peut être invoquée puisque nous sommes scellés du Saint-Esprit qui nous enseigne toutes choses (1 Jn. 2:20,27).} contre l'un des commandements de Yahweh, en commettant des choses qui ne doivent point se faire, et qu'elle aura fait une de ces choses~;
\VS{3} si c'est le prêtre oint qui a commis un péché semblable à quelque faute du peuple, il offrira à Yahweh pour son péché qu'il aura fait, un jeune taureau sans défaut, pris du troupeau en sacrifice pour l'expiation.
\VS{4}Il amènera le taureau à l'entrée de la tente d'assignation, devant Yahweh, il posera sa main sur la tête du taureau, et l'égorgera devant Yahweh.
\VS{5}Et le prêtre oint prendra du sang du taureau, et l'apportera dans la tente d'assignation.
\VS{6}Le prêtre trempera son doigt dans le sang, et fera sept fois l'aspersion du sang devant Yahweh, en face du voile du lieu saint\FTNT{No. 19:4.}.
\VS{7}Le prêtre mettra aussi devant Yahweh du sang sur les cornes de l'autel des parfums odoriférants, qui est dans la tente d'assignation~; et il répandra tout le reste du sang du taureau au pied de l'autel de l'holocauste, qui est à l'entrée de la tente d'assignation.
\VS{8}Il enlèvera toute la graisse du taureau du sacrifice pour l'expiation, à savoir, la graisse qui couvre les entrailles, et toute la graisse qui est sur les entrailles,
\VS{9}et les deux rognons avec la graisse qui les entoure, qui couvre les flancs, et il ôtera le grand lobe qui est sur le foie, pour le mettre sur les rognons~;
\VS{10}comme on les enlève du taureau du sacrifice d'offrande de paix\FTNT{Voir commentaire en Lé. 3:1.}, et le prêtre brûlera toutes ces choses-là sur l'autel de l'holocauste.
\VS{11}Mais quant à la peau du taureau et toute sa chair, avec sa tête, ses jambes, ses entrailles, et ses excréments,
\VS{12}et même tout le taureau, il l'emportera hors du camp, dans un lieu pur, où l'on répand les cendres, et il le brûlera au feu sur du bois. Il sera brûlé au lieu où l'on répand les cendres.
\VS{13}Et si toute l'assemblée d'Israël a péché involontairement, et que la chose soit restée cachée aux yeux de l'assemblée, et qu'ils aient violé l'un des commandements de Yahweh, en commettant des choses qui ne doivent pas se faire, et s'en soit rendu coupable,
\VS{14}et que le péché qu'ils ont fait vienne en évidence, l'assemblée offrira en sacrifice pour l'expiation un jeune taureau pris du troupeau, et on l'amènera devant la tente d'assignation.
\VS{15}Les anciens de l'assemblée poseront leurs mains sur la tête du taureau devant Yahweh, et on égorgera le taureau devant Yahweh.
\VS{16}Et le prêtre oint, apportera du sang du taureau dans la tente d'assignation~;
\VS{17}ensuite le prêtre trempera son doigt dans le sang, et en fera aspersion devant Yahweh en face du voile, par sept fois.
\VS{18}Et il mettra du sang sur les cornes de l'autel, qui est devant Yahweh dans la tente d'assignation~; et il répandra tout le reste du sang au pied de l'autel de l'holocauste, qui est à l'entrée de la tente d'assignation.
\VS{19}Il enlèvera toute sa graisse et la brûlera sur l'autel.
\VS{20}Et il fera de ce taureau comme il l'a fait du taureau pour le sacrifice d'expiation. Le prêtre fera ainsi~; il fera propitiation pour eux, et il leur sera pardonné.
\VS{21}Puis il emportera le taureau hors du camp, et le brûlera comme il a brûlé le premier taureau. Car c'est le sacrifice pour l'expiation de l'assemblée.
\VS{22}Que si un chef a péché involontairement, en violant l'un des commandements de Yahweh, son Dieu, ce qui ne doit point se faire, et s'en soit rendu coupable,
\VS{23}et qu'on vienne à connaître le péché qu'il a commis, il amènera pour sacrifice un jeune bouc, mâle, sans défaut~;
\VS{24}et il posera sa main sur la tête du bouc, et l'égorgera au lieu où l'on égorge l'holocauste devant Yahweh. C'est un sacrifice pour expiation.
\VS{25}Puis le prêtre prendra avec son doigt du sang de l'offrande pour l'expiation, et le mettra sur les cornes de l'autel de l'holocauste, et il répandra le reste de son sang au pied de l'autel de l'holocauste.
\VS{26}Et il brûlera toute sa graisse sur l'autel, comme la graisse du sacrifice d'offrande de paix. Ainsi le prêtre fera propitiation pour lui de son péché, et il lui sera pardonné.
\VS{27}Et si quelqu'un du peuple du pays a péché involontairement, en violant l'un des commandements de Yahweh, et en commettant des choses qui ne doivent point se faire, et s'en soit rendu coupable,
\VS{28}et qu'on vienne à connaître le péché qu'il a commis, il amènera pour offrande une jeune chèvre, femelle, sans défaut, pour le péché qu'il a commis.
\VS{29}Et il posera sa main sur la tête de l'offrande pour le péché, et égorgera l'offrande pour l'expiation au lieu où l'on égorge l'holocauste.
\VS{30}Puis le prêtre prendra du sang de la chèvre avec son doigt, et le mettra sur les cornes de l'autel de l'holocauste, et il répandra tout le reste de son sang au pied de l'autel.
\VS{31}Et il ôtera toute sa graisse, comme on ôte la graisse de dessus le sacrifice d'offrande de paix, et le prêtre la brûlera sur l'autel, en bonne odeur à Yahweh. Il fera propitiation pour lui, et il lui sera pardonné.
\VS{32}Et s'il amène un agneau comme offrande, pour le sacrifice d'expiation, il amènera une femelle sans défaut.
\VS{33}Et il posera sa main sur la tête de l'offrande d'expiation, et on l'égorgera en sacrifice pour l'expiation au lieu où l'on égorge l'holocauste.
\VS{34}Puis le prêtre prendra avec son doigt du sang de l'offrande pour l'expiation, et le mettra sur les cornes de l'autel de l'holocauste, et il répandra tout le reste de son sang au pied de l'autel.
\VS{35}Et il ôtera toute sa graisse, comme on ôte la graisse de l'agneau du sacrifice d'offrande de paix, et le prêtre la brûlera sur l'autel, par-dessus les sacrifices de Yahweh consumés par le feu, et il fera propitiation pour lui, pour son péché qu'il aura commis, et il lui sera pardonné.
\Chap{5}
\TextTitle{Le sacrifice de culpabilité\FTNTT{Lé. 7:1-7.}}
\VerseOne{}Et quand quelqu'un, étant témoin, après avoir entendu la parole du serment, aura péché en ne déclarant pas ce qu'il a vu ou ce qu'il sait, il portera son iniquité\FTNT{Pr. 29:24.}.
\VS{2}Et quand quelqu'un, à son insu, aura touché une chose souillée, soit le cadavre d'un animal impur, soit le cadavre d'une bête sauvage impure, soit le cadavre d'un reptile impur, il sera souillé et coupable\FTNT{Ag. 2:14~; 2 Co. 6:17.}.
\VS{3}Ou quand il aura touché à l'impureté d'un homme, quelle que soit son impureté par laquelle il se rend impur, et que cela lui soit resté caché, quand il le sait, alors il est coupable.
\VS{4}Ou quand quelqu'un, parlant légèrement de ses lèvres, a juré de faire du mal ou du bien, selon tout ce que l'homme profère légèrement en jurant, et que cela lui soit resté caché, quand il le sait, alors il est coupable dans l'un de ces points-là.
\VS{5}Quand donc quelqu'un sera coupable sur l'un de ces points là, il confessera ce en quoi il aura péché.
\VS{6}Et il amènera son sacrifice de culpabilité à Yahweh pour le péché qu'il a commis, à savoir, une femelle du menu bétail, soit une brebis, soit une chèvre, pour l'offrande d'expiation. Et le prêtre fera pour lui propitiation de son péché.
\VS{7}Et s'il n'a pas le moyen de trouver une brebis ou une chèvre, il apportera en offrande pour le péché à Yahweh, pour sa culpabilité, deux tourterelles ou deux jeunes pigeons, l'un comme sacrifice pour l'expiation, l'autre pour l'holocauste\FTNT{Lu. 2:24.}.
\VS{8}Il les apportera au prêtre, qui offrira premièrement celui qui est pour l'offrande d'expiation. Il leur ouvrira la tête avec l'ongle, près du cou, sans la séparer~;
\VS{9}puis il fera l'aspersion du sang du sacrifice d'expiation sur un côté de l'autel, et ce qui restera du sang sera exprimé au pied de l'autel. C'est un sacrifice pour l'expiation.
\VS{10}Et il fera de l'autre un holocauste, selon l'ordonnance. Et le prêtre fera pour lui la propitiation pour son péché qu'il aura commis, et il lui sera pardonné.
\VS{11}Si celui qui aura péché n'a pas le moyen de trouver deux tourterelles ou deux jeunes pigeons, il apportera pour son offrande un dixième d'épha de fine farine en offrande pour le sacrifice d'expiation~; il ne mettra ni huile ni encens, car c'est un sacrifice d'expiation.
\VS{12}Il l'apportera au prêtre, et le prêtre qui en prendra une pleine poignée pour souvenir\FTNT{Lé. 2:2.}, la brûlera sur l'autel, comme offrande consumée par le feu à Yahweh. C'est un sacrifice d'expiation.
\VS{13}Ainsi le prêtre fera propitiation pour lui, pour le péché qu'il a commis dans l'une de ces choses, et il lui sera pardonné. Le reste sera pour le prêtre, comme étant une offrande de gâteau.
\VS{14}Yahweh parla aussi à Moïse, en disant~:
\VS{15}Quand quelqu'un aura commis une transgression et péchera involontairement, en retenant des choses consacrées à Yahweh, il amènera un sacrifice de culpabilité à Yahweh, à savoir un bélier sans défaut, pris du troupeau, avec l'estimation que tu feras de la chose sainte, la faisant en sicles d'argent, selon le sicle du sanctuaire, à cause de son péché.
\VS{16}Il restituera donc ce en quoi il aura péché en retenant de la chose sainte et il y ajoutera un cinquième par dessus, et le donnera au prêtre~; et le prêtre fera propitiation pour lui par le bélier du sacrifice de culpabilité, et il lui sera pardonné.
\VS{17}Lorsque quelqu'un aura péché, en violant, sans le savoir, l'un des commandements de Yahweh, des choses qu'on ne doit point faire, il sera coupable et portera son iniquité.
\VS{18}Il amènera donc en sacrifice de culpabilité au prêtre, un bélier sans tâche, pris du troupeau, avec l'estimation que tu feras du péché involontaire~; et le prêtre fera propitiation pour lui du péché involontaire qu'il a commis et dont il ne se sera point aperçu~; et ainsi il lui sera pardonné.
\VS{19}C'est un sacrifice de culpabilité. Il s'est rendu coupable contre Yahweh.
\TextTitle{La restitution au jour du sacrifice de culpabilité\FTNTT{Lé. 7:1-7.}}
\VS{20}Yahweh parla aussi à Moïse, en disant~:
\VS{21}Quand quelqu'un aura péché et aura commis une transgression contre Yahweh, en mentant à son prochain pour un dépôt, pour une chose qu'on aura mise entre ses mains, un vol, ou qu'il ait extorqué son prochain,
\VS{22}ou s'il a trouvé quelque chose perdue, et qu'il mente à ce sujet, ou s'il jure faussement concernant l'une des choses qu'un homme fait en péchant~;
\VS{23}quand il péchera et se rendra coupable, il rendra la chose qu'il a volée ou extorquée, ou le dépôt qui lui a été donné en garde, ou la chose perdue qu'il a trouvée,
\VS{24}ou tout ce dont il aura juré faussement. Il le restituera totalement, et il y ajoutera un cinquième~; il le donnera à celui à qui il appartenait, le jour de son sacrifice de culpabilité.
\VS{25}Et il amènera pour Yahweh, au prêtre le sacrifice de culpabilité, à savoir un bélier sans défaut, pris du troupeau, avec l'estimation que tu feras de la culpabilité.
\VS{26}Et le prêtre fera propitiation pour lui devant Yahweh, et il lui sera pardonné, quelle que soit la faute dont il se sera rendu coupable. 
\Chap{6}
\TextTitle{Loi de l'holocauste\FTNTT{Lé. 1:1-17.}}
\VerseOne{}Yahweh parla aussi à Moïse, en disant~:
\VS{2}Ordonne à Aaron et à ses fils, et dis-leur~: C'est ici la loi de l'holocauste. L'holocauste demeurera sur le foyer de l'autel toute la nuit jusqu'au matin, et le feu brûlera sur l'autel.
\VS{3}Et le prêtre revêtira sa tunique de lin, mettra ses caleçons de lin sur son corps, et il enlèvera la cendre de l'holocauste que le feu aura consumé sur l'autel, puis il la mettra près de l'autel.
\VS{4}Alors il ôtera ses vêtements et portera d'autres vêtements pour transporter les cendres hors du camp, dans un lieu pur.
\VS{5}Et quant au feu qui brûle sur l'autel, il continuera de brûler, on ne l'éteindra point~; le prêtre y brûlera du bois tous les matins, il préparera l'holocauste sur le bois, et y brûlera les graisses des offrandes de paix.
\VS{6}Le feu brûlera continuellement sur l'autel, on ne le laissera point s'éteindre.
\TextTitle{Loi de l'offrande de gâteau\FTNTT{Lé. 2:1-16.}}
\VS{7}Et c'est ici la loi de l'offrande de gâteau. Les fils d'Aaron l'offriront devant Yahweh sur l'autel\FTNT{No. 15:4.}.
\VS{8}Et on lèvera une poignée de la fine farine du gâteau et de son huile, avec tout l'encens qui est sur le gâteau, et on le brûlera sur l'autel, en bonne odeur, en mémorial à Yahweh.
\VS{9}Aaron et ses fils mangeront ce qui en restera~; ils le mangeront sans levain dans un lieu saint, ils le mangeront dans le parvis de la tente d'assignation\FTNT{Ex. 29:26-37.}.
\VS{10}On ne le cuira point avec du levain. Je leur ai donné cela pour leur portion d'entre mes offrandes consumées par le feu. C'est une chose très sainte, comme le sacrifice d'expiation et le sacrifice de culpabilité.
\VS{11}Tout mâle d'entre les fils d'Aaron en mangera. C'est une ordonnance perpétuelle pour vos descendants concernant les offrandes consumées par le feu à Yahweh. Quiconque les touchera sera sanctifié.
\VS{12}Yahweh parla aussi à Moïse, en disant~:
\VS{13}C'est ici l'offrande d'Aaron et de ses fils, qu'ils offriront à Yahweh le jour où il sera oint~: Un dixième d'épha de fine farine, comme offrande de gâteau perpétuelle, une moitié le matin et une moitié le soir.
\VS{14}Elle sera apprêtée sur une plaque avec de l'huile, tu l'apporteras mélangée, et tu offriras les morceaux cuits du gâteau en bonne odeur à Yahweh.
\VS{15}Et le prêtre d'entre ses fils qui sera oint à sa place fera cela. C'est une ordonnance perpétuelle devant Yahweh. On le brûlera tout entier.
\VS{16}Tout le gâteau du prêtre sera entièrement consumé~; on n'en mangera pas.
\TextTitle{Loi du sacrifice d'expiation\FTNTT{Lé. 4:1-35.}}
\VS{17}Yahweh parla aussi à Moïse, en disant~:
\VS{18}Parle à Aaron et à ses fils, et dis-leur~: C'est ici la loi du sacrifice d'expiation. L'offrande pour l'expiation sera égorgée devant Yahweh, dans le même lieu où l'on égorge l'holocauste. C'est une chose très sainte.
\VS{19}Le prêtre qui offrira l'offrande pour l'expiation la mangera~; elle se mangera dans un lieu saint, dans le parvis de la tente d'assignation\FTNT{No. 18:10.}.
\VS{20}Quiconque touchera sa chair sera saint. Et s'il en jaillit du sang sur le vêtement, ce sur quoi il aura jailli sera lavé dans un lieu saint.
\VS{21}Et le vase de terre dans lequel on l'aura fait cuire sera brisé~; mais si on l'a fait cuire dans un vase d'airain, il sera nettoyé et lavé dans l'eau.
\VS{22}Tout mâle d'entre les prêtres en mangera~; car c'est une chose très sainte.
\VS{23}Aucune offrande pour le sacrifice d'expiation, dont on portera le sang dans la tente d'assignation pour faire la propitiation dans le sanctuaire, ne sera mangée, mais elle sera brûlée au feu\FTNT{Hé. 13:11.}.
\Chap{7}
\TextTitle{Loi du sacrifice de culpabilité\FTNTT{Lé. 5:1-26.}}
\VerseOne{}Or c'est ici la loi du sacrifice de culpabilité. C'est une chose très sainte.
\VS{2}Au même lieu où l'on égorgera l'holocauste, on égorgera le sacrifice de culpabilité. On en répandra le sang sur l'autel tout autour.
\VS{3}Puis on en offrira toute la graisse, avec la queue, et toute la graisse qui couvre les entrailles,
\VS{4}les deux rognons, la graisse qui est dessus, sur les flancs, et le grand lobe qui est sur le foie, qu'on ôtera jusqu'aux rognons.
\VS{5}Le prêtre brûlera toutes ces choses sur l'autel comme offrande consumée par le feu à Yahweh. C'est un sacrifice pour la culpabilité.
\VS{6}Tout mâle parmi les prêtres en mangera~; il sera mangé dans un lieu saint~; car c'est une chose très sainte.
\VS{7}Le sacrifice pour l'expiation sera semblable au sacrifice de culpabilité, il y aura une même loi pour les deux~; et la victime appartiendra au prêtre qui aura fait propitiation par elle.
\VS{8}Et le prêtre qui offrira l'holocauste de quelqu'un aura la peau de l'holocauste qu'il aura offert.
\VS{9}Et toute offrande de gâteau cuit au four, apprêtée sur le gril ou sur la plaque, appartiendra au prêtre qui l'offre.
\VS{10}Et toute offrande pétrie à l'huile, ou sèche, sera pour tous les fils d'Aaron, pour l'un comme pour l'autre.
\TextTitle{Loi du sacrifice d'offrande de paix\FTNTT{Lé. 3:1-17.}} 
\VS{11}Et c'est ici, la loi du sacrifice d'offrande de paix\FTNT{Voir commentaire en Lé. 3:1.} qu'on offrira à Yahweh.
\VS{12}Si quelqu'un l'offre pour un sacrifice de reconnaissance, il offrira avec le sacrifice de reconnaissance, des gâteaux sans levain pétris à l'huile, des galettes sans levain ointes d'huile, et des gâteaux de fine farine mêlés et pétris à l'huile.
\VS{13}En plus des gâteaux, il offrira pour son offrande du pain levé avec le sacrifice de reconnaissance de ses offrandes de paix.
\VS{14}Il présentera une part de chaque offrande, qu'il offrira comme offrande élevée à Yahweh~; elle sera pour le prêtre qui a répandu le sang du sacrifice d'offrande de paix.
\VS{15}Mais la chair du sacrifice de reconnaissance de ses offrandes de paix sera mangée le jour où elle sera offerte~; on n'en laissera rien jusqu'au matin.
\VS{16}Et si le sacrifice de son offrande est un vœu ou une offrande volontaire, son sacrifice sera mangé le jour où il l'aura offert~; ce qui en restera sera mangé le lendemain.
\VS{17}Mais ce qui restera de la chair du sacrifice sera brûlé au feu le troisième jour.
\VS{18}Et si on mange de la chair du sacrifice d'offrande de paix le troisième jour, celui qui l'aura offert ne sera point agréé, il ne lui sera point imputé, ce sera une chose infâme, et la personne qui en mangera portera son iniquité\FTNT{Ez. 4:14.}.
\VS{19}Et la chair de ce sacrifice qui a touché quelque chose d'impur ne sera point mangée, elle sera brûlée au feu. Mais quiconque sera pur, mangera de cette chair.
\VS{20}Car une personne qui mangera de la chair du sacrifice d'offrande de paix, laquelle appartient à Yahweh, et qui aura sur elle son impureté, cette personne-là sera retranchée de son peuple.
\VS{21}Si une personne touche quelque chose d'impur, soit une impureté d'homme, soit une bête impure, ou quelque  chose d'autre d'impure, et qu'il mange de la chair du sacrifice d'offrande de paix qui appartient à Yahweh, cette personne-là sera retranchée d'entre son peuple.
\VS{22}Yahweh parla à Moïse, en disant~:
\VS{23}Parle aux enfants d'Israël, et dis-leur~: Vous ne mangerez aucune graisse de bœuf, ni d'agneau, ni de chèvre.
\VS{24}On pourra se servir pour un usage quelconque de la graisse d'une bête morte ou de la graisse d'une bête déchirée~; mais vous n'en mangerez point.
\VS{25}Car quiconque mangera de la graisse d'une bête que l'on offre comme offrande consumée par le feu à Yahweh, la personne qui en mangera sera retranchée de son peuple.
\VS{26}Vous ne mangerez point de sang, ni d'oiseaux, ni d'autres bêtes, dans aucune de vos demeures.
\VS{27}Toute personne, qui aura mangé de quelque sang que ce soit, sera retranchée de son peuple.
\VS{28}Yahweh parla à Moïse, en disant~:
\VS{29}Parle aux enfants d'Israël, et dis-leur~: Celui qui offrira son sacrifice d'offrande de paix à Yahweh, apportera son offrande à Yahweh, prise sur son sacrifice d'offrande de paix.
\VS{30}Il apportera de ses mains les offrandes consumées par le feu devant Yahweh. Il apportera la graisse avec la poitrine, la poitrine pour l'agiter d'un côté et de l'autre devant Yahweh.
\VS{31}Puis le prêtre brûlera la graisse sur l'autel, mais la poitrine sera pour Aaron et ses fils.
\VS{32}Vous donnerez aussi au prêtre pour offrande élevée, l'épaule droite de vos sacrifices d'offrande de paix\FTNT{No. 18:18.}.
\VS{33}Celui des fils d'Aaron qui offrira le sang et la graisse de l'offrande de paix aura pour sa part l'épaule droite.
\VS{34}Car je prends sur les enfants d'Israël, la poitrine qu'on agite d'un côté et de l'autre, et l'épaule qu'on présente par élévation, de tous les sacrifices d'offrande de paix, et je les donne à Aaron le prêtre et à ses fils, par une ordonnance perpétuelle, de la part des fils d'Israël.
\VS{35}C'est là, le droit de l'onction d'Aaron et de l'onction de ses fils sur ces offrandes consumées par le feu devant Yahweh, depuis le jour où on les aura présentés pour exercer la prêtrise à Yahweh.
\VS{36}Et c'est ce que Yahweh ordonne aux enfants d'Israël de leur donner, depuis le jour où on les aura oints~; par une loi perpétuelle parmi leurs descendants\FTNT{Ex. 40:15.}.
\VS{37}Telle est donc la loi de l'holocauste, du gâteau, du sacrifice pour l'expiation, du sacrifice pour la culpabilité, de la consécration et du sacrifice d'offrande de paix.
\VS{38}Yahweh l'ordonna à Moïse sur la montagne de Sinaï, le jour où il ordonna aux enfants d'Israël d'offrir leurs offrandes à Yahweh dans le désert de Sinaï.
\Chap{8}
\TextTitle{Consécration d'Aaron et de ses fils}
\VerseOne{}Yahweh parla aussi à Moïse, en disant~:
\VS{2}Prends Aaron et ses fils avec lui, les vêtements, l'huile d'onction, un jeune taureau pour le sacrifice d'expiation, deux béliers et une corbeille de pains sans levain\FTNT{Ex. 29:1-2~; Ex. 30:25.}~;
\VS{3}et convoque toute l'assemblée à l'entrée de la tente d'assignation.
\VS{4}Et Moïse fit comme Yahweh lui avait ordonné~; et l'assemblée se rassembla à l'entrée de la tente d'assignation.
\VS{5}Moïse dit à l'assemblée~: Voici ce que Yahweh a ordonné de faire.
\TextTitle{La purification avec l'eau} 
\VS{6}Et Moïse fit approcher Aaron et ses fils, et les lava avec de l'eau.
\TextTitle{Les vêtements d'Aaron} 
\VS{7}Et il mit sur Aaron la tunique, il le ceignit de la ceinture, le revêtit de la robe, mit sur lui l'éphod, et le ceignit avec la ceinture de l'éphod dont il le lia.
\VS{8}Puis il mit sur lui le pectoral, après avoir mis au pectoral l'urim et le thummim.
\VS{9}Il lui mit aussi la tiare sur la tête, et il mit sur le devant de la tiare la lame d'or, la couronne de sainteté, comme Yahweh l'avait ordonné à Moïse\FTNT{Ex. 28.}.
\TextTitle{L'onction d'huile} 
\VS{10}Puis Moïse prit l'huile d'onction, et oignit le tabernacle et toutes les choses qui y étaient, et les sanctifia.
\VS{11}Et il en fit l'aspersion sur l'autel par sept fois, et il oignit l'autel, tous ses ustensiles, et la cuve avec sa base, pour les sanctifier.
\VS{12}Il versa aussi de l'huile d'onction sur la tête d'Aaron, et l'oignit pour le sanctifier\FTNT{Ps. 133:2.}.
\TextTitle{Les vêtements des fils d'Aaron}
\VS{13}Puis Moïse fit approcher les fils d'Aaron, les revêtit des tuniques, les ceignit des ceintures et leur attacha des turbans, comme Yahweh l'avait ordonné à Moïse.
\TextTitle{Les offrandes et les sacrifices}
\VS{14}Alors il fit approcher le jeune taureau pour le sacrifice d'expiation, et Aaron et ses fils posèrent leurs mains sur la tête du taureau pour le sacrifice d'expiation.
\VS{15}Et Moïse l'égorgea, prit de son sang, et en mit avec son doigt sur les cornes de l'autel tout autour, et purifia l'autel~; et il répandit le reste du sang au pied de l'autel, ainsi il le sanctifia pour faire la propitiation sur lui.
\VS{16}Puis il prit toute la graisse qui était sur les entrailles, le grand lobe du foie, les deux rognons avec leur graisse, et Moïse les brûla sur l'autel.
\VS{17}Mais il brûla au feu, hors du camp, le jeune taureau avec sa peau, sa chair, et ses excréments, comme Yahweh l'avait ordonné à Moïse.
\VS{18}Il fit aussi approcher le bélier de l'holocauste, et Aaron et ses fils posèrent leurs mains sur la tête du bélier.
\VS{19}Et Moïse l'égorgea et répandit le sang sur l'autel tout autour.
\VS{20}Puis il coupa le bélier en morceaux, et Moïse brûla la tête, les morceaux, et la graisse.
\VS{21}Et il lava dans l'eau les entrailles et les jambes, et brûla tout le bélier sur l'autel. Ce fut un holocauste d'une agréable odeur, c'était une offrande consumée par le feu à Yahweh, comme Yahweh l'avait ordonné à Moïse.
\VS{22}Il fit aussi approcher l'autre bélier, le bélier de consécration, et Aaron et ses fils posèrent les mains sur la tête du bélier.
\VS{23}Et Moïse l'égorgea, prit de son sang, et le mit sur le lobe de l'oreille droite d'Aaron, et sur le pouce de sa main droite et sur le gros orteil de son pied droit.
\VS{24}Il fit aussi approcher les fils d'Aaron, et mit du même sang sur le lobe de leur oreille droite, et sur le pouce de leur main droite, et sur le gros orteil de leur pied droit, et Moïse répandit le reste du sang sur l'autel tout autour.
\VS{25}Après, il prit la graisse, la queue, toute la graisse qui est sur les entrailles, et le grand lobe du foie, et les deux rognons avec leur graisse, et l'épaule droite.
\VS{26}Il prit aussi de la corbeille des pains sans levain, qui étaient devant Yahweh, un gâteau sans levain, et un gâteau de pain fait à l'huile et une galette, et il les mit sur les graisses, et sur l'épaule droite.
\VS{27}Puis il mit toutes ces choses sur les paumes des mains d'Aaron et sur les paumes des mains de ses fils, et les agita d'un côté et de l'autre devant Yahweh.
\VS{28}Puis Moïse les prit de leurs mains et les brûla sur l'autel, sur l'holocauste. Ce fut l'offrande de consécration de bonne odeur, c'est une offrande consumée par le feu devant Yahweh.
\VS{29}Moïse prit aussi la poitrine du bélier de consécration, et l'agita d'un côté et de l'autre devant Yahweh. Ce fut la part de Moïse, comme Yahweh l'avait ordonné à Moïse.
\TextTitle{L'aspersion d'huile et de sang}
\VS{30}Moïse prit de l'huile d'onction et du sang qui était sur l'autel, et il en fit l'aspersion sur Aaron et sur ses vêtements, sur ses fils et sur les vêtements de ses fils~; ainsi il sanctifia Aaron et ses vêtements, les fils d'Aaron et les vêtements de ses fils.
\TextTitle{La nourriture de consécration\FTNTT{Ex. 29:26~; Lé. 7:31-34~; 8:29.}}
\VS{31}Après cela, Moïse dit à Aaron et à ses fils~: Faites cuire la chair à l'entrée de la tente d'assignation, et vous la mangerez là, avec le pain qui est dans la corbeille de consécration, comme je l'ai ordonné, en disant~: Aaron et ses fils la mangeront.
\VS{32}Mais vous brûlerez au feu ce qui restera de la chair et du pain.
\TextTitle{Les prêtres mis à part}
\VS{33}Et vous ne sortirez point pendant sept jours, de l'entrée de la tente d'assignation, jusqu'à ce que vos jours de consécration soient accomplis~; car on emploiera sept jours à vous consacrer.
\VS{34}Yahweh a ordonné de faire en ces autres jours comme on a fait en celui-ci, pour faire la propitiation en votre faveur.
\VS{35}Vous resterez donc pendant sept jours à l'entrée de la tente d'assignation, jour et nuit, et vous observerez ce que Yahweh vous a ordonné d'observer, afin que vous ne mouriez pas~; car il m'a été ainsi ordonné.
\VS{36}Ainsi Aaron et ses fils firent toutes les choses que Yahweh avait ordonnées par Moïse.
\Chap{9}
\TextTitle{Aaron et ses fils commencent leur service dans le tabernacle}
\VerseOne{}Et il arriva au huitième jour, que Moïse appela Aaron et ses fils, et les anciens d'Israël.
\VS{2}Et il dit à Aaron~: Prends un jeune taureau du troupeau pour l'offrande d'expiation, et un bélier pour l'holocauste, tous deux sans défaut, et offre-les devant Yahweh.
\VS{3}Et tu parleras aux enfants d'Israël, en disant~: Prenez un bouc pour l'offrande d'expiation, un jeune taureau et un agneau, tous deux d'un an et sans défaut, pour l'holocauste~;
\VS{4}un bœuf et un bélier pour l'offrande de paix\FTNT{Voir commentaire en Lé. 3:1.}, pour les sacrifier devant Yahweh~; et un gâteau pétri à l'huile. Car aujourd'hui Yahweh vous apparaîtra.
\VS{5}Ils prirent donc les choses que Moïse avait ordonnées et les amenèrent devant la tente d'assignation, et toute l'assemblée s'approcha, et se tint devant Yahweh.
\VS{6}Et Moïse dit~: Faites ce que Yahweh vous a ordonné, et la gloire de Yahweh vous apparaîtra.
\VS{7}Moïse dit à Aaron~: Approche-toi de l'autel, fais ton sacrifice pour l'expiation et ton holocauste, et fais propitiation pour toi et pour le peuple~; présente l'offrande pour le peuple, et fais propitiation pour eux, comme Yahweh l'a ordonné\FTNT{Hé. 7:26-27.}.
\VS{8}Alors Aaron s'approcha de l'autel et égorgea le veau de son sacrifice d'expiation.
\VS{9}Et les fils d'Aaron lui présentèrent le sang, et il trempa son doigt dans le sang et le mit sur les cornes de l'autel~; puis il répandit le reste du sang au pied de l'autel.
\VS{10}Mais il brûla sur l'autel la graisse, et les rognons, et le grand lobe du foie de l'offrande pour le péché, comme Yahweh l'avait ordonné à Moïse.
\VS{11}Et il brûla au feu la chair et la peau hors du camp.
\VS{12}Il égorgea aussi l'holocauste. Les fils d'Aaron lui présentèrent le sang, lequel il répandit sur l'autel tout autour.
\VS{13}Puis ils lui présentèrent l'holocauste coupé en morceaux, avec la tête, et il les brûla sur l'autel.
\VS{14}Et il lava les entrailles et les jambes, qu'il brûla sur l'holocauste, sur l'autel.
\VS{15}Il offrit l'offrande du peuple. Il prit le bouc pour le sacrifice d'expiation du peuple, il l'égorgea et l'offrit pour le péché, comme la première offrande.
\VS{16}Il l'offrit en holocauste, faisant selon l'ordonnance.
\VS{17}Ensuite, il offrit l'offrande du gâteau, et il en remplit la paume de sa main, et la brûla sur l'autel, outre l'holocauste du matin.
\VS{18}Il égorgea aussi le bœuf et le bélier pour le sacrifice d'offrande de paix qui était pour le peuple. Les fils d'Aaron lui présentèrent le sang, lequel il répandit sur l'autel tout autour.
\VS{19}Ils présentèrent la graisse du bœuf et du bélier, la queue, ce qui couvre les entrailles, les rognons, et le grand lobe du foie~;
\VS{20}ils mirent les graisses sur les poitrines, et il brûla les graisses sur l'autel.
\VS{21}Et Aaron agita d'un côté et de l'autre devant Yahweh les poitrines et l'épaule droite, comme Yahweh l'avait ordonné à Moïse.
\VS{22}Aaron éleva aussi ses mains vers le peuple, et le bénit. Puis il descendit, après avoir offert le sacrifice pour l'expiation, l'holocauste et l'offrande de paix.
\VS{23}Moïse donc et Aaron entrèrent dans la tente d'assignation, puis ils sortirent et ils bénirent le peuple. Et la gloire de Yahweh apparut à tout le peuple.
\VS{24}Car le feu sortit de devant Yahweh, et consuma sur l'autel l'holocauste et les graisses. Tout le peuple le vit et ils poussèrent des cris de joie et tombèrent sur leur face\FTNT{1 R. 18:38~; 2 Ch. 7:1.}.
\Chap{10}
\TextTitle{Un feu étranger présenté à Yahweh}
\VerseOne{}Or les fils d'Aaron, Nadab et Abihu, prirent chacun leur encensoir, mirent du feu, et ils posèrent dessus du parfum~; ils offrirent devant Yahweh un feu étranger\FTNT{Ce passage nous avertit du danger auquel s'exposent ceux qui apportent un feu étranger dans le temple. Les feux étrangers sont~: Les fausses doctrines, le péché, les conceptions cartésiennes, pernicieuses, mercantiles, destinés à remplacer la Parole de Dieu et à conduire le chrétien dans les ténèbres, loin de la présence de Yahweh.}, ce qu'il ne leur avait point ordonné.
\VS{2}Et le feu sortit de devant Yahweh, et les dévora~; ils moururent devant Yahweh\FTNT{No. 3:4.}.
\VS{3}Moïse dit à Aaron~: C'est ce dont Yahweh avait parlé, en disant~: Je serai sanctifié par ceux qui s'approchent de moi, et je serai glorifié en présence de tout le peuple. Et Aaron se tut.
\VS{4}Et Moïse appela Mischaël et Eltsaphan, les fils d'Uziel, oncle d'Aaron, et leur dit~: Approchez-vous, emportez vos frères de devant le sanctuaire, hors du camp.
\VS{5}Alors ils s'approchèrent et les emportèrent avec leurs tuniques hors du camp, comme Moïse l'avait dit.
\TextTitle{Instructions données par Moïse} 
\VS{6}Puis Moïse dit à Aaron, à Eléazar et à Ithamar, ses fils~: Ne découvrez point vos têtes, et ne déchirez point vos vêtements, de peur que vous ne mouriez, et que Yahweh ne se mette en colère contre toute l'assemblée. Mais que vos frères, toute la maison d'Israël, pleurent à cause de l'embrasement que Yahweh a allumé\FTNT{Ez. 24:17.}.
\VS{7}Et ne sortez point de l'entrée de la tente d'assignation, de peur que vous ne mouriez, car l'huile de l'onction de Yahweh est sur vous. Et ils firent selon la parole de Moïse.
\VS{8}Et Yahweh parla à Aaron, en disant~:
\VS{9}Vous ne boirez point de vin, ni de boisson forte, ni toi ni tes fils avec toi, quand vous entrerez dans la tente d'assignation, de peur que vous ne mouriez~; c'est une ordonnance perpétuelle pour vos descendants\FTNT{No. 6:3~; Jg. 13:7.},
\VS{10}afin que vous puissiez discerner entre ce qui est saint et ce qui est profane, entre ce qui est impur et ce qui est pur,
\VS{11}afin que vous enseigniez aux enfants d'Israël toutes les ordonnances que Yahweh leur a prononcées par Moïse.
\VS{12}Puis Moïse parla à Aaron, à Eléazar et à Ithamar, ses fils qui lui restaient~: Prenez l'offrande de gâteau, leur dit-il, ce qui reste des offrandes de Yahweh consumées par le feu, et mangez-la avec des pains sans levain auprès de l'autel, car c'est une chose très sainte.
\VS{13}Vous la mangerez dans un lieu saint, parce que c'est la portion qui est assignée à toi et à tes fils sur les offrandes consumées par le feu à Yahweh~; car il m'a été ainsi ordonné.
\VS{14}Vous mangerez aussi la poitrine offerte par agitation et l'épaule présentée par élévation dans un lieu pur, toi, tes fils et tes filles avec toi~; car ces choses-là t'ont été données, dans les sacrifices d'offrande de paix\FTNT{Voir commentaire en Lé. 3:1.} des enfants d'Israël, comme ton droit et le droit de tes fils.
\VS{15}Ils apporteront l'épaule présentée par élévation et la poitrine offerte par agitation, avec les offrandes consumées par le feu, qui sont les graisses, pour les agiter en offrande çà et là devant Yahweh~: Cela t'appartiendra, et à tes fils avec toi, par une ordonnance perpétuelle, comme Yahweh l'a ordonné.
\VS{16}Or Moïse cherchait soigneusement le bouc de l'offrande pour l'expiation mais voici, il avait été brûlé. Et Moïse se mit en grande colère contre Eléazar et Ithamar, les fils d'Aaron qui lui restaient, et leur dit~:
\VS{17}Pourquoi n'avez-vous point mangé l'offrande pour l'expiation dans un lieu saint~? Car c'est une chose très sainte~; vu qu'elle vous a été donnée pour porter l'iniquité de l'assemblée, afin de faire propitiation pour eux devant Yahweh.
\VS{18}Voici, son sang n'a point été porté à l'intérieur du sanctuaire~; vous la mangerez, vous la mangerez dans le lieu saint, comme je l'avais ordonné.
\VS{19}Alors Aaron répondit à Moïse~: Voici, ils ont offert aujourd'hui leur offrande pour l'expiation et leur holocauste devant Yahweh, et ces choses-ci me sont arrivées. Si j'avais mangé aujourd'hui l'offrande pour le péché, cela aurait-il plu à Yahweh~?
\VS{20}Et Moïse l'entendit, et cela fut bon à ses yeux.
\Chap{11}
\TextTitle{Lois de purification~: Les bêtes pures et les bêtes impures}
\VerseOne{}Et Yahweh parla à Moïse et à Aaron, et leur dit~:
\VS{2}Parlez aux enfants d'Israël, et dites-leur~: Ce sont ici les bêtes dont vous mangerez d'entre toutes les bêtes qui sont sur la terre\FTNT{De. 14:4~; Ac. 10:11-14.}.
\VS{3}Vous mangerez d'entre les bêtes toutes celles qui ont le sabot fendu, qui ont le pied fourchu, et qui ruminent.
\VS{4}Mais vous ne mangerez point de celles qui ruminent uniquement, ou qui ont uniquement le sabot fendu, comme le chameau, car il rumine mais il n'a point le sabot fendu. Il vous sera impur.
\VS{5}Et le lapin, car il rumine mais il n'a point le sabot fendu. Il vous sera impur.
\VS{6}Le lièvre, car il rumine mais il n'a point le sabot fendu. Il vous sera impur.
\VS{7}Le porc, car il a bien le sabot fendu et le pied fourchu, mais il ne rumine pas. Il vous sera impur.
\VS{8}Vous ne mangerez point de leur chair, même vous ne toucherez point leur cadavre. Ils vous seront impurs.
\VS{9}Vous mangerez de ceci d'entre tout ce qui est dans les eaux~; vous mangerez de tout ce qui a des nageoires et des écailles dans les eaux, soit dans la mer, soit dans les fleuves.
\VS{10}Mais vous ne mangerez rien de ce qui n'a point de nageoires et d'écailles, soit dans la mer, soit dans les fleuves, tant des reptiles des eaux, que de toute chose vivante qui est dans les eaux, cela vous sera en abomination.
\VS{11}Elles vous seront donc en abomination, vous ne mangerez point de leur chair, et vous tiendrez pour une chose abominable leur cadavre.
\VS{12}Tout ce donc qui vit dans les eaux et qui n'a point de nageoires et d'écailles vous sera en abomination.
\VS{13}Et d'entre les oiseaux vous tiendrez ceux-ci pour abominables, on n'en mangera point, ils vous seront en abomination~: L'aigle, l'orfraie, l'aigle de mer~;
\VS{14}le vautour, et le milan, selon leur espèce~;
\VS{15}tout corbeau, selon son espèce~;
\VS{16}l'autruche, le hibou, la mouette, et l'épervier selon leur espèce~;
\VS{17}le chat-huant, le plongeon, la chouette~;
\VS{18}le cygne, le cormoran, le pélican~;
\VS{19}la cigogne, le héron selon leur espèce, la huppe et la chauve-souris,
\VS{20}et tout reptile volant qui marche sur quatre pattes vous sera en abomination.
\VS{21}Mais vous pourrez manger de toute chose rampante qui vole et qui va sur quatre pattes qui ont des jambes au-dessus de leurs pieds pour sauter avec celles-ci sur la terre.
\VS{22}Ce sont donc ici ceux dont vous mangerez~: La sauterelle selon son espèce, le solam\FTNT{«~Solam~», «~hargol~» et «~hagab~» sont diverses espèces de sauterelles.} selon son espèce, le hargol, selon son espèce et le hagab, selon son espèce.
\VS{23}Mais tout autre reptile volant qui a quatre pattes vous sera en abomination.
\VS{24}Vous serez donc impurs par ces bêtes~; quiconque touchera leur cadavre sera impur jusqu'au soir,
\VS{25}et quiconque aussi portera leur cadavre lavera ses vêtements et sera impur jusqu'au soir.
\VS{26}Toute bête qui a le sabot fendu, et qui n'a point le pied fourchu et ne rumine point, vous sera impur. Quiconque les touchera sera impur.
\VS{27}Tout ce qui marche sur ses pattes, entre tous les animaux qui marchent à quatre pieds, vous sera impur. Quiconque touchera leur cadavre sera impur jusqu'au soir,
\VS{28}et celui qui portera leur cadavre lavera ses vêtements et sera impur jusqu'au soir. Ils vous seront impurs.
\VS{29}Ceci aussi vous sera impur entre les reptiles, qui rampent sur la terre~: La taupe, la souris et la tortue, selon leur espèce~;
\VS{30}le hérisson, la grenouille, le lézard, la limace et le caméléon.
\VS{31}Ces choses vous seront impures entre les reptiles. Quiconque les touchera mortes sera impur jusqu'au soir.
\VS{32}Aussi, tout ce sur quoi il en tombera quelque chose quand elles seront mortes sera impur, soit ustensile de bois, soit vêtement, soit peau, ou sac, quelque objet que ce soit dont on se sert pour faire quelque chose~; il sera mis dans l'eau, et sera impur jusqu'au soir~; puis il sera pur.
\VS{33}Mais s'il en tombe quelque chose dans quelque vase de terre que ce soit, tout ce qui est dedans sera impur, et vous casserez le vase.
\VS{34}Et tout aliment qu'on mange, sur lequel il y aura eu de cette eau, sera impur~; tout breuvage qu'on boit dans quelque vase que ce soit, en sera impur.
\VS{35}Et s'il tombe quelque chose de leur cadavre sur quoi que ce soit, cela sera impur~; le four et le foyer seront détruits. Ils seront impurs, et ils vous seront impurs.
\VS{36}Toutefois, la source, le puits ou tel autre amas d'eaux resteront purs~; mais celui donc qui touchera leur cadavre sera impur.
\VS{37}Et s'il est tombé de leur cadavre sur quelque semence qui se sème, elle restera pure.
\VS{38}Mais si on avait mis de l'eau sur la semence, et que quelque chose de leur cadavre tombe sur elle, elle vous sera impure.
\VS{39}Et si une des bêtes qui vous servent de nourriture meurt, celui qui en touchera le cadavre sera impur jusqu'au soir~;
\VS{40}celui qui mangera de son cadavre lavera ses vêtements et sera impur jusqu'au soir, et celui aussi qui portera le cadavre de cette bête lavera ses vêtements et sera impur jusqu'au soir.
\VS{41}Tout reptile donc qui rampe sur la terre vous sera en abomination~; et on n'en mangera point\FTNT{cp. Ge. 3:14.}.
\VS{42}Vous ne mangerez point de tout ce qui rampe sur la poitrine, ni de tout ce qui marche sur les quatre pieds, ni de tout ce qui a plusieurs pieds entre tous les reptiles qui se traînent sur la terre~; car ils seront en abomination.
\VS{43}Ne rendez point vos personnes abominables par tous ces reptiles qui grouillent~; ne vous rendez point impurs par eux, ne vous souillez point par eux.
\VS{44}Car je suis Yahweh, votre Dieu~; vous vous sanctifierez donc et vous serez saints, car je suis saint\FTNT{1 Pi. 1:16.}~! Ainsi, vous ne rendrez point vos personnes impures par aucun reptile qui se traîne sur la terre.
\VS{45}Car je suis Yahweh, qui vous ai fait monter du pays d'Egypte, afin que je sois votre Dieu et que vous soyez saints, car je suis saint~!
\VS{46}Telle est la loi concernant les animaux, les oiseaux, tout être vivant, qui se meut dans les eaux, et toute être vivant, qui rampe sur la terre,
\VS{47}afin de discerner entre la chose impure et la chose pure, entre les animaux qu'on peut manger et les animaux dont on ne doit point manger.
\Chap{12}
\TextTitle{Lois de purification~: Le flux de sang\FTNTT{Ps. 51:7.}}
\VerseOne{}Yahweh parla aussi à Moïse, en disant~:
\VS{2}Parle aux enfants d'Israël, et dis-leur~: Si la femme après avoir conçu, enfante un mâle, elle sera impure pendant sept jours~; elle sera impure comme au temps de son indisposition menstruelle.
\VS{3}Et au huitième jour, on circoncira la chair du prépuce de l'enfant\FTNT{Les parents de Jésus ont observé cette loi (Lu. 2:21-24). Jn. 7:22.}.
\VS{4}Et elle demeurera trente-trois jours à se purifier de son sang~; elle ne touchera aucune chose sainte et ne viendra point au sanctuaire, jusqu'à ce que les jours de sa purification soient accomplis.
\VS{5}Si elle enfante une fille, elle sera impure deux semaines, comme au temps de son indisposition menstruelle, et elle restera soixante-six jours à se purifier de son sang.
\VS{6}Après que le temps de sa purification sera accompli, soit pour un fils ou pour une fille, elle présentera au prêtre un agneau d'un an en holocauste, et un jeune pigeon ou une tourterelle en sacrifice d'expiation, à l'entrée de la tente d'assignation\FTNT{No. 6:10.}.
\VS{7}Et le prêtre offrira ces choses devant Yahweh, et fera propitiation pour elle~; et elle sera purifiée du flux de son sang. Telle est la loi pour celle qui enfante un fils ou une fille.
\VS{8}Et que, si elle n'a pas le moyen de trouver un agneau, alors elle prendra deux tourterelles ou deux jeunes pigeons, l'un pour l'holocauste, et l'autre pour le sacrifice d'expiation. Le prêtre fera propitiation pour elle, et elle sera pure.
\Chap{13}
\TextTitle{Lois de purification~: La lèpre}
\VerseOne{}Yahweh parla aussi à Moïse et à Aaron, en disant~:
\VS{2}L'homme qui aura sur la peau de son corps une tumeur, une dartre, ou une tache blanche, et que cela paraîtra sur la peau de son corps comme une plaie de lèpre, on l'amènera à Aaron, le prêtre, ou à l'un de ses fils prêtres.
\VS{3}Et le prêtre examinera la plaie qui est sur la peau du corps. Si le poil de la plaie est devenu blanc, et si la plaie, à la voir, est plus profonde que la peau du corps, c'est une plaie de lèpre. Le prêtre donc l'examinera et le jugera impur.
\VS{4}Mais si la tache est blanche sur la peau du corps, et qu'à la voir, elle n'est point plus profonde que la peau, et si son poil n'est pas devenu blanc, le prêtre enfermera pendant sept jours celui qui a la plaie.
\VS{5}Et le prêtre l'examinera le septième jour. Si à ses yeux la plaie s'est arrêtée, et qu'elle ne s'est point étendue sur la peau, le prêtre l'enfermera une seconde fois pendant sept jours.
\VS{6}Et le prêtre l'examinera une seconde fois le septième jour. Si la plaie est devenue pâle, et qu'elle ne s'est point étendue sur la peau, le prêtre le jugera pur. C'est de la dartre~; il lavera ses vêtements, et sera pur.
\VS{7}Mais si la dartre s'est étendue sur la peau, après avoir été vu par le prêtre pour être jugé pur, il se fera examiner pour la seconde fois par le prêtre.
\VS{8}Le prêtre l'examinera encore. S'il aperçoit que la dartre s'est étendue sur la peau, le prêtre le jugera impur. C'est de la lèpre.
\VS{9}Quand il y aura une plaie de lèpre sur un homme, on l'amènera au prêtre.
\VS{10}Le prêtre l'examinera. Et s'il aperçoit qu'il y a une tumeur blanche sur la peau, que le poil est devenu blanc, et qu'il y a une trace de chair vive dans la tumeur,
\VS{11}c'est une lèpre invétérée dans la peau du corps. Le prêtre le jugera impur~; il ne l'enfermera pas, car il est jugé impur.
\VS{12}Si la lèpre fait une éruption sur la peau, et qu'elle couvre toute la peau de celui qui a la plaie, depuis la tête de cet homme jusqu'à ses pieds, partout où pourra voir le prêtre, le prêtre l'examinera,
\VS{13}et si le prêtre voit que la lèpre couvre tout le corps de cet homme, alors il jugera pur celui qui a la plaie. La plaie est devenue toute blanche, il est pur.
\VS{14}Mais le jour où l'on apercevra de la chair vive, il sera impur~;
\VS{15}alors le prêtre examinera la chair vive, et le jugera impur. La chair vive est impure, c'est de la lèpre.
\VS{16}Si la chair vive se change et devient blanche, alors il viendra vers le prêtre~;
\VS{17}et le prêtre l'examinera, et s'il aperçoit que la plaie est devenue blanche, le prêtre jugera pur celui qui a la plaie. Il est pur.
\VS{18}Si le corps a eu sur la peau un ulcère qui soit guéri,
\VS{19}et qu'à l'endroit où était l'ulcère il y ait une tumeur blanche, ou une tache blanche rougeâtre, cet homme se montrera au prêtre.
\VS{20}Le prêtre donc l'examinera. Et s'il aperçoit, qu'à la voir, elle paraît plus enfoncée que la peau, et que son poil est devenu blanc, alors le prêtre le jugera impur. C'est une plaie de lèpre qui a fait éruption dans l'ulcère.
\VS{21}Mais si le prêtre voit qu'il n'y a pas de poil blanc dans la tâche, et qu'elle n'est pas plus enfoncée que la peau, mais qu'elle est devenue pâle, le prêtre l'enfermera pendant sept jours.
\VS{22}Si elle s'est étendue sur la peau en quelque sorte que ce soit, le prêtre le jugera impur. C'est une plaie.
\VS{23}Mais si la tache est restée à la même place et ne s'est pas étendue, c'est une cicatrice d'ulcère. Ainsi le prêtre le jugera pur.
\VS{24}Si le corps a sur la peau une brûlure par le feu, et que la chair vive de la partie brûlée soit une tache blanche rougeâtre ou blanc seulement, le prêtre la regardera,
\VS{25}et si le poil est devenu blanc dans la tache, et qu'à la voir, elle est plus profonde que la peau, c'est de la lèpre, elle a fait éruption dans la brûlure~; le prêtre donc le jugera impur. C'est une plaie de lèpre.
\VS{26}Mais si le prêtre voit qu'il n'y a pas de poil blanc dans la tache, et qu'elle n'est point plus enfoncée que la peau, qu'elle est devenue pâle, le prêtre l'enfermera pendant sept jours.
\VS{27}Puis le prêtre l'examinera le septième jour. Si la tache s'est étendue sur la peau, le prêtre le jugera impur. C'est une plaie de lèpre.
\VS{28}Si la tache est restée à la même place, ne s'est pas étendue, et est devenue pâle, c'est la tumeur de la brûlure~; et le prêtre le jugera pur~; c'est la cicatrice de la brûlure.
\VS{29}Si l'homme ou la femme a une plaie à la tête, ou l'homme à la barbe,
\VS{30}le prêtre examinera la plaie, et si, à la voir, elle est plus profonde que la peau, et qu'il y ait en elle du poil jaunâtre et fin, le prêtre le jugera impur. C'est de la teigne, c'est une lèpre de la tête ou de la barbe.
\VS{31}Si le prêtre voit que la plaie de la teigne ne paraît pas plus profonde que la peau, et qu'il n'y a pas de poil noir, alors le prêtre enfermera pendant sept jours celui qui a la plaie de la teigne.
\VS{32}Et le septième jour le prêtre examinera la plaie. Si la teigne ne s'est point étendue, qu'elle n'a aucun poil jaunâtre, et, qu'à voir la teigne, elle n'est pas plus profonde que la peau,
\VS{33}celui qui a la plaie de la teigne se rasera, mais il ne se rasera point à l'endroit de la teigne, et le prêtre enfermera pendant sept autres jours celui qui a la teigne.
\VS{34}Puis le prêtre examinera la teigne au septième jour. Si la teigne ne s'est point étendue sur la peau et, qu'à la voir, elle n'est point plus profonde que la peau, le prêtre le jugera pur, et cet homme lavera ses vêtements, et il sera pur.
\VS{35}Mais si la teigne s'est étendue sur la peau, après sa purification, le prêtre l'examinera,
\VS{36}et si la teigne s'est étendue sur la peau, le prêtre ne cherchera point de poil jaunâtre. Il est impur.
\VS{37}Mais si la teigne s'est arrêtée, et qu'il y ait poussé du poil noir, la teigne est guérie. Il est pur, et le prêtre le jugera pur.
\VS{38}Si l'homme ou la femme ont sur la peau de leur corps des taches, des taches qui sont blanches,
\VS{39}le prêtre l'examinera. Si sur la peau de leur corps il y a des taches d'un blanc pâle, c'est une tache blanche qui a fait éruption sur la peau. Il est donc pur.
\VS{40}Si l'homme a la tête dépouillée de cheveux, c'est un chauve. Il est pur.
\VS{41}Et si sa tête est dépouillée de cheveux du côté de son visage, c'est un front chauve. Il est pur.
\VS{42}Et si dans la partie chauve de devant ou de derrière, il y a une plaie d'un blanc rougeâtre, c'est une lèpre qui a fait éruption dans sa partie chauve de derrière ou de devant.
\VS{43}Et le prêtre le regardera. S'il aperçoit que la tumeur de la plaie est d'un blanc rougeâtre dans sa partie chauve de derrière ou de devant, semblable à la lèpre de la peau du corps,
\VS{44}l'homme est lépreux, il est impur. Impur, le prêtre le jugera impur~; c'est à la tête qu'est sa plaie.
\VS{45}Or le lépreux qui sera atteint de la plaie aura ses vêtements déchirés, et sa tête nue~; et il se couvrira la moustache et criera~: Impur~! Impur~!
\VS{46}Pendant tout le temps qu'il aura cette plaie, il sera jugé impur. Il est impur. Il demeurera seul~; sa demeure sera hors du camp\FTNT{2 R. 7:3~; La. 4:15~; Lu. 17:12-13.}.
\VS{47}Et si le vêtement est infecté de la plaie de la lèpre, soit sur un vêtement de laine, soit sur un vêtement de lin,
\VS{48}à la chaîne ou à la trame du lin, ou de laine, sur la peau ou sur quelque ouvrage de peau,
\VS{49}et si cette plaie est verdâtre ou rougeâtre sur le vêtement ou sur la peau, à la chaîne ou à la trame, ou sur un objet quelconque de peau, ce sera une plaie de lèpre, et elle sera montrée au prêtre.
\VS{50}Et le prêtre examinera la plaie, et enfermera pendant sept jours celui qui a la plaie.
\VS{51}Et au septième jour, il examinera la plaie. Si la plaie s'est étendue sur le vêtement, à la chaîne ou à la trame, sur la peau ou sur quelque ouvrage de peau, la plaie est une lèpre invétérée. La chose est impure.
\VS{52}Il brûlera le vêtement, la chaîne ou la trame de laine ou de lin, et toutes les choses de peau, qui auront cette plaie, car c'est une lèpre rongeuse. Cela sera brûlé au feu.
\VS{53}Mais si le prêtre voit que la plaie ne s'est point étendue sur le vêtement, sur la chaîne ou sur la trame, ou sur quelque objet de peau,
\VS{54}le prêtre ordonnera qu'on lave la chose où est la plaie, et il l'enfermera une seconde fois pendant sept jours.
\VS{55}Si le prêtre observe que la plaie n'a point changé sa couleur après qu'on l'aura lavée, et qu'elle ne s'est point étendue, c'est une chose impure. Tu la brûleras au feu~; c'est une partie de l'endroit ou de l'envers qui a été rongée.
\VS{56}Si le prêtre voit que la plaie est devenue pâle après avoir été lavée, il la déchirera du vêtement ou de la peau, de la chaîne ou de la trame.
\VS{57}Si elle paraît encore sur le vêtement, à la chaîne ou à la trame, ou sur quelque chose de peau, c'est une lèpre qui a fait éruption. Vous brûlerez au feu la chose où est la plaie.
\VS{58}Mais si tu as lavé le vêtement, la chaîne ou la trame, ou quelque chose de peau, et que la plaie s'en est allée, il sera lavé une seconde fois, puis il sera pur.
\VS{59}Telle est la loi sur la plaie de la lèpre sur un vêtement de laine ou de lin, la chaîne ou la trame, ou quelque chose de peau, pour la juger pure ou impure.
\Chap{14}
\TextTitle{Loi du lépreux pour le jour de sa purification}
\VerseOne{}Yahweh parla aussi à Moïse, en disant~:
\VS{2}C'est ici la loi du lépreux pour le jour de sa purification. Il sera amené au prêtre\FTNT{Mt. 8:2-4~; Mc. 1:42-44~; Lu. 5:12-14.}.
\VS{3}Le prêtre sortira hors du camp et l'examinera. Si la plaie de la lèpre du lépreux est guérie,
\VS{4}le prêtre ordonnera qu'on prenne pour celui qui doit être purifié, deux oiseaux vivants et purs, avec du bois de cèdre, du cramoisi et de l'hysope\FTNT{Ex. 12:22.}.
\VS{5}Et le prêtre ordonnera qu'on égorge l'un des oiseaux sur un vase de terre, sur de l'eau vive.
\VS{6}Puis il prendra l'oiseau vivant, le bois de cèdre, le cramoisi et l'hysope~; et il trempera toutes ces choses avec l'oiseau vivant, dans le sang de l'autre oiseau qui aura été égorgé sur de l'eau vive.
\VS{7}Il en fera sept fois l'aspersion sur celui qui doit être purifié de la lèpre. Il le déclarera pur, et il laissera aller par les champs l'oiseau vivant.
\VS{8}Et celui qui doit être purifié lavera ses vêtements, rasera tout son poil, et se lavera dans l'eau~; et il sera pur. Ensuite il entrera dans le camp, mais il demeurera sept jours hors de sa tente.
\VS{9}Au septième jour, il rasera tout son poil, sa tête, sa barbe, les sourcils de ses yeux, tout son poil~; il rasera tout son poil~; puis il lavera ses vêtements et son corps, et il sera pur.
\VS{10}Et au huitième jour, il prendra deux agneaux sans défaut, une brebis d'un an sans défaut, et trois dixièmes de fine farine en offrande de gâteau, pétrie à l'huile, et un log d'huile.
\VS{11}Le prêtre qui fait la purification présentera celui qui doit être purifié et ces choses-là devant Yahweh, à l'entrée de la tente d'assignation.
\VS{12}Puis le prêtre prendra l'un des agneaux et l'offrira en sacrifice pour la culpabilité avec un log d'huile~; il agitera ces choses devant Yahweh, en offrande agitée.
\VS{13}Et il égorgera l'agneau au lieu où l'on égorge l'offrande pour l'expiation et l'holocauste, dans le lieu saint~; car le sacrifice pour la culpabilité appartient au prêtre, comme le sacrifice pour l'expiation~; c'est une chose très sainte.
\VS{14}Le prêtre prendra du sang de l'offrande pour la culpabilité~; il le mettra sur le lobe de l'oreille droite de celui qui doit être purifié, sur le pouce de sa main droite et sur le gros orteil de son pied droit.
\VS{15}Puis le prêtre prendra du log d'huile et en versera dans la paume de sa main gauche.
\VS{16}Et le prêtre trempera le doigt de sa main droite dans l'huile qui est dans sa paume gauche, et fera l'aspersion de l'huile avec son doigt sept fois devant Yahweh.
\VS{17}Et du reste de l'huile qui sera dans sa paume, le prêtre en mettra sur le lobe de l'oreille droite de celui qui doit être purifié, sur le pouce de sa main droite et sur le gros orteil de son pied droit, sur le sang pris de l'offrande pour la culpabilité.
\VS{18}Mais ce qui restera de l'huile sur la paume du prêtre, il le mettra sur la tête de celui qui doit être purifié~; et ainsi le prêtre fera propitiation pour lui devant Yahweh.
\VS{19}Ensuite le prêtre offrira le sacrifice pour l'expiation et fera propitiation pour celui qui doit être purifié de sa souillure, puis il égorgera l'holocauste.
\VS{20}Le prêtre offrira l'holocauste et le gâteau sur l'autel, et fera propitiation pour celui qui doit être purifié, et il sera pur.
\VS{21}Mais s'il est pauvre et s'il n'a pas le moyen de fournir ces choses, il prendra un agneau en offrande agitée pour la culpabilité, afin de faire propitiation pour lui~; et un dixième de fine farine pétrie à l'huile pour le gâteau, avec un log d'huile~;
\VS{22}et deux tourterelles ou deux jeunes pigeons, selon ce qu'il pourra fournir, dont l'un sera pour le péché et l'autre pour l'holocauste.
\VS{23}Et le huitième jour de sa purification, il les apportera au prêtre, à l'entrée de la tente d'assignation, devant Yahweh.
\VS{24}Et le prêtre recevra l'agneau du sacrifice pour la culpabilité et le log d'huile, et les agitera devant Yahweh en offrande agitée.
\VS{25}Et il égorgera l'agneau du sacrifice pour la culpabilité. Puis le prêtre prendra du sang de l'offrande pour la culpabilité, il le mettra sur le lobe de l'oreille droite de celui qui doit être purifié, sur le pouce de sa main droite et sur le gros orteil de son pied droit.
\VS{26}Puis le prêtre versera de l'huile dans la paume de sa main gauche.
\VS{27}Et avec le doigt de sa main droite, il fera l'aspersion de l'huile qui est dans sa main gauche sept fois devant Yahweh.
\VS{28}Il mettra de cette huile qui est dans sa paume, sur le lobe de l'oreille droite de celui qui doit être purifié et sur le pouce de sa main droite et sur le gros orteil de son pied droit, sur le lieu du sang pris de l'offrande pour la culpabilité.
\VS{29}Après il mettra le reste de l'huile qui est dans sa paume sur la tête de celui qui doit être purifié, afin de faire propitiation pour lui devant Yahweh.
\VS{30}Puis il sacrifiera l'une des tourterelles ou l'un des jeunes pigeons, selon ce qu'il aura pu fournir.
\VS{31}De ce donc qu'il aura pu fournir, l'un sera pour le sacrifice d'expiation et l'autre pour l'holocauste, avec le gâteau~; ainsi le prêtre fera propitiation devant Yahweh pour celui qui doit être purifié.
\VS{32}Telle est la loi de celui qui a une plaie de lèpre, et dont les ressources sont insuffisantes à sa purification.
\TextTitle{Lois de purification d'une maison lépreuse}
\VS{33}Puis Yahweh parla à Moïse et à Aaron, en disant~:
\VS{34}Quand vous serez entrés dans le pays de Canaan, que je vous donne en possession, si j'envoie une plaie de lèpre sur une maison du pays que vous posséderez,
\VS{35}celui à qui la maison appartiendra viendra et le fera savoir au prêtre, en disant~: Il me semble que j'aperçois comme une plaie dans ma maison.
\VS{36}Alors le prêtre ordonnera qu'on vide la maison avant qu'il y entre pour regarder la plaie, afin que rien de ce qui est dans la maison ne soit impur, puis le prêtre entrera pour voir la maison.
\VS{37}Et il regardera la plaie. Si la plaie qui est sur les murs de la maison a des creux verdâtres ou rougeâtres, qui soient, à les voir, plus enfoncés que le mur~;
\VS{38}le prêtre sortira de la maison, à l'entrée, et fera fermer la maison pendant sept jours.
\VS{39}Au septième jour, le prêtre retournera et la regardera. Si la plaie s'est étendue sur les murs de la maison,
\VS{40}alors il ordonnera de retirer les pierres sur lesquelles est la plaie, et de les jeter hors de la ville, dans un lieu impur.
\VS{41}Il fera aussi racler l'enduit de la maison à l'intérieur, tout autour~; et l'enduit qu'on aura raclé, on le jettera hors de la ville, dans un lieu impur.
\VS{42}Puis on prendra d'autres pierres, et on les mettra à la place des premières pierres~; et on prendra d'autres mortiers pour recrépir la maison.
\VS{43}Mais si la plaie revient et fait éruption dans la maison, après avoir retiré les pierres, après avoir raclé et recrépi la maison,
\VS{44}le prêtre y entrera et l'examinera. Si la plaie s'est étendue dans la maison, c'est une lèpre invétérée dans la maison. Elle est impure.
\VS{45}On démolira la maison, ses pierres, son bois, et tout le mortier de la maison~; et on les transportera hors de la ville, dans un lieu impur.
\VS{46}Si quelqu'un est entré dans la maison pendant tout le temps que le prêtre l'avait faite fermer, il sera impur jusqu'au soir.
\VS{47}Celui qui dormira dans cette maison lavera ses vêtements. Celui aussi qui mangera dans cette maison lavera ses vêtements.
\VS{48}Mais quand le prêtre y sera entré, et qu'il aura aperçu que la plaie ne s'est point étendue dans cette maison, après l'avoir recrépie, il jugera la maison pure, car sa plaie est guérie.
\VS{49}Alors il prendra pour purifier la maison deux oiseaux, du bois de cèdre, du cramoisi et de l'hysope.
\VS{50}Il égorgera l'un des oiseaux sur un vase de terre, sur de l'eau vive.
\VS{51}Il prendra le bois de cèdre, l'hysope, le cramoisi et l'oiseau vivant~; il trempera le tout dans le sang de l'oiseau qu'on aura égorgé et dans l'eau vive, puis il fera sept fois l'aspersion sur la maison.
\VS{52}Il purifiera la maison avec le sang de l'oiseau, avec l'eau vive, avec l'oiseau vivant, le bois de cèdre, l'hysope et le cramoisi.
\VS{53}Puis il laissera aller hors de la ville par les champs l'oiseau vivant. C'est ainsi qu'il fera propitiation pour la maison, et elle sera pure.
\VS{54}Telle est la loi pour toute plaie de lèpre et de teigne,
\VS{55}de lèpre de vêtement et de maison,
\VS{56}de tumeur, de dartre, et de tache~;
\VS{57}pour enseigner quand une chose est impure et quand elle est pure. Telle est la loi sur la lèpre.
\Chap{15}
\TextTitle{Lois de purification~: Gonorrhée et flux menstruel\FTNTT{Jn. 13:3-10~; Ep. 5:25-27~; 1 Jn. 1:9.}}
\VerseOne{}Yahweh parla aussi à Moïse et à Aaron, en disant~:
\VS{2}Parlez aux enfants d'Israël et dites-leur~: Tout homme qui a une gonorrhée\FTNT{Gonorrhée~: Infection des organes génito-urinaires.} sera impur à cause de son flux.
\VS{3}Et telle sera l'impureté de son flux~: Quand sa chair laissera aller son flux, ou que sa chair retiendra son flux, c'est son impureté.
\VS{4}Tout lit sur lequel se couchera celui qui est atteint d'un flux sera impur~; et toute chose sur laquelle il se sera assis sera impure.
\VS{5}L'homme aussi qui touchera son lit lavera ses vêtements et se lavera avec de l'eau~; et il sera impur jusqu'au soir.
\VS{6}Et celui qui s'assiéra sur quelque chose sur laquelle celui qui a ce flux s'est assis, lavera ses vêtements et se lavera dans l'eau, et il sera impur jusqu'au soir.
\VS{7}Et celui qui touchera la chair de celui qui a ce flux lavera ses vêtements et se lavera dans l'eau, et il sera impur jusqu'au soir.
\VS{8}Si celui qui a ce flux crache sur celui qui est pur, celui qui était pur lavera ses vêtements et se lavera dans l'eau, et il sera impur jusqu'au soir.
\VS{9}Toute monture que celui qui a ce flux aura montée sera impure.
\VS{10}Quiconque touchera quelque chose qui aura été sous lui sera impur jusqu'au soir~; et quiconque portera une telle chose lavera ses vêtements, et se lavera dans l'eau~; il sera impur jusqu'au soir.
\VS{11}Quiconque aura été touché par celui qui a ce flux, sans qu'il ait lavé ses mains dans l'eau, lavera ses vêtements et il se lavera dans l'eau, et il sera impur jusqu'au soir.
\VS{12}Et le vase de terre que celui qui a ce flux aura touché sera cassé, mais tout vase de bois sera lavé dans l'eau.
\VS{13}Or quand celui qui a ce flux sera purifié de son flux, il comptera sept jours pour sa purification~; il lavera ses vêtements et sa chair avec de l'eau vive, et ainsi il sera pur.
\VS{14}Au huitième jour, il prendra pour lui deux tourterelles ou deux jeunes pigeons, et il viendra devant Yahweh à l'entrée de la tente d'assignation, et les donnera au prêtre.
\VS{15}Et le prêtre les sacrifiera, l'un en sacrifice pour l'expiation et l'autre en holocauste~; ainsi le prêtre fera propitiation pour lui devant Yahweh à cause de son flux.
\VS{16}L'homme aussi dont sera sortie de la semence lavera dans l'eau tout son corps, et il sera impur jusqu'au soir.
\VS{17}Et tout vêtement, et toute peau, sur lequels il y aura de la semence seront lavés dans l'eau, et seront impurs jusqu'au soir.
\VS{18}Même la femme qui couchera avec un tel homme se lavera dans l'eau avec son mari, et ils seront impurs jusqu'au soir.
\VS{19}Et quand la femme aura un flux, un flux de sang en sa chair, elle sera séparée sept jours. Quiconque la touchera sera impur jusqu'au soir\FTNT{Mt. 9:18-22~; Mc. 5:21-34~; Lu. 8:41-48.}.
\VS{20}Toute chose sur laquelle elle aura couché durant sa séparation sera impure, toute chose aussi sur laquelle elle aura été assise sera impure.
\VS{21}Quiconque aussi touchera le lit de cette femme lavera ses vêtements et se lavera dans l'eau, et il sera impur jusqu'au soir.
\VS{22}Et quiconque touchera quelque chose sur laquelle elle se sera assise lavera ses vêtements et se lavera dans l'eau, et il sera impur jusqu'au soir.
\VS{23}Même si la chose que quelqu'un aura touchée était sur le lit ou sur quelque chose sur laquelle elle était assise, quand quelqu'un aura touché cette chose-là, il sera impur jusqu'au soir.
\VS{24}Et si un homme a couché avec elle et que son impureté soit sur lui, il sera impur sept jours, et toute couche sur laquelle il dormira sera impure.
\VS{25}La femme qui aura un flux de sang pendant plusieurs jours, hors de l'époque de ses menstruations, ou dont le flux durera plus longtemps que l'époque de ses menstruations, sera impure tout le temps du flux de son impureté, comme au temps de sa séparation.
\VS{26}Toute couche sur laquelle elle couchera tous les jours de son flux lui sera comme la couche de sa séparation, et toute chose sur laquelle elle s'assiéra sera impure comme pour l'impureté de sa séparation.
\VS{27}Et quiconque aura touché ces choses-là sera impur~; il lavera ses vêtements et se lavera dans l'eau, et il sera impur jusqu'au soir.
\VS{28}Mais si elle est purifiée de son flux, elle comptera sept jours, et après elle sera pure.
\VS{29}Au huitième jour, elle prendra deux tourterelles ou deux jeunes pigeons, et les apportera au prêtre à l'entrée de la tente d'assignation.
\VS{30}Et le prêtre en sacrifiera l'un en sacrifice pour l'expiation et l'autre en holocauste~; ainsi le prêtre fera propitiation pour elle devant Yahweh, à cause du flux de son impureté.
\VS{31}Ainsi, vous séparerez les enfants d'Israël de leurs impuretés, et ils ne mourront point à cause de leurs impuretés, en rendant impur mon tabernacle, qui est au milieu d'eux.
\VS{32}Telle est la loi pour celui qui a une gonorrhée ou de celui dont sort la semence qui le rend impur.
\VS{33}Telle est aussi la loi pour celle qui a son indisposition menstruelle ou de toute personne qui découle et qui a son flux, soit mâle, soit femelle, et de celui qui couche avec celle qui est impure.
\Chap{16}
\TextTitle{Expiation pour le prêtre, sa maison et le peuple\FTNTT{Hé. 9:1-14.}}
\VerseOne{}Or Yahweh parla à Moïse après la mort des deux fils d'Aaron, qui moururent lorsqu'ils s'étaient approchés de la présence de Yahweh.
\VS{2}Yahweh donc dit à Moïse~: Parle à Aaron, ton frère, et dis-lui qu'il n'entre point en tout temps dans le sanctuaire, au-dedans du voile, devant le propitiatoire qui est sur l'arche, afin qu'il ne meure point~; car j'apparaîtrai dans une nuée sur le propitiatoire.
\VS{3}Aaron entrera dans le sanctuaire de cette manière, après avoir offert un jeune taureau du troupeau pour le péché, et un bélier pour l'holocauste.
\VS{4}Il se revêtira de la sainte tunique de lin, et portera les caleçons de lin sur son corps~; il se ceindra de la ceinture de lin\FTNT{La ceinture de vérité (Ep. 6:14).}, et se couvrira la tête de la tiare\FTNT{La tiare, le casque du salut (Ep. 6:17).} de lin, qui sont les saints vêtements, et il s'en vêtira après avoir lavé son corps avec de l'eau\FTNT{Le lavement préfigure ici la régénération (Tit. 3:5).}.
\VS{5}Et il prendra de l'assemblée des enfants d'Israël deux jeunes boucs en offrande pour le péché et un bélier pour l'holocauste.
\VS{6}Puis Aaron offrira son veau en sacrifice pour l'expiation, et fera propitiation tant pour lui que pour sa maison.
\TextTitle{Les deux boucs expiatoires\FTNTT{2 Co. 5:21.}}
\VS{7}Et il prendra les deux boucs, et les présentera devant Yahweh, à l'entrée de la tente d'assignation.
\VS{8}Puis Aaron jettera le sort sur les deux boucs, un sort pour Yahweh et un sort pour le bouc qui doit être Azazel.
\VS{9}Et Aaron offrira le bouc sur lequel le sort sera échu pour Yahweh, et l'offrira en sacrifice pour l'expiation.
\VS{10}Mais le bouc sur lequel le sort sera tombé pour être Azazel, sera présenté vivant devant Yahweh pour faire propitiation par lui, et on l'enverra dans le désert pour être Azazel.
\VS{11}Aaron donc, présentera le veau en sacrifice pour l'expiation, et fera propitiation pour lui et pour sa maison. Et il égorgera, dis-je, son veau qui est le sacrifice pour l'expiation.
\VS{12}Puis il prendra un encensoir plein de charbons ardents, de dessus l'autel devant Yahweh, et deux poignées de parfum odoriférant en poudre~; et il les apportera au-dedans du voile~;
\VS{13}et il mettra le parfum sur le feu devant Yahweh, afin que la nuée du parfum couvre le propitiatoire qui est sur le témoignage, ainsi il ne mourra point.
\VS{14}Il prendra aussi du sang du veau, et il en fera l'aspersion avec son doigt au-devant du propitiatoire vers l'orient~; il fera l'aspersion de ce sang-là sept fois avec son doigt devant le propitiatoire.
\VS{15}Il égorgera aussi le bouc du peuple, qui est l'offrande pour l'expiation, et il apportera son sang au-dedans du voile. Il fera de son sang comme il a fait du sang du veau, en faisant l'aspersion sur le propitiatoire et sur le devant du propitiatoire.
\VS{16}Et il fera propitiation pour le sanctuaire, le purifiant des impuretés des enfants d'Israël, et de leurs transgressions, selon tous leurs péchés. Il fera la même chose pour la tente d'assignation, qui demeure avec eux au milieu de leurs impuretés.
\VS{17}Et personne ne sera dans la tente d'assignation quand le prêtre y entrera pour faire propitiation dans le sanctuaire, jusqu'à ce qu'il en sorte, lorsqu'il fera propitiation pour lui et pour sa maison, et pour toute l'assemblée d'Israël.
\VS{18}Puis il sortira vers l'autel qui est devant Yahweh, et fera propitiation pour lui~; il prendra du sang du veau et du sang du bouc, il le mettra sur les cornes de l'autel tout autour.
\VS{19}Et il fera par sept fois l'aspersion du sang avec son doigt sur l'autel, et le purifiera et le sanctifiera des impuretés des enfants d'Israël.
\VS{20}Et quand il achèvera de faire propitiation pour le sanctuaire, pour la tente d'assignation et pour l'autel, alors il offrira le bouc vivant.
\VS{21}Et Aaron posera ses deux mains sur la tête du bouc vivant, et il confessera sur lui toutes les iniquités des enfants d'Israël et toutes leurs transgressions, selon tous leurs péchés~; et il les mettra sur la tête du bouc, et l'enverra au désert par un homme prêt pour cela.
\VS{22}Et le bouc portera sur lui toutes leurs iniquités dans une terre inhabitable, puis cet homme laissera aller le bouc par le désert.
\VS{23}Et Aaron reviendra dans la tente d'assignation~; il quittera les vêtements de lin dont il s'était vêtu quand il était entré dans le sanctuaire, et les posera là.
\VS{24}Il lavera aussi son corps avec de l'eau dans le lieu saint, et se revêtira de ses vêtements. Puis il sortira, il offrira son holocauste et l'holocauste du peuple, et fera propitiation pour lui et pour le peuple.
\VS{25}Il brûlera aussi sur l'autel la graisse de l'offrande pour le péché.
\VS{26}Et celui qui aura conduit le bouc pour être Azazel lavera ses vêtements et son corps avec de l'eau~; après cela, il rentrera dans le camp.
\VS{27}Mais on tirera hors du camp le veau et le bouc qui auront été offerts en sacrifice pour l'expiation, et desquels le sang aura été porté dans le sanctuaire pour y faire propitiation, et on brûlera au feu leurs peaux, leur chair et leurs excréments\FTNT{Hé. 13:11.}.
\VS{28}Et celui qui les aura brûlés lavera ses vêtements et son corps avec de l'eau~; après cela, il rentrera dans le camp.
\VS{29}Et ceci sera pour vous une ordonnance perpétuelle~: Le dixième jour du septième mois, vous affligerez vos âmes, et vous ne ferez aucune œuvre, tant celui qui est du pays que l'étranger qui fait son séjour parmi vous\FTNT{La fête des expiations (ou yom kippour) avait lieu une fois par an, le dixième jour du septième mois (Ex. 30:10~; Lé. 16:29). A cette occasion, le grand-prêtre jetait le sort sur deux boucs~: Un sort pour Yahweh et un sort pour Azazel (Lé. 16:8-10). Le bouc pour Yahweh était sacrifié, il préfigurait la mort expiatoire de Christ. Le bouc émissaire, pour Azazel, n'avait lui-même rien fait de mal, mais il était choisi par Dieu pour porter le péché du peuple afin qu'il soit dégagé de toute accusation. Ce que l'on faisait de ce bouc, préfigurait l'œuvre de Jésus-Christ. Il symbolisait le Seigneur qui s'est chargé de nos péchés pour les emporter loin de nous (Es. 53~; Ps. 103:12~; Hé. 10:17~; Hé. 13:12-14). Christ est mort et ressuscité hors du camp et c'est là qu'il nous appelle à le rejoindre~: Hors du monde et des systèmes religieux (Hé. 13:10-14).}.
\VS{30}Car en ce jour-là le prêtre fera propitiation pour vous, afin de vous purifier. Ainsi vous serez purifiés de tous vos péchés devant Yahweh.
\VS{31}Ce sera pour vous donc un sabbat, un jour de repos, et vous affligerez vos âmes. C'est une ordonnance perpétuelle.
\VS{32}Et le prêtre qu'on aura oint, et qu'on aura consacré pour exercer la prêtrise à la place de son père, fera propitiation, s'étant revêtu des vêtements de lin, qui sont les saints vêtements.
\VS{33}Et il fera propitiation pour le saint sanctuaire et il fera propitiation pour la tente d'assignation et pour l'autel, et pour les prêtres et pour tout le peuple de l'assemblée.
\VS{34}Ceci donc sera pour vous une ordonnance perpétuelle, afin de faire propitiation pour les enfants d'Israël de tous leurs péchés une fois par an. On fit comme Yahweh l'avait ordonné à Moïse.
\Chap{17}
\TextTitle{Les sacrifices apportés à l'entrée de la tente d'assignation}
\VerseOne{}Yahweh parla aussi à Moïse, en disant~:
\VS{2}Parle à Aaron et à ses fils, et à tous les enfants d'Israël, et dis-leur~: C'est ici ce que Yahweh a ordonné, en disant~:
\VS{3}Quiconque de la maison d'Israël aura égorgé un bœuf, un agneau ou une chèvre dans le camp, ou qui l'aura égorgé hors du camp\FTNT{De. 12:6.},
\VS{4}et ne l'aura point amené à l'entrée de la tente d'assignation, pour en faire une offrande à Yahweh, devant le tabernacle de Yahweh, le sang sera imputé à cet homme-là~; il a répandu du sang, c'est pourquoi cet homme-là sera retranché du milieu de son peuple.
\VS{5}C'est afin que les enfants d'Israël amènent leurs sacrifices, qu'ils sacrifient dans les champs, qu'ils les amènent dis-je à Yahweh, à l'entrée de la tente d'assignation, vers le prêtre, et qu'ils les sacrifient en sacrifices d'offrande de paix\FTNT{Voir commentaire en Lé. 3:1.} à Yahweh~;
\VS{6}et que le prêtre en répande le sang sur l'autel de Yahweh, à l'entrée de la tente d'assignation, et en brûle la graisse en bonne odeur à Yahweh~;
\VS{7}et qu'ils n'offrent plus leurs sacrifices aux démons\FTNT{Le mot hébreu traduit par «~démon~»  est «~sa`iyr~» qui signifie «~velu, poilu, mâle de la chèvre, bouc ; comme animal de sacrifice ; satyre~». Dans la mythologie grecque, les satyres  étaient des créatures lubriques, couvertes de poils, mi-homme, mi bouc, faisant partie du cortège de Dionysos, dieu du vin et des excès (Bacchus chez les Romains). Souvent sujets de représentations ithyphalliques, les œuvres d’art les montrent souvent en train de pourchasser de leurs assiduités les nymphes et les ménades. Aujourd’hui, ce terme sert à désigner les exhibitionnistes et ceux qui commettent des attentats à la pudeur sur la voie publique.}, avec lesquels ils se sont prostitués. Ceci leur sera une ordonnance perpétuelle pour eux et leurs descendants\FTNT{De. 32:17~; Ps. 106:37.}.
\VS{8}Tu leur diras donc~: Si un homme de la maison d'Israël, ou des étrangers qui font leur séjour parmi eux, aura offert un holocauste ou un sacrifice,
\VS{9}et qui ne l'aura point amené à l'entrée de la tente d'assignation, pour le sacrifier à Yahweh, cet homme-là sera retranché d'entre ses peuples.
\TextTitle{Importance du sang}
\VS{10}Quiconque de la maison d'Israël ou des étrangers qui font leur séjour parmi eux aura mangé de quelque sang que ce soit, je mettrai ma face contre cette personne qui aura mangé du sang, et je la retrancherai du milieu de son peuple\FTNT{Ge. 9:4~; De. 12:16-23~; 1 S. 14:33.}.
\VS{11}Car l'âme de la chair est dans le sang. C'est pourquoi je vous ai ordonné qu'il soit mis sur l'autel, afin de faire propitiation pour vos âmes, car c'est le sang qui fera propitiation pour l'âme.
\VS{12}C'est pourquoi j'ai dit aux enfants d'Israël~: Que personne d'entre vous ne mange du sang, que même l'étranger qui fait son séjour parmi vous ne mange pas de sang.
\VS{13}Et quiconque des enfants d'Israël, et des étrangers qui font leur séjour parmi eux, aura pris à la chasse une bête sauvage ou un oiseau que l'on mange, il répandra leur sang et le couvrira de poussière.
\VS{14}Car l'âme de toute chair est dans son sang, c'est son âme. C'est pourquoi j'ai dit aux enfants d'Israël~: Vous ne mangerez le sang d'aucune chair~; car l'âme de toute chair est son sang~: Quiconque en mangera sera retranché.
\VS{15}Et toute personne qui aura mangé de la chair de quelque bête morte d'elle-même ou déchirée par les bêtes sauvages, tant celui qui est né dans le pays que l'étranger, lavera ses vêtements et se lavera avec de l'eau, et il sera impur jusqu'au soir~; puis il sera pur.
\VS{16}S'il ne lave pas ses vêtements et son corps, il portera son iniquité.
\Chap{18}
\TextTitle{Condamnation de l'inceste}
\VerseOne{}Yahweh parla encore à Moïse, en disant~:
\VS{2}Parle aux enfants d'Israël et dis-leur~: Je suis Yahweh, votre Dieu.
\VS{3}Vous ne ferez point ce qui se fait dans le pays d'Egypte où vous avez habité, ni ce qui se fait dans le pays de Canaan, où je vous amène. Vous ne vivrez point selon leurs statuts\FTNT{Jé. 10:2.}.
\VS{4}Vous pratiquerez mes ordonnances, et vous garderez mes lois pour les suivre. Je suis Yahweh, votre Dieu.
\VS{5}Vous garderez donc mes statuts et mes ordonnances, l'homme qui les pratiquera vivra par elles. Je suis Yahweh\FTNT{Ez. 20:11-13~; Ga. 3:12~; Ro. 10:5.}.
\VS{6}Que nul ne s'approche de celle qui est sa proche parente pour découvrir sa nudité. Je suis Yahweh.
\VS{7}Tu ne découvriras point la nudité de ton père, ni la nudité de ta mère. C'est ta mère~; tu ne découvriras point sa nudité.
\VS{8}Tu ne découvriras point la nudité de la femme de ton père. C'est la nudité de ton père\FTNT{De. 22:30~; 1 Co. 5:1.}.
\VS{9}Tu ne découvriras point la nudité de ta sœur, fille de ton père ou fille de ta mère, née dans la maison ou hors de la maison. Tu ne découvriras point leur nudité.
\VS{10}Quant à la nudité de la fille de ton fils ou de la fille de ta fille, tu ne découvriras point leur nudité. Car elles sont ta nudité.
\VS{11}Tu ne découvriras point la nudité de la fille de la femme de ton père, née de ton père. C'est ta sœur.
\VS{12}Tu ne découvriras point la nudité de la sœur de ton père. Elle est la proche parente de ton père.
\VS{13}Tu ne découvriras point la nudité de la sœur de ta mère~; car elle est la proche parente de ta mère.
\VS{14}Tu ne découvriras point la nudité du frère de ton père. Et tu ne t'approcheras point de sa femme. Elle est ta tante.
\VS{15}Tu ne découvriras point la nudité de ta belle-fille. Elle est la femme de ton fils~; tu ne découvriras point sa nudité.
\VS{16}Tu ne découvriras point la nudité de la femme de ton frère. C'est la nudité de ton frère.
\VS{17}Tu ne découvriras point la nudité d'une femme et de sa fille. Et tu ne prendras point la fille de son fils, ni la fille de sa fille pour découvrir leur nudité. Elles sont tes proches parentes~: C'est un crime.
\VS{18}Tu ne prendras point aussi une femme avec sa sœur pour exciter une rivalité en découvrant sa nudité à côté d'elle pendant sa vie.
\TextTitle{Condamnation des abominations}
\VS{19}Tu ne t'approcheras point d'une femme durant son impureté menstruelle, pour découvrir sa nudité.
\VS{20}Tu ne coucheras point avec la femme de ton prochain pour te souiller avec elle\FTNT{Ex. 20:17~; De. 5:21~; Mt. 5:28.}.
\VS{21}Tu ne donneras point tes enfants pour les faire passer par le feu devant Moloc\FTNT{Moloc est le nom du dieu auquel les Ammonites, peuple issu de la relation incestueuse de Loth et sa fille, sacrifiaient leurs premiers-nés en les jetant dans un brasier. De. 18:9-10~; 1 R. 11:5-7~; 2 R. 23:10~; Jé. 32:35.}, et tu ne profaneras point le Nom de ton Dieu. Je suis Yahweh.
\VS{22}Tu ne coucheras pas aussi avec un homme, comme on couche avec une femme. C'est une abomination\FTNT{1 Co. 6:9-10~; Ge. 13:13~; Ro. 1:26-27.}.
\VS{23}Tu ne coucheras point aussi avec une bête pour te souiller avec elle~; et la femme ne se prostituera point à une bête~; c'est une confusion\FTNT{1 Co. 6:9-10~; Ro. 1:26-27.}.
\VS{24}Ne vous rendez point impurs par aucune de ces choses, car les nations que je vais chasser de devant vous se sont rendues impures par toutes ces choses.
\VS{25}Le pays a été rendu impur~; et je punirai sur lui son iniquité, et le pays vomira ses habitants.
\VS{26}Mais quant à vous, vous garderez mes ordonnances et mes jugements, et vous ne ferez aucune de ces abominations, tant celui qui est né dans le pays que l'étranger qui fait son séjour parmi vous.
\VS{27}Car les gens de ce pays-là qui ont été avant vous, ont fait toutes ces abominations, et le pays en a été rendu impur.
\VS{28}Prenez garde que le pays ne vous vomisse, si vous le rendez impur, comme il aura vomi les nations qui y étaient avant vous.
\VS{29}Car tous ceux qui feront l'une de toutes ces abominations, seront retranchés du milieu de leur peuple.
\VS{30}Vous garderez donc ce que j'ai ordonné de garder, et vous ne pratiquerez aucune de ces coutumes abominables qui ont été pratiquées avant vous, et vous ne vous rendrez point impurs par elles. Je suis Yahweh, votre Dieu.
\Chap{19}
\TextTitle{Mise en garde contre l'idolâtrie}
\VerseOne{}Yahweh parla aussi à Moïse, en disant~:
\VS{2}Parle à toute l'assemblée des enfants d'Israël, et dis-leur~: Soyez saints, car je suis saint, moi, Yahweh, votre Dieu.
\VS{3}Chacun de vous craindra sa mère et son père, et vous garderez mes sabbats. Je suis Yahweh, votre Dieu\FTNT{Ex. 20:12~; De. 5:16~; Mt. 15:4.}.
\VS{4}Vous ne vous tournerez point vers les idoles, et vous ne vous ferez aucun dieu de fonte. Je suis Yahweh, votre Dieu\FTNT{Ex. 20:3-5.}.
\TextTitle{Recommandation pour les sacrifices}
\VS{5}Si vous offrez un sacrifice d'offrande de paix\FTNT{Voir commentaire en Lé. 3:1.} à Yahweh, vous le sacrifierez de votre bon gré.
\VS{6}II se mangera le jour où vous l'aurez sacrifié, et le lendemain, mais ce qui restera jusqu'au troisième jour sera brûlé au feu.
\VS{7}Si on en mange au troisième jour, ce sera une abomination. Il ne sera point agréé.
\VS{8}Quiconque aussi en mangera portera son iniquité~; car il aura profané la chose sainte de Yahweh. Cette personne-là sera retranchée d'entre ses peuples.
\TextTitle{La justice de Yahweh, l'amour pour son prochain}
\VS{9}Quand vous ferez la moisson de votre pays, tu n'achèveras point de moissonner le bout de ton champ, et tu ne glaneras point ce qui restera à cueillir de ta moisson.
\VS{10}Tu ne grappilleras point ta vigne, ni ne recueilleras point les grains tombés de ta vigne, mais tu les laisseras au pauvre et à l'étranger\FTNT{De. 24:19.}. Je suis Yahweh, votre Dieu.
\VS{11}Vous ne déroberez point, et vous ne vous tromperez point les uns les autres~; et aucun de vous ne mentira à son prochain\FTNT{Ex. 20:15~; Ep. 4:25~; Col. 3:9.}.
\VS{12}Vous ne jurerez point par mon Nom en mentant, car tu profanerais le Nom de ton Dieu\FTNT{Ex. 20:7~; De. 5:11.}. Je suis Yahweh.
\VS{13}Tu n'opprimeras point ton prochain, et tu ne le pilleras point\FTNT{De. 24:14-15~; Ja. 5:4.}. Le salaire de ton mercenaire ne demeurera point chez toi jusqu'au lendemain.
\VS{14}Tu ne maudiras point le sourd, et tu ne mettras point d'achoppement devant l'aveugle, mais tu craindras ton Dieu. Je suis Yahweh.
\VS{15}Vous ne ferez point d'iniquité dans vos jugements. Tu n'auras point d'égard à la personne du pauvre, et tu n'honoreras point la personne du grand, mais tu jugeras ton prochain selon la justice.
\VS{16}Tu ne répandras point de calomnies parmi ton peuple. Tu ne t'élèveras point contre le sang de ton prochain. Je suis Yahweh.
\VS{17}Tu ne haïras point ton frère dans ton cœur. Tu reprendras soigneusement ton prochain\FTNT{Ge. 4:8~; Mt. 18:15~; 1 Jn. 2:9-11.}, et tu ne te chargeras point d'un péché à cause de lui.
\VS{18}Tu n'useras point de vengeance, et tu ne la garderas point aux enfants de ton peuple~; mais tu aimeras ton prochain comme toi-même\FTNT{Mt. 7:12~; Mc. 12:28-34.}. Je suis Yahweh.
\VS{19}Vous garderez mes ordonnances. Tu n'accoupleras point tes bêtes de deux espèces différentes~; tu ne sèmeras point ton champ de diverses sortes de grains~; et tu ne mettras point sur toi de vêtements de diverses espèces, comme de la laine et du lin.
\VS{20}Si un homme couche et a commerce avec une femme, si c'est une esclave, fiancée à un homme, qui n'a pas été rachetée, et que la liberté ne lui a pas été donnée, ils auront le fouet, mais on ne les fera point mourir, parce qu'elle n'a pas été affranchie.
\VS{21}L'homme amènera son sacrifice pour la culpabilité à Yahweh à l'entrée de la tente d'assignation, à savoir un bélier pour la culpabilité.
\VS{22}Et le prêtre fera propitiation pour lui devant Yahweh par le bélier du sacrifice pour la culpabilité, à cause de son péché qu'il aura commis, et son péché qu'il aura commis lui sera pardonné.
\TextTitle{Ordonnances diverses}
\VS{23}Et quand vous serez entrés dans le pays, et que vous y aurez planté quelque arbre fruitier, vous considérerez son fruit comme incirconcis~; il vous sera incirconcis pendant trois ans, on n'en mangera point.
\VS{24}Mais à la quatrième année, tout son fruit sera une chose sainte à la louange de Yahweh.
\VS{25}Et à la cinquième année, vous mangerez son fruit, afin qu'il vous multiplie son produit. Je suis Yahweh, votre Dieu.
\VS{26}Vous ne mangerez rien avec le sang. Vous n'userez point de divinations, et vous ne pronostiquerez point le temps\FTNT{De. 12:23.}.
\VS{27}Vous ne couperez point en rond les coins de votre chevelure, et vous ne raserez point les coins de votre barbe.
\VS{28}Vous ne ferez point d'incisions dans votre chair pour un mort, et vous n'imprimerez point de caractères sur vous. Je suis Yahweh.
\VS{29}Tu ne profaneras point ta fille en la prostituant~; afin que le pays ne se prostitue point et ne se remplisse point de crimes.
\VS{30}Vous garderez mes sabbats et vous aurez en révérence mon sanctuaire. Je suis Yahweh.
\VS{31}Ne vous tournez point vers ceux qui évoquent les morts, ni vers les devins\FTNT{Ac. 16:16.}~; ne cherchez point à vous rendre impurs avec eux. Je suis Yahweh, votre Dieu.
\VS{32}Lève-toi devant les cheveux blancs, et tu honoreras la personne du vieillard. Tu craindras ton Dieu. Je suis Yahweh.
\VS{33}Si quelque étranger séjourne dans votre pays, vous ne lui ferez point de tort.
\VS{34}L'étranger qui séjourne parmi vous sera pour vous comme celui qui est né parmi vous, et vous l'aimerez comme vous-mêmes, car vous avez été étrangers dans le pays d'Egypte. Je suis Yahweh, votre Dieu.
\VS{35}Vous ne ferez point d'iniquité dans les jugements, ni dans les mesures de dimension, ni dans les poids, ni dans les mesures de capacité.
\VS{36}Vous aurez des balances justes, des poids justes, un épha juste et un hin juste. Je suis Yahweh, votre Dieu, qui vous ai fait sortir du pays d'Egypte.
\VS{37}Gardez donc toutes mes ordonnances et mes jugements, et pratiquez-les. Je suis Yahweh.
\Chap{20}
\TextTitle{Abominations diverses et leurs châtiments}
\VerseOne{}Yahweh parla aussi à Moïse, en disant~:
\VS{2}Tu diras aux enfants d'Israël~: Quiconque des enfants d'Israël ou des étrangers qui demeurent en Israël, qui donnera l'un de ses enfants à Moloc, il mourra, il mourra. Le peuple du pays le lapidera.
\VS{3}Et je tournerai ma face contre un tel homme, et je le retrancherai du milieu de son peuple, parce qu'il a donné de ses enfants à Moloc, pour rendre impur mon sanctuaire et profaner le Nom de ma sainteté.
\VS{4}Si le peuple du pays ferme les yeux en quelque manière que ce soit sur cet homme-là, qui donne de ses enfants à Moloc, et s'il ne le fait pas mourir,
\VS{5}je tournerai ma face contre cet homme-là, contre sa famille, et je le retrancherai du milieu de mon peuple, avec tous ceux qui se prostituent comme lui, en se prostituant à Moloc.
\VS{6}Quant à la personne qui se tournera vers ceux qui évoquent les morts, vers les devins, en se prostituant après eux, je tournerai ma face contre cette personne-là, et je la retrancherai du milieu de son peuple.
\VS{7}Sanctifiez-vous donc, et soyez saints, car je suis Yahweh, votre Dieu.
\VS{8}Gardez aussi mes lois et pratiquez-les. Je suis Yahweh, qui vous sanctifie.
\VS{9}Un homme qui maudit son père ou sa mère, il mourra, il mourra~; il a maudit son père ou sa mère. Son sang retombera sur lui.
\VS{10}Quant à l'homme qui commet un adultère avec la femme d'un autre, parce qu'il a commis un adultère avec la femme de son prochain, l'homme et la femme adultères mourront, ils mourront.
\VS{11}L'homme qui couche avec la femme de son père, découvre la nudité de son père, les deux seront mis à mort, leur sang est sur eux.
\VS{12}Quand un homme couche avec sa belle-fille, ils mouront, ils mourront, tous deux~; ils ont fait une confusion. Leur sang est sur eux.
\VS{13}Quand un homme couche avec un homme comme on couche avec une femme, ils ont tous deux fait une chose abominable~; ils mourront, ils mourront. Leur sang est sur eux.
\VS{14}Et si un homme prend pour femmes la fille et la mère, c'est un crime. Il sera brûlé au feu avec elles, afin que ce crime n'existe pas au milieu de vous.
\VS{15}Si un homme couche avec une bête, il sera puni de mort~; et vous tuerez aussi la bête.
\VS{16}Et si une femme s'approche d'une bête, tu tueras cette femme et la bête~; ils mourront, ils mourront. Leur sang sera sur eux.
\VS{17}Si un homme prend sa sœur, fille de son père ou fille de sa mère, et voit sa nudité, et qu'elle voit la nudité de cet homme, c'est une chose infâme~; ils seront donc retranchés sous les yeux des fils de leur peuple. Il a découvert la nudité de sa sœur, il portera son iniquité.
\VS{18}Si un homme couche avec une femme qui a son indisposition menstruelle, et qu'il découvre la nudité de cette femme, en découvrant son flux, et qu'elle découvre le flux de son sang, ils seront tous deux retranchés du milieu de leur peuple.
\VS{19}Tu ne découvriras point la nudité de la sœur de ta mère, ni de la sœur de ton père, car c'est découvrir sa proche parente, ils porteront tous deux leur iniquité.
\VS{20}Si un homme couche avec sa tante, il a découvert la nudité de son oncle~; ils porteront leur péché, et ils mourront privés d'enfants.
\VS{21}Si un homme prend la femme de son frère, c'est une impureté~; il a découvert la nudité de son frère, ils seront privés d'enfants.
\VS{22}Vous garderez toutes mes ordonnances et mes jugements et vous les pratiquerez, afin que le pays où je vous fais entrer pour y habiter ne vous vomisse point.
\VS{23}Vous ne suivrez point les statuts des nations que je vais chasser devant vous~; car elles ont fait toutes ces choses-là, et je les ai eues en abomination.
\VS{24}Et je vous ai dit~: Vous posséderez leur pays, je vous le donnerai en possession. C'est un pays où coulent le lait et le miel. Je suis Yahweh, votre Dieu, qui vous ai séparés des autres peuples.
\VS{25}C'est pourquoi séparez les bêtes pures de celles qui sont impures, les oiseaux purs de ceux qui sont impurs, et ne rendez point abominables vos personnes en mangeant des bêtes et des oiseaux impurs, ni rien qui rampe sur la terre, rien de ce que je vous ai défendu comme une chose impure.
\VS{26}Vous me serez donc saints, car je suis saint, moi, Yahweh~; je vous ai séparés des autres peuples afin que vous soyez à moi.
\VS{27}Si un homme ou une femme évoquent les morts ou se livrent à la divination, ils mourront, ils mourront~; on les lapidera. Leur sang sera sur eux.
\Chap{21}
\TextTitle{Recommandations aux prêtres}
\VerseOne{}Yahweh dit aussi à Moïse~: Parle aux prêtres, fils d'Aaron, et dis-leur~: Aucun d'eux ne se rendra impur parmi son peuple pour un mort,
\VS{2}excepté pour son proche parent, pour sa mère, pour son père, pour son fils, pour sa fille, et pour son frère,
\VS{3}et aussi pour sa sœur vierge, qui lui est proche, et qui n'aura point eu de mari, il se rendra impur pour elle.
\VS{4}Chef parmi son peuple, il ne se rendra point impur en se profanant.
\VS{5}Ils ne se feront point de place chauve sur la tête, ils ne raseront point les coins de leur barbe, ni ne feront d'incisions dans leur chair.
\VS{6}Ils seront consacrés à leur Dieu, et ils ne profaneront point le Nom de leur Dieu~; car ils offrent à Yahweh les sacrifices consumés par le feu, qui sont la nourriture de leur Dieu. C'est pourquoi ils seront très saints.
\VS{7}Ils ne prendront point une femme prostituée ou déshonorée~; ils ne prendront point une femme répudiée par son mari, car ils sont saints pour leur Dieu.
\VS{8}Tu regarderas chacun d'eux comme saint, parce qu'ils offrent la nourriture de ton Dieu~; ils seront saints, car je suis saint, moi, Yahweh, qui vous sanctifie.
\VS{9}Si la fille du prêtre se profane en se prostituant, elle déshonore son père. Qu'elle soit brûlée au feu~!
\VS{10}Le grand-prêtre d'entre ses frères, sur la tête duquel l'huile d'onction a été répandue, et qui se sera consacré pour vêtir les saints vêtements, ne découvrira point sa tête et ne déchirera point ses vêtements.
\VS{11}Il n'ira vers aucune personne morte, il ne se rendra point impur pour son père ni pour sa mère.
\VS{12}Il ne sortira point du sanctuaire, et ne profanera point le sanctuaire de son Dieu~; car l'huile d'onction de son Dieu est une couronne sur lui. Je suis Yahweh.
\VS{13}Il prendra pour femme une vierge.
\VS{14}Il ne prendra point une veuve, ni une répudiée, ni une femme déshonorée ou prostituée~; mais il prendra pour femme une vierge parmi son peuple.
\VS{15}Il ne profanera point sa postérité parmi son peuple~; car je suis Yahweh qui le sanctifie.
\VS{16}Yahweh parla aussi à Moïse, en disant~:
\VS{17}Parle à Aaron, et dis-lui~: Si quelqu'un de ta postérité, parmi tes descendants, a quelque défaut corporel, il ne s'approchera point pour offrir la nourriture de son Dieu.
\VS{18}Car tout homme en qui il y aura un défaut n'en approchera point~; l'homme aveugle, boiteux, ayant le nez camus ou qui aura un membre allongé~;
\VS{19}ou l'homme qui aura une fracture aux pieds ou aux mains~;
\VS{20}ou qui sera bossu ou grêle, qui aura une tache à l'œil, qui aura une gale sèche, une dartre, ou qui aura les testicules écrasés.
\VS{21}Nul homme de la postérité d'Aaron, le prêtre, en qui il y aura un défaut corporel, ne s'approchera pour offrir les offrandes consumées par le feu à Yahweh~; il y a un défaut en lui, il ne s'approchera donc point pour offrir la nourriture de son Dieu.
\VS{22}Il pourra manger la nourriture de son Dieu, des choses très saintes et des choses saintes,
\VS{23}mais il n'entrera point vers le voile, ni ne s'approchera point de l'autel, car il a un défaut corporel, et il ne profanera point mes sanctuaires, car je suis Yahweh, qui les sanctifie.
\VS{24}Moïse parla ainsi à Aaron et à ses fils, et à tous les enfants d'Israël.
\Chap{22}
\TextTitle{Consécration d'Aaron et de ses fils}
\VerseOne{}Puis Yahweh parla à Moïse, en disant~:
\VS{2}Parle à Aaron et à ses fils, afin qu'ils s'abstiennent des choses saintes des enfants d'Israël, et qu'ils ne profanent point le Nom de ma sainteté dans les choses qu'ils me consacrent. Je suis Yahweh.
\VS{3}Dis-leur donc~: Tout homme parmi votre génération et de vos descendants qui, étant impur, s'approchera des choses saintes que les enfants d'Israël auront sanctifiées à Yahweh, cette personne-là sera retranchée de devant moi. Je suis Yahweh.
\VS{4}Tout homme de la postérité d'Aaron, qui aura la lèpre ou une gonorrhée, ne mangera point des choses saintes jusqu'à ce qu'il soit pur. Il en sera de même pour celui qui touchera quelqu'un s'étant rendu impur en touchant un mort, ou celui qui aura une perte séminale,
\VS{5}et celui qui touchera un reptile et qui en aura été souillé, ou un homme atteint d'une impureté quelconque, il en sera rendu impur.
\VS{6}La personne qui touchera ces choses sera rendu impur jusqu'au soir~; il ne mangera point des choses saintes s'il n'a point lavé son corps dans l'eau~;
\VS{7}ensuite il sera pur après le coucher du soleil, et il mangera des choses saintes, car c'est sa nourriture.
\VS{8}Il ne mangera de la chair d'aucune bête morte d'elle-même ou déchirée par les bêtes sauvages, pour se rendre impur par elle. Je suis Yahweh.
\VS{9}Ils garderont ce que j'ai ordonné de garder, et ils ne commettront point de péché au sujet de la nourriture sainte, afin qu'ils ne meurent point, pour l'avoir profanée. Je suis Yahweh, qui les sanctifie.
\VS{10}Aucun étranger ne mangera des choses saintes~; l'étranger logé chez le prêtre et le mercenaire ne mangeront point des choses saintes.
\VS{11}Mais si le prêtre achète une personne avec son argent, elle en mangera, de même pour celui qui sera né dans sa maison~; ils mangeront de sa nourriture.
\VS{12}Si la fille du prêtre est mariée à un homme étranger, elle ne mangera point des choses saintes présentées en offrande par élévation.
\VS{13}Mais si la fille du prêtre est veuve ou répudiée, et si elle n'a point d'enfants, et est retournée dans la maison de son père, comme dans sa jeunesse, elle mangera de la nourriture de son père. Mais aucun étranger n'en mangera.
\VS{14}Si quelqu'un pèche involontairement en mangeant d'une chose sainte, il y ajoutera un cinquième et le donnera au prêtre avec la chose sainte.
\VS{15}Et ils ne profaneront point les choses sanctifiées des enfants d'Israël, qu'ils auront offertes à Yahweh.
\VS{16}Mais on leur fera porter la peine du péché, parce qu'ils auront mangé de leurs choses saintes. Car je suis Yahweh, qui les sanctifie.
\TextTitle{Animaux sans défaut pour les sacrifices\FTNTT{Hé. 9:14.}}
\VS{17}Yahweh parla encore à Moïse, en disant~:
\VS{18}Parle à Aaron, à ses fils, et à tous les enfants d'Israël, et dis-leur~: Quiconque de la maison d'Israël ou des étrangers qui sont en Israël, offrira son offrande, selon tous ses vœux, ou toutes ses offrandes volontaires, qu'on offre en holocauste à Yahweh,
\VS{19}il offrira de son bon gré, un mâle sans défaut, parmi les bœufs, les agneaux ou les chèvres.
\VS{20}Vous n'offrirez aucune chose qui ait un défaut, car elle ne serait point agréée pour vous.
\VS{21}Si un homme offre à Yahweh un sacrifice d'offrande de paix\FTNT{Voir commentaire en Lé. 3:1.} en s'acquittant d'un vœu, ou en faisant une offrande volontaire, soit de gros ou de menu bétail, elle sera sans défaut pour être agréée~; il ne doit y avoir aucun défaut.
\VS{22}Vous n'offrirez point à Yahweh ce qui sera aveugle, estropié, ou mutilé, qui ait un ulcère, une gale sèche ou une dartre~; et vous n'en ferez point sur l'autel un sacrifice consumé par le feu pour Yahweh.
\VS{23}Tu pourras bien faire une offrande volontaire d'un bœuf, ou d'une brebis, ou d'une chèvre ayant quelques membres allongés, ou quelque défaut dans ses membres, mais ils ne seront point agréés pour le vœu.
\VS{24}Vous n'offrirez point à Yahweh, et ne sacrifierez point dans votre pays un animal qui ait les testicules froissés, cassés, arrachés ou taillés.
\VS{25}Vous ne prendrez point de la main de l'étranger aucune de toutes ces choses pour les offrir comme nourriture à votre Dieu~; car la corruption qui est en eux est un défaut en elles. Elles ne seront point agréées pour vous.
\TextTitle{Ordonnances diverses sur les sacrifices}
\VS{26}Yahweh parla encore à Moïse, en disant~:
\VS{27}Quand un veau, un agneau ou une chèvre seront nés, et qu'ils auront été sept jours sous leur mère, depuis le huitième jour et les suivants, ils seront agréables pour l'offrande du sacrifice consumé par le feu à Yahweh.
\VS{28}Vous n'égorgerez point aussi en un même jour la vache, ou la brebis, ou la chèvre avec son petit.
\VS{29}Quand vous offrirez un sacrifice de remerciement à Yahweh, vous le sacrifierez de votre bon gré.
\VS{30}Il sera mangé le jour même~; vous n'en laisserez rien jusqu'au matin. Je suis Yahweh.
\VS{31}Gardez mes commandements et pratiquez-les. Je suis Yahweh.
\VS{32}Ne profanez point le Nom de ma sainteté, car je serai sanctifié au milieu des enfants d'Israël. Je suis Yahweh, qui vous sanctifie,
\VS{33}et qui vous ai fait sortir du pays d'Egypte, pour être votre Dieu. Je suis Yahweh.
\Chap{23}
\TextTitle{Les fêtes de Yahweh}
\VerseOne{}Yahweh parla aussi à Moïse, en disant~:
\VS{2}Parle aux enfants d'Israël et dis-leur~: Les fêtes\FTNT{Les fêtes de Yahweh étaient des jours solennels, c'est-à-dire des temps fixés pour s'approcher de Dieu et présenter des sacrifices (Voir le tableau en annexe «~Les 7 fêtes de Yahweh~» et également le dictionnaire).} solennelles de Yahweh, que vous publierez, seront de saintes convocations. Ce sont ici mes fêtes solennelles.
\VS{3}On travaillera six jours, mais au septième jour, qui est le sabbat, le jour du repos, il y aura une sainte convocation. Vous ne ferez aucune œuvre, car c'est le sabbat à Yahweh, dans toutes vos demeures.
\TextTitle{La Pâque}
\VS{4}Ce sont ici les fêtes solennelles de Yahweh, qui seront de saintes convocations, que vous publierez en leur saison.
\VS{5}Au premier mois, le quatorzième jour du mois, entre les deux soirs, sera la Pâque\FTNT{La pâque était une fête qui commémorait la sortie d'Egypte (Ex. 12:1-14). Elle préfigurait la rédemption en Jésus-Christ, notre Pâque (1 Co. 5:7). Elle était fixée au 14ème jour du mois de Nisan, le premier mois.} à Yahweh.
\TextTitle{La fête des pains sans levain\FTNTT{Ex. 12:18~; 13:6-8~; 1 Co. 11:23-26.}}
\VS{6}Et le quinzième jour de ce même mois sera la fête solennelle des pains sans levain\FTNT{La fête des pains sans levain commençait le 15ème jour du même mois (Nisan) et durait sept jours. Elle annonçait Christ, notre Pain descendu du ciel (Jn. 6:32-35). Seul le Seigneur Jésus a été sans levain, c'est-à-dire sans aucun péché. Le croyant est sauvé à la Pâque de Christ et doit vivre une vie sans péché (la fête des pains sans levain).} à Yahweh~; vous mangerez des pains sans levain pendant sept jours.
\VS{7}Le premier jour, vous aurez une sainte convocation. Vous ne ferez aucune œuvre servile.
\VS{8}Mais vous offrirez à Yahweh pendant sept jours des offrandes consumées par le feu. Et au septième jour, il y aura une sainte convocation. Vous ne ferez aucune œuvre servile.
\TextTitle{La fête des prémices\FTNTT{1 Co. 15:23.}}
\VS{9}Yahweh parla aussi à Moïse, en disant~:
\VS{10}Parle aux enfants d'Israël et dis-leur~: Quand vous serez entrés dans le pays que je vous donne, et que vous en aurez fait la moisson, vous apporterez alors au prêtre une gerbe des premiers fruits\FTNT{La fête des prémices annonce d'abord la résurrection du Seigneur Jésus-Christ, ensuite celle de tous ceux qui lui appartiennent (1 Th. 4:13-18~; 1 Co. 15:23). Elle commençait le premier jour de la semaine suivant le sabbat de la Pâque, au mois de Nisan.} de votre moisson.
\VS{11}Et il agitera cette gerbe-là devant Yahweh, afin qu'elle soit agréée pour vous. Le prêtre l'agitera le lendemain du sabbat.
\VS{12}Et le jour où vous agiterez cette gerbe, vous sacrifierez un agneau sans défaut et d'un an, en holocauste à Yahweh~;
\VS{13}et le gâteau de cet holocauste sera de deux dixièmes de fine farine, pétrie à l'huile, pour offrande consumée par le feu, en bonne odeur à Yahweh~; et sa libation de vin sera le quart d'un hin.
\VS{14}Vous ne mangerez ni pain, ni grain rôti, ni grain en épi, jusqu'à ce jour-là, même jusqu'à ce que vous ayez apporté l'offrande à votre Dieu. C'est une loi perpétuelle pour vos descendants, dans toutes vos demeures.
\TextTitle{La Pentecôte ou la fête des semaines}
\VS{15}Vous compterez aussi dès le lendemain du sabbat, à savoir dès le jour où vous aurez apporté la gerbe qu'on doit agiter, sept semaines entières.
\VS{16}Vous compterez donc cinquante jours\FTNT{La fête des semaines ou fête de la moisson est également désignée comme la Pentecôte. Elle avait lieu au mois de Sivan et préfigurait l'effusion du Saint-Esprit et l'inauguration de la Nouvelle Alliance (Ac. 2:1-4). Le levain autorisé lors de cette fête évoquait par avance la présence de l'ivraie, symbole du péché et des fils du malin, parmi le blé, c'est-à-dire les enfants de Dieu (Mt. 13:24-41). Cinquante jours séparent la Pâque de la Pentecôte. Cet intervalle correspond exactement à la période séparant la résurrection du Seigneur Jésus-Christ de la naissance de l'Eglise (Ac. 2:1-4).} jusqu'au lendemain du septième sabbat~; et vous offrirez à Yahweh un gâteau nouveau.
\VS{17}Vous apporterez de vos demeures deux pains pour en faire une offrande agitée, ils seront de deux dixièmes, et de fine farine, pétris avec du levain. Ce sont les premiers fruits à Yahweh.
\VS{18}Vous offrirez aussi avec ce pain-là sept agneaux sans défaut et d'un an, un jeune taureau pris du troupeau et deux béliers, qui seront un holocauste à Yahweh, avec leurs gâteaux et leurs libations, des sacrifices consumés par le feu, en bonne odeur à Yahweh.
\VS{19}Vous sacrifierez aussi un jeune bouc en sacrifice pour l'expiation, et deux agneaux d'un an pour le sacrifice d'offrande de paix\FTNT{Voir commentaire en Lé. 3:1.}.
\VS{20}Et le prêtre les agitera avec le pain des premiers fruits, et avec les deux agneaux, en offrande agitée devant Yahweh. Ils seront saints à Yahweh, pour le prêtre.
\VS{21}Vous publierez donc, en ce même jour-là, une sainte convocation. Vous ne ferez aucune œuvre servile. C'est une ordonnance perpétuelle dans toutes vos demeures, pour vos descendants.
\VS{22}Et quand vous ferez la moisson de votre pays, tu n'achèveras point de moissonner le bout de ton champ, et tu ne glaneras point les épis qui resteront de ta moisson. Mais tu les laisseras pour le pauvre et pour l'étranger. Je suis Yahweh, votre Dieu.
\TextTitle{La fête des trompettes}
\VS{23}Yahweh parla aussi à Moïse, en disant~:
\VS{24}Parle aux enfants d'Israël et dis-leur~: Au septième mois, le premier jour du mois, il y aura un jour de repos pour vous, un mémorial de jubilation\FTNT{La fête des trompettes préfigure le rassemblement futur du peuple d'Israël après sa longue dispersion et l'enlèvement de l'Eglise. Cette fête était fixée au premier jour du septième mois (Tishri).}, et une sainte convocation.
\VS{25}Vous ne ferez aucune œuvre servile, et vous offrirez à Yahweh des offrandes consumées par le feu.
\TextTitle{Le jour des expiations\FTNTT{Hé. 9:1-16.}}
\VS{26}Yahweh parla aussi à Moïse, en disant~:
\VS{27}Pareillement en ce même mois, qui est le septième, le dixième jour sera le jour des expiations\FTNT{Le jour des expiations ou du grand pardon (Voir Lé. 16) était célébré le dixième jour du septième mois (Tishri). Le Seigneur Jésus-Christ a fait l'expiation de nos péchés afin de nous amener à Dieu. Le propitiatoire, au lieu d'être le trône du jugement, devenait ainsi le lieu de rencontre de Dieu avec le croyant (Ex. 25:22). Christ est la propitiation pour nos péchés (1 Jn. 2:2), mais il est aussi lui-même le propitiatoire (Ro. 3:25). Le péché ôté, les fautes confessées, le pardon acquis, l'holocauste offert, le chemin est ouvert pour la joie de la fête des tabernacles.}. Vous aurez une sainte convocation, vous humilierez vos âmes, et vous offrirez à Yahweh des sacrifices consumés par le feu.
\VS{28}En ce jour-là, vous ne ferez aucune œuvre, car c'est le jour des expiations, afin de faire propitiation pour vous devant Yahweh, votre Dieu.
\VS{29}Toute personne qui ne s'humiliera point en ce jour-là sera retranchée d'entre son peuple.
\VS{30}Et toute personne qui aura fait quelque œuvre en ce jour-là, je ferai périr cette personne-là du milieu de son peuple.
\VS{31}Vous ne ferez donc aucune œuvre. C'est une ordonnance perpétuelle pour vos descendants dans toutes vos demeures.
\VS{32}Ce sera pour vous un sabbat, un jour de repos, et vous humilierez vos âmes. Le neuvième jour du mois, au soir, depuis le soir jusqu'à l'autre soir, vous célébrerez votre sabbat.
\TextTitle{La fête des tabernacles\FTNTT{Esd. 3:4.}}
\VS{33}Yahweh parla aussi à Moïse, en disant~:
\VS{34}Parle aux enfants d'Israël, et dis-leur~: Au quinzième jour de ce septième mois sera la fête solennelle des tabernacles\FTNT{La fête des tabernacles ou des récoltes, était la fête du souvenir et de la joie. Célébrée au mois de Tishri, elle était aussi celle du repos, dans l'accomplissement des promesses. Elle préfigure le Royaume millénaire (Za. 14).} pendant sept jours, à Yahweh.
\VS{35}Au premier jour, il y aura une sainte convocation. Vous ne ferez aucune œuvre servile.
\VS{36}Pendant sept jours, vous offrirez à Yahweh des offrandes consumées par le feu. Et au huitième jour, vous aurez une sainte convocation, et vous offrirez à Yahweh des offrandes consumées par le feu~; ce sera une assemblée solennelle. Vous ne ferez aucune œuvre servile.
\VS{37}Ce sont là les fêtes solennelles de Yahweh, que vous publierez pour être des convocations saintes, afin d'offrir à Yahweh des offrandes consumées par le feu~; à savoir un holocauste, un gâteau, un sacrifice et une libation, chacune de ces choses en son jour~;
\VS{38}outre les sabbats de Yahweh, et outre vos dons, outre tous vos vœux, outre toutes les offrandes volontaires que vous présenterez à Yahweh.
\VS{39}Et aussi au quinzième jour du septième mois, quand vous aurez recueilli le produit du pays, vous célébrerez la fête solennelle de Yahweh pendant sept jours. Le premier jour sera un jour de repos, le huitième aussi sera un jour de repos.
\VS{40}Et au premier jour, vous prendrez du fruit d'un bel arbre, des branches de palmier, des rameaux d'arbres touffus et des saules de rivière~; et vous vous réjouirez pendant sept jours, devant Yahweh, votre Dieu.
\VS{41}Et vous célébrerez à Yahweh cette fête solennelle pendant sept jours dans l'année. C'est une loi perpétuelle pour vos descendants. Vous la célébrerez le septième mois.
\VS{42}Vous demeurerez sept jours sous des tentes~; tous ceux qui seront nés entre les Israélites demeureront sous des tentes,
\VS{43}afin que votre postérité sache que j'ai fait habiter les enfants d'Israël sous des tentes, quand je les ai fait sortir du pays d'Egypte. Je suis Yahweh, votre Dieu.
\VS{44}Moïse déclara ainsi aux enfants d'Israël les fêtes solennelles de Yahweh.
\Chap{24}
\TextTitle{L'huile du chandelier\FTNTT{Ex. 25:6.}}
\VerseOne{}Yahweh parla aussi à Moïse, en disant~:
\VS{2}Ordonne aux enfants d'Israël de t'apporter de l'huile pure d'olives pressées pour le chandelier, afin de faire brûler les lampes continuellement.
\VS{3}Aaron les arrangera devant Yahweh continuellement, depuis le soir jusqu'au matin, en dehors du voile du témoignage dans la tente d'assignation. C'est une ordonnance perpétuelle pour vos descendants.
\VS{4}Il arrangera, dis-je, continuellement les lampes sur le chandelier pur, devant Yahweh.
\TextTitle{Les pains de proposition\FTNTT{Ex. 25:23-30.}}
\VS{5}Tu prendras aussi de la fine farine\FTNT{La fine farine est une farine de blé très pure, la première qui passe à travers les tamis de bluterie.}, et tu en feras cuire douze gâteaux\FTNT{Les pains de proposition étaient au nombre de douze et ne pouvaient être consommés que par les prêtres (Lé. 24:9). Ils préfiguraient Christ, le véritable Pain de vie descendu du ciel (Jn. 6:48-51). Sous la Nouvelle Alliance, chaque enfant de Dieu est également un prêtre (Ap. 1:6), et est invité par conséquent à manger ce pain. Le nombre douze nous parle du fondement sur lequel nous devons êtres bâtis, à savoir Jésus-Christ lui-même et l'enseignement des apôtres et des prophètes (1 Co. 3:11~; Ep. 2:20).}, chaque gâteau sera de deux dixièmes.
\VS{6}Et tu les exposeras devant Yahweh en deux rangées sur la table d'or pur, six à chaque rangée.
\VS{7}Et tu mettras de l'encens pur sur chaque rangée, qui sera comme un souvenir\FTNT{Voir commentaire en Lé. 2:2.} pour le pain, c'est une offrande consumée par le feu à Yahweh.
\VS{8}On les arrangera chaque jour de sabbat continuellement devant Yahweh, de la part des enfants d'Israël. C'est une alliance perpétuelle.
\VS{9}Et ils appartiendront à Aaron et à ses fils, qui les mangeront dans un lieu saint~; car ce sera pour eux une chose très sainte d'entre les offrandes de Yahweh consumées par le feu. C'est une ordonnance perpétuelle.
\TextTitle{Le blasphème contre le Nom de Yahweh\FTNTT{Jn. 8:59~; 10:31.}}
\VS{10}Or le fils d'une femme israélite, qui était aussi fils d'un homme égyptien, sortit parmi les enfants d'Israël, et ce fils de la femme israélite se querella dans le camp avec un homme israélite.
\VS{11}Et le fils de la femme israélite blasphéma et maudit le Nom de Yahweh. On l'amena à Moïse. Or sa mère s'appelait Schelomith, fille de Dibri, de la tribu de Dan.
\VS{12}Et on le mit en prison, jusqu'à ce que Moïse ait déclaré ce qu'il devrait faire selon la parole de Yahweh.
\VS{13}Et Yahweh parla à Moïse, en disant~:
\VS{14}Fais sortir du camp celui qui a maudit~; et que tous ceux qui l'ont entendu mettent les mains sur sa tête, et que toute l'assemblée le lapide.
\VS{15}Tu parleras aux enfants d'Israël, et tu leur diras~: Quiconque aura maudit son Dieu, portera la peine de son péché.
\VS{16}Et celui qui aura blasphémé le Nom de Yahweh, il mourra, il mourra. Toute l'assemblée le lapidera, le maidera. On fera mourir tant l'étranger que celui qui est né au pays, quand il aura blasphémé le Nom de Yahweh.
\TextTitle{La violence punie}
\VS{17}Celui qui aura frappé mortellement un homme, quel qu'il soit, il mourra, il mourra.
\VS{18}Celui qui aura frappé une bête mortellement, la remplacera~: Vie pour vie.
\VS{19}Et quand quelque homme aura fait une blessure à son prochain, on lui fera comme il a fait~:
\VS{20}Fracture pour fracture, œil pour œil, dent pour dent. Selon le mal qu'il aura fait à un homme, il lui sera fait de même.
\VS{21}Celui qui frappera une bête mortellement, la remplacera~; mais on fera mourir celui qui aura tué un homme.
\VS{22}Vous rendrez un même jugement. Vous traiterez l'étranger comme celui qui est né au pays~; car je suis Yahweh, votre Dieu.
\VS{23}Moïse parla aux enfants d'Israël, qui firent sortir hors du camp celui qui avait maudit, et le lapidèrent. Ainsi les enfants d'Israël firent comme Yahweh l'avait ordonné à Moïse.
\Chap{25}
\TextTitle{L'année sabbatique}
\VerseOne{}Yahweh parla aussi à Moïse sur la montagne de Sinaï, en disant~:
\VS{2}Parle aux enfants d'Israël, et dis-leur~: Quand vous serez entrés dans le pays que je vous donne, la terre se reposera. Ce sera un sabbat à Yahweh.
\VS{3}Pendant six ans tu sèmeras ton champ, et pendant six ans tu tailleras ta vigne~; et tu en recueilleras le produit.
\VS{4}Mais la septième année il y aura un sabbat, un temps de repos pour la terre, ce sera un sabbat à Yahweh. Tu ne sèmeras point ton champ, et tu ne tailleras point ta vigne.
\VS{5}Tu ne moissonneras point ce qui proviendra des grains tombés dans ta moisson, et tu ne vendangeras point les raisins de ta vigne non taillée. Ce sera une année de repos total pour la terre.
\VS{6}Mais ce qui proviendra de la terre l'année du sabbat vous servira de nourriture, à toi, à ton serviteur et à ta servante, à ton mercenaire et à l'étranger qui demeurent avec toi,
\VS{7}à ton bétail et aux animaux qui sont dans ton pays~; tout son produit servira de nourriture.
\TextTitle{L'année du jubilé}
\VS{8}Tu compteras aussi sept sabbats d'années, à savoir sept fois sept ans, et les jours de sept sabbats feront quarante-neuf ans.
\VS{9}Puis tu feras sonner le shofar du jubilé le dixième jour du septième mois~; le jour, dis-je, des expiations, vous ferez sonner le shofar dans tout votre pays.
\VS{10}Et vous sanctifierez la cinquantième année, et publierez la liberté dans le pays à tous ses habitants. Ce sera pour vous l'année du jubilé~; et vous retournerez chacun dans sa possession, et chacun dans sa famille.
\VS{11}Cette cinquantième année vous sera l'année du jubilé. Vous ne sèmerez point et vous ne moissonnerez point ce que la terre rapportera d'elle-même, et vous ne vendangerez point les fruits de la vigne non taillée.
\VS{12}Car c'est l'année du jubilé, elle vous sera sainte. Vous mangerez ce que les champs rapporteront cette année-là.
\VS{13}En cette année du jubilé chacun de vous retournera dans sa possession.
\VS{14}Et si tu fais une vente à ton prochain, ou si tu achètes quelque chose de ton prochain, que nul de vous ne trompe son frère.
\VS{15}Mais tu achèteras de ton prochain selon le nombre des années après le jubilé. Pareillement on te fera les ventes selon le nombre des années de rapport.
\VS{16}Selon qu'il y aura plus d'années, tu augmenteras le prix de ce que tu achètes~; et selon qu'il y aura moins d'années, tu le diminueras~; car on te vend le nombre des récoltes.
\VS{17}Que nul de vous ne trompe son prochain, mais craignez votre Dieu~; car je suis Yahweh, votre Dieu.
\VS{18}Pratiquez mes ordonnances, gardez mes jugements et observez-les, et vous habiterez en sécurité dans le pays.
\VS{19}Et le pays vous donnera ses fruits, vous en mangerez, vous en serez rassasiés, et vous y habiterez en sécurité.
\VS{20}Et si vous dites~: Que mangerons-nous la septième année si nous ne semons point, et si nous ne recueillons point notre récolte~?
\VS{21}J'ordonnerai à ma bénédiction de se répandre sur vous dans la sixième année, et la terre rapportera pour trois ans.
\VS{22}Puis vous sèmerez la huitième année, et vous mangerez de l'ancienne récolte jusqu'à la neuvième année~; jusqu'à ce que sa récolte soit venue, vous mangerez de l'ancienne.
\VS{23}La terre ne sera point vendue à perpétuité~; car le pays est à moi, et vous êtes étrangers et forains\FTNT{Forain~: Quelqu'un d'extérieur, d'étranger à un lieu.} chez moi.
\VS{24}C'est pourquoi dans tout le pays dont vous aurez la possession, vous donnerez le droit de rachat\FTNT{Pour voir un exemple de ce droit de rachat, voir Ru. 4:1-13.} pour la terre.
\TextTitle{Le droit de rachat}
\VS{25}Si ton frère est devenu pauvre et vend quelque chose de ce qu'il possède, celui qui a le droit de rachat, à savoir son plus proche parent, viendra et rachètera la chose vendue par son frère.
\VS{26}Si cet homme n'a personne qui ait le droit de rachat, et qu'il ait trouvé de lui-même suffisamment de quoi faire le rachat de ce qu'il a vendu,
\VS{27}il comptera les années du temps qu'il a fait la vente, et il restituera le surplus à l'homme auquel il l'avait faite, et ainsi il retournera dans sa possession.
\VS{28}Mais s'il n'a pas trouvé suffisamment de quoi lui rendre, la chose qu'il aura vendue sera dans les mains de celui qui l'aura achetée, jusqu'à l'année du jubilé~; puis l'acheteur en sortira au jubilé, et le vendeur retournera dans sa possession.
\VS{29}Et si quelqu'un a vendu une maison d'habitation dans quelque ville entourée de murs, il aura le droit de rachat jusqu'à la fin de l'année de sa vente~; son droit de rachat sera d'une année.
\VS{30}Mais si elle n'est point rachetée dans l'année accomplie, la maison qui est dans la ville entourée de murs, demeurera à l'acheteur absolument et à ses descendants~; il n'en sortira point au jubilé.
\VS{31}Mais les maisons des villages, qui ne sont point entourés de murs, seront comptées comme des fonds de terre~; le vendeur aura droit de rachat, et l'acheteur sortira au jubilé.
\VS{32}Et quant aux villes des Lévites, les Lévites auront un droit de rachat perpétuel des maisons des villes de leur possession.
\VS{33}Et celui qui achètera une maison des Lévites, sortira au jubilé de la maison vendue, qui est dans la ville de sa possession~; car les maisons des villes des Lévites sont leur possession parmi les enfants d'Israël.
\VS{34}Mais les champs situés autour des villes des Lévites ne seront point vendus~; car c'est leur possession perpétuelle.
\TextTitle{Les traitements du frère pauvre}
\VS{35}Quand ton frère sera devenu pauvre, et qu'il tendra vers toi ses mains tremblantes, tu le soutiendras, tu soutiendras aussi l'étranger, et le forain, afin qu'il vive avec toi.
\VS{36}Tu ne prendras point de lui d'usure ni d'intérêt, mais tu craindras ton Dieu, et ton frère vivra avec toi.
\VS{37}Tu ne lui prêteras point ton argent à intérêt ni ne lui prêteras de tes vivres pour en tirer du profit.
\VS{38}Je suis Yahweh, votre Dieu qui vous ai fait sortir du pays d'Egypte, pour vous donner le pays de Canaan, afin d'être votre Dieu.
\VS{39}Pareillement, quand ton frère sera devenu pauvre auprès de toi, et qu'il se sera vendu à toi, tu ne te serviras point de lui comme on se sert des esclaves.
\VS{40}Mais il sera chez toi, comme serait le mercenaire et l'étranger, et il te servira jusqu'à l'année du jubilé.
\VS{41}Alors il sortira de chez toi avec ses fils, il s'en retournera dans sa famille, et rentrera dans la possession de ses pères.
\VS{42}Car ils sont mes serviteurs, parce que je les ai retirés du pays d'Egypte~; c'est pourquoi ils ne seront point vendus comme on vend les esclaves.
\VS{43}Tu ne domineras point sur lui avec dureté, et tu craindras ton Dieu.
\VS{44}C'est des nations qui vous entourent que tu prendras ton esclave et ta servante qui t'appartiendront~; c'est d'elles que vous achèterez l'esclave et la servante.
\VS{45}Vous pourrez aussi en acheter des fils des étrangers qui demeureront chez toi, et même de leurs familles qui seront parmi vous, qui auront engendré dans votre pays, et vous les posséderez.
\VS{46}Vous les aurez comme un héritage pour les laisser à vos enfants après vous, afin qu'ils en héritent la possession, et vous vous servirez d'eux à perpétuité. Mais quant à vos frères, les fils d'Israël, nul ne dominera avec dureté sur son frère.
\VS{47}Et lorsque l'étranger ou celui qui demeure avec toi deviendra riche, et que ton frère qui est avec lui deviendra si pauvre qu'il se soit vendu à l'étranger qui demeure avec toi, ou à quelqu'un issu de la famille de l'étranger,
\VS{48}après s'être vendu, il y aura pour lui le droit de rachat. Un de ses frères le rachètera.
\VS{49}Son oncle, ou le fils de son oncle, ou quelque autre proche parent de son sang d'entre ceux de sa famille, le rachètera~; ou lui-même, s'il en trouve le moyen, se rachètera.
\VS{50}Et il comptera avec son acheteur depuis l'année qu'il s'est vendu à lui, jusqu'à l'année du jubilé~; de sorte que l'argent du prix pour lequel il s'est vendu, se comptera à raison du nombre des années, le temps qu'il aura servi lui sera compté comme les journées d'un mercenaire.
\VS{51}S'il y a encore plusieurs années, il restituera le prix de son achat à raison de ces années, selon le prix pour lequel il a été acheté~;
\VS{52}et s'il reste peu d'années jusqu'à l'année du jubilé, il comptera avec lui, et restituera le prix de son achat à raison des années qu'il a servi.
\VS{53}Il aura été avec lui comme un mercenaire qui se loue d'année en année, et cet étranger ne dominera point sur lui avec dureté en ta présence.
\VS{54}S'il n'est pas racheté par quelqu'un de ces moyens, il sortira l'année du jubilé, lui et ses fils avec lui.
\VS{55}Car c'est de moi que les enfants d'Israël sont esclaves~; ce sont mes esclaves que j'ai fait sortir du pays d'Egypte. Je suis Yahweh, votre Dieu.
\Chap{26}
\TextTitle{Mise en garde contre le péché}
\VerseOne{}Vous ne vous ferez point d'idole, vous ne vous dresserez point d'image taillée, ni de statue, et vous ne mettrez point de pierre sculptée dans votre pays, pour vous prosterner devant elles~; car je suis Yahweh, votre Dieu.
\VS{2}Vous garderez mes sabbats et vous révérerez mon sanctuaire. Je suis Yahweh.
\TextTitle{La bénédiction conditionnelle à l'obéissance à Yahweh}
\VS{3}Si vous marchez dans mes ordonnances et si vous gardez mes commandements et les pratiquez,
\VS{4}je vous donnerai les pluies en leur temps, la terre donnera ses produits, et les arbres des champs donneront leurs fruits.
\VS{5}Le foulage des grains atteindra la vendange chez vous, et la vendange atteindra les semailles~; vous mangerez votre pain à satiété et vous habiterez en sécurité dans votre pays.
\VS{6}Je donnerai la paix au pays, vous dormirez sans que personne ne vous trouble~; je ferai disparaître les bêtes méchantes du pays, et l'épée ne passera point par votre pays.
\VS{7}Vous poursuivrez vos ennemis, et ils tomberont par l'épée devant vous.
\VS{8}Cinq d'entre vous en poursuivront cent, et cent en poursuivront dix mille, et vos ennemis tomberont par l'épée devant vous.
\VS{9}Je me tournerai vers vous, je vous ferai fructifier et multiplier, et j'établirai mon alliance avec vous.
\VS{10}Vous mangerez de vieilles provisions, et vous sortirez le vieux pour y loger le nouveau.
\VS{11}Même, je mettrai mon tabernacle au milieu de vous, et mon âme ne vous aura point en horreur.
\VS{12}Mais je marcherai au milieu de vous, je serai votre Dieu, et vous serez mon peuple.
\VS{13}Je suis Yahweh, votre Dieu, qui vous ai fait sortir du pays d'Egypte, afin que vous ne soyez point leurs esclaves~; j'ai brisé les liens de votre joug, et je vous ai fait marcher la tête levée.
\TextTitle{Les châtiments en cas de désobéissance à Yahweh}
\VS{14}Mais si vous ne m'écoutez point et que vous ne pratiquez pas tous ces commandements,
\VS{15}et si vous rejetez mes ordonnances, et que votre âme a en horreur mes jugements, afin de ne point pratiquer tous mes commandements, et que vous rompiez mon alliance,
\TextTitle{La domination par les ennemis}
\VS{16}voici alors ce que je vous ferai~: Je répandrai sur vous la frayeur, la langueur et l'ardeur, qui vous consumeront les yeux et feront languir votre âme~; et vous sèmerez en vain votre semence car vos ennemis la mangeront.
\VS{17}Je tournerai ma face contre vous, vous serez battus devant vos ennemis~; ceux qui vous haïssent domineront sur vous~; et vous fuirez sans que personne ne vous poursuive.
\TextTitle{Le manque de fertilité de la terre}
\VS{18}Si après ces choses vous ne m'écoutez point, je vous châtierai sept fois plus à cause de vos péchés.
\VS{19}Je briserai l'orgueil de votre force et je ferai que votre ciel soit pour vous comme du fer, et votre terre comme de l'airain.
\VS{20}Votre force se consumera en vain, votre terre ne donnera point ses produits, et les arbres de la terre ne donneront point leurs fruits.
\TextTitle{Les attaques des bêtes des champs}
\VS{21}Si vous marchez en opposition avec moi et que vous ne voulez point m'écouter, je vous frapperai sept fois plus, selon vos péchés.
\VS{22}J'enverrai contre vous les bêtes des champs, qui vous priveront de vos enfants, qui détruiront votre bétail, et vous réduiront à un petit nombre, et vos chemins seront déserts.
\TextTitle{La peste}
\VS{23}Si après ces choses, vous ne recevez pas ma correction, et que vous marchiez en opposition avec moi,
\VS{24}je marcherai aussi en opposition avec vous, et je vous frapperai sept fois plus, selon vos péchés.
\VS{25}Et je ferai venir sur vous l'épée qui fera la vengeance de mon alliance~; et quand vous vous rassemblerez dans vos villes, j'enverrai la peste au milieu de vous, et vous serez livrés entre les mains de l'ennemi.
\TextTitle{La famine}
\VS{26}Lorsque je vous briserai le bâton du pain, dix femmes cuiront votre pain dans un seul four, et vous rendront votre pain au poids~; vous en mangerez, et vous n'en serez point rassasiés.
\VS{27}Si avec cela vous ne m'écoutez point, et que vous marchiez en opposition avec moi,
\VS{28}je marcherai aussi en opposition avec vous, avec fureur, et je vous châtierai aussi sept fois plus, selon vos péchés~;
\VS{29}vous mangerez la chair de vos fils, et vous mangerez aussi la chair de vos filles\FTNT{La. 4:10.}.
\VS{30}Je détruirai vos hauts lieux, j'abattrai vos statues consacrées au soleil, je mettrai vos cadavres sur les cadavres de vos idoles, et mon âme vous aura en horreur.
\VS{31}Je réduirai vos villes en désert, je dévasterai vos sanctuaires, et je ne respirerai plus l'agréable odeur de vos parfums.
\TextTitle{La dispersion dans les nations\FTNTT{De. 28:58-67.}}
\VS{32}Je dévasterai le pays, et vos ennemis qui l'habiteront, en seront étonnés.
\VS{33}Je vous disperserai parmi les nations, et je tirerai l'épée après vous, et votre pays sera dévasté, et vos villes désertes.
\VS{34}Alors la terre prendra plaisir à ses sabbats\FTNT{2 Ch. 36:21.}, tout le temps qu'elle sera dévastée, et lorsque vous serez dans le pays de vos ennemis, la terre se reposera et prendra plaisir à ses sabbats.
\VS{35}Tout le temps qu'elle sera dévastée, elle se reposera parce qu'elle ne s'était point reposée dans vos sabbats, lorsque vous y habitiez.
\VS{36}Et quant à ceux d'entre vous qui survivront dans le pays de leurs ennemis, je rendrai leur cœur lâche, de sorte que le bruit d'une feuille agitée les poursuivra, ils fuiront comme on fuit devant l'épée, et ils tomberont sans que personne ne les poursuive.
\VS{37}Et ils trébucheront les uns sur les autres comme devant l'épée, sans que personne ne les poursuive~; et vous ne tiendrez point devant vos ennemis~;
\VS{38}vous périrez parmi les nations, et le pays de vos ennemis vous consumera.
\VS{39}Et ceux d'entre vous qui survivront, se fondront à cause de leurs iniquités, dans les pays de vos ennemis~; ils se fondront aussi à cause des iniquités de leurs pères.
\TextTitle{Repentance et restauration de l'alliance d'Abraham, d'Isaac et de Jacob}
\VS{40}Alors, ils confesseront leurs iniquités et les iniquités de leurs pères, selon les transgressions qu'ils auront commises contre moi~; et aussi parce qu'ils auront marché en opposition avec moi.
\VS{41}Moi aussi, je marcherai en opposition avec eux, je les amènerai dans le pays de leurs ennemis. Et alors, leur cœur incirconcis s'humiliera, et ils recevront la peine de leur iniquité.
\VS{42}Et alors je me souviendrai de mon alliance avec Jacob, et de mon alliance avec Isaac, et je me souviendrai aussi de mon alliance avec Abraham, et je me souviendrai de la terre.
\VS{43}Quand donc la terre sera abandonnée par eux, et prendra plaisir à ses sabbats, ayant été désolée à cause d'eux~; et qu'ils recevront la peine de leur iniquité, parce qu'ils ont rejeté mes ordonnances, et que leur âme a eu mes ordonnances en horreur~;
\VS{44}je m'en souviendrai, dis-je, lorsqu'ils seront dans le pays de leurs ennemis. Je ne les rejetterai point, et je ne les aurai point en horreur pour les consumer entièrement jusqu'à rompre mon alliance avec eux~; car je suis Yahweh, leur Dieu.
\VS{45}Je me souviendrai en leur faveur de la Première Alliance, par laquelle je les ai fait sortir du pays d'Egypte, aux yeux des nations, pour être leur Dieu. Je suis Yahweh.
\VS{46}Ce sont là les statuts, les ordonnances, et les lois que Yahweh établit entre lui et les enfants d'Israël sur la montagne de Sinaï, par Moïse.
\Chap{27}
\TextTitle{Lois sur les personnes et les biens voués à Yahweh}
\VerseOne{}Yahweh parla aussi à Moïse, en disant~:
\VS{2}Parle aux enfants d'Israël, et dis-leur~: Quand quelqu'un aura fait un vœu important, les personnes vouées à Yahweh seront mises à ton estimation.
\VS{3}Et l'estimation que tu feras d'un homme, depuis l'âge de vingt ans jusqu'à l'âge de soixante ans, sera du prix de cinquante sicles d'argent, selon le sicle du sanctuaire.
\VS{4}Mais si c'est une femme, alors ton estimation sera de trente sicles.
\VS{5}Si c'est un homme de cinq ans jusqu'à vingt ans, alors ton estimation sera de vingt sicles~; et quant à la femme, de dix sicles.
\VS{6}Et si c'est un homme d'un mois jusqu'à cinq ans, ton estimation sera de cinq sicles d'argent~; et l'estimation d'une femme sera de trois sicles d'argent.
\VS{7}Et lorsque c'est un homme de soixante ans et au-dessus, ton estimation sera de quinze sicles~; et si c'est une femme, de dix sicles.
\VS{8}Et si celui qui a fait le vœu est plus pauvre que ton estimation, on le présentera devant le prêtre, qui en fera l'estimation, et le prêtre fera l'estimation selon les ressources de celui qui a fait le vœu.
\VS{9}Si c'est d'une des bêtes que l'on présente en offrande à Yahweh, tout ce qu'on donnera à Yahweh de la sorte sera saint.
\VS{10}On ne la changera point, et on n'en mettra point une autre à la place, d'une bonne pour une mauvaise, ou une mauvaise pour une bonne~; si l'on remplace une bête par une autre bête, elles seront l'une et l'autre chose sainte.
\VS{11}Si c'est d'une bête impure, qu'on ne peut présenter en offrande à Yahweh, on présentera la bête devant le prêtre,
\VS{12}qui en fera l'évaluation selon qu'elle sera bonne ou mauvaise, et il en sera fait ainsi, selon l'estimation du prêtre.
\VS{13}Mais si on veut la racheter, on ajoutera un cinquième à ton estimation.
\VS{14}Et quand quelqu'un sanctifiera sa maison pour être sainte à Yahweh, le prêtre l'estimera selon qu'elle sera bonne ou mauvaise, et on se tiendra à l'estimation que le prêtre en aura faite.
\VS{15}Mais si celui qui l'a sanctifiée veut racheter sa maison, il ajoutera par-dessus un cinquième de l'argent de ton estimation, et elle lui appartiendra.
\VS{16}Et si l'homme sanctifie à Yahweh une partie du champ de sa possession, ton estimation sera selon ce qu'on y sème, l'homer de semence d'orge à cinquante sicles d'argent.
\VS{17}S'il a sanctifié son champ dès l'année du jubilé, on s'en tiendra à ton estimation~;
\VS{18}mais s'il sanctifie son champ après le jubilé, le prêtre estimera l'argent selon le nombre des années qui restent jusqu'à l'année du jubilé, et il sera fait une réduction sur ton estimation.
\VS{19}Et si celui qui a sanctifié le champ veut le racheter en quelque sorte que ce soit, il ajoutera par-dessus un cinquième de l'argent de ton estimation, et il lui restera.
\VS{20}Mais s'il ne rachète point le champ, et que le champ se vende à un autre homme, il ne se rachètera plus.
\VS{21}Et ce champ-là ayant passé le jubilé sera consacré à Yahweh, comme un champ d'interdit, la possession en sera au prêtre.
\VS{22}Et s'il sanctifie à Yahweh un champ qu'il ait acheté, qui ne soit point des champs de sa possession,
\VS{23}le prêtre lui comptera la somme de ton estimation jusqu'à l'année du jubilé, et il donnera en ce jour-là ton estimation, afin que ce soit une chose consacrée à Yahweh.
\VS{24}Mais l'année du jubilé, le champ retournera à celui de qui il avait été acheté, et auquel était la possession de la terre.
\VS{25}Et toute estimation que tu auras faite, sera selon le sicle du sanctuaire~: Le sicle est de vingt guéras.
\TextTitle{Consécration des premiers-nés du bétail}
\VS{26}Toutefois, nul ne pourra consacrer le premier-né d'entre les bêtes, car il appartient à Yahweh par droit de primogéniture, soit de bœuf, soit d'agneau, il est à Yahweh.
\VS{27}Mais s'il s'agit d'une bête impure, il le rachètera selon ton estimation, et il ajoutera à ton estimation un cinquième~; et s'il n'est point racheté, il sera vendu selon ton estimation.
\TextTitle{Consécration des choses et des personnes dévouées par interdit à Yahweh}
\VS{28}Or toute chose dévouée, que quelqu'un dévouera à la façon de l'interdit à Yahweh, de tout ce qui est sien, soit homme, ou bête, ou champ de sa possession, ne se revendra ni ne se rachètera~; toute chose dévouée sera entièrement consacrée à Yahweh.
\VS{29}Nul interdit, dévoué par interdit d'entre les hommes, ne pourra être racheté, mais on le fera mettre à mort.
\TextTitle{Consécration de la dîme de la terre et du bétail}
\VS{30}Toute dîme de la terre, tant du grain de la terre que du fruit des arbres, est à Yahweh~; c'est une chose consacrée à Yahweh.
\VS{31}Mais si quelqu'un veut racheter en quelque sorte que ce soit quelque chose de sa dîme, il y ajoutera un cinquième par-dessus.
\VS{32}Mais toute dîme de bœufs, de brebis et de chèvres, à savoir tout ce qui passe sous la verge, le dixième en sera consacré à Yahweh.
\VS{33}On ne choisira point le bon ou le mauvais, et l'on ne fera point d'échange~; si on l'échange, la bête changée et l'autre seront consacrées, et ne seront point rachetées.
\VS{34}Ce sont là les commandements que Yahweh donna à Moïse sur la montagne de Sinaï, pour les enfants d'Israël.
\PPE{}
\end{multicols}
