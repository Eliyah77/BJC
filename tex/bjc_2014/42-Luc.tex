\ShortTitle{Lu.}\BookTitle{Luc}\BFont
\noindent\hrulefill
{\footnotesize
\textit{
\bigskip
{\centering{}
\\Auteur~: Luc
\\(Gr.~: Loukas)
\\Signification~: Qui donne la lumière
\\Thème~: Jésus le Fils de l'homme
\\Date de rédaction~: Env. 60 ap. J.-C.\\}
}
\textit{
\\D'origine grecque, Luc fut l'auteur de l'Evangile éponyme et du livre des «~Actes des apôtres~». Celui que Paul appelait le «~médecin bien-aimé~» et qui fut son compagnon d'œuvre avait entrepris des investigations visant à narrer avec exactitude la vie terrestre de Jésus-Christ dont il était devenu le disciple, probablement à la suite d'une prédication de Paul. Adressés initialement à Théophile, Luc était loin de penser que ses écrits constitueraient avec le temps une véritable richesse pour l'Eglise et pour le monde.
\\L'évangile de Luc présente l'humanité parfaite de Jésus, sa compassion et sa miséricorde à l'égard des plus faibles. Rédigé avec rigueur et soin, il retrace le parcours du Fils de l'homme, de sa naissance à son adolescence, puis de sa mort à sa résurrection, et enfin son ascension. Il souligne aussi sa vie de prière et son fardeau pour le salut de l'homme. Par ailleurs, il fait ressortir la manière dont les femmes ont assisté Jésus par leurs biens durant son service.
\\Fruit de recherches minutieuses, le récit de Luc présente certaines similitudes avec ceux de Matthieu et de Marc. Il est toutefois le seul à relater la célèbre parabole du fils prodigue, profonde représentation de l'amour du Père.\bigskip
}
}
\par\nobreak\noindent\hrulefill
\begin{multicols}{2}
\Chap{1}
\TextTitle{Introduction}
\VerseOne{}Parce que plusieurs se sont appliqués à mettre par ordre un récit des événements qui ont été pleinement certifiés parmi nous,
\VS{2}suivant ce que nous ont transmis ceux qui ont été des témoins oculaires, dès le commencement, et sont devenus des serviteurs de la parole.
\VS{3}Il m'a aussi semblé bon, après avoir examiné exactement toutes choses depuis le commencement jusqu'à la fin, très excellent Théophile, de te les mettre en ordre par écrit,
\VS{4}afin que tu connaisses la certitude des choses dans lesquelles tu as été instruit de vive voix.
\TextTitle{Un ange annonce la naissance de Jean-Baptiste}
\VS{5}Au temps d'Hérode, roi de Judée, il y avait un prêtre nommé Zacharie, de la classe d'Abia~; et sa femme était d'entre les filles d'Aaron, et s'appelait Elisabeth.
\VS{6}Et ils étaient tous deux justes devant Dieu, marchant dans tous les commandements et dans toutes les ordonnances du Seigneur, sans reproche.
\VS{7}Et ils n'avaient point d'enfants, parce qu'Elisabeth était stérile et qu'ils étaient fort avancés en âge.
\VS{8}Or il arriva que, comme Zacharie exerçait la prêtrise devant Dieu, selon le tour de sa classe, il fut appelé par le sort,
\VS{9}selon la coutume d'exercer la prêtrise, à entrer dans le temple du Seigneur pour offrir le parfum.
\VS{10}Toute la multitude du peuple était dehors en prière, à l'heure du parfum.
\VS{11}Et l'ange du Seigneur lui apparut, et se tint debout à droite de l'autel des parfums.
\VS{12}Zacharie fut troublé quand il le vit, et il fut saisi de crainte.
\VS{13}Mais l'ange lui dit~: Zacharie, ne crains point, car ta prière est exaucée, et Elisabeth, ta femme t'enfantera un fils, et tu lui donneras le nom de Jean.
\VS{14}Et il sera pour toi le sujet d'une grande joie et d'allégresse, et plusieurs se réjouiront de sa naissance,
\VS{15}car il sera grand devant le Seigneur, et il ne boira ni vin, ni boisson forte, et il sera rempli du Saint-Esprit dès le ventre de sa mère.
\VS{16}Et il ramènera plusieurs des enfants d'Israël au Seigneur, leur Dieu~;
\VS{17}car il marchera devant lui animé de l'esprit et de la puissance d'Elie, pour ramener les cœurs des pères vers les enfants\FTNT{Mal. 4:6.}, et les rebelles à la sagesse des justes, pour préparer au Seigneur un peuple bien disposé.
\VS{18}Alors Zacharie dit à l'ange~: A quoi reconnaîtrai-je cela~? Car je suis vieux et ma femme est fort âgée.
\VS{19}L'ange répondant, lui dit~: Je suis Gabriel, je me tiens devant Dieu et j'ai été envoyé pour te parler, et pour t'annoncer cette bonne nouvelle.
\VS{20}Et voici, tu seras muet et tu ne pourras point parler jusqu'au jour où ces choses arriveront, parce que tu n'as pas cru à mes paroles qui s'accompliront en leur temps.
\VS{21}Or le peuple attendait Zacharie et on s'étonnait de ce qu'il tardait tant dans le temple.
\VS{22}Mais, quand il fut sorti, il ne pouvait pas leur parler, et ils comprirent qu'il avait eu une vision dans le temple~; car il leur faisait des signes, et il resta muet.
\VS{23}Et il arriva que, quand les jours de son service furent achevés, il retourna dans sa maison.
\VS{24}Et après ces jours-là, Elisabeth, sa femme, conçut, et elle se cacha l'espace de cinq mois, en disant~:
\VS{25}Certes, le Seigneur en a agi avec moi, ainsi aux jours qu'il m'a regardée, pour ôter mon opprobre d'entre les hommes.
\TextTitle{L'ange Gabriel annonce la naissance de Jésus-Christ}
\VS{26}Or au sixième mois, l'ange Gabriel fut envoyé par Dieu dans une ville de Galilée, appelée Nazareth,
\VS{27}vers une vierge fiancée à un homme nommé Joseph, qui était de la maison de David. Et le nom de la vierge était Marie.
\VS{28}Et l'ange étant entré dans le lieu où elle était, lui dit~: Je te salue, toi à qui une grâce a été faite~; le Seigneur est avec toi~; tu es bénie parmi les femmes.
\VS{29}Troublée par cette parole, Marie se demandait ce que pouvait signifier une telle salutation.
\VS{30}L'ange lui dit~: Marie, ne crains point~; car tu as trouvé grâce devant Dieu.
\VS{31}Et voici, tu concevras en ton ventre, et tu enfanteras un fils, et tu lui donneras le Nom de JESUS.
\VS{32}Il sera grand, et sera appelé le Fils du Très-Haut, et le Seigneur Dieu lui donnera le trône de David, son père.
\VS{33}Il régnera sur la maison de Jacob éternellement, et son règne n'aura pas de fin.
\TextTitle{Naissance miraculeuse de Jésus-Christ}
\VS{34}Alors Marie dit à l'ange~: Comment cela se fera-t-il, puisque je ne connais point d'homme~?
\VS{35}L'ange lui répondit, et dit~: Le Saint-Esprit viendra sur toi, et la puissance du Très-Haut te couvrira de son ombre. C'est pourquoi, le Saint qui naîtra de toi sera appelé Fils de Dieu.
\VS{36}Voici, Elisabeth, ta cousine, a conçu elle aussi, un fils en sa vieillesse, et celle qui était appelée stérile est dans son sixième mois de grossesse.
\VS{37}Car rien n'est impossible à Dieu.
\VS{38}Et Marie dit~: Voici la servante du Seigneur~; qu'il me soit fait selon ta parole~! Et l'ange la quitta.
\TextTitle{Marie se rend chez Elisabeth}
\VS{39}Dans ce même temps, Marie se leva, et s'en alla en hâte au pays des montagnes, dans une ville de Juda.
\VS{40}Elle entra dans la maison de Zacharie et salua Elisabeth.
\VS{41}Et il arriva, comme Elisabeth entendait la salutation de Marie, que le petit enfant tressaillit dans son ventre et Elisabeth fut remplie de l'Esprit Saint.
\VS{42}Elle s'écria d'une voix forte et dit~: Tu es bénie entre les femmes, et béni est le fruit de ton ventre.
\VS{43}Comment m'est-il accordé que la mère de mon Seigneur vienne vers moi~?
\VS{44}Car voici, dès que la voix de ta salutation est parvenue à mes oreilles, le petit enfant a tressailli de joie dans mon ventre.
\VS{45}Heureuse celle qui a cru, parce que les choses qui lui ont été dites par le Seigneur auront leur accomplissement.
\TextTitle{Cantique de Marie\FTNTT{Cp. 1 S. 2:1-10.}}
\VS{46}Alors Marie dit~: Mon âme magnifie le Seigneur,
\VS{47}et mon esprit se réjouit en Dieu, mon Sauveur.
\VS{48}Car il a jeté les yeux sur la bassesse de sa servante. Voici, certes désormais toutes les générations me diront bienheureuse,
\VS{49}parce que le Tout-Puissant a fait pour moi de grandes choses et son Nom est Saint,
\VS{50}et sa miséricorde s'étend de génération en génération en faveur de ceux qui le craignent.
\VS{51}Il a puissamment opéré par son bras~; il a dissipé les desseins que les orgueilleux formaient dans leurs cœurs.
\VS{52}Il a renversé de dessus leurs trônes les puissants, et il a élevé les petits.
\VS{53}Il a rassasié de biens les affamés et il a renvoyé les riches à vide.
\VS{54}Il a pris sous sa protection Israël, son serviteur, et il s'est souvenu de sa miséricorde,
\VS{55}selon qu'il en avait parlé à nos pères, à savoir à Abraham et sa postérité à jamais.
\VS{56}Marie demeura avec elle environ trois mois. Puis elle retourna dans sa maison.
\TextTitle{Naissance de Jean}
\VS{57}Le temps où Elisabeth devait accoucher arriva, et elle enfanta un fils.
\VS{58}Ses voisins et ses parents ayant appris que le Seigneur avait fait éclater sa miséricorde envers elle, s'en réjouissaient avec elle.
\VS{59}Et il arriva qu'au huitième jour, ils vinrent pour circoncire le petit enfant, et ils l'appelaient Zacharie, du nom de son père.
\VS{60}Mais sa mère prit la parole, et dit~: Non, mais il sera appelé Jean.
\VS{61}Et ils lui dirent~: Il n'y a personne dans ta parenté qui soit appelé de ce nom.
\VS{62}Alors ils firent signe à son père pour savoir comment il voulait qu'on l'appelle.
\VS{63}Et Zacharie ayant demandé des tablettes écrivit, disant~: Jean est son nom. Et tous furent dans l'étonnement.
\VS{64}Au même instant, sa bouche s'ouvrit et sa langue se délia, et il parlait, bénissant Dieu.
\VS{65}Tous ses voisins furent saisis de crainte et toutes ces choses furent divulguées dans tout le pays des montagnes de Judée.
\VS{66}Tous ceux qui les apprirent les gardèrent dans leur cœur, disant~: Que sera donc cet enfant~? Et la main du Seigneur était avec lui.
\TextTitle{Cantique de Zacharie}
\VS{67}Alors Zacharie, son père, fut rempli du Saint-Esprit, et il prophétisa, en ces mots~:
\VS{68}Béni soit le Seigneur, le Dieu d'Israël, de ce qu'il a visité et délivré son peuple,
\VS{69}et de ce qu'il nous a suscité un puissant Sauveur dans la maison de David, son serviteur,
\VS{70}selon ce qu'il avait dit par la bouche de ses saints prophètes des temps anciens~;
\VS{71}un Sauveur qui nous délivre de nos ennemis et de la main de tous ceux qui nous haïssent~!
\VS{72}C'est ainsi qu'il manifeste sa miséricorde envers nos pères, et se souvient de sa sainte alliance,
\VS{73}selon le serment par lequel il avait juré à Abraham, notre père,
\VS{74}de nous permettre, après que nous serions délivrés de la main de nos ennemis, de le servir sans crainte,
\VS{75}en marchant devant lui dans la sainteté et dans la justice tous les jours de notre vie.
\VS{76}Et toi, petit enfant, tu seras appelé prophète du Très-Haut~; car tu marcheras devant la face du Seigneur, pour préparer ses voies,
\VS{77}afin de donner à son peuple la connaissance du salut, par la rémission de leurs péchés,
\VS{78}grâce aux entrailles de la miséricorde de notre Dieu, en vertu de laquelle le Soleil Levant nous a visités d'en haut,
\VS{79}pour éclairer ceux qui sont assis dans les ténèbres et dans l'ombre de la mort et pour conduire nos pas dans le chemin de la paix.
\VS{80}Or le petit enfant croissait et se fortifiait en esprit. Et il demeura dans les déserts jusqu'au jour où il se présenta à Israël.
\Chap{2}
\TextTitle{Naissance de Jésus à Bethléhem\FTNT{Mt. 1:18-25~; 2:1~; Jn. 1:14.}}
\VerseOne{}En ces jours-là fut publié un édit par César Auguste, ordonnant un recensement de toute la terre.
\VS{2}Ce premier recensement eut lieu pendant que Quirinius était gouverneur de Syrie.
\VS{3}Ainsi, tous allaient pour s'inscrire, chacun dans sa ville.
\VS{4}Joseph aussi monta de Galilée en Judée, de la ville de Nazareth, la ville de David, appelée Bethléhem, parce qu'il était de la maison et de la famille de David,
\VS{5}afin de se faire inscrire avec Marie, sa fiancée, qui était enceinte.
\VS{6}Pendant qu'ils étaient là, le temps où Marie devait accoucher arriva,
\VS{7}et elle enfanta son fils, premier-né. Et elle l'emmaillota, et le coucha dans une crèche, parce qu'il n'y avait point de place pour eux dans l'hôtellerie.
\TextTitle{L'Ange du Seigneur annonce la naissance de Jésus}
\VS{8}Et il y avait, dans cette même contrée, des bergers qui couchaient dans les champs, et qui gardaient leur troupeau pendant les veilles de la nuit.
\VS{9}Et voici, l'Ange du Seigneur survint vers eux, et la gloire du Seigneur resplendit autour d'eux. Et ils furent saisis d'une grande peur.
\VS{10}Mais l'Ange leur dit~: Ne craignez point~; car, voici je vous annonce une bonne nouvelle, qui sera un sujet de joie pour tout le peuple~:
\VS{11}C'est qu'aujourd'hui, dans la ville de David, vous est né le Sauveur, qui est le Christ, le Seigneur.
\VS{12}Et voici à quel signe vous le reconnaîtrez~: Vous trouverez le petit enfant emmailloté et couché dans une crèche.
\VS{13}Et aussitôt il se joignit à l'Ange une multitude de l'armée céleste, louant Dieu et disant~:
\VS{14}Gloire soit à Dieu dans les lieux très-hauts, que la paix soit sur la terre et la bonne volonté dans les hommes~!
\TextTitle{Les bergers de Bethléhem}
\VS{15}Et il arriva qu'après que les anges s'en furent allés d'avec eux au ciel, les bergers se dirent les uns aux autres~: Allons donc jusqu'à Bethléhem, et voyons cette chose qui est arrivée, ce que le Seigneur nous a fait connaître.
\VS{16}Ils y allèrent donc en hâte, et ils trouvèrent Marie et Joseph, et le petit enfant couché dans une crèche.
\VS{17}Après l'avoir vu, ils divulguèrent ce qui leur avait été dit au sujet de ce petit enfant.
\VS{18}Tous ceux qui les entendirent furent dans l'étonnement de ce que leur disaient les bergers.
\VS{19}Et Marie gardait soigneusement toutes ces choses, et les repassait dans son esprit.
\VS{20}Puis les bergers s'en retournèrent, glorifiant et louant Dieu pour tout ce qu'ils avaient entendu et vu, et qui était conforme à ce qui leur avait été annoncé.
\TextTitle{Jésus circoncis et présenté au temple de Jérusalem\FTNTT{Ex. 13:12,15.}}
\VS{21}Et quand les huit jours furent accomplis pour circoncire l'enfant, on lui donna le Nom de Jésus, nom qu'avait indiqué l'Ange avant qu'il soit conçu dans le sein de sa mère.
\VS{22}Et, quand les jours de la purification\FTNT{Lé. 12:2-6.} de Marie furent accomplis, selon la loi de Moïse, Joseph et Marie le portèrent à Jérusalem, pour le présenter au Seigneur,
\VS{23}selon ce qui est écrit dans la loi du Seigneur~: Tout mâle premier-né sera appelé Saint au Seigneur\FTNT{Ex. 13:2~; Ex. 13:12~; No. 3:13~; No. 8:17.}~; 
\VS{24}et pour offrir en sacrifice deux tourterelles ou deux jeunes pigeons, comme cela est prescrit dans la loi du Seigneur\FTNT{Lé. 12:8.}.
\TextTitle{Prophétie de Siméon}
\VS{25}Et voici, il y avait à Jérusalem un homme appelé Siméon. Et cet homme était juste et pieux, il attendait la consolation d'Israël, et le Saint-Esprit était sur lui.
\VS{26}Il avait été averti divinement par le Saint-Esprit qu'il ne mourrait point avant d'avoir vu le Christ du Seigneur.
\VS{27}Il vint au temple, poussé par l'Esprit. Et, comme les parents apportaient dans le temple l'enfant Jésus, pour accomplir à son égard ce qu'ordonnait la loi,
\VS{28}il le prit dans ses bras, bénit Dieu, et dit~:
\VS{29}Seigneur, tu laisses maintenant ton serviteur s'en aller en paix, selon ta parole.
\VS{30}Car mes yeux ont vu ton salut,
\VS{31}lequel tu as préparé devant la face de tous les peuples~;
\VS{32}la lumière pour éclairer les nations et pour être la gloire de ton peuple d'Israël.
\VS{33}Son père et sa mère s'étonnaient des choses qui étaient dites de lui.
\VS{34}Siméon les bénit, et dit à Marie, sa mère~: Voici, cet enfant est établi pour la chute et pour la résurrection de beaucoup en Israël, et pour être un signe qui provoquera la contradiction,
\VS{35}en sorte que les pensées de beaucoup de cœurs seront découvertes. Et pour toi, une épée te transpercera l'âme.
\TextTitle{Anne témoigne du Messie}
\VS{36}Il y avait aussi Anne, la prophétesse, fille de Phanuel, de la tribu d'Aser, qui était déjà avancée en âge, et qui avait vécu avec son mari sept ans depuis sa virginité.
\VS{37}Restée veuve, et âgée d'environ quatre-vingt-quatre ans, elle ne quittait pas le temple, et elle servait Dieu nuit et jour dans le jeûne et dans les prières.
\VS{38}Etant arrivée à cette heure, elle louait aussi Dieu, et parlait de lui à tous ceux qui attendaient la délivrance de Jérusalem.
\TextTitle{Retour à Nazareth\FTNTT{Suite aux événements de Mt. 2.}}
\VS{39}Et quand ils eurent accompli tout ce qui est ordonné par la loi du Seigneur, ils s'en retournèrent en Galilée, à Nazareth, leur ville.
\VS{40}Et le petit enfant croissait et se fortifiait en esprit. Il était rempli de sagesse, et la grâce de Dieu était sur lui.
\TextTitle{Jésus dans le temple de Jérusalem, assis au milieu des docteurs}
\VS{41}Ses parents allaient tous les ans à Jérusalem, à la fête de Pâque.
\VS{42}Lorsqu'il fut âgé de douze ans, ses parents montèrent à Jérusalem selon la coutume de la fête.
\VS{43}Puis, quand les jours furent écoulés, et qu'ils s'en retournèrent, l'enfant Jésus resta à Jérusalem. Et son père et sa mère ne s'en aperçurent point.
\VS{44}Mais croyant qu'il était avec leurs compagnons de voyage, ils marchèrent une journée, puis ils le cherchèrent parmi leurs parents et parmi leurs connaissances.
\VS{45}Et ne le trouvant point, ils retournèrent à Jérusalem, pour le chercher.
\VS{46}Or il arriva que trois jours après, ils le trouvèrent dans le temple, assis au milieu des docteurs, les écoutant et les interrogeant.
\VS{47}Tous ceux qui l'entendaient s'étonnaient de sa sagesse et de ses réponses.
\VS{48}Quand ses parents le virent, ils furent saisis d'étonnement, et sa mère lui dit~: Mon enfant, pourquoi nous as-tu fait ainsi~? Voici, ton père et moi te cherchions avec angoisse.
\VS{49}Et il leur dit~: Pourquoi me cherchiez-vous~? Ne saviez-vous pas qu'il faut que je m'occupe des affaires de mon Père~?
\VS{50}Mais ils ne comprirent point ce qu'il leur disait.
\VS{51}Alors il descendit avec eux et vint à Nazareth, et il leur était soumis. Et sa mère gardait toutes ces paroles dans son cœur.
\VS{52}Et Jésus croissait en sagesse, en stature, et en grâce, auprès de Dieu et devant les hommes.
\Chap{3}
\TextTitle{Jean-Baptiste, l'envoyé de Dieu\FTNTT{Mt. 3:1-12~; Mc. 1:1-8~; Jn. 1:6-8,15-37.}}
\VerseOne{}La quinzième année du règne de Tibère César, lorsque Ponce Pilate était gouverneur de la Judée, Hérode tétrarque de la Galilée, et son frère Philippe tétrarque de l'Iturée et du territoire de la Trachonite, et Lysanias tétrarque de l'Abilène,
\VS{2}et du temps des grands-prêtres Anne et Caïphe, la parole de Dieu fut adressée à Jean, fils de Zacharie, dans le désert.
\VS{3}Et il alla dans tout le pays des environs du Jourdain, prêchant le baptême de repentance, pour la rémission des péchés,
\VS{4}comme il est écrit dans le livre des paroles d'Esaïe, le prophète, disant~: C'est la voix de celui qui crie dans le désert~: Préparez le chemin du Seigneur, aplanissez ses sentiers.
\VS{5}Toute vallée sera comblée, et toute montagne et toute colline seront abaissées~; et ce qui est tortueux sera redressé, et les chemins raboteux seront aplanis.
\VS{6}Et toute chair verra le salut de Dieu\FTNT{Es. 40:3-5.}.
\VS{7}Il disait donc à ceux qui venaient en foule pour être baptisés par lui~: Races de vipères, qui vous a appris à fuir la colère à venir~?
\VS{8}Produisez donc des fruits dignes de la repentance, et ne vous mettez point à dire en vous-mêmes~: Nous avons Abraham pour père~! Car je vous dis que Dieu peut faire naître, même de ces pierres, des enfants à Abraham.
\VS{9}Or la cognée est déjà mise à la racine des arbres~; tout arbre donc qui ne produit pas de bons fruits, sera coupé et jeté au feu.
\VS{10}Alors la foule l'interrogeait, disant~: Que ferons-nous donc~?
\VS{11}Et il répondit, et leur dit~: Que celui qui a deux tuniques partage avec celui qui n'en a point et que celui qui a de quoi manger en fasse de même.
\VS{12}Il vint aussi à lui des publicains pour être baptisés, et ils lui dirent~: Maître, que ferons-nous~?
\VS{13}Et il leur dit~: N'exigez rien au-delà de ce qui vous a été ordonné.
\VS{14}Des soldats l'interrogèrent aussi, disant~: Et nous, que ferons-nous~? Et il leur répondit~: Ne commettez ni extorsion ni fraude envers personne, mais contentez-vous de votre solde.
\VS{15}Et comme le peuple était dans l'attente, et que tous se demandaient dans leurs cœurs si Jean n'était pas le Christ,
\VS{16}Jean prit la parole, et dit à tous~: Moi, je vous baptise d'eau~; mais il vient, celui qui est plus puissant que moi, et je ne suis pas digne de délier la courroie de ses souliers. Lui, il vous baptisera du Saint-Esprit et de feu.
\VS{17}Il a son van à la main~; et il nettoiera entièrement son aire, et amassera le froment dans son grenier, mais il brûlera la paille dans un feu qui ne s'éteint point.
\VS{18}Et faisant aussi plusieurs autres exhortations, il évangélisait le peuple.
\VS{19}Mais Hérode le tétrarque, étant repris par Jean au sujet d'Hérodias, femme de Philippe, son frère, et à cause de toutes les choses méchantes qu'il faisait,
\VS{20}ajouta encore à toutes les autres celle de mettre Jean en prison.
\TextTitle{Le baptême de Jésus-Christ\FTNTT{Mt. 3:13-17~; Mc. 1:9-11~; Jn. 1:31-34.}}
\VS{21}Tout le peuple se faisait baptiser, Jésus aussi fut baptisé et pendant qu'il priait, le ciel s'ouvrit,
\VS{22}et le Saint-Esprit descendit sur lui sous une forme corporelle, comme celle d'une colombe. Et une voix fit entendre du ciel ces paroles~: Tu es mon Fils bien-aimé, en toi j'ai trouvé mon plaisir.
\TextTitle{La généalogie de Jésus-Christ\FTNTT{v. 31~; Mt. 1:1-16.}}
\VS{23}Jésus avait environ trente ans, lorsqu'il commença son service, étant comme on l'estimait, fils de Joseph, fils d'Héli,
\VS{24}fils de Matthat, fils de Lévi, fils de Melchi, fils de Jannaï, fils de Joseph,
\VS{25}fils de Mattathias, fils d'Amos, fils de Nahum, fils d'Esli, fils de Naggaï,
\VS{26}fils de Maath, fils de Mattathias, fils de Sémeï, fils de Josech, fils de Joda,
\VS{27}fils de Joanan, fils de Rhésa, fils de Zorobabel, fils de Salathiel, fils de Néri,
\VS{28}fils de Melchi, fils d'Addi, fils de Kosam, fils d'Elmadam, fils d'Er,
\VS{29}fils de Jésus, fils d'Eliézer, fils de Jorim, fils de Matthat, fils de Lévi,
\VS{30}fils de Siméon, fils de Juda, fils de Joseph, fils de Jonam, fils d'Eliakim,
\VS{31}fils de Méléa, fils de Menna, fils de Mattatha, fils de Nathan, fils de David,
\VS{32}fils d'Isaï, fils d'Obed, fils de Boaz, fils de Salmon, fils de Naasson,
\VS{33}fils d'Aminadab, fils d'Arni, fils d'Esrom, fils de Pharès, fils de Juda,
\VS{34}fils de Jacob, fils d'Isaac, fils d'Abraham, fils de Thara, fils de Nachor,
\VS{35}fils de Seruch, fils de Ragau, fils de Phalek, fils d'Eber, fils de Sala,
\VS{36}fils de Kaïnam, fils d'Arphaxad, fils de Sem, fils de Noé, fils de Lamech,
\VS{37}fils de Mathusala, fils d'Hénoc, fils de Jared, fils de Maléléel, fils de Kaïnan,
\VS{38}fils d'Enos, fils de Seth, fils d'Adam, fils de Dieu.
\Chap{4}
\TextTitle{La tentation de Jésus-Christ\FTNTT{Mt. 4:1~; Mc. 1:12-13~; cp. Ge. 3:6~; 1 Jn. 2:16.}}
\VerseOne{}Jésus, rempli du Saint-Esprit, revint du Jourdain, et il fut conduit par l'Esprit dans le désert,
\VS{2}où il fut tenté par le diable quarante jours. Et il ne mangea rien durant ces jours-là, et après qu'ils furent écoulés, il eut faim.
\VS{3}Le diable lui dit~: Si tu es le Fils de Dieu, ordonne à cette pierre qu'elle devienne du pain.
\VS{4}Jésus lui répondit, en disant~: Il est écrit que l'homme ne vivra pas seulement de pain, mais de toute parole de Dieu\FTNT{De. 8:3.}.
\VS{5}Alors le diable l'emmena sur une haute montagne, et lui montra en un instant tous les royaumes de la terre,
\VS{6}et le diable lui dit~: Je te donnerai toute cette puissance et leur gloire~; car elle m'a été donnée, et je la donne à qui je veux.
\VS{7}Si donc tu m'adores, elle sera à toi.
\VS{8}Jésus lui répondit~: Va arrière de moi, Satan~! Car il est écrit~: Tu adoreras le Seigneur ton Dieu, et tu le serviras lui seul\FTNT{De. 6:13.}.
\VS{9}Le diable le conduisit encore à Jérusalem, et le plaça sur le haut du temple, et lui dit~: Si tu es le Fils de Dieu, jette-toi d'ici en bas~;
\VS{10}car il est écrit~: Il ordonnera à ses anges à ton sujet, afin qu'ils te gardent~;
\VS{11}et ils te porteront sur leurs mains, de peur que ton pied ne heurte contre une pierre\FTNT{Ps. 91:11-12.}.
\VS{12}Mais Jésus répondant, lui dit~: Il est écrit~: Tu ne tenteras pas le Seigneur, ton Dieu\FTNT{De. 6:16.}.
\VS{13}Après l'avoir tenté de toutes ces manières, le diable s'éloigna de lui pour un temps.
\TextTitle{Jésus-Christ retourne en Galilée\FTNTT{Mt. 4:12-17~; Mc. 1:14-15.}}
\VS{14}Jésus retourna en Galilée dans la puissance de l'Esprit, et sa renommée se répandit dans tout le pays d'alentour.
\VS{15}Il enseignait dans leurs synagogues, et il était glorifié par tous.
\TextTitle{Jésus lit le rouleau d'Esaïe\FTNTT{Cp. Mt. 13:54-58~; Mc. 6:1-6.}}
\VS{16}Il se rendit à Nazareth, où il avait été élevé, et selon sa coutume, il entra dans la synagogue le jour du sabbat. Et il se leva pour faire la lecture,
\VS{17}et on lui donna le livre du prophète Esaïe. Et l'ayant déroulé, il trouva le passage où il est écrit~:
\VS{18}L'Esprit du Seigneur est sur moi, parce qu'il m'a oint pour évangéliser les pauvres~; il m'a envoyé pour guérir ceux qui ont le cœur brisé,
\VS{19}pour proclamer aux captifs la délivrance, et aux aveugles le recouvrement de la vue, pour mettre en liberté les opprimés, pour publier une année de grâce du Seigneur\FTNT{Es. 61:1-2.}.
\VS{20}Ensuite, il roula le livre, le rendit au serviteur, et s'assit. Les yeux de tous ceux qui étaient dans la synagogue étaient fixés sur lui.
\VS{21}Alors il commença à leur dire~: Aujourd'hui, cette parole de l'Ecriture, que vous venez d'entendre, est accomplie.
\VS{22}Et tous lui rendaient témoignage, et s'étonnaient des paroles pleines de grâce qui sortaient de sa bouche~; et ils disaient~: Celui-ci n'est-il pas le fils de Joseph~?
\VS{23}Et il leur dit~: Assurément vous me direz ce proverbe~: Médecin, guéris-toi toi-même. Et fais ici, dans ton pays, tout ce que nous avons appris que tu as fait à Capernaüm.
\VS{24}Mais, il leur dit~: En vérité je vous dis qu'aucun prophète n'est reçu dans son pays.
\VS{25}Je vous le dis en vérité~: Il y avait plusieurs veuves en Israël du temps d'Elie, lorsque le ciel fut fermé trois ans et six mois et qu'il y eut une grande famine dans tout le pays.
\VS{26}Toutefois Elie ne fut envoyé vers aucune d'elles, mais seulement vers une femme veuve à Sarepta, dans le pays de Sidon.
\VS{27}Il y avait aussi plusieurs lépreux en Israël du temps d'Elisée, le prophète~; toutefois aucun d'eux ne fut purifié, si ce n'est Naaman, le Syrien.
\VS{28}Ils furent tous remplis de colère dans la synagogue lorsqu'ils entendirent ces choses.
\VS{29}Et s'étant levés, ils le chassèrent hors de la ville, et le menèrent jusqu'au bord de la montagne sur laquelle leur ville était bâtie, pour le jeter du haut en bas.
\VS{30}Mais il passa au milieu d'eux, et s'en alla.
\TextTitle{Guérison d'un démoniaque\FTNTT{Mc. 1:21-28.}}
\VS{31}Il descendit à Capernaüm, ville de Galilée, et il les enseignait les jours de sabbat.
\VS{32}Ils étaient frappés de sa doctrine~; car il parlait avec autorité.
\VS{33}Il y avait dans la synagogue un homme qui avait un esprit de démon impur, et qui s'écria d'une voix forte,
\VS{34}en disant~: Ah~! Qu'y a-t-il entre nous et toi, Jésus de Nazareth~? Es-tu venu pour nous détruire~? Je sais qui tu es, le Saint de Dieu.
\VS{35}Jésus le menaça, en lui disant~: Tais-toi, et sors de cet homme. Et le démon, l'ayant jeté avec impétuosité au milieu de l'assemblée, sortit de cet homme, sans lui faire aucun mal.
\VS{36}Et tous furent saisis de stupeur, et ils parlaient entre eux, et disaient~: Quelle est cette parole~? Car il commande avec autorité et puissance aux esprits impurs, et ils sortent~! 
\VS{37}Et sa renommée se répandit dans tous les lieux d'alentour.
\TextTitle{Guérison de la belle-mère de Pierre et de plusieurs malades\FTNTT{Mt. 8:14-17~; Mc. 1:29-34.}}
\VS{38}Et quand Jésus se fut levé de la synagogue, il se rendit à la maison de Simon. Et la belle-mère de Simon avait une violente fièvre, et ils le prièrent en sa faveur.
\VS{39}Et s'étant penché sur elle, il menaça la fièvre, et la fièvre la quitta. A l'instant elle se leva, et les servit.
\VS{40}Et après le coucher du soleil, tous ceux qui avaient des malades atteints de diverses maladies, les lui amenèrent. Et il imposa les mains à chacun d'eux, et il les guérit.
\VS{41}Les démons aussi sortirent de beaucoup de personnes, en criant et en disant~: Tu es le Christ, le Fils de Dieu. Mais il les menaçait fortement, et ne leur permettait pas de dire qu'ils savaient qu'il était le Christ.
\VS{42}Dès que le jour parut, il sortit et alla dans un lieu désert. Et une foule de gens se mirent à sa recherche, et arrivèrent jusqu'à lui et ils voulaient le retenir, afin qu'il ne les quittât point.
\VS{43}Mais il leur dit~: Il faut que j'annonce aux autres villes l'Evangile du Royaume de Dieu, car c'est pour cela que j'ai été envoyé.
\VS{44}Et il prêchait dans les synagogues de la Galilée.
\Chap{5}
\TextTitle{L'appel des premiers disciples\FTNTT{Mt. 4:18-22~; Mc. 1:16-20~; cp. Jn. 1:35-51~; 21:1-8.}}
\VerseOne{}Or il arriva, comme la foule se jetait toute sur lui pour entendre la parole de Dieu, qu'il se tenait sur le bord du lac de Génézareth.
\VS{2}Et voyant deux barques qui étaient au bord du lac, et dont les pêcheurs étaient descendus et lavaient leurs filets, il monta dans l'une de ces barques, qui était à Simon,
\VS{3}et il le pria de s'éloigner un peu de terre. Puis il s'assit, et de la barque il enseignait la foule.
\VS{4}Et quand il eut cessé de parler, il dit à Simon~: Avance en pleine eau, et jetez vos filets pour pêcher.
\VS{5}Et Simon répondant, lui dit~: Maître, nous avons travaillé toute la nuit, et nous n'avons rien pris~; toutefois à ta parole je jetterai les filets.
\VS{6}Et ayant fait cela, ils prirent une si grande quantité de poissons que leurs filets se rompaient.
\VS{7}Et ils firent signe à leurs compagnons qui étaient dans l'autre barque, de venir les aider~; et étant venus, ils remplirent les deux barques, tellement qu'elles s'enfonçaient.
\VS{8}Et quand Simon Pierre vit cela, il se jeta aux genoux de Jésus, en lui disant~: Seigneur, retire-toi de moi, car je suis un homme pécheur.
\VS{9}Parce que la frayeur l'avait saisi, lui et tous ceux qui étaient avec lui, à cause de la prise de poissons qu'ils venaient de faire. Il en fut de même pour Jacques et Jean, fils de Zébédée, qui étaient compagnons de Simon.
\VS{10}Alors Jésus dit à Simon~: Ne crains point~; désormais tu seras un pêcheur d'hommes vivants.
\VS{11}Et quand ils eurent ramené les barques à terre, ils abandonnèrent tout et le suivirent.
\TextTitle{Guérison d'un lépreux\FTNTT{Mt. 8:2-4~; Mc. 1:40-45.}}
\VS{12}Et il arriva que comme il était dans une des villes, voici un homme plein de lèpre, voyant Jésus, se jeta sur sa face et le supplia, disant~: Seigneur, si tu veux, tu peux me rendre pur.
\VS{13}Jésus étendit la main, et le toucha, en disant~: Je le veux, sois pur. Aussitôt la lèpre le quitta.
\VS{14}Et il lui commanda de ne le dire à personne. Mais va, lui dit-il, et montre-toi au prêtre, et offre pour ta purification ce que Moïse a commandé\FTNT{Lé. 13 et 14.} pour leur servir de témoignage.
\VS{15}Et sa renommée se répandait de plus en plus, tellement que de grandes foules s'assemblaient pour l'entendre, et pour être guéries par lui de leurs maladies.
\VS{16}Mais il se tenait retiré dans les déserts, et priait.
\TextTitle{Guérison d'un paralytique\FTNTT{Mt. 9:2-8~; Mc. 2:3-12.}}
\VS{17}Un jour Jésus enseignait. Et des pharisiens et des docteurs de la loi étaient là assis, venus de tous les villages de la Galilée, et de la Judée et de Jérusalem~; et la puissance du Seigneur se manifestait par des guérisons.
\VS{18}Et voici, des hommes qui portaient sur un lit un homme qui était paralytique, cherchaient le moyen de le porter dans la maison et de le mettre devant lui.
\VS{19}Comme ils ne savaient pas par où l'introduire, à cause de la foule, ils montèrent sur le toit, et ils le descendirent par une ouverture, avec son lit, au milieu de la foule, devant Jésus.
\VS{20}Voyant leur foi, il dit au paralytique~: Homme, tes péchés te sont pardonnés.
\VS{21}Alors les scribes et les pharisiens commencèrent à raisonner en eux-mêmes, disant~: Qui est celui-ci qui profère des blasphèmes~? Qui est-ce qui peut pardonner les péchés, si ce n'est Dieu seul~?
\VS{22}Mais Jésus, connaissant leurs pensées, prit la parole et leur dit~: Pourquoi raisonnez-vous ainsi en vous-mêmes~?
\VS{23}Lequel est le plus aisé de dire~: Tes péchés te sont pardonnés~; ou de dire~: Lève-toi et marche~?
\VS{24}Or afin que vous sachiez que le Fils de l'homme a le pouvoir sur la terre de pardonner les péchés, il dit au paralytique~: Je te l'ordonne, lève-toi, prends ton lit, et va dans ta maison.
\VS{25}Et à l'instant, le paralytique s'étant levé devant eux, prit le lit sur lequel il était couché, et s'en alla dans sa maison, glorifiant Dieu.
\VS{26}Ils furent tous saisis d'étonnement, et ils glorifiaient Dieu~; et étant remplis de crainte, ils disaient~: certainement nous avons vu aujourd'hui des choses étranges.
\TextTitle{Appel de Lévi\FTNTT{Mt. 9:9~; Mc. 2:13-14.}}
\VS{27}Après cela, Jésus sortit, et il vit un publicain nommé Lévi, assis au bureau des péages, et il lui dit~: Suis-moi.
\VS{28}Et abandonnant tout, il se leva, et le suivit.
\TextTitle{Jésus appelle les pécheurs à la repentance\FTNTT{Mt. 9:10-15~; Mc. 2:13-14.}}
\VS{29}Et Lévi lui fit un grand festin dans sa maison~; et il y avait une grande foule de publicains et d'autres gens qui étaient avec eux à table.
\VS{30}Les scribes de ce lieu-là et les pharisiens, murmuraient contre ses disciples en disant~: Pourquoi mangez-vous et buvez-vous avec les publicains et les gens de mauvaise vie~?
\VS{31}Mais Jésus, prenant la parole, leur dit~: Ceux qui sont en santé n'ont pas besoin de médecin, mais ceux qui se portent mal.
\VS{32}Je ne suis point venu appeler à la repentance les justes, mais les pécheurs.
\VS{33}Ils lui dirent aussi~: Pourquoi est-ce que les disciples de Jean jeûnent souvent, et font des prières, et également ceux des Pharisiens~; mais les tiens mangent et boivent~?
\VS{34}Il leur répondit~: Pouvez-vous faire jeûner les amis de l'Epoux pendant que l'Epoux est avec eux~?
\VS{35}Mais les jours viendront où l'Epoux leur sera enlevé alors ils jeûneront en ces jours-là.
\TextTitle{Parabole du drap neuf et des outres neuves\FTNTT{Mt. 9:16-17~; Mc. 2:21-22.}}
\VS{36}Puis il leur dit cette parabole~: Personne ne met une pièce d'un habit neuf à un vieil habit~; autrement le neuf déchire le vieux, et la pièce du neuf ne s'accorde pas avec le vieux.
\VS{37}Et personne ne met du vin nouveau dans de vieilles outres~; autrement le vin nouveau fait rompre les outres, et il se répand, et les outres sont perdues.
\VS{38}Mais le vin nouveau doit être mis dans des outres neuves~; et ainsi ils se conservent l'un et l'autre.
\VS{39}Et personne, après avoir bu du vin vieux, ne veut du nouveau, car il dit~: Le vieux est meilleur.
\Chap{6}
\TextTitle{Jésus, le Maître du sabbat\FTNTT{Mt. 12:1-8~; Mc. 2:23-28.}}
\VerseOne{}Or il arriva un jour de sabbat, appelé second-premier, qu'il passait par des blés, et ses disciples arrachaient des épis et les froissant dans leurs mains, ils les mangeaient.
\VS{2}Et quelques pharisiens leur dirent~: Pourquoi faites-vous ce qu'il n'est pas permis de faire les jours du sabbat~?
\VS{3}Et Jésus prenant la parole, leur dit~: N'avez-vous pas lu ce que fit David quand il eut faim, lui et ceux qui étaient avec lui~;
\VS{4}comment il entra dans la maison de Dieu, et prit les pains de proposition, et en mangea, et en donna aussi à ceux qui étaient avec lui, bien qu'il ne soit permis qu'aux seuls prêtres d'en manger~?\FTNT{1 S. 21:1-7.}
\VS{5}Puis il leur dit~: Le Fils de l'homme est Maître même du sabbat.
\TextTitle{Guérison de l'homme à la main sèche\FTNTT{Mt. 12:9-13~; Mc. 3:1-5.}}
\VS{6}Et il arriva, un autre jour de sabbat, qu'il entra dans la synagogue, et qu'il enseignait. Et il s'y trouvait là un homme dont la main droite était sèche.
\VS{7}Or les scribes et les pharisiens l'observaient, pour voir s'il ferait une guérison le jour du sabbat~; c'était afin d'avoir sujet de l'accuser.
\VS{8}Mais il connaissait leurs pensées, et il dit à l'homme qui avait la main sèche~: Lève-toi et tiens-toi debout au milieu. Et il se leva et se tint debout.
\VS{9}Puis Jésus leur dit~: Je vous demande une chose~: Est-il permis de faire du bien les jours de sabbat, ou de faire du mal, de sauver une personne, ou de la laisser mourir~?
\VS{10}Et ayant regardé tous ceux qui étaient autour de lui, il dit à l'homme~: Etends ta main. Ce qu'il fit, et sa main fut rendue saine comme l'autre.
\VS{11}Et ils furent remplis de fureur, et ils s'entretenaient ensemble touchant ce qu'ils pourraient faire à Jésus.
\VS{12}Or il arriva en ces jours-là, qu'il s'en alla sur une montagne pour prier, et qu'il passa toute la nuit à prier Dieu.
\TextTitle{Choix des douze apôtres\FTNTT{Mt. 10:2-4~; Mc. 3:13-19.}}
\VS{13}Et quand le jour fut venu, il appela ses disciples, et en ayant choisi douze d'entre eux, il les nomma apôtres~:
\VS{14}Simon, qu'il nomma Pierre, et André son frère, Jacques et Jean, Philippe et Barthélemy~;
\VS{15}Matthieu et Thomas, Jacques, fils d'Alphée, et Simon, surnommé zélote\FTNT{Le mot zélote signifie «~celui qui est zélé~». Les zélotes avaient parmi leurs modèles Phinées, le fils zélé d'Eléazar, fils d'Aaron. Ce dernier s'était illustré en tuant un prince d'une tribu d'Israël qui avait provoqué la colère de Dieu en se débauchant ouvertement avec une Madianite (No. 25:6-11). Issus des milieux populaires, les zélotes étaient à la fois des adeptes d'une secte religieuse espérant hâter l'accomplissement des promesses divines et des militants politiques qui s'opposaient à l'occupant romain. Pour eux, leur pays ne pouvait être gouverné que par un descendant du roi David. Ils furent à l'origine de la Grande Révolte Juive (66-70) en déclenchant en août 66 un soulèvement sanglant au cours duquel ils s'emparèrent de Jérusalem et décimèrent les grands-prêtres qui collaboraient avec l'occupant. En représailles, Vespasien (9-79) puis Titus (41-81) organisèrent un siège de la ville qui aboutit au saccage de la cité et à la destruction du temple en 70.}~;
\VS{16}Jude, frère de Jacques, et Judas Iscariot, qui devint traître.
\TextTitle{L'enseignement de Jésus sur la montagne\FTNTT{Mt. 5-7.}}
\VS{17}Puis descendant avec eux, il s'arrêta sur une plaine avec la foule de ses disciples et une grande multitude de peuple de toute la Judée, et de Jérusalem, et de la contrée maritime de Tyr et de Sidon, qui étaient venus pour l'entendre, et pour être guéris de leurs maladies.
\VS{18}Ceux aussi qui étaient tourmentés par des esprits impurs furent guéris.
\VS{19}Et toute la foule cherchait à le toucher, parce qu'une force sortait de lui et les guérissait tous.
\VS{20}Alors Jésus, levant les yeux vers ses disciples, leur dit~: Bénis, êtes-vous, vous qui êtes pauvres, car le Royaume de Dieu vous appartient~!
\VS{21}Bénis, êtes-vous, vous qui avez faim maintenant, car vous serez rassasiés~! Bénis, êtes-vous, vous qui pleurez maintenant, car vous serez dans la joie~!
\VS{22}Bénis, serez-vous quand les hommes vous haïront, et vous chasseront, et vous outrageront, et rejetteront votre nom comme infâme, à cause du Fils de l'homme~!
\VS{23}Réjouissez-vous en ce jour-là, et tressaillez d'allégresse, parce que votre récompense sera grande dans le ciel~; car leurs pères en faisaient de même aux prophètes.
\VS{24}Mais, malheur à vous, riches, car vous avez votre consolation~!
\VS{25}Malheur à vous qui êtes rassasiés, car vous aurez faim~! Malheur à vous qui riez maintenant, car vous serez dans le deuil et dans les larmes~!
\VS{26}Malheur à vous quand tous les hommes diront du bien de vous, car leurs pères en faisaient de même aux faux prophètes.
\VS{27}Mais à vous qui m'entendez, je vous dis~: Aimez vos ennemis~; faites du bien à ceux qui vous haïssent,
\VS{28}bénissez ceux qui vous maudissent, et priez pour ceux qui vous maltraitent.
\VS{29}Si quelqu'un te frappe sur une joue, présente-lui aussi l'autre. Et si quelqu'un prend ton manteau, ne l'empêche pas de prendre aussi ta tunique.
\VS{30}Donne à quiconque te demande, et ne réclame pas ton bien à celui qui s'en empare.
\VS{31}Ce que vous voulez que les hommes fassent pour vous, faites-le de même pour eux.
\VS{32}Mais si vous aimez seulement ceux qui vous aiment, quel gré vous en saura-t-on~? Les pécheurs aussi aiment ceux qui les aiment.
\VS{33}Et si vous faites du bien à ceux qui vous font du bien, quel gré vous en saura-t-on~? Les pécheurs aussi font de même.
\VS{34}Et si vous prêtez à ceux de qui vous espérez recevoir, quel gré vous en saura-t-on~? Les pécheurs aussi prêtent aux pécheurs, afin de recevoir la pareille.
\VS{35}C'est pourquoi aimez vos ennemis, et faites-leur du bien, et prêtez sans rien espérer, et votre récompense sera grande, et vous serez les fils du Très-Haut, car il est bon envers les ingrats et les méchants.
\VS{36}Soyez donc miséricordieux, comme aussi votre Père est miséricordieux.
\VS{37}Ne jugez point, et vous ne serez point jugés~; ne condamnez point, et vous ne serez point condamnés~; absolvez, et vous serez absous.
\VS{38}Donnez, et il vous sera donné~: On versera dans votre sein une bonne mesure, serrée, et secouée et qui déborde~; car on vous mesurera avec la mesure dont vous vous serez servis.
\VS{39}Il leur disait aussi cette parabole~: Un aveugle peut-il conduire un aveugle~? Ne tomberont-ils pas tous deux dans la fosse~?
\VS{40}Le disciple n'est pas au-dessus de son maître~; mais tout disciple accompli sera comme son maître.
\VS{41}Pourquoi regardes-tu la paille qui est dans l'œil de ton frère, et n'aperçois-tu pas la poutre qui est dans ton propre œil~?
\VS{42}Ou comment peux-tu dire à ton frère~: Mon frère, laisse-moi enlever la paille qui est dans ton œil, toi qui ne vois pas la poutre qui est dans ton œil~? Hypocrite, ôte premièrement la poutre de ton œil, et après cela tu verras comment ôter la paille qui est dans l'œil de ton frère.
\VS{43}Ce n'est pas un bon arbre qui porte du mauvais fruit, ni un mauvais arbre qui porte du bon fruit.
\VS{44}Car chaque arbre se reconnaît à son fruit. On ne cueille pas des figues sur des épines, et l'on ne vendange pas des raisins sur des ronces.
\VS{45}L'homme de bien tire de bonnes choses du bon trésor de son cœur, et l'homme méchant tire de mauvaises choses du mauvais trésor de son cœur~; car c'est de l'abondance du cœur que la bouche parle.
\TextTitle{Parabole des deux maisons\FTNTT{Mt. 7:24-27.}}
\VS{46}Mais pourquoi m'appelez-vous Seigneur, Seigneur~! Et ne faites-vous pas ce que je dis~?
\VS{47}Je vous montrerai à qui est semblable celui qui vient à moi, entend mes paroles, et les met en pratique.
\VS{48}Il est semblable à un homme qui bâtissant une maison, a creusé, creusé profondément, et a mis le fondement sur le roc. Mais une inondation est venue, et le torrent s'est jeté contre cette maison, sans pouvoir l'ébranler, parce qu'elle était bâtie sur le roc.
\VS{49}Mais celui qui entend mes paroles, et ne les met pas en pratique est semblable à un homme qui a bâti sa maison sur la terre, sans fondement. Le torrent s'est jeté contre elle~; et aussitôt elle est tombée, et la ruine de cette maison a été grande.
\Chap{7}
\TextTitle{Guérison du serviteur d'un centenier\FTNTT{Mt. 8:5-13.}}
\VerseOne{}Et quand il eut achevé tout ce discours devant le peuple qui l'écoutait, il entra dans Capernaüm.
\VS{2}Un centenier avait un serviteur, auquel il était très attaché, et qui était malade, sur le point de mourir.
\VS{3}Ayant entendu parler de Jésus, il envoya vers lui quelques anciens des Juifs pour le prier de venir guérir son serviteur.
\VS{4}Et étant venu à Jésus, ils le prièrent instamment, disant~: Il mérite que tu lui accordes cela~;
\VS{5}car, disaient-ils, il aime notre nation, et c'est lui qui a bâti notre synagogue.
\VS{6}Jésus s'en alla donc avec eux, et il n'était guère éloigné de la maison quand le centenier envoya ses amis au-devant de lui pour lui dire~: Seigneur, ne te fatigue point~; car je ne suis pas digne que tu entres sous mon toit.
\VS{7}C'est pourquoi aussi je ne me suis pas cru digne d'aller moi-même vers toi. Mais dis seulement une parole, et mon serviteur sera guéri.
\VS{8}Car, moi-même qui suis un homme soumis à des supérieurs, j'ai des soldats sous mes ordres~; et je dis à l'un~: Va~! Et il va~; et à un autre~: Viens~! Et il vient~; et à mon serviteur~: Fais cela~! Et il le fait.
\VS{9}Lorsque Jésus entendit ces paroles, il s'étonna à son sujet. Et se tournant vers la foule qui le suivait, il dit~: Je vous le dis, je n'ai pas trouvé, même en Israël, une si grande foi.
\VS{10}Et quand ceux qui avaient été envoyés furent de retour à la maison, ils trouvèrent le serviteur qui avait été malade, se portant bien.
\TextTitle{Résurrection du fils de la veuve de Naïn}
\VS{11}Le jour suivant, Jésus alla dans une ville appelée Naïn~; et plusieurs de ses disciples et une grande foule allaient avec lui.
\VS{12}Et comme il approchait de la porte de la ville, voici, on portait en terre un mort, fils unique de sa mère, qui était veuve~; et il y avait avec elle un grand nombre de gens de la ville.
\VS{13}Le Seigneur, l'ayant vue, fut ému de compassion pour elle, et il lui dit~: Ne pleure pas~!
\VS{14}Il s'approcha, et toucha le cercueil. Et ceux qui le portaient s'arrêtèrent. Et il dit~: Jeune homme, je te dis, lève-toi~!
\VS{15}Et le mort s'assit, et se mit à parler. Et Jésus le rendit à sa mère.
\VS{16}Et ils furent tous saisis de crainte, et ils glorifiaient Dieu, disant~: Certainement un grand prophète a paru parmi nous~; et Dieu a visité son peuple.
\VS{17}Cette parole sur ce miracle se répandit dans toute la Judée, et dans tout le pays d'alentour.
\TextTitle{Jean-Baptiste, le messager envoyé pour préparer la voie du Seigneur\FTNTT{Mt. 11:1-19.}}
\VS{18}Jean fut informé de toutes ces choses par ses disciples.
\VS{19}Il en appela deux, et les envoya vers Jésus pour lui dire~: Es-tu celui qui devait venir, ou devons-nous en attendre un autre~?
\VS{20}Et étant venus à lui, ils lui dirent~: Jean-Baptiste nous a envoyés auprès de toi, pour te dire~: Es-tu celui qui devait venir, ou devons-nous en attendre un autre~?
\VS{21}A l'heure même, Jésus guérit plusieurs personnes de maladies, et d'infirmités, et d'esprits malins, et il rendit la vue à plusieurs aveugles.
\VS{22}Ensuite Jésus leur répondit, et leur dit~: Allez, et rapportez à Jean ce que vous avez vu et entendu~: Que les aveugles recouvrent la vue, les boiteux marchent, les lépreux sont purifiés, les sourds entendent, les morts ressuscitent, l'Evangile est annoncé aux pauvres.
\VS{23}Bénis est celui qui n'aura point été scandalisé à cause de moi~!
\VS{24}Lorsque les messagers de Jean furent partis, Jésus se mit à dire à la foule, au sujet de Jean~: Qu'êtes-vous allés voir au désert~? Un roseau agité par le vent~?
\VS{25}Mais, qu'êtes-vous allés voir~? Un homme vêtu d'habits précieux~? Voici, ceux qui portent des habits magnifiques, et qui vivent dans les délices, sont dans les maisons des rois.
\VS{26}Mais qu'êtes-vous donc allés voir~? Un prophète~? Oui, vous dis-je, et plus qu'un prophète.
\VS{27}C'est de lui qu'il est écrit~: Voici, j'envoie mon messager devant ta face, et il préparera ta voie devant toi\FTNT{Mal. 3:1.}.
\VS{28}Car je vous dis, parmi ceux qui sont nés de femmes, il n'y a aucun prophète plus grand que Jean-Baptiste. Cependant, le plus petit dans le Royaume de Dieu est plus grand que lui.
\VS{29}Et tout le peuple qui entendait cela, et les publicains, justifiaient Dieu, ayant été baptisés du baptême de Jean.
\VS{30}Mais les pharisiens et les docteurs de la loi, qui n'avaient point été baptisés par lui, rendirent le dessein de Dieu inutile à leur égard.
\VS{31}Alors le Seigneur dit~: A qui donc comparerai-je les hommes de cette génération, et à quoi ressemblent-ils~?
\VS{32}Ils sont semblables aux enfants qui sont assis sur la place publique, et qui se parlant les uns aux autres, disent~: Nous vous avons joué de la flûte, et vous n'avez pas dansé~; nous vous avons chanté des complaintes, et vous n'avez pas pleuré.
\VS{33}Car Jean-Baptiste est venu ne mangeant point de pain, et ne buvant point de vin~; et vous dites~: Il a un démon.
\VS{34}Le Fils de l'homme est venu, mangeant et buvant~; et vous dites~: Voici un mangeur et un buveur, un ami des publicains et des pécheurs.
\VS{35}Mais la sagesse a été justifiée par tous ses enfants.
\TextTitle{Une femme pécheresse pardonnée par Jésus}
\VS{36}Un des pharisiens pria Jésus de manger chez lui. Et Jésus entra dans la maison de ce pharisien, et se mit à table.
\VS{37}Et voici, il y avait dans la ville une femme pécheresse, qui ayant su que Jésus était à table dans la maison du pharisien, apporta un vase d'albâtre plein de parfum.
\VS{38}Et se tenant derrière à ses pieds en pleurant, elle les mouilla de ses larmes, puis elle les essuya avec ses propres cheveux, et lui baisa les pieds, et les oignit de cette huile odoriférante.
\VS{39}Mais le pharisien qui l'avait invité, voyant cela, dit en lui-même~: Si cet homme était prophète, certes il saurait qui et de quelle espèce est la femme qui le touche, il saurait que c'est une pécheresse.
\TextTitle{Parabole des deux débiteurs}
\VS{40}Et Jésus prenant la parole, lui dit~: Simon, j'ai quelque chose à te dire. Maître, parle, répondit-il.
\VS{41}Un créancier avait deux débiteurs~: L'un lui devait cinq cents deniers, et l'autre cinquante.
\VS{42}Et comme ils n'avaient pas de quoi payer, il leur remit à tous deux leur dette. Lequel l'aimera le plus~?
\VS{43}Et Simon répondant lui dit~: Celui, je pense, à qui il a le plus remis. Jésus lui dit~: Tu as droitement jugé.
\VS{44}Alors se tournant vers la femme, il dit à Simon~: Vois-tu cette femme~? Je suis entré dans ta maison, et tu ne m'as point donné d'eau pour laver mes pieds~; mais elle, elle les a mouillés de ses larmes et elle les a essuyés avec ses propres cheveux.
\VS{45}Tu ne m'as point donné un baiser, mais elle, depuis que je suis entré, n'a cessé d'embrasser mes pieds.
\VS{46}Tu n'as pas oint ma tête d'huile~; mais elle, elle a oint mes pieds d'une huile odoriférante.
\VS{47}C'est pourquoi je te le dis, ses nombreux péchés ont été pardonnés, car elle a beaucoup aimé. Or celui à qui on pardonne peu, aime peu.
\VS{48}Puis il dit à la femme~: Tes péchés sont pardonnés.
\VS{49}Ceux qui étaient avec lui à table, se mirent à dire en eux-mêmes~: Qui est celui-ci qui pardonne même les péchés~?
\VS{50}Mais il dit à la femme~: Ta foi t'a sauvée. Va en paix.
\Chap{8}
\VerseOne{}Or il arriva après cela, qu'il allait de ville en ville, et de villages en villages, prêchant et annonçant l'Evangile du Royaume de Dieu.
\VS{2}Les douze disciples étaient auprès de lui avec quelques femmes aussi qu'il avait délivrées d'esprits malins et de maladies~: Marie de Magdala, de laquelle étaient sortis sept démons,
\VS{3}et Jeanne, femme de Chuza, intendant d'Hérode, Susanne, et plusieurs autres qui l'assistaient de leurs biens.
\TextTitle{Parabole du semeur\FTNTT{Mt. 13:1-23~; Mc. 4:1-20.}}
\VS{4}Une grande foule s'étant assemblée, et des gens étant venus de diverses villes auprès de lui, il leur dit cette parabole~:
\VS{5}Un semeur sortit pour semer sa semence. Et en semant, une partie de la semence tomba le long du chemin: Et elle fut foulée aux pieds, et les oiseaux du ciel la mangèrent toute.
\VS{6}Une autre partie tomba dans un endroit pierreux: Et quand elle fut levée, elle sécha, parce qu'elle n'avait point d'humidité.
\VS{7}Une autre partie tomba au milieu des épines: Et les épines crûrent avec elle, et l'étouffèrent.
\VS{8}Une autre partie tomba dans une bonne terre: Et quand elle fut levée, elle donna du fruit au centuple. En disant ces choses, Jésus dit à haute voix~: Que celui qui a des oreilles pour entendre, qu'il entende~!
\VS{9}Et ses disciples l'interrogèrent pour savoir ce que signifiait cette parabole.
\VS{10}Il répondit~: Il vous a été donné de connaître les mystères du Royaume de Dieu, mais pour les autres, cela leur est dit en paraboles, afin qu'en voyant ils ne voient point, et qu'en entendant ils ne comprennent point.
\VS{11}Voici donc ce que signifie cette parabole~: La semence, c'est la parole de Dieu.
\VS{12}Ceux qui ont reçu la semence le long du chemin, ce sont ceux qui entendent la parole~; mais ensuite le diable vient, et ôte la parole de leur cœur, de peur qu'ils ne croient et soient sauvés.
\VS{13}Et ceux qui ont reçu la semence dans un endroit pierreux, ce sont ceux qui, lorsqu'ils entendent la parole, la reçoivent avec joie, mais ils n'ont point de racine~; ils croient pour un temps, mais au moment de la tentation ils se retirent.
\VS{14}Et ce qui est tombé parmi les épines, ce sont ceux qui ayant entendu la parole, s'en vont, et la laissent étouffer par les soucis, les richesses, et les plaisirs de la vie~; et ils ne portent point de fruit qui vienne à maturité.
\VS{15}Mais ce qui est tombé dans une bonne terre, ce sont ceux qui ayant entendu la parole, la retiennent dans un cœur honnête et bon, et portent du fruit avec persévérance.
\TextTitle{Parabole du chandelier\FTNTT{Mt. 5:15-16~; Mc. 4:21-23~; Lu. 11:33-36.}}
\VS{16}Personne, après avoir allumé la lampe, ne la couvre d'un vase, ni ne la met sous un lit, mais il la met sur un chandelier afin que ceux qui entrent voient la lumière.
\VS{17}Car il n'est rien de secret qui ne doive être découvert, rien de caché qui ne doive être connu et qui ne vienne en évidence.
\VS{18}Prenez donc garde à la manière dont vous écoutez~; car on donnera à celui qui a, mais à celui qui n'a pas, on ôtera même ce qu'il croit avoir.
\TextTitle{La famille spirituelle\FTNTT{Mt. 12:46-50~; Mc. 3:31-35.}}
\VS{19}Alors sa mère et ses frères vinrent vers lui~; mais ils ne pouvaient l'aborder à cause de la foule.
\VS{20}Et on vint lui dire~: Ta mère et tes frères sont dehors, et ils désirent te voir.
\VS{21}Mais il répondit~: Ma mère et mes frères sont ceux qui écoutent la parole de Dieu, et qui la mettent en pratique.
\TextTitle{Jésus calme la tempête\FTNTT{Mt. 8:23-27~; Mc. 4:35-41.}}
\VS{22}Or il arriva qu'un jour, Jésus monta dans une barque avec ses disciples, et il leur dit~: Passons de l'autre côté du lac. Et ils partirent.
\VS{23}Pendant qu'ils naviguaient, il s'endormit. Un vent impétueux se leva sur le lac, la barque se remplissait d'eau, et ils étaient en danger.
\VS{24}Ils s'approchèrent et le réveillèrent, en disant~: Maître, Maître~! Nous périssons~! S'étant réveillé, il menaça le vent et les flots qui s'apaisèrent, et le calme revint.
\VS{25}Alors il leur dit~: Où est votre foi~? Saisis de frayeur et d'étonnement, ils se dirent les uns aux autres~: Quel est donc celui-ci, qui commande même aux vents et à l'eau, et à qui ils obéissent~?
\TextTitle{Le démoniaque de Gadara délivré\FTNTT{Mt. 8:28-34~; Mc. 5:1-20.}}
\VS{26}Puis ils abordèrent dans le pays des Gadaréniens qui est vis-à-vis de la Galilée.
\VS{27}Et quand il fut descendu à terre, il vint à sa rencontre un homme de cette ville, qui depuis longtemps était possédé de plusieurs démons. Il ne portait point de vêtements et avait sa demeure, non dans une maison, mais dans les sépulcres.
\VS{28}Ayant vu Jésus, il s'écria et se prosterna devant lui, disant à haute voix~: Qu'y a-t-il entre moi et toi, Jésus, Fils du Dieu Très-Haut~? Je te prie, ne me tourmente point.
\VS{29}Car Jésus commandait à l'esprit impur de sortir de cet homme, dont il s'était emparé depuis longtemps. Et quoique cet homme fût lié de chaînes et gardé dans les fers, il rompait les liens, et il était entrainé par le démon dans les déserts.
\VS{30}Jésus lui demanda~: Quel est ton nom~? Légion\FTNT{Voir commentaire en Mc. 5:9.} répondit-il. Car plusieurs démons étaient entrés en lui.
\VS{31}Et ils priaient Jésus de ne pas leur ordonner d'aller dans l'abîme.
\VS{32}Or il y avait là, dans la montagne, un grand troupeau de pourceaux qui paissaient. Et les démons supplièrent Jésus de leur permettre d'entrer dans ces pourceaux. Il le leur permit.
\VS{33}Les démons sortirent de cet homme et entrèrent dans les pourceaux, et le troupeau se précipita des pentes escarpées dans le lac, et se noya.
\VS{34}Ceux qui les faisaient paître, voyant ce qui était arrivé, s'enfuirent et allèrent le raconter dans la ville et dans les campagnes.
\VS{35}Les gens sortirent pour voir ce qui était arrivé. Ils vinrent auprès de Jésus, et ils trouvèrent l'homme de qui étaient sortis les démons assis aux pieds de Jésus, vêtu et dans son bon sens~; et ils furent saisis de frayeur.
\VS{36}Et ceux qui avaient vu ce qui s'était passé leur racontèrent comment le démoniaque avait été délivré.
\VS{37}Alors toute cette multitude venue de divers endroits voisins des Gadaréniens, le pria de se retirer de chez eux, car ils étaient saisis d'une grande crainte. Jésus monta donc dans la barque, et s'en retourna.
\VS{38}L'homme de qui étaient sortis les démons lui demanda la permission de rester avec lui~; mais Jésus le renvoya, en lui disant~:
\VS{39}Retourne dans ta maison, et raconte tout ce que Dieu t'a fait. Et il s'en alla donc, et publia par toute la ville tout ce que Jésus avait fait pour lui.
\VS{40}Quand Jésus fut de retour, la foule le reçut avec joie, car tous l'attendaient.
\TextTitle{Résurrection de la fille de Jaïrus et guérison de la femme à la perte de sang\FTNTT{Mt. 9:18-26~; Mc. 5:21-43.}}
\VS{41}Et voici, un homme appelé Jaïrus, qui était chef de la synagogue, vint et se jetant aux pieds de Jésus, le pria d'entrer dans sa maison
\VS{42}parce qu'il avait une fille unique, âgée d'environ douze ans, qui se mourait. Pendant que Jésus y allait, il était pressé par la foule.
\VS{43}Or il y avait une femme atteinte d'une perte de sang depuis douze ans, et qui avait dépensé tout son bien pour les médecins, sans qu'aucun n'ait pu la guérir.
\VS{44}S'approchant de lui par derrière, elle toucha le bord de son vêtement. Au même instant la perte de sang s'arrêta.
\VS{45}Jésus dit~: Qui m'a touché~? Comme tous le niaient, Pierre et ceux qui étaient avec lui, dirent~: Maître, la foule qui t'entoure te presse, et tu dis~: Qui m'a touché~?
\VS{46}Mais Jésus dit~: Quelqu'un m'a touché, car j'ai connu qu'une force est sortie de moi.
\VS{47}Alors la femme, se voyant découverte, vint toute tremblante se jeter à ses pieds, lui déclara devant tout le peuple pour quelle raison elle l'avait touché, et comment elle avait été guérie à l'instant.
\VS{48}Jésus lui dit~: Ma fille, rassure-toi. Ta foi t'a guérie. Va en paix.
\VS{49}Et comme il parlait encore, quelqu'un vint de chez le chef de la synagogue, qui lui dit~: Ta fille est morte, n'importune pas le Maître.
\VS{50}Mais Jésus ayant entendu cela, dit au père de la fille~: Ne crains point, crois seulement, et elle sera guérie.
\VS{51}Et quand il fut arrivé à la maison, il ne permit à personne d'entrer avec lui, si ce n'est à Pierre, à Jacques et à Jean, et au père et à la mère de la fille.
\VS{52}Or ils la pleuraient tous, et de douleur, ils se frappaient la poitrine. Mais il leur dit~: Ne pleurez point, elle n'est pas morte, mais elle dort.
\VS{53}Et ils se moquaient de lui, sachant bien qu'elle était morte.
\VS{54}Mais les ayant tous fait sortir, il prit la main de la fille, et dit d'une voix forte~: Enfant, lève-toi~!
\VS{55}Et son esprit revint en elle, et à l'instant elle se leva~; et Jésus ordonna qu'on lui donne à manger.
\VS{56}Les parents de la fille furent dans l'étonnement, et il leur commanda de ne dire à personne ce qui était arrivé.
\Chap{9}
\TextTitle{Mission des douze apôtres\FTNTT{Mt. 10:1-42~; Mc. 6:7-13.}}
\VerseOne{}Puis, Jésus ayant assemblé ses douze disciples, leur donna puissance et autorité sur tous les démons, avec le pouvoir de guérir les malades.
\VS{2}Il les envoya prêcher le Royaume de Dieu et guérir les malades.
\VS{3}Il leur dit~: Ne prenez rien pour le voyage, ni bâton, ni sac, ni pain, ni argent~; et n'ayez pas chacun deux tuniques.
\VS{4}Dans quelque maison que vous entriez, demeurez-y jusqu'à ce que vous partiez de là.
\VS{5}Et partout où l'on ne vous recevra pas, en partant de cette ville, secouez la poussière de vos pieds, en témoignage contre eux.
\VS{6}Ils partirent, et ils allèrent de village en village, évangélisant et opérant des guérisons partout.
\VS{7}Or Hérode le Tétrarque entendit parler de toutes les choses que Jésus faisait, et il ne savait que penser. Car quelques-uns disaient que Jean était ressuscité des morts~;
\VS{8}d'autres, qu'Elie était apparu~; et d'autres, que quelqu'un des anciens prophètes était ressuscité.
\VS{9}Mais Hérode dit~: J'ai fait décapiter Jean. Qui est donc celui-ci de qui j'entends dire de telles choses~? Et il cherchait à le voir.
\VS{10}Puis les apôtres étant de retour, lui racontèrent toutes les choses qu'ils avaient faites. Jésus les prit avec lui, et se retira dans un lieu désert, près de la ville appelée Bethsaïda.
\VS{11}Les foules l'ayant su, le suivirent. Jésus les accueillit, et il leur parlait du Royaume de Dieu~; il guérit aussi ceux qui avaient besoin d'être guéris.
\TextTitle{Multiplication des pains pour cinq mille hommes\FTNTT{Mt. 14:15-21~; Mc. 6:32-44~; Jn. 6:1-14.}}
\VS{12}Comme le jour commençait à baisser, les douze disciples s'approchèrent, et lui dirent~: Renvoie la foule, afin qu'elle aille dans les villages et dans les campagnes des environs, pour se loger et pour trouver à manger~; car nous sommes ici dans un pays désert.
\VS{13}Et il leur dit~: Donnez-leur vous-mêmes à manger. Et ils dirent~: Nous n'avons que cinq pains et deux poissons~; à moins que nous n'allions nous-mêmes acheter des vivres pour tout ce peuple.
\VS{14}Or il y avait environ cinq mille hommes. Jésus dit aux disciples~: Faites-les asseoir par rangées de cinquante chacune.
\VS{15}Ils le firent ainsi, et les firent tous asseoir.
\VS{16}Jésus prit les cinq pains et les deux poissons, et levant les yeux au ciel, il les bénit. Puis, il les rompit, et il les donna à ses disciples afin qu'ils les distribuent à la foule.
\VS{17}Tous mangèrent et furent rassasiés, et l'on remporta douze paniers pleins de morceaux de pain qui restaient.
\TextTitle{Pierre reconnaît Jésus comme étant le Messie\FTNTT{Mt. 16:13-16~; Mc. 8:27-30~; Jn. 6:66-71.}}
\VS{18}Or il arriva que, comme il était dans un lieu retiré pour prier, et que les disciples étaient avec lui, il les interrogea, disant~: Qui suis-je aux dires des foules~?
\VS{19}Ils lui répondirent~: Les uns disent que tu es Jean-Baptiste~; et les autres, Elie~; et les autres, qu'un des anciens prophètes est ressuscité.
\VS{20}Il leur dit alors~: Et vous, qui dites-vous que je suis~? Et Pierre répondit~: Tu es le Christ de Dieu.
\VS{21}Jésus leur commanda sévèrement de ne le dire à personne.
\TextTitle{Jésus annonce sa mort et sa résurrection\FTNTT{Mt. 16:21-23~; Mc. 8:31-33.}}
\VS{22}Et il leur dit~: Il faut que le Fils de l'homme souffre beaucoup, qu'il soit rejeté par les anciens, par les principaux prêtres, par les scribes, qu'il soit mis à mort et qu'il ressuscite le troisième jour.
\TextTitle{Le renoncement à soi-même\FTNTT{Mt. 16:24-28~; Mc. 8:34-38.}}
\VS{23}Puis il dit à tous~: Si quelqu'un veut venir après moi, qu'il renonce à lui-même, et qu'il se charge chaque jour de sa croix, et qu'il me suive.
\VS{24}Car celui qui voudra sauver sa vie, la perdra, mais celui qui perdra sa vie à cause de son amour pour moi, la sauvera.
\VS{25}Et que servirait-il à un homme de gagner tout le monde, s'il se détruisait ou se perdait lui-même~?
\VS{26}Car quiconque aura honte de moi et de mes paroles, le Fils de l'homme aura honte de lui, quand il viendra dans sa gloire, et dans celle du Père et des saints anges.
\TextTitle{La transfiguration\FTNTT{Mt. 17:1-8~; Mc. 9:1-8.}}
\VS{27}Je vous le dis en vérité, quelques-uns de ceux qui sont ici présents, ne mourront point qu'ils n'aient vu le Royaume de Dieu\FTNT{Voir commentaire en Mt. 16:28.}.
\VS{28}Or il arriva environ huit jours après ces paroles, qu'il prit avec lui Pierre, Jean, et Jacques, et qu'il monta sur une montagne pour prier.
\VS{29}Et comme il priait, l'aspect de son visage changea, et son vêtement devint blanc et resplendissant comme un éclair.
\VS{30}Et voici, deux hommes savoir Moïse et Elie, parlaient avec lui,
\VS{31}et ils apparurent environnés de gloire, et ils parlaient de sa mort\FTNT{Mort~: Du grec «~exodos~», ce qui signifie «~départ~», «~mort~», «~sortie~», «~hors de~». Jésus est le prophète de l'Exode dont Moïse a parlé dans De. 18:15.} qu'il allait accomplir à Jérusalem.
\VS{32}Or Pierre et ceux qui étaient avec lui étaient accablés de sommeil~; et quand ils furent réveillés, ils virent sa gloire, et les deux hommes qui étaient avec lui.
\VS{33}Et il arriva qu'au moment où ces hommes se séparaient de Jésus, Pierre dit~: Maître, il est bon que nous soyons ici~; dressons trois tentes, une pour toi, une pour Moïse, et une pour Elie. Il ne savait pas ce qu'il disait.
\VS{34}Et comme il parlait ainsi, une nuée vint les couvrir de son ombre~; et les disciples furent saisis de frayeur en les voyant entrer dans la nuée.
\VS{35}Et une voix vint de la nuée, disant~: Celui-ci est mon Fils bien-aimé~; écoutez-le~!
\VS{36}Quand la voix se fit entendre, Jésus se trouva seul. Et les disciples gardèrent le silence, et ils ne rapportèrent rien à personne en ce temps-là de ce qu'ils avaient vu.
\TextTitle{Une génération incrédule et perverse}
\VS{37}Or il arriva le jour suivant, lorsqu'ils furent descendus de la montagne, qu'une grande foule vint à sa rencontre.
\VS{38}Et voici, du milieu de la foule un homme s'écria~: Maître, je t'en prie, porte les regards sur mon fils, car c'est mon fils unique.
\VS{39}Et voici un esprit le saisit, et aussitôt le fait crier, et l'agite avec violence en le faisant écumer, et c'est à peine s'il se retire de lui après l'avoir broyé.
\VS{40}J'ai prié tes disciples de le chasser, mais ils n'ont pas pu.
\VS{41}Jésus répondit~: Ô génération incrédule et perverse, jusqu'à quand serai-je avec vous et vous supporterai-je~? Amène ici ton fils.
\VS{42}Comme il approchait, le démon l'agita violemment comme s'il voulait le déchirer. Mais Jésus menaça fortement l'esprit impur, et guérit l'enfant, et le rendit à son père.
\VS{43}Et tous furent étonnés de la puissance magnifique de Dieu. Et comme ils étaient tous dans l'admiration de tout ce que Jésus faisait, il dit à ses disciples~:
\VS{44}Vous, écoutez bien ces discours~: Le Fils de l'homme sera livré entre les mains des hommes.
\VS{45}Mais les disciples ne comprirent pas cette parole, et elle était voilée pour eux, afin qu'ils n'en aient pas le sens~; et ils craignaient de l'interroger à ce sujet.
\TextTitle{L'humilité, le secret de la véritable grandeur\FTNTT{Mt. 18:1-6~; Mc. 9:33-37.}}
\VS{46}Or une pensée leur vint à l'esprit, à savoir lequel d'entre eux était le plus grand.
\VS{47}Mais Jésus voyant la pensée de leur cœur, prit un petit enfant et le mit auprès de lui.
\VS{48}Puis il leur dit~: Quiconque reçoit ce petit enfant en mon Nom, me reçoit~; et quiconque me reçoit, reçoit celui qui m'a envoyé. Car celui qui est le plus petit d'entre vous tous, c'est celui-là qui est grand.
\TextTitle{Jésus condamne le sectarisme de Jacques et Jean\FTNTT{Mc. 9:38-40.}}
\VS{49}Et Jean prit la parole, et dit~: Maître, nous avons vu quelqu'un qui chassait les démons en ton Nom~; et nous l'en avons empêché, parce qu'il ne nous suit pas.
\VS{50}Mais Jésus lui dit~: Ne l'en empêchez pas~; car celui qui n'est pas contre nous est pour nous.
\TextTitle{Mission de Jésus~: Sauver les âmes}
\VS{51}Lorsque le temps où il devait être enlevé du monde approcha, Jésus prit la résolution d'aller à Jérusalem.
\VS{52}Il envoya devant lui des messagers, qui se mirent en route, et entrèrent dans un village des Samaritains, pour lui préparer un logement.
\VS{53}Mais les Samaritains ne le reçurent pas, parce qu'il se dirigeait vers Jérusalem.
\VS{54}Et quand Jacques et Jean, ses disciples virent cela, ils dirent~: Seigneur, veux-tu que nous commandions que le feu descende du ciel, et les consume, comme fit Elie~?
\VS{55}Mais Jésus se tourna vers eux et les réprimanda fortement, en leur disant~: Vous ne savez pas de quel esprit vous êtes animés.
\VS{56}Car le Fils de l'homme n'est pas venu pour perdre les âmes des hommes, mais pour les sauver. Ainsi, ils allèrent dans un autre village.
\TextTitle{Le prix à payer pour suivre Jésus\FTNTT{Mt. 8:19-22.}}
\VS{57}Pendant qu'ils étaient en chemin, un homme lui dit~: Seigneur, je te suivrai partout où tu iras.
\VS{58}Mais Jésus lui répondit~: Les renards ont des tanières, et les oiseaux du ciel ont des nids, mais le Fils de l'homme n'a pas où reposer sa tête.
\VS{59}Puis il dit à un autre~: Suis-moi. Et il répondit~: Permets-moi d'aller d'abord ensevelir mon père.
\VS{60}Mais Jésus lui dit~: Laisse les morts ensevelir leurs morts~; mais toi, va, et annonce le Royaume de Dieu.
\VS{61}Un autre aussi lui dit~: Seigneur, je te suivrai~; mais permets-moi de prendre d'abord congé de ceux de ma maison.
\VS{62}Mais Jésus lui répondit~: Quiconque met la main à la charrue, et regarde en arrière, n'est pas bien disposé pour le Royaume de Dieu.
\Chap{10}
\TextTitle{Soixante-dix disciples envoyés en mission}
\VerseOne{}Or après ces choses, le Seigneur désigna soixante-dix autres disciples, et il les envoya deux à deux devant lui, dans toutes les villes et dans tous les lieux où il devait aller.
\VS{2}Il leur dit~: La moisson est grande, mais il y a peu d'ouvriers~; priez donc le Maître de la moisson qu'il envoie des ouvriers dans sa moisson.
\VS{3}Allez, voici, je vous envoie comme des agneaux au milieu des loups.
\VS{4}Ne portez ni bourse, ni sac, ni souliers, et ne saluez personne en chemin.
\VS{5}En quelque maison que vous entriez, dites premièrement~: Que la paix soit sur cette maison~!
\VS{6}Et s'il y a là quelqu'un qui soit digne de paix, votre paix reposera sur lui~; sinon, elle retournera à vous.
\VS{7}Et demeurez dans cette maison, mangeant et buvant de ce qui sera mis devant vous~; car l'ouvrier mérite son salaire. N'allez pas de maison en maison.
\VS{8}Dans quelque ville que vous entriez, et où l'on vous recevra, mangez ce qui sera mis devant vous,
\VS{9}guérissez les malades qui s'y trouveront, et dites-leur~: Le Royaume de Dieu s'est approché de vous.
\VS{10}Mais dans quelque ville que vous entriez, et où l'on ne vous recevra pas, sortez dans ses rues et dites~:
\VS{11}Nous secouons contre vous-mêmes la poussière de votre ville qui s'est attachée à nous~; toutefois sachez que le Royaume de Dieu s'est approché de vous.
\VS{12}Je vous le dis qu'en ce jour Sodome sera traitée moins rigoureusement que cette ville-là.
\TextTitle{Jésus dénonce les indifférents\FTNTT{Mt. 11:20-24.}}
\VS{13}Malheur à toi, Chorazin~! Malheur à toi, Bethsaïda~! Car si les miracles qui ont été faits au milieu de vous avaient été faits dans Tyr et dans Sidon, il y a longtemps qu'elles se seraient repenties, couvertes d'un sac, et assises sur la cendre.
\VS{14}C'est pourquoi Tyr et Sidon seront traitées moins rigoureusement que vous au jour du jugement.
\VS{15}Et toi, Capernaüm, qui as été élevée jusqu'au ciel, tu seras précipitée jusqu'en enfer\FTNT{Voir commentaire en Mt. 16:18.}.
\VS{16}Celui qui vous écoute, m'écoute~; et celui qui vous rejette, me rejette. Or celui qui me rejette, rejette celui qui m'a envoyé.
\VS{17}Or les soixante-dix revinrent avec joie, disant~: Seigneur, les démons mêmes nous sont soumis en ton Nom.
\VS{18}Jésus leur dit~: Je voyais Satan tomber du ciel comme un éclair.
\VS{19}Voici, je vous ai donné le pouvoir de marcher sur les serpents et sur les scorpions, et sur toute la force de l'ennemi~; et rien ne pourra vous nuire.
\VS{20}Toutefois, ne vous réjouissez pas de ce que les esprits vous sont soumis, mais réjouissez-vous plutôt de ce que vos noms sont écrits dans les cieux.
\VS{21}En ce moment même, Jésus se réjouit en esprit, et dit~: Je te loue, ô Père, Seigneur du ciel et de la terre, de ce que tu as caché ces choses aux sages et aux intelligents, et que tu les as révélées aux petits enfants. Oui, Père, parce que telle a été ta bonne volonté.
\VS{22}Toutes choses m'ont été données en main par mon Père~; et personne ne connaît qui est le Fils, si ce n'est le Père~; ni qui est le Père, si ce n'est le Fils et celui à qui le Fils veut le révéler.
\VS{23}Puis, se tournant vers ses disciples, il leur dit en particulier~: Bénis sont les yeux qui voient ce que vous voyez~!
\VS{24}Car je vous dis que beaucoup de prophètes et de rois ont désiré voir ce que vous voyez, et ne l'ont pas vu, et entendre ce que vous entendez, et ne l'ont pas entendu.
\TextTitle{Un docteur de la loi tente d'éprouver Jésus\FTNTT{Mt. 22:34-40~; Mc. 12:28-34.}}
\VS{25}Alors voici, un docteur de la loi s'étant levé pour l'éprouver lui dit~: Maître, que dois-je faire pour avoir la vie éternelle~?
\VS{26}Et il lui dit~: Qu'est-il écrit dans la loi~? Qu'y lis-tu~?
\VS{27}Il répondit~: Tu aimeras le Seigneur, ton Dieu, de tout ton cœur, et de toute ton âme, et de toute ta force, et de toute ta pensée~; et ton prochain comme toi-même.
\VS{28}Jésus lui dit~: Tu as bien répondu. Fais cela, et tu vivras.
\VS{29}Mais lui, voulant se justifier, dit à Jésus~: Et qui est mon prochain~?
\TextTitle{Parabole du Samaritain}
\VS{30}Jésus reprit la parole et dit~: Un homme descendait de Jérusalem à Jéricho. Il tomba entre les mains des brigands qui le dépouillèrent, le chargèrent de plusieurs coups, et s'en allèrent, le laissant à demi mort.
\VS{31}Un prêtre, qui par hasard descendait par le même chemin, ayant vu cet homme, passa outre.
\VS{32}Un Lévite, qui arriva aussi dans ce lieu, l'ayant vu, passa outre.
\VS{33}Mais un Samaritain qui voyageait, étant venu là, fut ému de compassion lorsqu'il le vit.
\VS{34}Il s'approcha, banda ses plaies, en y versant de l'huile et du vin~; puis le mit sur sa propre monture, le conduisit à une hôtellerie, et prit soin de lui.
\VS{35}Et le lendemain, en partant il tira de sa bourse deux deniers, et les donna à l'hôte, en lui disant~: Aie soin de lui~; et tout ce que tu dépenseras de plus, je te le rendrai à mon retour.
\VS{36}Lequel donc de ces trois te semble-t-il avoir été le prochain de celui qui était tombé entre les mains des brigands~?
\VS{37}Il répondit~: C'est celui qui a usé de miséricorde envers lui. Jésus donc lui dit~: Va, et toi aussi, fais de même.
\TextTitle{Marthe et Marie}
\VS{38}Et il arriva comme ils s'en allaient, qu'il entra dans un village, et une femme, nommée Marthe, le reçut dans sa maison.
\VS{39}Elle avait une sœur nommée Marie, qui se tenant assise aux pieds du Seigneur, écoutait sa parole.
\VS{40}Mais Marthe était distraite par divers soins domestiques, et étant venue à Jésus, elle dit~: Seigneur, ne te soucies-tu point que ma sœur me laisse servir toute seule~? Dis-lui donc de m'aider de son côté.
\VS{41}Le Seigneur lui répondit, et dit~: Marthe, Marthe, tu t'inquiètes et tu t'agites pour beaucoup de choses.
\VS{42}Mais une chose est nécessaire~; et Marie a choisi la bonne part, qui ne lui sera point ôtée.
\Chap{11}
\TextTitle{Enseignement de Jésus sur la prière\FTNTT{Mt. 6:9-15.}}
\VerseOne{}Et il arriva, comme il était en prière en un certain lieu, qu'après qu'il eut cessé de prier, un de ses disciples lui dit~: Seigneur, enseigne-nous à prier, comme Jean l'a enseigné à ses disciples.
\VS{2}Il leur dit~: Quand vous prierez, dites~: Notre Père qui es aux cieux~! Que ton Nom soit sanctifié, que ton règne vienne, que ta volonté soit faite sur la terre comme au ciel.
\VS{3}Donne-nous chaque jour notre pain quotidien.
\VS{4}Et pardonne-nous nos péchés, car nous aussi, nous remettons les dettes à tous ceux qui nous doivent. Et ne nous induis point en tentation, mais délivre-nous du mal.
\TextTitle{Parabole des trois amis et de la prière importune}
\VS{5}Puis il leur dit~: Si l'un de vous a un ami, et qu'il aille le trouver à minuit pour lui dire~: Mon ami, prête-moi trois pains,
\VS{6}car un de mes amis est arrivé de voyage chez moi, et je n'ai rien à lui offrir,
\VS{7}et si, de l'intérieur de sa maison, cet ami lui répond~: Ne m'importune pas, car ma porte est déjà fermée, mes enfants et moi nous sommes au lit, je ne puis me lever pour t'en donner.
\VS{8}Je vous le dis, même s'il ne se levait pas pour les lui donner parce que c'est son ami, il se lèverait à cause de son importunité, et lui donnerait tout ce dont il a besoin.
\VS{9}Ainsi je vous dis~: Demandez, et il vous sera donné~; cherchez, et vous trouverez~; frappez, et l'on vous ouvrira.
\VS{10}Car quiconque demande, reçoit, et celui qui cherche, trouve, et l'on ouvre à celui qui frappe.
\TextTitle{Parabole du père}
\VS{11}Quel est parmi vous le père qui donnera une pierre à son fils, s'il lui demande du pain~? Ou, s'il lui demande un poisson, lui donnera-t-il un serpent au lieu d'un poisson~?
\VS{12}Ou, s'il demande un œuf, lui donnera-t-il un scorpion~?
\VS{13}Si donc vous qui êtes méchants, vous savez donner à vos enfants des choses bonnes, combien plus le Père qui est du ciel donnera-t-il l'Esprit Saint à ceux qui le lui demandent.
\TextTitle{Jésus guérit un démoniaque}
\VS{14}Alors il chassa un démon qui était muet. Lorsque le démon fut sorti, le muet parla et la foule fut dans l'admiration.
\TextTitle{Le blasphème contre le Saint-Esprit\FTNTT{Mt. 12:24-32~; Mc. 3:22-30.}}
\VS{15}Mais quelques-uns d'entre eux dirent~: C'est par Béelzebul\FTNT{Béelzebul, prince des démons, est connu dans le Tanakh sous le nom de Baal Zebub, dieu d'Ekron (2 R. 1:2-16). Son nom signifie «~seigneur des mouches~».}, prince des démons, qu'il chasse les démons.
\VS{16}Mais les autres pour l'éprouver, lui demandaient un miracle venant du ciel.
\VS{17}Mais lui, connaissant leurs pensées, leur dit~: Tout royaume divisé contre lui-même sera réduit en désert~; et toute maison divisée contre elle-même tombe en ruine.
\VS{18}Si donc Satan est divisé contre lui-même, comment son royaume subsistera-t-il~? Car vous dites que je chasse les démons par Béelzebul.
\VS{19}Et si moi, je chasse les démons par Béelzebul, vos fils par qui les chassent-ils~? C'est pourquoi ils seront eux-mêmes vos juges.
\VS{20}Mais si je chasse les démons par le doigt de Dieu, alors le Royaume de Dieu est parvenu jusqu'à vous.
\VS{21}Lorsqu'un homme fort et bien armé garde sa bergerie\FTNT{Bergerie~: Chez les Grecs, du temps d'Homère, c'était un espace découvert autour de la maison, fermé par un mur, tandis que chez les Orientaux, il s'agissait d'un espace dans la campagne, entouré d'un mur, où les troupeaux passaient la nuit. La bergerie désigne aussi la partie non couverte d'une maison. Dans la Première Alliance, il s'agit particulièrement du «~parvis~» du tabernacle et du temple à Jérusalem. Les demeures des gens de la haute société possédaient généralement deux de ces «~cours~»~: une entre la porte et la rue, l'autre entourant l'immeuble lui-même. C'est ce qui est mentionné en Mt. 26:69. Enfin, ce terme fait allusion à la maison elle-même, un palais.}, les biens qu'il a sont en sûreté.
\VS{22}Mais si un plus fort que lui survient et le vainc, il lui enlève toutes ses armes dans lesquelles il se confiait, et il partage ses dépouilles.
\VS{23}Celui qui n'est point avec moi est contre moi~; et celui qui n'assemble pas avec moi, il disperse.
\TextTitle{Le retour de l'esprit impur\FTNTT{Mt. 12:43-45.}}
\VS{24}Quand l'esprit impur est sorti d'un homme, il va par des lieux secs, cherchant du repos. N'en trouvant point, il dit~: Je retournerai dans ma maison, d'où je suis sorti,
\VS{25}et quand il arrive, il la trouve balayée et parée.
\VS{26}Alors il s'en va, et prend avec lui sept autres esprits plus méchants que lui, et ils entrent et demeurent là, de sorte que la dernière condition de cet homme-là est pire que la première.
\VS{27}Or il arriva comme il disait ces choses, qu'une femme élevant sa voix du milieu de la foule, lui dit~: Bénis est le ventre qui t'a porté, et les mamelles que tu as tétées~!
\VS{28}Et il répondit~: Bénis sont plutôt ceux qui écoutent la parole de Dieu, et qui la gardent~!
\TextTitle{Le miracle de Jonas, le prophète\FTNTT{Mt. 12:38-41.}}
\VS{29}Et comme les foules s'amassaient ensemble, il se mit à dire~: Cette génération est méchante~; elle demande un miracle, mais il ne lui sera donné d'autre miracle que celui de Jonas, le prophète.
\VS{30}Car, de même que Jonas fut un miracle pour les Ninivites, de même le Fils de l'homme en sera un pour cette génération.
\VS{31}La reine du Midi se lèvera, au jour du jugement, contre les hommes de cette génération et les condamnera, parce qu'elle vint des extrémités de la terre pour entendre la sagesse de Salomon~; et voici, il y a ici plus que Salomon.
\VS{32}Les gens de Ninive se lèveront, au jour du jugement, contre cette génération et la condamneront, parce qu'ils se sont repentis à la prédication de Jonas~; et voici, il y a ici plus que Jonas.
\TextTitle{Parabole de la lampe\FTNTT{Mt. 5:14-16~; Mc. 4:21-23~; Lu. 8:16-18.}}
\VS{33}Or personne n'allume une lampe pour la mettre dans un lieu caché ou sous le boisseau, mais sur un chandelier, afin que ceux qui entrent voient la lumière.
\VS{34}La lumière du corps c'est l'œil. Si donc ton œil est sain, tout ton corps aussi sera éclairé~; mais s'il est mauvais, ton corps aussi sera ténébreux.
\VS{35}Prends donc garde que la lumière qui est en toi ne soit pas ténèbres.
\VS{36}Si donc ton corps est éclairé, n'ayant aucune partie dans les ténèbres, il sera entièrement éclairé, comme lorsque la lampe t'éclaire de sa lumière.
\TextTitle{Jésus censure les pharisiens et les docteurs de la loi\FTNTT{cp. Mt. 12:38-41.}}
\VS{37}Comme il parlait, un pharisien le pria de dîner chez lui. Il entra et se mit à table.
\VS{38}Mais le pharisien vit avec étonnement qu'il ne s'était pas premièrement lavé avant le dîner.
\VS{39}Mais le Seigneur lui dit~: Vous autres pharisiens, vous nettoyez le dehors de la coupe et du plat~; et à l'intérieur vous êtes pleins de rapine et de méchanceté.
\VS{40}Insensés~! Celui qui a fait le dehors, n'a-t-il pas fait aussi le dedans~?
\VS{41}Donnez plutôt en aumône ce qui est dedans, et voici, toutes choses seront pures pour vous.
\VS{42}Mais malheur à vous, pharisiens~! Car vous payez la dîme de la menthe, de la rue\FTNT{La rue~: Il s'agit d'un arbuste ayant des propriétés médicinales. Les pharisiens poussaient leur zèle jusqu'à payer la dîme sur certaines herbes. Toutefois, en négligeant la justice et l'amour de Dieu, ils passaient à côté de l'essentiel. Toutes leurs œuvres étaient par conséquent vaines.}, et de toutes sortes d'herbes, et vous négligez la justice et l'amour de Dieu. C'est là ce qu'il fallait pratiquer, sans négliger les autres choses.
\VS{43}Malheur à vous, pharisiens, qui aimez les premières places dans les synagogues, et les salutations sur les places publiques~!
\VS{44}Malheur à vous, scribes et pharisiens hypocrites~! Car vous êtes comme les sépulcres qui ne paraissent pas, et sur lesquels on marche sans le voir.
\VS{45}Alors un des docteurs de la loi prit la parole, et lui dit~: Maître, en disant ces choses, tu nous outrages aussi.
\VS{46}Et il dit~: A vous aussi, malheur, docteurs de la loi~! Car vous chargez les hommes de fardeaux difficiles à porter, et vous-mêmes vous ne touchez pas ces fardeaux d'un seul de vos doigts.
\VS{47}Malheur à vous~! Car vous bâtissez les sépulcres des prophètes, que vos pères ont tués.
\VS{48}Vous rendez donc témoignage aux œuvres de vos pères, et vous y prenez plaisir~; car eux, ils les ont tués, et vous, vous bâtissez leurs sépulcres.
\VS{49}C'est pourquoi aussi la sagesse de Dieu a dit~: Je leur enverrai des prophètes et des apôtres, et ils tueront les uns, et persécuteront les autres,
\VS{50}afin que le sang de tous les prophètes qui a été répandu dès la fondation du monde, soit redemandé à cette nation,
\VS{51}depuis le sang d'Abel, jusqu'au sang de Zacharie, qui fut tué entre l'autel et le temple. Oui, je vous le dis qu'il sera redemandé à cette nation.
\VS{52}Malheur à vous, docteurs de la loi~! Parce que vous avez enlevé la clef de la science. Vous n'êtes pas entrés vous-mêmes, et vous avez empêché ceux qui entraient.
\VS{53}Et comme il leur disait ces choses, les scribes et les pharisiens commencèrent à le presser violemment, et à le faire parler sur beaucoup de choses,
\VS{54}lui dressant des pièges, et cherchant à tirer quelque chose de sa bouche, afin de l'accuser.
\Chap{12}
\TextTitle{Enseignements divers de Jésus\FTNT{Mt. 16:6-12~; Mc. 8:14-21.}}
\VerseOne{}Cependant les gens s'étaient rassemblés par milliers, au point de s'écraser les uns les autres. Jésus se mit à dire à ses disciples~: Avant tout, gardez-vous surtout du levain des pharisiens qui est l'hypocrisie.
\VS{2}Car il n'y a rien de caché, qui ne doive être révélé, ni de secret, qui ne doive être connu.
\VS{3}C'est pourquoi tout ce que vous aurez dit dans les ténèbres, sera entendu dans la lumière~; et ce que vous aurez dit à l'oreille dans les chambres, sera prêché sur les toits.
\VS{4}Je vous dis à vous qui êtes mes amis~: Ne craignez pas ceux qui tuent le corps, et qui après cela ne peuvent rien faire de plus.
\VS{5}Je vous montrerai qui vous devez craindre. Craignez celui qui, après avoir tué, a le pouvoir de jeter dans la géhenne~; oui, vous dis-je, craignez celui-là.
\VS{6}Ne vend-on pas cinq petits passereaux pour deux sous~? Cependant, aucun d'eux n'est oublié devant Dieu.
\VS{7}Et même les cheveux de votre tête sont tous comptés. Ne craignez donc point~; vous valez plus que beaucoup de passereaux.
\VS{8}Or je vous dis, quiconque me confessera devant les hommes, le Fils de l'homme le confessera aussi devant les anges de Dieu.
\VS{9}Mais quiconque me reniera devant les hommes, il sera renié devant les anges de Dieu.
\VS{10}Et quiconque parlera contre le Fils de l'homme, il lui sera pardonné~; mais celui qui aura blasphémé contre le Saint-Esprit\FTNT{Voir commentaire en Mt. 12:32.}, il ne lui sera point pardonné.
\VS{11}Quand ils vous mèneront devant les synagogues, les magistrats et les autorités, ne vous inquiétez pas de la manière dont vous vous défendrez ni de ce que vous aurez à dire.
\VS{12}Car le Saint-Esprit vous enseignera à l'heure même ce qu'il faudra dire.
\TextTitle{Parabole du riche insensé}
\VS{13}Et quelqu'un de la foule lui dit~: Maître, dis à mon frère qu'il partage avec moi notre héritage.
\VS{14}Mais il lui répondit~: Ô homme~! Qui m'a établi sur vous pour être votre juge, et pour faire vos partages~?
\VS{15}Puis il leur dit~: Gardez-vous avec soin de toute avarice~; car quoique les biens de quelqu'un abondent, il n'a pas la vie par ses biens.
\VS{16}Et il leur dit cette parabole~: Les champs d'un homme riche avaient beaucoup rapporté.
\VS{17}Et il raisonnait en lui-même, disant~: Que ferai-je, car je n'ai pas assez de place pour recueillir mes fruits~?
\VS{18}Puis il dit~: Voici ce que je ferai~: J'abattrai mes greniers, j'en bâtirai de plus grands, j'y amasserai toute ma récolte et tous mes biens.
\VS{19}Puis je dirai à mon âme~: Mon âme, tu as beaucoup de biens assemblés pour beaucoup d'années, repose-toi, mange, bois, et réjouis-toi.
\VS{20}Mais Dieu lui dit~: Insensé~! Cette même nuit ton âme te sera redemandée~; et ces choses que tu as préparées, à qui seront-elles~?
\VS{21}Il en est ainsi de celui qui amasse des biens pour lui-même, et qui n'est pas riche en Dieu.
\TextTitle{Exhortation à se confier en Dieu}
\VS{22}Jésus dit à ses disciples~: C'est pourquoi je vous dis~: Ne vous inquiétez pas pour votre vie, de ce que vous mangerez, ni pour votre corps, de quoi vous serez vêtus.
\VS{23}La vie est plus que la nourriture, et le corps est plus que le vêtement.
\VS{24}Considérez les corbeaux, ils ne sèment, ni ne moissonnent, et ils n'ont point de cellier, ni de grenier, et cependant Dieu les nourrit. Combien ne valez-vous pas plus que les oiseaux~?
\VS{25}Qui de vous par ses inquiétudes peut ajouter une coudée à la durée de sa vie~?
\VS{26}Si donc vous ne pouvez pas même la moindre chose, pourquoi êtes-vous inquiets du reste~?
\VS{27}Considérez comment croissent les lis, ils ne travaillent, ni ne filent, et cependant je vous dis que Salomon même, dans toute sa gloire, n'a pas été vêtu comme l'un d'eux.
\VS{28}Si Dieu revêt ainsi l'herbe qui est aujourd'hui au champ, et qui demain sera jetée au four, à combien plus forte raison vous vêtira-t-il, ô gens de petite foi~?
\VS{29}Ne dites donc point~: Que mangerons-nous, ou que boirons-nous~? Et ne soyez pas inquiets,
\VS{30}car toutes ces choses, ce sont les Gentils du monde qui les recherchent. Votre Père sait que vous en avez besoin.
\VS{31}Mais cherchez plutôt le Royaume de Dieu, et toutes ces choses vous seront données par-dessus.
\VS{32}Ne crains point petit troupeau, car il a plu à votre Père de vous donner le Royaume.
\VS{33}Vendez ce que vous avez, et donnez-le en aumône. Faites-vous des bourses qui ne s'usent point, un trésor dans les cieux qui ne défaille jamais, et où le voleur n'approche point, et où la teigne ne gâte rien.
\VS{34}Car là où est votre trésor, là aussi sera votre cœur.
\TextTitle{Veiller en attendant le Maître\FTNTT{Mt. 24:36-25:30.}}
\VS{35}Que vos reins soient ceints, et vos lampes allumées.
\VS{36}Et soyez semblables aux serviteurs qui attendent que leur maître revienne des noces, afin de lui ouvrir dès qu'il frappera.
\VS{37}Bénis sont ces serviteurs que le maître à son arrivée, trouvera veillant~! En vérité, je vous le dis, il se ceindra, les fera mettre à table et s'approchera pour les servir.
\VS{38}Qu'il arrive à la seconde veille ou à la troisième veille, bénis sont ces serviteurs, s'il les trouve veillant~!
\VS{39}Or sachez ceci, si le père de famille savait à quelle heure le voleur doit venir, il veillerait et ne laisserait pas percer sa maison.
\VS{40}Vous donc aussi tenez-vous prêts, car le Fils de l'homme viendra à l'heure où vous n'y penserez pas.
\TextTitle{Parabole des deux serviteurs}
\VS{41}Pierre lui dit~: Seigneur, dis-tu cette parabole pour nous, ou aussi pour tous~?
\VS{42}Et le Seigneur dit~: Quel est donc l'économe fidèle et prudent, que le maître établira sur les domestiques de sa maison pour leur donner la nourriture au temps convenable~?
\VS{43}Bénis est ce serviteur, que son maître, à son arrivée, trouvera faisant ainsi~!
\VS{44}Je vous le dis en vérité, il l'établira sur tous ses biens.
\VS{45}Mais si ce serviteur dit en son cœur~: Mon maître tarde longtemps à venir. S'il se met à battre les serviteurs et les servantes, à manger, à boire et à s'enivrer,
\VS{46}le maître de cet esclave-là viendra en un jour qu'il n'attend pas, et à une heure qu'il ne sait pas, et il le coupera en deux\FTNT{Le mot grec «~dichotomeo~» signifie «~couper en deux parts~», «~couper quelqu'un en deux~», «~châtiant en coupant~», «~fléau sévère~». Certains peuples, dont les Hébreux, employaient cette méthode comme châtiment corporel.}, et lui donnera sa part avec les infidèles.
\VS{47}Or le serviteur qui a connu la volonté de son maître, qui ne s'est pas tenu prêt, et n'a point fait selon sa volonté, sera battu de plusieurs coups.
\VS{48} Mais celui qui ne l'a point connue et qui a fait des choses dignes de châtiment sera battu de peu de coups. Et il sera beaucoup redemandé à quiconque il aura été beaucoup donné~; et on exigera plus de celui à qui on aura beaucoup confié.
\TextTitle{Jésus suscite la division}
\VS{49}Je suis venu jeter un feu sur la terre, et qu'ai-je à désirer, s'il est déjà allumé~?
\VS{50}Il est un baptême dont je dois être baptisé, et combien suis-je pressé jusqu'à ce qu'il soit accompli.
\VS{51}Pensez-vous que je sois venu apporter la paix sur la terre~? Non, vous dis-je~; mais plutôt la division.
\VS{52}Car désormais, cinq dans une maison seront divisés, trois contre deux, et deux contre trois.
\VS{53}Le père sera divisé contre le fils et le fils contre le père, la mère contre la fille et la fille contre la mère, la belle-mère contre sa belle-fille et la belle-fille contre sa belle-mère.
\VS{54}Puis il dit encore aux foules~: Quand vous voyez un nuage se lever à l'occident, vous dites aussitôt~: La pluie vient. Et cela arrive ainsi.
\VS{55}Et quand vous voyez souffler le vent du midi, vous dites qu'il fera chaud. Et cela arrive.
\VS{56}Hypocrites~! Vous savez bien discerner l'aspect du ciel et de la terre~; et comment ne discernez-vous point cette saison~?
\VS{57}Et pourquoi aussi ne reconnaissez-vous pas de vous-mêmes ce qui est juste~?
\VS{58}Or quand tu vas avec ton adversaire devant le magistrat, tâche en chemin de t'en délivrer, de peur qu'il ne te traîne devant le juge, et que le juge ne te livre à l'officier de justice, et que celui-ci ne te mette en prison.
\VS{59}Je te le dis, tu ne sortiras pas de là que tu n'aies payé jusqu'au dernier pite\FTNT{Petite pièce de monnaie en laiton. Voir le «~tableau des monnaies au temps de Jésus-Christ~» en annexe.}.
\Chap{13}
\TextTitle{Exhortation à la repentance}
\VerseOne{}En ce même temps, quelques-uns qui se trouvaient là présents racontèrent à Jésus ce qui était arrivé à des Galiléens, dont Pilate avait mêlé le sang avec celui de leurs sacrifices.
\VS{2}Et Jésus répondant leur dit~: Croyez-vous que ces Galiléens étaient de plus grands pécheurs que tous les autres Galiléens, parce qu'ils ont souffert de la sorte~?
\VS{3}Non, vous dis-je~; mais si vous ne vous repentez pas, vous périrez tous de la même manière.
\VS{4}Ou bien, ces dix-huit personnes sur qui est tombée la tour de Siloé et qu'elle a tuées, croyez-vous qu'elles étaient plus coupables que tous les habitants de Jérusalem~?
\VS{5}Non, vous dis-je~; mais si vous ne vous repentez pas, vous périrez tous de la même manière.
\TextTitle{Parabole du figuier stérile et le jugement différé d'Israël\FTNTT{cp. Mt. 21:18-21.}}
\VS{6}Il disait aussi cette parabole~: Un homme avait un figuier planté dans sa vigne, et il vint pour y chercher du fruit, mais il n'en trouva point.
\VS{7}Et il dit au vigneron~: Voilà trois ans que je viens chercher du fruit à ce figuier, et je n'en trouve point. Coupe-le~; pourquoi occupe-t-il inutilement la terre~?
\VS{8}Et le vigneron lui répondit, et dit~: Seigneur, laisse-le encore pour cette année, je creuserai tout autour, et j'y mettrai du fumier.
\VS{9}Peut-être portera-t-il du fruit~; sinon, tu le couperas après cela.
\TextTitle{Guérison de la femme courbée le jour du sabbat}
\VS{10}Or comme il enseignait dans une de leurs synagogues un jour de sabbat,
\VS{11}voici, il y avait là une femme qui était possédée d'un démon qui la rendait infirme depuis dix-huit ans~; elle était courbée, et ne pouvait nullement se redresser.
\VS{12}Et quand Jésus la vit, il l'appela, et lui dit~: Femme, tu es délivrée de ton infirmité.
\VS{13}Et il lui imposa les mains~; et à l'instant elle se redressa, et glorifia Dieu.
\VS{14}Mais le chef de la synagogue, indigné de ce que Jésus avait opéré cette guérison un jour du sabbat, prenant la parole dit à l'assemblée~: Il y a six jours pour travailler~; venez donc vous faire guérir ces jours-là, et non pas le jour du sabbat.
\VS{15}Hypocrites~! Lui répondit le Seigneur, chacun de vous ne détache-t-il pas son bœuf ou son âne de la crèche le jour du sabbat, et ne les mène-t-il pas boire~?
\VS{16}Et ne fallait-il pas délier de ce lien le jour du sabbat cette femme qui est fille d'Abraham, et que Satan tenait liée depuis dix-huit ans~?
\VS{17}Comme il disait ces choses, tous ses adversaires étaient confus~; mais toutes les foules se réjouissaient de toutes les choses glorieuses qu'il opérait.
\TextTitle{Parabole du grain de moutarde et du levain\FTNTT{voir Mt. 13:31,33.}}
\VS{18}Il disait aussi~: A quoi est semblable le Royaume de Dieu, et à quoi le comparerai-je~?
\VS{19}Il est semblable au grain de semence de moutarde qu'un homme a pris et jeté dans son jardin~; et il pousse, et devient un grand arbre, et les oiseaux du ciel font leurs nids dans ses branches.
\VS{20}Il dit encore~: A quoi comparerai-je le Royaume de Dieu~?
\VS{21}Il est semblable au levain qu'une femme a pris et mis dans trois mesures de farine, pour faire lever toute la pâte.
\TextTitle{Enseignements de Jésus sur le chemin de Jérusalem}
\VS{22}Puis il s'en allait par les villes et les villages, enseignant, et faisant route vers Jérusalem.
\VS{23}Quelqu'un lui dit~: Seigneur, n'y a-t-il que peu de gens qui soient sauvés~? Il leur répondit~:
\VS{24}Efforcez-vous d'entrer par la porte étroite. Car je vous le dis que beaucoup chercheront à entrer, et ne le pourront pas.
\VS{25}Quand le père de famille se sera levé, et aura fermé la porte, et que vous, étant dehors, vous vous mettrez à frapper à la porte, en disant~: Seigneur, Seigneur~! Ouvre-nous~! Il vous répondra, en disant~: Je ne sais pas d'où vous êtes.
\VS{26}Alors vous vous mettrez à dire~: Nous avons mangé et bu en ta présence, et tu as enseigné dans nos rues.
\VS{27}Mais il dira~: Je vous le dis, je ne sais pas d'où vous êtes. Retirez-vous de moi, vous tous qui faites le métier d'iniquité.
\VS{28}C'est là qu'il y aura des pleurs et des grincements de dents, quand vous verrez Abraham, et Isaac, et Jacob, et tous les prophètes dans le Royaume de Dieu, et que vous serez jetés dehors.
\VS{29}Il en viendra aussi d'orient et d'occident, du nord et du sud, et ils se mettront à table dans le Royaume de Dieu.
\VS{30}Et voici, ceux qui sont les derniers seront les premiers, et ceux qui sont les premiers seront les derniers.
\VS{31}En ce même jour, quelques pharisiens vinrent à lui et lui dirent~: Retire-toi et va-t'en d'ici, car Hérode veut te tuer.
\VS{32}Il leur répondit~: Allez, et dites à ce renard~: Voici, je chasse les démons et j'achève de faire des guérisons aujourd'hui et demain, et le troisième jour je prends fin.
\VS{33}C'est pourquoi il me faut marcher aujourd'hui et demain, et le jour suivant~; car il ne convient pas qu'un prophète meure hors de Jérusalem.
\TextTitle{Lamentations de Jésus sur Jérusalem\FTNTT{Mt. 23:37-39~; Lu. 19:41-44~; cp. Jé. 22:5.}}
\VS{34}Jérusalem, Jérusalem, qui tues les prophètes et qui lapides ceux qui te sont envoyés~; combien de fois ai-je voulu rassembler tes enfants, comme la poule rassemble ses poussins sous ses ailes, et vous ne l'avez pas voulu~!
\VS{35}Voici, votre maison va être déserte~; et je vous le dis en vérité, que vous ne me verrez plus, jusqu'à ce que vous disiez~: Béni soit celui qui vient au Nom du Seigneur~!
\Chap{14}
\TextTitle{Jésus guérit un hydropique le jour du sabbat\FTNTT{cp. Mt. 12:9-13.}}
\VerseOne{}Jésus entra un jour de sabbat dans la maison d'un des chefs des pharisiens pour prendre un repas, et les pharisiens l'observaient.
\VS{2}Et voici, un homme hydropique était là devant lui.
\VS{3}Jésus prit la parole, et dit aux docteurs de la loi et aux pharisiens~: Est-il permis, ou non, de faire une guérison le jour du sabbat~?
\VS{4}Ils gardèrent le silence. Alors Jésus prit le malade, le guérit, et le renvoya.
\VS{5}Puis s'adressant à eux, il leur dit~: Lequel de vous, si son fils ou son bœuf tombe dans un puits, ne l'en retirera pas aussitôt, le jour du sabbat~?
\VS{6}Et ils ne pouvaient répliquer à ces choses.
\TextTitle{Parabole de l'invité}
\VS{7}Il proposa aussi aux conviés une parabole, en voyant qu'ils choisissaient les premières places~; et il leur dit~:
\VS{8}Quand tu seras convié par quelqu'un à des noces, ne te mets pas à la première place à table, de peur qu'il ne se trouve parmi les conviés une personne plus honorable que toi,
\VS{9}et que celui qui vous a conviés l'un et l'autre ne vienne te dire~: Cède ta place à cette personne-là. Et alors tu aurais honte d'aller occuper la dernière place.
\VS{10}Mais lorsque tu seras convié, va te mettre à la dernière place, afin que quand celui qui t'a convié viendra, il te dise~: Mon ami, monte plus haut. Alors cela te fera honneur devant tous ceux qui seront à table avec toi.
\VS{11}Car quiconque s'élève sera abaissé, et quiconque s'abaisse sera élevé.
\VS{12}Il dit aussi à celui qui l'avait convié~: Lorsque tu fais un dîner ou un souper, n'invite pas tes amis, ni tes frères, ni tes parents, ni tes riches voisins, de peur qu'ils ne te convient à leur tour, et qu'on ne te rende la pareille.
\VS{13}Mais, lorsque tu donneras un festin, convie les pauvres, les impotents, les boiteux et les aveugles.
\VS{14}Et tu seras bénis de ce qu'ils n'ont pas de quoi te rendre la pareille~; car elle te sera rendue à la résurrection des justes.
\TextTitle{Parabole du grand festin\FTNTT{Mt. 22:1-14.}}
\VS{15}Un de ceux qui étaient à table, ayant entendu ces paroles, lui dit~: Bénis est celui qui mangera du pain dans le Royaume de Dieu~!
\VS{16}Et Jésus lui répondit~: Un homme fit un grand festin, et il convia beaucoup de gens.
\VS{17}Et à l'heure du souper, il envoya son serviteur pour dire aux conviés~: Venez, car tout est déjà prêt.
\VS{18}Mais ils commencèrent tous unanimement à s'excuser. Le premier lui dit~: J'ai acheté un champ, et il me faut nécessairement partir pour aller le voir~; je te prie, excuse-moi.
\VS{19}Un autre dit~: J'ai acheté cinq paires de bœufs, et je vais les essayer~; je te prie, excuse-moi.
\VS{20}Et un autre dit~: J'ai épousé une femme, c'est pourquoi je ne puis aller.
\VS{21}Le serviteur, de retour, rapporta ces choses à son maître. Alors le père de famille irrité, dit à son serviteur~: Va promptement dans les places et dans les rues de la ville, et amène ici les pauvres, les impotents, les boiteux et les aveugles.
\VS{22}Puis le serviteur dit~: Maître, ce que tu as commandé a été fait, et il y a encore de la place.
\VS{23}Et le maître dit au serviteur~: Va dans les chemins et le long des haies, et ceux que tu trouveras, contrains-les d'entrer, afin que ma maison soit remplie.
\VS{24}Car je vous dis, qu'aucun de ces hommes qui avaient été conviés ne goûtera de mon souper.
\TextTitle{Le prix de la consécration du disciple}
\VS{25}Or de grandes foules faisaient route avec Jésus. Il se retourna et leur dit~:
\VS{26}Si quelqu'un vient à moi, et ne hait pas son père et sa mère, sa femme et ses enfants, ses frères et ses sœurs, et même sa propre vie, il ne peut être mon disciple.
\VS{27}Et quiconque ne porte pas sa croix, et ne me suit pas, ne peut être mon disciple.
\VS{28}Car lequel de vous, s'il veut bâtir une tour, ne s'assied pas premièrement pour calculer la dépense et voir s'il a de quoi l'achever~?
\VS{29}De peur qu'après avoir posé les fondements, il ne puisse pas l'achever, et que tous ceux qui le verront ne commencent à se moquer de lui,
\VS{30}en disant~: Cet homme a commencé à bâtir, et il n'a pas pu achever.
\VS{31}Ou, quel roi, s'il va faire la guerre à un autre roi, ne s'assied pas premièrement pour examiner s'il peut, avec dix mille hommes, aller à la rencontre de celui qui vient contre lui avec vingt mille~?
\VS{32}Autrement, pendant que cet autre roi est encore loin, il lui envoie une ambassade pour demander la paix.
\VS{33}Ainsi donc, quiconque d'entre vous ne renonce pas à tout ce qu'il possède ne peut être mon disciple.
\VS{34}Le sel est bon~; mais si le sel perd sa saveur, avec quoi l'assaisonnera-t-on~?
\VS{35}Il n'est bon ni pour la terre, ni pour le fumier~; mais on le jette dehors. Que celui qui a des oreilles pour entendre, qu'il entende~!
\Chap{15}
\TextTitle{Trois paraboles sur la repentance}
\VerseOne{}Or tous les publicains et les pécheurs s'approchaient de Jésus pour l'entendre.
\VS{2}Mais les pharisiens et les scribes murmuraient, disant~: Cet homme reçoit les pécheurs, et mange avec eux.
\TextTitle{Parabole de la brebis perdue\FTNTT{Mt. 18:12-14.}}
\VS{3}Mais il leur proposa cette parabole, disant~:
\VS{4}Lequel d'entre vous, s'il a cent brebis, et qu'il en perd une, ne laisse pas les quatre-vingt-dix-neuf dans le désert, pour aller à la recherche de celle qui est perdue, jusqu'à ce qu'il la trouve~?
\VS{5}Et l'ayant retrouvée, il la met avec joie sur ses épaules,
\VS{6}et, de retour à la maison, il appelle ses amis et ses voisins, et il leur dit~: Réjouissez-vous avec moi~; car j'ai trouvé ma brebis qui était perdue.
\VS{7}De même, je vous le dis il y aura plus de joie dans le ciel pour un seul pécheur qui se repent, que pour les quatre-vingt-dix-neuf justes qui n'ont pas besoin de repentance.
\TextTitle{Parabole de la drachme perdue}
\VS{8}Ou quelle femme, si elle a dix drachmes, et qu'elle en perde une, n'allume pas une lampe, ne balaie la maison, et ne cherche avec soin, jusqu'à ce qu'elle la trouve~?
\VS{9}Lorsqu'elle l'a trouvée, elle appelle ses amies et ses voisines, en leur disant~: Réjouissez-vous avec moi~; car j'ai trouvé la drachme que j'avais perdue.
\VS{10}Ainsi je vous le dis, il y a de la joie devant les anges de Dieu pour un seul pécheur qui vient à se repentir.
\TextTitle{Parabole du fils perdu}
\VS{11}Il leur dit aussi~: Un homme avait deux fils.
\VS{12}Et le plus jeune dit à son père~: Mon père, donne-moi la part de bien qui m'appartient. Et il leur partagea ses biens.
\VS{13}Et peu de jours après, le plus jeune fils, ayant tout ramassé, partit pour un pays éloigné, où il dissipa son bien en vivant dans la débauche.
\VS{14}Et après qu'il eut tout dépensé, une grande famine survint dans ce pays-là, et il commença à se trouver dans la disette.
\VS{15}Alors il alla se mettre au service d'un des habitants du pays, qui l'envoya dans ses possessions pour paître les pourceaux.
\VS{16}Il aurait bien voulu se rassasier des carouges que les pourceaux mangeaient, mais personne ne lui en donnait.
\VS{17}Or étant revenu à lui-même, il dit~: Combien d'ouvriers chez mon père ont du pain en abondance, et moi je meurs de faim~!
\VS{18}Je me lèverai, j'irai vers mon père, et je lui dirai~: Mon père, j'ai péché contre le ciel et devant toi,
\VS{19}et je ne suis plus digne d'être appelé ton fils~; traite-moi comme l'un de tes ouvriers.
\VS{20}Il se leva donc, et alla vers son père. Et comme il était encore loin, son père le vit et fut ému de compassion, et il courut se jeter à son cou et le baisa.
\VS{21}Mais le fils lui dit~: Mon père, j'ai péché contre le ciel et devant toi~; et je ne suis plus digne d'être appelé ton fils.
\VS{22}Et le père dit à ses serviteurs~: Apportez la plus belle robe et revêtez-le, et mettez-lui un anneau au doigt, et des souliers aux pieds.
\VS{23}Amenez-moi le veau gras, et tuez-le. Mangeons et réjouissons-nous.
\VS{24}Car mon fils que voici était mort, mais il est ressuscité~; il était perdu, mais il est retrouvé. Et ils commencèrent à se réjouir.
\VS{25}Or son fils aîné était dans les champs. Lorsqu'il revint et approcha de la maison, il entendit la musique et les danses.
\VS{26}Il appela un des serviteurs, et il lui demanda ce que c'était.
\VS{27}Ce serviteur lui dit~: Ton frère est de retour, et ton père a tué le veau gras, parce qu'il l'a recouvré sain et sauf.
\VS{28}Mais il se mit en colère, et ne voulut pas entrer. Son père sortit et le pria d'entrer.
\VS{29}Mais il répondit, et dit à son père~: Voici, il y a tant d'années que je te sers, et jamais je n'ai transgressé ton commandement, et cependant tu ne m'as jamais donné un chevreau pour que je me réjouisse avec mes amis.
\VS{30}Mais quand ton fils est arrivé, celui qui a mangé ton bien avec des prostituées, c'est pour lui que tu as tué le veau gras~!
\VS{31}Et le père lui dit~: Mon enfant, tu es toujours avec moi, et tous mes biens sont à toi.
\VS{32}Or il fallait bien s'égayer et se réjouir, parce que ton frère que voici était mort et qu'il est ressuscité, parce qu'il était perdu et qu'il est retrouvé.
\Chap{16}
\TextTitle{Parabole de l'économe infidèle}
\VerseOne{}Il disait aussi à ses disciples~: Il y avait un homme riche qui avait un économe, qui fut accusé devant lui comme dissipant ses biens.
\VS{2}Il l'appela et lui dit~: Qu'est-ce que j'entends dire de toi~? Rends compte de ton administration, car tu n'auras plus le pouvoir d'administrer mes biens.
\VS{3}Alors l'économe dit en lui-même~: Que ferai-je, puisque mon maître m'ôte l'administration~? Travailler à la terre~? Je ne le puis. Mendier~? J'en ai honte.
\VS{4}Je sais ce que je ferai, afin que les gens me reçoivent dans leurs maisons quand mon administration me sera ôtée.
\VS{5}Alors il appela chacun des débiteurs de son maître, et il dit au premier~: Combien dois-tu à mon maître~?
\VS{6}Il dit~: Cent mesures d'huile. Et il lui dit~: Prends ton billet, et assied-toi vite, et écris cinquante.
\VS{7}Puis il dit à un autre~: Et toi, combien dois-tu~? Il dit~: Cent mesures de froment. Et il lui dit~: Prends ton billet, et écris quatre-vingts.
\VS{8}Et le maître loua l'économe infidèle de ce qu'il avait agi prudemment. Ainsi les enfants de ce siècle sont plus prudents dans leur génération, que les enfants de lumière.
\VS{9}Et moi aussi je vous dis~: Faites-vous des amis avec les richesses injustes, afin que quand vous viendrez à manquer, ils vous reçoivent dans les tabernacles éternels.
\VS{10}Celui qui est fidèle en très peu de choses, est fidèle aussi dans les grandes choses~; et celui qui est injuste en très peu de choses, est injuste aussi dans les grandes choses.
\VS{11}Si donc vous n'avez pas été fidèles dans les richesses injustes, qui vous confiera les véritables richesses~?
\VS{12}Et si en ce qui est à autrui vous n'avez pas été fidèles, qui vous donnera ce qui est vôtre~?
\VS{13}Nul serviteur ne peut servir deux maîtres. Car, ou il haïra l'un, et aimera l'autre~; ou il s'attachera à l'un, et méprisera l'autre. Vous ne pouvez pas servir Dieu et Mamon\FTNT{Voir commentaire en Mt. 6:24.}.
\TextTitle{L'avarice condamnée par Jésus}
\VS{14}Or les pharisiens aussi, qui étaient avares, entendaient toutes ces choses, et ils se moquaient de lui.
\VS{15}Et il leur dit~: Vous, vous cherchez à paraître justes devant les hommes, mais Dieu connaît vos cœurs~; c'est pourquoi ce qui est élevé parmi les hommes est une abomination devant Dieu.
\VS{16}La loi et les prophètes ont duré jusqu'à Jean~; depuis lors, le Royaume de Dieu est prêché, et chacun y fait violence.
\VS{17}Or il est plus aisé que le ciel et la terre passent, qu'il ne l'est qu'un trait de la lettre de la loi vienne à tomber.
\TextTitle{Enseignement de Jésus sur le divorce\FTNTT{Mt. 5:31-32~; 19:1-9~; Mc. 10:2-12.}}
\VS{18}Quiconque répudie sa femme et se marie à une autre, commet un adultère, et quiconque prend celle qui a été répudiée par son mari, commet un adultère.
\TextTitle{Histoire de l'homme riche et de Lazare}
\VS{19}Il y avait un homme riche, qui était vêtu de pourpre et de fin lin, et qui tous les jours se réjouissait d'une vie somptueuse.
\VS{20}Il y avait aussi un pauvre, nommé Lazare, couché à la porte du riche, tout couvert d'ulcères,
\VS{21}et qui désirait se rassasier des miettes qui tombaient de la table du riche~; et même les chiens venaient encore lécher ses ulcères.
\VS{22}Et il arriva que le pauvre mourut, et il fut porté par les anges dans le sein d'Abraham\FTNT{Le sein d'Abraham n'est pas le ciel car le Seigneur Jésus-Christ a dit que personne n'était monté au ciel si ce n'est celui qui était descendu du ciel, c'est-à-dire lui-même (Jn. 3:13). En effet, le chemin vers le saint des saints n'a pas été manifesté (ouvert) avant la croix. C'est donc Jésus-Christ qui l'a inauguré (Hé. 10:19-22). De plus, selon Ep. 4:8-10, Christ est descendu «~dans les régions inférieures de la terre~» et en est remonté avec des captifs, à savoir tous les croyants ayant vécu avant la croix. Le sein d'Abraham était une partie du scheol (également appelé enfer, séjour des morts ou Hadès~; voir commentaire en Mt. 16:18). L'histoire de Lazare et de l'homme riche démontre que le séjour des morts était divisé en deux parties séparées par un abîme infranchissable. L'une était réservée aux impies qui y subissaient des tourments, et l'autre aux croyants qui se reposaient en attendant leur délivrance. Ce récit est confirmé par l'histoire de Saül, un homme rejeté par Dieu, qui avait rejoint dans le scheol Samuel, un prophète intègre (1 S. 28:16-19).}. Le riche mourut aussi, et il fut enseveli.
\VS{23}Etant en enfer\FTNT{Le mot traduit par «~enfer~» vient du grec «~Hadès~». Voir commentaire en Mt. 16:18.}, il leva ses yeux~; et, tandis qu'il était dans les tourments, il vit de loin Abraham et Lazare dans son sein.
\VS{24}Il s'écria, et dit~: Père Abraham aie pitié de moi, et envoie Lazare, pour qu'il trempe le bout de son doigt dans l'eau et me rafraichisse la langue~; car je suis grièvement tourmenté dans cette flamme.
\VS{25}Abraham répondit~: Mon enfant, souviens-toi que tu as reçu tes biens pendant ta vie, et que Lazare a eu ses maux pendant la sienne~; maintenant il est ici consolé, et toi, tu es grièvement tourmenté.
\VS{26}D'ailleurs, il y a entre nous et vous un grand abîme, en sorte que ceux qui veulent passer d'ici vers vous ne le peuvent, et que ceux qui veulent passer de là ne traversent pas non plus vers nous.
\VS{27}Et il dit~: Je te prie donc, père, de l'envoyer dans la maison de mon père~; car j'ai cinq frères.
\VS{28}Afin qu'il leur rende témoignage de l'état où je suis~; de peur qu'eux aussi ne viennent dans ce lieu de tourment.
\VS{29}Abraham lui répondit~: Ils ont Moïse et les prophètes~; qu'ils les écoutent.
\VS{30}Mais il dit~: Non, père Abraham, mais si quelqu'un des morts va vers eux, ils se repentiront.
\VS{31}Et Abraham lui dit~: S'ils n'écoutent pas Moïse et les prophètes, ils ne seront pas non plus persuadés quand quelqu'un des morts ressusciterait.
\Chap{17}
\TextTitle{Instructions de Jésus au sujet des scandales, du pardon et de la foi\FTNTT{Mt. 5:31-32~; 19:1-9~; Mc. 10:2-12.}}
\VerseOne{}Or il dit à ses disciples~: Il est impossible qu'il n'arrive pas de scandales~; mais malheur à celui par qui ils arrivent.
\VS{2}Il vaudrait mieux pour lui qu'on lui mette une pierre de moulin autour de son cou, et qu'on le jette dans la mer, que de scandaliser un seul de ces petits.
\VS{3}Prenez garde à vous-mêmes. Si donc ton frère a péché contre toi, reprends-le~; et s'il se repent, pardonne-lui.
\VS{4}Et s'il a péché contre toi sept fois dans un jour et que sept fois il revienne à toi, disant~: Je me repens, tu lui pardonneras.
\VS{5}Alors les apôtres dirent au Seigneur~: Augmente-nous la foi.
\VS{6}Et le Seigneur dit~: Si vous aviez de la foi aussi gros qu'un grain de semence de moutarde, vous diriez à ce sycomore~: Déracine-toi, et plante-toi dans la mer~; et il vous obéirait.
\TextTitle{Les serviteurs inutiles}
\VS{7}Mais qui de vous, ayant un serviteur qui laboure ou paît les troupeaux, lui dira, quand il revient des champs~: Approche-toi vite, et mets-toi à table~?
\VS{8}Ne lui dira-t-il pas plutôt~: Prépare-moi à souper, ceins-toi, et sers-moi jusqu'à ce que j'aie mangé et bu~; et après cela tu mangeras et tu boiras~?
\VS{9}Doit-il de la reconnaissance à ce serviteur parce qu'il a fait ce qui lui était ordonné~? Je ne le pense pas.
\VS{10}Vous de même, quand vous aurez fait tout ce qui vous a été ordonné, dites~: Nous sommes des serviteurs inutiles~; et ce que nous étions obligés de faire, nous l'avons fait.
\TextTitle{Guérison des dix lépreux}
\VS{11}Et il arriva qu'en allant à Jérusalem, il passait par le milieu de la Samarie, et de la Galilée.
\VS{12}Et comme il entrait dans un village, dix hommes lépreux vinrent à sa rencontre. Se tenant à distance, ils élevèrent la voix, et dirent~:
\VS{13}Jésus, Maître, aie pitié de nous~!
\VS{14}Et quand il les eut vus, il leur dit~: Allez, montrez-vous aux prêtres\FTNT{Lé. 13.}. Et, pendant qu'ils y allaient, ils furent purifiés.
\VS{15}L'un d'eux se voyant guéri, revint sur ses pas, glorifiant Dieu à haute voix.
\VS{16}Et il se jeta en terre sur sa face aux pieds de Jésus, lui rendant grâces. Or c'était un Samaritain.
\VS{17}Alors Jésus, prenant la parole, dit~: Les dix n'ont-ils pas été rendus purs~? Et les neuf autres, où sont-ils~?
\VS{18}Il n'y a eu que cet étranger qui soit revenu pour rendre gloire à Dieu.
\VS{19}Alors il lui dit~: Lève-toi, va, ta foi t'a sauvé.
\TextTitle{Les pharisiens demandent à voir le Royaume\FTNTT{Lu. 19:11-27.}}
\VS{20}Or les pharisiens demandèrent à Jésus quand viendrait le Royaume de Dieu. Il leur répondit, et leur dit~: Le Royaume de Dieu ne vient pas de manière à attirer l'attention.
\VS{21}Et on ne dira point~: Il est ici~; ou~: Il est là. Car voici, le Royaume de Dieu est au milieu de vous.
\TextTitle{Jésus annonce sa seconde venue\FTNTT{voir De. 30:3.}}
\VS{22}Il dit aussi à ses disciples~: Des jours viendront où vous désirerez voir un des jours du Fils de l'homme, mais vous ne le verrez point. On vous dira~:
\VS{23}Il est ici~; ou~: Il est là. N'y allez pas, et ne les suivez point.
\VS{24}Car, comme l'éclair brille et resplendit d'une extrémité du ciel à l'autre, ainsi sera le Fils de l'homme en son jour.
\VS{25}Mais il faut premièrement qu'il souffre beaucoup, et qu'il soit rejeté par cette génération.
\VS{26}Ce qui arriva aux jours de Noé, arrivera de même aux jours du Fils de l'homme.
\VS{27}On mangeait et on buvait, on prenait et on donnait des femmes en mariage jusqu'au jour où Noé entra dans l'arche~; le déluge vint, et les fit tous périr.
\VS{28}C'est encore ce qui arriva aux jours de Lot~: On mangeait, on buvait, on achetait, on vendait, on plantait et on bâtissait.
\VS{29}Mais le jour où Lot sortit de Sodome, une pluie de feu et de soufre tomba du ciel, et les fit tous périr.
\VS{30}Il en sera de même au jour où le Fils de l'homme paraîtra.
\VS{31}En ce jour-là, que celui qui sera sur le toit, et qui aura ses effets dans la maison, ne descende point pour les prendre~; et que celui qui sera dans les champs, ne retourne pas non plus à ce qui est resté en arrière.
\VS{32}Souvenez-vous de la femme de Lot.
\VS{33}Quiconque cherchera à sauver sa vie la perdra, et quiconque la perdra la retrouvera.
\VS{34}Je vous dis, qu'en cette nuit-là deux seront dans un même lit~: L'un sera pris, et l'autre laissé~;
\VS{35}deux femmes moudront ensemble: L'une sera prise et l'autre laissée~;
\VS{36}deux seront aux champs: L'un sera pris et l'autre laissé.
\VS{37}Les disciples lui dirent~: Où sera-ce, Seigneur~? Et il leur dit~: Là où est le corps, là aussi s'assembleront les aigles.
\Chap{18}
\TextTitle{Parabole du juge inique}
\VerseOne{}Et il leur proposa une parabole, pour montrer qu'il faut toujours prier, et ne point se relâcher,
\VS{2}disant~: Il y avait dans une ville un juge qui ne craignait point Dieu et qui ne respectait personne.
\VS{3}Et dans la même ville, il y avait une veuve, qui venait souvent lui dire~: Fais-moi justice de ma partie adverse.
\VS{4}Pendant longtemps il refusa. Mais après cela il dit en lui-même~: Quoique je ne craigne point Dieu, et que je ne respecte personne,
\VS{5}néanmoins, parce que cette veuve me donne de la peine, je lui ferai justice, de peur qu'elle ne vienne sans cesse me casser la tête.
\VS{6}Et le Seigneur dit~: Ecoutez ce que dit le juge inique.
\VS{7}Et Dieu ne ferait-il point justice à ses élus, qui crient à lui jour et nuit, quoiqu'il use de patience avant d'intervenir pour eux~?
\VS{8}Je vous le dis que bientôt il les vengera. Mais quand le Fils de l'homme viendra, pensez-vous qu'il trouvera la foi sur la terre~?
\TextTitle{Parabole du pharisien et du publicain}
\VS{9}Il dit aussi cette parabole au sujet de certaines personnes se persuadant qu'elles étaient justes, et ne faisant aucun cas des autres~:
\VS{10}Deux hommes montèrent au temple pour prier, l'un était pharisien, et l'autre, publicain.
\VS{11}Le pharisien, se tenant debout, priait en lui-même en ces termes~: Ô Dieu, je te rends grâces de ce que je ne suis pas comme le reste des hommes, qui sont ravisseurs, injustes, adultères, ni même comme ce publicain.
\VS{12}Je jeûne deux fois la semaine, et je donne la dîme de tout ce que je possède.
\VS{13}Mais le publicain se tenant loin, n'osait même pas lever les yeux vers le ciel, mais il se frappait la poitrine, en disant~: Ô Dieu, sois apaisé envers moi qui suis pécheur~!
\VS{14}Je vous dis que celui-ci descendit dans sa maison justifié, plutôt que l'autre. Car quiconque s'élève, sera abaissé, et quiconque s'abaisse, sera élevé.
\TextTitle{Jésus et les petits enfants\FTNTT{Mt. 19:13-15~; Mc. 10:13-16.}}
\VS{15}Et quelques-uns lui présentèrent aussi de petits enfants, afin qu'il les touchât, mais les disciples voyant cela, reprenaient ceux qui les présentaient.
\VS{16}Mais Jésus les appela, et dit~: Laissez venir à moi les petits enfants, et ne les en empêchez pas~; car le Royaume de Dieu est pour ceux qui leur ressemblent.
\VS{17}Je vous le dis en vérité, quiconque ne recevra point comme un enfant le Royaume de Dieu n'y entrera point.
\TextTitle{Jésus dénonce l'attachement aux richesses\FTNTT{Mt. 19:16-30~; Mc. 10:17-31~; cp. Lu. 10:25-37.}}
\VS{18}Un chef interrogea Jésus et dit~: Maître qui est bon~! Que dois-je faire pour hériter la vie éternelle~?
\VS{19}Jésus lui dit~: Pourquoi m'appelles-tu bon~? Il n'y a de bon que Dieu seul\FTNT{Voir commentaire Mc. 10:18.}.
\VS{20}Tu connais les commandements~: Tu ne commettras point d'adultère. Tu ne tueras point. Tu ne déroberas point. Tu ne diras point de faux témoignage. Honore ton père et ta mère.
\VS{21}Et il lui dit~: J'ai observé toutes ces choses dès ma jeunesse.
\VS{22}Et quand Jésus eut entendu cela, il lui dit~: Il te manque encore une chose~: Vends tout ce que tu as, et distribue-le aux pauvres, et tu auras un trésor dans les cieux. Puis viens, et suis-moi.
\VS{23}Lorsqu'il entendit ces choses, il devint tout triste, car il était extrêmement riche.
\VS{24}Jésus voyant qu'il était devenu tout triste, dit~: Qu'il est difficile à ceux qui ont des richesses d'entrer dans le Royaume de Dieu~!
\VS{25}Car il est plus facile à un chameau de passer par le trou d'une aiguille, qu'à un riche d'entrer dans le Royaume de Dieu\FTNT{Voir commentaire Mt. 19:24.}.
\VS{26}Ceux qui entendirent cela, dirent~: Et qui peut donc être sauvé~?
\VS{27}Jésus leur répondit~: Ce qui est impossible aux hommes est possible à Dieu.
\TextTitle{La récompense des disciples de Jésus}
\VS{28}Pierre dit~: Voici, nous avons tout quitté, et nous t'avons suivi.
\VS{29}Et il leur dit~: Je vous le dis en vérité, il n'est personne qui, ayant quitté pour l'amour du Royaume de Dieu, sa maison, ou ses parents, ou ses frères, ou sa femme, ou ses enfants,
\VS{30}ne reçoive beaucoup plus dans ce siècle-ci, et dans le siècle à venir la vie éternelle.
\TextTitle{Jésus annonce à nouveau sa mort et sa résurrection\FTNTT{Mt. 20:17-19~; Mc. 10:32-34.}}
\VS{31}Jésus prit à part les douze, et il leur dit~: Voici, nous montons à Jérusalem, et tout ce qui est écrit par les prophètes au sujet du Fils de l'homme, s'accomplira.
\VS{32}Car il sera livré aux Gentils~; et on se moquera de lui, on l'outragera, et on lui crachera au visage,
\VS{33}et, après l'avoir battu de verges, on le fera mourir~; mais il ressuscitera le troisième jour.
\VS{34}Mais ils ne comprirent rien à cela, et ce discours était si obscur pour eux qu'ils ne comprirent point ce qu'il leur disait.
\TextTitle{Guérison de Bartimée \FTNTT{cp. Mt. 20:29-34~; Mc. 10:46-53.}}
\VS{35}Or comme il approchait de Jéricho, un aveugle était assis au bord du chemin, et mendiait.
\VS{36}Et entendant la foule qui passait, il demanda ce que c'était.
\VS{37}Et on lui dit~: C'est Jésus de Nazareth qui passe.
\VS{38}Alors il cria, disant~: Jésus, Fils de David, aie pitié de moi~!
\VS{39}Ceux qui marchaient devant le reprenaient, pour le faire taire~; mais il criait beaucoup plus fort~: Fils de David, aie pitié de moi~!
\VS{40}Et Jésus s'étant arrêté ordonna qu'on le lui amène~; et, quand il se fut approché,
\VS{41}il lui demanda~: Que veux-tu que je te fasse~? Il répondit~: Seigneur, que je recouvre la vue.
\VS{42}Jésus lui dit~: Recouvre la vue~; ta foi t'a sauvé.
\VS{43}Et à l'instant il recouvra la vue et suivit Jésus, glorifiant Dieu. Et tout le peuple voyant cela, loua Dieu.
\Chap{19}
\TextTitle{Conversion de Zachée}
\VerseOne{}Jésus, étant entré dans Jéricho, traversait la ville.
\VS{2}Et voici, un homme riche, appelé Zachée, chef des publicains, cherchait à voir qui était Jésus,
\VS{3}mais il ne le pouvait pas à cause de la foule, car il était de petite taille.
\VS{4}C'est pourquoi il accourut devant, et monta sur un sycomore pour le voir~; car il devait passer par là.
\VS{5}Et quand Jésus fut arrivé à cet endroit-là, il leva les yeux, le vit, et lui dit~: Zachée, hâte-toi de descendre~; car il faut que je demeure aujourd'hui dans ta maison.
\VS{6}Zachée se hâta de descendre, et le reçut avec joie.
\VS{7}Et tous voyant cela murmuraient, et disaient~: Il est entré chez un homme pécheur pour y loger.
\VS{8}Et Zachée, se présentant devant le Seigneur, lui dit~: Voici, Seigneur, je donne la moitié de mes biens aux pauvres~; et si j'ai fait tort de quelque chose à quelqu'un, je lui rends le quadruple\FTNT{Lé. 5:20-24.}.
\VS{9}Et Jésus lui dit~: Aujourd'hui le salut est entré dans cette maison~; parce que celui-ci aussi est fils d'Abraham.
\VS{10}Car le Fils de l'homme est venu chercher et sauver ce qui était perdu.
\TextTitle{Parabole des dix mines\FTNTT{Lu. 17:21.}}
\VS{11}Et comme ils entendaient ces choses, Jésus poursuivit son discours, et proposa une parabole, parce qu'il était près de Jérusalem, et qu'ils pensaient que le Royaume de Dieu allait immédiatement paraître.
\VS{12}Il dit donc~: Un homme noble s'en alla dans un pays éloigné, pour prendre possession d'un royaume, et revenir ensuite.
\VS{13}Il appela dix de ses serviteurs, il leur donna dix mines et leur dit~: Faites-les valoir jusqu'à ce que je revienne.
\VS{14}Or ses concitoyens le haïssaient, c'est pourquoi ils envoyèrent après lui une ambassade, pour dire~: Nous ne voulons pas que cet homme règne sur nous.
\VS{15}Il arriva donc, après qu'il fut de retour, et après avoir pris possession du royaume, qu'il fit appeler auprès de lui les serviteurs auxquels il avait confié son argent, afin de connaître comment chacun l'avait fait valoir.
\VS{16}Alors le premier vint, et dit~: Seigneur, ta mine a produit dix autres mines.
\VS{17}Il lui dit~: C'est bien, bon serviteur~; parce que tu as été fidèle en peu de choses, reçois le gouvernement de dix villes.
\VS{18}Et le second vint, et dit~: Seigneur, ta mine a produit cinq autres mines.
\VS{19}Il dit aussi à celui-ci~: Toi aussi, sois établi sur cinq villes.
\VS{20}Un autre vint, et dit~: Seigneur, voici ta mine que j'ai gardée enveloppée dans un linge~;
\VS{21}car j'avais peur de toi, parce que tu es un homme sévère~; tu prends ce que tu n'as point mis, et tu moissonnes ce que tu n'as pas semé.
\VS{22}Et il lui dit~: Méchant serviteur, je te jugerai par ta propre parole~; tu savais que je suis un homme sévère, prenant ce que je n'ai point mis, et moissonnant ce que je n'ai point semé.
\VS{23}Pourquoi donc n'as-tu pas mis mon argent dans une banque, afin qu'à mon retour je le retire avec un intérêt~?
\VS{24}Alors il dit à ceux qui étaient présents~: Ôtez-lui la mine, et donnez-la à celui qui a les dix.
\VS{25}Et ils lui dirent~: Seigneur, il a dix mines.
\VS{26}Ainsi je vous le dis, on donnera à celui qui a, mais à celui qui n'a pas, on ôtera ce qu'il a.
\VS{27}Au reste, amenez ici mes ennemis qui n'ont pas voulu que je règne sur eux, et tuez-les devant moi.
\TextTitle{Jésus fait son entrée à Jérusalem\FTNTT{Za. 9:9~; Mt. 21:1-11~; Mc. 11:1-11~; Jn. 12:12-19.}}
\VS{28}Et ayant dit ces choses, il allait devant eux, montant à Jérusalem.
\VS{29}Lorsqu'il approcha de Bethphagé et de Béthanie, vers la montagne appelée Montagne des Oliviers, Jésus envoya deux de ses disciples,
\VS{30}en leur disant~: Allez au village qui est en face~; quand vous y serez entrés, vous trouverez un ânon attaché, sur lequel aucun homme n'est monté~; détachez-le, et amenez-le-moi.
\VS{31}Si quelqu'un vous demande pourquoi le détachez-vous, vous lui répondrez~: Le Seigneur en a besoin.
\VS{32}Et ceux qui étaient envoyés s'en allèrent, et le trouvèrent comme il le leur avait dit.
\VS{33}Et comme ils le détachaient, ses maîtres leur dirent~: Pourquoi détachez-vous cet ânon~?
\VS{34}Ils répondirent~: Le Seigneur en a besoin.
\VS{35}Ils emmenèrent à Jésus l'ânon, sur lequel ils jetèrent leurs vêtements, et firent monter Jésus dessus.
\VS{36}Quand il fut en marche, les gens étendirent leurs vêtements sur le chemin.
\VS{37}Et lorsque déjà il approchait de Jérusalem, vers la descente de la Montagne des Oliviers, toute la multitude des disciples saisie de joie, se mit à louer Dieu à haute voix, pour tous les miracles qu'ils avaient vus.
\VS{38}Ils disaient~: Béni soit le Roi qui vient au Nom du Seigneur\FTNT{Ps. 118:26.}~! Paix dans le ciel, et gloire dans les lieux très hauts.
\VS{39}Quelques pharisiens, du milieu de la foule, lui dirent~: Maître, reprends tes disciples.
\VS{40}Et Jésus répondit~: Je vous le dis, s'ils se taisent, les pierres crieront.
\TextTitle{Nouvelles lamentations de Jésus sur Jérusalem\FTNTT{cp. Mt. 23:37-39~; Lu. 13:34-35.}}
\VS{41}Comme il approchait de la ville, Jésus, en la voyant, pleura sur elle, et dit~:
\VS{42}Ô, si toi aussi, au moins en ce jour qui t'est donné, tu connaissais les choses qui appartiennent à ta paix~! Mais maintenant elles sont cachées devant tes yeux.
\VS{43}Car des jours viendront sur toi, où tes ennemis t'entoureront de tranchées, et t'environneront, et te serreront de tous côtés,
\VS{44}ils te raseront, toi et tes enfants qui sont au milieu de toi, et ils ne laisseront pas en toi pierre sur pierre, parce que tu n'as pas connu le temps de ta visitation.
\TextTitle{Jésus chasse les marchands du temple}
\VS{45}Il entra dans le temple, et il se mit à chasser dehors ceux qui vendaient et qui achetaient,
\VS{46}leur disant~: Il est écrit~: Ma maison sera appelée la maison de prière. Mais vous, vous en avez fait une caverne de voleurs\FTNT{Es. 56:7~; Jé. 7:11.}.
\VS{47}Il enseignait tous les jours dans le temple. Et les principaux prêtres et les scribes, et les principaux du peuple cherchaient à le faire mourir.
\VS{48}Mais ils ne savaient comment s'y prendre~; car tout le peuple s'attachait à ses paroles.
\Chap{20}
\TextTitle{L'autorité de Jésus et celle de Jean-Baptiste\FTNTT{Mt. 21:23-27~; Mc. 11:27-33.}}
\VerseOne{}Et il arriva un de ces jours-là, comme Jésus enseignait le peuple dans le temple, et qu'il évangélisait, les principaux prêtres, les scribes et les anciens survinrent,
\VS{2}et lui parlèrent en disant~: Dis-nous par quelle autorité fais-tu ces choses, ou qui est celui qui t'a donné cette autorité~?
\VS{3}Et Jésus leur répondit~: Je vous demanderai, moi aussi, une chose. Et répondez-moi,
\VS{4}le baptême de Jean était-il du ciel ou des hommes~?
\VS{5}Ils raisonnaient entre eux, disant~: Si nous répondons~: Du ciel~; il dira~: Pourquoi n'avez-vous pas cru en lui~?
\VS{6}Et si nous répondons~: Des hommes, tout le peuple nous lapidera~; car il est persuadé que Jean était un prophète.
\VS{7}C'est pourquoi ils répondirent qu'ils ne savaient d'où il était.
\VS{8}Et Jésus leur dit~: Moi non plus, je ne vous dirai pas par quelle autorité je fais ces choses.
\TextTitle{Parabole des vignerons\FTNTT{Es. 5:1-7~; Mt. 21:33-46~; Mc. 12:1-12.}}
\VS{9}Alors il se mit à dire au peuple cette parabole~: Un homme planta une vigne, et la loua à des vignerons, et s'en alla hors du pays pour longtemps.
\VS{10}Et à la saison de la récolte, il envoya un serviteur vers les vignerons, afin qu'ils lui donnent du fruit de la vigne. Les vignerons le battirent, et le renvoyèrent à vide.
\VS{11}Et il leur envoya encore un autre serviteur~; mais ils le battirent aussi, et après l'avoir traité indignement, ils le renvoyèrent à vide.
\VS{12}Et il en envoya encore un troisième, mais ils le blessèrent aussi, et le jetèrent dehors.
\VS{13}Alors le maître de la vigne dit~: Que ferai-je~? J'enverrai mon fils bien-aimé~; peut-être que quand ils le verront, ils le respecteront.
\VS{14}Mais quand les vignerons le virent, ils raisonnèrent entre eux, et dirent~: Voici l'héritier~; venez, tuons-le, afin que l'héritage soit à nous.
\VS{15}Et ils le jetèrent hors de la vigne, et le tuèrent. Que leur fera donc le maître de la vigne~?
\VS{16}Il viendra, et fera périr ces vignerons-là, et il donnera la vigne à d'autres. Lorsqu'ils entendirent cela, ils dirent~: A Dieu ne plaise~!
\VS{17}Alors il les regarda, et dit~: Que veut donc dire ce qui est écrit~: La pierre qu'ont rejetée ceux qui bâtissaient est devenue la principale de l'angle\FTNT{Ps. 118:22.}~?
\VS{18}Quiconque tombera sur cette pierre, sera brisé, et elle écrasera celui sur qui elle tombera.
\VS{19}Les principaux prêtres et les scribes cherchèrent à mettre la main sur lui à l'heure même, mais ils craignirent le peuple. Ils avaient compris que c'était pour eux que Jésus avait dit cette parabole.
\TextTitle{Le tribut à César\FTNTT{Mt. 22:15-22~; Mc. 12:13-17.}}
\VS{20}Ils se mirent à observer Jésus~; et ils envoyèrent des espions, qui feignaient être des hommes justes, pour lui tendre des pièges et saisir de lui quelque parole afin de le livrer au magistrat et à l'autorité du gouverneur.
\VS{21}Et ils l'interrogèrent, en disant~: Maître, nous savons que tu parles et enseignes conformément à la justice, et que tu ne regardes pas à l'apparence des personnes, mais que tu enseignes la voie de Dieu selon la vérité.
\VS{22}Nous est-il permis de payer le tribut à César, ou non~?
\VS{23}Jésus, apercevant leur ruse, leur dit~: Pourquoi me tentez-vous~?
\VS{24}Montrez-moi un denier. De qui a-t-il l'image et l'inscription~? Ils lui répondirent~: De César.
\VS{25}Alors il leur dit~: Rendez donc à César ce qui est à César, et à Dieu ce qui est à Dieu.
\VS{26}Ainsi ils ne purent le surprendre dans ses paroles devant le peuple~; mais, étonnés de sa réponse, ils gardèrent le silence.
\TextTitle{Les preuves de la résurrection\FTNTT{Mt. 22:23-33~; Mc. 12:18-27.}}
\VS{27}Alors quelques-uns des sadducéens, qui nient formellement la résurrection, s'approchèrent et l'interrogèrent,
\VS{28}disant~: Maître, voici ce que Moïse nous a laissé par écrit~: Si le frère de quelqu'un meurt, ayant une femme et pas d'enfants, son frère épousera la femme, et suscitera une postérité à son frère.
\VS{29}Or il y avait sept frères. Le premier se maria, et mourut sans enfants.
\VS{30}Le deuxième épousa la femme et mourut sans enfants.
\VS{31}Puis le troisième l'épousa aussi, et tous les sept de même~; et ils moururent sans laisser d'enfants.
\VS{32}Enfin, la femme mourut aussi.
\VS{33}Duquel d'entre eux donc sera-t-elle la femme à la résurrection~? Car les sept l'ont eue pour femme.
\VS{34}Jésus leur répondit~: Les enfants de ce siècle prennent des femmes et des maris~;
\VS{35}mais ceux qui seront trouvés dignes d'avoir part au siècle à venir et à la résurrection des morts, ne prendront ni femmes ni maris.
\VS{36}Car ils ne pourront plus mourir, parce qu'ils seront semblables aux anges, et qu'ils seront fils de Dieu, étant fils de la résurrection.
\VS{37}Que les morts ressuscitent, c'est ce que Moïse a fait connaître quand, à propos du buisson, il appelle le Seigneur le Dieu d'Abraham, et le Dieu d'Isaac, et le Dieu de Jacob.
\VS{38}Or il n'est pas le Dieu des morts, mais des vivants~; car tous vivent en lui.
\TextTitle{Jésus dénonce l'attitude des scribes\FTNTT{cp. Mt. 22:41-23:36~; Mc. 12:35-40.}}
\VS{39}Quelques-uns des scribes prenant la parole, dirent~: Maître, tu as bien parlé.
\VS{40}Et ils n'osaient plus lui poser aucune question.
\VS{41}Mais il leur dit~: Comment dit-on que le Christ est Fils de David~?
\VS{42}Car David lui-même dit au livre des psaumes~: Le Seigneur a dit à mon Seigneur~: Assieds-toi à ma droite,
\VS{43}jusqu'à ce que j'aie mis tes ennemis pour le marchepied de tes pieds\FTNT{Ps. 110:1.}.
\VS{44}David donc l'appelle son Seigneur, comment est-il son Fils~?
\VS{45}Comme tout le peuple écoutait, il dit à ses disciples~:
\VS{46}Gardez-vous des scribes, qui se plaisent à se promener en robes longues, qui aiment les salutations dans les places publiques, et les premiers sièges dans les synagogues, et les premières places dans les festins~;
\VS{47}qui dévorent entièrement les maisons des veuves, et qui font pour l'apparence de longues prières. Ils en recevront une plus grande condamnation.
\Chap{21}
\TextTitle{Offrande de la pauvre veuve\FTNTT{Mc. 12:41-44.}}
\VerseOne{}Comme Jésus regardait, il vit des riches qui mettaient leurs offrandes dans le tronc.
\VS{2}Il vit aussi une pauvre veuve qui y mettait deux petites pièces de monnaie.
\VS{3}Et il dit~: Je vous le dis en vérité, cette pauvre veuve a mis plus que tous les autres.
\VS{4}Car tous ceux-ci ont mis aux offrandes de Dieu, de leur superflu~; mais elle a mis de son nécessaire, tout ce qu'elle avait pour vivre.
\TextTitle{Enseignement sur le Mont des Oliviers\FTNTT{Mt. 24-25~; Mc. 13.}}
\VS{5}Et comme quelques-uns disaient du temple, qu'il était orné de belles pierres, et d'offrandes, il dit~:
\VS{6}Est-ce cela que vous regardez~? Les jours viendront où, il ne restera pas pierre sur pierre qui ne soit démolie.
\TextTitle{Les disciples posent deux questions à Jésus\FTNTT{Mt. 24:3~; Mc. 13:3-4.}}
\VS{7}Et ils lui demandèrent~: Maître, quand donc cela arrivera-t-il, et à quel signe connaîtra-t-on que ces choses vont arriver~?
\TextTitle{Les temps de la fin\FTNTT{Mt. 24:4-14~; Mc. 13:5-13.}}
\VS{8}Et il dit~: Prenez garde que vous ne soyez point séduits. Car plusieurs viendront en mon Nom, disant~: C'est moi qui suis le Christ et le temps approche. Ne les suivez pas.
\VS{9}Quand vous entendrez parler des guerres et des soulèvements, ne vous épouvantez point, car il faut que ces choses arrivent premièrement. Mais ce ne sera pas encore la fin.
\VS{10}Alors il leur dit~: Une nation s'élèvera contre une autre nation, et un royaume contre un autre royaume.
\VS{11}Il y aura de grands tremblements de terre en divers lieux, des famines et des pestes~; il y aura des choses terribles, et de grands signes dans le ciel.
\TextTitle{Souffrance des croyants}
\VS{12}Mais, avant toutes ces choses, ils mettront les mains sur vous, et vous persécuteront, vous livrant aux synagogues, et vous mettant en prison~; et ils vous mèneront devant les rois et les gouverneurs, à cause de mon Nom.
\VS{13}Et cela vous arrivera pour que vous serviez de témoignage.
\VS{14}Mettez donc dans vos cœurs de ne pas préméditer votre défense.
\VS{15}Car je vous donnerai une bouche et une sagesse à laquelle vos adversaires ne pourront contredire, ni résister.
\VS{16}Vous serez aussi livrés par vos parents, par vos frères, par vos proches et par vos amis, et ils feront mourir plusieurs d'entre vous.
\VS{17}Et vous serez haïs de tous à cause de mon Nom.
\VS{18}Mais il ne se perdra pas un cheveu de votre tête.
\VS{19}Possédez vos âmes par votre patience.
\TextTitle{La destruction de Jérusalem prophétisée}
\VS{20}Et quand vous verrez Jérusalem environnée par les armées, sachez alors que sa désolation est proche.
\VS{21}Alors, que ceux qui seront en Judée, fuient dans les montagnes, et que ceux qui seront au milieu de Jérusalem, en sortent, et que ceux qui seront dans les champs, n'entrent pas en elle.
\VS{22}Car ce seront là des jours de vengeance, afin que toutes les choses qui sont écrites soient accomplies.
\VS{23}Or malheur à celles qui seront enceintes, et à celles qui allaiteront en ces jours-là~! Car il y aura une grande calamité sur le pays, et une grande colère contre ce peuple.
\VS{24}Et ils tomberont sous le tranchant de l'épée, ils seront emmenés captifs\FTNT{Les Romains, installés en Palestine depuis 65 av. J.-C., essuyèrent plusieurs révoltes juives. L'une des plus connues est celle organisée par les zélotes (voir commentaire en Lu. 6:15) en août 66 et qui aboutit en 70 à la ruine de Jérusalem et à la destruction du temple par l'armée de Titus (41-81). Soixante-deux ans plus tard, en 132, Bar Kokhba organisa une armée après avoir établi un état Juif en Judée et déclencha une rébellion qui deviendra la seconde guerre judéo-romaine. En 135, les légionnaires de l'empereur Hadrien (76-138) matèrent définitivement l'insurrection dans un bain de sang qui couta la vie à quelques 580 000 Judéens. Jérusalem fut alors totalement détruite et interdite aux Juifs. Ces événements marquèrent le début de la diaspora juive dans le monde entier.} parmi toutes les nations, et Jérusalem sera foulée par les nations, jusqu'à ce que les temps des nations soient accomplis.
\TextTitle{Retour du Messie sur la terre\FTNTT{Mt. 24:29-31~; Mc. 13:24-27.}}
\VS{25}Et il y aura des signes dans le soleil, dans la lune, et dans les étoiles, et une telle détresse des nations, qu'on ne saura que devenir sur la terre, au bruit de la mer et des flots,
\VS{26}de sorte que les hommes seront comme rendant l'âme de frayeur, dans l'attente des choses qui surviendront sur toute la terre~; car les puissances des cieux seront ébranlées.
\VS{27}Et alors on verra le Fils de l'homme venant sur une nuée avec puissance et grande gloire.
\VS{28}Quand ces choses commenceront à arriver, regardez en haut et levez vos têtes, parce que votre délivrance approche.
\TextTitle{Parabole du figuier\FTNTT{Mt. 24:29-31~; Mc. 13:24-27.}}
\VS{29}Et il leur proposa cette comparaison~: Voyez le figuier et tous les autres arbres.
\VS{30}Quand ils commencent à pousser, vous connaissez de vous-mêmes, en regardant, que l'été est déjà proche.
\VS{31}Vous aussi de même, quand vous verrez arriver ces choses, sachez que le Royaume de Dieu est proche.
\VS{32}En vérité je vous le dis, que cette génération ne passera point, que toutes ces choses ne soient arrivées.
\VS{33}Le ciel et la terre passeront, mais mes paroles ne passeront point.
\TextTitle{Exhortation à veiller\FTNTT{Mt. 24:36-51~; Mc. 13:32-37.}}
\VS{34}Prenez donc garde à vous-mêmes, de peur que vos cœurs ne soient appesantis par la gourmandise et l'ivrognerie, et par les soucis de cette vie~; et que ce jour-là ne vous surprenne subitement.
\VS{35}Car il surprendra comme un filet tous ceux qui habitent sur la surface de toute la terre.
\VS{36}Veillez donc, et priez en tout temps, afin que vous soyez trouvés dignes d'échapper à toutes ces choses qui arriveront, et de vous tenir devant le Fils de l'homme.
\VS{37}Et il passait les jours dans le temple à enseigner, et la nuit, il sortait et demuerait à la montagne appelée des Oliviers.
\VS{38}Et dès le point du jour, tout le peuple venait vers lui au temple pour l'entendre.
\Chap{22}
\TextTitle{Trahison de Judas\FTNTT{Mt. 26:14-16~; Mc. 14:1-2,10-11.}}
\VerseOne{}Or la fête des pains sans levain, qu'on appelle Pâque, approchait.
\VS{2}Les principaux prêtres et les scribes cherchaient comment ils pourraient le faire mourir~; mais ils craignaient le peuple.
\VS{3}Mais Satan entra dans Judas, surnommé Iscariot, qui était du nombre des douze.
\VS{4}Et il s'en alla et parla avec les principaux prêtres et les chefs des gardes, sur la manière de le leur livrer.
\VS{5}Et ils se réjouirent, et convinrent de lui donner de l'argent.
\VS{6}Et il s'engagea~; et il cherchait une occasion favorable pour le leur livrer à l'insu de la foule.
\TextTitle{La dernière Pâque\FTNTT{Mt. 26:17-25~; Mc. 14:12-21~; Jn. 13:1-12.}}
\VS{7}Or le jour des pains sans levain, où l'on devait immoler la Pâque, arriva.
\VS{8}Et Jésus envoya Pierre et Jean, en leur disant~: Allez, et apprêtez-nous l'agneau de Pâque, afin que nous le mangions.
\VS{9}Et ils lui dirent~: Où veux-tu que nous l'apprêtions~?
\VS{10}Il leur dit~: Voici, quand vous serez entrés dans la ville vous rencontrerez un homme portant une cruche d'eau, suivez-le dans la maison où il entrera.
\VS{11}Et dites au maître de la maison~: Le Maître te dit~: Où est le lieu où je mangerai l'agneau de Pâque avec mes disciples~?
\VS{12}Et il vous montrera une grande chambre haute, meublée~; c'est là que vous apprêterez l'agneau de Pâque.
\VS{13}Ils partirent, et trouvèrent les choses comme il leur avait dit~; et ils apprêtèrent l'agneau de Pâque.
\VS{14}Et quand l'heure fut venue, il se mit à table, et les douze apôtres avec lui.
\VS{15}Et il leur dit~: J'ai désiré vivement manger cet agneau de Pâque avec vous, avant de souffrir~;
\VS{16}car, je vous dis, que je n'en mangerai plus jusqu'à ce qu'il soit accompli dans le Royaume de Dieu.
\VS{17}Et, ayant pris la coupe, il rendit grâces, et il dit~: Prenez cette coupe, et distribuez-la entre vous.
\VS{18}Car, je vous dis, que je ne boirai plus du fruit de la vigne, jusqu'à ce que le Royaume de Dieu soit venu.
\TextTitle{Le repas de la Pâque\FTNTT{Mt. 26:26-29~; Mc. 14:22-25~; cp. Jn. 13:12-30~; 1 Co. 11:23-26.}}
\VS{19}Puis il prit du pain, et après avoir rendu grâces, il le rompit et le leur donna, en disant~: Ceci est mon corps, qui est donné pour vous~; faites ceci en mémoire de moi.
\VS{20}De même aussi, il prit la coupe après le souper, et la leur donna, en disant~: Cette coupe est la Nouvelle Alliance en mon sang, qui est répandu pour vous.
\TextTitle{Jésus annonce qu'il sera livré\FTNTT{Mt. 26:21-25~; Mc. 14:18-21~; Jn. 13:18-30.}}
\VS{21}Cependant voici, la main de celui qui me trahit est avec moi à table.
\VS{22}Le Fils de l'homme s'en va selon ce qui est déterminé~: Toutefois malheur à cet homme par qui il est trahi~!
\VS{23}Alors ils commencèrent à se demander l'un à l'autre, qui donc serait celui d'entre eux qui allait faire cela.
\TextTitle{Leçon d'humilité\FTNTT{Mt. 20:20-28~; Mc. 9:33-37~; 10:35-45~; Jn. 13:1-17.}}
\VS{24}Il s'éleva une contestation parmi eux, pour savoir lequel d'entre eux serait estimé le plus grand.
\VS{25}Mais il leur dit~: Les rois des nations les maîtrisent, et ceux qui exercent l'autorité sur elles sont appelés bienfaiteurs.
\VS{26}Mais il n'en sera pas ainsi de vous~: Au contraire, que le plus grand parmi vous soit comme le plus petit, et celui qui gouverne, comme celui qui sert.
\VS{27}Car lequel est le plus grand, celui qui est à table, ou celui qui sert~? N'est-ce pas celui qui est à table~? Or je suis au milieu de vous comme celui qui sert.
\TextTitle{Le Royaume, une récompense}
\VS{28}Vous, vous êtes ceux qui avez persévéré avec moi dans mes épreuves~;
\VS{29}c'est pourquoi je vous confie le Royaume comme mon Père me l'a confié,
\VS{30}afin que vous mangiez et que vous buviez à ma table dans mon Royaume, et que vous soyez assis sur des trônes, pour juger les douze tribus d'Israël.
\TextTitle{Jésus prophétise le triple reniement de Pierre\FTNTT{Mt. 26:30-35~; Mc. 14:26-31~; Jn. 13:36-38.}}
\VS{31}Le Seigneur dit aussi~: Simon, Simon, voici, Satan vous a réclamés pour vous cribler comme le froment~;
\VS{32}mais j'ai prié pour toi afin que ta foi ne défaille point~; et toi donc, quand tu seras un jour converti, affermis tes frères.
\VS{33}Pierre lui dit~: Seigneur, je suis prêt à aller avec toi en prison et à la mort.
\VS{34}Mais Jésus lui dit~: Pierre, je te dis que le coq ne chantera pas aujourd'hui, que premièrement tu n'aies nié trois fois de me connaître.
\TextTitle{Recommandation aux disciples\FTNTT{cp. Jn. 14-16~; contraste Mt. 10:9-13.}}
\VS{35}Puis il leur dit~: Quand je vous ai envoyés sans bourse, sans sac, et sans souliers, avez-vous manqué de quelque chose~? Ils répondirent~: De rien.
\VS{36}Et il leur dit~: Mais maintenant, que celui qui a une bourse la prenne, et de même celui qui a un sac~; et que celui qui n'a point d'épée vende son vêtement, et achète une épée.
\VS{37}Car, je vous le dis, il faut que cette parole qui est écrite s'accomplisse en moi~: Il a été mis au nombre des malfaiteurs\FTNT{Es. 53:12.}. Parce qu'en effet, ce qui me concerne est sur le point d'arriver.
\VS{38}Ils dirent~: Seigneur, voici ici deux épées. Et il leur dit~: C'est assez.
\TextTitle{Gethsémané\FTNTT{Mt. 26:36-46~; Mc. 14:32-42~; Jn. 18:1~; cp. Hé. 5:7-8.}}
\VS{39}Après être sorti, il alla, selon sa coutume, au Mont des Oliviers~; et ses disciples le suivirent.
\VS{40}Et quand il fut arrivé en ce lieu-là, il leur dit~: Priez afin que vous ne tombiez pas en tentation.
\VS{41}Puis s'étant éloigné d'eux à la distance d'environ un jet de pierre, et s'étant mis à genoux, il pria,
\VS{42}disant~: Père, si tu voulais éloigner cette coupe loin de moi~! Toutefois, que ma volonté ne soit point faite, mais la tienne.
\VS{43}Et un ange lui apparut du ciel, pour le fortifier.
\VS{44}Etant en agonie, il priait plus instamment, et sa sueur devint comme des grumeaux de sang qui tombaient à terre.
\VS{45}Après avoir prié, il revint vers ses disciples, qu'il trouva endormis de tristesse,
\VS{46}et il leur dit~: Pourquoi dormez-vous~? Levez-vous, et priez, afin que vous ne tombiez pas en tentation.
\TextTitle{Trahison de Judas\FTNTT{Mt. 26:47-54~; Mc. 14:43-47~; Jn. 18:2-11.}}
\VS{47}Et comme il parlait encore, voici une foule arriva~; et celui qui s'appelait Judas, l'un des douze, marchait devant elle. Il s'approcha de Jésus pour le baiser.
\VS{48}Et Jésus lui dit~: Judas, c'est par un baiser que tu trahis le Fils de l'homme~?
\VS{49}Alors ceux qui étaient autour de lui, voyant ce qui allait arriver, lui dirent~: Seigneur, frapperons-nous de l'épée~?
\VS{50}Et l'un d'eux frappa le serviteur du grand-prêtre, et lui emporta l'oreille droite.
\VS{51}Mais Jésus prenant la parole dit~: Laissez-les faire jusqu'ici. Et, ayant touché son oreille, il le guérit.
\VS{52}Puis Jésus dit aux principaux prêtres, aux chefs des gardes du temple, et aux anciens qui étaient venus contre lui~: Etes-vous venus comme après un brigand avec des épées et des bâtons~?
\VS{53}J'étais tous les jours avec vous dans le temple, et vous n'avez pas mis la main sur moi. Mais c'est ici votre heure, et la puissance des ténèbres.
\TextTitle{Pierre renie Jésus\FTNTT{Mt. 26:55-58,69-75~; Mc. 14:48-54,66-72~; Jn. 18:15-18,25-27.}}
\VS{54}Après avoir saisi Jésus, ils l'emmenèrent, et le conduisirent dans la maison du grand-prêtre~; et Pierre suivait de loin.
\VS{55}Et lorsqu'ils eurent allumé un feu au milieu de la cour et qu'ils se furent assis ensemble, Pierre s'assit parmi eux.
\VS{56}Et une servante le voyant assis auprès du feu, fixa sur lui les regards, et dit~: Celui-ci aussi était avec lui.
\VS{57}Mais il le nia, disant~: Femme, je ne le connais point.
\VS{58}Peu après, un autre le voyant, dit~: Tu es aussi de ces gens-là, mais Pierre dit~: Ô homme~! Je n'en suis point.
\VS{59}Et environ une heure plus tard, un autre affirmait, et disait~: Certainement celui-ci aussi était avec lui, car il est Galiléen.
\VS{60}Et Pierre dit~: Ô homme~! Je ne sais pas ce que tu dis. Au même instant, comme il parlait encore, le coq chanta.
\VS{61}Et le Seigneur, s'étant retourné, regarda Pierre. Et Pierre se souvint de la parole que le Seigneur lui avait dite~: Avant que le coq chante, tu me renieras trois fois.
\VS{62}Alors Pierre étant sorti dehors, pleura amèrement.
\TextTitle{Jésus est outragé\FTNTT{Mt. 26:67-68~; Mc. 14:65~; Jn. 18:22-23.}}
\VS{63}Les hommes qui tenaient Jésus se moquaient de lui, et le frappaient.
\VS{64}Ils lui bandèrent les yeux, ils lui donnaient des coups sur le visage, et l'interrogeaient, disant~: Devine qui est celui qui t'a frappé~?
\VS{65}Et ils proféraient contre lui beaucoup d'autres injures.
\TextTitle{Jésus déclare qu'il est le Fils de Dieu\FTNTT{Mt. 26:59-68~; 27:1~; Mc. 14:55-65~; 15:1~; Jn. 18:19-24.}}
\VS{66}Et quand le jour fut venu, les anciens du peuple, les principaux prêtres et les scribes, s'assemblèrent, et firent amener Jésus dans le sanhédrin.
\VS{67}Ils dirent~: Si tu es le Christ, dis-le-nous. Et il leur répondit~: Si je vous le dis, vous ne le croirez point~;
\VS{68}et si je vous interroge, vous ne me répondrez pas, et vous ne me laisserez pas aller.
\VS{69}Désormais le Fils de l'homme sera assis à la droite de la puissance de Dieu.
\VS{70}Alors ils dirent tous~: Tu es donc le Fils de Dieu~? Et il leur répondit~: Vous le dites vous-mêmes, je le suis.
\VS{71}Alors ils dirent~: Qu'avons-nous besoin encore de témoignage~? Nous l'avons entendu nous-mêmes de sa bouche.
\Chap{23}
\TextTitle{Jésus devant Pilate\FTNTT{Mt. 27:2,11-14~; Mc. 15:1-5~; Jn. 18:28-38.}}
\VerseOne{}Puis ils se levèrent tous, et ils conduisirent Jésus devant Pilate.
\VS{2}Et ils se mirent à l'accuser, disant~: Nous avons trouvé cet homme excitant notre nation à la révolte, et défendant de payer le tribut à César, et se disant lui-même Christ, Roi.
\VS{3}Pilate l'interrogea, disant~: Es-tu le Roi des Juifs~? Et Jésus lui répondit~: Tu le dis.
\VS{4}Alors Pilate dit aux principaux prêtres et à la foule~: Je ne trouve aucun crime en cet homme.
\VS{5}Mais ils insistèrent encore davantage, et dirent~: Il soulève le peuple, enseignant par toute la Judée, depuis la Galilée où il a commencé, jusqu'ici.
\TextTitle{Jésus envoyé devant Hérode}
\VS{6}Or quand Pilate entendit parler de la Galilée, il demanda si cet homme était Galiléen,
\VS{7}et, ayant appris qu'il était de la juridiction d'Hérode, il le renvoya à Hérode, qui se trouvait aussi à Jérusalem.
\VS{8}Et lorsque Hérode vit Jésus, il en eut une grande joie~; car depuis longtemps il désirait le voir, à cause de ce qu'il avait entendu dire de lui, et il espérait qu'il le verrait faire quelque miracle.
\VS{9}Il lui adressa beaucoup de questions~; mais Jésus ne lui répondit rien.
\VS{10}Et les principaux prêtres et les scribes étaient là, et l'accusaient avec une grande véhémence.
\VS{11}Mais Hérode, avec ses gardes, le traita avec mépris~; et, après s'être moqué de lui et l'avoir revêtu d'un vêtement éclatant, il le renvoya à Pilate.
\VS{12}Et ce même jour, Pilate et Hérode devinrent amis~; car auparavant ils étaient ennemis.
\TextTitle{Hérode renvoie Jésus à Pilate\FTNTT{Mt. 27:15-26~; Mc. 15:6-15~; Jn. 18:39-19:15.}}
\VS{13}Alors Pilate, ayant assemblé les principaux prêtres, les magistrats, et le peuple, leur dit~:
\VS{14}Vous m'avez présenté cet homme comme soulevant le peuple. Et voici, je l'ai interrogé devant vous, et je ne l'ai trouvé coupable d'aucun des crimes dont vous l'accusez~;
\VS{15}Hérode non plus~; car il nous l'a renvoyé, et voici, cet homme n'a rien fait qui soit digne de mort.
\VS{16}Je le relâcherai donc, après l'avoir châtié.
\VS{17}Or il était obligé de leur relâcher quelqu'un à la fête.
\VS{18}Toutes les foules s'écrièrent ensemble, disant~: Ôte celui-ci, et relâche-nous Barabbas.
\VS{19}Cet homme avait été mis en prison pour une sédition qui avait eu lieu dans la ville, et pour un meurtre.
\VS{20}Pilate leur parla de nouveau, désirant relâcher Jésus.
\VS{21}Et ils crièrent~: Crucifie, crucifie-le~!
\VS{22}Et il leur dit pour la troisième fois~: Mais quel mal a fait cet homme~? Je ne trouve rien en lui qui soit digne de mort. Après l'avoir fait battre de verges, je le relâcherai.
\VS{23}Mais ils insistèrent à grands cris, demandant qu'il soit crucifié. Et leurs cris et ceux des principaux prêtres l'emportèrent.
\VS{24}Alors Pilate prononça que ce qu'ils demandaient serait fait.
\VS{25}Et il leur relâcha celui qui, pour sédition et pour meurtre, avait été mis en prison, celui qu'ils demandaient~; et il abandonna Jésus à leur volonté.
\TextTitle{Sur le chemin de Golgotha\FTNTT{Mt. 27:31-32~; Mc. 15:20-21~; Jn. 19:16-17.}}
\VS{26}Et comme ils l'emmenaient, ils prirent un certain Simon de Cyrène, qui revenait des champs, et le chargèrent de la croix pour qu'il la porte derrière Jésus.
\VS{27}Il était suivi d'une grande multitude des gens du peuple et de femmes, qui se frappaient la poitrine, et se lamentaient sur lui.
\VS{28}Mais Jésus se tourna vers elles, leur dit~: Filles de Jérusalem, ne pleurez point sur moi, mais pleurez sur vous-mêmes, et sur vos enfants.
\VS{29}Car voici, des jours viendront où l'on dira~: Heureuses les stériles, les entrailles qui n'ont point enfanté, et les mamelles qui n'ont point allaité~!
\VS{30}Alors ils se mettront à dire aux montagnes~: Tombez sur nous. Et aux collines~: Couvrez-nous~!
\VS{31}Car s'ils font ces choses au bois vert, que sera-t-il fait au bois sec~?
\VS{32}Deux autres aussi, qui étaient des malfaiteurs, furent menés pour les faire mourir avec lui.
\TextTitle{Crucifixion de Jésus\FTNTT{Mt. 27:33-43~; Mc. 15:24-32~; Jn. 19:17-37.}}
\VS{33}Et quand ils furent arrivés au lieu qui est appelé Calvaire (le Crâne), ils le crucifièrent là, et les malfaiteurs aussi, l'un à la droite, et l'autre à la gauche.
\VS{34}Jésus dit~: Père, pardonne-leur, car ils ne savent pas ce qu'ils font. Ils se partagèrent ensuite ses vêtements, en tirant au sort.
\VS{35}Et le peuple se tenait là, et regardait. Les magistrats se moquaient de Jésus disant~: Il a sauvé les autres, qu'il se sauve lui-même, s'il est le Christ, l'élu de Dieu~!
\VS{36}Les soldats aussi se moquaient de lui~; s'approchant et lui présentant du vinaigre,
\VS{37}et disant~: Si tu es le Roi des Juifs, sauve-toi toi-même.
\VS{38}Or il y avait au-dessus de lui un écriteau en lettres Grecques, Romaines et Hébraïques, en ces mots~: CELUI-CI EST LE ROI DES JUIFS.
\TextTitle{Repentance du malfaiteur crucifié\FTNTT{cp. Mt. 27:44~; Mc. 15:32.}}
\VS{39}L'un des malfaiteurs qui étaient crucifiés, l'outrageait, disant~: Si tu es le Christ, sauve-toi toi-même, et sauve-nous~!
\VS{40}Mais l'autre le reprenait, et disait~: Ne crains-tu pas Dieu, car tu es condamné au même supplice~?
\VS{41}Pour nous, c'est juste, car nous recevons ce qu'ont mérité nos crimes~; mais celui-ci n'a fait aucun mal.
\VS{42}Et il dit à Jésus~: Seigneur, souviens-toi de moi quand tu viendras dans ton règne.
\VS{43}Jésus lui dit~: Je te le dis en vérité, aujourd'hui tu seras avec moi dans le paradis.
\TextTitle{Jésus remet son esprit à son Père\FTNTT{Mt. 27:45-56~; Mc. 15:33-41~; Jn. 19:30-37.}}
\VS{44}Or il était déjà environ la sixième heure, et il y eut des ténèbres sur toute la terre jusqu'à la neuvième heure.
\VS{45}Et le soleil s'obscurcit, et le voile du temple se déchira par le milieu.
\VS{46}Et Jésus criant à haute voix, dit~: Père, je remets mon esprit entre tes mains~! Et, en disant cela, il expira\FTNT{C'est la fin de la Première Alliance. Voir commentaire Jn. 19:30.}.
\TextTitle{FIN DE LA LOI MOSAIQUE OU DE LA PREMIERE ALLIANCE\FTNT{Hé. 9:16-18.}}
\VS{47}Le centenier, voyant ce qui était arrivé, glorifia Dieu, et dit~: Certes, cet homme était juste.
\VS{48}Et tous ceux qui assistaient en foule à ce spectacle, après avoir vu ce qui était arrivé, s'en retournèrent, se frappant la poitrine.
\VS{49}Et tous ceux qui connaissaient Jésus, et les femmes qui l'avaient suivi de Galilée, se tenaient dans l'éloignement et regardaient ces choses.
\TextTitle{Sépulture de Jésus\FTNTT{Mt. 27:57-61~; Mc. 15:42-47~; Jn. 19:38-42.}}
\VS{50}Et voici, il y avait un conseiller, nommé Joseph, homme bon et juste,
\VS{51}qui n'avait point participé au conseil et aux actes des autres~; il était d'Arimathée, ville des Juifs, et il attendait le Royaume de Dieu.
\VS{52}Etant venu à Pilate, il lui demanda le corps de Jésus.
\VS{53}Et l'ayant descendu, il l'enveloppa d'un linceul, et le mit dans un sépulcre taillé dans le roc, où personne n'avait jamais été mis.
\VS{54}Or c'était le jour de la préparation, et le sabbat allait commencer.
\VS{55}Les femmes qui étaient venues de Galilée avec Jésus, l'accompagnèrent, virent le sépulcre, et comment le corps de Jésus y avait été déposé.
\VS{56}Et s'en étant retournées, elles préparèrent des aromates et des parfums. Et le jour du sabbat elles se reposèrent selon le commandement.
\Chap{24}
\TextTitle{Résurrection du Messie\FTNTT{Mt. 28:1-15~; Mc. 16:1-11~; Jn. 20:1-18.}}
\VerseOne{}Mais le premier jour de la semaine, elles se rendirent au sépulcre de grand matin, apportant les aromates qu'elles avaient préparés.
\VS{2}Et elles trouvèrent la pierre roulée à côté du sépulcre.
\VS{3}Et, étant entrées, elles ne trouvèrent point le corps du Seigneur Jésus.
\VS{4}Comme elles ne savaient que penser de cela, voici, deux hommes leur apparurent en habits resplendissants.
\VS{5}Saisies de frayeur, elles baissèrent le visage contre terre, mais ils leur dirent~: Pourquoi cherchez-vous parmi les morts celui qui est vivant~?
\VS{6}Il n'est point ici, mais il est ressuscité. Souvenez-vous comment il vous a parlé quand il était encore en Galilée,
\VS{7}et qu'il disait~: Il faut que le Fils de l'homme soit livré entre les mains des pécheurs, et qu'il soit crucifié, et qu'il ressuscite le troisième jour.
\VS{8}Et elles se souvinrent de ses paroles.
\VS{9}A leur retour du sépulcre, elles annoncèrent toutes ces choses aux onze disciples, et à tous les autres.
\VS{10}Or c'étaient Marie de Magdala, Jeanne, Marie, mère de Jacques, et les autres qui étaient avec elles, qui dirent ces choses aux apôtres.
\VS{11}Mais les paroles de ces femmes leur semblèrent comme des paroles futiles, et ils ne les crurent point.
\VS{12}Mais Pierre s'étant levé, courut au sépulcre et s'étant courbé pour regarder, il ne vit que les linges là tout seuls, puis il s'en alla chez lui, dans l'étonnement de ce qui était arrivé.
\TextTitle{Jésus et les deux disciples sur le chemin d'Emmaüs\FTNTT{Mc. 16:12-13.}}
\VS{13}Or voici, deux d'entre eux étaient ce jour-là en chemin, pour aller à un village nommé Emmaüs, éloigné de Jérusalem de soixante stades.
\VS{14}Et ils s'entretenaient ensemble de toutes ces choses qui étaient arrivées.
\VS{15}Et il arriva que, comme ils s'entretenaient et discutaient entre eux, Jésus lui-même s'approcha et se mit à marcher avec eux.
\VS{16}Mais leurs yeux étaient retenus de sorte qu'ils ne le reconnaissaient pas.
\VS{17}Et il leur dit~: Quels sont ces discours que vous tenez ensemble en marchant~? Et pourquoi êtes-vous tout tristes~?
\VS{18}Et l'un d'eux, nommé Cléopas, lui répondit, et lui dit~: Es-tu le seul étranger dans Jérusalem qui ne sache point les choses qui s'y sont passées ces jours-ci~?
\VS{19}Et il leur dit~: Lesquelles~? Ils répondirent~: Celles concernant Jésus de Nazareth, qui était un prophète puissant en œuvres et en paroles devant Dieu, et devant tout le peuple.
\VS{20}Et comment les principaux prêtres et nos magistrats l'ont livré pour être condamné à mort, et l'ont crucifié.
\VS{21}Or nous espérions que ce serait lui qui délivrerait Israël~; mais avec tout cela, c'est aujourd'hui le troisième jour que ces choses sont arrivées.
\VS{22}Toutefois quelques femmes d'entre nous nous ont fort étonnés, car elles ont été de grand matin au sépulcre
\VS{23}et n'ayant point trouvé son corps, elles sont venues dire que même elles avaient vu une apparition d'anges, qui disaient qu'il est vivant.
\VS{24}Et quelques-uns des nôtres sont allés au sépulcre, et ont trouvé les choses comme les femmes l'avaient dit~; mais lui, ils ne l'ont point vu.
\VS{25}Alors Jésus leur dit~: Ô gens sans intelligence, et dont le cœur est lent à croire tout ce que les prophètes ont annoncé~!
\VS{26}Ne fallait-il pas que le Christ souffrît ces choses, et qu'il entra dans sa gloire~?
\VS{27}Puis commençant par Moïse, et continuant par tous les prophètes, il leur expliquait dans toutes les Ecritures ce qui le concernait.
\VS{28}Et comme ils furent près du village où ils allaient, il faisait comme s'il voulait aller plus loin.
\VS{29}Mais ils le forcèrent, en lui disant~: Reste avec nous, car le soir approche et le jour commence à baisser. Et il entra donc pour rester avec eux.
\VS{30}Et il arriva comme il était à table avec eux, il prit le pain, et il le bénit~; et l'ayant rompu, il leur distribua.
\VS{31}Alors leurs yeux s'ouvrirent, et ils le reconnurent~; mais il disparut de devant eux.
\VS{32}Et ils se dirent l'un à l'autre~: Notre cœur ne brûlait-il pas au-dedans de nous lorsqu'il nous parlait en chemin, et qu'il nous ouvrait\FTNTT{Le verbe «~ouvrir~» vient du grec «~dianoigo~» qui signifie «~ouvrir complètement ce qui a été fermé~»~; «~un enfant ouvrant la matrice, le premier-né~»~; «~ouvrir les yeux et les oreilles~»~; «~ouvrir l'esprit de quelqu'un, lui faire comprendre une chose~» ou encore «~donner la faculté de comprendre ou le désir d'apprendre~».} les Ecritures~?
\TextTitle{Nouvelles apparitions du ressuscité\FTNTT{Mc. 16:14~; Jn. 20:19-25~; Jn. 20:26~; 21:25.}}
\VS{33}Et se levant à l'heure même, ils retournèrent à Jérusalem, et ils trouvèrent assemblés les onze et ceux qui étaient avec eux,
\VS{34}qui disaient~: Le Seigneur est véritablement ressuscité, et il est apparu à Simon.
\VS{35}A leur tour, ils racontèrent ce qui leur était arrivé en chemin, et comment il avait été reconnu d'eux en rompant le pain.
\VS{36}Comme ils tenaient ces discours, Jésus se présenta lui-même au milieu d'eux, et leur dit~: Que la paix soit avec vous~!
\VS{37}Mais eux, tout terrifiés et effrayés croyaient voir un esprit.
\VS{38}Et il leur dit~: Pourquoi êtes-vous troublés, et pourquoi monte-t-il des pensées dans vos cœurs~?
\VS{39}Voyez mes mains et mes pieds, c'est bien moi. Touchez-moi, et voyez~: Car un esprit n'a ni chair ni os, comme vous voyez que j'ai.
\VS{40}Et en disant cela, il leur montra ses mains et ses pieds.
\VS{41}Mais comme de joie, ils ne croyaient point encore, et qu'ils s'étonnaient, il leur dit~: Avez-vous ici quelque chose à manger~?
\VS{42}Et ils lui présentèrent un morceau de poisson rôti, et un rayon de miel.
\VS{43}Et l'ayant pris, il mangea devant eux.
\TextTitle{La nouvelle mission des onze\FTNTT{Mt. 28:18-20~; Mc. 16:15-18~; Jn. 17:18~; 20:21~; Ac. 1:8.}}
\VS{44}Puis il leur dit~: Ce sont ici les paroles que je vous disais lorsque j'étais encore avec vous, qu'il fallait que s'accomplisse tout ce qui est écrit de moi dans la loi de Moïse, dans les prophètes, et dans les psaumes.
\VS{45}Alors il leur ouvrit l'esprit\FTNT{Pour comprendre les Ecritures, nous avons besoin de l'aide de l'Esprit de Dieu. La vraie connaissance ne vient pas des hommes, mais de Dieu (Da. 9:22).} afin qu'ils comprennent les Ecritures.
\VS{46}Et il leur dit~: Il est ainsi écrit, et ainsi il fallait que le Christ souffre, et qu'il ressuscite des morts le troisième jour,
\VS{47}et que la repentance et le pardon des péchés seraient prêchés en son Nom à toutes les nations, à commencer par Jérusalem\FTNT{Es. 53.}.
\VS{48}Et vous êtes témoins de ces choses. 
\VS{49}Et voici, j'enverrai sur vous la promesse de mon Père, mais vous donc restez dans la ville de Jérusalem, jusqu'à ce que vous soyez revêtus de la puissance d'en haut.
\TextTitle{Jésus enlevé au ciel\FTNTT{Mc. 16:19-20~; Ac. 1:9-11.}}
\VS{50}Après quoi il les conduisit dehors jusqu'en Béthanie, et levant ses mains en haut, il les bénit.
\VS{51}Et pendant qu'il les bénissait, il se sépara d'eux, et fut élevé au ciel.
\VS{52}Pour eux, après l'avoir adoré, ils retournèrent à Jérusalem avec une grande joie.
\VS{53}Et ils étaient toujours dans le temple, louant et bénissant Dieu. Amen~!
\PPE{}
\end{multicols}
