\ShortTitle{Na.}\BookTitle{Nahum}\BFont
\noindent\hrulefill
{\footnotesize
\textit{
\bigskip
{\centering{}
\\Auteur~: Nahum
\\(Heb.~: Nachuwm)
\\Signification~: Consolation, qui a compassion
\\Thème~: La ruine de Ninive
\\Date de rédaction~: 7\up{ème} siècle av. J.-C.\\}
}
\textit{
\\Nahum d'Elkosch, contemporain d'Habakuk, exerça son service dans le royaume de Juda. Il fut chargé d'annoncer la chute de Ninive, qui s'était repentie quelques décennies plus tôt, suite à la prédication de Jonas, mais elle multiplia de nouveau ses actes d'injustice et de violences au point de terroriser tous les peuples des alentours.\bigskip
}
}
\par\nobreak\noindent\hrulefill
\begin{multicols}{2}
\Chap{1}
\VerseOne{}Prophétie sur Ninive\FTNT{Ninive était la capitale de l'ancien empire assyrien. Voir Jon. 1:1-2.}, qui est le livre de la vision de Nahum d'Elkosch.
\TextTitle{Jugement annoncé sur Ninive}
\VS{2}Yahweh est un Dieu jaloux, il se venge, Yahweh se venge, il est plein de fureur~; Yahweh se venge de ses adversaires, et il garde sa colère contre ses ennemis.
\VS{3}Yahweh est lent à la colère, et grand par sa force, mais il ne tient nullement le coupable pour innocent. Yahweh marche parmi les tourbillons et les tempêtes, les nuées sont la poussière de ses pieds.
\VS{4}Il réprimande la mer et la fait tarir, il dessèche tous les fleuves~; le Basan et le Carmel languissent, la fleur du Liban languit.
\VS{5}Les montagnes tremblent à cause de lui, et les collines se fondent~; la terre se soulève devant sa présence, dis-je, et tous ceux qui y habitent.
\VS{6}Qui subsistera devant son indignation~? Et qui demeurera ferme dans l'ardeur de sa colère~? Sa fureur se répand comme un feu, et les rochers se brisent devant lui.
\VS{7}Yahweh est bon, il est une forteresse au jour de la détresse, et il connaît ceux qui se confient en lui.
\VS{8}Il s'en va passer comme un débordement d'eaux~; il réduira son lieu à néant, et fera que les ténèbres poursuivront ses ennemis.
\TextTitle{Jugement contre les ennemis de Yahweh}
\VS{9}Que projetez-vous contre Yahweh~? C'est lui qui réduit à néant~; la détresse ne se lèvera pas deux fois~;
\VS{10}car entrelacés comme des épines, et ivres de leur vin, ils seront consumés entièrement comme la paille sèche.
\VS{11}De toi est sorti celui qui méditait du mal contre Yahweh, et qui avait de mauvais desseins.
\VS{12}Ainsi parle Yahweh~: Bien qu'ils soient en paix et en grand nombre, ils seront certainement retranchés, et on passera outre. Or je t'ai affligé, mais je ne t'affligerai plus.
\VS{13}Maintenant je briserai son joug de dessus toi, et je détacherai tes liens.
\VS{14}Voici ce qu'a ordonné Yahweh contre toi~: Tu n'auras plus de semence qui porte ton nom~; je retrancherai de la maison de tes dieux les images taillées et en fonte~; j'en ferai ta tombe parce que tu es insignifiant.
\Chap{2}
\TextTitle{Délivrance et réjouissance}
\VerseOne{}Voici sur les montagnes les pieds de celui qui apporte de bonnes nouvelles\FTNT{Es. 40:9~; 52:7~; Ro. 10:15.}, qui publie la paix~! Toi Juda, célèbre tes fêtes solennelles, accomplis tes vœux~; car à l'avenir les hommes violents ne passeront plus au milieu de toi, ils sont entièrement retranchés.
\TextTitle{Récit de la destruction de Ninive}
\VS{2}Le destructeur est monté contre toi~; garde la forteresse~! Veille sur la route~! Affermis tes reins~! Consolide toute ta force~!
\VS{3}Car Yahweh se détourne de la majesté de Jacob et d'Israël~; parce que les dévastateurs les ont vidés, et qu'ils ont détruit leurs sarments.
\VS{4}Le bouclier de ses hommes forts est rouge~; ses hommes puissants sont teints de pourpre~; le fer des chars étincelle au jour qu'il a fixé pour la bataille, et les lances sont agitées.
\VS{5}Les chars s'élancent avec rapidité dans les rues, ils se précipitent sur les places, ils sont comme des flambeaux, et courent comme des éclairs.
\VS{6}Il se souvient de ses hommes vaillants, mais ils chancellent dans leur marche. Ils se hâtent vers les murs, et ils se préparent à la défense.
\VS{7}Les portes des fleuves sont ouvertes, et le palais se fond.
\VS{8}C'est fixé~: Elle est découverte et emportée~; ses servantes gémissent de leur voix comme des colombes, frappant leurs poitrines comme un tambour.
\VS{9}Or Ninive, depuis qu'elle a été bâtie, a été comme un vivier d'eaux~; mais ils s'enfuient… Arrêtez-vous~! Arrêtez-vous~! Mais il n'y a personne qui tourne le visage…
\VS{10}Pillez l'argent~! Pillez l'or~! Il y a des dispositions sans fin, des richesses en objets précieux.
\VS{11}On pille, on dévaste, on ravage~! Et les cœurs se fondent, leurs genoux se heurtent l'un contre l'autre. Que le tourment soit dans les reins de tous~! Et que leurs visages deviennent noirs comme un pot qui a été mis sur le feu~!
\VS{12}Où est le repaire des lions, le pâturage des lionceaux, dans lequel se retiraient les lions, et où se tenaient le lion et la lionne, et le petit du lion, sans qu'aucun ne les effraie~?
\VS{13}Les lions ravissaient tout ce qu'il fallait pour leurs petits, et étranglaient les bêtes pour leurs lionnes, ils remplissaient leurs tanières de proies, et leurs repaires de dépouilles.
\VS{14}Voici, j'en veux à toi, dit Yahweh des armées, je brûlerai tes chars, et ils s'en iront en fumée, l'épée consumera tes lionceaux. Je retrancherai de la terre ta proie, et la voix de tes messagers ne sera plus entendue.
\Chap{3}
\TextTitle{Méchanceté de Ninive}
\VerseOne{}Malheur à la ville sanguinaire qui est pleine de mensonge, pleine de violence~; la rapine ne s'en retirera point,
\VS{2}ni le bruit du fouet, ni le bruit impétueux des roues, ni le galop des chevaux, ni le saut des chars~;
\VS{3}ni les cavaliers montant leurs chevaux, ni l'épée flamboyante, ni la lance étincelante, ni la multitude des blessés, ni le grand nombre de cadavres. Des corps morts à l'infini, on trébuche sur un grand nombre de corps morts~!
\VS{4}à cause de la multitude des prostitutions de cette prostituée, pleine de charmes, experte en sortilèges, qui vendait les nations par ses prostitutions, et les familles par ses enchantements.
\VS{5}Voici, j'en veux à toi, dit Yahweh des armées, je relèverai ta robe jusqu'à ton visage~; je manifesterai ta nudité aux nations, et ton ignominie aux royaumes.
\VS{6}Je ferai tomber sur ta tête la peine de tes abominations, je te consumerai et je te donnerai en spectacle.
\VS{7}Et il arrivera que quiconque te verra, s'éloignera de toi et dira~: Ninive est détruite~! Qui aura compassion d'elle~? D'où te chercherai-je des consolateurs~?
\VS{8}Vaux-tu mieux que No-Amon, qui est assise au milieu des fleuves, qui a des eaux autour d'elle, dont la mer est le rempart, et à qui la mer sert de murailles~?
\VS{9}L'Ethiopie et l'Egypte étaient sa force, et une infinité d'autres peuples~; Puth et les Lybiens sont allés à son secours.
\VS{10}Elle-même aussi est transportée hors de sa terre, elle s'en est allée en captivité~; ses enfants ont été écrasés aux carrefours de toutes les rues, et on a jeté le sort sur ses gens honorables, et tous ses grands ont été liés de chaînes.
\VS{11}Toi aussi, tu seras enivrée, tu te tiendras cachée, et tu chercheras un refuge contre l'ennemi.
\VS{12}Toutes tes forteresses seront comme des figues, et comme des premiers fruits qui étant secoués, tombent dans la bouche de celui qui veut les manger.
\VS{13}Voici, ton peuple sera comme autant de femmes au milieu de toi~; les portes de ton pays seront toutes ouvertes, elles seront ouvertes à tes ennemis~; le feu consumera tes verrous.
\VS{14}Puise-toi de l'eau pour le siège~! Fortifie tes remparts~! Enfonce le pied dans la boue, foule l'argile~! Et fortifie le four à brique~!
\VS{15}Là, le feu te consumera, l'épée te retranchera, elle te dévorera comme la sauterelle dévore les arbres. Multiplie-toi comme les sauterelles~! Multiplie-toi comme les sauterelles~!
\VS{16}Tu as multiplié le nombre de tes marchands plus que les étoiles des cieux~; les sauterelles s'étant répandues, ont tout ravagé, et puis se sont envolées.
\VS{17}Tes princes et leurs scribes sont comme des sauterelles qui campent dans les murs au temps de la froidure~: Le soleil paraît, elles s'envolent et on ne reconnaît plus le lieu où elles étaient.
\VS{18}Tes bergers se sont endormis, ô roi d'Assyrie~! Tes grands hommes se tiennent dans leurs tentes~; ton peuple est dispersé par les montagnes, et il n'y a personne qui le rassemble.
\VS{19}Il n'y a point de remède à ta blessure, ta plaie est douloureuse~; tous ceux qui entendront parler de toi battront des mains sur toi~; car qui n'a pas continuellement éprouvé les effets de ta méchanceté~?
\PPE{}
\end{multicols}
