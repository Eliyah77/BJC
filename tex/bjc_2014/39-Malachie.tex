\ShortTitle{Malachie}\BookTitle{Malachie}\BFont
\noindent\hrulefill
\textit{
\bigskip
{\centering{}
\\(Malakhi)
\\Signifie : Mon messager, mon ange
\\Thème : Message final de l'ancienne alliance à une nation désobéissante
\\Auteur : Malachie
\\Date de rédaction : Vème siècle av. J.-C.\\}
}
%\bigskip
\textit{
\\Dernier prophète de l’ancienne alliance, Malachie exerça son ministère en Juda après la reconstruction du temple et la reprise des cultes. Il annonça la venue du Messie et du messager qui devait le précéder, le nouvel Elie que Jésus-Christ reconnut en Jean-Baptiste. Ses écrits mettent en évidence l’importance de l’obéissance à la loi de Yahweh et la justice divine.
\bigskip
\\message. 
\bigskip
\\message.\bigskip
}
\par\nobreak\noindent\hrulefill
\begin{multicols}{2}
\TextTitle{[L'amour de Yahweh pour son peuple]}
\Chap{1}
\VerseOne{}Oracle, parole de Yahweh contre Israël, par le moyen de Malachie.
\VS{2}Je vous ai aimés, dit Yahweh ; et vous dites : En quoi nous as-tu aimés ? Esaü n'était-il pas frère de Jacob ? dit Yahweh. Or j'ai aimé Jacob,
\VS{3}Mais j'ai eu de la haine pour Esaü, et j'ai fait de ses montagnes une solitude, j’ai livré son héritage aux chacals du désert.
\VS{4}Si Edom dit : Nous sommes détruits, nous rebâtirons les lieux ruinés ! Ainsi parle Yahweh des armées : Ils rebâtiront, mais je détruirai, et on les appellera pays de méchanceté, peuple contre lequel Yahweh est irrité pour toujours.
\VS{5}Vos yeux le verront, et vous direz : Yahweh est grand par-delà les frontières d'Israël !
\TextTitle{[Le péché des sacrificateurs après le retour d'exil]}
\VS{6}Un fils honore son père, et un serviteur son maître. Si donc je suis Père, où est l'honneur qui m'appartient ? Si je suis maître, où est la crainte qu'on a de moi ? dit Yahweh des armées, à vous sacrificateurs, qui méprisez mon Nom, et qui dites : En quoi avons-nous méprisé ton Nom ?
\VS{7}Vous offrez sur mon autel des aliments souillés, et vous dites : En quoi t'avons-nous profané ? C'est en disant : La table de Yahweh est méprisable !
\VS{8}Et quand vous amenez une bête aveugle pour la sacrifier, n'y a-t-il point de mal en cela ? Quand vous en offrez une boiteuse ou malade, n’est-ce pas mal ? Offre-la à ton gouverneur ! T’agréera-t-il, te recevra-t-il favorablement ? dit Yahweh des armées.
\VS{9}Maintenant, donc suppliez Dieu, pour qu'il ait pitié de nous ! Cela vient de vos mains : Vous recevra-t-il favorablement ? dit Yahweh des armées.
\VS{10}Lequel de vous fermera les portes pour que vous n’allumiez pas en vain le feu sur mon autel ? Je ne prends aucun plaisir en vous, dit Yahweh des armées, et je n’agrée pas l'offrande de vos mains.
\VS{11}Car depuis le soleil levant jusqu'au soleil couchant, mon Nom est grand parmi les nations, et en tous lieux on brûle de l’encens en l'honneur de mon Nom, et des offrandes pures ; car mon Nom est grand parmi les nations, dit Yahweh des armées.
\VS{12}Mais vous, vous le profanez, en disant : La table de Yahweh est souillée, et ce qu’elle rapporte est un aliment méprisable.
\VS{13}Vous dites aussi : Quelle fatigue ! Et vous le dédaignez, dit Yahweh des armées ; vous amenez ce qui a été dérobé, ce qui est boiteux, et malade, ce sont là les offrandes que vous faites ! Accepterai-je cela de vos mains ? dit Yahweh.
\VS{14}C'est pourquoi, maudit soit l'homme trompeur, qui a dans son troupeau un mâle, et qui voue et sacrifie à Yahweh ce qui est corrompu ! Car je suis un grand roi, dit Yahweh des armées, et mon Nom est redoutable parmi les nations.
\TextTitle{[Mise en garde de Yahweh aux sacrificateurs]}
\Chap{2}
\VerseOne{}Maintenant, c’est à vous, sacrificateurs que s'adresse ce commandement :
\VS{2}Si vous n'écoutez pas, et que vous ne preniez point pas à cœur de donner gloire à mon Nom, dit Yahweh des armées, j'enverrai sur vous la malédiction, et je maudirai vos bénédictions ; et déjà même je les ai maudites, parce que vous ne prenez pas cela à cœur.
\VS{3}Voici, je vais détruire vos semences, et je répandrai les excréments de vos victimes sur vos visages, les excréments, dis-je, de vos solennités, et on vous emportera avec eux.
\VS{4}Alors vous saurez que je vous ai adressé ce commandement, afin que mon alliance avec Lévi subsiste, dit Yahweh des armées.
\VS{5}Mon alliance avec lui était la vie et la paix, c’est ce que je lui accordai pour qu’il me craigne ; il a eu pour moi de la crainte, et il a tremblé devant mon Nom.
\VS{6}La loi de la vérité était dans sa bouche, et il ne s'est point trouvé de perversité sur ses lèvres ; il a marché avec moi dans la paix et dans la droiture, et il en a détourné beaucoup de l'iniquité.
\VS{7}Car les lèvres du sacrificateur doivent garder la science, et c’est de sa bouche qu’on demande la loi, parce qu'il est un messager de Yahweh des armées.
\VS{8}Mais vous, vous vous êtes écartés de la voie, vous avez fait de la loi une occasion de chute pour beaucoup, et vous avez corrompu l'alliance de Lévi, dit Yahweh des armées.
\TextTitle{[Infidélités envers des frères et envers Yahweh]}
\VS{9}C'est pourquoi je vous rendrai méprisables et abjects aux yeux de tout le peuple. Parce que vous n’avez pas gardé mes voies, et vous avez égard à l'apparence des personnes quand vous enseignez la loi.
\VS{10}N'avons-nous pas tous un seul Père ? N’est-ce pas un seul Dieu qui nous a créés ? Pourquoi donc agissons-nous avec perfidie l’un avec l’autre, en violant l'alliance de nos pères ?
\VS{11}Juda s’est montré infidèle, et une abomination a été commise en Israël et à Jérusalem ; car Juda a profané ce qui est consacré à Yahweh, ce qu’il aime, il s'est marié à la fille d'un dieu étranger.
\VS{12}Yahweh retranchera l’homme qui fait cela, celui qui veille et qui répond, il le retranchera des tentes de Jacob, et il retranchera celui qui présente une offrande à Yahweh des armées.
\VS{13}Voici une autre chose que vous faites : Vous couvrez l'autel de Yahweh de larmes, de plaintes et de gémissements, en sorte qu’il n’a plus égard aux offrandes et qu’il ne peut rien agréer de vos mains.
\VS{14}Et vous dites : Pourquoi ?... C'est parce que Yahweh est intervenu comme témoin entre toi et la femme de ta jeunesse, envers laquelle tu as été infidèle, bien qu’elle soit ta compagne et la femme de ton alliance.
\VS{15}Nul n’a fait cela, avec un reste de bon esprit. Un seul l’a fait, et pourquoi ? Parce qu'il cherchait une postérité de Dieu. Prenez donc garde en votre esprit, et qu’aucun ne soit infidèle à la femme de sa jeunesse !
\VS{16}Car je hais la répudiation, dit Yahweh, le Dieu d'Israël, et celui qui couvre de violence son vêtement, dit Yahweh des armées. Prenez donc garde en votre esprit, et ne soyez pas infidèles !
\TextTitle{[Fausse profession religieuse]}
\VS{17}Vous fatiguez Yahweh par vos paroles, et vous dites : En quoi l'avons-nous fatigué ? C'est quand vous dites : Quiconque fait le mal plaît à Yahweh, et il prend plaisir à de tels gens ! Autrement : Où est le Dieu du jugement ?
\TextTitle{[Venue du précurseur du messie]}
\Chap{3}
\VerseOne{}Voici, j'enverrai mon messager\FTNT{Ce messager, ou Elie le prophète, est Jean-Baptiste (Es. 40:1-3 ; Mal. 3:1 ; Mt. 3:1-15 ; Mt. 11:14 ; Mt. 17:10-13 ; Mc. 1:1-11 ; Mc. 9:11-13 ; Lu. 1:17 ; Lu. 3:1-5).} ; il préparera le chemin devant moi. Et soudain entrera dans son temple le Seigneur que vous cherchez ; l’ange de l'alliance\FTNT{L’ange de l’alliance est le Seigneur Jésus-Christ. Voir aussi commentaire en Mt. 1:20}, que vous désirez, voici, il vient, dit Yahweh des armées.
\VS{2}Mais qui pourra soutenir le jour de sa venue ? Qui pourra subsister quand il paraîtra ? Car il sera comme le feu du fondeur et comme la potasse des foulons.
\VS{3}Et il sera assis comme celui qui raffine et purifie l'argent ; il nettoiera les fils de Lévi, il les épurera comme l’or et l'argent, et ils présenteront à Yahweh des offrandes avec justice.
\VS{4}Alors l’offrande de Juda et de Jérusalem sera agréable à Yahweh, comme aux anciens jours, comme aux années d'autrefois.
\VS{5}Je m'approcherai de vous pour le jugement, et je me hâterai de témoigner contre les enchanteurs et les adultères, contre ceux qui jurent faussement, et contre ceux qui retiennent le salaire du mercenaire, qui oppriment la veuve et l'orphelin, qui font tort à l'étranger, et qui ne me craignent point, dit Yahweh des armées.
\VS{6}Parce que je suis Yahweh et que je n'ai point changé ; à cause de cela, enfants de Jacob, vous n'avez point été consumés.
\TextTitle{[Le peuple infidèle qui vole Yahweh]}
\VS{7}Depuis le temps de vos pères, vous vous êtes écartés de mes ordonnances, vous ne les avez point observées. Revenez à moi, et je reviendrai à vous, dit Yahweh des armées. Et vous dites : En quoi nous convertirons-nous ?
\VS{8}L'homme pillera-t-il Dieu, que vous me pilliez ? Et vous dites : En quoi t'avons-nous pillé ? Vous l'avez fait dans les dîmes et dans les offrandes.
\VS{9}Vous êtes certainement maudits, parce que vous me pillez, vous, toute la nation !
\VS{10}Apportez toutes les dîmes\FTNT{Voici le verset favori de tous ceux qui enseignent le paiement de la dîme ! C’est sur ce passage que repose l’essentiel de leur doctrine. Toutefois, en l’étudiant de plus près, nous découvrirons quelque chose de très intéressant. 
Rappelez-vous qu’il existait quatre dîmes en Israël (voir commentaire sur les différentes sortes de dîmes en No. 18:21.) Il convient donc de se demander à quelle dîme Malachie faisait-il allusion dans ce passage. Né. 10:38 nous donne la réponse : «~Le sacrificateur, fils d'Aaron, sera avec les lévites, lorsque les lévites lèveront la dîme ; et les lévites apporteront la dîme de la dîme à la maison de notre Dieu, dans les chambres de la maison du trésor~». Il s’agit de la dîme que devaient payer les Lévites et non les dîmes dues par le peuple ! Ainsi, les malédictions annoncées par Malachie, et dont nous menacent certains «~hommes de Dieu~» qui insistent pour maintenir la perception de la dîme de nos jours, ne concernent pas les fidèles. Malachie ne fustige pas le peuple en général, mais il reprend sévèrement les Lévites qui ne payaient pas la dîme de la dîme. Par conséquent, ceux qui utilisent ce verset pour vous faire payer la dîme ne se rendent pas compte que si ce passage devait concerner quelqu'un, ce serait eux-mêmes.
La lecture de Mal. 4:4 permet de mettre en évidence que l’objet de ce livre concerne la stricte application de la loi de Moïse : «~Souvenez-vous de la loi de Moïse, mon serviteur, auquel j’ai prescrit en Horeb, pour tout Israël, des statuts et des ordonnances.~»
Or, nous chrétiens, ne sommes plus sous la première alliance (la loi de Moïse), mais sous la nouvelle alliance. Si vous décidez de vous soumettre ne serait-ce qu’à une seule des dispositions de la loi de Moïse, vous allez avoir un sérieux problème ! D’abord parce qu’elle est impossible à respecter en entier, ensuite parce que c’est contraire à l’ordre de Jésus. «~Car tous ceux qui s’attachent aux œuvres de la loi, sont sous la malédiction ; car il est écrit : maudit est quiconque ne persévère pas dans toutes les choses qui sont écrites dans le livre de la loi et ne les met pas en pratique.~» Ga. 3:10.
Si nous observons la loi pour être sauvés, nous devons l’observer en entier, sinon nous sommes sous la malédiction ! Le but de la loi était d’agir comme un tuteur. Elle était «~notre pédagogue pour nous amener à Christ, afin que nous soyons justifiés par la foi.~» (Ga. 3:24).
JESUS A-T-IL ENSEIGNE LA DIME ?
«~Malheur à vous, scribes et pharisiens hypocrites ! Parce que vous payez la dîme de la menthe, de l'aneth et du cumin ; et vous laissez les choses les plus importantes de la loi, c'est-à-dire la justice, la miséricorde et la fidélité. Il fallait pratiquer ces choses-là, sans négliger les autres choses.~» Mt. 23:23.
Quand Jésus disait aux pharisiens «~C’est là ce qu’il fallait pratiquer, sans négliger les autres choses.~» (Lu. 11:42), cela signifie-t-il que les chrétiens doivent payer la dîme ? 
Observez bien les paroles de Jésus dans leur contexte. 
A qui Jésus s’adressait-il ? — Aux pharisiens. Quelle était la particularité des pharisiens ? — Ils se considéraient eux-mêmes comme consacrés à la loi, définition littérale du mot «~pharisien~». Paul était l'un d'eux «~Quant au zèle, persécutant l'Eglise ; et quant à la justice à l’égard de la loi, étant sans reproche~» (Ph. 3:6). Jésus s’adressait donc à des hommes qui se vantaient d’observer parfaitement la loi. Aussi, il leur avait dit de continuer à la respecter, mais sans négliger la justice et l’amour de Dieu. Toutefois, dans tout le chapitre 23 de Matthieu, Jésus exprima sa colère contre les hypocrites qu’il dénonçait, en disant notamment au verset 19 : «~Insensés et aveugles ! Car lequel est le plus grand, l’offrande, ou l'autel qui sanctifie l’offrande ?~» Les reproches de Jésus aux pharisiens se concluent par l’annonce de sa crucifixion, de la destruction du temple et de l’avènement d’une nouvelle alliance. Avant sa mort et sa résurrection, Jésus demandait qu’on observe la loi (Mt. 8:1-4 ; Mt. 23:1-3), or la dîme faisait partie des ordonnances à respecter. Les données ont changé lorsque le rideau du temple s'est déchiré pour inaugurer la dispensation de la grâce, et quand le Seigneur a dit : «~Tout est accompli~» (Jn. 19:30). En effet, Jésus n'est pas venu supprimer la loi, mais l'accomplir (Mt. 5:17). Au travers de cet acte symbolique, Dieu montrait que désormais tout homme pouvait s'approcher de son trône avec assurance (Hé. 4:16), Jésus nous ayant justifiés par son sacrifice (Ro. 3:24). Jésus raconta également, l’histoire de deux hommes ; l’un payait la dîme, et l’autre ne la payait pas. Voyons lequel fut justifié. «~Deux hommes montèrent au Temple pour prier, l'un était pharisien, et l'autre, publicain. Le pharisien se tenant à l'écart priait en lui-même en ces termes : O Dieu ! Je te rends grâces de ce que je ne suis pas comme le reste des hommes, qui sont ravisseurs, injustes, adultères, ou même comme ce publicain. Je jeûne deux fois la semaine, et je donne la dîme de tout ce que je possède. Mais le publicain se tenant loin, n'osait même pas lever les yeux vers le ciel, mais il se frappait la poitrine, en disant : O Dieu ! Sois apaisé envers moi qui suis un pécheur ! Je vous dis que celui-ci descendit dans sa maison justifié, plutôt que l'autre ; car quiconque s'élève, sera abaissé, et quiconque s'abaisse, sera élevé.~» Lu. 18:10-14.
Qui fut donc justifié devant le Seigneur, celui qui payait la dîme, ou celui qui ne la payait pas ? «~Vous rejetez bien le commandement de Dieu, afin de garder 
votre tradition. Car Moïse a dit : Honore ton père et ta mère ; et : celui qui maudira son père ou sa mère, sera puni de mort. Mais vous, vous dites : Si quelqu'un dit à son père ou à sa mère : Tout ce dont je pourrais t’assister est corban, c’est-à-dire, une offrande à Dieu, il ne sera point coupable. 
Et vous ne lui permettez plus de rien faire pour son père ou pour sa mère, anéantissant ainsi la parole de Dieu par votre tradition que vous avez établie. Et vous faites encore beaucoup d’autres choses semblables.~» Mc. 7:9-13. Si votre argent peut servir à pourvoir aux besoins de votre famille et que vous l’utilisez pour le donner à votre église en tant que «~dîme~», vous faites exactement ce que faisaient ces pharisiens repris par Jésus ! Vous dites que votre argent est «~corban~» ! En faisant cela, vous annulez donc la parole de Dieu par votre tradition, comme Jésus le reprochait aux pharisiens.
Que dit le testament de Jésus en ce qui concerne l’argent et les dons ?
«~Rendez donc à tous ce qui leur est dû : l’impôt à qui vous devez l’impôt, le tribut à qui vous devez le tribut, péage, la crainte, à qui vous devez la crainte, l’honneur à qui vous devez l'honneur.~» Ro.13:7. Nous devons continuer à payer nos impôts à l’Etat. De ce point de vue, rien n’est changé par rapport à la première alliance. Nous devons toujours contribuer à financer la Sécurité sociale et la Fonction publique. Mais, en ce qui concerne nos dons, nous devons reconnaître que nous appartenons entièrement au Seigneur, avec tout ce que nous possédons. Quand nous nous présentons devant lui, nous ne devons jamais oublier cette vérité.
COMMENT ET POURQUOI UN CHRETIEN PEUT-IL FAIRE DES DONS ET DES OFFRANDES ?
— Donner pour répondre à un besoin
Nous devons donner en réponse à un besoin et non pour répondre à la cupidité de ceux qui nous font des appels de fonds. Aujourd’hui, beaucoup de chrétiens font l’objet d’incessantes sollicitations. Les professionnels de la religion qui les dirigent leur demandent sans cesse des dons toujours plus importants afin de se payer des propriétés luxueuses, acheter les derniers modèles de voitures, voyager dans le monde entier et se bâtir d’immenses empires financiers contrôlés par leur famille ; tout cela, bien entendu, «~pour la gloire de Dieu~» ! Tous ces bâtisseurs d’empires ont bien soin de demander aux chrétiens de verser la dîme à leur ministère, en les menaçant des pires châtiments de Dieu s’ils ne s’exécutent pas ! La Bible ne nous demande absolument pas d’encourager la cupidité de tels hommes. Elle nous demande plutôt de pourvoir aux besoins véritables. «~Car il n'y avait parmi eux aucun indigent ; parce que tous ceux qui possédaient des champs ou des maisons, les vendaient, et ils apportaient le prix des choses vendues ; Et le mettaient aux pieds des apôtres ; et il était distribué à chacun selon ses besoins.~» Ac. 4:34-35.
«~En ce temps-là, quelques prophètes descendirent de Jérusalem à Antioche. L’un d'eux, nommé Agabus, se leva, et déclara par l'Esprit, qu'une grande famine devait arriver sur toute la terre. Elle arriva, en effet, sous Claude César. 
Les disciples résolurent d’envoyer chacun selon ses moyens, quelques secours pour subvenir aux besoins des frères qui habitaient la Judée. Ils le firent parvenir aux anciens par les mains de Barnabas et de Saul.~» Ac. 11:27-30. Remarquez que les véritables prophètes prédisent la famine ! Aujourd’hui, les faux prophètes modernes prédisent toujours la «~prospérité~». Ils poussent ainsi les fidèles dans des enthousiasmes délirants, avant de les soulager d’énormes 
offrandes, sous prétexte de «~soutenir leur ministère~». A l’époque des Actes des apôtres, les chrétiens subvenaient à des besoins véritables.
— Donner secrètement et humblement
«~Gardez-vous de pratiquer votre justice devant les hommes, pour en être vus ; autrement, vous ne recevrez point la récompense de votre Père qui est dans les cieux. Donc, lorsque tu fais ton aumône, ne fais point sonner la trompette 
devant toi, comme font les hypocrites dans les synagogues, et dans les rues, afin d’être glorifiés par les hommes. Je vous le dis en vérité, ils reçoivent leur récompense. Mais quand tu fais ton aumône, que ta main gauche ne sache pas ce que fait ta droite. Afin que ton aumône se fasse en secret, et ton Père, qui voit ce qui se fait dans le secret, te récompensera publiquement.~» Mt. 6:1-4.
Le Seigneur nous demande donc de donner secrètement et humblement.
— Donner selon ses moyens
«~La bonne volonté, quand elle existe, est agréable en raison de ce qu’elle peut avoir à sa disposition, et non de ce qu’elle n’a pas.~» 2 Co. 8:12.
Si vous disposez de cent euros et que vous devez cent euros à quelqu’un, et si, au lieu de payer votre dette, vous donnez votre argent à une organisation religieuse, Dieu n’acceptera pas votre offrande. Ne donnez que ce dont vous disposez réellement. Ne vous laissez pas avoir par tous ceux qui vous manipulent par leurs boniments, comme c’est le cas dans beaucoup d’églises pentecôtistes ou charismatiques, quand on vous demande de «~donner par la foi~», en croyant que Dieu va multiplier par cent votre don ! La Bible dit clairement que le Seigneur considère de tels dons comme inacceptables !
— Donner avec joie
«~Mais que chacun contribue comme il a résolu en son cœur, sans tristesse ni contrainte ; car Dieu aime celui qui donne avec joie. Et Dieu est Tout-Puissant pour vous combler de toutes sortes de grâces, afin que possédant toujours en toutes choses de quoi satisfaire à tous vos besoins, vous ayez encore en abondance pour toute bonne œuvre, Selon ce qui est écrit : il a fait des largesses, il a donné aux pauvres ; sa justice demeure éternellement. Que celui qui fournit de la semence au semeur, veuille aussi vous donner du pain à manger, et multiplier votre semence, et augmenter les revenus de votre justice. Afin que vous soyez pleinement enrichis pour exercer une parfaite libéralité, et qu’ainsi nous ayons sujet de rendre des actions de grâces à Dieu. Car le secours de cette assistance non seulement pourvoit aux besoins des saints, mais il est encore une source abondante de nombreuses actions de grâces envers Dieu.~» 2 Co. 9:7-12.
Voici ce que la Bible nous demande de faire ici : Ne donnez que si vous êtes réellement heureux de donner ! Elle ne nous demande pas de donner plus que nous ne pouvons nous le permettre, en nous forçant à être joyeux de le faire ! Le verset 7 résume le mieux ce que le testament de Jésus nous demande de faire quand nous donnons : «~Mais que chacun contribue comme il a résolu en son cœur, sans tristesse ni contrainte ; car Dieu aime celui qui donne avec joie.~». Dieu désire que vous donniez uniquement ce que vous voulez donner, et que vous le donniez sans contraintes et avec joie. Si vous ne pouvez pas donner avec joie, ne donnez rien ! 
Dieu ne l’exige pas, et il n’acceptera pas un tel don ! Dieu agrée une offrande faite de bon cœur (Ex. 25:2).}, et qu'il y ait provision dans ma maison ; et dès maintenant éprouvez moi en cela, a dit Yahweh des armées, si je ne vous ouvre pas les écluses des cieux, et si je ne répands pas en votre faveur la bénédiction, en sorte que vous n'y pourrez point suffire.
\VS{11}Et je réprimerai pour l'amour de vous le dévorateur, et il ne vous ravagera pas les fruits de la terre, et vos vignes ne seront pas stériles dans vos campagnes, a dit Yahweh des armées.
\VS{12}Toutes les nations vous diront heureux, car vous serez un pays de délices, dit Yahweh des armées.
\VS{13}Vos paroles sont rudes contre moi, a dit Yahweh. Et vous dites : Qu'avons-nous donc dit contre toi ?
\VS{14}Vous avez dit : C'est en vain que l’on sert Dieu ; et qu'avons-nous gagné à observer ses ordonnances, et à marcher en pauvre état pour l'amour de Yahweh des armées ?
\VS{15}Et maintenant nous tenons pour heureux les orgueilleux ; et même ceux qui commettent la mechanceté, sont avancés ; et s'ils ont tenté Dieu, ils ont été délivrés !
\TextTitle{[Le "reste d'Israël" demeure fidèle à Yahweh"]}
\VS{16}Alors ceux qui craignent Yahweh se parlèrent l'un à l’autre ; et Yahweh fut attentif, et il écouta ; et un livre de souvenir fut écrit devant lui pour ceux qui craignent Yahweh et qui pensent à son Nom.
\VS{17}Ils seront à moi, a dit Yahweh des armées, le jour où je mettrai à part mes plus précieux joyaux, et je leur pardonnerai comme un homme pardonne à son fils qui le sert.
\VS{18}Convertissez-vous donc, et vous verrez la différence qu'il y a entre le juste et le méchant, entre celui qui sert Dieu et celui qui ne le sert pas.
\TextTitle{[Avènement du jour de Yahweh]}
\Chap{4}
\VerseOne{}Car voici, le jour vient, ardent comme une fournaise. Tous les orgueilleux et tous les méchants seront comme du chaume ; et ce jour qui vient, dit Yahweh des armées, les embrasera, il ne leur laissera ni racine ni rameau.
\VS{2}Mais pour vous qui craignez mon Nom, se lèvera le Soleil de justice\FTNT{Le Soleil de justice : Jésus-Christ est notre Soleil (Lu. 1:78-79). Cet aspect de Jésus-Christ nous parle de la grâce de Dieu : « il fait lever son soleil sur les méchants, et sur les gens de bien » (Mt. 5:45). Le soleil évoque aussi le jugement de Dieu. Ainsi, en plein midi, il est le feu de la justice et de la colère de Dieu. Son pardon et son amour pour nous sont alors comparés à une ombre fraîche qui nous sauve de sa chaleur ardente (Ps. 121 ; Es. 25:4). Dans Ps. 19:6, le soleil est comparé à un époux. Or Jésus-Christ est notre époux et le soleil qui nous apporte la guérison.}, et la guérison sera sous ses ailes ; vous sortirez, et bondirez comme les veaux d’une étable.
\VS{3}Et vous foulerez les méchants, car ils seront comme de la cendre sous les plantes de vos pieds, au jour où je ferai mon œuvre, dit Yahweh des armées.
\VS{4}Souvenez-vous de la loi de Moïse, mon serviteur, auquel j’ai prescrit en Horeb, pour tout Israël, des statuts et des ordonnances.
\TextTitle{[Retour d'Elie avant le jour de Yahweh]}
\VS{5}Voici, je vous enverrai Elie, le prophète\FTNT{Voir commentaire en Mal. 3:1.}, avant que le jour grand et redoutable de Yahweh vienne.
\VS{6}Il ramènera le cœur des pères à leurs enfants, et le cœur des enfants à leurs pères, de peur que je ne vienne et que je ne frappe la terre d’interdit.
\PPE{}
\end{multicols}
