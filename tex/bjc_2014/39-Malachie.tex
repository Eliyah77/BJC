\ShortTitle{Malachie}\BookTitle{Malachie}\BFont
\noindent\hrulefill
{\footnotesize
\textit{
\bigskip
{\centering{}
\\(Malakhi)
\\Signifie : Mon messager, mon ange
\\Thème : Message final de l'ancienne alliance à une nation désobéissante
\\Auteur : Malachie
\\Date de rédaction : 5ème siècle av. J.-C.\\}
}
%\bigskip
\textit{
\\Dernier prophète de l’ancienne alliance, Malachie exerça son ministère en Juda après la reconstruction du temple et la reprise des cultes. Il annonça la venue du Messie et du messager qui devait le précéder, le nouvel Elie que Jésus-Christ reconnut en Jean-Baptiste. Ses écrits mettent en évidence l’importance de l’obéissance à la loi de Yahweh et la justice divine.\bigskip
}
}
\par\nobreak\noindent\hrulefill
\begin{multicols}{2}
\TextTitle{[L'amour de Yahweh pour son peuple]}
\Chap{1}
\VerseOne{}Oracle, parole de Yahweh contre Israël, par le moyen de Malachie.
\VS{2}Je vous ai aimés, dit Yahweh ; et vous dites : En quoi nous as-tu aimés ? Esaü n'était-il pas frère de Jacob ? dit Yahweh. Or j'ai aimé Jacob,
\VS{3}Mais j'ai eu de la haine pour Esaü, et j'ai fait de ses montagnes une solitude, j’ai livré son héritage aux chacals du désert.
\VS{4}Si Edom dit : Nous sommes détruits, nous rebâtirons les lieux ruinés ! Ainsi parle Yahweh des armées : Ils rebâtiront, mais je détruirai, et on les appellera pays de méchanceté, peuple contre lequel Yahweh est irrité pour toujours.
\VS{5}Vos yeux le verront, et vous direz : Yahweh est grand par-delà les frontières d'Israël !
\TextTitle{[Le péché des sacrificateurs après le retour d'exil]}
\VS{6}Un fils honore son père, et un serviteur son maître. Si donc je suis Père, où est l'honneur qui m'appartient ? Si je suis maître, où est la crainte qu'on a de moi ? dit Yahweh des armées, à vous sacrificateurs, qui méprisez mon Nom, et qui dites : En quoi avons-nous méprisé ton Nom ?
\VS{7}Vous offrez sur mon autel des aliments souillés, et vous dites : En quoi t'avons-nous profané ? C'est en disant : La table de Yahweh est méprisable !
\VS{8}Et quand vous amenez une bête aveugle pour la sacrifier, n'y a-t-il point de mal en cela ? Quand vous en offrez une boiteuse ou malade, n’est-ce pas mal ? Offre-la à ton gouverneur ! T’agréera-t-il, te recevra-t-il favorablement ? dit Yahweh des armées.
\VS{9}Maintenant, donc suppliez Dieu, pour qu'il ait pitié de nous ! Cela vient de vos mains : Vous recevra-t-il favorablement ? dit Yahweh des armées.
\VS{10}Lequel de vous fermera les portes pour que vous n’allumiez pas en vain le feu sur mon autel ? Je ne prends aucun plaisir en vous, dit Yahweh des armées, et je n’agrée pas l'offrande de vos mains.
\VS{11}Car depuis le soleil levant jusqu'au soleil couchant, mon Nom est grand parmi les nations, et en tous lieux on brûle de l’encens en l'honneur de mon Nom, et des offrandes pures ; car mon Nom est grand parmi les nations, dit Yahweh des armées.
\VS{12}Mais vous, vous le profanez, en disant : La table de Yahweh est souillée, et ce qu’elle rapporte est un aliment méprisable.
\VS{13}Vous dites aussi : Quelle fatigue ! Et vous le dédaignez, dit Yahweh des armées ; vous amenez ce qui a été dérobé, ce qui est boiteux, et malade, ce sont là les offrandes que vous faites ! Accepterai-je cela de vos mains ? dit Yahweh.
\VS{14}C'est pourquoi, maudit soit l'homme trompeur, qui a dans son troupeau un mâle, et qui voue et sacrifie à Yahweh ce qui est corrompu ! Car je suis un grand roi, dit Yahweh des armées, et mon Nom est redoutable parmi les nations.
\TextTitle{[Mise en garde de Yahweh aux sacrificateurs]}
\Chap{2}
\VerseOne{}Maintenant, c’est à vous, sacrificateurs que s'adresse ce commandement :
\VS{2}Si vous n'écoutez pas, et que vous ne preniez point pas à cœur de donner gloire à mon Nom, dit Yahweh des armées, j'enverrai sur vous la malédiction, et je maudirai vos bénédictions ; et déjà même je les ai maudites, parce que vous ne prenez pas cela à cœur.
\VS{3}Voici, je vais détruire vos semences, et je répandrai les excréments de vos victimes sur vos visages, les excréments, dis-je, de vos solennités, et on vous emportera avec eux.
\VS{4}Alors vous saurez que je vous ai adressé ce commandement, afin que mon alliance avec Lévi subsiste, dit Yahweh des armées.
\VS{5}Mon alliance avec lui était la vie et la paix, c’est ce que je lui accordai pour qu’il me craigne ; il a eu pour moi de la crainte, et il a tremblé devant mon Nom.
\VS{6}La loi de la vérité était dans sa bouche, et il ne s'est point trouvé de perversité sur ses lèvres ; il a marché avec moi dans la paix et dans la droiture, et il en a détourné beaucoup de l'iniquité.
\VS{7}Car les lèvres du sacrificateur doivent garder la science, et c’est de sa bouche qu’on demande la loi, parce qu'il est un messager de Yahweh des armées.
\VS{8}Mais vous, vous vous êtes écartés de la voie, vous avez fait de la loi une occasion de chute pour beaucoup, et vous avez corrompu l'alliance de Lévi, dit Yahweh des armées.
\TextTitle{[Infidélités envers des frères et envers Yahweh]}
\VS{9}C'est pourquoi je vous rendrai méprisables et abjects aux yeux de tout le peuple. Parce que vous n’avez pas gardé mes voies, et vous avez égard à l'apparence des personnes quand vous enseignez la loi.
\VS{10}N'avons-nous pas tous un seul Père ? N’est-ce pas un seul Dieu qui nous a créés ? Pourquoi donc agissons-nous avec perfidie l’un avec l’autre, en violant l'alliance de nos pères ?
\VS{11}Juda s’est montré infidèle, et une abomination a été commise en Israël et à Jérusalem ; car Juda a profané ce qui est consacré à Yahweh, ce qu’il aime, il s'est marié à la fille d'un dieu étranger.
\VS{12}Yahweh retranchera l’homme qui fait cela, celui qui veille et qui répond, il le retranchera des tentes de Jacob, et il retranchera celui qui présente une offrande à Yahweh des armées.
\VS{13}Voici une autre chose que vous faites : Vous couvrez l'autel de Yahweh de larmes, de plaintes et de gémissements, en sorte qu’il n’a plus égard aux offrandes et qu’il ne peut rien agréer de vos mains.
\VS{14}Et vous dites : Pourquoi ? C'est parce que Yahweh est intervenu comme témoin entre toi et la femme de ta jeunesse, envers laquelle tu es infidèle, bien qu’elle soit ta compagne et la femme de ton alliance.
\VS{15}Nul n’a fait cela, avec un reste de bon esprit. Un seul l’a fait, et pourquoi ? Parce qu'il cherchait une postérité de Dieu. Prenez donc garde en votre esprit, et qu’aucun ne soit infidèle à la femme de sa jeunesse !
\VS{16}Car je hais la répudiation, dit Yahweh, le Dieu d'Israël, et celui qui couvre de violence son vêtement, dit Yahweh des armées. Prenez donc garde en votre esprit, et ne soyez pas infidèles !
\TextTitle{[Fausse profession religieuse]}
\VS{17}Vous fatiguez Yahweh par vos paroles, et vous dites : En quoi l'avons-nous fatigué ? C'est quand vous dites : Quiconque fait le mal plaît à Yahweh, et il prend plaisir à de tels gens ! Autrement où est le Dieu du jugement ?
\TextTitle{[Venue du précurseur du Messie]}
\Chap{3}
\VerseOne{}Voici, j'enverrai mon messager\FTNT{Ce messager, ou Elie le prophète, est Jean-Baptiste (Es. 40:1-3 ; Mal. 3:1 ; Mt. 3:1-15 ; Mt. 11:14 ; Mt. 17:10-13 ; Mc. 1:1-11 ; Mc. 9:11-13 ; Lu. 1:17 ; Lu. 3:1-5).} ; il préparera le chemin devant moi. Et soudain entrera dans son temple le Seigneur que vous cherchez ; l’Ange de l'alliance, que vous désirez, voici, il vient, dit Yahweh des armées.
\VS{2}Mais qui pourra soutenir le jour de sa venue ? Qui pourra subsister quand il paraîtra ? Car il sera comme le feu du fondeur et comme la potasse des foulons.
\VS{3}Et il sera assis comme celui qui raffine et purifie l'argent ; il nettoiera les fils de Lévi, il les épurera comme l’or et l'argent, et ils présenteront à Yahweh des offrandes avec justice.
\VS{4}Alors l’offrande de Juda et de Jérusalem sera agréable à Yahweh, comme aux anciens jours, comme aux années d'autrefois.
\VS{5}Je m'approcherai de vous pour le jugement, et je me hâterai de témoigner contre les enchanteurs et les adultères, contre ceux qui jurent faussement, et contre ceux qui retiennent le salaire du mercenaire, qui oppriment la veuve et l'orphelin, qui font tort à l'étranger, et qui ne me craignent point, dit Yahweh des armées.
\VS{6}Parce que je suis Yahweh et que je n'ai point changé ; à cause de cela, enfants de Jacob, vous n'avez point été consumés.
\TextTitle{[Le peuple infidèle qui vole Yahweh]}
\VS{7}Depuis le temps de vos pères, vous vous êtes écartés de mes ordonnances, vous ne les avez point observées. Revenez à moi, et je reviendrai à vous, dit Yahweh des armées. Et vous dites : En quoi nous convertirons-nous ?
\VS{8}L'homme trompera-t-il Dieu ? Car vous me trompez ? Et vous dites : En quoi t'avons-nous trompé ? Vous l'avez fait dans les dîmes et dans les offrandes.
\VS{9}Vous êtes certainement maudits, parce que vous me trompez, vous, toute la nation !
\VS{10}Apportez toutes les dîmes\FTNT{Il est question ici de la dîme de la dîme que les Lévites donnaient aux sacrificateurs. Cette dîme était rapportée aux magasins, ou greniers (Né. 10:35-39), là aussi, était stocké toute sorte de trésors. Pour les autres dîmes, voir le commentaire dans Dt. 14:22-29.} aux magasins, afin qu'il y ait provision dans ma maison ; et dès maintenant éprouvez moi en cela, a dit Yahweh des armées, si je ne vous ouvre pas les écluses des cieux, et si je ne répands pas en votre faveur la bénédiction, jusqu'à ce qu'il n'y ait plus assez de place.
\VS{11}Et je réprimerai pour l'amour de vous le dévorateur, et il ne vous ravagera pas les fruits de la terre, et vos vignes ne seront pas stériles dans vos campagnes, a dit Yahweh des armées.
\VS{12}Toutes les nations vous diront heureux, car vous serez un pays de délices, dit Yahweh des armées.
\VS{13}Vos paroles sont rudes contre moi, a dit Yahweh. Et vous dites : Qu'avons-nous donc dit contre toi ?
\VS{14}Vous avez dit : C'est en vain que l’on sert Dieu ; et qu'avons-nous gagné à observer ses ordonnances, et à marcher en pauvre état pour l'amour de Yahweh des armées ?
\VS{15}Et maintenant nous tenons pour heureux les orgueilleux ; et même ceux qui commettent la méchanceté, sont établis ; même ils tentent Dieu et sont délivrés.
\TextTitle{[Le «~reste d'Israël~» demeure fidèle à Yahweh]}
\VS{16}Alors ceux qui craignent Yahweh se parlèrent l'un à l’autre ; et Yahweh fut attentif, et il écouta ; et un livre de souvenir fut écrit devant lui pour ceux qui craignent Yahweh et qui pensent à son Nom.
\VS{17}Ils seront à moi, a dit Yahweh des armées, le jour où je mettrai à part mes plus précieux joyaux, et je leur pardonnerai comme un homme pardonne à son fils qui le sert.
\VS{18}Convertissez-vous donc, et vous verrez la différence qu'il y a entre le juste et le méchant, entre celui qui sert Dieu et celui qui ne le sert pas.
\TextTitle{[Avènement du jour de Yahweh]}
\Chap{4}
\VerseOne{}Car voici, le jour vient, ardent comme une fournaise. Tous les orgueilleux et tous les méchants seront comme du chaume ; et ce jour qui vient les embrasera, dit Yahweh des armées, il ne leur laissera ni racine ni rameau.
\VS{2}Mais pour vous qui craignez mon Nom, se lèvera le Soleil de justice\FTNT{Le Soleil de justice : Jésus-Christ est notre Soleil (Lu. 1:78-79). Cet aspect de Jésus-Christ nous parle de la grâce de Dieu : «~il fait lever son soleil sur les méchants, et sur les gens de bien~» (Mt. 5:45). Le soleil évoque aussi le jugement de Dieu. Ainsi, en plein midi, il est le feu de la justice et de la colère de Dieu. Son pardon et son amour pour nous sont alors comparés à une ombre fraîche qui nous sauve de sa chaleur ardente (Ps. 121 ; Es. 25:4). Dans Ps. 19:6, le soleil est comparé à un époux. Or Jésus-Christ est notre Epoux et le soleil qui nous apporte la guérison.}, et la guérison sera sous ses ailes ; vous sortirez, et bondirez comme les veaux d’une étable.
\VS{3}Et vous foulerez les méchants, car ils seront comme de la cendre sous les plantes de vos pieds, au jour où je ferai mon œuvre, dit Yahweh des armées.
\VS{4}Souvenez-vous de la loi de Moïse, mon serviteur, auquel j’ai prescrit en Horeb, pour tout Israël, des statuts et des ordonnances.
\TextTitle{[Retour d'Elie avant le jour de Yahweh]}
\VS{5}Voici, je vous enverrai Elie, le prophète\FTNT{Voir commentaire en Mal. 3:1.}, avant que le jour grand et redoutable de Yahweh vienne.
\VS{6}Il ramènera le cœur des pères à leurs enfants, et le cœur des enfants à leurs pères, de peur que je ne vienne et que je ne frappe la terre d’interdit.
\PPE{}
\end{multicols}
