\ShortTitle{1 Pi.}\BookTitle{1 Pierre}\BFont
\noindent\hrulefill
{\footnotesize
\textit{
\bigskip
{\centering{}
\\Auteur~: Pierre
\\(Gr.~: Petro)
\\Signification~: Roc, pierre
\\Thème~: La victoire sur la souffrance
\\Date de rédaction~: Env. 65 ap. J.-C.\\}
}
\textit{
\\Cette lettre semble avoir été écrite à Rome, même si Pierre y parlait de «~Babylone~». En ces temps de persécutions, les chrétiens devaient être prudents quant à la manière dont ils parlaient du pouvoir en place, c'est pourquoi ils utilisaient souvent des codes. C'est donc durant une période difficile que fut rédigée cette épître qui s'adressait à des églises d'Asie Mineure dont la plupart furent fondées par Paul. A travers de ces quelques lignes, Pierre exhorte les frères et sœurs à tenir ferme dans la foi malgré les souffrances liées aux épreuves, et les encourage à espérer en Jésus-Christ, leur salut. Il finit cette épître en donnant des conseils quant à l'attitude à avoir au sein de l'église.\bigskip
}
}
\par\nobreak\noindent\hrulefill
\begin{multicols}{2}
\Chap{1}
\TextTitle{Introduction}
\VerseOne{}Pierre, apôtre de Jésus-Christ, à ceux qui sont étrangers et dispersés dans le Pont\FTNT{Le Pont~: Province formant presque la totalité de l'Asie Mineure.}, la Galatie, la Cappadoce, l'Asie et la Bithynie,
\VS{2}élus selon la prescience de Dieu le Père, par la sanctification de l'Esprit afin d'obéir à Jésus-Christ, et qui participent à l'aspersion de son sang~: Que la grâce et la paix vous soient multipliées~!
\TextTitle{Les souffrances du chrétien et sa conduite pour un salut parfait}
\VS{3}Béni soit Dieu, le Père de notre Seigneur Jésus-Christ, qui par sa grande miséricorde nous a régénérés pour une espérance vivante, par la résurrection de Jésus-Christ d'entre les morts,
\VS{4}pour un héritage incorruptible, et qui ne peut ni se souiller, ni se flétrir, qui est conservé dans les cieux pour nous
\VS{5}qui sommes gardés par la puissance de Dieu, par la foi, afin que nous obtenions le salut, qui est prêt à être révélé dans les derniers temps~!
\VS{6}En quoi vous vous réjouissez, quoique vous soyez maintenant affligés pour un peu de temps par diverses épreuves, vu que cela est convenable,
\VS{7}afin que l'épreuve de votre foi, beaucoup plus précieuse que l'or périssable, et qui toutefois est éprouvé par le feu, ait pour résultat la louange, l'honneur et la gloire, lorsque Jésus-Christ sera révélé~;
\VS{8}lequel vous aimez quoique vous ne l'ayez pas vu ; en qui vous croyez, quoique maintenant vous ne le voyiez pas ; et vous vous réjouissez d'une joie ineffable et glorieuse,
\VS{9}remportant le but de votre foi, à savoir le salut de vos âmes.
\VS{10}C'est au sujet de ce salut que les prophètes, qui ont prophétisé concernant la grâce qui vous était destinée, ont fait leurs recherches et leurs investigations.
\VS{11}Ils voulaient sonder l'époque et les circonstances marquées par l'Esprit prophétique de Christ qui était en eux, et qui rendait à l'avance témoignage, leur faisant connaître les souffrances de Christ et la gloire dont elles seraient suivies.
\VS{12}Mais il leur fut révélé que ce n'était pas pour eux-mêmes, mais pour nous, qu'ils administraient ces choses que vous ont annoncées maintenant ceux qui vous ont prêché l'Evangile par le Saint-Esprit envoyé du ciel, et dans lesquelles les anges désirent plonger leurs regards.
\VS{13}C'est pourquoi, ceignez les reins de votre entendement, soyez sobres, et ayez une entière espérance dans la grâce qui vous est présentée, jusqu'à ce que Jésus-Christ soit révélé\FTNT{Révélé, voir commentaire en 2 Th. 1:7.}.
\VS{14}Comme des enfants obéissants, ne vous conformez pas à vos convoitises d'autrefois, pendant votre ignorance. 
\VS{15}Mais, comme celui qui vous a appelés est saint, vous aussi de même soyez saints dans toute votre conduite,
\VS{16}selon ce qu'il est écrit~: Soyez saints, car je suis saint\FTNT{Lé. 11:44.}.
\VS{17}Et si vous invoquez comme votre Père celui qui juge selon l'œuvre de chacun, sans favoritisme, conduisez-vous avec crainte pendant le temps de votre séjour sur la terre,
\VS{18}sachant que vous avez été rachetés de votre vaine manière de vivre, qui vous avait été enseignée par vos pères, non  par des choses corruptibles, comme l'argent ou l'or,
\VS{19}mais par le sang précieux de Christ, comme d'un agneau sans défaut et sans tache,
\VS{20}prédestiné avant la fondation du monde, et manifesté dans les derniers temps pour vous.
\VS{21}Par lui, vous croyez en Dieu, qui l'a ressuscité des morts et lui a donné la gloire, afin que votre foi et votre espérance reposent sur Dieu.
\VS{22}Ayant donc purifié vos âmes en obéissant à la vérité par le Saint-Esprit, afin que vous ayez un amour fraternel qui soit sans hypocrisie, aimez-vous ardemment les uns les autres d'un cœur pur,
\VS{23}puisque vous avez été régénérés, non par une semence corruptible, mais par une semence incorruptible, par la parole vivante de Dieu et qui demeure éternellement.
\VS{24}Car toute chair est comme l'herbe, et toute la gloire de l'homme comme la fleur de l'herbe. L'herbe sèche, et sa fleur tombe~;
\VS{25}mais la parole du Seigneur demeure éternellement\FTNT{Es. 40:6-8.}. Et cette parole est celle qui vous a été annoncée par l'Evangile.
\Chap{2}
\VerseOne{}Ayant donc renoncé à toute sorte de malice, et de toute fraude, et de dissimulation, et d'envie et de toute médisance,
\VS{2}désirez ardemment, comme des enfants nouveau-nés, le lait spirituel et pur, afin que vous croissiez par lui,
\VS{3}si toutefois vous avez goûté combien le Seigneur est bon.
\VS{4}Et vous approchant de lui, pierre vivante, rejetée par les hommes, mais choisie et précieuse devant Dieu~;
\VS{5}et vous aussi, comme des pierres vivantes, vous êtes édifiés pour être une maison spirituelle, et une sainte prêtrise, afin d'offrir des sacrifices spirituels, agréables à Dieu par Jésus-Christ. 
\VS{6}C'est pourquoi aussi, il est dit dans l'Ecriture~: Voici, je mets en Sion la principale pierre\FTNT{Jésus-Christ est la Pierre rejetée par les bâtisseurs. Voir Es. 28:16~; Ps. 118:22.} de l'angle, choisie et précieuse~; et celui qui croit en elle ne sera pas confus.
\VS{7}Elle est donc précieuse pour vous qui croyez. Mais par rapport aux rebelles, il est dit~: La pierre que ceux qui bâtissaient ont rejetée est devenue la principale de l'angle, 
\VS{8}et une pierre d'achoppement, et un rocher de scandale~; ils se heurtent contre la parole, et sont rebelles; et c'est à cela qu'ils sont destinés.
\TextTitle{La position du croyant}
\VS{9}Mais vous, vous êtes la race élue, vous êtes la prêtrise royale, la nation sainte, le peuple acquis, afin que vous annonciez les vertus de celui qui vous a appelés des ténèbres à sa merveilleuse lumière.
\VS{10}Vous qui autrefois n'étiez pas son peuple, mais qui maintenant êtes le peuple de Dieu; vous qui n'aviez pas obtenu miséricorde, mais qui maintenant avez obtenu miséricorde.
\VS{11}Mes bien-aimés, je vous exhorte, comme étrangers et voyageurs, à vous abstenir des convoitises charnelles qui font la guerre à l'âme~;
\VS{12}ayant une conduite honnête avec les Gentils, afin que là même où ils vous calomnient comme si vous étiez des malfaiteurs, ils remarquent vos bonnes œuvres et glorifient Dieu, au jour où il les visitera.
\VS{13}Soyez donc soumis à tout établissement humain, pour l'amour de Dieu~: Soit au roi, comme à celui qui est au-dessus des autres,
\VS{14}soit aux gouverneurs, comme à ceux qui sont envoyés de sa part pour punir les méchants et pour honorer les gens de bien.
\VS{15}Car c'est là la volonté de Dieu, qu'en faisant le bien vous fermiez la bouche à l'ignorance des hommes insensés~;
\VS{16}comme libres, et non pas comme ayant la liberté pour servir de voile à la méchanceté, mais agissant comme des serviteurs de Dieu.
\VS{17}Honorez tout le monde~; aimez tous vos frères~; craignez Dieu~; honorez le roi.
\VS{18}Serviteurs, soyez soumis en toute crainte à vos maîtres, non seulement à ceux qui sont bons et équitables, mais aussi à ceux qui sont méchants.
\VS{19}Car c'est une chose agréable à Dieu si quelqu'un, à cause de la conscience qu'il a envers Dieu, endure des afflictions en souffrant injustement. 
\VS{20}Autrement, quelle gloire en aurez-vous, si lorsque vous péchez et qu'on vous frappe, vous le supportez patiemment~? Mais si quand vous faites le bien et que vous souffrez, vous le supportez patiemment, voilà où Dieu prend plaisir. 
\TextTitle{Les souffrances de Christ}
\VS{21}Car vous êtes appelés à cela, vu même que Christ a souffert pour nous, nous laissant un modèle, afin que vous suiviez ses traces, 
\VS{22}lui qui n'a pas commis de péché, et dans la bouche duquel il ne s'est pas trouvé de fraude~;
\VS{23}qui, lorsqu'on lui disait des outrages, n'en rendait pas; et quand on lui faisait du mal, n'usait pas de menaces, mais il se remettait à celui qui juge justement~; 
\VS{24}lui qui a porté lui-même nos péchés en son corps sur le bois, afin qu'étant morts au péché, nous vivions pour la justice~; lui par la meurtrissure\FTNT{Es. 53:5.} duquel même vous avez été guéris.
\VS{25}Car vous étiez comme des brebis errantes, mais maintenant vous êtes convertis au Pasteur et à l'Evêque de vos âmes. 
\Chap{3}
\TextTitle{La conduite du chrétien dans le mariage}
\VerseOne{}Femmes, soyez de même soumises à vos maris, afin que si quelques-uns n'obéissent pas à la parole, ils soient gagnés sans paroles par la conduite de leurs femmes,
\VS{2}lorsqu'ils verront la pureté de votre conduite, accompagnée de crainte.
\VS{3}Et que votre ornement ne soit pas celui de dehors, qui consiste dans la frisure des cheveux, et dans une parure d'or, et dans la magnificence des habits,
\VS{4}mais que votre parure consiste dans l'homme caché dans le cœur, c'est-à-dire dans l'incorruptibilité d'un esprit doux et paisible, qui est d'un grand prix devant Dieu.
\VS{5}Car c'est ainsi que se paraient aussi autrefois les saintes femmes qui espéraient en Dieu, étant soumises à leurs maris,
\VS{6}comme Sara, qui obéissait à Abraham et l'appelait son seigneur. C'est d'elle que vous êtes devenues les filles, en faisant ce qui est bien et sans vous laisser troubler par aucune crainte.
\VS{7}Et vous, maris, de même comportez-vous selon la sagesse avec vos femmes, comme avec un vase\FTNT{Pierre utilise une métaphore connue des grecs pour parler du corps~: le vase.} plus fragile, c'est-à-dire féminin~; les traitant avec honneur comme étant aussi ensemble héritiers de la grâce de la vie, afin que vos prières ne soient pas interrompues. 
\VS{8}Enfin, soyez tous d'un même sentiment, remplis de compassion les uns envers les autres, d'amour fraternel, miséricordieux et doux.
\VS{9}Ne rendez pas mal pour mal, ou injure pour injure\FTNT{Mt. 5:44.}~; mais au contraire, bénissez~; sachant que c'est à cela que vous êtes appelés, afin d'hériter la bénédiction.
\VS{10}Car celui qui veut aimer sa vie et voir des jours heureux, qu'il préserve sa langue du mal, et ses lèvres de prononcer aucune fraude,
\VS{11}qu'il se détourne du mal et fasse le bien, qu'il recherche la paix, et qu'il tâche de se la procurer~;
\VS{12}car les yeux du Seigneur sont sur les justes, et ses oreilles sont attentives à leurs prières, mais la face du Seigneur est contre ceux qui se conduisent mal.
\TextTitle{Souffrir en faisant le bien}
\VS{13}Et qui vous maltraitera, si vous êtes les imitateurs de celui qui est bon~?
\VS{14}Mais si toutefois vous souffrez quelque chose pour la justice, vous êtes bénis. Et ne craignez pas les maux dont ils veulent vous faire peur, et n'en soyez pas troublés~;
\VS{15}mais sanctifiez le Seigneur dans vos cœurs, et soyez toujours prêts à répondre avec douceur et avec respect, à quiconque vous demande une parole concernant l'espérance qui est en vous, 
\VS{16}et ayant une bonne conscience, afin que ceux qui blâment votre bonne conduite en Christ soient confus de ce qu'ils médisent de vous, comme si vous étiez des malfaiteurs.
\VS{17}Car il vaut mieux, si telle est la volonté de Dieu, que vous souffriez en faisant le bien qu'en faisant le mal.
\TextTitle{Les souffrances de Christ}
\VS{18}Car Christ aussi a souffert une fois pour les péchés, lui juste pour les injustes, afin de nous amener à Dieu, étant mort en la chair, mais vivifié par l'Esprit~;
\VS{19}par lequel aussi il est allé prêcher aux esprits qui sont en prison\FTNT{La possibilité du salut après la mort n'a aucun fondement biblique (Hé. 9:27). Dans ce passage, il est fait mention des pécheurs qui ont vécu du temps de Noé et auxquels le Seigneur Jésus a confirmé la condamnation lorsqu'il est descendu dans l'Hadès ( l'enfer~; Ep. 4:9). Voir aussi commentaire en Mt. 16:18.},
\VS{20}et qui avaient été autrefois incrédules, quand la patience de Dieu les attendait une fois, durant les jours de Noé, tandis que l'arche se préparait dans laquelle un petit nombre, à savoir huit personnes, furent sauvées par l'eau.
\VS{21}A quoi aussi maintenant répond la figure qui nous sauve, c'est-à-dire le baptême~; non pas celui par lequel les ordures de la chair sont nettoyées, mais la promesse faite à Dieu d'une conscience pure, par la résurrection de Jésus-Christ,
\VS{22}qui est à la droite de Dieu, étant allé au Ciel; et auquel sont assujettis les anges, et les dominations et les puissances.
\Chap{4}
\TextTitle{Souffrir dans la chair}
\VerseOne{}Puisque donc Christ a souffert pour nous dans la chair, vous aussi armez-vous de la même pensée. Car celui qui a souffert dans la chair a cessé de pécher,
\VS{2}afin de vivre, non plus selon les convoitises des hommes, mais selon la volonté de Dieu, pendant le temps qui lui reste à vivre dans la chair.
\VS{3}Car il nous suffit d'avoir accompli la volonté des Gentils, pendant le temps de notre vie passée, quand nous nous abandonnions aux impudicités, aux convoitises, à l'ivrognerie, aux excès dans le manger et dans le boire, et aux idolâtries abominables;
\VS{4}ce que ces Gentils trouvent fort étrange, ils vous calomnient de ce que vous ne courez pas avec eux dans un même débordement de dissolution. 
\VS{5}Mais ils rendront compte à celui qui est prêt à juger les vivants et les morts.
\VS{6}Car c'est aussi pour cela que les morts ont été évangélisés, afin qu'ils soient jugés selon les hommes dans la chair, et qu'ils vivent selon Dieu dans l'esprit. 
\TextTitle{L'exercice des dons de l'Esprit}
\VS{7}Or la fin de toutes choses est proche~: Soyez donc sobres et vigilants pour prier. 
\VS{8}Mais surtout, ayez les uns pour les autres une ardente charité, car la charité couvre une multitude de péchés.
\VS{9}Soyez hospitaliers les uns envers les autres, sans murmures. 
\VS{10}Que chacun selon le don qu'il a reçu, l'emploie pour le service des autres, comme de bons gestionnaires des diverses grâces de Dieu. 
\VS{11}Si quelqu'un parle, qu'il parle comme annonçant les paroles de Dieu~; si quelqu'un administre, qu'il administre comme par la puissance que Dieu lui a communiquée, afin qu'en toutes choses Dieu soit glorifié par Jésus-Christ, auquel appartiennent la gloire et la force, aux siècles des siècles. Amen~! 
\TextTitle{Se réjouir dans la souffrance}
\VS{12}Mes bien-aimés, ne trouvez pas étrange quand vous êtes comme dans une fournaise pour votre épreuve, comme s'il vous arrivait quelque chose d'extraordinaire. 
\VS{13}Mais réjouissez-vous de ce que vous participez aux souffrances de Christ, afin que lorsque sa gloire sera révélée, vous vous réjouissiez avec allégresse.
\VS{14}Si on vous dit des injures pour le Nom de Christ, vous êtes bénis, car l'Esprit de gloire, l'Esprit de Dieu repose sur vous, lequel est blasphémé par ceux qui vous noircissent, mais pour vous, vous le glorifiez.
\VS{15}Que nul de vous ne souffre comme meurtrier, ou voleur, ou malfaiteur ou comme se mêlant des affaires d'autrui,
\VS{16}mais si quelqu'un souffre comme chrétien, qu'il n'en ait pas de honte, mais qu'il glorifie Dieu en cela.
\VS{17}Car il est temps que le jugement commence par la maison de Dieu\FTNT{Le jugement commence par la maison de Dieu. Ez. 9:1-11.}. Or s'il commence premièrement par nous, quelle sera la fin de ceux qui n'obéissent pas à l'Evangile de Dieu~?
\VS{18}Et si le juste est difficilement sauvé, où comparaîtra le méchant et le pécheur~? 
\VS{19}Que ceux-là donc aussi, qui souffrent par la volonté de Dieu, puisqu'ils font ce qui est bon, lui recommandent leurs âmes, comme au fidèle Créateur. 
\Chap{5}
\TextTitle{Recommandations}
\VerseOne{}Je prie les anciens qui sont parmi vous, moi qui suis ancien avec eux, et témoin des souffrances de Christ et participant de la gloire qui doit être révélée, et je leur dis~: 
\VS{2}Paissez le troupeau de Dieu qui vous est avec vous, se chargeant de le surveiller; non par contrainte, mais volontairement~; non pour un gain déshonnête, mais par un principe d'affection. 
\VS{3}Et non comme ayant la domination sur les héritages du Seigneur, mais de telle manière que vous soyez les modèles du troupeau. 
\VS{4}Et quand le souverain Pasteur\FTNT{Jésus est notre Souverain Pasteur. Voir Ps. 23~; Jn. 10.} apparaîtra, vous obtiendrez la couronne incorruptible de la gloire.
\VS{5}De même, vous jeunes gens, soyez soumis aux anciens. Et ayant tous de la soumission les uns pour les autres, revêtez-vous\FTNT{Le verbe grec «~egkomboomai~» signifie lien ou bande par laquelle deux choses sont liées ensemble, se revêtir ou se ceindre. A l’époque, les esclaves avaient une écharpe blanche attachée à la ceinture afin de les distinguer des hommes libres. Pierre dans ce passage fait allusion à cet usage, pour montrer la soumission réciproque des chrétiens. C'est aussi une référence au très humble vêtement, espèce de tablier, que les esclaves devaient porter pour ne pas se salir pendant leur travail.} d'humilité~; parce que Dieu résiste aux orgueilleux, mais il fait grâce aux humbles. 
\VS{6}Humiliez-vous donc sous la puissante main de Dieu, afin qu'il vous élève quand le temps sera venu.
\VS{7}Remettez-lui tout ce qui peut vous inquiéter, car il prend soin de vous.
\VS{8}Soyez sobres et veillez~: Car le diable, votre adversaire, tourne autour de vous comme un lion rugissant, cherchant qui il pourra dévorer. 
\VS{9}Résistez-lui donc en demeurant fermes dans la foi, sachant que les mêmes souffrances s'accomplissent dans la compagnie de vos frères qui sont dans le monde. 
\TextTitle{Salutations}
\VS{10}Et que le Dieu de toute grâce, qui nous a appelés à sa gloire éternelle en Jésus-Christ, après que vous aurez souffert un peu de temps, vous rende parfaits, vous affermisse, vous fortifie et vous établisse~! 
\VS{11}A lui soient la gloire et la force, aux siècles des siècles~! Amen~!
\VS{12}Je vous ai écrit brièvement par Silvain, notre frère, que je crois vous être fidèle, vous déclarant et vous protestant que la grâce de Dieu dans laquelle vous êtes est la véritable. 
\VS{13}L'église qui est à Babylone, élue avec vous, et Marc, mon fils, vous saluent. 
\VS{14}Saluez-vous les uns les autres par un baiser de charité. Que la paix soit avec vous tous qui êtes en Jésus-Christ~! Amen~!
\PPE{}
\end{multicols}
