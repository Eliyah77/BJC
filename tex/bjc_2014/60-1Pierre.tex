\ShortTitle{1 Pierre}\BookTitle{1 Pierre}\BFont
\noindent\hrulefill
\textit{
\bigskip
{\centering{}
\\Signifie : Roc, pierre
\\Thème : La victoire sur la souffrance
\\Auteur : Pierre
\\Date de rédaction : Env. 65 apr. J.-C.\\}
}
%\bigskip
\textit{
\\Cette lettre semble avoir été écrite à Rome même si Pierre y parla de « Babylone ». En ces temps de persécutions, les chrétiens devaient être prudents quant à la manière dont ils parlaient du pouvoir en place et utilisaient souvent des codes. C’est donc durant une période difficile que fut rédigée cette épître qui s’adressait à des églises d’Asie Mineure dont la plupart furent fondées par Paul. Au travers de ces quelques lignes, Pierre exhortait les frères et sœurs à tenir ferme dans la foi malgré les souffrances liées aux épreuves et les encourageait à espérer en Jésus Christ, leur salut. Il finit cette épître en donnant des conseils quant à l’attitude à avoir au sein de l’église.\bigskip
}
\par\nobreak\noindent\hrulefill
\begin{multicols}{2}
\TextTitle{[Introduction]}
\Chap{1}
\VerseOne{}Pierre, apôtre de Jésus-Christ, à ceux qui sont étrangers et dispersés dans le Pont{\FTNT{Le Pont : province formant presque la totalité de l’Asie mineure.}}, la Galatie, la Cappadoce, l’Asie et la Bithynie,
\VS{2}et qui sont élus selon la prescience de Dieu le Père, par la sanctification de l’Esprit, afin d’obéir à Jésus-Christ, et qu’ils participent à l'aspersion de son sang : Que la grâce et la paix vous soient multipliées !
\TextTitle{[Les souffrances du chrétien et sa conduite à la lumière d'un salut parfait]}
\VS{3}Béni soit Dieu, le Père de notre Seigneur Jésus-Christ, qui par sa grande miséricorde nous a régénérés pour une espérance vivante, par la résurrection de Jésus-Christ d'entre les morts ;
\VS{4}pour un héritage incorruptible, qui ne peut ni se souiller, ni se corrompre, ni se flétrir ; qui est réservé dans les cieux pour nous,
\VS{5}qui sommes gardés par la puissance de Dieu, par la foi, afin que nous obtenions le salut, qui est prêt à être révélé dans les derniers temps.
\VS{6}C’est là ce qui fait votre joie, quoique maintenant, puisqu’il le faut, vous soyez affligés pour un peu de temps par diverses épreuves, vu que cela est convenable ;
\VS{7}afin que l'épreuve de votre foi, beaucoup plus précieuse que l'or périssable, qui toutefois est éprouvé par le feu, ait pour résultat la louange, l’honneur et la gloire, lorsque Jésus-Christ sera révélé.
\VS{8}Vous l’aimez sans l’avoir vu, vous croyez en lui sans le voir encore, vous réjouissant d’une joie ineffaçable et glorieuse,
\VS{9}parce que vous obtiendrez le salut de vos âmes pour prix de votre foi.
\VS{10}C’est au sujet de ce salut que les prophètes qui ont prophétisé touchant la grâce qui vous était destinée, ont fait leurs recherches et leurs investigations.
\VS{11}Ils voulaient sonder l’époque et les circonstances marquées par l'Esprit prophétique de Christ qui était en eux, et qui rendait à l’avance témoignage, leur faisant connaître les souffrances de Christ et la gloire dont elles seraient suivies.
\VS{12}Mais il leur fut révélé que ce n'était pas pour eux-mêmes, mais pour nous, qu'ils étaient les dispensateurs de ces choses que vous ont annoncées maintenant ceux qui vous ont prêché l'Evangile, par le Saint-Esprit envoyé du ciel, et dans lesquelles les anges désirent plonger leurs regards.
\VS{13}C’est pourquoi, ceignez les reins de votre entendement, soyez sobres, et ayez une entière espérance dans la grâce qui vous est présentée, jusqu'à ce que Jésus-Christ soit révélé{\FTNT{Révélé, du grec «~apokalupsis~», signifie «~mettre à nu~». C’est l’usage d'événements par lequel les choses ou la nature de certains, jusqu'ici cachés deviennent visibles à tous. Voir 1 Jn. 3:2.}}.
\VS{14}Comme des enfants obéissants, ne vous conformez pas aux convoitises que vous aviez autrefois, quand vous étiez dans l’ignorance.
\VS{15}Mais comme celui qui vous a appelés est saint, vous aussi de même soyez saints dans toute votre conduite,
\VS{16}selon ce qu'il est écrit : Soyez saints, car je suis saint{\FTNT{Lé. 11:44.}}.
\VS{17}Et si vous invoquez comme votre Père celui qui juge selon l’œuvre de chacun, sans favoritisme, conduisez-vous avec crainte pendant le temps de votre séjour sur la terre ;
\VS{18}sachant que ce n’est pas par des choses périssables, par l’argent ou l’or, que vous avez été rachetés de la vaine manière de vivre que vous aviez apprise de vos pères,
\VS{19}mais par le sang précieux de Christ, comme d’un agneau sans défaut et sans tache,
\VS{20}prédestiné avant la fondation du monde, et manifesté dans les derniers temps pour vous.
\VS{21}Par lui, vous croyez en Dieu qui l'a ressuscité des morts, et lui a donné la gloire, afin que votre foi et votre espérance reposent sur Dieu.
\VS{22}Ayant donc purifié vos âmes en obéissant à la vérité par le Saint-Esprit, afin que vous ayez un amour fraternel et sans hypocrisie, aimez-vous tendrement les uns les autres d'un cœur pur,
\VS{23}puisque vous avez été régénérés, non par une semence corruptible, mais par une semence incorruptible, par la Parole de Dieu qui vit et demeure éternellement.
\VS{24}Car toute chair est comme l'herbe, et toute la gloire de l'homme comme la fleur de l'herbe. L'herbe sèche, et sa fleur tombe ;
\VS{25}mais la parole du Seigneur demeure éternellement{\FTNT{Es. 40 : 6-8.}}. Et cette parole est celle qui vous a été annoncée par l’Evangile.
\Chap{2}
\VerseOne{}Ayant donc renoncé à toute sorte de malice, de toute fraude, de dissimulation, d'envie et de médisance,
\VS{2}désirez ardemment, comme des enfants nouveau-nés, le lait spirituel et pur afin que par lui vous croissiez,
\VS{3}si vous avez goûté combien le Seigneur est bon.
\VS{4}Approchez-vous de lui, pierre vivante, rejetée par les hommes, mais choisie et précieuse devant Dieu ;
\VS{5}et vous-mêmes, comme des pierres vivantes, entrez dans la structure de l’édifice pour être une maison spirituelle, et de saints sacrificateurs, pour offrir des sacrifices spirituels et agréables à Dieu par Jésus-Christ.
\VS{6}C'est pourquoi il est dit dans l'Ecriture : Voici, je mets en Sion la principale pierre{\FTNT{Jésus-Christ est la Pierre rejetée par les bâtisseurs. Voir Es. 28:16 ; Ps. 118:22.}} de l’angle, choisie et précieuse ; et celui qui croit en elle, ne sera point confus.
\VS{7}Elle est donc précieuse pour vous qui croyez ; mais quant aux rebelles, il est dit : La pierre qu’ont rejetée ceux qui bâtissaient est devenue la principale de l’angle, et une pierre d'achoppement, et un rocher de scandale ;
\VS{8}Ils heurtent contre la parole, et sont rebelles, c’est à cela qu’ils sont destinés.
\TextTitle{[La vie chrétienne en rapport avec la position du croyant et les souffrances de Christ, notre substitut]}
\VS{9}Vous, au contraire, vous êtes la race élue, vous êtes la sacrificature royale, la nation sainte, le peuple acquis, afin que vous annonciez les vertus de celui qui vous a appelés des ténèbres à sa merveilleuse lumière,
\VS{10}vous qui autrefois n'étiez pas son peuple, mais qui maintenant êtes le peuple de Dieu, vous qui n'aviez point obtenu miséricorde, mais qui maintenant avez obtenu miséricorde.
\VS{11}Mes bien-aimés, je vous exhorte, comme étrangers et voyageurs, à vous abstenir des convoitises charnelles qui font la guerre à l'âme.
\VS{12}Ayant une conduite honnête au milieu des gentils, afin que là même où ils vous calomnient comme si vous étiez des malfaiteurs, ils remarquent vos bonnes œuvres et glorifient Dieu au jour où il les visitera.
\VS{13}Soyez donc soumis, par amour pour le Seigneur, à toute autorité établie parmi les hommes, soit au roi, comme souverain,
\VS{14}soit aux gouverneurs, comme envoyés par lui pour punir les malfaiteurs et pour honorer les gens de bien.
\VS{15}Car c'est la volonté de Dieu, qu'en faisant le bien, vous fermiez la bouche des hommes ignorants et insensés ;
\VS{16}étant libres, sans faire de la liberté un voile qui couvre la méchanceté, mais agissant comme des serviteurs de Dieu.
\VS{17}Honorez tout le monde ; aimez tous vos frères ; craignez Dieu ; honorez le roi.
\VS{18}Serviteurs, soyez soumis en toute crainte à vos maîtres, non seulement à ceux qui sont bons et équitables, mais aussi à ceux qui sont méchants.
\VS{19}Car c’est une grâce de supporter des afflictions par motif de conscience envers Dieu, quand on souffre injustement.
\VS{20}Autrement, quelle gloire y a-t-il à supporter de mauvais traitements pour avoir commis des fautes ? Mais si vous supportez la souffrance lorsque vous faites ce qui est bien, c’est une grâce devant Dieu.
\TextTitle{[Les souffrances de Christ, le Substitut des hommes]}
\VS{21}Et c’est à cela que vous avez été appelés, parce que Christ aussi a souffert pour vous, vous laissant un modèle, afin que vous suiviez ses traces.
\VS{22}Lui qui n'a point commis de péché, et dans la bouche duquel il ne s’est point trouvé de fraude ;
\VS{23}lui qui injurié, ne rendait point d’injures, maltraité, ne faisait point de menaces, mais s’en remettait à celui qui juge justement ;
\VS{24}lui qui a porté lui-même nos péchés en son corps sur le bois, afin que morts aux péchés, nous vivions pour la justice ; lui par la meurtrissure duquel vous avez été guéris.
\VS{25}Car vous étiez comme des brebis errantes. Mais maintenant vous êtes retournés vers le Pasteur et à l'Evêque de vos âmes.
\TextTitle{[La sainteté de la conduite chrétienne à la maison et à l'église]}
\Chap{3}
\VerseOne{}Femmes, soyez de même soumises à vos maris, afin que si quelques-uns n'obéissent point à la parole, ils soient gagnés sans paroles, par la conduite de leurs femmes,
\VS{2}lorsqu'ils verront la pureté de votre conduite, accompagnée de respect.
\VS{3}Que votre parure ne soit point celle de l’extérieur, qui consiste dans les cheveux tressés, les ornements d’or, ou les habits somptueux ;
\VS{4}mais la parure intérieure et cachée dans le cœur, la pureté incorruptible d'un esprit doux et paisible, qui est d'un grand prix devant Dieu.
\VS{5}Car c'est ainsi que se paraient autrefois les saintes femmes qui espéraient en Dieu, étant soumises à leurs maris,
\VS{6}comme Sara, qui obéissait à Abraham et l’appelait son seigneur. C’est d’elle que vous êtes devenues les filles, en faisant bien ce qui est bien, sans vous laisser troubler par aucune crainte.
\VS{7}Et vous, maris, de même conduisez-vous avec prudence envers vos femmes, comme un vase{\FTNT{Vase plus fragile. Le «~Vase~» était une métaphore grecque commune pour parler du corps car les Grecs pensaient que l'âme vivait temporairement dans les corps.}} plus fragile, c’est-à-dire, féminin ; honorez-les, comme devant aussi hériter avec vous de la grâce de vie, afin que vos prières ne soient point interrompues.
\VS{8}Enfin, soyez tous animés des mêmes pensées, remplis de compassion les uns envers les autres, vous aimant fraternellement, étant miséricordieux et doux.
\VS{9}Ne rendez point mal pour mal, ou injure pour injure{\FTNT{(2) Mt. 5:44.}} ; mais au contraire, bénissez ; sachant que c’est à cela que vous êtes appelés, afin d’hériter la bénédiction.
\VS{10}Si quelqu’un veut aimer sa vie et voir des jours heureux, qu'il préserve sa langue du mal, et ses lèvres de prononcer aucune fraude,
\VS{11}qu'il se détourne du mal, et fasse le bien, qu'il recherche la paix et la poursuive ;
\VS{12}car les yeux du Seigneur sont sur les justes, et ses oreilles sont attentives à leurs prières ; mais la face du Seigneur est contre ceux qui font le mal.
\TextTitle{[La sainteté de la conduite chrétienne aux yeux du monde, par amour pour la justice]}
\VS{13}Et qui vous maltraitera, si vous êtes les imitateurs de celui qui est bon ?
\VS{14}D’ailleurs, même si vous souffriez pour la justice, vous seriez heureux. N’ayez d’eux aucune crainte, et ne soyez pas troublés ;
\VS{15}mais sanctifiez le Seigneur dans vos cœurs, et soyez toujours prêts à répondre avec douceur et respect à tous ceux qui vous demanderont raison de l'espérance qui est en vous,
\VS{16}et ayant une bonne conscience, afin que ceux qui blâment votre bonne conduite en Christ, soient confus de ce qu'ils parlent mal de vous, comme si vous étiez des malfaiteurs.
\VS{17}Car il vaut mieux souffrir, si telle est la volonté de Dieu, en faisant le bien qu’en faisant le mal.
\TextTitle{[Les souffrances de Christ, le substitut, prêchées par l'Esprit de Christ en Noé]}
\VS{18}Car Christ aussi a souffert une fois pour les péchés, lui juste pour les injustes, afin de nous amener à Dieu ; étant mort selon la chair, mais ayant été vivifié par l'Esprit,
\VS{19}par lequel aussi il est allé prêcher aux esprits en prison{\FTNT{Les esprits en prison. La possibilité du salut après la mort n’a aucun fondement biblique. Voir Hé. 9:27.}},
\VS{20}qui autrefois avaient été incrédules, lorsque la patience de Dieu se prolongeait, aux jours de Noé, pendant la construction de l'arche, dans laquelle un petit nombre de personnes, c’est-à-dire huit, furent sauvées à travers l'eau.
\VS{21}Cette eau était une figure du baptême, qui n’est pas la purification des souillures du corps, mais l’engagement d’une bonne conscience envers Dieu, et qui maintenant vous sauve, vous aussi, par la résurrection de Jésus-Christ,
\VS{22}lequel est à la droite de Dieu, depuis qu’il est allé au ciel, et auquel sont assujettis les anges, les dominations et les autorités.
\TextTitle{[Christ a souffert : pourquoi ne souffririons-nous pas ?]}
\Chap{4}
\VerseOne{}Ainsi donc, puisque Christ a souffert pour nous dans la chair, vous aussi armez-vous de la même pensée. Car celui qui a souffert dans la chair, a cessé de pécher,
\VS{2}afin de vivre, non plus selon les convoitises des hommes, mais selon la volonté de Dieu, pendant le temps qui lui reste à vivre dans la chair.
\VS{3}C’est assez, en effet, d’avoir accompli la volonté des gentils, en marchant dans le dérèglement, les convoitises, l’ivrognerie, les orgies, et les idolâtries criminelles.
\VS{4}C’est pourquoi, ils trouvent étrange que vous ne vous précipitiez pas avec eux dans le même débordement de débauche, et ils vous calomnient.
\VS{5}Mais ils rendront compte à celui qui est prêt à juger les vivants et les morts.
\VS{6}C’est pour cela que l’Evangile fut annoncé aux morts, afin que, après avoir été jugés comme les hommes dans la chair, ils vivent selon Dieu par l'Esprit.
\TextTitle{[La conduite chrétienne en fonction des temps dans lesquels nous vivons]}
\VS{7}La fin de toutes choses est proche : Soyez donc sobres, et vigilants pour vaquer à la prière.
\VS{8}Mais surtout, ayez les uns pour les autres une ardente charité, car la charité couvre une multitude de péchés.
\VS{9}Exercez l’hospitalité les uns envers les autres, sans murmures.
\VS{10}Comme de bons dispensateurs des diverses grâces de Dieu, que chacun de vous mette au service des autres le don qu'il a reçu.
\VS{11}Si quelqu'un parle, que ce soit comme annonçant les oracles de Dieu, si quelqu'un remplit un ministère, qu'il le remplisse selon la vertu que Dieu communique, afin qu'en toutes choses Dieu soit glorifié par Jésus-Christ, à qui appartiennent la gloire et la force aux siècles des siècles, Amen !
\VS{12}Mes bien-aimés, ne trouvez point étrange si vous êtes comme dans une fournaise pour être éprouvés, comme s’il vous arrivait quelque chose d'extraordinaire.
\VS{13}Réjouissez-vous, au contraire, de ce que vous ayez part aux souffrances de Christ, afin que vous soyez aussi dans la joie et dans l’allégresse lorsque sa gloire apparaîtra.
\VS{14}Si vous êtes outragés pour le Nom de Christ, vous êtes heureux, parce que l'Esprit de gloire, qui est l’Esprit de Dieu, repose sur vous, lequel est blasphémé par eux ; mais il est glorifié par vous.
\VS{15}Que personne d’entre vous ne souffre comme meurtrier, ou voleur, ou malfaiteur, ou pour s’être ingéré dans les affaires d'autrui.
\VS{16}Mais si quelqu'un souffre comme chrétien, qu'il n'en ait point de honte, mais qu'il glorifie Dieu à cause de cela.
\VS{17}Car il est temps que le jugement commence par la maison de Dieu{\FTNT{Le jugement commence par la maison de Dieu. Ez. 9:1-11.}}. Or s'il commence premièrement par nous, quelle sera la fin de ceux qui n'obéissent pas à l'Evangile de Dieu ?
\VS{18}Et si le juste est sauvé difficilement, que deviendront l’impie et le pécheur ?
\VS{19}Ainsi donc, que ceux qui souffrent selon la volonté de Dieu, recommandent leurs âmes au fidèle Créateur, en faisant ce qui est bon.
\TextTitle{[Le service chrétien inspiré par la perspective du retour de Christ]}
\Chap{5}
\VerseOne{}Voici les exhortations que j’adresse aux anciens qui sont parmi vous, moi, ancien comme eux, témoin des souffrances de Christ, et participant de la gloire qui doit être révélée :
\VS{2}Paissez le troupeau de Dieu qui est avec vous, le surveillant, non par contrainte, mais volontairement ; non pour un gain sordide, mais avec dévouement.
\VS{3}Et non comme dominant sur l’héritage du Seigneur, mais en étant les modèles du troupeau.
\VS{4}Et lorsque le souverain Pasteur{\FTNT{Jésus est notre Souverain Pasteur. Voir Ps. 23 ; Jn. 10.}} apparaîtra, vous obtiendrez la couronne incorruptible de la gloire.
\VS{5}De même, jeunes gens soyez soumis aux anciens. Et tous, soumettez-vous les uns les autres, soyez revêtus d’humilité ; car Dieu résiste aux orgueilleux, mais il fait grâce aux humbles.
\VS{6}Humiliez-vous donc sous la puissante main de Dieu, afin qu'il vous élève au temps convenable ;
\VS{7}et déchargez-vous sur lui de tous vos soucis, car lui-même prend soin de vous.
\VS{8}Soyez sobres, et veillez. Car votre adversaire, le diable, rôde comme un lion rugissant, cherchant qui il dévorera.
\VS{9}Résistez-lui avec une foi ferme, sachant que les mêmes souffrances sont imposées à vos frères sont dans le monde.
\TextTitle{[Conclusion]}
\VS{10}Le Dieu de toute grâce, qui nous a appelés à sa gloire éternelle en Jésus-Christ, après que vous aurez souffert un peu de temps, vous perfectionnera lui-même, vous affermira, vous fortifiera, vous rendra inébranlables.
\VS{11}A lui soient la gloire et la force, aux siècles des siècles, Amen !
\VS{12}C’est par Silvain, qui est comme je l’estime un frère fidèle, que je vous écris ce peu de mots, pour vous exhorter et pour vous attester que la grâce de Dieu à laquelle vous êtes attachés est la véritable.
\VS{13}L’Eglise des élus qui est à Babylone vous salue, ainsi que Marc, mon fils.
\VS{14}Saluez-vous les uns les autres par un baiser de charité. Que la paix soit avec vous tous qui êtes en Jésus-Christ, Amen !
\PPE{}
\end{multicols}
