\ShortTitle{Actes}\BookTitle{Actes}\BFont
\noindent\hrulefill
\textit{
\bigskip
{\centering{}
\\Thème : Les missions du 1er siècle
\\Auteur : Luc
\\Date de rédaction : Env. 60 apr. J.-C.\\}
}
%\bigskip
\textit{
\\D’origine grecque, Luc fut l’auteur du livre communément appelé «actes des apôtres» et de l’évangile éponyme ; tous deux adressés à Théophile. Ce livre retrace la genèse de l’Église, de l’ascension de Jésus à la Pentecôte, de la prédication vivante et fructueuse de Pierre à la conversion de Paul, jusqu’au voyage de celui-ci à Rome en tant que prisonnier. On y découvre des apôtres déterminés, des ouvriers de Christ qui acceptèrent de subir l’humiliation et la persécution par amour de la vérité. Sont également présentés des hommes et des femmes qui - touchés par la simplicité de l’évangile du Royaume - se convertirent puis se firent baptiser.
\bigskip
\\Bien plus qu’un recueil relatant de banales manifestations, le livre des «actes du Saint-Esprit» témoigne de la résurrection et de la puissance de Jésus-Christ manifestée au travers de son corps. Il retrace l’origine et le développement du premier réveil qui fut un véritable bouleversement au sein d’un empire en proie à l’impiété.\bigskip
}
\par\nobreak\noindent\hrulefill
\begin{multicols}{2}
\TextTitle{[I. Les disciples attendent l'effusion du Saint-Esprit
\\Introduction : le Messie ressuscité parle des choses qui concernent le Royaume de Dieu pendant 40 jours]}
\Chap{1}
\VerseOne{}Nous avons rempli le premier traité, Théophile ! De toutes les choses que Jésus a faites et enseignées,
\VS{2}jusqu'au jour où il fut enlevé au ciel, après avoir donné par le Saint-Esprit, ses ordres aux apôtres qu'il avait choisis.
\VS{3}Après qu’il eut souffert, il leur apparut vivant, et leur en donna plusieurs preuves, se montrant à eux pendant quarante jours, et parlant des choses qui concernent le Royaume de Dieu.
\VS{4}Et les ayant assemblés, il leur recommanda de ne pas partir de Jérusalem, mais d’attendre la promesse du Père, ce que je vous ai annoncé leur dit-il ;
\VS{5}car Jean a baptisé d'eau, mais vous serez baptisés du Saint-Esprit dans peu de jours.
\VS{6}Eux donc étant assemblés l'interrogèrent, disant : Seigneur, est-ce en ce temps-ci que tu rétabliras le royaume d'Israël ?
\VS{7}Mais il leur dit : Ce n'est pas à vous de connaître les temps ou les moments que le Père a fixés de sa propre autorité.
\TextTitle{[Mission d'évangélisation]
\\(Mt. 28:18-20 ; Mc. 16:15-18 ; Lu. 24:47-48 ; Jn. 20:21-22)}
\VS{8}Mais vous recevrez la puissance du Saint-Esprit qui viendra sur vous, et vous serez mes témoins à Jérusalem, dans toute la Judée, dans la Samarie, et jusqu'aux extrémités de la terre.
\VS{9}Après avoir dit ces choses, il fut élevé au ciel pendant qu’ils le regardaient, et une nuée le déroba à leurs yeux.
\TextTitle{[Promesse du retour de Jésus]}
\VS{10}Et comme ils avaient les yeux fixés vers le ciel, pendant qu’il s’en allait, voici, deux hommes en vêtements blancs se présentèrent devant eux,
\VS{11}et leur dirent : Hommes Galiléens, pourquoi vous arrêtez-vous à regarder au ciel ? Ce Jésus qui a été enlevé au ciel du milieu de vous, reviendra de la même manière que vous l'avez contemplé montant au ciel\FTNT{Jésus-Christ est monté au ciel depuis la montagne des Oliviers et lors de son retour, ses pieds se poseront sur cette montagne. Voir Za. 14.}.
\TextTitle{[Attente de l'effusion de l'Esprit]
\\(v.5)}
\VS{12}Alors ils retournèrent à Jérusalem de la montagne appelée la Montagne des oliviers, qui est près de Jérusalem, à la distance d’un chemin de sabbat\FTNT{Chemin de sabbat : C’est la distance qu’il est permis à un juif de parcourir le jour de sabbat (Ex. 16:29). Elle correspond à deux mille coudées ou 1100 m.}.
\VS{13}Et quand ils furent entrés dans la ville, ils montèrent dans une chambre haute où demeuraient Pierre et Jacques, Jean et André, Philippe et Thomas, Barthélemy et Matthieu, Jacques fils d'Alphée, et Simon le zélote, et Jude frère de Jacques.
\VS{14}Tous d’un commun accord persévéraient dans la prière et dans l’oraison avec les femmes, avec Marie, mère de Jésus, et avec ses frères.
\TextTitle{[Matthias désigné pour remplacer Judas]}
\VS{15}Et en ces jours-là, Pierre se leva au milieu des disciples, qui étaient là assemblés au nombre d'environ cent vingt personnes, et il leur dit :
\VS{16}Hommes frères ! Il fallait que s’accomplisse ce qui a été écrit, ce que le Saint-Esprit a annoncé d’avance par la bouche de David, au sujet de Judas, qui a été le guide de ceux qui ont saisi Jésus.
\VS{17}Il était compté parmi nous, et il avait part au même ministère.
\VS{18}Mais après avoir acquis un champ avec le salaire du crime qui lui avait été donné, il est tombé, s’est rompu par le milieu, et toutes ses entrailles ont été répandues.
\VS{19}La chose a été si connue de tous les habitants de Jérusalem, que ce champ a été appelé dans leur propre langue Hakeldama, c'est-à-dire le champ du sang.
\VS{20}Car il est écrit dans le livre des Psaumes : Que sa demeure soit déserte, et que personne ne l’habite\FTNT{Ps. 69:26.}. Et, qu'un autre prenne sa charge\FTNT{Charge : Du grec ~episkope~, il s’agit de la fonction d’un ancien. Ps.109:8.}.
\VS{21}Il faut donc que parmi ceux qui ont été avec nous pendant tout le temps que le Seigneur Jésus a vécu avec nous,
\VS{22}depuis le baptême de Jean jusqu'au jour où il a été enlevé du milieu de nous, qu’il y en ait un qui soit témoin avec nous de sa résurrection.
\VS{23}Et ils en présentèrent deux : Joseph, appelé Barsabbas, surnommé Justus, et Matthias.
\VS{24}Et en priant, ils dirent : Toi, Seigneur, qui connais les cœurs de tous, désigne lequel de ces deux tu as choisi,
\VS{25}afin qu'il prenne part au ministère et à l’apostolat que Judas a abandonné pour aller en son lieu.
\VS{26}Puis ils tirèrent au sort, et le sort tomba sur Matthias, qui d'une commune voix fut mis au rang des onze apôtres.
\TextTitle{[l'Eglise ; Pentecôte : l'Esprit descend du ciel]}
\Chap{2}
\VerseOne{}Le jour de la Pentecôte étant arrivé, ils étaient tous ensemble dans un même lieu.
\VS{2}Tout à coup, il vint du ciel un bruit, comme le bruit d'un vent qui souffle avec impétuosité, et il remplit toute la maison où ils étaient assis.
\VS{3}Des langues, semblables à des langues de feu, leur apparurent, séparées les unes des autres, et se posèrent sur chacun d’eux.
\VS{4}Et ils furent tous remplis du Saint-Esprit, et commencèrent à parler des langues étrangères selon que l'Esprit leur donnait de s’exprimer.
\VS{5}Or, il y avait en séjour à Jérusalem des Juifs, hommes pieux, de toutes les nations qui sont sous le ciel.
\VS{6}Au bruit qui eut lieu, la multitude accourut, et elle fut confondue parce que chacun les entendait parler dans sa propre langue.
\VS{7}Ils étaient tous dans l’étonnement et la surprise, et ils se disaient les uns les autres : Voici, ces gens qui parlent ne sont-ils pas tous Galiléens ?
\VS{8}Comment les entendons-nous dans notre propre langue à chacun, dans notre langue maternelle ?
\VS{9}Parthes, Mèdes, Elamites, et ceux qui habitent la Mésopotamie, la Judée, la Cappadoce, le Pont, l’Asie,
\VS{10}la Phrygie, la Pamphylie, l’Egypte, le territoire de la Libye qui est près de Cyrène, et ceux qui sont venus de Rome ?
\VS{11}Juifs et Prosélytes, Crétois et Arabes, comment les entendons-nous parler chacun dans notre langue des merveilles de Dieu ?
\VS{12}Ils étaient tous dans l’étonnement, et ils ne savaient que penser, ils se disaient les uns les autres : Que veut dire ceci ?
\VS{13}Mais les autres se moquaient, et disaient : C’est qu'ils sont pleins de vin doux.
\TextTitle{[Prédication de Pierre à la Pentecôte]}
\VS{14}Alors Pierre, se présentant avec les onze, éleva sa voix, et leur dit : Hommes Juifs, et vous tous qui habitez à Jérusalem, sachez ceci, et écoutez avec attention mes paroles.
\TextTitle{[Explication : l'effusion de l'Esprit]
\\(Joë. 2:28-32)}
\VS{15}Ces gens ne sont pas ivres, comme vous le pensez, car c'est la troisième heure\FTNT{Neuf heures du matin} du jour.
\VS{16}Mais c'est ici ce qui a été dit par le prophète Joël :
\VS{17}Dans les derniers jours, dit Dieu, je répandrai de mon Esprit sur toute chair ; vos fils et vos filles prophétiseront, vos jeunes gens auront des visions, et vos vieillards auront des songes.
\VS{18}Et dans ces jours-là je répandrai de mon Esprit sur mes serviteurs et sur mes servantes, et ils prophétiseront.
\VS{19}Et je ferai des choses merveilleuses en haut dans le ciel, et des prodiges en bas sur la terre, du sang, du feu, et une vapeur de fumée.
\VS{20}Le soleil se changera en ténèbres, et la lune en sang, avant l’arrivée du jour du Seigneur, de ce jour grand et glorieux.
\VS{21}Alors quiconque invoquera le Nom du Seigneur sera sauvé\FTNT{Jo. 2:28-32.}.
\TextTitle{[Le Messie ressuscité est proclamé]
\\(v. 25-31 ; Ps. 16:8-11)}
\VS{22}Hommes Israélites, écoutez ces paroles ! Jésus de Nazareth, homme à qui Dieu a rendu témoignage devant vous par les miracles, les merveilles, et les prodiges qu’il a opérés par lui au milieu de vous, comme aussi vous le savez ;
\VS{23}cet homme ayant été livré selon le dessein arrêté et selon la prescience de Dieu, vous l'avez attaché à la croix, vous l’avez fait mourir par les mains des impies.
\VS{24}Mais Dieu l'a ressuscité, ayant brisé les liens de la mort, parce qu'il n'était pas possible qu'il soit retenu par elle.
\VS{25}Car David dit de lui : Je contemplais constamment le Seigneur devant moi : Parce qu’il est à ma droite, afin que je ne sois point ébranlé\FTNT{Ps. 16:8-11.}.
\VS{26}C'est pourquoi mon cœur est dans la joie, et ma langue dans l’allégresse ; et de plus, ma chair reposera avec espérance.
\VS{27}Car tu ne laisseras point mon âme dans le scheol et tu ne permettras point que ton Saint voie la corruption.
\VS{28}Tu m'as fait connaître le chemin de la vie, tu me rempliras de joie par ta présence\FTNT{Ps. 16:11.}.
\VS{29}Hommes frères, qu’il me soit permis de vous dire librement, au sujet du patriarche David, qu'il est mort, qu'il a été enseveli, et que son sépulcre existe encore parmi nous jusqu’à ce jour.
\VS{30}Mais comme il était prophète, et qu'il savait que Dieu lui avait promis avec serment, qu’il ferait naître le Christ de sa postérité selon la chair, pour le faire asseoir sur son trône,
\VS{31}c’est la résurrection du Christ qu’il a prévue et annoncée, en disant qu’il ne serait pas abandonné dans le scheol et que sa chair ne verrait pas la corruption.
\VS{32}Dieu a ressuscité ce Jésus ; nous en sommes tous témoins.
\VS{33}Elevé au ciel à la droite de Dieu, il a reçu de son Père le Saint-Esprit qui avait été promis, et il l’a répandu comme vous le voyez et l’entendez maintenant.
\VS{34}Car David n'est pas monté au ciel ; mais lui-même dit : Le Seigneur a dit à mon Seigneur : Assieds-toi à ma droite,
\VS{35}jusqu'à ce que je fasse de tes ennemis ton marchepied\FTNT{Ps. 110:1.}. 
\VS{36}Que toute la maison d'Israël sache donc avec certitude que Dieu a fait Seigneur et Christ, ce Jésus, que vous avez crucifié.
\TextTitle{[Pierre exhorte les gens à la repentance]}
\VS{37}Après avoir entendu ces choses, ils eurent le cœur touché de componction\FTNT{Componction : Tristesse produite par les effets du repentir, le regret d'avoir offensé Dieu.}, et ils dirent à Pierre et aux autres apôtres : Hommes frères, que ferons-nous ?
\VS{38}Pierre leur dit : Repentez-vous, et que chacun de vous soit baptisé au Nom de Jésus-Christ, pour obtenir le pardon de vos péchés, et vous recevrez le don du Saint-Esprit.
\VS{39}Car la promesse est pour vous, pour vos enfants, et pour tous ceux qui sont loin, en aussi grand nombre que le Seigneur notre Dieu les appellera.
\VS{40}Et par plusieurs autres paroles il les conjurait et les exhortait, en disant : Sauvez-vous de cette génération perverse.
\TextTitle{[Trois mille personnes convertis reçoivent le baptême]}
\VS{41}Ceux qui reçurent de bon cœur sa parole, furent baptisés ; et en ce jour-là furent ajoutées à l'Eglise environ trois mille âmes.
\VS{42}Ils persévéraient tous dans la doctrine des apôtres, dans la communion fraternelle, dans la fraction du pain, et dans les prières.
\VS{43}Et tout le monde avait de la crainte, et beaucoup de miracles et de prodiges se faisaient par les apôtres.
\VS{44}Tous ceux qui croyaient étaient ensemble dans le même lieu, et ils avaient tout en commun ;
\VS{45}et ils vendaient leurs possessions et leurs biens, et les distribuaient à tous, selon les besoins de chacun.
\VS{46}Et tous les jours, ils persévéraient tous d'un commun accord dans le temple ; et rompant le pain de maison en maison, ils prenaient leur repas avec joie et simplicité de cœur ;
\VS{47}louant Dieu, et se rendant agréables à tout le peuple. Et le Seigneur ajoutait tous les jours à l'Eglise des gens pour être sauvés.
\TextTitle{[Guérison de l'homme boiteux de naissance]}
\Chap{3}
\VerseOne{}Pierre et Jean montaient ensemble au temple à l'heure de la prière ; c’était la neuvième heure.
\VS{2}Il y avait un homme boiteux de naissance, qu’on portait et qu’on plaçait tous les jours à la porte du temple appelée la Belle, pour demander l'aumône à ceux qui entraient au temple.
\VS{3}Cet homme voyant Pierre et Jean qui allaient entrer au temple, les pria de lui donner l'aumône.
\VS{4}Alors Pierre, de même que Jean, fixa les yeux sur lui, et lui dit : Regarde-nous.
\VS{5}Et il les regardait attentivement, s'attendant à recevoir d'eux quelque chose.
\VS{6}Mais Pierre lui dit : Je n'ai ni argent ni or ; mais ce que j'ai, je te le donne : Au Nom de Jésus-Christ de Nazareth, lève-toi et marche.
\VS{7}Et le prenant par la main droite, il le fit lever ; et aussitôt les plantes et les chevilles de ses pieds devinrent fermes.
\VS{8}D’un saut, il fut debout et marcha. Et il entra avec eux dans le temple, marchant, sautant, et louant Dieu.
\VS{9}Et tout le peuple le vit marchant et louant Dieu.
\VS{10}Ils reconnaissaient que c'était celui qui était assis à la Belle porte du temple, pour demander l'aumône. Ils furent remplis d'admiration et d'étonnement de ce qui lui était arrivé.
\VS{11}Et comme le boiteux, qui avait été guéri, tenait par la main Pierre et Jean, tout le peuple étonné accourut vers eux, au portique dit de Salomon.
\TextTitle{[Prédication de Pierre]}
\VS{12}Mais Pierre voyant cela, dit au peuple : Hommes Israélites, pourquoi vous étonnez-vous de ceci ? Ou pourquoi avez-vous les regards fixés sur nous, comme si par notre puissance ou par notre piété, nous avions fait marcher cet homme ?
\VS{13}Le Dieu d'Abraham, d'Isaac, et de Jacob, le Dieu de nos pères, a glorifié son Fils Jésus, que vous avez livré et renié devant Pilate, qui était d’avis qu’on le relâche.
\VS{14}Mais vous avez renié le Saint et le Juste, et vous avez demandé qu'on vous rende un meurtrier.
\VS{15}Vous avez fait mourir le Prince de la vie, que Dieu a ressuscité des morts ; nous en sommes témoins.
\VS{16}C’est par la foi en son Nom, que son Nom a raffermi les pieds de cet homme que vous voyez et connaissez. La foi que nous avons en lui a donné à cet homme cette entière guérison de tous ses membres, en présence de vous tous.
\VS{17}Et maintenant, mes frères, je sais que vous avez agi par ignorance, ainsi que vos chefs.
\VS{18}Mais Dieu a ainsi accompli les choses qu'il avait prédites par la bouche de tous ses prophètes, que le Christ devait souffrir\FTNT{Es. 53.}.
\VS{19}Repentez-vous donc, et convertissez-vous, afin que vos péchés soient effacés ;
\VS{20}afin que des temps de rafraîchissement viennent par la présence du Seigneur, et qu'il envoie celui qui vous a été auparavant annoncé, Jésus-Christ,
\VS{21}que le ciel doit recevoir, jusqu'au temps du rétablissement de toutes les choses que Dieu a prononcées par la bouche de tous ses saints prophètes, dès le commencement du monde.
\VS{22}Moïse a dit à nos pères : Le Seigneur votre Dieu, vous suscitera d'entre vos frères un Prophète comme moi ; vous l'écouterez dans tout ce qu'il vous dira,
\VS{23}et quiconque n’écoutera pas ce Prophète, sera exterminé du milieu du peuple\FTNT{De. 18:15-19.}.
\VS{24}Tous les prophètes qui ont parlé depuis Samuel, et ceux qui l'ont suivi, ont aussi annoncé ces jours-là.
\VS{25}Vous êtes les fils des prophètes et de l'alliance que Dieu a traitée avec nos pères, en disant à Abraham : Toutes les familles de la terre seront bénies en ta postérité\FTNT{Ge. 12:2.}.
\VS{26}C'est à vous premièrement que Dieu, ayant suscité son Fils Jésus, l'a envoyé pour vous bénir, en détournant chacun de vous de ses iniquités.
\TextTitle{[Première persécution de l'Eglise]}
\Chap{4}
\VerseOne{}Mais comme Pierre et Jean parlaient au peuple, survinrent les sacrificateurs, et le commandant du temple, et les sadducéens,
\VS{2}mécontents de ce qu'ils enseignaient le peuple, et annonçaient la résurrection des morts au Nom de Jésus.
\VS{3}Ils mirent la main sur eux, et ils les jetèrent en prison jusqu'au lendemain, parce qu'il était déjà tard.
\VS{4}Cependant, plusieurs de ceux qui avaient entendu la parole crurent ; et le nombre des personnes fut d'environ cinq mille.
\TextTitle{[Pierre et Jean convoqués au sanhédrin]}
\VS{5}Le lendemain, les chefs, les anciens et les scribes s'assemblèrent à Jérusalem ;
\VS{6}avec Anne, le souverain sacrificateur, Caïphe, Jean, Alexandre, et tous ceux qui étaient de la race des principaux sacrificateurs.
\VS{7}Ils firent placer au milieu d’eux Pierre et Jean, et leur demandèrent : Par quel pouvoir, ou au Nom de qui avez-vous fait cette guérison ?
\VS{8}Alors Pierre étant rempli du Saint-Esprit, leur dit : Chefs du peuple, et vous anciens d'Israël :
\VS{9}Puisque nous sommes jugés aujourd'hui sur un bienfait accordé à un homme impotent, afin que nous disions comment il a été guéri,
\VS{10}sachez-le tous et que tout le peuple d'Israël le sache ! C’est par le Nom de Jésus-Christ de Nazareth, que vous avez crucifié, et que Dieu a ressuscité des morts, c’est en son Nom, que cet homme qui se présente ici devant vous, a été guéri.
\VS{11}Jésus est la pierre rejetée par vous qui bâtissez, et qui est devenue la principale de l’angle\FTNT{Ps. 118:22.}.
\VS{12}Il n'y a de salut en aucun autre : Car il n'y a sous le ciel aucun autre Nom qui ait été donné aux hommes par lequel nous devions être sauvés.
\TextTitle{[Le sanhédrin interdit aux apôtres de prêcher sur le Nom de Jésus]}
\VS{13}Eux, voyant la hardiesse de Pierre et de Jean, et sachant aussi qu'ils étaient des hommes sans lettres et idiots ; s'en étonnaient, et ils reconnaissaient bien qu'ils avaient été avec Jésus.
\VS{14}Et voyant que l'homme qui avait été guéri, était présent avec eux, ils ne pouvaient contredire en rien.
\VS{15}Ils leur ordonnèrent de sortir du sanhédrin, et ils délibérèrent entre eux, disant : Que ferons-nous à ces gens ?
\VS{16}Car il est manifeste pour tous les habitants de Jérusalem, qu'un miracle a été fait par eux, et cela est si évident que nous ne pouvons le nier.
\VS{17}Mais afin qu'il ne soit plus divulgué parmi le peuple, défendons-leur avec menaces expresses, qu'ils n'aient plus à parler en ce Nom à qui que ce soit.
\VS{18}Et les ayant donc appelés, ils leur défendirent de ne plus parler ni d’enseigner en aucune manière au Nom de Jésus.
\VS{19}Mais Pierre et Jean leur répondirent : Jugez s'il est juste devant Dieu de vous obéir plutôt qu'à Dieu.
\VS{20}Car nous ne pouvons pas ne pas parler de ce que nous avons vu et entendu.
\VS{21}Alors ils les relâchèrent avec menaces, ne trouvant point comment ils pourraient les punir, à cause du peuple, parce que tous glorifiaient Dieu de ce qui avait été fait.
\VS{22}Car l'homme en qui cette miraculeuse guérison avait été faite, avait plus de quarante ans.
\TextTitle{[L'Eglise demande l'assistance de Dieu]}
\VS{23}Après avoir été relâchés, ils allèrent vers les leurs, et leur racontèrent tout ce que les principaux sacrificateurs et les anciens leur avaient dit.
\VS{24}Lorsqu’ils l’eurent entendu, ils élevèrent tous ensemble la voix à Dieu, et dirent : Seigneur ! Tu es le Dieu qui as fait le ciel et la terre, la mer, et toutes les choses qui y sont ;
\VS{25}et qui as dit par la bouche de David ton serviteur : Pourquoi ce tumulte parmi les nations, et ces vaines pensées parmi les peuples ?
\VS{26}Les rois de la terre se sont soulevés en personne, et les princes se sont ligués ensemble contre le Seigneur, et contre son Christ\FTNT{Ps. 2:1-2.}.
\VS{27}En effet, contre ton Saint Fils Jésus, que tu as oint, se sont assemblés Hérode et Ponce Pilate, avec les gentils, et le peuple d'Israël,
\VS{28}pour faire toutes les choses que ta main et ton conseil avaient auparavant déterminé qui seraient faites.
\VS{29}Maintenant donc, Seigneur, fais attention à leurs menaces, et donne à tes serviteurs d'annoncer ta parole avec toute hardiesse ;
\VS{30}en étendant ta main afin qu'il se fasse des guérisons, des prodiges, et des merveilles, par le Nom de ton Saint Fils Jésus.
\VS{31}Et quand ils eurent prié, le lieu où ils étaient assemblés trembla ; et ils furent tous remplis du Saint-Esprit, et ils annonçaient la parole de Dieu avec hardiesse.
\TextTitle{[Les chrétiens unis comme un seul corps]
\\(Ac. 2:42-47)}
\VS{32}Or la multitude de ceux qui croyaient, n'était qu'un cœur et qu'une âme ; et nul ne disait d'aucune des choses qu'il possédait, qu'elle fût à lui, mais toutes choses étaient communes entre eux.
\VS{33}Aussi les apôtres rendaient témoignage avec une grande force à la résurrection du Seigneur Jésus ; et une grande grâce était sur eux tous.
\VS{34}Car il n'y avait parmi eux aucun indigent ; parce que tous ceux qui possédaient des champs ou des maisons, les vendaient, et ils apportaient le prix des choses vendues,
\VS{35}et le mettaient aux pieds des apôtres ; et il était distribué à chacun selon qu'il en avait besoin.
\VS{36}Or Joseph, surnommé par les apôtres Barnabas, c'est-à-dire, fils de consolation, Lévite, et originaire de Chypre,
\VS{37}ayant une possession, la vendit, et en apporta le prix, et le mit aux pieds des apôtres.
\TextTitle{[Mort d'Ananias et de Saphira]}
\Chap{5}
\VerseOne{}Mais un homme appelé Ananias, et Saphira sa femme, vendit une propriété,
\VS{2}et retint une partie du prix, sa femme le sachant ; puis il apporta le reste, et le déposa aux pieds des apôtres.
\VS{3}Mais Pierre lui dit : Ananias comment Satan s'est-il emparé de ton cœur jusqu’à t'inciter à mentir au Saint-Esprit, et à soustraire une partie du prix du champ ?
\VS{4}S’il n’avait pas été vendu, ne te restait-il pas ? Et après qu’il ait été vendu, le prix n’était-il pas à ta disposition ? Comment as-tu pu mettre en ton cœur un pareil dessein ? Ce n’est pas aux hommes que tu as menti, mais à Dieu.
\VS{5}Ananias, entendant ces paroles, tomba et rendit l'âme ; ce qui causa une grande crainte à tous ceux qui en entendirent parler.
\VS{6}Et quelques jeunes hommes se levant le prirent, et l'emportèrent dehors, et l’ensevelirent.
\VS{7}Environ trois heures plus tard, sa femme entra, sans savoir ce qui était arrivé.
\VS{8}Pierre prenant la parole, lui dit : Dis-moi, est-ce à un tel prix que vous avez vendu le champ ? Et elle dit : Oui, répondit-elle, c’est à ce prix-là.
\VS{9}Alors Pierre lui dit : Pourquoi avez-vous fait un complot entre vous pour tenter l'Esprit du Seigneur ? Voici à la porte les pieds de ceux qui ont enterré ton mari, et ils t'emporteront.
\VS{10}Et au même instant, elle tomba à ses pieds et rendit l'âme. Et quand les jeunes hommes furent entrés, ils la trouvèrent morte, et ils l'emportèrent dehors, et l’ensevelirent auprès de son mari.
\VS{11}Et cela donna une grande crainte à toute l'Eglise, et à tous ceux qui entendaient ces choses.
\TextTitle{[Miracles à Jérusalem]}
\VS{12}Beaucoup de prodiges et de miracles se faisaient parmi le peuple par les mains des apôtres ; et ils étaient tous d'un commun accord au portique de Salomon.
\VS{13}Cependant aucun des autres n'osait se joindre à eux, mais le peuple les louait hautement.
\VS{14}Et le nombre de ceux qui croyaient au Seigneur, hommes et femmes, augmentait de plus en plus.
\VS{15}Et on apportait les malades dans les rues, et on les mettait sur de petits lits et sur des couchettes, afin que quand Pierre viendrait, au moins son ombre passe sur quelqu'un d'eux.
\VS{16}La multitude accourait aussi des villes voisines à Jérusalem, amenant des malades, et ceux qui étaient tourmentés des esprits impurs ; et tous étaient guéris.
\TextTitle{[Deuxième persécution de l'Eglise]}
\VS{17}Alors le souverain sacrificateur se leva, lui et tous ceux qui étaient avec lui, à savoir la secte des sadducéens, et ils furent remplis de jalousie ;
\VS{18}et mettant la main sur les apôtres, ils les jetèrent dans la prison publique.
\VS{19}Mais l’Ange du Seigneur\FTNT{Voir commentaire Mt. 1:20.} ouvrit pendant la nuit les portes de la prison, les fit sortir, et leur dit :
\VS{20}Allez, et présentez-vous dans le temple, annoncez au peuple toutes les paroles de cette vie.
\VS{21}Ayant entendu cela, ils entrèrent dès le matin dans le temple, et se mirent à enseigner. Mais le souverain sacrificateur et ceux qui étaient avec lui étant arrivés, ils convoquèrent le sanhédrin et tous les anciens des fils d'Israël, et ils envoyèrent chercher les apôtres à la prison.
\VS{22}Mais, les huissiers à leur arrivée, ne les trouvèrent point dans la prison. Ils retournèrent, et firent leur rapport,
\VS{23}en disant : Nous avons trouvé la prison fermée avec toute sûreté, et les gardes aussi qui étaient devant les portes ; mais après l'avoir ouverte, nous n'avons trouvé personne dedans.
\VS{24}Lorsque le souverain sacrificateur, le commandant du temple, et les principaux sacrificateurs, eurent entendu ces paroles, ils ne savaient que penser au sujet des apôtres, ne sachant ce qui arriverait de tout cela.
\VS{25}Mais quelqu'un vint leur dire : Voici, les hommes que vous avez mis en prison sont dans le temple, et ils enseignent le peuple.
\VS{26}Alors le commandant du temple partit avec les huissiers, et il les conduisit sans violence, car ils avaient peur d'être lapidés par le peuple.
\VS{27}Après qu’ils les eurent amenés, ils les présentèrent au sanhédrin. Et le souverain sacrificateur les interrogea,
\VS{28}disant : Ne vous avons-nous pas défendu expressément d'enseigner en ce Nom-là ? Et voici, vous avez rempli Jérusalem de votre doctrine, et vous voulez faire retomber sur nous le sang de cet homme.
\TextTitle{[Les apôtres refusent d'obéir au sanhédrin]}
\VS{29}Alors Pierre et les autres apôtres répondirent : Il faut plutôt obéir à Dieu qu'aux hommes.
\VS{30}Le Dieu de nos pères a ressuscité Jésus, que vous avez fait mourir en le pendant au bois.
\VS{31}Dieu l’a élevé à sa droite comme Prince et Sauveur, afin de donner à Israël la repentance et la rémission des péchés.
\VS{32}Nous sommes témoins de ce que nous disons, de même que le Saint-Esprit que Dieu a donné à ceux qui lui obéissent.
\TextTitle{[Conseil de Gamaliel aux chefs du peuple]}
\VS{33}Mais eux, ayant entendu ces choses, grinçaient des dents, et voulaient les faire mourir.
\VS{34}Mais un pharisien nommé Gamaliel, docteur de la loi, honoré de tout le peuple, se leva dans le sanhédrin, et ordonna de faire sortir un instant les apôtres.
\VS{35}Puis il leur dit : Hommes Israélites, prenez garde à ce que vous allez faire à l’égard de ces gens.
\VS{36}Car il n’y a pas longtemps que Theudas s'éleva, se disant être quelque chose, et auquel se joignit un nombre d’environ quatre cents hommes ; mais il fut tué, et tous ceux qui s'étaient joints à lui ont été dissipés et réduits à rien.
\VS{37}Après lui parut Judas le Galiléen au temps du recensement, et il attira à lui un grand peuple ; il périt aussi, et tous ceux qui s'étaient joints à lui ont été dispersés.
\VS{38}Maintenant donc je vous dis : Ne continuez plus vos poursuites contre ces hommes, et laissez-les. Car si cette entreprise ou cette œuvre vient des hommes, elle sera détruite ;
\VS{39}mais si elle vient de Dieu, vous ne pourrez pas la détruire. Et prenez garde qu’il ne se trouve que vous combattiez contre Dieu.
\TextTitle{[Les apôtres battus de verge]}
\VS{40}Et ils furent de son avis. Et ayant appelé les apôtres, ils les firent battre de verges, ils leur défendirent de parler au Nom de Jésus, et ils les relâchèrent.
\VS{41}Et les apôtres se retirèrent de devant le sanhédrin, joyeux d'avoir été jugés dignes de subir des outrages pour le Nom de Jésus.
\VS{42}Et tous les jours, ils ne cessaient d'enseigner, et d'annoncer l’Evangile de Jésus-Christ dans le temple, et de maison en maison.
\TextTitle{[Sept hommes choisis pour le service]}
\Chap{6}
\VerseOne{}En ces jours-là, le nombre des disciples augmentant, il s'éleva un murmure des Hellénistes\FTNT{Les Hellénistes étaient des chrétiens d'origine juive venant de la diaspora, et dont les ancêtres juifs avaient émigré ou avaient été déportés de Palestine depuis longtemps vers les rives de la Méditerranée, la Mésopotamie ou ailleurs. Ils étaient de culture et de langue grecques.} contre les Hébreux, parce que leurs veuves étaient méprisées dans le service ordinaire.
\VS{2}C'est pourquoi les douze, ayant convoqué la multitude des disciples, leur dirent : Il n'est pas raisonnable que nous laissions la prédication de la parole de Dieu pour servir aux tables.
\VS{3}Choisissez donc, frères, sept hommes parmi vous, de qui l’on rende un bon témoignage, qui soient pleins du Saint-Esprit et de sagesse, et que nous chargerons de cet emploi.
\VS{4}Et nous, nous continuerons à vaquer à la prière, et au ministère de la parole.
\VS{5}Cette proposition plut à toute l'assemblée qui était là présente ; et ils élurent Etienne, homme plein de foi et du Saint-Esprit, Philippe, Prochore, Nicanor, Timon, Parménas, et Nicolas, prosélyte d'Antioche.
\VS{6}Ils les présentèrent aux apôtres ; qui, après avoir prié, leur imposèrent les mains.
\VS{7}La parole de Dieu se répandait, et le nombre des disciples augmentait beaucoup à Jérusalem ; une grande foule de sacrificateurs obéissaient à la foi.
\VS{8}Etienne, plein de foi et de puissance, faisait de grands miracles et de grands prodiges parmi le peuple.
\TextTitle{[Troisième persécution de l'Eglise ; Etienne convoqué au sanhédrin]}
\VS{9}Quelques-uns de la synagogue appelée la synagogue des Affranchis\FTNT{Affranchis : Du grec ~libertinos~, c’est-à-dire ~libertins~ : Hommes libres. Fraction de la communauté Juive qui avait sa propre synagogue à Jérusalem. Probablement des Juifs qui avaient été faits prisonniers par Pompée et d'autres généraux romains, qui avaient été déportés à Rome, puis libérés.}, de celle des Cyréniens et de celle des Alexandrins, avec ceux de Cilicie et d'Asie, se mirent à discuter avec Etienne.
\VS{10}Mais ils ne pouvaient pas résister à la sagesse et à l'Esprit par lequel il parlait.
\VS{11}Alors ils soudoyèrent des hommes qui dirent : Nous l’avons entendu proférer des paroles blasphématoires contre Moïse et contre Dieu.
\VS{12}Et ils soulevèrent le peuple, les anciens, et les scribes, et se jetant sur lui, ils l'enlevèrent et l'amenèrent au sanhédrin.
\VS{13}Et ils présentèrent de faux témoins qui dirent : Cet homme ne cesse de proférer des paroles blasphématoires contre ce saint lieu et contre la loi.
\VS{14}Car nous l’avons entendu dire que Jésus, ce Nazaréen, détruira ce lieu, et changera les coutumes que Moïse nous a données.
\VS{15}Tous ceux qui siégeaient au sanhédrin avaient les yeux fixés sur Etienne, son visage leur parut comme celui d'un ange.
\TextTitle{[Discours d'Etienne devant le sanhédrin]}
\Chap{7}
\VerseOne{}Alors le souverain sacrificateur lui dit : Ces choses sont-elles ainsi ?
\VS{2}Etienne répondit : Hommes frères et pères, écoutez-moi : Le Dieu de gloire apparut à notre père Abraham, lorsqu’il était en Mésopotamie, avant qu'il s’établisse à Charan, et lui dit :
\VS{3}Sors de ton pays, et de ta famille, et va dans le pays que je te montrerai.
\VS{4}Il sortit donc du pays des Chaldéens, et alla demeurer à Charan. De là, après la mort de son père, Dieu le fit passer dans ce pays que vous habitez maintenant.
\VS{5}Il ne lui donna aucun héritage dans ce pays, pas même de quoi poser le pied, mais il promit de lui en donner la possession, ainsi qu’à sa postérité après lui, quoiqu’il n’ait point d’enfants.
\VS{6}Dieu lui parla ainsi : Ta postérité séjournera dans une terre étrangère pendant quatre cents ans ; et on la réduira à la servitude et on la maltraitera.
\VS{7}Mais je jugerai la nation à laquelle ils auront été asservis, dit Dieu ; et après cela ils sortiront, et me serviront en ce lieu-ci\FTNT{Ge. 15:13-14.}.
\VS{8}Puis Dieu donna à Abraham l'alliance de la circoncision ; et ainsi Abraham engendra Isaac et le circoncit le huitième jour. Isaac engendra Jacob, et Jacob les douze patriarches.
\VS{9}Les patriarches, jaloux de Joseph, le vendirent pour être emmené en Egypte.
\VS{10}Mais Dieu fut avec lui et le délivra de toutes ses tribulations ; et par la sagesse qu’il lui donna, il le rendit agréable à Pharaon, roi d'Egypte, qui l'établit gouverneur d’Egypte et de toute sa maison.
\VS{11}Il survint dans tout le pays d'Egypte, et dans celui de Canaan, une famine et une grande détresse, en sorte que nos pères ne pouvaient trouver des vivres.
\VS{12}Mais Jacob apprit qu'il y avait du blé en Egypte, il y envoya une première fois nos pères.
\VS{13}Et la seconde fois, Joseph fut reconnu par ses frères, et la famille de Joseph fut déclarée à Pharaon.
\VS{14}Alors Joseph envoya chercher Jacob son père, et toute sa famille, composée de soixante-quinze personnes.
\VS{15}Jacob descendit en Egypte, où il mourut, lui et nos pères ;
\VS{16}et ils furent transportés à Sichem, et déposés dans le sépulcre qu'Abraham avait acheté à prix d'argent des fils d’Hamor, père de Sichem.
\VS{17}Le temps approchait où devait s’accomplir la promesse que Dieu avait faite avec serment à Abraham, et le peuple s’accrut et se multiplia beaucoup en Egypte.
\VS{18}Jusqu'à ce que parut en Egypte un autre roi, qui n'avait pas connu Joseph.
\VS{19}Ce roi, usant d’artifice contre notre race, maltraita nos pères jusqu'à leur faire exposer leurs enfants à l'abandon, afin d'en faire périr la race.
\VS{20}En ce temps-là naquit Moïse, qui était beau aux yeux de Dieu. Il fut nourri trois mois dans la maison de son père ;
\VS{21}et quand il eut été exposé à l'abandon, la fille de Pharaon le recueillit et l’éleva comme son fils.
\VS{22}Moïse fut instruit dans toute la sagesse des Egyptiens ; et il était puissant en paroles et en œuvres.
\VS{23}Il avait quarante ans lorsqu’il eut à cœur d'aller visiter ses frères, les fils d'Israël.
\VS{24}Et voyant l’un d’eux à qui l’on faisait tort, il le défendit, et vengea celui qui était outragé en tuant l'Egyptien.
\VS{25}Il croyait que ses frères comprendraient par là que Dieu les délivrerait par son moyen ; mais ils ne le comprirent point.
\VS{26}Et le jour suivant, il parut entre eux comme ils se querellaient, et il tâcha de les mettre d'accord en leur disant : Hommes, vous êtes frères, pourquoi vous faites-vous tort l'un à l'autre ?
\VS{27}Mais celui qui maltraitait son prochain le repoussa en disant : Qui t'a établi prince et juge sur nous ?
\VS{28}Veux-tu me tuer, comme tu as tué hier l'Egyptien ?
\VS{29}A cette parole, Moïse s'enfuit ; et il demeura comme étranger dans le pays de Madian où il eut deux fils.
\VS{30}Quarante ans plus tard, l'Ange du Seigneur lui apparut au désert de la montagne de Sinaï, dans la flamme d’un buisson en feu.
\VS{31}Et quand Moïse le vit, il fut étonné de la vision, et comme il approchait pour considérer ce que c'était, la voix du Seigneur lui fut adressée,
\VS{32}disant : Je suis le Dieu de tes pères, le Dieu d'Abraham, le Dieu d'Isaac, et le Dieu de Jacob. Et Moïse tout tremblant n'osait pas regarder.
\VS{33}Le Seigneur lui dit : Ôte tes souliers de tes pieds, car le lieu sur lequel tu te tiens est une terre sainte.
\VS{34}J'ai vu l'affliction de mon peuple qui est en Egypte, et j'ai entendu leur gémissement, et je suis descendu pour les délivrer. Maintenant donc, va, je t'enverrai en Egypte.
\VS{35}Ce Moïse, qu’ils avaient rejeté en disant : Qui t'a établi prince et juge ? C’est lui que Dieu envoya comme prince et comme libérateur par le moyen de l'Ange qui lui était apparu dans le buisson.
\VS{36}C'est celui qui les fit sortir d’Egypte, en opérant des miracles et des prodiges au pays d’Egypte, au sein de la mer Rouge, et au désert pendant quarante ans.
\VS{37}C'est ce Moïse qui a dit aux enfants d'Israël : Le Seigneur votre Dieu vous suscitera d'entre vos frères un Prophète comme moi ; écoutez-le\FTNT{De. 18:15.}.
\VS{38}C'est lui, qui, lors de l'assemblée au désert, étant avec l'Ange qui lui parlait sur la montagne de Sinaï et avec nos pères, reçut les oracles vivants pour nous les donner.
\VS{39}Nos pères ne voulurent pas lui obéir, mais ils le rejetèrent, et ils tournèrent leur cœur vers l’Egypte,
\VS{40}en disant à Aaron : Fais-nous des dieux qui marchent devant nous ; car nous ne savons point ce qui est arrivé à ce Moïse qui nous a amenés hors du pays d'Egypte.
\VS{41}Ils firent donc en ces jours-là un veau, et ils offrirent des sacrifices à l'idole, et se réjouirent de l’œuvre de leurs mains.
\VS{42}C'est pourquoi aussi Dieu se détourna d'eux, et les livra au culte de l'armée du ciel, ainsi qu'il est écrit dans le livre des prophètes : Maison d'Israël, m'avez-vous offert des sacrifices et des victimes pendant quarante ans au désert ?
\VS{43}Mais vous avez porté la tente de Moloc\FTNT{Lé. 18:21.}, et l'étoile de votre dieu Remphan ; qui sont des figures que vous avez faites pour les adorer ; c'est pourquoi je vous transporterai au-delà de Babylone.
\VS{44}Nos pères avaient au désert le tabernacle du témoignage, comme l’avait ordonné celui qui avait dit à Moïse de le faire selon le modèle qu'il avait vu.
\VS{45}Et nos pères avaient reçu ce tabernacle, ils le portèrent sous la conduite de Josué dans le pays qui était possédé par les nations que Dieu chassa de devant eux, et il y resta jusqu'aux jours de David.
\VS{46}David trouva grâce devant Dieu, et demanda de pouvoir dresser une tente pour le Dieu de Jacob.
\VS{47}Et ce fut Salomon qui lui bâtit une maison.
\VS{48}Mais le Très-Haut n'habite pas dans des temples faits de main d’homme, selon ces paroles du prophète :
\VS{49}Le ciel est mon trône, et la terre est mon marchepied : Quelle maison me bâtirez-vous, dit le Seigneur, ou quel pourrait être le lieu de mon repos ?
\VS{50}Ma main n'a-t-elle pas fait toutes ces choses\FTNT{Es. 66:1.} ?
\VS{51}Hommes au cou raide, et incirconcis de cœur et d'oreilles, vous vous obstinez toujours contre le Saint-Esprit ; vous faites comme vos pères ont fait.
\VS{52}Lequel des prophètes vos pères n'ont-ils pas persécuté ? Ils ont même tué ceux qui annonçaient d’avance l'avènement du Juste, dont vous avez été les traîtres et les meurtriers,
\VS{53}vous qui avez reçu la loi par le ministère des anges, et qui ne l'avez point gardée.
\TextTitle{[Etienne, premier martyr, meurt lapidé]}
\VS{54}En entendant ces choses, ils étaient furieux dans leur cœur, et ils grinçaient des dents contre lui.
\VS{55}Mais Etienne, rempli du Saint-Esprit, et fixant les yeux vers le ciel, vit la gloire de Dieu, et Jésus qui était à la droite de Dieu.
\VS{56}Et il dit : Voici, je vois les cieux ouverts, et le Fils de l'homme qui est à la droite de Dieu.
\VS{57}Alors ils s'écrièrent à haute voix, et bouchèrent leurs oreilles, et tous d'un commun accord se jetèrent sur lui.
\VS{58}Et l'ayant tiré hors de la ville, ils le lapidèrent ; et les témoins déposèrent leurs vêtements aux pieds d'un jeune homme nommé Saul.
\VS{59}Et ils lapidaient Etienne qui priait et disait : Seigneur Jésus ! Reçois mon esprit\FTNT{Dans Ec. 12:9, il est dit qu’à la mort, l’esprit retourne à Dieu qui l’a donné. Jésus est donc Dieu puisqu’il a reçu l’esprit d’Etienne.}.
\VS{60}Et s'étant mis à genoux, il cria à haute voix : Seigneur, ne leur impute point ce péché ! Et quand il eut dit cela, il s'endormit.
\TextTitle{[Quatrième persécution de l'Eglise : Saul persécute les saints]}
\Chap{8}
\VerseOne{}Saul avait approuvé le meurtre d'Etienne, et en ce temps-là, il y eut une grande persécution contre l'Eglise de Jérusalem. Et tous, excepté les apôtres, se dispersèrent dans les contrées de la Judée et de la Samarie.
\VS{2}Et quelques hommes pieux emportèrent Etienne pour l'ensevelir, et le pleurèrent à grand bruit.
\VS{3}Mais Saul ravageait l'Eglise, entrant dans toutes les maisons, et traînant par force hommes et femmes, il les mettait en prison.
\TextTitle{[Les premiers missionnaires]
\\(Ac. 11 : 19-21)}
\VS{4}Ceux qui avaient été dispersés allaient de lieu en lieu, annonçant la parole de Dieu.
\TextTitle{[Philippe évangélise la Samarie]}
\VS{5}Philippe, étant descendu dans la ville de Samarie, leur prêcha Christ.
\VS{6}Et les foules tout entières étaient attentives à ce que Philippe disait, l'écoutant, lorsqu’elles virent les miracles qu'il faisait.
\VS{7}Car les esprits impurs sortirent de plusieurs démoniaques, en poussant de grands cris, et beaucoup de paralytiques et de boiteux furent guéris,
\VS{8}ce qui causa une grande joie dans cette ville-là.
\VS{9}Il y avait auparavant dans la ville un homme nommé Simon qui se prenant pour un personnage important, exerçait la magie et provoquait l’étonnement du peuple de la Samarie.
\VS{10}Tous, depuis le plus petit jusqu’au plus grand étaient attentifs, et ils disaient : Celui-ci est la grande puissance de Dieu.
\VS{11}Et ils étaient attachés à lui, parce que depuis longtemps il les avait éblouis par sa magie.
\VS{12}Mais quand ils eurent cru ce que Philippe leur annonçait, l’Evangile du Royaume de Dieu, et du Nom de Jésus-Christ, hommes et femmes se firent baptiser.
\VS{13}Simon lui-même crut, et après avoir été baptisé, il ne quittait plus Philippe, et il voyait avec étonnement les prodiges et les grands miracles qui s’opéraient.
\VS{14}Les apôtres qui étaient à Jérusalem, ayant appris que la Samarie avait reçu la parole de Dieu, y envoyèrent Pierre et Jean.
\VS{15}Ceux-ci, arrivés chez les Samaritains, prièrent pour eux afin qu'ils reçoivent le Saint-Esprit.
\VS{16}Car il n'était pas encore descendu sur aucun d'eux, mais seulement ils étaient baptisés au Nom du Seigneur Jésus.
\VS{17}Alors les apôtres leur imposèrent les mains, et ils reçurent le Saint-Esprit.
\VS{18}Lorsque Simon vit que le Saint-Esprit était donné par l'imposition des mains des apôtres, il leur offrit de l'argent,
\VS{19}en leur disant : Donnez-moi aussi ce pouvoir, afin que tous ceux à qui j'imposerai les mains reçoivent le Saint-Esprit.
\VS{20}Mais Pierre lui dit : Que ton argent périsse avec toi, puisque tu as cru que le don de Dieu s'acquérait avec de l'argent.
\VS{21}Il n’y a pour toi ni part ni héritage dans cette affaire, car ton cœur n'est pas droit devant Dieu.
\VS{22}Repens-toi donc de cette méchanceté, et prie Dieu, afin que s'il est possible, la pensée de ton cœur te soit pardonnée.
\VS{23}Car je vois que tu es dans un fiel très amer, et dans un lien d'iniquité.
\VS{24}Alors Simon répondit : Priez vous-mêmes le Seigneur pour moi, afin qu’il ne m’arrive rien de ce que vous avez dit.
\VS{25}Eux donc après avoir prêché et annoncé la parole du Seigneur, retournèrent à Jérusalem et annoncèrent l'Evangile dans plusieurs villages des Samaritains.
\TextTitle{[Conversion et baptême de l'eunuque éthiopien]}
\VS{26}L’Ange du Seigneur s’adressant à Philippe, lui dit : Lève-toi et va du côté du Midi, sur le chemin qui descend de Jérusalem à Gaza, celui qui est désert.
\VS{27}Il se leva et partit. Et voici, un homme Ethiopien, un eunuque, un des principaux seigneurs de la cour de Candace, reine des Ethiopiens, surintendant de toutes ses richesses, venu à Jérusalem pour adorer,
\VS{28}s’en retournait, assis dans son char, et lisait le prophète Esaïe.
\VS{29}L'Esprit dit à Philippe : Avance, et approche-toi de ce char.
\VS{30}Philippe accourut et entendit l’Ethiopien qui lisait le prophète Esaïe ; et il lui dit : Comprends-tu ce que tu lis ?
\VS{31}Et il lui dit : Comment pourrais-je le comprendre, si quelqu'un ne me guide pas ? Et il pria Philippe de monter et s'asseoir avec lui.
\VS{32}Le passage de l'Ecriture qu'il lisait était celui-ci : Il a été mené comme une brebis à la boucherie, et comme un agneau muet devant celui qui le tond ; en sorte qu'il n'a point ouvert sa bouche.
\VS{33}Dans son humiliation, son jugement a été levé ; mais qui racontera sa durée ? Car sa vie est retranchée de la terre\FTNT{Es. 53:7-8.}.
\VS{34}Et l'eunuque prenant la parole, dit à Philippe : Je te prie, de qui est-ce que le prophète dit cela : Est-ce de lui-même, ou de quelque autre ?
\VS{35}Alors Philippe, ouvrant sa bouche, et commençant par cette Ecriture, lui annonça l’Evangile de Jésus.
\VS{36}Comme ils continuaient leur chemin, ils arrivèrent à un endroit où il y avait de l'eau. Et l'eunuque dit : Voici de l'eau, qu'est-ce qui empêche que je ne sois baptisé ?
\VS{37}Philippe dit : Si tu crois de tout ton cœur, cela t'est permis ; et l' eunuque répondit : Je crois que Jésus-Christ est le Fils de Dieu.
\VS{38}Il fit arrêter le char ; Philippe et l’eunuque descendirent tous deux dans l'eau, et Philippe le baptisa.
\VS{39}Quand ils furent sortis de l'eau, l'Esprit du Seigneur enleva Philippe, et l'eunuque ne le vit plus. Tandis que tout joyeux il continua son chemin,
\VS{40}Philippe se trouva dans Azot, d’où il alla jusqu’à Césarée, en évangélisant toutes les villes par lesquelles il passait.
\TextTitle{[Conversion de Saul]
\\(Ac. 22:1-16 ; 26:9-18)}
\Chap{9}
\VerseOne{}Cependant Saul, respirant encore la menace et le carnage contre les disciples du Seigneur, s’adressa au souverain sacrificateur,
\VS{2}et lui demanda des lettres de sa part pour les porter aux synagogues de Damas, afin que, s'il trouvait quelques-uns de cette secte, hommes ou femmes, il les amène liés à Jérusalem.
\VS{3}Comme il était en chemin, et qu’il approchait de Damas, tout à coup une lumière resplendit du ciel comme un éclair autour de lui.
\VS{4}Il tomba par terre et il entendit une voix qui lui disait : Saul, Saul, pourquoi me persécutes-tu ?
\VS{5}Il répondit : Qui es-tu, Seigneur ? Et le Seigneur lui dit : Je suis Jésus, que tu persécutes. Il te serait dur de regimber contre les aiguillons.
\VS{6}Alors, tout tremblant et tout effrayé, il dit : Seigneur, que veux-tu que je fasse ? Et le Seigneur lui dit : Lève-toi, et entre dans la ville, et on te dira ce que tu dois faire.
\VS{7}Les hommes qui l’accompagnaient s'arrêtèrent tout épouvantés, entendant bien la voix, mais ne voyant personne.
\VS{8}Et Saul se leva de terre, et ouvrant ses yeux, il ne voyait personne ; c'est pourquoi ils le conduisirent par la main, et le menèrent à Damas,
\VS{9}où il fut trois jours sans voir, sans manger ni boire.
\VS{10}Or il y avait à Damas un disciple, nommé Ananias, à qui le Seigneur dit en vision : Ananias ! Et il répondit : Me voici Seigneur.
\VS{11}Et le Seigneur lui dit : Lève-toi, va dans la rue appelée la droite, et cherche dans la maison de Judas un homme appelé Saul, de Tarse.
\VS{12}Car il prie. Or Saul avait vu en vision un homme appelé Ananias, entrant et lui imposant les mains, afin qu'il recouvre la vue.
\VS{13}Et Ananias répondit : Seigneur ! J'ai entendu parler plusieurs fois de cet homme-là ; et combien de maux il a faits à tes saints dans Jérusalem.
\VS{14}Il a même ici le pouvoir de la part des principaux sacrificateurs, de lier tous ceux qui invoquent ton Nom.
\VS{15}Mais le Seigneur lui dit : Va, car cet homme est un instrument que j'ai choisi, pour porter mon Nom devant les gentils, devant les rois, et devant les fils d'Israël.
\VS{16}Car je lui montrerai combien il aura à souffrir pour mon Nom.
\TextTitle{[Saul rempli du Saint-Esprit]}
\VS{17}Ananias sortit ; et lorsqu’il fut arrivé dans la maison, il imposa les mains à Saul, et lui dit : Saul mon frère, le Seigneur Jésus, qui t'est apparu sur le chemin par lequel tu venais, m'a envoyé, afin que tu recouvres la vue, et que tu sois rempli du Saint-Esprit.
\TextTitle{[Saul est baptisé et évangélise Damas]}
\VS{18}Et aussitôt il tomba de ses yeux comme des écailles ; et à l'instant il recouvra la vue. Puis il se leva, et fut baptisé.
\VS{19}Et ayant mangé, il reprit ses forces. Et Saul fut quelques jours avec les disciples qui étaient à Damas.
\VS{20}Et aussitôt il prêcha dans les synagogues que Jésus était le Fils de Dieu.
\VS{21}Et tous ceux qui l'entendaient, étaient dans l’étonnement, et ils disaient : N'est-ce pas celui qui persécutait à Jérusalem ceux qui invoquaient ce Nom, et qui est venu ici exprès pour les amener liés devant les principaux sacrificateurs ?
\VS{22}Mais Saul se fortifiait de plus en plus, et confondait les Juifs qui habitaient à Damas, prouvant que Jésus était le Christ.
\TextTitle{[Paul fuit Damas et se rend à Jérusalem]}
\VS{23}Longtemps après, les Juifs conspirèrent ensemble pour le faire mourir ;
\VS{24}et leur complot parvint à la connaissance de Saul. Ils gardaient les portes jour et nuit, afin de le faire mourir.
\VS{25}Mais pendant une nuit, les disciples le prirent, et le descendirent par la muraille dans une corbeille.
\VS{26}Lorsqu’il se rendit à Jérusalem, Saul tâcha de se joindre aux disciples ; mais tous le craignaient, ne croyant pas qu'il fût un disciple.
\VS{27}Alors Barnabas l’ayant pris avec lui, le conduisit vers les apôtres, et leur raconta comment sur le chemin Saul avait vu le Seigneur, qui lui avait parlé, et comment à Damas il avait prêché franchement au Nom de Jésus.
\VS{28}Il allait et venait avec eux à Jérusalem, et parlant avec hardiesse au Nom du Seigneur Jésus.
\VS{29}Il parlait aussi et discutait avec les Hellénistes ; mais ils cherchaient à lui ôter la vie.
\TextTitle{[Saul retourne à Tarse]}
\VS{30}Les frères l’ayant découvert, l’emmenèrent à Césarée, et le firent partir à Tarse.
\VS{31}Les Eglises étaient en paix dans toute la Judée, la Galilée, et la Samarie, étant édifiées et marchant dans la crainte du Seigneur ; et elles s’accroissaient par l’assistance du Saint-Esprit.
\TextTitle{[Guérison d'Enée, le paralytique]}
\VS{32}Comme Pierre les visitait toutes, il descendit aussi vers les saints qui habitaient à Lydde.
\VS{33}Il y trouva un homme appelé Enée, qui était couché dans un petit lit depuis huit ans, car il était paralytique.
\VS{34}Et Pierre lui dit : Enée, Jésus-Christ te guérit ! Lève-toi et arrange ton lit. Et aussitôt il se leva.
\VS{35}Tous ceux qui habitaient à Lydde et à Saron le virent, et ils se convertirent au Seigneur.
\TextTitle{[Résurrection de Tabitha]}
\VS{36}Il y avait à Joppé une femme disciple, appelée Tabitha, qui signifie en grec Dorcas ; elle faisait beaucoup de bonnes œuvres et d'aumônes.
\VS{37}Elle tomba malade en ce temps-là, et mourut. Après l’avoir lavée, on la déposa dans une chambre haute.
\VS{38}Comme Lydde était près de Joppé, les disciples ayant appris que Pierre était à Lydde, ils envoyèrent vers lui deux hommes, pour le prier de venir chez eux sans tarder.
\VS{39}Pierre se leva, et partit avec ces hommes. Lorsqu’il fut arrivé, on le conduisit dans la chambre haute. Toutes les veuves l’entourèrent en pleurant, et lui montrèrent les tuniques et les vêtements que faisait Dorcas quand elle était avec elles.
\VS{40}Pierre fit sortir tout le monde, se mit à genoux, et pria ; puis se tournant vers le corps, il dit : Tabitha, lève-toi. Et elle ouvrit ses yeux, et voyant Pierre, elle s’assit.
\VS{41}Il lui donna la main, et la fit lever ; puis ayant appelé les saints et les veuves, il la leur présenta vivante.
\VS{42}Cela fut connu dans tout Joppé ; et plusieurs crurent au Seigneur.
\VS{43}Pierre demeura plusieurs jours à Joppé, chez un corroyeur nommé Simon.
\TextTitle{[L'Evangile parvient aux gentils]}
\Chap{10}
\VerseOne{}Il y avait à Césarée un homme, nommé Corneille, centenier d'une cohorte de la légion appelée Italienne.
\VS{2}Cet homme était pieux et craignait Dieu avec toute sa famille. Il faisait aussi beaucoup d'aumônes au peuple, et priait Dieu continuellement.
\VS{3}Vers la neuvième heure du jour, il vit clairement dans une vision un ange de Dieu qui entra chez lui, et qui lui dit : Corneille !
\VS{4}Corneille ayant les yeux fixés sur lui, et tout effrayé, lui dit : Qu'y a-t-il Seigneur ? Et il lui dit : Tes prières et tes aumônes sont montées devant Dieu, et il s’en est souvenu.
\VS{5}Maintenant donc envoie des gens à Joppé, et fais venir Simon, surnommé Pierre.
\VS{6}Il est logé chez un certain Simon, corroyeur, qui a sa maison près de la mer ; c'est lui qui te dira ce qu'il faut que tu fasses.
\VS{7}Dès que l'ange qui lui parlait fut parti, Corneille appela deux de ses serviteurs, et un soldat craignant Dieu, d'entre ceux qui se tenaient près de lui.
\VS{8}Et après leur avoir tout raconté, il les envoya à Joppé.
\TextTitle{[Vision de Pierre : une nappe descend du ciel]}
\VS{9}Le lendemain, comme ils marchaient et qu'ils approchaient de la ville, Pierre monta sur le toit, vers la sixième heure, pour prier.
\VS{10}Il eut faim, et il voulut prendre son repas. Pendant qu’on lui préparait à manger, il tomba en extase.
\VS{11}Il vit le ciel ouvert, et un vase descendant sur lui semblable à une grande nappe, attachée par les quatre coins, qui descendait vers la terre,
\VS{12}où se trouvaient tous les quadrupèdes, les bêtes sauvages, les reptiles et les oiseaux du ciel.
\VS{13}Et une voix lui dit : Pierre, lève-toi, tue, et mange.
\VS{14}Mais Pierre répondit : Non Seigneur, car je n’ai jamais rien mangé de souillé ni d’impur.
\VS{15}Et la voix lui dit encore pour la seconde fois : Les choses que Dieu a purifiées, ne les tiens point pour souillées.
\VS{16}Et cela arriva jusqu’à trois fois, et puis le vase fut retiré au ciel.
\VS{17}Comme Pierre ne savait pas en lui-même que penser du sens de la vision qu’il avait eue, voici, les hommes envoyés par Corneille s’étant mis en quête de la maison de Simon, se présentèrent à la porte,
\VS{18}et demandèrent à haute voix si c’était là que logeait Simon, surnommé Pierre.
\VS{19}Et comme Pierre pensait à la vision, l'Esprit lui dit : Voici trois hommes qui te demandent.
\VS{20}Lève-toi donc, et descends, et pars avec eux sans hésiter, car c'est moi qui les ai envoyés.
\VS{21}Pierre donc, descendit vers les gens qui lui avaient été envoyés par Corneille et leur dit : Voici, je suis celui que vous cherchez ; pour quel sujet êtes-vous venus ?
\VS{22}Et ils dirent : Corneille, centenier, homme juste et craignant Dieu, et à qui toute la nation des Juifs rend un bon témoignage, a été averti de Dieu par un saint ange de te faire venir dans sa maison et d’entendre tes paroles.
\TextTitle{[Pierre se rend à Césarée]}
\VS{23}Alors Pierre les fit entrer, et les logea. Le lendemain il s'en alla avec eux, et quelques-uns des frères de Joppé l’accompagnèrent.
\VS{24}Ils arrivèrent à Césarée le jour suivant. Corneille les attendait, et avait invité ses parents et ses amis.
\VS{25}Lorsque Pierre entra, Corneille qui était allé au-devant de lui, se jeta à ses pieds, et se prosterna.
\VS{26}Mais Pierre le releva en lui disant : Lève-toi, moi aussi je suis un homme.
\VS{27}Et s’entretenant avec lui, il entra et trouva plusieurs personnes réunies.
\VS{28}Vous savez, leur dit-il, qu’il n’est pas permis à un Juif de se lier avec un étranger ou d'aller chez lui, mais Dieu m'a appris à ne regarder aucun homme comme souillé et impur.
\VS{29}C'est pourquoi, ayant été appelé, je suis venu sans difficulté. Je vous demande donc pour quel sujet vous m'avez fait venir.
\VS{30}Corneille lui dit : Il y a quatre jours, à cette heure-ci, j'étais en jeûne et en prière dans ma maison, et tout à coup, un homme, vêtu d’un habit resplendissant, se présenta devant moi et me dit :
\VS{31}Corneille, ta prière est exaucée, et Dieu s'est souvenu de tes aumônes.
\VS{32}Envoie donc quelqu’un à Joppé, et fais venir Simon, surnommé Pierre, qui est logé dans la maison de Simon le corroyeur, près de la mer. Quand il sera venu, il te parlera.
\VS{33}Aussitôt j’ai envoyé quelqu’un vers toi, et tu as bien fait de venir. Maintenant donc nous sommes tous présents devant Dieu pour entendre tout ce que Dieu t'a ordonné de nous dire.
\TextTitle{[Pierre évangélise les gentils : Corneille]
\\(Ac. 2:14-41)}
\VS{34}Alors Pierre prenant la parole, dit : En vérité je reconnais que Dieu n'a point égard à l'apparence des personnes,
\VS{35}mais qu'en toute nation celui qui le craint et qui pratique la justice, lui est agréable.
\VS{36}C'est ce qu'il a fait entendre aux enfants d'Israël, en leur annonçant la paix par Jésus-Christ, qui est le Seigneur de tous.
\VS{37}Vous savez ce qui est arrivé dans toute la Judée, après avoir commencé en Galilée, à la suite du baptême que Jean a prêché ;
\VS{38}vous savez comment Dieu a oint du Saint-Esprit et de force Jésus de Nazareth, qui allait de lieu en lieu, faisant du bien et guérissant tous ceux qui étaient sous l’empire du diable, car Dieu était avec lui.
\VS{39}Nous sommes témoins de toutes les choses qu'il a faites, dans le pays des Juifs et à Jérusalem. Cependant ils l'ont fait mourir en le pendant au bois.
\VS{40}Dieu l'a ressuscité le troisième jour, et il a permis qu’il apparaisse,
\VS{41}non à tout le peuple, mais aux témoins choisis d’avance par Dieu, à nous, qui avons mangé et bu avec lui après qu'il fut ressuscité des morts.
\VS{42}Et il nous a ordonné de prêcher au peuple, et d’attester que c'est lui qui a été établi par Dieu Juge des vivants et des morts.
\VS{43}Tous les prophètes rendent de lui le témoignage que quiconque croit en lui, reçoit la rémission de ses péchés par son Nom.
\TextTitle{[Le Saint-Esprit accordé aux gentils]}
\VS{44}Comme Pierre prononçait encore ce discours, le Saint-Esprit descendit sur tous ceux qui écoutaient la parole.
\VS{45}Tous les fidèles circoncis qui étaient venus avec Pierre, furent étonnés de ce que le don du Saint-Esprit était aussi répandu sur les gentils.
\VS{46}Car ils les entendaient parler diverses langues et glorifier Dieu.
\VS{47}Alors Pierre prenant la parole, dit : Quelqu’un pourrait-il empêcher qu’on baptise dans l’eau ceux qui ont reçu le Saint-Esprit aussi bien que nous ?
\VS{48}Et il ordonna qu'ils soient baptisés au Nom du Seigneur. Après cela, ils le prièrent de rester quelques jours auprès d’eux.
\TextTitle{[Les apôtres et les frères de Judée font des reproches à Pierre]}
\Chap{11}
\VerseOne{}Or les apôtres et les frères qui étaient en Judée, apprirent que les gentils aussi avaient reçu la parole de Dieu.
\VS{2}Et quand Pierre fut monté à Jérusalem, ceux de la circoncision disputaient contre lui,
\VS{3}disant : Tu es entré chez des hommes incirconcis, et tu as mangé avec eux.
\VS{4}Alors Pierre commençant leur exposa le tout par ordre, disant :
\VS{5}J’étais dans la ville de Joppé, et pendant que je priais, je tombai en extase et j’eus une vision : Un vase semblable à une grande nappe attachée par les quatre coins, descendit du ciel, et vint jusqu'à moi.
\VS{6}Les regards fixés sur cette nappe, j’examinai, et je vis les quadrupèdes, les bêtes sauvages, les reptiles, et les oiseaux du ciel.
\VS{7}Et j’entendis une voix qui me disait : Pierre, lève-toi, tue, et mange.
\VS{8}Et je répondis : Non Seigneur, car jamais rien de souillé ni d’impur n’est entré dans ma bouche.
\VS{9}La voix me parla du ciel une seconde fois : Ce que Dieu a déclaré pur, ne le regarde pas comme souillé.
\VS{10}Cela arriva jusqu'à trois fois, puis toutes ces choses furent retirées dans le ciel.
\VS{11}Et voici, aussitôt trois hommes qui avaient été envoyés de Césarée vers moi, se présentèrent à la maison où j'étais.
\VS{12}L’Esprit me dit de partir avec eux sans hésiter. Les six frères que voici m’accompagnèrent, et nous entrâmes dans la maison de Corneille.
\VS{13}Cet homme nous raconta comment il avait vu dans sa maison un ange qui s'était présenté à lui, et lui avait dit : Envoie des gens à Joppé, et fais venir Simon, surnommé Pierre,
\VS{14}qui te dira des choses par lesquelles tu seras sauvé, toi et toute ta maison.
\VS{15}Lorsque je me fus mis à parler, le Saint-Esprit descendit sur eux, comme il était descendu sur nous au commencement.
\VS{16}Et je me souvins de cette parole du Seigneur, et comment il avait dit : Jean a baptisé d'eau, mais vous, vous serez baptisés du Saint-Esprit.
\VS{17}Or, puisque Dieu leur a accordé le même don qu'à nous qui avons cru au Seigneur Jésus-Christ, pouvais-je, moi, m'opposer à Dieu ?
\VS{18}Après avoir entendu ces choses, ils s'apaisèrent, et ils glorifièrent Dieu en disant : Dieu a donc accordé la repentance aussi aux gentils, afin qu’ils aient la vie.
\TextTitle{[L'Eglise d'Antioche ; les disciples appelés chrétiens]}
\VS{19}Ceux qui avaient été dispersés par la persécution survenue à cause d'Etienne, allèrent jusqu'en Phénicie, dans l’île de Chypre, et à Antioche\FTNT{Antioche : Capitale de la Syrie située sur le fleuve Oronte, fondée en 300 av. J.-C., et ainsi nommée en l'honneur de son fondateur Antiochus. De nombreux Juifs grecs y vivaient et c'est là que les disciples de Christ furent appelés pour la première fois, chrétiens.}, n’annonçant la parole à personne, seulement aux Juifs.
\VS{20}Mais il y eut parmi eux quelques hommes de Chypre et de Cyrène, qui étant venus à Antioche, parlèrent aussi aux Grecs, et leur annoncèrent l’Evangile du Seigneur Jésus.
\VS{21}La main du Seigneur était avec eux, et un grand nombre de personnes crurent et se convertirent au Seigneur.
\VS{22}Le bruit en parvint aux oreilles de l'Eglise qui était à Jérusalem, et ils envoyèrent Barnabas jusqu’à Antioche.
\VS{23}Lorsqu’il fut arrivé, et qu’il eut vu la grâce de Dieu, il s'en réjouit, et il les exhortait tous à demeurer attachés au Seigneur de tout leur cœur.
\VS{24}Car c’était un homme de bien, plein du Saint-Esprit et de foi. Et un grand nombre de personnes se joignirent au Seigneur.
\VS{25}Barnabas s'en alla à Tarse pour chercher Saul ;
\VS{26}et l'ayant trouvé, il l’amena à Antioche. Pendant toute une année, ils se réunirent aux assemblées de l’Eglise, et ils enseignèrent beaucoup de personnes. Ce fut à Antioche que, pour la première fois, les disciples furent appelés chrétiens.
\TextTitle{[Prophétie d'Agabus]}
\VS{27}En ce temps-là, quelques prophètes descendirent de Jérusalem à Antioche.
\VS{28}L’un d'eux, nommé Agabus, se leva et déclara par l'Esprit qu'une grande famine devait arriver sur toute la terre. Elle arriva, en effet, sous Claude César.
\VS{29}Les disciples résolurent d’envoyer, chacun selon ses moyens, quelque secours pour subvenir aux besoins des frères qui habitaient la Judée.
\VS{30}Ils le firent parvenir aux anciens par les mains de Barnabas et de Saul.
\TextTitle{[Cinquième persécution de l'Eglise ; Pierre en prison]}
\Chap{12}
\VerseOne{}En ce même temps, le roi Hérode se mit à maltraiter quelques membres de l'Eglise ;
\VS{2}et il fit mourir par l'épée Jacques, frère de Jean.
\VS{3}Voyant que cela était agréable aux Juifs, il fit aussi arrêter Pierre. C’était pendant les jours des pains sans levain.
\VS{4}Après l’avoir saisi et jeté en prison, il le mit sous la garde de quatre bandes, de quatre soldats chacune, avec l’intention de le faire comparaître devant le peuple après la fête de Pâque.
\TextTitle{[Pierre miraculeusement délivré de prison]}
\VS{5}Pierre était donc gardé dans la prison ; mais l'Eglise faisait sans cesse des prières à Dieu pour lui.
\VS{6}La nuit qui précéda le jour où Hérode devait l’envoyer au supplice, Pierre dormait entre deux soldats, lié de deux chaînes ; et les gardes qui étaient devant la porte gardaient la prison.
\VS{7}Et voici, l’Ange du Seigneur survint, et une lumière resplendit dans la prison. L’Ange réveilla Pierre en le frappant au côté, et en disant : Lève-toi promptement ! Et les chaînes tombèrent de ses mains.
\VS{8}Et l'Ange lui dit : Mets ta ceinture et tes sandales. Et il fit ainsi. L’Ange lui dit encore : Enveloppe-toi de ton manteau et suis-moi.
\VS{9}Pierre sortit et le suivit, ne sachant pas que ce qui se faisait par l’Ange était réel, car il croyait qu’il avait une vision.
\VS{10}Lorsqu’ils eurent passé la première et la seconde garde, ils arrivèrent à la porte de fer qui mène à la ville, et qui s’ouvrit d’elle-même devant eux ; et ils sortirent et s’avancèrent dans une rue. Et subitement, l'Ange quitta Pierre.
\VS{11}Revenu à lui-même, Pierre dit : Je vois à présent d’une manière certaine que le Seigneur a envoyé son Ange, et qu'il m'a délivré de la main d'Hérode, et de toute l'attente du peuple Juif.
\VS{12}Après avoir réfléchi, il alla à la maison de Marie, mère de Jean, surnommé Marc, où plusieurs personnes étaient assemblées et priaient.
\VS{13}Il frappa à la porte du vestibule, une servante, appelée Rhode, vint pour écouter.
\VS{14}Elle reconnut la voix de Pierre, et dans sa joie elle n'ouvrit pas la porte du vestibule, mais elle courut dans la maison et annonça que Pierre était devant la porte.
\VS{15}Ils lui dirent : Tu es folle. Mais elle affirma que ce qu'elle disait était vrai.
\VS{16}Et ils dirent : C’est son ange. Cependant Pierre continuait à frapper. Et quand ils eurent ouvert, ils le virent, et furent étonnés de le voir.
\VS{17}Pierre, leur ayant fait signe de se taire de la main, leur raconta comment le Seigneur l'avait fait sortir de la prison, et il leur dit : Annoncez ces choses à Jacques et aux frères. Puis sortant de là il s'en alla dans un autre lieu.
\VS{18}Quand il fit jour, les soldats furent dans une grande agitation, pour savoir ce que Pierre était devenu.
\VS{19}Hérode s’étant mis à sa recherche et ne l’ayant pas trouvé, fit faire le procès aux gardes, et donna l’ordre de les mener au supplice. Puis il descendit de la Judée à Césarée où il séjourna.
\TextTitle{[Mort d'Hérode]}
\VS{20}Hérode avait le dessein de faire la guerre aux Tyriens et aux Sidoniens ; mais ils vinrent le trouver d'un commun accord ; et ayant gagné Blaste, son Chambellan, ils demandèrent la paix, parce que leur pays tirait sa subsistance de celui du roi.
\VS{21}A un jour marqué, Hérode, revêtu de ses habits royaux, s'assit sur son trône et les harangua publiquement.
\VS{22}Le peuple s'écria : Voix d'un dieu et non point d'un homme !
\VS{23}Et à l'instant l’Ange du Seigneur le frappa, parce qu'il n'avait pas donné gloire à Dieu. Et il expira, rongé des vers.
\VS{24}Cependant la parole de Dieu se répandait de plus en plus, et le nombre des disciples augmentait.
\VS{25}Barnabas et Saul, après s’être acquittés de leur ministère, s’en retournèrent de Jérusalem, ayant aussi pris avec eux Jean, surnommé Marc.
\TextTitle{[PREMIER VOYAGE MISSIONNAIRE]}
\TextTitle{[Saul et Barnabas mis à part par le Saint-Esprit]}
\Chap{13}
\VerseOne{}Il y avait dans l'Eglise d’Antioche des prophètes et des docteurs, Barnabas, Siméon, appelé Niger, Lucius le Cyrénien, Manahen, qui avait été élevé avec Hérode le tétrarque, et Saul.
\VS{2}Comme ils servaient le Seigneur dans leur ministère et qu'ils jeûnaient, le Saint-Esprit dit : Mettez-moi à part Barnabas et Saul pour l’œuvre à laquelle je les ai appelés.
\VS{3}Alors après avoir jeûné et prié, ils leur imposèrent les mains, et les laissèrent partir.
\TextTitle{[Paul et Barnabas évangélise l'île de Chypre et Paphos, Bar-Jésus aveuglé et conversion du proconsul Sergius Paulus]}
\VS{4}Barnabas et Saul, envoyés par le Saint-Esprit, descendirent à Séleucie, et de là ils s’embarquèrent pour l’île de Chypre.
\VS{5}Et lorsqu’ils furent à Salamine, ils annoncèrent la parole de Dieu dans les synagogues des Juifs ; ils avaient Jean avec eux pour les aider.
\TextTitle{[Bar-Jésus s'oppose à l'Evangile]}
\VS{6}Ayant ensuite traversé l'île jusqu'à Paphos, ils trouvèrent là un certain magicien, faux prophète Juif, nommé Bar-Jésus,
\VS{7}qui était avec le proconsul Sergius Paulus, homme intelligent qui fit appeler Barnabas et Saul, désirant entendre la parole de Dieu.
\VS{8}Mais Elymas, le magicien, car c'est ce que signifie ce nom d'Elymas, leur résistait, cherchant à détourner de la foi le proconsul.
\VS{9}Alors Saul, appelé aussi Paul, rempli du Saint-Esprit, fixa les yeux sur lui et dit :
\VS{10}Ô homme plein de toute fraude et de toute ruse, fils du diable, ennemi de toute justice, ne cesseras-tu point de renverser les voies droites du Seigneur ?
\VS{11}C'est pourquoi, voici la main du Seigneur est sur toi, tu seras aveugle, et pour un temps tu ne verras pas le soleil. Aussitôt l’obscurité et des ténèbres tombèrent sur lui, et il cherchait, en tâtonnant, des personnes pour le guider.
\VS{12}Alors le proconsul voyant ce qui était arrivé, crut, étant rempli d'admiration pour la doctrine du Seigneur.
\VS{13}Et quand Paul et ceux qui étaient avec lui furent partis de Paphos, ils vinrent à Perge, ville de Pamphylie. Jean se sépara d’eux et retourna à Jérusalem.
\TextTitle{[Paul prêche à Antioche de Pisidie]}
\VS{14}De Perge, ils poursuivirent leur route, et arrivèrent à Antioche, ville de Pisidie\FTNT{Antioche de Pisidie : Ville de Pisidie (en Turquie), à la frontière de Phrygie, fondée par Seleucus Nicanor. Elle devint une colonie romaine et fut aussi appelée Césarée.}, et étant entrés dans la synagogue le jour du sabbat, ils s'assirent.
\VS{15}Après la lecture de la loi et des prophètes, les chefs de la synagogue leur envoyèrent dire : Hommes frères, si vous avez quelque parole d'exhortation pour le peuple, dites-la.
\VS{16}Alors Paul s'étant levé, et ayant fait signe de la main qu'on fasse silence, dit : Hommes Israélites, et vous qui craignez Dieu, écoutez.
\VS{17}Le Dieu de ce peuple d'Israël a choisi nos pères. Il a distingué glorieusement ce peuple pendant son séjour au pays d’Egypte, et il l’en fit sortir par son bras élevé.
\VS{18}Il les supporta\FTNT{Le verbe supporter vient du grec ~tropophoreo~ qui signifie supporter les manières, endurer le caractère de quelqu'un.} au désert environ quarante ans.
\VS{19}Et ayant détruit sept nations au pays de Canaan, il leur distribua le pays par le sort.
\VS{20}Après cela, durant quatre cent cinquante ans, il leur donna des juges, jusqu'à Samuel le prophète.
\VS{21}Puis ils demandèrent un roi, et Dieu leur donna Saül fils de Kis, homme de la tribu de Benjamin ; et ainsi se passèrent quarante ans.
\VS{22}Et Dieu l'ayant rejeté, il leur suscita pour roi David, auquel il a rendu ce témoignage : J’ai trouvé David fils d’Isaï, homme selon mon cœur, qui exécutera toute ma volonté.
\VS{23}C’est de la postérité de David que Dieu, selon sa promesse, a suscité Jésus pour être le Sauveur d’Israël.
\VS{24}Avant la venue de Jésus, Jean avait prêché le baptême de repentance à tout le peuple d'Israël.
\VS{25}Et lorsque Jean achevait sa course, il disait : Je ne suis pas celui que vous pensez ; je ne suis point le Christ ; mais voici, après moi vient celui dont je ne suis pas digne de délier le soulier de ses pieds.
\VS{26}Hommes frères, fils de la race d'Abraham, et vous qui craignez Dieu, c'est à vous que la parole de ce salut a été envoyée.
\VS{27}Car les habitants de Jérusalem et leurs chefs ont méconnu Jésus, et en le condamnant, ils ont accompli les paroles des prophètes qui se lisent chaque sabbat.
\VS{28}Quoiqu'ils n’aient rien trouvé en lui qui soit digne de mort, ils demandèrent à Pilate de le faire mourir.
\VS{29}Et après qu'ils eurent accompli toutes les choses qui avaient été écrites de lui, ils le descendirent du bois, et le déposèrent dans un sépulcre.
\VS{30}Mais Dieu l'a ressuscité des morts.
\VS{31}Il est apparu pendant plusieurs jours à ceux qui étaient montés avec lui de Galilée à Jérusalem, et qui sont ses témoins devant le peuple.
\VS{32}Et nous, nous vous annonçons cette bonne nouvelle que la promesse faite à nos pères,
\VS{33}Dieu l'a accomplie pour nous, leurs enfants, en ressuscitant Jésus, selon qu'il est écrit dans le deuxième psaume : Tu es mon Fils, je t'ai aujourd'hui engendré\FTNT{Ps. 2:7.}.
\VS{34}Et pour montrer qu'il l'a ressuscité des morts, pour ne plus devoir retourner au sépulcre, il a dit ainsi : Je vous donnerai les grâces saintes promises à David, ces grâces qui sont assurées.
\VS{35}C'est pourquoi il a dit aussi dans un autre endroit : Tu ne permettras point que ton Saint voie la corruption\FTNT{Ps. 16:10.}.
\VS{36}Or David, après avoir servi en son temps au dessein de Dieu, est mort, a été réuni à ses pères, et a vu la corruption.
\VS{37}Mais celui que Dieu a ressuscité n'a pas vu la corruption.
\VS{38}Sachez donc, hommes frères, que c'est par lui que la rémission des péchés vous est annoncée,
\VS{39}et que quiconque croit est justifié par lui, de tout ce dont vous n'avez pas pu être justifiés par la loi de Moïse.
\VS{40}Prenez donc garde qu'il ne vous arrive ce qui est dit dans les prophètes :
\VS{41}Voyez, vous les arrogants, soyez étonnés et disparaissez : Car je vais faire une œuvre en votre temps, une œuvre que vous ne croiriez pas si quelqu'un vous la racontait.
\VS{42}Lorsqu’ils sortirent de la synagogue des Juifs, les gentils les prièrent de parler le sabbat suivant sur les mêmes choses.
\VS{43}Et quand l'assemblée fut séparée, beaucoup de Juifs et de prosélytes craignant Dieu, suivirent Paul et Barnabas qui les exhortèrent à persévérer dans la grâce de Dieu.
\TextTitle{[Les Juifs d'Antioche s'opposent à Paul qui se tourne vers les gentils]
\\(Ac. 18:6 ; 28:25)}
\VS{44}Le sabbat suivant, presque toute la ville s'assembla pour entendre la parole de Dieu.
\VS{45}Mais les Juifs voyant toute cette foule, furent remplis de jalousie, et ils s’opposaient à ce que Paul disait, en le contredisant et en blasphémant.
\VS{46}Alors Paul et Barnabas leur dirent avec assurance : C’est à vous premièrement qu'il fallait annoncer la parole de Dieu, mais puisque vous la rejetez, et que vous vous jugez vous-mêmes indignes de la vie éternelle, voici, nous nous tournons vers les gentils.
\VS{47}Car ainsi nous l’a ordonné le Seigneur : Je t'ai établi pour être la lumière des gentils, pour porter le salut jusqu’aux extrémités de la terre.
\VS{48}Les gentils en entendant cela, se réjouissaient et ils glorifiaient la parole du Seigneur ; et tous ceux qui étaient destinés à la vie éternelle crurent.
\VS{49}Ainsi la parole du Seigneur se répandait dans tout le pays.
\VS{50}Mais les Juifs excitèrent quelques femmes dévotes et distinguées, et les principaux de la ville, et ils provoquèrent une persécution contre Paul et Barnabas, et les chassèrent de leur territoire.
\VS{51}Paul et Barnabas secouèrent contre eux la poussière de leurs pieds et allèrent à Icone,
\VS{52}tandis que les disciples étaient remplis de joie et du Saint-Esprit.
\TextTitle{[Paul et Barnabas à Icone]}
\Chap{14}
\VerseOne{}A Icone, Paul et Barnabas entrèrent ensemble dans la synagogue des Juifs, et ils parlèrent d'une telle manière qu'une grande multitude de Juifs et de Grecs crurent.
\VS{2}Mais ceux des Juifs qui furent rebelles, émurent et irritèrent les esprits des gentils contre les frères.
\VS{3}Ils restèrent cependant assez longtemps à Icone, parlant avec assurance du Seigneur, qui rendait témoignage à la parole de sa grâce, en faisant par leurs mains des prodiges et des miracles.
\VS{4}La population de la ville fut partagée en deux, et les uns étaient du côté des Juifs, et les autres du côté des apôtres.
\TextTitle{[Paul prêche à Derbe et à Lystre ; guérison d'un boiteux de naissance]}
\VS{5}Et comme il se faisait une émeute des gentils et des Juifs, avec leurs principaux chefs, pour outrager et lapider les apôtres,
\VS{6}Paul et Barnabas en ayant eu connaissance, se réfugièrent dans les villes de Lycaonie, à Lystre, à Derbe, et dans les contrées d'alentour.
\VS{7}Et ils y annoncèrent l'Evangile.
\VS{8}A Lystre, se tenait assis un homme impotent des pieds, boiteux dès sa naissance, et qui n'avait jamais marché.
\VS{9}Cet homme écoutait parler Paul. Et Paul fixant ses yeux sur lui, et voyant qu'il avait la foi pour être guéri,
\VS{10}lui dit à haute voix : Lève-toi droit sur tes pieds. Et il se leva en sautant, et marcha.
\VS{11}Et les gens qui étaient là assemblés, ayant vu ce que Paul avait fait, élevèrent leur voix, disant en langue lycaonienne : Les dieux sous une forme humaine, sont descendus vers nous.
\VS{12}Et ils appelaient Barnabas Jupiter, et Paul Mercure, parce que c'était lui qui portait la parole.
\VS{13}Le prêtre de Jupiter, qui était à l’entrée de leur ville, ayant amené des taureaux et des couronnes jusqu'à l'entrée de la porte voulait, de même que la foule, offrir un sacrifice.
\VS{14}Mais les apôtres Barnabas et Paul ayant appris cela, déchirèrent leurs vêtements et se précipitèrent au milieu de la foule,
\VS{15}en s’écriant : Ô hommes, pourquoi faites-vous cela ? Nous aussi, nous sommes des hommes, sujets aux mêmes infimités que vous, et vous apportant l’Evangile, nous vous exhortons à renoncer à ces choses vaines, pour vous convertir au Dieu vivant, qui a fait le ciel et la terre, la mer, et tout ce qui s’y trouve.
\VS{16}Ce Dieu, dans les siècles passés, a laissé toutes les nations marcher dans leurs voies,
\VS{17}quoiqu'il n’ait cessé de rendre témoignage de ce qu’il est, en faisant du bien, en nous dispensant du ciel les pluies et les saisons fertiles, en nous donnant la nourriture avec abondance, et en remplissant nos cœurs de joie.
\VS{18}A peine purent-ils, par ces paroles, empêcher la foule de leur offrir un sacrifice.
\TextTitle{[Paul lapidé à Lystre]}
\VS{19}Alors survinrent quelques Juifs d'Antioche et d'Icone qui gagnèrent la foule, et qui après avoir lapidé Paul, le traînèrent hors de la ville, croyant qu'il était mort.
\VS{20}Mais les disciples s'étant assemblés autour de lui, il se leva et entra dans la ville ; et le lendemain il s'en alla avec Barnabas à Derbe.
\TextTitle{[Vote et établissement des anciens dans les églises]}
\VS{21}Quand ils eurent évangélisé cette ville, et fait un certain nombre de disciples, ils retournèrent à Lystre, à Icone, et à Antioche ;
\VS{22}fortifiant l'esprit des disciples, et les exhortant à persévérer dans la foi, disant que c'est par beaucoup de tribulations qu'il nous faut entrer dans le Royaume de Dieu.
\VS{23}Après le vote à main levée des assemblées, ils établirent des anciens dans chaque Eglise, et après avoir prié et jeûné, ils les recommandèrent au Seigneur, en qui ils avaient cru.
\VS{24}Traversant ensuite la Pisidie, ils allèrent en Pamphylie,
\VS{25}annoncèrent la parole à Perge, et descendirent à Attalie.
\TextTitle{[Retour à Antioche]}
\VS{26}De là, ils s’embarquèrent pour Antioche, d'où ils avaient été recommandés à la grâce de Dieu, pour l’œuvre qu'ils venaient d’accomplir.
\VS{27}Et quand ils furent arrivés, ils convoquèrent l’Eglise, et ils racontèrent toutes les choses que Dieu avait faites par eux, et comment il avait ouvert aux gentils la porte de la foi.
\VS{28}Et ils demeurèrent assez longtemps avec les disciples.
\TextTitle{[Discussion entre les chrétiens de l'Eglise de Jérusalem sur la question de l'observation ou non des rites mosaïque par les chrétiens issus des nations]}
\TextTitle{[Des hommes venus de Judée veulent imposer la circoncision]}
\Chap{15}
\VerseOne{}Quelques hommes qui étaient descendus de Judée, enseignaient les frères en disant : Si vous n'êtes pas circoncis selon le rite de Moïse, vous ne pouvez pas être sauvés.
\TextTitle{[Paul et Barnabas montent à Jérusalem]}
\VS{2}Paul et Barnabas eurent avec eux un débat et une vive discussion ; et les frères décidèrent que Paul et Barnabas, avec quelques-uns des leurs, monteraient à Jérusalem vers les apôtres et les anciens, pour traiter cette question.
\VS{3}Après avoir été accompagnés par l’assemblée, ils traversèrent la Phénicie et la Samarie, racontant la conversion des gentils ; et ils causèrent une grande joie à tous les frères.
\VS{4}Arrivés à Jérusalem, ils furent reçus par l'Eglise, les apôtres et les anciens, et ils racontèrent toutes les choses que Dieu avait faites par leur moyen.
\VS{5}Mais quelques-uns, de la secte des pharisiens qui avaient cru, se levèrent en disant qu'il fallait circoncire les gentils et leur ordonner de garder la loi de Moïse.
\TextTitle{[Réfutation de Pierre]}
\VS{6}Alors les apôtres et les anciens se réunirent pour examiner cette affaire.
\VS{7}Et après une grande discussion, Pierre se leva et leur dit : Hommes frères, vous savez que depuis longtemps Dieu m’a choisi parmi nous, afin que par ma bouche, les gentils entendent la parole de l'Evangile, et qu'ils croient.
\VS{8}Et Dieu, qui connaît les cœurs, leur a rendu témoignage en leur donnant le Saint-Esprit, de même qu'à nous.
\VS{9}Il n'a fait aucune différence entre nous et eux, ayant purifié leurs cœurs par la foi.
\VS{10}Maintenant donc pourquoi tentez-vous Dieu en voulant imposer aux disciples un joug que ni nos pères ni nous n'avons pu porter ?
\VS{11}Mais nous croyons que nous serons sauvés par la grâce du Seigneur Jésus-Christ, comme eux aussi.
\TextTitle{[Témoignages de Paul et de Barnabas]}
\VS{12}Alors toute l'assemblée garda le silence, et l’on écouta Barnabas et Paul qui racontèrent tous les miracles et les prodiges que Dieu avait faits par leur moyen au milieu des gentils.
\TextTitle{[Discours de Jacques]} 
\VS{13}Lorsqu’ils eurent cessé de parler, Jacques prit la parole et dit : Hommes frères, écoutez-moi !
\VS{14}Simon a raconté comment Dieu a premièrement jeté les regards sur les nations pour choisir du milieu d’elles un peuple consacré à son Nom. 
\VS{15}Et avec cela s'accordent les paroles des prophètes, selon qu'il est écrit :
\VS{16}Après cela, je reviendrai, et je rebâtirai le tabernacle de David qui est tombé, je réparerai ses ruines et je le relèverai\FTNT{Am. 9:11.}.
\VS{17}Afin que le reste des hommes recherche le Seigneur, et toutes les nations aussi sur lesquelles mon Nom est invoqué, dit le Seigneur, qui fait toutes ces choses.
\VS{18}Toutes les œuvres de Dieu lui sont connues de toute éternité.
\TextTitle{[Les chrétiens issus des nations ne sont pas soumis à la loi mosaïque]}
\VS{19}C'est pourquoi je suis d'avis qu’on ne crée pas des difficultés à ceux des gentils qui se convertissent à Dieu ;
\VS{20}mais qu’on leur écrive de s’abstenir des souillures des idoles et de la fornication, des animaux étouffés et du sang.
\VS{21}Car depuis bien des générations, Moïse a dans chaque ville des gens qui le prêchent, puisqu’on le lit tous les jours de sabbat dans les synagogues.
\VS{22}Alors il parut bon aux apôtres et aux anciens avec toute l'Eglise, de choisir parmi eux et d'envoyer à Antioche avec Paul et Barnabas, Jude, appelé Barsabas, et Silas, hommes considérés entre les frères.
\VS{23}Ils écrivirent par eux en ces termes : Les apôtres, les anciens, et les frères, aux frères d'entre les gentils qui sont à Antioche, en Syrie, et en Cilicie, salut.
\VS{24}Ayant appris que quelques hommes partis de chez nous, et auxquels nous n’avons donné aucun ordre, vous ont troublés par leurs discours et ont ébranlé vos âmes, en vous disant qu’il faut être circoncis et garder la loi,
\VS{25}nous avons été d'avis, étant assemblés tous d'un commun accord, d'envoyer vers vous, avec nos très chers Barnabas et Paul, des hommes que nous avons choisis ;
\VS{26}et qui sont des hommes qui ont abandonné leurs vies pour le Nom de notre Seigneur Jésus-Christ.
\VS{27}Nous avons donc envoyé Jude et Silas, qui vous feront entendre les mêmes choses de vive voix.
\TextTitle{[les gentils devenus croyant ne doivent pas scandaliser les Juifs]}
\VS{28}Car il a paru bon au Saint-Esprit et à nous, de ne vous imposer d’autre charge que ce qui est nécessaire,
\VS{29}savoir, de vous abstenir des viandes sacrifiées aux idoles, du sang, des animaux étouffés, et de la fornication ; choses contre lesquelles vous vous trouverez bien de vous tenir en garde. Adieu.
\TextTitle{[Mission de Jude et Silas à Antioche]}
\VS{30}Après avoir donc pris congé de l’Eglise, ils allèrent à Antioche, et ayant assemblé l'Eglise, ils remirent la lettre.
\VS{31}Après l’avoir lue, les frères d’Antioche furent réjouis de la consolation qu’elle leur apportait.
\VS{32}Jude et Silas, qui étaient eux-mêmes prophètes, exhortèrent les frères par plusieurs discours, et les fortifièrent.
\VS{33}Au bout de quelque temps, les frères les laissèrent en paix retourner vers les apôtres qui les avaient envoyés.
\VS{34}Toutefois Silas trouva bon de rester.
\VS{35}Et Paul et Barnabas demeurèrent aussi à Antioche, enseignant et annonçant, avec plusieurs autres, la parole du Seigneur.
\TextTitle{[DEUXIEME VOYAGE MISSIONNAIRE]}
\TextTitle{[Paul et Barnabas se séparent]}
\VS{36}Quelques jours après, Paul dit à Barnabas : Retournons visiter nos frères dans toutes les villes où nous avons annoncé la parole du Seigneur, pour voir quel est leur état.
\VS{37}Barnabas voulait emmener avec eux Jean, surnommé Marc.
\VS{38}Mais Paul jugea plus convenable de ne pas prendre avec eux celui qui les avait quittés depuis la Pamphylie, et qui ne les avait point accompagnés dans leur œuvre.
\VS{39}Il y eut donc entre eux une contestation, en sorte qu’ils se séparèrent l’un de l’autre. Barnabas, prenant Marc avec lui, s’embarqua pour l’île de Chypre.
\VS{40}Mais Paul ayant choisi Silas pour l'accompagner, partit après avoir été recommandé à la grâce de Dieu par les frères.
\VS{41}Il traversa la Syrie et la Cilicie, fortifiant les Eglises.
\TextTitle{[Timothée, circoncis, rejoint l'équipe de missionnaire]}
\Chap{16}
\VerseOne{}Il se rendit à Derbe et à Lystre, et voici, il y avait là un disciple, nommé Timothée, fils d'une femme Juive fidèle et d'un père Grec.
\VS{2}Les frères de Lystre et d’Icone rendaient de lui un bon témoignage.
\VS{3}C'est pourquoi Paul voulut l’emmener avec lui ; et l'ayant pris, il le circoncit, à cause des Juifs qui étaient dans ces lieux-là, car ils savaient tous que son père était Grec.
\VS{4}En passant par les villes, ils recommandaient aux frères d’observer les ordonnances établies par les apôtres et les anciens de Jérusalem.
\VS{5}Ainsi les Eglises étaient affermies dans la foi et augmentaient en nombre chaque jour.
\TextTitle{[La vision de Paul]}
\VS{6}Ayant traversé la Phrygie et le pays de Galatie, le Saint-Esprit leur défendit d'annoncer la parole dans l’Asie.
\VS{7}Arrivés près de la Mysie, ils se disposaient à entrer en Bithynie ; mais l'Esprit de Jésus\FTNT{Le Saint-Esprit est aussi appelé l’Esprit de Jésus. Il ne faut donc pas le séparer du Seigneur pour créer une troisième personne afin d’abonder dans le sens du dogme trinitaire. Dieu est UN.} ne le leur permit pas.
\VS{8}Ils traversèrent ensuite la Mysie, et descendirent à Troas.
\VS{9}Pendant la nuit, Paul eut une vision d’un homme macédonien qui se présenta devant lui, et le pria, disant : Passe en Macédoine et secours-nous !
\VS{10}Après cette vision de Paul, nous cherchâmes aussitôt à nous rendre en Macédoine, concluant que le Seigneur nous appelait à les évangéliser.
\TextTitle{[Paul à Philippes]}
\VS{11}Ainsi étant partis de Troas, nous fîmes voile directement vers la Samothrace, et le lendemain à Néapolis.
\VS{12}De là nous allâmes à Philippes, qui est la première ville d’un district de Macédoine, et une colonie romaine. Nous séjournâmes quelque temps dans la ville.
\VS{13}Le jour du sabbat nous nous rendîmes hors de la ville, vers une rivière, où nous pensions que se trouvait un lieu de prière. Nous nous assîmes, et nous parlâmes aux femmes qui étaient assemblées.
\TextTitle{[Conversion de Lydie]}
\VS{14}L’une d’elles, appelée Lydie, marchande de pourpre, de la ville de Thyatire, était une femme craignant Dieu, et elle nous écoutait. Le Seigneur lui ouvrit le cœur, afin qu'elle soit attentive à ce que disait Paul.
\VS{15}Lorsqu’elle eut été baptisée, avec sa famille, elle nous fit cette demande : Si vous me jugez fidèle au Seigneur, entrez dans ma maison, et demeurez-y. Et elle nous pressa par ses instances.
\TextTitle{[L'esprit de python chassé ; Paul et Silas sont battus de vierge et mis en prison]}
\VS{16}Comme nous allions à la prière, une servante qui avait un esprit de python, et qui en devinant apportait un grand profit à ses maîtres, nous rencontra,
\VS{17}et elle se mit à nous suivre, Paul et nous, en criant et disant : Ces hommes sont les serviteurs du Dieu Très-Haut, et ils vous annoncent la voie du salut.
\VS{18}Elle fit cela pendant plusieurs jours. Mais Paul, fatigué, se retourna et dit à l'esprit : Je t’ordonne au Nom de Jésus-Christ de sortir de cette fille. Et il sortit au même instant.
\VS{19}Mais les maîtres de la servante voyant disparaître l’espoir de leur gain, se saisirent de Paul et de Silas, et les traînèrent sur la place publique devant les magistrats.
\VS{20}Ils les présentèrent aux préteurs en disant : Ces hommes, qui sont Juifs, troublent notre ville :
\VS{21}Car ils annoncent des coutumes qu'il ne nous est pas permis de recevoir ni de suivre, à nous qui sommes Romains.
\VS{22}La foule se souleva aussi contre eux, et les préteurs, ayant fait déchirer leurs vêtements, ordonnèrent qu'ils soient battus de verges.
\VS{23}Après qu’on les eut chargés de coups de fouet, ils les mirent en prison, en recommandant au geôlier de les garder sûrement.
\VS{24}Le geôlier ayant reçu cet ordre, les jeta dans la prison intérieure, et leur mit les ceps aux pieds.
\TextTitle{[Conversion du geôlier]}
\VS{25}Vers minuit, Paul et Silas priaient et chantaient les louanges de Dieu, et les prisonniers les entendaient.
\VS{26}Tout à coup, il se fit un grand tremblement de terre, en sorte que les fondements de la prison furent ébranlés ; au même instant, toutes les portes s'ouvrirent et les liens de tous furent rompus.
\VS{27}Le geôlier se réveilla, et voyant les portes de la prison ouvertes, il tira son épée et allait se tuer, croyant que les prisonniers s’étaient enfuis.
\VS{28}Mais Paul cria d’une voix forte : Ne te fais point de mal, nous sommes tous ici.
\VS{29}Alors le geôlier, ayant demandé de la lumière, entra précipitamment dans le cachot, et se jeta tout tremblant aux pieds de Paul et de Silas ;
\TextTitle{[La foi, indispensable au salut]}
\VS{30}il les fit sortir, et dit : Seigneur, que faut-il que je fasse pour être sauvé ?
\VS{31}Paul et Silas répondirent : Crois au Seigneur Jésus-Christ et tu seras sauvé, toi et ta famille.
\VS{32}Et ils lui annoncèrent la parole du Seigneur, et à tous ceux qui étaient dans sa maison.
\VS{33}Il les prit avec lui, à cette heure même de la nuit, il lava leurs plaies, et aussitôt après il fut baptisé, avec tous ceux de sa maison.
\VS{34}Les ayant amenés dans sa maison, il leur servit à manger, et il se réjouit avec toute sa famille de ce qu’il avait cru en Dieu.
\TextTitle{[Paul et Silas relâchés]}
\VS{35}Quand il fit jour, les préteurs envoyèrent des huissiers pour dire au geôlier : Relâche ces hommes.
\VS{36}Et le geôlier rapporta ces paroles à Paul, disant : Les préteurs ont envoyé dire qu'on vous relâche ; maintenant donc sortez, et allez en paix.
\VS{37}Mais Paul dit aux huissiers : Après nous avoir battus de verges publiquement et sans jugement, nous qui sommes Romains, ils nous ont jetés en prison, et maintenant ils nous font sortir secrètement ! Il n'en sera pas ainsi. Qu’ils viennent eux-mêmes nous mettre en liberté.
\VS{38}Les huissiers rapportèrent ces paroles aux préteurs qui furent effrayés en apprenant qu'ils étaient Romains.
\VS{39}Ils vinrent vers eux et leur firent des excuses, et ils les mirent en liberté en les priant de quitter la ville.
\VS{40}Quand ils furent sortis de la prison, ils entrèrent chez Lydie, et après avoir vu et consolé les frères, ils partirent.
\TextTitle{[Mission de Paul à Thessalonique]}
\Chap{17}
\VerseOne{}Paul et Silas passèrent par Amphipolis et par Apollonie, et ils arrivèrent à Thessalonique, où les Juifs avaient une synagogue.
\VS{2}Paul y entra, selon sa coutume. Pendant trois sabbats, il discuta avec eux d’après les Ecritures ;
\VS{3}expliquant et établissant que le Christ devait souffrir et ressusciter des morts. Et ce Jésus, que je vous annonce, disait-il, c’est lui qui est le Christ.
\VS{4}Quelques-uns d'entre eux crurent, et se joignirent à Paul et à Silas, ainsi qu’une grande multitude de Grecs craignant Dieu, et beaucoup de femmes de qualité.
\TextTitle{[Emeute à Thessalonique]}
\VS{5}Mais les Juifs rebelles et jaloux, prirent avec eux quelques hommes méchants et fainéants de la populace, provoquèrent des attroupements, et répandirent l’agitation dans la ville. Ils se rendirent à la maison de Jason, et ils cherchèrent Paul et Silas, pour les amener vers le peuple.
\VS{6}Ne les ayant pas trouvés, ils traînèrent Jason et quelques frères devant les magistrats de la ville, en criant : Ces gens qui ont bouleversé le monde sont aussi venus ici, et Jason les a reçus chez lui.
\VS{7}Ils sont tous rebelles aux édits de César, disant qu'il y a un autre Roi, qu'ils nomment Jésus.
\VS{8}Ils soulevèrent donc le peuple et les magistrats de la ville, qui entendant ces choses,
\VS{9}ne laissèrent aller Jason et les autres qu’après avoir obtenu d’eux une caution. Aussitôt les frères firent partir de nuit Paul et Silas pour Bérée. Lorsqu’ils furent arrivés, ils entrèrent dans la synagogue des Juifs.
\TextTitle{[Paul et Silas à Bérée]}
\VS{10}Au même instant les frères firent partir de nuit hors de la ville Paul et Silas, pour aller à Bérée, où étant arrivés ils entrèrent dans la synagogue des Juifs.
\VS{11}Ces Juifs avaient des sentiments plus nobles que ceux de Thessalonique ; ils reçurent la parole avec beaucoup de promptitude, et ils examinaient tous les jours les Ecritures, pour voir si ce qu’on leur disait était exact.
\VS{12}Plusieurs d'entre eux crurent, ainsi que des femmes Grecques de distinction, et des hommes en assez grand nombre.
\VS{13}Mais quand les Juifs de Thessalonique surent que Paul annonçait aussi à Bérée la parole de Dieu, ils vinrent y agiter la foule.
\VS{14}Alors les frères firent aussitôt partir Paul du côté de la mer ; Silas et Timothée restèrent à Bérée.
\VS{15}Ceux qui avaient pris la charge de mettre Paul en sûreté, le conduisirent jusqu'à Athènes. Puis ils s’en retournèrent, après avoir reçu l’ordre de Paul de dire à Silas et à Timothée de le rejoindre au plus tôt.
\TextTitle{[Paul à Athènes]}
\VS{16}Comme Paul les attendait à Athènes, il sentit au-dedans son esprit s’irriter à la vue de cette ville entièrement adonnée à l'idolâtrie.
\VS{17}Il s’entretenait donc dans la synagogue avec les Juifs et les hommes craignant Dieu, et tous les jours sur la place publique avec ceux qui s'y rencontraient.
\VS{18}Quelques philosophes épicuriens\FTNT{L’épicurisme a été fondé par Epicure (341 av. J.-C.- 270 av. J.-C.). Cette philosophie est axée sur la recherche du bonheur par l’évitement de la souffrance et des inquiétudes (ataraxie). Pour l’épicurien, seules les sensations immédiates de plaisir et de déplaisir sont vraies, tout le reste est vain et inexistant. Le «~quadruple remède~» des épicuriens affirme que : - «~les dieux ne sont pas à craindre~» parce que leur félicité divine les rend étrangers à l'envie ou à la colère. - «~la mort n'est rien pour nous~» parce qu'elle est la destruction de la sensation : je ne sentirai jamais ma propre mort, elle n'est donc rien pour moi, puisque la sensation est l'origine de toute chose. - «~le plaisir n'est pas susceptible de degré~», il ne faut donc pas regretter d'être mortel : un homme immortel ne connaîtrait pas plus de plaisir pour autant. - «~la douleur ne dure pas~» : le bien est donc facile à obtenir, et les maux faciles à écarter.} et stoïciens\FTNT{Les stoïciens étaient disciples de Zénon (336-264 av. J.-C.). Leur philosophie se fondait sur la conception d’un homme se suffisant à lui-même, sur une discipline rigoureuse, et sur la solidarité du genre humain.} se mirent à parler avec lui. Et les uns disaient : Que veut dire ce discoureur ? Les autres disaient : Il semble qu’il annonce des divinités étrangères ; parce qu'il leur annonçait Jésus et la résurrection.
\VS{19}Alors ils le prirent et le menèrent à l'Aréopage\FTNT{A l’origine, l’aréopage désignait le tribunal d'Athènes qui siégeait sur la colline d'Arès. Le sens figuré est le suivant : Assemblée de juges, de savants, d'hommes de lettres très compétents.}, et lui dirent : Pourrions-nous savoir quelle est cette nouvelle doctrine que tu enseignes ?
\VS{20}Car tu nous remplis les oreilles de certaines choses étranges ; nous voudrions donc savoir ce que veulent dire ces choses.
\VS{21}Or tous les Athéniens et les étrangers qui demeuraient à Athènes, ne passaient leur temps qu'à dire ou à écouter des nouvelles.
\TextTitle{[Prédication de Paul à l'Aréopage]}
\VS{22}Paul, debout au milieu de l'Aréopage, leur dit : Hommes Athéniens, je vous trouve à tous égards extrêmement religieux.
\VS{23}Car en passant et en regardant vos divinités, j'ai même trouvé un autel sur lequel était écrit : Au Dieu inconnu ! Celui que vous révérez sans le connaître, c'est celui que je vous annonce.
\VS{24}Le Dieu qui a fait le monde et tout ce qui s’y trouve, étant le Seigneur du ciel et de la terre, n'habite point dans des temples faits de main d’homme.
\VS{25}Il n'est point servi par les mains des hommes, comme s'il avait besoin de quoi que ce soit, lui qui donne à tous la vie, la respiration, et toutes choses.
\VS{26}Il a fait que tous les hommes, sortis d’un seul sang, habitent sur toute l'étendue de la terre, ayant déterminé la durée des temps et les bornes de leur habitation.
\VS{27}Il a voulu qu'ils cherchent le Seigneur, et qu’ils s’efforcent de le trouver en tâtonnant, quoiqu'il ne soit pas loin de chacun de nous,
\VS{28}car c’est par lui que nous avons la vie, le mouvement et l'être. C’est ce qu’ont dit quelques-uns même de vos poètes : De lui nous sommes la race.
\VS{29}Ainsi donc, étant de la race de Dieu, nous ne devons pas croire que la divinité soit semblable à de l'or, ou à de l'argent, ou à de la pierre taillée par l'art et l'industrie des hommes.
\VS{30}Mais Dieu, sans tenir compte du temps d’ignorance, annonce maintenant à tous les hommes en tous lieux qu'ils se repentent,
\VS{31}parce qu'il a arrêté un jour où il jugera le monde selon la justice, par l'homme qu'il a établi pour cela, ce dont il a donné à tous une preuve certaine, en le ressuscitant des morts.
\VS{32}Lorsqu’ils entendirent parler de la résurrection des morts, les uns se moquèrent, et les autres dirent : Nous t'entendrons là-dessus une autre fois.
\VS{33}Ainsi Paul se retira du milieu d'eux.
\VS{34}Quelques-uns néanmoins se joignirent à lui et crurent : Denys, juge de l’Aéropage, une femme nommée Damaris, et d’autres avec eux.
\TextTitle{[Paul enseigne à Corinthe un an et demi]}
\Chap{18}
\VerseOne{}Après cela, Paul partit d'Athènes, et se rendit à Corinthe.
\VS{2}Il y trouva un Juif, nommé Aquilas, originaire du Pont, récemment arrivé d'Italie, avec Priscille sa femme, parce que Claude avait ordonné à tous les Juifs de sortir de Rome. Il s’approcha d’eux,
\VS{3}et comme il était du même métier qu’eux, il demeura chez eux et y travailla. Et leur métier était de faire des tentes.
\VS{4}Paul discourait dans la synagogue chaque sabbat, et il persuadait des Juifs et des Grecs.
\VS{5}Quand Silas et Timothée furent arrivés de Macédoine, Paul étant poussé par l'Esprit, rendait témoignage aux Juifs que Jésus était le Christ.
\VS{6}Mais comme ils s’opposaient à lui et qu'ils blasphémaient, il secoua ses vêtements et leur dit : Que votre sang retombe sur votre tête ! J’en suis pur ! Dès maintenant, j’irai vers les gentils.
\VS{7}Et sortant de là, il entra dans la maison d'un homme appelé Justus, homme craignant Dieu, et dont la maison était contiguë à la synagogue.
\VS{8}Cependant Crispus, le chef de la synagogue, crut au Seigneur avec toute sa famille. Et plusieurs Corinthiens qui avaient entendu Paul, crurent aussi, et ils furent baptisés.
\VS{9}Le Seigneur dit à Paul dans une vision pendant la nuit : Ne crains point, mais parle et ne te tais point,
\VS{10}parce que je suis avec toi, et personne ne mettra la main sur toi pour te faire du mal. Parle, car j'ai un peuple nombreux dans cette ville.
\VS{11}Il y demeura un an et six mois, enseignant parmi eux la parole de Dieu.
\TextTitle{[Indifférence de Gallion]}
\VS{12}Pendant que Gallion était proconsul de l’Achaïe, les Juifs se soulevèrent d'un commun accord contre Paul, et le menèrent devant le tribunal
\VS{13}en disant : Cet homme incite les gens à servir Dieu d’une manière contraire à la loi.
\VS{14}Et comme Paul voulait ouvrir la bouche pour parler, Gallion dit aux Juifs : Ô Juifs ! S’il s’agissait de quelque injustice, ou de quelque crime, je vous écouterais patiemment, autant qu’il serait raisonnable.
\VS{15}Mais il s’agit de discussions sur une parole, sur des noms, et sur votre loi, vous y mettrez de l’ordre vous-mêmes, car je ne veux pas être juge de ces choses.
\VS{16}Et il les renvoya du tribunal.
\VS{17}Alors tous les Grecs se saisirent de Sosthène, le chef de la synagogue, le battirent devant le tribunal, sans que Gallion s'en mette en peine.
\TextTitle{[Paul fait un voeu]
\\(Ro. 6:14 ; 2 Co. 3:7-14 ; Ga. 3:23-28)}
\VS{18}Paul resta encore assez longtemps à Corinthe. Ensuite, il prit congé des frères et s’embarqua pour la Syrie, avec Priscille et Aquilas, après s’être fait raser la tête à Cenchrées, car il avait fait un vœu.
\VS{19}Ils arrivèrent à Ephèse, et Paul y laissa ses compagnons. Etant entré dans la synagogue, il s’entretint avec les Juifs,
\VS{20}qui le prièrent de rester encore plus longtemps avec eux.
\VS{21}Mais il n’y consentit point, et il prit congé d'eux en leur disant : Il faut absolument que je célèbre la fête prochaine à Jérusalem. Je reviendrai vers vous, s'il plaît à Dieu. Ainsi il partit d'Ephèse.
\VS{22}Etant débarqué à Césarée, il monta à Jérusalem, et après avoir salué l'Eglise, il descendit à Antioche.
\TextTitle{[TROISIEME VOYAGE MISSIONNAIRE]}
\VS{23}Lorsqu’il eut passé quelque temps à Antioche, Paul parcourut de suite la Galatie et la Phrygie, fortifiant tous les disciples.
\TextTitle{[Apollos annonce l'Evangile à Ephèse et à Corinthe]}
\VS{24}En ce temps-là, un Juif, nommé Apollos, originaire d’Alexandrie, homme éloquent et puissant dans les Ecritures, vint à Ephèse.
\VS{25}Il était en quelque sorte instruit dans la voie du Seigneur, et fervent d’esprit, il expliquait et enseignait avec exactitude ce qui concerne Jésus, bien qu’il ne connaisse que le baptême de Jean.
\VS{26}Il se mit à parler avec hardiesse dans la synagogue. Aquilas et Priscille l’ayant entendu, le prirent avec eux et lui exposèrent plus exactement la voie de Dieu.
\VS{27}Et comme il voulut passer en Achaïe, les frères qui l'y encouragèrent écrivirent aux disciples de bien le recevoir. Quand il fut arrivé, il se rendit, par la grâce de Dieu, très utile à ceux qui avaient cru.
\VS{28}Car il réfutait publiquement les Juifs avec une grande véhémence, démontrant par les Ecritures que Jésus était le Christ.
\TextTitle{[Paul à Ephèse]}
\Chap{19}
\VerseOne{}Pendant qu’Apollos était à Corinthe, Paul, après avoir parcouru toutes les hautes provinces de l’Asie, arriva à Ephèse. Ayant rencontré quelques disciples, il leur dit :
\VS{2}Avez-vous reçu le Saint-Esprit quand vous avez cru ? Ils lui répondirent : Nous n'avons même pas entendu dire qu’il y ait un Saint-Esprit.
\VS{3}Et il leur dit : De quel baptême donc avez-vous été baptisés ? Ils répondirent : Du baptême de Jean.
\VS{4}Alors Paul dit : Il est vrai que Jean a baptisé du baptême de repentance, disant au peuple de croire en celui qui venait après lui, c'est-à-dire en Jésus-Christ.
\VS{5}Après avoir entendu ces choses, ils furent baptisés au nom du Seigneur Jésus.
\VS{6}Lorsque Paul leur eut imposé les mains, le Saint-Esprit descendit sur eux, et ils parlaient diverses langues et prophétisaient.
\VS{7}Ils étaient en tout environ douze hommes.
\TextTitle{[Paul enseigne à Ephèse]
\\(v. 9-10 ; Ac. 20:31)}
\VS{8}Ensuite, Paul entra dans la synagogue où il parla librement. Pendant trois mois, il discourut sur les choses qui concernent le Royaume de Dieu avec persuasion.
\VS{9}Mais comme quelques-uns restaient endurcis et rebelles, décriant devant la multitude la voie du Seigneur, il se retira d’eux, sépara les disciples, et enseigna tous les jours dans l'école d'un nommé Tyrannus.
\VS{10}Cela dura deux ans, de sorte que tous ceux qui habitaient l’Asie, Juifs et Grecs, entendirent la parole du Seigneur Jésus.
\TextTitle{[Réveil et Miracles à Ephèse]}
\VS{11}Et Dieu faisait des prodiges extraordinaires par les mains de Paul,
\VS{12}au point qu’on appliquait sur les malades des mouchoirs ou des linges qui avaient touché son corps, et ils étaient guéris de leurs maladies, et les esprits malins sortaient.
\VS{13}Alors quelques exorcistes Juifs ambulants essayèrent d'invoquer le Nom du Seigneur Jésus sur ceux qui étaient possédés d’esprits malins, en disant : Nous vous conjurons par ce Jésus que Paul prêche !
\VS{14}Ceux qui faisaient cela étaient sept fils de Scéva, un homme Juif, l’un des principaux sacrificateurs
\VS{15}Mais l’esprit malin leur répondit : Je connais Jésus, et je sais qui est Paul ; mais vous, qui êtes-vous ?
\VS{16}Et l'homme dans lequel était l’esprit malin se jeta sur eux, se rendit maître de deux d’entre eux, et les maltraita de telle sorte qu’ils s'enfuirent de cette maison nus et blessés.
\VS{17}Cela fut connu de tous les Juifs et de tous les Grecs qui demeuraient à Ephèse ; et ils furent tous saisis de crainte, et le Nom du Seigneur Jésus était glorifié.
\VS{18}Plusieurs de ceux qui avaient cru venaient, confessant et déclarant ce qu'ils avaient fait.
\VS{19}Et un grand nombre de ceux qui s’étaient adonnés à des pratiques magiques, apportèrent leurs livres et les brûlèrent devant tous : On en estima la valeur à cinquante mille pièces d'argent.
\VS{20}C’est ainsi que la parole du Seigneur se répandait sensiblement, et produisait de grands effets.
\VS{21}Après que ces choses se furent passées, Paul se proposa par un mouvement de l'Esprit\FTNT{Paul fut conduit par le Saint-Esprit (Jn 3:8).} d'aller à Jérusalem, en traversant la Macédoine et l’Achaïe. Quand j’y serai allé, se disait-il, il faut aussi que je voie Rome.
\VS{22}Il envoya en Macédoine deux de ceux qui l'assistaient, Timothée et Eraste, et il resta lui-même quelque temps en Asie.
\TextTitle{[Emeute excitée par l'orfèvre Démétrius à Ephèse]}
\VS{23}Or il arriva en ce temps-là un grand trouble au sujet de la doctrine.
\VS{24}Car un certain homme, nommé Démétrius, orfèvre, fabriquait de petits temples d'argent de Diane, et apportait beaucoup de profit aux ouvriers du métier.
\VS{25}Il les rassembla, avec ceux du même métier, et dit : Ô hommes, vous savez que tout notre gain vient de cet ouvrage,
\VS{26}et vous voyez et entendez que, non seulement à Ephèse, mais dans presque toute l'Asie, ce Paul a persuadé et détourné beaucoup de monde, en disant que les dieux faits de main d’hommes ne sont pas des dieux.
\VS{27}Le danger qui en résulte, ce n’est pas seulement que notre métier ne soit décrié, mais que le temple de la grande déesse Diane ne tombe dans le mépris, et que sa majesté, que toute l'Asie et le monde entier adorent, ne soit réduite à néant.
\VS{28}Ayant entendu ces choses, ils furent tous remplis de colère, et ils se mirent à crier : Grande est la Diane des Ephésiens !
\VS{29}Toute la ville fut remplie de confusion. Ils se précipitèrent tous ensemble au théâtre, entrainant avec eux Gaïus et Aristarque, Macédoniens, compagnons de voyage de Paul.
\VS{30}Paul voulait se présenter devant le peuple, mais les disciples l’en empêchèrent.
\VS{31}Quelques-uns même des Asiarques, qui étaient ses amis, envoyèrent quelqu’un vers lui pour le prier de ne pas se présenter au théâtre.
\VS{32}Les uns criaient d'une manière, les autres d'une autre, car l'assemblée était confuse, et la plupart ne savaient pas pourquoi ils s’étaient assemblés.
\VS{33}Alors Alexandre fut contraint de sortir de la foule, les Juifs le poussant en avant ; et Alexandre, faisant signe de la main, voulait présenter quelque excuse au peuple.
\VS{34}Mais quand ils reconnurent qu'il était Juif, tous d’une seule voix crièrent pendant deux heures : Grande est la Diane des Ephésiens !
\VS{35}Cependant, le secrétaire de la ville, ayant apaisé la foule, dit : Hommes Ephésiens, quel est celui des hommes qui ignore que la ville d’Ephèse est la gardienne de la grande déesse Diane et de son image tombée de Jupiter\FTNT{Tombée de Jupiter : C’est-à-dire du ciel.} ?
\VS{36}Cela étant donc incontestable, vous devez vous apaiser et ne rien faire avec précipitation.
\VS{37}Car ces gens que vous avez amenés ne sont ni sacrilèges ni blasphémateurs de votre déesse.
\VS{38}Mais si Démétrius et ses ouvriers ont à se plaindre de quelqu’un, il y a des jours d’audience et des proconsuls ; qu’ils s’appellent en justice les uns les autres.
\VS{39}Et si vous avez quelque autre chose à réclamer, on pourra en décider dans une assemblée légale.
\VS{40}Car nous risquons d’être accusés de sédition pour ce qui s'est passé aujourd'hui, n’ayant aucune raison pour justifier ce rassemblement. Après ces paroles, il congédia l’assemblée.
\TextTitle{[Paul annonce l'Evangile en Macédoine et en Grèce]}
\Chap{20}
\VerseOne{}Lorsque le tumulte eut cessé, Paul fit venir les disciples, et après les avoir embrassés, il partit pour aller en Macédoine.
\VS{2}Il parcourut cette contrée, en adressant aux disciples de nombreuses exhortations.
\VS{3}Puis il se rendit en Grèce où il séjourna trois mois. Il était sur le point de s’embarquer pour la Syrie, quand les Juifs lui dressèrent des embûches. Alors il se décida à reprendre la route de la Macédoine.
\VS{4}Il avait pour l’accompagner jusqu’en Asie : Sopater de Bérée, Aristarque et Second de Thessalonique, Gaïus de Derbe, Timothée, ainsi que Tychique et Trophime, originaires d’Asie.
\VS{5}Ceux-ci prirent les devants et nous attendirent à Troas.
\TextTitle{[Paul à Troas pendant sept jours]}
\VS{6}Pour nous, après les jours des pains sans levain, nous nous embarquâmes à Philippes, et nous les rejoignîmes à Troas où nous séjournâmes sept jours.
\VS{7}Le premier jour de la semaine, les disciples étant assemblés pour rompre le pain, Paul, qui devait partir le lendemain, leur fit un discours qu'il étendit jusqu'à minuit.
\VS{8}Or il y avait beaucoup de lampes dans la chambre haute où ils étaient assemblés.
\VS{9}Or un jeune homme nommé Eutychus, qui était assis sur une fenêtre, s’endormit profondément pendant le long discours de Paul ; entraîné par le sommeil, il tomba du troisième étage en bas, et quand on voulut le relever, il était mort.
\VS{10}Mais Paul, étant descendu, se pencha sur lui, le prit dans ses bras, et dit : Ne vous troublez pas, car son âme est en lui.
\VS{11}Quand il fut remonté, il rompit le pain et mangea, et il parla longtemps encore jusqu'au jour. Après quoi il partit.
\VS{12}Ils ramenèrent le jeune homme vivant, et ce fut le sujet d’une grande consolation.
\TextTitle{[Voyage de Troas à Milet]}
\VS{13}Pour nous, étant montés sur un navire, nous fîmes voile vers Assos, où nous avions convenu de reprendre Paul, parce qu’il devait faire la route à pied.
\VS{14}Lorsqu’il nous eut rejoints à Assos, nous le prîmes avec nous, et nous allâmes à Mytilène.
\VS{15}Puis étant partis de là, le jour suivant nous abordâmes vis-à-vis de Chios. Le lendemain, nous arrivâmes vers Samos, et nous nous arrêtâmes à Trogyle ; le jour d’après, nous vînmes à Milet.
\VS{16}Car Paul avait résolu de passer devant Ephèse sans s’y arrêter, afin de ne pas perdre de temps en Asie ; parce qu'il se hâtait pour être, si cela lui était possible, à Jérusalem le jour de la Pentecôte.
\TextTitle{[Paul exhorte et prend congé des anciens d'Ephèse]}
\VS{17}Cependant de Milet, il envoya chercher à Ephèse les anciens de l'Eglise.
\VS{18}Lorsqu’ils furent arrivés vers lui, il leur dit : Vous savez de quelle manière je me suis toujours conduit avec vous dès le premier jour où je suis entré en Asie ;
\VS{19}servant le Seigneur en toute humilité, avec beaucoup de larmes, et au milieu des épreuves que me suscitaient les embûches des Juifs.
\VS{20}Vous savez que je n’ai rien caché de ce qui vous était utile, et que je n’ai pas craint de vous prêcher et de vous enseigner publiquement et dans les maisons,
\VS{21}prêchant tant aux Juifs qu’aux Grecs la repentance envers Dieu, et la foi en Jésus-Christ notre Seigneur.
\VS{22}Et maintenant voici, étant lié par l'Esprit, je vais à Jérusalem, ignorant ce qui m’y arrivera ;
\VS{23}seulement, de ville en ville le Saint-Esprit m'avertit que des liens et des tribulations m'attendent.
\VS{24}Mais je ne fais pour moi-même aucun cas de ma vie, comme si elle m’était précieuse, pourvu que j'achève ma course avec joie, et le ministère que j'ai reçu du Seigneur Jésus, pour rendre témoignage à l'Evangile de la grâce de Dieu.
\VS{25}Et maintenant voici, je sais que vous ne verrez plus mon visage, vous tous au milieu desquels j’ai passé en prêchant le Royaume de Dieu.
\VS{26}C'est pourquoi je vous prends aujourd'hui à témoin que je suis net du sang de tous.
\VS{27}Car je vous ai annoncé tout le conseil de Dieu, sans en rien cacher.
\VS{28}Prenez donc garde à vous-mêmes, et à tout le troupeau sur lequel le Saint-Esprit vous a établis évêques\FTNT{Evêque, ~episcopos~ en grec : surveillant, gardien. Ce terme désigne la fonction des anciens. Dans la Nouvelle Alliance, les évêques (ou anciens) sont des personnes dont la mission est de veiller et enseigner les âmes du Seigneur au sein de l'assemblée locale. Jésus-Christ, notre Dieu, est l’évêque par excellence (1 Pi. 2:25).}, pour paître l'Eglise de Dieu, qu’il a acquise par son propre sang.
\VS{29}Car je sais qu'après mon départ il s’introduira parmi vous des loups cruels, qui n'épargneront pas le troupeau,
\VS{30}et qu'il se lèvera du milieu de vous des hommes qui enseigneront des doctrines pernicieuses dans le but d'attirer les disciples vers eux.
\VS{31}C'est pourquoi veillez, vous souvenant que durant trois ans, je n'ai cessé nuit et jour d'avertir chacun de vous avec larmes.
\VS{32}Et maintenant, mes frères, je vous recommande à Dieu, et à la parole de sa grâce, à celui qui est puissant pour achever de vous édifier, et pour vous donner l'héritage avec tous les saints.
\VS{33}Je n'ai désiré ni l'argent, ni l'or, ni les vêtements de personne.
\VS{34}Et vous savez vous-mêmes que ces mains ont pourvu à mes besoins et à ceux des personnes qui étaient avec moi.
\VS{35}Je vous ai montré de toutes manières que c’est en travaillant ainsi qu’il faut soutenir les faibles, et se rappeler les paroles du Seigneur Jésus, qui a dit lui-même : Il y a plus de bonheur à donner qu’à recevoir\FTNT{Lu. 14:12.}.
\VS{36}Après avoir ainsi parlé, il se mit à genoux et il pria avec eux tous.
\VS{37}Alors tous fondirent en larmes, et se jetant au cou de Paul,
\VS{38}ils l’embrassèrent, étant principalement affligés de ce qu’il avait dit qu’ils ne verraient plus son visage. Et ils l’accompagnèrent jusqu’au navire.
\TextTitle{[Paul voyage jusqu'à Tyr]}
\Chap{21}
\VerseOne{}Nous nous embarquâmes, après nous être séparés d’eux, et nous allâmes directement à Cos, et le jour suivant à Rhodes, et de là à Patara.
\VS{2}Et ayant trouvé un navire qui faisait la traversée vers la Phénicie, nous montâmes et partîmes.
\VS{3}Puis ayant découvert l’île de Chypre, nous la laissâmes à gauche, nous fîmes route vers la Syrie, nous arrivâmes à Tyr, car le navire devait y décharger sa cargaison.
\TextTitle{[Escale de sept jours à Tyr ; L'Esprit avertit Paul de la persécution]}
\VS{4}Nous trouvâmes les disciples et nous restâmes là sept jours. Les disciples, poussés par l'Esprit, disaient à Paul de ne pas monter à Jérusalem.
\VS{5}Mais ces jours étant passés, nous partîmes et nous nous acheminâmes pour partir de Tyr, et tous nous accompagnèrent avec leurs femmes et leurs enfants, jusqu’à l’extérieur de la ville. Nous nous mîmes à genoux sur le rivage et nous fîmes la prière.
\VS{6}Et après nous être embrassés les uns les autres, nous montâmes sur le navire, et les autres retournèrent chez eux.
\TextTitle{[Paul à Ptolémaïs, puis à Césarée ; le prophète Agabus]}
\VS{7}Et ainsi achevant notre navigation, nous allâmes de Tyr à Ptolémaïs ; et après avoir salué les frères, nous passâmes un jour avec eux.
\VS{8}Nous partîmes le lendemain, et nous arrivâmes à Césarée. Etant entrés dans la maison de Philippe l'évangéliste, qui était l'un des sept, nous restâmes chez lui.
\VS{9}Il avait quatre filles vierges qui prophétisaient.
\VS{10}Comme nous étions là depuis plusieurs jours, un prophète, nommé Agabus, arriva de Judée
\VS{11}et vint nous trouver. Il prit la ceinture de Paul, se lia les mains et les pieds, et il dit : Voici ce que déclare le Saint-Esprit : L’homme à qui appartient cette ceinture, les Juifs le lieront de la même manière à Jérusalem, et le livreront entre les mains des gentils.
\VS{12}Quand nous entendîmes ces choses, nous et ceux de l’endroit, nous priâmes Paul de ne pas monter à Jérusalem.
\VS{13}Mais Paul répondit : Que faites-vous en pleurant et en affligeant mon cœur ? Je suis prêt, non seulement à être lié, mais aussi à mourir à Jérusalem pour le Nom du Seigneur Jésus.
\VS{14}Comme il ne se laissait pas persuader, nous n’insistâmes pas, et nous dîmes : Que la volonté du Seigneur soit faite !
\TextTitle{[QUATRIEME VOYAGE : DE JERUSALEM A ROME]}
\TextTitle{[Paul monte à Jérusalem]}
\VS{15}Quelques jours après, nous fîmes nos préparatifs et nous montâmes à Jérusalem.
\VS{16}Quelques disciples de Césarée vinrent avec nous, amenant avec eux un homme appelé Mnason, de l’île de Chypre, disciple de longue date, chez qui nous devions loger.
\TextTitle{[Paul, à Jerusalem, se conforme aux rites de la loi mosaïque au sujet d'un voeu et entre dans le temple]
\\(Hé. 10:2,4,9-12)}
\VS{17}Lorsque nous arrivâmes à Jérusalem, les frères nous reçurent avec joie.
\VS{18}Et le jour suivant, Paul se rendit avec nous chez Jacques, et tous les anciens s’y réunirent.
\VS{19}Après les avoir embrassés, il raconta en détail les choses que Dieu avait faites au milieu des gentils par son ministère.
\VS{20}Quand ils l’eurent entendu, ils glorifièrent le Seigneur. Puis ils dirent à Paul : Tu vois frère, combien de milliers de Juifs ont cru ; mais ils sont tous zélés pour la loi.
\VS{21}Or ils ont appris que tu enseignes à tous les Juifs qui sont parmi les gentils, à renoncer à Moïse, en leur disant qu’ils ne doivent pas circoncire leurs enfants et de ne pas vivre selon les ordonnances de la loi.
\VS{22}Que faut-il donc faire ? Il faut absolument rassembler la multitude des fidèles, car ils apprendront que tu es venu.
\VS{23}C’est pourquoi fais ce que nous allons te dire : Nous avons quatre hommes qui ont fait un vœu,
\VS{24}prends-les avec toi, purifie-toi avec eux, et pourvois à leurs besoins afin qu'ils se rasent la tête. Et ainsi tous sauront que ce qu’ils ont entendu sur ton compte est faux, mais que toi aussi tu te conduis en observateur de la loi.
\VS{25}A l'égard des gentils qui ont cru, nous avons décidé et nous leur avons écrit qu’ils doivent s’abstenir des viandes sacrifiées aux idoles, du sang, des animaux étouffés, et de la débauche.
\VS{26}Alors Paul prit ces hommes, se purifia, et entra le lendemain dans le temple avec eux, pour annoncer quel jour leur purification devait s'achever, et quand l’offrande devait être présentée pour chacun d’eux.
\TextTitle{[Paul saisi et frappé par les Juifs]}
\VS{27}A la fin des sept jours, les Juifs d'Asie ayant vu Paul dans le temple, soulevèrent tout le peuple, et mirent la main sur lui,
\VS{28}en criant : Hommes Israélites, au secours ! Voici l’homme qui prêche partout et à tout le monde contre le peuple, contre la loi, et contre ce lieu. Il a même introduit des Grecs dans le temple, et a profané ce saint lieu.
\VS{29}Car ils avaient vu auparavant Trophime d’Ephèse avec lui dans la ville, et ils croyaient que Paul l'avait fait entrer dans le temple.
\VS{30}Toute la ville fut émue, et le peuple accourut de toutes parts. Ils se saisirent de Paul et le traînèrent hors du temple, dont les portes furent aussitôt fermées.
\TextTitle{[Intervention des soldats et des centeniers]}
\VS{31}Comme ils cherchaient à le tuer, le bruit vint au tribun de la cohorte que tout Jérusalem était en trouble.
\VS{32}A l’instant, il prit des soldats et des centeniers, et courut vers eux. Voyant le tribun et les soldats, ils cessèrent de frapper Paul.
\VS{33}Alors le tribun s’approcha, se saisit de Paul, et le fit lier de deux chaînes. Puis il demanda qui il était et ce qu’il avait fait.
\VS{34}Les uns criaient d'une manière, et les autres d'une autre, dans la foule. Ne pouvant donc rien apprendre de certain à cause du tumulte, il ordonna de mener Paul dans la forteresse.
\VS{35}Lorsque Paul fut sur les degrés, il dut être porté par les soldats, à cause de la violence de la foule ;
\VS{36}car la multitude du peuple le suivait en criant : Fais-le mourir !
\VS{37}Comme on allait faire entrer Paul dans la forteresse, il dit au tribun : M’est-il permis de te dire quelque chose ? Et le tribun répondit : Tu sais parler le grec ?
\VS{38}Tu n’es donc pas cet Egyptien qui a excité une sédition dernièrement, et qui a emmené dans le désert quatre mille brigands ?
\VS{39}Paul lui dit : Je suis Juif de Tarse, citoyen de la ville renommée de la Cilicie. Permets-moi, je te prie, de parler au peuple.
\VS{40}Le tribun le lui permit, Paul débout sur les degrés, fit signe de la main au peuple. Un profond silence s’établit, et Paul, parlant en langue hébraïque, dit :
\TextTitle{[Discours de Paul aux Juifs sur sa conversion]
\\(Ac. 9:1-18 ; 26:9-18)}
\Chap{22}
\VerseOne{}Hommes frères et pères, écoutez ce que j’ai maintenant à vous dire pour ma défense.
\VS{2}Lorsqu’ils entendirent qu’il leur parlait en langue hébraïque, ils redoublèrent de silence. Et Paul leur dit :
\VS{3}Je suis Juif, né à Tarse en Cilicie ; mais j’ai été élevé dans cette ville-ci aux pieds de Gamaliel et instruit dans la connaissance exacte de la loi de nos pères, étant plein de zèle pour la loi de Dieu, comme vous l'êtes tous aujourd'hui.
\VS{4}J’ai persécuté à mort cette doctrine, liant et mettant en prison hommes et femmes.
\VS{5}Le souverain sacrificateur lui-même et toute l'assemblée des anciens m'en sont témoins. J’ai même reçu d’eux des lettres pour les frères de Damas, où je me rendis afin d’amener liés à Jérusalem ceux qui se trouvaient là et de les faire punir.
\VS{6}Comme j’étais en chemin, et que j'approchais de Damas, tout à coup, vers midi, une grande lumière venant du ciel resplendit comme un éclair autour de moi.
\VS{7}Je tombai par terre, et j'entendis une voix qui me dit : Saul, Saul, pourquoi me persécutes-tu ?
\VS{8}Je répondis : Qui es-tu Seigneur ? Et il me dit : Je suis Jésus de Nazareth, que tu persécutes.
\VS{9}Ceux qui étaient avec moi furent tout effrayés, ils virent bien la lumière, mais ils ne comprirent pas la voix de celui qui me parlait. Alors je dis : Que ferai-je Seigneur ?
\VS{10}Et le Seigneur me dit : Lève-toi, va à Damas, et là on te dira tout ce que tu dois faire.
\VS{11}Comme je ne voyais rien, à cause de l’éclat de cette lumière, ceux qui étaient avec moi me prirent par la main, et j’arrivai à Damas.
\VS{12}Or un nommé Ananias, homme pieux selon la loi, et de qui tous les Juifs demeurant à Damas rendaient un bon témoignage, vint me trouver
\VS{13}et me dit : Saul mon frère, recouvre la vue. Au même instant, je recouvrai la vue et je le regardai.
\VS{14}Et il me dit : Le Dieu de nos pères t'a destiné à connaître sa volonté, à voir le Juste, et à entendre les paroles de sa bouche.
\VS{15}Car tu lui serviras de témoin auprès de tous les hommes des choses que tu as vues et entendues.
\VS{16}Et maintenant, pourquoi tardes-tu ? Lève-toi, et sois baptisé et purifié de tes péchés, en invoquant le Nom du Seigneur.
\TextTitle{[Le Seigneur appelle Paul à quitter Jérusalem et l'envoie dans les nations]}
\VS{17}De retour à Jérusalem, comme je priais dans le temple, je fus ravi en extase,
\VS{18}et je vis le Seigneur qui me disait : Hâte-toi, et sors promptement de Jérusalem, parce qu’ils ne recevront pas le témoignage que tu leur rendras de moi.
\VS{19}Et je dis : Seigneur, ils savent eux-mêmes que je faisais mettre en prison et battre de verges dans les synagogues ceux qui croyaient en toi,
\VS{20}et que lorsque le sang d'Etienne ton martyr fut répandu, j’étais moi-même présent, je consentais à sa mort, et je gardais les vêtements de ceux qui le faisaient mourir.
\VS{21}Alors il me dit : Va, car je t'enverrai au loin vers les gentils.
\TextTitle{[Les Juifs demandent la mort de Paul]}
\VS{22}Ils l'écoutèrent jusqu'à cette parole. Mais alors ils élevèrent leur voix en disant : Ote de la terre un tel homme, car il n'est pas digne de vivre !
\VS{23}Et comme ils criaient à haute voix, secouaient leurs vêtements, et jetaient de la poussière en l'air,
\VS{24}le tribun commanda de faire entrer Paul dans la forteresse, et de lui donner la question par le fouet, afin de savoir pour quel sujet ils criaient ainsi contre lui.
\TextTitle{[Paul revendique ses droits de citoyen romain]}
\VS{25}Comme on l’attachait pour le frapper, Paul dit au centenier qui était près de lui : Vous est-il permis de fouetter un homme romain, et qui n'est même pas condamné ?
\VS{26}A ces mots, le centenier alla vers le tribun pour l'avertir, disant : Prends garde à ce que tu feras, car cet homme est Romain.
\VS{27}Et le tribun, étant venu, dit à Paul : Dis-moi, es-tu Romain ? Et il répondit : Oui, je le suis.
\VS{28}Le tribun lui dit : J'ai acquis ce droit de citoyen pour une grande somme d’argent. Et moi, dit Paul, je l'ai par ma naissance.
\VS{29}Aussitôt, ceux qui devaient lui donner la question se retirèrent, et le tribun, voyant que Paul était Romain, fut dans la crainte parce qu’il l’avait fait lier.
\TextTitle{[Paul devant le sanhédrin]}
\VS{30}Le lendemain, voulant savoir avec certitude de quoi les Juifs l’accusaient, le tribun lui fit ôter ses liens, et donna l’ordre aux principaux sacrificateurs et à tout le sanhédrin de se réunir ; puis, il fit descendre Paul, et il le plaça au milieu d’eux.
\Chap{23}
\VerseOne{}Paul regardant fixement le sanhédrin, dit : Hommes frères ! Je me suis conduit en toute bonne conscience devant Dieu jusqu'à ce jour.
\VS{2}Le souverain sacrificateur Ananias ordonna à ceux qui étaient près de lui de le frapper sur la bouche.
\VS{3}Alors Paul lui dit : Dieu te frappera, muraille blanchie ! Tu es assis pour me juger selon la loi, et tu violes la loi en ordonnant qu’on me frappe !
\VS{4}Ceux qui étaient présents lui dirent : Tu insultes le souverain sacrificateur de Dieu ?
\VS{5}Et Paul dit : Je ne savais pas mes frères, que c’était le souverain sacrificateur ; car il est écrit : Tu ne parleras pas mal du chef de ton peuple.
\TextTitle{[Dissension entre pharisiens et sadducéens]}
\VS{6}Paul, sachant qu'une partie de l’assemblée était composée de sadducéens et l'autre de pharisiens, s'écria dans le sanhédrin : Hommes frères ! Je suis pharisien, fils de pharisien, c’est à cause de l’espérance et de la résurrection des morts que je suis mis en jugement.
\VS{7}Quand il eut dit cela, il s’éleva un débat entre les pharisiens et les sadducéens ; et l'assemblée se divisa.
\VS{8}Car les sadducéens disent qu'il n'y a point de résurrection, ni d'ange, ni d'esprit, mais les pharisiens soutiennent les deux choses.
\VS{9}Il y eut une grande clameur. Alors les scribes du parti des pharisiens se levèrent et contestèrent, disant : Nous ne trouvons aucun mal en cet homme ; peut-être un esprit ou un ange lui a parlé, ne combattons point contre Dieu.
\VS{10}Comme la division allait croissant, le tribun craignant que Paul ne soit mis en pièces par eux, fit descendre les soldats pour l’enlever du milieu d'eux, et le conduire dans la forteresse.
\TextTitle{[Le Seigneur fortifie Paul]}
\VS{11}La nuit suivante, le Seigneur apparut à Paul et lui dit : Prends courage, car de même que tu as rendu témoignage de moi dans Jérusalem, il faut aussi que tu rendes témoignage à Rome.
\TextTitle{[Plus de quarante Juifs conspirent contre Paul pour le tuer]}
\VS{12}Quand le jour fut venu, les Juifs formèrent un complot et firent des imprécations contre eux-mêmes, en disant qu'ils ne mangeraient pas ni ne boiraient pas jusqu'à ce qu'ils aient tué Paul.
\VS{13}Ceux qui formèrent ce complot étaient plus de quarante,
\VS{14}et ils s'adressèrent aux principaux sacrificateurs et aux anciens, et leur dirent : Nous nous sommes engagés, avec des imprécations contre nous-mêmes, à ne rien manger jusqu’à ce que nous ayons tué Paul.
\VS{15}Vous donc, maintenant, adressez-vous avec le sanhédrin au tribun pour le faire descendre demain au milieu de vous, comme si vous vouliez examiner sa cause plus exactement ; et nous, avant qu'il approche, nous sommes tous prêts à le tuer.
\VS{16}Le fils de la sœur de Paul, ayant eu connaissance de ce complot, alla dans la forteresse et le rapporta à Paul.
\VS{17}Paul appela l’un des centeniers et lui dit : Mène ce jeune homme au tribun, car il a quelque chose à lui rapporter.
\VS{18}Il le prit donc et le mena au tribun, et il lui dit : Le prisonnier Paul m'a appelé et m'a prié de t'amener ce jeune homme qui a quelque chose à te dire.
\VS{19}Et le tribun le prenant par la main, se retira à part, et lui demanda : Qu'est-ce que tu as à me rapporter ?
\VS{20}Et il lui dit : Les Juifs ont conspiré de te prier que demain tu envoies Paul au sanhédrin, comme s'ils voulaient s'enquérir de lui plus exactement de quelque chose.
\VS{21}Mais ne les écoute pas, car plus de quarante hommes d'entre eux lui adressent un guet-apens, et se sont engagés, avec des imprécations contre eux-mêmes, à ne rien manger ni boire jusqu'à ce qu'ils l'aient tué ; maintenant ils sont tous prêts, et n’attendent que ton consentement.
\VS{22}Le tribun donc renvoya le jeune homme, en lui recommandant de ne parler à personne de ce rapport qu’il lui avait fait.
\TextTitle{[Paul transféré de nuit à Césarée]}
\VS{23}Ensuite, il appela deux des centeniers, et il leur dit : Tenez prêts, dès la troisième heure de la nuit, deux cents soldats, soixante-dix cavaliers, et deux cents archers, pour aller jusqu’à Césarée.
\VS{24}Et ayez soin qu'il y ait aussi des montures pour Paul, afin qu’on le mène sain et sauf au gouverneur Félix\FTNT{Marcus Antonuis Félix était procurateur de la province romaine de la Judée de 52 à 60 ap. J.-C.}.
\VS{25}Et il lui écrivit une lettre en ces termes :
\TextTitle{[Lettre de Claude Lysias]}
\VS{26}Claude Lysias au très excellent gouverneur Félix, salut !
\VS{27}Les Juifs s’étaient saisis de cet homme et allaient le tuer, lorsque je survins avec des soldats et le leur enlevai, ayant appris qu’il était Romain.
\VS{28}Voulant connaître le motif pour lequel ils l'accusaient, je l’amenai devant leur sanhédrin.
\VS{29}J’ai trouvé qu’il était accusé au sujet de questions relatives à leur loi, mais qu’il n’avait commis aucun crime qui mérite la mort ou la prison.
\VS{30}Ayant été averti des embûches que les Juifs avaient dressées contre lui, je te l'ai aussitôt envoyé, en ordonnant à ses accusateurs de te dire eux-mêmes ce qu’ils ont contre lui. Adieu.
\VS{31}Les soldats prirent Paul, selon l’ordre qu’ils avaient reçu, et le conduisirent pendant la nuit jusqu’à Antipatris.
\VS{32}Le lendemain, laissant les cavaliers poursuivre la route avec Paul, ils retournèrent à la forteresse.
\VS{33}Arrivés à Césarée, les cavaliers remirent la lettre au gouverneur, et lui présentèrent aussi Paul.
\VS{34}Le gouverneur, après avoir lu la lettre, demanda à Paul de quelle province il était. Ayant appris qu'il était de Cilicie :
\VS{35}Je t'entendrai, lui dit-il, plus amplement quand tes accusateurs seront venus. Et il ordonna qu'il soit gardé dans le prétoire d'Hérode.
\TextTitle{[Paul devant le gouverneur Félix]}
\Chap{24}
\VerseOne{}Cinq jours après, arriva Ananias, le souverain sacrificateur, avec les anciens, et un certain orateur, nommé Tertulle. Ils portèrent plainte au gouverneur contre Paul.
\TextTitle{[Paul est accusé par les juifs]}
\VS{2}Paul fut appelé, et Tertulle commença à l'accuser, en disant :
\VS{3}Très excellent Félix, nous reconnaissons en tout et partout, et avec une entière gratitude, que c’est grâce à toi et aux heureux succès survenus à cette nation par ta prévoyance que nous jouissons d’une grande paix.
\VS{4}Mais, pour ne pas te retenir plus longtemps, je te prie d’écouter, dans ta bonté, ce que nous allons te dire en peu de paroles.
\VS{5}Nous avons trouvé cet homme, qui est une peste, qui sème des divisions parmi tous les Juifs du monde entier, et qui est le chef de la secte des Nazaréens.
\VS{6}Il a même tenté de profaner le temple ; et nous l'avons saisi, et avons voulu le juger selon notre loi.
\VS{7}Mais le tribun Lysias étant arrivé, l'a arraché de nos mains avec une grande violence,
\VS{8}en ordonnant à ses accusateurs de venir vers toi. Tu pourras toi-même, en l'interrogeant, apprendre de lui tout ce dont nous l'accusons.
\VS{9}Les Juifs se joignirent aussi à l’accusation, en disant que les choses étaient ainsi.
\TextTitle{[Paul se défend devant Félix, à Césarée]}
\VS{10}Après que le gouverneur eut fait signe à Paul de parler, il répondit : Sachant que tu es juge de cette nation depuis plusieurs années, c’est avec confiance que je prends la parole pour défendre ma cause.
\VS{11}Il n’y a pas plus de douze jours, tu peux t’en assurer, que je suis monté à Jérusalem pour adorer Dieu.
\VS{12}Ils ne m’ont pas trouvé ni dans le temple, ni dans les synagogues, ni dans la ville, discutant avec quelqu’un ou provoquant un rassemblement séditieux de la foule.
\VS{13}Et ils ne sauraient prouver les choses dont ils m'accusent maintenant.
\VS{14}Je te confesse bien que je sers le Dieu de mes pères selon la voie qu’ils appellent une secte, croyant tout ce qui est écrit dans la loi et dans les prophètes,
\VS{15}et ayant en Dieu cette espérance, comme ils l’ont eux-mêmes, qu’il y aura une résurrection des justes et des injustes.
\VS{16}C'est pourquoi je m’efforce d’avoir constamment une conscience sans reproche devant Dieu, et devant les hommes.
\VS{17}Après plusieurs années, je suis venu pour faire des aumônes et des offrandes à ma nation.
\VS{18}Et comme je m’occupais de ces choses, quelques Juifs d’Asie m’ont trouvé purifié dans le temple, sans attroupement ni tumulte.
\VS{19}C’était à eux de paraître en ta présence et de se porter accusateurs, s’ils avaient quelque chose contre moi.
\VS{20}Ou bien, que ceux-ci eux-mêmes disent, s'ils ont trouvé en moi quelque injustice, quand j'ai été présenté au sanhédrin ;
\VS{21}à moins que ce ne soit uniquement de ce cri que j’ai fait entendre au milieu d’eux : C’est à cause de la résurrection des morts que je suis aujourd’hui mis en jugement devant vous.
\VS{22}Félix, qui était parfaitement au courant de ce qui concerne cette secte, les ajourna, en disant : Quand le tribun Lysias sera venu, j’examinerai votre affaire.
\VS{23}Et il donna l’ordre au centenier de garder Paul, en lui laissant une certaine liberté, et n'empêchant aucun des siens de le servir, ou de venir vers lui.
\TextTitle{[Paul prêche Christ au gouverneur et à sa femme]}
\VS{24}Quelques jours après, Félix vint avec Drusille, sa femme, qui était Juive, et il envoya chercher Paul. Il l’entendit sur la foi en Christ.
\VS{25}Et comme il parlait de la justice, de la tempérance, et du jugement à venir, Félix tout effrayé répondit : Pour le moment retire-toi ; et quand j'aurai la commodité, je te rappellerai.
\VS{26}Il espérait en même temps que Paul lui donnerait de l’argent afin de le délivrer, c'est pourquoi il l'envoyait chercher souvent, et s'entretenait avec lui.
\TextTitle{[Paul, prisonnier deux ans à Césarée]}
\VS{27}Deux ans s’écoulèrent ainsi, et Félix eut pour successeur Porcius Festus\FTNT{Porcius Festus était procurateur de Judée d'environ 60 à 62, succédant à Antonius Félix.}, qui voulant faire plaisir aux Juifs, laissa Paul en prison.
\TextTitle{[Paul devant le gouverneur Festus]}
\Chap{25}
\VerseOne{}Festus, étant arrivé dans la province, monta trois jours après de Césarée à Jérusalem.
\VS{2}Le souverain sacrificateur, et les principaux d'entre les Juifs portèrent plainte contre Paul devant lui. Ils firent des instances auprès de Festus, et dans des vues hostiles,
\VS{3}lui demandèrent une faveur qu’il le fasse venir à Jérusalem. Ils avaient dressé des embûches pour le tuer en chemin.
\VS{4}Mais Festus leur répondit que Paul était bien gardé à Césarée, et que lui-même devait partir sous peu.
\VS{5}Et il ajouta : Que les principaux d’entre vous descendent avec moi, et s’il y a quelque chose de coupable contre cet homme, qu’ils l’accusent.
\VS{6}Festus ne passa que dix jours parmi eux, puis il descendit à Césarée. Le lendemain, siégeant au tribunal, il ordonna que Paul soit amené.
\VS{7}Quand il fut amené, les Juifs qui étaient descendus de Jérusalem l’entourèrent et portèrent contre lui de nombreuses et graves accusations qu’ils ne pouvaient pas prouver.
\VS{8}Tandis que Paul parlait pour sa défense : Je n’ai rien fait de coupable, ni contre la loi des Juifs, ni contre le temple, ni contre César.
\VS{9}Mais Festus voulant faire plaisir aux Juifs, répondit à Paul et dit : Veux-tu monter à Jérusalem et y être jugé sur ces choses devant moi ?
\TextTitle{[Paul en appelle à César]}
\VS{10}Paul dit : Je comparais devant le tribunal de César, où il faut que je sois jugé. Je n'ai fait aucun tort aux Juifs, comme tu le sais très bien.
\VS{11}Si j’ai commis quelque injustice, ou un crime digne de mort, je ne refuse pas de mourir ; mais si les choses dont ils m’accusent sont fausses, personne n’a le droit de me livrer à eux. J’en appelle à César.
\VS{12}Alors Festus, après avoir délibéré avec le conseil, lui répondit : Tu en as appelé à César, tu iras devant César.
\TextTitle{[Festus expose le cas de Paul au roi Agrippa]}
\VS{13}Quelques jours après, le roi Agrippa\FTNT{Agrippa : Agrippa II (27-28 ap. J.-C. – 93-101 ap. J.-C.) est le fils d’Agrippa 1er (10 av. J.-C. – 44 ap. J.-C.) lui-même petit-fils d’Hérode le Grand (73 av. J.-C. – 4 av. J.-C.). En 48, il se voit confier l’autorité sur les affaires du temple de Jérusalem. En 50, il devient roi de Chalcis, région située entre le Liban et l’Anti-Liban. Dès 53, il échange ce royaume contre l’ancienne tétrarchie d’Hérode Philippe, constituée de Batanée et de Trachonite, dans le sud de la Syrie. En 54, le nouvel empereur Néron (37 ap. J.-C. – 68 ap. J.-C.) ajoute au royaume d’Agrippa II la Galilée. A l’instar de son père, Agrippa II collabore avec Rome et fait tout ce qui est en son pouvoir pour empêcher la rupture entre Rome et les Juifs, mais en vain.} et Bérénice\FTNT{Bérénice : Aussi connue comme Julia Bérénice, née vraisemblablement à Rome vers 28, est une fille du roi Agrippa 1er et la sœur d’Agrippa II. Elle fut la maîtresse de Titus (39 ap. J.-C. – 81 ap. J.-C.), le fils de l'empereur Vespasien (9 ap. J.-C. – 79 ap. J.-C.). Bérénice reste cependant avec son frère Agrippa II pour y jouer le rôle de reine. On ne sait rien de l'épouse d'Agrippa II, pendant tout son règne c'est Bérénice qui fut présentée comme reine à ses côtés.} arrivèrent à Césarée pour saluer Festus.
\VS{14}Comme ils passèrent là plusieurs jours, Festus fit mention au roi de l'affaire de Paul, en disant : Félix a laissé prisonnier un homme
\VS{15}contre lequel, lorsque j'étais à Jérusalem, les principaux sacrificateurs et les anciens des Juifs ont porté plainte, en demandant sa condamnation.
\VS{16}Mais je leur ai répondu que ce n'est pas la coutume des Romains de livrer quelqu'un à la mort, avant que l’inculpé ait été mis en présence de ses accusateurs, et qu'il ait eu la liberté de se défendre sur le crime dont on l’accuse.
\VS{17}Ils sont donc venus ici, et sans différer, je siégeai le lendemain, et je donnai l’ordre qu’on amène cet homme.
\VS{18}Ses accusateurs s’étant présentés, ne lui imputèrent aucun des crimes dont je pensais qu’ils l’accuseraient.
\VS{19}Mais ils avaient avec lui des discussions relatives à leurs superstitions, et à un certain Jésus qui est mort, que Paul affirmait être vivant.
\VS{20}Ne sachant quel parti prendre dans ce débat, je demandai à cet homme s'il voulait aller à Jérusalem et y être jugé sur ces choses.
\VS{21}Mais Paul en ayant appelé, pour que sa cause soit réservée à la connaissance de l’empereur, j’ai ordonné qu’on le garde jusqu’à ce que je l’envoie à César.
\VS{22}Alors Agrippa dit à Festus : Je voudrais bien aussi entendre cet homme. Demain, dit-il, tu l'entendras.
\TextTitle{[Paul est amené dans la salle d'audience]}
\VS{23}Le lendemain donc, Agrippa et Bérénice étant venus en grande pompe, et étant entrés dans la salle d’audience avec les tribuns et les principaux de la ville, Paul fut amené sur l’ordre de Festus.
\VS{24}Et Festus dit : Roi Agrippa, et vous tous qui êtes ici avec nous, vous voyez cet homme au sujet duquel toute la multitude des Juifs s’est adressée à moi, soit à Jérusalem soit ici, en s’écriant qu'il ne devait plus vivre.
\VS{25}Pour moi, ayant trouvé qu'il n'avait rien fait qui mérite la mort, et lui-même en ayant appelé à Auguste, j'ai résolu de le faire partir.
\VS{26}Comme je n'ai rien de certain à écrire à l'empereur sur son compte, je vous l'ai présenté, et principalement à toi, roi Agrippa, afin qu'après en avoir fait l'examen, j'aie de quoi écrire.
\VS{27}Car il me semble qu'il n'est pas raisonnable d'envoyer un prisonnier sans marquer les faits dont on l'accuse.
\TextTitle{[Discours de Paul devant Agrippa]
\\(Ac. 9:1-18 ; 22:1-16)}
\Chap{26}
\VerseOne{}Agrippa dit à Paul : Il t'est permis de parler pour toi-même. Alors Paul ayant étendu la main, parla ainsi pour sa défense.
\VS{2}Roi Agrippa ! Je m'estime heureux de ce que je dois me défendre aujourd'hui devant toi, de toutes les choses dont les Juifs m’accusent.
\VS{3}Car tu connais parfaitement leurs coutumes et leurs discussions. Je te prie donc de m’écouter avec patience.
\VS{4}Ma vie, dès les premiers temps de ma jeunesse, est connue de tous les Juifs, puisqu’elle s’est passée à Jérusalem, au milieu de ma nation.
\VS{5}Car ils savent depuis longtemps, s'ils veulent en rendre témoignage, que j'ai vécu en pharisien, selon la secte la plus rigide de notre religion.
\VS{6}Et maintenant, je suis mis en jugement parce que j’espère l’accomplissement de la promesse que Dieu a faite à nos pères,
\VS{7}et à laquelle nos douze tribus, qui servent Dieu continuellement nuit et jour, espèrent parvenir ; et c'est pour cette espérance, ô roi Agrippa, que je suis accusé par les Juifs.
\VS{8}Quoi, jugez-vous incroyable que Dieu ressuscite les morts ?
\VS{9}Pour moi, j’avais cru devoir agir vigoureusement contre le Nom de Jésus de Nazareth.
\VS{10}C’est ce que j’ai fait à Jérusalem. J’ai mis en prison plusieurs des saints, après en avoir reçu le pouvoir des principaux sacrificateurs, et quand on les faisait mourir, je joignais mon suffrage à celui des autres.
\VS{11}Je les ai souvent châtiés dans toutes les synagogues, et les forçais à blasphémer. Dans mes excès de fureur contre eux, je les persécutais même jusque dans les villes étrangères.
\VS{12}Comme j'allais aussi à Damas dans ce dessein, avec l’autorisation et la permission des principaux sacrificateurs,
\VS{13}en plein midi, ô roi, je vis en chemin resplendir autour de moi et de mes compagnons, une lumière venant du ciel et dont l’éclat surpassait celui du soleil.
\VS{14}Nous tombâmes tous par terre, et j’entendis une voix qui me parlait en langue hébraïque : Saul, Saul, pourquoi me persécutes-tu ? Il te serait dur de regimber contre les aiguillons.
\VS{15}Je répondis : Qui es-tu Seigneur ? Et il répondit : Je suis Jésus que tu persécutes.
\VS{16}Mais lève-toi, et tiens-toi sur tes pieds ; car je te suis apparu pour t'établir ministre et témoin des choses que tu as vues et de celles pour lesquelles je t'apparaîtrai.
\VS{17}Je t’ai arraché du milieu du peuple et des gentils, vers qui je t'envoie maintenant,
\VS{18}pour ouvrir leurs yeux afin qu'ils passent des ténèbres à la lumière, et de la puissance de Satan à Dieu ; afin que par la foi qu’ils auront en moi, ils reçoivent la rémission de leurs péchés et qu’ils aient part à l’héritage des saints.
\VS{19}Ainsi, ô roi Agrippa, je n’ai pas été désobéissant à la vision céleste.
\VS{20}A ceux de Damas d’abord, puis à Jérusalem, dans toute la Judée, et chez les gentils, j’ai prêché la repentance et la conversion à Dieu, avec la pratique d’œuvres dignes de la repentance.
\VS{21}C'est pour cela que les Juifs se sont saisis de moi dans le temple, et ont tâché de me tuer.
\VS{22}Mais ayant été secouru par l'aide de Dieu, je suis vivant jusqu'à ce jour, rendant témoignage aux petits et aux grands, sans m’écarter en rien de ce que les prophètes et Moïse ont prédit devoir arriver,
\VS{23}à savoir que le Christ souffrirait, et que ressuscité le premier d’entre les morts, il annoncerait la lumière au peuple et aux nations.
\TextTitle{[Paul exhorte Agrippa]}
\VS{24}Comme il parlait ainsi pour sa défense, Festus dit à haute voix : Tu es fou Paul ! Ton grand savoir dans les lettres te fait déraisonner.
\VS{25}Et Paul dit : Je ne suis point fou, très excellent Festus, mais je dis des paroles de vérité et de bon sens.
\VS{26}Car le roi est bien informé de ces choses ; et je lui en parle librement, parce que je suis persuadé qu'il n’en ignore aucune, puisque ce n’est pas en cachette qu’elles se sont passées.
\VS{27}Ô Roi Agrippa ! Crois-tu aux prophètes ? Je sais que tu y crois.
\VS{28}Et Agrippa répondit à Paul : Tu vas bientôt me persuader de devenir chrétien !
\VS{29}Et Paul lui dit : Je souhaiterais devant Dieu que non seulement toi, mais aussi tous ceux qui m'écoutent aujourd'hui, vous deveniez tels que je suis à l’exception de ces liens !
\VS{30}Paul ayant dit ces choses, le roi se leva, avec le gouverneur et Bérénice, et ceux qui étaient assis avec eux.
\VS{31}Et s’étant retirés à part, ils se disaient les uns les autres : Cet homme n'a rien fait qui mérite la mort ou la prison.
\VS{32}Et Agrippa dit à Festus : Cet homme aurait pu être relâché s'il n'avait pas appelé à César.
\TextTitle{[Paul envoyé à Rome]}
\Chap{27}
\VerseOne{}Lorsqu’il fut décidé que nous embarquerions pour l’Italie, on remit Paul avec quelques autres prisonniers à un nommé Julius, centenier d'une cohorte de la légion appelée Auguste.
\VS{2}Nous montâmes sur un navire d'Adramytte, nous partîmes prenant notre route vers les côtes de l'Asie, ayant avec nous Aristarque, un Macédonien de la ville de Thessalonique.
\VS{3}Le jour suivant, nous arrivâmes à Sidon ; et Julius, qui traitait Paul avec bienveillance, lui permit d'aller vers ses amis afin de recevoir leurs soins.
\VS{4}Puis étant partis de là, nous longeâmes l’île de Chypre, parce que les vents étaient contraires.
\VS{5}Après avoir traversé la mer de Cilicie et de Pamphylie, nous arrivâmes à Myra, ville de Lycie.
\VS{6}Et là, le centenier trouva un navire d'Alexandrie qui allait en Italie, dans lequel il nous fit monter.
\VS{7}Pendant plusieurs jours, nous naviguâmes lentement, et ce ne fut pas sans difficulté que nous atteignîmes la hauteur de Cnide, où le vent ne nous permit pas d’aborder. Nous passâmes au-dessous de l’île de Crète, du côté de Salmone.
\VS{8}Nous la côtoyâmes avec peine, nous arrivâmes à un lieu qui est appelé Beaux-Ports, près duquel était la ville de Lasée.
\VS{9}Il s’était écoulé beaucoup de temps, et la navigation devenait dangereuse, car le temps du jeûne était déjà passé\FTNT{Ce jeûne correspondait au jour de l'expiation célébré le dixième jour du septième mois. Lé 23:27.}.
\VS{10}C’est pourquoi Paul les avertit en disant : Ô hommes, je vois que la navigation ne se fera pas sans péril et sans dommage, non seulement pour la cargaison et pour le navire, mais aussi pour nos propres vies.
\VS{11}Mais le centenier écouta plus le pilote et le maître du navire, plutôt que les paroles de Paul.
\VS{12}Et comme le port n'était pas bon pour y passer l'hiver, la plupart furent d'avis de partir de là, pour tâcher de gagner Phénix, qui est un port de Crète, qui regarde le vent d’Afrique et le couchant septentrional, afin d’y passer l'hiver.
\VS{13}Un vent du midi commença à souffler doucement, et se croyant maîtres de leur dessein, ils levèrent l’ancre et côtoyèrent de près l’île de Crète.
\TextTitle{[La tempête]}
\VS{14}Mais bientôt un vent impétueux, du nord-est, qu'on appelle Euraquilon\FTNT{Euraquilon : Vagues et vent d’Est.}, nous écarta de l'île.
\VS{15}Le navire fut emporté par la violence de la tempête, et ne pouvant résister, nous nous laissâmes aller au gré du vent.
\VS{16}Nous passâmes au-dessous d’une petite île nommée Clauda, et nous eûmes de la peine à nous rendre maîtres de la chaloupe ;
\VS{17}après l’avoir hissée, les matelots se servirent des moyens de secours pour ceindre le navire, et dans la crainte de tomber sur la Syrte\FTNT{Syrte : Il s’agit de la Grande Syrte et de la Petite Syrte : deux bancs de sables mouvants très redoutés.}, ils abaissèrent les voiles. C’est ainsi qu’on se laissa emporter par le vent.
\VS{18}Comme nous étions violemment battus par la tempête, le jour suivant, ils jetèrent la cargaison à la mer ;
\VS{19}et le troisième jour, nous jetâmes de nos propres mains les agrès du navire.
\VS{20}Le soleil et les étoiles ne parurent pas pendant plusieurs jours, et la tempête nous agitait si violemment que nous perdîmes enfin toute espérance de nous sauver.
\TextTitle{[Paul exhorte les prisonniers à s'alimenter]}
\VS{21}On n’avait pas mangé depuis longtemps. Paul se tenant alors debout au milieu d'eux, leur dit : Ô hommes, il fallait m’écouter et ne pas partir de Crète, afin d’éviter cette tempête et ce dommage.
\VS{22}Maintenant je vous exhorte à prendre courage ; car aucun de vous ne perdra la vie, et il n’y aura de perte que celle du navire.
\VS{23}Car un ange du Dieu à qui j’appartiens et que je sers m’est apparu cette nuit,
\VS{24}et m’a dit : Paul, ne crains point ; il faut que tu comparaisses devant César ; et voici, Dieu t'a donné tous ceux qui naviguent avec toi.
\VS{25}C'est pourquoi, ô hommes, prenez courage, car j'ai cette confiance en Dieu que la chose arrivera comme elle m'a été dite.
\VS{26}Mais nous devons échouer sur une île.
\VS{27}La quatorzième nuit, vers minuit, tandis que nous étions ballotés sur l’Adriatique, les matelots soupçonnèrent qu’on approchait de quelque terre.
\VS{28}Ayant jeté la sonde, ils trouvèrent vingt brasses ; puis étant passés un peu plus loin, et ayant encore jeté la sonde, ils trouvèrent quinze brasses.
\VS{29}Mais craignant de heurter contre des écueils, ils jetèrent quatre ancres de la poupe, et attendirent le jour avec impatience.
\VS{30}Mais comme les matelots cherchaient à s’échapper du navire, et mettaient la chaloupe à la mer, sous prétexte de jeter les ancres de la proue,
\VS{31}Paul dit au centenier et aux soldats : Si ces hommes ne restent pas dans le navire, vous ne pouvez pas être sauvés.
\VS{32}Alors les soldats coupèrent les cordes de la chaloupe, et la laissèrent tomber.
\VS{33}Avant que le jour paraisse, Paul les exhorta tous à prendre de la nourriture, en leur disant : C’est aujourd'hui le quatorzième jour que vous êtes en attente et que vous persistez à vous abstenir de manger.
\VS{34}Je vous exhorte donc à prendre de la nourriture, car cela est nécessaire pour votre salut, et aucun de vos cheveux ne se perdra.
\VS{35}Ayant ainsi parlé, il prit du pain, et rendit grâces à Dieu en présence de tous ; il le rompit et se mit à manger.
\VS{36}Et tous, reprenant courage, mangèrent aussi.
\VS{37}Nous étions dans le navire deux cent soixante-seize personnes.
\VS{38}Quand ils eurent mangé jusqu'à être rassasiés, ils allégèrent le navire en jetant le blé dans la mer.
\TextTitle{[Naufrage du navire]}
\VS{39}Lorsque le jour fut venu, ils ne reconnurent point la terre ; mais ayant aperçu un golfe avec un rivage, ils résolurent d'y faire échouer le navire, s’ils le pouvaient.
\VS{40}Ayant donc retiré les ancres, ils abandonnèrent le navire à la mer, lâchant en même temps les attaches des gouvernails ; et ayant tendu la voile de l'artimon, ils tâchaient de se diriger vers le rivage.
\VS{41}Mais ils rencontrèrent une langue de terre, où ils firent échouer le navire ; et la proue, s’étant engagée, resta immobile, tandis que la poupe se brisait par la violence des vagues.
\VS{42}Les soldats furent d’avis de tuer les prisonniers, de peur que quelqu’un d’eux ne s’échappe à la nage.
\VS{43}Mais le centenier, voulant sauver Paul, les empêcha d'exécuter ce conseil. Il ordonna à ceux qui savaient nager de se jeter les premiers dans l’eau pour gagner la terre,
\VS{44}et aux autres, de se mettre sur des planches ou sur des débris du navire. Et ainsi tous parvinrent à terre sains et saufs.
\TextTitle{[Paul mordu par une vipère sur l'île de Malte]}
\Chap{28}
\VerseOne{}Une fois hors de danger, ils reconnurent alors que l'île s'appelait Malte.
\VS{2}Les barbares nous traitèrent avec beaucoup d’humanité ; ils nous recueillirent tous auprès d’un grand feu, qu’ils avaient allumé parce que la pluie tombait et qu’il faisait très froid.
\VS{3}Paul ayant ramassé un tas de broussailles et l’ayant mis au feu, une vipère en sortit à cause de la chaleur et s’attacha à sa main.
\VS{4}Quand les barbares virent cette bête suspendue à sa main, ils se dirent les uns les autres : Assurément, cet homme est un meurtrier ; puisque la justice n’a pas voulu le laisser vivre, après qu’il a été sauvé de la mer.
\VS{5}Mais Paul ayant secoué la bête dans le feu, ne ressentit aucun mal.
\VS{6}Les barbares s’attendaient à le voir enfler ou tomber subitement ; mais après avoir longtemps attendu, voyant qu’il ne lui arrivait aucun mal, ils changèrent de langage et dirent que c’était un Dieu.
\TextTitle{[Guérison du père de Publius]}
\VS{7}Il y avait dans cet endroit-là des terres qui appartenaient au principal personnage de l'île, nommé Publius, qui nous reçut et nous logea pendant trois jours avec beaucoup de bonté.
\VS{8}Le père de Publius était au lit, malade de la fièvre et de la dysenterie ; Paul s’étant rendu vers lui, pria, lui imposa les mains, et le guérit.
\VS{9}Là-dessus, vinrent tous les autres malades de l'île, et ils furent guéris.
\VS{10}Ils nous rendirent de grands honneurs, et à notre départ, on nous fournit ce qui nous était nécessaire.
\TextTitle{[Paul arrive à Rome]}
\VS{11}Trois mois après, nous partîmes sur un navire d'Alexandrie qui avait passé l’hiver dans l'île, et qui avait pour enseigne Castor et Pollux.
\VS{12}Ayant abordé à Syracuse, nous y restâmes trois jours.
\VS{13}De là, en suivant la côte, nous arrivâmes à Reggio ; et un jour après, le vent du Midi s'étant levé, nous fîmes en deux jours le trajet jusqu’à Pouzzoles,
\VS{14}où nous trouvâmes des frères qui nous prièrent de passer sept jours avec eux. Et ensuite, nous arrivâmes à Rome.
\VS{15}Et les frères qui y étaient, ayant appris de nos nouvelles, vinrent à notre rencontre jusqu’au Forum d'Appius et aux Trois-Tavernes. En les voyant, Paul rendit grâces à Dieu et prit courage.
\TextTitle{[Paul rend témoignage aux juif de Rome et y prêche pendant deux ans]}
\VS{16}Lorsque nous fûmes arrivés à Rome, le centenier mit les prisonniers entre les mains du préfet du Prétoire ; mais quant à Paul, il lui permit de demeurer dans un domicile particulier avec un soldat qui le gardait.
\VS{17}Trois jours après, Paul convoqua les principaux des Juifs ; et quand ils furent réunis, il leur dit : Hommes frères ! Sans avoir rien fait contre le peuple ni contre les coutumes des pères, j'ai été mis en prison à Jérusalem, et livré entre les mains des Romains,
\VS{18}qui après m'avoir examiné, voulaient me relâcher parce qu'il n'y avait en moi aucun crime qui mérite la mort.
\VS{19}Mais les Juifs s'y opposèrent, j'ai été contraint d'en appeler à César ; n’ayant du reste aucun dessein d'accuser ma nation.
\VS{20}C’est pour ce sujet que je vous ai appelés, afin de vous voir et vous parler ; car c'est pour l'espérance d'Israël que je porte cette chaîne.
\VS{21}Mais ils lui répondirent : Nous n'avons reçu de Judée aucune lettre à ton sujet, et il n’est venu aucun frère qui ait rapporté ou dit quelque mal de toi.
\VS{22}Cependant, nous voudrions bien apprendre de toi ce que tu penses, car nous savons que cette secte rencontre de partout de l’opposition.
\VS{23}Ils lui fixèrent un jour, et plusieurs vinrent auprès de lui dans son logis. Et depuis le matin jusqu’au soir, Paul leur annonçait le Royaume de Dieu, en rendant témoignage, et en cherchant, par la loi de Moïse et par les prophètes, à les persuader de ce qui concerne Jésus.
\VS{24}Les uns furent persuadés par les choses qu'il disait ; et les autres n'y crurent point.
\TextTitle{[Paul se tourne vers les gentils]
\\(Ap. 13:44 ; 18:6)}
\VS{25}Comme ils se retiraient en désaccord, Paul n’ajouta que ces paroles : C’est avec raison que le Saint-Esprit, parlant à nos pères par le prophète Esaïe, a dit :
\VS{26}Va vers ce peuple et dis : Vous entendrez de vos oreilles, et vous ne comprendrez point ; vous regarderez de vos yeux, et vous ne verrez point.
\VS{27}Car le cœur de ce peuple est devenu insensible ; ils ont endurci leurs oreilles, et ils ont fermé leurs yeux ; de peur qu'ils ne voient des yeux, qu'ils n'entendent des oreilles, qu'ils ne comprennent de leur cœur, qu'ils ne se convertissent, et que je ne les guérisse\FTNT{Es. 6:10.}.
\VS{28}Sachez donc que ce salut de Dieu est envoyé aux gentils, et ils l’écouteront.
\VS{29}Lorsqu’il eut dit cela, les Juifs s’en allèrent, discutant vivement entre eux.
\VS{30}Paul demeura deux ans entiers dans une maison qu'il avait louée. Il recevait tous ceux qui venaient le voir,
\VS{31}prêchant le Royaume de Dieu, et enseignant ce qui concerne le Seigneur Jésus-Christ en toute liberté et sans obstacle.
\PPE{}
\end{multicols}
