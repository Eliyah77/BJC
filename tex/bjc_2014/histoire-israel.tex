\section*{Israël : l'histoire d'une nation}

\subsection*{Les hébreux}

L'origine du mot « Hébreu » demeure incertaine à ce jour. Il est néanmoins probable que ce terme dérive du nom « Héber », l'un des ancêtres d'Abraham (Genèse 10 : 24-25). Le nom Hébreux, qui signifie « ceux qui passent », se rapporterait donc à des peuples nomades, de passage.
Certains pensent que ce nom pourrait venir de « Habirou », terme de l’Égypte antique utilisé pour désigner les « nomades ».

Concernant Abraham, il est aussi très vraisemblable qu'il vécut sous la dynastie des rois babyloniens qui régnèrent sur la Mésopotamie au IIe millénaire av JC.

\subsection*{Abraham : la promesse}

Abram, dont le nom signifie « père élevé » ou « haut père » était d'origine araméenne. Il vécut à Ur, dans la région de Chaldée située au sud de la Mésopotamie (actuelle Irak) aux environs du IIe millénaire avant J.C. La région était principalement composée de peuples nomades. A cette époque extrêmement reculée, ces peuples mésopotamiens vivaient au sein d'un clan composé de la famille du patriarche, de serviteurs et de servantes, et du bétail. Guidé par son père et patriarche de son clan, Térach, Abraham quitta sa région natale chaldéenne avec sa femme Saraï, pour migrer vers le pays de Canaan. Cette région correspond aujourd'hui à la partie du moyen orient regroupant la Palestine, le Liban (ancienne Phénicie), l'Ouest de la Jordanie, l'Est de l'Égypte et le sud de la Syrie. Durant le voyage, le clan s'arrêta et habita quelque temps dans le pays de Charan ou Haran (nord de la Syrie) où Térach mourut âgé de deux-cents cinq ans (Genèse 11 : 32).
La signification du nom de Térach signifie « station ». Ce dernier était donc symboliquement un poids pour Abram car il l'obligeait à stationner à Charan alors que l'objectif était d'arriver en Canaan. D'autres parts, l’Éternel avait d'autres plans pour Abram et ce n'est qu'après la disparition de Térach que Dieu, nous le verrons plus bas, ordonna à Abram de suivre la voie qu'il lui indiqua.

Remarque : Notons qu'Abram, bien que marié et adulte, partit de Chaldée sous l'autorité de son père, Térach. A cette époque antique, tout le clan était soumis à l'autorité patriarcale jusqu'à ce que ce dernier meure ou qu'il délègue son autorité à un tiers. Les fils, bien qu'âgés, étaient soumis à l'autorité parentale. Celle-ci n'était pas remise en cause et il était d'usage que chaque membre du clan, quel que soit son âge ou sa position, se plie à la volonté du patriarche.

C'est alors qu'à Charan, l’Éternel s'adressa à Abram et lui dit : « Va-t-en de ton pays, de ta patrie, et de la maison de ton père, dans le pays que je te montrerai. Je ferai de toi une grande nation, et je te bénirai; je rendrai ton nom grand, et tu seras une source de bénédiction. Je bénirai ceux qui te béniront, et je maudirai ceux qui te maudiront; et toutes les familles de la terre seront bénies en toi » Genèse 12 : 1-3.

Il est très intéressant de noter que la première fois que l’Éternel s'adressa à Abram fût pour lui dire que de lui sortirai une grande nation. En effet, Abram n'était encore que le nomade chaldéen quand Dieu vit déjà en lui nation qu'il eût choisi de bâtir.
Âgé de soixante-quinze ans (Genèse 12 : 4), et prenant foi aux paroles de l’Éternel, il s'en alla donc vers Canaan, accompagné de sa femme, de ses serviteurs, et de son neveu, Lot. Une fois arrivé à Sichem, petite localité située à environ deux kilomètres de l'actuelle Naplouse (en Palestine), l’Éternel fit une seconde promesse à Abram : « Je donnerai ce pays à ta postérité » (Genèse 12 : 7). Après lui avoir promis que de lui sortirait une grande nation, l’Éternel dévoila donc au nomade le futur lieu de résidence de sa postérité : l'immense région de Canaan. Imaginons l'effet que cette promesse eut dans le cœur du pèlerin. Lui qui n'avait jamais eu de terre, engendrera une nation qui sera amenée à résider dans cette région prospère et fertile (Exode 3 : 8). A défaut de pouvoir voir contempler cette nation de ses propres yeux, Dieu fît grâce à Abram de contempler le pays de son habitation. En signe de soumission, d'acceptation, et d'adoration, Abram érigea là, à Sichem, un autel à l’Éternel, puis il descendit au sud et s'en alla dresser ses tentes dans la petite région de Luz, située à dix kilomètres au nord de l'actuelle Jérusalem et y bâtit un second autel à Dieu et continua sa marche vers le sud.

Remarque : Là où Térach avait décidé de dresser ses tentes à Charan, Abraham dressa les siennes à Luz. Ce stationnement d'Abraham fût prophétique dans la mesure où, deux générations plus tard, Jacob, son petit-fils, recevant une visitation de Dieu, changera le nom de cette petite ville en Béthel qui signifie « maison de Dieu » (voir chapitre sur Jacob).

Il y eut une famine dans le pays de Canaan ce qui obligea Abram à descendre jusqu'au pays d’Égypte, puis il retourna à Luz où il avait dressé l'autel à l’Éternel. Il se sépara de son neveu Lot (Genèse 13 : 1-13), l’Éternel lui confirma sa promesse (Genèse 13 : 14-18) et le pèlerin stationna Mamré, au nord d'Hébron. Plus tard, Abram reçut la révélation d'El Elyon (Dieu Très-Haut) auprès du roi de Salem, Melchisédeck. Ce fût dans la vallée de Schavé prêt de Sodome (Genèse 14 : 17-22).
Abram qui était très âgé, eût un moment de doute quant à la réalisation de la promesse mais Dieu le lui confirma de nouveau (Genèse 15 : 1-6). En outre, il lui annonça la future captivité égyptienne de son peuple (voir chapitre sur la captivité égyptienne) durant quatre cent ans et la délivrance qui s'en suivra (Genèse 14 : 13-14).

Manque de foi d'Abram et naissance d'Ismaël
Après ces événements, Saraï, submergée par le découragement quant à son infertilité, se résigna à donner Agar, sa servante égyptienne à Abram pour qu'elle lui donne un héritier. Agar devint enceinte et méprisa Saraï. Cette dernière, excédée, la maltraita et Agar fuit dans le désert. Là, l’Éternel se révéla à elle et lui promit une grande postérité à travers Ismaël, son fils. Ce dernier naquit alors qu'Abram avait quatre-vingt-six ans (Genèse 16 : 1-16).

Remarque : Ismaël et les Arabes
La naissance d'Ismaël fût le résultat du manque de foi d'Abram et de Saraï. Mais Dieu dans sa bonté décida de bénir l'enfant et de le rendre prospère. Néanmoins, sa naissance eut pour conséquence de créer un conflit entre la postérité d'Isaac, fils de la promesse, et celle d'Ismaël. En effet, Ismaël est l'ancêtre des Arabes, qui sont aujourd'hui majoritairement hostiles envers les hébreux. En outre, ils créèrent l'islam, une religion en totale inadéquation avec les vérités de l’Évangile et qui n'accepte pas la divinité de Jésus-Christ.

« Père d'une multitude » : la vision se précise
Treize ans après la naissance d'Ismaël, et à l'âge de quatre-vingt-dix-neuf ans, Abram reçu la révélation d'El Shaddaï, le Dieu Tout Puissant, qui lui confirma sa promesse et lui dit :

« Voici mon alliance, que je fais avec toi. Tu deviendras père d'une multitude de nations. On ne t'appellera plus Abram; mais ton nom sera Abraham, car je te rends père d'une multitude de nations. Je te rendrai fécond à l'infini, je ferai de toi des nations; et des rois sortiront de toi. J'établirai mon alliance entre moi et toi, et tes descendants après toi, selon leurs générations: ce sera une alliance perpétuelle, en vertu de laquelle je serai ton Dieu et celui de ta postérité après toi. Je te donnerai, et à tes descendants après toi, le pays que tu habites comme étranger, tout le pays de Canaan, en possession perpétuelle, et je serai leur Dieu » Genèse 17 : 4-8.

Le changement du nom d'Abraham symbolisa son changement de statut. Il passa de « père élevé » à « père d'une multitude ». Aussi, un élément s'ajouta à ce changement de nom : l'apparition de rois dans un futur proche (voir chapitre des Rois).

La Circoncision
Aussi, le Seigneur scella cette alliance par un signe, la circoncision. Il ordonna à toute la maison d'Abraham d'être circoncise en signe de l'accomplissement de la promesse. Par ce moyen, l’Éternel marqua son alliance dans la chair de son peuple (Genèse 17 : 9-14).

Par la suite, l’Éternel donna cette ordonnance à Abraham concernant Saraï et lui confirma la naissance du fils de la promesse :

« Tu ne donneras plus à Saraï, ta femme, le nom de Saraï; mais son nom sera Sara. Je la bénirai, et je te donnerai d'elle un fils; je la bénirai, et elle deviendra des nations; des rois de peuples sortiront d'elle. Abraham tomba sur sa face; il rit, et dit en son cœur: Naîtrait-il un fils à un homme de cent ans ? et Sara, âgée de quatre-vingt-dix ans, enfanterait-elle ? Et Abraham dit à Dieu: Oh ! qu'Ismaël vive devant ta face ! Dieu dit: Certainement Sara, ta femme, t'enfantera un fils; et tu l'appelleras du nom d'Isaac. J'établirai mon alliance avec lui comme une alliance perpétuelle pour sa postérité après lui » Genèse 17 : 15-19.

Abraham pensait qu'Ismaël était le fils de la promesse, néanmoins ce dernier fût conçu à cause de l'incrédulité d'Abram et de Saraï. Sa naissance ne fût pas dans le plan de Dieu, mais comme nous l'avons vu précédemment, Dieu béni l'enfant et le rendit prospère. Dans ce passage, l’Éternel attesta son choix d'Isaac comme héritier d'Abraham et vecteur de la promesse.
Plus tard, trois anges de Dieu confirmèrent à Sara et Abraham la naissance prochaine d'Isaac (Genèse 18 : 10-15).

\subsection*{Isaac : fils de la promesse}

« L'Éternel se souvint de ce qu'il avait dit à Sara, et l'Éternel accomplit pour Sara ce qu'il avait promis. Sara devint enceinte, et elle enfanta un fils à Abraham dans sa vieillesse, au temps fixé par Dieu. Abraham donna le nom d'Isaac au fils qui lui était né, que Sara lui avait enfanté. Abraham circoncit son fils Isaac, âgé de huit jours, comme Dieu le lui avait ordonné. Abraham était âgé de cent ans, à la naissance d'Isaac, son fils. Et Sara dit: Dieu m'a fait un sujet de rire; quiconque l'apprendra rira de moi. Elle ajouta: Qui aurait dit à Abraham: Sara allaitera des enfants ? Cependant je lui ai enfanté un fils dans sa vieillesse. L'enfant grandit, et fut sevré; et Abraham fit un grand festin le jour où Isaac fut sevré » Genèse 21 : 1-8.

Après vingt-cinq ans d'attente, la promesse d'un fils de l'alliance se concrétisa. L’Éternel se montra fidèle envers son serviteur et démontra en parallèle que sa fidélité restera à jamais intacte vis à vis de la nation qui sortira d'Abraham (Habakuk 2 : 3).

Remarque : Le nom d'Isaac, qui signifie « il rit », symbolisa le bonheur d'Abraham et de Sara, mais aussi l'incrédulité des époux face aux promesses de Dieu (Genèse 17 : 17 ; Genèse 18 : 12-15)

Par la suite Dieu éprouva Abraham, en lui demandant le sacrifice d'Isaac (Genèse 22 : 1-14). Imaginons ce qu'il dut ressentir à l'idée de sacrifier son unique fils, celui par lequel l’Éternel devait passer pour accomplir la promesse. Fini la joie d'avoir un fils, fini la promesse d'une nation puissante et prospère. Abraham emmena donc Isaac à Morija pour le sacrifier, mais l’Éternel l'en empêcha :

« Puis Abraham étendit la main, et prit le couteau, pour égorger son fils. Alors l'ange de l'Éternel l'appela des cieux, et dit: Abraham ! Abraham ! Et il répondit: Me voici ! L'ange dit: N'avance pas ta main sur l'enfant, et ne lui fais rien; car je sais maintenant que tu crains Dieu, et que tu ne m'as pas refusé ton fils, ton unique. Abraham leva les yeux, et vit derrière lui un bélier retenu dans un buisson par les cornes; et Abraham alla prendre le bélier, et l'offrit en holocauste à la place de son fils. Abraham donna à ce lieu le nom de Jehova Jiré. C'est pourquoi l'on dit aujourd'hui: A la montagne de l'Éternel il sera pourvu » Genèse 22 : 10-14.

L'épreuve d'Abraham, fît transparaître la volonté de Dieu que sa nation lui soit obéissante quoiqu'il arrive. Nous verrons, en avançant dans l'histoire, que l'obéissance des fils d'Abraham, sera la condition sine qua none pour que ces derniers obtiennent la victoire, la délivrance, et la paix.

Remarque : C'est à Jérusalem, sur le mont Morija, que Salomon construisit le temple de l’Éternel (II Chroniques 3 : 1).

Isaac grandit et Abraham confia à son plus fidèle serviteur le soin de lui trouver une épouse parmi celles de son peuple :

« Abraham dit à son serviteur, le plus ancien de sa maison, l'intendant de tous ses biens: Mets, je te prie, ta main sous ma cuisse; et je te ferai jurer par l'Éternel, le Dieu du ciel et le Dieu de la terre, de ne pas prendre pour mon fils une femme parmi les filles des Cananéens au milieu desquels j'habite, mais d'aller dans mon pays et dans ma patrie prendre une femme pour mon fils Isaac » Genèse 24 : 2-4.

Le serviteur retourna donc en Chaldée dans le clan d'Abraham. Conduit par le Seigneur il rencontra Rebecca, fille de Béthuel, fils de Nachor, frère d'Abraham. Elle le conduisit dans la maison de son père et ce dernier accepta la requête du serviteur d'Abraham. Il revint avec elle à Canaan. Isaac, âgé de quarante ans, épousa Rébecca (Genèse 24 : 15-67).

Remarque : A l'instar d'Abraham, l’Éternel apparut à Isaac pour lui confirmer la promesse (Genèse 26 : 2-5). Il apparut à tous les patriarches pour la leur confirmer directement.

\subsection*{Jacob : un Israël en devenir}

Rebecca donna naissance à deux fils : Esaü et Jacob. Bien que les deux frères furent jumeaux, ils furent différents d'un point de vue physique et psychologique. Esaü fût poilu et habile dans la chasse, ce fût un homme des champs. Quant à Jacob, il fût paisible et aimait demeurer dans les tentes (Genèse 25 : 24-26).

Remarque : Esaü, dont le nom signifie « poilu, velu » fût le fondateur d'Edom, pays situé dans l'actuelle Jordanie. Au cours de leur histoire les israélites affrontèrent à plusieurs reprises les édomites (Nombres 20 : 14-21). Les deux peuples furent en inimitié l'un envers l'autre conformément à la prophétie d'Isaac (Genèse 27 : 39-40).

La vente du droit d’aînesse

En rentrant d'une chasse, Esaü vendit son droit d'aînesse à son frère Jacob pour un plat de lentille (Genèse 25 : 27-34).

La vente du droit d’aînesse symbolisa le changement de statut de Jacob. Non seulement il avait maintenant autorité, et prééminence sur son frère, mais surtout, il devint le premier héritier de la maison de son père, Isaac, fils d'Abraham. La promesse faite à Abraham et à Isaac (Genèse 26 : 3-5) se réalisera et se poursuivra avec Jacob. Il deviendra l’instrument de Dieu pour fonder la nation d’Israël. Esaü négligea son statut d'héritier et ne s'attacha pas aux promesses de l’Éternel. Nous pouvons néanmoins dire que dans son plan parfait, Dieu s'était réservé Jacob. Même si Esaü était un fin chasseur et un homme vigoureux, l’Éternel avait choisi Jacob, un homme tranquille et paisible.

Par la suite, Isaac, fort en âge, promis à Esaü la bénédiction s'il lui amenait un plat de viande. Esaü s'employa à aller chasser pour confectionner le plat. Entre temps, Rebecca ayant ouïe les paroles d'Isaac, appela Jacob et mit au point un plan pour que la bénédiction lui soit attribuée (Genèse 27 : 1-27). Certains commentateurs arguent que Jacob vola la bénédiction à son frère Esaü. En réalité il n'en est rien car comme nous l'avons vu précédemment, même si Jacob, déguisé, se fit passé pour Esaü, son frère ainé, il avait la prédominance sur lui et était devenu l'héritier légitime des promesses de l’Éternel en vertu de l'acquisition du droit d’aînesse. Nous ne savons pas si Isaac fût au courant de la perte de ce droit par Esaü au profit de Jacob, mais Dieu le savait et prit en considération ce passage de témoin. Ainsi, Jacob reçu de manière légitime devant Dieu la bénédiction. Au-delà des hommes, Dieu rétabli l'ordre des choses et la promesse bénéficia à celui qui en eut droit.

Voici les paroles qu'Isaac déclara à Jacob :

« Que Dieu te donne de la rosée du ciel Et de la graisse de la terre, Du blé et du vin en abondance ! Que des peuples te soient soumis, Et que des nations se prosternent devant toi ! Sois le maître de tes frères, Et que les fils de ta mère se prosternent devant toi ! Maudit soit quiconque te maudira, Et béni soit quiconque te bénira » Genèse 27 : 28-26.

Ces paroles dites à Jacob, étaient prophétiques, destinées au peuple qui sortirait de ses entrailles. Deux points sont à noter concernant cette prophétie :

- Confirmation de la terre d'abondance (verset 28)
- Domination du peuple sur les autres nations (versets 29)

Plus tard, Isaac confirma à Jacob l'alliance que l’Éternel fît avec Abraham :

« Que le Dieu tout-puissant te bénisse, te rende fécond et te multiplie, afin que tu deviennes une multitude de peuples ! Qu'il te donne la bénédiction d'Abraham, à toi et à ta postérité avec toi, afin que tu possèdes le pays où tu habites comme étranger, et qu'il a donné à Abraham ! » Genèse 28 : 3-4.

Puis il ordonna à Jacob de ne pas prendre de femmes parmi les cananéennes et selon l'ordre de son père, Jacob se mit en chemin pour aller à Charan, dans la plaine de Paddan-Aram, située au Nord de la Mésopotamie (dans l'actuelle Syrie).

Remarque : A l'instar d'Abraham, Isaac demanda à ce que son fils prenne une femme parmi celles de son peuple, resté à Charan.

Sur la route, dans la petite ville de Luz, il fît un songe dans lequel Dieu s'adressa à lui et lui déclara ces paroles :

« Je suis l'Éternel, le Dieu d'Abraham, ton père, et le Dieu d'Isaac. La terre sur laquelle tu es couché, je la donnerai à toi et à ta postérité. Ta postérité sera comme la poussière de la terre; tu t'étendras à l'occident et à l'orient, au septentrion et au midi; et toutes les familles de la terre seront bénies en toi et en ta postérité. Voici, je suis avec toi, je te garderai partout où tu iras, et je te ramènerai dans ce pays; car je ne t'abandonnerai point, que je n'aie exécuté ce que je te dis » Genèse 28 : 12.

Plusieurs choses sont à noter :

- Pour la première fois, Dieu s'adressa directement à Jacob pour lui confirmer la promesse et l'alliance qu'il avait fait avec Abraham et Isaac.
- L’Éternel affirma par ses paroles qu'il serait le protecteur de cette nation, qu'il ne l'abandonnerait jamais. Il révèle ainsi sa fidélité.

Après cet événement, Jacob consacra le lieu en utilisant la pierre qui lui servit de chevet pour dresser un monument à l’Éternel. Ce geste, très significatif, symbolisa le fait que Dieu serait l'appui et le soutien de cette nation. Aussi, il appela le lieu Béthel, littéralement « maison de l’Éternel » à l'endroit même où Abraham son père avait dressé ses tentes (Genèse 28 : 17).

Néanmoins, après cette visitation de l’Éternel, voici ce que déclara Jacob :

« Si Dieu est avec moi et me garde pendant ce voyage que je fais, s'il me donne du pain à manger et des habits pour me vêtir, et si je retourne en paix à la maison de mon père, alors l'Éternel sera mon Dieu » Genèse 28 : 20-21.

Le conditionnel qu'utilisa Jacob, mit en exergue sa dubitation et son manque de foi vis à vis des promesses de l’Éternel.
Puis, Jacob arriva à Charan chez son oncle Laban. Il demeura et travailla en tout vingt ans chez ce dernier (Genèse 31 : 38). Il prit pour épouse ses deux filles, Léa l'aînée, et Rachel la cadette. Cette dernière qui fût temporairement stérile, donna à Jacob sa servante afin qu'elle lui donne des héritiers. Léa eut aussi recourt à sa servante, Zilpa pour augmenter le nombre d’héritiers qu'elle donna à Jacob et ce dernier eut en tout douze fils :

Fils de Léa, 1ère femme de Jacob
Fils de Bilha servante de Rachel
Fils de Zilpa servante de Léa
Fils de Rachel, 2ème femme de Jacob
Ruben, Siméon, Lévi,
Juda (Genèse 29 : 32-35)

Isaacar, Zabulon (Genèse 29 : 18-20)
Dan, Nephtali (Genèse 29 : 6-8)
Gad, Aser (Genèse 29 : 11-13)
Joseph (Genèse 29 : 22-24)
Benjamin (Genèse 35 : 16)

Remarque : Léa appela son quatrième fils Juda, car elle dit : « cette fois je louerai l’Éternel ». Il est à ce titre intéressant de noter que de la descendance de Juda naîtra le roi David (voir chapitre des rois) qui fût connut pour être un grand musicien, jouant de harpe et étant chantre de l’Éternel. Ce dernier fût d'ailleurs à l'origine de bon nombres de Psaumes et il apaisait Saül quand il était agité par un démon (I Samuel 16 : 13). C'est aussi de la descendance de Juda que naquit le Seigneur Jésus-Christ, le Roi de Gloire loué dans toutes les nations (Philippiens 2 : 9-11).

Rappelons-nous de cette parole : « toutes les familles de la terre seront bénies en toi ». Durant le séjour de Jacob à Charan, l’Éternel accru les biens de Laban, et il en fût de même pour Jacob dont la richesse augmenta de manière exceptionnelle (Genèse 30 : 27-43). Ainsi, à l'image de cette future nation, Jacob fût une source de bénédiction pour Laban, son oncle. Cet événement marqua les prémices de l'accomplissement de cette prophétie.

Par la suite, l’Éternel demanda à Jacob de retourner à Béthel : « Je suis le Dieu de Béthel, où tu as oint un monument, où tu m'as fait un vœu. Maintenant, lève-toi, sors de ce pays, et retourne au pays de ta naissance » Genèse 31 : 13.

Jacob et sa maison se mirent en route, mais Rachel déroba les térébinthes de son père. Ce dernier, apprenant le départ de son gendre et le vol de ses idoles, se mit en marche et le poursuivit (Genèse 31 : 31-55). Notons que l'épisode du vol des térébinthes de Laban préfigurait aussi les infidélités de ce peuple vis à vis du Dieu Unique, en se tournant vers des idoles. Rachel, mère de cette nation, démontra son attachement aux dieux païens de son père, alors que l’Éternel s'était révélé à Jacob et attendait de lui une entière soumission.

Laban fût presque un obstacle mais l’Éternel délivra Jacob et les deux hommes se séparèrent en paix (Genèse 31 : 14-55).

Jacob, une fois arrivé à la rivière Jabbok (située à l'est du fleuve Jourdain, dans l'actuelle Jordanie), rappela la promesse que l’Éternel lui avait faite. Rappelons-nous que, comme nous l'avons vu précédemment, Jacob fût dubitatif quant aux paroles que l’Éternel lui donna à Béthel. Mais vingt ans après, dans l'adversité, il mit foi en ces paroles et invoqua l’Éternel :

« Délivre-moi, je te prie, de la main de mon frère, de la main d'Ésaü ! Car je crains qu'il ne vienne, et qu'il ne me frappe, avec la mère et les enfants. Et toi, tu as dit: Je te ferai du bien, et je rendrai ta postérité comme le sable de la mer, si abondant qu'on ne saurait le compter » Genèse 32 : 12-13.

\subsection*{Israël : naissance d'une nation}

La nuit même, Dieu lutta avec Jacob jusqu'à l'aurore :

« Jacob demeura seul. Alors un homme lutta avec lui jusqu'au lever de l'aurore. Voyant qu'il ne pouvait le vaincre, cet homme le frappa à l'emboîture de la hanche; et l'emboîture de la hanche de Jacob se démit pendant qu'il luttait avec lui. Il dit: Laisse-moi aller, car l'aurore se lève. Et Jacob répondit: Je ne te laisserai point aller, que tu ne m'aies béni. Il lui dit: Quel est ton nom ? Et il répondit: Jacob. Il dit encore: ton nom ne sera plus Jacob, mais tu seras appelé Israël; car tu as lutté avec Dieu et avec des hommes, et tu as été vainqueur. Jacob l'interrogea, en disant: Fais-moi je te prie, connaître ton nom. Il répondit: Pourquoi demandes-tu mon nom ? Et il le bénit là. Jacob appela ce lieu du nom de Peniel: car, dit-il, j'ai vu Dieu face à face, et mon âme a été sauvée. Le soleil se levait, lorsqu'il passa Peniel. Jacob boitait de la hanche. C'est pourquoi jusqu'à ce jour, les enfants d'Israël ne mangent point le tendon qui est à l'emboîture de la hanche; car Dieu frappa Jacob à l'emboîture de la hanche, au tendon » Genèse 32 : 24-32.

Cette nuit-là, Jacob ne fût plus Jacob. Il ne fût plus « celui qui prend par le talon », ni « celui qui supplante », mais il devint « celui qui lutte avec Dieu ». Depuis la migration d'Abraham de Chaldée, jusqu'à ce retour en Canaan, l'Eternel termina de poser les bases de cette alliance, et enfin, il donna un nom à la promesse : Israël.

A partir de cet instant, l'homme unique, le personnage singulier de Jacob devint le père en puissance de toute une nation. La naissance d'Israël s'est concrétisée à ce moment. Nous savons qu'Abraham est le patriarche de ce peuple car c'est à lui que le Seigneur à fait la promesse qu'il serait une grande nation. Néanmoins, la promesse en question n'était pas encore tangible, ni palpable. A travers Jacob, l’Éternel donna un nom à la promesse. Nous sommes donc passé des qualificatifs « une grande nation » et « postérité », à « tu seras appelé Israël ».

« Je t'ai établi père d'un grand nombre de nations. Il est notre père devant celui auquel il a cru, Dieu, qui donne la vie aux morts, et qui appelle les choses qui ne sont point comme si elles étaient » Romains 4 : 17.

La parole de Dieu est créatrice et agissante. Dans la genèse, nous remarquons que les choses se sont créés et ont commencé à exister rien à partir de sa parole. Quand il a appelé Jacob « Israël », il a appelé à la naissance de cette nation. Et en effet, de Jacob devenu Israël, sortira directement et de manière palpable les douze tribus qui composeront le peuple que l’Éternel s'est choisi.

Après ces événements, Israël alla à la rencontre d'Esaü qui, contre toute attente, le reçu en paix. Les deux frères se séparèrent et Israël arriva à Sichem où, il érigea un autel qu'il appela El-Elohé-Israël, « Dieu, le Dieu d'Israël » (Genèse 35 : 17-20). Auparavant quand il invoquait l’Éternel, Jacob le définissait toujours comme « le Dieu d'Abraham et d'Isaac » ou « le Dieu que sert Isaac ». Mais, en élevant cet autel, Israël assume son nouveau nom, accepte de servir l’Éternel son Dieu, et prend à son compte toutes les promesses faites à ses pères. Cette révélation précéda ce que l'on peut considérer comme le premier réveil d'Israël. En effet, il monta à Béthel où, après avoir fait la purge de tous les dieux étrangers présents dans sa maison (Genèse 35 : 1-7), il érigea un nouvel autel qu'il appela El-Béthel, « le Dieu de la maison de Dieu ». Ainsi, Israël se détourna entièrement des idoles pour se consacrer à l’Éternel, le Dieu véritable. Il montra son attachement non pas seulement à la maison de Dieu, mais au Dieu de la maison, se tournant ainsi vers Lui seul.

Par la suite, Joseph, 17 ans,le dernier fils d'Israël (Benjamin n'étant pas encore né) fut vendu par ses frères et emmené captif en Égypte par des marchand madianites où il devint esclave (Genèse 37). Néanmoins, la main de Dieu fût sur le jeune Joseph et ce dernier, au fil des années, fut prospère dans le pays d’Égypte. Il eut deux fils, Ephraïm et Manassé (Genèse 46 : 20). Après avoir reçu du Seigneur l'interprétation des songes de Pharaon, prévoyant une famine, Joseph fît faire des réserves en sorte que la famine ne toucha point le pays d’Égypte (Genèse 41). La famine fît son apparition et Joseph, ayant retrouvé les traces de sa famille, les fît tous venir en Égypte, y compris Israël. Le nombre total des gens de la maison de Jacob qui vinrent en Égypte fût de soixante-dix personnes (Exode 1 : 5).

\subsection*{Captivité égyptienne}

Joseph, proche de la mort, déclara ces paroles :
« Je vais mourir ! Mais Dieu vous visitera, et il vous fera remonter de ce pays-ci dans le pays qu'il a juré de donner à Abraham, Isaac, et Jacob » (Genèse 50 : 24).

Néanmoins, avant l'accomplissement de cette prophétie, va se jouer en Égypte l'une des périodes les plus sombres du peuple hébreux : la captivité égyptienne.

Comme vu précédemment, le nombre des gens de la maison de Jacob venus en Égypte fut de soixante-dix personnes en tout. La prophétie de Joseph commença à se concrétiser en Egypte. En effet, après la mort de Joseph, les hébreux se multiplièrent. Ils devinrent plus nombreux que les égyptiens eux-mêmes (Exode 1 : 6-9). Le pharaon d'alors, prit ainsi la décision de les réduire en esclavage.

Remarque : Plusieurs hypothèses ont été évoquées quant à l'identité du pharaon oppresseur qui s'opposa à Moïse. Certains estiment que ce dernier fût Thoutmôsis III (1479 à -1425), mais la majeure partie des historiens et exégètes évoquent le nom de Ramsès II (1304 – 1213 av JC). En effet, lors de son règne, ce dernier entreprit la construction d'une nouvelle capitale située au nord-est du pays, Pi-Ramsès. Il pourrait très certainement s'agir de la ville évoquée en Exode 1 : 11, alors en construction.

Le peuple élu de Dieu se retrouva donc sous la servitude égyptienne. Les esclaves de l'Egypte antique étaient principalement destinés aux travaux forcés, et à la construction de grands édifices égyptiens. Cependant, malgré leur condition, les enfants d'Israël augmentèrent encore. Le pharaon, tenta par deux fois de faire limiter le nombre des hébreux :

- Il demanda aux sages-femmes de tuer les enfants mâles, en vain (Exode 1 : 15-20)
- Il demanda à tout le peuple de noyer les enfants mâles (Exode 1 : 20-22)

\subsection*{Délivrance}

Durant ce génocide, une femme hébreu cacha son enfant, mais elle finit par le mettre dans un berceau et le laissa flotter sur le Nil. Il fût trouvé par la fille du pharaon qui l'appela Moïse, « sauvé des eaux », et l'éleva comme son propre fils à la cour du pharaon (Exode 2 : 1-10). Une fois adulte, ce dernier eût compassion de son peuple et tua un égyptien qui frappait un Hébreu. La chose se su et Moïse, se sachant menacé par pharaon, il s'enfuit jusqu'à l'est du désert d'Arabie, au pays de Madian (Exode 2 : 11-22).

Par la suite, « le Dieu d'Abraham, le Dieu d'Isaac et le Dieu de Jacob » se révéla à Moïse sur le mont Horeb lui indiquant qu'il serait l'instrument qu'il utiliserait pour délivrer son peuple (Exode 3 : 6-22). Et lorsque Moïse lui demanda plus précisément son nom, il lui répondit : « C'est ainsi que tu répondras aux enfants d'Israël : Celui qui s'appelle YHWH m'a envoyé vers vous ».

Remarque : (voir YHWH dans dictionnaire biblique)

Malgré les quatre-cents ans de servitude et d'esclavage, l’Éternel n'avait pas abandonné ni oublié son peuple. Il rappela à Moïse la promesse qu'il avait faite aux patriarches et la lui confirma.

Puis Moïse s'en alla auprès des Hébreux pour les délivrer au nom de l’Éternel. Face au refus de pharaon de faire partir les Hébreux, Dieu opéra de grands prodiges connus sous le nom des dix plaies d’Égypte (Exode 7 à 12). Dieu montra ainsi à son peuple qu'il est le Tout-puissant, El Shaddaï. A la fin de la neuvième plaie, Moïse donna un ultimatum à pharaon mais ce dernier refusa une nouvelle fois. Alors selon l'ordre de Dieu, les enfants d'Israël sacrifièrent un agneau et de son sang, en couvrirent les linteaux de leur portes (Exode 12 : 1-13).

Remarque : Ce sacrifice effectué par les Hébreux préfigurait le salut des hommes par le sacrifice de l'Agneau parfait, Jésus-Christ (Jean 1 : 29).

L'ange passa en Égypte et il prit la vie de tous les premiers nés égyptiens, et passa outre les Hébreux qui avaient mis le sang sur leurs portes. Pharaon, dépité par ce nouveau prodige, laissa aller les fils d'Israël et la prophétie de Genèse 15 : 13-14 s'accomplit.

Après le départ des Hébreux, pharaon les poursuivit mais l’Éternel opéra encore de grand miracles et ils traversèrent la Mer Rouge à pied sec pour finalement arriver dans le désert d'Arabie (Exode 14 à 16), où ils passèrent quarante-ans. Durant ces années, il y eut plusieurs événements majeurs comme la manne (Exode 16), l'instauration du sabbat (Exode 16), le don de la loi (Exode 20 à 24), ou encore l'édification du tabernacle, habitation de Dieu au milieu du peuple (Exode 25 à 31). Ce fut aussi un moyen pour Dieu d'éprouver les enfants d'Israël afin de voir s'ils garderaient ses commandements (Deutéronome 8 : 2).

\subsection*{Les douzes tribus}

Souvenons-nous qu'Israël eut douze fils (voir tableau précédemment). De ces fils naquirent les douze tribus de la nation d'Israël. A la fin sa vie, il cita chacun de ses fils (les tribus) et les bénis comme suit :

Ruben
Ruben, toi, mon premier-né, Ma force et les prémices de ma vigueur, Supérieur en dignité et supérieur en puissance, Impétueux comme les eaux, tu n'auras pas la prééminence ! Car tu es monté sur la couche de ton père, Tu as souillé ma couche en y montant (Genèse 49 : 3-4).

Siméon
Siméon et Lévi sont frères; Leurs glaives sont des instruments de violence. Que mon âme n'entre point dans leur conciliabule, Que mon esprit ne s'unisse point à leur assemblée ! Car, dans leur colère, ils ont tué des hommes, Et, dans leur méchanceté, ils ont coupé les jarrets des taureaux. Maudite soit leur colère, car elle est violente, Et leur fureur, car elle est cruelle ! Je les séparerai dans Jacob, Et je les disperserai dans Israël (Genèse 49 : 5-7).

Juda
Juda, tu recevras les hommages de tes frères; Ta main sera sur la nuque de tes ennemis. Les fils de ton père se prosterneront devant toi. Juda est un jeune lion. Tu reviens du carnage, mon fils ! Il ploie les genoux, il se couche comme un lion, Comme une lionne: qui le fera lever ? Le sceptre ne s'éloignera point de Juda, Ni le bâton souverain d'entre ses pieds, Jusqu'à ce que vienne le Schilo, Et que les peuples lui obéissent. Il attache à la vigne son âne, Et au meilleur cep le petit de son ânesse; Il lave dans le vin son vêtement, Et dans le sang des raisins son manteau. Il a les yeux rouges de vin, Et les dents blanches de lait (Genèse 49 : 8-12).
 
Zabulon
Zabulon habitera sur la côte des mers, Il sera sur la côte des navires, Et sa limite s'étendra du côté de Sidon (Genèse 49 : 13).

Issacar
Issacar est un âne robuste, Qui se couche dans les étables. Il voit que le lieu où il repose est agréable, Et que la contrée est magnifique; Et il courbe son épaule sous le fardeau, Il s'assujettit à un tribut (Genèse 49 : 14-15).

Dan
Dan jugera son peuple, Comme l'une des tribus d'Israël. Dan sera un serpent sur le chemin, Une vipère sur le sentier, Mordant les talons du cheval, Pour que le cavalier tombe à la renverse. J'espère en ton secours, ô Éternel ! (Genèse 49 : 16-18).

Gad
Gad sera assailli par des bandes armées, Mais il les assaillira et les poursuivra (Genèse 49 : 19).

Aser
Aser produit une nourriture excellente; Il fournira les mets délicats des rois (Genèse 49 : 20).

Nephthali
Nephthali est une biche en liberté; Il profère de belles paroles (Genèse 49 : 21).
 
Joseph (Ephraïm et Manassé ; voir ci-dessous)
Joseph est le rejeton d'un arbre fertile, Le rejeton d'un arbre fertile près d'une source; Les branches s'élèvent au-dessus de la muraille. Ils l'ont provoqué, ils ont lancé des traits; Les archers l'ont poursuivi de leur haine. Mais son arc est demeuré ferme, Et ses mains ont été fortifiées Par les mains du Puissant de Jacob: Il est ainsi devenu le berger, le rocher d'Israël. C'est l’œuvre du Dieu de ton père, qui t'aidera; C'est l’œuvre du Tout puissant, qui te bénira Des bénédictions des cieux en haut, Des bénédictions des eaux en bas, Des bénédictions des mamelles et du sein maternel. Les bénédictions de ton père s'élèvent Au-dessus des bénédictions de mes pères Jusqu'à la cime des collines éternelles: Qu'elles soient sur la tête de Joseph, Sur le sommet de la tête du prince de ses frères (Genèse 49 : 22-26).

Benjamin
Benjamin est un loup qui déchire; Le matin, il dévore la proie, Et le soir, il partage le butin. Ce sont là tous ceux qui forment les douze tribus d'Israël. Et c'est là ce que leur dit leur père, en les bénissant. Il les bénit, chacun selon sa bénédiction (Genèse 49 : 27).

Remarque : Selon Josué 14 : 4, Ephraïm et Manasée constituèrent les deux tribus issues de Joseph. Les fils de Lévi furent consacrés au service dans le tabernacle (Josué 13 : 14) raison pour laquelle ils ne constituèrent pas de tribu

\subsection*{Conquête de la terre promise}

Après les quarante années passées dans le désert d'Arabie, les Hébreux arrivèrent à Jéricho et ils prirent la ville (Josué 2 à 6).

Remarque : Le récit de cette première conquête est extraordinaire. En effet, ce fut à travers les louanges des assaillants et les sons des shofars des sacrificateurs que les murailles de la ville s'écroulèrent. Cela rappelle ce verset de la Parole : « Ce n'est point par l'armée, ni par force, mais par mon Esprit, a dit l’Éternel des armées » (Zacharie 4:6). En d'autres termes : « Vous vainquez, non par votre propre force, mais parce que c'est moi qui combat pour vous ».

Puis, après une première tentative infructueuse due au péché d'Acan, les Hébreux réussirent à conquérir la ville d'Aï, située à l'est de Canaan (Josué 7 : 8).
Par la suite, après la soumission des habitants de Gabaon (ville située à environ dix kilomètres au nord de Jérusalem), Josué et les israélites vainquirent une coalition de cinq rois et leurs armées :

- Adoni-Tsédek, roi de Jérusalem
- Hoham, roi d'Hébron (ville située au sud de l'actuelle Palestine)
- Piream, roi de Jarmuth (ville située à l'ouest de Bethléem)
- Japhia, roi de Lakis (ville située à l'ouest d'Hébron)
- Debir, roi d'Eglon (ville située au sud-est de Lakis)

C'est à Makkeda, (ville située entre Lakis et Eglon) que Josué exécuta les cinq rois (Josué 10 : 1-28). Puis, par une succession de victoires, il conquit le sud et le nord de Canaan.

Répartition des terres pour chaque tribu : Josué 13-19 (voir illustration)

« C'est ainsi que l’Éternel donna à Israël tout le pays qu'il avait juré de donner à leur pères ; ils en prirent possession et s'y établirent » Josué 21 : 43.

Le peuple avait enfin sa terre, celle que Dieu lui avait promise depuis le temps d'Abraham.

« Maintenant, craignez l’Éternel, et servez-le avec intégrité et fidélité. Faite disparaître les dieux qu'ont servis vos pères de l'autre côté du fleuve et en Égypte, et servez l’Éternel. Et si vous ne trouvez pas bon de servir l’Éternel, choisissez aujourd'hui qui vous voulez servir, ou les dieux que servaient vos pères au-delà du fleuve, ou les dieux des Amoréens dans le pays desquels vous habitez. Moi et ma maison, nous servirons l’Éternel. Le peuple répondit, et dit : Loin de nous la pensée d'abandonner l’Éternel, et de servir d'autres dieux ! […] Ôtez donc les dieux étrangers qui sont au milieu de vous, et tournez votre cœur vers l’Éternel, le Dieu d'Israël » Josué 24 : 14-16.

Après ces ordonnances, et la promesse du peuple de servir l’Éternel, Josué mourut âgé de cent dix ans.

\subsection*{Période des Juges}

La période des Juges marque une ère de troubles spirituels. Elle décrit les premières années de la vie des israélites dans la terre promise.
Après la mort de Josué, les tribus d'Israël connurent plusieurs victoires en Canaan, mais elles désobéirent à l’Éternel en laissant les cananéens demeurer dans le pays. C'est ainsi que du fait de leur désobéissance (Juges 2 : 1-3), les enfants d'Israël habitèrent au milieu de leurs ennemis.
Pendant la vie de Josué, ils servirent l’Éternel. Mais à la mort de ce dernier, ils se corrompirent et se mirent à servir les dieux des nations qu'ils avaient laissés habiter dans le pays de Canaan (Josué 16 : 10). Les hommes de la génération de Josué moururent, et leur descendants abandonnèrent le Dieu de leur père et se tournèrent vers Baal et les Astartés (Josué 2 : 10-15). Notons que l'obéissance et la fidélité du peuple induisait immanquablement la victoire et la paix parce que l’Éternel était avec eux. La désobéissance au contraire, impliquait automatiquement la défaite et les guerres à répétitions (Juges 2 : 13-15). Ainsi, Dieu les punis en les assujettissants à leur ennemis d'alentour parmi lesquels les Philistins, les Sidoniens, les Héviens, et les autres cananéens. Au lieu de profiter paisiblement du pays de la promesse qui leur était maintenant acquis, ils vécurent des années de troubles et de guerres à cause de leur corruption et ils furent parfois soumis aux peuples païens qui devaient leur être soumis.
Néanmoins Dieu, dans son amour, suscita en Israël des juges (voir tableau ci-dessous) capables d'amener un réveil spirituel au sein de la nation et de délivrer le peuple de ses ennemis. Quand ce dernier mourrait, le peuple retombait dans la désobéissance, ce qui impliquait la servitude aux nations païennes, jusqu'à qu'à la venue d'un nouveau juge. Cette période fut donc marquée par les défaites et les délivrances d'Israël en fonction de leur obéissance envers l’Éternel.


Ci-dessous la liste des juges par ordre chronologique :

I - Othniel
(Juges 3 : 9-11)
II - Ehud
(Juges 3 : 12- 30)
III - Schamgar
(Juges 3 : 31)
IV – Déborah / V - Balak
(Juges 4 et 5)
VI - Gédéon
(Juges 6 à 9)
VII - Thola
(Juges 10 : 1-2)
VIII - Jaïr
(Juges 10 : 3-5)
IX - Jephté
(Juges 11 et 12)
X - Ibtsan
(Juges 12 : 8-10)
XI - Elom
(Juges 12 : 11-13)
XII - Abdon
(Juges 12 : 13-15)
XIII - Samson
(Juges 13 à 16)

Othniel délivra les israélites du roi de Mésopotamie, Cushan-Risheathaïm. Ehud, libéra les Hébreux des moabites et Schamgar les délivra des Philistins. Barak et Déborah les libéra de Jabin, roi de Canaan, tandis que Gédéon les délivra des madianites.
Le plus connu de tous les juges est sans conteste Samson, dont la force donné par l'Esprit de Dieu lui permit de commencer à libérer Israël d'une nouvelle servitude des Philistins.

\subsection*{Période des Rois}

Après les événements consécutifs à la période des juges, Israël, alors gouverné par l’Éternel, demanda à être gouverné par un roi, au même titre que les nations païennes. La volonté du peuple de se détourner d'un gouvernement divin au profit d'un gouvernement humain montre bien l'apostasie qui avait gagné le cœur du peuple au fil des siècles. Samuel, dernier et juge authentique de Dieu (il établit ses fils juges mais ces derniers furent apostats ; I Samuel 8 : 1-5), s'en attrista mais l’Éternel lui dit ces paroles :

« L’Éternel dit à Samuel : Écoute la voix du peuple dans tout ce qu'il te dira, car ce n'est pas toi qu'ils rejettent, c'est moi qu'ils rejettent, afin que je ne règne pas sur eux. Ils agissent à ton égard comme ils ont toujours agi depuis que je les fais monter d’Égypte jusqu'à ce jour ; ils m'ont abandonné, pour servir d'autres dieux. Ecoute donc leur voix ; mais donne leur des avertissements, et fais-leur connaître le droit du roi qui régnera sur eux » I Samuel 8 : 7-9.

Saül : l'oint rejeté
Selon l'ordre de Dieu, Samuel s'en alla à Tsuph oindre Saül, fils de Kis de la tribu de Benjamin, et ce dernier, fût proclamé roi devant toute l'assemblée d'Israël à Mitspa. Ce fut le premier roi de l'histoire d'Israël. Selon la description qu'en font les écrits, Saül fût un jeune homme grand, fort, vaillant et beau (I Samuel 9 : 1-2).
Son premier fait d'arme fût sa victoire contre les ammonites qui avaient auparavant assiégés Jabès, une localité située au nord-est d'Israël, dans la tribu de Manassé (I Samuel 11 : 1-11). Ainsi sa royauté s'affirma aux yeux du peuple.
Néanmoins, lors d'une bataille contre les philistins à Guilgal, Saül désobéit à l’Éternel, et par la voix de Samuel, Dieu le prévint de la proche fin de son règne, et d'un autre homme qu'il s'est réservé pour lui succéder (I Samuel 13 : 13-14).
Il est en effet important de noter que l'instauration de la royauté en Israël ne voulait pas dire que Dieu n'en était plus le chef, bien au contraire. Saül, et nous le verront plus tard, David, furent tous deux établis rois par l’Éternel. Mais l'un choisit de suivre sa propre voix, et de marcher sans Dieu, et l'autre considéra les intérêts et la loi de l’Éternel. Le résultat qui en découla fut que la maison de Saül perdit la royauté, et que le règne de David fût affermi à jamais (II Chroniques 7 : 17-18). L'obéissance et la droiture d'un roi furent les gages d'un règne réussi et prospère. A contrario, les rois apostats furent déchus et rejetés par Dieu.
Saül, par orgueil et par crainte du peuple, détourna son regard de l’Éternel et continua à lui désobéir. Dieu lui parla de nouveau à travers Samuel le prophète en lui affirmant son rejet (I Samuel 15).

Remarque : Le règne de Saül préfigura celui de bon nombre de rois (Achab, Omri, Manassé, Sédécias etc) qui se détournèrent de l’Éternel.

David : le roi selon Dieu

Plus tard, à l'insu de Saül, l’Éternel envoya Samuel en Juda, dans la maison d'Isaï, à Bethléem pour oindre son plus jeune fils, David, comme nouveau roi (I Samuel 16 : 1-12). Notons que contrairement à Saül, David fût choisi par l’Éternel en fonction de son cœur, et non pas par sa force ni par sa taille (I Samuel 13 : 14). Le critère d'intégrité morale et l'attachement à l’Éternel furent cruciaux et primordiaux lors de la désignation de David. Dieu voulait un roi droit, et qui le suive, et non plus un roi fort, mais dont le cœur est éloigné de lui. En plus de sa droiture de cœur, David était vaillant, fort, et prêt à la guerre (I Samuel 16 : 18 ; I Samuel 17 : 34-36). Il eut l'occasion de démontrer sa bravoure lorsque les philistins montèrent contre Israël à Azéka, ville située à la frontière nord de Juda. En effet, lorsque Goliath, un guerrier de près de trois mètres de haut s'avança pour défier les israélites, David fût le seul qui osa s'approcher : « David dit au Philistin: Tu marches contre moi avec l'épée, la lance et le javelot; et moi, je marche contre toi au nom de l'Éternel des armées, du Dieu de l'armée d'Israël, que tu as insultée. Aujourd'hui l'Éternel te livrera entre mes mains, je t'abattrai et je te couperai la tête; aujourd'hui je donnerai les cadavres du camp des Philistins aux oiseaux du ciel et aux animaux de la terre. Et toute la terre saura qu'Israël a un Dieu. Et toute cette multitude saura que ce n'est ni par l'épée ni par la lance que l'Éternel sauve. Car la victoire appartient à l'Éternel. Et il vous livre entre nos mains. Il mit la main dans sa gibecière, y prit une pierre, et la lança avec sa fronde; il frappa le Philistin au front, et la pierre s'enfonça dans le front du Philistin, qui tomba le visage contre terre. Ainsi, avec une fronde et une pierre, David fut plus fort que le Philistin; il le terrassa et lui ôta la vie, sans avoir d'épée à la main » I Samuel 17 : 45-50.

Cette victoire offrit à David une renommée retentissante dans tout Israël, et sa célébrité augmenta car il revenait à chaque fois victorieux de toutes ses campagnes militaires. Saül, rempli de jalousie à son égard chercha plusieurs fois à le faire mourir. David dû s'enfuir loin de la maison du roi et lorsque ce dernier le poursuivait, plusieurs fois David l'épargna alors que Saül fut à sa merci.
Jamais Saül ne réussit à faire mourir David, et le roi, rejeté par Dieu mourut à Guilboa, lors d'une bataille contre les philistins (I Samuel 31).
A la mort de Saül, David retourna à Juda où il fût couronné roi par les habitants, tandis que dans le reste du pays, ce fût Isch-Boscheth, fils de Saül qui régnait. Une guerre civile éclata entre Juda et le reste d'Israël, et David fut vainqueur. Peu de temps après, Récab et Baana, deux chefs de bandes appartenant au fils de Saül se retournèrent contre lui et l'assassinèrent. David fût irrité de l'apprendre et punit les meurtriers (II Samuel 4).

C'est alors que tout Israël vint vers David, et le peuple oignit le nouveau roi, alors âgé de trente ans. Il régna en tout quarante ans : sept ans à Juda, et trente-trois ans sur tout Israël (II Samuel 5 : 4-5). Il battit les Jébusiens à Jérusalem et en fît la capitale du pays (II Samuel 5 : 6-9). Notons aussi que l'arche de l'alliance fût transportée dans la capitale (II Samuel 6).

Sous le règne de David, les israélites connurent une succession de victoires sur leurs ennemis. Tant sur les philistins, les ammonites, les moabites, syriens ou encore les édomites, les hébreux furent à de nombreuses reprises vainqueurs. L’Éternel affermit le royaume de David, bien que son règne fût secoué par la révolte de son fils Absalom. Cette dernière se soldera par la mort du fils rebelle (II Samuel 15 à 18). 

Salomon : Grandeur et décadence

Après la mort de David, Salomon, son fils, accéda au trône. Le royaume fût alors au paroxysme de sa puissance. Tous les ennemis alentours furent soumis à Israël et le territoire de Salomon s’étendit de l'Euphrate, jusqu'à la frontière égyptienne. Israël n'eût jamais été aussi puissant et affermi qu'au règne de Salomon. Aussi, de part la sagesse que l’Éternel lui donna, le roi su administrer le pays et faire asseoir le royaume de son père (I Rois 3 : 3-14). Salomon régna durant quarante ans (I Rois 11 : 42).
C'est alors que pendant le règne de Salomon, une autre partie de la promesse faite à Abraham se réalisa : « je te bénirai et je multiplierai ta postérité, comme les étoiles du ciel et comme sur le sable qui est sur le bord de la mer ; et ta postérité possédera la porte de ses ennemis » Genèse 22 : 17.

En effet, le peuple était devenu nombreux, comme le sable de la mer (I Rois 4 : 20) et Israël avait vaincu tous ses ennemis (I Rois 4 : 21). Aussi, Salomon fût même celui qui construit le temple de l’Éternel (I Rois 6). Cette période correspond à l'âge d'or de l'Israël antique.

Néanmoins, Salomon aima beaucoup de femmes étrangères parmi lesquelles des moabites, des ammonites, des sidoniennes, des édomites ou encore des hétiennes. Et il fût très certainement l'un des rois les plus polygames de l'histoire avec ses sept cent femmes et ses trois cent concubines ! Jusqu'alors, Salomon avait servi l’Éternel comme son père David avant lui et comme nous l'avons vu, son royaume connut un âge de prospérité sans précédents. Mais son idolâtrie des femmes, amena celles-ci à détourner son cœur de l’Éternel (I Rois 11 : 1-7). Pour un homme qui aima les femmes à l'excès, il n'est pas étonnant que celles-ci aient réussi à tourner son cœur vers Astarté, déesse de la fertilité (voir DICO). Aussi ses femmes ammonites tournèrent son cœur vers Milcolm, dieu païen auprès duquel les fidèles effectuaient des sacrifices d'enfants, ce que l’Éternel réprouvait totalement (Lévitique 18 : 21).

Remarque : Bien des siècles plus tard, pour avertir les enfants d'Israël, Néhémie prendra l'exemple de la déchéance de Salomon (Néhémie 13 : 26).

En conséquence de sa corruption, l’Éternel prévint le roi de l'effondrement de son royaume par le biais d'un schisme entre son fils, Roboam, et le serviteur de Salomon, Jéroboam (I Rois 11 : 9-13).

L’Éternel avait amené Israël à ne pas se mélanger aux peuples païens car il savait très bien que le peuple se corromprait. C'est parce que Salomon désobéit à l'ordre de Dieu de ne pas s'allier avec des faux dieux, que le schisme se produisit.

\subsection*{Schisme : division du royaume d'Israël et de Juda}

Après la mort de Salomon, Israël fut divisé en deux royaumes

1) Royaume d'Israël, objet d'idolâtrie dès sa fondation.

Le royaume d'Israël fût composé des dix tribus suivantes : Ruben, Gad, Aser, Nephtali, Siméon, Isaacar, Zabulon, Dan, Ephraïm, Manassé.
 
Jérusalem, la ville sainte, fût le centre de la vie politique et religieuse de tout le royaume avant le schisme. Les enfants d'Israël adoraient l’Éternel aux abords du temple. Or, Jérusalem se trouva en Juda, royaume de Roboam.
Les juifs du nord se retrouvèrent donc sans temple pour adorer Dieu. Jéroboam, dans un souci de conservation du pouvoir, développa le culte du veau d'or en Israël et en fît deux hauts lieux.
Pour donner une légitimité à ces faux dieux, il utilisa le lieu saint de Béthel à Sichem, pour ériger un des veaux, et érigea l'autre dans la tribu de Dan. Aussi, il se fît des sacrificateurs qui n'étaient pas Lévites et ces derniers officièrent auprès de ces autels païens. Sa volonté fût qu'Israël ne retourna pas en Jérusalem de peur que son cœur ne se tourne vers le royaume de Juda. Ainsi, le royaume d'Israël devint une terre d'idolâtrie ou la crainte de l’Éternel avait disparue. La très grande majorité des rois qui succédèrent à Jéroboam suivirent son apostasie et entraînèrent davantage le peuple dans le péché et la transgression.
Le résultat de cette idolâtrie excessive, fût l'annonce de la prochaine dispersion et déportation du royaume d'Israël (I Rois 14 : 15-16).

Remarque : Ce fût Omri, roi d'Israël qui bâtit la ville de Samarie (II Rois 16 : 23-24).

Rois du royaume d'Israël
Jéroboam (régna 22ans) x
Nadab (régna 2 ans) x
Baescha (régna 24 ans) x
Ela (régna 2 ans) x
Zimri (régna 7 jours) x
Omri (régna 12 ans) x
Achab (régna 22 ans) x
Achazia (régna 2 ans) x
Joram (régna 12 ans) x
Jéhu (régna 28 ans) x
Joachaz (régna 17 ans) x
Joas (régna 16 ans) x
Jéroboam II (régna 41 ans)
Zacharie (régna 6 mois) x
Schallum (régna 1 mois) x
Menahem (régna 10 ans) x
Pekachia (régna 2 ans) x
Pékash (régna 20 ans) x
Osée (régna 9 ans) x

Sur les dix-neuf rois qui régnèrent sur Israël, tous furent apostats et s'éloignèrent de l’Éternel.

Déportation
Au IXe siècle AV JC, le roi de Syrie, Hazaël commença à entamer les frontières d'Israël (II Rois 10 : 32-33).
Mais c'est en 721 AV JC, alors qu'Osée était roi d'Israël, que le roi d'Assyrie, Salmanasar V assiégea la Samarie, capitale d'Israël et tout le peuple fût emmené captif en Assyrie (II Rois 17 : 1-23). Le pays fût colonisé par des nations païennes, notamment des babyloniens (II Rois 17 : 24). D'après les archéologues, la population totale du royaume d'Israël fût d'environ deux-cents mille personnes.

Royaume de Juda : Entre apostasie et réveil spirituel

Le royaume de Juda fût composé des deux tribus suivantes : Juda et Benjamin.

Le schisme fût le vecteur d'une grande détresse spirituelle, particulièrement en Israël ou la très grande majorité des rois abandonnèrent l’Éternel. Juda connut lui aussi des périodes de profondes apostasies et Roboam, qui fût le premier roi de ce nouveau royaume, s'éloigna de l’Éternel (I Rois 14 : 21-24).
Sous son règne, les enfants d'Israël élevèrent des hauts lieux pour la vénération des idoles païennes que Salomon avait déjà introduites dans le pays. Les rois qui succédèrent à Roboam s'élancèrent eux aussi dans cet élan d'apostasie bien que certains d'entre eux entreprirent des réformes morales pour restaurer le véritable culte de l’Éternel à l'image d’Ézéchias ou encore Josias.

Rois du royaume de Juda
Roboam (régna 17ans) x
Abijam (régna 3ans) x
Asa (régna 41 ans) V
Josaphat (régna 25 ans) V
Joram (régna 8 ans) x
Achazia (régna 1 an) x
Athalie (régna 6 ans) x
Joas (régna 40 ans) V
Amastia (régna 29 ans) V
Azaria (régna 52 ans) V
 Jotham (régna 16 ans) V
Achaz (régna 16 ans) x apostasie
Ézéchias (régna 29 ans) V réveil
Manassé (régna 55 ans) x
Amon (régna 21 ans) x
Josias (régna 31 ans) V
Joachaz (régna 3 mois) x
Jojakim (régna 11 ans) x
Jojakin (régna 3 mois) x
Sédécias (régna 11 ans) x

Sur les vingt rois qui régnèrent sur le royaume de Juda, seuls sept firent « ce qui est droit aux yeux de l’Éternel ». Ézéchias fut un roi intègre devant Dieu (II Rois 18 : 1-6). C'est d'ailleurs sous son règne que l'on détruisit les autels et haut lieux païens qu'avaient érigés Roboam. Malheureusement, après ce règne marqué par la justice et la restauration spirituelle, Manassé, fils d’Ézéchias, rétablit les cultes païens dans le royaume de Juda, et replongea le peuple dans les ténèbres de l'idolâtrie (II Rois 21 : 1-18). Ses descendants firent de même jusqu'aux déportations babyloniennes (voir chapitre sur les déportations), à l’exception du roi Josias. Il fut le seul roi véritablement intègre tout au long de sa vie. Il restaura le temple (II Rois 22 : 3-7) ; il lut la loi devant le peuple et fit alliance avec Dieu (II Rois 23 : 1-3). Il chassa les prêtres de Baal et d'Ashéra, et ôta du temple tout les ustensiles païens (II Rois 23 : 4-6) et exerça d'autres réformes en ce sens (II Rois 23 : 7-20).

\subsection*{Déportations}

Captivité : -609 à -539 = 70 ans

1ere déportation :

Première déportation : 609 av JC (II Chroniques 36 : 6-7 ; Daniel 1 : 1).
A partir de l'an -609, les juifs subirent leur première déportation.
Nebucadnetsar appelé aussi Nabuchodonosor II roi de Babylone, se leva avec son armée contre Jojakim, roi de Juda. Il le battit et le fit prisonnier. Il emmena le roi, et toute l'aristocratie juive en captivité à Babylone. Les nobles furent aussi déportés et l'armée babylonienne déroba les ustensiles utilisés pour le service dans le Temple. Ces derniers furent emmenés à Babylone et placés dans les temples babyloniens. Après sa victoire, Nebucadnetsar mit sur le trône de Juda Jojakin, fils de Jojakim. C'est d'ailleurs pendant cette période que Daniel fût déporté à Babylone (Daniel 1:1).

Deuxième déportation : -597 – (II Rois 24 : 8-16)
Jojakin, se rebella contre le roi babylonien. Nebucadnetsar monta contre lui, mata la rébellion et déporta en tout, près de dix-huit mille hommes et femmes (II Rois 24 : 14-16) dont des guerriers, des hommes vaillants, des charpentiers, et des serruriers. Ils furent emmenés captifs à Babylone. Nebucadnetsar dépouilla une nouvelle fois le Temple de ses trésors et ustensiles (II Rois 24 : 13-16). Il emmena aussi Jojakin en captivité et mit son fils, Sédécias, sur le trône de Juda.

Troisième déportation : -586 (II Chroniques 36 : 11-21)
En 588 av JC, après neuf ans de règne et malgré les avertissements du prophète Jérémie (Jérémie 38 : 17-28), le roi de Juda, Sédécias, se révolta contre Nebucadnetsar. Ce dernier, monta la même année contre le roi de Juda, et assiégea la ville de Jérusalem pendant deux ans. Le siège, très éprouvant pour les hébreux, se solda par la prise de la ville et Jérusalem fût entièrement brûlée en 586 av JC. Le temple quant à lui, ne fut plus seulement pillé comme lors des précédents déportations, mais aussi entièrement détruit (II Rois 25 : 1-9). Les ustensiles d'airain utilisés pour le service dans le Temple furent de nouveau dérobés et emmenés par l'armée babylonienne (II Rois 25 : 14-17). Quant au reste du peuple qui resta dans la ville et qui se rendit à Nebucadnetsar, il fût à son tour déporté vers Babylone. Seuls les pauvres et les indigents restèrent dans une Jérusalem en ruine (II Rois 25 : 11-12).

\subsection*{Retour à Jérusalem}

En -539, Cyrus II (580 av. JC. - 530 av. J-C) roi de Perse, entreprit une campagne militaire contre l'empire babylonien. Il entra dans Babylone, et son armée battit les babyloniens. Il prit le contrôle de la ville et destitua Nabonide le roi babylonien, scellant ainsi la fin du règne de Babylone selon les prophéties du prophète Jérémie (Jérémie 25 : 1-12 ; Jérémie 50 à 52). Un an après sa victoire, en 538 av JC, Cyrus, convaincu par l’Éternel (Esdras 1 : 1-4) édita un décret permettant aux juifs de retourner à Jérusalem pour reconstruire la ville et le Temple. Il accomplit ainsi la prophétie d'Isaïe qui, deux siècles avant la naissance du roi de Perse, avait prophétisé son action (Isaïe 45 : 1-4) ! A cette occasion, les exilés reçurent en grande quantité des provisions, de l'argent, de l'or, des effets, du bétail et des offrandes volontaires.
Le nombre des hébreux revenus de l'exil fût de quarante-deux mille trois cent soixante personnes, dont quatre mille deux-cent quatre-vingt-neuf sacrificateurs (Esdras 1 : 36-39), des Lévites (Esdras 1 : 40-54), et s'ajoutèrent à ce nombre sept mille trois cent trente-sept serviteurs et servantes (Esdras 1 : 54-67). Aussi, dans l'élan de la reconstruction du Temple, le roi de Perse rendit aux juifs tous les ustensiles nécessaires au sacerdoce que Nebucadnetsar avait emmené à Babylone soixante-dix ans plus tôt.
La reconstruction du Temple débuta en -535 av JC dans une Jérusalem en ruine, sous l'impulsion de Zorobabel, le nouveau gouverneur de Juda, nommé par Cyrus. Ils construisirent l'autel, effectuèrent les premiers sacrifices en l'honneur de l’Éternel, et posèrent les fondements du Temple (Esdras 3 : 1-13). Aussi, plusieurs prophètes tels que Aggée et Zacharie, furent témoins de la reconstruction du Temple. Ils encouragèrent Zorobabel et Josué, le grand sacrificateur dans leurs efforts pour la reconstruction (Aggée 1 à 2 ; Zacharie 1 à 14 ; Esdras 5 : 1-5).
Après plusieurs oppositions (Esdras 4) la reconstruction du Temple fut achevée en 516 av JC, soit plus de vingt ans après le retour des exilés à Jérusalem (Esdras 6 : 13-22).
En 455 av JC, plus de cinquante ans après la reconstruction du Temple, Artaxerxes, roi de Perse de 465 à 424 av JC, autorisa Esdras le scribe, à officier à Jérusalem. Ce dernier, érudit et attaché à la loi de Dieu, était accompagné de plusieurs sacrificateurs, Lévites, et chantres (Esdras 7). Il s'appliqua à rétablir les sacrifices au sein du Temple et suscita un grand réveil spirituel à Jérusalem (Néhémie 8 à 10). Plus tard, en -445, le roi de Perse permis à Néhémie de retourner à Jérusalem pour reconstruire la muraille de la ville (Néhémie 1 : 1-8).
 
Remarque : Fait remarquable, le retour à Jérusalem fut motivé par la reconstruction du Temple selon l'édit de Cyrus. Cet événement, marqua l'envie du peuple de se réconcilier avec l’Éternel, son Dieu. Aussi, pour qu'ils puissent se sentir chez eux, il fallait qu'ils commencent par l'honorer. Quand une personne s'installe dans une maison, elle commence à y installer les éléments essentiels à la vie. De même, les juifs une fois de retour commencèrent par reconstruire le Temple, élément essentiel de la vie religieuse en Israël. Malgré les soixante-dix ans de persécution, la loi de l’Éternel n'avait pas quitté leur cœur, même exilés à Babylone. Certains juifs comme Daniel, Esdras, Zorobabel, ou encore Josué, ne se détournèrent point des commandements du Seigneur, et ils symbolisèrent le reste d'Israël resté fidèle à Dieu. Le culte de l’Éternel est le fondement de l'identité juive. L'histoire du peuple est lié à Dieu qui les a choisi et fait monter hors d’Égypte. En effet il déclare en Exode 6 : 7 : « Je vous prendrai pour mon peuple, je serai votre Dieu, et vous saurez que c'est moi, l'Éternel, votre Dieu, qui vous affranchis des travaux dont vous chargent les Égyptiens ». Dieu est la source de l'identité juive, et le sceau de leur liberté. Le désir de la reconstruction du temple, avant même la ville et les murailles, représente bien le lien qui unit le peuple juif à l’Éternel. C'est donc parce que les juifs se sont tournés vers l’Éternel que ce dernier commença à les délivrer de la servitude babylonienne. Si la principale motivation aurait été la reconstruction de la ville, le retour en terre promise n'aurait probablement pas eu lieu car la volonté du Seigneur était que le peuple se détourne de ses mauvaises voies, et revienne tout entier à lui. Néanmoins, les infidélités passées du peuple vis à vis du Seigneur eurent des conséquences désastreuses sur la vie religieuse. En effet, l'arche de l'alliance qui résidait dans la partie la plus sainte du temple, le saint des saints, avait disparue. Celle-ci, emportée par les babyloniens lors de la prise de Jérusalem ne fût jamais retrouvée. Plusieurs hypothèses existent quant à sa disparition mais jusqu’aujourd’hui, personne ne sait ce qu'il en advint. La disparition de l'arche fût un réel désastre quand l'on considère le fait c'était le lieu où l’Éternel faisait manifester sa gloire. De part de son caractère unique, l'arche ne put être remplacée. Le Temple était reconstruit, mais la gloire de l’Éternel ne revint pas à l'intérieur de ce dernier.

\subsection*{Entre Malachie et Matthieu}

Entre le IVe siècle et l'an 20 av JC, l'histoire juive continua à se dessiner.
En 331 av JC, Alexandre le Grand (356-323 av JC) défit le roi de Perse Darius III (380-330 av JC) mettant fin à l'empire perse fondé par Cyrus II au VIe siècle av JC. Les juifs de Jérusalem qui avaient trouvé une semi indépendance sous l'empire perse furent maintenant sous la domination de l'empire gréco-macédonien fondé par Alexandre le Grand. Néanmoins, le maître de ce nouvel empire n'opprima pas les juifs et ces derniers purent continuer à pratiquer le culte dans le Temple de Jérusalem. Après la mort d'Alexandre, ses principaux généraux appelés « Diaboques » se partagèrent l'immense empire et fondèrent leurs propres dynasties. Antiochus IV Epiphane (vers 215-173 av JC) était l'un des descendants de Séleucos, un diaboque qui fonda la dynastie des Séleucides (305-64 AV JC). Antiochus voulut uniformiser son empire sous l'égide de la culture grecque, et pour ce faire il tenta d'imposer le mode de vie et la pensée grecque aux peuples de son empire. Aussi, il les obligea à se conformer à la religion polythéiste grecque et donc à adorer les dieux de son panthéon. Mais en 168 av JC, face au refus des juifs, attachés à loi de Moïse, de se soumettre à ces dogmes païens, Epiphane entra à Jérusalem et massacra toute une partie de la population. Il déroba les ustensiles utilisés pour le service dans le temple puis, comble de l'ignominie, il le profana en y faisant sacrifier une truie (animal considéré comme étant impur, Deutéronome 14 : 8) sur l'autel, qu’il érigea au dieu païen Jupiter ! Le culte de l’Éternel fut dès lors interdit et les juifs survivants furent contraints de se soumettre à l'adoration des dieux grecs.
C'est alors que survint la révolte des Macchabées, une famille de prêtres juifs déterminés à redonner l'indépendance à Israël et à rétablir le véritable culte à Jérusalem. En 165 av JC, Matthias Macchabée parvint à lever une petite armée et avec ses fils dont Juda, il reconquit la ville de Jérusalem. Il ôta toutes les idoles du Temple, et le consacra de nouveau à l’Éternel.
En 63 av JC, le pouvoir romain, dominant dans la région, plaça Antipater comme procurateur de Judée. En 37, son fils, Hérode le Grand (173 – 4 AV JC), se proclama « roi des juifs » et régna sur le territoire de Judée de -37 jusqu'à sa mort en l'an 4 av JC. Ce dernier, pour se donner une légitimité au sein des juifs, qui le haïssaient, entreprit une refonte du temple en 19 av JC.

Remarque :

1. C'est le fils de Hérode le Grand qui ordonna le massacre des enfants juifs à l'époque de Jésus-Christ (Matthieu 2 : 16-18). Son fils, Hérode Archélaüs (Matthieu 2 : 22) régna huit ans après la mort de son père, mais ce fut son frère, Hérode Antipas qui régna en Judée pendant le ministère de Jésus. C'est ce roi qui ordonna le meurtre de Jean-Baptise (Marc 6 : 14-29).

2. A l'époque de Jésus, Israël fut toujours sous le contrôle des romains. Le pays fut divisé en deux parties : Au nord la Galilée et au sud la Judée. La vie religieuse juive était dominée par une secte juive, les pharisiens. Jésus eut à les confronter de nombreuses fois (Jean 8). Le temple en pleine reconstruction faisait la fierté des religieux juifs bien que la gloire de l’Éternel n'y était plus depuis la vision d'Ezéchiel (Ezéchiel 10). Il s'agissait donc d'une période de disette spirituelle avant la venue du Libérateur. Accomplissant la prophétie de Michée 5 : 1, Jésus, le messie, naquit à Béthléem dans le territoire de Judée, mais il grandit à Nazareth, en Galilée (Matthieu 2 : 22-23). Ainsi, la partie sud d'Israël eut la grâce de voir naître en son sein le messie, tandis que la partie nord eut le privilège de le voir grandir. Néanmoins, le peuple juif ne le reconnu pas et et Jésus prophétisa la destruction de Jérusalem comme conséquence de son incrédulité (Luc 19 : 41-44).

\subsection*{Exil massif des juifs}

Après la résurrection de Jésus et pendant le ministère des apôtres, la reconstruction du temple se poursuivit et prit fin en 63 ap JC sous le règne d'Agrippa II. En 66, les juifs de Judée se révoltèrent contre l'empire romain. Quatre ans plus tard en avril 70, Titus (39-81 AP JC) général romain et fils de l'empereur Vespasien (9-79 AP JC), assiégea la ville de Jérusalem. Le 30 août, ses troupes parvinrent à entrer dans le temple qui fût entièrement brûlé, accomplissant ainsi la parole prophétique du Seigneur Jésus-Christ (Matthieu 24 : 1-2). En septembre de la même année, les romains rasèrent complètement la ville et Titus fier de sa victoire, fît mettre à mort des milliers de juifs.
En 132 ap JC, une nouvelle révolte éclata à l'initiative de Shimon bar Koziba, un patriote juif voulant reconstruire le temple. L'insurrection fût écrasée trois ans plus tard par l'empereur Hadrien qui décida de détruire une nouvelle fois Jérusalem pour y bâtir une cité romaine qu'il nomma Aelia Capitiolina. Tous les juifs furent alors expulsés de la ville et remplacés par des citoyens romains, grecs et syriens. En outre, Hadrien érigea un temple dédié au dieu Jupiter (Zeus grec) à l'emplacement même des ruines du temple de Jérusalem et il appela la région « Palestine » en hommage à l'ancien peuple Philistin. Ce fût le commencement de l'exil massif des juifs hors de Judée.

Au IVe siècle, l'empereur Constantin se convertit au christianisme et fît de la Palestine une région dominée par les chrétiens. Il redonne à la ville son nom d'origine, Jérusalem. Trois siècles plus tard, en 638, les arabes musulmans conquirent la région et prirent la ville. A l'emplacement de l'ancien temple de Jérusalem, ils construisirent en 691 le Dôme du rocher et en 702 la mosquée Al aqsa qui sont considérés jusqu'aujourd'hui comme des lieux saints musulmans. Entre 1096 et 1244, Jérusalem fût disputée entre les rois chrétiens et les souverains musulmans. Cela donna lieu à de nombreuses guerres mais les arabes restèrent maîtres de la région. Entre temps, les juifs d'Europe, persécutés par une Église catholique de plus en plus inquisitoire, furent nombreux à se réfugier dans une Palestine où ils furent minoritaires depuis l'époque d'Hadrien. 
Par la suite, au début du XVIe siècle, les ottomans s'emparèrent de la Palestine sous l'égide de Selim Ier, sultan de l'empire. La Palestine fut sous le contrôle de l'empire ottoman jusqu'au début du XXe siècle.

\subsection*{Epoque moderne}

Malgré la dispersion des juifs, et le fait que la terre promise fût occupée durant deux millénaires ces derniers restèrent attachés à la terre que Dieu leur avait donnée et qu'ils avaient perdue. Israël et Jérusalem ne furent jamais très loin dans la pensée de la diaspora juive. Aussi, les nombreuses révoltes du peuple contre les occupants d'alors démontrent un attachement sans faille à la terre de leurs ancêtres. C'est ainsi qu'à la fin du 19e siècle, Théodore Herzl, un intellectuel juif, commença à porter l'idéologie sioniste. Cette dernière est un plaidoyer en faveur de l'unification des juifs du monde en terre d'Israël et de l'édification d'un Etat juif en Palestine. La persécution de ces derniers motiva davantage l'engagement des sionistes en faveur d'un foyer national juif, autrement dit, un État indépendant juif. Le premier congrès sioniste eut lieu à Bâle en 1897, et permit aux différents représentants de traiter de la question. Mais au début du 20e siècle, la Palestine est sous le contrôle de l'empire ottoman et la majeur partie des habitants sont arabes musulmans. L'idée d'un retour massif paraît alors peu vraisemblable...
Mais un événement tragique va néanmoins conduire à l'avancée de la création d'une nation juive. En effet, en 1939 éclata la seconde guerre mondiale au cours de laquelle près de 6,5 millions de juifs furent exterminés par les nazis d'Adolf Hitler. Après la victoire des alliés le 17 août 1945, il devint urgent de créer un État pour les juifs.

Naissance de L’État d'Israël

« Qui entendit jamais une telle chose, et qui en a jamais vu de semblables ? Ferait-on qu'un pays fût enfanté en un jour ? ou une nation naîtrait-elle tout d'un coup, que Sion ait enfanté ses fils aussitôt qu'elle a été en travail d'enfant ? » Esaïe 66 : 8.

Plus tard, la parole du prophète Esaïe se réalisa le 29 novembre 1947, lors du vote l'Assemblée des Nations Unis (ONU) pour le plan de partage de la Palestine. Ce plan prévit l'obtention de la moitié de la Palestine au peuple juif. Ainsi, en un jour, l'établissement d'une terre nationale juive en Palestine fût proclamé. L'année d'après, le 14 mai 1948 David Ben Gourion, futur premier ministre du pays, proclama la naissance de l’État d'Israël. Ceci révèle du miracle. Jamais, dans toute l'histoire de l'humanité une nation ne revint dans le pays de ses ancêtres après plus de deux millénaires d'exil. Les fils d'Abraham, d'Isaac, et de Jacob, retrouvèrent la terre promise et purent de nouveau habiter le pays où coulent de lait et le miel (verset). Malgré les millénaires d'exode et la dispersion des juifs autour du globe, l’Éternel montra sa fidélité et permit à Israël de retourner dans le lieu de son enfance.
L'officialisation de la naissance de l’État Hébreu eut pour conséquence le retour d'un grand nombre de juifs issus de la diaspora. De petites vagues de retour commencèrent dès la fin du 19e siècle, mais entre 1948 et 1952, des centaines de milliers de juifs immigrèrent en masse vers le nouvel État. Des vagues d'immigrations similaires eurent lieu dans les décennies à venir et on nota un pic dans les années 80-90 avec plus d'1 million d'immigrants !

La naissance d'Israël fût un succès, mais les populations arabes de Palestine et du moyen orient se levèrent contre le jeune État. Dès 1948, l’Égypte, l'Irak, la Jordanie et la Syrie se levèrent contre l’État Hébreu lors de la Guerre israélo-arabe. En 1967, l’Égypte, la Syrie et la Jordanie se liguèrent de nouveau contre Israël. Ils dépêchèrent leurs armées, mais les Hébreux, en sous nombre et moins armés, défirent la coalition en l'espace de six jours ! Plus qu'une réussite militaire, ce fût là une preuve supplémentaire que la main de Dieu était sur ce peuple. Cet événement resta dans les mémoires sous le nom de Guerre des six jours.

Avec la création de l’État Hébreu, la revendication de la reconstruction du temple se fit de plus en plus pressante.
En effet, les juifs furent toujours désireux de reconstruire le temple qui fut par deux fois détruit au cours de l'histoire. Cette reconstruction nécessite la destruction du Dôme du rocher et de la mosquée Al aqsa. De nos jours, d'un point de vue géopolitique, cette éventualité semble incertaine, mais la Parole de Dieu indique bel et bien que le temple sera reconstruit. D'ailleurs, la reconstruction du temple de Jérusalem est l'un des signes du début de la soixante-dixième semaine de Daniel (équivalente à sept ans) et qui correspond à l'entrée dans la tribulation à venir. Nous voyons là qu'Israël est sans conteste l'horloge de Dieu concernant les nations et l'accomplissement des prophéties bibliques. N'oublions pas que Dieu s'était révélé à ce peuple il y a bien des millénaire, et qu'il le choisit pour entrer dans le monde et sauver les nations à travers Jésus-Christ qui, rappelons-le, était juif, et issu de la descendance du roi David à propos duquel l’Éternel déclara : « Ta maison et ton règne seront pour toujours assurés, ton trône sera pour toujours affermi » (II Samuel 7 : 16).
Toutes ces expériences spirituelles, de l'appel d'Abraham, au don de la loi, en passant par la période des rois et celle des déportations, furent pour les chrétiens des signes que Dieu plaça en vue de la rédemption par la foi en Jésus-Christ (Colossiens 2 : 17).

\subsection*{Israël aujourd'hui}

L’État Hébreux compte aujourd'hui près de sept millions d'habitants pour une superficie d'environ 20 770 km². Riche, et doté d'une des plus grandes forces de frappe militaire, il dispose d'une grande influence internationale que ce soir sur le plan politique et économique.
En outre, Jérusalem est devenue une ville qui concentre les lieux saints des trois grandes religions monothéistes :

- Le mur des lamentations pour les juifs
- Le saint sépulcre (construit au IVe siècle après JC) pour les catholiques
- La dôme du rocher et la mosquée Al Aqsa (tous deux construits au VIIe siècle ap JC) pour les musulmans.

Ce « melting pot » religieux n'est pas sans conséquence. Comme nous l'avons vu précédemment, Jérusalem fut disputée par les juifs, les musulmans, et les catholiques au cours de ces deux derniers millénaires. Les différents sites considérés comme saints mettent davantage en exergue les tensions inter religieuses. Les incidents entre arabes et juifs sont fréquents aux abords du dôme du rocher, sous lequel repose les ruines du temple de Jérusalem.

\subsection*{Evénements post-enlèvement}

Après le retour du Seigneur, ceux qui auront cru en son témoignage et qui l'auront accepté comme Seigneur et Sauveur (Actes 4 : 12), seront enlevés dans les cieux (I Thessaloniciens 4 : 13-17).
Par la suite commencera la soixante-dixième semaine de Daniel (Matthieu 24 : 4-28) marquée par l'avènement de l'homme impie et une dégradation spirituelle grandissante. Entre temps, le temple de Jérusalem sera reconstruit et les juifs rendront de nouveau le culte à l'intérieur de celui-ci (Daniel 9 : 26-27).

Remarque : La prophétie de Daniel 9 : 26 sous-entend que le temple sera reconstruit. Cela implique la destruction du dôme du rocher et de la mosquée Al Aqsa qui se trouvent au-dessus des ruines du temple.

Néanmoins, Jérusalem et le temple seront détruits par l'homme impie (Daniel 9 : 26-27) et Israël sera attaquée de toutes parts (Ezekiel 38 ; 39 : 1-24 ; Zacharie 12 : 1-9). Ce sera alors une grande détresse dans le pays, et la terre connaîtra les jours les plus sombres de son histoire (Matthieu 24 : 15-28). Après ces événements, Israël reconnaîtra Jésus, le Christ, comme le Messie et se tournera vers lui (Zacharie 12 : 10-14 ; 13 : 1-9). Ensuite se déroulera la bataille dite d'Harmaguédon : Le Seigneur Jésus et son armée constituée des saints descendront du ciel et vaincront les armées impies de la terre (Zacharie 14 : 1-3), Israël sera délivrée (Matthieu 24 : 29- 31 ; Apocalypse 19 : 11-21).
Le diable sera lié pendant une période de mille ans pendant laquelle Jésus-Christ et tous ceux qui auront porté son témoignage régneront avec lui. Un nouveau temple sera reconstruit, et Jérusalem sera le centre du culte d'adoration à l’Éternel (Esaïe 2 : 1-5). Ainsi s'installera le royaume messianique, royaume de paix, de sérénité, de crainte de l’Éternel et de sainteté (Esaïe 11 et 12). Un royaume théocratique, ou seul le Seigneur sera maître.
Après ce millénaire, le diable sera relâché et séduira des habitants de la terre pour qu'ils fassent la guerre aux saints, réunis à Jérusalem. Néanmoins, Dieu les vaincra de nouveau et le diable, la bête et le faux prophète seront jetés dans l'étang de feu pour la damnation éternelle (Apocalypse 20 : 7-10). Par la suite, les morts seront jugés (Apocalypse 21 : 11-15).

La terre que nous connaissons aujourd'hui sera alors remplacée par une nouvelle terre, façonnée de Dieu et sans aucun péché. Une nouvelle Jérusalem sera édifiée, plus belle et magnifique qu'elle ne l'a jamais été. Le Messie et les saints régneront avec lui dans cette nouvelle ville pour toujours et à perpétuité (Apocalypse 21 et 22).
