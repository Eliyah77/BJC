\ShortTitle{Ephésiens}\BookTitle{Ephésiens}\BFont
\begin{multicols}{2}
\TextTitle{[Salutations]}
\Chap{1}
\VerseOne{}Paul, apôtre de Jésus-Christ, par la volonté de Dieu, aux saints et fidèles en Jésus-Christ qui sont à Ephèse.
\VS{2}Que la grâce et la paix vous soient données de la part de Dieu notre Père, et du Seigneur Jésus-Christ.
\TextTitle{[Le croyant en Christ dans les lieux célestes]}
\VS{3}Béni soit Dieu, le Père de notre Seigneur Jésus-Christ, qui nous a bénis de toutes sortes de bénédictions spirituelles dans les lieux célestes en Christ.
\VS{4}En lui, Dieu nous a élus avant la fondation du monde, afin que nous soyons saints et irrépréhensibles devant lui dans la charité.
\VS{5}Il nous a prédestinés dans son amour à être ses enfants d’adoption par Jésus-Christ, selon le bon plaisir de sa volonté,
\VS{6}à la louange de la gloire de sa grâce, qu’il nous a gratuitement accordée en son Bien-Aimé.
\VS{7}En lui nous avons la rédemption par son sang, la rémission des offenses, selon les richesses de sa grâce,
\VS{8}qu’il a abondamment répandue sur nous en toute sagesse et intelligence.
\VS{9}Il nous a fait connaître le mystère de sa volonté, selon le bienveillant dessein qu’il avait formé en lui-même,
\VS{10}pour le mettre à exécution lorsque les temps seraient accomplis, de réunir toutes choses en Christ, celles qui sont dans les cieux et celles qui sont sur la terre.
\VS{11}C’est en lui que nous sommes aussi devenus héritiers, ayant été prédestinés, suivant la résolution de celui qui opère toutes choses avec efficacité, selon le dessein de sa volonté,
\VS{12}afin que nous servions à la louange de sa gloire, nous qui avons les premiers espéré en Christ.
\VS{13}En lui vous êtes aussi, après avoir entendu la parole de la vérité, l'Evangile de votre salut ; en lui vous avez cru, et vous avez été scellés du Saint-Esprit qui avait été promis,
\VS{14}lequel est un gage de notre héritage jusqu'à l’entière rédemption de ceux qu'il s’est acquis à la louange de sa gloire.
\TextTitle{[Prière pour recevoir sagesse et puissance]}
\VS{15}C'est pourquoi, ayant aussi entendu parler de la foi que vous avez en notre Seigneur Jésus, et de la charité que vous avez envers tous les saints,
\VS{16}je ne cesse de rendre grâces pour vous dans mes prières,
\VS{17}afin que le Dieu de notre Seigneur Jésus-Christ, le Père de gloire, vous donne l'Esprit de sagesse et de révélation, dans sa connaissance ;
\VS{18}qu'il illumine les yeux de votre cœur, afin que vous sachiez quelle est l'espérance de sa vocation, et quelles sont les richesses de la gloire de son héritage qu’il réserve aux saints,
\VS{19}et quelle est l’infinie grandeur de sa puissance envers nous qui croyons, selon l’efficacité de la puissance de sa force.
\VS{20}Il l’a déployée avec efficacité en Christ, en le ressuscitant des morts, et il l’a fait asseoir à sa droite dans les lieux célestes,
\VS{21}au-dessus de toute principauté, de toute puissance, de toute dignité et de toute domination, et au-dessus de tout nom qui peut être nommé, non seulement dans le siècle présent, mais aussi dans le siècle à venir.
\TextTitle{[Christ, Chef suprême de l'Eglise qui est son corps]}
\VS{22}Et il a mis toutes choses sous ses pieds, et l'a établi sur toutes choses pour être le Chef de L’Eglise,
\VS{23}qui est son Corps, la plénitude de celui qui accomplit tout en tous.
\TextTitle{[Dieu sauve par la foi seule]}
\Chap{2}
\VerseOne{}Vous étiez morts par vos offenses et par vos péchés,
\VS{2}dans lesquels vous marchiez autrefois, suivant le train de ce monde, selon le prince de la puissance de l'air, qui est l'esprit qui agit maintenant avec efficacité dans les fils rebelles à Dieu,
\VS{3}parmi lesquels nous vivions tous autrefois, selon les convoitises de notre chair, accomplissant les désirs de la chair et de nos pensées ; et nous étions par nature des enfants de colère, comme les autres.
\VS{4}Mais Dieu, qui est riche en miséricorde, par sa grande charité dont il nous a aimés,
\VS{5}lorsque nous étions morts dans nos offenses, il nous a vivifiés ensemble avec Christ ; c’est par grâce que vous êtes sauvés.
\VS{6}Et il nous a ressuscités ensemble, et nous a fait asseoir ensemble dans les lieux célestes en Jésus-Christ,
\VS{7}afin de montrer dans les siècles à venir l’infinie richesse de sa grâce par sa bonté envers nous en Jésus-Christ.
\VS{8}Car c’est par la grâce que vous êtes sauvés, par la foi ; et cela ne vient point de vous, c'est le don de Dieu.
\VS{9}Ce n’est point par les œuvres, afin que personne ne se glorifie.
\VS{10}Car nous sommes son ouvrage, ayant été créés en Jésus-Christ pour les bonnes œuvres que Dieu a préparées d’avance afin que nous les pratiquions.
\TextTitle{[Condition des gentils par nature]}
\VS{11}C'est pourquoi souvenez-vous que vous qui étiez autrefois païens dans la chair, et appelés incirconcis par ceux qu’on appelle circoncis, et qui le sont en la chair par la main de l’homme,
\VS{12}vous étiez en ce temps-là sans Christ, privés du droit de cité en Israël, étrangers aux alliances de la promesse, sans espérance, et sans Dieu dans le monde.
\VS{13}Mais maintenant, par Jésus-Christ, vous qui étiez jadis éloignés, vous avez été rapprochés par le sang de Christ.
\TextTitle{[Juifs et païens, un seul corps en Christ]}
\VS{14}Car il est notre paix, lui qui des deux peuples n’en a fait qu’un, en abattant le mur de séparation, l’inimitié ;
\VS{15}ayant détruit par sa chair la loi des ordonnances dans ses prescriptions, afin de créer en lui-même, avec les deux peuples, un seul homme nouveau, en établissant la paix,
\VS{16}et en réconciliant les uns et les autres avec Dieu pour former un seul corps par sa croix, ayant détruit par elle l'inimitié.
\VS{17}Ainsi, il est venu annoncer la paix à vous qui étiez loin, et à ceux qui étaient près.
\VS{18}Car nous avons par lui les uns et les autres accès auprès du Père dans un même Esprit.
\TextTitle{[L'Eglise : un temple pour l'habitation de Dieu en Esprit]}
\VS{19}Vous n'êtes donc plus des étrangers ni des gens de dehors, mais les concitoyens des saints, et gens de la maison de Dieu.
\VS{20}Vous avez été édifiés sur le fondement (1) des apôtres et des prophètes, et Jésus-Christ lui-même étant la pierre angulaire ;
\VS{21}sur qui tout l'édifice bien coordonné s'élève pour être un temple saint dans le Seigneur.
\VS{22}En lui vous êtes édifiés ensemble pour être une habitation de Dieu en Esprit.
\TextTitle{[L'Eglise : un "mystère" caché de tout temps]
\\(Col. 1:24-27)}
\Chap{3}
\VerseOne{}C’est à cause de cela que moi, Paul, je suis le prisonnier de Jésus-Christ pour vous païens.
\VS{2}Si du moins vous avez appris quelle est la dispensation de la grâce de Dieu qui m'a été donnée pour vous.
\VS{3}C’est par révélation que ce mystère m'a été manifesté ainsi que je l'ai écrit ci-dessus en peu de mots.
\VS{4}En le lisant, vous pouvez comprendre l'intelligence que j'ai du mystère de Christ.
\VS{5}Il n'a pas été manifesté aux enfants des hommes dans les autres générations, comme il a été révélé maintenant par l'Esprit aux saints apôtres et prophètes de Christ.
\VS{6}A savoir que les païens sont cohéritiers, et forment un même corps, et participent ensemble à la promesse en Christ, par l'Evangile,
\VS{7}dont j'ai été fait ministre, selon le don de la grâce de Dieu, qui m'a été donnée par l’efficacité de sa puissance.
\VS{8}Cette grâce, m'a été donnée à moi qui suis le moindre de tous les saints, pour annoncer parmi les païens les richesses incompréhensibles de Christ,
\VS{9}et pour mettre en évidence devant tous quel est le moyen de faire connaître le mystère caché de tout temps en Dieu, qui a créé toutes choses par Jésus-Christ,
\VS{10}afin que les dominations et les autorités dans les lieux célestes connaissent aujourd’hui par L’Eglise la sagesse infiniment variée de Dieu,
\VS{11}selon le dessein éternel qu’il a exécuté par Jésus-Christ notre Seigneur.
\VS{12}En lui, nous avons la liberté de nous approcher de Dieu avec confiance, par la foi que nous avons en lui.
\TextTitle{[Prière pour recevoir révélation]}
\VS{13}C'est pourquoi je vous prie de ne pas vous décourager à cause de mes afflictions que j’endure par amour pour vous ; elles sont votre gloire.
\VS{14}A cause de cela, je fléchis mes genoux devant le Père de notre Seigneur Jésus-Christ,
\VS{15}de qui toute famille tire son nom dans les cieux et sur la terre,
\VS{16}afin que selon les richesses de sa gloire, il vous donne d'être puissamment fortifiés par son Esprit dans l'homme intérieur,
\VS{17}en sorte que Christ habite dans vos cœurs par la foi ; afin qu’étant enracinés et fondés dans la charité,
\VS{18}vous puissiez comprendre avec tous les saints, quelle est la largeur et la longueur, la profondeur et la hauteur,
\VS{19}et connaître la charité de Christ qui surpasse toute connaissance ; afin que vous soyez remplis de toute la plénitude de Dieu.
\VS{20}Or à celui qui peut faire, par la puissance qui agit en nous avec efficacité, infiniment au-delà de tout ce que nous demandons et pensons,
\VS{21}à lui soit la gloire dans L’Eglise, et en Jésus-Christ, dans toutes les générations, aux siècles des siècles ! Amen !
\TextTitle{[Une marche digne de la vocation]}
\Chap{4}
\VerseOne{}Je vous prie donc, moi le prisonnier dans le Seigneur, à marcher d'une manière digne de la vocation à laquelle vous êtes appelés,
\VS{2}avec toute humilité et douceur, avec un esprit patient, vous supportant les uns les autres avec charité,
\VS{3}vous efforçant de garder l'unité de l'Esprit par le lien de la paix.
\TextTitle{[Sept verité fondamentales]}
\VS{4}Il y a un seul corps, un seul Esprit, comme aussi vous êtes appelés à une seule espérance par votre vocation.
\VS{5}Il y a un seul Seigneur, une seule foi, un seul baptême ;
\VS{6}un seul Dieu et Père de tous, qui est au-dessus de tous, parmi tous, et en vous tous.
\TextTitle{[Les dons de Christ ressuscité et leurs buts]
(1 Co. 12:4-11)}
\VS{7}Mais à chacun de nous la grâce a été donnée, selon la mesure du don de Christ.
\VS{8}C'est pourquoi il est dit : Etant monté dans les hauteurs, il a emmené une grande multitude de captifs, et il a offert des dons aux hommes (1).
\VS{9}Or que signifie : Il est monté, sinon qu’il est premièrement descendu dans les parties les plus basses de la terre ?
\VS{10}Celui qui est descendu, c'est le même qui est monté au-dessus de tous les cieux, afin de remplir toutes choses.
\VS{11}Lui-même donc a donné les uns pour être apôtres, les autres pour être prophètes, les autres pour être évangélistes, les autres pour être pasteurs et docteurs (2),
\VS{12}pour le perfectionnement (3) des saints, pour l’œuvre du ministère, pour l'édification du corps de Christ,
\VS{13}jusqu'à ce que nous soyons tous parvenus à l'unité de la foi, et de la connaissance du Fils de Dieu, à l'état d'homme parfait, à la mesure de la parfaite stature de Christ,
\VS{14}afin que nous ne soyons plus des enfants flottants, et emportés çà et là à tout vent de doctrine, par la tromperie des hommes et par leur ruse à séduire artificieusement,
\VS{15}mais afin que suivant la vérité avec la charité, nous croissions en toutes choses en celui qui est le Chef, c'est-à-dire Christ,
\VS{16}duquel tout le corps, bien proportionné et bien joint par la liaison de ses parties qui communiquent les unes aux autres, tire son accroissement selon la force qu’il distribue à chaque membre, afin qu’il soit édifié dans la charité.
\TextTitle{[La marche du croyant, homme nouveau en Christ]}
\VS{17}Voici donc ce que je vous dis et ce que je vous déclare de la part du Seigneur : Vous ne devez plus marcher comme le reste des païens qui suivent la vanité de leurs pensées.
\VS{18}Ils ont l’intelligence obscurcie par les ténèbres, et ils sont étrangers à la vie de Dieu, à cause de l'ignorance qui est en eux par l'endurcissement de leur cœur.
\VS{19}Ayant perdu tout sentiment, ils se sont abandonnés à la dissolution, pour commettre toute espèce d’impureté jointe à la cupidité.
\VS{20}Mais vous, ce n’est ainsi que vous avez appris Christ,
\VS{21}si toutefois vous l'avez entendu, et si conformément à la vérité qui est Jésus, vous avez été instruits par lui à vous dépouiller,
\VS{22}par rapport à votre vie passée, du vieil homme qui se corrompt par les convoitises,
\VS{23}à être renouvelés dans l'esprit de votre intelligence,
\VS{24}et à revêtir le nouvel homme, créé selon Dieu dans une justice et une sainteté véritables.
\VS{25}C'est pourquoi renoncez au mensonge, et que chacun de vous parle selon la vérité à son prochain, car nous sommes membres les uns des autres.
\VS{26}Si vous vous mettez en colère, ne péchez point. Que le soleil ne se couche point sur votre colère,
\VS{27}ne donnez point accès au diable pour vous perdre.
\VS{28}Que celui qui dérobait ne dérobe plus ; mais plutôt qu’il travaille en faisant de ses mains ce qui est bon, pour avoir de quoi donner à celui qui est dans le besoin.
\VS{29}Qu’il ne sorte de votre bouche aucune parole malhonnête, mais que vos discours servent à l’édification, et qu’ils communiquent une grâce à ceux qui les entendent.
\TextTitle{[La marche du croyant rempli de l'Esprit]}
\VS{30}N'attristez pas le Saint-Esprit de Dieu, par lequel vous avez été scellés pour le jour de la rédemption.
\VS{31}Que toute amertume, toute colère, toute irritation, toute clameur, toute médisance, et toute malice soient bannies du milieu de vous.
\VS{32}Soyez bons les uns envers les autres, pleins de compassion, et vous pardonnant les uns aux autres, ainsi que Dieu vous a pardonné par Christ.
\TextTitle{[La marche du croyant en enfant bien-aimé de Dieu]}
\Chap{5}
\VerseOne{}Soyez donc les imitateurs de Dieu, comme ses enfants bien-aimés.
\VS{2}Et marchez dans la charité, à l’exemple de Christ, qui nous a aimés et qui s'est livré lui-même à Dieu pour nous comme offrande et sacrifice de bonne odeur.
\VS{3}Que la fornication, ni aucune impureté, ni la cupidité, ne soient pas même nommées parmi vous, ainsi qu’il est convenable à des saints.
\VS{4}Qu’on n’entende ni parole grossière, ni propos insensés, ni plaisanterie, choses qui sont contraires à la bienséance ; qu’on entende plutôt des actions de grâces.
\VS{5}Car sachez-le bien qu’aucun fornicateur, ni impur, ni cupide, qui est un idolâtre, n'a d'héritage dans le Royaume de Christ et de Dieu.
\VS{6}Que personne ne vous séduise par de vains discours, car à cause de ces choses la colère de Dieu vient sur les fils de la rébellion.
\VS{7}Ne soyez donc point leurs associés.
\VS{8}Car vous étiez autrefois ténèbres, mais maintenant vous êtes lumière dans le Seigneur ; conduisez-vous donc comme des enfants de lumière.
\VS{9}Car le fruit de l'Esprit consiste en toutes sortes de bonté, de justice et de vérité.
\VS{10}Examinez ce qui est agréable au Seigneur.
\VS{11}Et ne prenez pas part aux œuvres infructueuses des ténèbres, mais au contraire condamnez-les.
\VS{12}Car il est honteux de dire les choses qu'ils font en secret.
\VS{13}Mais toutes ces choses étant condamnées par la lumière sont manifestées, car c’est la lumière qui manifeste tout.
\VS{14}C'est pourquoi il est dit : Réveille-toi, toi qui dors, et relève-toi d'entre les morts, et Christ t'éclairera (1).
\VS{15}Prenez donc garde afin de vous conduire avec circonspection, non comme des insensés, mais comme des sages.
\VS{16}Rachetez le temps, car les jours sont mauvais.
\VS{17}C'est pourquoi ne soyez pas inconsidérés, mais comprenez bien quelle est la volonté du Seigneur.
\TextTitle{[La vie intérieur du croyant rempli de l'Esprit]}
\VS{18}Et ne vous enivrez pas de vin, c’est la débauche ; soyez au contraire remplis de l'Esprit.
\VS{19}Entretenez-vous par des psaumes, par des hymnes et par des cantiques spirituels, chantant et célébrant de tout votre cœur les louanges du Seigneur.
\VS{20}Rendez continuellement grâces à Dieu le Père pour toutes choses au Nom de notre Seigneur Jésus-Christ,
\VS{21}en vous soumettant les uns aux autres, dans la crainte de Christ.
\TextTitle{[La vie des époux croyants remplis de l'Esprit : illustration de Christ et de l'Eglise]}
\VS{22}Femmes, soyez soumises à vos maris, comme au Seigneur.
\VS{23}Car le mari est le chef de la femme, comme Christ est le Chef de L’Eglise qui est son corps, et dont il est le Sauveur.
\VS{24}Or de même que L’Eglise est soumise à Christ, les femmes doivent l’être à leurs maris en toutes choses.
\VS{25}Et vous maris, aimez vos femmes, comme Christ a aimé L’Eglise, et s'est livré lui-même pour elle,
\VS{26}afin de la sanctifier, en la purifiant et en la lavant par l’eau de la parole,
\VS{27}pour faire paraître devant lui cette Egliseglorieuse, sans tache, ni ride, ni rien de semblable, mais sainte et irréprochable.
\VS{28}C’est ainsi que les maris doivent aimer leurs femmes comme leurs propres corps. Celui qui aime sa femme s'aime lui-même,
\VS{29}car personne n'a jamais haï sa propre chair, mais il la nourrit et l'entretient, comme le Seigneur entretient L’Eglise.
\VS{30}Car nous sommes membres de son Corps, étant de sa chair et de ses os.
\VS{31}C'est pourquoi l'homme quittera son père et sa mère et s’attachera à sa femme, et les deux deviendront une seule chair.
\VS{32}Ce mystère est grand ; je dis cela par rapport à Christ et à L’Eglise.
\VS{33}Ainsi, que chacun de vous aime sa femme comme lui-même, et que la femme respecte son mari.
\TextTitle{[La vie domestique des croyants remplis de l'Esprit, dans la famille et la société]}
\Chap{6}
\VerseOne{}Enfants, obéissez à vos parents, selon le Seigneur ; car cela est juste.
\VS{2}Honore ton père et ta mère - c’est le premier commandement avec une promesse -
\VS{3}afin que tu sois heureux et que tu vives longtemps sur la terre.
\VS{4}Et vous pères, n'irritez pas vos enfants, mais élevez-les (1) en les instruisant et les avertissant selon le Seigneur.
\VS{5}Serviteurs, obéissez à vos maîtres selon la chair, avec crainte et tremblement, dans la simplicité de votre cœur, comme à Christ.
\VS{6}Ne les servez pas seulement sous leurs yeux, comme cherchant à plaire aux hommes ; mais comme serviteurs de Christ, faisant de bon cœur la volonté de Dieu,
\VS{7}servez-les avec empressement comme servant le Seigneur et non les hommes.
\VS{8}Sachant que chacun, soit esclave, soit libre, recevra du Seigneur le bien qu'il aura fait.
\VS{9}Et vous, maîtres, agissez de même à leur égard, et abstenez-vous des menaces, sachant que leur Seigneur et le vôtre est dans les cieux, et que devant lui il n'y a point de favoritisme.
\TextTitle{[Le combat des croyants remplis de l'Esprit
\\1. la puissance du combattant]}
\VS{10}Au reste, mes frères, fortifiez-vous dans le Seigneur, et par sa force toute-puissante.
\TextTitle{[2. l'armure du chrétien]}
\VS{11}Revêtez-vous de toutes les armes de Dieu, afin de pouvoir résister aux embûches du diable.
\VS{12}Car nous n'avons pas à lutter contre la chair et le sang, mais contre les dominations, contre les autorités, contre les princes des ténèbres de ce siècle, contre les esprits méchants qui sont dans les lieux célestes.
\VS{13}C'est pourquoi prenez toutes les armes de Dieu, afin de pouvoir résister dans le mauvais jour, et tenir ferme après avoir tout surmonté.
\VS{14}Soyez donc fermes : Ayez à vos reins la vérité pour ceinture ; revêtez la cuirasse de la justice.
\VS{15}Mettez pour chaussures à vos pieds le zèle de l'Evangile de paix.
\VS{16}Prenez par-dessus tout le bouclier de la foi, avec lequel vous pourrez éteindre tous les dards enflammés du malin.
\VS{17}Prenez aussi le casque du salut, et l'épée de l'Esprit, qui est la parole de Dieu.
\TextTitle{[3. la ressource du chrétien]}
\VS{18}Faites en tout temps par l’Esprit toutes sortes de prières et de supplications, veillez à cela avec une entière persévérance, et priez pour tous les saints.
\VS{19}Priez pour moi, afin qu'il me soit donné de parler en toute liberté, et avec hardiesse, pour faire connaître le mystère de l'Evangile,
\VS{20}pour lequel je suis ambassadeur dans les chaînes, afin que je parle librement, comme je dois en parler.
\TextTitle{[Envoi de l'Epître par Tychique et salutations]}
\VS{21}Afin que vous aussi, vous sachiez ce qui me concerne, ce que je fais, Tychique, notre frère bien-aimé, et fidèle ministre du Seigneur, vous informera de tout.
\VS{22}Je l’envoie exprès vers vous, afin que vous connaissiez notre situation, et pour qu'il console vos cœurs.
\VS{23}Que la paix et la charité, avec la foi, soient avec les frères, de la part de Dieu le Père, et du Seigneur Jésus-Christ.
\VS{24}Que la grâce soit avec tous ceux qui aiment notre Seigneur Jésus-Christ en pureté. Amen !
\PPE{}
\end{multicols}
