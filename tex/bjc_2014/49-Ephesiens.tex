\ShortTitle{Ephésiens}\BookTitle{Ephésiens}\BFont
\noindent\hrulefill
{\footnotesize
\textit{
\bigskip
{\centering{}
\\Auteur : Paul
\\Thème : L'Eglise, corps de Christ
\\Date de rédaction : Env. 60 ap. J.-C.\\}
}
%\bigskip
\textit{
\\Ephèse figurait parmi les principales villes de l'Empire romain sous le règne de l'empereur Claude Ier (10 av. J.-C. – 54 ap. J.-C.). Bien que Pergame était considérée comme la capitale de l'Asie Mineure en raison de sa position géographique et grâce à ses affluents, Ephèse possédait le plus grand port de la région, ce qui lui attribuait le contrôle du trafic commercial. Richissime et prospère, elle était renommée pour son faste et sa liberté de parole, et constituait donc un endroit privilégié pour les philosophes. C'était une ville où l'activité culturelle tenait une grande place (Jeux olympiques, théâtres, cirques, etc.) et où chacun pouvait y pratiquer la religion de son choix (croyances gréco-romaines, égyptiennes, judaïque etc.).
%\bigskip
\\Ephèse, dont le nom signifie « désirable », était la gardienne de l'Artémision, temple dédié à la déesse grecque Artémis,
la Diane des Ephésiens.
%\bigskip
\\L'église d'Ephèse vit le jour lors du second voyage missionnaire de Paul (50-52). Quand il repartit, il laissa à Aquilas et Priscille la charge de la toute jeune assemblée. Paul s'installa à Ephèse lors de son troisième voyage (53-57) et y demeura presque trois ans. Il discourut pendant trois mois dans la synagogue sur le Royaume de Dieu, mais se retrouva confronté à l'endurcissement de certains. C'est alors qu'il se retira pour enseigner dans l'école d'un certain Tyrannus durant deux ans, de sorte que tous ceux qui habitaient l'Asie, Juifs et Grecs, entendirent parler de Jésus-Christ.
%\bigskip
\\Plusieurs de ceux qui avaient crû confessèrent leurs péchés et un certain nombre de ceux qui avaient pratiqué la magie allèrent même jusqu'à brûler leurs livres publiquement. C'est ainsi que l'église d'Ephèse croissait en puissance et en force. La prédication de Paul vint troubler le marché fructueux des fabricants d'idoles au point que Démétrius ( orfèvre tirant un grand profit de cette industrie ) entraina une émeute contre lui. Paul était cependant soutenu par des amis influents : les asiarques.
%\bigskip
\\Rédigée en prison, cette épître a pour vocation d'enseigner les chrétiens d'Ephèse sur la manière dont il convient de vivre les uns avec les autres au sein de l'Eglise corps du Christ.\bigskip
}
}
\par\nobreak\noindent\hrulefill
\begin{multicols}{2}
\Chap{1}
\TextTitle{Introduction}
\VerseOne{}Paul, apôtre de Jésus-Christ par la volonté de Dieu, aux saints et fidèles en Jésus-Christ qui sont à Ephèse.
\VS{2}Que la grâce et la paix vous soient données par Dieu notre Père, et par le Seigneur Jésus-Christ.
\TextTitle{La position du croyant dans le Royaume de Dieu}
\VS{3}Béni soit Dieu, qui est le Père de notre Seigneur Jésus-Christ, qui nous a bénis de toutes bénédictions spirituelles dans les lieux célestes en Christ !
\VS{4}Selon qu'il nous a élus en lui avant la fondation du monde, afin que nous soyons saints et irrépréhensibles devant lui dans la charité, 
\VS{5}nous ayant prédestinés dans son amour à être ses enfants d'adoption par Jésus-Christ, selon le bon plaisir de sa volonté,
\VS{6}à la louange de la gloire de sa grâce, par laquelle il nous a rendus agréables en son bien-aimé.
\VS{7}En lui nous avons la rédemption par son sang, savoir la rémission des offenses, selon les richesses de sa grâce,
\VS{8}qu'il a fait abonder sur nous en toute sagesse et intelligence,
\VS{9}nous ayant donné à connaître le mystère de sa volonté, qu'il avait premièrement arrêté en lui-même,
\VS{10}afin que dans l'accomplissement des temps qu'il avait réglés, il réunît tout en Christ, tant ce qui est dans les cieux, que ce qui est sur la terre. 
\VS{11} En qui nous sommes aussi devenus héritiers, ayant été prédestinés, suivant la résolution de celui qui accomplit toutes choses avec efficacité, selon le conseil de sa volonté,
\VS{12}afin que nous soyons à la louange de sa gloire, nous qui avons les premiers espéré en Christ.
\VS{13}En qui vous êtes aussi, ayant entendu la parole de la vérité, qui est  l’Evangile de votre salut, et auquel ayant cru vous avez été scellés du Saint-Esprit qui avait été promis ;
\VS{14}qui est un gage de notre héritage jusqu'à la rédemption de ceux qu'il s'est acquis, à la louange de sa gloire.
\VS{15}C'est pourquoi, ayant aussi entendu parler de la foi que vous avez en notre Seigneur Jésus, et de la charité que vous avez envers tous les saints,
\VS{16}je ne cesse de rendre grâces pour vous dans mes prières,
\VS{17}afin que le Dieu de notre Seigneur Jésus-Christ, le Père de gloire, vous donne l'Esprit de sagesse et de révélation, dans ce qui regarde sa connaissance ;
\VS{18}qu'il illumine les yeux de votre esprit, afin que vous sachiez quelle est l'espérance de sa vocation, et quelles sont les richesses de la gloire de son héritage qu'il réserve aux saints,
\VS{19}et quelle est l’excellente grandeur de sa puissance envers nous qui croyons selon l’efficacité de la puissance de sa force, 
\VS{20}qu'il a déployée avec efficacité en Christ, quand il l'a ressuscité des morts et qu'il l'a fait asseoir à sa droite dans les lieux célestes,
\VS{21}au-dessus de toute principauté, de toute puissance, de toute dignité et de toute domination, et au-dessus de tout nom qui se nomme, non seulement dans le siècle présent, mais aussi dans celui qui est à venir.
\TextTitle{Le Messie est le Chef suprême de l'Eglise}
\VS{22}Et il a assujetti toutes choses sous ses pieds, et l'a établi sur toutes choses pour être le Chef de l'Eglise,
\VS{23}qui est son Corps, et la plénitude de celui qui remplit tout en tous.
\Chap{2}
\TextTitle{Le salut par la foi}
\VerseOne{}Vous étiez morts par vos offenses et par vos péchés,
\VS{2}dans lesquels vous marchiez autrefois, suivant le train de ce monde, selon le prince de la puissance de l'air, qui est l'esprit qui agit maintenant avec efficacité dans les fils rebelles à Dieu,
\VS{3}parmi lesquels nous vivions tous autrefois selon les convoitises de notre chair, accomplissant les désirs de la chair et de nos pensées, et nous étions par nature des enfants de colère, comme les autres.
\VS{4}Mais Dieu, qui est riche en miséricorde, à cause de sa grande charité dont il nous a aimés,
\VS{5}lorsque nous étions morts dans nos offenses, il nous a vivifiés ensemble avec Christ ; c'est par grâce que vous êtes sauvés.
\VS{6}Et il nous a ressuscités ensemble, et nous a fait asseoir ensemble dans les lieux célestes, en Jésus-Christ,
\VS{7}afin qu'il montre dans les siècles à venir les immenses richesses de sa grâce par sa bonté envers nous en Jésus-Christ.
\VS{8}Car vous êtes sauvés par la grâce, par la foi. Et cela ne vient point de vous, c'est le don de Dieu.
\VS{9}Non pas par les œuvres, afin que personne ne se glorifie ;
\VS{10}car nous sommes son ouvrage, ayant été créés en Jésus-Christ pour les bonnes œuvres, que Dieu a préparées d'avance, afin que nous marchions en elles.
\VS{11}C’est pourquoi souvenez-vous que vous, qui étiez autrefois Gentils dans la chair, et qui étiez appelés prépuce par ceux qu'on appelle circoncis dans la chair faite par la main des hommes, 
\VS{12}étiez en ce temps-là sans Christ, privés du droit de cité en Israël, étant étrangers des alliances de la promesse, n'ayant point d'espérance, et étant sans Dieu, dans le monde.
\VS{13}Mais maintenant, par Jésus-Christ, vous qui étiez autrefois éloignés, vous avez été rapprochés par le sang de Christ.
\TextTitle{Juifs et Gentils forment un seul corps}
\VS{14}Car il est notre paix, lui qui des deux n'en a fait qu'un, en détruisant le mur de séparation,
\VS{15}ayant aboli dans sa chair l'inimitié, savoir la loi des commandements qui consiste en ordonnances, afin de créer en lui-même avec les deux un seul homme nouveau, en faisant la paix,
\VS{16}et de réconcilier les uns et les autres avec Dieu pour former un seul corps par sa croix, ayant détruit par elle l'inimitié.
\VS{17}Et il est venu prêcher la paix à vous qui étiez loin, et à ceux qui étaient près.
\VS{18}Car nous avons par lui les uns et les autres accès auprès du Père dans un même Esprit.
\TextTitle{L'Eglise véritable}
\VS{19}C'est pourquoi vous n'êtes plus des étrangers ni des gens de dehors, mais concitoyens des saints et gens de la maison de Dieu ;
\VS{20}étant édifiés sur le fondement\FTNT{Le fondement a été posé une fois pour toutes par les apôtres et les prophètes. Et ce fondement est notre Seigneur Jésus-Christ (1 Co. 3:11).} des apôtres et des prophètes, et Jésus-Christ lui-même étant la pierre angulaire ;
\VS{21}en qui tout l’édifice, bien ajusté ensemble, s'élève pour être un temple saint dans le Seigneur.
\VS{22}En qui vous êtes édifiés ensemble, pour être une habitation de Dieu en Esprit.
\Chap{3}
\TextTitle{L'Eglise : Un « mystère » caché de tout temps\FTNTT{Col. 1:24-27}}
\VerseOne{}C'est pour cela que moi, Paul, je suis prisonnier de Jésus-Christ pour vous Gentils.
\VS{2}Si toutefois vous avez entendu quel est le ministère de la grâce de Dieu, qui m'a été donnée pour vous.
\VS{3}Comment par révélation ce mystère m'a été manifesté, ainsi que je l'ai écrit ci-dessus en peu de mots.
\VS{4}D'où vous pouvez voir, en le lisant, quelle est l'intelligence que j'ai du mystère de Christ.
\VS{5}Lequel n'a pas été manifesté aux fils des hommes dans les autres générations, comme il a été révélé maintenant par l'Esprit à ses saints apôtres et à ses prophètes,
\VS{6}savoir que les Gentils sont cohéritiers, et un même corps, et qu'ils participent ensemble à sa promesse en Christ, par l'Evangile.
\VS{7}Dont j'ai été fait serviteur, selon le don de la grâce de Dieu, qui m'a été donnée selon l'efficacité de sa puissance.
\VS{8}Cette grâce, dis-je, m'a été donnée à moi qui suis le moindre de tous les saints, pour annoncer parmi les Gentils les richesses incompréhensibles de Christ,
\VS{9}et pour mettre en évidence devant tous quelle est la communion qui nous a été accordée du mystère qui était caché de tout temps en Dieu, lequel a créé toutes choses par Jésus-Christ ;
\VS{10}afin que les dominations et les autorités dans les lieux célestes connaissent aujourd'hui par l'Eglise la sagesse infiniment variée de Dieu,
\VS{11}Suivant le dessein arrêté dès les siècles, qu'il a établi en Jésus-Christ notre Seigneur.
\VS{12}Par lequel nous avons hardiesse et accès avec confiance, par la foi que nous avons en lui.
\VS{13} C'est pourquoi je vous prie de ne pas vous relâcher à cause de mes afflictions que je souffre pour l'amour de vous, ce qui est votre gloire.
\VS{14}A cause de cela, je fléchis mes genoux devant le Père de notre Seigneur Jésus-Christ,
\VS{15}duquel toute parenté est nommée dans les cieux et sur la terre,
\VS{16}afin que selon les richesses de sa gloire, il vous donne d'être puissamment fortifiés par son Esprit, dans l'homme intérieur ;
\VS{17}tellement que Christ habite dans vos cœurs par la foi ; afin qu'étant enracinés et fondés dans la charité,
\VS{18}vous puissiez comprendre avec tous les saints, quelle est la largeur et la longueur, la profondeur et la hauteur ;
\VS{19}et connaître la charité de Christ qui surpasse toute connaissance ; afin que vous soyez remplis de toute la plénitude de Dieu.
\VS{20}Or à celui qui par la puissance qui agit en nous avec efficacité, peut faire infiniment au-delà de tout ce que nous demandons et pensons,
\VS{21}à lui soit la gloire dans l'Eglise, et en Jésus-Christ, dans toutes les générations, aux siècles des siècles ! Amen !
\Chap{4}
\TextTitle{La marche et le service des croyants}
\VerseOne{}Je vous prie donc, moi, le prisonnier dans le Seigneur, à marcher d'une manière digne de la vocation à laquelle vous êtes appelés,
\VS{2}avec toute humilité et douceur, avec patience, vous supportant les uns les autres dans la charité,
\VS{3}vous efforçant de garder l'unité de l'Esprit par le lien de la paix.
\TextTitle{Sept vérités fondamentales}
\VS{4}Il y a un seul corps, un seul Esprit, comme aussi vous êtes appelés à une seule espérance de votre vocation.
\VS{5}Il y a un seul Seigneur, une seule foi, un seul baptême ;
\VS{6}un seul Dieu et Père de tous, qui est au-dessus de tous, parmi tous, et en vous tous.
\TextTitle{Les dons du Messie et leurs buts\FTNTT{1 Co. 12:4-11.}}
\VS{7}Mais la grâce est donnée à chacun de nous, selon la mesure du don de Christ.
\VS{8}C'est pourquoi il est dit : Etant monté en haut, il a emmené captive une grande multitude de captifs, et il a donné des dons aux hommes\FTNT{Ps. 68:19.}.
\VS{9}Or que signifie : Il est monté, sinon qu'il est premièrement descendu dans les parties les plus basses de la terre ?
\VS{10}Celui qui est descendu, c'est le même qui est monté au-dessus de tous les cieux, afin de remplir toutes choses.
\VS{11}Lui-même donc a donné les uns pour être apôtres, les autres pour être prophètes, les autres pour être évangélistes, les autres pour être pasteurs et docteurs\FTNT{Le mot « ministère » vient du grec « diakonos », il signifie « service, ministère, de ceux qui répondent aux besoins des autres ». Jésus-Christ lui-même a pris la forme d'un serviteur pour nous servir et non pour être servi (Mt. 20:28).},
\VS{12}pour travailler au perfectionnement\FTNT{Le mot « perfectionnement » vient du grec « katartismos », qui tire son origine du terme « katartizo » : « redresser, ajuster, compléter, raccommoder (ce qui a été abîmé), réparer ». Ainsi, les divers ministères ont vocation, d'une part à réparer les dégâts causés par le péché dans les âmes, et d'autre part à préparer les disciples à rentrer à leur tour dans leur propre ministère.} des saints, pour l'œuvre du ministère, pour l'édification du Corps de Christ,
\VS{13}jusqu'à ce que nous soyons tous parvenus à l'unité de la foi et de la connaissance du Fils de Dieu, à l'état d'homme parfait, à la mesure de la parfaite stature de Christ,
\VS{14}afin que nous ne soyons plus des enfants, flottants et emportés çà et là à tout vent de doctrine, par la tromperie des hommes, et par leur habile ruse avec laquelle ils se tiennent en embuscade pour égarer ;
\VS{15}mais afin que, suivant la vérité avec la charité, nous croissions en toutes choses en celui qui est le Chef, c'est-à-dire Christ,
\VS{16}duquel tout le corps bien ajusté et lié ensemble par toutes les jointures de son assistance, tire son accroissement selon la force qu'il distribue à chaque membre, afin qu'il soit édifié dans la charité.
\TextTitle{La marche des croyants dans le Royaume de Dieu}
\VS{17}Je vous dis donc, et je vous conjure de la part du Seigneur, de ne plus vous conduire comme le reste des Gentils, qui suivent la vanité de leurs pensées.
\VS{18}Ils ont l'intelligence obscurcie par les ténèbres, et sont étrangers à la vie de Dieu, à cause de l'ignorance qui est en eux, par l'endurcissement de leur cœur.
\VS{19}Ils ont perdu tout sentiment, et se sont abandonnés à la dissolution, pour commettre toute sorte d'impureté avec cupidité.
\VS{20}Mais vous n'avez pas ainsi appris Christ,
\VS{21}si toutefois vous l'avez entendu, et si vous avez été enseignés par lui, puisque la vérité est en Jésus ; 
\VS{22}savoir vous dépouiller, pour ce qui est de votre conduite précédente, du vieil homme, qui se corrompt par les convoitises qui séduisent ; 
\VS{23}et que vous soyez renouvelés dans l'esprit de votre entendement,
\VS{24}et que vous soyez revêtus du nouvel homme, créé selon Dieu dans une justice et une sainteté véritables.
\VS{25}C'est pourquoi ayant dépouillé le mensonge, parlez en vérité chacun avec son prochain ; car nous sommes membres les uns des autres.
\VS{26}Si vous vous mettez en colère, ne péchez point. Que le soleil ne se couche pas sur votre colère,
\VS{27}ne donnez pas lieu au diable de vous perdre.
\VS{28}Que celui qui dérobait ne dérobe plus ; mais plutôt qu'il travaille en faisant de ses mains ce qui est bon, pour avoir de quoi donner à celui qui est dans le besoin.
\VS{29}Qu'aucun discours malhonnête ne sorte de votre bouche, mais seulement celui qui est propre à édifier, afin qu'il soit agréable à ceux qui l'écoutent.
\VS{30}Et n'attristez pas le Saint-Esprit de Dieu, par lequel vous avez été scellés pour le jour de la rédemption.
\VS{31}Que toute amertume, toute colère, toute irritation, toute clameur, toute médisance, et toute malice soient bannies du milieu de vous.
\VS{32}Mais soyez doux les uns envers les autres, pleins de compassion, et vous pardonnant les uns aux autres, ainsi que Dieu vous a pardonné par Christ.
\Chap{5}
\TextTitle{La marche des croyants (suite)}
\VerseOne{}Soyez donc les imitateurs de Dieu, comme ses enfants bien-aimés ;
\VS{2}et marchez dans la charité, ainsi que Christ nous a aimés, et s'est livré lui-même pour nous comme une offrande et un sacrifice de bonne odeur à Dieu.
\VS{3}Que la fornication, ni aucune impureté, ni la cupidité, ne soient pas même nommées parmi vous, ainsi qu'il est convenable à des saints.
\VS{4}Qu'on n'entende ni parole grossière, ni propos insensés, ni plaisanterie, choses qui sont contraires à la bienséance, mais plutôt des actions de grâces.
\VS{5}Car sachez-le bien qu'aucun fornicateur, ni impur, ni cupide, qui est un idolâtre, n'a d'héritage dans le Royaume de Christ et de Dieu.
\VS{6}Que personne ne vous séduise par de vains discours ; car à cause de ces choses la colère de Dieu vient sur les fils de la rébellion.
\VS{7}Ne soyez donc point leurs associés.
\VS{8}Car vous étiez autrefois ténèbres, mais maintenant vous êtes lumière dans le Seigneur. Conduisez-vous donc comme des enfants de la lumière.
\VS{9}Car le fruit de la lumière consiste en toute bonté, justice et vérité,
\VS{10}éprouvant ce qui est agréable au Seigneur.
\VS{11}Et ne participez pas aux œuvres infructueuses des ténèbres, mais au contraire condamnez-les.
\VS{12}Car il est honteux de dire les choses qu'ils font en secret ;
\VS{13}mais toutes choses, étant mises en évidence par la lumière, sont rendues manifestes, car la lumière est celle qui manifeste tout.
\VS{14}C'est pourquoi il est dit : Réveille-toi, toi qui dors, et relève-toi d'entre les morts, et Christ t'éclairera\FTNT{Es. 60:1.}.
\VS{15}Prenez donc garde de vous conduire soigneusement, non pas comme étant dépourvus de sagesse, mais comme étant sages.
\VS{16}Rachetez le temps, car les jours sont mauvais.
\VS{17}C'est pourquoi ne soyez pas sans intelligence, mais comprenez bien quelle est la volonté du Seigneur.
\VS{18}Et ne vous enivrez pas du vin dans lequel il y a de la dissolution, mais soyez remplis de l'Esprit.
\VS{19}Entretenez-vous par des psaumes, des hymnes et des cantiques spirituels, chantant et psalmodiant de votre coeur au Seigneur.
\VS{20}Rendez en tous temps grâces pour toutes choses à Dieu et Père, au Nom de notre Seigneur Jésus-Christ ;
\VS{21}en vous soumettant les uns aux autres, dans la crainte de Christ.
\TextTitle{La vie du couple croyant}
\VS{22}Femmes, soyez soumises à vos maris, comme au Seigneur ;
\VS{23}car le mari est le chef de la femme, comme Christ est le Chef de l'Eglise, qui est son corps, et dont il est le Sauveur.
\VS{24}Or de même que l'Eglise est soumise à Christ, les femmes aussi doivent l'être à leurs maris en toutes choses.
\VS{25}Et vous maris, aimez vos femmes, comme Christ a aimé l'Eglise, et s'est livré lui-même pour elle,
\VS{26}afin de la sanctifier, en la purifiant et en la lavant par l'eau de la parole.
\VS{27}Afin de faire paraître devant lui cette Eglise glorieuse, sans tache, ni ride, ni rien de semblable, mais afin qu'elle soit sainte et irréprochable.
\VS{28}C'est ainsi que les maris doivent aimer leurs femmes comme leurs propres corps ; celui qui aime sa femme s'aime lui-même,
\VS{29}car personne n'a jamais eu en haine sa propre chair, mais il la nourrit et l'entretient, comme le Seigneur entretient l'Eglise.
\VS{30}Car nous sommes membres de son Corps, étant de sa chair et de ses os.
\VS{31}C'est pourquoi l'homme quittera son père et sa mère et s'attachera à sa femme, et les deux deviendront une seule chair.
\VS{32}Ce mystère est grand, or je parle de Christ et de l'Eglise.
\VS{33}Que chacun de vous donc aime sa femme comme lui-même, et que la femme respecte son mari.
\Chap{6}
\TextTitle{La vie familiale des croyants}
\VerseOne{}Enfants, obéissez à vos pères et à vos mères, dans ce qui est selon le Seigneur, car cela est juste.
\VS{2}Honore ton père et ta mère, c'est le premier commandement avec une promesse,
\VS{3}afin que tu sois heureux et que tu vives longtemps sur la terre.
\VS{4}Et vous pères, n'irritez pas vos enfants, mais élevez-les\FTNT{Le verbe « élever » vient du grec « ektrepho » qui signifie nourrir jusqu'à maturité.} en les instruisant et les avertissant selon le Seigneur.
\VS{5}Serviteurs, obéissez à vos maîtres selon la chair, avec crainte et tremblement, dans la simplicité de votre cœur, comme à Christ.
\VS{6}Ne les servant pas seulement sous leurs yeux, comme cherchant à plaire aux hommes ; mais comme serviteurs de Christ, faisant de bon cœur la volonté de Dieu,
\VS{7}servant avec bienveillance comme servant le Seigneur et non pas les hommes.
\VS{8}Sachant que chacun, soit esclave, soit libre, recevra du Seigneur le bien qu'il aura fait.
\VS{9}Et vous maîtres, faites envers eux la même chose, et renoncez aux menaces, sachant que leur Seigneur et le vôtre est dans les cieux et qu'il n'y a point en lui acception de personnes.
\TextTitle{Le combat spirituel}
\VS{10}Au reste, mes frères, fortifiez-vous dans le Seigneur, et dans la puissance de sa force.
\VS{11}Revêtez-vous de toutes les armes de Dieu, afin de pouvoir résister aux embûches du diable.
\VS{12}Car nous n'avons pas à lutter contre la chair et le sang, mais contre les dominations, contre les autorités, contre les princes des ténèbres de ce siècle, contre les esprits méchants qui sont dans les lieux célestes.
\VS{13}C'est pourquoi prenez toutes les armes de Dieu\FTNT{Voir en annexe « Les armes du chrétien ».}, afin de pouvoir résister dans le mauvais jour, et tenir ferme après avoir tout surmonté.
\VS{14}Soyez donc fermes, ayant à vos reins la vérité pour ceinture, ayant revêtu la cuirasse de la justice,
\VS{15}et ayant chaussé vos pieds de la préparation de l’Evangile de paix.
\VS{16}Par-dessus tout, prenez le bouclier de la foi, avec lequel vous pourrez éteindre tous les dards enflammés du malin ;
\VS{17}prenez aussi le casque du salut, et l'épée de l'Esprit, qui est la parole de Dieu.
\VS{18}Priant en votre esprit par toutes sortes de prières et de supplications en tout temps, veillant à cela avec une entière persévérance, et priant pour tous les saints,
\VS{19}et pour moi aussi, afin qu'il me soit donné de parler en toute liberté, et avec hardiesse, pour faire connaître le mystère de l'Evangile,
\VS{20}pour lequel je suis ambassadeur quoique chargé de chaînes, afin, dis-je, que je parle librement, ainsi qu'il faut que je parle.
\TextTitle{Salutations}
\VS{21}Or afin que vous aussi, vous sachiez ce qui me concerne, et ce que je fais, Tychique, notre frère bien-aimé, et fidèle serviteur du Seigneur, vous fera tout savoir.
\VS{22}Je l'envoie exprès vers vous, afin que vous connaissiez notre situation, et pour qu'il console vos cœurs.
\VS{23}Que la paix soit avec les frères, et la charité avec la foi, de la part de Dieu le Père, et du Seigneur Jésus-Christ.
\VS{24}Que la grâce soit avec tous ceux qui aiment notre Seigneur Jésus-Christ dans l’incorruptibilité. Amen !
\PPE{}
\end{multicols}
