\ShortTitle{Ga.}\BookTitle{Galates}\BFont
\noindent\hrulefill
{\footnotesize
\textit{
\bigskip
{\centering{}
\\Auteur~: Paul
\\Thème~: Le salut par la grâce
\\Date de rédaction~: Env. 50 ap. J.-C.\\}
}
\textit{
\\Province antique de l'Asie Mineure, la Galatie se situait en Anatolie. Elle devait son nom aux Galates, Celtes provenant des Balkans.
\\La lettre de Paul aux Galates est la seule épître dont le début ne contient pas de témoignage d'affection. Paul commence par justifier l'origine de son appel, en employant un ton sec et sévère. Les Galates, qu'il avait lui-même évangélisés lors de son premier voyage, s'étaient promptement détournés de l'Evangile qu'ils avaient reçu. Ils ne l'avaient pas totalement abandonné, mais y avaient ajouté ce qui ne leur avait point été prescrit. Troublés par les enseignements des judaïsants - des juifs ayant cru en Jésus-Christ, mais persistant toujours dans la pratique de la loi - les Galates avaient repris à leur compte leurs traditions, annihilant ainsi l'œuvre de la croix. Par cette lettre, Paul les exhorte d'une part à revenir à l'Evangile véritable et d'autre part à marcher par l'Esprit afin d'en porter le fruit.\bigskip
}
}
\par\nobreak\noindent\hrulefill
\begin{multicols}{2}
\Chap{1}
\TextTitle{Introduction}
\VerseOne{}Paul, apôtre, non de la part des hommes, ni de la part d'aucun homme, mais de la part de Jésus-Christ, et de la part de Dieu le Père, qui l'a ressuscité des morts,
\VS{2}et tous les frères qui sont avec moi, aux églises de Galatie\FTNT{La Galatie, ou Gallo-Grèce, était une province de l'Asie Mineure (région de la Turquie actuelle). Au nord, elle était délimitée par la Bithynie et la
Paphlagonie, à l'est par le Pont et la Cappadoce, au sud par la Cappadoce, la Lycaonie et la Phrygie, et à l'ouest par la Phrygie et la Bithynie. Son nom vient des Gaulois qui s'étaient installés dans la région en 279 av. J.-C. Conquise par les Romains en 189 av. J.-C., elle devint une province de l'Empire en 25 av. J.-C.}.
\VS{3}Que la grâce et la paix vous soient données de la part de Dieu le Père, et de la part de notre Seigneur Jésus-Christ,
\VS{4}qui s'est donné lui-même pour nos péchés, afin de nous arracher du présent siècle mauvais, selon la volonté de Dieu notre Père.
\VS{5}A lui soit la gloire aux siècles des siècles. Amen~!
\TextTitle{Les Galates se détournent de l'Evangile véritable}
\VS{6}Je m'étonne que vous abandonniez si promptement celui qui vous avait appelés à la grâce de Christ, pour passer à un autre évangile. 
\VS{7}Non qu'il y ait un autre Evangile, mais il y a des gens qui vous troublent, et qui veulent renverser l'Evangile de Christ.
\VS{8}Mais quand nous-mêmes, ou quand un ange venu du ciel vous évangéliserait, outre\FTNT{Voir \vref{1 R. 13:11-34}.} ce que nous vous avons évangélisé, qu'il soit anathème~!
\VS{9}Comme nous l'avons déjà dit, je le dis encore maintenant~: Si quelqu'un vous évangélise outre ce que vous avez reçu, qu'il soit anathème~!
\TextTitle{Paul reçoit la révélation de l'Evangile}
\VS{10}Car est-ce les hommes que je prêche ou Dieu~? Ou est-ce que je cherche à plaire aux hommes~? Certes si je plaisais encore aux hommes, je ne serais pas le serviteur de Christ.
\VS{11}Je vous le déclare donc, mes frères, que l'Evangile que j'ai annoncé n'est pas selon l'homme,
\VS{12}parce que je ne l'ai ni reçu ni appris d'aucun homme, mais par la révélation de Jésus-Christ.
\VS{13}Car vous avez appris quelle a été autrefois ma conduite dans le judaïsme, et comment je persécutais à outrance l'Eglise de Dieu et la ravageais,
\VS{14}et comment j'étais plus avancé dans le judaïsme que beaucoup de ceux de mon âge et de ma nation, étant le plus ardent zélateur des traditions de mes pères.
\VS{15}Mais quand il a plu à Dieu, qui m'avait choisi dès le ventre de ma mère, et qui m'a appelé par sa grâce,
\VS{16}de révéler en moi son Fils, afin que je le prêche parmi les Gentils, aussitôt, je ne consultai ni la chair ni le sang,
\VS{17}et je ne montai point à Jérusalem vers ceux qui furent apôtres avant moi, mais je partis pour l'Arabie, puis je revins encore à Damas.
\VS{18}Ensuite, trois ans après, je montai à Jérusalem pour visiter Pierre, et je demeurai chez lui quinze jours.
\VS{19}Et je ne vis aucun des autres apôtres, sinon Jacques, le frère du Seigneur.
\VS{20}Or, dans les choses que je vous écris, voici, devant Dieu je vous dis que je ne mens point.
\VS{21}J'allai ensuite dans les pays de Syrie et de Cilicie.
\VS{22}Or j'étais inconnu de visage aux églises de Judée qui sont en Christ,
\VS{23}mais elles avaient seulement entendu dire~: Celui qui autrefois nous persécutait, annonce maintenant la foi qu'il détruisait autrefois.
\VS{24}Et elles glorifiaient Dieu à cause de moi.
\Chap{2}
\TextTitle{Paul et Barnabas se rendent à Jérusalem\FTNTT{\vref{Ac. 15}.}}
\VerseOne{}Quatorze ans après, je montai de nouveau à Jérusalem\FTNT{La grande assemblée de Jérusalem. Voir \vref{Ac. 15}.}, avec Barnabas, et je pris aussi avec moi Tite.
\VS{2}Et ce fut d'après une révélation que j'y montai. J'exposai l'Evangile que je prêche parmi les Gentils à ceux de Jérusalem, en particulier à ceux qui sont les plus considérés, afin de ne pas courir ou avoir couru en vain.
\VS{3}Et même on n'obligea pas Tite, qui était avec moi, de se faire circoncire quoiqu'il fût Grec.
\VS{4}Et cela à cause des faux frères qui s'étaient furtivement introduits et glissés dans l'église, pour épier la liberté que nous avons en Jésus-Christ, afin de nous ramener dans la servitude.
\VS{5}Nous ne leur cédâmes pas un instant et nous résistâmes à leurs exigences, afin que la vérité de l'Evangile soit maintenue parmi vous.
\VS{6}Et je ne suis différent en rien de ceux qui sont les plus estimés, quels qu'ils aient été autrefois, Dieu n'ayant point d'égard à l'apparence extérieure de l'homme car ceux qui sont en estime ne m'ont rien communiqué de plus.
\VS{7}Au contraire, quand ils virent que la prédication de l'Evangile pour les incirconcis m'avait été confiée, comme à Pierre pour les circoncis,
\VS{8}car celui qui a opéré avec efficacité par Pierre dans la charge d'apôtre pour les circoncis, a aussi opéré avec efficacité par moi envers les Gentils.
\VS{9}Jacques, dis-je, Céphas, et Jean, qui sont estimés comme des colonnes, ayant reconnu la grâce que j'avais reçue, me donnèrent, à moi et à Barnabas, la main d'association, afin que nous allions, nous vers les Gentils, et qu'ils aillent eux vers les circoncis.
\VS{10}Ils nous recommandèrent seulement de nous souvenir des pauvres, ce que j'ai eu bien soin de faire.
\TextTitle{Paul reprend Pierre à Antioche}
\VS{11}Mais lorsque Pierre vint à Antioche, je lui résistai en face parce qu'il méritait d'être repris.
\VS{12}Car avant l'arrivée de quelques personnes envoyées par Jacques, il mangeait avec les Gentils, mais quand elles furent venues, il s'esquiva et se sépara des Gentils, craignant les circoncis.
\VS{13}Les autres Juifs aussi usèrent de dissimulation comme lui, de sorte que Barnabas même se laissait entraîner par leur hypocrisie.
\VS{14}Mais quand je vis qu'ils ne marchaient pas droit selon la vérité de l'Evangile, je dis à Pierre devant tous~: Si toi qui es Juif, tu vis comme les Gentils, et non pas comme les Juifs, pourquoi contrains-tu les Gentils à judaïser~?
\TextTitle{Le chrétien est mort à la loi mosaïque}
\VS{15}Nous qui sommes Juifs de naissance, et non point pécheurs d'entre les Gentils,
\VS{16}sachant que l'homme n'est pas justifié par les œuvres de la loi, mais seulement par la foi en Jésus-Christ\FTNT{La justification. Voir \vref{Ro. 5:1}.}, nous, dis-je, nous avons cru en Jésus-Christ, afin que nous soyons justifiés par la foi en Christ, et non point par les œuvres de la loi~; parce que personne ne sera justifié par les œuvres de la loi.
\VS{17}Or si en cherchant à être justifiés par Christ, nous sommes aussi trouvés pécheurs, Christ est-il pourtant serviteur du péché~? A Dieu ne plaise~!
\VS{18}Car si je rebâtis les choses que j'ai renversées, je montre que je suis moi-même un transgresseur.
\VS{19}Car c'est par la loi que je suis mort à la loi, afin de vivre pour Dieu.
\TextTitle{La vie chrétienne doit refléter la vie de Christ\FTNTT{\vref{Ga. 5:15-23}.}}
\VS{20}Je suis crucifié avec Christ~; et si je vis, ce n'est plus moi qui vis, c'est Christ qui vit en moi~; si je vis maintenant dans la chair, je vis dans la foi au Fils de Dieu, qui m'a aimé et qui s'est livré lui-même pour moi.
\VS{21}Je n'anéantis point la grâce de Dieu, car si la justice vient de la loi, Christ est donc mort inutilement.
\Chap{3}
\TextTitle{L'Esprit s'acquière par la foi}
\VerseOne{}Ô Galates insensés~! Qui vous a ensorcelés pour faire que vous n'obéissiez point à la vérité, vous, aux yeux de qui Jésus-Christ a été auparavant dépeint, crucifié au milieu de vous~?
\VS{2}Je voudrais seulement entendre ceci de vous~: Avez-vous reçu l'Esprit par les œuvres de la loi, ou par la prédication de la foi~?
\VS{3}Etes-vous si insensés, qu'après avoir commencé par l'Esprit, voulez-vous maintenant finir par la chair~?
\VS{4}Avez-vous tant souffert en vain~? Si toutefois c'est en vain.
\VS{5}Celui donc qui vous donne l'Esprit, et qui produit en vous les dons miraculeux, le fait-il par les œuvres de la loi ou par la prédication de la foi~?
\TextTitle{L'alliance avec Abraham, une promesse fondée sur la foi\FTNTT{\vref{Ro. 4}.}}
\VS{6}Comme Abraham crut à Dieu, et cela lui fut imputé à justice,
\VS{7}sachez donc que ce sont ceux qui ont la foi qui sont fils d'Abraham.
\VS{8}Aussi, l'Ecriture prévoyant que Dieu justifierait les Gentils par la foi, a auparavant évangélisé à Abraham, en lui disant~: Toutes les nations seront bénies en toi\FTNT{\vref{Ge. 12:3}.}.
\VS{9}C'est pourquoi ceux qui ont la foi sont bénis avec Abraham le croyant.
\TextTitle{L'attachement aux œuvres de la loi produit la malédiction}
\VS{10}Car tous ceux qui s'attachent aux œuvres de la loi sont sous la malédiction~; car il est écrit~: Maudit est quiconque ne persévère pas dans toutes les choses qui sont écrites dans le livre de la loi et ne les met pas en pratique\FTNT{\vref{De. 27:26}.}.
\VS{11}Et que nul ne soit justifié devant Dieu par la loi, cela est évident, puisqu'il est dit~: Le juste vivra de la foi\FTNT{\vref{Ha. 2:4}.}.
\VS{12}Or la loi ne procède pas de la foi, mais elle dit~: L'homme qui mettra ces choses en pratique vivra par elles\FTNT{\vref{Lé. 18:5}.}.
\TextTitle{Le Messie a racheté les chrétiens de la malédiction de la loi}
\VS{13}Christ nous a rachetés de la malédiction de la loi quand il a été fait malédiction pour nous~; car il est écrit~: Maudit est quiconque est pendu au bois\FTNT{\vref{De. 21:23}.},
\VS{14}afin que la bénédiction d'Abraham ait son accomplissement pour les Gentils en Jésus-Christ, et que nous recevions par la foi l'Esprit qui avait été promis.
\VS{15}Mes frères, je parle à la manière des hommes, un testament en bonne forme, bien que fait par un homme, n'est annulé par personne, et personne n'y ajoute.
\VS{16}Or les promesses ont été faites à Abraham et à sa postérité. Il n'est pas dit~: Et aux postérités, comme s'il avait parlé de plusieurs, mais comme parlant d'une seule, et à sa postérité, c'est-à-dire Christ.
\VS{17}Voici ce que j'entends~: Une alliance, que Dieu a confirmée antérieurement, ne peut pas être annulée, et ainsi la promesse rendue vaine, par la loi survenue quatre cent trente ans plus tard.
\VS{18}Car si l'héritage venait de la loi, il ne viendrait plus de la promesse. Or, c'est par la promesse que Dieu a fait à Abraham ce don de sa grâce.
\TextTitle{La loi~: Pédagogue révélant le péché et conduisant à Christ}
\VS{19}A quoi donc sert la loi~? Elle a été donnée ensuite à cause des transgressions, jusqu'à ce que vienne la postérité à qui la promesse avait été faite~; et elle a été promulguée par des anges, au moyen d'un médiateur.
\VS{20}Or le médiateur n'est pas médiateur d'un seul, mais Dieu est un seul.
\VS{21}La loi a-t-elle donc été ajoutée contre les promesses de Dieu~? Nullement. Car s'il avait été donné une loi qui puisse procurer la vie, la justice viendrait réellement de la loi.
\VS{22}Mais l'Ecriture a renfermé tous les hommes sous le péché, afin que ce qui avait été promis soit donné par la foi en Jésus-Christ à ceux qui croient.
\VS{23}Or avant que la foi vienne, nous étions renfermés sous la garde de la loi, en vue de la foi qui devait être révélée.
\VS{24}Ainsi la loi a donc été notre pédagogue\FTNT{Le mot «~pédagogue~» du grec «~paidagogos~»~: «~celui qui dirige un garçon~». Un pédagogue était un tuteur, un gardien et un guide de garçons. Parmi les Grecs et les Romains, le mot était appliqué aux esclaves dignes de confiance qui étaient chargés de veiller à la vie et à la moralité des garçons appartenant aux classes supérieures. Les garçons ne pouvaient faire le moindre pas hors de la maison sans ces tuteurs tant qu'ils n'avaient pas atteint leur majorité.} pour nous amener à Christ, afin que nous soyons justifiés par la foi.
\VS{25}Mais la foi étant venue, nous ne sommes plus sous ce pédagogue.
\TextTitle{Ceux qui croient au Messie sont justifiés}
\VS{26}Parce que vous êtes tous fils de Dieu par la foi en Jésus-Christ,
\VS{27}car vous tous qui avez été baptisés en Christ, vous avez revêtu Christ.
\VS{28}Il n'y a plus ni Juif ni Grec, il n'y a plus ni esclave ni libre, il n'y a plus ni homme ni femme~; car vous êtes tous un en Jésus-Christ\FTNT{\vref{Ro. 10:12}~; \vref{Col. 3:11}.}.
\VS{29}Et si vous êtes de Christ, vous êtes donc la postérité d'Abraham, et héritiers selon la promesse.
\Chap{4}
\VerseOne{}Or aussi longtemps que l'héritier est enfant\FTNT{«~Enfant~», du grec «~nepios~», signifie aussi «~ignorant~».}, je dis qu'il ne diffère en rien d'un esclave, quoiqu'il soit le maître de tout.
\VS{2}Mais il est sous des tuteurs et des administrateurs jusqu'au temps déterminé par le Père.
\VS{3}Nous aussi, lorsque nous étions enfants, nous étions sous l'esclavage des rudiments du monde.
\VS{4}Mais lorsque les temps ont été accomplis, Dieu a envoyé son Fils, né d'une femme, né sous la loi,
\VS{5}afin qu'il rachète ceux qui étaient sous la loi, afin que nous recevions l'adoption.
\VS{6}Et parce que vous êtes fils, Dieu a envoyé l'Esprit de son Fils dans vos cœurs, lequel crie~: Abba, c'est-à-dire Père.
\VS{7}Maintenant donc tu n'es plus esclave, mais fils~; or si tu es fils, tu es aussi héritier de Dieu par Christ.
\TextTitle{Le légalisme et la religiosité privent de la grâce}
\VS{8}Autrefois, ne connaissant pas Dieu, vous serviez des dieux qui ne le sont pas de leur nature.
\VS{9}Et maintenant que vous avez connu Dieu, ou plutôt que vous avez été connus de Dieu, comment retournez-vous encore à ces faibles et misérables éléments, auxquels vous voulez encore vous asservir comme auparavant~?
\VS{10}Vous observez les jours, les mois, les temps et les années.
\VS{11}Je crains d'avoir travaillé inutilement pour vous.
\VS{12}Soyez comme moi~; car je suis aussi comme vous~; je vous en prie mes frères.
\VS{13}Vous ne m'avez fait aucun tort. Et vous savez que ce fut à cause d'une infirmité de la chair\FTNT{Les Ecritures ne donnent pas de précisions au sujet de l'infirmité de la chair dont souffrait Paul. On suppose toutefois qu'il avait un handicap au niveau de ses yeux. Quatre arguments viennent renforcer cette hypothèse. Tout d'abord, l'allusion de Paul aux Galates qui étaient prêts à «~s'arracher les yeux~» pour les lui donner (\vref{Ga. 4:15}) et le fait qu'il ait lui-même écrit cette épître avec de «~grandes lettres~» (\vref{Ga. 6:11}). Ensuite, lors de sa comparution devant le sanhédrin à Jérusalem, Paul n'a pas reconnu le premier prêtre pourtant facilement identifiable par sa tenue vestimentaire (\vref{Ac. 23:5}). Enfin, l'apôtre avait l'habitude de dicter ses lettres, ce qui constitue un argument majeur. L'épître aux Galates était une exception parce qu'il n'avait sans doute pas de secrétaire à disposition.} que je vous ai pour la première fois évangélisés.
\VS{14}Et vous ne m'avez point méprisé ni rejeté à cause de ces épreuves que j'ai dans ma chair~; mais vous m'avez reçu comme un ange de Dieu, et comme Jésus-Christ.
\VS{15}Où donc est l'expression de votre bonheur~? Car je vous atteste que, si cela avait été possible, vous vous seriez arrachés les yeux pour me les donner.
\VS{16}Suis-je donc devenu votre ennemi en vous disant la vérité~?
\VS{17}Ils ont du zèle pour vous, mais non loyalement. Au contraire, ils veulent vous détacher de nous afin que vous soyez zélés pour eux.
\VS{18}Il est bon d'être zélé pour le bien en tout temps, et non pas seulement quand je suis présent parmi vous.
\TextTitle{La loi et la grâce ne peuvent cohabiter~: Agar et Sara représentent deux alliances}
\VS{19}Mes petits enfants, pour qui j'éprouve de nouveau les douleurs de l'enfantement, jusqu'à ce que Christ soit formé en vous,
\VS{20}je voudrais être maintenant avec vous, et changer de langage, car je suis dans une grande inquiétude à votre sujet.
\VS{21}Dites-moi, vous qui voulez être sous la loi, ne comprenez-vous point la loi~?
\VS{22}Car il est écrit qu'Abraham eut deux fils, un de l'esclave, et un de la femme libre.
\VS{23}Mais celui de l'esclave naquit selon la chair~; et celui de la femme libre naquit en vertu de la promesse.
\VS{24}Ces faits ont une valeur allégorique, car ces deux femmes sont deux alliances~: L'une du Mont Sinaï, qui n'enfante que des esclaves, et c'est Agar.
\VS{25}Car le nom d'Agar veut dire Sinaï, qui est une montagne en Arabie correspondant à la Jérusalem actuelle qui est dans la servitude avec ses enfants.
\VS{26}Mais la Jérusalem d'en haut est la femme libre, et c'est notre mère à nous tous.
\VS{27}Car il est écrit~: Réjouis-toi, stérile, toi qui n'enfantes point~! Eclate et pousse des cris, toi qui n'as pas éprouvé les douleurs de l'enfantement~! Car les enfants de la délaissée seront plus nombreux que les enfants de celle qui était mariée\FTNT{\vref{Es. 54:1}.}.
\VS{28}Or pour nous mes frères, nous sommes enfants de la promesse comme Isaac.
\VS{29}Et de même qu'alors celui qui était né selon la chair persécutait celui qui était né selon l'Esprit, il en est de même maintenant.
\VS{30}Mais que dit l'Ecriture~? Chasse l'esclave et son fils, car le fils de l'esclave n'héritera pas avec le fils de la femme libre\FTNT{\vref{Ge. 21:10}.}.
\VS{31}C'est pourquoi, mes frères, nous ne sommes pas enfants de l'esclave, mais de la femme libre.
\Chap{5}
\TextTitle{Le Messie nous a libéré de la servitude}
\VerseOne{}Demeurez donc fermes dans la liberté pour laquelle Christ nous a affranchis, et ne vous mettez plus sous le joug de la servitude.
\VS{2}Moi, Paul, je vous dis que si vous vous faites circoncire, Christ ne vous servira de rien.
\VS{3}Et j'affirme encore une fois à tout homme qui se fait circoncire qu'il est tenu de pratiquer la loi tout entière.
\VS{4}Vous êtes séparés de Christ, vous tous qui cherchez la justification dans la loi~; vous êtes déchus de la grâce.
\VS{5}Mais pour nous, nous attendons par l'Esprit l'espérance d'être justifiés par la foi.
\VS{6}Car en Jésus-Christ ni la circoncision ni le prépuce\FTNT{Voir le commentaire en \vref{1 Co. 7:18}} n'ont de valeur, mais seulement la foi qui opère par la charité.
\VS{7}Vous couriez bien~: Qui vous a arrêtés pour vous empêcher d'obéir à la vérité~?
\VS{8}Cette influence ne vient pas de celui qui vous appelle.
\VS{9}Un peu de levain fait lever toute la pâte\FTNT{\vref{1 Co. 5:6}.}.
\VS{10}J'ai cette confiance en vous dans le Seigneur que vous n'aurez pas d'autre sentiment~; mais celui qui vous trouble, quel qu'il soit, en portera la condamnation.
\VS{11}Quant à moi, mes frères, si je prêche encore la circoncision, pourquoi suis-je encore persécuté~? Le scandale de la croix est donc aboli.
\VS{12}Plaise à Dieu que ceux qui vous troublent soient retranchés~!
\VS{13}Car mes frères, vous avez été appelés à la liberté, seulement ne faites pas de cette liberté une occasion de vivre selon la chair, mais servez-vous les uns les autres avec charité.
\VS{14}Car toute la loi est accomplie dans cette seule parole~: Tu aimeras ton prochain comme toi-même\FTNT{\vref{Lé. 19:18}~; \vref{Mt. 22:39}.}.
\VS{15}Mais si vous vous mordez et vous dévorez les uns les autres, prenez garde que vous ne soyez détruits les uns par les autres.
\VS{16}Je vous dis donc~: Marchez selon l'Esprit, et vous n'accomplirez point les désirs de la chair.
\TextTitle{La chair et ses œuvres s'opposent à l'Esprit de Dieu\FTNTT{\vref{Ro. 8:2}.}}
\VS{17}Car la chair a des désirs contraires à ceux de l'Esprit, et l'Esprit en a des contraires à ceux de la chair~; et ils sont opposés entre eux afin que vous ne fassiez point ce que vous voudriez.
\VS{18}Or si vous êtes conduits par l'Esprit, vous n'êtes point sous la loi.
\VS{19}Car les œuvres de la chair sont évidentes~: Ce sont l'adultère, la fornication, l'impureté, l'impudicité,
\VS{20}l'idolâtrie, la sorcellerie\FTNT{La sorcellerie~: du grec «~pharmakeia~»~: «~usage~» ou «~administration~» de drogues, «~empoisonnement~», «~sorcellerie~», «~arts magiques~», souvent trouvés en liaison avec l'idolâtrie et nourrie par celle-ci.}, les inimitiés, les querelles, les jalousies, les animosités, les disputes, les divisions, les sectes,
\VS{21}les envies, les meurtres, l'ivrognerie, les excès de table, et les choses semblables à celles-là, au sujet desquelles je vous prédis, comme je vous l'ai déjà dit, que ceux qui commettent de telles choses n'hériteront point le Royaume de Dieu.
\TextTitle{Le fruit de l'Esprit\FTNTT{\vref{Jn. 15:1-5}~; \vref{Ga. 2:20}.}}
\VS{22}Mais le fruit de l'Esprit c'est la charité\FTNT{Il est question ici de l'amour «~agape~»~: l'amour fraternel, la charité désintéressée.}, la joie, la paix, la patience, la bonté, la bienveillance, la foi, la douceur, la tempérance.
\VS{23}La loi n'est pas contre ces choses.
\VS{24}Ceux qui sont à Christ ont crucifié la chair avec ses passions et ses désirs.
\VS{25}Si nous vivons par l'Esprit, marchons aussi par l'Esprit.
\VS{26}Ne cherchons pas une vaine gloire, en nous provoquant les uns les autres et en nous portant envie les uns aux autres.
\Chap{6}
\TextTitle{La mise en pratique de la vie nouvelle en Jésus-Christ}
\VerseOne{}Mes frères, lorsqu'un homme est surpris en quelque faute, vous qui êtes spirituels, redressez-le avec un esprit de douceur. Prends garde à toi-même, de peur que tu ne sois aussi tenté.
\VS{2}Portez les fardeaux les uns des autres, et vous accomplirez ainsi la loi de Christ.
\VS{3}Car si quelqu'un pense être quelque chose, quoiqu'il ne soit rien, il s'abuse lui-même.
\VS{4}Que chacun examine ses propres œuvres, et alors il aura de quoi se glorifier pour lui-même seulement, et non par rapport aux autres.
\VS{5}Car chacun portera son propre fardeau.
\VS{6}Que celui à qui l'on enseigne la parole fasse part en tous biens à celui qui l'enseigne\FTNT{Le mot «~bien~» vient du grec «~agathos~» qui donne en français~: «~de bonne constitution ou nature~», «~utile~», «~salutaire~», «~bon~», «~agréable~», «~plaisant~», «~joyeux~», «~heureux~», «~excellent~», «~distingué~», «~droit~», «~honorable~» et n'a rien à voir avec les biens matériels (Voir \vref{Ga. 6:10}). Il ne doit en aucun cas servir de prétexte à ceux qui enseignent la Parole de Dieu pour exiger l'argent et les biens matériels des chrétiens. Ces derniers doivent donner sans contrainte, s'ils le veulent et comme ils le veulent (\vref{2 Co. 9:7}). Le salaire de l'ouvrier du Seigneur c'est avant tout le gîte et le couvert (\vref{Mt. 10:10}~; \vref{Lu. 10:8}~; \vref{1 Ti. 6:8}). Ainsi, malgré le droit qu'il avait de moissonner les biens matériels pour avoir semé des biens spirituels (\vref{1 Co. 9:11-12}), Paul «~n'a désiré ni l'or ni l'argent~» mais a travaillé de ses propres mains afin de pourvoir à ses besoins et de n'être à la charge de personne (\vref{Ac. 20:33-35}~; \vref{1 Th. 2:9}~; \vref{2 Th. 3:8}~; \vref{2 Co. 12:14} ).}.
\VS{7}Ne vous séduisez pas, on ne se moque pas de Dieu. Ce qu'un homme aura semé, il le moissonnera aussi.
\VS{8}C'est pourquoi celui qui sème pour sa chair moissonnera de la chair la corruption~; mais celui qui sème pour l'Esprit moissonnera de l'Esprit la vie éternelle.
\VS{9}Ne nous lassons pas de faire le bien~; car nous moissonnerons au temps convenable, si nous ne nous relâchons pas.
\VS{10}C'est pourquoi, pendant que nous en avons le temps, faisons du bien envers tous, mais principalement envers ceux qui sont de la famille de la foi.
\VS{11}Vous voyez avec quelles grandes lettres je vous ai écrit de ma propre main.
\VS{12}Tous ceux qui veulent se rendre agréables selon la chair vous contraignent à vous faire circoncire, uniquement afin de ne pas être persécutés pour la croix de Christ.
\VS{13}Car les circoncis eux-mêmes n'observent pas la loi~; mais ils veulent que vous soyez circoncis pour se glorifier dans votre chair.
\VS{14}Pour ce qui me concerne, loin de moi la pensée de me glorifier d'autre chose que de la croix de notre Seigneur Jésus-Christ, par qui le monde est crucifié pour moi, comme je le suis pour le monde~!
\VS{15}Car ce n'est rien que d'être circoncis ou incirconcis~; ce qui est quelque chose c'est d'être une nouvelle créature.
\VS{16}Que la paix et la miséricorde soient sur tous ceux qui suivront cette règle, et sur l'Israël de Dieu.
\TextTitle{Conclusion}
\VS{17}Au reste, que personne ne me fasse de la peine, car je porte sur mon corps les marques du Seigneur Jésus.
\VS{18}Mes frères, que la grâce de notre Seigneur Jésus-Christ soit avec votre esprit~! Amen~!
\PPE{}
\end{multicols}
