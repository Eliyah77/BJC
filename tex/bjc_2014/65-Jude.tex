\ShortTitle{Jude}\BookTitle{Jude}\BFont
\begin{multicols}{2}
\TextTitle{[Introduction]}
\Chap{1}
\VerseOne{}Jude serviteur de Jésus-Christ, et frère de Jacques, à ceux qui ont été appelés par l'Evangile, que Dieu a sanctifiés et gardés pour Jésus-Christ :
\VS{2}Que la miséricorde, la paix et l'amour vous soient multipliés.
\TextTitle{[Mise en garde contre l'apostasie]}
\VS{3}Mes bien-aimés, comme je désirais vous écrire avec empressement au sujet de notre salut commun, j’ai jugé nécessaire de le faire pour vous exhorter à combattre pour la foi qui a été transmise aux saints une fois pour toutes.
\VS{4}Car il s’est glissé parmi vous, certains hommes dont la condamnation est écrite depuis longtemps, des impies qui changent la grâce de notre Dieu en dissolution, et qui renient le seul Dominateur Jésus-Christ, notre Dieu et Seigneur.
\textit{[Exemples historiques d'incrédulité et de révolte]}
\VS{5}Je veux vous rappeler une chose que vous savez déjà : C'est que le Seigneur après avoir délivré le peuple du pays d'Egypte, fit ensuite périr les incrédules,
\VS{6}qu’il a réservés pour le jugement du grand jour, enchainés éternellement par les ténèbres, les anges qui n'ont pas gardé leur origine, mais qui ont abandonné leur propre demeure ;
\VS{7}que Sodome et Gomorrhe, et les villes voisines qui s'étaient abandonnées comme eux à l'impureté et à des vices contre nature, sont données en exemples, subissant la peine d’un feu éternel.
\textit{[Description des faux docteurs]}
\VS{8}Malgré cela, ces hommes aussi, plongés dans leurs rêveries, souillent leur chair, méprisent l’autorité, et blasphèment contre les dignités.
\VS{9}Or, l'archange Michel, lorsqu’il contestait avec le diable et lui disputait le corps de Moïse, n'osa pas prononcer contre lui un jugement blasphématoire, mais il dit seulement : Que le Seigneur te réprime !
\VS{10}Eux, au contraire, ils blasphèment contre tout ce qu'ils ignorent, et ils se corrompent dans tout ce qu'ils savent naturellement, comme font les bêtes brutes.
\VS{11}Malheur à eux ! Car ils ont suivi la voie de Caïn, et ils se sont jetés dans l’égarement de Balaam, pour l’amour du gain, ils se sont perdus par la rébellion de Koré (1).
\VS{12}Ce sont des écueils dans vos agapes, lorsqu’ils prennent leurs repas avec vous sans aucune retenue, et se repaissant eux-mêmes ; ce sont des nuées sans eau, emportées par des vents çà et là ; des arbres d’automne dont le fruit se pourrit, et sans fruits, deux fois morts, et déracinés ;
\VS{13}des vagues impétueuses de la mer, jetant l'écume de leurs impuretés ; des étoiles errantes, à qui l'obscurité des ténèbres est réservée éternellement.
\VS{14}C’est aussi pour eux qu’Hénoc, le septième homme après Adam, a prophétisé en disant :
\VS{15}Voici, le Seigneur est venu avec ses saintes myriades, pour exercer un jugement contre tous les hommes, et pour convaincre tous les impies parmi eux de tous les actes d'impiété qu’ils ont commis et de toutes les paroles blasphématoires qu’ont proférées contre lui des pécheurs impies.
\VS{16}Ce sont des gens qui murmurent, qui se plaignent toujours, qui marchent selon leurs convoitises, qui ont à la bouche des discours hautains, qui admirent les personnes pour le profit qui leur en revient.
\VS{17}Mais vous, mes bien-aimés, souvenez-vous des choses qui ont été prédites par les apôtres de notre Seigneur Jésus-Christ.
\VS{18}Ils vous disaient que dans les derniers temps il y aurait des moqueurs, qui marcheraient selon leurs convoitises impies.
\VS{19}Ce sont ceux qui provoquent des divisions, des gens sensuels, n'ayant pas l'Esprit.
\TextTitle{[Exhortation aux chrétiens]}
\VS{20}Mais vous, mes bien-aimés, vous édifiant vous-mêmes sur votre très sainte foi, et priant par le Saint-Esprit,
\VS{21}maintenez-vous les uns les autres dans l'amour de Dieu, en attendant la miséricorde de notre Seigneur Jésus-Christ, pour obtenir la vie éternelle.
\VS{22}Et ayez pitié des uns en usant de discernement ;
\VS{23}sauvez-en d’autres avec crainte, en les arrachant hors du feu, haïssant jusqu’à la tunique souillée par la chair.
\textit{[Conclusion]}
\VS{24}Or, à celui qui est puissant pour vous préserver de toute chute et vous faire paraître devant sa gloire irréprochables et dans l’allégresse,
\VS{25}à Dieu, seul sage, notre Sauveur, par Jésus-Christ notre Seigneur, soient gloire et magnificence, force et puissance, dès maintenant et dans tous les siècles, Amen !
\PPE{}
\end{multicols}
