\ShortTitle{Jud.}\BookTitle{Jude}\BFont
\noindent\hrulefill
{\footnotesize
\textit{
\bigskip
{\centering{}
\\Auteur~: Jude
\\(Gr.~: Ioudas / Origine héb.~: Yehuwdah)
\\Signification~: Qu'il (Dieu) soit loué
\\Thème~: Le combat de la foi
\\Date de rédaction~: Env. 68 ap. J.-C.\\}
}
\textit{
\\Cette épître, écrite par Jude, l'un des frères de Jésus-Christ homme, était destinée aux églises d'Asie Mineure. Il les avertit contre les faux docteurs et les exhorte à se confier en Christ pour leur salut éternel.\bigskip
}
}
\par\nobreak\noindent\hrulefill
\begin{multicols}{2}
\Chap{1}
\TextTitle{Introduction}
\VerseOne{}Jude, serviteur de Jésus-Christ, et frère de Jacques, à ceux qui ont été appelés par l'Evangile, que Dieu a sanctifiés et gardés pour Jésus-Christ~:
\VS{2}Que la miséricorde, la paix et l'amour vous soient multipliés~!
\TextTitle{Exhortation à combattre pour la foi}
\VS{3}Mes bien-aimés, comme je désirais vous écrire avec empressement au sujet de notre salut commun, j'ai jugé nécessaire de le faire pour vous exhorter à combattre pour la foi qui a été donnée une fois pour toutes aux saints.
\VS{4}Car il s'est glissé parmi vous certains hommes, dont la condamnation est écrite depuis longtemps, des impies, qui changent la grâce de notre Dieu en dissolution, et qui renient le seul dominateur, Jésus-Christ, notre Dieu et Seigneur.
\TextTitle{Exemples d'incrédulité et de révolte}
\VS{5}Or je veux vous rappeler une chose, que vous savez déjà~: C'est que le Seigneur après avoir délivré le peuple du pays d'Egypte, détruisit ensuite ceux qui n'avaient pas cru,
\VS{6}et qu'il a réservé sous l'obscurité, dans des liens éternels, jusqu'au jugement du grand jour, les anges qui n'ont pas gardé leur origine, mais qui ont abandonné leur propre demeure~;
\VS{7}et que Sodome et Gomorrhe et les villes voisines, qui s'étaient abandonnées de la même manière que celles-ci à l'impureté et qui avaient couru après les péchés contre nature, ont été mises pour servir d'exemple, ayant reçu la punition du feu éternel. 
\TextTitle{Caractéristiques des faux docteurs}
\VS{8}Malgré cela, ces hommes aussi, plongés dans leurs rêveries, souillent leur chair, méprisent l'autorité et blasphèment contre les dignités.
\VS{9}Or l'archange Michel, lorsqu'il contestait avec le diable et lui disputait le corps de Moïse, n'osa pas prononcer contre lui un jugement blasphématoire, mais il dit seulement~: Que le Seigneur te censure fortement~!
\VS{10}Eux, au contraire, ils blasphèment contre tout ce qu'ils ignorent, et ils se corrompent dans tout ce qu'ils savent naturellement, comme font les bêtes destituées d'intelligence.
\VS{11}Malheur à eux~! Car ils ont suivi la voie de Caïn, et ont couru par un égarement tel que celui de Balaam, après la récompense, et ont péri par une rébellion semblable à celle de Koré\FTNT{No. 16:1-35~; 2 Pi. 2:15~; Ap. 2:14.}.
\VS{12}Ceux-ci sont des écueils dans vos agapes, lorsqu'ils prennent leurs repas avec vous sans crainte, et se repaissant eux-mêmes. Ce sont des nuées sans eau, emportées par des vents çà et là~; des arbres au déclin de l'automne sans fruits, deux fois morts, et déracinés~;
\VS{13}des vagues impétueuses de la mer, jetant l'écume de leurs impuretés~; des étoiles errantes, à qui l'obscurité des ténèbres est réservée éternellement.
\VS{14}C'est aussi pour eux qu'Hénoc, le septième homme après Adam, a prophétisé, en disant~:
\VS{15}Voici, le Seigneur est venu avec ses saints, qui sont par millions, pour juger tous les hommes, et pour convaincre tous les méchants d'entre eux de toutes leurs méchantes actions qu'ils ont commises méchamment, et de toutes les paroles injurieuses que les pécheurs impies ont proférées contre lui.
\VS{16}Ce sont des gens qui murmurent, qui se plaignent toujours, qui marchent selon leurs convoitises, dont la bouche prononce des discours fort enflés, et qui admirent les personnes pour le profit qui leur en revient.
\VS{17}Mais vous, mes bien-aimés, souvenez-vous des paroles qui ont été dites auparavant par les apôtres de notre Seigneur Jésus-Christ.
\VS{18}Et comment ils vous disaient que dans les derniers temps il y aurait des moqueurs qui marcheraient selon leurs convoitises impies.
\VS{19}Ce sont ceux qui se séparent eux-mêmes, des gens sensuels, n'ayant pas l'Esprit.
\TextTitle{Exhortations}
\VS{20}Mais vous, mes bien-aimés, vous appuyant vous-mêmes sur votre très sainte foi, et priant par le Saint-Esprit,
\VS{21}conservez-vous les uns les autres dans l'amour de Dieu, en attendant la miséricorde de notre Seigneur Jésus-Christ pour obtenir la vie éternelle.
\VS{22}Et ayez pitié des uns en usant de discrétion~;
\VS{23}et sauvez les autres par la frayeur, les arrachant comme hors du feu, et haïssez même la robe souillée par la chair.
\TextTitle{Conclusion}
\VS{24}Or à celui qui est puissant pour vous garder sans que vous fassiez aucune chute et vous faire paraître devant sa gloire, irréprochables dans l'allégresse,
\VS{25}à Dieu, seul sage, notre Sauveur, par Jésus-Christ, notre Seigneur, soient gloire, magnificence, force et puissance, dès maintenant et dans tous les siècles~! Amen~!
\PPE{}
\end{multicols}
