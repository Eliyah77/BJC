\ShortTitle{1 Samuel}\BookTitle{1 Samuel}\BFont
\noindent\hrulefill
{\footnotesize
\textit{
\bigskip
{\centering{}
\\Auteur : Inconnu
\\(Heb. : Shemuw'el)
\\Signification : Entendu, exaucé de Dieu
\\Thème : Histoire de Samuel, Saül et David
\\Date de rédaction : 10\up{ème} siècle av. J.-C.\\}
}
%\bigskip
\textit{
\\Samuel était le fils d'Elkana, de la montagne d'Ephraïm. Anne, sa mère, avait longtemps désiré un enfant. Elle fit donc une alliance avec Dieu en lui promettant de lui consacrer son premier fils s'il la rendait féconde. Ainsi, dès son plus jeune âge, Samuel fut amené à la maison de Dieu où il grandit aux côtés d'Eli, le sacrificateur. A la mort de ce dernier, Samuel exerça les fonctions de juge, sacrificateur et prophète. C'est en son temps qu'Israël exprima le désir d'avoir un roi, marquant ainsi la fin de l'ère des juges et le début de la monarchie en Israël.
%\bigskip
\\Ce livre relate l'histoire de Saül, premier roi d'Israël, à qui Yahweh accorda de puissantes victoires notamment sur les Philistins, grands ennemis du peuple de Dieu. Mais très vite, Saül s'écarta de la volonté de Dieu, aussi Yahweh le disqualifia et choisit pour lui succéder au trône un homme de la tribu de Juda : David, fils d'Isaï. Son accession à la royauté ne fut pas immédiate. David dut faire preuve de patience, de courage et de foi en son Dieu au milieu de nombreuses persécutions. L'expérience des deux premiers rois d'Israël est une exhortation à l'obéissance à Dieu.\bigskip
}
}
\par\nobreak\noindent\hrulefill
\begin{multicols}{2}
\Chap{1}
\TextTitle{Stérilité d'Anne}
\VerseOne{}Il y avait un homme de Ramathaïm-Tsophim, de la montagne d'Ephraïm, nommé Elkana, fils de Jeroham, fils d'Elihu, fils de Thohu, fils de Tsuph, Ephratien.
\VS{2}Il avait deux femmes, dont l'une s'appelait Anne, et l'autre Peninna. Peninna avait des enfants, mais Anne n'en avait pas.
\VS{3}Or cet homme-là montait tous les ans, de sa ville à Silo\FTNT{Jos. 18:1.}, pour adorer Yahweh des armées, et lui offrir des sacrifices. Là étaient les deux fils d'Eli, Hophni et Phinées, sacrificateurs de Yahweh.
\VS{4}Le jour où Elkana offrait son sacrifice, il donnait des portions à Peninna, sa femme, à tous les fils et à toutes les filles qu'il avait d'elle.
\VS{5}Mais il donnait à Anne une portion double ; car il aimait Anne, mais Yahweh avait fermé sa matrice\FTNT{Dieu est celui qui ferme et ouvre les portes des bénédictions.}.
\VS{6}Sa rivale lui portait envie et la provoquait fort aigrement afin de l'irriter, car Yahweh avait fermé sa matrice.
\VS{7}Et Elkana faisait donc ainsi tous les ans. Mais quand Anne montait à la maison de Yahweh, Peninna la provoquait de la même manière, et Anne pleurait et ne mangeait pas.
\VS{8}Elkana, son mari, lui disait : Anne, pourquoi pleures-tu et pourquoi ne manges-tu pas ? Pourquoi ton cœur est-il triste ? Est-ce que je ne vaux pas pour toi mieux que dix fils ?
\TextTitle{Prière et voeu d'Anne à Yahweh}
\VS{9}Anne se leva après avoir mangé et bu à Silo. Et le sacrificateur Eli était assis sur un siège, près de l'un des poteaux du temple de Yahweh.
\VS{10}Elle donc, ayant le coeur rempli d'amertume, pria Yahweh en pleurant abondamment.
\VS{11}Et elle fit un vœu, en disant : Yahweh des armées ! Si tu regardes attentivement l'affliction de ta servante, et si tu te souviens de moi, et n'oublies pas ta servante, et que tu donnes à ta servante un enfant mâle, je le donnerai à Yahweh pour tous les jours de sa vie, et aucun rasoir ne passera sur sa tête.
\VS{12}Il arriva, comme elle continuait à prier devant Yahweh, qu'Eli observait sa bouche.
\VS{13}Or Anne parlait dans son cœur, elle ne faisait que remuer ses lèvres et on n'entendait pas sa voix. C'est pourquoi Eli estima qu'elle était ivre,
\VS{14}et Eli lui dit : Jusqu'à quand seras-tu ivre ? Eloigne-toi du vin.
\VS{15}Mais Anne répondit, et dit : Je ne suis pas ivre, mon seigneur, je suis une femme affligée en son esprit, je n'ai bu ni vin ni boisson forte ; mais je répandais mon âme devant Yahweh.
\VS{16}Ne mets pas ta servante au rang d'une fille de Bélial\FTNT{Voir commentaire en 1 S. 2:12.}, car c'est l'excès de ma douleur et de mon affliction qui m'a fait parler jusqu'à présent.
\VS{17}Alors Eli répondit, et dit : Va en paix, et que le Dieu d'Israël veuille t'accorder la demande que tu lui as faite !
\VS{18}Et elle dit : Que ta servante trouve grâce à tes yeux ! Puis cette femme poursuivit son voyage. Elle mangea, et son visage ne fut plus le même.
\TextTitle{Naissance de Samuel}
\VS{19}Après cela, ils se levèrent de bon matin, et se prosternèrent devant Yahweh, puis ils s'en retournèrent et revinrent dans leur maison à Rama. Elkana connut Anne, sa femme, et Yahweh se souvint d'elle.
\VS{20}Il arriva donc, quelque temps après, qu'Anne conçut et enfanta un fils, elle le nomma Samuel, parce que, dit-elle, je l'ai demandé à Yahweh.
\VS{21}Puis Elkana, son mari, monta avec toute sa maison, pour offrir à Yahweh le sacrifice annuel et son vœu.
\VS{22}Mais Anne n'y monta pas, car elle dit à son mari : Je n'irai pas jusqu'à ce que le petit enfant soit sevré, et alors je le mènerai, afin qu'il soit présenté devant Yahweh et qu'il demeure toujours-là.
\VS{23}Elkana, son mari, lui dit : Fais ce qui te semblera bon, reste jusqu'à ce que tu l'aies sevré. Seulement que Yahweh accomplisse sa parole ! Ainsi cette femme resta et allaita son fils, jusqu'à ce qu'elle l'ait sevré.
\TextTitle{Samuel chez Eli ; Anne accomplit son vœu}
\VS{24}Et dès qu'elle l'eut sevré, elle le fit monter avec elle, et ayant pris trois veaux, un épha de farine et une outre de vin, elle le mena dans la maison de Yahweh à Silo ; l'enfant était très jeune.
\VS{25}Puis ils égorgèrent un veau, et ils amenèrent l'enfant à Eli.
\VS{26}Elle dit : Pardon, mon seigneur ! Aussi vrai que ton âme vit, mon seigneur, je suis cette femme qui me tenais en ta présence pour prier Yahweh.
\VS{27}J'ai prié pour avoir cet enfant, et Yahweh m'a accordé la demande que je lui ai faite.
\VS{28}C'est pourquoi je le prête à Yahweh : Il sera prêté à Yahweh pour tous les jours de sa vie. Et ils se prosternèrent là devant Yahweh.
\Chap{2}
\TextTitle{Prière d'Anne}
\VerseOne{}Alors Anne pria, et dit : Mon cœur se réjouit en Yahweh, ma force\FTNT{Littéralement « corne ».} a été relevée par Yahweh ; ma bouche s'est ouverte contre mes ennemis, parce que je me suis réjouie de ton salut\FTNT{Le mot « salut » vient de l'hébreu « yeshuw'ah » c'est-à-dire « Jésus ». Voir commentaire en Es. 26:1.}.
\VS{2}Nul n'est saint comme Yahweh ; car il n'y en a pas d'autre que toi ; et il n'y a pas de rocher\FTNT{Voir commentaire en Es. 8:13-17.} tel que notre Dieu.
\VS{3}Ne proférez pas tant de paroles hautaines ; qu'il ne sorte pas de votre bouche des paroles arrogantes ; car Yahweh est le Dieu qui sait tout, c'est lui qui pèse toutes les actions.
\VS{4}L'arc des puissants est brisé, mais ceux qui chancellent ont la force pour ceinture.
\VS{5}Ceux qui étaient rassasiés se louent pour du pain, mais les affamés ont cessé de l'être ; même la stérile en a enfanté sept, et celle qui avait beaucoup de fils est devenue languissante.
\VS{6}Yahweh est celui qui fait mourir et qui fait vivre, qui fait descendre au scheol et qui en fait remonter.
\VS{7}Yahweh appauvrit et il enrichit, il abaisse et il élève.
\VS{8}Il élève le pauvre de la poussière, et il tire le misérable de dessus le fumier, pour le faire asseoir avec les nobles. Et il leur donne en héritage un trône de gloire ; car les fondements de la terre sont à Yahweh, et il a posé sur eux la terre habitable.
\VS{9}Il gardera les pieds de ses bien-aimés, et les méchants se tairont dans les ténèbres ; car l'homme ne triomphera pas par sa force.
\VS{10}Ceux qui contestent contre Yahweh seront effrayés ; des cieux il lancera son tonnerre sur chacun d'eux ; Yahweh jugera les extrémités de la terre. Et il donnera la force à son Roi\FTNT{Le Roi dont il est question ici est le Seigneur Jésus-Christ, le Roi des rois (Za. 14:9 ; Ap. 19:16).}, et élèvera la force de son Messie\FTNT{Anne a annoncé la glorification ou la résurrection du Seigneur Jésus, le Messie (Jn. 3:14).}.
\VS{11}Puis Elkana s'en alla à Rama dans sa maison, et le jeune garçon vaquait au service de Yahweh, en présence du sacrificateur Eli.
\TextTitle{Corruption des fils d'Eli}
\VS{12}Or les fils d'Eli\FTNT{Les fils d'Eli, Hophni et Phinées étaient corrompus. Ils volaient les offrandes de Dieu, couchaient avec les femmes qui venaient servir et adorer Dieu. L'esprit qui animait ces sacrificateurs n'a pas disparu après leur mort, mais il opère encore dans beaucoup d'institutions religieuses actuelles. Beaucoup de dirigeants d'églises continuent à s'approprier ce qui appartient à Dieu (l'adoration, les âmes… ). Ils ne craignent pas Yahweh. Ils abusent de leur position et de leur autorité pour contraindre leurs fidèles à leur donner la dîme et toutes sortes d'offrandes. Ils font payer les entretiens, les prières, et les divers dons qu'ils peuvent avoir. Non seulement l'esprit qui animait les fils d'Eli existe encore, mais il s'est accru en ces temps actuels.} étaient des fils de Bélial\FTNT{Les fils d'Eli étaient qualifiés de « fils de Bélial ». Ce mot vient de l'hébreu « beliya'al » qui signifie « indigne », « bon à rien », « méchant », « ruine », « destruction ». Il est à noter que Bélial est aussi un des noms de Satan. (2 Co. 6:15). Les fils d'Eli servaient Dieu sans le connaître. En fait, ils étaient au service de Satan. Ce terme est également utilisé au sujet des méchants qui incitèrent les Israélites à servir les dieux étrangers (De. 13:12-14), les hommes iniques de Guibea (Jg. 19:22 ; Jg. 20:13), les deux vauriens qui accusèrent Naboth (1 R. 21:10-13) et les individus qui s'opposèrent à la monarchie (1 S. 10:27 ; 2 S. 20:1 ; 2 Ch. 13:7). Voir aussi De. 13:13 ; 15:9 ; Job. 34:18 ; Ps. 18:4 ; 34:17 ; Pr. 6:12 ; 16:27 ; 19:28 ; Na. 1:11.} et ils ne connaissaient pas Yahweh.
\VS{13}Et voici la coutume de ces sacrificateurs envers le peuple : Lorsque quelqu'un faisait quelque sacrifice, le serviteur du sacrificateur venait lorsqu'on faisait bouillir la chair, ayant à la main une fourchette à trois dents,
\VS{14}avec laquelle il piquait dans la chaudière, dans le chaudron, dans la marmite, dans le pot ; et le sacrificateur prenait pour lui tout ce que la fourchette enlevait. C'est ainsi qu'ils agissaient envers tous ceux d'Israël qui venaient à Silo.
\VS{15}Même avant qu'on fasse brûler la graisse, le serviteur du sacrificateur venait et disait à l'homme qui sacrifiait : Donne-moi de la chair à rôtir pour le sacrificateur ; car il ne prendra pas de toi de chair bouillie, mais de la chair crue.
\VS{16}Et si l'homme lui répondait : On va d'abord faire brûler la graisse, et après cela tu prendras ce que ton âme souhaitera, alors le serviteur lui disait : Quoi qu'il en soit, tu en donneras maintenant, sinon j'en prendrai de force.
\VS{17}Et le péché de ces jeunes hommes fut très grand devant Yahweh, car les hommes méprisaient l'offrande de Yahweh.
\TextTitle{Samuel au service de Yahweh}
\VS{18}Samuel faisait le service en présence de Yahweh, étant jeune garçon, vêtu d'un éphod de lin.
\VS{19}Sa mère lui faisait une petite tunique, qu'elle lui apportait tous les ans, quand elle montait avec son mari pour offrir le sacrifice annuel.
\VS{20}Eli bénit Elkana, et sa femme, et dit : Que Yahweh te donne des enfants de cette femme, pour le prêt qu'elle a fait à Yahweh ! Et ils s'en retournèrent chez eux.
\VS{21}Et Yahweh visita Anne, elle conçut, et enfanta trois fils et deux filles. Et le jeune garçon Samuel grandissait en présence de Yahweh.
\TextTitle{Eli averti des péchés commis par ses fils}
\VS{22}Or Eli était très vieux, il apprit tout ce que faisaient ses fils à tout Israël, et qu'ils couchaient avec les femmes qui s'assemblaient à la porte de la tente d'assignation.
\VS{23}Et il leur dit : Pourquoi commettez-vous de telles choses ? Car j'apprends vos méchantes actions de tout le peuple.
\VS{24}Ne faites pas ainsi, mes fils, car ce que j'entends dire de vous n'est pas bon ; vous faites pécher le peuple de Yahweh.
\VS{25}Si un homme a péché contre un autre homme, le Juge\FTNT{Ici, le mot « juge » signifie aussi « Dieu » (« elohim » en hébreux). Dieu est le juste juge (Ec. 3:17 ; Ac. 10:42).} interviendra ; mais si quelqu'un pèche contre Yahweh, qui interviendra pour lui ? Mais ils n'obéirent pas à la voix de leur père parce que Yahweh voulait les faire mourir.
\VS{26}Cependant le jeune garçon Samuel croissait et il était agréable à Yahweh et aux hommes.
\VS{27}Or un homme de Dieu vint auprès d'Eli, et lui dit : Ainsi parle Yahweh : Ne me suis-je pas clairement manifesté à la maison de ton père, quand ils étaient en Egypte, dans la maison de Pharaon ?
\VS{28}Je l'ai choisi parmi toutes les tribus d'Israël pour être mon sacrificateur, afin d'offrir sur mon autel, et faire brûler les parfums, et porter l'éphod devant moi, et j'ai donné à la maison de ton père tous les holocaustes des enfants d'Israël.
\VS{29}Pourquoi avez-vous foulé aux pieds mes sacrifices et mes offrandes que j'ai ordonné de faire dans ma demeure ? Et pourquoi as-tu honoré tes fils plus que moi, afin de vous engraisser du meilleur de toutes les offrandes d'Israël, mon peuple ?
\VS{30}C'est pourquoi voici ce que dit Yahweh, le Dieu d'Israël : J'avais dit et promis que ta maison et la maison de ton père marcheraient devant moi éternellement. Et Yahweh dit : Il n'en sera pas ainsi ! Car j'honorerai ceux qui m'honorent, mais ceux qui me méprisent seront méprisés.
\VS{31}Voici, les jours viennent où je couperai ton bras, et le bras de la maison de ton père, de telle sorte qu'il n'y ait plus de vieillards dans ta maison.
\VS{32}Et tu verras un adversaire dans ma demeure, au temps où Dieu enverra toutes sortes de biens à Israël ; et il n'y aura plus jamais de vieillards dans ta maison.
\VS{33}Celui de tes descendants que je n'aurai pas retranché d'auprès de mon autel subsistera pour consumer tes yeux et affliger ton âme ; et tous les enfants de ta maison mourront dans la fleur de l'âge.
\VS{34}Et ceci sera pour toi un signe, à savoir ce qui arrivera à tes deux fils, Hophni et Phinées ; ils mourront tous les deux le même jour.
\VS{35}Et je m'établirai un sacrificateur fidèle\FTNT{Hé. 2:17 ; Hé 7:26-28.}, qui agira selon mon cœur, et selon mon âme ; et je lui édifierai une maison stable\FTNT{La maison stable fait premièrement allusion à Israël (Mi. 4) et ensuite à l'Eglise (Mt. 16:18). Cette prophétie sera pleinement réalisée lors du millénium (Za. 14).}, et il marchera à toujours devant mon Messie.
\VS{36}Et il arrivera que quiconque sera resté de ta maison viendra se prosterner devant lui pour avoir une pièce d'argent et un morceau de pain et dira : Place-moi, je te prie, dans une des charges du sacerdoce pour manger un morceau de pain.
\Chap{3}
\TextTitle{Yahweh appelle Samuel}
\VerseOne{}Le jeune garçon Samuel servait Yahweh en présence d'Eli. La parole de Yahweh était rare en ce temps-là, et les visions n'étaient pas fréquentes.
\VS{2}Il arriva en ce temps qu'Eli était couché à sa place, ses yeux commençaient à se ternir et il ne pouvait plus voir.
\VS{3}Et avant que les lampes\FTNT{Le chandelier d'or à sept branches du tabernacle et du temple de Jérusalem a été décrit avec une extrême minutie dans plusieurs passages de la Bible. Il a été réalisé selon le modèle imposé par Dieu à Moïse au Sinaï (Ex. 25:31-40 ; Ex. 37:17-24 ; No. 8:4).} de Dieu soient éteintes, Samuel était aussi couché dans le temple de Yahweh, où était l'arche de Dieu.
\VS{4}Yahweh appela Samuel. Et il répondit : Me voici !
\VS{5}Et il courut vers Eli, et lui dit : Me voici, car tu m'as appelé. Mais Eli dit : Je ne t'ai pas appelé ; retourne te coucher. Et il s'en alla, et se coucha.
\VS{6}Yahweh appela encore Samuel. Et Samuel se leva, et s'en alla vers Eli, et lui dit : Me voici, car tu m'as appelé ! Et Eli dit : Mon fils, je ne t'ai pas appelé, retourne, et couche-toi.
\VS{7}Or Samuel ne connaissait pas encore Yahweh, et la parole de Yahweh ne lui avait pas encore été révélée.
\VS{8}Et Yahweh appela encore Samuel pour la troisième fois. Et Samuel se leva, et s'en alla vers Eli, et dit : Me voici, car tu m'as appelé. Eli reconnut que Yahweh appelait ce jeune garçon,
\VS{9}alors Eli dit à Samuel : Va et couche-toi ; et si on t'appelle, tu diras : Parle, Yahweh, car ton serviteur écoute. Samuel donc s'en alla, et se coucha à sa place.
\VS{10}Yahweh donc vint, et se tint là, et appela comme les autres fois : Samuel, Samuel ! Et Samuel dit : Parle, car ton serviteur écoute.
\TextTitle{Jugement de Yahweh sur la maison d'Eli}
\VS{11}Alors Yahweh dit à Samuel : Voici, je vais faire une chose en Israël, qui étourdira les oreilles de quiconque l'entendra.
\VS{12}En ce jour-là, j'accomplirai sur Eli tout ce que j'ai déclaré contre sa maison ; je commencerai et j'achèverai.
\VS{13}Car je l'ai averti que je vais punir sa maison à perpétuité, à cause de l'iniquité dont il a connaissance, par laquelle ses fils se sont rendus infâmes, sans qu'ils les ait réprimés.
\VS{14}C'est pourquoi j'ai juré contre la maison d'Eli ; si jamais il se fait propitiation pour l'iniquité de la maison d'Eli, par sacrifice ou par offrande.
\VS{15}Et Samuel resta couché jusqu'au matin, puis il ouvrit les portes de la maison de Yahweh. Or Samuel craignait de rapporter cette vision à Eli.
\VS{16}Mais Eli appela Samuel, et lui dit : Samuel, mon fils ! Il répondit : Me voici !
\VS{17}Et Eli dit : Quelle est la parole qui t'a été adressée ? Je te prie ne me la cache pas. Ainsi Dieu te fasse, et ainsi il y ajoute, si tu me caches un seul mot de tout ce qui t'a été dit.
\VS{18}Samuel lui déclara donc toutes ces paroles, et ne lui en cacha rien. Et Eli répondit : C'est Yahweh, qu'il fasse ce qui lui semblera bon !
\VS{19}Samuel grandissait. Et Yahweh était avec lui, il ne laissa pas tomber à terre une seule de ses paroles.
\VS{20}Tout Israël, depuis Dan jusqu'à Beer-Schéba, reconnut que Samuel était établi prophète de Yahweh.
\VS{21}Yahweh continuait de se manifester dans Silo ; car Yahweh se manifestait à Samuel, dans Silo, par la parole de Yahweh.
\Chap{4}
\TextTitle{Les Philistins s'emparent de l'arche ; Yahweh juge la maison d'Eli}
\VerseOne{}La parole de Samuel s'adressait à tout Israël. Car Israël sortit en bataille pour aller à la rencontre des Philistins. Ils campèrent près d'Eben-Ezer, et les Philistins campaient à Aphek.
\VS{2}Les Philistins se rangèrent en bataille contre Israël, et le combat s'engagea. Israël fut battu par les Philistins, qui en tuèrent environ quatre mille hommes sur le champ de bataille.
\VS{3}Quand le peuple rentra au camp, les anciens d'Israël dirent : Pourquoi Yahweh nous a-t-il battus aujourd'hui par les Philistins ? Ramenons de Silo l'arche de l'alliance de Yahweh, et qu'elle vienne au milieu de nous, et nous délivre de la main de nos ennemis.
\VS{4}Le peuple envoya donc à Silo, d'où l'on apporta l'arche de l'alliance de Yahweh des armées qui habite entre les chérubins. Les deux fils d'Eli, Hophni et Phinées, étaient là, avec l'arche de l'alliance de Dieu.
\VS{5}Et il arriva que, comme l'arche de Yahweh entrait dans le camp, tout Israël poussa de grands cris de joie et la terre en fut ébranlée.
\VS{6}Les Philistins entendirent le bruit de ces cris de joie, et ils dirent : Que veut dire ce bruit, et que signifient ces grands cris de joie dans le camp de ces Hébreux ? Et ils apprirent que l'arche de Yahweh était arrivée dans le camp.
\VS{7}Les Philistins eurent peur, car ils disaient : Dieu est entré dans le camp. Et ils dirent : Malheur à nous ! Car il n'en a pas été ainsi auparavant.
\VS{8}Malheur à nous ! Qui nous délivrera de la main de ces dieux puissants\FTNT{Le terme hébreu « elohim », généralement traduit par « dieu » ou « dieux », signifie également « dirigeants », « juges » ou encore « anges ». Dans les textes bibliques, « Elohim » est employé pour désigner Moïse, qui a été fait « dieu » (« elohim ») pour Pharaon (Ex. 7:1), ainsi que pour les dieux païens Baal, Kemosh et Dagaon (Jg. 6:31 ; Jg. 11:24 ; 1 S. 5:7) Les Philistins avaient une vision polythéiste de la divinité et n'avaient pas la révélation du Dieu des Hébreux qui est Un (De. 6:4).} ? C'est le Dieu qui a frappé les Egyptiens de toutes sortes de plaies dans le désert.
\VS{9}Philistins, prenez courage et agissez en hommes, de peur que vous ne soyez esclaves des Hébreux, comme ils vous ont été asservis ; agissez en hommes, et combattez !
\VS{10}Les Philistins donc combattirent, et Israël fut battu. Et chacun s'enfuit dans sa tente. La défaite fut très grande, trente mille hommes de pied d'Israël périrent.
\VS{11}L'arche de Dieu fut prise, et les deux fils d'Eli, Hophni et Phinées, moururent.
\VS{12}Un homme de Benjamin s'enfuit de la bataille, et arriva à Silo ce même jour, ayant ses vêtements déchirés et la tête recouverte de terre.
\VS{13}Et comme il arrivait, voici, Eli était assis sur un siège au bord du chemin, étant attentif, car son cœur tremblait à cause de l'arche de Dieu. Cet homme entra donc dans la ville, et donna les nouvelles, et toute la ville se mit à crier.
\VS{14}Eli, entendant les cris, dit : Que veut dire ce grand tumulte ? Et aussitôt cet homme vint à Eli, et lui raconta tout.
\VS{15}Or Eli était âgé de quatre-vingt-dix-huit ans, ses yeux étaient fixes, il ne pouvait plus voir.
\VS{16}L'homme dit à Eli : Je viens de la bataille, car je me suis enfui aujourd'hui de la bataille. Et Eli dit : Qu'est-il arrivé, mon fils ?
\VS{17}Celui qui apportait les nouvelles répondit : Israël a fui devant les Philistins, et il y a eu une grande défaite du peuple ; tes deux fils, Hophni et Phinées, sont morts et l'arche de Dieu a été prise.
\VS{18}Et il arriva qu'aussitôt qu'il eut fait mention de l'arche de Dieu, Eli tomba à la renverse, de dessus son siège, à côté de la porte, se rompit le cou et mourut ; car cet homme était vieux et pesant. Il avait été juge en Israël pendant quarante ans.
\VS{19}Sa belle-fille, femme de Phinées, était enceinte, et sur le point d'accoucher. Lorsqu'elle apprit la nouvelle de la prise de l'arche de Dieu, de la mort de son beau-père et de son mari, elle se coucha et enfanta, car les douleurs la surprirent.
\VS{20}Comme elle mourait, celles qui l'assistaient lui dirent : Ne crains pas, car tu as enfanté un fils ! Mais elle ne répondit rien et n'en tint pas compte.
\VS{21}Mais elle appela l'enfant I-Kabod, en disant : La gloire s'en est allée d'Israël ! Parce que l'arche de Yahweh était prise à cause de son beau-père et de son mari.
\VS{22}Elle dit donc : La gloire s'en est allée d'Israël, car l'arche de Dieu est prise !
\Chap{5}
\TextTitle{Jugements de Yahweh sur les Philistins}
\VerseOne{}Les Philistins prirent l'arche de Dieu, et l'emmenèrent d'Eben-Ezer à Asdod.
\VS{2}Les Philistins donc prirent l'arche de Dieu, et l'emmenèrent dans la maison de Dagon\FTNT{L'étymologie du nom « Dagon » avait justifié la représentation qu'on faisait de ce dieu : une sorte de sirène mâle ou un homme avec une queue de poisson. En effet, « dâg », en hébreu signifie « poisson ». Il était le dieu des semences et de l'agriculture chez les peuples d'origine sémite, mais également l'un des principaux dieux des Philistins.}, et la posèrent auprès de Dagon.
\VS{3}Le lendemain les Asdodiens, s'étant levés de bon matin, trouvèrent Dagon le visage contre terre, devant l'arche de Yahweh. Mais ils le prirent et le remirent à sa place.
\VS{4}Ils se levèrent encore le lendemain de bon matin, et voici, Dagon était tombé le visage contre terre, devant l'arche de Yahweh ; la tête de Dagon et les deux paumes de ses mains découpées étaient sur le seuil, et il ne lui restait que le tronc.
\VS{5}C'est pour cela que les sacrificateurs de Dagon, et tous ceux qui entrent dans la maison de Dagon, à Asdod, ne marchent pas sur le seuil jusqu'à aujourd'hui.
\VS{6}Puis la main de Yahweh s'appesantit sur les Asdodiens et les dévasta ; et il les frappa d'hémorroïdes à Asdod et dans tout son territoire.
\VS{7}Ceux donc d'Asdod, voyant qu'il en allait ainsi, dirent : L'arche du Dieu d'Israël ne demeurera pas chez nous, car sa main s'est appesantie sur nous, et sur Dagon, notre dieu.
\VS{8}Et ils firent appeler et assemblèrent auprès d'eux tous les princes des Philistins, et dirent : Que ferons-nous de l'arche du Dieu d'Israël ? Et ils répondirent : Qu'on transporte à Gath l'arche du Dieu d'Israël. Ainsi on transporta l'arche du Dieu d'Israël.
\VS{9}Mais il arriva après qu'on l'eut transportée, la main de Yahweh fut sur la ville, et il y eut une très grande terreur ; et il frappa les gens de la ville, depuis le plus petit jusqu'au plus grand, par une éruption d'hémorroïdes.
\VS{10}Ils envoyèrent donc l'arche de Dieu à Ekron. Or comme l'arche de Dieu entrait à Ekron, ceux d'Ekron s'écrièrent, en disant : Ils ont transporté vers nous l'arche du Dieu d'Israël, pour nous faire mourir, nous et notre peuple !
\VS{11}C'est pourquoi ils firent appeler, et assemblèrent tous les princes des Philistins, en disant : Renvoyez l'arche du Dieu d'Israël, et qu'elle retourne en son lieu, afin qu'elle ne nous fasse pas mourir, nous et notre peuple. Car il régnait une terreur mortelle dans toute la ville ; et la main de Dieu s'y appesantissait fortement.
\VS{12}Les hommes qui n'en mouraient pas étaient frappés d'hémorroïdes, de sorte que le cri de la ville montait jusqu'au ciel.
\Chap{6}
\TextTitle{L'arche de Yahweh revient en Israël}
\VerseOne{}L'arche de Yahweh ayant été pendant sept mois dans le pays des Philistins,
\VS{2}les Philistins appelèrent les sacrificateurs et les devins, et leur dirent : Que ferons-nous de l'arche de Yahweh ? Dites-nous comment nous devons la renvoyer en son lieu.
\VS{3}Ils répondirent : Si vous renvoyez l'arche du Dieu d'Israël, ne la renvoyez pas à vide, et n'oubliez pas de lui payer le sacrifice pour la culpabilité ; alors vous serez guéris, et vous saurez pourquoi sa main ne s'est pas retirée de dessus vous.
\VS{4}Et ils dirent : Quelle offrande lui payerons-nous pour le délit\FTNT{Le mot « délit » vient de l'hébreu « asham » qui signifie « culpabilité », « offense », « ce qui est acquis par un délit, mal acquis ». Il est question ici de l'arche qui avait été volée par les Philistins. Aux yeux de Dieu, cet acte était un délit.} ? Et ils répondirent : Selon le nombre des princes des Philistins, vous donnerez cinq hémorroïdes d'or, et cinq souris d'or, car une même plaie a été sur vous tous, et sur vos princes.
\VS{5}Vous ferez donc des figures de vos hémorroïdes, et des figures des souris qui ravagent le pays, et vous donnerez gloire au Dieu d'Israël : Peut-être retirera-t-il sa main de dessus vous, et de dessus vos dieux, et de dessus votre pays.
\VS{6}Et pourquoi endurciriez-vous votre cœur, comme l'Egypte et Pharaon ont endurci leur cœur ? Après qu'il eut fait de merveilleux exploits parmi eux, ne les laissèrent-ils pas partir et s'en aller ?
\VS{7}Maintenant, donc prenez de quoi faire un char tout neuf, et deux jeunes vaches qui allaitent leurs veaux et qui n'aient point porté le joug ; et attelez au char les deux jeunes vaches, et ramenez leurs petits à la maison.
\VS{8}Puis prenez l'arche de Yahweh et mettez-la sur le char ; mettez les ouvrages d'or, que vous lui aurez payés pour le délit, dans un petit coffre à côté de l'arche, puis renvoyez-la, et elle s'en ira.
\VS{9}Et vous observerez : Si l'arche monte vers Beth-Schémesch, par le chemin de sa frontière, c'est Yahweh qui nous a fait tout ce grand mal ; si elle n'y va pas, nous saurons alors que sa main ne nous a pas touchés, mais que ceci nous est arrivé par hasard.
\VS{10}Ces gens firent ainsi. Ils prirent donc deux jeunes vaches qui allaitaient, ils les attelèrent au char, et ils enfermèrent leurs petits dans l'étable.
\VS{11}Ils mirent sur le char l'arche de Yahweh, et le coffre avec les souris d'or, et les figures de leurs hémorroïdes.
\VS{12}Alors les jeunes vaches prirent tout droit le chemin de Beth-Schémesch, elles suivirent toujours le même chemin, en marchant et en mugissant, et elles ne se détournèrent ni à droite ni à gauche. Les princes des Philistins allèrent après elles jusqu'à la frontière de Beth-Schémesch.
\VS{13}Or ceux de Beth-Schémesch moissonnaient les blés dans la vallée ; et ayant élevé leurs yeux, ils virent l'arche, et se réjouirent en la voyant.
\VS{14}Le char arriva dans le champ de Josué de Beth-Schémesch, et s'arrêta là. Or il y avait là une grande pierre, et on fendit le bois du char, et on offrit les jeunes vaches en holocauste à Yahweh.
\VS{15}Les Lévites descendirent l'arche de Yahweh, et le coffre dans lequel étaient les objets d'or ; et ils les mirent sur cette grande pierre. En ce même jour, ceux de Beth-Schémesch offrirent des holocaustes et des sacrifices à Yahweh.
\VS{16}Les cinq princes des Philistins, après avoir vu cela, retournèrent le même jour à Ekron.
\VS{17}Voici les hémorroïdes d'or que les Philistins donnèrent à Yahweh en offrande pour le délit : Un pour Asdod, un pour Gaza, un pour Askalon, un pour Gath, un pour Ekron.
\VS{18}Les souris d'or, selon le nombre de toutes les villes des Philistins, appartenant aux cinq princes, tant des villes fortifiées que des villages sans murailles. Et ils les amenèrent jusqu'à la grande pierre sur laquelle on posa l'arche de Yahweh, et qui jusqu'à ce jour est dans le champ de Josué de Beth-Schémesch.
\VS{19}Yahweh frappa des gens de Beth-Schémesch parce qu'ils avaient regardé dans l'arche de Yahweh ; il frappa (cinquante mille) et soixante-dix hommes\FTNT{Ce nombre est généralement considéré comme une erreur des copistes.}, et le peuple mena le deuil parce que Yahweh l'avait frappé d'une grande plaie.
\VS{20}Alors ceux de Beth-Schémesch dirent : Qui pourrait subsister en présence de Yahweh, ce Dieu Saint ? Et vers qui montera-t-il en s'éloignant de nous ?
\VS{21}Et ils envoyèrent des messagers aux habitants de Kirjath-Jearim, en disant : Les Philistins ont ramené l'arche de Yahweh ; descendez, et faites-la monter vers vous.
\Chap{7}
\TextTitle{Un réveil après l'apostasie}
\VerseOne{}Ceux donc de Kirjath-Jearim vinrent, et firent monter l'arche de Yahweh, et la mirent dans la maison d'Abinadab, sur la colline, et ils consacrèrent Eléazar, son fils, pour garder l'arche de Yahweh.
\VS{2}Il s'écoula un long moment depuis le jour où l'arche de Yahweh fut déposée à Kirjath-Jearim. Vingt années s'étaient écoulées. Toute la maison d'Israël soupira après Yahweh.
\VS{3}Et Samuel parla à toute la maison d'Israël, en disant : Si vous revenez à Yahweh de tout votre cœur, ôtez du milieu de vous les dieux étrangers, et les Astartés, dirigez votre cœur vers Yahweh, et servez-le, lui seul ; et il vous délivrera de la main des Philistins.
\VS{4}Alors les enfants d'Israël ôtèrent les Baals et les Astartés, et ils servirent Yahweh seul.
\VS{5}Samuel dit : Assemblez tout Israël à Mitspa, et je prierai Yahweh pour vous.
\VS{6}Ils s'assemblèrent donc à Mitspa. Ils puisèrent de l'eau qu'ils répandirent devant Yahweh et ils jeûnèrent ce jour-là, en disant : Nous avons péché contre Yahweh ! Et Samuel jugea les enfants d'Israël à Mitspa.
\VS{7}Or quand les Philistins eurent appris que les enfants d'Israël étaient assemblés à Mitspa, les princes des Philistins montèrent contre Israël. Les enfants d'Israël l'apprirent et ils eurent peur des Philistins.
\VS{8}Les enfants d'Israël dirent à Samuel : Ne cesse pas de crier pour nous à Yahweh, notre Dieu, afin qu'il nous délivre de la main des Philistins.
\TextTitle{Victoire d'Israël contre les Philistins}
\VS{9}Alors Samuel prit un agneau de lait, et l'offrit tout entier à Yahweh en holocauste. Et Samuel cria à Yahweh pour Israël, et Yahweh l'exauça.
\VS{10}Il arriva donc, comme Samuel offrait l'holocauste, les Philistins s'approchèrent pour combattre contre Israël, mais Yahweh fit gronder, en ce jour-là, un grand tonnerre sur les Philistins, et les mit en déroute, et ils furent battus devant Israël.
\VS{11}Les hommes d'Israël sortirent de Mitspa, et poursuivirent les Philistins, et les frappèrent jusqu'au-dessous de Beth-Car.
\VS{12}Alors Samuel prit une pierre, et la mit entre Mitspa et Schen, et il appela ce lieu Eben-Ezer, en disant : Yahweh nous a secourus jusqu'en ce lieu-ci.
\VS{13}Les Philistins furent humiliés, et ils ne vinrent plus sur le territoire d'Israël. La main de Yahweh fut contre les Philistins durant la vie de Samuel.
\VS{14}Les villes que les Philistins avaient prises sur Israël retournèrent à Israël, depuis Ekron jusqu'à Gath, avec leurs territoires ; Israël les délivra donc de la main des Philistins. Et il y eut paix entre Israël et les Amoréens.
\TextTitle{Samuel, juge en Israël}
\VS{15}Samuel fut juge en Israël tous les jours de sa vie.
\VS{16}Il allait tous les ans faire le tour de Béthel, de Guilgal et de Mitspa, et il jugeait Israël dans tous ces lieux.
\VS{17}Puis il revenait à Rama, où était sa maison ; et là il jugeait Israël, et il y bâtit un autel à Yahweh.
\Chap{8}
\TextTitle{Les fils de Samuel corrompus ; Israël demande un roi}
\VerseOne{}Et il arriva que quand Samuel était devenu vieux, il établit ses fils pour juges sur Israël.
\VS{2}Son fils premier-né s'appelait Joël, et le second Abija ; ils jugeaient à Beer-Schéba.
\VS{3}Mais ses fils ne marchèrent pas dans ses voies, ils s'en détournèrent pour les profits acquis par la violence ; ils recevaient des présents et violaient la justice.
\VS{4}C'est pourquoi tous les anciens d'Israël s'assemblèrent, et vinrent auprès de Samuel à Rama.
\VS{5}Ils lui dirent : Voici, tu es devenu vieux, et tes fils ne suivent pas tes voies ; maintenant, établis sur nous un roi pour nous juger, comme il y en a chez toutes les nations.
\TextTitle{Yahweh accepte la requête du peuple}
\VS{6}Samuel fut affligé de ce qu'ils lui avaient dit : Etablis sur nous un roi pour nous juger. Et Samuel pria Yahweh.
\VS{7}Yahweh dit à Samuel : Obéis à la voix du peuple dans tout ce qu'il te dira ; car ce n'est pas toi qu'ils ont rejeté, mais c'est moi qu'ils ont rejeté, afin que je ne règne plus sur eux.
\VS{8}Ils agissent à ton égard comme ils ont agi depuis le jour où je les ai fait monter hors d'Egypte jusqu'à ce jour ; ils m'ont abandonné, pour servir d'autres dieux.
\VS{9}Maintenant donc, obéis à leur voix ; mais ne manque pas de les avertir, en leur déclarant comment le roi qui régnera sur eux les traitera.
\TextTitle{Avertissement : Le roi sera un joug pour le peuple}
\VS{10}Ainsi Samuel dit toutes les paroles de Yahweh au peuple qui lui avait demandé un roi.
\VS{11}Il leur dit donc : Voici comment vous traitera le roi qui régnera sur vous. Il prendra vos fils et les mettra sur ses chars et parmi ses cavaliers, afin qu'ils courent devant son char ;
\VS{12}il en établira des chefs de mille, et des chefs de cinquante, pour labourer ses terres, pour récolter ses moissons, et pour fabriquer ses armes de guerre et l'équipement de ses chars.
\VS{13}Il prendra aussi vos filles pour en faire des parfumeuses, des cuisinières, et des boulangères.
\VS{14}Il prendra ce qu'il y a de meilleur parmi vos champs, vos vignes et vos oliviers, et il les donnera à ses serviteurs.
\VS{15}Il prélèvera la dîme de ce que vous aurez semé et de ce que vous aurez vendangé, et il la donnera à ses eunuques, et à ses serviteurs.
\VS{16}Il prendra vos serviteurs et vos servantes, l'élite de vos jeunes gens, vos ânes, et les emploiera à ses ouvrages.
\VS{17}Il prélèvera la dîme de vos troupeaux, et vous serez ses esclaves.
\VS{18}En ce jour-là, vous crierez à cause du roi que vous vous serez choisi, mais Yahweh ne vous exaucera pas.
\VS{19}Mais le peuple refusa d'écouter la voix de Samuel, et ils dirent : Non ! Mais il y aura un roi sur nous.
\VS{20}Nous serons aussi comme toutes les nations ; et notre roi nous jugera, il sortira devant nous, et il conduira nos guerres.
\VS{21}Samuel entendit donc toutes les paroles du peuple, et les rapporta à Yahweh.
\VS{22}Et Yahweh dit à Samuel : Obéis à leur voix, et établis un roi sur eux. Et Samuel dit aux hommes d'Israël : Allez-vous-en chacun dans sa ville.
\Chap{9}
\TextTitle{Saül choisi pour devenir le premier roi d'Israël}
\VerseOne{}Il y avait un homme de Benjamin, nommé Kis, fort et vaillant, fils d'Abiel, fils de Tseror, fils de Becorath, fils d'Aphiach, fils d'un Benjamite.
\VS{2}Il avait un fils nommé Saül, jeune et beau, et aucun des enfants d'Israël n'était plus beau que lui, des épaules en haut, il dépassait tout le peuple.
\VS{3}Les ânesses de Kis, père de Saül, s'égarèrent ; et Kis dit à Saül, son fils : Prends maintenant avec toi un des serviteurs et lève-toi, et va chercher les ânesses.
\VS{4}Il passa donc par la montagne d'Ephraïm et traversa le pays de Schalischa, mais ils ne les trouvèrent pas ; puis ils passèrent par le pays de Schaalim, mais elles n'y étaient pas ; ils passèrent ensuite par le pays de Benjamin, mais ils ne les trouvèrent pas.
\VS{5}Quand ils furent arrivés dans le pays de Tsuph, Saül dit à son serviteur qui était avec lui : Viens, et retournons, de peur que mon père oublie les ânesses, et s'inquiète pour nous.
\VS{6}Le serviteur lui dit : Voici, je te prie, il y a dans cette ville un homme de Dieu, qui est un homme très honoré ; tout ce qu'il déclare ne manque pas d'arriver. Allons y maintenant, peut-être nous renseignera-t-il sur le chemin que nous devons prendre.
\VS{7}Et Saül dit à son serviteur : Mais si nous y allons, que porterons-nous à l'homme de Dieu ? Nous n'avons plus de provisions, et nous n'avons aucun présent pour l'homme de Dieu. Qu'est-ce que nous avons ?
\VS{8}Le serviteur reprit la parole et dit à Saül : Voici, j'ai encore entre mes mains le quart d'un sicle d'argent, et je le donnerai à l'homme de Dieu, et il nous indiquera notre chemin.
\VS{9}Autrefois en Israël, quand on allait consulter Dieu, on se disait l'un à l'autre : Venez, allons vers le voyant ! Car le prophète s'appelait autrefois le voyant.
\VS{10}Saül dit à son serviteur : Tu as bien dit : Viens, allons ! Et ils s'en allèrent dans la ville où était l'homme de Dieu.
\VS{11}Et comme ils montaient à la ville, ils trouvèrent des jeunes filles qui sortaient pour puiser de l'eau, et ils leur dirent : Le voyant n'est-il pas ici ?
\VS{12}Elles leur répondirent, et dirent : Il y est, le voilà devant toi ; hâte-toi maintenant, car il est venu aujourd'hui à la ville, parce qu'il y a aujourd'hui un sacrifice pour le peuple sur le haut lieu.
\VS{13}Quand vous entrerez dans la ville, vous le trouverez avant qu'il monte au haut lieu pour manger ; car le peuple ne mangera pas jusqu'à ce qu'il soit venu, parce qu'il doit bénir le sacrifice ; après quoi, les conviés mangeront. Montez donc maintenant, car vous le trouverez aujourd'hui.
\VS{14}Ils montèrent donc à la ville. Comme ils entraient dans la ville, Samuel, qui sortait pour monter au haut lieu, les rencontra.
\VS{15}Or un jour avant l'arrivée de Saül, Yahweh avait fait une révélation à Samuel, en disant :
\VS{16}Demain, à cette même heure, je t'enverrai un homme du pays de Benjamin, et tu l'oindras pour être le conducteur de mon peuple d'Israël. Il délivrera mon peuple de la main des Philistins ; car j'ai regardé mon peuple, parce que son cri est venu jusqu'à moi.
\VS{17}Et dès que Samuel eut aperçu Saül, Yahweh lui dit : Voici l'homme dont je t'ai parlé ; c'est lui qui dominera sur mon peuple.
\VS{18}Et Saül s'approcha de Samuel au milieu de la porte, et dit : Indique-moi, je te prie, où est la maison du voyant.
\VS{19}Et Samuel répondit à Saül, et dit : Je suis le voyant. Monte devant moi au haut lieu, et vous mangerez aujourd'hui avec moi. Je te laisserai partir demain, et je te dirai tout ce que tu as sur le cœur.
\VS{20}Mais quant aux ânesses que tu as perdues il y a trois jours, ne t'en inquiète pas, parce qu'elles ont été retrouvées. Et vers qui tend tout le désir d'Israël ? N'est-ce pas vers toi, et vers toute la maison de ton père ?
\VS{21}Saül répondit : Ne suis-je pas de Benjamin, l'une des moindres tribus d'Israël, et ma famille n'est-elle pas la plus petite de toutes les familles de la tribu de Benjamin ? Pourquoi m'as-tu tenu de tels discours ?
\VS{22}Samuel prit Saül et son serviteur, et les fit entrer dans la salle, et les plaça à la tête des conviés, qui étaient environ trente hommes.
\VS{23}Et Samuel dit au cuisinier : Apporte la portion que je t'ai donnée, en te disant : Mets-la à part.
\VS{24}Le cuisinier prit l'épaule, et ce qui l'entoure, et il la servit à Saül. Et Samuel dit : Voici ce qui a été réservé ; mets-le devant toi, et mange, car il t'a été gardé expressément pour cette heure, lorsque j'ai résolu de convier le peuple. Et Saül mangea avec Samuel ce jour-là.
\VS{25}Puis ils descendirent du haut lieu dans la ville, et Samuel parla avec Saül sur le toit.
\VS{26}Puis ils se levèrent de bon matin ; et, dès l'aurore, Samuel appela Saül sur le toit, et lui dit : Lève-toi, et je te laisserai aller. Saül donc se leva, et ils sortirent tous deux dehors, lui et Samuel.
\VS{27}Et comme ils descendaient à l'extrémité de la ville, Samuel dit à Saül : Dis au serviteur de passer devant nous. Et le serviteur passa devant. Arrête-toi maintenant, afin que je te fasse entendre la parole de Dieu.
\Chap{10}
\TextTitle{Samuel oint Saül comme roi}
\VerseOne{}Or Samuel prit une fiole d'huile, qu'il répandit sur la tête de Saül. Il l'embrassa, et lui dit : Yahweh ne t'a-t-il pas oint pour être le conducteur de son héritage ?
\VS{2}Aujourd'hui, après m'avoir quitté, tu trouveras deux hommes près du sépulcre de Rachel, sur la frontière de Benjamin à Tseltsach, qui te diront : Les ânesses que tu es allé chercher sont retrouvées ; et voici, ton père ne pense plus aux ânesses, mais il s'inquiète pour vous, disant : Que dois-je faire à propos de mon fils ?
\VS{3}En allant plus loin, tu arriveras au chêne de Thabor, où tu seras rencontré par trois hommes qui montent vers Dieu, à Béthel, et l'un porte trois chevreaux, l'autre trois pains, et l'autre une outre de vin.
\VS{4}Ils te demanderont comment tu te portes, et ils te donneront deux pains, que tu recevras de leurs mains.
\VS{5}Après cela tu arriveras à Guibea-Elohim, où se trouve une garnison de Philistins. Et il arrivera qu'en entrant dans la ville, tu rencontreras une troupe de prophètes descendant du haut lieu, précédés du luth, du tambourin, de la flûte, et de la harpe, et qui prophétisent.
\VS{6}Alors l'Esprit de Yahweh te saisira, et tu prophétiseras avec eux, et tu seras changé en un autre homme.
\VS{7}Et quand ces signes te seront arrivés, fais avec force ce que tu trouveras, car Dieu est avec toi.
\VS{8}Puis tu descendras devant moi à Guilgal ; et voici, je descendrai vers toi pour offrir des holocaustes, et des sacrifices d'offrande de paix\FTNT{Voir commentaire en Lé. 3:1.}. Tu m'attendras là sept jours, jusqu'à ce que je vienne, et que je te déclare ce que tu devras faire.
\VS{9}Il arriva donc qu'aussitôt que Saül eut tourné le dos pour se séparer de Samuel, Dieu changea son cœur, et tous ces signes s'accomplirent le même jour.
\VS{10}Quand ils arrivèrent à Guibea, voici, une troupe de prophètes vint à sa rencontre. L'Esprit de Dieu le saisit, et il prophétisa au milieu d'eux.
\VS{11}Et il arriva que, quand toux ceux qui l'avaient connu auparavant virent qu'il était avec les prophètes, et qu'il prophétisait, ceux du peuple se dirent l'un à l'autre : Qu'est-il arrivé au fils de Kis ? Saül est-il aussi parmi les prophètes ?
\VS{12}Un homme répondit : Et qui est leur père ? De là le proverbe : Saül est-il aussi parmi les prophètes ?
\VS{13}Lorsqu'il eut cessé de prophétiser, il se rendit au haut lieu.
\VS{14}L'oncle de Saül dit à Saül et à son serviteur : Où êtes-vous allés ? Et il répondit : Chercher les ânesses, mais ne les trouvant pas, nous sommes allés vers Samuel.
\VS{15}Et l'oncle de Saül dit : Déclare-moi, je te prie, ce que vous a dit Samuel.
\VS{16}Saül répondit à son oncle : Il nous a assuré que les ânesses étaient retrouvées. Mais il ne lui déclara rien concernant la royauté dont Samuel lui avait parlé.
\VS{17}Samuel convoqua le peuple devant Yahweh à Mitspa.
\VS{18}Et il dit aux enfants d'Israël : Ainsi parle Yahweh, le Dieu d'Israël : J'ai fait monter Israël hors d'Egypte, et je vous ai délivrés de la main des Egyptiens, et de la main de tous les royaumes qui vous opprimaient.
\VS{19}Mais aujourd'hui, vous avez rejeté votre Dieu, celui qui vous a délivrés de tous vos malheurs, et de vos afflictions, et vous avez dit : Non, établis-nous un roi ! Présentez-vous donc maintenant, devant Yahweh, par tribus et par familles.
\VS{20}Ainsi Samuel fit approcher toutes les tribus d'Israël, et la tribu de Benjamin fut désignée.
\VS{21}Après, il fit approcher la tribu de Benjamin selon ses familles, et la famille de Matri fut désignée. Puis Saül\FTNT{Voir en annexe le tableau : « Chronologie : Les rois et les prophètes ».}, fils de Kis, fut désigné. On le chercha, mais on ne le trouva pas.
\VS{22}On consulta de nouveau Yahweh : Est-il encore venu quelqu'un ici ? Yahweh répondit : Il est caché parmi les bagages.
\VS{23}Ils coururent donc le chercher, et il se présenta au milieu du peuple, et il était plus grand que tout le peuple, depuis les épaules en haut.
\VS{24}Et Samuel dit à tout le peuple : Voyez-vous celui que Yahweh a choisi ? Il n'y a personne dans tout le peuple qui soit semblable à lui. Et le peuple poussa des cris de joie, et dit : Vive le roi !
\VS{25}Alors Samuel fit connaître au peuple les règles de la royauté, et les écrivit dans un livre, qu'il déposa devant Yahweh. Puis Samuel renvoya le peuple, chacun dans sa maison.
\VS{26}Saül aussi s'en alla chez lui à Guibea. Il fut accompagné par des vaillants hommes dont Dieu avait touché le cœur.
\VS{27}Mais il y eut des fils de Bélial\FTNT{1 S. 2:12.} qui dirent : Comment celui-ci nous délivrerait-il ? Et ils le méprisèrent, et ne lui apportèrent pas de présent. Mais Saül fit le sourd.
\Chap{11}
\TextTitle{Saül bat les Ammonites}
\VerseOne{}Nachasch, l'Ammonite, vint et assiégea Jabès en Galaad. Les habitants de Jabès dirent à Nachasch : Traite alliance avec nous et nous te servirons.
\VS{2}Mais Nachasch, l'Ammonite, leur répondit : Je traiterai avec vous à la condition que je vous crève à tous l'œil droit, et que je mette cet opprobre sur tout Israël.
\VS{3}Les anciens de Jabès lui dirent : Donne-nous sept jours de trêve, et nous enverrons des messagers dans tout le territoire d'Israël ; et s'il n'y a personne qui nous délivre, nous nous rendrons à toi.
\VS{4}Les messagers arrivèrent à Guibea de Saül, et dirent ces paroles devant le peuple. Tout le peuple éleva sa voix, et pleura.
\VS{5}Et voici, Saül revenait des champs derrière ses bœufs, et il dit : Qu'est-ce qu'a ce peuple pour pleurer ainsi ? Et on lui raconta ce qu'avaient dit ceux de Jabès.
\VS{6}Et l'Esprit de Dieu saisit Saül, lorsqu'il entendit ces paroles, et sa colère s'enflamma fortement.
\VS{7}Il prit une paire de bœufs, et les coupa en morceaux, qu'il envoya dans tout le territoire d'Israël, par des messagers, en disant : Les bœufs de tous ceux qui ne sortiront pas pour suivre Saül et Samuel, seront traités de la même manière. Et la frayeur de Yahweh tomba sur le peuple, et ils sortirent comme un seul homme.
\VS{8}Saül en fit la revue à Bézek ; les fils d'Israël étaient trois cents mille et ceux de Juda trente mille.
\VS{9}Puis, ils dirent aux messagers qui étaient venus : Vous parlerez ainsi à ceux de Jabès en Galaad : Vous serez délivrés demain, quand le soleil sera dans sa force. Les messagers rapportèrent donc cela à ceux de Jabès, qui s'en réjouirent ;
\VS{10}et ils dirent aux Ammonites : Demain nous nous rendrons à vous, et vous nous traiterez selon votre bon plaisir.
\VS{11}Le lendemain, Saül disposa le peuple en trois corps. Ils entrèrent dans le camp des Ammonites à la veille du matin, et ils les battirent jusqu'à la chaleur du jour. Ceux qui échappèrent furent dispersés si bien qu'il n'en resta pas deux ensemble.
\TextTitle{Le peuple reconnaît Saül comme roi}
\VS{12}Le peuple dit à Samuel : Qui est-ce qui dit : Saül régnera-t-il sur nous ? Donnez-nous ces hommes-là, et nous les ferons mourir.
\VS{13}Saül répondit : Personne ne sera mis à mort en ce jour, car Yahweh a délivré Israël aujourd'hui.
\VS{14}Et Samuel dit au peuple : Venez, allons à Guilgal, et nous y renouvellerons la royauté.
\VS{15}Et tout le peuple se rendit à Guilgal, et là, ils établirent Saül pour roi devant Yahweh, à Guilgal. Et ils offrirent des sacrifices d'offrande de paix devant Yahweh ; Saül et tous ceux d'Israël se réjouirent beaucoup.
\Chap{12}
\TextTitle{Le peuple rend un bon témoignage de Samuel}
\VerseOne{}Alors Samuel dit à tout Israël : Voici, j'ai obéi à votre voix dans tout ce que vous m'avez dit, et j'ai établi un roi sur vous.
\VS{2}Et maintenant, voici le roi qui marchera devant vous. Car moi, je suis vieux et tout blanc, et voici, mes fils aussi sont avec vous ; pour moi j'ai marché devant vous, depuis ma jeunesse jusqu'à ce jour.
\VS{3}Me voici ! Témoignez contre moi, devant Yahweh, et devant son oint. De qui ai-je pris le bœuf ? Et de qui ai-je pris l'âne ? Qui ai-je opprimé ? Qui ai-je traité durement ? Et de la main de qui ai-je reçu des présents, afin de fermer les yeux sur lui ? Et je vous le rendrai.
\VS{4}Et ils répondirent : Tu ne nous as pas opprimés, tu ne nous as pas traités durement et tu n'as rien reçu de la main de personne.
\VS{5}Il leur dit encore : Yahweh est témoin contre vous, et son oint aussi est témoin aujourd'hui, que vous n'avez rien trouvé entre mes mains. Et ils répondirent : Il en est témoin.
\TextTitle{Rappel des péchés du peuple ; exhortation à craindre Yahweh}
\VS{6}Alors Samuel dit au peuple : Yahweh est celui qui a établi Moïse et Aaron, et qui a fait monter vos pères hors du pays d'Egypte.
\VS{7}Maintenant donc, présentez-vous, et je vous jugerai devant Yahweh sur tous les bienfaits que Yahweh vous a accordés, à vous et à vos pères.
\VS{8}Après que Jacob fut entré en Egypte, vos pères crièrent à Yahweh, et Yahweh envoya Moïse et Aaron qui firent sortir vos pères hors d'Egypte, et les firent habiter en ce lieu.
\VS{9}Mais ils oublièrent Yahweh, leur Dieu, et il les livra entre les mains de Sisera, chef de l'armée de Hatsor, et entre les mains des Philistins, et entre les mains du roi de Moab, qui leur firent la guerre.
\VS{10}Ils crièrent encore à Yahweh, et dirent : Nous avons péché, car nous avons abandonné Yahweh, et nous avons servi les Baals et les Astartés. Maintenant donc, délivre-nous de la main de nos ennemis, et nous te servirons.
\VS{11}Et Yahweh envoya Jerubbaal, Bedan, Jephthé et Samuel, et il vous délivra de la main de tous vos ennemis d'alentour, et vous demeurâtes en sécurité.
\VS{12}Mais voyant que Nachasch, roi des fils d'Ammon, marchait contre vous, vous m'avez dit : Non ! Mais un roi régnera sur nous. Alors que Yahweh, votre Dieu, était votre Roi.
\VS{13}Maintenant donc, voici le roi que vous avez choisi, que vous avez demandé ; et voici Yahweh l'a établi roi sur vous.
\VS{14}Si vous craignez Yahweh, si vous le servez, et obéissez à sa voix, et que vous n'êtes pas rebelles au commandement de Yahweh, alors vous et votre roi qui règne sur vous, vous serez sous la conduite de Yahweh, votre Dieu.
\VS{15}Mais si vous n'obéissez pas à la voix de Yahweh, et si vous êtes rebelles au commandement de Yahweh, la main de Yahweh sera aussi contre vous, comme elle a été contre vos pères.
\VS{16}Maintenant, préparez-vous, et voyez cette grande chose que Yahweh va opérer sous vos yeux.
\VS{17}N'est-ce pas aujourd'hui la moisson des blés ? Je crierai à Yahweh, et il enverra des tonnerres et de la pluie. Sachez alors et voyez combien vous avez mal agi aux yeux de Yahweh en demandant un roi.
\VS{18}Alors Samuel cria à Yahweh, et Yahweh envoya des tonnerres et de la pluie ce même jour. Tout le peuple eut une grande crainte de Yahweh, et de Samuel.
\VS{19}Et tout le peuple dit à Samuel : Prie Yahweh, ton Dieu, pour tes serviteurs, afin que nous ne mourions pas ; car nous avons ajouté à nos péchés, celui d'avoir demandé un roi.
\VS{20}Alors Samuel dit au peuple : Ne craignez pas ! Vous avez fait tout ce mal, néanmoins ne vous détournez pas de Yahweh, mais servez Yahweh de tout votre cœur.
\VS{21}Ne vous en détournez pas, car vous iriez après des choses de néant, qui ne vous apportent ni profit ni délivrance, puisque ce sont des choses de néant.
\VS{22}Car Yahweh n'abandonne pas son peuple, pour l'amour de son grand Nom, car Yahweh a résolu de faire de vous son peuple.
\VS{23}Et pour moi, Dieu me garde de pécher contre Yahweh, et de cesser de prier pour vous ! Je vous enseignerai le bon et le droit chemin.
\VS{24}Craignez seulement Yahweh, et servez-le en vérité, de tout votre cœur ; car vous avez vu les choses magnifiques qu'il a faites pour vous.
\VS{25}Mais si vous persévérez à faire le mal, vous serez détruits vous et votre roi.
\Chap{13}
\TextTitle{Impatience et désobéissance de Saül ; la royauté lui sera enlevée}
\VerseOne{}Saül régna un an sur Israël et après deux années,
\VS{2}Saül choisit trois mille hommes d'Israël, deux mille avec lui à Micmasch, et sur la montagne de Béthel, et mille étaient avec Jonathan à Guibea de Benjamin. Il renvoya le reste du peuple, chacun à sa tente.
\VS{3}Et Jonathan battit le poste des Philistins qui était à Guéba, et les Philistins en furent informés. Et Saül fit sonner le shofar dans tout le pays, en disant : Que les Hébreux écoutent !
\VS{4}Tout Israël apprit donc que Saül avait battu le poste des Philistins, et Israël se rendit odieux aux Philistins. Et le peuple fut convoqué auprès de Saül, à Guilgal.
\VS{5}Les Philistins s'assemblèrent pour combattre Israël, ayant trente mille chars et six mille cavaliers, et le peuple était aussi nombreux que le sable au bord de la mer, tant il était en grand nombre. Ils allèrent prendre position à Micmasch, à l'orient de Beth-Aven.
\VS{6}Les hommes d'Israël furent pris d'une grande angoisse, car ils étaient oppressés, c'est pourquoi le peuple se cacha dans les cavernes, dans les buissons, dans les rochers, dans les tours et dans des citernes.
\VS{7}Les Hébreux passèrent le Jourdain pour aller au pays de Gad et de Galaad. Saül était encore à Guilgal, aussi tout le peuple effrayé le rejoignit.
\VS{8}Il attendit sept jours selon le terme fixé par Samuel. Mais Samuel ne venait pas à Guilgal et le peuple se dispersait.
\VS{9}Et Saül dit : Amenez-moi un holocauste et des sacrifices d'offrande de paix. Et il offrit l'holocauste.
\VS{10}Comme il achevait d'offrir l'holocauste, Samuel arriva, et Saül sortit au-devant de lui pour le saluer.
\VS{11}Et Samuel lui dit : Qu'as-tu fait ? Saül répondit : Lorsque j'ai vu que le peuple se dispersait, que tu ne venais pas au jour fixé, et que les Philistins étaient assemblés à Micmasch,
\VS{12}j'ai dit : Les Philistins descendront maintenant contre moi à Guilgal, et je n'ai pas supplié Yahweh ! Je me suis maîtrisé un temps, mais j'ai fini par offrir l'holocauste.
\VS{13}Samuel répondit à Saül : C'est en insensé que tu as agi, car tu n'as pas gardé le commandement que Yahweh, ton Dieu, t'avait donné ; car Yahweh aurait maintenu à jamais ta royauté sur Israël.
\VS{14}Et maintenant ta royauté ne subsistera pas. Yahweh s'est choisi un homme selon son cœur, et Yahweh l'a destiné à être le chef de son peuple parce que tu n'as pas respecté le commandement de Yahweh.
\TextTitle{Saül et ses hommes à Guibea de Benjamin}
\VS{15}Puis Samuel se leva, et monta de Guilgal à Guibea de Benjamin. Et Saül passa en revue le peuple qui se trouvait avec lui, qui fut d'environ six cents hommes.
\VS{16}Or Saül vint s'établir avec son fils Jonathan, et le peuple qui était sous ses ordres à Guéba de Benjamin, et les Philistins étaient campés à Micmasch.
\VS{17}Les Philistins sortirent du camp en trois divisions pour ravager : L'une de ces divisions prit le chemin d'Ophra, vers le pays de Schual ;
\VS{18}l'autre division prit le chemin de Beth-Horon ; et la troisième prit le chemin de la frontière qui regarde vers la vallée de Tseboïm, du côté du désert.
\VS{19}Or dans tout le pays d'Israël, il ne se trouvait aucun forgeron ; car les Philistins avaient dit : Empêchons les Hébreux de faire des épées ou des lances.
\VS{20}C'est pourquoi chaque homme descendait vers les Philistins pour aiguiser son soc, son hoyau, sa hache, et sa bêche,
\VS{21}lorsque le tranchant des bêches, des hoyaux, des tridents, et des haches était émoussé, même pour redresser un aiguillon.
\VS{22}De sorte qu'il arriva qu'au jour du combat, nul n'avait d'épée ni de lance dans toute l'armée qui était avec Saül et Jonathan ; si ce n'est Saül lui-même et Jonathan, son fils.
\VS{23}Un poste de Philistins s'établit au passage de Micmasch.
\Chap{14}
\TextTitle{Courage de Jonathan}
\VerseOne{}Jonathan, fils de Saül, dit un jour au garçon qui portait ses armes : Viens et allons jusqu'au poste de garde des Philistins qui est au-delà de ce lieu-là. Mais il ne dit rien à son père.
\VS{2}Saül se tenait à l'extrémité de Guibea sous un grenadier, à Migron, entouré d'environ six cents hommes.
\VS{3}Achija, fils d'Achithub, frère d'I-Kabod, fils de Phinées, fils d'Eli, sacrificateur de Yahweh à Silo, portait l'éphod. Et le peuple ignorait que Jonathan s'en était allé.
\VS{4}Or entre les passages par lesquels Jonathan voulait arriver au poste de garde des Philistins, il y avait une dent de rocher d'un côté et une dent de rocher de l'autre, l'une s'appelait Botsets et l'autre Séné.
\VS{5}L'une de ces dents était située du côté nord vis-à-vis de Micmasch, et l'autre, du côté sud vis-à-vis de Guéba.
\VS{6}Jonathan dit au garçon qui portait ses armes : Viens, poursuivons jusqu'au poste de garde de ces incirconcis. Peut-être Yahweh agira-t-il pour nous, car on ne saurait empêcher Yahweh de délivrer avec peu ou beaucoup de gens.
\VS{7}Et celui qui portait ses armes lui dit : Fais tout ce que tu as dans le cœur, vas-y, voici je serai avec toi où tu voudras.
\VS{8}Et Jonathan lui dit : Allons vers ces hommes, et montrons-nous à eux.
\VS{9}S'ils nous disent : Attendez jusqu'à ce que nous venions à vous ! Alors nous resterons sur place, et nous ne monterons pas vers eux.
\VS{10}Mais s'ils disent : Montez vers nous ! Nous irons, car Yahweh les aura livrés entre nos mains. Que cela soit pour nous un signe.
\VS{11}Ils se montrèrent donc tous deux au poste de garde des Philistins, et les Philistins dirent : Voici, les Hébreux sortent des trous où ils s'étaient cachés.
\VS{12}Et ceux du poste de garde dirent à Jonathan, et à celui qui portait ses armes : Montez vers nous, nous avons quelque chose à vous apprendre. Alors Jonathan dit à celui qui portait ses armes : Monte avec moi, car Yahweh les a livrés entre les mains d'Israël.
\VS{13}Et Jonathan monta en s'aidant des mains et des pieds ; celui qui portait ses armes le suivit. Puis ceux du poste de garde tombèrent sous les coups de Jonathan, et celui qui portait ses armes les tuait à sa suite.
\VS{14}Dans cette première victoire, Jonathan et celui qui portait ses armes, tuèrent environ vingt hommes, dans un espace d'environ une moitié d'un arpent de terre.
\VS{15}Et il y eut un grand effroi au camp, à la campagne, et parmi tout le peuple ; le poste de garde aussi, et ceux qui avaient ravagé furent effrayés et le pays fut tellement troublé que cela fut comme une frayeur de Dieu.
\TextTitle{Victoire d'Israël}
\VS{16}Et les sentinelles de Saül, qui étaient à Guibea de Benjamin, regardèrent, et voici, la multitude était en un si grand désordre qu'elle s'écroulait et s'en allait en s'entre-tuant.
\VS{17}Alors Saül dit au peuple qui était avec lui : Faites donc la revue et voyez qui s'en est allé du milieu de nous. Ils firent donc la revue, et voici Jonathan n'y était pas, ni celui qui portait ses armes.
\VS{18}Et Saül dit à Achija : Fais approcher l'arche de Dieu ! Car l'arche de Dieu était, en ce jour-là, avec les enfants d'Israël.
\VS{19}Pendant que Saül parlait au sacrificateur, le tumulte venant du camp des Philistins augmentait de plus en plus ; et Saül dit au sacrificateur : Retire ta main !
\VS{20}Saül et tout le peuple se rassemblèrent, et vinrent au champ de bataille ; les Philistins tournaient les épées les uns contre les autres, la confusion était extrême.
\VS{21}Les Hébreux, qui étaient montés auparavant dans le camp des Philistins et qui étaient dispersés, se joignirent aux Israëlites qui étaient avec Saül et Jonathan.
\VS{22}Et tous les Israëlites qui s'étaient cachés dans la montagne d'Ephraïm, ayant appris que les Philistins s'enfuyaient, les poursuivirent aussi pour les combattre.
\VS{23}Ce jour-là, Yahweh délivra Israël, et le combat s'étendit jusqu'à Beth-Aven.
\TextTitle{Jonathan épargné des conséquences du vœu de Saül}
\VS{24}Les hommes d'Israël furent épuisés cette journée-là. Mais Saül avait fait jurer le peuple, en disant : Maudit soit l'homme qui prendra de la nourriture avant le soir, avant que je me sois vengé de mes ennemis ! Et le peuple ne gouta pas de nourriture.
\VS{25}Tout le peuple arriva dans une forêt, où il y avait du miel à la surface du sol.
\VS{26}Lorsque le peuple entra dans la forêt, il vit le miel qui coulait ; mais nul ne porta la main à sa bouche, car le peuple craignait le serment.
\VS{27}Or Jonathan n'avait pas entendu son père lorsqu'il avait fait juré le peuple ; il étendit le bout du bâton qu'il avait à la main, le trempa dans un rayon de miel et porta sa main à sa bouche ; et ses yeux furent éclaircis.
\VS{28}Alors quelqu'un du peuple lui dit : Ton père a fait jurer expressément le peuple en disant : Maudit soit l'homme qui mangera aujourd'hui quelque chose ! Quoique le peuple soit très fatigué.
\VS{29}Et Jonathan dit : Mon père trouble le peuple ; voyez comment mes yeux sont éclaircis après avoir goûté un peu de ce miel.
\VS{30}Combien plus si le peuple s'était aujourd'hui restauré du butin de ses ennemis, la défaite des Philistins n'en aurait-elle pas été plus considérable ?
\VS{31}En ce jour-là donc ils frappèrent les Philistins de Micmasch à Ajalon. Le peuple était très fatigué.
\VS{32}Puis il se jeta sur le butin, il prit des brebis, des bœufs, et des veaux, et les égorgea sur la terre, et le peuple les mangeait avec le sang.
\VS{33}On le rapporta à Saül, en disant : Voici, le peuple pèche contre Yahweh, en mangeant avec le sang. Et il dit : Vous avez péché ; roulez-moi ici une grosse pierre.
\VS{34}Allez parmi le peuple, ajouta-t-il, et dites à chacun d'amener son bœuf et ses brebis, vous les égorgerez ici. Vous les mangerez, et vous ne pécherez plus contre Yahweh, en mangeant avec le sang. Et chacun amena, cette nuit-là, son bœuf à la main, et ils les égorgèrent.
\VS{35}Saül bâtit un autel à Yahweh ; ce fut le premier autel qu'il bâtit à Yahweh.
\VS{36}Puis Saül dit : Descendons et poursuivons de nuit les Philistins, afin de les piller jusqu'au matin, et n'en laissons pas un de reste. Ils lui répondirent : Fais tout ce qui te semble bon. Mais le sacrificateur dit : Approchons-nous d'abord de Dieu.
\VS{37}Saül consulta donc Dieu : Descendrai-je à la poursuite des Philistins ? Les livreras-tu entre les mains d'Israël ? Mais il ne lui répondit pas ce jour-là.
\VS{38}Et Saül dit : Approchez ici, vous tous les chefs du peuple, recherchez et voyez par qui ce péché est arrivé aujourd'hui.
\VS{39}Car Yahweh est vivant, lui qui délivre Israël, quand il s'agirait de mon fils, Jonathan, il en mourrait. Mais du peuple, personne ne répondit.
\VS{40}Puis il dit à tout Israël : Mettez-vous d'un côté, et nous serons de l'autre, moi et mon fils, Jonathan. Le peuple répondit à Saül : Fais ce qui te semble bon.
\VS{41}Et Saül dit à Yahweh, le Dieu d'Israël : Fais connaître la vérité. Jonathan et Saül furent désignés, et le peuple fut écarté.
\VS{42}Et Saül dit : Jetez le sort entre moi et Jonathan, mon fils. Et Jonathan fut désigné.
\VS{43}Alors Saül dit à Jonathan : Déclare-moi ce que tu as fait. Et Jonathan lui déclara et dit : Il est vrai que j'ai goûté un peu de miel avec le bout de mon bâton que j'avais à la main : Me voici, je mourrai.
\VS{44}Et Saül dit : Que Dieu agisse à mon égard comme il le veut, tu mourras, tu mourras\FTNT{Répétition des mots « tu mourras, tu mourras », voir commentaire en Ge. 2:17.}, Jonathan.
\VS{45}Mais le peuple dit à Saül : Jonathan qui a accompli cette grande délivrance en Israël mourrait-il ? Garde-toi bien ! Yahweh est vivant ! Il ne tombera pas à terre un seul des cheveux de sa tête, car c'est avec Dieu qu'il a agi en ce jour. Le peuple délivra Jonathan de la mort.
\VS{46}Saül renonça à poursuivre les Philistins, qui regagnèrent leur pays.
\TextTitle{Les guerres sous le règne de Saül}
\VS{47}Après que Saül eut pris possession de la royauté sur Israël, il fit la guerre de tous côtés contre ses ennemis, Moab, les fils d'Ammon, Edom, les rois de Tsoba et les Philistins ; partout où il se tournait, il était vainqueur.
\VS{48}Il manifesta sa puissance en frappant Amalek et délivra Israël de la main de ceux qui le pillaient.
\VS{49}Les fils de Saül étaient Jonathan, Jischvi et Malkischua ; et quant aux noms de ses deux filles, le nom de l'aînée était Mérab, et la plus jeune, Mical.
\VS{50}Et le nom de la femme de Saül était Achinoam, fille d'Achimaats ; et le nom du chef de son armée était Abner, fils de Ner, oncle de Saül.
\VS{51}Kis, père de Saül, et Ner, père d'Abner, étaient fils d'Abiel.
\VS{52}La guerre contre les Philistins fut violente durant toute la vie de Saül ; et chaque fois que Saül remarquait un homme fort et vaillant, il le prenait auprès de lui.
\Chap{15}
\TextTitle{Saül désobéit une fois de plus}
\VerseOne{}Samuel dit à Saül : Yahweh m'a envoyé pour t'oindre afin que tu sois roi sur son peuple, sur Israël ; maintenant donc, écoute les paroles de Yahweh.
\VS{2}Ainsi parle Yahweh des armées : Je me rappelle de ce qu'Amalek a fait à Israël, comment il s'opposa à lui sur le chemin, à sa sortie d'Egypte.
\VS{3}Va maintenant, et frappe Amalek, et dévouez par interdit tout ce qui lui appartient ; ne l'épargne pas, mais fais mourir hommes et femmes, enfants et nourrissons, bœufs et menu bétail, chameaux et ânes.
\VS{4}Saül donc convoqua le peuple, et en fit la revue à Thelaïm : Il y avait deux cent mille hommes de pied, et dix mille hommes de Juda.
\VS{5}Et Saül marcha jusqu'à la ville d'Amalek, et mit une embuscade dans la vallée.
\VS{6}Et Saül dit aux Kéniens : Allez, retirez-vous, séparez-vous des Amalécites, de peur que je ne vous détruise avec eux ; car vous avez agi avec bonté envers tous les enfants d'Israël, quand ils montèrent d'Egypte. Et les Kéniens se séparèrent des Amalécites.
\VS{7}Et Saül frappa les Amalécites depuis Havila jusqu'à Schur, qui est face à l'Egypte.
\VS{8}Il fit passer tout le peuple au fil de l'épée, le dévouant par interdit ; mais il épargna Agag, roi d'Amalek.
\VS{9}Saül et le peuple épargnèrent Agag, les meilleures brebis, les meilleurs boeufs, les bêtes grasses, les agneaux, ce qu'il y avait de meilleur ; ils ne voulurent pas les dévouer par interdit, détruisant seulement tout ce qui est chétif et méprisable.
\VS{10}Alors la parole de Yahweh fut adressée à Samuel en disant :
\VS{11}Je me repens d'avoir établi Saül pour roi, car il s'est détourné de moi et n'a pas exécuté mes paroles. Samuel fut très irrité, et il cria à Yahweh toute la nuit.
\TextTitle{Yahweh rejette Saül}
\VS{12}Puis Samuel se leva de bon matin pour aller rencontrer Saül. On lui rapporta que Saül, venu à Carmel, s'est érigé un monument, puis s'en est retourné, pour enfin descendre à Guilgal.
\VS{13}Samuel se rendit auprès de Saül, et Saül lui dit : Sois béni de Yahweh ! J'ai exécuté la parole de Yahweh.
\VS{14}Samuel dit : Quel est donc ce bêlement de brebis qui parvient à mes oreilles, et ce mugissement de bœufs que j'entends ?
\VS{15}Et Saül répondit : Ils les ont amenés de chez les Amalécites, car le peuple a épargné les meilleures brebis et les meilleurs bœufs, pour les sacrifier à Yahweh, ton Dieu ; et nous avons détruit le reste, nous l'avons dévoué par interdit.
\VS{16}Samuel dit à Saül : Laisse-moi et je te déclarerai ce que Yahweh m'a dit cette nuit. Et il lui répondit : Parle !
\VS{17}Samuel dit : N'est-il pas vrai que, quand tu étais petit à tes yeux, tu as été fait chef des tribus d'Israël, et Yahweh t'a oint pour roi sur Israël ?
\VS{18}Yahweh t'avait envoyé dans cette expédition, et t'avait dit : Va, et détruis ces pécheurs, les Amalécites, et fais-leur la guerre jusqu'à ce qu'ils soient exterminés.
\VS{19}Pourquoi n'as-tu pas obéi à la voix de Yahweh, t'es-tu jeté sur le butin, et as-tu fait ce qui déplaît à Yahweh ?
\VS{20}Et Saül répondit à Samuel : J'ai pourtant obéi à la voix de Yahweh, et je suis allé par le chemin par lequel Yahweh m'a envoyé. Et j'ai amené Agag, roi des Amalécites, et j'ai dévoué les Amalécites, par interdit ;
\VS{21}mais le peuple a pris des brebis, des bœufs, du butin, comme prémices de ce qui devait être dévoué, pour le sacrifier à Yahweh, ton Dieu, à Guilgal.
\VS{22}Samuel répondit : Yahweh prend-il plaisir aux holocaustes et aux sacrifices, autant qu'à l'obéissance à sa voix ? Voici, l'obéissance vaut mieux que les sacrifices, et l'observation de sa parole vaut mieux que la graisse des béliers.
\VS{23}Car la rébellion est un péché autant que la divination, et la résistance ne l'est pas moins que l'idolâtrie et les théraphim. Puisque tu as rejeté la parole de Yahweh, il te rejette aussi afin que tu ne sois plus roi.
\VS{24}Et Saül répondit à Samuel : J'ai péché parce que j'ai transgressé le commandement de Yahweh, ainsi que tes paroles ; car je craignais le peuple et j'ai obéi à sa voix.
\VS{25}Mais maintenant, je te prie, pardonne-moi mon péché, et reviens avec moi, que je me prosterne devant Yahweh.
\VS{26}Et Samuel dit à Saül : Je n'irai pas avec toi ; parce que tu as rejeté la parole de Yahweh, Yahweh te rejette afin que tu ne sois plus roi d'Israël.
\VS{27}Comme Samuel se détournait pour s'en aller, Saül le saisit par le pan de son manteau qui se déchira.
\VS{28}Alors Samuel lui dit : Yahweh déchire aujourd'hui le royaume d'Israël de dessus toi, et le donne à un autre, qui est meilleur que toi.
\VS{29}En effet, le Puissant d'Israël ne ment pas, il ne se repent pas, car il n'est pas un homme pour se repentir.
\VS{30}Et Saül répondit : J'ai péché ! Mais honore-moi maintenant, je te prie, en présence des anciens de mon peuple, et en présence d'Israël, et reviens avec moi, et je me prosternerai devant Yahweh, ton Dieu.
\VS{31}Samuel retourna et suivit Saül, et Saül se prosterna devant Yahweh.
\VS{32}Puis Samuel dit : Amenez-moi Agag, roi d'Amalek. Et Agag s'avança vers lui faisant le gracieux ; car Agag disait : Certainement l'amertume de la mort est passée.
\VS{33}Mais Samuel dit : Comme ton épée a privé les femmes de leurs enfants, ainsi ta mère entre les femmes sera privée d'enfants. Et Samuel mit Agag en pièces devant Yahweh à Guilgal.
\VS{34}Puis il s'en alla à Rama, et Saül monta dans sa maison à Guibea de Saül.
\VS{35}Et Samuel n'alla plus voir Saül jusqu'au jour de sa mort ; car Samuel pleurait sur Saül, de ce que Yahweh s'était repenti d'avoir établi Saül, roi sur Israël.
\Chap{16}
\TextTitle{Samuel envoyé à Bethléhem pour oindre David}
\VerseOne{}Yahweh dit à Samuel : Jusqu'à quand mèneras-tu deuil sur Saül, vu que je l'ai rejeté, afin qu'il ne règne plus sur Israël ? Remplis ta corne d'huile, et viens ; je t'enverrai chez Isaï, Bethléhémite, car je me suis pourvu d'un de ses fils pour roi.
\VS{2}Et Samuel dit : Comment irai-je ? Car Saül l'apprendra et il me tuera. Et Yahweh répondit : Tu emmèneras avec toi une jeune vache du troupeau, et tu diras : Je suis venu pour sacrifier à Yahweh.
\VS{3}Et tu inviteras Isaï au sacrifice, et je te ferai savoir ce que tu auras à faire, et tu m'oindras celui que je te dirai.
\VS{4}Samuel fit donc comme Yahweh lui avait dit, et il alla à Bethléhem. Les anciens de la ville tout effrayés accoururent au-devant de lui et lui dirent : Ton arrivée annonce-t-elle la paix ?
\VS{5}Et il répondit : Soyez en paix ; je suis venu pour sacrifier à Yahweh. Sanctifiez-vous, et venez avec moi au sacrifice. Il fit sanctifier aussi Isaï et ses fils, et les invita au sacrifice.
\VS{6}A son entrée, il remarqua Eliab, et se dit : L'oint de Yahweh est certainement devant lui.
\VS{7}Mais Yahweh dit à Samuel : Ne prête pas attention à son apparence ni à la hauteur de sa taille, car je l'ai rejeté. Yahweh ne considère pas ce que l'homme considère ; car l'homme considère ce que voient ses yeux, mais Yahweh regarde au cœur.
\VS{8}Puis Isaï appela Abinadab, et le fit passer devant Samuel ; et il dit : Yahweh n'a pas non plus choisi celui-ci.
\VS{9}Isaï fit passer Schamma ; et Samuel dit : Yahweh n'a pas non plus choisi celui-ci.
\VS{10}Ainsi Isaï fit passer ses sept fils devant Samuel ; et Samuel dit à Isaï : Yahweh n'a pas choisi ceux-ci.
\VS{11}Puis Samuel dit à Isaï : Sont-ce là tous tes garçons ? Et il dit : Il reste encore le plus jeune, seulement, il fait paître les brebis. Alors Samuel dit à Isaï : Envoie-le chercher, car nous ne retournerons pas avant qu'il ne soit venu ici.
\VS{12}Il le fit donc venir. Il était roux, avec de beaux yeux et une belle apparence. Et Yahweh dit à Samuel : Lève-toi, et oins-le, car c'est celui que j'ai choisi !
\VS{13}Alors Samuel prit la corne d'huile, et l'oignit au milieu de ses frères. Et depuis ce jour-là, l'Esprit de Yahweh saisit David. Et Samuel se leva, et s'en alla à Rama.
\TextTitle{David entre au service de Saül}
\VS{14}L'Esprit de Yahweh se retira de Saül, et un mauvais esprit\FTNT{Saül a été frappé d'un esprit d'égarement (2 Th. 2:9-12). Voir commentaires en Ge. 6:3 et Mt. 12:31.} envoyé par Yahweh le terrifiait.
\VS{15}Les serviteurs de Saül lui dirent : Voici, un mauvais esprit envoyé de Dieu te tourmente.
\VS{16}Que le roi, notre seigneur, parle ! Tes serviteurs sont devant toi. Ils chercheront un homme qui sache jouer de la harpe ; et quand le mauvais esprit envoyé par Dieu sera sur toi, il jouera de sa main, et tu seras soulagé.
\VS{17}Saül répondit à ses serviteurs : Trouvez-moi un homme qui sache bien jouer et amenez-le-moi.
\VS{18}L'un des serviteurs répondit : Voici, j'ai vu l'un des fils d'Isaï, le Bethléhémite, qui sait jouer des instruments, il est fort et vaillant, c'est un guerrier qui parle bien, bel homme, et Yahweh est avec lui.
\VS{19}Alors Saül envoya des messagers à Isaï, pour lui dire : Envoie-moi David, ton fils, qui est avec les brebis.
\VS{20}Isaï prit un âne, qu'il chargea de pain, et une outre de vin, et un jeune chevreau, et les envoya par David, son fils, à Saül.
\VS{21}David arrivé chez Saül, se présenta devant lui ; et Saül l'aima beaucoup, et il lui servit à porter ses armes.
\VS{22}Saül fit dire à Isaï : Je te prie que David demeure à mon service, car il a trouvé grâce devant moi.
\VS{23}Il arrivait donc que quand le mauvais esprit envoyé de Dieu était sur Saül, David prenait la harpe, et en jouait de sa main ; et Saül en était soulagé, parce que le mauvais esprit se retirait de lui.
\Chap{17}
\TextTitle{Goliath sème la terreur dans le camp d'Israël}
\VerseOne{}Or les Philistins réunirent leurs armées pour faire la guerre, et ils se rassemblèrent à Soco, qui est de Juda ; et ils campèrent entre Soco et Azéka, à Ephès-Dammim.
\VS{2}Saül et ceux d'Israël se rassemblèrent aussi ; et ils campèrent dans la vallée du chêne, et ils se mirent en ordre de bataille contre les Philistins.
\VS{3}Les Philistins étaient sur une montagne d'un côté, et les Israëlites sur une montagne de l'autre côté ; de sorte que la vallée les séparait.
\VS{4}Il sortit du camp des Philistins un homme qui se présentait entre les deux armées, il s'appelait Goliath, de la ville de Gath, haut de six coudées et d'un empan.
\VS{5}Il avait un casque d'airain sur sa tête et était armé d'une cuirasse à écailles pesant cinq mille sicles d'airain.
\VS{6}Il avait aussi des jambières d'airain, et un javelot d'airain entre ses épaules.
\VS{7}Le bois de sa lance était comme une ensouple d'un tisserand, et le fer de sa lance pesait six cents sicles de fer. Celui qui portait son bouclier marchait devant lui.
\VS{8}Il se présenta donc, et cria aux troupes d'Israël rangées en bataille, et leur disait : Pourquoi sortez-vous pour vous ranger en bataille ? Ne suis-je pas Philistin, et n'êtes-vous pas esclaves de Saül ? Choisissez l'un d'entre vous, et qu'il descende contre moi.
\VS{9}S'il peut me battre et qu'il me tue, nous serons vos esclaves ; mais si j'ai l'avantage sur lui, et que je le tue, vous serez nos esclaves, et vous nous serez asservis.
\VS{10}Le Philistin disait : Je jette un défi en ce jour aux troupes rangées d'Israël : Donnez-moi un homme, et nous combattrons ensemble.
\VS{11}Mais Saül et tous les Israëlites ayant entendu les paroles du Philistin furent épouvantés et saisis d'une grande frayeur.
\VS{12}Or David était le fils d'un homme Ephratien, de Bethléhem de Juda, nommé Isaï, qui avait huit fils, et qui du temps de Saül, était vieux et mis au rang des personnes de qualité.
\VS{13}Et les trois fils aînés d'Isaï avaient suivi Saül à la guerre. Les noms de ses trois fils qui s'en étaient allés à la guerre étaient Eliab, le premier-né, Abinadab, le second, et Schamma, le troisième.
\VS{14}David était le plus jeune, et les trois plus grands suivaient Saül.
\VS{15}David allait et revenait d'auprès de Saül pour paître les brebis de son père à Bethléhem.
\VS{16}Et le Philistin, s'approchant le matin et le soir, se présenta pendant quarante jours.
\TextTitle{David prêt à affronter Goliath}
\VS{17}Isaï dit à David, son fils : Prends maintenant pour tes frères un épha de ce blé rôti, et ces dix pains, et porte-les promptement au camp, à tes frères.
\VS{18}Tu porteras aussi ces dix fromages au chef de leur millier, tu t'informeras du bien-être de tes frères et tu m'en apporteras des nouvelles sûres.
\VS{19}Or Saül, et eux, et tous ceux d'Israël étaient dans la vallée du chêne, combattant contre les Philistins.
\VS{20}Et David se leva de bon matin, et laissa les brebis aux soins d'un gardien ; puis ayant pris sa charge, s'en alla, comme son père Isaï le lui avait ordonné, et il arriva au retranchement où l'armée sortait pour se ranger en bataille, et on poussait des cris de guerre.
\VS{21}Car les Israëlites et les Philistins se rangèrent armée contre armée.
\VS{22}Alors David se déchargea de son bagage, le laissant entre les mains de celui qui gardait le bagage, et courut vers les rangs de l'armée. Aussitôt arrivé, il demanda à ses frères s'ils se portaient bien.
\VS{23}Et comme il parlait avec eux, le Philistin de Gath, nommé Goliath, sortit des rangs de l'armée des Philistins, se présenta entre les deux armées et proféra les mêmes paroles qu'il avait proférées auparavant, et David les entendit.
\VS{24}A la vue de cet homme, tous ceux d'Israël s'enfuirent devant lui, saisis d'une grande frayeur.
\VS{25}Et les Israélites disaient : Avez-vous vu s'avancer cet homme ? Il est monté pour jeter un défi à Israël ! Mais si quelqu'un le tue, le roi le comblera de richesses, et lui donnera sa fille, et affranchira la maison de son père en Israël.
\VS{26}Alors David parla aux personnes qui étaient là avec lui, en disant : Quel bien fera-t-on à l'homme qui frappera ce Philistin, et qui ôtera l'opprobre de dessus Israël ? Car qui est ce Philistin, cet incirconcis, pour insulter l'armée du Dieu vivant ?
\VS{27}Et le peuple lui répéta ces mêmes paroles, et lui dit : C'est le bien qu'on fera à l'homme qui l'aura tué.
\VS{28}Et quand Eliab, son frère aîné, entendit qu'il parlait à ces personnes, sa colère s'enflamma contre David, et il lui dit : Pourquoi es-tu descendu, et à qui as-tu laissé ce peu de brebis au désert ? Je connais ton orgueil et la malice de ton cœur, car tu es descendu pour voir la bataille.
\VS{29}Et David répondit : Qu'ai-je donc fait ? Ne puis-je pas parler ainsi ?
\VS{30}Puis il se détourna de lui vers un autre, et lui posa les mêmes questions. Et le peuple lui répondit comme la première fois.
\VS{31}Les paroles que David avait dites furent entendues et rapportées devant Saül qui le fit venir.
\VS{32}David dit à Saül : Que personne ne perde courage à cause de ce Philistin ! Ton serviteur ira et se battra contre lui.
\VS{33}Mais Saül dit à David : Tu ne peux aller te battre contre ce Philistin, car tu n'es qu'un enfant, et il est un homme de guerre depuis sa jeunesse.
\VS{34}David répondit à Saül : Ton serviteur faisait paître les brebis de son père, quand un lion ou un ours venait emporter une brebis du troupeau,
\VS{35}je le poursuivais, je le frappais, et j'arrachais la brebis de sa gueule. S'il se jetait sur moi, je le saisissais par la mâchoire, je le frappais, et je le tuais.
\VS{36}Ton serviteur a tué et le lion, et l'ours, et ce Philistin, cet incirconcis, sera comme l'un d'eux, car il a déshonoré l'armée du Dieu vivant.
\VS{37}David dit encore : Yahweh qui m'a délivré de la griffe du lion et de la patte de l'ours, me délivrera de la main de ce Philistin. Alors Saül dit à David : Va, et que Yahweh soit avec toi !
\TextTitle{David tue Goliath ; les Philistins sont battus}
\VS{38}Saül fit revêtir David de ses vêtements, et lui mit son casque d'airain sur sa tête, et lui fit endosser une cuirasse.
\VS{39}Puis David ceignit l'épée par-dessus ses vêtements, et voulut marcher, car il n'avait pas encore essayé. Et David dit à Saül : Je ne saurais marcher ainsi, je ne l'ai jamais essayé. Et il s'en débarrassa.
\VS{40}Alors il prit en main son bâton, et se choisit dans le torrent cinq pierres bien polies, et les mit dans le sac de berger et dans sa poche. Puis sa fronde en main, il s'approcha du Philistin.
\VS{41}Le Philistin aussi s'avança et s'approcha lentement de David, précédé de l'homme qui portait son bouclier.
\VS{42}Le Philistin regarda, et lorsqu'il vit David, il le méprisa, car ce n'était qu'un jeune garçon, roux et beau de figure.
\VS{43}Le Philistin dit à David : Suis-je un chien, pour que tu viennes contre moi avec des bâtons ? Et le Philistin maudit David par ses dieux.
\VS{44}Le Philistin ajouta : Viens vers moi et je donnerai ta chair aux oiseaux du ciel et aux bêtes des champs.
\VS{45}Et David dit au Philistin : Tu marches contre moi avec l'épée, la lance, et le javelot ; mais moi, je marche contre toi au Nom de Yahweh des armées, le Dieu de l'armée d'Israël, que tu as blasphémé.
\VS{46}Aujourd'hui Yahweh te livrera entre mes mains, je t'abattrai, je te couperai la tête ; aujourd'hui je donnerai les cadavres du camp des Philistins aux oiseaux du ciel, et aux animaux de la terre. Et toute la terre saura qu'Israël a un Dieu.
\VS{47}Et toute cette assemblée saura que Yahweh ne délivre pas par l'épée ni par la lance. Car la victoire est à Yahweh, qui vous livrera entre nos mains.
\VS{48}Et il arriva que comme le Philistin se fut levé, et qu'il s'approchait pour rencontrer David, David se hâta, et courut au lieu du combat pour rencontrer le Philistin.
\VS{49}Alors David mit la main à son sac, prit une pierre, et la lança avec sa fronde ; il frappa tellement le Philistin au front que la pierre s'enfonça dans son front, il tomba le visage contre terre.
\VS{50}Ainsi avec une fronde et une pierre, David fut plus fort que le Philistin ; il le frappa, et le tua, sans avoir une épée à la main.
\VS{51}Alors David courut, se jeta sur le Philistin, prit son épée, la tira de son fourreau, le tua, et lui coupa la tête. Les Philistins, voyant que leur héros était mort, prirent la fuite.
\VS{52}Alors les hommes d'Israël et de Juda se levèrent, et poussèrent des cris de guerre, et poursuivirent les Philistins jusqu'à la vallée, et jusqu'aux portes d'Ekron. Les Philistins blessés à mort tombèrent dans le chemin de Schaaraïm, jusqu'à Gath, et jusqu'à Ekron.
\VS{53}Et les fils d'Israël revinrent de la poursuite des Philistins, et pillèrent leurs camps.
\VS{54}David prit la tête du Philistin et la porta à Jérusalem, et il mit aussi dans sa tente les armes du Philistin.
\VS{55}Quand Saül vit David sortant à la rencontre du Philistin, il dit à Abner, chef de l'armée : Abner, de qui ce jeune homme est-il le fils ? Abner répondit : Que ton âme vive, ô roi ! Je n'en sais rien.
\VS{56}Le roi lui dit : Informe-toi de qui ce jeune garçon est fils.
\VS{57}Et quand David fut de retour après avoir tué le Philistin, Abner le prit, et le mena devant Saül. David avait la tête du Philistin à la main.
\VS{58}Et Saül lui dit : Jeune garçon, de qui es-tu fils ? David répondit : Je suis fils d'Isaï, Bethléhémite, ton serviteur.
\Chap{18}
\TextTitle{Alliance entre Jonathan et David}
\VerseOne{}Or il arriva qu'aussitôt que David eut achevé de parler à Saül, l'âme de Jonathan fut attachée à l'âme de David, et Jonathan l'aima comme son âme.
\VS{2}Ce jour-là donc Saül le retint, et ne lui permit plus de retourner à la maison de son père.
\VS{3}Alors Jonathan fit alliance avec David, parce qu'il l'aimait comme son âme.
\VS{4}Jonathan se dépouilla du manteau qu'il portait, et le donna à David, avec ses habits, jusqu'à son épée, son arc, et sa ceinture.
\TextTitle{Saül est jaloux de David et cherche à le tuer}
\VS{5}David, envoyé par Saül, réussissait partout où il allait, de sorte que Saül l'établit sur son armée, et il plaisait à tout le peuple, même aux serviteurs de Saül.
\VS{6}Or il arriva que, comme ils revenaient, lors du retour de David après qu'il eut tué le Philistin, des femmes sortirent de toutes les villes d'Israël, en chantant et dansant devant le roi Saül, avec des tambourins, des triangles et en poussant des cris de joie.
\VS{7}Les femmes chantaient, se répondant les unes aux autres, en disant : Saül a frappé ses mille, et David ses dix mille.
\VS{8}Saül fut très irrité, car cette parole lui déplut. Il dit : Elles en ont donné dix mille à David, et à moi mille ! Il ne lui manque plus que le royaume.
\VS{9}Depuis ce jour-là, Saül regardait David d'un mauvais œil.
\VS{10}Et il arriva, dès le lendemain que l'esprit malin envoyé de Dieu saisit Saül, et il faisait le prophète au milieu de la maison, et David joua de sa main, comme les autres jours, et Saül avait une lance dans sa main. 
\VS{11}Saül jeta sa lance, se disant : Je frapperai David, contre le mur. Mais David l'évita deux fois.
\VS{12}Saül avait peur de la présence de David, parce que Yahweh était avec David, et qu'il s'était retiré de Saül.
\VS{13}C'est pourquoi Saül éloigna David de lui, et l'établit chef de mille. Et David allait et venait devant le peuple.
\VS{14}David réussissait dans tout ce qu'il entreprenait, car Yahweh était avec lui.
\VS{15}Saül, voyant que David réussissait beaucoup, avait peur de sa présence.
\VS{16}Mais tout Israël et Juda aimaient David, parce qu'il allait et venait devant eux.
\TextTitle{David épouse Mical, fille de Saül}
\VS{17}Saül dit à David : Voici, je te donnerai Mérab, ma fille aînée, pour femme ; sois pour moi un fils vaillant, et conduis les guerres de Yahweh. Car Saül disait : Que ma main ne le touche pas, mais que ce soit celle des Philistins.
\VS{18}David répondit à Saül : Qui suis-je, et quelle est ma vie, et la famille de mon père en Israël, pour que je devienne gendre du roi ?
\VS{19}Or il arriva qu'au temps où l'on devait donner Mérab, fille de Saül, à David, elle fut donnée pour femme à Adriel de Mehola.
\VS{20}Mais Mical, fille de Saül, aima David ; ce qu'on rapporta à Saül, et la chose lui plut.
\VS{21}Et Saül dit : Je la lui donnerai, afin qu'elle soit pour lui un piège, et que par ce moyen la main des Philistins l'atteigne. Saül donc dit à David pour la seconde fois : Tu seras aujourd'hui mon gendre.
\VS{22}Et Saül ordonna à ses serviteurs de parler à David en secret, et de lui dire : Voici, le roi prend plaisir en toi, et tous ses serviteurs t'aiment ; sois donc maintenant gendre du roi.
\VS{23}Les serviteurs de Saül répétèrent toutes ces paroles à David, et David répondit : Pensez-vous qu'il soit facile de devenir le gendre du roi, moi qui suis un homme pauvre, et peu important ?
\VS{24}Et les serviteurs de Saül le lui rapportèrent, et lui dirent : David a tenu tel discours.
\VS{25}Saül dit : Vous parlerez ainsi à David : Le roi ne désire pas de dot, mais cent prépuces de Philistins, afin d'être vengé de ses ennemis. Or Saül avait pour but de faire tomber David aux mains des Philistins.
\VS{26}Les serviteurs de Saül rapportèrent tous ces discours à David, à qui il plut de devenir gendre du roi. Le temps n'était pas encore écoulé,
\VS{27}que David se leva, et s'en alla, lui et ses gens, et tua deux cents hommes parmi les Philistins ; il apporta leurs prépuces, et on les livra au complet au roi, afin qu'il devienne gendre du roi. Alors Saül lui donna pour femme Mical, sa fille.
\VS{28}Saül vit et comprit que Yahweh était avec David ; et Mical, fille de Saül, l'aimait.
\VS{29}Saül craignait David de plus en plus, et devint son ennemi toute sa vie durant.
\VS{30}Les chefs des Philistins firent des incursions, mais chaque fois qu'ils sortaient, David remportait du succès mieux que tous les serviteurs de Saül et son nom devint fort estimé.
\Chap{19}
\TextTitle{David échappe aux assauts de Saül}
\VerseOne{}Saül parla à Jonathan, son fils, et à tous ses serviteurs de faire mourir David.
\VS{2}Mais Jonathan, fils de Saül, avait une grande affection pour David. C'est pourquoi Jonathan le fit savoir à David, et lui dit : Saül, mon père, cherche à te faire mourir. Maintenant donc, tiens-toi sur tes gardes jusqu'au matin, demeure dans un lieu secret, et cache-toi.
\VS{3}Je me tiendrai auprès de mon père, je sortirai dans le champ où tu seras ; car je parlerai de toi à mon père, je verrai ce qu'il en sera, et je te le rapporterai.
\VS{4}Jonathan parla favorablement de David à Saül, son père, et lui dit : Que le roi ne pèche pas contre son serviteur David, car il n'a pas péché contre toi. Au contraire, il a agi pour ton bien ;
\VS{5}car il a exposé sa vie, il a tué le Philistin, et Yahweh a opéré une grande délivrance pour tout Israël. Tu l'as vu, et tu t'en es réjoui. Pourquoi donc pécherais-tu contre le sang innocent en faisant mourir David sans cause ?
\VS{6}Saül écouta la voix de Jonathan et jura : Yahweh est vivant ! Il ne mourra pas.
\VS{7}Alors Jonathan appela David, et lui répéta toutes ces choses. Jonathan l'introduisit auprès de Saül, et il fut à son service comme auparavant.
\VS{8}La guerre ayant recommencé, David se mit en campagne et frappa les Philistins, et leur infligea une grande défaite, de sorte qu'ils prirent la fuite.
\VS{9}Le mauvais esprit envoyé de Yahweh fut sur Saül, comme il était assis dans sa maison, ayant sa lance à la main. David jouait de sa main,
\VS{10}Saül voulut frapper David avec sa lance contre le mur ; mais il se glissa de devant Saül, qui frappa le mur de la lance. David s'enfuit et s'échappa cette nuit-là.
\VS{11}Saül envoya des messagers à la maison de David pour le garder, et le faire mourir au matin. Mical, femme de David, l'en informa, en disant : Si tu ne te sauves pas, demain on te fera mourir.
\VS{12}Mical fit descendre David par une fenêtre, et ainsi il s'en alla et s'enfuit.
\VS{13}Ensuite Mical prit un théraphim, qu'elle plaça dans le lit ; elle mit une peau de chèvre à son chevet et l'enveloppa d'une couverture.
\VS{14}Lorsque Saül envoya des gens pour prendre David, elle dit : Il est malade.
\VS{15}Saül envoya encore des gens pour prendre David, en leur disant : Apportez-le-moi dans son lit, afin que je le fasse mourir.
\VS{16}Ces gens donc vinrent, et voici, un théraphim était au lit, et la peau de chèvre à son chevet.
\VS{17}Saül dit à Mical : Pourquoi m'as-tu trompé de la sorte, et as-tu laissé aller mon ennemi, de sorte qu'il s'est échappé ? Et Mical répondit à Saül : Il m'a dit : Laisse-moi aller ; pourquoi te tuerais-je ?
\VS{18}C'est ainsi que David prit la fuite et qu'il s'échappa. Il se rendit auprès de Samuel à Rama, et lui raconta tout ce que Saül lui avait fait. Puis il s'en alla avec Samuel, et ils demeurèrent à Najoth.
\VS{19}On le rapporta à Saül, en lui disant : Voici, David est à Najoth, en Rama.
\VS{20}Alors Saül envoya des gens pour s'emparer de David. Ils virent une assemblée de prophètes qui prophétisaient, et Samuel, à leur tête, se tenait là. L'Esprit de Dieu saisit les envoyés de Saül, qui prophétisèrent aussi.
\VS{21}On le rapporta à Saül, qui envoya d'autres gens, et eux aussi prophétisèrent. Saül en envoya encore pour la troisième fois et ils prophétisèrent également.
\VS{22}Alors il alla lui-même à Rama. Arrivé à la grande citerne qui est à Sécou, il s'informa disant : Où sont Samuel et David ? Et on lui répondit : Ils sont à Najoth, en Rama.
\VS{23}Il se dirigea vers Najoth, en Rama. Et l'Esprit de Dieu le saisit à son tour, et il continua son chemin en prophétisant, jusqu'à son arrivée à Najoth, en Rama.
\VS{24}Il se dépouilla lui aussi de ses vêtements et prophétisa devant Samuel ; et il se jeta à terre nu, tout ce jour-là et toute la nuit. C'est pourquoi on dit : Saül est-il aussi parmi les prophètes ?
\Chap{20}
\TextTitle{Renouvellement de l'alliance entre David et Jonathan}
\VerseOne{}David s'enfuit de Najoth, qui est en Rama. Il alla voir Jonathan et lui dit : Qu'ai-je fait ? Quelle est mon iniquité, et quel est mon péché devant ton père, pour qu'il en veuille à ma vie ?
\VS{2}Jonathan lui dit : A Dieu ne plaise ! Tu ne mourras pas. Voici, mon père ne fait aucune chose, ni grande ni petite, qu'il ne m'en informe ; pourquoi mon père me cacherait-il cette chose-là ? Il n'en est rien.
\VS{3}Alors David jurant, dit encore : Ton père sait certainement que j'ai trouvé grâce à tes yeux, et il aura dit : Que Jonathan ne sache rien de ceci, de peur qu'il n'en soit attristé. Mais Yahweh est vivant, et ton âme vit, il n'y a qu'un pas entre moi et la mort.
\VS{4}Alors Jonathan dit à David : Que désires-tu que je fasse ? Et je le ferai pour toi.
\VS{5}Et David dit à Jonathan : Voici, c'est demain la nouvelle lune, et je devrais m'asseoir auprès du roi pour manger, laisse-moi donc aller et je me cacherai aux champs, jusqu'au troisième soir.
\VS{6}Si ton père me cherche, tu lui répondras : David m'a demandé la permission de courir à Bethléhem, sa ville, parce que toute sa famille fait un sacrifice annuel.
\VS{7}S'il dit ainsi : C'est bien ! Ton serviteur n'a rien à craindre. Mais s'il se met en colère, sache qu'il a résolu mon malheur.
\VS{8}Use donc de bonté envers ton serviteur, puisque tu as conclu une alliance avec ton serviteur devant Yahweh. S'il y a de l'iniquité en moi, tue-moi toi-même, car pourquoi me mènerais-tu jusqu'à ton père ?
\VS{9}Jonathan lui dit : A Dieu ne plaise que cela t'arrive ! Si je savais ta perte arrêtée dans la pensée de mon père, ne t'en informerais-je pas ?
\VS{10}David répondit à Jonathan : Qui m'avertira si la réponse que t'aura faite ton père est sévère ?
\VS{11}Et Jonathan dit à David : Viens et sortons dans les champs. Ils sortirent donc eux deux dans les champs.
\VS{12}Alors Jonathan dit à David : Par Yahweh, le Dieu d'Israël ! Je sonderai mon père demain, environ à cette heure, ou après-demain, et s'il est favorable envers David, et que je n'envoie personne vers toi pour t'en informer,
\VS{13}que Yahweh traite Jonathan dans toute sa rigueur ! Si mon père a résolu de te faire du mal, je t'en informerai, et je te laisserai aller, et tu t'en iras en paix, de sorte que Yahweh sera avec toi comme il a été avec mon père.
\VS{14}Si je vis encore, tu useras de la bonté de Yahweh envers moi, en sorte que je ne meure pas.
\VS{15}Ne retire jamais ta bonté de ma maison, pas même quand Yahweh retranchera tous les ennemis de David de dessus la surface de la terre.
\VS{16}Ainsi Jonathan traita alliance avec la maison de David, en disant : Que Yahweh tire vengeance des ennemis de David !
\VS{17}Jonathan se lia encore par serment à David pour l'amour qu'il lui portait, car il l'aimait comme son âme.
\VS{18}Puis Jonathan lui dit : C'est demain la nouvelle lune, et on s'informera sur toi, car ta place sera vide.
\VS{19}Le troisième jour, au soir, tu descendras en hâte, jusqu'au fond du lieu où tu t'étais caché le jour de l'affaire et tu resteras près de la pierre d'Ezel.
\VS{20}Je tirerai trois flèches à côté de cette pierre, comme si je visais un but.
\VS{21}Et voici, j'enverrai un jeune homme, et je lui dirai : Va, trouve les flèches. Si je dis au jeune homme : Voici, les flèches sont au deçà de toi, prends-les ! Alors viens, car la paix est avec toi et tu n'as rien à craindre, Yahweh est vivant !
\VS{22}Mais si je dis ainsi au jeune homme : Voici, les flèches sont au-delà de toi ! Va-t'en, car Yahweh te renvoie.
\VS{23}Et quant à la parole que nous nous sommes donnée, toi et moi ; voici, Yahweh est entre moi et toi, à jamais.
\TextTitle{Saül en colère contre Jonathan}
\VS{24}David donc se cacha dans le champ. La nouvelle lune étant venue, le roi s'assit pour prendre son repas.
\VS{25}Et le roi s'assit à sa place, comme à l'ordinaire, sur son siège près du mur, Jonathan se leva, et Abner s'assit à côté de Saül ; mais la place de David resta vide.
\VS{26}Saül ne dit rien ce jour-là, car il se disait : Il lui est arrivé quelque chose, il n'est pas pur, certainement il n'est pas pur.
\VS{27}Mais le lendemain, le second jour de la nouvelle lune, la place de David était encore vide. Et Saül dit à Jonathan, son fils : Pourquoi le fils d'Isaï n'a-t-il été ni hier ni aujourd'hui au repas ?
\VS{28}Et Jonathan répondit à Saül : David m'a instamment demandé la permission d'aller à Bethléhem.
\VS{29}Même il m'a dit : Je te prie, laisse-moi aller, car notre famille fait un sacrifice dans la ville, et mon frère m'a ordonné de m'y trouver ; maintenant donc si j'ai trouvé grâce à tes yeux, je te prie que j'y aille, afin que je voie mes frères. C'est pour cela qu'il n'est pas venu à la table du roi.
\VS{30}Alors la colère de Saül s'enflamma contre Jonathan et il lui dit : Fils de la perfide et rebelle, ne sais-je pas que tu as choisi le fils d'Isaï à ta honte et à la honte de ta mère ?
\VS{31}Car aussi longtemps que le fils d'Isaï sera vivant sur la terre, tu ne seras pas stable, ni toi, ni ta royauté. C'est pourquoi, maintenant, amène-le-moi, car il est digne de mort.
\VS{32}Et Jonathan répondit à Saül, son père, et lui dit : Pourquoi le ferait-on mourir ? Qu'a-t-il fait ?
\VS{33}Et Saül lança sa lance contre lui pour le frapper. Alors Jonathan reconnut que son père avait résolu la mort de David.
\VS{34}Jonathan se leva de table dans une ardente colère, et ne mangea pas le pain le deuxième jour de la nouvelle lune ; car il était affligé à cause de David, parce que son père l'avait insulté.
\VS{35}Le matin venu, Jonathan sortit dans les champs, au lieu convenu avec David, et il amena avec lui un petit garçon.
\VS{36}Et il dit à son garçon : Cours, trouve maintenant les flèches que je m'en vais tirer. Et le garçon courut, et Jonathan tira une flèche qui le dépassa.
\VS{37}Lorsque le garçon arriva au lieu où était la flèche que Jonathan avait tirée, Jonathan cria après lui, et lui dit : La flèche n'est-elle pas plus loin de toi ?
\VS{38}Jonathan cria encore après le garçon : Hâte-toi, ne t'arrête pas ! Et le garçon ramassa les flèches, et revint vers son maître.
\VS{39}Le garçon ne savait rien de cette affaire ; seuls David et Jonathan le savaient.
\VS{40}Jonathan remit ses armes au garçon et lui dit : Va, porte-les à la ville.
\VS{41}Le garçon parti, David se leva du côté du sud, se jeta le visage contre terre et se prosterna à trois reprises. Ils s'embrassèrent et pleurèrent ensemble, David versa d'abondantes larmes.
\VS{42}Jonathan dit à David : Va en paix, comme nous l'avons juré au Nom de Yahweh, en disant : Que Yahweh soit entre moi et toi, entre ma postérité et ta postérité.
\VS{43}David donc se leva, s'en alla, et Jonathan rentra dans la ville.
\Chap{21}
\TextTitle{David et Achimélec le sacrificateur}
\VerseOne{}David se rendit à Nob, vers Achimélec, le sacrificateur, qui tout effrayé courut au-devant de David, et lui dit : Pourquoi es-tu seul et n'y a-t-il personne avec toi ?
\VS{2}David répondit à Achimélec, le sacrificateur : Le roi m'a donné un ordre et m'a dit : Que personne ne sache rien de l'affaire pour laquelle je t'envoie, ni de l'ordre que je t'ai donné. J'ai donné rendez-vous à mes hommes en un certain lieu.
\VS{3}Maintenant donc qu'as-tu sous la main ? Donne-moi cinq pains ou ce qui s'y trouvera.
\VS{4}Le sacrificateur répondit à David, et dit : Je n'ai pas de pain ordinaire sous la main, mais du pain sacré\FTNT{Mt. 12:4.} ; pourvu que tes gens se soient abstenus de femmes !
\VS{5}David répondit au sacrificateur : Il est vrai que depuis que je suis parti, il y a trois jours, les femmes ont été éloignées de nous, et les vases des serviteurs sont restés purs ; et ce pain est tenu pour commun, vu qu'aujourd'hui on en consacre de nouveau pour le mettre dans les vases.
\VS{6}Alors le sacrificateur lui donna du pain sacré, car il n'y avait pas là d'autre pain que les pains de proposition qui avaient été ôtés de devant Yahweh, pour le remplacer par du pain chaud le jour où on l'avait pris.
\VS{7}Or il y avait là un homme d'entre les serviteurs de Saül, retenu ce jour-là devant Yahweh ; il s'appelait Doëg, un Edomite, le plus puissant de tous les pasteurs de Saül.
\VS{8}David dit à Achimélec : Mais n'as-tu pas ici sous la main quelque lance, ou quelque épée ? Car je n'ai pas pris mon épée ni mes armes sur moi, parce que l'ordre du roi était pressant.
\VS{9}Et le sacrificateur dit : Voici l'épée de Goliath, le Philistin, que tu as tué dans la vallée du chêne ; elle est enveloppée d'un drap, derrière l'éphod ; si tu veux la prendre pour toi, prends-la, car il n'y en a pas ici d'autre que celle-là. Et David dit : Il n'y en a pas de pareille ; donne-la-moi.
\TextTitle{David s'enfuit à Gath}
\VS{10}Alors David se leva, et s'enfuit ce jour-là, loin de Saül, et s'en alla vers Akisch, roi de Gath.
\VS{11}Et les serviteurs d'Akisch lui dirent : N'est-ce pas là David, roi du pays ? N'est-ce pas celui duquel on chantait et répondait en dansant : Saül a tué ses mille, et David ses dix mille ?
\VS{12}David mit ces paroles dans son cœur, et eut une grande peur à cause d'Akisch, roi de Gath.
\VS{13}Il changea son comportement à leurs yeux, il agit devant eux comme un insensé ; et il faisait des marques sur les battants des portes, et laissait couler sa salive sur sa barbe.
\VS{14}Et Akisch dit à ses serviteurs : Ne voyez-vous pas que c'est un homme insensé ? Pourquoi me l'avez-vous amené ?
\VS{15}Est-ce que je manque d'hommes insensés, pour que vous m'ameniez celui-ci pour faire l'insensé devant moi ? Faudrait-il qu'il entre dans ma maison ?
\Chap{22}
\TextTitle{David se réfugie dans la caverne d'Adullam\FTNTT{1 Ch. 12:16-18.}}
\VerseOne{}Et David partit de là, et se sauva dans la caverne d'Adullam. Ses frères et toute la maison de son père l'ayant appris, ils descendirent vers lui.
\VS{2}Tous ceux qui étaient dans la détresse, qui avaient des créanciers, et qui avaient le cœur rempli d'amertume, se rassemblèrent auprès de lui, et il devint leur chef. Et il y eut avec lui environ quatre cents hommes.
\VS{3}David s'en alla de là à Mitspé dans le pays de Moab. Il dit au roi de Moab : Permets, je te prie, à mon père et à ma mère de se retirer chez vous jusqu'à ce que je sache ce que Dieu fera de moi.
\VS{4}Il les amena devant le roi de Moab, et ils demeurèrent chez lui, tout le temps que David fut dans cette forteresse.
\VS{5}Or Gad le prophète, dit à David : Ne demeure pas dans cette forteresse, mais va-t'en, et entre dans le pays de Juda. David donc s'en alla, et vint dans la forêt de Héreth.
\TextTitle{Saül fait tuer les sacrificateurs}
\VS{6}Saül apprit qu'on avait découvert David et ses gens. Or Saül était assis sous le tamaris, à Guibea, sur la hauteur ; il avait sa lance à la main, et tous ses serviteurs se tenaient devant lui.
\VS{7}Et Saül dit à ses serviteurs qui se tenaient près de lui : Ecoutez Benjamites ! Le fils d'Isaï vous donnera-t-il à vous tous des champs et des vignes ? Vous établira-t-il tous chefs de mille, et chefs de cent ?
\VS{8}Pourquoi avez-vous tous conspiré contre moi, et n'y a-t-il personne qui m'informe de l'alliance que mon fils a faite avec le fils d'Isaï ? Pourquoi n'y a-t-il personne de vous qui souffre à mon sujet et qui m'avertisse que mon fils a suscité mon serviteur contre moi pour me dresser des embûches, comme il le fait aujourd'hui ?
\VS{9}Alors Doëg, l'Edomite, qui était établi sur les serviteurs de Saül, répondit et dit : J'ai vu le fils d'Isaï venir à Nob, auprès d'Achimélec, fils d'Achithub.
\VS{10}Il a consulté Yahweh pour lui, il lui a donné des vivres ainsi que l'épée de Goliath, le Philistin.
\VS{11}Alors le roi envoya appeler Achimélec, le sacrificateur, fils d'Achithub, la maison de son père, et les sacrificateurs qui étaient à Nob. Et ils vinrent tous vers le roi.
\VS{12}Saül dit : Ecoute, fils d'Achithub ! Et il répondit : Me voici, mon seigneur !
\VS{13}Alors Saül lui dit : Pourquoi avez-vous conspiré contre moi, toi et le fils d'Isaï ? Pourquoi lui as-tu donné du pain et une épée, et as-tu consulté Dieu pour lui, pour qu'il s'élève contre moi comme il le fait aujourd'hui, pour me dresser des embûches ?
\VS{14}Et Achimélec répondit au roi et dit : Entre tous tes serviteurs y en a-t-il un comme David, fidèle, et gendre du roi, qui est parti sur ton commandement, et qui est si honoré dans ta maison ?
\VS{15}Est-ce aujourd'hui que j'ai commencé à consulter Dieu pour lui ? A Dieu ne plaise ! Que le roi n'impute aucun tort à son serviteur, à personne de la maison de mon père, car ton serviteur ne sait rien de tout cela, petite ou grande.
\VS{16}Le roi lui dit : Tu mourras, tu mourras\FTNT{Le mot est répété deux fois. Voir commentaire en Ge. 2:16.}, Achimélec, toi et toute la maison de ton père.
\VS{17}Alors le roi dit aux coureurs qui se tenaient devant lui : Approchez-vous, et mettez à mort les sacrificateurs de Yahweh ; car leur main est avec David, parce qu'ils savaient qu'il s'enfuyait, et qu'ils ne m'ont pas averti. Mais les serviteurs du roi ne voulurent pas étendre leurs mains pour frapper les sacrificateurs de Yahweh.
\VS{18}Alors Le roi dit à Doëg : Approche-toi, et frappe les sacrificateurs. Et Doëg, l'Edomite, se tourna, et frappa les sacrificateurs ; il tua en ce jour-là quatre-vingt-cinq hommes qui portaient l'éphod de lin.
\VS{19}Il frappa encore du tranchant de l'épée Nob, ville des sacrificateurs ; hommes et femmes, enfants et nourrissons, bœufs, ânes, et brebis, tombèrent sous le tranchant de l'épée.
\VS{20}Toutefois un des fils d'Achimélec, fils d'Achithub, qui s'appelait Abiathar, se sauva, et s'enfuit auprès de David.
\VS{21}Abiathar rapporta à David que Saül avait tué les sacrificateurs de Yahweh.
\VS{22}David dit à Abiathar : Je savais bien ce jour-là que Doëg, l'Edomite, qui était présent, ne manquerait pas d'informer Saül. Je suis la cause de la mort de toutes les personnes de la maison de ton père.
\VS{23}Reste avec moi, ne crains rien, car celui qui cherche ma vie cherche la tienne ; avec moi, tu seras bien gardé.
\Chap{23}
\TextTitle{David libère Keïla}
\VerseOne{}On fit ce rapport à David, en disant : Voici, les Philistins font la guerre à Keïla, et pillent les aires.
\VS{2}Et David consulta Yahweh\FTNT{La clé du succès de David était Yahweh. Il consultait régulièrement Dieu avant de s'engager dans une guerre (Ps. 60:14.).} en disant : Irai-je, et frapperai-je ces Philistins ? Et Yahweh répondit à David : Va, et tu frapperas les Philistins, et tu délivreras Keïla.
\VS{3}Les gens de David lui dirent : Voici, nous avons peur ici en Juda ; que sera-ce donc quand nous irons à Keïla contre les troupes des Philistins ?
\VS{4}C'est pourquoi David consulta encore Yahweh, et Yahweh lui répondit et dit : Lève-toi, descends à Keïla, car je vais livrer les Philistins entre tes mains.
\VS{5}Alors David s'en alla avec ses gens à Keïla, et combattit contre les Philistins ; et emmena leur bétail, et fit un grand carnage. Ainsi David délivra les habitants de Keïla.
\VS{6}Or il était arrivé que quand Abiathar, fils d'Achimélec, s'était enfui vers David à Keïla, il avait en main l'éphod.
\VS{7}On rapporta à Saül que David était venu à Keïla ; Saül dit : Dieu l'a livré entre mes mains car il s'est enfermé en entrant dans une ville qui a des portes et des barres.
\VS{8}Saül convoqua tout le peuple pour aller à la guerre, afin de descendre à Keïla, et d'assiéger David et ses gens.
\VS{9}David ayant eu connaissance des mauvais desseins de Saül à son égard, dit au sacrificateur Abiathar : Apporte l'éphod !
\VS{10}Puis David dit : Ô Yahweh, Dieu d'Israël ! Ton serviteur apprend que Saül cherche à venir à Keïla, pour détruire la ville à cause de moi.
\VS{11}Les chefs de Keïla me livreront-ils entre ses mains ? Saül descendra-t-il comme ton serviteur l'a entendu dire ? Ô Yahweh, Dieu d'Israël, je te prie, révèle-le à ton serviteur. Et Yahweh répondit : Il descendra.
\VS{12}David dit encore : Les chefs de Keïla me livreront-ils, moi et mes gens, entre les mains de Saül ? Et Yahweh répondit : Ils te livreront.
\TextTitle{David échappe encore à Saül}
\VS{13}Alors David se leva avec ses gens au nombre d'environ six cents hommes ; et ils sortirent de Keïla, et s'en allèrent où ils purent. On rapporta à Saül que David s'était sauvé de Keïla, c'est pourquoi il cessa sa marche.
\VS{14}David resta au désert, dans des lieux forts, et il se tint sur la montagne au désert de Ziph. Et Saül le cherchait tous les jours, mais Dieu ne le livra pas entre ses mains.
\VS{15}David, sachant que Saül était sorti pour chercher à sa vie, se tint au désert de Ziph, dans la forêt.
\VS{16}Alors Jonathan, fils de Saül, se leva, et s'en alla dans la forêt vers David, et fortifia ses mains en Dieu.
\VS{17}Et lui dit : Ne crains pas, car Saül, mon père, ne t'atteindra pas. Mais tu régneras sur Israël, et moi je serai le second après toi ; et même Saül, mon père, le sait bien.
\VS{18}Ils firent tous les deux alliance devant Yahweh ; et David resta dans la forêt, mais Jonathan retourna dans sa maison.
\VS{19}Or les Ziphiens montèrent auprès de Saül à Guibea, et lui dirent : David ne se tient-il pas caché parmi nous dans des lieux forts, dans la forêt, sur la colline de Hakila, qui est au sud du désert ?
\VS{20}Maintenant donc, ô roi, puisque tout le désir de ton âme est de descendre, descends, et ce sera à nous de le livrer entre les mains du roi.
\VS{21}Et Saül dit : Que Yahweh vous bénisse de ce que vous avez eu pitié de moi !
\VS{22}Allez donc, je vous prie, assurez-vous encore davantage pour savoir et trouver le lieu où il a dirigé ses pas et qui l'a vu, car, m'a-t-on dit, il est fort rusé.
\VS{23}Examinez donc et reconnaissez tous les lieux où il se tient caché, puis retournez vers moi quand vous en serez assurés, et j'irai avec vous. S'il est dans le pays, je le chercherai soigneusement parmi tous les milliers de Juda.
\VS{24}Ils se levèrent donc et s'en allèrent à Ziph avant Saül. David et ses gens étaient dans le désert de Maon, dans la plaine, au sud du désert.
\VS{25}Saül et ses gens partirent à la recherche de David. Et l'on en informa David, qui descendit le rocher, et resta dans le désert de Maon. Saül, l'ayant appris, poursuivit David au désert de Maon.
\VS{26}Saül marchait d'un côté de la montagne, et David et ses gens de l'autre côté de la montagne. David fuyait précipitamment pour échapper à Saül. Mais Saül et ses gens entouraient David et ses gens pour s'emparer d'eux,
\VS{27}lorsqu'un messager vint à Saül, en disant : Hâte-toi de venir, car les Philistins envahissent le pays.
\VS{28}Alors Saül cessa de poursuivre David, et s'en retourna au-devant des Philistins. C'est pourquoi on appela ce lieu Séla-Hammachlekoth.
\Chap{24}
\TextTitle{David épargne la vie de Saül à En-Guédi}
\VerseOne{}Puis David monta de là et demeura dans les lieux forts d'En-Guédi.
\VS{2}Lorsque Saül fut revenu de la poursuite des Philistins, on lui fit ce rapport, disant : Voilà David au désert d'En-Guédi.
\VS{3}Alors Saül prit trois mille hommes d'élite de tout Israël, et il s'en alla chercher David et ses gens jusque sur le rocher des boucs sauvages.
\VS{4}Saül arriva à des parcs de brebis qui étaient près du chemin, où il y avait une caverne dans laquelle il entra pour se couvrir les pieds. David et ses gens se tenaient au fond de la caverne.
\VS{5}Et les gens de David lui dirent : Voici le jour où Yahweh te dit : Je te livre ton ennemi entre tes mains, afin que tu lui fasses selon ce qu'il te semblera bon. David se leva et coupa tout doucement le pan du manteau de Saül.
\VS{6}Après cela, le cœur de David battit, parce qu'il avait coupé le pan du manteau de Saül.
\VS{7}Et il dit à ses gens : Que Yahweh me garde de commettre une telle action contre mon seigneur, l'oint de Yahweh, en mettant ma main sur lui ! Car il est l'oint de Yahweh\FTNT{David épargne Saül parce qu'il fait confiance à Yahweh. Il laisse Dieu agir plutôt que d'agir lui-même.}.
\VS{8}Ainsi, David détourna ses gens par ses paroles, et il ne leur permit pas de s'élever contre Saül. Puis Saül se leva de la caverne et poursuivit son chemin.
\VS{9}Après cela, David se leva, sortit de la caverne, et cria après Saül, en disant : Mon seigneur le roi ! Saül regarda derrière lui, et David s'inclina le visage contre terre et se prosterna.
\VS{10}David dit à Saül : Pourquoi écouterais-tu les paroles des gens qui te disent : Voici, David cherche ton malheur ?
\VS{11}Aujourd'hui, tes yeux ont vu que Yahweh t'avait livré entre mes mains dans la caverne, et on m'a dit de te tuer ; mais je t'ai épargné, et j'ai dit : Je ne porterai pas la main sur mon seigneur, car il est l'oint de Yahweh.
\VS{12}Regarde donc, mon père, regarde, dis-je, le pan de ton manteau dans ma main. Car j'ai coupé le pan de ton manteau et je ne t'ai pas tué. Sache et reconnais qu'il n'y a ni mal ni injustice dans ma conduite, et que je n'ai pas péché contre toi. Cependant tu me dresses des embûches pour m'ôter la vie !
\VS{13}Yahweh sera juge entre moi et toi, et Yahweh me vengera de toi, mais ma main ne sera pas sur toi.
\VS{14}C'est des méchants que vient la méchanceté, comme dit le proverbe des anciens ; c'est pourquoi ma main ne sera point sur toi.
\VS{15}Contre qui est sorti le roi d'Israël ? Qui poursuis-tu ? Un chien mort, une puce !
\VS{16}Yahweh sera donc juge, et jugera entre moi et toi ; il regardera et plaidera ma cause, il me rendra justice en me délivrant de ta main.
\VS{17}Or il arriva qu'aussitôt que David eut achevé d'adresser ces paroles à Saül, Saül dit : N'est-ce pas là ta voix, mon fils David ? Et Saül éleva la voix, et pleura.
\VS{18}Et il dit à David : Tu es plus juste que moi ; car tu m'as rendu le bien pour le mal que je t'ai fait,
\VS{19}et tu m'as fait connaître aujourd'hui comment tu as usé de bonté envers moi, car Yahweh m'avait livré entre tes mains, et cependant tu ne m'as pas tué.
\VS{20}Si quelqu'un rencontre son ennemi, le laisse-t-il poursuivre tranquillement son chemin ? Que Yahweh donc te récompense pour la grâce que tu m'as faite aujourd'hui !
\VS{21}Et maintenant voici, je sais que tu régneras certainement et que le royaume d'Israël sera ferme entre tes mains.
\VS{22}C'est pourquoi maintenant, jure-moi par Yahweh, que tu ne détruiras pas ma race après moi, et que tu n'extermineras pas mon nom de la maison de mon père.
\VS{23}Et David le jura à Saül. Puis Saül s'en alla dans sa maison, et David et ses gens montèrent au lieu fort.
\Chap{25}
\TextTitle{Israël pleure la mort de Samuel}
\VerseOne{}Samuel mourut. Et tout Israël s'assembla, et le pleura, et on l'enterra dans sa maison à Rama. David se leva, et descendit au désert de Paran.
\TextTitle{Méchanceté de Nabal ; bon sens d'Abigaïl}
\VS{2}Or il y avait à Maon un homme qui avait ses biens à Carmel, et cet homme-là était très puissant car il avait trois mille brebis, et mille chèvres, et il se trouvait à Carmel quand on tondait ses brebis.
\VS{3}Le nom de l'homme était Nabal, et le nom de sa femme, Abigaïl ; elle était une femme de bon sens et belle de visage, mais l'homme était cruel et méchant dans toutes ses actions. Il était de la race de Caleb.
\VS{4}Or David apprit au désert que Nabal tondait ses brebis.
\VS{5}Et il envoya dix jeunes gens, et leur dit : Montez à Carmel, et rendez-vous auprès de Nabal. Vous le saluerez en mon nom,
\VS{6}et vous lui direz : Puisses-tu faire autant l'année prochaine à la même saison, et que la paix soit avec ta maison et tout ce qui est à toi.
\VS{7}Et maintenant, j'ai appris que tu as les tondeurs. Or tes bergers ont été avec nous, et nous ne leur avons fait aucune injure, et ils n'ont subi aucune perte pendant tout le temps qu'ils ont été à Carmel.
\VS{8}Demande-le à tes serviteurs, et ils te le diront. Que ces jeunes gens trouvent donc grâce à tes yeux, puisque nous venons dans un jour favorable. Nous te prions de donner à tes serviteurs, et à David, ton fils, ce que tu trouveras sous ta main.
\VS{9}Les gens de David arrivèrent et dirent à Nabal, au nom de David, toutes ces paroles. Puis ils se turent.
\VS{10}Nabal répondit aux serviteurs de David, et dit : Qui est David, et qui est le fils d'Isaï ? Aujourd'hui le nombre des serviteurs qui s'échappent de leurs maîtres se multiplie.
\VS{11}Et prendrais-je mon pain et mon eau, et la viande que j'ai apprêtée pour mes tondeurs, afin de les donner à des gens qui viennent je ne sais d'où ?
\VS{12}Ainsi les gens de David rebroussèrent chemin. Ils s'en retournèrent, et lui firent leur rapport selon toutes ces paroles.
\VS{13}Et David dit à ses gens : Que chacun de vous ceigne son épée. Et ils ceignirent chacun leur épée. David aussi ceignit son épée et environ quatre cents hommes montèrent avec David. Il en resta deux cents près des bagages.
\VS{14}Or un des serviteurs de Nabal fit ce rapport à Abigaïl, femme de Nabal, et lui dit : Voici, David a envoyé du désert des messagers pour saluer notre maître, qui les a traités rudement.
\VS{15}Cependant ces hommes ont été très bons envers nous, et ne nous ont fait aucune injure, et rien ne nous a été enlevé, tout le temps que nous avons été avec eux lorsque nous étions dans les champs.
\VS{16}Ils nous ont servi de muraille nuit et jour, tout le temps que nous avons été avec eux, faisant paître les troupeaux.
\VS{17}Sache maintenant et vois ce que tu as à faire, car le mal est résolu contre notre maître et contre toute sa maison, et il est si méchant\FTNT{les termes utilisés en hébreu donnent littéralement « Fils de Bélial»} qu'on n'ose lui parler.
\VS{18}Abigaïl se hâta donc, et prit deux cents pains, deux outres de vin, cinq pièces de menu bétail, cinq mesures de grain rôti, cent paquets de raisins secs, deux cents de figues sèches, et les mit sur des ânes.
\VS{19}Puis elle dit à ses gens : Passez devant moi, je vais vous suivre. Elle n'en dit rien à Nabal, son mari.
\VS{20}Et étant montée sur un âne, elle descendait de la montagne par un chemin couvert ; voici, David et ses gens descendaient en face d'elle, et elle les rencontra.
\VS{21}David avait dit : Certainement c'est en vain que j'ai gardé tout ce que cet homme a dans le désert, en sorte qu'il ne s'est rien perdu de tout ce qu'il possède ; il m'a rendu le mal pour le bien.
\VS{22}Que Dieu traite son serviteur David dans toute sa rigueur, si d'ici au matin je laisse subsister de tout ce qui appartient à Nabal.
\VS{23}Lorsque Abigaïl aperçut David, elle se hâta de descendre de son âne, et tomba sur sa face devant David, et se prosterna contre terre.
\VS{24}Elle se jeta donc à ses pieds et lui dit : A moi la faute, mon seigneur ! Permets à ta servante de parler devant toi, et écoute les paroles de ta servante.
\VS{25}Que mon seigneur ne prenne pas garde à ce méchant homme, à Nabal, car il est comme son nom ; Nabal est son nom, et il y a de la folie en lui. Et moi, ta servante, je n'ai pas vu les gens que mon seigneur a envoyés.
\VS{26}Maintenant donc, mon seigneur, aussi vrai que Yahweh est vivant, et que ton âme vit, Yahweh t'a empêché d'en venir au sang, et il a retenu ta main. Or que tes ennemis, et ceux qui cherchent à nuire à mon seigneur soient comme Nabal.
\VS{27}Mais maintenant voici un présent que ta servante a apporté à mon seigneur, afin qu'on le donne aux gens qui sont à la suite de mon seigneur.
\VS{28}Pardonne, je te prie, le crime de ta servante, vu que Yahweh ne manquera pas d'établir une maison ferme à mon seigneur ; car mon seigneur conduit les batailles de Yahweh, et il ne s'est trouvé en toi aucun mal pendant toute ta vie.
\VS{29}Si les hommes se lèvent pour te persécuter, et pour chercher ton âme, l'âme de mon seigneur sera liée au faisceau des vivants auprès de Yahweh, ton Dieu ; mais il lancera au loin, avec la fronde, l'âme de tes ennemis.
\VS{30}Il arrivera que Yahweh fera à mon seigneur selon tout le bien qu'il t'a prédit, et qu'il t'établira conducteur d'Israël,
\VS{31}ceci ne sera pas un obstacle, ni un sujet de regret dans l'âme de mon seigneur, pour avoir répandu le sang inutilement, et pour s'être vengé lui-même. Aussi, lorsque Yahweh aura fait du bien à mon seigneur, tu te souviendras de ta servante.
\VS{32}Alors David dit à Abigaïl : Béni soit Yahweh, le Dieu d'Israël, qui t'a aujourd'hui envoyée à ma rencontre !
\VS{33}Et béni soit ton bon sens, et bénie sois-tu, toi qui m'as aujourd'hui empêché d'en venir au sang, et qui as retenu ma main !
\VS{34}Car certainement Yahweh, le Dieu d'Israël, qui m'a empêché de te faire du mal, est vivant ! Si tu ne t'étais hâtée de venir à ma rencontre, il ne serait resté qui que ce soit à Nabal d'ici à la lumière du matin.
\VS{35}David prit donc de sa main ce qu'elle lui avait apporté, et lui dit : Remonte en paix dans ta maison ; regarde, j'ai écouté ta voix, et j'ai répondu favorablement à ta demande.
\TextTitle{Mort de Nabal ; Abigaïl devient la femme de David}
\VS{36}Puis Abigaïl revint auprès de Nabal. Et voici, il faisait un festin dans sa maison, comme un festin de roi ; et Nabal avait le cœur joyeux, et il était complètement ivre. C'est pourquoi elle ne lui dit aucune chose petite ou grande, jusqu'au matin.
\VS{37}Il arriva donc au matin, après que Nabal fut désenivré, que sa femme lui déclara toutes ces choses. Et son cœur s'amortit en lui, de sorte qu'il devint comme une pierre.
\VS{38}Or, il arriva qu'environ dix jours après, Yahweh frappa Nabal, et il mourut.
\VS{39}Et quand David eut appris que Nabal était mort, il dit : Béni soit Yahweh, qui m'a vengé de l'outrage que j'avais reçu de la main de Nabal, et qui a préservé son serviteur de faire du mal, et a fait retomber le mal de Nabal sur sa tête ! Puis David envoya des gens pour parler à Abigaïl, afin de la prendre pour sa femme.
\VS{40}Les serviteurs de David vinrent auprès d'Abigaïl à Carmel, et lui parlèrent, en disant : David nous a envoyés vers toi, afin de te prendre pour femme.
\VS{41}Alors elle se leva, et se prosterna le visage contre terre, et dit : Voici, ta servante sera à ton service, afin de laver les pieds des serviteurs de mon seigneur.
\VS{42}Puis Abigaïl se leva promptement et monta sur un âne, et cinq de ses servantes la suivirent ; elle suivit les messagers de David, et fut sa femme.
\VS{43}Or David avait pris aussi Achinoam de Jizreel, et toutes les deux furent ses femmes.
\VS{44}Car Saül avait donné Mical, sa fille, femme de David, à Palthi, fils de Laïsch, qui était de Gallim.
\Chap{26}
\TextTitle{David épargne encore la vie de Saül}
\VerseOne{}Les Ziphiens allèrent encore auprès de Saül à Guibea, en disant : David ne se tient-il pas caché sur la colline de Hakila, en face du désert ?
\VS{2}Saül se leva, et descendit au désert de Ziph, avec trois mille hommes de l'élite d'Israël, pour chercher David dans le désert de Ziph.
\VS{3}Saül campa sur la colline de Hakila, en face du désert, près du chemin. Or David se tenait dans le désert, et il aperçut que Saül marchait à sa poursuite au désert,
\VS{4}alors il envoya des espions, et apprit avec certitude que Saül était arrivé.
\VS{5}Alors David se leva, et alla au lieu où Saül campait, et David vit la place où couchait Saül, avec Abner, fils de Ner, chef de son armée. Saül couchait au milieu du camp, et le peuple campait autour de lui.
\VS{6}David prit la parole et dit à Achimélec le Héthien, et à Abischaï, fils de Tseruja et frère de Joab, il dit : Qui veut descendre avec moi dans le camp vers Saül ? Et Abischaï répondit : J'y descendrai avec toi.
\VS{7}David et Abischaï allèrent de nuit vers le peuple. Et voici, Saül dormait étant couché au milieu du camp, et sa lance était plantée en terre à son chevet. Et Abner, et le peuple étaient couchés autour de lui.
\VS{8}Alors Abischaï dit à David : Aujourd'hui, Dieu a livré ton ennemi entre tes mains ; laisse-moi donc le frapper avec la lance, jusqu'en terre d'un seul coup, et je n'y retournerai pas une seconde fois.
\VS{9}Et David dit à Abischaï : Ne le tue pas ! Car qui porterait impunément sa main sur l'oint de Yahweh ?
\VS{10}David dit encore : Yahweh est vivant ! C'est Yahweh seul qui le frappera, soit que son jour vienne, soit qu'il descende au combat et qu'il y périsse.
\VS{11}Que Yahweh me garde de mettre ma main sur l'oint de Yahweh ! Mais prends maintenant la lance qui est à son chevet et la cruche d'eau, et allons-nous-en.
\VS{12}David donc prit la lance et la cruche d'eau qui étaient au chevet de Saül, puis ils s'en allèrent. Personne ne les vit, ni ne s'aperçut de rien, ni ne se réveilla, car ils dormaient tous d'un profond sommeil dans lequel Yahweh les avait plongés.
\VS{13}David passa de l'autre côté, et s'arrêta au loin sur le sommet de la montagne, et il y avait une grande distance entre eux.
\VS{14}Et il cria au peuple, et à Abner, fils de Ner, en disant : Ne répondras-tu pas, Abner ? Abner répondit, et dit : Qui es-tu, toi, qui cries vers le roi ?
\VS{15}Alors David dit à Abner : N'es-tu pas un vaillant homme ? Qui est semblable à toi en Israël ? Pourquoi donc n'as-tu pas gardé le roi, ton seigneur ? Car quelqu'un du peuple est venu pour tuer le roi, ton seigneur.
\VS{16}Ce que tu as fait n'est pas bien. Yahweh est vivant ! Vous méritez la mort, pour avoir si mal gardé votre seigneur, l'oint de Yahweh. Et maintenant, regarde où sont la lance du roi et la cruche d'eau qui étaient à son chevet.
\VS{17}Alors Saül reconnut la voix de David, et dit : N'est-ce pas là ta voix, mon fils David ? Et David dit : C'est ma voix, ô roi, mon seigneur.
\VS{18}Il dit encore : Pourquoi mon seigneur poursuit-il son serviteur ? Car qu'ai-je fait, et quel mal y a t-il dans ma main ?
\VS{19}Maintenant donc je te prie que le roi, mon seigneur, écoute les paroles de son serviteur : Si c'est Yahweh qui te pousse contre moi, que ton offrande lui soit agréable ; mais si ce sont les hommes, qu'ils soient maudits devant Yahweh, car aujourd'hui ils m'ont chassé, afin que je ne puisse me joindre à l'héritage de Yahweh, et ils m'ont dit : Va, sers les dieux étrangers !
\VS{20}Que mon sang ne tombe pas en terre loin de la face de Yahweh ! Car le roi d'Israël est sorti pour chercher une puce, comme on poursuivrait une perdrix dans les montagnes.
\TextTitle{Saül se repent devant David}
\VS{21}Saül dit : J'ai péché ; reviens, mon fils David, car je ne te ferai plus de mal, parce qu'aujourd'hui ma vie t'a été précieuse. Voici, j'ai agi en insensé, et j'ai commis une très grande faute.
\VS{22}David répondit et dit : Voici la lance du roi ; que l'un de tes gens vienne la prendre.
\VS{23}Que Yahweh rende à chacun selon sa justice et selon sa fidélité ; car il t'avait livré aujourd'hui entre mes mains, mais je n'ai pas voulu mettre ma main sur l'oint de Yahweh.
\VS{24}Voici, comme ta vie a été aujourd'hui de grand prix à mes yeux, ainsi ma vie sera de grand prix aux yeux de Yahweh, et il me délivrera de toutes les angoisses.
\VS{25}Saül dit à David : Béni sois-tu, mon fils David ! Tu auras du succès dans tes entreprises. Alors David continua son chemin, et Saül s'en retourna chez lui.
\Chap{27}
\TextTitle{David se réfugie dans le pays des Philistins}
\VerseOne{}Mais David dit en son cœur : Certes je périrai un jour par les mains de Saül ; ne vaut-il pas mieux que je me sauve en hâte au pays des Philistins, afin que Saül renonce à me chercher encore dans tout le territoire d'Israël ? Ainsi j'échapperai à sa main.
\VS{2}David se leva, lui et les six cents hommes qui étaient avec lui, et il passa chez Akisch, fils de Maoc, roi de Gath.
\VS{3}David et ses gens restèrent à Gath auprès d'Akisch ; ils avaient chacun leur famille, David et ses deux femmes, Achinoam de Jizreel, et Abigaïl, qui avait été femme de Nabal, lequel était de Carmel.
\VS{4}Alors on informa Saül que David s'était enfui à Gath ; et il cessa de le chercher.
\VS{5}David dit à Akisch : Si j'ai trouvé grâce à tes yeux, qu'on me donne dans l'une des villes du pays, un lieu où je puisse habiter ; car pourquoi ton serviteur habiterait-il dans la ville royale avec toi ?
\VS{6}Akisch lui donna ce même jour, Tsiklag. C'est pourquoi Tsiklag appartient aux rois de Juda jusqu'à ce jour.
\VS{7}Le temps que David demeura dans le pays des Philistins fut d'un an et quatre mois.
\VS{8}David montait avec ses gens faire des incursions chez les Gueschuriens, les Guirziens, et les Amalécites ; car ces nations-là habitaient dans le territoire dès les temps anciens, depuis Schur jusqu'au pays d'Egypte.
\VS{9}David ravageait ce territoire, il ne laissait en vie ni homme ni femme, et il prenait les brebis, les bœufs, les ânes, les chameaux, et les vêtements, puis il s'en retournait, et allait chez Akisch.
\VS{10}Akisch disait : Où avez-vous fait vos incursions aujourd'hui ? Et David répondait : Vers le sud de Juda, vers le sud des Jerachmeélites et vers le sud des Kéniens.
\VS{11}Mais David ne laissait en vie ni homme ni femme pour les amener à Gath, de peur, disait-il, qu'ils ne rapportent quelque chose contre nous, disant : Ainsi a fait David. Et il agit ainsi tout le temps qu'il demeura dans le pays des Philistins.
\VS{12}Akisch croyait David, et il disait : Il se rend odieux à Israël, son peuple, c'est pourquoi il sera mon serviteur à jamais.
\Chap{28}
\TextTitle{Les Philistins vont en guerre contre Saül}
\VerseOne{}Or il arriva qu'en ces jours-là, les Philistins rassemblèrent leurs armées pour faire la guerre, pour combattre Israël. Akisch dit à David : Sache certainement que vous viendrez avec moi au camp, toi et tes gens.
\VS{2}David répondit à Akisch : Certainement, tu verras ce que ton serviteur fera. Et Akisch dit à David : C'est pour cela que je te confierai toujours la garde de ma personne.
\VS{3}Or Samuel était mort, et tout Israël avait fait le deuil, et on l'avait enseveli à Rama qui était sa ville. Saül avait ôté du pays ceux qui évoquaient les morts, et les devins.
\VS{4}Les Philistins se rassemblèrent et vinrent camper à Sunem ; Saül aussi rassembla tout Israël, et ils campèrent à Guilboa.
\VS{5}A la vue du camp des Philistins, Saül eut peur, et son cœur fut saisi de crainte.
\VS{6}Saül consulta Yahweh ; mais Yahweh ne lui répondit rien, ni par des songes, ni par l'urim, ni par les prophètes.
\TextTitle{Saül chez la femme qui évoque les morts}
\VS{7}Saül dit à ses serviteurs : Cherchez-moi une femme qui évoque les morts, et j'irai vers elle, et je la consulterai. Ses serviteurs lui dirent : Voilà, il y a une femme à En-Dor qui évoque les morts.
\VS{8}Alors Saül se déguisa, prit d'autres vêtements, et il partit avec deux hommes. Ils arrivèrent de nuit chez cette femme et Saül lui dit : Je te prie évoque les morts et fais monter vers moi celui que je te dirai.
\VS{9}Mais la femme lui répondit : Voici, tu sais ce que Saül a fait, et comment il a exterminé du pays ceux qui évoquent les morts et les devins ; pourquoi donc dresses-tu un piège à mon âme pour me faire mourir ?
\VS{10}Saül lui jura par Yahweh, et lui dit : Yahweh est vivant ! Il ne t'arrivera pas de mal pour cela.
\VS{11}Alors la femme dit : Qui veux-tu que je te fasse monter ? Et il répondit : Fais-moi monter Samuel.
\VS{12}Et la femme voyant Samuel s'écria à haute voix, en disant à Saül : Pourquoi m'as-tu trompée ? Car tu es Saül !
\VS{13}Et le roi lui répondit : Ne crains pas ; mais que vois-tu ? La femme dit à Saül : Je vois un dieu qui monte de la terre.
\VS{14}Il lui dit encore : Comment est-il fait ? Elle répondit : C'est un vieillard qui monte, et il est couvert d'un manteau. Et Saül comprit que c'était Samuel, il s'inclina le visage contre terre et se prosterna.
\VS{15}Samuel dit à Saül : Pourquoi m'as-tu troublé en me faisant monter ? Et Saül répondit : Je suis dans une grande angoisse ; car les Philistins me font la guerre, et Dieu s'est retiré de moi, et ne m'a plus répondu ni par les prophètes ni par des songes ; c'est pourquoi je t'ai invoqué\FTNT{Dieu interdit formellement ces pratiques (De. 18:9-14 ; Es. 8:19.).}, afin que tu me fasses entendre ce que j'aurai à faire.
\VS{16}Samuel dit : Pourquoi donc me consultes-tu, puisque Yahweh s'est retiré de toi, et qu'il est devenu ton ennemi ?
\VS{17}Yahweh te traite comme je te l'avais annoncé de sa part ; car Yahweh a déchiré le royaume d'entre tes mains, et l'a donné à un autre, à David.
\VS{18}Parce que tu n'as pas obéi à la voix de Yahweh, et que tu n'as pas exécuté l'ardeur de sa colère contre Amalek, à cause de cela, Yahweh te traite de cette manière aujourd'hui.
\VS{19}Et même Yahweh livrera Israël avec toi entre les mains des Philistins, et vous serez demain avec moi, toi et tes fils ; Yahweh livrera aussi le camp d'Israël entre les mains des Philistins.
\VS{20}Aussitôt Saül tomba à terre tout étendu, car il fut très effrayé des paroles de Samuel, et même les forces lui manquèrent parce qu'il n'avait rien mangé ce jour, ni toute cette nuit.
\VS{21}Alors la femme vint auprès de Saül, et voyant qu'il avait été très effrayé, elle lui dit : Voici, ta servante a obéi à ta voix, j'ai exposé ma vie, et j'ai obéi aux paroles que tu m'as dites.
\VS{22}Maintenant, je te prie, écoute, toi aussi, ce que ta servante te dira : Laisse-moi te servir avant un morceau de pain, afin que tu manges pour avoir la force de te remettre en route.
\VS{23}Et il le refusa et dit : Je ne mangerai pas. Mais ses serviteurs et la femme aussi le pressèrent tellement qu'il écouta leur voix. Il se leva de terre, et s'assit sur un lit.
\VS{24}Or cette femme avait dans sa maison un veau qu'elle engraissait, et elle se hâta de le tuer ; puis elle prit de la farine, et la pétrit, et en cuisit des pains sans levain.
\VS{25}Elle les mit devant Saül et devant ses serviteurs. Et ils mangèrent. Puis s'étant levés, ils s'en allèrent cette nuit-là.
\Chap{29}
\TextTitle{Les Philistins refusent que David combatte Israël}
\VerseOne{}Or les Philistins rassemblèrent toutes leurs armées à Aphek, et Israël campa près de la fontaine de Jizreel.
\VS{2}Les princes des Philistins s'avancèrent avec leurs centaines et leurs milliers, et David et ses gens marchèrent à l'arrière-garde avec Akisch.
\VS{3}Les princes des Philistins dirent : Que font ici ces Hébreux ? Et Akisch répondit aux princes des Philistins : N'est-ce pas ce David, serviteur de Saül, roi d'Israël ? Il y a longtemps qu'il est avec moi, même quelques années, et je n'ai pas trouvé quelque chose à lui reprocher depuis son arrivée, jusqu'à ce jour.
\VS{4}Mais les princes des Philistins se mirent en colère contre lui, et lui dirent : Renvoie cet homme, et qu'il retourne dans le lieu où tu l'as établi, et qu'il ne descende pas avec nous dans la bataille, de peur qu'il ne se tourne contre nous dans la bataille. Car comment pourrait-il se remettre en grâce auprès de son maître ? Ne serait-ce pas par le moyen des têtes de nos hommes ?
\VS{5}N'est-ce pas ce David pour qui l'on chantait et répondait en dansant : Saül a frappé ses mille, et David ses dix mille ?
\VS{6}Akisch appela David, et lui dit : Yahweh est vivant ! Tu es certainement un homme droit, et ta conduite dans le camp m'a paru bonne, car je n'ai pas trouvé de mal en toi depuis le jour où tu es arrivé auprès de moi jusqu'à ce jour ; mais tu ne plais pas aux princes.
\VS{7}Maintenant donc, retourne et va-t'en en paix, afin que tu ne fasses aucune chose qui déplaise aux princes des Philistins.
\VS{8}Et David dit à Akisch : Mais qu'ai-je fait ? Et qu'as-tu trouvé en ton serviteur depuis que je suis avec toi jusqu'à ce jour, pour que je n'aille pas combattre contre les ennemis du roi, mon seigneur ?
\VS{9}Akisch répondit, et dit à David : Je le sais, car tu es agréable à mes yeux, comme un ange de Dieu ; mais c'est seulement les chefs des Philistins qui disent : Il ne montera pas avec nous dans la bataille.
\VS{10}C'est pourquoi lève-toi de bon matin, avec les serviteurs de ton maître qui sont venus avec toi ; levez-vous de bon matin, et partez dès que vous verrez le jour, allez-vous-en.
\VS{11}Ainsi David se leva de bonne heure, lui et ses gens, pour partir dès le matin, et retourner dans le pays des Philistins. Et les Philistins montèrent à Jizreel.
\Chap{30}
\TextTitle{David libère Tsiklag}
\VerseOne{}Or trois jours après, David et ses gens étant revenus à Tsiklag, trouvèrent que les Amalécites avaient fait une incursion dans le midi et à Tsiklag. Et ils avaient frappé et brûlé au feu Tsiklag,
\VS{2}après avoir fait prisonniers les femmes et tous ceux qui étaient là, petits et grands. Ils n'avaient tué personne, mais ils les avaient emmenés, et s'étaient remis en chemin.
\VS{3}David et ses gens revinrent dans la ville et voici, elle était brûlée, et leurs femmes, leurs fils, et leurs filles avaient été faits prisonniers.
\VS{4}C'est pourquoi David et le peuple qui était avec lui élevèrent leur voix, et pleurèrent tellement qu'il n'y avait plus en eux de force pour pleurer.
\VS{5}Les deux femmes de David avaient été emmenées, Achinoam de Jizreel, et Abigaïl de Carmel, qui avait été femme de Nabal.
\VS{6}David fut dans une grande angoisse, parce que le peuple parlait de le lapider, car tout le peuple avait de l'amertume dans l'âme à cause de leurs fils et de leurs filles. Toutefois, David se fortifia en Yahweh, son Dieu.
\VS{7}Et il dit à Abiathar, le sacrificateur, fils d'Achimélec : Apporte-moi, je te prie, l'éphod ! Abiathar apporta l'éphod à David.
\VS{8}Et David consulta Yahweh, en disant : Poursuivrai-je cette troupe ? L'atteindrai-je ? Et il lui répondit : Poursuis, car tu l'atteindras, et tu délivreras.
\VS{9}David s'en alla avec les six cents hommes qui étaient avec lui, et ils arrivèrent au torrent de Besor, où s'arrêtèrent ceux qui restaient en arrière.
\VS{10}Ainsi David et quatre cents hommes continuèrent la poursuite, mais deux cents hommes s'arrêtèrent, trop fatigués pour pouvoir passer le torrent de Besor.
\VS{11}Ayant trouvé un homme égyptien dans les champs, ils l'amenèrent à David, et lui donnèrent du pain, il mangea, puis ils lui donnèrent de l'eau à boire.
\VS{12}Ils lui donnèrent aussi quelques figues sèches, et deux grappes de raisins secs. Et il mangea, et le coeur lui revint, car cela faisait trois jours et trois nuits qu'il n'avait pas mangé de pain ni bu d'eau.
\VS{13}Et David lui dit : A qui es-tu ? Et d'où es-tu ? Et il répondit : Je suis un garçon égyptien, serviteur d'un homme amalécite, et mon maître m'a abandonné, parce que j'étais malade il y a trois jours.
\VS{14}Nous avons envahi le sud des Kéréthiens, et sur ce qui est à Juda, et sur le sud de Caleb, et nous avons brûlé Tsiklag par le feu.
\VS{15}David lui dit : Me conduiras-tu vers cette troupe ? Et il répondit : Jure-moi par le Nom de Dieu que tu ne me feras point mourir, et que tu ne me livreras point entre les mains de mon maître, et je te conduirai vers cette troupe.
\VS{16}Et il le conduisit. Et voici, ils étaient dispersés sur toute la contrée, mangeant, buvant, et dansant, à cause de ce grand butin qu'ils avaient pris au pays des Philistins, et au pays de Juda.
\VS{17}Et David les frappa depuis l'aube du jour, jusqu'au soir du lendemain, et il n'en échappa aucun d'eux, hormis quatre cents jeunes hommes qui montèrent sur des chameaux, et s'enfuirent.
\VS{18}David recouvra tout ce que les Amalécites avaient emporté, il délivra aussi ses deux femmes.
\VS{19}Il ne leur manqua personne, depuis le plus petit jusqu'au plus grand, ni fils ni filles, ni butin, ni rien de ce qu'ils leur avaient emporté : David ramena tout.
\VS{20}David reprit aussi tout le gros et menu bétail, qu'on mena devant les troupeaux ; et on disait : C'est ici le butin de David.
\TextTitle{David partage le butin}
\VS{21}Puis David arriva auprès des deux cents hommes qui avaient été tellement fatigués qu'ils n'avaient pu suivre David, et qu'on avait laissés au torrent de Besor. Ils sortirent au-devant de David, et au-devant du peuple qui était avec lui. David s'étant approché du peuple, il les salua aimablement.
\VS{22}Mais tous les mauvais et méchants hommes, qui étaient allés avec David, prirent la parole, et dirent : Puisqu'ils ne sont point venus avec nous, nous ne leur donnerons rien du butin que nous avons récupéré, sinon à chacun sa femme et ses enfants, et qu'ils les emmènent, et s'en aillent.
\VS{23}Mais David dit : Mes frères, n'agissez pas ainsi au sujet de ce que Yahweh nous a donné ; il nous a gardés, et a livré entre nos mains la troupe qui était venue contre nous.
\VS{24}Qui vous écouterait dans cette affaire ? Car celui qui est resté près des bagages doit avoir autant de part que celui qui est descendu sur le champ de bataille : Ils partageront ensemble.
\VS{25}Il en fut ainsi depuis ce jour et dans la suite, il en fut fait une ordonnance et une loi en Israël, jusqu'à ce jour.
\VS{26}David revint à Tsiklag, et envoya une partie du butin aux anciens de Juda, à ses amis, en disant : Voici, un présent pour vous du butin des ennemis de Yahweh !
\VS{27}Il en envoya à ceux de Béthel, à ceux qui étaient à Ramoth du sud, à ceux de Jatthir,
\VS{28}à ceux d'Aroër, à ceux de Siphmoth, à ceux de Eschthemoa,
\VS{29}à ceux de Racal, à ceux des villes des Jerachmeélites, à ceux des villes des Kéniens,
\VS{30}à ceux d'Horma, à ceux de Cor-Aschan, à ceux d'Athac,
\VS{31}à ceux d'Hébron, et dans tous les lieux où David avait demeuré, lui et ses gens.
\Chap{31}
\TextTitle{Les Philistins battent Israël ; Saül et ses fils meurent\FTNTT{1 Ch. 10:1-14.}}
\VerseOne{}Or les Philistins livrèrent bataille à Israël, et les hommes d'Israël s'enfuirent devant les Philistins, et furent tués sur la montagne de Guilboa.
\VS{2}Les Philistins atteignirent Saül et ses fils, et tuèrent Jonathan, Abinadab et Malkischua, fils de Saül.
\VS{3}L'effort du combat se porta sur Saül, et les archers l'atteignirent et le blessèrent grièvement.
\VS{4}Alors Saül dit à celui qui portait ses armes : Tire ton épée, et transperce-moi, de peur que ces incirconcis ne viennent, ne me transpercent, et ne m'outragent. Mais celui qui portait ses armes refusa, parce qu'il était saisi de crainte. Saül prit l'épée, et se jeta dessus.
\VS{5}Alors celui qui portait les armes de Saül, voyant que Saül était mort, se jeta aussi sur son épée, et mourut avec lui.
\VS{6}Ainsi périrent en ce jour, Saül et ses trois fils, celui qui portait ses armes, et tous ses gens.
\VS{7}Ceux d'Israël qui étaient de ce côté de la vallée et de ce côté du Jourdain, ayant vu que les Israëlites s'étaient enfuis, que Saül et ses fils étaient morts, abandonnèrent les villes et s'enfuirent de sorte que les Philistins y entrèrent et s'y établirent.
\VS{8}Or il arriva que, dès le lendemain, les Philistins vinrent pour dépouiller les morts, et ils trouvèrent Saül et ses trois fils, étendus sur la montagne de Guilboa.
\VS{9}Ils coupèrent la tête de Saül et le dépouillèrent de ses armes, qu'ils envoyèrent au pays des Philistins, dans tous les environs, pour en faire savoir les nouvelles dans les maisons de leurs idoles et parmi le peuple.
\VS{10}Ils déposèrent les armes de Saül dans le temple d'Astarté, et ils attachèrent son cadavre sur les murs de Beth-Schan.
\VS{11}Lorsque les habitants de Jabès en Galaad apprirent ce que les Philistins avaient fait à Saül,
\VS{12}tous les vaillants hommes, se levèrent et marchèrent toute la nuit, et ils enlevèrent des murs de Beth-Schan le cadavre de Saül et les cadavres de ses fils. Ils revinrent à Jabès, où ils les brûlèrent ;
\VS{13}puis ils prirent leurs os, les ensevelirent sous un tamaris près de Jabès, et ils jeûnèrent sept jours.
\PPE{}
\end{multicols}
