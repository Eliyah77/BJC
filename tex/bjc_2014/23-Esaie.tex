\ShortTitle{Esaïe}\BookTitle{Esaïe}\BFont
\noindent\hrulefill
{\footnotesize
\textit{
\bigskip
{\centering{}
\\Auteur : Esaïe
\\(Heb. : Yesha'yah)
\\Signification : YAHWEH a sauvé
\\Thème : Le Messie d'Israël
\\Date de rédaction : 8\up{ème} siècle av. J.-C.\\}
}
%\bigskip
\textit{
\\Prophète en Israël et contemporain des rois, Ozias, Jotham, Achaz et Ezéchias, Esaïe fut une figure marquante en raison du contenu et de l'impact de son message.
%\bigskip
\\Prophète en Israël, Esaïe fut une figure marquante en raison du contenu et de l'impact de son message. Véritable porte-parole de Dieu, Esaïe parla de la ruine morale d'Israël, de la déportation à Babylone et des jugements de Dieu sur son peuple. Il prophétisa également sur le retour de l'exil, la restauration finale et la reconstruction de Jérusalem. Plus qu'aucun autre livre, les écrits d'Esaïe annoncent clairement la naissance du Messie, son ministère, sa mission rédemptrice, son sacrifice et son futur règne millénaire. 
%\bigskip
\\L'autorité et l'exactitude de ses prophéties ont été une source d'édification au fil des siècles.\bigskip
}
}
\par\nobreak\noindent\hrulefill
\begin{multicols}{2}
\Chap{1}
\TextTitle{Prophéties concernant Juda}
\VerseOne{}La vision d'Esaïe, fils d'Amots, qu'il a vue touchant Juda et Jérusalem, au jour d'Ozias, de Jotham, d'Achaz, et d'Ezéchias, rois de Juda.
\VS{2}Cieux, écoutez ! Et toi terre prête l'oreille ! Car Yahweh parle. J'ai nourri des enfants, je les ai élevés, mais ils se sont rebellés contre moi.
\VS{3}Le bœuf connaît son possesseur et l'âne la crèche de son maître, mais Israël n'a point de connaissance, mon peuple n'a point d'intelligence.
\VS{4}Ha! nation pécheresse, peuple chargé d'iniquités, race de gens méchants, enfants qui ne font que se corrompre ! Ils ont abandonné Yahweh, ils ont irrité par leur mépris le Saint d'Israël, ils se sont retirés en arrière.
\VS{5}Pourquoi seriez-vous encore frappés ? Vous ajouteriez la révolte ! La tête entière est malade et tout le cœur est languissant.
\VS{6}De la plante du pied jusqu'à la tête, il n'y a rien de sain en lui : il n'y a que que blessures, meurtrissures et plaies pourie, qui n'ont été ni netoyées, ni bandées, et donc acune n'a été adoucies par l'huile.
\VS{7}Votre pays n'est que désolation et vos villes sont en feu, des étrangers dévorent votre terre sous vos yeux, et cette désolation est comme un bouleversement fait par des étrangers.
\VS{8}Car la fille de Sion est restera comme une cabane dans une vigne, comme une cabane dans un champ de concombres ; comme une ville assiégée.
\VS{9}Si Yahweh des armées ne nous avait pas laissé un petit reste, qui est même bien peu, nous serions comme Sodome, nous ressemblerions à Gomorrhe.
\TextTitle{Yahweh rejette la religiosité et recherche la justice}
\VS{10}Ecoutez la parole de Yahweh, chefs de Sodome, prêtez l'oreille à la loi de notre Dieu, peuple de Gomorrhe !
\VS{11}Qu'ai-je à faire, dit Yahweh, de la multitude de vos sacrifices ? Je suis rassasié des holocaustes de béliers et de la graisse des veaux ; je ne prends point plaisir au sang des taureaux ni des agneaux ni des boucs\FTNT{1 S. 15:22 ; Os. 8:13 ; Mt. 9:13.}.
\VS{12}Quand vous entrez pour vous présenter devant ma face, qui ai requis cela de votre main, que vous fouliez mes ?
\VS{13}Ne continuez plus à m'apporter de vaines offrandes : J'ai en horreur le parfum, quant aux nouvelles lunes,  aux sabbats et à la convocation des assemblées ; je ne puis supporter la méchanceté ni vos convocations solennelles.
\VS{14}Mon âme hait vos nouvelles lunes et vos fêtes solennelles ; elles me sont fâcheuses, je suis las de les supporter.
\VS{15}C'est pourquoi quand vous étendez vos mains, je cache mes yeux de vous ; quand vous multipliez vos prières, je ne les exauce pas ; vos mains sont pleines de sang\FTNT{Es. 59:1-3 ;Mi. 3:4.}.
\VS{16}Lavez-vous, purifiez-vous, ôtez de devant mes yeux la méchanceté de vos actions ; cessez de faire le mal.
\VS{17}Apprenez à bien faire, recherchez la droiture, redressez celui est foulé ; faites justice à l'orphelin, défendez la cause de la veuve.
\TextTitle{Mise en garde ; appel à la justice de Yahweh}
\VS{18}Venez maintenant, dit Yahweh, et débattons nos droits. Si vos péchés sont comme l'écarlate, ils seront blanchis comme la neige ; s'ils sont rouges comme le cramoisi ils seront blanchis comme la laine.
\VS{19}Si vous obéissez volontairement, vous mangerez le meilleur du pays.
\VS{20}Mais si vous refusez d'obéir et si vous êtes rebelles, vous serez dévorés par l'épée car la bouche de Yahweh a parlé.
\VS{21}Comment la cité fidèle est-elle devenue une prostituée ? Elle était pleine de droiture et la justice y habitait ; mais maintenant elle est pleine de meurtriers !
\VS{22}Ton argent s'est changé en scories; ton breuvage est mêlé d'eau. 
\VS{23}Les chefs de ton peuple sont rebelles et compagnons des voleurs ; chacun d'eux aime les présents, ils courent après les récompenses ; ils ne font point droit à l'orphelin, et la cause de la veuve ne vient point devant eux. 
\VS{24}C'est pourquoi le Seigneur, Yahweh des armées, le Puissant d'Israël dit ; Ha ! je me satisferai en punissant mes adversaires, et je me vengerai de mes ennemis. 
\VS{25}Et je remettrai ma main sur toi, je refondrai tes scories comme avec la potasse, et j'ôterai tout ton étain. 
\VS{26}Je rétablirai tes juges, tels qu'ils étaient autrefois, et tes conseillers, tels qu'ils étaient au commencement\FTNT{Dans le royaume messianique, le gouvernement théocratique sera restauré et la fonction des juges sera rétablie (voir livre des Juges ; Mt. 19:28 ; 1 Co. 6:2-3).}. Après cela, on t'appellera, cité de la justice, ville fidèle.
\VS{27}Sion sera rachetée par la droiture et ceux qui s'y convertiront seront rachetés par la justice.
\VS{28}Mais les rebelles et les pécheurs seront détruits ensemble et ceux qui ont abandonné Yahweh serons consumés.
\VS{29}Car on sera honteux à cause des térébinthes que vous avez désirés et vous rougirez à cause des jardins que vous avez choisis\FTNT{Des cultes idolâtres avaient lieu autour des térébinthes et dans des jardins (De. 16:21 ; Es. 57:4-5 ; Es. 65:3 ; Jé. 2:20 ; Ez. 20:28 ; Os. 4:13).}.
\VS{30}Car vous serez comme le térébinthe dont le feuillage tombe, comme un jardin qui n'a pas d'eau.
\VS{31}Et le fort sera de l'étoupe, et son œuvre une étincelle ; et tous deux brûleront ensemble et il n'y aura personne pour éteindre le feu.
\Chap{2}
\TextTitle{Vision du règne messianique}
\VerseOne{}La parole qu'Esaïe, fils d'Amots a vue touchant Juda et Jérusalem.
\VS{2}Or il arrivera, dans les derniers jours\FTNT{Voir Ge. 49:1-2.}, que la montagne de la maison de Yahweh sera affermie au  sommet des montagnes, qu'elle sera élevée par-dessus les collines et que toutes les nations y afflueront.
\VS{3}Et plusieurs peuples iront et diront ; venez, et montons à la montagne de Yahweh, à la maison du Dieu de Jacob ; et il nous instruira de ses voies, et nous marcherons dans ses sentiers ; car la loi sortira de Sion, et la parole de Yahweh sortira de Jérusalem. 
\VS{4}Il exercera le jugement parmi les nations et reprendra plusieurs peuples. De leurs épées ils forgeront des hoyaux, et de leurs lances des serpes ; une nation ne lèvera plus l'épée contre une autre et ils ne s'adonneront plus à la guerre.
\VS{5}Venez, ô maison de Jacob, et marchons dans la lumière de Yahweh.
\TextTitle{L'orgueilleux abaissé au jour de Yahweh}
\VS{6}Certes tu as rejeté ton peuple, la maison de Jacob, parce qu'ils se sont remplis d'orient et adonnés à la divination comme les Philistins, et parce qu'ils s'allient aux fils des étrangers\FTNT{De. 18:8-13 ; Os. 13:2 ; Mi. 5:11-13.}.
\VS{7}Son pays est rempli d'argent et d'or, et il n'y a pas de fin à ses trésors ; son pays est rempli de chevaux, et il n'y a pas de fin à ses chars.
\VS{8}Son pays est rempli d'idoles ; ils se sont prosternés devant l'ouvrage de leurs mains et devant ce que leurs doigts ont fait.
\VS{9}Les hommes du commun se sont inclinés, et les hommes de qualité se sont baissés ; ne leur pardonne donc point.
\VS{10}Entre dans les rochers et cache-toi dans la poussière, à cause de la frayeur de Yahweh, et à cause de la gloire de sa majesté\FTNT{Ap. 6:15-16.}.
\VS{11}Les regards hautains seront abaissés et les hommes qui s'élèvent seront humiliés, Yahweh sera seul haut élevé en ce jour-là.
\VS{12}Car il y a un jour assigné par Yahweh des armées contre tout homme orgueilleux et hautain, et contre tout homme qui s'élève, afin qu'il soit abaissé ;
\VS{13}contre tous les cèdres du Liban, hauts et élevés, et contre tous les chênes de Basan ;
\VS{14}contre toutes les hautes montagnes, et contre toutes les collines élevées ;
\VS{15}contre toutes les hautes tours, et contre toutes les murailles fortes ;
\VS{16}contre tous les navires de Tarsis, et contre toutes les peintures de plaisance.
\VS{17}L'arrogance des hommes sera humiliée et les hommes qui s'élèvent seront abaissés :
\VS{18}Yahweh seul sera élevé en ce jour-là. Quant aux idoles, elles tomberont toutes.
\VS{19}Les hommes entreront dans les cavernes des rochers et dans les trous de la terre, à cause de la frayeur de Yahweh et à cause de sa gloire magnifique, lorsqu'il se lèvera pour frapper la terre.
\VS{20}En ce jour-là, les hommes jetteront aux taupes et aux chauves-souris leurs idoles d'argent et leurs idoles d'or, qu'ils s'étaient faites pour se prosterner devant elles ;
\VS{21}et ils entreront dans les fentes des rochers et dans les creux des rochers, à cause de la frayeur de Yahweh, et à cause de sa gloire magnifique, quand il se lèvera pour punir la terre.
\VS{22}Retirez-vous de l'homme dans les narines duquel il n'y a qu'un souffle : Car quel cas mérite-t-il qu'on en fasse ?
\Chap{3}
\TextTitle{Le péché, cause de dissolution nationale}
\VerseOne{}Car voici, le Seigneur, Yahweh des armées, va ôter de Jérusalem et de Juda tout appui et toute ressource, toute ressource de pain et toute ressource d'eau.
\VS{2}L'homme fort et l'homme de guerre, le juge et le prophète, le devin et l'ancien,
\VS{3}le chef de cinquante et le magistrat, l'artisan distingué et l'homme éloquent.
\VS{4}Et je leur donnerai de jeunes gens pour chefs et des enfants domineront sur eux.
\VS{5}Le peuple sera opprimé ; l'un opprimera l'autre, chacun son prochain. Le jeune homme se portera arrogamment contre le vieillard et l'homme de rien contre l'honorable.
\VS{6}Même un homme ira jusqu'à saisir son frère dans la maison paternelle et lui dira : Tu as un manteau, sois notre chef et prends en main ces ruines.
\VS{7}Ce jour même il répondra : Je ne suis pas médecin et dans ma maison il n'y a ni pain ni manteau ; ne m'établissez donc pas chef du peuple.
\VS{8}Certes Jérusalem est renversée, et Juda est tombée, parce que leurs langues et leurs actions sont contre Yahweh, pour braver les regards de sa gloire.
\VS{9}L'aspect de leur visage témoigne contre eux, ils publient leur péché comme Sodome, ils ne le cachent pas. Malheur à leur âme, car ils ont attiré le mal sur eux !
\VS{10}Dites au juste que du bien lui arrivera, car il mangera le fruit de ses œuvres.
\VS{11}Malheur au méchant qui ne cherche qu'à faire le mal, car la rétribution de ses mains lui sera rendue.
\VS{12}Quant à mon peuple, il a pour oppresseur des enfants, et des femmes dominent sur lui. Mon peuple, ceux qui te conduisent t'égarent, ils corrompent le chemin dans lequel tu marches.
\VS{13}Yahweh se présente pour plaider, il se tient debout pour juger les peuples.
\VS{14}Yahweh entre en jugement avec les anciens de son peuple et avec ses chefs ; car vous avez brouté la vigne, et ce que vous avez ravi au pauvre est dans vos maisons.
\VS{15}Que vous revient-il de fouler mon peuple, et d'écraser le visage des affligés ? dit le Seigneur, Yahweh des armées.
\TextTitle{Les filles hautaines de Sion}
\VS{16}Yahweh dit aussi : Parce que les filles de Sion sont hautaines et qu'elles marchent le cou tendu et les yeux pleins de convoitise, parce qu'elles marchent avec une fière démarche faisant du bruit avec leurs pieds,
\VS{17}Yahweh rendra chauve le sommet de la tête des filles de Sion, Yahweh découvrira leur nudité.
\VS{18}En ce temps-là, le Seigneur ôtera l'ornement de leurs anneaux de cheville, et les filets et les croissants.
\VS{19}Les pendants d'oreilles, les bracelets et les voiles ;
\VS{20}les parures de la tête, les chaînettes des pieds et les ceintures, les boîtes à parfum et les amulettes ;
\VS{21}les anneaux et les bagues qui leur pendent sur le nez ;
\VS{22}les vêtements de fête et les larges tuniques, les manteaux et les gibecières ;
\VS{23}les miroirs et les chemises fines, les tiares et les voiles légers.
\VS{24}Et il arrivera qu'au lieu de parfum, il y aura de la puanteur ; au lieu de ceintures, des cordes ; au lieu de cheveux bouclés, des têtes chauves ; au lieu de robes flottantes, des sacs étroits ; et au lieu d'un beau teint, un teint tout hâlé.
\VS{25}Tes hommes tomberont par l'épée et ta force par la guerre.
\VS{26}Et ses portes gémiront et mèneront deuil ; désolée, elle s'assiéra par terre.
\Chap{4}
\TextTitle{Vision du règne messianique\FTNTT{Es. 11:1-16.}}
\VerseOne{}Et en ce jour sept femmes saisiront un seul homme, et diront : Nous mangerons notre pain et nous nous vêtirons de nos habits ; seulement fais-nous porter ton nom ; ôte notre opprobre.
\VS{2}En ce temps-là, le germe de Yahweh\FTNT{Jésus est le « germe » de Yahweh (Es. 4:2) et le germe de David (Jé. 23:5 ; Za. 3:8 ; Za. 6:12). Ce germe a été placé par la vertu du Saint-Esprit dans le sein d'une vierge (Es. 7:14 ; Lu. 1:34-35) et l'enfant qui naquit d'elle fut appelé « Fils de Dieu » tout en étant le Dieu Tout-Puissant. Il existe de toute éternité en forme de Dieu (Jn. 1:1 ; Es. 9:5), mais il a été fait chair pour nous sauver (Jn. 1:14. 1 Ti. 3:16).
C'est le plus grand des miracles et la démonstration de sa divinité, de sa sagesse et de son amour envers les hommes.
} sera plein de noblesse et de gloire, et le fruit de la terre plein de grandeur et d'excellence pour les réchappés d'Israël.
\VS{3}Et il arrivera que les restes de Sion, et les restes de Jérusalem, seront appelés saints; et ceux de Jérusalem qui seront inscrits parmi les vivants\FTNT{Es. 10:20-22 ; Ro. 9:27 ; Ro. 11:5 ;}.
\VS{4}Quand le Seigneur aura lavé la souillure des filles de Sion et purifié Jérusalem du sang qui est au milieu d'elle, par l'esprit de jugement et par l'esprit qui consume;
\VS{5}aussi Yahweh créera sur toute l'étendue du mont Sion et sur ses assemblées une nuée avec une fumée pendant le jour, et une splendeur de feu flamboyant pendant la nuit, car la gloire se répandra partout.
\VS{6}Et il y aura un tabernacle pour donner de l'ombre contre la chaleur du jour, pour servir de refuge et d'asile contre la tempête et la pluie\FTNT{Ap. 21:3.}.
\Chap{5}
\TextTitle{Israël, vigne de Yahweh}
\VerseOne{}Je chanterai maintenant pour mon bien-aimé le cantique de mon bien-aimé sur sa vigne. Mon bien-aimé avait une vigne sur un coteau fertile.
\VS{2}Il l'environna d'une haie, en ôta les pierres et y planta des ceps exquis ; il bâtit une tour au milieu d'elle et il y creusa aussi une cuve. Puis il espéra qu'elle produirait des raisins, mais elle a produit des grappes sauvages\FTNT{Lu. 13:6-9.}.
\VS{3}Maintenant donc, vous habitants de Jérusalem et vous hommes de Juda, jugez, je vous prie, entre moi et ma vigne.
\VS{4}Qu'y avait-il encore à faire à ma vigne que je ne lui aie fait ? Pourquoi, quand j'ai attendu qu'elle produirait des raisins, a-t-elle produit des grappes sauvages ?
\VS{5}Maintenant donc je vous dirai ce que je vais faire à ma vigne : J'ôterai sa haie et elle sera broutée ; je romprai sa clôture et elle sera foulée.
\VS{6}Et je la réduirai en désert, elle ne sera plus taillée, ni cultivée ; les ronces et les épines y croîtront ; et je commanderai aux nuées qu'elles ne laissent plus tomber de pluie sur elle.
\VS{7}Or la maison d'Israël est la vigne de Yahweh des armées, et les hommes de Juda sont la plante en laquelle il prenait plaisir. Il en attendait de la droiture, et voici du saccagement ! De la justice, et voici des cris de détresse !
\TextTitle{Six malheurs en punition de l'infidélité d'Israël}
\VS{8}Malheur à ceux qui ajoutent maison à maison, et qui joignent champ à champ, jusqu'à ce qu'il n'y ait plus d'espace et qu'ils habitent seuls au milieu du pays.
\VS{9}Yahweh des armées m'a fait entendre : Certainement, ces maisons nombreuses seront réduites en désolation, ces maisons grandes et belles seront sans d'habitants.
\VS{10}Même dix arpents de vigne ne produiront qu'un bath, et un homer de semence ne produira qu'un épha.
\VS{11}Malheur à ceux qui se lèvent de bon matin, qui recherchent les boissons enivrantes, qui demeurent jusqu'au soir, et jusqu'à ce que le vin les échauffe !
\VS{12}La harpe et le luth, le tambourin, la flûte et le vin sont dans leurs festins ; mais ils ne regardent pas l'œuvre de Yahweh et ils ne voient pas l'ouvrage de ses mains.
\VS{13}C'est pourquoi mon peuple sera emmené captif, parce qu'il n'a pas de connaissance\FTNT{2 R. 24:14-16 ; Os. 4:6.} ; et les plus honorables parmi eux seront des pauvres qui mourront de faim, et leur multitude sera asséchée par la soif.
\VS{14}C'est pourquoi le scheol s'élargit, il ouvre sa gueule outre mesure ; et sa magnificence y descend, sa multitude, sa pompe et tous ceux qui s'y réjouissent.
\VS{15}Ceux du commun seront humiliés, et les personnes de qualité seront abaissées, et les yeux des hautains seront abaissés.
\VS{16}Et Yahweh des armées sera haut élevé en jugement, et le Dieu Saint sera sanctifié dans la justice.
\VS{17}Les agneaux paîtront selon qu'ils seront parqués, et les étrangers dévoreront les champs désolés des riches.
\VS{18}Malheur à ceux qui tirent l'iniquité avec des cordes de vanité, et le péché avec les traits d'un char,
\VS{19}et qui disent : Qu'il hâte et qu'il fasse venir son œuvre bientôt, afin que nous la voyions ! Que le conseil du Saint d'Israël s'avance et vienne, afin que nous le connaissions !
\VS{20}Malheur à ceux qui appellent le mal bien et le bien mal\FTNT{Mi. 7:2.} ; qui font les ténèbres lumière, et la lumière ténèbres ; qui font l'amertume douceur, et la douceur amertume.
\VS{21}Malheur à ceux qui sont sages à leurs yeux, et se considérant eux-mêmes intelligents !
\VS{22}Malheur à ceux qui sont forts pour boire le vin et vaillants pour mêler des liqueurs fortes ;
\VS{23}qui justifient le méchant pour des présents et qui ôtent à chacun des justes sa justice.
\VS{24}C'est pourquoi, comme le flambeau de feu consume le chaume, et la flamme consume l'herbe sèche, ainsi leur racine sera comme la pourriture, et leur fleur sera détruite comme la poussière ; parce qu'ils ont rejeté la loi de Yahweh des armées, et ils ont méprisé la parole du Saint d'Israël.
\VS{25}C'est pourquoi la colère de Yahweh s'enflamme contre son peuple, il étend sa main sur lui, et il le frappe ; les montagnes tremblent et leurs cadavres ont été mis en pièces au milieu des rues. Malgré tout cela, sa colère ne se détourne pas, mais sa main est encore étendue.
\VS{26}Il élève une bannière pour les nations éloignées et il siffle à chacune d'elles depuis les extrémités de la terre ; et voici chacune viendra promptement et légèrement.
\VS{27}Nul n'est fatigué, nul ne chancelle de lassitude, personne ne sommeille ni ne dort ; et la ceinture de leurs reins ne sera point déliée, et la courroie de leurs souliers ne sera point rompue.
\VS{28}Leurs flèches sont aiguës et tous leurs arcs tendus ; les sabots de leurs chevaux ressemblent à des cailloux et les roues de leurs chars à un tourbillon.
\VS{29}Leur rugissement est comme celui d'un vieux lion ; ils rugissent comme des lionceaux ; ils grondent et saisissent la proie, il l'emportent et personne ne vient à son secours.
\VS{30}En ce jour-là, on mènera un bruit sur lui, semblable au mugissement de la mer ; en regardant la terre, on ne verra que ténèbres et détresse ; la lumière sera obscurcie dans le ciel.
\Chap{6}
\TextTitle{Révélation de Yahweh à Esaïe}
\VerseOne{}L'année de la mort du roi Ozias, je vis le Seigneur assis sur un trône haut et élevé, et les pans de sa robe remplissaient le temple\FTNT{2 Ch. 26:23.}.
\VS{2}Les séraphins se tenaient au-dessus de lui ; et chacun d'eux avait six ailes ; deux dont ils se couvraient la face, deux dont ils se couvraient les pieds et deux dont ils se servaient pour voler.
\VS{3}Et ils criaient l'un à l'autre, et disaient : Saint, saint, saint est Yahweh des armées ! Toute la terre est pleine de sa gloire !
\VS{4}Et les poteaux des seuils furent ébranlés dans leurs fondements par la voix de celui qui criait ; et la maison fut remplie de fumée.
\VS{5}Alors je dis : Malheur à moi ! Je suis perdu, car je suis un homme dont les lèvres sont impures, j'habite au milieu d'un peuple dont les lèvres sont impures et mes yeux ont vu le Roi, Yahweh des armées\FTNT{Jg. 13:21-22.}.
\VS{6}Mais l'un des séraphins vola vers moi, tenant à la main un charbon ardent, qu'il avait pris sur l'autel avec des pincettes.
\VS{7}Il en toucha ma bouche et dit : Voici, ceci a touché tes lèvres, c'est pourquoi ton iniquité est ôtée et la propitiation est faite pour ton péché.
\VS{8}Puis j'entendis la voix du Seigneur, disant : Qui enverrai-je et qui marchera pour nous ? Je répondis : Me voici, envoie-moi.
\TextTitle{Mission d'Esaïe}
\VS{9}Et il dit : Va et dis à ce peuple : En attendant vous entendrez, mais vous ne comprendrez point ; et en voyant vous verrez, mais vous n'apercevrez point.
\VS{10}Engraisse le cœur de ce peuple, et rends ses oreilles pesantes, et bouche-lui les yeux ; de peur qu'il ne voie de ses yeux, et qu'il n'entende de ses oreilles, et que son cœur ne comprenne, et qu'il ne se convertisse, et qu'il soit guéri\FTNT{Mt. 13:15 ; Mc. 4:12 ; Jn. 12:40 ; Ac. 28:27.}.
\VS{11}Je dis : Jusqu'à quand, Seigneur ? Et il répondit : Jusqu'à ce que les villes soient dévastées, jusqu'à ce qu'il n'y ait plus d'habitants, ni d'hommes dans les maisons, et que la terre soit mise en entière désolation ;
\VS{12} et que Yahweh ait dispersé au loin les hommes, et que celle qu'il aura abandonnée ait demeuré longtemps au milieu du pays.
\VS{13}Toutefois s'il y reste un dixième des habitants, ils reviendront pour être la proie des flammes. Mais comme le térébinthe et le chêne conservent leur tronc quand ils sont abattus, une sainte postérité renaîtra de ce peuple\FTNT{Ro. 11:17-25.}.
\Chap{7}
\TextTitle{Retsin et Pékach complote contre Juda}
\VerseOne{}Il arriva du temps d'Achaz, fils de Jotham, fils d'Ozias, roi de Juda, que Retsin, roi de Syrie, et Pékach, fils de Remalia, roi d'Israël, montèrent contre Jérusalem pour lui faire la guerre ; mais ils ne purent l'assiéger.
\VS{2}Et on rapporta à la maison de David : La Syrie s'est reposée sur Ephraïm. Et le cœur d'Achaz et le cœur de son peuple furent ébranlés comme les arbres des forêts qui sont ébranlés par le vent.
\VS{3}Alors Yahweh dit à Esaïe : Sors maintenant au devant d'Achaz, toi et Schear-Jaschub, ton fils, vers l'extrémité de l'aqueduc de l'étang supérieur, sur la route du champ du foulon.
\VS{4}Et dis-lui : Prends garde à toi, et demeure tranquille, ne crains point, et que ton cœur ne devienne point lâche à cause des deux queues de ces tisons fumants, à cause de l'ardeur de la colère de Retsin et de la Syrie, et du fils de Remalia, 
\VS{5}de ce que la Syrie délibère avec Ephraïm et le fils de Remalia de te faire du mal, en disant :
\VS{6}Montons contre Juda, assiégeons la ville, battons-la en brèche, et établissons pour roi le fils de Tabeel au milieu d'elle.
\VS{7}Ainsi parle le Seigneur, Yahweh : Cela n'aura point d'effet et cela ne se fera point.
\VS{8}Car la tête de la Syrie c'est Damas, et le chef de Damas c'est Retsin. Encore soixante-cinq ans, Ephraïm sera froissé pour n'être plus un peuple.
\VS{9}Et la tête d'Ephraïm c'est la Samarie, et le chef de la Samarie c'est le fils de Remalia. Si vous ne croyez pas, certainement vous ne serez point affermis.
\TextTitle{Annonce de la naissance d'Emmanuel}
\VS{10}Et Yahweh parla de nouveau à Achaz, en disant :
\VS{11}Demande pour toi un signe à Yahweh ton Dieu, demande-le, soit dans les bas lieux, soit dans les lieux élevés.
\VS{12}Et Achaz répondit : Je ne demanderai rien et je ne tenterai point Yahweh.
\VS{13}Alors Esaïe dit : Ecoutez maintenant, ô maison de David ! Est-ce trop peu pour vous de lasser les hommes, que vous lassiez aussi mon Dieu ?
\VS{14}C'est pourquoi le Seigneur lui-même vous donnera un signe : Voici, une vierge sera enceinte, et elle enfantera un fils, et elle lui donnera le nom d'Emmanuel\FTNT{Le nom « Emmanuel » est dérivé de l'hébreu « Immanuw'el » qui signifie « Dieu est avec nous ». Jésus a dit aux disciples dans Mt. 28:20 : « Et moi, je suis avec vous tous les jours jusqu'à la fin des temps ». Jésus est Emmanuel, Dieu avec nous jusqu'à la fin des temps.}.
\VS{15}Il mangera du beurre et du miel, jusqu'à ce qu'il sache rejeter le mal et choisir le bien.
\VS{16}Mais avant que l'enfant sache rejeter le mal et choisir le bien, la terre que tu as en détestation sera abandonnée par ses deux rois.
\TextTitle{Prophétie sur l'imminente invasion de Juda\FTNTT{2 Ch. 28:1-20.}}
\VS{17}Yahweh fera venir sur toi, sur ton peuple et sur la maison de ton père, par le roi d'Assyrie, des jours tels qu'il n'y en a point eu de semblable depuis le jour où Ephraïm s'est séparé de Juda.
\VS{18}Et il arrivera qu'en ce jour-là, Yahweh sifflera aux mouches qui sont à l'extrémité des ruisseaux d'Egypte, et aux abeilles qui sont au pays d'Assyrie.
\VS{19}Elles viendront et se poseront toutes dans les vallées désertes, et dans les fentes des rochers, et par tous les buissons, et par tous les halliers.
\VS{20}En ce jour-là, le Seigneur rasera avec le rasoir pris à louage au-delà du fleuve, avec le roi d'Assyrie, la tête et les poils des pieds, et il enlèvera aussi la barbe\FTNT{2 R. 16:5-9.}.
\VS{21}Et il arrivera, en ce jour-là, qu'un homme nourrira une jeune vache et deux brebis.
\VS{22}Et il arrivera que de l'abondance du lait qu'elles rendront, il mangera du beurre ; car tous ceux qui seront restés dans le pays se mangeront du beurre et du miel.
\VS{23}Et il arrivera, en ce jour-là, que tout lieu où il y aura mille vignes, valant mille sicles d'argent, sera réduit en ronces et en épines.
\VS{24}On y entrera avec des flèches et avec l'arc, car tout le pays ne sera que ronces et épines.
\VS{25}Et dans toutes les montagnes que l'on cultivait avec la bêche, on ne craindra plus de voir des ronces et des épines ; mais on y lâchera les bœufs, et la brebis en foulera le sol.
\Chap{8}
\TextTitle{Annonce de la défaite de Damas et de la Samarie}
\VerseOne{}Et Yahweh me dit : Prends un grand rouleau et écris dessus en grosses lettres : Qu'on se dépêche de butiner, qu'on se hâte de piller.
\VS{2}De quoi je pris avec moi des témoins fidèles : Urie, le sacrificateur, et Zacharie, fils de Bérékia.
\VS{3}Puis je m'étais approché de la prophétesse ; elle conçut et elle enfanta un fils. Et Yahweh me dit : Donne-lui pour nom Maher-Schalal-Chasch-Baz\FTNT{« Maher-Schalal-Chasch-Baz » signifie « rapide au butin, rapide sur la proie ».}.
\VS{4}Car avant que l'enfant sache dire : Mon père ! Ma mère ! On enlèvera la puissance de Damas et le butin de Samarie, devant le roi d'Assyrie.
\VS{5}Et Yahweh continua encore de me parler, en disant :
\VS{6}Parce que ce peuple a rejeté les eaux de Siloé qui coulent doucement, et qu'il s'est réjoui au sujet de Retsin et du fils de Remalia,
\VS{7}à cause de cela, voici, le Seigneur va faire monter contre eux les puissantes et grandes eaux du fleuve, le roi d'Assyrie et toute sa gloire. Il s'élèvera partout au-dessus de son lit, et il se répandra sur toutes ses rives.
\VS{8}Et il pénètrera dans Juda, il débordera et inondera, il atteindra jusqu'au cou. Et les étendues de ses ailes rempliront la largeur de ton pays, ô Emmanuel !
\TextTitle{Exhortation aux disciples de Yahweh à rester fidèles}
\VS{9}Alliez-vous, peuples ! Et vous serez brisés ; prêtez l'oreille, vous tous qui êtes d'un pays éloigné ! Equipez-vous, et vous serez brisés ; Equipez-vous, et vous serez brisés.
\VS{10}Prenez conseil, et il sera dissipé ; dites la parole, et elle sera sans effet : Car Dieu est avec nous.
\VS{11}Car ainsi m'a parlé Yahweh, avec une main forte, et il m'instruisit de n'aller point par le chemin de ce peuple-ci, en me disant :
\VS{12}Ne dites point : Conjuration, toutes les fois que ce peuple dit conjuration ; ne craignez point ce qu'il craint, et ne vous en épouvantez point.
\VS{13}Sanctifiez Yahweh des armées, lui-même, c'est lui que vous devez craindre et redouter.
\VS{14}Et il sera un sanctuaire, mais aussi une pierre d'achoppement\FTNT{Yahweh s'est présenté comme une pierre d'achoppement et un rocher de scandale. En Es. 44:8 il affirme d'ailleurs ne pas connaître d'autre rocher que lui. Esaïe n'est pas le seul prophète à qui le Seigneur s'est révélé comme étant une pierre et un rocher. Dans le Ps. 118:22-23, il est dit : « La pierre qu'ont rejetée ceux qui bâtissaient est devenue la principale de l'angle ». Daniel et Zacharie ont également prophétisé au sujet de cette pierre : « Tu regardais, lorsqu'une pierre se détacha sans le secours d'aucune main, frappa les pieds de fer et d'argile de la statue, et les mit en pièces. Mais la pierre qui avait frappé la statue devint une grande montagne, et remplit toute la terre » (Da. 2:34-35). « Car voici, pour ce qui est de la pierre que j'ai placée devant Josué, il y a sept yeux sur cette seule pierre ; voici, je graverai moi-même ce qui doit y être gravé, dit Yahweh des armées; et j'enlèverai l'iniquité de ce pays, en un jour » (Za. 3:9). Ces prophéties se sont accomplies en Jésus-Christ, l'Agneau de Dieu qui ôte le péché du monde (Jn. 1:29). Le Seigneur s'est d'ailleurs clairement identifié à la pierre angulaire, affirmant ainsi sa divinité (Lu. 20:17-19). En Mt. 16:18, il s'est présenté comme le rocher inébranlable sur lequel il allait bâtir son Eglise. De plus, il est à noter que dans le livre de l'Apocalypse, l'Agneau possède sept yeux comme la pierre vue par Zacharie (Ap. 5:6). Ces sept yeux sont aussi les sept lampes du chandelier d'or que Zacharie et Jean avaient également vues (Za. 4:2 ; Ap. 4:5 ). Or le chiffre sept symbolise la plénitude et la perfection divines. Esaïe prophétisa encore en ces termes : « Voici, j'ai mis pour fondement en Sion une pierre, une pierre éprouvée, une pierre angulaire de prix, solidement posée; celui qui la prendra pour appui n'aura point hâte de fuir » (Es. 28:16). Les écrits de la nouvelle alliance attestent l'accomplissement de cette prophétie en Jésus-Christ, notamment par la bouche de Paul et de Pierre : « Vous avez été édifiés sur le fondement des apôtres et des prophètes, Jésus-Christ lui-même étant la pierre angulaire » (Ep. 2:20). « Car personne ne peut poser un autre fondement que celui qui a été posé, savoir Jésus-Christ » (1 Co. 3:11). « Approchez-vous de Lui, pierre vivante, rejetée par les hommes, mais choisie et précieuse devant Dieu ; et vous-mêmes, comme des pierres vivantes, édifiez-vous pour former une maison spirituelle, un saint sacerdoce, afin d'offrir des victimes spirituelles, agréables à Dieu, par Jésus-Christ ». (1 Pi. 2:4-5).}, un rocher de scandale pour les deux maisons d'Israël, un filet et un piège pour les habitants de Jérusalem.
\VS{15}Plusieurs d'entre eux trébucheront, ils tomberont et se briseront, ils seront enlacés et pris.
\VS{16}Enveloppe cet oracle, scelle cette révélation parmi mes disciples.
\VS{17}Je m'attends à Yahweh, qui cache sa face à la maison de Jacob, et je m'attendrai à lui.
\VS{18}Me voici, avec les enfants que Yahweh m'a donnés, pour être un signe et un miracle en Israël, de la part de Yahweh des armées, qui habite sur la montagne de Sion.
\VS{19}Si l'on vous dit : Consultez ceux qui évoquent les morts et les diseurs de bonne aventure, qui poussent des sifflements et des soupirs, répondez : Un peuple ne consultera-t-il pas son Dieu ? S'adressera-t-il aux morts en faveur des vivants ?
\VS{20}A la loi et au témoignage ! Si l'on ne parle pas ainsi, il n'y aura certainement point d'aurore pour le peuple.
\VS{21}Et il sera errant dans le pays, accablé et affamé ; et il arrivera que dans sa faim, il s'irritera, maudira son roi et son Dieu, et tournera les yeux en haut ;
\VS{22}puis il regardera vers la terre, et voici, il n'y aura que détresse, ténèbres et de sombres angoisses : Il sera enfoncé dans l'obscurité.
\VS{23}Car il n'y a plus d'obscurité pour celle qui a été affligée : Si le temps passé a couvert de malédiction le pays de Zabulon et le pays de Nephthali, le dernier temps couvrira de gloire la terre voisine de la mer, au-delà du Jourdain, dans la Galilée des Gentils.
\Chap{9}
\TextTitle{Annonce de la naissance et du règne du Messie}
\VerseOne{}Le peuple qui marchait dans les ténèbres voit une grande lumière, et la lumière resplendit sur ceux qui habitaient le pays de l'ombre de la mort\FTNT{Mt. 4:15-16.}.
\VS{2}Tu multiplies la nation, tu lui accordes de grandes joies, ils se réjouissent devant toi, comme on se réjouit à la moisson, comme on s'égaye quand on partage le butin.
\VS{3}Car tu as mis en pièces le joug dont il était chargé, et le bâton dont on lui battait ordinairement les épaules, et la verge de celui qui l'opprimait, comme au jour de Madian.
\VS{4}Parce que toute bataille de guerrier se fait dans un bruit confus, et que le vêtement est vautré dans le sang ; mais ceci sera comme un embrasement, quand le feu dévore quelque chose.
\VS{5}Car un enfant nous est né, un Fils nous a été donné\FTNT{La Bible enseigne que Jésus-Christ est Dieu (Ro. 9:5) et en même temps homme (Ph. 2:7). Jésus n'est pas un homme devenu Dieu, mais Dieu devenu homme. Jésus-Christ n'est pas seulement Dieu, il est homme et il n'est pas seulement homme, il est Dieu. Il n'est pas non plus 50\% homme et 50\% Dieu, il est 100\% homme et 100\% Dieu.}, et l'empire reposera sur son épaule - on l'appellera l'Admirable, le Conseiller, le Dieu Puissant, le Père d'éternité\FTNT{Philippe, disciple de Jésus-Christ voulait rencontrer le Père. Il posa au Seigneur cette question « Seigneur, montre-nous le Père, et cela nous suffit » (Jn. 14:8). Jésus lui répondit : « Il y a si longtemps que je suis avec vous, et tu ne m'as pas connu, Philippe ! » (Jn. 14:9).}, le Prince de paix -
\VS{6}pour accroître l'empire, et une paix sans fin au trône de David et à son royaume, pour l'affermir et le soutenir par le droit et par la justice, dès maintenant et à toujours\FTNT{Lu. 1:32-33.}. Voilà ce que fera le zèle de Yahweh des armées.
\TextTitle{Jugement sur le royaume du Nord}
\VS{7}Le Seigneur envoie une parole à Jacob, et elle tombe sur Israël\FTNT{Ge. 32:28.}.
\VS{8}Et tout le peuple en aura connaissance, Ephraïm et les habitants de Samarie qui disent avec orgueil et avec un cœur hautain :
\VS{9}Des briques sont tombées, mais nous bâtirons en pierres de taille ; des sycomores ont été coupés, mais nous les changerons en cèdres.
\VS{10}Yahweh élèvera contre eux les ennemis de Retsin, et il armera les ennemis d'Israël ;
\VS{11}la Syrie à l'orient et les Philistins à l'occident ; et ils dévoreront Israël à gueule ouverte. Malgré tout cela, sa colère ne s'apaise point, et sa main est encore étendue.
\VS{12}Parce que le peuple ne revient pas à celui qui le frappe, et il ne cherche pas Yahweh des armées.
\VS{13}A cause de cela Yahweh retranchera d'Israël en un seul jour la tête et la queue, la branche de palmier et le roseau.
\VS{14}L'ancien et le magistrat, c'est la tête ; et le prophète qui enseigne le mensonge, c'est la queue.
\VS{15}Ceux donc qui font croire à ce peuple qu'il est heureux sont des séducteurs\FTNT{1 Ti. 4:1 ; Tit. 1:10.}; et ceux qui se laissent diriger par eux se perdent.
\VS{16}C'est pourquoi le Seigneur ne saurait prendre plaisir à leurs jeunes hommes ni avoir pitié de leurs orphelins et de leurs veuves, car tous sont des hypocrites et des méchants, et toute bouche ne profère que des infamies. Malgré tout cela, sa colère ne s'apaise point et sa main est encore étendue.
\VS{17}Car la méchanceté consume comme un feu, elle dévore les ronces et les épines ; elle embrase l'épaisseur de la forêt d'où s'élèvent des colonnes de fumée.
\VS{18}A cause de la fureur de Yahweh des armées, la terre est obscurcie et le peuple est comme la proie du feu ; nul n'a compassion de son frère.
\VS{19}On pille à droite, et l'on a faim ; on dévore à gauche, et l'on n'est pas rassasié ; chacun mange la chair de son bras.
\VS{20}Manassé dévore Ephraïm, Ephraïm dévore Manassé, et ensemble ils fondent sur Juda. Malgré tout cela, sa colère ne s'apaise point, et sa main est encore étendue.
\Chap{10}
\VerseOne{}Malheur à ceux qui font des ordonnances iniques, et à ceux qui écrivent pour ordonner l'oppression,
\VS{2}pour refuser la justice aux pauvres et ravir leur droit aux malheureux de mon peuple, afin d'avoir les veuves pour leur butin, et de piller les orphelins !
\VS{3}Et que ferez-vous au jour de la visitation, et de la ruine éclatante qui viendra de loin ? Vers qui fuirez-vous pour avoir du secours et où laisserez-vous votre gloire\FTNT{Os. 9:7 ; Mt. 24:17-21 ; Lu. 19:41-44.} ?
\VS{4}Les uns seront courbés parmi les prisonniers, les autres tomberont parmi les morts. Malgré tout cela, sa colère ne s'apaise point, et sa main est encore étendue.
\TextTitle{Jugement sur l'Assyrie}
\VS{5}Malheur à l'Assyrie, verge de ma colère ! Le verge dans leur main c'est l'instrument de ma colère.
\VS{6}Je l'ai envoyé contre une nation impie, et je l'ai fait marcher contre le peuple de ma fureur, afin qu'il se livre au pillage et fasse du butin, pour qu'il le foule aux pieds comme la boue des rues.
\VS{7}Mais il n'en juge pas ainsi, et ce n'est pas là la pensée de son cœur ; mais il ne songe qu'à détruire, qu'à exterminer beaucoup de nations.
\VS{8}Car il dit : Mes princes ne sont-ils pas autant de rois ?
\VS{9}Calno n'est-elle pas comme Carkemisch ? Hamath n'est-elle pas comme Arpad ? Et Samarie n'est-elle pas comme Damas ?
\VS{10}Puisque ma main a soumis les royaumes qui avaient des idoles, où il y avait plus d'images taillées qu'à Jérusalem et à Samarie,
\VS{11}ne ferai-je pas aussi à Jérusalem et à ses dieux, comme j'ai fait à Samarie et à ses idoles ?
\VS{12}Mais il arrivera que quand le Seigneur aura achevé toute son œuvre sur la montagne de Sion et à Jérusalem, je punirai le roi d'Assyrie pour le fruit de son cœur orgueilleux et pour l'arrogance de ses regards hautains.
\VS{13}Parce qu'il dit : C'est par la force de ma main que j'ai agi, c'est par ma sagesse, car je suis intelligent ; j'ai reculé les bornes des peuples et j'ai pillé ce qu'ils avaient de plus précieux ; et comme un homme vaillant, j'ai fait descendre ceux qui étaient assis.
\VS{14}Ma main a trouvé les richesses des peuples comme on trouve un nid ; comme on rassemble des œufs délaissés, ainsi ai-je rassemblé toute la terre ; nul n'a remué l'aile, ni ouvert le bec, ni poussé un cri.
\VS{15}La hache se glorifie-t-elle envers celui qui s'en sert ? Ou la scie s'élève-t-elle au-dessus de celui qui la manie ? Comme si la verge faisait mouvoir celui qui la lève, et que le bâton se levait comme s'il n'était pas du bois.
\VS{16}C'est pourquoi le Seigneur, Yahweh des armées, enverra la maigreur sur ses hommes gras ; et sous sa gloire éclatera l'embrasement d'un feu.
\VS{17}Car la lumière d'Israël deviendra un feu, et son Saint une flamme qui embrasera et consumera ses épines et ses ronces tout en un jour ;
\VS{18}Et il consumera la gloire de sa forêt et de ses campagnes, depuis l'âme jusqu'à la chair. Il en sera comme quand celui qui porte l'étendard est défait.
\VS{19}Le reste des arbres de sa forêt pourra être compté, et un enfant en écrirait le nombre.
\TextTitle{Conversion et délivrance du reste d'Israël}
\VS{20}Et il arrivera en ce jour-là que le reste d'Israël et les réchappés de la maison de Jacob ne s'appuieront plus sur celui qui les frappait, mais ils s'appuieront avec confiance sur Yahweh, le Saint d'Israël.
\VS{21}Le reste se convertira, le reste de Jacob se convertira au Dieu puissant.
\VS{22}Car quand ton peuple, ô Israël, serait comme le sable de la mer, un reste seulement se convertira ; la destruction est résolue, elle fera déborder la justice.
\VS{23}Car la destruction qu'il a résolue, le Seigneur Yahweh des armées va l'exécuter au milieu de toute la terre.
\VS{24}C'est pourquoi ainsi parle le Seigneur, Yahweh des armées : Mon peuple qui habites en Sion, ne crains pas le roi d'Assyrie ; il te frappe de la verge et il lève son bâton sur toi comme faisait l'Egypte.
\VS{25}Mais encore un peu de temps, un peu de temps, et le châtiment cessera, puis ma colère se tournera contre lui pour l'exterminer.
\VS{26}Et Yahweh des armées lèvera le fouet contre lui, comme il frappa Madian au rocher d'Oreb ; et de même qu'il leva son bâton sur la mer, il le lèvera aussi comme contre les Egyptiens.
\VS{27}En ce jour-là, son fardeau sera ôté de dessus ton épaule et son joug de dessus ton cou ; et l'onction fera rompre le joug.
\TextTitle{Défaite des Assyriens\FTNTT{Es. 35-36 ; 37.7.}}
\VS{28}Il marche sur Ajjath, traverse Migron et il met ses bagages à Micmasch.
\VS{29}Ils passent le défilé, ils couchent à Guéba ; Rama est effrayée ; Guibea de Saül prend la fuite.
\VS{30}Pousse des cris, fille de Gallim ! Malheur à toi Anathoth ! Prends garde Laïs !
\VS{31}Madména se disperse, les habitants de Guébim se sauvent en foule.
\VS{32}Encore un jour d'arrêt à Nob, et il menace de sa main la montagne de la fille de Sion, la colline de Jérusalem.
\VS{33}Voici, le Seigneur, Yahweh des armées, brise les rameaux avec force ; et ceux qui sont les plus hauts élevés sont coupés, et les haut montés sont abaissés.
\VS{34}Et il taille avec le fer les lieux les plus épaix de la forêt, et le Liban tombe sous le Puissant.
\Chap{11}
\TextTitle{Rétablissement du règne de David par le Messie}
\VerseOne{}Mais il sortira un rameau du tronc d'Isaï, et un rejeton naîtra de ses racines\FTNT{Mt. 1:6-16 ; Lu. 1:31-32 ; Ro. 15:12 ; Ap. 5:5.}.
\VS{2}L'Esprit de Yahweh reposera sur lui, Esprit de sagesse et d'intelligence, Esprit de conseil et de force, Esprit de connaissance et de crainte de Yahweh\FTNT{Es. 61:1 ; Lu. 4:18}.
\VS{3}Il respirera la crainte de Yahweh, il ne jugera point sur l'apparence et il ne reprendra point sur un ouï-dire\FTNT{Jé. 11:20 ; Mt. 22:16 ; Ap. 2:23.}.
\VS{4}Mais il jugera les pauvres avec justice, et il prononcera avec droiture un jugement sur les malheureux de la terre, et il frappera la terre par la verge de sa bouche, et fera mourir le méchant par le souffle de ses lèvres\FTNT{Job. 4:9 ; Job. 15:30 ; 2 Thess. 2:8.}.
\VS{5}La justice sera la ceinture de ses reins, et la fidélité, la ceinture de ses flancs\FTNT{Ep. 6:14.}.
\VS{6}Le loup habitera avec l'agneau, et le léopard se couchera avec le chevreau ; le veau, et le lionceau, et le bétail qu'on engraisse seront ensemble, et un petit enfant les conduira.
\VS{7}La jeune vache paîtra avec l'ourse, leurs petits auront un même gîte, et le lion comme le bœuf mangera de la paille\FTNT{Es. 65:25.}.
\VS{8}Le nourrisson s'ébattra sur l'antre de l'aspic, et l'enfant sevré mettra sa main dans la caverne du basilic.
\VS{9}Il ne se fera ni tort ni dommage sur toute ma montagne sainte, car la terre sera remplie de la connaissance de Yahweh, comme le fond de la mer des eaux qui le couvrent.
\VS{10}En ce jour-là, les nations rechercheront le rejeton d'Isaï qui sera comme une bannière\FTNT{Jésus, notre bannière doit être élevé afin que les pécheurs soient sauvés (Jn. 3:14-15). Le livre du Cantique des cantiques est une superbe image de l'amour de Dieu manifesté en Jésus-Christ, pour nous son Église, qui sommes sa bien-aimée. « Il m'a fait entrer dans la maison du vin ; et la bannière qu'il déploie sur moi, c'est l'amour » (Ca. 2:4). Le vin dont il est question c'est le Saint-Esprit que le Seigneur déverse sur nous jour après jour et qui nous désaltère spirituellement. « Moïse bâtit un autel, et lui donna pour nom : Yahweh ma bannière » (Ex. 17:15). Autrefois, lors des combats, les différentes armées portaient bien haut leur bannière en tête des troupes pour témoigner de leur appartenance et pour indiquer pour quel pays elles combattaient. En portant en nous Jésus, nous proclamons notre appartenance à Dieu et à son Royaume. Le Ps. 20:6 dit ceci : « Nous nous réjouirons de ton salut, nous lèverons l'étendard au nom de notre Dieu ; Yahweh exaucera tous tes vœux ». Et dans le Ps. 60:6 il est dit : « Tu as donné à ceux qui te craignent une bannière pour qu'elle s'élève à cause de la vérité ». La vérité se trouve en Jésus-Christ qui est le chemin, la vérité et la vie (Jn. 14:6). En élevant Jésus comme notre bannière, nous proclamons l'œuvre parfaite accomplie à la croix. Jésus, notre bannière, est le point de rassemblement des chrétiens de tous horizons. Ce rassemblement forme le corps de Christ, l'Eglise dont nous sommes les membres. Tout comme les douze tribus d'Israël se réunissaient pour combattre, nous nous réunissons tous sous la même bannière. Jésus, notre bannière, est le signe de la victoire contre les puissances des ténèbres. En élevant le nom de Jésus comme une bannière, nous faisons fuir toute l'armée de Satan. Dans Jn. 12:32 le Seigneur nous dit : « Et moi, quand j'aurai été élevé de la terre, j'attirerai tous les hommes à moi ». Jésus a été élevé comme étant la bannière qui a réconcilié Dieu avec les pécheurs. Lorsque cette bannière est élevée, les pécheurs sont attirés vers Dieu, ils passent des ténèbres à la lumière, de la mort à la vie.} pour les peuples, et son séjour ne   sera que gloire.
\TextTitle{Etablissement du règne du Messie}
\VS{11}Et il arrivera en ce jour-là, que le Seigneur mettra encore sa main une seconde fois sa main pour acquérir le reste de son peuple, dispersé en Assyrie, en Egypte, à Pathros, en Ethiopie, à Elam, à Schinear, à Hamath et dans les îles de la mer.
\VS{12}Il élèvera une bannière parmi les nations, il rassemblera les exilés d'Israël qui auront été chassés, et il recueillera les dispersés de Juda des quatre extrémités de la terre.
\VS{13}Et la jalousie d'Ephraïm sera ôtée, et les oppresseurs de Juda seront retranchés ; Ephraïm ne sera plus jaloux de Juda, et Juda n'opprimera plus Ephraïm.
\VS{14}Mais ils voleront sur l'épaule des Philistins vers la mer ; ils pilleront ensemble les fils de l'orient ; Edom et Moab seront la proie de leurs mains et les filss d'Ammon leur obéiront.
\VS{15}Yahweh exterminera aussi à la façon de l'interdit la langue de la mer d'Egypte, et il lèvera sa main contre le fleuve par la force de son vent, et il le frappera sur les sept rivières, et fera qu'on y marchera avec des souliers.
\VS{16}Et il y aura un chemin pour le reste de son peuple, qui sera échappé de l'Assyrie, comme il y en eut un pour Israël le jour où il remonta du pays d'Egypte.
\Chap{12}
\TextTitle{Louange au sein du royaume}
\VerseOne{}Tu diras en ce jour-là : Je te loue, ô Yahweh ! Car tu as été irrité contre moi, ta colère s'est apaisée, et tu m'as consolé.
\VS{2}Voici, Dieu est ma délivrance, j'aurai confiance et je ne craindrai rien ; car Yahweh, Yahweh est ma force et ma louange ; il est mon Sauveur.
\VS{3}Et vous puiserez de l'eau avec joie aux sources du salut\FTNT{Jn. 4:10-14.},
\VS{4}et vous direz en ce jour-là : Louez Yahweh, invoquez son Nom, publiez ses œuvres parmi les peuples, rappelez que son Nom est une haute retraite !
\VS{5}Psalmodiez à Yahweh car il a fait des choses magnifiques : Cela est connu dans toute la terre !
\VS{6}Habitante de Sion, égaye-toi, et réjouis-toi avec chant de triomphe ! Car le Saint d'Israël est grand au milieu de toi.
\Chap{13}
\TextTitle{Yahweh lève une armée}
\VerseOne{}Oracle sur Babylone, révélé à Esaïe, fils d'Amots.
\VS{2}Elevez la bannière sur la haute montagne, élevez la voix vers eux, faites des signes avec la main, et qu'on entre dans les portes des magnifiques !
\VS{3}C'est moi qui ai donné des ordres à ceux qui me sont consacrés, j'ai appelé mes hommes forts pour exécuter ma colère, ceux qui se réjouissent de ma grandeur.
\VS{4}Il y a sur les montagnes un bruit d'une multitude, comme celui d'un grand peuple ; on entend un tumulte de royaumes, de nations rassemblées : Yahweh des armées passe en revue l'armée pour le combat.
\VS{5}D'un pays éloigné, de l'extrémité des cieux, Yahweh vient avec les instruments de sa colère pour détruire tout le pays.
\TextTitle{Jugement de Yahweh sur Babylone}
\VS{6}Hurlez, car le jour de Yahweh est proche, il vient comme un ravage du Tout-Puissant.
\VS{7}C'est pourquoi toutes les mains deviennent lâches et tout cœur d'homme se fond.
\VS{8}Ils sont épouvantés, les détresses et les douleurs les saisissent ; ils sont en travail comme celle qui enfante ; ils se regardent les uns les autres avec stupeur, leurs visages sont comme des visages enflammés.
\VS{9}Voici, le jour de Yahweh arrive, jour cruel, jour de colère et d'ardente fureur\FTNT{Mal. 4:1 ; Ap. 19:15}, qui réduira le pays en désolation, et en exterminera les pécheurs.
\VS{10}Même les étoiles des cieux et leurs astres ne feront plus briller leur lumière ; le soleil s'obscurcira dès son lever, et la lune ne fera plus resplendir sa lueur\FTNT{Joë. 2:31 ; Mt. 24:29 ; Mc. 13:24.}.
\VS{11}Je punirai le monde habitable à cause de sa malice, et les méchants à cause de leur iniquité ; je ferai cesser l'arrogance des hautains et j'abaisserai l'arrogance des tyrans.
\VS{12}Je ferai qu'un homme sera plus précieux que l'or fin, et une personne plus que l'or d'Ophir.
\VS{13}C'est pourquoi j'ébranlerai les cieux, et la terre sera secouée de sa base\FTNT{Ag. 2:6} à cause de la fureur de Yahweh des armées, et à cause du jour de son ardente colère.
\VS{14}Et chacun sera comme un chevreuil qui est chassé et comme une brebis que personne ne retire, chacun se tournera vers son peuple, chacun fuira vers son pays.
\VS{15}Quiconque sera trouvé, sera transpercé ; et quiconque s'y sera joint, tombera par l'épée.
\VS{16}Et leurs petits enfants seront écrasés sous leurs yeux\FTNT{Na. 3:10.}, leurs maisons seront pillées et leurs femmes violées.
\TextTitle{Yahweh envoie les Mèdes contre Babylone}
\VS{17}Voici, je vais susciter contre eux les Mèdes, qui ne font point cas de l'argent et qui ne convoitent point l'or.
\VS{18}Leurs arcs écraseront les jeunes gens et ils seront sans pitié pour le fruit des entrailles, leur œil n'épargnera point les enfants.
\VS{19}Ainsi Babylone, l'ornement des royaumes, la parure et l'orgueil des Chaldéens, sera comme Sodome et Gomorrhe que Dieu détruisit.
\VS{20}Elle ne sera plus jamais habitée, elle ne sera point habitée de génération en génération ; même les Arabes n'y dresseront point leurs tentes, et les bergers n'y feront plus reposer leurs troupeaux.
\VS{21}Mais les bêtes sauvages des déserts y prendront leur gîte, et les hiboux rempliront ses maisons, les autruches en feront leur demeure, et les boucs y sauteront.
\VS{22}Les chacals hurleront dans ses palais, et les dragons dans ses maisons de plaisance. Son temps est près d'arriver et ses jours ne se prolongeront pas.
\Chap{14}
\TextTitle{Chant d'Israël après la chute de Babylone}
\VerseOne{}Car Yahweh aura pitié de Jacob, il choisira encore Israël, et il les rétablira dans leur terre ; les étrangers se joindront à eux et s'attacheront à la maison de Jacob.
\VS{2}Et les peuples les prendront, et les ramèneront à leur demeure, et la maison d'Israël les possédera en droit d'héritage sur la terre de Yahweh, comme serviteurs et comme servantes ; ils retiendront captifs ceux qui les avaient tenus captifs, et ils domineront sur leurs oppresseurs.
\VS{3}Et il arrivera qu'au jour que Yahweh fera cesser ton travail, ton tourment, et la dure servitude qui te fut imposée,
\VS{4}alors tu prononceras ce proverbe sur le roi de Babylone, et tu diras : Comment a-t-il fini le tyran ? Comment se repose celle qui était si avide de richesses ?
\VS{5}Yahweh a brisé le bâton des méchants et la verge des dominateurs.
\VS{6}Celui qui frappait avec fureur les peuples de coups qu'on ne pouvait point détourner, qui dominait sur les nations avec colère, est poursuivi sans ménagement.
\VS{7}Toute la terre jouit du repos et de la paix ; on éclate en chants de triomphe à gorge déployée.
\VS{8}Même les cyprès et les cèdres du Liban se réjouissent de toi en disant : Depuis que tu es tombé, personne n'est monté pour nous abattre.
\TextTitle{Le roi de Babylone dépouillé de sa gloire}
\VS{9}Le scheol s'émeut jusque dans ses profondeurs, pour t'accueillir à ton arrivée ; il réveille à cause de toi les morts, et il fait lever de leurs sièges tous les principaux de la terre.
\VS{10}Tous prennent la parole pour te dire : Toi aussi, tu es sans force comme nous, tu es devenu semblable à nous !
\VS{11}Ta hauteur est descendue dans le scheol, avec le son de tes luths ; tu es couché sur une couche de vers, et la vermine est ta couverture.
\TextTitle{Orgueil, rébellion et chute de Satan}
\VS{12}Comment es-tu tombée du ciel, astre brillant, fils de l'aurore ? Toi qui foulais les nations, tu es abattu jusqu'à terre !
\VS{13}Tu disais en ton cœur : Je monterai aux cieux, je placerai mon trône au-dessus des étoiles de Dieu ; je m'assiérai sur la montagne de l'assemblée, du côté nord ;
\VS{14}je monterai au dessus des hauts lieux des nuées, je serai semblable au Très-Haut.
\VS{15}Et cependant tu as été précipité dans le scheol, dans les profondeurs de la fosse\FTNT{Voir commentaire Ge. 1:1-2.}.
\VS{16}Ceux qui te voient fixent sur toi leurs regards, ils te considèrent attentivement en disant : N'est-ce pas celui qui faisait trembler la terre, qui ébranlait les royaumes,
\VS{17}qui réduisait le monde habitable en désert, qui détruisait les villes, et ne relâchait pas ses prisonniers, ni ne les renvoyait chez eux ?
\TextTitle{Babylone anéantie}
\VS{18}Tous les rois des nations, oui, tous, reposent avec honneur, chacun dans sa maison.
\VS{19}Mais toi, tu as été jeté loin de ton sépulcre, comme un rejeton pourri, comme une dépouille de gens tués, transpercés avec l'épée, qu'on jette sous les pierres d'une fosse, comme un cadavre foulé aux pieds.
\VS{20}Tu ne seras point rangé comme eux dans le sépulcre, car tu as ravagé ta terre, tu as tué ton peuple. La race des méchants ne sera point renommée à toujours.
\VS{21}Préparez la tuerie pour ses enfants, à cause de l'iniquité de leurs pères ; afin qu'ils ne se relèvent point, et qu'ils n'héritent point la terre et ne remplissent point de villes le dessus de la terre habitable.
\VS{22}Je m'élèverai contre eux, dit Yahweh des armées, et je retrancherai à Babylone le nom et le reste qu'elle a, ses descendants et sa postérité\FTNT{Ap. 14:8 ; Ap. 18:2.}, dit Yahweh.
\VS{23}J'en ferai l'habitation du butor et un marécage, et je la balayerai avec le balai de la destruction, dit Yahweh des armées.
\TextTitle{Jugement sur le roi d'Assyrie}
\VS{24}Yahweh des armées l'a juré, en disant : Certainement ce que j'ai décidé arrivera, ce que j'ai résolu s'accomplira.
\VS{25}Je briserai le roi d'Assyrie dans ma terre, je le foulerai aux pieds sur mes montagnes ; et son joug leur sera ôté, et son fardeau sera ôté de dessus leurs épaules.
\VS{26}C'est là le conseil arrêté contre toute la terre, c'est là la main étendue sur toutes les nations.
\TextTitle{Jugement sur le pays des Philistins}
\VS{27}Car Yahweh des armées a arrêté en son conseil : Qui l'empêchera ? Sa main est étendue : Qui la détournera\FTNT{Ec. 7:13.} ?
\VS{28}L'année de la mort du roi Achaz, cet oracle fut prononcé : 29 Ne te réjouis pas, toi pays des Philistins, de ce que la verge de celui qui te frappait est brisée ! Car de la racine du serpent sortira un basilic, et son fruit sera un serpent brûlant qui vole.
\VS{30}Alors les plus misérables seront repus, et les pauvres reposeront en assurance ; mais je ferai mourir de faim ta racine, et ce qui restera de toi sera tué.
\VS{31}Porte, hurle ! Ville, crie ! Tremble, pays tout entier des Philistins ! Car du nord, vient une fumée, et il ne restera pas un homme dans des habitations.
\VS{32}Et que répondra-t-on aux envoyés de cette nation ? On répondra que Yahweh a fondé Sion, et que les affligés de son peuple y trouvent un refuge.
\Chap{15}
\TextTitle{Jugement sur Moab}
\VerseOne{}Oracle sur Moab. La nuit même où elle est ravagée, Ar-Moab est détruite ! La nuit même où elle est saccagée, Kir-Moab est détruite !
\VS{2}Il monte à Bajith et à Dibon, dans les hauts lieux, pour pleurer ; Moab est en lamentations sur Nebo et sur Médeba : Toutes les têtes sont rasées et toutes les barbes sont coupées.
\VS{3}On sera couvert de sacs dans les rues ; chacun hurle fondant en larmes sur ses toits et dans ses places\FTNT{Jé. 48:38.}.
\VS{4}Hesbon et Elealé poussent des cris, on entend leur voix jusqu'à Jahats ; c'est pourquoi les guerriers de Moab se lamentent, ils ont l'effroi dans l'âme.
\VS{5}Mon cœur crie à cause de Moab, dont les fugitifs s'enfuient jusqu'à Tsoar, comme une génisse de trois ans ; car ils montent par la montée de Luchith avec des pleurs, et ils jettent des cris de détresse sur le chemin de Choronaïm.
\VS{6}Même les eaux de Nimrim ne sont que désolations, même le foin est déjà séché, l'herbe est consumée, et il n'y a point de verdure.
\VS{7}C'est pourquoi ils surveillent les richesses abondantes qu'ils ont acquises, afin que ce qu'ils ont réservé soit porté dans la vallée des saules.
\VS{8}Car les cris environnent les frontières de Moab, ses lamentations retentissent jusqu'à Eglaïm, ses lamentations retentissent jusqu'à Beer-Elim.
\VS{9}Même les eaux de Dimon sont pleines de sang ; car j'ajouterai un surcroît sur Dimon : Des lions contre les réchappés de Moab, et le reste du pays.
\Chap{16}
\TextTitle{Moab recherche un refuge}
\VerseOne{}Envoyez l'agneaux au souverain du pays, envoyez-le du rocher du désert, à la montagne de la fille de Sion.
\VS{2}Car il arrivera que les filles de Moab seront au passage de l'Arnon, comme un oiseau volant ça et là, comme une nichée chassée de son nid .
\VS{3}Mets en avant le conseil, fais l'ordonnance, sers d'ombre comme une nuit au milieu de midi ; cache ceux qui ont été chassés, et ne trahis pas ceux qui sont errants.
\VS{4}Que ceux de mon peuple qui ont été chassés séjournent chez toi, ô Moab ! Sois pour eux un refuge contre le dévastateur ! Car celui qui use d'extorsion cessera, la dévastation finira, celui qui foule le pays sera consumé de dessus la terre.
\VS{5}Et le trône s'affermira par la clémence ; et sur ce trône sera assis en vérité, dans le tabernacle de David, un juge recherchant le droit, et se hâtant de faire justice\FTNT{Mi. 4:7 ; Da. 7:14 ; Lu. 1:33 ; Ap. 11:15}.
\VS{6}Nous avons entendu l'orgueil de Moab, le peuple extrêmement orgueilleux, sa fierté, son orgueil, son arrogance et ses vains discours.
\VS{7}C'est pourquoi Moab gémit sur Moab, chacun gémit ; vous soupirez pour les fondements de Kir-Haréseth, il n'y aura que des gens blessés à mort.
\VS{8}Car les campagnes de Hesbon et le vignoble de Sibma languissent ; les maîtres des nations ont foulé ses meilleurs ceps, qui s'étendaient jusqu'à Jaezer, qui couraient ça et là par le désert ; ses rameaux s'étendaient et passaient au-delà de la mer.
\VS{9}C'est pourquoi je pleure sur la vigne de Sibma, comme sur Jaezer ; je vous arrose de mes larmes, ô Hesbon et Elealé ! Car l'ennemi avec des cris s'est jeté sur tes fruits d'été et sur ta moisson.
\VS{10}Et la joie et l'allégresse s'est retirée du champ fertile ; on ne se réjouit plus et on ne s'égaye plus dans les vignes, le vendangeur ne foule plus dans les cuves, j'ai fait cesser la chanson de la vendange\FTNT{Jé. 48:31-34.}.
\VS{11}C'est pourquoi mes entrailles gémissent sur Moab, comme une harpe, et mon intérieur sur Kir-Harès.
\VS{12}Et on voit Moab qui se fatigue sur les hauts lieux ; il entre dans son sanctuaire pour prier mais il ne peut rien obtenir.
\VS{13}Telle est la parole que Yahweh a prononcée depuis longtemps sur Moab.
\VS{14}Et maintenant Yahweh a parlé, en disant : Dans trois ans, comme les années d'un mercenaire, la gloire de Moab sera avilie, avec toute cette grande multitude ; et le reste sera petit, ce sera peu de chose, ce ne sera rien de considérable.
\Chap{17}
\TextTitle{Oracle sur la chute de Damas et de ses alliés}
\VerseOne{}Oracle sur Damas. Voici, Damas est détruite pour ne plus être une ville, et elle ne sera qu'un monceau de ruines\FTNT{Jé. 49:23-27.}.
\VS{2}Les villes d'Aroër sont abandonnées, elles sont livrées aux troupeaux qui s'y reposent, et il n'y a personne qui les effraie.
\VS{3}Il n'y aura plus de forteresse en Ephraïm, ni de royaume à Damas et dans le reste de la Syrie ; ils seront comme la gloire des enfants d'Israël, dit Yahweh des armées.
\VS{4}Et il arrivera en ce jour-là que la gloire de Jacob sera affaiblie et la graisse de sa chair sera fondue.
\VS{5}Il en sera comme quand le moissonneur cueille les blés, et qu'il moissonne les épis avec son bras\FTNT{Joë. 3:13 ; Mt. 13:24-30.} ; comme quand on ramasse les épis dans la vallée de Rephaïm.
\VS{6}Mais il en restera quelques grappillages, comme quand on secoue l'olivier, et qu'il reste deux ou trois olives en haut de la cime, et qu'il y en a quatre ou cinq que l'olivier a produites dans ses branches fruitières, dit Yahweh, le Dieu d'Israël.
\VS{7}En ce jour-là, l'homme regardera vers celui qui l'a fait, et ses yeux se tourneront vers le Saint d'Israël.
\VS{8}Et il ne regardera plus vers les autels, qui sont l'ouvrage de ses mains, et il ne regardera plus ce que ses doigts ont fabriqué, ni les images d'Asherah, ni les statues du soleil.
\VS{9}En ce jour-là, ses villes fortes seront abandonnés à cause des enfants d'Israël, ils seront comme un bois taillis et des rameaux abandonnés, et ce sera un désert.
\VS{10}Parce que tu as oublié le Dieu de ton salut, et que tu ne t'es pas souvenue du rocher\FTNT{Voir commentaire Es. 8:13-14.} de ta force, à cause de cela tu as transplanté des plantes de plaisance, et tu as planté des ceps étrangers.
\VS{11}De jour tu as fais croître ce que tu as planté, et le matin tu as fait levé ta semence; mais la moisson a été enlevée au jour que l'on voulait en jouir, et il y a eu une douleur désespérée.
\VS{12}Malheur à la multitude de peuples nombreux, qui font un bruit comme le bruit des mers ; et à la tempête éclatante des nations, qui font du bruit comme une tempête éclatante d'eaux impétueuses !
\VS{13}Les nations font un bruit comme une tempête éclatante de grosses eaux, mais il les menace et elles s'enfuient ; elles seront poursuivies comme la balle des montagnes chassée par le vent, et comme une boule poussée par un tourbillon.
\VS{14}Au temps du soir, voici une terreur soudaine ; mais avant le matin, ils ne sont plus ! C'est là le partage de ceux qui nous dépouillent, et le lot de ceux qui nous pillent.
\Chap{18}
\TextTitle{Jugement sur l'Ethiopie}
\VerseOne{}Malheur à la terre qui fait ombre avec des ailes, qui est au-delà des fleuves de l'Ethiopie ;
\VS{2}qui envoie par mer des messagers, dans des navires de jonc, voguant à la surface des eaux ! Allez, messagers rapides, vers la nation robuste et vigoureuse, vers le peuple redoutable, depuis là où il est et par delà ; nation puissante et qui écrase tout, et dont les fleuves ravagent son pays.
\VS{3}Vous tous, habitants du monde, et vous qui habitez dans le pays, quand la bannière sera élevée sur les montagnes, regardez ; et quand le shofar sonnera, écoutez !
\VS{4}Car ainsi m'a parlé Yahweh : Je me tiens tranquillement et je regarde de ma demeure, par la chaleur de la lumière, et par la vapeur de la rosée, au temps de la chaude moisson.
\VS{5}Car avant la moisson, quand le bouton vient en sa perfection, et que la fleur devient un raisin qui mûrit, il coupe les sarments avec des serpes, il enlève les sarments, les ayant retranchés.
\VS{6}Ils seront tous ensemble abandonnés aux oiseaux de proie qui demeurent dans les montagnes, et aux bêtes de la terre ; les oiseaux de proie seront sur eux tout le long de l'été, et toutes les bêtes de la terre y passeront l'hiver.
\VS{7}En ce temps-là, un présent sera apporté à Yahweh des armées, par le peuple robuste et vigoureux, par le peuple terrible depuis là où il est et par delà, nation puissante et qui écrase tout, et dont le pays est ravagé par ses fleuves ; il sera apporté dans la demeure du Nom de Yahweh des armées, sur la montagne de Sion.
\Chap{19}
\TextTitle{Chute de l'Egypte}
\VerseOne{}Oracle sur l'Egypte. Voici, Yahweh est monté sur une nuée rapide, il entre en Egypte ; et les idoles d'Egypte s'enfuient de toutes parts devant sa face, et le cœur des Egyptiens se fond au milieu d'elle\FTNT{Jé. 43:12.}.
\VS{2}Et je ferai venir pêle-mêle l'Egyptien contre l'Egyptien, et chacun fera la guerre contre son frère, et chacun contre son ami, ville contre ville, et royaume contre royaume.
\VS{3}L'esprit de l'Egypte disparaîtra du milieu d'elle et je dissiperai son conseil ; et ils consulteront les idoles et les enchanteurs, ceux qui évoquent les morts et ceux qui prédisent l'avenir.
\VS{4}Et je livrerai l'Egypte entre les mains d'un maître sévère ; et un roi cruel dominera sur eux, dit le Seigneur, Yahweh des armées.
\VS{5}Les eaux de la mer tariront, le fleuve séchera et tarira\FTNT{Jé. 51:36.}.
\VS{6}Et on fera détourner les fleuves ; les ruisseaux des digues s'abaisseront et sécheront ; les roseaux et les joncs seront coupés.
\VS{7}Les prairies qui sont près des ruisseaux, et sur l'embouchure du fleuve, tout ce qui aura été semé le long des ruisseaux, séchera, sera jeté au loin, et ne sera plus.
\VS{8}Et les pêcheurs gémiront, tous ceux qui jettent l'hameçon dans le fleuve mèneront deuil, et ceux qui étendent des filets sur les eaux languiront.
\VS{9}Ceux qui travaillent en fin lin et en fin crêpe, et ceux qui tissent les filets seront confus.
\VS{10}Les soutiens du pays seront brisés, tous les mercenaires auront l'âme attristée.
\VS{11}Certes les princes de Tsoan ne sont que des insensés, les sages conseillers de Pharaon forment un conseil stupide. Comment osez-vous dire à Pharaon : Je suis fils des sages, fils des anciens rois ?
\VS{12}Où sont-ils maintenant? Où sont tes sages ? Qu'ils t'annoncent, je te prie, s'ils le savent, ce que Yahweh des armées a décrété contre l'Egypte.
\VS{13}Les princes de Tsoan sont devenus insensés, les princes de Noph se sont trompés, les chefs des tribus font égarer l'Egypte.
\VS{14}Yahweh a versé au milieu d'elle un esprit de vertige\FTNT{1 R. 22:18-22.}, pour qu'ils fassent chanceler les Egyptiens dans toutes leurs actions, comme un homme ivre se vautre dans son vomissement.
\VS{15}Et l'Egypte sera hors d'état de faire ce que font la tête et la queue, la branche de palmier et le roseau.
\TextTitle{L'Egypte et l'Assyrie dans le royaume du Messie}
\VS{16}En ce jour-là, l'Egypte sera comme des femmes : Elle sera étonnée et épouvantée à cause de la main de Yahweh des armées, quand il élèvera la main contre elle.
\VS{17}Et la terre de Juda sera pour l'Egypte un objet d'effroi ; quiconque fera mention d'elle, en sera épouvanté en lui-même, à cause du conseil décrété contre elle par Yahweh des armées.
\VS{18}En ce jour-là, il y aura cinq villes au pays d'Egypte qui parleront la langue de Canaan, et qui jureront par Yahweh des armées : L'une sera appelée ville de la destruction.
\VS{19}En ce jour-là, il y aura un autel à Yahweh au milieu du pays d'Egypte et un monument dressé à Yahweh sur la frontière.
\VS{20}Et ce sera un signe et un témoignage pour Yahweh des armées dans le pays d'Egypte ; car ils crieront à Yahweh à cause des oppresseurs, et il leur enverra un sauveur, quelqu'un de grand, et il les délivrera\FTNT{Es. 43:11.}.
\VS{21}Et Yahweh se fera connaître aux Egyptiens, et les Egyptiens connaîtront Yahweh en ce jour-là ; ils le serviront, ils offriront des sacrifices et des offrandes, et ils feront des vœux à Yahweh et les accompliront.
\VS{22}Ainsi Yahweh frappera les Egyptiens, il les frappera, mais il les guérira ; et ils retourneront à Yahweh qui les exaucera et les guérira.
\VS{23}En ce jour-là, il y aura un chemin battu de l'Egypte en Assyrie ; et l'Assyrie viendra en Egypte, et l'Egypte en Assyrie, et l'Egypte servira avec l'Assyrie.
\VS{24}En ce même temps, Israël sera, lui troisième, uni à l'Egypte et à l'Assyrie, et la bénédiction sera au milieu de la terre.
\VS{25}Ce que Yahweh des armées bénira, en disant : Bénis soit l'Egypte mon peuple, et l'Assyrie œuvre de mes mains, et Israël mon héritage !
\Chap{20}
\TextTitle{Conquête de l'Egypte et de l'Ethiopie}
\VerseOne{}L'année où Tharthan, envoyé par Sargon, roi d'Assyrie, vint et combattit contre Asdod, et la prit.
\VS{2}En ce temps-là, Yahweh parla par le ministère d'Esaïe, fils d'Amots, et lui dit : Va, délie le sac de dessus tes reins et ôte tes souliers de tes pieds. Il fit ainsi, marchant nu et déchaussé.
\VS{3}Puis Yahweh dit : De même que mon serviteur Esaïe marche nu et déchaussé, ce qui sera dans trois ans un signe et un prodige contre l'Egypte et contre l'Ethiopie,
\VS{4}de même le roi d'Assyrie emmènera de l'Egypte et de l'Ethiopie prisonniers et captifs les jeunes et les vieux, nus et déchaussés, ayant les hanches découvertes, ce qui sera l'opprobre de l'Egypte\FTNT{2 S. 10:4 ; Es. 3:17 ; Jé. 13:22-26.}.
\VS{5}Ils seront effrayés, et ils seront honteux à cause de l'Ethiopie à qui ils s'attendaient, et à cause de l'Egypte dont ils se glorifiaient.
\VS{6}Et les habitants de cette côte diront en ce jour-là : Voilà ce qu'est devenu le peuple à qui nous nous attendions, celui vers qui nous courions chercher du secours, afin d'être délivrés du roi d'Assyrie ! Comment pourrons-nous échapper ?
\Chap{21}
\TextTitle{Annonce de la conquête de Babylone}
\VerseOne{}Oracle sur le désert de la mer. Il vient du désert, de la terre redoutable, comme s'avance l'ouragan du pays du midi.
\VS{2}Une vision terrible m'a été révélée. L'oppresseur opprime, le dévastateur dévaste. Monte, Elam ! Assiège, Médie ! Je fais cesser tous les soupirs.
\VS{3}C'est pourquoi mes reins sont remplis de douleur ; les angoisses me saisissent comme les douleurs de celle qui enfante ; je suis tourmenté à cause de ce que j'ai entendu, et j'ai été tout troublé à cause de ce que j'ai vu.
\VS{4}Mon cœur est troublé, la terreur s'empare de moi ; la nuit de mes plaisirs devient une nuit d'épouvante.
\VS{5}On dresse une table, la sentinelle veille, on mange, on boit. Levez-vous, princes ! Oignez le bouclier !
\VS{6}Car ainsi m'a parlé le Seigneur : Va, place la sentinelle, et qu'elle rapporte ce qu'elle verra\FTNT{Ez. 33:1-19.}.
\VS{7}Elle vit de la cavalerie, des cavaliers deux à deux, des cavaliers sur des ânes, des cavaliers sur des chameaux ; et elle était attentive, très attentive.
\VS{8}Puis elle s'écria comme un lion : Seigneur, je me tiens sur la tour toute la journée et je suis à mon poste toutes les nuits ;
\VS{9}Et voici, il vient de la cavalerie, des cavaliers deux à deux ! Elle prit encore la parole, et dit : Elle est tombée, elle est tombée, Babylone\FTNT{Prophétie sur la chute de Babylone. Voir Jé. 50 et 51 ; Ap.18.}, et toutes les images taillées de ses dieux sont brisées par terre.
\VS{10}C'est ce que j'ai foulé, et le grain que j'ai battu dans mon aire. Je vous ai annoncé ce que j'ai entendu de Yahweh des armées, du Dieu d'Israël.
\VS{11}Oracle sur Duma. On me crie de Séir : Ô sentinelle ! Qu'en est-il de la nuit ? Ô sentinelle ! Qu'en est-il de la nuit ?
\VS{12}La sentinelle répond : Le matin vient, la nuit aussi. Si vous voulez interroger, interrogez. Convertissez-vous et revenez.
\TextTitle{Jugement sur l'Arabie}
\VS{13}Oracle sur l'Arabie. Vous passerez pêle-mêle la nuit dans la forêt, caravanes de Dedan !
\VS{14}Portez de l'eau à ceux qui ont soif ; les habitants du pays de Théma portent du pain aux fugitifs.
\VS{15}Car ils fuient devant les épées, devant l'épée nue, devant l'arc tendu, devant une bataille acharnée.
\VS{16}Car ainsi m'a parlé le Seigneur : Encore une année, comme les années d'un mercenaire, et toute la gloire de Kédar prendra fin.
\VS{17}Il ne restera qu'un petit nombre des vaillants archers, fils des Kédar, car Yahweh le Dieu d'Israël l'a déclaré.
\Chap{22}
\TextTitle{Malédiction sur la vallée des visions, Jérusalem}
\VerseOne{}Oracle sur la vallée des visions. Qu'as-tu maintenant, que tu sois toute montée sur les toits ?
\VS{2}Ville bruyante, pleine de tumulte, ville joyeuse ! Tes morts ne périront pas par l'épée, ils ne mourront pas par la guerre.
\VS{3}Tous tes chefs fuient ensemble, ils sont faits prisonniers par les archers ; tous tes habitants qui sont trouvés sont faits prisonniers ensemble tandis qu'ils prennent au loin la fuite.
\VS{4}C'est pourquoi je dis : Détournez de moi les regards, que je pleure amèrement. N'insistez pas pour me consoler du désastre de la fille de mon peuple.
\VS{5}Car c'est un jour de trouble, d'oppression et de confusion\FTNT{Lam. 1:5 ; Lam. 2:2.}, envoyé par le Seigneur, Yahweh des armées, dans la vallée des visions. Il s'en va démolir la muraille et les cris retentissent jusqu'à la montagne.
\VS{6}Même Elam porte le carquois, avec des chars pleins d'hommes et des cavaliers ; Kir découvre le bouclier.
\VS{7}Tes plus belles vallées sont remplies de chars, et les cavaliers se rangent tous en bataille à tes portes.
\VS{8}Et on a ôté la couverture de Juda, et en ce jour là tu regardes les armures de la maison de la forêt.
\VS{9}Vous voyez les brèches nombreuses faites à la cité de David, et vous retenez les eaux de l'étang inférieur.
\VS{10}Vous comptez les maisons de Jérusalem, et vous démolissez les maisons pour fortifier la muraille.
\VS{11}Vous faites un réservoir d'eau entre les deux murailles, pour les eaux de l'ancien étang. Mais vous ne regardez pas vers celui qui a voulu ces choses, vous ne voyez pas celui qui les a préparées de loin.
\VS{12}Le Seigneur, Yahweh des armées, vous appelle en ce jour à pleurer et à vous frapper la poitrine, à vous raser la tête et à ceindre le sac\FTNT{Ez. 7:18 ; Joë 1:13}.
\VS{13}Et voici il y a de la joie et de l'allégresse ! On égorge des bœufs et l'on tue des moutons, on mange la viande et l'on boit du vin ; puis on dit : Mangeons et buvons, car demain nous mourrons\FTNT{Es. 56:12 ; 1 Co. 15:32.} !
\VS{14}Or Yahweh des armées a révélé dans mes oreilles, disant : certainement cette iniquité ne vous sera pas pardonnée jusqu'à ce que vous mouriez, a dit le Seigneur, Yahweh des armées.
\TextTitle{Eliakim succède à Schebna}
\VS{15}Ainsi parle le Seigneur, Yahweh des armées : Va, entre chez ce trésorier, chez Schebna, gouverneur du palais et dis-lui :
\VS{16}Qu'as-tu à faire ici,et qu'as-tu ici qui t'appartienne, que tu creuses ici un sépulcre ? Il se creuse un sépulcre en hauteur, il se taille une demeure dans le rocher.
\VS{17}Voici, ô homme ! Yahweh te jettera au loin avec un bras vigoureux ; il t'enveloppera entièrement.
\VS{18}Il te fera rouler, rouler comme une balle sur une terre large et spacieuse ; là tu mourras, là seront tes chars magnifiques, ô toi qui es la honte de la maison de ton Seigneur !
\VS{19}Je te chasserai de ton poste, Yahweh t'arrachera de ton service.
\VS{20}En ce jour-là, j'appellerai mon serviteur Eliakim, fils de Hilkija.
\VS{21}Je le revêtirai de ta tunique, je le ceindrai de ta ceinture, et je remettrai ton pouvoir entre ses mains, il sera un père pour les habitants de Jérusalem et pour la maison de Juda.
\VS{22}Je mettrai la clef de la maison de David sur son épaule ; quand il ouvrira, nul ne fermera ; quand il fermera, nul n'ouvrira\FTNT{La clé de David est le symbole de l'autorité du Messie (Es. 9:5 ; Mt. 28:18 ; Ap. 3:7-8)}.
\VS{23}Je l'enfoncerai comme un clou dans un lieu sûr, et il sera un trône de gloire pour la maison de son père.
\VS{24}Il sera le soutien de toute la gloire de la maison de son père, des rejetons grands et petits ; de tous les petits ustensiles, des bassins jusqu'à tous les instruments de musique.
\VS{25}En ce jour-là, dit Yahweh des armées, le clou enfoncé dans un lieu sûr sera enlevé, il sera abattu et tombera, et le fardeau qui était sur lui sera retranché, car Yahweh a parlé.
\Chap{23}
\TextTitle{Effondrement de Tyr}
\VerseOne{}Oracle sur Tyr. Lamentez-vous, navires de Tarsis ! Car elle est détruite, il n'y a plus de maisons, on n'y entre plus ! C'est du pays de Kittim que la nouvelle leur est venue.
\VS{2}Habitants de la côte, taisez-vous, toi qui étais remplie de marchands de Sidon et de ceux qui parcouraient la mer !
\VS{3}A travers les grandes eaux, les grains de Shichor, la moisson du Nil était pour elle son revenu ; elle était le marché des nations\FTNT{Ez. 27.}.
\VS{4}Sois confuse, ô Sidon ! Car ainsi parle la mer, la forteresse de la mer, en disant : Je n'ai point eu de douleurs, je n'ai point enfanté, je n'ai point nourri de jeunes gens ni élevé de vierges.
\VS{5}Quand les Egyptiens sauront la nouvelle, ils seront dans l'angoisse en apprenant la chute de Tyr.
\VS{6}Passez à Tarsis, lamentez-vous, habitants de la côte !
\VS{7}Est-ce là votre ville joyeuse ? Elle avait une origine antique et ses propres pieds la mènent séjourner dans un pays étranger.
\VS{8}Qui a pris cette résolution contre Tyr, la distributrice des couronnes, elle dont les marchands étaient des princes, dont les trafiquants étaient les grands de la terre\FTNT{Ap. 18:9-18.} ?
\VS{9}C'est Yahweh des armées qui a pris cette résolution, pour flétrir l'orgueil de toute la noblesse, et pour avilir les plus grands de la terre.
\VS{10}Parcours ton pays, comme un fleuve, ô fille de Tarsis ! Plus de ceinture qui te retienne !
\VS{11}Yahweh a étendu sa main sur la mer, il a fait trembler les royaumes ; il a ordonné la destruction des forteresses de Canaan.
\VS{12}Il a dit : Tu ne te livreras plus à la joie, vierge déshonorée, fille de Sidon ! Lève-toi, passe au pays de Kittim ! Même là, il n'y a pas de repos pour toi.
\VS{13}Vois le pays des Chaldéens ; ce peuple n'existait pas autrefois ; Assur\FTNT{Assur : Le second fils de Sem (Ge. 10:22). L'ancêtre des Assyriens.} l'a fondé pour les gens du désert ; ils élèvent des tours, ils renversent les palais de Tyr, ils les mettent en ruines.
\VS{14}Lamentez-vous, navires de Tarsis ! Car votre forteresse est détruite !
\VS{15}En ce jour-là, Tyr tombera dans l'oubli soixante-dix ans, selon les jours d'un roi. Mais au bout de soixante-dix ans\FTNT{Jé. 25 : 11-12.}, il en sera de Tyr comme de la femme prostituée dont parle la chanson :
\VS{16}Prends la harpe, fais le tour de la ville, ô prostituée qu'on oublie ! Joue bien, répète tes chants, afin qu'on se souvienne de toi.
\VS{17}Au bout de soixante-dix ans, Yahweh visitera Tyr, mais elle retournera au salaire de sa prostitution, et elle se prostituera à tous les royaumes de la terre, sur la face du sol.
\VS{18}Mais son gain et son salaire seront consacrés à Yahweh ; ils ne seront ni entassés ni conservés ; car son gain fournira pour ceux qui habitent dans la présence de Yahweh, une nourriture abondante et des vêtements magnifiques.
\Chap{24}
\TextTitle{Désastre après l'invasion babylonienne}
\VerseOne{}Voici, Yahweh dévaste la terre et la rend déserte, il en bouleverse la face et disperse les habitants\FTNT{Ge. 11:1-8.}.
\VS{2}Et il en est du sacrificateur comme du peuple, du maître comme de son serviteur, de la maîtresse comme de sa servante, du vendeur comme de l'acheteur, du prêteur comme de l'emprunteur, du créancier comme du débiteur.
\VS{3}La terre est entièrement dévastée et livrée au pillage, car Yahweh a prononcé cet arrêt.
\VS{4}La terre mène le deuil, elle est déchue ; la terre habitée est devenue languissante, elle est déchue ; les plus distingués du peuple de la terre sont languissants.
\VS{5}La terre était profanée par ses habitants qui marchent sur elle ; car ils ont transgressé les lois, ils ont changé les ordonnances et ont enfreint l'alliance éternelle\FTNT{Da. 7:25.}.
\VS{6}C'est pourquoi la malédiction dévore la terre, et ses habitants portent la peine de leurs crimes ; c'est pourquoi les habitants de la terre sont consumés et il n'en reste qu'un petit nombre.
\VS{7}Le vin excellent pleure, la vigne languit, tous ceux qui avaient le cœur joyeux soupirent.
\VS{8}La joie des tambours a cessé ; la gaîté bruyante a pris fin, la joie de la harpe a cessé.
\VS{9}On ne boit plus de vin en chantant ; les liqueurs fortes sont amères au buveur.
\VS{10}La ville informe est en ruines ; toutes les maisons sont fermées, on n'y entre plus.
\VS{11}On crie dans les rues parce que le vin manque ; toute réjouissance a disparu, l'allégresse est bannie de la terre.
\VS{12}La désolation est restée dans la ville et les portes abattues sont en ruines.
\VS{13}Car il en est au milieu de la terre et parmi les peuples, comme quand on secoue l'olivier, comme quand on grappille après la vendange.
\TextTitle{Un reste de rescapés célèbre Yahweh}
\VS{14}Ils élèvent leur voix, ils poussent des cris d'allégresse ; et des bords de la mer, ils célèbrent la majesté de Yahweh.
\VS{15}C'est pourquoi glorifiez Yahweh dans les lieux où brille la lumière, le Nom de Yahweh, le Dieu d'Israël, dans les îles de la mer.
\VS{16}De l'extrémité de la terre,nous entendons des cantiques : Gloire au Juste ; mais moi je dis : Je suis perdu ! Je suis perdu ! Malheur à moi ! Les pillards pillent, et les pillards ont les vêtements des perfides.
\TextTitle{Manifestation des jugements de Yahweh}
\VS{17}La terreur, la fosse et le piège sont sur toi, habitant de la terre.
\VS{18}Celui qui fuit devant les cris de la terreur tombe dans la fosse ; et celui qui remonte hors de la fosse se prend au filet ; car les écluses d'en haut s'ouvrent et les fondements de la terre tremblent.
\VS{19}La terre est entièrement déchirée, la terre se brise entièrement, la terre chancelle.
\VS{20}La terre chancelle comme un homme ivre, elle vacille comme une cabane ; son péché pèse sur elle, elle tombe et ne se relève plus.
\VS{21}En ce jour-là, Yahweh châtiera en haut, l'armée d'en haut, et sur la terre les rois de la terre.
\VS{22}Ils seront assemblés captifs dans une prison, ils seront enfermés dans des cachots, et après plusieurs jours ils seront châtiés.
\VS{23}La lune rougira et le soleil sera honteux quand Yahweh des armées règnera sur la montagne de Sion et à Jérusalem, resplendissant de gloire en présence de ses anciens\FTNT{Mt. 24:29-30 ; 2 Pi. 3:10-12 ; Ap. 6:12.}.
\Chap{25}
\TextTitle{Le royaume de Yahweh}
\VerseOne{}Ô Yahweh, tu es mon Dieu ; je t'exalterai, je célébrerai ton nom, car tu as fait des choses merveilleuses ; tes desseins conçus d'avance se sont fidèlement accomplis.
\VS{2}Car tu as réduit la ville en un monceau de pierres, la cité forte en ruines ; le palais des étrangers qui était dans la ville ne sera jamais rebâti.
\VS{3}C'est pourquoi les peuples puissants te glorifient, les villes des nations redoutables te craignent.
\VS{4}Tu as été un refuge pour le faible, un refuge pour le malheureux dans sa détresse, un refuge contre la tempête, un ombrage contre la chaleur ; car le souffle des tyrans est comme l'ouragan qui frappe une muraille.
\VS{5}Tu rabaisses le bruit tumultueux des étrangers ; comme la chaleur dans un pays brûlant, comme la chaleur est étouffée par l'ombre d'un nuage, ainsi ont été étouffés les chants de triomphe des tyrans.
\VS{6}Yahweh des armées prépare à tous les peuples sur cette montagne un festin de mets succulents, un festin de vins vieux, de mets succulents, pleins de mœlle, de vins vieux, bien épurés\FTNT{Mt. 22:2 ; Ap. 3:20.}.
\VS{7}Et il détruit sur cette montagne le voile qui est sur tous les peuples, la couverture qui couvre toutes les nations.
\VS{8}Il engloutit la mort par sa victoire\FTNT{1 Co. 15:54.} ; le Seigneur Yahweh essuie les larmes de tous les visages\FTNT{Ap. 7:17.}, et il ôte l'opprobre de son peuple de toute la terre\FTNT{Lu. 1:25.}, car Yahweh a parlé.
\VS{9}Et l'on dira en ce jour-là : Voici, c'est notre Dieu, en qui nous avons confiance, et c'est lui qui nous sauve ; c'est Yahweh, en qui nous avons confiance ; soyons dans l'allégresse, et réjouissons-nous de son salut !
\VS{10}Car la main de Yahweh repose sur cette montagne ; et Moab est foulé aux pieds sous lui, comme on foule la paille dans une mare à fumier.
\VS{11}Au milieu de cette mare, il étend ses mains, comme le nageur les étend pour nager ; et Yahweh abat son orgueil, ainsi que l'artifice de ses mains.
\VS{12}Il renverse, il précipite les fortifications élevées de tes murailles, il les fait crouler à terre, et les réduit en poussière.
\Chap{26}
\TextTitle{Adoration à Yahweh}
\VerseOne{}En ce jour-là, on chantera ce cantique dans le pays de Juda : Nous avons une ville forte ; il nous donne le salut\FTNT{Le mot salut vient du mot « Yeshuw'ah ». Cette même racine a donné le prénom Jésus qui signifie Yahweh sauve. Jésus est notre muraille et notre rempart. Dans Ex. 15:2, Moïse identifie Yahweh à « Yeshuw'ah » c'est-à-dire à Jésus. Dans 1 Ch. 16:23, il est dit que « Yeshuw'ah » doit être annoncé tous les jours. Dans Ps. 62:2, il est présenté comme Dieu et le Rocher. Dans Es. 12:2, il est le Dieu qui sauve. Jacob et David avaient mis en lui leur espoir (Ge. 49:18 ; Ps. 119:166). Dans Es. 49:6, il est dit que le salut (« Yeshuw'ah » ou Jésus) doit être annoncé aux extrémités de la terre, et cela est répété et confirmé en Mt. 28:18-20. Es. 56:1 nous apprend que celui qui vient s'appelle « Yeshuw'ah ». Es. 59:17 le présente comme notre casque, ce qui fait écho au casque du salut en Ep. 6:17. Les murs de la Nouvelle Jérusalem portent son Nom (Es. 60:18). Ha. 3:8 nous dit que « Yeshuw'ah » montera sur ses chevaux, corroborant le récit de son retour en gloire dans Ap. 19:11-20. « Yeshuw'ah » est notre flambeau selon Es. 62:1 et Ap. 21:23.} pour murailles et pour rempart.
\VS{2}Ouvrez les portes, laissez entrer la nation juste et fidèle.
\VS{3}A celui qui s'appuie sur toi, tu assures la vraie paix, parce qu'on se confie en toi\FTNT{Es. 57:19 ; Ph. 4:6-7.}.
\VS{4}Confiez-vous en Yahweh à perpétuité, car Yahweh Dieu est le rocher\FTNT{Voir commentaire en Es. 8:13-14. } des siècles.
\VS{5}Il a renversé ceux qui habitaient les hauteurs, il a abaissé la ville de haute retraite, il l'a abaissée jusqu'à terre, il l'a fait descendre jusqu'à la poussière.
\VS{6}Elle est foulée aux pieds, aux pieds, dis-je, des pauvres, sous les pas des misérables.
\VS{7}Le chemin du juste est la droiture ; tu aplanis le chemin du juste.
\VS{8}Aussi t'avons-nous attendu, ô Yahweh, sur la voie de tes jugements ! Ton Nom et ton souvenir sont le désir de notre âme.
\VS{9}Mon âme te désire pendant la nuit et mon esprit te cherche au-dedans de moi ; car lorsque tes jugements s'exercent sur la terre, les habitants du monde apprennent la justice.
\VS{10}Si l'on fait grâce au méchant, il n'apprend pas la justice, mais il agit méchamment sur la terre de la droiture, et il n'a point égard à la majesté de Yahweh.
\VS{11}Yahweh, ta main est élevée, mais ils ne l'aperçoivent pas. Qu'ils voient ton zèle pour ton peuple et qu'ils soient confus ! Et le feu dont tu punis tes ennemis les dévorera.
\VS{12}Yahweh ! Tu nous procures la paix car tout ce que nous faisons, c'est toi qui l'accomplis pour nous.
\VS{13}Yahweh, notre Dieu, d'autres maîtres que toi ont dominé sur nous, mais c'est par toi seul que nous pouvons invoquer ton Nom.
\VS{14}Ils sont morts, ils ne revivront plus, des ombres ne se relèveront pas ; car tu les as châtiés et exterminés, et tu en as détruit tout souvenir\FTNT{Ec. 9:5.}.
\VS{15}Multiplie le peuple, ô Yahweh ! Multiplie le peuple, tu as été glorifié, mais tu les as jetés loin dans toutes les extrémités de la terre.
\TextTitle{Un reste épargné de la colère de Yahweh}
\VS{16}Yahweh, ils t'ont cherché quand ils étaient dans la détresse ; ils se sont répandus en prières quand tu les as châtiés.
\VS{17}Comme une femme enceinte sur le point d'accoucher se tord et crie au milieu de ses douleurs, ainsi avons-nous été loin de ta face ô Yahweh !
\VS{18}Nous avons conçu et nous avons éprouvé des douleurs et nous n'avons enfanté que du vent : La terre n'est pas sauvée et ses habitants ne tomberaient point par notre force.
\VS{19}Tes morts vivront ! Même mon corps mort vivra ! Ils se relèveront. Réveillez-vous et réjouissez-vous avec des chants de triomphe, vous, habitants de la poussière ; car ta rosée est comme la rosée des herbes, et la terre jettera dehors les morts\FTNT{Os. 13:14 ; Da. 12:2 ; 1 Co. 15:52.}.
\VS{20}Va, mon peuple, entre dans tes chambres et ferme ta porte derrière toi\FTNT{Mt. 6:6.} ; cache-toi pour un petit moment, jusqu'à ce que l'indignation soit passée.
\VS{21}Car voici, Yahweh s'en va sortir de son lieu pour visiter l'iniquité des habitants de la terre, commise contre lui ; alors la terre découvrira le sang qu'elle aura reçu et ne couvrira plus ceux qu'on a mis à mort.
\Chap{27}
\TextTitle{Israël rétabli}
\VerseOne{}En ce jour-là, Yahweh frappera de sa dure, grande et forte épée le léviathan\FTNT{Ps. 104:26 ; Job. 40:20}, serpent fuyard, le léviathan, dis-je, serpent tortueux, et il tuera le monstre qui est dans la mer.
\VS{2}En ce jour-là, chantez sur la vigne désirable\FTNT{Esaïe annonce ici le rétablissement d'Israël. Voir également Ro. 11:1-24.}.
\VS{3}C'est moi Yahweh, j'en suis le gardien, je l'arrose à chaque instant, je la garde nuit et jour, afin que personne ne lui fasse du mal.
\VS{4}Il n'y a point de colère en moi ; qu'on me donne des ronces, des épines à combattre ! Je marcherai contre elles, je les consumerai toutes ensemble.
\VS{5}A moins qu'on ne me prenne pour un refuge, qu'on fasse la paix avec moi, qu'on fasse la paix avec moi.
\VS{6}Dans les temps à venir, Jacob prendra racine, Israël poussera des fleurs et des rejetons, et il remplira le monde de ses fruits.
\VS{7}Dieu l'a-t-il frappé comme il a frappé ceux qui le frappaient ? L'a-t-il tué comme il a tué ceux qui le tuaient ?
\VS{8}C'est avec mesure que tu l'as châtié en le rejetant, lorsqu'il fut emporté par le souffle impétueux du vent d'orient.
\VS{9}C'est pourquoi la propitiation de l'iniquité de Jacob sera faite par ce moyen, et ceci en sera le fruit entier, que son péché sera ôté ; quand il aura transformé toutes les pierres des autels comme des pierres de chaux réduites en poussière ; et lorsque les idoles d'Asherah et les statues consacrées au soleil ne se relèveront plus.
\VS{10}Car la ville forte est solitaire, la demeure agréable est abandonnée et délaissée comme le désert. Là pâture le veau, il s'y couche et broute les branches.
\VS{11}Quand les rameaux sèchent, on les brise ; des femmes viennent pour les brûler. C'était un peuple sans intelligence\FTNT{De. 32:28 ; Es. 1:3.} ; aussi celui qui l'a fait n'a point eu pitié de lui, celui qui l'a formé ne lui a point fait grâce.
\VS{12}En ce jour-là, Yahweh secouera des fruits, depuis le cours du fleuve jusqu'au torrent d'Egypte ; et vous serez ramassés un à un, ô enfants d'Israël.
\VS{13}En ce jour-là, on sonnera du grand shofar, et ceux qui étaient exilés au pays d'Assyrie, ou fugitifs au pays d'Egypte, reviendront et se prosterneront devant Yahweh, sur la sainte montagne, à Jérusalem.
\Chap{28}
\TextTitle{Malheur et captivité d'Ephraïm En assyrie}
\VerseOne{}Malheur à la couronne orgueilleuse des ivrognes d'Ephraïm, à la fleur fanée qui fait l'éclat de sa parure, qui est sur le sommet de la fertile vallée, de ceux qui sont vaincus par le vin !
\VS{2}Voici, le Seigneur tient dans sa main un homme fort et puissant, semblable à un orage de grêle, à un ouragan destructeur, comme une tempête de grosses eaux débordées ; il la fait tomber à terre avec force.
\VS{3}Elle sera foulée aux pieds, la couronne orgueilleuse des ivrognes d'Ephraïm.
\VS{4}Et la fleur fanée, qui fait l'éclat de sa parure sur le sommet de la fertile vallée, sera comme une figue précoce qu'on aperçoit avant la récolte et qui à peine dans la main, est aussitôt dévorée.
\VS{5}En ce jour-là, Yahweh des armées sera une couronne éclatante et un diadème de gloire pour le reste de son peuple ;
\VS{6}Un esprit de justice pour celui qui est assis au siège de la justice, et une force pour ceux qui dans le combat repoussent l'ennemi jusqu'à ses portes.
\VS{7}Mais eux aussi, ils chancellent dans le vin, et les boissons enivrantes leur donnent des vertiges ; sacrificateurs et prophètes chancellent dans les boissons fortes ; ils sont vaincus par le vin, ils ont des vertiges à cause des boissons enivrantes ; ils chancellent en prophétisant, ils vacillent en rendant la justice.
\VS{8}Toutes leurs tables sont pleines de vomissements et d'ordures ; il n'y a plus de place !
\VS{9}A qui veut-on enseigner la sagesse ? A qui veut-on donner des leçons ? Est-ce à des enfants qui viennent d'être sevrés, qui viennent de quitter la mamelle ?
\VS{10}Car il faut leur donner précepte sur précepte, précepte sur précepte, règle sur règle, règle sur règle, un peu ici, un peu là\FTNT{Hé. 5:12.}.
\VS{11}C'est pourquoi, c'est par des lèvres qui balbutient et par une langue étrangère qu'il parlera à ce peuple.
\VS{12}Il leur disait : Voici le repos, laissez reposer celui qui est fatigué ; voici le lieu du repos ! Mais ils n'ont point voulu écouter.
\VS{13}Ainsi la parole de Yahweh sera pour eux précepte sur précepte, précepte sur précepte, règle sur règle, règle sur règle, un peu ici, un peu là ; afin qu'en marchant ils tombent à la renverse, afin qu'ils soient brisés, qu'ils tombent dans le piège et qu'ils soient pris.
\TextTitle{Yahweh rompt le pacte du scheol par une pierre angulaire}
\VS{14}C'est pourquoi écoutez la parole de Yahweh, vous hommes moqueurs, qui dominez sur ce peuple qui est à Jérusalem !
\VS{15}Car vous dites : Nous avons fait un pacte avec la mort, nous avons réalisé une vision avec le scheol ; quand le fléau débordé passera, il ne nous atteindra pas, car nous avons la fausseté pour refuge et le mensonge pour abri.
\VS{16}C'est pourquoi ainsi parle le Seigneur Yahweh : Voici, je mettrai pour fondement en Sion une pierre\FTNT{Voir commentaire en Es. 8:13-16.}, une pierre éprouvée, la pierre angulaire, la plus précieuse, pour être un fondement solide ; celui qui croira n'aura point hâte de fuir.
\VS{17}Je ferai de la droiture une règle, et de la justice un niveau ; et la grêle emportera le refuge de la fausseté, et les eaux inonderont l'abri du mensonge.
\VS{18}Et votre alliance avec la mort sera détruite, votre vision avec le scheol ne subsistera pas ; quand le fléau débordé passera, vous serez foulés par lui.
\VS{19}Chaque fois qu'il passera, il vous emportera car il passera tous les matins, le jour et la nuit ; et dès qu'on entendra son bruit, il y aura de l'épouvante.
\VS{20}Car le lit sera trop court pour s'y étendre, et la couverture trop étroite pour s'en envelopper.
\VS{21}Car Yahweh se lèvera comme à la montagne de Peratsim, et il sera ému comme dans la vallée de Gabaon, pour faire son œuvre, son œuvre extraordinaire, et pour exécuter son travail, son travail inconnu.
\VS{22}Maintenant donc, ne vous livrez plus à la moquerie, de peur que vos liens ne soient renforcés, car la destruction de tout le pays est résolue ; je l'ai appris du Seigneur, de Yahweh des armées.
\VS{23}Prêtez l'oreille, et écoutez ma voix ; soyez attentifs, et écoutez mon discours !
\VS{24}Celui qui laboure pour semer, laboure-t-il tous les jours ? Ouvre-t-il et brise-t-il toujours son terrain ?
\VS{25}N'est-ce pas après en avoir aplani la surface qu'il répand le cumin, qu'il met le froment par rangées, l'orge à une place marquée, et l'épeautre\FTNT{L'épeautre est une espèce de blé} sur les bords ?
\VS{26}Son Dieu lui a enseigné la règle à suivre et l'a instruit.
\VS{27}Car on ne foule pas la nielle avec le traîneau, et on ne tourne point la roue du chariot sur le cumin ; mais on bat la nielle avec la verge et le cumin avec le bâton.
\VS{28}On bat le blé avec lequel on fait le pain, mais le laboureur ne le bat pas toujours ; il y pousse la roue du chariot et les chevaux, mais il ne l'écrase pas.
\VS{29}Cela aussi vient de Yahweh des armées qui est admirable en conseil et magnifique en sagesse.
\Chap{29}
\TextTitle{Avertissement d'un châtiment imminent}
\VerseOne{}Malheur à Ariel\FTNT{Ariel : Lion de Dieu, nom appliqué à Jérusalem.}, à Ariel, la ville dont David fit sa demeure ! Ajoutez année à année, qu'on égorge des victimes pour les fêtes.
\VS{2}Mais je mettrai Ariel à l'étroit, il y aura des plaintes et des gémissements ; et la ville sera pour moi comme un Ariel.
\VS{3}Car je t'investirai de toutes parts, je t'assiégerai au moyen des postes armés, j'élèverai contre toi des retranchements.
\VS{4}Tu seras abaissée, et tu parleras depuis la terre, et ta parole sortira étouffée par la poussière, ta voix sortira de terre comme celle d'un évocateur d'esprits, et c'est de la poussière que tu murmureras tes discours.
\VS{5}La multitude de tes étrangers sera comme une fine poussière ; cette multitude de guerriers sera comme la balle qui vole, et cela tout à coup, en un instant.
\VS{6}C'est de Yahweh des armées que viendra le châtiment, avec des tonnerres, des tremblements de terre, et un grand bruit\FTNT{Za. 14:13-14 ; Ap. 16:18-19.} ; avec l'ouragan et la tempête, et avec la flamme d'un feu dévorant.
\VS{7}Et comme il en est d'un songe, d'une vision nocturne, ainsi en sera-t-il de la multitude de toutes les nations qui feront la guerre à Ariel, de tous ceux qui l'attaqueront, elle et sa forteresse, et qui la serreront de près.
\VS{8}Comme celui qui a faim rêve qu'il mange, mais quand il s'éveille son âme est vide ; et comme celui qui a soif rêve qu'il boit, mais quand il s'éveille il est épuisé, et son âme est altérée, ainsi sera-t-il de la multitude de toutes les nations qui combattront contre la montagne de Sion.
\TextTitle{Yahweh donne les raisons du châtiment}
\VS{9}Soyez étonnés et stupéfaits ! Fermez les yeux et devenez aveugles ! Ils sont ivres, mais non de vin ; ils chancellent, mais ce n'est pas l'effet des liqueurs fortes.
\VS{10}Car Yahweh a répandu sur vous un esprit d'assoupissement\FTNT{Ro. 11:8.} ; il a fermé vos yeux, les prophètes ; il a voilé vos têtes, les voyants.
\VS{11}Toute vision est pour vous comme les paroles d'un livre cacheté que l'on donne à un homme qui sait lire en lui disant : Nous te prions, lis donc cela ! Et qui répond : Je ne le puis, car il est cacheté.
\VS{12}Ou comme un livre que l'on donne à un homme qui ne sait pas lire, en lui disant : Nous te prions, lis donc cela ! Et qui répond : Je ne sais pas lire.
\VS{13}C'est pourquoi le Seigneur dit : Quand ce peuple s'approche de moi, il m'honore de sa bouche et des lèvres, mais son cœur est éloigné de moi ; et la crainte qu'il a de moi n'est qu'un commandement enseigné par des hommes\FTNT{Mt. 15:8-9 ; Mc. 7:6-7.}.
\VS{14}A cause de cela, voici, je frapperai encore ce peuple par des prodiges et des miracles ; et la sagesse de ses sages périra, et l'intelligence de ses hommes intelligents disparaîtra.
\VS{15}Malheur à ceux qui cachent profondément leurs desseins, pour les dérober à Yahweh, qui font leurs œuvres dans les ténèbres, et qui disent : Qui nous voit, et qui nous connaît\FTNT{Es. 47:10 ; Ez. 8:12 ; Ps. 10:11 ; Ps. 94:7.} ?
\VS{16}Pervers que vous êtes ! L'argile doit-elle être estimée à l'égale du potier, pour que l'ouvrage dise : Il ne m'a point fait ? Pour que le vase dise du potier : Il n'a point d'intelligence\FTNT{Ps. 100:3.} ?
\TextTitle{Yahweh rachète Jacob}
\VS{17}Encore un peu de temps, et le Liban se changera en verger, et le verger sera considéré comme une forêt.
\VS{18}En ce jour-là, les sourds entendront les paroles du livre, et les yeux des aveugles, délivrés de l'obscurité et des ténèbres, verront\FTNT{Mt. 11:5 ; Lu. 7:22.}.
\VS{19}Les malheureux se réjouiront de plus en plus en Yahweh, et les pauvres d'entre les hommes feront du Saint d'Israël le sujet de leur allégresse\FTNT{Mt. 5:3-11.}.
\VS{20}Car l'oppresseur ne sera plus, le moqueur sera consumé, et tous ceux qui veillaient pour commettre l'iniquité seront retranchés\FTNT{Ap. 20:10.},
\VS{21}ceux qui condamnaient les autres par leur parole, qui tendaient des pièges à celui qui défendait sa cause à la porte et violaient par leur fraude les droits de l'innocent.
\VS{22}C'est pourquoi ainsi parle Yahweh, à la maison de Jacob, lui qui a racheté Abraham : Jacob ne sera plus dans la honte, et sa face ne pâlira plus.
\VS{23}Car lorsque ses enfants verront au milieu d'eux l'œuvre de mes mains, ils sanctifieront mon Nom ; ils sanctifieront le Saint de Jacob et ils craindront le Dieu d'Israël.
\VS{24}Ceux dont l'esprit s'égarait deviendront intelligents, et ceux qui murmuraient apprendront la bonne doctrine.
\Chap{30}
\TextTitle{Mise en garde contre les alliances étrangères}
\VerseOne{}Malheur aux enfants rebelles, dit Yahweh, qui prennent des conseils, et non pas de moi, et qui se forgent des idoles où mon esprit n'est point, afin d'accumuler péché sur péché.
\VS{2}Qui sans avoir consulté ma bouche, marchent pour descendre en Egypte, pour se réfugier sous la protection de Pharaon et chercher un abri sous l'ombre de l'Egypte\FTNT{Jé. 42:19}.
\VS{3}La protection de Pharaon sera pour vous une honte, et l'abri sous l'ombre de l'Egypte une ignominie.
\VS{4}Car ses princes sont à Tsoan, et ses messagers ont atteint Hanès.
\VS{5}Tous seront confus au sujet d'un peuple qui ne leur sera point utile, ni pour les secourir ni pour les aider, mais qui fera leur honte et leur opprobre.
\VS{6}Les bêtes sont chargées pour aller au midi, dans le pays de détresse et d'angoisse, d'où viennent le lion et la lionne, la vipère et le dragon volant ; ils portent à dos d'ânes leurs richesses, et sur la bosse des chameaux leurs trésors, vers le peuple qui ne leur sera point utile.
\VS{7}Car le secours de l'Egypte n'est que vanité et néant ; c'est pourquoi j'appelle cela du bruit orgueilleux qui n'aboutit à rien.
\VS{8}Entre donc maintenant et écris ces choses devant eux sur une table, et rédige-les par écrit dans un livre, afin que cela subsiste dans les temps à venir, éternellement et à perpétuité.
\VS{9}Car c'est ici un peuple rebelle, des enfants menteurs, des enfants qui ne veulent point écouter la loi de Yahweh\FTNT{No. 20: 3-5 ; De. 9:7 ; Ac. 7:51.} ;
\VS{10}qui disent aux voyants : Ne voyez pas ! Et à ceux qui voient des visions : Ne voyez pas de visions de justice, mais dites-nous des choses agréables, voyez des choses trompeuses\FTNT{2 Ti. 4:3-4 ; Mi. 2:6.}.
\VS{11}Détournez-vous du chemin, détournez-vous du sentier, éloignez de notre présence le Saint d'Israël\FTNT{Jn. 14:6.}.
\VS{12}C'est pourquoi ainsi parle le Saint d'Israël : Parce que vous rejetez cette parole et que vous vous confiez dans l'oppression et dans les détours, et que vous les prenez pour appuis,
\VS{13}à cause de cela, cette iniquité sera pour vous comme la fente d'une muraille qui va tomber, faisant saillie dans un mur élevé, dont la ruine arrive soudainement, en un instant.
\VS{14}Il se brise comme on brise un vase de terre, que l'on casse sans ménagement, et dont les débris ne laissent pas un morceau pour prendre du feu au foyer, ou pour puiser de l'eau à la citerne.
\TextTitle{La confiance en Yahweh, la vraie force}
\VS{15}Car ainsi a parlé le Seigneur, Yahweh, le Saint d'Israël : C'est dans la tranquillité et le repos que vous serez délivrés, c'est dans le calme et la confiance que sera votre force. Mais vous ne l'avez point voulu.
\VS{16}Vous avez dit : Non, mais nous nous enfuirons sur des chevaux ; à cause de cela vous vous enfuirez. Et vous avez dit : Nous monterons sur des chevaux légers ; à cause de cela ceux qui vous poursuivront seront légers.
\VS{17}Mille d'entre vous fuiront à la menace d'un seul ; vous fuirez à la menace de cinq ; jusqu'à ce que vous restiez comme un signal tout ébranché au sommet de la montagne, comme un étendard sur la colline.
\VS{18}Cependant Yahweh attend pour vous faire grâce, et il se lèvera pour vous faire miséricorde ; car Yahweh est le Dieu juste ; heureux tous ceux qui se confient en lui !
\VS{19}Car le peuple habitera dans Sion et dans Jérusalem. Tu ne pleureras plus ! Certes, il te fera grâce quand tu crieras ; dès qu'il aura entendu, il t'exaucera.
\VS{20}Le Seigneur vous donnera du pain dans la détresse, et de l'eau dans l'angoisse, ceux qui t'enseignent ne se cacheront plus, mais tes yeux verront ceux qui t'enseignent.
\VS{21}Tes oreilles entendront la parole de celui qui sera derrière toi, disant : Voici le chemin, marchez-y ! Car vous iriez à droite, ou vous iriez à gauche.
\VS{22}Vous tiendrez pour souillés l'argent qui couvre vos idoles et l'or dont elles sont revêtues ; tu les jetteras loin, comme du sang impur : Hors d'ici ! leur diras-tu.
\VS{23}Alors il répandra la pluie sur la semence que tu auras mise en terre, et le pain que produira la terre sera savoureux et nourrissant ; en ce jour-là, ton bétail paîtra dans de vastes pâturages\FTNT{Jn. 14:6.}.
\VS{24}Les bœufs et les ânes qui labourent la terre mangeront un fourrage salé, qu'on aura vanné avec la pelle et le van.
\VS{25}Il y aura des ruisseaux, des courants d'eaux sur toute haute montagne et sur toute colline élevée, au jour du grand carnage, à la chute des tours.
\VS{26}La lumière de la lune sera comme la lumière du soleil ; et la lumière du soleil sera sept fois plus grande, comme si c'était la lumière de sept jours, le jour où Yahweh aura bandé la blessure de son peuple, et qu'il guérira la blessure de sa plaie.
\TextTitle{Jugement de Yahweh sur les Assyriens}
\VS{27}Voici, le Nom de Yahweh vient de loin, sa colère est ardente, c'est un violent incendie ; ses lèvres sont pleines de fureur et sa langue est comme un feu dévorant.
\VS{28}Son Esprit est comme un torrent débordé qui atteint jusqu'au milieu du cou, pour cribler les nations avec le crible de la destruction, et comme un mors trompeur entre les mâchoires des peuples, qui les fera aller à travers champs.
\VS{29}Vous chanterez un cantique comme la nuit où l'on célèbre une fête solennelle ; vous aurez le cœur joyeux comme celui qui marche au son de la flûte, pour aller à la montagne de Yahweh, vers le rocher d'Israël.
\VS{30}Et Yahweh fera entendre sa voix, pleine de majesté, et il montrera son bras prêt à frapper, dans l'ardeur de sa colère, au milieu de la flamme d'un feu dévorant, de l'orage, de la pluie violente et des pierres de grêle.
\VS{31}Car à la voix de Yahweh, l'Assyrien tremblera ; Yahweh le frappera de sa verge.
\VS{32}A chaque coup de verge qui lui est destiné, et que Yahweh fera tomber sur lui, on entendra les tambourins et les harpes ; Yahweh combattra contre lui à main levée.
\VS{33}Depuis longtemps Topheth\FTNT{Topheth : Lieu pour brûler. Un lieu à l'extrémité sud-est de la vallée de Hinnom au sud de Jérusalem.} est déjà préparé, et même il est préparé pour le roi ; on a fait son bûcher profond et large ; son bûcher c'est du feu et du bois en abondance ; le souffle de Yahweh l'enflamme comme un torrent de soufre.
\Chap{31}
\TextTitle{Le secours de Yahweh préférable à celui de l'Egypte}
\VerseOne{}Malheur à ceux qui descendent en Egypte pour avoir du secours, qui s'appuient sur les chevaux, et qui se fient à la multitude de leurs chars et à la force de leurs cavaliers, mais qui ne regardent pas vers le Saint d'Israël, et ne recherchent pas Yahweh.
\VS{2}Lui aussi, cependant, il est sage, il fait venir le malheur et ne révoque point sa parole ; il s'élève contre la maison des méchants et contre le secours de ceux qui commettent l'iniquité.
\VS{3}Les Egyptiens sont des hommes et non Dieu ; et leurs chevaux sont chair et non esprit. Quand Yahweh étendra sa main, celui qui donne du secours sera renversé ; et celui à qui le secours est donné tombera ; et tous ensemble seront consumés.
\VS{4}Car ainsi m'a parlé Yahweh : Comme le lion, comme le lionceau rugit sur sa proie, et, malgré tous les bergers rassemblés contre lui, ne se laisse ni effrayer par leur voix, ni dompter par leur nombre ; de même Yahweh des armées descendra pour combattre en faveur de la montagne de Sion et de sa colline.
\VS{5}Comme des oiseaux déploient les ailes sur leur couvée, ainsi Yahweh des armées protégera Jérusalem, il protégera, il épargnera et sauvera\FTNT{De. 32:11 ; Ps. 91:4 ; Mt. 23:37.}.
\VS{6}Retournez vers celui de qui les enfants d'Israël se sont étrangement éloignés.
\VS{7}En ce jour-là, chacun rejettera ses idoles d'argent et ses idoles d'or que vous vous êtes fabriquées par vos mains pour vous faire pécher.
\VS{8}Et l'Assyrien tombera par l'épée qui n'est pas celle d'un vaillant homme, et l'épée qui n'est pas celle d'un homme le dévorera ; il s'enfuira devant l'épée, et ses jeunes guerriers seront asservis.
\VS{9}Et saisi de frayeur, il s'enfuira à sa forteresse, et ses chefs seront effrayés à cause de la bannière, dit Yahweh, qui a son feu dans Sion et sa fournaise dans Jérusalem.
\Chap{32}
\TextTitle{La venue de l'Esprit annonce la paix et la justice}
\VerseOne{}Voici, un roi régnera selon la justice, et les princes gouverneront avec équité.
\VS{2}Chacun d'eux sera comme un abri contre le vent et un refuge contre la tempête ; comme des ruisseaux d'eau dans un pays aride, comme l'ombre d'un grand rocher dans une terre altérée.
\VS{3}Alors les yeux de ceux qui voient ne seront plus bouchés, et les oreilles de ceux qui entendent seront attentives.
\VS{4}Le cœur des hommes légers entendra la science, et la langue de ceux qui balbutient parlera aisément et nettement.
\VS{5}L'insensé ne sera plus appelé noble, et le trompeur ne sera plus nommé magnifique.
\VS{6}Car l'insensé profère des folies, et son cœur s'adonne à la fausseté, pour exécuter son déguisement et proférer des faussetés contre Yahweh, pour rendre vide l'âme de celui qui a faim, et enlever le breuvage de celui qui a soif\FTNT{Jn. 10:10.}.
\VS{7}Les armes du fourbe sont pernicieuses ; il forme des desseins pleins de machinations, pour attraper par des paroles trompeuses les malheureux, même quand la cause du pauvre est juste\FTNT{2 Pi. 2:3.}.
\VS{8}Mais le noble forme des desseins de libéralité et se lève pour user de libéralité.
\VS{9}Femmes insouciantes, levez-vous, écoutez ma voix ! Filles indolentes, prêtez l'oreille à ma parole !
\VS{10}Dans un an et quelques jours, vous tremblerez, indolentes ; car c'en est fait de la vendange, la récolte n'arrivera plus.
\VS{11}Soyez dans l'effroi, insouciantes ! Déshabillez-vous, mettez-vous à nu et ceignez vos reins !
\VS{12}On se frappe la poitrine à cause de la beauté des champs et de la fécondité des vignes.
\VS{13}Les épines et les ronces croissent sur la terre de mon peuple ; même dans toutes les maisons de plaisance de la ville joyeuse.
\VS{14}Car le palais est abandonné, la cité bruyante est délaissée ; la colline et la tour serviront à jamais de cavernes ; les ânes sauvages y joueront, et les troupeaux y paîtront,
\VS{15}jusqu'à ce que l'Esprit soit répandu d'en haut sur nous\FTNT{Joë. 2:28 ; Za.12:10 ; Ac. 2:17-18.}, et que le désert se change en verger, et que le verger soit considéré comme une forêt.
\VS{16}Alors la droiture habitera dans le désert et la justice fera sa demeure dans le verger.
\VS{17}La justice produira de la paix, et le fruit de la justice sera le repos et la sécurité pour toujours.
\VS{18}Mon peuple habitera dans une demeure paisible, dans des habitations sûres, dans des asiles tranquilles.
\VS{19}Mais la grêle tombera sur la forêt, et la ville sera entièrement abaissée.
\VS{20}Heureux vous qui partout semez le long des eaux, et qui laissez sans entraves le pied du bœuf et de l'âne !
\Chap{33}
\TextTitle{Yahweh se lève}
\VerseOne{}Malheur à toi qui ravages et qui n'as pas été ravagé ! Qui pilles et qu'on n'a pas encore pillé ! Quand tu auras fini de ravager, tu seras ravagé ; quand tu auras achevé de piller, on te pillera.
\VS{2}Yahweh, aie pitié de nous ! Nous espérons en toi. Sois notre aide dès le matin et notre délivrance au temps de la détresse !
\VS{3}Quand ta voix retentit, les peuples fuient ; quand tu te lèves, les nations se dispersent.
\VS{4}Votre butin est ramassé comme on ramasse la sauterelle ; on se précipite dessus comme se précipitent les sauterelles.
\VS{5}Yahweh est élevé, car il habite dans les lieux élevés ; il remplit Sion de droiture et de justice\FTNT{Ps. 97:9.}.
\VS{6}Tes jours seront en sûreté ; la sagesse et la connaissance sont une source de délivrance ; la crainte de Yahweh est son trésor.
\VS{7}Voici, leurs héros poussent des cris au-dehors, et les messagers de paix pleurent amèrement.
\VS{8}Les routes sont désertes, on ne passe plus dans les chemins. Il a rompu l'alliance, il méprise les villes, il n'a de respect pour personne.
\VS{9}On mène le deuil, la terre languit. Le Liban est confus et languissant. Le Saron est comme un désert. Le Basan et le Carmel secouent leur feuillage.
\VS{10}Maintenant je me lèverai, dit Yahweh, maintenant je serai exalté, maintenant je serai élevé.
\VS{11}Vous avez conçu du foin et vous enfanterez de la paille ; votre souffle vous dévorera comme le feu.
\VS{12}Les peuples seront des fournaises de chaux ; ils seront brûlés au feu comme des épines coupées.
\VS{13}Vous qui êtes loin, écoutez ce que j'ai fait ! Et vous qui êtes près, sachez quelle est ma force !
\TextTitle{Yahweh assure la paix aux justes}
\VS{14}Les pécheurs sont effrayés dans Sion, et le tremblement saisit les hypocrites, tellement qu'ils disent : Qui de nous pourra séjourner avec le feu dévorant\FTNT{Hé. 12:29.} ? Qui de nous pourra séjourner avec les flammes éternelles ?
\VS{15}Celui qui observe la justice et qui parle selon la droiture ; qui rejette le gain acquis par extorsion, qui secoue ses mains pour ne pas accepter un présent ; qui ferme ses oreilles pour ne pas entendre des propos sanguinaires, et qui ferme ses yeux pour ne pas voir le mal.
\VS{16}Celui-là habitera dans des lieux élevés, des rochers fortifiés seront sa retraite ; du pain lui sera donné, de l'eau lui sera assurée\FTNT{Jn. 4:14 ; Jn. 6:33-35 ; Ap 21:6.}.
\VS{17}Tes yeux contempleront le roi dans sa beauté ; ils verront la terre éloignée.
\VS{18}Ton cœur se souviendra de la terreur en disant : Où est le secrétaire, où est le trésorier ? Où est celui qui tient le compte des tours ?
\VS{19}Tu ne verras plus le peuple audacieux, le peuple au langage inconnu qu'on n'entend pas, à la langue barbare qu'on ne comprend pas.
\VS{20}Regarde Sion, la ville de nos fêtes solennelles ! Tes yeux verront Jérusalem, séjour tranquille, tente qui ne sera plus transportée, et dont les pieux ne seront jamais ôtés, et dont les cordages ne seront point détachés\FTNT{Ap. 21:2.}.
\VS{21}Car c'est là vraiment que Yahweh est magnifique ; il nous tient lieu de fleuves, de vastes rivières, où ne pénètrent pas des navires à rames et que ne traverse aucun grand navire.
\VS{22}Car Yahweh est notre Juge, Yahweh est notre Législateur, Yahweh est notre Roi\FTNT{Jésus-Christ exerce toutes les fonctions gouvernementales : législatives, exécutives et judiciaires.} ; c'est lui qui vous sauvera.
\VS{23}Tes cordages sont relâchés, ils ne serrent plus le pied du mât et ne tendent plus les voiles. Alors on partage la dépouille d'un grand butin ; même les boiteux prennent part au butin.
\VS{24}Aucun de ceux qui y demeurent ne dira : Je suis malade ! Le peuple de Jérusalem reçoit le pardon de ses iniquités.
\Chap{34}
\TextTitle{Le jugement des nations\FTNTT{Ap. 19:17-21.}}
\VerseOne{}Approchez-vous nations, pour écouter ! Et vous peuples, soyez attentifs ! Que la terre écoute, et tout ce qui la remplit ! Que le monde et tout ce qu'il produit écoute !
\VS{2}Car la colère de Yahweh va fondre sur toutes les nations et sa fureur sur toute leur armée ; il les voue à l'extermination, il les livre au massacre.
\VS{3}Leurs morts sont jetés là, leurs cadavres exhalent la puanteur et les montagnes se fondent dans leur sang.
\VS{4}Toute l'armée des cieux se dissout ; les cieux sont roulés comme un livre\FTNT{Ap. 6:14.}, et toute leur armée tombe, comme tombe la feuille de la vigne, et comme tombe celle du figuier\FTNT{Mt. 24:28 ; Mc. 13:25.}.
\VS{5}Parce que mon épée s'est enivrée dans les cieux, voici, elle va descendre en jugement contre Edom, et contre le peuple que j'ai voué à l'extermination.
\VS{6}L'épée de Yahweh est pleine de sang ; couverte de graisse, du sang des agneaux et des boucs, de la graisse des reins des béliers ; car il y a des sacrifices de Yahweh à Botsra, et un grand massacre dans le pays d'Edom.
\VS{7}Les buffles tombent avec eux, et les bœufs avec les taureaux ; leur terre est enivrée de sang, leur poussière imprégnée de graisse.
\VS{8}Car c'est un jour de vengeance pour Yahweh, une année de rétribution pour la cause de Sion\FTNT{Jé. 46:10 ; Joë. 2:2 ; So. 1:15.}.
\VS{9}Les torrents d'Edom seront changés en poix, et sa poussière en soufre, et sa terre deviendra de la poix qui brûle.
\VS{10}Elle ne s'éteindra ni jour ni nuit ; sa fumée s'en élèvera éternellement, elle sera désolée de génération en génération ; à tout jamais personne n'y passera.
\VS{11}Le pélican et le hérisson la posséderont, la chouette et le corbeau l'habiteront. On y étendra le cordeau de la désolation et le niveau de la destruction.
\VS{12}Ses grands ne seront plus là pour proclamer un roi, tous ses princes seront réduits à néant.
\VS{13}Les épines croîtront dans ses palais, les ronces et les chardons dans ses forteresses. Ce sera la demeure des serpents, le repaire des filles de l'autruche, le parvis des hiboux.
\VS{14}Les bêtes sauvages du désert et les chiens s'y rencontreront ; les boucs s'y appelleront les uns les autres ; là, la Lilith\FTNT{Lilith est le nom d'une déesse de la nuit connue pour être un démon nocturne qui hantait les lieux déserts d'Edom.} aura sa demeure et trouvera son lieu de repos ;
\VS{15}là le serpent fera son nid, déposera ses œufs, les couvera, et recueillera ses petits à son ombre ; là se rassembleront tous les vautours.
\VS{16}Consultez le livre de Yahweh et lisez : Aucun d'eux ne fera défaut, ni l'un ni l'autre ne manqueront ; car c'est sa bouche qui l'a ordonné, c'est son Esprit qui les rassemblera.
\VS{17}C'est lui qui a jeté le sort pour eux ; c'est sa main qui leur a partagé cette terre au cordeau, ils la posséderont toujours, ils l'habiteront d'âge en âge.
\Chap{35}
\TextTitle{Yahweh se révèle et sauve son peuple}
\VerseOne{}Le désert et le lieu aride se réjouiront ; le lieu solitaire s'égaiera et fleurira comme un narcisse.
\VS{2}Il fleurira abondamment et tressaillira de joie, avec des chants d'allégresse et des cris de triomphe ; la gloire du Liban lui sera donnée, avec la magnificence du Carmel et de Saron ; ils verront la gloire de Yahweh et la magnificence de notre Dieu.
\VS{3}Fortifiez les mains languissantes, et affermissez les genoux qui chancellent\FTNT{Hé. 12:12.}.
\VS{4}Dites à ceux qui ont le cœur troublé : Prenez courage et ne craignez plus\FTNT{Jn. 14:1 ; Jn. 16:33.} ; voici votre Dieu, la vengeance viendra, la rétribution de Dieu ; il viendra lui-même et vous délivrera.
\VS{5}Alors s'ouvriront les yeux des aveugles et les oreilles des sourds seront ouvertes.
\VS{6}Alors le boiteux sautera comme un cerf, et la langue du muet chantera de joie\FTNT{Esaïe a annoncé la venue de Yahweh lui-même. Cette prophétie s'est parfaitement accomplie en Jésus-Christ qui a réalisé tout ce qui avait été prédit. « Allez rapporter à Jean ce que vous entendez et ce que vous voyez : Les aveugles voient, les boiteux marchent, les lépreux sont purifiés, les sourds entendent, les morts ressuscitent, et l'Evangile est annoncé aux pauvres » (Mt. 11:4-5).}. Car des eaux jailliront dans le désert et des torrents dans la solitude.
\VS{7}Le mirage se changera en étang, et la terre desséchée en sources d'eaux ; dans le repaire qui servait de gîte aux serpents, croîtront des roseaux et des joncs.
\VS{8}Il y aura là un chemin, une route qu'on appellera la voie sainte ; nul impur n'y passera ; elle sera pour eux seuls ; ceux qui la suivront, même les insensés, ne pourront s'égarer\FTNT{Mt. 7:13-14 ; Jn. 14:6.}.
\VS{9}Sur cette route, point de lion ; nulle bête féroce ne la prendra, nulle ne s'y rencontrera ; mais les rachetés y marcheront.
\VS{10}Ceux dont Yahweh aura payé la rançon\FTNT{Jésus-Christ est Yahweh qui a payé notre rançon (Mc. 10:45).}, retourneront, ils iront à Sion avec chant de triomphe, et une joie éternelle couronnera leur tête ; ils obtiendront la joie et l'allégresse ; la douleur et le gémissement s'enfuiront.
\Chap{36}
\TextTitle{Invasion de Sanchérib, menaces de Rabschaké\FTNTT{2 R. 18:9-37 ; 2 Ch. 32:1-19.}}
\VerseOne{}La quatorzième année du roi Ezéchias, Sanchérib, roi d'Assyrie, monta contre toutes les villes fortes de Juda et s'en empara\FTNT{2 R. 18:17.}.
\VS{2}Puis le roi d'Assyrie envoya de Lakis à Jérusalem, vers le roi Ezéchias, Rabschaké avec une puissante armée. Rabschaké s'arrêta à l'aqueduc de l'étang supérieur, sur le chemin du champ du foulon.
\VS{3}Alors Eliakim, fils de Hilkija, chef de la maison du roi, se rendit auprès de lui, avec Schebna, le secrétaire, et Joach, fils d'Asaph, l'archiviste.
\VS{4}Rabschaké leur dit : Dites maintenant à Ezéchias : Ainsi parle le grand roi, le roi d'Assyrie : Quelle est cette confiance que tu as ?
\VS{5}Je te le dis, ce ne sont que des paroles en l'air ; il faut pour la guerre de la prudence et de la force. En qui donc as-tu placé ta confiance pour t'être rebellé contre moi ?
\VS{6}Voici, tu l'as placée dans l'Egypte, tu as pris pour soutien ce bâton, ce roseau cassé qui perce et traverse la main de celui qui s'appuie dessus ; tel est Pharaon, roi d'Egypte, pour tous ceux qui se confient en lui.
\VS{7}Peut-être me diras-tu : Nous nous confions en Yahweh, notre Dieu. Mais n'est-ce pas lui dont Ezéchias a fait disparaître les hauts lieux et les autels, en disant à Juda et à Jérusalem : Vous vous prosternerez devant cet autel-ci ?
\VS{8}Maintenant donc, donne des otages au roi d'Assyrie, mon maître ; et je te donnerai deux mille chevaux, si tu peux fournir des cavaliers pour les monter.
\VS{9}Comment ferais-tu tourner le visage à un seul gouverneur d'entre les moindres serviteurs de mon maître ? Mais tu mets ta confiance dans l'Egypte pour les chars et pour les cavaliers.
\VS{10}D'ailleurs, est-ce sans la volonté de Yahweh que je suis monté contre ce pays pour le détruire ? Yahweh m'a dit : Monte contre ce pays et détruis-le.
\VS{11}Alors Eliakim, Schebna et Joach dirent à Rabschaké : Nous te prions de parler en langue araméenne à tes serviteurs, car nous la comprenons ; mais ne parle pas en langue judaïque aux oreilles du peuple qui est sur la muraille.
\VS{12}Rabschaké répondit : Est-ce à ton maître et à toi que mon maître m'a envoyé dire ces paroles ? N'est-ce pas à ces hommes assis sur la muraille pour manger leurs excréments et pour boire leur urine avec vous ?
\VS{13}Puis Rabschaké se dressa et s'écria à haute voix en langue judaïque, et dit : Ecoutez les paroles du grand roi, du roi d'Assyrie !
\VS{14}Ainsi parle le roi : Qu'Ezéchias ne vous séduise pas, car il ne pourra pas vous délivrer.
\VS{15}Qu'Ezéchias ne vous fasse pas mettre votre confiance en Yahweh, en disant : Yahweh nous délivrera certainement ; cette ville ne sera point livrée entre les mains du roi d'Assyrie.
\VS{16}N'écoutez point Ezéchias ; car ainsi parle le roi d'Assyrie : Faites un accord avec moi pour votre bénédiction et rendez-vous à moi, et chacun de vous mangera de sa vigne et de son figuier, et chacun boira de l'eau de sa citerne,
\VS{17}jusqu'à ce que je vienne et que je vous emmène dans un pays pareil à votre pays, dans un pays de blé et de bon vin, un pays de pain et de vignes.
\VS{18}Qu'Ezéchias ne vous séduise point en disant : Yahweh nous délivrera. Les dieux des nations ont-ils délivré chacun son pays de la main du roi d'Assyrie ?
\VS{19}Où sont les dieux de Hamath et d'Arpad ? Où sont les dieux de Sepharvaïm ? Ont-ils délivré Samarie de ma main ?
\VS{20}Parmi tous les dieux de ces pays, quels sont ceux qui ont délivré leur pays de ma main, pour que Yahweh délivre Jérusalem de ma main ?
\VS{21}Mais ils se turent et ne lui répondirent pas un mot ; car le roi avait donné cet ordre, disant : Vous ne lui répondrez pas.
\TextTitle{Ezéchias informé des menaces}
\VS{22}Après cela, Eliakim fils de Hilkija, chef de la maison du roi, Schebna, le secrétaire, et Joach, fils d'Asaph l'archiviste, vinrent auprès d'Ezéchias, les vêtements déchirés, et lui rapportèrent les paroles de Rabschaké.
\Chap{37}
\TextTitle{Ezéchias recherche Yahweh auprès d'Esaïe\FTNTT{2 R. 19:1-7 ; 2 Ch. 32:20.}}
\VerseOne{}Lorsque le roi Ezéchias eut entendu ces choses, il déchira ses vêtements, se couvrit d'un sac et entra dans la maison de Yahweh\FTNT{2 R. 19:1-7 ; 2 Ch. 32:20.}.
\VS{2}Puis il envoya Eliakim, chef de la maison du roi, et Schebna, le secrétaire, et les plus anciens des sacrificateurs couverts de sacs, vers Esaïe, le prophète, fils d'Amots.
\VS{3}Et ils lui dirent : Ainsi parle Ezéchias : Ce jour est un jour d'angoisse, de châtiment et de blasphème ; car les enfants sont près de sortir du sein maternel, mais il n'y a point de force pour enfanter.
\VS{4}Peut-être que Yahweh, ton Dieu, a-t-il entendu les paroles de Rabschaké, que le roi d'Assyrie, son maître, a envoyé pour blasphémer le Dieu vivant et lui faire outrage ; et peut-être que Yahweh, ton Dieu, exercera-t-il ses châtiments à cause des paroles qu'il a entendues. Fais une prière pour le reste qui subsiste encore.
\VS{5}Les serviteurs du roi Ezéchias allèrent vers Esaïe.
\VS{6}Et Esaïe leur dit : Voici ce que vous direz à votre maître : Ainsi parle Yahweh : Ne crains point pour les paroles que tu as entendues, par lesquelles les serviteurs du roi d'Assyrie m'ont blasphémé.
\VS{7}Voici, je vais mettre en lui un esprit tel que, sur une nouvelle qu'il recevra, il retournera dans son pays, et je le ferai tomber par l'épée dans son pays.
\TextTitle{Provocation et menace de Sanchérib\FTNTT{2 R. 19:8-13 ; 2 Ch. 32:17-19.}}
\VS{8}Rabschaké s'étant retiré, trouva le roi d'Assyrie qui attaquait Libna, car il avait appris qu'il était parti de Lakis.
\VS{9}Alors le roi d'Assyrie ayant entendu dire au sujet de Tirhaka, roi d'Ethiopie : Il est sorti pour te faire la guerre. Dès qu'il eut entendu cela, il envoya des messagers à Ezéchias, en leur disant :
\VS{10}Vous parlerez ainsi à Ezéchias, roi de Juda : Que ton Dieu, auquel tu te confies, ne t'abuse point, en disant : Jérusalem ne sera point livrée entre les mains du roi d'Assyrie.
\VS{11}Voici, tu as appris ce que les rois d'Assyrie ont fait à tous les pays, et comment ils les ont détruits entièrement ; et toi, tu échapperais ?
\VS{12}Les dieux des nations que mes ancêtres ont détruites, à savoir Gozan, Charan, Retseph, et les fils d'Eden, qui sont à Telassar, les ont-ils délivrées ?
\VS{13}Où sont le roi de Hamath, le roi d'Arpad, et le roi de la ville de Sepharvaïm, d'Héna et d'Ivva ?
\TextTitle{Prière d'Ezéchias à Yahweh\FTNTT{2 R. 19:14-19 ; 2 Ch. 32:20.}}
\VS{14}Ezéchias prit les lettres de la main des messagers et les lut. Puis il monta à la maison de Yahweh, et Ezéchias les déploya devant Yahweh.
\VS{15}Puis Ezéchias fit sa prière à Yahweh, en disant :
\VS{16}Ô Yahweh des armées ! Dieu d'Israël qui es assis entre les chérubins ! C'est toi qui es le seul Dieu de tous les royaumes de la terre, c'est toi qui as fait les cieux et la terre.
\VS{17}Ô Yahweh ! Incline ton oreille et écoute ! Ô Yahweh ! Ouvre tes yeux et regarde ! Entends toutes les paroles de Sanchérib, qu'il m'a envoyé dire pour blasphémer le Dieu vivant.
\VS{18}Il est vrai, ô Yahweh, que les rois d'Assyrie ont détruit tous les pays et leurs contrées ;
\VS{19}et qu'ils ont jeté dans le feu leurs dieux ; mais ce n'étaient point des dieux, mais un ouvrage de mains d'homme, du bois et de la pierre ; c'est pourquoi ils les ont détruits.
\VS{20}Maintenant donc, ô Yahweh notre Dieu ! Délivre-nous de la main de Sanchérib, afin que tous les royaumes de la terre sachent que toi seul es Yahweh.
\TextTitle{Esaïe transmet la réponse de Yahweh\FTNTT{2 R. 19:20-34.}}
\VS{21}Alors Esaïe, fils d'Amots, envoya dire à Ezéchias : Ainsi parle Yahweh, le Dieu d'Israël : J'ai entendu la prière que tu m'as faite au sujet de Sanchérib, roi d'Assyrie.
\VS{22}C'est ici la parole que Yahweh a prononcée contre lui : Elle te méprise, elle se moque de toi, la vierge, fille de Sion ; elle hoche la tête après toi, la fille de Jérusalem.
\VS{23}Qui as-tu outragé et blasphémé ? Contre qui as-tu élevé ta voix ? Tu as porté tes yeux en haut, sur le Saint d'Israël.
\VS{24}Tu as insulté le Seigneur par le moyen de tes serviteurs, et tu as dit : J'ai gravi le sommet des montagnes avec la multitude de mes chars, les extrémités du Liban ; je couperai ses plus hauts cèdres et ses plus beaux cyprès, et j'atteindrai sa dernière cime, la forêt de son verger.
\VS{25}J'ai creusé des sources et j'en ai bu les eaux, et je tarirai avec la plante de mes pieds tous les fleuves de l'Egypte.
\VS{26}N'as-tu pas appris que j'ai déjà préparé cette ville depuis longtemps, et que dès les temps anciens je l'ai ainsi formée ? Et maintenant l'aurais-je conservée pour être réduite en désolation, et les villes fortes en monceaux de ruines ?
\VS{27}Leurs habitants sont impuissants, épouvantés et confus ; ils sont devenus comme l'herbe des champs, la tendre verdure ; comme le gazon des toits, et le blé brûlé avant la formation de sa tige.
\VS{28}Mais je sais quand tu t'assieds, quand tu sors et quand tu entres, et comment tu es furieux contre moi\FTNT{Ps. 139:2.}.
\VS{29}Parce que tu es furieux contre moi, et que ton insolence est montée à mes oreilles, je mettrai ma boucle à tes narines, et mon mors entre tes lèvres, et je te ferai retourner par le chemin par lequel tu es venu.
\VS{30}Que ceci soit un signe pour toi, ô Ezéchias : On mangera cette année le produit du grain tombé, et une deuxième année ce qui croit de soi-même ; mais la troisième année, vous sèmerez, vous moissonnerez, vous planterez des vignes, et vous en mangerez le fruit.
\VS{31}Ce qui aura été épargné de la maison de Juda, ce qui sera resté poussera encore des racines par-dessous et produira du fruit par-dessus.
\VS{32}Car il sortira de Jérusalem un reste, et de la montagne de Sion des réchappés. Voilà ce que fera le zèle de Yahweh des armées.
\VS{33}C'est pourquoi ainsi parle Yahweh sur le roi d'Assyrie : Il n'entrera point dans cette ville, il n'y lancera aucune flèche, il ne se présentera point contre elle avec le bouclier, et il n'élèvera point de retranchements contre elle.
\VS{34}Il s'en retournera par le chemin par lequel il est venu, et il n'entrera point dans cette ville, dit Yahweh.
\VS{35}Car je protégerai cette ville pour la délivrer pour l'amour de moi, et pour l'amour de David, mon serviteur.
\TextTitle{Yahweh frappe Sanchérib\FTNTT{2 R. 19:35-37 ; 2 Ch. 32:21.}}
\VS{36}L'ange de Yahweh\FTNT{Ge. 16:7.} sortit et frappa cent quatre-vingt-cinq mille hommes dans le camp des Assyriens. Et quand on se leva le matin, voici, ils étaient tous morts.
\VS{37}Alors Sanchérib, roi d'Assyrie, partit de là ; il s'en alla et s'en retourna, et il resta à Ninive.
\VS{38}Or comme il était prosterné dans la maison de Nisroc\FTNT{Le nom Nisroc signifie « le grand aigle ». C'était une idole de Ninive adorée par Sanchérib, symbolisée par un aigle à figure humaine.}, son dieu, Adrammélec et Scharetser, ses fils, le tuèrent avec l'épée ; puis ils s'enfuirent au pays d'Ararat. Et Esar-Haddon, son fils, régna à sa place.
\Chap{38}
\TextTitle{Maladie et guérison d'Ezéchias\FTNTT{2 R. 20:1-11 ; 2 Ch. 32:24-30.}}
\VerseOne{}En ces jours-là, Ezéchias fut malade à la mort\FTNT{2 R. 20:1-11 ; 2 Ch. 32:24-30.}. Et Esaïe le prophète, fils d'Amots, vint auprès de lui, et lui dit : Ainsi parle Yahweh : Donne tes ordres à ta maison, car tu vas mourir et tu ne vivras plus.
\VS{2}Alors Ezéchias tourna sa face contre la muraille et fit sa prière à Yahweh,
\VS{3}et dit : Ô Yahweh, souviens-toi maintenant je te prie que j'ai marché devant ta face avec fidélité et intégrité de cœur, et que j'ai fait ce qui est agréable à tes yeux ! Et Ezéchias pleura abondamment.
\VS{4}Puis la parole de Yahweh fut adressée à Esaïe, en disant :
\VS{5}Va, et dis à Ezéchias ainsi parle Yahweh, le Dieu de David, ton père : J'ai exaucé ta prière, j'ai vu tes larmes. Voici, j'ajouterai à tes jours quinze années.
\VS{6}Je te délivrerai de la main du roi d'Assyrie, toi et cette ville, et je protègerai cette ville.
\VS{7}Et voici, de la part de Yahweh, le signe auquel tu connaîtras que Yahweh accomplira la parole qu'il a prononcée.
\VS{8}Voici, je m'en vais faire retourner de dix degrés en arrière avec le soleil l'ombre des degrés qui est descendue sur les degrés d'Achaz. Et le soleil recula de dix degrés sur les degrés où il était descendu.
\VS{9}Or c'est ici l'écrit d'Ezéchias, roi de Juda, sur sa maladie et sur son rétablissement.
\VS{10}Je disais : Quand mes jours sont en repos, je m'en irai aux portes du scheol, je suis privé du reste de mes années.
\VS{11}Je disais : Je ne contemplerai plus Yahweh, Yahweh sur la terre des vivants ; je ne verrai plus aucun homme parmi les habitants du monde !
\VS{12}Ma génération est enlevée et transportée loin de moi comme une tente de berger ; ma vie est coupée comme par un tisserand qui me retrancherait de sa trame. Du matin au soir tu m'auras enlevé\FTNT{Aux versets 12 et 13, le mot qui a été traduit par « enlevé » est « shalam » : « être dans une alliance de paix, être en paix ».} !
\VS{13}Je me suis contenu jusqu'au matin ; comme un lion, il brisait ainsi tous mes os ; du matin au soir tu m'auras enlevé.
\VS{14}Je poussais des cris comme une grue et comme une hirondelle ; je gémissais comme la colombe ; mes yeux se lassaient à force de regarder en haut : Ô Yahweh, je suis en détresse, sois mon garant !
\VS{15}Que dirai-je ? Il m'a répondu et il m'a exaucé. Je marcherai humblement jusqu'au terme de mes années, après que mon âme ait été affligée.
\VS{16}Seigneur, c'est par ces choses qu'on a la vie, c'est en elles que consiste la vie de mon esprit. Ainsi tu me rétabliras et me feras revivre.
\VS{17}Voici, dans ma paix, une grande amertume m'est survenue, mais tu as pris plaisir à retirer mon âme de la fosse de la destruction, car tu as jeté tous mes péchés derrière ton dos.
\VS{18}Car ce n'est pas le scheol qui te loue, ce n'est pas la mort qui te célèbre ; ceux qui sont descendus dans la fosse ne s'attendent plus à ta vérité\FTNT{Ps. 115:17.}.
\VS{19}Mais le vivant, le vivant est celui qui te loue, comme moi aujourd'hui ; le père conduira ses enfants à la connaissance de ta vérité\FTNT{Pr. 22:6 ; Ep. 6:4.}.
\VS{20}Yahweh est venu me délivrer, et à cause de cela, nous jouerons sur les instruments mes cantiques, tous les jours de notre vie dans la maison de Yahweh.
\VS{21}Esaïe avait dit : Qu'on prenne une masse de figues sèches et qu'on les frotte sur l'ulcère ; et Ezéchias guérira.
\VS{22}Et Ezéchias avait dit : A quel signe connaîtrais-je que je monterai à la maison de Yahweh ?
\Chap{39}
\TextTitle{Ezéchias montre toutes ses richesses aux babyloniens\FTNTT{2 R. 20:12-19}}
\VerseOne{}En ce temps-là\FTNT{2 R. 20:12-19.}, Mérodac-Baladan, fils de Baladan, roi de Babylone, envoya des lettres avec un présent à Ezéchias, parce qu'il avait appris qu'il avait été malade et qu'il avait guéri.
\VS{2}Ezéchias en eut de la joie, et il leur montra les cabinets où étaient ses choses précieuses, l'argent, l'or, et les aromates, et l'huile précieuse, tout son arsenal, et tout ce qui se trouvait dans ses trésors ; il n'y eut rien qu'Ezéchias ne leur fît voir dans sa maison et dans tous ses domaines.
\VS{3}Puis le prophète Esaïe vint vers le roi Ezéchias, et lui dit : Qu'ont dit ces hommes-là, et d'où sont-ils venus vers toi ? Et Ezéchias répondit : Ils sont venus vers moi d'un pays éloigné, de Babylone.
\VS{4}Esaïe dit encore : Qu'ont-ils vu dans ta maison ? Ezéchias répondit : Ils ont vu tout ce qui est dans ma maison ; il n'y a rien dans mes trésors que je ne leur aie fait voir.
\VS{5}Et Esaïe dit à Ezéchias : Ecoute la parole de Yahweh des armées :
\VS{6}Voici, les jours viennent où l'on emportera à Babylone tout ce qui est dans ta maison et ce que tes pères ont amassé dans leurs trésors jusqu'à ce jour ; il n'en restera rien, dit Yahweh\FTNT{2 R. 24:13 ; 2 R. 25:13-15 ; Jé. 20:5.}.
\VS{7}Et l'on prendra de tes fils qui sortiront de toi, et que tu auras engendrés pour en faire des eunuques dans le palais du roi de Babylone\FTNT{Da. 1:3-4.}.
\VS{8}Ezéchias répondit à Esaïe : La parole de Yahweh que tu as prononcée est bonne ; car, ajouta-t-il, au moins qu'il y ait paix et sécurité pendant mes jours.
\Chap{40}
\TextTitle{Un nouveau message pour Esaïe}
\VerseOne{}Consolez, consolez mon peuple, dit votre Dieu.
\VS{2}Parlez à Jérusalem selon son cœur, et criez-lui que son temps de guerre est fini, que son iniquité est tenue pour acquittée, qu'elle a reçu de la main de Yahweh au double de tous ses péchés.
\TextTitle{Mission de Jean-Baptiste\FTNTT{Mt. 3:3.}}
\VS{3}La voix de celui qui crie au désert\FTNT{L'accomplissement de cette prophétie se trouve en Mt 3:3, où il nous est dit que la voix qui devait crier ces choses était celle de Jean-Baptiste (voir aussi Mal. 3:1 ; Mal. 4:5-6 ; Mt. 17:10-13).} est : Préparez le chemin de Yahweh\FTNT{Les évangiles nous enseignent que Jean-Baptiste a été envoyé pour préparer le chemin du Seigneur Jésus (Jn. 1:19-27 ; Jn. 1:29-34 ; Jn. 3:28-31).}, aplanissez parmi les lieux arides un chemin pour notre Dieu.
\VS{4}Toute vallée sera comblée, toute montagne et toute colline seront abaissées, et les lieux tortueux seront redressés, et les lieux raboteux seront aplanis.
\VS{5}Alors la gloire de Yahweh sera manifestée, et toute chair en même temps la verra, car la bouche de Yahweh a parlé.
\TextTitle{La grandeur de Dieu échappe à l'homme}
\VS{6}La voix dit : Crie ! Et on a répondu : Que crierai-je ? Toute chair est comme l'herbe, et toute sa grâce est comme la fleur d'un champ\FTNT{Ja. 1:10 ; 1 Pi. 1:24-25.}.
\VS{7}L'herbe sèche, et la fleur tombe quand le vent de Yahweh souffle dessus. Certainement le peuple est comme l'herbe.
\VS{8}L'herbe sèche, et la fleur tombe, mais la parole de notre Dieu demeure éternellement.
\VS{9}Sion, qui annonce les nouvelles, monte sur une haute montagne ; Jérusalem, qui annonce les nouvelles, élève ta voix avec force ; élève-la, ne crains point ; dis aux villes de Juda : Voici votre Dieu !
\VS{10}Voici, le Seigneur Yahweh\FTNT{Jésus-Christ est Yahweh qui vient (Es. 35:4 ; Es. 40:10-11 ; Es. 60:1 ; Es. 62:11-12 ; Es. 66:15-16 ; Za. 14:1-7 ; Mt. 24 ; Jn. 14:1-3; Ac. 1:10-12 ; Ap. 3:11 ; Ap. 19:11-12 ; Ap. 22:7 ; Ap. 22:12 ; Ap. 22:20).} viendra contre le fort, et son bras dominera sur lui ; voici son salaire est avec lui, et ses rétributions le précèdent.
\VS{11}Il paîtra son troupeau comme un berger, il rassemblera les agneaux dans ses bras, il les portera dans son sein ; il conduira celles qui allaitent\FTNT{Jn. 10.}.
\VS{12}Qui est celui qui a mesuré les eaux dans le creux de sa main, pris les dimensions des cieux avec la paume, qui a rassemblé toute la poussière de la terre dans un boisseau, et qui a pesé au crochet les montagnes et les collines à la balance ?
\VS{13}Qui a mesuré l'Esprit de Yahweh, ou qui a été son conseiller pour l'instruire\FTNT{1 Co. 2:16 ; Ro. 11:34.} ?
\VS{14}De qui a-t-il pris conseil pour en recevoir de l'instruction ? Qui lui a appris le chemin de la justice ? Qui lui a enseigné la science et fait connaître le chemin de l'intelligence ?
\VS{15}Voici, les nations sont comme une goutte qui tombe d'un seau, et elles sont comme de la poussière sur une balance ; voici, les îles sont comme une fine poussière qui s'envole.
\VS{16}Le Liban ne suffirait pas pour le feu, et ses animaux ne seraient pas suffisants pour l'holocauste.
\VS{17}Toutes les nations sont devant lui comme un rien, et il ne les considère que comme de la poussière, et comme un néant.
\VS{18}A qui voulez-vous comparer Dieu ? Et par quelle image le représenterez-vous ?
\VS{19}L'ouvrier fond l'idole, et l'orfèvre la couvre d'or et y soude des chaînettes d'argent.
\VS{20}Celui qui est trop pauvre pour faire une offrande choisit un bois qui ne pourrisse pas ; il se cherche un habile ouvrier pour faire une image taillée qui ne bouge pas\FTNT{Es. 44:9-20.}.
\VS{21}Ne le savez-vous pas ? Ne l'avez-vous pas appris ? Cela ne vous a-t-il pas été déclaré dès le commencement ? N'avez-vous jamais réfléchi à la fondation de la terre ?
\VS{22}C'est lui qui est assis au-dessus du globe de la terre, et ceux qui l'habitent sont comme des sauterelles ; il étend les cieux comme un voile, il les déploie comme une tente pour en faire sa demeure.
\VS{23}C'est lui qui réduit les princes à rien, et qui fait des juges de la terre une vanité.
\VS{24}Ils ne sont pas même plantés, pas même semés, leur tronc n'a pas même de racine en terre ; il souffle sur eux, et ils se dessèchent, et un tourbillon les emporte comme de la paille.
\VS{25}A qui me ferez-vous ressembler, et à qui serais-je égalé ? dit le Saint.
\VS{26}Levez vos yeux en haut et regardez ! Qui a créé ces choses ? C'est lui qui fait sortir leur armée par ordre, et qui les appelle toutes par leur nom ; il n'en est pas une qui fait défaut, à cause de la grandeur de sa force, et parce qu'il excelle en puissance.
\VS{27}Pourquoi donc dis-tu, ô Jacob, pourquoi dis-tu, ô Israël : Ma destinée est cachée à Yahweh, et mon Dieu ne soutient plus ma cause ?
\VS{28}Ne le sais-tu pas ? Ne l'as-tu pas appris ? C'est le Dieu d'éternité, Yahweh, qui a créé les extrémités de la terre ; il ne se fatigue point, il ne se lasse point, et il n'y a pas moyen de sonder son intelligence.
\VS{29}C'est lui qui donne de la force à celui qui est fatigué, et il multiplie la force de celui qui n'a aucune vigueur.
\VS{30}Les jeunes gens se lassent et se fatiguent, et les jeunes hommes chancellent.
\VS{31}Mais ceux qui s'attendent à Yahweh renouvellent leur force. Ils s'élèvent avec des ailes, comme des aigles, ils courent, et ne se lassent point, ils marchent, et ne se fatiguent point.
\Chap{41}
\TextTitle{Dénonciation des idoles}
\VerseOne{}Iles, faites silence pour m'écouter ! Que les peuples reprennent de nouvelles forces ; qu'ils s'avancent et qu'ils parlent ; approchons pour plaider ensemble.
\VS{2}Qui a suscité de l'orient celui que la justice appelle à sa suite ? Qui lui a donné les nations ? Qui lui a donné la domination sur les rois ? Qui les a livrés à son épée comme de la poussière, et à son arc comme de la paille qui s'envole ?
\VS{3}Il les a poursuivis, il a parcouru avec paix un chemin que son pied n'avait jamais foulé.
\VS{4}Qui a fait et exécuté ces choses ? C'est celui qui a appelé les âges dès le commencement. Moi Yahweh je suis le premier, et je suis avec les derniers\FTNT{Ap. 1:8 ; Ap. 21:6 ; Ap. 22:13.}.
\VS{5}Les îles le voient, et sont dans la crainte, les extrémités de la terre tremblent, ils s'approchent, ils viennent.
\VS{6}Ils s'aident l'un l'autre, et chacun dit à son frère : Fortifie-toi.
\VS{7}L'ouvrier encourage le fondeur ; celui qui polit au marteau encourage celui qui frappe sur l'enclume, il dit de la soudure : Elle est bonne ! Et il fixe l'idole avec des clous, afin qu'elle ne bouge pas.
\VS{8}Mais toi, Israël, tu es mon serviteur, et toi, Jacob, tu es celui que j'ai choisi, la race d'Abraham qui m'a aimé !
\VS{9}Toi, que j'ai pris aux extrémités de la terre, en te préférant aux plus excellents qui sont en elle, à qui j'ai dit : Tu es mon serviteur, je t'ai choisi, et je ne te rejette point\FTNT{De. 7:6 ; Ps. 77:8.}.
\VS{10}Ne crains rien, car je suis avec toi ; ne sois pas inquiet, car je suis ton Dieu ; je te fortifie, et je viens à ton secours, je te soutiens de la droite de ma justice.
\VS{11}Voici, tous ceux qui sont irrités contre toi seront honteux et confus ; ils seront réduits à néant, et les hommes qui disputent avec toi périront.
\VS{12}Tu les chercheras, et tu ne les trouveras plus, ceux qui te suscitaient querelle ; ils seront réduits à néant, ils ne seront plus, ceux qui te faisaient la guerre.
\VS{13}Car je suis Yahweh, ton Dieu, qui fortifie ta main droite, et qui te dis : Ne crains rien, c'est moi qui te secours.
\VS{14}Ne crains rien, vermisseau de Jacob, hommes mortels d'Israël ; je viens à ton secours dit Yahweh, le Saint d'Israël, ton rédempteur.
\VS{15}Voici, je fais de toi un traîneau aigu, tout neuf, ayant des pointes ; tu fouleras les montagnes, tu les écraseras, et tu rendras les collines semblables à de la balle.
\VS{16}Tu les vanneras, et le vent les emportera, et un tourbillon les dispersera. Mais toi, tu te réjouiras en Yahweh, tu mettras ta gloire dans le Saint d'Israël.
\VS{17}Les malheureux et les indigents cherchent de l'eau, et il n'y en a point ; leur langue est desséchée par la soif. Moi, Yahweh, je les exaucerai ; moi, le Dieu d'Israël, je ne les abandonnerai pas\FTNT{Ge. 28:15 ; Jos. 1:5 ; Hé. 13:5.}.
\VS{18}Je ferai jaillir des fleuves sur les hauteurs et des sources au milieu des vallées ; je changerai le désert en étang et la terre aride en sources d'eaux.
\VS{19}Je ferai croître dans le désert le cèdre, l'acacia, le myrte et l'olivier ; je mettrai dans les lieux stériles le cyprès, l'orme et le buis ensemble,
\VS{20}afin qu'ils voient, qu'ils sachent, qu'ils observent et comprennent tous, que la main de Yahweh a fait ces choses, que le Saint d'Israël en est le créateur.
\VS{21}Plaidez votre cause, dit Yahweh ; produisez votre argument, dit le roi de Jacob.
\VS{22}Qu'ils le produisent et qu'ils nous déclarent ce qui doit arriver. Quelles sont les prédictions que vous avez faites auparavant ? Dites-le, pour que nous y prenions garde, et que nous en reconnaissions l'accomplissement ; ou bien annoncez-nous l'avenir.
\VS{23}Déclarez ce qui arrivera plus tard, et nous saurons que vous êtes des dieux ; faites seulement quelque chose de bien ou de mal, pour que nous le voyions et le regardions ensemble.
\VS{24}Voici, vous n'êtes rien, et votre œuvre est le néant ; c'est une abomination que de s'appuyer sur vous.
\VS{25}Je l'ai suscité du nord, et il est venu ; il invoque mon Nom de l'orient ; il foule les princes comme de la boue, comme le potier foule l'argile.
\VS{26}Qui l'a déclaré dès le commencement, pour que nous le sachions, et longtemps d'avance, pour que nous disions : Il est juste ? Nul ne l'a déclaré, nul ne l'a prédit, et personne n'a entendu vos paroles.
\VS{27}Le premier sera pour Sion, disant ; voici, les voici ; et je donnerai quelqu'un à Jérusalem qui annoncera des nouvelles\FTNT{Es. 52:7 ; Ap. 14:6.}.
\VS{28}Je regarde, et il n'y a point d'homme notable parmi eux, et il n'y a aucun homme de conseil qui puisse répondre si je l'interroge.
\VS{29}Voici, ils ne sont tous que vanité, leurs œuvres ne sont que néant, leurs idoles de fonte sont du vent et de la confusion.
\Chap{42}
\TextTitle{Le messie, serviteur de Yahweh}
\VerseOne{}Voici mon serviteur, que je soutiens, c'est mon élu, en qui mon âme prend plaisir ; j'ai mis mon Esprit sur lui, il annoncera la justice aux nations\FTNT{Mt. 3:17 ; Mt. 17:5 ; Mc. 9:7.}.
\VS{2}Il ne criera point, et il n'élèvera point la voix ; il ne la fera point entendre dans les rues.
\VS{3}Il ne brisera point le roseau cassé, et il n'éteindra point le lumignon qui fume\FTNT{Mt. 12:18-20.} ; il annoncera la justice selon la vérité.
\VS{4}Il ne se retirera point et ne s'affaiblira point, jusqu'à ce qu'il ait établi la justice sur la terre et que les îles espèrent en sa loi.
\VS{5}Ainsi parle Dieu, Yahweh, qui a créé les cieux, et qui les a étendus, qui a aplani la terre avec ce qu'elle produit, qui a donné la respiration à ceux qui la peuplent, et l'esprit à ceux qui y marchent.
\VS{6}Moi Yahweh, je t'ai appelé pour la justice, et je te prendrai par la main, je te garderai, et je t'établirai pour traiter alliance avec le peuple, pour être la lumière des nations\FTNT{Voir commentaire en Ge. 1:3-5.},
\VS{7}pour ouvrir les yeux des aveugles, pour faire sortir de prison les prisonniers, et de leur cachot ceux qui habitent dans les ténèbres.
\TextTitle{Israël n'a pas été attentif à Yahweh}
\VS{8}Je suis Yahweh, c'est là mon Nom ; et je ne donnerai pas ma gloire à un autre, ni mon honneur aux images taillées\FTNT{Es. 48:11.}.
\VS{9}Voici, les choses qui ont été prédites auparavant se sont accomplies. Et je vous en annonce de nouvelles ; avant qu'elles arrivent, je vous les prédis.
\VS{10}Chantez à Yahweh un cantique nouveau, et que ses louanges éclatent aux extrémités de la terre, vous qui voguez sur la mer et vous qui la peuplez, îles et habitants des îles.
\VS{11}Que le désert et ses villes élèvent la voix ! Que les villages où habite Kédar élèvent la voix ! Que ceux qui habitent dans les rochers tressaillent d'allégresse ! Que des cris de joie soient poussés du sommet des montagnes !
\VS{12}Qu'on donne gloire à Yahweh, et qu'on publie sa louange dans les îles !
\VS{13}Yahweh s'avance comme un vaillant homme, il réveille son zèle comme un homme de guerre, il jette des cris de joie, il jette, dis-je, de grands cris, il manifeste sa force contre ses ennemis.
\VS{14}J'ai longtemps gardé le silence, je me suis tu, je me suis contenu ; je crierai comme une femme en travail, je serai haletant et j'engloutirai tout à la fois.
\VS{15}Je réduirai les montagnes et les collines en désert, j'en dessécherai toute la verdure, je changerai les fleuves en îles, et je ferai tarir les étangs.
\VS{16}Je conduirai les aveugles sur un chemin qu'ils ne connaissent pas, je les ferai marcher par des sentiers qu'ils ignorent ; je changerai devant eux les ténèbres en lumière, et les endroits tortueux en plaines ; voilà ce que je ferai, et je ne les abandonnerai point.
\VS{17}Ils reculeront, ils seront confus, ceux qui se confient aux images taillées, ceux qui disent aux idoles de métal fondu : Vous êtes nos dieux !
\VS{18}Sourds, écoutez ! Et vous aveugles, regardez et voyez !
\VS{19}Qui est aveugle, sinon mon serviteur ? Et qui est sourd, comme mon messager que j'ai envoyé ? Qui est aveugle, comme celui que j'ai comblé de grâces ? Qui, dis-je, est aveugle, comme le serviteur de Yahweh ?
\VS{20}Vous voyez beaucoup de choses, mais vous ne prenez garde à rien ; vous avez les oreilles ouvertes, mais vous n'entendez rien.
\VS{21}Yahweh prenait plaisir en lui à cause de sa justice ; il a publié une loi grande et magnifique.
\VS{22}Et c'est un peuple pillé et dépouillé ! On les a tous enchaînés dans des cavernes et plongés dans des prisons ; ils ont été mis au pillage, et personne ne les délivre ! Dépouillés et personne ne dit : Restituez !
\VS{23}Qui parmi vous prêtera l'oreille à ces choses ? Qui voudra s'y rendre attentif et écouter à l'avenir ?
\VS{24}Qui a livré Jacob au pillage, et Israël aux pillards\FTNT{Jg. 2:13-16.} ? N'est-ce pas Yahweh, contre lequel nous avons péché ? Ils n'ont point voulu marcher dans ses voies et ils n'ont point écouté sa loi.
\VS{25}Aussi a-t-il répandu sur Israël l'ardeur de sa colère, la violence et la guerre ; la guerre l'a embrasé de toutes parts, et il n'a point compris ; elle l'a consumé, et il n'y a point pris garde.
\Chap{43}
\TextTitle{Yahweh veut racheter Israël}
\VerseOne{}Ainsi parle maintenant Yahweh, qui t'a créé, ô Jacob ! Celui qui t'a formé, ô Israël ! Ne crains rien, car je te rachète, je t'appelle par ton nom : Tu es à moi !
\VS{2}Si tu traverses les eaux, je serai avec toi ; et les fleuves, ils ne te submergeront pas ; si tu marches dans le feu, tu ne seras pas brûlé ; et la flamme ne t'embrasera pas.
\VS{3}Car je suis Yahweh, ton Dieu, le Saint d'Israël, ton Sauveur. Je donne l'Egypte pour ta rançon, l'Ethiopie et Saba à ta place ;
\VS{4}parce que tu es précieux à mes yeux, parce que tu es honoré et que je t'aime, je donne des hommes à ta place, et des peuples pour ta vie.
\VS{5}Ne crains rien, car je suis avec toi ; je ramènerai ta postérité de l'orient, et je te rassemblerai de l'occident.
\VS{6}Je dirai au nord : Donne ! Et au midi : Ne retiens point ! Fais venir mes fils des pays lointains, et mes filles de l'extrémité de la terre,
\VS{7}tous ceux qui s'appellent de mon Nom\FTNT{Dans les Ecritures, le Nom de Dieu le plus cité est YHWH. Jésus, dont le nom signifie « YHWH est salut » correspond au nom et à l'identité que Dieu a révélé à tous ceux qui l'ont rencontré quand il était sur cette terre. Dans sa dernière prière à Gethsémané, Jésus dit : « J'ai fait connaître ton Nom » (Jn. 17:6), et « Je leur ai fait connaître ton Nom » (Jn. 17:26). Ce nom n'est autre que le sien puisque Jésus (YHWH est salut) était et est le Nom de Dieu. Moïse n'avait pas reçu la révélation de ce Nom (Ex. 3:13-14) car cette révélation était réservée à l'Eglise. En tant qu'épouse de Christ, l'Eglise porte le Nom du Seigneur et bénéficie de l'autorité qu'il confère. Ainsi, Jésus est le seul Nom par lequel nous pouvons être sauvés (Ac. 4:12). C'est aussi en son Nom que nous devons être baptisés (Ac. 8:16 ; Ac. 19:5), que nous recevons l'exaucement de nos prières (Jn. 14:13-14 ; Jn. 16:24), que nous sommes délivrés de l'ennemi et que nous obtenons la victoire sur le camp de l'ennemi (Mc. 16:17 ; Ph. 2:9-11).} ; car je les ai créés pour ma gloire ; je les ai formés et les ai faits.
\TextTitle{Yahweh appelle ses témoins}
\VS{8}Qu'on fasse sortir dehors le peuple aveugle qui a des yeux, et les sourds qui ont des oreilles.
\VS{9}Que toutes les nations se rassemblent, et que les peuples se réunissent. Lequel d'entre eux a annoncé ces choses ? Et qui sont ceux qui nous ont fait entendre les premières prédictions ? Qu'ils produisent leurs témoins et qu'ils se justifient ; qu'on les écoute et qu'on dise : C'est vrai !
\VS{10}Vous êtes mes témoins\FTNT{Ac. 1:8.}, dit Yahweh, vous, et mon serviteur que j'ai choisi, afin que vous le sachiez, que vous me croyiez et compreniez que c'est moi. Avant moi il n'a pas été formé de Dieu, et après moi il n'y en aura point.
\VS{11}C'est moi, c'est moi qui suis Yahweh, et à part moi il n'y a point de sauveur\FTNT{Yahweh dit qu'à part lui, il n'y a pas d'autres sauveurs. Or les écrits de la Nouvelle Alliance affirment que Jésus-Christ est le seul Sauveur (Lu. 1:67-80 ; Ac. 4:11-12).}.
\VS{12}C'est moi qui ai prédit ce qui devait arriver, c'est moi qui vous ai sauvés, et qui vous ai fait entendre l'avenir ; ce n'est point parmi vous un dieu étranger qui ait fait ces choses ; et vous êtes mes témoins, dit Yahweh, je suis Dieu.
\VS{13}Je le suis avant que le jour fût, et nul ne délivre de ma main ; je ferai l'œuvre, qui m'en empêchera ?
\TextTitle{Yahweh fera une chose nouvelle car Jacob ne l'a pas honoré}
\VS{14}Ainsi parle Yahweh, votre rédempteur\FTNT{Es. 60:16 ; 1 Co. 1:30 ; Ro. 3:24 ; Ep. 1:7.}, le Saint d'Israël : Par amour pour vous, j'envoie l'ennemi contre Babylone, et je fais descendre tous les fugitifs, et on entendra le cri des Chaldéens sur les navires.
\VS{15}Moi, Yahweh, je suis votre Saint, le créateur d'Israël, votre Roi.
\VS{16}Ainsi parle Yahweh, qui fraya un chemin dans la mer, et un sentier parmi les eaux puissantes ;
\VS{17}qui mit en campagne des chars et des chevaux, une armée et de vaillants guerriers, pour être couchés ensembles et ne plus se relever, pour être anéantis, éteints comme une mèche de lin :
\VS{18}Ne pensez plus aux choses passées, et ne considérez plus ce qui est ancien.
\VS{19}Voici, je vais faire une chose nouvelle\FTNT{2 Co. 5:17.} qui paraîtra bientôt, ne la connaîtrez-vous pas ? Je mettrai un chemin dans le désert et des fleuves dans la solitude.
\VS{20}Les bêtes des champs me glorifieront, les serpents et les autruches, parce que j'aurai mis des eaux dans le désert et des fleuves dans la solitude, pour abreuver mon peuple que j'ai élu.
\VS{21}Le peuple que je me suis formé publiera mes louanges.
\VS{22}Mais toi, Jacob, tu ne m'as pas invoqué, car tu t'es lassé de moi, ô Israël !
\VS{23}Tu ne m'as pas offert tes brebis en holocauste, et tu ne m'as pas glorifié par tes sacrifices ; je ne t'ai point tourmenté pour me faire des offrandes, et je ne t'ai point fatigué pour me présenter de l'encens.
\VS{24}Tu ne m'as pas acheté à prix d'argent le roseau aromatique, et tu ne m'as pas rassasié de la graisse de tes sacrifices ; mais tu m'as tourmenté par tes péchés, et tu m'as fatigué par tes iniquités.
\VS{25}C'est moi, c'est moi qui efface tes transgressions pour l'amour de moi, et je ne me souviendrai plus de tes péchés.
\VS{26}Réveille ma mémoire et plaidons ensemble ; énumère toi-même tes raisons pour te justifier.
\VS{27}Ton premier père a péché, et tes interprètes se sont rebellés contre moi.
\VS{28}C'est pourquoi j'ai profané les chefs du lieu saint, et j'ai livré Jacob à la destruction et Israël aux outrages.
\Chap{44}
\TextTitle{Promesse de l'Esprit, folie de l'idolâtrie}
\VerseOne{}Ecoute maintenant, ô Jacob, mon serviteur, et toi Israël que j'ai choisi.
\VS{2}Ainsi parle Yahweh, qui t'a fait, et qui t'a formé dès le ventre, celui qui te soutient ; ne crains rien, ô Jacob, mon serviteur ! Et toi Israël que j'ai choisi.
\VS{3}Car je répandrai des eaux sur celui qui est altéré, et des ruisseaux sur la terre desséchée ; je répandrai mon esprit sur ta postérité, et ma bénédiction sur tes rejetons.
\VS{4}Ils germeront comme au milieu de l'herbe, comme les saules auprès des courants d'eau.
\VS{5}Celui-ci dira : Je suis à Yahweh ; celui-là se réclamera du nom de Jacob ; et un autre écrira de sa main : Je suis à Yahweh, et se nommera du nom d'Israël.
\VS{6}Ainsi parle Yahweh, le Roi d'Israël et son Rédempteur, Yahweh des armées : Je suis le premier, et je suis le dernier ; et hors moi il n'y a point de Dieu.
\VS{7}Qui a fait entendre sa voix comme moi, qu'il le déclare et me le prouve, depuis que j'ai établi le peuple ancien ? Qu'ils annoncent l'avenir et ce qui doit arriver !
\VS{8}Ne soyez point effrayés et ne soyez point troublés ; ne te l'ai-je pas fait entendre et déclarer dès longtemps ? Vous êtes mes témoins ; y a-t-il un autre Dieu que moi ? Certes il n'y a pas d'autre Rocher\FTNT{Yahweh dit qu'il ne connaît pas d'autre rocher. Jésus-Christ est ce rocher qui suivait les Hébreux dans le désert (Mt. 16:18 ; 1 Co. 10:1-4. Voir aussi commentaire en Es. 8:14). }, je n'en connais pas.
\VS{9}Les ouvriers d'images taillées ne sont tous que vanité, et leurs plus belles œuvres ne servent à rien ; elles le témoignent elles-mêmes ; elles n'ont ni la vue ni la connaissance, afin qu'ils soient dans la confusion.
\VS{10}Mais qui est-ce qui fabrique un dieu, ou fond une image taillée, pour n'en retirer aucun profit ?
\VS{11}Voici, tous ses compagnons seront confus, et les ouvriers ne sont que des hommes ; qu'ils se réunissent tous, qu'ils se présentent, et tous ensemble ils seront effrayés et couverts de honte.
\VS{12}Le forgeron fait une hache, il travaille avec le charbon, et il la façonne à coups de marteau ; il la forge d'un bras vigoureux. Mais a-t-il faim ? le voilà sans force ; ne boit-il pas d'eau ? le voilà épuisé.
\VS{13}Le charpentier étend le cordeau, il trace sa forme au crayon avec de la craie ; il façonne le bois avec des équerres, et marque ses dimensions avec le compas, et il fabrique une figure d'homme, une belle forme humaine, pour qu'elle habite dans une maison.
\VS{14}Il se coupe des cèdres, il prend des rouvres et des chênes, qu'il a laissés croître parmi les arbres de la forêt ; il plante des pins, et la pluie les fait croître.
\VS{15}Ces arbres servent à l'homme pour brûler, car il en prend et il se chauffe. Il y met aussi du feu pour cuire du pain ; et il en fait également un dieu devant lequel il se prosterne ; il en fait une image taillée qu'il adore.
\VS{16}Il brûle au feu la moitié de son bois, avec cette moitié il cuit de la viande, il apprête un rôti, et se rassasie ; il se chauffe aussi, et il dit : Ha ! Ha ! Je me chauffe, je vois la flamme !
\VS{17}Puis avec le reste il fait un dieu, son image taillée ; il se prosterne devant elle, il l'adore, il l'invoque, et lui dit : Délivre-moi, car tu es mon dieu !
\VS{18}Ces gens n'ont ni intelligence ni entendement, car on leur a plâtré les yeux pour qu'ils ne voient point, et les cœurs pour qu'ils ne comprennent point.
\VS{19}Il ne rentre pas en lui-même\FTNT{So. 2:1 ; 2 Co. 13:5.}, et il n'a ni la connaissance ni l'intelligence pour dire : J'en ai brûlé la moitié au feu, j'ai cuit du pain sur les charbons, j'ai rôti de la viande et je l'ai mangée ; et avec le reste je ferai une abomination ! Je me prosternerais devant un morceau de bois !
\VS{20}Il se repaît de cendres, et son cœur abusé l'égare, et il ne délivrera point son âme, et ne dira point : N'est-ce pas du mensonge que j'ai dans ma main droite ?
\TextTitle{Yahweh rachète son peuple}
\VS{21}Souviens-toi de ces choses, ô Jacob ! Ô Israël, car tu es mon serviteur ; je t'ai formé, tu es mon serviteur, ô Israël ! Je ne t'oublierai pas.
\VS{22}J'efface tes transgressions comme une nuée épaisse, et tes péchés comme une nuée ; reviens à moi, car je t'ai racheté.
\VS{23}Ô cieux ! Réjouissez-vous avec chants de triomphe, car Yahweh a opéré ; profondeurs de la terre, retentissez d'allégresse ! Montagnes, éclatez en cris de joie ! Et vous aussi forêts, avec tous vos arbres ! Car Yahweh a racheté Jacob, et s'est manifesté glorieusement en Israël.
\VS{24}Ainsi parle Yahweh, ton Rédempteur, et celui qui t'a formé dès le ventre : Je suis Yahweh qui ai fait toutes choses, seul j'ai déployé les cieux, seul j'ai étendu la terre ;
\VS{25}j'anéantis les signes des menteurs, et je rends insensés les devins ; je renverse l'esprit des sages, et je tourne leur science en folie.
\VS{26}Je confirme la parole de mon serviteur, et j'accomplis les résolutions de mes messagers ; je dis de Jérusalem : Elle sera encore habitée ; et des villes de Juda : Elles seront rebâties ; et je relèverai leurs ruines.
\VS{27}Je dis à l'abîme : Dessèche-toi, et je tarirai tes fleuves.
\TextTitle{Prophétie sur le rétablissement d'Israël par Cyrus}
\VS{28}Je dis de Cyrus\FTNT{Vers 732 avant notre ère, Esaïe prophétisa la destruction de Babylone. Il alla jusqu'à préciser qu'elle serai conquise par un commandant nommé Cyrus. Environ 200 ans plus tard, le 5 octobre 539 avant notre ère, la prophétie se réalisa dans tous ses détails. L'historien grec Hérodote (Ve siècle avant notre ère) confirma la façon dont la ville a été prise.} : Il est mon berger ; et il accomplira toute ma volonté ; il dira de Jérusalem : Qu'elle soit rebâtie ! Et du temple : Qu'il soit fondé.
\Chap{45}
\TextTitle{Cyrus scucité par Yahweh}
\VerseOne{}Ainsi parle Yahweh à son oint, à Cyrus\FTNT{Cyrus le Grand (580 av. J.-C. - 530 av. J.-C.). Voir Esd. 1.},
\VS{2}que je tiens par la main droite, pour terrasser les nations devant lui, et pour délier les ceintures des rois, pour ouvrir devant lui les portes, afin qu'elles ne soient plus fermées.
\VS{3}Je marcherai devant toi, et j'aplanirai les chemins montueux ; je romprai les portes d'airain, et je briserai les verrous de fer. Et je te donnerai des trésors cachés, et des richesses le plus secrètement gardées, afin que tu saches que je suis Yahweh, le Dieu d'Israël, qui t'appelle par ton nom.
\VS{4}Pour l'amour de Jacob, mon serviteur, et d'Israël mon élu ; je t'ai, dis-je, appelé par ton nom, et je t'ai appelé par ton nom avant que tu me connaisses.
\TextTitle{Yahweh, le seul Dieu}
\VS{5}Je suis Yahweh, et il n'y en a point d'autre ; à part moi, il n'y a point de Dieu. Je t'ai ceint avant que tu me connaisses.
\VS{6}Afin que l'on sache, du soleil levant au soleil couchant, qu'à part moi il n'y a point de Dieu. Je suis Yahweh, et il n'y en a point d'autre.
\VS{7}Je forme la lumière et je crée les ténèbres ; je donne la paix et je crée l'adversité ; c'est moi Yahweh qui fais toutes ces choses.
\VS{8}Ô cieux ! Répandez la rosée d'en haut, et que les nuées laissent couler la justice ! Que la terre s'ouvre, qu'elle produise le salut, qu'elle fasse germer la justice ! Moi, Yahweh, je crée ces choses.
\VS{9}Malheur à celui qui conteste avec celui qui l'a façonné ! Vase parmi des vases de terre ! L'argile dit-elle à celui qui la façonne : Que fais-tu ? Et l'œuvre dit-elle à l'ouvrier : Tu n'as point de mains\FTNT{Jé. 18:6 ; Ro. 9:21.} ?
\VS{10}Malheur à celui qui dit à son père : Pourquoi m'as-tu engendré ? Et à sa mère : Pourquoi m'as-tu enfanté ?
\VS{11}Ainsi parle Yahweh, le Saint d'Israël, qui est son Créateur : Veut-on m'interroger sur les choses à venir, me donner des ordres sur mes fils, et sur l'œuvre de mes mains ?
\VS{12}C'est moi qui ai fait la terre et qui ai créé l'homme sur elle ; c'est moi qui ai étendu les cieux de mes mains, et qui ai donné la loi à toute leur armée.
\VS{13}C'est moi qui ai suscité Cyrus dans ma justice, et j'aplanirai toutes ses voies ; il rebâtira ma ville, et libèrera mes captifs\FTNT{Cyrus le grand libéra les Juifs après 70 ans de captivité (Esd. 1).}, sans rançon ni présents, dit Yahweh des armées.
\TextTitle{Les autres peuples reconnaîtront la main de Yahweh sur Israël}
\VS{14}Ainsi parle Yahweh : Les richesses de l'Egypte, et les profits de l'Ethiopie, et ceux des Sabéens, gens de grande stature, passeront chez toi Jérusalem, et seront à toi ; ils marcheront à ta suite, ils passeront enchaînés, ils se prosterneront devant toi, ils te diront en suppliant : Certes, Dieu est au milieu de toi, et il n'y a point d'autre Dieu que lui.
\VS{15}Certainement, tu es le Dieu qui te caches, le Dieu d'Israël, le Sauveur.
\VS{16}Ils sont tous honteux et confus, ils s'en vont tous avec honte, les fabricants d'idoles.
\VS{17}Mais Israël a été sauvé par Yahweh, d'un salut éternel ; vous ne serez ni honteux ni confus jusque dans l'éternité.
\VS{18}Car ainsi parle Yahweh, le Créateur des cieux, le seul Dieu, qui a formé la terre, qui l'a faite et qui l'a affermie ; qui l'a créée pour qu'elle ne soit pas informe\FTNT{Informe : de l'hébreu « tohuw » qui signifie « informe, confusion, solitude, désert, néant ». On retrouve ce mot dès Ge. 1:2.}, qui l'a formée pour qu'elle soit habitée ; je suis Yahweh, et il n'y en a point d'autre.
\VS{19}Je n'ai point parlé en secret ni dans quelque lieu ténébreux de la terre ; je n'ai point dit à la postérité de Jacob : Cherchez-moi vainement ! Je suis Yahweh, qui prononce ce qui est juste, qui déclare ce qui est droit.
\VS{20}Assemblez-vous et venez, approchez-vous ensemble, vous les réchappés des nations ! Ceux qui portent leur idole de bois et qui invoquent un dieu qui ne sauve pas.
\VS{21}Déclarez-le, et faites-les venir ! Qu'ils prennent conseil les uns des autres ! Qui a fait entendre ces choses dès l'origine et les a déclarées dès longtemps ? N'est-ce pas moi, Yahweh ? Il n'y a point d'autre Dieu que moi ; à part moi, il n'y a point de Dieu juste et sauveur.
\VS{22}Vous tous qui êtes aux extrémités de la terre, regardez vers moi, et vous serez sauvés ; car je suis Dieu, et il n'y en a point d'autre.
\VS{23}Je le jure par moi-même, la vérité sort de ma bouche et ma parole ne sera point révoquée : Tout genou fléchira devant moi, et toute langue jurera par moi\FTNT{Ph. 2:9-11.}.
\VS{24}En Yahweh seul, me dira-t-on, sont la justice et la force ; à lui viendront, pour être confondus, tous ceux qui étaient irrités contre lui.
\VS{25}Toute la postérité d'Israël sera justifiée et elle se glorifiera en Yahweh.
\Chap{46}
\TextTitle{La puissance de Yahweh, l'incapacité des idoles}
\VerseOne{}Bel s'écroule, Nebo tombe ; ils mettent leurs faux dieux sur les animaux, sur des bêtes ; les idoles que vous portiez, les voilà chargées, devenues un fardeau pour l'animal fatigué !
\VS{2}Ils sont tombés, ils se sont inclinés ensemble sur leurs genoux, ils ne peuvent sauver le fardeau, et ils s'en vont eux-mêmes en captivité.
\VS{3}Ecoutez-moi, maison de Jacob, et vous tous, reste de la maison d'Israël, vous que j'ai pris à ma charge dès le sein maternel, que j'ai portés dès votre naissance.
\VS{4}Je serai le même jusqu'à votre vieillesse, et je vous soutiendrai jusqu'à la blanche vieillesse ; je l'ai fait, et je veux encore vous porter, vous soutenir et vous sauver.
\VS{5}A qui me comparerez-vous pour le faire mon égal ? Et à qui me ferez-vous ressembler pour que nous soyons semblables ?
\VS{6}Ils tirent l'or de leur bourse, et pèsent l'argent à la balance, et ils paient un orfèvre pour qu'il en fasse un dieu ; ils l'adorent, et se prosternent devant lui.
\VS{7}Ils le portent sur les épaules, et il y reste ; il ne bouge pas de sa place ; puis on crie vers lui, mais il ne répond pas, il ne sauve pas de la détresse ceux qui crient vers lui.
\VS{8}Souvenez-vous de cela, et reprenez courage, vous transgresseurs, et soyez des hommes !
\VS{9}Souvenez-vous des premières choses, celles des temps anciens ; car c'est moi qui suis Dieu, et il n'y en a point d'autre, je suis Dieu, et nul n'est semblable à moi.
\VS{10}J'annonce dès le commencement ce qui doit arriver, et longtemps d'avance, les choses qui ne sont pas encore accomplies ; je dis : Mes desseins subsisteront, et j'exécuterai toute ma volonté.
\VS{11}C'est moi qui appelle de l'orient un oiseau de proie, et d'une terre éloignée un homme pour exécuter mes desseins. Je l'ai dit et je le réaliserai ; je l'ai formé et je l'exécuterai.
\VS{12}Ecoutez-moi, vous qui avez le cœur endurci et qui êtes éloignés de la justice.
\VS{13}Je fais approcher ma justice, elle n'est point loin ; et mon salut, il ne tardera pas. Je mettrai le salut en Sion pour Israël, qui est ma gloire.
\Chap{47}
\TextTitle{Jugement sur Babylone}
\VerseOne{}Descends, et assieds-toi dans la poussière, vierge, fille de Babylone ! Assieds-toi à terre, il n'y a plus de trône pour la fille des Chaldéens ! Car tu ne te feras plus appeler la délicate et la voluptueuse.
\VS{2}Prends les meules et mouds de la farine ; ôte ton voile, relève les pans de ta robe, découvre tes jambes, traverse les fleuves !
\VS{3}Ta nudité sera découverte et ta honte sera vue ; j'exécuterai ma vengeance, je n'épargnerai personne.
\VS{4}Notre Rédempteur s'appelle Yahweh des armées, le Saint d'Israël.
\VS{5}Assieds-toi sans dire mot, et entre dans les ténèbres, fille des Chaldéens, car tu ne te feras plus appeler la souveraine des royaumes.
\VS{6}J'étais irrité contre mon peuple, j'ai profané mon héritage, c'est pourquoi je les ai livrés entre tes mains, mais tu n'as point usé de miséricorde envers eux, tu as durement appesanti ton joug sur le vieillard.
\VS{7}Tu disais : Je serai souveraine à toujours ! Tu n'as point mis dans ton cœur, tu n'as pas pensé que cela prendrait fin.
\VS{8}Maintenant donc écoute ceci, voluptueuse qui t'assieds avec assurance, et qui dis en ton cœur : C'est moi, et il n'y en a point d'autre que moi ; je ne serai jamais veuve, et je ne saurai jamais ce que c'est que d'être privée d'enfants.
\VS{9}Ces deux choses t'arriveront subitement, au même jour, la privation d'enfants et le veuvage ; elles fondront en plein sur toi, malgré la multitude de tes sortilèges, malgré le grand nombre de tes enchantements\FTNT{Ap. 18:7-8.}.
\VS{10}Tu t'es confiée dans ta méchanceté, tu disais : Personne ne voit ! Ta sagesse et ta science t'ont séduite. Et tu disais en ton cœur : C'est moi, et il n'y en a point d'autre que moi.
\VS{11} C'est pourquoi le mal viendra sur toi, et tu ne sauras pas quand il sera près d'arriver, et le malheur qui tombera sur toi sera tel, que tu ne pourras pas le détourner ; et la ruine éclatante que tu n'as pas soupçonnée viendra sur toi subitement.
\VS{12}Reste donc au milieu de tes enchantements, et du grand nombre de tes sortilèges auxquels tu as consacré ton travail dès ta jeunesse ; peut-être pourras-tu en tirer profit ; peut-être te rendras-tu redoutable.
\VS{13}Tu t'es lassée à force de demander des conseils. Que les spectateurs des cieux qui contemplent les étoiles, et qui font leurs prédictions selon les lunes, comparaissent maintenant, et qu'ils te délivrent des choses qui viendront sur toi.
\VS{14}Voici, ils sont devenus comme de la paille, le feu les consume, ils ne sauveront pas leur vie du pouvoir de la flamme ; ce ne sera pas du charbon dont on se chauffe ni un feu auprès duquel on s'assied.
\VS{15}Tel sera le sort de ceux que tu te lassais à consulter. Et ceux avec qui tu as trafiqué dès ta jeunesse, ils errent chacun de son côté ; il n'y a personne pour te sauver.
\Chap{48}
\TextTitle{Yahweh rappelle ses promesses}
\VerseOne{}Ecoutez ceci, maison de Jacob, vous qui êtes appelés du nom d'Israël, et qui êtes sortis des eaux de Juda ; vous qui jurez par le Nom de Yahweh, et qui faites mention du Dieu d'Israël, mais non pas conformément à la vérité et à la justice\FTNT{Jé. 5:2.}.
\VS{2}Car ils prennent leur nom de la sainte cité, et ils s'appuient sur le Dieu d'Israël, dont le nom est Yahweh des armées\FTNT{Ex. 20:7.}.
\VS{3}J'ai déclaré dès longtemps les premières choses, elles sont sorties de ma bouche et je les ai publiées ; je les ai faites subitement et elles se sont accomplies.
\VS{4}Sachant que tu es endurci, que ton cou est une barre de fer, et que tu as un front d'airain,
\VS{5}je t'ai déclaré dès longtemps ces choses, je te les ai fait entendre avant qu'elles arrivent, afin que tu ne dises pas : C'est mon idole qui les a faites, c'est mon image taillée ou mon image en métal fondu qui les a ordonnées.
\VS{6}Tu entends ! Considère tout cela ! Et vous, ne l'annoncerez-vous pas ? Maintenant, je t'annonce des choses nouvelles, cachées, inconnues de toi.
\VS{7}Elles sont créées maintenant et n'appartiennent pas au passé ; avant ce jour tu n'en avais aucune connaissance, afin que tu ne dises pas : Voici, je le savais.
\VS{8}Tu n'en as rien appris, tu n'en as rien su, et jadis ton oreille n'en a point été frappée ; car je savais que tu agirais avec infidélité, et que dès le ventre tu fus appelé transgresseur.
\VS{9}Pour l'amour de mon Nom je diffère ma colère, et pour l'amour de ma gloire je me contiens envers toi pour ne pas t'exterminer.
\VS{10}Voici, je t'ai épuré, mais non pas comme on épure l'argent ; je t'ai éprouvé au creuset de l'affliction.
\VS{11}C'est pour l'amour de moi, pour l'amour de moi que je le fais, car comment mon Nom serait-il profané ? Certes, je ne donnerai pas ma gloire à un autre
\VS{12}Ecoute-moi, Jacob ! Et toi Israël, que j'ai appelé ; c'est moi, c'est moi qui suis le premier, c'est aussi moi qui suis le dernier.
\VS{13}Ma main a fondé la terre et ma droite a mesuré les cieux à l'empan ; je les appelle, et aussitôt ils se présentent.
\VS{14}Vous tous, assemblez-vous et écoutez ! Qui d'entre eux a annoncé ces choses ? Celui que Yahweh aime exécutera sa volonté contre Babylone, et son bras s'appesantira sur les Chaldéens.
\VS{15}C'est moi, c'est moi qui ai parlé, je l'ai aussi appelé, je l'ai fait venir, et son œuvre réussira.
\VS{16}Approchez-vous de moi et écoutez ceci ! Dès le commencement, je n'ai point parlé en cachette, dès l'origine de ces choses, j'ai été là. Et maintenant, le Seigneur, Yahweh, m'a envoyé avec son Esprit.
\VS{17}Ainsi parle Yahweh, ton Rédempteur, le Saint d'Israël : Je suis Yahweh, ton Dieu, je t'instruis pour ton bien, je te guide dans le chemin où tu dois marcher.
\VS{18}Oh ! Si tu étais attentif à mes commandements ! Ta paix serait comme un fleuve et ta justice comme les flots de la mer\FTNT{Jos. 1:8 ; Ps. 1:2 ; Jn. 14:21 ; Ja. 1:22.}.
\VS{19}Ta postérité serait comme le sable et les fruits de tes entrailles comme les grains de sable\FTNT{Ge. 15:5 ; Ge. 22:17 ; Ge. 32:12.} ; ton nom ne serait point retranché ni effacé devant ma face.
\VS{20}Sortez de Babylone, fuyez loin des Chaldéens ! Avec une voix d'allégresse annoncez-le, publiez-le, faites-le savoir jusqu'à l'extrémité de la terre, dites : Yahweh a racheté son serviteur Jacob !
\VS{21}Et ils n'auront pas soif dans les déserts où il les fera marcher ; il fera jaillir pour eux l'eau du rocher, il fendra le rocher, et les eaux couleront.
\VS{22}Il n'y a point de paix pour les méchants, dit Yahweh.
\Chap{49}
\TextTitle{Le Messie, la lumière de tous les peuples}
\VerseOne{}Iles, écoutez-moi ! Peuples éloignés, soyez attentifs ! Yahweh m'a appelé dès le ventre, il a fait mention de mon nom dès les entrailles de ma mère\FTNT{Jé. 1:5 ; Ps. 139:16.}.
\VS{2}Il a rendu ma bouche semblable à une épée aiguë ; il m'a caché dans l'ombre de sa main, et m'a rendu semblable à une flèche bien polie, il m'a caché dans son carquois.
\VS{3}Il m'a dit : Tu es mon serviteur, Israël en qui je me glorifierai.
\VS{4}Et moi j'ai dit : C'est en vain que j'ai travaillé, c'est pour le vide et le néant que j'ai consumé ma force ; toutefois mon droit est auprès de Yahweh, et mon salaire auprès de mon Dieu.
\VS{5}Maintenant, Yahweh parle, lui qui m'a formé dès le ventre pour lui être serviteur, pour ramener à lui Jacob, et Israël encore dispersé ; car je suis glorifié aux yeux de Yahweh, et mon Dieu sera ma force.
\VS{6}Il me dit : C'est peu que tu sois serviteur pour relever les tribus de Jacob et pour ramener les restes d'Israël ; c'est pourquoi je t'établis pour être lumière des nations, pour que tu sois mon salut jusqu'aux extrémités de la terre.
\VS{7}Ainsi parle Yahweh, le Rédempteur, le Saint d'Israël, à celui qu'on méprise, à celui que la nation dédaigne, au serviteur de ceux qui dominent : Des rois le verront, et ils se lèveront, des princes, et ils se prosterneront devant lui, pour l'amour de Yahweh, qui est fidèle, et du Saint d'Israël qui t'a choisi.
\VS{8}Ainsi parle Yahweh : Je t'exaucerai au temps de la grâce et je te secourrai au jour du salut ; je te garderai et je t'établirai pour traiter alliance avec le peuple, pour relever la terre, pour que tu possèdes les héritages désolés.
\VS{9}Pour dire aux captifs : Sortez ! Et à ceux qui sont dans les ténèbres : Paraissez ! Ils paîtront sur les chemins, et leurs pâturages seront sur tous les coteaux.
\VS{10}Ils n'auront pas faim et ils n'auront pas soif ; la chaleur et le soleil ne les frapperont plus, car celui qui a pitié d'eux sera leur guide, et les conduira vers des sources d'eaux\FTNT{Ps. 121:6 ; Lu. 1:67-79.}.
\VS{11}Je changerai toutes mes montagnes en chemins, et mes routes seront frayées.
\VS{12}Les voici, ils viennent de loin, les uns du nord et de l'occident, les autres du pays de Sinim.
\VS{13}Ô cieux ! Réjouissez-vous avec des chants de triomphe, et toi terre, sois dans l'allégresse ! Et vous, montagnes, éclatez de joie avec des chants de triomphe, car Yahweh console son peuple, il a compassion de ses malheureux.
\VS{14}Mais Sion disait : Yahweh me délaisse, le Seigneur m'oublie !
\VS{15}Une femme peut-elle oublier son enfant qu'elle allaite ? N'a-t-elle pas pitié du fils de ses entrailles ? Quand elle l'oublierait, moi je ne t'oublierai point.
\VS{16}Voici, je t'ai gravé sur les paumes de mes mains ; tes murs sont continuellement devant moi.
\VS{17}Tes fils accourent, mais tes destructeurs et tes dévastateurs sortiront du milieu de toi.
\VS{18}Jette les yeux autour de toi et regarde : Tous ils s'assemblent, ils viennent vers toi. Je suis vivant ! dit Yahweh, tu les revêtiras tous comme un ornement, et tu t'en ceindras comme une épouse.
\VS{19}Car tes déserts, tes ruines, et ton pays ravagé seront désormais trop étroits pour ses habitants, et ceux qui dévoraient s'éloigneront.
\VS{20}Ils répèteront à tes oreilles, ces fils dont tu fus privée : L'espace est trop étroit pour moi, fais-moi de la place pour que je puisse m'établir.
\VS{21}Et tu diras en ton cœur : Qui me les a engendrés ? Car j'avais perdu mes enfants et j'étais stérile. J'étais exilée et chassée, qui les a élevés ? Voici, j'étais restée toute seule, et ceux-ci où étaient-ils ?
\VS{22}Ainsi parle le Seigneur Yahweh : Voici, je lèverai ma main vers les nations et je dresserai ma bannière vers les peuples ; et ils ramèneront tes fils entre leurs bras, et ils porteront tes filles sur les épaules.
\VS{23}Des rois seront tes nourriciers et leurs princesses tes nourrices ; ils se prosterneront devant toi le visage contre terre, et ils lécheront la poussière de tes pieds ; et tu sauras que je suis Yahweh, et que ceux qui se confient en moi ne seront point confus\FTNT{Ps. 22:5-6 ; Ps. 69:7 ; Ro. 9:33 ; 1 Pi. 2:6.}.
\VS{24}Le butin du puissant lui sera-t-il enlevé ? Et les captifs du juste seront-ils délivrés ?
\VS{25}Car ainsi parle Yahweh : Oui, la capture du puissant lui sera enlevée, et le butin de l'impie lui sera enlevé ; car je combattrai moi-même tes adversaires, et je délivrerai tes fils.
\VS{26}Je ferai manger à tes oppresseurs leur propre chair ; ils s'enivreront de leur sang comme du moût, et toute chair connaîtra que je suis Yahweh, ton Sauveur, ton Rédempteur, le puissant de Jacob.
\Chap{50}
\TextTitle{Avertissements de Yahweh par son serviteur}
\VerseOne{}Ainsi parle Yahweh : Où est la lettre de divorce par laquelle j'ai répudié votre mère\FTNT{De. 24:1 ; Jé. 3:8 ; Mt. 5:31.} ? Ou bien, auquel de mes créanciers vous ai-je vendus ? Voici, vous avez été vendus à cause de vos iniquités, et votre mère a été répudiée à cause de vos péchés.
\VS{2}Je suis venu : Pourquoi n'y avait-il personne ? J'ai appelé : Pourquoi personne n'a-t-il répondu ? Ma main est-elle trop courte pour racheter\FTNT{No. 11:23 ; Es. 59:1.} ? N'y a-t-il plus de force en moi pour délivrer ? Par ma menace, je dessèche la mer, je réduis les fleuves en désert ; leurs poissons se corrompent faute d'eau, et ils meurent de soif.
\VS{3}Je revêts les cieux d'obscurité, et je fais d'un sac leur couverture.
\VS{4}Le Seigneur, Yahweh, m'a donné la langue des savants, pour que je sache soutenir par la parole celui qui est accablé de maux\FTNT{Job. 6:14 ; 1 Th. 5:14. } ; chaque matin il me réveille soigneusement afin que je prête l'oreille aux discours des sages.
\VS{5}Le Seigneur Yahweh m'a ouvert l'oreille et je n'ai pas été rebelle, je ne me suis pas retiré en arrière.
\VS{6}J'ai exposé mon dos à ceux qui me frappaient et mes joues à ceux qui m'arrachaient la barbe ; je n'ai pas caché mon visage aux opprobres et aux crachats\FTNT{Mt. 5:39 ; Mt. 26:67 ; Lu. 6:29 ; Lu. 18:32.}.
\VS{7}Mais le Seigneur, Yahweh m'a secouru, c'est pourquoi je n'ai point été confus, c'est pourquoi j'ai rendu mon visage semblable à un caillou\FTNT{Ez. 3:8-9.}, sachant que je ne serais point confondu.
\VS{8}Celui qui me justifie est proche ; qui disputera contre moi ? Comparaissons ensemble ! Qui est mon adversaire ? Qu'il s'approche de moi.
\VS{9}Voici, le Seigneur, Yahweh me secourra, qui me condamnera ? Voici, ils tomberont tous en lambeaux comme un vêtement, la teigne les dévorera.
\VS{10}Qui d'entre vous craint Yahweh ? Qu'il écoute la voix de son serviteur ! Quiconque marche dans les ténèbres et manque de lumière, qu'il se confie dans le Nom de Yahweh, et qu'il s'appuie sur son Dieu.
\VS{11}Voici, vous tous qui allumez un feu et qui êtes armés de flambeaux, tombez dans les flammes de votre feu et des flambeaux que vous avez allumés ! C'est par ma main que ces choses vous arriveront ; vous vous coucherez dans les tourments.
\Chap{51}
\TextTitle{Exhortation à ceux qui recherchent Yahweh}
\VerseOne{}Ecoutez-moi, vous qui poursuivez la justice et qui cherchez Yahweh ! Regardez au rocher d'où vous avez été taillés, et au creux de la citerne d'où vous avez été tirés.
\VS{2}Regardez à Abraham, votre père, et à Sara qui vous a enfantés ; car lui seul je l'ai appelé, je l'ai béni et multiplié\FTNT{Ro. 4:1-16 ; Hé. 11:8-12.}.
\VS{3}Ainsi Yahweh va consoler Sion, il aura pitié de toutes ses ruines, il rendra son désert semblable à Eden, et sa terre aride à un jardin de Yahweh. La joie et l'allégresse se trouveront en elle, la reconnaissance et le bruit de mélodie.
\VS{4}Ecoutez-moi donc attentivement, mon peuple, et prêtez-moi l'oreille, vous ma nation ; car la loi sortira de moi, et j'établirai ma loi pour être la lumière des peuples.
\VS{5}Ma justice est proche, mon salut va paraître, et mes bras jugeront les peuples ; les îles espéreront en moi, elles se confieront en mon bras.
\VS{6}Levez les yeux vers les cieux et regardez en bas sur la terre ! Car les cieux s'évanouiront comme une fumée, et la terre tombera en lambeaux comme un vêtement, et ses habitants périront comme des poux ; mais mon salut demeurera éternellement, et ma justice ne sera point anéantie.
\VS{7}Ecoutez-moi, vous qui connaissez la justice, peuple dans le cœur duquel est ma loi ! Ne craignez point l'opprobre des hommes et ne soyez point effrayés devant leurs outrages.
\VS{8}Car la teigne les rongera comme un vêtement\FTNT{Mt. 6:19 ; Lu. 12:33 ; Ja. 5:2.}, et la gerce les dévorera comme de la laine ; mais ma justice durera éternellement, et mon salut s'étendra d'âge en âge.
\VS{9}Réveille-toi, réveille-toi, revêts-toi de force, bras de Yahweh ! Réveille-toi comme aux jours anciens, aux siècles passés. N'est-ce pas toi qui abattis l'Egypte, et qui blessas mortellement le dragon ?
\VS{10}N'est-ce pas toi qui fis tarir la mer, les eaux du grand abîme ? Qui réduisit les lieux les plus profonds de la mer en un chemin pour le passage des rachetés ?
\VS{11}Ainsi ceux dont Yahweh aura payé la rançon, retourneront, ils iront à Sion avec chants de triomphe ; et une allégresse éternelle couronnera leurs têtes ; ils obtiendront la joie et l'allégresse, la douleur et le gémissement s'enfuiront.
\VS{12}C'est moi, c'est moi qui vous console. Qui es-tu pour avoir peur de l'homme mortel qui mourra, et du fils de l'homme qui deviendra comme du foin ?
\VS{13}Et tu oublierais Yahweh qui t'a fait, qui a étendu les cieux et fondé la terre ; et tu tremblerais constamment tout le jour devant la fureur de l'oppresseur parce qu'il cherche à détruire ! Où est maintenant la fureur de l'oppresseur ?
\VS{14}Bientôt celui qui est courbé sous les fers sera mis en liberté. Il ne mourra pas dans la fosse, et son pain ne lui manquera pas.
\VS{15}Car je suis Yahweh, ton Dieu, qui soulève la mer et fais mugir ses flots. Yahweh des armées est son Nom.
\VS{16}Je mets mes paroles dans ta bouche, et je te couvre de l'ombre de ma main, pour établir les cieux, et pour fonder la terre, et pour dire à Sion : Tu es mon peuple !
\VS{17}Réveille-toi, réveille-toi ! Lève-toi, Jérusalem, qui as bu de la main de Yahweh la coupe de sa fureur ; tu as bu, tu as sucé jusqu'à la lie la coupe d'étourdissement\FTNT{Ps. 60:5 ; Ap. 14:10.} !
\VS{18}Il n'y en a aucun pour la conduire de tous les fils qu'elle a enfantés, il n'y en a aucun pour la prendre par la main de tous les fils qu'elle a élevés.
\VS{19}Ces deux choses te sont arrivées ; qui te plaindra ? Le ravage et la ruine, la famine et l'épée ; par qui te consolerai-je ?
\VS{20}Tes fils en défaillance gisaient à tous les carrefours de toutes les rues, comme un bœuf sauvage pris dans les filets, pleins de la fureur de Yahweh, de la répréhension de ton Dieu.
\VS{21}C'est pourquoi, écoute maintenant ceci, ô malheureuse, ivre, mais non pas de vin :
\VS{22}Ainsi parle Yahweh, ton Seigneur et ton Dieu, qui plaide la cause de son peuple : Voici, je prends de la main la coupe d'étourdissement, la lie de la coupe de ma fureur, tu ne la boiras plus !
\VS{23}Car je la mettrai dans la main de tes oppresseurs, qui disaient à ton âme : Courbe-toi, et nous passerons ! Tu faisais alors de ton dos comme une terre, comme une rue pour les passants.
\Chap{52}
\TextTitle{Le réveil de Jérusalem, la ville sainte}
\VerseOne{}Réveille-toi, réveille-toi, Sion ! Revêts-toi de ta force ! Jérusalem, ville sainte ! Revêts-toi de tes vêtements magnifiques ! Car il n'entrera plus chez toi ni incirconcis ni impur.
\VS{2}Jérusalem, secoue ta poussière, lève-toi, mets-toi sur ton séant ! Détache les liens de ton cou, captive, fille de Sion !
\VS{3}Car ainsi parle Yahweh : Vous avez été vendus gratuitement, et vous serez aussi rachetés sans argent.
\VS{4}Car ainsi parle le Seigneur, Yahweh : Mon peuple descendit jadis en Egypte pour y séjourner ; puis l'Assyrien l'opprima sans cause.
\VS{5}Et maintenant, qu'ai-je à faire ici, dit Yahweh, quand mon peuple a été enlevé gratuitement ? Ceux qui dominent sur lui le font hurler, dit Yahweh, et mon Nom est continuellement blasphémé chaque jour.
\VS{6}C'est pourquoi mon peuple connaîtra mon Nom ; c'est pourquoi il saura, en ce jour, que c'est moi qui parle : Me voici !
\VS{7}Combien sont beaux sur les montagnes les pieds de celui qui apporte de bonnes nouvelles, qui proclame la paix\FTNT{Na. 2:1 ; Ro. 10:15.}, qui apporte des nouvelles de bonheur, qui annonce le salut, qui dit à Sion, Ton Dieu règne !
\VS{8}Tes sentinelles élèvent leurs voix, elles se réjouissent ensemble avec chants de triomphe ; car de leurs propres yeux elles voient que Yahweh ramène Sion.
\VS{9}Déserts de Jérusalem, éclatez, réjouissez-vous ensemble avec chants de triomphe ; car Yahweh console son peuple, il rachète Jérusalem.
\VS{10}Yahweh manifeste le bras de sa sainteté aux yeux de toutes les nations\FTNT{Es. 53:1.}, et toutes les extrémités de la terre verront le salut\FTNT{Toutes les extrémités de la terre verront le salut de Yahweh, c'est-à-dire Jésus (Mt. 28:18-20). } de notre Dieu.
\VS{11}Retirez-vous, retirez-vous, sortez de là ! Ne touchez rien d'impur ! Sortez du milieu d'elle\FTNT{Jé. 51:45 ; 2 Co. 6:17 ; Ap. 18:4.} ! Purifiez-vous, vous qui portez les vases de Yahweh.
\VS{12}Car vous ne sortirez pas avec précipitation, et vous ne marcherez pas en fuyant, car Yahweh ira devant vous, et le Dieu d'Israël sera votre arrière-garde.
\TextTitle{Le serviteur de Yahweh}
\VS{13}Voici, mon serviteur prospérera, il sera fort exalté, élevé et glorifié.
\VS{14}De même qu'il a été pour plusieurs un sujet d'étonnement, tant son visage était défiguré, tant son aspect différait de celui des fils de l'homme,
\VS{15}de même il fera tressaillir de joie beaucoup de nations ; devant lui des rois fermeront la bouche ; car ils verront ce qui ne leur avait point été raconté, ils apprendront ce qu'ils n'avaient point entendu.
\Chap{53}
\TextTitle{Le sacrifice du Messie, serviteur de Yahweh}
\VerseOne{}Qui a cru à notre prédication ? Et à qui le bras de Yahweh\FTNT{Jésus-Christ homme est le bras de Yahweh. Le bras de Yahweh est le symbole de la puissance divine. Cette puissance s'est manifestée dans l'œuvre du Messie accomplissant le salut du monde. Le prophète est transporté au moment où le peuple juif, après avoir rejeté son Messie, ouvrira enfin les yeux et acceptera celui qu'il a percé (Za. 12:10 ; Ap. 1:7). Voir aussi Jé. 27:4-5 ; Jé. 32:17.} a-t-il été révélé ?
\VS{2}Toutefois il s'est élevé devant lui comme une jeune plante, comme un rejeton qui sort d'une terre desséchée ; il n'y a en lui ni beauté, ni splendeur, quand nous le regardons, ni apparence qui nous le fasse désirer.
\VS{3}Il est le méprisé et le rejeté des hommes\FTNT{Ps. 22:6-7 ; Mt. 27:27-31 ; Mc. 9:12 ; Jn. 16:32.}, homme de douleur, et sachant ce que c'est que la douleur : et nous avons comme caché notre visage arrière de lui, tant il était méprisé ; et nous ne l'avons pas estimé.
\VS{4}En vérité, il a porté nos maladies, et il s'est chargé de nos douleurs\FTNT{Mt. 8:17 ; 1 Pi. 2:24.} ; et nous l'avons considéré comme frappé, battu par Dieu et humilié.
\VS{5}Mais il était transpercé pour nos péchés, brisé pour nos iniquités, le châtiment qui nous apporte la paix est tombé sur lui, et c'est par ses meurtrissures que nous avons la guérison.
\VS{6}Nous avons tous été errants\FTNT{Pierre, apôtre de l'Agneau, confirme que le Messie est bel et bien le Bon Berger (1 Pi. 2:25).} comme des brebis, nous nous sommes détournés, chacun suivait son propre chemin, et Yahweh a fait venir sur lui l'iniquité de nous tous.
\VS{7}Opprimé et humilié, il n'a point ouvert sa bouche\FTNT{Mt. 26:62-63 ; Mc. 15:3-5 ; Jn. 19:9 ; Ac. 8:32-33.}, semblable à un agneau qu'on mène à la boucherie, à une brebis muette devant celui qui la tond, et il n'a point ouvert sa bouche.
\VS{8}Il a été enlevé par la force de l'angoisse et par la condamnation ; et sa génération qui la racontera ? Car il a été retranché de la terre des vivants, et la plaie lui a été faite pour les péchés de mon peuple.
\VS{9}On a mis son sépulcre parmi les méchants, et dans sa mort, il a été avec le riche, quoiqu'il n'ait point commis de violence, et qu'il n'y ait point eu de fraude dans sa bouche\FTNT{Mc. 15:28 ; Lu. 23:32-33.}.
\VS{10}Toutefois il a plu à Yahweh de le briser ; il l'a mis dans la souffrance. Après qu'il aura mis son âme en sacrifice pour le péché, il verra une postérité et prolongera ses jours ; et le bon plaisir de Yahweh prospérera en sa main\FTNT{Jé. 23:5.}.
\VS{11}Il jouira du travail de son âme et en sera rassasié ; mon serviteur juste justifiera beaucoup d'hommes par la connaissance qu'ils auront de lui ; et lui-même portera leurs iniquités.
\VS{12}C'est pourquoi je lui donnerai sa part parmi les grands ; il partagera le butin avec les puissants, parce qu'il aura livré son âme à la mort, qu'il aura été mis au rang des transgresseurs, et que lui-même aura porté les péchés de plusieurs, et qu'il aura intercédé pour les transgresseurs.
\Chap{54}
\TextTitle{Yahweh réhabilite Israël la délaissée}
\VerseOne{}Réjouis-toi avec chants de triomphe, stérile, toi qui n'enfantes plus ! Fais éclater ton allégresse et ta joie, toi qui n'as plus de douleurs ! Car les fils de la délaissée seront plus nombreux que les fils de celle qui est mariée, dit Yahweh.
\VS{2}Elargis le lieu de ta tente, qu'on déploie les courtines de ton tabernacle : Ne retiens pas ! Allonge tes cordages et affermis tes pieux !
\VS{3}Car tu te répandras à droite et à gauche, et ta postérité possédera les nations et peuplera des villes désertes.
\VS{4}Ne crains pas, car tu ne seras point confondue ; ne rougis pas, car tu ne seras pas confuse ; mais tu oublieras la honte de ta jeunesse, et tu ne te souviendras plus de l'opprobre de ton veuvage.
\VS{5}Car ton Créateur est ton époux : Yahweh des armées est son Nom ; et ton Rédempteur est le Saint d'Israël : Il se nomme le Dieu de toute la terre.
\VS{6}Car Yahweh t'appelle comme une femme délaissée et à l'esprit attristé et affligé, comme une femme qu'on aurait épousée dans la jeunesse et qui aurait été répudiée, dit ton Dieu.
\VS{7}Je t'avais délaissée pour un petit moment, mais je t'accueillerai avec une grande affection.
\VS{8}Dans un instant de colère, je t'avais un moment dérobé ma face, mais avec une bonté éternelle j'aurai compassion de toi, dit Yahweh, ton Rédempteur.
\VS{9}Il en sera pour moi comme les eaux de Noé : J'avais juré que les eaux de Noé ne se répandraient plus sur la terre\FTNT{Ge. 9:11 ; Ge. 8:21.} ; je jure de même de ne plus m'irriter contre toi et de ne plus te menacer.
\VS{10}Quand les montagnes s'éloigneraient, quand les collines chancelleraient, ma bonté ne s'éloignera point de toi, et mon alliance de paix ne chancellera point, dit Yahweh, qui a compassion de toi.
\VS{11}Malheureuse, battue de la tempête, dénuée de consolation, voici, je garnirai tes pierres d'antimoine, et je te donnerai des fondements de saphir ;
\VS{12}je ferai tes créneaux de rubis, et tes portes d'escarboucles, et toute ton enceinte de pierres précieuses.
\VS{13}Aussi tous tes enfants seront enseignés de Yahweh, et grande sera la paix de tes fils.
\VS{14}Tu seras affermie par la justice, tu seras loin de l'oppression, car tu n'as rien à craindre de la ruine, car elle n'approchera pas de toi.
\VS{15}Voici, on ne manquera pas de comploter contre toi, cela ne viendra pas de moi ; quiconque complotera contre toi tombera devant toi\FTNT{Ps. 91:7 ; Ge. 37.}.
\VS{16}Voici, c'est moi qui ai créé le forgeron qui souffle le charbon au feu, et qui fabrique une arme pour son travail, mais j'ai créé aussi le destructeur pour la briser.
\VS{17}Aucune arme forgée contre toi ne réussira, et toute langue qui se lèvera en justice contre toi, tu la condamneras\FTNT{Ps. 23:4.}. Tel est l'héritage des serviteurs de Yahweh, telle est la justice qui leur viendra de moi, dit Yahweh.
\Chap{55}
\TextTitle{Le salut gratuit par la grâce de Dieu}
\VerseOne{}Vous tous qui avez soif, venez aux eaux, et vous qui n'avez pas d'argent, venez, achetez et mangez ; venez, dis-je, achetez du vin et du lait sans argent et sans rien payer !
\VS{2}Pourquoi dépensez-vous de l'argent pour ce qui ne nourrit pas ? Pourquoi travaillez-vous pour ce qui ne rassasie pas\FTNT{Ro. 14:17.} ? Ecoutez-moi attentivement, et vous mangerez de ce qui est bon, et votre âme se délectera de mets succulents.
\VS{3}Prêtez l'oreille, et venez à moi\FTNT{Mt. 11:28.}, écoutez, et votre âme vivra ; je traiterai avec vous une alliance éternelle, les grâces immuables promises à David.
\VS{4}Voici, je l'ai établi comme témoin auprès des peuples, comme chef et dominateur des peuples.
\VS{5}Voici, tu appelleras des nations que tu ne connais pas, et les nations qui ne te connaissent pas accourront vers toi, à cause de Yahweh, ton Dieu, et du Saint d'Israël qui te glorifie.
\VS{6}Cherchez Yahweh pendant qu'il se trouve, invoquez-le tandis qu'il est près.
\VS{7}Que le méchant abandonne sa voie, et l'homme d'iniquité ses pensées ; qu'il retourne à Yahweh qui aura pitié de lui, à notre Dieu qui ne se lasse pas de pardonner\FTNT{Jé. 18:11 ; Ez. 33:11 ; Jon. 3:10 ; 1 Ti. 2:1-4 ; 2 Pi. 3:9.}.
\VS{8}Car mes pensées ne sont pas vos pensées, et mes voies ne sont pas vos voies, dit Yahweh.
\VS{9}Mais autant que les cieux sont élevés au-dessus de la terre, autant mes voies sont élevées au-dessus de vos voies, et mes pensées au-dessus de vos pensées.
\VS{10}Car comme la pluie et la neige descendent des cieux et n'y retournent plus, mais arrosent la terre, la fécondent et font germer les plantes, donnent de la semence au semeur, et du pain à celui qui mange,
\VS{11}ainsi en est-il de ma parole qui sort de ma bouche, elle ne retourne point à moi sans effet, sans avoir exécuté toute ma volonté et fait réussir l'œuvre pour laquelle je l'ai envoyée.
\VS{12}Car vous sortirez avec joie, et vous serez conduits en paix ; les montagnes et les collines éclateront de joie avec chants de triomphe devant vous, et tous les arbres des champs battront des mains.
\VS{13}Au lieu de l'épine s'élèvera le cyprès, au lieu de la ronce croîtra le myrte ; et ceci fera connaître le nom de Yahweh, et ce sera un signe perpétuel qui ne sera jamais retranché.
\Chap{56}
\TextTitle{Exhortation à s'attacher à Yahweh}
\VerseOne{}Ainsi parle Yahweh : Observez ce qui est droit, pratiquez ce qui est juste, car mon salut ne tardera pas à venir et ma justice à être révélée.
\VS{2}Heureux l'homme qui fait cela, et le fils de l'homme qui y demeure ferme, gardant le sabbat pour ne pas le profaner, et veillant sur ses mains pour ne commettre aucun mal.
\VS{3}Que l'enfant de l'étranger qui s'attache à Yahweh ne parle pas en disant : Yahweh me séparera entièrement de son peuple ! Et que l'eunuque ne dise pas : Voici, je suis un arbre sec.
\VS{4}Car ainsi a dit Yahweh : Aux eunuques qui garderont mes sabbats, qui choisiront ce en quoi je prends plaisir et qui persévéreront dans mon alliance,
\VS{5}je donnerai dans ma maison et dans mes murailles une place et un nom préférables à des fils et à des filles ; je leur donnerai à chacun un nom éternel qui ne périra jamais\FTNT{Ap. 2:17.}.
\VS{6}Et les fils des étrangers qui s'attacheront à Yahweh pour le servir, pour aimer le Nom de Yahweh, pour être ses serviteurs, tous ceux qui garderont le sabbat pour ne pas le profaner et qui persévéreront dans mon alliance\FTNT{Ex. 31:14.},
\VS{7}je les amènerai sur ma montagne sainte, et je les réjouirai dans ma maison de prière ; leurs holocaustes et leurs sacrifices seront agréés sur mon autel, car ma maison sera appelée la maison de prière\FTNT{Mt. 21:13 ; Mc. 11:17 ; Lu. 19:46.} pour tous les peuples.
\VS{8}Le Seigneur, Yahweh, parle, lui qui rassemble les exilés d'Israël. Je réunirai d'autres peuples à lui, aux siens déjà rassemblés.
\VS{9}Vous toutes, bêtes des champs, venez manger, vous, toutes bêtes des forêts !
\VS{10}Tous ses gardiens sont aveugles, ils ne connaissent rien ; ils sont tous des chiens muets qui ne peuvent aboyer, ils ont des rêveries, se tiennent couchés, aiment à sommeiller.
\VS{11}Ce sont des chiens voraces, insatiables ; ce sont des bergers qui ne savent rien comprendre ; tous suivent leur propre voie, chacun à son gain injuste dans son quartier, en disant\FTNT{Mt. 23:24 ; Tit. 1:7-11 ; 1 Pi. 5:2.} :
\VS{12}Venez, je vais chercher du vin, et nous boirons des liqueurs fortes ! Nous en ferons autant demain, et beaucoup plus encore !
\Chap{57}
\TextTitle{Yahweh expose la fausseté et défend le juste}
\VerseOne{}Le juste périt, et nul ne le prend à cœur ; et les gens de bien sont enlevés, nul ne comprend que c'est par suite de la méchanceté que le juste est enlevé\FTNT{Mi. 7:2 ; Ec. 7:15.}.
\VS{2}Il entrera dans la paix, il reposera sur sa couche, celui qui aura marché dans le droit chemin\FTNT{Mt. 25:23 ; Lu. 19:17.}.
\VS{3}Mais vous, approchez ici, fils de l'enchanteresse, race de l'adultère et de la prostituée !
\VS{4}De qui vous êtes-vous moqués ? Contre qui avez-vous ouvert une large bouche et tirez-vous la langue ? N'êtes-vous pas des enfants de rébellion, une race de mensonge ?
\VS{5}S'échauffant près des térébinthes, sous tout arbre vert ; égorgeant les enfants dans les vallées, sous les fentes des rochers\FTNT{Lé. 18:21 ; 1 R. 14:23 ; Jé. 2:20 ; Jé. : 32:35.}.
\VS{6}C'est dans les pierres polies des torrents qu'est ton partage, ce sont elles, ce sont elles qui sont ton lot ; c'est à elles que tu verses des libations, que tu fais des offrandes ; puis-je être content de ces choses ?
\VS{7}Tu dresses ta couche sur les montagnes hautes et élevées ; c'est aussi là que tu montes pour offrir des sacrifices.
\VS{8}Tu mets ton souvenir derrière la porte et les poteaux ; car tu lèves la couverture loin de moi et tu montes, tu élargis ta couche, et c'est avec ceux-là que tu t'allies ; tu aimes leur commerce, tu choisis une place.
\VS{9}Tu voyages vers le roi avec de l'huile, et tu ajoutes parfums sur parfums ; tu envoies au loin tes messagers, tu t'abaisses jusqu'aux enfers.
\VS{10}Tu te fatigues à force de marcher, et tu ne dis pas : C'est en vain ! Tu trouves encore de la vigueur dans ta main ; c'est pourquoi tu n'as pas été languissante.
\VS{11}Et qui redoutais-tu, qui craignais-tu pour que tu me mentes, pour ne pas te souvenir, te soucier de moi ? Est-ce que je ne garde pas le silence, et depuis longtemps ? C'est pourquoi tu ne me crains pas.
\VS{12}Je vais publier ta justice et tes œuvres, et elles ne te profiteront pas.
\VS{13}Quand tu crieras, qu'elles te délivrent, les idoles que tu as amassées ! Le vent les emportera toutes, un souffle les enlèvera ; mais celui qui se confie en moi, héritera la terre et possédera ma montagne sainte\FTNT{Es. 2:3 ; Ps. 2:6 ;  Hé. 12:22.}.
\VS{14}On dira : Aplanissez, aplanissez, préparez le chemin, enlevez tout obstacle loin du chemin de mon peuple.
\TextTitle{Yahweh aime l'homme contrit}
\VS{15}Car ainsi parle le Très-Haut, dont la demeure est éternelle et dont le nom est le Saint : J'habiterai dans le lieu haut et saint ; je suis avec l'homme qui a le cœur brisé et qui est humble d'esprit, afin de vivifier l'esprit des humbles, afin de vivifier ceux qui ont le cœur brisé\FTNT{Ps. 34:19 ; Ps. 51:19.}.
\VS{16}Je ne veux pas contester à toujours, ni garder une éternelle colère, quand devant moi languissent les esprits, les âmes que j'ai faites\FTNT{Mi. 7:18 ; Ps. 85:6 ; Ps. 103:9.}.
\VS{17}A cause de l'iniquité de ses gains déshonnêtes, je me suis irrité et je l'ai frappé, j'ai caché ma face dans mon indignation ; et le rebelle a suivi la voie de son cœur.
\VS{18}J'ai vu ses voies, toutefois je le guérirai ; je le conduirai et je le consolerai, lui et ceux qui mènent deuil avec lui.
\VS{19}Je crée ce qui est proféré par les lèvres : Paix, paix à celui qui est loin et à celui qui est près ! dit Yahweh, car je le guérirai.
\VS{20}Mais les méchants sont comme la mer agitée qui ne peut se calmer, et dont les eaux rejettent la boue et le bourbier.
\VS{21}Il n'y a point de paix pour les méchants, dit mon Dieu.
\Chap{58}
\TextTitle{Le vrai et le faux jeûne}
\VerseOne{}Crie à plein gosier, ne te retiens pas, élève ta voix comme un shofar, et annonce à mon peuple leur iniquité et à la maison de Jacob ses péchés !
\VS{2}Car ils me cherchent tous les jours, ils prennent plaisir à connaître mes voies ; comme une nation qui aurait pratiqué la justice, et qui n'aurait pas abandonné la loi de son Dieu ; ils me demandent des jugements justes, ils prennent plaisir à s'approcher de Dieu, et puis ils disent :
\VS{3}Que nous sert de jeûner, si tu ne le vois pas ? D'affliger nos âmes, si tu n'y as point égard ? Voici, le jour de votre jeûne, vous faites votre volonté, et vous traitez durement tous vos mercenaires.
\VS{4}Voici, vous jeûnez pour faire des procès et des querelles, et pour frapper du poing méchamment ; vous ne jeûnez pas comme le veut ce jour, pour que votre voix soit exaucée en haut.
\VS{5}Est-ce là le jeûne que j'ai choisi, que l'homme afflige son âme un jour ? Est-ce en courbant sa tête comme le jonc et en étendant le sac et la cendre ? Appelleras-tu cela un jeûne et un jour agréable à Yahweh ?
\VS{6}Voici le jeûne que j'ai choisi : Détache les liens de la méchanceté, dénoue les cordages du joug, renvoie libres les opprimés, et que l'on rompe toute espèce de joug ;
\VS{7}partage ton pain avec celui qui a faim, et fais entrer dans ta maison les malheureux errants ; si tu vois un homme nu, couvre-le, et ne te détourne pas de ton semblable.
\TextTitle{Bénédiction pour ceux qui pratiquent le bien}
\VS{8}Alors ta lumière jaillira comme l'aurore, et ta guérison germera promptement ; ta justice marchera devant toi, et la gloire de Yahweh sera ton arrière-garde.
\VS{9}Alors tu appelleras, et Yahweh t'exaucera ; tu crieras, et il dira : Me voici ! Si tu ôtes du milieu de toi le joug, si tu cesses de lever le doigt et de dire des outrages ;
\VS{10}si tu ouvres ton âme à celui qui a faim, si tu rassasies l'âme affligée ; ta lumière se lèvera sur les ténèbres, et l'obscurité sera comme le midi.
\VS{11}Yahweh te conduira continuellement, il rassasiera ton âme dans les grandes sécheresses, il redonnera de la vigueur à tes os, et tu seras comme un jardin arrosé, comme une source dont les eaux ne tarissent pas\FTNT{Jn. 4:14 ; Ap. 21:6.}.
\VS{12}Les tiens rebâtiront les antiques ruines, tu rétabliras les fondements ruinés depuis plusieurs générations ; et on t'appellera le réparateur des brèches et le restaurateur des chemins qui rend le pays habitable.
\VS{13}Si tu retiens ton pied pendant le sabbat pour ne pas faire ta volonté en mon saint jour ; si tu appelles le sabbat tes délices, et honorable ce qui est saint à Yahweh, et si tu l'honores en ne suivant point tes voies, en ne te livrant pas à tes désirs et à des vains discours,
\VS{14}alors tu mettras ton plaisir en Yahweh, et je te ferai monter comme à cheval sur les hauteurs du pays, je te ferai jouir de l'héritage de Jacob, ton père ; car la bouche de Yahweh a parlé.
\Chap{59}
\TextTitle{Le péché sépare de Yahweh}
\VerseOne{}Voici, la main de Yahweh n'est pas trop courte pour ne pas pouvoir délivrer, ni son oreille trop dure pour entendre.
\VS{2}Mais ce sont vos iniquités qui mettent une séparation entre vous et votre Dieu ; ce sont vos péchés qui vous cachent sa face, afin qu'il ne vous écoute point\FTNT{De. 31:17-18 ; Ez. 39:23-24.}.
\VS{3}Car vos mains sont souillées de sang, et vos doigts d'iniquité ; vos lèvres profèrent le mensonge, et votre langue fait entendre la perversité.
\VS{4}Nul ne se plaint avec justice, nul ne plaide selon la vérité ; ils s'appuient sur des choses vaines et disent des faussetés, ils conçoivent le mal et enfantent l'iniquité.
\VS{5}Ils font éclore des œufs de basilic, et ils tissent des toiles d'araignée ; celui qui mange de leurs œufs meurt ; et si on les écrase, il en sort une vipère.
\VS{6}Leurs toiles ne servent point à faire des vêtements, et ils ne peuvent se couvrir de leurs ouvrages ; car leurs œuvres sont des œuvres d'iniquité, et les actions de violence sont dans leurs mains.
\VS{7}Leurs pieds courent au mal, et ils ont hâte de répandre le sang innocent ; leurs pensées sont des pensées d'iniquité ; le ravage et la ruine sont sur leurs voies.
\VS{8}Ils ne connaissent point le chemin de la paix, et il n'y a point de justice dans leurs voies, ils se sont pervertis dans leurs sentiers, tous ceux qui y marchent ignorent la paix\FTNT{Pr. 1:16 ; Pr. 6:16-19.}.
\VS{9}C'est pourquoi le jugement favorable s'est éloigné de nous, et la justice ne parvient pas jusqu'à nous ; nous attendions la lumière, et voici les ténèbres, la clarté, et nous marchons dans l'obscurité.
\VS{10}Nous tâtonnons comme des aveugles le long du mur, nous tâtonnons comme ceux qui n'ont pas d'yeux ; nous chancelons en plein midi comme la nuit, au milieu de l'abondance nous ressemblons à des morts.
\VS{11}Nous grondons tous comme des ours, et nous ne cessons de gémir comme des colombes ; nous attendions le jugement, et il n'est pas là, la délivrance, et elle s'est éloignée de nous.
\VS{12}Car nos transgressions se sont multipliées devant toi, et nos péchés témoignent contre nous ; parce que nos transgressions sont avec nous, et nous connaissons nos iniquités.
\VS{13}Nous avons été rebelles et menteurs envers Yahweh, nous nous sommes éloignés de notre Dieu, nous avons proféré la violence et la révolte, conçu et médité dans le cœur des paroles de mensonge.
\VS{14}C'est pourquoi le jugement s'est éloigné et la justice se tient éloignée ; car la vérité trébuche sur la place publique, et la droiture ne peut entrer.
\VS{15}Même la vérité a disparu, et quiconque se retire du mal est exposé au pillage ; Yahweh voit, d'un regard indigné, parce qu'il n'y a plus de droiture.
\TextTitle{Yahweh cherche un homme, il suscite le Messie}
\VS{16}Il voit qu'il n'y a pas un homme, il s'étonne de ce que personne n'intercède ; alors son bras lui vient en aide, et sa propre justice lui sert d'appui\FTNT{Es. 53:1 ; Es. 63:5 ; Ps. 77:15-16 ; Ac. 13:17.}.
\VS{17}Car il se revêt de la justice comme d'une cuirasse, et il met sur sa tête le casque du salut\FTNT{Ep. 6:14-17.} ; il se revêt de la vengeance comme d'un vêtement, et se couvre de la jalousie comme d'un manteau.
\VS{18}Il rendra à chacun selon ses œuvres\FTNT{Jé. 17:10 ; Job. 34:11 ; Mt. 16:27 ; Ap. 2:23 ; Ap. 20:13.}, la fureur à ses adversaires, la pareille à ses ennemis ; il rendra la pareille aux îles.
\VS{19}On craindra le Nom de Yahweh depuis l'occident, et sa gloire depuis le soleil levant ; car l'ennemi viendra comme un fleuve, mais l'Esprit de Yahweh lèvera la bannière\FTNT{En hébreu « Yahweh Nissi », c'est-à-dire « Yahweh est ma bannière ». C'est le nom donné par Moïse à l'autel qu'il construisit pour célébrer la défaite d'Amalek (Ex. 17:15). En No. 21:8-9, Moïse éleva une bannière sur laquelle il avait fixé un serpent d'airain pour la guérison des malades. } contre lui.
\VS{20}Le Rédempteur\FTNT{Le Rédempteur qui viendra pour Sion est le Seigneur Jésus-Christ (Ro. 11:26). Voir aussi Es. 60 : 16. } viendra pour Sion, pour ceux de Jacob qui se convertiront de leur péché, dit Yahweh.
\VS{21}Et quant à moi, c'est ici mon alliance que je ferai avec eux, dit Yahweh : Mon Esprit qui est sur toi, et mes paroles que j'ai mises dans ta bouche, ne se retireront point de ta bouche, ni de la bouche de ta postérité, ni de la bouche de la postérité de ta postérité, dit Yahweh, dès maintenant et à jamais.
\Chap{60}
\TextTitle{Yahweh lèvera sa gloire dans Sion}
\VerseOne{}Lève-toi, sois illuminée, car ta lumière arrive, et la gloire de Yahweh se lève sur toi.
\VS{2}Car voici, les ténèbres couvrent la terre, et l'obscurité couvre les peuples ; mais Yahweh se lève sur toi, et sa gloire apparaît sur toi.
\VS{3}Des nations marchent à ta lumière, et des rois à la splendeur de tes rayons\FTNT{Ap. 21:24.}.
\VS{4}Porte tes yeux alentour et regarde : Tous ils s'assemblent, ils viennent vers toi ; tes fils viennent de loin, et tes filles sont nourries par des nourriciers, étant portées sur les bras.
\VS{5}Alors tu verras et tu seras éclairée, et ton cœur s'étonnera et s'épanouira de joie, quand les richesses de la mer se tourneront vers toi, et que la puissance des nations viendra à toi.
\VS{6}Tu seras couverte d'une foule de chameaux, des dromadaires de Madian et d'Epha ; et tous ceux de Séba viendront, ils apporteront de l'or et de l'encens, et publieront les louanges de Yahweh.
\VS{7}Toutes les brebis de Kédar seront assemblées vers toi, les béliers de Nebajoth seront à ton service ; ils monteront à mon autel et me seront agréables, et je glorifierai la maison de ma gloire.
\VS{8}Qui sont ceux-là qui volent comme des nuées, comme des colombes vers leur colombier ?
\VS{9}Car les îles espèrent en moi, et les navires de Tarsis sont en tête pour ramener de loin tes enfants, avec leur argent et leur or, à cause du Nom de Yahweh, ton Dieu, et du Saint d'Israël qui te glorifie.
\VS{10}Les fils des étrangers rebâtiront tes murailles, et leurs rois seront tes serviteurs ; car je t'ai frappée dans ma colère, mais j'ai eu pitié de toi au temps de ma miséricorde.
\VS{11}Tes portes seront continuellement ouvertes, elles ne seront fermées ni nuit ni jour, afin de laisser entrer chez toi les richesses des nations, et leurs rois avec leur suite\FTNT{Ap. 21:25-26.}.
\VS{12}Car la nation et le royaume qui ne te serviront pas périront, ces nations-là seront exterminées.
\VS{13}La gloire du Liban viendra chez toi, le cyprès, l'orme, et le buis, tous ensemble pour orner le lieu de mon sanctuaire ; et je glorifierai le lieu où reposent mes pieds.
\VS{14}Les fils de tes oppresseurs viendront s'humilier devant toi, et tous ceux qui te méprisaient se prosterneront à tes pieds et t'appelleront la ville de Yahweh, la Sion du Saint d'Israël.
\VS{15}Alors que tu étais délaissée et haïe, et que personne ne te parcourait, je ferai de toi un ornement pour toujours, un sujet de joie de génération en génération.
\VS{16}Et tu suceras le lait des nations, tu suceras la mamelle des rois, et tu sauras que je suis Yahweh, ton Sauveur, ton Rédempteur\FTNT{Le verbe « ga'al » et le nom correspondant « go'el », ont été traduits respectivement en français par « racheter » et « rédempteur ». Selon la loi de Moïse, si quelqu'un perdait son héritage à cause d'une dette ou s'il se vendait comme esclave, lui et ses biens pouvaient être rachetés par un proche parent qui devait payer le prix de la rédemption (Lé. 25:23-55). Yahweh se présente comme le Rédempteur par excellence (Es. 49:26 ; Es. 60:16 ; Ps. 78:35 ; Ps. 130:7; Job. 19:25). Or Jésus-Christ «[...] a été fait pour nous sagesse, justice, sanctification et rédemption » (1 Co. 1:30). Les épîtres nous révèlent la rédemption qu'il a acquise pour nous : « nous avons la rédemption par son sang » (Ep. 1:7). La rédemption est le paiement d'une rançon, or il est écrit : « Jésus-Christ s'est donné en rançon pour nous tous » (1 Ti. 2:6). « Vous avez été rachetés à grand prix » (1 Co. 6:20).}, le Puissant de Jacob.
\VS{17}Je ferai venir de l'or au lieu de l'airain, je ferai venir de l'argent au lieu du fer, de l'airain au lieu du bois, et du fer au lieu des pierres ; je ferai régner la paix et dominer la justice.
\VS{18}On n'entendra plus parler de violence dans ton pays ni de ravage et de ruine dans ton territoire ; tu donneras à tes murailles le nom de Salut, et tes portes celui de Louange.
\VS{19}Ce ne sera plus le soleil qui te servira de lumière pendant le jour, ni la lune qui t'éclairera de sa lueur, mais Yahweh sera pour toi la lumière éternelle\FTNT{Voir le commentaire en Ge 1:3.}, et ton Dieu sera ta gloire.
\VS{20}Ton soleil ne se couchera plus, et ta lune ne se retirera plus, car Yahweh sera pour toi la lumière éternelle, et les jours de ton deuil seront passés.
\VS{21}Il n'y aura plus que des justes parmi ton peuple, ils posséderont la terre à toujours ; c'est le rejeton que j'ai planté, l'œuvre de mes mains pour servir à ma gloire\FTNT{Es. 11:1 ; Ro. 15:12 ; Ap. 5:5 ; Ap. 22:16.}.
\VS{22}La plus petite famille deviendra un millier de personnes, et la moindre deviendra une nation puissante. Je suis Yahweh, je hâterai ces choses en leur temps.
\Chap{61}
\TextTitle{La mission du messie}
\VerseOne{}L'Esprit du Seigneur Yahweh est sur moi, car Yahweh m'a oint pour évangéliser les malheureux ; il m'a envoyé pour guérir ceux qui ont le cœur brisé, pour proclamer aux captifs la liberté, et aux prisonniers l'ouverture de la prison ;
\VS{2}pour publier une année de grâce de Yahweh, et le jour de vengeance de notre Dieu ; pour consoler tous ceux qui mènent deuil\FTNT{Lu. 4:14-19.} ;
\VS{3}pour annoncer à ceux de Sion qui mènent deuil, que le diadème leur sera donné au lieu de la cendre, une huile de joie au lieu du deuil, un manteau de louange au lieu d'un esprit abattu\FTNT{Job. 29:14 ; Ja. 1:12 ; 1 Co.9:25 ; 2 Ti. 4:8.}, afin qu'on les appelle des térébinthes de la justice, une plantation de Yahweh, pour servir à sa gloire.
\VS{4}Ils rebâtiront les ruines antiques, ils relèveront d'antiques décombres, ils renouvelleront des villes ravagées, dévastées depuis longtemps.
\VS{5}Des étrangers seront là et feront paître vos troupeaux, des fils de l'étranger seront vos laboureurs et vos vignerons.
\VS{6}Mais vous, vous serez appelés sacrificateurs de Yahweh, on vous nommera serviteurs de notre Dieu\FTNT{Ap. 1:6 ; Ap. 5:10.} ; vous mangerez les richesses des nations, et vous vous glorifierez de leur gloire.
\VS{7}Au lieu de la honte que vous avez eue, les nations en auront le double, et elles crieront tout haut que la confusion est leur portion ; c'est pourquoi ils posséderont le double dans leur pays, et leur joie sera éternelle.
\VS{8}Car je suis Yahweh, j'aime la justice, je hais la rapine avec l'holocauste ; je leur donnerai fidèlement leur récompense, et je traiterai avec eux une alliance éternelle.
\VS{9}Leur race sera connue parmi les nations, et leur postérité parmi les peuples ; tous ceux qui les verront reconnaîtront qu'ils sont la race bénie de Yahweh.
\VS{10}Je me réjouirai extrêmement en Yahweh, et mon âme sera ravie d'allégresse en mon Dieu ; car il m'a revêtu des vêtements du salut, il m'a couvert du manteau de la justice, comme un époux qui se pare de magnificence, et comme une épouse qui s'orne de ses joyaux\FTNT{Os. 2:21-22 ; Ap. 19:7-8.}.
\VS{11}Car comme la terre fait éclore son germe, et comme un jardin fait germer ses semences, ainsi le Seigneur Yahweh fera germer la justice et la louange en présence de toutes les nations.
\Chap{62}
\TextTitle{Yahweh proclamme la restauration d'Israël}
\VerseOne{}Pour l'amour de Sion, je ne me tairai point, pour l'amour de Jérusalem je ne prendrai point de repos, jusqu'à ce que sa justice paraisse comme l'aurore, et sa délivrance comme un flambeau qui s'allume.
\VS{2}Alors les nations verront ta justice, et tous les rois ta gloire ; et on t'appellera d'un nouveau nom\FTNT{Ap. 2:17.}, que la bouche de Yahweh aura expressément déclaré.
\VS{3}Tu seras une couronne de gloire dans la main de Yahweh, un turban royal dans la main de ton Dieu.
\VS{4}On ne te nommera plus la délaissée, on ne nommera plus ta terre la désolation ; mais on t'appellera mon bon plaisir en elle ; on appellera ta terre l'épouse ; car Yahweh met son bon plaisir en toi, et ta terre aura un époux.
\VS{5}Comme un jeune homme épouse une vierge, comme tes fils se marient chez toi, ainsi ton Dieu se réjouira de toi, de la joie qu'un époux a de son épouse.
\VS{6}Jérusalem, j'ai placé des gardes sur tes murailles ; ils ne se tairont ni jour ni nuit. Vous qui la rappelez au souvenir de Yahweh, n'ayez point de repos !
\VS{7}Et ne vous arrêtez pas de l'invoquer jusqu'à ce qu'il rétablisse Jérusalem et la rende glorieuse sur la terre.
\VS{8}Yahweh l'a juré par sa droite et par son bras puissant : Je ne donnerai plus ton blé pour nourriture à tes ennemis, et les fils des étrangers ne boiront plus ton vin excellent pour lequel tu as travaillé.
\VS{9}Mais ceux qui auront amassé le blé le mangeront et loueront Yahweh, et ceux qui auront récolté le vin le boiront dans les parvis de ma sainteté.
\VS{10}Franchissez, franchissez les portes ! Préparez le chemin du peuple ! Frayez, frayez la route, ôtez les pierres ! Elevez une bannière vers les peuples.
\VS{11}Voici ce que Yahweh proclame aux extrémités de la terre : Dites à la fille de Sion : Voici, ton Sauveur vient\FTNT{De nombreux passages, notamment dans le livre d'Esaïe, présentent Dieu comme le sauveur, le seul sauveur (Es. 43:3 ; Es. 43:11 ; Os. 13:4) qui viendra pour délivrer son peuple (Es. 35:4 ; Es. 60:1 ; Za. 14:1-7). Jésus-Christ a accompli en tous points les prophéties relatives à la venue de Yahweh. Dieu est bel et bien venu sur terre il y a plus de 2000 ans et ce même Dieu revient bientôt (Ac. 1:11 ; Ap. 1:7).} ; voici, son salaire est avec lui, et les récompenses le précèdent.
\VS{12}On les appellera, peuple saint, les rachetés de Yahweh\FTNT{1 Pi. 2:9 ; Ap. 5:9.} ; et toi, on t'appellera, la recherchée, la ville non abandonnée.
\Chap{63}
\TextTitle{Le jour de vengeance du messie\FTNTT{Es. 2:10-22 ; Ap. 19:11-21.}}
\VerseOne{}Qui est celui-ci qui vient d'Edom, de Botsra, en vêtements rouges, en habits éclatants, marchant selon la grandeur de sa force ? C'est moi qui parle avec justice et qui ai tout pouvoir de délivrer.
\VS{2}Pourquoi tes habits sont-ils rouges, et tes vêtements comme les vêtements de ceux qui foulent dans la cuve ?
\VS{3}J'ai été seul à fouler au pressoir, et nul homme d'entre les peuples n'était avec moi ; je les ai foulés dans ma colère, je les ai écrasés dans ma fureur ; leur sang a rejailli sur mes vêtements, et j'ai souillé tous mes habits.
\VS{4}Car le jour de la vengeance était dans mon cœur, et l'année de mes rachetés est venue.
\VS{5}Je regardais, personne pour m'aider ; j'étais étonné, personne pour me soutenir ; alors mon bras m'a sauvé et ma fureur m'a soutenu.
\VS{6}Ainsi j'ai foulé des peuples dans ma colère, et je les ai enivrés dans ma fureur ; et j'ai répandu leur sang sur la terre.
\TextTitle{Esaïe confesse les péchés du peuple}
\VS{7}Je publierai les bontés de Yahweh, les louanges de Yahweh, pour tous les bienfaits que Yahweh nous a faits ; je dirai sa grande bonté envers la maison d'Israël, qu'il a traitée selon ses compassions et la richesse de sa miséricorde.
\VS{8}Il avait dit : Certainement, ils sont mon peuple, des enfants qui ne seront pas infidèles ! Et il a été pour eux un Sauveur.
\VS{9}Dans toutes leurs détresses, il a été en angoisse, et l'ange qui est devant sa face les a délivrés\FTNT{Ge. 16:7-10 ; Jg. 6:11-14 ; Za.1:11.} ; lui-même les a rachetés dans son amour et sa miséricorde, et constamment il les a soutenus et portés, aux anciens jours.
\VS{10}Mais ils ont été rebelles, et ils ont attristé son Esprit saint\FTNT{Ep. 4:30.}, c'est pourquoi il est devenu leur ennemi, il a lui-même combattu contre eux.
\VS{11}Alors son peuple se souvint des anciens jours de Moïse. Où est celui, a-t-on dit, qui les fit monter de la mer, avec les bergers de son troupeau ? Où est celui qui mettait au milieu d'eux son Esprit saint ?
\VS{12}Qui dirigea la droite de Moïse par son bras glorieux ; qui fendit les eaux devant eux pour se faire un nom éternel ?
\VS{13}Qui les dirigea à travers les flots, comme un cheval dans le désert, sans qu'ils bronchent ?
\VS{14}L'Esprit de Yahweh les a menés au repos comme on mène une bête qui descend dans la vallée. C'est ainsi que tu as conduit ton peuple, pour te faire un nom glorieux.
\VS{15}Regarde du ciel et vois de ta demeure sainte et glorieuse : Où sont ton zèle et ta puissance ? Le frémissement de tes entrailles et tes compassions se retiennent-ils envers moi ?
\VS{16}Cependant tu es notre Père, car Abraham ne nous connaît pas, et Israël ignore qui nous sommes ; Yahweh, c'est toi qui es notre Père, et ton Nom est notre Rédempteur de tout temps.
\VS{17}Pourquoi nous as-tu fait égarer loin de tes voies, ô Yahweh, et endurcis-tu notre cœur contre ta crainte ? Reviens, pour l'amour de tes serviteurs, des tribus de ton héritage !
\VS{18}Ton peuple saint n'a possédé le pays que peu de temps ; nos ennemis ont foulé ton sanctuaire.
\TextTitle{Prière du reste d'Israël à Yahweh pour sa délivrance}
\VS{19}Nous sommes depuis longtemps comme ceux que tu ne gouvernes pas, et ceux qui ne sont point appelés de ton Nom. Oh ! Si tu fendais les cieux, et si tu descendais, les montagnes s'ébranleraient devant toi !
\Chap{64}
\VerseOne{}Comme s'allume un feu de bois sec, comme s'évapore l'eau qui bouillonne, tes ennemis connaîtront ton Nom, et les nations trembleront devant toi.
\VS{2}Lorsque tu fis les choses redoutables que nous n'attendions pas, tu descendis et les montagnes s'ébranlèrent devant toi.
\VS{3}Jamais on n'a appris ni entendu dire, et jamais l'œil n'a vu qu'un autre dieu que toi fît de telles choses pour ceux qui se confient en lui\FTNT{1 Co. 2:9.}.
\VS{4}Tu vas au-devant de celui qui pratique avec joie la justice, de ceux qui marchent dans tes voies et se souviennent de toi. Mais tu as été irrité parce que nous avons péché, tu t'es irrité longtemps ; tes compassions sont éternelles, c'est pourquoi nous serons sauvés.
\VS{5}Nous sommes tous devenus comme des impurs, et toute notre justice est comme le linge le plus souillé\FTNT{Ap. 19:8.} ; nous sommes tous flétris comme une feuille, et nos iniquités nous emportent comme le vent.
\VS{6}Il n'y a personne qui invoque ton Nom, qui se réveille pour s'attacher fortement à toi ; c'est pourquoi tu nous as caché ta face, et tu nous laisses fondre par l'effet de nos iniquités.
\VS{7}Cependant, ô Yahweh, tu es notre Père ; nous sommes l'argile, et c'est toi qui nous as formés, et nous sommes tous l'ouvrage de ta main\FTNT{Es. 29:16 ; Es. 45:9 ; Jé. 18:6 ; Ro. 9:20-21.}.
\VS{8}Ne t'irrite pas à l'extrême, ô Yahweh, et ne te souviens pas à toujours de notre iniquité. Voici, regarde, nous te prions, nous sommes tous ton peuple.
\VS{9}Tes villes saintes sont devenues un désert ; Sion est devenue un désert, et Jérusalem une solitude.
\VS{10}Notre maison sainte et glorieuse, où nos pères célébraient tes louanges, a été consumée par le feu ; tout ce que nous avions de précieux a été dévasté.
\VS{11}Après cela, ô Yahweh, te contiendras-tu ? Garderas-tu le silence, et nous affligeras-tu à l'excès ?
\Chap{65}
\TextTitle{Réponse de Yahweh}
\VerseOne{}J'ai exaucé ceux qui ne me demandaient rien, et je me suis laissé trouver par ceux qui ne me cherchaient pas\FTNT{Mt. 7:7 ; Lu. 11:9.} ; j'ai dit à une nation qui ne s'appelait pas de mon Nom : Me voici, me voici !
\VS{2}J'ai tendu mes mains tous les jours vers un peuple rebelle, qui marche dans une mauvaise voie, au gré de ses pensées ;
\VS{3}vers un peuple qui m'irrite continuellement en face, sacrifiant dans les jardins, et brûlant de l'encens sur les autels de briques,
\VS{4}qui habite les sépulcres et passe la nuit dans les lieux désolés, mangeant la chair de porc, et ayant dans ses vases des mets abominables.
\VS{5}Qui dit : Retire-toi, ne m'approche pas, car je suis plus saint que toi ! De pareilles choses, c'est une fumée dans mes narines, c'est un feu ardent tout le jour.
\VS{6}Voici, cela est écrit devant moi, je ne me tairai point, mais je leur ferai porter la peine, oui je leur ferai porter la peine
\VS{7}de vos iniquités, dit Yahweh, et les iniquités de vos pères, qui ont brûlé de l'encens sur les montagnes, et qui m'ont outragé sur les collines ; c'est pourquoi je leur mesurerai aussi dans leur sein le salaire de leurs actions passées.
\VS{8}Ainsi parle Yahweh : Quand on trouve du vin dans une grappe, on dit : Ne la détruis pas, car il y a là une bénédiction ! J'agirai de même pour l'amour de mes serviteurs, afin de ne pas tout détruire.
\VS{9}Je ferai sortir de Jacob une postérité, et de Juda un héritier de mes montagnes ; et mes élus posséderont le pays, et mes serviteurs y habiteront.
\VS{10}Le Saron servira de pâturage au menu bétail, et la vallée d'Acor servira de gîte au gros bétail, pour mon peuple qui m'aura recherché.
\VS{11}Mais vous, qui abandonnez Yahweh et qui oubliez ma montagne sainte, qui dressez la table pour Gad\FTNT{Gad : Dieu de la fortune.}, et qui remplissez une coupe pour Meni\FTNT{Meni : divinité païenne assimilée à la lune et dont le nom signifie « destin, sort ou fortune ».},
\VS{12}je vous destine aussi à l'épée, et vous fléchirez les genoux pour être égorgés ; parce que j'ai appelé, et vous n'avez point répondu ; j'ai parlé, et vous n'avez point écouté ; mais vous avez fait ce qui me déplaît, et vous avez choisi ce qui me déplaît.
\VS{13}C'est pourquoi, ainsi parle le Seigneur, Yahweh : Voici, mes serviteurs mangeront, et vous aurez faim ; voici, mes serviteurs boiront, et vous aurez soif ; voici mes serviteurs se réjouiront, et vous serez confondus.
\VS{14}Voici, mes serviteurs se réjouiront avec chants de triomphe dans la joie de leur cœur ; mais vous, vous crierez dans la douleur de votre cœur, et vous vous lamenterez dans la douleur de votre esprit.
\VS{15}Vous laisserez votre nom en imprécation à mes élus ; le Seigneur Yahweh vous fera mourir ; et il donnera à ses serviteurs un autre nom.
\VS{16}Celui qui souhaitera être béni sur la terre voudra l'être par le Dieu de vérité ; et celui qui jurera sur la terre jurera par le Dieu de vérité ; car les détresses du passé seront oubliées, et même elles seront cachées à mes yeux.
\TextTitle{De nouveaux cieux et une nouvelle terre}
\VS{17}Car voici, je vais créer de nouveaux cieux et une nouvelle terre\FTNT{Es. 66:22 ; 2 Pi. 3:13 ; Ap. 21:1.} ; et on ne se souviendra plus des choses précédentes, elles ne reviendront plus au cœur.
\VS{18}Réjouissez-vous plutôt et soyez à toujours dans l'allégresse, à cause de ce que je vais créer ; car je vais créer Jérusalem pour l'allégresse, et son peuple pour la joie.
\VS{19}Je ferai de Jérusalem mon allégresse, et de mon peuple ma joie ; on n'y entendra plus le bruit des pleurs et le bruit des cris.
\VS{20}Il n'y aura plus ni enfant ni vieillard qui n'accomplissent leurs jours ; car celui qui mourra âgé de cent ans sera encore jeune ; mais le pécheur âgé de cent ans sera maudit.
\VS{21}Ils bâtiront des maisons et les habiteront ; ils planteront des vignes et en mangeront le fruit.
\VS{22}Ils ne bâtiront pas des maisons pour qu'un autre les habite ; ils ne planteront pas des vignes pour qu'un autre en mange le fruit ; car les jours de mon peuple seront comme les jours des arbres ; et mes élus jouiront de l'œuvre de leurs mains.
\VS{23}Ils ne travailleront plus en vain, et ils n'auront pas des enfants pour les voir exposés à la frayeur ; car ils seront la race bénie de Yahweh, et leurs postérités seront avec eux.
\VS{24}Avant qu'ils m'invoquent, je répondrai ; avant qu'ils aient cessé de parler, j'exaucerai.
\VS{25}Le loup et l'agneau paîtront ensemble, le lion comme le bœuf mangeront de la paille, et la poussière sera la nourriture du serpent\FTNT{Es. 2:4 ; Es. 11:6-7.}. Il ne se fera ni tort ni dommage sur toute ma montagne sainte dit Yahweh.
\Chap{66}
\TextTitle{Yahweh réprouve l'hypocrisie et agrée ceux qui le craignent}
\VerseOne{}Ainsi parle Yahweh : Le ciel est mon trône, et la terre est le marchepied de mes pieds\FTNT{Mt. 5:34-35 ; Ac. 7:49.}. Quelle maison me bâtiriez-vous, et quel serait le lieu de mon repos ?
\VS{2}Car ma main a fait toutes ces choses, et toutes ont reçu l'existence, dit Yahweh. Voici sur qui je porterai mes regards : Sur celui qui est affligé et qui a l'esprit abattu, sur celui qui craint ma parole.
\VS{3}Celui qui immole un bœuf est comme celui qui tuerait un homme ; celui qui sacrifie une brebis est comme celui qui romprait la nuque à un chien, celui qui présente une offrande est comme celui qui répandrait du sang de porc, celui qui brûle de l'encens est comme celui qui adorerait des idoles ; tous ceux-là ont choisi leurs voies, et leur âme trouve du plaisir dans leurs abominations.
\VS{4}Moi aussi je ferai attention à leurs tromperies, et je ferai venir sur eux ce qu'ils redoutent ; parce que j'ai appelé, et qu'ils n'ont point répondu, parce que j'ai parlé, et qu'ils n'ont point écouté ; mais ils ont fait ce qui est mal à mes yeux, et ils ont choisi ce qui me déplaît.
\VS{5}Ecoutez la parole de Yahweh, vous qui craignez sa parole. Voici ce que disent vos frères, qui vous haïssent et vous repoussent comme une chose abominable, à cause de mon Nom : Que Yahweh montre sa gloire ! Et que nous voyions votre joie ! Mais ils seront confondus.
\VS{6}Une voix éclatante sort de la ville, une voix sort du temple, c'est la voix de Yahweh, qui rend à ses ennemis selon leurs œuvres.
\TextTitle{Israël renaît en un jour}
\VS{7}Avant d'éprouver les douleurs, elle a enfanté ; avant que les souffrances lui viennent, elle a donné naissance à un enfant mâle.
\VS{8}Qui a jamais entendu pareille chose ? Qui a jamais vu rien de semblable ? Un pays peut-il naître en un jour ? Une nation est-elle enfantée d'un seul coup\FTNT{Cette prophétie fait allusion à la création de l'Etat d'Israël le 14 mai 1948.} ? A peine en travail, Sion a enfanté ses fils !
\VS{9}Moi qui fais enfanter les autres, ne ferais-je point enfanter Sion ? dit Yahweh. Moi, qui fais naître les autres, l'empêcherais-je d'enfanter ? dit ton Dieu.
\TextTitle{Réjouissance à Jérusalem et consolation}
\VS{10}Réjouissez-vous avec Jérusalem, faites d'elle le sujet de votre allégresse, vous tous qui l'aimez ; vous tous qui menez deuil sur elle, tressaillez avec elle de joie ;
\VS{11}afin que vous soyez allaités et rassasiés du lait de ses consolations, afin que vous suciez le lait et que vous savouriez avec bonheur la plénitude de sa gloire.
\VS{12}Car ainsi parle Yahweh : Voici, je dirigerai vers elle la paix comme un fleuve, et la gloire des nations comme un torrent débordé, et vous serez allaités, vous serez portés sur les bras et caressés sur les genoux.
\VS{13}Je vous consolerai pour vous apaiser, comme un homme que sa mère caresse pour l'apaiser, vous serez consolés dans Jérusalem.
\VS{14}Vous le verrez et votre cœur se réjouira, et vos os germeront comme l'herbe ; et la main de Yahweh sera connue de ses serviteurs ; mais il fera sentir sa colère à ses ennemis.
\TextTitle{Jugement de Yahweh}
\VS{15}Car voici, Yahweh viendra dans un feu, et ses chars seront comme un tourbillon ; il convertit sa colère en fureur, et ses menaces en flamme de feu.
\VS{16}C'est par le feu que Yahweh exercera ses jugements, c'est par son épée qu'il châtiera toute chair ; et ceux que tuera Yahweh seront en grand nombre.
\VS{17}Ceux qui se sanctifient et se purifient dans les jardins, au milieu desquels ils vont un à un, qui mangent de la chair de porc, des choses abominables, et des souris, tous ceux-là périront dit Yahweh.
\VS{18}Mais pour moi, voyant leurs œuvres et leurs pensées, le temps est venu de rassembler toutes les nations et les langues ; ils viendront et verront ma gloire.
\TextTitle{Toutes les nations adoreront Yahweh}
\VS{19}Car je mettrai un signe parmi elles, et j'enverrai leurs réchappés vers les nations, à Tarsis, à Pul, à Lud, gens tirant de l'arc, à Tubal et à Javan, aux îles lointaines, qui n'ont jamais entendu parler de moi, et qui n'ont pas vu ma gloire ; et ils publieront ma gloire parmi les nations.
\VS{20}Ils amèneront tous vos frères du milieu de toutes les nations, en offrande à Yahweh, sur des chevaux, sur des chars et des litières, sur des mulets et des dromadaires, à ma montagne sainte, à Jérusalem, dit Yahweh, comme lorsque les enfants d'Israël apportent l'offrande dans un vase pur, à la maison de Yahweh.
\VS{21}Et je prendrai aussi parmi eux des sacrificateurs, des Lévites, dit Yahweh.
\VS{22}Car comme les nouveaux cieux et la nouvelle terre que je vais faire subsisteront devant moi, dit Yahweh, ainsi subsistera votre postérité et votre nom.
\VS{23}Et de nouvelle lune à nouvelle lune, et de sabbat en sabbat, toute chair viendra se prosterner devant ma face, dit Yahweh.
\VS{24}Et quand ils sortiront dehors, ils verront les cadavres des hommes qui se sont rebellés contre moi ; car leur ver ne mourra point, et leur feu ne s'éteindra point\FTNT{Mc. 9:48.} ; et ils seront méprisés de tout le monde.
\PPE{}
\end{multicols}
