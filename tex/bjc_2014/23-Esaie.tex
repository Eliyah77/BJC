\ShortTitle{Es.}\BookTitle{Esaïe}\BFont
\noindent\hrulefill
{\footnotesize
\textit{
\bigskip
{\centering{}
\\Auteur~: Esaïe
\\(Heb.~: Yesha'yah)
\\Signification~: YAHWEH a sauvé
\\Thème~: Le Messie d'Israël
\\Date de rédaction~: 8\up{ème} siècle av. J.-C.\\}
}
\textit{
\\Prophète en Israël, Esaïe fut une figure marquante en raison du contenu et de l'impact de son message. Véritable porte-parole de Dieu, il parla de la ruine morale d'Israël, de la déportation à Babylone et des jugements de Dieu sur son peuple. Il prophétisa également sur le retour de l'exil, la restauration finale et la reconstruction de Jérusalem. Plus qu'aucun autre livre, les écrits d'Esaïe annoncent clairement la naissance du Messie, son service, sa mission rédemptrice, son sacrifice et son futur règne millénaire.
\\L'autorité et l'exactitude de ses prophéties ont été une source d'édification au fil des siècles.\bigskip
}
}
\par\nobreak\noindent\hrulefill
\begin{multicols}{2}
\Chap{1}
\TextTitle{Prophéties concernant Juda}
\VerseOne{}La vision d'Esaïe, fils d'Amots, qu'il a vue sur Juda et Jérusalem, au jour d'Ozias, de Jotham, d'Achaz, et d'Ezéchias, rois de Juda.
\VS{2}Cieux, écoutez~! Et toi, terre, prête l'oreille~! Car Yahweh parle. J'ai nourri des enfants, je les ai élevés, mais ils se sont rebellés contre moi.
\VS{3}Le bœuf connaît son possesseur, et l'âne la crèche de son maître, mais Israël n'a point de connaissance, mon peuple n'a point d'intelligence.
\VS{4}Ah~! Nation pécheresse, peuple chargé d'iniquités, race de gens méchants, enfants qui ne font que se corrompre~! Ils ont abandonné Yahweh, ils ont irrité par leur mépris le Saint d'Israël, ils se sont retirés en arrière.
\VS{5}Pourquoi serez-vous encore frappés~? Vous ajouterez la révolte~! La tête entière est malade, et tout le cœur est languissant.
\VS{6}Depuis la plante du pied jusqu'à la tête, il n'y a rien de sain en lui~: Il n'y a que blessures, meurtrissures et plaies pourries, qui n'ont été ni nettoyées, ni bandées, et dont aucune n'a été adoucie par l'huile.
\VS{7}Votre pays n'est que désolation, et vos villes sont en feu~; des étrangers dévorent votre terre sous vos yeux, et cette désolation est comme un bouleversement fait par des étrangers.
\VS{8}Car la fille de Sion est restée comme une cabane dans une vigne, comme une cabane dans un champ de concombres, comme une ville assiégée.
\VS{9}Si Yahweh des armées ne nous avait pas laissé un petit reste, qui est même bien peu, nous serions comme Sodome, nous ressemblerions à Gomorrhe.
\VS{10}Ecoutez la parole de Yahweh, chefs de Sodome, prêtez l'oreille à la loi de notre Dieu, peuple de Gomorrhe~!
\VS{11}Qu'ai-je à faire, dit Yahweh, de la multitude de vos sacrifices~? Je suis rassasié des holocaustes de béliers et de la graisse des veaux~; je ne prends point plaisir au sang des taureaux, ni des agneaux, ni des boucs\FTNT{1 S. 15:22~; Os. 8:13~; Mt. 9:13.}.
\VS{12}Quand vous venez pour vous présenter devant ma face, qui a requis cela de votre main, que vous fouliez de vos pieds mes parvis~?
\VS{13}Ne continuez plus à m'apporter de vaines offrandes~: Le parfum m'est en abomination, quant aux nouvelles lunes, aux sabbats et à la publication de vos convocations~; je ne puis plus supporter votre méchanceté ni vos assemblées solennelles.
\VS{14}Mon âme hait vos nouvelles lunes et vos fêtes solennelles~; elles me sont fâcheuses, je suis las de les supporter.
\VS{15}C'est pourquoi, quand vous étendez vos mains, je cache mes yeux de vous~; quand vous multipliez vos prières, je ne les exauce pas~; vos mains sont pleines de sang\FTNT{Es. 59:1-3~; Mi. 3:4.}.
\VS{16}Lavez-vous, purifiez-vous, ôtez de devant mes yeux la méchanceté de vos actions~; cessez de faire le mal.
\VS{17}Apprenez à bien faire, recherchez la droiture, redressez celui qui est foulé~; faites justice à l'orphelin, défendez la cause de la veuve.
\TextTitle{Invitation à l'obéissance~; restauration de la justice de Yahweh}
\VS{18}Venez maintenant, dit Yahweh, et débattons nos droits. Si vos péchés sont comme l'écarlate, ils seront blanchis comme la neige~; s'ils sont rouges comme le vermillon ils seront blanchis comme la laine.
\VS{19}Si vous obéissez volontairement, vous mangerez le meilleur du pays.
\VS{20}Mais si vous refusez d'obéir et si vous êtes rebelles, vous serez dévorés par l'épée, car la bouche de Yahweh a parlé.
\VS{21}Comment la cité fidèle est-elle devenue une prostituée~? Elle était pleine de droiture et la justice y habitait~; mais maintenant elle est pleine de meurtriers~!
\VS{22}Ton argent s'est changé en scories~; ton breuvage est mêlé d'eau.
\VS{23}Les chefs de ton peuple sont rebelles et compagnons des voleurs~; chacun d'eux aime les présents, ils courent après les récompenses~; ils ne font point droit à l'orphelin, et la cause de la veuve ne vient point devant eux.
\VS{24}C'est pourquoi le Seigneur, Yahweh des armées, le Puissant d'Israël dit~: Ah~! Je me satisferai en punissant mes adversaires, et je me vengerai de mes ennemis.
\VS{25}Et je remettrai ma main sur toi, je refondrai tes scories comme avec la potasse, et j'ôterai tout ton étain~;
\VS{26}mais je rétablirai tes juges, tels qu'ils étaient autrefois, et tes conseillers, tels qu'ils étaient au commencement\FTNT{Dans le royaume messianique, le gouvernement théocratique sera restauré et la fonction des juges sera rétablie (voir livre des Juges~; Mt. 19:28~; 1 Co. 6:2-3).}. Après cela, on t'appellera cité de la justice, ville fidèle.
\VS{27}Sion sera rachetée par la droiture et ceux qui s'y convertiront seront rachetés par la justice.
\VS{28}Mais les rebelles et les pécheurs seront détruits ensemble, et ceux qui abandonnent Yahweh seront consumés.
\VS{29}Car on sera honteux à cause des térébinthes que vous avez désirés, et vous rougirez à cause des jardins que vous avez choisis\FTNT{Des cultes idolâtres avaient lieu autour des térébinthes et dans des jardins (De. 16:21~; Es. 57:4-5~; Es. 65:3~; Jé. 2:20~; Ez. 20:28~; Os. 4:13).}.
\VS{30}Car vous serez comme le térébinthe dont le feuillage tombe, et comme un jardin qui n'a pas d'eau.
\VS{31}Et le fort sera de l'étoupe, et son œuvre une étincelle~; et tous deux brûleront ensemble, et il n'y aura personne pour éteindre le feu.
\Chap{2}
\TextTitle{Vision du règne messianique}
\VerseOne{}La parole qu'Esaïe, fils d'Amots, a vue sur Juda et Jérusalem.
\VS{2}Or il arrivera, dans les derniers jours\FTNT{Voir Ge. 49:1-2.}, que la montagne de la maison de Yahweh sera affermie au sommet des montagnes, qu'elle sera élevée par-dessus les collines et que toutes les nations y afflueront.
\VS{3}Et plusieurs peuples iront et diront~: Venez et montons à la montagne de Yahweh, à la maison du Dieu de Jacob~; il nous instruira ses voies, et nous marcherons dans ses sentiers~; car la loi sortira de Sion, et la parole de Yahweh sortira de Jérusalem.
\VS{4}Il exercera le jugement parmi les nations, et reprendra plusieurs peuples. De leurs épées ils forgeront des hoyaux, et de leurs lances des serpes~; une nation ne lèvera plus l'épée contre une autre et ils ne s'adonneront plus à la guerre.
\VS{5}Venez, ô maison de Jacob, et marchons dans la lumière de Yahweh.
\TextTitle{L'orgueilleux abaissé au jour de Yahweh}
\VS{6}Certes tu as rejeté ton peuple, la maison de Jacob, parce qu'ils se sont remplis d'orient et adonnés à la divination comme les Philistins, et parce qu'ils s'allient aux enfants des étrangers\FTNT{De. 18:8-13~; Os. 13:2~; Mi. 5:11-13.}.
\VS{7}Son pays est rempli d'argent et d'or, et il n'y a pas de fin à ses trésors~; son pays est rempli de chevaux, et il n'y a pas de fin à ses chars.
\VS{8}Son pays est rempli d'idoles~; ils se prosternent devant l'ouvrage de leurs mains et devant ce que leurs doigts ont fabriqué.
\VS{9}Et ceux du commun sont abattus, et les personnes de qualité sont abaissées~; ne leur pardonne donc point.
\VS{10}Entre dans les rochers et cache-toi dans la poussière, à cause de la frayeur de Yahweh, et à cause de la gloire de sa majesté\FTNT{Ap. 6:15-16.}.
\VS{11}Les yeux hautains des hommes seront abaissés et les hommes qui s'élèvent seront humiliés, Yahweh sera seul haut élevé en ce jour-là.
\VS{12}Car il y a un jour assigné par Yahweh des armées contre tout homme orgueilleux et hautain, et contre tout homme qui s'élève, afin qu'il soit abaissé~;
\VS{13}contre tous les cèdres du Liban, hauts et élevés, et contre tous les chênes de Basan~;
\VS{14}contre toutes les hautes montagnes, et contre toutes les collines élevées~;
\VS{15}contre toutes les hautes tours, et contre toutes les murailles fortes~;
\VS{16}contre tous les navires de Tarsis, et contre toutes les peintures de plaisance.
\VS{17}Et l'arrogance des hommes sera humiliée, et les hommes qui s'élèvent seront abaissés~:
\VS{18}Yahweh seul sera élevé en ce jour-là. Quant aux idoles, elles tomberont toutes.
\VS{19}Et les hommes entreront dans les cavernes des rochers et dans les trous de la terre, à cause de la frayeur de Yahweh et à cause de sa gloire magnifique, lorsqu'il se lèvera pour faire trembler la terre.
\VS{20}En ce jour-là, les hommes jetteront aux taupes et aux chauves-souris leurs idoles d'argent et leurs idoles d'or, qu'ils s'étaient faites pour se prosterner devant elles~;
\VS{21}et ils entreront dans les fentes des rochers et dans les creux des rochers, à cause de la frayeur de Yahweh, et à cause de sa gloire magnifique, quand il se lèvera pour punir la terre.
\VS{22}Retirez-vous de l'homme, dans les narines duquel il n'y a qu'un souffle~: Car quel cas mérite-t-il qu'on en fasse~?
\Chap{3}
\TextTitle{Le péché, cause de dissolution nationale}
\VerseOne{}Car voici, le Seigneur, Yahweh des armées, va ôter de Jérusalem et de Juda tout appui et toute ressource, toute ressource de pain et toute ressource d'eau~;
\VS{2}l'homme fort et l'homme de guerre, le juge et le prophète, le devin et l'ancien,
\VS{3}le chef de cinquante et l'homme d'autorité, le conseiller, l'expert d'entre les artisans et l'habile enchanteur.
\VS{4}Et je leur donnerai de jeunes gens pour chefs, et des enfants domineront sur eux.
\VS{5}Le peuple sera opprimé~; l'un opprimera l'autre, chacun son prochain. Le jeune homme se portera arrogamment contre le vieillard, et l'homme de rien contre l'honorable.
\VS{6}Même un homme ira jusqu'à saisir son frère dans la maison de son père et lui dira~: Tu as un manteau, sois notre chef~! Et prends en main ces ruines~!
\VS{7}Ce jour même il répondra~: Je ne suis pas médecin, et dans ma maison il n'y a ni pain ni manteau~; ne m'établissez donc pas chef du peuple.
\VS{8}Certes Jérusalem est renversée, et Juda est tombée, parce que leurs langues et leurs actions sont contre Yahweh, pour braver les regards de sa gloire.
\VS{9}L'aspect de leur visage témoigne contre eux, ils publient leur péché comme Sodome, ils ne le cachent pas. Malheur à leur âme, car ils ont attiré le mal sur eux~!
\VS{10}Dites au juste que du bien lui arrivera, car il mangera le fruit de ses œuvres.
\VS{11}Malheur au méchant qui ne cherche qu'à faire le mal, car la rétribution de ses mains lui sera rendue.
\VS{12}Quant à mon peuple, il a pour oppresseur des enfants, et des femmes dominent sur lui. Mon peuple, ceux qui te conduisent t'égarent, ils corrompent le chemin dans lequel tu marches.
\VS{13}Yahweh se présente pour plaider, il se tient debout pour juger les peuples.
\VS{14}Yahweh entre en jugement avec les anciens de son peuple et avec ses chefs~; car vous avez brouté la vigne, et ce que vous avez ravi au pauvre est dans vos maisons.
\VS{15}Que vous revient-il de fouler mon peuple, et d'écraser le visage des affligés~? dit le Seigneur, Yahweh des armées.
\TextTitle{Les filles hautaines de Sion}
\VS{16}Yahweh dit aussi~: Parce que les filles de Sion sont hautaines, et qu'elles marchent le cou tendu et les yeux pleins de convoitise, parce qu'elles marchent avec une fière démarche faisant du bruit avec leurs pieds,
\VS{17}Yahweh rendra chauve le sommet de la tête des filles de Sion, Yahweh découvrira leur nudité.
\VS{18}En ce temps-là, le Seigneur ôtera l'ornement de leurs anneaux de cheville, et les filets et les croissants~;
\VS{19}les pendants d'oreilles, les bracelets et les voiles~;
\VS{20}les parures de la tête, les chaînettes des pieds et les ceintures, les boîtes à parfum et les amulettes~;
\VS{21}les anneaux et les bagues qui leur pendent sur le nez~;
\VS{22}les vêtements de fête et les larges tuniques, les manteaux et les gibecières~;
\VS{23}les miroirs et les chemises fines, les tiares et les voiles légers.
\VS{24}Et il arrivera qu'au lieu du parfum, il y aura de la puanteur~; au lieu de ceintures, des cordes~; au lieu de cheveux bouclés, des têtes chauves~; au lieu de robes flottantes, des sacs étroits~; et au lieu d'un beau teint, un teint tout hâlé.
\VS{25}Tes hommes tomberont par l'épée et ta force par la guerre.
\VS{26}Et ses portes gémiront et mèneront deuil~; désolée, elle s'assiéra par terre.
\Chap{4}
\TextTitle{Vision du règne messianique\FTNTT{Es. 11:1-16.}}
\VerseOne{}Et en ce jour sept femmes saisiront un seul homme, et diront~: Nous mangerons notre pain, et nous nous vêtirons de nos habits~; seulement fais-nous porter ton nom~; ôte notre opprobre.
\VS{2}En ce temps-là, le germe de Yahweh\FTNT{Jésus est le «~germe~» de Yahweh (Es. 4:2) et le germe de David (Jé. 23:5~; Za. 3:8~; Za. 6:12). Ce germe a été placé par la vertu du Saint-Esprit dans le sein d'une vierge (Es. 7:14~; Lu. 1:34-35) et l'enfant qui naquit d'elle fut appelé «~Fils de Dieu~» tout en étant le Dieu Tout-Puissant. Il existe de toute éternité en forme de Dieu (Jn. 1:1~; Es. 9:5), mais il a été fait chair pour nous sauver (Jn. 1:14.~; 1 Ti. 3:16). C'est le plus grand des miracles et la démonstration de sa divinité, de sa sagesse et de son amour envers les hommes.} sera plein de noblesse et de gloire, et le fruit de la terre plein de grandeur et d'excellence pour les réchappés d'Israël.
\VS{3}Et il arrivera que les restes de Sion, et les restes de Jérusalem, seront appelés saints~; et ceux de Jérusalem seront inscrits parmi les vivants\FTNT{Es. 10:20-22~; Ro. 9:27~; Ro. 11:5.}.
\VS{4}Quand le Seigneur aura lavé la souillure des filles de Sion, et purifié Jérusalem du sang qui est au milieu d'elle, par l'esprit de jugement et par l'esprit qui consume~;
\VS{5}aussi Yahweh créera, sur toute l'étendue du mont Sion et sur ses assemblées, une nuée avec une fumée pendant le jour, et une splendeur de feu flamboyant pendant la nuit, car la gloire se répandra partout.
\VS{6}Et il y aura un tabernacle pour donner de l'ombre contre la chaleur du jour, pour servir de refuge et d'asile contre la tempête et la pluie\FTNT{Ap. 21:3.}.
\Chap{5}
\TextTitle{Israël, vigne de Yahweh}
\VerseOne{}Je chanterai maintenant pour mon bien-aimé le cantique de mon bien-aimé sur sa vigne. Mon bien-aimé avait une vigne sur un coteau fertile.
\VS{2}Il l'environna d'une haie, en ôta les pierres, et y planta des ceps exquis~; il bâtit une tour au milieu d'elle, et il y creusa aussi une cuve. Puis il espéra qu'elle produirait des raisins, mais elle a produit des grappes sauvages\FTNT{Lu. 13:6-9.}.
\VS{3}Maintenant donc, vous habitants de Jérusalem et vous hommes de Juda, jugez, je vous prie, entre moi et ma vigne.
\VS{4}Qu'y avait-il encore à faire à ma vigne que je ne lui aie fait~? Pourquoi, quand j'ai attendu qu'elle produirait des raisins, a-t-elle produit des grappes sauvages~?
\VS{5}Maintenant donc je vous dirai ce que je vais faire à ma vigne~: J'ôterai sa haie, et elle sera broutée~; je romprai sa clôture, et elle sera foulée.
\VS{6}Et je la réduirai en désert, elle ne sera plus taillée, ni cultivée~; les ronces et les épines y croîtront~; et je commanderai aux nuées qu'elles ne laissent plus tomber de pluie sur elle.
\VS{7}Or la maison d'Israël est la vigne de Yahweh des armées, et les hommes de Juda sont la plante en laquelle il prenait plaisir. Il en attendait de la droiture, et voici du saccagement~! De la justice, et voici des cris de détresse~!
\TextTitle{Six malheurs en punition de l'infidélité d'Israël}
\VS{8}Malheur à ceux qui ajoutent maison à maison, et qui joignent champ à champ, jusqu'à ce qu'il n'y ait plus d'espace et qu'ils habitent seuls au milieu du pays.
\VS{9}Yahweh des armées a dit à mes oreilles~: Certainement, ces maisons nombreuses seront réduites en désolation, ces maisons grandes et belles seront sans habitants.
\VS{10}Même dix arpents de vigne ne produiront qu'un bath, et un homer de semence ne produira qu'un épha.
\VS{11}Malheur à ceux qui se lèvent de bon matin, qui recherchent les boissons fortes, qui demeurent jusqu'au soir, et jusqu'à ce que le vin les échauffe~!
\VS{12}La harpe et le luth, le tambourin, la flûte et le vin sont dans leurs festins~; mais ils ne regardent pas l'œuvre de Yahweh, et ils ne voient pas l'ouvrage de ses mains.
\VS{13}C'est pourquoi mon peuple sera emmené captif, parce qu'il n'a pas de connaissance\FTNT{2 R. 24:14-16~; Os. 4:6.}~; et les plus honorables parmi eux seront des pauvres qui mourront de faim, et leur multitude sera asséchée par la soif.
\VS{14}C'est pourquoi le scheol s'élargit, il ouvre sa gueule outre mesure~; et sa magnificence y descend, sa multitude, sa pompe et tous ceux qui s'y réjouissent.
\VS{15}Ceux du commun seront abattus, les personnes de qualité seront humiliées, et les yeux des hautains seront humiliés.
\VS{16}Et Yahweh des armées sera haut élevé en jugement, et le Dieu saint sera sanctifié dans la justice.
\VS{17}Les agneaux paîtront selon qu'ils seront parqués, et les étrangers dévoreront les champs désolés des riches.
\VS{18}Malheur à ceux qui tirent l'iniquité avec des cordes de vanité, et le péché avec les traits d'un char,
\VS{19}et qui disent~: Qu'il hâte et qu'il fasse venir son œuvre bientôt, afin que nous la voyions~! Que le conseil du Saint d'Israël s'avance et vienne, afin que nous le connaissions~!
\VS{20}Malheur à ceux qui appellent le mal bien et le bien mal\FTNT{Mi. 7:2.}~; qui font les ténèbres lumière, et la lumière ténèbres~; qui font l'amertume douceur, et la douceur amertume.
\VS{21}Malheur à ceux qui sont sages à leurs yeux, en se considérant eux-mêmes intelligents~!
\VS{22}Malheur à ceux qui sont forts pour boire le vin et vaillants pour mêler des boissons fortes~;
\VS{23}qui justifient le méchant pour des présents, et qui ôtent à chacun des justes sa justice.
\VS{24}C'est pourquoi, comme le flambeau de feu consume le chaume, et la flamme consume l'herbe sèche, ainsi leur racine sera comme la pourriture, et leur fleur sera détruite comme la poussière~; parce qu'ils ont rejeté la loi de Yahweh des armées, et ils ont méprisé la parole du Saint d'Israël.
\VS{25}C'est pourquoi la colère de Yahweh s'enflamme contre son peuple, il étend sa main sur lui, et il le frappe~; les montagnes tremblent, et leurs cadavres ont été mis en pièces au milieu des rues. Malgré tout cela, sa colère ne se détourne pas, mais sa main est encore étendue.
\VS{26}Il élève une bannière pour les nations éloignées, et il siffle à chacune d'elles depuis les extrémités de la terre~; et voici chacune viendra promptement et légèrement.
\VS{27}Nul n'est fatigué, nul ne chancelle de lassitude, personne ne sommeille ni ne dort~; la ceinture de leurs reins ne sera point déliée, et la courroie de leurs souliers ne sera point rompue.
\VS{28}Leurs flèches sont aiguës et tous leurs arcs tendus~; les sabots de leurs chevaux ressemblent à des cailloux, et les roues de leurs chars à un tourbillon.
\VS{29}Leur rugissement est comme celui d'un vieux lion~; ils rugissent comme des lionceaux~; ils grondent et saisissent la proie, ils l'emportent et personne ne vient à son secours.
\VS{30}En ce jour-là, on mènera un bruit sur lui, semblable au mugissement de la mer~; en regardant la terre, on ne verra que ténèbres et détresse~; la lumière sera obscurcie dans le ciel.
\Chap{6}
\TextTitle{Révélation de Yahweh à Esaïe}
\VerseOne{}L'année de la mort du roi Ozias, je vis le Seigneur assis sur un trône haut et élevé, et les pans de sa robe remplissaient le temple\FTNT{2 Ch. 26:23.}.
\VS{2}Les séraphins se tenaient au-dessus de lui~; et chacun d'eux avait six ailes~; deux dont ils se couvraient la face, deux dont ils se couvraient les pieds et deux dont ils se servaient pour voler.
\VS{3}Et ils criaient l'un à l'autre, et disaient~: Saint, saint, saint est Yahweh des armées~! Toute la terre est pleine de sa gloire~!
\VS{4}Et les poteaux des seuils furent ébranlés dans leurs fondements par la voix de celui qui criait~; et la maison fut remplie de fumée.
\VS{5}Alors je dis~: Malheur à moi~! Je suis perdu, car je suis un homme dont les lèvres sont impures, j'habite au milieu d'un peuple dont les lèvres sont impures et mes yeux ont vu le Roi, Yahweh des armées\FTNT{Jg. 13:21-22.}.
\VS{6}Mais l'un des séraphins vola vers moi, tenant à la main un charbon ardent, qu'il avait pris sur l'autel avec des pincettes.
\VS{7}Il en toucha ma bouche, et dit~: Voici, ceci a touché tes lèvres, c'est pourquoi ton iniquité est ôtée, et la propitiation est faite pour ton péché.
\VS{8}Puis j'entendis la voix du Seigneur, disant~: Qui enverrai-je et qui marchera pour nous~? Je répondis~: Me voici, envoie-moi.
\TextTitle{Mission d'Esaïe}
\VS{9}Et il dit~: Va et dis à ce peuple~: Entendez, entendez, mais vous ne comprendrez point~; et voyez, voyez, mais vous n'apercevrez point.
\VS{10}Engraisse le cœur de ce peuple, et rends ses oreilles pesantes, et bouche-lui les yeux~; de peur qu'il ne voie de ses yeux, et qu'il n'entende de ses oreilles, et que son cœur ne comprenne, et qu'il ne se convertisse, et qu'il ne recouvre la santé\FTNT{Mt. 13:15~; Mc. 4:12~; Jn. 12:40~; Ac. 28:27.}.
\VS{11}Je dis~: Jusqu'à quand, Seigneur~? Et il répondit~: Jusqu'à ce que les villes soient dévastées, jusqu'à ce qu'il n'y ait plus d'habitants, ni d'hommes dans les maisons, et que la terre soit mise en entière désolation~;
\VS{12} et que Yahweh ait dispersé au loin les hommes, et que l'abandon ait été grand au milieu du pays.
\VS{13}Toutefois s'il y reste un dixième des habitants, ils reviendront pour être la proie des flammes. Mais comme le térébinthe et le chêne conservent leur tronc quand ils sont abattus, une sainte postérité renaîtra de ce peuple\FTNT{Ro. 11:17-25.}.
\Chap{7}
\TextTitle{Retsin et Pékach complote contre Juda}
\VerseOne{}Or il arriva du temps d'Achaz, fils de Jotham, fils d'Ozias, roi de Juda, que Retsin, roi de Syrie, et Pékach, fils de Remalia, roi d'Israël, montèrent contre Jérusalem pour lui faire la guerre~; mais ils ne purent l'assiéger.
\VS{2}Et on rapporta à la maison de David~: La Syrie s'est reposée sur Ephraïm. Et le cœur d'Achaz, et le cœur de son peuple furent ébranlés comme les arbres des forêts qui sont ébranlés par le vent.
\VS{3}Alors Yahweh dit à Esaïe~: Sors maintenant au devant d'Achaz, toi et Schear-Jaschub, ton fils, vers l'extrémité de l'aqueduc de l'étang supérieur, sur la route du champ du foulon.
\VS{4}Et dis-lui~: Prends garde à toi, et demeure tranquille, ne crains point, et que ton cœur ne devienne point lâche à cause des deux queues de ces tisons fumants, à cause de l'ardeur, dis-je, de la colère de Retsin et de la Syrie, et du fils de Remalia,
\VS{5}de ce que la Syrie délibère avec Ephraïm et le fils de Remalia de te faire du mal, en disant~:
\VS{6}Montons contre Juda, assiégeons la ville, battons-la en brèche, et établissons pour roi le fils de Tabeel au milieu d'elle.
\VS{7}Ainsi parle le Seigneur, Yahweh~: Cela n'aura point d'effet, et cela ne se fera point.
\VS{8}Car la tête de la Syrie c'est Damas, et le chef de Damas c'est Retsin. Encore soixante-cinq ans, Ephraïm sera froissé pour n'être plus un peuple.
\VS{9}Et la tête d'Ephraïm c'est la Samarie, et le chef de la Samarie c'est le fils de Remalia. Si vous ne croyez pas, certainement vous ne serez point affermis.
\TextTitle{Annonce de la naissance d'Emmanuel}
\VS{10}Et Yahweh parla de nouveau à Achaz, en disant~:
\VS{11}Demande pour toi un signe à Yahweh, ton Dieu, demande-le, soit dans les bas lieux, soit dans les lieux élevés.
\VS{12}Et Achaz répondit~: Je ne demanderai rien, et je ne tenterai point Yahweh.
\VS{13}Alors Esaïe dit~: Ecoutez maintenant, ô maison de David~! Est-ce trop peu pour vous de lasser les hommes, que vous lassiez aussi mon Dieu~?
\VS{14}C'est pourquoi le Seigneur lui-même vous donnera un signe~: Voici, une vierge sera enceinte, et elle enfantera un fils, et elle lui donnera le nom d'Emmanuel\FTNT{Le nom «~Emmanuel~» est dérivé de l'hébreu «~Immanuw'el~» qui signifie «~Dieu est avec nous~». Mt. 28:20~: «~Et moi, je suis avec vous tous les jours jusqu'à la fin des temps~». Jésus est Emmanuel, Dieu avec nous jusqu'à la fin des temps.}.
\VS{15}Il mangera du lait et du miel, jusqu'à ce qu'il sache rejeter le mal et choisir le bien.
\VS{16}Mais avant que l'enfant sache rejeter le mal et choisir le bien, la terre que tu as en détestation sera abandonnée par ses deux rois.
\TextTitle{Prophétie sur l'imminente invasion de Juda\FTNTT{2 Ch. 28:1-20.}}
\VS{17}Yahweh fera venir sur toi, sur ton peuple et sur la maison de ton père, par le roi d'Assyrie, des jours tels qu'il n'y en a point eu de semblable depuis le jour où Ephraïm s'est séparé de Juda.
\VS{18}Et il arrivera qu'en ce jour-là, Yahweh sifflera aux mouches qui sont à l'extrémité des ruisseaux d'Egypte, et aux abeilles qui sont au pays d'Assyrie.
\VS{19}Elles viendront, et se poseront dans toutes les vallées désertes, et dans les fentes des rochers, et par tous les buissons, et par tous les halliers.
\VS{20}En ce jour-là, le Seigneur rasera avec le rasoir pris à louage au-delà du fleuve, avec le roi d'Assyrie, la tête et les poils des pieds, et il enlèvera aussi la barbe\FTNT{2 R. 16:5-9.}.
\VS{21}Et il arrivera, en ce jour-là, qu'un homme nourrira une jeune vache et deux brebis.
\VS{22}Et il arrivera que de l'abondance du lait qu'elles rendront, il mangera du beurre~; car tous ceux qui seront restés dans le pays mangeront du beurre et du miel.
\VS{23}Et il arrivera, en ce jour-là, que tout lieu où il y aura mille vignes, valant mille sicles d'argent, sera réduit en ronces et en épines.
\VS{24}On y entrera avec des flèches et avec l'arc, car tout le pays ne sera que ronces et épines.
\VS{25}Et dans toutes les montagnes que l'on cultivait avec la bêche, on ne craindra plus de voir des ronces et des épines~; mais on y lâchera les bœufs, et la brebis en foulera le sol.
\Chap{8}
\TextTitle{Annonce de la défaite de Damas et de la Samarie}
\VerseOne{}Et Yahweh me dit~: Prends un grand rouleau et écris dessus en grosses lettres~: Qu'on se dépêche de butiner, qu'on se hâte de piller.
\VS{2}Et je pris avec moi des témoins fidèles~: Urie, le prêtre, et Zacharie, fils de Bérékia.
\VS{3}Puis je m'étais approché de la prophétesse~; elle conçut et elle enfanta un fils. Et Yahweh me dit~: Donne-lui pour nom Maher-Schalal-Chasch-Baz\FTNT{«~Maher-Schalal-Chasch-Baz~» signifie «~rapide au butin, rapide sur la proie~».}.
\VS{4}Car avant que l'enfant sache dire~: Mon père~! Ma mère~! On enlèvera la puissance de Damas et le butin de Samarie, devant le roi d'Assyrie.
\VS{5}Et Yahweh continua encore de me parler, en disant~:
\VS{6}Parce que ce peuple a rejeté les eaux de Siloé qui coulent doucement, et qu'il s'est réjoui au sujet de Retsin, et du fils de Remalia,
\VS{7}à cause de cela, voici, le Seigneur va faire monter contre eux les puissantes et grandes eaux du fleuve~: Le roi d'Assyrie et toute sa gloire. Il s'élèvera partout au-dessus de son lit, et il se répandra sur toutes ses rives.
\VS{8}Et il pénétrera dans Juda, il débordera et inondera, il atteindra jusqu'au cou. Et les étendues de ses ailes rempliront la largeur de ton pays, ô Emmanuel~!
\TextTitle{Exhortation aux disciples de Yahweh à rester fidèles}
\VS{9}Alliez-vous, peuples~! Et vous serez brisés~; prêtez l'oreille, vous tous qui êtes d'un pays éloigné~! Equipez-vous, et vous serez brisés~; équipez-vous, et vous serez brisés.
\VS{10}Prenez conseil, et il sera dissipé~; dites la parole, et elle sera sans effet~: Car Dieu est avec nous.
\VS{11}Car ainsi m'a parlé Yahweh, avec une main forte, et il m'instruisit de ne point aller par le chemin de ce peuple-ci, en me disant~:
\VS{12}Ne dites point~: Conjuration, toutes les fois que ce peuple dit conjuration~; ne craignez point ce qu'il craint, et ne vous en épouvantez point.
\VS{13}Sanctifiez Yahweh des armées, lui-même, c'est lui que vous devez craindre et redouter.
\VS{14}Et il sera un sanctuaire, mais aussi une pierre d'achoppement\FTNT{Yahweh s'est présenté comme une pierre d'achoppement et un rocher de scandale. En Es. 44:8 il affirme ne pas connaître d'autre rocher que lui. Esaïe n'est pas le seul prophète à qui le Seigneur s'est révélé comme étant une pierre et un rocher. Dans le Ps. 118:22-23, il est dit~: «~La pierre qu'ont rejetée ceux qui bâtissaient est devenue la principale de l'angle~». Daniel et Zacharie ont également prophétisé au sujet de cette pierre~: «~Tu regardais, lorsqu'une pierre se détacha sans le secours d'aucune main, frappa les pieds de fer et d'argile de la statue, et les mit en pièces. Mais la pierre qui avait frappé la statue devint une grande montagne, et remplit toute la terre~» (Da. 2:34-35). «~Car voici, pour ce qui est de la pierre que j'ai placée devant Josué, il y a sept yeux sur cette seule pierre~; voici, je graverai moi-même ce qui doit y être gravé, dit Yahweh des armées~; et j'enlèverai l'iniquité de ce pays, en un jour~» (Za. 3:9). Ces prophéties se sont accomplies en Jésus-Christ, l'Agneau de Dieu qui ôte le péché du monde (Jn. 1:29). Le Seigneur s'est d'ailleurs clairement identifié à la pierre angulaire, affirmant ainsi sa divinité (Lu. 20:17-19). En Mt. 16:18, il s'est présenté comme le rocher inébranlable sur lequel il allait bâtir son Eglise. De plus, il est à noter que dans le livre de l'Apocalypse, l'Agneau possède sept yeux comme la pierre vue par Zacharie (Ap. 5:6). Ces sept yeux sont aussi les sept lampes du chandelier d'or que Zacharie et Jean avaient également vues (Za. 4:2~; Ap. 4:5 ). Or le chiffre sept symbolise la plénitude et la perfection divines. Esaïe prophétisa encore en ces termes~: «~Voici, j'ai mis pour fondement en Sion une pierre, une pierre éprouvée, une pierre angulaire de prix, solidement posée~; celui qui la prendra pour appui n'aura point hâte de fuir~» (Es. 28:16). Les écrits de la Nouvelle Alliance attestent l'accomplissement de cette prophétie en Jésus-Christ, notamment par la bouche de Paul et de Pierre~: «~Vous avez été édifiés sur le fondement des apôtres et des prophètes, Jésus-Christ lui-même étant la pierre angulaire~» (Ep. 2:20). «~Car personne ne peut poser un autre fondement que celui qui a été posé, savoir Jésus-Christ~» (1 Co. 3:11). «~Approchez-vous de lui, pierre vivante, rejetée par les hommes, mais choisie et précieuse devant Dieu~; et vous-mêmes, comme des pierres vivantes, édifiez-vous pour former une maison spirituelle, une sainte prêtrise, afin d'offrir des victimes spirituelles, agréables à Dieu, par Jésus-Christ~». (1 Pi. 2:4-5).}, un rocher de scandale pour les deux maisons d'Israël, un filet et un piège pour les habitants de Jérusalem.
\VS{15}Plusieurs d'entre eux trébucheront, ils tomberont et se briseront, ils seront enlacés et pris.
\VS{16}Enveloppe ce témoignage, scelle cette loi\FTNT{«~towrah~» ou «~torah~» en hébreu.} parmi mes disciples.
\VS{17}Je m'attends à Yahweh, qui cache sa face à la maison de Jacob, et je regarde à lui.
\VS{18}Me voici, avec les enfants que Yahweh m'a donnés, pour être un signe et un miracle en Israël, de la part de Yahweh des armées, qui habite sur la montagne de Sion.
\VS{19}Si l'on vous dit~: Consultez ceux qui évoquent les morts et les diseurs de bonne aventure, qui poussent des sifflements et des soupirs, répondez~: Un peuple ne consultera-t-il pas son Dieu~? S'adressera-t-il aux morts en faveur des vivants~?
\VS{20}A la loi et au témoignage~! Si l'on ne parle pas ainsi, il n'y aura certainement point d'aurore pour le peuple.
\VS{21}Et il sera errant dans le pays, accablé et affamé~; et il arrivera que dans sa faim, il s'irritera, maudira son roi et son Dieu, et tournera les yeux en haut~;
\VS{22}puis il regardera vers la terre, et voici, il n'y aura que détresse, ténèbres et de sombres angoisses~: Il sera enfoncé dans l'obscurité.
\VS{23}Mais les ténèbres ne seront pas autant qu'elles avaient été là où il y a de la détresse~; si au commencement Dieu affligea légèrement le pays de Zabulon et le pays de Nephthali, dans l'avenir, il couvrra de gloire la route de la mer, au-delà du Jourdain, dans la Galilée des Gentils.
\Chap{9}
\TextTitle{Annonce de la naissance et du règne du Messie}
\VerseOne{}Le peuple qui marchait dans les ténèbres voit une grande lumière, et la lumière resplendit sur ceux qui habitaient le pays de l'ombre de la mort\FTNT{Mt. 4:15-16.}.
\VS{2}Tu multiplies la nation, tu lui accordes de grandes joies, ils se réjouissent devant toi, comme on se réjouit à la moisson, comme on se réjouit quand on partage le butin.
\VS{3}Car tu as mis en pièces le joug dont il était chargé, et le bâton dont on lui battait ordinairement les épaules, et la verge de celui qui l'opprimait, comme au jour de Madian.
\VS{4}Parce que toute bataille de guerrier se fait dans un bruit confus, et que le vêtement est vautré dans le sang~; mais ceci sera comme un embrasement, quand le feu dévore quelque chose.
\VS{5}Car un enfant nous est né, un Fils nous a été donné\FTNT{Jésus-Christ est 100\% Dieu et 100\% homme. Il existe depuis toute éternité en tant que Dieu. Il est devenu homme au moment de son incarnation (Ph. 2:5-7).}, et l'empire reposera sur son épaule~: On l'appellera l'Admirable, le Conseiller, le Dieu Puissant, le Père d'éternité\FTNT{Philippe, disciple de Jésus-Christ voulait rencontrer le Père. Il posa au Seigneur cette question «~Seigneur, montre-nous le Père, et cela nous suffit~» (Jn. 14:8). Jésus lui répondit~: «~Il y a si longtemps que je suis avec vous, et tu ne m'as pas connu, Philippe~!~» (Jn. 14:9).}, le Prince de paix,
\VS{6}pour accroître l'empire, et une paix sans fin au trône de David et à son royaume, pour l'affermir et le soutenir par le droit et par la justice, dès maintenant et à toujours\FTNT{Lu. 1:32-33.}. Voilà ce que fera le zèle de Yahweh des armées.
\TextTitle{Jugement sur le royaume du nord}
\VS{7}Le Seigneur envoie une parole à Jacob, et elle tombe sur Israël\FTNT{Ge. 32:28.}.
\VS{8}Et tout le peuple en aura connaissance, Ephraïm et les habitants de Samarie, qui disent avec orgueil et avec un cœur hautain~:
\VS{9}Des briques sont tombées, mais nous bâtirons en pierres de taille~; des sycomores ont été coupés, mais nous les changerons en cèdres.
\VS{10}Yahweh élèvera contre eux les ennemis de Retsin, et il armera les ennemis d'Israël~;
\VS{11}la Syrie à l'orient, et les Philistins à l'occident~; et ils dévoreront Israël à gueule ouverte. Malgré tout cela, sa colère ne s'apaise point, et sa main est encore étendue.
\VS{12}Parce que le peuple ne revient pas à celui qui le frappe, et il ne cherche pas Yahweh des armées.
\VS{13}A cause de cela Yahweh retranchera d'Israël en un seul jour la tête et la queue, la branche de palmier et le roseau.
\VS{14}L'ancien et le magistrat, c'est la tête~; et le prophète qui enseigne le mensonge, c'est la queue.
\VS{15}Ceux donc qui font croire à ce peuple qu'il est heureux sont des séducteurs\FTNT{1 Ti. 4:1~; Tit. 1:10.}~; et ceux qui se laissent diriger par eux se perdent.
\VS{16}C'est pourquoi le Seigneur ne saurait prendre plaisir à leurs jeunes hommes ni avoir pitié de leurs orphelins et de leurs veuves, car tous sont des hypocrites et des méchants, et toute bouche ne profère que des infamies. Malgré tout cela, sa colère ne s'apaise point et sa main est encore étendue.
\VS{17}Car la méchanceté consume comme un feu, elle dévore les ronces et les épines~; elle embrase l'épaisseur de la forêt, d'où s'élèvent des colonnes de fumée.
\VS{18}A cause de la fureur de Yahweh des armées, la terre est obscurcie, et le peuple est comme la proie du feu~; nul n'a compassion de son frère.
\VS{19}On pille à droite, et l'on a faim~; on dévore à gauche, et l'on n'est pas rassasié~; chacun mange la chair de son bras.
\VS{20}Manassé dévore Ephraïm, Ephraïm dévore Manassé, et ensemble ils fondent sur Juda. Malgré tout cela, sa colère ne s'apaise point, et sa main est encore étendue.
\Chap{10}
\VerseOne{}Malheur à ceux qui décrètent des ordonnances iniques, et à ceux qui écrivent pour ordonner l'oppression,
\VS{2}pour refuser la justice aux pauvres et ravir leur droit aux malheureux de mon peuple, afin d'avoir les veuves pour leur butin, et de piller les orphelins~!
\VS{3}Et que ferez-vous au jour du châtiment, et de la ruine éclatante qui viendra de loin~? Vers qui fuirez-vous pour avoir du secours et où laisserez-vous votre gloire\FTNT{Os. 9:7~; Mt. 24:17-21~; Lu. 19:41-44.}~?
\VS{4}Les uns seront courbés parmi les prisonniers, les autres tomberont parmi les morts. Malgré tout cela, sa colère ne s'apaise point, et sa main est encore étendue.
\TextTitle{Jugement sur l'Assyrie}
\VS{5}Malheur à l'Assyrie, verge de ma colère~! La verge dans leur main c'est l'instrument de ma colère.
\VS{6}Je l'ai envoyé contre une nation impie, et je l'ai fait marcher contre le peuple de ma fureur, afin qu'il se livre au pillage et fasse du butin, pour qu'il le foule aux pieds comme la boue des rues.
\VS{7}Mais il n'en juge pas ainsi, et ce n'est pas là la pensée de son cœur~; il ne songe qu'à détruire, qu'à exterminer beaucoup de nations.
\VS{8}Car il dit~: Mes princes ne sont-ils pas autant de rois~?
\VS{9}Calno n'est-elle pas comme Carkemisch~? Hamath n'est-elle pas comme Arpad~? Et Samarie n'est-elle pas comme Damas~?
\VS{10}Puisque ma main a soumis les royaumes qui avaient des idoles, où il y avait plus d'images taillées qu'à Jérusalem et à Samarie,
\VS{11}ne ferai-je pas aussi à Jérusalem et à ses dieux, comme j'ai fait à Samarie et à ses idoles~?
\VS{12}Mais il arrivera que, quand le Seigneur aura achevé toute son œuvre sur la montagne de Sion et à Jérusalem, je punirai le roi d'Assyrie pour le fruit de son cœur orgueilleux, et pour la gloire de ses regards hautains.
\VS{13}Parce qu'il dit~: C'est par la force de ma main que j'ai agi, c'est par ma sagesse, car je suis intelligent~; j'ai reculé les bornes des peuples, et j'ai pillé ce qu'ils avaient de plus précieux~; et comme un homme vaillant, j'ai fait descendre ceux qui étaient assis.
\VS{14}Ma main a trouvé les richesses des peuples, comme on trouve un nid~; comme on rassemble des œufs délaissés, ainsi ai-je rassemblé toute la terre~; nul n'a remué l'aile, ni ouvert le bec, ni poussé un cri.
\VS{15}La hache se glorifie-t-elle envers celui qui s'en sert~? Ou la scie s'élève-t-elle au-dessus de celui qui la manie~? Comme si la verge faisait mouvoir celui qui la lève, et que le bâton se levait comme s'il n'était pas du bois~!
\VS{16}C'est pourquoi le Seigneur, Yahweh des armées, enverra la maigreur sur ses hommes gras~; et sous sa gloire éclatera l'embrasement d'un feu.
\VS{17}Car la lumière d'Israël deviendra un feu, et son Saint une flamme qui embrasera et consumera ses épines et ses ronces tout en un jour~;
\VS{18}et il consumera la gloire de sa forêt et de ses campagnes, depuis l'âme jusqu'à la chair. Il en sera comme quand celui qui porte la bannière est défait.
\VS{19}Le reste des arbres de sa forêt pourra être compté, et un enfant en écrirait le nombre.
\TextTitle{Conversion et délivrance du reste d'Israël}
\VS{20}Et il arrivera en ce jour-là, que le reste d'Israël et les réchappés de la maison de Jacob ne s'appuieront plus sur celui qui les frappait, mais ils s'appuieront avec confiance sur Yahweh, le Saint d'Israël.
\VS{21}Le reste se convertira, le reste, dis-je, de Jacob se convertira au Dieu puissant.
\VS{22}Car quand ton peuple, ô Israël, serait comme le sable de la mer, un reste seulement se convertira~; la destruction est résolue, elle fera déborder la justice.
\VS{23}Car la destruction qu'il a résolue, le Seigneur, Yahweh des armées, va l'exécuter au milieu de toute la terre.
\VS{24}C'est pourquoi ainsi parle le Seigneur, Yahweh des armées~: Mon peuple qui habites en Sion, ne crains pas le roi d'Assyrie~; il te frappe de la verge, et il lève son bâton sur toi comme faisait l'Egypte.
\VS{25}Mais encore un peu de temps, un peu de temps, et le châtiment cessera, puis ma colère se tournera contre lui pour l'exterminer.
\VS{26}Et Yahweh des armées lèvera le fouet contre lui, comme il frappa Madian au rocher d'Oreb~; et de même qu'il leva son bâton sur la mer, il le lèvera aussi comme contre les Egyptiens.
\VS{27}En ce jour-là, son fardeau sera ôté de dessus ton épaule et son joug de dessus ton cou~; et l'onction fera rompre le joug.
\TextTitle{Défaite des Assyriens\FTNTT{Es. 35-36~; 37:7.}}
\VS{28}Il marche sur Ajjath, traverse Migron et il met ses bagages à Micmasch.
\VS{29}Ils passent le défilé, ils couchent à Guéba~; Rama est effrayée~; Guibea de Saül prend la fuite.
\VS{30}Pousse des cris, fille de Gallim~! Malheur à toi Anathoth~! Prends garde Laïs~!
\VS{31}Madména se disperse, les habitants de Guébim se sauvent en foule.
\VS{32}Encore un jour d'arrêt à Nob, et il menace de sa main la montagne de la fille de Sion, la colline de Jérusalem.
\VS{33}Voici, le Seigneur, Yahweh des armées, brise les rameaux avec force~; et ceux qui sont les plus hauts élevés sont coupés, et les hauts montés sont abaissés.
\VS{34}Et il taille avec le fer les lieux les plus épais de la forêt, et le Liban tombe sous le Puissant.
\Chap{11}
\TextTitle{Rétablissement du règne de David par le Messie}
\VerseOne{}Mais il sortira un rameau du tronc d'Isaï, et un rejeton naîtra de ses racines\FTNT{Mt. 1:6-16~; Lu. 1:31-32~; Ro. 15:12~; Ap. 5:5.}.
\VS{2}L'Esprit de Yahweh reposera sur lui, Esprit de sagesse et d'intelligence, Esprit de conseil et de force, Esprit de connaissance et de crainte de Yahweh\FTNT{Es. 61:1~; Lu. 4:18}.
\VS{3}Il respirera la crainte de Yahweh, il ne jugera point sur l'apparence et il ne reprendra point sur un ouï-dire\FTNT{Jé. 11:20~; Mt. 22:16~; Ap. 2:23.}.
\VS{4}Mais il jugera les pauvres avec justice, et il prononcera avec droiture un jugement sur les malheureux de la terre, et il frappera la terre par la verge de sa bouche, et il fera mourir le méchant par le souffle de ses lèvres\FTNT{Job 4:9~; Job 15:30~; 2 Th. 2:8.}.
\VS{5}La justice sera la ceinture de ses reins, et la fidélité, la ceinture de ses flancs\FTNT{Ep. 6:14.}.
\VS{6}Le loup habitera avec l'agneau, et le léopard se couchera avec le chevreau~; le veau, le lionceau, et le bétail qu'on engraisse seront ensemble, et un petit enfant les conduira.
\VS{7}La jeune vache paîtra avec l'ourse, leurs petits auront un même gîte, et le lion, comme le bœuf, mangera de la paille\FTNT{Es. 65:25.}.
\VS{8}Le nourrisson s'ébattra sur l'antre de l'aspic, et l'enfant sevré mettra sa main dans la caverne de la vipère.
\VS{9}Il ne se fera ni tort ni dommage sur toute ma montagne sainte, car la terre sera remplie de la connaissance de Yahweh, comme le fond de la mer des eaux qui le couvrent.
\VS{10}En ce jour-là, les nations rechercheront le rejeton d'Isaï qui sera comme une bannière\FTNT{Voir dictionnaire.} pour les peuples, et son séjour ne sera que gloire.
\TextTitle{Etablissement du règne du Messie}
\VS{11}Et il arrivera en ce jour-là, que le Seigneur mettra encore sa main une seconde fois pour acquérir le reste de son peuple dispersé en Assyrie, en Egypte, à Pathros, en Ethiopie, à Elam, à Schinear, à Hamath et dans les îles de la mer.
\VS{12}Il élèvera une bannière parmi les nations, il rassemblera les exilés d'Israël qui auront été chassés, et il recueillera les dispersés de Juda des quatre extrémités de la terre.
\VS{13}Et la jalousie d'Ephraïm sera ôtée, et les oppresseurs de Juda seront retranchés~; Ephraïm ne sera plus jaloux de Juda, et Juda n'opprimera plus Ephraïm.
\VS{14}Mais ils voleront sur l'épaule des Philistins vers la mer~; ils pilleront ensemble les fils de l'orient~; Edom et Moab seront la proie de leurs mains et les enfants d'Ammon leur obéiront.
\VS{15}Yahweh exterminera aussi à la façon de l'interdit la langue de la mer d'Egypte, et il lèvera sa main contre le fleuve par la force de son vent, et il le frappera sur les sept rivières, et fera qu'on y marche avec des souliers.
\VS{16}Et il y aura un chemin pour le reste de son peuple, qui sera échappé de l'Assyrie, comme il y en eut un pour Israël le jour où il remonta du pays d'Egypte.
\Chap{12}
\TextTitle{Louange au sein du royaume}
\VerseOne{}Tu diras en ce jour-là~: Je te loue, ô Yahweh~! Car tu as été irrité contre moi, ta colère s'est apaisée, et tu m'as consolé.
\VS{2}Voici, Dieu est mon salut\FTNT{Voir commentaire en Gn. 49:18.}, j'aurai confiance et je ne craindrai rien~; car Yahweh, Yahweh est ma force et ma louange~; il est mon Sauveur.
\VS{3}Et vous puiserez de l'eau avec joie aux sources du salut\FTNT{Jn. 4:10-14.},
\VS{4}et vous direz en ce jour-là~: Louez Yahweh, invoquez son Nom, publiez ses œuvres parmi les peuples, rappelez que son Nom est une haute retraite~!
\VS{5}Psalmodiez à Yahweh car il a fait des choses magnifiques~: Cela est connu dans toute la terre~!
\VS{6}Habitante de Sion, égaye-toi, et réjouis-toi avec chant de triomphe~! Car le Saint d'Israël est grand au milieu de toi.
\Chap{13}
\TextTitle{Yahweh lève une armée}
\VerseOne{}Prophétie sur Babylone, révélée à Esaïe, fils d'Amots.
\VS{2}Elevez la bannière sur la haute montagne, élevez la voix vers eux, faites des signes avec la main, et qu'on entre dans les portes des magnifiques~!
\VS{3}C'est moi qui ai donné des ordres à ceux qui me sont consacrés, j'ai appelé mes hommes forts pour exécuter ma colère, ceux qui se réjouissent de ma grandeur.
\VS{4}Il y a sur les montagnes un bruit d'une multitude, comme celui d'un grand peuple~; on entend un tumulte de royaumes, de nations rassemblées~: Yahweh des armées passe en revue l'armée pour le combat.
\VS{5}D'un pays éloigné, de l'extrémité des cieux, Yahweh vient avec les instruments de sa colère pour détruire tout le pays.
\TextTitle{Jugement de Yahweh sur Babylone}
\VS{6}Hurlez, car le jour de Yahweh est proche, il vient comme un ravage du Tout-Puissant.
\VS{7}C'est pourquoi toutes les mains deviennent lâches, et tout cœur d'homme se fond.
\VS{8}Ils sont épouvantés~; les détresses et les douleurs les saisissent~; ils sont en travail comme celle qui enfante~; ils se regardent les uns les autres avec stupeur, leurs visages sont comme des visages enflammés.
\VS{9}Voici, le jour de Yahweh arrive, jour cruel, jour de colère et d'ardente fureur\FTNT{Mal. 4:1~; Ap. 19:15.}, qui réduira le pays en désolation, et en exterminera les pécheurs.
\VS{10}Même les étoiles des cieux et leurs astres ne feront plus briller leur lumière~; le soleil s'obscurcira dès son lever, et la lune ne fera plus resplendir sa lueur\FTNT{Joë. 2:31~; Mt. 24:29~; Mc. 13:24.}.
\VS{11}Je punirai le monde habitable à cause de sa malice, et les méchants à cause de leur iniquité~; je ferai cesser l'orgueil des hautains et j'abaisserai l'arrogance des tyrans.
\VS{12}Je ferai qu'un homme sera plus précieux que l'or fin, et une personne plus que l'or d'Ophir.
\VS{13}C'est pourquoi j'ébranlerai les cieux, et la terre sera secouée de sa base\FTNT{Ag. 2:6.}, à cause de la fureur de Yahweh des armées, et à cause du jour de son ardente colère.
\VS{14}Et chacun sera comme un chevreuil qui est chassé, et comme une brebis que personne ne retire, chacun se tournera vers son peuple, chacun fuira vers son pays.
\VS{15}Quiconque sera trouvé, sera transpercé~; et quiconque s'y sera joint, tombera par l'épée.
\VS{16}Et leurs petits enfants seront écrasés sous leurs yeux\FTNT{Na. 3:10.}, leurs maisons seront pillées, et leurs femmes violées.
\TextTitle{Yahweh envoie les Mèdes contre Babylone}
\VS{17}Voici, je vais susciter contre eux les Mèdes, qui ne font point cas de l'argent, et qui ne convoitent point l'or.
\VS{18}Leurs arcs écraseront les jeunes gens, et ils seront sans pitié pour le fruit des entrailles, leur œil n'épargnera point les enfants.
\VS{19}Ainsi Babylone, l'ornement des royaumes, la parure et l'orgueil des Chaldéens, sera comme Sodome et Gomorrhe que Dieu détruisit.
\VS{20}Elle ne sera plus jamais habitée, elle ne sera point habitée de génération en génération~; même les Arabes n'y dresseront point leurs tentes, et les bergers n'y feront plus reposer leurs troupeaux.
\VS{21}Mais les bêtes sauvages des déserts y prendront leur gîte, et les hiboux rempliront ses maisons, les autruches en feront leur demeure, et les boucs y sauteront.
\VS{22}Les chacals hurleront dans ses palais, et les dragons dans ses maisons de plaisance. Son temps est près d'arriver et ses jours ne se prolongeront pas.
\Chap{14}
\TextTitle{Chant d'Israël après la chute de Babylone}
\VerseOne{}Car Yahweh aura pitié de Jacob, il choisira encore Israël, et il les rétablira dans leur pays~; les étrangers se joindront à eux et s'attacheront à la maison de Jacob.
\VS{2}Et les peuples les prendront, et les ramèneront à leur demeure, et la maison d'Israël les possédera en droit d'héritage dans le pays de Yahweh, comme serviteurs et comme servantes~; ils retiendront captifs ceux qui les avaient tenus captifs, et ils domineront sur leurs oppresseurs.
\VS{3}Et il arrivera qu'au jour où Yahweh te fera reposer de ton travail, de ton tourment, et de la dure servitude qui te fut imposée,
\VS{4}alors tu prononceras ce proverbe sur le roi de Babylone, et tu diras~: Comment a-t-il fini le tyran~? Comment se repose celle qui était si avide de richesses~?
\VS{5}Yahweh a brisé le bâton des méchants, et la verge des dominateurs.
\VS{6}Celui qui frappait avec fureur les peuples de coups qu'on ne pouvait point détourner, qui dominait sur les nations avec colère, est poursuivi sans ménagement.
\VS{7}Toute la terre jouit du repos et de la paix~; on éclate en chants de triomphe à gorge déployée.
\VS{8}Même les cyprès et les cèdres du Liban se réjouissent de toi, en disant~: Depuis que tu es tombé, personne n'est monté pour nous abattre.
\TextTitle{Le roi de Babylone dépouillé de sa gloire}
\VS{9}Le scheol s'émeut jusque dans ses profondeurs, pour t'accueillir à ton arrivée~; il réveille à cause de toi les morts, et il fait lever de leurs sièges tous les principaux de la terre.
\VS{10}Tous prennent la parole pour te dire~: Toi aussi, tu es sans force comme nous, tu es devenu semblable à nous~!
\VS{11}Ta hauteur est descendue dans le scheol, avec le son de tes luths~; tu es couché sur une couche de vers, et la vermine est ta couverture.
\TextTitle{Orgueil, rébellion et chute de Satan}
\VS{12}Comment es-tu tombé du ciel, astre brillant, fils de l'aurore~? Toi qui foulais les nations, tu es abattu jusqu'à terre~!
\VS{13}Tu disais en ton cœur~: Je monterai aux cieux, je placerai mon trône au-dessus des étoiles de Dieu~; je m'assiérai sur la montagne de l'assemblée, du côté d'Aquilon\FTNT{Aquilon est un dieu des vents septentrionaux, froids et violents, dans la mythologie romaine.}~;
\VS{14}je monterai au dessus des hauts lieux des nuées, je serai semblable au Très-Haut.
\VS{15}Et cependant tu as été précipité dans le scheol, dans les profondeurs de la fosse\FTNT{Voir commentaire Ge. 1:1-2.}.
\VS{16}Ceux qui te voient fixent sur toi leurs regards, ils te considèrent attentivement, en disant~: N'est-ce pas celui qui faisait trembler la terre, qui ébranlait les royaumes,
\VS{17}qui réduisait le monde habitable en désert, qui détruisait les villes, et ne relâchait pas ses prisonniers, ni ne les renvoyait chez eux~?
\TextTitle{Babylone anéantie}
\VS{18}Tous les rois des nations, oui, tous, reposent avec honneur, chacun dans sa maison.
\VS{19}Mais toi, tu as été jeté loin de ton sépulcre, comme un rejeton pourri, comme une dépouille de gens tués, transpercés avec l'épée, qu'on jette sous les pierres d'une fosse, comme un cadavre foulé aux pieds.
\VS{20}Tu ne seras point rangé comme eux dans le sépulcre, car tu as ravagé ta terre, tu as tué ton peuple. La race des méchants ne sera point renommée à toujours.
\VS{21}Préparez la tuerie pour ses enfants, à cause de l'iniquité de leurs pères~; afin qu'ils ne se relèvent point, et qu'ils n'héritent point la terre, et ne remplissent point de villes le dessus de la terre habitable.
\VS{22}Je m'élèverai contre eux, dit Yahweh des armées, et je retrancherai à Babylone le nom et le reste qu'elle a, ses descendants et sa postérité\FTNT{Ap. 14:8~; Ap. 18:2.}, dit Yahweh.
\VS{23}J'en ferai l'habitation du butor et un marécage, et je la balayerai avec le balai de la destruction, dit Yahweh des armées.
\TextTitle{Jugement sur le roi d'Assyrie}
\VS{24}Yahweh des armées l'a juré, en disant~: Certainement ce que j'ai décidé arrivera, ce que j'ai résolu s'accomplira.
\VS{25}Je briserai le roi d'Assyrie dans ma terre, je le foulerai aux pieds sur mes montagnes~; et son joug leur sera ôté, et son fardeau sera ôté de dessus leurs épaules.
\VS{26}C'est là le conseil arrêté contre toute la terre, c'est là la main étendue sur toutes les nations.
\VS{27}Car Yahweh des armées l'a arrêté en son conseil~: Qui l'empêchera~? Sa main est étendue~: Qui la détournera\FTNT{Ec. 7:13.}~?
\TextTitle{Jugement sur le pays des Philistins}
\VS{28}L'année de la mort du roi Achaz, cette prophétie fut prononcée~:
\VS{29}Ne te réjouis pas, toi pays des Philistins, de ce que la verge de celui qui te frappait est brisée~! Car de la racine du serpent sortira une vipère, et son fruit sera un serpent brûlant qui vole.
\VS{30}Alors les plus misérables seront repus, et les pauvres reposeront en assurance~; mais je ferai mourir de faim ta racine, et ce qui restera de toi sera tué.
\VS{31}Porte, hurle~! Ville, crie~! Tremble, pays tout entier des Philistins~! Car d'Aquilon vient une fumée, et il ne restera pas un homme dans ses habitations.
\VS{32}Et que répondra-t-on aux envoyés de cette nation~? On répondra que Yahweh a fondé Sion, et que les affligés de son peuple y trouvent un refuge.
\Chap{15}
\TextTitle{Jugement sur Moab}
\VerseOne{}Prophétie sur Moab. La nuit même où elle est ravagée, Ar-Moab est détruite~! La nuit même où elle est saccagée, Kir-Moab est détruite~!
\VS{2}Il monte à Bajith et à Dibon, dans les hauts lieux, pour pleurer~; Moab est en lamentations sur Nebo et sur Médeba~: Toutes les têtes sont rasées et toutes les barbes sont coupées.
\VS{3}On sera couvert de sacs dans les rues~; chacun hurle, fondant en larmes sur ses toits et dans ses places\FTNT{Jé. 48:38.}.
\VS{4}Hesbon et Elealé poussent des cris, et l'on entend leur voix jusqu'à Jahats~; c'est pourquoi les guerriers de Moab se lamentent, ils ont l'effroi dans l'âme.
\VS{5}Mon cœur crie à cause de Moab, dont les fugitifs s'enfuient jusqu'à Tsoar, comme une génisse de trois ans~; car ils montent par la montée de Luchith avec des pleurs, et ils jettent des cris de détresse sur le chemin de Choronaïm.
\VS{6}Même les eaux de Nimrim ne sont que désolations, même le foin est déjà séché, l'herbe est consumée, et il n'y a point de verdure.
\VS{7}C'est pourquoi ils surveillent les richesses abondantes qu'ils ont acquises, afin que ce qu'ils ont réservé soit porté dans la vallée des saules.
\VS{8}Car les cris environnent les frontières de Moab, ses lamentations retentissent jusqu'à Eglaïm, ses lamentations retentissent jusqu'à Beer-Elim.
\VS{9}Même les eaux de Dimon sont pleines de sang~; car j'ajouterai un surcroît sur Dimon~: Des lions contre les réchappés de Moab, et le reste du pays.
\Chap{16}
\TextTitle{Lamentation sur Moab}
\VerseOne{}Envoyez l'agneau au souverain du pays, envoyez-le du rocher du désert, à la montagne de la fille de Sion.
\VS{2}Car il arrivera que les filles de Moab seront au passage de l'Arnon, comme un oiseau volant ça et là, comme une nichée chassée de son nid.
\VS{3}Mets en avant le conseil, fais l'ordonnance, sers d'ombre comme une nuit au milieu de midi~; cache ceux qui ont été chassés, et ne trahis pas ceux qui sont errants.
\VS{4}Que ceux de mon peuple qui ont été chassés séjournent chez toi, ô Moab~! Sois pour eux un refuge contre le dévastateur~! Car celui qui use d'extorsion cessera, la dévastation finira, celui qui foule le pays sera consumé de dessus la terre.
\VS{5}Et le trône s'affermira par la clémence~; et sur ce trône sera assis en vérité, dans le tabernacle de David, un juge recherchant le droit, et se hâtant de faire justice\FTNT{Mi. 4:7~; Da. 7:14~; Lu. 1:33~; Ap. 11:15.}.
\VS{6}Nous avons entendu l'orgueil de Moab, le peuple extrêmement orgueilleux, sa fierté, son orgueil, son arrogance et ses vains discours.
\VS{7}C'est pourquoi Moab gémit sur Moab, chacun gémit~; vous soupirez pour les fondements de Kir-Haréseth, il n'y aura que des gens blessés à mort.
\VS{8}Car les campagnes de Hesbon et le vignoble de Sibma languissent~; les maîtres des nations ont foulé ses meilleurs ceps, qui s'étendaient jusqu'à Jaezer, qui couraient ça et là par le désert~; ses rameaux s'étendaient et passaient au-delà de la mer.
\VS{9}C'est pourquoi je pleure sur la vigne de Sibma, comme sur Jaezer~; je vous arrose de mes larmes, ô Hesbon et Elealé~! Car l'ennemi avec des cris s'est jeté sur tes fruits d'été et sur ta moisson.
\VS{10}Et la joie et l'allégresse se sont retirées du champ fertile~; on ne se réjouit plus et on ne s'égaye plus dans les vignes, le vendangeur ne foule plus dans les cuves, j'ai fait cesser la chanson de la vendange\FTNT{Jé. 48:31-34.}.
\VS{11}C'est pourquoi mes entrailles gémissent sur Moab, comme une harpe, et mon intérieur sur Kir-Harès.
\VS{12}Et on voit Moab qui se fatigue sur les hauts lieux~; il entre dans son sanctuaire pour prier mais il ne peut rien obtenir.
\VS{13}Telle est la parole que Yahweh a prononcée depuis longtemps sur Moab.
\VS{14}Et maintenant Yahweh a parlé, en disant~: Dans trois ans, comme les années d'un mercenaire, la gloire de Moab sera avilie, avec toute cette grande multitude~; et le reste sera petit, ce sera peu de chose, ce ne sera rien de considérable.
\Chap{17}
\TextTitle{Prophétie sur la chute de Damas et de ses alliés}
\VerseOne{}Prophétie sur Damas. Voici, Damas est détruite pour ne plus être une ville, et elle ne sera qu'un monceau de ruines\FTNT{Jé. 49:23-27.}.
\VS{2}Les villes d'Aroër sont abandonnées, elles sont livrées aux troupeaux qui s'y reposent, et il n'y a personne qui les effraie.
\VS{3}Il n'y aura plus de forteresse en Ephraïm, ni de royaume à Damas et dans le reste de la Syrie~; ils seront comme la gloire des enfants d'Israël, dit Yahweh des armées.
\VS{4}Et il arrivera en ce jour-là que la gloire de Jacob sera affaiblie et la graisse de sa chair sera fondue.
\VS{5}Il en sera comme quand le moissonneur cueille les blés, et qu'il moissonne les épis avec son bras\FTNT{Joë. 3:13~; Mt. 13:24-30.}~; comme quand on ramasse les épis dans la vallée de Rephaïm.
\VS{6}Mais il en restera quelques grappillages, comme quand on secoue l'olivier, et qu'il reste deux ou trois olives en haut de la cime, et qu'il y en a quatre ou cinq que l'olivier a produites dans ses branches fruitières, dit Yahweh, le Dieu d'Israël.
\VS{7}En ce jour-là, l'homme regardera vers celui qui l'a fait, et ses yeux se tourneront vers le Saint d'Israël.
\VS{8}Et il ne regardera plus vers les autels, qui sont l'ouvrage de ses mains, et il ne regardera plus ce que ses doigts ont fabriqué, ni les images d'Asherah\FTNT{Voir commentaire en Ex. 34:13.}, ni les statues du soleil.
\VS{9}En ce jour-là, ses villes fortes seront abandonnées à cause des enfants d'Israël, ils seront comme un bois taillis et des rameaux abandonnés, et ce sera un désert.
\VS{10}Parce que tu as oublié le Dieu de ton salut, et que tu ne t'es pas souvenue du rocher\FTNT{Voir commentaire Es. 8:13-14.} de ta force, à cause de cela tu as transplanté des plantes de plaisance, et tu as planté des ceps étrangers.
\VS{11}De jour tu as fait croître ce que tu as planté, et le matin tu as fait lever ta semence~; mais la moisson a été enlevée au jour que l'on voulait en jouir, et il y a eu une douleur désespérée.
\VS{12}Malheur à la multitude de peuples nombreux, qui font un bruit comme le bruit des mers~; et à la tempête éclatante des nations, qui font du bruit comme une tempête éclatante d'eaux impétueuses~!
\VS{13}Les nations font un bruit comme une tempête éclatante de grosses eaux, mais il les menace, et elles s'enfuient~; elles seront poursuivies comme la balle des montagnes chassée par le vent, et comme une boule poussée par un tourbillon.
\VS{14}Au temps du soir, voici une terreur soudaine~; mais avant le matin, ils ne sont plus~! C'est là le partage de ceux qui nous dépouillent, et le lot de ceux qui nous pillent.
\Chap{18}
\TextTitle{Jugement sur l'Ethiopie}
\VerseOne{}Malheur à la terre qui fait ombre avec des ailes, qui est au-delà des fleuves de l'Ethiopie~;
\VS{2}qui envoie par mer des messagers, dans des navires de jonc, voguant à la surface des eaux~! Allez, messagers rapides, vers la nation robuste et vigoureuse, vers le peuple redoutable, depuis là où il est et par delà~; nation puissante et qui écrase tout, et dont les fleuves ravagent son pays.
\VS{3}Vous tous, habitants du monde, et vous qui habitez dans le pays, quand la bannière sera élevée sur les montagnes, regardez~; et quand le shofar sonnera, écoutez~!
\VS{4}Car ainsi m'a parlé Yahweh~: Je me tiens tranquillement, et je regarde de ma demeure, par la chaleur de la lumière, et par la vapeur de la rosée, au temps de la chaude moisson.
\VS{5}Car avant la moisson, quand le bourgeon vient en sa perfection, et que la fleur devient un raisin qui mûrit, il coupe les sarments avec des serpes, il enlève les sarments, les ayant retranchés.
\VS{6}Ils seront tous ensemble abandonnés aux oiseaux de proie qui demeurent dans les montagnes, et aux bêtes de la terre~; les oiseaux de proie seront sur eux tout le long de l'été, et toutes les bêtes de la terre y passeront l'hiver.
\VS{7}En ce temps-là, un présent sera apporté à Yahweh des armées, par le peuple robuste et vigoureux, de la part, dis-je, du peuple terrible depuis là où il est et au-delà, nation puissante et qui écrase tout, et dont le pays est ravagé par ses fleuves~; il sera apporté dans la demeure du Nom de Yahweh des armées, sur la montagne de Sion.
\Chap{19}
\TextTitle{Chute de l'Egypte}
\VerseOne{}Prophétie sur l'Egypte. Voici, Yahweh est monté sur une nuée rapide, il entre en Egypte~; et les idoles d'Egypte s'enfuient de toutes parts devant sa face, et le cœur des Egyptiens se fond au milieu d'elle\FTNT{Jé. 43:12.}.
\VS{2}Et je ferai venir pêle-mêle l'Egyptien contre l'Egyptien, et chacun fera la guerre contre son frère, et chacun contre son ami, ville contre ville, et royaume contre royaume.
\VS{3}L'esprit de l'Egypte disparaîtra du milieu d'elle, et je dissiperai son conseil~; et ils consulteront les idoles et les enchanteurs, ceux qui évoquent les morts et ceux qui prédisent l'avenir.
\VS{4}Et je livrerai l'Egypte entre les mains d'un maître sévère~; et un roi cruel dominera sur eux, dit le Seigneur, Yahweh des armées.
\VS{5}Les eaux de la mer tariront, le fleuve séchera et tarira\FTNT{Jé. 51:36.}.
\VS{6}Et on fera détourner les fleuves~; les ruisseaux des digues s'abaisseront et sécheront~; les roseaux et les joncs seront coupés.
\VS{7}Les prairies qui sont près des ruisseaux, et sur l'embouchure du fleuve, tout ce qui aura été semé le long des ruisseaux, séchera, sera jeté au loin, et ne sera plus.
\VS{8}Et les pêcheurs gémiront, tous ceux qui jettent l'hameçon dans le fleuve mèneront deuil, et ceux qui étendent des filets sur les eaux languiront.
\VS{9}Ceux qui travaillent en fin lin et en fin crêpe, et ceux qui tissent les filets seront confus.
\VS{10}Les fondements du pays seront rompus, et tous ceux qui font des écluses de viviers auront l'âme attristée.
\VS{11}Certes les chefs de Tsoan ne sont que des insensés, les sages d'entre les conseillers de Pharaon forment un conseil stupide. Comment osez-vous dire à Pharaon~: Je suis fils des sages, fils des anciens rois~?
\VS{12}Où sont-ils maintenant~? Où sont, dis-je, tes sages~? Qu'ils t'annoncent, je te prie, s'ils le savent, ce que Yahweh des armées a décrété contre l'Egypte.
\VS{13}Les chefs de Tsoan sont devenus insensés, les chefs de Noph se sont trompés, les chefs des tribus font égarer l'Egypte.
\VS{14}Yahweh a versé au milieu d'elle un esprit de vertige\FTNT{1 R. 22:18-22.}, pour qu'ils fassent chanceler les Egyptiens dans toutes leurs actions, comme un homme ivre se vautre dans son vomissement.
\VS{15}Et l'Egypte sera hors d'état de faire ce que font la tête et la queue, la branche de palmier et le roseau.
\TextTitle{L'Egypte et l'Assyrie dans le royaume du Messie}
\VS{16}En ce jour-là, l'Egypte sera comme des femmes~: Elle sera étonnée et épouvantée à cause de la main de Yahweh des armées, quand il élèvera la main contre elle.
\VS{17}Et la terre de Juda sera pour l'Egypte un objet d'effroi~; quiconque fera mention d'elle, en sera épouvanté en lui-même, à cause du conseil décrété contre elle par Yahweh des armées.
\VS{18}En ce jour-là, il y aura cinq villes au pays d'Egypte, qui parleront la langue de Canaan, et qui jureront par Yahweh des armées~; l'une sera appelée ville de la destruction.
\VS{19}En ce jour-là, il y aura un autel à Yahweh au milieu du pays d'Egypte, et un monument dressé à Yahweh sur la frontière.
\VS{20}Et ce sera un signe et un témoignage pour Yahweh des armées dans le pays d'Egypte~; car ils crieront à Yahweh à cause des oppresseurs, il leur enverra un sauveur, quelqu'un de grand, et il les délivrera\FTNT{Es. 43:11.}.
\VS{21}Et Yahweh se fera connaître aux Egyptiens, et les Egyptiens connaîtront Yahweh en ce jour-là~; ils le serviront, ils offriront des sacrifices et des offrandes, et ils feront des vœux à Yahweh et les accompliront.
\VS{22}Ainsi Yahweh frappera les Egyptiens, il les frappera, mais il les guérira~; et ils retourneront à Yahweh, qui les exaucera et les guérira.
\VS{23}En ce jour-là, il y aura un chemin battu de l'Egypte en Assyrie~; l'Assyrie viendra en Egypte, et l'Egypte en Assyrie, et l'Egypte servira avec l'Assyrie.
\VS{24}En ce même temps, Israël sera, lui troisième, uni à l'Egypte et à l'Assyrie, et la bénédiction sera au milieu de la terre.
\VS{25}Yahweh des armées les bénira, en disant~: Bénis soient l'Egypte mon peuple, et l'Assyrie œuvre de mes mains, et Israël mon héritage~!
\Chap{20}
\TextTitle{Conquête de l'Egypte et de l'Ethiopie}
\VerseOne{}L'année où Tharthan, envoyé par Sargon, roi d'Assyrie, vint et combattit contre Asdod, et la prit.
\VS{2}En ce temps-là, Yahweh parla par Esaïe, fils d'Amots, et lui dit~: Va, délie le sac de dessus tes reins et ôte tes souliers de tes pieds. Il fit ainsi, marchant nu et déchaussé.
\VS{3}Puis Yahweh dit~: De même que mon serviteur Esaïe marche nu et déchaussé, ce qui sera dans trois ans un signe et un prodige contre l'Egypte et contre l'Ethiopie,
\VS{4}de même le roi d'Assyrie emmènera de l'Egypte et de l'Ethiopie prisonniers et captifs les jeunes et les vieux, nus et déchaussés, ayant les hanches découvertes, ce qui sera l'opprobre de l'Egypte\FTNT{2 S. 10:4~; Es. 3:17~; Jé. 13:22-26.}.
\VS{5}Ils seront effrayés, et ils seront honteux à cause de l'Ethiopie à qui ils s'attendaient, et à cause de l'Egypte dont ils se glorifiaient.
\VS{6}Et les habitants de cette côte diront en ce jour-là~: Voilà ce qu'est devenu le peuple à qui nous nous attendions, celui vers qui nous courions chercher du secours, afin d'être délivrés du roi d'Assyrie~! Comment pourrons-nous échapper~?
\Chap{21}
\TextTitle{Annonce de la conquête de Babylone}
\VerseOne{}Prophétie sur le désert de la mer. Il vient du désert, de la terre redoutable, comme des tourbillons qui s'élèvent au pays du midi pour traverser.
\VS{2}Une vision terrible m'a été révélée. Le traître demeure traître, celui qui saccage, saccage toujours. Monte, Elam~! Assiège, Médie~! Je fais cesser tous les soupirs.
\VS{3}C'est pourquoi mes reins sont remplis de douleur~; les angoisses me saisissent comme les douleurs de celle qui enfante~; je suis tourmenté à cause de ce que j'ai entendu, et j'ai été tout troublé à cause de ce que j'ai vu.
\VS{4}Mon cœur est agité de toutes parts, la terreur s'empare de moi~; la nuit de mes plaisirs devient une nuit de crainte.
\VS{5}Qu'on dresse la table, que la sentinelle veille, qu'on mange, qu'on boive~! Levez-vous, chefs~! Oignez le bouclier~!
\VS{6}Car ainsi m'a parlé le Seigneur~: Va, place la sentinelle, et qu'elle rapporte ce qu'elle verra\FTNT{Ez. 33:1-19.}.
\VS{7}Et elle vit un char, un couple de cavaliers, un char tiré par des ânes, un char tiré par des chameaux~; et elle les considéra fort attentivement.
\VS{8}Et elle s'écria~: C'est un lion~! Seigneur, je me tiens en sentinelle toute la journée et je suis à mon poste toutes les nuits~;
\VS{9}et voici venir le char d'un homme et un couple de cavaliers~! Alors elle parla, et dit~: Elle est tombée, elle est tombée, Babylone\FTNT{Prophétie sur la chute de Babylone. Voir Jé. 50-51~; Ap. 18.}, et toutes les images taillées de ses dieux sont brisées par terre.
\VS{10}C'est ce que j'ai foulé, et le grain que j'ai battu dans mon aire. Je vous ai annoncé ce que j'ai entendu de Yahweh des armées, du Dieu d'Israël.
\VS{11}Prophétie sur Duma. On me crie de Séir~: Ô sentinelle~! Qu'en est-il de la nuit~? Ô sentinelle~! Qu'en est-il de la nuit~?
\VS{12}La sentinelle répond~: Le matin vient et la nuit aussi. Si vous demandez, demandez. Retournez, venez.
\TextTitle{Jugement sur l'Arabie}
\VS{13}Prophétie contre l'Arabie. Vous passerez pêle-mêle la nuit dans la forêt, caravanes de Dedan~!
\VS{14}Les habitants du pays de Théma portent de l'eau à ceux qui ont soif~; ils viennent au-devant du fugitif avec du pain pour lui.
\VS{15}Car ils fuient devant les épées, devant l'épée dégainée, devant l'arc tendu, devant le fort de la bataille.
\VS{16}Car ainsi m'a parlé le Seigneur~: Encore une année, comme les années d'un mercenaire, et toute la gloire de Kédar prendra fin.
\VS{17}Et le reste du nombre des forts archers des fils de Kédar sera diminué, car Yahweh, le Dieu d'Israël, a parlé.
\Chap{22}
\TextTitle{Malédiction sur la vallée des visions, Jérusalem}
\VerseOne{}Prophétie sur la vallée des visions. Qu'as-tu maintenant, que tu sois toute montée sur les toits~?
\VS{2}Toi ville bruyante, pleine de tumulte, ville joyeuse~! Tes blessés à morts ne seront pas blessés à mort par l'épée, et ils ne mourront pas par la guerre.
\VS{3}Tous tes chefs fuient ensemble, ils sont liés par les archers~; tous ceux des tiens qui sont trouvés sont liés ensemble tandis qu'ils s'enfuient au loin.
\VS{4}C'est pourquoi je dis~: Détournez de moi vos regards, que je pleure amèrement. Ne vous empressez pas pour me consoler du désastre de la fille de mon peuple.
\VS{5}Car c'est le jour de trouble, d'oppression et de confusion\FTNT{La. 1:5~; La. 2:2.}, envoyé par le Seigneur, Yahweh des armées, dans la vallée des visions. Il démolit la muraille et les cris retentissent jusqu'à la montagne.
\VS{6}Même Elam prend son carquois, il y a des hommes montés sur des chars et des cavaliers~; Kir découvre le bouclier.
\VS{7}Et tes plus belles vallées sont remplies de chars, et les cavaliers se rangent tous en bataille à tes portes.
\VS{8}Et on découvre ce qui couvrait Juda, et en ce jour là tu regardes vers les armes de la maison de la forêt.
\VS{9}Vous voyez que les brèches de la cité de David sont nombreuses~; et vous assemblez les eaux de l'étang inférieur.
\VS{10}Vous faites le dénombrement des maisons de Jérusalem, et vous démolissez les maisons pour fortifier la muraille.
\VS{11}Et vous faites aussi un réservoir d'eau entre les deux murailles, pour les eaux de l'ancien étang. Mais vous ne regardez pas à celui qui a fait ces choses, qui les a formées il y a longtemps.
\VS{12}Le Seigneur, Yahweh des armées, vous appelle ce jour-là aux pleurs et au deuil, à vous raser la tête, et à ceindre le sac\FTNT{Ez. 7:18~; Joë. 1:13.}.
\VS{13}Et voici il y a de la joie et de l'allégresse~! On égorge des bœufs et l'on tue des moutons, on mange la viande et l'on boit du vin~; puis on dit~: Mangeons et buvons, car demain nous mourrons\FTNT{Es. 56:12~; 1 Co. 15:32.}~!
\VS{14}Or il m'a été révélé à l'oreille, par Yahweh des armées~: Sûrement cette iniquité ne vous sera pas pardonnée jusqu'à ce que vous mouriez, a dit le Seigneur, Yahweh des armées.
\TextTitle{Eliakim succède à Schebna}
\VS{15}Ainsi parle le Seigneur, Yahweh des armées~: Va, entre chez ce trésorier, chez Schebna, gouverneur du palais et dis-lui~:
\VS{16}Qu'as-tu à faire ici, et qu'as-tu ici qui t'appartienne, que tu te tailles ici un sépulcre~? Il taille un sépulcre en hauteur, il se taille une demeure dans le rocher.
\VS{17}Voici, ô homme~! Yahweh te chassera au loin d'un bras vigoureux~; il t'enveloppera entièrement.
\VS{18}Il te fera rouler fort vite, comme une balle sur une terre large et spacieuse~; là tu mourras, là seront les chars de ta gloire, ô toi qui es la honte de la maison de ton Seigneur~!
\VS{19}Je te jetterai hors de ton rang, et on t'arrachera de ton service.
\VS{20}Et il arrivera en ce jour-là que j'appellerai mon serviteur Eliakim, fils de Hilkija.
\VS{21}Je le revêtirai de ta tunique, je le ceindrai de ta ceinture, et je remettrai ton autorité entre ses mains, il sera un père pour les habitants de Jérusalem et pour la maison de Juda.
\VS{22}Et je mettrai la clef de la maison de David sur son épaule~; et il ouvrira, et il n'y aura personne qui ferme~; et il fermera, et il n'y aura personne qui ouvre\FTNT{La clé de David est le symbole de l'autorité du Messie (Es. 9:5~; Mt. 28:18~; Ap. 3:7-8).}.
\VS{23}Je l'enfoncerai comme un clou dans un lieu sûr, et il sera un trône de gloire pour la maison de son père.
\VS{24}Et on y pendra toute la gloire de la maison de son père, de ses parents et de celles qui lui appartiennent~; tous les ustensiles des plus petites choses, des bassins comme des vases.
\VS{25}En ce jour-là, dit Yahweh des armées, le clou enfoncé dans un lieu sûr sera ôté~; et étant retranché il tombera, et le fardeau qui était sur lui sera retranché, car Yahweh a parlé.
\Chap{23}
\TextTitle{Effondrement de Tyr}
\VerseOne{}Prophétie sur Tyr. Hurlez, navires de Tarsis~! Car elle est détruite, il n'y a plus de maisons, on n'y entre plus~! Ceci leur a été révélé du pays de Kittim.
\VS{2}Vous qui habitez dans l'île, taisez-vous~! Toi qui étais remplie de marchands de Sidon, et de ceux qui traversaient la mer~!
\VS{3}A travers les grandes eaux, les grains de Shichor, la moisson du Nil était pour elle son revenu~; elle était le marché des nations\FTNT{Ez. 27.}.
\VS{4}Sois honteuse, ô Sidon~! Car la mer, la forteresse de la mer, a parlé, en disant~: Je n'ai point eu de douleurs, je n'ai point enfanté, je n'ai point nourri de jeunes gens ni élevé aucune vierge.
\VS{5}Quand le bruit parviendra en Egypte, ainsi ils seront dans l'angoisse quand ils entendront le bruit concernant Tyr.
\VS{6}Passez à Tarsis, hurlez, vous qui habitez dans l'île~!
\VS{7}N'est-ce pas ici votre ville joyeuse~? Elle avait une origine antique et ses propres pieds la mènent séjourner dans un pays étranger.
\VS{8}Qui a pris ce conseil contre Tyr, celle qui couronnait les siens, dont les marchands étaient des princes, et dont les trafiquants étaient les plus honorables de la terre\FTNT{Ap. 18:9-18.}~?
\VS{9}Yahweh des armées a pris ce conseil, pour flétrir l'orgueil de toute la noblesse, et pour avilir tous les honorables de la terre.
\VS{10}Traverse ton pays, comme une rivière, ô fille de Tarsis~! Il n'y a plus de ceinture.
\VS{11}Il a étendu sa main sur la mer, il a fait trembler les royaumes~; Yahweh a ordonné la destruction des forteresses de Canaan.
\VS{12}Il a dit~: Tu ne te livreras plus à la joie, vierge opprimée, fille de Sidon~! Lève-toi, passe au pays de Kittim~! Même là, il n'y aura pas de repos pour toi.
\VS{13}Voilà le pays des Chaldéens~; ce peuple-là n'était pas autrefois~; Assur\FTNT{Assur~: Le second fils de Sem (Ge. 10:22). L'ancêtre des Assyriens.} l'a fondé pour les gens du désert~; on a dressé ses forteresses, on a élevé ses palais, et il l'a mis en ruines.
\VS{14}Hurlez, navires de Tarsis~! Car votre force est détruite~!
\VS{15}Et il arrivera en ce jour-là que Tyr tombera dans l'oubli durant soixante-dix ans, selon les jours d'un roi. Mais au bout de soixante-dix ans\FTNT{Jé. 25:11-12.}, on chantera une chanson à Tyr comme à une femme prostituée~:
\VS{16}Prends la harpe, fais le tour de la ville, ô prostituée qu'on oublie~! Sonne avec force, chante et rechante, afin qu'on se ressouvienne de toi~!
\VS{17}Et il arrivera au bout de soixante-dix ans que Yahweh visitera Tyr, mais elle retournera au salaire de sa prostitution, et elle se prostituera avec tous les royaumes de la terre, sur le dessus de la terre.
\VS{18}Mais son trafic et son salaire seront sanctifiés à Yahweh~; il n'en sera rien réservé, ni serré~; car son trafic sera pour ceux qui habitent dans la présence de Yahweh, pour en manger à satiété, et pour avoir des vêtements durables.
\Chap{24}
\TextTitle{Désastre après l'invasion babylonienne}
\VerseOne{}Voici, Yahweh s'en va rendre le pays vide et l'épuiser, il en renverse le dessus, et disperse ses habitants\FTNT{Ge. 11:1-8.}.
\VS{2}Et il en est du prêtre comme du peuple, du maître comme de son serviteur, de la dame comme de sa servante, du vendeur comme de l'acheteur, de celui qui prête comme de celui qui emprunte, du créancier comme du débiteur.
\VS{3}Le pays est entièrement vidé et entièrement pillé, car Yahweh a prononcé cet arrêt.
\VS{4}La terre mène le deuil, elle est déchue~; le pays habité est devenu languissant, il est déchu~; les plus distingués du peuple de la terre sont languissants.
\VS{5}Le pays était profané par ses habitants qui marchent sur lui~; car ils ont transgressé les lois, ils ont changé les ordonnances et ont enfreint l'alliance éternelle\FTNT{Da. 7:25.}.
\VS{6}C'est pourquoi la malédiction dévore le pays, et ses habitants portent la peine de leurs crimes~; c'est pourquoi les habitants du pays sont brûlés et il n'en reste qu'un petit nombre.
\VS{7}Le vin excellent pleure, la vigne languit, et tous ceux qui avaient le cœur joyeux soupirent.
\VS{8}La joie des tambours a cessé~; le bruit de ceux qui s'égayent a pris fin, la joie de la harpe a cessé.
\VS{9}On ne boit plus de vin en chantant~; les boissons fortes sont amères à ceux qui les boivent.
\VS{10}La ville confuse est en ruines~; toutes les maisons sont fermées, on n'y entre plus.
\VS{11}On crie dans les rues parce que le vin manque~; toute la joie est tournée en obscurité, l'allégresse du pays s'en est allée.
\VS{12}La désolation est restée dans la ville et la porte est frappée d'une ruine éclatante.
\VS{13}Car il arrivera au milieu de la terre et parmi les peuples, comme quand on secoue l'olivier, et comme quand on grappille après la vendange.
\TextTitle{Un reste de réchappés célèbre Yahweh}
\VS{14}Ils élèvent leur voix, ils se réjouissent avec chant de triomphe~; et s'égayent du côté de la mer, ils célèbrent la majesté de Yahweh.
\VS{15}C'est pourquoi glorifiez Yahweh dans les vallées, le Nom de Yahweh, le Dieu d'Israël, dans les îles de la mer~!
\VS{16}De l'extrémité de la terre, nous entendons des cantiques à la gloire du Juste~; mais moi je dis~: Maigreur sur moi~! Maigreur sur moi~! Malheur à moi~! Les perfides ont agi perfidement~; et ils ont imité la mauvaise foi des perfides.
\TextTitle{Manifestation des jugements de Yahweh}
\VS{17}La frayeur, la fosse, et le piège sont sur toi, habitant du pays~!
\VS{18}Et il arrivera que celui qui fuit à cause du bruit de la frayeur tombe dans la fosse, et celui qui remonte hors de la fosse se prend au filet~; car les écluses d'en haut s'ouvrent et les fondements de la terre tremblent.
\VS{19}La terre est entièrement brisée, la terre s'écrase entièrement, la terre se remue de sa place.
\VS{20}La terre chancelle entièrement comme un homme ivre, elle est transportée comme une cabane~; son péché pèse sur elle, elle tombe et ne se relève plus.
\VS{21}Et il arrivera en ce jour là, que Yahweh punira dans le lieu élevé l'armée d'en haut, et sur la terre les rois de la terre.
\VS{22}Ils seront assemblés en troupes comme des prisonniers dans une fosse, et ils seront enfermés dans une prison, et après plusieurs jours ils seront visités.
\VS{23}La lune rougira et le soleil sera honteux quand Yahweh des armées régnera sur la montagne de Sion et à Jérusalem, resplendissant de gloire en présence de ses anciens\FTNT{Mt. 24:29-30~; 2 Pi. 3:10-12~; Ap. 6:12.}.
\Chap{25}
\TextTitle{Le royaume de Yahweh}
\VerseOne{}Ô Yahweh, tu es mon Dieu~; je t'exalterai, je célébrerai ton Nom, car tu as fait des choses merveilleuses~; tes conseils conçus d'avance sont fidèlement accomplis.
\VS{2}Car tu as fait de la ville un monceau de pierres, et de la cité forte une ruine~; le palais des étrangers qui était dans la ville ne sera jamais rebâti.
\VS{3}C'est pourquoi le peuple fort te glorifie, la ville des nations redoutables te révère.
\VS{4}Parce que tu as été la force du faible, la force du misérable dans sa détresse, le refuge contre la tempête, l'ombrage contre la chaleur~; car le souffle des tyrans est comme la tempête qui abat une muraille.
\VS{5}Tu as rabaissé la tempête éclatante des étrangers~; comme la chaleur, dis-je, dans un pays sec, comme la chaleur par l'ombre d'une nuée, le branchage des tyrans sera abattu.
\VS{6}Et Yahweh des armées prépare à tous les peuples sur cette montagne un banquet de choses grasses, un banquet de vins vieux, un banquet, dis-je, de choses grasses et mœlleuses, et de vins vieux bien purifiés\FTNT{Mt. 22:2~; Ap. 3:20.}.
\VS{7}Et il détruit sur cette montagne l'enveloppe redoublée qu'on voit sur tous les peuples, et la couverture qui est étendue sur toutes les nations.
\VS{8}Il détruit la mort par sa victoire\FTNT{1 Co. 15:54.}~; et le Seigneur Yahweh essuie les larmes de tous les visages\FTNT{Ap. 7:17.}, et il ôte l'opprobre de son peuple de toute la terre\FTNT{Lu. 1:25.}, car Yahweh a parlé.
\VS{9}Et l'on dira en ce jour-là~: Voici, c'est ici notre Dieu, auquel nous nous attendons, aussi c'est lui qui nous sauve~; c'est ici Yahweh, auquel nous nous attendons~; soyons dans l'allégresse, et réjouissons-nous de son salut~!
\VS{10}Car la main de Yahweh repose sur cette montagne~; mais Moab est foulé aux pieds sous lui, comme on foule la paille pour en faire du fumier.
\VS{11}Et il étend ses mains au milieu d'eux, comme le nageur étend ses mains pour nager~; et Yahweh abat son orgueil, ainsi que l'artifice de ses mains.
\VS{12}Il abaisse la forteresse des plus hautes retraites de tes murailles, il les renverse, il les fait crouler à terre, et les réduit en poussière.
\Chap{26}
\TextTitle{Adoration à Yahweh}
\VerseOne{}En ce jour-là, ce cantique sera chanté dans le pays de Juda~: Nous avons une ville forte~; le salut\FTNT{Le mot salut vient du mot «~Yeshuw'ah~». Cette même racine a donné le prénom Jésus qui signifie Yahweh sauve. Jésus est notre muraille et notre rempart. Dans Ex. 15:2, Moïse identifie Yahweh à «~Yeshuw'ah~» c'est-à-dire à Jésus. Dans 1 Ch. 16:23, il est dit que «~Yeshuw'ah~» doit être annoncé tous les jours. Dans Ps. 62:2, il est présenté comme Dieu et le Rocher. Dans Es. 12:2, il est le Dieu qui sauve. Jacob et David avaient mis en lui leur espoir (Ge. 49:18~; Ps. 119:166). Dans Es. 49:6, il est dit que le salut («~Yeshuw'ah~» ou Jésus) doit être annoncé aux extrémités de la terre, et cela est répété et confirmé en Mt. 28:18-20. Es. 56:1 nous apprend que celui qui vient s'appelle «~Yeshuw'ah~». Es. 59:17 le présente comme notre casque, ce qui fait écho au casque du salut en Ep. 6:17. Les murs de la Nouvelle Jérusalem portent son Nom (Es. 60:18). Ha. 3:8 nous dit que «~Yeshuw'ah~» montera sur ses chevaux, corroborant le récit de son retour en gloire dans Ap. 19:11-20. «~Yeshuw'ah~» est notre flambeau selon Es. 62:1 et Ap. 21:23.} y sera mis pour muraille et pour rempart.
\VS{2}Ouvrez les portes, et la nation juste, celle qui garde la fidélité, y entrera.
\VS{3}Tu gardes dans la paix, dans la paix\FTNT{Voir commentaire en Gn. 2:16-17.}, celui dont les desseins s'appuient sur toi, parce qu'il se confie en toi\FTNT{Es. 57:19~; Ph. 4:6-7.}.
\VS{4}Confiez-vous en Yahweh à perpétuité, car le Rocher \FTNT{Voir commentaire en Es. 8:13-14.} des siècles est en Yahweh Dieu.
\VS{5}Car il a abaissé ceux qui habitaient aux lieux haut élevés, il a renversé la ville de haute retraite, il l'a renversée jusqu'à terre, il l'a réduite jusqu'à la poussière.
\VS{6}Le pied marchera dessus~; les pieds, dis-je, des pauvres, les plantes des misérables marcheront dessus.
\VS{7}Le sentier du juste est la droiture~; toi qui est juste, tu dresses au niveau le chemin du juste.
\VS{8}Aussi t'avons-nous attendu, ô Yahweh, dans le sentier de tes jugements~! Ton Nom et ton souvenir sont le désir de notre âme.
\VS{9}De nuit, je te désire de mon âme, et dès le point du jour, mon esprit qui est en moi te recherche~; car lorsque tes jugements s'exercent sur la terre, les habitants du monde apprennent la justice.
\VS{10}Est-il fait grâce au méchant~? Il n'en apprend point la justice, mais il agit méchamment sur la terre de la droiture, et il ne regarde pas à la majesté de Yahweh.
\VS{11}Yahweh, quand ta main est élevée, ils ne le voient pas. Mais ils verront et seront honteux à cause de leur jalousie pour ton peuple~; et le feu dont tu punis tes ennemis les dévorera.
\VS{12}Yahweh, tu ordonnes la paix pour nous, car aussi tout ce que nous faisons, c'est toi qui l'accomplis en nous.
\VS{13}Yahweh, notre Dieu, d'autres seigneurs que toi nous ont maîtrisés, mais c'est par toi seul que nous pouvons faire mention de ton Nom.
\VS{14}Ils sont morts, ils ne revivront plus, ils sont trépassés, ils ne se relèveront pas~; car tu les as châtiés et exterminés, et tu as fait périr toute mémoire d'eux\FTNT{Ec. 9:5.}.
\VS{15}Yahweh, tu avais accru la nation, tu avais accru la nation, tu as été glorifié, mais tu les as jetés loin dans toutes les extrémités de la terre.
\TextTitle{Un reste épargné de la colère de Yahweh}
\VS{16}Yahweh, étant en détresse ils se sont rendus auprès de toi~; ils se sont répandus en prières quand ton châtiment a été sur eux.
\VS{17}Comme celle qui est enceinte est en travail, et crie dans ses tranchées, lorsqu'elle est prête d'enfanter, ainsi avons-nous été devant ta face ô Yahweh~!
\VS{18}Nous avons conçu et nous avons éprouvé des douleurs et nous avons comme enfanté du vent. Nous ne saurions en aucune manière délivrer le pays, et les habitants de la terre habitable ne tomberaient point par notre force.
\VS{19}Tes morts vivront~! Même mon corps mort vivra~! Ils se relèveront. Réveillez-vous et réjouissez-vous avec des chants de triomphe, vous, habitants de la poussière~; car ta rosée est comme la rosée des herbes, et la terre jettera dehors les morts\FTNT{Os. 13:14~; Da. 12:2~; 1 Co. 15:52.}.
\VS{20}Va, mon peuple, entre dans tes cabinets et ferme ta porte derrière toi\FTNT{Mt. 6:6.}~; cache-toi pour un petit moment, jusqu'à ce que l'indignation soit passée.
\VS{21}Car voici, Yahweh s'en va sortir de son lieu pour visiter l'iniquité des habitants de la terre, commise contre lui~; alors la terre découvrira le sang qu'elle aura reçu et ne couvrira plus ceux qu'on a mis à mort.
\Chap{27}
\TextTitle{Israël rétabli}
\VerseOne{}En ce jour-là, Yahweh frappera de sa dure, grande et forte épée le Léviathan\FTNT{Ps. 104:26~; Job 40:20.}, le serpent fuyard, le Léviathan, dis-je, le serpent tortueux, et il tuera le monstre qui est dans la mer.
\VS{2}En ce jour-là, chantez sur la vigne désirable\FTNT{Esaïe annonce ici le rétablissement d'Israël. Voir également Ro. 11:1-24.}.
\VS{3}C'est moi Yahweh qui la garde, je l'arrose à chaque instant, je la garde nuit et jour, afin que personne ne lui fasse du mal.
\VS{4}Il n'y a point de fureur en moi~; qu'on me donne des ronces, des épines pour les combattre~! Je marcherai contre elles, je les brulerai toutes ensemble.
\VS{5}Ou bien, qu'il saisisse ma force, qu'il fasse la paix avec moi, qu'il fasse la paix avec moi.
\VS{6}Il fera que Jacob prendra racine, Israël fleurira, et s'épanouira~; et il remplira de fruits le dessus de la terre habitable.
\VS{7}L'a-t-il frappé comme il a frappé ceux qui le frappaient~? L'a-t-il tué comme il a tué ceux qui le tuaient~?
\VS{8}Tu as plaidé avec elle modérément, quand tu l'as renvoyée~; en l'emportant par le vent rude au jour du vent d'orient.
\VS{9}C'est pourquoi l'expiation de l'iniquité de Jacob sera faite par ce moyen, et ceci en sera le fruit entier, que son péché sera ôté~; quand il aura transformé toutes les pierres des autels comme des pierres de chaux réduites en poussière~; et lorsque les idoles d'Asherah et les statues consacrées au soleil ne seront plus debout.
\VS{10}Car la ville fortifiée est désolée, la demeure agréable est abandonnée et délaissée comme le désert. Là pâture le veau, il y gîte et broute les branches.
\VS{11}Quand son branchage est sec, il est brisé~; et les femmes y venant en allument un feu. Car c'est un peuple sans intelligence\FTNT{De. 32:28~; Es. 1:3.}, c'est pourquoi celui qui l'a fait n'a point eu pitié de lui, et celui qui l'a formé ne lui a point fait grâce.
\VS{12}Il arrivera en ce jour-là que Yahweh secouera, depuis le cours du fleuve jusqu'au torrent d'Egypte~; mais vous serez glanés un à un, ô enfants d'Israël.
\VS{13}Et il arrivera en ce jour-là qu'on sonnera du grand shofar, et ceux qui étaient exilés au pays d'Assyrie, et ceux qui avaient été chassés au pays d'Egypte, reviendront et se prosterneront devant Yahweh, sur la sainte montagne, à Jérusalem.
\Chap{28}
\TextTitle{Malheur et captivité d'Ephraïm en Assyrie}
\VerseOne{}Malheur à la couronne de fierté des ivrognes d'Ephraïm, la noblesse de la gloire qui n'est qu'une fleur qui tombe~; ceux qui sont sur le sommet de la grasse vallée sont étourdis de vin~!
\VS{2}Voici, le Seigneur a dans sa main un homme fort et puissant, semblable à une tempête de grêle, à un tourbillon destructeur, à une tempête de grosses eaux débordées~; il la fera tomber à terre avec la main.
\VS{3}Elles seront foulées aux pieds, la couronne de fierté et les ivrognes d'Ephraïm.
\VS{4}Et la noblesse de sa gloire qui est sur le sommet de la fertile vallée, ne sera qu'une fleur qui tombe~; ils seront comme les fruits précoces avant l'été, aussitôt que celui qui regarde les voit, à peine ils sont dans sa main, il les dévore.
\VS{5}En ce jour-là, Yahweh des armées sera une couronne de noblesse et un diadème de gloire pour le reste de son peuple~;
\VS{6}et un esprit de jugement pour celui qui sera assis au siège de jugement, et une force à ceux qui dans le combat repousseront l'ennemi jusqu'à la porte.
\VS{7}Mais eux aussi, s'oublient dans le vin, et se fourvoient dans les boissons fortes~; le prêtre et le prophète s'oublient dans les boissons fortes~; ils sont engloutis par le vin, ils se fourvoient à cause des boissons fortes~; ils s'oublient dans la vision, ils vacillent dans le jugement.
\VS{8}Car toutes leurs tables sont couvertes de vomissements et d'ordures~; aussi il n'y a plus de place~!
\VS{9}A qui enseigne-t-on la connaissance~? A qui fait-on comprendre l'enseignement~? Est-ce à ceux qu'on vient de sevrer et de retirer de la mamelle~?
\VS{10}Car il faut leur donner précepte après précepte, précepte après précepte, règle après règle, règle après règle, un peu ici, un peu là\FTNT{Hé. 5:12.}.
\VS{11}C'est pourquoi, il parlera à ce peuple par des lèvres qui balbutient et une langue étrangère.
\VS{12}Il leur disait~: Voici le repos, donnez du repos à celui qui est fatigué~; voici le soulagement~! Mais ils n'ont point voulu écouter.
\VS{13}Ainsi la parole de Yahweh sera pour eux précepte après précepte, précepte après précepte, règle après règle, règle après règle, un peu ici, un peu là~; afin qu'ils aillent et tombent à la renverse, et qu'ils soient brisés, et afin qu'ils tombent dans le piège et qu'ils soient pris.
\TextTitle{Yahweh rompt le pacte du scheol par une pierre angulaire}
\VS{14}C'est pourquoi écoutez la parole de Yahweh, vous hommes moqueurs, qui dominez sur ce peuple qui est à Jérusalem~!
\VS{15}Car vous dites~: Nous avons fait un pacte avec la mort, et nous avons un accord avec le scheol~; quand le fléau débordé passera, il ne viendra pas sur nous, car nous avons le mensonge pour refuge et nous nous sommes cachés sous la fausseté.
\VS{16}C'est pourquoi ainsi parle le Seigneur Yahweh~: Voici, je mettrai pour fondement en Sion une pierre\FTNT{Voir commentaire en Es. 8:13-16.}, une pierre éprouvée, la pierre angulaire la plus précieuse, pour être un fondement solide~; celui qui croira ne se hâtera point.
\VS{17}Et je mettrai le jugement à l'équerre, et la justice au niveau~; et la grêle détruira le refuge du mensonge, et les eaux inonderont le lieu où l'on se retirait.
\VS{18}Et votre pacte avec la mort sera détruit, votre accord avec le scheol ne tiendra pas~; quand le fléau débordé passera, vous en serez foulés.
\VS{19}Dès qu'il passera, il vous emportera. Or il passera tous les matins, le jour et la nuit~; et dès qu'on en entendra le bruit, il n'y aura que terreur.
\VS{20}Car le lit sera trop court, et on ne pourra pas s'y étendre, et la couverture trop étroite pour s'en envelopper.
\VS{21}Car Yahweh se lèvera comme à la montagne de Peratsim, et il sera ému comme dans la vallée de Gabaon, pour faire son œuvre, son œuvre extraordinaire, et pour faire son travail, son travail non accoutumé.
\VS{22}Maintenant donc, ne vous moquez plus, de peur que vos liens ne soient renforcés, car j'ai entendu de par le Seigneur, Yahweh des armées, que la destruction est déterminée sur tout le pays.
\VS{23}Prêtez l'oreille, et écoutez ma voix~; soyez attentifs, et écoutez mon discours~!
\VS{24}Celui qui laboure pour semer, laboure-t-il tous les jours~? Ne casse-t-il pas et ne rompt-il pas les mottes de sa terre~?
\VS{25}Quand il en aura aplani la surface, ne sèmera-t-il pas la vesce\FTNT{La vesce est un genre de plantes herbacées de la famille des légumineuses.}~; ne répandra-t-il pas le cumin, ne mettra-t-il pas le froment au meilleur endroit, et l'orge en son lieu assigné, et l'épeautre\FTNT{L'épeautre est une espèce de blé.} en son quartier~?
\VS{26}Parce que son Dieu l'a instruit, et lui a enseigné ce qu'il faut faire.
\VS{27}Car on ne foule pas la vesce avec la herse\FTNT{La herse est un instrument agricole permettant de travailler la terre en surface.}, et on ne tourne point la roue du chariot sur le cumin~; mais on bat la vesce avec la verge, et le cumin avec le bâton.
\VS{28}Le blé avec lequel on fait le pain se menuise, car le laboureur ne le foule pas entièrement~; et quoiqu'il l'écrase avec la roue de son chariot, néanmoins il ne le menuisera pas avec ses chevaux.
\VS{29}Cela aussi vient de Yahweh des armées qui est admirable en conseil et magnifique en moyens.
\Chap{29}
\TextTitle{Avertissement d'un châtiment imminent}
\VerseOne{}Malheur à Ariel\FTNT{Ariel~: Lion de Dieu, nom appliqué à Jérusalem.}, à Ariel, la ville dont David fit sa demeure~! Ajoutez année à année, qu'on égorge des victimes pour les fêtes.
\VS{2}Mais je mettrai Ariel à l'étroit, il n'y aura que tristesse et deuil~; et elle sera pour moi comme Ariel.
\VS{3}Car je camperai en rond contre toi, et je t'assiégerai avec des tours, et je dresserai contre toi des retranchements.
\VS{4}Et tu seras abaissée, et tu parleras depuis la terre, et ta parole sortira étouffée par la poussière~; et ta voix sortira de terre comme celle d'un esprit de Python, et ta parole sera comme un murmure sortant de la poussière.
\VS{5}La multitude de tes étrangers sera comme une fine poussière~; et la multitude des guerriers sera comme la balle qui passe, et cela sera pour un petit moment.
\VS{6}Elle sera visitée par Yahweh des armées avec des tonnerres, des tremblements de terre, et un grand bruit\FTNT{Za. 14:13-14~; Ap. 16:18-19.}~; avec la tempête, le tourbillon, et avec la flamme d'un feu dévorant.
\VS{7}Et la multitude de toutes les nations qui feront la guerre à Ariel, et tous ceux qui la combattront, et ceux qui la serreront de près seront comme un songe d'une vision de nuit.
\VS{8}Et il arrivera que, comme celui qui a faim rêve qu'il mange, mais quand il se réveille son âme est vide~; et comme celui qui a soif rêve qu'il boit, mais quand il se réveille il est épuisé, et son âme est altérée~; ainsi sera-t-il de la multitude de toutes les nations qui combattront contre la montagne de Sion.
\TextTitle{Yahweh donne les raisons du châtiment}
\VS{9}Arrêtez-vous et soyez étonnés~! Bouchez les yeux et soyez aveugles~! Ils sont ivres, mais non de vin~; ils chancellent, mais non pas à cause des boissons fortes.
\VS{10}Car Yahweh a répandu sur vous un esprit d'un profond sommeil\FTNT{Ro. 11:8.}~; il a fermé vos yeux, il a bandé ceux de vos prophètes et de vos principaux voyants.
\VS{11}Et toute vision est pour vous comme les paroles d'un livre cacheté que l'on donne à un homme de lettres en lui disant~: Nous te prions, lis donc cela~! Et qui répond~: Je ne le puis, car il est cacheté~;
\VS{12}puis si on le donne à quelqu'un qui n'est pas un homme de lettres, en lui disant~: Nous te prions, lis donc cela~! Et qui répond~: Je ne sais pas lire.
\VS{13}C'est pourquoi le Seigneur dit~: Parce que ce peuple s'approche de moi de sa bouche et qu'il m'honore de ses lèvres, mais que son cœur est éloigné de moi~; et parce que la crainte qu'il a de moi lui a été enseigné par un commandement d'hommes\FTNT{Mt. 15:8-9~; Mc. 7:6-7.}~;
\VS{14}à cause de cela, voici, je continuerai de faire à l'égard de ce peuple-ci des merveilles et des prodiges étranges~; et la sagesse de ses sages périra, et l'intelligence de ses hommes intelligents disparaîtra.
\VS{15}Malheur à ceux qui cachent profondément leurs desseins, pour les dissimuler à Yahweh, et dont les œuvres sont dans les ténèbres, et qui disent~: Qui nous voit, et qui nous connaît\FTNT{Es. 47:10~; Ez. 8:12~; Ps. 10:11~; Ps. 94:7.}~?
\VS{16}Ce que vous renversez ne sera-t-il pas réputé comme l'argile d'un potier~? Même l'ouvrage dira-t-il de celui qui l'a fait~: Il ne m'a point fait~? Et la chose formée dira-t-elle de celui qui l'a formée~: Il n'a point d'intelligence\FTNT{Ps. 100:3.}~?
\TextTitle{Yahweh rachète Jacob}
\VS{17}Le Liban ne sera-t-il pas encore dans très peu de temps changé en un Carmel~? Et Carmel ne sera-t-il pas considéré comme une forêt~?
\VS{18}En ce jour-là, les sourds entendront les paroles du livre, et les yeux des aveugles, étant délivrés de l'obscurité et des ténèbres, verront\FTNT{Mt. 11:5~; Lu. 7:22.}.
\VS{19}Les humbles auront joie sur joie en Yahweh, et les pauvres d'entre les hommes se réjouiront dans le Saint d'Israël\FTNT{Mt. 5:3-11.}.
\VS{20}Car l'oppresseur prendra fin, le moqueur sera consumé, et tous ceux qui veillaient pour commettre l'iniquité seront retranchés\FTNT{Ap. 20:10.},
\VS{21}ceux qui rendaient coupables les hommes pour une parole, qui tendaient des pièges à celui qui les reprenait à la porte, et qui faisaient tomber le juste en confusion.
\VS{22}C'est pourquoi ainsi parle Yahweh, lui qui a racheté Abraham, à la maison de Jacob~: Jacob ne sera plus honteux, et sa face ne pâlira plus.
\VS{23}Car quand il verra ses fils, ouvrage de mes mains, au milieu de lui, ils sanctifieront mon Nom~; ils sanctifieront, dis-je, le Saint de Jacob, et ils craindront le Dieu d'Israël.
\VS{24}Et ceux dont l'esprit s'était fourvoyé deviendront intelligents, et ceux qui murmuraient apprendront la doctrine.
\Chap{30}
\TextTitle{Mise en garde contre les alliances étrangères}
\VerseOne{}Malheur aux enfants rebelles, dit Yahweh, qui prennent des conseils, et non pas de moi, et qui se forgent des idoles de métal où mon esprit n'est point, afin d'ajouter péché sur péché.
\VS{2}Qui sans avoir interrogé ma bouche, marchent pour descendre en Egypte, afin de se fortifier de la force de Pharaon et se retirer sous l'ombre de l'Egypte\FTNT{Jé. 42:19.}.
\VS{3}Car la force de Pharaon sera pour vous une honte, et le refuge sous l'ombre de l'Egypte votre confusion.
\VS{4}Car ses princes sont à Tsoan, et ses messagers ont atteint Hanès.
\VS{5}Tous seront rendus honteux par un peuple qui ne leur profitera de rien, ils n'en recevront aucun secours ni aucun avantage, il sera leur honte et leur opprobre.
\VS{6}Les bêtes sont chargées pour aller au midi, ils portent leurs richesses sur les dos des ânons, et leurs trésors sur la bosse des chameaux, vers le peuple qui ne leur profitera point dans le pays de détresse et d'angoisse, d'où viennent le vieux lion et le lion, la vipère et le serpent volant.
\VS{7}Car le secours de l'Egypte n'est que vanité et néant~; c'est pourquoi je crie ceci~: Leur force est de se tenir tranquille.
\VS{8}Va maintenant, et écris-le en leur présence sur une table, et rédige-le par écrit dans un livre, afin que cela demeure pour le temps à venir, à perpétuité, à jamais~;
\VS{9}que c'est ici un peuple rebelle, des enfants menteurs, des enfants qui ne veulent point écouter la loi de Yahweh\FTNT{No. 20:3-5~; De. 9:7~; Ac. 7:51.}~;
\VS{10}qui disent aux voyants~: Ne voyez pas~! Et aux prophètes~: Ne nous prophétisez pas des choses droites, mais dites-nous des choses agréables, voyez des choses trompeuses\FTNT{2 Ti. 4:3-4~; Mi. 2:6.}~!
\VS{11}Retirez-vous du chemin, détournez-vous du sentier, éloignez de notre présence le Saint d'Israël\FTNT{Jn. 14:6.}.
\VS{12}C'est pourquoi ainsi dit le Saint d'Israël~: Parce que vous rejetez cette parole et que vous vous confiez dans l'oppression et dans les détours, et que vous vous êtes appuyés sur ces choses,
\VS{13}à cause de cela, cette iniquité sera pour vous comme la fente d'une muraille qui va tomber, un renflement dans un mur élevé, dont la ruine vient soudainement, et en un instant.
\VS{14}Il la brise donc comme on brise un vase de terre, que l'on n'épargne point, et de ses pièces, il ne se trouve pas un tesson pour prendre du feu au foyer, ou pour puiser de l'eau à la citerne.
\TextTitle{La confiance en Yahweh, la vraie force}
\VS{15}Car ainsi a parlé le Seigneur Yahweh, le Saint d'Israël~: En vous tenant tranquilles et en repos vous serez sauvés~; votre force sera en vous tenant en repos et en espérance. Mais vous ne l'avez point voulu.
\VS{16}Et vous avez dit~: Non, mais nous nous enfuirons sur des chevaux~; à cause de cela vous vous enfuirez. Et vous avez dit~: Nous monterons sur des chevaux rapides~; à cause de cela ceux qui vous poursuivront seront rapides.
\VS{17}Mille d'entre vous s'enfuiront à la menace d'un seul~; vous vous enfuirez à la menace de cinq~; jusqu'à ce que vous soyez abandonnés comme un arbre tout ébranché au sommet d'une montagne, et comme une bannière sur la colline.
\VS{18}Cependant Yahweh attend pour vous faire grâce, et ainsi il sera exalté pour vous faire miséricorde~; car Yahweh est le Dieu de jugement~: Ô bienheureux sont tous ceux qui se confient en lui~!
\VS{19}Car le peuple demeurera dans Sion et dans Jérusalem. Tu ne pleureras point~! Certes, il te fera grâce dès qu'il entendra ton cri~; dès qu'il aura entendu, il t'exaucera.
\VS{20}Le Seigneur vous donnera du pain de détresse, et de l'eau d'angoisse, mais tes enseignants ne s'envoleront plus, et tes yeux verront tes enseignants.
\VS{21}Et tes oreilles entendront la parole de celui qui sera derrière toi, disant~: Voici le chemin, marchez-y, soit que vous tiriez à droite, soit que vous tiriez à gauche~!
\VS{22}Et vous tiendrez pour souillés les chapiteaux des images taillées faites d'argent, et les ornements faits d'or fondu~; tu les jetteras au loin comme un sang impur, et tu leur diras~: Hors d'ici~!
\VS{23}Alors il donnera la pluie sur la semence que tu auras semée en terre, et le grain du revenu de la terre sera abondant et bien nourri~; en ce jour-là, ton bétail paîtra dans un pâturage spacieux\FTNT{Jn. 14:6.}.
\VS{24}Les bœufs et les ânes qui labourent la terre mangeront le pur fourrage de ce qui aura été vanné avec la pelle et le van.
\VS{25}Et il y aura des ruisseaux d'eau courante sur toute haute montagne, et sur toute colline haut élevée, au jour de la grande tuerie, quand les tours tomberont.
\VS{26}Et la lumière de la lune sera comme la lumière du soleil~; et la lumière du soleil sera sept fois plus grande, comme si c'était la lumière de sept jours, le jour où Yahweh bandera la blessure de son peuple, et qu'il guérira la blessure de sa plaie.
\TextTitle{Jugement de Yahweh sur les Assyriens}
\VS{27}Voici, le Nom de Yahweh vient de loin, sa colère est ardente, et une pesante charge~; ses lèvres sont pleines d'indignation, et sa langue est comme un feu dévorant.
\VS{28}Son Esprit est comme un torrent qui déborde et atteint jusqu'au milieu du cou, pour disperser les nations d'une telle dispersion qu'elles seront réduites à néant, et il est comme une bride aux mâchoires des peuples, qui les fera errer.
\VS{29}Vous aurez un cantique comme la nuit où l'on célèbre une fête solennelle~; vous aurez le cœur joyeux comme celui qui marche au son de la flûte, pour aller à la montagne de Yahweh, vers le Rocher d'Israël.
\VS{30}Et Yahweh fera entendre sa voix, pleine de majesté, et il montrera où aura assené son bras dans l'indignation de sa colère, avec une flamme de feu dévorant, avec éclat, tempête, et pierres de grêle.
\VS{31}Car l'Assyrien, qui frappait du bâton, sera effrayé par la voix de Yahweh.
\VS{32}Et partout où passe le bâton dont Yahweh l'a assené, et par lequel il combattra dans les batailles à bras élevé, on entendra les tambourins et les harpes.
\VS{33}Car Topheth\FTNT{Topheth~: Lieu pour brûler. Un lieu à l'extrémité sud-est de la vallée de Hinnom au sud de Jérusalem.} est déjà préparée, et même elle est apprêtée pour le roi~; on a fait son bûcher profond et large~; son bûcher c'est du feu et du bois en abondance~; le souffle de Yahweh l'allume comme un torrent de soufre.
\Chap{31}
\TextTitle{Le secours de Yahweh préférable à celui de l'Egypte}
\VerseOne{}Malheur à ceux qui descendent en Egypte pour avoir de l'aide, et qui s'appuient sur les chevaux, et qui mettent leur confiance dans leurs chars parce qu'ils sont nombreux, et en leurs cavaliers quand ils sont bien forts, mais qui ne regardent pas vers le Saint d'Israël, et ne recherchent pas Yahweh.
\VS{2}Et cependant, c'est lui qui est sage, et il fait venir le malheur et ne révoque point sa parole~; il s'élève contre la maison des méchants et contre ceux qui aident les ouvriers d'iniquité.
\VS{3}Or les Egyptiens sont des hommes et non Dieu~; et leurs chevaux sont chair et non esprit. Quand Yahweh étendra sa main, celui qui donne du secours sera renversé, celui à qui le secours est donné tombera, et eux tous ensemble seront consumés.
\VS{4}Mais ainsi m'a dit Yahweh~: Comme le lion, comme le lionceau rugit sur sa proie, et quoiqu'on appelle contre lui un grand nombre de bergers, il ne se laisse ni effrayer par leur cri, ni abaisser par leur bruit~; ainsi Yahweh des armées descendra pour combattre en faveur de la montagne de Sion et de sa colline.
\VS{5}Comme les oiseaux volent, ainsi Yahweh des armées défendra Jérusalem, la défendant et la délivrant, passant outre et la sauvant\FTNT{De. 32:11~; Ps. 91:4~; Mt. 23:37.}.
\VS{6}Retournez vers celui de qui les enfants d'Israël se sont étrangement éloignés.
\VS{7}Car en ce jour-là, chacun rejettera ses idoles d'argent et ses idoles d'or que vos propres mains ont fabriquées pour vous faire pécher.
\VS{8}Et l'Assyrien tombera par l'épée qui n'est pas celle d'un vaillant homme, et l'épée qui n'est pas celle d'un homme le dévorera~; et il s'enfuira devant l'épée, et ses jeunes hommes seront rendus tributaires.
\VS{9}Et saisi de frayeur, il s'enfuira à sa forteresse, et ses chefs seront effrayés à cause de la bannière, dit Yahweh, qui a son feu dans Sion et son fourneau dans Jérusalem.
\Chap{32}
\TextTitle{La venue de l'Esprit annonce la paix et la justice}
\VerseOne{}Voici, un roi régnera selon la justice, et les princes gouverneront avec équité.
\VS{2}Et ce personnage sera comme une cachette contre le vent et comme un asile contre la tempête~; comme des ruisseaux d'eau dans un pays sec, et l'ombre d'un grand rocher dans une terre altérée.
\VS{3}Alors les yeux de ceux qui voient ne seront point retenus, et les oreilles de ceux qui entendent seront attentives.
\VS{4}Et le cœur des étourdis entendra la science, et la langue de ceux qui balbutient parlera aisément et nettement.
\VS{5}Le chiche ne sera plus appelé libéral, et l'avare trompeur ne sera plus nommé magnifique.
\VS{6}Car l'homme vil dira des choses viles, et son cœur ne machine qu'iniquité, pour exécuter son hypocrisie et pour proférer des faussetés contre Yahweh, pour rendre vide l'âme de celui qui a faim, et faire tarir la boisson de celui qui a soif\FTNT{Jn. 10:10.}.
\VS{7}Les instruments de l'avare sont pernicieux~; il prend des conseils pleins de machinations, pour attraper par des paroles de mensonge les affligés, même quand la cause du pauvre est juste\FTNT{2 Pi. 2:3.}.
\VS{8}Mais le libéral forme des conseils de libéralité et se lève pour user de libéralité.
\VS{9}Femmes qui êtes à votre aise, levez-vous, écoutez ma voix~! Filles qui vous tenez assurées, prêtez l'oreille à ma parole~!
\VS{10}Dans un an et quelques jours, vous qui vous tenez assurées serez troublées~; car la vendange a manqué, la récolte n'arrivera plus.
\VS{11}Vous qui êtes à votre aise, tremblez~! Vous qui vous tenez assurées, soyez troublées~! Dépouillez-vous, quittez vos habits et ceignez de sacs vos reins~!
\VS{12}On se frappe la poitrine à cause de la vigne abondante en fruits.
\VS{13}Les épines et les ronces montent sur la terre de mon peuple, même sur toutes les maisons où il y a de la joie et sur la ville joyeuse.
\VS{14}Car le palais est abandonné, la multitude de la cité est délaissée~; les lieux inaccessibles du pays et les forteresses serviront de cavernes à toujours~; les ânes sauvages y joueront, et les troupeaux y paîtront,
\VS{15}jusqu'à ce que l'Esprit soit répandu d'en haut sur nous\FTNT{Joë. 2:28~; Za. 12:10~; Ac. 2:17-18.}, et que le désert devienne un Carmel et que Carmel soit considéré comme une forêt.
\VS{16}Le jugement habitera dans le désert et la justice se tiendra en Carmel.
\VS{17}L'oeuvre de la justice sera la paix, et le fruit de la justice, repos et sécurité pour toujours.
\VS{18}Mon peuple habitera dans une demeure paisible, et dans des habitations assurées, et dans un repos fort tranquille.
\VS{19}Mais la grêle tombera sur la forêt, et la ville sera entièrement abaissée.
\VS{20}Heureux vous qui semez sur toutes les eaux, et qui laissez sans entraves le pied du bœuf et de l'âne~!
\Chap{33}
\TextTitle{Yahweh se lève}
\VerseOne{}Malheur à toi qui dépouilles et qui n'as pas été dépouillé~; qui pilles et qu'on n'a pas encore pillé~! Quand tu auras fini de dépouiller, tu seras dépouillé~; et quand tu auras achevé de piller, on te pillera.
\VS{2}Yahweh, aie pitié de nous~! Nous nous attendons à toi~! Sois leur bras dès le matin et notre délivrance au temps de la détresse~!
\VS{3}Au son du tumulte, les peuples s'enfuient~; quand tu te lèves, les nations se dispersent.
\VS{4}Et votre butin est recueilli comme on rassemble les sauterelles~; on saute dessus comme sautellent les sauterelles.
\VS{5}Yahweh est élevé, car il habite dans les lieux élevés~; il remplit Sion de jugement et de justice\FTNT{Ps. 97:9.}.
\VS{6}Et la sagesse et la science seront la certitude de ta durée, et la force de ton salut~; la crainte de Yahweh est son trésor.
\VS{7}Voici, leurs hérauts poussent des cris au-dehors, et les messagers de paix pleurent amèrement.
\VS{8}Les routes sont réduites en désolation, les passants n'y passent plus. Il a rompu l'alliance, il rejette les villes, il ne fait plus cas des hommes.
\VS{9}On mène le deuil, la terre languit. Le Liban est honteux et flétri. Le Saron est comme un désert. Le Basan et le Carmel secouent leur feuillage.
\VS{10}Maintenant je me lèverai, dit Yahweh, maintenant je serai exalté, maintenant je serai élevé.
\VS{11}Vous avez conçu du foin, et vous enfanterez de la paille~; votre souffle vous dévorera comme le feu.
\VS{12}Et les peuples seront des fourneaux de chaux~; ils seront brûlés au feu comme des épines coupées.
\VS{13}Vous qui êtes loin, écoutez ce que j'ai fait~! Et vous qui êtes près, connaissez ma force~!
\TextTitle{Yahweh assure la paix aux justes}
\VS{14}Les pécheurs sont effrayés dans Sion, et le tremblement saisit les hypocrites, tellement qu'ils diront~: Qui de nous pourra séjourner avec le feu dévorant\FTNT{Hé. 12:29.}~? Qui de nous pourra séjourner avec les flammes éternelles~?
\VS{15}Celui qui observe la justice et qui profère des choses droites~; celui qui rejette le gain déshonnête d'extorsion, et qui secoue ses mains pour ne pas accepter un présent~; celui qui bouche ses oreilles pour ne pas entendre des propos sanguinaires, et qui ferme ses yeux pour ne pas voir le mal.
\VS{16}Celui-là habitera dans des lieux élevés, des forteresses assises sur des rochers seront sa haute retraite~; son pain lui sera donné, et ses eaux ne lui manqueront point\FTNT{Jn. 4:14~; Jn. 6:33-35~; Ap. 21:6.}.
\VS{17}Tes yeux contempleront le roi dans sa beauté~; et ils regarderont la terre éloignée.
\VS{18}Ton cœur méditera-il la frayeur, en disant~: Où est le secrétaire, où est le trésorier~? Où est celui qui tient le compte des tours~?
\VS{19}Tu ne verras plus le peuple fier, le peuple au langage inconnu qu'on n'entend pas, et de langue bégayante qu'on ne comprend pas.
\VS{20}Regarde Sion, la ville de nos fêtes solennelles~! Que tes yeux voient Jérusalem, séjour tranquille, tabernacle qui ne sera pas transporté, et dont les pieux ne seront jamais ôtés, et dont les cordages ne seront point rompus\FTNT{Ap. 21:2.}.
\VS{21}C'est là que Yahweh nous est glorieux~; c'est le lieu de fleuves, de vastes rivières, où n'ira pas de navire à rame et où aucun gros navire ne passera.
\VS{22}Parce que Yahweh est notre Juge, Yahweh est notre Législateur, Yahweh est notre Roi\FTNT{Jésus-Christ exerce toutes les fonctions gouvernementales~: législatives, exécutives et judiciaires.}~; c'est lui qui vous sauvera.
\VS{23}Tes cordages sont lâchés~; et ainsi ils ne tiennent point ferme leur mât et on n'étendra point la voile. Alors la dépouille d'un grand butin est partagée~; même les boiteux pillent le butin.
\VS{24}Et celui qui fait sa demeure dans la maison ne dit point~: Je suis malade~! Le peuple qui habite en elle reçoit le pardon de ses iniquités.
\Chap{34}
\TextTitle{Le jugement des nations\FTNTT{Ap. 19:17-21.}}
\VerseOne{}Approchez-vous nations, pour écouter~! Et vous peuples, soyez attentifs~! Que la terre et tout ce qui la remplit écoute~! Que le monde habitable et tout ce qui y est produit écoute~!
\VS{2}Car l'indignation de Yahweh est sur toutes les nations, et sa fureur sur toute leur armée~; il les voue à l'interdit, il les livre pour être tuées.
\VS{3}Leurs blessés à mort sont jetés là, et la puanteur de leurs corps morts se répand et les montagnes découlent de leur sang.
\VS{4}Et toute l'armée des cieux se fond~; les cieux sont roulés comme un livre\FTNT{Ap. 6:14.}, et toute leur armée tombe, comme tombe la feuille de la vigne, et comme tombe celle du figuier\FTNT{Mt. 24:28~; Mc. 13:25.}.
\VS{5}Parce que mon épée s'est enivrée dans les cieux, voici, elle va descendre en jugement contre Edom, et contre le peuple que j'ai voué à l'interdit.
\VS{6}L'épée de Yahweh est pleine de sang~; engraissée de graisse, et du sang des agneaux et des boucs, et de la graisse des reins de béliers~; car il y a des sacrifices de Yahweh à Botsra, et une grande tuerie dans le pays d'Edom.
\VS{7}Les licornes descendent avec eux, et les bœufs avec les taureaux~; leur terre est enivrée de sang, et leur poussière engraissée de graisse.
\VS{8}Car c'est un jour de vengeance pour Yahweh, une année de rétribution pour maintenir la cause de Sion\FTNT{Jé. 46:10~; Joë. 2:2~; So. 1:15.}.
\VS{9}Et ces torrents d'Edom seront changés en poix, et sa poussière en soufre, et sa terre deviendra de la poix ardente.
\VS{10}Elle ne sera point éteinte ni jour ni nuit~; sa fumée montera éternellement, elle sera désolée de génération en génération~; il n'y aura personne qui passe par elle à jamais.
\VS{11}Le pélican et le hérisson la posséderont, la chouette et le corbeau y habiteront~; et on étendra sur elle la ligne de la désolation et le niveau de désordre.
\VS{12}Ses magistrats crieront qu'il n'y a plus là de royaume, et tous ses princes seront réduits à néant.
\VS{13}Les épines croîtront dans ses palais, les chardons et les buissons dans ses forteresses~; elle sera la demeure des dragons, et le parvis des hiboux.
\VS{14}Les bêtes sauvages des déserts rencontreront les bêtes sauvages des îles~; et les boucs s'y appelleront les uns les autres~; là aussi, la Lilith\FTNT{Lilith est le nom d'une déesse de la nuit connue pour être un démon nocturne qui hantait les lieux déserts d'Edom.} aura sa demeure et trouvera son lieu de repos~;
\VS{15}là le martinet fera son nid, déposera ses œufs, les couvera, et recueillera ses petits à son ombre~; et là aussi se rassembleront tous les vautours.
\VS{16}Consultez le livre de Yahweh et lisez~: Il n'en manquera pas un seul point~; ni l'un ni l'autre ne manqueront~; car c'est ma bouche qui l'a ordonné, et son Esprit qui les rassemblera.
\VS{17}Car il leur a jeté le sort, et sa main leur a partagé cette terre au cordeau, ils la posséderont toujours, ils l'habiteront d'âge en âge.
\Chap{35}
\TextTitle{Yahweh se révèle et sauve son peuple}
\VerseOne{}Le désert et le lieu aride seront dans la joie~; le lieu solitaire se réjouira et fleurira comme une rose.
\VS{2}Il fleurira abondamment, et se réjouira, se réjouissant même et chantant en triomphe. La gloire du Liban lui est donnée, avec la magnificence de Carmel et de Saron~; ils verront la gloire de Yahweh et la magnificence de notre Dieu.
\VS{3}Renforcez les mains lâches, et fortifiez les genoux tremblants\FTNT{Hé. 12:12.}.
\VS{4}Dites à ceux qui ont le cœur troublé~: Prenez courage et ne craignez plus\FTNT{Jn. 14:1~; Jn. 16:33.}~; voici votre Dieu, la vengeance viendra, la rétribution de Dieu~; il viendra lui-même et vous délivrera.
\VS{5}Alors les yeux des aveugles seront ouverts, et les oreilles des sourds seront débouchées.
\VS{6}Alors le boiteux sautera comme un cerf, et la langue du muet chantera en triomphe\FTNT{Mt. 11:4-5.}. Car des eaux jailliront dans le désert, et des torrents dans le lieu solitaire.
\VS{7}Et les lieux secs deviendront des étangs, et la terre desséchée deviendra des sources d'eaux~; et dans les repaires où des dragons faisaient leur gîte, il y aura un parvis à roseaux et à joncs.
\VS{8}Il y aura là un sentier et un chemin, qu'on appellera le chemin de sainteté~; celui qui est souillé n'y passera point, mais il sera pour ceux-là~; celui qui va son chemin, et les insensés ne s'y égareront point\FTNT{Mt. 7:13-14~; Jn. 14:6.}.
\VS{9}Là il n'y aura point de lion~; et aucune des bêtes qui ravissent les autres, n'y montera, et ne s'y trouvera~; mais les rachetés y marcheront.
\VS{10}Ceux dont Yahweh a payé la rançon\FTNT{Jésus-Christ est Yahweh qui a payé notre rançon (Mc. 10:45).}, retourneront, et viendront en Sion avec chant de triomphe, et une joie éternelle sera sur leur tête~; ils obtiendront la joie et l'allégresse~; la douleur et le gémissement s'enfuiront.
\Chap{36}
\TextTitle{Invasion de Sanchérib, menaces de Rabschaké\FTNTT{2 R. 18:9-37~; 2 Ch. 32:1-19.}}
\VerseOne{}La quatorzième année du roi Ezéchias, Sanchérib, roi d'Assyrie, monta contre toutes les villes fortes de Juda et les prit\FTNT{2 R. 18:17.}.
\VS{2}Puis le roi d'Assyrie envoya de Lakis à Jérusalem, vers le roi Ezéchias, Rabschaké avec une puissante armée. Rabschaké s'arrêta à l'aqueduc de l'étang supérieur, sur le chemin du champ du foulon.
\VS{3}Alors Eliakim, fils de Hilkija, chef de la maison du roi, Schebna, le secrétaire, et Joach, fils d'Asaph, l'archiviste, sortirent vers lui.
\VS{4}Rabschaké leur dit~: Dites maintenant à Ezéchias~: Ainsi parle le grand roi, le roi d'Assyrie~: Quelle est cette confiance que tu as~?
\VS{5}Je te le dis, ce ne sont là que des paroles~; mais il faut pour la guerre de la prudence et de la force. Or maintenant en qui t'es tu confié pour t'être rebellé contre moi~?
\VS{6}Voici, tu t'es confié sur ce bâton qui n'est qu'un roseau cassé, sur l'Egypte, qui perce et traverse la main de celui qui s'appuie dessus~; tel est Pharaon, roi d'Egypte, à tous ceux qui se confient en lui.
\VS{7}Que si tu me dis~: Nous nous confions en Yahweh, notre Dieu. Mais n'est-ce pas lui dont Ezéchias a ôté les hauts lieux et les autels, en disant à Juda et à Jérusalem~: Vous vous prosternerez devant cet autel-ci~?
\VS{8}Maintenant donc, donne des otages au roi d'Assyrie, mon maître~; et je te donnerai deux mille chevaux, si tu peux donner autant d'hommes pour monter dessus.
\VS{9}Et comment ferais-tu détourner le visage à un seul gouverneur d'entre les moindres serviteurs de mon maître~? Mais tu te confies en l'Egypte pour les chars et pour les cavaliers.
\VS{10}Mais suis-je monté sans Yahweh dans ce pays pour le détruire~? Yahweh m'a dit~: Monte contre ce pays et détruis-le.
\VS{11}Alors Eliakim, Schebna et Joach dirent à Rabschaké~: Nous te prions de parler en langue araméenne à tes serviteurs, car nous la comprenons~; mais ne parle pas en langue judaïque, pendant que le peuple qui est sur la muraille l'écoute.
\VS{12}Et Rabschaké répondit~: Mon maître m'a-t-il envoyé vers ton maître ou vers toi, pour dire ces paroles là~? Ne m'a-t-il pas envoyé vers les hommes qui se tiennent sur la muraille, pour leur dire qu'ils mangeront leur propre fiente, et qu'ils boiront leur urine avec vous~?
\VS{13}Puis Rabschaké se dressa et s'écria à haute voix en langue judaïque, et dit~: Ecoutez les paroles du grand roi, du roi d'Assyrie~!
\VS{14}Ainsi parle le roi~: Qu'Ezéchias ne vous séduise pas, car il ne pourra pas vous délivrer.
\VS{15}Qu'Ezéchias ne vous fasse pas confier en Yahweh, en disant~: Yahweh nous délivrera, nous délivrera~; cette ville ne sera point livrée entre les mains du roi d'Assyrie.
\VS{16}N'écoutez point Ezéchias~; car ainsi parle le roi d'Assyrie~: Faites un accord avec moi pour votre bien, et sortez vers moi, et vous mangerez chacun de sa vigne, et chacun de son figuier, et vous boirez chacun de l'eau de sa citerne,
\VS{17}jusqu'à ce que je vienne, et que je vous emmène dans un pays qui est comme votre pays, un pays de blé et de bon vin, un pays de pain et de vignes.
\VS{18}Qu'Ezéchias donc ne vous séduise point, en disant~: Yahweh nous délivrera. Les dieux des nations ont-ils délivré chacun leur pays de la main du roi d'Assyrie~?
\VS{19}Où sont les dieux de Hamath et d'Arpad~? Où sont les dieux de Sepharvaïm~? Ont-ils délivré Samarie de ma main~?
\VS{20}Qui sont ceux d'entre tous les dieux de ces pays qui aient délivré leur pays de ma main, pour que Yahweh délivre Jérusalem de ma main~?
\VS{21}Mais ils se turent et ne lui répondirent pas un mot~; car le roi avait donné cet ordre, disant~: Vous ne lui répondrez pas.
\TextTitle{Ezéchias informé des menaces}
\VS{22}Après cela, Eliakim fils de Hilkija, chef de la maison du roi, Schebna, le secrétaire, et Joach, fils d'Asaph l'archiviste, s'en revinrent auprès d'Ezéchias, les vêtements déchirés, et lui rapportèrent les paroles de Rabschaké.
\Chap{37}
\TextTitle{Ezéchias recherche Yahweh auprès d'Esaïe\FTNTT{2 R. 19:1-7~; 2 Ch. 32:20.}}
\VerseOne{}Et il arriva qu'aussitôt que le roi Ezéchias eut entendu ces choses, il déchira ses vêtements, se couvrit d'un sac, et entra dans la maison de Yahweh\FTNT{2 R. 19:1-7~; 2 Ch. 32:20.}.
\VS{2}Puis il envoya Eliakim, chef de la maison du roi, et Schebna, le secrétaire, et les plus anciens des prêtres couverts de sacs, vers Esaïe, le prophète, fils d'Amots.
\VS{3}Et ils lui dirent~: Ainsi parle Ezéchias~: Ce jour est un jour d'angoisse, de répréhension et de blasphème~; car les enfants sont près de sortir du sein maternel, mais il n'y a point de force pour enfanter.
\VS{4}Peut-être que Yahweh, ton Dieu, a-t-il entendu les paroles de Rabschaké, que le roi d'Assyrie, son maître, a envoyé pour blasphémer le Dieu vivant et lui faire outrage~; selon les paroles que Yahweh, ton Dieu, a entendues~; fais donc requête pour le reste qui subsiste encore.
\VS{5}Les serviteurs du roi Ezéchias vinrent vers Esaïe.
\VS{6}Et Esaïe leur dit~: Voici ce que vous direz à votre maître~: Ainsi parle Yahweh~: Ne crains point pour les paroles que tu as entendues, par lesquelles les serviteurs du roi d'Assyrie m'ont blasphémé.
\VS{7}Voici, je vais mettre en lui un esprit tel qu'ayant entendu une certaine rumeur, il retournera dans son pays, et je le ferai tomber par l'épée dans son pays.
\TextTitle{Provocation et menace de Sanchérib\FTNTT{2 R. 19:8-13~; 2 Ch. 32:17-19.}}
\VS{8}Or quand Rabschaké s'en fut retourné, il alla trouver le roi d'Assyrie qui attaquait Libna, car il avait appris qu'il était parti de Lakis.
\VS{9}Alors le roi d'Assyrie entendit dire au sujet de Tirhaka, roi d'Ethiopie~: Il est sorti pour te faire la guerre. Dès qu'il eut entendu cela, il envoya des messagers à Ezéchias, en leur disant~:
\VS{10}Vous parlerez ainsi à Ezéchias, roi de Juda~: Que ton Dieu, auquel tu te confies, ne te séduise point, en disant~: Jérusalem ne sera point livrée entre les mains du roi d'Assyrie.
\VS{11}Voilà, tu as entendu ce que les rois d'Assyrie ont fait à tous les pays, en les détruisant entièrement~; et toi, tu échapperais~?
\VS{12}Les dieux des nations que mes ancêtres ont détruites, à savoir Gozan, Charan, Retseph, et les fils d'Eden, qui sont à Telassar, les ont-ils délivrées~?
\VS{13}Où sont le roi de Hamath, le roi d'Arpad, et le roi de la ville de Sepharvaïm, d'Héna et d'Ivva~?
\TextTitle{Prière d'Ezéchias à Yahweh\FTNTT{2 R. 19:14-19~; 2 Ch. 32:20.}}
\VS{14}Et quand Ezéchias reçut les lettres de la main des messagers et les lut, il monta à la maison de Yahweh, et Ezéchias les déploya devant Yahweh.
\VS{15}Puis Ezéchias fit sa prière à Yahweh, en disant~:
\VS{16}Ô Yahweh des armées~! Dieu d'Israël qui es assis entre les chérubins~! C'est toi qui es le seul Dieu de tous les royaumes de la terre, c'est toi qui as fait les cieux et la terre.
\VS{17}Ô Yahweh~! incline ton oreille et écoute~! Ô Yahweh~! ouvre tes yeux et regarde~! Ecoute les paroles de Sanchérib, qu'il m'a envoyé dire pour blasphémer le Dieu vivant.
\VS{18}Il est bien vrai, ô Yahweh, que les rois d'Assyrie ont détruit tous les pays et leurs contrées~;
\VS{19}et qu'ils ont jeté dans le feu leurs dieux~; mais ce n'étaient point des dieux, mais un ouvrage de mains d'homme, du bois et de la pierre~; c'est pourquoi ils les ont détruits.
\VS{20}Maintenant donc, ô Yahweh, notre Dieu~! Délivre-nous de la main de Sanchérib, afin que tous les royaumes de la terre sachent que toi seul es Yahweh.
\TextTitle{Esaïe transmet la réponse de Yahweh\FTNTT{2 R. 19:20-34.}}
\VS{21}Alors Esaïe, fils d'Amots, envoya dire à Ezéchias~: Ainsi parle Yahweh, le Dieu d'Israël~: J'ai entendu la prière que tu m'as faite au sujet de Sanchérib, roi d'Assyrie.
\VS{22}C'est ici la parole que Yahweh a prononcée contre lui~: La vierge, fille de Sion, te méprise et se moque de toi~; la fille de Jérusalem hoche la tête après toi.
\VS{23}Contre qui as-tu élevé ta voix, et levé tes yeux en haut~? C'est contre le Saint d'Israël.
\VS{24}Tu as outragé le Seigneur par le moyen de tes serviteurs, et tu as dit~: Je suis monté avec la multitude de mes chars sur le haut des montagnes, aux côtés du Liban, je couperai les plus hauts cèdres, et les plus beaux cyprès qui y soient, et j'entrerai jusqu'en son plus haut bout, et en la forêt de son Carmel.
\VS{25}J'ai creusé des sources, et j'en ai bu les eaux, et je tarirai avec la plante de mes pieds tous les fleuves de l'Egypte.
\VS{26}N'as-tu pas appris, qu'il y a déjà longtemps, j'ai fait cette ville, et que dès les temps anciens je l'ai ainsi formée~? Et maintenant l'aurais-je conservée pour être réduite en désolation, et les villes fortes en monceaux de ruines~?
\VS{27}Or leurs habitants, étant dénués de force, ont été épouvantés et confus~; ils sont devenus comme l'herbe des champs~; et l'herbe verte, comme le foin des toits, et le blé brûlé avant la formation de sa tige.
\VS{28}Mais je sais quand tu t'assieds, quand tu sors et quand tu entres, et comment tu es furieux contre moi\FTNT{Ps. 139:2.}.
\VS{29}Parce que tu es furieux contre moi, et que ton insolence est montée à mes oreilles, je mettrai ma boucle à tes narines, et mon mors en ta bouche, et je te ferai retourner par le chemin par lequel tu es venu.
\VS{30}Et ceci te sera pour signe, ô Ezéchias, c'est qu'on mangera cette année ce qui viendra de soi-même aux champs~; et en la deuxième année ce qui croîtra encore sans semer~; mais la troisième année, vous sèmerez, vous moissonnerez, vous planterez des vignes, et vous en mangerez le fruit.
\VS{31}Et ce qui est réchappé, et demeuré de reste dans la maison de Juda, étendra sa racine par-dessous, et elle produira du fruit par-dessus.
\VS{32}Car il sortira de Jérusalem un reste, et de la montagne de Sion quelques réchappés, la jalousie de Yahweh des armées fera cela.
\VS{33}C'est pourquoi ainsi parle Yahweh sur le roi d'Assyrie~: Il n'entrera point dans cette ville, il n'y jettera aucune flèche, il ne se présentera point contre elle avec le bouclier, et il ne dressera point de retranchements contre elle.
\VS{34}Il s'en retournera par le chemin par lequel il est venu, et il n'entrera point dans cette ville, dit Yahweh.
\VS{35}Car je protégerai cette ville pour la délivrer pour l'amour de moi, et pour l'amour de David, mon serviteur.
\TextTitle{Yahweh frappe Sanchérib\FTNTT{2 R. 19:35-37~; 2 Ch. 32:21.}}
\VS{36}L'ange de Yahweh\FTNT{Ge. 16:7.} sortit et frappa cent quatre-vingt-cinq mille hommes dans le camp des Assyriens. Et quand on se leva le matin, voici, ils étaient tous morts.
\VS{37}Alors Sanchérib, roi d'Assyrie, partit de là~; il s'en alla et s'en retourna, et il se tint à Ninive.
\VS{38}Et il arriva qu'étant prosterné dans la maison de Nisroc\FTNT{Le nom Nisroc signifie «~le grand aigle~». C'était une idole de Ninive adorée par Sanchérib, symbolisée par un aigle à figure humaine.}, son dieu, Adrammélec et Scharetser, ses fils, le tuèrent avec l'épée~; puis ils s'enfuirent au pays d'Ararat. Et Esar-Haddon, son fils, régna à sa place.
\Chap{38}
\TextTitle{Maladie et guérison d'Ezéchias\FTNTT{2 R. 20:1-11~; 2 Ch. 32:24-30.}}
\VerseOne{}En ces jours-là, Ezéchias fut malade à la mort\FTNT{2 R. 20:1-11~; 2 Ch. 32:24-30.}. Et Esaïe, le prophète, fils d'Amots, vint auprès de lui, et lui dit~: Ainsi parle Yahweh~: Donne tes ordres à ta maison, car tu vas mourir et tu ne vivras plus.
\VS{2}Alors Ezéchias tourna sa face contre la muraille et fit sa prière à Yahweh,
\VS{3}et dit~: Ô Yahweh, souviens-toi maintenant je te prie que j'ai marché devant toi en vérité et en intégrité de cœur, et que j'ai fait ce qui est agréable à tes yeux~! Et Ezéchias pleura abondamment.
\VS{4}Puis la parole de Yahweh fut adressée à Esaïe, en disant~:
\VS{5}Va, et dis à Ezéchias ainsi parle Yahweh, le Dieu de David, ton père~: J'ai exaucé ta prière, j'ai vu tes larmes. Voici, j'ajouterai à tes jours quinze années.
\VS{6}Et je te délivrerai de la main du roi d'Assyrie, toi et cette ville, et je défendrai cette ville.
\VS{7}Et ce signe t'est donné par Yahweh, pour voir que Yahweh accomplira la parole qu'il a prononcée.
\VS{8}Voici, je ferai retourner de dix degrés en arrière avec le soleil l'ombre des degrés qui est descendue sur les degrés d'Achaz. Et le soleil retourna de dix degrés par les degrés par lesquels il était descendu.
\VS{9}Or c'est ici l'écrit d'Ezéchias, roi de Juda, sur sa maladie et sur son rétablissement.
\VS{10}J'avais dit dans le retranchement de mes jours~: Je m'en irai aux portes du scheol, je suis privé de ce qui restait de mes années.
\VS{11}Je disais~: Je ne contemplerai plus Yahweh, Yahweh sur la terre des vivants~; je ne verrai plus aucun homme parmi les habitants du monde~!
\VS{12}Ma durée s'en est allée, et a été transportée loin de moi, comme une cabane de berger~; ma vie est coupée, je suis retranché comme la toile que le tisserand détache de sa trame. Du matin au soir tu m'auras enlevé\FTNT{Aux versets 12 et 13, le mot qui a été traduit par «~enlevé~» est «~shalam~»~: «~être dans une alliance de paix, être en paix~».}~!
\VS{13}Je pensais en moi-même jusqu'au matin~; comme un lion, qui briserait ainsi tous mes os~; du matin au soir tu m'auras enlevé~!
\VS{14}Je grommelais comme la grue et l'hirondelle~; je gémissais comme la colombe~; mes yeux défaillaient à force de regarder en haut~: Ô Yahweh, je suis opprimé, sois mon garant~!
\VS{15}Que dirai-je~? Il m'a parlé et lui-même l'a fait. Je m'en irai tout doucement tous les ans de ma vie, dans l'amertume de mon âme.
\VS{16}Seigneur, par ces choses-là on a la vie, et dans toutes ces choses est la vie de mon esprit. Ainsi tu me rétabliras et me feras revivre.
\VS{17}Voici, dans ma paix, une grande amertume m'est survenue, mais tu as embrassé mon âme afin qu'elle ne tombe pas dans la fosse de la pourriture, car tu as jeté tous mes péchés derrière ton dos.
\VS{18}Car le scheol ne te loue point, la mort ne te célèbre point~; ceux qui sont descendus dans la fosse ne s'attendent plus à ta vérité\FTNT{Ps. 115:17.}.
\VS{19}Mais le vivant, le vivant est celui qui te célèbre, comme moi aujourd'hui~; le père conduira ses enfants à la connaissance de ta vérité\FTNT{Pr. 22:6~; Ep. 6:4.}.
\VS{20}Yahweh est venu me délivrer, et à cause de cela, nous jouerons sur les instruments mes cantiques, tous les jours de notre vie dans la maison de Yahweh.
\VS{21}Or Esaïe avait dit~: Qu'on prenne une masse de figues sèches et qu'on en fasse un emplâtre sur l'ulcère~; et Ezéchias guérira.
\VS{22}Et Ezéchias avait dit~: Quel est le signe que je monterai à la maison de Yahweh~?
\Chap{39}
\TextTitle{Ezéchias montre ses richesses aux Babyloniens\FTNTT{2 R. 20:12-19.}}
\VerseOne{}En ce temps-là\FTNT{2 R. 20:12-19.}, Mérodac-Baladan, fils de Baladan, roi de Babylone, envoya des lettres avec un présent à Ezéchias, parce qu'il avait entendu qu'il avait été malade, et qu'il était guéri.
\VS{2}Et Ezéchias en eut de la joie, et il leur montra les cabinets où étaient ses choses précieuses, l'argent, l'or, et les aromates, et l'huile précieuse, tout son arsenal, et tout ce qui se trouvait dans ses trésors~; il n'y eut rien qu'Ezéchias ne leur montra dans sa maison et dans tous ses domaines.
\VS{3}Puis le prophète Esaïe vint vers le roi Ezéchias, et lui dit~: Qu'ont dit ces hommes-là, et d'où sont-ils venus vers toi~? Et Ezéchias répondit~: Ils sont venus vers moi d'un pays éloigné, de Babylone.
\VS{4}Puis Esaïe dit~: Qu'ont-ils vu dans ta maison~? Ezéchias répondit~: Ils ont vu tout ce qui est dans ma maison~; il n'y a rien dans mes trésors que je ne leur aie montré.
\VS{5}Et Esaïe dit à Ezéchias~: Ecoute la parole de Yahweh des armées~:
\VS{6}Voici, les jours viennent où l'on emportera à Babylone tout ce qui est dans ta maison, et ce que tes pères ont amassé dans leurs trésors jusqu'à aujourd'hui~; il n'en restera rien, dit Yahweh\FTNT{2 R. 24:13~; 2 R. 25:13-15~; Jé. 20:5.}.
\VS{7}Même on prendra de tes fils qui sortiront de toi, et que tu auras engendrés afin qu'ils soient eunuques dans le palais du roi de Babylone\FTNT{Da. 1:3-4.}.
\VS{8}Et Ezéchias répondit à Esaïe~: La parole de Yahweh, que tu as prononcée, est bonne~; et il ajouta, au moins qu'il y ait paix et sécurité pendant mes jours.
\Chap{40}
\TextTitle{Un nouveau message pour Esaïe}
\VerseOne{}Consolez, consolez mon peuple, dit votre Dieu.
\VS{2}Parlez à Jérusalem selon son cœur, et criez-lui que son temps marqué est accompli, que son iniquité est tenue pour acquittée, qu'elle a reçu de la main de Yahweh le double pour tous ses péchés.
\VS{3}La voix de celui qui crie au désert\FTNT{L'accomplissement de cette prophétie se trouve en Mt. 3:3, où il nous est dit que la voix qui devait crier ces choses était celle de Jean-Baptiste (voir aussi Mal. 3:1~; Mal. 4:5-6~; Mt. 17:10-13).} est~: Préparez le chemin de Yahweh\FTNT{Les évangiles nous enseignent que Jean-Baptiste a été envoyé pour préparer le chemin du Seigneur Jésus (Jn. 1:19-27~; Jn. 1:29-34~; Jn. 3:28-31).}, aplanissez parmi les lieux arides un chemin pour notre Dieu.
\VS{4}Toute vallée sera comblée, toute montagne et toute colline seront abaissées, et les lieux tortueux seront redressés, et les lieux raboteux seront aplanis.
\VS{5}Alors la gloire de Yahweh sera manifestée, et toute chair en même temps la verra, car la bouche de Yahweh a parlé.
\TextTitle{La grandeur de Dieu échappe à l'homme}
\VS{6}La voix dit~: Crie~! Et on a répondu~: Que crierai-je~? Toute chair est comme l'herbe, et toute sa grâce est comme la fleur d'un champ\FTNT{Ja. 1:10~; 1 Pi. 1:24-25.}.
\VS{7}L'herbe sèche, et la fleur tombe, parce que le vent de Yahweh souffle dessus. Certainement le peuple est comme l'herbe.
\VS{8}L'herbe sèche, et la fleur tombe, mais la parole de notre Dieu demeure éternellement.
\VS{9}Sion, qui annonce de bonnes nouvelles, monte sur une haute montagne~; Jérusalem, qui annonce de bonnes nouvelles, élève ta voix avec force~; élève-la, ne crains point~; dis aux villes de Juda~: Voici votre Dieu~!
\VS{10}Voici, le Seigneur Yahweh\FTNT{Jésus-Christ est Yahweh qui vient (Es. 35:4~; Es. 40:10-11~; Es. 60:1~; Es. 62:11-12~; Es. 66:15-16~; Za. 14:1-7~; Mt. 24~; Jn. 14:1-3~; Ac. 1:10-12~; Ap. 3:11~; Ap. 19:11-12~; Ap. 22:7~; Ap. 22:12~; Ap. 22:20).} viendra contre le fort, et son bras dominera sur lui~; voici son salaire est avec lui, et ses rétributions sont devant lui.
\VS{11}Il paîtra son troupeau comme un berger, il rassemblera les agneaux dans ses bras, il les placera dans son sein~; il conduira celles qui allaitent\FTNT{Jn. 10.}.
\VS{12}Qui est celui qui a mesuré les eaux avec le creux de sa main, et qui a pris les dimensions des cieux avec la paume, qui a rassemblé toute la poussière de la terre dans un boisseau, et qui a pesé au crochet les montagnes et les collines à la balance~?
\VS{13}Qui a dirigé l'Esprit de Yahweh, ou qui a été son conseiller pour l'enseigner\FTNT{1 Co. 2:16~; Ro. 11:34.}~?
\VS{14}Avec qui a-t-il pris conseil, et qui l'a instruit, et lui a enseigné le sentier de jugement~? Qui lui a enseigné la science, et lui a montré le chemin de l'intelligence~?
\VS{15}Voilà, les nations sont comme une goutte qui tombe d'un seau, et elles sont réputées comme la menue poussière d'une balance~; voila, il a jeté çà et là les îles comme de la poudre.
\VS{16}Et le Liban ne suffirait pas pour faire le feu, et les bêtes qui y sont ne seraient pas suffisantes pour l'holocauste.
\VS{17}Toutes les nations sont devant lui comme un rien, et il ne les considère que comme de la poussière, et comme un néant.
\VS{18}A qui donc ferez-vous ressembler Dieu~? Et à quelle ressemblance l'égalerez-vous~?
\VS{19}L'ouvrier fond l'image, et l'orfèvre la couvre d'or, et y soude des chaînettes d'argent.
\VS{20}Celui qui est si pauvre qu'il n'a pas de quoi faire une offrande, choisit un bois qui ne pourrisse point~; il se cherche un habile ouvrier pour faire une image taillée qui ne bouge pas\FTNT{Es. 44:9-20.}.
\VS{21}Ne le savez-vous pas~? Ne l'avez-vous pas entendu~? Cela ne vous a-t-il pas été déclaré dès le commencement~? Ne l'avez-vous pas entendu dès les fondements de la terre~?
\VS{22}C'est lui qui est assis au-dessus du globe de la terre, et à qui ses habitants sont comme des sauterelles~; c'est lui qui étend les cieux comme un voile, il les déploie même comme une tente pour y demeurer.
\VS{23}C'est lui qui réduit les princes à rien, et qui fait des chefs de la terre une chose de néant.
\VS{24}Ils ne sont pas même plantés, pas même semés, même leur tronc n'a point de racine en terre~; il souffle sur eux, et ils sèchent, et le tourbillon les emporte comme de la paille.
\VS{25}A qui donc me ferez-vous ressembler, et à qui serai-je égalé~? dit le Saint.
\VS{26}Elevez vos yeux en haut et regardez~! Qui a créé ces choses~? C'est lui qui fait sortir leur armée par ordre, et qui les appelle toutes par leur nom~; il n'y en a pas une qui fait défaut, à cause de la grandeur de sa force, et parce qu'il excelle en puissance.
\VS{27}Pourquoi donc dis-tu, ô Jacob, pourquoi dis-tu, ô Israël~: Ma voie est cachée à Yahweh, et mon jugement passe inaperçu devant mon Dieu~?
\VS{28}Ne sais-tu pas~? N'as-tu pas entendu que le Dieu d'éternité, Yahweh, a créé les extrémités de la terre~; il ne se fatigue point, il ne se lasse point, et il n'y a pas moyen de sonder son intelligence.
\VS{29}C'est lui qui donne de la force à celui qui est las, et il multiplie la force de celui qui n'a aucune vigueur.
\VS{30}Les jeunes gens se lassent et se fatiguent, même les jeunes hommes tombent sans force.
\VS{31}Mais ceux qui s'attendent à Yahweh renouvellent leur force. Ils s'élèvent avec des ailes, comme des aigles~; ils courent, et ne se fatiguent point~; ils marchent, et ne se lassent point.
\Chap{41}
\TextTitle{Dénonciation des idoles}
\VerseOne{}Iles, faites moi silence~! Que les peuples renouvellent leurs forces~; qu'ils s'approchent et qu'alors ils parlent~; allons ensemble en jugement.
\VS{2}Qui a fait lever l'homme droit de l'orient~? Qui l'a appelé à sa suite~? Qui a soumis à son commandement les nations~? Qui lui a donné la domination sur les rois~? Qui les a livrés à son épée comme de la poussière, et à son arc comme de la paille poussée par le vent~?
\VS{3}Il les a poursuivis, il est passé en paix par le chemin que son pied n'avait jamais foulé.
\VS{4}Qui est celui qui a opéré et fait ces choses~? C'est celui qui a appelé les âges dès le commencement. Moi, Yahweh, JE SUIS le premier, et JE SUIS avec les derniers\FTNT{Ap. 1:8~; Ap. 21:6~; Ap. 22:13.}.
\VS{5}Les îles voient, et sont dans la crainte, les extrémités de la terre sont effrayées~; ils s'approchent, ils viennent.
\VS{6}Chacun aide son prochain, et chacun dit à son frère~: Fortifie-toi.
\VS{7}L'ouvrier encourage le fondeur~; celui qui frappe doucement du marteau encourage celui qui frappe sur l'enclume, et il dit~: Cela est bon pour souder, puis il fixe l'idole avec des clous, afin qu'elle ne bouge pas.
\VS{8}Mais toi, Israël, tu es mon serviteur, et toi, Jacob, tu es celui que j'ai élu, la race d'Abraham qui m'a aimé~!
\VS{9}Car je t'ai pris aux extrémités de la terre, je t'ai appelé en te préférant aux plus excellents qui sont en elle, et je t'ai dit~: C'est toi qui est mon serviteur, je t'ai élu, et je ne te rejette point\FTNT{De. 7:6~; Ps. 77:8.}.
\VS{10}Ne crains rien, car je suis avec toi~; ne sois pas étonné, car je suis ton Dieu~; je te fortifie, et je t'aide, même je te soutiens par la droite de ma justice.
\VS{11}Voici, tous ceux qui sont indignés contre toi seront honteux et confus~; ils seront réduits à néant, et les hommes qui ont querelle avec toi périront.
\VS{12}Tu les chercheras, et tu ne les trouveras plus, ceux qui te suscitaient querelle~; ils seront réduits à néant, et ceux qui te font la guerre seront comme ce qui n'est plus.
\VS{13}Car je suis Yahweh, ton Dieu, qui soutiens ta main droite, et te dis~: Ne crains rien, c'est moi qui te secours.
\VS{14}Ne crains point, vermisseau de Jacob, hommes mortels d'Israël~; je viens à ton secours dit Yahweh, et ton défenseur, le Saint d'Israël.
\VS{15}Voici, je fais de toi un traîneau aigu, tout neuf, ayant des dents~; tu fouleras les montagnes et les broieras, et tu rendras les collines semblables à de la balle.
\VS{16}Tu les vanneras, et le vent les emportera, et le tourbillon les dispersera. Mais toi, tu te réjouiras en Yahweh, tu te glorifieras dans le Saint d'Israël.
\VS{17}Quant aux affligés et aux misérables qui cherchent des eaux, et n'en ont point~; dont la langue est tellement altérée qu'elle n'en peut plus~; moi, Yahweh, je les exaucerai~; moi, le Dieu d'Israël, je ne les abandonnerai pas\FTNT{Ge. 28:15~; Jos. 1:5~; Hé. 13:5.}.
\VS{18}Je ferai jaillir des fleuves sur les hauteurs et des fontaines au milieu des vallées~; et je ferai du désert des étangs d'eaux et de la terre sèche des sources d'eaux.
\VS{19}Je ferai croître au désert le cèdre, l'acacia, le myrte et l'olivier~; je mettrai dans les lieux stériles le cyprès, l'orme et le buis ensemble,
\VS{20}afin qu'on voie, qu'on sache, qu'on pense, et qu'on comprenne que la main de Yahweh a fait cela, et que le Saint d'Israël a créé cela.
\VS{21}Plaidez votre cause, dit Yahweh~; et mettez en avant les fondements de votre cause, dit le Roi de Jacob.
\VS{22}Qu'ils les amènent et qu'ils nous déclarent ce qui doit arriver. Déclarez-nous que veulent dire les choses qui ont été auparavant et nous y prendrons garde, et nous saurons leur issue, ou faites-nous entendre ce qui est prêt à arriver.
\VS{23}Déclarez les choses qui doivent arriver dorénavant, et nous saurons que vous êtes des dieux~; faites aussi du bien ou du mal, et nous en serons tout étonnés puis nous regarderons ensemble.
\VS{24}Voici, vous n'êtes rien, et votre œuvre est le néant~; celui qui vous choisit n'est qu'abomination.
\VS{25}Je l'ai suscité du nord, et il est venu~; il invoque mon Nom de devant le soleil levant~; et marche sur les princes comme sur le mortier, et les foule comme le potier foule la boue.
\VS{26}Qui est celui qui a manifesté ces choses dès le commencement, afin que nous le connaissions~? Et longtemps d'avance, que nous puissions dire~: Il est juste. Mais il n'y a personne qui les annonce, même il n'y a personne qui les donne à entendre, même il n'y a personne qui entende vos paroles.
\VS{27}Le premier sera pour Sion, disant~: Voici, les voici~! Et je donnerai quelqu'un à Jérusalem qui annoncera de bonnes nouvelles\FTNT{Es. 52:7~; Ap. 14:6.}.
\VS{28}Je regarde, et il n'y a point d'homme même entre ceux-là, et il n'y a aucun homme de conseil~; je les interroge aussi afin qu'ils répondent quelque chose.
\VS{29}Voici, quant à eux tous, leurs œuvres ne sont que vanité, leurs idoles de fonte sont du vent et de la confusion.
\Chap{42}
\TextTitle{Le messie, serviteur de Yahweh}
\VerseOne{}Voici mon serviteur, que je soutiens, c'est mon élu, en qui mon âme prend son bon plaisir~; j'ai mis mon Esprit sur lui, il manifestera le jugement aux nations\FTNT{Mt. 3:17~; Mt. 17:5~; Mc. 9:7.}.
\VS{2}Il ne criera point, et il ne haussera, ni ne fera entendre sa voix dans les rues.
\VS{3}Il ne brisera point le roseau cassé, et il n'éteindra point le lumignon qui fume\FTNT{Mt. 12:18-20.}~; il mettra en avant le jugement selon la vérité.
\VS{4}Il ne se retirera point et ne s'affaiblira point, jusqu'à ce qu'il ait établi la justice sur la terre, et que les îles s'attendent à sa loi.
\VS{5}Ainsi parle Dieu, Yahweh, qui a créé les cieux, et qui les a étendus, qui a aplani la terre avec ce qu'elle produit, qui donne la respiration au peuple qui est sur elle, et l'esprit à ceux qui y marchent.
\VS{6}Moi Yahweh, je t'ai appelé en justice, et je prendrai ta main et te garderai, et je te ferai être l'alliance du peuple et la lumière des nations\FTNT{Voir commentaire en Ge. 1:3-5.},
\VS{7}afin d'ouvrir les yeux des aveugles, et de faire sortir les prisonniers hors du lieu où on les tient enfermés, et ceux qui habitent dans les ténèbres hors de la prison.
\TextTitle{Israël n'a pas été attentif à Yahweh}
\VS{8}Je suis Yahweh, c'est là mon Nom~; et je ne donnerai pas ma gloire à un autre, ni ma louange aux images taillées\FTNT{Es. 48:11.}.
\VS{9}Voici, les choses qui ont été prédites auparavant se sont accomplies. Et je vous en annonce de nouvelles~; et je vous les fais entendre avant qu'elles arrivent.
\VS{10}Chantez à Yahweh un cantique nouveau, et que sa louange éclate aux extrémités de la terre, vous qui descendez en la mer, et tout ce qui est en elle, les îles et leurs habitants~!
\VS{11}Que le désert et ses villes élèvent la voix~! Que les villages où habite Kédar et ceux qui habitent dans les rochers éclatent en chant de triomphe~! Qu'ils s'écrient du sommet des montagnes~!
\VS{12}Qu'on donne gloire à Yahweh, et qu'on publie sa louange dans les îles~!
\VS{13}Yahweh sort comme un homme vaillant, il réveille sa jalousie comme un homme de guerre, il jette, dis-je, des cris de joie, il jette de grands cris, et il prévaut sur ses ennemis.
\VS{14}Je me suis tu dès longtemps~; me tiendrais-je en repos~? Me retiendrais-je~? Je crierai comme celle qui enfante, je détruirai, et j'engloutirai tout à la fois.
\VS{15}Je réduirai les montagnes et les collines en désert, et j'en dessécherai toute la verdure, je réduirai les fleuves en îles, et je ferai tarir les étangs.
\VS{16}Je conduirai les aveugles sur un chemin qu'ils ne connaissent pas, je les ferai marcher par des sentiers qu'ils ne connaissent pas~; je réduirai devant eux les ténèbres en lumière, et les choses tortues en choses droites~; voilà ce que je ferai, et je ne les abandonnerai point.
\VS{17}Ils se retireront en arrière, et ils seront tout honteux, ceux qui se confient aux images taillées, et qui disent aux images de fonte~: Vous êtes nos dieux~!
\VS{18}Sourds, écoutez~! Et vous aveugles, regardez et voyez~!
\VS{19}Qui, dis-je, est aveugle, sinon mon serviteur~? Et qui est sourd, comme mon messager que j'envoie~? Qui est aveugle, comme celui que j'ai comblé de grâces~? Qui est aveugle, comme le serviteur de Yahweh~?
\VS{20}Vous voyez beaucoup de choses, mais vous ne prenez garde à rien~; vous avez les oreilles ouvertes, mais vous n'entendez rien.
\VS{21}Yahweh a plaisir en lui à cause de sa justice~; il a magnifié la loi et l'a rendu honorable.
\VS{22}Mais c'est ici un peuple pillé et dépouillé~! Ils sont enlacés dans les cavernes, et sont cachés dans des prisons~; ils sont un butin, et il n'y a personne qui les délivre~; une proie, et il n'y a personne qui dise~: Restituez~!
\VS{23}Qui est celui d'entre vous qui prêtera l'oreille à ces choses~? Qui s'y rendra attentif et l'écoutera à l'avenir~?
\VS{24}Qui est-ce qui a livré Jacob au pillage, et Israël aux pillards\FTNT{Jg. 2:13-16.}~? N'est-ce pas Yahweh, contre lequel nous avons péché~? Car on n'a point voulu marcher dans ses voies et on n'a point obéi à sa loi.
\VS{25}C'est pourquoi il a répandu sur lui la fureur de sa colère, et une forte guerre~; et il l'a embrasé tout alentour, mais Israël ne l'a point connu~; et il l'a brûlé, mais il n'y a point pris garde.
\Chap{43}
\TextTitle{Yahweh veut racheter Israël}
\VerseOne{}Mais maintenant ainsi parle Yahweh, qui t'a créé, ô Jacob~! Celui qui t'a formé, ô Israël~! Ne crains point, car je te rachète, je t'appelle par ton nom, tu es à moi~!
\VS{2}Si tu passes par les eaux, je serai avec toi~; et si tu passes par les fleuves, ils ne te noieront pas~; si tu marches dans le feu, tu ne seras pas brûlé, et la flamme ne t'embrasera pas.
\VS{3}Car je suis Yahweh, ton Dieu, le Saint d'Israël, ton Sauveur. Je donne l'Egypte pour ta rançon, l'Ethiopie et Saba à ta place.
\VS{4}Parce que tu es précieux à mes yeux, tu es rendu honorable et je t'aime, je donne des hommes à ta place, et des peuples pour ta vie.
\VS{5}Ne crains point, car je suis avec toi~; je ferai venir ta postérité de l'orient, et je t'assemblerai de l'occident.
\VS{6}Je dirai au nord~: Donne~! Et au midi~: Ne retiens point~! Fais venir mes fils de loin, et mes filles du bout de la terre,
\VS{7}à savoir tous ceux qui s'appellent de mon Nom\FTNT{Dans les Ecritures, le Nom de Dieu le plus cité est YHWH. Jésus, dont le nom signifie «~YHWH est salut~» correspond au nom et à l'identité que Dieu a révélés à tous ceux qui l'ont rencontré quand il était sur cette terre. Dans sa dernière prière à Gethsémané, Jésus dit~: «~J'ai fait connaître ton Nom~» (Jn. 17:6), et «~Je leur ai fait connaître ton Nom~» (Jn. 17:26). Ce nom n'est autre que le sien puisque Jésus (YHWH est salut) était et est le Nom de Dieu. Moïse n'avait pas reçu la révélation de ce Nom (Ex. 3:13-14) car cette révélation était réservée à l'Eglise. En tant qu'épouse de Christ, l'Eglise porte le Nom du Seigneur et bénéficie de l'autorité qu'il confère. Ainsi, Jésus est le seul Nom par lequel nous pouvons être sauvés (Ac. 4:12). C'est aussi en son Nom que nous devons être baptisés (Ac. 8:16~; Ac. 19:5), que nous recevons l'exaucement de nos prières (Jn. 14:13-14~; Jn. 16:24), que nous sommes délivrés de l'ennemi et que nous obtenons la victoire sur le camp de l'ennemi (Mc. 16:17~; Ph. 2:9-11).}~; car je les ai créés pour ma gloire~; je les ai formés et les ai faits.
\TextTitle{Yahweh appelle ses témoins}
\VS{8}Amène dehors le peuple aveugle qui a des yeux, et les sourds qui ont des oreilles.
\VS{9}Que toutes les nations soient ramassées ensemble, et que les peuples soient assemblés. Lequel d'entre eux a annoncé ces choses-là~? Et qui sont ceux qui nous ont fait entendre les choses qui ont été ci-devant~? Qu'ils produisent leurs témoins et qu'ils se justifient~; qu'on les entende et qu'on dise~: C'est vrai~!
\VS{10}Vous êtes mes témoins\FTNT{Ac. 1:8.}, dit Yahweh, et mon serviteur que j'ai élu, afin que vous connaissiez, que vous me croyiez et que vous compreniez que JE SUIS. Avant moi il n'a pas été formé de Dieu, et il n'y en aura point après moi.
\VS{11}Moi, Je suis Yahweh, et à part moi il n'y a point de Sauveur\FTNT{Yahweh dit qu'à part lui, il n'y a pas d'autres sauveurs. Or les écrits de la nouvelle alliance affirment que Jésus-Christ est le seul Sauveur (Lu. 1:67-80~; Ac. 4:11-12).}.
\VS{12}C'est moi qui ai prédit ce qui devait arriver, qui vous ai sauvés, et qui vous ai fait entendre l'avenir, quand il n'y avait point de dieu étranger parmi vous~; et vous êtes mes témoins, dit Yahweh, que je suis Dieu.
\VS{13}Et même avant que le jour fût, JE SUIS, et il n'y a personne qui puisse délivrer de ma main~; je ferai l'œuvre, qui m'en empêchera~?
\TextTitle{Yahweh fera une chose nouvelle car Jacob ne l'a pas honoré}
\VS{14}Ainsi parle Yahweh, votre Rédempteur\FTNT{Es. 60:16~; 1 Co. 1:30~; Ro. 3:24~; Ep. 1:7.}, le Saint d'Israël~: J'envoie pour l'amour de vous contre Babylone, et je les fais descendre tous fugitifs, et le cri des Chaldéens sera dans les navires.
\VS{15}Je suis Yahweh, votre Saint, le Créateur d'Israël, votre Roi.
\VS{16}Ainsi parle Yahweh, qui fraya un chemin dans la mer, et un sentier parmi les eaux impétueuses~;
\VS{17}qui amena des chars et des chevaux, et de grandes forces~; ils ont été étendus ensemble, et ils ne se relèveront point, ils ont été étouffés, ils ont été éteints comme un lumignon~:
\VS{18}Ne pensez plus aux choses passées, et ne considérez point les choses anciennes.
\VS{19}Voici, je m'en vais faire une chose nouvelle\FTNT{2 Co. 5:17.}, qui paraîtra bientôt, ne la connaîtrez-vous pas~? Je mettrai un chemin dans le désert, et des fleuves dans le lieu de désolation.
\VS{20}Les bêtes des champs me glorifieront, les serpents et les autruches, parce que j'aurai mis des eaux dans le désert, et des fleuves dans la solitude, pour abreuver mon peuple que j'ai élu.
\VS{21}Ce peuple que je me suis formé racontera mes louanges.
\VS{22}Mais toi, Jacob, tu ne m'as pas invoqué, car tu t'es lassé de moi, ô Israël~!
\VS{23}Tu ne m'as pas offert le menu bétail de tes holocaustes, et tu ne m'as pas glorifié dans tes sacrifices~; je ne t'ai point asservi pour me faire des offrandes, et je ne t'ai point fatigué pour de l'encens.
\VS{24}Tu ne m'as pas acheté à prix d'argent du roseau aromatique, et tu ne m'as pas rassasié de la graisse de tes sacrifices~; mais tu m'as asservi par tes péchés, et tu m'as peiné par tes iniquités.
\VS{25}Moi, JE SUIS celui qui efface tes transgressions pour l'amour de moi, et je ne me souviendrai plus de tes péchés.
\VS{26}Réveille ma mémoire, et plaidons ensemble~; toi, déclare pour que tu puisses être justifié.
\VS{27}Ton premier père a péché, et tes docteurs se sont rebellés contre moi.
\VS{28}C'est pourquoi j'ai profané les chefs du lieu saint, et j'ai livré Jacob à la destruction, et Israël à l'opprobre.
\Chap{44}
\TextTitle{Promesse de l'Esprit, folie de l'idolâtrie}
\VerseOne{}Ecoute maintenant, ô Jacob, mon serviteur, et toi Israël que j'ai choisi~!
\VS{2}Ainsi parle Yahweh, qui t'a fait et formé dès le ventre, celui qui te soutient~: Ne crains point, ô Jacob, mon serviteur~! Et toi Jeshurun\FTNT{Jeshurun : «~Celui qui est droit~». Il s'agit du nom symbolique de Israël décrivant son caractère idéal.} que j'ai élu.
\VS{3}Car je répandrai des eaux sur celui qui est altéré, et des rivières sur la terre sèche~; je répandrai mon Esprit sur ta postérité, et ma bénédiction sur ta descendance.
\VS{4}Et ils germeront comme au milieu de l'herbe, comme les saules auprès des courants d'eau.
\VS{5}L'un dira~: Je suis à Yahweh~; et l'autre se réclamera du nom de Jacob~; et un autre écrira de sa main~: Je suis à Yahweh, et se nommera du nom d'Israël.
\VS{6}Ainsi parle Yahweh, le Roi d'Israël et son Rédempteur, Yahweh des armées~: Je suis le premier, et je suis le dernier~; et à part moi il n'y a point de Dieu.
\VS{7}Et qui, comme moi, a appelé, déclaré et ordonné cela, depuis que j'ai établi le peuple ancien~? Qu'ils déclarent les choses à venir, les choses qui arriveront ci-après~!
\VS{8}Ne soyez point effrayés et ne soyez point troublés~; ne te l'ai-je pas fait entendre, et déclaré dès ce temps-là~? Vous êtes mes témoins~; y a-t-il un autre Dieu que moi~? Certes il n'y a pas d'autre Rocher\FTNT{Yahweh dit qu'il ne connaît pas d'autre rocher. Jésus-Christ est ce rocher qui suivait les Hébreux dans le désert (Mt. 16:18~; 1 Co. 10:1-4. Voir aussi commentaire en Es. 8:14).}, je n'en connais pas.
\VS{9}Les ouvriers d'images taillées ne sont tous que vanité, et leurs choses les plus désirables ne sont d'aucun profit~; elles le témoignent elles-mêmes, elles ne voient point, et ne connaissent point, afin qu'ils soient honteux.
\VS{10}Mais qui est-ce qui fabrique un dieu, ou fond une image taillée, pour n'en avoir aucun profit~?
\VS{11}Voici, tous ses compagnons seront honteux, car ces ouvriers-là sont d'entre les hommes. Qu'ils s'assemblent tous, qu'ils se tiennent là~! Ils seront effrayés et rendus honteux tous ensemble.
\VS{12}Le forgeron fait une hache, et il travaille avec le charbon, et il le forme à coups de marteau~; il le fait à force de bras, même il a faim et il est sans force, il ne boit point d'eau, et il est tout fatigué.
\VS{13}Le charpentier étend sa règle, il trace sa forme au crayon avec de la craie~; il le fait avec des équerres, et le forme au compas, et le fait à la ressemblance d'un homme, selon la beauté d'un homme, afin qu'il demeure dans la maison.
\VS{14}Il se coupe des cèdres, et prend un cyprès, ou un chêne, qu'il a laissé croître parmi les arbres de la forêt~; il plante des pins, et la pluie les fait croître.
\VS{15}Ces arbres servent à l'homme pour brûler, car il en prend et il s'en chauffe. Il en fait du feu, dis-je, et en cuit du pain~; et il en fait aussi un dieu et se prosterne devant lui~; il en fait une image taillée et l'adore.
\VS{16}Il en brûle au feu une partie, et d'une autre partie il mange sa chair, laquelle il rôtit, et s'en rassasie~; il s'en chauffe aussi, et il dit~: Ah~! Ah~! Je me chauffe, je vois la flamme~!
\VS{17}Puis avec le reste il fait un dieu pour être son image taillée~; il se prosterne devant elle, il l'adore, il lui fait sa requête, et dit~: Délivre-moi, car tu es mon dieu~!
\VS{18}Ils ne savent et n'entendent rien, car on leur a plâtré les yeux afin qu'ils ne voient point, et les cœurs pour qu'ils ne comprennent point.
\VS{19}Nul ne rentre en lui-même\FTNT{So. 2:1~; 2 Co. 13:5.}, et il n'a ni la connaissance ni l'intelligence pour dire~: J'en ai brûlé une partie au feu, et même j'ai cuit du pain sur les charbons, j'ai rôti de la viande et je l'ai mangée~; et avec le reste ferais-je une abomination~? Adorerais-je une branche de bois~?
\VS{20}Il se repaît de cendres, et son cœur abusé l'égare, et il ne délivrera point son âme, et ne dira point~: N'est-ce pas du mensonge que j'ai dans ma main droite~?
\TextTitle{Yahweh rachète son peuple}
\VS{21}Souviens-toi de ces choses, ô Jacob~! Ô Israël, car tu es mon serviteur~; je t'ai formé, tu es mon serviteur, ô Israël~! je ne t'oublierai pas.
\VS{22}J'efface tes transgressions comme une nuée épaisse, et tes péchés comme une nuée~; reviens à moi, car je t'ai racheté.
\VS{23}Ô cieux~! Réjouissez-vous avec chants de triomphe, car Yahweh a agi~; profondeurs de la terre, poussez des cris de réjouissance~! Montagnes, éclatez de joie avec chant de triomphe~! Et vous aussi forêts et tous les arbres qui êtes en elles~! Parce que Yahweh a racheté Jacob, et s'est manifesté glorieusement en Israël.
\VS{24}Ainsi parle Yahweh, ton Rédempteur, celui qui t'a formé dès le ventre~: Je suis Yahweh qui ai fait toutes choses, qui seul ai étendu les cieux, et qui ai par moi-même étendu la terre~;
\VS{25}qui dissipe les signes des menteurs, qui rends insensés les devins~; qui renverse l'esprit des sages, et qui change leur science en folie.
\VS{26}C'est lui qui confirme la parole de son serviteur, et accomplit le conseil de ses messagers~; qui dit à Jérusalem~: Tu seras encore habitée~! Et aux villes de Juda~: Vous serez rebâties~! Et je redresserai ses lieux déserts.
\VS{27}Qui dit à l'abîme~: Sois asséchée, et je tarirai tes fleuves.
\TextTitle{Prophétie sur le rétablissement d'Israël par Cyrus}
\VS{28}Qui dit de Cyrus\FTNT{Esaïe prophétisa la destruction de Babylone deux siècles avant la réalisation de cet événement, le 5 octobre 539 av. J.-C. Fait remarquable~: il précisa même le nom du commandant Cyrus qui dompta le lion babylonien. L'historien Hérodote donnera par ailleurs raison au prophète sur le déroulement de la prise de Babylone.}~: Il est mon berger, et il accomplira tout mon bon plaisir~; disant même à Jérusalem~: Tu seras rebâtie~! Et au temple~: Tu seras fondé.
\Chap{45}
\TextTitle{Cyrus suscité par Yahweh}
\VerseOne{}Ainsi parle Yahweh à son oint, à Cyrus\FTNT{Cyrus le Grand (580 av. J.-C. - 530 av. J.-C.). Voir Esd. 1.},
\VS{2}que je tiens par la main droite, pour terrasser les nations devant lui, et pour délier les ceintures des rois, pour ouvrir devant lui les portes, afin qu'elles ne soient point fermées.
\VS{3}J'irai devant toi, et j'aplanirai les lieux tortueux~; je romprai les portes d'airain, et je mettrai en pièces les barres de fer. Et je te donnerai des trésors cachés, et des richesses le plus secrètement gardées, afin que tu saches que je suis Yahweh, le Dieu d'Israël, qui t'appelle par ton nom.
\VS{4}Pour l'amour de Jacob, mon serviteur, et d'Israël mon élu~; je t'ai, dis-je, appelé par ton nom, et je t'ai surnommé avant que tu me connaisses.
\TextTitle{Yahweh, le seul Dieu}
\VS{5}Je suis Yahweh, et il n'y en a point d'autre~; à part moi, il n'y a point de Dieu. Je t'ai ceint avant que tu me connaisses,
\VS{6}afin que l'on sache, du soleil levant au soleil couchant, qu'à part moi, il n'y a point de Dieu. Je suis Yahweh, et il n'y en a point d'autre.
\VS{7}Je forme la lumière, et je crée les ténèbres~; je fais la paix et je crée l'adversité~; moi, Yahweh, je fais toutes ces choses.
\VS{8}Ô cieux~! Répandez la rosée d'en haut, et que les nuées laissent couler la justice~! Que la terre s'ouvre, qu'elle produise le salut, qu'elle fasse également germer la justice~! Moi, Yahweh, je crée ces choses.
\VS{9}Malheur à celui qui plaide contre celui qui l'a façonné~! Que le vase plaide contre les autres vases de terre~! Mais l'argile dira-t-elle à celui qui l'a façonnée~: Que fais-tu~? Et tu n'as pas d'adresse pour ton ouvrage\FTNT{Jé. 18:6~; Ro. 9:21.}~?
\VS{10}Malheur à celui qui dit à son père~: Pourquoi engendres-tu~? Et à sa mère~: Pourquoi enfantes-tu~?
\VS{11}Ainsi parle Yahweh, le Saint d'Israël, qui est son Créateur~: Interrogez-moi sur les choses à venir, mes fils~; me commanderez-vous sur l'œuvre de mes mains~?
\VS{12}C'est moi qui ai fait la terre et qui ai créé l'homme sur elle~; c'est moi qui ai étendu les cieux de mes mains, et qui ai donné la loi à toute leur armée.
\VS{13}C'est moi qui ai suscité Cyrus dans ma justice, et j'aplanirai toutes ses voies~; il rebâtira ma ville, et libérera mes captifs\FTNT{Cyrus le grand libéra les Juifs après 70 ans de captivité (Esd. 1).}, sans rançon ni présents, dit Yahweh des armées.
\TextTitle{Les autres peuples reconnaîtront la main de Yahweh sur Israël}
\VS{14}Ainsi parle Yahweh~: Le travail de l'Egypte, et le trafic de l'Ethiopie, et ceux des Sabéens, gens de grande stature, passeront chez toi Jérusalem, et seront à toi~; ils marcheront à ta suite, ils passeront enchaînés, ils se prosterneront devant toi, ils te diront en suppliant~: Certainement, Dieu est au milieu de toi, et il n'y a point d'autre Dieu que lui.
\VS{15}En vérité, tu es le Dieu qui te caches, le Dieu d'Israël, le Sauveur.
\VS{16}Ils sont tous honteux et confus, ils s'en vont tous avec ignominie, les fabricants d'idoles.
\VS{17}Mais Israël a été sauvé par Yahweh, d'un salut éternel~; vous ne serez ni honteux ni confus jusque dans l'éternité.
\VS{18}Car ainsi parle Yahweh qui a créé les cieux, lui qui est le Dieu qui a formé la terre, qui l'a faite et qui l'a affermie~; qui l'a créée pour qu'elle ne soit pas informe\FTNT{Voir commentaire en Ge. 1:2.}, qui l'a formée pour qu'elle soit habitée~; JE SUIS Yahweh, et il n'y en a point d'autre.
\VS{19}Je n'ai point parlé en secret ni dans quelque lieu ténébreux de la terre~; je n'ai point dit à la postérité de Jacob~: Cherchez-moi vainement~! JE SUIS Yahweh, qui prononce ce qui est juste, qui déclare ce qui est droit.
\VS{20}Assemblez-vous et venez, approchez-vous ensemble, vous les réchappés des nations~! Ceux qui portent le bois de leur image taillée ne savent rien, et invoquent un dieu qui ne sauve pas.
\VS{21}Déclarez-le, et faites-les approcher~! Qu'ils prennent conseil ensemble~! Qui a fait entendre ces choses dès l'origine, et les a déclarées dès longtemps~? N'est-ce pas moi, Yahweh~? Or il n'y a point d'autre Dieu à part moi~; un Dieu juste et un Sauveur, il n'y en a pas d'autre à part moi.
\VS{22}Vous tous qui êtes aux extrémités de la terre, regardez vers moi, et soyez sauvés~; car JE SUIS Dieu, et il n'y en a point d'autre.
\VS{23}Je le jure par moi-même, la parole est sortie avec justice de ma bouche, et elle ne sera point révoquée~: Tout genou fléchira devant moi, et toute langue jurera par moi\FTNT{Ph. 2:9-11.}.
\VS{24}Certainement, on dira de moi~: En Yahweh seul sont la justice et la force~; à lui viendront, pour être confondus, tous ceux qui étaient irrités contre lui.
\VS{25}Toute la postérité d'Israël sera justifiée, et elle se glorifiera en Yahweh.
\Chap{46}
\TextTitle{La puissance de Yahweh, l'incapacité des idoles}
\VerseOne{}Bel s'incline sur ses genoux, Nebo est renversé~; leurs faux dieux sont mis sur leurs bêtes et sur leur bétail~; les idoles que vous portiez, ont été chargées, elles sont un fardeau pour la bête fatiguée~!
\VS{2}Elles se sont courbées, elles se sont inclinées ensemble sur leurs genoux, et ne peuvent échapper au fardeau, et elles-mêmes s'en vont en captivité.
\VS{3}Ecoutez-moi, maison de Jacob, et vous tous, tout le reste de la maison d'Israël, dont je me suis chargé dès le ventre, et que j'ai porté dès le sein maternel.
\VS{4}Jusqu'à votre vieillesse, JE SUIS~; et je vous chargerai sur moi jusqu'à votre blanche vieillesse, je l'ai fait, et je vous porterai encore, je vous chargerai sur moi et vous sauverai.
\VS{5}A qui me ferez-vous ressembler, et à qui m'égalerez-vous~? A qui me comparerez-vous pour que nous soyons semblables~?
\VS{6}Ils tirent l'or de la bourse, et pèsent l'argent à la balance, et ils engagent un orfèvre pour en faire un dieu~; ils l'adorent, et se prosternent devant lui.
\VS{7}On le porte sur les épaules, on s'en charge~; on le pose en sa place où il se tient debout et ne bouge point de son lieu, puis on crie à lui, mais il ne répond pas, et il ne délivre pas de la détresse ceux qui crient vers lui.
\VS{8}Souvenez-vous de cela, et montrez-vous des hommes~; rappelez-le à votre pensée, ô vous transgresseurs~!
\VS{9}Souvenez-vous des premières choses d'autrefois~; JE SUIS Dieu, et il n'y en a point d'autre, JE SUIS Dieu et il n'y en a point comme moi~;
\VS{10}qui déclare dès le commencement ce qui doit arriver à la fin, et longtemps auparavant, les choses qui n'ont pas encore été faites~; qui dis~: Mon conseil tiendra, et j'exécuterai tout mon bon plaisir~;
\VS{11}qui appelle de l'orient l'oiseau de proie, et d'une terre éloignée un homme pour exécuter mon conseil. Oui, j'ai parlé, aussi je ferai venir la chose~; je l'ai formée, aussi je l'accomplirai.
\VS{12}Ecoutez-moi, vous qui avez le cœur endurci et qui êtes éloignés de la justice.
\VS{13}Je fais approcher ma justice, elle ne s'éloignera point loin~; et mon salut, il ne tardera pas. Je mettrai le salut en Sion pour Israël, qui est ma gloire.
\Chap{47}
\TextTitle{Jugement sur Babylone}
\VerseOne{}Descends, et assieds-toi dans la poussière, vierge, fille de Babylone~! Assieds-toi à terre, il n'y a plus de trône pour la fille des Chaldéens~! Car tu ne te feras plus appeler la délicate et la voluptueuse.
\VS{2}Mets la main aux meules, et fais moudre la farine~; délie tes tresses, déchausse-toi, découvre tes jambes et traverse les fleuves~!
\VS{3}Ta honte sera découverte et ton opprobre sera vu~; je prendrai vengeance, je n'irai point contre toi en homme.
\VS{4}Quant à notre Rédempteur, son Nom est Yahweh des armées, le Saint d'Israël.
\VS{5}Assieds-toi sans dire mot, et entre dans les ténèbres, fille des Chaldéens, car tu ne te feras plus appeler la dame des royaumes.
\VS{6}J'ai été embrasé de colère contre mon peuple, j'ai profané mon héritage, c'est pourquoi je les ai livrés entre tes mains, mais tu n'as point usé de miséricorde envers eux, tu as durement appesanti ton joug sur le vieillard.
\VS{7}Et tu as dit~: Je serai dame à toujours~! de sorte que tu n'as point mis ces choses-là dans ton cœur, tu ne t'es point souvenue ce qu'en serait la fin.
\VS{8}Maintenant donc écoute ceci, toi voluptueuse qui habite avec assurance, et qui dis en ton cœur~: C'est moi, et il n'y en a point d'autre que moi~; je ne deviendrai point veuve, et je ne saurai point ce que c'est que d'être privée d'enfants.
\VS{9}Mais ces deux choses t'arriveront en un moment, en un même jour, la privation d'enfants et le veuvage~; elles viendront sur toi dans leur perfection, pour le grand nombre de tes sortilèges, et pour la grande abondance de tes enchantements\FTNT{Ap. 18:7-8.}.
\VS{10}Et tu t'es confiée dans ta méchanceté, et tu disais~: Personne ne me voit~! Ta sagesse et ta science t'ont pervertie, et tu disais en ton cœur~: C'est moi, et il n'y en a point d'autre que moi.
\VS{11} C'est pourquoi le mal viendra sur toi, et tu ne sauras pas quand il sera près d'arriver, et le malheur qui tombera sur toi sera tel, que tu ne pourras pas le détourner~; et la ruine éclatante que tu n'as pas soupçonnée viendra sur toi subitement.
\VS{12}Tiens-toi maintenant avec tes enchantements, et avec le grand nombre de tes sortilèges, après lesquels tu as travaillé dès ta jeunesse~; peut-être pourras-tu en tirer quelque profit~; peut-être en seras-tu renforcée.
\VS{13}Tu t'es lassée à force de demander des conseils. Que les spectateurs des cieux qui contemplent les étoiles, et qui font leurs prédictions selon les lunes, comparaissent maintenant, et qu'ils te délivrent des choses qui viendront sur toi.
\VS{14}Voici, ils sont devenus comme de la paille, le feu les consume, ils ne délivreront pas leur vie du pouvoir de la flamme~; il n'y a point de charbon pour se chauffer et il n'y a point de lueur de feu pour s'asseoir vis-à-vis.
\VS{15}Tels sont devenus pour toi ceux avec lesquels tu as travaillé et avec lesquels tu as trafiqué dès ta jeunesse, ils erreront chacun de son côté comme un vagabon~; il n'y a personne pour te sauver.
\Chap{48}
\TextTitle{Yahweh rappelle ses promesses}
\VerseOne{}Ecoutez ceci, maison de Jacob, qui êtes appelés du nom d'Israël, et qui êtes sortis des eaux de Juda~; qui jurez par le Nom de Yahweh, et qui faites mention du Dieu d'Israël, mais non pas conformément à la vérité et à la justice\FTNT{Jé. 5:2.}.
\VS{2}Car ils prennent leur nom de la sainte cité, et ils s'appuient sur le Dieu d'Israël, dont le Nom est Yahweh des armées\FTNT{Ex. 20:7.}.
\VS{3}J'ai déclaré les premières choses dès le commencement, elles sont sorties de ma bouche et je les ai publiées~; je les ai faites subitement et elles se sont accomplies.
\VS{4}Parce que je savais que tu es obstiné, que ton cou est une barre de fer, et que ton front est d'airain,
\VS{5}je t'ai déclaré ces choses dès lors, et je les ai faites entendre avant qu'elles arrivent, de peur que tu ne dises~: Mes dieux ont fait ces choses~; mon image taillée, et mon image de fonte les ont ordonnées.
\VS{6}Tu l'entends~! Vois tout cela~! Et vous, ne l'annoncerez-vous pas~? Je te fais entendre dès maintenant des choses nouvelles, et qui étaient en réserve et que tu ne savais pas.
\VS{7}Elles sont créées maintenant, et non pas depuis le commencement~; et avant ce jour-ci tu n'en avais rien entendu, afin que tu ne dises pas~: Voici, je les savais bien.
\VS{8}Oui, tu n'en avais pas entendu parler, oui, tu ne savais pas~; oui, depuis ce temps ton oreille n'a pas été ouverte~; car je savais que tu trahirais, que tu trahirais~; aussi tu as été appelé transgresseur dès le ventre.
\VS{9}Pour l'amour de mon Nom, je diffère ma colère~; et pour l'amour de ma louange, je retiens mon courroux contre toi, afin de ne pas te retrancher.
\VS{10}Voici, je t'ai épuré, mais non pas comme on épure l'argent~; je t'ai éprouvé au creuset de l'affliction.
\VS{11}Pour l'amour de moi, pour l'amour de moi, je le ferai, car comment mon Nom serait-il profané~? Certes, je ne donnerai pas ma gloire à un autre
\VS{12}Ecoute-moi, Jacob~! Et toi Israël, mon appelé~; moi, JE SUIS le premier, JE SUIS aussi le dernier.
\VS{13}Ma main aussi a fondé la terre et ma droite a étendu les cieux~; quand je les appelle, ils comparaissent ensemble.
\VS{14}Vous tous, assemblez-vous et écoutez~! Lequel parmi eux a déclaré ces choses~? Yahweh l'aime et exécutera son bon plaisir contre Babylone, et son bras sera contre les Chaldéens.
\VS{15}Moi, JE SUIS celui qui ai parlé, je l'ai aussi appelé, je l'ai amené, et ses desseins réussiront.
\VS{16}Approchez-vous de moi et écoutez ceci~! Dès le commencement, je n'ai point parlé en secret, depuis l'origine de ces choses, JE SUIS là. Or maintenant, le Seigneur, Yahweh, et son Esprit m'ont envoyé.
\VS{17}Ainsi parle Yahweh, ton Rédempteur, le Saint d'Israël~: JE SUIS Yahweh, ton Dieu, qui t'enseigne pour ton profit, et qui te guide dans le chemin où tu dois marcher.
\VS{18}Ô~! Si tu étais attentif à mes commandements, ta paix serait comme un fleuve, et ta justice comme les flots de la mer\FTNT{Jos. 1:8~; Ps. 1:2~; Jn. 14:21~; Ja. 1:22.},
\VS{19}ta postérité serait comme le sable, et ceux qui sortent de tes entrailles comme les grains de sable\FTNT{Ge. 15:5~; Ge. 22:17~; Ge. 32:12.}~; son nom ne serait point retranché ni effacé de devant ma face.
\VS{20}Sortez de Babylone, fuyez loin des Chaldéens~! Publiez ceci avec une voix de chant de triomphe, annoncez-le, portez-ceci jusqu'aux extrémités de la terre, dites~: Yahweh a racheté son serviteur Jacob~!
\VS{21}Et ils n'auront pas soif quand il les fera marcher dans les déserts~; il fera découler pour eux l'eau hors du rocher, même il leur fendra le rocher, et les eaux couleront.
\VS{22}Il n'y a point de paix pour les méchants, dit Yahweh.
\Chap{49}
\TextTitle{Le Messie, la lumière de tous les peuples}
\VerseOne{}Iles, écoutez-moi~! Soyez attentifs, vous peuples éloignés~! Yahweh m'a appelé dès le ventre, il a fait mention de mon nom dès les entrailles de ma mère\FTNT{Jé. 1:5~; Ps. 139:16.}.
\VS{2}Et il a rendu ma bouche semblable à une épée aiguë~; il m'a caché dans l'ombre de sa main, et m'a rendu semblable à une flèche bien polie, il m'a serré dans son carquois.
\VS{3}Et il m'a dit~: Tu es mon serviteur, ô Israël, en qui je serai glorifié.
\VS{4}Et moi j'ai dit~: J'ai travaillé en vain, j'ai consumé ma force pour néant et sans fruit~; toutefois mon jugement est auprès de Yahweh, et ma récompense est auprès de mon Dieu.
\VS{5}Maintenant donc, Yahweh, qui m'a formé dès le ventre pour être à son service, m'a dit que je lui ramène Jacob, mais Israël ne se rassemble point~; toutefois je serai honoré aux yeux de Yahweh, et mon Dieu sera ma force.
\VS{6}Il me dit~: C'est peu de chose que tu sois mon serviteur pour relever les tribus de Jacob et pour ramener les restes d'Israël~; c'est pourquoi je te donne pour lumière aux nations, afin que tu sois mon salut jusqu'aux extrémités de la terre.
\VS{7}Ainsi parle Yahweh, le Rédempteur, le Saint d'Israël, à celui qu'on méprise, à celui qui est abominable au peuple, au serviteur de ceux qui dominent~; les rois le verront, et se lèveront, et les princes aussi, et ils se prosterneront devant lui, pour l'amour de Yahweh, qui est fidèle, et du Saint d'Israël qui t'a élu.
\VS{8}Ainsi parle Yahweh~: Je t'ai exaucé au temps de la bienveillance, et je t'ai aidé au jour du salut~; je te garderai, et je te donnerai pour être l'alliance du peuple, pour relever la terre, afin que tu possèdes les héritages désolés~;
\VS{9}disant à ceux qui sont emprisonnés~: Sortez~! Et à ceux qui sont dans les ténèbres~: Montrez-vous~! Ils paîtront sur les chemins, et leurs pâturages seront sur tous les lieux élevés.
\VS{10}Ils n'auront pas faim et ils n'auront pas soif~; la chaleur et le soleil ne les frapperont plus, car celui qui a pitié d'eux sera leur guide, et les conduira vers des sources d'eaux\FTNT{Ps. 121:6~; Lu. 1:67-79.}.
\VS{11}Et je réduirai toutes mes montagnes en chemins, et mes sentiers seront relevés.
\VS{12}Voici, ceux-ci viennent de loin, et voici ceux-là viennent du nord et de l'occident, et les autres du pays de Sinim.
\VS{13}Ô cieux, réjouissez-vous avec des chants de triomphe~! Et toi, ô terre, sois dans l'allégresse~! Et vous, ô montagnes, éclatez de joie avec des chants de triomphe~! Car Yahweh console son peuple, il a compassion de ceux qu'il a affligés.
\VS{14}Mais Sion disait~: Yahweh me délaisse, le Seigneur m'oublie~!
\VS{15}Une femme peut-elle oublier son enfant qu'elle allaite de sorte qu'elle n'ait pas pitié du fils de ses entrailles~? Mais quand les femmes les oublieraient, moi je ne t'oublierai point.
\VS{16}Voici, je t'ai gravé sur les paumes de mes mains~; tes murs sont continuellement devant moi.
\VS{17}Tes enfants viennent à grande hâte, mais ceux qui te détruisaient et ceux qui te réduisaient en désert, sortiront du milieu de toi.
\VS{18}Elève tes yeux autour de toi, et regarde~: Tous ceux-ci s'assemblent, ils viennent à toi. JE SUIS vivant, dit Yahweh, tu te revêtiras de tous comme d'une parure, et tu t'en orneras comme une épouse.
\VS{19}Car tes déserts, tes ruines, et ton pays détruit seront désormais trop étroits pour ses habitants, et ceux qui t'engloutissaient s'éloigneront.
\VS{20}Les enfants que tu auras après avoir perdu les autres diront encore, à tes oreilles~: Le lieu est trop étroit pour moi, fais-moi de la place pour que je puisse y demeurer.
\VS{21}Et tu diras en ton cœur~: Qui m'a engendré ceux-ci vu que j'avais perdu mes enfants et que j'étais stérile, emmenée en captivité et agitée~? Et qui m'a nourri ceux-ci~? Voici, j'étais restée toute seule, et ceux-ci où étaient-ils~?
\VS{22}Ainsi parle le Seigneur Yahweh~: Voici, je lèverai ma main vers les nations et je dresserai ma bannière vers les peuples~; et ils ramèneront tes fils entre leurs bras, et ils porteront tes filles sur les épaules.
\VS{23}Et les rois seront tes nourriciers et leurs princesses, leurs femmes, tes nourrices~; ils se prosterneront devant toi le visage contre terre, et ils lécheront la poussière de tes pieds~; et tu sauras que JE SUIS Yahweh, et que ceux qui se confient en moi ne seront point confus\FTNT{Ps. 22:5-6~; Ps. 69:7~; Ro. 9:33~; 1 Pi. 2:6.}.
\VS{24}Le butin sera-t-il ôté à l'homme puissant~? Et les captifs du juste seront-ils délivrés~?
\VS{25}Car ainsi parle Yahweh~: Même les captifs pris par l'homme puissant lui seront ôtés, et le butin de l'homme fort lui sera enlevé~; car je plaiderai moi-même avec ceux qui plaident contre toi, et je délivrerai tes enfants.
\VS{26}Et je ferai manger leur propre chair à ceux qui t'oppriment~; et ils s'enivreront de leur sang comme du moût, et toute chair connaîtra que je suis Yahweh, ton Sauveur, ton Rédempteur, le Puissant de Jacob.
\Chap{50}
\TextTitle{Avertissements de Yahweh par son serviteur}
\VerseOne{}Ainsi parle Yahweh~: Où est la lettre de divorce par laquelle j'ai répudié votre mère\FTNT{De. 24:1~; Jé. 3:8~; Mt. 5:31.}~? Ou bien, auquel de mes créanciers vous ai-je vendus~? Voici, vous avez été vendus à cause de vos iniquités, et votre mère a été répudiée à cause de vos transgressions.
\VS{2}JE SUIS venu~: Pourquoi ne s'est-il trouvé personne~? J'ai appelé~: Pourquoi personne n'a-t-il répondu~? Ma main est-elle trop courte pour racheter\FTNT{No. 11:23~; Es. 59:1.}~? Ou n'y a-t-il plus de force en moi pour délivrer~? Voici, par ma menace, je dessèche la mer, je réduis les fleuves en désert~; leurs poissons se corrompent faute d'eau, et ils meurent de soif.
\VS{3}Je revêts les cieux de noirceur, et je fais d'un sac leur couverture.
\VS{4}Le Seigneur Yahweh m'a donné la langue des savants, pour que je sache soutenir par la parole celui qui est accablé de maux\FTNT{Job 6:14~; 1 Th. 5:14.}~; chaque matin il me réveille soigneusement afin que je prête l'oreille aux discours des sages.
\VS{5}Le Seigneur Yahweh m'a ouvert l'oreille et je n'ai pas été rebelle, et je ne me suis pas retiré en arrière.
\VS{6}J'ai exposé mon dos à ceux qui me frappaient et mes joues à ceux qui me tiraient le poil~; je n'ai pas caché mon visage aux opprobres et aux crachats\FTNT{Mt. 5:39~; Mt. 26:67~; Lu. 6:29~; Lu. 18:32.}.
\VS{7}Mais le Seigneur Yahweh m'a aidé, c'est pourquoi je n'ai point été confus, et ainsi j'ai rendu mon visage semblable à un caillou\FTNT{Ez. 3:8-9.}, car je sais que je ne serai point rendu honteux.
\VS{8}Celui qui me justifie est proche~; qui plaidera contre moi~? Comparaissons ensemble~! Qui est mon adversaire~? Qu'il s'approche de moi~!
\VS{9}Voici, le Seigneur Yahweh m'aidera, qui est celui qui me condamnera~? Voici, tous seront usés comme un vêtement, la teigne les dévorera.
\VS{10}Qui est celui d'entre vous qui craint Yahweh, et qui obéit à la voix de son serviteur~? Que celui qui marche dans les ténèbres, et qui n'a pas de clarté, se confie dans le Nom de Yahweh, et qu'il s'appuie sur son Dieu.
\VS{11}Voici, vous tous qui allumez le feu, et qui vous ceignez d'étincelles, marchez à la lueur de votre feu et dans les étincelles que vous avez embrasées~; voici ce que vous aurez de ma main~; vous vous coucherez dans les tourments.
\Chap{51}
\TextTitle{Exhortation à ceux qui recherchent Yahweh}
\VerseOne{}Ecoutez-moi, vous qui poursuivez la justice et qui cherchez Yahweh~! Regardez au rocher d'où vous avez été taillés, et au creux de la citerne dont vous avez été tirés.
\VS{2}Regardez à Abraham, votre père, et à Sara qui vous a enfantés~; car lui seul je l'ai appelé, je l'ai béni et multiplié\FTNT{Ro. 4:1-16~; Hé. 11:8-12.}.
\VS{3}Car Yahweh console Sion, il console toutes ses désolations, il rendra son désert semblable à Eden, et sa terre aride à un jardin de Yahweh. En elle sera trouvée la joie et l'allégresse, la reconnaissance et la voie de mélodie.
\VS{4}Ecoutez-moi donc attentivement, mon peuple, et prêtez-moi l'oreille, vous ma nation~; car la loi sortira de moi, et j'établirai mon jugement pour être la lumière des peuples.
\VS{5}Ma justice est proche, mon salut va paraître, et mes bras jugeront les peuples~; les îles espéreront en moi, elles se confieront en mon bras.
\VS{6}Levez les yeux vers les cieux et regardez en bas sur la terre~! Car les cieux s'évanouiront comme la fumée, et la terre tombera en lambeaux comme un vêtement, et ses habitants périront pareillement~; mais mon salut demeurera éternellement, et ma justice ne sera point anéantie.
\VS{7}Ecoutez-moi, vous qui connaissez la justice, peuple dans le cœur duquel est ma loi~! Ne craignez point l'opprobre des hommes et ne soyez point effrayés devant leurs outrages.
\VS{8}Car la teigne les rongera comme un vêtement\FTNT{Mt. 6:19~; Lu. 12:33~; Ja. 5:2.}, et la gerce les dévorera comme de la laine~; mais ma justice demeurera toujours, et mon salut d'âge en âge.
\VS{9}Réveille-toi, réveille-toi, revêts-toi de force, bras de Yahweh~! Réveille-toi comme aux jours anciens, aux siècles passés. N'es-tu pas celui qui tailla en pièce l'Egypte, et qui blessa mortellement le dragon~?
\VS{10}N'est-ce pas toi qui fis tarir la mer, les eaux du grand abîme~? Qui réduisit les lieux les plus profonds de la mer en un chemin afin que les rachetés y passent~?
\VS{11}Ainsi ceux dont Yahweh aura payé la rançon, retourneront, ils iront à Sion avec chants de triomphe~; et une allégresse éternelle couronnera leurs têtes~; ils obtiendront la joie et l'allégresse~; la douleur et le gémissement s'enfuiront.
\VS{12}C'est moi, c'est moi qui vous console. Qui es-tu pour avoir peur de l'homme mortel qui mourra, et du fils de l'homme qui deviendra comme du foin~?
\VS{13}Et tu oublierais Yahweh qui t'a fait, qui a étendu les cieux et fondé la terre~! Et chaque jour tu tremblerais continuellement à cause de la fureur de ton oppresseur parce qu'il s'apprête à détruire~! Et où est maintenant la fureur de ton oppresseur~?
\VS{14}Il se hâtera de faire que celui qui aura été transporté d'un lieu à l'autre, soit mis en liberté, afin qu'il ne meure point dans la fosse, et que son pain ne lui manque pas.
\VS{15}Car JE SUIS Yahweh, ton Dieu, qui fend la mer, et les flots rugissants. Yahweh des armées est son Nom.
\VS{16}Or je mets mes paroles dans ta bouche, et je te couvre de l'ombre de ma main, afin que j'affermisse les cieux, que je fonde la terre, et que je dise à Sion~: Tu es mon peuple~!
\VS{17}Réveille-toi, réveille-toi~! Lève-toi, Jérusalem, qui as bu de la main de Yahweh la coupe de sa fureur~; tu as bu, tu as sucé la lie de la coupe d'étourdissement\FTNT{Ps. 60:5~; Ap. 14:10.}~!
\VS{18}Il n'y a pas un de tous les enfants qu'elle a enfantés qui te conduise, et de tous les enfants qu'elle a nourris, il n'y en a pas un qui la prenne par la main.
\VS{19}Ces deux choses te sont arrivées~; qui te plaindra~? Le ravage et la ruine, la famine et l'épée~; par qui te consolerai-je~?
\VS{20}Tes enfants en défaillance gisaient aux carrefours de toutes les rues, comme un bœuf sauvage pris dans les filets, pleins de la fureur de Yahweh, de la répréhension de ton Dieu.
\VS{21}C'est pourquoi, écoute maintenant ceci, ô affligée, ivre, mais non pas de vin.
\VS{22}Ainsi parle Yahweh, ton Seigneur et ton Dieu, qui plaide la cause de son peuple~: Voici, je prends de la main la coupe d'étourdissement, la lie de la coupe de ma fureur, tu n'en boiras plus désormais~!
\VS{23}Car je la mettrai dans la main de ceux qui t'ont affligée, et qui disaient à ton âme~: Courbe-toi, et nous passerons~! C'est pourquoi tu as exposé ton corps comme la terre, comme une rue pour les passants.
\Chap{52}
\TextTitle{Le réveil de Jérusalem, la ville sainte}
\VerseOne{}Réveille-toi, réveille-toi, Sion~! Revêts-toi de ta force~! Jérusalem, ville sainte~! Revêts-toi de tes vêtements magnifiques~! Car l'incirconcis et le souillé ne passeront plus désormais chez toi.
\VS{2}Jérusalem, secoue ta poussière, lève-toi, et assieds-toi~! Détache les liens de ton cou, captive, fille de Sion~!
\VS{3}Car ainsi parle Yahweh~: Vous avez été vendus pour rien, et vous serez aussi rachetés sans argent.
\VS{4}Car ainsi parle le Seigneur Yahweh~: Mon peuple descendit jadis en Egypte pour y séjourner~; mais les Assyriens l'opprimèrent sans cause.
\VS{5}Et maintenant, qu'ai-je à faire ici, dit Yahweh, quand mon peuple a été enlevé pour rien~? Ceux qui dominent sur lui le font hurler, dit Yahweh, et mon Nom est blasphémé continuellement chaque jour.
\VS{6}C'est pourquoi mon peuple connaîtra mon Nom~; c'est pourquoi il saura, en ce jour-là, que JE SUIS parle~: Voici JE SUIS~!
\VS{7}Combien sont beaux sur les montagnes les pieds de celui qui apporte de bonnes nouvelles, qui publie la paix\FTNT{Na. 2:1~; Ro. 10:15.}, qui apporte de bonnes nouvelles concernant le bien, qui publie le salut, qui dit à Sion~: Ton Dieu règne~!
\VS{8}Tes sentinelles élèvent leurs voix, elles se réjouissent ensemble avec chants de triomphe~; car de leurs propres yeux elles voient comment Yahweh ramène Sion.
\VS{9}Déserts de Jérusalem, éclatez, réjouissez-vous ensemble avec chants de triomphe~! Car Yahweh console son peuple, il rachète Jérusalem.
\VS{10}Yahweh manifeste le bras de sa sainteté aux yeux de toutes les nations\FTNT{Es. 53:1.}, et toutes les extrémités de la terre verront le salut\FTNT{Toutes les extrémités de la terre verront le salut de Yahweh, c'est-à-dire Jésus (Mt. 28:18-20).} de notre Dieu.
\VS{11}Retirez-vous, retirez-vous, sortez de là~! Ne touchez rien d'impur~! Sortez du milieu d'elle\FTNT{Jé. 51:45~; 2 Co. 6:17~; Ap. 18:4.}~! Nettoyez-vous, vous qui portez les vases de Yahweh.
\VS{12}Car vous ne sortirez pas en hâte, et vous ne marcherez pas en fuyant, car Yahweh ira devant vous, et le Dieu d'Israël sera votre arrière-garde.
\TextTitle{Le serviteur de Yahweh}
\VS{13}Voici, mon serviteur prospérera, il sera fort exalté, élevé et glorifié.
\VS{14}Comme plusieurs ont été étonnés en te voyant, son visage était défiguré plus que celui d'aucun homme, et son apparence plus que celle d'aucun fils d'homme~;
\VS{15}ainsi, il aspergera plusieurs nations, et les rois fermeront la bouche sur lui~; car ceux auxquels on n'en avait point parlé le verront~; et ceux qui ne l'avaient point entendu l'entendront.
\Chap{53}
\TextTitle{Le sacrifice du Messie, serviteur de Yahweh}
\VerseOne{}Qui a cru à notre prédication~? Et à qui le bras de Yahweh\FTNT{Jésus-Christ homme est le bras de Yahweh. Le bras de Yahweh est le symbole de la puissance divine. Cette puissance s'est manifestée dans l'œuvre du Messie accomplissant le salut du monde. Le prophète est transporté au moment où le peuple juif, après avoir rejeté son Messie, ouvrira enfin les yeux et acceptera celui qu'il a percé (Za. 12:10~; Ap. 1:7). Voir aussi Jé. 27:4-5~; Jé. 32:17.} a-t-il été révélé~?
\VS{2}Toutefois il s'est élevé devant lui comme une jeune plante, comme un rejeton qui sort d'une terre desséchée~; il n'y avait en lui ni beauté, ni splendeur, quand nous le regardions, ni apparence qui nous le fasse désirer.
\VS{3}Il était le méprisé et le rejeté des hommes\FTNT{Ps. 22:6-7~; Mt. 27:27-31~; Mc. 9:12~; Jn. 16:32.}, homme de douleur, et sachant ce que c'est que la maladie~; et nous avons comme caché notre visage arrière de lui, tant il était méprisé~; et nous ne l'avons pas estimé.
\VS{4}En vérité, il a porté nos maladies, et il s'est chargé de nos douleurs\FTNT{Mt. 8:17~; 1 Pi. 2:24.}~; et nous l'avons considéré comme frappé, battu par Dieu et humilié.
\VS{5}Mais il était transpercé pour nos péchés, brisé pour nos iniquités, le châtiment qui nous apporte la paix est tombé sur lui, et c'est par ses meurtrissures que nous avons la guérison.
\VS{6}Nous avons tous été errants\FTNT{Pierre, apôtre de l'Agneau, confirme que le Messie est bel et bien le Bon Berger (1 Pi. 2:25).} comme des brebis, nous nous sommes détournés, chacun suivait son propre chemin, et Yahweh a fait venir sur lui l'iniquité de nous tous.
\VS{7}Opprimé et humilié, il n'a point ouvert sa bouche\FTNT{Mt. 26:62-63~; Mc. 15:3-5~; Jn. 19:9~; Ac. 8:32-33.}, semblable à un agneau qu'on mène à la boucherie, à une brebis muette devant celui qui la tond, et il n'a point ouvert sa bouche.
\VS{8}Il a été enlevé de la force de l'angoisse et de la condamnation~; mais qui racontera sa durée~? Car il a été retranché de la terre des vivants, et la plaie lui a été faite pour les péchés de mon peuple.
\VS{9}On a mis son sépulcre parmi les méchants, et dans sa mort, il a été avec le riche, quoiqu'il n'ait point commis de violence, et qu'il n'y ait point eu de fraude dans sa bouche\FTNT{Mc. 15:28~; Lu. 23:32-33.}.
\VS{10}Toutefois il a plu à Yahweh de le briser~; il l'a mis dans la souffrance. Après avoir mis son âme en sacrifice pour le péché, il verra une postérité et prolongera ses jours~; et le bon plaisir de Yahweh prospérera en sa main\FTNT{Jé. 23:5.}.
\VS{11}Il jouira du travail de son âme et en sera rassasié~; mon serviteur juste justifiera beaucoup d'hommes par la connaissance qu'ils auront de lui~; et lui-même portera leurs iniquités.
\VS{12}C'est pourquoi je lui donnerai sa part parmi les grands~; il partagera le butin avec les puissants, parce qu'il a livré son âme à la mort, qu'il a été mis au rang des transgresseurs, et que lui-même a porté les péchés de plusieurs, et qu'il a intercédé pour les transgresseurs.
\Chap{54}
\TextTitle{Yahweh réhabilite Israël la délaissée}
\VerseOne{}Réjouis-toi avec chants de triomphe, stérile, toi qui n'enfantes point, toi qui n'as pas connu les douleurs de l'accouchement~! Eclate de joie avec chant de triomphe et réjouis-toi~! Car les enfants de la délaissée seront plus nombreux que les enfants de celle qui est mariée, dit Yahweh.
\VS{2}Elargis l'espace de ta tente, et qu'on étende les couvertures de ton tabernacle~: Ne retiens rien~! Allonge tes cordages et affermis tes pieux~!
\VS{3}Car tu te répandras à droite et à gauche, et ta postérité possédera les nations et peuplera les villes désertes.
\VS{4}Ne crains pas, car tu ne seras point honteuse, ni confuse, et tu ne rougiras pas~; mais tu oublieras la honte de ta jeunesse, et tu ne te souviendras plus de l'opprobre de ton veuvage.
\VS{5}Car ton créateur est ton époux~: Yahweh des armées est son Nom~; et ton Rédempteur est le Saint d'Israël~: Il sera appelé le Dieu de toute la terre.
\VS{6}Car Yahweh t'appelle comme une femme délaissée et à l'esprit affligé, comme une femme qu'on a épousée dans la jeunesse, et qui a été répudiée, dit ton Dieu.
\VS{7}Je t'avais délaissée pour un petit moment, mais je te rassemblerai avec de grandes compassions.
\VS{8}Dans une courte colère, je t'avais un moment caché ma face, mais j'aurai compassion de toi avec une bonté éternelle, dit Yahweh, ton rédempteur.
\VS{9}Car il en sera pour moi comme les eaux de Noé~: De même que j'avais juré que les eaux de Noé ne se répandraient plus sur la terre\FTNT{Ge. 9:11~; Ge. 8:21.}~; je jure de ne plus m'irriter contre toi, et de ne plus te menacer.
\VS{10}Car quand les montagnes s'en iraient, quand les collines chancelleraient, ma bonté ne s'en ira point de toi, et mon alliance de paix ne chancellera point, dit Yahweh, qui a compassion de toi.
\VS{11}Ô affligée, agitée de la tempête, dénuée de consolation, voici, je coucherai tes pierres d'antimoine, et je te fonderai sur des saphirs~;
\VS{12}et je ferai tes fenêtrages d'agates, et tes portes de rubis, et toute ton enceinte de pierres précieuses.
\VS{13}Aussi tous tes enfants seront enseignés de Yahweh, et grande sera la paix de tes fils.
\VS{14}Tu seras affermie par la justice, tu seras loin de l'oppression, et tu ne craindras rien~; tu seras, dis-je, loin de la frayeur, car elle n'approchera pas de toi.
\VS{15}Voici, on ne manquera pas de comploter contre toi, cela ne viendra pas de moi~; quiconque complotera contre toi tombera pour l'amour de toi\FTNT{Ps. 91:7~; Ge. 37.}.
\VS{16}Voici, c'est moi qui ai créé le forgeron soufflant le charbon au feu, et formant un instrument pour son travail, et j'ai créé aussi le destructeur pour détruire.
\VS{17}Aucune arme forgée contre toi ne réussira, et toute langue qui se lèvera en jugement contre toi, tu la condamneras\FTNT{Ps. 23:4.}. Tel est l'héritage des serviteurs de Yahweh, et telle est la justice qui leur viendra de moi, dit Yahweh.
\Chap{55}
\TextTitle{Le salut par la grâce de Dieu}
\VerseOne{}Vous tous qui avez soif, venez aux eaux, et vous qui n'avez pas d'argent, venez, achetez et mangez~; venez, dis-je, achetez du vin et du lait sans argent, et sans rien payer~!
\VS{2}Pourquoi dépensez-vous de l'argent pour ce qui ne nourrit pas~? Pourquoi travaillez-vous pour ce qui ne rassasie pas\FTNT{Ro. 14:17.}~? Ecoutez-moi attentivement, et vous mangerez de ce qui est bon, et votre âme se délectera de la graisse.
\VS{3}Inclinez l'oreille, et venez à moi\FTNT{Mt. 11:28.}, écoutez, et votre âme vivra~; et je traiterai avec vous une alliance éternelle, les miséricordes immuables promises à David.
\VS{4}Voici, je l'ai donné comme témoin auprès des peuples, comme chef et dominateur des peuples.
\VS{5}Voici, tu appelleras des nations que tu ne connais pas, et les nations qui ne te connaissent pas accourront vers toi, à cause de Yahweh, ton Dieu, et du Saint d'Israël, qui t'aura glorifié.
\VS{6}Cherchez Yahweh pendant qu'il se trouve, invoquez-le tandis qu'il est près.
\VS{7}Que le méchant abandonne sa voie, et l'homme injuste ses pensées~; et qu'il retourne à Yahweh, qui aura pitié de lui, et à notre Dieu qui pardonne abondamment\FTNT{Jé. 18:11~; Ez. 33:11~; Jon. 3:10~; 1 Ti. 2:1-4~; 2 Pi. 3:9.}.
\VS{8}Car mes pensées ne sont pas vos pensées, et mes voies ne sont pas vos voies, dit Yahweh.
\VS{9}Mais autant les cieux sont élevés au-dessus de la terre, autant mes voies sont élevées au-dessus de vos voies, et mes pensées au-dessus de vos pensées.
\VS{10}Car comme la pluie et la neige descendent des cieux et n'y retournent plus, mais arrosent la terre, et la font produire et germer, afin de donner de la semence au semeur, et du pain à celui qui mange,
\VS{11}ainsi en est-il de ma parole qui sort de ma bouche, elle ne retourne point vers moi sans effet, mais elle fait tout ce en quoi je prends plaisir, et prospérera dans l'œuvre pour laquelle je l'ai envoyée.
\VS{12}Car vous sortirez avec joie, et vous serez conduits en paix~; les montagnes et les collines éclateront de joie avec chants de triomphe devant vous, et tous les arbres des champs battront des mains.
\VS{13}Au lieu de l'épine s'élèvera le cyprès, au lieu de la ronce croîtra le myrte~; et ceci fera connaître le Nom de Yahweh, et ce sera un signe perpétuel, qui ne sera jamais retranché.
\Chap{56}
\TextTitle{Exhortation à s'attacher à Yahweh}
\VerseOne{}Ainsi parle Yahweh~: Observez la justice, faites ce qui est juste, car mon salut ne tardera pas à venir, et ma justice à être révélée.
\VS{2}Béni est l'homme qui fait cela, et le fils de l'homme qui s'y tient, observant le sabbat pour ne pas le profaner, et gardant ses mains pour ne faire aucun mal.
\VS{3}Et que l'enfant de l'étranger qui se joint à Yahweh ne parle pas en disant~: Yahweh me séparera entièrement de son peuple~! Et que l'eunuque ne dise pas~: Voici, je suis un arbre sec.
\VS{4}Car ainsi parle Yahweh concernant les eunuques~: Ceux qui garderont mes sabbats, et qui choisiront ce en quoi je prends plaisir, et qui tiendront dans mon alliance,
\VS{5}je leur donnerai dans ma maison et dans mes murailles une place et un nom meilleur que le nom de fils ou de filles~; je leur donnerai à chacun un nom éternel qui ne périra jamais\FTNT{Ap. 2:17.}.
\VS{6}Et les enfants des étrangers qui se joindront à Yahweh pour le servir, pour aimer le Nom de Yahweh, pour être ses serviteurs, savoir tous ceux qui garderont le sabbat pour ne pas le profaner et qui tiendront dans mon alliance\FTNT{Ex. 31:14.},
\VS{7}je les amènerai sur ma montagne sainte, et je les réjouirai dans ma maison de prière~; leurs holocaustes et leurs sacrifices seront agréés sur mon autel, car ma maison sera appelée une maison de prière\FTNT{Mt. 21:13~; Mc. 11:17~; Lu. 19:46.} pour tous les peuples.
\VS{8}Le Seigneur Yahweh, parle, lui qui rassemble les exilés d'Israël~: Je réunirai d'autres peuples à lui, outre ceux déjà rassemblés.
\VS{9}Bêtes des champs, bêtes des forêts, venez toutes pour manger~!
\VS{10}Toutes ses sentinelles sont aveugles, elles ne connaissent rien~; ce sont tous des chiens muets, qui ne peuvent aboyer, dormant et demeurant couchés, et aimant à sommeiller.
\VS{11}Ce sont des chiens voraces et insatiables~; ce sont des pasteurs qui ne savent rien comprendre~; tous suivent leur propre voie, chacun à son gain injuste dans son quartier, en disant\FTNT{Mt. 23:24~; Tit. 1:7-11~; 1 Pi. 5:2.}~:
\VS{12}Venez, je vais chercher du vin, et nous nous enivrerons de boissons fortes~! Nous en ferons autant demain, et même beaucoup plus encore~!
\Chap{57}
\TextTitle{Yahweh expose la fausseté et défend le juste}
\VerseOne{}Le juste périt, et nul ne le prend à cœur~; et les gens de bien sont recueillis, sans qu'on y soit attentif, sans qu'on considère que le juste a été recueilli devant le mal\FTNT{Mi. 7:2~; Ec. 7:15.}.
\VS{2}Il entrera en paix, il reposera sur sa couche, celui qui aura marché dans la droiture\FTNT{Mt. 25:23~; Lu. 19:17.}.
\VS{3}Mais vous, approchez ici, enfants de l'enchanteresse, race de l'adultère et de la prostituée~!
\VS{4}De qui vous êtes-vous moqués~? Contre qui avez-vous ouvert la bouche et tirez-vous la langue~? N'êtes-vous pas des enfants de rébellion, une race de mensonge,
\VS{5}s'échauffant près des faux dieux, sous tout arbre vert~; égorgeant les enfants dans les vallées, sous les fentes des rochers\FTNT{Lé. 18:21~; 1 R. 14:23~; Jé. 2:20~; Jé. 32:35.}~?
\VS{6}Parmi les pierres polies des torrents est ta portion, ce sont elles, ce sont elles qui sont ton lot~; tu leur as aussi répandu ton aspersion, tu leur as aussi offert des offrandes~; puis-je être content de ces choses~?
\VS{7}Tu dresses ta couche sur les montagnes hautes et élevées~; c'est aussi là que tu montes pour offrir des sacrifices.
\VS{8}Et tu mets ton souvenir derrière la porte et les poteaux~; car tu te découvres loin de moi et tu montes, tu élargis ta couche, et tu te l'es taillée plus grande que n'ont fait ceux-là~; tu as aimé leur couche, tu as pris garde aux belles places.
\VS{9}Tu voyages vers le roi avec de l'huile précieuse, et tu ajoutes parfums sur parfums~; tu envoies au loin tes ambassades, tu t'abaisses jusqu'au scheol.
\VS{10}Tu te fatigues par la longueur du chemin, et tu ne dis pas~: C'est sans espoir~! Tu trouves encore de la vigueur dans ta main~; c'est pourquoi tu n'as pas été languissante.
\VS{11}Et qui redoutais-tu, qui craignais-tu pour que tu me mentes, pour ne pas te souvenir et te soucier de moi~? N'ai-je pas gardé le silence, et même depuis longtemps, et tu ne me crains pas.
\VS{12}Je vais déclarer ta justice et tes œuvres, qui ne te profiteront pas.
\VS{13}Quand tu crieras, que ceux que tu assembles te délivrent~! Mais le vent les emmènera tous, la vanité les enlèvera~; mais celui qui met sa confiance en moi, héritera la terre et possédera ma montagne sainte\FTNT{Es. 2:3~; Ps. 2:6~; Hé. 12:22.}.
\VS{14}On dira~: Frayez, frayez, préparez le chemin, enlevez tout obstacle loin du chemin de mon peuple~!
\TextTitle{Yahweh aime l'homme contrit}
\VS{15}Car ainsi parle celui qui est haut et élevé, qui habite dans l'éternité et dont le Nom est le Saint~: J'habiterai dans les lieux hauts et saints, avec celui qui a le cœur brisé et qui est humble d'esprit, afin de vivifier l'esprit des humbles, et afin de vivifier ceux qui ont le cœur brisé\FTNT{Ps. 34:19~; Ps. 51:19.}.
\VS{16}Parce que je ne veux pas contester à toujours, et que je ne serai pas irrité à jamais~; car devant moi tombent en défaillance les esprits, et les âmes que j'ai faites\FTNT{Mi. 7:18~; Ps. 85:6~; Ps. 103:9.}.
\VS{17}A cause de l'iniquité de ses gains déshonnêtes, je me suis irrité et je l'ai frappé, je me suis caché dans ma colère~; et le rebelle a suivi la voie de son cœur.
\VS{18}J'ai vu ses voies, et toutefois je le guérirai~; je le conduirai et je le restaurerai, lui et ceux qui mènent deuil avec lui.
\VS{19}Je crée les fruits des lèvres. Paix, paix à celui qui est loin et à celui qui est près~! dit Yahweh, car je le guérirai.
\VS{20}Mais les méchants sont comme la mer agitée, quand elle ne peut se calmer, et dont les eaux rejettent la boue et le bourbier.
\VS{21}Il n'y a point de paix pour les méchants, dit mon Dieu.
\Chap{58}
\TextTitle{Le vrai et le faux jeûne}
\VerseOne{}Crie à plein gosier, ne te retiens pas, élève ta voix comme un shofar, et annonce à mon peuple ses iniquités et à la maison de Jacob ses péchés~!
\VS{2}Car ils me cherchent tous les jours, ils prennent plaisir à connaître mes voies~; comme une nation qui aurait pratiqué la justice, et qui n'aurait pas abandonné les ordonnances de son Dieu~; ils me demandent des jugements justes, ils prennent plaisir à s'approcher de Dieu, et puis ils disent~:
\VS{3}Pourquoi jeûnons-nous, et tu ne le vois pas~? Pourquoi affligeons-nous nos âmes, si tu n'y as point connaissance~? Voici, le jour de votre jeûne, vous trouvez votre plaisir, et vous oppressez tous vos travailleurs.
\VS{4}Voici, vous jeûnez pour faire des querelles et vous disputer, et pour frapper du poing méchamment~; vous ne jeûnez pas comme le veut ce jour, pour que votre voix soit exaucée d'en haut.
\VS{5}Est-ce là le jeûne que j'ai choisi, que l'homme afflige son âme un jour~? Est-ce en courbant sa tête comme le jonc et en étendant le sac et la cendre~? Appelleras-tu cela un jeûne et un jour agréable à Yahweh~?
\VS{6}N'est-ce pas plutôt ici le jeûne que j'ai choisi~: Que tu détaches les liens de la méchanceté, que tu délies les cordages du joug, que tu laisses aller libres les opprimés, et que l'on rompe toute espèce de joug~?
\VS{7}N'est-ce pas que tu partages ton pain avec celui qui a faim~? Et que tu fasses venir dans ta maison les affligés errants~? Quand tu vois un homme nu, que tu le couvres, et que tu ne te caches pas de ta propre chair~?
\TextTitle{Bénédiction pour ceux qui pratiquent le bien}
\VS{8}Alors ta lumière éclatera comme l'aurore, et ta guérison germera rapidement~; ta justice ira devant toi, et la gloire de Yahweh sera ton arrière-garde.
\VS{9}Alors tu prieras, et Yahweh t'exaucera~; tu crieras, et il dira~: Me voici~! Si tu ôtes du milieu de toi le joug, si tu cesses de lever le doigt et de dire des outrages~;
\VS{10}si tu ouvres ton âme à celui qui a faim, si tu rassasies l'âme affligée~; ta lumière se lèvera sur les ténèbres, et l'obscurité sera comme le midi.
\VS{11}Et Yahweh te conduira continuellement, il rassasiera ton âme dans les grandes sécheresses, il fortifiera tes os, et tu seras comme un jardin arrosé, et comme une source dont les eaux ne tarissent pas\FTNT{Jn. 4:14~; Ap. 21:6.}.
\VS{12}Et ceux qui sortiront de toi rebâtiront les lieux déserts depuis longtemps, tu rétabliras les fondements ruinés depuis plusieurs générations~; et on t'appellera le réparateur des brèches et le restaurateur des chemins, afin qu'on habite au pays.
\VS{13}Si tu détournes ton pied pendant le sabbat pour ne pas faire ta volonté en mon saint jour~; si tu appelles le sabbat tes délices, et honorable ce qui est saint à Yahweh, et si tu l'honores en ne suivant point tes voies, en ne te livrant pas à tes désirs et à des vains discours,
\VS{14}alors tu prendras plaisir en Yahweh, et je te ferai monter comme à cheval par-dessus les lieux haut élevés de la terre, et je te donnerai à manger de l'héritage de Jacob, ton père~; car la bouche de Yahweh a parlé.
\Chap{59}
\TextTitle{Le péché sépare de Yahweh}
\VerseOne{}Voici, la main de Yahweh n'est pas trop courte pour pouvoir sauver, ni son oreille trop pesante pour pouvoir entendre.
\VS{2}Mais ce sont vos iniquités qui mettent une séparation entre vous et votre Dieu~; ce sont vos péchés qui vous cachent sa face, afin qu'il ne vous entende point\FTNT{De. 31:17-18~; Ez. 39:23-24.}.
\VS{3}Car vos mains sont souillées de sang, et vos doigts d'iniquité~; vos lèvres profèrent le mensonge, et votre langue déclare la perversité.
\VS{4}Nul ne crie pour la justice, nul ne plaide pour la vérité~; ils s'appuient sur des choses vaines et disent des faussetés, ils conçoivent le mal et enfantent l'iniquité.
\VS{5}Ils font éclore des œufs de vipère, et ils tissent des toiles d'araignée~; celui qui mange de leurs œufs meurt~; et si on les écrase, il en sort une vipère.
\VS{6}Leurs toiles ne servent point à faire des vêtements, et on ne se couvre pas de leurs ouvrages~; car leurs ouvrages sont des ouvrages d'iniquité, et il y a en leurs mains des actions de violence.
\VS{7}Leurs pieds courent au mal, et se hâtent pour répandre le sang innocent~; leurs pensées sont des pensées d'iniquité~; le ravage et la ruine sont sur leurs voies.
\VS{8}Ils ne connaissent point le chemin de la paix, et il n'y a point de jugement dans leurs voies, ils se sont pervertis dans leurs sentiers, tous ceux qui y marchent ignorent la paix\FTNT{Pr. 1:16~; Pr. 6:16-19.}.
\VS{9}C'est pourquoi le jugement s'est éloigné de nous, et la justice ne parvient pas jusqu'à nous~; nous attendions la lumière, et voici les ténèbres, la clarté, et nous marchons dans l'obscurité.
\VS{10}Nous tâtonnons comme des aveugles le long du mur, nous tâtonnons comme ceux qui sont sans yeux~; nous chancelons en plein midi comme la nuit, et nous sommes dans les lieux abondants comme y sont des morts.
\VS{11}Nous rugissons tous comme des ours, et nous ne cessons de gémir comme des colombes~; nous attendons le jugement, et il n'y en a point, la délivrance, et elle est éloignée de nous.
\VS{12}Car nos transgressions se sont multipliées devant toi, et chacun de nos péchés témoignent contre nous~; parce que nos transgressions sont avec nous, et nous connaissons nos iniquités
\VS{13}qui sont de pécher et de mentir contre Yahweh, de s'éloigner de notre Dieu, de proférer l'oppression et la révolte, de concevoir et prononcer du cœur des paroles de mensonge.
\VS{14}C'est pourquoi le jugement s'est éloigné et la justice se tient éloignée~; car la vérité est tombée par les rues, et la droiture ne peut y entrer.
\VS{15}Même la vérité a disparu, et quiconque se retire du mal est exposé au pillage~; Yahweh voit, et cela lui a déplu, parce qu'il n'y a plus de droiture.
\TextTitle{Yahweh cherche un homme, il suscite le Messie}
\VS{16}Il voit aussi qu'il n'y a aucun homme, il s'étonne que personne ne se tienne à la brèche~; c'est pourquoi son bras lui vient en aide, et sa propre justice lui sert d'appui\FTNT{Es. 53:1~; Es. 63:5~; Ps. 77:15-16~; Ac. 13:17.}.
\VS{17}Car il se revêt de la justice comme d'une cuirasse, et le casque du salut est sur sa tête\FTNT{Ep. 6:14-17.}~; il se revêt de la vengeance comme d'un vêtement, et se couvre de la jalousie comme d'un manteau.
\VS{18}Selon leurs actes, il rendra à chacun la pareille\FTNT{Jé. 17:10~; Job 34:11~; Mt. 16:27~; Ap. 2:23~; Ap. 20:13.}, la fureur à ses adversaires, la rétribution à ses ennemis~; il rendra ainsi la rétribution aux îles.
\VS{19}Et on craindra le Nom de Yahweh depuis l'occident, et sa gloire depuis le soleil levant~; car l'ennemi viendra comme un fleuve, mais l'Esprit de Yahweh lèvera la bannière\FTNT{En hébreu «~Yahweh Nissi~», c'est-à-dire «~Yahweh est ma bannière~». C'est le nom donné par Moïse à l'autel qu'il construisit pour célébrer la défaite d'Amalek (Ex. 17:15). En No. 21:8-9, Moïse éleva une bannière sur laquelle il avait fixé un serpent d'airain pour la guérison des malades.} contre lui.
\VS{20}Et le Rédempteur\FTNT{Le Rédempteur qui viendra pour Sion est le Seigneur Jésus-Christ (Ro. 11:26). Voir aussi Es. 60:16.} viendra en Sion, et vers ceux de Jacob qui se convertiront de leur péché, dit Yahweh.
\VS{21}Et quant à moi, c'est ici mon alliance que je ferai avec eux, dit Yahweh~: Mon Esprit qui est sur toi, et mes paroles que j'ai mises dans ta bouche, ne se retireront point de ta bouche, ni de la bouche de ta postérité, ni de la bouche de la postérité de ta postérité, dit Yahweh, dès maintenant et à jamais.
\Chap{60}
\TextTitle{La gloire de Yahweh se lèvera sur Sion}
\VerseOne{}Lève-toi, sois illuminée, car ta lumière arrive, et la gloire de Yahweh se lève sur toi.
\VS{2}Car voici, les ténèbres couvrent la terre, et l'obscurité couvre les peuples~; mais Yahweh se lève sur toi, et sa gloire apparaît sur toi.
\VS{3}Des nations marchent à ta lumière, et des rois à la splendeur qui se lève sur toi\FTNT{Ap. 21:24.}.
\VS{4}Elève tes yeux alentour, et regarde~: Tous ceux-ci s'assemblent, ils viennent vers toi~; tes fils viennent de loin, et tes filles sont nourries par des nourriciers, étant portées sur les côtés.
\VS{5}Alors tu verras et tu seras éclairée, et ton cœur s'étonnera et s'épanouira de joie, quand l'abondance de la mer se sera tournée vers toi, et que la puissance des nations sera venue chez toi.
\VS{6}Tu seras couverte d'une foule de chameaux, des dromadaires de Madian et d'Epha~; et tous ceux de Séba viendront, ils apporteront de l'or et de l'encens, et publieront les louanges de Yahweh.
\VS{7}Toutes les brebis de Kédar seront assemblées vers toi, les béliers de Nebajoth seront à ton service~; ils seront agréables étant offerts sur mon autel, et je rendrai magnifique la maison de ma gloire.
\VS{8}Qui sont ceux-là qui volent comme des nuées, comme des colombes vers leur colombier~?
\VS{9}Car les îles s'attendent à moi, et les navires de Tarsis les premiers, afin d'amener de loin tes enfants, avec leur argent et leur or, à cause du Nom de Yahweh, ton Dieu, et du Saint d'Israël qui te glorifie.
\VS{10}Les fils des étrangers rebâtiront tes murailles, et leurs rois seront employés à ton service~; car je t'ai frappée dans ma colère, mais j'ai eu pitié de toi au temps de mon bon plaisir.
\VS{11}Tes portes seront continuellement ouvertes, elles ne seront fermées ni nuit ni jour, afin que les forces des nations te soient amenées et que leurs rois y soient conduits\FTNT{Ap. 21:25-26.}.
\VS{12}Car la nation et le royaume qui ne te serviront pas périront, et ces nations-là seront réduites en une entière désolation.
\VS{13}La gloire du Liban viendra vers toi, le cyprès, l'orme, et le buis, tous ensemble pour rendre honorable le lieu de mon sanctuaire~; et je rendrai glorieux le lieu de mes pieds.
\VS{14}Mais les enfants de tes oppresseurs viendront vers toi en se courbant, et tous ceux qui te méprisaient se prosterneront à tes pieds et t'appelleront la ville de Yahweh, la Sion du Saint d'Israël.
\VS{15}Au lieu d'avoir été délaissée et haïe, si bien que personne ne passait par toi, je te mettrai dans une élévation éternelle et dans une joie qui sera de génération en génération.
\VS{16}Et tu suceras le lait des nations, et tu suceras la mamelle des rois, et tu sauras que je suis Yahweh, ton Sauveur, ton Rédempteur\FTNT{Le verbe «~ga'al~» et le nom correspondant «~go'el~», ont été traduits respectivement en français par «~racheter~» et «~rédempteur~». Selon la loi de Moïse, si quelqu'un perdait son héritage à cause d'une dette ou s'il se vendait comme esclave, lui et ses biens pouvaient être rachetés par un proche parent qui devait payer le prix de la rédemption (Lé. 25:23-55). Yahweh se présente comme le Rédempteur par excellence (Es. 49:26~; Es. 60:16~; Ps. 78:35~; Ps. 130:7~; Job 19:25).}, le Puissant de Jacob.
\VS{17}Je ferai venir de l'or au lieu de l'airain, et de l'argent au lieu du fer, et de l'airain au lieu du bois, et du fer au lieu des pierres~; et je ferai régner la paix et dominer la justice.
\VS{18}On n'entendra plus parler de violence dans ton pays ni de ravage et de ruine dans ton territoire~; mais tu appelleras tes murailles~: Salut~; et tes portes~: Louange.
\VS{19}Tu n'auras plus le soleil pour la lumière du jour, et la lueur de la lune ne t'éclairera plus, mais Yahweh sera pour toi la lumière éternelle\FTNT{Voir le commentaire en Ge. 1:3.}, et ton Dieu sera ta gloire.
\VS{20}Ton soleil ne se couchera plus, et ta lune ne se retirera plus, car Yahweh sera pour toi une lumière perpétuelle, et les jours de ton deuil seront finis.
\VS{21}Quant à ton peuple, ils seront tous justes, ils posséderont la terre à toujours~; savoir le germe de mes plantes, l'œuvre de mes mains pour y être glorifié\FTNT{Es. 11:1~; Ro. 15:12~; Ap. 5:5~; Ap. 22:16.}.
\VS{22}La petite famille deviendra un millier de personnes, et la moindre deviendra une nation puissante. Je suis Yahweh, je hâterai ces choses en leur temps.
\Chap{61}
\TextTitle{La mission du Messie}
\VerseOne{}L'Esprit du Seigneur Yahweh est sur moi, car Yahweh m'a oint pour évangéliser les malheureux~; il m'a envoyé pour guérir ceux qui ont le cœur brisé, pour proclamer aux captifs la liberté, et aux prisonniers l'ouverture de la prison~;
\VS{2}pour publier une année de grâce de Yahweh, et le jour de vengeance de notre Dieu~; pour consoler tous ceux qui mènent deuil\FTNT{Lu. 4:14-19.}~;
\VS{3}pour annoncer à ceux de Sion qui mènent deuil, que la magnificence leur sera donnée au lieu de la cendre, une huile de joie au lieu du deuil, un manteau de louange au lieu d'un esprit abattu\FTNT{Job. 29:14~; Ja. 1:12~; 1 Co.9:25~; 2 Ti. 4:8.}, afin qu'on les appelle des térébinthes de la justice, une plantation de Yahweh, pour servir à sa gloire.
\VS{4}Et ils rebâtiront les ruines antiques, ils relèveront les lieux qui étaient auparavant désolés, et ils renouvelleront des villes ravagées, et les choses désolées d'âge en âge.
\VS{5}Et des étrangers s'y tiendront là et feront paître vos troupeaux, et les enfants de l'étranger seront vos laboureurs et vos vignerons.
\VS{6}Mais vous, vous serez appelés prêtres de Yahweh, et on vous nommera serviteurs de notre Dieu\FTNT{Ap. 1:6~; Ap. 5:10.}~; vous mangerez les richesses des nations, et vous vous glorifierez de leur gloire.
\VS{7}Au lieu de la honte que vous avez eue, les nations en auront le double, et elles crieront tout haut que la confusion est leur portion~; c'est pourquoi ils posséderont le double dans leur pays, et leur joie sera éternelle.
\VS{8}Car JE SUIS Yahweh qui aime la justice, qui hait la rapine et l'iniquité~; j'établirai leur œuvre dans la vérité et je traiterai avec eux une alliance éternelle.
\VS{9}Et leur race sera connue parmi les nations, et ceux qui seront sortis d'eux seront connus parmi les peuples~; tous ceux qui les verront reconnaîtront qu'ils sont la race que Yahweh aura bénie.
\VS{10}Je me réjouirai, je me réjouirai en Yahweh, et mon âme sera joyeuse en mon Dieu~; car il m'a revêtu des vêtements du salut, il m'a couvert du manteau de la justice, comme un époux qui se pare de magnificence, et comme une épouse qui s'orne de ses joyaux\FTNT{Os. 2:21-22~; Ap. 19:7-8.}.
\VS{11}Car comme la terre fait éclore son germe, et comme un jardin fait germer ses semences, ainsi le Seigneur Yahweh fera germer la justice, et la louange en présence de toutes les nations.
\Chap{62}
\TextTitle{Yahweh proclamme la restauration d'Israël}
\VerseOne{}Pour l'amour de Sion, je ne me tiendrai pas tranquille, et pour l'amour de Jérusalem je ne prendrai point de repos, jusqu'à ce que sa justice sorte dehors comme une splendeur, et que sa délivrance ne soit allumée comme une lampe.
\VS{2}Alors les nations verront ta justice, et tous les rois ta gloire~; et on t'appellera d'un nouveau nom\FTNT{Ap. 2:17.}, que la bouche de Yahweh aura expressément déclaré.
\VS{3}Tu seras une couronne de gloire dans la main de Yahweh, un turban royal dans la main de ton Dieu.
\VS{4}On ne te nommera plus la délaissée, et on ne nommera plus ta terre la désolation~; mais on t'appellera mon bon plaisir en elle~; et on appellera ta terre l'épouse~; car Yahweh prend son bon plaisir en toi, et ta terre aura un époux.
\VS{5}Car comme le jeune homme épouse la vierge, comme tes enfants se marient chez toi, ainsi ton Dieu se réjouira en toi, de la joie qu'un époux a de son épouse.
\VS{6}Jérusalem, j'ai placé des gardes sur tes murailles tout le jour et toute la nuit, et ils ne se tairont point. Vous qui faites mention de Yahweh, ne gardez point le silence~!
\VS{7}Et ne vous arrêtez pas de l'invoquer jusqu'à ce qu'il rétablisse Jérusalem et lui rende sa renommée sur la terre.
\VS{8}Yahweh l'a juré par sa droite et par son bras puissant~: Je ne donnerai plus ton froment pour nourriture à tes ennemis, et les enfants des étrangers ne boiront plus ton vin excellent pour lequel tu as travaillé.
\VS{9}Mais ceux qui auront amassé le froment le mangeront et loueront Yahweh, et ceux qui auront récolté le vin le boiront dans les parvis de ma sainteté.
\VS{10}Passez, passez les portes~! Disant~: Préparez le chemin du peuple~! Frayez, frayez la route, et ôtez-en les pierres~! Elevez une bannière vers les peuples.
\VS{11}Voici ce que Yahweh proclame aux extrémités de la terre~: Dites à la fille de Sion~: Voici, ton Sauveur vient\FTNT{De nombreux passages, notamment dans le livre d'Esaïe, présentent Dieu comme le sauveur, le seul sauveur (Es. 43:3~; Es. 43:11~; Os. 13:4) qui viendra pour délivrer son peuple (Es. 35:4~; Es. 60:1~; Za. 14:1-7). Jésus-Christ a accompli en tous points les prophéties relatives à la venue de Yahweh. Dieu est bel et bien venu sur terre il y a plus de 2000 ans et ce même Dieu revient bientôt (Ac. 1:11~; Ap. 1:7).}~; voici, son salaire est avec lui, et sa récompense marche devant lui.
\VS{12}Et on les appellera le peuple saint, les rachetés de Yahweh\FTNT{1 Pi. 2:9~; Ap. 5:9.}~; et toi, on t'appellera la recherchée, la ville non abandonnée.
\Chap{63}
\TextTitle{Le jour de vengeance du Messie\FTNTT{Es. 2:10-22~; Ap. 19:11-21.}}
\VerseOne{}Qui est celui-ci qui vient d'Edom, de Botsra, en habits rouges, magnifiquement paré en son vêtement, marchant selon la grandeur de sa force~? C'est moi qui parle en justice et qui ai tout pouvoir de sauver.
\VS{2}Pourquoi tes vêtements sont-ils rouges, et pourquoi tes habits sont comme les habits de ceux qui foulent dans la cuve~?
\VS{3}J'ai été seul à fouler au pressoir, et nul homme d'entre les peuples n'était avec moi. Cependant, j'ai marché sur eux dans ma colère, et je les ai foulés dans ma fureur~; et leur sang a rejailli sur mes vêtements, et j'ai souillé tous mes habits.
\VS{4}Car le jour de la vengeance était dans mon cœur, et l'année de mes rachetés est venue.
\VS{5}Je regardais donc, il n'y avait personne pour m'aider~; et j'étais étonné, et il n'y avait personne pour me soutenir~; mais mon bras m'a sauvé et ma fureur m'a soutenu.
\VS{6}Ainsi j'ai foulé des peuples dans ma colère, et je les ai enivrés dans ma fureur~; et j'ai abattu leur force par terre.
\TextTitle{Esaïe confesse les péchés du peuple}
\VS{7}Je ferai mention des bontés de Yahweh, qui sont les louanges de Yahweh, pour tous les bienfaits que Yahweh nous a faits~; car grande est sa bonté envers la maison d'Israël, qu'il a traitée selon ses compassions et la richesse de sa miséricorde.
\VS{8}Car il a dit~: Certainement, ils sont mon peuple, des enfants qui ne tricheront pas~! Et il a été pour eux un Sauveur.
\VS{9}Et dans toutes leurs détresses, il a été en détresse, et l'ange qui est devant sa face les a délivrés\FTNT{Ge. 16:7-10~; Jg. 6:11-14~; Za. 1:11.}~; lui-même les a rachetés dans son amour et sa miséricorde, et il les a soutenus et portés, tous les jours d'autrefois.
\VS{10}Mais ils ont été rebelles, et ils ont attristé son Esprit saint\FTNT{Ep. 4:30.}, c'est pourquoi il est devenu leur ennemi, et il a lui-même combattu contre eux.
\VS{11}Et on se souvint des anciens jours de Moïse et de son peuple. Où est celui, a-t-on dit, qui les fit monter de la mer, avec les pasteurs de son troupeau~? Où est celui qui mit au milieu d'eux son Esprit saint~;
\VS{12}qui les dirigea par la droite de Moïse et par son bras glorieux~; qui fendit les eaux devant eux pour se faire un nom éternel~;
\VS{13}qui les dirigea à travers les flots, comme un cheval dans le désert, sans qu'ils ne bronchent~?
\VS{14}L'Esprit de Yahweh les a menés au repos comme on mène une bête qui descend dans la vallée. C'est ainsi que tu as conduit ton peuple, afin de t'acquérir un nom glorieux.
\VS{15}Regarde du ciel et vois de ta demeure sainte et glorieuse~: Où sont ton zèle et ta puissance~? Le son de tes entrailles et de tes compassions se retiennent-ils envers moi~?
\VS{16}Certes tu es notre Père, car Abraham ne nous connaît pas, et Israël ignore qui nous sommes~; Yahweh, c'est toi qui es notre Père, et ton Nom est notre Rédempteur de tout temps.
\VS{17}Pourquoi nous as-tu fait égarer loin de tes voies, ô Yahweh, et endurcis-tu notre cœur pour ne pas te crainte~? Reviens, pour l'amour de tes serviteurs, des tribus de ton héritage~!
\VS{18}Ton peuple saint n'a possédé le pays que peu de temps~; nos ennemis ont foulé ton sanctuaire.
\TextTitle{Prière du reste d'Israël à Yahweh pour sa délivrance}
\VS{19}Nous sommes comme ceux sur lesquels tu ne domines pas depuis longtemps, et sur lesquels ton Nom n'est point réclamé. Ô~! Si tu fendais les cieux, et si tu descendais, les montagnes s'ébranleraient devant toi~!
\Chap{64}
\VerseOne{}Comme le feu embrase les broussailles, comme le feu fait bouillir l'eau, tu ferais connaître ton Nom à tes ennemis en sorte que les nations tremblent en ta présence.
\VS{2}Lorsque tu fis les choses redoutables que nous n'attendions pas, tu descendis et les montagnes tremblèrent devant toi.
\VS{3}Jamais on n'a appris ni entendu dire, et jamais l'œil n'a vu qu'un autre dieu que toi fît de telles choses pour ceux qui s'attendent à lui\FTNT{1 Co. 2:9.}.
\VS{4}Tu viens à la rencontre de celui qui se réjouit et qui agit avec justice, et de ceux qui se souviennent de toi dans tes voies. Voici tu as été irrité parce que nous avons péché~; tes compassions sont éternelles, c'est pourquoi nous serons sauvés.
\VS{5}Or nous sommes tous devenus comme une chose souillée, et toute notre justice est comme le linge le plus souillé\FTNT{Ap. 19:8.}~; nous sommes tous flétris comme la feuille, et nos iniquités nous emportent comme le vent.
\VS{6}Il n'y a personne qui invoque ton Nom, qui se réveille pour s'attacher fortement à toi~; c'est pourquoi tu nous as caché ta face, et tu nous fais fondre par l'effet de nos iniquités.
\VS{7}Cependant, ô Yahweh, tu es notre Père~; nous sommes l'argile, et c'est toi qui nous as formés, et nous sommes tous l'ouvrage de ta main\FTNT{Es. 29:16~; Es. 45:9~; Jé. 18:6~; Ro. 9:20-21.}.
\VS{8}Ne t'irrite pas à l'extrême, ô Yahweh, et ne te souviens pas à toujours de notre iniquité. Voici, regarde, nous te prions, nous sommes tous ton peuple.
\VS{9}Tes villes saintes sont devenues un désert~; Sion est devenue un désert, et Jérusalem une désolation.
\VS{10}Notre maison sainte et glorieuse, où nos pères te louaient, a été brûlée par le feu~; tout ce que nous avions de précieux a été dévasté.
\VS{11}Après cela, ô Yahweh, ne te retiendras-tu pas~? Ne cesseras-tu pas, et nous affligeras-tu à l'excès~?
\Chap{65}
\TextTitle{Réponse de Yahweh}
\VerseOne{}Je me suis laissé rechercher par ceux qui ne me demandaient pas, et je me suis laissé trouver par ceux qui ne me cherchaient pas\FTNT{Mt. 7:7~; Lu. 11:9.}~; j'ai dit à la nation qui ne s'appelait pas de mon Nom~: Me voici, me voici~!
\VS{2}J'ai tendu mes mains tous les jours vers un peuple rebelle, à celui qui marche dans une mauvaise voie, au gré de ses pensées~;
\VS{3}vers un peuple qui m'irrite continuellement en face, qui sacrifie dans les jardins, et qui fait des parfums sur les autels de briques,
\VS{4}qui habite les sépulcres et passe la nuit dans les lieux désolés, qui mange la chair de porc, et ayant dans ses vases le jus des choses abominables~;
\VS{5}qui dit~: Retire-toi, ne m'approche pas, car je suis plus saint que toi~! Ceux-là sont une fumée dans mes narines, un feu ardent tout le jour.
\VS{6}Voici, ceci est écrit devant moi, je ne me tairai point, mais je leur ferai porter la peine, oui je leur ferai porter la peine
\VS{7}de vos iniquités, dit Yahweh, et les iniquités de vos pères ensemble, qui ont brûlé de l'encens sur les montagnes, et qui m'ont blasphémé sur les collines~; c'est pourquoi je leur mesurerai aussi dans leur sein le salaire de ce qu'ils ont fait au commencement.
\VS{8}Ainsi parle Yahweh~: Comme quand on trouve du vin dans une grappe, on dit~: Ne la détruis pas, car il y a là une bénédiction~! J'agirai de même à cause de mes serviteurs, afin de ne pas tous les détruire.
\VS{9}Je ferai sortir de Jacob une postérité, et de Juda celui qui héritera de mes montagnes~; et mes élus hériteront le pays, et mes serviteurs y habiteront.
\VS{10}Et Saron servira de pâturage au menu bétail, et la vallée d'Acor sera le gîte du gros bétail, pour mon peuple qui m'aura recherché.
\VS{11}Mais vous, qui abandonnez Yahweh et qui oubliez ma montagne sainte, qui dressez la table pour Gad\FTNT{Gad~: Dieu de la fortune.}, et qui remplissez une coupe pour Meni\FTNT{Meni~: divinité païenne assimilée à la lune et dont le nom signifie «~destin, sort ou fortune~».},
\VS{12}je vous destine aussi à l'épée, et vous serez tous courbés pour être égorgés~; parce que j'ai appelé, et vous n'avez point répondu~; j'ai parlé, et vous n'avez point écouté~; mais vous avez fait ce qui me déplaît, et vous avez choisi les choses auxquelles je ne prends pas plaisir.
\VS{13}C'est pourquoi, ainsi parle le Seigneur Yahweh~: Voici, mes serviteurs mangeront, et vous aurez faim~; voici, mes serviteurs boiront, et vous aurez soif~; voici mes serviteurs se réjouiront, et vous serez honteux.
\VS{14}Voici, mes serviteurs se réjouiront avec chants de triomphe pour la joie qu'ils auront au cœur~; mais vous, vous crierez pour la douleur que vous aurez au cœur, et vous crierez à cause de l'accablement de votre esprit.
\VS{15}Et vous laisserez votre nom à mes élus comme malédiction~; et le Seigneur Yahweh vous fera mourir~; et il donnera à ses serviteurs un autre nom.
\VS{16}Celui qui se bénira sur la terre, se bénira par le Dieu de vérité~; et celui qui jurera sur la terre jurera par le Dieu de vérité~; car les détresses du passé seront oubliées, et même elles seront cachées devant mes yeux.
\TextTitle{De nouveaux cieux et une nouvelle terre}
\VS{17}Car voici, je vais créer de nouveaux cieux et une nouvelle terre\FTNT{Es. 66:22~; 2 Pi. 3:13~; Ap. 21:1.}~; et on ne se souviendra plus des choses précédentes, elles ne reviendront plus au cœur.
\VS{18}Réjouissez-vous plutôt et soyez à toujours dans l'allégresse, à cause de ce que je vais créer~; car voici je vais créer Jérusalem pour n'être que joie, et son peuple pour n'être qu'allégresse.
\VS{19}Je ferai de Jérusalem mon allégresse, et de mon peuple ma joie~; on n'y entendra plus le bruit des pleurs et le bruit des clameurs.
\VS{20}Il n'y aura plus désormais ni nourrisson, ni vieillard qui n'accomplisse leurs jours~; car celui qui mourra âgé de cent ans sera encore jeune~; mais le pécheur âgé de cent ans sera maudit.
\VS{21}Ils bâtiront des maisons et y habiteront~; ils planteront des vignes et ils en mangeront le fruit.
\VS{22}Ils ne bâtiront pas des maisons pour qu'un autre y habite~; ils ne planteront pas des vignes pour qu'un autre en mange le fruit~; car les jours de mon peuple seront comme les jours des arbres~; et mes élus jouiront de l'œuvre de leurs mains.
\VS{23}Ils ne travailleront plus en vain, et ils n'engendreront plus des enfants pour être exposés à la frayeur~; car ils seront la postérité des bénis de Yahweh, et ceux qui sortiront d'eux seront avec eux.
\VS{24}Et il arrivera qu'avant qu'ils crient, je les exaucerai~; et lorsqu'encore ils parleront, je les aurai déjà entendus.
\VS{25}Le loup et l'agneau paîtront ensemble, le lion comme le bœuf mangeront de la paille, et la poussière sera la nourriture du serpent\FTNT{Es. 2:4~; Es. 11:6-7.}. On ne nuira point et on ne fera aucun dommage sur toute ma montagne sainte, dit Yahweh.
\Chap{66}
\TextTitle{Yahweh réprouve l'hypocrisie et agrée ceux qui le craignent}
\VerseOne{}Ainsi parle Yahweh~: Le ciel est mon trône, et la terre est le marchepied de mes pieds\FTNT{Mt. 5:34-35~; Ac. 7:49.}. Quelle maison me bâtiriez-vous, et quel serait le lieu de mon repos~?
\VS{2}Car ma main a fait toutes ces choses, et c'est par moi que toutes ces choses ont eu leur existence, dit Yahweh. Mais à qui regarderai-je~? A celui qui est affligé, qui a l'esprit abattu, et qui tremble à ma parole.
\VS{3}Celui qui égorge un bœuf est comme celui qui tuerait un homme~; celui qui sacrifie une brebis est comme celui qui romprait la nuque à un chien~; celui qui présente une offrande est comme celui qui offrirait le sang d'un pourceau~; celui qui fait un parfum d'encens est comme celui qui bénirait une idole~; tous ceux-là ont choisi leurs voies, et leur âme trouve du plaisir dans leurs abominations.
\VS{4}Moi aussi je ferai attention à leurs tromperies, et je ferai venir sur eux les choses qu'ils craignent~; parce que j'ai appelé, et personne n'a répondu, parce que j'ai parlé, et qu'ils n'ont point écouté~; mais ils ont fait ce qui est mal à mes yeux, et ils ont choisi les choses auxquelles je ne prends pas de plaisir.
\VS{5}Ecoutez la parole de Yahweh, vous qui tremblez à sa parole~; vos frères, qui vous haïssent et qui vous repoussent comme une chose abominable, à cause de mon Nom disent~: Que Yahweh montre sa gloire~! Il sera donc vu à votre joie mais eux seront honteux.
\VS{6}Un son éclatant sort de la ville, un son sort du temple, le son de Yahweh, qui rend à ses ennemis selon leurs œuvres.
\TextTitle{Israël renaît en un jour}
\VS{7}Elle a enfanté, avant d'éprouver les douleurs de l'enfantement~; elle a donné naissance à un enfant mâle, avant que les souffrances lui viennent.
\VS{8}Qui a jamais entendu une telle chose~? Qui en a jamais vu de semblable~? Ferait-on qu'un pays naisse en un jour~? Ou une nation naîtrait-elle d'un seul coup\FTNT{Cette prophétie fait allusion à la création de l'Etat d'Israël le 14 mai 1948.}~? Car dès que Sion a été en travail, elle a enfanté ses enfants~!
\VS{9}Moi qui fais enfanter les autres, ne ferais-je point enfanter Sion~? dit Yahweh. Moi qui donne de la postérité aux autres, l'empêcherais-je d'enfanter~? dit ton Dieu.
\TextTitle{Réjouissance à Jérusalem et consolation}
\VS{10}Réjouissez-vous avec Jérusalem, faites d'elle le sujet de votre allégresse, vous tous qui l'aimez~; vous tous qui menez deuil sur elle, réjouissez-vous avec elle d'une grande joie,
\VS{11}afin que vous soyez allaités et rassasiés de la mamelle de ses consolations, afin que vous suciez le lait et que vous jouissiez à plaisir de la plénitude de sa gloire.
\VS{12}Car ainsi parle Yahweh~: Voici, je ferai couler vers elle la paix comme un fleuve, et la gloire des nations comme un torrent débordé, et vous serez allaités, vous serez portés sur les côtés et caressés sur les genoux.
\VS{13}Je vous consolerai pour vous apaiser, comme quelqu'un que sa mère caresse pour l'apaiser, vous serez consolés dans Jérusalem.
\VS{14}Vous le verrez et votre cœur se réjouira, et vos os germeront comme l'herbe~; et la main de Yahweh sera connue de ses serviteurs~; mais il sera indigné contre ses ennemis.
\TextTitle{Jugement de Yahweh}
\VS{15}Car voici, Yahweh viendra avec le feu, et ses chars seront comme la tempête~; afin qu'il tourne sa colère en fureur, et sa menace en flamme de feu.
\VS{16}Car Yahweh exercera ses jugements contre toute chair par le feu et avec son épée~; et le nombre de ceux qui seront mis à mort par Yahweh sera grand.
\VS{17}Ceux qui se sanctifient et se purifient au milieu des jardins, l'un après l'autre, qui mangent de la chair de porc et des choses abominables, comme des souris, seront ensemble consumés, dit Yahweh.
\VS{18}Mais pour moi, voyant leurs œuvres et leurs pensées, le temps est venu de rassembler toutes les nations et les langues~; ils viendront et verront ma gloire.
\TextTitle{Toutes les nations adoreront Yahweh}
\VS{19}Car je mettrai un signe en eux, et j'enverrai ceux d'entre eux qui seront réchappés, vers les nations, à Tarsis, à Pul, à Lud, gens tirant de l'arc, à Tubal et à Javan, et vers les îles lointaines, qui n'ont point entendu ma renommée, et qui n'ont pas vu ma gloire~; et ils annonceront ma gloire parmi les nations.
\VS{20}Et ils amèneront tous vos frères d'entre toutes les nations, sur des chevaux, sur des chars et dans des litières, sur des mulets et sur des dromadaires, en offrande à Yahweh, à la montagne sainte, à Jérusalem, dit Yahweh, comme lorsque les enfants d'Israël apportent l'offrande dans un vase pur, à la maison de Yahweh.
\VS{21}Et même je prendrai aussi parmi eux des prêtres, des Lévites, dit Yahweh.
\VS{22}Car comme les nouveaux cieux et la nouvelle terre que je vais faire subsisteront devant moi, dit Yahweh, ainsi subsistera votre postérité et votre nom.
\VS{23}Et il arrivera que de nouvelle lune en nouvelle lune, et de sabbat en sabbat, toute chair viendra se prosterner devant ma face, dit Yahweh.
\VS{24}Et quand ils sortiront dehors, ils verront les cadavres des hommes qui se sont rebellés contre moi~; car leur ver ne mourra point, et leur feu ne s'éteindra point\FTNT{Mc. 9:48.}~; et ils seront méprisés de tout le monde.
\PPE{}
\end{multicols}
