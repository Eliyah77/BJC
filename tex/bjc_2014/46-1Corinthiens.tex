\ShortTitle{1 Corinthiens}\BookTitle{1 Corinthiens}\BFont
\noindent\hrulefill
{\footnotesize
\textit{
\bigskip
{\centering{}
\\Auteur : Paul
\\Thème : Le comportement du chrétien
\\Date de rédaction : Env. 56 ap. J.-C.\\}
}
%\bigskip
\textit{
\\Dans l'antiquité, Corinthe, capitale de l'Achaïe, était la ville la plus prospère et la plus puissante de Grèce. Située sur
un isthme séparant la mer Egée de la mer Ionienne, Corinthe était au carrefour de l'Asie et de l'Italie et constituait un véritable centre commercial où les produits orientaux et occidentaux se croisaient.
%\bigskip
\\L'apôtre Paul arriva à Corinthe en 51, sous le règne de l'empereur romain Claude (10 av. J.-C. – 54 apr. J.-C.), et y demeura 18 mois. Il trouva une ville riche en pleine expansion, une population parlant diverses langues et rendant des cultes à une multitude de divinités. Rédigée au terme des trois ans passés à Ephèse, la première épître de Paul aux Corinthiens répond à une lettre dans laquelle ceux-ci s'interrogeaient sur le mariage et sur les aliments consacrés aux idoles. Ce fut aussi l'occasion pour lui de procéder à la correction de cette jeune église dont l'état charnel constituait un frein à l'avancée spirituelle. Les Corinthiens avaient en effet confondu le culte raisonnable et les pratiques liées aux cultes à mystères.\bigskip
}
}
\par\nobreak\noindent\hrulefill
\begin{multicols}{2}
\Chap{1}
\TextTitle{La grâce de Christ manifeste dans la vie des saints\FTNTT{Ro. 5:1-2 ; Ep. 1:3-14}}
\VerseOne{}Paul, appelé à être apôtre de Jésus-Christ, par la volonté de Dieu, et le frère Sosthène,
\VS{2}à l'église de Dieu qui est à Corinthe, aux sanctifiés en Jésus-Christ, appelés à être saints, et à tous ceux qui en quelque lieu que ce soit invoquent le Nom de notre Seigneur Jésus-Christ, leur Seigneur et le nôtre.
\VS{3}Que la grâce et la paix vous soient données de la part de Dieu notre Père et du Seigneur Jésus-Christ.
\VS{4}Je rends toujours grâces à mon Dieu à votre sujet, pour la grâce de Dieu qui vous a été donnée en Jésus-Christ.
\VS{5}Car en lui vous avez été enrichis de toutes les richesses qui concernent la parole et la connaissance,
\VS{6}selon que le témoignage de Jésus-Christ a été confirmé en vous,
\VS{7}de sorte qu'il ne vous manque aucun don, pendant que vous attendez la manifestation de notre Seigneur Jésus-Christ.
\VS{8}Qui vous affermira aussi jusqu'à la fin pour que vous soyez irrépréhensibles au jour de notre Seigneur Jésus-Christ.
\VS{9}Dieu qui vous a appelés à la communion de son Fils Jésus-Christ notre Seigneur est fidèle.
\TextTitle{Les rivalités, causes de divisions}
\VS{10}Je vous prie, mes frères, par le Nom de notre Seigneur Jésus-Christ, à tenir tous un même langage, et à ne point avoir de divisions parmi vous, mais à être parfaitement unis dans une même pensée et dans un même jugement.
\VS{11}Car mes frères, j'ai été informé par ceux de la maison de Chloé qu'il y a des dissensions parmi vous.
\VS{12}Je veux dire que chacun de vous parle ainsi : Moi je suis de Paul ! Et moi d'Apollos ! Et moi de Céphas ! Et moi de Christ !
\VS{13}Christ est-il divisé ? Paul a-t-il été crucifié pour vous ? Ou avez-vous été baptisés au nom de Paul ?
\VS{14}Je rends grâces à Dieu de ce que je n'ai baptisé aucun de vous, sinon Crispus et Gaïus,
\VS{15}afin que personne ne dise que j'ai baptisé en mon nom.
\VS{16}J'ai bien aussi baptisé la famille de Stéphanas ; du reste, je ne sais pas si j'ai baptisé quelque autre.
\VS{17}Car Christ ne m'a pas envoyé pour baptiser, mais pour évangéliser, non pas avec des discours de la sagesse humaine, afin que la croix de Christ ne soit pas anéantie.
\TextTitle{La sagesse de Dieu manifestée à la croix dépasse l'entendement humain}
\VS{18}Car la prédication de la croix est une folie pour ceux qui périssent, mais pour nous qui sommes sauvés, elle est la puissance de Dieu.
\VS{19}Car il est écrit : Je détruirai la sagesse des sages et j'anéantirai l'intelligence des hommes intelligents\FTNT{Es. 29:14.}.
\VS{20}Où est le sage ? Où est le scribe ? Où est le disputeur de ce siècle ? Dieu n'a-t-il pas convaincu de folie la sagesse de ce monde ?
\VS{21}Puisque le monde, avec sa sagesse, n'a pas connu Dieu, dans la sagesse de Dieu, il a plu à Dieu de sauver les croyants par la folie de la prédication.
\VS{22}Les Juifs demandent des miracles et les Grecs cherchent la sagesse,
\VS{23}mais pour nous, nous prêchons Christ crucifié, scandale pour les Juifs, et folie pour les Grecs,
\VS{24}à ceux qui sont appelés, tant Juifs que Grecs, nous leur prêchons Christ, la puissance de Dieu et la sagesse de Dieu.
\VS{25}Parce que la folie de Dieu est plus sage que les hommes, et la faiblesse de Dieu est plus forte que les hommes.
\TextTitle{Dieu se sert des choses viles pour confondre le monde et sa sagesse}
\VS{26}Considérez, mes frères, que parmi vous qui avez été appelés, il n'y a pas beaucoup de sages selon la chair, ni beaucoup de puissants, ni beaucoup de nobles.
\VS{27}Mais Dieu a choisi les choses folles de ce monde pour confondre les sages ; et Dieu a choisi les choses faibles de ce monde pour confondre les fortes ;
\VS{28}et Dieu a choisi les choses viles de ce monde et les méprisées, même celles qui ne sont point, pour réduire à néant celles qui sont,
\VS{29}afin que nulle chair ne se glorifie devant lui.
\VS{30}Or c'est par lui que vous êtes en Jésus-Christ, lequel, de par Dieu, a été fait pour nous sagesse, justice, sanctification et rédemption ;
\VS{31}afin que comme il est écrit, celui qui se glorifie se glorifie dans le Seigneur\FTNT{Jé. 9:24.}.
\Chap{2}
\TextTitle{La foi en Dieu ne se base pas sur la sagesse humaine}
\VerseOne{}Pour moi donc, mes frères, lorsque je suis allé chez vous, ce n'est pas avec des discours pompeux, remplis de la sagesse humaine, que je suis allé vous annoncer le témoignage de Dieu.
\VS{2}Car je n'ai pas eu la pensée de savoir parmi vous autre chose que Jésus-Christ et Jésus-Christ crucifié.
\VS{3}Et j'ai même été parmi vous dans la faiblesse, dans la crainte, et dans un grand tremblement.
\VS{4}Et ma parole et ma prédication ne reposaient pas sur les discours persuasifs de la sagesse humaine, mais sur une démonstration d'Esprit et de puissance ;
\VS{5}afin que votre foi ne soit pas fondée sur la sagesse des hommes, mais sur la puissance de Dieu.
\VS{6}Cependant, nous prêchons une sagesse parmi les parfaits, une sagesse, dis-je, qui n'est pas de ce monde, ni des chefs de ce siècle, qui vont être anéantis.
\VS{7}Mais nous prêchons la sagesse de Dieu, qui est un mystère, c'est-à-dire cachée, que Dieu avant les siècles, avait prédestinée pour notre gloire,
\VS{8}sagesse qu'aucun des chefs de ce siècle n'a connue, car s'ils l'avaient connue, ils n'auraient pas crucifié le Seigneur de gloire.
\TextTitle{C'est l'Esprit de Dieu qui revèle les profondeurs de Dieu}
\VS{9}Mais comme il est écrit : Ce sont des choses que l'œil n'a point vues, que l'oreille n'a point entendues, et qui ne sont point montées au cœur de l'homme, des choses que Dieu a préparées pour ceux qui l'aiment\FTNT{Es. 64:4.}.
\VS{10}Mais Dieu nous les a révélées par son Esprit. Car l'Esprit sonde toutes choses, même les choses profondes de Dieu.
\VS{11}Qui donc, parmi les hommes, connaît les choses de l'homme, sinon l'esprit de l'homme qui est en lui ? De même aussi, personne ne connaît les choses de Dieu, si ce n'est l'Esprit de Dieu.
\VS{12}Or nous, nous n'avons pas reçu l'esprit de ce monde, mais l'Esprit qui vient de Dieu, afin que nous connaissions les choses qui nous ont été données de Dieu.
\TextTitle{L'homme animal et l'homme spirituel}
\VS{13}Et nous en parlons, non avec des discours que la sagesse humaine enseigne, mais avec celle qu'enseigne le Saint-Esprit, communiquant des choses spirituelles à ceux qui sont spirituels.
\VS{14}Mais l'homme animal\FTNT{L'homme animal (ou naturel) est un homme incrédule. C'est un homme non-régénéré, ayant le principe de la vie animale, c'est-à-dire ce que les hommes ont en commun avec les brutes. Sa nature sensuelle est sujette aux appétits et aux passions (Jud. 1:19).} ne comprend pas les choses de l'Esprit de Dieu, car elles sont une folie pour lui ; et il ne peut même pas les entendre, parce c'est spirituellement qu'on en juge.
\VS{15}Mais l'homme spirituel\FTNT{L'homme spirituel est un homme dont l'esprit est régénéré et qui marche par l'Esprit. Il a la pensée de Christ et porte les fruits de l'Esprit.} juge de tout et il n'est jugé par personne.
\VS{16}Car qui a connu la pensée du Seigneur pour pouvoir l'instruire\FTNT{Es. 40:13.} ? Mais nous, nous avons la pensée de Christ.
\Chap{3}
\TextTitle{Les œuvres de la chair nuisent à la croissance en Christ}
\VerseOne{}Pour moi, mes frères, je n'ai pas pu vous parler comme à des hommes spirituels, mais comme à des hommes charnels\FTNT{L'homme charnel est gouverné par la nature humaine et non par l'Esprit de Dieu (Ga. 5:16-21). L'homme charnel est un enfant en Christ, littéralement « ignorant » (Ga. 4:1). Il est comparé à un esclave.}, c'est-à-dire comme à des enfants en Christ.
\VS{2}Je vous ai donné du lait à boire, et non pas de la viande, parce que vous ne pouviez pas la supporter ; et même maintenant vous ne le pouvez pas encore, parce que vous êtes encore charnels.
\VS{3}Car puisqu'il y a parmi vous de la jalousie, des disputes, et des divisions, n'êtes-vous pas charnels, et ne vous conduisez-vous pas à la manière des hommes ?
\VS{4}Car quand l'un dit : Moi je suis de Paul ; et l'autre : Moi je suis d'Apollos, n'êtes-vous pas charnels ?
\TextTitle{Condamnation du sectarisme ; Dieu est le Maître de tout}
\VS{5}Qu'est-ce donc Paul, et qui est Apollos ? Des ministres, par le moyen desquels vous avez cru, selon que le Seigneur l'a donné à chacun.
\VS{6}J'ai planté, Apollos a arrosé, mais c'est Dieu qui a donné l'accroissement,
\VS{7}en sorte que ce n'est pas celui qui plante qui est quelque chose, ni celui qui arrose, mais Dieu qui donne l'accroissement.
\VS{8}Celui qui plante et celui qui arrose sont égaux, et chacun recevra sa récompense selon son propre travail.
\VS{9}Car nous sommes ouvriers avec Dieu. Vous êtes le champ de Dieu et l'édifice de Dieu.
\VS{10}Selon la grâce de Dieu qui m'a été donnée, j'ai posé le fondement comme un sage architecte, et un autre édifie dessus. Mais que chacun prenne garde comment il édifie dessus.
\TextTitle{Jésus- Christ : Unique fondement}
\VS{11}Car personne ne peut poser un autre fondement que celui qui a été posé, à savoir Jésus-Christ.
\TextTitle{Deux types de construction}
\VS{12}Si quelqu'un édifie sur ce fondement avec de l'or, de l'argent, des pierres précieuses, du bois, du foin, du chaume, l'œuvre de chacun sera manifestée ;
\VS{13}car le jour la fera connaître, parce qu'elle sera manifestée par le feu ; et le feu éprouvera ce qu'est l'œuvre de chacun.
\VS{14}Si l'œuvre édifiée par quelqu'un sur le fondement subsiste, il recevra la récompense.
\VS{15}Si l'œuvre de quelqu'un est consumée, il perdra sa récompense ; mais pour lui, il sera sauvé, toutefois comme au travers du feu.
\VS{16}Ne savez-vous pas que vous êtes le temple\FTNT{Le temple de Dieu. Beaucoup veulent construire des bâtiments qu'ils appellent « temples ou maisons de Dieu » alors que chaque chrétien est le temple de Dieu. Voir Es. 66:1 ; Ac. 17:24 ; 1 Co. 6:19.} de Dieu et que l'Esprit de Dieu habite en vous ?
\VS{17}Si quelqu'un détruit le temple de Dieu, Dieu le détruira ; car le temple de Dieu est saint, et vous êtes ce temple.
\VS{18}Que personne ne s'abuse lui-même : Si quelqu'un d'entre vous croit être sage selon ce monde, qu'il devienne fou, afin de devenir sage.
\VS{19}Parce que la sagesse de ce monde est une folie devant Dieu ; car il est écrit : Il surprend les sages dans leur ruse\FTNT{Job 5:13.}.
\VS{20}Et encore : Le Seigneur connaît que les pensées des sages sont vaines\FTNT{Ps. 94:11.}.
\VS{21}Que personne donc ne mette sa gloire dans les hommes, car toutes choses sont à vous,
\VS{22}soit Paul, soit Apollos, soit Céphas, soit le monde, soit la vie, soit la mort, soit les choses présentes, soit les choses à venir, toutes choses sont à vous,
\VS{23}et vous à Christ, et Christ à Dieu.
\Chap{4}
\TextTitle{Le Seigneur est le seul véritable Juge}
\VerseOne{}Que chacun nous regarde comme des serviteurs de Christ et des dispensateurs des mystères de Dieu.
\VS{2}Du reste, il est exigé des dispensateurs que chacun soit trouvé fidèle.
\VS{3}Pour moi, il m'importe fort peu d'être jugé par vous, ou par un jugement d'homme. Je ne me juge pas non plus moi-même, car je ne me sens coupable de rien,
\VS{4}mais ce n'est pas pour cela que je suis justifié. Celui qui me juge, c'est le Seigneur.
\VS{5}C'est pourquoi ne jugez de rien avant le temps, jusqu'à ce que le Seigneur vienne, alors il mettra en lumière les choses cachées dans les ténèbres et manifestera les desseins des cœurs. Alors chacun recevra de Dieu la louange qui lui sera due.
\VS{6}Or mes frères, j'ai fait de ces choses une application à ma personne et à celle d'Apollos, à cause de vous ; afin que vous appreniez de nous à ne point aller au-delà de ce qui est écrit, et que nul de vous ne conçoive de l'orgueil en faveur de l'un contre l'autre.
\VS{7}Car qui est-ce qui met de la différence entre toi et un autre ? Qu'as-tu que tu n'aies reçu ? Et si tu l'as reçu, pourquoi te glorifies-tu comme si tu ne l'avais pas reçu\FTNT{Les diverses grâces que Dieu accorde à ses enfants doivent les amener à l'humilité.} ?
\VS{8}Vous êtes déjà rassasiés, vous êtes déjà enrichis, vous êtes devenus rois sans nous. Plaise à Dieu que vous régniez en effet, afin que nous aussi nous régnions avec vous !
\TextTitle{L'humilité et la patience}
\VS{9}Car je pense que Dieu nous a exposés publiquement, nous qui sommes les derniers des apôtres, comme des gens condamnés à la mort, puisque nous avons été en spectacle au monde, aux anges et aux hommes.
\VS{10}Nous sommes fous pour l'amour de Christ, mais vous êtes sages en Christ ; nous sommes faibles, et vous êtes forts ; vous êtes dans l'estime, et nous sommes dans le mépris.
\VS{11}Jusqu'à cette heure, nous souffrons la faim, la soif, la nudité ; on nous frappe au visage, et nous sommes errants çà et là ;
\VS{12}nous nous fatiguons à travailler de nos propres mains ; on dit du mal de nous, et nous bénissons ; nous sommes persécutés, et nous le supportons.
\VS{13}Nous sommes calomniés, et nous prions ; nous sommes devenus comme les balayures du monde, comme le rebut de tous, jusqu'à maintenant.
\VS{14}Je n'écris pas ces choses pour vous faire honte, mais je vous avertis comme mes chers enfants.
\VS{15}Car même si vous aviez dix mille maîtres en Christ, vous n'avez pourtant pas plusieurs pères, car c'est moi qui vous ai engendrés en Jésus-Christ par l'Evangile.
\VS{16}Je vous prie donc d'être mes imitateurs.
\VS{17}C'est pour cela que je vous ai envoyé Timothée, qui est mon fils bien-aimé, et qui est fidèle dans le Seigneur, afin qu'il vous rappelle quelles sont mes voies en Christ et comment j'enseigne partout dans toutes les églises.
\TextTitle{L'autorité de Paul}
\VS{18}Quelques-uns se sont enflés d'orgueil comme si je ne devais pas aller chez vous.
\VS{19}Mais j'irai bientôt chez vous, si le Seigneur le veut ; et je connaîtrai non les paroles, mais la puissance de ceux qui se sont glorifiés.
\VS{20}Car le Royaume de Dieu ne consiste pas en paroles, mais en puissance.
\VS{21}Que voulez-vous ? Que j'aille chez vous avec la verge, ou avec charité et dans un esprit de douceur ?
\Chap{5}
\TextTitle{Cas d'inceste parmi les Corinthiens}
\VerseOne{}On entend dire de toutes parts qu'il y a parmi vous de l'impudicité, et une impudicité telle qu'elle ne se rencontre même pas chez les Gentils ; c'est au point où l'un de vous a la femme de son père\FTNT{L'inceste est interdit par la loi (Lé. 18:6-8).}.
\TextTitle{Importance d'ôter le mal de l'assemblée des saints}
\VS{2}Et vous êtes enflés d'orgueil ! Et vous n'avez pas été plutôt dans le deuil, afin que celui qui a commis cette action soit retranché du milieu de vous.
\VS{3}Pour moi, étant absent de corps, mais présent en esprit, j'ai déjà jugé comme si j'étais présent, celui qui a commis une telle action.
\VS{4}Vous et mon esprit étant assemblés au nom de notre Seigneur Jésus-Christ, j'ai ordonné, avec la puissance de notre Seigneur Jésus-Christ,
\VS{5}qu'un tel homme soit livré à Satan\FTNT{Cette déclaration de Paul peut paraître choquante pour certains, mais elle nous rappelle l'histoire de Job, qui fut mis à l'épreuve par Yahweh qui l'avait livré à Satan (Job. 1:12). Paul espérait ainsi amener cet homme à la repentance en l'excluant de l'assemblée.} pour la destruction de la chair, afin que l'esprit soit sauvé au jour du Seigneur Jésus.
\VS{6}Votre vanité est mal fondée. Ne savez-vous pas qu'un peu de levain\FTNT{Le levain fait gonfler ou enfler. Il symbolise la cause principale de nombreux péchés : L'orgueil. Dans la Bible, le levain représente aussi des péchés spirituellement destructeurs comme la malice, la méchanceté, l'hypocrisie et les faux enseignements (Mt. 16:11-12 ; Lu. 12:1).} fait lever toute la pâte ?
\VS{7}Otez donc le vieux levain, afin que vous soyez une nouvelle pâte, puisque vous êtes sans levain ; car Christ, notre Pâque\FTNT{Ex. 12.}, a été sacrifié pour nous.
\VS{8}C'est pourquoi célébrons donc la fête, non avec du vieux levain, non avec un levain de méchanceté et de malice, mais avec les pains sans levain de la sincérité et de la vérité.
\TextTitle{Attitude du chrétien envers les faux frères et les non-croyants}
\VS{9}Je vous ai écrit dans ma lettre de ne pas vous mêler\FTNT{Mêler vient du grec « sunanamignumi » qui signifie : « mêler ensemble, se tenir en compagnie avec, être intime avec quelqu'un. Avoir des relations, être en communication » (Ps. 1:1 ; 1 Co. 15:33 ; Ro. 16:17-18 ; Tit. 3:10).} avec les fornicateurs,
\VS{10}non pas d'une manière absolue avec les fornicateurs de ce monde, ou avec les cupides, ou les ravisseurs, ou les idolâtres ; autrement, il vous faudrait sortir du monde.
\VS{11}Maintenant, ce que je vous ai écrit, c'est de ne pas avoir de relations avec quelqu'un qui, se nommant frère, est fornicateur, ou cupide, ou idolâtre, ou médisant, ou ivrogne, ou ravisseur, de ne même pas manger avec un tel homme.
\VS{12}Car qu'ai-je à juger ceux qui sont dehors ? N'est-ce pas ceux du dedans que vous avez à juger ?
\VS{13}Mais Dieu juge ceux qui sont du dehors. Otez donc le méchant du milieu de vous.
\Chap{6}
\TextTitle{Interdiction des procès entre chrétiens parmi les non-croyants}
\VerseOne{}Quand quelqu'un d'entre vous a une affaire contre un autre, ose-t-il bien aller en jugement devant les injustes, et il ne va pas devant les saints ?
\VS{2}Ne savez-vous pas que les saints jugeront le monde\FTNT{L'Eglise jugera les nations. Les douze apôtres jugeront Israël (Mt. 19:28 ; Lu. 22:30).} ? Or si le monde doit être jugé par vous, êtes-vous indignes de rendre les moindres jugements ?
\VS{3}Ne savez-vous pas que nous jugerons les anges\FTNT{Le mot ange vient du grec « aggelos » et veut dire « messager, envoyé, ange ». Ce terme s'applique donc aussi bien aux hommes qu'aux créatures spirituelles.} ? Et à plus forte raison les choses de cette vie ?
\VS{4}Si donc vous avez des procès pour les affaires de cette vie, prenez pour juge ceux qui sont des moins estimés dans l'Eglise !
\VS{5}Je le dis à votre honte. Ainsi il n'y a parmi vous pas un seul homme sage qui puisse prononcer un jugement entre frères.
\VS{6}Mais un frère a des procès contre son frère, et cela devant les infidèles.
\VS{7}C'est déjà un grand défaut chez vous que vous ayez des procès entre vous. Pourquoi ne souffrez-vous pas plutôt quelque injustice ? Pourquoi ne vous laissez-vous pas plutôt dépouiller ?
\VS{8}Mais c'est vous qui commettez l'injustice et qui dépouillez, et c'est envers des frères que vous agissez de la sorte !
\TextTitle{Le chrétien est sanctifié, lavé et justifié}
\VS{9}Ne savez-vous pas que les injustes n'hériteront point le Royaume de Dieu ? Ne vous y trompez pas : Ni les fornicateurs, ni les idolâtres, ni les adultères,
\VS{10}ni les efféminés, ni les homosexuels, ni les voleurs, ni les avares, ni les ivrognes, ni les médisants, ni les ravisseurs, n'hériteront le Royaume de Dieu.
\VS{11}Et c'est là ce que vous étiez ; mais vous avez été lavés, mais vous avez été sanctifiés, mais vous avez été justifiés au nom du Seigneur Jésus, et par l'Esprit de notre Dieu.
\VS{12}Toutes choses me sont permises, mais toutes choses ne conviennent pas ; toutes choses me sont permises, mais je ne me rendrai esclave d'aucune chose.
\TextTitle{Le chrétien est la propriété du Seigneur}
\VS{13}Les aliments sont pour le ventre, et le ventre pour les aliments ; et Dieu détruira l'un comme les autres. Or le corps n'est point pour la fornication, mais pour le Seigneur, et le Seigneur pour le corps.
\VS{14}Et Dieu qui a ressuscité le Seigneur, nous ressuscitera aussi par sa puissance.
\VS{15}Ne savez-vous pas que vos corps sont les membres de Christ ? Prendrai-je donc les membres de Christ pour en faire les membres d'une prostituée ? A Dieu ne plaise !
\VS{16}Ne savez-vous pas que celui qui s'unit à la prostituée, devient un même corps avec elle ? Car il est dit : Les deux deviendront une même chair\FTNT{Ge. 2:24.}.
\VS{17}Mais celui qui s'unit au Seigneur est avec lui un seul esprit.
\VS{18}Fuyez la fornication. Quelque autre péché qu'un homme commette, ce péché est hors du corps ; mais le fornicateur pèche contre son propre corps.
\TextTitle{Le chrétien est le temple du Saint-Esprit}
\VS{19}Ne savez-vous pas que votre corps est le temple du Saint-Esprit qui est en vous, et que vous avez reçu de Dieu, et que vous ne vous appartenez point à vous-mêmes ?
\VS{20}Car vous avez été achetés à un prix ; glorifiez donc Dieu dans votre corps et dans votre esprit, qui appartiennent à Dieu.
\Chap{7}
\TextTitle{La sanctification dans le mariage}
\VerseOne{}Pour ce qui concerne les choses au sujet desquelles vous m'avez écrit : Je vous dis qu'il est bon à l'homme de ne pas se marier.
\VS{2}Toutefois, pour éviter la fornication, que chacun ait sa femme, et que chaque femme ait son mari.
\VS{3}Que le mari rende à sa femme la bienveillance qui lui est due ; et que la femme de même la rende à son mari.
\VS{4}Car la femme n'a pas de pouvoir sur son propre corps, mais c'est son mari. De même, le mari n'a pas de pouvoir sur son propre corps, mais c'est sa femme.
\VS{5}Ne vous privez point l'un de l'autre, si ce n'est par un consentement mutuel, pour un temps, afin que vous vaquiez au jeûne et à la prière, mais après cela retournez ensemble, de peur que Satan ne vous tente par votre manque de contrôle.
\VS{6}Or je dis ceci par conseil, et non par commandement.
\VS{7}Car je voudrais que tous les hommes soient comme moi ; mais chacun a reçu de Dieu un don particulier, l'un d'une manière, l'autre d'une autre.
\VS{8}A ceux qui ne sont pas mariés, et aux veuves, je dis qu'il leur est bon de demeurer comme moi.
\VS{9}Mais s'ils manquent de maîtrise, qu'ils se marient ; car il vaut mieux se marier que de brûler.
\TextTitle{Recommandations à ceux qui sont mariés}
\VS{10}Et quant à ceux qui sont mariés, je leur ordonne, non pas moi, mais le Seigneur, que la femme ne se sépare point de son mari.
\VS{11}Et si elle s'en sépare, qu'elle demeure sans être mariée, ou qu'elle se réconcilie avec son mari ; que le mari aussi ne quitte point sa femme.
\VS{12}Mais aux autres je leur dis, et non pas le Seigneur : Si un frère a une femme incrédule et qu'elle consente d'habiter avec lui, qu'il ne la quitte point.
\VS{13}Et si une femme a un mari incrédule et qu'il consente d'habiter avec elle, qu'elle ne le quitte point.
\VS{14}Car le mari incrédule est sanctifié par la femme, et la femme incrédule est sanctifiée par le mari ; autrement vos enfants seraient impurs, or maintenant ils sont saints.
\VS{15}Que si l'incrédule se sépare, qu'il se sépare ; le frère ou la sœur ne sont point liés dans ce cas-là, car Dieu nous a appelés à la paix.
\VS{16}Car que sais-tu, femme, si tu sauveras ton mari ? Ou que sais-tu, mari, si tu sauveras ta femme ?
\TextTitle{La circoncision et l'incirconcision ne sont rien, Dieu est tout}
\VS{17}Toutefois, que chacun marche selon le don qu'il a reçu de Dieu, chacun selon l'appel qu'il a reçu du Seigneur. C'est ainsi que je l'ordonne dans toutes les églises.
\VS{18}Quelqu'un a-t-il été appelé étant circoncis ? Qu'il ne ramène point le prépuce\FTNT{Vient du grec « Epispaomai » qui a pour définition : Ne pas devenir incirconcis. Aux jours d'Antiochus IV, dit aussi Antioche Epiphane (voir commentaire en Da.8:9), certains Juifs, voulant échapper aux persécutions, cachaient le signe de leur nationalité, la circoncision, en se faisant reproduire artificiellement le prépuce par une opération chirurgicale qui étendait la peau restante.}. Quelqu'un a-t-il été appelé incirconcis ? Qu'il ne se fasse pas circoncire.
\VS{19}La circoncision n'est rien, et l'incirconcision aussi n'est rien, mais l'observation des commandements de Dieu est tout.
\VS{20}Que chacun demeure dans la condition où il était quand il a été appelé.
\VS{21}As-tu été appelé étant esclave ? Ne t'en inquiète pas ; mais si tu peux être mis en liberté, profites-en plutôt.
\VS{22}Car l'esclave qui a été appelé par notre Seigneur est un affranchi du Seigneur ; de même, celui qui est appelé étant libre, est un esclave de Christ.
\VS{23}Vous avez été rachetés à un prix, ne devenez pas les esclaves des hommes.
\VS{24}Mes frères, que chacun demeure devant Dieu dans l'état où il était quand il a été appelé.
\TextTitle{Conseils de Paul aux célibataires}
\VS{25}Pour ce qui concerne les vierges, je n'ai point de commandement du Seigneur, mais je donne un avis comme ayant obtenu miséricorde du Seigneur pour être fidèle.
\VS{26}Voici donc ce que j'estime bon, à cause des afflictions présentes : Il est avantageux à chacun de demeurer comme il est.
\VS{27}Es-tu lié à une femme ? Ne cherche pas à rompre ce lien. N'es-tu pas lié à une femme ? Ne cherche point de femme.
\VS{28}Si tu te maries, tu ne pèches point ; et si la vierge se marie, elle ne pèche point aussi ; mais ceux qui sont mariés auront des afflictions dans la chair ; or je voudrais vous les épargner.
\VS{29}Mais je vous dis ceci, mes frères : Le temps est court, que désormais ceux qui ont une femme soient comme n'en ayant pas ;
\VS{30}ceux qui pleurent comme ne pleurant pas, ceux qui se réjouissent comme ne se réjouissant pas, ceux qui achètent comme ne possédant pas,
\VS{31}et ceux qui usent de ce monde comme n'en usant pas, car la figure de ce monde passe.
\VS{32}Or je voudrais que vous soyez sans inquiétude. Celui qui n'est pas marié s'occupe des choses du Seigneur, cherchant à plaire au Seigneur.
\VS{33}Mais celui qui est marié s'occupe des choses de ce monde, cherchant à plaire à sa femme, et ainsi il est divisé.
\VS{34}Il y a de même une différence entre la femme mariée et la vierge : Celle qui n'est pas mariée s'occupe des choses du Seigneur, afin d'être sainte de corps et d'esprit ; mais celle qui est mariée s'occupe des choses du monde pour plaire à son mari.
\VS{35}Je dis cela dans votre intérêt, ce n'est pas pour vous tendre un piège, mais pour vous porter à ce qui est bienséant et propre à vous unir au Seigneur sans aucune distraction.
\VS{36}Mais si quelqu'un croit qu'il n'est pas honorable que sa fille dépasse la fleur de l'âge sans être mariée, et qu'il faille la marier, qu'il fasse ce qu'il veut, il ne pèche point ; qu'elle soit mariée.
\VS{37}Mais celui qui a pris une ferme résolution, sans contrainte, et avec l'exercice de sa propre volonté en son cœur, de garder sa fille vierge, celui-là fait bien.
\VS{38}Celui donc qui la marie fait bien, mais celui qui ne la marie pas fait mieux.
\VS{39}La femme est liée par la loi pendant tout le temps que son mari est en vie\FTNT{Dieu est contre le divorce. Pour le Seigneur, le mariage doit être un engagement à vie (Mal. 2:16 ; Ro. 7:1-3).}, mais si son mari meurt, elle est libre de se marier à qui elle veut ; seulement, que ce soit dans le Seigneur.
\VS{40}Elle est néanmoins plus heureuse si elle demeure ainsi, selon mon avis ; or j'estime que j'ai aussi l'Esprit de Dieu.
\Chap{8}
\TextTitle{Viandes sacrifiées aux idoles ; limites de la liberté chrétienne}
\VerseOne{}Pour ce qui concerne les choses qui sont sacrifiées aux idoles\FTNT{A Corinthe, on offrait rituellement des viandes sacrifiées aux idoles. A ces occasions, certaines parties des animaux sacrifiés étaient déposées sur l'autel de l'idole, d'autres étaient données aux prêtres et aux adorateurs, qui les mangeaient lors d'un repas ou d'un festin, soit dans le temple, soit dans une maison particulière. Certains morceaux de la chair offerte aux idoles étaient ensuite apportés au marché pour être vendus (Da. 1).}, nous savons que nous avons tous de la connaissance. La connaissance enfle, mais la charité édifie.
\VS{2}Et si quelqu'un croit savoir quelque chose, il n'a encore rien connu comme il faut connaître.
\VS{3}Mais si quelqu'un aime Dieu, il est connu de lui.
\VS{4}Pour ce qui est donc de manger des choses sacrifiées aux idoles, nous savons que l'idole n'est rien dans le monde et qu'il n'y a aucun autre Dieu qu'un seul\FTNT{Paul affirme avec force que le Dieu Créateur n'est pas mélangé avec d'autres divinités. Voir De. 6:4.}.
\VS{5}Car s'il est des êtres qui sont appelés dieux, soit dans le ciel, soit sur la terre, comme il existe réellement plusieurs dieux, et plusieurs seigneurs,
\VS{6}nous n'avons pourtant qu'un seul Dieu, qui est le Père, de qui viennent toutes choses, et nous pour lui ; et un seul Seigneur : Jésus-Christ, par qui sont toutes choses, et nous par lui.
\VS{7}Mais tous n'ont pas cette connaissance. Car quelques-uns, d'après la manière dont ils envisagent encore l'idole, mangent de ces choses comme étant sacrifiées aux idoles, et leur conscience qui est faible en est souillée.
\VS{8}Ce n'est pas une viande qui nous rend agréables à Dieu ; car si nous en mangeons, nous n'avons rien de plus ; si nous n'en mangeons pas, nous n'avons rien de moins.
\VS{9}Mais prenez garde que cette liberté que vous avez ne soit en quelque sorte un scandale pour les faibles.
\VS{10}Car si quelqu'un te voit, toi qui as de la connaissance, être à table dans le temple des idoles, sa conscience, à lui qui est faible, ne le portera-t-elle pas à manger des choses sacrifiées aux idoles ?
\VS{11}Et ainsi ton frère, qui est faible, et pour lequel Christ est mort, périra par ta connaissance.
\VS{12}Or quand vous péchez ainsi contre vos frères, et que vous blessez leur conscience qui est faible, vous péchez contre Christ.
\VS{13}C'est pourquoi, si la viande scandalise mon frère, je ne mangerai jamais de chair pour ne point scandaliser mon frère.
\Chap{9}
\TextTitle{Défense de Paul\FTNTT{Ga. 1:11}}
\VerseOne{}Ne suis-je pas apôtre ? Ne suis-je pas libre ? N'ai-je pas vu notre Seigneur Jésus-Christ ? N'êtes-vous pas mon ouvrage dans le Seigneur ?
\VS{2}Si je ne suis pas apôtre pour les autres, je le suis au moins pour vous, car vous êtes le sceau de mon apostolat dans le Seigneur.
\VS{3}C'est là ma défense contre ceux qui me condamnent.
\VS{4}N'avons-nous pas le droit de manger et de boire ?
\VS{5}N'avons-nous pas le droit de mener avec nous une sœur, notre femme, comme font les autres apôtres, et les frères du Seigneur, et Céphas ?
\VS{6}N'y a-t-il que Barnabas et moi qui n'ayons pas le droit de ne pas travailler ?
\TextTitle{Dieu prend soin de ses serviteurs}
\VS{7}Qui est-ce qui va à la guerre à ses propres frais ? Qui est-ce qui plante une vigne et n'en mange pas le fruit ? Qui est-ce qui fait paître un troupeau et ne se nourrit pas du lait du troupeau ?
\VS{8}Dis–je ces choses selon l'homme ? la loi ne dit–elle pas aussi la même chose ?
\VS{9}Car il est écrit dans la Loi de Moïse : Tu ne muselleras pas le bœuf qui foule le grain\FTNT{De. 25:4.}. Dieu se met-il en peine des bœufs ?
\VS{10}Ou parle-t-il uniquement pour nous ? Oui, c'est pour nous qu'il a été écrit que celui qui laboure doit labourer avec espérance, et celui qui foule le blé, le foule avec l'espérance d'y avoir part.
\VS{11}Si nous avons semé parmi vous des biens spirituels, est-ce une grosse affaire si nous moissonnons vos biens charnels ?
\VS{12}Si d'autres usent de ce droit à votre égard, pourquoi n'en userions-nous pas plutôt qu'eux ? Cependant nous n'avons point usé de ce droit, mais au contraire, nous supportons toutes sortes d'incommodités, afin de ne pas créer d'obstacle à l'Evangile de Christ.
\VS{13}Ne savez-vous pas que ceux qui font le service sacré mangent des choses sacrées ; et que ceux qui servent à l'autel participent à l'autel\FTNT{No. 18:8-31.} ?
\VS{14}Le Seigneur a ordonné que ceux qui annoncent l'Evangile vivent de l'Evangile.
\VS{15}Pour moi, je n'ai usé d'aucun de ces droits, et ce n'est pas afin de les réclamer en ma faveur que j'écris ainsi ; car j'aimerais mieux mourir que de me laisser enlever cette gloire.
\VS{16}Car si j'évangélise, ce n'est pas pour moi un sujet de gloire, c'est parce que la nécessité m'en est imposée ; et malheur à moi si je n'évangélise pas !
\VS{17}Si je le fais de bon cœur, j'en aurai la récompense ; mais si je le fais malgré moi, c'est une charge qui m'est confiée.
\VS{18}Quelle récompense en ai-je donc ? C'est qu'en prêchant l'Evangile, je prêche l'Evangile de Christ sans qu'il en coûte rien\FTNT{Paul annonçait l'Evangile gratuitement. Donnez gratuitement : C'est la suite logique des choses, on reçoit gratuitement et on donne gratuitement. Si nous sommes comme Christ (car là est le sens du mot disciple), nous devons agir comme lui. Il a donné ses enseignements et nourrit les gens gratuitement. Dans Ap. 21:6 et 22:17, le Seigneur invite toutes les personnes qui ont soif à venir s'abreuver gratuitement. Alors pourquoi vendre la parole c'est-à-dire l'eau qu'on a reçue gratuitement ? Nous devons donner gratuitement.}, afin que je n'abuse pas de mon autorité dans l'Evangile.
\TextTitle{L'attitude d'un vrai serviteur de Dieu}
\VS{19}Car bien que je sois libre à l'égard de tous, je me suis pourtant rendu le serviteur de tous, afin de gagner plus de personnes.
\VS{20}Avec les Juifs, j'ai été comme Juif, afin de gagner les Juifs ; avec ceux qui sont sous la loi, comme si j'étais sous la loi, afin de gagner ceux qui sont sous la loi ;
\VS{21}avec ceux qui sont sans loi, comme si j'étais sans loi (quoique je ne sois point sans la Loi quant à Dieu, étant sous la Loi de Christ), afin de gagner ceux qui sont sans loi.
\VS{22}J'ai été faible avec les faibles, afin de gagner les faibles ; je me suis fait tout à tous, afin d'en sauver au moins quelques-uns.
\VS{23}Je fais cela à cause de l'Evangile, afin que j'en sois fait aussi participant avec les autres.
\VS{24}Ne savez-vous pas que ceux qui courent dans le stade, courent tous, mais qu'un seul remporte le prix ? Courez de manière à le remporter.
\VS{25}Tout homme qui combat, vit entièrement de régime ; et ces gens-là le font pour obtenir une couronne corruptible\FTNT{Paul fait ici allusion aux athlètes vainqueurs des jeux olympiques panhélleniques qui recevaient pour récompense une couronne qui était effectivement corruptible puisqu’elle était faite, selon les sites, de feuillages  d’olivier, de laurier, de pin ou encore de céleri.} ; mais nous, faisons-le pour une couronne incorruptible.
\VS{26}Moi donc je cours, non pas comme à l'aventure ; je combats, mais non pas comme battant l'air.
\VS{27}Mais je traite durement mon corps et je le tiens assujetti, de peur d'être moi-même désapprouvé après avoir prêché aux autres.
\Chap{10}
\TextTitle{Israël dans le désert}
\VerseOne{}Mes frères, je ne veux pas que vous ignoriez que nos pères ont tous été sous la nuée, et qu'ils ont tous passé au travers de la mer,
\VS{2}et qu'ils ont tous été baptisés en Moïse dans la nuée et dans la mer ;
\VS{3}et qu'ils ont tous mangé la même viande spirituelle ;
\VS{4}et qu'ils ont tous bu le même breuvage spirituel : Car ils buvaient de l'eau du rocher spirituel qui les suivait, et ce rocher\FTNT{Jésus-Christ, le Rocher des âges. Voir Es. 8:13-17.} était Christ.
\VS{5}Mais la plupart d'entre eux ne furent point agréables à Dieu puisqu'ils périrent dans le désert.
\VS{6}Or ces choses ont été des exemples pour nous, afin que nous ne convoitions point des choses mauvaises, comme eux-mêmes les ont convoitées.
\VS{7}Ne devenez point idolâtres, comme quelques-uns d'entre eux, selon qu'il est écrit : Le peuple s'assit pour manger et pour boire, puis ils se levèrent pour jouer\FTNT{Ex. 32:6.}.
\VS{8}Ne nous livrons pas à la fornication, comme quelques-uns d'entre eux s'y livrèrent, de sorte qu'il en tomba vingt-trois mille en un jour\FTNT{No. 25:9.}.
\VS{9}Ne tentons\FTNT{Tenter : Du grec « ekpeirazo » : Mettre à l'épreuve ; éprouver le caractère de Dieu et son pouvoir.} point Christ, comme le tentèrent\FTNT{Tenter : Du grec « peirazo », essayer si une chose peut être faite ; éprouver malicieusement, astucieusement, pour prouver ses sentiments et ses jugements ; essayer ou éprouver la foi, la vertu, le caractère par la séduction du péché ; solliciter à pécher ; infliger des maux dans le but d'éprouver. Ce terme est aussi utilisé lorsque les hommes veulent tenter Dieu en montrant leur méfiance, par une conduite impie ou méchante, pour éprouver la justice et la patience de Dieu, et le défier, pour le pousser à donner une preuve de ses perfections.} quelques-uns d'entre eux qui périrent par les serpents\FTNT{No. 21:6-9.}.
\VS{10}Ne murmurez point, comme quelques-uns d'entre eux qui périrent par le destructeur\FTNT{No. 14:2-29 ; No. 26:63-65.}.
\TextTitle{L'expérience d'Israël est un enseignement pour l'Eglise}
\VS{11}Or toutes ces choses leur sont arrivées pour servir d'exemples, et elles ont été écrites pour notre instruction, comme étant ceux auxquels les derniers temps sont parvenus.
\VS{12}Que celui donc qui pense demeurer debout prenne garde qu'il ne tombe.
\VS{13}Aucune tentation ne vous a éprouvés, qui n'ait été une tentation humaine, et Dieu qui est fidèle ne permettra pas que vous soyez tentés au-delà de vos forces, mais avec la tentation il préparera aussi le moyen d'en sortir, afin que vous puissiez la supporter.
\VS{14}C'est pourquoi, mes bien-aimés, fuyez l'idolâtrie.
\VS{15}Je vous parle comme à des personnes intelligentes, jugez vous-mêmes de ce que je dis.
\TextTitle{Communion avec le Seigneur et non avec les démons}
\VS{16}La coupe de bénédiction, que nous bénissons, n'est-elle pas la communion du sang de Christ ? Et le pain que nous rompons, n'est-il pas la communion au corps de Christ ?
\VS{17}Parce qu'il n'y a qu'un seul pain, nous qui sommes plusieurs sommes un seul corps ; car nous sommes tous participants du même pain.
\VS{18}Voyez l'Israël selon la chair, ceux qui mangent les sacrifices ne sont-ils pas en communion avec l'autel ?
\VS{19}Que dis-je donc ? Que l'idole soit quelque chose ? Ou que ce qui est sacrifié à l'idole soit quelque chose ? Nullement.
\VS{20}Mais je dis que les choses que les Gentils sacrifient, ils les sacrifient aux démons, et non à Dieu ; or je ne veux pas que vous soyez en communion avec des démons.
\VS{21}Vous ne pouvez pas boire la coupe du Seigneur et la coupe des démons ; vous ne pouvez pas participer à la table du Seigneur et à la table des démons\FTNT{L'apôtre Paul nous parle de deux sortes de tables : La table de Jézabel (ou des démons) et la table du Seigneur. La table du Seigneur à été révélée à Moïse (Ex. 25:23-30 ; Lé. 24:5-9). Il y avait dessus 12 pains destinés à la consommation des sacrificateurs. Ces pains étaient renouvelés chaque sabbat et représentaient Christ, le Pain de Dieu, qui est l'aliment du croyant-sacrificateur (Jn. 6:33-58). La table de Jézabel nous est présentée dans 1 R. 18:19 : « Fais maintenant rassembler tout Israël auprès de moi, à la montagne du Carmel, et aussi les quatre cent cinquante prophètes de Baal et les quatre cents prophètes d'Astarté qui mangent à la table de Jézabel ». Jézabel avait à sa table 850 faux prophètes qui partageaient son repas. Voir Ap. 17. Satan est maître en matière de déguisement et d'imitation (2 Co. 11:13-15). Il a donc imité la table du Seigneur et propose aux hommes les mets du roi et le vin de la débauche (Da. 1). Il invite ceux qui cherchent Dieu à sa table afin de les détourner de la vision du ciel. Voir Mt. 6:24 ; Lu. 16:13.}.
\VS{22}Voulons-nous provoquer la jalousie du Seigneur ? Sommes-nous plus forts que lui ?
\TextTitle{La loi de l'amour s'applique dans le manger et le boire\FTNTT{Ro. 14:1-23}}
\VS{23}Toutes choses me sont permises, mais toutes ne sont pas utiles ; toutes choses me sont permises, mais toutes n'édifient pas.
\VS{24}Que personne ne cherche son propre intérêt, mais que chacun cherche celui d'autrui.
\VS{25}Mangez de tout ce qui se vend au marché, sans vous enquérir de rien par motif de conscience\FTNT{1 Ti. 4:3-5.}.
\VS{26}Car la terre avec tout ce qu'elle contient, est au Seigneur.
\VS{27}Si un incrédule vous invite et que vous vouliez aller, mangez de tout ce qui sera mis devant vous, sans vous enquérir par motif de conscience.
\VS{28}Mais si quelqu'un vous dit : Ceci a été sacrifié aux idoles, n'en mangez pas, à cause de celui qui vous a avertis, et à cause de la conscience ; car la terre avec tout ce qu'elle contient est au Seigneur.
\VS{29}Je parle ici, non de votre conscience, mais de celle de l'autre. Pourquoi ma liberté serait-elle condamnée par la conscience d'un autre ?
\VS{30}Et si par la grâce j'en suis participant, pourquoi suis-je blâmé pour une chose dont je rends grâces ?
\VS{31}Soit donc que vous mangiez, soit que vous buviez, ou que vous fassiez quelque autre chose, faites tout à la gloire de Dieu.
\VS{32}Soyez tels que vous ne donniez aucun scandale ni aux Juifs, ni aux Grecs, ni à l'Eglise de Dieu,
\VS{33}de la même manière que moi aussi, je m'efforce en toutes choses de complaire à tous, cherchant, non pas mon avantage, mais celui du plus grand nombre, afin qu'ils soient sauvés.
\Chap{11}
\VerseOne{}Soyez mes imitateurs comme je le suis moi-même de Christ.
\TextTitle{Position de l'homme et de la femme devant Dieu}
\VS{2}Or mes frères, je vous loue de ce que vous vous souvenez de tout ce qui me concerne, et de ce que vous retenez mes instructions comme je vous les ai données.
\VS{3}Mais je veux que vous sachiez que Christ est le chef\FTNT{Le mot « chef » vient du grec « kephal » qui signifie tête. Jésus-Christ est la seule tête et l'unique chef de l'Eglise (Ep. 1:22-23 ; Col. 1:18). Toute personne qui se proclame la tête de l'église devient naturellement antéchrist.} de tout homme, que l'homme est le chef de la femme, et que Dieu est le chef de Christ.
\VS{4}Tout homme qui prie ou qui prophétise, ayant quelque chose sur la tête, déshonore son chef.
\VS{5}Toute femme au contraire qui prie, ou qui prophétise sans avoir la tête couverte, déshonore son chef, c'est comme si elle était rasée.
\VS{6}Car si une femme n'est pas couverte, qu'on lui coupe aussi les cheveux. Or, s'il est honteux pour une femme d'avoir les cheveux coupés, ou d'être rasée, qu'elle se voile.
\VS{7}Car pour ce qui est de l'homme, il ne doit point couvrir sa tête, vu qu'il est l'image et la gloire de Dieu ; mais la femme est la gloire de l'homme.
\VS{8}Parce que l'homme n'a point été tiré de la femme, mais la femme a été tirée de l'homme.
\VS{9}Et aussi l'homme n'a pas été créé pour la femme, mais la femme pour l'homme.
\VS{10}C'est pourquoi la femme à cause des anges doit avoir sur la tête une marque de l'autorité de son mari dont elle dépend.
\VS{11}Toutefois, dans le Seigneur, l'homme n'est point sans la femme ni la femme sans l'homme.
\VS{12}Car comme la femme est par l'homme, de même l'homme est par la femme, et tout cela procède de Dieu.
\VS{13}Jugez-en vous-mêmes : Est-il convenable que la femme prie Dieu sans être couverte ?
\VS{14}La nature elle-même ne vous enseigne-t-elle pas que c'est une honte pour l'homme d'avoir de longs cheveux,
\VS{15}mais que c'est une gloire pour la femme de porter des longs cheveux, parce que la chevelure lui a été donnée pour lui servir de voile ?
\VS{16}Si quelqu'un aime à contester, nous n'avons pas une telle coutume, ni les églises de Dieu.
\TextTitle{Le repas du Seigneur et les abus dénoncés par Paul}
\VS{17}Or en ce que je vais vous dire, je ne vous loue point : C'est que vous vous assemblez, non pour devenir meilleurs, mais pour empirer.
\VS{18}Car premièrement, lorsque vous vous réunissez en assemblée, j'apprends qu'il y a des divisions parmi vous et j'en crois une partie,
\VS{19}car il faut qu'il y ait même des hérésies\FTNT{Le mot "hérésies" est aussi traduit par "sectes".} parmi vous, afin que ceux qui sont dignes d'être approuvés soient reconnus parmi vous.
\VS{20}Quand donc vous vous assemblez ainsi tous ensemble, ce n'est pas pour manger le repas du Seigneur ;
\VS{21}car, quand on se met à table, chacun commence par prendre son souper particulier, et l'un a faim tandis que l'autre est ivre.
\VS{22}N'avez-vous donc pas de maisons pour manger et pour boire ? Ou méprisez-vous l'Eglise de Dieu et faites-vous honte à ceux qui n'ont rien ? Que vous dirai-je ? Vous louerai-je ? Je ne vous loue point en cela.
\TextTitle{Le repas du Seigneur}
\VS{23}Car j'ai reçu du Seigneur ce qu'aussi je vous ai donné ; c'est que le Seigneur Jésus, la nuit où il fut trahi, prit du pain,
\VS{24}et après avoir rendu grâces, le rompit et dit : Prenez, mangez : Ceci est mon corps qui est rompu pour vous ; faites ceci en mémoire de moi.
\VS{25}De même aussi après le souper, il prit la coupe, en disant : Cette coupe est la Nouvelle Alliance en mon sang ; faites ceci toutes les fois que vous en boirez, en mémoire de moi\FTNT{Mt. 26:26-28 ; Mc. 14:22-24 ; Lu. 22:19-20.}.
\VS{26}Car toutes les fois que vous mangerez de ce pain, et que vous boirez de cette coupe, vous annoncerez la mort du Seigneur, jusqu'à ce qu'il vienne.
\VS{27}C'est pourquoi quiconque mangera de ce pain ou boira de la coupe du Seigneur indignement, sera coupable envers le corps et le sang du Seigneur.
\VS{28}Que chacun donc s'éprouve soi-même, et ainsi qu'il mange de ce pain, et qu'il boive de cette coupe.
\VS{29} Car celui qui en mange et qui en boit indignement, mange et boit sa condamnation, ne distinguant point le corps du Seigneur.
\VS{30}C'est pour cela qu'il y a parmi vous beaucoup d'infirmes et de malades, et que plusieurs dorment.
\VS{31}Car si nous nous jugions nous-mêmes, nous ne serions point jugés.
\VS{32}Mais quand nous sommes jugés, nous sommes enseignés par le Seigneur, afin que nous ne soyons point condamnés avec le monde.
\VS{33}C'est pourquoi, mes frères, quand vous vous assemblez pour manger, attendez-vous les uns les autres.
\VS{34}Et si quelqu'un a faim, qu'il mange dans sa maison, afin que vous ne vous assembliez pas pour votre condamnation. Touchant les autres points, je les réglerai quand je serai arrivé.
\Chap{12}
\TextTitle{L'œuvre de l'Esprit : Révèle Christ}
\VerseOne{}Pour ce qui concerne les dons spirituels, je ne veux point, mes frères, que vous soyez ignorants.
\VS{2}Vous savez que lorsque vous étiez Gentils, vous vous laissiez entraîner vers les idoles muettes, selon que vous étiez conduits.
\VS{3}C'est pourquoi je vous fais savoir que personne, s'il parle par l'Esprit de Dieu, ne dit : Jésus est anathème ! Et personne ne peut dire : Jésus est le Seigneur ! Si ce n'est par le Saint-Esprit.
\TextTitle{Les dons de l'Esprit\FTNTT{Ep. 4:7-16}}
\VS{4} Or il y a diversité de dons, mais il n'y a qu'un même Esprit.
\VS{5}Il y a aussi diversité de ministères, mais il n'y a qu'un même Seigneur.
\VS{6}Il y a aussi diversité d'opérations, mais il n'y a qu'un même Dieu qui opère toutes choses en tous.
\VS{7}Or à chacun est donnée la manifestation de l'Esprit pour l'utilité commune.
\VS{8}Car à l'un est donnée par l'Esprit, la parole de sagesse ; et à l'autre par le même Esprit, la parole de connaissance ;
\VS{9}et à un autre, la foi par ce même Esprit ; à un autre, les dons de guérison par ce même Esprit ;
\VS{10}et à un autre, les opérations des miracles ; à un autre, la prophétie ; à un autre, le don de discerner les esprits ; à un autre, la diversité de langues ; et à un autre, le don d'interpréter les langues.
\VS{11}Un seul et même Esprit opère toutes ces choses, distribuant à chacun ses dons en particulier comme il lui plaît.
\TextTitle{Utilité et rôle de chaque membre du corps de Christ}
\VS{12}Car comme le corps est un, et cependant a plusieurs membres, et comme tous les membres du corps, malgré leur nombre, ne forment qu'un seul corps, il en est de même de Christ.
\VS{13}Nous avons tous, en effet, été baptisés d'un même Esprit\FTNT{Le baptême du Saint-Esprit : Les signes du baptême du Saint-Esprit (la conversion) sont les fruits de l'Esprit et sont abordés en Ga. 5:22. A aucun endroit, les écritures stipulent que le parler en langues, qui est un don gratuit (Mt. 7:16-20), est en soi le signe du baptême du Saint-Esprit. Ainsi, il nous est dit que chaque croyant en Christ a le Saint-Esprit (1 Co. 12:13 ; Ro. 8:9 ; Ep. 1:13-14) mais que tous les croyants ne parlent pas forcément en langues (1 Co. 12:29-31).}, pour être un même corps, soit Juifs, soit Grecs, soit esclaves, soit libres, nous avons tous, dis-je, été abreuvés d'un seul Esprit.
\VS{14}Ainsi, le corps n'est pas un seul membre, mais il est formé de plusieurs membres.
\VS{15}Si le pied dit : Parce que je ne suis pas la main, je ne suis point du corps ; ne serait-il pas pourtant du corps ?
\VS{16}Et si l'oreille dit : Parce que je ne suis pas l'œil, je ne suis point du corps ; ne serait-elle pas pourtant du corps ?
\VS{17}Si tout le corps est l'œil, où serait l'ouïe ? Si tout est l'ouïe, où serait l'odorat ?
\VS{18}Mais maintenant Dieu a placé chaque membre dans le corps comme il a voulu.
\VS{19}Et si tous étaient un seul membre, où serait le corps ?
\VS{20}Maintenant donc, il y a plusieurs membres et un seul corps.
\VS{21}L'œil ne peut pas dire à la main : Je n'ai pas besoin de toi ; ni la tête dire aux pieds : Je n'ai pas besoin de vous.
\VS{22}Et qui plus est, les membres du corps qui semblent être les plus faibles sont beaucoup plus nécessaires ;
\VS{23}et ceux que nous estimons être les moins honorables au corps, nous les entourons d'un plus grand honneur. Ainsi, nos membres les moins décents reçoivent le plus d'honneur,
\VS{24}Car les parties qui sont belles en nous, n'en ont pas besoin. Mais Dieu a disposé le corps de manière à donner plus d'honneur à ce qui en manquait,
\VS{25}afin qu'il n'y ait pas de division dans le corps, mais que les membres aient un soin mutuel les uns des autres.
\VS{26}Et si l'un des membres souffre quelque chose, tous les membres souffrent avec lui ; si l'un des membres est honoré, tous les membres ensemble se réjouissent avec lui.
\VS{27}Vous êtes le corps de Christ, et vous êtes chacun l'un de ses membres.
\VS{28}Et Dieu a établi dans l'Eglise premièrement des apôtres, deuxièmement des prophètes, troisièmement des docteurs, ensuite ceux qui ont le don des miracles, puis ceux qui ont les dons de guérir, de secourir, de gouverner, de parler diverses langues.
\VS{29}Tous sont-ils apôtres ? Tous sont-ils prophètes ? Tous sont-ils docteurs ? Tous ont-ils le don des miracles ?
\VS{30}Tous ont-ils les dons de guérisons ? Tous parlent-ils diverses langues ? Tous interprètent-ils ?
\VS{31}Désirez avec ardeur des dons plus excellents, et je vais vous montrer la voie la plus excellente.
\Chap{13}
\TextTitle{L'amour, don par excellence}
\VerseOne{}Quand je parlerais toutes les langues des hommes\FTNT{Les langues des hommes. Les 120 Galiléens ont été rendus capables de s'exprimer dans diverses langues afin de pouvoir annoncer la vérité aux personnes en voyage à Jérusalem dans leurs propres langues. Voir Es. 28:11-12 ; Ac. 2:1-13.}, et même des anges\FTNT{La langue des anges ou langue inconnue est incompréhensible à notre intelligence, elle est un des moyens par lequel nous disons des mystères à Dieu. Voir 1 Co. 14:2 et 28 ; Ro. 8:25-26. Il faut une interprétation si l'on veut parler cette langue dans l'assemblée à cause des non croyants qui nous visitent (1 Co. 14:23). Voir Mc. 16:17.}, si je n'ai pas la charité\FTNT{Il est question ici de l'amour « agape » : L'amour divin et désintéressé, l'amour fraternel.}, je suis un airain qui résonne ou une cymbale qui retentit.
\VS{2}Et quand j'aurais le don de prophétie et que je connaîtrais tous les mystères et la science de toutes choses ; et quand j'aurais même toute la foi qu'on puisse avoir, jusqu'à transporter les montagnes, si je n'ai pas la charité, je ne suis rien.
\VS{3}Et quand je distribuerais tous mes biens pour la nourriture des pauvres, quand je livrerais mon corps pour être brûlé, si je n'ai pas la charité, cela ne me sert à rien.
\VS{4}La charité est patiente, la charité est douce, la charité n'est point envieuse, la charité n'use point d'insolence, elle ne s'enfle point d'orgueil,
\VS{5}elle ne fait rien de malhonnête, elle ne cherche point son intérêt, elle ne s'irrite point, elle n'impute pas le mal,
\VS{6}elle ne se réjouit point de l'injustice, mais elle se réjouit de la vérité.
\VS{7}Elle couvre\FTNT{Dans ce passage, le grec utilisé, « stego », signifie « toit, couverture, protéger ou garder en recouvrant, préserver » (Pr. 10:12 ; Pr. 17:9). La charité ne rappelle pas sans cesse les erreurs des uns et des autres, mais sait préserver en gardant le secret des fautes expiées. Par contre, en aucun cas elle ne permet la compromission avec le péché en ne dénonçant pas les œuvres des ténèbres (Mt. 18:15-18 ; Ja. 5:19-20).} tout, elle croit tout, elle espère tout, elle supporte tout.
\VS{8}La charité ne périt jamais. Les prophéties seront abolies et les langues cesseront, la connaissance sera abolie.
\VS{9}Car nous connaissons en partie et nous prophétisons en partie.
\VS{10}Mais quand la perfection sera venue, alors ce qui est en partie sera aboli.
\VS{11}Quand j'étais enfant, je parlais comme un enfant, je jugeais comme un enfant, je pensais comme un enfant ; mais quand je suis devenu homme, j'ai aboli ce qui était de l'enfance.
\VS{12}Car aujourd'hui nous voyons au moyen d'un miroir, de manière obscure, mais alors nous verrons face à face. Aujourd'hui je connais en partie, mais alors je connaîtrai comme j'ai été connu.
\VS{13}Maintenant ces trois choses demeurent : La foi, l'espérance et la charité ; mais la plus excellente de ces trois vertus c'est la charité.
\Chap{14}
\TextTitle{Importance du don de prophétie}
\VerseOne{}Recherchez la charité. Désirez avec ardeur les dons spirituels, mais surtout celui de prophétiser.
\VS{2}Parce que celui qui parle une langue inconnue ne parle point aux hommes, mais à Dieu, car personne ne le comprend, et c'est en esprit qu'il dit des mystères.
\VS{3}Mais celui qui prophétise, édifie, exhorte et console les hommes qui l'entendent.
\VS{4}Celui qui parle une langue inconnue s'édifie lui-même, mais celui qui prophétise édifie l'Eglise.
\VS{5}Je désire que vous parliez tous diverses langues, mais encore plus que vous prophétisiez. Celui qui prophétise est plus grand que celui qui parle diverses langues, à moins que ce dernier n'interprète, afin que l'Eglise en reçoive de l'édification.
\VS{6}Maintenant donc, mes frères, si je viens à vous et que je parle des langues inconnues, que vous servira cela si je ne vous parle pas par révélation, ou par science, ou par prophétie, ou par doctrine ?
\VS{7}De même, si les choses inanimées qui rendent un son, comme une flûte ou une harpe, ne rendent pas des sons distincts, comment reconnaîtra-t-on ce qui est joué sur la flûte ou sur la harpe ?
\VS{8}Et si la trompette rend un son confus, qui se préparera à la bataille ?
\VS{9}De même vous, si vous ne prononcez dans votre langue une parole distincte, comment saura-t-on ce que vous dites ? Car vous parlerez en l'air.
\VS{10}Et il y a, selon qu'il se rencontre, tant de divers sons dans le monde, et cependant aucun de ces sons n'est muet ;
\VS{11}mais si je ne sais point ce qu'on veut signifier par la parole, je serai un barbare pour celui qui parle, et celui qui parle sera un barbare pour moi.
\VS{12}Ainsi, puisque vous désirez avec ardeur les dons spirituels, que ce soit pour l'édification de l'Eglise que vous cherchiez à en posséder abondamment.
\VS{13}C'est pourquoi que celui qui parle une langue inconnue prie pour avoir le don d'interpréter.
\VS{14}Car si je prie dans une langue inconnue mon esprit est en prière, mais l'intelligence que j'en ai, est sans fruit.
\VS{15}Que faire donc ? Je prierai par l'esprit, mais je prierai aussi d'une manière à être entendu ; je chanterai par l'esprit, mais je chanterai aussi d'une manière à être entendu.
\VS{16}Autrement, si tu rends grâces par l'esprit, comment celui qui est du simple peuple dira-t-il Amen ! à ton action de grâces\FTNT{L'expression « actions de grâces » vient du grec « eucharisteo » ce qui signifie être reconnaissant, rendre grâces, remercier. Contrairement à ce que l'on enseigne dans beaucoup d'églises, il n'est pas question ici de faire une offrande d'argent mais de se montrer reconnaissant envers le Seigneur. Voir aussi commentaires en Lé. 3 et Lé. 7.}, puisqu'il ne sait pas ce que tu dis ?
\VS{17}Il est vrai que tu rends grâces, mais l'autre n'est pas édifié.
\VS{18}Je rends grâces à mon Dieu de ce que je parle plus de langues que vous tous.
\VS{19}Mais j'aime mieux prononcer dans l'Eglise cinq paroles d'une manière à être entendu, afin d'instruire aussi les autres, que dix mille paroles dans une langue inconnue.
\VS{20}Mes frères, ne soyez point des enfants sous le rapport du jugement, mais soyez des enfants à l'égard de la malice ; et à l'égard du jugement, soyez des hommes faits.
\VS{21}Il est écrit dans la loi : Je parlerai à ce peuple par des gens d'une autre langue, et par des lèvres étrangères, et ils ne m'écouteront pas même ainsi, dit le Seigneur\FTNT{Es. 28:11.}.
\VS{22}C'est pourquoi les langues sont un signe, non pour les croyants, mais pour les non-croyants ; la prophétie, au contraire, est un signe, non pour les non-croyants, mais pour les croyants.
\TextTitle{L'exercice des dons dans l'église locale}
\VS{23}Si donc, l'Eglise entière s'assemble en un corps, et que tous parlent des langues étrangères et qu'il entre des gens du commun peuple ou des non-croyants, ne diront-ils pas que vous êtes hors de sens ?
\VS{24}Mais si tous prophétisent, et qu'il entre un non-croyant ou quelqu'un du commun peuple, il est convaincu par tous et il est jugé de tous,
\VS{25}ainsi les secrets de son cœur sont manifestés, de telle sorte qu'il tombera sur sa face, il adorera Dieu et publiera que Dieu est véritablement parmi vous.
\VS{26}Que faire donc mes frères ? Lorsque vous vous assemblez, les uns ou les autres parmi vous ont-ils un cantique, une instruction, une langue étrangère, une révélation, une interprétation, que tout se fasse pour l'édification.
\VS{27}Et si quelqu'un parle une langue inconnue, que cela se fasse par deux, ou tout au plus par trois, chacun à son tour, et que quelqu'un interprète ;
\VS{28}s'il n'y a point d'interprète, que cet homme se taise dans l'Eglise, et qu'il parle à lui-même et à Dieu.
\VS{29}Et que deux ou trois prophètes parlent, et que les autres en jugent ;
\VS{30}et si quelque chose est révélé à un autre qui est assis, que le premier se taise.
\VS{31}Car vous pouvez tous prophétiser l'un après l'autre, afin que tous soient instruits et que tous soient consolés.
\VS{32}Et les esprits des prophètes sont soumis aux prophètes.
\VS{33}Car Dieu n'est point un Dieu de confusion, mais de paix, comme on le voit dans toutes les églises des saints.
\VS{34}Que les femmes qui sont parmi vous se taisent dans les églises ; car il ne leur est point permis d'y parler, mais elles doivent être soumises, comme le dit aussi la loi.
\VS{35}Et si elles veulent s'instruire sur quelque chose, qu'elles interrogent leurs maris à la maison ; car il est honteux à une femme de parler dans l'église.
\VS{36}Est-ce de chez vous que la parole de Dieu est sortie ? Ou est-elle parvenue seulement à vous ?
\VS{37}Si quelqu'un croit être prophète, ou spirituel, qu'il reconnaisse que les choses que je vous écris sont des commandements du Seigneur.
\VS{38}Et si quelqu'un l'ignore, qu'il l'ignore.
\VS{39}C'est pourquoi, mes frères, désirez avec ardeur de prophétiser, et n'empêchez point de parler diverses langues.
\VS{40}Que toutes choses se fassent avec bienséance, et avec ordre.
\Chap{15}
\TextTitle{L'Evangile basé sur la résurrection de Christ}
\VerseOne{}Or, mes frères, je vous rappelle l'Evangile que je vous ai annoncé, que vous avez reçu, et auquel vous vous tenez ferme,
\VS{2}et par lequel vous êtes sauvés, si vous le retenez tel je vous l'ai annoncé ; à moins que vous n'ayez cru en vain. 
\VS{3}Car avant toutes choses, je vous ai donné ce que j'avais aussi reçu, à savoir que Christ est mort pour nos péchés, selon les Ecritures,
\VS{4}et qu'il a été enseveli, et qu'il est ressuscité\FTNT{La Résurrection du Messie. La résurrection de Jésus est un espoir pour tous les êtres humains. Elle est un principe fondamental de la foi chrétienne. Contrairement à toutes les autres religions, la foi chrétienne est la seule qui apporte l'espérance face à la mort. Toutes les autres religions ont été fondées par des hommes, leurs prophètes ou fondateurs sont morts et aucun n'est revenu à la vie. En tant que disciples de Jésus, nous sommes réconfortés par le fait que notre Dieu s'est fait homme, afin de mourir pour nos péchés, et est ressuscité le troisième jour. L'Enfer ne pouvait pas le retenir, et il tient les clés de la mort et de l'Enfer (Ap. 1:18). Voir Jn. 11:25-26. Jésus-Christ est la Résurrection.} le troisième jour, selon les Ecritures ;
\VS{5}et qu'il a été vu de Céphas, et ensuite des douze.
\VS{6}Depuis, il a été vu de plus de cinq cents frères à la fois, dont plusieurs sont encore vivants, et quelques-uns sont morts.
\VS{7}Depuis, il est apparu à Jacques, puis à tous les apôtres.
\VS{8}Après eux tous, il a été vu aussi de moi, comme d'un avorton.
\VS{9}Car je suis le moindre des apôtres, je ne suis pas digne d'être appelé apôtre, parce que j'ai persécuté l'Eglise de Dieu.
\VS{10}Mais par la grâce de Dieu, je suis ce que je suis ; et sa grâce envers moi n'a pas été vaine, mais j'ai travaillé plus qu'eux tous, toutefois non pas moi, mais la grâce de Dieu qui est avec moi.
\VS{11}Soit donc moi, soit eux, nous prêchons ainsi et vous l'avez cru ainsi.
\TextTitle{Valeur de la résurrection de Christ}
\VS{12}Or si on prêche que Christ est ressuscité des morts, comment disent quelques-uns d'entre vous qu'il n'y a point de résurrection des morts ?
\VS{13}Car s'il n'y a point de résurrection des morts, Christ aussi n'est point ressuscité.
\VS{14}Et si Christ n'est pas ressuscité, notre prédication est donc vaine, et votre foi aussi est vaine.
\VS{15}Et même nous sommes de faux témoins de la part de Dieu, car nous avons rendu témoignage à l'égard de Dieu qu'il a ressuscité Christ, tandis qu'il ne l'aurait pas ressuscité, si les morts ne ressuscitent point.
\VS{16}Car si les morts ne ressuscitent point, Christ non plus n'est point ressuscité.
\VS{17}Et si Christ n'est pas ressuscité, votre foi est vaine, et vous êtes encore dans vos péchés,
\VS{18}et par conséquent aussi ceux qui dorment en Christ sont perdus.
\VS{19}Si nous n'avons d'espérance en Christ que pour cette vie seulement, nous sommes les plus misérables de tous les hommes.
\TextTitle{Détails sur les résurrections}
\VS{20}Mais maintenant Christ est ressuscité des morts, il est les prémices de ceux qui dorment.
\VS{21}Car puisque la mort est venue par un seul homme, c'est aussi par un homme qu'est venue la résurrection des morts.
\VS{22}Car comme tous meurent en Adam, de même aussi tous seront vivifiés en Christ.
\VS{23}Mais chacun en son rang, Christ comme prémices, puis ceux qui sont à Christ seront vivifiés lors de son avènement.
\VS{24}Ensuite viendra la fin, quand il aura remis le Royaume à Dieu le Père, après avoir aboli tout empire, toute puissance, et toute force.
\VS{25}Car il faut qu'il règne jusqu'à ce qu'il ait mis tous ses ennemis sous ses pieds\FTNT{Ps. 110:1.}.
\VS{26}L'ennemi qui sera détruit le dernier c'est la mort.
\VS{27}Car Dieu a tout mis sous ses pieds. Mais lorsqu'il dit que tout lui a été soumis, il est évident que celui qui lui a soumis toutes choses est excepté.
\VS{28}Et lorsque toutes choses lui auront été soumises, alors le Fils lui-même sera soumis à celui qui lui a soumis toutes choses, afin que Dieu soit tout en tous.
\VS{29}Autrement que feraient ceux qui se font baptiser pour les morts ? Si les morts ne ressuscitent absolument pas, pourquoi se font-ils baptiser pour les morts ?
\VS{30}Et nous, pourquoi sommes-nous en danger à toute heure ?
\VS{31}Tous les jours je suis exposé à la mort, je l'atteste, par la gloire de notre Seigneur Jésus-Christ.
\VS{32}Si j'ai combattu contre les bêtes à Ephèse dans des vues humaines, quel profit m'en revient-il ? Si les morts ne ressuscitent pas, mangeons et buvons, car demain nous mourrons.
\VS{33}Ne soyez point séduits : Les mauvaises compagnies corrompent les bonnes mœurs.
\VS{34}Réveillez-vous pour vivre justement, et ne péchez point ; car quelques-uns ne connaissent pas Dieu, je le dis à votre honte.
\TextTitle{Les corps de résurrection ou glorifiés}
\VS{35}Mais quelqu'un dira : Comment les morts ressuscitent-ils, et avec quel corps viennent-ils ?
\VS{36}Insensé ! Ce que tu sèmes ne reprend point vie s'il ne meurt pas\FTNT{Jn. 12:24.}.
\VS{37}Et ce que tu sèmes, tu ne sèmes point le corps qui naîtra, c'est un simple grain, de blé peut-être, ou d'une autre semence.
\VS{38}Mais Dieu lui donne le corps comme il veut, et à chacune des semences son propre corps.
\VS{39}Toute chair n'est pas de la même chair, mais autre est la chair des hommes, autre la chair des bêtes, autre celle des poissons, autre celle des oiseaux.
\VS{40}Il y a aussi des corps célestes, et des corps terrestres ; mais autre est l'éclat des corps célestes, et autre celui des corps terrestres.
\VS{41}Autre est l'éclat du soleil, autre l'éclat de la lune, autre l'éclat des étoiles ; même une étoile diffère d'une autre étoile en éclat.
\VS{42}Il en sera aussi de même à la résurrection des morts : Le corps est semé corruptible, il ressuscitera incorruptible.
\VS{43}Il est semé en déshonneur, il ressuscite glorieux ; il est semé en faiblesse, il ressuscite plein de force.
\VS{44}Il est semé corps animal, il ressuscitera corps spirituel. S'il y a un corps animal, il y a aussi un corps spirituel.
\VS{45}Comme aussi il est écrit : Le premier homme, Adam, a été fait en âme vivante\FTNT{Ge. 2:7.}. Le dernier Adam en Esprit vivifiant\FTNT{Jn. 5 : 21 ; Ro. 8 : 11. Jésus-Christ est le dernier Adam. Voir Ph. 2:7 ; 1 T. 3:16.}.
\VS{46}Or ce qui est spirituel n'est pas le premier, mais ce qui est animal ; et puis vient ce qui est spirituel.
\VS{47}Le premier homme, étant de la terre, est tiré de la poussière, mais le second homme, à savoir le Seigneur, est du ciel.
\VS{48}Tel qu'est celui qui est tiré de la poussière, tels aussi sont ceux qui sont tirés de la poussière ; et tel qu'est le céleste, tels aussi sont les célestes.
\VS{49}Et comme nous avons porté l'image de celui qui est tiré de la poussière, nous porterons aussi l'image du céleste.
\VS{50}Voici donc ce que je dis, mes frères, c'est que la chair et le sang ne peuvent hériter le Royaume de Dieu, et que la corruption n'hérite pas l'incorruptibilité.
\TextTitle{Mystère de la résurrection\FTNTT{1 Th. 4:14-17}}
\VS{51}Voici, je vous dis un mystère : Nous ne dormirons pas tous, mais tous nous serons changés,
\VS{52}en un instant, en un clin d'œil, à la dernière trompette\FTNT{Le mot dernier dans ce passage est « eschatos » qui signifie « dernier en temps ou en lieu, dernier dans des séries de lieux, dernier dans une succession dans le temps ». Paul associe le mystère de la résurrection à la dernière trompette. Or, dans le livre d'Apocalypse il n'y a que sept trompettes (Ap. 8 ; 9 et 11:15-19), et c'est à la dernière, c'est-à-dire la septième que le mystère de Dieu s'accomplit.}. Car la trompette sonnera, et les morts ressusciteront incorruptibles, et nous, nous serons changés.
\VS{53}Car il faut que ce corruptible revête l'incorruptibilité, et que ce mortel revête l'immortalité.
\TextTitle{La mort engloutie}
\VS{54}Lorsque ce corruptible aura revêtu l'incorruptibilité, et que ce mortel aura revêtu l'immortalité, alors cette parole de l'Ecriture sera accomplie : La mort a été détruite dans la victoire\FTNT{Es. 25:8.}.
\VS{55}Ô Hadès\FTNT{« Hadès » chez les grecs, et « Pluton » chez les romains, est l'enfer (voir en Mt. 16:18) le dieu des profondeurs de la terre, du monde du néant et du royaume de la mort. « Hadès », ou enfer, et la mort seront jetés dans le lac de feu après le règne de mille ans de Jésus-Christ (Ap. 20:13-14). En Grec biblique, Hadès est associé à Orcus, les régions infernales, endroit sombre, morne, dans les profondeurs même de la terre, l'endroit où se retrouvent les esprits. Le « Hadès » est la demeure des méchants, un endroit très inconfortable. Notre Seigneur Jésus-Christ a vaincu la divinité Hadès et a récupéré les clefs de l'enfer et de la mort. (Ap. 1:18)}, où est ta victoire ? Ô mort, où est ton aiguillon\FTNT{Os. 13:14.} ?
\VS{56}Or l'aiguillon de la mort c'est le péché ; et la puissance du péché c'est la loi.
\VS{57}Mais grâces soient rendues à Dieu qui nous donne la victoire par notre Seigneur Jésus-Christ !
\VS{58}C'est pourquoi, mes frères bien-aimés, soyez fermes, inébranlables, vous appliquant toujours avec un nouveau zèle à l'œuvre du Seigneur, sachant que votre travail ne sera pas inutile dans le Seigneur.
\Chap{16}
\TextTitle{Instructions et salutations de Paul}
\VerseOne{}De plus, concernant la collecte en faveur des saints, faites comme je l'ai ordonné aux églises de Galatie.
\VS{2}C'est que chaque premier jour de la semaine, chacun de vous mette à part chez lui ce qu'il pourra assembler, selon la prospérité que Dieu lui accordera, afin qu'on n'attende pas mon arrivée pour recueillir les dons.
\VS{3}Puis quand je serai arrivé, j'enverrai les personnes que vous aurez approuvées avec des lettres pour porter votre libéralité à Jérusalem.
\VS{4}Et s'il convient que j'y aille moi-même, ils viendront aussi avec moi.
\VS{5}J'irai donc chez vous quand j'aurai traversé la Macédoine, car je traverserai par la Macédoine.
\VS{6}Et peut-être que je séjournerai parmi vous, ou même que j'y passerai l'hiver, afin que vous me conduisiez partout où j'irai.
\VS{7}Car je ne veux pas cette fois vous voir en passant, mais j'espère demeurer quelque temps auprès de vous, si le Seigneur le permet.
\VS{8}Toutefois, je resterai à Ephèse jusqu'à la Pentecôte.
\VS{9}Car une grande porte et un accès efficace m'y est ouverte, et les adversaires sont nombreux.
\VS{10}Si Timothée arrive, faites en sorte qu'il soit en sûreté parmi vous, car il travaille à l'œuvre du Seigneur comme moi-même.
\VS{11}Que personne donc ne le méprise. Ayez soin qu'il soit sans crainte au milieu de vous, afin qu'il vienne vers moi, car je l'attends avec les frères.
\VS{12}Quant à Apollos, notre frère, je l'ai beaucoup exhorté à se rendre chez vous avec les frères, mais il n'a nullement eu la volonté d'y aller maintenant ; il partira quand il en aura l'occasion.
\VS{13}Veillez, soyez fermes dans la foi, agissez courageusement, fortifiez-vous.
\VS{14}Que tout ce que vous faites se fasse avec charité.
\VS{15}Or, mes frères, vous connaissez la famille de Stéphanas, et vous savez qu'elle est les prémices de l'Achaïe, et qu'ils se sont entièrement appliqués au service des saints.
\VS{16}Je vous prie de vous soumettre à de tels hommes, et à chacun qui s'emploient à l'œuvre du Seigneur, et qui travaillent avec nous.
\VS{17}Je me réjouis de l'arrivée de Stéphanas, de Fortunatus, et d'Achaïcus, parce qu'ils ont suppléé à votre absence.
\VS{18}Car ils ont tranquillisé mon esprit et le vôtre. Ayez donc de la considération pour de telles personnes.
\VS{19}Les églises d'Asie vous saluent. Aquilas et Priscille, avec l'église qui est dans leur maison, vous saluent affectueusement en notre Seigneur.
\VS{20}Tous les frères vous saluent. Saluez-vous les uns les autres par un saint baiser.
\VS{21}Je vous salue, moi Paul, de ma propre main.
\VS{22}Si quelqu'un n'aime pas le Seigneur Jésus-Christ, qu'il soit anathème ! Maranatha\FTNT{Maranatha signifie littéralement « Le Seigneur vient ! } !
\VS{23}Que la grâce de notre Seigneur Jésus-Christ soit avec vous !
\VS{24}Mon amour est avec vous tous en Jésus-Christ. Amen.
\PPE{}
\end{multicols}
