\ShortTitle{Ec.}\BookTitle{Ecclésiaste}\BFont
\noindent\hrulefill
{\footnotesize
\textit{
\bigskip
{\centering{}
\\Auteur~: Salomon
\\(Heb.~: Qohelet)
\\Signification~: Prédicateur
\\Thème~: Les raisonnements humains
\\Date de rédaction~: 10\up{ème} siècle av. J.-C.\\}
}
\textit{
\\Ce livre, du fait de sa date de rédaction, est généralement attribué à Salomon en raison de l'allusion faite au premier chapitre et du style adopté. L'Ecclésiaste figure d'ailleurs dans le canon des livres reconnus d'inspiration divine.
\\La problématique centrale du livre est de savoir si la vie vaut la peine d'être vécue ou non. L'auteur y répondit en connaissance de cause car il avait obtenu tout ce que l'homme pouvait désirer~: les richesses, le luxe, la volupté, la sagesse… Sans pour autant incriminer Dieu, il dressa le constat de ce qu'est l'expérience humaine. Selon lui, l'homme vit dans un cycle d'éternels recommencements où tout n'est que poursuite du vent et vanité.\bigskip
}
}
\par\nobreak\noindent\hrulefill
\begin{multicols}{2}
\Chap{1}
\TextTitle{Tout est vanité\FTNTT{Ec. 12:8.}}
\VerseOne{}Les Paroles de l'Ecclésiaste, fils de David, roi de Jérusalem.
\VS{2}Vanité des vanités, dit l'Ecclésiaste, vanité des vanités, tout est vanité.
\TextTitle{Le cycle du temps}
\VS{3}Quel avantage a l'homme de tout son travail auquel il s'occupe sous le soleil~?\FTNT{Ec. 2:22~; Ec. 3:9.}
\VS{4}Une génération passe et une autre génération vient, mais la terre demeure toujours ferme.
\VS{5}Le soleil aussi se lève et le soleil se couche~; il soupire après le lieu d'où il se lève.
\VS{6}Le vent va vers le midi, tourne vers le nord~; il va tournoyant çà et là, et il retourne après ses circuits.
\VS{7}Tous les fleuves vont à la mer, et la mer n'en est point remplie~; les fleuves retournent au lieu d'où ils étaient partis, pour revenir\FTNT{Job 38:8-11~; Ps. 104:9-10.} à la mer. 
\VS{8}Toutes choses travaillent plus que l'homme ne saurait dire~; l'œil n'est jamais rassasié de voir\FTNT{Pr. 27:20.} et l'oreille ne se lasse pas d'entendre 
\VS{9}Ce qui a été, c'est ce qui sera, et ce qui s'est fait, c'est ce qui se fera, il n'y a rien de nouveau sous le soleil\FTNT{Ec. 3:15.}.
\VS{10}Y~a-t-il quelque chose dont on puisse dire~: Regarde cela, il est nouveau~? Il a déjà été dans les siècles qui ont été avant nous.
\VS{11}On ne se souvient plus des choses d'autrefois~; de même on ne se souviendra point des choses à venir et ceux qui viendront n'en auront aucun souvenir. 
\TextTitle{La sagesse des hommes ne comble pas}
\VS{12}Moi l'Ecclésiaste, j'ai été roi sur Israël à Jérusalem.
\VS{13}Et j'ai appliqué mon cœur à rechercher et à sonder par la sagesse tout ce qui se fait sous les cieux~: C'est une occupation désagréable que Dieu a donnée aux hommes, afin qu'ils s'y occupent\FTNT{1 R. 4:30-34~; Ec. 7:25.}.
\VS{14}J'ai vu toutes les œuvres qui se font sous le soleil~; et voici tout est vanité et tourment d'esprit.
\VS{15}Ce qui est courbé ne peut se redresser, et ce qui manque ne peut être compté.
\VS{16}J'ai parlé en mon cœur, disant~: Voici, je suis devenu grand et j'ai surpassé en sagesse tous ceux qui ont été avant moi sur Jérusalem, et mon cœur a vu beaucoup de sagesse et de science.
\VS{17}Et j'ai appliqué mon cœur à connaître la sagesse, et à connaître les sottises et la stupidité~; et j'ai reconnu que cela aussi était un tourment d'esprit.
\VS{18}Car là où il y a beaucoup de sagesse, il y a beaucoup de chagrin, et celui qui augmente sa connaissance, augmente son chagrin.
\Chap{2}
\TextTitle{Les richesses ne comblent pas}
\VerseOne{}J'ai dit en mon cœur~: Allons, que je t'éprouve maintenant par la joie et prends du bon temps. Et voici, c'est encore une vanité\FTNT{Lu. 12:19.}.
\VS{2}J'ai dit touchant le rire~: Il est insensé~! Et touchant la joie~: A quoi sert-elle~?
\VS{3}J'ai recherché en moi-même le moyen de me traiter délicatement, de faire que mon cœur s'accoutume cependant à la sagesse, et qu'il comprenne ce que c'est que la folie, jusqu'à ce que je voie ce qu'il est bon aux hommes de faire sous les cieux, pendant les jours de leur vie. 
\VS{4}Je me suis fait des choses magnifiques~; je me suis bâti des maisons~; je me suis planté des vignes.
\VS{5}Je me suis fait des jardins et des vergers, et j'y plantai des arbres fruitiers de toutes sortes~;
\VS{6}je me suis fait des réservoirs d'eaux pour arroser la forêt où poussent les arbres.
\VS{7}J'ai acquis des hommes et des femmes esclaves~; et j'ai eu des esclaves nés dans ma maison, et j'ai eu plus de gros et de menu bétail que tous ceux qui ont été avant moi dans Jérusalem. 
\VS{8}Je me suis aussi amassé de l'argent et de l'or, et des plus précieux joyaux qui se trouvent chez les rois et dans les provinces\FTNT{1 R. 9:28~; 1 R. 10:10~; 2 Ch. 1:15.}. Je me suis acquis des chanteurs et des chanteuses, et les délices des hommes~; une harmonie d'instruments de musique, même plusieurs harmonies de toutes sortes d'instruments\FTNT{La plupart des bibles ont traduit la deuxième partie de ce verset par le mot«~femme~», or le sens du terme hébreu «~shiddah~» est incertain. Toutefois, le contexte de ce verset montre clairement que Salomon parlait des instruments de musique et des chanteurs et chanteuses qu'il a acquis, et non de ses conquêtes féminines}. 
\VS{9}Je me suis aggrandi et me suis accru plus que tous ceux qui ont été avant moi dans Jérusalem. Et ma sagesse est demeurée avec moi.
\VS{10}Enfin je n'ai rien refusé à mes yeux de tout ce qu'ils ont demandé~; et je n'ai épargné aucune joie à mon cœur~; car mon cœur s'est réjoui de tout mon travail et c'est là tout ce que j'ai eu de tout mon travail.
\VS{11}Mais ayant considéré toutes mes œuvres que mes mains avaient faites, et tout le travail auquel je m'étais occupé en les faisant, voilà tout était vanité et tourment d'esprit~; tellement que l'homme n'a aucun avantage de ce qui est sous le soleil.
\TextTitle{Le sage et l'insensé ont le même sort}
\VS{12}Puis je me suis mis à considérer tant la sagesse, que les sottises, et la folie, (or qui est l'homme qui pourrait suivre le Roi en ce qui a été déjà fait~?).
\VS{13}Et j'ai vu que la sagesse a beaucoup d'avantage sur la folie, comme la lumière a beaucoup d'avantage sur les ténèbres.
\VS{14}Le sage a ses yeux à sa tête, et l'insensé marche dans les ténèbres. Mais j'ai aussi reconnu qu'un même sort leur arrive à tous\FTNT{Ps. 49:11~; Ec. 3:17~; Ec. 9:2.}.
\VS{15}C'est pourquoi j'ai dit en mon cœur~: Il m'arrivera le même sort que l'insensé~; de quoi donc me servira-t-il alors d'avoir été plus sage~? C'est pourquoi j'ai dit en mon cœur, que cela aussi est une vanité. 
\VS{16}Car le souvenir du sage n'est pas plus éternel que celui de l'insensé, parce que ce qui est maintenant, va être oublié dans les jours qui suivent. Le sage meurt aussi bien que l'insensé\FTNT{Ec. 8:10~; Ec. 9:5.}~!
\VS{17}C'est pourquoi j'ai haï cette vie, car les choses qui se sont faites sous le soleil m'ont déplu~; car tout est vanité, et tourment d'esprit.
\VS{18}J'ai aussi haï tout mon travail, auquel je me suis occupé sous le soleil, parce que je le laisserai à l'homme qui sera après moi\FTNT{Ec. 4:8.}.
\VS{19}Et qui sait s'il sera sage ou insensé~? Cependant il sera maître de tout mon travail, auquel je me suis occupé et de ce en quoi j'ai été sage sous le soleil. Cela aussi est une vanité.
\VS{20}C'est pourquoi j'ai fait en sorte que mon cœur perde toute espérance de tout le travail auquel je m'étais occupé sous le soleil.
\VS{21}Car il y a tel homme, dont le travail a été avec sagesse, science, et adresse, qui néanmoins le laisse à celui qui n'y a point travaillé comme étant sa part~; cela aussi est une vanité et un grand mal. 
\VS{22}Car qu'est-ce que l'homme a de tout son travail et du désir de son cœur, dont il souffre sous le soleil~?
\VS{23}Puisque tous ses jours ne sont que douleur, et son occupation n'est que chagrin~; même la nuit son cœur ne repose point. Cela aussi est une vanité\FTNT{Ps. 90:9~; Job. 14:1.}.
\VS{24}N'est-ce donc pas un bien pour l'homme de manger, et de boire, et de faire que son âme jouisse du bien dans son travail~? J'ai vu aussi que cela vient de la main de Dieu\FTNT{Ec. 3:12~; Ec. 3:22~; Ec. 5:18~; Ec. 8:15.}.
\VS{25}Car qui en mangera, qui s'est réjoui plus que moi~?
\VS{26}Parce que Dieu donne à celui qui lui est agréable, de la sagesse, de la science et de la joie~; mais il donne au pécheur de l'occupation à recueillir et à assembler, afin que cela soit donné à celui qui est agréable à Dieu. Cela aussi est une vanité et tourment d'esprit\FTNT{Pr. 13:22~; Pr. 28:8~; Job 27:17}.
\Chap{3}
\TextTitle{Il y a un temps pour toute chose}
\VerseOne{}A toute chose sa saison, et à toute affaire sous les cieux, son temps.
\VS{2}Un temps pour naître et un temps pour mourir~; un temps pour planter et un temps pour arracher ce qui est planté~;
\VS{3}un temps pour tuer et un temps pour guérir~; un temps pour démolir et un temps pour bâtir~;
\VS{4}un temps pour pleurer et un temps pour rire~; un temps pour se lamenter et un temps pour sauter de joie~;
\VS{5}un temps pour jeter des pierres et un temps pour ramasser des pierres~; un temps pour embrasser et un temps pour s'éloigner des embrassements~;
\VS{6}un temps pour chercher et un temps pour perdre~; un temps pour garder et un temps pour jeter~;
\VS{7}un temps pour déchirer et un temps pour coudre~; un temps pour être silencieux et un temps pour parler~;
\VS{8}un temps pour aimer et un temps pour haïr~; un temps pour la guerre et un temps pour la paix.
\TextTitle{Dieu fait toute chose belle en son temps}
\VS{9}Quel avantage celui qui travaille a-t-il de sa peine~?
\VS{10}J'ai considéré cette occupation que Dieu a donnée aux fils des hommes pour s'y appliquer.
\VS{11}Il a fait que toutes choses sont belles en leur temps~; aussi a-t-il mis l'éternité dans leur cœur, sans toutefois que l'homme puisse comprendre du commencement à la fin\FTNT{Ec. 8:17.}l'œuvre que Dieu a faite. 
\VS{12}C'est pourquoi j'ai reconnu qu'il n'y a rien de meilleur aux hommes, que de se réjouir et de se faire du bien pendant leur vie. 
\VS{13}Et même si un homme mange et boit et jouit du bien-être de tout son travail, c'est un don de Dieu\FTNT{Ec. 5:18~; Ec. 8:15~; Ec. 9:7.}.
\VS{14}J'ai reconnu que tout ce que Dieu fait subsiste à toujours, il n'y a rien à y ajouter et rien à en retrancher, et Dieu le fait afin que devant lui, on le craigne.
\VS{15}Ce qui a été, est maintenant~; et ce qui doit être, a déjà été~; et Dieu rappelle ce qui est passé.
\VS{16}J'ai encore vu sous le soleil qu'au lieu établi pour juger, il y a de la méchanceté~; et qu'au lieu établi pour la justice, il y a de la méchanceté.
\VS{17}J'ai dit en mon cœur~: Dieu jugera le juste et le méchant~; car il y a là un temps pour toute chose et pour toute œuvre.
\VS{18}J'ai dit en mon cœur sur l'état des fils de l'homme, que Dieu les éprouverait, et qu'ils verraient qu'ils ne sont que des bêtes.
\VS{19}Car le sort des fils des hommes et le sort de la bête est un même sort~; telle qu'est la mort de l'un, telle est la mort de l'autre. Tous ont un même souffle et la supériorité de l'homme sur la bête est nulle. Car tout est vanité.
\VS{20}Tout va dans un même lieu~; tout a été fait de la poussière, et tout retourne à la poussière\FTNT{Ge. 3:19~; Job 34:15~; Ec. 6:6~; Ec. 12:9.}.
\VS{21}Qui sait si l'esprit des fils de l'homme monte en haut, et si l'esprit de la bête descend en bas dans la terre\FTNT{Les animaux comme les hommes ont une âme et un esprit (Ge. 1:20~; Ez. 1:1-28). A leur mort, leurs esprits quittent leurs corps (Ja. 2:26). L'homme régénéré reçoit le Saint-Esprit, ce qui n'est pas le cas des animaux (Ro. 8:16). Les animaux, tout comme la création tout entière, attendent leur rédemption, car ils ont été soumis à la corruption à cause du péché de l'homme (Ro. 8:19-22).Dans le royaume millénaire, il y aura des animaux (Es. 11:6-9).}~?
\VS{22}J'ai donc vu qu'il n'y a rien de meilleur pour l'homme que de se réjouir de ses œuvres~: C'est là sa part. Car qui le ramènera pour voir ce qui sera après lui~?
\Chap{4}
\TextTitle{Un monde injuste}
\VerseOne{}Puis je me suis mis à regarder toutes les injustices qui se font sous le soleil~; et voici les larmes de ceux à qui on fait tort, et ils n'ont point de consolation. Et la force est du côté de ceux qui leur font tort, et ils n'ont point de consolateur. 
\VS{2}C'est pourquoi j'estime plus les morts qui sont déjà morts, que les vivants qui sont encore vivants\FTNT{Ec. 7:1.}~;
\VS{3}même j'estime celui qui n'a pas encore été, plus heureux que les uns et les autres~; car il n'a pas vu les mauvaises actions qui se font sous le soleil.
\VS{4}Puis j'ai vu que tout travail et tout succès dans le travail n'est que jalousie de l'un à l'égard de l'autre. Cela aussi est une vanité et un tourment d'esprit. 
\VS{5}L'insensé se croise les mains et dévore sa propre chair\FTNT{Pr. 6:10~; Pr. 19:24~; Pr. 24:33~; Pr. 26:15.}.
\VS{6}Mieux vaut le creux de la main pleine avec repos, que les deux mains pleines avec travail et tourment d'esprit\FTNT{Ps. 37:16~; Pr. 15:16-17~; Pr. 16:8.}. 
\VS{7}Puis je me suis mis à regarder une autre vanité sous le soleil. 
\VS{8}C'est qu'il y a tel qui est seul, et qui n'a point de second, qui aussi n'a ni fils ni frère et qui cependant ne met nulle fin à son travail~; même son œil ne voit jamais assez de richesses, et il ne se dit point en lui-même~: Pour qui est-ce que je travaille, et que je prive mon âme du bien~? Cela aussi est une vanité et une fâcheuse occupation\FTNT{Ec. 2:26~; Ps. 39:7~; Lu. 12:20.}.
\VS{9}Deux valent mieux qu'un, car ils ont un meilleur salaire de leur travail.
\VS{10}Même si l'un des deux tombe, l'autre relèvera son compagnon~; mais malheur à celui qui est seul~; parce qu'étant tombé, il n'aura personne pour le relever. 
\VS{11}Si deux aussi couchent ensemble, ils en auront plus de chaleur~; mais celui qui est seul, comment aura-t-il chaud~? 
\VS{12}Et si quelqu'un a le dessus sur l'un ou sur l'autre les deux peuvent lui résister~; et la corde à trois cordons ne se rompt pas rapidement.
\VS{13}Un enfant pauvre et sage vaut mieux qu'un roi vieux et insensé, qui ne sait ce que c'est que d'être averti.
\VS{14}Car tel qui sort de prison pour régner, et de même tel étant né roi, devient pauvre dans son royaume.
\VS{15}J'ai vu tous les vivants qui marchent sous le soleil suivre le fils qui est la seconde personne après le roi, et qui doit être à sa place. 
\VS{16}Il n'y a pas de fin à tout le peuple, à tous ceux qui ont été devant eux~; cependant ceux qui viendront après ne se réjouiront point en lui. Certainement cela aussi est une vanité, et un tourment d'esprit. 
\TextTitle{Le sacrifice des insensés}
\VS{17}Quand tu entres dans la maison de Dieu, prends garde à ton pied, et approche-toi pour écouter, plutôt que pour donner ce que donnent les insensés, car ils ne savent pas qu'ils font mal.
\Chap{5}
\VerseOne{}Ne te précipite point à parler, et que ton cœur ne se hâte point de parler devant Dieu~; car Dieu est au ciel, et toi sur la terre~; c'est pourquoi use de peu de paroles.
\VS{2}Car comme le songe vient de la multitude des occupations~; ainsi la voix des insensés sort de la multitude des paroles\FTNT{Pr. 10:19.}.
\VS{3}Quand tu as fais quelque vœu à Dieu, ne diffère point de l'accomplir~; car il ne prend point de plaisir aux insensés~; accomplis donc le vœu que tu as fait\FTNT{No. 30:3. De. 23:21}.
\VS{4}Il vaut mieux que tu ne fasses point de vœux que d'en faire et de ne pas les accomplir\FTNT{De. 23:21-22.}.
\VS{5}Ne permets pas à ta bouche de faire pécher ta chair, et ne dis point devant le messager de Dieu que c'est un péché involontaire. Pourquoi Yahweh s'irriterait-il de tes paroles, et détruirait-il l'œuvre de tes mains~?
\VS{6}Car comme dans la multitude des songes il y a des vanités, aussi y en a-t-il beaucoup dans la multitude des paroles~; mais crains Dieu\FTNT{Ec. 10:14~; Pr. 10:19.}.
\VS{7}Si tu vois dans la Province qu'on fasse tort au pauvre, et que le droit et la justice y soient violés, ne t'étonne point de cela~; car celui qui est plus élevé que les plus hauts élevés y prend garde, et il y en a de plus élevés qu'eux\FTNT{Es. 3:14-15.}.
\TextTitle{Vanité des richesses}
\VS{8}C'est un avantage pour le pays, un roi qui travaille dans les champs.
\VS{9}Celui qui aime l'argent n'est point rassasié par l'argent\FTNT{Jé. 6:13~; Pr. 22:7~; Pr. 28:16~; Mt. 6:33~; Mt. 7:7-11~; Lu. 12:13-20~; Ac. 20:33~; 2 Co. 9:5~; Ep. 4:19~; Ep. 5:5~; Col. 3:5~; 1 Ti. 6:10~; Hé. 13:5.}, et celui qui aime les richesses, n'en est pas nourri~; cela aussi est une vanité. 
\VS{10}Où il y a beaucoup de bien, là il y a beaucoup de gens qui le mangent~; et quel avantage en revient-il à son maître, sinon qu'il le voit de ses yeux~? 
\VS{11}Le sommeil de celui qui travaille est doux, qu'il mange peu ou beaucoup~; mais le rassasiement du riche ne le laisse point dormir. 
\VS{12}Il y a un mal fâcheux que j'ai vu sous le soleil, c'est que des richesses sont conservées à leurs maîtres afin qu'ils en aient du mal. 
\VS{13}Ces richesses périssent par quelque fâcheux accident~; de sorte qu'on aura engendré un fils et il n'aura rien entre ses mains.
\VS{14}Et comme il est sorti du ventre de sa mère, il s'en retournera nu, s'en allant comme il était venu, et il n'emportera rien de son travail auquel il a employé ses mains\FTNT{1 Ti. 6:7.}.
\VS{15}Et c'est aussi un mal fâcheux, que comme il est venu, il s'en va de même~; et quel avantage a-t-il d'avoir travaillé pour du vent~?
\VS{16}Il mange aussi tous les jours de sa vie dans les ténèbres et se chagrine beaucoup, et son mal va jusqu'à la fureur.
\VS{17}Voilà donc ce que j'ai vu~; que c'est une chose bonne et agréable à l'homme de manger, de boire et de jouir du bien-être de tout son travail qu'il fait sous le soleil, pendant le nombre des jours de vie que Dieu lui a donnés~; car c'est là sa part.
\VS{18}Aussi ce que Dieu donne de richesses et de biens à un homme, quel qu'il soit~; ce dont il le fait maître pour en manger, pour en prendre sa part et pour se réjouir de son travail~; c'est là un don de Dieu. 
\VS{19}Car il ne se souviendra pas beaucoup des jours de sa vie, parce que Dieu lui répond par la joie de son cœur. 
\Chap{6}
\TextTitle{Vanité de la vie de l'homme}
\VerseOne{}Il y a un mal que j'ai vu sous le soleil, et qui est fréquent parmi les hommes.
\VS{2}C'est qu'il y a tel homme à qui Dieu a donné des richesses, des biens et des honneurs, en sorte qu'il ne manque rien pour son âme de tout ce qu'il peut souhaiter. Mais Dieu ne l'en fait pas le maître pour en manger, et un étranger le mangera. Cela est une vanité et un mal fâcheux. 
\VS{3}Quand un homme engendrerait cent fils, qu'il vivrait plusieurs années, en sorte que les jours de ses années se soient fort multipliés, cependant si son âme ne s'est point rassasiée de bien, et même s'il n'a point eu de sépulture, je dis qu'un avorton vaut mieux que lui.
\VS{4}Car il est venu en vain, et s'en va dans les ténèbres, et son nom est couvert de ténèbres~;
\VS{5}Il n'a même point vu le soleil~; il n'a rien connu~; il a plus de repos que cet homme-là\FTNT{Job 3:16.}.
\VS{6}Et s'il vivait deux fois mille ans, et qu'il ne jouit d'aucun bien, tous ne vont-ils pas en un même lieu\FTNT{Ec. 3:20~; Job 3:13-19~; Job 30:23~; Ps. 89:48~; Hé. 9:27.}~?
\VS{7}Tout le travail de l'homme est pour sa bouche, et cependant son âme n'est jamais satisfaite\FTNT{Les richesses de ce monde ne peuvent jamais combler le vide de l'âme. Seul l'amour de Dieu peut réellement inonder nos âmes (Pr. 13:4).}.
\VS{8}Car qu'est-ce que le sage a de plus que l'insensé~? Ou quel avantage a le malheureux qui sait se conduire devant les vivants~?
\VS{9}Mieux vaut ce qu'on voit de ses yeux, que si l'âme fait de grandes recherches. Cela aussi est une vanité, et un tourment d'esprit\FTNT{1 Ti. 6:9.}.
\VS{10}Le nom de ce qui existe a déjà été nommé\FTNT{Ec. 1:9~; Ec. 3:15.}~; et savait-on ce que devait être l'homme, et qu'il ne pourrait plaider avec celui qui est plus fort que lui. 
\VS{11}Quand on a beaucoup, on a beaucoup de vanité. Quel avantage en a l'homme~? 
\VS{12}Car qui est-ce qui connaît ce qui est bon à l'homme dans sa vie, pendant les jours de la vie de sa vanité, lesquels il passe comme une ombre~? Et qui est-ce qui déclarera à l'homme ce qui sera après lui sous le soleil\FTNT{Ps. 144:4~; Ec. 8:7~; Ec. 8:13~; Ec. 10:14~; Ja. 4:13-14.}~?
\Chap{7}
\TextTitle{La sagesse qu'enseigne la vie de l'homme}
\VerseOne{}La renommée vaut mieux que le bon parfum, et le jour de la mort que le jour de la naissance\FTNT{Pr. 22:1.}.
\VS{2}Il vaut mieux aller dans une maison de deuil que d'aller dans une maison de festin~; car c'est là la fin de tout homme, et le vivant met cela dans son cœur.
\VS{3}Il vaut mieux le chagrin que le rire~; car par la tristesse du visage le cœur devient joyeux\FTNT{Ec. 8:1~; 2 Co. 7:10.}.
\VS{4}Le cœur des sages est dans la maison du deuil, mais le cœur des insensés est dans la maison de joie.
\VS{5}Il vaut mieux entendre la réprimande du sage, que d'entendre la chanson des hommes insensés\FTNT{Ps. 141:5~; Pr. 13:18~; Pr. 15:31-32.}.
\VS{6}Car tel qu'est le bruit des épines sous la chaudière, tel est le rire de l'insensé. Cela aussi est une vanité. 
\VS{7}Certainement l'oppression fait perdre le sens au sage~; et le don fait perdre l'entendement. 
\VS{8}Mieux vaut la fin d'une chose, que son commencement. Mieux vaut l'homme qui est d'un esprit patient, que l'homme qui est d'un esprit hautain. 
\VS{9}Ne te précipite point en ton esprit de t'irriter, car l'irritation repose dans le sein des insensés.
\VS{10}Ne dis point~: D'où vient que les jours passés ont été meilleurs que ceux-ci~? Car ce n'est pas par sagesse que tu t'enquiers de cela. 
\VS{11}La sagesse est bonne avec un héritage, et ceux qui voient le soleil reçoivent de l'avantage d'elle.
\VS{12}Car on est à couvert à l'ombre de la sagesse, de même qu'à l'ombre de l'argent~; mais la science a cet avantage, que la sagesse fait vivre celui qui en est doué. 
\VS{13}Regarde l'œuvre de Dieu~: Qui pourra redresser ce qu'il a renversé~?
\VS{14}Au jour du bien, use du bien, et au jour de l'adversité, prends-y garde~; car Dieu a fait l'un vis-à-vis de l'autre, afin que l'homme ne trouve rien à redire après lui. 
\VS{15}J'ai vu tout ceci pendant les jours de ma vanité. Il y a tel juste qui périt dans sa justice, et il y a tel méchant qui prolonge ses jours dans sa méchanceté\FTNT{Ec. 8:14~; Job 21:7-8.}.
\VS{16}Ne te crois pas trop juste, et ne te fais pas plus sage qu'il ne faut~: Pourquoi t'exposer à la ruine\FTNT{Pr. 3:7~; Ro. 12:16.}~?
\VS{17}Ne sois point méchant à l'excès, et ne sois point insensé~: Pourquoi mourrais-tu avant ton temps\FTNT{Ec. 9:16.}~?
\VS{18}Il est bon que tu retiennes ceci, et que tu ne retires point ta main de cela~; car celui qui craint Dieu sort de tout.
\VS{19}La sagesse donne plus de force au sage que dix gouverneurs qui sont dans une ville.
\VS{20}Certainement il n'y a point d'homme juste sur la terre qui agisse toujours bien, et qui ne pèche point\FTNT{Ps. 14:3~; Pr. 20:9~; 2 Ch. 6:36~; Ja. 3:2~; Ro. 3:12~; 1 Jn. 1:8.}.
\VS{21}Ne mets point aussi ton cœur à toutes les paroles qu'on dira, afin que tu n'entendes pas ton serviteur médire de toi. 
\VS{22}Car aussi ton cœur a reconnu plusieurs fois que tu as pareillement mal parlé des autres. 
\VS{23}J'ai essayé tout ceci avec sagesse, et j'ai dit~: J'acquerrai de la sagesse~; mais elle s'est éloignée de moi. 
\VS{24}Ce qui est loin et ce qui est profond, qui le trouvera~?
\VS{25}Moi et mon cœur nous nous sommes agités pour savoir, pour épier, et pour chercher la sagesse et la raison de tout~; et pour connaître la méchanceté de la folie, de la bêtise et des sottises. 
\VS{26}Et j'ai trouvé plus amère que la mort, la femme dont le cœur est un piège et un filet, et dont les mains sont des liens~; celui qui est agréable à Dieu lui échappera~; mais le pécheur sera pris par elle\FTNT{Pr. 5:3-4~; Pr. 6:26~; Pr. 7:13-27~; Pr. 9:13-16~; Pr. 22:14.}.
\VS{27}Vois, dit l'Ecclésiaste, ce que j'ai trouvé en cherchant la raison de toutes choses, l'une après l'autre~;
\VS{28}C'est que jusqu'à présent, mon âme a cherché, mais que je n'ai point trouvé, c'est que j'ai bien trouvé un homme entre mille~; mais pas une femme entre elles toutes. 
\VS{29}Seulement voici ce que j'ai trouvé~; c'est que Dieu a créé l'homme juste~; mais ils ont cherché beaucoup d'inventions.
\Chap{8}
\TextTitle{L'obéissance aux autorités}
\VerseOne{}Qui est tel que le sage~? Et qui sait ce que veulent dire les choses~? La sagesse de l'homme fait briller son visage, et son regard farouche en est changé\FTNT{Ec. 7:3~; Pr. 15:13.}.
\VS{2}Je te le dis~: Prends garde aux ordres du roi, et cela à cause du serment fait à Dieu.
\VS{3}Ne te précipite point de te retirer de devant sa face~; et ne persévère point dans une chose mauvaise~; car il fera tout ce qu'il lui plaira. 
\VS{4}En quelque lieu qu'est la parole du roi, là est la puissance~; et qui lui dira~: Que fais-tu~? 
\VS{5}Celui qui garde le commandement, ne sentira aucun mal~; et le cœur du sage discerne le temps et ce qui est juste. 
\VS{6}Car dans toute affaire il y a un temps et un jugement, autrement mal sur mal tombe sur l'homme. 
\VS{7}Car il ne sait pas ce qui arrivera~; et même qui est-ce qui lui déclarera quand cela arrivera~? 
\VS{8}L'homme n'est point maître de son souffle\FTNT{Le souffle ou l'esprit de l'homme quitte son corps le jour de sa mort (Ps. 39:5~; Ja. 2:26).} pour pouvoir le retenir, il n'a aucune puissance sur le jour de la mort~; il n'y a point de délivrance dans ce combat, et la méchanceté ne délivrera point son maître.
\VS{9}J'ai vu tout cela, et j'ai appliqué mon cœur à toute œuvre qui se fait sous le soleil. Il y a un temps où l'homme domine sur l'autre pour son malheur.
\VS{10}Alors j'ai vu les méchants ensevelis et s'en aller~; et ceux qui avaient agi avec droiture s'en aller loin du lieu saint et être oubliés dans la ville. Cela aussi est une vanité\FTNT{Ec. 2:16~; Ec. 9:5.}.
\VS{11}Parce que la sentence contre les mauvaises œuvres ne s'exécute point promptement, à cause de cela le cœur des fils de l'homme se remplit en eux de l'envie de faire le mal\FTNT{Ec. 12:1.}.
\VS{12}Car bien que le pécheur fasse le mal cent fois, et qu'il y persévère longtemps, je sais aussi qu'il y aura du bonheur pour ceux qui craignent Dieu et qui révèrent sa face\FTNT{Job 22:21~; Pr. 1:33~; Es. 3:10.}.
\VS{13}Mais le bonheur n'est pas pour le méchant, et il ne prolongera point ses jours plus que l'ombre, parce qu'il n'a pas de crainte devant Dieu.
\VS{14}Il y a une vanité qui arrive sur la terre~: C'est qu'il y a des justes auxquels il arrive selon l'œuvre des méchants~; et il y a aussi des méchants auxquels il arrive selon l'œuvre des justes. Je dis que cela aussi est une vanité.
\VS{15}C'est pourquoi j'ai loué la joie, parce qu'il n'y a rien sous le soleil de meilleur à l'homme, que de manger et de boire et de se réjouir~; c'est aussi ce qui lui restera de son travail durant les jours de sa vie, que Dieu lui donne sous le soleil. 
\VS{16}Après avoir appliqué mon cœur à connaître la sagesse, et à regarder les occupations qu'il y a sur la terre, (car l'homme ne donne, ni jour ni nuit, de repos à ses yeux), 
\VS{17}après avoir vu, dis-je, toute l'œuvre de Dieu, j'ai vu que l'homme ne peut pas trouver l'œuvre qui se fait sous le soleil~; il a beau se fatiguer à chercher, il n'est pas capable de trouver~; et même si le sage dit la connaître, il ne peut la trouver.
\Chap{9}
\TextTitle{L'impuissance de la sagesse face à la mort}
\VerseOne{}Certainement j'ai appliqué mon cœur à tout cela~; et pour l'éclaircir à savoir que les justes, les sages et leurs actions sont dans la main de Dieu~; mais les hommes ne connaissent ni l'amour ni la haine de tout ce qui est devant eux. 
\VS{2}Tout arrive également à tous~; un même sort arrive au juste et au méchant~; au bon, au pur et au souillé~; à celui qui sacrifie et à celui qui ne sacrifie point~; le pécheur est comme l'homme de bien~; celui qui jure, comme celui qui craint de jurer. 
\VS{3}C'est un mal parmi tout ce qui se fait sous le soleil, c'est qu'il y a pour tous un même sort~; aussi le cœur des fils de l'homme est-il plein de méchanceté, et la folie est dans leur cœur pendant leur vie~; après cela, ils vont chez les morts. Qui est celui qui voudrait leur être associé\FTNT{Ec. 2:16~; Job 9:22}~?
\VS{4}Il y a de l'espérance pour tous ceux qui sont encore vivants~; et même un chien vivant vaut mieux qu'un lion mort.
\VS{5}Certainement les vivants savent qu'ils mourront, mais les morts ne savent rien, et ne gagnent plus rien~; car leur mémoire est mise en oubli. 
\VS{6}Aussi leur amour, leur haine, et leur envie ont déjà péri, et ils n'auront plus aucune part à tout ce qui se fait sous le soleil.
\VS{7}Va donc, mange ton pain avec joie, et bois gaiement ton vin~; car depuis longtemps Dieu prend plaisir à tes œuvres.
\VS{8}Que tes vêtements soient blancs en tout temps, et que le parfum ne manque point sur ta tête. 
\VS{9}Vis joyeusement tous les jours de ta vie de vanité avec la femme que tu aimes, qui t'a été donnée sous le soleil, tous les jours de ta vanité~; car c'est là ta part dans la vie, au milieu de ton travail que tu fais sous le soleil.
\VS{10}Tout ce que ta main trouve à faire, fais-le selon ton pouvoir~; car dans le scheol, où tu vas, il n'y a ni œuvre, ni pensée, ni connaissance, ni sagesse.
\VS{11}Je me suis tourné ailleurs, et j'ai vu sous le soleil que la course n'est point aux légers, ni la guerre aux héros, ni le pain aux sages, ni la richesse à ceux qui sont intelligents, ni la grâce aux savants~; mais que le temps et les circonstances décident de ce qui arrive à tous.
\VS{12}Car l'homme ne connaît pas son heure, comme les poissons qui sont pris au filet de malheur et les oiseaux qui sont pris au piège~; comme eux, les fils de l'homme sont enlacés au temps du malheur, lorsqu'il tombe subitement sur eux.
\VS{13}J'ai aussi vu cette sagesse sous le soleil, et elle m'a semblé grande.
\VS{14}Il y avait une petite ville, avec peu d'hommes dans son sein~; un roi puissant marcha contre elle, l'investit et bâtit de grands forts contre elle.
\VS{15}Mais il s'y trouvait un homme pauvre et sage qui délivra la ville par sa sagesse. Et personne ne s'est souvenu de cet homme pauvre.
\VS{16}Alors j'ai dit~: La sagesse vaut mieux que la force. Cependant, la sagesse du pauvre est méprisée, et ses paroles ne sont point écoutées.
\VS{17}Les paroles des sages doivent être écoutées plus paisiblement que le cri de celui qui domine parmi les insensés. 
\VS{18}Mieux vaut la sagesse que tous les instruments de guerre~; et un seul homme pécheur détruit beaucoup de bien.
\Chap{10}
\TextTitle{La sagesse vaut mieux que la folie}
\VerseOne{} Les mouches mortes font puer et fermenter les parfums du parfumeur~; et un peu de folie produit le même effet à l'égard de celui qui est estimé pour sa sagesse, et pour sa gloire.
\VS{2}Le cœur du sage est à sa droite, et le cœur de l'insensé est à sa gauche.
\VS{3}Et même quand l'insensé se met en chemin, le sens lui manque~; et il dit de chacun~: Il est insensé. 
\VS{4}Si l'esprit de celui qui domine s'élève contre toi, ne sors point de ta condition~; car la douceur fait pardonner de grandes fautes. 
\VS{5}Il y a un mal que j'ai vu sous le soleil, comme une erreur qui procède du prince~:
\VS{6}C'est que la folie est mise aux plus hauts lieux, et que les riches sont assis dans un lieu bas. 
\VS{7}J'ai vu des serviteurs sur des chevaux, et des princes marchant sur terre comme des serviteurs.
\VS{8}Celui qui creuse la fosse y tombera, et celui qui coupe la haie, le serpent le mordra\FTNT{Ps. 7:15~; Pr. 26:27~; Pr. 28:10.}.
\VS{9}Celui qui remue des pierres hors de leur place, en sera blessé, et celui qui fend du bois se met en danger.
\VS{10}Si le fer est émoussé, et qu'on n'en ait point aiguisé le tranchant, il devra redoubler de force~; mais la sagesse a l'avantage du succès.
\VS{11}Si le serpent mord sans faire du bruit, le médisant ne vaut pas mieux. 
\VS{12}Les paroles de la bouche du sage ne sont que grâce, mais les lèvres de l'insensé le réduisent à néant\FTNT{Pr. 10:21.}.
\VS{13}Le commencement des paroles de sa bouche est folie, et la fin de son discours est une méchante folie.
\VS{14}Or l'insensé multiplie les paroles. L'homme ne sait point ce qui arrivera, et qui lui déclarera ce qui sera après lui~?
\VS{15}Le travail de l'insensé le fatigue, parce qu'il ne sait pas aller à la ville.
\VS{16}Malheur à toi, pays dont le roi est un enfant, et dont les princes mangent dès le matin\FTNT{Es. 3:4.}~!
\VS{17}Que tu es béni, ô pays~! Si ton roi est de race illustre, et si tes gouverneurs mangent au temps convenable, pour leur réfection et non pour se livrer à la débauche~! 
\VS{18}A cause des mains paresseuses, la charpente s'affaisse~; et à cause des mains lâches, la maison a des gouttières.
\VS{19}On fait des pains pour se réjouir et le vin réjouit les vivants et l'argent répond à tout.
\VS{20}Ne maudis point le roi, même dans ta pensée, et ne maudis pas le riche dans la chambre où tu couches~; car l'oiseau du ciel emporterait ta voix, le Baal ailé\FTNT{Baal ailé~: Le terme sémitique «~baal~» (en hébreu ba'al) signifie à l'origine «~possesseur~», «~maître~» ou «~seigneur~». Le Baal ailé était une créature ailée. Utilisé au pluriel, l'expression «~baalim de flèches~» désignait des archers. Les écritures nous parlent de Baal-Zebub (seigneur des mouches), un démon adoré à Ekron, l'une des villes des Philistins (2 R. 1:1-16). Baal-Zebud à donné «~Béelzébul~» dans les Evangiles (Mt. 10:25~; Mt. 12:24~; Mt. 12:27~; Lu. 11:15-19). Ce passage nous enseigne clairement que les démons épient les enfants de Dieu et vont ensuite faire leurs rapports à Satan afin de mieux les attaquer. Ils agissent comme des espions. Ces esprits sont comme des mouches et essayent de s'infiltrer partout.} rapporterait tes paroles.
\Chap{11}
\TextTitle{L'homme travaille en tâtonnant}
\VerseOne{}Jette ton pain à la face des eaux, car avec le temps tu le retrouveras.
\VS{2}Donnes-en une part à sept et même à huit, car tu ne sais point quel mal viendra sur la terre.
\VS{3}Quand les nuages sont pleins, ils répandent la pluie sur la terre~; et quand un arbre tombe, au sud ou au nord, il reste à la place où il est tombé.
\VS{4}Celui qui prend garde au vent, ne sèmera point~; et celui qui regarde les nuées, ne moissonnera point. 
\VS{5}Comme tu ne sais point quel est le chemin du vent, ni comment se forment les os dans le ventre de celle qui est enceinte, ainsi tu ne connais pas l'œuvre de Dieu qui fait tout\FTNT{Ceux qui sont nés d'en-haut sont insaisissables comme le vent (Jn. 3:8).}.
\VS{6} Sème ta semence dès le matin, et ne laisse pas reposer tes mains le soir~; car tu ne sais point lequel sera le meilleur, ceci ou cela~; et si tous deux seront pareillement bons.
\VS{7}Il est vrai que la lumière est douce, et qu'il est agréable aux yeux de voir le soleil.
\VS{8}Mais si un homme vit de nombreuses années, qu'il se réjouisse, et qu'il se souvienne des jours de ténèbres qui seront en grand nombre, tout ce qui lui arrivera est vanité.
\Chap{12}
\TextTitle{Message à la jeunesse}
\VerseOne{}Jeune homme, réjouis-toi dans ton jeune âge, et que ton cœur te rende gai aux jours de ta jeunesse, et marche comme ton cœur te mène, et selon le regard de tes yeux~; mais sache que pour toutes ces choses Dieu t'amènera en jugement. 
\VS{2}Ôte le chagrin de ton cœur, et éloigne de toi le mal~; car le jeune âge et l'adolescence ne sont que vanité. 
\VS{3}Mais souviens-toi de ton Créateur pendant les jours de ta jeunesse, avant que les jours mauvais arrivent et que viennent les années où tu diras~: Je n'y prends point de plaisir~;
\VS{4}avant que le soleil et la lumière, la lune et les étoiles s'obscurcissent, et que les nuages reviennent après la pluie.
\VS{5}Lorsque les gardes de la maison tremblent\FTNT{«~Ceux qui gardent la maison~» représentent les mains.}, et que les hommes forts\FTNT{«~Les hommes forts~» sont les jambes.} se courbent, et que celles qui moulent\FTNT{Les dents sont celles qui moulent.} cessent de travailler parce qu'elles sont diminuées, et quand ceux qui regardent par les fenêtres\FTNT{Les yeux sont ceux qui regardent par les fenêtres.} sont obscurcis.
\VS{6}Et quand les deux battants de la porte\FTNT{Les oreilles sont les deux battants de la porte.} se ferment sur la rue quand s'abaisse le bruit de la meule, quand on se lève au chant de l'oiseau, et que toutes les chanteuses s'affaiblissent.
\VS{7}Quand aussi on craint ce qui est élevé, et qu'on tremble en chemin, quand l'amandier fleurit, et quand les cigales deviennent pesantes, et que l'appétit s'en ira, car l'homme s'en va vers sa maison éternelle\FTNT{La maison éternelle c'est la Nouvelle Jérusalem pour les chrétiens (Ap. 21.) et pour les païens le lac de feu (Ap. 20:11-15).}, et ceux qui pleurent font le tour des rues.
\VS{8}Avant que la corde d'argent\FTNT{Cette corde est comme le cordon ombilical, elle lie l'âme au corps. Lors de la mort, la corde d'argent est coupée.} se détache, que le vase d'or\FTNT{Le corps humain est comme un vase ou une tente qui renferme son esprit. Comme l'argile dans la main du potier, ainsi est l'homme dans celle de Dieu. Avec cette argile, il décide souverainement de fabriquer de la même masse un vase d'honneur et un autre pour un usage vil (Jé. 18:4-6~; Ro. 9:21~; 2 Ti. 2:20-21).} se brise, que la cruche se rompe sur la source, que la roue s'écrase sur la citerne~;
\VS{9}avant que la poussière retourne dans la terre, comme elle y avait été, et que l'esprit retourne à Dieu qui l'a donné.
\TextTitle{Conclusion}
\VS{10}Vanité des vanités, dit l'Ecclésiaste, tout est vanité.
\VS{11}Plus l'Ecclésiaste a été sage, plus il a enseigné la science au peuple~; il a fait entendre, il a recherché et mis en ordre plusieurs graves sentences. 
\VS{12}L'Ecclésiaste a cherché pour trouver des discours agréables~; mais ce qui en a été écrit ici, est la droiture même~; ce sont des paroles de vérité. 
\VS{13}Les paroles des sages sont comme des aiguillons, et les maîtres qui en ont fait des recueils, sont comme des clous plantés, et ces choses ont été données par un maître.
\VS{14}Mon fils, garde-toi de ce qui est au-delà de ceci~; car il n'y a point de fin à faire plusieurs livres, et tant d'étude n'est que travail qu'on se donne. 
\VS{15}Le but de tout le discours qui a été entendu c'est~: Crains Dieu, et garde ses commandements~; car c'est là le tout de l'homme.
\VS{16}Parce que Dieu amènera toute œuvre en jugement, au sujet de tout ce qui est caché, soit bien, soit mal.
\PPE{}
\end{multicols}
