\ShortTitle{Ecclésiaste}\BookTitle{Ecclésiaste}\BFont
\noindent\hrulefill
{\footnotesize
\textit{
\bigskip
{\centering{}
\\(Qohelet)
\\Signifie : Prédicateur
\\Thème : Les raisonnements humains
\\Auteur : Incertain (peut-être Salomon)
\\Date de rédaction : 10ème siècle av. J.-C.\\}
}
%\bigskip
\textit{
\\Même si  l’auteur de ce livre est incertain du fait de sa date de rédaction, ces écrits sont généralement attribués à Salomon en raison de l’allusion faite au premier chapitre et du style adopté. L’Ecclésiaste figure d’ailleurs dans le canon des livres reconnus d’inspiration divine.
%\bigskip
\\La problématique centrale du livre est de savoir si la vie vaut la peine d’être vécu ou non. L’auteur y répondit en connaissance de cause car il avait obtenu tout ce que l’homme pouvait désirer : les richesses, le luxe, la volupté, la sagesse… Sans pour autant incriminer Dieu, il dressa le constat de ce qu’est l’expérience humaine. Selon lui, l’homme vit dans un cycle d’éternels recommencements où tout n’est que poursuite du vent et vanité. 
%\bigskip
}
}
\par\nobreak\noindent\hrulefill
\begin{multicols}{2}
\TextTitle{[Tout est vanité]
\\(Ec. 12:8)}
\Chap{1}
\VerseOne{}Paroles de l'Ecclésiaste, fils de David, roi de Jérusalem.
\VS{2}Vanité des vanités, dit l'Ecclésiaste, vanité des vanités, tout est vanité.
\TextTitle{[Le cycle des temps]}
\VS{3}Quel avantage a l'homme de toute la peine qu’il se donne sous le soleil ?\FTNT{Ec. 2:22 ; Ec. 3:9.}
\VS{4}Une génération s’en va, et une autre génération vient, mais la terre subsiste toujours.
\VS{5}Le soleil se lève, le soleil se couche ; il soupire après le lieu d'où il se lève.
\VS{6}Le vent va vers le sud, tourne vers le nord ; puis il tourne encore, et reprend les mêmes circuits.
\VS{7}Tous les fleuves vont à la mer, et la mer n’est point remplie ; ils continuent à aller vers le lieu où ils se dirigent\FTNT{Job. 38:8-11 ; Ps. 104:9-10.}.
\VS{8}Toutes choses sont en travail au-delà de ce que l'homme peut dire ; l'œil ne se rassasie jamais de voir, et l'oreille ne se lasse pas d'entendre\FTNT{Pr. 27:20.}.
\VS{9}Ce qui a été, c'est ce qui sera, et ce qui s’est fait, c’est ce qui se fera, il n'y a rien de nouveau sous le soleil\FTNT{Ec. 3:15.}.
\VS{10}Y a-t-il une chose dont on dise : Vois ceci, c’est nouveau ? Cette chose existait déjà dans les siècles qui nous ont précédés.
\VS{11}On ne se souvient pas de ce qui est ancien ; et ce qui arrivera dans la suite ne laissera pas de souvenir chez ceux qui viendront plus tard.
\TextTitle{[La sagesse des hommes ne comble pas]}
\VS{12}Moi l'Ecclésiaste, j'ai été roi sur Israël à Jérusalem.
\VS{13}J'ai appliqué mon cœur à rechercher et à sonder par la sagesse tout ce qui se fait sous les cieux : C’est une occupation déplaisante à laquelle Dieu soumet les fils de l’homme pour s’occuper\FTNT{Ec. 2:12 ; Ec. 7:25 ; 1 R. 4:30-34 ; }.
\VS{14}J'ai vu tout le travail qui se fait sous le soleil ; et voici tout est vanité et poursuite du vent.
\VS{15}Ce qui est tordu ne peut se redresser, et ce qui manque ne peut être compté.
\VS{16}J'ai parlé en mon cœur, disant : Voici, je suis devenu grand et j’ai surpassé en sagesse tous ceux qui ont été avant moi sur Jérusalem, et mon cœur a vu beaucoup de sagesse et de science.
\VS{17}J'ai appliqué mon cœur à connaître la sagesse, et à connaître la sottise et la stupidité ; j'ai reconnu que cela aussi c’est la poursuite du vent.
\VS{18}Car avec beaucoup de sagesse, on a beaucoup de chagrin, et celui qui augmente sa connaissance, augmente sa douleur.
\TextTitle{[Les richesses ne comblent pas]}
\Chap{2}
\VerseOne{}J'ai dit en mon cœur : Allons, je t'éprouverai par la joie, tu rechercheras le bonheur. Et voici, c’est encore une vanité\FTNT{Lu. 12:19.}.
\VS{2}J'ai dit au rire : Tu es déraison ! Et à la joie : Que fait-elle ?
\VS{3}J'ai recherché en mon cœur le moyen de livrer ma chair au vin, tandis que mon cœur me conduirait avec sagesse, et de saisir la folie jusqu’à ce que je voie ce qu'il est bon aux fils de l’homme de faire sous les cieux, pendant le nombre des jours de leur vie.
\VS{4}J’ai fait de grands ouvrages : Je me suis bâti des maisons ; je me suis planté des vignes ;
\VS{5}je me fis des jardins et des vergers, et j'y plantai des arbres fruitiers de toutes sortes ;
\VS{6}je me créai des étangs, pour arroser la forêt où poussent les arbres.
\VS{7}J’achetai des serviteurs et des servantes, et j'eus leurs fils nés dans ma maison ; et je possédai plus de troupeaux de bœufs et de brebis que tous ceux qui étaient avant moi dans Jérusalem.
\VS{8}Je m’amassai de l'argent et de l'or, et les richesses des rois et des provinces. Je me procurai des chanteurs et des chanteuses, et les délices des fils de l’homme, des femmes en grand nombre\FTNT{1R. 9:28 ; 1 R. 10:10 ; 2Ch. 1:15.}.
\VS{9}Je devins grand, plus que tous ceux qui ont été avant moi dans Jérusalem. Et même ma sagesse resta avec moi.
\VS{10}Je ne refusai rien à mes yeux de tout ce qu'ils avaient désiré ; je n'ai refusé aucune joie à mon cœur ; car mon cœur s'est réjoui de tout mon travail, c'est la part que j'ai eue de tout mon travail.
\VS{11}Puis, j’ai considéré toutes les œuvres que mes mains avaient faites, et toute la peine que j’avais prise à les faire ; voici tout est vanité et poursuite du vent, il n’y a aucun avantage sous le soleil.
\TextTitle{[Le sage et l'insensé ont le même sort]}
\VS{12}Puis je me suis tourné pour considérer la sagesse, la sottise et la stupidité. Car que fera l’homme qui viendra après le roi ? Ce qu’on a déjà fait.
\VS{13}Et j'ai vu que la sagesse a de l’avantage sur la stupidité, comme la lumière a de l’avantage sur les ténèbres.
\VS{14}Le sage a ses yeux à sa tête, et l’insensé marche dans les ténèbres. Mais j'ai aussi reconnu qu'un même sort leur arrive à tous\FTNT{Ec. 3:17 ; Ec. 9:2 ; Ps. 49:11-12 ; Ps. 49:20.}.
\VS{15}Et j'ai dit en mon cœur : Il m'arrivera le même sort que l’insensé ; pourquoi donc ai-je été plus sage ? Et j'ai dit en mon cœur que cela est aussi une vanité.
\VS{16}Car le souvenir du sage n’est pas plus éternel que celui de l’insensé, puisque déjà les jours qui suivent, tout est oublié. Le sage meurt aussi bien que l’insensé\FTNT{Ec. 8:10 ; Ec. 9:5.} !
\VS{17}Et j'ai haï la vie, car les choses qui se font sous le soleil m'ont déplu, car tout est vanité et poursuite du vent.
\VS{18}J'ai haï tout le travail que j’ai fait sous le soleil, parce que je le laisserai à l'homme qui sera après moi\FTNT{Ec. 4:8.}.
\VS{19}Et qui sait s'il sera sage ou insensé ? Cependant il sera maître de tout mon travail, du travail de ma sagesse sous le soleil. Cela aussi est une vanité.
\VS{20}Et j’en suis venu à abandonner mon cœur au désespoir, à cause de tout le travail que j’ai fait sous le soleil.
\VS{21}Car il y a tel homme, qui a travaillé avec sagesse et connaissance et avec adresse, et il donne son héritage à un homme qui n'y a point travaillé. C’est encore là une vanité et un grand mal.
\VS{22}Que reste-t-il à l'homme de tout son travail et du désir de son cœur, dont il souffre sous le soleil ?
\VS{23}Tous ses jours ne sont que douleur, et son occupation n’est que chagrin ; même la nuit son cœur ne repose point. C’est encore là une vanité\FTNT{Job. 14:1 ; Ps. 90:9.}.
\VS{24}Il n’y a de bonheur pour l'homme qu’à manger et à boire, et à faire jouir son âme du bien-être de son travail ; mais j'ai vu que cela aussi vient de la main de Dieu\FTNT{Ec. 3:12 ; Ec. 3:22 ; E. 5:18 ; Ec. 8:15.}.
\VS{25}Qui, en effet, peut manger et jouir, en dehors de moi ?
\VS{26}Car il donne à l’homme qui lui est agréable, la sagesse, la connaissance et la joie ; mais il donne au pécheur la tâche de recueillir et d’amasser, afin de donner à celui qui est agréable devant Dieu. C’est encore une vanité et la poursuite du vent\FTNT{Job. 27:17 ; Pr. 13:22 ; Pr. 28:8.}.
\TextTitle{[Il y a un temps pour toute chose]}
\Chap{3}
\VerseOne{}Il y a un temps pour tout, un temps pour toute chose sous les cieux.
\VS{2}Un temps pour naître, et un temps pour mourir ; un temps pour planter, et un temps pour arracher ce qui est planté ;
\VS{3}un temps pour tuer, et un temps pour guérir ; un temps pour abattre, et un temps pour bâtir ;
\VS{4}un temps pour pleurer, et un temps pour rire ; un temps pour se lamenter, et un temps pour sauter de joie ;
\VS{5}un temps pour jeter des pierres, et un temps pour ramasser des pierres ; un temps pour embrasser, et un temps pour s'éloigner des embrassements ;
\VS{6}un temps pour chercher, et un temps pour perdre ; un temps pour garder, et un temps pour jeter ;
\VS{7}un temps pour déchirer, et un temps pour coudre ; un temps pour être silencieux, et un temps pour parler ;
\VS{8}un temps pour aimer, et un temps pour haïr ; un temps pour la guerre, et un temps pour la paix.
\TextTitle{[Dieu a tout fait]}
\VS{9}Quel avantage celui qui travaille a-t-il de sa peine ?
\VS{10}J'ai vu l’occupation à laquelle Dieu soumet les fils de l’homme pour les occuper.
\VS{11}Il fait toute chose belle en son temps ; même il a mis la pensée de l’éternité dans leur cœur, bien que l'homme ne puisse pas comprendre l’œuvre que Dieu fait, depuis le commencement jusqu’à la fin\FTNT{Ec. 8:17.}.
\VS{12}J'ai reconnu qu'il n'y a rien de bon pour eux, que de se réjouir et de se donner du bien-être pendant leur vie.
\VS{13}Et même si un homme mange et boit et jouit du bien-être de tout son travail, c'est un don de Dieu\FTNT{Ec. 5:18 ;Ec. 8:15 ; Ec. 9:7.}.
\VS{14}J'ai reconnu que tout ce que Dieu fait subsiste à toujours, il n’y a rien à y ajouter et rien à en retrancher, et Dieu le fait afin que devant lui, on le craigne.
\VS{15}Ce qui est a déjà été, et ce qui sera a déjà été, et Dieu ramène ce qui est passé.
\VS{16}J'ai encore vu sous le soleil qu'au lieu établi pour juger, il y a de la méchanceté ; et qu'au lieu établi pour la justice, il y a de la méchanceté.
\VS{17}J'ai dit en mon cœur : Dieu jugera le juste et le méchant ; car il y a là un temps pour toute chose et pour toute œuvre.
\VS{18}J'ai dit en mon cœur au sujet des fils de l’homme, que Dieu les éprouverait, et qu'ils verraient qu'ils ne sont que des bêtes.
\VS{19}Car le sort des fils de l’homme, et le sort des bêtes sont un même sort : Comme meurt l'un, ainsi meurt l'autre, ils ont tous un même souffle, et l'homme n'a pas la supériorité sur la bête ; car tout est vanité.
\VS{20}Tout va dans un même lieu ; tout a été fait de la poussière, et tout retourne à la poussière\FTNT{Ec. 6:6 ; Ec. 12:9 ; Ge. 3:19 ; Job. 34:15.}.
\VS{21}Qui sait si l’esprit des fils de l’homme monte en haut, et si l’esprit de la bête descend en bas dans la terre\FTNT{Les animaux comme les hommes ont une âme et un esprit (Ge. 1:20 ; Ez. 1:1-28). A leur mort, leurs esprits quittent leurs corps (Ja. 2:26). L’homme régénéré reçoit le Saint-Esprit, ce qui n’est pas le cas des animaux (Ro. 8:16). Les animaux, tout comme la création tout entière, attendent leur rédemption, car ils ont été soumis à la corruption à cause du péché de l’homme (Ro. 8: 19-22).Dans le royaume millénaire, il y aura des animaux  (Es. 11:6-9).}?
\VS{22}J'ai donc vu qu'il n'y a rien de meilleur pour l'homme que de se réjouir de ses œuvres : C'est là sa part. Car qui le ramènera pour voir ce qui sera après lui ?
\TextTitle{[Les opressions et les injustices]}
\Chap{4}
\VerseOne{}Puis je me suis mis à considérer toutes les oppressions qui se font sous le soleil ; et voici les opprimés sont dans les larmes, personne ne les console ! La main de leurs oppresseurs est puissante, et personne ne les console !
\VS{2}C'est pourquoi j'estime les morts qui sont déjà morts plus heureux que les vivants qui sont encore vivants\FTNT{Ec. 7:1.},
\VS{3}même plus heureux que les uns et les autres celui qui n’a pas encore été et qui n'a pas vu les mauvaises actions qui se font sous le soleil.
\VS{4}J'ai vu que tout travail et toute habileté dans le travail n’est que jalousie de l’homme à l’égard de son prochain. C’est encore une vanité et poursuite du vent.
\VS{5}L’insensé se croise les mains et consume sa propre chair\FTNT{Pr. 6:10 ; Pr. 19:24 ; Pr. 24:33 ; Pr. 26:15.}.
\VS{6}Mieux vaut le creux de la main pleine avec repos, que les deux mains pleines avec travail et poursuite du vent\FTNT{Ps. 37:16 ; Pr. 15:16-17 ; Pr. 16:8.}.
\VS{7}Je me suis mis à regarder une autre vanité sous le soleil.
\VS{8}Tel homme est seul, et n'a point de second, il n'a ni fils ni frère, et cependant son travail n’a point de fin et ses yeux ne sont jamais rassasiés de richesses. Pour qui est-ce que je travaille et que je prive mon âme du bonheur ? C’est encore là une vanité et une désagréable occupation\FTNT{Ec. 2:26 ; Ps. 39:6 ; Jé. 17:11 ; Lu. 12:20.}.
\VS{9}Deux valent mieux qu'un, car ils ont un bon salaire de leur travail.
\VS{10}Car s’ils tombent, l'un relève son compagnon ; mais malheur à celui qui est seul et qui tombe et qui n'a personne pour le relever.
\VS{11}De même, si deux couchent ensemble, ils auront chaud ; mais celui qui est seul comment aura-t-il chaud ?
\VS{12}Et si quelqu'un est plus fort qu’un seul, les deux peuvent lui résister ; et la corde à trois cordons ne se rompt pas rapidement.
\VS{13}Mieux vaut un enfant pauvre et sage qu'un roi vieux et insensé qui ne comprend pas l’instruction.
\VS{14}Car tel sort de prison pour régner, et de même tel étant né roi, devient pauvre dans son royaume.
\VS{15}J'ai vu tous les vivants qui marchent sous le soleil entourer l’enfant qui devait succéder au roi et régner à sa place.
\VS{16}Il n’y avait point de fin à tout ce peuple, à tous ceux à la tête desquels il était. Cependant, ceux qui viendront après ne se réjouiront point à son sujet. C’est encore là une vanité et la poursuite du vent.
\TextTitle{[Le sacrifice des insensés]}
\VS{17}Lorsque tu entres dans la maison de Dieu, prends garde à ton pied, et approche-toi pour écouter, plutôt que pour offrir le sacrifice des insensés, car ils ne savent pas qu'ils font mal.
\Chap{5}
\VerseOne{}Ne te précipite pas d’ouvrir la bouche, et que ton cœur ne se hâte point de prononcer une parole devant Dieu ; car Dieu est au ciel, et toi sur la terre : Que tes paroles soient peu nombreuses.
\VS{2}Car les songes naissent de la multitude des occupations, et la voix des insensés de la multitude des paroles\FTNT{Pr. 10:19.}.
\VS{3}Lorsque tu as fait un vœu à Dieu, ne tarde point à l'accomplir, car il ne prend point plaisir aux insensés : Accomplis le vœu que tu as fait\FTNT{No. 30:3. De. 23:21}.
\VS{4}Il vaut mieux que tu ne fasses point de vœux que d'en faire et de ne pas les accomplir\FTNT{De. 23:21-22.}.
\VS{5}Ne permets pas à ta bouche de faire pécher ta chair, et ne dis point devant le messager de Dieu que c'est un péché involontaire. Pourquoi Yahweh s’irriterait-il de tes paroles, et détruirait-il l'œuvre de tes mains ?
\VS{6}Car s’il y a des vanités dans la multitude des songes, il y en a aussi dans beaucoup de paroles ; mais crains Dieu\FTNT{Ec. 10:14 ; Pr. 10:19.}.
\VS{7}Si tu vois dans une province le pauvre opprimé et le droit et la justice violés, ne t’en étonne point ; car un homme élevé est sous l'observation d’un autre plus élevé, et au-dessus d’eux il en est de plus élevé encore\FTNT{Es. 3:14-15.}.
\TextTitle{[La vanité des richesses]}
\VS{8}C’est un avantage pour le pays, un roi qui travaille dans les champs.
\VS{9}Celui qui aime l'argent n'est point rassasié par l'argent\FTNT{Pr. 22:7 ; Pr. 28:16; Jé. 6:13; Mt. 6:33 ; Mt. 7:22; Lu. 12:1 ; Lu. 12:15 ; Ac. 20:33; 2 Co. 9:5; Ep. 4:19 ; Ep. 5:5. Col.  3:5 ; 1 Ti. 6:10 ; Hé. 13:5.}, et celui qui aime les richesses n’en profite pas. C’est encore là une vanité.
\VS{10}Où il y a beaucoup de bien, il y a beaucoup de gens qui le mangent ; et quel avantage en revient-il à son possesseur, sinon qu'il le voit de ses yeux ?
\VS{11}Le sommeil de celui qui travaille est doux, qu'il mange peu ou beaucoup ; mais l’abondance du riche ne le laisse point dormir.
\VS{12}Il y a un mal douloureux que j'ai vu sous le soleil : Des richesses conservées, pour son malheur, par celui qui les possède.
\VS{13}Ces richesses périssent par quelque fâcheux accident ; s’il a engendré un fils, il n'aura rien entre ses mains.
\VS{14}Comme il est sorti du ventre de sa mère, il s'en retournera nu, s'en allant comme il était venu, et il n'emportera rien de son travail qu’il puisse prendre dans sa main\FTNT{1Ti. 6:7.}.
\VS{15}C'est aussi un mal douloureux. Il s’en va comme il était venu ; et quel avantage a-t-il d'avoir travaillé pour du vent ?
\VS{16}De plus, tous les jours de sa vie il mange dans les ténèbres, et il a beaucoup de chagrins, des maux et des irritations.
\VS{17}Voici ce que j'ai vu : C'est pour l’homme une bonne et belle chose de manger, de boire, et de jouir du bien-être de tout son travail qu'il fait sous le soleil, pendant le nombre des jours de vie que Dieu lui a donnés ; car c'est là sa part.
\VS{18}Mais, si Dieu a donné à un homme des richesses et des biens, s’il lui a donné le pouvoir d’en manger, d’en prendre sa part, et de se réjouir de son travail, c'est là un don de Dieu.
\VS{19}Car il ne se souviendra pas beaucoup des jours de sa vie, parce que Dieu lui répond par la joie de son cœur.
\TextTitle{[La vie d'un homme, une vie de vanité]}
\Chap{6}
\VerseOne{}Il est un mal que j'ai vu sous le soleil, et qui est fréquent parmi les hommes.
\VS{2}C'est qu'il y a tel homme à qui Dieu a donné des richesses, des biens et de la gloire, et qui ne manque pour son âme rien de ce qu'il désire, mais Dieu ne lui laisse pas le pouvoir d’en manger, car c’est un étranger qui en jouira. C’est là encore une vanité et un mal douloureux.
\VS{3}Quand un homme engendrerait cent fils, vivrait un grand nombre d’années, et que les jours de ses années se multiplieraient, si son âme ne s'est point rassasiée de bonheur, et s'il n'a point de sépulture, je dis qu'un avorton est plus heureux que lui.
\VS{4}Car il est venu en vain, il s'en va dans les ténèbres, et son nom est couvert de ténèbres ;
\VS{5}il n’a point vu, il n’a point connu le soleil ; il a plus de repos que cet homme\FTNT{Job. 3:16.}.
\VS{6}Et quand celui-ci vivrait deux fois mille ans, sans avoir vu le bonheur, tous ne vont-ils pas dans un même lieu\FTNT{Ec. 3:20 ; Job. 3:13-19 ; Job. 30:23 ; Ps. 89:48 ; Hé. 9:27.} ?
\VS{7}Tout le travail de l'homme est pour sa bouche, et cependant son âme n’est jamais satisfaite\FTNT{Les richesses de ce monde ne peuvent jamais combler le vide de l’âme. Seul l’amour de Dieu peut réellement inonder nos âmes (Pr. 13:4).}.
\VS{8}Car quel avantage le sage a-t-il sur l’insensé ? Quel avantage a le malheureux qui sait se conduire devant les vivants ?
\VS{9}Ce que voient les yeux est préférable à l’agitation de l’âme ; c’est encore là une vanité et la poursuite du vent\FTNT{1 Tim. 6:9.}.
\VS{10}Ce qui existe a déjà été appelé par son nom\FTNT{Ec.1:9 ;Ec. 3:15.} ; et on sait que celui qui est homme ne peut contester avec celui qui est plus puissant que lui.
\VS{11}S’il y a beaucoup de choses, il y a beaucoup de vanité : Quel avantage en revient-il à l'homme ?
\VS{12}Car qui sait ce qui est bon pour l'homme dans la vie, pendant le nombre des jours de sa vie de vanité, qu’il passe comme une ombre ? Et qui peut dire à l'homme ce qui sera après lui sous le soleil\FTNT{Ec. 8:13 ; Ps. 144:4 ; Ja. 4:13-14 ; Ec. 8:7 ;Ec. 10:14.}?
\TextTitle{[La sagesse qu'enseigne la vie de l'homme]}
\Chap{7}
\VerseOne{}Une bonne réputation vaut mieux que le bon parfum, et le jour de la mort que le jour de la naissance\FTNT{Pr. 22:1.}.
\VS{2}Mieux vaut aller dans une maison de deuil que d'aller dans une maison de festin ; car là est la fin de tout homme, et celui qui vit prend la chose à cœur.
\VS{3}Mieux vaut le chagrin que le rire ; car par la tristesse du visage le cœur devient joyeux\FTNT{Ec. 8:1 ; 2 Co. 7:10.}.
\VS{4}Le cœur des sages est dans la maison du deuil, mais le cœur des insensés dans la maison de joie.
\VS{5}Mieux vaut entendre la réprimande du sage que d'entendre le chant des hommes insensés\FTNT{Ps. 141:5 ; Pr. 13:18 ; Pr. 15:31-32.}.
\VS{6}Car comme le bruit des épines sous la chaudière, ainsi est le rire de l’insensé. C’est encore là une vanité.
\VS{7}L'oppression rend insensé le sage, et les présents détruisent le cœur.
\VS{8}Mieux vaut la fin d'une chose que son commencement ; mieux vaut un esprit patient qu'un esprit hautain.
\VS{9}Ne te précipite point en ton esprit de t’irriter, car l’irritation repose dans le sein des insensés.
\VS{10}Ne dis point : D'où vient que les jours passés étaient meilleurs que ceux-ci ? Car ce n’est point par sagesse que tu demandes cela.
\VS{11}La sagesse est aussi bonne qu’un héritage, et même plus pour ceux qui voient le soleil.
\VS{12}Car à l'ombre de la sagesse on est abrité comme à l'ombre de l'argent ; mais l’avantage de la connaissance c’est que la sagesse fait vivre celui qui la possède.
\VS{13}Regarde l'œuvre de Dieu : Qui pourra redresser ce qu'il a renversé ?
\VS{14}Au jour du bonheur, sois heureux, et au jour de l'adversité prends garde : Dieu a fait l'un comme l'autre afin que l'homme ne découvre rien de ce qui sera après lui.
\VS{15}J'ai vu tout ceci pendant les jours de ma vanité. Il y a tel juste qui périt dans sa justice, et il y a tel méchant qui prolonge ses jours dans sa méchanceté\FTNT{Ec. 8:14 ; Job. 21:7-8.}.
\VS{16}Ne sois pas juste à l’excès, et ne te montre pas trop sage : Pourquoi causerais tu ta propre ruine\FTNT{Pr. 3:7 ; Ro. 12:16.} ?
\VS{17}Ne sois point méchant à l’excès, et ne sois point insensé : Pourquoi mourrais-tu avant ton temps\FTNT{Ec. 9:16.} ?
\VS{18}Il est bon que tu saisisses ceci, et que tu ne retires point ta main de cela ; car celui qui craint Dieu échappe à toutes ces choses.
\VS{19}La sagesse donne plus de force au sage que dix chefs qui sont dans une ville.
\VS{20}Certainement il n'y a point d'homme juste sur la terre qui fasse le bien, et qui ne pèche jamais\FTNT{1 R. 8:46 ; 2 Ch. 6:36 ; Ps. 14:3 ; Ps. 54:3 ; Pr. 20:9 ; Ro.3:12 ; Ja. 3:2 ; 1 Jn. 1:8.}.
\VS{21}Ne fais donc pas attention à toutes les paroles qu'on dit, de peur que tu n'entendes ton serviteur te maudire.
\VS{22}Car ton cœur reconnaît que bien des fois tu as toi-même maudit les autres.
\VS{23}J'ai éprouvé tout cela par la sagesse. J'ai dit : Je serai sage. Et la sagesse est restée loin de moi.
\VS{24}Ce qui est loin et ce qui est profond, qui l’atteindra ?
\VS{25}J’ai appliqué mon cœur à connaître, à sonder, à chercher la sagesse et la raison des choses, et à connaître la folie de la méchanceté, et la stupidité de la sottise.
\VS{26}Et j'ai trouvé plus amère que la mort la femme dont le cœur est un piège et un filet, et dont les mains sont des liens ; celui qui est agréable à Dieu lui échappe, mais celui qui pèche est pris par elle\FTNT{Pr. 5:3-4 ; Pr. 6:26 ; Pr. 7:13-27 ; Pr. 9:13-16 ; Pr. 22:14.}.
\VS{27}Voici, dit l'Ecclésiaste, ce que j'ai trouvé en examinant les choses une à une pour en trouver la raison ;
\VS{28}voici, ce que mon âme cherche encore, mais que je n'ai point trouvé. J'ai trouvé un homme entre mille ; mais je n’ai point trouvé une femme entre elles toutes.
\VS{29}Seulement voici ce que j'ai trouvé ; c'est que Dieu a fait les hommes justes ; mais ils ont cherché beaucoup de détours.
\TextTitle{[L'obéissance aux autorités]}
\Chap{8}
\VerseOne{}Qui est comme le sage, et qui connaît l’explication des choses ? La sagesse d’un homme éclaire son visage, et la sévérité de sa face est changée\FTNT{Ec. 7:3 ; Pr. 15:13.}.
\VS{2}Je te le dis : Prends garde aux ordres du roi, et cela à cause de la parole du serment fait à Dieu.
\VS{3}Ne te hâte pas de t’éloigner de lui, et ne persiste point dans une chose mauvaise : Car il peut faire tout ce qu'il lui plait,
\VS{4}parce que la parole du roi est puissante ; et qui lui dira : Que fais-tu ?
\VS{5}Celui qui observe le commandement ne connaît point de chose mauvaise, et le cœur du sage connaît le temps et le jugement.
\VS{6}Car pour toute chose il y a un temps et un jugement, quand le malheur accable l'homme.
\VS{7}Car il ne sait pas ce qui arrivera, et qui le lui déclarera ?
\VS{8}L'homme n'est point maître de son souffle\FTNT{Le souffle ou l’esprit de l’homme quitte son corps le jour de la mort (Ps. 39:5 ; Ja. 2:26).} pour pouvoir le retenir, il n'a aucune puissance sur le jour de la mort ; il n'y a point de délivrance dans ce combat, et la méchanceté ne délivrera point son maître.
\VS{9}J'ai vu tout cela, et j'ai appliqué mon cœur à toute œuvre qui se fait sous le soleil. Il y a un temps où l’homme domine sur l'autre pour son malheur.
\VS{10}Alors j'ai vu les méchants ensevelis et s’en aller ; et ceux qui avaient agi avec droiture s’en aller loin du lieu saint et être oubliés dans la ville. C’est encore là une vanité\FTNT{Ec. 2:16 ; Ec. 9:5. }.
\VS{11}Parce que la sentence contre les mauvaises œuvres ne s'exécute point promptement, le cœur des fils de l’homme se remplit en eux de l’envie de faire le mal\FTNT{Ec. 12:1.}.
\VS{12}Bien que le pécheur fasse le mal cent fois, et qu’il y persiste, je sais aussi que le bonheur est pour ceux qui craignent Dieu, parce qu’ils ont de la crainte devant lui\FTNT{Job. 22:21 ; Pr. 1:33 ; Es. 3:10.}.
\VS{13}Mais le bonheur n’est pas pour le méchant, et il ne prolongera point ses jours plus que l'ombre, parce qu'il n’a pas de crainte devant Dieu.
\VS{14}Il est une vanité qui a lieu sur la terre : C'est qu'il y a des justes auxquels il arrive selon l'œuvre des méchants ; et il y a des méchants auxquels il arrive selon l'œuvre des justes. Je dis que c’est encore là une vanité.
\VS{15}J'ai donc loué la joie, parce qu'il n'y a de bonheur pour l’homme sous le soleil qu’à manger et à boire et à se réjouir ; c'est là ce qui doit l’accompagner au milieu de son travail, durant les jours de sa vie que Dieu lui donne sous le soleil.
\VS{16}Lorsque j’ai appliqué mon cœur à connaître la sagesse et à considérer les évènements qui se passent sur la terre, car les yeux de l’homme ne voient le sommeil ni jour ni nuit,
\VS{17}j’ai vu toute l'œuvre de Dieu, j’ai vu que l'homme ne peut pas trouver l'œuvre qui se fait sous le soleil ; il a beau se fatiguer à chercher, il n’est pas capable de trouver ; et même si le sage dit la connaître, il ne peut la trouver.
\TextTitle{[La sagesse impuissante devant la mort]}
\Chap{9}
\VerseOne{}Car j'ai appliqué mon cœur à tout cela, j’ai tout examiné, j’ai vu que les justes, les sages, et leurs travaux sont dans la main de Dieu, l’homme ne connaît ni l’amour ni la haine ; les hommes ne connaissent rien de tout ce qui est devant eux.
\VS{2}Tout arrive également à tous ; même sort pour le juste et pour le méchant, pour celui qui est bon et pour celui est impur, pour celui qui sacrifie et pour celui qui ne sacrifie pas ; il en est du bon comme du pécheur, de celui qui jure comme de celui qui craint de jurer.
\VS{3}C'est un mal parmi tout ce qui se fait sous le soleil, c’est qu’il y a pour tous un même sort ; aussi le cœur des fils de l’homme est-il plein de méchanceté, et la folie est dans leur cœur pendant leur vie ; après cela ils vont chez les morts. Qui est celui qui voudrait leur être associé\FTNT{Ec. 2:16 ; Job. 9 :22 ; Ps. 49:10.} ?
\VS{4}Il y a de l'espérance pour tous ceux qui sont encore vivants ; et même un chien vivant vaut mieux qu'un lion mort.
\VS{5}Les vivants, en effet, savent qu'ils mourront, mais les morts ne savent rien, et il n’y a pour eux plus de récompense, car leur mémoire est oubliée.
\VS{6}Aussi leur amour, leur haine, et leur envie ont déjà péri, et ils n'auront plus aucune part à tout ce qui se fait sous le soleil.
\VS{7}Va, mange ton pain avec joie, et bois gaiement ton vin ; car depuis longtemps Dieu prend plaisir à tes œuvres.
\VS{8}Qu’en tout temps tes vêtements soient blancs, et que l’huile ne manque point sur ta tête.
\VS{9}Vis joyeusement tous les jours de ta vie de vanité avec la femme que tu aimes, qui t’a été donnée sous le soleil, tous les jours de ta vanité ; car c'est là ta part dans la vie, au milieu de ton travail que tu fais sous le soleil.
\VS{10}Tout ce que ta main trouve à faire, fais-le selon ton pouvoir ; car dans le scheol, où tu vas, il n'y a ni œuvre, ni pensée, ni connaissance, ni sagesse.
\VS{11}J’ai encore vu sous le soleil que la course n'est point aux agiles ni la guerre aux héros, ni le pain aux sages, ni la richesse à ceux qui sont intelligents, ni la grâce à ceux qui ont de la connaissance ; mais tous dépendent du temps et des circonstances.
\VS{12}Car l'homme ne connaît pas son heure\FTNT{La mort est à la fois imprévisible et inévitable (Lu. 12:13-21).  Lors de la mort, le corps, c’est-à-dire la chair,est endormi dans la tombe (le mot ~ scheol ~ signifie aussi tombe). Le corps est inconscient, inanimé, endormi comme le dit l’Ecclésiaste. L’esprit est dépouillé momentanément de son habitation terrestre qui n’est qu’une tente (2 Co. 5:1-8).  Une fois l’esprit délogé de son enveloppe terrestre, ceux qui appartiennent au Seigneur le rejoignent (Ph. 1:21-23). Les chrétiens entrent dans la gloire et jouissent pleinement de la gloire de Dieu (Mt. 22:32 ; Mc. 9:43 ; Lu. 16:19-31). Quant aux païens, ils vont en enfer (Lu.16:19-31).}, comme les poissons qui sont pris au filet de malheur et les oiseaux qui sont pris au piège ; comme eux, les fils de l’homme sont enlacés au temps du malheur, lorsqu'il tombe subitement sur eux.
\VS{13}J'ai aussi vu cette sagesse sous le soleil, et elle m'a semblé grande.
\VS{14}Il y avait une petite ville, avec peu d’hommes dans son sein ; un roi puissant marcha contre elle, l’investit et bâtit de grands forts contre elle.
\VS{15}Il s'y trouvait un homme pauvre et sage qui délivra la ville par sa sagesse. Et personne ne s'est souvenu de cet homme pauvre.
\VS{16}Alors j'ai dit : La sagesse vaut mieux que la force. Cependant, la sagesse du pauvre est méprisée, et ses paroles ne sont point écoutées.
\VS{17}Mieux vaut des paroles de sages tranquillement écoutées que le cri de celui qui domine parmi les insensés.
\VS{18}Mieux vaut la sagesse que tous les instruments de guerre ; et un seul homme pécheur détruit beaucoup de bien.
\TextTitle{[La sagesse vaut mieux que la folie]}
\Chap{10}
\VerseOne{}Les mouches mortes infectent et font fermenter l’huile du parfumeur ; ainsi un peu de folie produit le même effet à l'égard de celui qui est estimé pour sa sagesse et pour sa gloire.
\VS{2}Le cœur du sage est à sa droite, et le cœur de l’insensé est à sa gauche.
\VS{3}Quand l’insensé marche dans un chemin, le sens lui manque, et il dit de chacun : Voilà un insensé !
\VS{4}Si l'esprit de celui qui gouverne s'élève contre toi, ne quitte point ton poste ; car la douceur prévient de grands péchés.
\VS{5}Il y a un mal que j'ai vu sous le soleil, comme une erreur provenant de celui qui gouverne :
\VS{6}La folie occupe des postes très élevés, et des riches sont assis dans l’abaissement.
\VS{7}J'ai vu des serviteurs sur des chevaux, et des princes marchant sur terre comme des serviteurs.
\VS{8}Celui qui creuse la fosse y tombera, et celui qui renverse une clôture le serpent le mordra\FTNT{Ps. 7:15 ; Pr. 26:27 ; Pr. 28:10.}.
\VS{9}Celui qui remue des pierres en sera blessé, et celui qui fend du bois en éprouvera du danger.
\VS{10}Si le fer est émoussé, et qu'on n'en ait point aiguisé le tranchant, il devra redoubler de force ; mais la sagesse a l’avantage du succès.
\VS{11}Si le serpent mord faute d’enchantement, il n’y a pas de profit pour le maître enchanteur.
\VS{12}Les paroles de la bouche du sage ne sont que grâce ; mais les lèvres de l’insensé l’engloutissent\FTNT{Pr. 10:21. }.
\VS{13}Le commencement des paroles de sa bouche est folie, et la fin de son discours est une méchante folie
\VS{14}L’insensé multiplie les paroles. L'homme ne sait point ce qui arrivera, et qui lui déclarera ce qui sera après lui ?
\VS{15}Le travail de l’insensé le fatigue, parce qu’il ne sait pas aller à la ville.
\VS{16}Malheur à toi, pays dont le roi est un enfant, et dont les princes mangent dès le matin\FTNT{Es. 3:4. } !
\VS{17}Heureux toi, pays dont le roi est de race illustre, et dont les princes mangent au temps convenable, pour soutenir leurs forces, et non pour se livrer à la boisson!
\VS{18}A cause de la paresse, la charpente s'affaisse, et à cause des mains lâches, la maison a des gouttières.
\VS{19}On fait des pains pour se réjouir, le vin rend la vie joyeuse, et l'argent répond à tout\FTNT{L'argent est nécessaire pour financer l'œuvre de Dieu dans le monde (évangélisation, soutien aux nécessiteux…), pour payer notre nourriture, nos loyers, nos factures, nos vêtements. Il ne donne cependant pas accès à la vie éternelle (Mt. 16:26 ; Mc. 8:37). L’amour de l’argent est la racine de tous les maux (1 Ti. 6:10).}.
\VS{20}Ne maudis point le roi, même dans ta pensée, et ne maudis pas le riche dans la chambre où tu couches ; car l’oiseau du ciel emporterait ta voix, le Baal ailé\FTNT{Baal ailé : Le terme sémitique "baal" (en hébreu ba’al) signifie à l’origine ~"possesseur"~, ~"maître"~ ou ~"seigneur"~. Le Baal ailé était une créature ailée. Utilisé au pluriel, l’expression ~"baalim de flèches"~ désignait des archers. Les écritures nous parlent de Baal-Zebub (seigneur des mouches),un démon adoré à Ekron, l’une des villes des Philistins (2 R. 1:1-16). Baal-Zebud à donné ~"Béelzébul"~ dans les Evangiles (Mt. 10:25 ; Mt.12:24 ; Mt. 12:27 ; Lu. 11:15-19). Ce passage nous enseigne clairement que les démons épient les enfants de Dieu et vont ensuite faire leurs rapports à Satan afin de mieux les attaquer. Ils agissent comme des espions. Ces esprits sont comme des mouches et essayent de s’infiltrer partout.} rapporterait tes paroles.
\TextTitle{[L'homme travaille en tâtonnant]}
\Chap{11}
\VerseOne{}Jette ton pain à la face des eaux, car avec le temps tu le retrouveras.
\VS{2}Donnes-en une part à sept et même à huit, car tu ne sais point quel mal peut venir sur la terre.
\VS{3}Quand les nuages sont pleins, ils répandent la pluie sur la terre ; et quand un arbre tombe, au sud ou au nord, il reste à la place où il est tombé.
\VS{4}Celui qui observe le vent ne sèmera point, et celui qui regarde les nuages ne moissonnera point.
\VS{5}Comme tu ne sais point quel est le chemin du vent, ni comment se forment les os dans le ventre de celle qui est enceinte, de même tu ne connais pas l'œuvre de Dieu qui fait tout\FTNT{Ceux qui sont nés d’en-haut sont insaisissables comme le vent (Jn 3:8).}.
\VS{6}Sème ta semence dès le matin, et ne laisse pas reposer ta main le soir ; car tu ne sais point ce qui réussira, ceci ou cela, ou si tous deux seront également bons.
\VS{7}La lumière est douce, et il est agréable aux yeux de voir le soleil.
\VS{8}Mais si un homme vit de nombreuses années, qu'il se réjouisse, et qu'il se souvienne des jours de ténèbres qui seront nombreux ; tout ce qui lui arrivera est vanité.
\Chap{12}
\VerseOne{}Jeune homme, réjouis-toi dans ta jeunesse, et que ton cœur te rende joyeux pendant les jours de ta jeunesse, marche dans les voies de ton cœur et selon le regard de tes yeux ; mais sache que pour tout cela Dieu t’appellera en jugement.
\VS{2}Bannis le chagrin de ton cœur, et éloigne le mal de ton corps ; car l’enfance et la jeunesse ne sont que vanité.
\TextTitle{[Craindre Dieu pour garder ses commandements]}
\VS{3}Mais souviens-toi de ton Créateur pendant les jours de ta jeunesse, avant que les jours mauvais arrivent et que viennent les années où tu diras : Je n'y prends point de plaisir ;
\VS{4}avant que le soleil et la lumière, la lune et les étoiles s'obscurcissent, et que les nuages reviennent après la pluie,
\VS{5}jours où ceux qui gardent la maison tremblent\FTNT{~ Ceux qui gardent la maison ~ représentent les mains.}, où les hommes forts\FTNT{~ Les hommes forts ~ sont les jambes.} se courbent, où celles qui moulent\FTNT{Les dents sont ~ celles qui moulent ~.} cessent de travailler parce qu'elles sont diminuées, où ceux qui regardent par les fenêtres\FTNT{Les yeux sont ~ ceux qui regardent par les fenêtres ~.} sont obscurcis,
\VS{6}et où les deux battants de la porte\FTNT{Les oreilles sont ~ les deux battants de la porte ~.} se ferment sur la rue quand s’abaisse le bruit de la meule, où l’on se lève au chant de l'oiseau, et que toutes les filles du chant s’affaiblissent.
\VS{7}Où l’on craint ce qui est élevé, où l’on a des terreurs en chemin, quand l'amandier fleurit, où la sauterelle devient pesante, et que la câpre est sans effet, car l'homme s'en vers sa demeure éternelle\FTNT{~ La demeure éternelle ~  c’est la Nouvelle Jérusalem pour les chrétiens (Ap. 21.) et pour les païens le lac de feu (Ap. 20:11-15).}, et ceux qui pleurent font le tour des rues.
\VS{8}Avant que la corde d'argent\FTNT{Cette corde est comme le cordon ombilical, elle lie l’âme au corps. Lors de la mort, la corde d’argent est coupée.} se détache, que le vase d'or\FTNT{Le corps humain est comme un vase ou une tente qui renferme son esprit. Comme l'argile dans la main du potier, ainsi est l'homme dans celle de Dieu. Avec cette argile, il décide souverainement de fabriquer de la même masse un vase à honneur et un autre pour un usage vil (Jé. 18: 4-6 ; Ro. 9:21 ; 2 Tim. 2:20-21).} se brise, que la cruche se rompe sur la source, que la roue s’écrase sur la citerne ;
\VS{9}avant que la poussière retourne dans la terre\FTNT{Le corps de l’homme a été tiré de la poussière, à la mort il retourne à la poussière (Ge. 3:19). }, comme elle y avait été, et que l'esprit retourne à Dieu\FTNT{Voir commentaire en Ec. 9:12.} qui l'a donné.
\VS{10}Vanité des vanités, dit l'Ecclésiaste, tout est vanité.
\VS{11}Outre que l'Ecclésiaste fut sage, il a enseigné la connaissance au peuple, et il a examiné, sondé, mis en ordre beaucoup de proverbes.
\VS{12}L'Ecclésiaste a cherché pour trouver des paroles agréables ; et ce qui a été écrit avec droiture, ce sont des paroles de vérité.
\VS{13}Les paroles des sages sont comme des aiguillons ; et rassemblées en recueil, elles sont comme des clous plantés, données par un seul maître.
\VS{14}Du reste, mon fils, sois instruit par ces choses ; on ne finirait pas, si l’on voulait faire un grand nombre de livres, et beaucoup d'étude est une fatigue pour le corps.
\VS{15}Ecoutons la conclusion du discours : Crains Dieu, et garde ses commandements. C'est là le tout de l'homme.
\VS{16}Parce que Dieu amènera toute œuvre en jugement, au sujet de tout ce qui est caché, soit bien, soit mal.
\PPE{}
\end{multicols}
