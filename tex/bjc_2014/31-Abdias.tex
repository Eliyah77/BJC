\ShortTitle{Abdias}\BookTitle{Abdias}\BFont
\noindent\hrulefill
{\footnotesize
\textit{
\bigskip
{\centering{}
\\Auteur : Abdias
\\(Heb. : Obadyah)
\\Signification : Adorateur ou serviteur de Yahweh
\\Thème : Condamnation d'Edom
\\Date de rédaction : 6\up{ème} siècle av. J.-C.\\}
}
%\bigskip
\textit{
\\Prophète ayant exercé son ministère en Juda, Abdias reçut un message court, mais clair sur le jour de Yahweh, et plus particulièrement sur le jugement d'Edom à la suite de ses violences envers Israël.\bigskip
}
}
\par\nobreak\noindent\hrulefill
\begin{multicols}{2}
\Chap{1}
\TextTitle{[Malédiction prononcée sur Edom]}
\VerseOne{}La vision d'Abdias. Ainsi parle le Seigneur, Yahweh, sur Edom : Nous avons entendu une publication de la part de Yahweh, et un messager a été dépêché parmi les nations, et elles ont dit : Courage, levons-nous contre lui pour le combattre\FTNT{Jé. 49:14-16} !
\VS{2}Voici, je te rendrai petit parmi les nations, tu seras fort méprisé.
\VS{3}L'orgueil de ton coeur t'a séduit, toi qui habites dans les fentes des rochers, qui sont ta haute demeure, et qui dis en ton coeur : Qui me renversera par terre ?
\VS{4}Quand tu élèverais ton nid comme l'aigle, et quand bien même tu le mettrais entre les étoiles, je te jetterai de là par terre, dit Yahweh.
\VS{5}Sont–ce des larrons qui sont entrés chez toi, ou des voleurs de nuit ? Comment donc as–tu été rasé ? N'eussent–ils pas dérobé, jusqu'à ce qu'ils en eussent eu assez ? Si des vendangeurs fussent entrés chez toi, n'eussent–ils pas laissé quelque grappillage ?
\VS{6}Comment a été fouillé Esaü ? Comment ses trésors cachés ont été découverts ?
\VS{7}Tous tes alliés t'ont conduit jusqu'à la frontière, ceux qui étaient en paix avec toi t'ont trompé et ont eu le dessus sur toi, ceux qui mangeaient ton pain t'ont tendu des pièges, et tu ne t'en es pas aperçu.
\VS{8}N'est-ce pas en ce jour-là, dit Yahweh, que je ferai périr les sages au milieu d'Edom, et l'intelligence de la montagne d'Esaü ?
\VS{9}Tes guerriers seront effrayés, ô Théman ! Afin qu'ils soient tous retranchés de la montagne d'Esaü par le carnage\FTNT{Ez. 25:13 ; Mi. 7:8 ; So. 2:8}.
\TextTitle{Les causes de sa malédiction}
\VS{10}A cause de la violence que tu as faite à ton frère Jacob, la honte te couvrira, et tu seras retranché à jamais.
\VS{11}Le jour où tu te tenais vis-à-vis de lui, le jour où des étrangers emmenaient captive son armée, où des étrangers entraient dans ses portes et jetaient le sort sur Jérusalem, toi aussi, tu étais comme l'un d'eux.
\VS{12}Mais tu ne devais pas prendre plaisir à voir la journée de ton frère quand il a été livré aux étrangers, et tu ne devais pas te réjouir sur les fils de Juda le jour où ils ont été détruits et tu ne devais pas les braver au jour de la détresse.
\VS{13}Et tu ne devais pas entrer dans la porte de mon peuple au jour de sa calamité, et tu ne devais pas prendre plaisir, toi, dis–je, à voir son mal au jour de sa calamité, et tes mains ne se devaient pas d'avancer sur son bien, au jour de sa calamité.
\VS{14}Et tu ne te devais pas tenir sur les passages, pour exterminer ses réchappés ; ni livrer ceux qui étaient restés, au jour de la détresse.
\TextTitle{Châtiment d'Edom au jour de Yahweh}
\VS{15}Car le jour de Yahweh est proche sur toutes les nations ; comme tu as fait, il te sera ainsi fait ; ta récompense retournera sur ta tête\FTNT{Jé. 50:15-29 ; Ez. 35:15}.
\VS{16}Car comme vous avez bu sur la montagne de ma sainteté la coupe de ma fureur, ainsi toutes les nations la boiront sans interruption ; oui, elles la boiront et l'avaleront, et elles seront comme si elles n'avaient point été\FTNT{Jé. 25:15-28}.
\TextTitle{Edom jugé, Jacob délivré}
\VS{17}Mais la délivrance\FTNT{La délivrance sera sur la montagne de Sion. Cette prophétie fait allusion au Royaume messianique. Voir Ro. 11:26.} sera sur la montagne de Sion, elle sera sainte, et la maison de Jacob possédera ses possessions.
\VS{18}La maison de Jacob sera un feu, et la maison de Joseph une flamme, et la maison d'Esaü du chaume ; ils s'allumeront parmi eux et les consumeront ; et il n'y aura rien de reste dans la maison d'Esaü ; car Yahweh a parlé.
\VS{19}Ils posséderont le Midi, à savoir la montagne d'Esaü ; et la plaine, le pays des Philistins, et ils posséderont le territoire d'Ephraïm, et le territoire de Samarie ; et Benjamin possédera Galaad.
\VS{20}Et ces bandes des enfants d'Israël qui auront été transportés, posséderont ce qui était des Cananéens, jusqu'à Sarepta ; et ceux de Jérusalem qui auront été transportés, posséderont ce qui est jusqu'à Sépharad, ils le posséderont avec les villes du Midi.
\VS{21}Des libérateurs monteront sur la montagne de Sion, pour juger la montagne d'Esaü ; et le royaume sera à Yahweh.
\PPE{}
\end{multicols}
