\ShortTitle{Philippiens}\BookTitle{Philippiens}\BFont
\noindent\hrulefill
{\footnotesize
\textit{
\bigskip
{\centering{}
\\Auteur : Paul
\\Thème : Expérience chrétienne
\\Date de rédaction : Env. 60 ap. J.-C.\\}
}
%\bigskip
\textit{
\\Fondée par Philippe II (382 av. J.-C. – 336 av. J.-C.) en 356 av. J.-C., Philippes est une ville grecque de Macédoine orientale. Située sur une voie romaine qui traversait les Balkans (Via Egnatia), elle est restée de taille modeste en dépit de son fort taux de fréquentation.
%\bigskip
\\La première mention de l’assemblée de Philippes se trouve dans Actes 16 lors de la rencontre de Paul avec  des femmes réunies à l’extérieur  de la ville pour la prière. Au travers des paroles de Paul, le Seigneur toucha particulièrement Lydie qui, après avoir été baptisée avec sa famille,  reçut Paul et ses compagnons dans sa maison.
%\bigskip
\\C’est à Rome, sous le règne de Néron (37-68), que Paul alors captif rédigea cette lettre. Cet écrit de l’apôtre accusait réception d’un don monétaire que l’église de Philippes lui avait fait parvenir par le biais d’Epaphrodite. Paul y exprimait sa joie en dépit des souffrances et invitait les Philippiens à faire de même. Loin des erreurs doctrinales reprochées à d’autres, ces chrétiens recevaient ainsi l’expression de l’affection de Paul et ses encouragements à persévérer dans la foi en Christ en toutes circonstances.\bigskip
}
}
\par\nobreak\noindent\hrulefill
\begin{multicols}{2}
\Chap{1}
\TextTitle{Introduction}
\VerseOne{}Paul et Timothée, serviteurs de Jésus-Christ, à tous les saints en Jésus-Christ qui sont à Philippes, avec les évêques et les diacres.
\VS{2}Que la grâce et la paix vous soient données de la part de Dieu, notre Père et du Seigneur Jésus-Christ.
\VS{3}Je rends grâces à mon Dieu de tout le souvenir que je garde de vous,
\VS{4}en priant toujours pour vous tous avec joie dans toutes mes prières,
\VS{5}à cause de votre attachement à l'Evangile, depuis le premier jour jusqu'à maintenant.
\VS{6}Je suis persuadé que celui qui a commencé cette bonne œuvre en vous l'achèvera jusqu'au jour de Jésus-Christ.
\VS{7}Comme il est juste que je pense ainsi de vous tous, parce que je retiens dans mon cœur que vous avez tous été participants de la grâce avec moi dans mes liens, et dans la défense et la confirmation de l'Evangile.
\TextTitle{Les chrétiens encouragés par la souffrance de Paul}
\VS{8}Car Dieu m'est témoin que je vous aime tous tendrement, conformément à la charité de Jésus-Christ.
\VS{9}Et je lui demande cette grâce : Que votre charité abonde encore de plus en plus avec connaissance et toute intelligence,
\VS{10}pour le discernement des choses contraires, afin que vous soyez purs et irréprochables pour le jour de Christ,
\VS{11}étant remplis de fruits de justice, qui sont par Jésus-Christ, à la gloire et à la louange de Dieu.
\VS{12}Je veux bien que vous sachiez, mes frères, que ce qui m’est arrivé a plutôt contribué aux progrès de l'Evangile.
\VS{13}En effet, dans tout le prétoire et partout ailleurs, nul n’ignore que c’est pour Christ que je suis dans les liens.
\VS{14}Et plusieurs de nos frères en notre Seigneur, étant rassurés par mes liens, osent annoncer la parole résolument et sans crainte.
\VS{15}Il est vrai que quelques-uns prêchent Christ par envie et par esprit de dispute, et que les autres le font, au contraire, avec une intention sincère.
\VS{16}Les uns annoncent Christ par un esprit de dispute, et non pas purement, croyant ajouter de l'affliction à mes liens.
\VS{17}Mais les autres le font par charité, sachant que je suis établi pour la défense de l'Evangile.
\VS{18}Qu’importe ? De toute manière, que ce soit par ostentation, ou par amour de la vérité, Christ n’est pas moins annoncé. Je m’en réjouis, et je m’en réjouirai encore.
\VS{19}Car je sais que cela tournera à mon salut grâce à vos prières, et à l’assistance de l'Esprit de Jésus-Christ,
\VS{20}selon ma ferme attente et mon espérance, je n’aurai honte de rien, mais que parlant avec hardiesse, Christ, qui a toujours été glorifié dans mon corps, le sera encore à présent, soit par ma vie, soit par ma mort.
\VS{21}Car Christ est ma vie, et la mort m’est un gain.
\VS{22}Mais s'il est utile pour mon œuvre que je vive dans la chair, je ne saurais dire ce que je dois préférer.
\VS{23}Car je suis pressé des deux côtés : J’ai le désir de m’en aller et d’être avec Christ, ce qui de beaucoup est le meilleur.
\VS{24}Mais il est plus nécessaire pour vous que je demeure dans la chair.
\VS{25}Et je suis persuadé, je sais que je demeurerai et que je resterai avec vous tous, pour votre avancement et pour votre joie dans la foi,
\VS{26}afin que, par mon retour auprès de vous, vous ayez en moi un sujet de vous glorifier de plus en plus en Jésus-Christ.
\VS{27}Seulement conduisez-vous dignement comme il est séant selon l'Evangile de Christ ; afin que, soit que je vienne, et que je vous voie ; soit que je sois absent, j'entende quant à votre état, que vous persistez dans un même esprit, combattant ensemble d'un même courage par la foi de l’Evangile, et n'étant en rien épouvantés par les adversaires.
\VS{28}Ce qui est pour eux une preuve de perdition, mais pour vous, de salut ; et cela de la part de Dieu.
\VS{29}Parce qu'il vous a été gratuitement donné dans ce qui a du rapport à Christ, non seulement de croire en lui, mais aussi de souffrir pour lui,
\VS{30}en soutenant le même combat que vous m’avez vu soutenir, et que vous apprenez maintenant que je soutiens encore.
\Chap{2}
\TextTitle{Exhortation à l'unité}
\VerseOne{}Si donc il y a quelque consolation en Christ, s’il y a quelque soulagement dans la charité, s'il y a quelque communion d'esprit, s'il y a quelques cordiales affections et quelques compassions,
\VS{2}rendez ma joie parfaite, ayant un même sentiment, un même amour, une même âme, et consentant tous à une même chose.
\VS{3}Ne faites rien par esprit de parti\FTNT{Parti : Du grec «~eritheia~». Ce mot ne se trouve, avant la nouvelle alliance, que dans Aristote où il dénote une recherche personnelle, la poursuite d'une fonction politique par des moyens injustes. Ce mot signifie aussi faire une campagne électorale ou intriguer pour une fonction dans un esprit partisan, querelleur.}, ou par vaine gloire ; mais que l’humilité de cœur vous fasse regarder les autres comme étant au-dessus de vous-mêmes.
\VS{4}Que chacun de vous, au lieu de considérer ses propres intérêts, considère aussi ceux des autres.
\TextTitle{l'humanité de Christ}
\VS{5}Ayez donc en vous les sentiments qui étaient en Jésus-Christ,
\VS{6}lequel étant en forme de Dieu, n'a point regardé son égalité avec Dieu comme une usurpation.
\VS{7}Mais il s’est dépouillé lui-même, en prenant une forme de serviteur, en devenant semblable aux hommes ; et il parut comme un simple homme,
\VS{8}il s'est abaissé lui-même, en devenant obéissant jusqu’à la mort, même jusqu’à la mort de la croix.
\VS{9}C'est pourquoi aussi Dieu l'a souverainement élevé, et lui a donné le Nom qui est au-dessus de tout nom ;
\VS{10}afin qu'au Nom de Jésus, tout genou fléchisse dans les cieux, sur la terre, et sous la terre,
\VS{11}et que toute langue confesse que Jésus-Christ est Seigneur, à la gloire de Dieu le Père.
\VS{12}C'est pourquoi, mes bien-aimés, comme vous avez toujours obéi, mettez en œuvre votre propre salut avec crainte et tremblement, non seulement comme en ma présence, mais beaucoup plus maintenant que je suis absent.
\VS{13}Car c'est Dieu qui produit en vous avec efficacité le vouloir et le faire, selon son bon plaisir.
\VS{14}Faites toutes choses sans murmures et sans disputes,
\VS{15}afin que vous soyez sans reproche et purs, des enfants de Dieu irrépréhensibles au milieu de la génération perverse et corrompue, parmi laquelle vous brillez comme des flambeaux dans le monde, portant la parole de la vie.
\VS{16}En sorte qu’au jour de Christ, je puisse me glorifier de n'avoir point couru en vain, ni travaillé en vain.
\TextTitle{Paul témoigne de Timothée et d'Epaphrodite}
\VS{17}Et même si je sers de libation pour le sacrifice et pour le service de votre foi, je m’en réjouis, et je m'en réjouis avec vous tous.
\VS{18}Vous aussi pareillement, réjouissez-vous, et réjouissez-vous avec moi.
\VS{19}J'espère, avec la grâce du Seigneur Jésus, vous envoyer bientôt Timothée, afin d’être encouragé moi-même en apprenant ce qui vous concerne.
\VS{20}Car je n'ai personne ici qui partage mes sentiments pour prendre sincèrement votre situation à cœur.
\VS{21}Parce que tous cherchent leur intérêt particulier, et non les intérêts de Jésus-Christ.
\VS{22}Vous savez qu’il a été mis à l’épreuve, en se consacrant au service de l’Evangile avec moi, comme un enfant avec son père.
\VS{23}J'espère donc vous l'envoyer dès que j'aurai pourvu à mes affaires.
\VS{24}Et j’ai cette confiance en notre Seigneur que moi-même aussi j’irai bientôt.
\VS{25}Mais j'ai jugé nécessaire de vous envoyer Epaphrodite mon frère, mon compagnon d’œuvre et de combat, par qui vous m’aviez envoyé de quoi pourvoir à mes besoins.
\VS{26}Car il désirait ardemment vous voir tous, et il était fort affligé de ce que vous aviez appris sa maladie.
\VS{27}En effet, il a été malade et tout près de la mort ; mais Dieu a eu pitié de lui, et non seulement de lui, mais aussi de moi, afin que je n'eusse pas tristesse sur tristesse.
\VS{28}Je l'ai donc envoyé à cause de cela avec plus de soin, afin qu'en le revoyant vous ayez de la joie, et que j'aie moins de tristesse.
\VS{29}Recevez-le donc dans le Seigneur, avec une joie entière ; et ayez de la considération pour de tels hommes.
\VS{30}Car c’est pour l’œuvre de Christ qu’il a été près de la mort, ayant exposé sa propre vie, afin de suppléer à votre absence dans le service que vous me rendiez.
\Chap{3}
\TextTitle{Mise en garde contre les légalistes et la loi mosaïque}
\VerseOne{}Au reste, mes frères, réjouissez-vous dans le Seigneur. Je ne me lasse point de vous écrire les mêmes choses, et c’est votre sûreté.
\VS{2}Prenez garde aux chiens ; prenez garde aux mauvais ouvriers ; prenez garde aux faux circoncis.
\VS{3}Car c'est nous qui sommes les circoncis, qui rendons à Dieu notre culte en Esprit, et qui nous glorifions en Jésus-Christ, et qui ne mettons point notre confiance en la chair.
\VS{4}Moi aussi, cependant, j’aurais sujet de mettre ma confiance en la chair. Si quelqu'un estime qu'il a de quoi se confier en la chair, je le puis bien davantage :
\VS{5}Moi, circoncis le huitième jour, de la race d'Israël, de la tribu de Benjamin, Hébreu né d'Hébreux, pharisien en ce qui concerne la loi.
\VS{6}Quant au zèle, persécutant l'Eglise ; et quant à la justice à l’égard de la loi, étant sans reproche.
\TextTitle{Le Messie, objet de notre foi}
\VS{7}Mais ces choses qui étaient pour moi un gain, je les ai regardées comme une perte à cause de l'amour de Christ.
\VS{8}Et certes, je regarde toutes les autres choses comme m'étant nuisibles en comparaison de l'excellence de la connaissance de Jésus-Christ mon Seigneur, pour l'amour duquel je me suis privé de toutes ces choses, et je les estime comme du fumier, afin de gagner Christ,
\VS{9}et d’être trouvé en lui, non avec ma justice, celle qui vient de la loi, mais celle qui s’obtient par la foi en Christ, la justice qui vient de Dieu par la foi.
\VS{10}Ainsi, je connaîtrai Jésus-Christ et la puissance de sa résurrection, et la communion de ses souffrances, en devenant conforme à lui dans sa mort, pour parvenir,
\VS{11}si je puis, à la résurrection d’entre les morts.
\VS{12}Ce n’est pas que j'aie déjà atteint le but, ou que j’aie déjà atteint la perfection, mais je cours pour tâcher de le saisir, puisque moi aussi j'ai été saisi par Jésus-Christ.
\VS{13}Mes frères, pour moi, je ne me persuade pas d'avoir atteint le but ;
\VS{14}mais je fais une chose : Oubliant les choses qui sont en arrière, et me portant vers celles qui sont en avant, je cours vers le but, pour remporter le prix de la vocation céleste de Dieu en Jésus-Christ.
\TextTitle{Paul exhorte les croyants à l'unité}
\VS{15}C'est pourquoi, nous tous qui sommes parfaits, ayons ce même sentiment ; et si vous êtes en quelque point d’un autre avis, Dieu vous le révélera aussi.
\VS{16}Cependant marchons suivant une même règle pour les choses auxquelles nous sommes parvenus, et ayons un même sentiment.
\VS{17}Soyez tous ensemble mes imitateurs, mes frères, et portez les regards sur ceux qui marchent selon le modèle que vous avez en nous.
\VS{18}Car il en est plusieurs qui marchent en ennemis de la croix de Christ, je vous en ai souvent parlé, et j’en parle encore en pleurant.
\VS{19}Leur fin sera la perdition, ils ont pour dieu leur ventre, et ils mettent leur gloire dans ce qui fait leur honte, n'ayant d'affection que pour les choses de la terre.
\TextTitle{Le Messie, objet de notre espérance}
\VS{20}Mais pour nous, notre cité à nous est dans les cieux, d'où nous attendons aussi comme Sauveur, le Seigneur Jésus-Christ,
\VS{21}qui transformera le corps de notre humiliation, en le rendant semblable à son corps glorieux, par le pouvoir qu’il a de s’assujettir toutes choses.
\Chap{4}
\TextTitle{Paul exhorte les chrétiens à avoir le même sentiment}
\VerseOne{}C'est pourquoi, mes très chers frères bien-aimés, vous qui êtes ma joie et ma couronne, demeurez ainsi fermes dans le Seigneur, mes bien-aimés.
\VS{2}J’exhorte Evodie, et j’exhorte aussi Syntyche, à être d’un même sentiment dans le Seigneur.
\VS{3}Et toi aussi, mon vrai compagnon\FTNT{Le mot compagnon est la traduction du grec «~suzugos~» qui signifie littéralement: ensemble sous le joug, compagnon de peine, de joug. Cette expression renvoie à 2 corinthiens 6:14.}, oui je te prie de les aider, elles qui ont combattu avec moi pour l'Evangile, avec Clément, et mes autres compagnons d’œuvre, dont les noms sont écrits dans le livre de vie.
\VS{4}Réjouissez-vous toujours dans le Seigneur ; je vous le répète, réjouissez-vous.
\VS{5}Que votre douceur soit connue de tous les hommes. Le Seigneur est proche.
\VS{6}Ne vous inquiétez de rien, mais en toutes choses exposez vos besoins à Dieu par des prières et des supplications, avec des actions de grâces.
\VS{7}Et la paix de Dieu, qui surpasse toute intelligence, gardera vos cœurs et vos pensées en Jésus-Christ.
\TextTitle{L'objet de nos pensées}
\VS{8}Au reste, mes frères, que tout ce qui est vrai, tout ce qui est honorable, tout ce qui est juste, tout ce qui est pur, tout ce qui est aimable, tout ce qui mérite l’approbation, ce qui est vertueux et digne de louange, occupent vos pensées.
\VS{9}Ce que vous avez appris, reçu et entendu de moi, et ce que vous avez vu en moi, pratiquez-le. Et le Dieu de paix sera avec vous.
\TextTitle{Dieu soutient ses serviteurs}
\VS{10}J’ai éprouvé une grande joie dans le Seigneur de ce que vous avez pu enfin renouveler l’expression de vos sentiments pour moi. Vous y pensiez bien, mais l’occasion vous manquait.
\VS{11}Ce n’est pas en vue de mes besoins que je dis cela, car j’ai appris à être content de l’état où je me trouve.
\VS{12}Je sais vivre dans l’humiliation, je sais aussi vivre dans l'abondance ; en tout et partout j’ai appris à être rassasié, et à avoir faim, à être dans l'abondance, et à être dans la disette.
\VS{13}Je puis tout par Christ qui me fortifie.
\VS{14}Néanmoins, vous avez bien fait de prendre part à mon affliction.
\VS{15}Vous le savez vous-mêmes, Philippiens, au commencement de la prédication de l’Evangile, lorsque je partis de la Macédoine, aucune église n’entra en compte avec moi pour ce qu’elle donnait et recevait.
\VS{16}Vous fûtes les seuls à le faire, vous m’envoyâtes déjà à Thessalonique, une fois, et même deux fois, ce dont j'avais besoin.
\VS{17}Ce n'est pas que je recherche des présents, mais je cherche le fruit qui abonde pour votre compte.
\VS{18}J'ai tout reçu, et je suis dans l'abondance, et j'ai été comblé de biens en recevant d'Epaphrodite ce qui vient de vous, comme un parfum de bonne odeur, comme un sacrifice que Dieu accepte et qui lui est agréable.
\VS{19}Aussi mon Dieu pourvoira selon sa richesse à tous vos besoins, et avec gloire, en Jésus-Christ.
\TextTitle{Conclusion}
\VS{20}À notre Dieu et Père soit la gloire aux siècles des siècles. Amen !
\VS{21}Saluez tous les saints en Jésus-Christ. Les frères qui sont avec moi vous saluent.
\VS{22}Tous les saints vous saluent, et principalement ceux qui sont de la maison de César.
\VS{23}Que la grâce de notre Seigneur Jésus-Christ soit avec vous tous. Amen.
\PPE{}
\end{multicols}
