\ShortTitle{Ph.}\BookTitle{Philippiens}\BFont
\noindent\hrulefill
{\footnotesize
\textit{
\bigskip
{\centering{}
\\Auteur~: Paul
\\Thème~: Expérience chrétienne
\\Date de rédaction~: Env. 60 ap. J.-C.\\}
}
\textit{
\\Fondée par Philippe II (382 av. J.-C. – 336 av. J.-C.) en 356 av. J.-C., Philippes est une ville grecque de Macédoine orientale. Située sur une voie romaine qui traversait les Balkans (Via Egnatia), elle est restée de taille modeste en dépit de son fort taux de fréquentation.
\\La première mention de l'assemblée de Philippes se trouve dans Actes 16, lors de la rencontre de Paul avec des femmes réunies à l'extérieur de la ville pour la prière. A travers les paroles de Paul, le Seigneur toucha particulièrement Lydie qui, après avoir été baptisée avec sa famille, le reçut avec ses compagnons d'oeuvre dans sa maison.
\\C'est à Rome, sous le règne de Néron (37-68), que Paul, alors captif, rédigea cette lettre. Cet écrit de l'apôtre accusait réception d'un don monétaire que l'église de Philippes lui avait fait parvenir par le biais d'Epaphrodite. Paul y exprimait sa joie en dépit des souffrances et invitait les Philippiens à faire de même. Loin des erreurs doctrinales reprochées à d'autres, ces chrétiens recevaient ainsi l'expression de l'affection de Paul et ses encouragements à persévérer dans la foi en Christ en toutes circonstances.\bigskip
}
}
\par\nobreak\noindent\hrulefill
\begin{multicols}{2}
\Chap{1}
\TextTitle{Introduction}
\VerseOne{}Paul et Timothée, serviteurs de Jésus-Christ, à tous les saints en Jésus-Christ qui sont à Philippes, avec les évêques et les diacres~:
\VS{2}Que la grâce et la paix vous soient données de la part de Dieu notre Père, et du Seigneur Jésus-Christ~!
\VS{3}Je rends grâces à mon Dieu toutes les fois que je fais mention de vous,
\VS{4}en priant toujours pour vous tous avec joie dans toutes mes prières,
\VS{5}à cause de votre attachement à l'Evangile, depuis le premier jour jusqu'à maintenant~;
\VS{6}étant persuadé de cela même que celui qui a commencé cette bonne œuvre en vous, l'achèvera jusqu'au jour de Jésus-Christ.
\VS{7}Comme il est juste que je pense ainsi de vous tous, parce que je retiens dans mon cœur que vous avez tous été participants de la grâce avec moi dans mes liens, et dans la défense et la confirmation de l'Evangile.
\VS{8}Car Dieu m'est témoin que je vous aime tous tendrement, conformément à la charité de Jésus-Christ.
\VS{9}Et je lui demande cette grâce~: Que votre charité abonde encore de plus en plus avec connaissance et toute intelligence,
\VS{10}pour le discernement des choses contraires, afin que vous soyez purs et irréprochables pour le jour de Christ,
\VS{11}étant remplis de fruits de justice, qui sont par Jésus-Christ à la gloire et à la louange de Dieu.
\TextTitle{Les chrétiens encouragés par les souffrances de Paul}
\VS{12}Or mes frères, je veux bien que vous sachiez que les choses qui me sont arrivées sont arrivées pour un plus grand avancement de l'Evangile.
\VS{13}De sorte que mes liens en Christ ont été rendus célèbres dans tout le prétoire, et partout ailleurs. 
\VS{14}Et que plusieurs de nos frères en notre Seigneur, étant rassurés par mes liens, osent annoncer la parole plus hardiment, et sans crainte. 
\VS{15}Il est vrai que quelques-uns prêchent Christ par envie et par un esprit de dispute~; et que les autres le font, au contraire, par une bonne volonté. 
\VS{16}Les uns, dis-je, annoncent Christ par un esprit de dispute, et non pas purement, croyant ajouter de l'affliction à mes liens.
\VS{17}Mais les autres le font par charité, sachant que je suis établi pour la défense de l'Evangile\FTNT{Les versets 16 et 17 sont inversés dans les bibles basées sur les textes minoritaires qui sont moins précis. Le texte majoritaire (byzantin), beaucoup plus fiable et fidèle à la pensée des auteurs bibliques, présente les versets dans cet ordre.}.
\VS{18}Qu'importe~? De toute manière, que ce soit par ostentation, ou par amour de la vérité, Christ n'est pas moins annoncé~: Je me réjouis, et je me réjouirai.
\VS{19}Car je sais que cela tournera à mon salut par vos prières et par le secours de l'Esprit de Jésus-Christ,
\VS{20}selon ma ferme attente et mon espérance, je ne serai confus en rien, mais qu'en toute assurance, Christ sera maintenant, comme il l'a toujours été glorifié dans mon corps, soit par ma vie, soit par ma mort.
\VS{21}Car Christ est ma vie, et la mort m'est un gain.
\VS{22}Mais s'il est utile pour mon œuvre de vivre dans la chair, ce que je dois choisir, je n'en sais rien.
\VS{23}Car je suis pressé des deux côtés~: Mon désir tendant bien à déloger, et à être avec Christ, ce qui me serait beaucoup meilleur~;
\VS{24}mais il est plus nécessaire pour vous que je demeure dans la chair.
\VS{25}Et je suis persuadé, je sais que je demeurerai et que je resterai avec vous tous, pour votre avancement et pour votre joie dans la foi,
\VS{26}afin que vous ayez en moi un sujet de vous glorifier de plus en plus en Jésus-Christ, par mon retour au milieu de vous. 
\VS{27}Seulement, conduisez-vous dignement comme il est séant selon l'Evangile de Christ, afin que soit que je vienne et que je vous voie, soit que je sois absent, j'entende quant à votre état, que vous persistez dans un même esprit, combattant ensemble d'un même courage par la foi de l'Evangile, et n'étant en rien épouvantés par les adversaires.
\VS{28}Ce qui est pour eux une preuve de perdition, mais pour vous de salut~; et cela de la part de Dieu.
\TextTitle{Souffrir pour Christ: Une grâce}
\VS{29}Parce qu'il vous a été gratuitement donné en ce qui concerne Christ, non seulement de croire en lui, mais aussi de souffrir pour lui,
\VS{30}en soutenant le même combat dans lequel vous m'avez vu, et dans lequel vous entendez dire que je suis encore.
\Chap{2}
\TextTitle{Exhortation à l'unité}
\VerseOne{}Si donc il y a quelque consolation en Christ, s'il y a quelque soulagement dans la charité, s'il y a quelque communion d'esprit, s'il y a quelques cordiales affections et quelques compassions,
\VS{2}rendez ma joie parfaite, ayant un même sentiment, un même amour, une même âme, et consentant tous à une même chose.
\VS{3}Ne faites rien par esprit de parti\FTNT{Parti~: «~eritheia~» en grec. Avant la Nouvelle Alliance, ce mot ne se trouve que dans les écrits d'Aristote (philosophe grec, disciple de Platon, né en 384 et mort en 322 av. J.-C) où il dénote une «~recherche personnelle, la poursuite d'une fonction politique par des moyens injustes~». Ce mot signifie aussi «~faire une campagne électorale~» ou «~intriguer pour une fonction dans un esprit partisan, querelleur~». On retrouve le mot grec dans Ja. 3:14,16~; Ga. 5:20~; Ro. 2:8~; Ph. 1:16.}, ou par vaine gloire, mais que l'humilité de cœur vous fasse regarder les autres comme étant au-dessus de vous-mêmes.
\VS{4}Ne regardez pas chacun, à votre intérêt particulier, mais que chacun ait égard aussi à ce qui concerne les autres.
\TextTitle{L'humilité de Christ}
\VS{5}Qu'il y ait donc en vous le même sentiment qui a été en Jésus-Christ, 
\VS{6}lequel étant en forme de Dieu, n'a pas regardé son égalité avec Dieu comme une usurpation.
\VS{7}Cependant il s'est vidé lui-même, ayant pris la forme de serviteur, fait à la ressemblance des hommes~;
\VS{8}et, étant trouvé en apparence comme un homme, il s'est abaissé lui-même, en se rendant obéissant jusqu'à la mort, même jusqu'à la mort de la croix. 
\VS{9}C'est pourquoi aussi Dieu l'a souverainement élevé, et lui a donné le Nom qui est au-dessus de tout nom,
\VS{10}afin qu'au Nom de Jésus, tout genou fléchisse, tant de ceux qui sont dans les cieux, que de ceux qui sont sur la terre, et sous la terre,
\VS{11}et que toute langue confesse que Jésus-Christ est le Seigneur, à la gloire de Dieu le Père.
\VS{12}C'est pourquoi, mes bien-aimés, comme vous avez toujours obéi, mettez en œuvre votre propre salut avec crainte et tremblement, non seulement comme en ma présence, mais beaucoup plus maintenant que je suis absent.
\VS{13}Car c'est Dieu qui produit en vous avec efficacité le vouloir et le faire, selon son bon plaisir.
\VS{14}Faites toutes choses sans murmures et sans disputes,
\VS{15}afin que vous soyez sans reproche et purs, des enfants de Dieu irrépréhensibles au milieu de la génération corrompue et perverse, parmi laquelle vous brillez comme des flambeaux dans le monde, qui portent au devant d'eux la parole de la vie~; 
\VS{16}pour que je puisse me glorifier au jour de Christ de n'avoir pas couru en vain, ni travaillé en vain. 
\VS{17}Et même si je sers de libation sur le sacrifice et sur le service de votre foi, je m'en réjouis, et je m'en réjouis avec vous tous.
\VS{18}Vous aussi pareillement, réjouissez-vous, et réjouissez-vous avec moi.
\TextTitle{Paul témoigne de Timothée et d'Epaphrodite}
\VS{19}Et j'espère avec la grâce du Seigneur Jésus, vous envoyer bientôt Timothée, afin que j'aie aussi plus de courage quand j'aurai connu votre état. 
\VS{20}Car je n'ai personne d'un pareil courage, et qui soit vraiment soigneux de ce qui vous concerne,
\VS{21}parce que tous cherchent leur intérêt particulier, et non les intérêts de Jésus-Christ. 
\VS{22}Mais vous savez l'épreuve que j'ai faite de lui, puisqu'il a servi avec moi en l'Evangile, comme l'enfant sert son père.
\VS{23}J'espère donc vous l'envoyer dès que j'aurai pourvu à mes affaires.
\VS{24}Et j'ai cette confiance en notre Seigneur que moi-même aussi j'irai bientôt.
\VS{25}Mais j'ai cru nécessaire de vous envoyer Epaphrodite, mon frère, mon compagnon d'œuvre et mon compagnon d'armes, par qui vous m'aviez envoyé de quoi pourvoir à mes besoins.
\VS{26}Car aussi il désirait ardemment vous voir tous, et il était fort affligé de ce que vous aviez appris qu'il avait été malade.
\VS{27}En effet, il a été malade, et tout près de la mort~; mais Dieu a eu pitié de lui, et non seulement de lui, mais aussi de moi, afin que je n'aie pas tristesse sur tristesse.
\VS{28}Je l'ai donc envoyé à cause de cela avec plus de soin, afin qu'en le revoyant vous ayez de la joie, et que j'aie moins de tristesse.
\VS{29}Recevez-le donc en notre Seigneur, avec toute sorte de joie~; et ayez de l'estime pour ceux qui sont tels que lui. 
\VS{30}Car il a été proche de la mort pour l'œuvre de Christ, n'ayant eu aucun égard à sa propre vie, afin de suppléer au défaut de votre service envers moi.
\Chap{3}
\TextTitle{Le légalisme et la justice de la loi mosaïque}
\VerseOne{}Au reste, mes frères, réjouissez-vous dans le Seigneur. Je ne me lasse pas de vous écrire les mêmes choses, mais pour vous c'est une sécurité.
\VS{2}Prenez garde aux chiens, prenez garde aux mauvais ouvriers, prenez garde aux faux circoncis.
\VS{3}Car c'est nous qui sommes les circoncis, nous qui rendons à Dieu notre culte en Esprit, et qui nous glorifions en Jésus-Christ, et qui n'avons pas de confiance en la chair.
\VS{4}Moi aussi, cependant, j'aurais sujet de mettre ma confiance en la chair. Si quelqu'un estime qu'il a de quoi se confier en la chair, je le puis bien davantage~:
\VS{5}Moi, circoncis le huitième jour, de la race d'Israël, de la tribu de Benjamin, Hébreu né d'Hébreux, pharisien en ce qui concerne la loi~;
\VS{6}quant au zèle, persécutant l'Eglise~; et quant à la justice à l'égard de la loi, étant sans reproche.
\TextTitle{L'excellence de la connaissance de Jésus-Christ} 
\VS{7}Mais ces choses qui étaient pour moi un gain, je les ai regardées comme une perte à cause de l'amour de Christ.
\VS{8}Et certes, je regarde toutes les autres choses comme m'étant nuisibles en comparaison de l'excellence de la connaissance de Jésus-Christ, mon Seigneur, pour l'amour duquel je me suis privé de toutes ces choses, et je les estime comme du fumier, afin de gagner Christ,
\VS{9}et que je sois trouvé en lui, ayant non pas ma justice qui est de la loi, mais celle qui est par la foi en Christ, c'est-à-dire, la justice qui est de Dieu par la foi.
\VS{10}Ainsi, je connaîtrai Jésus-Christ et la puissance de sa résurrection, et la communion de ses souffrances, en devenant conforme à lui dans sa mort, pour parvenir,
\VS{11}si je puis, à la résurrection d'entre les morts.
\VS{12}Non que j'aie déjà atteint le but, ou que je sois déjà rendu parfait, mais je poursuis ce but pour tâcher d'y parvenir~; c'est pourquoi aussi j'ai été pris par Jésus-Christ.
\VS{13}Mes frères, pour moi, je ne me persuade pas d'avoir atteint le but~;
\VS{14}mais je fais une chose~: Oubliant les choses qui sont en arrière, et me portant vers celles qui sont en avant, je cours vers le but, pour remporter le prix de la vocation céleste de Dieu en Jésus-Christ.
\VS{15}C'est pourquoi, nous tous qui sommes parfaits, ayons ce même sentiment~; et si vous êtes en quelque point d'un autre avis, Dieu vous le révélera aussi.
\VS{16}Cependant, marchons suivant une même règle pour les choses auxquelles nous sommes parvenus, et ayons un même sentiment.
\VS{17}Soyez tous ensemble mes imitateurs, mes frères, et portez les regards sur ceux qui marchent selon le modèle que vous avez en nous.
\VS{18}Car il y en a plusieurs qui marchent d'une telle manière, que je vous ai souvent dit, et maintenant je vous le dis encore en pleurant, qu'ils sont ennemis de la croix de Christ. 
\VS{19}Eux dont la fin est la perdition, qui ont pour dieu leur ventre, et dont la gloire est dans leur confusion, n'ayant d'affection que pour les choses de la terre.
\VS{20}Mais pour nous, notre cité est dans les cieux, d'où nous aussi nous attendons le Sauveur, le Seigneur Jésus-Christ,
\VS{21}qui transformera notre corps vil, afin qu'il soit rendu conforme à son corps glorieux, selon cette efficacité\FTNT{Ep. 3:7.} par laquelle il peut même s'assujettir toutes choses. 
\Chap{4}
\TextTitle{Encouragements}
\VerseOne{}C'est pourquoi, mes très chers frères bien-aimés, vous qui êtes ma joie et ma couronne, demeurez ainsi fermes dans le Seigneur, mes bien-aimés.
\VS{2}J'exhorte Evodie, et j'exhorte aussi Syntyche, à être d'un même sentiment dans le Seigneur.
\VS{3}Et toi aussi, mon vrai compagnon\FTNT{Le mot «~compagnon~» est la traduction du grec «~suzugos~» qui signifie littéralement «~ensemble sous le joug, compagnon de peine, de joug~». Cette expression renvoie à 2 Co. 6:14.}, oui je te prie de les aider, elles qui ont combattu avec moi pour l'Evangile, avec Clément, et mes autres compagnons d'œuvre, dont les noms sont écrits dans le livre de vie.
\VS{4}Réjouissez-vous toujours dans le Seigneur~; je vous le répète, réjouissez-vous~!
\VS{5}Que votre douceur soit connue de tous les hommes. Le Seigneur est proche.
\VS{6}Ne vous inquiétez de rien, mais en toutes choses présentez vos demandes à Dieu par des prières et des supplications, avec des actions de grâces.
\VS{7}Et la paix de Dieu, qui surpasse toute intelligence, gardera vos cœurs et vos sentiments en Jésus-Christ.
\TextTitle{L'objet de nos pensées}
\VS{8} Au reste, mes frères, que toutes les choses qui sont véritables, toutes les choses qui sont vénérables, toutes les choses qui sont justes, toutes les choses qui sont pures, toutes les choses qui sont aimables, toutes les choses qui sont de bonne renommée, toutes celles où il y a quelque vertu et quelque louange~; pensez à ces choses.
\VS{9}Car vous les avez aussi apprises, reçues, entendues et vues en moi. Faites ces choses, et le Dieu de paix sera avec vous. 
\TextTitle{Le contentement dans l'abondance et dans la disette}
\VS{10}Or je me suis fort réjoui en notre Seigneur de ce qu'à la fin vous avez fait revivre le soin que vous aviez pour moi~; à quoi aussi vous pensiez, mais vous n'en aviez pas l'occasion. 
\VS{11}Je ne dis pas ceci à cause de mes besoins, car, moi, j'ai appris à être content en moi-même dans les circonstances où je me trouve.
\VS{12}Je sais être abaissé, je sais aussi être dans l'abondance~; partout et en toutes choses je suis instruit tant à être rassasié qu'à avoir faim, tant à être dans l'abondance que dans la disette.
\VS{13}Je puis toutes choses en Christ qui me fortifie.
\VS{14}Néanmoins, vous avez bien fait de prendre part à mon affliction.
\VS{15}Vous savez aussi, vous Philippiens, qu'au commencement de la prédication de l'Evangile, quand je partis de Macédoine, aucune église ne me communiqua rien en matière de donner et de recevoir, excepté vous seuls. 
\VS{16}Et même lorsque j'étais à Thessalonique, vous m'avez envoyé une fois, et même deux fois, ce dont j'avais besoin. 
\VS{17}Ce n'est pas que je recherche des présents, mais je cherche le fruit qui abonde pour votre compte.
\VS{18}J'ai tout reçu, et je suis dans l'abondance, et j'ai été comblé de biens en recevant d'Epaphrodite ce qui vient de vous, comme un parfum de bonne odeur, comme un sacrifice que Dieu accepte et qui lui est agréable.
\VS{19}Aussi mon Dieu pourvoira à tout ce dont vous aurez besoin selon ses richesses, avec gloire en Jésus-Christ. 
\TextTitle{Salutations}
\VS{20}Or à notre Dieu et Père soit la gloire aux siècles des siècles~! Amen~!
\VS{21}Saluez tous les saints en Jésus-Christ. Les frères qui sont avec moi vous saluent.
\VS{22}Tous les saints vous saluent, et principalement ceux qui sont de la maison de César.
\VS{23}Que la grâce de notre Seigneur Jésus-Christ soit avec vous tous~! Amen~!
\PPE{}
\end{multicols}
