\ShortTitle{Jonas}\BookTitle{Jonas}\BFont
\begin{multicols}{2}
\Chap{1}
\VerseOne{}La parole de Yahweh fut adressée à Jonas, fils d'Amitthaï, en ces mots :
\VS{2}Lève-toi, va à Ninive (1), la grande ville, et crie contre elle ! Car leur malice est montée jusqu'à moi.
\VS{3}Mais Jonas se leva pour s'enfuir à Tarsis, loin de la face de Yahweh. Il descendit à Japho, où il trouva un navire qui allait à Tarsis ; il paya le prix du transport et y entra pour aller à Tarsis, loin de la face de Yahweh.
\VS{4}Mais Yahweh fit lever un grand vent sur la mer, et il y eut une grande tempête sur la mer, de sorte que le navire semblait se briser.
\VS{5}Et les mariniers eurent peur, et ils crièrent chacun à son dieu, et jetèrent dans la mer les objets qui étaient dans le navire, pour l’alléger. Jonas descendit au fond du navire, se coucha et s’endormit profondément.
\VS{6}Le chef des marins s'approcha de lui, et lui dit : Qu’as-tu dormeur ? Lève-toi, invoque ton Dieu ! Peut-être ton Dieu pensera à nous et nous ne périrons pas.
\VS{7}Puis ils se dirent l'un à l'autre : Venez, tirons au sort pour savoir qui est la cause de ce malheur. Ils tirèrent au sort, et le sort tomba sur Jonas.
\VS{8}Alors ils lui dirent : Dis-nous quelle est la cause de ce malheur. Quel est ton métier, et d'où viens-tu ? Quel est ton pays, et de quel peuple es-tu ?
\VS{9}Il leur répondit : Je suis Hébreu, et je crains Yahweh, le Dieu des cieux, qui a fait la mer et la terre sèche.
\VS{10}Alors ces hommes furent saisis d'une grande crainte, et lui dirent : Pourquoi as-tu fait cela ? Car ces hommes savaient qu’il fuyait loin de la face de Yahweh, parce qu'il le leur avait déclaré.
\VS{11}Ils lui dirent : Que te ferons-nous pour que la mer se calme ? Car la mer était de plus en plus agitée.
\VS{12}Il leur répondit : Prenez-moi, et jetez-moi dans la mer, et la mer se calmera ; car je sais que c’est moi la cause de cette grande tempête.
\VS{13}Ces hommes ramaient pour revenir sur la terre sèche, mais ils ne le purent, car la mer s'agitait toujours plus contre eux.
\VS{14}Alors ils invoquèrent Yahweh, et dirent : O Yahweh, ne nous fais pas périr à cause de la vie de cet homme, et ne mets pas sur nous le sang innocent ! Car toi, Yahweh, tu fais comme il te plait (2).
\VS{15}Alors ils prirent Jonas, et le jetèrent dans la mer. Et la fureur de la mer s'arrêta.
\VS{16}Ces hommes furent saisis d’une grande crainte envers Yahweh, et ils offrirent des sacrifices à Yahweh, et firent des vœux.
\VS{17}Yahweh ordonna à un grand poisson d’engloutir Jonas, et Jonas fut dans le ventre du poisson trois jours et trois nuits.
\Chap{2}
\VerseOne{}Jonas pria Yahweh, son Dieu, dans le ventre du poisson.
\VS{2}Il dit : Dans ma détresse j’ai invoqué Yahweh, et il m'a exaucé ; du sein du scheol j’ai crié, et tu as entendu ma voix (1).
\VS{3}Tu m'as jeté dans les profondeurs, au cœur de la mer, et le courant m'a environné ; tous tes flots et toutes tes vagues ont passé sur moi.
\VS{4}Je disais : Je suis chassé loin de tes yeux ! Cependant je verrai encore le temple de ta sainteté.
\VS{5}Les eaux m'ont environné jusqu'à l'âme. L'abîme m'a enveloppé, les roseaux ont lié ma tête.
\VS{6}Je suis descendu jusqu'aux bases des montagnes, la terre fermait sur moi ses barres pour toujours ; mais tu m’as fait remonter vivant de la fosse, Yahweh, mon Dieu !
\VS{7}Quand mon âme s’était affaiblie en moi, je me suis souvenu de Yahweh, et ma prière est parvenue jusqu’à toi, dans le temple de ta sainteté.
\VS{8}Ceux qui s’adonnent à des vanités mensongères abandonnent ta miséricorde.
\VS{9}Mais moi, je t’offrirai des sacrifices avec un cri de louange, j’accomplirai les vœux que j’ai faits : Car le salut vient de Yahweh (2).
\VS{10}Alors Yahweh parla au poisson, et le poisson vomit Jonas sur la terre sèche.
\Chap{3}
\VerseOne{}La parole de Yahweh fut adressée à Jonas une seconde fois, en disant :
\VS{2}Lève-toi, va à Ninive, la grande ville, et proclames-y à haute voix ce que je t'ordonne !
\VS{3}Jonas se leva, et alla à Ninive, suivant la parole de Yahweh. Or Ninive était une grande ville devant Dieu, de trois jours de marche.
\VS{4}Jonas commença dans la ville le chemin d'une journée de marche ; il criait et disait : Encore quarante jours, et Ninive sera renversée !
\VS{5}Les hommes de Ninive crurent à Dieu, ils publièrent un jeûne, et se vêtirent de sacs, depuis le plus grand d'entre eux jusqu'au plus petit.
\VS{6}Cette parole parvint au roi de Ninive ; il se leva de son trône, ôta de dessus lui son manteau, se couvrit d'un sac, et s'assit sur la cendre.
\VS{7}Puis il fit faire une proclamation, et publier dans Ninive par décret du roi et de ses grands : Que les hommes, les bêtes, les bœufs et les brebis, ne goûtent de rien, ne paissent point, et ne boivent point d'eau !
\VS{8}Que les hommes et les bêtes soient couverts de sacs, qu'ils crient à Dieu avec force, et que chacun revienne de sa mauvaise voie, et des actions violentes que ses mains ont commises !
\VS{9}Qui sait si Dieu ne reviendra pas et ne se repentira pas, et s'il ne se détournera pas de son ardente colère, en sorte que nous ne périssions point ?
\VS{10}Dieu vit ce qu’ils faisaient et comment ils revenaient de leur mauvaise voie. Alors Dieu se repentit du mal qu'il avait déclaré de leur faire, et il ne le fit point.
\Chap{4}
\VerseOne{}Mais cela déplut fortement à Jonas, et il fut furieux.
\VS{2}Il pria Yahweh, et dit : Oh ! Yahweh, n'est-ce pas là ce que je disais quand j'étais encore dans mon pays ? C'est pourquoi j'ai voulu m'enfuir à Tarsis. Car je savais que tu es un Dieu compatissant, miséricordieux, lent à la colère et riche en bonté, et qui te repens du mal (1).
\VS{3}Maintenant, Yahweh, prends-moi donc la vie, car la mort m'est meilleure que la vie.
\VS{4}Et Yahweh répondit : Fais-tu bien de te mettre en colère ?
\VS{5}Alors Jonas sortit de la ville, et s'assit à l'orient de la ville, là il se fit une cabane, et y resta à l'ombre, jusqu'à ce qu'il vît ce qui arriverait à la ville.
\VS{6}Yahweh Dieu ordonna à un ricin de croître au-dessus de Jonas, pour donner de l’ombre sur sa tête et pour le délivrer de son mal. Jonas éprouva une grande joie à cause de ce ricin.
\VS{7}Mais le lendemain, à l’aurore, Dieu ordonna à un ver d’attaquer le ricin, et le ricin sécha.
\VS{8}Au lever du soleil, Dieu ordonna à un vent chaud d’orient de souffler, et le soleil frappa la tête de Jonas, au point qu’il s’évanouit. Il demanda la mort, et dit : La mort m'est meilleure que la vie.
\VS{9}Dieu dit à Jonas : Fais-tu bien de te mettre en colère à cause du ricin ? Et il répondit : Je fais bien de m’irriter jusqu’à la mort.
\VS{10}Et Yahweh dit : Tu as pitié du ricin pour lequel tu n'as point travaillé et que tu n'as point fait croître, qui est né dans une nuit et qui a péri dans une nuit.
\VS{11}Et moi, je n’aurais pas pitié de Ninive, la grande ville, dans laquelle il y a plus de cent vingt mille personnes qui ne savent point distinguer leur main droite de leur main gauche, et des animaux en grand nombre !
\PPE{}
\end{multicols}