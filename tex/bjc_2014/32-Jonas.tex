\ShortTitle{Jon.}\BookTitle{Jonas}\BFont
\noindent\hrulefill
{\footnotesize
\textit{
\bigskip
{\centering{}
\\Auteur~: Jonas
\\(Heb.~: Yonah)
\\Signification~: Colombe
\\Thème~: La miséricorde divine
\\Date de rédaction~: 8\up{ème} siècle av. J.-C.\\}
}
\textit{
\\Jonas, contemporain d'Amos, exerça son service dans le royaume du nord sous le règne de Jéroboam II. Alors que Yahweh lui ordonna d'aller prêcher la repentance à Ninive, sous peine de la détruire, il refusa et tenta de s'enfuir loin de lui. Toutefois, dans sa miséricorde, Dieu généra une situation pour pousser Jonas à accomplir sa mission. A travers ce récit, on comprend que la voie de la repentance est ouverte à tous et que Dieu veut sauver tous les hommes sans exception.\bigskip
}
}
\par\nobreak\noindent\hrulefill
\begin{multicols}{2}
\Chap{1}
\VerseOne{}Or la parole de Yahweh vint à Jonas, fils d'Amitthaï, en disant~:
\TextTitle{Désobéissance et fuite de Jonas}
\VS{2}Lève-toi, va à Ninive\FTNT{Ninive était la capitale de l'empire d'Assyrie.}, la grande ville, et crie contre elle~! Car leur malice est montée jusqu'à moi.
\VS{3}Mais Jonas se leva pour s'enfuir à Tarsis, loin de la face de Yahweh et il descendit à Japho, où il trouva un navire qui allait à Tarsis~; il paya le prix du transport et y entra pour aller à Tarsis, loin de la face de Yahweh.
\VS{4}Mais Yahweh fit lever un grand vent sur la mer, et il y eut une grande tempête sur la mer, de sorte que le navire semblait se briser.
\VS{5}Et les mariniers eurent peur, et ils crièrent chacun à son dieu, et jetèrent dans la mer les objets qui étaient dans le navire pour l'alléger. Jonas descendit au fond du navire, se coucha et s'endormit profondément.
\VS{6}Le chef des marins s'approcha de lui et lui dit~: Qu'as-tu dormeur~? Lève-toi, crie à ton Dieu~! Peut-être ton Dieu pensera à nous et nous ne périrons pas.
\VS{7}Puis ils se dirent l'un à l'autre~: Venez, tirons au sort pour savoir qui est la cause de ce malheur. Ils tirèrent au sort et le sort tomba sur Jonas.
\VS{8}Alors ils lui dirent~: Dis-nous quelle est la cause de ce malheur. Quel est ton métier et d'où viens-tu~? Quel est ton pays et de quel peuple es-tu~?
\VS{9}Et il leur répondit~: Je suis Hébreu et je crains Yahweh, le Dieu des cieux, qui a fait la mer et la terre sèche.
\VS{10}Alors ces hommes furent saisis d'une grande crainte, et lui dirent~: Pourquoi as-tu fait cela~? Car ces hommes savaient qu'il fuyait loin de la face de Yahweh, parce qu'il le leur avait déclaré.
\VS{11}Et ils lui dirent~: Que te ferons-nous pour que la mer se calme~? Car la mer était de plus en plus agitée.
\TextTitle{Jonas dans le ventre du poisson}
\VS{12}Il leur répondit~: Prenez-moi, et jetez-moi dans la mer, et la mer se calmera~; car je sais que c'est moi la cause de cette grande tempête.
\VS{13}Et ces hommes ramaient pour revenir sur la terre sèche, mais ils ne le purent, car la mer s'agitait toujours plus contre eux.
\VS{14}Alors ils invoquèrent Yahweh, et dirent~: Ô Yahweh, ne nous fais pas périr à cause de la vie de cet homme, et ne mets pas sur nous le sang innocent~! Car toi, Yahweh, tu fais comme il te plaît\FTNT{Ps. 115:3.}.
\VS{15}Alors ils prirent Jonas, et le jetèrent dans la mer. Et la fureur de la mer s'arrêta.
\VS{16}Et ces hommes furent saisis d'une grande crainte envers Yahweh, et ils offrirent des sacrifices à Yahweh, et firent des vœux.
\Chap{2}
\TextTitle{La prière de Jonas exaucée par Yahweh}
\VerseOne{}Or Yahweh ordonna à un grand poisson d'engloutir Jonas, et Jonas fut dans le ventre du poisson trois jours et trois nuits.
\VS{2}Jonas pria Yahweh, son Dieu, dans le ventre du poisson.
\VS{3}Et il dit~: Dans ma détresse, j'ai invoqué Yahweh, et il m'a exaucé~; du sein du scheol j'ai crié, et tu as entendu ma voix\FTNT{Ps. 18:5-7.}.
\VS{4}Tu m'as jeté dans les profondeurs, au cœur de la mer, et le courant m'a environné~; tous tes flots et toutes tes vagues ont passé sur moi.
\VS{5}Je disais~: Je suis chassé loin de tes yeux~! Cependant je verrai encore le temple de ta sainteté.
\VS{6}Les eaux m'ont environné jusqu'à l'âme. L'abîme m'a enveloppé, les roseaux ont lié ma tête.
\VS{7}Je suis descendu jusqu'aux bases des montagnes, la terre fermait sur moi ses barres pour toujours~; mais tu m'as fait remonter vivant de la fosse, Yahweh, mon Dieu~!
\VS{8}Quand mon âme s'était affaiblie en moi, je me suis souvenu de Yahweh, et ma prière est parvenue à toi, jusqu'au palais de ta sainteté.
\VS{9}Ceux qui s'adonnent à des vanités mensongères abandonnent ta miséricorde.
\VS{10}Mais moi, je t'offrirai des sacrifices avec un cri de louange, j'accomplirai les vœux que j'ai faits~: Car le salut vient de Yahweh\FTNT{Le mot «~salut~», de l'hébreu «~Yeshuw'ah~», a la même racine que le nom de notre Seigneur~: «~Jésus~». Jonas a compris que seul Jésus pouvait le sauver d'une mort certaine.}.
\VS{11}Alors Yahweh parla au poisson, et le poisson vomit Jonas sur la terre sèche.
\Chap{3}
\TextTitle{Repentance nationale à Ninive}
\VerseOne{}La parole de Yahweh vint à Jonas une seconde fois, en disant~:
\VS{2}Lève-toi, va à Ninive, la grande ville, et proclames-y à haute voix ce que je t'ordonne~!
\VS{3}Jonas se leva et alla à Ninive, suivant la parole de Yahweh. Or Ninive était une grande ville devant Dieu, de trois jours de marche.
\VS{4}Et Jonas commença dans la ville le chemin d'une journée de marche~; il criait et disait~: Encore quarante jours, et Ninive sera renversée~!
\VS{5}Les hommes de Ninive crurent à Dieu, ils publièrent un jeûne et se vêtirent de sacs, depuis le plus grand d'entre eux jusqu'au plus petit.
\VS{6}Et cette parole parvint au roi de Ninive~; il se leva de son trône, ôta de dessus lui son manteau, se couvrit d'un sac et s'assit sur la cendre.
\VS{7}Puis il fit faire une proclamation, et publier dans Ninive par décret du roi et de ses grands~: Que les hommes, les bêtes, les bœufs et les brebis, ne goûtent de rien, ne paissent point, et ne boivent point d'eau~!
\VS{8}Et que les hommes soient couverts de sacs, et les bêtes aussi, qu'ils crient à Dieu avec force, et que chacun revienne de sa mauvaise voie, des actions violentes que ses mains ont commises~!
\VS{9}Qui sait si Dieu viendra à se repentir, s'il se détournera de l'ardeur de sa colère, en sorte que nous ne périssions point~?
\VS{10}Et Dieu regarda à ce qu'ils avaient fait, comment ils s'étaient détournés de leur mauvaise voie. Alors Dieu se repentit du mal qu'il avait déclaré de leur faire, et il ne le fit point.
\Chap{4}
\TextTitle{La miséricorde est accordée à Ninive~; Jonas mécontent}
\VerseOne{}Mais cela déplut fortement à Jonas, et il fut en colère.
\VS{2}Il pria Yahweh et dit~: Oh~! Yahweh, n'est-ce pas là ce que je disais quand j'étais encore dans mon pays~? C'est pourquoi j'ai voulu m'enfuir à Tarsis. Car je savais que tu es un Dieu compatissant, miséricordieux, lent à la colère et riche en bonté, et qui te repens du mal\FTNT{Joë. 2:13.}.
\VS{3}Maintenant donc, Yahweh, prends-moi donc la vie, car la mort vaut mieux pour moi que la vie.
\TextTitle{Yahweh réprimande Jonas}
\VS{4}Et Yahweh répondit~: Fais-tu bien de te mettre en colère~?
\VS{5}Alors Jonas sortit de la ville et s'assit à l'orient de la ville, là il se fit une cabane et y resta à l'ombre, jusqu'à ce qu'il vît ce qui arriverait à la ville.
\VS{6}Yahweh Dieu ordonna à un ricin de croître au-dessus de Jonas, pour donner de l'ombre sur sa tête et pour le délivrer de son mal. Jonas éprouva une grande joie à cause de ce ricin.
\VS{7}Puis Dieu prépara pour le lendemain, lorsque l'aube du jour monta, un ver qui frappa le ricin, et il sécha.
\VS{8}Et il arriva que quand le soleil fut levé, Dieu prépara un vent chaud d'orient qu'on n'apercevait point, et le soleil frappa la tête de Jonas, au point qu'il s'évanouit. Il demanda la mort et dit~: La mort m'est meilleure que la vie.
\VS{9}Dieu dit à Jonas~:Fais-tu bien de t'être ainsi mis en colère au sujet de ce ricin~? Et il répondit~: Je fais bien de m'irriter jusqu'à la mort.
\VS{10}Et Yahweh dit~: Tu as pitié du ricin pour lequel tu n'as point travaillé et que tu n'as point fait croître, qui est né dans une nuit et qui a péri dans une nuit.
\VS{11}Et moi, n'épargnerais-je point Ninive, cette grande ville, dans laquelle il y a plus de cent vingt mille hommes qui ne savent point distinguer leur main droite de leur main gauche, et où il y a aussi une grande quantité de bêtes?
\PPE{}
\end{multicols}
