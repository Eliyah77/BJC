\ShortTitle{Lamentations de Jérémie}\BookTitle{Lamentations de Jérémie}\BFont
\noindent\hrulefill
{\footnotesize
\textit{
\bigskip
{\centering{}
\\Auteur : Jérémie
\\(Heb. : Eikha)
\\Signification : Où ?
\\Thème : Affliction pour Jérusalem
\\Date de rédaction : 6ème siècle av. J.-C\\}
}
%\bigskip
\textit{
\\Recueil de pièces poétiques, les lamentations de Jérémie furent composées selon un procédé visant à accentuer le caractère funèbre, de façon à ce qu'elles soient récitées avec gémissements. Ses complaintes exposent la profonde désolation du prophète face au fardeau du peuple qu'il portait dans ses entrailles tout comme la douleur et la tristesse de Yahweh face à Israël.
%\bigskip
\\Très différentes des prophéties retrouvées dans le livre de Jérémie, les Lamentations reflètent l'affliction convenant à la
gravité du châtiment subi : famine, pillage et ruine du temple, déportation, cessation du culte, diverses calamités… Jérémie rappelle ainsi les conséquences de l'endurcissement du cœur face aux appels à la repentance ; il présente aussi les bontés éternelles de Yahweh.\bigskip
}
}
\par\nobreak\noindent\hrulefill
\begin{multicols}{2}
\Chap{1}
\TextTitle{Pleurs et désolation de Jérusalem}
\VerseOne{}[Aleph.] Comment est-il arrivé que la ville si peuplée se trouve si solitaire ? Que celle qui était grande entre les nations est devenue comme une veuve ? Que celle qui était noble dame entre les provinces a été rendue tributaire ?
\VS{2}[Beth.] Elle ne cesse de pleurer pendant la nuit, et ses larmes sont sur joues ; il n'y a pas un de tous ses amis qui la console ; ses intimes amis ont agi perfidement contre elle, ils sont devenus ses ennemis.
\VS{3}[Guimel.] Juda a été emmenée captive tant elle est affligée, et tant est grande sa servitude ; elle demeure maintenant entre les nations, et ne trouve point de repos ; tous ses persécuteurs l'ont attrapée dans sa détresse\FTNT{Jé. 52:26.}.
\VS{4}[Daleth.] Les chemins de Sion sont dans le deuil, plus personne ne vient aux fêtes solennelles ; toutes ses portes sont dans la désolation, ses sacrificateurs gémissent, ses vierges sont affligées de tristesse ; et elle est remplie d'amertume.
\VS{5}[He.] Ses adversaires sont établis pour chefs, ses ennemis prospèrent ; car Yahweh l'a humiliée à cause de la multitude de ses transgressions ; ses petits enfants ont marché captifs devant l'adversaire\FTNT{Jé. 30:14.}.
\VS{6}[Vav.] Et tout l'honneur de la fille de Sion s'est retiré d'elle ; ses chefs sont devenus semblables à des cerfs qui ne trouvent pas de pâture, et qui fuient sans force devant celui qui les poursuit.
\VS{7}[Zayin.] Jérusalem dans les jours de son affliction et de son pauvre état s'est souvenue de toutes ses choses précieuses qu'elle avait depuis si longtemps, lorsque son peuple est tombé par la main de l'ennemi, sans aucun secours ; les ennemis l'ont vue, et se sont moqués de ses sabbats.
\VS{8}[Heth.] Jérusalem a grièvement péché ; c'est pourquoi elle est devenue un objet de dégoût ; tous ceux qui l'honoraient l'ont méprisée parce qu'ils ont vu son ignominie ; elle en a aussi sangloté, et s'est retournée en arrière.
\VS{9}[Teth.] Sa souillure était dans les pans de sa robe, et elle ne s'est pas souvenue de sa dernière fin ; elle a été extraordinairement abaissée, et elle n'a pas de consolateur. Vois ma misère, ô Yahweh ! Car l'ennemi s'est élevé avec orgueil !
\VS{10}[Yod.] L'oppresseur a étendu sa main sur tout ce qu'elle avait de précieux ; car elle a vu entrer dans son sanctuaire les nations auxquelles tu avais défendu d'entrer dans ton assemblée\FTNT{De. 23:3.}.
\VS{11}[Kaf.] Tout son peuple soupire, cherchant du pain\FTNT{Jé. 52:6.} ; ils ont donné leurs choses précieuses pour des aliments, afin de faire revenir leur vie. Vois, ô Yahweh ! Regarde comme je suis vile.
\VS{12}[Lamed.] Cela ne vous touche-t-il point ? Vous tous passants, contemplez, et voyez s'il est une douleur comme ma douleur, celle dont j'ai été frappée ! Moi que Yahweh a accablée de douleur au jour de l'ardeur de sa colère.
\VS{13}[Mem.] Il a envoyé d'en haut, dans mes os, un feu qui les domine ; il a tendu un filet sous mes pieds, et m'a fait revenir en arrière ; il m'a mise dans la désolation, dans une langueur de tous les jours.
\VS{14}[Nun.] Le joug de mes iniquités est lié par sa main ; elles sont entrelacées, et appliquées sur mon cou ; il a renversé ma force ; le Seigneur m'a livrée entre les mains de ceux contre qui je ne pourrai pas me lever.
\VS{15}[Samech.] Le Seigneur a abattu tous les hommes forts que j'avais au milieu de moi ; il a appelé contre moi, au temps fixé, une armée pour détruire mes jeunes hommes ; le Seigneur a foulé au pressoir la vierge, fille de Juda.
\VS{16}[Ayin.] À cause de ces choses, je pleure, mes yeux fondent en larmes ; car le consolateur qui restaurait ma vie est loin de moi. Mes fils sont dans la désolation parce que l'ennemi a été plus fort.
\VS{17}[Pe.] Sion a étendu les mains, et personne ne l'a consolée ; Yahweh a ordonné aux ennemis de Jacob de l'entourer de toutes parts. Jérusalem a été comme une impureté au milieu d'eux.
\VS{18}[Tsade.] Yahweh est juste car j'ai été rebelle à ses ordres. Ecoutez, vous tous, peuples, et voyez ma douleur ! Mes vierges et mes jeunes hommes sont allés en captivité.
\VS{19}[Qof.] J'ai appelé mes amis, mais ils m'ont trompé. Mes sacrificateurs et mes anciens sont morts dans la ville : Ils cherchaient de la nourriture afin de restaurer leur vie.
\VS{20}[Resh.] Vois ma détresse, ô Yahweh ! Mes entrailles bouillonnent, mon cœur est agité au dedans de moi car j'ai été rebelle. Au dehors l'épée m'a privée d'enfants, dans ma maison il y a comme la mort.
\VS{21}[Shin.] On a entendu mes soupirs, et personne ne m'a consolée ; tous mes ennemis ont entendu mon malheur et se sont réjouis de ce que tu l'as causé ; tu viendras, tu proclameras le jour où ils seront comme moi.
\VS{22}[Tav.] Que toute leur méchanceté vienne devant ta face, et traite-les avec sévérité comme tu m'as traité avec sévérité à cause de toutes mes transgressions ! Car mes soupirs sont nombreux, et mon cœur est souffrant.
\Chap{2}
\TextTitle{Le jour de la colère de Yahweh}
\VerseOne{}[Aleph.] Comment est-il arrivé que le Seigneur a couvert de sa colère la fille de Sion tout à l'entour, comme d'une nuée, et qu'il a précipité du ciel sur la terre la beauté d'Israël, et ne s'est pas souvenu du marchepied de ses pieds\FTNT{Ez. 43:7.} au jour de sa colère ?
\VS{2}[Beth.] Le Seigneur a englouti sans épargner toutes les habitations de Jacob ; il a dans sa fureur renversé les forteresses de la fille de Juda, il les a jetées par terre ; il a profané le royaume et ses chefs.
\VS{3}[Guimel.] Il a retranché toute la force d'Israël par l'ardeur de sa colère ; il a retiré sa droite en arrière devant l'ennemi ; il s'est allumé dans Jacob comme un feu flamboyant qui le consume de toutes parts.
\VS{4}[Daleth.] Il a tendu son arc comme un ennemi ; sa droite s'est dressée comme celle d'un adversaire ; il a tué tout ce qui était agréable à l'œil dans la tente de la fille de Sion ; il a répandu sa fureur comme un feu.
\VS{5}[He.] Le Seigneur a été comme un ennemi ; il a englouti Israël, il a englouti tous ses palais, il a détruit toutes ses forteresses ; il a multiplié chez la fille de Juda le deuil et les afflictions.
\VS{6}[Vav.] Il a mis en pièces avec violence sa tente comme un jardin ; il a détruit le lieu de son assemblée ; Yahweh a fait oublier dans Sion la fête solennelle et le sabbat, et dans sa violente colère, il a rejeté le roi et le sacrificateur.
\VS{7}[Zayin.] Le Seigneur a rejeté au loin son autel, il a dédaigné son sanctuaire ; il a livré entre les mains de l'ennemi les murailles de ses palais ; ils ont poussé des cris dans la maison de Yahweh, comme aux jours des fêtes solennelles.
\VS{8}[Heth.] Yahweh avait projeté de détruire les murailles de la fille de Sion ; il a étendu le cordeau, il n'a pas fait revenir sa main sans les avoir engloutis ; il a plongé dans le deuil remparts et murailles, ils ont été ruinés tous ensemble.
\VS{9}[Teth.] Ses portes sont enfoncées dans la terre ; il en a détruit et brisé les barres. Son roi et ses chefs sont parmi les nations ; la loi n'est plus. Même les prophètes ne reçoivent plus aucune vision de Yahweh\FTNT{Ez. 7:26.}.
\VS{10}[Yod.] Les anciens de la fille de Sion sont assis à terre, ils sont muets ; ils ont couvert leur tête de poussière, ils se sont ceints de sacs ; les vierges de Jérusalem baissent leurs têtes vers la terre.
\VS{11}[Kaf.] Mes yeux se consument à force de larmes, mes entrailles bouillonnent, ma bile se répand sur la terre. À cause des ruines de la fille de mon peuple, des enfants et des nourrissons qui tombent en défaillance dans les rues de la ville.
\VS{12}[Lamed.] Ils disaient à leurs mères : Où y a-t-il du blé et du vin ? Et ils tombaient comme morts dans les rues de la ville, comme un homme blessé à mort, ils rendaient l'âme sur le sein de leurs mères.
\VS{13}[Mem] Qui dois-je prendre à témoin ? À qui te comparer, fille de Jérusalem ? Qui pourrait t'égaler, et quelle consolation te donner, vierge, fille de Sion ? Car ta ruine est grande comme une mer : Qui pourrait te guérir\FTNT{Es. 51:19-20.} ?
\VS{14}[Nun.] Tes prophètes ont eu pour toi des visions vaines et insensées ; ils n'ont pas découvert ton iniquité, afin de te faire revenir captif en captivité ; ils t'ont prophétisé des oracles mensongers et trompeurs\FTNT{Jé. 2:8 ; Jé. 5:31 ; Jé. 14:14.}.
\VS{15}[Samech.] Tous les passants applaudissent sur toi, ils sifflent, ils secouent leur tête contre la fille de Jérusalem : Est-ce ici la ville de laquelle on disait : La parfaite en beauté, la joie de toute la terre\FTNT{Na. 3:19.} ?
\VS{16}[Pe.] Tous tes ennemis ouvrent la bouche contre toi, ils sifflent, ils grincent des dents, ils disent : Nous l'avons engloutie ! C'est ici le jour que nous attendions, nous l'avons atteint, nous le voyons !
\VS{17}[Ayin.] Yahweh a fait ce qu'il avait projeté, il a accompli sa parole qu'il avait ordonnée depuis longtemps, il a détruit sans épargner, il a fait de toi la joie de l'ennemi, il a donné de la force à tes adversaires.
\VS{18}[Tsade.] Leur cœur crie au Seigneur… Muraille de la fille de Sion, fais couler des larmes jour et nuit, comme un torrent\FTNT{Jé. 14:17.} ! Ne te donne pas de repos ; et que la prunelle de tes yeux ne se repose pas !
\VS{19}[Qof.] Lève-toi, pousse des cris dès le commencement des veilles de la nuit ! Répands ton cœur comme de l'eau en présence du Seigneur ! Lève tes mains vers lui pour l'âme de tes enfants qui meurent de faim aux coins de toutes les rues !
\VS{20}[Resh.] Vois, ô Yahweh ! Regarde qui tu as traité avec sévérité ! Les femmes n'ont-elles pas mangé leur fruit : leurs petits enfants objets de leur tendresse ? Le sacrificateur et le prophète n'ont-ils pas été tués dans le sanctuaire du Seigneur\FTNT{Lé. 26:29 ; De. 28:53 ; Jé. 19:9.} ?
\VS{21}[Shin.] Les jeunes gens et les vieillards sont couchés par terre dans les rues ; mes vierges et mes jeunes hommes sont tombés par l'épée ; tu as tué au jour de ta colère, tu as massacré sans épargner.
\VS{22}[Tav.] Tu as convié comme pour un jour solennel mes frayeurs de toutes parts. Au jour de la colère de Yahweh, il n'y a eu ni réchappé ni survivant. Ceux que j'avais langés et élevés, mon ennemi les a consumés.
\Chap{3}
\TextTitle{Jérémie partage l'affliction des siens}
\VerseOne{}[Aleph.] Je suis l'homme qui a vu l'affliction par la verge de sa fureur\FTNT{Jé. 15:15-18.}.
\VS{2}Il m'a conduit, mené dans les ténèbres, et non dans la lumière.
\VS{3}Certes c'est contre moi qu'il a tout le jour tourné et retourné sa main.
\VS{4}[Beth.] Il a fait vieillir ma chair et ma peau, il a brisé mes os\FTNT{Es. 38:13.}.
\VS{5}Il a bâti autour de moi, il m'a environné de venin et de peine.
\VS{6}Il me fait habiter dans les lieux ténèbreux, comme ceux qui sont morts depuis longtemps.
\VS{7}[Guimel.] Il a fait une cloison autour de moi, afin que je ne sorte point ; il a appesanti mes chaînes.
\VS{8}Même quand je crie et que j'élève ma voix, il rejette ma prière.
\VS{9}Il a fait un mur de pierres de taille pour fermer mes chemins, il a renversé mes sentiers.
\VS{10}[Daleth.] Il a été pour moi un ours en embuscade, un lion qui se tient dans un lieu caché\FTNT{Os. 13:8.}.
\VS{11}Il a détourné mes chemins, il m'a mis en pièces, il m'a mis dans la désolation.
\VS{12}Il a tendu son arc, et il m'a placé comme une cible pour sa flèche.
\VS{13}[He.] Il a fait entrer dans mes reins les flèches de son carquois.
\VS{14}Je suis la risée pour tout mon peuple, et leur chanson\FTNT{Ps. 69:13 ; Job. 30:9.} tout le jour.
\VS{15}Il m'a rassasié d'amertume, il m'a enivré d'absinthe.
\VS{16}[Vav.] Il a brisé mes dents avec du gravier, il m'a couvert de cendres.
\VS{17}Tellement que la paix s'est éloignée de mon âme, j'ai oublié ce que c'est que d'être à son aise.
\VS{18}Et j'ai dit : Ma force est perdue, et mon espérance aussi que j'avais en l'Eternel.
\VS{19}[Zayin.] Souviens-toi de mon affliction, et de mon pauvre état qui n'est qu'absinthe et que fiel ;
\VS{20}Mon âme s'en souvient sans cesse, et elle est abattue au-dedans de moi.
\VS{21}Mais je rappellerai ceci en mon coeur, et c'est pourquoi j'aurai de l'espérance :
\VS{22}[Heth.] Ce sont les bontés de Yahweh que nous n'avons point été consumés parce que ses compassions ne sont pas taries\FTNT{Ps. 103:10.} ;
\VS{23}elles se renouvellent chaque matin. C'est une chose grande que ta fidélité !
\VS{24}Yahweh est ma portion, dit mon âme ; c'est pourquoi j'aurai espérance en lui\FTNT{Ps. 16:5.}.
\VS{25}[Teth.] Yahweh est bon pour ceux qui s'attendent à lui, pour l'âme qui le cherche.
\VS{26}Il est bon d'espérer et d'attendre en silence la délivrance de Yahweh.
\VS{27}Il est bon pour l'homme de porter le joug dans sa jeunesse.
\VS{28}[Yod.] Il sera assis solitaire et silencieux parce qu'on le lui impose.
\VS{29}Il mettra sa bouche dans la poussière, peut-être y aura-t-il quelque espérance ?
\VS{30}Il présentera la joue à celui qui le frappe, il se rassasiera d'opprobres.
\VS{31}[Kaf.] Car le Seigneur ne rejette pas à toujours\FTNT{Es. 57:16 ; Ps. 77:8.}.
\VS{32}Mais s'il afflige quelqu'un, il a aussi compassion selon la grandeur de sa miséricorde.
\VS{33}Car ce n'est pas sa volonté d'affliger et d'humilier les fils des hommes.
\VS{34}[Lamed.] Lorsqu'on foule aux pieds tous les prisonniers de la terre,
\VS{35}lorsqu'on pervertit la justice humaine en la présence du Très-Haut,
\VS{36}lorsqu'on fait tort à quelqu'un dans son procès, le Seigneur ne le voit-il pas ?
\VS{37}[Mem.] Qui est-ce qui dit qu'une chose est arrivée sans que le Seigneur l'ait commandé ?
\VS{38}Les maux et les biens\FTNT{Es. 45:7 ; Am. 3:6 ; Job. 1:21.} ne procèdent-ils pas de la bouche du Très-Haut ?
\VS{39}Pourquoi un homme vivant se plaindrait-il, un homme, à cause de la peine de ses péchés ?
\TextTitle{Le peuple appelé à s'examiner pour revenir à Yahweh}
\VS{40}[Nun.] Recherchons nos voies, sondons-les, et retournons à Yahweh\FTNT{Ps. 119:59 ; 2 Co. 13:5.} ;
\VS{41}élevons nos cœurs et nos mains vers Dieu qui est au ciel :
\VS{42}Nous avons péché, nous avons été rebelles ! Tu n'as pas pardonné !
\VS{43}[Samech.] Tu nous as couverts de ta colère, et tu nous as poursuivis ; tu as tué sans épargner ;
\VS{44}tu t'es couvert d'une nuée pour que les prières ne te parviennent pas.
\VS{45}Tu nous as fait être la raclure et le rebut au milieu des peuples.
\VS{46}[Pe.] Tous nos ennemis ouvrent leur bouche contre nous.
\VS{47}La frayeur et la fosse, le dégât et la calamité nous sont arrivés\FTNT{Es. 24:18 ; Jé. 48:44.}.
\VS{48}De mes yeux coulent des torrents d'eau à cause de la ruine de la fille de mon peuple.
\VS{49}[Ayin.] Mon œil fond en larmes, sans repos, sans relâche,
\VS{50}jusqu'à ce que Yahweh regarde et voie des cieux\FTNT{Ps. 80:15 ; Ps. 102:20.} ;
\VS{51}mon œil fait souffrir mon âme à cause de toutes les filles de ma ville.
\TextTitle{Yahweh, le soutien de Jérémie dans la détresse}
\VS{52}[Tsade.] Ceux qui sont mes ennemis sans cause m'ont poursuivi à outrance, comme après un oiseau.
\VS{53}Ils ont voulu anéantir ma vie dans une fosse, et ils ont jeté une pierre sur moi.
\VS{54}Les eaux ont coulé par-dessus ma tête ; je disais : Je suis retranché !
\VS{55}[Qof.] J'ai invoqué ton nom, ô Yahweh, du fond de la fosse\FTNT{Jé. 38:6.}.
\VS{56}Tu as entendu ma voix : Ne ferme pas tes oreilles à mes soupirs, à mes cris !
\VS{57}Au jour où je t'ai invoqué, tu t'es approché, et tu as dit : Ne crains rien !
\VS{58}[Resh.] Ô Seigneur, tu as plaidé la cause de mon âme, tu as racheté ma vie.
\VS{59}Tu as vu, ô Yahweh ! le tort qu'on me fait, fais-moi justice !
\VS{60}Tu as vu toutes les vengeances dont ils ont usé, et toutes leurs machinations contre moi.
\VS{61}[Shin.] Yahweh, tu as entendu leurs outrages, toutes leurs machinations contre moi,
\VS{62}les discours de ceux qui se lèvent contre moi, et leur dessein qu'ils ont contre moi tout au long du jour.
\VS{63}Considère quand ils sont assis et quand ils se lèvent, car je suis leur chanson.
\VS{64}[Tav.] Rends-leur la pareille, ô Yahweh, selon l'œuvre de leurs mains ;
\VS{65}livre-les à l'endurcissement de leur cœur, à ta malédiction.
\VS{66}Poursuis-les dans ta colère, et extermine-les de dessous les cieux, ô Yahweh !
\Chap{4}
\TextTitle{Crimes et apostasie du peuple}
\VerseOne{}[Aleph.] Comment l'or est-il devenu obscur, et le fin or s'est-il altéré ? Comment les pierres du sanctuaire sont-elles répandues aux coins de toutes les rues ?
\VS{2}[Beth.] Comment les chers fils de Sion, qui étaient estimés à l'égal de l'or pur, sont-ils reputés comme des vases de terre, ouvrage des mains du potier !
\VS{3}[Guimel.] Il y a même des monstres marins qui présentent leurs mamelles et allaitent leurs petits ; mais la fille de mon peuple est devenue cruelle comme les autruches du désert.
\VS{4}[Daleth.] La langue de celui qui têtait s'est attachée à son palais dans sa soif ; les enfants demandent du pain, et personne ne leur en donne\FTNT{Jé. 52:6.}.
\VS{5}[He.] Ceux qui mangeaient des mets délicats sont en désolation dans les rues ; ceux qui étaient nourris sur l'étoffe écarlate embrassent le fumier.
\VS{6}[Vav.] L'iniquité de la fille de mon peuple est plus grande que le péché de Sodome, renversée en un instant, sans que personne n'ait tourné la main sur elle.
\VS{7}[Zayin.] Ses naziréens étaient plus purs que la neige, plus blancs que le lait ; leur teint était plus vermeil que les pierres précieuses ; ils étaient polis comme un saphir.
\VS{8}[Heth.] Leur apparence est plus sombre que le noir ; on ne les reconnaît pas dans les rues ; ils ont la peau collée sur les os ; elle est devenue sèche comme du bois\FTNT{Job. 30:30.}.
\VS{9}[Teth.]Ceux qui ont été mis à mort par l'épée, ont été plus heureux que ceux qui sont morts par la famine, qui eux sont consumés peu à peu, transpercés par le défaut du fruit des champs.
\VS{10}[Yod.] Les mains des femmes, naturellement tendres, font cuire leurs enfants ; ils leur servent de nourriture dans la ruine de la fille de mon peuple\FTNT{De. 28:57 ; 2 R. 6:29.}.
\VS{11}[Kaf.] Yahweh a accompli sa fureur, il a répandu l'ardeur de sa colère ; il a allumé dans Sion un feu qui en dévore les fondements.
\VS{12}[Lamed.] Les rois de la terre, et tous les habitants de la terre habitable n'auraient jamais cru que l'adversaire et l'ennemi entrerait dans les portes de Jérusalem.
\VS{13}[Mem.] Cela est arrivé à cause des péchés de ses prophètes, et des iniquités de ses sacrificateurs, qui répandaient le sang des justes au milieu d'elle\FTNT{Jé. 5:29-31. Le péché des conducteurs donne accès à l'ennemi pour les détruire ainsi que les biens qui leur ont été confiés (Lu. 11:21-22).}.
\VS{14}[Nun.] Ils erraient comme des aveugles dans les rues, souillés de sang, au point qu'on ne pouvait pas toucher leurs vêtements.
\VS{15}[Samech.] On leur criait : retirez-vous, souillés, retirez-vous, retirez-vous, ne nous touchez point. Quand ils se sont enfuis, ils ont erré ça et là ; on a dit parmi les nations : Ils n'auront plus leur demeure !
\VS{16}[Pe.] La face de Yahweh les a dispersés, il ne veut plus les regarder ; ils n'ont pas eu de respect pour les sacrificateurs, et n'ont pas été miséricordieux envers les vieillards.
\VS{17}[Ayin.] Pour nous, nos yeux se consumaient après un vain secours ; nous regardions du haut de nos lieux élevés vers une nation qui ne pouvait pas délivrer\FTNT{Jé. 18:15.}.
\VS{18}[Tsade.] Ils ont épié nos pas afin de nous empêcher d'aller sur nos places ; notre fin s'approchait, nos jours étaient accomplis… Notre fin est arrivée !
\VS{19}[Qof.] Nos persécuteurs étaient plus légers que les aigles des cieux ; ils nous ont poursuivis sur les montagnes, ils ont mis des embûches contre nous dans le désert.
\VS{20}[Resh.] Le souffle de nos narines, l'oint de Yahweh\FTNT{L'oint en question est le roi Josias (2 R. 21:24 ; 22 ; 23).}, a été pris dans leurs fosses, celui de qui nous disions : Nous vivrons sous son ombre parmi les nations.
\VS{21}[Shin.] Réjouis-toi, sois dans l'allégresse, fille d'Edom, habitante du pays d'Uts ! La coupe passera aussi vers toi ; tu en seras enivrée, et tu seras mise à nu\FTNT{Jé. 25:15-18 ; Ps. 137:7.}.
\VS{22}[Tav.] Fille de Sion, ton iniquité est expiée ; il ne t'enverra plus en exil. Fille d'Edom, il châtiera ton iniquité, il découvrira tes péchés.
\Chap{5}
\TextTitle{Supplications de Jérémie à Yahweh}
\VerseOne{}Souviens-toi, ô Yahweh, de ce qui nous est arrivé ! Regarde et vois notre opprobre !
\VS{2}Notre héritage a été renversé par des étrangers, nos maisons par des inconnus.
\VS{3}Nous sommes devenus comme des orphelins qui sont sans pères, et nos mères sont comme des veuves.
\VS{4}Nous buvons notre eau à prix d'argent, et notre bois nous est vendu.
\VS{5}Ceux qui nous poursuivent sont sur notre cou ; nous sommes épuisés, nous n'avons pas de repos.
\VS{6}Nous avons étendu la main vers l'Egypte, et vers l'Assyrie pour nous rassasier de pain.
\VS{7}Nos pères ont péché, ils ne sont plus, et c'est nous qui portons la peine de leurs iniquités\FTNT{Chaque homme naît pécheur et hérite de la nature pécheresse d'Adam (Ge. 3:20 ; Ac. 17:26 ; Ro. 5:12-21). De ce fait, les péchés commis par les parents ont des conséquences sur les enfants (Ex. 20:4-5 ). Jésus-Christ nous a délivrés du péché d'Adam et de celui de nos ancêtres à la croix (Col. 1:12). Lors de notre naissance d'en haut, les péchés de notre passé et de nos origines sont expiés (2 Co. 5:17 ; Col. 1:12-14 ; Col. 2:13-15 ; Ep. 1:7 ; Ep. 2:1-15 ; 1 Pi. 1:18-19 ; 1 Jn. 1:7 ; 1 Jn. 1:9). Les péchés des ancêtres et leurs conséquences touchent les personnes qui vivent dans les péchés de leurs ancêtres, c'est-à-dire ceux qui haïssent Dieu et ses commandements (De. 24:16 ; Jé. 31:29-34 ; Ez. 18:17-20).}.
\VS{8}Les esclaves dominent sur nous, et personne ne nous délivre de leurs mains.
\VS{9}Nous amenons notre pain au péril de notre vie, à cause de l'épée du désert.
\VS{10}Notre peau est brûlante comme un four, à cause l'ardeur de la faim.
\VS{11}Ils ont déshonoré les femmes dans Sion, les vierges dans les villes de Juda.
\VS{12}Des chefs ont été pendus par leurs mains ; et ils n'ont pas honoré la personne des vieillards.
\VS{13}Ils ont pris les jeunes gens pour moudre, et les enfants sont tombés sous le bois.
\VS{14}Les vieillards ont cessé de se trouver aux portes, et les jeunes gens de chanter.
\VS{15}La joie a disparu de notre cœur, et notre danse est changée en deuil.
\VS{16}La couronne de notre tête est tombée ! Malheur à nous, parce que nous avons péché !
\VS{17}C'est pourquoi notre coeur est languissant. À cause de ces choses, nos yeux sont obscurcis.
\VS{18}À cause de la montagne de Sion qui est désolée ; les renards s'y promènent.
\VS{19}Toi, ô Yahweh, tu demeures éternellement, et ton trône subsiste de génération en génération.
\VS{20}Pourquoi nous oublierais-tu à jamais ? pourquoi nous délaisserais-tu si longtemps ?
\VS{21}Convertis-nous à toi, ô Yahweh ! et nous serons convertis ; renouvelle nos jours comme ils étaient autrefois\FTNT{Jé. 30:20 ; Jé. 31:18 ; Ps. 80:3.}.
\VS{22}Ou bien, nous aurais-tu entièrement rejetés ? Serais-tu extrêmement courroucé contre nous ?
\PPE{}
\end{multicols}
