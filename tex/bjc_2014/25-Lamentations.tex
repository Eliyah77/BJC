\ShortTitle{Lamentations de Jérémie}\BookTitle{Lamentations de Jérémie}\BFont
\noindent\hrulefill
{\footnotesize
\textit{
\bigskip
{\centering{}
\\(Eikha)
\\Signifie : Où ?
\\Thème : Affliction pour Jérusalem
\\Auteur : Jérémie
\\Date de rédaction : 6ème siècle av. J.-C\\}
}
%\bigskip
\textit{
\\Recueil de pièces poétiques,  les lamentations de Jérémie furent composées selon un procédé visant à accentuer le caractère funèbre, de façon à ce qu’elles soient récitées avec gémissements. Ses complaintes exposèrent la profonde désolation du prophète face au fardeau du peuple qu’il portait dans ses entrailles tout comme la douleur et la tristesse de Yahweh face à Israël, responsable de son propre malheur.
\bigskip
\\Très différentes des prophéties retrouvées dans le livre de Jérémie, les lamentations reflètent l’affliction convenant à la gravité du châtiment subi : famine, pillage et ruine du temple, déportation, cessation du culte, diverses calamités... Jérémie rappela ainsi les conséquences de l’endurcissement du cœur face aux appels à la repentance ; il présenta aussi les bontés éternelles de Yahweh. 
\bigskip
}
}
\par\nobreak\noindent\hrulefill
\begin{multicols}{2}
\TextTitle{[La désolation de Jérusalem]}
\Chap{1}
\VerseOne{}[Aleph.] Comment est-il arrivé que la ville si peuplée se trouve si solitaire ? Que celle qui était grande entre les nations est devenue comme une veuve ? Que celle qui était noble Dame entre les Provinces a été rendue tributaire ?
\VS{2}[Beth.] Elle ne cesse de pleurer pendant la nuit, et ses joues sont couvertes de larmes ; de tous ceux qui l'aimaient nul ne la console ; tous ses amis intimes lui sont devenus infidèles, ils sont devenus ses ennemis.
\VS{3}[Guimel.] Juda est en exil victime de l'affliction et d'une grande servitude ; il habite maintenant au milieu des nations, et il n’y trouve pas de repos ; tous ses persécuteurs l'ont surpris dans l'angoisse\FTNT{Jé. 52:27.}.
\VS{4}[Daleth.] Les chemins de Sion sont dans le deuil, plus personne ne vient aux fêtes solennelles ; toutes ses portes sont dans la désolation, ses sacrificateurs gémissent, ses vierges sont affligées de tristesse ; et elle est remplie d'amertume.
\VS{5}[He.] Ses adversaires sont établis pour chefs, ses ennemis prospèrent ; car Yahweh l'a humiliée à cause de la multitude de ses transgressions ; ses enfants ont marché captifs devant l'adversaire\FTNT{Jé. 30:14.}.
\VS{6}[Vau.] Toute la gloire de la fille de Sion s'est éloignée d'elle ; ses chefs sont devenus semblables à des cerfs qui ne trouvent pas de pâture, et qui fuient sans force devant celui qui les poursuit.
\VS{7}[Zaïn.] Aux jours de sa détresse et de sa misère, Jérusalem s’est souvenue de toutes les choses précieuses qu’elle possédait autrefois, quand son peuple est tombé sans secours sous la main de l’oppresseur ; ses ennemis l’ont vue, et ils ont ri de sa chute.
\VS{8}[Heth.] Jérusalem a multiplié ses péchés, c'est pourquoi elle est un objet d'aversion ; tous ceux qui l'honoraient la méprisent en voyant sa nudité ; elle-même soupire, et retourne en arrière.
\VS{9}[Teth.] La souillure était dans les pans de sa robe, et elle ne songeait pas à sa fin ; elle est tombée d'une manière étonnante, et elle n'a pas de consolateur. Vois ma misère, ô Yahweh ! Quelle arrogance chez l'ennemi !
\VS{10}[Jod.] L’oppresseur a étendu sa main sur tout ce qu’elle avait de précieux ; car elle a vu entrer dans son sanctuaire les nations auxquelles tu avais défendu d’entrer dans ton assemblée\FTNT{De. 23:3.}.
\VS{11}[Caph.] Tout son peuple soupire, il cherche du pain ; ils ont donné leurs choses précieuses pour de la nourriture, afin de restaurer leur vie. Vois, ô Yahweh ! Regarde comme je suis vile\FTNT{Jé. 52:6.}.
\VS{12}[Lamed.] Je m'adresse à vous, à vous tous qui passez ici ! Regardez et voyez s'il est une douleur comme ma douleur, celle dont j'ai été châtiée ! Yahweh m'a affligée au jour de son ardente colère.
\VS{13}[Mem.] Il a envoyé d'en haut un feu dans mes os qui les dévore ; il a tendu un filet sous mes pieds, et m'a fait revenir en arrière ; il m'a mise dans la désolation, dans une souffrance de tous les jours.
\VS{14}[Nun.] Sa main a lié le joug de mes transgressions ; elles se sont entrelacées, elles pèsent sur mon cou ;  il a brisé ma force ; le Seigneur m’a livrée entre les mains auxquelles je ne puis résister.
\VS{15}[Samech.] Le Seigneur a terrassé tous mes guerriers que j'avais au milieu de moi ; il a appelé contre moi, au temps fixé, une armée pour détruire mes jeunes hommes ; le Seigneur a foulé au pressoir la vierge, fille de Juda\FTNT{Ez. 30:7 ; Mt. 22:7.}.
\VS{16}[Hajin.] C’est pour cela que je pleure, que mes yeux fondent en larmes ; car le consolateur qui restaurait ma vie est loin de moi.  Mes fils sont dans la désolation, parce que l'ennemi a été plus fort\FTNT{Jé. 13:17.}.
\VS{17}[Pe.] Sion a étendu les mains, et personne ne l'a consolée ; Yahweh a ordonné aux ennemis de Jacob de l’entourer de toutes parts. Jérusalem a été comme une impureté au milieu d'eux.
\VS{18}[Tsade.] Yahweh est juste, car j'ai été rebelle à ses ordres. Ecoutez, vous tous, peuples, et voyez ma douleur ! Mes vierges et mes jeunes hommes sont allés en captivité.
\VS{19}[Koph.] J'ai appelé mes amis, mais ils m'ont trompée. Mes sacrificateurs et mes anciens sont morts dans la ville : Ils cherchaient de la nourriture afin de restaurer leur vie.
\VS{20}[Resch.] Vois ma détresse, ô Yahweh ! Mes entrailles bouillonnent, mon cœur est agité au dedans de moi, car j'ai été rebelle. Au dehors l’épée m’a privée d’enfants, dans ma maison il y a comme la mort.
\VS{21}[Scin.] On a entendu mes soupirs, et personne ne m'a consolée ; tous mes ennemis ont entendu mon malheur et se sont réjouis de ce que tu l’as causé ; tu viendras, tu proclameras le jour où ils seront comme moi.
\VS{22}[Thau.] Que toute leur méchanceté vienne devant ta face, et traite-les avec sévérité  comme tu m'as traité avec sévérité à cause de toutes mes transgressions ! Car mes soupirs sont nombreux, et mon cœur est souffrant.
\TextTitle{[Le jour de la colère de Yahweh]}
\Chap{2}
\VerseOne{}[Aleph.] Comment est-il arrivé que le Seigneur a couvert  de sa colère la fille de Sion tout à l’entour, comme d’une nuée, et qu’il a précipité du ciel sur la terre la beauté d’Israël, et ne s’est pas souvenu de son marchepied au jour de sa colère\FTNT{Ez. 43:7.} ?
\VS{2}[Beth.] Le Seigneur a englouti sans épargner toutes les habitations de Jacob ; il a dans sa fureur renversé les forteresses de la fille de Juda, il les a jetées par terre ; il a profané le royaume et ses rois.
\VS{3}[Guimel.] Il a abattu toute la force d’Israël par l’ardeur de sa colère ; il a retiré sa droite en arrière devant l’ennemi ; il a allumé dans Jacob des flammes de feu, qui consument de toutes parts.
\VS{4}[Daleth.] Il a tendu son arc comme un ennemi ; sa droite s'est dressée comme celle d’un adversaire ; il a tué tout ce qui était agréable à l’œil ; il a répandu sa fureur comme un feu sur la tente de la fille de Sion.
\VS{5}[He.] Le Seigneur a été comme un ennemi ; il a englouti Israël, il a englouti tous ses palais, il a englouti toutes ses forteresses ; il a multiplié chez la fille de Juda le deuil et les gémissements\FTNT{2 R. 25:9.}.
\VS{6}[Vau.] Il a dépouillé sa tente comme un jardin ; il a détruit le lieu de son assemblée ; Yahweh a fait oublier dans Sion la fête solennelle et le sabbat, et dans sa violente colère, il a rejeté le roi et le sacrificateur.
\VS{7}[Zaïn.] Le Seigneur a rejeté son autel, il a repoussé son sanctuaire ; il a livré entre les mains de l’ennemi les murs des palais de Sion ; ils ont fait du bruit dans la maison de Yahweh, comme aux jours des fêtes solennelles.
\VS{8}[Heth.] Yahweh avait projeté de détruire les murs de la fille de Sion ; il a étendu le cordeau, il n’a pas retiré sa main sans les avoir engloutis ; il a plongé dans le deuil remparts et murailles, qui n'offrent plus ensemble qu'une triste ruine.
\VS{9}[Teth.] Ses portes sont enfoncées dans la terre ; il en a détruit et brisé les barres. Son roi et ses chefs sont parmi les nations ; la loi n’est plus. Même les prophètes ne reçoivent plus aucune vision de Yahweh\FTNT{Ez. 7:26.}.
\VS{10}[Jod.] Les anciens de la fille de Sion sont assis à terre, ils sont muets ; ils ont couvert leur tête de poussière, ils se sont ceints de sacs ; les vierges de Jérusalem baissent leurs têtes vers la terre.
\VS{11}[Caph.] Mes yeux se consument à force de larmes, mes entrailles bouillonnent, ma bile se répand sur la terre. A cause des ruines de la fille de mon peuple, des enfants et des nourrissons qui tombent en défaillance dans les rues de la ville.
\VS{12}[Lamed.] Ils disaient à leurs mères : Où y a-t-il du blé et du vin ? Et ils tombaient comme morts dans les rues de la ville, comme un homme blessé à mort, ils rendaient l'âme sur le sein de leurs mères.
\VS{13}[Mem] Qui dois-je prendre à témoin ? A quoi te comparer, fille de Jérusalem ? Qui pourrait t'égaler, et quelle consolation te donner, vierge, fille de Sion ? Car ta ruine est grande comme une mer : Qui pourrait te guérir\FTNT{Es. 51:19-20.} ?
\VS{14}[Nun.] Tes prophètes ont eu pour toi des visions vaines et fausses ; ils n’ont pas découvert ton iniquité, afin de détourner de toi la captivité ; ils t’ont prophétisé des oracles mensongers et trompeurs\FTNT{Jé. 2:8 ; Jé. 5:31 ; Jé. 14:14.}.
\VS{15}[Samech.] Tous les passants applaudissent sur toi, ils sifflent, ils secouent leur tête contre la fille de Jérusalem : Est-ce ici la ville de laquelle on disait : La parfaite en beauté, la joie de toute la terre\FTNT{Jé. 25:9 ; Na. 3:19.} ?
\VS{16}[Pe.] Tous tes ennemis ouvrent la bouche contre toi, ils sifflent, ils grincent des dents, ils disent : Nous l'avons engloutie ! C’est ici le jour que nous attendions, nous l’avons atteint, nous le voyons !
\VS{17}[Hajin.] Yahweh a exécuté ce qu’il avait projeté, il a accompli la parole qu’il avait dès longtemps arrêtée, il a détruit sans épargner, il a fait de toi la joie de l'ennemi, il a donné de la force à tes adversaires.
\VS{18}[Tsade.] Leur cœur crie au Seigneur... Mur de la fille de Sion, fais couler des larmes jour et nuit, comme un torrent ! Ne te donne pas de repos ; et que tes yeux de fille n’aient pas de repos\FTNT{Jé. 14:17.}  !
\VS{19}[Koph.] Lève-toi, pousse des gémissements dès le commencement des veilles de la nuit ! Répands ton cœur comme de l’eau en présence du Seigneur ! Lève tes mains vers lui pour la vie de tes enfants qui meurent de faim aux coins de toutes les rues\FTNT{Ps. 42:3.} !
\VS{20}[Resch.] Vois, ô Yahweh ! Regarde qui tu as traité avec sévérité ! Fallait-il que les femmes mangent le fruit de leurs entrailles, leurs petits enfants objets de leur tendresse ? Que le sacrificateur et le prophète soient égorgés dans le sanctuaire du Seigneur\FTNT{Lé. 26:29 ; De. 28:53 ; Jé. 19:9. Le siège de Jérusalem par les Chaldéens a eu lieu en 587 av. J.-C. et 586 av. J.-C.} ?
\VS{21}[Scin.] Les jeunes gens et les vieillards sont couchés par terre dans les rues ; mes vierges et mes jeunes gens sont tombés par l’épée ; tu as tué au jour de ta colère, tu as massacré sans épargner.
\VS{22}[Thau.] Tu as appelé sur moi la terreur de toutes parts, comme un jour solennel. Au jour de la colère de Yahweh, il n'y a eu ni réchappé ni survivant. Ceux que j’avais soignés et élevés, mon ennemi les a consumés.
\TextTitle{[Jérémie partage l'affliction de son peuple]}
\Chap{3}
\VerseOne{}[Aleph.] Je suis l’homme qui a vu l’affliction sous la verge de sa fureur\FTNT{Jé. 15:15-18.}.
\VS{2}Il m’a conduit, mené dans les ténèbres, et non dans la lumière.
\VS{3}Contre moi il tourne et retourne sa main tout le jour.
\VS{4}[Beth.] Il a fait vieillir ma chair et ma peau, il a brisé mes os\FTNT{Es. 38:13.}.
\VS{5}Il a bâti autour de moi, il m’a environné du venin et de la douleur.
\VS{6}Il me fait habiter dans les ténèbres, comme ceux qui sont morts depuis longtemps.
\VS{7}[Guimel.] Il m'a entouré d'un mur, afin que je ne sorte pas ; il m'a donné des chaînes pesantes.
\VS{8}J'ai beau crier et implorer du secours, il repousse ma prière.
\VS{9}Il a fait un mur de pierres de taille pour fermer mon chemin, il a détruit mes sentiers.
\VS{10}[Daleth.] Il a été pour moi un ours en embuscade, un lion qui se tient dans un lieu caché\FTNT{Os. 13:8.}.
\VS{11}Il a détourné mes chemins, il m’a mis en pièces, il m’a mis dans la désolation.
\VS{12}Il a tendu son arc, et il m’a placé comme une cible pour sa flèche.
\VS{13}[He.] Il a fait entrer dans mes reins les flèches de son carquois.
\VS{14}Je suis la risée pour tout mon peuple, et leur chanson\FTNT{Job. 30:9 ; Ps. 69:13.} chaque jour.
\VS{15}Il m’a rassasié d’amertume, il m’a enivré d’absinthe.
\VS{16}[Vau.] Il a brisé mes dents avec du gravier, il m’a couvert de cendres.
\VS{17}Tellement que la paix s’est éloignée de mon âme, j’ai oublié ce que c’est que d’être à son aise.
\VS{18}Et j’ai dit : Ma force est perdue, je n'ai plus d'espérance en Yahweh !
\VS{19}[Zaïn.] Souviens-toi de mon affliction, et de mon pauvre état, qui n’est qu’absinthe et que fiel ;
\VS{20}Mon âme s’en souvient sans cesse, et elle est abattue au dedans de moi.
\VS{21}Voici ce que je veux rappeler à mon cœur, et c’est pourquoi j’aurai de l'espérance :
\VS{22}[Heth.] C’est une grâce de Yahweh que nous n’avons point été consumés, parce que ses compassions ne sont pas terminées\FTNT{Ps. 103:10.} ;
\VS{23}elles se renouvellent chaque matin. C’est une chose grande que ta fidélité !
\VS{24}Yahweh est ma portion, dit mon âme ; c’est pourquoi j’aurai espérance en lui\FTNT{Ps. 16:5.}.
\VS{25}[Teth.] Yahweh est bon pour ceux s’attendent à lui, pour l’âme qui le consulte.
\VS{26}C’est une chose bonne qu’on attende, même en se tenant en repos, la délivrance de Yahweh.
\VS{27}C’est une chose bonne pour l’homme de porter le joug dans sa jeunesse.
\VS{28}[Jod.] Il s'assiéra solitaire et silencieux, parce que Yahweh le lui impose.
\VS{29}Il mettra sa bouche dans la poussière, sans perdre toute espérance.
\VS{30}Il présentera la joue à celui qui le frappe, il se rassasiera d’opprobres.
\VS{31}[Caph.] Car le Seigneur ne rejette pas à toujours\FTNT{Es. 57:16 ; Ps. 77:8.}.
\VS{32}Mais s’il afflige quelqu’un, il a aussi compassion selon sa grande miséricorde.
\VS{33}Car ce n’est pas sa volonté d'affliger et d'humilier les fils des hommes.
\VS{34}[Lamed.] Lorsqu’on foule aux pieds tous les prisonniers du pays,
\VS{35}lorsqu’on pervertit la justice humaine à la face du Très-Haut,
\VS{36}lorsqu’on fait tort à quelqu’un dans son procès, le Seigneur ne le voit-il pas ?
\VS{37}[Mem.] Qui dira qu'une chose arrive, sans que le Seigneur ne l’ait ordonnée ?
\VS{38}Les maux et les biens\FTNT{Job. 1:21 ; Es. 45:7 ; Am. 3:6.} ne procèdent-ils pas de la bouche du Très-Haut ?
\VS{39}Pourquoi l’homme vivant se plaindrait-il ? Que l’homme se plaigne à cause de ses péchés.
\TextTitle{[Le peuple appelé à s'examiner pour revenir à Yahweh]}
\VS{40}[Nun.] Recherchons nos voies, sondons-les, et retournons à Yahweh\FTNT{Ps. 119:59 ; 2 Co. 13:5.} ;
\VS{41}élevons nos cœurs et nos mains vers Dieu qui est au ciel :
\VS{42}Nous avons péché, nous avons été rebelles ! Tu n’as pas pardonné !
\VS{43}[Samech.] Tu nous as couverts dans ta colère, et tu nous as poursuivis ; tu as tué sans épargner ;
\VS{44}tu t’es enveloppé d'un nuage pour fermer l’accès à la prière.
\VS{45}Tu nous as rendus un objet de mépris et de dédain au milieu des peuples.
\VS{46}[Pe.] Tous nos ennemis ouvrent leur bouche contre nous.
\VS{47}Notre partage a été la terreur et la fosse, le ravage et la ruine\FTNT{Jé. 48:44 ; Es. 24:18.}.
\VS{48}De mes yeux coulent des torrents d’eau, à cause de la ruine de la fille de mon peuple.
\VS{49}[Hajin.] Mon œil fond en larmes, sans repos, sans relâche,
\VS{50}jusqu’à ce que Yahweh regarde et voie des cieux\FTNT{Ps. 80:15 ; Ps. 102:20.} ;
\VS{51}mon œil fait souffrir mon âme, à cause de toutes les filles de ma ville.
\TextTitle{[Yahweh soutient Jérémie en prison]}
\VS{52}[Tsade.] Ceux qui sont mes ennemis sans cause m’ont donné la chasse comme un oiseau.
\VS{53}Ils ont voulu mettre fin à ma vie dans une fosse, et ils ont jeté des pierres sur moi.
\VS{54}Les eaux ont inondé ma tête ; je disais : Je suis perdu !
\VS{55}[Koph.] J’ai invoqué ton nom, ô Yahweh, du fond de la fosse\FTNT{Jé. 38:6.}.
\VS{56}Tu as entendu ma voix : Ne ferme pas tes oreilles à mes soupirs, à mes cris !
\VS{57}Au jour où je t’ai invoqué, tu t'es approché, tu as dit : Ne crains rien !
\VS{58}[Resch.] Seigneur, tu as plaidé la cause de mon âme, tu as racheté ma vie.
\VS{59}Yahweh, tu as vu ce qu’on m'a fait souffrir : Rends-moi justice !
\VS{60}Tu as vu toutes les vengeances dont ils ont usé, et toutes leurs machinations contre moi.
\VS{61}[Scin.] Yahweh, tu as entendu leurs outrages, toutes leurs machinations contre moi,
\VS{62}les discours de ceux qui se lèvent contre moi, et leur dessein qu'ils  ont contre moi tout le long du jour.
\VS{63}Regarde, quand ils sont assis et quand ils se lèvent, car je suis leur chanson.
\VS{64}[Thau.] Rends-leur la pareille, ô Yahweh, selon l’œuvre de leurs mains ;
\VS{65}livre-les à l'endurcissement de leur cœur, donne-leur ta malédiction.
\VS{66}Poursuis-les dans ta colère, et extermine-les de dessous les cieux, ô Yahweh !
\TextTitle{[Les atrocités pendant le siège de Jérusalem]}
\Chap{4}
\VerseOne{}[Aleph.] Comment l’or a-t-il perdu son éclat ! L'or pur est altéré ! Les pierres du sanctuaire sont dispersées aux coins de toutes les rues !
\VS{2}[Beth.] Les nobles enfants de Sion, estimés à l'égal de l'or pur, sont regardés, hélas, comme des vases de terre, ouvrage des mains du potier !
\VS{3}[Guimel.] Il y a même des monstres marins qui présentent leurs mamelles et allaitent leurs petits ; mais la fille de mon peuple est devenue cruelle comme les autruches du désert.
\VS{4}[Daleth.] La langue du nourrisson s'attache à son palais, desséchée par la soif ; les enfants demandent du pain, et personne ne leur en donne\FTNT{Jé. 52:6.}.
\VS{5}[He.] Ceux qui mangeaient des mets délicats périssent dans les rues ; ceux qui étaient nourris sur l’étoffe écarlate embrassent le fumier.
\VS{6}[Vau.] L’iniquité de la fille de mon peuple est plus grande que le péché de Sodome, détruite en un instant, sans que personne n’ait tourné la main sur elle.
\VS{7}[Zaïn.] Ses princes étaient plus éclatants que la neige, plus blancs que le lait ; leur teint était plus vermeil que le corail ; ils étaient polis comme un saphir.
\VS{8}[Heth.] Leur aspect est plus sombre que le noir ; on ne les reconnaît pas dans les rues ; ils ont la peau collée sur les os ; elle est devenue sèche comme du bois\FTNT{Job. 30:30.}.
\VS{9}[Teth.] Ceux qui périssent par l’épée sont plus heureux que ceux qui périssent par la famine, qui tombent exténués, privés du fruit des champs.
\VS{10}[Jod.] Les mains des femmes, malgré leur tendresse, font cuire leurs enfants ; ils leur servent de nourriture, au milieu des ruines de la fille de mon peuple\FTNT{De. 28:57 ; 2 R. 6:29.}.
\VS{11}[Caph.] Yahweh a accompli sa fureur, il a répandu l’ardeur de sa colère ; il a allumé dans Sion un feu qui en dévore les fondements.
\VS{12}[Lamed.] Les rois de la terre n'auraient pas cru, aucun des habitants du monde n'aurait cru que l’adversaire, que l’ennemi entrerait dans les portes de Jérusalem.
\VS{13}[Mem.] Voilà le fruit des péchés de ses prophètes, des iniquités de ses sacrificateurs, qui ont répandu le sang des justes au milieu d’elle\FTNT{Jé. 5:31. Le péché des conducteurs donne accès à l’ennemi pour les détruire ainsi que les biens qui leur ont été confiés (Lu. 11:21-22).}.
\VS{14}[Nun.] Ils erraient en aveugles dans les rues, souillés de sang ; on ne pouvait pas toucher leurs vêtements.
\VS{15}[Samech.] Eloignez-vous, impurs ! Leur criait-on, Eloignez-vous, ne nous touchez pas ! Ils s'enfuient, ils errent çà et là ; on dit parmi les nations : Ils n’y retourneront plus pour y séjourner.
\VS{16}[Pe.] Yahweh les a dispersés de devant sa face, il ne les regarde plus ; ils n’ont pas eu de respect pour les sacrificateurs, et n'ont pas été miséricordieux envers les vieillards.
\VS{17}[Hajin.] Nos yeux consumaient encore, et nous attendions en vain du secours ; nous regardions avec espérance vers une nation qui ne peut pas nous délivrer\FTNT{Jé. 18:15.}.
\VS{18}[Tsade.] On épiait nos pas pour nous empêcher d'aller sur nos places ; notre fin s'approchait, nos jours étaient accomplis... Notre fin est arrivée !
\VS{19}[Koph.] Nos persécuteurs étaient plus légers que les aigles du ciel ; ils nous ont poursuivis sur les montagnes, ils ont mis des embûches contre nous dans le désert.
\VS{20}[Resch.] Le souffle de nos narines, l’oint de Yahweh, a été pris dans leurs fosses, celui de qui nous disions : Nous vivrons sous son ombre parmi les nations\FTNT{L’oint en question est le roi Josias (2 R. 21:24 ; 2 R. 22 ; 2 R. 23).}.
\VS{21}[Scin.] Réjouis-toi, sois dans l’allégresse, fille d’Edom, habitante du pays d'Uts ! La coupe passera aussi vers toi ; tu t'enivreras, et tu seras mise à nu\FTNT{Jé. 25:15-21 ; Ps. 137:7.}.
\VS{22}[Thau.] Fille de Sion, ton iniquité est expiée ; il ne t'enverra plus en captivité. Fille d’Edom, il châtiera ton iniquité, il mettra à découvert tes péchés.
\TextTitle{[Complainte adressée à Yahweh]}
\Chap{5}
\VerseOne{}Souviens-toi, ô Yahweh, de ce qui nous est arrivé ! Regarde et vois notre opprobre !
\VS{2}Notre héritage a été renversé par des étrangers, nos maisons par des inconnus.
\VS{3}Nous sommes orphelins, sans pères ; nos mères sont comme des veuves.
\VS{4}Nous buvons notre eau à prix d’argent, nous payons notre bois.
\VS{5}Nous sommes poursuivis, le joug sur le cou ; nous sommes épuisés, nous n’avons pas de repos.
\VS{6}Nous avons tendu la main vers l'Egypte, vers l'Assyrie pour nous rassasier de pain.
\VS{7}Nos pères ont péché, ils ne sont plus, et c’est nous qui portons la peine de leurs iniquités\FTNT{Chaque homme nait pécheur et hérite  de la nature pécheresse d’Adam (Ge. 3:20 ; Ac. 17:26 ; Ro. 5:12-21). De ce fait, les péchés commis par les parents ont des conséquences sur les enfants (Ex. 20:4-5 ). Jésus-Christ nous a délivrés du péché d’Adam et de celui de nos ancêtres à la croix (Col. 1:12). Lors de notre naissance d’en haut, les péchés de notre passé et de nos origines sont expiés (2 Co. 5:17 ; Col. 1:12-14 ; Col. 2:13-15 ; Ep. 1:7 ; Ep. 2:1-15 ; 1 Pi. 1:18-19 ; 1 Jn. 1:7 ; 1 Jn. 1:9). Les péchés des ancêtres et leurs conséquences touchent les personnes qui vivent dans les péchés des ancêtres, c'est-à-dire ceux qui haïssent Dieu et ses commandements (De. 24:16 ; Jé. 31:29-34 ; Ez. 18:17-20).}.
\VS{8}Les esclaves dominent sur nous, et personne ne nous délivre de leurs mains.
\VS{9}Nous cherchons notre pain au péril de notre vie, devant l’épée du désert.
\VS{10}Notre peau est brûlante comme un four, par l’ardeur de la faim.
\VS{11}Ils ont déshonoré les femmes dans Sion, les vierges dans les villes de Juda.
\VS{12}Des chefs ont été pendus par leurs mains ; et ils n'ont pas honoré la personne des vieillards.
\VS{13}Les jeunes hommes ont porté la meule, les enfants trébuchaient sous les fardeaux de bois.
\VS{14}Les vieillards ne vont plus à la porte, et les jeunes hommes ont cessé de chanter.
\VS{15}La joie a disparu de nos cœurs, et le deuil a remplacé nos danses.
\VS{16}La couronne de notre tête est tombée ! Malheur à nous, parce que nous avons péché !
\VS{17}Si notre cœur est souffrant, si nos yeux sont obscurcis,
\VS{18}c'est que la montagne de Sion est dans la désolation, c'est que les renards s'y promènent.
\VS{19}Toi, ô Yahweh, tu règnes éternellement ;  ton trône subsiste de génération en génération.
\VS{20}Pourquoi nous oublierais-tu pour toujours, nous abandonnerais-tu pour de longues années ?
\VS{21}Fais-nous revenir vers toi, ô Yahweh, et nous reviendrons ! Renouvelle nos jours comme ils étaient autrefois\FTNT{Jé. 30:20 ; Jé. 31:18 ; Ps. 80:3.}.
\VS{22}Nous aurais-tu rejetés, et te fâcherais-tu extrêmement contre nous ?
\PPE{}
\end{multicols}
